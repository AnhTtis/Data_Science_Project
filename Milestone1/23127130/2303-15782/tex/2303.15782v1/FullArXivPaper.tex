% CVPR 2023 Paper Template
% based on the CVPR template provided by Ming-Ming Cheng (https://github.com/MCG-NKU/CVPR_Template)
% modified and extended by Stefan Roth (stefan.roth@NOSPAMtu-darmstadt.de)
\pdfoutput=1
\pdfminorversion=6
\documentclass[10pt,twocolumn,letterpaper]{article}

%%%%%%%%% PAPER TYPE  - PLEASE UPDATE FOR FINAL VERSION
% \usepackage[review]{cvpr}      % To produce the REVIEW version
% \usepackage{cvpr}              % To produce the CAMERA-READY version
\usepackage[pagenumbers]{cvpr} % To force page numbers, e.g. for an arXiv version

\makeatletter
\@namedef{ver@everyshi.sty}{}
\makeatother

% Include other packages here, before hyperref.
\usepackage{graphicx}
\usepackage{amsmath}
\usepackage{amssymb}
\usepackage{booktabs}
\usepackage{flushend}

% My packages
\setlength {\marginparwidth }{1.5cm}
\setlength{\parskip}{1pt}

\usepackage{todonotes}
\usepackage{outlines}
\usepackage{caption}
\usepackage{subcaption}
\captionsetup[figure]{font=footnotesize,labelfont=footnotesize}
\captionsetup{font=footnotesize}
\usepackage{microtype}
\usepackage{hhline}
\usepackage{multirow}
\usepackage{siunitx}
\usepackage{makecell}
\usepackage{gensymb}
\usepackage[inline]{enumitem}
\usepackage{placeins}

\usepackage{algorithm}
% \usepackage{algorithmic}
% \usepackage{algorithmicx}
\usepackage{algpseudocode}
\usepackage{cuted}
\usepackage[accsupp]{axessibility}  % Improves PDF readability for those with disabilities.

\usepackage{ifthen}
\newboolean{reftosupp}
\setboolean{reftosupp}{true}
\newboolean{reftomain}
\setboolean{reftomain}{true}


\usepackage{pifont}% http://ctan.org/pkg/pifont
\newcommand{\cmark}{\ding{51}}%
\newcommand{\xmark}{\ding{55}}%
\newcommand{\sz}[1]{{\color{blue}{[Sergey: #1]}}}

% It is strongly recommended to use hyperref, especially for the review version.
% hyperref with option pagebackref eases the reviewers' job.
% Please disable hyperref *only* if you encounter grave issues, e.g. with the
% file validation for the camera-ready version.
%
% If you comment hyperref and then uncomment it, you should delete
% ReviewTempalte.aux before re-running LaTeX.
% (Or just hit 'q' on the first LaTeX run, let it finish, and you
%  should be clear).
% \usepackage[pagebackref,breaklinks,colorlinks]{hyperref}
\usepackage{hyperref}
\hypersetup{pagebackref,breaklinks,colorlinks}

% Support for easy cross-referencing
\usepackage[capitalize]{cleveref}
\crefname{section}{Sec.}{Secs.}
\Crefname{section}{Section}{Sections}
\Crefname{table}{Table}{Tables}
\crefname{table}{Tab.}{Tabs.}
\crefname{algorithm}{Algo.}{Algos.}
\Crefname{algorithm}{Algorithm}{Algorithms}
\crefname{appendix}{Sec.}{Secs.}
\Crefname{appendix}{Section}{Sections}
% \crefname{appsec}{Sec.}{Secs.}
% \Crefname{appsec}{Section}{Sections}


% Loading variable definitions
\newcommand{\ourName}{CARTO}
\newcommand{\ourNameFull}{Category and Joint Agnostic Reconstruction of ARTiculated Objects}
% \newcommand{\ourNameFull}{Category and joint agnostic reconstruction of ARTiculated Objects}

\newcommand{\latentVariable}{\boldsymbol{z}}
\newcommand{\latentShape}{\latentVariable_{\text{s}}}
\newcommand{\latentJoint}{\latentVariable_{\text{j}}}
\newcommand{\geometryDecoder}{\phi_{\text{geom}}}
\newcommand{\jointDecoder}{\phi_{\text{joint}}}

\newcommand{\jointType}{\textit{jt}}
\newcommand{\jointState}{q}

\newcommand{\objectCount}{M}
\newcommand{\objectIndex}{\MakeLowercase{\objectCount}}
\newcommand{\jointCount}{N}
\newcommand{\jointIndex}{\MakeLowercase{\jointCount}}

\newcommand{\object}{x}
\newcommand{\objectLoD}{l}

\newcommand{\spaceCoordinate}{\boldsymbol{x}}
\newcommand{\normal}{\boldsymbol{n}}
\newcommand{\sdfValue}{s}
\newcommand{\allSdfValues}{\boldsymbol{S}}


\newcommand{\realspace}{\mathbb{R}}
\newcommand{\dimensionsShapeSpace}{D_s}
\newcommand{\dimensionsJointSpace}{D_j}
\newcommand{\dimensionsIndex}{d}

\newcommand{\leakyClamp}{\text{clamp}_l}
\newcommand{\leakyThreshold}{\delta}
\newcommand{\leakySlope}{\alpha}

\newcommand{\lossGeneralFunction}{\mathcal{L}}
\newcommand{\lossScaling}{\delta}

\newcommand{\indexReconstruction}{\text{rec}}
\newcommand{\indexCodeRegularizer}{\text{reg}}
\newcommand{\indexJointRegularizer}{\text{jr}}
\newcommand{\indexJointType}{\jointType}
\newcommand{\indexJointState}{\jointState}

\newcommand{\similarityFunction}{\text{sim}}
\newcommand{\indexLatent}{\text{latent}}
\newcommand{\indexReal}{\text{real}}
\newcommand{\similarityFunctionLatent}{\similarityFunction_\indexLatent{}}
\newcommand{\similarityFunctionReal}{\similarityFunction_\indexReal{}}
\newcommand{\similarityMatrix}{\boldsymbol{S}}

\newcommand{\jointToCodeFunction}{\xi_{\text{code}}}
\newcommand{\polynomDegree}{p}

\newcommand{\importance}{\psi}
\newcommand{\importanceFull}{\Psi}
\newcommand{\depth}{d}
\newcommand{\depthMap}{D^{W \times H}}
\newcommand{\inputImage}{I^{W \times H \times 6}}


\newcommand{\sdfValueThreshold}{\epsilon}

% Spacer for math
\newcommand\+{\mkern2mu}


%%%%%%%%% PAPER ID  - PLEASE UPDATE
\def\cvprPaperID{9001} % *** Enter the CVPR Paper ID here
\def\confName{CVPR}
\def\confYear{2023}

%\renewcommand{\baselinestretch}{0.99}

%%%%%%%%% TITLE - PLEASE UPDATE
\title{\ourName: \ourNameFull}

\author{
Nick~Heppert$^{1,3}$,~
Muhammad~Zubair~Irshad$^{2}$,~
Sergey~Zakharov$^{3}$,~
Katherine~Liu$^{3}$,\\
Rares~Andrei~Ambrus$^{3}$,~
Jeannette~Bohg$^{4}$,~
Abhinav~Valada$^{1}$,~
Thomas~Kollar$^{3}$\vspace{2mm}\\
$^{1}$University~of~Freiburg\quad$^{2}$Georgia~Institute~of~Technology\\
$^{3}$Toyota~Research~Institute~(TRI)\quad$^{4}$Stanford~University
}

\begin{document}
% \maketitle
\twocolumn[{%
\renewcommand\twocolumn[1][]{#1}%
\maketitle
\vspace{-6mm}
\begin{center}
    \centering
    \includegraphics[width=0.98\textwidth]{content/assets/imgs/figure_1_low_res.pdf}
    \captionsetup{width=\linewidth}
    \captionof{figure}{
    Visualization of \ourName{} on unseen object instances. We first use \ourName{} to jointly detect all objects in the scene and then articulate them while keeping the predicted shape code constant. 
    }
   \label{fig:teaser}
\end{center}%
}]

%\let\thefootnote\relax\footnotetext{*{\tt heppert@cs.uni-freiburg.de}}


%%%%%%%%% ABSTRACT
\begin{abstract}
   \begin{abstract}
The current study investigated possible human-robot kinaesthetic interaction using a variational recurrent neural network model, called PV-RNN, which is based on the free energy principle.
Our prior robotic studies using PV-RNN showed that the nature of interactions between top-down expectation and bottom-up inference is strongly affected by a parameter, called the meta-prior, which regulates the complexity term in free energy.
% The current study examines how the behaviours of robots alter by changing the meta-prior $w$ in human-robot kinaesthetic interaction.
The current study examines how changing the meta-prior $w$ in the interaction phase affects the counter force generated when an experimenter attempts to induce movement pattern transitions familiar to the robot through its prior training.
The study also compares the counter force generated when trained transitions are induced by a human experimenter and when untrained transitions are induced.
Our experimental results indicated that (1) the human experimenter needs more/less force to induce trained transitions when $w$ is set with larger/smaller values, (2) the human experimenter needs more force to act on the robot when he attempts to induce untrained as opposed to trained movement pattern transitions.
Our analysis of time development of essential variables and values in PV-RNN during bodily interaction clarified the mechanism by which gaps in actional intentions between the human experimenter and the robot can be manifested as reaction forces between them.


%% Hiroki writing 2022-11-4
%Current study investigates the dynamics of the latent states during human-robot kinaesthetic interaction using PV-RNN.
%We have achieved to observe and analyse the internal state of an RNN model based on the free energy principle, during real-time human-robot interaction.
%Essential characteristics observed in the previous study of this variational recurrent neural network model, PV-RNN, is that by changing a meta prior $w$, the balance between the top-down intention and the bottom-up perceptual reality changes.
%In the current study, we examined how changing the weighting parameter $w$ between accuracy and complexity in free energy principle affects the humanoid robot's behaviour through human-robot interaction. We have conducted some human-robot kinaesthetic interaction experiments with various $w$ and quantitatively analysed the latent variable and the force applied to the humanoid robot. We have observed that the force required to change the robot's intention has increased, both when the top-down intention was strengthened by changing the $w$ and when corresponding switch of its primitive was against the experience of the RNN during its training. The study confirms through quantitative analysis that by increasing or decreasing the $w$ in PV-RNN, humanoid robot leads or follows the human counterpart during the human-robot kinaesthetic interaction.

\begin{comment}
Comment from Jun #2
・最後にQualitativeな結果(インパクト)が欲しい
・Current study investigates the problem on~と書き出すのが一般的
・最初の一文と最後の一文を対応させる
・最後の一文はもう少しAbstractかつ包括的に
\end{comment}

\begin{comment}
Comment from Jun #1
We investigated how the kinaesthetic human-robot interaction can affect the internal state of a model based on the free energy principle. 
=> how the internal state is affected is not the most important point in this study. This part should be rewritten.

The key function of this variational recurrent neural network model, PV-RNN, is that by changing a meta prior $w$, it takes a balance between the "complexity” term and the ”accuracy” term which corresponds to a top-down intention and a bottom-up perceptual reality in the free energy principle, respectively. 
=> This is not key function of PV-RNN. It is an essential characteristics observed in the previous study. The grammar after $w$ is something strange. Rewrite these.

This research has conducted a human-robot interaction experiment with a robotic agent in a kinaesthetic sense.
=> The sentence is not good. "in a kinaesthetic sense" is grammatically wrong.
MODIFIED => "In the current study human-robot interaction experiments using the kinaesthetic sense were conducted."

We investigated that when human forces the agent to switch primitives from one to another, larger force was required both when the human intention is conflictive against the top-down the intention of the agent and when the agent has a stronger top-down intention by modifying the $w$.
=> You should write the essential results of the experiments rather than what we investigated and also how these results could contribute to the studies on human-robot interaction.
\end{comment}

\end{abstract}
\end{abstract}

%%%%%%%%% BODY TEXT
% \begin{figure}[t]
%     % \begin{subfigure}{1\linewidth}
%     %   \centering
%     % %   \includegraphics[width=1\linewidth]{figs/fig_1_moti_textattn.pdf}  
%     % %   \includegraphics[width=1\linewidth]{figs/fig_1_moti_textattn_v2.pdf}  
%     %   \includegraphics[width=1\linewidth]{figs/fig_1_moti_textattn_v5.pdf}  
%     %   \vspace{-0.5cm}
%     %     \caption{Amount of attention added to each video clip from the source video and query text in the self-attention layers of Moment-DETR encoder.}
%     %     % \caption{Distribution of attention for source and query in Moment-DETR encoder}
%     %     % Visualization of video clip's self-attention score in Moment-DETR encoder.
%     %   \label{fig:fig1_text_attn_ex}
%     % \end{subfigure}%\hfill% or  or \hspace{0.3\textwidth}
%     \vspace{0.2cm}
%     % \begin{subfigure}{1\linewidth}
%       \centering
%     %   \includegraphics[width=1\linewidth]{figs/fig1_moti_negattn.pdf}  
%       \includegraphics[width=1\linewidth]{figs/fig1_moti_negattn_v3.pdf}  
%       \vspace{-0.4cm}
%     %   \caption{Correspondence of saliency scores on the relevance between video clips and the text query.}
%     % \caption{Predicted saliency scores against the video relevant positive query and video irrelevant negative query}
%       \label{fig:fig1_neg_attn_ex}
%     % \end{subfigure}%\hfill% or  or \hspace{0.3\textwidth}
%     \caption{
%     % 원준 원본
%     % (a) Comparison between attention scores of source and query for each video clip~(We sum the attention scores from video and text). 
%     % We observe that the attention scores are dominated by other clips in the source video. 
%     % Text queries do not account for much attention regardless of the relevance to the video clips.
%     % \textbf{(a)} Inspection of the query dependency in Moment-DETR encoder.
%     % % We visualize the attention score of video tokens in the transformer encoder and observe that text query accounts for only a low portion of attention.
%     % % This tendency occurs regardless of the relevance between the text query and video clips. 
%     % We visualize the attention score of video tokens in the transformer encoder and observe 1) text query only accounts for a low portion of attention, and 2) relevance between video-query pair does not affect the attention scores ratio of text.
%     \textbf{(b)} Comparison of highlight-ness when relevant and non-relevant queries are input.
%     As observed in , existing work only uses queries to play an insignificant role, thereby may not be capable of detecting false queries and considering the video-query relevance even when the problem in (a) is resolved. 
%     % \SE{} % 이 부분이 "not capable of" 란 용어가 세다는 피드백이 있는 듯 합니다. 이러한 능력이 없다는 것은 굉장히 강한 어조인거 같기는 하고, 이러한 경우들이 종종 있다거나 좀 약화시킬 필요가 있어보이긴 하네요.
%     On the other hand, our QD-DETR yields a query-dependent representation that the relevance between the source video and query text is updated in the saliency scores.
%     There is a large gap between positive and negative saliency scores, and scores are consistent since the clips are all highly correlated to others.
%     }
%     \label{fig:motivation_ex}
%     % \captionsetup{belowskip=13pt}
%     % \setlength{\belowcaptionskip}{-10pt}
% \end{figure}
\begin{figure}
    \centering
    \includegraphics[width=1\linewidth]{figs/fig1_moti_negattn_1111.pdf}
    % \includegraphics[width=1\linewidth]{figs/fig1_moti_negattn_1109.pdf}
    % \includegraphics[width=1\linewidth]{figs/fig1_moti_negattn_stat.pdf}
    \vspace{-0.6cm}
    \caption{
        % \SE{} % 수정 필요
        Comparison of highlight-ness~(saliency score) when relevant and non-relevant queries are given.
        We found that the existing work only uses queries to play an insignificant role, thereby may not be capable of detecting negative queries and video-query relevance; saliency scores for clips in ground-truth~(GT) moments are low and equivalent for positive and negative queries.
        % This also results in mispredicted moments when ground-truth~(GT) moment is dominated by clips unrelated to GT since their prediction is highly focused on the video.
        % \SE{} % 여기 한번 더 보면 좋을 듯 합니다. GT moment에 unrelated한 clip이 많으면? label이 틀렷을 경우를 말씀하시는건지?
        % As observed in saliency graph, existing work only uses queries to play an insignificant role, thereby may not be capable of detecting false queries and considering the video-query relevance.
        On the other hand, query-dependent representations of QD-DETR result in corresponding saliency scores to the video-query relevance and precisely localized moments.
        % On the other hand, our QD-DETR yields a query-dependent representation that the
        % saliency scores are in accordance with the relevance between the video and query.
        % text is in accordance with the saliency scores.
        % There is a large gap between positive and negative saliency scores, and scores are consistent since the clips are all highly correlated to others.
}
    \label{fig:motivation_ex}
\end{figure}


\section{Introduction}
% 원준 원본
% Along with the advance of digital devices and platforms, video is now one of the most desired data type for consumers. However, although the large information capacity of videos may be beneficial in many aspects, e.g., informative and entertaining, on the contrary perspective, videos are time-consuming, and hard to search for desirable moments. 
% This has led many creators to use extra manpower to crop and edit the video to generate highlight clips to gain the consumer’s attention.
Along with the advance of digital devices and platforms, video is now one of the most desired data types for consumers~\cite{apostolidis2021video,wu2017deep}.
% SE: Video aware deep learning application & survey papers?
Although the large information capacity of videos might be beneficial in many aspects, e.g., informative and entertaining, inspecting the videos is time-consuming, so that it is hard to capture the desired moments~\cite{anne2017localizing,apostolidis2021video}. 
% This has led many creators to use extra manpower to crop and edit the video to generate highlight clips to gain the consumer’s attention.


% On the other side, 
Indeed, the need to retrieve user-requested or highlight moments within videos is greatly raised.
Numerous research efforts were put into the search for the requested moments in the video~\cite{anne2017localizing, gao2017tall, liu2015multi, escorcia2019temporal} and summarizing the video highlights~\cite{zhang2016video, mahasseni2017unsupervised, badamdorj2022contrastive, wei2022learning}.
% Numerous research efforts were put into the search for the requested moments in the video~\cite{anne2017localizing, gao2017tall, liu2015multi, escorcia2019temporal}, summarizing the video to generate highlights was another popular topic~\cite{zhang2016video, mahasseni2017unsupervised, badamdorj2022contrastive, wei2022learning}.
Recently, Moment-DETR~\cite{momentdetr} further spotlighted the topic by proposing a QVHighlights dataset that enables the model to perform both tasks, retrieving the moments with their highlight-ness, simultaneously.

% 원준 원본
% To detect the desired moments, previous works employed transformer encoder-decoder architectural designs to fuse the text query into the video representations. Moment-DETR~\cite{mDETR} modified detection transformer to process capture the moment as a set, and UMT~\cite{umt} implemented transformer decoder as to output clip-wise saliency. 
% Yet to their outstanding breakthroughs in the literature of moment retrieval with the seminal architectures, their limitation is that the role of the given text query is insignificant in representing the query-conditioned video representation; the attention mechanism of moment DETR is not explicitly conditioned on the text query, and the text query is conditioned on multi-modal clips where the differences between the clips are smoothed after encoding process in UMT.



% \begin{figure}[t]
% \centering
%     \begin{subfigure}[l]{0.37\linewidth}
%       \centering
%       \vspace{0.20cm}
%     %   \includegraphics[width=1\linewidth]{figs/fig_1_moti_textattn.pdf}  
%     %   \includegraphics[width=1\linewidth]{figs/fig_1_moti_textattn_v2.pdf}  
%       \includegraphics[width=1\linewidth]{figs/fig1_moti_violin_a.pdf}  
%       \vspace{-0.60cm}
%     %   \caption{text attention}
%         \caption{Importance of queries in video representation}
%       \label{fig:fig1_text_attn}
%     \end{subfigure}%\hfill% or  or \hspace{0.3\textwidth}
%     \vspace{0.2cm}
%     \begin{subfigure}[r]{0.61\linewidth}
%       \centering
%     %   \includegraphics[width=1\linewidth]{figs/fig1_moti_negattn.pdf}  
%       \includegraphics[width=1\linewidth]{figs/fig1_moti_violin_b.pdf}  
%     %   \caption{neg attention}
%         % \caption{Relation between the highlight-ness and the relevance between videos and query texts.}
%         \caption{Highlight-ness~(saliency) histogram of positive and negative video-query pairs\SE{}}
%       \label{fig:fig1_neg_attn}
%     \end{subfigure}%\hfill% or  or \hspace{0.3\textwidth}
%     % \vspace{-0.2cm}
%     \caption{Overall statistics for attention scores in Fig.~\ref{fig:motivation_ex} in QVHighlights dataset. 
%     (a) For the attention scores that measure how much the text query is generally involved in video representation, we use violin plots to show the probability density. We plot the score for each layer in the encoder.
%     % (b) Using the histogram, we compare how the baseline and QD-DETR yield different salient scores given the positive and negative video-text pairs.
%     (b) Saliency histogram shows the distributional gap between positive and negative video-text query pairs of baseline~(Moment-DETR) and proposed QD-DETR.\SE{}
%     }
%     \label{fig:motivation}
%     % \captionsetup{belowskip=13pt}
%     % \setlength{\belowcaptionskip}{-10pt}
% \end{figure}

% \begin{figure}[t]
% \centering

%     \begin{subfigure}[r]{1\linewidth}
%       \centering
%       \hspace{-0.2cm}
%     %   \includegraphics[width=1\linewidth]{figs/fig1_moti_negattn.pdf}  
%       \includegraphics[width=1.1\linewidth]{figs/fig1_moti_violin_a_v2.pdf}  
%     %   \caption{neg attention}
%         % \caption{Relation between the highlight-ness and the relevance between videos and query texts.}
%         \vspace{-0.5cm}
%         % \caption{Saliency histogram of positive and negative video-query pairs}
%         \caption{We plot the histograms and its average value~(dotted line) to compare saliency scores when true and false text queries are given for each method. (left) Since the video representations do not include much textual information, both the true and false queries yield similar saliency scores. (Middle) Even when the video representation is enforced to be updated with the textual information, the issue is not much resolved. (Right) By extracting discriminative features in the text query, distributions are differentiated.
%         % \SE{} % R1@0.5 설명
%         Also, R1@0.5 indicates evaluation metric, Recall at 1 with IoU 0.5 threshold on QVhighlight \textit{val} set.
%         }
%       \label{fig:fig1_neg_attn}
%     \end{subfigure}%\hfill% or  or \hspace{0.3\textwidth}
%     \\
%     \begin{tabular}{cc}
%     \hspace{-0.2cm}
%         \begin{minipage}{.4\linewidth}
%             \begin{subfigure}[l]{1\linewidth}
%               \centering
%             %   \vspace{0.20cm}
%             %   \includegraphics[width=1\linewidth]{figs/fig_1_moti_textattn.pdf}  
%             %   \includegraphics[width=1\linewidth]{figs/fig_1_moti_textattn_v2.pdf}  
%               \includegraphics[width=1\linewidth]{figs/fig1_moti_violin_a.pdf}  
%               \vspace{-0.60cm}
%             %   \caption{text attention}
%                 \caption{Importance of queries in video representation}
%               \label{fig:fig1_text_attn}
%             \end{subfigure}%\hfill% or  or \hspace{0.3\textwidth}
%         \end{minipage}
        
%         \begin{minipage}{.6\linewidth}
%             \vspace{-0.2cm}
%             \caption{Overall statistics of Fig.~\ref{fig:motivation_ex} in QVHighlights dataset. 
%             (a) Saliency histogram shows the distributional gap between positive and negative video-text query pairs.
%             % (a) For the attention scores that measure how much the text query is generally involved in video representation, we use violin plots to show the probability density. We plot the score for each layer in the encoder.
%             % (b) Using the histogram, we compare how the baseline and QD-DETR yield different salient scores given the positive and negative video-text pairs.
%             % (b) Text ratio in self-attention layer to  of Moment-DETR
%             % (b) Ratio of text when representing video tokens in self-attention of Moment-DETR.
%             % (b) Magnitude of attention text query involved.
%             % (b) Attention score of video tokens
%             % (b) Magnitude of text query to refine the video tokens in self-attention layer of Moment-DETR.
%             (b) Probability density depicting the weight of the text query in attention score for video clips. Scores are from the self-attention layers in Moment-DETR encoder.
%             % (b) The text query ratio in attention score of video clips (Self-attention layer in Moment-DETR encoder). We use violin plots to show probability density.
%             % 텍스트 쿼리가, 비디오 피쳐에 얼만큼 attend 하는지
%             }
%         \end{minipage}
    
%     \end{tabular}
%     \vspace{-0.5cm}
%     \label{fig:moti}
%     % \captionsetup{belowskip=13pt}
%     % \setlength{\belowcaptionskip}{-10pt}
% \end{figure}


% \begin{figure}
%     \centering
%     % \includegraphics[width=1\linewidth]{figs/fig1_moti_negattn_1109.pdf}
%     \includegraphics[width=1\linewidth]{figs/fig1_moti_negattn_stat_v2.pdf}
%     \vspace{-0.8cm}
%     \caption{
%         Histogram of saliency when the positive and negative queries are given. We plot the histograms and its average value~(dotted line) to compare saliency scores when relevant~(positive) and irrelevant~(negative) text queries are given for each method. (Left) Since the video representations do not properly reflect textual information, both the positive and negative queries yield similar saliency scores. 
%         % (Middle) Even when the video representation is enforced to be updated with the textual information, the issue is not much resolved. 
%         (Right) By representing video clips in query-dependent manner, distributions are differentiated.
%     }
%     \vspace{-0.6cm}
%     \label{fig:motivation}
% \end{figure}


% One of the demanding task is moment retrieval task, which is detecting the desired moments from the given query, typically the text query.
When describing the moment, one of the most favored types of query is the natural language sentence~(text)\cite{anne2017localizing}. 
While early methods utilized convolution networks~\cite{zhang2020learning, gao2021fast, wang2020temporally}, recent approaches have shown that deploying the attention mechanism of transformer architecture is more effective to fuse the text query into the video representation.
% To handle these modalities, previous works simply employed the attention mechanism of transformer architecture to fuse the text query into the video representation.
For example, Moment-DETR~\cite{momentdetr} introduced the transformer architecture which processes both text and video tokens as input by modifying the detection transformer~(DETR), and UMT~\cite{umt} proposed transformer architectures to take multi-modal sources, e.g., video and audio. 
Also, they utilized the text queries in the transformer decoder.
Although they brought breakthroughs in the field of MR/HD with seminal architectures, they overlooked the role of the text query.
To validate our claim, we investigate the Moment-DETR~\cite{momentdetr} in terms of the impact of text query in MR/HD~(Fig.\ref{fig:motivation_ex}).
Given the video clips with a relevant positive query and an irrelevant negative query, we observe that the baseline often neglects the given text query when estimating the query-relevance scores, i.e., saliency scores, for each video clip.
% the output saliency score, i.e. query-relevance scores.
% Based on the observation, we traced the actual saliency prediction of the model against both the video-relevant query and the irrelevant dummy one where we find that the baseline often neglects the given text query when estimating the query-relevance scores of video clips.
% For example, in Fig.~\ref{fig:motivation_ex}, saliency scores are not affected even when the query is substituted with the dummy.
% % General statistics for Fig.~\ref{fig:motivation_ex} is shown in Fig.~\ref{fig:motivation}. 
% General statistics corresponding to Fig.~\ref{fig:motivation_ex} are also shown in Fig.~\ref{fig:motivation}.



% The limitation of the concrete baseline~\cite{momentdetr} is inspected in two different aspects; 1) Utilization of text-query in the encoding process and 2) the output saliency score, i.e. query-relevance scores.
% Firstly, we visualize the attention score when video clips are given as a query in self-attention. 
% We observe that the text queries have relatively small impacts compared to other video features, as shown in Fig.~\ref{fig:fig1_text_attn_ex}.
% That is, the text does not account for much in representing every video clip, although the goal of MR/HD is to detect query-relevant moments.
% Based on the observation, we traced the actual saliency prediction of the model against both the video-relevant query and the irrelevant dummy one where we find that the baseline often neglects the given text query when estimating the query-relevance scores of video clips.
% For example, in Fig.~\ref{fig:motivation_ex}, saliency scores are not affected even when the query is substituted with the dummy.
% % General statistics for Fig.~\ref{fig:motivation_ex} is shown in Fig.~\ref{fig:motivation}. 
% General statistics are also shown in Fig.~\ref{fig:motivation}.

% Consequently, in Fig.~\ref{fig:fig1_neg_attn_ex}~(b), we found that the baseline often neglects the given text query when estimating the query-relevance scores of video clips; 
% For example, 


% We validate the previous work sometimes neglects the given query when estimating the saliency of video clips.
% For example, there is an example that the saliency scores from positive and negative queries cannot be distinguishable, as shown in Fig.~\ref{fig:fig1_neg_attn_ex}.
% % 우리는 추가로 text attention을 추가도 해봤지만, 효과가 있긴 했으나, still 이슈가 있는 것을 확인하였다?
% % Still, we observe that assuring the high attendance of text queries does not resolve the overlap which motivates us to question the quality of the naive use of task-agnostic text representation~\cite{momentdetr, umt}.
% We found that introducing the text-attention for ensuring the high attendance of text queries relieve the overlap, but there still be a severe overlap.


% To validate their limitations, we inspect the impacts of text queries in the concrete baseline~\cite{momentdetr} with the two different aspects, 1) tendency of attention in self-attention layer and 2) saliency score, i.e. query-relevance scores. \SE{} % attention 이 갑자기 등장하는가?
% Firstly, we visualize the attention score when video clips are given as a query in self-attention. We observe the text queries have relatively low attention scores compared to the video features, as shown in Fig.~\ref{fig:fig1_text_attn_ex}.
% That is, the text does not account for much in representing every video clip, although the goal of MR/HD is to detect query-relevant moments.
% Based on this observation, we trace the actual saliency prediction of the model against both positive and negative text queries.
% We validate the previous work sometimes neglects the given query when estimating the saliency of video clips.
% For example, there is an example that the saliency scores from positive and negative queries cannot be distinguishable, as shown in Fig.~\ref{fig:fig1_neg_attn_ex}.
% % 우리는 추가로 text attention을 추가도 해봤지만, 효과가 있긴 했으나, still 이슈가 있는 것을 확인하였다?
% % Still, we observe that assuring the high attendance of text queries does not resolve the overlap which motivates us to question the quality of the naive use of task-agnostic text representation~\cite{momentdetr, umt}.
% We found that introducing the text-attention for ensuring the high attendance of text queries relieve the overlap, but there still be a severe overlap.



% Thus, we 
% query dependency를 높이기 위해 
% Cross-attention? text-attention? detailed explanation on text-attention should be needed?
% By handling these two issues, we find that more precise retrieval can be achieved.
% 
% 
%
% By projecting video-discriminative text features with high text attendance to source video, we f 
% We also find the need to improve the quality of query features since assuring high text attendance also results in...
% pairs are not finetuned to be discriminative that even the similarity within the pairs does not reflect the relevance between the query and the video clips.
% General statistics for Fig.~\ref{fig:motivation_ex} is shown in Fig.~\ref{fig:motivation}. 
% \SE{} % 이거 ??로 뜨는데, 위처럼 figure 그리면 label이 안되는걸까요
% \SE{}
% 형님 아래 사항 생각 좀 해보는게 좋을 거 같아요.
% fig 1. (a) 그림만 봤을 때 모든 clip에 대해 text attention이 일정이상 존재하긴 하니까, 뭔가 not assured to be conditioned가 와닿지 않는거 같아요.
% + 왜 text가 항상 attend 해야하나?
% not assured to be conditioned --> text shows relatively low affects compared to video 같이 실제 나타난 현상까지 같이 적으면 어떨까 싶어요.
% fig 1. (b) 덜 반영한다?

% \SU{}
% 일단 text가 attend 잘 되어야 한다는 것에 좀 궁금점이 생깁니다. 결국에는 text와 관련있는 frame들을 attend해서 higlight를 찾아야 하는게 아닐까요? 그리고, 현제 저희의 모델 구조상 text query가 Key와 Value로 거의 활용되고 있는데 그렇다면 결국에는 해당 모델은 text에 대한 attention이 전혀 없다고 봐도 무방하지 않을까요? 그런 면에서 text attention을 강조하는게 좀 걸리긴 합니다.

% Specifically, the text query is not assured to be explicitly conditioned on every clip of the video, and as the query texts are evenly treated, discriminative keywords may not be spotlighted.
% attention mechanism of Moment-DETR is not explicitly conditioned on the text query as shown in Fig~\ref{}(d), and in UMT, the text are only used for conditioning the queries while the video representation are refined itself by self-attention.

% \begin{figure}[t]
%     \begin{subfigure}{1\linewidth}
%       \centering
%     %   \includegraphics[width=1\linewidth]{figs/fig_1_moti_textattn.pdf}  
%     %   \includegraphics[width=1\linewidth]{figs/fig_1_moti_textattn_v2.pdf}  
%       \includegraphics[width=1\linewidth]{figs/fig_1_moti_textattn_v4.pdf}  
%       \vspace{-0.5cm}
%     %   \caption{text attention}
%         \caption{Distribution of attention scores in Moment-DETR encoder}
%       \label{fig:fig1_text_attn}
%     \end{subfigure}%\hfill% or  or \hspace{0.3\textwidth}
%     \vspace{0.2cm}
%     \begin{subfigure}{1\linewidth}
%       \centering
%     %   \includegraphics[width=1\linewidth]{figs/fig1_moti_negattn.pdf}  
%       \includegraphics[width=1\linewidth]{figs/fig1_moti_negattn_v2.pdf}  
%       \vspace{-0.5cm}
%     %   \caption{neg attention}
%         \caption{Saliency score against positive and negative text queries}
%       \label{fig:fig1_neg_attn}
%     \end{subfigure}%\hfill% or  or \hspace{0.3\textwidth}
%     \vspace{0.2cm}
%     \begin{subfigure}{1\linewidth}
%       \centering
%     %   \includegraphics[width=1\linewidth]{figs/fig1_moti_violin.pdf}  
%       \includegraphics[width=1\linewidth]{figs/fig1_moti_violin_v2.pdf}  
%       \vspace{-0.5cm}
%       \caption{violin}
%       \label{fig:fig1_violin}
%     \end{subfigure}%\hfill% or  or \hspace{0.3\textwidth}
%     \vspace{-0.2cm}
%     \caption{(a) 1. portion of text attention vs. video attention 2. relation with text query and content (e.g. fg, bg) of clip seems not to affect the attention score
%     (b) 1. high variability even though entire clips are highly correlated with the given text query 2. positive and negative query makes overlaps on saliency score distribution
%     (3) actual distribution on validation dataset.}
%     \label{fig:motivation}
%     % \captionsetup{belowskip=13pt}
%     % \setlength{\belowcaptionskip}{-10pt}
% \end{figure}

To this end, we propose Query-Dependent DETR~(QD-DETR) that produces query-dependent video representation.
% Our key focus is to ensure each clip in predicted moments is explicitly conditioned by the query, particularly on the video-descriptive portion of the text query.
% Our key focus is to ensure that query-relevant clips are predicted by enforcing each clip to be explicitly conditioned by the query.
%Our key focus is to ensure that the model prediction for each clip is highly relevant to the query.
Our key focus is to ensure that the model's prediction for each clip is highly dependent on the query.
% by enforcing each clip to be explicitly conditioned by the query. :)
% hmm...
% \SE {} % "query-relevant clips are predicted" 이 문장이 좀 애매한거 같습니다. relevant 클립을 놓지지 않고 찾는 것을 보장한다? 이런 느낌인지 아니면 높은 saliency 를 주는게 목적이다? model prediction이 query-relevance를 반영하는 것을 보장한다?
% Our key focus is to ensure that the model prediction reflects query-relevance of clips by enforcing each clip to be explicitly conditioned by the query.
First, to fully utilize the contextual information in the query, we revise the transformer encoder to be equipped with cross-attention layers at the very first layers.
% 상익's thought :  single video - query간의 관계만 고려 - 같은 word가 더 많이 쓰이는 것을 보고 
% 교수님's thought : neg pair 를 쓰면 쿼리를 보지 않고서는 video clip간만 고려하는 것이 사라짐. 왜냐면 0으로 내보내야 하기 때문. --> SE: relative difference 만 고려하다가, 
By inserting a video as the query and a text as the key and value of the cross-attention layers, our encoder enforces the engagement of the text query in extracting video representation.
% 원준 교수님 코멘트 반영해서 다시
Then, in order to not only inject a lot of textual information into the video feature but also make it fully exploited, we leverage the negative video-query pairs generated by mixing the original pairs.
Specifically, the model is learned to suppress the saliency scores of such  negative~(irrelevant) pairs.
Our expectation is the increased contribution of the text query in prediction since the videos will be sometimes required to yield high saliency scores and sometimes low ones depending on whether the text query is relevant or not.
% \SE{}
% learns to?
% By suppressing the saliency scores of the irrelevant video-query pairs, the model learns to spotlight only the video-specific discriminative words in the query.
% % \SE{} % ====================== 상익 수정 ========================
% However, this architectural design still lacks the capability of identifying the video-descriptive keywords in the query.
% % However, this architectural design still lacks in identifying proper query relevance.
% This is because the current training scheme only focuses on the interactions of video and clips within a single video while neglecting information shared throughout the entire video.
% % We argue the problem of the current training scheme that only focuses on distinguishing the clips in a single video while neglecting information shared throughout the entire video.
% Therefore, we leverage the negative video-query relationships to enhance the capability of identifying the contextual similarity of query and video clips.
% 
% 원준 원본 
% However, this architectural design heavily relies on the quality of the text query.
% Therefore, we leverage the negative video-query relationships to enable the model to emphasize key corresponding query features.
% By suppressing the saliency scores of the irrelevant video-query pairs, the model learns to spotlight only the video-specific discriminative words in the query.
% =========================================================
Lastly, to apply the dynamic criterion to mark highlights for each instance, we deploy a saliency token to represent the entire video and utilize it as an input-adaptive saliency criterion. 
With all components combined, our QD-DETR produces query-dependent video representation by integrating source and query modalities.
This further allows the use of positional queries~\cite{dabdetr} in the transformer decoder.
% Furthermore, we can exploit the advanced DETR decoder architectures using the positional information, e.g., DAB-DETR, since our encoded tokens consist of identical position representations from a single modality.
% \SE{} % ====================== 상익 수정 ========================
% Furthermore, we can exploit the advanced DETR decoder architectures using the positional information, e.g., DAB-DETR, since our video clip tokens consist of identical position representations from a single modality.
% 원준 원본
% It also enables the use of advanced DETR decoder architectures, e.g., DAB-DETR, for the first time, as these works exploit the position information within a single modality.
% =========================================================
Overall, our superior performances over the existing approaches validate the significance of the role of text query for MR/HD.
% Our extensive experiments on QVHighlights, TVSum, and Charades-STA datasets validate the significance of considering the role and the quality of text query.

% All components combined with dynamic anchor moments for the query of decoder, our FOQUE fosters the query-dependent video representation, thereby making the 
% All components combined, our modified transformer encoding process fosters the query-dependent video representation thereby achieving the state-of-the-art results on various benchmarks of moment-retrieval and highlight detection.
	
% -	Video Platform & Streamer & Consumer의 증가. 
% Video는 다른 데이터 타입보다 정보가 많아 유용하지만, 이는 다른 말로 해석하면 video를 보는 것은 time-consuming 하고, 원하는 것을 찾아보기에는 힘들 수 있음.
% 따라서, 많은 매체에서는 사람들의 더 많은 이목을 끌기 위해 highlight 비디오라는 것을 편집하여 공유도 함.
% 하지만, highlight video를 만들기 위해 사람의 노력이 필요한 현 시점에서, This spotlights the need to retrieve the user-requested / Highlight moments in the video.

% -	이전에도 이러한 문제를 해결하기 위해 (asdfasdf) for moment retrieval, (asdfasdf) for highlight detection 등이 제안 되었지만, 이들은 비디오의 특정 영역을 찾는다는 공통된 목적을 가지고 있으면서도, 데이터 셋의 한계로 인해 따로 연구되었음. 이를 문제 삼으며, 최근에는 두 task를 동시에 학습할 수 있는 dataset이 소개 되었는데, 컴퓨터비전에서 최근 각광을 받고 있는 Transformer 모델 도입과 함께 큰 발전을 거듭하고 있음.

% -	구체적으로, 이 두가지 task를 수행하기 위해서는 transformer를 두가지 방법으로 이용할 수 있는데, moment-DETR 처럼 moment 를 clip의 set 단위로 예측할 수 있고, UMT 처럼 clip-wise prediction을 할 수 있음. 하지만, 이들은 query를 condition이 아닌 video와 동등한 레벨로 취급하거나 [mDETR], 매 클립이 self-attention으로 mixing 된 후에 condition을 걸어주어 clip간의 차이를 확실하지 이용하지 못하였고, 또한, 확실하게 condition으로 주지 못하였고, video와 query 사이의 관계를 한정적으로만 이용하였다.

% -	따라서, we explore three different ways to fully exploit query information. First, we design one-way cross-attention layer to condition every clip with the query features. Then, we utilized the negative video-text pairs to better model the relationships between the video and the text embeddings. Lastly, we define the saliency token to be the video-query dependent saliency estimator.


















% ===================== neg pair 부분 ===========================
% Nevertheless, the current training scheme, only considering the given video-query pair, still disturbs the model from identifying proper query-relevance prediction.
% In detail, the model focus on learning the fine-grained discrepancy between video clips, while neglecting the information they share, which contains significant clues to understand the context of video.
% Therefore, we leverage the negative video-query relationships to enhance the capability of identifying the contextual similarity of query and video clips.
% Therefore, we leverage the negative video-query relationships by suppressing those pairs, so that enhance the capability of identifying the contextual similarity of query and video clips.
% We hypothsize the diversity in query-video pairs are insufficient to learn the general relationship between text query and video.
% Therefore, we leverage the negative video-query relationships by suppressing the saliency scores of the irrelevant video-query pairs.
% However, this architectural design still lacks in identifying proper query relevance.
% We argue that the current training scheme only focuses on learning the fine-grained discrepancy between clips in a single video, while neglecting the information they share, which contains significant clues to understand the context of the video.
% Therefore, we leverage the negative video-query relationships to enhance the capability of identifying the contextual similarity of query and video clips.
% However, this architectural design still lacks in identifying proper query relevance.
% We argue the problem of the current training scheme that only focuses on learning the fine-grained discrepancy between clips in a single video.
% That is, the current design neglects the information shared throughout the video, although it contains significant clues to understand the context of the video.
\section{Related Work}
\label{sec:related_work}
\subsection{Co-Speech Gesture Synthesis}
The early approaches for generating co-speech gestures often involve creating linguistic rules to translate speech input into a sequence of pre-collected gesture segments, which are typically referred to as rule-based methods \cite{cassell1994rulefullbody,cassell2001beat,kipp2004gesture,kopp2006bml}. \citet{wagner2014rulereview} provide a comprehensive review of these methods. Rule-based methods produce interpretable and controllable results, but creating gesture datasets and rules requires significant effort. To alleviate the manual effort of designing rules in rule-based methods, data-driven approaches have gradually become predominant in this field. \citet{nyatsanga2023data_driven_gesture_survey} offer a thorough survey of these methods. Early data-driven approaches aim to directly learn mapping rules from data through statistical models \cite{neff2008videogesture,levine2009prosodygesture,levine2010gesturecontroller} and combine them with predefined gesture units for gesture generation. Later, the powerful modeling capability of deep neural networks makes it possible to train complex end-to-end models using raw speech-gesture data directly. One option is deterministic models, such as MLP \cite{kucherenko2020gesticulator}, CNN \cite{habibie2021videogesture}, RNN \cite{yoon2019robot,yoon2020trimodalgesture,bhattacharya2021affectivegesture,liu2022hierarchicalgesture}, and Transformer \cite{bhattacharya2021text2gestures}. Another choice is generative models, including flow-based models \cite{alexanderson2020stylegesture,ye2022styleflowgesture}, VAEs \cite{li2021audio2gesture,ghorbani2022zeroeggs}, and VQ-VAE \cite{yi2022talkshow,yazdian2022gesture2vec,liu2022vqgesturevideo}. Due to the inherent many-to-many relationship between speech and gesture, end-to-end models can generate natural-looking gestures but face challenges in ensuring content matching between speech and generated gestures \cite{yoon2022genea}. To address this issue, some neural systems aim to explicitly model both rhythm and semantics from the perspective of model structure \cite{kucherenko2021speech2properties2gestures,ao2022rhythmicgesticulator,liu2022disco} or training supervision strategy \cite{liang2022seeg}. Furthermore, hybrid systems, such as the combination of deep features and motion graphs \cite{zhou2022gesturemaster}, have been proposed to harness the advantages of different approaches. Recently, diffusion models \cite{sohldickstein2015diffusion,song2020improvedscore,ho2020ddpm} have demonstrated impressive results in image synthesis \cite{ramesh2022dalle2} and human motion generation \cite{tevet2022humanmotiondiffusion, zhang2022motiondiffuse}. Inspired by these works, our system adapts the latent diffusion model \cite{rombach2022latentdiffusion} for the co-speech gesture generation task and achieves appealing results.

\subsection{Style Control for Human Motion}
A typical approach to style control for human motion involves specifying a motion clip as a reference and transferring the reference clip's style to the source motion. This task is also known as \emph{style transfer}. Early works in motion style transfer integrate traditional machine learning techniques with manually defined features to infer motion styles \cite{hsu2005motion_style_translation,ma2010motion_style_transfer,xia2015realtime_motion_style_transfer,yumer2016spectral_motion_style_transfer}. Recently, deep learning-based methods have significantly enhanced motion quality. \citet{holden2016deepmotion} first propose a learning framework enabling motion style control through optimization in the motion manifold space. \citet{du2019stylemotioncvae} improve transfer efficiency by training a conditional VAE. \citet{mason2018few-shot_motion_style_transfer} use few-shot learning to generate stylized locomotion. \citet{aberman2020adain} employ a temporally invariant adaptive instance normalization (AdaIN) layer for target style injection, eliminating the need for paired data during training. \citet{wen2021stylemotionflow} achieve unsupervised style transfer using a flow model. \citet{jang2022motionpuzzle} introduce a method capable of controlling styles for individual body parts.

Previous co-speech gesture synthesis systems with style control can be categorized based on whether or not they require style labels. For methods needing labeled data, early works can only learn an individual style for one generator \cite{levine2010gesturecontroller,neff2008videogesture,ginosar2019stylegesture}. \citet{ahuja2022lowresource} propose a strategy that efficiently adapts the source generator to another speaker style using low-resource data. Some works learn a speaker style embedding space with labeled speaker-motion data, enabling gesture style control by sampling from this space \cite{ahuja2020stylegesture,yoon2020trimodalgesture,bhattacharya2021affectivegesture}. \citet{alexanderson2020stylegesture} aimat controlling fine-grained styles, such as gesturing speed and spatial scope, using preprocessed control signal-motion data. Their later work \cite{alexanderson2022diffusiongesture} utilizes a diffusion model for audio-driven motion synthesis, achieving label-based style control by training the model on labeled data. For methods not requiring style labels, \citet{habibie2022motionmatching} propose a motion matching framework to achieve flexible style control. Other studies achieve arbitrary style control by imitating an example given as a video \cite{liu2022hierarchicalgesture} or a motion clip \cite{ghorbani2022zeroeggs,ye2022styleflowgesture,kuriyama2022tokenizedgestures}.  In this work, we utilize a CLIP-based encoder to extract a style embedding from an arbitrary text prompt and incorporate it into the generator via an AdaIN layer, guiding the synthesis of stylized gestures. Our system supports fine-grained multimodal style prompts as opposed to label-based style control. It employs a self-supervised learning scheme and eliminates the need for labeled data. Additionally, we use an autoregressive model rather than a parallel model, making it potentially suitable for real-time applications.
% \section{Method}
\label{sec: method}
% This section introduces the rendering pipeline of our proposed hierarchical compositional scene. 
% our pipeline consists of three processes, including decomposing the text into editable 3D layout, rendering the compositional views with local (object) NeRFs and global (scene) NeRF and the joint optimization on these hierarchical 3D representations.

% Note that the transformation between the object and the scene frame is defined by ${p}_o$ and ${D}_o$. 
%
% Next, we build a residual connection to add ${\sigma}_o$ and the referenced global color, and the rendering result will be used to calculate the SDS loss based on the global text.  
% Fig.~\ref{fig:framework} illustrates our pipeline, which consists of three main components, including the editable 3D scene layout based on multi-object text (Sec.~\ref{ssec:layout}), the scene rendering pipeline that composites the predictions from all local NeRFs (Sec.~\ref{ssec:render}), and the joint optimization on both local and global representation models (Sec.~\ref{sec:optimization}).
% To elaborate, our editable 3D scene layout represents a global frame of the scene by decomposing it into a set of local frames, where each is parameterized by a local NeRF, a 3D bounding box, and a corresponding local text prompt.
% For instance, the text prompt `A teddy bear and a stuffed monkey sit side by side' is interpreted as a 3D scene layout, as shown in Fig.~\ref{fig:framework}.  
% The whole 3D layout, \ie, scene frame, consists of two 3D bounding boxes, \ie local frames \#1 and \#2, with specific local text prompts, \ie, `a teddy bear' and `a stuffed monkey'. 
% %
% To render the scene view, we first calculate the ray-box intersections between the boxes and rays $({\boldsymbol{r}}_o, \boldsymbol{\phi}_d, {\boldsymbol{\theta}}_d)$, where the ${\boldsymbol{r}}_o$ is the ray origin and the $({\boldsymbol{r}}_o, \boldsymbol{\phi}_d)$ is its direction.
% Then, to infer each object's properties in local NeRFs, we sample the global points $({\boldsymbol{x}}_g, {\boldsymbol{y}}_g, {\boldsymbol{z}}_g)$ in the global frame within the ray-box intersection intervals and project them into the normalized local location $({\boldsymbol{x}}_l, {\boldsymbol{y}}_l, {\boldsymbol{z}}_l)$ in the local frame.
% %
% Given the local sampling points $({\boldsymbol{x}}_l, {\boldsymbol{y}}_l, {\boldsymbol{z}}_l)$, the implicit local NeRF ${\boldsymbol{\theta}}_l$ outputs four pseudo-color channels ${\boldsymbol{C}}_l$ and density $\boldsymbol{\sigma}$, which can be used to render a local view of the local frame to match its local text prompt.
% %
% We further calibrate the predicted pseudo-color $\boldsymbol{C}_l$ from local frames by adding the global embeddings ${\boldsymbol{emb}}_g$ to improve the global view consistency.
% Then, the calibrated predictions after composition are used to reconstruct the scene view by volumetric rendering along the rays.
% %
% Lastly, the rendered views based on local and global frames are guided by score distillation sampling loss $\nabla \mathcal{L}_{\text{SDS}}$~\cite{poole2022dreamfusion} to optimize all the learnable parameters. 
To resolve the issue of guidance collapse, our principal strategy is to \textit{decompose the scene into reusable components and compose/recompose them into a unified and consistent one}.
This enables flexible control over the generated content with direct use of prompts and box layouts, as illustrated in \cref{fig:teaser}.
%
Our proposed CompoNeRF confers several key benefits:
1) \textbf{Semantic Coherence}: It reliably creates 3D objects with detailed textures and global consistency, exemplified by authentic light interactions, such as reflections on the bed surface.
2) \textbf{Modularity and Reusability}: CompoNeRF functions as an ensemble of independently trained NeRF models. These can be efficiently stored and later retrieved from a cached dataset, enabling their reuse in various cases.
3) \textbf{Editability}: Our approach allows for flexible scene modification, such as interchanging the lamp for a vase filled with sunflowers or altering its scale, by simply adjusting the box dimensions for later finetuning. This feature enhances flexibility and creative possibilities. 


% Furthermore, the usage of layout boxes enables more flexible control over the generated content compared with the intricate sketch shape in Latent-NeRF\cite{metzer2022latent}. 
\begin{figure*}[t]
    \centering
    \includegraphics[width=0.9\linewidth]{figures/method.pdf}
    % \vspace{-12pt}
    \caption{\textbf{Framework Overview}.
The CompoNeRF model unfolds in three stages: 1) Editing 3D scene, which initiates the process by structuring the scene with 3D boxes and textual prompts; 2) Scene rendering, which encapsulates the composition/recomposition process, facilitating the transformation of NeRFs to a global frame, ensuring cohesive scene construction. Here, we specify design choices between density-based or color-based(without refining density) composition; 3) Joint Optimization, which leverages textual directives to amplify the rendering quality of both global and local views, while also integrating revised text prompts and NeRFs for refined scene depiction.
  % The model is structured into three components: Composition, Decomposition, and Recomposition. Composition deals with the foundational setup, detailed with choices for density-based and color-based composition. Decomposition utilizes the modularity of the CompoNeRF feature, caching each NeRF module offline for efficient recalibration. Recomposition reuses these cached NeRFs and adjusts the semantic context, providing a revised output with the inclusion of the offline NeRF enhancements.
    % Our model consists of two branches where the upper part is individual NeRFs, and the lower part denotes global calibration with our tailored composition model. The specific designs for density-based and color-based composition modules are highlighted. 
    % CompoNeRF consists of three parts: 1). The editable 3D scene layout configures the scene representations with 3D boxes and text prompts; 2).  The scene rendering includes the global calibration and the compositional process; 3). The joint optimization applies global and local text guidance on global and local render views.
    % The global frame (scene space) contains a set of local frames. Each is  represented by a local NeRF associated with a 3D box and text prompt defined by the editable 3D layout.
    % The scene view is volumetric rendered by sampling the points $({\boldsymbol{x}}_g, \boldsymbol{y}_g, \boldsymbol{z}_g)$ intersected with any local frame along the ray $(\boldsymbol{r}_o, {\boldsymbol{\phi}}_d, \boldsymbol{\theta}_d)$.
    % The sampling points are first inferred through the local NeRF with the local frame locations $({\boldsymbol{x}}_l, \boldsymbol{y}_l, \boldsymbol{z}_l)$ projected from the global location $({\boldsymbol{x}}_g, \boldsymbol{y}_g, \boldsymbol{z}_g)$.
    % And then, all the local predictions are calibrated by a global MLP with conditional input to render the scene view.
    % During the optimization, the text guidance is applied to both local views predicted by local frames only and global views predicted by the composition of all local frame predictions.
    }
    \label{fig:framework}
    % \vspace{-8pt}
\end{figure*}

\subsection{Preliminaries}
Defining individual object bounding boxes as \textit{local frames} and the overall scene coordinate system as the \textit{global frame}, we build the foundation of NeRF and diffusion processes.

\label{sec:background}
\noindent \textbf{3D Representation in Latent Space.}
Our methodology capitalizes on the state-of-the-art text-to-image generative model—Stable Diffusion as described by Rombach et al\cite{rombach2022high}.
We build upon the Latent-NeRF framework~\cite{metzer2022latent}, which computes latent colors for individual objects by considering their sample positions within a localized frame. Specifically, it maps a three-dimensional point in local coordinates \(\boldsymbol{x}_l = (x_l, y_l, z_l)\) to a volumetric density \(\boldsymbol{\sigma}_l\) and an associated color \(\boldsymbol{C}_l\), expressed as \((\boldsymbol{C}_l, \boldsymbol{\sigma}_l) = f_{\boldsymbol{\theta}_l}(x_l, y_l, z_l)\). Here, \(f\) represents a Multi-Layer Perceptron (MLP) characterized by parameters \(\boldsymbol{\theta}_l\).
 This NeRF-generated color is then assessed in the context of the Stable Diffusion model, using text prompts to guide NeRF toward spatially coherent inference with intricate context.
% to infer pseudo-color for each object using local NeRF.
% Specifically, the representation maps a point $\boldsymbol{x}_l = \left({x}_l, {y}_l, {z}_l\right)\in [-1, 1]$ in the local frame to its corresponding volumetric density $\boldsymbol{\sigma}_l$ and emitted color $\boldsymbol{C}_l$, \ie,  $\left(\boldsymbol{C}_l, {\boldsymbol{\sigma}_l}\right)=\boldsymbol{\theta}_{_l}\left({x_l}, {y}_l, {z}_l\right)$.
% The predicted pseudo-color is fed forward into the decoder of the Stable Diffusion model to obtain the final rendering result.

\noindent \textbf{Volume Rendering with Multiple Objects.}
% For each local frame $j$ with NeRF parameterized as $\theta_j$, we follow original NeRF design\cite{nerf} to integrate $(\boldsymbol{C}_l, \boldsymbol{\sigma}_l)$ of   sampled points from any hit ray $r_l=(\boldsymbol{o}_l, \boldsymbol{d}_l)$ by,
% For consistent scene rendering, object transmittance $T_k$ must be recalculated in the global frame based on independent properties inferred from local NeRFs. Hence, we sort predictions according to their distance to $\boldsymbol{o}_g$. 
% Similar to \cref{eq:volrend}, global color $\hat{\boldsymbol{C}}_g$ of ray $\boldsymbol{r}_g=(\boldsymbol{o}_g, \boldsymbol{d}_g)$ is predicted by the volumetric rendering integrating over $m$ objects,
We extend the volume rendering process to accommodate multiple objects by assigning each a local frame, denoted as $j$, with NeRF parameters $\boldsymbol{\theta}_{l, j}$. Drawing from the foundational NeRF approach \cite{nerf}, in each local frame, we integrate the color $\boldsymbol{C}_l$ and density $\boldsymbol{\sigma}_l$ for points $\boldsymbol{x}_l$ sampled along a ray $\boldsymbol{r}_l$, emanates from the camera origin $\boldsymbol{o}_l$ in direction $\boldsymbol{d}_l$. This is formalized in the predicted color integration for $\hat{\boldsymbol{C}}_l$ as:
{\setlength\abovedisplayskip{2pt}
\setlength\belowdisplayskip{2pt}
\begin{equation}
\label{eq:volrend}
{\hat{\boldsymbol{C}}_l}({\boldsymbol{r}_l})=\sum_{k=1}^{N} T_{l, k} \left(1-\exp \left(-\sigma_{l, k} \delta_k\right) \right) {\boldsymbol{C}}_{l,k},
\end{equation}}where $T_{l, k}=\exp \left(-\sum_{j=1}^{k-1} \sigma_{l,j} \delta_j\right)$ represents the transmittance to the $k$-th of total $N$ sample, calculated exponentially over the cumulative density along $\boldsymbol{r}_l$, and $\delta_k$ is the interval between adjacent samples.
%
To synthesize a coherent scene, we transition from processing individual local frames to a collective global frame. Within this global context, we reconcile object attributes inferred from their individual local NeRFs for refined $\boldsymbol{\sigma}_g, \boldsymbol{C}_g$ along with $T_{g, k}$. The samples $\boldsymbol{x}_g$ are ordered based on their spatial distances from the origin $\boldsymbol{o}_g$ following the coordinate transformation. We then express the volumetric rendering of a ray $\boldsymbol{r}_g$ integrating $m$ objects within the global frame as follows:
{
\setlength\abovedisplayskip{2pt}
\setlength\belowdisplayskip{2pt}
\begin{equation}
\label{eq:multi_volrend}
{\hat{\boldsymbol{C}}_g}({\boldsymbol{r}_g})=\sum_{k=1}^{m*N} T_{g, k} \left(1-\exp \left(-\sigma_{g, k} \delta_k\right) \right) {\boldsymbol{C}}_{g,k}. 
\end{equation}}

\noindent \textbf{Score Distillation Sampling.}
% During the SDS process, a noise image $\boldsymbol{X}_t$ is first generated by adding a sampled noise $\epsilon \sim \mathcal{N}(0, I)$ in noise level $t$ into a rendered view $\boldsymbol{X}$ from a NeRF.
To facilitate the conversion from text descriptions to 3D models, DreamFusion~\cite{poole2022dreamfusion} utilizes Score Distillation Sampling (SDS), leveraging the generative capabilities of a diffusion model, denoted as $\phi$, to guide the optimization of NeRF parameters, symbolized as $\boldsymbol{\theta}$.
%
Initially, SDS creates a noisy image $\boldsymbol{X}_t$ by infusing a randomly sampled noise $\epsilon$, which follows a normal distribution $\mathcal{N}(0, I)$, into a NeRF-rendered image $\boldsymbol{X}$ at a given noise level $t$.
The diffusion model $\phi$ then estimates the noise $\epsilon_\phi\left(\boldsymbol{X}_t, t, T\right)$ from this noisy image, conditioned by the noise level $t$ and an optional text prompt $T$. 
The key step in SDS involves calculating the gradient of the loss function, which measures the discrepancy between the estimated noise and the originally added noise:
{\setlength\abovedisplayskip{2pt}
\setlength\belowdisplayskip{2pt}
\begin{equation}
\label{eq:sds_loss}
\nabla_\theta \mathcal{L}_{\text{SDS}}(\boldsymbol{X}_t, T)=  w(t)\left(\epsilon_\phi\left(\boldsymbol{X}_t, t, T\right)-\epsilon\right),
\end{equation}}where $w(t)$ is a weighting function that adjusts the influence of the gradient based on the noise level. 
The gradients across all rendered views direct the update of $\boldsymbol{\theta}$, ensuring that the NeRF-generated images align with the text descriptions. Additionally, we incorporate the 'perturb and average' technique from SJC for more robust $\mathcal{L}_{\text{SDS}}$. For a comprehensive understanding of these methods, the reader is directed to the detailed explanations provided in \cite{poole2022dreamfusion,wang2022score}.

%
%
% \subsection{Editable 3D Scene Layout}
% \label{ssec:layout}
% The 3D scene layout explicitly combines language structures with 3D layouts in an editable way.
% Given the input text prompt $T$, the attribute-object pairs can be easily obtained based on user control.
% Note that the text prompt indicates the multi-object text prompt by default.
% % available for free in many structured representations, such as the constituency tree.
% As shown in Fig.~\ref{fig:framework}, we can extract multiple noun phrases with their binding attributes and map these local text prompts into corresponding regions.
% Specifically, we define the scene structure with $m$ local frames, each employs a local NeRF $\boldsymbol{\theta}_l$ as representation, the local text prompt $T_{l} \subseteq{T}$ and its spatial layout with 3D boxes $\mathbf{b} = \{\mathbf{p}, \mathbf{s}\} \in  \mathbb{R}^6$ of each object entity, where $\mathbf{p}=\{p_x, p_y, p_z\}$ refers to the center point and $\mathbf{s}=\{s_x, s_y, s_z\}$ denotes the box scale. 
% \textit{Our editable 3D layout is easy to be collected and edited with its simplicity, allowing for versatile and interactive user control by modifying the box's or text's properties to define a new scene}.
% Moreover, as depicted in Fig.~\ref{fig:teaser}, each component in a 3D scene layout can be replaced or re-composited with other trained local NeRFs, which is more friendly for flexible user editions compared with using only text prompts.
% We fine-tuned the new layout by global rendering, which enables scalable re-editing.
% Each relationship $r_k \in R$ is a triplet in a <subject-predictive object> format, where a subject node is. After we generate the scene graph from the complex prompts, we can sample the closest relationship with the 2d spatial layout as the initial 3D position. fine-tuned the new layout by global rendering, which enables scalable re-editing
%
% \subsection{Scene Rendering Pipeline}
% \label{ssec:render}
% In CompoNeRF, the scene images are rendered by a ray-casting approach following the design of NeRF.
% % Each ray to be cast is generated based on the camera pose, intrinsic, and transformation.
% The camera is defined by a pinhole camera model, casting a set of rays $(\boldsymbol{r}_o, \boldsymbol{\phi}_d, {\boldsymbol{\theta}}_d)=\boldsymbol{o}+t\boldsymbol{d}$ through each pixel on the frame of size $H \times W$, where the $\boldsymbol{r}_o \in  \mathbb{R}^3$ is the origin and the $(\boldsymbol{\phi}_d, \boldsymbol{\theta}_d)$ is the viewing direction.
% Along this ray, we sample all the points intersected with any layout box of local frames.
% For each hit sampled point, the color and volumetric density are computed through the local NeRF of the hit local frame.
% The ray color perdition is calculated by the differentiable integration applied on all the point-predicted colors and volumetric density along the ray.
%
% \noindent \textbf{Ray-box Intersection with Local Frames.}
% Given a ray $\boldsymbol{r}_i$, each box $\boldsymbol{b}_j$ of the local frame is applied with the AABB ray intersection test algorithm to check the intersections.
% When the ray $r_i$ is hit with a box $\boldsymbol{b}_j$ of the local frame, we use the entrance and exit points as near $\boldsymbol{t}_{in}$ and far $\boldsymbol{t}_{out}$ bounds to sample $N$ equidistant quadrature points, $
% \boldsymbol{t}_{i,j,n}=\frac{n-1}{N-1}\left(\boldsymbol{t}_{out}-\boldsymbol{t}_{in}\right)+\boldsymbol{t}_{in} , n \in \left[1, N\right]$
% % Despite each local frame only having a small number of hit rays compared to the scene, we observe that it is enough to represent each object accurately while maintaining short rendering times.
% Note that the coordinates of sampled points are first projected into normalized coordinates using the box scale of local frames to enable each local NeRF to learn the scale-independent representation.
% The bounding box $\mathbf{b}$ of the local frame in global coordinate can be transformed into a canonical bounding box by ${(\mathbf{b}} - \boldsymbol{p}) / \mathbf{s}$.
% Considering the rendering efficiency, we only calculate the valid points, interacted with the boxes, and set all the empty points with a constant background color.
%
% The appearance of a set object representations depends on its interaction with the scene and illumination which should be decided by the local frame location.
% To ensure the volumetric consistency, we only calibrate the emitted color with scene location, while the gradient still can be propagated.
% Since the overall color depends on both the global  positions $({x}_w, {y}_w, {z}_w)$ and ray directions $({\phi}_d, {\theta}_d)$, the global color embedding is learned based on both the positions and ray directions.
% Since the overall color depends on both the global  positions $({x}_w, {y}_w, {z}_w)$ and ray directions $({\phi}_d, {\theta}_d)$, the global color embedding is learned based on both the positions and ray directions.
% \subsection{The Proposed CompoNeRF}
% \subsubsection{Composition Module}
% CompoNeRF aims to composite multiple NeRFs to reconstruct multi-object scenes with both box and prompt guidance.
% %
% Our framework, as shown in \cref{fig:framework}, applies the AABB ray intersection test algorithm to check for intersections on each box in the global frame. We then samples $\boldsymbol{x}_g$ within the ray box intervals, and project them to $\boldsymbol{x}_l$ to infer  $\left(\boldsymbol{C}_l, {\boldsymbol{\sigma}_l}\right)$ in separate NeRF models. 
% %
% We then utilize volume rendering to obtain rendered views for each local frame respectively. 
% %
% After that, they would be passed on to our tailored composition Module to infer 
% $\left(\boldsymbol{C}_g, {\boldsymbol{\sigma}_g}\right)$
% for global rendering. 
% Next, we match local and global texts with their corresponding image outputs by SDS losses. 
% We also support recomposition by passing samples from cached models into $\boldsymbol{x}_l$ to continue the above process.
\begin{figure}[t!]
    \centering
    \includegraphics[width=\linewidth]{figures/abls.pdf}
    % \vspace{-22pt}
    % \caption{Ablation study on text guidance. (a) without local SDS losses. (b) without global SDS losses. (c) vanilla SDS losses without perturb and average scoring~\cite{wang2022score}. (d) full model.}
    \caption{\textbf{Design Impact Comparison: Density vs. Color-based Methods.} The top row illustrates the density-based approach's detailed rendering and quick convergence in the 'table wine' scene. The bottom row highlights the color-based method's enhancements and its drawbacks, such as geometric and shadow inaccuracies, particularly in close-up views and slow convergence.
    % \textbf{(a)} global text guidance(integrating local frames by \cref{eq:multi_volrend}) and global calibration(integrating local frames, then aligning the rendering result directly with the full text). 
    }
    \label{fig:abls}
    % \vspace{-20pt}
\end{figure}
\subsection{The Proposed CompoNeRF}
\subsubsection{Composition Module}
CompoNeRF is designed to composite multiple NeRFs to reconstruct scenes featuring multiple objects, utilizing guidance from both bounding boxes and textual prompts. Within our framework, depicted in \cref{fig:framework}, the Axis-Aligned Bounding Box (AABB) ray intersection test algorithm is applied to ascertain intersections across each box in the global frame. Subsequently, we sample points \(\boldsymbol{x}_g\) within the intervals of the ray-box and project them to \(\boldsymbol{x}_l\) to deduce the corresponding color \(\boldsymbol{C}_l\) and density \(\boldsymbol{\sigma}_l\) within individual NeRF models.
%
These properties are processed through our composition module to infer the global color \(\boldsymbol{C}_g\) and density \(\boldsymbol{\sigma}_g\), crucial for the global rendering.
%
Volume rendering techniques~\cite{kajiya1984ray} are then employed to procure the rendered views for both local and global frames. We propose dual SDS losses to ensure coherence between the image outputs and their corresponding textual descriptions. Additionally, our approach facilitates recomposition by channeling samples from cached models back into local frames along with the text revision, thereby streamlining the integration.

% As shown in \cref{fig:abls}(a), we verify its necessity by dropping $\nabla \mathcal{L}_{\text{SDS}_g}$. 
% %
% Compared with our full model, its layout does not fit our shared sense of a room, \ie, \emph{nightstand} is usually lower than \emph{bed}; \emph{lamp} needs a base to support it. Additionally,  it lacks global consistency, such as light reflection, to make it more realistic. 
% %
% Therefore, we leverage the full text semantics to ensure consistent global rendering across local frames. 
% %
% Instead of conditioning the global rendering view with the full prompt directly, we note that global calibration is necessary for geometry and color to be learned sufficiently.
% For example, we observe that geometric completeness and texture of \emph{nightstand} are not ideal. Although reflection appears around \emph{nightstand}, \emph{bed} is stripped of the light. 
% %
% Therefore, we opt to leverage the correlation between the rendering output of the combined NeRFs and the overall semantics to perform multi-object scene reconstruction.  
%

\noindent\textbf{Global Composition.}
The independent optimization of each local frame may inadvertently result in a lack of global coherence within the scene. To address this, our scene composition process is designed to integrate these frames, thereby achieving a more consistent result.
%
Before exploring the specifics of the module, it is imperative to discuss two critical design decisions within the composition module, as depicted in \cref{fig:framework}.
%
Upon integrating the properties inferred from \(\boldsymbol{x}_g\) into the composition module, they are fine-tuned through gradients derived from the global SDS loss.  This process leads to a critical consideration: the necessity and implications of refining the global density \(\boldsymbol{\sigma}_g\). This can be divided into two approaches: \textbf{1) Density-based:} The advantage of adjusting \(\boldsymbol{\sigma}_g\) is that it can adjust geometry, thus yielding a scene more congruent with the global text prompt. 
However, this comes at the cost of potentially compromising the optimal color \(\boldsymbol{C}_g\), as calibrating \(\boldsymbol{\sigma}_g\) introduces more uncertainty for subsequent color refinement as it requires prior density features $\boldsymbol{h}$ as shown at \cref{fig:compo}. 
\textbf{2) Color-based:} Conversely, directly employing \(\boldsymbol{\sigma}_l\) mitigates this uncertainty but at the expense of reduced geometric control, presenting a challenging balance to strike in the pursuit of precise scene composition.
% , which may lead to suboptimal outcomes.
%
After thorough experiments, exemplified in \cref{fig:abls}, we have opted for the density-based approach to refine \(\boldsymbol{\sigma}_g\)  prioritizing both \textbf{accuracy and efficiency}. The test revealed that it excels in rendering intricate details, such as enhanced wood grain textures and more naturally contoured 'salad', as accentuated by boxes. This method also demonstrated a swifter convergence rate. Conversely, while the color-based improved reflections and reduced flickering on the 'wine cup', it was plagued by issues such as sparse density, which adversely brings holes at the base of the 'cup' and the corner of the 'table'.
Furthermore, upon close examination, it becomes evident that shadow artifacts of 'wine' on the 'table' are pronounced, suggesting that its disadvantages outweigh its advantages.
%  in this context
% \textbf{Global Composition.}
% Each local frame is optimized independently, causing a lack of global connections for scene composition.
% Before delving into module details, there are two choices (see \cref{fig:framework}) on the composition module design we need to elaborate on first. 
% %
% In \cref{fig:framework}, by taking $\boldsymbol{x}_g$ into the composition module, their inferred properties are calibrated with gradients propagated from the global SDS loss. 
% However, it remains unclear whether $\boldsymbol{\sigma}_g$ should be refined or not. 
% %
% The trade-off on its usage is the density adjustment bringing a more reasonable layout and more geometric details that fit the global text prompt. While its potential downside is that $\boldsymbol{C}_g$ may not be optimal as $\boldsymbol{\sigma}_g$ has more uncertainty compared to $\boldsymbol{\sigma}_l$, bringing sub-optimal rendering results. 

% We choose the density-based method after comparing them with the experiment shown in \cref{fig:abls}. 
% %
% Specifically, we test both designs on the scene \emph{table wine} and discover that the density-based design provides more intrinsic details(as indicated by green boxes), \eg, enriched wood grains, and a more natural shape for \emph{salad} and has much faster convergence speed. In contrast, the color-based method enhances the reflection and smooths flickering on \emph{wine cup}, (as indicated by red boxes), but it suffers from 1) sparse density, resulting in poorly generated geometry at the base of  \emph{cup} and the wood \emph{table} corner. Additionally, shadow artifacts appeared on \emph{table} when viewed up close, outweighing benefits of the color-based method.

\begin{figure}[t!]
    \centering
    \includegraphics[width=\linewidth]{figures/compo_module.pdf}
    % \vspace{-24pt}
    % \caption{Ablation study on text guidance. (a) without local SDS losses. (b) without global SDS losses. (c) vanilla SDS losses without perturb and average scoring~\cite{wang2022score}. (d) full model.}
    \caption{\textbf{Detail of Composition module}: density-based design. 
    }
    \label{fig:compo}
    % \vspace{-18pt}
\end{figure}
\noindent\textbf{Network Design.}
The compositional framework of our network, as delineated in \cref{fig:compo}, is predicated on an architecture that employs a suite of MLPs, represented as \(\{\boldsymbol{\theta}_l\}_{l=1}^{m}\),  each dedicated to a distinct local frame. To harmonize \(\boldsymbol{\sigma}_l\) and \(\boldsymbol{C}_l\), we incorporate global MLPs, including density calibrator $f_{\boldsymbol{\theta}_{g_d}}$ and color calibrator $f_{\boldsymbol{\theta}_{g_c}}$.
%
A transformation module complements this system, tasked with maintaining the spatial coherence between the global and local frames. It governs the transformation of sampling points $\boldsymbol{x}$, ray directions $\boldsymbol{d}$, and adjacent sampling distances $\delta$. This module also orders the points $\{\boldsymbol{x}_{g,j}\}_j$ by their distance to the global camera origin $\boldsymbol{o}_g$, ensuring that each local point $\boldsymbol{x}_l$ is accurately matched with its corresponding global point $\boldsymbol{x}_g$ for subsequent volume rendering. 
%
The network design is:
{
\setlength\abovedisplayskip{4.5pt}
\setlength\belowdisplayskip{4.5pt}
\begin{align}
\label{eq:g_c_d}
{\boldsymbol{\sigma}_g}  &= \alpha_d f_{\boldsymbol{\theta}_{g_d}}({\boldsymbol{x}_g}) + \boldsymbol{\sigma}_l, \\
{\boldsymbol{C}_g}  &= \alpha_c f_{\boldsymbol{\theta}_{g_c}}(\boldsymbol{h}, {\boldsymbol{d}_g}) + \boldsymbol{C}_l. 
\end{align}}In contrast to the local frames, the global frame's color output $\boldsymbol{C}_g$ is inferred based on $\boldsymbol{h}$ and conditional on $\boldsymbol{d}_g$ to enable a view-dependent lighting effect.
% Denote the density features as $\boldsymbol{h}$. 
%
%
Residual learning is leveraged here, where \(\boldsymbol{\sigma}_l, \boldsymbol{C}_l\) serve as foundational elements that support the learning of global density \(\boldsymbol{\sigma}_g\) and color \(\boldsymbol{C}_g\). The parameters \(\alpha_d, \alpha_c\) are adjustable, allowing fine-tuning of the influence that local components exert on the global outputs.
%
It is imperative to acknowledge that in our color-based method, density calibration is intentionally excluded to concentrate solely on the refinement of color dynamics as shown at \cref{fig:framework}. This is achieved by conditioning the process on both spatial and directional global inputs \((\boldsymbol{x}_g, \boldsymbol{d}_g)\), as demonstrated in the following equations:
\begin{align}
\setlength\abovedisplayskip{4.5pt}
\setlength\belowdisplayskip{4.5pt}
\label{eq:g_c_c}
\boldsymbol{\sigma}_g = \boldsymbol{\sigma}_l, \quad
{\boldsymbol{C}_g} = \alpha_c f_{\boldsymbol{\theta}_{g_c}}({\boldsymbol{x}_g}, {\boldsymbol{d}_g}) + \boldsymbol{C}_l.
\end{align}
The integration of extra $\boldsymbol{x}_g$ aims to facilitate a fair comparison under same inputs with the density-based. It enhances the visual appeal of effects like the wine cup's reflection, as demonstrated in \cref{fig:abls}. However, this method is not without its compromises. It tends to produce artifacts and is characterized by a slower convergence rate. Additionally, this approach limits the ability to precisely control density, subsequently impacting the intricate geometric details.


\begin{figure*}[t!]
    \centering
    \includegraphics[width=\linewidth]{figures/sota.pdf}
    % \vspace{-24pt}
    \caption{\textbf{Qualitative comparison with other text-to-3D methods using multi-object text prompts}. Cases 1-3 demonstrate simpler settings characterized by compositions involving two objects. In contrast, Cases 4-8 delve into more intricate scenarios featuring compositions with more than two objects. Smaller images are presented to illustrate the generated local NeRFs(partially shown in Cases 4-8).}
    \label{fig:sota}
    % \vspace{-5pt}
\end{figure*}
%
% \begin{table*}[t!]
% \centering
% \resizebox{\textwidth}{!}
% {
% \begin{tabular}{cccccccc}
% \toprule
% Method            & \rotatebox{60}{table wine}  & \rotatebox{60}{teddy monkey} & \rotatebox{60}{computer mouse} & \rotatebox{60}{bed room}  & \rotatebox{60}{chess} & \rotatebox{60}{pisa tower} & \rotatebox{60}{astronaut} & \rotatebox{60}{tesla}  \\ \midrule
% LatentNeRF  & 21.55 & 27.38 & 17.13 & 21.86 & 31.19 & 24.31 & 27.07 & 25.16 \\
% SJC & 23.33 & 27.37 & 18.00 & 22.54 & 30.53 & \textbf{26.18 }& 27.84 & 23.55 \\
% CompoNeRF & \textbf{32.68} & \textbf{28.57}	 &\textbf{ 22.34} &\textbf{ 28.65} & \textbf{31.45} & \textbf{28.96} & 25.82 & 25.95 & 24.42 & \textbf{32.71} & \textbf{26.13 }& \textbf{26.38} & \textbf{30.98} & \textbf{33.37} \\
% \bottomrule
% \end{tabular}
% }
% \vspace{-10pt}
% \caption{Performance of our CompoNeRF in different 3D scenes. We use CLIP score \cite{parmar2023zero,zhang2023sine,wang2023imagen} as our evaluation metric, which is a common evaluation metric in text-to-image generation tasks to evaluate the similarity of the generated image to the text prompt. }
% \label{perclass}
% \end{table*}
%
\begin{table*}[t!]
% \scalebox{0.8}
\renewcommand{\arraystretch}{1.2}
\fontsize{4pt}{4pt}
\selectfont 
\centering
% \vspace{-8pt}
\resizebox{\textwidth}{!}
{
% \begin{tabular}{lcccccccc}
% \hline
% Method     & table\_wine    & tesla          & pyramid        & chess          & apple and banana      & astronaut      & glass\_balls   & Eiffel\_tower    \\ \hline
% LatentNeRF & 21.55          & 25.16          & 27.43          & 31.19          & 27.69          & 27.07          & 29.51          & 26.32          \\
% SJC        & 23.33          & 23.55          & 25.62          & 30.53          & 28.21          & 27.84          & 28.76          &27.41 \\
% \textbf{CompoNeRF(Ours)}     & \textbf{32.68} & \textbf{26.13} & \textbf{28.96} & \textbf{31.45} & \textbf{33.37} & \textbf{32.71} & \textbf{30.98} & \textbf{28.44}          \\ \hline
% \end{tabular}
\begin{tabular}{lcccccccc}
\hline
Method                   & Case 1         & Case 2         & Case 3         & Case 4         & Case 5         & Case 6         & Case 7         & Case 8         \\ 
\hlineB{1.1}
LatentNeRF               & 25.16          & 27.07          & 27.69          & 31.19          & 21.55          & 26.32          & 27.43          & 29.51          \\
SJC                      & 23.55          & 27.84          & 28.21          & 30.53          & 23.33          & 27.41          & 25.62          & 28.76          \\
\textbf{CompoNeRF (Ours)} & \textbf{26.13} & \textbf{32.71} & \textbf{33.37} & \textbf{31.45} & \textbf{36.06} & \textbf{28.44} & \textbf{28.96} & \textbf{30.98} \\ \hlineB{1.1}
\end{tabular}
}

% \vspace{-6pt}
\caption{\textbf{Performance comparison of our CompoNeRF in different 3D scenes}. For our evaluation metric, we utilize the average of CLIP scores~\cite{parmar2023zero,zhang2023sine,wang2023imagen} across different views, which serve to assess the similarity between the generated images and the global text prompt. }
\label{tb:perclass}
\end{table*}
% \cref{fig:framework} depicts the network architecture of the composition module. Denote $m$ as local MLP $\{\boldsymbol{\theta}_l\}_{l=1}^{m}$ for each local frame. Then, we introduce the global MLPs including density $\boldsymbol{\theta}_{g_d}$ and $\boldsymbol{\theta}_{g_c}$ calibrators to refine $\boldsymbol{\sigma}_l$ and $\boldsymbol{C}_l$. 
% %
% In detail, the network design is, 
% {
% % \setlength\abovedisplayskip{4.5pt}
% % \setlength\belowdisplayskip{4.5pt}
% \begin{align}
% \label{eq:g_c_d}
% {\boldsymbol{\sigma}_g}  &= \alpha_d \boldsymbol{\theta}_{g_d}({\boldsymbol{\sigma}_l}) + \boldsymbol{\sigma}_l, \\  
% {\boldsymbol{C}_g}  &= \alpha_c \boldsymbol{\theta}_{g_c}({\boldsymbol{C}_l},  {\boldsymbol{d}_g}) + \boldsymbol{C}_l, 
% \end{align}}
% %
% where residual $\boldsymbol{\sigma}_l, \boldsymbol{C}_l$ assist in learning $\boldsymbol{\sigma}_g$ and $\boldsymbol{C}_g$, while $\alpha_d, \alpha_c$ balance their contribution as learnable parameters.
% %
% Note that the color-based omits density calibration, and simply uses the shared color refinement.



% The 3D boxes are only used for the spatial configuration of local NeRFs, while the implicit representation of local NeRFs is inferred by the canonical samples inside the local frame without considering the global relationship across different objects.
% To relieve such location-dependent effects, we further calibrate the output color and density from the local NeRF with global coordinates $({\boldsymbol{x}}_g, {\boldsymbol{y}}_g, {\boldsymbol{z}}_g)$ and ray directions $\left({\boldsymbol{\phi}}_{d}, {\boldsymbol{\theta}}_{d}\right)$ as the conditional input.
% % to inject the global visual clues.
% %
% %
% Specifically, we adopt a shared MLP $\boldsymbol{\theta}_{g}$ to calibrate all the predicted object colors, that is,
% {\setlength\abovedisplayskip{4.5pt}
% \setlength\belowdisplayskip{4.5pt}
% \begin{align}
% \label{eq:MLP_dyn_2}
% {\boldsymbol{C}_g} = {\boldsymbol{C}_l} + \boldsymbol{emb}_{g} &= {\boldsymbol{C}_l} + \boldsymbol{\theta}_{g}({\boldsymbol{x}}_g, {\boldsymbol{y}}_g, {\boldsymbol{z}}_g, {\boldsymbol{\phi}}_{d}, {\boldsymbol{\theta}}_{d}),
% \end{align}}
% where ${\boldsymbol{C}_l}$ is the color predicted by the local NeRF.
% Therefore, the scene color can preserve the view-consistent behavior from the original architecture and add consistency across poses for the volumetric density.
% Since the color and density values share the same latent expression in $({\boldsymbol{x}}_l, {\boldsymbol{y}}_l, {\boldsymbol{z}}_l)$, we only calibrate the emitted scene color explicitly with the scene location, as the densities of local NeRFs also are implicitly adjusted during optimization.

% \noindent \textbf{Global and Local Volumetric Rendering.}
% After compositing all the interacted points, each ray $\boldsymbol{r}_i$ collects a set sampling points by $\{\boldsymbol{t}_{i,j,n} \}_{j=1, n=1}^{m_j, N}$, where $m_j$ is the number of the hit object.
% For each sampling point, the inference results with the respective 3D representations are the local color $\boldsymbol{c}_{l}$, global color $\boldsymbol{c}_{g}$, and density $\sigma$.

% In fact, the local view $\hat{C}_{l,j}$ of single object $j$ also can be rendered by the sampled points  belongs to the same local frames as shown at Fig.~\ref{fig:framework}.

\subsubsection{Recomposition}
Our architecture advances scene reconstruction by providing an intuitive interface for layout manipulation.  This capability is crucial for the reconfiguration of scene elements into novel scenes, as depicted in \cref{fig:framework}. Here, the input panel allows for adjustments in the attributes of bounding boxes, such as modifying the position and scale of the 'apple' bounding box prior to composition. The refinement process further involves sampling ray-box intervals from the global frame, leading to transformed coordinates with the corresponding ray samples that are then incorporated into the pipeline, as demonstrated in \cref{fig:compo}.
%
Each bounding box represents an individual NeRF, providing the flexibility to move, scale, or remove elements as needed. CompoNeRF's capabilities also extend to textual edits, exemplified by the transformation of 'wine' into 'juice'.
%
Since NeRFs have been well trained, we only finetune \(\theta_g, \theta_l\) to align text prompts to promote consistency of both local and global views.
%
Moreover, the NeRFs once retrained within the edited scene, are also structured to be decomposable and cacheable in future scene compositions.
% Our CompoNeRF architecture facilitates the seamless reconstruction of scenes leveraging existing models. It enables precise editing of bounding boxes parameterized by \(\{\boldsymbol{\theta}_l\}_{l=1}^{m}\), allowing for their reconfiguration into new layouts. Refer to \cref{fig:framework}, the input panel permits the modification of attributes such as the position and scale of the 'apple' node's bounding box prior to composition. The process is further refined by sampling from the updated ray-box intervals within the global frame, which are then projected onto \(\boldsymbol{x}_l\), ensuring a streamlined reconstruction that integrates the 'apple' effectively. This addition is executed with careful attention to color consistency, positioning the 'apple' adjacent to the 'French bread' to complement the scene's overall palette. Each bounding box represents an individual NeRF, which means they can be manipulated through moving, scaling, and removal operations. CompoNeRF also extends its editing prowess to textual modifications, as evidenced by the 'wine cup' now appearing filled with juice—a change propagated through both subtexts and the global test. 
% %
% Since NeRFs have been well trained, we only finetune $\theta_g, \theta_l$ to align text prompts to promote consistency of both local and global views . 
% %
% Moreover, the NeRFs, once retrained within the reimagined scene, are also structured to be decomposable and cacheable for subsequent scene compositions.

% , as shown in Fig.~\ref{fig:framework}.
% For each scene described by the multi-object text prompt $T$, we
% To enhance the guidance of local representations, we use the local text prompt $T_l \subseteq T$ of a single object to optimize the local NeRFs in local views.
% The scene views $\hat{\boldsymbol{X}}_g=\{\hat{\boldsymbol{C}}_{g,i}\}_{i=1}^{H\times W}$ is obtained from the predicted pixel values of $H \times W$ rays by compositing all the ray-box interaction values.
% Similarly, the rendered view $\hat{\boldsymbol{X}}_{l,j}$ of the local frame $\boldsymbol{\theta}_j$ without compositing other objects can be calculated by $\hat{\boldsymbol{C}}_{l,j}$, as depicted in Sec.~\ref{ssec:render}.
% We use the local color instead of the globally calibrated color to obtain a local view because the local NeRF should learn the object identity unrelated to its placed position, as the position can be different during user edition.
% % Compared to cropping the local region from a global view for training, separate rendering can avoid the undesired information from other objects brought by the occlusion and resolution adjustments.
% Formally, we employ the following loss as the learning objective,
\begin{figure*}[t!]
    \centering
    \includegraphics[width=\linewidth]{figures/editing.pdf}
    % \vspace{-23pt}
    \caption{\textbf{Scene Editing Outcome:} Demonstrated here are the stages of our recomposition, utilizing cached source scenes. Each NeRF is individually identified by colorful labels. These decomposed nodes are then positioned in the initial layout and subsequently calibrated to form the final composition. The detailed description of the ambient environment is underscored, enhancing the scene's realism.}
    \label{fig:app}
    % \vspace{-12pt}
\end{figure*} 
\subsubsection{Optimization}
\label{sec:optimization}
During optimization, our method employs dual text guidance to align rendering results with both global and local textual descriptions. The optimization objective is:
{
\small
\setlength\abovedisplayskip{2pt}
\setlength\belowdisplayskip{2pt}
\begin{equation}
\label{eqn:loss_f}
\mathcal{L}= {\alpha_g}\nabla\mathcal{L}_{\text{SDS}}(\hat{\boldsymbol{X}}_{g}, T) + {\alpha_l}\sum_{j=1}^{m} \nabla\mathcal{L}_{\text{SDS}}(\hat{\boldsymbol{X}}_{l,j}, T_{l,j}) + \beta\mathcal{L}_{\text{sparse}},\nonumber
\end{equation}
}where $T$ signifies the global text prompt, while $T_{l}$ pertains to a specific object within the global context. The hyperparameters $\alpha_{g}, \alpha_{l}$, and $\beta$ modulate the respective loss weights. 
% $\nabla \mathcal{L}_{\text{SDS}}$ is the score distillation sampling loss, as described in Sec.~\ref{sec:background}.
As suggested in~\cite{metzer2022latent}, we use $L_{\text{sparse}}$ included to penalize the binary entropy of local NeRFs' densities, thereby mitigating the issue of extraneous floating radiance.
Additionally, incorporating directional cues such as "front view" or "side view" into the input text, as suggested by \cite{poole2022dreamfusion,metzer2022latent} proves beneficial in specifying camera poses during the training phase, further enhancing the alignment of our generated scenes with the intended perspectives.
% Note that the global calibration in the scene frame can adaptively revise both $({C}_l, {\sigma})$ in local NeRF with $\nabla \mathcal{L}_{SDS}$ along with the back-propagating gradient.

\section{Technical Approach}
\label{sec:method}

In this section, we detail our proposed single-shot detector for articulated objects. Our method consists of two individually learned components: an encoder that predicts latent object codes as well as poses in the camera frame and a decoder that reconstructs objects in a canonical frame that can be transformed into the camera frame through the predicted pose. An overview of our approach is shown in \cref{fig:overview}.
\begin{figure*}
    \centering
    \includegraphics[width=0.9\textwidth]{content/assets/imgs/full_diagram_low_res.pdf}
    \caption{Overview of our proposed method. We first encode a stereo image using \cite{laskey2021simnet} to predict a depth and an importance value, a pose as well as a shape and joint code for each pixel using peak detection on the depth map allows us to detect objects which then can be reconstructed given the latent code. Last, to place the objects in camera frame we transform the reconstructed point cloud using the predicted poses at the peaks. On the right side we show the position of the predicted shape codes in a t-SNE visualization of the learned shape codes for the training used as input to our single category- and joint-agnostic decoder. We additionally project each categories mean shape code and use them to reconstruct the objects at the average prismatic and revolute joint state in the training set.}
    \label{fig:overview}
    %\vspace{-0.3cm}
\end{figure*}


\subsection{Encoder}
\label{subsec:encoder}
Our encoder builds upon CenterSnap~\cite{irshad_centersnap_2022} and SimNet~\cite{laskey2021simnet}. For each pixel in our input stereo image $\inputImage$ we predict an importance scalar $\importance$, where a higher value indicates closeness to a 2D spatial center in the image of the object. The full output map of $\importance$ represents a heatmap over objects. 
Additionally, we predict a dense pixel map of canonical 6D poses for each articulated object independent of their articulation state. 
Further, we extend \cite{irshad_centersnap_2022}, which predicted a shape code $\latentShape \in \realspace^{\dimensionsShapeSpace}$, to also predict a joint code $\latentJoint \in \realspace^{\dimensionsJointSpace}$ for each pixel.  These codes can be used to predict the articulation state of the object.

Additionally, while not needed for our full pipeline, to guide the network towards important geometric object features, we also predict a semantic segmentation mask as well as 3D bounding boxes, again on a pixel-level. We use these predictions for constructing our baseline as described in \cref{subsubsec:full_pipeline_baseline}. Last, we also predict a depth map $\depthMap$. The full network architecture is given in \ifthenelse{\boolean{reftosupp}}{\cref{supp:subsec:encoder_architecture}}{Sec.~S.1.1}. % in the supplementary material.

During inference of our full pipeline, given the predicted heatmap of importance values, we use non-maximum suppression to extract peaks in the image. At each peak, we then query the feature map to get the pose, shape, and joint code. We convert our 13-dimensional pose vector to a scale value $\in\realspace$ of the canonical object frame, a position $\in\realspace^{3}$ and using \cite{bregier2021deepregression} to an orientation $\in\realspace^{3 \times 3}$ in the camera frame.
We then use the shape and joint code to reconstruct each object in its canonical object frame using our decoder. After reconstruction, we use the predicted pose to place the object in the camera frame as shown in \cref{fig:overview}. 

\subsection{Decoder}
\label{subsec:decoder}
Given a latent code, the decoder reconstructs object geometry, classifies the discrete joint type (i.e., as prismatic or revolute), and predicts the continuous joint state.
To disentangle the shape of the object from its articulation state, we split the latent code in two separate codes: a shape and a joint code. We assign the same unique shape code  $\latentShape \in \realspace^{\dimensionsShapeSpace} $ to an object instance in different articulation states, where an articulation state is expressed through its own joint code variable $\latentJoint \in \realspace^{\dimensionsJointSpace}$. We structure our decoder as two sub-decoders, one for reconstructing the geometry (\cref{subsubsec:geometry_decoder}) and the other for predicting the joint type $\jointType$ and state $\jointState$ (\cref{subsubsec:joint_decoder}). See \ifthenelse{\boolean{reftosupp}}{\cref{supp:subsec:decoder_architecture}}{Sec.~S.1.2} for a full architecture description.

\subsubsection{Geometry Decoder}
\label{subsubsec:geometry_decoder}
The geometry decoder $\geometryDecoder$ reconstructs objects based on a shape code $\latentShape$ and joint code $\latentJoint$. In principle, the approach is agnostic to the specific decoder architecture as long as it is differentiable with respect to the input latent codes. While there are many potential options such as occupancy maps \cite{mescheder2019occupancy} as adopted in \cite{jiang_ditto_2022}, we use signed distance functions (SDFs) \cite{park_deepsdf_2019} due to the proven performance in \cite{mu_a-sdf_2021, park_deepsdf_2019}.
Specifically, in the case when using SDFs as our geometry decoder, the model takes as input a point in 3D space $\spaceCoordinate$ as well as a shape  $\latentShape$ and joint code $\latentJoint$
\begin{equation}
    \geometryDecoder(\latentShape, \latentJoint, \spaceCoordinate) = \hat{\sdfValue}_{\spaceCoordinate}
\end{equation}
and predicts a value $\hat{\sdfValue}_{\spaceCoordinate}$ that indicates the distance to the surface of the object.

For faster inference, we implement a multi-level refinement procedure~\cite{irshad_2022_shapo}. We first sample query points on a coarse grid and refine them around points that have a predicted distance within half of the boundary to the next point. This step can be repeated multiple times to refine the object prediction up to a level $\objectLoD$. Eventually, we extract the surface of the objects by selecting all query points $\spaceCoordinate$ for which $ |\hat{\sdfValue}_{\spaceCoordinate}| < \sdfValueThreshold $ holds. By taking the derivative 
\begin{equation}
    \normal_{\spaceCoordinate} = \frac{\partial \geometryDecoder(\latentShape, \latentJoint, \spaceCoordinate)}{\partial \spaceCoordinate}
\end{equation}
and normalizing it we get the normal $\hat{\normal}_{\spaceCoordinate}$ at each point $\spaceCoordinate$, which can then be used to project the points onto the surface of the object with $\hat{\spaceCoordinate} = \spaceCoordinate - \hat{\sdfValue}_{\spaceCoordinate} \hat{\normal}_{\spaceCoordinate}$.

\subsubsection{Joint Decoder}
\label{subsubsec:joint_decoder}
As we represent the articulation state of the object implicitly through a joint code $\latentJoint$, we additionally introduce an articulation state decoder $\jointDecoder$ to regress a discrete joint type $\jointType = \{ \text{prismatic}, \text{revolute} \}$ and a continuous joint state $q$:
% , which is dependent on the predicted joint type:
\begin{equation}
    \jointDecoder(\latentJoint) = \hat{\jointType}, \hat{\jointState}
\end{equation} 
We use a multi-layer perceptron with 64 neurons in one hidden layer. 

\begin{figure}
    \centering
    \includegraphics[width=0.7\linewidth]{content/assets/imgs/latent_joint_space_svd_with_imgs.pdf}
    \caption{Intuition for Latent Space Regularization.
    Our main idea is that the joint codes 
    % $\latentJoint^i$ and $\latentJoint^j$
    of two similarly articulated objects should be close. We define the similarity first through the joint type $\jointType$ and second through an exponential distance measure of the joint state $\jointState$. Here, the laptop (a) and the oven (b) have a revolute joint and are similarly wide open around $30\degree$. Compared to that, table (d) has a prismatic joint and thus should not be close to the revolute instances. Contrary to that, the dishwasher (c), has a revolute joint but is opened much more than the other revolute objects and therefore, should be relatively close. The visualization shows a lower dimensional projection of our learned latent joint space trained using our regularization.}
    \label{fig:latent_space_intuition}
    %\vspace{-0.3cm}
\end{figure}

\subsubsection{Backward Code Optimization}
\label{para:backward_code_optimiziation}
For the task of canonical object reconstruction from a set of SDF values at specific query points, we follow the optimization procedure from \cite{mu_a-sdf_2021} to retrieve a shape and joint code. Different from \cite{mu_a-sdf_2021}, we utilize GPU parallelization to optimize multiple code hypotheses at once and pick the best one in the end. Additionally, we do not reset the codes as done in \cite{mu_a-sdf_2021} but rather freeze them for some iterations. In early testing, we discovered that first optimizing for both codes jointly together gives a good initial guess. Freezing the joint code in a second stage of optimization helps such that the shape fits the static part and then last, freezing the shape code such that the joint code can do some fine adjustment to the articulation state of the object. To guide the gradient in the joint code space, we first transform the space using the singular value decomposition of the stacked training joint codes $\latentJoint^{\jointIndex} \+\forall\+ \jointIndex \in 1, \ldots, \jointCount$.
For further information, refer to \ifthenelse{\boolean{reftosupp}}{\cref{supp:sec:backward_optim}}{Sec.~S.2}. % in the supplementary material.

\subsection{Training Protocol}
To train \ourName{}, we first train the decoder as it provides ground truth training labels for the shape and joint code supervision of our encoder. Once shape and joint code labels are obtained for the objects in the dataset, we then train our encoder to predict the latent codes in addition to object pose. Thus, to adhere with our training procedure, we will first explain how to train our decoder and then our encoder. 
\subsubsection{Decoder}
Given a training set of $\objectCount$ objects each in $\jointCount$ articulation states, we denote the $\objectIndex$-th object in its $\jointIndex$-th articulation state as $\object_{\objectIndex, \jointIndex}$. As during training, a fixed association between each object and its latent codes is given, we can uniquely identify $\object_{\objectIndex, \jointIndex} = ( \latentShape^{\objectIndex}, \latentJoint^{\jointIndex})$ as a tuple of both. This allows us to pass the gradient all the way to the codes themselves and thus, the embedding spaces rearrange them accordingly. Similar to \cite{mu_a-sdf_2021}, during training we regularize the codes through minimizing the L2-norm
\begin{equation}
    \label{eqn:loss_code_reg}
    \lossGeneralFunction_{\indexCodeRegularizer}(\latentVariable) =
    \| \latentVariable \|,
\end{equation}
where $\latentVariable$ is either $\latentShape$ or $\latentJoint$.

{\parskip=5pt
\noindent\textbf{The geometry decoder} described in \cref{subsubsec:geometry_decoder} is trained on a set of query points $\spaceCoordinate$ close to the object surface sampled as in \cite{park_deepsdf_2019}.}
We define our reconstruction loss $\lossGeneralFunction_{\indexReconstruction}$ as in \cite{park_deepsdf_2019} but use a leaky clamping function \begin{equation}
    \leakyClamp(s|\leakyThreshold, \leakySlope) = 
    \begin{cases} 
        s & |s| \leq \leakyThreshold, \\
        \leakySlope s &  |s| > \leakyThreshold
    \end{cases}
\end{equation}
which is conceptually similar to a leaky ReLU by instead of hard clamping values above a threshold $\leakyThreshold$, we multiply it by a small factor $\leakySlope$. Initial testing revealed a more stable training. 
Our reconstruction loss at one query point $\spaceCoordinate$ is now given by:
\begin{align}
    \label{eqn:loss_reconstruction}
    & \lossGeneralFunction_{\indexReconstruction}\left( \latentShape, \latentJoint, \spaceCoordinate, \sdfValue_{\spaceCoordinate} \right) = \\
    & \quad \left| 
        \leakyClamp(
            \geometryDecoder(\latentShape, \latentJoint, \spaceCoordinate)
            | \leakyThreshold, \leakySlope
        )
        -
        \leakyClamp(
            \sdfValue_{\spaceCoordinate}
            | \leakyThreshold, \leakySlope
        )
    \right|, \nonumber
\end{align}
where $\sdfValue_{\spaceCoordinate}$ is the ground truth distance to the surface.

{\parskip=5pt
\noindent\textbf{The joint decoder} introduced in \cref{subsubsec:joint_decoder} is jointly trained with the aforementioned geometry decoder. For the joint type loss $\lossGeneralFunction_{\indexJointType}$, we use cross entropy between the predicted joint type $\hat{\jointType}$ and ground truth $\jointType$ and for the joint state loss $\lossGeneralFunction_{\indexJointState}$ the L2-norm between the predicted joint state $\hat{\jointState}$ and ground truth $\jointState$.}


{\parskip=5pt
\noindent\textbf{Joint Space Regularization}: 
\label{subsubsec:joint_space}
One core contribution of our approach is how we impose structure in our joint code space during our decoder training. Here, we enforce the same similarity of latent codes as their corresponding articulations states have. A visualization of the underlying idea is shown in \cref{fig:latent_space_intuition}.
Formally, given the joint codes $\latentJoint^{k}$ and $\latentJoint^{l}$ encoding two different articulation states $k, l \in 1, \ldots, \jointCount$, we define the similarity between them in latent space as 
\begin{equation}
    \similarityFunctionLatent \left( \latentJoint^{k}, \latentJoint^{l} \right) =
    \exp \left(- \frac{\| \latentJoint^{k} - \latentJoint^{l} \| }{\sigma} \right). % Sigma == 1
\end{equation}
Similarly, we define the respective similarity in real joint space, considering the joint types $\jointType^k$ and $\jointType^l$ and the joint states $\jointState^k$ and $\jointState^l$, through
\begin{align}
    \label{eqn:joint_sim_real}
    &\similarityFunctionReal \left( \left( \jointType^k, \jointState^k \right), \left( \jointType^l, \jointState^l \right) \right) = \\
    &\qquad\qquad \begin{cases}
        \exp \left(
            - \left( \frac{\jointState^k-\jointState^l}{\sigma_\jointType} \right)^2 
        \right) & \jointType^k = \jointType^l \\
        0 & \jointType^k \neq \jointType^l, \nonumber
    \end{cases}
\end{align}
where $\sigma_\jointType$ is a joint type specific scaling.
By minimizing the L1-norm between both similarity measurements 
\begin{align}
    \label{eqn:joint_code_reg_single}
    & \lossGeneralFunction_{\indexJointRegularizer} \left( 
        \latentJoint^{k}, \latentJoint^{l} \right) =   \\
    & \quad \left|
            \similarityFunctionLatent \left( \latentJoint^{k}, \latentJoint^{l} \right)
            -
            \similarityFunctionReal \left( \left( \jointType^k, \jointState^k \right), \left( \jointType^l, \jointState^l \right) \right)
        \right|  \nonumber
\end{align}
we enforce that the latent similarities are similarly scaled as their real similarities. 

We scale this formulation to all articulation states in the training set as described below. Calculating $\similarityFunctionReal$ can be done once in a pre-processing step for all articulation state pairs $k,l \in 1, \ldots, \jointCount$ resulting in a matrix $\similarityMatrix_\indexReal \in \realspace^{\jointCount \times \jointCount}$. Similarly, calculating all $\similarityFunctionLatent$-pairs can be efficiently implemented as a vector-vector product. We denote the resulting matrix as $\similarityMatrix_\indexLatent \in \realspace^{\jointCount \times \jointCount}$. \cref{eqn:joint_code_reg_single} now simplifies to\looseness=-1
\begin{align}
    \label{eqn:joint_code_reg}
    \lossGeneralFunction_{\indexJointRegularizer} =  
    \frac{
    \left|
        \similarityMatrix_\indexLatent
            -
        \similarityMatrix_\indexReal
    \right|}{\jointCount^2}.
\end{align}
Through this efficient calculation, optimizing this loss term comes with almost no overhead during training. This concept of similarity can be extended for arbitrary kinematic graphs.
}

{\parskip=5pt
\noindent\textbf{Pre-Training}: Before we start training our full decoder, we only optimize our joint codes $\latentJoint^{\jointIndex}\+\forall\+\jointIndex \in 1,\ldots, \jointCount$. The pre-training helps with learning the full decoder as our joint codes are already more structured and thus, it is easier to learn the shape and joint code disentanglement. 
In the pre-training, we minimize
\begin{equation}
    \lossGeneralFunction_{\text{pre}} = 
    \lossScaling_{\indexCodeRegularizer, \latentJoint, \text{pre}} \lossGeneralFunction_{\indexCodeRegularizer, \latentJoint}
    + \lossScaling_{\indexJointRegularizer, \text{pre}} \lossGeneralFunction_{\indexJointRegularizer},
\end{equation}
where $\lossGeneralFunction_{\indexCodeRegularizer, \latentJoint}$ is the default norm regularization from \cref{eqn:loss_code_reg} and $\lossGeneralFunction_{\indexJointRegularizer}$ was introduced in \cref{eqn:joint_code_reg}.
}

{\parskip=5pt
\noindent\textbf{Loss Function}: Given an object $\object_{\objectIndex, \jointIndex}$, we express our full decoder loss as} 
\begin{align}
    \label{eqn:full_loss}
    \lossGeneralFunction = &
    \lossScaling_{\indexCodeRegularizer, \latentShape} \lossGeneralFunction_{\indexCodeRegularizer, \latentShape}
    + \lossScaling_{\indexCodeRegularizer, \latentJoint} \lossGeneralFunction_{\indexCodeRegularizer, \latentJoint}
    + \lossScaling_{\indexReconstruction} \lossGeneralFunction_{\indexReconstruction} \\
    & \quad+ \lossScaling_{\indexJointType} \lossGeneralFunction_{\indexJointType}
    + \lossScaling_{\indexJointState} \lossGeneralFunction_{\indexJointState} , \nonumber
\end{align}
where $\lossGeneralFunction_{\indexCodeRegularizer, \latentVariable}$ are the shape and joint code regularization from \cref{eqn:loss_code_reg}, $\lossGeneralFunction_{\indexReconstruction}$ is the reconstruction loss introduced in \cref{eqn:loss_reconstruction}, $\lossGeneralFunction_{\indexJointType}$ and $\lossGeneralFunction_{\indexJointState}$ are the joint type and state loss.
We jointly optimize $\lossGeneralFunction$ for the latent shape and joint code as well as the network parameters of the geometry decoder and joint decoder
using ADAM \cite{Kingma2015AdamAM} for 5000 epochs.
Our new joint code regularizer loss $\lossGeneralFunction_{\indexJointRegularizer}$ introduced in \cref{eqn:joint_code_reg} is minimized at the end of each epoch separately scaled by $\lossScaling_{\indexJointRegularizer}$. All $\lossScaling$ variables are scalars to balance the different loss terms and are reported in \ifthenelse{\boolean{reftosupp}}{\cref{supp:tab:decoder_scaling_training}}{Tab.~S.1}. 


\subsubsection{Encoder}
Using \cite{laskey2021simnet} we generate a large-scale dataset in which we annotate each pixel with its respective ground truth value from the simulation as described in \cref{subsec:encoder}. For annotating the shape codes we directly use the results of our previous encoder training whereas for the joint code we use 
our inverse mapping explained in \cref{subsec:inverse_joint_decoder} to retrieve joint codes for arbitrary sampled articulation states.

\subsubsection{Inverse Joint Decoder}
\label{subsec:inverse_joint_decoder}
To solve the inverse problem, given an articulation state for which we want to retrieve a joint code, we fit polynomial functions in the learnt joint code space. With the help of this mapping, we can retrieve arbitrary joint codes which then can be combined with a shape code to reconstruct objects in novel articulation states which have not been seen during the decoder training. Additionally, the mapping provides joint code training labels for the encoder. 

We describe the full mapping as a function $\jointToCodeFunction (\jointType, \jointState) = \latentJoint$ that takes a joint type $\jointType$ and joint state $\jointState$ as input and outputs a joint code $\latentJoint$. We leverage the fact that after decoder training, we learned a joint code $\latentJoint^{\jointIndex}$ for each known training articulation state. We now define individual mappings for each joint type $\jointType$ the following way. We will treat each latent dimension $\dimensionsIndex$ separately. For each dimension $\dimensionsIndex$, we fit a polynomial function $\jointToCodeFunction^{\jointType, \dimensionsIndex} (\jointState)$ of varying degree $\polynomDegree$ through all point tuples $(\jointState^{\jointIndex}, \latentJoint^{\jointIndex}(\dimensionsIndex)) \+\forall\+ \jointIndex \in 1, \ldots, \jointCount$. 

The final function
\begin{equation}
    \jointToCodeFunction (\jointType, \jointState) = 
    \begin{bmatrix}
        \jointToCodeFunction^{\jointType, 1} (\jointState) \\
        \vdots \\
        \jointToCodeFunction^{\jointType, \dimensionsJointSpace} (\jointState)
    \end{bmatrix}
\end{equation}
is then given by evaluating the polynomials individually and stacking the results into a vector.
The exact choice of $\polynomDegree$ is not important as long as the amount of joint codes to fit to is much higher than the potential dimensions of the polynomial $\polynomDegree \ll \jointCount$. Thus, we fixed $\polynomDegree = 5$ for all of our experiments. 
A visualization of our learned latent joint space and the fitted polynomials is given in \ifthenelse{\boolean{reftosupp}}{\cref{supp:subfig:carto_latent_joint_space}}{Fig.~S.3a}.

\section{Experimental Setup}
\label{sec:experiments}
\begin{figure}[t]
    \centering 
    \hspace{-.04\columnwidth}
    \includegraphics[width=1.025\columnwidth]{results/VOC/figures/pareto_example.pdf}
    \caption{\textbf{Selecting models for evaluation.} For each configuration, we evaluate every model at every checkpoint and measure its performance across various metrics (\fone, \epg, \iou) on the validation set; \ie every point in the left graph corresponds to one model (for \bcos models optimized via the \epgloss loss at the input layer). Instead of evaluating a single model on the test set, we evaluate \emph{all Pareto-dominant} models, as indicated in the center and right plot.
    % \moritz{Did we not update the results to be consistent with this? I distinctly remember creating the plots for this. (The Pareto front here as a lot more points than those in the result figures...)}
    }
    \label{fig:pareto_example}
\end{figure}

In this section, we describe our experimental setup
and how we select the best models across metrics. {Full training details can be found in the supplement.} We evaluate across the full sweep of combinations of choices for each category, and discuss our results in \cref{sec:results}. 

\myparagraph{Datasets:} We evaluate on \voc \citeMain{everingham2009pascal} and \coco \citeMain{lin2014microsoft} for multi-label image classification. {In \cref{sec:results:waterbirds}, to understand the effectiveness of model guidance in mitigating spurious correlations, we also evaluate on the synthetically constructed Waterbirds-100 dataset \citeMain{sagawa2019distributionally,petryk2022guiding}, where landbirds are perfectly correlated with land backgrounds on the training and validation sets, but are equally likely to occur on land or water in the test set (similar for waterbirds and water). With this dataset, we evaluate model guidance for suppressing undesired features.}

\myparagraph{Attribution Methods and Architectures:} As described in \cref{sec:method:attributions}, we evaluate with \ixg \citeMain{shrikumar2017learning}, \intgrad \citeMain{sundararajan2017axiomatic}, \bcos \citeMain{bohle2022b}, and \gradcam \citeMain{selvaraju2017grad} using models with a \resnet \citeMain{he2016deep} backbone. For \intgrad, we use an \xdnn \resnet \citeMain{hesse2021fast} to reduce the computational cost, and a \bcos \resnet for the \bcos attributions. We optimize the attributions at the input and final layer\footnote{As typically used in \ixg (input) and \gradcam (final) respectively.}; for intermediate layer results, see supplement. Given the similarity of the results between \gradcam and \ixg, and since \bcos attributions performed better than \gradcam for \bcos models, we show \gradcam results in the supplement. 
All models were pretrained on \imagenet \citeMain{imagenet}, and model guidance was performed starting from a baseline model fine-tuned on the target dataset.

\myparagraph{Localization Losses:} As described in \cref{sec:method:losses}, we compare four localization losses in our evaluation: (i) \energyloss, (ii) \loneloss \citeMain{gao2022aligning,gao2022res}, (iii) \ppceloss \citeMain{shen2021human}, and (iv) \rrrloss (cf.~\cref{sec:method:losses}, \citeMain{ross2017right}).

\myparagraph{Evaluation Metrics:} As discussed in \cref{sec:method:metrics}, we evaluate both for classification and localization performance of the models. For classification, we report the F1 scores, similar results with \map scores can be found in the supplement. For localization, we evaluate using the \epg and \iou scores.

\myparagraph{Selecting the best models:} As we evaluate for two distinct objectives (classification and localization), it is non-trivial to decide which models to select during training. \Eg, a model that provides the best classification performance might provide significantly worse localization performance than a model that provides slightly lower classification performance but much better localization. Finding the right balance and deciding which of those models in fact constitutes the `better' model depends on the preference of the end user. 
Hence, instead of selecting models based on a single metric, we select the set of Pareto-dominant models \citeMain{pareto1894massimo,pareto2008maximum,backhaus1980pareto} across three metrics---F1, \epg, and \iou---for each training configuration, as defined by a combination of attribution method, layer, and loss. Specifically, as shown in \cref{fig:pareto_example}, we train for each configuration using three different choices of $\lambda_\text{loc}$, and select the set of Pareto-dominant models among all checkpoints (epochs and $\lambda_\text{loc}$). This provides a more holistic view of the general trends on the effectiveness of model guidance for each configuration.
% \section{Results}

% 这个部分展示了我们结果的性能优势,首先,我们通过多组对照试验确定了最优的架构参数,然后我们基于UNet与SwinUNETR预训练了通用模型与任务特定模型,在结果表中,*表示通用模型,否则为任务特定模型。同时,我们使用了多种来源的数据,涉及了不同模态,不同器官和不同的分割目标来验证HybridMIM的鲁棒性。此外,我们还验证了不同有标签数据比例下,HybridMIM依然能够有较高的性能优势。最后,我们还进行了消融实验,验证了HybridMIM中不同模块的有效性。
This section demonstrates the significance of our proposed HybridMIM method. 
%%
First, we make comparison with the current state-of-the-art approaches from four aspects: downstream segmentation performance (quantitative and qualitative), annotation cost reduction, and pre-training speed. 
%%
We then conduct ablation experiments to explain how to determine the optimal architectural parameters, and illustrate the contribution of each component to the performance of HybridMIM.

%This section demonstrates the performance advantage of our results. First, we determine the optimal architectural parameters by multiple controlled trials, and then we pre-train the generic and task-specific models based on UNet with SwinUNETR. In the result table, * indicates the generic model, otherwise the task-specific model. Also, we use data from multiple sources involving different modalities, organs, and segmentation targets to validate the robustness of HybridMIM. In addition, we verify that HybridMIM can still have high-performance advantages with different scales of labeled data. Finally, we also conduct ablation experiments to validate the effectiveness of different modules in HybridMIM.

\begin{table}[th]
    %\centering
    % 其中MSD Liver数据集需要分割肝脏和对应的肿瘤。MSD Spleen数据集需要分割脾脏。我们使用Dice和HD95来评估不同对比方法的性能。无论基于UNet架构还是SwinTransformer架构,MP-SSL方法都对其有很高的性能提升,并实现了state-of-the-art的结果。
    %\vspace{-2mm}
    \caption{The MSD Liver dataset requires segmentation of the liver and the corresponding tumor. and the MSD Spleen dataset requires segmentation of the spleen.}
    \label{tab:msd_segmentation}
    \renewcommand\arraystretch{1.3}
    \setlength\tabcolsep{3pt}%调列距
    \resizebox{\columnwidth}{!}{
    \begin{tabular}{c | c c c c c c | c c c}

    \hline
    Organ & \multicolumn{6}{c}{Liver} & \multicolumn{2}{c}{Spleen} \\
    \hline
    Metrics & Dice & Dice & Dice & HD & HD & HD & Dice & HD \\
     & liver & tumor & Avg & liver & tumor & Avg &  &  \\
    \hline
    SegresNet & 95.53 & 48.26 & 71.90 & 0.81 & {15.31} & 25.31 & 94.10 & 0.5\\
    UNETR & 93.07 & 33.59 & 63.33& 1.26 & 30.50 & 15.88 & 94.04 & 0.58\\
    SwinUNETR & 95.14 & 45.11 & 70.13 & 0.89 & 21.31 & 11.11 & 94.61 & 0.25\\
    \hline
    ModelGen & 95.22 & {52.53} & 73.87 & 0.67 & 18.83 & 9.75 & 94.43 & 0.63 \\
    TransVW & 95.67 & 52.10 & 73.88 & 0.60 & 21.36 & 10.98 & 95.55 & 0.41 \\
    UNetFormer* & 95.50 & 49.81 & 72.65 & {0.52} & 21.72 & 11.12 & 95.36 & 0.25 \\
    UNetFormer & 95.83 & 50.25 & 73.04 & 0.43 & 18.66 & 9.55 & 95.59 & 0.30 \\
    
    \hline
    HybridMIM*(Swin) & 95.45 & 50.19 & 72.82 & 0.69 & \textbf{15.21} & \textbf{7.95} & 95.87 & 0.25\\
    HybridMIM*(UNet) & \textbf{96.35} & 52.38 & \textbf{74.36} & 0.59 & 19.98 & 10.28 & {95.94} & \textbf{0.20} \\
    \hline
    HybridMIM(Swin) & 95.86 & 50.45 & 73.16 & 0.42 & 17.36 & 8.89 & 95.97 & 0.20 \\
    HybridMIM(UNet) & 95.70 & \textbf{52.81} & 74.26 & \textbf{0.27} & 18.25 & 9.26 & \textbf{96.05} & \textbf{0.20} \\
    \hline 
    \end{tabular}
    }
    \vspace{-2mm}
\end{table}




\subsection{Quantitative Comparison to Previous Methods} 
%
\textbf{BTCV multi-organ segmentation.} The multi-organ segmentation results are listed in Table \ref{tab:btcv_segmentation}, in which
the first, second, and third best dice scores are marked in red, blue, and green colors, respectively. 
%%
Among the comparative methods, we can see that those with self-supervised pre-training generally achieve averagely better results than those fully supervised methods. 
%%
TransVW obtains the best average Dice of 82.27\%,  
%%
while for UNetFormer, its generic pre-trained model presents an average Dice of 82.44\%, outperforming the task-specific pre-trained model UNetFormer* by 0.26\%. 

% 与其他对比方法相比,我们的基于UNet和SwinTransformer架构的方法均取得了有竞争力的结果。红色,蓝色,绿色分别代表最高的dice得分,第二高的dice得分与第三高的dice得分。可以清楚的发现,基于SwinUNETR架构的任务特定模型Swin(HybridMIM)在7项指标中均位于前三名,实现了82.41%的Dice平均值。而基于UNet架构的通用预训练模型UNet*(HybridMIM),在4项指标中位于前两名,相比于其他方法实现了最高的平均Dice,83.00。在BTCV多器官分割任务中,通用预训练模型的性能均高于任务特定预训练模型。
In comparison, our methods on both UNet and SwinTransformer architectures outperform most SOTA methods, and the generic pre-trained models get better performance than their task-specific pre-trained counterparts.  
%%
Specifically, the generic pre-trained model HybridMIM(UNet) presents the highest average Dice of 83.00\%,
%We can find that the task-specific model Swin (HybridMIM) based on SwinUNETR architecture is in the top three in all seven metrics, achieving an 82.41\% Dice average. 
% 拿性能最好的UNet*(HybridMIM)来说,它实现了最高的83.0%的平均Dice,比表现较好的同样在通用数据集上预训练的UNetFormer*模型提升了0.56%。并且UNet*(HybridMIM)在13个分割目标中有9个目标的分割结果均优于UNetFormer*。
which is 0.56\% better than the best SOTA model UNetFormer, and outperforms it in 9 out of 13 segmentation targets.
% 并且基于SwinUNETR架构的任务特定预训练模型在Lag器官上分割效果明显优于其他对比方法,达到了68.47%的dice值,比第二名UNETR高出1.82%。而基于UNet架构的任务特定预训练模型在Gall器官上分割效果显著,达到了 the dice of 78.67%,而第二名UNetFormer与第三名Segresnet方法的dice均没有超过76%。
%
Furthermore, the task-specific pre-trained model HybridMIM*(Swin) segmented significantly better than the other methods on the Lag organ, reaching the Dice of 68.47\%, which is 1.82\% higher than the second place UNETR, while HybridMIM*(UNet) reports a significantly better result on the Gall organ, reaching a Dice of 78.67\%. 
%In comparison, neither the second-place UNetFormer nor the third-place Segresnet method had more than 76\% Dice.

% 肝脏与肝脏肿瘤分割结果被展示在表3的左侧。加粗字体表示最优的指标。可以清晰的看到,我们提出的基于UNet架构的任务特定预训练模型UNet(HybridMIM)在肝脏的分割上有最好的Dice of 96.35%,比第二名TransVW提升了0.68%。同时其在肝脏肿瘤的分割中达到了Dice of 52.38%,仅次于ModelGen方法的52.53%。此外,UNet(HybridMIM)也实现了两个分割指标的最好的平均Dice,为74.36,比第二名TransVW方法提升了0.48%。
\textbf{Liver and liver tmuor segmentation.} As shown in Table \ref{tab:msd_segmentation}, 
%The bolded font indicates the best metrics.
our task-specific pre-trained model HybridMIM*(UNet) achieves the best average Dice of 74.36\%, with an improvement of 0.48\% over the second-place TransVW method.
Furthermore, it reports the best Dice of 96.35\% for the segmentation of the liver, which is 0.68\% better than the second place TransVW; and obtains a Dice of 52.38\% in the segmentation of liver tumors, only slightly lower than the second place ModelGen method with 52.53\%. 
% 对于HD95分割指标,基于UNet(HybridMIM)在肝脏的分割中位于第二名,HD95结果为0.59,略高于UNetFormer方法的0.52。在肝脏肿瘤的分割中为第三名,HD95为19.98。
For the HD95 segmentation metric, the HybridMIM*(UNet) gets an average HD95 of 10.28, ranked in the third place.
%is in second place in the segmentation of the liver with an HD95 result of 0.59, slightly higher than the UNetFormer method of 0.52. It was in third place in the segmentation of liver tumors with an HD95 result of 19.98, and the average HD95 was also in third place.
% 同时,Swin(HybridMIM)总体来说在HD95指标上表现更好。其在肝脏肿瘤的分割上拥有最好的HD95,为15.21,并且其在肝脏与肝脏肿瘤两个分割目标上实现了最好的的平均HD95,为7.95,比第二名ModelGen方法降低了1.8。相比于没有经过预训练SwinUNETR方法,Swin(MP-SSL)有更加明显的提升。其在肝脏与肝脏肿瘤的平均Dice得分达到了72.82%,比SwinUNETR方法提升了2.17%。
%Meanwhile, the Swin(HybridMIM) performed better overall on HD95 metrics. 
%It achieves the best HD95 of 15.21 for liver tumor segmentation and the best average HD95 of 7.95 for liver and liver tumor segmentation targets, which is 1.8 lower than the ModelGen method in second place. 
In addition, compared to the SwinUNETR method without pre-training, both HybridMIM*(Swin) and HybridMIM(Swin) which employ SwinUNETR as the underlying architecture, have more significant improvements in all the metrics. 
%%
%HybridMIM*(Swin) and HybridMIM(Swin) get an average Dice score of 72.82\% and 73.16\%, 2.69\% and 3.03\% higher than the SwinUNETR method, respectively.



% 脾脏的分割结果被展示在表3的右侧。可以看到,基于UNet与SwinUNETR架构的HybridMIM均表现出了优秀的性能,无论是在Dice还是在HD95上。基于UNet*(HybridMIM)获得了 state-of-the-art 的Dice与HD95,分别为96.05与0.20,在Dice得分上相比于同样表现较好的对比方法TransVW提升了0.50%,比基于Transformer架构的UNETR提升了2.1%。此外,Swin*(HybridMIM)实现了95.97%的Dice与0.20的HD95,仅次于UNet(HybridMIM)。
\textbf{Spleen segmentation.} The spleen segmentation results are listed on the right side of Table~\ref{tab:msd_segmentation}.
%%
The HybridMIM based on both UNet and SwinUNETR architectures presented improved performance, both on Dice and HD95. 
%%
HybridMIM(UNet) obtains Dice and HD95 with 96.05 and 0.20, respectively, improving the Dice score by 0.50\% compared to TransVW, and by 2.1\% compared to UNETR. 
%%
%In addition, Swin*(HybridMIM) achieves 95.97\% Dice and 0.20 HD95, second only to UNet (HybridMIM).
% 值得注意的是,SwinUNETR方法的Dice得分为94.61,而我们提出的通用预训练模型Swin* (HybridMIM)方法则达到了95.97的Dice得分,实现了1.36%的提升。通过我们提出的Hybrid的多层次自监督学习方式首先学习丰富的3D脾脏数据的空间解剖学特征,然后通过迁移学习在下游分割任务中训练,可以明显的提升原模型的效果。
Among the fully supervised methods, SwinUNETR gets the best Dice score of 94.61, and HD 0.25.
%%
Our generic pre-trained model HybridMIM(Swin) further improves SwinUNETR to achieve a Dice score of 95.97, realizing an increase of 1.36\%.
%%
%The original model can significantly improve by learning the spatial anatomical features of the rich 3D spleen data through our proposed Hybrid's multi-level self-supervised learning approach and then training it in the downstream segmentation task through transfer learning.

\begin{figure*}[tbp] %H为当前位置,!htb为忽略美学标准,htbp为浮动图形
\vspace{-4mm}
\centering %图片居中
\includegraphics[width=\textwidth]{figures/visual_1.pdf} %插入图片,[]中设置图片大小,{}中是图片文件名
% Ours为Swin*(HybridMIM)方法,三行视觉比较结果分别为BraTS2020,Liver和BTCV。我们提出的方法更够更好的分割细微的病灶(第一行),并且分割的完整度更高(第二行,第三行)。
\vspace{-3mm}
\caption{Qualitative visualizations of the proposed HybridMIM and baseline methods. "Ours" is the HybridMIM(Swin) method. The three rows of visual comparison results are from BraTS2020, Liver, and BTCV datasets. Our proposed method is better for segmenting tiny lesions (first row) and has higher segmentation integrity (second row, third row).} %最终文档中希望显示的图片标题
\label{fig:visual} %用于文内引用的标签
\end{figure*}

% 基于BraTS2020数据的脑胶质瘤的分割结果被展示在表4中。我们使用Dice来评测不同方法的性能。其中WT,TC,ET分别代表了全部肿瘤,肿瘤核心,增强肿瘤,Avg代表3个分割目标的Dice均值。
\textbf{Brain tumor segmentation.} The segmentation results of gliomas for BraTS2020 dataset are summarized in Table \ref{tab:brats_segmentation}. 
%We use Dice to evaluate the performance of different methods. 
WT, TC, ET represent whole tumor region, tumor core, and enhanced tumor region, respectively, and Avg is the Dice mean of the three segmentation targets.
% 我们提出的Swin(MP-SSL)方法实现了一个state-of-the-art的分割结果并且在WT,TC,ET三个分割目标中均达到了最优,分别为91.48%,86.88%,80.81%。相比于没有加入预训练的SwinUNETR方法,Swin(MP-SSL)在三个分割目标中均有较大幅度的提升,分别提升了1.4%,1.69%,0.8%,且三个分割目标的平均Dice得分比第二名TransVW方法提升了0.59%。
Our task-specific pre-trained model HybridMIM*(Swin) reports the best in WT, ET, and Avg with 91.48\%, 80.81\%, and 86.39\% respectively.
% 对比没有预训练的SwinUNETR方法,Swin(HybridMIM)与Swin* (HybridMIM)在三个分割目标上均有较大的提升,相比SwinUNETR,平均的Dice分别提升了1.3%, 1.24%。
%Compared with the SwinUNETR method without pre-training, Swin(HybridMIM) and Swin* (HybridMIM) show a considerable improvement in all three segmentation objectives, with an average Dice improvement of 1.3\%, 1.24\%, respectively, compared to SwinUNETR.
% 此外,UNet方法经过预训练后,也有了非常明显的提升,像表中最后一行展示的那样,UNet* (HybridMIM)方法在三个分割目标7分别实现了90.41%, 86.49%, 80.61%的Dice得分,相比于同样为UNet架构的ModelGen,三个分割指标的平均Dice提升了0.12%。以上的结果充分证明了MP-SSL方法良好的迁移学习和模型泛化能力。
%%In addition, the UNet method shows a significant improvement after pre-training, as shown in the last row of the table. 
As for UNet as the underlying architecture, the generic pre-trained model HybridMIM(UNet) achieves Dice scores of 90.41\%, 86.49\%, and 80.61\% for the three segmentation targets, respectively. Compared with ModelGen which is also built on UNet, we has the average Dice improved by 0.12\%. 
%%
%The above results fully demonstrate the good transfer learning and model generalization ability of the HybridMIM method.
It is also noted that on BraTS2020 dataset, the task-specific pre-trained mode gets better performance than the generic pre-trained mode. 

\begin{table}[t]
    \centering
    % BraTS2020数据集包含四个模态,三个分割目标。我们选择UNet和SwinTransformer作为backbone,分别于有监督学习方法跟自监督学习方法对比,结果展示了UniLearn对不同架构的有效性。
    \caption{Quantitative comparison on BraTS 2020 dataset, which contains four modalities and three segmentation targets. }
    % \vspace{-3mm}
    \label{tab:brats_segmentation}
    \renewcommand\arraystretch{1.3}
    \setlength\tabcolsep{10pt}%调列距
    \resizebox{0.48\textwidth}{!}{
    \begin{tabular}{c | c c c c}
    \hline
    Methods & WT & TC & ET & Avg\\
    \hline
    SegresNet & 90.04 & 85.08 & 78.81 & 84.64 \\
    
    UNETR & 89.92 & 84.79 & 79.51 & 84.74\\
    SwinUNETR & 90.08 & 85.19 & 80.01 & 85.09\\
    \hline
    ModelGen & 90.60 & 86.59 & 79.95 & 85.71\\
    TransVW & 90.96 & 86.26 & 80.20 & 85.80 \\
    UNetFormer* & 90.93 & 86.17 & 79.97 & 85.69\\
    UNetFormer & 90.71 & 86.22 & 80.19 & 85.71\\
    \hline
    HybridMIM*(Swin) & \textbf{91.48} & {86.88} & \textbf{80.81} & \textbf{86.39} \\
    HybridMIM*(UNet) & 90.62 & 86.28 & 80.17 & 85.69\\
    \hline
    HybridMIM(Swin) & 90.95 & \textbf{87.34} & 80.71 & 86.33\\
    HybridMIM(UNet) & 90.41 & 86.49 & 80.61 & 85.83 \\
    \hline
    \end{tabular}
    }
    \vspace{-2mm}
\end{table}



\begin{figure}[htbp] %H为当前位置,!htb为忽略美学标准,htbp为浮动图形
\centering %图片居中
\vspace{-2mm}
\includegraphics[width=0.8\columnwidth]{figures/data_proportion.pdf} %插入图片,[]中设置图片大小,{}中是图片文件名
% 不同有标签数据规模对迁移学习结果的影响。我们分别选择了BraTS2020数据集中训练数据的10%,20%,40%,60%,80%,100%,验证在不同自监督学习方法的迁移学习能力。
\caption{Effect of different labeled data sizes on migration learning results. We selected 10\%, 20\%, 40\%, 60\%, 80\%, and 100\% of the training data in the BraTS2020 dataset to verify the transfer learning ability in different self-supervised learning methods.} %最终文档中希望显示的图片标题
\label{fig:data_proportion}
\vspace{-2mm}
%用于文内引用的标签
\end{figure}

\vspace{-2mm}
\subsection{Qualitative Comparison to Previous Methods}

% 为了更加直观的对比不同方法的分割结果,我们选择Swin*(HybirdMIM)和其他六个性能较好的对比方法在BraTS2020,Liver和BTCV数据集上进行视觉比较。
To compare the segmentation results of different methods more intuitively, we choose HybridMIM(Swin) and four comparative methods with better performance on the BraTS2020, Liver, and BTCV datasets for visual comparison.
% 像Fig. 6. 所展示的,Swin*(HybridMIM)能够提升病灶识别的准确度和完整度,并且针对细微的病灶依然可以高效的识别出来。模型经过HybridMIM方法预训练后,对局部区域的感知能力更强。
As shown in Figure~\ref{fig:visual}, HybridMIM(Swin) can improve the accuracy and completeness of lesion identification,  and still perceive subtle lesions. 
%The model is pre-trained by the HybridMIM method and better perceives localized regions.
%在Fig. 6. 的第一行,可以明显看出我们的方法相比于其他对比方法可以更加精准的分割微小的病灶。在Liver数据集中(Fig. 6.第二行),Swin*(HybridMIM)分割的完整性更高,没有出现像其他对比方法中的分割区域不连续的情况。同时,在BTCV数据集中的可视化结果中,我们的方法的分割结果包含的空洞更少,与其他对比方法相比,有较高的完整度。 
To be specific, for brain tumor in BraTS2020 (the first row of Figure~\ref{fig:visual}), our method segments the whole tumor with more accurate boundary, while the comparative methods all enlarge the tumor region. 
%%
In the liver segmentation task (the second row), we can clear see that the comparative methods generate obvious discontinuity in the segmented areas. Especially UNETR and SegResNet fail to detect the lower part of the liver, while the detected liver region from our method exhibits a clearly higher integrity. 
%%
For the BTCV dataset, TransVW, UNetFormer, SiwnUNETR generates small holes in stomach; ModelGen even is subjected to a much large missing detected part. In contrast, our segmentation result is more close to the ground truth.

\vspace{-2mm}
\subsection{Reduce Manual Labeling Efforts}
% 为了验证随着有标签数据比例逐渐降低,HybridMIM方法相比于其他自监督学习方法依然能保持良好的迁移学习能力,我们选择UNetFormer与TransVW作为对比方法,BraTS2020作为下游分割任务数据集,采用10%,20%,40%,60%,80%,100%的数据比例进行对比实验。
To evaluate the transfer learning ability with annotation scarcity challenge in medical imaging, we conduct the experiment of finetuning using a subset of BraTS2020 data.  
%%
Figure~\ref{fig:data_proportion} demonstrates the comparison results between HybridMIM(Swin), TransVW and UNetFormer. 
%%
%In order to verify that as the proportion of labeled data gradually decreases, the HybridMIM method still maintains good transfer learning ability. We choose UNetFormer and TransVW as the comparison methods and BraTS2020 as the downstream segmentation task dataset and use 10\%, 20\%, 40\%, 60\%, 80\%, and 100\% data proportions for comparison experiments.
% Fig. 4. 展示了减少有标签数据比例的实验结果。实验结果表明,当有标签数据比例降低至60%时,UNetFormer与TransVW方法在BraTS2020分割数据集上的迁移学习能力明显降低。而通过HybridMIM方法预训练的通用模型SwinUNETR在有标签数据比例为20%时依然能够实现0.825的平均Dice。
%Fig. \ref{data_proportion} shows the experimental results of reducing the proportion of labeled data.
It is clear that the generic pre-trained model HybridMIM(Swin) presents the best performance when using the same portion of labelled data.
%%
On employing 20\% labelled data, HybridMIM(Swin) already achieves an average Dice of 82.55\%, with 1.42\% and 3.17\% higher than UNetFormer and TransVW, respectively.  
%%
The Dice 85.24\% can be achieved by using HybridMIM(Swin) with 60\% labelled data, while UNetFormer requires about 80\% data and TransVW requires nearly 90\% data.
%%
%%On employing 40\% labelled data, HybridMIM(Swin) obtains an average Dice of ??, even higher than UNetFormer and TransVW employing 60\% labelled data. 
 
%The experimental results show that the transfer learning ability of UNetFormer and TransVW methods declined significantly on the BraTS2020 segmented dataset when reducing the proportion of labeled data to 60\%. In contrast, Swin, a generic model pre-trained by the HybridMIM method, still achieves an average Dice of 0.825 when the proportion of labeled data is 20\%.
% 此外,当有标签数据的比例相同时,Swin*(HybridMIM)较其他对比方法均有明显的性能优势。并且Swin*(HybridMIM)需要更少的数据便可以实现其他对比方法需要更多数据才能实现的性能,例如Swin*(HybridMIM)利用60%的有标签数据达到的迁移学习的性能,UNetFormer需要80%的数据,TransVW需要90%的数据。
%In addition, the HybridMIM(Swin) has a significant performance advantage over other comparison methods when the proportion of labeled data is the same. For example, the HybridMIM(Swin) achieves transfer learning performance with 60\% of labeled data, while UNetFormer requires 80\% of data and TransVW requires 90\% of data.

\vspace{-2mm}
\subsection{Pre-training Speed Comparison}
% 在自监督学习的过程中,由于无标签数据的数据量通常较大,因此训练速度是一个影响自监督学习方法的非常重要的因素。MP-SSL通过灵活的选择局部的一级区域重建来提升预训练速度。我们与其他的自监督学习方法进行对比,像图3(d)中展示的那样,我们分别列举了基于UNet与SwinTransformer架构的MP-SSL方法与其他自监督方法的时间消耗。
In self-supervised learning, the training speed is a notable factor to consider, because the unlabeled data scale is usually large especially in the generic training mode. 
%%
Figure~\ref{fig:pretraining_time} demonstrates the time consumption of those self-supervised methods in the pre-training stage on BraTS2020 dataset.
%%
%The HybridMIM enhances the pre-training speed by flexibly selecting local first-level region reconstruction. We compare with other self-supervised learning methods, as shown in Fig. \ref{pretraining_time}, and we enumerate the time consumption of the HybridMIM method based on UNet and SWinUNETR architectures, respectively, with other self-supervised methods.
% 值得注意的是,为了更加公平的进行对比,我们对比了每个自监督学习方法运行一步的平均时间消耗。其中一步内包含了前向传播,反向传播,更新参数,而不包含数据读取,数据预处理等时间消耗不确定的操作。
For a fair comparison, we count the average time of running one step for each method, which contains forward prediction, backward propagation, and updating network parameters, but does not include data reading and preprocessing operations.
%

\begin{figure}[htbp] %H为当前位置,!htb为忽略美学标准,htbp为浮动图形
\centering %图片居中
\vspace{-2mm}
\includegraphics[width=0.8\columnwidth]{figures/time-1.pdf} %插入图片,[]中设置图片大小,{}中是图片文件名
% 不同自监督学习方法预训练时间消耗对比。横坐标为不同自监督学习方法和不同重建大小的HybridMIM方法,128是全局重建大小,96是我们提出的局部重建方式。纵坐标表示预训练时每步的时间消耗。
\caption{Comparison of pre-training time consumption for different SSL methods. 
%The horizontal coordinates are different self-supervised learning methods. 
``Not partial'' denotes that the partial region prediction scheme is not used.
%, which spend more time in pre-training. The vertical coordinate indicates the time consumption of each step during pre-training.
} %最终文档中希望显示的图片标题
\label{fig:pretraining_time} %用于文内引用的标签
\vspace{-2mm}
\end{figure}

% 因此,由图3(d)可以看出,TransVW与ModelGenesis方法时间消耗最多。Swin(HybridMIM)当使用(128,128,128)作为重构区域时,由于其包含更多的损失函数,因此时间消耗高于类似架构的UNetFormer方法。但是随着我们将需要重构的局部区域降低为(96,96,96),预训练时间大幅度降低,相比于TransVW与ModelGen方法,预训练速度提升48%,相比于UNetFormer方法,预训练速度提升36%。
As Figure \ref{fig:pretraining_time} shows, the TransVW and ModelGenesis methods with the same underlying architecture have the highest time consumption, both of which are 1.42s per step. 
%%
HybridMIM(Swin), when predicting all the masked sub-volumes (denoted as ``Not partial''; see the fourth bar), has a higher time consumption than the UNetFormer method. 
%%
It is because that although they have the similar underlying architecture, HybridMIM(Swin) involves  more loss functions. 
%%
On the other hand, when we apply the partial region prediction, the pre-training time of HybridMIM(Swin) decreases dramatically, in which the speedup is 48\% with respect to TransVW and ModelGen, and 36\% against UNetFormer.


% 类似的,当使用UNet(HybridMIM)方法时,此时虽然由于所使用的UNet本身的结构特殊性,有更低时间消耗,但通过选择局部区域重建,训练速度依然有显著的提升。像表3d中展示的那样,当使用(128,128,128)大小作为重构尺寸时,每步时间消耗为1.03s,而当使用(96,96,96)大小时,每步时间消耗降低0.35s,相比TransVW和ModelGen方法,预训练速度快52%,相比UNetFormer方法,预训练速度加快40%。
When using the HybridMIM(UNet) method, there is a lower time consumption due to the structural simplicity of the UNet (see the rightmost two bars). 
%%
The partial region prediction enables it to get a significant improvement in the pre-training speed, with the time consumption per step reduced by 0.35s.
%%
HybridMIM(UNet) achieves a pre-training  speed of 0.68s, 52\% faster than the TransVW and ModelsGenesis methods, and 40\% faster than the UNetFormer method.
%%
% It is worthy noting that the pre-training speed is close to the training speed in the finetuning, despite that the later has fewer losses to compute.
% %%
% Therefore, our method can also have faster time performance in the finetuning stage.



\vspace{-2mm}
\subsection{Ablation Study}
\subsubsection{Selection of the optimal architecture settings}
%\vspace{-4m}
\begin{figure}[htbp] %H为当前位置,!htb为忽略美学标准,htbp为浮动图形
\vspace{-4mm}
\centering %图片居中
\includegraphics[width=0.48\textwidth]{figures/architecture_3.pdf} %插入图片,[]中设置图片大小,{}中是图片文件名
% 不同架构参数对迁移学习性能与预训练时间的影响。(a)中横坐标中a-b-c分别代表一级区域大小,二级区域大小,重建区域大小。纵坐标表示在BraTS2020数据集迁移学习能力(三个分割目标的Dice平均值)。(b)中右侧纵坐标表示预训练时每个step消耗的时间。我们首先通过(a)确定最优的一级区域与二级区域,32-16-128迁移学习效果最好。之后,我们通过(b)改变重建区域的大小,兼顾性能与时间选择最优的架构参数设置。
\caption{Effect of different architecture parameters on transfer learning performance and pre-training time. The a-b-c in the horizontal coordinates in (a) represent the first-level, second-level, and reconstructed region sizes, respectively. The vertical coordinates represent the transfer learning capability in the BraTS2020 dataset (average Dice for the three segmentation targets). The right vertical coordinate in (b) indicates the time consumed per step during pre-training. The two red dashed boxes indicate the optimal architectural parameters we choose in (a) and (b), respectively. } %最终文档中希望显示的图片标题
%% 两个红色虚线框分别表示了我们在(a)和(b)中选择的最优架构参数。
%% We determine the optimal first-level and second-level regions by (a), and 32-16-128 migration learning works best. After that, we change the size of the reconstructed region by (b) choosing the optimal architecture parameter settings considering the performance and time.
\label{fig:pretraining_setting}
\vspace{-2mm}
%用于文内引用的标签
\end{figure}
%\label{pretraining_settings}

% 为了选择一个更好的架构参数,我们进行了多组对照实验。我们选择UNet架构预训练多组通用模型,see Fi. 3. 横坐标架构设置a-b-c中,a表示一级区域的大小,b表示二级区域的大小,c表示重建大小。纵坐标为通用模型在BraTS2020数据集中finetuning的Dice指标。

%%
In order to choose an optimal architecture setting, we conduct a multigroup control experiment. 
%%
We choose the UNet architecture to pre-train the possible settings (see Figure~\ref{fig:pretraining_setting}), where the three numbers under each bar represent the first-level sub-volume size, the second-level patch size, and the region size for partial region prediction.  
%%
The left vertical coordinates are the Dice metrics of finetuning the generic pre-trained model on the BraTS2020 dataset.
% Fig. 3. (b)中右侧纵坐标为每个step的时间消耗。
The right vertical coordinate in Figure~\ref{fig:pretraining_setting} (b) is the time consumption of each pre-training step.
% 像Fig. 3.(a)中所展示的那样,我们固定预训练的重构大小为128,选取了64-32,64-16,32-16,32-8四组参数预训练通用模型,之后在BraTS2020分割任务中进行finetuning,结果显示,32-16-128的参数设置表现最好,实现了最好的Dice。

As Figure~\ref{fig:pretraining_setting} (a) shows, we first fix the region size for partial region prediction to be 128, select four sets of parameters (64-32, 64-16, 32-16, and 32-8) for sub-volume and patch sizes.
%, and later perform finetuning in the BraTS2020 segmentation task. 
The results show that the parameter setting of 32-16-128 performs the best and achieves the best Dice of 85.79\%.

% 之后,我们选择32-16参数设置,逐步减小重构大小,see Fig. 3. (b),实验结果展示,重建大小由128降低到96时,每个step的时间由1.03s降低至0.68s。下游分割任务的Dice指标由0.875降低至0.860。当重建大小继续降低至64时,每个step的时间为0.50s,Dice指标为85.38。为了实现更快的预训练速度并使性能影响降低,我们选择32-16-96作为我们的架构参数设置。^^
Afterwards, we fix the optimal sub-volume and patch sizes (32-16), and gradually decrease the reconstruction region size; see Figure~\ref{fig:pretraining_setting} (b). 
%%
We can see that with a smaller reconstruction region size, the Dice score decreases a little bit, while the time performance reduces greatly. 
%%
For instance, when reducing the reconstruction size from 128 to 96, the Dice score for the downstream segmentation task decreases from 85.79\% to 85.57\%, and the time per step decreases from 1.03s to 0.68s. 
%when the reconstruction size decreases to 64, the time per step is 0.50s, and the Dice metric is 85.38. The Dice metric is 85.38 for 0.50s. 
Considering the trade-off between the segmentation accuracy and pre-training speed, we choose 32-16-96 as our architecture parameters for the case that the input sample has a size of $128\times128\times128$ (BraTS2020 dataset).
%%
Taking this experiments as guidance, we use an architectural parameter setting of 32-16-64 for the case that the input sample has a size of $96\times96\times96$ (BTCV, MSD Liver and MSD Spleen).

% 我们分别使用了UNet与SwinTransformer作为backbone,在BraTS2020数据集上通过消融实验充分的验证了我们提出的每个模块的有效性。实验结果被展示在表5中。Loss单元格包含五个不同的损失函数,分别为LR(local reconstruction), Num(number), Loc(location), Consis(consistency), CL(contrastive learning),其中LR代表了像素层次的3D医学图像表征的学习,Num,Loc,Consis代表了区域层次的表征学习,而CL代表了样本层次的学习。我们验证了MP-SSL在不同层次上的自监督学习对下游分割任务的性能提升。
\subsubsection{Efficiency of Self-Supervised Objectives}

We comprehensively validate the effectiveness of our modules through ablation experiments on the BraTS2020 dataset. 
%%
The experimental results using the generic pre-training mode are presented in Table~\ref{tab:ablation}. 
%%
We have five loss functions, namely $\mathcal{L}_{\mathrm{PR}}$ (partial region prediction), $\mathcal{L}_{\mathrm{Num}}$ (number prediction), $\mathcal{L}_{\mathrm{Loc}}$ (location prediction), $\mathcal{L}_{\mathrm{Con}}$ (consistency between number and location prediction), and $\mathcal{L}_{\mathrm{CL}}$ (contrastive learning).
%%
$\mathcal{L}_{\mathrm{PR}}$ facilitates the learning of 3D medical image latent representations at the pixel level; the combination of $\mathcal{L}_{\mathrm{Num}}$, $\mathcal{L}_{\mathrm{Loc}}$, and $\mathcal{L}_{\mathrm{Con}}$ facilitates the learning at the region level; and $\mathcal{L}_{\mathrm{CL}}$ facilitates the learning at the sample level. 
%We validate the performance improvement of the HybridMIM method at different levels of self-supervised learning for downstream segmentation tasks.
% Segmentation Target表示BraTS2020数据集不同的分割目标,Avg代表三个分割目标的平均指标。
%Segmentation Target represents the different segmentation targets of the BraTS2020 dataset, and Avg represents the average metric of the three segmentation targets.
% 表格中每个backbone的第一行结果为基线,不进行预训练,而是直接在下游分割任务上进行训练。之后,我们在预训练过程中逐渐添加不同的损失函数,来验证我们提出的不同模块对不同网络架构的性能提升能力。
We make comparison to the baseline with supervised training from scratch on the BraTS2020 dataset (see the first row for each backbone). 
%After that, we gradually add different loss functions during the pre-training process to verify the performance improvement capability of our proposed different modules for different network architectures.

\begin{table}[th]
    \centering
    \vspace{-3mm}
    % 在BraTS2020数据集上进行消融实验。我们选择UNet与SwinTransformer作为backbone,逐个添加我们提出的不同层次的损失函数。其中LR为局部重建损失,Num为数量分布预测损失,Loc为位置分布预测损失,Consis为一致性损失,CL为对比学习损失。下游任务的分割结果展示了我们提出的每个损失函数对于不同架构的有效性。
    \caption{Ablation experiments are performed on the BraTS2020 dataset. $\mathcal{L}_{\mathrm{LR}}$: the local reconstruction loss, $\mathcal{L}_{\mathrm{Num}}$: the number distribution prediction loss, $\mathcal{L}_{\mathrm{Loc}}$: the location distribution prediction loss, $\mathcal{L}_{\mathrm{Con}}$: the consistency loss, $\mathcal{L}_{\mathrm{CL}}$: the contrastive learning loss. }
    %The segmentation results of the downstream task demonstrate the effectiveness of each of our proposed loss functions for different architectures.
    % \vspace{-3mm}
    \label{tab:ablation}
    \renewcommand\arraystretch{1.2}
    \setlength\tabcolsep{5pt}%调列距
    \resizebox{\columnwidth}{!}{
    \begin{tabular}{c| l | c c c c c c}

    \hline
    \multirow{2}*{\makecell{Backbone}} & \multirow{2}*{Loss} & \multicolumn{4}{c}{Segmentation Target} & \\
    % \cline{3-7} \cline{10-13}
     & &  WT & TC & ET & Avg & \\
    \hline
    %% LR & Num & Loc & Consis & CL
    \multirow{6}{*}{UNet} & Supervised learning & 89.75 & 84.65 & 78.83 & 84.41 &\\
     & $\mathcal{L}_{\mathrm{PR}}$ & 90.19 & 85.50 & 79.48 & 85.06 & \\
     & $\mathcal{L}_{\mathrm{PR}} + \mathcal{L}_{\mathrm{Num}}$ & 90.05 & 85.48 & 79.97 & 85.17 & \\
     & $\mathcal{L}_{\mathrm{PR}} + \mathcal{L}_{\mathrm{Num}} + \mathcal{L}_{\mathrm{Loc}}$ & 90.15 & 85.65 & 80.10 & 85.30 & \\
     & $\mathcal{L}_{\mathrm{PR}} + \mathcal{L}_{\mathrm{Num}} + \mathcal{L}_{\mathrm{Loc}} + \mathcal{L}_{\mathrm{Con}}$ & 90.30 & 85.36 & \textbf{80.56} & 85.40 & \\
     & $\mathcal{L}_{\mathrm{PR}} + \mathcal{L}_{\mathrm{Num}} + \mathcal{L}_{\mathrm{Loc}} + \mathcal{L}_{\mathrm{Con}} + \mathcal{L}_{\mathrm{CL}}$ & \textbf{90.62} & \textbf{86.28} & {80.17} & \textbf{85.69} & \\
     \hline
     
     \multirow{6}{*}{Swin} & Supervised learning & 90.08 & 85.19 & 80.01 & 85.09 &\\
     & $\mathcal{L}_{\mathrm{PR}}$ & 90.95 & 86.17 & 80.22 & 85.78 & \\
     & $\mathcal{L}_{\mathrm{PR}} + \mathcal{L}_{\mathrm{Num}}$ & 90.93 & 86.94 & 80.48 & 86.12 & \\
     & $\mathcal{L}_{\mathrm{PR}} + \mathcal{L}_{\mathrm{Num}} + \mathcal{L}_{\mathrm{Loc}}$ & 91.18 & 86.33 & \textbf{81.10} & 86.20 & \\
     & $\mathcal{L}_{\mathrm{PR}} + \mathcal{L}_{\mathrm{Num}} + \mathcal{L}_{\mathrm{Loc}}  + \mathcal{L}_{\mathrm{Con}}$ & 90.98 & \textbf{87.06} & 80.71 & 86.24 & \\
     & $\mathcal{L}_{\mathrm{PR}} + \mathcal{L}_{\mathrm{Num}} + \mathcal{L}_{\mathrm{Loc}}  + \mathcal{L}_{\mathrm{Con}} + \mathcal{L}_{\mathrm{CL}}$ & \textbf{91.48} & {86.88} & {80.81} & \textbf{86.39} & \\
    %  \hline

     
    % \multirow{4}{*}{Swin} & & &  & & & & & & 90.08 & 85.19 & 80.01 & 85.09 & \\
    %  & & \checkmark & & & &  & & & 90.95 & 86.17 & 80.22 & 85.78 & \\
    %  & & \checkmark & \checkmark & & &  & & & 90.93 & 86.94 & 80.48 & 86.12 & \\
    %  & & \checkmark & \checkmark & \checkmark & &  & & & 91.18 & 86.33 & 81.10 & 86.20 & \\
    %  & & \checkmark & \checkmark & \checkmark & \checkmark &  & & & 90.98 & 87.06 & 80.71 & 86.24 & \\
    %  & & \checkmark & \checkmark &\checkmark & \checkmark & \checkmark & & & {91.48} & {86.88} & {80.81} & {86.39} & \\
    \hline
    \end{tabular}
    }
    \vspace{-2mm}
\end{table}




% 从表5中可以清晰的看出,当使用UNet架构在BraTS2020数据集上从零开始训练时,三个分割目标的Dice得分分别为89.75%, 84.65%, 78.83%, 平均值为84.41%。
\textbf{UNet architecture.} The baseline that is trained from scratch reports the Dice scores 89.75\%, 84.65\%, and 78.83\%, for the three segmentation targets respectively, with an average number of 84.41\%. 
% 此时加入第一个自监督学习损失LR(local reconstruction),该损失从像素层次来重建原图像被掩蔽区域的分布。在下游分割任务上加载由LR损失预训练得到的模型权重,使得每项分割目标均有不同程度的提升,平均值达到85.06%,较从零开始训练提升了0.65%。
At this point, we add the first self-supervised learning loss $\mathcal{L}_{\mathrm{PR}}$, which reconstructs the masked regions of the original image at the pixel level. The model weights fine-tuned onto the downstream segmentation task, result in a Dice average of 85.06\%, with an improvement of 0.65\% over the baseline.
% 之后,添加区域层次的自监督损失Num(number),Loc(location),Consis(consistency),提升模型表征空间区域分布的能力,分割目标的均值由85.06%提升至85.40%。
The addition of region-perception losses, i.e. $\mathcal{L}_{\mathrm{Num}}$, $\mathcal{L}_{\mathrm{Loc}}$, $\mathcal{L}_{\mathrm{Con}}$, improves the model's ability to characterize the distribution of spatial regions, and the mean Dice value is increased from 85.06\% to 85.40\%, getting an improvement of 0.34\%.
% 最后,添加样本层次的自监督损失CL(contrastive learning),提升模型对于不同样本表征的区分能力。通过CL损失,在下游分割任务中,三个分割目标的Dice得分均值达到了85.69%,并且在WT与TC上的Dice得分也达到了最高,分别为90.62%和86.28%。
Finally, we add the sample-level self-supervised loss $\mathcal{L}_{\mathrm{CL}}$ to enhance the model's ability to distinguish between different sample representations. With $\mathcal{L}_{\mathrm{CL}}$, the mean Dice score reaches 85.69\% in the downstream segmentation task, and the highest Dice scores of 90.62\% and 86.28\% on WT and TC, respectively. 
%%
In the end, the average Dice score with pre-training was 1.29\% higher than that without pre-training.

% 类似的,MP-SSL方法对于SwinTransformer架构也有较大程度的提升。三个分割目标的平均Dice得分由没有预训练时候的85.09%最终提升到了86.39%,在BraTS2020数据集上实现了SOTA的分割结果。
\textbf{SwinUNETR architecture.} Similarly, the HybridMIM method also achieves obvious improvements for the SwinUNETR architecture. The average Dice score of the three segmentation targets was finally improved from 85.09\% without pre-training to 86.39\%, achieving SOTA segmentation results on the BraTS2020 dataset. 

% \textbf{Analysis of self-supervised loss enhancement effects.} 对于UNet跟SwinTransformer架构,从表中可以看出,LR损失发挥了比较大的作用。UNet架构加入LR损失后,三个分割指标的平均Dice得分提升了0.65%,而SwinTransformer架构加入LR损失后,三个指标的平均Dice得分提升了0.69%。
\textbf{Analysis of self-supervised loss enhancement effects.} 
For the UNet and SwinTransformer architectures, Table~\ref{tab:ablation} shows that the $\mathcal{L}_{\mathrm{PR}}$ plays a larger role. The average Dice score of the three segmentation targets increases by 0.65\% with the aid of $\mathcal{L}_{\mathrm{PR}}$ upon the UNet architecture, while the average Dice score of the three metrics increased by 0.69\% upon the SwinTransformer architecture.
% 此外Consis损失由于具有保持预测的数量与位置信息一致的作用,提升自监督学习的可解释性,因此其对于下游分割任务的提升较小。对于UNet架构,平均Dice得分提升了0.1%,而对于SwinTransformer结构,平均Dice提升了0.04%。
%%
The region perception losses ($\mathcal{L}_{\mathrm{Num}}$, $\mathcal{L}_{\mathrm{Loc}}$, $\mathcal{L}_{\mathrm{Con}}$ together) are the second important. 
%%
Also note that although the $\mathcal{L}_{\mathrm{Con}}$ has a relatively small improvement for the downstream segmentation task, it has a role in keeping the predicted quantity consistent with the location information, improving the interpretability of the self-supervised learning. 
%For the UNet architecture, the average Dice score increased by 0.1\%, while for the SwinTransformer structure, the average Dice increased by 0.04\%.


 \section{Conclusion}
 In this paper, we have presented a tactile manipulation system that is able to rotate different objects without vision. We showed an end-to-end reinforcement learning framework to learn tactile dexterity over the proposed system. We carried out experiments both in simulation and real to demonstrate its effectiveness. Our work demonstrated that we are able to achieve tactile dexterity as humans in real for the first time. In the future, there are many promising future directions to investigate, such as exploring the use of a more dense contact sensor array and scaling up the system to solve more diverse tasks. We hope that our work can pave the way for more intelligent robot hands.

{\parskip=3pt
\noindent\textbf{Acknowledgements}: This work was partially funded by the Carl Zeiss Foundation with the ReScaLe project. 
% Toyota Research Institute (TRI) provided funds to assist the authors with their research but this article solely reflects the opinions and conclusions of its authors and not TRI or any other Toyota entity.
}


%%%%%%%%% REFERENCES
% \cleardoublepage
% \newpage
{\small
\bibliographystyle{ieee_fullname}
\bibliography{FullArXivPaper}
}


%%%%%%%%%%%%%%%%%%%%%%% Supplementary material
% Attach Supp Material

\cleardoublepage

\begin{strip}
\begin{center}
\vspace{-5ex}
\textbf{\Large \bf
% Panoptic Segmentation in the Bird's Eye View
% Learning and Aggregating Lane Graphs for Urban Automated Driving
\ourName: \ourNameFull
} \\
\vspace{2ex}

\Large{\bf Supplementary Material}\\
\vspace{0.4cm}
\large{
Nick~Heppert,~
Muhammad~Zubair~Irshad,~
Sergey~Zakharov,~
Katherine~Liu,\\
Rares~Andrei~Ambrus,~
Jeannette~Bohg,~
Abhinav~Valada,~
Thomas~Kollar
}
\end{center}
\end{strip}

%%%%%%%%%% Merge with supplemental materials %%%%%%%%%%
%%%%%%%%%% Prefix a "S" to all equations, figures, tables and reset the counter %%%%%%%%%%
\appendix
\setcounter{section}{0}
\setcounter{equation}{0}
\setcounter{figure}{0}
\setcounter{table}{0}
% \setcounter{page}{1}
\makeatletter
\renewcommand{\thesection}{S.\arabic{section}}
\renewcommand{\thesubsection}{S.\arabic{section}.\arabic{subsection}}
\renewcommand{\thetable}{S.\arabic{table}}
\renewcommand{\thefigure}{S.\arabic{figure}}
\renewcommand{\theequation}{S.\arabic{equation}}
%%%%%%%%%% Alternative continue the counting (done for submission)
% \setcounter{figure}{3}
% \setcounter{table}{3}
% \setcounter{equation}{12}


\normalsize
\let\cleardoublepage\clearpage

%%%%%%%%% ABSTRACT
\begin{abstract}
The current study investigated possible human-robot kinaesthetic interaction using a variational recurrent neural network model, called PV-RNN, which is based on the free energy principle.
Our prior robotic studies using PV-RNN showed that the nature of interactions between top-down expectation and bottom-up inference is strongly affected by a parameter, called the meta-prior, which regulates the complexity term in free energy.
% The current study examines how the behaviours of robots alter by changing the meta-prior $w$ in human-robot kinaesthetic interaction.
The current study examines how changing the meta-prior $w$ in the interaction phase affects the counter force generated when an experimenter attempts to induce movement pattern transitions familiar to the robot through its prior training.
The study also compares the counter force generated when trained transitions are induced by a human experimenter and when untrained transitions are induced.
Our experimental results indicated that (1) the human experimenter needs more/less force to induce trained transitions when $w$ is set with larger/smaller values, (2) the human experimenter needs more force to act on the robot when he attempts to induce untrained as opposed to trained movement pattern transitions.
Our analysis of time development of essential variables and values in PV-RNN during bodily interaction clarified the mechanism by which gaps in actional intentions between the human experimenter and the robot can be manifested as reaction forces between them.


%% Hiroki writing 2022-11-4
%Current study investigates the dynamics of the latent states during human-robot kinaesthetic interaction using PV-RNN.
%We have achieved to observe and analyse the internal state of an RNN model based on the free energy principle, during real-time human-robot interaction.
%Essential characteristics observed in the previous study of this variational recurrent neural network model, PV-RNN, is that by changing a meta prior $w$, the balance between the top-down intention and the bottom-up perceptual reality changes.
%In the current study, we examined how changing the weighting parameter $w$ between accuracy and complexity in free energy principle affects the humanoid robot's behaviour through human-robot interaction. We have conducted some human-robot kinaesthetic interaction experiments with various $w$ and quantitatively analysed the latent variable and the force applied to the humanoid robot. We have observed that the force required to change the robot's intention has increased, both when the top-down intention was strengthened by changing the $w$ and when corresponding switch of its primitive was against the experience of the RNN during its training. The study confirms through quantitative analysis that by increasing or decreasing the $w$ in PV-RNN, humanoid robot leads or follows the human counterpart during the human-robot kinaesthetic interaction.

\begin{comment}
Comment from Jun #2
・最後にQualitativeな結果(インパクト)が欲しい
・Current study investigates the problem on~と書き出すのが一般的
・最初の一文と最後の一文を対応させる
・最後の一文はもう少しAbstractかつ包括的に
\end{comment}

\begin{comment}
Comment from Jun #1
We investigated how the kinaesthetic human-robot interaction can affect the internal state of a model based on the free energy principle. 
=> how the internal state is affected is not the most important point in this study. This part should be rewritten.

The key function of this variational recurrent neural network model, PV-RNN, is that by changing a meta prior $w$, it takes a balance between the "complexity” term and the ”accuracy” term which corresponds to a top-down intention and a bottom-up perceptual reality in the free energy principle, respectively. 
=> This is not key function of PV-RNN. It is an essential characteristics observed in the previous study. The grammar after $w$ is something strange. Rewrite these.

This research has conducted a human-robot interaction experiment with a robotic agent in a kinaesthetic sense.
=> The sentence is not good. "in a kinaesthetic sense" is grammatically wrong.
MODIFIED => "In the current study human-robot interaction experiments using the kinaesthetic sense were conducted."

We investigated that when human forces the agent to switch primitives from one to another, larger force was required both when the human intention is conflictive against the top-down the intention of the agent and when the agent has a stronger top-down intention by modifying the $w$.
=> You should write the essential results of the experiments rather than what we investigated and also how these results could contribute to the studies on human-robot interaction.
\end{comment}

\end{abstract}
\section{Model Architecture}
\label{supp:sec:model_architectures}
We present our model architecture for the encoder in \cref{supp:fig:encoder_architecture} and for the decoder in \cref{supp:fig:decoder_architecture}. 

\subsection{Encoder}
\label{supp:subsec:encoder_architecture}
{\parskip=0pt
Our encoder builds upon the SimNet-architecture \cite{laskey2021simnet}. The input is a stereo RGB image pair of size $\realspace^{960 \times 512 \times 3}$. Each image gets passed through a shared feature encoder network that outputs a low-dimensional feature map of size $\realspace^{128 \times 240 \times 16}$. This output is then fed into a cost volume which performs approximate stereo matching. Based on the result of the cost volume of size $\realspace^{128 \times 240 \times 32}$, a lightweight head predicts an auxiliary disparity map of size $\realspace^{128 \times 240}$.  
%The output of this stage is a low-resolution disparity image.
Parallel to that, we feed the left image through a separate RGB encoder that also predicts a feature map of size $\realspace^{128 \times 240 \times 32}$. This map, as well as the output of the cost volume, get concatenated and fed into a feature pyramid network which predicts three feature maps of sizes $\realspace^{128 \times 240 \times 32}$, $\realspace^{64 \times 120 \times 64}$, $\realspace^{32 \times 60 \times 64}$. Finally, using these features, each quantity described in \ifthenelse{\boolean{reftomain}}{\cref{subsec:encoder}}{Sec.~3.1} is predicted by its respective output head, including a segmentation mask, 3D bounding box, object pose, full resolution disparity, shape code, and joint code heads.

As in~\cite{laskey2021simnet}, although the stereo input for sim-to-real transfer has benefits for perceiving objects in harsh lighting conditions and for transparent or reflective objects, a RGB-D version 
% can be trained similar to~\cite{xie2020}.
could be trained as well (see \cref{supp:subsec:rgbd_version}).
}
\begin{figure*}
    \centering
    \includegraphics[width=\linewidth]{content/supp/img/simnet.pdf}
    \caption{Encoder Architecture based on \cite{laskey2021simnet}}
    \label{supp:fig:encoder_architecture}
\end{figure*}

\subsection{Decoder}
\label{supp:subsec:decoder_architecture}
Our decoder is split into two sub-decoders. A geometry decoder (see \ifthenelse{\boolean{reftomain}}{\cref{subsubsec:geometry_decoder}}{Sec~3.2.1}) based on DeepSDF \cite{park_deepsdf_2019} and a joint decoder (see \ifthenelse{\boolean{reftomain}}{\cref{subsubsec:joint_decoder}}{Sec.~3.2.2}). We detail both the architecture in the subsequent paragraphs.

\noindent \textbf{The geometry decoder} is a deep multi-layer perceptron consisting of four layers. The first layer takes a shape code $\latentShape$ and joint code $\latentJoint$ as input. Before the second and last layer, we concatenate the space coordinate $\spaceCoordinate$ for which we want to retrieve the SDF-value $\sdfValue$ with the output of the previous layer. Following the findings in \cite{mu_a-sdf_2021}, we again input the joint code $\latentJoint$ before the second last layer. For the exact feature vector dimensions see \cref{supp:fig:decoder_architecture}. As an activation function, we use ReLU for all except the last layer which uses tanh. 
Exploring exact input positions for shape code $\latentShape$, joint code $\latentJoint$, and space coordinate $\spaceCoordinate$, could be a topic of further research.

\noindent \textbf{The joint decoder} only takes a joint code $\latentJoint$ as input and feeds it through a single layer outputting a feature vector with 64 dimensions. This vector is then used to regress the articulation state, consisting of the continuous joint state $\jointState$ (no activation) and the discrete joint type $\jointType$ (Sigmoid activation).

An overview of the used loss scaling hyperparameters is given in \cref{supp:tab:decoder_scaling_training}.

\begin{figure}
    \centering
    \includegraphics[width=\linewidth]{content/supp/img/decoder.pdf}
    \caption{Decoder Architecture. The numbers indicate the size of the respective feature vector. Each arrow represents a layer of a multi-layer perceptron. For the geometry decoder, except for the first layer, the input to a layer always has a size of 512. The output dimensions vary depending on auxiliary inputs.}
    \label{supp:fig:decoder_architecture}
\end{figure}

\begin{table}
    \centering
    \begin{tabular}{lc}
        \toprule
        Scaling Variable & Value \\
        \midrule
        $\lossScaling_{\indexCodeRegularizer, \latentJoint, \text{pre}}$ & 0.1 \\
        $\lossScaling_{\indexJointRegularizer, \text{pre}}$ & 1.0 \\
        $\lossScaling_{\indexCodeRegularizer, \latentShape}$ & 0.0001 \\
        $\lossScaling_{\indexCodeRegularizer, \latentJoint}$ & 0.001 \\
        $\lossScaling_{\indexReconstruction}$ & 1.0 \\
        $\lossScaling_{\indexJointType}$ & 0.001 \\
        $\lossScaling_{\indexJointState}$  & 0.1 \\
        $\lossScaling_{\indexJointRegularizer}$ & 0.1 \\
        \bottomrule
    \end{tabular}
    \caption{Scaling Hyperparameters for Decoder Training.}
    \label{supp:tab:decoder_scaling_training}
\end{table}

\section{Backward Optimization}
\label{supp:sec:backward_optim}

The goal of the backward optimization is to retrieve the shape code $\latentShape^{u}$ and joint code $\latentJoint^{u}$ of an unknown object. The object is given through sampled SDF values, in total $P$. We denote the set of all $P$ SDF-space coordinate tuples as $\allSdfValues^{u} = \{(\spaceCoordinate_{p}, \sdfValue_{p})_{p} \forall p \in P \}$ for this unknown object. The problem can then be formalized as 
\begin{align}
    \label{supp:eqn:backward_optim}
    &\latentShape^{u}, \latentJoint^{u} = \\
    &\quad \arg\min_{\latentShape^{u}, \latentJoint^{u}} 
    \frac{1}{|\allSdfValues^{u}|} \sum_{(\spaceCoordinate_{p}, \sdfValue_{p}) \in \allSdfValues^{u}} \left| \geometryDecoder(\latentShape^{u}, \latentJoint^{u}, \spaceCoordinate_{p}) -  \sdfValue_{p} \right|, \nonumber
\end{align}
a minimization of the distance between the given SDF values and the one predicted by our geometry decoder (see \ifthenelse{\boolean{reftomain}}{\cref{subsubsec:geometry_decoder}}{Sec.~3.2.1}).

\begin{algorithm*}
    \caption{Backward Optimization: The goal is to retrieve shape and joint code for an unknown shape $\allSdfValues^{u}$ with $n$ hypotheses in parallel. Here, for clarity, we show the optimization for a single $i \in [1, \ldots, n]$. Eventually, from all returned code pairs the pair having the lowest distance error (see \cref{supp:eqn:backward_optim}) is returned. This procedure can be efficiently parallelized on a GPU.}
    \label{supp:algo:backward_optimization}
    \begin{algorithmic}[1] % The number tells where the line numbering should start
        \Procedure{BackOptimSingle}{$\allSdfValues^{u}, \boldsymbol{Z}_{\text{j}}, i$}
            \State $ \text{project}(\bullet), \text{reproject}(\bullet) \gets \text{SVD}(\boldsymbol{Z}_{\text{j}}) $
                \Comment{Retrieve project (see \cref{supp:joint_code_projection}) and reproject (see \cref{supp:joint_code_reprojection}) function}
            \State $\latentShape^{i} \gets \mathcal{N}( \boldsymbol{0}, \Sigma )$ 
                \Comment{Initialize shape codes with $\boldsymbol{0} \in \realspace^{\dimensionsShapeSpace}, \Sigma = \text{diag}(0.5) \in \realspace^{\dimensionsShapeSpace \times \dimensionsShapeSpace}$ }
            \State $\latentJoint^{i} \gets 
            \begin{cases}
                \bar{\boldsymbol{Z}}_{\text{j}}^{\text{prismatic}} & i \mod 2 = 0   \\ 
                \bar{\boldsymbol{Z}}_{\text{j}}^{\text{revolute}}  & i \mod 2 = 1
            \end{cases} $
            \Comment{Initialize joint codes}
            \State $\hat{\latentJoint}^{i} \gets \text{project} (\latentJoint^{i}) $
            \Comment{Project joint codes}
            \For{$\textit{step} \in [1, \ldots, 800$]}
                \State $\begin{aligned} 
                    \textit{loss}^{i} \gets &  \frac{1}{| \allSdfValues^{u} |} \sum_{(\spaceCoordinate_{p}, \sdfValue_{p}) \in \allSdfValues^{u}} \left| \geometryDecoder(\latentShape^{i}, \text{reproject}(\hat{\latentJoint}^{i}), \spaceCoordinate_{p}) -  \sdfValue_{p} \right|  
                    \\ 
                    & + 5 \cdot 10^{-3} ||\latentShape^{i}|| \\
                    & + 10^{-2} \min( ||\text{reproject}(\hat{\latentJoint}^{i}) - \boldsymbol{Z}_{\text{j}}||)  
                \end{aligned}$
                \Comment{Sum distance loss and regularization terms}
                \If{$\textit{step} \leq 600$}
                    \State $\latentShape^{i}, \hat{\latentJoint}^{i} \gets \text{ADAM}(\textit{loss}^{i})  $
                    \Comment{Update shape and joint code}
                \ElsIf{$600 < \textit{step} \leq 700$}
                    \State $\latentShape^{i} \gets \text{ADAM}(\textit{loss}^{i})  $
                    \Comment{Update shape code}
                \Else
                    \State $\hat{\latentJoint}^{i} \gets \text{ADAM}(\textit{loss}^{i})  $
                    \Comment{Update joint code}
                \EndIf
            \EndFor\label{optim}
            \State \textbf{return} $ \latentShape^{i}, \text{reproject}(\hat{\latentJoint}^{i}) $
        \EndProcedure
    \end{algorithmic}
\end{algorithm*}

At the beginning of the optimization, we randomly sample a set of 16 random shape codes from a zero-mean Gaussian distribution with a variance of $0.5$ as well as a set of 16 corresponding joint codes. For the joint codes, we do not sample but rather take the mean from all final joint codes of the training set after training $\latentJoint^{\jointIndex} \forall \jointIndex \in N$. We split the joint codes, where one half is using the mean of all prismatic training joint codes and the other half uses the mean of all revolute training joint codes. To guide the optimization through our latent joint code space, we propose a projection of the space as well as bounding the joint code variables.

{\noindent \textbf{SVD Projection}: To facilitate optimization along significant axes we will construct a projection based on the singular value decomposition of our training joint codes. To that end, we stack all training joint codes}
\begin{equation}
    \boldsymbol{Z}_{\text{j}} = \begin{bmatrix}
        \vdots\\
        {\latentJoint^{\jointIndex}}^\text{T}\\
        \vdots
    \end{bmatrix} \in \realspace^{\jointCount \times \dimensionsJointSpace}
\end{equation}
and do a singular value decomposition
\begin{align}
    &\boldsymbol{Z}_{\text{j}} - \bar{\boldsymbol{Z}_{\text{j}}} = \boldsymbol{U} \boldsymbol{\Sigma} \boldsymbol{V}^\text{T}, \\
    &\boldsymbol{U} \in \realspace^{\jointCount \times \jointCount}, \boldsymbol{\Sigma} \in \realspace^{\jointCount \times \dimensionsJointSpace}, \boldsymbol{V}^\text{T} \in \realspace^{\dimensionsJointSpace \times \dimensionsJointSpace}.
\end{align}
We then use 
\begin{equation}
    \label{supp:joint_code_projection}
    \hat{\latentJoint}^{u} = \left( \latentJoint^{u} - \bar{\boldsymbol{Z}_{\text{j}}} \right) \boldsymbol{V}
\end{equation}
to \textit{project} any joint code $\latentJoint^{u}$ and 
\begin{equation}
    \label{supp:joint_code_reprojection}
    \latentJoint^{u} = \hat{\latentJoint}^{u}\boldsymbol{V}^\text{T} + \bar{\boldsymbol{Z}_{\text{j}}}
\end{equation}
to \textit{reproject} a joint code $\hat{\latentJoint}^{u}$. 

We carry out the optimization from \cref{supp:eqn:backward_optim} in our projected space and thus, initially we project our joint codes using \cref{supp:joint_code_projection}. As well as in each optimization step, before inputting the joint code in our geometry decoder, we first reproject it using \cref{supp:joint_code_reprojection}. In initial testing, we found that this projection-reprojection step greatly helps navigate the high-dimensional space in which our joint codes reside in.

{\noindent \textbf{Bound Joint Code Variables}: On top of the previously described projection procedure, we ensure that the joint code variable is close to final joint codes from the training examples $\boldsymbol{Z}_{\text{j}}$ through minimizing the minimum distance to any joint code in the training set:}
\begin{equation}
    \min( ||\latentJoint^{u} - \boldsymbol{Z}_{\text{j}}||),
\end{equation}
where $||\bullet||$ is the row-wise Euclidean norm and $\min(\bullet)$ is a differentiable operator returning the minimum of a vector. An outline of our full optimization is presented in \cref{supp:algo:backward_optimization}.

\section{Learned Joint Code Space}
\label{supp:sec:joint_code_space}
{\parskip=0pt
In this section, we visualize and compare the resulting learned latent joint space using our in \ifthenelse{\boolean{reftomain}}{\cref{subsubsec:joint_space}}{Sec.~3.3.1} introduced regularization against naively training it. In \cref{supp:fig:latent_joint_spaces}, we visualize the learned joint codes of the training results for \textit{\ourName} and \textit{\ourName-No-Enf} from \ifthenelse{\boolean{reftomain}}{\cref{subsec:canonical_reconstruction_task}}{Sec.~4.2}. When comparing both visualizations, we can explain the worse performance of \textit{\ourName-No-Enf} in \cref{tab:decoder_results}. 

\ourName{} trained without regularization struggles to correctly align the spaces such that joint codes belonging to the same articulation state, independent of the object, are close and show a low variance. The decoder rather learns to represent the final geometry of the articulated object jointly through both codes, the shape code $\latentShape$ and joint code $\latentShape$ instead of disentangling one from the other. One could argue that this case is similar to not splitting the codes. Compared to that, when using our proposed regularization, we learn a cleaner disentanglement between the shape and the articulation state of the object. Joint codes of similar articulation states in the training set are arranged closer in the latent joint embedding and thus, the variance in the y-direction across all plots on the left side of \cref{supp:fig:latent_joint_spaces} (a) is much lower when compared to training without our regularization in \cref{supp:fig:latent_joint_spaces} (b). Moreover, two distinct clusters are visible (prismatic and revolute) whereas without \ourName{}s regularization different joint types overlap. 
}
\begin{figure*}
    \centering
     \begin{subfigure}{\textwidth}
        \includegraphics[width=\textwidth]{content/supp/img/with_enforcement.png}
        \caption{Using \ourName{}s Regularization.}
        \label{supp:subfig:carto_latent_joint_space}
    \end{subfigure}\\
    \begin{subfigure}{\textwidth}
        \includegraphics[width=\textwidth]{content/supp/img/without_enforcement.png}
        \caption{Only norm Regularization.}
    \end{subfigure}
    \caption[]{Comparison of Learned Joint Code Space. We compare the learned embedding of training joint codes when using our proposed regularization (a) against naively just regularizing the norm (b). An articulation state is expressed two-fold. First, by its form to represent the joint type. Here upside-down triangles stand for revolute and cross for prismatic joint types. Second, the form is colored by its joint state according to the scale shown on the right. In the left figure, each plot represents one component of the joint code $\latentJoint \in \realspace^{16}$. In the $i$-th plot, we plot the $i$-th component of all training joint codes on the y-axis against their associated known joint state on the x-axis. Additionally, we overlay the in 
    \ifthenelse{\boolean{reftomain}}{\cref{subsec:inverse_joint_decoder}}{Sec.~3.3.3} explained polynomial functions. In the right figure, we show a two-dimensional projection based on singular value decomposition of all training joint codes.}
    \label{supp:fig:latent_joint_spaces}
\end{figure*}

\section{Datasets}
\label{supp:sec:datasets}
{\parskip=0pt
\cref{supp:fig:syn_images} presents exemplary images of our procedural generated kitchen dataset. In total, we collected roughly 100k images for training and 20k images for testing. Due to the long run-time of our A-SDF baseline, we only evaluated on the first 2000 samples in which according to our ground truth objects are present. See \cref{supp:fig:real_images} for exemplary images from our collected real-world dataset.

In \cref{tab:dataset_comparison} we compare our dataset against the RBO \cite{martin-martin_rbo_2018} and the BMVC \cite{michel_pose_2015} dataset.
}

\begin{table}[t]
\centering
\begin{tabular}{c|c|c|c}
\toprule
   \multirow{2}{*}{Data} &\multicolumn{3}{c}{ID: Pascal}  \\\cline{2-4}
&            Comic & Water & Cart \\\hline 
IN1K&10.9 (\textcolor{blue}{+3.6})&22.5 (\textcolor{blue}{+2.1}) &16.5 (\textcolor{blue}{+4.7})\\
IN1K+Augmix&    12.7 (\textcolor{blue}{+3.1}) &25.7 (\textcolor{blue}{+2.6}) &17.3 (\textcolor{blue}{+2.5}) \\
Instagram&16.8	(\textcolor{blue}{+9.3}) & 26.5	(\textcolor{blue}{+7.1}) &17.6 (\textcolor{blue}{+6.2})   \\
\bottomrule
\end{tabular}
\caption{Results of ResNet50 pre-trained with different datasets or augmentation. We show the results with DP-FT + WR and improvement over FT baseline with blue numbers.}
\label{tb:compare_dataset}
\end{table}

\begin{figure*}[ht!]
    \centering
    \includegraphics[width=0.95\linewidth]{content/supp/img/synthetic.jpg}
    \caption{Synthetic Example Images}
    \label{supp:fig:syn_images}
    \vspace{1cm}
\end{figure*}

\begin{figure*}[htb!]
    \centering
    \includegraphics[width=0.95\linewidth]{content/supp/img/real.jpg}
    \caption{Real Example Images}
    \label{supp:fig:real_images}
\end{figure*}


\section{Additional Experimental Results}
\label{supp:sec:additional_metric}

\begin{table*}
    \centering
    \caption{Decoder Optimization Results. Each object is sampled in 50 different joint states for training as well as for testing. \textit{$\dagger$ means the model is trained only on a single category.}}
    \label{tab:decoder_results}
    \begin{subtable}[h]{1.0\textwidth}
        \small
        \centering
        \footnotesize
        \begin{tabular}{l|ccccccc|c|c}
            \toprule
            Method & Dishwasher & Laptop & Microwave & Oven & \makecell{Refrigerator} & Table & \makecell{Washing\\Machine} & \makecell{Instance\\Mean} & \makecell{Category\\Mean} \\
%            \#Train & 900 & 1000 & 500 & 350 & 500 & 950 & 400 & 4600 & N/A\\
%            \#Test & 250 & 250 & 150 & 100 & 150 & 250 & 100 & 1250 & N/A\\
            \midrule
            A-SDF \cite{mu_a-sdf_2021} $\dagger$  & 2.162 & {0.264} & 2.256 & 1.540 & 1.409 & 1.465 & 0.960 & 1.437 & 1.418 \\
            \ourName{} $\dagger$  & 1.874 & 0.965 & 1.429 & 1.606 & {0.690} & {1.330} & {0.437} & 1.190 & 1.252\\
            A-SDF \cite{mu_a-sdf_2021} & \textbf{0.336} & \textbf{0.601} & \textbf{0.700} & \textbf{0.883} & 1.425 & 1.862 & \textbf{0.608} & \textbf{0.934} & \textbf{0.916} \\
            \ourName-No-Enf & 1.043 & 3.820 & 2.685 & 2.317 & 2.454 & \textbf{1.676} & 1.727 & 2.246 & 2.248\\
            \ourName{} & 0.554 & 1.448 & 0.782 & 2.056 & \textbf{0.988} & 1.688 & 0.830 & 1.192 & 1.181\\
            \bottomrule
        \end{tabular}
        \caption{Shape Reconstructions Results. We report the bi-directional L2-Chamfer distance (CD) ($\downarrow$) times 1000 between the original mesh and the reconstructed version.}
        \label{tab:shape_reconstruction_results}
    \end{subtable}%
    \newline
    \vspace*{0.2cm}
    \newline
    \begin{subtable}[h]{0.95\textwidth}
        \small
        \centering
        \footnotesize
        \begin{tabular}{l|ccccccc|c|c}
            \toprule
            Method & Dishwasher & Laptop & Microwave & Oven & \makecell{Refrigerator} & Table & \makecell{Washing\\Machine} & \makecell{Instance\\Mean$^*$} & \makecell{Category\\Mean$^*$} \\
            \midrule
%            \#Train & 900 & 1000 & 500 & 350 & 500 & 950 & 400 & 4600 & N/A\\
%            \#Test & 250 & 250 & 150 & 100 & 150 & 250 & 100 & 1250 & N/A\\
            A-SDF \cite{mu_a-sdf_2021} $\dagger$  & {1.616$\degree$} & $17.282\degree$ & $11.161\degree$ & {4.045$\degree$} & {19.254$\degree$} & $0.094\si{m}$ & $16.462\degree$ & {11.337$\degree$} & {11.636$\degree$} \\
            \ourName{} $\dagger$  & $6.264\degree$ & {6.818$\degree$} & $20.425\degree$ & $8.156\degree$ & $21.456\degree$ & {0.081$\si{m}$} & $21.057\degree$ & $12.474\degree$ & $14.029\degree$ \\
            A-SDF \cite{mu_a-sdf_2021} & \textbf{3.457$\degree$} & $30.740\degree$ & $7.189\degree$ & \textbf{3.884$\degree$} & $34.714\degree$ & $0.235\si{m}$ & $12.265\degree$ & $16.139\degree$ & $15.375\degree$ \\
            \ourName-No-Enf & $29.289\degree$ & $37.476\degree$ & $36.086\degree$ & $46.648\degree$ & $37.856\degree$ & \textbf{0.104$\si{m}$} & $34.451\degree$ & $35.892\degree$ & \makecell{$36.967\degree$} \\
            \ourName{} & $8.214\degree$ & \textbf{10.678$\degree$} & \textbf{6.815$\degree$} & $14.136\degree$ & \textbf{23.467$\degree$} & {0.141$\si{m}$} & \textbf{8.328$\degree$} & \textbf{11.512$\degree$} & \textbf{11.940$\degree$} \\
            \midrule
            A-SDF \cite{mu_a-sdf_2021} & \textbf{1.000} & \textbf{0.956} & 0.933 & \textbf{0.99} & \textbf{0.9} & \textbf{0.988} & 0.923 & \textbf{0.962} & \textbf{0.957} \\
            \ourName-No-Enf & 0.604 & 0.748 & 0.727 & {0.600} & 0.700 & 0.496 & 0.710 & 0.646 & 0.655 \\
            \ourName{} & {0.932} & {0.932} & \textbf{0.973} & 0.570 & {0.867} & {0.956} & \textbf{0.970} & {0.908} & {0.886} \\
            \bottomrule
        \end{tabular}
        \caption{Articulation State Prediction Results. We report the joint state error ($\downarrow$) in the first set of rows for all correctly classified joints and joint type accuracy ($\uparrow$) in the last set of rows. As A-SDF does not classify the joint type and \ourName{} trained on a single category always predicts the correct joint type, we do not report joint accuracy for those models. \textit{$^*$The joint state error mean is only reported across the revolute categories, as there is only one prismatic category.}}    
        \label{tab:joint_state_results}
    \end{subtable}
    \vspace{-0.3cm}
\end{table*}


\begin{table*}
    \small
    \caption{Reconstruction and Articulation State Prediction Results when using A-SDF \cite{mu_a-sdf_2021} with the Proposed Test-Time-Adaptation. \textit{$^*$The joint state error mean is only reported across the revolute categories, as there is only one prismatic category.}}    
    \label{supp:tab:asdf_tta}
    \centering
    \footnotesize
    \begin{tabular}{l|ccccccc|c|c}
        \toprule
        Method & Dishwasher & Laptop & Microwave & Oven & \makecell{Refrigerator} & Table & \makecell{Washing\\Machine} & \makecell{Instance\\Mean$^*$} & \makecell{Category\\Mean$^*$} \\
        \midrule
%            \#Train & 900 & 1000 & 500 & 350 & 500 & 950 & 400 & 4600 & N/A\\
%            \#Test & 250 & 250 & 150 & 100 & 150 & 250 & 100 & 1250 & N/A\\
        Chamfer Distance ($\downarrow$) & 0.101 & 1.035 & 0.529 & 0.451 & 1.383 & 64.097 & 0.332 & 13.339 & 9.704 \\
        %\midrule
        Joint State Error ($\downarrow$) & $19.387\degree$ & $18.675\degree$ & $58.088\degree$ & $25.432\degree$ & $20.700\degree$ & $0.552\si{m}$ & $51.467\degree$ & $29.024\degree$ & $32.292\degree$ \\
        \bottomrule
    \end{tabular}
\end{table*}

In this section, we present additional metrics for our experiments. Namely, using A-SDFs proposed test time adaption as well as the Chamfer distance and joint state error for the full pipeline experiment. Also, in addition to \ifthenelse{\boolean{reftomain}}{\cref{tab:decoder_results_short}}{Tab.~2}, we report the more fine-grained category-level metrics in \cref{tab:decoder_results}.

\subsection{Canonical Reconstruction Task: A-SDF TTA}
\label{supp:subsec:a-sdf_tta}
{\parskip=0pt
In our experiments (see \ifthenelse{\boolean{reftomain}}{\cref{subsec:canonical_reconstruction_task}}{Sec.~4.2}), the proposed test time adaptation (TTA) \cite{mu_a-sdf_2021} did not prove to be stable. We report the results in \cref{supp:tab:asdf_tta}. While for some object instances TTA reduces the Chamfer distance, for the table category TTA does not prove to be robust. Additionally, the joint state error increases substantially. Both behaviors are reasonable when reflecting on the proposed TTA. When jointly optimizing the input shape code, the joint state, and network weights, the entire network will overfit to the single given geometry. Thus, it is easier to achieve a lower Chamfer distance. Whereas, the joint state variable becomes unbound from other examples and can be optimized freely, losing its meaning and therefore, potentially resulting in a high joint state error.

The proposed TTA is still promising and with further investigation into how to mitigate the aforementioned problems, it can prove to be an ideal tool for reconstructing (articulated) objects in the wild \cite{irshad_2022_shapo}.
}

\subsection{Extended Metrics for Full Pipeline}
\label{supp:subsec:extend_metrics}
{\parskip=0pt
In addition to the tabular values reported in \ifthenelse{\boolean{reftomain}}{\cref{tab:full_pipeline_synthetic}}{Tab.~3a}, we present the respective mAP curve in \cref{supp:fig:3d_metrics}. The results highlight even more that an optimization-based two-stage approach suffers from its partial input. The two counter objects, laptops and microwaves, which are free-standing and thus much more points for reconstruction are available get reconstructed much better compared to other objects. On the other hand, for these objects, the predicted rotation is much worse. This can be rooted in the fact that for all other objects, we can learn a strong prior of the rotation being roughly camera facing, whereas, for laptops and microwaves the range of the possible rotation is much higher as they are placed freely on top of the counter.

While in \ifthenelse{\boolean{reftomain}}{\cref{tab:full_pipeline_synthetic}}{Tab.~3a} and previously we only discussed the overall 3D IoU and pose error, which gives a holistic evaluation of the full pipeline, we additionally report object-centric L2-Chamfer distances (similar to \cite{irshad_centersnap_2022, irshad_2022_shapo}) multiplied by $10^3$, as well as the joint state error in \cref{supp:fig:chamfer_joint_metrics}. Since this is an object-centric evaluation and should not evaluate the detection quality, we are very forgiving in selecting our detection matches. For each scene, we calculate our spatial 2D detections and retrieve the ground-truth spatial 2D detections from the heatmap, we then match the predicted and ground truth detections by solving a linear sum assignment problem, ignoring unmatched detections (either ground-truth or predicted). For each matched detection we then reconstruct the object as before using our geometry decoder and retrieve the joint through our joint decoder. We then calculate the Chamfer distance between the predicted points and the ground-truth points and compare the joint states.

In this experiment, we observe the same trend as for 3D IoU. One major difference is that \textit{A-SDF-GT} reconstructs laptops more accurately compared to \textit{A-SDF} and \textit{\ourName} which can be attributed to laptops having the least occlusion (either through self-occlusion or other objects).
}
\begin{figure*}[b]
    \centering
    \begin{subfigure}{0.85\textwidth}
        \includegraphics[width=\textwidth]{content/supp/img/asdf_gt_mAP.pdf}
        \caption{A-SDF-GT}
    \end{subfigure}\\
    \begin{subfigure}{0.85\textwidth}
        \includegraphics[width=\textwidth]{content/supp/img/asdf_no_gt_mAP.pdf}
        \caption{A-SDF}
    \end{subfigure}\\
    \begin{subfigure}{0.85\textwidth}
        \includegraphics[width=\textwidth]{content/supp/img/ours_mAP.pdf}
        \caption{\ourName}
    \end{subfigure}
    \caption[]{Detailed metrics for the experiment presented in \ifthenelse{\boolean{reftomain}}{\cref{subsec:full_pipeline_task}}{Sec.~4.3}. We report the average precision for the 3D IoU and the pose prediction for each category in our test set as well as the mean over all instances. It can be observed that overall the mean 3D IoU is lower for \textit{\ourName{}} compared to \textit{A-SDF-GT} and \textit{A-SDF}. For \textit{A-SDF-GT} the laptop and microwave category stands out as they are mostly placed on counters and thus they are less occluded than other objects. As expected, the poses predicted by \textit{A-SDF} and \textit{\ourName{}} are similar as they both use the same pose map predicted by our encoder.}
    \label{supp:fig:3d_metrics}
\end{figure*}

\begin{figure*}[b]
    \centering
    \begin{subfigure}{0.85\textwidth}
        \includegraphics[width=\textwidth]{content/supp/img/asdf_gt_mAP_chamfer_joint.pdf}
        \caption{A-SDF-GT}
    \end{subfigure}\\
    \begin{subfigure}{0.85\textwidth}
        \includegraphics[width=\textwidth]{content/supp/img/asdf_no_gt_mAP_chamfer_joint.pdf}
        \caption{A-SDF}
    \end{subfigure}\\
    \begin{subfigure}{0.85\textwidth}
        \includegraphics[width=\textwidth]{content/supp/img/ours_mAP_chamfer_joint.pdf}
        \caption{\ourName{}}
    \end{subfigure}
    \caption[]{Additional metrics for the experiment presented in \ifthenelse{\boolean{reftomain}}{\cref{subsec:full_pipeline_task}}{Sec.~4.3}. In addition to the metrics already reported in \ifthenelse{\boolean{reftomain}}{\cref{subsec:full_pipeline_task}}{Sec.~4.3}, we report the more fine-grained object-centric Chamfer distance as well as the joint state prediction error. Both metrics show a similar trend as the more coarse 3D IoU. One can observe though, that for the laptop category \textit{A-SDF-GT} performs significantly better than all other categories. Compared to that \textit{A-SDF}, which uses a predicted segmentation masks, does not show this special behavior for laptops. As laptops are small and the segmentation mask is very thin, this gap in performance highlights potential failure cases of an optimization-based method due to imperfect segmentation masks.}
    \label{supp:fig:chamfer_joint_metrics}
\end{figure*}

\subsection{\ourName{} RGB-D Version}
\label{supp:subsec:rgbd_version}
{\parskip=0pt
In addition to the proposed stereo-RGB input version of \textit{\ourName{}}, we also evaluated and tested an RGB-D version \textit{\ourName{}-D}. 
We report quantitative results on the same synthetic dataset in \cref{supp:tab:stereo_rgbd} as well as compare the detections qualitatively in \cref{supp:fig:stereo_rgbd_comparison}. 

Quantitatively, the \textit{\ourName{}-D} performs slightly better compared to our proposed stereo RGB version. This is to be expected given that \ourName{} needs to learn the notion of depth first whereas \textit{\ourName{}-D} does not. Contrary to this observation, in our real world experiments, we do not get a single meaningful detection using the RGB-D input version (see \cref{supp:fig:stereo_rgbd_comparison}). Thus, overall, we decided for the proposed stereo version of \ourName{}.
}
\begin{table}
    \centering
    \footnotesize
    \caption{Full Scene Reconstructions Results with RGB-D Input.}
    \label{supp:tab:stereo_rgbd}
    \begin{tabular}{l|cccc}
        \toprule
            Method & IOU25 $\uparrow$ & IOU50 $\uparrow$ & $10\si{\degree}10\si{cm} \uparrow$ & $20\si{\degree}30\si{cm} \uparrow$  \\
        \midrule
            \ourName & {64.0} & {31.5} & \textbf{28.7} & {76.6} \\
            \ourName-D & \textbf{67.8} & \textbf{38.2} & {27.0} & \textbf{84.7} \\
        \bottomrule
    \end{tabular}
\end{table}%

\begin{figure*}[b]
    \centering
    \includegraphics[width=\textwidth]{content/supp/img/qualitative_comparison_stereo_rgbd.jpg}
    \caption{Stereo-RGB Image (Left) vs. RGB-D Image (Right) Input. The first row shows a successful detection and reconstruction of \ourName{} in an office kitchen environment. Second row shows a reconstruction of a cabinet. Eventhough, \ourName{} has never seen objects from this category it highlight its generalization beyond the trained categories. The third and fourth row show two failure cases of either no detection at all (third row) or a misdetection of a laptop on the kitchen counter (fourth row). CARTO with RGB-D input is not able to reconstruct any objects.}
    \label{supp:fig:stereo_rgbd_comparison}
\end{figure*}


\end{document}
