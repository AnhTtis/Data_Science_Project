\documentclass[article,superscriptaddress]{revtex4-2}

\usepackage{graphicx}
\usepackage{amsmath,amsfonts,amssymb,amsthm,color}
\allowdisplaybreaks
%\renewcommand\Affilfont{\itshape\small}
%\usepackage{epstopdf,epsfig}
%\usepackage{newtxtext}
%\usepackage{newtxmath}
%\usepackage[square,numbers]{natbib}

\renewcommand{\figurename}{Supplementary Fig.}

\usepackage{hyperref}
\hypersetup{
    colorlinks = true,
    urlcolor   = blue,
    citecolor  = black,
}
\newtheorem{lemma}{Lemma}
\newtheorem{corollary}{Corollary}
\newcommand{\RomanNumeralCaps}[1]

%\usepackage{float}
%\usepackage{caption}
%\captionsetup{justification=justified,width=\textwidth}
\usepackage[linesnumbered,ruled,vlined]{algorithm2e}

\begin{document}

\title{Supplementary information for: \\ Detecting hidden structures from a static loading experiment: \\ topology optimization meets physics-informed neural networks}

\author{Saviz Mowlavi}
 \email{mowlavi@merl.com}
\affiliation{Mitsubishi Electric Research Laboratories, Cambridge, MA 02139, USA}
\affiliation{Department of Mechanical Engineering, MIT, Cambridge, MA 02139, USA}
\author{Ken Kamrin}%
 \email{kkamrin@mit.edu}
\affiliation{Department of Mechanical Engineering, MIT, Cambridge, MA 02139, USA}

\begin{abstract}
\end{abstract}

\maketitle

\section{Uniqueness of solutions}

We sketch a proof for the uniqueness of solutions to the elasticity imaging problem considered in this paper. For the specific case of a 2D linear elastic material with a single void, it has been proved that there exists at most one cavity which yields the same surface displacements and stresses on a finite portion of the external boundary \citep{ang1999}. Our approach is different and applicable to any 2D or 3D problem governed by analytic and elliptic PDEs, be they linear or nonlinear, heat conduction, elasticity, etc. Although we present the proof for the case of a single void or inclusion, the same reasoning readily generalizes to any number of voids or inclusions.

\subsection{Useful lemmas}

Before sketching the proof, we state a few lemmas that will be useful:
\begin{itemize}
\item \textbf{Lemma 1 (Cauchy–Kowalevski theorem)} Consider the Cauchy problem defined by a PDE in a connected domain $\Omega$ with boundary conditions specified by Cauchy data on a hypersurface $S \in \Omega$. (The Cauchy data defines the values of the solution and its normal derivatives of order up to $k-1$ on $S$, where $k$ is the order of the PDE.) If the functions defining the PDE, the Cauchy data, and the hypersurface are analytic, then the Cauchy problem has a unique analytic solution in a neighborhood of $S$ \citep{folland2020}.
\item \textbf{Lemma 2}. Any $C^2$ solution of a linear or nonlinear elliptic PDE is a real analytic function \citep{morrey1958}.
\item \textbf{Lemma 3 (identity theorem)}. Any two real analytic functions on a path-connected domain $\Omega$ that are equal on a finite and connected subset $\mathcal{S} \subset \Omega$, are necessarily equal in all of $\Omega$ \citep{krantz2002}. As a particular case, if a real analytic function defined on $\Omega$ vanishes in a subset $\mathcal{S} \subset \Omega$, then it necessarily vanishes in all of $\Omega$.
\end{itemize}

\subsection{Setup}

%%%%
\begin{figure}[tb]
\centering
\includegraphics[width=\textwidth]{Figures/NCS_Proof}
\caption{\textbf{Setup for the proof of the uniqueness of solutions.} \textbf{a},\textbf{b}, Two identically-shaped bodies with different inclusions are assumed to have the same surface displacements and tractions along a portion $\partial \mathcal{B}^m$ of their outer boundary. \textbf{c}, In the case of a void, the blue region $\partial \tilde{\mathcal{B}}^{(1)}$ of the first body has vanishing stress. \textbf{d}, In the case of a rigid inclusion, the blue region $\partial \tilde{\mathcal{B}}^{(1)}$ of the first body undergoes a rigid displacement.}
\label{fig:Proof}
\end{figure}
%%%%

Consider two bodies $\mathcal{B}^{(1)}$ and $\mathcal{B}^{(2)}$ with the same material properties and sharing the same external boundary $\partial \mathcal{B}^\mathrm{ext}$ (Supplementary Fig.~\ref{fig:Proof}a,b). Each body contains a single smooth void or rigid inclusion, characterized by the connected domains $\mathcal{I}^{(1)}$ and $\mathcal{I}^{(2)}$ with boundaries $\partial \mathcal{I}^{(1)}$ and $\partial \mathcal{I}^{(2)}$, respectively. 
%The two bodies are subjected to the same external loading, consisting of given displacements on portions of $\partial \mathcal{B}^\mathrm{ext}$ and given tractions on the remaining portions of $\partial \mathcal{B}^\mathrm{ext}$. This loading generates a displacement and a stress field within each body, denoted respectively by $\mathbf{u}^{(1)}$, $\boldsymbol{\sigma}^{(1)}$ and $\mathbf{u}^{(2)}$, $\boldsymbol{\sigma}^{(2)}$, and it is assumed that the boundary tractions $\mathbf{t}^{(1)}$ and $\mathbf{t}^{(2)}$ do not vanish everywhere on $\partial \mathcal{B}^\mathrm{ext}$. Finally, we measure identical surface displacements or tractions on a finite portion $\partial \mathcal{B}^m$ of $\partial \mathcal{B}^\mathrm{ext}$. Thus, we have Cauchy data $\mathbf{u}^{(1)} = \mathbf{u}^{(2)} = \bar{\mathbf{u}}$ and $\mathbf{t}^{(1)} = \mathbf{t}^{(2)} = \bar{\mathbf{t}}$ on $\partial \mathcal{B}^m$.
Assume that there exists a finite connected segment $\partial \mathcal{B}^m \subset \partial \mathcal{B}^\mathrm{ext}$ on which the two bodies have identical surface displacements and tractions, yielding the Cauchy data $\mathbf{u}^{(1)} = \mathbf{u}^{(2)} = \bar{\mathbf{u}}$ and $\mathbf{t}^{(1)} = \mathbf{t}^{(2)} = \bar{\mathbf{t}}$ on $\partial \mathcal{B}^m$. In practice, such Cauchy data is obtained by knowing the applied displacement or traction and measuring the other quantity. Further, assume that the displacement and stress fields within each body, denoted by $\mathbf{u}^{(1)}$, $\boldsymbol{\sigma}^{(1)}$ and $\mathbf{u}^{(2)}$, $\boldsymbol{\sigma}^{(2)}$, are $C^2$ over $\mathcal{B}^{(1)}$ and $\mathcal{B}^{(2)}$, respectively. Finally, assume that $\bar{\mathbf{u}}$ and $\bar{\mathbf{t}}$ are analytic functions and $\bar{\mathbf{t}}$ does not vanish everywhere on $\partial \mathcal{B}^m$. We will prove that there cannot be two distinct shapes $\mathcal{I}^{(1)}$ and $\mathcal{I}^{(2)}$ yielding the same displacement and tractions on $\partial \mathcal{B}^m$. 

\subsection{Sketch of the proof}
The displacement fields $\mathbf{u}^{(1)}$ and $\mathbf{u}^{(2)}$ solve the equilibrium equations of elasticity. These equations are elliptic, and they are also analytic as defined in the sense of Lemma 1. As a result, Lemma 2 stipulates that $\mathbf{u}^{(1)}$ and $\mathbf{u}^{(2)}$ are real analytic functions over $\mathcal{B}^{(1)}$ and $\mathcal{B}^{(2)}$, respectively, and in addition, from Lemma 1, they must be equal in a neighborhood of $\partial \mathcal{B}^m$. Then, Lemma 3 implies that $\mathbf{u}^{(1)} = \mathbf{u}^{(2)}$, and therefore $\boldsymbol{\sigma}^{(1)} = \boldsymbol{\sigma}^{(2)}$, over $\mathcal{B}^{(1)} \cap \mathcal{B}^{(2)}$. We now focus on the behavior of the solution in body $\mathcal{B}^{(1)}$ and treat separately the cases of a void and a rigid inclusion. 

\subsubsection{Void}

Consider that each body contains a void, and let us focus on body $\mathcal{B}^{(1)}$. We will analyze the region $\tilde{\mathcal{B}}^{(1)} = \mathcal{B}^{(1)} \cap \mathcal{I}^{(2)}$ pictured in blue in Supplementary Fig.~\ref{fig:Proof}c. Assume first that the voids $\mathcal{I}^{(1)}$ and $\mathcal{I}^{(2)}$ overlap; then, the boundary delineating $\tilde{\mathcal{B}}^{(1)}$ contains one segment $\partial \mathcal{I}^{(1)} \cap \mathcal{I}^{(2)}$ and another segment $\partial \mathcal{I}^{(2)} \cap \mathcal{B}^{(1)}$. The traction $\mathbf{t}^{(1)}$ trivially vanishes on $\partial \mathcal{I}^{(1)} \cap \mathcal{I}^{(2)}$. Since $\boldsymbol{\sigma}^{(1)} = \boldsymbol{\sigma}^{(2)}$ in $\mathcal{B}^{(1)} \cap \mathcal{B}^{(2)}$, $\mathbf{t}^{(1)}$ also vanishes on $\partial \mathcal{I}^{(2)} \cap \mathcal{B}^{(1)}$. Assume now that the voids $\mathcal{I}^{(1)}$ and $\mathcal{I}^{(2)}$ do not overlap; then, the entire boundary delineating $\tilde{\mathcal{B}}^{(1)}$ is equal to $\partial \mathcal{I}^{(2)} \cap \mathcal{B}^{(1)}$, along which we just showed that $\mathbf{t}^{(1)}$ vanishes. Therefore, in both cases, the region $\tilde{\mathcal{B}}^{(1)}$ has zero traction along its entire boundary, implying that $\boldsymbol{\sigma}^{(1)}$ vanishes in $\tilde{\mathcal{B}}^{(1)}$.

Then, Lemma 3 implies that $\boldsymbol{\sigma}^{(1)}$ must vanish not only in $\tilde{\mathcal{B}}^{(1)}$, but also in the entire body $\mathcal{B}^{(1)}$.  This violates the fact that the surface traction $\mathbf{t}^{(1)}$ cannot vanish everywhere on $\partial \mathcal{B}^m$, so $\mathcal{I}^{(1)}$ and $\mathcal{I}^{(2)}$ must be identical. 

Note that to use Lemma 3 in the prior paragraph, we must assume that a path-connection exists between all points of $\tilde{\mathcal{B}}^{(1)}$ and any point in $\mathcal{B}^{(1)}$.  This occurs if each void is a simply connected domain.  As a counter-example, suppose a void contains a disconnected island of material within it (a ``rattler'' so to speak). We should not expect to be able to determine the shape of the rattler based on data from $\partial\mathcal{B}^{ext}$ because the rattler is completely disconnected from the rest of the body.  The path-connectedness caveat of Lemma 3 removes these cases from the proof. That is, only those void boundaries that are path-connected to $\partial\mathcal{B}^{ext}$ can be  uniquely identified from data on $\partial\mathcal{B}^{ext}$.

\subsubsection{Rigid inclusion}

Consider that each body contains a rigid inclusion, and let us focus on body $\mathcal{B}^{(1)}$. We will analyze the region $\tilde{\mathcal{B}}^{(1)} = \mathcal{B}^{(1)} \cap \mathcal{I}^{(2)}$ pictured in blue in Supplementary Fig.~\ref{fig:Proof}d. Assume first that the voids $\mathcal{I}^{(1)}$ and $\mathcal{I}^{(2)}$ overlap; then, the boundary delineating $\tilde{\mathcal{B}}^{(1)}$ contains one segment $\partial \mathcal{I}^{(1)} \cap \mathcal{I}^{(2)}$ and another segment $\partial \mathcal{I}^{(2)} \cap \mathcal{B}^{(1)}$. The displacement $\mathbf{u}^{(1)}$ on $\partial \mathcal{I}^{(1)} \cap \mathcal{I}^{(2)}$ corresponds to that of a rigid motion, i.e.~$\mathbf{u}^{(1)} = \mathbf{u}_r^{(1)} + \boldsymbol{\theta}_r^{(1)} \times \mathbf{x}$ for some fixed $\mathbf{u}_r^{(1)}$ and $\boldsymbol{\theta}_r^{(1)}$. Since $\mathbf{u}^{(1)} = \mathbf{u}^{(2)}$ in $\mathcal{B}^{(1)} \cap \mathcal{B}^{(2)}$, we also know that $\mathbf{u}^{(1)} = \mathbf{u}^{(2)} = \mathbf{u}_r^{(2)} + \boldsymbol{\theta}_r^{(2)} \times \mathbf{x}$ on $\partial \mathcal{I}^{(2)} \cap \mathcal{B}^{(1)}$ for some other fixed $\mathbf{u}_r^{(2)}$ and $\boldsymbol{\theta}_r^{(2)}$. However, given that these two rigid motions must coincide at $\partial \mathcal{I}^{(1)} \cap \partial \mathcal{I}^{(2)}$, we must have $\mathbf{u}_r^{(1)} = \mathbf{u}_r^{(2)}$ and $\boldsymbol{\theta}_r^{(1)} = \boldsymbol{\theta}_r^{(2)}$. Assume now that the voids $\mathcal{I}^{(1)}$ and $\mathcal{I}^{(2)}$ do not overlap; then, the entire boundary delineating $\tilde{\mathcal{B}}^{(1)}$ is equal to $\partial \mathcal{I}^{(2)} \cap \mathcal{B}^{(1)}$, along which we just showed that $\mathbf{u}^{(1)} = \mathbf{u}_r^{(2)} + \boldsymbol{\theta}_r^{(2)} \times \mathbf{x}$ on $\partial \mathcal{I}^{(2)} \cap \mathcal{B}^{(1)}$. Therefore, in both cases, the entire boundary of the region $\tilde{\mathcal{B}}^{(1)}$ undergoes a single rigid motion, implying that $\boldsymbol{\sigma}^{(1)}$ vanishes in $\tilde{\mathcal{B}}^{(1)}$. 

As before, Lemma 3 then implies that $\boldsymbol{\sigma}^{(1)}$ must vanish not only in $\tilde{\mathcal{B}}^{(1)}$, but also in the entire body $\mathcal{B}^{(1)}$. This violates the fact that the surface traction $\mathbf{t}^{(1)}$ cannot vanish everywhere on $\partial \mathcal{B}^\mathrm{ext}$, so $\mathcal{I}^{(1)}$ and $\mathcal{I}^{(2)}$ must be identical.  Note that, as in the case of voids, we require path-connectedness between $\tilde{\mathcal{B}}^{(1)}$ and any point in $\mathcal{B}^{(1)}$, which is assured by the assumption that the rigid inclusions are simply connected, i.e. have fully-rigid interiors.

\section{Governing equations}

In this Appendix, we describe the governing equations, boundary conditions, and measurement data for all cases studied in this paper. Cases listed in Tabs.~I and II and involving a linear elastic matrix are described in Section \ref{app:SmallDeformationLinearElasticity}, while those involving a hyperelastic matrix are described in Section \ref{app:LargeDeformationNonlinearHyperelasticity}. Cases listed in Tab.~III are described in Section \ref{app:ThermalExperiments}.

\subsection{Mechanical experiments}
\label{app:MechanicalExperiments}

\subsubsection{Small-deformation linear elasticity} 
\label{app:SmallDeformationLinearElasticity}

\textbf{Physical quantities.} We first consider the case where the elastic body and inclusions consist of linear elastic materials, with Young's modulus $E$ and Poisson's ratio $\nu$ for the body, and Young's modulus $\bar{E}$ and Poisson's ratio $\bar{\nu}$ for the inclusions. Voids and rigid inclusions correspond to the limits $\bar{E} \rightarrow 0$ and $\bar{E} \rightarrow \infty$, respectively. The deformation of the elastic body is described by the vector field $\boldsymbol{\psi}(\mathbf{x}) = (\mathbf{u}(\mathbf{x}), \boldsymbol{\sigma}(\mathbf{x}))$, where $\mathbf{u}(\mathbf{x})$ is the planar displacement field with components $u_i(\mathbf{x})$ and $\boldsymbol{\sigma}(\mathbf{x})$ is the Cauchy stress tensor with components $\sigma_{ij}(\mathbf{x})$. Indices $i$ and $j$ will hereafter range from 1 to 2 for 2D cases, and from 1 to 3 for 3D cases.

\textbf{Governing PDEs.} The governing PDEs comprise the equilibrium equations
\begin{equation}
\sum_{j}\frac{\partial \sigma_{ij}}{\partial x_j} = 0, \quad \mathbf{x} \in \Omega,
\label{eq:LinearEquilibrium}
\end{equation}
as well as a linear elastic constitutive law $F(\boldsymbol{\sigma}, \nabla \mathbf{u}, \rho) = 0$ that we will express in two different but equivalent ways, depending on whether the inclusions are softer or stiffer than the matrix. For voids and soft inclusions, we consider the constitutive law in stress-strain form,
\begin{equation}
\boldsymbol{\sigma} = \rho \left[ \lambda \, \mathrm{tr} (\boldsymbol{\epsilon}) \, \mathbf{I} + 2 \mu \, \boldsymbol{\epsilon} \right] + (1 - \rho) \left[ \bar{\lambda} \, \mathrm{tr} (\boldsymbol{\epsilon}) \, \mathbf{I} + 2 \bar{\mu} \, \boldsymbol{\epsilon} \right], \quad \mathbf{x} \in \Omega,
\label{eq:LinearStressStrainVoid}
\end{equation}
where $\boldsymbol{\epsilon} = (\nabla \mathbf{u} + \nabla \mathbf{u}^T)/2$ is the infinitesimal strain tensor, $\mathrm{tr}(\cdot)$ denotes the trace, $\lambda = E \nu/(1+\nu)(1-2\nu)$ and $\mu = E/2(1+\nu)$ are the Lam\'e constants of the body, and $\bar{\lambda} = \bar{E} \bar{\nu}/(1+\bar{\nu})(1-2\bar{\nu})$ and $\bar{\mu} = \bar{E}/2(1+\bar{\nu})$ are the Lam\'e constants of the inclusions. Notice that in the case of voids, the stress vanishes in the $\rho = 0$ regions. For stiff and rigid inclusions, 
%the strain must vanish in the $\rho = 0$ regions, 
we consider the constitutive law in the inverted strain-stress form, which differs in 2D and 3D due to the plane strain assumption. In 2D,
\begin{equation}
\boldsymbol{\epsilon} = \rho \left[ \frac{1+\nu}{E} \boldsymbol{\sigma} - \frac{\nu(1+\nu)}{E} \, \mathrm{tr} (\boldsymbol{\sigma}) \, \mathbf{I} \right] + (1-\rho) \left[ \frac{1+\bar{\nu}}{\bar{E}} \boldsymbol{\sigma} - \frac{\bar{\nu}(1+\bar{\nu})}{\bar{E}} \, \mathrm{tr} (\boldsymbol{\sigma}) \, \mathbf{I} \right], \quad \mathbf{x} \in \Omega,
\label{eq:LinearStressStrainInclusion2D}
\end{equation}while in 3D,
\begin{equation}
\boldsymbol{\epsilon} = \rho \left[ \frac{1+\nu}{E} \boldsymbol{\sigma} - \frac{\nu}{E} \, \mathrm{tr} (\boldsymbol{\sigma}) \, \mathbf{I} \right] + (1-\rho) \left[ \frac{1+\bar{\nu}}{\bar{E}} \boldsymbol{\sigma} - \frac{\bar{\nu}}{\bar{E}} \, \mathrm{tr} (\boldsymbol{\sigma}) \, \mathbf{I} \right], \quad \mathbf{x} \in \Omega.
\label{eq:LinearStressStrainInclusion3D}
\end{equation}
Notice that in the case of rigid inclusions, the strain vanishes in the $\rho = 0$ regions.

\textbf{Boundary conditions.} For the 2D square elastic matrix (Fig.~1a), the domain is $\Omega = [-0.5,0.5] \times [-0.5,0.5]$. The boundary conditions are
\begin{subequations}
\begin{alignat}{2}
\boldsymbol{\sigma}(\mathbf{x}) \mathbf{n}(\mathbf{x}) &= \pm P_o \mathbf{e}_1, &\quad &\mathbf{x} \in \{ \pm 0.5 \} \times [-0.5,0.5], \\
\boldsymbol{\sigma}(\mathbf{x}) \mathbf{n}(\mathbf{x}) &= \mathbf{0}, \quad &&\mathbf{x} \in [-0.5,0.5] \times \{ \pm 0.5 \},
\end{alignat} \label{eq:2DMatrixBCs}%
\end{subequations}
where $P_o/E = 0.01$, and the displacement $\mathbf{u}(\mathbf{x})$ is measured at locations $\partial \Omega^m$ distributed along the entire external boundary $\partial \Omega$. In the case of the M, I, T inclusions, the domain is $\Omega = [-1,1] \times [-0.5,0.5]$ and the boundary conditions \eqref{eq:2DMatrixBCs} are changed to account for the fact that the matrix is pulled from the top and bottom boundaries.

For the 2D half-star elastic matrix (Fig.~6, top row), the domain $\Omega$ is delineated by the top boundary $\partial\Omega_\mathrm{top} = \{ \mathbf{x} = (x,y) : x \in [-r^*, r^*], y = 0 \}$ and the wavy boundary $\partial\Omega_\mathrm{wavy} = \{ \mathbf{x} = (x,y) : x = r(\theta) \cos \theta, y = r(\theta) \sin \theta, r(\theta) = (r^* + a^* \sin 7 \theta), \theta \in [\pi, 2 \pi] \}$, with $r^* = 0.45$ and $a^* = 0.07$. Because the half-star is pulled by gravity, a body force $-g \mathbf{e}_2$ is added to the right-hand side of the equilibrium relation \eqref{eq:LinearEquilibrium}. The boundary conditions are
\begin{subequations}
\begin{alignat}{2}
\mathbf{u}(\mathbf{x}) &= \mathbf{0}, &\quad &\mathbf{x} \in \partial\Omega_\mathrm{top}, \\
\boldsymbol{\sigma}(\mathbf{x}) \mathbf{n}(\mathbf{x}) &= \mathbf{0}, \quad &&\mathbf{x} \in \partial\Omega_\mathrm{wavy},
\end{alignat} \label{eq:HalfStarMatrixBCs}%
\end{subequations}
and the displacement $\mathbf{u}(\mathbf{x})$ is measured at locations $\partial \Omega^m$ distributed along the exposed wavy boundary $\partial\Omega_\mathrm{wavy}$.

For the 3D square elastic matrix (Fig.~6, middle row), the domain is $\Omega = [-0.5,0.5] \times [-0.5,0.5] \times [-0.5,0.5]$. The boundary conditions are
\begin{subequations}
\begin{alignat}{2}
\boldsymbol{\sigma}(\mathbf{x}) \mathbf{n}(\mathbf{x}) &= \pm P_o \mathbf{e}_1, &\quad &\mathbf{x} \in \{\pm 0.5\} \times [-0.5,0.5] \times [-0.5,0.5], \\
\boldsymbol{\sigma}(\mathbf{x}) \mathbf{n}(\mathbf{x}) &= \mathbf{0}, \quad &&\mathbf{x} \in [-0.5,0.5] \times \{\pm 0.5\} \times [-0.5,0.5], \\
\boldsymbol{\sigma}(\mathbf{x}) \mathbf{n}(\mathbf{x}) &= \mathbf{0}, \quad &&\mathbf{x} \in [-0.5,0.5] \times [-0.5,0.5] \times \{\pm 0.5\},
\end{alignat} \label{eq:3DMatrixBCs}%
\end{subequations}
and the displacement $\mathbf{u}(\mathbf{x})$ is measured at locations $\partial \Omega^m$ distributed along the entire external boundary $\partial \Omega$.

For the elastic layer (Fig.~1b), the domain is $\Omega = [0,1] \times [-0.5,0]$. The boundary conditions are
\begin{subequations}
\begin{alignat}{2}
\boldsymbol{\sigma}(\mathbf{x}) \mathbf{n}(\mathbf{x}) &= -P_o \mathbf{e}_2, &\quad& \mathbf{x} \in [0,1] \times \{0\}, \\
\mathbf{u} &= \mathbf{0}, && \mathbf{x} \in [0,1] \times \{-0.5\},
\end{alignat} \label{eq:LayerBCs}%
\end{subequations}
in addition to periodic displacement and traction boundary conditions on $\mathbf{x} \in \{0,1\} \times [-0.5,0]$. The displacement $\mathbf{u}(\mathbf{x})$ is measured at locations $\partial \Omega^m$ distributed along the top surface $\partial \Omega_t = [0,1] \times \{0\}$.

% The geometry identification problem that we solve can then be stated as follows. Given surface displacement measurements $\mathbf{u}_i^m$ at locations $\mathbf{x}_i \in \partial \Omega^m$, find the distribution of material density $\rho$ in $\Omega$ such that the difference between the predicted and measured surface displacements vanish, that is,
% \begin{equation}
% \mathbf{u}(\mathbf{x}_i) = \mathbf{u}_i^m, \quad \mathbf{x}_i \in \partial \Omega^m.
% \end{equation}
% The predicted displacement field must satisfy the equilibrium equation \eqref{eq:LinearEquilibrium}, the constitutive relation \eqref{eq:LinearStressStrainVoid}, \eqref{eq:LinearStressStrainInclusion2D}, or \eqref{eq:LinearStressStrainInclusion3D}, and the boundary conditions \eqref{eq:MatrixBCs} or \eqref{eq:LayerBCs}.

\subsubsection{Large-deformation nonlinear hyperelasticity} 
\label{app:LargeDeformationNonlinearHyperelasticity}

\textbf{Physical quantities.} Next, we consider the case where the elastic body consists of an incompressible Neo-Hookean hyperelastic material with shear modulus $\mu$.
We now have to distinguish between the reference (undeformed) and current (deformed) configurations. 
We denote by $\mathbf{x} = (x_1,x_2) \in \Omega$ and $\mathbf{y} = (y_1,y_2) \in \Omega^*$ the coordinates in the reference and deformed configurations, respectively, with $\Omega^*$ the deformed image of $\Omega$. 
The displacement field $\mathbf{u}(\mathbf{x})$ with components $u_i(\mathbf{x})$ moves an initial position $\mathbf{x} \in \Omega$ into its current location $\mathbf{y} = \mathbf{x} + \mathbf{u}(\mathbf{x}) \in \Omega^*$. 
In order to formulate the governing equations and boundary conditions in the reference configuration $\Omega$, we need to introduce the first Piola-Kirchhoff stress tensor $\mathbf{S}(\mathbf{x})$ with components $S_{ij}(\mathbf{x})$. Unlike the Cauchy stress tensor, the first Piola-Kirchhoff stress tensor is defined in $\Omega$ and is not symmetric. The deformation of the elastic body is then described by the vector field $\boldsymbol{\psi}(\mathbf{x}) = (\mathbf{u}(\mathbf{x}), \mathbf{S}(\mathbf{x}),p(\mathbf{x}))$ defined over $\Omega$, where $p(\mathbf{x})$ is a pressure field that serves to enforce the incompressibility constraint.

\textbf{Governing PDEs.} The equilibrium equations are
\begin{equation}
\sum_{j}\frac{\partial S_{ij}}{\partial x_j} = 0, \quad \mathbf{x} \in \Omega,
\label{eq:NonlinearEquilibrium}
\end{equation}
where the derivatives in $\nabla_\mathbf{x}$ are taken with respect to the reference coordinates $\mathbf{x}$. We only consider the presence of voids so that the nonlinear constitutive law $F(\mathbf{S}, \nabla_\mathbf{x} \mathbf{u}, p, \rho) = 0$ is simply expressed as
\begin{equation}
\mathbf{S} = \rho \left[ -p \mathbf{F}^{-T} + \mu \mathbf{F} \right], \quad \mathbf{x} \in \Omega,
\label{eq:NonlinearStressStrain}
\end{equation}
where $\mathbf{F}(\mathbf{x}) = \mathbf{I} + \nabla_\mathbf{x} \mathbf{u}(\mathbf{x})$ is the deformation gradient tensor. Notice that the stress vanishes in the $\rho = 0$ regions. Finally, we have the incompressibility constraint
\begin{equation}
\rho \left[ \det(\mathbf{F}) - 1 \right] = 0, \quad \mathbf{x} \in \Omega,
\label{eq:NonlinearIncompressibility}
\end{equation}
which turns itself off in the $\rho = 0$ regions since voids do not deform in a way that preserves volume.

\textbf{Boundary conditions.} We only treat the matrix problem (Fig.~1a) in this hyperelastic case. The domain is $\Omega = [-0.5,0.5] \times [-0.5,0.5]$ and the boundary conditions are
\begin{subequations}
\begin{alignat}{2}
\mathbf{S}(\mathbf{x}) \mathbf{n}_0(\mathbf{x}) &= -P_o \mathbf{e}_1, &\quad &\mathbf{x} \in \{-0.5,0.5\} \times [-0.5,0.5], \\
\mathbf{S}(\mathbf{x}) \mathbf{n}_0(\mathbf{x}) &= \mathbf{0}, \quad &&\mathbf{x} \in [-0.5,0.5] \times \{-0.5,0.5\},
\end{alignat} \label{eq:MatrixBCsHyperEla}%
\end{subequations}
where $P_o/E = 0.173$. As in the linear elastic case, the displacement $\mathbf{u}(\mathbf{x})$ is measured at locations $\partial \Omega^m$ distributed along the entire external boundary $\partial \Omega$.

% The geometry identification problem can then be stated identically as in the linear elastic case. This time, the predicted displacement field must satisfy the equilibrium equation \eqref{eq:NonlinearEquilibrium}, the constitutive relation \eqref{eq:NonlinearStressStrain}, the incompressibility condition \eqref{eq:NonlinearIncompressibility}, and the boundary conditions \eqref{eq:MatrixBCsHyperEla}.

\subsection{Thermal conduction experiments} 
\label{app:ThermalExperiments}

\textbf{Physical quantities.} In heat transfer problems, the response of the body is described by the vector field $\boldsymbol{\psi}(\mathbf{x}) = (T(\mathbf{x}), \mathbf{q}(\mathbf{x}))$, where $T(\mathbf{x})$ is the temperature field and $\mathbf{q}(\mathbf{x})$ is the heat flux with components $q_i(\mathbf{x})$. Since we only consider 2D thermal imaging cases, the index $i$ will hereafter range from 1 to 2.

\textbf{Governing PDEs.} The governing PDEs comprise the heat conservation law stipulating that, at steady-state,
\begin{equation}
\sum_{i}\frac{\partial q_i}{\partial x_i} = 0, \quad \mathbf{x} \in \Omega,
\label{eq:HeatConservation}
\end{equation}
as well as Fourier's law of heat conduction relating the heat flux and the temperature. For cases involving perfectly insulating inclusions, we express Fourier's law as
\begin{equation}
\mathbf{q} + \rho k(T) \nabla T = 0, \quad \mathbf{x} \in \Omega,
\label{eq:FourierLaw}
\end{equation}
while for cases involving perfectly conductive inclusions, we express Fourier's law as
\begin{equation}
\rho \mathbf{q} + k(T) \nabla T = 0, \quad \mathbf{x} \in \Omega.
\label{eq:InvertedFourierLaw}
\end{equation}
In both expressions, $k$ denotes the thermal conductivity of the material, which may be a function of temperature. For case 26 in Tab.~III, we assume that the matrix is made of steel and therefore consider a constant thermal conductivity $k = 45 \, \mathrm{W}/\mathrm{mK}$, resulting in a linear PDE. For cases 27 and 28 in Tab.~III, we assume that the matrix is made of a material with temperature-dependent thermal conductivity $k(T) = k_0(1+T/T_0)$, where $k_0 = 1 \, \mathrm{W}/\mathrm{mK}$ is a reference thermal conductivity and $T_0 = 1 \, \mathrm{K}$ is a reference temperature, resulting in a nonlinear PDE. Notice that \eqref{eq:FourierLaw} and \eqref{eq:InvertedFourierLaw} are formulated such that when $\rho = 0$, the heat flux vanishes for perfectly insulating inclusions, while the temperature becomes uniform for perfectly conductive inclusions.

\textbf{Boundary conditions.} For all thermal cases considered in this paper, the domain is $\Omega = [-0.5,0.5] \times [-0.5,0.5]$. For case 26 in Tab.~III, the left side of the body is heated to a prescribed temperature $T_0 = 478 \, \mathrm{K}$, while the three remaining sides are left exposed to air at ambient temperature $T_a = 278 \, \mathrm{K}$. The corresponding boundary conditions are
\begin{subequations}
\begin{alignat}{2}
T(\mathbf{x}) &= T_0, &\quad &\mathbf{x} \in \{ - 0.5 \} \times [-0.5,0.5], \\
\mathbf{q}(\mathbf{x}) \cdot \mathbf{n}(\mathbf{x}) &= \epsilon \sigma (T^4 - T_a^4), \quad &&\mathbf{x} \in \{ 0.5 \} \times [-0.5,0.5] \cup [-0.5,0.5] \times \{ \pm 0.5 \}.
\end{alignat} \label{eq:MatrixBCsThermal1}%
\end{subequations}
where the three exposed sides have been modeled using a radiation boundary condition that involves the emissivity $\epsilon = 0.9$ of a rough steel surface as well as the Stefan-Boltzmann constant $\sigma = 5.67 \cdot 10^{-8} \, \mathrm{W}/(\mathrm{m}^2 \mathrm{K}^4)$. The temperature $T(\mathbf{x})$ is measured at locations $\partial \Omega^m$ distributed along the three exposed sides.

For cases 27 and 28 in Tab.~III, the boundary conditions are
\begin{subequations}
\begin{alignat}{2}
T(\mathbf{x}) &= T_0, &\quad &\mathbf{x} \in \{ - 0.5 \} \times [-0.5,0.5], \\
T(\mathbf{x}) &= T_1, &\quad &\mathbf{x} \in \{ 0.5 \} \times [-0.5,0.5], \\
\mathbf{q}(\mathbf{x}) \cdot \mathbf{n}(\mathbf{x}) &= 0, \quad &&\mathbf{x} \in [-0.5,0.5] \times \{ \pm 0.5 \}.
\end{alignat} \label{eq:MatrixBCsThermal2}%
\end{subequations}
where $T_0 = 1 \, \mathrm{K}$ and $T_1 = 0 \, \mathrm{K}$. These cases are repeated four times, in each instance assuming that both the applied boundary condition and measurements are unavailable on one of the four sides, while the temperature $T(\mathbf{x})$ or normal heat flux $\mathbf{q}(\mathbf{x}) \cdot \mathbf{n}(\mathbf{x})$ resulting from the prescribed thermal loading are measured at locations $\partial \Omega^m$ distributed along the remaining three sides.

\section{Detailed solution methodology}
\label{app:AdditionalInformationSolutionMethodology}

\subsection{Mechanical experiments}

\subsubsection{Small-deformation linear elasticity}

We describe the implementation of the 2D cases, which is easily generalized to 3D. Since the physical quantities are $\boldsymbol{\psi} = (u_1,u_2,\sigma_{11},\sigma_{22},\sigma_{12})$, we introduce the neural network approximations
\begin{subequations}
\begin{align}
u_1(\mathbf{x}) &= \bar{u}_1(\mathbf{x}; \boldsymbol{\theta}_1), \\
u_2(\mathbf{x}) &= \bar{u}_2(\mathbf{x}; \boldsymbol{\theta}_2), \\
\sigma_{11}(\mathbf{x}) &= \bar{\sigma}_{11}(\mathbf{x}; \boldsymbol{\theta}_3), \\
\sigma_{22}(\mathbf{x}) &= \bar{\sigma}_{22}(\mathbf{x}; \boldsymbol{\theta}_4), \\
\sigma_{12}(\mathbf{x}) &= \bar{\sigma}_{12}(\mathbf{x}; \boldsymbol{\theta}_5), \\
\phi(\mathbf{x}) &= \bar{\phi}(\mathbf{x}; \boldsymbol{\theta}_\phi).
\end{align} \label{eq:LinearNN}%
\end{subequations}
The last equation represents the level-set neural network, which defines the material density as $\rho(\mathbf{x}) = \bar{\rho}(\mathbf{x};\boldsymbol{\theta}_\phi) = \mathrm{sigmoid}(\bar{\phi}(\mathbf{x};\boldsymbol{\theta}_\phi)/\delta)$. We then formulate the loss function $\mathcal{L}$ by specializing the loss term expressions described in the main text to the linear elasticity problem, using the governing equations given in Appendix \ref{app:SmallDeformationLinearElasticity}. Omitting the $\boldsymbol{\theta}$'s for notational simplicity, we obtain
\begin{subequations}
\begin{align}
\mathcal{L}_\mathrm{meas}(\boldsymbol{\theta}_{\boldsymbol{\psi}}) &= \frac{1}{|\partial \Omega^m|} \sum_{\mathbf{x}_i \in \partial \Omega^m}  |\bar{\mathbf{u}}(\mathbf{x}_i) - \mathbf{u}_i^m|^2, \label{eq:LinearLmeas} \\
\mathcal{L}_\mathrm{gov}(\boldsymbol{\theta}_{\boldsymbol{\psi}}, \boldsymbol{\theta}_\phi) &= \frac{1}{|\Omega^d|} \sum_{\mathbf{x}_i \in \Omega^d} |\mathbf{r}_\mathrm{eq}(\bar{\boldsymbol{\sigma}}(\mathbf{x}_i))|^2 + \frac{1}{|\Omega^d|} \sum_{\mathbf{x}_i \in \Omega^d} |\mathbf{r}_\mathrm{cr}(\bar{\mathbf{u}}(\mathbf{x}_i),\bar{\boldsymbol{\sigma}}(\mathbf{x}_i),\bar{\rho}(\mathbf{x}_i))|^2, \label{eq:LinearLF},
\end{align}
\end{subequations}
where $\bar{\mathbf{u}} = (\bar{u}_1,\bar{u}_2)$ and $\bar{\boldsymbol{\sigma}}$ has components $\bar{\sigma}_{i,j}$, $i,j=1,2$. In \eqref{eq:LinearLF}, the terms $\mathbf{r}_\mathrm{eq}$ and $\mathbf{r}_\mathrm{cr}$ refer to the residuals of the equilibrium equation \eqref{eq:LinearEquilibrium} and the constitutive relation \eqref{eq:LinearStressStrainVoid}, \eqref{eq:LinearStressStrainInclusion2D}, or \eqref{eq:LinearStressStrainInclusion3D}. The eikonal loss term is problem-independent and therefore identical to the expression given in (6) in the main text.

We note that instead of defining neural network approximations for the displacements and the stresses, we could define neural network approximations solely for the displacements, that is, $\boldsymbol{\psi} = (u_1,u_2)$. In this case, the loss term \eqref{eq:LinearLF} would only include the residual of the equilibrium equation \eqref{eq:LinearEquilibrium}, in which the stress components would be directly expressed in terms of the displacements and the material distribution using the constitutive relation \eqref{eq:LinearStressStrainVoid}. However, several recent studies \citep{rao2021,haghighat2021,henkes2022,rezaei2022,gladstone2022,harandi2023} have shown that the mixed formulation adopted in the present work results in superior accuracy and training performance, which could partly be explained by the fact that only first-order derivatives of the neural network outputs are involved since the displacements and stresses are only differentiated to first oder in \eqref{eq:LinearEquilibrium} and \eqref{eq:LinearStressStrainVoid}. In our case, the mixed formulation holds the additional advantage that it enables us to treat stiff and rigid inclusions using the inverted constitutive relation \eqref{eq:LinearStressStrainInclusion2D} instead of \eqref{eq:LinearStressStrainVoid}. Finally, the mixed formulation allows us to directly integrate both displacement and traction boundary conditions into the output of the neural network approximations, as we describe in the next paragraph.

We design the architecture of the neural networks in such a way that they inherently satisfy the boundary conditions, treating the latter as hard constraints \citep{dong2021,sukumar2022}. For the square elastic matrix, we do this through the transformations
\begin{subequations}
\begin{align}
\bar{u}_1(\mathbf{x}; \boldsymbol{\theta}_1) &= \bar{u}_1'(\mathbf{x}; \boldsymbol{\theta}_1), \\
\bar{u}_2(\mathbf{x}; \boldsymbol{\theta}_2) &= \bar{u}_2'(\mathbf{x}; \boldsymbol{\theta}_2), \\
\bar{\sigma}_{11}(\mathbf{x}; \boldsymbol{\theta}_3) &= (x-0.5)(x+0.5) \, \bar{\sigma}_{11}'(\mathbf{x}; \boldsymbol{\theta}_3) + P_o, \\
\bar{\sigma}_{22}(\mathbf{x}; \boldsymbol{\theta}_4) &= (y-0.5)(y+0.5) \, \bar{\sigma}_{22}'(\mathbf{x}; \boldsymbol{\theta}_4), \\
\bar{\sigma}_{12}(\mathbf{x}; \boldsymbol{\theta}_5) &= (x-0.5)(x+0.5)(y-0.5)(y+0.5) \, \bar{\sigma}_{12}'(\mathbf{x}; \boldsymbol{\theta}_5), \\
\bar{\phi}(\mathbf{x}; \boldsymbol{\theta}_\phi) &= (x-0.5)(x+0.5)(y-0.5)(y+0.5) \, \bar{\phi}'(\mathbf{x}; \boldsymbol{\theta}_\phi) + w,
\end{align}
\end{subequations}
where the quantities with a prime denote the raw output of the neural network. In this way, the neural network approximations defined in \eqref{eq:LinearNN} obey by construction the boundary conditions \eqref{eq:2DMatrixBCs}. Further, since we know that the elastic material is present all along the outer surface $\partial \Omega$, we define $\bar{\phi}$ so that $\rho = \mathrm{sigmoid}(\bar{\phi}/\delta) \simeq 1$ on $\partial \Omega$ (recall that $w = 10 \delta$). In the case of the M, I, T inclusions, these transformations are trivially changed to reflect the fact that the matrix is wider and pulled from the top and bottom. 

The half-star elastic matrix calls for more complex transformations due to its wavy boundary. Let
\begin{subequations}
\begin{align}
\bar{u}_1(\mathbf{x}; \boldsymbol{\theta}_1) &= y \, \bar{u}_1'(\mathbf{x}; \boldsymbol{\theta}_1), \\
\bar{u}_2(\mathbf{x}; \boldsymbol{\theta}_2) &= y \, \bar{u}_2'(\mathbf{x}; \boldsymbol{\theta}_2), \\
\bar{\sigma}_{11}(\mathbf{x}; \boldsymbol{\theta}_3, \boldsymbol{\theta}_6) &= \Phi(\mathbf{x}) \bar{\sigma}_{11}'(\mathbf{x}; \boldsymbol{\theta}_3) + f_{11}(\mathbf{x}) \bar{\sigma}_{mm}'(\mathbf{x}; \boldsymbol{\theta}_6), \label{eq:HalfStarStress11} \\
\bar{\sigma}_{22}(\mathbf{x}; \boldsymbol{\theta}_4, \boldsymbol{\theta}_6) &= \Phi(\mathbf{x}) \bar{\sigma}_{22}'(\mathbf{x}; \boldsymbol{\theta}_4) + f_{22}(\mathbf{x}) \bar{\sigma}_{mm}'(\mathbf{x}; \boldsymbol{\theta}_6), \\
\bar{\sigma}_{12}(\mathbf{x}; \boldsymbol{\theta}_5, \boldsymbol{\theta}_6) &= \Phi(\mathbf{x}) \bar{\sigma}_{12}'(\mathbf{x}; \boldsymbol{\theta}_5) + f_{12}(\mathbf{x}) \bar{\sigma}_{mm}'(\mathbf{x}; \boldsymbol{\theta}_6), \label{eq:HalfStarStress12} \\
\bar{\phi}(\mathbf{x}; \boldsymbol{\theta}_\phi) &= y \Phi(\mathbf{x}) \, \bar{\phi}'(\mathbf{x}; \boldsymbol{\theta}_\phi) + w,
\end{align}
\end{subequations}
where $\Phi(\mathbf{x})$ is a function that smoothly vanishes on the wavy boundary $\partial \Omega_\mathrm{wavy}$ and is positive elsewhere. Furthermore, $f_{ij}(\mathbf{x}) = \Psi(\mathbf{x}) m_i(\mathbf{x}) m_j(\mathbf{x})$ for $i,j = 1,2$, where $\Psi(\mathbf{x})$ is a function that equates 1 on $\partial \Omega_\mathrm{wavy}$ and smoothly vanishes elsewhere, and $m_i(\mathbf{x})$ are smooth functions equal on $\partial \Omega_\mathrm{wavy}$ to the components of the tangent unit vector. That way, the neural network approximations defined in \eqref{eq:LinearNN} obey by construction the boundary conditions \eqref{eq:HalfStarMatrixBCs}. Concretely, we use the expressions
\begin{subequations}
\begin{align}
\Phi(\mathbf{x}) &= 1 - \left( \frac{r}{r^* + a^* \sin 7 \theta} \right)^2, \\
\Psi(\mathbf{x}) &= \frac{1 + \cos ( \pi \min(|\Phi(\mathbf{x})/\Phi_c|, 1) )}{2}, \\
m_1(\mathbf{x}) &= 7 a^* \cos 7 \theta \cos \theta - (r^* + a^* \sin 7 \theta) \sin \theta, \\
m_2(\mathbf{x}) &= 7 a^* \cos 7 \theta \sin \theta + (r^* + a^* \sin 7 \theta) \cos \theta.
\end{align}
\end{subequations}
Note that the transformations defined in \eqref{eq:HalfStarStress11}-\eqref{eq:HalfStarStress12} require the introduction of an additional neural network $\bar{\sigma}_{mm}'$ with parameters $\boldsymbol{\theta}_6$ that represents the tangential stress along $\partial \Omega_\mathrm{wavy}$. Finally, since we know that the elastic material is present all along the outer perimeter $\partial \Omega$, we define $\bar{\phi}$ so that $\rho = \mathrm{sigmoid}(\bar{\phi}/\delta) \simeq 1$ on $\partial \Omega$.

For the periodic elastic layer, we introduce the transformations
\begin{subequations}
\begin{align}
\bar{u}_1(\mathbf{x}; \boldsymbol{\theta}_1) &= (y+0.5) \, \bar{u}_1'(\cos x, \sin x, y; \boldsymbol{\theta}_1), \\
\bar{u}_2(\mathbf{x}; \boldsymbol{\theta}_2) &= (y+0.5) \, \bar{u}_2'(\cos x, \sin x, y; \boldsymbol{\theta}_2), \\
\bar{\sigma}_{11}(\mathbf{x}; \boldsymbol{\theta}_3) &= \bar{\sigma}_{11}'(\cos x, \sin x, y; \boldsymbol{\theta}_3), \\
\bar{\sigma}_{22}(\mathbf{x}; \boldsymbol{\theta}_4) &= y \, \bar{\sigma}_{22}'(\cos x, \sin x, y; \boldsymbol{\theta}_4) - P_o, \\
\bar{\sigma}_{12}(\mathbf{x}; \boldsymbol{\theta}_5) &= y \, \bar{\sigma}_{12}'(\cos x, \sin x, y; \boldsymbol{\theta}_5), \\
\bar{\phi}(\mathbf{x}; \boldsymbol{\theta}_\phi) &= y(y+0.5) \, \bar{\phi}'(\cos x, \sin x, y; \boldsymbol{\theta}_\phi) + w(4y+1),
\end{align}
\end{subequations}
so that the neural network approximations defined in \eqref{eq:LinearNN} obey by construction the boundary conditions \eqref{eq:LayerBCs} and are periodic along the $x$ direction. Further, since we know that the elastic material is present all along the top surface $y = 0$ and the rigid substrate is present all along the bottom surface $y = -0.5$, we define $\bar{\phi}$ so that $\rho = \mathrm{sigmoid}(\bar{\phi}/\delta) \simeq 1$ for $y = 0$ and $\rho \simeq 0$ for $y=-0.5$.

\subsubsection{Large-deformation hyperelasticity}

The problem is now described by the physical quantities $\boldsymbol{\psi} = (u_1,u_2,S_{11},S_{22},S_{12},S_{21},p)$. We therefore introduce the neural network approximations
\begin{subequations}
\begin{align}
u_1(\mathbf{x}) &= \bar{u}_1(\mathbf{x}; \boldsymbol{\theta}_1), \\
u_2(\mathbf{x}) &= \bar{u}_2(\mathbf{x}; \boldsymbol{\theta}_2), \\
S_{11}(\mathbf{x}) &= \bar{S}_{11}(\mathbf{x}; \boldsymbol{\theta}_3), \\
S_{22}(\mathbf{x}) &= \bar{S}_{22}(\mathbf{x}; \boldsymbol{\theta}_4), \\
S_{12}(\mathbf{x}) &= \bar{S}_{12}(\mathbf{x}; \boldsymbol{\theta}_5), \\
S_{21}(\mathbf{x}) &= \bar{S}_{21}(\mathbf{x}; \boldsymbol{\theta}_6), \\
p(\mathbf{x}) &= \bar{p}(\mathbf{x}; \boldsymbol{\theta}_7), \\
\phi(\mathbf{x}) &= \bar{\phi}(\mathbf{x}; \boldsymbol{\theta}_\phi),
\end{align} \label{eq:NonlinearNN}%
\end{subequations}
and the material distribution is given by $\rho(\mathbf{x}) = \bar{\rho}(\mathbf{x};\boldsymbol{\theta}_\phi) = \mathrm{sigmoid}(\bar{\phi}(\mathbf{x};\boldsymbol{\theta}_\phi)/\delta)$. We then formulate the loss function $\mathcal{L}$ by specializing the loss term expressions described in the main text to the hyperelasticity problem, using the governing equations given in Appendix \ref{app:LargeDeformationNonlinearHyperelasticity}. Omitting the $\boldsymbol{\theta}$'s for notational simplicity, we obtain
\begin{subequations}
\begin{align}
\mathcal{L}_\mathrm{meas}(\boldsymbol{\theta}_{\boldsymbol{\psi}}) &= \frac{1}{|\partial \Omega^m|} \sum_{\mathbf{x}_i \in \partial \Omega^m}  |\bar{\mathbf{u}}(\mathbf{x}_i) - \mathbf{u}_i^m|^2, \label{eq:NonlinearLmeas} \\
\mathcal{L}_\mathrm{gov}(\boldsymbol{\theta}_{\boldsymbol{\psi}}, \boldsymbol{\theta}_\phi) &= \frac{1}{|\Omega^d|} \sum_{\mathbf{x}_i \in \Omega^d} |\mathbf{r}_\mathrm{eq}(\bar{\mathbf{S}}(\mathbf{x}_i))|^2 + \frac{1}{|\Omega^d|} \sum_{\mathbf{x}_i \in \Omega^d} |\mathbf{r}_\mathrm{cr}(\bar{\mathbf{u}}(\mathbf{x}_i),\bar{\mathbf{S}}(\mathbf{x}_i),\bar{p}(\mathbf{x}_i),\bar{\rho}(\mathbf{x}_i))|^2 \nonumber \\
&\quad+ \frac{1}{|\Omega^d|} \sum_{\mathbf{x}_i \in \Omega^d} |\mathbf{r}_\mathrm{inc}(\bar{\mathbf{u}}(\mathbf{x}_i),\bar{\rho}(\mathbf{x}_i))|^2, \label{eq:NonlinearLF}
\end{align}
\end{subequations}
where $\bar{\mathbf{u}} = (\bar{u}_1,\bar{u}_2)$ and $\bar{\mathbf{S}}$ has components $\bar{S}_{i,j}$, $i,j=1,2$. In \eqref{eq:NonlinearLF}, the terms $\mathbf{r}_\mathrm{eq}$, $\mathbf{r}_\mathrm{cr}$, and $\mathbf{r}_\mathrm{inc}$ refer to the residuals of the equilibrium equation \eqref{eq:NonlinearEquilibrium}, the constitutive relation \eqref{eq:NonlinearStressStrain}, and the incompressibility constraint \eqref{eq:NonlinearIncompressibility}. The eikonal loss term is problem-independent and therefore identical to the expression given in (6) in the main text.

As in the linear elasticity case, we design the architecture of the neural networks in such a way that they inherently satisfy the boundary conditions. For the elastic matrix problem,
\begin{subequations}
\begin{align}
\bar{u}_1(\mathbf{x}; \boldsymbol{\theta}_1) &= \bar{u}_1'(\mathbf{x}; \boldsymbol{\theta}_1), \\
\bar{u}_2(\mathbf{x}; \boldsymbol{\theta}_2) &= \bar{u}_2'(\mathbf{x}; \boldsymbol{\theta}_2), \\
\bar{S}_{11}(\mathbf{x}; \boldsymbol{\theta}_3) &= (x-0.5)(x+0.5) \, \bar{S}_{11}'(\mathbf{x}; \boldsymbol{\theta}_3) + P_o, \\
\bar{S}_{22}(\mathbf{x}; \boldsymbol{\theta}_4) &= (y-0.5)(y+0.5) \, \bar{S}_{22}'(\mathbf{x}; \boldsymbol{\theta}_4), \\
\bar{S}_{12}(\mathbf{x}; \boldsymbol{\theta}_5) &= (y-0.5)(y+0.5) \, \bar{S}_{12}'(\mathbf{x}; \boldsymbol{\theta}_5), \\
\bar{S}_{21}(\mathbf{x}; \boldsymbol{\theta}_6) &= (x-0.5)(x+0.5) \, \bar{S}_{21}'(\mathbf{x}; \boldsymbol{\theta}_6), \\
\bar{p}(\mathbf{x}; \boldsymbol{\theta}_7) &= \bar{p}'(\mathbf{x}; \boldsymbol{\theta}_7), \\
\bar{\phi}(\mathbf{x}; \boldsymbol{\theta}_\phi) &= (x-0.5)(x+0.5)(y-0.5)(y+0.5) \, \bar{\phi}'(\mathbf{x}; \boldsymbol{\theta}_\phi) + w,
\end{align}
\end{subequations}
where the quantities with a prime denote the raw output of the neural network. In this way, the neural network approximations defined in \eqref{eq:NonlinearNN} obey by construction the boundary conditions \eqref{eq:MatrixBCsHyperEla}. As before, since we know that the elastic material is present all along the outer surface $\partial \Omega$, we define $\bar{\phi}$ so that $\phi = w$ on $\partial \Omega$, which ensures that $\rho = \mathrm{sigmoid}(\phi/\delta) \simeq 1$ on $\partial \Omega$.

\subsection{Thermal conduction experiments}

Since the physical quantities are $\boldsymbol{\psi} = (T, q_1, q_2)$, we introduce the neural network approximations
\begin{subequations}
\begin{align}
T(\mathbf{x}) &= T(\mathbf{x}; \boldsymbol{\theta}_1), \\
q_1(\mathbf{x}) &= \bar{q}_1(\mathbf{x}; \boldsymbol{\theta}_2), \\
q_2(\mathbf{x}) &= \bar{q}_2(\mathbf{x}; \boldsymbol{\theta}_3), \\
\phi(\mathbf{x}) &= \bar{\phi}(\mathbf{x}; \boldsymbol{\theta}_\phi).
\end{align}%
\end{subequations}
The last equation represents the level-set neural network, which defines the material density as $\rho(\mathbf{x}) = \bar{\rho}(\mathbf{x};\boldsymbol{\theta}_\phi) = \mathrm{sigmoid}(\bar{\phi}(\mathbf{x};\boldsymbol{\theta}_\phi)/\delta)$. We then formulate the loss function $\mathcal{L}$ by specializing the loss term expressions described in the main text to the heat transfer problem, using the governing equations given in Appendix \ref{app:ThermalExperiments}. Omitting the $\boldsymbol{\theta}$'s for notational simplicity, we obtain
\begin{subequations}
\begin{align}
\mathcal{L}_\mathrm{meas}(\boldsymbol{\theta}_{\boldsymbol{\psi}}) &= \frac{1}{|\partial \Omega_T^m|} \sum_{\mathbf{x}_i \in \partial \Omega_T^m}  |\bar{T}(\mathbf{x}_i) - T_i^m|^2 + \frac{1}{|\partial \Omega_q^m|} \sum_{\mathbf{x}_i \in \partial \Omega_q^m}  |\bar{\mathbf{q}}(\mathbf{x}_i) \cdot \mathbf{n} - q_i^m|^2, \label{eq:ThermalLmeas} \\
\mathcal{L}_\mathrm{gov}(\boldsymbol{\theta}_{\boldsymbol{\psi}}, \boldsymbol{\theta}_\phi) &= \frac{1}{|\Omega^d|} \sum_{\mathbf{x}_i \in \Omega^d} |\mathbf{r}_\mathrm{hc}(\bar{T}(\mathbf{x}_i))|^2 + \frac{1}{|\Omega^d|} \sum_{\mathbf{x}_i \in \Omega^d} |\mathbf{r}_\mathrm{fl}(\bar{T}(\mathbf{x}_i),\bar{\mathbf{q}}(\mathbf{x}_i),\bar{\rho}(\mathbf{x}_i))|^2, \label{eq:ThermalLF},
\end{align}
\end{subequations}
where $\bar{\mathbf{q}} = (\bar{q}_1,\bar{q}_2)$. In \eqref{eq:ThermalLmeas}, $\partial \Omega_T^m$ and $\partial \Omega_q^m$ denote portions of the boundary with temperature and normal heat flux measurements, respectively. In \eqref{eq:ThermalLF}, the terms $\mathbf{r}_\mathrm{hc}$ and $\mathbf{r}_\mathrm{fl}$ refer to the residuals of the heat conservation equation \eqref{eq:HeatConservation} and Fourier's law \eqref{eq:FourierLaw} or \eqref{eq:InvertedFourierLaw}. The eikonal loss term is problem-independent and therefore identical to the expression given in (6) in the main text.

As with the other examples, we design the architecture of the neural networks in such a way that they inherently satisfy the boundary conditions. For case 26 in Tab.~III, we do this through the transformations
\begin{subequations}
\begin{align}
\bar{T}(\mathbf{x}; \boldsymbol{\theta}_1) &= (x+0.5) \, \bar{T}'(\mathbf{x}; \boldsymbol{\theta}_1) + T_0, \\
\bar{q}_1(\mathbf{x}; \boldsymbol{\theta}_2) &= (x-0.5) \, \bar{q}_1'(\mathbf{x}; \boldsymbol{\theta}_2) + q_n(\mathbf{x}), \\
\bar{q}_2(\mathbf{x}; \boldsymbol{\theta}_3) &= (y-0.5)(y+0.5) \, \bar{q}_2'(\mathbf{x}; \boldsymbol{\theta}_3) + (y-0.5) \, q_n(\mathbf{x}) + (y+0.5) \, q_n(\mathbf{x}), \\
\bar{\phi}(\mathbf{x}; \boldsymbol{\theta}_\phi) &= (x-0.5)(x+0.5)(y-0.5)(y+0.5) \, \bar{\phi}'(\mathbf{x}; \boldsymbol{\theta}_\phi) + w,
\end{align}
\end{subequations}
where $q_n(\mathbf{x}) = \epsilon \sigma (\mathtt{stop\_gradient}\{\bar{T}(\mathbf{x}; \boldsymbol{\theta}_1)\}^4 - T_a^4)$, and the quantities with a prime denote the raw output of the neural network. In this way, the neural network approximations defined in \eqref{eq:LinearNN} obey by construction the boundary conditions \eqref{eq:MatrixBCsThermal1}. Further, since we know that the elastic material is present all along the outer surface $\partial \Omega$, we define $\bar{\phi}$ so that $\rho = \mathrm{sigmoid}(\bar{\phi}/\delta) \simeq 1$ on $\partial \Omega$ (recall that $w = 10 \delta$).

For cases 27 and 28 in Tab.~III, we utilize the transformations
\begin{subequations}
\begin{align}
\bar{T}(\mathbf{x}; \boldsymbol{\theta}_1) &= (x-0.5)(x+0.5) \, \bar{T}'(\mathbf{x}; \boldsymbol{\theta}_1) - (x-0.5) \, T_0, \\
\bar{q}_1(\mathbf{x}; \boldsymbol{\theta}_2) &= \bar{q}_1'(\mathbf{x}; \boldsymbol{\theta}_2), \\
\bar{q}_2(\mathbf{x}; \boldsymbol{\theta}_3) &= (y-0.5)(y+0.5) \, \bar{q}_2'(\mathbf{x}; \boldsymbol{\theta}_3), \\
\bar{\phi}(\mathbf{x}; \boldsymbol{\theta}_\phi) &= (x-0.5)(x+0.5)(y-0.5)(y+0.5) \, \bar{\phi}'(\mathbf{x}; \boldsymbol{\theta}_\phi) + w,
\end{align}
\end{subequations}
where the quantities with a prime denote the raw output of the neural network. In this way, the neural network approximations defined in \eqref{eq:LinearNN} obey by construction the boundary conditions \eqref{eq:MatrixBCsThermal2}. Note, however, that these transformations are modified to reflect the fact that the applied boundary condition is presumed unknown on one of the four sides. Further, since we know that the elastic material is present all along the outer surface $\partial \Omega$, we define $\bar{\phi}$ so that $\rho = \mathrm{sigmoid}(\bar{\phi}/\delta) \simeq 1$ on $\partial \Omega$ (recall that $w = 10 \delta$).

\subsection{Rescaling} 
\label{app:Rescaling}

The physical quantities involved in these mechanical and heat transfer problems span a wide range of scales; for instance, displacements may be orders of magnitude smaller than the length scale associated with the geometry. In order to obtain balanced gradients between the different components of the loss function during training, we rescale all physical quantities into nondimensional values of order one and implement the corresponding nondimensional equations, as also done, for example, in \cite{henkes2022}. In mechanical experiments (all cases in Tabs.~I and II), lengths are rescaled with the characteristic length $L$ of the geometry, tractions and stresses with the magnitude $P_o$ of the applied traction at the boundaries, and displacements with the ratio $L P_o / E$, where $E$ is the Young's modulus of the elastic material (in the hyperelastic case, we use the equivalent Young's modulus $E = 3 \mu$, where $\mu$ is the shear modulus of the hyperelastic material). In the thermal conduction experiment corresponding to case 26 in Tab.~III, we express the temperature through the deviation $\Delta T = T - T_a$, which we rescale with $T_0 - T_a$, and we rescale the heat flux with $\varepsilon \sigma (T_0^4 - T_a^4)$. In the thermal conduction experiment corresponding to cases 27 and 28 in Tab.~III, the temperature is rescaled with $T_0$ and the heat flux with $k T_0/L$.

\section{Architecture and training details}
\label{app:ImplementationDetails}

This section provides details on the architecture of the deep neural networks, the training procedure, and the parameter values considered in this study.

\subsection{Neural network architecture}
\label{app:NeuralNetworkFormulation}

State variable fields of the form $\psi(\mathbf{x})$ are approximated using deep fully-connected neural networks that map the location $\mathbf{x}$ to the corresponding value of $\psi$ at that location. This map can be expressed as $\psi(\mathbf{x}) = \bar{\psi}(\mathbf{x};\boldsymbol{\theta})$, and is defined by the sequence of operations
\begin{subequations}
\begin{align}
\mathbf{z}^0 &= \mathbf{x}, \label{eq:NNInput} \\
\mathbf{z}^k &= \sigma(\mathbf{W}^k \mathbf{z}^{k-1} + \mathbf{b}^k), \quad 1 \le k \le \ell-1, \label{eq:NNMiddleLayers} \\
\psi = \mathbf{z}^\ell &= \mathbf{W}^\ell \mathbf{z}^{\ell-1} + \mathbf{b}^\ell.
\end{align}
\end{subequations}
The input $\mathbf{x}$ is propagated through $\ell$ layers, all of which (except the last) take the form of a linear operation composed with a nonlinear transformation. Each layer outputs a vector $\mathbf{z}^k \in \mathbb{R}^{q_k}$, where $q_k$ is the number of `neurons', and is defined by a weight matrix $\mathbf{W}^k \in \mathbb{R}^{q_k \times q_{k-1}}$, a bias vector $\mathbf{b}^k \in \mathbb{R}^{q_k}$, and a nonlinear activation function $\sigma(\cdot)$. Finally, the output of the last layer is assigned to $\psi$. The weight matrices and bias vectors, which parametrize the map from $\mathbf{x}$ to $\psi$, form a set of trainable parameters $\boldsymbol{\theta} = \{\mathbf{W}^k,\mathbf{b}^k\}_{k=1}^\ell$.

The choice of the nonlinear activation function $\sigma(\cdot)$ and the initialization procedure for the trainable parameters $\boldsymbol{\theta}$ are both important factors in determining the performance of neural networks. While the tanh function has been a popular candidate in the context of PINNs \citep{lu2021a}, recent works have shown that using sinusoidal activation functions can lead to improved training performance by promoting the emergence of small-scale features \cite{sitzmann2020,wong2022}. In this work, we select the sinusoidal representation network (SIREN) architecture from Ref.~\cite{sitzmann2020}, which combines the use of the sine as an activation function with a specific way to initialize the trainable parameters $\boldsymbol{\theta}$ that ensures that the distribution of the input to each sine activation function remains unchanged over successive layers. Specifically, each component of  $\mathbf{W}^k$ is uniformly distributed between $- \sqrt{6/q_k}$ and $\sqrt{6/q_k}$ where $q_k$ is the number of neurons in layer $k$, and $\mathbf{b}^k = \mathbf{0}$, for $k=1, \dots, \ell$. Further, the first layer of the SIREN architecture is $\mathbf{z}^1 = \sigma(\omega_0 \mathbf{W}^1 \mathbf{z}^0 + \mathbf{b}^1)$ instead of \eqref{eq:NNMiddleLayers}, with the extra scalar $\omega_0$ promoting higher-frequency content in the output.

\textcolor{black}{We use the standard logistic function for the sigmoid function representing the material density field; that is, $\rho = \mathrm{sigmoid}(\phi/\delta) = 1/(1+\exp (-\phi/\delta))$, where $\phi = \bar{\phi}(\mathbf{x}; \boldsymbol{\theta}_\phi)$ is the output of the level-set neural network. Note that the presence of the sigmoid function means that $\rho$ can only get asymptotically close to the theoretical solution of the inverse problem, which is not a problem in practice as exemplified by the widespread adoption of sigmoid functions to learn bounded functions in machine learning \cite{goodfellow2016}.}

\subsection{Training procedure}

We calculate the loss and train the neural networks in TensorFlow 2. The training is performed using ADAM, a first-order gradient-descent-based algorithm with adaptive step size \cite{kingma2014}. In each case, we repeat the training over four random initializations of the neural networks parameters and report the best results. Three tricks resulted in noticeably improved training performance and consistency.

First, we found that pretraining the level-set neural network $\phi(\mathbf{x}) = \bar{\phi}(\mathbf{x}; \boldsymbol{\theta}_\phi)$ in a standard supervised setting leads to much more consistent results over different initializations of the neural networks. During this pretraining phase, carried out before the main optimization, we minimize the mean-square error 
\begin{equation}
\mathcal{L}_\mathrm{sup}(\boldsymbol{\theta}_\phi) = \frac{1}{|\Omega^d|} \sum_{\mathbf{x}_i \in \Omega^d} |\bar{\phi}(\mathbf{x}_i; \boldsymbol{\theta}_\phi) - \phi_i |,
\end{equation}
where $\Omega^d$ is the same set of collocation points as in Eq.~(5), the supervised labels $\phi_i = |\mathbf{x}_i| - 0.25$ for the elastic matrix, and $\phi_i = y_i + 0.25$ for the elastic layer. The material density $\bar{\rho}(\mathbf{x};\boldsymbol{\theta}_\phi) = \mathrm{sigmoid}(\bar{\phi}(\mathbf{x};\boldsymbol{\theta}_\phi)/\delta)$ obtained at the end of this pretraining phase is one outside a circle of radius 0.25 centered at the origin for the elastic matrix, and it is one above the horizontal line $y = -0.25$ for the elastic layer. This choice for the supervised labels is justified by the fact that $\rho$ is known to be one along the outer boundary of the domain $\Omega$ for the elastic matrix, and it is known to be one (zero) along the top (bottom) boundary of $\Omega$ for the elastic layer. 

Second, during the main optimization in which all neural networks are trained to minimize the total loss described in Eq.~(3) of the main text, we evaluate the loss component $\mathcal{L}_\mathrm{gov}$ in Eq.~(5) using a different subset, or mini-batch, of residual points from $\Omega^d$ at every iteration. Such a mini-batching approach has been reported to improve the convergence of the PINN training process \citep{wight2021,daw2022}, corroborating our own observations. Thus, each gradient update
\begin{subequations}
\begin{align}
\boldsymbol{\theta}_{\boldsymbol{\psi}}^{k+1} &= \boldsymbol{\theta}_{\boldsymbol{\psi}}^k - \alpha_{\boldsymbol{\psi}}(k) \nabla_{\boldsymbol{\theta}_{\boldsymbol{\psi}}} \mathcal{L}(\boldsymbol{\theta}_{\boldsymbol{\psi}}^k, \boldsymbol{\theta}_\phi^k), \\
\boldsymbol{\theta}_\phi^{k+1} &= \boldsymbol{\theta}_\phi^k - \alpha_\phi(k) \nabla_{\boldsymbol{\theta}_\phi} \mathcal{L}(\boldsymbol{\theta}_{\boldsymbol{\psi}}^k, \boldsymbol{\theta}_\phi^k)
\end{align} \label{eq:GradientUpdate}%
\end{subequations}
is evaluated using a subsequent mini-batch of residual points. The order of the mini-batches is shuffled after every epoch, i.e.~after passing through all the mini-batches constituting $\Omega^d$.  

Third, the initial nominal step size $\alpha_{\boldsymbol{\psi}}$ governing the learning rate of the physical quantities neural networks is set to be 10 times larger than its counterpart $\alpha_\phi$ governing the learning rate of the level-set neural network. This results in a separation of time scales between the rate of change of the physical quantities neural networks and that of the level-set neural network, which is motivated by the idea that physical quantities should be given time to adapt to a given geometry before the geometry itself changes.

\subsection{Parameter values}

In this study, the hyperparameters were manually optimized, starting from a base configuration and changing individual values in order for the training to avoid being stuck in bad local minima. We performed this process once for cases with the same physics and outer matrix geometry, leading to the same hyperparameter values among these cases. Note that there also exist systematic hyperparameter optimization methods, although we have not explored them \citep{akiba2019,yang2020}.

For all cases in Tabs.~I to III except cases 20, 21, 22, we opted for neural networks with 4 hidden layers of 50 neurons each, which we found to be a good compromise between expressivity and training time. For cases 20 and 22, we used 6 hidden layers with 100 neurons each. For case 21, we used 6 hidden layers with 150 neurons each. Further, we choose $\omega_0 = 10$ as the scalar appearing in the first layer of the SIREN architecture.

For all cases in Tabs.~I to III except cases 21 and 22, we considered measurements distributed over 100 equally-spaced locations along each straight external boundary segment (i.e., amounting to $|\partial \Omega^m| = 400$ in cases where measurements were taken along all sides of a square matrix, and $|\partial \Omega^m| = 100$ in the elastic layer cases). For case 21, we considered measurements distributed over 357 locations along the wavy boundary of the half-star. For case 22, we considered measurements distributed over 400 locations in a grid on each face of the 3D cube.

For all cases in Tabs.~I to III except cases 20, 21, 22, the set of collocation points $\Omega^d$ consists of 10000 points. For cases 20 and 22, $\Omega^d$ consists of 50000 points. For case 21, $\Omega^d$ consists of approximately 200000 points. In all cases, the collocation points are sampled from $\Omega$ using a Latin Hypercube Sampling (LHS) strategy.

The pretraining of the level-set neural network is carried out using the Adam optimizer with nominal step size $10^{-3}$ over 800 iterations, employing the whole set $\Omega^d$ to compute the gradient of $\mathcal{L}_\mathrm{sup}$ at each update step. The main optimization, during which all neural networks are trained to minimize the total loss described in Eq.~(3), is also carried out using the Adam optimizer with a mini-batch size of 1000 for all examples. For cases 1 to 19 in Tab.~I and cases 27, 28 in Tab.~III, we use a total of 1500k training iterations starting from a nominal step size of $10^{-3}$, reduced to $10^{-4}$ and $10^{-5}$ at 600k and 1200k iterations, respectively. The schedule is the same for cases 23 to 25 in Tab.~II, with the difference that we use a total of 2000k training iterations. For cases 20 and 22 in Tab.~I, we use a total of 2500k iterations starting from a nominal step size of $10^{-3}$, which is reduced to $10^{-4}$ and $10^{-5}$ at 800k and 2000k iterations, respectively. For case 21 in Tab.~I, we use a total of 2000k iterations starting from a nominal step size of $10^{-3}$, which is reduced to $10^{-4}$ and $10^{-5}$ at 600k and 1200k iterations, respectively. For case 26 in Tab.~III, we use a total of 3000k iterations starting from a nominal step size of $10^{-3}$, which is reduced to $10^{-4}$ and $10^{-5}$ at 600k and 2600k iterations, respectively. As mentioned in the previous section, the initial nominal step size for the level-set neural network is set to $10^{-4}$ in all examples, and decreases to $10^{-5}$ at the same time as the other neural networks. 

The scalar weights in the loss (3) in the main text are assigned the values $\lambda_\mathrm{meas} = 10$, $\lambda_\mathrm{gov} = 1$, and $\lambda_\mathrm{reg} = 1$ for all cases. We also multiply the second term of $\mathcal{L}_\mathrm{gov}$ in \eqref{eq:NonlinearLF} and \eqref{eq:LinearLF} of the main text with a scalar weight $\lambda_\mathrm{cr} = 10$.

Finally, the computations are carried out on a cluster, using 24 threads of an Intel Xeon
Platinum 8260 CPU for each case. For each case, the pretraining phase takes a couple minutes, while the main optimization takes between 2 and 6 hours.

{\color{black}
\section{Effect of measurement noise}
\label{app:EffectMeasurementNoise}

In this appendix, we analyze the robustness of solutions provided by our TO framework in the presence of measurement noise, focusing on mechanical experiments on a linear elastic matrix containing voids (cases 4, 9, 18 in Tab.~I). To this effect, we simulate noisy measurements by adding white Gaussian noise of standard deviation $\sigma_\mathrm{noise}$ to the displacement data obtained from Abaqus simulations. 

Our TO framework identifies the correct structures for noise levels $\sigma_\mathrm{noise}$ up to 10\% of the characteristic displacement magnitude $P_0 L/E$, where $P_0$ is the applied normal traction, $L$ the size of the square matrix, and $E$ the Young's modulus (Supplementary Fig.~\ref{fig:NCS_Noise}a). 

For a more quantitative analysis of the effect of measurement noise on the accuracy of the identified voids, consider an accuracy metric such as the IoU (intersection over union, a non-negative scalar equal to 1 in the perfect case). In general, we can expect the IoU to depend on $\sigma_\mathrm{noise}$, the geometry of the matrix and voids, and the material properties of the matrix. Since the critical quantity affecting the accuracy of the results is the signal-to-noise ratio, we may write the IoU as
\begin{equation}
\mathrm{IoU} = f \left( \frac{\sigma_\mathrm{noise}}{u_\mathrm{meas}^*}, \mathcal{G} \right),
\label{eq:NoiseScaling}
\end{equation}
where $f$ is some unknown function, $u_\mathrm{meas}^*$ is an estimate for the signal amplitude, that is, the average magnitude of boundary displacement perturbation caused by the voids, and $\mathcal{G}$ is a scalar quantity that depends on the normalized geometric shape of the voids, such as their aspect ratio or boundary curvature. The intuition behind including $\mathcal{G}$ is that more complex void shapes are harder to detect accurately than simpler shapes of similar size, even though both induce boundary displacement perturbations of comparable magnitude. To derive an estimate for $u_\mathrm{meas}^*$, consider a circle-shaped (in 2D) or spherical-shaped (in 3D) stress-free void centered inside a disk (in 2D) or a sphere (in 3D) subject to a normal traction $P_0$ on its outer boundary. In this case, it is well-known that the boundary displacement perturbation induced by the presence of the void scales as $(P_0 L/E) (a/L)^d$, where $a$ is the void diameter and $d$ is the spatial dimension (2 in 2D, 3 in 3D) \cite{bower2009}. Thus, we choose $u_\mathrm{meas}^* = (P_0 L/E) (a/L)^d$, with $a$ equal to the average equivalent size of all voids in the matrix, the equivalent size of a given void being defined as the size of a circle-shaped (in 2D) or spherical-shaped (in 3D) void with identical area or volume. 

For all three void configurations tested above, the IoU plotted as a function of $\sigma_\mathrm{noise}/u_\mathrm{meas}^*$ remains close to 1 for low $\sigma_\mathrm{noise}/u_\mathrm{meas}^*$ and starts to decrease noticeably around $\sigma_\mathrm{noise}/u_\mathrm{meas}^* = 1$ (Supplementary Fig.~\ref{fig:NCS_Noise}b). The faster decrease of the IoU for the slit-shaped void may be attributable to its high aspect ratio, which is accounted for by the geometric parameter $\mathcal{G}$ in \eqref{eq:NoiseScaling}. 

In practice, the measurement noise level $\sigma_\mathrm{noise}$ can often be known ahead of time (from the sensor specifications), but $u_\mathrm{meas}^*$ and $\mathcal{G}$ are unknown since the voids are hidden. Nevertheless, we can derive two useful guidelines from the scaling relationship \eqref{eq:NoiseScaling}. First, increasing $P_0$ will always reduce the ratio $\sigma_\mathrm{noise}/u_\mathrm{meas}^*$, leading to better results for any noise level and any void configuration. Second, since the accuracy of the identified voids decreases sharply for $\sigma_\mathrm{noise}/u_\mathrm{meas}^*$ on the order of 1 or greater (Supplementary Fig.~\ref{fig:NCS_Noise}b), we can expect that, for a given $\sigma_\mathrm{noise}$, the minimum detectable void size will be on the order of $(\sigma_\mathrm{noise} E / P_0 L)^{1/d} L$.  This minimum detectable void formula is a worst-case estimate in the limit of otherwise refined numerics; it assumes that voids are situated far from the outer boundaries while numerical parameters such as the interfacial thickness $\delta$ and collocation point separation are sufficiently small compared to the void size, and that the optimizer is perfect.}

%%%%
\begin{figure*}
\centering
\includegraphics[width=\textwidth]{Figures/NCS_ElaVoid_Iters.pdf}
\caption{\textbf{Identification of voids in a linear elastic matrix.} Evolution of the material density during the training process for the cases reported in Fig.~4a-d.}
\label{fig:NCS_ElaVoid_Iters}
\end{figure*}
%%%%

%%%%
\begin{figure*}
\centering
\includegraphics[width=\textwidth]{Figures/NCS_ElaVoid_Stress.pdf}
\caption{\textbf{Identification of voids in a linear elastic matrix.} Final Cauchy stress components $\sigma_{xx}$ (\textbf{a}), $\sigma_{yy}$ (\textbf{b}), and $\sigma_{xy}$ (\textbf{c}), displayed in the deformed configuration obtained from the final displacement components $u_1$ and $u_2$, for the cases reported in Fig.~4a-d. The grey dotted lines show the outline of the matrix surface in the reference configuration.}
\label{fig:NCS_ElaVoid_Stress}
\end{figure*}
%%%%

%%%%
\begin{figure*}
\centering
\includegraphics[width=\textwidth]{Figures/NCS_ElaInclusion_Stress.pdf}
\caption{\textbf{Identification of inclusions in a linear elastic matrix.} Final Cauchy stress components $\sigma_{xx}$ (\textbf{a}), $\sigma_{yy}$ (\textbf{b}), and $\sigma_{xy}$ (\textbf{c}), displayed in the deformed configuration obtained from the final displacement components $u_1$ and $u_2$, for the cases reported in Fig.~4e. The grey dotted lines show the outline of the matrix surface in the reference configuration.}
\label{fig:NCS_ElaInclusion_Stress}
\end{figure*}
%%%%

%%%%
\begin{figure*}
\centering
\includegraphics[width=\textwidth]{Figures/NCS_HyperElaVoid_Stress.pdf}
\caption{\textbf{Identification of voids in a nonlinear hyperelastic matrix.} Final Cauchy stress components $\sigma_{xx}$ (\textbf{a}), $\sigma_{yy}$ (\textbf{b}), and $\sigma_{xy}$ (\textbf{c}), displayed in the deformed configuration obtained from the final displacement components $u_1$ and $u_2$, for the cases reported in Fig.~4f. The grey dotted lines show the outline of the matrix surface in the reference configuration.}
\label{fig:NCS_HyperElaVoid_Stress}
\end{figure*}
%%%%

%%%%
\begin{figure*}
\centering
\includegraphics[width=\textwidth]{Figures/NCS_Noise.pdf}
\caption{\textbf{Effect of noise on the identification of voids in a linear elastic matrix.} \textbf{a}, Final material density $\rho$ obtained for various values of measurement noise standard deviation $\sigma_\mathrm{noise}$. \textbf{b}, IoU (intersection over union, a geometry detection accuracy metric equal to 1 in the perfect case) versus scaled standard deviation $\sigma_\mathrm{noise}/u_\mathrm{meas}^*$ of the measurement noise, where $u_\mathrm{meas}^*$ is an analytical estimate for the magnitude of boundary displacement perturbation caused by the voids based on their average size as well as the size and material properties of the elastic matrix. Each plot corresponds to the voids topology shown in \textbf{a} in the same column. The circles report the average value and the shade report the highest and lowest values obtained for 4 random initializations of the neural networks parameters.}
\label{fig:NCS_Noise}
\end{figure*}
%%%%

%%%%
\begin{figure*}
\centering
\includegraphics[width=\textwidth]{Figures/NCS_SparseMeasCircle.pdf}
\caption{\textbf{Effect of sparse measurements on the identification of voids in a linear elastic matrix with one circle-shaped void.} Final material density $\rho$ obtained in our framework when using fewer measurement locations and restricting the number of measurements to a subset of the outer surfaces.}
\label{fig:NCS_SparseMeasCircle}
\end{figure*}
%%%%

%%%%
\begin{figure*}
\centering
\includegraphics[width=\textwidth]{Figures/NCS_SparseMeasTwo.pdf}
\caption{\textbf{Effect of sparse measurements on the identification of voids in a linear elastic matrix with one star-shaped and one rectangle-shaped void.} Final material density $\rho$ obtained in our framework when using fewer measurement locations and restricting the number of measurements to a subset of the outer surfaces.}
\label{fig:NCS_SparseMeasTwo}
\end{figure*}
%%%%

%%%%
\begin{figure*}
\centering
\includegraphics[width=\textwidth]{Figures/NCS_SparseMeasSlit.pdf}
\caption{\textbf{Effect of sparse measurements on the identification of voids in a linear elastic matrix with one slit-shaped void.} Final material density $\rho$ obtained in our framework when using fewer measurement locations and restricting the number of measurements to a subset of the outer surfaces.}
\label{fig:NCS_SparseMeasSlit}
\end{figure*}
%%%%

%%%%
\begin{figure*}
\centering
\includegraphics[width=\textwidth]{Figures/NCS_AbaqusStress.pdf}
\caption{\textbf{Stress distribution in a linear elastic matrix for different types of U-shaped inclusions.} The von Mises stress obtained in Abaqus for U-shaped inclusions with different constitutive properties reveals that the concave part of the matrix is subject to very little stress in the case of a void or rigid inclusion. Note also that stiff and rigid inclusions `strengthen' the matrix, as opposed to the void and soft inclusion that `soften' the matrix.}
\label{fig:NCS_AbaqusStress}
\end{figure*}
%%%%

%%%%
\begin{figure}
\centering
\includegraphics[width=\textwidth]{Figures/NCS_HeatTransfer}
\caption{\textbf{Identification of inclusions in a nonlinearly conducting matrix with incomplete information.} \textbf{a}, Setup of the geometry and applied boundary conditions during the loading. \textbf{b}, The geometry identification inverse problem is solved assuming that one of the four sides of the matrix is inaccessible to the user, meaning that both the applied boundary condition and the measurements are unavailable on that side. \textbf{c}, The final inferred material density $\rho$ and temperature $T$ in the case of a perfectly insulating inclusion. \textbf{d}, The final inferred material density $\rho$ and temperature $T$ in the case of a perfectly conducting inclusion.}
\label{fig:HeatTransfer}
\end{figure}
%%%%

\bibliography{bibliography}

\end{document}
