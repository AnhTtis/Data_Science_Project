\documentclass[article,superscriptaddress]{revtex4-2}

\usepackage{graphicx}
\usepackage{amsmath,amsfonts,amssymb,amsthm,color}
\allowdisplaybreaks
%\renewcommand\Affilfont{\itshape\small}
%\usepackage{epstopdf,epsfig}
%\usepackage{newtxtext}
%\usepackage{newtxmath}
%\usepackage[square,numbers]{natbib}

\renewcommand{\figurename}{Supplementary Fig.}

\usepackage{hyperref}
\hypersetup{
    colorlinks = true,
    urlcolor   = blue,
    citecolor  = black,
}
\newtheorem{lemma}{Lemma}
\newtheorem{corollary}{Corollary}
\newcommand{\RomanNumeralCaps}[1]

%\usepackage{float}
%\usepackage{caption}
%\captionsetup{justification=justified,width=\textwidth}
\usepackage[linesnumbered,ruled,vlined]{algorithm2e}

\begin{document}

\preprint{APS/123-QED}

\title{Supplementary information for: \\ Topology optimization with physics-informed neural networks: \\ application to noninvasive detection of hidden geometries}

\author{Saviz Mowlavi}
 \email{mowlavi@merl.com}
\affiliation{Department of Mechanical Engineering, MIT, Cambridge, MA 02139, USA}
\affiliation{Mitsubishi Electric Research Laboratories, Cambridge, MA 02139, USA}
\author{Ken Kamrin}%
 \email{kkamrin@mit.edu}
\affiliation{Department of Mechanical Engineering, MIT, Cambridge, MA 02139, USA}

\begin{abstract}
\end{abstract}

\maketitle

\section*{Uniqueness of solutions}

We sketch a proof for the uniqueness of solutions to the geometry detection elasticity problem considered in this paper. For the specific case of a two-dimensional linear elastic material with a single void, it has been proved that there exists at most one cavity which yields the same surface displacements and stresses on an finite portion of the external boundary \citep{ang1999}. Our approach is different and applicable to any three-dimensional problem governed by elliptic PDEs (be they linear or nonlinear, heat conduction, elasticity, etc) and containing multiple voids or rigid inclusions. Before sketching the proof, we state a few lemmas that will be useful:
\begin{itemize}
\item \textbf{Lemma 1}. Any $C^2$ displacement field that solves the equilibrium Navier-Cauchy equations of linear elasticity with constant elastic moduli is a real analytic function of space. Ths holds because the Navier-Cauchy equations are elliptic, so any $C^2$ solution must be real analytic \citep{morrey1958}.
\item \textbf{Lemma 2}. Any real analytic function on a connected domain $\Omega$ that vanishes on a finite and connected subset $\mathcal{S} \subset \Omega$ necessarily vanishes in all of $\Omega$. This is a specialization of the identity theorem for analytic functions \citep{mityagin2020}.
\item \textbf{Lemma 3}. If $\mathbf{u}$ is a $C^2$ displacement solution to the Navier-Cauchy equations on a connected domain $\Omega$, and both $\mathbf{u}$ and traction $\mathbf{t}$ vanish on a smooth finite portion of the boundary $S\subset\partial \Omega$, 
then $\mathbf{u}$ must necessarily vanish in all of $\Omega$. This fact can be seen by extending the domain $\Omega$ along $S$ by some amount $\Omega^{ext}$ and defining $\mathbf{u}=\mathbf{0}$ there. Then  $\mathbf{u}$ satisfies the Navier-Cauchy equations at each point in $\Omega\cup\Omega^{ext}$, and it can be shown that all second derivatives exist and are continuous on $S$.   Thus, on $\Omega\cup\Omega^{ext}$, $\mathbf{u}$ is an analytic function (by Lemma 1) that vanishes in $\Omega^{ext}$, and therefore $\mathbf{u}$ vanishes in all of $\Omega\cup\Omega^{ext}$ (by Lemma 2) and thus in $\Omega$.
%One can indeed show that all second-order derivatives of $f$ vanish on $\partial \Omega$ \textcolor{blue}{[how?]}, hence one can construct an extension of $f$ that is zero outside of $\Omega$ and $C^2$ everywhere. Since the extended $f$ is real analytic everywhere \textcolor{blue}{[is it? note that $f$ is not necessarily harmonic in $\Omega$]}, one can appeal to Lemma 2 to show that $f$ vanishes everywhere.
\end{itemize}

%%%%
\begin{figure}[tb]
\centering
\includegraphics[width=\textwidth]{Figures/NCS_Proof}
\caption{\textbf{Setup for the proof of the uniqueness of solutions.} \textbf{a},\textbf{b}, Two bodies with different inclusions are subjected to the same loading and are assumed to generate identical surface displacements on a portion $\partial \mathcal{B}^m$ of the outer traction boundary. \textbf{c}, The difference between the temperature and stress fields of the two bodies has vanishing displacement and traction on $\partial \mathcal{B}^m$. \textbf{d}, In the case of a void, the blue region $\partial \tilde{\mathcal{B}}^{(1)}$ of the first body has vanishing stress. \textbf{e}, In the case of a rigid inclusion, the blue region $\partial \tilde{\mathcal{B}}^{(1)}$ of the first body undergoes a rigid displacement.}
\label{fig:Proof}
\end{figure}
%%%%

We sketch the proof for the case of a single void or inclusion, but the same reasoning generalizes to any number of voids or inclusions. Consider two bodies $\mathcal{B}^{(1)}$ and $\mathcal{B}^{(2)}$ with the same material properties and sharing the same external boundary $\partial \mathcal{B}^\mathrm{ext}$ (Supplementary Fig.~\ref{fig:Proof}a,b). Each body contains a single smooth void or rigid inclusion, characterized by the connected domains $\mathcal{I}^{(1)}$ and $\mathcal{I}^{(2)}$ with boundaries $\partial \mathcal{I}^{(1)}$ and $\partial \mathcal{I}^{(2)}$, respectively. The two bodies are subjected to the same external loading, which consists of an applied displacement $\bar{\mathbf{u}}$ on a portion $\partial \mathcal{B}_u^\mathrm{ext}$ of $\partial \mathcal{B}^\mathrm{ext}$ and an applied traction $\bar{\mathbf{t}}$ on a portion $\partial \mathcal{B}_t^\mathrm{ext}$ of $\partial \mathcal{B}^\mathrm{ext}$, with $\partial \mathcal{B}^\mathrm{ext} = \partial \mathcal{B}_u^\mathrm{ext} \cup \partial \mathcal{B}_t^\mathrm{ext}$. This loading generates a displacement and stress solution within each body, which we denote by $\mathbf{u}^{(1)}$, $\boldsymbol{\sigma}^{(1)}$ and $\mathbf{u}^{(2)}$, $\boldsymbol{\sigma}^{(2)}$, respectively. These solutions must be nontrivial, meaning that the surface tractions $\mathbf{t}^{(1)}$ and $\mathbf{t}^{(2)}$ do not vanish everywhere on $\partial \mathcal{B}^\mathrm{ext}$. Finally, we assume that we measure identical surface displacements, i.e.~$\mathbf{u}^{(1)} = \mathbf{u}^{(2)} = \mathbf{u}^m$, on a finite portion $\partial \mathcal{B}^m$ of $\partial \mathcal{B}_t^\mathrm{ext}$.

We will prove that there cannot be two distinct shapes $\mathcal{I}^{(1)}$ and $\mathcal{I}^{(2)}$ yielding nontrivial solutions that are identical on $\partial \mathcal{B}^m$. Let us first subtract the displacement and stress solutions of the two bodies, which yields a displacement field $\Delta \mathbf{u} = \mathbf{u}^{(1)} - \mathbf{u}^{(2)}$ and a stress field $\Delta \boldsymbol{\sigma} = \boldsymbol{\sigma}^{(1)} - \boldsymbol{\sigma}^{(2)}$ defined over the intersection $\mathcal{B}^{(1)} \cap \mathcal{B}^{(2)}$ between the two bodies (Supplementary Fig.~\ref{fig:Proof}c). This `difference solution' itself satisfies the governing equations of linear elasticity owing to their linearity, and its displacement and traction vanish on $\partial \mathcal{B}^m$. Therefore, we can show using Lemmas 1 and 3 that the difference solution vanishes in its domain of definition, meaning that $\mathbf{u}^{(1)} = \mathbf{u}^{(2)}$ and $\boldsymbol{\sigma}^{(1)} = \boldsymbol{\sigma}^{(2)}$ in $\mathcal{B}^{(1)} \cap \mathcal{B}^{(2)}$. We now focus on the behavior of the solution in body $\mathcal{B}^{(1)}$ and treat separately the cases of a void or rigid inclusion. 

In the case of a void (Supplementary Fig.~\ref{fig:Proof}d), the traction $\mathbf{t}^{(1)}$ on the internal boundary $\partial \mathcal{I}^{(1)}$ vanishes. Using the fact that $\boldsymbol{\sigma}^{(1)} = \boldsymbol{\sigma}^{(2)}$ in $\mathcal{B}^{(1)} \cap \mathcal{B}^{(2)}$, we know that $\mathbf{t}^{(1)}$ also vanishes on $\mathcal{B}^{(1)} \cap \partial \mathcal{I}^{(2)}$. As a result, there exists a finite region $\tilde{\mathcal{B}}^{(1)} = \mathcal{B}^{(1)} \cap \mathcal{I}^{(2)}$ with zero boundary traction. Thus, $\boldsymbol{\sigma}^{(1)}$ must vanish not only in $\tilde{\mathcal{B}}^{(1)}$, but also in the entire body $\mathcal{B}^{(1)}$ as a consequence of Lemma 2. This violates the fact that the surface traction $\mathbf{t}^{(1)}$ cannot vanish everywhere on $\partial \mathcal{B}^\mathrm{ext}$. 

In the case of a rigid inclusion (Supplementary Fig.~\ref{fig:Proof}e), the displacement $\mathbf{u}^{(1)}$ on the internal boundary $\partial \mathcal{I}^{(1)}$ corresponds to that of a rigid motion, i.e.~$\mathbf{u}^{(1)} = \mathbf{u}_r^{(1)} + \boldsymbol{\theta}_r^{(1)} \times \mathbf{x}$ for some fixed $\mathbf{u}_r^{(1)}$ and $\boldsymbol{\theta}_r^{(1)}$. Using the fact that $\mathbf{u}^{(1)} = \mathbf{u}^{(2)}$ in $\mathcal{B}^{(1)} \cap \mathcal{B}^{(2)}$, we know that $\mathbf{u}^{(1)} = \mathbf{u}^{(2)} = \mathbf{u}_r^{(2)} + \boldsymbol{\theta}_r^{(2)} \times \mathbf{x}$ on $\mathcal{B}^{(1)} \cap \partial \mathcal{I}^{(2)}$ for some other fixed $\mathbf{u}_r^{(2)}$ and $\boldsymbol{\theta}_r^{(2)}$. However, given that these two rigid motions must coincide at $\partial \mathcal{I}^{(1)} \cap \partial \mathcal{I}^{(2)}$, we must have $\mathbf{u}_r^{(1)} = \mathbf{u}_r^{(2)}$ and $\boldsymbol{\theta}_r^{(1)} = \boldsymbol{\theta}_r^{(2)}$. As a result, the entire surface of the finite region $\tilde{\mathcal{B}}^{(1)} = \mathcal{B}^{(1)} \cap \mathcal{I}^{(2)}$ undergoes a rigid motion. Thus, $\boldsymbol{\sigma}^{(1)}$ must vanish not only in $\tilde{\mathcal{B}}^{(1)}$, but also in the entire body $\mathcal{B}^{(1)}$ as a consequence of Lemma 2. As in the case of the void, this violates the fact that the surface traction $\mathbf{t}^{(1)}$ cannot vanish everywhere on $\partial \mathcal{B}^\mathrm{ext}$.

\section*{Ill-posed nonlinear thermal problem}

To illustrate the flexibility of our TO framework with respect to the physical model and the sparsity of the data, in this appendix we apply our method to an ill-posed nonlinear thermal imaging problem. Consider a nonlinearly conducting matrix (Supplementary Fig.~\ref{fig:HeatTransfer}a) whose heat flux $\mathbf{q}$ relates to the temperature $T$ through a nonlinear Fourier's law $\mathbf{q} = - k(1+T/T_0) \nabla T$, where $k$ is akin to a thermal conductivity and $T_0$ is a reference temperature. The matrix contains a hidden inclusion that is either perfectly insulating (no inside heat flux) or perfectly conducting (no inside temperature gradient). A thermal loading is applied, consisting of a prescribed unit temperature on the left boundary, zero temperature on the right boundary, and insulated top and bottom boundaries (Supplementary Fig.~\ref{fig:HeatTransfer}a). The goal of the inverse problem is to identify the location and shape of the inclusion under different ill-posed scenarios where the applied boundary condition and the measurements are both assumed unavailable on one of the four sides, while the temperature or flux resulting from the prescribed thermal loading are measured on the remaining three sides (Supplementary Fig.~\ref{fig:HeatTransfer}b).

%%%%
\begin{figure}
\centering
\includegraphics[width=\textwidth]{Figures/NCS_HeatTransfer}
\caption{\textbf{Identification of inclusions in a nonlinearly conducting matrix with incomplete information.} \textbf{a}, Setup of the geometry and applied boundary conditions during the loading. \textbf{b}, The geometry identification inverse problem is solved assuming that one of the four sides of the matrix is inaccessible to the user, meaning that both the applied boundary condition and the measurements are unavailable on that side. \textbf{c}, The final inferred material density $\rho$ and temperature $T$ in the case of a perfectly insulating inclusion. \textbf{d}, The final inferred material density $\rho$ and temperature $T$ in the case of a perfectly conducting inclusion.}
\label{fig:HeatTransfer}
\end{figure}
%%%%

The experiment is simulated in the FEM software Abaqus using biquadratic DC2D8 diffusive heat transfer elements, and considering $k = 1$ and $T_0 = 1$. The inverse problem is solved with our PINN-based TO framework by constructing neural network approximations for the physical quantities $\boldsymbol{\psi} = (T,\mathbf{q})$ and the density field $\rho$. Similarly to the elasticity examples, the neural networks are designed to inherently satisfy the boundary conditions (on the three sides where they are assumed to be known). The governing equations included in the loss function comprise the conservation law $\nabla \cdot \mathbf{q} = 0$ as well as the nonlinear Fourier's law $F(\mathbf{q},T,\rho) = 0$. The latter, expressed over both the matrix and the inclusion, takes the form $\mathbf{q} + \rho k(1+T/T_0) \nabla T = 0$ in the presence of a perfectly-insulating inclusion, or $\rho \mathbf{q}/k + (1+T/T_0) \nabla T = 0$ in the presence of a perfectly-conducting inclusion. We use the same training parameters and the same weights in the loss function as in the square elastic matrix examples. 

Despite the complete lack of information along an entire side, with the applied boundary condition and measurements both missing, our TO framework is able to detect the slit-shaped inclusion with reasonable accuracy in both the perfectly-insulating case (Supplementary Fig.~\ref{fig:HeatTransfer}c) and the perfectly-conducting case (Supplementary Fig.~\ref{fig:HeatTransfer}d). This example showcases the ability of the framework to generate good results when even the forward problem is ill-posed.


\bibliographystyle{abbrvnat}
\bibliography{bibliography_SI}

\end{document}