% This is samplepaper.tex, a sample chapter demonstrating the
% LLNCS macro package for Springer Computer Science proceedings;
% Version 2.20 of 2017/10/04
%
\documentclass[runningheads]{llncs}
%
% used for displaying figures
\usepackage{graphicx}

% for \mathbb
\usepackage{amsfonts}

% colored text in figures
\usepackage{xcolor} 

% eps figures
\usepackage{epstopdf}

% maths
\usepackage{amsmath}

% for font color
\usepackage{xcolor}

% roman numerals in bullets
% help from: https://tex.stackexchange.com/a/54057
%\usepackage{enumerate}
\usepackage[shortlabels]{enumitem}

\usepackage{caption}

% cell width in table: https://tex.stackexchange.com/a/5020
\usepackage{array}
\newcolumntype{L}[1]{>{\raggedright\arraybackslash}p{#1}}
\newcolumntype{C}[1]{>{\centering\arraybackslash}p{#1}}
\newcolumntype{R}[1]{>{\raggedleft\arraybackslash}p{#1}}

\usepackage[backend=biber,
sorting=nyt,
url=false,
doi=false,
maxnames=4,
minnames=1,
style=lncs]{biblatex}

% FIXES FOR BROKEN lncs.bbx
\DeclareNameFormat{author}{%
  \ifdefvoid{\namepartprefix}{}{\namepartprefix\space}\namepartfamily, \namepartgiveni%
  \ifthenelse{\value{listcount}<\value{liststop}}
    {\addcomma\space}%
    {}%
  \usebibmacro{name:andothers}%
}
\DeclareNameFormat{editor}{%
  \ifdefvoid{\namepartprefix}{}{\namepartprefix\space}\namepartfamily, \namepartgiveni%
  \ifthenelse{\value{listcount}<\value{liststop}}
    {\addcomma\space}%
    {\space\ifthenelse{\value{listcount}>1}
      {(\bibstring{editors})}
      {(\bibstring{editor})}}%
}
\DeclareBibliographyAlias{misc}{article}
\DeclareFieldFormat[article]{volume}{\textbf{#1}}
% END FIXES FOR BROKEN lncs.bbx
\addbibresource{references.bib}

% for check and x marks: https://tex.stackexchange.com/a/42620
\usepackage{pifont}
\newcommand{\cmark}{\ding{51}} % checkmark
\newcommand{\xmark}{\ding{55}} % cross mark 

% for subfigure
%\usepackage{subcaption}

\definecolor{yelloworange}{HTML}{FFC000}
\DeclareMathOperator*{\argmin}{arg\,min}
\DeclareMathOperator{\DICE}{DICE}
\DeclareMathOperator{\ASSD}{ASSD}

\newcommand\figref{Fig.~\ref}
\newcommand\tabref{Table~\ref}
\newcommand\secref{Section~\ref}

%\newcommand{\myparagraph}[1]{\paragraph{#1}}
%\newcommand{\myparagraph}[1]{\textbf{\textit{#1}}}
\newcommand{\myparagraph}[1]{\textit{#1.}}
%\linepenalty=1000

\usepackage{hyperref}
% If you use the hyperref package, please uncomment the following line
% to display URLs in blue roman font according to Springer's eBook style:
\renewcommand\UrlFont{\color{blue}\rmfamily}
% \newcommand*{\ANONYMIZED}{}%

\begin{document}
% adaptive multi-scale online likelihood learning network neural model

% Multi-scale Adaptive Online Learning for Interactive Segmentation of COVID-19 Lung Lesions
  
% AMONet: Adaptive Multi-scale Online Likelihood Network for AI-assisted Interactive Segmentation
%
\title{Supplementary Material: Adaptive Multi-scale Online Likelihood Network for AI-assisted Interactive Segmentation}
% \title{Contribution Title\thanks{Supported by organization x.}}
%
\titlerunning{Supplementary Material: Adaptive Multi-scale Online Likelihood Network}
\ifdefined\ANONYMIZED
\author{Anonymous MICCAI Submission, Paper 2125}
\else
\author{Muhammad Asad\inst{1}
% \orcidID{0000-0002-3672-2414} 
\and
Helena Williams \inst{2}
% \orcidID{1111-2222-3333-4444} 
\and
Indrajeet Mandal\inst{3} \and
Sarim Ather \inst{3} \and
Jan Deprest \inst{2} \and
Jan D'hooge \inst{4} \and
Tom Vercauteren \inst{1}
% \orcidID{0000-0003-1794-0456}
}
% %
\authorrunning{M Asad et al.}
% index{Asad, Muhammad}
% index{Williams, Helena}
% index{Mandal, Indrajeet}
% index{Ather, Sarim}
% index{Deprest, Jan}
% index{D'hooge, Jan}
% index{Vercauteren, Tom}

% First names are abbreviated in the running head.
% If there are more than two authors, 'et al.' is used.
%
\institute{School of Biomedical Engineering \& Imaging Sciences, King’s College London, UK \and Department of Development \& Regeneration, KU Leuven, Belgium \and Radiology Department, Oxford University Hospitals NHS Foundation Trust, UK \and
Department of Cardiovascular Sciences, KU Leuven, Belgium
% \email{*******}
}
\fi

%
\maketitle              % typeset the header of the contribution
%
\begin{figure}[htb!]
  \centering
  \includegraphics[width=0.75\linewidth,trim={0cm, 1cm, 0cm, 1cm}]{figures/synthetic_eval/full_adaptive_onlinemlp_tau_tauonly_.png}
  \caption{
  Line search for finding optimal $\tau$ in Eq.~(2) in the paper. Shows mean and standard deviation of Dice (\%) accuracy for a given $\tau \in [0,1]$. The selected optimal value $\tau=0.3$ results in the best Dice with the least standard deviation.
  }
  \label{fig:syneval_accuracy1}
\end{figure}

\begin{figure}[htb!]
  \centering
  \includegraphics[width=0.8\linewidth,trim={0cm, 2cm, 0cm, 2cm}]{figures/synthetic_eval/full_adaptive_onlinemlp_graphcut_25x25.eps}
  \caption{
  Grid search for finding the optimal parameters for GraphCut regularization, where $\lambda$ and $\sigma$ are shown along with corresponding Dice (\%) accuracy. $\times$ shows the optimal values ($\lambda=2.5$, $\sigma=0.15$) used in our paper experiments.
  }
  \label{fig:syneval_accuracy2}
\end{figure}

\begin{figure}[t!]
  \centering
  \includegraphics[width=1.0\linewidth]{figures/qualitative_analysis_v2/Picture2.jpg}
  \caption{
  Visual comparison of two consecutive slices showing results using scribbles from an expert annotator with MONet and MIDeepSegTuned. The advantage of using proposed MONet is visible where it even addresses discrepancies within ground truth labels by applying learned knowledge to infer possible lesion and non-lesion segmentations. 
  }
  \label{fig:user_scrib_qualit_results_analysis}
\end{figure}





%
% ---- Bibliography ----
%
% BibTeX users should specify bibliography style 'splncs04'.
% References will then be sorted and formatted in the correct style.
%
\end{document}
