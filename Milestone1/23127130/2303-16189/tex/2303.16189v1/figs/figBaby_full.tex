\begin{figure*}[t!]
 
  \vspace*{-0.0in}
  \centering
  \scalebox{0.85}{
    \begin{tikzpicture}
     \node[anchor=north west](main) at (0in,0in)
      {{\includegraphics[width=0.9\textwidth,clip=true,trim=0in
      7.8in 0.0in 0.5in]{figs/baby_6.pdf}}};
      % Iteration number
     \node[anchor=north] at ($ (main.south) + (-4.45, 0.3) $){\footnotesize \IterNum=1};
     \node[anchor=north] at ($ (main.south) + (-2.65, 0.3) $){\footnotesize \IterNum=5};
     \node[anchor=north] at ($ (main.south) + (-0.85, 0.3) $){\footnotesize \IterNum=10};
%     \node[anchor=north west] at (0.92in,-0.05in) {\textbf{(a)}};
%     \node[anchor=north west] at (2.00in,-0.05in) {\textbf{(b)}};
%     \node[anchor=north west] at (3.09in,-0.05in) {\textbf{(c)}};
    \end{tikzpicture}
  }
  % \vspace*{-0.0in}
  \vspace{-8pt}
  \caption{
    \textbf{\small Qualitative Visualization of Planning and Execution Procedure in BabyAI}.
    \textit{Left} depicts the planning through iterative energy minimization where $N$ is the sample iteration number.
    \textit{Right} shows the execution of the concatenate action sequences.
    Two task settings are illustrated:
    \textbf{(a)}: \textbf{Trajectory planning}, where the task is to solely plan a trajectory leading to the goal location.
    \textbf{(b)}: \textbf{Instruction completion}, where a sequence of tasks are commanded, and an additional "Open" is involved to get through the doors.
    Target locations are marked with \protect{\raisebox{-.05cm}{\includegraphics[height=.30cm]{figs/target_highlight.png}}}.
  }
  \vspace*{-10pt}
 \label{fig:babyai_plan_execute}
\end{figure*}
