\begin{figure*}[t!]
 
  \vspace*{-0.0in}
  \centering
  \scalebox{0.85}{
    \begin{tikzpicture}
     \node[anchor=north west] at (0in,0in)
      {{\includegraphics[width=1.0\textwidth,clip=true,trim=0in
     6.8in 0.3in 0.5in]{figs/property_result.pdf}}};
    % \node[yshift=-0pt,anchor=north west] at (-0.05in,-0.08in) {\bf (a)};
    % \node[yshift=-0pt,anchor=north west] at (2.72in,-0.08in) {\bf (b)};
    % \node[yshift=-0pt,anchor=north west] at (-0.15in,-1.47in) {\bf (c)};
%     \node[anchor=north west] at (0.92in,-0.05in) {\textbf{(a)}};
%     \node[anchor=north west] at (2.00in,-0.05in) {\textbf{(b)}};
%     \node[anchor=north west] at (3.09in,-0.05in) {\textbf{(c)}};
    \end{tikzpicture}
  }
  \vspace{-18pt}
  \caption{\small
    \textbf{Qualitative Visualization of Generalization Tests.} 
    (a): Online adaptation (\textit{medium}), trained in plane world and tested in world with lavas;
    (b): Generalization (\textit{hard}), trained in plane world and tested in maze world;
    (c): Task composition (\textit{easy}), each model only perceive half number of obstacles.
   Target locations and unperceivable obstacles are marked with \protect{\raisebox{-.05cm}{\includegraphics[height=.30cm]{figs/target_highlight.png}}} 
   and
   \protect{\raisebox{-.05cm}{\includegraphics[height=.30cm]{figs/invisible.png}}},
   respectively.
}
  \vspace{-20pt}
 \label{fig:property}
\end{figure*}
