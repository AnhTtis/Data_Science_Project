
\documentclass{article} % For LaTeX2e
\usepackage{iclr2023/iclr2023_conference,times}

% Optional math commands from https://github.com/goodfeli/dlbook_notation.
\newcommand{\bbox}{\text{bbox}}
\newcommand{\alphapck}{\alpha_\bbox}
\newcommand{\kcycle}{\text{k-CyPCK}}
\newcommand{\cycle}{\text{-CyPCK}}

\newcommand{\I}{\mathbf{I}}
\newcommand{\Ia}{\I^\text{a}}
\newcommand{\Ib}{\I^\text{b}}
\newcommand{\Iatob}{\I^\text{a $\rightarrow$ b}}
\newcommand{\F}{\mathbf{F}}
\newcommand{\Fa}{\F^\text{a}}
\newcommand{\Fb}{\F^\text{b}}
\newcommand{\f}{\mathbf{f}}
\newcommand{\fa}{\f^\text{a}}
\newcommand{\fb}{\f^\text{b}}
\newcommand{\p}{\mathbf{p}}
\newcommand{\pa}{\p^\text{a}}
\newcommand{\pb}{\p^\text{b}}
\newcommand{\A}{\boldsymbol{\Phi}_\text{align}}
\newcommand{\G}{\mathbf{G}}
\newcommand{\C}{\mathbf{C}}
\newcommand{\Ca}{\C^\text{a}}
\newcommand{\Cb}{\C^\text{b}}
\newcommand{\cc}{\mathbf{c}}
\newcommand{\cca}{\cc^\text{a}}
\newcommand{\ccb}{\cc^\text{b}}
\newcommand{\Irec}{\I_\text{Recon}}
\newcommand{\M}{\mathbf{M}}
\newcommand{\Mrec}{\M_\text{Recon}}
\newcommand{\loss}{\mathcal{L}}
\newcommand{\T}{\mathcal{T}}
\newcommand{\W}{\mathcal{W}}
\newcommand{\Id}{\mathcal{I}}



%%%%%%%%%%%%%%%%%%%%%%%%%%%%%%%%%%%%%%%%%%%%%%%%%%%%%%%%%%%%%%%
%%%     packages
%%%%%%%%%%%%%%%%%%%%%%%%%%%%%%%%%%%%%%%%%%%%%%%%%%%%%%%%%%%%%%%

\ifdefined\siggraph
\usepackage{times}
\fi

%\usepackage{parskip}
\usepackage{color}
\usepackage{ifthen}
\usepackage{float}
\usepackage{alltt}
\usepackage{newlfont} % for Box
\usepackage{array}
\usepackage{wrapfig}
\usepackage{booktabs}
\usepackage{multirow}
\usepackage{amsfonts}
\usepackage{dsfont}
\usepackage[linesnumbered,ruled,vlined]{algorithm2e}
 
%%% Coloring the comment as blue
\newcommand\mycommfont[1]{\footnotesize\ttfamily\textcolor{blue}{#1}}
\SetCommentSty{mycommfont}
 



\let\labelindent\relax
% 
\usepackage{enumitem}

%%%%%%%%%%%%%%%%%%%%%%%%%%%%%%%%%%%%%%%%%%%%%%%%%%%%%%%%%%%%%%%
%%%     newcommands
%%%%%%%%%%%%%%%%%%%%%%%%%%%%%%%%%%%%%%%%%%%%%%%%%%%%%%%%%%%%%%%
% utils
\makeatletter
\newcommand{\printfnsymbol}[1]{%
  \textsuperscript{\@fnsymbol{#1}}%
}

\newcommand{\ycReb}[1]{\textcolor{black}{{#1}}}
\newcommand{\ycCITE}[1]{\textcolor{cyan}{[CITE: {#1}]}}
\newcommand{\ycREF}[1]{\textcolor{blue}{[REFER: {#1}]}}
\newcommand{\ycMOD}[1]{\textcolor{green}{ycMOD: #1}}
\newcommand{\ycSUG}[1]{\textcolor{magenta}{ycSUGGEST: #1}}
\newcommand{\hongyi}[1]{\textcolor{black}{#1}}
\newcommand{\fix}{\marginpar{FIX}}
\newcommand{\new}{\marginpar{NEW}}


\newcommand{\netName}{LEAP\xspace}

\newcommand{\st}{\mathbf{s}_t}
\newcommand{\stp}{\mathbf{s}_{t+1}}
\newcommand{\stpp}{\mathbf{s}_{t+2}}
\newcommand{\at}{\mathbf{a}_t}
\newcommand{\atp}{\mathbf{a}_{t+1}}
\newcommand{\atpp}{\mathbf{a}_{t+2}}
\newcommand{\bs}{\mathbf{s}}
\newcommand{\ba}{\mathbf{a}}
\newcommand{\br}{\mathbf{r}}
\newcommand{\bz}{\mathbf{z}}
\newcommand{\bx}{\mathbf{x}}
\newcommand{\btau}[1]{\bm{\tau}^{#1}}
%\newcommand{\btau}[1][{}]{\bm{\tau}^{#1}}
\newcommand{\opt}{\mathcal{O}}
\newcommand{\bopt}{\bm{\mathcal{O}}}
\newcommand\undermat[2]{%
  \makebox[0pt][l]{$\smash{\underbrace{\phantom{%
    \begin{matrix}#2\end{matrix}}}_{\text{$#1$}}}$}#2}
\def\NoNumber#1{{\def\alglinenumber##1{}\STATE #1}\addtocounter{ALG@line}{-1}}
\def\x{\mathbf{x}}
\newcommand{\expect}[2]{\mathbb{E}_{#1} \left[ #2 \right] }
\newcommand{\te}[1]{\texttt{#1}}

\newcommand{\Hrz}{T}
\newcommand{\DataNum}{M}
\newcommand{\IterNum}{N}
\newcommand{\idSet}{I}

\newcommand{\enModel}{E}
\newcommand{\mParam}{\theta}
\definecolor{purple}{rgb}{1, 0, 1}

\newcommand{\ie}{\emph{i.e.,}\xspace}
\newcommand{\eg}{\emph{e.g.,}\xspace}
\newcommand{\abr}{\emph{abbr.}\xspace}
\newcommand{\ea}{\emph{et al.}\xspace}
\newcommand{\gensync}{\emph{GenSync}\xspace}
\newcommand{\colosseum}{\emph{Colosseum}\xspace}
\newcommand{\srep}{\emph{SREP}\xspace} % Set Reconciliation Enhances
\newcommand{\srepsim}{\emph{SREPSim}\xspace}
% Propagation
\newcommand{\esrep}{\emph{E-SREP}\xspace}
\newcommand{\epsrep}{\emph{EP-SREP}\xspace}
\newcommand{\mesrep}{\emph{ME-SREP}\xspace}
\newcommand{\mempoolsync}{\emph{MempoolSync}}

\newcommand{\fref}[1]{Fig.~\ref{#1}}
\newcommand{\tref}[1]{Table~\ref{#1}}
\newcommand{\aref}[1]{Algorithm~\ref{#1}}
\newcommand{\procref}[1]{Procedure~\ref{#1}}
\newcommand{\sref}[1]{Section~\ref{#1}}
\newcommand{\lineref}[1]{line~\ref{#1}}
\newcommand{\appref}[1]{Appendix~\ref{#1}}

% Change \eqref
\LetLtxMacro{\originaleqref}{\eqref}
\renewcommand{\eqref}{Eq.~\originaleqref}

% Theorems and corollaries
\newcounter{theoremcount}
\setcounter{theoremcount}{0}
\DeclareRobustCommand{\theorem}[1]{%
  \refstepcounter{theoremcount}%
  \noindent\textit{\textbf{Theorem \thetheoremcount\label{theorem:#1}: }}%
}
\DeclareRobustCommand{\theoremref}[1]{Theorem~\ref{theorem:#1}}

\DeclareRobustCommand{\proof}{\emph{Proof:}\xspace}
\DeclareRobustCommand{\qqed}{\hfill$\blacksquare$}

\newcounter{corollcount}
\setcounter{corollcount}{0}
\DeclareRobustCommand{\coroll}[1]{%
  \refstepcounter{corollcount}%
  \noindent\textit{\textbf{Corollary \thecorollcount\label{coroll:#1}: }}%
}
\DeclareRobustCommand{\corollref}[1]{Corollary~\ref{coroll:#1}}

\newcounter{lemmacount}
\setcounter{lemmacount}{0}
\DeclareRobustCommand{\lemma}[1]{%
  \refstepcounter{lemmacount}%
  \noindent\textit{\textbf{Lemma \thelemmacount\label{lemma:#1}: }}%
}
\DeclareRobustCommand{\lemmaref}[1]{Lemma~\ref{lemma:#1}}

\newcounter{definitioncount}
\setcounter{definitioncount}{0}
\DeclareRobustCommand{\definition}[1]{%
  \refstepcounter{definitioncount}%
  \noindent\textit{\textbf{Definition \thedefinitioncount\label{definition:#1}: }}%
}
\DeclareRobustCommand{\defref}[1]{Definition~\ref{definition:#1}}

%notes of different authors
\newif\ifnotes
\notestrue
\notesfalse

\newif\ifdiff
\difftrue
\difffalse

\newcommand{\anote}[1]{\ifnotes $\ll$\textsf{\textcolor{purple}{Ari: {#1}}}$\gg$ \fi}
\newcommand{\nnote}[1]{\ifnotes $\ll$\textsf{\textcolor{orange}{Novak: {#1}}}$\gg$ \fi}
\newcommand{\diff}[1]{\ifdiff\textcolor{orange}{#1}\else#1\fi}

%%% Local Variables:
%%% mode: latex
%%% TeX-master: "main"
%%% End:


%%%%%%%%%%%%%%%%%%%%%%%%%%%%%%%%%%%%%%%%%%%%%%%%%%%%%%%%%%%%%%%
%%%     Get started
%%%%%%%%%%%%%%%%%%%%%%%%%%%%%%%%%%%%%%%%%%%%%%%%%%%%%%%%%%%%%%%
\title{Planning with \hongyi{Sequence} Models through Iterative Energy Minimization}

% Authors must not appear in the submitted version. They should be hidden
% as long as the \iclrfinalcopy macro remains commented out below.
% Non-anonymous submissions will be rejected without review.

\author{
\qquad Hongyi Chen\thanks{\fontsize{6}{7} denotes equal contribution. Correspondence to \texttt{hchen657@gatech.edu},\texttt{ yilundu@mit.edu}, \texttt{yychen2019@gatech.edu}}\;\;\printfnsymbol{2}, ~Yilun Du \printfnsymbol{1}\printfnsymbol{3}, ~Yiye Chen\printfnsymbol{1}\printfnsymbol{2}, ~Joshua Tenenbaum\printfnsymbol{3}, ~Patricio Vela\printfnsymbol{2}\\
\qquad\qquad~Georgia Institute of Technology\printfnsymbol{2}\qquad ~Massachusetts Institute of Technology\printfnsymbol{3}\\
}

% The \author macro works with any number of authors. There are two commands
% used to separate the names and addresses of multiple authors: \And and \AND.
%
% Using \And between authors leaves it to \LaTeX{} to determine where to break
% the lines. Using \AND forces a linebreak at that point. So, if \LaTeX{}
% puts 3 of 4 authors names on the first line, and the last on the second
% line, try using \AND instead of \And before the third author name.


%\iclrfinalcopy % Uncomment for camera-ready version, but NOT for submission.
\begin{document}


\maketitle

\begin{abstract}
Recent works have shown that \hongyi{sequence} modeling can be effectively used to train reinforcement learning (RL) policies.  
However, the success of applying existing sequence models to planning, 
in which we wish to obtain a trajectory of actions to reach some goal,
is less straightforward. 
The typical autoregressive generation procedures of \hongyi{sequence} models preclude sequential refinement of earlier steps, which limits the effectiveness of a predicted plan. 
In this paper, we suggest an approach towards integrating planning with \hongyi{sequence} models based on the idea of iterative energy minimization, and illustrate how such a procedure leads to improved RL performance across different tasks.
We train a masked language model to capture an implicit energy function over trajectories of actions, 
and formulate planning as finding a trajectory of actions with minimum energy. 
We illustrate how this procedure enables improved performance over recent approaches across BabyAI and Atari environments. 
We further demonstrate unique benefits of our iterative optimization procedure,
involving new task generalization, test-time constraints adaptation, and the ability to compose plans together. Project website: \href{https://hychen-naza.github.io/projects/LEAP/index.html}{https://hychen-naza.github.io/projects/LEAP}.

% enabling us to generalize to harder tasks at test time, as well as to adapt to novel test-time constraints.

% effective online task adaptation and generalization to harder tasks at test time. 

% Many previous works integrate reinforcement learning (RL) with widely used language models for generic sequence modeling tasks. However, existing approaches typically rely on

% But usually they rely on autoregressive decoding which prevents iterative refinement and planning and often have poor long-horizon accuracy due to compounding errors. In this paper, we present a principled approach towards iterative planning with language model based on the idea of energy minimization. We adopt masked language modeling as our energy models and draw action samples from the masked locations conditioned on the surrounding context. The iterative refining process will reject the action samples with high energy values according to the energy model and optimized the planned action sequence. 

% We validate the effectiveness of the proposed framework on sequence decision making problems including Atari, Babyai and Graph-walking that involve reward-sparse tasks and high-dimensional observation inputs. Moreover, we explore its unusual and useful properties in test-time flexibility and generalization.

\end{abstract}

%%%% introduction
\section{Introduction}

The increasing complexity of source code poses a key challenge to the reliability of large-scale software systems. Software bugs in these systems can lead to safety issues~\cite{bug_safety} for users around the world as well as cause non-negligible financial losses~\cite{bug_loss}. As such, developers have to spend a large amount of time and effort on bug fixing. Consequently, \aprfull (\apr), designed to automatically generate patches to fix software bugs, has attracted wide attention from both academia and industry~\cite{long2016prophet, legoues2012genprog, long2015spr, lou2020can, tufano2018empstudy}. 


To achieve \apr, one popular approach is known as Generate-and-Validate (G\&V)~\cite{qi2015gv, ghanbari2019prapr, lou2020can, le2016hdrepair, legoues2012genprog, wen2018capgen, hua2018sketchfix, martinez2016astor, koyuncu2020fixminder, liu2019tbar, liu2019avatar}, which is typically based on the following pipeline: First, fault localization techniques~\cite{wong2016fl, abreu2007ochiai, zhang2013injecting, papadakis2015metallaxis, li2019deepfl, li2017transforming} are applied to determine the suspicious locations in programs where bugs are likely to exist. Then, the buggy locations are used by the \apr tools to generate a list of patches that replace buggy lines with correct lines. Afterward, each patch is validated against the original test suite to identify any \emph{plausible patches} (i.e., passing all tests in the test suite). Finally, to determine the \emph{correct patches}, developers examine the list of plausible patches to see if any of them can correctly fix the bug. 

Traditional \apr tools can mainly be categorized into heuristic-based~\cite{legoues2012genprog, le2016hdrepair, wen2018capgen}, constraint-based~\cite{mechtaev2016angelix, le2017s3, demacro2014nopol, long2015spr} and \template~\cite{ghanbari2019prapr, hua2018sketchfix, martinez2016astor, liu2019tbar, liu2019avatar}. Among these traditional tools, \template \apr tools~\cite{ghanbari2019prapr, liu2019tbar, benton2020effectiveness} have been able to achieve state-of-the-art results. \Template \apr tools typically leverage pre-defined templates (e.g., adding a nullness check) for bug fixing. However, since these fix templates are typically handcrafted, the number and types of bugs they are able to fix can be limited. 



To address the limitations of traditional \apr, researchers have proposed various \learning \apr tools~\cite{li2020dlfix, chen2018sequencer, jiang2021cure, lutellier2020coconut, zhu2021recoder, ye2022rewardrepair} based on the \nmtfull (\nmt) architecture~\cite{sutskever2014mt} where the input is the buggy code snippets and the goal is to translate the buggy code snippets into a fixed version. To accomplish this, \learning \apr tools require supervised training datasets with pairs of both buggy and fixed code snippets in order to learn how to perform this translation step. These training data are usually obtained by mining historical bug fixes using heuristics/keywords~\cite{dallmeier2007benchmark}, which can be imprecise for identifying bug-fixing commits; even the actual bug-fixing commits can include irrelevant code changes, leading to further pollution in the dataset~\cite{xia2022alpharepair}.
% 
Moreover, it can be hard for such \apr tools to generalize and fix bug types unseen during training. 



To better leverage recent advances in \plmfull{s} (\plm{s}), researchers~\cite{xia2022alpharepair, xia2023repairstudy, kolak2022patch, prenner2021codexws} have directly applied \plm{s} to generate patches without bug-fixing datasets. These \llm-based \apr tools work by either directly generating a complete code function~\cite{prenner2021codexws, xia2023repairstudy} or predict/infill the correct code snippet given its surrounding context~\cite{xia2022alpharepair, xia2023repairstudy}. By directly using \llm{s} that are pre-trained on billions of open-source code snippets, \llm-based \apr tools can achieve state-of-the-art performance on many repair datasets~\cite{xia2022alpharepair}. 


% 
%
%

Traditional \apr tools have long used the insight of the \emph{plastic surgery hypothesis}~\cite{barr2014plastic} where it states that the code ingredients to fix a bug already exist within the same project. Traditional \apr tools have manually designed pattern-~\cite{ghanbari2019prapr, saha2017elixir} or heuristic-based~\cite{jiang2018simfix, legoues2012genprog} approaches to finding and using such relevant code ingredients to generate fixes for bugs. However, the plastic surgery hypothesis has been largely ignored in \llm-based \apr. In fact, \llm provides a unique opportunity to fully automate the plastic surgery hypothesis idea via fine-tuning (learning project-specific information via model updates from the buggy project) and prompting (directly providing relevant code ingredients to the model), and make it directly applicable to different languages (since the \llm{s} are typically multi-lingual).%
Moreover, despite the intensive manual efforts involved, traditional \apr tools still cannot fully leverage project-specific information due to large search space for leveraging/composing existing code ingredients. In contrast, the project-specific information can effectively leveraged by \llm{s} due to their power in code understanding/vectorization, e.g., even partial/imprecise information may still guide \llm{s} in correct patch generation!
 To this end, we ask the question: \emph{How useful is the plastic surgery hypothesis in the era of \plm{s}}?








\mypara{Our Work.} To answer the question, we present \ourtech{\xspace} -- a \llm-based approach that automatically utilizes the plastic surgery hypothesis by systematically combining multiple fine-tuning and prompting strategies for \apr. \ourtech fine-tunes \plm{s} using two novel domain-specific training strategies: \textbf{\epfinetune} -- we fine-tune using the original buggy project by aggressively masking out a high percentage of tokens, which allows \plm to learn project-specific code tokens and programming styles; and \textbf{\rofinetune} -- which only masks out a single continuous code sequence per training sample, allowing the model to get used to the final \csapr task of predicting a single continuous code sequence. Furthermore, we directly leverage the ability for \plm{s} to understand natural language instructions and introduce a novel prompting strategy, \textbf{\idprompting}, which uses information retrieval and static analysis to obtain a list of relevant identifiers for the buggy lines. While such relevant identifiers are critical for fixing some difficult bugs, they may not be seen by the \llm during inference due to limited context window size. Through the use of prompting, we directly tell the model to use these extracted identifiers (relevant code ingredients) to generate the correct code. Finally, to perform repair, we combine all four model variants (including the base model, both fine-tuned models and the base model with prompting) for the final repair.





While our insight of leveraging the plastic surgery hypothesis for \llm-based \apr is generalizable across different types of \plm{s}, to implement \ourtech, we choose a recent \plm{\xspace}, \ctfive~\cite{wang2021codet5}, which is pre-trained on millions of open-source code snippets. \ctfive is an encoder-decoder model trained using \mspfull (\msp) objective where a percentage of tokens are masked out and each continuous masked token sequence is referred to as a masked span. Also, although we only extract relevant identifiers from the current buggy project (since this paper focuses on the plastic surgery hypothesis), our work can be easily extended to obtain other code information (such as relevant statements or functions) from other sources, such as  the massive pre-training corpora~\cite{husain2020codesearchnet} or historical bug-fixing datasets~\cite{jiang2019infer}, which can provide more coding knowledge for \llm{s}. Besides, although we mainly focus on using traditional string comparison algorithms for information retrieval in this paper, these techniques can be easily replaced by other frequency-based retrieval~\cite{robertson2009probabilistic} and neural search (or embedding-based search)~\cite{reimers2019sentence}.
  In summary, this paper makes the following contributions:


%


\begin{itemize}[noitemsep, leftmargin=*, topsep=0pt]
    \item \textbf{Dimension.} This paper is the first to revisit the important plastic surgery hypothesis in the era of \llm{s}. It opens up a new dimension for \llm-based \apr to incorporate previously neglected information from the buggy project itself to boost \apr performance. Furthermore, it demonstrates the promising future of retrieval-based prompting for modern \llm-based \apr.
    \item \textbf{Implementation.} We implement \ourtech based on the recent \ctfive model. We augment the model using two novel fine-tuning strategies: \epfinetune and \rofinetune, along with a novel prompting strategy based on information retrieval and static analysis: \idprompting. We combine the patches generated by all four models together and perform patch ranking to speed up \apr.% 
    \item \textbf{Evaluation Study.} We conduct an extensive evaluation against state-of-the-art \apr tools. On the widely studied \dfj 1.2 and 2.0 datasets~\cite{just2014dfj}, \ourtech is able to achieve the new state-of-the-art results of 89 and 44 correct bug fixes (15 and 8 more than best baseline) respectively.  Furthermore, we perform a broad ablation study to justify our design. \ourtech demonstrates for the first time that the plastic surgery hypothesis can substantially boost \llm-based \apr and advance state-of-the-art \apr, while being fully automated and general. Moreover, even partial/imprecise code ingredients may still effectively guide \llm{s} for \apr!
\end{itemize}



%%%% related works
\vspace{-5pt}
\section{Related Work}
\vspace{3pt}
%Transformer architectures with self-attention \citep{vaswani2017attention}
%which is formalized as sequential Markov decision processes,
%They demonstrate that supervised learning alone, without temporal difference (TD) learning, can be remarkably effective for offline RL, as the effectiveness of these models is determined by the representational capacity of the sequence model rather than algorithmic sophistication 
\myparagraph{\hongyi{Sequence} Models and Reinforcement Learning.} Sequence modeling with deep networks, from sequence-to-sequence models \citep{LSTM1997,sutskever2014sequence} to BERT \citep{devlin-etal-2019-bert} and XLnet \citep{NEURIPS2019_dc6a7e65}, have shown promising results in a series of language modeling problems \citep{dai2019transformer,sutskever2014sequence,liu2019text, dehghani2018universal}. With these advances, people start applying sequence models to represent components in standard RL such as policies, value functions, and models to improved performance \citep{rl2018lstm,RL2020Transformer,kapturowski2018RLrecurrent}. While the sequence models provide memory information to make the agent predictions temporally and spatially coherent, they still rely on standard RL algorithm to fit value functions or compute policy gradients. Furthermore, recent works replace as much of the RL pipeline as possible with sequence modeling to leverage its scalability, flexible representations and causally reasoning \citep{janner2021tt, chen2021dt,furuta2021generalized, zheng2022online,emmons2021rvs,li2022pre}. However, those methods adopt autoregressive modeling objectives and the predicted trajectory sequences have no easy way to be optimized, which will inevitably lower the long-horizon accuracy. \hongyi{Recent studies point out that using sequence models \citep{chen2021dt, emmons2021rvs} rather than typical value-based approaches have difficulty converging in stochastic environments \citep{paster2020planning, dtStoch}.}


% A number of methods have been used for inference and sampling in EBMs, from Gibbs Sampling \citep{hinton2006fast}, to Langevin Dynamics \citep{du2019implicit}, and Metropolis Hastings \citep{goyal2021exposing}. 


\myparagraph{Planning in Reinforcement Learning.} Planning has been explored extensively in model-based RL, which learns how the environment respond to actions \citep{sutton1991dyna}. The learned world dynamic model is exploited to predict the conditional distribution over the immediate next state or autoregressively reason the long-term future \citep{chiappa2017recurrent_env,ke2018modeling}. However, due to error compounding, plans generated by this procedure often look more like adversarial examples than optimal trajectories when the planning horizon is extended \citep{bengio2015scheduled, talvitie2014model, asadi2018lipschitz}. To avoid the aforementioned issues, simple gradient-free method like Monte Carlo tree search \citep{10.1007/978-3-540-75538-8_7}, random shooting \citep{nagabandi2018neural} and \hongyi{beam search \citep{plate}} are explored. Another line of works studied how to break the barrier between model learning and planning, and plan with an imperfect model, include training an autoregressive latent-space model to predict values for abstract future states \citep{tamar2016value, oh2017value, schrittwieser2020mastering, plate}; energy-based models of policies for model-free reinforcement learning \citep{haarnoja2017reinforcement}; \hongyi{improve the offline policies by planning with learned models for model-based reinforcement learning \citep{mopo, muzeroUnp}}; directly applying collocation techniques for direct trajectory optimization \citep{erez2012trajectory, du2019model}; and folding planning into the generative modeling process \citep{janner2021tt}. In contrast to these works, we explore having planning directly integrated in a language modeling framework.

\myparagraph{Energy Based Learning.} Energy-Based Models (EBMs) capture dependencies between variables by associating a scalar energy to each configuration of the variables, and provide a unified theoretical framework for many probabilistic and
non-probabilistic approaches to learning \citep{lecun2006tutorial}. Prior works have explored EBMs for policy training in model-free RL \citep{haarnoja2017reinforcement}, modeling the environment dynamics in model-based RL \citep{du2019model, janner2022planning} and natural images \citep{du2019implicit, dai2019exponential}, as well as energy values over text \citep{goyal2021exposing}. Most similar to our work, \citet{du2019model} illustrates how energy optimization in EBMs naturally support planning given start and goal state distributions. However, the 
 underlying training relies on a constrastive divergence, which is difficult to train \citep{du2020improved}. 
 On the other hand, our training approach relies on a more stable masked language modeling objective.


%%%% method
\section{Method}
\label{sec:method}

% \ml{``Inconsistent'' to ``large variation''}

% In this section, we propose our methods based on the observations in Section \ref{sec:motivation}.
In this section, we propose two techniques to further enhance the strong baseline to capture the variation of activation distributions better.
We first introduce spatial re-scaling to adapt the network to pixel-to-pixel variation.
We then propose channel-wise shifting and re-scaling to better capture the channel-to-channel variation.
Meanwhile, as both of the two methods are image-dependent, the image-to-image variation can be captured naturally.
By combining the two methods with our strong baseline, we build our enhanced BNN for SR, named EBSR.

% Because the activation distributions among pixels, channels and images have large variations \red{**are highly inconsistent} in SR networks, we introduce spatial re-scaling to adapt to pixel-wise variations and channel shift and re-scaling to adapt to channel-wise variations. And both of them are image-dependent to adapt to image-wise variations, which means during inference our network re-scales and shifts the distributions of activations flexibly for different input images. Based on these methods, we build an enhanced binary neural network for image super-resolution (EBSR).

% According to [3], the difference of activation magnitudes indicates different scaling factors are needed for each pixel.

\subsection{Spatial Re-scaling}
% It is better to use different scaling factors for different pixels to reduce the quantization error and retain more detailed information for image super-resolution. 

% \ml{In the main method, we do not need to introduce the previous works but can focus on introducing our own method. Channel rescaling in Real-to-binary Net is not relevant in this context.}

% Re-scaling the output of binary convolutions was proposed at the birth of BNN in XNOR-Net \cite{rastegari2016xnor} to reduce quantization error and improve accuracy for image classification tasks.
% It is computed as below:
% \begin{equation}
% \mathcal{A} * \mathcal{W} \approx(\operatorname{sign}(\mathcal{A}) \circledast \operatorname{sign}(\mathcal{W})) \odot \mathcal{K} \alpha
% \label{eq:xnor-net rescale}
% \end{equation}
% where $\circledast$ denotes the binary convolution and $\odot$ denotes the element-wise multiplication.
% $\mathcal{A}$, $\mathcal{W}$, $\alpha$, and $\mathcal{K}$ denote the activation, weight, weight scaling factor, and activation scaling factor, respectively.
%  Later in XNOR-Net++ \cite{bulat2019xnor}, Bulat et al. fuse the activation and weight scaling factors into a single one that is learned end-to-end based on gradients and this improves the classification accuracy on ImageNet dataset.

% % It is computed as Eq.~\ref{eq:xnor-net rescale}, where $\circledast$ denotes 
% %  the binary convolution and $\odot$ denotes the element-wise multiplication. The binary convolution of $\mathcal{A}$ and $\mathcal{W}$ is rescaled by the weight scaling factor $\alpha$ and the activation scaling factor $\mathcal{K}$, both of which are calculated analytically.


% \zc{Similarly, you should explain the meaning of A, W and the operators $\circledast$ in the formula}
% Then in Real-to-binary Net \cite{martinez2020training}, Martinez et al. used a data-driven channel re-scaling module that takes the pre-convolution activations as input to predict the activation scaling factor. Unlike that in XNOR-Net++ \cite{bulat2019xnor}, these scaling factors are not fixed during inference but rather inferred from data. By doing this, they further improved the classification accuracy on ImageNet over XNOR-Net++. 
As is shown in Figure \ref{fig:pixel}, activation distributions have large pixel-to-pixel variation in SR networks
and the difference of activation magnitudes indicates different scaling factors are preferred for different pixels.
Inspired by \cite{martinez2020training}, we propose spatial re-scaling to better adapt the network to the spatial variation
of activation distributions in SR networks.
% fit the various pixel-wise distributions in SR networks.
We take the real-valued activations $A$ before convolution as input and predict pixel-wise scaling factors $S(A)$, which re-scale the binary convolution output. Spatial re-scaling process can be formulated as follows:
\begin{equation}
A * W \approx(\operatorname{sign}(A) \circledast \operatorname{sign}(W)) \odot \alpha \odot S(A)
\label{eq:spatial rescale}
\end{equation}
where $\circledast$ denotes 
the binary convolution and $\odot$ denotes the element-wise multiplication. $A$, $W$, $\alpha$, and $S\left(A\right)$ denote real-valued activations, weights, the scaling factor of weights, and the spatial-wise scaling factor of activations respectively. $S\left(A\right) \in \mathbb{R}^{1\times H\times W}$ can be calculated with a convolution and a sigmoid function.
% as $\sigma\left( CONV\left(A\right)\right)$. 
As shown in Figure \ref{fig:method}(a), real-valued activations first go through a convolution layer,
which has an input channel of $C$ and an output channel of 1, 
and then pass through a sigmoid function to produce the scaling factors $S(A)$ along the spatial dimension.
During inference, the scaling factor will change dynamically according to different input feature maps.
By re-scaling binary convolution output using $S(A)$, we can reduce the quantization error and the original pixel-wise information in FP activation
will be preserved much better.
Spatial re-scaling leads to a large PSNR improvement of 0.24 dB (from 30.30 dB to 31.54 dB) on Set5 and 0.22 dB (from 25.09 dB to 25.31 dB)
on Urban100 compared with our strong baseline. 

\subsection{Channel-wise Shifting and Re-scaling}

\begin{table}[!tb]
\centering
\caption{Comparison between whether to fuse channel-wise shifting and re-scaling or not based on our baseline with spatial re-scaling. }
\label{tab:fusing}

\scalebox{0.65}{
\begin{tabular}{c|cc|cc|cc}
\hline
\multirow{2}{*}{Method}     & \multirow{2}{*}{OPs} & \multirow{2}{*}{Params} & \multicolumn{2}{c|}{Set5} & \multicolumn{2}{c}{Urban100} \\ \cline{4-7} 
                            &                      &                         & PSNR        & SSIM        & PSNR          & SSIM         \\ \hline
Baseline + spatial re-scale & 2.16G                & 0.05M                   & 31.54       & 0.883       & 25.31         & 0.759        \\
+ channel-wise shift and re-scale             & 2.34G                & 0.09M                   & 31.61       & 0.885       & 25.35         & 0.761        \\
+ w/ fusing                   & 2.27G                & 0.08M                   & \textbf{31.64}       & \textbf{0.885}       & \textbf{25.36}         & \textbf{0.761}        \\ \hline
\end{tabular}
}
\end{table}

In SR networks, activation distributions exhibit larger channel-to-channel variation (Figure \ref{fig:chl}).
Both the mean and magnitude of the activation distributions vary significantly across channels.
% Thus we use channel-wise shifting and re-scaling to adapt to various channel-wise distributions. 
\cite{martinez2020training} has proposed the data-driven channel re-scaling, 
but our method differs from them in further introducing data-driven thresholds to handle the channel-wise variation of both mean and magnitude.
Since the blocks to generate the scaling factors and thresholds are very similar, we further propose to fuse them into one module.
% and fusing channel-wise shifting and re-scaling into one module.
We evaluate the effect of fusing the two blocks in Table \ref{tab:fusing}.
With channel-wise shifting and re-scaling fused, our models have fewer operations and parameters overhead and slightly higher performance.

For the specific process, we take the real-valued activations as input and predict different thresholds and scaling factors for each channel. They are also image dependent, e.g., $\beta_{i}$ in Eq.\ref{eq:act_binarize} is no longer fixed during inference but generated according to different input feature maps. Channel-wise shifting and re-scaling can be formulated as follows:
\begin{equation}
A * W \approx(\operatorname{sign}(A-C_s(A)) \circledast \operatorname{sign}(W)) \odot \alpha \odot C_r(A)
\label{eq:channel-wise_shift_and_rescale}
\end{equation}
where $\circledast$ denotes 
the binary convolution and $\odot$ denotes the element-wise multiplication. $C_s(A), C_r(A) \in \mathbb{R}^{C\times1\times1}$ denote the channel-wise threshold and scaling factor, respectively. 
We show the block diagram in Figure \ref{fig:method}(b).
The real-valued input feature map is first squeezed to a ${C\times1\times1}$ vector by a global average pooling (GAP) layer.
The subsequent fully connected layers and ReLU learn the channel-wise information and output a ${2C\times1\times1}$ vector.
Then the ${2C\times1\times1}$ vector is split into two ${C\times1\times1}$ vectors.
We use the first $C$ channels as the channel-wise bias and pass the last $C$ channels through a sigmoid layer 
as the channel-wise scaling factor, which are used to shift the real-valued activations and re-scale the binary convolution output, respectively. 


% \ml{We can mention previously, channel-wise re-scale has been proposed. We propose to fuse them. Add the comparison between fuse v.s. no fuse.}

\begin{figure}[!tbp]%
  \centering
    \includegraphics[width=0.4\textwidth]{fig/methods.png}
  
% \subfloat[channel-wise shifting\&re-scale]{
%     \label{subfig:channel-wise shifting and re-scale}
%     \includegraphics[width=0.2\textwidth]{fig/chl shift and rescale.png}
%   }

  \caption{Block diagram for spatial re-scaling, and channel-wise shifting and re-scaling.} 
  % Input A is the real-valued activation tensor and C, H, and W denote its dimension. GAP stands for global average pooling. The reduction ratio r is set to 16 for a better trade-off between the performance and the number of operations and parameters.}
  \label{fig:method}
\end{figure}


\subsection{Network Structure}

Combining the spatial re-scaling and the channel-wise shifting and re-scaling methods, we construct the enhanced convolution layer (E-Conv).
Then we build our EBSR model based on E-Conv.
In Figure \ref{fig:E-conv}, we compare the binary convolution layer used in the baseline network and our proposed E-Conv.
We use spatial and channel-wise scaling factors to re-scale the binary convolution output,
and use channel-wise shifting to learn appropriate thresholds for each channel before binarization.
The scaling factors and threshold used in E-Conv are learnable and depend on the real-valued input activations.
In this way, our proposed EBSR can adapt to pixel-to-pixel, channel-to-channel, and image-to-image variations
to reduce the large binarization error and preserve more details.
% In this way, our proposed E-Conv reduces the large quantization error caused by binarization and keeps the original information of input feature maps to a large extent.


\begin{figure}[!tb]%
  \centering

    \includegraphics[width=0.5\textwidth]{fig/E-conv.png}

  \caption{Comparison of (a) the binary convolution layer with a skip connection used in our baseline network and (b) the proposed E-Conv.}
  \label{fig:E-conv}
\end{figure}


Figure \ref{fig:network} shows the basic block based on the E-Conv and our EBSR composed of the basic blocks. Following existing works, the convolution layers in the head and tail modules are not binarized. We choose the lightweight EDSR which has 16 basic blocks and 64 channels, and EDSR which has 32 basic blocks and 256 channels as our backbones, which correspond to EBSR-light and EBSR, respectively.

\begin{figure}[!tb]%
  \centering
  {
    \includegraphics[width=0.35\textwidth]{fig/network.png}
  }
  
  \caption{The structure of our proposed EBSR.  Convolution layers in purple are real-valued vanilla 3x3 convolutions.}
  \label{fig:network}
\end{figure}
\subsection{Soundness and Completeness}
\label{s:sound-complete-lambda-gs}

In this section we show the main properties of the type system $\sysgs$ with respect to the operational semantics of the $\lam$-calculus with global state. The properties of type system $\sysgs$ are similar to the ones for $\syscbv$, but now with respect to configurations instead of terms. \emph{Soundness} does not only state that a (tightly) typable configuration $(t,s)$ is terminating, but also gives exact (and split) measures concerning the reduction sequence from $(t,s)$ to a final form. \emph{Completeness} guarantees that a terminating configuration $(t,s)$ is tightly typable, where the measures of the associated reduction sequence of $(t,s)$ to final form are reflected in the counters of the resulting type derivation of $(t,s)$. This is the first work providing a model for a language with global memory being able to count the number of memory accesses.

We start by noting that type system $\sysgs$ does not type blocked configurations, which is exactly the notion that we want to capture.

\begin{restatable}[]{proposition}{proptypedunblock}
    \label{prop:typed-unblock}
    If $\Phi \tr \seqi{\Gam}{(t,s)}{\ctype}{(b,m,d)}$, then $(t,s)$ is unblocked.
\end{restatable}

We also show that counters capture the notion of normal form correctly, both for terms and states.

\begin{restatable}[]{lemma}{lemzerocounters} \mbox{}
  \label{lem:zero-counters-size-store}
  \begin{enumerate}
    \item \label{lem:zero-counters} Let $\Phi \tr \seqi{\Gam}{t}{\del}{(0,0,d)}$ be tight. Then, (1) $t \in \normal$ and (2) $d = \size{t}$.
    \item \label{lem:zero-size-store} Let $\Phi \tr \seqi{\Del}{s}{\stype}{(0,0,d)}$ be tight. Then $d = 0$.
  \end{enumerate}
\end{restatable}

In fact, we can show the following stronger property with respect to the counters for the number of $\betav$- and $\getname/\setname$-steps.

\begin{restatable}[]{lemma}{lemzeronfs}
    \label{lem:zero-nfs}
    Let $\Phi \tr \seqi{\Gam}{t}{\del}{(b,m,d)}$ be tight. Then, $b = m = 0$ iff $t \in \normal$.
\end{restatable}

The following property is essential for tight type systems~\cite{Accattoli2020}, and it shows that tightness of types spreads throughout type derivations of neutral terms, just as long as the environments are tight. 

\begin{restatable}[{\bf Tight Spreading}]{lemma}{lemcomtightspreading}
    \label{lem:comp-tight-spreading}
    Let $\Phi \tr \seqi{\Gam}{t}{\tcomptype{\stype}{\tau}{\stype'}}{(b,m,d)}$, such that $\Gam$ is tight. If  $t \in \neutral$, then $\tau \in \tightt$.
\end{restatable}

The two following properties ensure tight typability of final configurations. For that we need to be able to \emph{tightly} type any state, as well as any normal form. In fact, normal forms do not depend on a particular state since they are irreducible, so we can type them with any state type.

\begin{restatable}[{\bf Typability of States and Normal Forms}]{lemma}{typstates} \mbox{}
  \label{lem:typestatesnfs}
  \begin{enumerate}
      \item \label{lem:typ-states} Let $s$ be a state. Then, there exists $\Phi\ \tr \seqi{}{s}{\stype}{(0,0,0)}$ tight.
      \item \label{lem:comp-typ-nfs} Let  $t \in \normal$. Then for any tight $\stype$ there exists $\Phi \tr \seqi{\Gam}{t}{\tcomptype{\stype}{\tightt}{\stype}}{(\cmiguel{b,m}{0,0},d)}$ tight s.t. $d = \size{t}$.
  \end{enumerate}
\end{restatable}

Finally, we state the usual basic properties.

\begin{restatable}[{\bf Substitution} and {\bf Anti-Substitution}]{lemma}{lemcompsubsantisubs} \mbox{}
    \label{lem:comp-subs-antisubs}    
    \begin{enumerate}
        \item {\bf (Substitution)} \label{lem:comp-subs} If $\Phi_t \tr \seqi{\Gam_t; x : \M}{t}{\del}{(b_t,m_t,d_t)}$ and $\Phi_v \tr \seqi{\Gam_v}{v}{\M}{(b_v,m_v,d_v)}$, then $\Phi_{t \subs{x}{v}} \tr \seqi{\Gam_t + \Gam_v}{t \subs{x}{v}}{\del}{(b_t+b_v,m_t+m_v,d_t+d_v)}$.
        \item {\bf (Anti-Substitution)} \label{lem:comp-antisubs} If $\Phi_{t \subs{x}{v}} \tr \seqi{\Gam_{t \subs{x}{v}}}{t \subs{x}{v}}{\del}{(b,m,d)}$, then $\Phi_t \tr \seqi{\Gam_t; x : \M}{t}{\del}{(b_t,m_t,d_t)}$ and $\Phi_v \tr \seqi{\Gam_v}{v}{\M}{(b_v,m_v,d_v)}$, such that $\Gam_{t \subs{x}{v}} = \Gam_t + \Gam_v$, $b = b_t+b_v$, $m = m_t+m_v$, and $d = d_t + d_v$.
    \end{enumerate}
\end{restatable}

\begin{restatable}[{\bf Split Exact Subject Reduction} and {\bf Expansion}]{lemma}{lemexactredexp}
  \label{lem-exact-red-exp} \mbox{}
  \begin{enumerate}
    \item {\bf (Subject Reduction)} \label{lem:subj-comp-red} Let $(t,s) \red[\gname] (u,q)$. If $\Phi \tr \seqi{\Gam}{(t,s)}{\ctype}{(b,m,d)}$ is tight, then $\Phi' \tr \seqi{\Gam}{(u,q)}{\ctype}{(b',m',d)}$, where $\gname =\beta$ implies $b' = b - 1$ and $m' = m$, while $\gname \in \{\getname, \setname\}$ implies $b'=b$ and  $m' = m - 1$.
    \item {\bf (Subject Expansion)} \label{lem:comp-subj-exp} Let $(t,s) \red[\gname] (u,q)$. If $\Phi' \tr \seqi{\Gam}{(u,q)}{\ctype}{(b',m',d)}$ is tight, then $\Phi \tr \seqi{\Gam}{(t,s)}{\ctype}{(b,m,d)}$, where $\gname =\beta$ implies $b' = b - 1$ and $m' = m$, while $\gname \in \{\getname, \setname\}$ implies $b'=b$ and  $m' = m - 1$.
  \end{enumerate}
\end{restatable}

Soundness (resp. completeness) is based on exact subject reduction (resp. expansion) respectively, in turn based on the previous substitution (resp. anti-substitution) lemma.

\begin{restatable}[{\bf Quantitative Soundness} and {\bf Completeness}]{theorem}{compsoundness} \mbox{}
    \begin{enumerate}
        \item {\bf (Soundness)} If $\Phi \tr \seqi{\Gam}{(t,s)}{\kap}{(b,m,d)}$ tight, then there exists $(u,q)$ such that $u \in \normal$ and $(t,s) \gsrred^{(b,m)} (u,q)$ with $b$ $\beta$-steps, $m$ $\getname/\setname$-steps, and $\size{(u,q)} = d$.
        \item {\bf (Completeness)} If $(t,s) \rra^{(b,m,d)} (u,q)$ and $u \in \normal$, then there exists $\Phi \tr \seqi{\Gam}{(t,s)}{\ctype}{(b,m,\size{(u,q)})}$ tight.
    \end{enumerate}
\end{restatable}

\begin{example}
  Consider again configuration $c_0$ from~\cref{ex:globalstate} and its associated tight derivation $\Phi_{c_0}$. The first two counters of $\Phi_c$ are different from $0$: this means that $c$ is not a final configuration, but normalizes in one $\betav$-step ($b = 1$) and two $\getname/\setname$-steps ($m = 2$), to a final configuration having size $d = 0 = \size{z} = \size{(z, \upd{l}{I}{\estate})}$.
\end{example}


%%%% experiments
\section{Experiments}
\label{sec:exp}

In this section, we demonstrate the wide range of applications and the high capabilities of Uni-Fusion. 
First, we evaluate Uni-Fusion in application 1) Incremental surface and color reconstruction, comparing its performance with SOTAs.
%
For applications 2) and 5), which are new topics, no specific benchmarks are available. 
Therefore, we showcase the performance on existing results.
%
Next, we implement application 3) and compare it with SOTA zero-shot semantic segmentation models.
%
Finally, for application 4), since infrared data is not commonly used, we collect our own dataset containing infrared values and show all applications on this data.

\subsection{Implementation Details}
\label{sec:exp:details}

In the experiments, we use our sample-based GPIS for local geometry encoding.
For each point, two additional points are sampled along normal direction, one positive and one negative, with distance $d_s=0.1$ in the local voxel's normalized space. 
Compared to derivative-based GPIS, our sample-based GPIS is more efficient in both space and time. 
For the encoder, we randomly sample $256$ anchor points from the range $[-0.5,0.5]^3$.
We utilize the first $20$ eigenpairs, resulting in a feature dimension of $20$.
The model selection process is discussed in the ablation study.

Different latent maps use different granularities.
For the surface LIM, we use a voxel size of $5\si{\centi\meter}$. 
For color which requires later comparison to NeRF, we use a voxel size of $2\si{\centi\meter}$.
For other property LIM and feature LIM, we use a voxel size of $10\si{\centi\meter}$.

For smooth reconstruction, the encoded voxel is designed overlapped following~\cite{huang2021di}.
The encoded voxel uses twice the voxel size, resulting in a half-space overlap with each neighboring voxel.
During meshing, SDFs are retrieved and interpolated from the overlapped voxels~\cite{huang2021di}.
While for the remaining properties, we sample only from its own voxel part.

The implementation runs on a PC with AMD Ryzen 9 5950X 16-core CPU and an Nvidia Geforce RTX 3090 GPU (24 GB).

\subsection{Datasets}

We evaluate incremental reconstruction on the ScanNet dataset~\cite{dai2017scannet}, TUM RGB-D dataset~\cite{sturm2012benchmark}, and Replica dataset~\cite{sucar2021imap}.
Using MSG-Net~\cite{zhang2018multi}'s material set, we transfer styles to the 3D canvas.
For open-vocabulary scene understanding, we evaluate on ScanNet segmentation data~\cite{qi2017pointnet++} and S3DIS dataset~\cite{armeni20163d}.

\subsubsection{ScanNet~\cite{dai2017scannet}}

ScanNet is a densely annotated RGB-D video dataset.
It is captured with the structure sensor~\cite{occipital} and contains 1513 scenes for training and validation.
For each scene, both images and a 3D mesh is provided, along with their 2D and 3D semantic annotations. 

ScanNet provides 312 scenes for validation, which contains a wide range of different room structures.
It has now been widely used in the thorough evaluation of the performance of reconstruction and semantic segmentation.

\subsubsection{TUM RGB-D~\cite{sturm2012benchmark}}

TUM RGB-D is a benchmark to mainly evaluate the tracking performance.
It is captured with Microsoft Kinect sensor together with ground-truth trajectory from the sensor.

\subsubsection{Replica~\cite{sucar2021imap}}

The Replica dataset refers to iMAP's pre-processed dataset~\cite{sucar2021imap}.
It is a synthetic rendered RGB-D dataset from given 3D models.
The advantage of including this dataset is that Replica does not have motion blur. 
This is better to evaluate the capability of the algorithms on reconstructing surface color.

\subsubsection{MSG-Net Style~\cite{zhang2018multi}}

MSG-Net provides material images for transfering the styles.
We select 21style fold for demonstration.
These images are given in \cref{fig:style} together with our result.

\subsubsection{ScanNet Point Cloud Segmentation Data~\cite{qi2017pointnet++}}

For point cloud semantic segmentation benchmarking, PointNet++~\cite{qi2017pointnet++} preprocesses the original ScanNet~\cite{dai2017scannet} and generates subsampled point clouds and corresponding annotations for each scene.

\subsubsection{S3DIS~\cite{armeni20163d} and 2D-3D-S~\cite{armeni2017joint}}

S3DIS is a semantic segmentation dataset for 3D point clouds.
Which is also a subset of the 2D-3D-S dataset.
The 2D-3D-S dataset is a multi-modality dataset containing 2D, 2.5D and 3D domains. 
This dataset is densely annotated with semantic labels.

Note that 2D-3D-S's 2D captures is not a RGB-D video as ScanNet.
2D-3D-S's images only have small overlap. 
Therefore, it is only suitable for semantic segmentation and not for incremental reconstruction.

\subsection{Baselines}

For online surface mapping evaluation, we select TSDF-Fusion~\cite{curless1996volumetric}, iMAP~\cite{sucar2021imap}, SOTA DI-Fusion~\cite{huang2021di} and BNV-Fusion~\cite{li2022bnv} as four baseline methods.

For the color field, we choose TSDF-Fusion~\cite{curless1996volumetric}, $\sigma$-Fusion~\cite{rosinol2023probabilistic}, iMAP~\cite{sucar2021imap}, NICE-SLAM~\cite{zhu2022nice} and even the recent hot Neural Radiance Fields model NeRF-SLAM~\cite{rosinol2022nerf} as five baselines.
While including NeRF in the comparison may not be entirely fair, we want to show how Uni-Fusion narrows the performance gap.

For the scene understanding application, we evaluate generalized zero-shot point cloud semantic segmentation with ZSLPC~\cite{cheraghian2019zero}, DeViSe~\cite{frome2013devise} and SOTA 3DGenZ~\cite{michele2021generative} for comparison.

\subsection{Metrics}

For incremental reconstruction, we evaluate the geometric reconstruction using \textbf{Accuracy}, \textbf{Completeness}, and \textbf{F1 score} according to SOTA BNV-Fusion. It firstly uniformly samples $100,000$ points from the reconstruction and ground truth meshes respectively.
Then \textbf{Accuracy} (\textbf{Completeness}) measures the percentage of reconstruction-to-groundtruth (groundtruth-to-reconstruction) distances that are lower than $2.5\si{\centi\meter}$ threshold. \textbf{F1 score} is the harmonic mean of accuracy and completeness.
For tracking performance, we use \textbf{ATE RMSE}.

To evaluate color reconstruction, we follow SOTA on this topic, NeRF to render both depth and RGB images to compare the image level \textbf{Depth L1} and \textbf{RGB PSNR}.

To compare scene understanding, we follow zero-shot point cloud semantic segmentation SOTA 3DGenZ to evaluate the \textbf{Intersection-of-Union (IoU)} and \textbf{Accuracy}.


\subsection{Reconstruction Results}

For evaluation, we first use the ScanNet validation set with 312 sequences to thoroughly test the geometric reconstruction on a large variant of scenes.
%
Then, we use TUM RGB-D to compare our modified tracking model with related works.
Because this part is not the main contribution, we give a rough overview of the tracking results.
%
To fairly evaluate the color reconstruction, we use the high quality rendered Replica dataset to compare with related works, including NeRF.

%\subsubsection{Object}
% on instance-gp
% Objective data usually has more fine detail
% 1. for detail precision
% A: no, object reconstruction is not as good as instance-ngp, so cancelled.
\begin{table*}[!]
	\centering
	\caption{Comparison to ScanNet~\cite{dai2017scannet}.
       Our method generalizes better to various scenes.
       $^*$ indicates the result from our runs of the official BNV-Fusion code.}
	\small
	%\setlength{\tabcolsep}{5mm}
	\setlength{\tabcolsep}{0.9em}
		%\resizebox{\textwidth}{!}{
		\begin{tabular}{l  c c c| c c c }
			\toprule
			Method & \begin{tabular}{@{}c@{}}Pre-Train\\ with extra dataset\end{tabular} & \begin{tabular}{@{}c@{}}Train \\ with sequences\end{tabular} & Real-time & Accuracy (\%) $\uparrow$ & Completeness (\%) $\uparrow$ & F1 score $\uparrow$ \\
			\midrule
			TSDF Fusion~\cite{zhou2018open3d} & None & None & $\checkmark$ &73.83 & 85.85 & 78.84 \\
			iMAP~\cite{sucar2021imap} & None & Online train& &68.96 & 82.12 & 74.96 \\
			DI-Fusion~\cite{huang2021di} &Object Pretrain & None & $\checkmark$&66.34 & 79.65 & 72.97 \\
			BNV-Fusion~\cite{li2022bnv} &Object Pretrain &  Post Optimization& &{74.90} & \textbf{88.12} & {80.56} \\
			BNV-Fusion$^{*}$~\cite{li2022bnv} &Object Pretrain & Post Optimization &&{73.42} & {81.75} & {77.18} \\
			\textbf{Uni-Fusion (Ours)} &None &None &$\checkmark$&\textbf{80.43} & {84.91} & \textbf{82.44} \\
			\bottomrule
		\end{tabular}
	  %}
	\label{tab:scannet}
	\vspace{-.6cm}
\end{table*}
\begin{figure*}[t]
	\subfloat[width=.33\textwidth][Accuracy]{
		\centering
		\includegraphics[width=.22\linewidth]{im/exp/recons/scannet/scannet_acc.png}
		\includegraphics[width=.1\linewidth]{im/exp/recons/scannet/scannet_acc_box.png}
	}
	\subfloat[width=.33\textwidth][Completeness]{
		\centering
		\includegraphics[width=.22\linewidth]{im/exp/recons/scannet/scannet_comp.png}
		\includegraphics[width=.1\linewidth]{im/exp/recons/scannet/scannet_comp_box.png}
	}
	\subfloat[width=.33\textwidth][F1 score]{
		\centering
		\includegraphics[width=.22\linewidth]{im/exp/recons/scannet/scannet_F1.png}
		\includegraphics[width=.1\linewidth]{im/exp/recons/scannet/scannet_F1_box.png}
	}
	\label{fig:recon:scannet:elementwise}
	\caption{Quantitative comparison on 312 scenes of the ScanNet validation set.
       We demonstrate the performance of SOTA BNV-Fusion and our Uni-Fusion.
       We sort our evaluation value and reordered all of the scores.
       The zigzag pink is the BNV-Fusion result;
       we also plot a deep-pink smoothed curve for better visualization.}
\end{figure*}

\subsubsection{Evaluation on ScanNet Dataset~\cite{dai2017scannet}}
\label{sec:exp:scannet}

We use the 312 diversified scenes from the ScanNet validation set to evaluate the performance of surface reconstruction. 
We follow the pure mapping SOTA BNV-Fusion to take every 10th posed frame as input. 
%
Without using any learning (in contrast iMAP, DI-Fusion, and BNV-Fusion do) or any post optimization (as BNV-Fusion does), our Uni-Fusion is capable to achieve precise continuous mapping performance. 

As shown in~\cref{tab:scannet}, our Uni-Fusion achieves \textbf{$+6$ higher accuracy} than the incremental surface reconstruction SOTA BNV-Fusion.
Our model does not exceed on completeness comparing to BNV-Fusion that support completion in post-optimization.
Though, Uni-Fusion's completion is still much higher than one other optimization based model iMAP.
%We consider it because our model does not support hole-completion as the optimization based models iMap and BNV-Fusion.
Overall, our Uni-Fusion model achieves higher F1-scores that quantifies the overall quality.

Please note that, SOTA BNV-Fusion is not real-time capable, since it requires post optimization of all fed frames.
Without the post-optimization, the real-time model Di-Fusion shows much worse results.
However, our \textbf{real-time} model \textbf{Uni-Fusion} is able to achieves \textbf{much better} reconstruction quality than these approaches even without post-optimization. 

\newcommand{\scannetImSize}{.16}
\begin{figure*}[t!]
	\centering
	\setlength{\tabcolsep}{0.1em}
	\renewcommand{\arraystretch}{.1}
	\begin{tabular}{|c | c |c |||c |c | c|}
		\hline
		{\Large{BNV-Fusion}} & {\Large{Uni-Fusion}} &{\Large{Ground Truth}} & {\Large{BNV-Fusion}} &{\Large{Uni-Fusion}} & {\Large{Ground Truth}} \\ \hline \hline
		
\includegraphics[width=\scannetImSize\linewidth]{im/exp/recons/scannet_qualifi/scene0568_00_bnv.png}
		&\includegraphics[width=\scannetImSize\linewidth]{im/exp/recons/scannet_qualifi/scene0568_00_mine.png}
		&\includegraphics[width=\scannetImSize\linewidth]{im/exp/recons/scannet_qualifi/scene0568_00_gt.png}
		&		\includegraphics[width=\scannetImSize\linewidth]{im/exp/recons/scannet_qualifi/scene0164_00_bnv.png}
		&\includegraphics[width=\scannetImSize\linewidth]{im/exp/recons/scannet_qualifi/scene0164_00_mine.png}
		&\includegraphics[width=\scannetImSize\linewidth]{im/exp/recons/scannet_qualifi/scene0164_00_gt.png}\\
		
\includegraphics[width=\scannetImSize\linewidth]{im/exp/recons/scannet_qualifi/scene0249_00_bnv.png}
		&\includegraphics[width=\scannetImSize\linewidth]{im/exp/recons/scannet_qualifi/scene0249_00_mine.png}
		&\includegraphics[width=\scannetImSize\linewidth]{im/exp/recons/scannet_qualifi/scene0249_00_gt.png}
		&		\includegraphics[width=\scannetImSize\linewidth]{im/exp/recons/scannet_qualifi/scene0435_00_bnv.png}
		&\includegraphics[width=\scannetImSize\linewidth]{im/exp/recons/scannet_qualifi/scene0435_00_mine.png}
		&\includegraphics[width=\scannetImSize\linewidth]{im/exp/recons/scannet_qualifi/scene0435_00_gt.png}\\
		
\includegraphics[width=\scannetImSize\linewidth]{im/exp/recons/scannet_qualifi/scene0046_00_bnv.png}
		&\includegraphics[width=\scannetImSize\linewidth]{im/exp/recons/scannet_qualifi/scene0046_00_mine.png}
		&\includegraphics[width=\scannetImSize\linewidth]{im/exp/recons/scannet_qualifi/scene0046_00_gt.png}
		&		\includegraphics[width=\scannetImSize\linewidth]{im/exp/recons/scannet_qualifi/scene0050_00_bnv.png}
		&\includegraphics[width=\scannetImSize\linewidth]{im/exp/recons/scannet_qualifi/scene0050_00_mine.png}
		&\includegraphics[width=\scannetImSize\linewidth]{im/exp/recons/scannet_qualifi/scene0050_00_gt.png}\\
		\hline
	\end{tabular}
	%\captionof{figure}
	\caption{Surface reconstruction on ScanNet dataset.}
	\label{fig:recons:scannet_demo}
	\vspace{-.5cm}
\end{figure*}

We additionally run BNV-Fusion's official implementation (emphasized with $^*$) on the 312 videos of ScanNet and do a post element-wise comparison in \cref{fig:recon:scannet:elementwise}. 
Our result is the {\color{Cyan}light blue} curve, BNV-Fusion's result is colored with {\color{Lavender}pink}.
Scene index is sorted corresponding to the score value of Uni-Fusion.
For better visualization, we smooth BNV-Fusion's curve and plot it with dark pink.
It is obvious that the score of Uni-Fusion is overall higher than BNV-Fusion's. 
Moreover, we use box-plot to conclude the statistics besides the curve plot. Uni-Fusion's scores are distributed in a higher region. For completeness which is less obvious better, Uni-Fusion's box is smaller while in a relative higher position. This means that Uni-Fusion has more stable completeness result while BNV-Fusion is more likely to get low completeness in some cases.

To summarize, our model is almost better on all 312 scenes on all accuracy, completeness and F1-score.
Which is also revealed in \cref{tab:scannet} with BNV-Fusion$^*$, that the BNV-Fusion's official implementation does not exceed Uni-Fusion on all metrics.

We plot reconstruction on selected scenes from ScanNet in~\cref{fig:recons:scannet_demo}. 
Both BNV-Fusion and our Uni-Fusion are able to produce high quality reconstruction.
We see that BNV-Fusion gives lots of small meshes on walls, which are shown as small particles in the reconstruction. 
We consider it is because BNV-Fusion use very small voxel size ($0.02\si{\meter}$) to get a high score.
This is also revealed by their \textbf{\SI{247}{MB}} mesh in average, while ours is only \textbf{\SI{54}{Mb}} in average.
Furthermore, our Uni-Fusion's mesh is more smooth and
%Both BNV-Fusion and Uni-Fusion demonstrate high quality result.
also provides high-precise color to the mesh which is not available for the Surface SOTA.

%In this test, we purely evaluate the surface reconstruction capacity with SOTAs. 
%While reconstruction is not merely surface.  
%Thus in the following, we find benchmarks for both surface and color.


\subsubsection{Tracking Evaluation on TUM RGB-D Dataset~\cite{sturm2012benchmark}}
% follow nice-slam

In the above test, we compare the performance of pure mapping.
While tracking is not the contribution focus in our paper, it is part of the reconstruction model. We follow the novel reconstruction model NICE-SLAM~\cite{zhu2022nice} to evaluate the camera tracking on the small-scale TUM RGB-D dataset.
Our Uni-Fusion uses a coarse-to-fine strategy for 3D reconstruction tracking.
From~\cref{tab:tum_rmse}, it demonstrates overall better ATE RMSE than other implicit representation models.

\begin{table}[]
		\caption{Tracking on TUM RGB-D~\cite{sturm2012benchmark}.
		ATE RMSE [$\si{\centi\meter}$] ($\downarrow$) is used as the evaluation metric.
	}
	\centering
	\footnotesize
	\setlength{\tabcolsep}{0.7em}
	\resizebox{\linewidth}{!}{
		\begin{tabular}{l|ccc}
			\hline
			& \tt{fr1/desk} &  \tt{fr2/xyz} &  \tt{fr3/office} \\
			
			\hline
			{iMAP}~\cite{sucar2021imap}      & 4.9 & 2.0 & 5.8  \\
			{iMAP$^*$}~\cite{sucar2021imap} & 7.2 & 2.1  & 9.0 \\
			{DI-Fusion~\cite{huang2021di}} & 4.4 & 2.3 & 15.6 \\
			NICE-SLAM~\cite{zhu2022nice}           & 2.7 & 1.8 & 3.0 \\
			Ours& 1.8& 0.5& 2.1 \\
			\hline
			{BAD-SLAM}\cite{schops2019bad} & 1.7  & 1.1  & 1.7 \\
			{Kintinuous}\cite{whelan2012kintinuous} & 3.7  &  2.9  & 3.0 \\
			{ORB-SLAM2}\cite{mur2017orb} & \bf 1.6  & \bf 0.4  & \bf 1.0 \\
			\hline
	\end{tabular}}
	\vspace{2pt}

	\label{tab:tum_rmse}
\end{table}

On the other hand, there also exist high accuracy algorithms from SLAM. 
By additional using Bundle Adjustment and Loop-closing techniques, their tracking quality is much better than all of the implicit based models.

%But it is dangerous to directly apply SLAM result on reconstruction. Please find our demonstration in Fig [?]. Which explains the more widely used frame-to-model strategy in 3D reconstruction.
Even though, our coarse-to-fine strategy firstly ensure not easy to lose track. Secondly, it is more suitable for surface fitting.

Which further support our test in Replica dataset.



\begin{table*}[t!]
	\centering
	\caption{Geometric (L1) and Photometric (PSNR) evaluation on the Replica dataset~\cite{sucar2021imap}.}
	\footnotesize
	\setlength{\tabcolsep}{0.36em}
	\renewcommand{\arraystretch}{1.2}
	\begin{tabular}{clcccccccccccccccccc}
		\toprule
		& & \multicolumn{1}{c}{\makecell{\tt{office-0}}} & \multicolumn{1}{c}{\makecell{\tt{office-1}}} & \multicolumn{1}{c}{\makecell{\tt{office-2}}}& \multicolumn{1}{c}{\makecell{\tt{office-3}}} & \multicolumn{1}{c}{\makecell{\tt{office-4}}} & \multicolumn{1}{c}{\makecell{\tt{room-0}}} & \multicolumn{1}{c}{\makecell{\tt{room-1}}} &  \multicolumn{1}{c}{\makecell{\tt{room-2}}} & Avg. \\
		\midrule
		\multicolumn{5}{l}{\textit{Non-continuous mapping method}}\\
		\multirow{2}{*}{\makecell{\textbf{TSDF-Fusion}~\cite{curless1996volumetric}}}
		& {\bf Depth L1} [$\si{\centi\meter}$] $\downarrow$
	 & 14.11 & 10.50 & 30.89 & 28.92 & 22.83	& 23.51 & 20.94 & 23.34 & 21.88 \\
		& {\bf PSNR } [$\si{\dB}$] $\uparrow$
		& 11.16 & 15.92 & 4.86 & 5.68 & 5.46 & 3.43 & 4.51 & 5.57 & 7.07 \\
		
		\midrule
		\multirow{2}{*}{\makecell{\textbf{$\sigma$-Fusion}\cite{rosinol2023probabilistic} }}
		& {\bf Depth L1} [$\si{\centi\meter}$] $\downarrow$
		 & 13.80 & 10.21 & 22.27 & 28.70 & 22.21& 21.92 & 19.28 & 22.40 & 20.10 \\
		& {\bf PSNR } [$\si{\dB}$] $\uparrow$
		 & 11.16 & 15.92 & 4.86 & 5.69 & 5.46& 3.45  & 4.51 & 5.57 & 7.08 \\
		
		
		
		
		
		\midrule
		\midrule
		\multicolumn{5}{l}{\textit{Continuous mapping method}}\\
		\multirow{2}{*}{\makecell{\textbf{iMAP$^*$}~\cite{sucar2021imap}}}
		& {\bf Depth L1} [$\si{\centi\meter}$] $\downarrow$
		 & 6.43 & 7.41 & 14.23 & 8.68 & 6.80& 5.70 & 4.93 & 6.94 & 7.64\\
		& {\bf PSNR } [$\si{\dB}$] $\uparrow$
		& 7.39 & 11.89 & 8.12 & 5.62 & 5.98& 5.66 & 5.31 & 5.64  & 6.95\\
		\midrule
		\multirow{2}{*}{{\makecell{\textbf{Nice-SLAM}~\cite{zhu2022nice} }}}
		& {\bf Depth L1} [$\si{\centi\meter}$] $\downarrow$
		& { 1.51 } & { 0.93 } & { 8.41 } & { 10.48 } & {2.43} & { 2.53 } & { 3.45 } & { 2.93 }  & { 4.08 } \\
		& {\bf PSNR } [$\si{\dB}$] $\uparrow$
		 & { 22.44 } & { 25.22 } & { 22.79 } & { 22.94 } & { 24.72 } & \textbf{ 29.90 } & \textbf{ 29.12 } & { 19.80 }& { 24.61 } \\
		
		
		
		
		\midrule	
		\multirow{2}{*}{{\makecell{\textbf{Uni-Fusion} (Ours) }}}
		% using abs(diff)
		%	& {\bf Depth L1} [$\si{\centi\meter}$] $\downarrow$ &\textbf{1.98}&\textbf{1.18}&\textbf{1.64}&\textbf{1.23}&\textbf{0.84}&\textbf{1.61}&\textbf{3.01}&\textbf{1.60} &\textbf{1.64}
		% follow nerf-slam to remove outlier gt first
		& {\bf Depth L1} [$\si{\centi\meter}$] $\downarrow$
		& \textbf{0.79}&\textbf{0.56}&\textbf{1.59}&\textbf{2.71}&\textbf{1.66}&\textbf{1.94}&\textbf{0.69}&\textbf{1.80}& \textbf{1.47}
		\\
		& {\bf PSNR } [$\si{\dB}$] $\uparrow$ &\textbf{33.88}&\textbf{33.31}&\textbf{25.84}&\textbf{26.01}&\textbf{28.14}&24.02&26.20&\textbf{27.17} &\textbf{28.07}
		\\
		
		\midrule
		\midrule
		\multicolumn{5}{l}{\textit{Neural radiance field method}}\\
		\multirow{2}{*}{{\makecell{\textbf{NeRF-SLAM}~\cite{rosinol2022nerf} }}}
		& {\bf Depth L1} [$\si{\centi\meter}$] $\downarrow$
	 & {2.49}   & {1.98}  & {9.13}  & {10.58} & {3.59}	& {2.97}  & {2.63}  & {2.58}  & {4.49} \\
		& {\bf PSNR } [$\si{\dB}$] $\uparrow$
	 & \textbf{48.07}  & \textbf{53.44} & \textbf{39.30} & \textbf{38.63} & \textbf{39.21} 	& \textbf{34.90} & \textbf{36.95} & \textbf{40.75}& \textbf{41.40} \\
		
		\bottomrule
	\end{tabular}%
	
	\label{tab:replica_per_scene}
\end{table*}


\begin{table*}[t!]
	\centering
	\caption{Differences among different Surface \& Color reconstruction models.}
	\small
	\setlength{\tabcolsep}{.6em}
	%{
		%\resizebox{\textwidth}{!}{
			\begin{tabular}{l | c c c c c c }
				\toprule
				Method & 
				\begin{tabular}{@{}c@{}}Pre-Train\\ with extra dataset\end{tabular}
				& \begin{tabular}{@{}c@{}}Train\\ with sequences\end{tabular}
				& Real-time	
				& Direct Output &  \begin{tabular}{@{}c@{}}Light\\ direction\end{tabular} 
				&Render\\
				\hline 
				TSDF-Fusion & None & None & $\checkmark$& Discrete TSDF &  &Ray Rasterization\\\hline
				$\sigma$-Fusion & None & None &$\checkmark$&Discrete TSDF  && Ray Rasterization\\\hline
				iMAP & None & Online Train && MLPs  & &Volumetric Rendering\\\hline
				NICE-SLAM & \begin{tabular}{@{}c@{}}Pretrain\\ with indoor dataset\end{tabular} & Online Train&& Neural Implicit Grid&  & Volumetric Rendering\\\hline
				
				NeRF-SLAM & None & Train hundred epochs &-&NeRF & $\checkmark$ &Volumetric Rendering \\\hline
				
				\textbf{Uni-Fusion} & None & None&$\checkmark$& Latent Implicit Map && Ray Rasterization\\				
				\hline
			\end{tabular}
		%}
	%}
	\label{tab:replica_diff}
\end{table*}
\subsubsection{Evaluation on Replica RGB-D Dataset~\cite{sucar2021imap}}
In this evaluation, we compare with implicit reconstruction (TSDF-Fusion, $\sigma$-Fusion) and latent implicit reconstruction models (iMAP, NICE-SLAM) that support colors. 
We also add a large-scale NeRF model, NeRF-SLAM in to the table.  
NeRF is SOTA in view-synthesis task, which is unfair to direct compare with the rest. As the rest model does not even model light directions.
We add NeRF in this part to demonstrate that Uni-Fusion strongly reduce the gap.
Note that, NeRF-SLAM embeds external tracking model ~\cite{teed2021droid} to provide poses while using SOTA NeRF implementation Instance-ngp~\cite{muller2022instant} for NeRF construction.
%Therefore it is considered the SOTA to model the colors.

Uni-Fusion track and follow our previous setting in ScanNet test to take every 10 frames for mapping.
NICE-SLAM and NeRF-SLAM produce depth and color by rendering,
To obtain result from Uni-Fusion, we cast rays from virtual camera to our result surface mesh for depth image. 
Then Uni-Fusion infer the cast points in Uni-Fusion's color LIM for color result.

From~\cref{tab:replica_per_scene}, Uni-Fusion demonstrate
best Depth L1 on all scenes with an average of \textbf{$\pmb{1.47}$$\si{\centi\meter}$ depth L1}. Which is \textbf{$\pmb{177\%}$ boost} comparing to the second best.

Moreover, excluding NeRF, our Uni-Fusion also shows the best performance to model the colors with an average of $28.07$$\si{\dB}$ PSNR.

However, it is strange that NICE-SLAM lost details while in two cases, it shows better PSNR than Uni-Fusion. 
To highlight the true result,
we plot the rendered image in \cref{fig:replica_render}.
It is obvious that our Uni-Fusion models the details of painting, carpet and quilt well. 
While NICE-SLAM just roughly models the average color.

Moreover, from the  \cref{fig:replica_render}, our Uni-Fusion's rendering quality is as precise as NeRF. 
Please also find the painting, carpet and quilt, Uni-Fusion recovered the original appearance well.
Please find the {\color{green} green window} for the emphasized region.
Uni-Fusion reproduce the high-quality appearance which is very close to NeRF on qualitative evaluation.
%It can hardly find difference between the results from NeRF-SLAM, Uni-Fusion and Ground Truth.
%
But, Uni-Fusion still has a quantitative score gap to the color rending of NeRF ($41.4$$\si{\dB}$).
Though the Uni-Fusion's rendering result is highly close to NeRF and ground truth.
%
We consider the main reasons are that \textbf{1.} Uni-Fusion does not model the light directions to points, which is essential to NeRF.
\textbf{2.} NeRF optimizes on the rendering image quality by focussing mainly on color while less on depth.
It can be revealed by the higher color rendering score with much worse depth rendering than our Uni-Fusion.
\textbf{3.} our Uni-Fusion does not support hole filling.
This directly leads to black holes in our rendered images.

We summarize the differences in \cref{tab:replica_diff}.
Similar to TSDF-Fusion and $\sigma$-Fusion, our Uni-Fusion is a forward method which, does not need any training, i.e., pre- or online training. 
Uni-Fusion also produces similar to NICE-SLAM and NeRF-SLAM an implicit map with set of latent that outputs results at arbitrary resolution.
However, we differ on the extracting of the signed distance field.
%FIXME: I do not understand the next sentence.
Uni-Fusion's latent feature rule its own region independently.
Each query value is directly inferred with the corresponding ruling latent.
While NICE-SLAM and NeRF-SLAM use a much denser grid to interpolate query features. This requires volumetric rendering for inference.

Similar to TSDF-Fusion, $\sigma$-Fusion, our Uni-Fusion is also a real-time algorithm.
iMAP, NICE-SLAM and NeRF-SLAM run hardly in real-time.
NeRF-SLAM is claiming to be real-time, which is questionable as they still need hundreds of epochs training after feeding the data.

Nevertheless, optimization with backpropagation learns pixel-to-pixel well.
It is theoretically advanced for a regression-and-fusion strategy. 
Though Uni-Fusion demonstrates its high capability to model the color, NeRF-like post-optimization would still be a good direction for further improvements of Uni-Fusion.

\newcommand{\replicaImSize}{.24}
\begin{figure*}[t]
	\centering
	\setlength{\tabcolsep}{0.1em}
	\renewcommand{\arraystretch}{.1}
	\begin{tabular}{|c | c |c |c| }
		 \hline
		{\Large{NICE-SLAM}} &{\Large{NeRF-SLAM}}&\textbf{\Large{Uni-Fusion}}&\Large{Ground Truth}\\
		%		\hline
		%		\includegraphics[width=\replicaImSize\linewidth]{im/exp/recons/replica/nice-slam/of2_1286.png} &
		%		\includegraphics[width=\replicaImSize\linewidth]{im/exp/recons/replica/mine/of2_1286.jpg} &
		%		\includegraphics[width=\replicaImSize\linewidth]{im/exp/recons/replica/mine/of2_1286.jpg} &
		%		\includegraphics[width=\replicaImSize\linewidth]{im/exp/recons/replica/gt/of2_1286.jpg} \\
		
		\hline
		\includegraphics[width=\replicaImSize\linewidth]{im/exp/recons/replica/nice-slam/rm0_769_window.png} &
		\includegraphics[width=\replicaImSize\linewidth]{im/exp/recons/replica/nerf-slam/rm0_769_window.jpg} &
		\includegraphics[width=\replicaImSize\linewidth]{im/exp/recons/replica/mine/rm0_769_window.jpg} &
		\includegraphics[width=\replicaImSize\linewidth]{im/exp/recons/replica/gt/rm0_769_window.jpg} \\
		\hline
		\includegraphics[width=\replicaImSize\linewidth]{im/exp/recons/replica/nice-slam/of3_575_window.png} &
		\includegraphics[width=\replicaImSize\linewidth]{im/exp/recons/replica/nerf-slam/of3_575_window.jpg} &
		\includegraphics[width=\replicaImSize\linewidth]{im/exp/recons/replica/mine/of3_575_window.jpg} &
		\includegraphics[width=\replicaImSize\linewidth]{im/exp/recons/replica/gt/of3_575_window.jpg} \\
		\hline
		\includegraphics[width=\replicaImSize\linewidth]{im/exp/recons/replica/nice-slam/rm1_425_window.png} &
		\includegraphics[width=\replicaImSize\linewidth]{im/exp/recons/replica/nerf-slam/rm1_425_window.jpg} &
		\includegraphics[width=\replicaImSize\linewidth]{im/exp/recons/replica/mine/rm1_425_window.jpg} &
		\includegraphics[width=\replicaImSize\linewidth]{im/exp/recons/replica/gt/rm1_425_window.jpg} \\
		\hline
		\includegraphics[width=\replicaImSize\linewidth]{im/exp/recons/replica/nice-slam/rm2_1085_window.png} &
		\includegraphics[width=\replicaImSize\linewidth]{im/exp/recons/replica/nerf-slam/rm2_1085_window.jpg} &
		\includegraphics[width=\replicaImSize\linewidth]{im/exp/recons/replica/mine/rm2_1085_window.jpg} &
		\includegraphics[width=\replicaImSize\linewidth]{im/exp/recons/replica/gt/rm2_1085_window.jpg} \\		
		
	\end{tabular}
	%\captionof{figure}
	\caption{Demonstration of color rendering on the Replica dataset. Fine appearances are highlighted in {\color{green}green window}. Small flaws are in a {\color{red}red} box.}
	\label{fig:replica_render}
	\vspace{-.5cm}
\end{figure*}

%(2) NeRF model learning radiance field that model the light on different direction on surface. While Uni-Fusion naturally treat different directional light the same color.

%\begin{table*}[t!]
%	\centering
%	\setlength{\tabcolsep}{0.1em}
%	\renewcommand{\arraystretch}{.1}
%	\begin{tabular}{c | c |c |c |c }
%		\hline 
%		\rotatebox{90}{\large{NICE-SLAM}} &
%		\includegraphics[width=\replicaImSize\linewidth]{im/exp/recons/replica/nice-slam/of3_575.png} &
%		\includegraphics[width=\replicaImSize\linewidth]{im/exp/recons/replica/nice-slam/rm0_769.png} &
%		\includegraphics[width=\replicaImSize\linewidth]{im/exp/recons/replica/nice-slam/rm1_425.png} &
%		\includegraphics[width=\replicaImSize\linewidth]{im/exp/recons/replica/nice-slam/rm2_1085.png} \\
%		\hline
%		\rotatebox{90}{\large{NeRF-SLAM}} &
%		\includegraphics[width=\replicaImSize\linewidth]{im/exp/recons/replica/mine/of3_575.jpg} &
%		\includegraphics[width=\replicaImSize\linewidth]{im/exp/recons/replica/mine/rm0_769.jpg} &
%		\includegraphics[width=\replicaImSize\linewidth]{im/exp/recons/replica/mine/rm1_425.jpg} &
%		\includegraphics[width=\replicaImSize\linewidth]{im/exp/recons/replica/mine/rm2_1085.jpg} \\	
%		\hline
%		\rotatebox{90}{\textbf{\Large{Uni-Fusion}}} &
%		\includegraphics[width=\replicaImSize\linewidth]{im/exp/recons/replica/mine/of3_575.jpg} &
%		\includegraphics[width=\replicaImSize\linewidth]{im/exp/recons/replica/mine/rm0_769.jpg} &
%		\includegraphics[width=\replicaImSize\linewidth]{im/exp/recons/replica/mine/rm1_425.jpg} &
%		\includegraphics[width=\replicaImSize\linewidth]{im/exp/recons/replica/mine/rm2_1085.jpg} \\
%		\hline
%		\rotatebox{90}{\large{Ground Truth}} &
%		\includegraphics[width=\replicaImSize\linewidth]{im/exp/recons/replica/gt/of3_575.jpg} &
%		\includegraphics[width=\replicaImSize\linewidth]{im/exp/recons/replica/gt/rm0_769.jpg} &
%		\includegraphics[width=\replicaImSize\linewidth]{im/exp/recons/replica/gt/rm1_425.jpg} &
%		\includegraphics[width=\replicaImSize\linewidth]{im/exp/recons/replica/gt/rm2_1085.jpg} \\
%		\hline		
%		
%	\end{tabular}
%	\captionof{figure}{Demonstration of color rendering on Replica dataset.}
%\end{table*}

\subsection{Ablation study}
\label{exp:surface:ablation}

\begin{figure}[]
	\centering
%		\subfloat[width=\textwidth][Sample based]{
%		\centering
%		\includegraphics[width=.7\linewidth]{im/exp/ablation/GPIS/seq3_sample_color.png}
%	}\\
%	\subfloat[width=\textwidth][Derivative based]{
%		\centering
%		\includegraphics[width=.7\linewidth]{im/exp/ablation/GPIS/seq3_derivative_color.png}
%	}
		\includegraphics[width=.7\linewidth]{im/exp/ablation/GPIS/seq3_sample_color_a.png}
		\includegraphics[width=.7\linewidth]{im/exp/ablation/GPIS/seq3_derivative_color_b.png}
	\caption{Ablation study on surface construction basis. (a) Sample based. (b) Derivative based.}
	\label{fig:ablation:GPIS}
\end{figure}


\begin{table}[]
	\caption{Ablation study on tracking.
	}
	\centering
	\footnotesize
	\setlength{\tabcolsep}{0.7em}
	\resizebox{\linewidth}{!}{
		\begin{tabular}{l|ccc}
			\hline
			& \tt{fr1/desk} &  \tt{fr2/xyz} &  \tt{fr3/office} \\
			\hline
			External& 2.1& 0.5& 2.5 \\
			External+Internal&1.8& 0.5& 2.1 \\
			\hline
	\end{tabular}}
	\vspace{-2pt}
	%\vspace{-1cm}
	\label{tab:tum_rmse2}
\end{table}

\begin{figure}
	\centering
	\includegraphics[width=.7\linewidth]{im/exp/ablation/voxel_size/seq_voxel_size.png}
	%	\subfloat[width=.33\textwidth][0.1]{
		%		\centering
		%		\includegraphics[width=.33\linewidth]{im/exp/ablation/voxel_size/seq3_0_1_color.png}
		%	}
	%	\subfloat[width=.3\textwidth][0.05]{
		%		\centering
		%		\includegraphics[width=.33\linewidth]{im/exp/ablation/GPIS/seq3_sample_color.png}
		%	}
	%	\subfloat[width=.33\textwidth][0.02]{
		%	\centering
		%	\includegraphics[width=.33\linewidth]{im/exp/ablation/voxel_size/seq3_0_02_color.png}
		%	}
	\caption{Ablation study on voxel size.}
	\label{fig:ablation:voxel_size}
\end{figure}




 \newcommand{\styleImSize}{.2}
\begin{figure*}[b!]
	\vspace{-.5cm}
	\centering
	\setlength{\tabcolsep}{0.1em}
	\renewcommand{\arraystretch}{.1}
	\resizebox{\textwidth}{!}{\begin{tabular}{ccccc}
			%		\includegraphics[width=\styleImSize\line]{im/exp/style/style/0} &
			%		\includegraphics[width=\styleImSize\linewidth]{im/exp/style/style/1} &
			%		\includegraphics[width=\styleImSize\linewidth]{im/exp/style/style/2} &
			%		\includegraphics[width=\styleImSize\linewidth]{im/exp/style/style/3} &
			%		\includegraphics[width=\styleImSize\linewidth]{im/exp/style/style/4} &
			%		\includegraphics[width=\styleImSize\linewidth]{im/exp/style/style/5} &
			%		\includegraphics[width=\styleImSize\linewidth]{im/exp/style/style/6} \\
			\hline\hline
			\includegraphics[width=\styleImSize\linewidth]{im/exp/style/processed/office0_0.png} &
			\includegraphics[width=\styleImSize\linewidth]{im/exp/style/processed/office0_1.png} &
			\includegraphics[width=\styleImSize\linewidth]{im/exp/style/processed/office0_2.png} &
			\includegraphics[width=\styleImSize\linewidth]{im/exp/style/processed/office0_3.png} &
			\includegraphics[width=\styleImSize\linewidth]{im/exp/style/processed/office0_4.png} \\
			\includegraphics[width=\styleImSize\linewidth]{im/exp/style/processed/office0_5.png} &
			\includegraphics[width=\styleImSize\linewidth]{im/exp/style/processed/office0_6.png} &
			\includegraphics[width=\styleImSize\linewidth]{im/exp/style/processed/office0_7.png} &
			\includegraphics[width=\styleImSize\linewidth]{im/exp/style/processed/office0_8.png} &
			\includegraphics[width=\styleImSize\linewidth]{im/exp/style/processed/office0_9.png} \\
			\includegraphics[width=\styleImSize\linewidth]{im/exp/style/processed/office0_10.png} &
			\includegraphics[width=\styleImSize\linewidth]{im/exp/style/processed/office0_11.png} &
			\includegraphics[width=\styleImSize\linewidth]{im/exp/style/processed/office0_12.png} &
			\includegraphics[width=\styleImSize\linewidth]{im/exp/style/processed/office0_13.png} &
			\includegraphics[width=\styleImSize\linewidth]{im/exp/style/processed/office0_14.png} \\
			\includegraphics[width=\styleImSize\linewidth]{im/exp/style/processed/office0_15.png} &
			%\includegraphics[width=\styleImSize\linewidth]{im/exp/style/processed/office0_16.png} &
			\includegraphics[width=\styleImSize\linewidth]{im/exp/style/processed/office0_17.png} &
			\includegraphics[width=\styleImSize\linewidth]{im/exp/style/processed/office0_18.png} &
			\includegraphics[width=\styleImSize\linewidth]{im/exp/style/processed/office0_19.png} &
			\includegraphics[width=\styleImSize\linewidth]{im/exp/style/processed/office0_20.png} 
			\\	\hline
		\end{tabular}
	}
	%\captionof{figure}
	\caption{Style transfer on 3D canvas.}
	\label{fig:style}
\end{figure*}


\subsubsection{ Sample-based or Derivative-based}

We select the surface model with our own captured sequences. 
All settings are detailed in \cref{sec:exp:details}.
As shown in~\cref{fig:ablation:GPIS}, reconstruction of Yijun's office is demonstrated. 
Both models are able to construct, but the derivative-based model produces a lot of noise on the surface.
This is because for smoothness purpose, we build voxels that are overlapped to its neighbor, which causes redundant voxels near the surface.
For those redundant voxels, no center sample is provided and thus the derivative based surface construction builds bad SDFs on unknow region of the voxels.


Instead, sample-based surface construction does not have this problem as it adds more points in voxels and is able to construct highly-smooth surfaces.
From which, we find well constructed and colored white board, chair, school bag and even the oranges.

\subsubsection{Tracking}


Our Uni-Fusion use a coarse-to-fine strategy for tracking. 
An external tracking model is running in one thread aside from the mapping thread.
In the mapping thread, it takes pose result from the external tracking and applies the internal tracking for colored point cloud.

The result is demonstrated in~\cref{tab:tum_rmse2}. 
The coarse-to-fine is relatively better on trajectory estimation.

\subsubsection{Voxel size}

Testing the office scene, we vary the voxel size from low to high. 
From~\cref{fig:ablation:voxel_size}, when low voxel size $0.02$m is used, the surface is rough.
Then when voxel size goes larger, the smoothness is improved.
However, when we use $0.1$m voxel size, the surface color is blur. 
Considering Uni-Fusion produces a surface color field, the quality of surface directly affect the coloring.
Thus, continuing enlarging the voxel size also results in worse surface results.

Therefore, in the above experiments, $0.05$m voxel size is utilized for surface construction.
In addition, each voxel for encoding are actually with size $0.1$m, since we use overlapped voxel.

%\subsubsection{Anchor number and feature dimension}


\subsection{Application: 2D-to-3D Transfer}
\label{sec:fabircated_prop}

Applications such as 2) and 4) can be easily integrated with application 1) incremental reconstruction (\cref{sec:incremental_reconstruction}) by incorporating the fabricated result together with the point cloud.
%
For instance, given RGB-D frames, we detect saliency or transfer image styles to generate a fabricated $X$ image. Here, $X$ represents saliency, style, or other properties. 
By combining $X$ with depth information through unprojection,
we assign
the fabricated values to corresponding points, resulting in point pairs ($\V X$, $\V Q_{X}$).

Similar to the reconstruction pipeline in~\cref{fig:recons_and_scene_understanding}, we employ encoding (\cref{sec:encoder}) and fusion (\cref{eq:fuse}) to construct a global LIM for the fabricated properties $X$.
This global LIM represents a surface $X$ fields that is utilized for subsequent inference.

While it is possible to similarly transfer a 2D semantic image to 3D,
it may not be feasible in practice due to the need for multiple passes of different categories of semantic information 
 on the same dataset (such as object, usability, etc.).
Therefore, in the following section, we demonstrate the construction of a surface feature field for scene understanding application that satisfies various 
requirements through a single mapping pass.

\begin{table*}[b!]
	%\renewcommand{\arraystretch}{0.9}
	%\setlength{\tabcolsep}{3pt}
	\caption{GZSL semantic segmentation results. Scores are in \%.
	  $^\dagger$ indicate 3DGenZ's adaption of the method.
       Note that, Uni-Fusion-SU does not even train with the seen classes.}
	\centering
	\begin{tabular}{l|c|c |c ||ccc|ccc}
		\toprule
		\multicolumn{1}{c}{}& \multicolumn{2}{c|}{Training set} & Inference input &\multicolumn{3}{c|}{ScanNet } & \multicolumn{3}{c}{S3DIS}\\
		& Backbone & Classifier & &$Seen$& $Unseen$ & $All$&$Seen$& $Unseen$ & $All$
%		\multicolumn{3}{c|}{mIoU} & 
%		\multicolumn{3}{c|}{mIoU} \\ 
%		&&&& $Seen$& $Unseen$ & $All$&$Seen$& $Unseen$ & $All$
		%\cellcolor{white}{}  
		%\cellcolor{white}{}
		\\
		\midrule
		
		\multicolumn{5}{l}{\textit{Supervised methods with different levels of supervision}}\\
		
		Full supervision & $seen \cup unseen$ & $seen \cup unseen$ & Point Cloud &43.3&51.9 &45.1&74.0&50.0&66.6 \\
		
		ZSL backbone & $seen$ & $seen \cup unseen$  &Point Cloud&41.5&39.2 & 40.3&60.9& 21.5&  48.7 \\
		
		ZSL-trivial & $seen$ & $seen$ &Point Cloud&39.2&0.0&31.3&70.2 &0.0&48.6  \\
		\midrule
		\multicolumn{5}{l}{\textit{Generalized zero-shot-learning methods}}\\
		
		ZSLPC-Seg~\cite{cheraghian2019zero}$^\dagger$ & $seen$ & $unseen$  &Point Cloud&28.2&0.0& 22.6&65.6 &0.0& 45.3\\
		
		DeViSe-3DSeg~\cite{frome2013devise}$^\dagger$ & $seen$ & $unseen$   &Point Cloud &20.0&0.0&16.0&70.2&0.0& 48.6\\ 
		%ZSLPC-Seg~\cite{cheraghian2019zero}$^\dagger$ & $seen$ & $unseen$  &  4.0&13.9\\
		%DeViSe-3DSeg~\cite{frome2013devise}$^\dagger$ & $seen$ & $unseen$   &  3.0&10.9\\
		3DGenZ~\cite{michele2021generative} & $seen$ & $seen \cup \hat{unseen}$  &Point Cloud &32.8&7.7& {27.8}&53.1&7.3&   \textbf{39.0} \\
		\midrule
		\multicolumn{5}{l}{\textit{Zero-shot learning + map fusion}}\\
		Uni-Fusion-SU (Ours) &None&None&Sparse Frames&31.0&\textbf{41.9}&\textbf{32.9} &31.3&\textbf{24.0}&29.0\\
		\bottomrule
		\multicolumn{1}{l}{}\\[-7pt]
	\end{tabular}

	\label{tab:sem_seg_overview}
\end{table*}

\begin{figure*}[t!]
	\centering
	\setlength{\tabcolsep}{0.1em}
	\renewcommand{\arraystretch}{.1}
	\begin{tabular}{|c | c |c |||c |c | c|}
		\toprule
		{\Large{3DGenZ}} & {\Large{Uni-Fusion}} &{\Large{Ground Truth}} & {\Large{3DGenZ}} &{\Large{Uni-Fusion-SU}} & {\Large{Ground Truth}} \\ \midrule
		
		\includegraphics[width=\scannetImSize\linewidth]{im/exp//ss/gen3dz_0568.png}
		&\includegraphics[width=\scannetImSize\linewidth]{im/exp//ss/mine_0568.png}
		&\includegraphics[width=\scannetImSize\linewidth]{im/exp//ss/gt_0568.png}
		&		\includegraphics[width=\scannetImSize\linewidth]{im/exp//ss/gen3dz_0164.png}
		&\includegraphics[width=\scannetImSize\linewidth]{im/exp//ss/mine_0164.png}
		&\includegraphics[width=\scannetImSize\linewidth]{im/exp//ss/gt_0164.png}\\
		
		
		\includegraphics[width=\scannetImSize\linewidth]{im/exp//ss/gen3dz_0249.png}
		&\includegraphics[width=\scannetImSize\linewidth]{im/exp//ss/mine_0249.png}
		&\includegraphics[width=\scannetImSize\linewidth]{im/exp//ss/gt_0249.png}
		&		\includegraphics[width=\scannetImSize\linewidth]{im/exp//ss/gen3dz_0435.png}
		&\includegraphics[width=\scannetImSize\linewidth]{im/exp//ss/mine_0435.png}
		&\includegraphics[width=\scannetImSize\linewidth]{im/exp//ss/gt_0435.png}\\
		
		
		\includegraphics[width=\scannetImSize\linewidth]{im/exp//ss/gen3dz_0046.png}
		&\includegraphics[width=\scannetImSize\linewidth]{im/exp//ss/mine_0046.png}
		&\includegraphics[width=\scannetImSize\linewidth]{im/exp//ss/gt_0046.png}
		&		\includegraphics[width=\scannetImSize\linewidth]{im/exp//ss/gen3dz_0050.png}
		&\includegraphics[width=\scannetImSize\linewidth]{im/exp//ss/mine_0050.png}
		&\includegraphics[width=\scannetImSize\linewidth]{im/exp//ss/gt_0050.png}\\
		\bottomrule
		
	\end{tabular}
	\includegraphics[width=\linewidth]{im/ss_colorbar}
	%\captionof{figure}
	\caption{Demonstration of semantic segmentation on the ScanNet dataset.
       Selected scenes are consistent with~\cref{fig:recons:scannet_demo}}
	\label{fig:segmentation_demo}
	
\end{figure*}

\subsection{Scene Understanding Results}

Saliency detection effectively highlights the objects of interest.
This is also considered part of 3D semantic understanding.
However, as the semantics categories vary, fusing different categories of semantics into multiple LIMs can be inefficient.
%
Therefore, in this section, we utilize Uni-Fusion to fuse and construct a surface field for high-dimensional CLIP embeddings.
With a single LIM, we can generate different semantic results based on corresponding commands.
%
Since now our Uni-Fusion works with OpenSeg for scene understanding purposes, we call it Uni-Fusion-SU.

\subsubsection{Semantic Segmentation}
\label{sec:exp:semantic}

We first evaluate our model on generalized zero-shot point cloud semantic segmentation application.
Generalized Zero-Shot Learning (GZSL) differs from Zero-Shot Learning (ZSL) in that ZSL only predicts classes unseen during training, while GZSL predicts both unseen and seen classes~\cite{michele2021generative}.
Therefore, comparing our results with GZSL SOTAs provides a better understanding of the potential of Uni-Fusion-SU, as it does not train on both seen and unseen. 

This test uses ScanNet and S3DIS datasets for benchmarking. 
It is important to note that the \textbf{compared baselines are trained on the corresponding datasets}.
Our Uni-Fusion-SU uses OpenSeg to provide the 2D image level feature ebmedding.
Although \textbf{Uni-Fusion-SU} is also zero-shot, \textbf{it does not touch any ScanNet or S3DIS annotations}.

We demonstrate the mIoU scores in~\cref{tab:sem_seg_overview}.
In particular, our model achieves best results among the zero-shot learning methods on the ScanNet dataset and remains competitive with fully supervised methods.

Furthermore, we provide results specifically for the unseen classes in~\cref{sup:tab:sn_acc_miou}.
Although not as good as the fully supervised approach, Uni-Fusion-SU performs much better than 3DGenZ.
In addition, our Uni-Fusion-SU demonstrates high precision in classes such as sofa and Toilet, even when compared to the fully supervised model.

\begin{table}[htbp]
		\caption{Classwise GZSL semantic segmentation performance (\%) on the ScanNet unseen split.}
	\centering
	\newcommand*\rotext{\multicolumn{1}{R{45}{1em}}}
	\setlength{\tabcolsep}{1.7pt}
	\begin{tabular}{@{}l@{~}c|rrrr|r@{}}
		\toprule		
		& &
		{\textbf{Bookshelf}} & {\textbf{Desk}} & {\textbf{Sofa}} & {\textbf{Toilet}} & \stackbox{mean} \\
		
		\midrule
		FSL (Fully supervise) & IoU & 	56.9&	30.0&	57.4&	63.4 & 51.9
		\\ 
		3DGenZ (Zero-shot) & IoU & 	6.3&	3.3&	13.1&	8.1 & 7.7
		\\
		Uni-Fusion-SU (Ours) & IoU &38.3&16.8&51.7&60.9&41.9
	\\ \midrule 
	3DGenZ (Zero-shot)& Acc. & 	13.4&	5.9&	49.6&	26.3 &23.8
	\\
	Uni-Fusion-SU (Ours) & Acc. &61.9&29.6&67.4&91.6& 62.6
		\\
		\bottomrule
	\end{tabular}

	\label{sup:tab:sn_acc_miou}
\end{table}

However, in the S3DIS dataset, our model does not outperform 3DGenZ and other methods as shown in~\cref{tab:sem_seg_overview}.

Even in the result of unsceened data, as presented in \cref{sup:tab:s3dis_acc_miou}, we observe that Uni-Fusion-SU hardly finds some classed, e.g. Beam and Column, which are not commonly annotated objects. 
However, for common objects like sofa and window, our model performs much better.

\begin{table}[htbp]
		\caption{Classwise GZSL semantic segmentation performance (\%) on the S3DIS unseen split.}
	\centering
	\newcommand*\rotext{\multicolumn{1}{R{45}{1em}}}
	\setlength{\tabcolsep}{1.7pt}
	\begin{tabular}{@{}l@{~}c|rrrr|r@{}}
		\toprule		
		& &
		{\textbf{Beam}} & {\textbf{Column}} & {\textbf{Sofa}} & {\textbf{Window}} & \stackbox{mean} \\
		
		\midrule
		FSL (Fully supervise) & IoU & 	63.1&	10.2&	54.1&	72.4 & 50.0
		\\ 
		3DGenZ (Zero-shot) & IoU & 	13.9&	2.4&4.9&	8.1 &7.3
		\\
		Uni-Fusion-SU (Ours) & IoU &5.5&0.02&57.4&32.9&	24.0
		\\ \midrule 
		3DGenZ (Zero-shot) & Acc. & 	20.0&	9.1&	62.4&	23.7 &28.8
		\\
		Uni-Fusion-SU (Ours) & Acc. &41.5&0.02&78.3&42.1& 40.5
		\\	
		\bottomrule
	\end{tabular}

	\label{sup:tab:s3dis_acc_miou}
\end{table}

We present the results of the semantic segmentation in~\cref{fig:segmentation_demo}. 
It is evident that, 3DGenZ's result contains more noise, as seen in the spotted sofa, bed and other objects.
Conversely, Uni-Fusion-SU's results are generally smoother and more precise.

%
%\begin{figure*}[htbp]
%	\centering
%	\includegraphics[width=.3\linewidth]{example-image-golden}
%	\includegraphics[width=.3\linewidth]{example-image-golden}
%	\includegraphics[width=.3\linewidth]{example-image-golden}
%	\\
%	\includegraphics[width=.3\linewidth]{example-image-golden}
%	\includegraphics[width=.3\linewidth]{example-image-golden}
%	\includegraphics[width=.3\linewidth]{example-image-golden}
%	
%	\caption{Semantic segmentation result on ScanNet.}
%\end{figure*}
%
%\begin{figure*}[htbp]
%	\centering
%	\includegraphics[width=.3\linewidth]{example-image-golden}
%	\includegraphics[width=.3\linewidth]{example-image-golden}
%	\includegraphics[width=.3\linewidth]{example-image-golden}
%	\\
%	\includegraphics[width=.3\linewidth]{example-image-golden}
%	\includegraphics[width=.3\linewidth]{example-image-golden}
%	\includegraphics[width=.3\linewidth]{example-image-golden}
%	
%	\caption{Semantic segmentation result on S3DIS.}
%\end{figure*}

\subsubsection{Scene Understanding with Different Properties}

\begin{figure*}[t!]
	\centering
	\setlength{\tabcolsep}{0.1em}
	\renewcommand{\arraystretch}{.1}
	\resizebox{\textwidth}{!}{\begin{tabular}{|c | c | c | c | c | c|}
			\toprule 
			& \textbf{scene0568\_00} & \textbf{scene0249\_00} & \textbf{scene0435\_00} & \textbf{office3} & \textbf{room0}\\
			\midrule
			{} &
			\raisebox{-.5\height}{\includegraphics[width=\fabImSize\linewidth]{im/exp/fab/scannet/0568_color.png}} & %\raisebox{-.5\height}{\includegraphics[width=\fabImSize\linewidth]{im/exp/fab/scannet/0164_color.png}} &
			\raisebox{-.5\height}{\includegraphics[width=\fabImSize\linewidth]{im/exp/fab/scannet/0249_color.png}} & \raisebox{-.5\height}{\includegraphics[width=\fabImSize\linewidth]{im/exp/fab/scannet/0435_color.png}}
			&
			\raisebox{-.5\height}{\includegraphics[width=\fabImSize\linewidth]{im/exp/fab/replica/office3_color.png}}
			&
			\raisebox{-.5\height}{\includegraphics[width=\fabImSize\linewidth]{im/exp/fab/replica/room0_color.png}}\\ %\raisebox{-.5\height}{\includegraphics[width=\fabImSize\linewidth]{im/exp/fab/scannet/0050_color.png}} %\includegraphics[width=\fabImSize\linewidth]{im/exp/fab/replica/office3_color.png}
			\\
			\textbf{Desk}  &
			\raisebox{-.5\height}{\includegraphics[width=\fabImSize\linewidth]{im/exp/fab/scannet/0568_lt_desk.png}}&
			%\raisebox{-.5\height}{\includegraphics[width=\fabImSize\linewidth]{im/exp/fab/scannet/0164_lt_desk.png}}&
			\raisebox{-.5\height}{\includegraphics[width=\fabImSize\linewidth]{im/exp/fab/scannet/0249_lt_desk.png}}&
			\raisebox{-.5\height}{\includegraphics[width=\fabImSize\linewidth]{im/exp/fab/scannet/0435_lt_desk.png}}
			&
			\raisebox{-.5\height}{\includegraphics[width=\fabImSize\linewidth]{im/exp/fab/replica/office3_lt_desk.png}}
			&
			\raisebox{-.5\height}{\includegraphics[width=\fabImSize\linewidth]{im/exp/fab/replica/room0_lt_desk.png}}\\
			%\raisebox{-.5\height}{\includegraphics[width=\fabImSize\linewidth]{im/exp/fab/scannet/0050_lt_desk.png}}
			%\includegraphics[width=\fabImSize\linewidth]{im/exp/fab/replica/office3_saliency.png}
			\\
			
			\textbf{Sofa} &
			\raisebox{-.5\height}{\includegraphics[width=\fabImSize\linewidth]{im/exp/fab/scannet/0568_lt_sofa.png}} &
			%\raisebox{-.5\height}{\includegraphics[width=\fabImSize\linewidth]{im/exp/fab/scannet/0164_lt_sofa.png}} &
			\raisebox{-.5\height}{\includegraphics[width=\fabImSize\linewidth]{im/exp/fab/scannet/0249_lt_sofa.png}} &
			\raisebox{-.5\height}{\includegraphics[width=\fabImSize\linewidth]{im/exp/fab/scannet/0435_lt_sofa.png}}&
			\raisebox{-.5\height}{\includegraphics[width=\fabImSize\linewidth]{im/exp/fab/replica/office3_lt_sofa.png}}
			&
			\raisebox{-.5\height}{\includegraphics[width=\fabImSize\linewidth]{im/exp/fab/replica/room0_lt_sofa.png}}\\
			%\raisebox{-.5\height}{\includegraphics[width=\fabImSize\linewidth]{im/exp/fab/scannet/0050_lt_sofa.png}}
			%\includegraphics[width=\fabImSize\linewidth]{im/exp/fab/replica/office3_style.png}
			\\
			\textbf{Work} &
			\raisebox{-.5\height}{\includegraphics[width=\fabImSize\linewidth]{im/exp/fab/scannet/0568_lt_work.png}} &
			%\raisebox{-.5\height}{\includegraphics[width=\fabImSize\linewidth]{im/exp/fab/scannet/0164_lt_work.png}} &
			\raisebox{-.5\height}{\includegraphics[width=\fabImSize\linewidth]{im/exp/fab/scannet/0249_lt_work.png}} &
			\raisebox{-.5\height}{\includegraphics[width=\fabImSize\linewidth]{im/exp/fab/scannet/0435_lt_work.png}}&
			\raisebox{-.5\height}{\includegraphics[width=\fabImSize\linewidth]{im/exp/fab/replica/office3_lt_work.png}}
			&
			\raisebox{-.5\height}{\includegraphics[width=\fabImSize\linewidth]{im/exp/fab/replica/room0_lt_work.png}}\\
			%\raisebox{-.5\height}{\includegraphics[width=\fabImSize\linewidth]{im/exp/fab/scannet/0050_lt_work.png}}
			%\includegraphics[width=\fabImSize\linewidth]{im/exp/fab/replica/office3_style.png}
			\\
			\textbf{Sittable} &
			\raisebox{-.5\height}{\includegraphics[width=\fabImSize\linewidth]{im/exp/fab/scannet/0568_lt_sit.png}} &
			%\raisebox{-.5\height}{\includegraphics[width=\fabImSize\linewidth]{im/exp/fab/scannet/0164_lt_sit.png}} &
			\raisebox{-.5\height}{\includegraphics[width=\fabImSize\linewidth]{im/exp/fab/scannet/0249_lt_sit.png}} &
			\raisebox{-.5\height}{\includegraphics[width=\fabImSize\linewidth]{im/exp/fab/scannet/0435_lt_sit.png}}&
			\raisebox{-.5\height}{\includegraphics[width=\fabImSize\linewidth]{im/exp/fab/replica/office3_lt_sit.png}}
			&
			\raisebox{-.5\height}{\includegraphics[width=\fabImSize\linewidth]{im/exp/fab/replica/room0_lt_sit.png}}\\
			%\raisebox{-.5\height}{\includegraphics[width=\fabImSize\linewidth]{im/exp/fab/scannet/0050_lt_sit.png}}
			%\includegraphics[width=\fabImSize\linewidth]{im/exp/fab/replica/office3_style.png}
			\\
			\textbf{Wood} &
			\raisebox{-.5\height}{\includegraphics[width=\fabImSize\linewidth]{im/exp/fab/scannet/0568_lt_wood.png}} &
			%\raisebox{-.5\height}{\includegraphics[width=\fabImSize\linewidth]{im/exp/fab/scannet/0164_lt_wood.png}} &
			\raisebox{-.5\height}{\includegraphics[width=\fabImSize\linewidth]{im/exp/fab/scannet/0249_lt_wood.png}} &
			\raisebox{-.5\height}{\includegraphics[width=\fabImSize\linewidth]{im/exp/fab/scannet/0435_lt_wood.png}}&
			\raisebox{-.5\height}{\includegraphics[width=\fabImSize\linewidth]{im/exp/fab/replica/office3_lt_wood.png}}
			&
			\raisebox{-.5\height}{\includegraphics[width=\fabImSize\linewidth]{im/exp/fab/replica/room0_lt_wood.png}}\\
			%\raisebox{-.5\height}{\includegraphics[width=\fabImSize\linewidth]{im/exp/fab/scannet/0050_lt_wood.png}}
			%\includegraphics[width=\fabImSize\linewidth]{im/exp/fab/replica/office3_style.png}
			\\
			
			
			\bottomrule
		\end{tabular}
	}
	%\captionof{figure}
	\caption{Demonstration of the original mesh, highlighted semantic mesh given various queries.}
	\label{fig:fab_lt}
	\vspace{-.5cm}
\end{figure*}

The main contribution of this application is that, Uni-Fusion is the first model to construct a continuous mapping of high-dimensional embeddings onto the surface without the need for any training of the map representation.
%
In the previous experiment (\cref{sec:exp:semantic}), we evaluate the performance of generalized zero-shot semantic segmentation.
However, the potential of Uni-Fusion goes beyond semantic segmentation.
%
By constructing a LIM, we obtain a surface CLIP feature field.
This enables us to query various semantic categories such as 
%without the need of multiple LIMs or rerun for other properties, we query 
\textbf{Object, Room Type, Material, Affordance and Activity} without requiring multiple LIMs or re-running the model.

We present the results in \cref{fig:fab_lt}, 
where we query object (desk, sofa), activity (work), affordance (sittable), and material (wood).
Uni-Fusion-SU accurately identifies and highlights the object and material regions.
However, for less specific commands such as work or sittable, the model provides a wider range of results with less confidence (indicated by dull yellow).
Nevertheless, the suggested options are also roughly correct.









\subsection{Time}

We run all of the applications in a single pass using our captured office sequences and evaluate the time cost of construction and fusion of each LIM. 
The average time cost across frames is shown in~\cref{tab:time}.

\begin{table}[htbp]
	\caption{Time required for each frame.
	}
	\centering
	\footnotesize
	\setlength{\tabcolsep}{0.7em}
	\resizebox{\linewidth}{!}{
		\begin{tabular}{l|ccccccc}
			\toprule
			&Surface & Color & Infrared & Style & Saliency & Latent&Internal Track \\ \midrule
			Time ($\si{\second}$)&0.100 & 0.038 & 0.045 & 0.048 & 0.045 &0.011 &0.225 \\ \bottomrule
	\end{tabular}}
	
	\label{tab:time}
\end{table}

\newcommand{\mineImSize}{.32}
%\begin{table*}[t!]
%	\centering
%	\setlength{\tabcolsep}{0.1em}
%	\renewcommand{\arraystretch}{.1}
%	\begin{tabular}{|c | c |c |}
%		\hline 
%		{Color} &{Infrared} & {Saliency} \\
%		\includegraphics[width=\mineImSize\linewidth]{im/exp/fab/mine/office/seq3_color.png} &
%		\includegraphics[width=\mineImSize\linewidth]{im/exp/fab/mine/office/seq3_color.png} &
%		\includegraphics[width=\mineImSize\linewidth]{im/exp/fab/mine/office/seq3_saliency.png} \\
%		{Style 1}&{Style 1}&{Style 1}\\
%		\includegraphics[width=\mineImSize\linewidth]{im/exp/fab/mine/office/seq3_style.png}&
%		\includegraphics[width=\mineImSize\linewidth]{im/exp/fab/mine/office/seq3_style.png}&
%		\includegraphics[width=\mineImSize\linewidth]{im/exp/fab/mine/office/seq3_style.png}\\
%		{Sofa}&{Desk}&{Soft}\\
%		\includegraphics[width=\mineImSize\linewidth]{im/exp/fab/mine/office/seq3_lt_sofa.png}&
%		\includegraphics[width=\mineImSize\linewidth]{im/exp/fab/mine/office/seq3_lt_desk.png}&
%		\includegraphics[width=\mineImSize\linewidth]{im/exp/fab/mine/office/seq3_lt_soft.png}\\		
%		\hline
%	\end{tabular}
%	\captionof{figure}{Demonstration on captured office data.}
%	\label{fig:mine_demo}
%\end{table*}
%\begin{figure*}[t!]
%	\centering
%	\setlength{\tabcolsep}{0.1em}
%	\renewcommand{\arraystretch}{.1}
%	\begin{tabular}{|c | c |c |}
%		\hline \hline
%		\includegraphics[width=\mineImSize\linewidth]{im/exp/fab/mine/office/seq3_w_slam_color.png}&	\includegraphics[width=\mineImSize\linewidth]{im/exp/fab/mine/office/seq3_w_slam_ir.png}&	\includegraphics[width=\mineImSize\linewidth]{im/exp/fab/mine/office/seq3_w_slam_saliency.png}\\
%		{Color} &{Infrared} & {Saliency}\\
%<<<<<<< HEAD


%=======
%		
%		\includegraphics[width=\mineImSize\linewidth]{im/exp/fab/mine/office/seq3_w_slam_style.png}
%		&\includegraphics[width=\mineImSize\linewidth]{im/exp/fab/mine/office/seq3_w_slam_lt_desk.png}
%		&\includegraphics[width=\mineImSize\linewidth]{im/exp/fab/mine/office/seq3_w_slam_lt_wood.png}\\
%		{Style} & {Object-desk} & {Material-wood} \\\hline
%>>>>>>> e014bc950c14dec9ffa1d2d7a6de9b7abfefabdd
%	\end{tabular}
%	%\captionof{figure}
%	\caption{Demonstration on captured Office data.}
%	\label{fig:office}
%\end{figure*}

\begin{figure*}[]
	\centering
	\setlength{\tabcolsep}{0.1em}
	\renewcommand{\arraystretch}{.1}
	\begin{tabular}{|c | c |c |}
	\hline \hline
	\includegraphics[width=\mineImSize\linewidth]{im/exp/fab/mine/appartment2/appartment2_color.png}&	\includegraphics[width=\mineImSize\linewidth]{im/exp/fab/mine/appartment2/appartment2_ir.png}&	\includegraphics[width=\mineImSize\linewidth]{im/exp/fab/mine/appartment2/appartment2_saliency.png}\\
		{Color} &{Infrared} & {Saliency}\\
	\includegraphics[width=\mineImSize\linewidth]{im/exp/fab/mine/appartment2/appartment2_style.png}
&\includegraphics[width=\mineImSize\linewidth]{im/exp/fab/mine/appartment2/appartment2_lt_sofa.png}
&\includegraphics[width=\mineImSize\linewidth]{im/exp/fab/mine/appartment2/appartment2_lt_desk.png}
\\
{Style} & {Object-sofa} & {Object-desk}\\
\includegraphics[width=\mineImSize\linewidth]{im/exp/fab/mine/appartment2/appartment2_lt_coat.png}
&\includegraphics[width=\mineImSize\linewidth]{im/exp/fab/mine/appartment2/appartment2_lt_sit.png}
&\includegraphics[width=\mineImSize\linewidth]{im/exp/fab/mine/appartment2/appartment2_lt_wood.png}\\
{Object-coat} & {Affordance-sit} & {Material-wood} \\\hline
	\end{tabular}
%\captionof{figure}
\caption{Demonstration on the captured apartment data.}
\label{fig:appartment}
%\vspace{-.5cm}
\end{figure*}


Using depth and property images of size $720\times1280$ as input, it is evident from the table, that our model operates at a frequency of $\sim10\si{\hertz}$ for  surface (sample mode) LIM construction and integration. 
It alse achieves a frequency of over $20\si{\hertz}$ for color, infrared, style, and saliency.
These results demonstrate the suitability of Uni-Fusion for real-time applications.

However, our internal tracking process takes around $0.225\si{\second}$ per frame, which is relatively slower compared to the mapping module. 
Nevertheless, Uni-Fusion uses external tracking to prevent tracking loss, enabling our internal tracking and mapping to operate at a lower frequency.
As a result, the entire model can be effectively applied in real-time in various scenarios.

\section{Extensive experiment on our own dataset}

In previous experiments, we evaluate the capabilities of Uni-Fusion in different applications. 
To further demonstrate its effectiveness in robotic environmental understanding, we capture our own dataset to show all applications together.

We capture two scenes: The office and apartment of the first author using a Microsoft Kinect Azure. 
%
RGB-D and infrared video are captured. After calibration, RGB, depth, infrared inputs have resolution of $720\times1280$.
Uni-Fusion tracks and reconstructs all applications in one pass.
%
While office data has been involved in ablation study (\cref{exp:surface:ablation}), we showcase all applications using the apartment dataset, as depicted in~\cref{fig:appartment}.

For better visualization, the ceiling of reconstruction is removed.
The top row of images presents the colored mesh with room details, the infrared mesh revealing the lighting effect, and the saliency reconstruction highlighting objects crucial for navigation.
Additionally, we select the second style from~\cref{fig:style} for style transfer to the apartment canvas.
%
As a result, the wooden floor in the room is colored with dark green.
The whole apartment is in a warm style.

The remaining results are generated from the surface field of the CLIP embeddings. 
We issue commands to locate objects, e.g., where is the sofa, desk and coat.
In addition, it easily identifies affordances such as being sittable.
For material, it successfully detects the wooden floor in each room.


\vspace{-3pt}
\subsection{BabyAI}
\vspace{3pt}
\myparagraph{Setup} The BabyAI comprises an extensible suite of tasks in environments with different sizes and shapes, where the reward is given only when the task is successfully finished. We evaluate models in trajectory planning that requires the agent move to the goal object through turn left, right and move forward actions, and general instruction completion tasks in which the agent needs to take extra pickup, drop and open actions to complete the instruction. For more detailed experimental settings and parameters, please refer to Table \ref{tbl:babyai_env_setting} in Appendix. 

% In instruction completion tasks, the agent needs to perform the extra pickup and open actions on top of trajectory planning in PickUpLoc and GoToObjMazeS4R2Close. Besides, GoToSeqS5R2 task requires go to a sequence of objects in correct order, thus no one fixed goal location is available. The planning horizon $T$ and Gibbs sampling iteration $N$ are varied in different environment. Usually, the larger size environment requires longer horizon and correspondingly more sampling iterations. After $N$ iteration, all $T$ planned actions will be executed. For more detailed experimental settings and parameters, please refer to Table \ref{tbl:babyai_env_setting} in Appendix. 

\myparagraph{Baselines} 
We compare our approach with a set of different baselines: Behavior Cloning algorithm (BC); model free reinforcement learning (RL) algorithms Batch-Constrained deep Q-Learning (BCQ) \citep{fujimoto2019off}, and Implicit Q-Learning (IQL) \citep{kostrikov2021offline}; \hongyi{model based RL algorithm Model-based Offline Policy (MOPO) \citep{mopo}}; return-conditioning approach Decision Transformer (DT) \citep{chen2021dt}; \hongyi{model based Planning Transformer (PlaTe) \citep{plate}}. In our experiments, we use, as model inputs, the full observation of the environment, the instruction, the agent's current location and the goal object location (if available).


\myparagraph{Results}
Across all environments, \netName achieves highest success rate, with the margin magnified on larger, harder tasks, see Table~\ref{tbl:babyai_main}. In particular, \netName could solve the easy tasks like GoToLocalS8N7 with nearly 90\% success rate, and has huge advantages over baselines in the large maze worlds (GoToObjMazeS7) and complex tasks (GoToSeqS5R2) which require going to a sequence of objects in correct order. In contrast, baselines perform poorly solving these difficult tasks. 

Next, we visualize the underlying iterative planing and execution procedure in GoToLocalS8N7 and GoToSeqS5R2 tasks. On the left side of Figure~\ref{fig:babyai_plan_execute}, we present how the trajectory is optimized and its energy is minimized at single time step. Through iterative refinement, the final blue trajectory is closer to the optimal solution than the original red one, which follows the correct direction with higher efficiency and perform the actions like open the door in correct situation. On the right side, we present the entire task completion process through many time steps. \netName successfully plans a efficient trajectory to visit the two objects and opens the door when blocked. \hongyi{We also explore the model performance in the stochastic settings, please refer to Appendix \ref{appendix:stochastic}.}

\begin{table*}[h]
\centering
\small
\scalebox{0.8}{
\begin{tabular}{@{}ccccccccc@{}}
\toprule
\multicolumn{1}{c}{\bf Task} & \multicolumn{1}{c}{\bf Env} & \multicolumn{1}{c}{\bf BC} & \multicolumn{1}{c}{\bf BCQ} & \multicolumn{1}{c}{\bf IQL} & \multicolumn{1}{c}{\bf DT} & \multicolumn{1}{c}{\bf \hongyi{PlaTe}} & \multicolumn{1}{c}{\bf \hongyi{MOPO}}& \multicolumn{1}{c}{\bf \netName} \\
 \midrule
\multirow{4}*{\makecell[c]{Trajectory\\Planning}} 
&{GoToLocalS7N5} & 71.0\% & 71.5\% & 84.5\%&73.0\% & \hongyi{42.5\%} & \hongyi{87.0\%}& \bf{91.0\%} \\
&{GoToLocalS8N7} & 61.5\%&63.0\% & 71.5\%&63.5\%&\hongyi{45.0\%} & \hongyi{81.5\%}&\bf{95.0\%}\\
&{GoToObjMazeS4} & 24.0\%&23.0\%&52.5\%&46.5\% & \hongyi{35.5\%} & \hongyi{60.0\%}& \bf{62.5\%}\\
&{GoToObjMazeS7} & 18.0\%&10.5\% & 29.0\%&22.0\%&\hongyi{27.5\%} & \hongyi{30.5\%}&\bf{42.5\%}\\\midrule
\multirow{3}*{\makecell[c]{Instruction\\ Completion}} 
&{PickUpLoc} & 57.5\% & 58.5\%& 41.0\%&59.5\% & \hongyi{7.5\%} &\hongyi{43.5\%}& \bf{67.0\%}\\
&{GoToSeqS5R2} & 13.5\% & 10.0\%& 28.5\%&26.5\% & \hongyi{25.0\%}&\hongyi{30.0\%}& \bf{33.0\%} \\
&{GoToObjMazeS4R2Close} & 24.0\%&22.5\%&31.5\% &48.5\% & \hongyi{32.5\%}& \hongyi{36.0\%}& \bf{55.0\%}\\
\bottomrule
\end{tabular}
}
\caption{\small \textbf{BabyAI Quantitative Performance.} The task success rate of \netName and a variety of prior algorithms on BabyAI env. Models are trained with 500 optimal trajectory demos in each environment, and results are averaged over 5 random seeds. The SX and NY in environment name means its size and the number of obstacles.}
\label{tbl:babyai_main}
\vspace{-10pt}
\end{table*}





\myparagraph{Effect of Iterative Refinement.} 
We investigate the effect of iterative refinement by testing the  success  rate of our approach under different sample iteration number in GoToLocalS7N5 environment. 
From the left side of Figure~\ref{fig:anlsBaby}, the task success rate continues to
improve as we increase the number of sample iteration.  
% In detail, iterative refinement helps more substantially when going from 1 iteration to 5 iterations, and the benefit gradually decreases as the iteration number goes from 20 to 50. 



\myparagraph{Energy Landscape Verification.}
We further verify our approach by visualizing the energy   
 assignment on various trajectories in the same environment as above. 
More specifically, we compare the estimated energy of the labeled optimal trajectory with the noisiness of trajectories,
produced by randomizing a percentage of steps in the optimal action sequence.
The right Figure~\ref{fig:anlsBaby} depicts the energy assignment to trajectories  with various corruption levels as a function of training  time.
With the progress of training, \netName learns to
(a) reduce the energy value assigned to the optimal trajectory;
(b) increase the energy value assigned to the corrupted trajectories.
\begin{wraptable}{r}{0.4\textwidth}
    \vspace{-5pt}
    \small
    \centering
%   \setlength{\tabcolsep}{0.5em}
  \scalebox{0.68}{
    \begin{tabular}{@{}ccccc@{}}
    \toprule
     & \multicolumn{2}{c}{\bf Optimal} & \multicolumn{2}{c}{\bf Suboptimal(25\%)} \\
    \multicolumn{1}{c}{\bf Env}& {\bf DT} & {\bf \netName} & {\bf DT} & {\bf \netName} \\
     \midrule
    {GoToObjMazeS4}& 46.5\% & \bf{65.0\%} & 45\% & \bf{63.0\%} \\ % &38.5\%&46.0\% \\
    {GoToObjMazeS7}& 22.0\%&\bf{45.5\%} & 19.5\%&  \bf{44.0\%} \\% & 26.0\%&21.5\%\\
    {GoToSeqS5R2} & 26.5\%& \bf{38.0\%} & 29.5\%& \bf{37.5\%} \\%& 34.0\%&28.0\%\\
    \bottomrule
    \end{tabular}
  }
  \vspace{-5pt}
 . \caption{\small \textbf{Performance on Suboptimal Data} The task success rate of \netName and DT, using optimal trajectories and suboptimal trajectories containing $25\%$ random actions respectively as training data.}
  \label{tbl:suboptimal_data}
  \vspace{-15pt}
\end{wraptable}
% \vspace{-6pt} 
This result justifies the performance of \netName. 



\myparagraph{Effect of Training Data.} In BabyAI, we utilize a set of demonstrations generated  using an oracle planner. We further investigate the performance of \netName when the training data is not optimal. To achieve it, we randomly swap the decisions in the optimal demonstration with an arbitrary action with the probability of $25\%$. We compare against DT, the autoregressive sequential planner. Despite a small performance drop in Table~\ref{tbl:suboptimal_data}, \netName still substantially outperforms DT, indicating \netName works well with non-perfect data.




\begin{figure*}[t!]
 
  \vspace*{-0.0in}
  \centering
  \scalebox{0.85}{
    \begin{tikzpicture}
     \node[anchor=north west](main) at (0in,0in)
      {{\includegraphics[width=0.9\textwidth,clip=true,trim=0in
      7.8in 0.0in 0.5in]{figs/baby_6.pdf}}};
      % Iteration number
     \node[anchor=north] at ($ (main.south) + (-4.45, 0.3) $){\footnotesize \IterNum=1};
     \node[anchor=north] at ($ (main.south) + (-2.65, 0.3) $){\footnotesize \IterNum=5};
     \node[anchor=north] at ($ (main.south) + (-0.85, 0.3) $){\footnotesize \IterNum=10};
%     \node[anchor=north west] at (0.92in,-0.05in) {\textbf{(a)}};
%     \node[anchor=north west] at (2.00in,-0.05in) {\textbf{(b)}};
%     \node[anchor=north west] at (3.09in,-0.05in) {\textbf{(c)}};
    \end{tikzpicture}
  }
  % \vspace*{-0.0in}
  \vspace{-8pt}
  \caption{
    \textbf{\small Qualitative Visualization of Planning and Execution Procedure in BabyAI}.
    \textit{Left} depicts the planning through iterative energy minimization where $N$ is the sample iteration number.
    \textit{Right} shows the execution of the concatenate action sequences.
    Two task settings are illustrated:
    \textbf{(a)}: \textbf{Trajectory planning}, where the task is to solely plan a trajectory leading to the goal location.
    \textbf{(b)}: \textbf{Instruction completion}, where a sequence of tasks are commanded, and an additional "Open" is involved to get through the doors.
    Target locations are marked with \protect{\raisebox{-.05cm}{\includegraphics[height=.30cm]{figs/target_highlight.png}}}.
  }
  \vspace*{-10pt}
 \label{fig:babyai_plan_execute}
\end{figure*}

\begin{figure*}[t!]
    \centering
    \begin{minipage}{0.43\textwidth}
        \centering
        \includegraphics[width=\textwidth]{figs/anlsIter.pdf}
        % \caption{Caption}
        \label{fig:anlsBaby_sample}
    \end{minipage}
    \begin{minipage}{0.40\textwidth}
        \centering
        \includegraphics[width=\textwidth]{figs/anlsEngyNoise.pdf}
        % \caption{Caption}
        \label{fig:anlsBaby_noise}
    \end{minipage}

    \vspace{-20pt}
    \caption{\small
        \textbf{Analysis of \netName in the BabyAI Environments}.
        \textit{Left}: Success rate increases with more sampling steps, suggesting the importance of iterative refinement in \netName.
        \small \textit{Right}: \netName captures the correct energy landscape.
        It assigns low energy to the optimal trajectory (noise level=0\%) and high energy to noisy paths.
    }
    \label{fig:anlsBaby}
    \vspace{-15pt}
\end{figure*}



\vspace{-5pt}
\subsection{Atari}

\vspace{3pt}

\myparagraph{Setup} We further evaluate our approach on the dynamically changing Atari environment, with higher-dimensional visual state. 
Due to above features , we train and test our model without the goal state, and update the plan after each step of execution to avoid unexpected changes in the world. We compare our model to BC, DT \citep{chen2021dt}, CQL \citep{kumar2020conservative}, REM \citep{agarwal2020optimistic}, and QR-DQN \citep{dabney2018distributional}.
Following \citet{chen2021dt}, the evaluation is conducted on four Atari games (Breakout, Qbert, Pong, and Seaquest), where $1\%$ of data is used for training. 
Human normalized score is utilized for the performance evaluation.


\myparagraph{Results} 
Table \ref{tbl:atari_main} shows the result comparison.
\netName achieves the best average performance.
Specifically, it achieves better or comparable result in 3 games out of 4,
whereas baselines typically perform poorly in more than one games.



\begin{comment}
When using only 1\% of the data, our method is competitive with the-state-of-art models like DT and CQL in 3 out of 4 games and achieves highest averaged scores over 4 games, as shown in Table \ref{tbl:atari_main}. The scores are normalized based on a professional gamer, following the protocol in \citet{chen2021dt}. 

\end{comment}


\myparagraph{Energy Landscape Verification}
In Atari environment, the training trajectories are generated by an online DQN agent during training, whose accumulated rewards are varied a lot. \netName is trained to estimate the energy values of trajectories depending on their rewards. In Figure~\ref{fig:anlsAtar}, we visualize the estimated energy of different training trajectories and their corresponding rewards in Breakout and Pong games. We notice that the underlying energy value estimated to a trajectory is well correlated with its reward, with low energy value assigned to high reward trajectory. This justifies the correctness of our trained model and further gives a natural objective to assess the relative of quality planned trajectory. In Qbert and Seaquest games that \netName gets low scores, this negative correlation is not obvious showing that the model is not well-trained.
% \begin{figure*}[t!]
    \centering
    \begin{minipage}{0.4\textwidth}
        \centering
        \includegraphics[width=\textwidth]{figs/anls_atar_Breakout.pdf}
        % \caption{Caption}
        % \label{fig:sample_figure}
    \end{minipage}
    \begin{minipage}{0.4\textwidth}
        \centering
        \includegraphics[width=\textwidth]{figs/anls_atar_Pong.pdf}
        % \caption{Caption}
        % \label{fig:sample_figure}
    \end{minipage}
    % \begin{minipage}{0.24\textwidth}
    %     \centering
    %     \includegraphics[width=\textwidth]{figs/anls_atar_Seaquest.pdf}
    %     % \caption{Caption}
    %     % \label{fig:sample_figure}
    % \end{minipage}
    % \begin{minipage}{0.24\textwidth}
    %     \centering
    %     \includegraphics[width=\textwidth]{figs/anls_atar_Qbert.pdf}
    %     % \caption{Caption}
    %     % \label{fig:sample_figure}
    % \end{minipage}
    \vspace{-10pt}
    \caption{
        \textbf{Energy vs. Reward on Atari}.
        Energies and rewards are normalized to $[0, 1]$.
        We demonstrate negative correspondence between the achieved rewards and estimated energy by \netName,
        which justifies our method.
    }
    \label{fig:anlsAtar}
    \vspace{-10pt}
\end{figure*}
\begin{table*}[t!]
    \centering
    \small
    % \hspace{-35pt}
    \begin{minipage}[t]{0.5\linewidth}
        \centering
        \setlength\tabcolsep{1.5pt} 
        \scalebox{0.76}{
        \begin{tabular}{lrrrrrr}
            \toprule
            \multicolumn{1}{c}{\bf Game} & \multicolumn{1}{c}{\bf \netName} & \multicolumn{1}{c}{\bf DT} & \multicolumn{1}{c}{\bf CQL} & \multicolumn{1}{c}{\bf QR-DQN} & \multicolumn{1}{c}{\bf REM} & \multicolumn{1}{c}{\bf BC} \\ 
            \midrule
            Breakout & $\bf{400.9}\hongyi{\pm 54.0}$& $267.5\hongyi{\pm 56.3}$  & $211.1$ & $17.1$ & $8.9$ & $138.9$ \\ 
            Qbert   &  $19.5\hongyi{\pm 1.6}$ & $15.4\hongyi{\pm 6.6}$ & $\bf{104.2}$ & $0.0$ & $0.0$ & $17.3$ \\
            Pong    & $108.9\hongyi{\pm 1.6}$ & $106.1\hongyi{\pm 4.7}$ & $\bf{111.9}$ & $18.0$ & $0.5$ & $85.2$ \\ 
            Seaquest & $1.3\hongyi{\pm 0.2}$ & $\bf{2.5}\hongyi{\pm 0.2}$ & $1.7$ &  $0.4$ & $0.7$  & $2.1$ \\ 
            \midrule
            Avg & $\bf{132.6}$ & $97.9$ & $107.2$ &  $8.9$ & $2.5$  & $60.9$ \\ 
            \bottomrule
        \end{tabular}
        }
        \caption{\small
            \textbf{Quantitative Comparison on Atari.}
            Gamer-normalized scores for the 1\% DQN-replay Atari dataset \citep{agarwal2020optimistic}. We report the mean \hongyi{and standard error} score across 5 seeds. \netName achieves best averaged scores over 4 games and performs comparably to DT and CQL over all games.}
        \label{tbl:atari_main}
    \end{minipage}
    \hfill
    \begin{minipage}[t]{0.46\linewidth}
        % \hspace{-35pt}
        \centering
        \begin{minipage}{0.49\textwidth}
            \centering
            \includegraphics[width=\textwidth]{figs/anls_atar_Breakout.pdf}
            % \caption{Caption}
            % \label{fig:sample_figure}
        \end{minipage}
        \begin{minipage}{0.49\textwidth}
            \centering
            \includegraphics[width=\textwidth]{figs/anls_atar_Pong.pdf}
            % \caption{Caption}
            % \label{fig:sample_figure}
        \end{minipage}
        \vspace{-10pt}
        \captionof{figure}{ \small 
            \textbf{Energy vs. Reward on Atari}.
            Energies and rewards are normalized to $[0, 1]$.
            We demonstrate negative correspondence between the achieved rewards and estimated energy by \netName,
            which justifies our method.
        }
        \label{fig:anlsAtar}
    \end{minipage}
    \vspace{-18pt}
\end{table*}
% \begin{wraptable}{r}{0.4\textwidth}
    \vspace{-5pt}
    \small
    \centering
%   \setlength{\tabcolsep}{0.5em}
  \scalebox{0.68}{
    \begin{tabular}{@{}ccccc@{}}
    \toprule
     & \multicolumn{2}{c}{\bf Optimal} & \multicolumn{2}{c}{\bf Suboptimal(25\%)} \\
    \multicolumn{1}{c}{\bf Env}& {\bf DT} & {\bf \netName} & {\bf DT} & {\bf \netName} \\
     \midrule
    {GoToObjMazeS4}& 46.5\% & \bf{65.0\%} & 45\% & \bf{63.0\%} \\ % &38.5\%&46.0\% \\
    {GoToObjMazeS7}& 22.0\%&\bf{45.5\%} & 19.5\%&  \bf{44.0\%} \\% & 26.0\%&21.5\%\\
    {GoToSeqS5R2} & 26.5\%& \bf{38.0\%} & 29.5\%& \bf{37.5\%} \\%& 34.0\%&28.0\%\\
    \bottomrule
    \end{tabular}
  }
  \vspace{-5pt}
 . \caption{\small \textbf{Performance on Suboptimal Data} The task success rate of \netName and DT, using optimal trajectories and suboptimal trajectories containing $25\%$ random actions respectively as training data.}
  \label{tbl:suboptimal_data}
  \vspace{-15pt}
\end{wraptable}
% \vspace{-6pt}




%%%% experiments
We provide some comments on the growth conditions which constituted the majority of our analysis in sections \ref{sec:Hmixing} and \ref{sec:Hsigma}. In the simplest cases of Lemma \ref{lemma:unstableGrowth}, growth was established in an analogous fashion to the old one-step expansion condition (\ref{eq:oldOneStepExpansion}), finding the relevant Jacobians $M_j$ and checking that their expansion factors $K(M_j)$ satisfy
\begin{equation}
    \label{eq:discussionOneStep}
    \sum_j \frac{1}{K(M_j)} <1.
\end{equation}
For the more complicated cases, the inductive method used to establish growth near the accumulation points in Lemma \ref{lemma:unstableGrowth} and the weakened one-step expansion condition (\ref{eq:oneStep}) both address the same fundamental issue: the splitting of unstable curves by singularities into an unbounded number of small components. They circumvent this obstacle in rather different ways, however. While (\ref{eq:oneStep}) generalises (\ref{eq:discussionOneStep}) to ensure an growth of unstable curves `on average' (see \cite{chernov_statistical_2009} for a precise statement), our inductive method is a more direct adaptation of (\ref{eq:discussionOneStep}), using it to generate contradictory geometric conditions which a hypothetical non-growing unstable curve must satisfy. It may be possible to prove Theorem \ref{sec:Hmixing} using (\ref{eq:oneStep}) as the basis for growth. Since we required (\ref{eq:oneStep}) anyway for proving Theorem \ref{thm:HsigmaExp}, this could potentially condense our analysis, but only to a minor extent. A convenience of the method used in section \ref{sec:Hmixing} is that, by way of the `simple intersection' property, it naturally gives geometric information on the images of manifolds, useful for proving the property \textbf{(M)} of Theorem \ref{thm:katok-strelcyn}.

We expect that essentially analogous analysis can be applied to establish mixing properties in a wide class of piecewise linear non-uniformly hyperbolic maps, including those (like the OTM) which sit on the boundary of ergodicity and beyond. While we have relied on the precise partition structure of $H_\sigma$, its fundamental feature (self-similar sequences of elements $A^k$, sharing boundaries with its neighbours $A^{k-1},A^{k+1}$ and accumulating onto some point $p$) is quite typical to return map systems. See, for example, those of various stadium billiards \cite{chernov_chaotic_2006,chernov_improved_2008,chernov_statistical_2009} and LTMs \cite{springham_polynomial_2014}. Indeed, the same method can be used to prove the Bernoulli property for non-monotonic LTMs \cite{myers_hill_mixing_2022}, where monotonicity of the manifold images cannot be assumed and the classical argument \cite{sturman_mathematical_2006} fails. The OTM is the pointwise limit of these maps as the boundary shrinks to null measure. It further has utility in proving growth conditions for maps which are uniformly hyperbolic but possess regions $A_j$ where the hyperbolicity is very weak, signified by $K(M_j) \approx 1$, so that (\ref{eq:discussionOneStep}) fails. Typically this leads to suboptimal bounds on mixing windows, see e.g. \cite{wojtkowski_model_1981,przytycki_ergodicity_1983,myers_hill_family_2022}. The map $H_{(\eta,\eta)}$ for $\eta \approx 1/2$ is another example, possessing weak hyperbolicity over $A_2, A_3$. Letting $\varepsilon = |\eta-1/2|>0$, there is an upper bound $N = N(\varepsilon)$ on escape times from the intersections $A_2\cap \sigma, A_3 \cap \sigma$. The growth lemma then follows by applying the inductive step roughly $N$ times and can be established for arbitrarily small $\varepsilon$, opening the door to establishing optimal mixing windows.

The above gives two examples of piecewise linear perturbations to $H$ where mixing with respect to Lebesgue is preserved and our methods can be applied. Nonlinear perturbations to the shear profiles complicate the analysis in several ways. Firstly as the map's Jacobians takes on a broader range of values, cone invariance becomes an increasingly harder condition to establish. Cones must be widened, giving looser bounds on expansion factors, which may already be weak due to new regions of weaker stretching. This, together with the change from polygonal to curvilinear return time partition elements and nonlinear local manifolds, adds some complexity to showing growth conditions. This does not rule out certain (small) nonlinear perturbations however. There is some leeway in the inequalities which govern cone invariance and growth of local manifolds, the latter of which is not too dissimilar from the piecewise linear setting (see Lemmas \ref{lemma:piecewiseApprox}, \ref{lemma:componentLength}). Certain small perturbations would not alter the \emph{topological} structure of the return time partition, i.e. which elements share boundaries, the key information needed for setting up the induction. Finally while the partition elements would no longer be polygonal, only coarse geometric information is required for verifying each inductive step. Following the above, a potential perturbation could be to replace the linear portions of each shear by a cubic, perturbing the tent profile
\[  f(t) = \begin{cases} 2t & 0 \leq t \leq 1/2, \\ 2(1-t) & 1/2 \leq t \leq 1 ,\end{cases} \]
of the OTM shears to
\[  f_a(t) = \begin{cases} \frac{1}{8} t \left(16 - a + 6at - 8at^{2} \right) & 0 \leq t \leq 1/2, \\ \frac{1}{8}\left(1-t\right)\left( 16 - a + 6a\left(1-t\right) - 8a\left(1-t\right)^{2}\right)  & 1/2 \leq t \leq 1, \end{cases}   \]
for $a>0$. For small enough $a$ the gradient range $f'(t)$ is restricted to small neighbourhoods of $\{ 2, -2\}$ and the escape time partition retains a similar structure. We illustrate this in Figure \ref{fig:perturbations}, showing escapes from the square $S_3$ under the map $G \circ F$, equivalent to escapes from the perturbed $A_3$ under the $G \circ F$, but with a cleaner geometry for comparison. When $a$ is too large the analogy to the OTM breaks down. At $a=16$ the map is twice differentiable everywhere and features a new source of slowed mixing, the Jacobian is the identity at the corner points $x,y \in \{  0, 1/2 \}$ giving locally parabolic behaviour (visible in the escape time partition). 

\begin{figure}
    \centering
    \includegraphics[width=0.24 \linewidth]{0.png}
    \includegraphics[width=0.24 \linewidth]{4.png}
    \includegraphics[width=0.24 \linewidth]{8.png}
    \includegraphics[width=0.24 \linewidth]{16.png}
    \caption{Partition of escape times from $S_3$ under the mapping $F \circ G$ for $a= 0,4,8,16$. }
    \label{fig:perturbations}
\end{figure}




%%%% Conclusion
\section{Conclusion}
High Dynamic Range (HDR) Imaging is a desirable function for modern digital cameras. To overcome the hardware limitation of current camera sensors in dynamic range, we have described a new neural model to blend low dynamic range images captured at tri-exposures into HDR images. We have shown that, by integrating the gating mechanism with the shifted-window self-attention mechanism, our model progressively aligns the cross-exposure images and screen outliers caused by large motion from HDR fusion. We have also demonstrated that, in the framework of multi-scale encoder-decoder, the Gated Swin Transformer achieves global reception, long-range matching and adaptive fusion. Extensive ablation study has been presented to validate the effectiveness of our model in addressing large motion in HDR imaging.

\begin{comment}

\subsection{Footnotes}

Indicate footnotes with a number\footnote{Sample of the first footnote} in the
text. Place the footnotes at the bottom of the page on which they appear.
Precede the footnote with a horizontal rule of 2~inches
(12~picas).\footnote{Sample of the second footnote}

\subsection{Figures}

All artwork must be neat, clean, and legible. Lines should be dark
enough for purposes of reproduction; art work should not be
hand-drawn. The figure number and caption always appear after the
figure. Place one line space before the figure caption, and one line
space after the figure. The figure caption is lower case (except for
first word and proper nouns); figures are numbered consecutively.

Make sure the figure caption does not get separated from the figure.
Leave sufficient space to avoid splitting the figure and figure caption.

You may use color figures.
However, it is best for the
figure captions and the paper body to make sense if the paper is printed
either in black/white or in color.
\begin{figure}[h]
\begin{center}
%\framebox[4.0in]{$\;$}
\fbox{\rule[-.5cm]{0cm}{4cm} \rule[-.5cm]{4cm}{0cm}}
\end{center}
\caption{Sample figure caption.}
\end{figure}

\subsection{Tables}

All tables must be centered, neat, clean and legible. Do not use hand-drawn
tables. The table number and title always appear before the table. See
Table~\ref{sample-table}.

Place one line space before the table title, one line space after the table
title, and one line space after the table. The table title must be lower case
(except for first word and proper nouns); tables are numbered consecutively.

In an attempt to encourage standardized notation, we have included the
notation file from the textbook, \textit{Deep Learning}
\cite{goodfellow2016deep} available at
\url{https://github.com/goodfeli/dlbook_notation/}.  Use of this style
is not required and can be disabled by commenting out
\texttt{math\_commands.tex}.


\centerline{\bf Numbers and Arrays}
\bgroup
\def\arraystretch{1.5}
\begin{tabular}{p{1in}p{3.25in}}
$\displaystyle a$ & A scalar (integer or real)\\
$\displaystyle \va$ & A vector\\
$\displaystyle \mA$ & A matrix\\
$\displaystyle \tA$ & A tensor\\
$\displaystyle \mI_n$ & Identity matrix with $n$ rows and $n$ columns\\
$\displaystyle \mI$ & Identity matrix with dimensionality implied by context\\
$\displaystyle \ve^{(i)}$ & Standard basis vector $[0,\dots,0,1,0,\dots,0]$ with a 1 at position $i$\\
$\displaystyle \text{diag}(\va)$ & A square, diagonal matrix with diagonal entries given by $\va$\\
$\displaystyle \ra$ & A scalar random variable\\
$\displaystyle \rva$ & A vector-valued random variable\\
$\displaystyle \rmA$ & A matrix-valued random variable\\
\end{tabular}
\egroup
\vspace{0.25cm}

\centerline{\bf Sets and Graphs}
\bgroup
\def\arraystretch{1.5}

\begin{tabular}{p{1.25in}p{3.25in}}
$\displaystyle \sA$ & A set\\
$\displaystyle \R$ & The set of real numbers \\
$\displaystyle \{0, 1\}$ & The set containing 0 and 1 \\
$\displaystyle \{0, 1, \dots, n \}$ & The set of all integers between $0$ and $n$\\
$\displaystyle [a, b]$ & The real interval including $a$ and $b$\\
$\displaystyle (a, b]$ & The real interval excluding $a$ but including $b$\\
$\displaystyle \sA \backslash \sB$ & Set subtraction, i.e., the set containing the elements of $\sA$ that are not in $\sB$\\
$\displaystyle \gG$ & A graph\\
$\displaystyle \parents_\gG(\ervx_i)$ & The parents of $\ervx_i$ in $\gG$
\end{tabular}
\vspace{0.25cm}


\centerline{\bf Indexing}
\bgroup
\def\arraystretch{1.5}

\begin{tabular}{p{1.25in}p{3.25in}}
$\displaystyle \eva_i$ & Element $i$ of vector $\va$, with indexing starting at 1 \\
$\displaystyle \eva_{-i}$ & All elements of vector $\va$ except for element $i$ \\
$\displaystyle \emA_{i,j}$ & Element $i, j$ of matrix $\mA$ \\
$\displaystyle \mA_{i, :}$ & Row $i$ of matrix $\mA$ \\
$\displaystyle \mA_{:, i}$ & Column $i$ of matrix $\mA$ \\
$\displaystyle \etA_{i, j, k}$ & Element $(i, j, k)$ of a 3-D tensor $\tA$\\
$\displaystyle \tA_{:, :, i}$ & 2-D slice of a 3-D tensor\\
$\displaystyle \erva_i$ & Element $i$ of the random vector $\rva$ \\
\end{tabular}
\egroup
\vspace{0.25cm}

\end{comment}

\bibliography{iclr2023_conference}
\bibliographystyle{iclr2023/iclr2023_conference}

% \section{Appendix

\newpage


\section{Stochastic Allocations of Indivisible Goods}\label{sec:app}
% title: what is the purpose of this section

In this section we consider a particular application of our results, for the problem of \textit{stochastic allocations of indivisible goods}. 
% title: basic model  
The setting postulates a set of $n$ agents $1,\ldots,n,$ and $m$ items, $1,\ldots,m,$ to be distributed amongst the agents.
A \emph{deterministic allocation} of the items to the agents is a mapping $A:[m]\rightarrow [n]$, determining which agent gets each item. 
% Given an allocation $A$, we denote by $A_i=A^{-1}(i)$ - the set of items allocated to $i$ in $A$. 
We denote by $\mathcal{A}$ the set of deterministic allocations. 
A \emph{stochastic allocation}, $d$, is a distribution over the deterministic allocations.  The set of all possible stochastic allocations is: 
\begin{align*}
    \mathcal{D} = \{d \mid p_d \colon \mathcal{A} \to [0,1], \sum_{A \in \mathcal{A}} p_d(A) = 1\}
\end{align*}   
Each agent $j$ is associated with a function $u_j \colon \mathcal{A} \to \mathbb{R}_{\geq 0}$ that describes its utility from a deterministic allocation.
Agents are assumed to care only about their own share (allowing us to use the following abuse of notation in which $u_j$ takes a bundle $b$ of items), their utilities are assumed to be normalized ($u_j(\emptyset) = 0$), monotone ($u_j(b_1) \leq u_j(b_2)$ if $b_1 \subseteq b_2$), and submodular ($u_j(b_1) + u_j(b_2) \geq u_j(b_1 \cup b_2) + u_j(b_1 \cap b_2)$ for any bundles $b_1,b_2$).
We also assume that utilities $(u_i)_{i=1}^n$ are given in the \emph{value oracle model}, meaning that we do not have a direct access to them, but only to an oracle that indicates the value of an agent from a given deterministic allocation.
Lastly, it is assumed that each agent has a positive utility from the set of all items.
% \eden{z1 > 0}
% \eden{should we explain somewhere that submodularity in the context of people's utilities makes a lot of sense? Maybe to explain what it is with diminishing returns?}
% \eden{should we say something about the value-oracle model? if so, where?}

% \eden{maybe to change "when agents have submodular utilities" to: "under these settings"?}
We prove that, in this setting, an approximately-optimal leximin solution with \emph{only} a multiplicative error can be obtained in polynomial time.
Specifically, we prove that a $\frac{1}{3}$-approximation\footnote{Throughout this section, we only discuss multiplicative approximations; so, for brevity, we use the term "$\multApprox$-approximation" to refer to "$(\multApprox,0)$-approximation".} can be obtained deterministically, whereas a $\frac{(e-1)^2}{e^2-e+1} \approx 0.52$-approximation can be obtained w.h.p.
As a reference point, it is worth noting that the problem of maximizing the egalitarian welfare in these settings has been shown to be NP-hard to approximate to a (multiplicative) factor better than than $1-\frac{1}{e} \approx 0.632$ \cite{kawase_max-min_2020}.
% \eden{should we say, as \textcite{kawase_max-min_2020}, that given an approximation algorithm with a multiplicative error of $\multError$ for welfare maximization, we obtain a approximate leximin with a multiplicative error of $\frac{\multError}{1-\multError +\multError^2}$?}.

% \footnote{Recall that a $(\multError,\additiveError)$-approximation has at most a multiplicative error of $\multError$ and at most an additive error of $\additiveError$. Therefore, the lower the parameters, the higher the accuracy.}




Given a stochastic allocation $d$, the expected utility of agent $j$ is given by
\begin{align*}
	E_j(d) = \sum_{A\in \mathcal{A}}p_d(A)\cdot u_j(A)
\end{align*}
The goal is to find a stochastic allocation that maximizes the set of functions $E_1,\ldots,E_n$. 
Formally, we consider the following problem:
\begin{align*}
	\lexmaxmin \quad &\{E_1(d), E_2(x), \dots E_n(d)\} \\
	s.t. \quad  & d \in \mathcal{D}
\end{align*}
% \eden{maybe $\maxlexmin$?} 
\eden{should write something about the output size, as \textcite{kawase_max-min_2020}}
That is, the feasible region is the set of stochastic allocations ($S = \mathcal{D}$) and the objective functions are the expected utilities ($f_i = E_i$ for any $i\in [N]$).

However, we shall see that an $\multApprox$-approximation to leximin is first and foremost an $\multApprox$-approximation to the egalitarian welfare. Therefore, the same hardness result applies to our problem as well.

\begin{lemma}
    If a solution is an $\multApprox$-approximation to leximin, then it is also an $\multApprox$-approximation to the egalitarian welfare.
\end{lemma}

% \eden{I removed the proof for now due to lack of space}
% \begin{proof}
%     Let $d \in \mathcal{D}$ be a stochastic allocation, and assume that it is an $\multApprox$-approximation to leximin. 
%     By definition, there is no solution that is $(\multApprox,0)$-preferred over it --- $d' \nAlphaBetaPreferredParams{\multApprox}{0} d$ for any $d' \in \mathcal{D}$.
%     Suppose by contradiction that $d$ is \emph{not} an $\multApprox$-approximation to the egalitarian welfare, and let $d^*$ be the optimal solution to this problem.
%     This means that the smallest objective value of $d'$ is less than smallest objective value of $d^*$ times $\multApprox$ --- $\valBy{1}{d} < \multApprox \cdot\valBy{1}{d^*}$.
%     But it follows that $d^*\alphaBetaPreferredParams{\multApprox}{0} d $;
%     for $k=1$, the required for $j<k$ is vacuously true 
%     and $\valBy{1}{d^*} > \frac{1}{\multApprox}\valBy{1}{d}$.
%     However, we know that the $d' \nAlphaBetaPreferredParams{\multApprox}{0} d$ for any $d' \in \mathcal{D}$, so it is true in particular for $d^*$. This is a contradiction.
% \end{proof}
As the proof is straightforward, it is omitted.


\textcite{kawase_max-min_2020} present an approximation algorithm that relates the problem of finding a stochastic allocation that approximates the egalitarian welfare, to the problem of finding a \emph{deterministic} allocation that approximates the \emph{utilitarian welfare} (i.e., the sum of utilities):
\begin{align*}
 \max \quad &\sum_{i=1}^n u_i(A)   \;\;
        \quad s.t. \quad   A \in \mathcal{A}  \tag{U1}\label{eq:utilitarian}
\end{align*}
% \erel{Can you write the maximization problem for utilitarian welfare?}

In this paper, we adapt their algorithm and prove the following relation to leximin:
\begin{theorem}
\label{th:app-main}
Suppose we are given a randomized algorithm that returns a deterministic allocation that approximates the utilitarian welfare with multiplicative error $\multError$ (with success probability $p$).
Then, Algorithm \ref{alg:basic-ordered-Outcomes} can be used to obtain a stochastic allocation that approximates leximin with a multiplicative error of at most $\frac{\multError}{1-\multError +\multError^2}$ (with the same probability).
\end{theorem}

\noindent Proving this theorem will yield two immediate results since there are known algorithms to approximate the utilitarian welfare when the agents' utility functions are monotone and submodular.
First, there are deterministic approximation algorithms with a multiplicative error $\frac{1}{2}$ \cite{Fisher1978, Buchbinder2019}, and therefore:
\begin{corollary}
    Algorithm \ref{alg:basic-ordered-Outcomes} can be used to obtain a stochastic allocation that approximates leximin with a multiplicative error at most 
    $
    \frac{0.5}{1-0.5+0.5^2} = 
    \frac{2}{3}$.
\end{corollary}
\noindent Second, there is a randomized approximation algorithm with a multiplicative error of  $\frac{1}{e}$ w.h.p \cite{vondrak_optimal_2008}, and therefore:
\begin{corollary}
    Algorithm \ref{alg:basic-ordered-Outcomes} can be used to obtain a stochastic allocation that approximates leximin with a multiplicative error at most $\frac{e}{e^2-e+1} \approx 0.48$ w.h.p.
\end{corollary}

The proof of Theorem \ref{th:app-main} is provided in Appendix \ref{sec:app-sec-proofs}. 



\end{document}
