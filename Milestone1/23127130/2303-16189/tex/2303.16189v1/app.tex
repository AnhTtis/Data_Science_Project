\appendix
\textbf{\Large{Appendix}}
\vspace{5pt}

\section{Experimental Details}

\subsection{BabyAI Environment Details}
We categorize the environments tested in the trajectory planning and instruction completion into the single-room plane world and multi-room maze world which are connect by doors.
\begin{enumerate}[leftmargin=*]
    \item Trajectory planning: 
    \begin{itemize}
        \item Plane world: GoToLocalS7N5 (7$\times$7), GoToLocalS8N7 (8$\times$8)
        \item Maze world: GoToObjMazeS4 (10$\times$10), GoToObjMazeS7 (19$\times$19)
    \end{itemize}
    \item Instruction completion:
    \begin{itemize}
        \item Plane world: PickUpLoc (8$\times$8)
        \item Maze world: GoToObjMazeS4R2Close (7$\times$7), GoToSeqS5R2 (9$\times$9)
    \end{itemize}
\end{enumerate}

The Table~\ref{tbl:babyai_env_setting} presents the detailed BabyAI environment settings including the environment size, the number of rooms, the number of obstacles, the status of doors and one example instruction in that environment.
\begin{table*}[h]
\centering
\small
\scalebox{0.8}{
\begin{tabular}{lrrrrc}
\toprule
\multicolumn{1}{c}{\bf Env} & \multicolumn{1}{c}{\bf Size} & \multicolumn{1}{c}{\bf \# Room} & \multicolumn{1}{c}{\bf \# Obs} & \multicolumn{1}{c}{\bf Door} & \multicolumn{1}{c}{\bf Inst} \\ 
\midrule
GoToLocalS7N5 & $7\times7$ & 1 & $5$  & $-$ & go to the green key \\ 
GoToLocalS8N7   &  $8\times8$ & 1& $7$ & $-$ & go to the blue box  \\
GoToObjMazeS4    & $10\times10$& 9 & $1$ & $9$ ~$Open$ & go to the blue key \\ 
GoToObjMazeS7 & $19\times19$& 9 & $1$ & $9$ ~$Open$ &  go to the grey ball \\ 
GoToObjMazeS4R2Close & $7\times7$& 4 & $1$ & $3$~ $Closed$ & go to the blue ball \\ 
PickUpLoc   &  $8\times8$& 1 & $8$ & $-$ & pick up the yellow ball \\
GoToSeqS5R2    & $9\times9$& 4 & $4$ & $3$ ~$Closed$ &  \makecell[c]{go to the green door and go to the green door,\\then go to a red door and go to the green door}  \\ 
\bottomrule
\end{tabular}
}
\caption{
BabyAI environment setting details and example instruction.}
\label{tbl:babyai_env_setting}
\end{table*}

\subsection{Network Details} \label{appendix:atari_hyperparameters}

We build our \netName implementation based on Decision Transformer (\url{https://github.com/kzl/decision-transformer}) and exploit the instruction encoder from the BabyAI agent model (\url{https://github.com/mila-iqia/babyai/blob/iclr19/babyai/model.py}). 
In detail, we use the Gated Recurrent Units (GRU) encoder to process the instruction and then apply ExpertControllerFiLM (inspired by FiLMedBlock from \url{https://arxiv.org/abs/1709.07871})to fuse the instruction embedding with state embedding. \hongyi{For all our experiments we use bidirectional mask in transformer attention layer, except for Atari where we found casual attention to perform better.} The full list of hyperparameters can be found in Table \ref{tbl:babyai_hyperparameters} and Table \ref{tbl:atari_hyperparameters}, most of the hyperparameters are taken from Decision Transformer and BabyAI agent model. 


\begin{table*}[ht]
\vskip 0.15in
\begin{center}
\begin{small}
\begin{tabular}{ll}
\toprule
\textbf{Hyperparameter} & \textbf{Value}  \\
\midrule
Number of layers & $3$  \\ 
Number of attention heads    & $4$  \\
Embedding dimension    & $128$  \\ 
Batch size   & $64$\\ 
Image Encoder & nn.Conv2d\\
Image Encoder channels & $128, 128$\\
Image Encoder filter sizes & $2 \times 2, 3 \times 3$\\
Image Encoder maxpool strides & $2, 2$ (Image Encoder may vary a little \\ & depending on the environment size)\\
Instruction Encoder & nn.GRU\\
Instruction Encoder channels & $128$\\
State Encoder & nn.Linear\\
State Encoder channels & $256, 256, 128$\\
Max epochs & $200$ \\
Dropout & $0.1$ \\
Learning rate & $6*10^{-4}$ \\
Adam betas & $(0.9, 0.95)$ \\
Grad norm clip & $1.0$ \\
Weight decay & $0.1$ \\
Learning rate decay & Linear warmup and cosine decay (see code for details) \\
\bottomrule
\end{tabular}
\caption{Hyperparameters of \netName for BabyAI experiments.}
\label{tbl:babyai_hyperparameters}
\end{small}
\end{center}
\end{table*} 


% not using CQL numbers for conditioning
\begin{table}[ht]
\vskip 0.15in
\begin{center}
\begin{small}
\begin{tabular}{ll}
\toprule
\textbf{Hyperparameter} & \textbf{Value}  \\
\midrule
Number of layers & $6$  \\ 
Number of attention heads    & $8$  \\
Embedding dimension    & $128$  \\ 
Batch size   & $64$ Breakout, Qbert\\ 
            & $128$ Seaquest\\ 
            & $256$ Pong\\ 
Image Encoder & nn.Conv2d\\
Image Encoder channels & $32, 64, 64$\\
Image Encoder filter sizes & $8 \times 8, 4 \times 4, 3 \times 3$\\
Image Encoder strides & $4, 2, 1$ \\
Max epochs & $10$ \\
Dropout & $0.1$ \\
Learning rate & $6*10^{-4}$ \\
Adam betas & $(0.9, 0.95)$ \\
Grad norm clip & $1.0$ \\
Weight decay & $0.1$ \\
Learning rate decay & Linear warmup and cosine decay (see code for details) \\
\bottomrule
\end{tabular}
\end{small}
\caption{Hyperparameters of \netName for Atari experiments.}
\label{tbl:atari_hyperparameters}
\end{center}
\end{table} 

\subsection{Baseline Models}
\paragraph{BabyAI Baseline Models}
We ran BCQ and IQL based on the following implementation
\begin{center}
{
    \small
    \href{https://github.com/sfujim/BCQ}
    {\texttt{https://github.com/sfujim/BCQ}}.
}
\end{center}
\begin{center}
{
    \small
    \href{https://github.com/BY571/Implicit-Q-Learning/tree/main/discrete_iql}
    {\texttt{https://github.com/BY571/Implicit-Q-Learning/tree/main/discrete\_iql}}.
}
\end{center}
For BC and DT, we use the author's original implementation
\begin{center}
{
    \small
    \href{https://github.com/kzl/decision-transformer}
    {\texttt{https://github.com/kzl/decision-transformer}}.
}
\end{center}
\hongyi{For PlaTe, we use the author's original implementation
\begin{center}
{
    \small
    \href{https://github.com/Jiankai-Sun/plate-pytorch}
    {\texttt{https://github.com/Jiankai-Sun/plate-pytorch}}.
}
\end{center}
For MOPO, we use the author's original implementation of dynamic model training and policy learning. For RL policy, we adopt the IQL discussed above.
\begin{center}
{
    \small
    \href{https://github.com/tianheyu927/mopo}
    {\texttt{https://github.com/tianheyu927/mopo}}.
}
\end{center}}
The actor network and policy network of BCQ and IQL use the transformer architecture which is the same as architecture in our model, see details above. The original DT and BC already use the transformer architecture so we didn't change. For all baselines, we add the same instruction encoder and image encoder described above to process instruction and image observations. 


\paragraph{Atari Baseline Models} The scores for DT, BC, CQL, QR-DQN, and REM in Table \ref{tbl:atari_main} can be found in \citet{chen2021dt}.


\subsection{Experiment details}\label{app:ExpDetails}
\label{appendix:experiment}

\vspace{3pt}
\myparagraph{BabyAI}

For \netName, the larger size environment requires longer horizon $T$ and correspondingly more sampling iterations $N$. After $N$ iteration, all $T$ planned actions will be executed. For DT model, it's beneficial of using a longer context length in more complex environments as shown in its original paper \citep{chen2021dt}. We list out these parameters for \netName and DT models in Table~\ref{tbl:babyai_experiment_setting}. We didn't use context information in \netName in most BabyAI environments as we expect the iterative planning could generate a correct trajectory based solely on the current state observation. While the GoToSeqS5R2 environment requires go to a sequence of objects in correct order and \netName needs to remember what objects have been visited from the context. During training, we randomly select and mask one action in a trajectory. 

\begin{table*}[h]
\centering
\small
\scalebox{0.8}{
\begin{tabular}{ccccc}
\toprule
& \multicolumn{3}{c}{\bf \netName} & \multicolumn{1}{c}{\bf DT} \\ 
\multicolumn{1}{c}{\bf Env} & \multicolumn{1}{c}{\bf context} & \multicolumn{1}{c}{\bf plan } & \multicolumn{1}{c}{\bf sample iteration} & \multicolumn{1}{c}{\bf context} \\
\midrule
GoToLocalS7N5 & 0 & 5 & 10  & 5 \\ 
GoToLocalS8N7   &  0 & 5 & 10  & 5 \\
GoToObjMazeS4    & 0& 10 & 30 & 10 \\ 
GoToObjMazeS7 & 0& 15 & 50 & 15\\ 
GoToObjMazeS4R2Close & 0& 5 & 10 & 5\\ 
PickUpLoc & 0& 5 & 10 & 5\\ 
GoToSeqS5R2    & 20 & 5 & 10 & 20 \\ 
\bottomrule
\end{tabular}
}
\caption{
BabyAI environment experiment details for \netName and DT.}
\label{tbl:babyai_experiment_setting}
\end{table*}

The input to DT model includes the instruction, state context sequence, action context sequence and return-to-go sequence in which the target reward is set to 1 initially. The input to other baseline models are the same except they use normal reward sequence instead of return-to-go sequence. While \netName only use the instruction, state context sequence and action context sequence.
Inside state sequence, each state $\bs_n$ contains the $[x,y,dir,g_x,g_y]$ meaning the agent's x position, y position, direction and goal object's x position, y position (if the goal location is available).

\myparagraph{Atari}
\vspace{3pt}
In dynamically changing Atari environment, \netName use context information in all four games and only execute the first planned action to avoid the unexpected changes in the world, see details in Table~\ref{tbl:atari_experiment_setting}. During training, we randomly  sample and mask one action in a trajectory. 

%\ycReb{To deal with varied quality of demonstrations for training, we added the return-to-go, the remaining reward at each step towards the achieved final reward following DT , as input to \netName. We test \netName without the reward input, which results in deteriorate performance as shown in Table. \ref{tbl:LEAP_reward}.}



\begin{table*}[h!]
    \centering
    \small
    % \hspace{-35pt}
    \begin{minipage}[t]{0.45\linewidth}
        \centering\color{black}
        % \setlength\tabcolsep{1.5pt} 
        \scalebox{1.0}{
        \begin{tabular}{cccc}
        \toprule
        \multicolumn{1}{c}{\bf Env} & \multicolumn{1}{c}{\bf context} & \multicolumn{1}{c}{\bf plan } & \multicolumn{1}{c}{\bf sample iteration}\\
        \midrule
        Breakout & 25 & 5 & 10  \\ 
        Qbert   &  25 & 5 & 10 \\
        Pong    & 25& 5& 10 \\ 
        Seaquest & 25& 10 & 30\\ 
        \bottomrule
        \end{tabular}
        }
        \caption{
        Atari environment experiment details for \netName.}
        \label{tbl:atari_experiment_setting}
    \end{minipage}
    \hfill
    \begin{minipage}[t]{0.45\linewidth}
        % \hspace{-35pt}
        \centering\color{black}
        \scalebox{1.0}{
            \begin{tabular}{ccc}
            \toprule
            \multicolumn{1}{c}{\bf Env} & \multicolumn{1}{c}{\bf w/o return} & \multicolumn{1}{c}{\bf w return}\\
            \midrule
            Breakout & 182.0 & 378.9  \\ 
            Qbert   &  41.0 & 19.6 \\
            Pong    & 100.7& 108.9 \\ 
            Seaquest & 0.5& 1.3\\ 
            \bottomrule
            \end{tabular}
        }
        \captionsetup{labelfont={color=black},font={color=black}}
        \caption{\netName performance in Atari environment with and w/o return input.}
        \label{tbl:LEAP_reward}
    \end{minipage}
    \vspace{-0pt}
\end{table*}

\hongyi{
Note that our approach can easily be conditioned on total reward, by simply concatenating the reward as input in the sequence model. One hypothesis is that when demonstration set contained trajectories of varying quality, taking reward as input following will enable the model to recognize the quality of training trajectories and potentially improve the performance. To further validate the importance of the rewards, we test the \netName with and without return-to-go inputs, which sum of future rewards \cite{chen2021dt}. The results show that the performance degradation without the return-to-go inputs, which is shown in Table \ref{tbl:LEAP_reward}.}


\section{\ycReb{Stochastic Environment Testing}}
\label{appendix:stochastic}
\ycReb{
In this section we demonstrate the possibility of extending our method into stochastic settings.
Although \cite{dtStoch} reveals that planning by sampling from the learnt policy conditioned on desired reward could lead to suboptimal outcome due to the existence of stochastic factors,
our model circumvents the problem by formulating the planning as an optimization problem - we use the Gibbs sampling method to find the trajectory with the lowest energy evaluated by the trained model.
Assuming that the frequency of successful actions dominates in the dataset, our model is trained to assign lower energy to trajectories with higher likelihood of reaching the goal.
Consequently, in the stochastic environments, \netName constructs a sequence of actions that has the best opportunity to accomplish the target.
When executing this plan in a stochastic environment, we may also choose to replan our sequence of actions after each actual action in the environment (to deal with stochasticity of the next state given an action). Note then that this sequence of actions will be optimal in the stochastic environment, as we always choose the action that has the maximum likelihood of reaching the final state. 
Also note that multi-step planning can potentially provide advantage over a simple policy to predict the next action in stochastic environments, as such policy simply assigns probability distribution to the immediate next step without the awareness of future step adjustments facing stochastic factors.
}

\ycReb{
To verify the assumptions, we constructed a stochastic testing in BabyAI environment.
The test is created by adopting a stochastic dynamic model, where the agent fails to execute the turning actions \textit{turn left/right} with $20\%$ chance, and instead performs the remaining actions, including \textit{turn right/left}, \textit{forward}, \textit{pickup}, \textit{drop}, and \textit{open}, with uniform probability.
The remaining settings follow BabyAI experiments detailed in Appendix. \ref{app:ExpDetails}, except that we train models using demonstrations generated with the above dynamic model.
Those training data are noisy in the sense that the actions taken are not optimal, and corrections are required from future actions.
We believe \netName, as a multi-step planner, can learn the above correlations between the consecutive actions. 
We compare \netName with the baseline DT, the results of which is collected in the Table. \ref{tbl:stoch_babyai}.
It can be observed that \netName has a superior performance compared to DT on both tested environments,
which indicates both the possibility of applying our approach in the stochastic settings, and the advantage of multi-step planning when facing stochastic factors. 
}


\begin{table*}[h!]
    \centering\color{black}
    \small
        \setlength\tabcolsep{20pt} 
        \scalebox{1.0}{
            \begin{tabular}{ccc}
            \toprule
            \multicolumn{1}{c}{\bf Env} & \multicolumn{1}{c}{\bf \netName} &  \multicolumn{1}{c}{\bf DT}\\
            \midrule
            GoToObjMazeS4 & \textbf{57.5\%} & 30.8\%   \\ 
            GoToObjMazeS7   &  \textbf{33.3\%} & 28.3\%
            \\
            \bottomrule
            \end{tabular}
        }
        \captionsetup{labelfont={color=black},font={color=black}}
        \caption{Comparison of \netName and DT on stochastic settings}
        \label{tbl:stoch_babyai}
   
    \vspace{-0pt}
\end{table*}




\section{\ycReb{Ablation on Energy model and Optimization method}}

\ycReb{We further ablate on our design choices, including the energy model and sampling methods.
We consider Masked Language Model (MLM) and sequence model classifier as the energy model, and random-shooting, Cross-Entropy Method (CEM), and Gibbs Sampling as the optimization approach. 
All combinations are tested, for which the results are collected in Table~\ref{tbl:samplings}.
We observe that the Gibbs sampling gives the best outcome with MLM model and that defining an energy value using a sequence model classifier doesn't work well in all settings.}

\begin{enumerate}[leftmargin=*]
    \item \hongyi{Sampling methods:} 
    \begin{itemize}
        \item \hongyi{Random-shooting: randomly generated 30 action sequences and pick up the one with lowest estimated energy value.}
        \item \hongyi{Cross-Entropy method: randomly generated 30 action sequences, keep the three best sequences with lowest estimated energy values in each iteration. Then we randomly update one action token in the elite sequences to get 30 new sequences for next iteration. The sequence with lowest energy is selected in the final iteration.}
        \item \hongyi{Gibbs sampling: discussed in the main text.}
    \end{itemize}
    \item \hongyi{Energy models:} 
    \begin{itemize}
        \item \hongyi{Sequence model classifier: LSTM sequence model predicts the scalar energy value given the entire trajectory $\btau{}$, and train the loss between ground truth trajectory energy and estimated energy. The optimal trajectories in Babyai are assigned with lowest energy value 0 and the generates suboptimal trajectories are assigned with higher values depending on the degree of randomness.}
        \item \hongyi{MLM: discussed in the main text.}
    \end{itemize}
\end{enumerate}


\begin{table*}[t]
    \centering
    \small
    % \hspace{-35pt}
        \scalebox{1.0}{
            \color{black}\begin{tabular}{cccc}
            \toprule
            \multicolumn{1}{c}{\bf Energy model} & \multicolumn{1}{c}{\bf Random-shooting} &  \multicolumn{1}{c}{\bf CEM} &  \multicolumn{1}{c}{\bf Gibbs sampling}\\
            \midrule
            MLM & 23.3\% & 57.5\%  & \textbf{62.5\%}\\ 
            Classifier & 25.0\% & 12.5\%  & 15.5\% 
            \\
            \bottomrule
            \end{tabular}
        }
        \captionsetup{labelfont={color=black},font={color=black}}
        \caption{Comparison of different combination between energy models and sampling methods}
        \label{tbl:samplings}
    \vspace{-0pt}
\end{table*}