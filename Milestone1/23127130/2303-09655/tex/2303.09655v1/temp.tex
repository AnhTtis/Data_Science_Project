

\author{\IEEEauthorblockN{Vani Nagarajan}
\IEEEauthorblockA{\textit{Electrical and Computer Engineering} \\
\textit{Purdue University}\\
West Lafayette, USA \\
nagara16@purdue.edu}
\and
\IEEEauthorblockN{Milind Kulkarni}
\IEEEauthorblockA{\textit{Electrical and Computer Engineering} \\
\textit{Purdue University}\\
West Lafayette, USA \\
milind@purdue.edu}
}

%Eval Fig 7
\begin{subfigure}[b]{0.43\textwidth}
         \centering
         \includegraphics[width=\textwidth]{figures/3diono-scalable-num-intersections.pdf}
         \caption{Impact of dataset size on number of intersections}
         \label{fig:num-intersections}
     \end{subfigure}
     \caption{Scalability of results}
     \label{fig:scalability}

     II B 2):The Optix API \cite{prog-guide} provides programmable kernels (programs) to accelerate ray tracing applications by leveraging the RT cores in GPUs. These programs allow the user to write customize shader programs.

%ref.bib
@article{Evangelou2021RadiusSearch,
  author =       {I. Evangelou and G. Papaioannou and K. Vardis and A. A. Vasilakis}, 
  title =        {Fast Radius Search Exploiting Ray Tracing Frameworks},
  year =         2021,
  month =        {February},
  day =          5,
  journal =      {Journal of Computer Graphics Techniques (JCGT)},
  volume =       10,
  number =       1,
  pages =        {25--48},
  url =          {http://jcgt.org/published/0010/01/02/},
  issn =         {2331-7418}
} 

@inproceedings {wald19,
booktitle = {High-Performance Graphics - Short Papers},
editor = {Steinberger, Markus and Foley, Tim},
title = {{RTX Beyond Ray Tracing: Exploring the Use of Hardware Ray Tracing Cores for Tet-Mesh Point Location}},
author = {Wald, Ingo and Usher, Will and Morrical, Nathan and Lediaev, Laura and Pascucci, Valerio},
year = {2019},
publisher = {The Eurographics Association},
ISSN = {2079-8687},
ISBN = {978-3-03868-092-5},
DOI = {10.2312/hpg.20191189}
}

%NGSIM fig
\begin{figure*}
     \centering
     \begin{subfigure}[b]{0.45\textwidth}
         \centering
         \includegraphics[scale=0.5]{figures/eps-ngsim.pdf}
         \vspace{-0.5em}
         \caption{Speedup on varying search radius ($\varepsilon$)}
         \label{fig:ngsim_eps}
     \end{subfigure}    
     \hfill
     \begin{subfigure}[b]{0.45\textwidth}
         \centering
         \includegraphics[scale=0.5]{figures/datasize-ngsim.pdf}
         \vspace{-0.5em}
         \caption{Speedup on varying dataset size}
         \label{fig:ngsim_ds}
     \end{subfigure}
     \caption{Speedup over FDBSCAN on varying $\varepsilon$ and dataset size for NGSIM}
     \label{fig:ngsim}
     \vspace{-0.5em}
\end{figure*}

%NGSIM Table
\begin{table}[h]
%\centering
    \begin{tabular}{|c| c| c| c|} 
    \hline
       \textbf{Dataset size } & \textbf{FDBSCAN(s)} & \textbf{RT-DBSCAN(s)} & \textbf{Speedup} \\
       \hline
       500K & 12.7 & 0.03 & 423.33\\
       \hline
       1M & 72.8 & 0.06 & 1213.33 \\
       \hline
       2M & 364.6  & 0.13 & 2804.61\\
       \hline
       4M & 1631.4 & 0.3 & 5438\\
       \hline
       8M & 6964.1 & 1.26 & 5527.06\\
       \hline
   \end{tabular}
   \caption{Execution time (in seconds) for NGSIM dataset on varying dataset size}
    \label{table:ngsim-dataset-size}
     \vspace{-1.5em}
\end{table}

\begin{table}[h]
%\centering
    \begin{tabular}{|c| c| c| c|} 
    \hline
    \textbf{Search radius ($\varepsilon$) } & \textbf{FDBSCAN(s)} & \textbf{RT-DBSCAN(s)} & \textbf{Speedup} \\         
       \hline
        0.0001 & 64.72 & 0.0257 & 2518.28\\
       \hline
        0.00025 & 64.77 & 0.0259 & 2500.77\\
       \hline
        0.0005 & 64.74 & 0.0259 & 2499.61\\
       \hline
        0.00075 & 64.71 & 0.026 & 2488.84\\
       \hline
       0.001 & 64.74 & 0.0259 & 2499.61\\
       \hline
   \end{tabular}
   \caption{Execution time (in seconds) for NGSIM dataset on varying search radius ($\varepsilon$)}
    \label{table:ngsim-eps}
     \vspace{-1.5em}
\end{table}

%BVH image
\begin{figure}
    \centering
    \includegraphics[width=\linewidth]{figures/bvh_img.png}
    \caption{(a) 2D rectangular bounding boxes for the objects and intermediate bounding volumes (b) Bounding Volume Hierarchy built from the bounding boxes in (a)}
    \label{fig:bvh_img}
    \vspace{-1.6em}
\end{figure}