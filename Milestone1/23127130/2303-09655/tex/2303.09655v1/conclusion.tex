In this work, we implemented RT-DBSCAN, where we accelerated the nearest neighbor searches in DBSCAN using Ray Tracing cores. We found that the hardware acceleration led to performance improvements by as much as 4.5x over current state-of-the-art GPU-based DBSCAN implementations. Future work entails removing the fixed-radius constraint for neighbor searches to accelerate a wider range of applications. It would also be interesting to see if RT cores can be used to accelerate more general tree traversal algorithms.
%more general, distance-based problems. We are now able to solve {\em fixed-radius} nearest neighbor problems efficiently and a natural extension is to add support for cases where the search radius is not known. This would open up opportunities to accelerate a wider range of applications such as k-nearest neighbors, k-means clustering etc... We are also working on accelerating spatial queries in databases. It would also be interesting to see if RT cores can be used to accelerate more general tree traversal algorithms.