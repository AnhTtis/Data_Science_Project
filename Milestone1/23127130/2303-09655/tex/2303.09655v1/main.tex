\documentclass[conference]{IEEEtran}
\IEEEoverridecommandlockouts
% The preceding line is only needed to identify funding in the first footnote. If that is unneeded, please comment it out.
\usepackage{cite}
\usepackage{amsmath,amssymb,amsfonts}
\usepackage{algorithmic}
\usepackage{graphicx}
\usepackage{textcomp}
\usepackage{xcolor}
\usepackage{caption}
\usepackage{subcaption}
\usepackage[algo2e,ruled,linesnumbered]{algorithm2e}
\usepackage{balance}
\usepackage{booktabs}
% \def\BibTeX{{\rm B\kern-.05em{\sc i\kern-.025em b}\kern-.08em
%     T\kern-.1667em\lower.7ex\hbox{E}\kern-.125emX}}

\usepackage[english]{babel} %for definition environment
\usepackage{amsthm}
\theoremstyle{definition}
\newtheorem{definition}{Definition}[section]

\usepackage{enumitem} %for description environment

\newcommand{\vani}[1]{\textcolor{blue}{\sl{\bf Vani:}#1}}
\newcommand{\milind}[1]{\textcolor{red}{\sl{\bf Milind:}#1}}
\newcommand{\krish}[1]{\textcolor{teal}{\sl{\bf Krish:}#1}}

\newcommand{\eg}{\textit{e.g.,}}
\newcommand{\ie}{\textit{i.e.,}}
\newcommand{\etal}{\textit{et. al. }}


\begin{document}

\title{RT-DBSCAN: Accelerating DBSCAN using Ray Tracing Hardware\\
%\thispagestyle{plain}
%\pagestyle{plain}
\thanks{We thank the anonymous IPDPS reviewers for their valuable feedback. We thank Dr. Eleazar Leal for providing the G-DBSCAN code and Dr. Andrey Prokopenko for answering our questions about FDBSCAN. We are grateful to Kirshanthan Sundararajah for his insightful comments that helped improve the paper. This work was funded by NSF grants CCF-1908504, CCF-1919197 and CCF-2216978.}
}
\author{\IEEEauthorblockN{Vani Nagarajan}
\IEEEauthorblockA{
\textit{Purdue University, USA} \\
nagara16@purdue.edu}
\and
\IEEEauthorblockN{Milind Kulkarni}
\IEEEauthorblockA{
\textit{Purdue University, USA} \\
milind@purdue.edu}
}
\thispagestyle{plain}
\pagestyle{plain}
\maketitle
\begin{abstract}
General Purpose computing on Graphical Processing Units (GPGPU) has resulted in unprecedented levels of speedup over its CPU counterparts, allowing programmers to harness the computational power of GPU shader cores to accelerate other computing applications. But this style of acceleration is best suited for {\em regular} computations (e.g., linear algebra). Recent GPUs feature new {\em Ray Tracing (RT)} cores that instead speed up the irregular process of ray tracing using Bounding Volume Hierarchies. While these cores seem limited in functionality, they can be used to accelerate n-body problems by leveraging RT cores to accelerate the required distance computations. In this work, we propose RT-DBSCAN, the first RT-accelerated DBSCAN implementation. We use RT cores to accelerate Density-Based Clustering of Applications with Noise (DBSCAN) by translating fixed-radius nearest neighbor queries to ray tracing queries. We show that leveraging the RT hardware results in speedups between 1.3x to 4x over current state-of-the-art, GPU-based DBSCAN implementations.
\end{abstract}

\begin{IEEEkeywords}
DBSCAN, clustering, ray tracing
\end{IEEEkeywords}

\section{Introduction}
\section{Introduction}
\label{sec:introduction}
% \begin{itemize}
%     % Diffusion of FL
%     \item {\st{Diffusion of FL}}
%     % Security threats to FL
%     \item {\st{Security threats to FL with particular focus on model poisoning}}
%     % Limitations of existing countermeasures
%     \item {\st{Current countermeasures (e.g., KRUM) and their limitations}}
%     % Proposed method and its advantages
%     \item {\st{Intuitive description of the proposed method and its difference (i.e., advantages) w.r.t. state of the art}}
%     % Main contributions
%     \item {\st{Summary of the main contributions of this work}}
%     % Paper's structure and organization
%     \item {\st{Paper's structure and organization}}
% \end{itemize}

% Diffusion of FL
Recently, {\em federated learning} (FL) has emerged as the leading paradigm for training distributed, large-scale, and privacy-preserving machine learning (ML) systems~\cite{mcmahan2017googleai,mcmahan2017aistats}. 
The core idea of FL is to allow multiple edge clients to collaboratively train a shared, global model without disclosing their local private training data.
%Specifically, an FL system consists of a central server and many edge clients; 
A typical FL round involves the following steps: {\em(i)} the server randomly picks some clients and sends them the current, global model; {\em(ii)} each selected client locally trains its model with its own private data; then, it sends the resulting local model to the server;\footnote{Whenever we refer to global/local model, we mean global/local model {\em parameters}.} {\em(iii)} the server updates the global model by computing an \emph{aggregation function}, usually the average (FedAvg), on the local models received from clients.
% \begin{enumerate}
%     \item[{\em(i)}] the server sends the current, global model to the clients and appoints some of them for training;
%     \item[{\em(ii)}] each selected client locally trains its copy of the global model with its own private data; then, it sends the resulting local model back to the server;\footnote{Whenever we refer to global/local model, we mean global/local model {\em parameters}.}
%     \item[{\em(iii)}] the server updates the global model by computing an \emph{aggregation function} on the local models received from clients (by default, the average, also referred to as FedAvg~\cite{mcmahan2017aistats}).
% \end{enumerate}
This process goes on until the global model converges. %(e.g., after a certain number of rounds or other similar stopping criteria).
%\\
% The advantages of FL over the traditional, centralized learning paradigm are undoubtedly clear in terms of flexibility/scalability (clients can join/disconnect from the FL network dynamically), network communications (only model weights\footnote{We will use \textit{parameters} and \textit{weights} interchangeably.} are exchanged between clients and server), and privacy (each client's private training data is kept local at the client's end and not uploaded to the server).
\\
% Security threats to FL
%However, the growing adoption of FL also raises security concerns~\cite{costa2022covert}, particularly about its confidentiality, integrity, and availability.
Although its advantages over standard ML, FL also raises security concerns~\cite{costa2022covert}. %, particularly about its confidentiality, integrity, and availability~\cite{costa2022covert}.
% OLD, LONG VERSION
% Indeed, some work deals with privacy leakage that may expose the local data of some clients~\cite{melis2019sp}. 
% A large body of work, instead, investigates attacks that usually aim to detriment the predictive accuracy of the learned global model. For instance, \emph{data poisoning} attacks achieve this goal by letting an adversary pollute the training set of some corrupt FL clients with maliciously crafted examples~\cite{jagielski2018sp}.
% Similarly, in \emph{model poisoning} the attacker attempts to tweak the global model weights~\cite{bhagoji2019pmlr} by directly perturbing the local model's weights of some infected FL clients before these are sent to the central server for aggregation, usually via so-called Byzantine attacks. 
% It turns out that Byzantine model poisoning attacks severely impact standard FedAvg; therefore, more robust aggregation functions must be designed to make FL systems secure.
Here, we focus on \emph{untargeted model poisoning} attacks~\cite{bhagoji2019pmlr}, where an adversary attempts to tweak the global model weights %\footnote{We will use the terms \textit{parameters} and \textit{weights} interchangeably.} 
by directly perturbing the local model's parameters of some infected clients before these are sent to the central server for aggregation.
In doing so, the adversary aims to jeopardize the global model \textit{indiscriminately} at inference time.
Such model poisoning attacks severely impact standard FedAvg; therefore, more robust aggregation functions must be designed to secure FL systems.
\\
% In this paper, we focus on designing a novel robust aggregation scheme at the server's end to contrast the effect of Byzantine model poisoning attacks.
%
% Current countermeasures and their limitations
%Several countermeasures have been proposed in the literature to combat model poisoning attacks on FL systems.
% Some methods use simple statistics more robust than plain average to smooth the impact of malicious updates (e.g., Trimmed Mean and FedMedian~\cite{yin2018icml}). 
% Other defenses implement outlier detection techniques to discard malicious updates from the aggregation performed at the server's end. Those are either based on heuristics (e.g., Krum/Multi-Krum~\cite{blanchard2017nips} and Bulyan~\cite{mhamdi2018pmlr}) or data-driven approaches (e.g., K-means clustering~\cite{shen2016acm} or DnC via spectral analysis~\cite{shejwalkar2021ndss}). 
% Finally, some strategies rely on a centralized ``source of trust'' to spot potential malicious updates (e.g., FLTrust~\cite{cao2020fltrust}).
% Several countermeasures have been proposed in the literature to combat model poisoning attacks on FL systems, i.e., to discard possible malicious local updates from the aggregation performed at the server's end. 
% These techniques range from simple statistics more robust than plain average (e.g., Trimmed Mean and FedMedian~\cite{yin2018icml}) to outlier detection heuristics (e.g., Krum/Multi-Krum~\cite{blanchard2017nips} and Bulyan~\cite{mhamdi2018pmlr}) or data-driven approaches (e.g., spectral analysis via K-means clustering~\cite{shen2016acm} or spectral analysis), or methods based on ``source of trust'' (e.g., FLTrust~\cite{cao2020fltrust}).
% OLD, LONG VERSION
%Several countermeasures have been proposed in the literature to combat Byzantine model poisoning attacks on FL systems.
% Descriptive statistics
% For example, Trimmed Mean and FedMedian aggregate local model updates using more robust statistics than standard average~\cite{yin2018icml}.
%
% % Heuristics for outlier detection
% Many existing Byzantine-resilient strategies implement some outlier detection heuristics to discard the model updates sent by potentially malicious clients from the input of the aggregation function.
% One of the most popular heuristics is Krum~\cite{blanchard2017nips}.
% This strategy tries to mitigate the impact of Byzantine attacks by selecting as a global model the local model with the smallest sum of Euclidean distances to {\em all} the other local models.
% Although powerful, Krum requires the server to know (or, at least, estimate) the number of malicious FL clients upfront, which is generally impossible in a realistic attack scenario. %
% Moreover, Krum may become ineffective for complex, high-dimensional model parameter spaces due to the curse of dimensionality.
% Bulyan~\cite{mhamdi2018pmlr} tries to overcome this issue by combining Krum with a variant of Trimmed Mean.
% % Data-driven outlier detection
% Other strategies use data-driven outlier detection techniques -- e.g., via K-means clustering~\cite{shen2016acm} -- to spot potential malicious local model updates. 
% %For instance, Shen et al. propose to cluster local model updates with K-means and thus identify outliers.
%
% % Other techniques
% As far as the server is concerned, any local model received can be from a potential malicious client. 
% FLTrust~\cite{cao2020fltrust} assumes the server acts as a client, i.e., trains a local model on an additional {\em trustworthy} dataset at the server's end and compares it against all the local models from other clients. 
% This way, the server can rely on some ``source of trust'' when discarding potentially malicious clients.
%\\
% Limitations of existing Byzantine-resilient strategies
Unfortunately, existing defense mechanisms either rely on simple heuristics (e.g., Trimmed Mean and FedMedian by~\cite{yin2018icml}) or need strong and unrealistic assumptions to work effectively (e.g., foreknowledge or estimation of the number of malicious clients in the FL system, as for Krum/Multi-Krum~\cite{blanchard2017nips} and Bulyan~\cite{mhamdi2018pmlr}, which, however, cannot exceed a fixed threshold).
Furthermore, outlier detection methods using K-means clustering~\cite{shen2016acm} or spectral analysis like DnC~\cite{shejwalkar2021ndss} do not directly consider the temporal evolution of local model updates received.
Finally, strategies like FLTrust~\cite{cao2020fltrust} require the server to collect its own dataset and act as a proper client, thereby altering the standard FL protocol.
\\
% OLD, LONG VERSION
% Overall, existing Byzantine-resilient strategies are either simple heuristics (e.g., FedMedian) or, if they are more complex, they rely on strong and unrealistic assumptions to work effectively (e.g., knowing the number of malicious clients in the FL system in advance, as for Krum and alike).
% Furthermore, data-driven outlier detection methods do not consider the temporary evolution of local model updates received (e.g., K-means clustering). 
% Finally, strategies like FLTrust requires the server to collect its own dataset and act as a proper client, thereby altering the standard FL protocol.
%
% Description of the proposed method
This work introduces a novel pre-aggregation \textit{filter} robust to untargeted model poisoning attacks. Notably, this filter $(i)$ operates without requiring prior knowledge or constraints on the number of malicious clients and $(ii)$ inherently integrates temporal dependencies. 
The FL server can employ this filter as a preprocessing step before applying \textit{any} aggregation function, be it standard like FedAvg or robust like Krum or Bulyan.
Specifically, we formulate the problem of identifying corrupted updates as a multidimensional (i.e., matrix-valued) time series anomaly detection task. 
The key idea is that legitimate local updates, resulting from well-calibrated iterative procedures like stochastic gradient descent (SGD) with an appropriate learning rate, show \textit{higher predictability} compared to malicious updates. This hypothesis stems from the fact that the sequence of gradients (thus, model parameters) observed during legitimate training exhibit regular patterns, as validated in Section~\ref{subsec:intuition}. %until convergence. 
%This regularity may be more pronounced for smooth convex loss functions, but it can still be captured within an appropriate time window, even for more complex and convoluted loss surfaces. 
%We provide evidence of this claim in Appendix~B, where we show that the average mutual information (i.e., ``predictability''), calculated over pairs of legitimate model updates sent at different FL rounds, is significantly higher than the corresponding computation for a malicious client.
\\
Inspired by the matrix autoregressive (MAR) framework for multidimensional time series forecasting~\cite{chen2021je}, we propose the FLANDERS ({\em \textbf{F}ederated \textbf{L}earning meets \textbf{AN}omaly \textbf{DE}tection for a \textbf{R}obust and \textbf{S}ecure}) filter.
The main advantages of FLANDERS over existing strategies like FLDetector~\cite{zhao2020multivariate} are its resilience to large-scale attacks, where $50\%$ or more FL participants are hostile, and the capability of working under realistic non-iid scenarios.
We attribute such a capability to two key factors: $(i)$ FLANDERS works without knowing a priori the ratio of corrupted clients, and $(ii)$ it embodies temporal dependencies between intra- and inter-client updates, quickly recognizing local model drifts caused by evil players. Below, we summarize our main contributions:

\begin{itemize}
\item[{\em(i)}]
We provide empirical evidence that the sequence of models sent by legitimate clients is more predictable than those of malicious participants performing untargeted model poisoning attacks.
\\
\item[{\em(ii)}] 
We introduce FLANDERS, the first pre-aggregation filter for FL robust to untargeted model poisoning based on multidimensional time series anomaly detection.
\\
\item[{\em(iii)}] 
We integrate FLANDERS into Flower,\footnote{\scriptsize{\url{https://flower.dev/}}} a popular FL simulation framework for reproducibility.
\\
\item[{\em(iv)}] 
We show that FLANDERS improves the robustness of the existing aggregation methods under multiple settings: different datasets, client's data distribution (non-iid), models, and attack scenarios.
\\
\item[{\em(v)}] 
We publicly release all the implementation code of FLANDERS along with our experiments.\footnote{\scriptsize{\url{https://anonymous.4open.science/r/flanders_exp-7EEB}}}
\end{itemize}

% Paper's structure and organization
The remainder of the paper is structured as follows. %some related work and the current state-of-the-art solutions to security issues that FL entails. 
Section~\ref{sec:background} covers background and preliminaries. 
In Section~\ref{sec:related}, we discuss related work.
Section~\ref{sec:problem} and Section~\ref{sec:method} describe the problem formulation and the method proposed. % to tackle it. 
Section~\ref{sec:experiments} gathers experimental results. %, and Section~\ref{sec:limitations} discusses some limitations of this work.
Finally, we conclude in Section~\ref{sec:conclusion}.
 %discusses the limitations of this work and draws future research directions.
%reports conclusions and draws perspectives for future research directions.

%%%%%%% OLD %%%%%%%
%to overcome the resilience of Byzantine failures in distributed Stochastic Gradient Descent computations. 
% The strength of Krum is its time complexity, which is linear in the gradient dimension. 
% However, the robustness of the approach is guaranteed for gradient-based learning applications only when the majority of the clients are not compromised. 
% Besides, the aggregation mechanism of Krum, as well as that of similar methods, is robust from a coarse-grained perspective and does not provide solutions to errors and perturbations that may occur at inference time.
%A related approach to~\cite{blanchard2017nips} is the work of Su et al.~\cite{su2016dc}. Here, the authors propose an iterated approximate agreement to tackle a multi-layer scenario attacked by Byzantine agents. 
%However, the method works efficiently on the sole discrete context and it is inapplicable to continuous state environments.
%\gabri{Maybe, we should just talk about the main limitations of existing countermeasures without digging into their details (or, we can just mention Krum as this is the most popular one). I will move the description of all these methods to the Related Work section.}

\section{Background}
\section{Background on Network Calculus}
\label{sec: background}


\begin{figure*}[tbh]
\centering
\begin{subfigure}[b]{0.3\textwidth}
    \centering
    \includegraphics[width=\linewidth]{images/in-out.png}
    \caption{Arrival and departure data and their relation with delay $d(t)$ and backlog $b(t)$. For a FIFO system, the delay is the horizontal distance between $R(t)$ and $R^*(t)$ but some other multiplexing techniques may shift the data to a later priority, causing a longer delay.}
    \label{fig: data in-out}
\end{subfigure}
\hfill
\begin{subfigure}[b]{0.35\textwidth}
    \centering
    \includegraphics[width=\linewidth]{images/arrival-service.png}
    \caption{Characteristics of an arrival curve and a service curve. From any point of observation, the arriving data never exceeds its arrival curve; the departure data is also never less than the service curve with respect to the data arrival.}
    \label{fig: arrival-service curves}
\end{subfigure}
\hfill
\begin{subfigure}[b]{0.33\textwidth}
    \centering
    \includegraphics[width=\linewidth]{images/bound.png}
    \caption{Delay and backlog bounds of a system. Backlog is the maximum vertical distance between $\alpha(t)$ and $\beta(t)$; FIFO delay is their maximum horizontal distance; but for arbitrary multiplexing, the delay guarantee is when the system clears its buffer, thus it's the intersection of $\alpha(t)$ and $\beta(t)$.}
    \label{fig: system bounds}
\end{subfigure}
\caption{Network calculus framework. We let $R(t)$ and $R^*(t)$ be the arrival and departure data flow of a system; $\alpha(t)$ be the piecewise linear concave arrival curve and $\beta(t)$ be the piecewise linear convex service curve of a system.}
% \hossein{Better to show piece-wise linear concave arrival curve and piece-wise linear convex service curve instead of token-bucket and rate-latency.}}
\end{figure*}

We recall some of the network calculus essentials for a better understanding of the framework used in Saihu. In the following context, we use the following notation: $\mbb{R}^+$ is the set of non-negative real numbers; $[x]_+$ denotes $\max(0, x)$

The data flow is by convention modeled as a left-continuous wide-sense increasing function $R(t): \mbb{R}^+ \mapsto \mbb{R}^+$ with respect to time $t$~\cite{ncbook2001leboudec}. 

A system $\mcal{S}$ receives arrival data described as a cumulative function $R(t)$ and delivers departure data as another cumulative function $R^*(t)$. Figure~\ref{fig: data in-out} illustrates such a system $\mcal{S}$. The benefit of representing a system like this is that we can observe system backlog and delay with such a model. 

\begin{definition}[Backlog and Delay~\cite{ncbook2001leboudec}]
    The backlog of a system at time~$t$ is
    \begin{equation}
        b(t) = R(t) - R^*(t)
    \end{equation}
    
    The virtual delay of a FIFO system at time $t$ is
    \begin{equation}
        d_{FIFO}(t) = \inf \lbp \tau \geq 0 : R(t) \leq R^*(t+\tau) \rbp
    \end{equation}
\end{definition}



The backlog of a system can be viewed as the vertical distance between $R$ and $R^*$. The FIFO (\textit{First-in First-out}) delay is the horizontal distance between $R$ and $R^*$. One may obtain other delay values if the multiplexing technique is not FIFO.

% \begin{figure}
%     \centering
%     \includegraphics[width=0.9\linewidth]{images/in-out.png}
%     \caption{In/out data flow; delay and backlog}
%     \label{fig: data in-out}
% \end{figure}

Since we are interested in the system guarantee instead of a single instance of data flow, we would like to have general bounds to the arrival and departure data flows. Therefore, we define \textit{arrival curve} and \textit{service curve} as the bounds of arrival and departure data flows.

\begin{definition}[Arrival Curve~\cite{ncbook2001leboudec}]
    Given a wide-sense increasing function $\alpha: \mbb{R}^+ \mapsto \mbb{R}^+$, we say that a flow $R(t)$ is $\alpha$-constrained if and only if for all $s \leq t$:
    \begin{equation}
        R(t) - R(s) \leq \alpha(t-s)
    \end{equation}
    We say $R(t)$ has $\alpha$ as an arrival curve.
\end{definition}

\begin{definition}[Service Curve~\cite{ncbook2001leboudec}]
    Given a wide-sense increasing function $\beta: \mbb{R}^+ \mapsto \mbb{R}^+$ and $\beta(0) = 0$. A system $\mcal{S}$ having $R(t)$ and $R^*(t)$ as its arrival and departure flows. We say $\mcal{S}$ offers a service curve $\beta$ if and only if
    \begin{equation}
        R^*(t) \geq (R \otimes \beta)(t) =: \inf_{s \leq t} \lbp R(s) + \beta(t-s) \rbp
    \end{equation}
    where $\otimes$ denotes the min-plus convolution
\end{definition}

Figure~\ref{fig: arrival-service curves} illustrates the arrival and service curves. Any segment of arrival flow $R(t)$ is constrained by arrival curve $\alpha$ and the output curve $R^*(t)$ is always no less than the curve $R\otimes\beta$. As a result, an arrival curve upper bounds the incoming traffic, and a service curve lower bounds the outgoing traffic.

% \begin{figure}
%     \centering
%     \includegraphics[width=\linewidth]{images/arrival-service.png}
%     \caption{Arrival/Service curve}
%     \label{fig: arrival-service curves}
% \end{figure}

We consider 2 special types of curves throughout this paper, \textit{token-bucket} (or sometimes called \textit{leaky-bucket}) curve and \textit{rate-Latency} curve.

\begin{definition}[Token-bucket and Rate-latency~\cite{ncbook2001leboudec}]
    A token-bucket curve $\gamma_{r,b}$ with arrival rate $r$ and burst $b$ is defined as
    \begin{equation}
        \gamma_{r,b}(t) = b + rt
    \end{equation}

    A rate-latency curve $\beta_{R,T}$ with service rate $R$ and latency $T$ is defined as
    \begin{equation}
        \beta_{R,T}(t) = R \lb t - T \rb_+
    \end{equation}
\end{definition}

A token-bucket curve is determined by a burst $b$ and an arrival rate~$r$. Burst represents the maximum possible data volume that can arrive simultaneously, and arrival rate represents the maximum long-term data rate~\cite{bouillard2022tradeoff}.
A rate-latency curve is determined by a latency~$T$ and a service rate~$R$. Latency represents the time a server needs before starting to process the incoming data, and service rate represents the minimum rate to process data after the initial latency.

With the help of arrival and service curves, we can derive delay and backlog bounds for a system $\mcal{S}$ illustrated in Figure~\ref{fig: system bounds}. Suppose a system $\mcal{S}$ has arrival curve $\alpha$ and service curve~$\beta$, its worst-case backlog $b^*$ is the maximum vertical distance between~$\alpha$ and~$\beta$. Similarly, depending on the multiplexing technique applied to the system, its worst-case delay bound $d^*$ is the maximum horizontal distance between $\alpha$ and $\beta$ if $\mcal{S}$ is a FIFO system. If we don't have any information about its multiplexing technique, referred to as arbitrary multiplexing, the best we can say is that when $\alpha$ and $\beta$ intersect each other, where all data has been delivered out of the system. Consequently, the worst-case delay bound for arbitrary multiplexing is the time required for $\mcal{S}$ to clear its buffer.

% \begin{figure}
%     \centering
%     \includegraphics[width=\linewidth]{images/bound.png}
%     \caption{System delay/backlog bounds}
%     \label{fig: system bounds}
% \end{figure}

While a service curve captures the slowest possible output speed of a system, a link's transmission capacity limits the speed as well. Hence, we model this phenomenon using a \textit{greedy shaper} with a sub-additive function $\sigma: \mbb{R}^+ \mapsto \mbb{R}^+$ concatenated with a server. We consider a concatenation as shown in Figure \ref{fig: system}. By convention we assume $\sigma(0) = 0$ and $\beta(t) \leq \sigma(t), \forall t \in \mbb{R}^+$, meaning that the buffer is cleared at the beginning and the service never exceed its physical limitation. With the above definition, such greedy shaper conserves the service provided by the system due to theorem \ref{thm: shaping}.

\begin{figure}[thb]
    \centering
    \includegraphics[width=0.7\linewidth]{images/system.png}
    \caption{Shaping of departure data. A flow that has an arrival curve $\alpha$ feeds into a server with an arrival data flow $R(t)$. The server having service curve $\beta$ takes $R(t)$ and gives a departure data flow $R^*(t)$ to a shaper with shaping function $\sigma$. The shaper takes $R^*(t)$ and shape the data flow as another departure $D(t)$.}
    \label{fig: system}
\end{figure}


\begin{theorem}[Shaping conserves service \cite{ncbook2001leboudec}]
\label{thm: shaping}
Following the system shown in Figure \ref{fig: system}, we have
\begin{equation}
     D = R^* \otimes \sigma \geq \lp R \otimes \beta \rp \otimes \sigma = R \otimes \lp \beta \otimes \sigma \rp = R \otimes \beta
\end{equation}
\end{theorem}

In the following context, we model the shaping function $\sigma$ as a token-bucket curve $\gamma_{C,L}$ with transmission capacity $C$ and the packet size $L$ to capture the link capacity and packetization~\cite{bouillard2022tradeoff}.


\section{Design} \label{sec:design}
\section{Design}
\label{s:design}
In this section, we will first present the core of our system. Then we present some analysis of the system along with some extensions to address a few practical concerns. We will present details of our cloud implementation separately in the next section.

\subsection{Delivery Based Ordering}
Our solution is composed of three parts. 
\subsubsection{Delivery Clock\\}
\noindent\textbf{What we do.}
Each RB maintains a delivery clock. This delivery clock essentially tracks time relative to when market data was delivered to the participant. We use $DC(i,a)$ to represent delivery clock of participant $i$ at time when trade $(i,a)$ is submitted. Delivery clock is a lexicographical tuple.
\begin{align}
    DC(i,a) = \langle ld(i,a), S(i,a)-D(i, ld(i,a))\rangle.
\end{align}
where $ld(i,a)$ is the latest data point that was delivered to MP$_i$ by time S(i,a), i.e., $D(i,ld(i,a)) \leq S(i,a) < D(i,ld(i,a)+1)$). 
Interval, $S(i,a)-D(i, ld(i,a))$, corresponds to the time that has elapsed since the last delivery and can be measured locally at the RB without requiring any clock synchronization (challenge 1). 

\noindent
\textit{Monotonicity:} Delivery clocks advance monotonically with submission time. As a result, DBO trivially satisfies the causality condition (Equation~\ref{eq:causality}). Further, it incentivizes the participants to submit trades as early as possible. Therefore, \emph{a participant cannot gain any advantage by delaying trades.} %\pg{should this point have a heading of its own}
Finally, we also leverage the monotonic property to overcome challenge 3 (\S\ref{ss:enforcing_ordering}). Figure~\ref{fig:delivery_clock} shows how delivery clock advances with time.

%\pg{I tried to reduce the notation here. I defined delivery clock slightly differently.}

\begin{figure}[t]
\centering
    \includegraphics[width=0.8\columnwidth]{figures/delivery_clock.pdf}
    \caption{\small{\bf Delivery Clock.}}% \pg{Redraw}}% \pg{Eashan see Ranveer's comment}}% \pg{Eashan can you redraw this figure in powerpoint or something.}}}
    \label{fig:delivery_clock}
    \vspace{-2.5mm}
\end{figure}

All incoming trades are marked with the delivery clock at the trade submission time. The ordering buffer uses this delivery clock time to order trades. Formally, the ordering in DBO is given by,  

\vspace{-2mm}
\begin{align}
    O(i,a) = DC(i, a). 
    \label{eq:ordering_with_dc}
\end{align}


\begin{figure}[t]
\centering
    \includegraphics[trim={0 0 0 2mm},clip,width=0.8\columnwidth]{figures/dbo_correct.pdf}
    \vspace{-4mm}
    \caption{\small{{\bf DBO can help correct for late delivery of data.} Delivery of market data to MP$_i$ is lagging behind MP$_j$. There are two trades $(i,a)$ and $(j,b)$ generated in response to the same market data $x$. $(j,b)$ was submitted before $(i,a)$ but
    %, i.e., $S_j(l) < A_i(k)$. 
    response time of $(i,a)$ is less than $(j,b)$.
    %, i.e., $rt_i(k) < rt_j(l)$. 
    In this example, $DC(i,a) (= \langle x, RT(i,a)\rangle) < DC(j,b) (= \langle x, RT(j,b)\rangle)$ and trade $(i,a)$ is correctly ordered ahead of $(j,b)$.}} %Ordering based on the submission time leads to incorrect ordering.}
    %\pg{Correct figure}}
    \label{fig:dbo_correction}
    \vspace{-3mm}
\end{figure}


\noindent\textbf{Why it works.}
When the trigger point of trade $(i,a)$ is indeed the last data point (i.e., $x = TP(i,a) = ld(i, a)$), then, DBO respects condition C2 for LRTF. Figure~\ref{fig:dbo_correction} shows an illustrative example of this.
This is because, the delivery clock directly tracks the response time of $i,a$ in this case and $O(i,a) = DC(i, a) = \langle x, RT(i,a)\rangle$. For a competing trade $(j,b)$ with higher response time, the delivery clock at time of submission will either read $O(j,b) = DC(j, b) = \langle x, RT(j,b)\rangle$ (if S(j,b)<D(j,x+1)) or $DC(j, b) = \langle y, S(j,b)-D(j,y)\rangle$ with $y>x$. In both cases, $O(i,a) < O(j,b)$.


At a high level, in our ordering we are correcting for latency differences in data delivery by using the delivery time of the last data point. When the last data point is not the trigger point for trade $(i,a)$, DBO satisfies the LRTF condition C2, if the following condition holds, 
\begin{align}
    D(i,ld(i,a))-D(i,x) = D(j,ld(i,a))-D(j,x),
    \label{eq:cond_delivery_lrtf}
\end{align}
where $x = TP(i,a)$.  
While it is impossible to ensure that inter-delivery times remain the same for all participants for all points, by pacing data at the RB it is indeed possible to ensure that the above condition is always met.% \radhika{you meant C2 or the above condition?}. \pg{the above condition only}
The main reason why we can meet the above condition is that condition C2 limits that the trigger point $x$ cannot be any arbitrary data point in the past, and that the trigger point must have been delivered recently  $S(i,a)-D(i,x) < \delta$.
%and we only need to ensure same inter-delivery times for. 
In the next subsection, we will show how we can achieve this and solve challenge 2. %\pg{Is this easy to follow?}



%\pg{FIX: say delivery clocks helps correct has static differences in latency. Why are delivery clocks so good on their own, give more intuition and experimentation. Potential things to include, see 6.1. Maybe make a section of.delivery clock on its own. correct the equation here in terms of response time as well.}
%\pg{Should we include results on necessary conditions on delivery times for achieving LRTF. Maybe its a bit of an overkill.}

\noindent
\textit{Remark:} In our cloud experiments, we find that DBO achieves fairness with very high probability. This is because network latency (from CES to any given participant) exhibits temporal correlation in latency especially over  short periods of time. When temporal correlation is high, inter-delivery time at any participant is close to the inter-generation time at the CES. In such cases, condition given by Equation~\ref{eq:cond_delivery_lrtf} is satisfied with high probability.

\noindent
\textbf{Difference with traditional logical clocks:} Logical clocks are commonly used in distributed systems. The most famous ones are lamport clocks~\cite{lamportSeminalPaper} and vector clocks. These clocks can be used for achieving total causal ordering of events. While these clocks can track causality of events, they cannot be used to achieve response time fairness. In particular, these clocks don't say anything about how two competing trades generated using the same market data should be ordered as these two trades have no direct causality relation. Unlike delivery clocks, such logical clocks also have no notion of measuring time between occurrences of two events. Time difference between events is critical to achieve fairnesss for exchanges. 

\noindent\textit{Note:} Several major financial exchanges already rely on heartbeats~\cite{nyse-client} for liveness when traffic is low.


\begin{figure}[t]
\centering
    \includegraphics[width=0.8\columnwidth]{figures/batching_pacing.pdf}
    \vspace{-2mm}
    \caption{\small{\bf Batching and Pacing. Inter-delivery time for consecutive batches is equal to or more than $\delta$.}}% \pg{Redraw}}% \pg{Eashan see Ranveer's comment}}% \pg{Eashan can you redraw this figure in powerpoint or something.}}}
    \label{fig:batching_pacing}
    \vspace{-4.5mm}
\end{figure}

\subsubsection{Batching and Pacing\\}
\noindent
\textbf{What we do.}
In DBO, the CES breaks data into batches. Each new batch contains all data points in the duration $(1+\kappa) \cdot \delta$ after the previous batch. Here $\kappa > 0$. Each release buffer delivers all data points in a batch at the same time. %Two points $x,y$ belonging to the same batch are delivered simultaneously to each participant, i.e., $D(j,y)=D(j,x), \forall j$.
The release buffer delivers batches as quickly as possible while ensuring that the time between delivery of two consecutive batches is atleast $\delta$. Figure~\ref{fig:batching_pacing} shows an illustration of batching. Both batching and pacing increase the delivery time of data points. In the next subsection we will analyze the impact of the two on latency. Note that in the event of queue build up at the RB, since the batch generation rate ($\frac{1}{(1+\kappa) \cdot \delta}$) is slower than the batch dequeue rate($\frac{1}{\delta}$), the queue at the RB eventually gets drained(\S\ref{ss:understanding_latency}).


\noindent
\textbf{Why it works.} With batching and pacing, DBO achieves LRTF. In particular, 
consider a trade $(i,a)$ with response time less than $\delta$. Because of pacing, consecutive batches are separated atleast by $\delta$. This means that the trigger point ($x=TP(i,a)$) must be within the last received batch. The point $ld(i,a)$ is also the last point in this batch and $D(i,ld(i,a)) = D(i,x)$. \emph{With batching and pacing, the delivery clock again directly tracks the response time of $(i,a)$} and $O(i,a) = DC(i,a) = <ld(i,a), RT(i,a)>$.
With batching, for participant $j$, $x$ and $ld(i,a)$ also belong to the same batch $D(j,ld(i,a)) = D(j,x)$.
For a competing trade $(j,b)$ with higher response time, the delivery clock at the time of submission will either read $O(j,b) = DC(j,b)) = \langle ld(i,a)), RT(j,b)\rangle$ (if $(j,b)$ was submitted before the next batch, i.e., $S(j,b) < D(j,ld(i,a)+1)$) or $DC(j, b) = \langle y, S(j,b)-D(j,y)\rangle$ with $y>ld(i,a)$. In both cases, $O(i,a) < O(j,b)$.

\if 0
\begin{figure}[t]
\centering
    \includegraphics[width=0.8\columnwidth, angle = -90]{images/pq_hb.jpg}
    \vspace{-2.5mm}
    \caption{\small{\bf Enforcing the ordering.} \pg{Redraw}}% \pg{Eashan see Ranveer's comment}}% \pg{Eashan can you redraw this figure in powerpoint or something.}}}
    \label{fig:pq_hb}
    \vspace{-2.5mm}
\end{figure}
\fi

\subsubsection{Enforcing the ordering\\}
\label{ss:enforcing_ordering}
OB contains a priority queue where all incoming trades are sorted based on the delivery clock timestamp (Equation~\ref{eq:ordering_with_dc}). A trade $(i,a)$ at the head of the priority queue should be forwarded to the CES only when the OB has received all trades $(j,b)$ with lower ordering $DC(j,b) < DC(i,a)$. 

\noindent
\textit{OB's Heartbeat Handler:} In DBO, each RB sends a heartbeat periodically every $\tau$ seconds to the CES. The heartbeat $(i,h)$, from participant $i$ contains the delivery clock timestamp at the time the heartbeat was generated ($DC(i,h)$). Since data in delivered in order and because delivery clock advances monotonically with time, heartbeat $(i,h)$ tells the OB that it has received all trades from participant $i$ with delivery clock less than $DC(i,h)$. The ordering buffer forwards trade $(i,a)$ if it has received heartbeats from all the participants with delivery clock timestamp higher than $DC(i,a)$. 


\subsection{Understanding DBO}

\subsubsection{Latency, parameter setting and straggler mitigation\\}
\label{ss:understanding_latency}

We will first derive the optimal latency for any ordering system that achieves response time fairness. We will then discuss how DBO compares to  optimal latency. We will also present guidelines for setting parameters and how to mitigate stragglers that can impact latency.

We define latency for trade $(i,a)$, $L(i,a)$, as the sum of latency in delivering data (from generation time) and latency in trade forwarding to the CES (from trade submission time). Formally,
\begin{align}
    L(i,a) = (D(i, x) - G(x)) + (F(i,a) - S(i,a)),\nonumber\\
    L(i,a) = F(i,a) - G(x) - RT(i,a),
    \label{eq:latency_def}
\end{align}
where $x=TP(i,a)$.

\noindent
\textbf{Optimal Latency:} Formally trade $(i,a)$ can only be forwarded to the CES's ME only when the CES has received all potential competing trades $(j,b)$ with lower response times ($RT(j,b) < RT(i,a)$). Let $R(i, x, RT)$ represent the time when the CES receives trade $(i,a)$ whose whose trigger point is x and response time is RT. Formally, 
\begin{align}
    F(i,a) = \max_{j}(R(j, x=TP(i,a), RT=RT(i,a))). 
\end{align}
A subtle point to note here is that even if participant $j$ does not produce any trades, we still need to wait for that participant till $R(j, x=TP(i,a), RT(i,a))$. Before this time, fundamentally the CES cannot be sure that it will not receive a trade from participant $j$ with a lower response time. 

We use $RTT(i, x, RT)$ to represent the sum of raw network latency for point x from CES to MP $_i$ and latency of trade from MP$_i$ to the CES (whose trigger point is x and response time RT).  In the best case scenario for latency (no buffering at any point in the path) we get
\begin{align}
    R(i, x, RT) = G(x) + RTT(i, x, RT) + RT.
\end{align}


Using the above two equations, we can write the following theorem.
\begin{theorem}
For any ordering system that achieves response time fairness, the minimum latency for trade $(i,a)$ is given by,
\begin{align}
    L(i,a) = \max_{j}(RTT(j, x=TP(i,a), RT=RT(i,a))).
\end{align}
\vspace{-2mm}
\label{thm:latency}
\end{theorem}

Put it simply, the above theorem states for achieving response time fairness, the minimum latency is bounded by the maximum round trip time across all participants. This means that fundamentally bad latency for a participant affects the latency of all trades. To achieve low latency consistently, we would like to ensure that latency of all the participants is well behaved majority of the times. How to better achieve this goal is left as a subject for future work.

%This theorem implies that even in cloud settings exchanges should ask for  network latency  

%With a very large number of participants thus pose a 
%\pg{fundamental issue with scalability}

\noindent
\textbf{How does DBO compare with the optimal?} DBO achieves close to optimal latency.  Compared to the optimal, batching and pacing introduce additional delay in delivery of market data points.  Since heartbeats are  generated only periodically they can  introduce an additional delay of $\tau$ at the ordering buffer. We now discuss the delay due to each of these components and how do the parameters $\kappa$, $\delta$ and $\tau$ affect latency. %\pg{Include a table here for parameters?}

\noindent
\textbf{Impact of batching:} Batching can introduce an additional delay of $(1+\kappa)\cdot \delta$ in the worst case. 

\noindent
\textit{Setting $\delta$:} $\delta$ thus presents a trade-off between latency and fairness (how large of a horizon can we pick). The right trade-off really depends on the needs of the exchange. Ideally, the exchange should pick the minimum value of $\delta$ that accommodates the response time of the fastest participants in a race. Our conversations reveal that fastest participants typically respond within a few microseconds and majority of the speed races last 5-10 $\mu s$. For our cloud experiments we  use $\delta = 20 \mu s$.

\begin{figure}[t]
    \centering
    \includegraphics[trim={0 0 0 0mm},clip,width=0.8\linewidth]{images/latency_b+p.pdf}
    \vspace{-5mm}
    \caption{\small{\textbf{Latency in data delivery:} x-axis shows the generation time of the market data. y-axis shows the latency from generation time to data delivery. $\kappa$  governs the average slope of the orange line immediately after latency spike (slope = $\frac{\kappa}{1+\kappa}$}).} %\pg{Include orange line and the base latency. Change labels to DBO and direct-delivery. Slope is $\kappa/(1+kappa)$}}
    %\pg{Eashan: Include the drain rate, make the colored lines thicker and use different linestyles for the three schemes..}}% \pg{Maybe label the drain rate in the figure for S1 and S2.}}
    \label{fig:latency_b+p}
    \vspace{-5mm}
\end{figure}

\noindent
\textbf{Impact of pacing.} Pacing restricts the batch dequeue rate at the RB. When network latency to a participant is not varying, the batch arrival/enqueue rate at the RB ($\frac{1}{(1+\kappa) \cdot \delta}$) is higher than the batch dequeue rate limit ($\frac{1}{\delta}$) and there is no queue build up. However, when network latency to a participant is decreasing (e.g., after a latency spike), batch arrival rate at the RB can exceed the dequeue rate limit leading to a queue build up. The overall queue - dequeue rate can be given by $\text{batch size} \cdot \text{batch rate limit} = 1+\kappa$. Figure~\ref{fig:latency_b+p} shows the impact of batching and pacing on latency in delivery of data in the event of a queue build up. The figure also shows the latency when data is delivered directly (raw network latency). The smaller sawtooths in the batching + pacing are because of batching. The deviation in direct delivery and batching + pacing is because of the rate limit imposed by pacing.

\noindent
\textit{Setting $\kappa$:} Increasing $\kappa$ increases batching delay but also increases the queue drain rate in the event of queue build up due to tail latency spikes. Increasing $\kappa$ thus presents a trade-off between reducing tail latency and increasing average latency. In our experiments we use $\kappa = 0.25$.
 
\noindent\textbf{Impact of heartbeats:} Heartbeats present a trade-off. Too frequent heartbeats can overwhelm the network, the ordering buffer or the release buffer. 
Infrequent heartbeats can increase the time OB has to wait of the participants. In particular, hearbeats can introduce an additional wait time of $\tau$. Note that the number of heartbeats, the OB needs to process increases linearly with the number of participants. In the next section we show how the heartbeat handler can be sharded for scalability.

\noindent\textit{Setting $\tau$:} Ideally we want to pick as low of a value as possible for the heartbeats without overwhelming the system. This number is very much dependent on the capabilities of the network and the processing power of the RB and the OB. In our cloud implementation we use $\tau = 20 \mu s$.

\noindent\textit{A note on latency:} When the network latency to participants is not varying with time, there is no queue build up at the release buffers. In such cases, DBO adds maximum of $((1+\kappa)\cdot \delta) + \tau$ additional latency over the optimal.

\noindent\textbf{Straggler Mitigation and RB/MP failure} In the event a  participant or release buffer crashes, DBO can stall processing trades. Further, the overall system latency also gets impacted when a certain participant is experiencing unusually high network latency (see Theorem~\ref{thm:latency}). Here we have the option to wait for the delayed participant and take a latency hit but not let the fairness be impacted. Ideally, we want to let the system continue with low latency with only the affected participant incurring unfairness. In DBO, we use a simple strategy to mitigate this. Using the heartbeats and the generation time of data points, the OB tracks the round trip latency to each participant. If this latency goes beyond a certain threshold for a participant, then the OB does not wait for heartbeats from such straggler participant before forwarding trades. When the round trip latency goes down, OB again starts waiting for heartbeats from the straggler. In the event of crashes, OB might not hear any heartbeats. If the OB does not hear a heartbeat from a particular participant for the above threshold, then it concludes that round trip latency exceeds the threshold and the OB deems the participant a straggler. 
 
\noindent\textit{OB failure:} In the event, the OB crashes all trades in the priority queue will be lost. System will incur unfairness in such cases. 

%The above strategy is also helpful in controlling overall system latency when a certain participant is experiencing unusually high network latency.


\subsubsection{Is Batching and Pacing necessary?\\}
\textbf{Batching and pacing contribute delays; are they necessary?} The answer is yes. Similar to Lemma~\ref{lemma:inter_delivery_imp}, we can derive the necessary conditions for achieving LRTF. 
\begin{corollary}
When trigger points are unknown, the \textit{necessary} conditions on the delivery processes for achieving response time fairness with any ordering system is given by,
\vspace{-1mm}
\begin{align*}
    \text{If }  D(i,y) - D(i,x) &< \delta, \text{ then},\nonumber\\
    D(i, y) - D(i,x) &= D(i,y) - D(i,x), & \forall i,j.
\end{align*}
\label{cor:inter_delivery_lrtf}
\vspace{-6mm}
\end{corollary}

\begin{proof}
Please see Appendix~\ref{app:cor_inter_delivery_lrtf}.
\end{proof}
\vspace{-1mm}
In contrast to Lemma~\ref{lemma:inter_delivery_imp}, the above condition states that the inter-delivery time of two points should be same across all participants only if they are separated by less than $\delta$ for some participant. Batching and pacing indeed satisfies this, for two points x and y in a batch, the inter-delivery times across all participants is indeed zero and hence equal. For point $x$ and $y$ belonging to different batches, since the inter-delivery time is greater than $\delta$ across all participants, there is no additional contraint on inter-delivery times being equal.
 
\subsubsection{Impact of RB to MP latency\\}
In scenarios where RB and the participant cannot be colocated, DBO can incur unfairness. If this latency is unbounded, then, it might be impossible to achieve fairness. If latency is bounded, however, then DBO provides the following fairness guarantees.

\begin{theorem}
    If round trip network latency from release buffer $i$ to it's corresponding participant is bounded between $B_l(i)$ and $B_h(i)$, then, DBO achieves the following guarantee for ordering trades.
    \begin{align*}
    C3: &\text{ if } TP(i,a)= TP(j,b) = x\\ 
    &\land RT(i,a) < RT(j,b) - (B_h(i)-B_l(j)), \\
    & \land RT(i,a) < \delta - B_h(i),\\
    &\text{ then, }O(i,a) < O(j,b).
\end{align*}
    \label{thm:rb_to_mp_latency}
    \vspace{-5mm}
\end{theorem}

\vspace{-1mm}
\begin{proof}
See Appendix~\ref{app:rb_to_mp_latency}.
\end{proof}
\vspace{-1mm}

Compared to LRTF, the above condition reduces the bound on response time for the faster trade $(i,a)$ to $\delta - B_h(i)$.
Additionally, the above condition states that trades are ordered fairly only when the response time of the faster trade is lower than the response time of the competing trade by atleast the variability in latency ($B_h(i)-B_l(j)$). This theorem essentially states that when RB and MP cannot be colocated, for better fairness we should ensure that latency between them is both consistent (across participants) and the upper bound is small.



\subsubsection{Impact of Losses\\}

Although infrequent, packet losses can occur in cloud environments. Such losses can impact fairness in DBO. However, only the fairness for trades that are lost and trades  whose trigger point is lost is impacted (see Appendix~\ref{app:impact_losses}).



\if 0
\subsubsection{Excessive queing at RB and OB\\}
\pg{This can be cut?}

Even though DBO employs straggler mitigation to limit the latency at the OB, it can build up a large queue if it receives a very large number of trades (little's law). The RB can also overflow in scenarios where the network latency is decreasing (Figure~\ref{fig:latency_b+p}) for a large period of time. 

\noindent
\textbf{RB:} In the event a release buffer's queue fills up (exceeds a certain threshold), to avoid overflow the release buffer forgoes pacing and starts releasing data as fast as possible to reduce the queue. In such cases, the delivery clock advances faster than as dictated by pacing. As a result, trades from such a participant might unfairly get ordered behind. The fairness for trades from other participants remains unaffected. When the queue goes down the RB resumes normal operation.

\noindent
\textbf{OB overflow:} In the event the order buffer's queue fills up, the OB starts releasing trades as fast as possible without waiting for heartbeats from participants. Once the queue goes down, OB resumes normal operation. In such cases, fairness of all trades are impacted. 
\fi

\subsubsection{Thwarting front-running attacks\\}

%Monotonicity of delivery clocks ensures that participants are incentivized to submit trades as early as possible and delaying trades does not offer any competitive advantage.% and participants are incentivized to be honest.
There is a front-running attack possible in our system. In particular, if a participant receives a market data point $x$ through some other way before RB delivers the data point $x$ to the participant then the participant has a competitive advantage. This scenario (though unlikely) is still possible. 

A simple to avoid this is to limit that a participant cannot talk to anyone beyond the CES. 
%\pg{External participants}
However, we would like the participant machine to use other  ``helper'' machines in the cloud, e.g.,  to aid computation. We also want to allow the participants to be able to talk to machines outside the cloud, e.g., to get a news stream. %stream.%\footnote{Participants use external news streams update trading strategies and make trading decisions.} 

In Appendix~\ref{app:front_running}, we show how we can prevent such front running attacks. In our solution, the participant and its helpers cannot communicate with any other participants or their helpers using the cloud network. 
To prevent scenarios where a participant uses a proxy machine outside the cloud to send market data to other  participants (faster than the network), we precisely add additional latency for data being sent outside the cloud.
While our solution introduces latency for data going out, the latency of speed trades remains unaffected.

\if 0

While monotonicity of delivery clocks ensure that participants are incentivized to submit trades as early as possible an delaying trades does offer any competitive advantage, there is still a potential front-running attack possible in our system. In particular, if a participant receives a market data point $x$ through some other way before RB delivers the data point $x$ to the participant then it has a competitive advantage. This scenario though unlikely is still possible.
A simple to avoid this is to limit that participant cannot talk to anyone beyond the CES. 

However, we would like the participant machine to use other  ``helper'' machines in the cloud to aid computation. We also want to allow the participants to be able to talk to machines outside the cloud. Participants do use external news streams and feeds from other exchanges to update trading strategies and make trading decisions. We will discuss fairness with respect to such streams shortly.  

Allowing such communication naively can lead to attacks.
By restricting communication, it is possible to ensure that no participant gets early access to market data %(at the cost of introducing latency in messages from the front-end to helpers outside the cloud)
and thwart such front-running attacks. 

%
%\pg{Which of two alternatives is better?}
%
To this end, we impose two simple constraints on communication. \begin{enumerate*}[label=(\arabic*)]\item A participant machine and its helper machines can communicate with each other freely but they cannot communicate with any other machines in the cloud. This restriction can be imposed easily by cloud providers today using security groups. This restriction ensures that a participant machine cannot get market data from other participant machines in the cloud directly. Next, we will ensure that a participant machine cannot get an earlier market data feed from outside the cloud. 
We will do so by restricting that a participant can only send data point x out of the cloud, when x has been delivered to all participants in the cloud. This way, market data points can only be available outside the cloud once they have been delivered to all the participants.
\item The helper machines cannot send data outside the cloud. Any data (excluding the trade orders) from a participant being sent outside the cloud is tagged by the delivery clock at the RB and buffered at a gateway. The data sent by the participant could potentially be a market data point with id less than or equal to the last point id (first tuple) of the delivery clock time stamp. The gateway thus buffers this data until it is sure that the all data points with id less than the last data point id in the delivery clock time stamp have been delivered. For this purpose, RB's periodically communicate their delivery clock to the gateway. 
%
%A simple way to achieve this is for each RB to send other RBs periodic beacons communicating the status of its delivery clock. This way each RB can maintain a lower bound on the delivery clocks at other RBs. 
\end{enumerate*}
\pg{include this? a bit hand-wavy and not clean. There is one challenge to be solved though. If data delivery to a particular participant is straggling then the gateway buffer can get bloated. It is not necessary for the gateway to wait for such straggler if we disable the incoming data to the straggler. The gateway can identify such stragglers and then disable any data coming from outside the cloud.}

Note that the above solution adds additionaly latency for data being sent outside the cloud. However, the latency of speed trades remains unaffected.
%There are other ways to thwart front-running that impose weaker restrictions on communication or are easier to implement. We chose to present this one for its simplicity.


\fi



\subsubsection{Limtations of DBO: Fairness beyond LRTF\\}
\label{ss:beyond_fairness}

With DBO, it is not guaranteed that trades that do not directly follow the LRTF model (Theorem~\ref{thm:1} and Equation~\ref{eq:cm})are ordered fairly. However, DBO still ensures that fairness for the most latency-sensitive speed trades. While ensuring guaranteed fairness for trades that do not follow the might be impossible, we will discuss potential some solutions.


%This will impose some system challenges. Another challenge is that different participants might be requesting different external streams. 
%


\noindent\textbf{Trades with response time > $\delta$:} DBO does not provide any guarantees for trades with response time greater than $\delta$. %If the inter-delivery times for batches across participants are same then DBO provides response time fairness for such trades. Again achieving the same inter-delivery times for all the batches is impossible. 
In case we have access to synchronized clocks, we can try and ensure (to the extent possible) that batches are indeed delivered at the same time across participants. 
When batches are delivered simultaneously, delivery clocks also get synchronized and DBO simply orders trades in the order of submission time. DBO thus ensures better fairness for such trades (when data is delivered simultaneous) while always guaranteeing LRTF. %\pg{Is this clear?}


%Regardless of whether using clocksync or not for deliverying the data, the performance of DBO for such trades is comparable to 


\noindent\textbf{Generalized compute model for trades:} A trade's submission time might be governed by delivery times of multiple data points. Again in such cases if we have access to synchronized clocks, we can try and ensure simultaneous delivery to the extent possible and achieve better fairness for such trades.


\noindent\textbf{External data streams:} In theory, external data streams like news events or market data from a competing exchange can trigger speed races. While DBO does not delay delivery of such streams to the participants (Appendix~\ref{app:front_running}), as described it does not guarantee fairness with respect to such streams. Existing exchanges do not provide any simultaneous delivery guarantees with respect to such external streams. Such streams typically traverse the internet, and the variability is network latency is substantially higher (order of milliseconds) than the market data stream (order of microseconds). Potentially, the exchange can serialize such external streams with the market data stream and ensure LRTF with respect to such a super stream. Such a serialization might not be trivial. Participants are requesting different data streams. We need to think carefully about what constitutes a fair serialization.
%\pg{Talk about how  further system challenges.}


%\subsubsection{\pg{Miscellaneous, do if time:}}
%\pg {Radhika advidce here would be helpful}

%\pg{1. Impact of clock drift rate, 3. Is batching and pacing necessary 4. Discussion, sharding for scalability, a separate RB for each asset class}













\if 0

\subsubsection{Delivery Clock\\}
Each RB maintains a delivery clock. This delivery clock essentially tracks time relative to when market data was delivered to the participant. We use $DC(i,t)$ to represent deliver clock of participant $i$ at time $t$. Delivery clock is a lexicographical tuple.
\begin{align}
    DC(i,t) = \langle ld(i,t), t-D(i, ld(i,t))\rangle.
\end{align}
where $ld(i,t)$ is the latest data point that was delivered to MP$_i$ at time t.% (i.e., $D_i(x_l(t)) \leq t < D_i(x_l(t)+1)$). 
Interval, $t-D(i, ld(i,t))$, corresponds to the time that has elapsed since the last delivery and can be measured locally at the RB without requiring any clock synchronization (challenge 1). Delivery clock advance monotonically with time. This property will help us overcome challenge 3 and also guard us against certain attack. (\pg{forward pointers}). Figure~\ref{fig:delivery_clock} shows how delivery clock advances with time.

\begin{figure}[t]
\centering
    \includegraphics[width=0.8\columnwidth]{images/delivery_clock.jpg}
    \vspace{-2.5mm}
    \caption{\small{\bf Delivery Clock.} \pg{Redraw}}% \pg{Eashan see Ranveer's comment}}% \pg{Eashan can you redraw this figure in powerpoint or something.}}}
    \label{fig:delivery_clock}
    \vspace{-2.5mm}
\end{figure}

All incoming trades are market with the delivery clock at the trade submission time. The ordering buffer uses this delivery clock time to order trades. Formally, the ordering in DBO is given by,  

\begin{align}
    O(i,a) = DC(i, S(i,a)). 
    \label{eq:ordering_with_dc}
\end{align}


\begin{figure}[t]
\centering
    \includegraphics[trim={0 0 0 2mm},clip,width=0.9\columnwidth]{hotnets-images/time series visualization (3).pdf}
    \vspace{-3mm}
    \caption{\small{{\bf DBO can help correct for late delivery of data.} Delivery of market data to MP$_i$ is lagging behind MP$_j$. There are two trades $(i,a)$ and $(j,b)$ generated in response to the same market data $x$. $(j,l)$ was submitted before $(i,k)$ but
    %, i.e., $S_j(l) < A_i(k)$. 
    response time of $(i,k)$ is less than $(j,l)$.
    %, i.e., $rt_i(k) < rt_j(l)$. 
    With DBO, $O(i,a) (= \langle x, RT(i,a)\rangle) < O(j,b) (= \langle x, RT(j,b)\rangle)$ and trade $(i,a)$ is correctly ordered ahead of $(j,b)$.} %Ordering based on the submission time leads to incorrect ordering.}
    \pg{Correct figure}}
    \label{fig:dbo_correction}
    \vspace{-4mm}
\end{figure}


When the trigger point of trade $(i,a)$ is indeed the last data point (i.e., $x = TP(i,a) = ld(i, S(i,a))$), then, DBO respects condition C2 for LRTF. Figure~\ref{fig:dbo_correction} shows an illustrative example of this.
This is because $O(i,a) = DC(i, S(i,a)) = \langle x, RT(i,a)\rangle$. For, a competing trade $(j,b)$ with higher response time, the delivery clock at time of submission will either read $O(j,b) = DC(j, S(j,b)) = \langle x, RT(j,b)\rangle$ (if D(j,x+1)>S(j,b)) or $DC(j, S(j,b) = \langle y, S(j,b)-D(j,y)\rangle$ with $y>x$. In both cases, $O(i,a) < O(j,b)$.


\noindent
\t
At a high level, in our ordering we are correcting for latency differences in data delivery by using the delivery time of the last data point. When the last data point is not the trigger point for trade $(i,a)$, DBO satisfies the LRTF condition C2, if the following condition holds, 
\begin{align}
    D(i,ld(i,t))-D(i,x) = D(j,ld(i,t))-D(j,x),
    \label{eq:cond_delivery_lrtf}
\end{align}
where $x = TP(i,a)$.  
While it is impossible to ensure that inter-delivery times remain the same for all participants for all points, by pacing data at the RB it is indeed possible to ensure that the above condition is always met. 
The main reason why we can do so is thaat condition C2 limits that the trigger point $x$ cannot be any arbitrary data point in the past ($S(i,a)-D(i,x) < \delta$).
%and we only need to ensure same inter-delivery times for. 
In the next subsection, we will show how we can achieve this and solve challenge 2. \pg{Is this easy to follow?}

\pg{Should we include results on necessary conditions on delivery times for achieving LRTF}

\noindent
\textit{Remark:} In our cloud experiments, we find that DBO achieves fairness with very high probability. This is because network latency (from CES to any given participant) exhibits temporal correlation in latency especially over  short periods of time. When temporal correlation is high, inter-delivery time at any participant is close to the inter-generation time at the CES. In such cases, condition given by Equation~\ref{eq:cond_delivery_lrtf} is satisfied with high probability.

\begin{figure}[t]
\centering
    \includegraphics[width=0.8\columnwidth]{images/batching_pacing.jpg}
    \vspace{-2.5mm}
    \caption{\small{\bf Batching and Pacing.} \pg{Redraw}}% \pg{Eashan see Ranveer's comment}}% \pg{Eashan can you redraw this figure in powerpoint or something.}}}
    \label{fig:batching_pacing}
    \vspace{-2.5mm}
\end{figure}

\subsubsection{Batching and Pacing\\}
In DBO, the CES breaks data into batches. Each new batch contains all data points in the duration $(1+\kappa) \cdot \delta$ after the previous batch. Here $\kappa > 0$. Each release buffer delivers all data points in a batch at the same time. %Two points $x,y$ belonging to the same batch are delivered simultaneously to each participant, i.e., $D(j,y)=D(j,x), \forall j$.
The release buffer delivers batches as quickly as possible while ensuring that the time between delivery of two consecutive batches is atleast $\delta$. Figure~\ref{fig:batching_pacing} shows an illustration of batching. Both batching and pacing increase the delivery time of data points. In the next subsection we will analyze the impact of the two on latency. Note that since $\kappa > 0$ batch generation rate is slower than batch drain rate and build up queue because of pacing will eventually get drained. 



With batching and pacing, DBO achieves LRTF. In particular, 
consider a trade $(i,a)$ with response time less than $\delta$. Because of pacing, batches are separated by $\delta$. This means that the trigger point ($x=TP(i,a)$) must be within the last received batch. The point $ld(i,S(i,a))$ is also the last point in this batch and $D(i,ld(i,S(i,a)) = D(i,x)$. $O(i,a) = DC(i,S(i,a)) = <ld(i,S(i,a)), RT(i,a)>$.
With batching, for participant $j$, $x$ and $ld(i,S(i,a))$ also belong to the same batch $D(j,ld(i,S(i,a)) = D(j,x)$.
For, a competing trade $(j,b)$ with higher response time, the delivery clock at the time of submission will either read $O(j,b) = DC(j, S(j,b)) = \langle ld(i,S(i,a)), RT(j,b)\rangle$ (if $(j,b)$ was submitted before the next batch, i.e., $D(j,ld(i,S(i,a))+1) > S(j,b)$,) or $DC(j, S(j,b) = \langle y, S(j,b)-D(j,y)\rangle$ with $y>ld(i,S(i,a))$. In both cases, $O(i,a) < O(j,b)$.

\fi

\if 0
\subsection{Compute Model of the HFT Trader and Definition of Fairness}

\begin{enumerate}
    \item $MD_R(i, x):$ Receive time of market data at the gateway/RBi
    \item $TO_G(i, a):$ Generation time of trade order a by trader i
    \item $TP(i,a):$ Trigger/stimuli for trade (i,a)
    \item $RT(i,a):$ Response time of for trade (i,a) 
\end{enumerate}


\textbf{Compute Model:}
Time of generation of trade= time participant received the market point that triggered the trade + response time (or time it took to generate the trade)
\begin{equation}
    TO_G(i,a) = MD_R(i,TP(i,a)) + RT(i,a)
\end{equation}


\textbf{Perceived Fairness with respect to participant i}
If all other participants received the market data at the same time as i, then how should the trades be ordered
\begin{align*}
    \text{Trade (i,a) should be ordered ahead if}\\
    TO_G(i,a) &< MD_R(i,y) + RT(j,b)\\
    TO_G(i,a) - MD_R(i,y) &< TO_G(j,b) - MD_R(j,y)
\end{align*}
This definition states for two orders trades we need to measure time relative to event y

alternatively what if i goes into j's time domain
\begin{align*}
    &\text{Trade (i,a) should be ordered ahead iff O(i,a)<O(j,b)}\\
    MD_R(j,x) + RT(i,a) &< TO_G(j,b)\\
    TO_G(i,a)-MD_R(i,x) &< TO_G(j,b) - MD_R(j,x)
\end{align*}

Correction, relative ordering




\textbf{Achieving fairness}
There are two challenges,
\begin{outline}
    \1 How do you decide how to order these trades when TP y is unknown. \pg{Three options 1) Delivery Clocks 2) Equal RTT 3) Directly to limited fairness} \pg{Time domain: two options a) I's domain b) zero latency time doman. Fairness for trades using different data points.}
        \2 Don't know which x, recency \pg{equivalence between equal inter-delivery and correcting one way latency}
        \2 Clocks are not synced
        \2 Monotonic ordering with time
    \1 How do you enforce the ordering process. In particular, trades may take an arbitrary amount of time to reach the OB.
\end{outline}

What is the lowest RTT possible with this system?\\
Say you knew the trigger points x,y what then, \\
Say you didn't know the trigger points\\
Enforcing the ordering: key insight Enforcing an ordering at a single point is easier than controlling things at multiple RBs\\
What about trades with response time greater than delta\\


Question: Fairness wrt to external data stream

\textbf{Practical Considerations}

\begin{enumerate}
    \item Collusion attacks: Ensure that any market data point is delivered only after all participants have received it
    \item external participants: Have all participants submit trade via a dummy MP machine (we dont support fairness for such particpants)
    \item External data streams:
    \item Stragglers: 
\end{enumerate}


\textit{Correction by latency pitch}
\begin{align*}
    TO_G(i,a) - MD_R(i,y) &< TO_G(j,b) - MD_R(j,y)\\
    TO_G(i,a) - (G(y) - MD_R(i,y))) &< TO_G(j,b) +(G(y)- MD_R(j,y))
\end{align*}

\pg{Alternatively fairness in the same or equal or zero latency time domain?}
\begin{align*}
    &\text{Trade (i,a) should be ordered ahead iff O(i,a)<O(j,b)}\\
    G(x) + RT(i,a) &< G(y) + RT(j,b))\\
    TO_G(i,a) + (G(x)-MD_R(i,x)) &< TO_G(j,b) + (G(y) - MD_R(j,y))
\end{align*}


\textbf{Final Pitch Attempt}
\begin{enumerate}
    \item Introduce generalized compute model
    \item Talk about zero latency model for fairness. Three problems clocksync, which x to use, how to enforce ordering. \pg{Introduce C1 from strong fairness here?}
    \item clocksync: We are interested in competing trades that are generated using the same data point \pg{is clocksync really necessary to force this}
    \item which x to use: the last x since trades are fast. What about latency for trades with response time greater than delta
    \item how to enforce ordering: monotonic ordering process \pg{unclear if monotonic is time property is even needed (if )} 
    \item part of above? No fooling: C1 property of strong fairness
    \item \pg{Limitations: Our solution doesn't work with this model for trades generated using different data points. What about approx fairness? This is kind of nice because it talks about latency/}
\end{enumerate}
\fi

\section{Implementation} 
\section{Implementation}
\label{sec:impl}

At \company, we have deployed \sysname in our internal clusters to serve daily DL workloads.
The internal clusters consist of heterogeneous GPUs, including NVIDIA T4 GPU and NVIDIA A10 GPU.
Integrated with Kubernetes~\cite{k8s}, \sysname manages thousands of GPUs in each cluster and more than 20,000 GPUs in all.

\parabf{Service manager.}
For online workloads, we use the existing service manager at \company which deploys containers, discovers service, and autoscales horizontal pods.

\parabf{Global manager.}
We modify the Kubernetes scheduler to schedule offline workloads.
The workload profiler takes several dedicated GPUs, whose number is negligible to the total number of GPUs.
When a new offline workload comes, the workload profiler performs a few dry runs of the workload and utilizes the NVIDIA Data Center GPU Manager (DCGM) tools~\cite{dcgm} and NVIDIA Management Library (NVML)~\cite{nvml} libraries to collect GPU metrics.
We collect about 2,000 data for each GPU type to train the speed predictor.
The MLPs of the speed predictor have four layers with hidden size $64\times 64$.
The MLPs are trained with momentum SGD optimizer~\cite{ruder2016overview} in PyTorch v1.8.0~\cite{paszke2019pytorch} until they converge.
\sysname invokes the scheduler periodically to schedule all offline workloads.
When moving workloads, we record checkpoints of offline workloads and restart the workloads after transmitting the models and checkpoints.
As the datasets are usually colossal, we store the datasets in a remote file system and fetch data during the execution.
We implement the scheduler as a third-party plugin to the Kubernetes scheduler.


\parabf{Local executor.}
Each local executor executes online workloads according to the service manager and offline workloads according to the global manager.
DL workloads are executed in Docker containers with our customized components.
We add Best-Effort GPU DevicePlugin in Kubernetes and relevant control paths with Kubelet and \sysprobe for offline workloads.
To control SM percentage, we leverage the environment variable $CUDA\_MPS\_ACTIVE\_THREAD\_PERCENTAGE$ provided by MPS.
The GPU monitor collects resource metrics through DCGM~\cite{dcgm} and NVML~\cite{nvml} for NVIDIA GPU.
The \sysprobe updates the state machine with the collected resource metrics and empirically-set thresholds.
When the state is unhealthy, the \sysprobe will ask the NodeManager in Kubernetes to evict offline workloads.
\bytecuda intercepts nearly 800 CUDA driver APIs for GPU memory allocation and kernel launch.
The GPU memory quota of offline workloads is fixed to $40\%$ as Figure~\ref{fig:motiv_gpu_resource} reports that most online workloads use less than $60\%$ GPU memory.
We adopt the cpuset of Cgroup for CPU isolation.
For memory, \sysname will evict offline workloads if memory usage is higher than a threshold or the kernel swap daemon is busy for a long time.
The parameters to calculate GPU load in Equation~\ref{equ:gpu_load}$\&$\ref{equ:clock_factor} are empirically selected through trial-and-error.


\section{Evaluation}
 \section{Benchmarks and Evaluation}
\label{sec:eval}

We evaluate \krakenSpace to answer the following set of questions:
\begin{itemize}
\item How much improvement does partial evaluation and our implemented compiler optimizations give \kraken? %(\S \ref{sec:eval2})
\item How much faster is our purely functional f-expr language, \krakenSpace, compared to other implementations of fexprs? %(\S \ref{sec:eval1} - \ref{sec:eval2})
\item How does \kraken's performance, with its fexprs, compare to macros? %(\S \ref{sec:eval1}, \S \ref{sec:eval3})
\item How do the different partial evaluation mechanisms/optimizations in \krakenSpace contribute towards reduction in overall runtime?
%\item What does \krakenSpace do internally when we create a data structure and evaluate it for some function? (\S \ref{sec:casestudy})
\end{itemize}

\textbf{Experimental Setup}: 
We ran these experiments in a reproducible Nix environment on a NixOS install \cite{10.1145/1411203.1411255} (Kernel 6.0.0) on a laptop with 8 cores / 16 threads and 64 GB of RAM.
Our code contains the scripts and Nix Flakes needed to reproduce the exact set of dependencies to run our tests.
%The code can be found at \url{https://github.com/limvot/kraken}.

The Kraken benchmarks were run using both the Wasmtime and WAVM WebAssembly engines for most benchmarks.
The Wasmtime WebAssembly engine is one of the most popular, developed by the Bytecode Alliance itself, and uses the CraneLift code generation backend.
The WAVM WebAssembly engine is interesting for its use of LLVM, and it often produces the fastest code on benchmarks but has a higher startup time.
We eliminated the Cfold Wasmtime benchmark due to problems running out of stack space (a known property of the Cfold benchmark).

\textbf{Benchmarks}: 
To showcase the capability of Kraken, we created benchmarks that are commonly implemented in functional languages and have been used as benchmarks in other papers \cite{reinking2021perceus, 10.1145/3547646}.
The benchmarks are
\begin{itemize}
\item Fib - Calculating the nth Fibonacci number
\item RB-Tree - Inserting n items into a red-black tree, then traversing the tree to sum its values
\item Deriv - Computing a symbolic derivative of a large expression
\item Cfold - Constant-folding a large expression
\item NQueens - Placing n number of queens on the board such that no two queens are diagonal, vertical, or horizontal from each other
\end{itemize}
All benchmarks besides Fibonacci use the fexpr version of match for pattern matching in \kraken, which is equivalent to the macro version in NewLisp. We also RB-Tree using NewLisp's~\cite{mueller2018newlisp} version of fexpr match. We modified the sizes of the problems presented to the benchmark to account for the longer running times of some of the less-optimized implementations.
The code for Kraken and NewLisp is very similar, and we should note that it is very unidiomatic NewLisp.
Our goal was not to compare Kraken and NewLisp as implementation languages for Red-Black Trees, but to stress test a single reasonably complex fexpr/macro, namely pattern matching.
% \textbf{Comparison with other languages}: We evaluated \krakenSpace against a language that contains f-exprs, as well as against itself with various optimizations disabled. The only other language we could find which contains a real f-expr mechanism is NewLisp~\cite{mueller2018newlisp} and so we ported \kraken's benchmark implementation to NewLisp.

%The six state-of-the-art languages are Java 17.0.1, Swift 5.4.2, Koka 2.3.2, C++, Haskell 8.10.7, and OCaml 4.12.
%The language choices were taken directly from Perceus reference-counting paper \cite{reinking2021perceus}.
%The Fibonacci benchmark additionally tests Python 3.9.11 and Chez Scheme 9.5.4.
%Koka, Ocaml and Haskell are good comparison points as statically-typed, compiled, functional programming languages, while Chez Scheme is a good comparison point as a mature and industrial strength dynamically-typed Scheme implementation known for its performance. 
%\subsection{Basic Level Comparison}
\subsection{The Effect of Partial Evaluation on Eval Calls}

\begin{table}[h]
\caption{Number of eval calls with no partial evaluation for Fexprs}
	\begin{tabular}{||c | c c c c c ||} 
		\hline
		&Evals & Eval w1 Calls & Eval w0 Calls & Comp Dyn & Comp Dyn\\ 
        & & & & w1 Calls & w0 Calls\\ [0.5ex] 
		\hline\hline
		Cfold 5 & 10897376 & 2784275 & 879066  & 1 & 0 \\ 
		\hline
		  Deriv 2  & 11708558 & 2990090 & 946500 & 1 & 0 \\ 
        \hline
		  NQueens 7 & 13530241 & 3429161 & 1108393 & 1 & 0 \\ 
    \hline
		  Fib 30 & 119107888 & 30450112 & 10770217 & 1 & 0 \\ 
    \hline
		  RB-Tree 10 & 5032297 & 1291489 & 398104 & 1 & 0 \\ 
		\hline
	\end{tabular}
    \label{npe:calls}
 \end{table}

As mentioned before, using fexprs without partial evaluation will prelude optimization and cause a massive amount of repeated work. Table \ref{npe:calls} and Table \ref{pe:calls} show the number of calls to the \krakenSpace runtime's eval function, the number of times the runtime's eval function executed a call to an applicative with wrap\_level=1, the number of times the runtime's eval function executed a call to an operative with wrap\_level=0, the number of compiled dynamic calls to applicatives with wrap\_level=1, and the number of compiled dynamic calls to operatives with wrap\_level=0.
These are shown for \krakenSpace test cases with partial evaluation turned off and turned on. 
\begin{table}[h]
\caption{Number of eval calls in Partially Evaluated Fexprs}
	\begin{tabular}{||c | c c c c c ||} 
		\hline
		&Evals & Eval w1 Calls & Eval w0 Calls & Comp Dyn & Comp Dyn\\ 
        & & & & w1 Calls & w0 Calls\\ [0.5ex] 
		\hline\hline
		Cfold 5 & 0 & 0 & 0  & 0 & 0 \\ 
		\hline
		  Deriv 2  & 0 & 0 & 0 & 2 & 0 \\ 
        \hline
		  NQueens 7 & 0 & 0 & 0 & 0 & 0 \\ 
    \hline
		  Fib 30 & 0 & 0 & 0 & 0 & 0 \\ 
    \hline
		  RB-Tree 10 & 0 & 0 & 0 & 10 & 0 \\ 
		\hline
	\end{tabular}
    \label{pe:calls}
 \end{table}

\begin{table}[h]
\caption{Number of calls to the runtime's eval function for RB-Tree. The table shows the non-partial evaluation numbers -> partial evaluation numbers.}
	\begin{tabular}{||c | c c c c c ||} 
		\hline
		&Evals & Eval w1 Calls & Eval w0 Calls & Comp Dyn & Comp Dyn\\ 
        & & & & w1 Calls & w0 Calls\\ [0.5ex] 
		\hline\hline
		  RB-Tree 7 & 2952848 -> 0 & 757932 -> 0 & 233513 -> 0 & 1 -> 7 & 0 -> 0\\ 
        \hline
		  RB-Tree 8 & 3532131 -> 0 & 906548 -> 0 & 279379 -> 0 & 1 -> 8 & 0 -> 0\\ 
        \hline
		  RB-Tree 9 & 4278001 -> 0 & 1097965 -> 0 & 3383831 -> 0 & 1 -> 9 & 0 -> 0\\ 
		\hline
	\end{tabular}
    \label{pe:rb}
    \vspace{-4mm}
 \end{table}

Without partial evaluation, no compilation can be done because it is impossible to tell if arguments to calls will be evaluated. In all benchmarks, partial evaluation removed all calls to the runtime's eval function, resulting in a completely compiled program. Looking at RB-Tree, there are over a million calls to combiners with wrap level 1 (normal functions), and 398,000 calls to combiners with wrap level 0 (operatives replacing macros). This massive blowup in the number of calls is due to the repeated and exponential re-execution of macro-like-combiners in the definition of other macro-like-combiners, as discussed in the Introduction.

The non-partially-evaluated benchmarks show 1 compiled dynamic call to an applicative (its the first call into eval) and 0 compiled dynamic calls to operatives, because there is no compilation at all. For the partially evaluated benchmarks, there are a few compiled dynamic calls to applicatives due to higher-order function use in the benchmarks, and there are no compiled dynamic calls to operatives, as all operative use has been eliminated.
We also varied the inputs for RB-Tree shown in Table \ref{pe:rb} to give a sense for how the number scale with respect to input size.

The incredible slowdown implied by these tables comes to full fruition in our RB-Tree test in Fig.~\ref{fig:kraken_nqueens_rbtree}.
We kept this run shorter because Kraken's non-partial-evaluating interpreter takes an incredibly long time even for 100 insertions (40 minutes).
The compounding layers of repeated macro-like operative calls in the non-partially-evaluated Kraken version cause a ~70,000x slowdown relative to the partial evaluated, optimized, and compiled version.
For the remaining benchmarks, we remove the naive interpreted \krakenSpace version, as in each case its performance is so bad as to blow out the graph and make it impossible to do any comparison.
In our optimized Kraken, our partial evaluation algorithm is able to fully collapse these levels of inefficiency, evaluate and inline the results, and give the backend more specialized code to optimize, emitting a compiled version that handily beats not only the NewLisp-fexpr implementation but even the NewLisp-macro implementation, as can be seen in Fig.~\ref{fig:kraken_vs_world_fib}.
We kept the benchmark sizes small in this test because the stack limits of NewLisp prevent sizes larger then ~880, while the Tail Call Elimination performed by the \krakenSpace compiler allows us to run much larger benchmarks, including the run of 4,800,000 inserts to the RB-Tree.
This result shows the dramatic effect of partial evaluation and compiler optimizations on runtime for \kraken. Our technique takes the performance of a fully fexpr based language from being completely infeasible to being faster than a macro-based dynamic scripting language currently in use.
% \begin{center}
% \begin{table}[ht]
% \caption{Number of call to the runtime's eval function for Fib. The table shows the non-partial evaluation numbers -> partial evaluation numbers}
% 	\begin{tabular}{||c | c c c c c ||} 
% 		\hline
% 		&Evals & Eval w1 Calls & Eval w0 Calls & Comp Dyn w1 Calls & Comp Dyn w0 Calls\\ [0.5ex] 
% 		\hline\hline
% 		Fib 10 & 8468 -> 0 & 2167 -> 0  & 777 -> 0 & 1 -> 0 & 0 -> 0 \\ 
% 		\hline
% 		  Fib 15  & 87916 -> 0 & 22478 -> 0 & 7961 -> 0 & 1 -> 0 & 0 -> 0 \\ 
%         \hline
% 		  Fib 20 & 969010 -> 0 & 247731 -> 0 & 87633 -> 0 & 1 -> 0 & 0 -> 0 \\ 
%     \hline
% 		  Fib 25 & 10740492 -> 0 & 2745825 -> 0  & 971209 -> 0 & 1 -> 0 & 0 -> 0 \\ 
% 		\hline
% 	\end{tabular}
%     \label{pe:fib}
%  \end{table}
% \end{center}

\begin{figure}[h]
\caption{Constant Fold and Deriv}
\includegraphics[width=0.45\textwidth]{cfold_table.csv_}
\includegraphics[width=0.45\textwidth]{deriv_table.csv_}
\label{fig:kraken_const_deriv}
\vspace{-6mm}
\end{figure}
\subsection{Comparison between Kraken Versions}
Beyond the massive speedup from partial-evaluation, Fig. \ref{fig:kraken_const_deriv} and \ref{fig:kraken_nqueens_rbtree} show the effect of the various compiler optimizations we described by disabling them one by one.
 Our main four optimizations have a strong positive effect on runtime, with the exception of lazy environment instantiation. Lazy environment instantiation helps massively on fib, and some on Deriv, but generally hurts the rest slightly.


\begin{figure}[h]
\caption{N-Queens}
\includegraphics[width=0.45\textwidth]{nqueens_table.csv_}
\includegraphics[width=0.45\textwidth]{slow_rbtree_table.csv_}
\label{fig:kraken_nqueens_rbtree}
\vspace{-4mm}
\end{figure}


\subsection{Comparison against Others}


To give a general idea of our current performance, we also show a Fibonacci benchmark that mostly exercises pure function-call speed and inlining as seen in Fig. ~\ref{fig:kraken_vs_world_fib}.
We include Python and Chez Scheme to give a general idea for where an exemplar slow and an exemplar fast dynamic language would fall.
With the benefit of our partial evaluation, compilation, and leaning upon mature WebAssembly implementations, we beat both, but this should be taken with a grain of salt, as this is a very limited micro-benchmark only meant to give a general sense of the order of magnitude of our performance.



\label{sec:eval1}
\begin{figure}[h]
\caption{Kraken vs. Others. Ordered by fastest to slowest}
\includegraphics[width=0.45\textwidth]{fib_table.csv_}
\includegraphics[width=0.45\textwidth]{rbtree_table.csv_}
\label{fig:kraken_vs_world_fib}
\end{figure}

%\label{sec:eval_nqueens}
%\begin{figure}[h]
%\caption{N-Queens}
%\includegraphics[width=0.45\textwidth]{nqueens_table.csv_}
%\includegraphics[width=0.45\textwidth]{slow_nqueens_table.csv_}
%\label{fig:kraken_nqueens}
%\end{figure}

%\label{sec:eval_nqueens}
%\begin{figure}[h]
%\caption{Kraken, N-Queens, absolute value and log-scale}
%\includegraphics[width=0.45\textwidth]{nqueens_table.csv_}
%\includegraphics[width=0.45\textwidth]{nqueens_table.csv_log}
%\label{fig:kraken_nqueens}
%\end{figure}
%\label{sec:eval_nqueensp}
%\begin{figure}[h]
%\caption{Kraken, N-Queens, absolute value and log-scale}
%\includegraphics[width=0.45\textwidth]{slow_nqueens_table.csv_}
%\includegraphics[width=0.45\textwidth]{slow_nqueens_table.csv_log}
%\label{fig:kraken_nqueensp}
%\end{figure}

%\label{sec:eval_cfold}
%\begin{figure}[h]
%\caption{C-Fold}
%\includegraphics[width=0.45\textwidth]{cfold_table.csv_}
%\includegraphics[width=0.45\textwidth]{slow_cfold_table.csv_}
%\label{fig:kraken_cfold}
%\end{figure}
%\label{sec:eval_cfold}
%\begin{figure}[h]
%\caption{Kraken, C-Fold, absolute value and log-scale}
%\includegraphics[width=0.45\textwidth]{cfold_table.csv_}
%\includegraphics[width=0.45\textwidth]{cfold_table.csv_log}
%\label{fig:kraken_cfold}
%\end{figure}
%\label{sec:eval_cfoldp}
%\begin{figure}[h]
%\caption{Kraken, C-Fold, absolute value and log-scale}
%\includegraphics[width=0.45\textwidth]{slow_cfold_table.csv_}
%\includegraphics[width=0.45\textwidth]{slow_cfold_table.csv_log}
%\label{fig:kraken_cfoldp}
%\end{figure}

%\label{sec:eval_deriv}
%\begin{figure}[h]
%\caption{Deriv}
%\includegraphics[width=0.45\textwidth]{deriv_table.csv_}
%\includegraphics[width=0.45\textwidth]{slow_deriv_table.csv_}
%\label{fig:kraken_deriv}
%\end{figure}
%\label{sec:eval_deriv}
%\begin{figure}[h]
%\caption{Kraken, Deriv, absolute value and log-scale}
%\includegraphics[width=0.45\textwidth]{deriv_table.csv_}
%\includegraphics[width=0.45\textwidth]{deriv_table.csv_log}
%\label{fig:kraken_deriv}
%\end{figure}
%\label{sec:eval_derivp}
%\begin{figure}[h]
%\caption{Kraken, Deriv, absolute value and log-scale}
%\includegraphics[width=0.45\textwidth]{slow_deriv_table.csv_}
%\includegraphics[width=0.45\textwidth]{slow_deriv_table.csv_log}
%\label{fig:kraken_derivp}
%\end{figure}

%\subsection{Comparison against state-of-the-art languages}
%\label{sec:eval3}

%\begin{figure}[h]
%\caption{Kraken vs. S.o.t.A.}
%\includegraphics[width=0.45\textwidth]{cfold_table.csv_}
%\includegraphics[width=0.45\textwidth]{rbtree_table.csv_}
%\label{fig:kraken_vs_world1}
%\end{figure}

%\begin{figure}[h]
%\caption{Kraken vs. S.o.t.A.}
%\includegraphics[width=0.45\textwidth]{deriv_table.csv_}
%\includegraphics[width=0.45\textwidth]{nqueens_table.csv_}
%\label{fig:kraken_vs_world2}
%\end{figure}

% \begin{figure}[h]
% \caption{Kraken vs. S.o.t.A. (Log)}
% \includegraphics[width=0.45\textwidth]{cfold_table.csv_log}
% \includegraphics[width=0.45\textwidth]{rbtree_table.csv_log}
% \label{fig:kraken_vs_world_log_1}
% \end{figure}
% \begin{figure}[h]
% \caption{Kraken vs. S.o.t.A. (Log)}
% \includegraphics[width=0.45\textwidth]{deriv_table.csv_log}
% \includegraphics[width=0.45\textwidth]{nqueens_table.csv_log}
% \label{fig:kraken_vs_world_log_2}
% \end{figure}

%As we noted before with the Fib(30) microbenchmark in Section \ref{sec:eval1}, we remain significantly slower than state-of-the-art compiled languages.
%This is particularly true for memory-intensive benchmarks due to our naive reference-counting and malloc/free implementations.
%However, our results are of a similar order of magnitude to the difference between the state-of-the-art compiled languages and dynamic scripting languages, like Python's results in the Fib(30) microbenchmark.
%We assert that is not a fundamental limitation because the classic f-expr slowness is being eliminated, as shown by Fig. \ref{fig:kraken_vs_newlisp1} and Fig. \ref{fig:kraken_vs_newlisp2}.
%In future work, we plan to expand our compile-time analysis and optimization to implement a modified, dynamic-language version of Perceus reference counting.
%With this change, we belive \krakenSpace can be competitive with these state-of-the-art languages.

%\subsection{Case Study: Red-Black Tree}
%\label{sec:casestudy}

%\begin{figure}[h]
%\caption{Kraken vs. S.o.t.A. - RB-Tree Focus}
%\includegraphics[width=0.4\textwidth]{rbtree_table.csv_}
%\includegraphics[width=0.4\textwidth]{rbtree_table.csv_log}
%\label{fig:kraken_vs_world_rbtree}
%\end{figure}


%To evaluate our partial evaluation algorithm and compiler, we extracted the benchmarks used by the Koka language project from their code repository and added Kraken versions, as well as implementing a naive Fibonacci microbenchmark ourselves to evaluate pure function call speed.\\
%With partial evaluation and the compiler optimizations listed above, we get fairly strong performance on purely numerical computations, such as the naive Fibonacci microbenchmark.
%Unfortunately, the overhead of our unsophisticated reference counting, dynamic type checking, and bounds checking causes poor performance on benchmarks involving data structures relative to mainstream programming language implementations.
%This is not a fundamental limitation, and will be addressed in future work, as recounted in the next section.
%It should be noted, however, that while the performance relative to established language implementations is very poor for the memory-intensive benchmarks (600-900x slower), we still realize a massive speedup compared to an unoptimized and non-partial-evaluated f-expr implementation (100,000x faster)!


\section{Discussion}
We provide some comments on the growth conditions which constituted the majority of our analysis in sections \ref{sec:Hmixing} and \ref{sec:Hsigma}. In the simplest cases of Lemma \ref{lemma:unstableGrowth}, growth was established in an analogous fashion to the old one-step expansion condition (\ref{eq:oldOneStepExpansion}), finding the relevant Jacobians $M_j$ and checking that their expansion factors $K(M_j)$ satisfy
\begin{equation}
    \label{eq:discussionOneStep}
    \sum_j \frac{1}{K(M_j)} <1.
\end{equation}
For the more complicated cases, the inductive method used to establish growth near the accumulation points in Lemma \ref{lemma:unstableGrowth} and the weakened one-step expansion condition (\ref{eq:oneStep}) both address the same fundamental issue: the splitting of unstable curves by singularities into an unbounded number of small components. They circumvent this obstacle in rather different ways, however. While (\ref{eq:oneStep}) generalises (\ref{eq:discussionOneStep}) to ensure an growth of unstable curves `on average' (see \cite{chernov_statistical_2009} for a precise statement), our inductive method is a more direct adaptation of (\ref{eq:discussionOneStep}), using it to generate contradictory geometric conditions which a hypothetical non-growing unstable curve must satisfy. It may be possible to prove Theorem \ref{sec:Hmixing} using (\ref{eq:oneStep}) as the basis for growth. Since we required (\ref{eq:oneStep}) anyway for proving Theorem \ref{thm:HsigmaExp}, this could potentially condense our analysis, but only to a minor extent. A convenience of the method used in section \ref{sec:Hmixing} is that, by way of the `simple intersection' property, it naturally gives geometric information on the images of manifolds, useful for proving the property \textbf{(M)} of Theorem \ref{thm:katok-strelcyn}.

We expect that essentially analogous analysis can be applied to establish mixing properties in a wide class of piecewise linear non-uniformly hyperbolic maps, including those (like the OTM) which sit on the boundary of ergodicity and beyond. While we have relied on the precise partition structure of $H_\sigma$, its fundamental feature (self-similar sequences of elements $A^k$, sharing boundaries with its neighbours $A^{k-1},A^{k+1}$ and accumulating onto some point $p$) is quite typical to return map systems. See, for example, those of various stadium billiards \cite{chernov_chaotic_2006,chernov_improved_2008,chernov_statistical_2009} and LTMs \cite{springham_polynomial_2014}. Indeed, the same method can be used to prove the Bernoulli property for non-monotonic LTMs \cite{myers_hill_mixing_2022}, where monotonicity of the manifold images cannot be assumed and the classical argument \cite{sturman_mathematical_2006} fails. The OTM is the pointwise limit of these maps as the boundary shrinks to null measure. It further has utility in proving growth conditions for maps which are uniformly hyperbolic but possess regions $A_j$ where the hyperbolicity is very weak, signified by $K(M_j) \approx 1$, so that (\ref{eq:discussionOneStep}) fails. Typically this leads to suboptimal bounds on mixing windows, see e.g. \cite{wojtkowski_model_1981,przytycki_ergodicity_1983,myers_hill_family_2022}. The map $H_{(\eta,\eta)}$ for $\eta \approx 1/2$ is another example, possessing weak hyperbolicity over $A_2, A_3$. Letting $\varepsilon = |\eta-1/2|>0$, there is an upper bound $N = N(\varepsilon)$ on escape times from the intersections $A_2\cap \sigma, A_3 \cap \sigma$. The growth lemma then follows by applying the inductive step roughly $N$ times and can be established for arbitrarily small $\varepsilon$, opening the door to establishing optimal mixing windows.

The above gives two examples of piecewise linear perturbations to $H$ where mixing with respect to Lebesgue is preserved and our methods can be applied. Nonlinear perturbations to the shear profiles complicate the analysis in several ways. Firstly as the map's Jacobians takes on a broader range of values, cone invariance becomes an increasingly harder condition to establish. Cones must be widened, giving looser bounds on expansion factors, which may already be weak due to new regions of weaker stretching. This, together with the change from polygonal to curvilinear return time partition elements and nonlinear local manifolds, adds some complexity to showing growth conditions. This does not rule out certain (small) nonlinear perturbations however. There is some leeway in the inequalities which govern cone invariance and growth of local manifolds, the latter of which is not too dissimilar from the piecewise linear setting (see Lemmas \ref{lemma:piecewiseApprox}, \ref{lemma:componentLength}). Certain small perturbations would not alter the \emph{topological} structure of the return time partition, i.e. which elements share boundaries, the key information needed for setting up the induction. Finally while the partition elements would no longer be polygonal, only coarse geometric information is required for verifying each inductive step. Following the above, a potential perturbation could be to replace the linear portions of each shear by a cubic, perturbing the tent profile
\[  f(t) = \begin{cases} 2t & 0 \leq t \leq 1/2, \\ 2(1-t) & 1/2 \leq t \leq 1 ,\end{cases} \]
of the OTM shears to
\[  f_a(t) = \begin{cases} \frac{1}{8} t \left(16 - a + 6at - 8at^{2} \right) & 0 \leq t \leq 1/2, \\ \frac{1}{8}\left(1-t\right)\left( 16 - a + 6a\left(1-t\right) - 8a\left(1-t\right)^{2}\right)  & 1/2 \leq t \leq 1, \end{cases}   \]
for $a>0$. For small enough $a$ the gradient range $f'(t)$ is restricted to small neighbourhoods of $\{ 2, -2\}$ and the escape time partition retains a similar structure. We illustrate this in Figure \ref{fig:perturbations}, showing escapes from the square $S_3$ under the map $G \circ F$, equivalent to escapes from the perturbed $A_3$ under the $G \circ F$, but with a cleaner geometry for comparison. When $a$ is too large the analogy to the OTM breaks down. At $a=16$ the map is twice differentiable everywhere and features a new source of slowed mixing, the Jacobian is the identity at the corner points $x,y \in \{  0, 1/2 \}$ giving locally parabolic behaviour (visible in the escape time partition). 

\begin{figure}
    \centering
    \includegraphics[width=0.24 \linewidth]{0.png}
    \includegraphics[width=0.24 \linewidth]{4.png}
    \includegraphics[width=0.24 \linewidth]{8.png}
    \includegraphics[width=0.24 \linewidth]{16.png}
    \caption{Partition of escape times from $S_3$ under the mapping $F \circ G$ for $a= 0,4,8,16$. }
    \label{fig:perturbations}
\end{figure}

\section{Related Work}
\section{Related work}
% There is extensive recent work on speeding up analytical queries due to the need for consistent execution times in the face of the explosive growth in the volume of available data.
% In this section, we divide existing work into two categories: maintaining data freshness in MVs (\Cref{sec:server_side}) and optimizations for minimizing ad-hoc query latency (\Cref{sec:client_side}).

% \subsection{Maintaining Data Freshness in MVs}
% \label{sec:server_side}
% There exists a variety of data warehousing applications aimed at supporting low-latency analytical queries on fresh data.
% In particular, these applications require efficiency in the propagation of newly ingested data into downstream MVs.
 
\mypara{Efficient MV Refresh}
Incremental view maintenance (IVM) aims to update MVs to reflect newly ingested data, taking advantage of already computed results to perform the update in a manner more efficient than computing from scratch (full refresh)
~\cite{ahmad2012dbtoaster,mcsherry2013differential,armbrust2013generalized,zeng2016iolap, palpanas2002incremental, griffin1995incremental, agiwal2021napa, braun2015analytics}. 
There is an abundance of work in IVM, including incremental updates on duplicate values~\cite{griffin1995incremental}, non-distributive aggregate functions~\cite{palpanas2002incremental}, higher-order views~\cite{ahmad2012dbtoaster}, and sliding windows~\cite{braun2015analytics}. 
More recent works also investigate the scalability aspect of IVM, proposing scale-independent updates~\cite{armbrust2013generalized} and sampled views~\cite{zeng2016iolap}. Since \system is applicable to arbitrary SQL statements, \system is orthogonal to and is fully compatible with existing IVM techniques.

\mypara{MV Refresh Scheduling}
There exist works on scheduling the refresh of a MV set focusing on resolving cyclic dependencies~\cite{folkert2005optimizing}, minimizing weighted average staleness~\cite{golab2009scheduling}, and prioritizing MVs with the highest speedups on predicted future queries~\cite{ahmed2020automated}.
\system's scheduling to speed up the end-to-end refresh of the MV set is not addressed in existing works.

\mypara{DAG Workflow Scheduling}
The execution of workloads consisting of individual jobs with acyclic dependencies is a well-studied topic~\cite{apacheoozie,sparkdag,marchal2018parallel,bathie2020revisiting,baruah2022ilp}; many of these techniques can be applied to MV refresh runs studied in this paper.
Existing workflow scheduling systems such as Apache Oozie~\cite{apacheoozie}, Apache Airflow~\cite{airflow}, and Spark DAG scheduler~\cite{sparkdag} automate the execution of user-defined workflows following a topological order.
There are recent works aimed at finding more optimal execution orders in terms of peak memory usage~\cite{marchal2018parallel, bathie2020revisiting} and execution time on parallel platforms~\cite{baruah2022ilp}.
While \system is designed for use with MV refresh runs/workloads, our technique on joint scheduling and optimization can be reasonably applied to general workloads as a possible future direction.

% \paragraph{Incremental MV indexing}
% Update-optimized indices such as the log-structured merge-trees (LSM)~\cite{o1996log} are used for indexing MVs due to frequent updates induced by data ingestion~\cite{gupta2016mesa,agiwal2021napa}.
% \system is orthogonal to indexing: \system is capable of efficiently performing MV refresh runs regardless of whether the individual MVs are indexed or not.

% \subsection{Ad-hoc Query Latency Reduction}
% \label{sec:client_side}

% The minimization of ad-hoc analytical query response times is a well-studied topic due to latency being negatively correlated with the productivity of a data analyst during a data analysis session~\cite{liu2014effects}.
% Sessions are commonly conducted within visualization systems that contain a variety of optimization techniques to ensure that query response times fall within a certain latency tolerance.

% \mypara{Data prefetching}
% Data is often loaded into memory on a by-need basis in visualization systems to minimize interference with user-issued query computations~\cite{mani2017effective,xin2021enhancing,galakatos2017revisiting, yan2020auto, battle2016dynamic, crotty2016case, jalaparti2018netco}. 
% Query-time data retrieval can be significantly expedited by anticipating the data usage of the user in future queries and pre-loading the data into memory during the downtime between user queries (`think time'). SMART~\cite{mani2017effective} prefetches data for modified versions of current user-issued queries with different filters and dimensions. A-WARE~\cite{crotty2016case} maintains a sample store constantly refined through ingesting data based on speculations of future plots.
% ForeCache~\cite{battle2016dynamic} uses an SVM to predict the user's current analysis phase and accordingly prefetches data tiles partitioned based on different numerical values. NetCo predicts future queries via log analysis, and solves an ILP formulation to prefetch data to maximize the number of SLO-meeting queries~\cite{jalaparti2018netco}.
% In the case of MV refresh workloads, `think time' is nonexistent as individual MVs are refreshed back-to-back, rendering data prefetching techniques non-applicable.

\mypara{Intermediate Data Caching}
Some existing data visualization systems cache user-defined variables to support the typical incremental construction of data visualizations~\cite{zgraggen2016progressive, eichmann2020idebench} during data analysis sessions~\cite{jupyter, rstudio, colab}. 
Recent work proposes a management scheme for these cached variables under a memory constraint that greedily keeps variables with the highest estimated time savings based on predicted future user behavior via neural networks~\cite{xin2021enhancing}.
While useful for data visualization, a greedy approach to memory management fails to achieve satisfactory results compared to \system.

\mypara{Intermediate Result Reuse}

There exist works on storing intermediate results from computations to speedup future computations~\cite{yang2018intermediate, dursun2017revisiting, nagel2013recycling, michiardi2019memory, galakatos2017revisiting}.
Studied topics include the identification of reuse opportunities by finding overlaps in computation graphs of successive jobs~\cite{yang2018intermediate, michiardi2019memory},
selective storage under a space constraint with heuristics such as reuse probability~\cite{dursun2017revisiting}, expected savings~\cite{yang2018intermediate}, and recompute-storage cost difference~\cite{nagel2013recycling},
and rewriting incoming jobs to take advantage of stored intermediates~\cite{galakatos2017revisiting}.
These works share similarity with \system in their selection of items to store under a memory constraint, however, \system's problem setting requires it to uniquely consider the joint (re)ordering of job executions along with the selection of items.

% work that considers both job execution (re)order as well as intermediate result caching with a bounded amount of memory. but notably lack the joint aspect of \system and cannot be used to achieve immediate speedup on an incoming MV refresh run if no intermediates are stored beforehand. 

\mypara{Incremental Query Processing} Incremental processing (IQP) is useful for cases where not all data required for a query is immediately available. Similar to online aggregation~\cite{hellerstein1997online}, initial results of a query are computed on a subset of required data and progressively refined as the rest of the required data arrives in a predictable pattern~\cite{tang2019intermittent,wangtempura}. Tang et al. propose a dynamic programming formulation to pick intermediate states to store in memory given a limited memory budget~\cite{tang2019intermittent}. Tempura rewrites the query plan for more efficient execution based on predicted data arrival patterns~\cite{wangtempura}. While similarities exist between the problem setting of IQP and \system, such as management of bounded memory, \system notably includes additional joint optimization for the order of MV updates.

% \paragraph{Sampling}
% Sampling has seen wide use in visualization systems for reducing the computation time of ad-hoc queries by computing an approximate result over a subset of data as exact results are not always required by the user~\cite{crotty2016case, mani2017effective, zgraggen2014panoramicdata, kraska2021northstar, galakatos2017revisiting, kandula2016quickr}. 
% Commonly studied topics in sampling for ad-hoc queries include complex query sampling~\cite{kandula2016quickr}, rare event aggregation~\cite{kraska2021northstar, galakatos2017revisiting}, and maintaining consistency between related sampled visualizations~\cite{zgraggen2014panoramicdata}.
% Sampling server-side at the MV level compromises the assumptions of downstream applications and is thus not considered in \system.

% \paragraph{Progressive visualization}
% The latency tolerance for time-consuming queries can be circumvented by presenting a partially-computed visualization to the user within the tolerance, which is then incrementally refined until it is fully accurate~\cite{rahman2017ve, zgraggen2016progressive, crotty2015vizdom, kraska2021northstar, kamat2017infiniviz}.
% Example plots which benefit from progressive visualization include bar charts~\cite{kamat2017infiniviz} and heatmaps~\cite{rahman2017ve}.
% Similar to sampling, study on this topic is orthogonal to \system as pushing out partially-updated MVs compromises downstream assumptions.

\section{Conclusion}
\section{Conclusion}\label{sec:conclusion}
In this work, we focus on addressing the fundamental challenge of OOD detection tasks, which is how to fully understand the semantic discrepancy between the ID/OOD samples. We reveal that the key to success in the realistic SCOOD task is to allocate as many ID samples in the unlabeled set correctly as possible. To this end, we propose a novel uncertainty-aware optimal transport scheme that introduces class-specific energy scores as guidance for effective label assignment. Experimental results show that our method achieves better performance than previous state-of-the-art methods on SCOOD benchmarks.

\textbf{Limitations.} In addition to temperature scaling, other techniques such as feature clipping applied in ReAct~\cite{sun2021react} also enhance the performance of energy score, so how to obtain an OOD score that best fits the SCOOD task can be further explored. Moreover, a setting highly related to SCOOD has been proposed in \cite{katz2022training} and formulated as a constrained optimization problem. We will also theoretically analyze these practical OOD settings in our feature work.

% \section*{Acknowledgments}
\textbf{Acknowledgments.} 
This work is supported by National Key R\&D Program of China under Grant 2020AAA0105701, National Natural Science Foundation of China (NSFC) under Grants 61872327, Major Special Science and Technology Project of Anhui, National Natural Science Foundation of China (62033012) and Ant Group through Ant Research Intern Program.


\balance
\bibliography{ref}
\bibliographystyle{IEEEtran}

\end{document}
