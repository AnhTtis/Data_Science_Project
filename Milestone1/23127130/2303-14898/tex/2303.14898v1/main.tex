%%
%% This is file `sample-sigconf.tex',
%% generated with the docstrip utility.
%%
%% The original source files were:
%%
%% samples.dtx  (with options: `sigconf')
%% 
%% IMPORTANT NOTICE:
%% 
%% For the copyright see the source file.
%% 
%% Any modified versions of this file must be renamed
%% with new filenames distinct from sample-sigconf.tex.
%% 
%% For distribution of the original source see the terms
%% for copying and modification in the file samples.dtx.
%% 
%% This generated file may be distributed as long as the
%% original source files, as listed above, are part of the
%% same distribution. (The sources need not necessarily be
%% in the same archive or directory.)
%%
%% The first command in your LaTeX source must be the \documentclass command.

%\documentclass[sigconf,anonymous=true, review]{acmart}
\documentclass[sigconf]{acmart}

\makeatletter
\def\@ACM@checkaffil{% Only warnings
    \if@ACM@instpresent\else
    \ClassWarningNoLine{\@classname}{No institution present for an affiliation}%
    \fi
    \if@ACM@citypresent\else
    \ClassWarningNoLine{\@classname}{No city present for an affiliation}%
    \fi
    \if@ACM@countrypresent\else
        \ClassWarningNoLine{\@classname}{No country present for an affiliation}%
    \fi
}
\makeatother

\usepackage[ruled,linesnumbered]{algorithm2e}
\usepackage{enumitem}
\usepackage{booktabs}
\usepackage{xcolor}
\usepackage{caption}
\usepackage{subcaption}
\usepackage{amsmath}
\usepackage{multirow}
\usepackage{amsfonts}
\newcommand{\cmark}{\checkmark}%
\newcommand{\xmark}{\text{\sffamily X}}%
\usepackage{color}
\usepackage{colortbl}
\definecolor{mygray}{gray}{.9}
\definecolor{LightCyan}{rgb}{0.88,1,1}
\newcommand{\nop}[1]{}

\newcommand{\tarek}[1]{{\color{red}[Tarek: #1]}}
\newcommand{\ruijie}[1]{{\color{blue}[Ruijie: #1]}}
\newcommand{\chao}[1]{{\color{magenta}[Chao: #1]}}
\newcommand{\zheng}[1]{{\color{teal}[Zheng: #1]}}
\newcommand{\jingfeng}[1]{{\color{green}[Jingfeng: #1]}}
\newcommand{\tianyu}[1]{{\color{violet}[Tianyu: #1]}}

\newcommand{\model}{{MP-KD}\xspace }

\settopmatter{printacmref=true}

%%
%% \BibTeX command to typeset BibTeX logo in the docs
\AtBeginDocument{%
  \providecommand\BibTeX{{%
    \normalfont B\kern-0.5em{\scshape i\kern-0.25em b}\kern-0.8em\TeX}}}

%% Rights management information.  This information is sent to you
%% when you complete the rights form.  These commands have SAMPLE
%% values in them; it is your responsibility as an author to replace
%% the commands and values with those provided to you when you
%% complete the rights form.
\copyrightyear{2023} 
\acmYear{2023} 
\setcopyright{acmlicensed}\acmConference[WWW '23]{Proceedings of the ACM Web Conference 2023}{April 30-May 4, 2023}{Austin, TX, USA}
\acmBooktitle{Proceedings of the ACM Web Conference 2023 (WWW '23), April 30-May 4, 2023, Austin, TX, USA}
\acmPrice{15.00}
\acmDOI{10.1145/3543507.3583407}
\acmISBN{978-1-4503-9416-1/23/04}

% Authors, replace the red X's with your assigned DOI string during the rightsreview eform process.
% \settopmatter{printacmref=false} % Removes citation information below abstract
% \renewcommand\footnotetextcopyrightpermission[1]{} % removes footnote with conference information in first column
% \pagestyle{plain} % removes running headers
%\renewcommand\footnotetextcopyrightpermission[1]{}

\begin{document}
\title{Mutually-paced Knowledge Distillation for Cross-lingual Temporal Knowledge Graph Reasoning}
% Authors must not appear in the submitted version. They should be hidden
% as long as the \iclrfinalcopy macro remains commented out below.
% Non-anonymous submissions will be rejected without review.

\author{Ruijie Wang, Zheng Li$^{2\dagger}$, Jingfeng Yang$^{2}$, Tianyu Cao$^{2}$, Chao Zhang$^{3}$, \\Bing Yin$^{2}$ and Tarek Abdelzaher$^{1\dagger}$}
\thanks{$^{\dagger}$Corresponding authors in UIUC and Amazon.com Inc}
\affiliation{%
  \institution{$^{1}$Department of Computer Science, University of Illinois Urbana-Champaign,~$^{2}$Amazon.com Inc}
  \institution{$^{3}$School of Computational Science and Engineering, Georgia Institute of Technology}
}

\email{{ruijiew2, zaher}@illinois.edu,\space chaozhang@gatech.edu}
\email{{amzzhe, jingfe, caoty, alexbyin}@amazon.com}

\begin{CCSXML}
<ccs2012>
   <concept>
       <concept_id>10010147.10010178.10010187.10010193</concept_id>
       <concept_desc>Computing methodologies~Temporal reasoning</concept_desc>
       <concept_significance>500</concept_significance>
       </concept>
   % <concept>
   %     <concept_id>10010147.10010257</concept_id>
   %     <concept_desc>Computing methodologies~Machine learning</concept_desc>
   %     <concept_significance>500</concept_significance>
   %     </concept>
 </ccs2012>
\end{CCSXML}

\ccsdesc[500]{Computing methodologies~Temporal reasoning}
%\ccsdesc[500]{Computing methodologies~Machine learning}

\keywords{Temporal Knowledge Graph, Cross-lingual Transfer, Knowledge Distillation, Self-training}

%%
%% By default, the full list of authors will be used in the page
%% headers. Often, this list is too long, and will overlap
%% other information printed in the page headers. This command allows
%% the author to define a more concise list
%% of authors' names for this purpose.
\renewcommand{\shortauthors}{Wang, et al.}

% The \author macro works with any number of authors. There are two commands
% used to separate the names and addresses of multiple authors: \And and \AND.
%
% Using \And between authors leaves it to \LaTeX{} to determine where to break
% the lines. Using \AND forces a linebreak at that point. So, if \LaTeX{}
% puts 3 of 4 authors names on the first line, and the last on the second
% line, try using \AND instead of \And before the third author name.

\newcommand{\fix}{\marginpar{FIX}}
\newcommand{\new}{\marginpar{NEW}}

\begin{abstract}
This paper investigates cross-lingual temporal knowledge graph reasoning problem, which aims to facilitate reasoning on Temporal Knowledge Graphs (TKGs) in low-resource languages by transfering knowledge from TKGs in high-resource ones. The cross-lingual distillation ability across TKGs becomes increasingly crucial, in light of the unsatisfying performance of existing reasoning methods on those severely incomplete TKGs, especially in low-resource languages. However, it poses tremendous challenges in two aspects. First, the cross-lingual alignments, which serve as bridges for knowledge transfer, are usually too scarce to transfer sufficient knowledge between two TKGs. Second, temporal knowledge discrepancy of the aligned entities, especially when alignments are unreliable, can mislead the knowledge distillation process. We correspondingly propose a mutually-paced knowledge distillation model \model, where a teacher network trained on a source TKG can guide the training of a student network on target TKGs with an alignment module. Concretely, to deal with the scarcity issue, \model generates pseudo alignments between TKGs based on the temporal information extracted by our representation module. To maximize the efficacy of knowledge transfer and control the noise caused by the temporal knowledge discrepancy, we enhance \model with a temporal cross-lingual attention mechanism to dynamically estimate the alignment strength. The two procedures are mutually paced along with model training. Extensive experiments on twelve cross-lingual TKG transfer tasks in the EventKG benchmark demonstrate the effectiveness of the proposed \model method. 

% compared with ten baselines from three related areas, in both scarce and noisy settings.

%This paper investigates a practical but underexplored problem, namely cross-lingual temporal knowledge graph reasoning. It aims to facilitate reasoning on Temporal Knowledge Graphs (TKGs) in low-resource languages, by distilling knowledge from TKGs in high-resource ones, through a few cross-lingual entity alignments. This new task is important in applications that require timely knowledge from low-resource language TKGs, in light of the unsatisfying performance of existing reasoning methods on those severely incomplete TKGs. However, it poses tremendous challenges in two aspects. First, cross-lingual alignments are usually too scarce to transfer sufficient knowledge to TKGs in low-resource languages. Second, temporal knowledge discrepancy of the aligned entities, especially when alignments are unreliable, can mislead knowledge distillation. We correspondingly propose a mutually-paced knowledge distillation model \model, where a teacher network trained on a source TKG can guide the training of a student network on target TKGs with an alignment module. Concretely, to deal with the co-existing scarcity issue, \model alternatively generates pseudo alignments between TKGs and pseudo events in the target TKG, where two generation procedures are mutually paced along with model training. To maximize the efficacy of knowledge transfer and control the noise caused by knowledge mismatch, we enhance \model with a temporal cross-lingual attention mechanism to dynamically estimate the alignment strength. Extensive experiments on twelve cross-lingual TKG transfer tasks in the EventKG benchmark  demonstrate the effectiveness of the proposed \model method. 
% compared with ten baselines from three related areas, in both scarce and noisy settings.

 %describe time-varying events in real-world. Predicting future events on TKGs, namely temporal knowledge graph reasoning, is important to various knowledge-intensive applications. However, most existing TKG reasoning methods yield limited performance in  low-resource languages due to the scarcity of TKGs. We study cross-lingual temporal knowledge reasoning, which aims to facilitate low-resource temporal reasoning by transferring knowledge from high-resource TKGs to low-resource ones with only a small set of cross-lingual alignments. Transferring knowledge for most entities in low-resource TKGs through such limited alignments is non-trivial. It is also challenging to determine the suitable alignment strength adaptively given the temporal knowledge discrepancy and even unreliable alignments. In light of these challenges, we propose a mutually-paced knowledge distillation model \model, which progressively generates pseudo data to deal with data scarcity for both cross-lingual alignment and  low-resource TKG reasoning. During training, we iteratively generate pseudo alignments to expand the cross-lingual connection, as well as pseudo temporal events to facilitate student model training in low-resource languages.  Our alignment module is learned to dynamically control the alignment strength for different entities at different time, thereby maximizing the benefits of knowledge transferring. Moreover, our theoretical analysis provides convergence guarantee on the new training objective on both groundtruth data and pseudo data. Extensive experiments on 12 language pairs of EventKG data demonstrate the superiority of the proposed model compared with ten baselines from three related areas, in both scarce and noisy settings.


\end{abstract}
\maketitle
\section{introduction}

% 1. importance of TKGs and reasoning on TKGs. 
% 2. low resource languages, main main idea.
% 3. relations and limitations of current works.
% 4. summarize our solutions and contributions.

Temporal Knowledge Graphs (TKGs)~\cite{YAGO,ICEWS18,WIKI,acekg} characterize temporally evolving events, where each event, represented as ({\em subject}, {\em relation}, {\em object}), is associated with temporal information ({\em time}), e.g., ({\em Macron}, {\em reelected}, {\em French president}, {\em 2022}). TKGs has facilitated various knowledge-intensive Web applications with timeliness, such as question answering~\cite{KBQA}, product recommendation~\cite{RippleNet,TKG4Rec,TKG4Rec2,RETE}, and social event forecasting~\cite{KG4Social,DyDiff-VAE,andgan,belief,misinfo,polarization}. 

As new events are continually emerging, modern TKGs are still far from being complete. Conventionally, the TKG construction process relies primarily on information extraction from unstructured corpus~\cite{WIKI,YAGO, EventKG}, which necessitates extensive manual annotations to keep up with changing events. For instance, the recent transition from Trump to Biden as the President of the United States has not been reflected in many TKGs, highlighting the need for timely updates. This spurs research on temporal knowledge graph reasoning to automate evolving events prediction over time~\cite{TA-DistMult,Know-Evolve,Renet,RE-GCN}. Unfortunately, the problem of TKG incompleteness is particularly pronounced in low-resource languages, where it is unable to collect enough corpus and annotations to support robust TKG construction. This results in suboptimal reasoning performance and distinctly unsatisfying accuracy in predicting recent and future events.

% whose performance can degrade significantly in low-resource language TKGs that suffer from severe incompleteness over time. 
% \jingfeng{why don't people  study cross-lingual TKG previously, (i.e. use language alignment to improve TKG). Is it really helpful intuitively to use high resource language to help TKGC? For instance, is it enough to use static langauge-alignment to help KGC, ignoring the temporal information? Are those langauge-alignment changing across time?}



\begin{figure}
    \centering
    \includegraphics[width = 1.0\linewidth]{fig/task.pdf}
    \caption{An illustrative example of cross-lingual reasoning on TKGs. 1) We aim to transfer knowledge from English TKG to Japanese TKG, where the English version provides more complete information; 2) Cross-lingual alignments only cover a small ratio of entities, e.g., Apple Inc; 3) Cross-lingual alignments can be noisy and misleading, e.g., A city called Ventura is linked to new macOS Ventura at $t_2$, introducing noise for reasoning in Japanese.}
    \label{fig:illustration}
    %\vspace{-6mm}
\end{figure}

Inspired by the incompleteness issue facing low-resource languages in constructing TKGs, we introduce a novel task named Cross-Lingual Temporal Knowledge Graph Reasoning (as shown in Figure~\ref{fig:illustration}). This task aims to alleviate the reliance on supervision for TKGs in low-resource languages (referred to as the target language) by transferring temporal knowledge from high-resource languages (referred to as the source language)~\footnote{In this paper, for the sake of brevity, we interchangeably use the terms high-resource/low-resource and source/target.}. In contrast, all the existing efforts are either limited to reasoning in monolingual TKGs (usually high-resource languages, e.g., English)~\cite{TA-DistMult,Know-Evolve,Renet,RE-GCN}, or multilingual static KGs~\cite{KEnS,AlignKGC,SS-AGA}. To the best of our knowledge, cross-lingual TKG reasoning that transfers temporal knowledge between TKGs has not been investigated. 

%Motivated by this, we study a new task named {\em cross-lingual temporal knowledge graph reasoning} as shown in Figure~\ref{fig:illustration}, to alleviate the heavy dependence on supervision for any resource-poor language TKGs by distilling the temporal knowledge from resource-rich ones. Differently, all the existing efforts are either limited to reasoning in monolingual (usually high-resource languages, e.g., English) temporal KGs~\cite{TA-DistMult,Know-Evolve,Renet,RE-GCN}, or multilingual static KG~\cite{KEnS,AlignKGC,SS-AGA}, but neglecting the reasoning in a both temporal and cross-lingual manner that highly requires capturing time-evolving patterns and language discrepancy. To the best of our knowledge, this problem, regarding how to transfer cross-lingual knowledge between TKGs, has still not been formally investigated. 

% Unlike conventional TKG reasoning, 
The fulfillment of this task poses tremendous challenges in two aspects: 1) \textbf{Scarcity of cross-lingual alignment}: as the informative bridge of two separate TKGs, cross-lingual alignment is imperative for cross-lingual knowledge transfer~\cite{AlignKGC,KEnS,SS-AGA}. However, obtaining alignments between languages is a time-consuming and resource-intensive process that heavily relies on human annotations. The transfer of knowledge through a limited number of alignments is often insufficient to fully enhance the TKG in the target language. 2) \textbf{Temporal knowledge discrepancy}: the information associated with two aligned entities is not necessarily identical, especially with regards to temporal patterns. Utilizing a rough approach to equate the aligned entities at all times can result in the transfer of misleading knowledge and negatively impact performance. This becomes more pronounced when the alignments are noisy and unreliable. For example, at the time step $t_2$, a new event about operating system ``{\it Ventura}'' from Apple company occurs in the source English TKG, and meanwhile there is a noisy aligned entity ``{\it Ventura city}'' in the target Japanese TKG. Directly pulling those two entities at this point, can inevitably introduce  noise and fail to predict a set of related events in the target TKG. Therefore, it is crucial to dynamically regulate the alignment strength of each local graph structure over time in order to maximize the effectiveness of cross-lingual knowledge distillation.

% Pulling those entities together cannot augment information in target languages. Small alignment strength is beneficial in the unreliable alignment cases, otherwise the misleading knowledge transferring can even hurt the performance.

% Moreover, in a case that the alignments are not fully reliable, directly pulling the two aligned entities together 


% optimally dynamic alignment strength
% {\em Optimal alignment strength to maximize the benefits of knowledge distillation is difficult to obtain, especially in the temporal manner.} 
% In practical, although the aligned entities can share similar information, they may still differ in other perspectives, including but not limited to frequency, interactions, and temporal patterns. How to adjust the alignment strength (i.e., the distance constrains of the aligned entities in the uni-space) accordingly for different entities at different time is unclear. \zheng{Ruijie TODO: add Ventura case}Moreover, in a case that the alignments are not fully reliable, directly pulling the two aligned entities together can even hurt the performance.



% scarcity of hinders the efficient
% knowledge transfer across languages. 
% {\em Transferring knowledge through a small set of alignments is hard to augment information for all entities.} 

% Aligning the same entities across languages rely heavily on manual labeling or rule-based inference~\cite{EA1,EA2,EA3,selfKG}, which is too time-consuming and impractical to obtain the alignments covering most of the entities in target language. 

% In this paper, we study how to boost the TKG reasoning performance in low-resource languages by explicitly increasing the completeness of those TKGs in history. Instead of improving the underlying information extraction techniques in low-data regime, we propose a new task called {\em Cross-lingual Temporal Knowledge Graph Reasoning}, motivated by the facts that there exists common or complementary knowledge shared by the TKGs in different languages under similar topics. The new task aims to facilitate TKG reasoning in low-resource languages (target languages) by distilling knowledge from a corresponding TKG in high-resource language (source language)  through a small set of entity alignments as bridges~\footnote{In this paper, we interchangeably use the terminology high-resource/low-resource and source/target for briety.}. Figure~\ref{fig:illustration} provides an illustrative example of the proposed task.


% Unfortunately, recent breakthroughs in temporal knowledge graph reasoning model~\cite{TA-DistMult,Know-Evolve,Renet,RE-GCN} highly rely on the completeness of the TKGs, especially for the most recent events. 

% However, the completeness of TKGs varies a lot across different languages, even under similar topics. Conventionally, the TKG construction process relies primarily on information extraction techniques built on the unstructured corpus~\cite{WIKI,YAGO, EventKG}. Therefore, the amount of corpus and human annotations in different languages significantly influence the quality of the corresponding TKGs . 
% Therefore, automatically completing/updating TKGs has been attracting enormous interests in recently years, which aims to predict recent/future events on TKGs based on historical events~\cite{TA-DistMult,Know-Evolve,Renet,RE-GCN}, namely temporal knowledge graph reasoning~\footnote{Broadly speaking, TKG reasoning includes interpolation to predict historical events and extrapolation to predict future events. In this paper, we refer to extrapolation task as TKG reasoning, since it is more vital for time-sensitive downstream tasks.}.


% For languages with large-scale and carefully labeled corpus (we refer to as high-resource languages, e.g., English), the constructed TKGs are more comprehensive than TKGs in other languages that lack the high-quality corpus (we refer to as low-resource languages, e.g., Spanish, Slovene, Danish, etc). Such completeness discrepancy leads to distinctly uneven TKG reasoning performances in different languages, which in turn affects the quality of service of the downstream applications. 


% Compared with the traditional TKG reasoning task, the new task imposes non-trivial challenges. An intuitive solution is to construct a unified graph including two TKGs in both source and target languages, and the knowledge distillation can be fulfilled by pulling the aligned entities from two languages close to each other in the uni-space~\cite{AlignKGC,KEnS}. However, there are still two challenges to be addressed. 

% \zheng{Ruijie TODO, Place this part to related works.}
% Existing works in related areas fail to address the aforementioned challenges. Monolingual reasoning methods on static/temporal knowledge graphs~\cite{TransE,TranR,ComplEX,RotatE,TA-DistMult,Know-Evolve,Renet,RE-GCN} is incapable of the desired knowledge transferring due to the insufficient alignment modeling. Although they can be extended on the cross-lingual scenario by viewing the alignments as a new relation on the merged TKGs, the limited amount of alignments prevent them from augmenting information for most of the entities. Entity alignment methods on KGs~\cite{EA1,EA2,EA3,EA4,EA5,selfKG} can automatically enlarge the alignments by  predicting the correspondence between the two TGs. But most of them, if not all, require the relatively even completeness of two TGs to capture the structural similarities, which can not be satisfied in our case, as target TKGs are far from complete. Some recent works start to study the multilingual TK reasoning on static graphs~\cite{AlignKGC,KEnS,SS-AGA}, which similarly aim to extract knowledge from several source KGs to boost the reasoning performance in the target KG, while they still require a sufficient amount of cross-lingual alignments and totally ignore the temporal perspective in our task.

% to facilitate temporal knowledge graph reasoning in low-resource languages. 
% increase the TKG connection and target TKG capacity
% In light of the mutual benefits, we iteratively generate pseudo alignment pairs and pseudo temporal events to address the co-existing scarcity issue in both cross-lingual alignment and target TKGs. 


In this paper, we propose a novel Mutually-paced Knowledge Distillation (\model) framework, where a teacher network learns more enriched temporal knowledge and reasoning skills from the source TKG to facilitate the learning of a student network in the low-data target one. The knowledge transfer is enabled via an alignment module, which estimates entity correspondence across languages based on temporal patterns. Firstly, to alleviate the limited language alignments (\textbf{Challenge \#1}), such a knowledge distillation process is mutually paced over time. This means, on one hand, we encourage the mutually interactive learning between the teacher and student. Concretely, the alignment module between the teacher and the student learns to generate pseudo alignment between TKGs to maximally expand the upper bound of knowledge transfer. And subsequently, it empowers the student to encode more informative knowledge in target TKG, which can in turn boost the alignment module to explore more reasonable alignments as the bridge across TKGs. One the other hand, inspired by self-paced learning~\cite{spl-1,spl-2}, we make the generations as a progressively easy-to-hard process over time. We start from generating reliable pseudo data with high confidence. As time goes by, we then gradually increase the generation amount by relieving the restriction over time. Secondly, to inhibit the temporal knowledge mismatch (\textbf{Challenge \#2}), the attention module can estimate the graph alignment strength distribution over time. This is achieved by a temporal cross-lingual attention in terms of the local graph structure and temporal-evolving patterns of aligned entities. As such, it can dynamically control the negative effect and suppress noise  propagation from the source TKG. Moreover, we provide a theoretical convergence guarantee for the training objective on both initial ground-truth data and pseudo data. To evaluate \model, we conduct extensive experiments of 12 cross-lingual TKG transfer tasks in multilingual EventKG dataset~\cite{EventKG}. Our empirical results show that the \model method outperforms state-of-the-art baselines in both with and without alignment noise settings, where only $20\%$ of temporal events in the target KG and $10\%$ of cross-lingual alignments are preserved.

% To validate the effectiveness of \model, we conduct extensive experiments of 12 cross-lingual TKG transfer tasks in multilingual EventKG benchmark dataset~\cite{EventKG} . Our experimental results empirically demonstrate the superiority of the \model method over state-of-the-art baselines, ranging from static KG embedding~\cite{TransE,TransR,DistMult,RotatE}, temporal KG reasoning~\cite{TA-DistMult,Renet,RE-GCN} to multilingual KG completion~\cite{KEnS,AlignKGC,SS-AGA}, in both with and without alignment noise settings. We further conduct comprehensive ablation and hyperparameter studies to validate the effectiveness of each design choices. Moreover, we provide theoretical analysis of convergence guarantee for the training objective on both initial groundtruth data and pseudo generative data.



To sum up, our contributions are three-fold:

\begin{itemize}[leftmargin = 15pt]
    \item \textbf{Problem formulation}: We propose the cross-lingual temporal knowledge graph reasoning task, to boost the temporal reasoning performance in target TKG by transferring knowledge from source TKG;
    \item \textbf{Novel framework}: We propose a novel \model framework, which enables the mutually-paced learning between the teacher and student networks, to promote both pseudo alignments and knowledge transfer reliability. Besides, \model involves a dynamic alignment estimation across TKGs that inhibits the influence of temporal knowledge discrepancy.
    \item \textbf{Extensive evaluations}: Empirically, extensive experiments on 12 cross-lingual TKG transfer tasks in multilingual EventKG benchmark dataset demonstrate the effectiveness of \model.
\end{itemize}
% pseudo data generation technique to progressively enhance the training data. The generated pseudo alignments can help the training of the representation modules by the knowledge distillation, and in turn adding pseudo events in the target TKG can improves alignment module by providing high-quality representations. 




% interactively
% TKGs in a source language and a target language are represented by a teacher representation module and a student one into a uni-space, respectively. 
% The knowledge distillation is enabled by a cross-lingual alignment module which pulls the aligned entities close to each other and push other entities far away. 
% To address the challenge caused by the scarcity of cross-lingual alignment, 


\section{Preliminaries and Notations}
In this section, we formally define the cross-lingual temporal knowledge graph reasoning task, and summarize the notations in Table~\ref{tb:notation}. A temporal knowledge graph can be defined as follows:

\begin{table}[t]
    \centering
    \caption{Symbols and Notations.}
    \label{tb:notation}
    \small
    \resizebox{1.0\linewidth}{!}{
    \fontsize{8.5}{11}\selectfont
    \begin{tabular}{c|c}
    \toprule
    \textbf{Symbol} & \textbf{Definition} \\ \midrule
    $(e, r, e^\prime, t)$ & A quadruple in TKG. \\
    $\mathcal{G}_s$, $\mathcal{G}_t$  & Source TKG and Target TKG.\\
    $e_s$, $e_t$ & Entities in the source and target TKGs. \\
    $\Gamma_{s\leftrightarrow t}$ & Alignments between the source and target TKGs. \\
    $\Tilde{\mathcal{G}}_t$, $\Tilde{\Gamma}_{s\leftrightarrow t}$ & Incomplete target TKG and alignments. \\
    $\Tilde{\mathcal{G}}_t^{ST}$, $\Tilde{\Gamma}_{s\leftrightarrow t}^{ST}$ & Pseudo target TKG and  pseudo alignment. \\
    $f(\cdot; \Theta_s)$ & Teacher network on the source TKG.  \\
    $f(\cdot; \Theta_t)$ & Student network on the target TKG.  \\
    $g(e_s, e_t, t; \Phi)$ & Alignment module measuring correspondence of $(e_s, e_t)$ at $t$. \\
    $\mathcal{L}_{\Tilde{\mathcal{G}}_t}$, $\mathcal{L}_{\Tilde{\mathcal{G}}_t^{ST}}$ & Reasoning loss on groundtruth/pseudo target TKG. \\
    $\mathcal{L}_{\Tilde{\Gamma}_{s\leftrightarrow t}}$, $\mathcal{L}_{\Tilde{\Gamma}_{s\leftrightarrow t}^{ST}}$ & Alignment loss on groundtruth/pseudo alignment pairs. \\
    $\mathcal{L}_{s \rightarrow t}$ & Cross-lingual reasoning loss from source TKG to target TKG. \\
    $\mathcal{L}_{s \rightarrow t}^{ST}$ & Cross-lingual reasoning loss on both groundtruth and pseudo data. \\
    \bottomrule
    \end{tabular}}
\end{table}

\begin{definition}[{\bf Temporal Knowledge Graph}]  A temporal knowledge graph (TKG) is denoted as $\mathcal{G} = \{(e, r, e^\prime, t) | t \leq T\} \subseteq \mathcal{E} \times \mathcal{R} \times \mathcal{E} \times \mathcal{T}$, where $\mathcal{E}$ denotes the entities set, $\mathcal{R}$ denotes the relation set, $\mathcal{T}$ denotes the timestamp set, and $T$ denotes the latest update time. Each quadruple $(e, r, e^\prime, t)$ refers to an event that a subject entity $e \in \mathcal{E}$ has a relation $r \in \mathcal{R}$ with an object entity $e^\prime \in \mathcal{E}$ at timestamp $t \in \mathcal{T}$.
\end{definition}

% A temporal knowledge graph (TKG) collected before time $T$ is denoted as $\mathcal{G} = \{(e, r, e^\prime, t) | t \leq T\} \subseteq \mathcal{E} \times \mathcal{R} \times \mathcal{E} \times \mathcal{T}$,

\begin{definition}[{\bf Multilingual TKGs and Alignments}] To denote multilingual TKGs, we further utilize subscript to represent specific languages, i.e., $\mathcal{G}_s$ denotes TKG in the source language and $\mathcal{G}_t$ denotes TKG in the target language. The corresponding entities can be denoted as $e_s$ and $e_t$ respectively. Given two different languages $s$,$t$, we have the cross-lingual alignment set $\Gamma_{s\leftrightarrow t}$. To be more practical, we further assume the TKG in target language $\mathcal{G}_t$ and alignment set $\Gamma_{s\leftrightarrow t}$ are incomplete: $\Tilde{\mathcal{G}}_t$, $\Tilde{\Gamma}_{s\leftrightarrow t}$. 
\end{definition}

% ~\footnote{Following the existing works~\cite{EA1,EA2,EA3,selfKG,AlignKGC,KEnS,SS-AGA}, we assume relations are perfectly aligned, because a few relations (hundred-level) are much easier to processed.}

\begin{figure*}
    \centering
    \includegraphics[width = 0.9\linewidth]{fig/framework.pdf}
    \vspace{-2mm}
    \caption{An overview of \model. (a) The source TKG is more complete than the target TKG, and the cross-lingual alignments are also scarce; (b) A teacher/student representation module to represent source/target TKG, and an alignment module for knowledge transfer; (c) Mutually-paced knowledge distillation between knowledge transfer and pseudo alignment generation.}
    \label{fig:framework}
    \vspace{-3mm}
\end{figure*}

Based on the definition above, we formalize our cross-lingual reasoning task on TKGs as follows:
\begin{definition} [{\bf Cross-lingual reasoning on TKGs}]  Given the TKG $\mathcal{G}_s$ in the source language and the incomplete TKG $\Tilde{\mathcal{G}}_t$ in the target language before the latest update time $T$, and the incomplete cross-lingual alignment $\Tilde{\Gamma}_{s\leftrightarrow t}$, we aim to predict future events in the target TKG after time $T$. Concretely, we aim to predict missing entity in each future quadruple: $\{(e_t,r,?,t)~\text{or}~(?,r,e_t^\prime,t) | t > T\}$ in the target TKG.
\end{definition}

% \noindent
% {\bf Discussion}: Following the existing works in entity alignment~\cite{EA1,EA2,EA3,selfKG} and multilingual knowledge graph reasoning~\cite{AlignKGC,KEnS,SS-AGA}, we assume relation sets are perfectly aligned, because of hundred-level size of the relation set. We hereby utilize $r \in \mathcal{R}$ to denote relations across languages and omit the language subscript for simplicity.

\begin{figure*}[t]
\includegraphics[width=\linewidth]{figure/structure/OPDFormer-figure-V4.2.pdf}
\vspace{-15pt}
\caption{
Our \opdformerbaseline architecture is based on the Mask2Former~\cite{cheng2021masked} architecture.
The left side shows the overall network while the right shows the mask module in detail.  The Mask2Former employs an image backbone and pixel decoder to obtain pixel-level embeddings, which are passed to a transformer decoder with masked attention together with learnable part queries to learn embeddings that are used to predict the part type and mask (by the mask module).
To obtain a high-resolution mask, Mask2Former uses a multi-scale strategy with visual feature maps at increasing resolutions, each of which are fed into the transformer decoder.  The transformer decoder unit is then stacked for $L$ layers.
We enhance the mask module to predict the part motion parameters (motion type, origin, axis) in addition to the part type and mask (see green boxes). The part bounding box is computed directly from the mask.
We investigate three variants of the architecture that predict motion parameters either directly in camera coordinates (-C), or in object coordinates which are then transformed to camera coordinates via a global pose or a per-part object pose.
The pose prediction variants are indicated in \textcolor{orange}{orange} dashed boxes: global pose (`-O' at middle bottom), and per-part object pose (`-P' top right).
The detected parts in the center correspond to the part queries.
}
\label{fig:network-structure}
\end{figure*}

\section{Approach}

We adopt the detect-and-predict strategy for openable part detection and motion parameter estimation, following \citet{jiang2022opd}.
Our architecture replaces the Mask R-CNN~\cite{he2017mask} detection component with Mask2Former~\cite{cheng2021masked}. 
Mask2Former uses the transformer decoder to predict instance masks and classes, matching against ground truth using the Hungarian algorithm during training.
We extend the Mask2Former architecture to predict part motion parameters in the mask module, and create three variants of the architecture that predict the motion parameters using different coordinate frames.
Our key differences from \citet{jiang2022opd} are that: 1) we replace the MaskRCNN segmentation architecture with Mask2Former; and 2) we introduce a per-part object pose prediction (instead of a global object pose prediction).


\subsection{Model variants}

All models eventually predict motion parameters in camera coordinates (\textbf{C}).
However, as noted in prior work~\cite{li2020category,jiang2022opd}, it is useful to predict motion parameters in the object coordinate frame as the motion axes are often parallel to one of the main axes of the object (see supplement).
The object pose is used as a bridge to transform between the object coordinate frame and the camera coordinate frame.


\citet{jiang2022opd} used the entire image to predict a single object pose, ignoring the fact that there could be multiple objects with different poses. 
To alleviate this limitation, we develop two variants of our architecture for pose prediction, predicting a single \emph{global} pose vs predicting a different object pose for each \emph{part}.
By predicting the object pose per part, we can handle multiple objects without explicitly detecting each object.
This allows us to have an object-agnostic method that can generalize across object categories.
In addition, we consider a base variant that predicts directly in the camera coordinates.

\mypara{Camera coordinates.} 
The base variant \opdformerc does not predict the object pose, and predict the motion parameters in the camera coordinate directly (see \Cref{fig:network-structure}, without orange dashed boxes).  Note that it is the direct analogue of the Mask R-CNN based baseline \opdrcnn-C from \citet{jiang2022opd}.

\mypara{Single global pose.}
In this variant, we predict a single global pose for all objects and parts in the input image.
This is predicted directly from the image features of the entire image (see \Cref{fig:network-structure} dashed box with label `-O').
We call this variant \opdformero as it is the direct analogue of the OPDRCNN-O introduced in OPD~\cite{jiang2022opd}.
For \ourdatamulti, we train with the scene coordinates defining a global pose, and transform relevant motion parameters from camera coordinates to these scene coordinates.


\mypara{Per-part object pose.}
When there are multiple objects in an image, each of the objects can have a different pose and its openable parts would have motion parameters strongly correlated with that object's pose.
To account for this, we add an additional head for each part that predicts the object's pose (see \Cref{fig:network-structure} dashed box with `-P' label).
We call this variant \opdformerp and compare it against \opdrcnnop, an extension of the MaskRCNN based model from \citet{jiang2022opd} to predict per-part object pose.
For \ourdatamulti, we leverage the object oriented bounding boxes in MultiScan~\cite{mao2022multiscan} to obtain object poses and transform motion parameters to object coordinates for training.

\mypara{Parameterization.}
In all variants, the motion parameters and object pose are parameterized in the same way as \citet{jiang2022opd}'s \opdrcnn: motion axis and motion origin are 3-dim vectors and object pose is a 12-dim vector (9 for rotation and 3 for translation).
Motion type prediction is trained with a cross-entropy loss and other motion parameters and object pose use a smooth L1 loss with $\beta=1$.

\subsection{Network Architecture and Losses}

The overall architecture and mask module with per-part prediction heads are shown in the left and right sides of \Cref{fig:network-structure}.
Our architecture uses a Mask2Former module for part segmentation and self-attention over parts.
We use the same multiscale pixel decoder and transformer decoder (with 100 queries) as in \citet{cheng2021masked}, and an R50 backbone for fair comparison with \citet{jiang2022opd}'s \opdnet.

For the segmentation and motion losses, we add the auxiliary loss after each transformer decoder.
The object pose loss is determined by the specific architecture variant and is either a single loss term or one loss term per part.

\mypara{Segmentation losses.}
We use the same set of losses as Mask2Former~\cite{cheng2021masked}, including the binary cross-entropy loss ($L_\text{ce}$) and the dice loss ($L_\text{dice}$)~\cite{milletari2016v} for the mask segmentation, and cross-entropy loss ($L_\text{cls}$) for the mask classification: $L_\text{seg} = \lambda_{ce} L_\text{ce} + \lambda_{dice} L_\text{dice} + \lambda_{cls} L_\text{cls}$.
We adopt the loss weights proposed in Mask2Former, $\lambda_{ce} = 5, \lambda_{dice} = 5$ and $\lambda_{cls} = 2$ for matched predictions and $0.1$ for unmatched.

\mypara{Motion losses.}
Motion prediction losses are based on \opdrcnn~\cite{jiang2022opd}.
We use a cross entropy loss for the motion type ($L_\text{c}$), combined with smooth L1 losses for the motion axis ($L_\text{a}$) and motion origin ($L_\text{o}$): $L_\text{mot} = \lambda_c L_c + \lambda_a L_a + \lambda_o L_o$.
We also use the same loss weight ratios.
Specifically, we set $\lambda_c = 2, \lambda_a = 16, \lambda_o = 16$ for our experiments.

\mypara{Object pose loss.}
Object pose prediction is trained under the smooth L1 loss ($L_\text{pose}$) with $\lambda_\text{pose} = 30$.

We sum all of the above losses to obtain the overall loss used during training: $L = L_\text{seg} + L_\text{mot} + \lambda_\text{pose} L_\text{pose}$.

\section{Experiment}

\begin{table}[t]
\caption{Statistics of the datasets.}
\label{tb:data}
\small
\centering
\resizebox{1.0\linewidth}{!}{
    \fontsize{8.5}{11}\selectfont
\begin{tabular}{c|ccccc}
\toprule
\textbf{Languages} & \textbf{Entity} & \textbf{Relation} & \textbf{Quadruple} & \textbf{Train/Val/Test} & \textbf{Time} \\ \midrule
\textbf{English (EN)} & 34,416 & 105 & 602K & 602K/0K/0K & 28 \\
\textbf{French (FR)} & 32,546 & 105 & 580K & 580K/0K/0K & 28 \\ \midrule
\textbf{Spanish (ES)} & 31,808 & 105 & 316K & 114K/136K/66K & 40 \\
\textbf{German (DE)} & 27,657 & 105 & 268K & 97K/114K/56K & 40 \\
\textbf{Italian (IT)} & 23,734 & 94 & 236K & 84K/100K/51K & 40 \\
\textbf{Danish (DA)} & 15,710 & 94 & 125K & 48K/50K/26K & 40 \\
\textbf{Slovene (SL)} & 13,250 & 94 & 55K & 24K/21K/10K & 40 \\
\textbf{Bulgarian (BG)} & 3,508 & 105 & 23K & 8K/9K/6K & 40 \\ 
\bottomrule
\end{tabular}}
\end{table}
\begin{table*}[t]
\caption{Overall Performance without alignment noise. Average results on $5$ independent runs are reported. $*$ indicates the statistically significant improvements over the best baseline, with $p$-value smaller than $0.01$.}
\label{tb:clean}
\small
\centering
\resizebox{0.85\textwidth}{!}{
\fontsize{8.5}{11}\selectfont
\begin{tabular}{c|c|cccccccccccccc}
\toprule
\textbf{Models} & \textbf{Target} & \multicolumn{2}{c}{\textbf{ES}} & \multicolumn{2}{c}{\textbf{DE}} & \multicolumn{2}{c}{\textbf{IT}} & \multicolumn{2}{c}{\textbf{DA}} & \multicolumn{2}{c}{\textbf{BG}} & \multicolumn{2}{c}{\textbf{SL}} & \multicolumn{2}{c}{\textbf{Avg.}} \\ \hline
\textbf{} & \textbf{Source} & \textbf{MRR} & \textbf{H@10} & \textbf{MRR} & \textbf{H@10} & \textbf{MRR} & \textbf{H@10} & \textbf{MRR} & \textbf{H@10} & \textbf{MRR} & \textbf{H@10} & \textbf{MRR} & \textbf{H@10} & \textbf{MRR} & \textbf{H@10} \\ \hline
\rowcolor{LightCyan}
\textbf{RE-GCN w/o source} & \textbf{NA} & 14.31 & 31.85 & 16.32 & 34.19 & 14.59 & 31.64 & 14.19 & 31.24 & 10.27 & 23.44 & 9.33 & 21.63 & 13.17 & 29.00 \\ \hline
\multicolumn{16}{c}{ \textbf{Static KG embedding methods}} \\ \hline
\multirow{3}{*}{\textbf{TransE}~\cite{TransE}} & \textbf{EN} & 11.67 & 26.73 & 15.19 & 31.37 & 9.15 & 21.44 & 12.71 & 23.31 & 10.17 & 23.72 & 9.73 & 21.83 & 11.44 & 24.73 \\
 & \textbf{FR} & 12.37 & 27.79 & 14.01 & 28.30 & 11.38 & 23.19 & 10.05 & 22.10 & 11.88 & 23.01 & 10.63 & 22.44 & 11.72 & 24.47 \\
 \rowcolor{mygray}
 & \textbf{T.R.} & 0.84 & 0.86 & 0.89 & 0.87 & 0.70 & 0.71 & 0.80 & 0.73 & 1.07 & 1.00 & 1.09 & 1.02 & 0.88 & 0.85 \\ \hline
\multirow{3}{*}{\textbf{TransR}~\cite{TransR}} & \textbf{EN} & 11.88 & 28.66 & 16.01 & 32.01 & 8.14 & 22.07 & 13.34 & 24.73 & 10.33 & 23.51 & 8.89 & 22.12 & 11.43 & 25.52 \\
 & \textbf{FR} & 12.01 & 28.32 & 14.58 & 29.51 & 9.93 & 24.66 & 11.90 & 22.64 & 11.98 & 23.44 & 9.27 & 23.88 & 11.61 & 25.41 \\
 \rowcolor{mygray}
 & \textbf{T.R.} & 0.83 & 0.89 & 0.94 & 0.90 & 0.62 & 0.74 & 0.89 & 0.76 & 1.09 & 1.00 & 0.97 & 1.06 & 0.87 & 0.88 \\ \hline
\multirow{3}{*}{\textbf{DistMult}~\cite{DistMult}} & \textbf{EN} & 13.66 & 29.77 & 17.46 & 33.19 & 11.63 & 26.63 & 14.63 & 25.91 & 9.97 & 22.92 & 9.08 & 20.44 & 12.74 & 26.48 \\
 & \textbf{FR} & 12.58 & 28.73 & 16.03 & 31.81 & 12.12 & 27.76 & 11.64 & 22.97 & 9.01 & 23.77 & 10.13 & 21.07 & 11.92 & 26.02 \\
 \rowcolor{mygray}
 & \textbf{T.R.} & 0.92 & 0.92 & 1.03 & 0.95 & 0.81 & 0.86 & 0.93 & 0.78 & 0.92 & 1.00 & 1.03 & 0.96 & 0.94 & 0.91 \\ \hline
\multirow{3}{*}{\textbf{RotatE}~\cite{RotatE}} & \textbf{EN} & 12.99 & 28.89 & 19.87 & 35.46 & 15.62 & 30.14 & 13.44 & 25.79 & 11.10 & 22.98 & 11.37 & 23.99 & 14.07 & 27.88 \\
 & \textbf{FR} & 13.01 & 29.33 & 17.63 & 34.81 & 14.99 & 31.04 & 11.62 & 23.17 & 10.73 & 23.14 & 11.10 & 24.66 & 13.18 & 27.69 \\
 \rowcolor{mygray}
 & \textbf{T.R.} & 0.91 & 0.91 & 1.15 & 1.03 & 1.05 & 0.97 & 0.88 & 0.78 & 1.06 & 0.98 & 1.20 & 1.12 & 1.03 & 0.96 \\ \hline \multicolumn{16}{c}{ \textbf{Temporal KG embedding methods}} \\ \hline
\multirow{3}{*}{\textbf{TA-DistMult}~\cite{TA-DistMult}} & \textbf{EN} & 15.83 & 34.77 & 18.99 & 37.46 & 14.98 & 29.99 & 14.97 & 30.01 & 9.02 & 21.10 & 8.74 & 17.76 & 13.75 & 28.51 \\
 & \textbf{FR} & 16.61 & 35.83 & 17.81 & 37.96 & 15.58 & 31.21 & 13.21 & 28.58 & 9.63 & 22.91 & 9.03 & 18.83 & 13.65 & 29.22 \\
  \rowcolor{mygray}
 & \textbf{T.R.} & 1.13 & 1.11 & 1.13 & 1.10 & 1.05 & 0.97 & 0.99 & 0.94 & 0.91 & 0.94 & 0.95 & 0.85 & 1.04 & 1.00 \\ \hline
\multirow{3}{*}{\textbf{RE-Net}~\cite{Renet}} & \textbf{EN} & 17.58 & 37.97 & 19.03 & 39.46 & 15.88 & 33.69 & 15.03 & 34.77 & 12.01 & 25.72 & 11.07 & 25.64 & 15.10 & 32.88 \\
 & \textbf{FR} & 17.01 & 36.79 & 18.32 & 38.07 & 15.47 & 34.83 & 15.63 & 33.86 & 12.31 & 25.03 & 11.79 & 24.97 & 15.09 & 32.26 \\
  \rowcolor{mygray}
 & \textbf{T.R.} & 1.21 & 1.17 & 1.14 & 1.13 & 1.07 & 1.08 & 1.08 & 1.10 & 1.18 & 1.08 & 1.23 & 1.17 & 1.15 & 1.12 \\ \hline
\multirow{3}{*}{\textbf{RE-GCN}~\cite{RE-GCN}} & \textbf{EN} & 16.88 & 36.54 & 19.84 & 40.17 & 16.17 & 34.84 & 15.99 & 35.62 & 12.22 & 26.02 & 10.63 & 23.38 & 15.29 & 32.76 \\
 & \textbf{FR} & 17.14 & 37.01 & 19.63 & 41.01 & 16.44 & 35.61 & 15.03 & 33.19 & 11.91 & 25.13 & 11.09 & 22.77 & 15.21 & 32.45 \\ 
  \rowcolor{mygray}
 & \textbf{T.R.} & 1.19 & 1.15 & 1.21 & 1.19 & 1.12 & 1.11 & 1.09 & 1.10 & 1.17 & 1.09 & 1.16 & 1.07 & 1.16 & 1.12 \\ \hline 
 \multicolumn{16}{c}{ \textbf{Multilingual KG embedding methods}} \\ \hline
\multirow{3}{*}{\textbf{KEnS}~\cite{KEnS}} & \textbf{EN} & 15.98 & 33.91 & 17.33 & 37.62 & 14.41 & 31.44 & 14.47 & 29.61 & 12.88 & 26.77 & 11.03 & 24.99 & 14.35 & 30.72 \\
 & \textbf{FR} & 17.02 & 34.07 & 16.61 & 37.99 & 15.57 & 33.82 & 13.62 & 30.24 & 12.03 & 24.32 & 10.51 & 23.86 & 14.23 & 30.72 \\
  \rowcolor{mygray}
 & \textbf{T.R.} & 1.15 & 1.07 & 1.04 & 1.11 & 1.03 & 1.03 & 0.99 & 0.96 & 1.21 & 1.09 & 1.15 & 1.13 & 1.09 & 1.06 \\ \hline
\multirow{3}{*}{\textbf{AlignKGC}~\cite{AlignKGC}} & \textbf{EN} & 13.59 & 33.19 & 16.44 & 33.14 & 13.71 & 34.07 & 12.13 & 31.07 & 11.33 & 26.63 & 8.32 & 20.77 & 12.59 & 29.81 \\
 & \textbf{FR} & 13.90 & 34.71 & 17.14 & 34.81 & 14.97 & 33.65 & 12.07 & 30.44 & 10.92 & 25.31 & 9.64 & 21.28 & 13.11 & 30.03 \\
  \rowcolor{mygray}
 & \textbf{T.R.} & 0.96 & 1.07 & 1.03 & 0.99 & 0.98 & 1.07 & 0.85 & 0.98 & 1.08 & 1.11 & 0.96 & 0.97 & 0.98 & 1.03 \\ \hline
\multirow{3}{*}{\textbf{SS-AGA}~\cite{SS-AGA}} & \textbf{EN} & 15.11 & 32.19 & 16.49 & 36.14 & 14.83 & 33.31 & 12.27 & 30.68 & 12.99 & 27.03 & 11.55 & 25.07 & 13.87 & 30.74 \\
 & \textbf{FR} & 16.54 & 33.99 & 18.32 & 37.19 & 15.02 & 32.99 & 11.73 & 29.98 & 11.13 & 25.62 & 11.01 & 23.64 & 13.96 & 30.57 \\ 
  \rowcolor{mygray}
 & \textbf{T.R.} & 1.11 & 1.04 & 1.07 & 1.07 & 1.02 & 1.05 & 0.85 & 0.97 & 1.17 & 1.12 & 1.21 & 1.13 & 1.06 & 1.06 \\ \hline \midrule
\multirow{3}{*}{\textbf{\model*}} & \textbf{EN} & \textbf{19.51} & 41.55 & \textbf{22.84} & \textbf{49.30} & 17.18 & 37.62 & \textbf{18.79} & \textbf{40.01} & \textbf{14.33} & \textbf{30.13} & \textbf{13.87} & \textbf{30.30} & \textbf{17.75} & \textbf{38.15} \\
 & \textbf{FR} & 19.05 & \textbf{42.86} & 21.67 & 46.57 & \textbf{17.92} & \textbf{39.18} & 17.95 & 37.95 & 13.85 & 29.27 & 12.54 & 27.36 & 17.16 & 37.20 \\
  \rowcolor{mygray}
 & \textbf{T.R.} & 1.35 & 1.33 & 1.36 & 1.40 & 1.20 & 1.21 & 1.29 & 1.25 & 1.37 & 1.27 & 1.42 & 1.33 & 1.33 & 1.3 \\ \hline
\rowcolor{LightCyan}  \textbf{Gains} & & 11\% &	13\%&	15\%&	20\%&	9\%&	10\%&	18\%&	12\%&	10\%&	11\%&	18\%&	18\%&	16\%&	16\% \\
 \bottomrule
\end{tabular}}
\end{table*}

We evaluate \model on EventKG data~\cite{EventKG} including 2 source languages and 6 target languages, and we aim to answer the following research questions:
\begin{itemize}[leftmargin = 15pt]
    \item \textbf{RQ1}: How does \model perform compared with state-of-the-art models on the low-resource target languages?
    \item \textbf{RQ2}: How do reliability of alignment information (with various noise ratio) affect model performances?
    \item \textbf{RQ3}: How do each component and important parameters affect \model performance?
\end{itemize}

\subsection{Datasets} % ~\footnote{https://eventkg.l3s.uni-hannover.de/index.html}
We evaluate \model by 12 cross-lingual TKG transfer tasks on EventKG data~\cite{EventKG}, which is a multilingual TKG including 2 source languages and 6 target languages. For each language, we collect events during 1980 to 2022 and split the time span into 40 time steps for training, validation and testing (28/4/8). Table~\ref{tb:data} shows the dataset statistics. We describe the dataset details in Appendix~\ref{ap:data}.

\subsection{Experimental Setup}

\noindent \textbf{Baselines}.
We compare ten state-of-the-art baselines from three related areas. We describe the baseline details in Appendix~\ref{ap:baseline}.
\begin{itemize}[leftmargin = 15pt]
    \item Static KG embedding methods: \textbf{TransE}~\cite{TransE}, \textbf{TransR}~\cite{TransR}, \textbf{DistMult}~\cite{DistMult}, \textbf{RotatE}~\cite{RotatE};
    \item Temporal KG embedding methods: \textbf{TA-DistMult}~\cite{TA-DistMult}, \textbf{RE-NET}~\cite{Renet}, \textbf{RE-GCN}~\cite{RE-GCN}; 
    \item Multilingual KG embedding methods: \textbf{KEnS}~\cite{KEnS}; \textbf{AlignKGC}~\cite{AlignKGC}; \textbf{SS-AGA}~\cite{SS-AGA}. 
\end{itemize}


\noindent \textbf{Evaluation Protocol and Metrics}.
For each prediction $(e, r, ?, t)$ or $(?, r, e, t)$, we rank missing entities to evaluate the performance. Following~\cite{RE-GCN}, we adopt raw mean reciprocal rank ({\em MRR}) and raw Hits at 10 ({\em H@10}) as evaluation metrics. To quantitatively compare how well the transferred knowledge from the source languages can improve predictions on the low-resource languages, we adopt Transfer Ratio ({\em T.R.}) to evaluate the average improvement of each 
method over the best baseline without knowledge transferring, i.e.:
\begin{equation}
    \small
    T.R. (t_i) = \frac{1}{|S|} \sum_{s_i \in \mathcal{S}} \frac{\text{Model}(s_i \rightarrow t_i)}{\text{BestBaseline}(t_i)}
\end{equation}
\noindent
where $t_i$ denotes each target language, $s_i \in \mathcal{S}$ denotes each source language, and $\text{BestBaseline}(t_i)$ denotes the best baseline performance on the target language $t_i$ without any knowledge transferring, i.e., {\em RE-GCN w/o source}.

\noindent \textbf{Implementation}.
To simulate scarce setting, by default, we utilize $10\%$ alignments and $20\%$ events of target TKG by random selection. For static/temporal KG embedding methods, we merge source graph and target graph by adding one new type of relation (alignment). For multilingual baselines, we train them on 1-to-1 knowledge transferring (instead of the original setting) for fair comparison. We introduce implementation details of baseline models and \model in Appendix~\ref{ap:implementation}. Code and data are open-source and available at \url{https://github.com/amzn/mpkd-thewebconf-2023}.

\begin{figure}[t]
    \centering
    \includegraphics[width = 1.0 \linewidth]{fig/noisy.pdf}
    \caption{Experimental results under various alignment noise ratios. Average H@10 on 6 target languages are reported. \model achieves relatively robust results, with only $3.7\%$ relative drop, others have over $10\%$ drop.}
    \label{fig:noise}
    \vspace{-5mm}
\end{figure}

\subsection{Experiments on Cross-lingual Reasoning (RQ1)}
We first evaluate the model performance with incomplete cross-lingual alignments, where we randomly preserve $10\%$ alignments of the target entities for distilling knowledge. Table~\ref{tb:clean} reports the overall results for the cross-lingual experiments. By utilizing only $10\%$ cross-lingual alignments, \model achieves $33\%$ ({\em MRR}) and $30\%$ ({\em H@10}) relative improvement over best baseline without the knowledge transferring ({\em RE-GCN w/o source}) on average, demonstrating the effectiveness of \model in modeling alignments for knowledge transferring. Compared with ten baselines using alignments, \model still achieves  relative $14\%$ relative improvements over the second best results. Specifically, we have the following observations:
\begin{itemize}[leftmargin = 5pt]
\item Static baseline ({\em TransE, TransR, DistMult, RotatE}) fail to beat {\em RE-GCN w/o source}, although using alignments, due the insufficient modeling of temporal information. Similarly, multilingual methods ({\em KEnS, AlignKGC, SS-AGA}) also produce unsatisfying results; 

\item All temporal baselines ({\em TA-DistMult, RE-Net, RE-GCN}) manage to beat {\em RE-GCN w/o source}, as the modeling of both temporal evolution and cross-lingual alignment can facilitate the representation learning of target entities. But the improvements are marginal compared with our model, as the effect of knowledge distillation is constrained by the limited amount of cross-lingual alignments; 

\item Our model consistently achieves the best performance. Through $10\%$ alignments, \model can progressively transfer temporal knowledge and generate pseudo alignments with high confidence to boost the effect and range of the knowledge distillation;

\item We also notice the uneven improvements across languages, (e.g., ~$40\%$ improvements for German, ~$20\%$ for Italian). We hypothesize it is because of various language dependencies with source languages.
\end{itemize}

% \subsubsection{Performance among entity groups}.
% \ruijie{waiting for results.} To dig into the knowledge distillation effects for different groups of entities, we cluster entities in term of event frequency into four groups, and present the detailed performance distribution among groups in Figure~\ref{fig:group}.

\subsection{Experiments under Alignment Noises (RQ2)}
In reality, cross-lingual alignments can be obtained by human labeling or rule-based inference modules, which may introduce indispensable noises. We evaluate how the reliability of alignment information affects baseline models and \model. In this experiment, we still utilize $10\%$ alignments. To simulate unreliable alignments, we select a subset of alignments (measured by {\em Noise Ratio}) and randomly change the aligned target entity to another entity without alignment information.

\begin{table}[t]
\caption{Ablation Studies.}
\label{tb:ablation}
\small
\centering
\resizebox{0.9\linewidth}{!}{
    \fontsize{8.5}{11}\selectfont
\begin{tabular}{c|c|cccccc}
\toprule
\textbf{Ablations} & \textbf{Target} & \multicolumn{2}{c}{\textbf{ES}} & \multicolumn{2}{c}{\textbf{SL}} & \multicolumn{2}{c}{\textbf{Avg.}} \\ \hline
\textbf{} & \textbf{Source} & \textbf{MRR} & \textbf{H@10} & \textbf{MRR} & \textbf{H@10} & \textbf{MRR} & \textbf{H@10} \\ \hline
\multirow{3}{*}{\textbf{\begin{tabular}[c]{@{}c@{}}\model w/o \\ Align. Strength \\ Control\end{tabular}}} & \textbf{EN} & 17.61 & 38.59 & 13.07 & 28.51 & 16.24 & 37.04 \\
\multicolumn{1}{l|}{} & \textbf{FR} & 17.07 & 39.46 & 13.03 & 28.14 & 16.33 & 36.61 \\
\multicolumn{1}{l|}{} &   \textbf{T.R.} & 1.21 & 1.23 & 1.27 & 1.21 & 1.21 & 1.22 \\ \hline
\multirow{3}{*}{\textbf{\begin{tabular}[c]{@{}c@{}}\model w \\ Pure Training\end{tabular}}} & \textbf{EN} & 17.09 & 37.03 & 11.89 & 26.25 & 14.81 & 31.90 \\
 & \textbf{FR} & 16.99 & 37.10 & 11.78 & 25.91 & 14.97 & 32.15 \\
 &   \textbf{T.R.} & 1.19 & 1.16 & 1.15 & 1.11 & 1.13 & 1.09 \\ \hline
\multirow{3}{*}{\textbf{\begin{tabular}[c]{@{}c@{}}\model w/o \\  Pseudo Align.\end{tabular}}} & \textbf{EN} & 17.97 & 38.04 & 12.31 & 27.83 & 15.98 & 34.83 \\
 & \textbf{FR} & 17.55 & 38.45 & 12.19 & 26.32 & 15.79 & 35.27 \\
 &  \textbf{T.R.} & 1.24 & 1.20 & 1.19 & 1.16 & 1.22 & 1.19 \\ \hline
\multirow{3}{*}{\textbf{\begin{tabular}[c]{@{}c@{}}\model w/o \\ Event Transfer\end{tabular}}} & \textbf{EN} & 18.79 & 39.03 & 13.07 & 28.07 & 16.03 & 36.74 \\
 & \textbf{FR} & 18.83 & 39.88 & 12.95 & 28.79 & 15.99 & 36.95 \\
 & \textbf{T.R.} & 1.31 & 1.24 & 1.27 & 1.21 & 1.27 & 1.24 \\ \hline
% \multirow{3}{*}{\textbf{Static Training}} & \textbf{EN} & 19.24 & 40.99 & 13.85 & 29.19 & 17.01 & 37.24 \\
%  & \textbf{FR} & 19.17 & 41.74 & 14.01 & 28.78 & 16.69 & 36.73 \\
%  &   \textbf{T.R.} & 1.34 & 1.30 & 1.36 & 1.24 & 1.3 & 1.28 \\ \midrule
\multirow{3}{*}{\textbf{\model}} & \textbf{EN} & \textbf{19.51} & 41.55 & \textbf{14.33} & \textbf{30.13} & \textbf{17.75} & \textbf{38.15} \\
 & \textbf{FR} & 19.05 & \textbf{42.86} & 13.85 & 29.27 & 17.16 & 37.20 \\
 & \textbf{T.R.} & 1.35 & 1.33 & 1.37 & 1.27 & 1.33 & 1.3  \\
 \bottomrule
\end{tabular}}
\vspace{-5mm}
\end{table}
We vary the noise ratio from $0.0$ to $0.2$ to evaluate the models performance, as shown in Figure~\ref{fig:noise}. We report the average H@10 on 6 target languages by utilizing English TKG and French TKG respectively. As expected, with the increase of noise ratio, the performances of all compared models degrade, as the wrong alignment links mislead the knowledge transfer process. Most baselines fail to beat {\em RE-GCN w/o Source} even with $10\%$ noise, and all lose with $20\% noise$, which indicates that the quality of alignments significantly influences the model effectiveness in the cross-lingual TKG reasoning task. Notably, \model achieves relatively robust results, with only $3.7\%$ performance drop, while other strong baselines have over $10\%$ drop. This is because during the generation of pseudo alignments, \model can automatically replace those unreliable ones based on the confidence score. Also, in the alignment module, \model can assign small alignment strength to unreliable alignments.

\subsection{Model Analysis (RQ3)}
\noindent \textbf{Ablation Study}.
We evaluate performance improvements brought by the \model framework by following ablations: 
\begin{itemize}[leftmargin = 15pt]
\item {\bf \model w/o Align. Strength Control} uniformly set the alignment strength for all entities across all time steps; 
\item {\bf \model w Pure Training} optimizes the teacher-student framework without pseudo alignment generation and temporal event transfer; 
\item {\bf \model w/o Pseudo Align} eliminates the pseudo alignments generation process; 
\item {\bf \model w/o Event Transfer} eliminates the explicit transfer of temporal events.
% \item {\bf Static Training} optimizes the framework in a static manner, i.e., it replaces line 7-10 in Algorithm~\ref{al:training} with static training on all quadruples instead of step by step.
\end{itemize}
Table~\ref{tb:ablation} reports the results measured by $H@10$. Each component leads to performance boost. \model with uniform alignment strength largely degrades performance, due to temporal knowledge discrepancy. \model without pseudo data generation achieves similar performance with temporal baselines {\em RE-Net, RE-GCN}, because of the limited amount of cross-lingual alignments. Bother generating pseudo alignments and explicitly transferring temporal events increase the performance, and combining them together in a mutually-paced procedure (in \model) can achieve the best results. 

\noindent \textbf{The Effect of Pseudo Alignments Ratio}.
To investigate the effects of the pseudo alignments on the reasoning performance, we vary the amount of pseudo alignments during training period and compare the corresponding performance measured by {\em H@10}, as shown in Figure~\ref{fig:pseudoratio}. The blue line and red line show the performances of single model on complete target TKG and single model on $20\%$ target TKG (our setting) respectively. From $0.1$, \model starts to generate and expand the initially available alignments. We observe a significant performance improvement, demonstrating the positive effects of the pseudo alignments. As expected, we find a performance decrease at $50\%$, as the added pseudo data with relatively low confidence start to introduce noise that hurt the performance.


\begin{figure}[t]
    \centering
    \begin{subfigure}[b]{0.495\linewidth}
    \centering
    \includegraphics[width = \linewidth]{fig/alignratio.pdf}
    \caption{Pseudo align. analysis.}
    \label{fig:pseudoratio}
    \end{subfigure}
    \begin{subfigure}[b]{0.495\linewidth}
    \centering
    \includegraphics[width = \linewidth]{fig/time_analysis.pdf}
    \caption{Efficiency analysis.}
    \label{fig:time}
    \end{subfigure}
    \caption{Efficiency analysis and pseudo alignment analysis.}
    \label{fig:analysis}
\end{figure}


\subsection{Efficiency Comparison}
To demonstrate the efficiency of \model framework, we train \model and baseline models from scratch on both target language and source language, and compare the training time. Figure~\ref{fig:time} shows that \model significantly outperforms baseline models with reasonable training time. More details are provided in Appendix~\ref{ap:time}.


\section{Related Work}
\label{sec:related}

We now briefly summarize prior work on standard video compression, learned image compression, and learned video compression.

\textbf{Standard Video Compressors:} Video compression has been attracting attention from the early 2000s due to increasing video content in live streaming and real-time communication. More recently, with the wide spread of COVID-19 across the world, we have seen the importance of reliable and fast communication of high-resolution video content for remote lectures and remote meetings. Standard video compression frameworks typically consist of transform coding, intra-prediction, motion prediction, motion compensation, and entropy coding blocks. The large number of blocks to be designed and optimized for the best rate-distortion performance resulted in many different video coding standards such as ISO/IEC MPEG series \cite{tudor1995mpeg, haskell1996digital, sikora1997mpeg, li2001overview}, ITU-T H.26x series \cite{wiegand2003overview, sullivan2012overview, sze2014high, sullivan2004h, vetro2011overview}, AVS series \cite{yu2009overview, ma2013overview, zhang2019recent}, VP9 \cite{mukherjee2013latest}, and AV1 \cite{chen2018overview, han2021technical}. In our experiments, we use H.264 as the standard video codec between the neural pre-processor and post-processor. Hence, the differential proxy in Figure~\ref{fig:diagram} is designed to best approximate H.264 while maintaining the computational efficiency. \berivan{revisit the baselines once we get the new results.} 

\textbf{Learned Image Compression:}
Since the early work on learned image compression \cite{toderici2015variable}, the rate-distortion performance of learned compressors have gradually outperformed the standard image codecs \cite{balle2016end, balle2016end2, balle2018efficient, balle2016end_pcs, balle2018variational, hu2021learning, minnen2018joint, toderici2017full, mentzer2020high}. This was achieved thanks to the non-linear transforms \cite{balle2020nonlinear} learned through the end-to-end optimization of the Lagrangian loss function $L(\theta)=D(\theta)+\lambda R(\theta)$, where $D(\theta)$ is an expected distortion, $R(\theta)$ is an expected rate, and $\lambda >0$ is a Lagrange multiplier, using a differentiable proxy for the quantizer, mostly modeled as additive uniform noise or bypassed through a straight-through estimator. While they exceed the performance of the standard image codec, learned image compressors require a much greater complexity. Sandwiched image compression \cite{guleryuz2021sandwiched} is one way of breaking the tension between the flexibility and the computational complexity that the neural networks bring. Having a standard image codec between two lightweight neural pre- and post-processor networks, sandwiched image model improved the rate-distortion performance over the standard codec used alone by learning non-linear transforms without increasing the computational complexity much compared to other learned image compression frameworks. Our experiment on compressing a video clip by applying the sandwiched image model on each frame individually verifies that sandwiched image model gives more than 25 dB better reconstructions than the image codec applied alone. See Figure~\ref{fig:intra-comparison} for the experiments with JPEG and H264 (intra codec version) used with grayscale coding. With the sandwiched video model, our goal is to achieve better rate-distortion points than the sandwiched image baseline by training the model with a carefully designed video codec proxy.


\begin{figure}
    \centering
    \subfigure[\em ]{\includegraphics[width=0.45\linewidth]{ICIP/figures/jpeg.png}}
    \subfigure[\em ]{\includegraphics[width=0.45\linewidth]{ICIP/figures/h264_intra.png}}
    \caption{Comparison of (a) JPEG vs. sandwiched JPEG and (b) H264-intra vs sandwiched H264-intra.}
    \label{fig:intra-comparison}
    \vspace{-0.2in}
\end{figure}

\textbf{Learned Video Compression:}
\berivan{Fernando's paper?}


\section{Conclusion}\label{conclusion}
% Restate problem
On heterogeneous systems, it is pragmatic and, therefore, attractive to use a hybrid parallelization, i.e., different simulation modules running on different hardware.
However, hybrid implementations increase the complexity of achieving good performance and scalability, especially on large-scale systems.
% Repeat what has been done
In this paper, we have examined a hybrid coupled fluid-particle simulation with geometrically resolved particles. We use \glspl{gpu} for the fluid dynamics, whereas the particle simulation runs on the \glspl{cpu}.
% Summarize arguments and findings
We have reported and studied the performance of this approach for two cases of a fluidized bed simulation that differ in terms of the number of particles per volume.
The overhead introduced by the hybrid implementation (i.e., \gls{cpu}-\gls{gpu} communication) is negligible because we are transferring only a small amount of data per particle but no fluid cells.
The performance of the fluid simulation is close to utilizing the full memory bandwidth of the A100, implying that using the \gls{gpu} is a good choice for the fluid simulation.
In both cases, the \gls{gpu} routines take most of the run time.
In a weak scaling benchmark, the hybrid fluid-particle implementation reaches a parallel efficiency of 71\% in the dilute case and 53\% in the dense case when using 1024 \gls{cpu}-\gls{gpu} pairs.
The current \gls{pd} methodology requires 32 \gls{cpu}-\gls{cpu} communication steps per time step, which is the driving force for the decrease of the overall parallel efficiency. Our results are limited insofar as different numbers of particle sub-cycles, fluid cells per diameter, etc., will result in different performance results.
% Key takeaways from the paper
We have formulated four criteria that a hybrid implementation must meet to be suitable for the responsible use of heterogeneous supercomputers.
The performance results have shown that our hybrid implementation fulfills all criteria, making it suitable for large-scale simulations on heterogeneous supercomputers.
% Future work / Outlook
In the future, we plan to investigate the particle communication steps in more detail regarding the bottleneck and optimization possibilities. We are employing sub-cycles to increase stability for stiff systems. Using other integrators may permit longer time steps and thus less communication due to sub-cycles. We have shown the acceleration potential of hybrid implementations. Therefore, we plan to run coupled fluid-particle simulations of even larger scenarios to better analyze, among others, the physical phenomena of erosion in sediment beds.


%\newpage
%\bibliographystyle{ACM-Reference-Format}
\bibliographystyle{plain}
\bibliography{main}
\newpage
\appendix
\balance
\section{Appendix}
% \begin{algorithm}[t]
\caption{Generating Pseudo Alignments.}
\label{al:pseudoA}
\small
\KwIn{Current alignment $\Tilde{\Gamma}_{s\leftrightarrow t}$, representation modules $f_s(\cdot;\Theta_s)$ and $f_t(\cdot;\Theta_t)$, alignment module $g(\cdot;\Phi)$}
\KwOut{Pseudo alignment $\Tilde{\Gamma}_{s\leftrightarrow t}^{ST}$.}
Initialize $\Tilde{\Gamma}_{s\leftrightarrow t}^{ST}$ $\leftarrow$ $\{\}$;\\
Collect entity set $\mathcal{N}(\tilde{\mathcal{E}}_t)$ to be aligned based on $\Tilde{\Gamma}_{s\leftrightarrow t}$; \\
Solve Eq.~\eqref{eq:pseudoalign} to get $\{(e_s, e_t) | \phi(e_s, e_t) = 1\}$; \\
Add $(e_s, e_t)$ to $\Tilde{\Gamma}_{s\leftrightarrow t}^{ST}$ if $sim(e_s, e_t) > \text{threshold}_1$.\\
\end{algorithm}
\vspace{-3pt}

% \begin{algorithm}[t]
\caption{Generating Pseudo Events.}
\label{al:pseudoG}
\small
\KwIn{Current alignment $\Tilde{\Gamma}_{s\leftrightarrow t}$, source TKG $\mathcal{G}_s$, representation module $f(\cdot;\Theta_t)$;}
\KwOut{Pseudo Events $\Tilde{\mathcal{G}}_s^{ST}$.}
Initialize $\Tilde{\mathcal{G}}_s^{ST}$ $\leftarrow$ $\{\}$;\\
\For{Each $(e_s, e_t) \in \Tilde{\Gamma}_{s\leftrightarrow t}$}{
    \For{Each $(e_s, r,  e_s^?, t)~\text{or}~(e_s^?, r, e_s, t) \in \mathcal{G}_s$}{
    Add $(e_t, r, e_t^?, t)~\text{or}~(e_t^?, r, e_t, t)$ to $\Tilde{\mathcal{G}}_s^{ST}$ \textbf{if}: \\
        $~~$\textbf{Case 1}: $(e_s^?, e_t^?) \in \Tilde{\Gamma}_{s\leftrightarrow t}$ \\
        $~~$\textbf{Case 2}:
        $~~~~$Select $e_t^?$ with highest ranking via utilizing $f(\cdot;\Theta_n)$; and $f(e_t, r, e_t^?, t;\Theta_t) > \text{threshold}_2$;\\
    }
}
\end{algorithm}

\subsection{Model Description}
\label{ap:attn}
In this section, we introduce the temporal attention layer and cross-lingual attention layer for entity alignments utilized in Section~\ref{sec:align}. We first introduce the general attention mechanism we utilized, then specify the two layers respectively.

Given two representation sequence from temporal domain: key sequence $\boldsymbol{H}_K = \{\boldsymbol{h}_K^{1}, \boldsymbol{h}_K^{2}, \cdots, \boldsymbol{h}_K^{T}\}$ and query sequence $\boldsymbol{H}_Q = \{\boldsymbol{h}_Q^{1}, \boldsymbol{h}_Q^{2}, \cdots, \boldsymbol{h}_Q^{T}\}$  from all time steps, we propose the following attention to calculate the pairwise importance:
\begin{equation}
    \small
    \mathbf{\beta} = \text{Attn}(key = \boldsymbol{H}_K, query = \boldsymbol{H}_Q) = \operatorname{softmax}\left(\frac{\boldsymbol{H}_Q \boldsymbol{W}^Q(\boldsymbol{H}_K \boldsymbol{W}^K)^T}{\sqrt{d}} + \boldsymbol{M}\right),
\end{equation}
\noindent
where $\boldsymbol{W}^Q$, $\boldsymbol{W}^L$ are trainable temporal parameters, $\mathbf{\beta}$ is learned temporal weight indicating pairwise importance, $d$ denotes dimension of input representations, and $\boldsymbol{M}$ is added to ensure auto-regressive setting, i.e., preventing future information affecting current state. We define $\boldsymbol{M}_{ij}=0$ if $i\leq j$, otherwise $-\infty$.

For {\em temporal attention} layer, we use $\boldsymbol{h}_e = \{\boldsymbol{h}_e(1), \boldsymbol{h}_e(2), \cdots, \boldsymbol{h}_e(T)\}$ for both query and key sequence to obtain the temporal attention weights $\beta$:
\begin{equation}
    \small
    \mathbf{\beta} = \text{Attn}(key = \boldsymbol{h}_e, query = \boldsymbol{h}_e),
\end{equation}
\noindent
then the desired $\boldsymbol{H}_e(t)$ is leaned as the combination of input sequence, where $\boldsymbol{W}^V$ is a trainable matrix.:
\begin{equation}
    \begin{aligned}
    \small
    \boldsymbol{H}_e(t) &= \text{Temporal-Attn}(\boldsymbol{h}_e(1), \cdots, \boldsymbol{h}_e(t)) \\
    &= \sum_{i = 1}^{t} \beta_{it} \boldsymbol{h}_e(i)\boldsymbol{W}^V
    \end{aligned}
\end{equation}

For {\em cross-lingual attention} layer, we use $\boldsymbol{H}^s_e = \{\boldsymbol{H}^s_e(1), \cdots, \boldsymbol{H}^s_e(T)\}$ in source language as query sequence and $\boldsymbol{H}^t_e = \{\boldsymbol{H}^t_e(1), \cdots, \boldsymbol{H}^t_e(T)\}$ in target language as key sequence to obtain the attention weights $\beta$:
\begin{equation}
    \small
    \mathbf{\beta}_{e,t} = \text{Attn}(key = \boldsymbol{H}^t_e, query = \boldsymbol{H}^s_e)_{tt},
\end{equation}
\noindent
where $\mathbf{\beta}_{e,t}$ is trainable weight to adjust the alignment strength of different entities at different time.

\subsection{Theorem Proof}
\label{ap:proof}
\begin{theorem}
Let $N$ denote the number of negative samples for optimization, $\epsilon$ denotes the ratio of correct pseudo data, $\beta$ denotes the ratio of pseudo data amount to the initial groundtruth data amount. As the number of negative samples $N \rightarrow \infty$, the $\mathcal{L}_{s\rightarrow t}^{ST}$ converges to its limit with an absolute deviation decaying in $O(\frac{1+\epsilon}{1+\beta}\cdot N ^{-2/3})$.
\end{theorem}
\begin{proof}
In representation learning, the margin loss has been widely adopted as the similarity metric. Without loss of generality, they can be expressed in the form of Noise Contrastive Estimation (NCE)~\cite{NCE}. Therefore, we express $\mathcal{L}_{\mathcal{G}}$ and $\mathcal{L}_{\Gamma_{s\leftrightarrow t}}$ in the form of Noise Contrastive Estimation (NCE) by introducing the negative sampling:
\begin{equation}
\scriptsize
    \mathcal{L}_{\mathcal{G}} \triangleq \mathbb{E}\left[-\log \frac{e^{f(\cdot;\Theta) / \tau}}{e^{f(\cdot;\Theta)/ \tau}+\sum_- e^{\left(f^-(\cdot;\Theta)\right) / \tau}}\right],
\end{equation}

\begin{equation}
\scriptsize
    \mathcal{L}_{\Gamma_{s\leftrightarrow t}} \triangleq \mathbb{E} \left[-\log \frac{e^{g(\cdot;\Phi)) / \tau}}{e^{g(\cdot;\Phi) / \tau}+\sum_- e^{g^-(\cdot;\Phi) / \tau}}\right],
\end{equation}
\noindent
for simplicity, $f^-(\cdot;\Theta)$ denotes the score for negative quadruple, and  $g^-(\cdot;\Phi)$ denotes score for negative alignment pair.

For our training objective $\mathcal{L}_{s \rightarrow t}^{ST}$, we show the convergence analysis of four terms one by one, then prove the overall convergence results. First of all, following~\cite{proof,selfKG}, let $N$ denote the number of negative samples per each quadruple, and we have:
\begin{equation}
\scriptsize
\begin{aligned}
&\lim _{N \rightarrow \infty}\left[\mathcal{L}_{\Tilde{\mathcal{G}}_t}-\log M\right]\\
&=-\frac{1}{\tau} \underset{(e_t, r, e_t^\prime, t) \in \Tilde{\mathcal{G}}_t}{\mathbb{E}}\left[f(\cdot;\Theta)\right] \\
&+\lim _{N \rightarrow \infty} \underset{e^-_t\in \mathcal{E}_t}{\underset{(e_t, r, e_t^\prime, t) \in \Tilde{\mathcal{G}}_t}{\mathbb{E}}}\left[\log \left(\frac{\lambda}{N} e^{f(\cdot;\Theta)/ \tau}+\frac{1}{N} \sum_- e^{f^-(\cdot;\Theta) / \tau}\right)\right]\\ 
&= -\frac{1}{\tau} \underset{(e_t, r, e_t^\prime, t) \in \Tilde{\mathcal{G}}_t}{\mathbb{E}}\left[f(\cdot;\Theta)\right] + \underset{(e_t, r, e_t^\prime, t) \in \Tilde{\mathcal{G}}_t}{\mathbb{E}} \left[\log \underset{e^-_t\in \mathcal{E}_t}{\mathbb{E}}\left[e^{f^-(\cdot;\Theta)}\right]\right],
\end{aligned}
\end{equation}
\noindent
where $\lambda$ denotes the duplicate quadruples co-existing in both incomplete $\Tilde{\mathcal{G}}_t$ and negative samples. The convergence speed is derived as follows:

For one side:
\begin{equation}
\scriptsize
    \mathcal{L}_{\Tilde{\mathcal{G}}_t}-\log N - \lim _{N \rightarrow \infty}\left[\mathcal{L}_{\Tilde{\mathcal{G}}_t}-\log N\right] \leq \frac{\lambda}{N}e^{\frac{2}{\tau}}.
\end{equation}

For another side:
\begin{equation}
    \scriptsize
    \lim _{N \rightarrow \infty}\left[\mathcal{L}_{\Tilde{\mathcal{G}}_t}-\log N\right] - \left[\mathcal{L}_{\Tilde{\mathcal{G}}_t}-\log N\right] \leq \frac{\lambda}{N}e^{2/\tau} + \frac{5}{4} N^{-\frac{2}{3}} e^{\frac{1}{\tau}}(e^{\frac{1}{\tau}}-e^{-\frac{1}{\tau}}).
\end{equation}

We then generalize the above results to the loss term on pseudo data. Suppose $\epsilon$ of the pseudo data are correct. Then we can have the following two inequality. For one side:
\begin{equation}
\scriptsize
\begin{aligned}
    &\mathcal{L}_{\Tilde{\mathcal{G}}_t^{ST}}-\log N - \lim _{N \rightarrow \infty}\left[\mathcal{L}_{\Tilde{\mathcal{G}}_t^{ST}}-\log N\right] \\
    &\leq \epsilon \underset{e\in\mathcal{E}_t}{\mathbb{E}}\left[\log \frac{\underset{e^{-}\in\mathcal{E}_t}{\mathbb{E}}\left[\frac{\lambda}{N}e^{1/\tau} + e^{f^-(\cdot;\Theta_t)/\tau}\right]}{\underset{e^{-}\in\mathcal{E}_t}{\mathbb{E}}e^{f^-(\cdot;\Theta_t)/\tau}} \right] \leq \epsilon \frac{\lambda}{N}e^{\frac{2}{\tau}}
\end{aligned}
\end{equation}

Therefore, for $\mathcal{L}_{s \rightarrow t}^{ST}$ in this side, we have:

\begin{equation}
    \scriptsize
    \mathcal{L}_{s \rightarrow t}^{ST}-\log N - \lim _{N \rightarrow \infty}\left[\mathcal{L}_{s \rightarrow t}^{ST}-\log N\right] \leq \frac{1+\epsilon}{1+\beta} \frac{\lambda}{N}e^{\frac{2}{\tau}},
\end{equation}
where $\beta$ is the ratio of pseudo data amount to groundtruth data amount during training.

Similarly, for another side, we have:
\begin{equation}
\scriptsize
\begin{aligned}
    \lim _{N \rightarrow \infty}\left[\mathcal{L}_{s \rightarrow t}^{ST}-\log N\right] &- \left[\mathcal{L}_{s \rightarrow t}^{ST}-\log N\right] \leq \frac{1+\epsilon}{1+\beta} \frac{\lambda}{N}e^{2/\tau} \\
    &+ \frac{1+\epsilon}{1+\beta} \frac{5}{4} N^{-\frac{2}{3}} e^{\frac{1}{\tau}}(e^{\frac{1}{\tau}}-e^{-\frac{1}{\tau}}).
\end{aligned}
\end{equation}

Therefore, we conclude that the $\mathcal{L}_{s \rightarrow t}^{ST}$ converges to its limit with an absolute deviation decaying in $O(\frac{1+\epsilon}{1+\beta}\cdot N ^{-2/3})$


\end{proof}

\subsection{Datasets}
\label{ap:data}
\noindent \textbf{Dataset Information}.
The commonly utilized benchmark TKGs are divided into two categories: temporal event graphs~\cite{ICEWS18} and knowledge graphs where temporally associated facts have valid periods~\cite{WIKI,YAGO,DBPedia}. In this paper, we mainly evaluate \model on the EventKG~\cite{EventKG}, which is a multilingual resource incorporating event-centric information extracted from several large-scale knowledge graphs such as Wikidata~\cite{WIKI}, DBpedia~\cite{DBPedia} and YAGO~\cite{YAGO}. Each temporal event is organized as $(e, r, e^\prime, t_s, t_e)$, where each piece of data is attached with a valid time period from start time $t_s$ to end time $t_e$. Following~\cite{Renet}, we preprocess the format such that each fact is converted to a sequence $\{(e, r, e^\prime, t_s), (e, r, e^\prime, t_s+1), \cdots, (e, r, e^\prime, t_e)\}$ from $t_s$ to $t_e$, with the minimum time unit as one step.

\noindent \textbf{Splitting Scheme}.
We collect events during 1980 to 2022, and noisy events of early years are removed. To construct multilingual TKGs, we first preserve important entities and relations by excluding infrequent ones that have less than $20$ events in each language. Then we collect the events and cross-lingual alignments.To guarantee the relation match, we only preserve relations appearing in English TKG. We split the time span into 40 equal time steps for training, validation and testing (28/4/8), where each time step roughly lasts for one year. To focus on the prediction on existing entities during training period, and eliminate the negative effects possibly caused by the randomly appearning new entities in val/test period, we only preserve entities having events during training period, following~\cite{RE-GCN}. Table~\ref{tb:data} shows the dataset statistics, including 2 source languages and 6 target languages. We purposefully choose 6 different target languages with diverse characteristics in term of the TKG size, which can evaluate \model from different data granularity. It is worth noting that to simulate the scarcity issue in target TKGs, the training quadruples presented in Table~\ref{tb:data} are randomly selected from original TKGs, with random ratio $20\%$.

\subsection{Baselines}
\label{ap:baseline}
We describe the baselines utilized in the experiments in detail:
\begin{itemize}[leftmargin = 15pt]
    \item \textbf{TransE}~\cite{TransE} is a translation-based embedding model, where both entities and relations are represented as vectors in the latent space. The relation is utilized as a translation operation between the subject and the object entity;
    \item \textbf{TransR}~\cite{TransR}  advances TransE by optimizing modeling of n-n relations, where each entity embedding can be projected to hyperplanes defined by relations;
    \item \textbf{DistMult}~\cite{DistMult} is a general framework with bilinear objective for multi-relational learning that unifies most multi-relational embedding models;
    \item \textbf{RotatE}~\cite{RotatE} represents entities as complex vectors and relations as rotation operations in a complex vector space;
    \item \textbf{TA-DistMult}~\cite{TA-DistMult} is a temporal knowledge graph reansoing method aiming at predicting missing events in history. We utilize it for predicting future events; 
    \item \textbf{RE-NET}~\cite{Renet} is a generative model to predict future facts on temporal knowledge graphs, which employs a recurrent neural network to model the entity evolution, and utilizes a neighborhood aggregator to consider the connection of facts at the same time intervals;
    \item \textbf{RE-GCN}~\cite{RE-GCN} learns the temporal representations of both entities and relations by modeling the KG sequence recurrently;
    \item \textbf{KEnS}~\cite{KEnS} starts to directly improve KGR performance on static KGs given a set of seed alignment, and proposes an ensemble-based approach for the task;
    \item \textbf{AlignKGC}~\cite{AlignKGC} jointly optimizes entity alignment loss and knowledge graph reasoning loss to improve the performance;
    \item \textbf{SS-AGA}~\cite{SS-AGA} views alignments as new edge type and employ a relation-aware GNN with learnable attention weight to model the influence of the aligned entities.
\end{itemize}


\subsection{Reproducibility}
\label{ap:implementation}
\subsubsection{Baseline Setup}
 For static knowledge graph reasoning methods, i.e., TransE, TransR, DistMult, and RotatE, we ignore all time information in quadruples, and view temporal knowledge graphs as static, cumulative ones. For static/temporal KG embedding methods, we merge source graph and target graph by adding one new type of relation (alignment), as they do not explicitly model cross-lingual entity alignment. For multilingual baselines, we train them on 1-to-1 knowledge transferring (instead of the original setting) for fair comparison. For static baselines, we utilize the static embeddings for predictions in all time steps. For fair comparisons, we keep the dimension of all embeddings as $128$, we feed pre-trained TransE embeddings on the merge graph including both source and target TKGs to those that require initial entity/relation embeddings. We tune learning rate of baselines based on {\em MRR} on validation set, and we train all baseline models and \model on same GPUs (Nvidia A100) and CPUs (Intel(R) Xeon(R) Platinum 8275CL).
 
\subsubsection{\model Setup}
We first utilize the source TKG to train the teacher representation module. Then we initialize the student module with the parameters of the teacher. During the training procedure, we first optimize the objective without generating pseudo data in the first 10 epoch. After that, we start to generate high-quality pseudo data. For the generation in each epoch, we gradually increase the amount of pseudo alignments from $10\%$ to $40\%$, and transfer all temporal events that meet the requirement. During evaluation, we tune hyperparameters based on {\em MRR} on validation set, and report the performance on the test set. Next, we report the choices of hyperparameters. For model training, we utilize Adam optimizer, and set maximum number of epochs as $50$. We set batch size as $256$, the dimension of all embeddings as $128$, and dropout rate as $0.5$. For the sake of efficiency,  we set number of temporal neighbors $b$ as $8$, and employ $1$ neighborhood aggregation layer in temporal encoder. For TKG reasoning, we set negative sampling factor as 10. For entity alignment, we set negative sampling factor as 50. For temporal generation process, We divide time span into $4$ time intervals. For model training, we mainly tune margin value $\lambda_1$, $\lambda_2$ in score functions in range $\{0.1,0.2,0.3,0.4,0.5,0.6,0.7,0.8,0.9\}$, learning rate  in range $\{0.02,0.01,0.005,0.001,0.0005\}$. 

\subsubsection{Efficiency Comparison}
\label{ap:time}
To demonstrate the efficiency of \model framework, we train \model and baseline models from scratch on both target language and source language, and compare the training time. We train all baseline models and \model on same GPUs (Nvidia A100) and CPUs (Intel(R) Xeon(R) Platinum 8275CL). Figure~\ref{fig:time} shows that \model significantly outperforms baseline models with reasonable training time. Notably, we include the pseudo data generation time. Compared with slow temporal models {\em RE-NET, RE-GCN} for knowledge graph reasoning, \model is more efficient because our temporal encoder can learn temporal entity embeddings via sampled temporal neighbors at each time without using RNNs.

\end{document}
