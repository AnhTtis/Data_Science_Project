\section{Preliminaries and Notations}
In this section, we formally define the cross-lingual temporal knowledge graph reasoning task, and summarize the notations in Table~\ref{tb:notation}. A temporal knowledge graph can be defined as follows:

\begin{table}[t]
    \centering
    \caption{Symbols and Notations.}
    \label{tb:notation}
    \small
    \resizebox{1.0\linewidth}{!}{
    \fontsize{8.5}{11}\selectfont
    \begin{tabular}{c|c}
    \toprule
    \textbf{Symbol} & \textbf{Definition} \\ \midrule
    $(e, r, e^\prime, t)$ & A quadruple in TKG. \\
    $\mathcal{G}_s$, $\mathcal{G}_t$  & Source TKG and Target TKG.\\
    $e_s$, $e_t$ & Entities in the source and target TKGs. \\
    $\Gamma_{s\leftrightarrow t}$ & Alignments between the source and target TKGs. \\
    $\Tilde{\mathcal{G}}_t$, $\Tilde{\Gamma}_{s\leftrightarrow t}$ & Incomplete target TKG and alignments. \\
    $\Tilde{\mathcal{G}}_t^{ST}$, $\Tilde{\Gamma}_{s\leftrightarrow t}^{ST}$ & Pseudo target TKG and  pseudo alignment. \\
    $f(\cdot; \Theta_s)$ & Teacher network on the source TKG.  \\
    $f(\cdot; \Theta_t)$ & Student network on the target TKG.  \\
    $g(e_s, e_t, t; \Phi)$ & Alignment module measuring correspondence of $(e_s, e_t)$ at $t$. \\
    $\mathcal{L}_{\Tilde{\mathcal{G}}_t}$, $\mathcal{L}_{\Tilde{\mathcal{G}}_t^{ST}}$ & Reasoning loss on groundtruth/pseudo target TKG. \\
    $\mathcal{L}_{\Tilde{\Gamma}_{s\leftrightarrow t}}$, $\mathcal{L}_{\Tilde{\Gamma}_{s\leftrightarrow t}^{ST}}$ & Alignment loss on groundtruth/pseudo alignment pairs. \\
    $\mathcal{L}_{s \rightarrow t}$ & Cross-lingual reasoning loss from source TKG to target TKG. \\
    $\mathcal{L}_{s \rightarrow t}^{ST}$ & Cross-lingual reasoning loss on both groundtruth and pseudo data. \\
    \bottomrule
    \end{tabular}}
\end{table}

\begin{definition}[{\bf Temporal Knowledge Graph}]  A temporal knowledge graph (TKG) is denoted as $\mathcal{G} = \{(e, r, e^\prime, t) | t \leq T\} \subseteq \mathcal{E} \times \mathcal{R} \times \mathcal{E} \times \mathcal{T}$, where $\mathcal{E}$ denotes the entities set, $\mathcal{R}$ denotes the relation set, $\mathcal{T}$ denotes the timestamp set, and $T$ denotes the latest update time. Each quadruple $(e, r, e^\prime, t)$ refers to an event that a subject entity $e \in \mathcal{E}$ has a relation $r \in \mathcal{R}$ with an object entity $e^\prime \in \mathcal{E}$ at timestamp $t \in \mathcal{T}$.
\end{definition}

% A temporal knowledge graph (TKG) collected before time $T$ is denoted as $\mathcal{G} = \{(e, r, e^\prime, t) | t \leq T\} \subseteq \mathcal{E} \times \mathcal{R} \times \mathcal{E} \times \mathcal{T}$,

\begin{definition}[{\bf Multilingual TKGs and Alignments}] To denote multilingual TKGs, we further utilize subscript to represent specific languages, i.e., $\mathcal{G}_s$ denotes TKG in the source language and $\mathcal{G}_t$ denotes TKG in the target language. The corresponding entities can be denoted as $e_s$ and $e_t$ respectively. Given two different languages $s$,$t$, we have the cross-lingual alignment set $\Gamma_{s\leftrightarrow t}$. To be more practical, we further assume the TKG in target language $\mathcal{G}_t$ and alignment set $\Gamma_{s\leftrightarrow t}$ are incomplete: $\Tilde{\mathcal{G}}_t$, $\Tilde{\Gamma}_{s\leftrightarrow t}$. 
\end{definition}

% ~\footnote{Following the existing works~\cite{EA1,EA2,EA3,selfKG,AlignKGC,KEnS,SS-AGA}, we assume relations are perfectly aligned, because a few relations (hundred-level) are much easier to processed.}

\begin{figure*}
    \centering
    \includegraphics[width = 0.9\linewidth]{fig/framework.pdf}
    \vspace{-2mm}
    \caption{An overview of \model. (a) The source TKG is more complete than the target TKG, and the cross-lingual alignments are also scarce; (b) A teacher/student representation module to represent source/target TKG, and an alignment module for knowledge transfer; (c) Mutually-paced knowledge distillation between knowledge transfer and pseudo alignment generation.}
    \label{fig:framework}
    \vspace{-3mm}
\end{figure*}

Based on the definition above, we formalize our cross-lingual reasoning task on TKGs as follows:
\begin{definition} [{\bf Cross-lingual reasoning on TKGs}]  Given the TKG $\mathcal{G}_s$ in the source language and the incomplete TKG $\Tilde{\mathcal{G}}_t$ in the target language before the latest update time $T$, and the incomplete cross-lingual alignment $\Tilde{\Gamma}_{s\leftrightarrow t}$, we aim to predict future events in the target TKG after time $T$. Concretely, we aim to predict missing entity in each future quadruple: $\{(e_t,r,?,t)~\text{or}~(?,r,e_t^\prime,t) | t > T\}$ in the target TKG.
\end{definition}

% \noindent
% {\bf Discussion}: Following the existing works in entity alignment~\cite{EA1,EA2,EA3,selfKG} and multilingual knowledge graph reasoning~\cite{AlignKGC,KEnS,SS-AGA}, we assume relation sets are perfectly aligned, because of hundred-level size of the relation set. We hereby utilize $r \in \mathcal{R}$ to denote relations across languages and omit the language subscript for simplicity.
