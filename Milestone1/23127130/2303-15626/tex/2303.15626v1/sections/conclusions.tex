\section{Conclusions and Outlooks}\label{s:conclusions}
In this paper, we have established a race between classical and quantum generative models in terms of quality-based generalization and defined four types of practical quantum advantage (PQA). Here, we focus on what we referred to as \textit{potential PQA} (PPQA), which aims to compare quantum models with the best-known classical algorithms to the best of our efforts and compute capabilities for the specific task at hand. We have proposed two different competition rules for comparing different models and defining PPQA. We denote these rules as tracks based on the race analogy. We have used QCBMs, RNNs, TFs, VAEs and WGANs to provide a first instance of this comparison on the two tracks. The first track (T1) relies on assuming a fixed sampling budget at the evaluation stage while allowing for an arbitrary number of cost function evaluations. In contrast, the second track (T2) assumes we only have access to a limited number of cost function evaluations, which is the case for applications where the cost estimation is expensive. We also study the impact of the degree of data available to the models for their training. Our results have demonstrated that QCBMs are the most efficient in the scarce-data regime and, in particular, in T2. In general, QCBMs showcase a competitive diversity of solutions compared to the other state-of-the-art generative model in all the tracks and datasets considered here.

It is important to note that the two tracks we chose for this study are not comprehensive, even though they are well motivated by plausible real-world scenarios. One could also use different rules of the game where, for example, the training data can be updated for each training step, as it is customary in the generator-enhanced optimization (GEO) framework~\cite{alcazar2021enhancing}, or where the overall budget takes into account the number of samples required during training. The two tracks introduced here serve the purpose of illustrating the possibilities ahead from this formal approach, In particular, such an approach helps to unambiguously specify the criteria for establishing PQA for generative models in real-world use cases, especially in the context of generative modeling with the goal of generating diverse and valuable solutions, which could boost in turn the solution to combinatorial optimization problems. This characterization is a long-sought-after milestone by many application scientists in the quantum information community, and we believe this framework can provide valuable insights when analyzing the suitability of the adoption of quantum or quantum-inspired models against state-of-the-art classical ones.

Despite the encouraging results obtained from our quantum-based models, we foresee a significant space for potential improvements regarding all the generative models used in this study and some not explored here. In particular, one can embed constraints into generative models such as in $U(1)$-symmetric tensor networks~\cite{Lopez-Piqueres2022} and $U(1)$-symmetric RNNs~\cite{Hibat_Allah_2020, morawetz2021u}. Furthermore, including other state-of-the-art generative models with different variations is vital for establishing a more comprehensive comparison. Lastly, the extension of this work to more realistic datasets is also crucial in the quest to investigate generalization-based PQA. We hope that our work will encourage more comparisons with a broader class of generative models and that it will be diversified to include more criteria for comparison into account.\\