\begin{abstract}
Generative modeling has seen a rising interest in both classical and quantum machine learning, and it represents a promising candidate to obtain a practical quantum advantage in the near term. In this study, we build over the framework proposed in Gili et al.~\cite{gili2022evaluating} for evaluating the generalization performance of generative models, and we establish the first quantitative comparative race towards practical quantum advantage (PQA) between classical and quantum generative models, namely Quantum Circuit Born Machines (QCBMs), Transformers (TFs), Recurrent Neural Networks (RNNs), Variational Autoencoders (VAEs), and Wasserstein Generative Adversarial Networks (WGANs). After defining four types of PQAs scenarios, we focus on what we refer to as \textit{potential PQA}, aiming to compare quantum models with the best-known classical algorithms for the task at hand.  We let the models race on a well-defined and application-relevant competition setting, where we illustrate and demonstrate our framework on 20 variables (qubits) generative modeling task. Our results suggest that QCBMs are more efficient in the data-limited regime than the other state-of-the-art classical generative models. Such a feature is highly desirable in a wide range of real-world applications where the available data is scarce.

\end{abstract}