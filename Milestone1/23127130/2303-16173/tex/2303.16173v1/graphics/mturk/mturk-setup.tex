% \begin{figure*}
%     \centering
%     \includegraphics[width=\textwidth]{graphics/mturk/mturk-task-setup.pdf}
%     \caption{Annotation task set-up for human studies.}
%     \label{fig:tasksetup}
% \end{figure*}

\begin{figure}
    \centering
    \begin{subfigure}[b]{\columnwidth}
        \centering
        \fbox{\includegraphics[width=.95\textwidth]{graphics/mturk/annotation-task.pdf}}
        \caption{Input presentation for \textit{post+stereo} setting. The statement was removed for the \textit{stereo} setting and the stereotype was removed in the \textit{post} setting.}
        \label{fig:task}
    \end{subfigure}
    \begin{subfigure}[b]{\columnwidth}
        \centering
        \fbox{\includegraphics[width=.95\textwidth]{graphics/mturk/annotation-example.pdf}}
        \caption{Example presentation. All five types of counter statements are listed in the same manner.}
        \label{fig:taskexample}
    \end{subfigure}
    \begin{subfigure}[b]{\columnwidth}
        \fbox{\includegraphics[width=.95\textwidth]{graphics/mturk/mturk-task-setup.pdf}}
        \caption{Annotation questions and attention check.}
        \label{fig:taskquestions}
    \end{subfigure}
    \caption{Details of the annotation task for human studies.}
    \label{fig:mturk}
\end{figure}