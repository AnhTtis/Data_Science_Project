\section{Discussion and Conclusion}\label{sec:discussion}
Through our online studies, we find that broadening statements are the most preferred type of counterstatement, while statements with direct counter-evidence are consistently least preferred. In addition, we observe variation across our three settings. Below, we discuss how are findings related to work in psychology (\S\ref{ssec:psych}) and content moderation (\S\ref{ssec:hate-speech}), and finally, outline challenges, limitations, and future directions (\S\ref{ssec:future}).


\subsection{Stereotypes and Psychology}\label{ssec:psych}
Generic language, with its quasi-unique implications, readily conveys essentialist beliefs. 
Indeed, psychological research shows that generic language is a powerful mechanism by which social essentialist beliefs are \textit{transmitted} between people, and even across generations \cite{rhodes2012cultural,leshin2021does}.
Such implications can have a profound impact on children  --- e.g., girls as young as 6 years old have absorbed the stereotype that males are more likely than females to be ``really, really smart'' \cite{bian2017gender}. 
In order to challenge such essentialist beliefs, we argue that it is important to consider the complexities of generics and associated inferences. 

Through reasoning directly about the implications of generics, we construct counterstatements that directly challenge essentialist implications. In particular,
our results highlight the value of broadening statements (\textsc{Lots} and \textsc{Alt}), which counter
the implication that a particular negative quality
is distinctive of a particular group (e.g., ``\textit{Only} women are vain''). This finding is consistent with recent work in psychology, in particular \cite{foster-hanson_leslie_rhodes_2019}.
These statements thereby challenge the cognitive \textit{value} of the stereotype as an information-processing short-cut~\cite{Devine1989StereotypesAP}, since the wide applicability of the stereotyped quality may result in many incorrect inferences (e.g., assuming someone is not vain because they are not a woman). 

Furthermore, our results corroborate findings from psychology that individuals who do not fit a stereotype are not viewed as invalidating that stereotype, since they are categorized as special~\cite[e.g.,][]{kunda1995maintaining}. In particular,  the consistently low preference for direct exception statements comports with that finding (\textsc{Dir-Ind} and \textsc{Dir-Grp}). Although providing facts (e.g., exceptional individuals) has been previously studied as a strategy to counter hate-speech~\cite[e.g.,][]{chung-etal-2019-conan,mathew2019thou}, our work specifically isolates the \textit{type} of facts (i.e., direct counter-evidence versus broadening statements) as a variable for investigation. 
As such, we can observe that providing broadening facts is much more effective than counter-evidence. This further highlights the importance of reasoning about the specific implications of a text to counter essentialist beliefs.

\subsection{Essentialism, Counter Hate-Speech, and Content Moderation}\label{ssec:hate-speech}
Although countering essentialism is similar in spirit to countering hate-speech and content moderation, common strategies in the latter are often inapplicable to countering essentialist beliefs.
In content moderation, discursive actions such as answering clarifying questions or providing additional details are common~\cite{Ziegele2018JournalisticCI}.
However, since essentialist beliefs are often conveyed implicitly (e.g., see statements in Figure~\ref{fig:intro}), discursive actions aimed at a text may not actually address its essentialist implications. For example, the additional detail \textit{``libt$^*$rd is not a real language''} does not actually counter the implication that \textit{liberals are stupid} in Fig~\ref{fig:intro}. Similarly, while humor, expressing affiliation with the targeted group (e.g., \textit{``us Scots only having a wee cuppa tea''}), and pointing out hypocrisy or contradictions (e.g., \textit{``it needs to involve food to be a meal''}) are common when countering hate-speech~\cite{chung-etal-2019-conan,mathew2019thou}, they also do not address the essentialist beliefs implicit in a text (e.g., that \textit{Scots are drunkards}, Figure~\ref{fig:intro}).
As such, we argue that it is important to investigate effective ways to counter essentialist implications, as distinct from general counter-speech and content moderation.

\begin{figure}
    \centering
    \begin{subfigure}[b]{\columnwidth}
        \includegraphics[width=\textwidth]{graphs/images/racial-demo-wwhite.pdf}
        \caption{Racial demographics.}% ``White'' (the majority) is excluded from the graph for clarity.}
        \label{subfig:race}
    \end{subfigure}
    \begin{subfigure}[b]{\columnwidth}
        \includegraphics[width=\textwidth]{graphs/images/gender-demo.pdf}
        \caption{Gender demographics.}
        \label{subfig:gender}
    \end{subfigure}
    \caption{Self-reported annotator demographics (percentage) across settings.
    % \maarten{While i understand why we removed white annotators, I think we should put them back in here, cause the main point to takeaway from this graph should be that most of the annotators were white, and that we have limited number of non-white annotators. This may have been why SJ asked for the results looking at annotators annotating statements where their own minority group is targeted.}
    }
    \label{fig:annotatordemos}
\end{figure}
\subsection{Limitations, Challenges, and Future Directions}\label{ssec:future}

Along with promising preliminary findings, our results highlighted several limitations and challenges that should be tackled in future work.

\paragraph{Human-annotated implications}
Since this work constitutes preliminary investigation on the promise of using NLP tools for combating essentialism, we used a corpus of statements paired with gold human-annotated implications. 
However, such annotations will not always be available. 
Future work should examine whether our findings would hold with machine-generated implications \cite[e.g., using the neural model from][]{sap2020socialbiasframes}, on various types of source domains and overtness levels \cite[e.g., the corpus of implicit toxicity from][]{hartvigsen2022toxigen}.
Furthermore, future research could investigate how the quality and specificity of the implications affects the counterstatement generation and effectiveness.


\begin{table}[t]
    \centering
    \scalebox{0.8}{
    \begin{tabular}{l|r}
        \hline
        \textbf{Group} & \textbf{Nb Examples} \\
        \hline
        Black folks & 66\\
        Women & 60\\
        \hdashline
        Muslim folks & 18\\
        Jewish folks & 16\\
        Asian folks & 15\\
        \hdashline
        Gay men & 7\\
        Latino/Latina folks & 6\\
        Liberals & 5\\
        Feminists & 4\\
        \hdashline
        % Africans & 3\\
        African folks & 3\\
        Mentally disabled folks & 3\\
        % Indians & 3\\
        Indian folks & 3\\
        Lesbian women & 3\\
        Immigrants & 3\\
        Ethiopian folks & 3\\
        % Ethiopians & 3\\
        % Americans & 2\\
        American folks & 2\\
        Mexican folks & 2\\
        % Mexicans & 2\\
        \hdashline
        Physically disabled folks & 1\\
        Folks with mental illness/disorder & 1\\
        % Japanese people & 1\\
        Japanese folks & 1\\
        Polish folks & 1\\
        Arabic folks & 1\\
        % Italians & 1\\
        Italian folks & 1\\
        Christian folks & 1\\
        Native American/First Nation folks & 1\\
        \hline
    \end{tabular}
    }
    \caption{Counts for number of examples per group. There are 227 examples total across 25 unique groups.}
    \label{tab:groupstats}
\end{table}


\paragraph{Targeted group and annotator identity}
Our studies are conducted on Amazon Mechanical Turk which can be lacking in diversity among annotators. For example, the majority of annotators in our study were white (Fig~\ref{subfig:race}) or male (Fig~\ref{subfig:gender}). In contrast, targeted groups are often \textit{not} white or male (see Table~\ref{tab:groupstats}). Since an annotator's identity and beliefs may impact their perceptions of how effective a counterstatement is~\cite[as they do with perceptions of toxicity;][]{sap2022annotatorsWithAttitudes}, homogeneity in the annotator population limits our results. Additionally, how deeply rooted an essentialist belief is for an annotator may impact what they consider effective counterstatements. 
Our results, which show large variation in annotator preference depending on whether they endorse a statement, corroborate these findings.
Therefore, future work should investigate more diverse annotator pools or matching annotators to targeted groups, as well as examining how annotator's familiarity with essentialist beliefs and identities affect their judgements. 


Furthermore, prior work in countering hate-speech has show that effective strategies can vary widely depending on the target group~\citep{mathew2019thou,chung-etal-2019-conan}. 
In our work, we consider results aggregated across all groups.
However, community-specific investigations are an important future step towards developing effective counter-statements. 

\paragraph{Accuracy of generated exceptions}
The selection of specific individuals for direct exceptions presents an ongoing challenge, based on the high number of \textsc{Dir-Ind} marked incorrect. 
Since language models often encode biases and stereotypes derived from training corpora \cite{sheng2019woman}, they may have difficulty producing \textit{relevant} individuals who are not prototypical (i.e., they do not have a particular stereotype).
We illustrate incorrect individuals and subgroups in the bottom two examples of Table~\ref{tab:statementexs}. 
Additionally, as mentioned in \S\ref{ssec:results}, many stereotypes are subjective (e.g., ``women are vain''). 
Therefore, individuals who are counterexamples to the stereotype may be judged differently by different people (e.g., our system proposes that ``taylor swift, sarah palin, and scarlett johansson'' are not vain).
Producing accurate and relevant direct exceptions to a stereotype is important for understanding the role of such examples to counter essentialist beliefs.


Our results and discussion highlight the complexity of countering essentialist beliefs.
We propose that future work should improve the factuality of counterstatements, particularly of direct counter-evidence, and consider both variation in respondent demographics and community-specific needs. Therefore, we argue that working at the intersection of NLP and psychology is crucial for further investigations in this area.
