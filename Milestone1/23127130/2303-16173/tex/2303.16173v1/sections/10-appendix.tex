\section{Data Processing}\label{appsec:dataprocess}
To construct the hedged counterstatments, if the main verb is `is' or `are' we convert it to `can also be'. For example `men are vain' becomes `men can also be vain'. If the main verb is `should' we convert it to `should also'. Otherwise, we insert `may also' before the quality. For example, `men think they know everything' becomes `men may also think they know everything'.

We also note that the group names in Table~\ref{tab:groupstats} have been normalized. We will include both the normalized and unnormalized names in the released data.


\subsection{GPT-3 Generation}\label{appsec:gpt3}
We access GPT-3 using the API from OpenAI\footnote{\url{https://beta.openai.com/docs/introduction}}. To obtain subtypes from GPT-3 we use the \textit{`davinci'}
model and top-$p$ sampling with $p = 0.9$, temperature 0.8 and maximum length $100$ tokens. The presence and frequency penalties are both $0$. We kept the top 5 generations from GPT-3. 
We filter out generations that are the same as the queried group. The prompts are shown in Table~\ref{tab:gpt3prompts}. We randomized the order of the 5 examples in each prompt for every group. 

\begin{table*}[t]
    \centering
    \begin{tabular}{l|l}
        \textbf{Counterstatement Type} & \textbf{Prompt} \\ \hline
        \textsc{Dir-Grp} & 
        \begin{tabular}[t]{@{}l@{}} Consider the following groups of men:\\1. male students\\2. male authors\\3. male atheletes\\4. businessmen\\5. male movie stars\\\#\#\\\#\#\\Consider the following groups of \textsc{GROUP}:\\\end{tabular}\\
        \hline
        \textsc{Dir-Ind} &
        \begin{tabular}[t]{@{}l@{}} Consider the following groups of men:\\1. Barack Obama\\2. Sherlock Holmes\\3. Usain Bolt\\4. Ryan Reynolds\\5. Stephan Hawking\\\#\#\\\#\#\\Consider the following groups of \textsc{GROUP}:\\\end{tabular}\\
        \hline
    \end{tabular}
    \caption{Prompts for generating subtypes for \textsc{GROUP} from GPT3 (e.g., \textsc{GROUP}=women).}
    \label{tab:gpt3prompts}
\end{table*}


\section{Human Studies}\label{appsec:annotation}
For our user studies, we recruit annotators from Amazon Mechanical Turk who were qualified for a toxicity explanation task from our previous work \cite{anon}.\footnote{Anonymized to preserve double-blindness of reviewing, will be de-anonymized upon public release.}
Racial and gender breakdowns of our annotator pool are in Figure~\ref{fig:annotatordemos}.
Annotators were paid \$0.27 per task. 
For each instance in each of the three settings we have 3 annotators.
This study was approved by our institution's ethics board (IRB).

We show the detailed task instructions in Figure~\ref{fig:annotationinstructions}. 
An example of the task setup is shown in Figure~\ref{fig:mturk}.
Before choosing the most convincing counter statements, annotators have the option to mark each statement as incorrect or ungrammatical (Figure~\ref{fig:taskexample}).
Note that before asking annotators to select their second choice, we include an attention check (in Figure~\ref{fig:taskquestions}). The attention check was randomly set in each HIT. Annotations where the attention check incorrect were discarded. As a result, we removed 3 annotations from the \textit{post} setting, 5 from the \textit{stereo} setting, and 4 from \textit{post+stereo}. 
\begin{figure}
    \centering
    \fbox{\includegraphics[width=.95\columnwidth]{graphics/mturk/mturk-instructions.pdf}}
    \caption{Detailed annotation instructions for human studies.}
    \label{fig:annotationinstructions}
\end{figure}
% \begin{figure*}
%     \centering
%     \includegraphics[width=\textwidth]{graphics/mturk/mturk-task-setup.pdf}
%     \caption{Annotation task set-up for human studies.}
%     \label{fig:tasksetup}
% \end{figure*}

\begin{figure}
    \centering
    \begin{subfigure}[b]{\columnwidth}
        \centering
        \fbox{\includegraphics[width=.95\textwidth]{graphics/mturk/annotation-task.pdf}}
        \caption{Input presentation for \textit{post+stereo} setting. The statement was removed for the \textit{stereo} setting and the stereotype was removed in the \textit{post} setting.}
        \label{fig:task}
    \end{subfigure}
    \begin{subfigure}[b]{\columnwidth}
        \centering
        \fbox{\includegraphics[width=.95\textwidth]{graphics/mturk/annotation-example.pdf}}
        \caption{Example presentation. All five types of counter statements are listed in the same manner.}
        \label{fig:taskexample}
    \end{subfigure}
    \begin{subfigure}[b]{\columnwidth}
        \fbox{\includegraphics[width=.95\textwidth]{graphics/mturk/mturk-task-setup.pdf}}
        \caption{Annotation questions and attention check.}
        \label{fig:taskquestions}
    \end{subfigure}
    \caption{Details of the annotation task for human studies.}
    \label{fig:mturk}
\end{figure}

For each annotation, we also collected demographic information (Figure~\ref{fig:demoquestions}). The demographic information is associated only with an annonymized annotator ID. Additionally, before annotators select counter-statements, we ask annotators to indicate their own belief in or agreement with the provided statement and stereotype (Figure~\ref{fig:agreequestions}).
\begin{figure}
    \centering
    \fbox{\includegraphics[width=.95\columnwidth]{graphics/mturk/demographic-qs.png}}
    \caption{Demographic questionnaire in human studies.}
    \label{fig:demoquestions}
\end{figure}
\begin{figure}
    \centering
    \fbox{\includegraphics[width=.95\columnwidth]{graphics/mturk/annotation-agree-qs.pdf}}
    \caption{Questions about stereotype belief of annotators.}
    \label{fig:agreequestions}
\end{figure}









