\section{Introduction}
\textit{Essentialism}, i.e., the belief that members of the same group are fundamentally alike, plays a crucial role in how prejudices and biases about social and demographic groups are formed and expressed \citep{leslie2014carving}.
For example, the statement ``\textit{I speak English, I don't speak libt*rd}'' implies the belief that all ``\textit{liberals are stupid}.''
If left unchallenged,
statements with such essentializing implications can cause harm by perpetuating and reifying stereotypical beliefs about social groups~\cite{greenwald1995implicit,steele2011whistling,prentice2007psychological,rhodes2012cultural,leshin2021does}.

\begin{figure}[t]
    \centering
    \includegraphics[scale=0.265]{images/intro_bear.PNG}
    \caption{\textbf{How does language context impact region-word alignment and downstream detection?} In captions, objects (\eg bear) are often described with rich contextual information (\eg attributes like large, furry, brown) which we hypothesize can impact the effectiveness of object grounding and therefore downstream detection. We show that prior work in detection with region-word alignment does not fully benefit from context such as attributes. We thus provide strategies to better leverage important object context. 
     }
    \label{intro_fig}
\end{figure}

% Such strategy can enable better understanding of fine-grained categories (\eg tuk tuk).

% Multimodal pretraining for object-level tasks involves learning to ground visual regions to caption word embeddings. 

In this work, we investigate the task of combating essentialist statements and beliefs through psychologically and linguistically informed counterstatement generation. 
We examine these essentialist beliefs through the lens of \textit{generics}~\citep{rhodes2012cultural}, i.e., beliefs that attribute a quality to a target group without explicit quantification~\citep[``liberals are stupid'';][]{abelson1966subjective,carlson1995generic}.
In the context of toxic or hateful language, these generic beliefs can be both expressed directly or conveyed through subtle implications~\cite{gelman2003essential,sap2020socialbiasframes}.


Automatically countering essentialism is challenging because it requires deep psychological reasoning about the linguistic implications of statements -- for example, changing people's beliefs about stereotypes only through counterexamples is difficult~\cite{kunda1995maintaining}. Therefore, we examine five different strategies for combating essentializing stereotypes, combining insights from psychology~\cite{foster2016does,foster-hanson_leslie_rhodes_2019,wodak2015loaded} and NLP~\cite{allaway2022penguins}. We craft five types of statements (see Fig~\ref{fig:intro}): \textit{broadening} the scope of a stereotype by generalizing to ``all people'' or an alternative group (\textsc{Lots} and \textsc{Alt}), providing \textit{direct counter-evidence} through specific individuals or groups (\textsc{Dir-Ind} and \textsc{Dir-Grp}), and simply \textit{calling out} the generalization (\textsc{Tol}).
In contrast to prior studies on countering hate-speech which use uncontrolled end-to-end generation approaches ~\cite[e.g.,]{qian2019benchmark,Zhu2021GeneratePS,Chung2020ItalianCN}, we generate counterstatements by reasoning directly about the targeted group, attributed quality, and linguistic expression of a stereotype.


Since our work provides a preliminary exploration of this task, we conduct online studies in three settings where counterstatements are paired with human-written implications from the Social Bias Frames Inference Corpus (SBIC)~\cite{sap2020socialbiasframes}. In these settings, we explore variation in counterstatement effectiveness when the beliefs are conveyed either implicitly, explicitly without context, or as an explicit inference from provided context.
We find that challenging a stereotype by applying it broadly (e.g., to ``lots of people''; \textsc{Lots} and \textsc{Alts}; Figure~\ref{fig:intro}) is generally the most preferred strategy. 
In contrast, statements containing direct counter-evidence (e.g., \textsc{Dir-Ind} and \textsc{Dir-Grp}; Figure~\ref{fig:intro}) are the least popular. 
Additionally, we observe that the most favored strategy varies depending on whether the stereotype is explicitly presented to annotators (e.g., providing the essentialist belief in Figure~\ref{fig:intro}) or only conveyed implicitly (e.g., only providing the first statement in Figure~\ref{fig:intro}). For example, direct counter-evidence is more popular when the stereotype is explicitly provided. Our results highlight the complexity of countering essentialist beliefs and the importance of further investigation at the intersection of NLP and psychology. 
