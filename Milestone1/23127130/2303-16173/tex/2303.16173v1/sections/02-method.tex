\section{
Automatically Countering Essentialism
} \label{sec:method}
We operationalize our counterstatement generation by focusing on the expression of stereotypes through generics (\S\ref{ssec:link-generics-stereotypes}). 
Inspired by work in psychology and philosophy, we construct five types of counterstatements to a stereotype (\S\ref{sec:countertypes}).

\begin{table*}[t]
    \centering
    \scalebox{0.85}{
    \begin{tabular}{ll}
        \hline
        \textbf{TEXT:} \textit{RT @Vbomb20: Got these hoes on my dick like brad pitt} & \textbf{GENERIC:} Women are sex objects.\\
        \hdashline
        \multicolumn{2}{l}{
            \begin{tabular}[t]{@{}l@{}}Actually this is a generalization about women. +\\
            \quad\quad \textbf{(\ref{template1}-\textsc{Grp})}\ The following women are not sex objects: businesswomen, female atheletes, and female movie stars.\\
            \quad\quad \textbf{(\ref{template1}-\textsc{Ind})}\ The following women are not sex objects: ellen degeneres, sarah palin, and rachel maddow.\\
            \quad\quad \textbf{(\ref{template2})}\ Men can also be sex objects.\\
            \quad\quad \textbf{(\ref{template3})}\ Lots of people can be sex objects.\\
            \quad\quad \textbf{(\textsc{Tol})}\ All groups of people deserve tolerance.\end{tabular}
        }\\
        \hline
        % \begin{tabular}[t]{@{}l@{}}\textbf{TEXT:} \textit{They say science disproves Islam But they also say}\\\quad \textit{the world started with a BANG}\end{tabular} & \textbf{GENERIC.:}  muslims blow things up\\
        % \hdashline
        % \multicolumn{2}{l}{
        % \begin{tabular}[t]{@{}l@{}} Actually, this is a generalization about muslims. +\\
        % \quad\quad \textbf{(1a)}\ The following muslims do not blow things: male muslim businessmen, muslims who attend mosque,\\\quad\quad\quad\quad and male muslim movie stars.\\
        % \quad\quad \textbf{(1b)}\ The following muslims do not blow things: adult muslim men, salman rushdie, and mohammad amir.\\
        % \quad\quad \textbf{(2)}\ Christian folks may also blow things.\\
        % \quad\quad \textbf{(3)}\ Lots of people may blow things.\\
        % \quad\quad \textbf{(4)}\ All groups of people deserve tolerance.\end{tabular}
        % }\\
        % \hline
        \textbf{TEXT:} \textit{What's black and doesn't work? Half of London} & \textbf{GENERIC:}  Black people don't work\\
        \hdashline
        \multicolumn{2}{l}{
        \begin{tabular}[t]{@{}l@{}}Actually, this is a generalization about black people. + \\ \quad\quad \textbf{(\ref{template1}-\textsc{Grp})}\ The following black people work: black businessmen, famous black people, and black movie stars.\\
        \quad\quad \textbf{(\ref{template1}-\textsc{Ind})}\ The following black people work: barack obama, misty copeland, and usain bolt.\\
        \quad\quad \textbf{(\ref{template2})}\ White folks may also not work.\\
        \quad\quad \textbf{(\ref{template3})}\ Lots of people don't work.\\
        \quad\quad \textbf{(\textsc{Tol})}\ All groups of people deserve tolerance. \end{tabular}
        }
        \\
        \hline
        \hline
        \begin{tabular}[t]{@{}l@{}}\textbf{TEXT:} \textit{How do you kill a thousand flies in one hit?}\\ \textit{Slap an Ethiopian in the face.}\end{tabular} & \textbf{GENERIC:}  Ethiopian people are dirty.\\
        \hdashline
        \multicolumn{2}{l}{
        \begin{tabular}[t]{@{}l@{}}Actually, this is a generalization about ethiopian people. + \\ \quad\quad \textbf{(\ref{template1}-\textsc{Grp})}\ The following ethiopian people are not dirty: male atheletes, female movie stars, \\\quad\quad\quad\quad\quad\quad\quad and people who practice judaism.\\
        \quad\quad \textbf{(\ref{template1}-\textsc{Ind})}\  The following ethiopian people are not dirty: kenyan marathon runners, michael jackson,\\\quad\quad\quad\quad\quad\quad\quad and ryan reynolds.\\
        % \quad\quad \textbf{(\ref{template2})}\ American folks can also be dirty.\\
        % \quad\quad \textbf{(\ref{template3})}\ Lots of people can be dirty.\\
        % \quad\quad \textbf{(\textsc{Tol})}\ All groups of people deserve tolerance. 
        \end{tabular}
        }
        \\
        \hline
        \begin{tabular}[t]{@{}l@{}}\textbf{TEXT:} \textit{A muslim enters a building..}\\ \textit{With 500 passengers and a plane}\end{tabular} & \textbf{GENERIC:}  Muslims are terrorists.\\
        \hdashline
        \multicolumn{2}{l}{
        \begin{tabular}[t]{@{}l@{}}Actually, this is a generalization about muslims. + \\ \quad\quad \textbf{(\ref{template1}-\textsc{Grp})}\ The following muslims are not terrorists: male muslim businessmen, muslims businessmen, \\\quad\quad\quad\quad\quad\quad\quad  and male muslim movie stars.\\
        \quad\quad \textbf{(\ref{template1}-\textsc{Ind})}\  The following muslims are not terrorists: adult muslim men, all muslims, and malala yousafzai.\\
        \quad\quad\quad\quad\quad\quad\quad $\bm{\hdots}$
        % \quad\quad \textbf{(\ref{template2})}\ Christian folks can also be terrorists.\\
        % \quad\quad \textbf{(\ref{template3})}\ Lots of people can be terrorists.\\
        % \quad\quad \textbf{(\textsc{Tol})}\ All groups of people deserve tolerance. 
        \end{tabular}
        }
        \\
        \hline
    \end{tabular}
    }
    \caption{Automatically generated counterstatements (\S\ref{sec:countertypes}) from our system. The bottom two examples illustrate challenges with factuality in the \textsc{Dir} counterstatements.}
    \label{tab:statementexs}
\end{table*}
\subsection{Stereotypes as Generics}\label{ssec:link-generics-stereotypes}
Many negative stereotypes are expressed as generics; they generalize a dangerous or harmful quality (e.g., being a drunkard) to an entire group (e.g., Scots) based on the behavior of only a few individuals. ~\citet{leslie2008generics,leslie2017original} termed such generics \textbf{striking} and argued that such generalizations are based upon an assumption that all members of the group in question (e.g., Scots) are \textit{disposed} to possess the dangerous or harmful quality. We argue that many stereotypes can also be interpreted as asserting a \textbf{quasi-unique} association between the group and quality. For example, ``Scots are drunkards''  also implies that Scots are distinctly more likely than other groups (e.g., the English) to exhibit drunkenness.
In our work, we assume that all stereotypes under consideration are generics and have both interpretations. 

Since generics are unquantified, they naturally allow for \textbf{exceptions} (i.e., counterexamples to the generic).
While these exceptions may provide a relevant source of counter-statements for a stereotype, some evidence from psychology suggests that people are adept at maintaining their stereotyped beliefs in the face of such specific exceptions \cite[e.g.,][]{kunda1995maintaining}. Therefore, we experiment with a variety of different counter-statements.

\subsection{Generating Counter-Speech}
\label{sec:countertypes}
To generate counter-speech to stereotypes, we produce five types of outputs in three broad categories (see Table~\ref{tab:statementexs}). Since the stereotypes we consider are expressed as generics (e.g., ``Scots are drunkards''), they can be separated into three components: a \textit{group} (e.g., Scots), a \textit{relation} (e.g., are), and a \textit{quality} (e.g., ``drunkards''), which we use to construct the counter-speech. Additionally, we prepend the sentence ``Actually, this is a generalization about \textsc{GROUP}'' to each type of statement we generate, in order to contextualize the statements as counter-speech.


\paragraph{Direct Exceptions (\textsc{Dir})}
Direct exceptions present subgroups or individuals that do not have the quality specified in the generic, and thereby counter the striking or extrapolating implications of the stereotype. For example, for ``Scots are drunkards'', the extrapolating implication is that ``\textit{All} Scots are drunkards''; thus, direct exceptions would be either individual Scots (e.g., Ewan McGregor\footnote{\url{https://fherehab.com/learning/celebrities-who-dont-drink}}) or sub-groups of Scots (e.g., Scottish babies) who are not drunkards. We follow~\citet{allaway2022penguins} who propose that these exceptions can be constructed with the following template:
\setlength{\abovedisplayskip}{3pt}
\setlength{\belowdisplayskip}{3pt}
\begin{align}
    &\textsc{Group}(x) + \text{not } \textit{relation} + \textsc{Quality}. \label{template1}\tag{\textsc{Dir}}
\end{align}
We say that $\textsc{Group}(x)$ is satisfied if $x$ is either a specific member of the group or a subgroup. 
We generate subtypes (i.e., subgroups and specific group members) using GPT-3~\citep{brown2020language}. In particular, we prompt GPT-3 with a list of subtypes for an example group not in our data and query the model to produce subtypes for \textsc{Group} as the prompts completion. We choose as our example group ``men'' (see Appendix~\ref{appsec:gpt3} for prompts).
We then construct exceptions following template~\ref{template1} using each generated subtype. 
In order to select the most truthful and relevant subtypes, we apply a truth discriminator from \citet{allaway2022penguins} 
to each exception, and rank the subtypes by the probability of being true and relevant. 
We construct the final statements by combining the top three ranked subgroups into a single exception ((\ref{template1}-\textsc{Grp}) in Table~\ref{tab:statementexs}) and combining the top three individuals into a single exception ((\ref{template1}-\textsc{Ind}) in Table~\ref{tab:statementexs}).

\paragraph{Broadening Exceptions (\textsc{Alts})}
Broadening exceptions challenge the quasi-unique implication of the generic by attributing the quality in question to a different social group (e.g., ``Americans can also be drunkards''). ~\citet{allaway2022penguins} propose that these exceptions follow the template:
\begin{align}
    &{\nsim}\textsc{Group}(x) + \textit{relation} + \textsc{Quality}.\label{template2}\tag{\textsc{Alt}}
\end{align}
where ${\nsim}\textsc{Group}$ indicates a contextually relevant alternative group. For example, if \textsc{Group} = \textsc{Scots}, then a contextually relevant alternative would be ${\nsim}\textsc{Group}$ = \textsc{Americans}. In our work, we define the relevant alternative group ${\nsim}\textsc{Group}$ to be the perceived oppressing group. For example, if the generic is ``women are vain'', then ``men'' would be the relevant alternative group ${\nsim}\textsc{Women}$ (i.e., the oppressing group). To avoid generating stereotypes about the oppressing group, we convert the relation into a hedged form (see Appendix~\ref{appsec:dataprocess}). For example, if the relation is ``are'', the hedged form of the relation would be ``can be''. 

\begin{figure*}[t!]
    \centering
    \begin{subfigure}[b]{0.5\textwidth}
        \centering
        \includegraphics[width=\textwidth]{graphs/images/first-choice.pdf}
        \caption{First choice.}
        \label{subfig:firstchoice}
    \end{subfigure}%
    \begin{subfigure}[b]{0.5\textwidth}
        \centering
        \includegraphics[width=\textwidth]{graphs/images/second-choice.pdf}
        \caption{Second choice.}
        \label{subfig:secondchoice}
    \end{subfigure}
    \caption{Percentage of annotators that selected each counterstatement type (\S\ref{sec:method}) across all three settings.
    % \maarten{Can we add the raw percentage number (e.g., `15') in the bars?}
    }
    \label{fig:choicecharts}
\end{figure*}


\paragraph{Broadening Universals (\textsc{Lots})}
In addition to broadening exceptions, we generate \textit{broadening universals}, which maximize the scope of the quality so that it includes people in general, rather than any specific social group. That is, we generate statements following:
\begin{align}
    \text{Lots of people} + \textit{relation} + \textsc{Quality}.\label{template3}\tag{\textsc{Lots}}
\end{align}
For example, ``Lots of people are drunkards" is a broadening universal for the stereotype ``Scots are drunkards''. See (\ref{template3}) in Table~\ref{tab:statementexs}. Similarly to the statements following template~\ref{template2}, we also hedge the relation in template~\ref{template3}. 



\paragraph{Tolerance (\textsc{Tol})}
Finally, we include the denouncing statement, ``All groups of people deserve tolerance'', since denouncing is a common strategy in countering hate-speech~\cite[e.g.,][]{mathew2019thou,qian2019benchmark,Ziegele2018JournalisticCI}. This form of counter-speech does not depend on the details of the generic in question and so is the same for all stereotypes. See (\textsc{Tol}) in Table~\ref{tab:statementexs}. 
