% \section{Plug-and-Play Compression in Applications.}
% \label{sec:applications}

% The main distinction between the compression techniques of Gentry et al.~\cite{gentryCompressibleFHEApplications2019} and Brakerski et al.~\cite{brakerskiLeveragingLinearDecryption2019} and what we proposed is that our technique can be plugged into existing implementations of encryption schemes and existing applications.
% In contrast, Gentry et al. designed a custom PIR scheme that can use their compression method.
% In this section, we provide examples of how our compression technique can be integrated into existing protocols and elaborate on the advantages it offers.

% % Moreover, we show how compression can make a practical difference in costs for these applications.
% % The biggest difference is in applications where the download cost is the main burden on communication.
% % This is commonly the case because upload costs can be compressed with some practical techniques such as seeding the randomness and or transciphering, as mentioned in \Cref{sec:related-work}.

% % We will measure the savings, particularly in terms of dollars in some cases.
% % Less download always saves money for the client.
% % But we will show how in some applications, this will translate to direct cost savings even for the server.
% % Specifically, the cost of compression + sending the compressed response is less than that of sending a large response.

% \subsection{Private Decision Tree Evaluation}
% In Private Decision Tree Evaluation (PDTE), the server infers a private decision tree over the client's encrypted attributes.
% This application (and other forms of classification) benefits significantly from compression because the response of a classification task is usually small, e.g., the class corresponding to the prediction.

% % This application is good because the input is a list of features that can be compressed with methods to compress uploaded ciphertexts.
% % Then the download can be compressed using our methods.
% % Two examples which our technique can be integrated into is:
% 1) SortingHats with transciphering~\cite{congSortingHatEfficientPrivate2022a}
% 2) XXCMP-PDTE 3) PDTE using Concrete and LWE ciphertexts~\cite{zamaConcreteTFHECompiler2022}.

% \paragraph{SortingHats~\cite{congSortingHatEfficientPrivate2022a}.}
% In SortingHats, the server returns the result of the classification as one LWE ciphertext~\cite[Algorithm 8]{congSortingHatEfficientPrivate2022a} which holds the class corresponding to the prediction.
% Moreover, the authors propose using transciphering to reduce the client upload. By combining transciphering and compression, the overall communication can be significantly reduced.

% \paragraph{XXCMP-PDTE~\cite{mahdaviLevelPrivateNonInteractive2023}}
% Mahdavi et al. propose a PDTE protocol based on a generalized version of the XCMP comparison operator, which they denote as XXCMP-PDTE.
% In XXCMP-PDTE, instead of the result being one scalar which indicates the class of the prediction, two scalars are produced for each leaf of the tree.
% Specifically, for each leaf, the sum of all comparisons from the root to that leaf is added up and placed in the constant term of an RLWE ciphertext.
% The second scalar for that leaf contains the masked value corresponding to that leaf.
% The authors include RLWE packing using a trivial method, i.e. scaling and adding, to compress all leaves within the same RLWE ciphertext.
% We can instead compress the constant terms of the polynomials using our proposed technique.

% Using our compression technique in this specific instance offers an additional computational advantage.
% One step by the server is extracting the constant coefficient and zeroing out all other coefficients, in preparation for RLWE packing.
% By compressing the constant terms into Paillier ciphertexts the coefficient extraction step can also be eliminated.
% As an example, in this work, the decision tree trained over the Breast, Heart, and Spam dataset with 16-bit precision has 17, 66, and 236 leaf nodes, respectively.
% We observe that $N=8192$ in these cases, based on their implementation, and the RLWE packing takes about 10 ms, for each value that is packed.
% The size of the response is equal to two RLWE ciphertexts or 86 KB.
% Using our technique, we reduce the response size to 1.5 KB, 5 KB, and 17 KB, respectively.

% % \begin{figure*}
% %     \centering
% %     \begin{minipage}{0.5\textwidth}
% %         \begin{algorithm}[H]
% %         	 \caption[]{XXCMP-PDTE (via RLWE Packing)}
% %         	 \label{alg:lwe-encrypt}
% %         	 \begin{algorithmic}[1]
% %                 \State Packed Cipher
% %                 \For{$\ell$}
% %             	\State $\textsc{ExtractConstant}(\ell)$
% %                     \State 
% %                 \EndFor
% %                 \Return Packed Cipher
% %         	 \end{algorithmic}
% %         \end{algorithm}
% %     \end{minipage}%
% %     \begin{minipage}{0.5\textwidth}
% %         \begin{algorithm}[H]
% %         	 \caption[]{XXCMP-PDTE + Compression}
% %         	 \label{alg:lwe-encrypt}
% %         	 \begin{algorithmic}[1]
% %         	 	\For{Leaf}
% %                         \State $x_i \leftarrow \textsc{CompressRLWECoefficient}(sk,C,0)$
% %                     \EndFor
% %                 \Return $\{x_i\}$
% %         	 \end{algorithmic}
% %         \end{algorithm}
% %     \end{minipage}
% % \end{figure*}


% % \begin{minipage}{0.45\textwidth}
% % \begin{algorithm*}[H]
% % 	 \caption[]{LWEEncrypt$_{\texttt{sk}}$}
% % 	 \label{alg:lwe-encrypt}
% % 	 \begin{flushleft}
% % 		 \textbf{Input:} $\mu\in\ZZ_p$
% % 	 \end{flushleft}
% % 	 \begin{algorithmic}[1]
% %         \vspace{-3mm}
% % 	 	\State Sample $\textbf{a}\xleftarrow{\$} \ZZ_q^{n}$
% %         \State Sample $e \leftarrow \chi$
% %         \State $b = \sum_{i=1}^{n} a[i] \cdot \texttt{sk}[i] + \Delta \cdot \mu + e \mod q$
% % 	 \end{algorithmic}
% %   \vspace{-2mm}
% % 	 \begin{flushleft}
% %     	 \textbf{Output:}  $c=(\textbf{a},b)$\\
% % 	 \end{flushleft}
% %   \vspace{-3mm}
% % \end{algorithm*}
% % \end{minipage}
% % ~~~
% % \begin{minipage}{0.45\textwidth}
% % \begin{algorithm*}[H]
% % 	 \caption[]{LWEDecrypt$_{\texttt{sk}}$}
% % 	 \label{alg:lwe-decrypt}
% % 	 \begin{flushleft}
% % 		 \textbf{Input:} $c\in \ZZ_q^{n}\times\ZZ_q$
% % 	 \end{flushleft}
% % 	 \begin{algorithmic}[1]
% %         \vspace{-3mm}
% %         \State $\mu^* = (b - \sum_{i=1}^{n} a[i] \cdot \texttt{sk}[i]) \mod q $
% %         \State $\mu'=\lfloor\mu^* / \Delta\rceil$
% % 	 \end{algorithmic}
% %   \vspace{3mm}
% % 	 \begin{flushleft}
% %     	 \textbf{Output:}  $\mu'$\\
% % 	 \end{flushleft}
% %   \vspace{-3mm}
% % \end{algorithm*}
% % \end{minipage}

% \subsection{Private Information Retrieval}
% In Private Information Retrieval (PIR), a client retrieves an element from a database such that the server holding the database does not learn anything about the client's query. State-of-the-art solutions to PIR use FHE and some can benefit from our compression technique.

% \paragraph{MulPIR~\cite{aliCommunicationComputationTradeoffs2021a}.}
% Ali et al. propose a PIR protocol, based on SealPIR~\cite{angelPIRCompressedQueries2018a} which takes advantage of homomorphic multiplications instead of only relying on additive encryption.
% The encrypted result is stored in a B/FV style ciphertext, in the coefficients of a polynomial.
% If the size of the payload is large and fills the entire plaintext space, compression is not necessary.
% However, in some applications such as contact tracing, contact discovery, and checking for leaked credentials, the response could be small, in some cases as small as one bit.
% In these cases, compression is very advantageous to reduce the response size.

% \paragraph{Constant-weight PIR~\cite{mahdaviConstantweightPIRSingleround2022}.}
% Mahdavi and Kerschbaum propose a keyword PIR protocol which has output similar to that of MulPIR.
% Hence, under the same assumptions (small response sizes), compression is effective in producing a small response size for Constant-weight PIR as well.

% % \paragraph{OnionPIR~\cite{mugheesOnionPIRResponseEfficient2021}}

% % \subsection{Circuit-PSI}

% % \paragraph{Encrypted Email Spam Filtering}
% % Encrypted email services such as Proton Mail are common.
% % However, offering spam filtering is not possible if the server does not have access to the emails.
% % Using HE, the server can classify emails as spam/not spam in a manner that only the client sees the result.
% % One approach might be to delegate the spam filtering algorithm to the client, but this would require computational effort on the client.
% % Moreover, any updates to the spam filtering algorithm would require updating all clients.
% % In a centralized setting, with the use of HE, the server could update the spam filtering algorithm on the fly without the need to update any clients.

% % Large ciphertext sizes are an obstacle in the practical deployment of many applications using FHE. Here we identify two specific use cases of FHE where our compression techniques can be applied.

% % \subsection{Filters over Encrypted Images.}

% % \begin{itemize}
% %     \item Using the seeded randomness described in the background, we can reduce the upload a lot
% %     \item So in applications where the download is a bottleneck in communication, our approach could help a lot
% %     \item In this section, we will outline some applications which fall have this feature.
% % \end{itemize}

% % The compression of LWE ciphertexts is suitable for applications where the output is large, such as applying a filter on an encrypted image. For example, Signal offers a tool to blur faces for the privacy and safety of individuals in images~\cite{signal-blur}. Currently, this feature runs locally on the client's device. However, to reduce overhead on small devices, the client can send the encrypted image to the server to process and receive the response. This situation is a suitable fit for FHE since the client device can establish cryptographic keys with the server and reuse them many times.

% % Applying filters over encrypted images has also been implemented using the Concrete library~\cite{concrete-image}. They encode the image as an array of LWE ciphertexts, which makes it ideal for compression using our proposed algorithm.

% % \subsection{Private Data Analysis.}
% % Circuit-PSI and Private Decision Tree Evaluation are two examples of data analysis over private data.

% % In Circuit-PSI, the server computes a function of the intersection of the client encrypted set with its own private set. In the protocol proposed by Kacsmar et al.~\cite{dipsi}, the client's set is encrypted as an RLWE ciphertext. The response, however, is small and only occupies one coefficient slot. This is an ideal application of the compression technique over RLWE ciphertexts.

% % In Private Decision Tree Evaluation, the server evaluates a decision tree over the client's private encrypted input. Cong et al. \cite{sortinghats} propose a protocol which encrypts the client's input as multiple LWE ciphertexts and returns the result of the classification as an LWE ciphertext. Compression can be applied to this application as well.

% % \subsection{Private Information Retrieval}
% % In Private Information Retrieval (PIR), a client queries an element from a database in a manner such that the query is not revealed to the server holding the database.
% % The most efficient solutions for PIR use homomorphic encryption and suffer from large response sizes, particularly when retrieving small items.

% % One PIR protocol is OnionPIR~\cite{mugheesOnionPIRResponseEfficient2021}.
% % As shown by their experiments~\cite[Table 3]{onionpir}, two thirds of the communication consists of a large response, 128 KB out of 192 KB.
% % Each database entry can be up to 30 KB in OnionPIR, which is encoded in the coefficients of an RLWE plaintext.
% % If the database occupies fewer coefficients of the plaintext, our method can be used to get a more compressed response.

% % % \begin{table}[]
% % %     \centering
% % %     \begin{tabular}{c|c|c|c|c}
% % %     \toprule
% % %          & Spiral & Us & Compressed\\
% % %     \midrule
% % %         $2^{14} \times 100 KB$
% % %         & Offline Comm. & 14-18 MB & 315 KB & \\
% % %         & Request & 14 KB & 384 KB \\
% % %         & Response & 1.8 MB \\
% % %     \bottomrule
% % %     \end{tabular}
% % %     \caption{Caption}
% % %     \label{tab:my_label}
% % % \end{table}


% % \subsection{Small Output Applications}
% % \begin{itemize}
% %     \item LWE compression is best when the output is small, e.g. one ciphertext.
% %     \item One good example is sentiment analysis over text, or spam filtering of emails (pgp maybe)
% %     \item Encrypted DNS is another example
% %     \item Helping with spam is hard when you can't see the data \footnote{\url{https://proton.me/blog/encrypted-email-spam-filtering}}
% %     \item Compression reduces the network activity for these applications, but it will also reduce costs
% %     \item Here are some examples of reducing costs that we have estimated
% %     \item Costs are calculated assuming the server is running on a AWS server [specs] and the prices are based on pricing
% % \end{itemize}

% % \paragraph{Sentiment Analysis}
% % The request for sentiment analysis is the encryption of quantized transformer representation. The output is one number, indicating the statement is neutral, positive or negative. The output is one LWE ciphertext (verify).

% % \begin{table}[]
% %     \centering
% %     \begin{tabular}{c|c|c}
% %     \toprule
% %         Params & $n=??, \log_q = 64$ & \\
% %     \midrule
% %         Request Size (bits) & 128 (seed) + $\ell$ (64 (body)) bits & $128 + 64 \ell$ \\
% %         Cost to send uncompressed response & (it should be 5 KB) (?? \$) &  \\
% %         Cost to compress & a few ms (? \$) \\
% %         Cost to send compressed response & 0.5 KB (?? \$) \\
% %         Total cost of compressed response method & Add \$ \\
% %     \bottomrule
% %     \end{tabular}
% %     \caption{Costs associated with the different methods to send the response for Sentiment Analysis. $\ell$ denotes the number of characters in the message that is being evaluated.}
% %     \label{tab:sentiment-analysis-costs}
% % \end{table}

% % \paragraph{Health Prediction}

% % \paragraph{Private Decision Tree Evaluation}

% % The input in private decision tree evaluation is some features that have been represented as LWE ciphertexts.
% % If we use some specific methods, the output is only one ciphertext which indicates the prediction (be more specific)
% % For example, we can combine it with the PDTE algorithm called SortingHats~\cite{sortinghats} which also implements transciphering to reduce the input size. Overall, the comm. will be very low. 

% % \begin{table}[]
% %     \centering
% %     \begin{tabular}{c|c}
% %     \toprule
% %         Request Size & 128 (seed) + \# chars * (64 (body))\\
% %         Cost to send uncompressed response & 500 B??? (it should be 5 KB) (?? \$) \\
% %         Cost to compress & a few ms (? \$) \\
% %         Cost to send compressed response & 0.5 KB (?? \$) \\
% %         Total cost of compressed response method & Add \$ \\
% %     \bottomrule
% %     \end{tabular}
% %     \caption{Costs for PDTE using SortingHats, with and without compression.}
% %     \label{tab:pdte-costs}
% % \end{table}

