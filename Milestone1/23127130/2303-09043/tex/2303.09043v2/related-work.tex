\section{Related Work}
\label{sec:related-work}

\begin{table}[t]
    \centering
    \caption{
        Taxonomy of different techniques which involve conversion between encryption schemes.
        Techniques which could offer compression for the downloaded ciphertexts are indicated in bold.
    } 
    \resizebox{\columnwidth}{!}{%
    \begin{tabular}{cc|c}
    \toprule
    \textbf{Source} & \textbf{Dest.} & \textbf{Technique} \\
    \midrule
    \midrule
    \multirow{2}{*}{\begin{tabular}{c}Symmetric\\Ciphers\end{tabular}}
    & LWE & Hybrid HE~\cite{dobraunigPastaCaseHybrid2021}\\\cmidrule{2-3}
    & RLWE & Hybrid HE~\cite{canteautStreamCiphersPractical2018, albrechtCiphersMPCFHE2015, dobraunigRastaCipherLow2018, choTranscipheringFrameworkApproximate2021a, dobraunigPastaCaseHybrid2021} \\
    \midrule
        LTV & LWE & \textbf{Compression~\cite{huImprovingEfficiencyHomomorphic2013}}\\
    \midrule
        \multirow{4}{*}{LWE} & LWE &
        \begin{tabular}{c}
            Keyswitching~\cite{chillottiTFHEFastFully2020, ducasFHEWBootstrappingHomomorphic2015} \\
            \textbf{Dim. Reduction~\cite{brakerskiEfficientFullyHomomorphic2011,naehrigCanHomomorphicEncryption2011a}}
        \end{tabular}\\\cmidrule{2-3}
         & RLWE & Scheme Switching~\cite{bouraCHIMERACombiningRingLWEbased2020} \\\cmidrule{2-3}
         & Paillier & \textbf{Our work} \\
    \midrule
        \multirow{2}{*}{\begin{tabular}{c}Multiple\\LWEs\end{tabular}}
        & RLWE & \textbf{RLWE Packing~\cite{chenEfficientHomomorphicConversion2021}} \\\cmidrule{2-3}
        & Damgard-Jurik & \textbf{Our work} \\
    \midrule
        \multirow{5}{*}{RLWE} & LWE & Coefficient Extraction~\cite{chillottiTFHEFastFully2020,bouraCHIMERACombiningRingLWEbased2020} \\\cmidrule{2-3}
        & RLWE$^*$ & Oblivious Expand~\cite{angelPIRCompressedQueries2018a,chenOnionRingORAM2019} \\\cmidrule{2-3}
        & RLWE & \textbf{Modulus Switching~\cite{brakerskiLeveledFullyHomomorphic2012,cheonHomomorphicEncryptionArithmetic2017a}} \\\cmidrule{2-3}
        & Paillier & \textbf{Our work} \\
    \midrule
        \multirow{2}{*}{GSW} & ElGamal & \cite{gentryFullyHomomorphicEncryption2011} \\\cmidrule{2-3}
        & PVW & \textbf{\cite{gentryCompressibleFHEApplications2019}}\\
    \bottomrule
    \end{tabular}%
    }
    \label{tab:sok}
\end{table}


Our technique is predicated on converting the scheme under which the plaintexts are encoded.
This concept has been proposed in the literature before but for different purposes.
In this section, we examine the existing literature which involves scheme switching in any form and outline how our work differs from them.
We specifically point out scheme-switching techniques that result in any compression, but we emphasize that our work achieves better compression in a plug-and-play fashion, which has not been the case in previous work. \Cref{tab:sok} summarizes the related work.

\subsection{Identity Scheme Switching}
Identity scheme switching implies that the destination ciphertext encrypts the exact same message as the original plaintext.
There are many techniques in the literature for conversions, either between schemes or within a scheme, with different purposes besides compression.
Some might also come with the benefit of compression, but the compression rate is not high.
The common feature of these works is that the destination ciphertext accurately encrypts the same message as the initial ciphertext.
As we will see, in other forms of scheme scheme switching, this may not hold.

\subsubsection{Parameter Switching}
There are many ways that the parameters of the encryption system may change. Here we discuss three of such methods:
Key switching, Dimension reduction, and Modulus Switching. We also describe how each method is relevant to the compression.

Key switching is a very common technique and as the name suggests, allows the secret key under which the ciphertext is encrypted, to change.
However, \cite{brakerskiEfficientFullyHomomorphic2011} observed that, for LWE-based schemes in particular, the ciphertext dimension can also change whilst changing the key.
They referred to this as dimension reduction and used it to reduce the size of the decryption circuit for better bootstrapping.
Dimension reduction can be used for compression, by simply switching to smaller parameters, i.e., smaller $n$ and $q$ as per the notation of \Cref{sec:lwe}.
Ciphertexts encrypted with the new parameters are smaller but the compression achieved by this approach is limited given that the ciphertext is still a vector in $\ZZ_q^{n+1}$, albeit with smaller parameters.

Another relevant technique is modulus switching which is primarily used in RLWE-based schemes such as BGV~\cite{brakerskiLeveledFullyHomomorphic2012}, B/FV~\cite{brakerskiFullyHomomorphicEncryption2012,fan2012somewhat}, and CKKS~\cite{cheonHomomorphicEncryptionArithmetic2017a}.
Generally, there are two purposes to modulus switching in these schemes:
First, to limit noise growth with homomorphic operations.
This technique was first proposed by Brakerski and Vaikuntanathan~\cite{brakerskiEfficientFullyHomomorphic2011} and subsequently used by Brakerski et al. to construct the BGV cryptosystem~\cite{brakerskiLeveledFullyHomomorphic2012}.
This technique is also used in the CKKS scheme to switch between levels~\cite{cheonHomomorphicEncryptionArithmetic2017a}.

The second benefit of modulus switching is size reduction before communicating the result to the client. Using our notation from \Cref{sec:lwe}, this technique results in a smaller $q$.
However, this technique is limited by the fact that the size of the ciphertext is still linear in $N$, which significantly impacts the size of the ciphertext.

Note that the ciphertext compression technique mentioned in this work can be used in conjunction with modulus switching. After the modulus has been switched to the smallest value, our compression to an additive scheme is performed. Switching the smaller parameters via modulus switching improves the compression that we can gain from our technique because more ciphertexts can fit within one additive ciphertext.

\subsubsection{Bootstrapping \& Extended Functionality}

% Gentry Squashing
Gentry and Halevi~\cite{gentryFullyHomomorphicEncryption2011} use scheme switching as an alternative to squashing the decryption circuit in the bootstrapping process.
They start with ciphertexts encrypted under a Somewhat Homomorphic Encryption (SWHE) scheme.
They express the decryption function of that scheme as a depth-3 circuit of a particular form.
For one step of the decryption, which involves multiplications, they switch to the Elgamal scheme~\cite{elgamalPublicKeyCryptosystem1985} which is a multiplicative encryption scheme.
They perform the multiplications in Elgamal and then switch back to the SWHE scheme by evaluating the decryption circuit of Elgamal.
This approach to bootstrapping avoids the squashing step proposed by Gentry in the original blueprint~\cite{gentryFullyHomomorphicEncryption2009} and does not require the additional assumption that the sparse subset problem is hard.

% Chimera
Boura et al.~\cite{bouraCHIMERACombiningRingLWEbased2020} propose scheme switching as a method to benefit from the features of many cryptosystems.
The authors provide procedures to switch between three Ring-LWE based schemes, B/FV~\cite{fan2012somewhat,brakerskiFullyHomomorphicEncryption2012}, TFHE~\cite{chillottiTFHEFastFully2020}, and CKKS~\cite{cheonHomomorphicEncryptionArithmetic2017a}.

% Sample Extract/Coeff Extract.
\paragraph{Coefficient Extraction.}
Coefficient Extraction~\cite{chillottiTFHEFastFully2020,bouraCHIMERACombiningRingLWEbased2020} is a specific instance of scheme switching which we elaborate on due to the relevance to compression. Coefficient extraction generates a LWE-based ciphertext which encrypts one coefficient of an RLWE plaintext.
Given that RLWE ciphertexts are usually larger than LWE ciphertexts, conversion from RLWE to LWE can offer compression when only one RLWE coefficient is of interest.s

% Ciphertext packing
\paragraph{RLWE Packing.}
The reverse process of coefficient extraction is RLWE packing, which encodes many LWE plaintexts into the coefficients of an RLWE plaintext.
Due to the smaller expansion factor of RLWE ciphertexts, this technique is suitable for compression if enough coefficients in the RLWE plaintext are utilized.
Chen et al.~\cite{chenEfficientHomomorphicConversion2021} demonstrated an efficient method to perform this conversion.
% If the original LWE ciphertext is in $\ZZ_q^n$, then the RLWE ciphertext is in $R_q$.
% While this approach is feasible, it has a few restrictions
% 1) It is computationally slow
% 2) It additionally requires the key switching key from LWE to RLWE
% 3) is consumes a lot of noise budget so you either need larger parameters or you need to resort to bootstrapping


% \paragraph{RLWE Coefficient Extraction.}
% TFHE~\cite{chillottiTFHEFastFully2020a} supports coefficient extraction or sample extraction~\cite{bouraCHIMERACombiningRingLWEbased2020} over RLWE ciphertexts.
% This allows the server to extract one coefficient of the underlying plaintext of an RLWE ciphertext as an LWE ciphertext.
% This technique is useful for the purpose of bootstrapping. It has not been proposed as a method for size reduction.

% Coefficient extraction can be used in combination with our techniques.
% For example, the desired coefficient in an RLWE ciphertext can be extracted into an LWE ciphertext, and then LWE compression can be used.
% However, using compression over RLWE ciphertexts, we can directly compress the RLWE coefficient without the need to initially perform coefficient extraction.
% Converting from RLWE to LWE requires an additional large keyswitching key. 

% The main distinction between scheme switching and the approach of this paper is that scheme switching produces a ciphertext encrypting the same value within a different scheme, whereas compression does not require the new ciphertext to encrypt the same value. It is sufficient for the new ciphertext to decode the same message, without encrypting the exact same value.

% \paragraph{Oblivious Expansion.}
% Angel et al.~\cite{angel_pir_2018} proposed a method, which was extended by Chen et al.~\cite{chen_onion_2019}, to extract the coefficients of an RLWE ciphertext into separate ciphertexts.
% The result is still a valid RLWE-based ciphertext.
% Oblivious expansion has been used to compress PIR queries in many works~\cite{ali_communicationcomputation_2021,angel_pir_2018,mughees_onionpir_2021}.

% \paragraph{RLWE Packing.}

% This is a special case of scheme switching which we separate due to its relevance.
% Another possibility is to pack the RLWE ciphertext with LWE instances and send that back.
% Chen et al.~\cite{chenEfficientHomomorphicConversion2021} demonstrated an efficient method to perform this conversion.
% If the original LWE ciphertext is in $\ZZ_q^n$, then the RLWE ciphertext is in $R_q$
% While this may be feasible, it has a few restrictions
% 1) It is computationally slow
% 2) It additionally requires the key switching key from LWE to RLWE
% 2) is consumes a lot of noise budget so you either need larger parameters or you need to resort to bootstrapping

\begin{table*}
    \centering
    \caption{Evaluation of the ciphertext compression technique for a single LWE ciphertext. Three sample parameter sets are chosen for LWE-based ciphertexts.
    The first three columns are common parameter sets used in the Concrete library~\cite{zamaConcreteTFHECompiler2022}.
    The last configuration is the STD128 configuration for CGGI in OpenFHE~\cite{albadawiOpenFHEOpenSourceFully2022}.} \resizebox{\textwidth}{!}{
    \begin{tabular}{c|c|c|c|c|c|c|c|c}
    \toprule
    \multirow{2}{*}{Parameters}
    & \multicolumn{4}{c|}{LWE $(n,\log_2 q)$} 
    & \multicolumn{4}{c}{RLWE $(N,\log_2 q)$} 
    \\
        & (630, 64)
        & (742, 64)
        & (870, 64)
        & (1305, 11)
        & (1024,27)
        & (2048,54)
        & (4096,36)
        & (8192,43)
         \\
    \midrule                     
        Compression Time   & 9.7 ms & 11.0 ms & 12.9 ms & 16.6 ms & 7.2 ms & 23.8 ms & 33.8 ms & 83.3 ms \\
        Compressed Ciphertext    & 768 B   & 768 B  & 768 B & 768 B   & 768 B  & 768 B & 768 B & 768 B \\
        Uncompressed Ciphertext  & 5.05 KB & 5.94 KB & 6.97 KB & 1.80 KB & 3.46 KB & 13.83 KB & 18.44 KB & 44.04 KB \\
        Size Reduction            & \textbf{84.78 \%} & \textbf{87.08} \% & \textbf{88.98\%} & \textbf{57.23\%} & \textbf{77.80\%} & \textbf{94.45\%} & \textbf{95.83\%} & \textbf{98.26\%} \\
    \bottomrule
    \end{tabular}
    }
    \label{tab:evaluation-lwe-compress}
\end{table*}



\subsubsection{Transciphering/Hybrid HE}
The concept of Hybrid Homomorphic Encryption (HHE) was first introduced by Naehrig et al.~\cite{naehrigCanHomomorphicEncryption2011a}. 
The client encrypts their input using a symmetric encryption scheme, which has an expansion factor of one.
The server, having access to the symmetrically encrypted ciphertext and homomorphically encrypted secret key, can perform a homomorphic decryption of the symmetric ciphertext to get a homomorphic ciphertext of the intended message.
The communication burden of sending a large ciphertext is substituted with a computational effort by the server to perform the conversion.
Recent works have attempted to reduce the computational burden on the server by proposing alternative symmetric encryption schemes that are more \textit{HE-friendly}~\cite{dobraunigPastaCaseHybrid2021, albrechtCiphersMPCFHE2015,dobraunigRastaCipherLow2018,dobraunigPastaCaseHybrid2021, canteautStreamCiphersPractical2018, meauxStreamCiphersEfficient2016}

To summarize, this technique involves conversion from a symmetric encryption scheme to a homomorphic encryption scheme.
While this technique is extremely effective in reducing the upload cost, it can not be used in the opposite direction, from homomorphic ciphertexts to symmetric ciphertexts, to reduce the download cost.
The destination ciphertext must be homomorphic so that the decryption function of the source ciphertext can be computed. 


\subsection{Posthoc, Approximate Conversions}
Imprecise conversions are helpful in cases when the result does not need to be operated anymore.
Compression is a prime example.
In posthoc conversion, the destination ciphertext doesn't have to encrypt the same message, as long as the original message can be retrieved, given the new message.
In our compression, we skip the second step of decryption in the source encryption, i.e., the rounding step, which is deferred to the client and only the linear step is performed. 
This way, the compression is done with small computational effort, while also enabling the client to retrieve the correct message.

\subsubsection{Compressing LTV Ciphertexts}
Hu~\cite{huImprovingEfficiencyHomomorphic2013} introduced the concept of \textit{secure converters} for converting between cryptographic schemes.
This is achieved by homomorphically evaluating (part of) the decryption circuit of the source scheme under the destination scheme. 
Within that framework, the author proposed homomorphically converting from LTV ciphertexts to Paillier ciphertexts to reduce bandwidth usage from the server to the client.
The conversion, however, is not precise and the Paillier ciphertexts encrypt a noisy version of the initial plaintexts.
Using this approach, a 256x compression rate is achieved whilst communicating ciphertexts back to the client.
However, the LTV cryptographic scheme is not adopted as a practical homomorphic encryption scheme.

\subsubsection{High-rate Compression}
Brakerski et al.~\cite{brakerskiLeveragingLinearDecryption2019} showed how a high-rate compression, arbitrarily close to one, can be achieved over ciphertexts with the \textit{linear-decrypt-and-multiply} characteristic.
Cryptosystems with linear-decrypt-and-multiply can decrypt to any multiple of the message.
Based on the authors, among prevalent encryption schemes, only GSW falls into that category.
Assuming the goal is to encrypt $\{m_0,m_1,\cdots,m_{\ell-1}\}$, then the compression is done by homomorphically decrypting these messages to $\{m_0+e_0,\Delta m_1+e_1,\cdots,\Delta^{\ell-1}+e_{\ell-1} m_{\ell-1}\}$, where $e_i$'s are noise introduced from the homomorphic cryptosystem, similar to LWE.
By adding these messages together, the server obtains one large plaintext, encrypted under an additive ciphertext which is sends to the client.

\subsubsection{GSW Compression}
Gentry et al.~\cite{gentryCompressibleFHEApplications2019} also proposed a method to compress many GSW ciphertexts into high-rate PVW ciphertexts.
The ratio between the plaintext and ciphertext can be arbitrarily close to zero in their construction.
However, this can only be achieved if the underlying aggregate plaintext is very large.
Specifically, for the ratio to be $1-\epsilon$, the aggregate plaintext must be proportional to $1/\epsilon^3$.
The authors described how to construct a PIR protocol from this technique, but their compression techniques is not applicable to any other type of ciphertext.

