%% bare_conf_compsoc.tex
%% V1.4b
%% 2015/08/26
%% by Michael Shell
%% See:
%% http://www.michaelshell.org/
%% for current contact information.
%%
%% This is a skeleton file demonstrating the use of IEEEtran.cls
%% (requires IEEEtran.cls version 1.8b or later) with an IEEE Computer
%% Society conference paper.
%%
%% Support sites:
%% http://www.michaelshell.org/tex/ieeetran/
%% http://www.ctan.org/pkg/ieeetran
%% and
%% http://www.ieee.org/

%%*************************************************************************
%% Legal Notice:
%% This code is offered as-is without any warranty either expressed or
%% implied; without even the implied warranty of MERCHANTABILITY or
%% FITNESS FOR A PARTICULAR PURPOSE! 
%% User assumes all risk.
%% In no event shall the IEEE or any contributor to this code be liable for
%% any damages or losses, including, but not limited to, incidental,
%% consequential, or any other damages, resulting from the use or misuse
%% of any information contained here.
%%
%% All comments are the opinions of their respective authors and are not
%% necessarily endorsed by the IEEE.
%%
%% This work is distributed under the LaTeX Project Public License (LPPL)
%% ( http://www.latex-project.org/ ) version 1.3, and may be freely used,
%% distributed and modified. A copy of the LPPL, version 1.3, is included
%% in the base LaTeX documentation of all distributions of LaTeX released
%% 2003/12/01 or later.
%% Retain all contribution notices and credits.
%% ** Modified files should be clearly indicated as such, including  **
%% ** renaming them and changing author support contact information. **
%%*************************************************************************


% *** Authors should verify (and, if needed, correct) their LaTeX system  ***
% *** with the testflow diagnostic prior to trusting their LaTeX platform ***
% *** with production work. The IEEE's font choices and paper sizes can   ***
% *** trigger bugs that do not appear when using other class files.       ***                          ***
% The testflow support page is at:
% http://www.michaelshell.org/tex/testflow/



\documentclass[conference,compsoc]{IEEEtran}
% Some/most Computer Society conferences require the compsoc mode option,
% but others may want the standard conference format.
%
% If IEEEtran.cls has not been installed into the LaTeX system files,
% manually specify the path to it like:
% \documentclass[conference,compsoc]{../sty/IEEEtran}





% Some very useful LaTeX packages include:
% (uncomment the ones you want to load)


% *** MISC UTILITY PACKAGES ***
%
%\usepackage{ifpdf}
% Heiko Oberdiek's ifpdf.sty is very useful if you need conditional
% compilation based on whether the output is pdf or dvi.
% usage:
% \ifpdf
%   % pdf code
% \else
%   % dvi code
% \fi
% The latest version of ifpdf.sty can be obtained from:
% http://www.ctan.org/pkg/ifpdf
% Also, note that IEEEtran.cls V1.7 and later provides a builtin
% \ifCLASSINFOpdf conditional that works the same way.
% When switching from latex to pdflatex and vice-versa, the compiler may
% have to be run twice to clear warning/error messages.






% *** CITATION PACKAGES ***
%
\ifCLASSOPTIONcompsoc
  % IEEE Computer Society needs nocompress option
  % requires cite.sty v4.0 or later (November 2003)
  \usepackage[nocompress]{cite}
\else
  % normal IEEE
  \usepackage{cite}
\fi
% cite.sty was written by Donald Arseneau
% V1.6 and later of IEEEtran pre-defines the format of the cite.sty package
% \cite{} output to follow that of the IEEE. Loading the cite package will
% result in citation numbers being automatically sorted and properly
% "compressed/ranged". e.g., [1], [9], [2], [7], [5], [6] without using
% cite.sty will become [1], [2], [5]--[7], [9] using cite.sty. cite.sty's
% \cite will automatically add leading space, if needed. Use cite.sty's
% noadjust option (cite.sty V3.8 and later) if you want to turn this off
% such as if a citation ever needs to be enclosed in parenthesis.
% cite.sty is already installed on most LaTeX systems. Be sure and use
% version 5.0 (2009-03-20) and later if using hyperref.sty.
% The latest version can be obtained at:
% http://www.ctan.org/pkg/cite
% The documentation is contained in the cite.sty file itself.
%
% Note that some packages require special options to format as the Computer
% Society requires. In particular, Computer Society  papers do not use
% compressed citation ranges as is done in typical IEEE papers
% (e.g., [1]-[4]). Instead, they list every citation separately in order
% (e.g., [1], [2], [3], [4]). To get the latter we need to load the cite
% package with the nocompress option which is supported by cite.sty v4.0
% and later.





% *** GRAPHICS RELATED PACKAGES ***
%
\ifCLASSINFOpdf
  % \usepackage[pdftex]{graphicx}
  % declare the path(s) where your graphic files are
  % \graphicspath{{../pdf/}{../jpeg/}}
  % and their extensions so you won't have to specify these with
  % every instance of \includegraphics
  % \DeclareGraphicsExtensions{.pdf,.jpeg,.png}
\else
  % or other class option (dvipsone, dvipdf, if not using dvips). graphicx
  % will default to the driver specified in the system graphics.cfg if no
  % driver is specified.
  % \usepackage[dvips]{graphicx}
  % declare the path(s) where your graphic files are
  % \graphicspath{{../eps/}}
  % and their extensions so you won't have to specify these with
  % every instance of \includegraphics
  % \DeclareGraphicsExtensions{.eps}
\fi
% graphicx was written by David Carlisle and Sebastian Rahtz. It is
% required if you want graphics, photos, etc. graphicx.sty is already
% installed on most LaTeX systems. The latest version and documentation
% can be obtained at: 
% http://www.ctan.org/pkg/graphicx
% Another good source of documentation is "Using Imported Graphics in
% LaTeX2e" by Keith Reckdahl which can be found at:
% http://www.ctan.org/pkg/epslatex
%
% latex, and pdflatex in dvi mode, support graphics in encapsulated
% postscript (.eps) format. pdflatex in pdf mode supports graphics
% in .pdf, .jpeg, .png and .mps (metapost) formats. Users should ensure
% that all non-photo figures use a vector format (.eps, .pdf, .mps) and
% not a bitmapped formats (.jpeg, .png). The IEEE frowns on bitmapped formats
% which can result in "jaggedy"/blurry rendering of lines and letters as
% well as large increases in file sizes.
%
% You can find documentation about the pdfTeX application at:
% http://www.tug.org/applications/pdftex





% *** MATH PACKAGES ***
%
\usepackage{amsmath}
% A popular package from the American Mathematical Society that provides
% many useful and powerful commands for dealing with mathematics.
%
% Note that the amsmath package sets \interdisplaylinepenalty to 10000
% thus preventing page breaks from occurring within multiline equations. Use:
%\interdisplaylinepenalty=2500
% after loading amsmath to restore such page breaks as IEEEtran.cls normally
% does. amsmath.sty is already installed on most LaTeX systems. The latest
% version and documentation can be obtained at:
% http://www.ctan.org/pkg/amsmath





% *** SPECIALIZED LIST PACKAGES ***
%
% \usepackage{algorithmic}
% algorithmic.sty was written by Peter Williams and Rogerio Brito.
% This package provides an algorithmic environment fo describing algorithms.
% You can use the algorithmic environment in-text or within a figure
% environment to provide for a floating algorithm. Do NOT use the algorithm
% floating environment provided by algorithm.sty (by the same authors) or
% algorithm2e.sty (by Christophe Fiorio) as the IEEE does not use dedicated
% algorithm float types and packages that provide these will not provide
% correct IEEE style captions. The latest version and documentation of
% algorithmic.sty can be obtained at:
% http://www.ctan.org/pkg/algorithms
% Also of interest may be the (relatively newer and more customizable)
% algorithmicx.sty package by Szasz Janos:
% http://www.ctan.org/pkg/algorithmicx




% *** ALIGNMENT PACKAGES ***
%
% \usepackage{array}
% Frank Mittelbach's and David Carlisle's array.sty patches and improves
% the standard LaTeX2e array and tabular environments to provide better
% appearance and additional user controls. As the default LaTeX2e table
% generation code is lacking to the point of almost being broken with
% respect to the quality of the end results, all users are strongly
% advised to use an enhanced (at the very least that provided by array.sty)
% set of table tools. array.sty is already installed on most systems. The
% latest version and documentation can be obtained at:
% http://www.ctan.org/pkg/array


% IEEEtran contains the IEEEeqnarray family of commands that can be used to
% generate multiline equations as well as matrices, tables, etc., of high
% quality.




% *** SUBFIGURE PACKAGES ***
%\ifCLASSOPTIONcompsoc
%  \usepackage[caption=false,font=footnotesize,labelfont=sf,textfont=sf]{subfig}
%\else
%  \usepackage[caption=false,font=footnotesize]{subfig}
%\fi
% subfig.sty, written by Steven Douglas Cochran, is the modern replacement
% for subfigure.sty, the latter of which is no longer maintained and is
% incompatible with some LaTeX packages including fixltx2e. However,
% subfig.sty requires and automatically loads Axel Sommerfeldt's caption.sty
% which will override IEEEtran.cls' handling of captions and this will result
% in non-IEEE style figure/table captions. To prevent this problem, be sure
% and invoke subfig.sty's "caption=false" package option (available since
% subfig.sty version 1.3, 2005/06/28) as this is will preserve IEEEtran.cls
% handling of captions.
% Note that the Computer Society format requires a sans serif font rather
% than the serif font used in traditional IEEE formatting and thus the need
% to invoke different subfig.sty package options depending on whether
% compsoc mode has been enabled.
%
% The latest version and documentation of subfig.sty can be obtained at:
% http://www.ctan.org/pkg/subfig




% *** FLOAT PACKAGES ***
%
\usepackage{fixltx2e}
% fixltx2e, the successor to the earlier fix2col.sty, was written by
% Frank Mittelbach and David Carlisle. This package corrects a few problems
% in the LaTeX2e kernel, the most notable of which is that in current
% LaTeX2e releases, the ordering of single and double column floats is not
% guaranteed to be preserved. Thus, an unpatched LaTeX2e can allow a
% single column figure to be placed prior to an earlier double column
% figure.
% Be aware that LaTeX2e kernels dated 2015 and later have fixltx2e.sty's
% corrections already built into the system in which case a warning will
% be issued if an attempt is made to load fixltx2e.sty as it is no longer
% needed.
% The latest version and documentation can be found at:
% http://www.ctan.org/pkg/fixltx2e


% \usepackage{stfloats}
% stfloats.sty was written by Sigitas Tolusis. This package gives LaTeX2e
% the ability to do double column floats at the bottom of the page as well
% as the top. (e.g., "\begin{figure*}[!b]" is not normally possible in
% LaTeX2e). It also provides a command:
%\fnbelowfloat
% to enable the placement of footnotes below bottom floats (the standard
% LaTeX2e kernel puts them above bottom floats). This is an invasive package
% which rewrites many portions of the LaTeX2e float routines. It may not work
% with other packages that modify the LaTeX2e float routines. The latest
% version and documentation can be obtained at:
% http://www.ctan.org/pkg/stfloats
% Do not use the stfloats baselinefloat ability as the IEEE does not allow
% \baselineskip to stretch. Authors submitting work to the IEEE should note
% that the IEEE rarely uses double column equations and that authors should try
% to avoid such use. Do not be tempted to use the cuted.sty or midfloat.sty
% packages (also by Sigitas Tolusis) as the IEEE does not format its papers in
% such ways.
% Do not attempt to use stfloats with fixltx2e as they are incompatible.
% Instead, use Morten Hogholm'a dblfloatfix which combines the features
% of both fixltx2e and stfloats:
%
% \usepackage{dblfloatfix}
% The latest version can be found at:
% http://www.ctan.org/pkg/dblfloatfix




% *** PDF, URL AND HYPERLINK PACKAGES ***
%
% \usepackage{url}
% url.sty was written by Donald Arseneau. It provides better support for
% handling and breaking URLs. url.sty is already installed on most LaTeX
% systems. The latest version and documentation can be obtained at:
% http://www.ctan.org/pkg/url
% Basically, \url{my_url_here}.




% *** Do not adjust lengths that control margins, column widths, etc. ***
% *** Do not use packages that alter fonts (such as pslatex).         ***
% There should be no need to do such things with IEEEtran.cls V1.6 and later.
% (Unless specifically asked to do so by the journal or conference you plan
% to submit to, of course. )


% correct bad hyphenation here
\hyphenation{op-tical net-works semi-conduc-tor}

% \usepackage{algorithm}
% \usepackage{algorithmicx}
% \usepackage{algorithm2e}/

% \usepackage[ruled,vlined,linesnumbered,lined,boxed,commentsnumbered]{algorithm2e}

\usepackage{algorithm}
\usepackage[noend]{algpseudocode}

% Redefine the line number command to make it smaller
\makeatletter
\renewcommand{\ALG@beginalgorithmic}{\scriptsize}
\renewcommand{\alglinenumber}[1]{\scriptsize #1:}
\makeatother


\usepackage{booktabs}
\usepackage{amsfonts, mathtools}
\usepackage{amsthm}
\usepackage{amsmath}
\usepackage[lambda ,advantage, operators, sets , adversary, landau, probability, notions, logic, ff , mm, primitives, events, complexity, asymptotics, keys]{cryptocode}
\usepackage{xurl}
\usepackage{multirow}
\usepackage{multicol}
\usepackage{graphicx}
\usepackage{rotating}
% \usepackage{pgfplots}
% \pgfplotsset{compat=1.17}
\usepackage{colortbl}% http://ctan.org/pkg/xcolor
\usepackage{makecell}
\usepackage{tikz}
\usepackage{caption}
\usepackage{subcaption}
\usepackage{diagbox}
\usepackage[first=0,last=9]{lcg}
\usepackage{stfloats}
\usepackage{diagbox}

\usepackage{pgfplots}
% \usetikzlibrary{pgfplots.fillbetween} % Load the fillbetween library
\pgfplotsset{compat=1.15}


\newcommand{\TODO}[1]{\textcolor{red}{TODO: #1}}
\newtheorem{theorem}{Theorem}
\newtheorem{definition}{Definition}
\newtheorem{proposition}{Proposition}
\newtheorem{corollary}{Corollary}
\newtheorem{lemma}{Lemma}
% \newtheorem*{remark}{Remark}

\newcommand{\round}[1]{\lfloor #1 \rceil}
\newcommand{\db}{\pckeystyle{db}}
\newcommand{\qu}{\pckeystyle{qu}}
\newcommand{\ck}{\pckeystyle{ck}}
\newcommand{\eck}{\pckeystyle{eck}}
\newcommand{\pck}{\pckeystyle{pck}}
\newcommand{\ct}{\pckeystyle{ct}}
\newcommand{\ans}{\pckeystyle{ans}}
\newcommand{\pirstate}{\pckeystyle{st}}
\newcommand{\hint}{\textbf{H}}
% \newcommand{\adv}{\mathcal{A}}

\newcommand{\protocol}{ZipPIR}
\newcommand{\protocolsingle}{ZipPIR$_{C}$}
\newcommand{\protocolbatched}{ZipPIR$_{B}$}
\newcommand{\addkey}{\sk_{A}}
\newcommand{\paillierkey}{\sk_{P}}
\newcommand{\paillierpk}{\pk_{P}}
\newcommand{\lwekey}{\sk}

\newcommand{\mask}{\textbf{A}}
\newcommand{\eudist}[1]{\lVert #1 \rVert}

\newcommand{\appendixsectionname}{Appendix}
\newcommand{\appsection}[1]{\appendixsectionname~\ref{#1}}


% \algblockdefx[procedure]{procedure}{EndProcedure}[2]{\textbf{procedure} \textsc{#1}(#2)}{\algpx@endIndent}
% \algtext*{EndProcedure}
% \let\oldReturn\Return
% \renewcommand{\Return}{\State\oldReturn}

\usepackage{stackengine}
\stackMath
\def\stackalignment{l}

% \newtheorem{theorem}{Theorem}
% \newtheorem{proof}{Proof}

%% MUST COME AFTER OTHER PACKAGES
\usepackage{cleveref}

\begin{document}
%
\title{HE is all you need: Smaller FHE Responses via Additive HE}
%
%\titlerunning{Abbreviated paper title}
% If the paper title is too long for the running head, you can set
% an abbreviated paper title here
%

\author{
  \IEEEauthorblockN{
    Rasoul Akhavan Mahdavi\IEEEauthorrefmark{1}, 
    Abdulrahman Diaa\IEEEauthorrefmark{1}, 
    Florian Kerschbaum\IEEEauthorrefmark{1}
  }
  \IEEEauthorblockA{\IEEEauthorrefmark{1}University of Waterloo}
}

% \author{
%     Anonymous Authors
% }

% \authorrunning{Rasoul Akhavan Mahdavi \and Abdulrahman Diaa \and Florian Kerschbaum}
% % First names are abbreviated in the running head.
% % If there are more than two authors, 'et al.' is used.
% %
% \institute{University of Waterloo, Waterloo, Ontario, Canada}
%

\maketitle
% \thispagestyle{plain}
% \pagestyle{plain}

\begin{abstract}
% Many privacy-preserving services are built using homomorphic encryption.
Homomorphic Encryption (HE) is a commonly used tool for building privacy-preserving applications.
However, in scenarios with many clients and high-latency networks, communication costs due to large ciphertext sizes are the bottleneck.
In this paper, we present a new compression technique that uses an additive homomorphic encryption scheme with small ciphertexts to compress large homomorphic ciphertexts based on Learning with Errors (LWE).
Our technique exploits the linear step in the decryption of such ciphertexts to delegate part of the decryption to the server.
We achieve compression ratios up to 90\% which only requires a small compression key.
By compressing multiple ciphertexts simultaneously, we can over 99\% compression rate.
Our compression technique can be readily applied to applications which transmit LWE ciphertexts from the server to the client as the response to a query.
Furthermore, we apply our technique to private information retrieval (PIR) where a client accesses a database without revealing its query.
Using our compression technique, we propose \protocol{}, a PIR protocol which achieves the lowest overall communication cost among all protocols in the literature.
\protocol{} does not require any communication with the client in the preprocessing phase, making it a great solution for use cases of PIR with ephemeral clients or high-latency networks.

\end{abstract}

% \IEEEpeerreviewmaketitle

\section{Introduction}
\label{sec:introduction}
% \begin{itemize}
%     % Diffusion of FL
%     \item {\st{Diffusion of FL}}
%     % Security threats to FL
%     \item {\st{Security threats to FL with particular focus on model poisoning}}
%     % Limitations of existing countermeasures
%     \item {\st{Current countermeasures (e.g., KRUM) and their limitations}}
%     % Proposed method and its advantages
%     \item {\st{Intuitive description of the proposed method and its difference (i.e., advantages) w.r.t. state of the art}}
%     % Main contributions
%     \item {\st{Summary of the main contributions of this work}}
%     % Paper's structure and organization
%     \item {\st{Paper's structure and organization}}
% \end{itemize}

% Diffusion of FL
Recently, {\em federated learning} (FL) has emerged as the leading paradigm for training distributed, large-scale, and privacy-preserving machine learning (ML) systems~\cite{mcmahan2017googleai,mcmahan2017aistats}. 
The core idea of FL is to allow multiple edge clients to collaboratively train a shared, global model without disclosing their local private training data.
%Specifically, an FL system consists of a central server and many edge clients; 
A typical FL round involves the following steps: {\em(i)} the server randomly picks some clients and sends them the current, global model; {\em(ii)} each selected client locally trains its model with its own private data; then, it sends the resulting local model to the server;\footnote{Whenever we refer to global/local model, we mean global/local model {\em parameters}.} {\em(iii)} the server updates the global model by computing an \emph{aggregation function}, usually the average (FedAvg), on the local models received from clients.
% \begin{enumerate}
%     \item[{\em(i)}] the server sends the current, global model to the clients and appoints some of them for training;
%     \item[{\em(ii)}] each selected client locally trains its copy of the global model with its own private data; then, it sends the resulting local model back to the server;\footnote{Whenever we refer to global/local model, we mean global/local model {\em parameters}.}
%     \item[{\em(iii)}] the server updates the global model by computing an \emph{aggregation function} on the local models received from clients (by default, the average, also referred to as FedAvg~\cite{mcmahan2017aistats}).
% \end{enumerate}
This process goes on until the global model converges. %(e.g., after a certain number of rounds or other similar stopping criteria).
%\\
% The advantages of FL over the traditional, centralized learning paradigm are undoubtedly clear in terms of flexibility/scalability (clients can join/disconnect from the FL network dynamically), network communications (only model weights\footnote{We will use \textit{parameters} and \textit{weights} interchangeably.} are exchanged between clients and server), and privacy (each client's private training data is kept local at the client's end and not uploaded to the server).
\\
% Security threats to FL
%However, the growing adoption of FL also raises security concerns~\cite{costa2022covert}, particularly about its confidentiality, integrity, and availability.
Although its advantages over standard ML, FL also raises security concerns~\cite{costa2022covert}. %, particularly about its confidentiality, integrity, and availability~\cite{costa2022covert}.
% OLD, LONG VERSION
% Indeed, some work deals with privacy leakage that may expose the local data of some clients~\cite{melis2019sp}. 
% A large body of work, instead, investigates attacks that usually aim to detriment the predictive accuracy of the learned global model. For instance, \emph{data poisoning} attacks achieve this goal by letting an adversary pollute the training set of some corrupt FL clients with maliciously crafted examples~\cite{jagielski2018sp}.
% Similarly, in \emph{model poisoning} the attacker attempts to tweak the global model weights~\cite{bhagoji2019pmlr} by directly perturbing the local model's weights of some infected FL clients before these are sent to the central server for aggregation, usually via so-called Byzantine attacks. 
% It turns out that Byzantine model poisoning attacks severely impact standard FedAvg; therefore, more robust aggregation functions must be designed to make FL systems secure.
Here, we focus on \emph{untargeted model poisoning} attacks~\cite{bhagoji2019pmlr}, where an adversary attempts to tweak the global model weights %\footnote{We will use the terms \textit{parameters} and \textit{weights} interchangeably.} 
by directly perturbing the local model's parameters of some infected clients before these are sent to the central server for aggregation.
In doing so, the adversary aims to jeopardize the global model \textit{indiscriminately} at inference time.
Such model poisoning attacks severely impact standard FedAvg; therefore, more robust aggregation functions must be designed to secure FL systems.
\\
% In this paper, we focus on designing a novel robust aggregation scheme at the server's end to contrast the effect of Byzantine model poisoning attacks.
%
% Current countermeasures and their limitations
%Several countermeasures have been proposed in the literature to combat model poisoning attacks on FL systems.
% Some methods use simple statistics more robust than plain average to smooth the impact of malicious updates (e.g., Trimmed Mean and FedMedian~\cite{yin2018icml}). 
% Other defenses implement outlier detection techniques to discard malicious updates from the aggregation performed at the server's end. Those are either based on heuristics (e.g., Krum/Multi-Krum~\cite{blanchard2017nips} and Bulyan~\cite{mhamdi2018pmlr}) or data-driven approaches (e.g., K-means clustering~\cite{shen2016acm} or DnC via spectral analysis~\cite{shejwalkar2021ndss}). 
% Finally, some strategies rely on a centralized ``source of trust'' to spot potential malicious updates (e.g., FLTrust~\cite{cao2020fltrust}).
% Several countermeasures have been proposed in the literature to combat model poisoning attacks on FL systems, i.e., to discard possible malicious local updates from the aggregation performed at the server's end. 
% These techniques range from simple statistics more robust than plain average (e.g., Trimmed Mean and FedMedian~\cite{yin2018icml}) to outlier detection heuristics (e.g., Krum/Multi-Krum~\cite{blanchard2017nips} and Bulyan~\cite{mhamdi2018pmlr}) or data-driven approaches (e.g., spectral analysis via K-means clustering~\cite{shen2016acm} or spectral analysis), or methods based on ``source of trust'' (e.g., FLTrust~\cite{cao2020fltrust}).
% OLD, LONG VERSION
%Several countermeasures have been proposed in the literature to combat Byzantine model poisoning attacks on FL systems.
% Descriptive statistics
% For example, Trimmed Mean and FedMedian aggregate local model updates using more robust statistics than standard average~\cite{yin2018icml}.
%
% % Heuristics for outlier detection
% Many existing Byzantine-resilient strategies implement some outlier detection heuristics to discard the model updates sent by potentially malicious clients from the input of the aggregation function.
% One of the most popular heuristics is Krum~\cite{blanchard2017nips}.
% This strategy tries to mitigate the impact of Byzantine attacks by selecting as a global model the local model with the smallest sum of Euclidean distances to {\em all} the other local models.
% Although powerful, Krum requires the server to know (or, at least, estimate) the number of malicious FL clients upfront, which is generally impossible in a realistic attack scenario. %
% Moreover, Krum may become ineffective for complex, high-dimensional model parameter spaces due to the curse of dimensionality.
% Bulyan~\cite{mhamdi2018pmlr} tries to overcome this issue by combining Krum with a variant of Trimmed Mean.
% % Data-driven outlier detection
% Other strategies use data-driven outlier detection techniques -- e.g., via K-means clustering~\cite{shen2016acm} -- to spot potential malicious local model updates. 
% %For instance, Shen et al. propose to cluster local model updates with K-means and thus identify outliers.
%
% % Other techniques
% As far as the server is concerned, any local model received can be from a potential malicious client. 
% FLTrust~\cite{cao2020fltrust} assumes the server acts as a client, i.e., trains a local model on an additional {\em trustworthy} dataset at the server's end and compares it against all the local models from other clients. 
% This way, the server can rely on some ``source of trust'' when discarding potentially malicious clients.
%\\
% Limitations of existing Byzantine-resilient strategies
Unfortunately, existing defense mechanisms either rely on simple heuristics (e.g., Trimmed Mean and FedMedian by~\cite{yin2018icml}) or need strong and unrealistic assumptions to work effectively (e.g., foreknowledge or estimation of the number of malicious clients in the FL system, as for Krum/Multi-Krum~\cite{blanchard2017nips} and Bulyan~\cite{mhamdi2018pmlr}, which, however, cannot exceed a fixed threshold).
Furthermore, outlier detection methods using K-means clustering~\cite{shen2016acm} or spectral analysis like DnC~\cite{shejwalkar2021ndss} do not directly consider the temporal evolution of local model updates received.
Finally, strategies like FLTrust~\cite{cao2020fltrust} require the server to collect its own dataset and act as a proper client, thereby altering the standard FL protocol.
\\
% OLD, LONG VERSION
% Overall, existing Byzantine-resilient strategies are either simple heuristics (e.g., FedMedian) or, if they are more complex, they rely on strong and unrealistic assumptions to work effectively (e.g., knowing the number of malicious clients in the FL system in advance, as for Krum and alike).
% Furthermore, data-driven outlier detection methods do not consider the temporary evolution of local model updates received (e.g., K-means clustering). 
% Finally, strategies like FLTrust requires the server to collect its own dataset and act as a proper client, thereby altering the standard FL protocol.
%
% Description of the proposed method
This work introduces a novel pre-aggregation \textit{filter} robust to untargeted model poisoning attacks. Notably, this filter $(i)$ operates without requiring prior knowledge or constraints on the number of malicious clients and $(ii)$ inherently integrates temporal dependencies. 
The FL server can employ this filter as a preprocessing step before applying \textit{any} aggregation function, be it standard like FedAvg or robust like Krum or Bulyan.
Specifically, we formulate the problem of identifying corrupted updates as a multidimensional (i.e., matrix-valued) time series anomaly detection task. 
The key idea is that legitimate local updates, resulting from well-calibrated iterative procedures like stochastic gradient descent (SGD) with an appropriate learning rate, show \textit{higher predictability} compared to malicious updates. This hypothesis stems from the fact that the sequence of gradients (thus, model parameters) observed during legitimate training exhibit regular patterns, as validated in Section~\ref{subsec:intuition}. %until convergence. 
%This regularity may be more pronounced for smooth convex loss functions, but it can still be captured within an appropriate time window, even for more complex and convoluted loss surfaces. 
%We provide evidence of this claim in Appendix~B, where we show that the average mutual information (i.e., ``predictability''), calculated over pairs of legitimate model updates sent at different FL rounds, is significantly higher than the corresponding computation for a malicious client.
\\
Inspired by the matrix autoregressive (MAR) framework for multidimensional time series forecasting~\cite{chen2021je}, we propose the FLANDERS ({\em \textbf{F}ederated \textbf{L}earning meets \textbf{AN}omaly \textbf{DE}tection for a \textbf{R}obust and \textbf{S}ecure}) filter.
The main advantages of FLANDERS over existing strategies like FLDetector~\cite{zhao2020multivariate} are its resilience to large-scale attacks, where $50\%$ or more FL participants are hostile, and the capability of working under realistic non-iid scenarios.
We attribute such a capability to two key factors: $(i)$ FLANDERS works without knowing a priori the ratio of corrupted clients, and $(ii)$ it embodies temporal dependencies between intra- and inter-client updates, quickly recognizing local model drifts caused by evil players. Below, we summarize our main contributions:

\begin{itemize}
\item[{\em(i)}]
We provide empirical evidence that the sequence of models sent by legitimate clients is more predictable than those of malicious participants performing untargeted model poisoning attacks.
\\
\item[{\em(ii)}] 
We introduce FLANDERS, the first pre-aggregation filter for FL robust to untargeted model poisoning based on multidimensional time series anomaly detection.
\\
\item[{\em(iii)}] 
We integrate FLANDERS into Flower,\footnote{\scriptsize{\url{https://flower.dev/}}} a popular FL simulation framework for reproducibility.
\\
\item[{\em(iv)}] 
We show that FLANDERS improves the robustness of the existing aggregation methods under multiple settings: different datasets, client's data distribution (non-iid), models, and attack scenarios.
\\
\item[{\em(v)}] 
We publicly release all the implementation code of FLANDERS along with our experiments.\footnote{\scriptsize{\url{https://anonymous.4open.science/r/flanders_exp-7EEB}}}
\end{itemize}

% Paper's structure and organization
The remainder of the paper is structured as follows. %some related work and the current state-of-the-art solutions to security issues that FL entails. 
Section~\ref{sec:background} covers background and preliminaries. 
In Section~\ref{sec:related}, we discuss related work.
Section~\ref{sec:problem} and Section~\ref{sec:method} describe the problem formulation and the method proposed. % to tackle it. 
Section~\ref{sec:experiments} gathers experimental results. %, and Section~\ref{sec:limitations} discusses some limitations of this work.
Finally, we conclude in Section~\ref{sec:conclusion}.
 %discusses the limitations of this work and draws future research directions.
%reports conclusions and draws perspectives for future research directions.

%%%%%%% OLD %%%%%%%
%to overcome the resilience of Byzantine failures in distributed Stochastic Gradient Descent computations. 
% The strength of Krum is its time complexity, which is linear in the gradient dimension. 
% However, the robustness of the approach is guaranteed for gradient-based learning applications only when the majority of the clients are not compromised. 
% Besides, the aggregation mechanism of Krum, as well as that of similar methods, is robust from a coarse-grained perspective and does not provide solutions to errors and perturbations that may occur at inference time.
%A related approach to~\cite{blanchard2017nips} is the work of Su et al.~\cite{su2016dc}. Here, the authors propose an iterated approximate agreement to tackle a multi-layer scenario attacked by Byzantine agents. 
%However, the method works efficiently on the sole discrete context and it is inapplicable to continuous state environments.
%\gabri{Maybe, we should just talk about the main limitations of existing countermeasures without digging into their details (or, we can just mention Krum as this is the most popular one). I will move the description of all these methods to the Related Work section.}
\section{Background on Network Calculus}
\label{sec: background}


\begin{figure*}[tbh]
\centering
\begin{subfigure}[b]{0.3\textwidth}
    \centering
    \includegraphics[width=\linewidth]{images/in-out.png}
    \caption{Arrival and departure data and their relation with delay $d(t)$ and backlog $b(t)$. For a FIFO system, the delay is the horizontal distance between $R(t)$ and $R^*(t)$ but some other multiplexing techniques may shift the data to a later priority, causing a longer delay.}
    \label{fig: data in-out}
\end{subfigure}
\hfill
\begin{subfigure}[b]{0.35\textwidth}
    \centering
    \includegraphics[width=\linewidth]{images/arrival-service.png}
    \caption{Characteristics of an arrival curve and a service curve. From any point of observation, the arriving data never exceeds its arrival curve; the departure data is also never less than the service curve with respect to the data arrival.}
    \label{fig: arrival-service curves}
\end{subfigure}
\hfill
\begin{subfigure}[b]{0.33\textwidth}
    \centering
    \includegraphics[width=\linewidth]{images/bound.png}
    \caption{Delay and backlog bounds of a system. Backlog is the maximum vertical distance between $\alpha(t)$ and $\beta(t)$; FIFO delay is their maximum horizontal distance; but for arbitrary multiplexing, the delay guarantee is when the system clears its buffer, thus it's the intersection of $\alpha(t)$ and $\beta(t)$.}
    \label{fig: system bounds}
\end{subfigure}
\caption{Network calculus framework. We let $R(t)$ and $R^*(t)$ be the arrival and departure data flow of a system; $\alpha(t)$ be the piecewise linear concave arrival curve and $\beta(t)$ be the piecewise linear convex service curve of a system.}
% \hossein{Better to show piece-wise linear concave arrival curve and piece-wise linear convex service curve instead of token-bucket and rate-latency.}}
\end{figure*}

We recall some of the network calculus essentials for a better understanding of the framework used in Saihu. In the following context, we use the following notation: $\mbb{R}^+$ is the set of non-negative real numbers; $[x]_+$ denotes $\max(0, x)$

The data flow is by convention modeled as a left-continuous wide-sense increasing function $R(t): \mbb{R}^+ \mapsto \mbb{R}^+$ with respect to time $t$~\cite{ncbook2001leboudec}. 

A system $\mcal{S}$ receives arrival data described as a cumulative function $R(t)$ and delivers departure data as another cumulative function $R^*(t)$. Figure~\ref{fig: data in-out} illustrates such a system $\mcal{S}$. The benefit of representing a system like this is that we can observe system backlog and delay with such a model. 

\begin{definition}[Backlog and Delay~\cite{ncbook2001leboudec}]
    The backlog of a system at time~$t$ is
    \begin{equation}
        b(t) = R(t) - R^*(t)
    \end{equation}
    
    The virtual delay of a FIFO system at time $t$ is
    \begin{equation}
        d_{FIFO}(t) = \inf \lbp \tau \geq 0 : R(t) \leq R^*(t+\tau) \rbp
    \end{equation}
\end{definition}



The backlog of a system can be viewed as the vertical distance between $R$ and $R^*$. The FIFO (\textit{First-in First-out}) delay is the horizontal distance between $R$ and $R^*$. One may obtain other delay values if the multiplexing technique is not FIFO.

% \begin{figure}
%     \centering
%     \includegraphics[width=0.9\linewidth]{images/in-out.png}
%     \caption{In/out data flow; delay and backlog}
%     \label{fig: data in-out}
% \end{figure}

Since we are interested in the system guarantee instead of a single instance of data flow, we would like to have general bounds to the arrival and departure data flows. Therefore, we define \textit{arrival curve} and \textit{service curve} as the bounds of arrival and departure data flows.

\begin{definition}[Arrival Curve~\cite{ncbook2001leboudec}]
    Given a wide-sense increasing function $\alpha: \mbb{R}^+ \mapsto \mbb{R}^+$, we say that a flow $R(t)$ is $\alpha$-constrained if and only if for all $s \leq t$:
    \begin{equation}
        R(t) - R(s) \leq \alpha(t-s)
    \end{equation}
    We say $R(t)$ has $\alpha$ as an arrival curve.
\end{definition}

\begin{definition}[Service Curve~\cite{ncbook2001leboudec}]
    Given a wide-sense increasing function $\beta: \mbb{R}^+ \mapsto \mbb{R}^+$ and $\beta(0) = 0$. A system $\mcal{S}$ having $R(t)$ and $R^*(t)$ as its arrival and departure flows. We say $\mcal{S}$ offers a service curve $\beta$ if and only if
    \begin{equation}
        R^*(t) \geq (R \otimes \beta)(t) =: \inf_{s \leq t} \lbp R(s) + \beta(t-s) \rbp
    \end{equation}
    where $\otimes$ denotes the min-plus convolution
\end{definition}

Figure~\ref{fig: arrival-service curves} illustrates the arrival and service curves. Any segment of arrival flow $R(t)$ is constrained by arrival curve $\alpha$ and the output curve $R^*(t)$ is always no less than the curve $R\otimes\beta$. As a result, an arrival curve upper bounds the incoming traffic, and a service curve lower bounds the outgoing traffic.

% \begin{figure}
%     \centering
%     \includegraphics[width=\linewidth]{images/arrival-service.png}
%     \caption{Arrival/Service curve}
%     \label{fig: arrival-service curves}
% \end{figure}

We consider 2 special types of curves throughout this paper, \textit{token-bucket} (or sometimes called \textit{leaky-bucket}) curve and \textit{rate-Latency} curve.

\begin{definition}[Token-bucket and Rate-latency~\cite{ncbook2001leboudec}]
    A token-bucket curve $\gamma_{r,b}$ with arrival rate $r$ and burst $b$ is defined as
    \begin{equation}
        \gamma_{r,b}(t) = b + rt
    \end{equation}

    A rate-latency curve $\beta_{R,T}$ with service rate $R$ and latency $T$ is defined as
    \begin{equation}
        \beta_{R,T}(t) = R \lb t - T \rb_+
    \end{equation}
\end{definition}

A token-bucket curve is determined by a burst $b$ and an arrival rate~$r$. Burst represents the maximum possible data volume that can arrive simultaneously, and arrival rate represents the maximum long-term data rate~\cite{bouillard2022tradeoff}.
A rate-latency curve is determined by a latency~$T$ and a service rate~$R$. Latency represents the time a server needs before starting to process the incoming data, and service rate represents the minimum rate to process data after the initial latency.

With the help of arrival and service curves, we can derive delay and backlog bounds for a system $\mcal{S}$ illustrated in Figure~\ref{fig: system bounds}. Suppose a system $\mcal{S}$ has arrival curve $\alpha$ and service curve~$\beta$, its worst-case backlog $b^*$ is the maximum vertical distance between~$\alpha$ and~$\beta$. Similarly, depending on the multiplexing technique applied to the system, its worst-case delay bound $d^*$ is the maximum horizontal distance between $\alpha$ and $\beta$ if $\mcal{S}$ is a FIFO system. If we don't have any information about its multiplexing technique, referred to as arbitrary multiplexing, the best we can say is that when $\alpha$ and $\beta$ intersect each other, where all data has been delivered out of the system. Consequently, the worst-case delay bound for arbitrary multiplexing is the time required for $\mcal{S}$ to clear its buffer.

% \begin{figure}
%     \centering
%     \includegraphics[width=\linewidth]{images/bound.png}
%     \caption{System delay/backlog bounds}
%     \label{fig: system bounds}
% \end{figure}

While a service curve captures the slowest possible output speed of a system, a link's transmission capacity limits the speed as well. Hence, we model this phenomenon using a \textit{greedy shaper} with a sub-additive function $\sigma: \mbb{R}^+ \mapsto \mbb{R}^+$ concatenated with a server. We consider a concatenation as shown in Figure \ref{fig: system}. By convention we assume $\sigma(0) = 0$ and $\beta(t) \leq \sigma(t), \forall t \in \mbb{R}^+$, meaning that the buffer is cleared at the beginning and the service never exceed its physical limitation. With the above definition, such greedy shaper conserves the service provided by the system due to theorem \ref{thm: shaping}.

\begin{figure}[thb]
    \centering
    \includegraphics[width=0.7\linewidth]{images/system.png}
    \caption{Shaping of departure data. A flow that has an arrival curve $\alpha$ feeds into a server with an arrival data flow $R(t)$. The server having service curve $\beta$ takes $R(t)$ and gives a departure data flow $R^*(t)$ to a shaper with shaping function $\sigma$. The shaper takes $R^*(t)$ and shape the data flow as another departure $D(t)$.}
    \label{fig: system}
\end{figure}


\begin{theorem}[Shaping conserves service \cite{ncbook2001leboudec}]
\label{thm: shaping}
Following the system shown in Figure \ref{fig: system}, we have
\begin{equation}
     D = R^* \otimes \sigma \geq \lp R \otimes \beta \rp \otimes \sigma = R \otimes \lp \beta \otimes \sigma \rp = R \otimes \beta
\end{equation}
\end{theorem}

In the following context, we model the shaping function $\sigma$ as a token-bucket curve $\gamma_{C,L}$ with transmission capacity $C$ and the packet size $L$ to capture the link capacity and packetization~\cite{bouillard2022tradeoff}.

\section{Additive HE for Smaller FHE Responses}
\label{sec:main}

Ciphertexts that have been processed by the server can not be compressed using the technique mentioned in \Cref{sec:background}. We propose a technique to compress LWE/RLWE ciphertexts using auxiliary information provided by the client.
% These techniques apply to cryptosystems that use LWE/RLWE ciphertexts such as TFHE~\cite{chillottiTFHEFastFully2020} and B/FV~\cite{fan2012somewhat,brakerskiFullyHomomorphicEncryption2012}, and BGV~\cite{brakerskiLeveledFullyHomomorphic2012}.

\emph{Exploiting Linear Phase Evaluation.}
In LWE and RLWE decryption, we compute an intermediate value which is commonly referred to as the \textit{phase}, i.e., $\mu^*$ and $\mu^*(X)$ in \Cref{alg:lwe-encrypt-decrypt}.
Phase evaluation is linear in both schemes and the phase is much smaller than the ciphertext itself.
The main insight behind our solution is to homomorphically compute the phase on the server using encrypted values of the secret key, encrypted under an additive encryption scheme.
Since the phase is much smaller than the original ciphertext, this results in a smaller response size.
In general, our technique can be applied to any encryption scheme that has a linear phase evaluation. Examples of encryption schemes with this property are Regev~\cite{regevLatticesLearningErrors2009}, FHEW~\cite{ducasFHEWBootstrappingHomomorphic2015}, TFHE~\cite{chillottiTFHEFastFully2020}, BFV~\cite{fan2012somewhat,brakerskiFullyHomomorphicEncryption2012}, and BGV~\cite{brakerskiLeveledFullyHomomorphic2012}.

% The main insight behind our solution is that the first step of LWE/RLWE decryption is linear in the secret key. Hence, if the client sends encryptions of the bits of the secret key to the server, encrypted under an additive encryption scheme, it can compute the first step of decryption homomorphically and send back only an encrypted scalar to the client. If the additive encryption has small ciphertexts, there is an overall size reduction.

\emph{The Additive Encryption Scheme.}
For the compression protocol, we require an additive encryption scheme which we denote $\mathcal{E}_A$ such that the plaintext space is $\ZZ_m$, for some $m$.
Also, denote the ciphertext space of $\mathcal{E}_A$ as $\mathcal{C}$.
$\mathcal{E}_A$ supports addition and plaintext multiplications.
We denote addition and plaintext multiplication with $\oplus$ and $\otimes$, respectively.
Moreover, denote the secret key generated by $\mathcal{E}_A$ as $\addkey$ and the corresponding encryption and decryption algorithms as $\texttt{AEnc}$ and $\texttt{ADec}$.

Paillier~\cite{paillierPublicKeyCryptosystemsBased1999}, Damgard-Jurik~\cite{damgardGeneralisationSimplificationApplications2001a}, Exponential ElGamal~\cite{elgamalPublicKeyCryptosystem1985}, and Benaloh~\cite{benalohDenseProbabilisticEncryption1994} are examples of cryptosystems that can be used for this purpose.

\subsection{Compressing LWE Ciphertexts}

% Let $s$ be a secret key for the Paillier cryptosystem. Plaintexts in this scheme are in $\ZZ_{m}$ for some $m$.
% Let $\texttt{AEnc}_{s}$ and $\texttt{ADec}_{s}$ denote the encryption and decryption algorithm of Paillier.
The ciphertext compression algorithm for LWE and the corresponding modified decryption algorithm is given in \Cref{alg:lwe-compress}.

\begin{algorithm}[H]
    \caption{LWE compression, performed by the server and the corresponding modified decryption process, performed by the client over a compressed ciphertext. The compression key $\ck\in\mathcal{C}^{n}$ is such that $\ck[i]=\texttt{AEnc}(\addkey, \sk[i])$.}
    \label{alg:lwe-compress}
    \begin{algorithmic}[1]
        \Procedure{LWECompress$_{q}$}{$\ck, \ct=(\textbf{a},b)$}
        \Comment{$\ct\in\ZZ^{n}\times\ZZ$}
        \State $x=b$
        \For{$i \in [n]$} 
            \State $x \leftarrow x \oplus \left( (q-\textbf{a}[i]) \otimes \ck[i] \right)$\label{alg:line-multiply}
        \EndFor
        \State \Return $x$ \Comment{$\mu^*=\texttt{ADec}(\addkey, x)$}
        \EndProcedure
    \vspace{3mm}
    \Procedure{ModifiedLWEDecrypt$_{q, p}$}{$\addkey,x$}
	 	\State $\mu^{**} = \texttt{ADec}(\addkey, x) \mod q$
            \label{alg:lwe-mod-decrypt}
	 	\vspace{1mm}
	 	\State $ \mu'' = \lfloor \mu^{**}/\Delta\rceil$
            \Comment{$\Delta=\round{q/p}$}
            \vspace{2mm}
            \State \Return $\mu'' \in \ZZ_{p}$
   \EndProcedure
   \end{algorithmic}
\end{algorithm}

\begin{theorem}[Correctness]
\label{thm:lwe-compress-correct}
    For an LWE ciphertext $\ct\in\ZZ_q^{n+1}$, if $m>q+nq^2$, then $\textsc{LWECompress}_q$ produces a compressed ciphertext which decrypts to the correct message if decrypted using \textsc{ModifiedLWEDecrypt}. More formally, if 

\begin{align*}
    x\leftarrow\textsc{LWECompress}_{q}(\ck, \ct) \\
    \mu'' \leftarrow \textsc{ModifiedLWEDecrypt}_{q,p}(\addkey, x)
\end{align*}
then
$\mu'' = \textsc{LWEDecrypt}(\sk, \ct)$
\end{theorem}

\begin{proof}
In the \textsc{ModifiedLWEDecrypt}$_{q,p}$ procedure (\Cref{alg:lwe-mod-decrypt} of \Cref{alg:lwe-compress}), we calculate 
$b + \sum_{i\in[n]} (q-\textbf{a}[i]) \cdot \sk[i]$, encrypted under additive encryption, which is achievable due to the linear properties of the additive encryption. We know that $\sk[i], \textbf{a}[i]$ and $b$ are elements in $\ZZ_q$ so $0 \leq \sk[i], \textbf{a}[i], b < q$ and 

{\footnotesize
\begin{align}
    b + \sum_{i\in[n]} (q-\textbf{a}[i]) \cdot \sk[i] \leq q + \sum_{i\in[n]} q  \cdot q = q + nq^2 < m .
\end{align}
}

so there is no overflow in the plaintext space of the additive ciphertext.
In $\textsc{ModifiedLWEDecrypt}_{q,p}$ (\Cref{alg:lwe-compress}), we have

{\footnotesize
\begin{align*}
    \mu^{**}\mod q &= \texttt{ADec}(\addkey, x) \mod q \\
    &= \left((b + \sum_{i\in[n]} (q-\textbf{a}[i]) \cdot \sk[i]) \mod m \right) \mod q \\
    &= \left(b + \sum_{i\in[n]} (q-\textbf{a}[i]) \cdot \sk[i] \right) \mod q \\
    &= b - \sum_{i\in[n]} \textbf{a}[i] \cdot \sk[i] \mod q
\end{align*}
}

This is identical to $\mu^*$ in line 1 of \Cref{alg:lwe-encrypt-decrypt}, hence, since the subsequent steps of \textsc{LWEDecrypt} and \textsc{ModifiedLWEDecrypt} are identical, they produce the same response, and the theorem is proven.
\end{proof}

In cryptosystems such as TFHE~\cite{chillottiTFHEFastFully2020}, the secret key is sampled from a binary distribution.
In such a case, we can tighten the inequality required in \Cref{thm:lwe-compress-correct} for correctness because $0\leq\sk[i]\leq 1$. The following corollary summarizes this fact.

\begin{corollary}
    If the LWE secret key is binary and $m>q+nq$, \textsc{LWECompress} produces a compressed ciphertext which decrypts to the correct message if decrypted using \textsc{ModifiedLWEDecrypt}.
\end{corollary}

% \emph{Security.}
In Gentry's original construction of a bounded depth encryption scheme, he proposed the idea of using a chain of semantically secure cryptosystems, such that each cryptosystem encrypts the secret key of the next~\cite{gentryFullyHomomorphicEncryption2009}. Gentry proved that if the secret key of each cryptosystem is sampled independently, the composed scheme is also semantically secure.

Let $\mathcal{E}'$ denote the cryptosystem which is the chaining of $\mathcal{E}_{\text{LWE}}$ and $\mathcal{E}_{A}$. The encryption and decryption procedure of $\mathcal{E}'$ is shown in \Cref{alg:lwe-encrypt-decrypt} and \Cref{alg:lwe-compress}, respectively. The secret key of $\mathcal{E}'$ is the combination of the secret keys of $\mathcal{E}_{\text{LWE}}$ and $\mathcal{E}_{A}$. The same holds for the public key as well. Moreover, we also release encryptions of the bits of the secret key of $\mathcal{E}_{\text{LWE}}$ under the secret key of $\mathcal{E}_{A}$. 

\begin{proposition}[Security]
    If $\mathcal{E}_{\text{LWE}}$ and $\mathcal{E}_A$ are semantically secure, then $\mathcal{E}'$ is also semantically secure.
\end{proposition}

\subsection{Using Smaller Compression Keys}
\label{sec:smaller-compression-key}
In practice, the plaintext space of the additive encryption system could be much larger than is required for the correctness of the compression technique to hold, i.e., $m \gg q + n q^2$.
For example, the plaintext space of Paillier for 128-bit security is 3072 bits, which is much larger than $q + n q^2$ for any common choice of LWE parameters.
We can use this gap to pack multiple bits of the LWE secret key within one additive ciphertext.
Instead of encrypting each bit of the LWE key separately, we encrypt the first $t$ bits of the secret key together into one packed additive ciphertext as $\texttt{pck}_{0-t} = \texttt{AEnc}(\addkey, \sum_{i\in[t]}\sk[i] \cdot \delta^{i})$ for a large enough $\delta$.
Specifically, $\delta$ should be such that $\delta > q + n q^2$ (or $\delta > q + n q$ in the case of binary keys).
On the server side, the server unpacks the secret key by computing $\ck[i] = \delta^{t-1-i} \otimes \texttt{pck}_{0-t}$ for $i\in[t]$.
Compression proceeds as before, with the only difference being that the encrypted phase, calculated by the server in the additive ciphertext, is scaled by a factor of $\delta^{t-1}$.
\appsection{sec:lwe-compress-packed-keys} details the procedures for generating the packed key, unpacking it, and the corresponding modified LWE decryption function.
We use the same function for compressing the ciphertext.

% Hence, \textsc{ModifiedLWEDecrypt} is modified to account for this scale.
% Specifically, line 7 of \Cref{alg:lwe-compress} changes to 
% \begin{align}
%     \mu^{**} = \left(\floor{\frac{\texttt{ADec}(\addkey, x)}{\delta^{t-1}}} \mod \delta\right) \mod q 
% \end{align}

\subsection{Batched Compression}
\label{sec:batched-compression}

To achieve better compression, multiple LWE ciphertexts (encrypted using the same secret key) can be compressed within the same additive ciphertext, which we denote as \textit{batched compression}.
Each LWE ciphertext takes up $\log_2 (q+nq^2)$ bits of the total bitwidth of the plaintext space.
So, if $m$ is the modulus of the plaintext space, then $\floor{\log_2 m / \log_2 (q+nq^2)}$ LWE ciphertexts can be compressed into one ciphertext from the additive cryptosystem.

\Cref{alg:batched-compress} illustrates how to compress $\ell$ LWE ciphertexts within one additive ciphertexts. The corresponding decryption procedure is also shown.
Using \textsc{LWECompress} as a subprocedure allows for better parallelization when compressing many LWE ciphertexts.

\begin{algorithm}[t]
	 \caption{Batch Compression of LWE ciphertexts by the server and the modified decryption procedure, performed by the client. The compression key $\ck$ is such that $\sk[i]={\texttt{ADec}(\addkey, \ck[i])}$ and $cts=\{c_j\}_{j\in [\ell]}$ such that $c_j = (\boldsymbol{\textbf{a}_j}, b_j)\in \ZZ_{q}^{n}\times\ZZ$ and $\gamma = q + n q^2$.}
	 \label{alg:batched-compress}
	 \begin{algorithmic}[1]
    \Procedure{BatchedLWECompress$_{q, \gamma}$}{$\ck, cts=\{c_j\}_{j\in [\ell]}$}
        \For{$j\in{[\ell]}$}
            \State $x_j \leftarrow \textsc{LWECompress}_q(\ck,c_j)$
            \State $x \leftarrow x \oplus \gamma^{j} x_j $
        \EndFor
        \Return $x$
    \EndProcedure
    \vspace{3mm}
    \Procedure{ModifiedBatchedLWEDecrypt$_{q,p, \gamma}$}{$\addkey, x$}
	 	\State $\mu^{**} = \texttt{ADec}(\addkey, x)$
            \For{$j \in [\ell]$}
                \State $\mu_j^{**} =\floor{\mu^{**}/\gamma^j} \mod \gamma$
	       \State $\mu''_j = \lfloor \frac{\mu_j^{**} \mod q}{\Delta} \rceil$\Comment{$\Delta=\round{q/p}$}
            \EndFor
        \Return $\{\mu_j''\in \ZZ_{p}\}_{j\in[\ell]}$
    \EndProcedure
	 \end{algorithmic}
\end{algorithm}


\begin{theorem}[Correctness]
    Let $\ct=\{c_j\}_{j\in [\ell]}$ be a vector of $\ell$ LWE ciphertexts.
    For $\gamma\geq q + n q^2$, if  $m > \gamma^{\ell}$, then \textsc{BatchedLWECompress}$_{q, \gamma}$ produces a compressed ciphertext which, if decrypted using the corresponding modified decryption, decrypts to the vector of $\ell$ plaintexts. More formally, if
    \begin{align*}
        x \leftarrow \textsc{BatchedLWECompress}(\ck, \ct, k) \\
        \{\mu'_j\}_{j\in\ell} \leftarrow \textsc{ModifiedBatchedLWEDecrypt}(
        \addkey, x)
    \end{align*}
    then $\mu'_j = \text{LWEDecrypt}(\sk,c_j)$.
\end{theorem}
\begin{proof}
    By the proof of \Cref{thm:lwe-compress-correct} we know that if $\mu^{**}_j=\texttt{ADec}_{\texttt{s}}(x_j)$, then $0\leq \mu^{**}_j < \gamma = q + nq^2$. Hence, we have
    \begin{align*}
        \mu^{**} = \sum_{j\in[\ell]} \gamma^j \mu^{**}_j \leq \sum_{j\in[\ell]} \gamma^j (\gamma-1) = \gamma^{\ell} - 1 < \gamma^{\ell} < m .
    \end{align*}
    Hence, the plaintext corresponding to $x$, i.e., $\mu^{**}$, does not overflow in the plaintext space of the additive ciphertext.
    If $\mu^*_j$ us equivalent to $\mu^*$ in the \textsc{LWEEncrypt} procedure, then for some value $t$,
    \begin{align*}
        \mu''_j =\floor{\mu^{**}/\gamma^j} \mod \gamma = (\mu^*_j + \gamma \cdot t ) \mod \gamma = \mu^*_j
    \end{align*}
    and the subsequent steps are similar, which proves the theorem.
\end{proof}

\subsubsection{Faster Batched Compression with Expanded Key}
Compression makes use of expensive operations in the additive scheme.
The plaintext multiplication in \Cref{alg:line-multiply} of \Cref{alg:lwe-compress} is the most expensive operation.
In additive schemes such as Paillier and ElGamal, this is equivalent to a modular exponentiation in a large group.

In the batched setting, we can reduce the overhead by precomputing and reusing multiples of the bits of the secret key. If we decompose $(q-\textbf{a}[i])$ as $(b_{t-1}\cdots b_1 b_0)_2 = (q-\textbf{a}[i]) \mod q$ we compute the plaintext multiplication as follows
\begin{align}
    &(q-\textbf{a}[i]) \otimes \ck[i] \\
    &= 2^{t-1} b_{t-1} \ck[i] + \cdots + 2b_{1} \ck[i] + b_{0} \ck[i]
\end{align}
and we can precompute and \textit{extended compression key}, $\eck$, such that $\eck[i][j] = 2^j \ck[i]$ for $j \in [t]$, which can reused for all LWE ciphertexts we want to compress.

\begin{algorithm}[t]
	 \caption{Batch compression of LWE ciphertexts using precomputed powers. The compression key $\ck$ is such that $\sk[i]={\texttt{ADec}(\addkey, \ck[i])}$ and $cts=\{c_j\}_{j\in [\ell]}$ such that $c_j = (\boldsymbol{\textbf{a}_j}, b_j)\in \ZZ_{q}^{n}\times\ZZ$.}
	 \label{alg:batched-compress-precompute}
	 \begin{algorithmic}[1]
    \Procedure{ExpandCompressionKey$_{q}$}{$\ck$}
        \State $\eck[0] = \ck$
        \For {$i \in [t-1]$} \Comment{$t = \ceil{\log_2 q}$}
            \For {$j \in [n]$}
                \State $\eck[i+1][j] = \eck[i][j] \oplus \eck[i][j]$
            \EndFor
        \EndFor
        \Return $\eck$
    \EndProcedure
    \vspace{3mm}
    \Procedure{FastLWECompress$_q$}{$\eck, \ct=(\textbf{a},b)$}
    \State $x=b$
    \For{$i \in [n]$} 
        \State $(b_{t-1}\cdots b_1 b_0)_2 \leftarrow (q-\textbf{a}[i]) \mod q$
        \Comment{$t = \ceil{\log_2 q}$}
        \For {$j \in [t]$}
            \If {$b_j = 1$}
                \State $x \leftarrow x \oplus \eck[j][i]$
            \EndIf
        \EndFor
    \EndFor
    \Return $x$ \Comment{$\mu^*=\texttt{ADec}(\addkey, x)$}
    \EndProcedure
    \vspace{3mm}
    \Procedure{FastBatchedLWECompress$_{q,\gamma}$}{$\ck, cts=\{c_j\}_{j\in [\ell]}$}
        \State $\eck \leftarrow \textsc{ExpandCompressionKey}_{q}(\ck)$
        \State $\gamma = q + n q^2$
        \For{$j\in{[\ell]}$}
            \State $x_j \leftarrow \textsc{FastLWECompress}_q(\eck,c_j)$
            \State $x \leftarrow x \oplus \gamma^{j} x_j $
        \EndFor
        \Return $x$
    \EndProcedure
	 \end{algorithmic}
\end{algorithm}

\begin{corollary}
    Let $\ct=\{c_j\}_{j\in [\ell]}$ be a vector of $\ell$ LWE ciphertexts.
    For $\gamma\geq q + n q^2$, if  $m > \gamma^{\ell}$, if
    \begin{align*}
        x \leftarrow \textsc{FastBatchedLWECompress}(\eck, \ct, k) \\
        \{\mu'_j\}_{j\in\ell} \leftarrow \textsc{ModifiedBatchedLWEDecrypt}(
        \addkey, x)
    \end{align*}
    then $\mu'_j = \text{LWEDecrypt}(\sk,c_j)$.
\end{corollary}


\subsubsection{Rescaling for Compression}

In some instances, it is possible to rescale the elements in the ciphertext to a smaller modulus without altering the underlying message.
This technique, also called modulus switching, is commonly used in the literature to simplify the decryption procedure or control noise growth~\cite{brakerskiFullyHomomorphicEncryption2012}.
However, rescaling is only possible if the noise of the underlying LWE ciphertext is less than a given bound. 
In \appsection{sec:modulus-switching-theorem}, we prove how rescaling is possible for LWE ciphertexts with binary keys, if the noise is less than a certain bound, i.e., less than $\Delta/4$.
Rescaling to a smaller modulus accelerates our compression technique since the scalar multiplication in the additive encryption scheme is done with a smaller scalar.

\subsubsection{Better compression with a smaller scale}
\label{sec:overlapping-noise}

The number of LWE ciphertexts that fit within each additive ciphertext is determined by the scale, i.e., $\gamma=q+nq^2$.
Using a smaller scale would allow us to pack more LWE ciphertexts within each additive ciphertext.
There are two instances where we can use a smaller scale.
First, when the LWE secret key is binary.
In that case, we can use $\gamma=q+nq$ as the scale.
This follows from the fact that in the case of binary keys, $0<\mu_j^{**} \leq \gamma = q+nq^2$.

The second instance where we can reduce the scale is by allowing $\mu_{j}^{**}$ and $\mu_{j+1}^{**}$ to overlap in the additive scheme.
Intuitively, this is possible because the high-order bits of $\mu_{j}^{**}$ are removed when it is modulized by $q$ as part of the modified decryption.
The lower order bits of $\mu_{j+1}^{**}$ are also rounded during the modified decryption so it is possible to add additional error, as long as it does not interfere with the message.
Specifically, if $|e|<\Delta/4$ (instead of the usual condition where $|e|<\Delta/2$ for correct decryption), we can reduce the scale to $\gamma=q^2$ and $\gamma=q$ in the case of non-binary and binary keys, respectively.
Due to space restrictions, we provide proof of the correctness of this technique using a smaller scale under these conditions in the full version of the paper.

% \begin{corollary}
%     If the LWE secret key is binary and $m > (q + n q)^{\ell}$, \textsc{BatchedLWECompress} produces a compressed ciphertext which decrypts to the vector of $\ell$ plaintexts if decrypted using \textsc{ModifiedBatchedLWEDecrypt}. More formally, for a compression key $\ck$, any list of ciphertexts $c=\{c_j\}_{j\in[\ell]}$ and any $k\in[\ell]$, if
%     \begin{align*}
%         x \leftarrow \text{BatchedLWECompress}(\ck, c, k)\\
%         \{\mu'_j\}_{j\in\ell} \leftarrow \text{ModifiedBatchedLWEDecrypt}_s(x)
%     \end{align*}
% then $\mu'_j = \text{LWEDecrypt}_{\sk}(c_j)$
% \end{corollary}

% The second method for reducing the scale is by allowing $\mu_{j}^{**}$ and $\mu_{j+1}^{**}$ to overlap in the additive scheme. Intuitively, this is possible because the high-order bits of $\mu_{j}^{**}$ are removed when it is modulized by $q$ as part of the modified decryption.
% The lower order bits of $\mu_{j+1}^{**}$ are also rounded during the modified decryption so it is possible to add additional error, as long as it does not interfere with the message.
% These are all summarized in the following theorem.

% \begin{theorem}
%     For $\gamma = \frac{q+nq^2}{\Delta/2}$ and $m > \gamma^{\ell}$, \textsc{BatchedLWECompress} (with the given $\gamma$) produces a compressed ciphertext which decrypts to the vector of $\ell$ plaintexts if decrypted using \textsc{ModifiedBatchedLWEDecrypt}. More formally, for a compression key $\ck$, any list of ciphertexts $c=\{c_j\}_{j\in[\ell]}$ and any $k\in[\ell]$, if
%     \begin{align*}
%         x \leftarrow \textsc{BatchedLWECompress}(\ck, c, k) \\
%         \{\mu'_j\}_{j\in\ell} \leftarrow \textsc{ModifiedBatchedLWEDecrypt}_s(x)
%     \end{align*}
%     then $\mu'_j = \textsc{LWEDecrypt}(\sk, c_j)$.
% \end{theorem}

% We provide the full proof of the fact that we can use a smaller scale in the appendix.

\subsection{Compressing RLWE Coefficients}
\label{sec:rlwe-compression}

RLWE ciphertexts also have a linear phase evaluation and hence, can benefit from our compression technique.
However, an RLWE ciphertext is only twice as large as the phase so the compression technique, applied naively, would not yield a significant improvement.
Our approach is beneficial if the user is only interested in some coefficients of the RLWE plaintext and not all of them.

The main observation is that each coefficient of $\mu'(X)$ in the \textsc{RLWEDecrypt} procedure can be calculated separately. Specifically, for $0\leq k \leq N-1$

%changing the equation number to 'normalsize'
\makeatletter
    \def\tagform@#1{\maketag@@@{\normalsize(#1)\@@italiccorr}}
\makeatother

{\tiny
\begin{align}
    \mu'&[k] = \lfloor\frac{\mu^*[k]}{\Delta}\rceil \\
    &= \left\lfloor \frac{B[k] - \sum_{i=0}^{k} A[k-i] \cdot S[i] + \sum_{i=k+1}^{N-1} A[N+k-i] \cdot S[i]}{\Delta} \right\rceil
    \label{eq:rlwe-extract-general}
\end{align}
}%

Note that the operations in the numerator are happening modulo $q$. The numerator of \Cref{eq:rlwe-extract-general} is a linear combination of the coefficients of the secret key, hence it can be computed given the encrypted coefficients of the secret key. The complete procedure to compress the coefficient of $X^k$ in an RLWE plaintext and the corresponding decryption function is shown in \Cref{alg:rlwe-compress-response}. Compression of RLWE coefficients is fully compatible with the compact compression keys of \Cref{sec:smaller-compression-key} and batched compression of \Cref{sec:batched-compression}.

\begin{algorithm}[H]
    \caption{
      Compressing Extracted RLWE Coefficient, performed by the server and the corresponding modified decryption process, for the client.
      The compression key is $\ck$ such that $\ck[i]={\texttt{AEnc}_{\texttt{s}}(S[i])}$ and $C\in R_q\times R_q$
    }
	 \label{alg:rlwe-compress-response}
	 \begin{algorithmic}[1]
    \Procedure{RLWECompressCoefficient}{$\ck, C, k$}
        \State $x=B[k]$
        \For{$i \in \{0,1,\cdots,k\}$}
            \State $x \leftarrow x \oplus \left( (q-A[k-i]) \otimes \ck[i]\right)$
        \EndFor
        \For{$i \in \{k+1,\cdots,N-1\}$}
            \State $x \leftarrow x \oplus \left(A[N+k-i] \otimes \ck[i]\right)$
        \EndFor
        
        \State \Return $x$ %\Comment{$\mu^*(X)=\texttt{ADec}_{\texttt{s}}(x)$}
    \EndProcedure
    \vspace{3mm}
    \Procedure{ModifiedRLWEDecrypt$_{q,p}$}{$\addkey, x$}
        \State $\mu^{**}_k = \texttt{ADec}(\addkey, x) \mod q$
        \vspace{1mm}
        \State $\mu''_k= \lfloor \mu^{**}_k / \Delta \rceil$
        \Comment{$\Delta=\round{q/p}$}
        \vspace{2mm}
        
        \State \Return  $\mu''_k \in \ZZ_{p}$
    \EndProcedure
	 \end{algorithmic}
\end{algorithm}


\begin{theorem}[Correctness]
\label{thm:rlwe-compress-correct}
    If $m > q + N q^2$, \Cref{alg:rlwe-compress-response} produces a compressed ciphertext which decrypts to the coefficient of $X^k$ if decrypted using \textsc{ModifiedRLWEDecrypt}$_{q,p}$. More formally, 
    \begin{align*}\normalfont
        x \leftarrow\textsc{RLWECompressCoefficient}(\ck, c, k) \\
        \mu_k'' \leftarrow\textsc{ModifiedRLWEDecrypt}_{q,p}(\texttt{s}, x)
    \end{align*}
    then $\mu_k''$ is equal to the coefficient of $X^k$ in 
    \begin{align*}
        \mu'(X) = \textsc{RLWEDecrypt}(\sk, c)
    \end{align*}
\end{theorem}

We provide the proof of \Cref{thm:rlwe-compress-correct} in \appsection{sec:prove-rlwe-compress}.
Similar to the case of LWE ciphertexts, if the coefficients of the RLWE secret key are binary, we can tighten the condition on $m$ in \Cref{thm:rlwe-compress-correct} such that $m > q + N q$.
The following corollary summarizes this fact.

\begin{corollary}
    If the coefficients of the secret key are binary and $m > q + N q$, \Cref{alg:rlwe-compress-response} produces a compressed ciphertext which decrypts to the coefficient of $X^k$ if decrypted using \textsc{ModifiedRLWEDecrypt}$_{q,p}$.
\end{corollary}

\emph{Security.} A similar argument can be made about the security of compression over RLWE. Let $\mathcal{E}''$ denote the cryptosystem which is the combination of $\mathcal{E}_{RLWE}$ and $\mathcal{E}_{A}$. The following proposition holds regarding security.

\begin{proposition}[Security]
    If $\mathcal{E}_{RLWE}$ and $\mathcal{E}_A$ are semantically secure, then $\mathcal{E}''$ is also semantically secure.
\end{proposition}


\section{Related Work}\label{sec:related-work}



Over the last few years, several benchmarks for stream processing frameworks have been proposed and stream processing benchmarking studies have been conducted. The differentiation between benchmarks and experimental studies applying them is sometimes blurry. Many publications that present benchmarks perform also an experimental study with them. On the other hand, many experimental studies utilize existing benchmarks, but modify them.
Nevertheless, we structure this section into two parts: First, we give an overview of stream processing benchmarks to justify our benchmark selection for this study. Second, we discuss related stream processing benchmarking studies.

\subsection{Related Work on Stream Processing Benchmarks}

Besides the Theodolite benchmarks for event-driven microservices used in this study, several other benchmarks for stream processing frameworks have been proposed.
\cref{tab:related-benchmarks} summarizes characteristics of the discussed benchmarks. 


\begin{table*}
	\begin{threeparttable}[b]
		\caption{Overview of the characteristics and implementations of stream processing benchmarks.}
		\label{tab:related-benchmarks}
		\footnotesize
		\newcommand{\cmark}{\ding{51}}%
		\newcommand{\xmark}{\ding{55}}%
		\newcommand{\qmark}{\makebox[0pt][l]{\textbf{\textit{?}}}\phantom{\cmark}}%
		
		\newcommand{\txnote}[1]{\makebox[0pt][l]{\tnote{#1}}}
		
		\newcommand\undefcolumntype[1]{\expandafter\let\csname NC@find@#1\endcsname\relax}
		\newcommand\forcenewcolumntype[1]{\undefcolumntype{#1}\newcolumntype{#1}}
		
		\newcommand*\rot{\rotatebox{90}}
		\newcolumntype{L}{>{\raggedright\arraybackslash}X}
		\newcolumntype{R}{>{\raggedleft\arraybackslash}X}
		\newcolumntype{C}{>{\centering\arraybackslash}X}
		\newcolumntype{o}{p{0pt}}
		\renewcommand{\arraystretch}{1.2}
		\newcommand{\fnoptional}{a}
		\newcommand{\fnbeam}{b}
		\newcommand{\fnriottasksamples}{d}
		\newcommand{\fnbeamnexmark}{c}
		\begin{tabularx}{\textwidth}{ll o C o C o CCC o CCCCCCC o C o CCC}
			\toprule
			&&& && && \multicolumn{3}{c}{Messaging} && \multicolumn{7}{c}{Stream processing framework} && && \multicolumn{3}{c}{Cloud-native} \\
			\cmidrule{8-10} \cmidrule{12-18} \cmidrule{22-24}
			Benchmark & Published && \rot{Task samples} && \rot{Open source} && \rot{Kafka} & \rot{Others} & \rot{None} && \rot{Flink} & \rot{Spark} & \rot{Storm} & \rot{Samza} & \rot{Kafka Streams} & \rot{Hazelcast Jet} & \rot{Others} && \rot{Database} && \rot{Containers} & \rot{Kubernetes} & \rot{Others} \\
			\midrule
			Theodolite \cite{BDR2021} & \citeyear{BDR2021}
			& %
			& 4
			& %
			& \cmark %
			& %
			& \cmark %
			& %
			& %
			& %
			& \cmark %
			& %
			& %
			& \cmark\txnote{\fnbeam} %
			& \cmark %
			& \cmark %
			& \cmark\txnote{\fnbeam} %
			& %
			& \phantom{\cmark}\txnote{\fnoptional} %
			& %
			& \cmark %
			& \cmark %
			& %
			\\
			Beam Nexmark \cite{BeamNexmark2022} & \citeyear{BeamNexmark2022}\txnote{\fnbeamnexmark}
			& %
			& 13
			& %
			& \cmark %
			& %
			& \cmark %
			& \cmark %
			& %
			& %
			& \cmark\txnote{\fnbeam} %
			& \cmark\txnote{\fnbeam} %
			& %
			& \qmark\txnote{\fnbeam} %
			& %
			& \qmark\txnote{\fnbeam} %
			& \cmark\txnote{\fnbeam} %
			& %
			& %
			& %
			& %
			& %
			& %
			\\
			ESPBench \cite{Hesse2021} & \citeyear{Hesse2021}
			& %
			& 5
			& %
			& \cmark %
			& %
			& \cmark %
			& %
			& %
			& %
			& \cmark\txnote{\fnbeam} %
			& \cmark\txnote{\fnbeam} %
			& %
			& \qmark\txnote{\fnbeam} %
			& %
			& \cmark\txnote{\fnbeam} %
			& \qmark\txnote{\fnbeam} %
			& %
			& \cmark %
			& %
			& %
			& %
			& %
			\\
			OSPBench \cite{vanDongen2020} & \citeyear{vanDongen2020}
			& %
			& 5
			& %
			& \cmark %
			& %
			& \cmark %
			& %
			& %
			& %
			& \cmark %
			& \cmark %
			& %
			& %
			& \cmark %
			& %
			& %
			& %
			& %
			& %
			& \cmark %
			& %
			& \cmark %
			\\
			DSPBench \cite{Bordin2020} & \citeyear{Bordin2020}
			& %
			& 5
			& %
			& \cmark %
			& %
			& \cmark %
			& %
			& %
			& %
			& %
			& \cmark %
			& \cmark %
			& %
			& %
			& %
			& %
			& %
			& \cmark %
			& %
			& %
			& %
			& %
			\\
			\citet{Shahverdi2019} & \citeyear{Shahverdi2019}
			& %
			& 1
			& %
			& \cmark %
			& %
			& \cmark %
			& %
			& %
			& %
			& \cmark %
			& \cmark %
			& \cmark %
			& %
			& \cmark %
			& \cmark %
			& %
			& %
			& \cmark %
			& %
			& %
			& %
			& %
			\\
			\citet{Karimov2018} & \citeyear{Karimov2018}
			& %
			& 2
			& %
			& %
			& %
			& %
			& %
			& \cmark %
			& %
			& \cmark %
			& \cmark %
			& \cmark %
			& %
			& %
			& %
			& %
			& %
			& %
			& %
			& %
			& %
			& %
			\\
			RIoTBench \cite{Shukla2017} & \citeyear{Shukla2017}
			& %
			& 4\txnote{\fnriottasksamples} %
			& %
			& \cmark %
			& %
			& %
			& \cmark %
			& %
			& %
			& %
			& %
			& \cmark %
			& %
			& %
			& %
			& %
			& %
			& \cmark %
			& %
			& %
			& %
			& %
			\\
			YSB \cite{Chintapalli2016} & \citeyear{Chintapalli2016}
			& %
			& 1
			& %
			& \cmark %
			& %
			& \cmark %
			& %
			& %
			& %
			& \cmark %
			& \cmark %
			& \cmark %
			& %
			& %
			& %
			& %
			& %
			& \cmark %
			& %
			& %
			& %
			& %
			\\
			SparkBench \cite{Li2015} & \citeyear{Li2015}
			& %
			& 10
			& %
			& \cmark %
			& %
			& %
			& %
			& \cmark %
			& %
			& %
			& \cmark %
			& %
			& %
			& %
			& %
			& %
			& %
			& %
			& %
			& %
			& %
			& %
			\\
			StreamBench \cite{Lu2014} & \citeyear{Lu2014}
			& %
			& 7
			& %
			& %
			& %
			& \cmark %
			& %
			& %
			& %
			& %
			& \cmark %
			& \cmark %
			& %
			& %
			& %
			& %
			& %
			& %
			& %
			& %
			& %
			& %
			\\
			Linear Road \cite{Arasu2004} & \citeyear{Arasu2004}
			& %
			& 5
			& %
			& %
			& %
			& %
			& %
			& \cmark %
			& %
			& %
			& %
			& %
			& %
			& %
			& %
			& \cmark %
			& %
			& %
			& %
			& %
			& %
			& %
			\\
			\bottomrule
		\end{tabularx}
		\begin{tablenotes}\footnotesize
			\item[\fnoptional] optional
			\item[\fnbeam] using Apache Beam
			\item[\fnbeamnexmark] the Beam Nexmark benchmarks are based on the Nexmark paper \cite{Tucker2010} published in \citeyear{Tucker2010}
			\item[\fnriottasksamples] RIoTBench's 4 application benchmarks are composed of 27 microbenchmarks
		\end{tablenotes}
	\end{threeparttable}
\end{table*}



StreamBench~\cite{Lu2014} is one of the earliest benchmarks for modern stream processing frameworks. While originally only implemented for Spark and Storm, it has later been used to benchmark Apache Apex, Beam, Flink, and Samza as well \cite{Hesse2019, Qian2016}.
As its name suggests, SparkBench~\cite{Li2015} is a benchmark tailored to Apache Spark.
The Yahoo Streaming Benchmark (YSB) \cite{Chintapalli2016} is frequently used and adapted in research \cite{Lopez2016, Yang2017, Karakaya2017, Nasiri2019, Zeuch2019, Chu2020, vanDongen2020}.
Worth highlighting is the work of \citet{Shahverdi2019}, who extend YSB with implementations for the frameworks Kafka Streams and Hazelcast Jet. As discussed in \cref{sec:frameworks}, these frameworks are particularly suited for building event-driven microservices.
RIoTBench \cite{Shukla2017} provides four application benchmarks for Storm composed of 27~small task samples. \citet{Nasiri2019} adopt RIoTBench for Flink and Spark.
\citet{Karimov2018} present a benchmark with two task samples, derived from a real industrial context, yet without providing open-source implementations.

More recently, DSPBench \cite{Bordin2020}, OSPBench~\cite{vanDongen2020}, and ESPBench \cite{Hesse2021} have been proposed.
DSPBench contains 15~benchmarks, which resample typical stream processing applications, derived from reviewing the literature.
OSPBench provides benchmarks for analyzing traffic sensor data. Besides evaluations of latency, throughput, and resource usage, \citeauthor{vanDongen2020} used OSPBench to also evaluate scalability~\cite{vanDongen2021b} and fault recovery~\cite{vanDongen2021a}.
In contrast to most other benchmarks, OSPBench provides implementations for the rather new framework Kafka Streams, which is also evaluated in this study.
The Enterprise Stream Processing Benchmark (ESPBench) builds upon the Senska benchmark \cite{Hesse2018}.
It is special in the sense that it integrates a relational database management system.
In contrast to most other benchmarks, ESPBench's task samples are implemented with Apache Beam. While \citet{Hesse2021} only perform evaluations with Spark, Flink, and Hazelcast Jet, we expect that also other Beam runners can be used to run the benchmark.

The Nexmark benchmark \cite{Tucker2010} has originally been proposed as the \textit{Niagara Extension to the XMark benchmark} addressed to first-generation stream processing systems (see the survey of \citet{Fragkoulis2023} for a discussion of first and second-generation stream processing systems).
The Apache Beam community adapted and extended Nexmark with implementations for Beam to benchmark the performance of different runners~\cite{BeamNexmark2022}.
Documentation and benchmark results are provided for the direct runner as well as for the Flink, the Spark, and the Google Cloud Dataflow runners.
However, running the benchmark with other runners should be possible as well.
Recently, there seems to be an effort to implement the Nexmark task samples with other frameworks in an open-source project.\footnote{\url{https://github.com/nexmark/nexmark}}
However, currently this project only provides implementations for Apache Flink.
Moreover, \citet{Gencer2021} implemented the Nexmark benchmark for their performance evaluation of Hazelcast Jet.

Worth mentioning is also the Linear Road benchmark presented by \citet{Arasu2004}. Although published years before all modern stream processing frameworks considered in this work have been released, it is still used in research \cite{Zhang2017,Zeuch2019,Sax2020} and compared to newer benchmarks \cite{Bordin2020,Hesse2021}.
\citet{Pagliari2020} and \citet{Garcia2022a, Garcia2022b} present approaches to generate benchmarks.






From \cref{tab:related-benchmarks}, we can see that a lot of open-source benchmarks have been proposed. Apart from the Theodolite benchmarks, none of these benchmarks is particularly addressed to scalability.
Often originating in data management research, many benchmarks are defined as ``queries'' over data streams~\cite{Tucker2010,Karimov2018,Hesse2021}.
Most benchmarks include a messaging system as a middleware component between workload generation and stream processing framework. In the vast majority of cases, this is Apache Kafka.
\citet{Karimov2018} exclude such a system to not let it become the benchmark's bottleneck. Our Theodolite benchmarks purposely include Kafka to represent more realistic event-driven microservice deployments~\cite{BDR2021}.
Flink, Spark, and Storm are by far the most supported frameworks. Only a few benchmarks exist for Samza, Kafka Streams, and Hazelcast Jet, which are frameworks particularly suited for implementing event-driven microservice. Our Theodolite benchmarks are the only ones providing implementations for all of them.
While some benchmarks include an interaction with a database in their setup, others do not.
With the Theodolite benchmarks, a database can optionally be used as we did in a previous study~\cite{IC2E2022FaaSStreaming}.
Besides our Theodolite benchmarks, there is only one other benchmark (OSPBench) that is provided as container images to be used in a cloud-native setting. No other benchmark provides Kubernetes manifests.





\subsection{Related Work on Stream Processing Benchmarking}


\begin{table*}
	\begin{threeparttable}[b]
		\caption{Overview of employed benchmarks, frameworks, and experimental setup of stream processing benchmarking studies.}
		\label{tab:related-experiments}
		\footnotesize
		\newcommand{\cmark}{\ding{51}}%
		\newcommand{\xmark}{\ding{55}}%
		\newcommand{\qmark}{\makebox[0pt][l]{\textbf{\textit{?}}}\phantom{\cmark}}%
		
		\newcommand{\txnote}[1]{\makebox[0pt][l]{\tnote{#1}}}
		
		\newcommand\undefcolumntype[1]{\expandafter\let\csname NC@find@#1\endcsname\relax}
		\newcommand\forcenewcolumntype[1]{\undefcolumntype{#1}\newcolumntype{#1}}
		
		\newcommand*\rot{\rotatebox{90}}
		\newcolumntype{L}{>{\raggedright\arraybackslash}X}
		\newcolumntype{R}{>{\raggedleft\arraybackslash}X}
		\newcolumntype{C}{>{\centering\arraybackslash}X}
		\newcolumntype{o}{p{0pt}}
		\renewcommand{\arraystretch}{1.2}
		\newcommand{\fnvandenpoel}{a}
		\newcommand{\fnbeam}{b}
		\begin{tabularx}{\textwidth}{ll o CCCCCCCCCCCCC o CCCCCCC o CCCCCC}
			\toprule
			&&& \multicolumn{13}{c}{Benchmark} && \multicolumn{7}{c}{Framework} && \multicolumn{6}{c}{Execution} \\
			\cmidrule{4-16} \cmidrule{18-24} \cmidrule{26-31}
			Publication & Year &&
			\rot{Theodolite \cite{BDR2021}} &
			\rot{Beam Nexmark \cite{BeamNexmark2022}} &
			\rot{ESPBench \cite{Hesse2021}} &
			\rot{OSPBench \cite{vanDongen2020}} &
			\rot{DSPBench \cite{Bordin2020}} &
			\rot{\citet{Shahverdi2019}} &
			\rot{\citet{Karimov2018}} &
			\rot{RIoTBench \cite{Shukla2017}} &
			\rot{YSB \cite{Chintapalli2016}} &
			\rot{SparkBench \cite{Li2015}} &
			\rot{StreamBench \cite{Lu2014}} &
			\rot{Linear Road \cite{Arasu2004}} &
			\rot{Others}
			&&
			\rot{Flink} &
			\rot{Spark} &
			\rot{Storm} &
			\rot{Samza} &
			\rot{Kafka Streams} &
			\rot{Hazelcast Jet} &
			\rot{Others}
			&&
			\rot{Cloud environment} &
			\rot{Distributed} &
			\rot{Different resource amounts} &
			\rot{\dots in isolated experiments} &
			\rot{Different load intensities} &
			\rot{\dots in isolated experiments}
			\\
			\midrule
			This work &
				& %
				& \cmark %
				& %
				& %
				& %
				& %
				& %
				& %
				& %
				& %
				& %
				& %
				& %
				& %
				& %
				& \cmark %
				& %
				& %
				& \cmark\txnote{\fnbeam} %
				& \cmark %
				& \cmark %
				& %
				& %
				& \cmark %
				& \cmark %
				& \cmark %
				& \cmark %
				& \cmark %
				& \cmark %
			\\
			\citet{IC2E2022FaaSStreaming} & \citeyear{IC2E2022FaaSStreaming}
				& %
				& \cmark %
				& %
				& %
				& %
				& %
				& %
				& %
				& %
				& %
				& %
				& %
				& %
				& %
				& %
				& \cmark\txnote{\fnbeam} %
				& %
				& %
				& \cmark\txnote{\fnbeam} %
				& %
				& %
				& \cmark\txnote{\fnbeam} %
				& %
				& \cmark %
				& \cmark %
				& \cmark %
				& \cmark %
				& \cmark %
				& \cmark %
			\\
			\citet{Hesse2021} & \citeyear{Hesse2021}
				& %
				& %
				& %
				& \cmark %
				& %
				& %
				& %
				& %
				& %
				& %
				& %
				& %
				& %
				& %
				& %
				& \cmark\txnote{\fnbeam} %
				& \cmark\txnote{\fnbeam} %
				& %
				& %
				& %
				& \cmark\txnote{\fnbeam} %
				& %
				& %
				& %
				& \cmark %
				& \cmark %
				& \cmark %
				& %
				& %
			\\
			van Dongen\tnote{\fnvandenpoel} \cite{vanDongen2021b} & \citeyear{vanDongen2021b}
				& %
				& %
				& %
				& %
				& \cmark %
				& %
				& %
				& %
				& %
				& %
				& %
				& %
				& %
				& %
				& %
				& \cmark %
				& \cmark %
				& %
				& %
				& \cmark %
				& %
				& %
				& %
				& \cmark %
				& \cmark %
				& \cmark %
				& %
				& \cmark %
				& \cmark %
			\\
			van Dongen\tnote{\fnvandenpoel} \cite{vanDongen2021a} & \citeyear{vanDongen2021a}
				& %
				& %
				& %
				& %
				& \cmark %
				& %
				& %
				& %
				& %
				& %
				& %
				& %
				& %
				& %
				& %
				& \cmark %
				& \cmark %
				& %
				& %
				& \cmark %
				& %
				& %
				& %
				& \cmark %
				& \cmark %
				& %
				& %
				& \cmark %
				& %
			\\
			\citet{Bordin2020} & \citeyear{Bordin2020}
				& %
				& %
				& %
				& %
				& %
				& \cmark %
				& %
				& %
				& %
				& %
				& %
				& %
				& %
				& %
				& %
				& %
				& \cmark %
				& \cmark %
				& %
				& %
				& %
				& %
				& %
				& \cmark %
				& \cmark %
				& %
				& %
				& \cmark %
				& \cmark %
			\\
			\citet{Chu2020} & \citeyear{Chu2020}
				& %
				& %
				& %
				& %
				& %
				& %
				& %
				& %
				& %
				& \cmark %
				& %
				& %
				& %
				& \cmark %
				& %
				& \cmark %
				& %
				& \cmark %
				& %
				& %
				& %
				& \cmark %
				& %
				& %
				& \cmark %
				& \cmark %
				& %
				& %
				& %
			\\
			\citet{Vikash2020} & \citeyear{Vikash2020}
				& %
				& %
				& %
				& %
				& %
				& %
				& %
				& %
				& %
				& %
				& %
				& %
				& %
				& \cmark %
				& %
				& \cmark %
				& \cmark %
				& \cmark %
				& %
				& %
				& %
				& \cmark %
				& %
				& %
				& \cmark %
				& %
				& %
				& \cmark %
				& \cmark %
			\\
			van Dongen\tnote{\fnvandenpoel} \cite{vanDongen2020} & \citeyear{vanDongen2020}
				& %
				& %
				& %
				& %
				& \cmark %
				& %
				& %
				& %
				& %
				& %
				& %
				& %
				& %
				& %
				& %
				& \cmark %
				& \cmark %
				& %
				& %
				& \cmark %
				& %
				& %
				& %
				& \cmark %
				& \cmark %
				& \cmark %
				& %
				& %
				& %
			\\
			\citet{Nasiri2019} & \citeyear{Nasiri2019}
				& %
				& %
				& %
				& %
				& %
				& %
				& %
				& %
				& \cmark %
				& \cmark %
				& %
				& %
				& %
				& %
				& %
				& \cmark %
				& \cmark %
				& \cmark %
				& %
				& %
				& %
				& %
				& %
				& %
				& \cmark %
				& \cmark %
				& \cmark %
				& \cmark %
				& \cmark %
			\\
			\citet{Shahverdi2019} & \citeyear{Shahverdi2019}
				& %
				& %
				& %
				& %
				& %
				& %
				& \cmark %
				& %
				& %
				& %
				& %
				& %
				& %
				& %
				& %
				& \cmark %
				& \cmark %
				& \cmark %
				& %
				& \cmark %
				& \cmark %
				& %
				& %
				& \cmark %
				& \cmark %
				& \cmark %
				& \cmark %
				& %
				& %
			\\
			\citet{Zeuch2019} & \citeyear{Zeuch2019}
				& %
				& %
				& %
				& %
				& %
				& %
				& %
				& %
				& %
				& \cmark %
				& %
				& %
				& \cmark %
				& \cmark %
				& %
				& \cmark %
				& \cmark %
				& \cmark %
				& %
				& %
				& %
				& \cmark %
				& %
				& %
				& \cmark %
				& %
				& %
				& \cmark %
				& \cmark %
			\\
			\citet{Karimov2018} & \citeyear{Karimov2018}
				& %
				& %
				& %
				& %
				& %
				& %
				& %
				& \cmark %
				& %
				& %
				& %
				& %
				& %
				& %
				& %
				& \cmark %
				& \cmark %
				& \cmark %
				& %
				& %
				& %
				& %
				& %
				& %
				& \cmark %
				& \cmark %
				& %
				& \cmark %
				& \cmark %
			\\
			\citet{Truong2018} & \citeyear{Truong2018}
				& %
				& %
				& %
				& %
				& %
				& %
				& %
				& %
				& %
				& %
				& %
				& %
				& %
				& \cmark %
				& %
				& %
				& %
				& %
				& %
				& %
				& %
				& \cmark %
				& %
				& \cmark %
				& \cmark %
				& %
				& %
				& \cmark %
				& \cmark %
			\\
			\citet{Karakaya2017} & \citeyear{Karakaya2017}
				& %
				& %
				& %
				& %
				& %
				& %
				& %
				& %
				& %
				& \cmark %
				& %
				& %
				& %
				& %
				& %
				& \cmark %
				& \cmark %
				& \cmark %
				& %
				& %
				& %
				& %
				& %
				& %
				& \cmark %
				& %
				& %
				& \cmark %
				& \cmark %
			\\
			\citet{Shukla2017} & \citeyear{Shukla2017}
				& %
				& %
				& %
				& %
				& %
				& %
				& %
				& %
				& \cmark %
				& %
				& %
				& %
				& %
				& %
				& %
				& %
				& %
				& \cmark %
				& %
				& %
				& %
				& %
				& %
				& \cmark %
				& \cmark %
				& \cmark %
				& %
				& %
				& %
			\\
			\citet{Yang2017} & \citeyear{Yang2017}
				& %
				& %
				& %
				& %
				& %
				& %
				& %
				& %
				& %
				& \cmark %
				& %
				& %
				& %
				& \cmark %
				& %
				& \cmark %
				& \cmark %
				& \cmark %
				& %
				& %
				& %
				& %
				& %
				& \cmark %
				& \cmark %
				& %
				& %
				& %
				& %
			\\
			\citet{Chintapalli2016} & \citeyear{Chintapalli2016}
				& %
				& %
				& %
				& %
				& %
				& %
				& %
				& %
				& %
				& \cmark %
				& %
				& %
				& %
				& %
				& %
				& \cmark %
				& \cmark %
				& \cmark %
				& %
				& %
				& %
				& %
				& %
				& %
				& \cmark %
				& \cmark %
				& \cmark %
				& %
				& %
			\\
			\citet{Lopez2016} & \citeyear{Lopez2016}
				& %
				& %
				& %
				& %
				& %
				& %
				& %
				& %
				& %
				& %
				& %
				& %
				& %
				& \cmark %
				& %
				& \cmark %
				& \cmark %
				& \cmark %
				& %
				& %
				& %
				& %
				& %
				& %
				& \cmark %
				& %
				& %
				& \cmark %
				& \cmark %
			\\
			\citet{Qian2016} & \citeyear{Qian2016}
				& %
				& %
				& %
				& %
				& %
				& %
				& %
				& %
				& %
				& %
				& %
				& \cmark %
				& %
				& %
				& %
				& %
				& \cmark %
				& \cmark %
				& \cmark %
				& %
				& %
				& %
				& %
				& %
				& \cmark %
				& \cmark %
				& \cmark %
				& %
				& %
			\\
			\citet{Lu2014} & \citeyear{Lu2014}
				& %
				& %
				& %
				& %
				& %
				& %
				& %
				& %
				& %
				& %
				& %
				& \cmark %
				& %
				& %
				& %
				& %
				& \cmark %
				& \cmark %
				& %
				& %
				& %
				& %
				& %
				& %
				& \cmark %
				& \cmark %
				& \cmark %
				& %
				& %
			\\
			\bottomrule
		\end{tabularx}
		\begin{tablenotes}\footnotesize
			\item[\fnvandenpoel] and van den Poel
			\item[\fnbeam] using Apache Beam
		\end{tablenotes}
	\end{threeparttable}
\end{table*}

\cref{tab:related-experiments} provides an overview of experimental performance evaluation and benchmarking studies. It indicates the applied benchmark, the evaluated stream processing, and information regarding the experiment setup and method. The latter includes whether the respective study was performed in a cloud environment, in a distributed fashion with multiple instances of the framework deployed. Moreover, it shows whether the benchmarks have been executed with different resource amounts and different load intensities and whether different resource amounts and load intensities are evaluated in isolated experiments. In previous work, we argued that scalability should be evaluated with isolated experiments for different combinations of load and resources~\cite{LTB2021,EMSE2022}.

We can observe that there is no established stream processing benchmark. Only YSB is used in several studies. However, YSB can be considered a micro-benchmark~\cite{Bermbach2017} and, hence, is less suited to benchmark entire microservices.
Except for the preliminary evaluation of our Theodolite benchmarks~\cite{BDR2021}, there is no benchmarking study addressed to stream processing frameworks employed within microservice architectures.

Flink, Spark, and Strom are by far the most frequently benchmarked frameworks. Kafka Stream, Hazelcast Jet, and Samza, which are particularly suited for implementing event-driven microservices, are only benchmarked in a few studies and there is no study benchmarking all of them.

9 out of 20 studies report on experiments in public or private clouds.
Except for this and our previous study~\cite{IC2E2022FaaSStreaming}, there are no evaluations in Kubernetes.
Likewise, there are no further studies evaluating scalability with a systematic approach as we do in this study. \citet{Vikash2020}, \citet{Nasiri2019}, \citet{Karakaya2017}, and \citet{vanDongen2021b} explicitly evaluate scalability, however, without testing different load intensities against different resource amounts in isolated experiments. \citet{Nasiri2019} conduct independent evaluations of scaling load and computing resources and, thus, address another aspect than our study.
Our previous study~\cite{IC2E2022FaaSStreaming} applies our Theodolite method as well, but benchmarks scalability with respect to costs and is addressed to comparing stream processing deployments against Function-as-a-Service offerings.



 \section{Benchmarks and Evaluation}
\label{sec:eval}

We evaluate \krakenSpace to answer the following set of questions:
\begin{itemize}
\item How much improvement does partial evaluation and our implemented compiler optimizations give \kraken? %(\S \ref{sec:eval2})
\item How much faster is our purely functional f-expr language, \krakenSpace, compared to other implementations of fexprs? %(\S \ref{sec:eval1} - \ref{sec:eval2})
\item How does \kraken's performance, with its fexprs, compare to macros? %(\S \ref{sec:eval1}, \S \ref{sec:eval3})
\item How do the different partial evaluation mechanisms/optimizations in \krakenSpace contribute towards reduction in overall runtime?
%\item What does \krakenSpace do internally when we create a data structure and evaluate it for some function? (\S \ref{sec:casestudy})
\end{itemize}

\textbf{Experimental Setup}: 
We ran these experiments in a reproducible Nix environment on a NixOS install \cite{10.1145/1411203.1411255} (Kernel 6.0.0) on a laptop with 8 cores / 16 threads and 64 GB of RAM.
Our code contains the scripts and Nix Flakes needed to reproduce the exact set of dependencies to run our tests.
%The code can be found at \url{https://github.com/limvot/kraken}.

The Kraken benchmarks were run using both the Wasmtime and WAVM WebAssembly engines for most benchmarks.
The Wasmtime WebAssembly engine is one of the most popular, developed by the Bytecode Alliance itself, and uses the CraneLift code generation backend.
The WAVM WebAssembly engine is interesting for its use of LLVM, and it often produces the fastest code on benchmarks but has a higher startup time.
We eliminated the Cfold Wasmtime benchmark due to problems running out of stack space (a known property of the Cfold benchmark).

\textbf{Benchmarks}: 
To showcase the capability of Kraken, we created benchmarks that are commonly implemented in functional languages and have been used as benchmarks in other papers \cite{reinking2021perceus, 10.1145/3547646}.
The benchmarks are
\begin{itemize}
\item Fib - Calculating the nth Fibonacci number
\item RB-Tree - Inserting n items into a red-black tree, then traversing the tree to sum its values
\item Deriv - Computing a symbolic derivative of a large expression
\item Cfold - Constant-folding a large expression
\item NQueens - Placing n number of queens on the board such that no two queens are diagonal, vertical, or horizontal from each other
\end{itemize}
All benchmarks besides Fibonacci use the fexpr version of match for pattern matching in \kraken, which is equivalent to the macro version in NewLisp. We also RB-Tree using NewLisp's~\cite{mueller2018newlisp} version of fexpr match. We modified the sizes of the problems presented to the benchmark to account for the longer running times of some of the less-optimized implementations.
The code for Kraken and NewLisp is very similar, and we should note that it is very unidiomatic NewLisp.
Our goal was not to compare Kraken and NewLisp as implementation languages for Red-Black Trees, but to stress test a single reasonably complex fexpr/macro, namely pattern matching.
% \textbf{Comparison with other languages}: We evaluated \krakenSpace against a language that contains f-exprs, as well as against itself with various optimizations disabled. The only other language we could find which contains a real f-expr mechanism is NewLisp~\cite{mueller2018newlisp} and so we ported \kraken's benchmark implementation to NewLisp.

%The six state-of-the-art languages are Java 17.0.1, Swift 5.4.2, Koka 2.3.2, C++, Haskell 8.10.7, and OCaml 4.12.
%The language choices were taken directly from Perceus reference-counting paper \cite{reinking2021perceus}.
%The Fibonacci benchmark additionally tests Python 3.9.11 and Chez Scheme 9.5.4.
%Koka, Ocaml and Haskell are good comparison points as statically-typed, compiled, functional programming languages, while Chez Scheme is a good comparison point as a mature and industrial strength dynamically-typed Scheme implementation known for its performance. 
%\subsection{Basic Level Comparison}
\subsection{The Effect of Partial Evaluation on Eval Calls}

\begin{table}[h]
\caption{Number of eval calls with no partial evaluation for Fexprs}
	\begin{tabular}{||c | c c c c c ||} 
		\hline
		&Evals & Eval w1 Calls & Eval w0 Calls & Comp Dyn & Comp Dyn\\ 
        & & & & w1 Calls & w0 Calls\\ [0.5ex] 
		\hline\hline
		Cfold 5 & 10897376 & 2784275 & 879066  & 1 & 0 \\ 
		\hline
		  Deriv 2  & 11708558 & 2990090 & 946500 & 1 & 0 \\ 
        \hline
		  NQueens 7 & 13530241 & 3429161 & 1108393 & 1 & 0 \\ 
    \hline
		  Fib 30 & 119107888 & 30450112 & 10770217 & 1 & 0 \\ 
    \hline
		  RB-Tree 10 & 5032297 & 1291489 & 398104 & 1 & 0 \\ 
		\hline
	\end{tabular}
    \label{npe:calls}
 \end{table}

As mentioned before, using fexprs without partial evaluation will prelude optimization and cause a massive amount of repeated work. Table \ref{npe:calls} and Table \ref{pe:calls} show the number of calls to the \krakenSpace runtime's eval function, the number of times the runtime's eval function executed a call to an applicative with wrap\_level=1, the number of times the runtime's eval function executed a call to an operative with wrap\_level=0, the number of compiled dynamic calls to applicatives with wrap\_level=1, and the number of compiled dynamic calls to operatives with wrap\_level=0.
These are shown for \krakenSpace test cases with partial evaluation turned off and turned on. 
\begin{table}[h]
\caption{Number of eval calls in Partially Evaluated Fexprs}
	\begin{tabular}{||c | c c c c c ||} 
		\hline
		&Evals & Eval w1 Calls & Eval w0 Calls & Comp Dyn & Comp Dyn\\ 
        & & & & w1 Calls & w0 Calls\\ [0.5ex] 
		\hline\hline
		Cfold 5 & 0 & 0 & 0  & 0 & 0 \\ 
		\hline
		  Deriv 2  & 0 & 0 & 0 & 2 & 0 \\ 
        \hline
		  NQueens 7 & 0 & 0 & 0 & 0 & 0 \\ 
    \hline
		  Fib 30 & 0 & 0 & 0 & 0 & 0 \\ 
    \hline
		  RB-Tree 10 & 0 & 0 & 0 & 10 & 0 \\ 
		\hline
	\end{tabular}
    \label{pe:calls}
 \end{table}

\begin{table}[h]
\caption{Number of calls to the runtime's eval function for RB-Tree. The table shows the non-partial evaluation numbers -> partial evaluation numbers.}
	\begin{tabular}{||c | c c c c c ||} 
		\hline
		&Evals & Eval w1 Calls & Eval w0 Calls & Comp Dyn & Comp Dyn\\ 
        & & & & w1 Calls & w0 Calls\\ [0.5ex] 
		\hline\hline
		  RB-Tree 7 & 2952848 -> 0 & 757932 -> 0 & 233513 -> 0 & 1 -> 7 & 0 -> 0\\ 
        \hline
		  RB-Tree 8 & 3532131 -> 0 & 906548 -> 0 & 279379 -> 0 & 1 -> 8 & 0 -> 0\\ 
        \hline
		  RB-Tree 9 & 4278001 -> 0 & 1097965 -> 0 & 3383831 -> 0 & 1 -> 9 & 0 -> 0\\ 
		\hline
	\end{tabular}
    \label{pe:rb}
    \vspace{-4mm}
 \end{table}

Without partial evaluation, no compilation can be done because it is impossible to tell if arguments to calls will be evaluated. In all benchmarks, partial evaluation removed all calls to the runtime's eval function, resulting in a completely compiled program. Looking at RB-Tree, there are over a million calls to combiners with wrap level 1 (normal functions), and 398,000 calls to combiners with wrap level 0 (operatives replacing macros). This massive blowup in the number of calls is due to the repeated and exponential re-execution of macro-like-combiners in the definition of other macro-like-combiners, as discussed in the Introduction.

The non-partially-evaluated benchmarks show 1 compiled dynamic call to an applicative (its the first call into eval) and 0 compiled dynamic calls to operatives, because there is no compilation at all. For the partially evaluated benchmarks, there are a few compiled dynamic calls to applicatives due to higher-order function use in the benchmarks, and there are no compiled dynamic calls to operatives, as all operative use has been eliminated.
We also varied the inputs for RB-Tree shown in Table \ref{pe:rb} to give a sense for how the number scale with respect to input size.

The incredible slowdown implied by these tables comes to full fruition in our RB-Tree test in Fig.~\ref{fig:kraken_nqueens_rbtree}.
We kept this run shorter because Kraken's non-partial-evaluating interpreter takes an incredibly long time even for 100 insertions (40 minutes).
The compounding layers of repeated macro-like operative calls in the non-partially-evaluated Kraken version cause a ~70,000x slowdown relative to the partial evaluated, optimized, and compiled version.
For the remaining benchmarks, we remove the naive interpreted \krakenSpace version, as in each case its performance is so bad as to blow out the graph and make it impossible to do any comparison.
In our optimized Kraken, our partial evaluation algorithm is able to fully collapse these levels of inefficiency, evaluate and inline the results, and give the backend more specialized code to optimize, emitting a compiled version that handily beats not only the NewLisp-fexpr implementation but even the NewLisp-macro implementation, as can be seen in Fig.~\ref{fig:kraken_vs_world_fib}.
We kept the benchmark sizes small in this test because the stack limits of NewLisp prevent sizes larger then ~880, while the Tail Call Elimination performed by the \krakenSpace compiler allows us to run much larger benchmarks, including the run of 4,800,000 inserts to the RB-Tree.
This result shows the dramatic effect of partial evaluation and compiler optimizations on runtime for \kraken. Our technique takes the performance of a fully fexpr based language from being completely infeasible to being faster than a macro-based dynamic scripting language currently in use.
% \begin{center}
% \begin{table}[ht]
% \caption{Number of call to the runtime's eval function for Fib. The table shows the non-partial evaluation numbers -> partial evaluation numbers}
% 	\begin{tabular}{||c | c c c c c ||} 
% 		\hline
% 		&Evals & Eval w1 Calls & Eval w0 Calls & Comp Dyn w1 Calls & Comp Dyn w0 Calls\\ [0.5ex] 
% 		\hline\hline
% 		Fib 10 & 8468 -> 0 & 2167 -> 0  & 777 -> 0 & 1 -> 0 & 0 -> 0 \\ 
% 		\hline
% 		  Fib 15  & 87916 -> 0 & 22478 -> 0 & 7961 -> 0 & 1 -> 0 & 0 -> 0 \\ 
%         \hline
% 		  Fib 20 & 969010 -> 0 & 247731 -> 0 & 87633 -> 0 & 1 -> 0 & 0 -> 0 \\ 
%     \hline
% 		  Fib 25 & 10740492 -> 0 & 2745825 -> 0  & 971209 -> 0 & 1 -> 0 & 0 -> 0 \\ 
% 		\hline
% 	\end{tabular}
%     \label{pe:fib}
%  \end{table}
% \end{center}

\begin{figure}[h]
\caption{Constant Fold and Deriv}
\includegraphics[width=0.45\textwidth]{cfold_table.csv_}
\includegraphics[width=0.45\textwidth]{deriv_table.csv_}
\label{fig:kraken_const_deriv}
\vspace{-6mm}
\end{figure}
\subsection{Comparison between Kraken Versions}
Beyond the massive speedup from partial-evaluation, Fig. \ref{fig:kraken_const_deriv} and \ref{fig:kraken_nqueens_rbtree} show the effect of the various compiler optimizations we described by disabling them one by one.
 Our main four optimizations have a strong positive effect on runtime, with the exception of lazy environment instantiation. Lazy environment instantiation helps massively on fib, and some on Deriv, but generally hurts the rest slightly.


\begin{figure}[h]
\caption{N-Queens}
\includegraphics[width=0.45\textwidth]{nqueens_table.csv_}
\includegraphics[width=0.45\textwidth]{slow_rbtree_table.csv_}
\label{fig:kraken_nqueens_rbtree}
\vspace{-4mm}
\end{figure}


\subsection{Comparison against Others}


To give a general idea of our current performance, we also show a Fibonacci benchmark that mostly exercises pure function-call speed and inlining as seen in Fig. ~\ref{fig:kraken_vs_world_fib}.
We include Python and Chez Scheme to give a general idea for where an exemplar slow and an exemplar fast dynamic language would fall.
With the benefit of our partial evaluation, compilation, and leaning upon mature WebAssembly implementations, we beat both, but this should be taken with a grain of salt, as this is a very limited micro-benchmark only meant to give a general sense of the order of magnitude of our performance.



\label{sec:eval1}
\begin{figure}[h]
\caption{Kraken vs. Others. Ordered by fastest to slowest}
\includegraphics[width=0.45\textwidth]{fib_table.csv_}
\includegraphics[width=0.45\textwidth]{rbtree_table.csv_}
\label{fig:kraken_vs_world_fib}
\end{figure}

%\label{sec:eval_nqueens}
%\begin{figure}[h]
%\caption{N-Queens}
%\includegraphics[width=0.45\textwidth]{nqueens_table.csv_}
%\includegraphics[width=0.45\textwidth]{slow_nqueens_table.csv_}
%\label{fig:kraken_nqueens}
%\end{figure}

%\label{sec:eval_nqueens}
%\begin{figure}[h]
%\caption{Kraken, N-Queens, absolute value and log-scale}
%\includegraphics[width=0.45\textwidth]{nqueens_table.csv_}
%\includegraphics[width=0.45\textwidth]{nqueens_table.csv_log}
%\label{fig:kraken_nqueens}
%\end{figure}
%\label{sec:eval_nqueensp}
%\begin{figure}[h]
%\caption{Kraken, N-Queens, absolute value and log-scale}
%\includegraphics[width=0.45\textwidth]{slow_nqueens_table.csv_}
%\includegraphics[width=0.45\textwidth]{slow_nqueens_table.csv_log}
%\label{fig:kraken_nqueensp}
%\end{figure}

%\label{sec:eval_cfold}
%\begin{figure}[h]
%\caption{C-Fold}
%\includegraphics[width=0.45\textwidth]{cfold_table.csv_}
%\includegraphics[width=0.45\textwidth]{slow_cfold_table.csv_}
%\label{fig:kraken_cfold}
%\end{figure}
%\label{sec:eval_cfold}
%\begin{figure}[h]
%\caption{Kraken, C-Fold, absolute value and log-scale}
%\includegraphics[width=0.45\textwidth]{cfold_table.csv_}
%\includegraphics[width=0.45\textwidth]{cfold_table.csv_log}
%\label{fig:kraken_cfold}
%\end{figure}
%\label{sec:eval_cfoldp}
%\begin{figure}[h]
%\caption{Kraken, C-Fold, absolute value and log-scale}
%\includegraphics[width=0.45\textwidth]{slow_cfold_table.csv_}
%\includegraphics[width=0.45\textwidth]{slow_cfold_table.csv_log}
%\label{fig:kraken_cfoldp}
%\end{figure}

%\label{sec:eval_deriv}
%\begin{figure}[h]
%\caption{Deriv}
%\includegraphics[width=0.45\textwidth]{deriv_table.csv_}
%\includegraphics[width=0.45\textwidth]{slow_deriv_table.csv_}
%\label{fig:kraken_deriv}
%\end{figure}
%\label{sec:eval_deriv}
%\begin{figure}[h]
%\caption{Kraken, Deriv, absolute value and log-scale}
%\includegraphics[width=0.45\textwidth]{deriv_table.csv_}
%\includegraphics[width=0.45\textwidth]{deriv_table.csv_log}
%\label{fig:kraken_deriv}
%\end{figure}
%\label{sec:eval_derivp}
%\begin{figure}[h]
%\caption{Kraken, Deriv, absolute value and log-scale}
%\includegraphics[width=0.45\textwidth]{slow_deriv_table.csv_}
%\includegraphics[width=0.45\textwidth]{slow_deriv_table.csv_log}
%\label{fig:kraken_derivp}
%\end{figure}

%\subsection{Comparison against state-of-the-art languages}
%\label{sec:eval3}

%\begin{figure}[h]
%\caption{Kraken vs. S.o.t.A.}
%\includegraphics[width=0.45\textwidth]{cfold_table.csv_}
%\includegraphics[width=0.45\textwidth]{rbtree_table.csv_}
%\label{fig:kraken_vs_world1}
%\end{figure}

%\begin{figure}[h]
%\caption{Kraken vs. S.o.t.A.}
%\includegraphics[width=0.45\textwidth]{deriv_table.csv_}
%\includegraphics[width=0.45\textwidth]{nqueens_table.csv_}
%\label{fig:kraken_vs_world2}
%\end{figure}

% \begin{figure}[h]
% \caption{Kraken vs. S.o.t.A. (Log)}
% \includegraphics[width=0.45\textwidth]{cfold_table.csv_log}
% \includegraphics[width=0.45\textwidth]{rbtree_table.csv_log}
% \label{fig:kraken_vs_world_log_1}
% \end{figure}
% \begin{figure}[h]
% \caption{Kraken vs. S.o.t.A. (Log)}
% \includegraphics[width=0.45\textwidth]{deriv_table.csv_log}
% \includegraphics[width=0.45\textwidth]{nqueens_table.csv_log}
% \label{fig:kraken_vs_world_log_2}
% \end{figure}

%As we noted before with the Fib(30) microbenchmark in Section \ref{sec:eval1}, we remain significantly slower than state-of-the-art compiled languages.
%This is particularly true for memory-intensive benchmarks due to our naive reference-counting and malloc/free implementations.
%However, our results are of a similar order of magnitude to the difference between the state-of-the-art compiled languages and dynamic scripting languages, like Python's results in the Fib(30) microbenchmark.
%We assert that is not a fundamental limitation because the classic f-expr slowness is being eliminated, as shown by Fig. \ref{fig:kraken_vs_newlisp1} and Fig. \ref{fig:kraken_vs_newlisp2}.
%In future work, we plan to expand our compile-time analysis and optimization to implement a modified, dynamic-language version of Perceus reference counting.
%With this change, we belive \krakenSpace can be competitive with these state-of-the-art languages.

%\subsection{Case Study: Red-Black Tree}
%\label{sec:casestudy}

%\begin{figure}[h]
%\caption{Kraken vs. S.o.t.A. - RB-Tree Focus}
%\includegraphics[width=0.4\textwidth]{rbtree_table.csv_}
%\includegraphics[width=0.4\textwidth]{rbtree_table.csv_log}
%\label{fig:kraken_vs_world_rbtree}
%\end{figure}


%To evaluate our partial evaluation algorithm and compiler, we extracted the benchmarks used by the Koka language project from their code repository and added Kraken versions, as well as implementing a naive Fibonacci microbenchmark ourselves to evaluate pure function call speed.\\
%With partial evaluation and the compiler optimizations listed above, we get fairly strong performance on purely numerical computations, such as the naive Fibonacci microbenchmark.
%Unfortunately, the overhead of our unsophisticated reference counting, dynamic type checking, and bounds checking causes poor performance on benchmarks involving data structures relative to mainstream programming language implementations.
%This is not a fundamental limitation, and will be addressed in future work, as recounted in the next section.
%It should be noted, however, that while the performance relative to established language implementations is very poor for the memory-intensive benchmarks (600-900x slower), we still realize a massive speedup compared to an unoptimized and non-partial-evaluated f-expr implementation (100,000x faster)!

For this chapter, fix a prime $p$. We first discuss deformations of coalgebras from $\F_{p}$
to the $p$-adic integers and further to the $p$-completed sphere $\S_{p}^{\wedge}$ which leads
us to the question of how coalgebras behave with respect to $p$-completion. We introduce the
notion of a $p$-complete coalgebra and show that this is well behaved with respect to the
deformation theory discussed in the previous chapter. We then use this to iterate
Proposition~\ref{witt} and prove our main results, namely the existence of Witt Vectors
and spherical Witt Vectors for formally \'etale coalgebras. Then we specialize to the case
of homology coalgebras, show that for a finite space $X$ the coalgebra $\F_{p}[X]$ is formally
\'etale, and answer our initial question about the relation between $\S[X]^{\wedge}_{p}$
and $\F_{p}[X]$

\subsection{Coalgebras and $p$-completion}

We have seen that the functors that interest us are all \textit{nilcomplete}. For a nilcomplete
functor $X:\rm{CAlg}^{\rm{cn}} \to \cl{S}$ and a connective $\bb{E}_{\infty}$-ring $R$, we can construct
lifts from $X(\pi_{0}R)$ to $X(R)$ inductively along the Postnikov tower
\[ \dots \to \tau_{\leq2}R \to \tau_{\tau\leq 1}R \to \tau_{\leq0} R =\pi_{0}R.\]
This is however not quite enough to obtain our goal of lifting from $\F_{p}$ to the
$p$-completed sphere, we first need to pass to $\Z_{p}= \pi_{0}\S_{p}^{\wedge}$.
Explicitly, this means constructing lifts against the tower
\[\dots \to \Z/p^{3}\to \Z/p^{2}\to \Z/p\to \F_{p}\]
which is clearly presents a different problem. With the machinery developed thus far, we can already
prove the following for a general deformation problem.

\begin{proposition}\label{liftpgen}
  Let $X: \rm{CAlg}^{\rm{cn}} \to \cl{S}$ be a cohesive functor and $A\in X(\F_{p})$
  such that $T_{X_{A}}\simeq 0$. Then there exists a unique lift of $A$ to a point in
  $\flim_{n}X(\Z/p^{n})$.
\end{proposition}
\begin{proof}
  Set $A_{0}= A$, we inductively construct lifts against the tower of square zero extensions
  \[\dots \to \Z/p^{3} \to \Z/p^{2}\to \F_{p}.\]
  Suppose we have already constructed lifts $A_{k}$ for $k\le n$ for some $n$.
  Applying Proposition~\ref{bc} inductively, we get that
  \[T_{X_{A_{n}}}^{\F_{p}} \simeq T^{\F_{p}}_{X_{A_{0}}} \simeq 0.\]
  Thus, since $\Z/p^{n+1}\to \Z/p^{n}$ is a square zero extension with fiber $\F_{p}$,
  Proposition~\ref{deformations} implies that the fiber
  \[X_{A_{n}}^{\Z/p^{n+1}}=\rm{fib}_{A_{n}}(X(\Z/p^{n+1})\to \Z/p^{n})\]
  is contractible and we find an essentially unique lift $A_{n+1}$. This proves the claim.
\end{proof}
 Of course, for an arbitrary functor $X:\rm{CAlg}^{\rm{cn}} \to \cl{S}$ the natural map
$X\to \flim_{n}X(\Z/p^{n})$ might not be an equivalence, meaning that in this generality
we can only construct pro-$p$ objects of $X$ using this inductive method.
In fact, we have that $\rm{cCAlg}_{\Z_{p}}\neq  \flim_{n} \rm{cCAlg}_{\Z/p^{n}}$. To remedy
this problem we show that this limit admits a description via \textit{$p$-complete} coalgebras.
To do this, we first recall some facts about $p$-complete modules.

\begin{definition}
Let $R$ be an $\bb{E}_{\infty}$-ring, then $M \in \rm{Mod}_{R}$ is called
$p$-\textit{complete} if the limit
\[ \lim \left(\dots \rar{\cdot p} M \rar{\cdot p}M \right)\]
vanishes. We denote the full subcategory spanned by the $p$-complete modules by $(\rm{Mod}_{R})_{p}^{\wedge}$.
\end{definition}

\begin{remark}
The inclusion $(\rm{Mod}_{R})_{p}^{\wedge} \rari{} \rm{Mod_{R}}$ admits a left adjoint which takes a module $M$
to its \textit{$p$-completion} given by the limit
\[ \lim \left( \dots \to M/p^{2} \to M/p \right).\]
In fact, $M$ is $p$-complete if and only if the natural map $M \to \lim M/p^{n}$ is an equivalence.
This inherits a natural $R^{\wedge}_{p}$-module structure, thus $p$-completion also gives
an equivalence of categories $(\rm{Mod}_{R})^{\wedge}_{p} \simeq (\rm{Mod}_{R^{\wedge}_{p}})^{\wedge}_{p}$ which
allows us to identify these in what follows.\\
The tensor product of $p$-complete modules is in general not $p$-complete. However, the
category $(\rm{Mod}_{R})_{p}^{\wedge}$ admits a symmetric monoidal structure given by the formula
 \[ M \otimes_{(\rm{Mod}_{R})_{p}^{\wedge}} N := ( M \otimes N )^{\wedge}_{p}.\]
 With this monoidal structure the $p$-completion functor $\rm{Mod}_{R}\to (\rm{Mod}_{R})_{p}^{\wedge}$
 is strong monoidal, while the inclusion is only lax monoidal.
\end{remark}

 \begin{definition}
   Let $R$ be an $\bb{E}_{\infty}$-ring. We define the $\infty$-category of $p$-complete
   $R$-coalgebras is given by.
   \[ {(\rm{cCAlg}_{R})}^{\wedge}_{p}:= \rm{cCAlg}({(\rm{Mod}_{R})}^{\wedge}_{p}).\]
 \end{definition}

 \begin{warning}
   Let $R$ be a $\bb{E}_{\infty}$-ring. Notice that by our definition a $p$-complete $R$-coalgebra
   is the same as a $p$-complete $R^{\wedge}_{p}$-coalgebra and so we do not differentiate between
   the two notions.
   However, this is \textit{not} the same as an $R^{\wedge}_{p}$-coalgebra whose underlying
   spectrum is $p$-complete. The process of $p$-completion does refine to a functor
   $\rm{cCAlg}_{R} \to (\rm{cCAlg}_{R^{\wedge}_{p}})^{\wedge}_{p}$,
   but it does not factor through the category $\rm{cCAlg}_{R^{\wedge}_{p}}$.
 \end{warning}

 We now show check that the assignment $R \mapsto \rm{cCAlg}_{R}^{\rm{cn}}$ is subject to the machinery
 of deformation theory.

 \begin{lemma}\label{conil2}
   The following statements hold:
   \begin{enumerate}
     \item   Suppose we have a pullback diagram of connective $\bb{E}_{\infty}$-rings
   \[\begin{tikzcd}
	R\p & S\p \\
	R & S
	\arrow[from=1-1, to=2-1]
	\arrow[from=2-1, to=2-2]
	\arrow[from=1-2, to=2-2]
	\arrow[from=1-1, to=1-2]
\end{tikzcd}\]
such that the map $\pi_{0}R \to \pi_{0}S$ is surjective. Then the natural map
\[ (\rm{cCAlg}_{R\p}^{\rm{cn}})^{\wedge}_{p} \to (\rm{cCAlg}_{R}^{\rm{cn}})^{\wedge}_{p}\times_{(\rm{cCAlg}_{S}^{\rm{cn}})^{\wedge}_{p}} (\rm{cCAlg}_{S\p}^{\rm{cn}})^{\wedge}_{p}\]
is an equivalence.
     \item For every connective $\bb{E}_{\infty}$-ring $R$, the natural map
           \[ (\rm{cCAlg}_{R}^{\rm{cn}})^{\wedge}_{p} \to\flim_{n} (\rm{cCAlg}_{\tau_{\le n}R}^{\rm{cn}})^{\wedge}_{p}\]
           is an equivalence.
   \end{enumerate}
 \end{lemma}
 \begin{proof}
   Ad 1.: Arguing as in the proof of Proposition~\ref{Mod}, it suffices to show that the
   strong monoidal functor
   \begin{align*}
    (\rm{Mod}_{R\p})^{\wedge}_{p} \to (\rm{Mod}_{R})^{\wedge}_{p}\times_{(\rm{Mod}_{S})^{\wedge}_{p}} (\rm{Mod}_{S\p})^{\wedge}_{p}
   \end{align*}
   is an equivalence. Indeed, given a point $(M,N,h)$ in the pullback, the $R\p$-module $M \times_{M \otimes_{R} S}N$
   is again $p$-complete since $p$-completion commutes with limits. Thus, the inverse functor of
   Proposition~\ref{Mod} also induces a functor on the categories of $p$-complete modules. Moreover,
   we have that
   \[ ((M\times_{M\otimes_{R}S}N)\otimes_{R\p} R)^{\wedge}_{p} \simeq M^{\wedge}_{p} \simeq M\]
   \[ ((M \times_{M\otimes_{R}}N)\otimes_{R\p}S\p)^{\wedge}_{p}\simeq N^{\wedge}_{p} \simeq N,\]
   where the first equivalences hold by Proposition~\ref{Mod}, and the latter since $M$ and $N$ are
   to be $p$-complete. Finally, for $M\in (\rm{Mod}_{R\p})^{\wedge}_{p}$, we compute that
   \[ (M \otimes_{R\p} R)^{\wedge}_{p}\times_{(M \otimes_{R\p} S)^{\wedge}_{p}}(M \otimes_{R\p}S\p)^{\wedge}_{p}
     \simeq \left( M \otimes_{R\p} R \times_{M\otimes_{R\p} S} M \otimes_{R\p} S\p\right)^{\wedge}_{p}
   \simeq M^{\wedge}_{p} \simeq M,\]
 where we have again used the result of Proposition~\ref{Mod} and the fact that $p$-completion commutes
 with limits.\\
 Ad 2: This uses the exact same arguments applied to the equivalence of Corollary~\ref{nilcomplete}.
 \end{proof}

 \begin{corollary}
   For any $n\in \bb{N}$, the functor
   \[ \rm{CAlg}^{\rm{cn}} \to \cl{S} \qquad R \mapsto [(\rm{cCAlg}_{R}^{\rm{cn}})^{\wedge}_{p}]^{\Delta^{n}}\]
   is coherent and nilcomplete.
 \end{corollary}

 We now prove the crucial $p$-completeness result for $\Z_{p}$-modules. As before
 this will enable us to deduce the same result for coalgebras and allow us to tackle the
 actual problem of comparing coalgebras over $\F_{p}$, $\Z_{p}$ and $\S_{p}^{\wedge}$.
\begin{proposition}\label{pcomp}
  Let $\rm{Mod}^{\wedge}_{\Z_p} \subseteq \rm{Mod}_{\Z_{p}}$ denote the full subcategory spanned by the
  $p$-complete $\Z_{p}$-module spectra. Then the natural map
  \[ \rm{Mod}_{\Z_{p}} \to \flim_{n} \rm{Mod}_{\Z/p^{n}} \quad N \mapsto (N\otimes_{\Z_{p}}\Z/p^{n})\]
  restricts to a strong monoidal equivalence
  \[(\rm{Mod}_{\Z_{p}})^{\wedge}_{p} \simeq \flim_{n}\rm{Mod}_{\Z/p^{n}}. \]
\end{proposition}
\begin{proof}
  The functor admits a right adjoint which takes $(M_{n})\in \flim_{n}\rm{Mod}_{\Z/p^{n}}$ to the limit
  $\lim_{n}M_{n}$ taken in the category of $\Z_{p}$-modules. Since $p$-complete modules are closed under
  limits, the essential image of this functor is contained in $\rm{Mod}_{\Z_{p}}^{\wedge}$. Moreover,
  if $M\in \rm{Mod}_{\Z_{p}}^{\wedge}$, then we have that
  \[ \flim_{n}(M \otimes_{\Z_{p}} \Z/p^{n}) \simeq \flim_{n} M/p^{n} \simeq M^{\wedge}_{p}\simeq M.\]
  Hence, the counit of the adjunction is an equivalence on $p$-complete modules.
  Conversely, given $(N_{k})\in \flim_{k}\rm{Mod}_{\Z/p^{k}}$ write $N= \lim_{k}N$. We want
  to show that, for every $n$ the natural map
  \[ N \otimes_{\Z_{p}} \Z/p^{n}\rar{\sim}N_{n}\]
  is an equivalence. Since $N \otimes_{\Z_{p}}Z/p^{n}\simeq N/p^{n}$ and limits are exact, we have an equivalence
  \[N \otimes_{\Z_{p}}\Z/p^{n}\simeq \lim_{k >n}(N_{k}\otimes_{\Z_{p}}\Z/p^{n}).\]
  Thus, the unit of the adjunction may be written as
  \[ \lim_{k>n}(N_{k} \otimes_{\Z_{p}}\Z/p^{n}) \to \lim_{k>n}(N_{k}\otimes_{\Z/p^{k}}\Z/p^{n})\simeq N_{n}\]
  and so has fiber given by
  \[ F_{n}:=\lim_{k>n}\left(N_{k}\otimes_{\Z/p^{k}}\rm{fib}(\Z/p^{k}\otimes_{\Z_{p}}\Z/p^{n}\to \Z/p^{n}) \right).\]
  Now we compute the fiber of $\Z/p^{k}\otimes_{\Z_{p}}\Z/p^{n}\to \Z/p^{n}$ as the module
  \[ \rm{Tor}^{\Z_{p}}(\Z/p^{k}, \Z/p^{n})[1]\simeq \Z/p^{n}[1].\]
  The reduction map $\Z/p^{k}\to \Z/p^{k-1}$ is induced by the map of projective resolutions
\[\begin{tikzcd}
	{\Z_p} & {\Z_p} \\
	{\Z_p} & {\Z_p}
	\arrow["{\cdot p^k}", from=1-1, to=1-2]
	\arrow["\id", from=1-2, to=2-2]
	\arrow["{\cdot p}"', from=1-1, to=2-1]
	\arrow["{\cdot p^{k-1}}"', from=2-1, to=2-2],
\end{tikzcd}\]
hence, on Tor it induces the multiplication by $p$ map
\[ \Z/p^{n}=\rm{Tor}^{\Z_{p}}(\Z/p^{k}, \Z/p^{n})\rar{\cdot p} \rm{Tor}^{\Z_{p}}(\Z/p^{k-1}, \Z/p^{n}) =\Z/p^{n}.\]
Thus, if we have $k\p > k > n$ such that $k\p -k > n$, the transition map
\[ F_{k\p}=N_{k\p} \otimes \rm{Tor}^{\Z_{p}}(\Z/p^{k}, \Z/p^{n})\to N_{k} \otimes \rm{Tor}^{\Z_{p}}(\Z/p^{k-1}, \Z/p^{n})= F_{k}\]
vanishes since the Tor-groups are $p^{n}$-torsion. Choosing a cofinal subset $S\subseteq \bb{N}_{>n}$ such that
$\abs{k\p -k}> n$ for any distinct $k\p,k\in S$, we see that
\[ \lim_{k>n} F_{k}\simeq \lim_{k\in S} F_{k} \simeq 0 \]
vanishes. Thus, since limits are exact, the map $N \otimes_{\Z_{p}} \Z/p^{n}\rar{\sim}N_{n}$ is an equivalence.\\
To see that the functor $\rm{Mod}_{\Z_{p}}^{\wedge} \to \flim_n \rm{Mod}_{\Z/p^{n}}$ is strong monoidal,
we observe that since cofibers and limits are exact, we have for each $n$ equivalences
\begin{align*}
  (M \otimes_{\Z_{p}} N)^{\wedge}_{p} \otimes_{\Z_{p}}\Z/p^{n} &\simeq \lim_{k}(M/p^{k} \otimes_{\Z_{p}}N/p^{k})/p^{n}\\
                                              &\simeq \lim_{k}\left((M/p^{n} \otimes_{\Z_{p}} N/p^{n})\otimes_{Z_{p}}\Z/p^{k}\right) \\
  &\simeq ((N\otimes_{\Z_{p}}\Z/p^{n}) \otimes_{\Z_{p}} (M \otimes_{\Z_{p}}\Z/p^{n}))^{\wedge}_{p}.
\end{align*}
This proves the claim.
\end{proof}

\begin{corollary}\label{pcomp1}
  We have an equivalence of categories
  \[ (\rm{cCAlg}_{\Z_{p}})_{p}^{\wedge} \rar{\sim} \flim_{n} \rm{cCAlg}_{\Z/p^{n}} \quad A \mapsto (A\otimes_{\Z_{p}}\Z/p^{n})\]
  with inverse taking a system of coalgebras $(B_{n})$ to the limit $\lim_{n}B_{n}$ taken in the
  category of ($p$-complete) $\Z_{p}$-modules, equipped with the induced $p$-complete
  $\Z_{p}$-coalgebra structure.
\end{corollary}
\begin{proof}
This follows from Proposition~\ref{pcomp}, arguing as in the proof of Proposition~\ref{Mod}.
\end{proof}

\begin{corollary}\label{obliftzp}
  Let $X(\blank)= (\rm{cCAlg}_{\blank}^{\rm{cn}})^{\Delta^{0}}$ and $A\in X(\F_{p})$ such that $T_{X_{A}}\simeq 0$.
  Then the space of lifts of $A$ to a $p$-complete $\Z_{p}$-coalgebra is contractible
\end{corollary}
 \begin{proof}
 Combine Proposition~\ref{liftpgen} and Corollary~\ref{pcomp1}.
 \end{proof}

\begin{corollary}\label{mapliftzp}
  Let $\varphi: B\to A$ be a map of connective, formally \'etale $\F_{p}$-coalgebras. Then the space of
  lifts of $\varphi$ to a map of $p$-complete $\Z_{p}$-coalgebras $B\p \to A\p$ is contractible.
\end{corollary}
\begin{proof}
    Let $ \cl{X}(\blank)=\rm{cCAlg}_{\blank}^{\rm{cn}}$. By Proposition~\ref{etalchar} the natural map
    \[ T_{\cl{X}^{\Delta^{1}}_{\varphi}} \to T_{\cl{X}^{\Delta^{0}}_{B}}\]
    is an equivalence, but since $B$ is formally \'etale we have $T_{\cl{X}^{\Delta^{0}}_{B}} \simeq 0$.
    Hence, the claim follows by applying Proposition~\ref{liftpgen} to the functor $\cl{X}^{\Delta^{1}}$
    and using Corollary~\ref{pcomp1}.
\end{proof}

Having shown this, we can now construct a functor which is analogous to the classical
Witt-Vectors, which allow us to pass from \'etale $\F_{p}$-algebras to $\Z_{p}$-algebras.

\begin{theorem}
  Let $\cl{C}\subseteq (\rm{cCAlg}_{\Z_{p}}^{\rm{cn}})^{\wedge}_{p}$ denote the full subcategory spanned by those
  coalgebras $A$ for which $A\otimes_{\Z_{p}} \F_{p}$ is formally \'etale. Then the base change functor
  \[ \cl{C} \to \rm{cCAlg}_{\F_{p}}^{\rm{cn}, \rm{f\acute{e}t}}  \qquad A \mapsto A\otimes_{\Z_{p}}\F_{p}\]
  is fully faithful and essentially surjective. In particular, the quasi inverse defines a functor
  \[ W_{p}: \rm{cCAlg}_{\F_{p}}^{\rm{cn,f\acute{e}t}} \to (\rm{cCAlg}_{\Z_{p}}^{\rm{cn}})^{\wedge}_{p}\]
  which is fully faithful and satisfies $W_{p}(A)\otimes_{\Z_{p}}\F_{p} \simeq A$ for every connective, formally
  \'etale $\F_{p}$-coalgebra $A$.
\end{theorem}

\begin{proof}
  Combine Corollary~\ref{obliftzp} and Corollary~\ref{mapliftzp}.
\end{proof}

We now turn our attention to the leap from $\Z_{p}$ to $\S_{p}^{\wedge}$. The following proposition shows that,
for an arbitrary cohesive and nilcomplete functor, a $\Z_{p}$-valued point which has vanishing $\F_{p}$-tangent
complex admits a unique lift to a $\S_{p}^{\wedge}$-valued point. This is surprising, as we do not
actually require any information about the $\Z_{p}$-tangent complex, everything is determined by
what happens modulo $p$.

\begin{proposition}\label{spherelift}
  Let $X: \rm{CAlg}^{\rm{cn}} \to \cl{S}$ be a cohesive and nilcomplete functor and let $A \in X(\Z_{p})$
  such that $T_{X_{A\otimes_{\Z_{p}}\F_{p}}}\simeq 0$. Then $A$ admits an essentially unique lift to $X(\S_{p}^{\wedge})$.
\end{proposition}

\begin{proof}
  We inductively construct lifts against the Postnikov Tower
  \[ \dots \to \tau_{\leq2} \S_{p}^{\wedge}  \to \tau_{\leq 1} \S_{p}^{\wedge} \to \tau_{\leq 0} \S_{p}^{\wedge} \simeq \Z_{p}. \]
  Write $A=A_{0},~S_{n}= \tau_{\leq n}\S_{p}^{\wedge},~ M_{n} = \pi_{n}S_{n}$ and assume we have already constructed
  a unique lift $A_{n}$ to $X(S_{n})$. Consider the square zero extension
  \[ M_{n+1}[n+1] \to S_{n+1}\to S_{n}.\]
  Since $M_{n+1} = \pi_{n+1}S_{n+1}$ is concentrated in a single degree, the $S_{n}$-action factors
  through $S_{0}=\Z_{p}$. Moreover, since $\pi_{n+1}S_{n+1}$ is of finite $p$-torsion, the action
  further factors through $\Z/p^{k}$ for some $k\geq 0$. Thus, Proposition~\ref{bc} implies that
  we have an equivalence
  \[ T_{X_{A_{n}}}^{M_{n+1}[n+1]} \simeq \Sigma^{n}T_{X_{A_{n}}}^{M_{n+1}} \simeq T_{X_{A_{n} \otimes_{S_{n}} \Z/p^{k}}}^{M_{n+1}}.\]
  Arguing as in Proposition~\ref{cofib} with respect to the square zero extension
  \[ \F_{p} \to \Z/p^{k}\to \Z/p^{k-1},\]
  we see that we have a cofiber sequence
  \[  T^{M_{n+1}\otimes_{\Z/p^{k}}\F_{p}}_{X_{A_{n} \otimes_{S_{n}} \Z/p^{k-1}}}
    \to T_{X_{A_{n} \otimes_{S_{n}} \Z/p^{k}}}^{M_{n+1}}
    \to T^{M_{n+1}\otimes_{\Z/p^{k}}\Z/p^{{k-1}}}_{X_{A_{n} \otimes_{S_{n}} \Z/p^{k-1}}}.\]
  For the left hand term, Proposition~\ref{bc} gives the equivalence
  \[ T_{X_{A_{n}\otimes_{S_{n}}\Z/p^{k-1}}}^{M_{n+1}\otimes_{\Z/p^{k}}\F_{p}}
    \simeq T_{X_{A \otimes_{\Z_{p}}\F_{p}}}^{{M_{n+1}\otimes_{\Z/p^{k}}\F_{p}}}
    \simeq T_{X_{A\otimes_{\Z_{p}}\F_{p}}}\otimes_{\F_{p}}( M_{n+1}\otimes_{\Z/p^{k}}\F_{p} ) \simeq 0,\]
  where we have used that, since $M_{n+1}$ is finitely generated, the $\F_{p}$-module
  $M_{n+1}\otimes_{\Z/p^{k}}\F_{p}$ is perfect. For the right hand term we
  replace $M_{n+1}$ with $M_{n+1} \otimes_{\Z/p^{k}}\Z/p^{k-1}$ and repeat the argument,
  inductively yielding equivalences
  \[ T^{M_{n+1}}_{X_{A_{n}\otimes_{S_{n}}\Z/p^{k}}}
    \simeq T^{M_{n+1}\otimes_{\Z/p^{k}}\Z/p^{{k-1}}}_{X_{A_{n-1} \otimes_{S_{n-1}} \Z/p^{k-1}}}
  \simeq \cdots \simeq T^{M_{n+1}\otimes_{\Z/p^{k}} \F_{p}}_{X_{A \otimes_{\Z_{p}}\F_{p}}} \simeq 0.\]
In total, this shows that $T_{X_{A_{n}}}^{M_{n+1}[n+1]} \simeq 0$, and hence $A_{n}$ admits an essentially
unique lift to $X(S_{n+1})$. Thus, the fiber over $A$ of the map
\[ X(\S_{p}^{\wedge})\simeq \flim_{n}X(S_{n})\to X( \Z_{p})\]
is contractible and we are done.
  \end{proof}

  \begin{lemma}\label{pcomparison}
    Write $\cl{X}(\blank)=\rm{cCAlg}^{\rm{cn}}_{\blank}$ and $\cl{Y}(\blank)=
    (\rm{cCAlg}^{\rm{cn}}_{\blank})^{\wedge}_{p}$. Then the $p$-completion map $f:\cl{X}\to \cl{X}\p$
    induces an equivalence
    \[ T^{M}_{(\cl{X}^{\Delta^{n}})_{\xi}} \to  T^{M}_{(\cl{Y}^{\Delta^{n}})_{f(\xi)}}\]
        for every $\F_{p}$-module $M$, $n\in \bb{N}$ and $\xi \in \cl{X}(\F_{p})^{\Delta^{n}}$.
  \end{lemma}
  \begin{proof}
    For any $\F_{p}$-algebra $R$ the $p$-completion map gives an equivalence
    $\rm{Mod}_{R}\rar{\sim} (\rm{Mod}_{R})^{\wedge}_{p}$, since multiplication by some power of $p$
    is nullhomotopic over $\F_{p}$. In particular, this applies to the split square zero
    extension $\F_{p}\oplus M$ for any $M \in \rm{Mod}_{\F_{p}}$ and so the natural map
    $\cl{X}(\F_{p}\oplus M) \to \cl{Y}(\F_{p}\oplus M)$ is an equivalence as well.
    Consequently, we also obtain natural equivalences between the fibers
    \[ (\cl{X}^{\Delta^{n}})_{\xi}^{\F_{p}\oplus M} \to  (\cl{Y}^{\Delta^{n}})_{f(\xi_)}^{\F_{p}\oplus M},\]
    which induces the equivalence of spectra
    \[ T^{M}_{(\cl{X}^{\Delta^{n}})_{\xi}} \to  T^{M}_{(\cl{Y}^{\Delta^{n}})_{f(\xi)}}\]
      as claimed.
  \end{proof}

  \begin{corollary}\label{obliftsp}
    Let $X(\blank)=(\rm{cCAlg}^{\rm{cn}}_{\blank})^{\Delta^{0}}$ and $A \in X(\F_{p})$ such that
    $T_{X_{A}}\simeq 0$, then the space of lifts of $A$ to a $p$-complete $\S_{p}^{\wedge}$-coalgebra
    is contractible.
  \end{corollary}

  \begin{proof}
    Write $Y(\blank)= ((\rm{cCAlg}^{\rm{cn}}_{\blank})^{\wedge}_{p})^{\Delta^{0}}$. Then by Lemma~\ref{pcomparison}
    we have an equivalence $T_{X_{A}}\simeq T_{Y_{A}} \simeq 0$. Hence, we can apply Proposition~\ref{obliftzp} to
    obtain an essentially unique lift $A\p\in Y(Z_{p})$. Further applying Proposition~\ref{spherelift}
    to $A\p$ yields our claim.
  \end{proof}
  Thus, we can pointwise lift $\F_{p}$-coalgebras with vanishing tangent complex to $\S_{p}^{\wedge}$. If
  we moreover consider \textit{formally \'etale coalgebras}, we can make this lifting functorial
  in a coalgebraic analogue of the \textit{Spherical Witt Vectors} construction for
  $\bb{E}_{\infty}$-algebras over $\F_{p}$.

\begin{corollary}\label{mapliftsp}
  Let $\varphi:B\to A$ be a map of $\F_{p}$-coalgebras such that $A$ and $B$ are formally \'etale.
  Then the space of lifts of $\varphi$ to a map $\varphi\p: B\p \to A\p$ of $p$-complete
  $\S_{p}^{\wedge}$-coalgebras is contractible.
\end{corollary}

\begin{proof}
  Let $ \cl{X}(\blank)=\rm{cCAlg}_{\blank}^{\rm{cn}}$ and $\cl{Y}(\blank) =
  (\rm{cCAlg}_{\blank}^{\rm{cn}})^{\wedge}_{p}$. By Proposition~\ref{mapliftzp} the map $\varphi$ admits
  an essentially unique lift to a point $\psi \in \cl{Y}(\Z_{p})^{\Delta^{1}}$. Moreover, Lemma~\ref{pcomparison}
  yields an equivalence $T_{\cl{X}^{\Delta^{1}}_{\varphi}}\simeq T_{\cl{Y}^{\Delta^{1}}_{\varphi}}$. Since both $A$ and $B$ are
  formally \'etale Proposition~\ref{etalchar} gives equivalences
  \[ T_{\cl{X}^{\Delta^{1}}_{\varphi}} \rar{\sim} T_{\cl{X}^{\Delta^{0}}_{B}} \simeq 0\]
  Hence, we can apply Proposition~\ref{spherelift} to the functor $\cl{Y}^{\Delta^{1}}$ and the point
  $\psi \in \cl{Y}^{\Delta^{1}}$, proving the claim.
\end{proof}

\begin{theorem}\label{wittsp}
  Denote by $\cl{C}\subseteq (\rm{cCAlg}_{\S_{p}^{\wedge}}^{\rm{cn}})^{\wedge}_{p} $ the full subcategory spanned by those
  coalgebras $A$ such that $A\otimes_{\S_{p}^{\wedge}}\F_{p}$ is formally \'etale. Then the base change functor
  \[ \cl{C} \to \rm{cCAlg}_{\F_{p}}^{\rm{cn}, \rm{f\acute{e}t}} \qquad A \mapsto A \otimes_{\S_{p}^{\wedge}} \F_{p}\]
  is fully faithful and essentially surjective.
\end{theorem}
\begin{proof}
  Combine Corollary~\ref{obliftsp} and Corollary~\ref{mapliftsp}.
\end{proof}

\begin{remark}
  In the setting of Theorem~\ref{wittsp} the quasi-inverse to $\blank \otimes_{\S^{\wedge}_{p}}\F_{p}$ defines
  a fully faithful functor
  \[ W_{\S_{p}^{\wedge}}: \rm{cCAlg}_{\F_{p}}^{\rm{cn}, \rm{f\acute{e}t}}
    \to (\rm{cCAlg}_{\S_{p}^{\wedge}}^{\rm{cn}})^{\wedge}_{p}\]
  which satisfies $W_{\S_{p}^{\wedge}}(A)\otimes_{\S^{\wedge}_{p}}\F_{p} \simeq A$ for every connective, formally \'etale
  $\F_{p}$-coalgebra $A$. We call $W_{\S_{p}^{\wedge}}(A)$ the \textit{spherical Witt vectors} of $A$.
\end{remark}


\subsection{Homology coalgebras}

As observed in Example~\ref{homology}, for every space $X$ and every $\bb{E}_{\infty}$-ring $R$, the
$R$-homology $R[X]$ carries a natural $R$-coalgebra structure, which is a stronger invariant than its
underlying $R$-module. We now want to apply our results and see what can be said about the deformation
theoretic behavior of homology coalgebras. To do this, we first need to compute the cotangent complex of the
$\F_{p}$-cohomology.

\begin{definition}
  A space $X\in \cl{S}$ is called $p$-finite if the following conditions hold:
  \begin{enumerate}
    \item The space $X$ is truncated.
    \item The set $\pi_{0}X$ is finite.
    \item For each $n\geq 1$ and $x\in X$, we have that $\pi_{n}(X,x)$ is a finite $p$-group.
  \end{enumerate}
  We denote the full subcategory of $\cl{S}$ spanned by the $p$-finite spaces as $\cl{S}_{p}$ and call
 $\cl{S}^{\vee}_{p} =: \rm{Pro}(\cl{S}_{p})$ the category of $p$-\textit{profinite} spaces.
\end{definition}

\begin{remark}
We can regard $\cl{S}_{p}^{\vee}$ as the category of ``formal limits'' of $p$-finite spaces $\varprojlim X_{\alpha}$.
As such there is a functor $\cl{S}^{\vee}_{p}\to \cl{S}$ which takes a formal limit to the actual limit in $\cl{S}$.
This functor admits a left adjoint given by $Y \mapsto \flim_{Y_{\alpha} \to Y} Y_{\alpha}$, where the limit runs over all maps
from a $p$-finite space $Y_{\alpha}$ to $Y$.
\end{remark}

\begin{lemma}
  Let $X$ be a space and $\flim X_{\alpha}$ be its $p$-profinite completion. Then the natural map
  of cohomology rings
  \[ \fcolim \F_{p}^{X_{\alpha}} \to \F_{p}^{X} \]
  is an equivalence.
\end{lemma}
\begin{proof}
  This is immediate since the Eilenberg-MacLane spaces $K(\F_{p},n)$ are $p$-finite.
\end{proof}

\begin{proposition}[Mandell, Lurie]\label{coetal}
  Let $X$ be a space, then the $\F_{p}$-cohomology $\F_{p}^{X}$ is a formally \'etale $\F_{p}$-algebra.
\end{proposition}
\begin{proof}
  Since the functor $R \mapsto L_{R/\F_{p}}$ commutes with colimits, the claim follows from the fact that
  $L_{\F_{p}^{X}/\F_{p}}\simeq 0$ for every $p$-finite space $X$ which is proven
  in~\cite[][Proposition 2.4.12]{dag8}.
\end{proof}

Thus we obtain the following result about the homology coalgebra of a finite space $X$
with coefficients in a connective $\F_{p}$-algebra $R$:

\begin{corollary}\label{goal}
  Let $X$ be a finite space and $R$ be an $\F_{p}$-algebra, then $R[X]$ is a formally
  \'etale $R$-coalgebra.
\end{corollary}
\begin{proof}
  From Proposition~\ref{coetal} we get that
  \[ L_{R^{X}/R}\simeq L_{\F_{p}^{X}/\F_{p}}\otimes_{\F_{p}}R \simeq 0.\]
  Since $X$ is finite, the coalgebra $R[X]$ is dualizable with dual given by $R^{X}$, so the claim
  follows from Proposition~\ref{dualetal}.
\end{proof}

Moreover, for the case $R=\F_{p}$, we can use Theorem~\ref{wittsp} to give a partial answer to our
initial question about lifts of the coalgebra $\F_{p}[X]$.

\begin{corollary}
  Let $X$ be a finite space, then $\F_{p}[X]$ admits a unique lift to a $p$-complete $\S_{p}^{\wedge}$-coalgebra
  given by $W_{\S_{p}^{\wedge}}(\F_{p}[X]) \simeq (\S[X])^{\wedge}_{p}$. Moreover, for any other finite space $Y$
  the natural map
  \[\rm{Map}_{(\rm{cCAlg}_{\S_{p}^{\wedge}})^{\wedge}_{p}}((\S[Y])^{\wedge}_{p}, (\S[X])^{\wedge}_{p})
    \to \rm{Map}_{\rm{cCAlg}_{\F_{p}}}(\F_{p}[Y], \F_{p}[X])\]
  is a homotopy equivalence.
\end{corollary}
\begin{proof}
 Combine Corollary~\ref{goal} and Theorem~\ref{wittsp}.
\end{proof}

\section{Where to go from here}

We finish our discussion by explaining some of the shortcomings of our results and sketch a possible
way to proceed towards a coalgebraic analogue of Mandell's Theorem. The first missing puzzle piece is
the cotangent complex of a coalgebra $A$, which we have been unable to give a solid definition of.
The second and more important one is the relation to the \textit{coalgebra Frobenius}. We conjecture
that the class of \textit{perfect} coalgebras defined via this map give examples of non-dualizable
formally \'etale coalgebras. In particular, this conjecture would imply that the $\F_{p}$-homology
of \textit{any} space $X$ is formally \'etale.

\subsection{The cotangent complex of a coalgebra}
One of the first questions that arose during this project turned out to be one of the most subtle and
tricky ones, namely:

\begin{question}
  What is the cotangent complex of a coalgebra $A$?
\end{question}

Clearly, the existence of a single spectrum controlling the deformation theory of $A$ would be immensely
useful. However, it is not immediately clear what the universal property of such a spectrum should be,
i.e.~which space of derivations it should (co)represent.
Some inspiration can be gleamed from Proposition~\ref{cotangentder}. There we had seen that, for
$\varphi: B \to A$ a map of $R$-coalgebras with $A$ dualizable and $M$ an $R$-module, we have an equivalence
\[ \rm{Der}_{\varphi}(B, C_{A}(M)) \simeq \rm{Map}_{A^{\vee}}(L_{A^{\vee}/R}, \varphi^{\vee}_{\pt}\rm{map}_{R}(B, M)).\]
To get rid of the dependence on the second coalgebra $B$ one is tempted to take $B=R$ such that
$\rm{map}_{R}(B,M)\simeq M$. However, not every coalgebra $A$ admits a map $R\to A$, much less a canonical
one. The only natural choice for a map that is not the initial map would yield the following:

\begin{definition}[Preliminary 1.]
  Let $R$ be an $\bb{E}_{\infty}$-ring and $A\in \rm{cCAlg}_{R}$. The cotangent complex of $A$, if it exists,
  is the $R$-module $L_{A}$ corepresenting the functor
  \[ \rm{Mod}_{R}\to \rm{Mod}_{R} \qquad M \mapsto \rm{der}_{\id}(A, C_{A}(M))\]
\end{definition}

There are however several problems with this. Firstly, it is entirely unclear from the definition
whether $L_{A}$ vanishing would actually imply $A$ being formally \'etale. Moreover, in the dualizable
case it would lead to the rather awkward formula
\[ L_{A} \simeq L_{A^{\vee}/R}\otimes_{A^{\vee}}A.\]
Although somewhat plausible, this again gives us little information about what can actually be
deduced in the case that $L_{A}\simeq 0$.
This leaves us with several options, lest we accept that there is no good notion of one singular
cotangent complex. For one we could work with \textit{coaugmented} coalgebras, namely coalgebras
together with a map $R \to A$. For the purpose of understanding homology coalgebras this would correspond
to considering pointed spaces instead of just spaces, an entirely acceptable compromise, but beyond the
scope of this paper. \\
A different  approach would be to give up on the idea of corepresentability
and instead hope for a colimit preserving functor. For example, the functor
\[ \rm{Mod}_{R}\to \rm{Mod}_{R} \qquad M \mapsto C_{A}(M):=\rm{cofib}( A \rar{\eps} \Omega^{\infty}_{A}M).\]
seems to have no chance of preserving limits, but since colimits of coalgebras are formed underlying,
colimits are not out of the race. This leads us to the following idea:

\begin{definition}[Preliminary 2]\label{dream}
  Let $R$ be an $\bb{E}_{\infty}$-ring and $A\in \rm{cCAlg}_{R}$. We say that $A$ admits a cotangent
  complex $L_{A}:= C_{A}(R)$ if the functor $C_{A}(\blank):\rm{Mod}_{R} \to \rm{Mod}_{R}$ commutes
  with colimits. In this case we have $C_{A}(M)\simeq L_{A}\otimes M$ for every $ M \in \rm{Mod}_{R}$
\end{definition}

This definition is highly speculative, as the only coalgebras we know to admit a cotangent complex
in this sense are the formally \'etale coalgebras, for which the functor $C_{\blank}(A)$ is constant.
Conversely, if $A$ admits a cotangent complex then $L_{A}$ vanishes if and only if $A$ is formally
\'etale. Hence, the spectrum $L_{A}$ is precisely the obstruction to $A$ being formally \'etale,
which is the kind of conceptual clarity we are looking for.
While we lose any direct comparison to the cotangent complex of $A^{\vee}$ this is not entirely surprising,
since the property of being formally \'etale is defined very differently for $A^{\vee}$.
This leaves us with the following:

\begin{question}\label{cotangentdream}
  Let $R$ be an $\bb{E}_{\infty}$-ring. Does every $A \in \rm{cCAlg}_{R}$ admit a cotangent complex in the sense
  of Definition~\ref{dream}?
\end{question}

Regardless of the answer, the takeaway should be that the modules
$C_{A}(M)$ are exactly the obstruction towards $A$ being formally \'etale. Moreover, while the functor
$A\mapsto C_{A}(M)$ is very complicated, the dependence on $M$ should be relatively tame. That is,
for fixed $A$ it should be possible to describe the functor $M \mapsto C_{A}(M)$ in terms of a
formula involving $C_{A}(R)$. However, because $C_{A}(M)$ no longer has a direct relation to any
space of derivations or tangent complex, we cannot leverage results like Proposition~\ref{structure}
to obtain such a formula. We understand this as an indication that for these questions, the formalism may
have reached its limit.

\subsection{The Frobenius}
The most lacking thing about our results is the class of coalgebras that we can currently apply them to.
As of now, we are unable to give examples of formally \'etale coalgebras which are not dualizable. In
particular, we cannot describe the deformation theory of $R[X]$ for spaces $X$ which are not finite.
Attempts to reduce to the dualizable case all seem to fail for the following reason: Even though
we may write $X= \fcolim_{i}X_{i}$ where each $X_{i}$ is finite, giving the formula
$R[X]= \fcolim_{i}R[X_{i}]$, there is no reason why the functor
$\Omega^{\infty}_{\blank}(M): \rm{cCAlg}_{R}\to \rm{cCAlg}_{R}$ should commute with colimits.
Indeed, write $f_{M}:R\to R\oplus M$ for inclusion, then by definition
$\Omega^{\infty}_{\blank}(M) = f_{M,!} f^{\pt}_{M}$. The functor $f^{\pt}_{M}$ commutes with colimits,
and from Proposition~\ref{present} and the converse of the adjoint functor theorem we can deduce
that $f_{M,!}$ commutes with $\kappa$-filtered colimits for some regular cardinal $\kappa$. Thus, the class
of formally \'etale coalgebras is closed under $\kappa$-filtered colimits, but $\kappa$ is, in general, not countable.
% Closely related is the fact the notion of compactness is strangely behaved for coalgebras. For example,
% one can show that $\bb{Q}$ is not a compact object of $\rm{cCAlg}_{\Q}$, see~\cite[][Warning 1.2.15.]{ellII}.
% In particular, this means that
% \[ \rm{cSpec}(\fcolim_{i}\S[X_{i}])(\bb{Q})\neq \fcolim_{i}\rm{cSpec}(\S[X_{i}])(\Q),\]
% so we cannot deduce things about the cospectrum of infinite spaces in this way either. \\
This goes to show that the deformation theory of non-dualizable coalgebras is richer and more
interesting than that of the Ind-completion of dualizable coalgebras and requires additional input.
One contender for this additional input is the \textit{Coalgebra Frobenius} constructed by
Nikolaus:

\begin{theorem}[Nikolaus]
  Let $\cl{C} = (\rm{cCAlg}^{\rm{cn}}_{\S^{\wedge}_{p}})^{\wedge}_{p}$, then there exists a natural transformation
  $\psi_{p}:\id_{\cl{C}}\to \id_{\cl{C}}$ which on an object $A\in \cl{C}$ is given by the composition
  \[ \psi_{p}: A \rar{\Delta_{A}^{\otimes p}} (A^{\otimes p})^{hC_{p}} \rar{\rm{can}} (A^{\otimes p})^{tC_{p}} \rar{\sim} A,\]
  where the final map is the inverse of the \textit{Tate Diagonal}, see~\cite[][Theorem III.1.7]{tch}.
\end{theorem}

Given this map, we are naturally led to define \textit{perfect} coalgebras as follows:

\begin{definition}
  We say that $A \in  (\rm{cCAlg}^{\rm{cn}}_{\S^{\wedge}_{p}})^{\wedge}_{p}$ is \textit{perfect} if the coalgebra
  Frobenius $\psi_{p}: A\to A$ is a homotopy equivalence. We denote the full subcategory spanned by
  the perfect coalgebras by $(\rm{cCAlg}^{\rm{cn}}_{\S^{\wedge}_{p}})^{\wedge ,\rm{perf}}_{p} \subseteq
  (\rm{cCAlg}^{\rm{cn}}_{\S^{\wedge}_{p}})^{\wedge}_{p}$.
\end{definition}

\begin{example}\label{frobchains}
  Let $X$ be any space. Then $(\S[X])^{\wedge}_{p}$ is a perfect coalgebra since we have that
  \[\S[X]^{\wedge}_{p} \simeq (\S_{p}^{\wedge}[\colim_{X}\pt])^{\wedge}_{p} \simeq (\colim_{X} \S_{p}^{\wedge})^{\wedge}_{p}.\]
  On $\S_{p}^{\wedge}$ the map $\psi_{p}$ is necessarily given by the identity, because $\S_{p}^{\wedge}$
  is the terminal $p$-complete $\S_{p}^{\wedge}$-coalgebra. Thus, by naturality $\psi_{p}$ is given
  by the identity on $(\S[X])^{\wedge}_{p}$ as well.
\end{example}

We conjecture that this Frobenius map is related to the deformation theory of coalgebras in a similar
way to the Algebra Frobenius, in that it provides a sufficient condition for a coalgebra to be formally
\'etale.

\begin{conjecture}\label{frobcof}
  Let $A \in (\rm{cCAlg}^{\rm{cn}}_{\S^{\wedge}_{p}})^{\wedge}_{p}$ and write $A\p= A\otimes_{\S^{\wedge}_{p}}\F_{p}$.
  Then for any $M \in \rm{Mod}_{\F_{p}}^{\rm{cn}}$, the coalgebra Frobenius $\psi_p:A\to A$ induces the zero map
  on the $R$-module  $C_{A\p}(M) = \rm{cofib}(A\p \rar{\eta_{A\p}} \Omega^{\infty}_{A}(M))$.
\end{conjecture}

\begin{corollary}
  If Conjecture~\ref{frobcof} holds, then the base change functor
  \[ (\rm{cCAlg}^{\rm{cn}}_{\S^{\wedge}_{p}})^{\wedge ,\rm{perf}}_{p} \to \rm{cCAlg}_{\F_{p}}^{\rm{cn}}
  \qquad A \mapsto A\otimes_{\S_{p}^{\wedge}}\F_{p}\]
is fully faithful and factors through the full subcategory
$\rm{cCAlg}_{\F_{p}}^{\rm{cn}, \rm{f\acute{e}t}}\subseteq \rm{cCAlg}_{\F_{p}}^{\rm{cn}}$.
\end{corollary}
\begin{proof}
  Since $\psi_{p}:A\rar{\sim} A$ is an equivalence it induces an equivalence on $A\otimes_{\S_{p}^{\wedge}}\F_{p}$ and
  thus on $C_{A\otimes_{\S_{p}^{\wedge}}\F_{p}}(M)$ as well. However, since it also induces the zero map on the latter
  we get that $C_{A\otimes_{\S_{p}^{\wedge}}\F_{p}}(M)\simeq 0$. Thus, $A\otimes_{\S_{p}^{\wedge}}\F_{p}$ is formally \'etale and the
  claim follows from Theorem~\ref{wittsp}.
\end{proof}

Combining this with Example~\ref{frobchains} would allow us to fully answer our initial question about
homology coalgebras.

\begin{corollary}\label{dream2}
  If Conjecture~\ref{frobcof} holds, then for any space $X$ the $\F_{p}$-chains $\F_{p}[X]$
  are formally \'etale. In particular $\F_{p}[X]$ admits a unique and functorial lift to a $p$-complete
  $\S_{p}^{\wedge}$-coalgebra given by $\S[X]^{\wedge}_{p}= W_{\S_{p}^{\wedge}}(\F_{p}[X])$.
\end{corollary}

The fact that Conjecture~\ref{frobcof} needs to be checked for every connective $\F_{p}$-module should
be understood as an extension of our failure to find a cotangent complex. Indeed, if $\F_{p}[X]$ admits
a cotangent complex in the sense of Definition~\ref{dream}, then to obtain Corollary~\ref{dream2} it
would suffice to show that $\psi_{p}$ induces the zero map on $C_{A\otimes_{\S_{p}^{\wedge}}\F_{p}}(\F_{p})
= L_{A\otimes_{\S_{p}^{\wedge}}\F_{p}}$. However, even for this specific module the conjecture is difficult
to attack from our present position. The problem is the tricky right adjoint
$\rm{cCAlg}_{\F_{p}\oplus \F_{p}}\to \rm{cCAlg}_{\F_{p}}$ appearing in the definition of
$C_{A\otimes_{\S^{\wedge}_{p}}\F_{p}}(\F_{p})$. Because there is no known formula for this functor, attempts to verify
the conjecture have thus far been unsuccessful in all non-trivial cases. This warrants further investigation
of the coalgebra Frobenius and Conjecture~\ref{dream2}.

\section{\protocol{} : PIR using Compression}

In this section, we present \textit{\protocol{}}, our new low-communication PIR protocol.
\protocol{} takes advantage of the compression technique in this work, in combination with techniques in the literature such as layered encryption~\cite{angelPIRCompressedQueries2018a, kiayiasOptimalRatePrivate2015, lipmaaObliviousTransferProtocol2005} and PIR using LWE~\cite{henzingerOneServerPrice2023, davidsonFrodoPIRSimpleScalable2023}.
This results in a PIR protocol with the lowest overall communication costs compared to all related work.
We first describe a simple version of \protocol{} and show how to adapt the construction to work with smaller keys and larger payloads.

\subsection{\protocol{} Description}

We restructure the database as a $(k+1)$-dimensional hypercube and perform PIR using LWE on the first dimension and PIR using Paillier on the subsequent dimensions.
Using LWE in the first dimension improves the runtime significantly, but we add some layers of Paillier to reduce the communication costs.


\Cref{alg:compresspir} shows the detailed description of \protocol{}. We follow the framework of PIR with preprocessing~\cite{liReducingCiphertextExpansion2015} and provide four routines.
The high-level steps for \protocol{} are as follows:
\begin{enumerate}
    \item In the setup phase~(\Cref{compresspir:hint-compute}), the server hint is calculated and stored by the server. This hint can be reused for all queries by any client and updated locally as the database changes.
    \item Upon receiving the query from the client, the server first expands the compression key to use throughout the process~(\Cref{compresspir:expand-key}).
    \item Using $\qu_0$, The server performs PIR using LWE on the first dimension, which is assisted by the hint generated in the setup~(\Cref{compresspir:online-lwe}). This step is similar to that of related work~\cite{henzingerOneServerPrice2023,davidsonFrodoPIRSimpleScalable2023}.
    \item The output of this step is rescaled and compressed using the compression functions and the expanded key to get Paillier ciphertexts.
    \item The Paillier ciphertexts, $\qu_1,\cdots, \qu_{k}$, are then used to do PIR using layered encryption, expanding the size by a factor of two in each layer. The result is sent to the client~(\Cref{compresspir:start-paillier-pir}-\ref{compresspir:paillier-pir-1}).
    \item Upon receiving the response, the client decrypts the Paillier ciphertexts and finally performs a modified LWE decryption to retrieve the response~(\Cref{compresspir:start-paillier-decrypt}-\ref{compresspir:paillier-merge}).
\end{enumerate}

\begin{algorithm}[]
% \scriptsize  % Make the font smaller
\caption{
  Complete description of \protocol{}. $q$ is the LWE ciphertext modulus, and $r$ is the smallest divisor of $q$ such that $r\geq 2(n+1)p$, $p$ is the plaintext modulus and $\Delta = q/p$.
  Database $\db\in\ZZ_p^{N_0\times d_0}$ where $N=d_0 d_1 \cdots d_k$ and $N_{\ell} = N / (d_0 d_1 \cdots d_{\ell})$ for all $\ell$. Also, as setup $\textbf{A} \sample \ZZ_q^{d_0 \times n}$. Paillier ciphertexts are in $\ZZ_{m^2}$. We denote Paillier homomorphic addition and scalar multiplication by $\oplus$ and $\otimes$, respectively.
}
	 \label{alg:compresspir}
	 \begin{algorithmic}[1]
    \Procedure{Setup}{$\db \in \ZZ_q^{N_0 \times d_0}$}
        \State $\hint = \db \times \textbf{A} \in \ZZ_q^{N_0 \times n}$ \label{compresspir:hint-compute}
        \State \Return $\hint$
    \EndProcedure
    \vspace{3mm}
    \Procedure{Query}{$(i,i_0)\in [N_0]\times[d_0]$}
        \State Generate Paillier keys $(\paillierpk, \paillierkey)$
        \State Sample LWE key $\lwekey\sample\{0,1\}^{n}$
        \State $\ck \leftarrow \textsc{PaillierEncrypt}(\paillierkey, \sk)$ \label{compresspir:compression-key}
        \For {$\ell \in \{1,2,\cdots,k\}$}
            \State $i_{\ell} = \floor{i/N_{\ell}} \mod d_{\ell}$
        \EndFor
        \State $u_j =$ selection vector for index $i_j$, $j\in[k+1]$
        \State Sample $e\leftarrow \chi^n$
        \State $\qu_0 = \textbf{A}\cdot \lwekey + e + \Delta\cdot u_0$ \Comment{$\qu_0\in\ZZ_q^{d_0}$}\label{compresspir:lwe-encrypt}
        \For {$\ell \in \{1,2,\cdots,k\}$}
                \State $\qu_{\ell} = \textsc{PaillierEnc}(\paillierkey, u_{\ell})$ 
            \Comment{$\qu_{\ell}\in\ZZ_{m^2}^{d_{\ell}}$}
                \label{compresspir:paillier-encrypt}
        \EndFor
        \State \Return $(\paillierkey, (\paillierpk, \ck, {\qu}_0, {\qu}_1, \cdots, {\qu}_k))$
    \EndProcedure
    \vspace{3mm}
    \Procedure{Response}{$\db, \hint,\qu=(\paillierpk, \ck,{\qu}_0, {\qu}_1, \cdots, {\qu}_k)$}
        \State $\eck \leftarrow \textsc{ExpandCompressionKey}_{r}(\ck)$ \label{compresspir:expand-key}
        \State $b = \db \cdot {\qu}_0$ \label{compresspir:online-lwe}
        \State $\textbf{D} = [\hint | b] \in \ZZ_r^{N_0 \times (n+1)}$
        \State $\textbf{D}_0\leftarrow$ Rescale elements in $\textbf{D}$ to modulus $r$ \label{compresspir:rescale-online}
        \For {$i\in [N_0]$}
            \Comment{$c_0\in\ZZ_{m^2}^{N_0\times 1}$}
            \State $c_0[i] = \textsc{FastLWECompress}_{r}(\eck, \textbf{D}_0[i])$
            \label{compresspir:fastcompress}
        \EndFor
        \For {$\ell = \{1, \cdots, k\}$} \label{compresspir:start-paillier-pir}
            \State Initialize $c_{\ell}$ with zeros
            \Comment{$c_{\ell} \in \ZZ_{m^2}^{N_{\ell}\times 2^{\ell}}$}
            \For {$t\in [N_{\ell}]$}
                \For {$h \in [2^{\ell-1}]$}
                    \For {$j \in [d_{\ell}]$}
                        \State $u_{0j} \leftarrow c_{\ell-1}[jN_{\ell}+t][h] \mod m$
                        \State $u_{1j} \leftarrow \floor{c_{\ell-1}[jN_{\ell}+t][h] / m}$
                        \State $c_{\ell}[t][2h] \leftarrow c_{\ell}[t][2h] \oplus (\qu_{\ell}[j] \otimes u_{0j}) $ \label{compresspir:paillier-pir-0}
                        \State $c_{\ell}[t][2h+1] \leftarrow c_{\ell}[t][2h+1] \oplus (\qu_{\ell}[j] \otimes u_{1j})$ \label{compresspir:paillier-pir-1}
                    \EndFor
                \EndFor
            \EndFor
        \EndFor
        \State \Return $c_k[0] \in \ZZ_{m^2}^{2^k}$
    \EndProcedure
    \vspace{3mm}
    \Procedure{Extract}{$\st = \paillierkey, f_k \in \ZZ_{m^2}^{2^k}$}
        \For {$\ell \in \{k, k-1, \cdots, 1\} $}
        \label{compresspir:start-paillier-decrypt}
            \State $p_{\ell} = \textsc{PaillierDec}(\paillierkey, f_{\ell})$
            \Comment{$p_{\ell}\in\ZZ_{m}^{2^{\ell}}$}
            \For {$h\in[2^{\ell-1}]$}
                \State $f_{\ell-1}[h] = m\cdot p_{\ell}[2h+1] + p_{\ell}[2h]$
                \label{compresspir:paillier-merge}
            \EndFor
        \EndFor
        \State $f=\textsc{ModifiedLWEDecrypt}_{r, p}(\paillierkey, f_0)$
        \label{compresspir:lwe-dec}
        \State\Return $f$
    \EndProcedure    
  \end{algorithmic}
\end{algorithm}



\begin{theorem}[\protocol{}]
    For LWE parameters $(n, q, \chi_e, \chi_s)$, assume $\chi_e$ is a discrete Gaussian with standard deviation $\sigma$ and $\chi_s$ is a binary distribution. Assume these parameters are $\epsilon_{L}$-secure for LWE with $d_0$ samples and take plaintext modulus $p$ such that $p|q$ and 
    \begin{align}
        q/p > 2p\sigma \sqrt{2d_0\ln (2/\delta)} 
        \label{eq:correctness}
    \end{align}
    Also, assume we instantiate the Paillier cryptosystem with modulus $m$ such that it is $\epsilon_{P}$-secure and $m > q+nq$. Then for a random matrix $\textbf{A}\in\ZZ_{q}^{d_0\times n}$, \protocol{} is a $2(\epsilon_{L}+\epsilon_{P})$-secure PIR scheme on database of size $N$ with items in $\ZZ_p$, with $1-\delta$ success rate.
\end{theorem}

The complete proof of correctness and security are provided in \appsection{appendix:proof}.

\subsubsection{Concrete Costs of \protocol{}}
\label{sec:concrete-costs}

The concrete number of operations in each routine in \protocol{}, along with the party that must perform those operations is listed below.
\begin{itemize}
    \item $\textsc{Setup}$ (Server): $N_0 d_0 n$ multiplications and additions in $\ZZ_q$ (\Cref{compresspir:hint-compute})
    \item $\textsc{Query}$ (Client): $d_0$ LWE encryptions (\Cref{compresspir:lwe-encrypt}) and $n+\sum_{\ell=1}^{k} d_{\ell}$ Paillier encryptions (\Cref{compresspir:compression-key} and \Cref{compresspir:paillier-encrypt})
    \item $\textsc{Response}$ (Server): $N=N_0d_0$ multiplications and additions in $\ZZ_q$ (\Cref{compresspir:online-lwe}), and $n\log r+\frac{1}{2} n\log_2 r$ multiplications in $\ZZ_{m^2}$, i.e., Paillier additions (\Cref{compresspir:expand-key} and \Cref{compresspir:fastcompress}) and $\sum_{\ell\in[k]} 2^{\ell} N_{\ell}$ exponentiations in $\ZZ_{m^2}$, i.e., Paillier scalar multiplications (\Cref{compresspir:paillier-pir-0} and \Cref{compresspir:paillier-pir-1}).
    \item $\textsc{Extract}$ (Client): $2^{k+1}$ Paillier decryptions
\end{itemize}

Similarly, the concrete communication costs of the protocol are listed below. 
\begin{itemize}
    \item Client to Server: $d_0 \log_2 q + 2 \log_2 m (n + \sum_{\ell=1}^{k} d_i)$
    \item Server to Client: $2^{k+1}\log_2 m$
\end{itemize}

% \begin{table}[H]
%     \caption{Concrete number of operations and concrete communication costs in \protocol{}}
%     \label{tab:my_label}
%     \centering
%     \begin{tabular}{rl}
%     \toprule
%         \textbf{Step} & \textbf{Number of Operations} \\
%     \midrule
%     \midrule
%         $\textsc{Setup}$ (S) & \begin{tabular}{l}$N_0 d_0 n$ multiplications and additions in $\ZZ_q$\end{tabular} \\
%     \midrule
%         $\textsc{Query}$ (C) & 
%         \begin{tabular}{l}
%              $d_0$ LWE encryptions (\Cref{compresspir:lwe-encrypt}) and \\
%              $n+\sum_{\ell=1}^{k} d_{\ell}$ Paillier encryptions (\Cref{compresspir:compression-key} and \ref{compresspir:paillier-encrypt})
%         \end{tabular} \\
%     \midrule
%         $\textsc{Response}$ (S) &
%         \begin{tabular}{l}
%              $N=N_0d_0$ mult. and add in $\ZZ_q$ (\Cref{compresspir:online-lwe})\\
%              $n\log r+\frac{1}{2} n\log_2 r$ multiplications in $\ZZ_{m^2}$, \\
%              ~~~~~~i.e., Paillier add (\Cref{compresspir:expand-key} and \ref{compresspir:fastcompress})\\
%              $\sum_{\ell\in[k]} 2^{\ell} N_{\ell}$ exponentiations in $\ZZ_{m^2}$,\\
%              ~~~~~~i.e., Paillier scalar mult. (\Cref{compresspir:paillier-pir-0} and \ref{compresspir:paillier-pir-1})
%         \end{tabular}
%         \\
%     \midrule
%         $\textsc{Extract}$ (C) & \begin{tabular}{l}
%              $2^{k+1}$ Paillier decryptions
%         \end{tabular} \\
%     \bottomrule
%     \toprule
%         \textbf{Step} & \textbf{Communication Cost} \\
%     \midrule
%     \midrule
%         Query & $d_0 \log_2 q + 2 \log_2 m (n + \sum_{\ell=1}^{k} d_i)$ \\
%     \midrule
%         Response & $2^{k+1}\log_2 m$ \\
%     \bottomrule
%     \end{tabular}
% \end{table}

\subsection{Updating the Hint} 
When the database changes, the hint must also be updated, but the hint can be updated locally by the server and does not require any communication with the clients.
This is in contrast to other works which rely on hints that require sending updates to the clients~\cite{henzingerOneServerPrice2023, davidsonFrodoPIRSimpleScalable2023}.
Moreover, small changes to the database can be handled with small changes to the hint to reduce the computation cost.
For example, assume $\db'$ is an updated database compared to $\db$. If we denote the hint for the new database by $\hint'$, then the relationship between the previous hint and the new hint would be as follows
\begin{align}
    \hint' = \mask\cdot\db' = \mask\cdot(\db +\db_{\Delta}) = \hint + \mask\cdot\db_{\Delta}
\end{align}
where $\db_{\Delta}$ is the difference between the two databases.
Assuming the change is small, many columns of $\db_{\Delta}$ will be zero. For example, assume that only column $k$ of $\db_{\Delta}$ has non-zero elements, then we only need to calculate $\mask\cdot\db_{\Delta}[:,k]$, which is a matrix-vector multiplication with only $d_0n$ multiplications and additions in $\ZZ_q$.

\subsection{Smaller Keys or Compressed Large Payloads}
\protocol{} can be modified in one of two ways to either reduce the size of the compression keys or produce a more compressed response.
The changes required for these two modifications can not combined, so we propose two variants of \protocol{} which we denote \protocolsingle{} (with smaller compression keys) and \protocolbatched{} (with more compressed responses).
We provide a high-level description of these modifications and leave the full detailed description for the full version.

In \protocolsingle{}, we use the technique from \Cref{sec:smaller-compression-key} to produce packed compression keys.
The server must then unpack the compressed keys before using them.
More precisely, in \Cref{alg:compresspir}, we change \Cref{compresspir:compression-key} to encrypt the key in a compressed way using \textsc{GeneratePackedKey} from \Cref{alg:packed-key-compress}.
We also add a step before \Cref{compresspir:expand-key} to unpack the key using \textsc{UnpackCompressionKey} and change the \Cref{compresspir:lwe-dec} to the corresponding function with packed keys, \textsc{ModifedLWEDecryptPackedKey}.

In \protocolbatched{}, we adapt the protocol to produce more compressed responses when the payload is large, which is done using the batched compression technique.
The high-level description of \protocolbatched{} is as follows:
Assume each database element is an element in $\ZZ_p^{\ell}$ for some $\ell\in\NN$.
Let $\db^j$ denote a database consisting of the $j^{th}$ component of all elements in this database.
Corresponding to this, we generate $\hint^j$, $\textbf{D}^j$, and $\textbf{D}_{0}^{j}$, as is done in \Cref{alg:compresspir}.
We replace \Cref{compresspir:fastcompress} to perform a batched compression in the following manner
\begin{align}
    \textsc{FastBatchedLWECompress}_{r, \gamma}(\eck, \{\mathbf{D}_0^{j}\}_{j\in[\ell]})
\end{align}

and proceed with the rest of the protocol as before.
We also change the decryption function in \Cref{compresspir:lwe-dec} to the corresponding decryption for modified batched decryption.

\subsection{Additional techniques for \protocol{}}

In addition to the techniques mentioned in the previous two section, we use two more techniques to further reduce communication costs.
Firstly, we can use the technique from \Cref{sec:overlapping-noise} in one of two ways 1) Pack more LWE ciphertexts within each Paillier ciphertext to produce a smaller response or 2) Pack more bits of the secret key with each Paillier ciphertext to have a smaller compression key.
As mentioned, to produce correct results using this technique, the error in the LWE ciphertext must be small enough, so we must consider this constraint for correctness.
This changes the correctness condition in \Cref{eq:correctness} to 
\begin{align}
    \label{eq:correctness-modified}
    q/p > 4p\sigma \sqrt{2d_0\ln (2/\delta)} 
\end{align}
which is what we use in our experimental evaluation.
We also use a technique proposed by Beck~\cite{beckRandomizedDecryptionRD2015} to reduce the size of uploaded Paillier ciphertexts, at the cost of small computational overhead for the server.
Due to space restrictions, we provide the proof of correctness and security of \protocol{} using these techniques in the full version of the paper.

\section{PIR Evaluation}

For our evaluation, we first detail the process for selecting the parameters of \protocol{}.
Given the many parameters that must be chosen, this amounts to a non-trivial optimization problem.
After that, we provide runtimes for \protocol{} to demonstrate the scalability. Finally, we compare with related work on PIR with no setup such as WhisPIR, HintlessPIR, and YPIR.
Our results demonstrate that \protocol{} introduces a new category of PIR protocols with low communication that 

\subsection{Parameter Selection for \protocol{}}
The LWE parameters directly effect the performance of \protocol{}.
However, the set of secure LWE parameters is large and experimenting with all parameters set is infeasible.
Hence, we instantiate \protocol{} with three LWE parameters $(n,q,\chi_e, \chi_s)$ which are representative of different tradeoffs.
A smaller $n$ and $q$ results in fewer operations in based on the analysis of \Cref{sec:concrete-costs}, but limit the choice of $p$.
In contrast, higher $n$ and $q$ allow for a larger $p$ and $d_0$.
We choose three $(n,q)$ pairs, and let $\chi_e$ and $\chi_s$ be a discrete Gaussian with standard deviation $\sigma = 6.4$ and uniform binary distribution, respectively.
While in other works based on LWE~\cite{henzingerOneServerPrice2023,davidsonFrodoPIRSimpleScalable2023}, $q$ is chosen as a power of two, e.g., $2^{32}$ or $2^{64}$, to leverage the native CPU word size, we opt for a small $q$ since the bitlength of $q$ determines the total communication cost and number of operations.
Our chosen parameters for LWE provide 128-bit security based on the lattice estimator~\cite{albrechtConcreteHardnessLearning2015}.
We set the failure rate to $\delta=2^{-40}$ and for every $p$ find the upper bound for $d_0$ based on the correctness constraint, \Cref{eq:correctness}.
The upper bound on $d_0$ for each parameter set and each value of $p$ is shown in \Cref{tab:d0-bound}.
For the Paillier cryptosystem, we use a 3072-bit modulus which provides 128-bit security~\cite{barkerRecommendationKeyManagement2020}.

% \begin{table}[H]
%     \centering
%     \caption{Upper bound on $d_0$ for every value of $p$, based on a variant of \Cref{eq:correctness}. In all cases, $\sigma=6.4$, and $\delta=2^{-40}$.}
%     \label{tab:d0-bound}
%     \begin{tabular}{c|c|c|c|c|c|c|c|c}
%     \toprule
%          $(n,q)$ & $2^1$ & $2^2$ & $2^3$ & $2^4$ & $2^5$ & $2^6$ & $2^7$ & $2^8$ \\
%     \midrule
% $(630, 17)$ & 28825 & 1801 & 112 & 7 & - & - & - & - \\
% $(840, 22)$ & 3.0e+07 & 1.8e+06 & 1.2e+05 & 7206 & 450 & 28 & 1 & - \\
% $(1023, 27)$ & 3.0e+10 & 1.9e+09 & 1.2e+08 & 7.4e+06 & 4.6e+05 & 28825 & 1801 & 112 \\
%     \bottomrule
%     \end{tabular}
% \end{table}

\begin{table}[H]
    \centering
    \caption{Upper bound on $d_0$ for every value of $p$, based on \Cref{eq:correctness-modified}. In all cases, $\sigma=6.4$, and $\delta=2^{-40}$. Dashes indicate cases where no $d_0$ satisfies the equation.}
    \label{tab:d0-bound}
    \begin{tabular}{c|c|c|c}
    \toprule
         \diagbox{$p$}{$(n,q)$} & $(630, 17)$ & $(840, 22)$ & $(1023, 27)$ \\
    \midrule
        $2^1$ & 28825 & 29517568 & 30225990335 \\
        $2^2$ & 1801 & 1844848 & 1889124395 \\
        $2^3$ & 112 & 115303 & 118070274 \\
        $2^4$ & 7 & 7206 & 7379392 \\
        $2^5$ & - & 450 & 461212 \\
        $2^6$ & - & 28 & 28825 \\
        % $2^7$ & - & 1 & 1801 \\
        % $2^8$ & - & - & 112 \\
    \bottomrule
    \end{tabular}
\end{table}


The two important metrics to evaluate performance are total communication cost and server online runtime.
The remainder of the parameters, such as the dimensions of the database $\{d_i\}$, $p$, and the choice of the protocol (\protocolsingle{} or \protocolbatched{}) are chosen to minimize these costs.
However, the selection of these parameters is a non-trivial optimization problem.
While the communication cost can be derived with a closed-form formula, this is not the case for the server online runtime.
Moreover, many parameter sets fall on the Pareto frontier, i.e., are dominant in either communication or computation.
Hence, for a fixed number of rows and payload size, we aim to find as many parameter sets that fall on the Pareto frontier.
For this, we iterate over the list of all parameter sets and maintain a list which is Pareto optimal.
To narrow down the search space, represent each parameter set by the number of operations in the different steps, as calculated in \Cref{sec:concrete-costs}.
Parameter sets which are worse than another parameter set in all steps are trivially excluded.
Moreover, we use logistic regression to predict if a parameter set A dominates another parameter set B, given the concrete number of operations in the steps of the protocol.
This further reduces the space of parameter sets to a manageable size, which we can run experimentally.

\subsection{Performance of \protocol{}}
\Cref{fig:zippir-costs-per-num-rows} visualizes the communication and computation cost of \protocol{} as a function of the database size, with the goal of retrieving at least one bit from the database.
Using the procedure described in the previous section, we maintain the list of Pareto optimal parameters for each database size and plot them.
For each database size, each point on the communication graph corresponds to a point on the runtime graph, i.e., the point with higher communication has lower computation and vice-versa.
Within the communication graph, we also include the minimum required communication for related work on low-communication PIR.

There are several important observations from this graph.
Firstly, we observe that \protocolsingle{} is the best option, given that the requested payload is small.
Cases which \protocolbatched{} only appear when the payload size is large.
Second, for small databases, the communication cost of \protocol{} is small, as opposed to all other protocols in the literature, which have a lower bound on communication due to the use of large cryptographic keys.
Lastly, the minimum communication cost of \protocol{} grows sublinearly, roughly proportional to $|\db|^{0.27}$, which demonstrates the scalability of the protocol.

\newcommand{\myScatterClasses}{
    scatter/classes={
        small single={mark=*,mark size=0.6pt,yellow},
        medium single={mark=*,mark size=0.8pt,orange},
        large single={mark=*,mark size=1pt,red},
        small batched={mark=*,mark size=0.6pt,white},
        medium batched={mark=*,mark size=0.8pt,white},
        large batched={mark=*,mark size=1pt,white}
    }
}

\newcommand{\protocolsinglesize}{1.5pt}
\newcommand{\protocolsinglecolor}{red}

\newcommand{\protocolbatchedsize}{1.5pt}
\newcommand{\protocolbatchedcolor}{blue}

\newcommand{\hintlesssize}{1.5pt}
\newcommand{\hintlesscolor}{cyan}

\newcommand{\whispirsize}{1.5pt}
\newcommand{\whispircolor}{teal}

\newcommand{\ypirsize}{1.5pt}
\newcommand{\ypircolor}{green}


\begin{figure}[t]
    \centering
    \begin{subfigure}{\columnwidth}
        \centering
        \begin{tikzpicture}
            \begin{axis}[
                % xlabel={Number of Rows},
                ylabel={\footnotesize Online Time (s)},
                \myScatterClasses,
                width=\textwidth,
                height=0.4\textwidth,
                xmode=log,
                log basis x=2,                   
                ymode=log,
                log basis y=10,                   
                xtick={
                        8388608, 33554432, 134217728, 536870912, 2147483648, 8589934592, 34359738368, 137438953472, 549755813888, 2199023255552    
                },
                xticklabels={}
            ]
            \pgfplotstableread[col sep=comma]{data/Total_Size_vs_N_s1.csv}\datatable
            \addplot[
                scatter, 
                only marks,
                scatter src=explicit symbolic
            ] table [meta=mode, x=N input, y=Server Online Time] {\datatable};
            \end{axis}
        \end{tikzpicture}
        % \caption{8MB $=2^{20}\times$ 8B}
    \end{subfigure}
    ~
    \begin{subfigure}{\columnwidth}
        \centering
        \begin{tikzpicture}
            \begin{axis}[
                xlabel={Database Size},
                ylabel={\footnotesize Total Size (KB)},
                \myScatterClasses,
                width=\textwidth,
                height=0.6\textwidth,
                xmode=log,
                log basis x=2,                   
                ymode=log,
                log basis y=10,       
                % ymax = 400,
                xtick={
                    % 8388608,
                    16777216,
                    % 33554432,
                    % 67108864,
                    134217728,
                    % 268435456,
                    % 536870912,
                    1073741824,
                    % 2147483648,
                    % 4294967296,
                    8589934592,
                    % 17179869184,
                    34359738368,
                    % 68719476736,
                    137438953472,
                    % 274877906944, 549755813888, 1099511627776, 2199023255552,
                    4398046511104
                    },
                xticklabels={
                    % 1,
                    2MB,
                    % 4MB,
                    % 8,
                    16MB,
                    % 32MB,
                    % 64MB,
                    128MB,
                    % 256MB,
                    % 512,
                    1GB,
                    % 2048,
                    4GB,
                    % 8192,
                    16GB,
                    % 32768, 65536, 131072, 262144,
                    524288
                },
                tick label style={font=\footnotesize}, % or \footnotesize, \scriptsize, \tiny
                legend pos = north west
            ]
            
            \addplot[mark=none, \hintlesscolor, samples=2, domain=8000000:160000000000] {387};
            \addlegendentry{\scriptsize HintlessPIR (Lower Bound)}

            \addplot[mark=none, \whispircolor, samples=2, domain=8000000:160000000000] {441};
            \addlegendentry{\scriptsize WhisPIR (Lower Bound)}

            \addplot[mark=none, \ypircolor, samples=2, domain=8000000:160000000000] {846};
            \addlegendentry{\scriptsize YPIR (Lower Bound)}
            
            \pgfplotstableread[col sep=comma]{data/Total_Size_vs_N_s1.csv}\datatable
            \addplot[
                scatter, 
                only marks,
                scatter src=explicit symbolic
            ] table [meta=mode, x=N input, y=Total Size (KB)] {\datatable};

            % \addplot+[mark=none,samples=200,unbounded coords=jump,domain=8000000:160000000000] {pow(x, 0.21)};
            
            \end{axis}
        \end{tikzpicture}
        % \caption{Payload = 8 B}
    \end{subfigure}
    \caption{
        Communication cost and Server Online Runtime as a function of the database size. Each point in the upper graph as a corresponding point in the lower graph.
        We also plot the minimum communication required for other protocols in the literature.
        The yellow, orange, and red points correspond to $n=(630,17),(840,22),(1023, 27)$, respectively.
    }
    \label{fig:zippir-costs-per-num-rows}
\end{figure}


% \begin{figure*}[t]
%     \centering
%     \begin{subfigure}{0.3\textwidth}
%         \centering
%         \begin{tikzpicture}
%             \begin{axis}[
%                 % xlabel={Number of Rows},
%                 ylabel={\footnotesize Online Time (s)},
%                 \myScatterClasses,
%                 width=\textwidth,
%                 height=0.6\textwidth,
%                 xmode=log,
%                 xtick={1024, 16384, 524288},
%                 xticklabels={$2^{10}$, $2^{14}$, $2^{19}$},
%             ]
%             \pgfplotstableread[col sep=comma]{data/Total_Size_vs_N_s64.csv}\datatable
%             \addplot[
%                 scatter, 
%                 only marks,
%                 scatter src=explicit symbolic
%             ] table [meta=mode, x=N, y=Server Online Time] {\datatable};
%             \end{axis}
%         \end{tikzpicture}
%         % \caption{8MB $=2^{20}\times$ 8B}
%     \end{subfigure}
%     \hfill
%     \begin{subfigure}{0.3\textwidth}
%         \centering
%         \begin{tikzpicture}
%             \begin{axis}[
%                 % xlabel={Number of Rows},
%                 ylabel={\footnotesize Online Time (s)},
%                 \myScatterClasses,
%                 width=\textwidth,
%                 height=0.6\textwidth,
%                 xmode=log,
%                 xtick={1024, 16384, 524288},
%                 xticklabels={$2^{10}$, $2^{14}$, $2^{19}$},
%             ]
%             \pgfplotstableread[col sep=comma]{data/Total_Size_vs_N_s2048.csv}\datatable
%             \addplot[
%                 scatter, 
%                 only marks,
%                 scatter src=explicit symbolic
%             ] table [meta=mode, x=N, y=Server Online Time] {\datatable};
%             \end{axis}
%         \end{tikzpicture}
%     \end{subfigure}
%     \hfill
%     \begin{subfigure}{0.3\textwidth}
%         \centering
%         \begin{tikzpicture}
%             \begin{axis}[
%                 % xlabel={Number of Rows},
%                 ylabel={\footnotesize Online Time (s)},
%                 \myScatterClasses,
%                 width=\textwidth,
%                 height=0.6\textwidth,
%                 xmode=log,
%                 xtick={1024, 16384, 524288},
%                 xticklabels={$2^{10}$, $2^{14}$, $2^{19}$},
%             ]
%             \pgfplotstableread[col sep=comma]{data/Total_Size_vs_N_s262144.csv}\datatable
%             \addplot[
%                 scatter, 
%                 only marks,
%                 scatter src=explicit symbolic
%             ] table [meta=mode, x=N, y=Server Online Time] {\datatable};
%             \end{axis}
%         \end{tikzpicture}
%     \end{subfigure}
%     ~
%     \begin{subfigure}{0.3\textwidth}
%         \centering
%         \begin{tikzpicture}
%             \begin{axis}[
%                 xlabel={\footnotesize Number of Rows},
%                 ylabel={\footnotesize Total Size (KB)},
%                 \myScatterClasses,
%                 width=\textwidth,
%                 height=0.6\textwidth,
%                 xmode=log,
%                 ymode=log,
%                 xtick={1024, 16384, 524288},
%                 xticklabels={$2^{10}$, $2^{14}$, $2^{19}$},
%             ]
%             \pgfplotstableread[col sep=comma]{data/Total_Size_vs_N_s64.csv}\datatable
%             \addplot[
%                 scatter, 
%                 only marks,
%                 scatter src=explicit symbolic
%             ] table [meta=mode, x=N, y=Total Size (KB)] {\datatable};
%             \end{axis}
%         \end{tikzpicture}
%         \caption{Payload = 8 B}
%     \end{subfigure}
%     \hfill    
%     \begin{subfigure}{0.3\textwidth}
%         \centering
%         \begin{tikzpicture}
%             \begin{axis}[
%                 xlabel={\footnotesize Number of Rows},
%                 ylabel={\footnotesize Total Size (KB)},
%                 \myScatterClasses,
%                 width=\textwidth,
%                 height=0.6\textwidth,
%                 xmode=log,
%                 ymode=log,
%                 xtick={1024, 16384, 524288},
%                 xticklabels={$2^{10}$, $2^{14}$, $2^{19}$},
%                 % title={Payload = 256 B},
%             ]
%             \pgfplotstableread[col sep=comma]{data/Total_Size_vs_N_s2048.csv}\datatable
%             \addplot[
%                 scatter, 
%                 only marks,
%                 scatter src=explicit symbolic
%             ] table [meta=mode, x=N, y=Total Size (KB)] {\datatable};
%             \end{axis}
%         \end{tikzpicture}
%         \caption{Payload = 256 B}
%     \end{subfigure}
%     \hfill
%     \begin{subfigure}{0.3\textwidth}
%         \centering
%         \begin{tikzpicture}
%             \begin{axis}[
%                 xlabel={\footnotesize Number of Rows},
%                 ylabel={\footnotesize Total Size (KB)},
%                 \myScatterClasses,
%                 width=\textwidth,
%                 height=0.6\textwidth,
%                 xmode=log,
%                 ymode=log,
%                 xtick={1024, 16384, 524288},
%                 xticklabels={$2^{10}$, $2^{14}$, $2^{19}$},
%             ]
%             \pgfplotstableread[col sep=comma]{data/Total_Size_vs_N_s262144.csv}\datatable
%             \addplot[
%                 scatter, 
%                 only marks,
%                 scatter src=explicit symbolic
%             ] table [meta=mode, x=N, y=Total Size (KB)] {\datatable};
%             \end{axis}
%         \end{tikzpicture}
%         \caption{Payload = 32 KB}
%     \end{subfigure}
%     \caption{Communication cost and Server Online Runtime as a function of the number of rows}
%     \label{fig:zippir-costs-per-num-rows}
% \end{figure*}


\subsection{Evaluating PIR with No Setup}
We compare \protocol{} with other PIR protocols without setup by measuring communication and computation costs for different database sizes, with the goal of retrieving at least one bit.
We report our measurements in \Cref{fig:eval-pir-comm-comp} in four graphs, corresponding to four different database sizes.
For each database size, we include points corresponding to related work such as HintlessPIR, WhisPIR, and YPIR. Other PIR protocols have communication costs that are much higher than these works.

From these graphs, we can make the following observations.
\protocol{} offers a low communication alternative to existing work, such that in some configurations, we require less than 25\% of the total communication cost of related work.
However, this low communication comes with higher computation costs, which can be addressed in future work.

% Our results show that \protocol{} has less than 100 KB communication cost for any configuration, which is less than 50\% of any related work.
% Hence, \protocol{} is the least communication-intensive PIR protocol in the literature.
% Regarding runtime, \protocol{} is slower than related work, particularly those that allow hints and offline communication, but this is an expected consequence.

\begin{figure*}[t]
    \centering
    \begin{tikzpicture}
        \begin{axis}[
            height=0.2\columnwidth,
            width=2*\columnwidth, % Ensure this axis is as wide as both subfigures together
            hide axis,
            xmin=0,
            xmax=1,
            ymin=0,
            ymax=1,
            legend columns=-1, % Horizontal legend
            legend style={/tikz/every even column/.append style={column sep=0.5cm}},
            legend to name=namedtradeoff, % Define legend name for referencing
        ]
            \addlegendimage{color=\protocolsinglecolor, mark size=2pt, mark=*}
            \addlegendentry{\protocolsingle{}}
    
            % \addlegendimage{color=\protocolbatchedcolor, mark size=2pt, mark=*}
            % \addlegendentry{\protocolbatched{}}
    
            \addlegendimage{color=\hintlesscolor, mark size=2pt, mark=*}
            \addlegendentry{HintlessPIR}
    
            \addlegendimage{color=\whispircolor, mark size=2pt, mark=*}
            \addlegendentry{WhisPIR}

            \addlegendimage{color=\ypircolor, mark size=2pt, mark=*}
            \addlegendentry{YPIR}
            
        \end{axis}
        \end{tikzpicture}
    
        \ref{namedtradeoff} % Referencing the named legend

    \begin{subfigure}{0.24\textwidth}
        \centering
        \begin{tikzpicture}
            \begin{axis}[
                xlabel={\footnotesize Server Online Time (s)},
                ylabel={\footnotesize Total Size (KB)},
                \myScatterClasses,
                width=1.1\textwidth,
                height=0.75\textwidth,
                ymode=log,
                log basis y=2,   
                % xmode=log,
                % xmax=100,
                ymax=2000,
                tick label style={font=\footnotesize}, % or \footnotesize, \scriptsize, \tiny
            ]
            \pgfplotstableread[col sep=comma]{data/Total_Size_vs_Server_Online_Time_2147483648x1=256.00MB_all.csv}\datatable
            \addplot[
                scatter, 
                only marks,
                scatter src=explicit symbolic
            ] table [meta=mode, x=Server Online Time, y=Total Size (KB)] {\datatable};
            % \addplot[
            %     scatter,
            %     only marks,
            %     mark=*, mark size=\whispirsize, \whispircolor
            % ] coordinates {
            %     (1.691,564)
            %     (1.35,620)
            %     (1.037,878)
            %     (1.662,1044)
            % };
            
            \addplot[
                scatter,
                only marks,
                mark=*, mark size=\hintlesssize, \hintlesscolor
            ] coordinates {
                (0.575,1260)
            };

            \end{axis}
        \end{tikzpicture}
        \caption{$|\db|=$ 0.25 GB}
    \end{subfigure}
    \hfill
    \begin{subfigure}{0.24\textwidth}
        \centering
        \begin{tikzpicture}
            \begin{axis}[
                xlabel={\footnotesize Server Online Time (s)},
                \myScatterClasses,
                width=1.2\textwidth,
                height=0.75\textwidth,
                ymode=log,
                log basis y=2,   
                % xmode=log,
                % xmax=100,
                ymax=2000,
                tick label style={font=\footnotesize}, % or \footnotesize, \scriptsize, \tiny
            ]
            \pgfplotstableread[col sep=comma]{data/Total_Size_vs_Server_Online_Time_4294967296x1=512.00MB_all.csv}\datatable
            \addplot[
                scatter, 
                only marks,
                scatter src=explicit symbolic
            ] table [meta=mode, x=Server Online Time, y=Total Size (KB)] {\datatable};
            
            \addplot[
                scatter,
                only marks,
                mark=*, mark size=\hintlesssize, \hintlesscolor
            ] coordinates {
                (0.768,1639)
            };

            \end{axis}
        \end{tikzpicture}
        \caption{$|\db|=$ 0.5 GB}
    \end{subfigure}
    \hfill  
    \begin{subfigure}{0.24\textwidth}
        \centering
        \begin{tikzpicture}
            \begin{axis}[
                xlabel={\footnotesize Server Online Time (s)},
                \myScatterClasses,
                width=1.2\textwidth,
                height=0.75\textwidth,
                ymode=log,
                log basis y=2,   
                % xmode=log,
                xmax=100,
                ymax=2450,
                tick label style={font=\footnotesize}, % or \footnotesize, \scriptsize, \tiny
            ]
            \pgfplotstableread[col sep=comma]{data/Total_Size_vs_Server_Online_Time_8589934592x1=1024.00MB_all.csv}\datatable
            \addplot[
                scatter, 
                only marks,
                scatter src=explicit symbolic
            ] table [meta=mode, x=Server Online Time, y=Total Size (KB)] {\datatable};
            \addplot[
                scatter,
                only marks,
                mark=*, mark size=\whispirsize, \whispircolor
            ] coordinates {
                (1.691,564)
                (1.077,620)
                (1.037,878)
                (0.814,1044)
            };
            
            \addplot[
                scatter,
                only marks,
                mark=*, mark size=\hintlesssize, \hintlesscolor
            ] coordinates {
                (1.03,2131)
            };

            \addplot[
                scatter,
                only marks,
                mark=*, mark size=\ypirsize, \ypircolor
            ] coordinates {
                (0.428,858)
            };
            \end{axis}
        \end{tikzpicture}
        \caption{$|\db|=$ 1 GB}
    \end{subfigure}
    \hfill
    \begin{subfigure}{0.24\textwidth}
        \centering
        \begin{tikzpicture}
            \begin{axis}[
                xlabel={\footnotesize Server Online Time (s)},
                \myScatterClasses,
                width=1.2\textwidth,
                height=0.75\textwidth,
                ymode=log,
                log basis y=2,   
                ymax=2500,
                tick label style={font=\footnotesize}, % or \footnotesize, \scriptsize, \tiny
            ]
            \pgfplotstableread[col sep=comma]{data/Total_Size_vs_Server_Online_Time_68719476736x1=8192.00MB_all.csv}\datatable
            \addplot[
                scatter, 
                only marks,
                scatter src=explicit symbolic
            ] table [meta=mode, x=Server Online Time, y=Total Size (KB)] {\datatable};
            \addplot[
                scatter,
                only marks,
                mark=*, mark size=\whispirsize, \whispircolor
            ] coordinates {
                (8.412,852)
                (7.432,963)
                (5.984,1183)
                (6.254,2047)
            };            

            \addplot[
                scatter,
                only marks,
                mark=*, mark size=\hintlesssize, \hintlesscolor
            ] coordinates {(2.3, 2128)};

            \addplot[
                scatter,
                only marks,
                mark=*, mark size=\ypirsize, \ypircolor
            ] coordinates {(0.992, 1512)};
            
            \end{axis}
        \end{tikzpicture}
        \caption{$|\db| = $ 8 GB}
    \end{subfigure}
    \caption{Communication cost vs. server online runtime for several database sizes. The red, orange, and yellow points correspond to ZipPIR with different parameters, as described \Cref{fig:eval-pir-comm-comp}.}
    \label{fig:eval-pir-comm-comp}
\end{figure*}


% \begin{table}[]
%     \centering
%     \label{tab:eval-pir}    
%     \caption{
%         Performance of PIR protocols for ephemeral clients.
%         We use \protocolsingle{} for small payloads and \protocolbatched{} for the large payloads.
%         % The top two sections are cases where the payload size is small (and we use \protocol{} with the packed compression key).
%         % In this case, the packed compression key is only 8 KB of the query.
%         % The two lower sections are cases with a larger payload and we use \protocol{} with batched compression.
%         % The compression key is about 400 KB in this case.
%     }
%     \begin{tabular}{c|c|c|c|c}
%     \toprule
%      Protocol & & HintlessPIR & WhisPIR & \protocol{} \\
%      & & \cite{liHintlessSingleServerPrivate2023} & \cite{castroWhisPIRStatelessPrivate2024} &  \\
%     \midrule
%         & Query & 399 KB & & 49 KB \\
%         & Response & 151 KB & & 48 KB \\
%         $2^{20}\times$ 8B
%         & Total & 450 KB & $\approx$ 300 KB & 97 KB \\
%         \cline{2-5}
%         (8 MB)
%         & Offline & 3.11 s & ?? & 3.8 s \\
%         & Online & 271 ms & ?? & 12 s \\
%     \midrule
%         & Query & 480 KB & ?? & 175 KB \\
%         & Response & 1159 KB & ?? & 48 KB \\
%         $2^{26}\times$ 8B
%         & Total & 1260 KB & 300 & 223 KB \\
%         \cline{2-5}
%         (537 MB)
%         & Offline & 93.58 s & & 65 s \\
%         & Online & 768 ms & & 188 s \\
%     \midrule
%     \midrule
%         & Query & 453 KB & & 667 KB \\
%         & Response & 807 KB & & 15 KB \\
%         $2^{20}\times$ 256 B
%         & Total & 1260 KB & $\approx$ 300 & 682 KB \\
%         \cline{2-5}
%         (268 MB)
%         & Offline & 51.57s & \\
%         & Online & 575 ms & \\
%     \midrule
%         & Query & 518~KB & & 676 KB \\
%         & Response & 1610~KB & & 826 KB \\
%         $2^{18}\times$ 32~KB
%         & Total & 2128~KB & $\approx$ 300 & 1502 KB \\
%         \cline{2-5}
%         (8.59 GB)
%         & Offline & 2128 s & & - \\
%         & Online & 2309 ms & & - \\
%     \bottomrule
%     \end{tabular}
% \end{table}


% \newcommand{\plotheight}{0.4}
% \begin{figure*}[t]
%     \begin{tikzpicture}
%         \begin{axis}[
%             height=0.2\columnwidth,
%             width=\linewidth,
%             hide axis,
%             xmin=0,
%             xmax=1,
%             ymin=0,
%             ymax=1,
%             legend columns=-1, % Horizontal legend
%             legend style={/tikz/every even column/.append style={column sep=0.5cm}},
%             legend to name=named, % Define legend name for referencing
%         ]
%             % \addlegendimage{color=blue, mark size=1pt, mark=*}
%             % \addlegendentry{Comm. Optimal}
    
%             % \addlegendimage{color=red, mark size=1pt, mark=*}
%             % \addlegendentry{Comp. Optimal}
            
%         \end{axis}
%     \end{tikzpicture}
%     \centering
%     \ref{named} % Referencing the named legend
%     \begin{subfigure}[b]{\columnwidth}
%         \begin{tikzpicture}[]
%             \begin{axis}[
%                 name=single_comm,
%                 at={(0,0)},
%                 xlabel = {\# of Rows},
%                 ylabel = {Communication (KB)},
%                 height = \plotheight \columnwidth,
%                 width = \columnwidth,
%                 xmode=log,
%                 ymode=log,
%                 log basis x={2},
%                 % ymax=500,
%                 legend pos=north west
%             ]
    
%                 \addplot [color=blue, mark size=1pt, mark=*] table [
%                     x={Items},
%                     y={Total Size (KB)},
%                     col sep=comma
%                 ] {data/results_single_comm.csv};
%                 % \addlegendentry{Comm Optimal}
%                 \addplot [color=red, mark size=1pt, mark=*] table [
%                     x={Items},
%                     y={Total Size (KB)},
%                     col sep=comma
%                 ] {data/results_single_comp.csv};
%                 % \addlegendentry{Comp Optimal}
%             \end{axis}    
%         \end{tikzpicture}
%         \caption{Small Payload (8B)}
%     \end{subfigure}
%     ~
%     \begin{subfigure}[b]{\columnwidth}
%         \begin{tikzpicture}[]
%             \begin{axis}[
%                 name=batched_comm,
%                 at={(0,0)},
%                 xlabel = {\# of Rows},
%                 ylabel = {Communication (KB)},
%                 height = \plotheight \columnwidth,
%                 width = \columnwidth,
%                 xmode=log,
%                 ymode=log,
%                 log basis x={2}
%             ]
    
%                 \addplot [color=blue, mark size=1pt, mark=*] table [
%                     x={Items},
%                     y={Total Size (KB)},
%                     col sep=comma
%                 ] {data/results_batched_comm.csv};
%                 \addplot [color=red, mark size=1pt, mark=*] table [
%                     x={Items},
%                     y={Total Size (KB)},
%                     col sep=comma
%                 ] {data/results_batched_comp.csv};
%             \end{axis}    
%         \end{tikzpicture}
%         \caption{Large Payload (32KB)}
%     \end{subfigure}
% \begin{subfigure}[b]{\columnwidth}
%     \begin{tikzpicture}[]
%         \begin{axis}[
%             name=single_comp,
%             at={(0,0)},
%             xlabel = {\# of Rows},
%             ylabel = {Server Time (s)},
%             height = \plotheight \columnwidth,
%             width = \columnwidth,
%             xmode=log,
%             ymode=log,
%             log basis x={2},
%             % ymax=200
%         ]

%             \addplot [color=blue, mark size=1pt, mark=*] table [
%                 x={Items},
%                 y expr=\thisrow{Server Online Time}/1000000,
%                 col sep=comma
%             ] {data/results_single_comm.csv};
%             \addplot [color=red, mark size=1pt, mark=*] table [
%                 x={Items},
%                 y expr=\thisrow{Server Online Time}/1000000,
%                 col sep=comma
%             ] {data/results_single_comp.csv};
%         \end{axis}    
%     \end{tikzpicture}
%     \caption{Small Payload (8B)}
% \end{subfigure}
% ~
% \begin{subfigure}[b]{\columnwidth}
%     \begin{tikzpicture}[]
%         \begin{axis}[
%             name=batched_comp,
%             at={(0,0)},
%             xlabel = {\# of Rows},
%             ylabel = {Server Time (s)},
%             height = \plotheight \columnwidth,
%             width = \columnwidth,
%             xmode=log,
%             ymode=log,
%             log basis x={2},
%             % ymax=200
%         ]

%             \addplot [color=blue, mark size=1pt, mark=*] table [
%                 x={Items},
%                 y expr=\thisrow{Server Online Time}/1000000,
%                 col sep=comma
%             ] {data/results_batched_comm.csv};
%             \addplot [color=red, mark size=1pt, mark=*] table [
%                 x={Items},
%                 y expr=\thisrow{Server Online Time}/1000000,
%                 col sep=comma
%             ] {data/results_batched_comp.csv};
%         \end{axis}    
%     \end{tikzpicture}
%     \caption{Large Payload (32KB)}
% \end{subfigure}
% \caption{Communication and computation costs of \protocol{} as a function of the number of rows in the database. On the left, we have small 8B payloads so we use $\protocol{}_C$ and on the right, we have large 32KB payloads so we use \protocolbatched{}.}
% \label{fig:zippir-costs-per-num-rows}
% \end{figure*}


\section{Related Work on PIR}
Computational PIR (CPIR) protocols follow one of three approaches:
1) the server gives a \textit{hint} to the client 2) the client sends cryptographic keys to the server 3) there is no setup, hint, or apriori key exchange.
We describe each approach briefly, the advantages and disadvantages of each approach and list related work.

\subsection{Hint-based PIR}
One approach is for the server to generate a database-dependant hint which is transmitted to the client before the query is issued.
The objective of the hint is to speed up subsequent queries.
SimplePIR~\cite{henzingerOneServerPrice2023} and FrodoPIR~\cite{davidsonFrodoPIRSimpleScalable2023} are two recent works that propose a PIR protocol based on LWE with a client-independent hint.
The hint size is $O(\sqrt{N}n)$ for $N$ database rows and LWE dimension of $n$.
All clients use the same hint which helps respond quickly to PIR queries and achieve very high throughput (up to 10 GB/s).
However, the hint is a high upfront cost (100 MB for a 1 GB database) and must be recalculated and redistributed to the clients every time the database is updated.
The authors show how to update the client hint with a small amount of communication.
DoublePIR extends SimplePIR so that the hint that must be sent to the client is smaller but the overall throughput is less.
In recent work, Henzinger et al. used an improved version of SimplePIR for a private web search application to avoid sending a large hint to the client in a method similar to our work, but with the use of an RLWE-based cryptosystem~\cite{henzingerPrivateWebSearch2023}.

\subsection{PIR with Setup}
Another category of works assumes auxiliary information is sent before the start of the protocol, usually in the form of cryptographic keys.
The cost of sending these keys is amortized over many queries but requires per-client storage on the server.
While this approach is good if there is an established connection between the client and server, it is a high upfront cost.
Moreover, the public keys allow the server to correlate different queries that the client makes so it is not suitable to combine with anonymity networks.
Henzinger et al. also showed that such long-term persistent keys expose the client to state-recovery attacks that could compromise past queries.
Despite these disadvantages, the online time in such protocols is very small and if sufficient queries are made, the runtime and communication cost of setup is amortized.


Works that follow this model include SealPIR~\cite{angelPIRCompressedQueries2018a}, MulPIR~\cite{Ali2019CommunicationComputationTI}, OnionPIR~\cite{mugheesOnionPIRResponseEfficient2021}, Constant-weight PIR~\cite{mahdaviConstantweightPIRSingleround2022}, Pantheon~\cite{ahmadPantheonPrivateRetrieval2022}, FastPIR~\cite{ahmadAddraMetadataprivateVoice2021}, Spiral (and its variants)~\cite{menonSPIRALFastHighRate2022}, and SparsePIR~\cite{patelDonBeDense2023}.

\subsection{PIR without Hints or Setup}
The previous approaches require an established connection between a client and server to amortize the cost of the hint or cryptographic keys across many queries.
For applications where the client only performs a few queries, previous solutions are impractical
Naively applying previous solutions would require the cryptographic keys to be sent as part of the query, resulting in large queries.
Hence, the third approach is to design a PIR protocol that does not require precomputed hints or large cryptographic keys.
Our work also falls in this category and achieves the lowest total communication cost of all PIR protocols in the literature.

HintlessPIR~\cite{liHintlessSingleServerPrivate2023} is a protocol which expands on SimplePIR to remove the need to send the hint.
In short, HintlessPIR retrieves the necessary row of the hint from the server, essentially delegating the step which requires the hint to the server.
YPIR~\cite{menonYPIRHighThroughputSingleServer2024} also takes a similar approach and retrieves the necessary row of the hint using high-rate RLWE ciphertexts.

WhisPIR~\cite{castroWhisPIRStatelessPrivate2024}, on the other hand, expands on the protocols with setup and aims to reduce the number of required cryptographic keys.
The authors propose a PIR protocol focused on being stateless, i.e., working well for ephemeral clients and having low communication.
Two main contributions of WhisPIR are
1) modifications to reduce the number of cryptographic keys that are required 2) not performing relinearization after homomorphic multiplications.
Using these techniques along with a careful choice of parameters, WhisPIR achieves a communication cost that is smaller than related work.


\section{Conclusion}
In this work, we proposed a method for reducing server response sizes in client-server protocols using homomorphic encryption.
Specifically, we showed how to compress LWE ciphertexts, sent from the server to the client, up to 90\% for single ciphertexts and 99\% for many ciphertexts.
Using our compression technique, we proposed \protocol{}, a low-communication PIR protocol, suitable for ephemeral clients and low-latency networks. 
We evaluated both our compression technique and \protocol{} and showed that \protocol{} can query large databases with only 200-500KB of communication.

% In this work, we proposed techniques for reducing server response sizes, generated using homomorphic encryption.
% We introduce and evaluate methods to compress commonly used ciphertexts in the literature, LWE and RLWE ciphertexts.
% Our technique is predicated on conversion from lattice-based FHE schemes with large ciphertexts to additive schemes with smaller ciphertexts.
% Within the additive scheme, we perform part of the decryption function which drastically reduces the size of the ciphertext.
% We benchmark multiple methods to perform this operation and present the runtime and compression gains that result from this technique.

% Our technique allows more flexibility compared to existing work as it can be plugged into any existing protocol which operates over LWE ciphertexts.
% We provide a detailed taxonomy of related work to distinguish our technique from existing work in the literature.
% We also describe how our technique can be integrated into existing protocols and the compression it can offer.

\bibliographystyle{plain}
\bibliography{references}

\section{Appendix for Proofs}

\paragraph{Proof of Theorem \ref{thm:main}.}

\begin{proof}
\label{proof:main}
Our proof has two steps. In Step 1, we will show that SimCLR is equivalent to minimizing the cross entropy loss defined in Eqn.~(\ref{eqn:cross-entropy}). 
In Step 2, we will show  that minimizing the cross-entropy loss 
is equivalent to spectral clustering on $\bfpi$. 
Combining the two steps together, we have proved our theorem. 

\textbf{Step 1: } SimCLR is equivalent to minimizing the cross entropy loss.

The cross-entropy loss takes expectation over 
$\bfW_\bfX\sim \mathbb{P}(\cdot ; \bfpi)$, 
which means $\bfW_\bfX$ has exactly one non-zero entry in each row $i$. By Lemma~\ref{lem:multinomial}, we know every row $i$ of $\bfW_\bfX$ is independent of other rows. Moreover, 
$\bfW_{\bfX,i}\sim \mathcal{M}(1, \bfpi_i/\sum_j \bfpi_{i,j})=\mathcal{M}(1, \bfpi_i)$, because $\bfpi_i$ itself is a probability distribution.
Similarly, we know $\bfW_\bfZ$ also has the row-independent property by sampling over $\mathbb{P}(\cdot;\bfK_\bfZ)$.
Therefore, by Lemma~\ref{lem:cross_split}, we know Eqn.~(\ref{eqn:cross-entropy}) is equivalent to:
\[
 -\sum_{i=1}^n \mathbb{E}_{\bfW_{\bfX,i}}[\log \mathbb{P}(\bfW_{\bfZ,i}=\bfW_{\bfX,i};\bfK_\bfZ)],
\]

This expression takes expectation over $\bfW_{\bfX,i}$ for the given row $i$. Notice that 
$\bfW_{\bfX,i}$ has exactly one non-zero entry, which equals $1$ (same for $\bfW_{\bfZ,i}$). 
As a result
we expand the above expression to be:
\begin{equation}
 -\sum_{i=1}^n \sum_{j\neq i} \Pr(\bfW_{\bfX,i,j}=1)\log \Pr(\bfW_{\bfZ,i,j}=1).
\label{eqn:detailed-expansion}    
\end{equation}


By Lemma~\ref{lem:multinomial}, $\Pr(\bfW_{\bfZ,i,j}=1)=\bfK_{\bfZ,i,j}/\|\bfK_{\bfZ,i}\|_1$ for $j\neq i$. Recall that $\bfK_\bfZ=(k(\bfZ_i-\bfZ_j))_{(i,j)\in[n]^2}$, which means 
$\bfK_{\bfZ,i,j}/\|\bfK_{\bfZ,i}\|_1=\frac{\exp(-\|\bfZ_i-\bfZ_j\|^2/{2\tau})}{\sum_{k\neq i}
\exp(-\|\bfZ_i-\bfZ_k\|^2/{2\tau})
}$ for $j\neq i$, when $k$ is the Gaussian kernel with variance $\tau$. 

Notice that $\bfZ_i=f(\bfX_i)$, so we know
\begin{equation}
-\log \Pr(\bfW_{\bfZ,i,j}=1)=
-\log \frac{\exp(-\|f(\bfX_i)-f(\bfX_j)\|^2/{2\tau})}{\sum_{k\neq i}
\exp(-\|f(\bfX_i)-f(\bfX_k)\|^2/{2\tau}),
}
\label{eqn:infonce-equivalence}    
\end{equation}


The right hand side is exactly the InfoNCE loss defined in Eqn.~(\ref{eqn:infonce}).
Inserting Eqn.~(\ref{eqn:infonce-equivalence}) into Eqn.~(\ref{eqn:detailed-expansion}), we get the SimCLR algorithm, which first samples augmentation pairs $(i,j)$ with $\Pr(\bfW_{\bfX,i,j}=1)$ for each row $i$, and then optimize the InfoNCE loss. 

\textbf{Step 2: } minimizing the cross entropy loss 
is equivalent to spectral clustering on $\bfpi$.


By Lemma~\ref{lem:convert_to_spectral}, we may further convert the loss to 
\begin{equation}
\label{eqn:main-theorem-repul-attr}
\min_{\bfZ}
-\sum_{(i,j)\in [n]^2} \mathbf{P}_{i,j}
\log k (\bfZ_i-\bfZ_j)+\log \mathbf{R}(\bfZ).
\end{equation}
Since $k$ is the Gaussian kernel, this reduces to \[
\min_\bfZ \mathrm{tr}(\bfZ^\top \mathbf{L}(\bfpi) \bfZ)
+\log \mathbf{R}(\bfZ),
\]

where we use the fact that $\mathbb{E}_{\bfW_\bfX\sim \mathbb{P}(\cdot; \bfpi)}[\mathbf{L}(\bfW_\bfX)]
=\mathbf{L}(\bfpi)
$, because the Laplacian operator is linear and $
\mathbb{E}_{\bfW_\bfX\sim \mathbb{P}(\cdot; \bfpi)}(\bfW_\bfX)=\bfpi
$.
\end{proof}

\paragraph{Proof of Theorem \ref{thm:clip}.}
\begin{proof}
Since $\bfW_\bfX\sim \mathbb{P}(\cdot;\bfpi_{\mathbf{A}, \mathbf{B}})$, we know 
$\bfW_\bfX$ has exactly one non-zero entry in each row, denoting the pair that got sampled. 
A notable difference compared to the previous proof is we now have $n_\mathcal{A}+n_\mathcal{B}$ objects in our graph. CLIP deals with this by taking a mini-batch of size $2N$, 
such that $n_\mathcal{A}=n_\mathcal{B}=N$, and adding the $2N$ InfoNCE losses together. We label the objects in $\mathcal{A}$ as $[n_\mathcal{A}]$, and the objects in $\mathcal{B}$ as $\{n_\mathcal{A}+1, \cdots, n_\mathcal{A}+n_\mathcal{B}\}$. 

Notice that $\bfpi_{\mathbf{A}, \mathbf{B}}$ is a bipartite graph, so the edges of objects in $\mathcal{A}$ will only connect to object in $\mathcal{B}$ and vice versa. We can define the similarity matrix in $\cZ$ as $\bfK_\bfZ$, 
where $\bfK_\bfZ(i, j+n_\mathcal{A})=\bfK_\bfZ(j+n_\mathcal{A},i)= k(\bfZ_i-\bfZ_j)$ for $i\in [n_\mathcal{A}], j\in [n_\mathcal{B}]$, and otherwise we set $\bfK_\bfZ(i,j)=0$. 
The rest is same as the previous proof. 
\end{proof}

\paragraph{Proof of Theorem \ref{thm:exponential}.}

\begin{proof}
\label{proof:exponential}
Since the objective function consists of a linear term combined with an entropy regularization, which is a strongly concave function, the maximization problem is a convex optimization problem. Owing to the implicit constraints provided by the entropy function, the problem is equivalent to having only the equality constraint. We then introduce the Lagrangian multiplier $\lambda$ and obtain the following relaxed problem:

$$
\widetilde{E}(\boldsymbol{\alpha})=\psi_{1}-\sum_{i=1}^n \alpha_{i} \psi_{i}+\tau \sum_{i=1}^n \alpha_{i}\log \alpha_{i}+\lambda\left(\boldsymbol{\alpha}^{\top} \mathbf{1}_n-1\right).
$$

As the relaxed problem is unconstrained, taking the derivative with respect to $\alpha_{i}$ yields

$$
\frac{\partial \widetilde{E}(\boldsymbol{\alpha})}{\partial \alpha_{i}}=-\psi_{i}+\tau\left(\log \alpha_{i}+\alpha_{i} \frac{1}{\alpha_{i}}\right)+\lambda=0.
$$

Solving the above equation implies that $\alpha_{i}$ takes the form
$
\alpha_{i}=\exp \left(\frac{1}{\tau} \psi_{i}\right) \exp \left(\frac{-\lambda}{\tau}-1\right).
$ Since $\alpha_{i}$ lies on the probability simplex, the optimal $\alpha_{i}$ is explicitly given by
$
\alpha^{*}_{i}=\frac{\exp \left(\frac{1}{\tau} \psi_{i}\right)}{\sum_{i^{\prime}=1}^n \exp \left(\frac{1}{\tau} \psi_{i^{\prime}}\right)} .
$ Substituting the optimal point into the objective function, we obtain
$$
\begin{aligned}
E\left(\boldsymbol{\alpha}^*\right)  &=\psi_1-\sum_{i=1}^n \frac{\exp \left(\frac{1}{\tau} \psi_{i}\right)}{\sum_{i^{\prime}=1}^n \exp \left(\frac{1}{\tau} \psi_{i^{\prime}}\right)} \psi_{i}+\tau \sum_{i=1}^n \frac{\exp \left(\frac{1}{\tau} \psi_{i}\right)}{\sum_{i^{\prime}=1}^n \exp \left(\frac{1}{\tau} \psi_{i^{\prime}}\right)}\log \frac{\exp \left(\frac{1}{\tau} \psi_{i}\right)}{\sum_{i^{\prime}=1}^n \exp \left(\frac{1}{\tau} \psi_{i^{\prime}}\right)} \\
& =\psi_1 - \tau \log \left(\sum_{i=1}^n \exp \left(\frac{1}{\tau} \psi_{i}\right)\right).
\end{aligned}
$$
Thus, the Lagrangian dual function is given by
\begin{equation*}
-E\left(\boldsymbol{\alpha}^*\right)= -\tau \log \frac{\exp \left(\frac{1}{\tau} \psi_{1}\right)}{\sum_{i=1}^n \exp \left(\frac{1}{\tau} \psi_{i}\right)}.\qedhere
\end{equation*}
\end{proof}



\section{More on Experiments} \label{section: experiment_details}

\paragraph{CIFAR-10 and CIFAR-100} CIFAR-10 ~\citep{krizhevsky2009learning} and CIFAR-100 ~\citep{krizhevsky2009learning} are well-known classic image classification datasets. Both CIFAR-10 and CIFAR-100 contain a total of 60k $32 \times 32$ labeled images of different classes, with 50k for training and 10k for testing. CIFAR-10 is similar to CIFAR-100, except there are 10 different classes in CIFAR-10 and 100 classes in CIFAR-100.

\paragraph{TinyImageNet} TinyImageNet ~\citep{le2015tiny} is a subset of ImageNet ~\citep{deng2009imagenet}. There are 200 different object classes in TinyImageNet, with 500 training images, 50 validation images, and 50 test images for each class. All the images in TinyImageNet are colored and labeled with a size of $64 \times 64$.

\textbf{Pseudo-code.} Algorithm \ref{alg:Training Procedure} presents the pseudo-code for our empirical training procedure.

\begin{algorithm}[!htbp]
\caption{Training Procedure}
\label{alg:Training Procedure}
\begin{algorithmic}[1]
\REQUIRE trainable encoder network $f$, batch size $N$, augmentation strategy \textit{aug}, loss function $L$ with hyperparameters \textit{args}
\FOR {sampled minibatch ${x_i}_{i=1}^N$}
\FORALL{$i \in { 1, ..., N }$}
\STATE draw two augmentations $t_i = \textit{aug}\left(x_i\right) $, $t_i' = \textit{aug}\left(x_i\right) $
\STATE $z_i = f\left(t_i\right)$, $z_i' = f\left(t_i'\right)$
\ENDFOR
\STATE compute loss $\mathcal{L} = L(N, z, z', \textit{args})$
\STATE update encoder network $f$ to minimize $\mathcal{L}$
\ENDFOR
\STATE \textbf{Return} encoder network $f$
\end{algorithmic}
\end{algorithm}

We also provide the pseudo-code for our core loss function used in the training procedure in Algorithm \ref{alg:Core loss}. The pseudo-code is almost identical to SimCLR's loss function, with the exception of an extra parameter $\gamma$.

\begin{algorithm}[!htbp]
\caption{Core loss function $\mathcal{C}$}
\label{alg:Core loss}
\begin{algorithmic}[1]
\REQUIRE batch size $N$, two encoded minibatches $z_1, z_2$, $\gamma$, temperature $\tau$
\STATE $z = \textit{concat}\left(z_1, z_2\right)$
\FOR {$i \in {1, ..., 2N }, j \in {1, ..., 2N}$ }
\STATE $s_{i,j} = \Vert z_i - z_j \Vert_2^{\gamma}$
\ENDFOR
\STATE \textbf{define} $l(i, j)$ \textbf{as} $l(i, j) = - \log \frac{exp\left(s_{i,j}/\tau \right)}{\sum_{k=1}^{2N} \mathbf{1}{[k \ne i]} exp\left(s{i, j} / \tau \right)} $
\STATE \textbf{Return} $\frac{1}{2N} \sum_{k=1}^N\left[l(i, i+N) + l(i+N, i)\right]$
\end{algorithmic}
\end{algorithm}

Utilizing the core loss function $\mathcal{C}$, we can define all kernel loss functions used in our experiments in Table \ref{table: loss definition}. For all $z_i \in z$ with even dimensions $n$, we define $z_{L_i} = z_i\left[0:n/2\right]$ and $z_{R_i} = z_i\left[n/2:n\right]$.

\begin{table}[ht]
\centering
\begin{tabular}{{@{}l|l@{}}}
Kernel  &  Loss function \\ \midrule
Laplacian & $\mathcal{C}\left(N, z, z', \gamma=1, \tau\right)$\\ \midrule
Sum       & $\lambda * \mathcal{C}\left(N, z, z', \gamma=1, \tau_1\right) + (1-\lambda) * \mathcal{C}\left(N, z, z', \gamma=2, \tau_2\right)$  \\ \midrule
Concatenation Sum&$\lambda * \mathcal{C}\left(N, z_L, z'_L, \gamma=1, \tau_1\right) + (1-\lambda) * \mathcal{C}\left(N, z_R, z'_R, \gamma=2, \tau_2\right)$\\ \midrule
$\gamma = 0.5$ & $\mathcal{C}\left(N, z, z', \gamma=0.5, \tau\right)$          \\ 

\end{tabular}

\caption{Definition of kernel loss functions in our experiments}
\label {table: loss definition}
\end{table}

\textbf{Baselines.} We reproduce the SimCLR algorithm using PyTorch Lightning~\citep{PytorchLightning}.

\textbf{Encoder details.}
The encoder $f$ consists of a backbone network and a projection network. We employ ResNet50~\citep{ResNet} as the backbone and a 2-layer MLP (connected by a batch normalization~\citep{ioffe2015batch} layer and a ReLU \cite{nair2010rectified} layer) with hidden dimensions 2048 and output dimensions 128 (or 256 in the concatenation kernel case).

\textbf{Encoder hyperparameter tuning.}
For each encoder training case, we randomly sample 500 hyperparameter groups (sample details are shown in Table \ref{table: Hyperparameter sample}) and train these samples simultaneously using Ray Tune ~\citep{RayTune}, with the ASHA scheduler~\citep{li2018massively}. Ultimately, the hyperparameter group that maximizes the online validation accuracy (integrated in PyTorch Lightning) within 5000 validation steps is chosen for the given encoder training case.

\begin{table}[ht]
\centering

\begin{tabular}{@{}l|l|l@{}}
\midrule
Hyperparameter  & Sample Range & Sample Strategy \\ \midrule
start learning rate & $\left[10^{-2}, 10\right]$ & log uniform \\ \midrule
$\lambda$       & $\left[0, 1\right]$ & uniform \\ \midrule
$\tau$, $\tau_1$, $\tau_2$ & $\left[0, 1\right]$ & log uniform \\ \midrule
\end{tabular}

\caption{Hyperparameters sample strategy}
\label {table: Hyperparameter sample}
\end{table}

\textbf{Encoder training.} 
We train each encoder using the LARS optimizer~\citep{LARSOptimizer}, LambdaLR Scheduler in PyTorch, momentum 0.9, weight decay $10^{-6}$, batch size 256, and the aforementioned hyperparameters for 400 epochs on a single A-100 GPU.

\textbf{Image transformation.} The image transformation strategy, including augmentation, is identical to the default transformation strategy provided by PyTorch Lightning.

\textbf{Linear evaluation.}
The linear head is trained using the SGD optimizer with a cosine learning rate scheduler, batch size 64, and weight decay $10^{-6}$ for 100 epochs. The learning rate starts at $0.3$ and ends at $0$.

\textbf{Moco Experiments.} We also tested our method based on MoCo~\citep{he2019moco}. The results are summarized in Table \ref{tab:results-moco}. Here we choose ResNet18~\citep{ResNet} as the backbone and set a temperature of $0.1$ as default. For our simple sum kernel, we set $\lambda=0.8$. The results show that our method outperforms the original MoCo method.

\begin{table}[thb]
\centering
\caption{MoCo Experiment Results on CIFAR-10 and CIFAR-100.}
\label{tab:results-moco}
\resizebox{\textwidth}{!}{%
\begin{tabular}{@{}c|ccc|ccc@{}}
\toprule
\multirow{3}{*}{Method} & \multicolumn{3}{c|}{CIFAR-10} & \multicolumn{3}{c}{CIFAR-100} \\ \cmidrule(lr){2-4} \cmidrule(lr){5-7} 
                        & 200 epochs & 400 epochs    & 1000 epochs   & 200 epochs & 400 epochs & 1000 epochs         \\ \midrule
MoCo (repro.)         & $76.41 \pm 0.12$    & $80.01 \pm 0.15$          & $84.45 \pm 0.08$    & $\mathbf{47.02 \pm 0.11}$ & $52.50 \pm 0.07$ & $57.62 \pm 0.15$            \\
\midrule
Laplacian Kernel        & ${78.09 \pm 0.10}$    & $\mathbf{83.85 \pm 0.09}$          & $\mathbf{88.34 \pm 0.16}$    & $46.12 \pm 0.22$   & $53.44 \pm 0.17$ & $59.10 \pm 0.14$        \\
Simple Sum Kernel & $\mathbf{78.12 \pm 0.15}$   & $83.23 \pm 0.18$ & $87.50 \pm 0.20$ & $46.65 \pm 0.06$ & $\mathbf{53.62 \pm 0.19}$ & $\mathbf{59.83 \pm 0.12}$\\
\bottomrule
\end{tabular}
}
\end{table}



\section{More Experiments on Synthetic Data}


Consider a scenario with $n$ clusters, each containing $k$ vertices. Let the probability of vertices $u$ and $v$ from the same cluster belonging to $\bfpi$ be $p$. Conversely, for vertices $u$ and $v$ from different clusters, let the probability of belonging to $\pi$ be $q$. We generate the graph $\bfpi$ randomly, based on $p$ and $q$. We experiment with values of $k=100$ and $n=6$ for ease of visualization, embedding all points in a two-dimensional space. Each vertex's initial position originates from a normal distribution. In each iteration, we sample a subgraph of $\bfpi$ uniformly, ensuring each vertex has an out-degree of $1$. We then optimize the corresponding vectors using InfoNCE loss with an SGD optimizer and iterate until convergence. Our experimental setup consists of an SGD learning rate of $1$, an InfoNCE loss temperature of $0.5$, and a batch size of $50$. We evaluate two scenarios with different $p$ and $q$ values: $p=1$, $q=0$, and $p=0.75$, $q=0.2$. The results of these experiments are visualized in Figure \ref{fig:vis-spectral-cluster}. The obtained embeddings exhibit the hallmark pattern of spectral clustering of graph $\bfpi$.

\begin{figure}[!tb]
\centering
\subfigure{
\includegraphics[width=1\textwidth]{Figures/cluster_pi.png}
\label{fig:vis-cluster}
}
\subfigure{
\includegraphics[width=1\textwidth]{Figures/noised_cluster_pi.png}
\label{fig:vis-noised-cluster}
}
\caption{Visualizations of the optimization process using InfoNCE Loss on the vectors corresponding to $\bfpi$. Points of identical color belong to the same cluster within $\bfpi$. To showcase the internal structure of $\bfpi$, we randomly select 10 vertices from each cluster to display the edge distribution of $\bfpi$.}
\label{fig:vis-spectral-cluster}
\end{figure}



\end{document}
