\documentclass[floatfix, showpacs,showkeys,aps,prd,twocolumn]{revtex4-1}

\usepackage{lineno}
\usepackage{amsmath}
\usepackage{amssymb}

\usepackage{tikz}
\usepackage{lipsum,adjustbox}
\usepackage{graphicx}
\usepackage[caption = false]{subfig}
\usepackage{dcolumn}
\usepackage{bm}
\usepackage{gensymb}

\usepackage{tabstackengine}

\setstackEOL{\cr}
\setstackgap{L}{\normalbaselineskip}

\let\oldequation\equation
\let\oldendequation\endequation
\renewcommand{\figurename}{\rm Fig.}
\renewcommand{\tablename}{\rm Table}
\renewcommand{\thetable}{\arabic{table}}


\renewenvironment{equation}
  {\linenomathNonumbers\oldequation}
  {\oldendequation\endlinenomath}

\newcolumntype{L}[1]{>{\raggedright\arraybackslash}p{#1}}
\newcolumntype{C}[1]{>{\centering\arraybackslash}p{#1}}
\newcolumntype{R}[1]{>{\raggedleft\arraybackslash}p{#1}}
\newcommand{\T}{\rule{0pt}{2.6ex}}
\newcommand{\B}{\rule[-1.2ex]{0pt}{0pt}}

\newcommand{\tagmodeone}{$K^{+}K^{-}\pi^{-}$}
\newcommand{\tagmodetwo}{$K^{+}K^{-}\pi^{-}\pi^{0}$ }
\newcommand{\tagmodethree}{$\pi^{+}\pi^{-}\pi^{-}$ }
\newcommand{\tagmodefour}{$K_{S}^{0}K^{-}$}
\newcommand{\tagmodefive}{$K_{S}^{0}K^{-}\pi^{0}$ }
\newcommand{\tagmodesix}{$K^{-}\pi^{+}\pi^{-}$}
\newcommand{\tagmodeeight}{$K_{S}^{0}K^{+}\pi^{-}\pi^{-}$}
\newcommand{\tagmodenine}{$K_{S}^{0}K^{-}\pi^{+}\pi^{-}$}
\newcommand{\tagmodeten}{$\pi^{-}\eta$}
\newcommand{\tagmodetwelve}{$\pi^{-}\eta^{\prime}_{\pi\pi\eta}$}
\newcommand{\tagmodethirteen}{$\pi^{-}\eta^{\prime}_{\gamma\rho}$}
\newcommand{\tagmodeforteen}{$\rho^{-}\eta$}
\newcommand{\tagmodefifteen}{$\rho^{-}\eta_{3\pi}$}

\usepackage{hyperref}
\hypersetup{
 pdftitle={},
 pdfauthor={},
 pdfsubject={},
 pdfkeywords={},
 pdfstartview={},
 bookmarksopen=true, breaklinks=true, debug=true,
 colorlinks=true, linkcolor=blue, citecolor=red, urlcolor=blue,
 hyperfigures=true
}

\def \b{{\cal B}}
\usepackage{relsize}
\def \babar{\mbox{\slshape B\kern-0.1em{\smaller a}\kern-0.1em
    B\kern-0.1em{\smaller a\kern-0.1em r}}}

\usepackage{diagbox}
\begin{document}

\title{\boldmath Updated measurement of the branching fraction of $D_s^+\to \tau^+\nu_\tau$ via $\tau^+\to\pi^+\bar{\nu}_\tau$}

\input{bes3author_Dec16_2022}

\date{\today}

\begin{abstract}
We report the updated study of $D_s^+\to \tau^+\nu$ via $\tau^+\to\pi^+\bar{\nu}_\tau$ with a boosted decision trees analysis, with $7.33$ fb$^{-1}$ of $e^{+}e^{-}$ collision data collected by the BESIII detector at center-of-mass energies between $4.128$ and $4.226$~GeV. The branching fraction of $D_s^+\to \tau^+\nu_\tau$ is determined to be $(5.41\pm0.17_{\rm stat}\pm0.13_{\rm syst})\%$.
The product of the $D_s^+$ decay constant $f_{D_s^+}$ and the Cabibbo-Kobayashi-Maskawa matrix element $|V_{cs}|$ is $f_{D_s^+}|V_{cs}| = (247.6\pm3.9_{\rm stat}\pm3.2_{\rm syst}\pm1.0_{\rm input})~\mathrm{MeV}$. Combining with the $|V_{cs}|$ obtained from the global fit in the Standard Model and the $f_{D_s^+}$ from the lattice quantum chromodynamics, we determine $|V_{cs}| =  0.991\pm0.015_{\rm stat}\pm0.013_{\rm syst}\pm0.004_{\rm input}$ and $f_{D_s^+} = (254.3\pm4.0_{\rm stat}\pm3.3_{\rm syst}\pm1.0_{\rm input})~\text{MeV}$, respectively.
All results of $D_s^+\to\tau^+\nu_\tau$ via $\tau^+\to\pi^+\bar{\nu}_\tau$ obtained in this work supersede the BESIII previous results based on 6.32 fb$^{-1}$ of $e^+e^-$ collision data taken at center-of-mass energies between $4.178$ and $4.226$~GeV.
\end{abstract}


\maketitle

\section{INTRODUCTION}\label{sec:intro}
In the leptonic decay $D_s^+ \rightarrow \ell^+ \nu_\ell$, the charm quark ($c$) and anti-strange quark ($\bar{s}$) annihilate through a virtual $W$ boson to a charged and neutral lepton pair. According to the Standard Model (SM) and ignoring radiative corrections, the partial decay width of $D_s^+ \rightarrow \ell^+ \nu_\ell$ can be written as~\cite{Li:2021iwf} 
\begin{equation}\label{eq:decayrate}
  \begin{aligned}
    \Gamma_{D_s^+\to\ell^+\nu_\ell} 
    &=\frac{{\mathcal B}_{D^+_s\to \ell^+\nu_\ell}}{\tau_{D^+_s}}\\
    &=\frac{G^2_F}{8\pi}f^2_{D_s^+}|V_{cs}|^2m_{D_s^+}^3\mu_\ell^2(1-\mu_\ell^2)^2,\\
  \end{aligned}
\end{equation}
\noindent where ${\mathcal B}_{D^+_s\to \ell^+\nu_\ell}$ is the branching fraction of $D^+_s\to \ell^+\nu_\ell$, $\tau_{D^+_s}$ is the lifetime of $D^+_s$, $\mu_\ell$ stands for the ratio of the $\ell^+$ lepton mass to the $D_s^+$ meson mass ($m_{D_s^+}$), $G_F$ is the Fermi coupling constant, $f_{D^+_s}$ is the $D^+_s$ decay constant describing strong effect between quarks, $|V_{cs}|$ is the $c\to s$ Cabibbo-Kobayashi-Maskawa (CKM) matrix element describing weak effect between quarks. Experimentally measuring the branching fraction of $D_s^+\to \ell^+\nu_\ell$ can help us to determine $f_{D_s^+}$ when taking the $|V_{cs}|$ from the global fit in the SM as input, thereby testing various theoretical predictions, especially those from lattice quantum chromodynamics (LQCD)~\cite{FNAL/MILC17, ETM14E, chiQCD20A, RBC/UKQCD18A, HPQCD12A, PACS-CS11, Balasubramamian19, Blossier18, TWQCD14, ALPHA13B}. Conversely, one can determine $|V_{cs}|$ by taking the LQCD calculation of $f_{D_s^+}$ as input, thereby providing a stricter test of the CKM matrix unitarity. Therefore, studies of leptonic $D_s^+$ decays are important to deeply explore the strong and weak effects existing between the charm quark and the strange quark.

In addition, the ratio of the partial decay widths between $D_s^+\to \tau^+\nu_\tau$ and $D_s^+\to \mu^+\nu_\mu$ is defined as\begin{equation}\label{eq:theR}
  R = \frac{\Gamma_{D_s^+\to\tau^+\nu_\tau}}{\Gamma_{D_s^+\to\mu^+\nu_\mu}} = \frac{\mu^2_\tau(1-\mu^2_\tau)^2}{\mu^2_\mu(1-\mu^2_\mu)^2},
\end{equation}
\noindent where $f_{D_s^+}$ and $|V_{cs}|$ have been canceled. The SM calculation gives a  very precise prediction of $R = 9.75\pm0.01$. Any significant deviation from this value would imply new physics beyond the SM.

Previously, CLEO~\cite{CLEOel,CLEOpi,CLEOro}, Belle~\cite{Bellelnu} and BaBar~\cite{BaBarlnu} have reported the measurements of the branching fraction of $D_s^+\to\tau^+\nu_\tau$. In recent years, BESIII reported the measurements of the branching fractions $D_s^+\to\tau^+\nu_\tau$ via $\tau^+\to\pi^+\bar{\nu}_\tau$~\cite{hajime}, $\tau^+\to\pi^+\pi^0\bar{\nu}_\tau$~\cite{BES3taupipi0}, $\tau^+\to e^+\bar{\nu}_\tau\nu_e$~\cite{BES3taue}, and $\tau^+\to \mu^+\bar{\nu}_\tau\nu_\mu$~\cite{BES3taumu} using $6.32$ fb$^{-1}$ of $e^+e^-$ collision data collected at center-of-mass energies ($E_{\rm cm}$) between 4.178 and 4.226 GeV, as well as the measurement via $\tau^+\to\pi^+\bar{\nu}_\tau$ using 0.48 fb$^{-1}$ of $e^+e^-$ collision data at $E_{\rm cm}=4.008$ GeV~\cite{BES34009}.
In this paper, we report an updated measurement of the branching fraction of $D_s^+\to \tau^+\nu_\tau$ via $\tau^+\to \pi^+\bar{\nu}_\tau$ with the boosted decision trees (BDT)~\cite{tmva4} method by analyzing 7.33 fb$^{-1}$ of $e^+e^-$ collision data taken at $E_{\text{cm}}$ = $4.128$~GeV, $4.157$~GeV, $4.178$~GeV, $4.189$~GeV, $4.199$~GeV, $4.209$~GeV, $4.219$~GeV, and $4.226$~GeV~\cite{ecm4230}. The integrated luminosities for these subsamples are $0.402$~fb$^{-1}$, $0.409$~fb$^{-1}$, $3.189$~fb$^{-1}$, $0.570$~fb$^{-1}$, $0.526$~fb$^{-1}$, $0.572$~fb$^{-1}$, $0.569$~fb$^{-1}$, and $1.092$~fb$^{-1}$~\cite{lumi4230}, respectively, with an uncertainty of 1\% dominated by systematic uncertainty. For simplicity, we refer to these datasets as $4128$, $4157$, $4178$, $4189$, $4199$, $4209$, $4219$, and $4226$ in this paper. The reported results in this work supersede the previous results reported in Ref.~\cite{hajime}. Throughout this paper, charge conjugate modes are always implied.

\section{BESIII EXPERIMENT AND DATA SETS}
The BESIII detector~\cite{2009MAblikimDet} records symmetric $e^+e^-$ collisions provided by the BEPCII storage ring~\cite{bepcii} in the center-of-mass energy range from 2.0~GeV to 4.95~GeV, with a peak luminosity of $1 \times 10^{33}\;\text{cm}^{-2}\text{s}^{-1}$ achieved at $\sqrt{s} = 3.77\;\text{GeV}$. BESIII has collected large data samples in this energy region~\cite{2009MAblikimDet}. The cylindrical core of the BESIII detector covers 93\% of the full solid angle and consists of a helium-based multilayer drift chamber~(MDC), a plastic scintillator time-of-flight system~(TOF), and a CsI(Tl) electromagnetic calorimeter~(EMC), which are all enclosed in a superconducting solenoidal magnet providing a 1.0~T magnetic field~\cite{detvis}. The solenoid is supported by an octagonal flux-return yoke with resistive plate counter muon identification modules interleaved with steel. The charged particle momentum resolution at $1~{\rm GeV}/c$ is $0.5\%$, and the specific ionization energy loss ${\rm d}E/{\rm d}x$ resolution is $6\%$ for electrons from Bhabha scattering. The EMC measures photon energies with a resolution of $2.5\%$ ($5\%$) at $1$~GeV in the barrel (end-cap) region. The time resolution in the TOF barrel region is 68~ps. The end-cap TOF system was upgraded in 2015 using multi-gap resistive plate chamber technology, providing a time resolution of 60~ps~\cite{bes3mrpc}.

Simulated data samples produced with a {\sc geant4}-based~\cite{geant} Monte Carlo (MC) package, which includes the geometric description of the BESIII detector and the detector response, are used to determine detection efficiencies and to estimate backgrounds. The simulation models the beam energy spread and initial-state radiation (ISR) in the $e^+e^-$ annihilations with the {\sc kkmc} generator~\cite{kkmc}. The inclusive MC sample includes the production of open charm processes, the ISR production of vector charmonium(-like) states, and the continuum processes incorporated in {\sc kkmc}~\cite{kkmc}. All particle decays are modeled with {\sc evtgen}~\cite{evtgen} using branching fractions either taken from the Particle Data Group (PDG)~\cite{pdg2022}, when available, or otherwise estimated with {\sc lundcharm}~\cite{lundcharm}. Final-state radiation from charged final state particles is incorporated using the {\sc photos} package~\cite{photos}. 	The input cross section line shape of	$e^+e^-\to D^{*\pm}_sD^{\mp}_s$ is based on the cross section measurement in the energy range from threshold to 4.7 GeV.

\section{DATA ANALYSIS}
At $E_{\rm cm}$ between 4.128~GeV and 4.226 GeV, the $D_s$ mesons are produced dominantly by the $e^+e^-\to D^{*\pm}_sD^\mp_s$ reaction. Therefore, we can employ the double-tag (DT) technique pioneered by the MARK III Collaboration~\cite{markiii} in our selection of $D_s^+\to \tau^+\nu_\tau$ via $\tau^+\to\pi^+\bar{\nu}_\tau$ decays. In this method, a $D_s^-$ is fully reconstructed via any of thirteen hadronic decay modes (tag side), while the $D_s^+$ is identified via one charged track (signal side).
The single-tag (ST) $D_s^-$ can either be directly produced from the $e^+e^-$ collision (direct $D_s^-$) or be the daughter of a $D_s^*$ (indirect $D_s^-$).
The branching fraction of the signal decay is determined by
\begin{equation}\label{eq:dteqtau}
  \small
  \b(D_s^+\to\tau^+\nu_\tau) = \frac{N_\text{DT}^{\tau\nu}}{\sum_{i}{N^i_\text{ST}(\epsilon^{\tau\nu,i}_\text{DT}/\epsilon^i_\text{ST})} \b(\tau^+\to\pi^+\bar{\nu}_\tau)},
\end{equation}
\noindent where $N_\text{DT}$ is the yield of DT events with the tag side and signal side simultaneously reconstructed, $N_\text{ST}^i$ is the ST yield of reconstructed $D_s^-$ for tag mode $i$, and $\epsilon^{\tau\nu,i}_\text{DT}$ and $\epsilon^i_\text{ST}$ are the corresponding reconstruction efficiencies. Here the efficiency $\epsilon^{\tau\nu,i}_\text{DT}$ includes $\b(D^{*+}_s\to\gamma(\pi^0) D^+_s)$, but not the branching fraction  
$\b(\tau^+\to\pi^+\bar{\nu}_\tau)$.

\subsection{\boldmath Selection of ST $D_s^-$ candidates}\label{sec:STselect}
 Thirteen hadronic decay channels are used as the tag modes.  These are 
$D_s^-\to K^0_SK^-$, $K^-K^+\pi^-$, $K^0_SK^-\pi^0$,
$K^-K^+\pi^-\pi^0$, $K^0_SK^-\pi^+\pi^-$, $K^0_SK^+\pi^-\pi^-$,
$\pi^-\pi^-\pi^+$, $\pi^-\eta$,
$\rho^-\eta$, $\rho^-\eta_{3\pi}$,
$\pi^-\eta^\prime_{\pi\pi\eta}$, $\pi^-\eta^\prime_{\gamma\rho}$, and
$K^-\pi^-\pi^+$, where
$\pi^0\to\gamma\gamma$,
$K^0_S\to\pi^+\pi^-$,
$\eta\to\gamma\gamma$,
$\eta_{3\pi}\to\pi^+\pi^-\pi^0$,
$\rho^{-(0)}\to\pi^{0(+)}\pi^-$,
$\eta^\prime_{\pi\pi\eta}\to\pi^+\pi^-\eta$, and
$\eta^\prime_{\gamma\rho}\to\gamma\rho^0$.  This selection of modes is chosen after performing  the full analysis procedure on simulated data samples, with the aim of maximizing the signal sensitivity while introducing minimum bias on the measurement.

The selection criteria for $D_s^-$ daughters and the reconstruction procedures are the same as those described in Refs.~\cite{zhangsf,BES3taupipi0}. Tracks must be within the fiducial region ($|\!\cos\theta| < 0.93$, where $\theta$ is the polar angle defined with respect to the $z$-axis, which is the symmetry axis of the MDC) and originate within $1$~cm ($10$~cm) of the interaction point in the plane transverse to the beam direction (in the beam direction).
This requirement on the primary vertex is not applied for the reconstruction of $K^0_S\to\pi^+\pi^-$ decays, for which the distances of the closest approach of the two charged pions to the interaction point are required to be less than 20~cm along the MDC axis. In addition, we constrain the charged-pion pair to have a common vertex with a loose fit-quality requirement of $\chi^2 < 200$. To form $K^0_S$ candidates, the invariant mass of the $\pi^+\pi^-$ combination is required to be within $(0.486,\,0.510)$\,GeV$/c^{2}$.

The $K/\pi$ particle identification (PID) is performed by using the TOF and ${\rm d}E/{\rm d}x$ information. Each charged track is assigned as a pion or kaon if the corresponding hypothesis has a higher likelihood. No PID is performed on the charged pions from the intermediate decay $K^0_S\to\pi^+\pi^-$. In addition, we require the reconstructed momentum for any charged or neutral pion to be greater than $0.1$~GeV$/c$ to suppress events from $D^*\to D\pi$ decays.

Photon candidates are chosen from  EMC showers unassociated with any charged track~\cite{2009MAblikimDet}. The shower must start between $0$ and $700$~ns after a beam crossing to suppress electronic noise and showers unrelated to the event. When forming $\pi^0$ and $\eta$ candidates, the showers must have an energy greater than $25$~MeV if they are detected in the barrel EMC and $50$~MeV for the end-cap EMC. The $\pi^0$ and $\eta$ candidates are formed by photon pairs with invariant masses lying within the intervals $(0.115,\,0.150)$\,GeV$/c^{2}$ and $(0.500,\,0.570)$\,GeV$/c^{2}$, respectively. To improve momentum resolution and suppress background, a kinematic fit is imposed on each chosen photon pair to constrain its invariant mass to the known $\pi^{0}$ or $\eta$ mass~\cite{pdg2022}. The $\chi^2$ of this kinematic fit is required to be less than 20.

For the tag modes $D_s^-\to \pi^-\eta$ and $D_s^-\to \rho^-\eta$, the $\eta$ candidates are also formed with the $\pi^+\pi^-\pi^0$ combinations with invariant masses within the interval $(0.530,\,0.570)~\mathrm{GeV}/c^2$. The $\eta^\prime$ candidates are formed from $\pi^+\pi^-\eta$ and $\gamma\rho^0$ combinations with invariant masses lying within the intervals $(0.946,\,0.970)~\mathrm{GeV}/c^2$ and $(0.940,\,0.976)~\mathrm{GeV}/c^2$, respectively. In addition, the minimum energy of the $\gamma$ from $\eta^\prime\to\gamma\rho^0$ decay must be greater than 0.1\,GeV. The $\rho^0$ and $\rho^+$ candidates are reconstructed from $\pi^+\pi^-$ and $\pi^+\pi^0$ combinations with invariant masses within the interval $(0.570,\,0.970)~\mathrm{GeV}/c^2$.

Once the ST $D_s^-$ is reconstructed, we calculate the recoil mass against the $D_s^-$ tag as $M_{\text{rec}}^2c^4 = \Big (E_{\text{cm}} - \sqrt{|\vec{p}_{D_s^-}|^2c^2+m^2_{D_s^-}c^4}\Big )^2 - |\vec{p}_{D_s^-}|^2c^2$ in the center-of-mass system of the initial $e^+e^-$, where $\vec{p}_{D_s^-}$ is the three-momentum of the reconstructed $D_s^-$, and $m_{D_s^-}$ is the known $D_s^-$ mass~\cite{pdg2022}. Figure~\ref{fig:strecoil4180} shows the $M_{\rm rec}$ distribution for $D_s^-\to K^-K^+\pi^-$ in the 4178 data. In order to illustrate the shapes of the direct or indirect tag $D_s^-$ and the backgrounds, these three MC contributions are not stacked. All $e^+e^-\to D^{*\pm}_sD^\mp_s$ events accumulate near $m_{D_s^*} = 2.1122$~GeV$/c^2$~\cite{pdg2022}, with the direct component populating the central peak and the indirect component distributed more broadly.

\begin{figure}[htbp]\centering
  \includegraphics[keepaspectratio=true,width=3.4in,angle=0]{mDs_rec_draft.eps}
  \caption{The $M_{\rm rec}$ distribution of the candidates for $D_s^-\to K^-K^+\pi^-$ in the $4178$ data and MC-simulated background. The dashed vertical lines show the ST signal region. The black points with error bars are data, the gray-solid histogram represents the inclusive MC backgrounds, while the crosshatched histograms show the direct tag $D_s^-$ (blue) and the indirect tag $D_s^-$ (red).}
  \label{fig:strecoil4180}
\end{figure}

To ensure that events are from $e^+e^-\to D^{*\pm}_sD^\mp_s$, the $M_{\rm rec}$ of the ST candidates is required to satisfy the $E_{\text{cm}}$-dependent requirement listed in Table~\ref{tab:mreccut}. This requirement retains most of the $D_s^-$ mesons from $e^+e^-\to D^{*\pm}_sD^\mp_s$, and maintain roughly constant various tag efficiencies for different $E_{\text{cm}}$ datasets. When multiple reconstructed candidates are found for a given $D_s^-$ tag mode and electric charge, only the one with the $M_{\rm rec}$ closest to the known $D_s^{*+}$ mass~\cite{pdg2022} is kept for further analysis.

As example, Fig.~\ref{fig:STFIT_DATA} shows the $M_{\rm tag}$ distributions of the ST candidates selected from the 4178 data. The ST yield for each tag mode is determined from an unbinned maximum likelihood fit to the corresponding $M_{\rm tag}$ distribution in the range of $1.90<M_{\rm tag}<2.03$~GeV$/c^2$, where $M_{\rm tag}$ is the mass of the reconstructed ST $D_s^-$ candidate. The signal shapes are derived from MC simulations, obtained by the Gaussian kernel estimation method~\cite{keypdf}, and convolved with a Gaussian function to account for the resolution difference between data and MC simulation. For the tag mode $D_s^-\to K_S^0K^-$, the peaking background from $D^-\to K^0_S\pi^-$ is described by the simulated shape convolved with the same Gaussian function used in the signal shape and its size is left free. The non-peaking background is modeled by a first- or second-order Chebychev polynomial function, which has been validated by analyzing the inclusive sample.

Figure~\ref{fig:STFIT_DATA} shows the results of fits to the $M_{\rm tag}$ distributions of the thirteen tag modes for the $4178$ data.
The obtained $\text{ST}$ yields from data for various tag modes and data samples within the $M_{\rm tag}$ window are shown in Table~\ref{tab:xyzstyields}. Table~\ref{tab:xyzSTeff} shows the ST efficiencies for various tag modes which are obtained by analyzing inclusive MC sample within the $M_{\rm tag}$ window.

\begin{table}[htb]\centering
  \caption{The requirement on $M_{\rm rec}$ for the ST candidates at each energy point.}
  \label{tab:mreccut}
    \begin{tabular}{C{2cm} C{3.5cm}}
    \hline\hline
    Data set & $M_{\rm rec}$~(GeV$/c^2$) \\\hline
    4128    & $(2.060,\,2.155)$ \\
    4157    & $(2.054,\,2.175)$ \\
    4178    & $(2.050,\,2.195)$ \\
    4189    & $(2.048,\,2.205)$ \\
    4199    & $(2.046,\,2.215)$ \\
    4209    & $(2.044,\,2.225)$ \\
    4219    & $(2.042,\,2.235)$ \\
    4226    & $(2.040,\,2.220)$ \\
    \hline\hline
    \end{tabular}
\end{table}

\begin{figure*}
\begin{tikzpicture}
\node [ above right, inner sep=0] (image) at (0,0) {\includegraphics[keepaspectratio=true,width=\textwidth,angle=0]{plot_3x4.eps}};
\node[align=left] at (5.6,18.8) {$\pmb{K^+K^-\pi^+}$};
\node[align=left] at (11.2,18.8) {$\pmb{K^+K^-\pi^+\pi^0}$};
\node[align=left] at (17,18.8) {$\pmb{\pi^-\pi^-\pi^+}$};
\node[align=left] at (5.8,15.1) {$\pmb{K^0_SK^-}$};
\node[align=left] at (11.3,15.1) {$\pmb{K^0_SK^-\pi^0}$};
\node[align=left] at (17,15.1) {$\pmb{K^-\pi^-\pi^+}$};
\node[align=left] at (5.5,11.4) {$\pmb{K^0_SK^+\pi^-\pi^-}$};
\node[align=left] at (11.2,11.4) {$\pmb{K^0_SK^-\pi^+\pi^-}$};
\node[align=left] at (17.2,11.4) {$\pmb{\pi^-\eta}$};
\node[align=left] at (5.8,7.7) {$\pmb{\pi^-\eta^\prime_{\pi\pi\eta}}$};
\node[align=left] at (11.5,7.7) {$\pmb{\pi^-\eta^\prime_{\gamma\rho}}$};
\node[align=left] at (17.2,7.7) {$\pmb{\rho^-\eta}$};
\node[align=left] at (5.8,4.4) {$\pmb{\rho^-\eta_{3\pi}}$};
\end{tikzpicture}
\caption{Fits to the $M_{\rm tag}$ distributions of the candidates for various ST modes from the 4178 data. The points with error bars are data, while red-solid curves represent the total fits, blue-dashed curves are the fitted signal shapes and the green-dotted curves are the fitted background shapes. The violet curve for the $K_S^0K^-$ tag mode is the fitted peaking background shape from $D^-\to K^0_S\pi^-$. For each fit, the $\chi^2$/NDOF ranges from 0.8 to 1.3, where NDOF is the number of degrees of freedom.}
\label{fig:STFIT_DATA}
\end{figure*}

\begin{table*}[htb]\centering
  \caption{The $\text{ST}$ yields ($N_\text{ST}$) for each tag mode and data sample, in units of $10^3$. ``SUM'' denotes the total ST yield summed over tag modes. The uncertainties are statistical only.}
  \label{tab:xyzstyields}
  \scalebox{1.0}
  {
    \begin{tabular}{c| c c c c c c c c}
    \hline\hline
    \diagbox{Tag mode}{Data set} & $4128$ & $4157$ & $4178$ & $4189$ & $4199$ & $4209$ &  $4219$ &  $4226$ \\
    \hline
    {\ensurestackMath{
    \alignCenterstack{
    K^{+}K^{-}\pi^{-}\cr
    K^{+}K^{-}\pi^{-}\pi^{0}\cr
    \pi^{+}\pi^{-}\pi^{-}\cr
    K_{S}^{0}K^{-}\cr
    K_{S}^{0}K^{-}\pi^{0}\cr
    K^{-}\pi^{+}\pi^{-}\cr
    K_{S}^{0}K^{+}\pi^{-}\pi^{-}\cr
    K_{S}^{0}K^{-}\pi^{+}\pi^{-}\cr
    \pi^{-}\eta\cr
    \pi^{-}\eta^{\prime}_{\pi\pi\eta}\cr
    \pi^{-}\eta^{\prime}_{\gamma\rho}\cr
    \rho^{-}\eta\cr
    \rho^{-}\eta_{3\pi}}}} &
    {\ensurestackMath{
    \alignCenterstack{
    11.6\pm0.2\cr
    4.0\pm0.4\cr
    2.5\pm0.2\cr
    2.5\pm0.1\cr
    0.9\pm0.1\cr
    1.8\pm0.3\cr
    1.3\pm0.1\cr
    0.8\pm0.1\cr
    1.6\pm0.1\cr
    0.8\pm0.0\cr
    2.0\pm0.2\cr
    3.4\pm0.4\cr
    1.0\pm0.2}}} &
    {\ensurestackMath{
    \alignCenterstack{
    18.2\pm0.2\cr
    6.4\pm0.4\cr
    5.0\pm0.4\cr
    4.2\pm0.1\cr
    1.5\pm0.1\cr
    2.2\pm0.2\cr
    1.9\pm0.1\cr
    1.0\pm0.1\cr
    2.7\pm0.2\cr
    1.3\pm0.1\cr
    2.9\pm0.3\cr
    5.8\pm0.6\cr
    1.4\pm0.3}}} &
    {\ensurestackMath{
    \alignCenterstack{
    146.2\pm0.7\cr
    49.8\pm1.1\cr
    41.1\pm1.0\cr
    32.2\pm0.3\cr
    12.8\pm0.4\cr
    18.5\pm0.7\cr
    16.8\pm0.3\cr
    9.0\pm0.3\cr
    20.6\pm1.0\cr
    10.2\pm0.0\cr
    25.8\pm0.8\cr
    42.0\pm2.2\cr
    10.6\pm1.0}}} &
    {\ensurestackMath{
    \alignCenterstack{
    25.1\pm0.3\cr
    8.8\pm0.5\cr
    7.1\pm0.5\cr
    5.6\pm0.1\cr
    2.2\pm0.2\cr
    3.6\pm0.4\cr
    2.9\pm0.1\cr
    1.5\pm0.1\cr
    3.2\pm0.2\cr
    1.8\pm0.1\cr
    4.0\pm0.2\cr
    6.0\pm0.7\cr
    1.8\pm0.3}}} &
    {\ensurestackMath{
    \alignCenterstack{
    22.9\pm0.3\cr
    8.0\pm0.5\cr
    6.3\pm0.5\cr
    5.1\pm0.1\cr
    2.6\pm0.0\cr
    2.9\pm0.1\cr
    2.7\pm0.1\cr
    1.6\pm0.2\cr
    3.2\pm0.2\cr
    1.5\pm0.1\cr
    3.6\pm0.3\cr
    7.1\pm0.5\cr
    1.7\pm0.6}}} &
    {\ensurestackMath{
    \alignCenterstack{
    23.9\pm0.3\cr
    7.7\pm0.5\cr
    6.4\pm0.5\cr
    5.0\pm0.1\cr
    2.0\pm0.0\cr
    2.6\pm0.2\cr
    2.4\pm0.1\cr
    1.4\pm0.2\cr
    3.4\pm0.2\cr
    1.6\pm0.1\cr
    4.3\pm0.4\cr
    6.6\pm0.7\cr
    2.4\pm0.6}}} &
    {\ensurestackMath{
    \alignCenterstack{
    20.8\pm0.3\cr
    6.9\pm0.5\cr
    4.9\pm0.5\cr
    4.3\pm0.1\cr
    1.5\pm0.2\cr
    2.4\pm0.3\cr
    2.1\pm0.1\cr
    1.3\pm0.2\cr
    2.6\pm0.2\cr
    1.4\pm0.1\cr
    3.9\pm0.4\cr
    5.0\pm0.5\cr
    2.5\pm1.4}}} &
    {\ensurestackMath{
    \alignCenterstack{
    31.4\pm0.4\cr
    10.7\pm0.6\cr
    8.0\pm1.2\cr
    6.9\pm0.1\cr
    2.7\pm0.2\cr
    5.2\pm0.5\cr
    3.2\pm0.1\cr
    1.6\pm0.2\cr
    4.5\pm0.5\cr
    2.2\pm0.1\cr
    5.0\pm0.4\cr
    9.7\pm1.3\cr
    2.8\pm0.3}}} \\
    \hline
    SUM &
    {\ensurestackMath{\alignCenterstack{~34.2\pm0.8}}} &
    {\ensurestackMath{\alignCenterstack{~54.6\pm1.0}}} &
    {\ensurestackMath{\alignCenterstack{~435.8\pm3.3}}} &
    {\ensurestackMath{\alignCenterstack{~73.7\pm1.2}}} &
    {\ensurestackMath{\alignCenterstack{~69.1\pm1.2}}} &
    {\ensurestackMath{\alignCenterstack{~69.8\pm1.3}}} &
    {\ensurestackMath{\alignCenterstack{~59.5\pm1.8}}} &
    {\ensurestackMath{\alignCenterstack{~94.1\pm2.2}}} \\
    \hline\hline
    \end{tabular}
  }
\end{table*}

\begin{table*}[htb]\centering
  \caption{The $\text{ST}$ efficiencies ($\epsilon_\text{ST}$ in $\%$) for each tag mode and data sample. The uncertainties are statistical only. Efficiencies do not include the branching fractions of the intermediate decays $K^0_S\to\pi^+\pi^-$, $\pi^0\to\gamma\gamma$, $\eta\to\gamma\gamma$, $\eta_{3\pi}\to\pi^+\pi^-\pi^0$, $\eta^\prime_{\pi\pi\eta}\to\pi^+\pi^-\eta$, $\eta^\prime_{\gamma\rho}\to\gamma\rho^0$, or $\rho\to\pi\pi$.
  }
  \label{tab:xyzSTeff}
  \scalebox{1.00}
  {
    \begin{tabular}{c| c c c c c c c c}
    \hline\hline
    \diagbox{Tag mode}{Data set} & $4128$ & $4157$ & $4178$ & $4189$ & $4199$ & $4209$ &  $4219$ &  $4226$ \\
    \hline
    {\ensurestackMath{
    \alignCenterstack{
    K^{+}K^{-}\pi^{-}\cr
    K^{+}K^{-}\pi^{-}\pi^{0}\cr
    \pi^{+}\pi^{-}\pi^{-}\cr
    K_{S}^{0}K^{-}\cr
    K_{S}^{0}K^{-}\pi^{0}\cr
    K^{-}\pi^{+}\pi^{-}\cr
    K_{S}^{0}K^{+}\pi^{-}\pi^{-}\cr
    K_{S}^{0}K^{-}\pi^{+}\pi^{-}\cr
    \pi^{-}\eta\cr
    \pi^{-}\eta^{\prime}_{\pi\pi\eta}\cr
    \pi^{-}\eta^{\prime}_{\gamma\rho}\cr
    \rho^{-}\eta\cr
    \rho^{-}\eta_{3\pi}}}} &
    {\ensurestackMath{
    \alignCenterstack{
    44.9\pm0.1\cr
    13.6\pm0.2\cr
    59.5\pm0.6\cr
    49.6\pm0.2\cr
    19.3\pm0.4\cr
    51.6\pm1.1\cr
    22.7\pm0.2\cr
    20.3\pm0.4\cr
    51.6\pm0.6\cr
    26.1\pm0.5\cr
    34.5\pm0.5\cr
    21.7\pm0.3\cr
    9.8\pm0.3}}} &
    {\ensurestackMath{
    \alignCenterstack{
    44.6\pm0.1\cr
    13.7\pm0.1\cr
    60.1\pm0.5\cr
    49.6\pm0.2\cr
    19.0\pm0.3\cr
    51.0\pm0.8\cr
    23.0\pm0.2\cr
    20.3\pm0.3\cr
    51.6\pm0.4\cr
    25.6\pm0.3\cr
    34.5\pm0.3\cr
    21.2\pm0.2\cr
    9.9\pm0.2}}} &
    {\ensurestackMath{
    \alignCenterstack{
    43.7\pm0.1\cr
    13.9\pm0.1\cr
    57.8\pm0.2\cr
    49.6\pm0.1\cr
    19.5\pm0.1\cr
    50.6\pm0.3\cr
    23.6\pm0.1\cr
    21.4\pm0.1\cr
    51.7\pm0.2\cr
    25.5\pm0.1\cr
    33.8\pm0.1\cr
    20.9\pm0.1\cr
    9.9\pm0.1}}} &
    {\ensurestackMath{
    \alignCenterstack{
    43.8\pm0.1\cr
    14.0\pm0.1\cr
    56.6\pm0.4\cr
    49.5\pm0.2\cr
    19.4\pm0.3\cr
    51.9\pm0.7\cr
    23.7\pm0.1\cr
    21.1\pm0.3\cr
    51.1\pm0.4\cr
    25.1\pm0.3\cr
    33.5\pm0.3\cr
    21.4\pm0.2\cr
    9.9\pm0.2}}} &
    {\ensurestackMath{
    \alignCenterstack{
    43.9\pm0.1\cr
    14.1\pm0.1\cr
    57.2\pm0.4\cr
    49.4\pm0.2\cr
    20.0\pm0.3\cr
    50.9\pm0.7\cr
    23.8\pm0.2\cr
    21.4\pm0.3\cr
    51.4\pm0.4\cr
    24.8\pm0.3\cr
    34.3\pm0.3\cr
    20.9\pm0.2\cr
    9.5\pm0.2}}} &
    {\ensurestackMath{
    \alignCenterstack{
    43.6\pm0.1\cr
    14.1\pm0.1\cr
    56.1\pm0.4\cr
    48.9\pm0.2\cr
    19.1\pm0.3\cr
    50.1\pm0.7\cr
    23.5\pm0.1\cr
    21.0\pm0.3\cr
    51.1\pm0.4\cr
    25.2\pm0.3\cr
    33.9\pm0.3\cr
    20.8\pm0.2\cr
    9.4\pm0.2}}} &
    {\ensurestackMath{
    \alignCenterstack{
    43.2\pm0.1\cr
    14.1\pm0.1\cr
    55.5\pm0.5\cr
    48.9\pm0.2\cr
    19.7\pm0.3\cr
    48.1\pm0.9\cr
    23.5\pm0.2\cr
    21.0\pm0.3\cr
    50.4\pm0.5\cr
    25.1\pm0.4\cr
    32.8\pm0.4\cr
    20.5\pm0.3\cr
    9.4\pm0.3}}} &
    {\ensurestackMath{
    \alignCenterstack{
    43.6\pm0.1\cr
    14.2\pm0.1\cr
    57.3\pm0.5\cr
    49.4\pm0.1\cr
    19.8\pm0.3\cr
    50.3\pm0.7\cr
    23.9\pm0.1\cr
    21.4\pm0.3\cr
    51.0\pm0.4\cr
    25.5\pm0.3\cr
    34.5\pm0.3\cr
    20.6\pm0.2\cr
    9.9\pm0.2}}} \\
    \hline\hline
  \end{tabular}
  }
\end{table*}

\subsection{\boldmath Selection of $D_s^+\to\tau^+\nu_\tau$}\label{sec:dtana}
The $D_s^{*+}\to D_s^+ \gamma(\pi^0)$ transition is distinguished from combinatorial backgrounds by a kinematic variable:\begin{equation}\label{eq:dele}
\Delta E \equiv E_{\rm cm}-E_{\rm tag}-E_{\rm miss}-E_{\gamma(\pi^0)}.
\end{equation}

\noindent Here $E_{\rm miss} \equiv \sqrt{|\vec{p}_{\rm miss}|^2+m_{D_s^+}^2}$ and $\vec{p}_{\rm miss} \equiv -\vec{p}_{D_s^-}-\vec{p}_{\gamma(\pi^0)}$ are the missing energy and momentum of the recoiling system of the transition $\gamma(\pi^0)$ and the single-tag $D_s^-$, respectively. Once the ST $D_s^-$ is reconstructed, we loop over all remaining $\gamma(\pi^0)$ candidates to construct the four-momentum of the $D_s^*$ candidate based on the one giving the least $|\Delta E|$.
In the $D_s^{*+}$ rest frame, the transition photon energy has a monochromatic energy of $(m^2_{D_s^*}c^4 - m^2_{D_s}c^4)/(2m_{D_s^*}c^2)=0.1389$~GeV. To further suppress background, the energy of transition photon is required to be within $0.119<E_\gamma<0.149$~GeV. This criterion is optimized via figure-of-merit (FoM) defined by $S/\sqrt{S+B}$, where $S$ and $B$ are the signal and background events from inclusive MC sample. The resulting photon selection efficiency is about 85\%.

The $D_s^+ \rightarrow \tau^+\nu_\tau$ signals are reconstructed via $\tau^+ \to \pi^+\bar{\nu}_\tau$ in events containing a $D_s^-$ hadronic tag. We require only one additional track that is not used in the tag reconstruction. The particle candidate of the signal side must have charge opposite to the tag side and satisfy the pion PID criteria described in Sec.~\ref{sec:STselect}.

Potential background events are suppressed by three additional variables that are sensitive to unreconstructed and misreconstructed particles such as electrons are misidentified as pions, photons correlated with $K^0_L$ or from $\pi^0$ and $\eta$. The first variable is the ratio of the energy deposited in the EMC over the MDC momentum of the charged pion from $\tau^+$ decays, labeled as EOP. Because the electron deposits most of its energy in the EMC, the EOP of the track has to be less than 0.9 to suppress those associated electron events. The second is the maximum energy of extra photons in events, labeled as $E^\text{max}_\text{neu}$. The $E^\text{max}_\text{neu}$ must be less than 0.3 GeV to ensure that there is no extra energetic photon in the selected events. The third is $\cos{\theta_\text{miss}}$, where $\theta_\text{miss}$ is the polar angle of $\vec{p}_{\text{miss},\nu} = -\vec{p}_{D_s^-} - \vec{p}_{\gamma(\pi^0)} - \vec{p}_{\pi^+}$ in the $e^+e^-$ center-of-mass frame. The value of $|\!\cos{\theta_\text{miss}}|$ is required to be less than 0.9 to restrict $\vec{p}_{\text{miss},\nu}$ to point into the fiducial volume of the BESIII detector.

The presence of neutrinos in the final states from the event missing mass-squared is inferred by a kinematic variable, $M^2_{\rm{miss}}$ $= E^2_{\rm{miss},\nu} - |\vec{p}_{\rm{miss},\nu}c|^2$, where $E_{\rm{miss},\nu} = E_{\rm{cm}} -\sqrt{|\vec{p}_{D_s^-}c|^2+m^2_{D_s^-}c^4} - E_{\gamma(\pi^0)} -E_{\pi^+}$ is calculated in the $e^+e^-$ center-of-mass frame. The selected candidates are studied in the interval of $-0.2<M^2_{\rm{miss}}<0.6$~${\rm GeV}^2/c^4$, to retain more signal events and suppress the background of $e^{+}e^{-}\to q\bar{q}~(q = u, d, s)$ at higher masses as shown in the first figure in Fig.~\ref{fig:inputvar}.

\subsection{\boldmath Further selection}\label{sec:furtherselection}

After these background suppression, the remaining background components are $D_s^{+}\to\mu^+\nu_\mu$, $D_s^{+}\to K_L^0K^+$, $D_s^{+}\to K^0\pi^+$, $D_s^{+}\to\eta\pi^+$, other $\tau$ decays, $e^+e^-\to\tau^+\tau^-$, $e^+e^-\to q\bar{q}$, $D^0\to K^-\pi^+\pi^0$, and $D^0\to K^-\pi^+$. A BDT~\cite{tmva4}, which is built using the TMVA package~\cite{tmva3}, is used to discriminate signal from the background. 

As well as the signal regions described above, several control regions are defined. The control regions are chosen with requirements that are orthogonal to the signal-region selection criteria in order to provide independent data samples enriched in particular backgrounds and are used to normalize the simulated background.
In this analysis, four control regions ($\eta\pi$, $\tau_{\rm other}$, $\mu\nu_\mu$ and $qq\tau\tau$) corresponding to the four main background sources ($D_s^+\to \eta\pi^+$, other $\tau^+$ decay, $D_s^+\to \mu^+\nu_{\mu}$ and $e^{+}e^{-}\to\tau^{+}\tau^{-}/q \bar{q}~(q = u, d, s)$) are defined to validate the reliability of background simulation. The BDT analysis procedure in the control region is the same as that in the signal region defined later.

Unlike other machine learning algorithms such as (deep) neural networks, the BDT is not able to learn how to derive new variables~\cite{tmva2,tmva1}. It is therefore necessary to preprocess the input variables and combine them to derive variables that capture the event topologies. These variables are $M_{\rm tag}$ and $m^{\rm tag}_{\rm BC}$ of the tag side; the $\theta_{\rm miss}^{~\gamma}$, $\rm{cos}\theta_{\rm{miss}}$, and $M_{\rm{miss}}^2$ of the missing track; the $E_{\gamma(\pi^0)}$ of the transition $\gamma(\pi^0)$; the $E_\gamma^{\rm sum}$, which is the summed energy of extra photons in event; and the $\cos\theta_{\pi^+}$ and $\vec{p}_{\pi^+}$ of $\pi^+$ of the signal side. Here, $\theta_{\rm miss}^{~\gamma}$ is the opening angle between $\vec{p}_{\rm{miss},\nu}$ and the most energetic photon and $m^{\rm tag}_{\rm BC}$ is the beam-constrained mass of the ST $D_s^-$ candidate defined as $m^{\rm tag}_{\rm BC} = \sqrt{E^2_\text{beam}-|\vec{p}_{D_s^-}|^2}$, where $E_\text{beam}$ is the beam energy~($\sqrt{s}$/2).

The resultant distributions of input variables are shown in Fig.~\ref{fig:inputvar}. To minimize potential bias, we have performed a two-fold cross validation, splitting the MC samples into two subsets based on the event number. A BDT was trained on the even-event set and was then applied on the odd-event set. Another BDT was trained on the odd-event set and applied on the even-event set, and the results of two BDT test sets are combined together. To test our choice of super-parameters, we choose the intuitive binned signal significance method, provided in the TMVA package~\cite{tmva3}. After scanning more than 1000 points, the two FoMs give the optimal values of the number of trees 800, maximal depths 5 and minimum node size of 0.5\%, with significances of 230.4 and 0.4, respectively.

Figure~\ref{fig:croutput} shows the comparisons of the BDT scores of the four background sources in the various control regions. We observe consistency between data and MC simulation, except for small differences in the $\eta\pi$ control region.
This small difference is compensated by correcting this background component by a factor determined from two linear fits to the ratio plots. The residual imperfect simulation for this background source is considered as a systematic uncertainty.

\begin{figure*}[htbp]\centering
  \includegraphics[keepaspectratio=true,width=0.325\textwidth,angle=0]{mm2m.eps}
  \includegraphics[keepaspectratio=true,width=0.325\textwidth,angle=0]{Ppip.eps}
  \includegraphics[keepaspectratio=true,width=0.325\textwidth,angle=0]{Cpi.eps}
  \includegraphics[keepaspectratio=true,width=0.325\textwidth,angle=0]{egam.eps}
  \includegraphics[keepaspectratio=true,width=0.325\textwidth,angle=0]{egamsum.eps}
  \includegraphics[keepaspectratio=true,width=0.325\textwidth,angle=0]{mDs_tag.eps}
  \includegraphics[keepaspectratio=true,width=0.325\textwidth,angle=0]{ANG.eps}
  \includegraphics[keepaspectratio=true,width=0.325\textwidth,angle=0]{Cmiss.eps}
  \includegraphics[keepaspectratio=true,width=0.325\textwidth,angle=0]{mBC_tag.eps}
  \caption{Distributions of various input variables of the BDT. The black points with error bars are data, the red-shaded histogram shows the signal, the purple-shaded histogram is the $D_s^+\to \mu^+\nu_{\mu}$ background, the green-shaded histogram is the $D_s^+\to K^0\pi^+$ background, the orange-shaded histogram is the $D_s^+\to \eta\pi^+$ background, the cyan-shaded histogram is the other $\tau^{+}$ decays background, the magenta-shaded histogram is the $e^{+}e^{-}\to \tau^{+}\tau^{-}/q \bar{q}~(q = u, d, s)$ background and the blue-shaded histogram is the remaining background.}
  \label{fig:inputvar}
\end{figure*}

\begin{figure}[htbp]\centering
  \includegraphics[keepaspectratio=true,width=1.65in,angle=0]{etapi_tmva_draft.eps}
  \includegraphics[keepaspectratio=true,width=1.65in,angle=0]{othertau_tmva_draft.eps}
  \includegraphics[keepaspectratio=true,width=1.65in,angle=0]{muv_tmva_draft.eps}
  \includegraphics[keepaspectratio=true,width=1.65in,angle=0]{qqtt_tmva_draft.eps}
  \caption{Distributions of BDT output score in the control regions $\eta\pi$ (left top), $\tau_{\rm other}$ (right top), $\mu\nu_\mu$ (left bottom) and $qq\tau\tau$ (right bottom). The black points with error bars are data, the green-solid curve represents the scaled inclusive MC sample, while the crosshatched histograms show the target background (red) and the other backgrounds (blue). The bottom panel shows the data and MC comparison, where the error band indicates the statistical uncertainty only. The signal contribution is negligible in the control region.}
  \label{fig:croutput}
\end{figure}

\subsection{\boldmath Branching fraction}
Figure~\ref{fig:bdtfit} shows the result of the fit to the BDT score on the accepted candidate events in data. The signal and background shapes are derived from the inclusive MC sample and their yields are free fit parameters. From the fit, we obtain $2371\pm 74$ $D_s^+\to\tau^+(\to\pi^+\bar{\nu}_\tau)\nu_\tau$ events.

\begin{figure}[htbp]\centering
  \includegraphics[keepaspectratio=true,width=3.4in,angle=0]{his_bdt_draft.eps}\\
  \caption{Fit result on the BDT score of the candidate events in data. The black points are data. The blue-dashed and green-dotted curves represent the fitted signal and background shapes, respectively. The red-solid curve represents the total fit.}
\label{fig:bdtfit}
\end{figure}

The DT efficiencies are shown in Table~\ref{tab:xyzDTefftau}. These efficiencies have been corrected by a factor $f^{\rm cor}=1.0033\times0.9890\times0.9917\times1.0003\times1.0058$ to take into account the data-MC efficiency differences for the requirements of $\pi^+$ tracking, $\pi^+$ PID, $E^\text{max}_\text{neu}$\&$N_{\rm extra}^{\rm charge}$\&$N_{\rm extra}^{\pi^{0}}$, EOP and the least $|\Delta E|$ as described in Sec.~\ref{sec:sys}.

\begin{table*}[htb]\centering
  \caption{The $\text{DT}$ efficiencies (in $\%$) of $D_s^+\to\tau^+(\to\pi^+\bar{\nu}_\tau)\nu_\tau$ for each tag mode and data sample. The uncertainties are statistical only. These efficiencies do not include the branching fractions of the intermediate decays of $K^0_S\to\pi^+\pi^-$, $\pi^0\to\gamma\gamma$, $\eta\to\gamma\gamma$, $\eta_{3\pi}\to\pi^+\pi^-\pi^0$, $\eta^\prime_{\pi\pi\eta}\to\pi^+\pi^-\eta$, $\eta^\prime_{\gamma\rho}\to\gamma\rho^0$ or $\rho\to\pi\pi$.}
  \label{tab:xyzDTefftau}
  \scalebox{1.00}
  {
    \begin{tabular}{c| c c c c c c c c}
      \hline\hline
      \diagbox{Tag mode}{Data set} & $4128$ & $4157$ & $4178$ & $4189$ & $4199$ & $4209$ &  $4219$ &  $4226$ \\
      \hline
      {\ensurestackMath{
      \alignCenterstack{
      K^{+}K^{-}\pi^{-}\cr
      K^{+}K^{-}\pi^{-}\pi^{0}\cr
      \pi^{+}\pi^{-}\pi^{-}\cr
      K_{S}^{0}K^{-}\cr
      K_{S}^{0}K^{-}\pi^{0}\cr
      K^{-}\pi^{+}\pi^{-}\cr
      K_{S}^{0}K^{+}\pi^{-}\pi^{-}\cr
      K_{S}^{0}K^{-}\pi^{+}\pi^{-}\cr
      \pi^{-}\eta\cr
      \pi^{-}\eta^{\prime}_{\pi\pi\eta}\cr
      \pi^{-}\eta^{\prime}_{\gamma\rho}\cr
      \rho^{-}\eta\cr
      \rho^{-}\eta_{3\pi}}}} &
      {\ensurestackMath{
      \alignCenterstack{
      19.9\pm0.1\cr
      5.9\pm0.1\cr
      29.8\pm0.1\cr
      24.7\pm0.1\cr
      10.0\pm0.1\cr
      25.3\pm0.1\cr
      10.0\pm0.1\cr
      8.6\pm0.1\cr
      27.5\pm0.1\cr
      13.4\pm0.1\cr
      17.8\pm0.1\cr
      12.6\pm0.1\cr
      5.6\pm0.1}}} &
      {\ensurestackMath{
      \alignCenterstack{
      19.4\pm0.1\cr
      6.2\pm0.1\cr
      29.2\pm0.1\cr
      24.1\pm0.1\cr
      9.9\pm0.1\cr
      24.7\pm0.1\cr
      10.1\pm0.1\cr
      8.8\pm0.1\cr
      27.1\pm0.1\cr
      13.4\pm0.1\cr
      17.5\pm0.1\cr
      12.3\pm0.1\cr
      5.4\pm0.1}}} &
      {\ensurestackMath{
      \alignCenterstack{
      18.4\pm0.1\cr
      6.0\pm0.1\cr
      27.4\pm0.1\cr
      23.4\pm0.1\cr
      9.6\pm0.1\cr
      23.6\pm0.1\cr
      10.2\pm0.1\cr
      8.7\pm0.1\cr
      25.8\pm0.1\cr
      12.7\pm0.1\cr
      16.6\pm0.1\cr
      11.9\pm0.1\cr
      5.3\pm0.1}}} &
      {\ensurestackMath{
      \alignCenterstack{
      18.2\pm0.1\cr
      5.9\pm0.1\cr
      27.1\pm0.2\cr
      23.0\pm0.2\cr
      9.7\pm0.1\cr
      23.1\pm0.2\cr
      9.9\pm0.1\cr
      8.5\pm0.1\cr
      25.3\pm0.2\cr
      12.3\pm0.1\cr
      16.3\pm0.1\cr
      11.5\pm0.1\cr
      5.1\pm0.1}}} &
      {\ensurestackMath{
      \alignCenterstack{
      18.2\pm0.1\cr
      5.8\pm0.1\cr
      26.3\pm0.2\cr
      22.5\pm0.1\cr
      9.4\pm0.1\cr
      22.5\pm0.1\cr
      9.8\pm0.1\cr
      8.6\pm0.1\cr
      25.2\pm0.2\cr
      12.4\pm0.1\cr
      16.3\pm0.1\cr
      11.5\pm0.1\cr
      5.2\pm0.1}}} &
      {\ensurestackMath{
      \alignCenterstack{
      17.2\pm0.1\cr
      5.6\pm0.1\cr
      25.2\pm0.2\cr
      21.6\pm0.1\cr
      8.9\pm0.1\cr
      21.5\pm0.1\cr
      9.4\pm0.1\cr
      8.2\pm0.1\cr
      24.2\pm0.2\cr
      11.7\pm0.1\cr
      15.5\pm0.1\cr
      11.0\pm0.1\cr
      4.9\pm0.1}}} &
      {\ensurestackMath{
      \alignCenterstack{
      17.1\pm0.1\cr
      5.5\pm0.1\cr
      25.1\pm0.2\cr
      20.9\pm0.1\cr
      8.8\pm0.1\cr
      21.2\pm0.1\cr
      9.0\pm0.1\cr
      7.9\pm0.1\cr
      23.9\pm0.2\cr
      11.4\pm0.1\cr
      15.4\pm0.1\cr
      11.1\pm0.1\cr
      4.8\pm0.1}}} &
      {\ensurestackMath{
      \alignCenterstack{
      17.0\pm0.1\cr
      5.7\pm0.1\cr
      25.8\pm0.2\cr
      21.9\pm0.1\cr
      9.1\pm0.1\cr
      21.9\pm0.1\cr
      9.5\pm0.1\cr
      8.3\pm0.1\cr
      24.3\pm0.2\cr
      12.1\pm0.1\cr
      15.7\pm0.1\cr
      11.1\pm0.1\cr
      4.9\pm0.1}}} \\
      \hline\hline
    \end{tabular}
}
\end{table*}

Based on these, we determine the branching fraction of $D_s^+\to \tau^+\nu_\tau$ to be $\b(D_s^+\to\tau^+\nu_\tau) = (5.41\pm0.17)\%$, where the uncertainty is statistical only and the branching fraction of $\tau^+\to \pi^+\bar \nu_\tau$ has been set to be 10.82\%~\cite{pdg2022}.

\section{SYSTEMATIC UNCERTAINTIES}\label{sec:sys}
The sources of systematic uncertainties in the measurement of the branching fraction are divided into two categories. One is from the BDT input variables that directly affect the measurement, the other is from the overall event reconstruction and other sources.

\subsection{\boldmath BDT input variables systematic uncertainties}\label{sec:systdire}
In the signal selection, we require only one charged track, no extra $\pi^0$ in addition to the daughters of the reconstructed tag and $E^\text{max}_\text{neu}<0.3$~GeV. The associated systematic uncertainty is estimated with DT hadronic events in which a $D_s^-$ is tagged in one of our thirteen modes, while the $D_s^+$ is reconstructed in either a $K^0_S K^+$ or $K^+K^-\pi^+$ decay. The ratio of the acceptance efficiencies between data and MC simulation is $0.9917\pm0.0043$. After correcting the MC signal efficiency to data, we assign 0.43\% as the systematic uncertainty.

The systematic uncertainties due to the requirements of $\text{EOP}<0.9$ and $|\!\cos{\theta_\text{miss}}|<0.90$ are estimated similarly. The ratios of the acceptance efficiencies between data and MC simulation are $1.0003\pm0.0020$ and $1.0000\pm0.0107$. After correcting the MC signal efficiency to data, we assign 0.20\% and 1.07\% as the systematic uncertainties, respectively.

The systematic uncertainty due to the $M_{\rm miss}^2$ range is estimated by varying the $M_{\rm miss}^2$ within the range of $\pm$0.05 GeV$^2/c^4$ at the high and low ends randomly. Among 24 alternative ranges, all changes of re-measured branching fraction are within $\pm 2\sigma$ after taking into account the correlations between signal samples.
Therefore, the associated systematic uncertainty is considered negligible, following the procedure advocated in Ref.~\cite{rbmethod}.

To examine the possible influence of the unobserved decays of $D_s^+\to \gamma \mu^+\nu_\mu$ and $D_s^+\to\pi^+\pi^0$, which have not been included in our nominal fit, we add a PDF for each of these two decays in an alternative fit and its yield is fixed to the experimental upper limit, $\b(D_s^+\to \gamma e^+\nu_e)<1.3\times10^{-4}$ or $\b(D_s^+\to\pi^+\pi^0)<3.4\times10^{-4}$~\cite{pdg2022}. Their effects on the branching fraction measurement are negligible.
 
We vary the branching fractions of the background components mentioned in Sec.~\ref{sec:furtherselection} within $\pm 1\sigma$, as given in  Ref.~\cite{pdg2022}, and the cross sections of the relevant processes, to account for the systematic uncertainties in these contributions. The uncertainty is estimated to be 1.63\%.

The reliability of the signal shape is validated by using a control sample of $D_s^+\to K^+K^-\pi^+$ decays. With the same analysis procedure as our nominal analysis, the branching fraction of $D_s^+\to K^+K^-\pi^+$ is determined to be consistent with the PDG value given in~\cite{pdg2022}. Similarly four control regions defined in Sec.~\ref{sec:furtherselection} are studied to verify the reliability of the background shape. A small data-MC difference is found for the $\eta\pi$ control region. The difference of the branching fractions measured with this background component corrected and left uncorrected, 0.21\%, is taken as the systematic uncertainty.

\subsection{\boldmath Overall event reconstruction and other systematic uncertainties}\label{sec:systother}
The systematic uncertainty associated with the $\pi^+$ tracking efficiency is estimated by using a control sample of $e^+e^-\to K^+K^-\pi^+\pi^-$ decays. The ratio of the $\pi^+$ tracking efficiencies of data and MC simulation, averaged over $\pi^+$ momentum in the signal decay is $1.0033\pm0.0035$. After correcting the MC signal efficiency to data, we assign 0.35\% as the systematic uncertainty.

The systematic uncertainty due to the $\pi^+$ PID efficiency is estimated by using control samples of $e^+e^-\to K^+K^-\pi^+\pi^-(\pi^0)$ and $\pi^+\pi^-\pi^+\pi^-(\pi^0)$ decays.
The ratio of the $\pi^+$ PID efficiencies of data and MC simulation, averaged over $\pi^+$ momentum in the signal decay, is $0.9890\pm0.0032$. After correcting the MC signal efficiency to data, we assign 0.32\% as the systematic uncertainty.

The systematic uncertainty in the $\gamma(\pi^0)$ selection is estimated by using a control sample of $J/\psi\to\pi^+\pi^-\pi^0$ decays~\cite{BES3GamSyst}, and the corresponding systematic uncertainty is assigned as 1\%.

The systematic uncertainty due to the least $|\Delta E|$ requirement is estimated with control samples of $D_s^+\to K^+K^-\pi^+$ and $D_s^+\to K^0_S K^+$ decays. The ratio of the acceptance efficiencies between data and MC simulation is $1.0058\pm0.0029$. After correcting the MC signal efficiency to data, we assign 0.29\% as the systematic uncertainty.

The uncertainty in the total number of the single-tag $D_s^-$ mesons is assigned to be 0.55\% by examining the changes of the fit yields when varying the signal and background shapes and taking into account the background fluctuation.

Both $\text{ST}$ and $\text{DT}$ selections involve reconstructing a hadronic $D_s^-$ decay, and the efficiencies for this tag reconstruction vary according to different event environments. The different topologies of leptonic $D_s^+$ decays (with only one track) and generic $D_s^+$ decays (most with multiple charged tracks and showers) lead to a mode-dependent bias in the tag reconstruction. This effect may not be well modeled in the MC simulation. The associated systematic uncertainty is estimated by studying tracking and PID efficiencies, and is assigned to be 0.42\%.

The uncertainty due to the limited sizes of the MC samples, which is most important for the determination of the DT efficiencies, is 0.19\%, and is assigned as a source of systematic uncertainty.

The systematic uncertainties associated with the input branching fractions, $\b(D^{*+}_s\to\gamma(\pi^{0}) D^+_s)$ and $\b(\tau^+\to\pi^+\bar{\nu}_\tau)$, are examined by varying individual PDG values by $\pm 1\sigma$~\cite{pdg2022}. The changes in the values of the  branching fraction are 0.2\% and 0.5\%, respectively. Combining these two effects in quadrature gives a total systematic uncertainty of 0.54\%.

By adding all systematic uncertainties in quadrature, as summarized in Table~\ref{tab:systsummary}, the total systematic uncertainty in the branching fraction measurement is determined to be 2.49\%.

\begin{table}[htb]\centering
  \caption{Relative systematic uncertainties in the branching fraction measurement.}
  \label{tab:systsummary}
    \begin{tabular}{C{5.5cm} C{2.4cm}}
    \hline\hline
    Source &  Uncertainty~(\%) \\
    \hline
    $E^\text{max}_\text{neu}$\&\&$N_{\rm extra}^{\rm char}\&\&N_{\rm extra}^{\pi^0}$ & $0.43$  \\
    $|\!\cos{\theta_\text{miss}}| < 0.9$ & $1.07$  \\
    $\text{EOP} < 0.9$ & $0.20$  \\
    $M_{\rm miss}^{2}$ range & Negligible \\
    $D_s^+\to\gamma\mu^+\nu_\mu$ background & Negligible \\
    $D_s^+\to\pi^+\pi^0$ background & Negligible \\
    Background estimate & $1.63$  \\
    Input shape for BDT & $0.21$ \\
    \hline
    $\pi^+$ tracking & $0.35$\\
    $\pi^+$ PID & $0.32$ \\
    $\gamma(\pi^0)$ reconstruction & $1.00$ \\
    Best $\gamma(\pi^0)$ selection & $0.29$ \\
    ST yield  & $0.55$ \\
    Tag bias & $0.42$  \\
    MC sample size  & $0.19$\\
    Input branching fractions   & $0.54$ \\
    \hline
    Total & $2.49$ \\
    \hline\hline
    \end{tabular}
\end{table}

\section{SUMMARY}\label{sec:summary}
Using 7.33 fb$^{-1}$ of $e^+e^-$ collision data taken at $E_{\rm cm}$ between 4.128 and 4.226 GeV, we report the updated study of $D_s^+\to\tau^+\nu_\tau$ via $\tau^+\to \pi^+\bar{\nu}_\tau$, where the candidates are isolated with a multivariate selection. The branching fraction of $D_s^+\to \tau^+\nu_\tau$ is determined to be $(5.41\pm0.17_{\rm stat}\pm0.13_{\rm syst})\%$. This result is consistent with the previous measurements by CLEO~\cite{CLEOel,CLEOpi,CLEOro}, BaBar~\cite{BaBarlnu}, Belle~\cite{Bellelnu}, and BESIII~\cite{BES3taupipi0,BES3taumu,BES3taue}. In particular, it supersedes the previous BESIII result of $(5.21\pm0.25_{\rm stat}\pm0.17_{\rm syst})\%$ published in Ref.~\cite{hajime}, which was measured via $\tau^+\to \pi^+\bar{\nu}_\tau$ by analyzing 6.32 fb$^{-1}$ of $e^+e^-$ collision data taken at $E_{\rm cm}$ between 4.178 GeV and 4.226 GeV. Table~\ref{tab:resultsummary} shows comparison of $\b(D^+_s\to\tau^+\nu_\tau)$ and $f_{D_s^+}|V_{cs}|$ obtained in this study and the previous measurements.

With the result for $\b(D_s^+\to\tau^+\nu_\tau)$ obtained in this study, we determine
\begin{equation}
  \nonumber
  \begin{array}{r c l c}
    f_{D_s^+}|V_{cs}| & = & (247.6\pm3.9_{\rm stat}\pm3.2_{\rm syst}\pm1.0_{\rm input})~\text{MeV}, &  \\
  \end{array}
\end{equation}
\noindent where the third uncertainty is from the external inputs of $m_\ell$, $m_{D_s^+}$, and $\tau_{D_s^+}$~\cite{pdg2022}.

By taking the $|V_{cs}| = 0.97349\pm0.00016$ given by the global fit  in the SM~\cite{pdg2022} as an input, we obtain
\begin{equation}
  \nonumber
  \begin{array}{r c l c}
    f_{D_s^+} & = & (254.3\pm4.0_{\rm stat}\pm3.3_{\rm syst}\pm1.0_{\rm input})~\text{MeV}, &  \\
  \end{array}
\end{equation}
which is in excellent agreement with the LQCD predictions~\cite{flag2021}. Conversely, by taking the LQCD calculation of $f_{D_s^+} = 249.9\pm0.5$~MeV~\cite{flag2021} as an input, we determine
\begin{equation}
  \nonumber
  \begin{array}{r c l c}
    |V_{cs}| & = & 0.991\pm0.015_{\rm stat}\pm0.013_{\rm syst}\pm0.004_{\rm input}, & \\
  \end{array}
\end{equation}
which agrees well with the result given by the global fit in the SM~\cite{pdg2022}.

Averaging the BESIII measurements of the branching fractions of $D_s^+\to\tau^+\nu_\tau$ measured via $\tau^+\to\pi^+\pi^0\bar{\nu}_\tau$~\cite{BES3taupipi0}, $\tau^+\to e^+\bar{\nu}_\tau\nu_e$~\cite{BES3taue}, $\tau^+\to \mu^+\bar{\nu}_\tau\nu_\mu$~\cite{BES3taumu}, and that via $\tau^+\to\pi^+\bar{\nu}_\tau$ from the current study, which are based on $e^+e^-\to D^\pm_s D^{*\mp}_s$ data, we obtain the weighted branching fraction to be $\b(D_s^+\to\tau^+\nu_\tau) = (5.32\pm0.07_{\rm stat}\pm0.07_{\rm syst})\%$. This gives $f_{D_s^+} = (252.1\pm1.7_{\rm stat}\pm1.7_{\rm syst}\pm1.0_{\rm input})~\text{MeV}$ and $|V_{cs}| = 0.982\pm0.007_{\rm stat}\pm0.007_{\rm syst}\pm0.004_{\rm input}$. Combining the world average of $\b(D^+_s\to \mu^+\nu_\mu) = (5.43\pm0.15)\text{\textperthousand}$~\cite{pdg2022}, we obtain $R = \Gamma(D_s^+\to\tau^+\nu_\tau)/\Gamma(D_s^+\to\mu^+\nu_\mu) = 9.79\pm0.33$. Here  the uncertainties from the ST yield, the $\pi^+$ tracking efficiency, the soft $\gamma(\pi^0)$ reconstruction, the best transition $\gamma(\pi^0)$ selection, and the tag bias are taken to be correlated for $\b(D_s^+\to\tau^+\nu_\tau)$. Additional common uncertainties come from $\tau_{D_s^+}$, $m_{D_s^+}$ and $m_\tau$ for $f_{D_s^+}$ and $|V_{cs}|$, while all the other uncertainties are independent.

Averaging the branching fractions of $D_s^+\to\tau^+\nu_\tau$ measured by CLEO~\cite{CLEOel,CLEOpi,CLEOro}, BaBar~\cite{BaBarlnu}, Belle~\cite{Bellelnu}, BESIII~\cite{BES3taupipi0,BES3taumu,BES3taue,BES34009} and from our study, we obtain the average branching fraction to be $\b(D_s^+\to\tau^+\nu_\tau) = (5.35\pm0.09)\%$. This gives $f_{D_s^+} = (253.0\pm2.0\pm1.0)~\text{MeV}$, $|V_{cs}| = 0.985\pm0.008\pm0.004$ and $R = \Gamma(D_s^+\to\tau^+\nu_\tau)/\Gamma(D_s^+\to\mu^+\nu_\mu) = 9.86\pm0.32$.

\begin{table*}[htb]\centering
  \caption{Comparison of the branching fractions and the corresponding products of $f_{D_s^+}|V_{cs}|$ from various experiments. ``Weighted'' are obtained by weighting with considering the correlated effects. ``Averages'' are obtained by weighting both statistical and systematic uncertainties, but not the third uncertainties, which are dominated by the uncertainty of the $D_s^+$ lifetime. The uncertainty of ``Averages'' of $\b$ and the first uncertainty of ``Average" of $f_{D^+_s}|V_{cs}|$ are
  the combined values of their statistical and systematic uncertainties, respectively, and the second uncertainty of ``Average" of $f_{D^+_s}|V_{cs}|$ due to the uncertainty of the quoted $D_s^+$ lifetime.}
  \label{tab:resultsummary}
  \def\1#1#2{\multicolumn{#1}{#2}}
    \begin{tabular}{L{2.0cm} C{2.0cm} C{2.5cm} C{3cm} C{3.8cm} C{3.5cm}}
    \hline\hline
    Experiment & $E_{\rm cm}$ (GeV) & Mode & $\tau^+$ decay & $\b~(\%)$ & $f_{D_s^+}|V_{cs}|$ (MeV)\\
    \hline
    {\bf This work}   & {\bf 4.128-4.226} & $\pmb{D^\pm_sD^{*\mp}_s}$  & $\pmb{\pi^+\bar{\nu}_\tau}$  & $\pmb{5.41\pm0.17\pm0.13}$ & $\pmb{247.6\pm3.9\pm3.2\pm1.0}$ \\
    BESIII~\cite{BES3taumu} & 4.128-4.226 &  $D^\pm_sD^{*\mp}_s$ & $\mu^+\bar{\nu}_\tau\nu_\mu$  & $5.34\pm0.16\pm0.09$ & $246.2\pm3.7\pm2.3\pm1.0$ \\
    BESIII~\cite{BES3taue} & 4.178-4.226 & $D^\pm_sD^{*\mp}_s$ & $e^+\bar{\nu}_\tau\nu_e$  & $5.27\pm0.10\pm0.13$ & $244.4\pm2.3\pm2.9\pm1.0$ \\
    BESIII~\cite{BES3taupipi0} & 4.178-4.226 &  $D^\pm_sD^{*\mp}_s$ & $\pi^+\pi^0\bar{\nu}_\tau$  & $5.30\pm0.25\pm0.20$ & $245.1\pm5.8\pm4.7\pm1.0$ \\
    BESIII~\cite{hajime} & 4.178-4.226 &  $D^\pm_sD^{*\mp}_s$  & $\pi^+\bar{\nu}_\tau$  & $5.21\pm0.25\pm0.17$ & $243.0\pm5.8\pm4.0\pm1.0$ \\
    \hline
    Weighted$^{\text{a}}$   & $\cdot\cdot\cdot$ &  $\cdot\cdot\cdot$  & $\cdot\cdot\cdot$   & $5.32\pm0.07\pm0.07$  & $245.4\pm1.7\pm1.7\pm1.0$ \\
    \hline
    BESIII~\cite{BES34009} & 4.008 &  $D_s^+D_s^-$  & $\pi^+\bar{\nu}_\tau$   & $3.28\pm1.83\pm0.37$ & $192.8\pm44.2\pm10.9\pm0.8$ \\
    CLEO~\cite{CLEOel}   & 4.170 & $D^\pm_sD^{*\mp}_s$   & $e^+\bar{\nu}_\tau\nu_e$ & $5.30\pm0.47\pm0.22$ & $245.1\pm10.9\pm5.1\pm1.0$ \\
    CLEO~\cite{CLEOpi}  & 4.170 & $D^\pm_sD^{*\mp}_s$   & $\pi^+\bar{\nu}_\tau$   & $6.42\pm0.81\pm0.18$ & $269.7\pm17.2\pm3.8\pm1.1$ \\
    CLEO~\cite{CLEOro}  & 4.170 &  $D^\pm_sD^{*\mp}_s$  & $\rho^+\bar{\nu}_\tau$ & $5.52\pm0.57\pm0.21$ & $250.1\pm13.0\pm4.8\pm1.0$ \\
    BaBar~\cite{BaBarlnu}  & 10.56 &  $DKX\gamma D^{-}_s$ & $e^+\bar{\nu}_\tau\nu_{e}, \mu^+\bar{\nu}_\tau\nu_{\mu}$ & $4.96\pm0.37\pm0.57$ & $237.1\pm8.9\pm13.7\pm1.0$ \\
    Belle~\cite{Bellelnu}  & 10.56 &  $DKX\gamma D^{-}_s$   & $\pi^+\bar{\nu}_\tau, e^+\bar{\nu}_\tau\nu_{e}, \mu^+\bar{\nu}_\tau\nu_{\mu}$ & $5.70\pm0.21^{+0.31}_{-0.30}$ & $254.1\pm4.7\pm7.0\pm1.0$\\
    \hline
    Average   &  $\cdot\cdot\cdot$ & $\cdot\cdot\cdot$   & $\cdot\cdot\cdot$   & $5.35\pm0.09$  & $246.3\pm2.0\pm1.0$ \\
    \hline\hline
    \1{4}{l}{$^{\text{a}}$It excludes ``BESIII~\cite{hajime}''.} \\
  \end{tabular}
\end{table*}

\acknowledgments
The BESIII Collaboration thanks the staff of BEPCII and the IHEP computing center for their strong support. This work is supported in part by National Key R\&D Program of China under Contracts Nos. 2020YFA0406400, 2020YFA0406300; National Natural Science Foundation of China (NSFC) under Contracts Nos. 11635010, 11735014, 11835012, 11935015, 11935016, 11935018, 11961141012, 12022510, 12025502, 12035009, 12035013, 12061131003, 12192260, 12192261, 12192262, 12192263, 12192264, 12192265; the Chinese Academy of Sciences (CAS) Large-Scale Scientific Facility Program; the CAS Center for Excellence in Particle Physics (CCEPP); Joint Large-Scale Scientific Facility Funds of the NSFC and CAS under Contract No. U1832207; CAS Key Research Program of Frontier Sciences under Contracts Nos. QYZDJ-SSW-SLH003, QYZDJ-SSW-SLH040; 100 Talents Program of CAS; The Institute of Nuclear and Particle Physics (INPAC) and Shanghai Key Laboratory for Particle Physics and Cosmology; ERC under Contract No. 758462; European Union's Horizon 2020 research and innovation programme under Marie Sklodowska-Curie grant agreement under Contract No. 894790; German Research Foundation DFG under Contracts Nos. 443159800, 455635585, Collaborative Research Center CRC 1044, FOR5327, GRK 2149; Istituto Nazionale di Fisica Nucleare, Italy; Ministry of Development of Turkey under Contract No. DPT2006K-120470; National Research Foundation of Korea under Contract No. NRF-2022R1A2C1092335; National Science and Technology fund of Mongolia; National Science Research and Innovation Fund (NSRF) via the Program Management Unit for Human Resources \& Institutional Development, Research and Innovation of Thailand under Contract No. B16F640076; Polish National Science Centre under Contract No. 2019/35/O/ST2/02907; The Royal Society, UK under Contract No. DH160214; The Swedish Research Council; U. S. Department of Energy under Contract No. DE-FG02-05ER41374.
%
\begin{thebibliography}{99}

\bibitem{Li:2021iwf} H.~B.~Li and X.~R.~Lyu,
%``Study of the standard model with weak decays of charmed hadrons at BESIII,'' Natl. Sci. Rev. \textbf{8}, no.11, nwab181 (2021) doi:10.1093/nsr/nwab181 
\href{https://doi.org/10.1093\%2Fnsr\%2Fnwab181}
{Natl.\ Sci.\ Rev. {\bf 8}, nwab181 (2021)}.

\bibitem{FNAL/MILC17} A.~Bazavov {\it et al.} (FNAL and MILC Collaboration),
\href{https://doi.org/10.1103/PhysRevD.98.074512}
{Phys.\ Rev.\ D {\bf 98}, 074512 (2018)}.

\bibitem{ETM14E} N.~Carrasco, P.~Dimopoulos, R.~Frezzotti, P.~Lami, V.~Lubicz {\it et al.} (ETM Collaboration),
\href{https://doi.org/10.1103/PhysRevD.91.054507}
{Phys.\ Rev.\ D {\bf 91}, 054507 (2015)}.

\bibitem{chiQCD20A} Y.~Chen, W.~F.~Chiu, M.~Gong, Z.~F.~Liu and Y.~H.~Ma, ($\chi$QCD Collaboration),
\href{https://doi.org/10.1088/1674-1137/abcd8f}
{Chin.\ Phys.\ C {\bf 45}, 023109 (2021)}.

\bibitem{RBC/UKQCD18A} P.~A.~Boyle, L.~Del Debbio, N.~Garron, A.~Juttner, A.~Soni, J.~T.~Tsang and O.~Witzel (RBC and UKQCD Collaboration),
\href{https://arxiv.org/abs/1812.08791}
{arXiv:1812.08791}.

\bibitem{HPQCD12A} H.~Na, C.~T.~H.~Davies, E.~Follana, G.~P.~Lepage and J.~Shigemitsu {\it et al.} (HPQCD Collaboration),
\href{https://doi.org/10.1103/PhysRevD.86.054510}
{Phys.\ Rev.\ D {\bf 86}, 054510 (2012)}.

\bibitem{PACS-CS11} Y.~Namekawa {\it et al.} (PACS-CS Collaboration),
\href{https://doi.org/10.1103/PhysRevD.84.074505}
{Phys.\ Rev.\ D {\bf 84}, 074505 (2011)}.

\bibitem{Balasubramamian19} R.~Balasubramamian and B.~Blossier,
\href{https://doi.org/10.1140/epjc/s10052-020-7965-z}
{Eur.\ Phys.\ J.\ C {\bf 80}, 412 (2020)}.

\bibitem{Blossier18} B.~Blossier, J.~Heitger and M.~Post,
\href{https://doi.org/10.1103/PhysRevD.98.054506}
{Phys.\ Rev.\ D {\bf 98}, 054506 (2018)}.

\bibitem{TWQCD14} W.~P.~Chen, Y.~C.~Chen, T.~W.~Chiu, H.Y.~Chou, T.S.~Guu and T.H.~Hsieh (TWQCD Collaboration),
\href{https://doi.org/10.1016/j.physletb.2014.07.025}
{Phys.\ Lett.\ B {\bf 734}, 231 (2014)}.

\bibitem{ALPHA13B} J.~Heitger, G.~M.~von Hippel, S.~Schaefer and F.~Virotta (ALPHA Collaboration),
\href{https://arxiv.org/abs/1312.7693}
{arXiv:1312.7693}.

\bibitem{CLEOel} P.~U.~E.~Onyisi  {\it et al.} (CLEO Collaboration),
\href{https://journals.aps.org/prd/abstract/10.1103/PhysRevD.79.052002}
{Phys.\ Rev.\ D {\bf 79}, 052002 (2009)}.

\bibitem{CLEOpi} J.~P.~Alexander  {\it et al.} (CLEO Collaboration),
\href{https://journals.aps.org/prd/abstract/10.1103/PhysRevD.79.052001}
{Phys.\ Rev.\ D {\bf 79}, 052001 (2009)}.

\bibitem{CLEOro} P.~Naik  {\it et al.} (CLEO Collaboration),
\href{https://journals.aps.org/prd/abstract/10.1103/PhysRevD.80.112004}
{Phys.\ Rev.\ D {\bf 80}, 112004 (2009)}.

\bibitem{Bellelnu} A.~Zupanc  {\it et al.} (Belle Collaboration),
\href{https://link.springer.com/article/10.1007/JHEP09(2013)139}
{J. High Energy Phys. {\bf 09} 139 (2013)}.

\bibitem{BaBarlnu} J.~P.~Lees  {\it et al.} (BaBar~Collaboration),
\href{https://journals.aps.org/prd/abstract/10.1103/PhysRevD.82.091103}
{Phys.\ Rev.\ D {\bf 82}, 091103(R) (2010)};
\href{https://journals.aps.org/prd/abstract/10.1103/PhysRevD.91.019901}
{{\bf 91}, 019901(E) (2015)}.

\bibitem{hajime} M.~Ablikim {\it et al.} (BESIII Collaboration),
\href{https://journals.aps.org/prd/abstract/10.1103/PhysRevD.104.052009}
{Phys.\ Rev.\ D {\bf 104}, 052009 (2021)}.

\bibitem{BES3taupipi0} M.~Ablikim  {\it et al.} (BESIII Collaboration),
\href{https://journals.aps.org/prd/abstract/10.1103/PhysRevD.104.032001}
{Phys.\ Rev.\ D {\bf 104}, 032001 (2021)}.

\bibitem{BES3taue} M.~Ablikim  {\it et al.} (BESIII Collaboration),
\href{https://journals.aps.org/prl/abstract/10.1103/PhysRevLett.127.171801}
{Phys.\ Rev.\ Lett. {\bf 127}, 171801 (2021)}.

\bibitem{BES3taumu} M.~Ablikim  {\it et al.} (BESIII Collaboration),
\href{https://docbes3.ihep.ac.cn/DocDB/0008/000854/009/BESIII_draft_tauv_muvv_v2.0.pdf}
{``Measurement of the branching fraction of $D_s^+\to \tau^+\nu_\tau$ via $\tau^+\to \mu^+\nu_\mu\bar{\nu}_\tau$", publication in preparation.}

\bibitem{BES34009} M.~Ablikim  {\it et al.} (BESIII Collaboration),
\href{https://journals.aps.org/prd/abstract/10.1103/PhysRevD.94.072004}
{Phys.\ Rev.\ D {\bf 94}, 072004 (2016)}.

\bibitem{tmva4} {Y.~Freund and R.~E.~Schapire},
\href{https://www.sciencedirect.com/science/article/pii/S002200009791504X}{Journal of Computer and System Sciences {\bf 55}, 119 (1997)}.

\bibitem{ecm4230} The $E_{\text{cm}}$ measurement is described in M. Ablikim {\it et al.} (BESIII Collaboration),
\href{http://hepnp.ihep.ac.cn/article/doi/10.1088/1674-1137/40/6/063001}
{Chin.\ Phys.\ C {\bf 40}, 063001 (2016)}, which includes the result for the 4230 data sample. $E_{\text{cm}}$ values for the other data samples have been obtained by a similar procedure.

\bibitem{lumi4230} M. Ablikim {\it et al.} (BESIII Collaboration),
\href{http://hepnp.ihep.ac.cn/article/doi/10.1088/1674-1137/39/9/093001}
{Chin.\ Phys.\ C {\bf 39}, 093001 (2015)};
\href{https://iopscience.iop.org/article/10.1088/1674-1137/ac80b4}
{Chin.\ Phys.\ C {\bf 46}, 113002 (2022)}. These articles described the integrated luminosity measurement for data taken at $\sqrt{s}$ = 4.189, 4.199, 4.209, 4.219, and 4.226 GeV. The integrated luminosity values for the other data samples have been obtained by a similar procedure.

\bibitem{2009MAblikimDet} M.~Ablikim {\it et al.} (BESIII Collaboration),
\href{http://www.sciencedirect.com/science/article/pii/S0168900209023870}
{Nucl.\ Instrum.\ Methods\ Phys.\ Res., \ Sec.\ A {\bf 614}, 345 (2010)};
\href{http://cpc.ihep.ac.cn/article/2020/4}
{Chin.\ Phys.\ C {\bf 44}, 040001 (2020)}.

\bibitem{bepcii} C.~H.~Yu {\it et al.},
\href{https://accelconf.web.cern.ch/ipac2016/doi/JACoW-IPAC2016-TUYA01.html}
{in {\it Proceedings of IPAC 2016, Busan, Korea, 2016}, ISBN 978-3-95450-147-2}.

\bibitem{detvis} K.~X.~Huang, {\it et al.},
\href{https://doi.org/10.1007/s41365-022-01133-8}
{Nucl.\ Sci.\ Tech. {\bf 33}, 142 (2022)}.

\bibitem{bes3mrpc} X.~Li {\it et al.},
\href{https://link.springer.com/article/10.1007/s41605-017-0014-2}
{Radiat.\ Detect.\ Technol.\ Methods {\bf 1}, 13 (2017)};
Y.~X.~Guo {\it et al.},
\href{https://link.springer.com/article/10.1007\%2Fs41605-017-0012-4}
{Radiat.\ Detect.\ Technol.\ Methods {\bf 1}, 15 (2017)}.

\bibitem{geant} S.~Agostinelli {\it et al.} (GEANT4 Collaboration),
\href{http://www.sciencedirect.com/science/article/pii/S0168900203013688}
{Nucl.\ Instrum.\ Methods\ Phys.\ Res., Sec.\ A {\bf 506}, 250 (2003)};
J.~Allison {\it et al.},
\href{http://ieeexplore.ieee.org/xpls/abs_all.jsp?arnumber=1610988\&tag=1}
{IEEE\ Trans.\ Nucl.\ Sci.\ {\bf 53}, 270 (2006)};
Z.~Y.~Deng  {\it et al.},
\href{http://hepnp.ihep.ac.cn/article/id/283d17c0-e8fa-4ad7-bfe3-92095466def1}
{Chin.\ Phys.\ C  {\bf 30}, 371 (2006)} (in Chinese).

\bibitem{kkmc} S.~Jadach, B.~F.~L.~Ward, and Z.~Was,
\href{http://journals.aps.org/prd/abstract/10.1103/PhysRevD.63.113009}
{Phys.\ Rev.\ D {\bf 63}, 113009 (2001)}.

\bibitem{evtgen} D.~J.~Lange,
\href{http://www.sciencedirect.com/science/article/pii/S0168900201000894}
{Nucl.\ Instrum.\ Methods Phys.\ Res., Sec.\ A {\bf 462}, 152 (2001)};
R.~G.~Ping,
\href{http://iopscience.iop.org/1674-1137/32/8/001}
{Chin.\ Phys.\ C {\bf 32}, 599 (2008)}.

\bibitem{pdg2022} R.~L.~Workman {\it et al.} (Particle Data Group),
\href{https://pdg.lbl.gov/}
{Prog. Theor. Exp. Phys. {\bf 2022}, 083C01 (2022)}.

\bibitem{lundcharm} J.~C.~Chen, G.~S.~Huang, X.~R.~Qi., D.~H.~Zhang, and Y.~S.~Zhu,
\href{https://journals.aps.org/prd/abstract/10.1103/PhysRevD.62.034003}
{Phys.\ Rev.\ D {\bf 62}, 034003 (2000)};
R.~L.~Yang, R.~G.~Ping and H.~Chen,
\href{https://iopscience.iop.org/article/10.1088/0256-307X/31/6/061301}
{Chin.\ Phys.\ Lett.\ {\bf 31}, 061301 (2014)}.

\bibitem{photos} E.~Barberio and Z.~Was,
\href{https://www.sciencedirect.com/science/article/pii/0010465594900744?via\%3Dihub}
{Comput.\ Phys.\ Commun.\ {\bf 79}, 291 (1994)}.

% \bibitem{crsDsDss} M. Ablikim {\it et al.} (BESIII Collaboration), Measuurement of the cross section for $e^+e^-\to D^{*\pm}_s D^{\mp}_s$ up to 4.7 GeV (to be published).

\bibitem{markiii} R.~M.~Baltrusaitis {\it et al.} (MARK III Collaboration),
\href{https://journals.aps.org/prl/abstract/10.1103/PhysRevLett.56.2140}
{Phys.\ Rev.\ Lett. {\bf 56}, 2140 (1986)};
J.~Adler {\it et al.} (MARK III Collaboration),
\href{https://journals.aps.org/prl/abstract/10.1103/PhysRevLett.60.89}
{Phys.\ Rev.\ Lett. {\bf 60}, 89 (1988)}.

\bibitem{zhangsf} M.~Ablikim {\it et al.} (BESIII Collaboration),
\href{https://journals.aps.org/prl/abstract/10.1103/PhysRevLett.122.071802}
{Phys.\ Rev.\ Lett. {\bf 122}, 071802 (2019)}.

\bibitem{keypdf} Kyle~S.~Cranmer,
\href{http://www.sciencedirect.com/science/article/pii/S0010465500002435}
{Comput.\ Phys.\ Commun.\ {\bf 136}, 198 (2001)}.

\bibitem{tmva3} A.~Hoecker {\it et al.},
{\it TMVA-Toolkit for Multivariate Data Analysis},
\href{https://arxiv.org/abs/physics/0703039v2}
{arXiv:physics/0703039}.

\bibitem{tmva1} C.~M.~Bishop,
{\it Pattern Recognition and Machine Learning},
\href{https://www.microsoft.com/en-us/research/uploads/prod/2006/01/Bishop-Pattern-Recognition-and-Machine-Learning-2006.pdf}
{Springer New York, NY (2006).}

\bibitem{tmva2} T.~Hastie,
{\it The Elements of Statistical Learning: Data Mining, Inference, and Prediction, Second Edition},
\href{https://hastie.su.domains/ElemStatLearn/printings/ESLII_print12.pdf}
{Springer New York, NY (2009).}

\bibitem{rbmethod} R.~Barlow,
{\it Systematic Errors: facts and fictions},
\href{https://arxiv.org/abs/hep-ex/0207026}
{arXiv:hep-ex/0207026}.

\bibitem{BES3GamSyst} M.~Ablikim {\it et al.} (BESIII Collaboration),
\href{https://journals.aps.org/prd/abstract/10.1103/PhysRevD.81.052005}
{Phys.\ Rev.\ D {\bf 81}, 052005 (2010)}.

\bibitem{flag2021} Y.~Aoki {\it et al.} (Flavour Lattice Averaging Group),
\href{https://link.springer.com/article/10.1140/epjc/s10052-022-10536-1}
{Eur.\ Phys.\ J.\ C {\bf 82}, (2022) 869}.

\end{thebibliography}

\end{document}