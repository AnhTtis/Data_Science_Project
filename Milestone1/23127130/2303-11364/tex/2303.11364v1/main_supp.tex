% CVPR 2022 Paper Template
% based on the CVPR template provided by Ming-Ming Cheng (https://github.com/MCG-NKU/CVPR_Template)
% modified and extended by Stefan Roth (stefan.roth@NOSPAMtu-darmstadt.de)

\documentclass[10pt,twocolumn,letterpaper]{article}

%%%%%%%%% PAPER TYPE  - PLEASE UPDATE FOR FINAL VERSION
% \usepackage[rdeview]{cvpr}       % To produce the REVIEW version
%\usepackage{cvpr}              % To produce the CAMERA-READY version
\usepackage{iccv}

% Include other packages here, before hyperref.
\usepackage{times}
\usepackage{graphicx}
\usepackage{epsfig}
\usepackage{svg}

\usepackage{amsmath}
\usepackage{amssymb}
\usepackage{booktabs}
\usepackage{epstopdf}
%%% What I add
\usepackage{soul}
\usepackage{multirow}
\usepackage{comment}
\usepackage{latexcolors}
\usepackage{subcaption}
\usepackage{textcomp}
\usepackage{siunitx}
\usepackage{tabulary}
\usepackage{float}

\makeatletter\if@twocolumn\PassOptionsToPackage{switch}{lineno}\else\fi\makeatother
% reduce caption size
\captionsetup{belowskip=1pt,aboveskip=1pt}
\usepackage[pagebackref,breaklinks,colorlinks]{hyperref}
\hypersetup{
    colorlinks=true,
    linkcolor=blue,
    urlcolor=magenta,
    citecolor=blue,
}

\usepackage{subcaption}
\usepackage{tabularx}
\usepackage[usenames,dvipsnames]{xcolor}
\usepackage{booktabs}  %
 %
\usepackage{fancyvrb}
\usepackage{fvextra}
%
\makeatletter
\@namedef{ver@everyshi.sty}{}
\makeatother
\usepackage{tikz}
\usetikzlibrary{shapes}

\usepackage[numbers,sort]{natbib}

%
%

\usepackage{xspace}
\def\eg{\emph{e.g.}\@\xspace}
\def\ie{\emph{i.e.}\@\xspace}
\def\etal{\emph{et al.}\@\xspace}
\def\cf{\emph{c.f.}\@\xspace}
\def\etc{\emph{etc}}

\renewcommand{\paragraph}[1]{\smallskip\noindent{\bf{#1}}}
%

\def\APall{mAP\textsubscript{all}\@\xspace}
\def\APrare{mAP\textsubscript{rare}\@\xspace}
\def\APcomm{mAP\textsubscript{comm}\@\xspace}
\def\APfreq{mAP\textsubscript{freq}\@\xspace}
\def\LVISminusrare{LVIS\textsubscript{-R}\@\xspace}

\def\zerorec{0Ways\@\xspace}
\def\onerec{1Ways\@\xspace}
\def\tworec{2Ways\@\xspace}
\def\threerec{3Ways\@\xspace}

\newcommand{\isArXiv}[2]{#1}
%



\iccvfinalcopy % *** Uncomment this line for the final submission
\def\iccvPaperID{6020} % *** Enter the ICCV Paper ID here
\def\httilde{\mbox{\tt\raisebox{-.5ex}{\symbol{126}}}}

% Pages are numbered in submission mode, and unnumbered in camera-ready
\ificcvfinal\pagestyle{empty}\fi


\begin{document}

%%%%%%%%% TITLE - PLEASE UPDATE
\title{\moniker{}: Multiple Image Haze Removal and 3D Shape Reconstruction using Neural Radiance Fields}

\author{
Wei-Ting Chen\textsuperscript{1,2*}\\
\and Wang Yifan\textsuperscript{1*}\\
\and Sy-Yen Kuo\textsuperscript{2} \\
\and Gordon Wetzstein\textsuperscript{1}\\
\and
\textsuperscript{1}Stanford University \hspace{1cm} \textsuperscript{2}National Taiwan University
}
\maketitle

%%%%%%%%% ABSTRACT
\begin{abstract}
Neural radiance fields (NeRFs) have demonstrated state-of-the-art performance for 3D computer vision tasks, including novel view synthesis and 3D shape reconstruction. Yet, these methods fail in adverse weather conditions that are crucial for applications such as autonomous driving. To address this challenge, we introduce DehazeNeRF as a framework that robustly operates in hazy conditions. \moniker{} extends the volume rendering equation by physically realistic terms that model atmospheric scattering. These act as inductive network biases in our pipeline and, together with several regularization strategies, allow DehazeNeRF to demonstrate successful multi-view haze removal, novel view synthesis, and 3D shape reconstruction where existing approaches fail.
\end{abstract}

\let\thefootnote\relax\footnotetext{* indicates equal contribution.}
\vspace{-10pt}
%%%%%%%%% BODY TEXT

\section{Introduction}
\label{section:introduction}
%% 1. why should someone care?

%The advent of advanced interactive computer vision systems~\cite{hololens} and recent progress in vision-language and multi-modal models~\cite{} opens doors for such next generation of assistive agents. 
% We envision that the future assistive agents would build up on these visual and language reasoning capabilities of today and empower users to achieve goals in their everyday lives. In particular, such agents would be able to reason about \emph{unseen} human goals... 
% We posit that such agents would require the ability to understand user goals described in natural language at high-level i.e., without complete details about as well as unseen user goals. 

%Recent progress in augmented reality systems~\cite{hololens, magicleap}, as well as vision-language and multi-modal models~\cite{}, opens doors for the next generation of assistive agents. 
Inspired by recent progress in visual systems~\cite{MagicLeap, ungureanu2020hololens}, we consider an assistive egocentric agent capable of reasoning about daily activities. When invoked via natural language commands, for e.g., while baking a cake, the agent understands the steps involved in baking, tracks progress through the various stages of the task, detects and proactively prevents mistakes by making suggestions. Such an agent would empower users to learn new skills and accomplish tasks efficiently.
% One could envision invoking such an agent merely through natural language descriptions of tasks similar to how present day assistants such as Alexa, Siri etc.~\cite{voice_assistants} are invoked. 
%We envision such agents to empower users in daily life by  invoking them naturally through 

%% 2. Why is it challenging? 
%While recent progress in vision-language and multi-modal models~\cite{} opens doors for such next generation of assistive agents, various challenges remain in making such agents a reality. 
%To make such agents a reality, 

Developing such an egocentric agent capable of tracking and verifying everyday tasks based on their natural language specification is challenging for multiple reasons. First, such an agent must reason about various ways of doing a \emph{multi-step} task specified in natural language. This entails decomposing the task into relevant actions, state changes, object interactions as well as any necessary causal and temporal relationships between these entities. Secondly, the agent must ground these entities in egocentric observations to track progress and detect mistakes. Lastly, to truly be useful, such an agent must support tracking and verification for a combination of tasks and, ideally, even unseen tasks. These three challenges -- causal and temporal reasoning about task structure from natural language, visual grounding of sub-tasks, and compositional generalization -- form the core goals of our work.

% %% 3. What are we doing? What is our approach?
% \aks{I think this is a matter of preference, but I personally don't like related work in intro. I would make this paragraph be about EgoTV and NSG. Starting with something like - "To this end, we propose...", ie, your next paragraph.}
% \nk{+1, we should move parts of this para to lit review and delete the rest.}
% Recent research on language modeling enables decomposing tasks into multiple steps from natural language descriptions~\cite{llm_zero_shot_planning,proscript}. However, such \emph{task decompositions} cannot directly be leveraged for task tracking in egocentric agents because of lack of grounding into the visual observations or context. In parallel, the computer vision community has advanced action recognition~\cite{}, object detection and tracking~\cite{}, hand object interaction and object state change detection~\cite{ego_4d,change_it,}, step classification in procedural tasks~\cite{}, and even vision language reasoning~\cite{nsvqa,nscl,star_situated_reasoning,clevrer}, which may help with the grounding challenge. However, majority of current research on identifying actions, objects, steps, or state changes does not account for the overall task structure. Likewise, predominant research on vision language understanding~\cite{} and multi-modal grounding~\cite{} does not consider the temporal and causal constraints that emerge in task tracking and verification. We therefore focus on the order-aware visual grounding problem in our work, with an eye towards compositional generalization to scale usability of these agents. In particular, we aim to achieve visual grounding of the actions and objects corresponding to each step or sub-task obtained from the task description decomposition in an order-aware manner.

%% 4. What are our results/contributions?
As our first contribution, we propose a benchmark -- \emph{\textbf{Ego}centric \textbf{T}ask \textbf{V}erification} (\etv \inlineimg{figures/TV}) -- and a corresponding dataset in the AI2-THOR~\cite{ai2thor} simulator. % \emoji{tv}
Given a natural language (NL) task description and a corresponding egocentric video of an agent, the goal of \etv is to verify whether the task was successfully completed in the video or not.
\etv contains multi-step tasks with \emph{ordering} constraints on the steps and \emph{abstracted} NL task descriptions with omitted low-level task details inspired by the needs of real-world assistants. We also provide splits of the dataset focused on different generalization aspects, e.g., unseen visual contexts, compositions of steps, and tasks (see Figure~\ref{figure:dataset}).
% Next, we create splits of the dataset focused on different aspects of generalization, ranging from generalization to unseen visual context to unseen compositions of steps and tasks. Figure~\ref{figure:dataset} shows an example task and overview of generalization splits from \etv. Succeeding at \etv tasks requires decomposing tasks into partially-ordered steps from the NL description and order-aware visual grounding of these steps into the video. 

Our second contribution is a novel approach for order-aware visual grounding~--~\emph{\textbf{N}euro-\textbf{S}ymbolic \textbf{G}rounding} (NSG), capable of compositional reasoning and generalizing to unseen tasks owing to its ability to leverage abstract NL descriptions and compositional structure of tasks (task decomposition, ordering).~In contrast, state-of-the-art vision-language models~\cite{coca,clip,videoclip,clip_hitchiker} struggle to ground NL descriptions in egocentric videos, and do not generalize to unseen tasks.~NSG outperforms these models by~$\mathbf{33.8}\%$~on compositional generalization and~$\mathbf{32.8}\%$~on abstractly described task verification. Finally, to evaluate \nsg on real-world data, we instantiate \etv on the CrossTask~\cite{cross_task} instructional video dataset. %Specifically, we synthetically create videos with mistakes in CrossTask. 
We find that it also outperforms state-of-the-art models at task verification on CrossTask. We hope that the \etv~benchmark and dataset will enable future research on egocentric agents capable of aiding in everyday tasks.

% We experiment with many for the \etv tasks. We find that while these models generalize well to unseen visual context, they struggle to perform grounding from abstracted task descriptions and to generalize to new compositions of tasks. To deal with these challenges, we take inspiration from recent research on and develop . ~\rd{unclear why neurosymbolic models would do well on abstraction.} 

% To summarize, our main contributions are:~1)~\etv: a benchmark and synthetic dataset to systematically study egocentric task verification.
% 2)~\nsg: a novel neuro-symbolic approach to enable the core reasoning capability for \etv -- order-aware visual grounding. We demonstrate \nsg's capability on our synthetic \etv dataset as well as a real-world dataset derived from CrossTask. We will release both of these datasets and our models for future research on egocentric task tracking and verification. 


% Assistive agents require the ability to track actions and state changes from an egocentric perspective for effective assistance in day-to-day tasks. For example, an agent helping a user prepare a recipe would need to both generate the steps of the recipe (\textit{plan generation}) and track the user's actions to ensure the plan is executed correctly (\textit{plan verification}). We formulate this as a Video Entailment task~\cite{violin_dataset,9710490} \rd{should we call our task video-based goal entailment?}, wherein, given an egocentric video of an agent (or human) performing a task (\textit{premise}) and a NL task description (\textit{hypothesis}), the objective is to learn a model to track whether the given task was successfully executed in the video. 
% An ideal model should also be able to seamlessly generalize to novel compositions (of actions and objects) unseen during training. \rd{add a line about what we mean by abstraction and why is it important.} To this end, we generate a novel Vision-Language dataset on the AI2-THOR simulator~\cite{ai2thor} to study compositional and abstraction-based generalization. Our dataset provides effective evaluation measures in a controlled setting, while closely reflecting the diversity of real-world events. We implement and train a variety of end-to-end models based on existing state-of-the-art approaches. We empirically demonstrate that neural models suffer from overfitting and cannot effectively generalize to novel compositions of actions, objects, and scenes. 
% To address this problem, we propose an end-to-end Neuro-Symbolic (NeSy) framework that performs plan generation and verification. At the heart of our approach is the hypothesis that symbolic reasoning models are good at generalization and capturing compositional substructure, while neural models are good at learning representations from sensory data~\cite{10.5555/3326943.3327039,nscl,clevrer}. \rd{summarize contributions in a bulleted list.} \rd{also add a line about the main result e.g., x\% improvement as compared to end-to-end models}. 

% \rd{we also evaluate NeSy with real-world data: add briefly about CrossTask experiments.}

% % \fbox{\begin{minipage}{\linewidth}
% % \textbf{Problem Statement}

% % Given:
% % (i) Premise: Egocentric video of an agent performing a task.
% % (ii) Hypothesis: NL description of the task.

% % Learn: A model to track whether the premise entails the hypothesis. The output of the model is True if the given task is executed successfully in the video.
% % \end{minipage}}

% \textbf{Contributions:} 
% \begin{itemize}
%     \item We generate a benchmark video-language dataset to study compositional and abstraction-based generalization.
%     \item We evaluate the performance of a variety of state-of-the-art models and show that these (baseline) models cannot effectively generalize to novel compositions of actions.
%     \item We propose a novel end-to-end NeSy approach that significantly outperforms the baselines on some compositional generalization splits while performing on par with them on the rest.
%     \item We also evaluate our NeSy approach with real-world data showing similar performance improvements.
% \end{itemize}


\section{Related Surveys}
\label{sec-relat}

There are many research directions relating to RISs, including channel modelling and estimation, signal processing, performance analysis, passive beamforming, and hardware designs. This work focuses on optimization techniques due to their paramount importance, and Table \ref{tab1} compares this work with existing surveys in terms of control and optimization-related contributions. 

Table \ref{tab1} shows that most existing works focus on model-based approaches, including AO, MM, SCA, and SDR. The main reason is that these techniques have been widely applied, e.g., using AO to decouple joint active and passive beamforming, and applying MM and SCA to approximate non-convex objectives. Then, heuristic algorithms are usually considered as low-complexity alternatives and supplements. For example, greedy algorithms are used for element-by-element RIS phase-shift control, and matching theory is applied for resource allocation. However, despite their importance, heuristic approaches are omitted in many existing surveys. Meanwhile, ML algorithms have been widely used for wireless network management, but existing surveys are limited in supervised learning and RL. In addition, some newly emerging techniques, such as graph learning and hierarchical learning, are not mentioned in existing surveys.       

More specifically, in many existing studies \cite{Almo,gong,moha,mohadz}, optimization techniques are very briefly discussed by introducing the algorithm titles that have been used in the literature, but the motivations and algorithm features are not included.  
Alghamdi \textit{et al.} overviewed optimization and performance analysis techniques of RISs, but it is limited in analyzing problem formulations \cite{rawa}. 
In \cite{kfai}, Faisal and Choi specialized in ML approaches for RIS-aided wireless networks, but model-based and heuristic approaches are not included. Besides, some state-of-the-art ML techniques, including graph learning and hierarchical learning, are not included in \cite{kfai}.  
By contrast, multiple model-based approaches are introduced in \cite{cunh} for signal processing of RISs, but many heuristic and ML techniques are not covered.    
Liu \textit{et al.} presented RIS beamforming, resource management and ML for RIS-aided wireless networks, but only RL is presented in detail \cite{yliu}. Supervised learning, unsupervised learning, and FL are briefly discussed in \cite{yliu}, while newer techniques, such as graph learning, transfer learning, and hierarchical learning, are not covered. In \cite{zheng2022survey}, Zheng \textit{et al.} surveyed the channel estimation and practical RIS control under imperfect/statistical/hybrid CSI, but some optimization techniques are not included. 




This work is different from existing studies in the following aspects: 
\begin{itemize}
    \item  Control and optimization have been included in many surveys, but this work is the first to systematically investigate optimization techniques of RIS-aided wireless networks, ranging from problem formulations to steps, features, advantages, and difficulties of nearly 20 techniques.    
    \item We present in-depth analyses to apply these optimization techniques to RISs. For example, deep neural network (DNN) and deep reinforcement learning (DRL) are included in many existing surveys, but some important questions are not discussed, i.e., dataset acquisition for neural network training in RIS-aided environments, and customizing the state, action, and reward function definitions for RL-enabled RIS control. The answers to these questions are critical to taking full advantage of RISs.
    \item Finally, we present the most state-of-the-art ML techniques for optimizing RIS-aided wireless networks, e.g., graph learning, transfer learning, and hierarchical learning, which are not included in existing surveys, to the best of our knowledge. These novel techniques may bring new research directions.  
\end{itemize}
To summarize, this survey answers the following: what are the state-of-the-art techniques for optimizing RIS-aided wireless networks, and how do they cover different aspects with respect to each other? 


\section{Method}
Our method, {\moniker}, extends the volume rendering equation to accurately reconstruct the geometry and appearance robust to hazy conditions.
Our key idea is to introduce a series of important biases in the network architecture along with regularizers in the loss function that together underpin physically based scattering phenomena.

\subsection{Preliminary on Neural Radiance Fields}\label{sec:nerf}
Neural Radiance Fields (NeRFs)~\cite{mildenhall2020nerf} map a 3D sample point \(\p\) into a color $\mathbf{c}$ and volume density $\sigma$.
Considering only emission from classic volume rendering~\cite{kajiya1984ray,tagliasacchi2022volume}, the expected color ${C}(\r)$ of a camera ray $\r(t)=\mathbf{o} + t\mathbf{d}$ with the near and far boundary $t_n$ and $t_f$ can be written as
\begin{gather}
	{C}(\r, \mathbf{d})=\int_{t_n}^{t_f}T(t)\sigma(\r(t))c(\r(t), \mathbf{d}) \ dt \;\textrm{with} \label{eq:nerf}\\
    T(t)=\mathrm{exp}\left( - \int_{t_n}^{t}\sigma(\r(t')) \ dt'\right),
	\label{eq:occlusion}
\end{gather}
where \(T(t)\) is the accumulated transmittance between the ray section \(t_{n}\) to \(t \).
The predicted pixel value is then compared to the ground truth $\widehat{C}(\r,\d)$ for optimization.

\subsection{3D Haze Formation}\label{sec:rte_haze}
To address the 3D dehazing problem, we propose an alternative rendering equation to the image formation model.
We start from the radiative transfer equation (RTE)~\cite{chandrasekhar2013radiative,van1999multiple}, which describes the behavior of light in a medium that absorbs, scatters and emits radiation.
Assuming, a ray \(\r\left( t \right) = \mathbf{o} + t\d\) hits a surface point at \(\r\left( t_{0} \right)\), the incident radiance at the near image plane \(t_{n}\) can be divided into three parts~\cite{pharr2016physically}:
{\small
\begin{align}
C(\r, \d) &= C_{\textrm{emission}}(\r) + C_{\textrm{surface}}(\r) + C_{\textrm{in-scattering}}(\r)\nonumber\\
C_{\textrm{emission}}(\r, \d) &=
\int_{t_{n}}^{t_{0}}\epsilon\left(\r\left( t\right),\d\right)T_{\sigma_{t}}\left( t\right)dt\nonumber\\
C_{\textrm{surface}}(\r, \d) & =C_e\left(\r\left( t_{0} \right),\d\right)T_{\sigma_{t}}\left( t_{0}\right)\nonumber\\
C_{\textrm{in-scattering}}(\r, \d) &=
\int_{t_{n}}^{t_{0}}c_{\textrm{s}}\left( \r\left( t \right), \d \right)\sigma_{s}\left(\r\left( t \right)\right)T_{\sigma_{t}}\left( t \right)dt,\nonumber
\end{align}
}
where \(\epsilon\) is the emission, \(C_{e}\) is the outgoing radiance at the surface intersection, \(c_{\textrm{s}}\left(\r\left( t \right), \d \right)\) is the in-scattered light and \(\sigma_{s}\) is the scattering coefficient.
In particular, the transmittance here is computed from the attenuation coefficient \(\sigma_{t}\), \ie,
\(T_{\sigma_{t}}\left( t\right)=\exp\left( -\int_{t_{n}}^{t}\sigma_{t}(t')dt' \right)\),
where \(\sigma_{t}=\sigma_{a} + \sigma_{s}\) including the absorption and out-scattering effect.
For common haze formation, the participating particles are considered non-luminous~\cite{narasimhan2003contrast}, therefore we can drop the emission part, which leads to
{
\small
\begin{align}
\begin{split}
C(\r,\d)= {} & C_e(\r\left( t_{0} \right),\d)T_{\sigma_{t}}\left( t_{0} \right)+\\
&\int_{t_{n}}^{t_{0}}c_{\textrm{s}}\left( \r\left( t \right), \d \right)\sigma_{s}\left(\r\left( t \right)\right)T_{\sigma_{t}}\left( t \right)dt.
\end{split}
\label{eq:RTE_Haze}
\end{align}
}

Following NeRF~\cite{mildenhall2020nerf}, we represent the surface as a continuous density field with emission \(\epsilon\left(\r\left(t\right), \d\right)\coloneqq c\left(\r\left( t \right),\d\right)\sigma\left(\r\left( t \right)\right)\).
Meanwhile, the absorption part in the attenuation \(\sigma_{t}\) can be interpreted as the surface density \(\sigma\), since the volume density $\sigma$ is equal to absorption coefficient $\sigma_{a}$ in that they both determine the probability of a photon or a ray terminating at a given location.
As a result, we can write the rendering equation as
{\small
\begin{align}\begin{split}
    C(\r,\d)=&
    \underbrace{\int_{t_{n}}^{t_{0}}c(\r(t),\d)\sigma(t)T_{\sigma+\sigma_{s}}\left( t \right)dt}_{C_{\textrm{Surface}}} +\\
    &\underbrace{\int_{t_{n}}^{t_{0}}c_{s}(\r(t))\sigma_{s}(t)T_{\sigma+\sigma_{s}}\left( t \right)dt}_{C_{\textrm{Haze}}}.
    \label{eq:3D_haze_formation}
\end{split}
\end{align}
}
\cref{eq:3D_haze_formation} formally disentangles the surface and haze, represented by \(\left\{ c, \sigma \right\}\) and \(\left\{  c_{s}, \sigma_{s} \right\}\) respectively, in a principled manner.
Once successfully optimized (see the next Section), the clear-view surfaces can be recovered using \(\left\{ c, \sigma \right\}\):
\begin{equation}
C(\r,\d)=
\int_{t_{n}}^{t_{0}}c(\r(t),\d)\sigma(t)T_{\sigma}\left( t \right)dt\label{eq:clear_view}.
\end{equation}

\begin{figure}[t!]
\centering \includegraphics[width=\linewidth]{images/architecture.pdf}
\makeatother
\caption{\textbf{\moniker{} architecture.} Given a set of hazy images, our method augments the existing NeRF pipeline (gray) with a haze module (yellow), which explicitly models the scattering phenomenon using atmospheric light and scattering coefficient. During training, we render the hazy reconstruction as a composition of surface and haze, which is compared to the input hazy images to optimize the learnable parameters (in green) jointly. During inference, we use the surface module (gray) to render clear views.}
\vspace{-0.5cm}\label{fig:architecture}
\end{figure}

\subsection{Haze-aware Neural Radiance Field}\label{sec:dehaze_nerf}
Given multiple images of a hazy scene, we aim to jointly optimize for the surface appearance and geometry, \(\left\{ c, \sigma \right\}\) as well as the haze's scattering coefficient and in-scattered light (atmospheric light),  \(\left\{c_{s}, \sigma_{s} \right\}\) based on the enhanced scattering-aware rendering equation~\cref{eq:3D_haze_formation}.
However, the effects of these variables are interdependent. In order to correctly disentangle them, our model adopts suitable architecture designs and training regularizers to capture the distinct physical properties of haze and surface.
An overview of \moniker{} is illustrated in \cref{fig:architecture}.

\paragraph{Architecture.} Now we introduce inductive biases to match the physical properties of haze and surface.
For clarity, we highlight the quantities directly modeled by neural networks in \nn{green}.

\emph{Modeling a Surface.} Recall our goal is to learn the surface appearance and geometry, \(\left\{ c, \sigma \right\}\).
Similar to previous works~\cite{mildenhall2020nerf}, we model the appearance \(\cnet\left( \p, \d \right)\) with an MLP, which takes the sample location \(\p\) and viewing direction \(\d\) as inputs.
However, in order to encourage volume density \(\sigma\) to form a well-defined solid surface, instead of directly learning the volume density, we adopt the reparameterization of the volume density using signed distance function (SDF), \(\sdf\left( \r\left( t \right) \right)\in \R\), as proposed in NeuS~\cite{wang2021neus,wang2022hfs}.
The modified surface volume density \(\sigma\left( \r\left( t \right) \right) \), referred to as opaque density, can be parameterized as \(\sdf\left( \r\left( t \right) \right)\):
\begin{equation}
\sigma\left( \r\left( t \right) \right) = s\left( \Phi_{s}\left( \sdf\left( \r\left( t \right) \right)\right) -1 \right)\nabla \sdf\left( \r\left( t \right) \right)\mathbf{d},\label{eq:hfneus-sigma}
\end{equation}
where $\Phi_{s}(x)$ is the sigmoid function $\Phi_s(x) = (1 + e^{-sx})^{-1}$, whose derivative is a bell-shaped density function centered at 0 and has a learnable standard deviation of \(\nicefrac{1}{s}\).
We derive the discrete approximate following~\cite{mildenhall2020nerf,tagliasacchi2022volume}.
It samples $n$ points $\left\{ \p_{i}=\mathbf{o}+t_n\mathbf{d}|n=1,...,N,t_n<t_{n+1} \right\}$ along the ray.
The approximate pixel color of the ray is computed based on quadrature rule~\cite{max1995optical}, yielding
\begin{align}\begin{gathered}
C_{\textrm{surface}}(\r,\d) = \sum_{n=1}^{N}\frac{\sigma^{n}}{\sigma_{t}^{n}} T_{t}^{n}\alpha_{t}^{n}\nn{c}^{n} \textrm{ with } T_{t}^{n}=\prod_{m=1}^{n-1}\left(1 - \alpha_{t}^{m}\right) \label{eq:C_surface},
\end{gathered}\end{align}
where \(\alpha_{t}\) denotes the discrete \(\alpha\)-compositional weight defined as~\cite{wang2021neus,wang2022hfs}
\begin{equation}
 \resizebox{1\hsize}{!}{
 $
    \alpha_{t}^{n}=\textsc{clamp}\left( 1-\exp\left( -\sigma_{t}^{n}\delta^{n} \right),0, 1 \right) \textrm{ with } \delta^{n}=t^{n+1}-t^{n}\label{eq:alpha}\nonumber,$}
\end{equation}
where \(\sigma_{t}^{n}=\sigma^{n}+\nn{\sigma_{s}}^{n}\) denotes the total attenuation at sample \(n\), including the attenuation due to surface occlusion and the out-scattering.

\emph{Modeling Haze.} We use a low-frequency prior to compute the scattering coefficient and atmospheric light, \(\left\{c_{s}, \sigma_{s} \right\}\), since these components usually vary slowly in a common hazy scenes~\cite{li2015simultaneous}.
In practice, we use a small band-limited \textsc{MLP}~\cite{lindell2022bacon} for the scattering coefficient \(\sigma_{s}\) to capture inhomogenous haze.
Analogous to \cref{eq:C_surface}, the haze color can be approximated as
% \begin{equation}
% \begin{gathered}
% C_{\textrm{haze}}(\r) = \sum_{i=1}^{n}\frac{\nn{\sigma_{s}}^{n}}{\sigma_{t}^{n}} T_{t}^{n}\alpha_{t}^{n}\nn{c_{s}}^{n}.\label{eq:C_haze}
% \end{gathered}
% \end{equation}
\begin{equation}
\begin{gathered}
C_{\textrm{haze}}(\r) = \sum_{n=1}^{N}\frac{\nn{\sigma_{s}}^{n}}{\sigma_{t}^{n}} T_{t}^{n}\alpha_{t}^{n}\nn{c_{s}}^{n}.\label{eq:C_haze}
\end{gathered}
\end{equation}
During optimization, the color for an arbitrary input hazy image can be written as $C = C_{\textrm{surface}} + C_{\textrm{haze}}$.
At test time, we can reconstruct the clear-view color by discretizing \cref{eq:clear_view}, namely:
\begin{gather}
 C_{\textrm{clear}}\left( \r,\d \right) = \sum_{n=1}^{N}T_{\sigma}^{n}\alpha^{n} \nn{c}^{n}, \label{eq:clear_view_discrete}\\
 \resizebox{1\hsize}{!}{
 $T_{\sigma}^{n} = \prod_{j=1}^{n-1}\left(1 - \alpha^{j}\right)\, \textrm{and }\, \alpha^{n} = \textsc{clamp}\left( 1 - \exp\left( -\sigma^{n}\delta^{n} \right),0, 1 \right).\nonumber$}
\end{gather}
\paragraph{Optimization.} While the inductive biases separate the high-frequency surface appearance and geometry from the low-frequency color and density of the scattering medium, we introduce further regularizers to guide the optimization process to converge to more plausible clear-view geometry and color.

\emph{Koschmieder Consistency.}
Given an accurate depth map \(D\), assuming globally constant scattering coefficient \(\bar{\sigma}_{s}\) and airlight \(\bar{c}_{s}\), the relation between a clear-view image \(C_{\textrm{clear}}\) and the hazy image \(C\) can be described by the Koschmieder law~\cite{israel1959koschmieders} as
\begin{equation}
\resizebox{0.88\hsize}{!}{
\(C(\r)=C_{\textrm{clear}}(\r)\exp(-\bar{\sigma}_{s} D(\r))+\bar{c}_{s}(1-\exp(-\bar{\sigma}_{s} D(\r)))\).
}\label{eq:koschmieder}
\end{equation}
This model is widely adopted as the basis for image-based single and multiview dehazing.
The Koschmider model is an approximation of our rendering equation~\cref{eq:3D_haze_formation} under the assumption of
spatially-invariant (i.e., homogeneous) scattering coefficient and an ideal surface
\begin{align}
C_{\textrm{surface}}\left( \r \right) & \approx C_{\textrm{clear}}(\r)\exp(-\bar{\sigma}_{s} D(\r)) = \tilde{C}_{\textrm{surface}}\left( \r \right)\\
C_{\textrm{haze}}\left( \r \right) & \approx \bar{c}_{s}(1-\exp(-\bar{\sigma}_{s} D(\r)) = \tilde{C}_{\textrm{haze}}\left( \r \right),
\end{align}

We promote this relation with
%
\begin{align}
&\loss_{\textrm{2D}} = \left\|C_{\textrm{surface}}\left( \r \right) -  \tilde{C}_{\textrm{surface}}\left( \r \right)\right\|_{1} \\+
&\left\| C_{\textrm{haze}}\left( \r \right)\! - \!\tilde{C}_{\textrm{haze}}\left( \r \right)\right\|_{1} \!\!+\!
 \left\| C\! -\! \tilde{C}_{\textrm{surface}}\left( \r \right)\! -\! \tilde{C}_{\textrm{haze}}\left( \r \right)\right\|_{1}\!, \nonumber
\end{align}
%
where \(\bar{\sigma}_{s}\) and \(\bar{c}_{s}\) are the average over the samples on the ray, while
the depth value \(D\left( \r \right)\) is computed via the learned surface geometry~\cite{mildenhall2020nerf,yu2022monosdf} by accumulating over ray-length over all the samples on a ray:
\begin{equation}
    D\left( \r \right) = \sum_{n=1}^{N} T_{\sigma}^{n}\alpha^{n}t^{n}.
\end{equation}

\emph{Color Prior.}
Without knowing the original image, the heavily attenuated color in the hazy image can be explained by the haze but also by a dull surface color.
In order to reconstruct plausible clear-view colors, we adopt the popular 2D prior widely used in image-based dehazing methods, Dark Channel Prior (DCP)~\cite{he2010single}, which arises from the observation, that for most pixels in a natural haze-free image, the minimum of three color channels is close to zero.
We apply this prior to the estimated clear image \(C_{\textrm{clear}}\)
\begin{align}
DC(C_{\textrm{clear}})\left(\x\right)&=\underset{\y\in\Omega\left(\x\right)}{\min}\left(\underset{c\in\left\{r,g,b\right\}}{\min}C_\textrm{clear}^{c}\left(\y\right)\right),
\label{eq:DCP_definition}\\
\loss_{\textrm{dcp}}&=\frac{1}{K}\sum\limits_{k=1}^{K}\Vert DC\left(C_{\textrm{clear}}\right)\Vert_{1}.
\label{eq:loss_dcp}
\end{align}


\subsection{Implementation Details}
We adopt the same setting as that in HF-NeuS~\cite{wang2021neus} wherever possible.
This includes the MLPs for the surface SDF, \(\sdf\) and the view-dependent surface color, \(\cnet\), as well as the sampling strategy, the background composition, and learning rate schedule.

\paragraph{Loss.}
Our loss is composed of several terms:
\begin{equation}
    \loss = \loss_{\textrm{color}} + \lambda\loss_{\textrm{eikonal}} + \alpha\loss_{\textrm{dcp}} + \beta\loss_{\textrm{2D}},\label{eq:total_loss}
\end{equation}
where \(\loss_{\textrm{dcp}}\) and \(\loss_{\textrm{2D}}\) are the regularizations introduced in \cref{sec:dehaze_nerf},
while the photo-consistency loss, $\loss_{\textrm{color}}$, is the standard NeRF loss, and the eikonal loss, \(\loss_{\textrm{eikonal}}\), is commonly used to regularize SDF~\cite{gropp2020implicit},
\begin{align}
    \loss_{\textrm{color}}& = \frac{1}{K}\sum_{k=1}^{K}\left\|\widehat{C}_{k}(\r,\d) - C_{k}(\r,\d)\right\|_{1},\\
    \loss_{\textrm{eikonal}} &= \frac{1}{KN}\sum_{k}^{K}\sum_{n}^{N}(\|\nabla f({\mathbf{r}}_{k}(t_n))\|_2 - 1)^2,
\label{eq:loss_color}
\end{align}
where $\widehat{C}_{k}(\r,\d)$ is the pixel color. $N$ and $K$ denote the total sampling points on a ray and the total number of rays sampled per training batch.

Finally, because of the surface representation using SDF, we can optionally adopt the object masks for supervision~\cite{yariv2021volume,wang2021neus,wang2022hfs}.
Specifically, given the object mask, \(M\), the mask loss $\loss_{\textrm{mask}}$ for a sampled ray $k$ is defined as
\begin{equation}
    \loss_{\textrm{mask}} = \text{BCE}(M_k, \hat{O}_k),\label{eq:mask_loss}
\end{equation}
where $\hat{O}_k = \sum_{i=1}^{N}T_{\sigma}^{i}\alpha^{i}$ is the total weight for the clear-view surface color along the camera ray, and $\text{BCE}$ is the binary cross entropy loss.


\begin{figure*}[htbp]
\setlength{\tabcolsep}{0pt}
\renewcommand{\arraystretch}{0.8}\footnotesize
\centering\begin{tabular}{*{6}{>{\centering\arraybackslash}M{0.166\textwidth}}}
NeuS & COLMAP + NeuS &ImDehaze+NeuS& VidDehaze + NeuS & \moniker{} & Ground Truth  \\
\includegraphics[width=\linewidth]{images/qualitative/syn_new/scan24/scan24_neus.jpg}&
\includegraphics[width=\linewidth]{images/qualitative/syn_new/scan24/scan24_colmap.jpg} &
\includegraphics[width=\linewidth]{images/qualitative/syn_new/scan24/scan24_dehaze.jpg}&
\includegraphics[width=\linewidth]{images/qualitative/syn_new/scan24/scan24_video.jpg}&
\includegraphics[width=\linewidth]{images/qualitative/syn_new/scan24/scan24_ours.jpg}&
\hspace*{+0.4cm}\includegraphics[width=\linewidth]{images/qualitative/syn_new/scan24/scan24_gt.jpg}
\\
\includegraphics[width=\linewidth]{images/qualitative/syn_new/scan97/scan97_neus.jpg}&
\includegraphics[width=\linewidth]{images/qualitative/syn_new/scan97/scan97_colmap.jpg} &
\includegraphics[width=\linewidth]{images/qualitative/syn_new/scan97/scan97_dehaze.jpg}&
\includegraphics[width=\linewidth]{images/qualitative/syn_new/scan97/scan97_video.jpg}&
\includegraphics[width=\linewidth]{images/qualitative/syn_new/scan97/scan97_ours.jpg}&
\hspace*{-0.2cm}\includegraphics[width=\linewidth]{images/qualitative/syn_new/scan97/scan97_gt.jpg}
\\
\includegraphics[width=\linewidth]{images/qualitative/syn_new/scan110/scan110_neus.jpg}&
\includegraphics[width=\linewidth]{images/qualitative/syn_new/scan110/scan110_colmap.jpg}&
\includegraphics[width=\linewidth]{images/qualitative/syn_new/scan110/scan110_dehaze.jpg}&
\includegraphics[width=\linewidth]{images/qualitative/syn_new/scan110/scan110_video.jpg}&
\includegraphics[width=\linewidth]{images/qualitative/syn_new/scan110/scan110_ours.jpg}&
\hspace*{+0.6cm}\includegraphics[width=\linewidth]{images/qualitative/syn_new/scan110/scan110_gt.jpg}
\\

\includegraphics[width=\linewidth]{images/qualitative/syn_new/scan118/scan118_neus.jpg}&
\includegraphics[width=\linewidth]{images/qualitative/syn_new/scan118/scan118_colmap.jpg}&
\includegraphics[width=\linewidth]{images/qualitative/syn_new/scan118/scan118_dehaze.jpg}&
\includegraphics[width=\linewidth]{images/qualitative/syn_new/scan118/scan118_video.jpg}&
\includegraphics[width=\linewidth]{images/qualitative/syn_new/scan118/scan118_ours.jpg}&
\hspace*{+0.9cm}\includegraphics[width=\linewidth]{images/qualitative/syn_new/scan118/scan118_gt.jpg}\\
\end{tabular}
\caption{\textbf{Qualitative comparison on synthetic data.} Our method successfully removes heterogeneous haze in the synthesized views, showing the best appearance fidelity compared with baseline methods. The reconstructed geometry is more accurate, less noisy, and contains more details.
}\vspace{-0.5cm}
\label{fig:dtu_qualitative}
\end{figure*}

\section{Experiments with Synthetic Scenes}
In this section, we detail our experiments using synthetic data.
Our goal is to quantifiably evaluate the contribution of each proposed component in a controlled setting.
We report our main findings here and refer the readers to the supplement for more detailed evaluations.
\subsection{Data Preparation}
We synthesize haze using 10 scenes from the DTU dataset~\cite{jensen2014large}.
% The haze synthesis follows the 2D haze generation method~\cite{li2018benchmarking} based on Koschmieder's model~\cref{eq:koschmider}.
% The depth used for haze generation is computed using the meshes reconstructed by public available NeuS models~\cite{wang2021neus}.
% We set the atmospheric light and scattering coefficient to a constant value sampled in range $\sigma_s\in[0.2, 0.6]$ and $c_{s}\in[0.5, 0.9]$.\todo{change to Gaussian blob}
The scattering coefficient is modeled using the sum of 4 scaled Gaussian blobs located inside the spatial bounding box with a standard deviation uniformly sampled from 1.0 to 3.0;
the 3-D atmospheric light is sampled from a uniform distribution in the range $[0.7, 0.9]$.
10\% of the images in each synthetic scene are held out as the test set.

\subsection{Comparisons}\label{sec:comparisons}
\paragraph{Baselines.}
We compare {\moniker} with the following baselines:
%
\begin{compactenum}
\item \textbf{NeuS}: train HF-NeuS~\cite{wang2022hfs} on hazy images,
\item \textbf{ImDehaze+NeuS}: train HF-NeuS on dehazed images obtained using the state-of-the-art single-image dehazing method~\cite{guo2022image},
\item \textbf{VidDehaze+NeuS}: train HF-NeuS on dehazed images obtained using the state-of-the-art video dehazing method~\cite{zhang2021learning},
\item \textbf{COLMAP+NeuS}: train HF-NeuS on dehazed images obtained  by estimating the transmission maps using the dense depth map from COLMAP~\cite{schoenberger2016sfm,schoenberger2016mvs}.
\end{compactenum}\label{lst:baseline5}
%
With HF-NeuS as the backbone surface model~\cite{wang2022hfs}, all approaches observe the same surface prior.
While the first baseline neglects haze entirely, baselines 2 to 4 increasingly incorporate more multiview information for haze modeling, with ours being the most 3D-aware and physically accurate, as it models the spatial-variant scattering coefficient in 3D space and optimizes the 3D geometry, surface appearance, and the haze parameters jointly.
For the last baseline, we use the method proposed by~\cite{he2010single} to estimate the global airlight from the object regions (masked by~\cite{yariv2020multiview}) in all images.
Then we use 300 pairs of feature correspondences to estimate the global scattering coefficient, where each pair computes a candidate scattering coefficient as follows
\(\frac{1}{D_{b}\left( \x_{b} \right) - D_{a}\left( \x_{a} \right)}\ln\left( \frac{I_{a}\left( \x_{a} \right) - \bar{c}_{s}}{I_{b}\left( \x_{b} \right) - \bar{c}_{s}} \right)\).
\(\left( I_{a}, D_{a} \right)\) and \(\left( I_{b}, D_{b} \right)\) are RGB images and depth maps in two views, and \(\x_{a}\) and \(\x_{b}\) are the image coordinates of a pair of matched SIFT features.
The final result is obtained after filtering out negative or invalid estimations, which may occur due to specularities and noisy depth estimation.

\noindent\textbf{Qualitative Evaluation.}
We demonstrate some examples of the dehazed results and reconstructed geometry in \cref{fig:dtu_qualitative}.
Despite having a surface prior, naïvely training HF-NeuS directly from hazy images is equivalent to averaging the scattering-induced geometry-dependent irradiance variance observed across different views and attributing it the surface color.
Consequently, the view synthesis is hazy and blurred.
For two-stage strategies, the rendered results have color distortion of various degrees, as indicated by the PSNR evaluation in \cref{tab:dtu_quantitative}, since it is difficult to accurately estimate the airlight and coefficient when the presented data does not comply with the specific assumptions or fall in the distribution of the training data.
Moreover, our method clearly reconstructs the geometry with more surface details compared to all other baselines that adopt the surface prior formulated in NeuS, since our method can dehaze different views more consistently thanks to the underlying geometry that is optimized jointly.

\noindent\textbf{Quantitative Evaluation.} We measure the image quality using peak signal-to-noise ratio (PSNR), structural similarity (SSIM), and perceptual similarity (LPIPS).
The geometry quality is measured using Chamfer Distances (CD) using the DTU standard protocol.
As shown in \cref{tab:dtu_quantitative},
\moniker~achieves the best results compared to all baselines,
with superior performance in PSNR, a metric sensitive to low-frequency color shift.
This indicates that while other methods struggle to estimate the true air light and scattering coefficient from either statistical or data priors, our method benefits from jointly optimizing these quantities along with the scene appearance and geometry.



\begin{figure}
    \centering
    \begin{subfigure}[b]{0.155\textwidth}
        \centering
        \includegraphics[width=\textwidth]{images/ablation/3D/ablation-nodc.png}{}
        \caption{Without $\loss_{\textrm{dcp}}$}
        \label{fig:ablation_dcp_baseline}
    \end{subfigure}
    \begin{subfigure}[b]{0.155\textwidth}
        \centering
        \includegraphics[width=\textwidth]{images/ablation/3D/ablation-dc.png}{}
        \caption{With $\loss_{\textrm{dcp}}$}
        \label{fig:ablation_dcp_ours}
    \end{subfigure}
        \begin{subfigure}[b]{0.155\textwidth}
        \centering
        \includegraphics[width=\textwidth]{images/ablation/3D/ablation-gt.png}{}
        \caption{Ground Truth}
        \label{fig:ablation_dcp_gt}
    \end{subfigure}
	% \label{fig:ablation_visual}
% \end{figure}

% \begin{figure}
%     \centering
    \begin{subfigure}[b]{0.155\textwidth}
        \centering
        \includegraphics[width=\textwidth]{images/ablation/5-1.png}{}
        \caption{Without $\loss_{\textrm{2D}}$}
        \label{fig:ablation_cyle_base}
    \end{subfigure}
    \begin{subfigure}[b]{0.155\textwidth}
        \centering
        \includegraphics[width=\textwidth]{images/ablation/5-2.png}{}
        \caption{With $\loss_{\textrm{2D}}$}
        \label{fig:ablation_cyle_ours}
    \end{subfigure}
        \begin{subfigure}[b]{0.155\textwidth}
        \centering
        \includegraphics[width=\textwidth]{images/ablation/5-3.png}{}
        \caption{Ground Truth}
        \label{fig:ablation_cyle_gt}
    \end{subfigure}
    \label{fig:ablation_3d}\vspace{-3ex}
    \caption{\textbf{Ablation:} $\loss_{\textrm{dcp}}$ and \(\loss_{\textrm{2D}}\) lead to more accurate clear-view color photometric details.}
    \vspace{-0.5cm}
	\label{fig:ablation_visual}
\end{figure}


\begin{figure}
    \centering
    \begin{subfigure}[b]{0.23\textwidth}
        \centering
        \includegraphics[width=\textwidth]{images/ablation/psnr-iter-eps-converted-to.pdf}
    \end{subfigure}
        \begin{subfigure}[b]{0.23\textwidth}
        \centering
        \includegraphics[width=\textwidth]{images/ablation/ssim-iter-eps-converted-to.pdf}
    \end{subfigure}
 \caption{\textbf{Ablation on \(\loss_{\textrm{2D}}\).} We compare the validation results during training with and without \(\loss_{\textrm{2D}}\): With \(\loss_{\textrm{2D}}\) the image quality continuously improve.}
 \label{fig:ablation_2D_loss_curve}
\end{figure}


\begin{table}[t!]
\centering\small
\resizebox*{\linewidth}{!}{
\begin{tabular}{ccccc}\toprule
Method & PSNR ($\uparrow$) & SSIM ($\uparrow$) & LPIPS ($\downarrow$) & Chamfer ($\downarrow$)\\\midrule
        %  NeRF & 19.038	& 0.865 & 0.156	& 3.584 \\
         NeuS & 16.722 & 0.879 & 0.087 & 2.635 \\
         COLMAP + NeuS & 16.750 & 0.785 & 0.225 & 5.971 \\
         ImDehaze\cite{guo2022image} + NeuS & 17.887 & 0.899 & \underline{0.078} & \underline{2.298} \\
         VidDehaze\cite{zhang2021learning} + NeuS & \underline{18.324} & \underline{0.901} & 0.087 & 2.349 \\
         \moniker{} & \textbf{25.702} & \textbf{0.920} & \textbf{0.052} & \textbf{2.066}
\\\bottomrule
\end{tabular}
}
\caption{\textbf{Quantitative comparison using synthetic data with heterogeneous haze}. \moniker{} yields better reconstruction both in image quality and geometry accuracy.}
\label{tab:dtu_quantitative}
\vspace{-3ex}
\end{table}

\subsection{Ablation Study}\label{sec:ablation}
We conduct ablation studies for the optimization regularizers on 3 scenes and report their average in \cref{tab:ablation_1}.

As \cref{tab:ablation_1} shows, both \(\loss_{\textrm{2D}}\) and \(\loss_{\textrm{dcp}}\) contribute positively to the image quality and surface reconstruction accuracy.
The \(\loss_{\textrm{dcp}}\) has a prominent effect in reducing global color shift as indicated by the PSNR value.
Similarly, from the visual comparison in \cref{fig:ablation_dcp_baseline,fig:ablation_dcp_ours,fig:ablation_dcp_gt}, we can observe that by adopting \(\loss_{\textrm{dcp}}\), the residual haze can be suppressed effectively.
Moreover, as shown in~\cref{fig:ablation_cyle_base,sub@fig:ablation_cyle_ours,sub@fig:ablation_cyle_gt}, the results optimized with \(\loss_{\textrm{2D}}\) show more accurate structure (see orange bounding boxes).
In addition, we plot the evolution of validation PSNR and SSIM in \cref{fig:ablation_2D_loss_curve}.
%One can observe that
\(\loss_{\textrm{2D}}\) improves the convergence behavior and yields better validation results.

\begin{table}[t!]
\centering
\resizebox*{\linewidth}{!}{
\begin{tabular}{ccccc}\toprule
Method & PSNR ($\uparrow$) & SSIM ($\uparrow$) & LPIPS ($\downarrow$) & Chamfer ($\downarrow$)\\\midrule
w/o $\loss_{\textrm{2D}} + \loss_{\textrm{dcp}}$ & 18.21 & 0.88 & 0.09 & 2.47 \\
w $\loss_{\textrm{2D}}$ & 18.53 & 0.89 & 0.09 & 2.44  \\
w $\loss_{\textrm{dcp}}$ & 24.02 & 0.90 & 0.06 & 2.44 \\
\moniker{} & \textbf{25.23} & \textbf{0.91} & \textbf{0.05} & \textbf{2.41} \\
\bottomrule
\end{tabular}
}
\caption{\textbf{Ablations on regularization}. Each of the proposed two regularizations can individually improve the image and geometry reconstruction, and the best quality is achieved with both. The results are averaged from 10 test scenes from the DTU dataset.}
\vspace{-0.2cm}
\label{tab:ablation_1}
\end{table}

\begin{figure*}[t!]
\setlength{\tabcolsep}{0pt}
\renewcommand{\arraystretch}{0.75}\footnotesize
\centering\begin{tabular}{*{6}{>{\centering\arraybackslash}M{0.165\textwidth}}}
 NeuS & COLMAP + NeuS& ImDehaze + NeuS & VidDehaze + NeuS & \moniker{} & Ground Truth \\
\includegraphics[width=0.98\linewidth, clip, trim={2cm 2cm 0cm 5cm}]{images/qualitative/real/bear/neus-APC_0287.jpg} &
\includegraphics[width=0.98\linewidth, clip, trim={2cm 2cm 0cm 5cm}]{images/qualitative/real/bear/colmap-APC_0287.jpg} &
\includegraphics[width=0.98\linewidth, clip, trim={2cm 2cm 0cm 5cm}]{images/qualitative/real/bear/dehaze-APC_0287.jpg} &
\includegraphics[width=0.98\linewidth, clip, trim={2cm 2cm 0cm 5cm}]{images/qualitative/real/bear/video-APC_0287.jpg} &
\includegraphics[width=0.98\linewidth, clip, trim={0.5cm 0.5cm 0cm 1.25cm}]{images/qualitative/real/bear/00050000_0_32_res8_clear.png} &
\includegraphics[width=0.98\linewidth, clip, trim={2cm 2cm 0cm 5cm}]{images/qualitative/real/bear/gt-APC_0287.jpg}
\\
\includegraphics[width=0.98\linewidth, clip, trim={0cm 1cm 0.5cm 0cm}]{images/qualitative/real/elephant/neus-APC0202.jpg} &
\includegraphics[width=0.98\linewidth, clip, trim={0cm 1cm 0.5cm 0cm}]{images/qualitative/real/elephant/colmap-APC0202.jpg} &
\includegraphics[width=0.98\linewidth, clip, trim={0cm 1cm 0.5cm 0cm}]{images/qualitative/real/elephant/dehaze-APC0202.jpg} &
\includegraphics[width=0.98\linewidth, clip, trim={0cm 1cm 0.5cm 0cm}]{images/qualitative/real/elephant/video-APC0202.jpg} &
\includegraphics[width=0.98\linewidth, clip, trim={0cm 1cm 0.5cm 0cm}]{images/qualitative/real/elephant/ours-58.png} &
\includegraphics[width=0.98\linewidth, clip, trim={0cm 1cm 0.5cm 0cm}]{images/qualitative/real/elephant/gt-APC0202.jpg}
\\
\includegraphics[width=0.98\linewidth, clip, trim={0cm 2.5cm 0cm 0cm}]{images/qualitative/real/lion/neus-0006.jpg} &
\includegraphics[width=0.98\linewidth, clip, trim={0cm 2.5cm 0cm 0cm}]{images/qualitative/real/lion/colmap-0006.jpg} &
\includegraphics[width=0.98\linewidth, clip, trim={0cm 2.5cm 0cm 0cm}]{images/qualitative/real/lion/dehaze-0006.jpg} &
\includegraphics[width=0.98\linewidth, clip, trim={0cm 2.5cm 0cm 0cm}]{images/qualitative/real/lion/video-0006.jpg} &
\includegraphics[width=0.98\linewidth, clip, trim={0cm 2.5cm 0cm 0cm}]{images/qualitative/real/lion/05.png} &
\includegraphics[width=0.98\linewidth, clip, trim={0cm 2.5cm 0cm 0cm}]{images/qualitative/real/lion/gt-0006.jpg}
\\
\end{tabular}
\caption{\textbf{Qualitative comparison on captured data}. Our method recovers the haze-free scenes with more image details and colors closer to the ground truth.}
\vspace{-3ex}
\label{fig:real_qualitative}
\end{figure*}

\begin{table}[t!]
\resizebox*{\linewidth}{!}{
\begin{tabular}{cccc}\toprule
Method & PSNR ($\uparrow$) & SSIM ($\uparrow$) & LPIPS ($\downarrow$) \\\midrule
% NeRF & 11.98&0.48 &0.37 \\
NeuS & 12.60& 0.48& 0.38\\
COLMAP + NeuS &9.27 & 0.38&0.47 \\
ImDehaze + NeuS  & 13.35 & 0.49& 0.36\\
VidDehaze + NeuS  &\underline{14.56} & \underline{0.50}& \underline{0.34} \\
\moniker{} & \textbf{17.47} & \textbf{0.68} & \textbf{0.16}\\\bottomrule
\end{tabular}
}
\caption{\textbf{Quantitative evaluation on experimentally captured data} averaged over 3 scenes. Our method outperforms other methods by a large margin.}
\label{tab:real_quantitative}
\vspace{-0.5cm}
\end{table}

\section{Experimentally Captured Results}
\subsection{Data Collection}
We captured three indoor hazy scenes using two professional haze machines and an iPhone 12 Pro.
The scenes contain 38, 82, and 68 hazy images and 79, 47, and 47 clear images for testing respectively.
To ensure temporal consistency (dynamic haze is out of scope for this paper), we capture the data after the haze has settled and appears steady.
After the capturing, we merge hazy images and clear images under the same scene and adopt COLMAP to register camera positions, which yields a consistent registration for the hazy and clear view for ease of evaluation.
We scale and translate the object such that it lies inside the unit sphere and roughly at the center of the world coordinates.
More details on the data capture are described in the supplement.

\subsection{Implementation details.}
Several changes are applied to the model to facilitate the optimization for captured data.
First, we remove the mask loss (~\cref{eq:mask_loss}).
Second, similar to NeuS~\cite{wang2021neus}, the region outside uses the parameterization of NeRF++\cite{zhang2020nerf++}, for which we also applied the haze attenuation following \cref{eq:C_surface,eq:alpha,eq:C_haze}.

\subsection{Comparison}
For evaluation, we adopt the same baselines described in \cref{sec:comparisons}.
The 2D photometric quality is evaluated using the same metrics, and geometry reconstruction quality is omitted due to the lack of ground truth.

\paragraph{Quantitative Evaluation.}
The results are reported in \cref{tab:real_quantitative}.
One can observe that our method outperforms other baselines in all metrics, demonstrating the robustness of the proposed method in the real-world scenario.

\paragraph{Qualitative Evaluation.}
As shown in \cref{fig:real_qualitative},
vanilla NeuS includes haze in the synthesized views.
On the other hand, adopting 2D image dehazing or video dehazing methods as pre-processing also cannot generate desired clear results since these learning-based models have limited generalization ability in real-world data.
Although using COLMAP as pre-processing can suppress residual haze, the dehazed results tend to have the color distortion problem and the limited structure (see the background at the $3^{rd}$ column of \cref{fig:real_qualitative}).
Our method achieves significantly better visual quality compared to other baselines.

\section{Discussion}\label{Discussion}

\subsection{Super Massive Black Holes}

\subsubsection{$M_{\rm BH}$--$\sigma_{\rm e}$ relation}

\begin{figure}
\centering
\includegraphics[width=0.47\textwidth]{bhmasses_jwn_v2.pdf}
\caption{
This work's measurements of Abell 1201's SMBH's mass in comparison to the black-hole mass versus velocity dispersion relation, from the compilation of \citet{Bosch2016}. Abell 1201's $\sigma_{\rm e}$ value is taken from \citet{Smith2017}. This work's measurement of $M_{\rm BH} = 3.27 \pm 2.12\times10^{10}$\,M$_{\rm \odot}$ is shown in black, which comes from averaging over all mass models. The upper limit of $M_{\rm BH} \leq 5.3 \times 10^{10}$\,M$_{\rm \odot}$ inferred for the broken power law mass model (without a SMBH) is shown for completeness, although we have argued this model is less trustworthy due to being nonphysical (see \cref{ResultSIE}). This figure is adapted from \citet{Smith2017a} and shows their inferred SMBH masses in grey, which come from independent analyses using either point-source based strong lens modeling \citep{Smith2017a} or stellar kinematics \citep{Smith2017}. Both works report that a SMBH with $M_{\rm BH} \geq 10^{10}$\,M$_{\rm \odot}$ fits the data, but neither work could break a degeneracy with models that assumed a radial gradient in the conversion of mass to light. Our inferred value of $M_{\rm BH}$ in Abell 1201 makes it one of the highest mass SMBH's measured. The grey dashed and dotted diagonal lines show 1$\sigma$ and 2$\sigma$ scatter of the mean $M_{\rm BH}$--$\sigma_{\rm e}$ relation, with Abell 1201's SMBH approximately a 2$\sigma$ positive outlier.
} 
\label{figure:SMBHRelation}
\end{figure}

\cref{figure:SMBHRelation} shows the inferred value of $M_{\rm BH} = 3.27 \pm 2.12  \times 10^{10}$\,M$_{\rm \odot}$ on the black-hole mass versus velocity dispersion relation.
This figure shows that Abell 1201 has one of the largest reported black hole masses measured so far, making it an ultramassive black hole \citep{Hlavacek-Larrondo2012}. Its mass is comparable to the SMBH of the brightest cluster galaxies NGC 3842 and NGC 4889 \citep{Mcconnell2013} and the field elliptical NGC 1600 \citep{Thomas2016}, all of which are measured via stellar orbit analysis. All three objects have similar values of $\sigma_{\rm e}$ to Abell 1201. 

The SMBH of Abell 1201 is a $\sim$ $2\sigma$ outlier above the scatter of the $M_{\rm BH}$--$\sigma_{\rm e}$ relation. Two other objects with similar $\sigma_{\rm e}$ values to Abell 1201, NGC3842 and NGC1600, are $\sim 1.5$-$2\sigma$ outliers above the mean relation. There are no corresponding outliers at $\sim 1.5\sigma$ below the mean relation, indicating that for $\sigma_{\rm e} > 250$km\,s$^{\rm -1}$ SMBH masses tend to be above the mean $M_{\rm BH}$--$\sigma_{\rm e}$ relation. Although there are too few objects to draw definitive conclusions,  such an upwards kink at high $\sigma_{\rm e}$ is a prediction of different physical processes. For example, binary SMBH scouring, which saturates $\sigma_{\rm e}$ whilst increasing $M_{\rm BH}$ \citep{Kormendy2013a, Thomas2014}, as well as AGN feedback processes \citep{Hlavacek-Larrondo2012}.  

\subsubsection{Stellar Core}

Massive ellipticals are often observed with a stellar core, quantified via the Nuker or cored Sersic models \citep{Hernquist1990, Trujillo2004, Dullo2013, Dullo2014}. BCGs like Abell 1201 may have extremely large and flat cores \citep{Postman2012}. It is posited that these cores form via SMBH scouring, whereby the dissipationless merging of two SMBHs in the centre of a galaxy preferentially ejects high mass stars via three-body interactions \citep{Faber1997, Merritt2006, Kormendy2009,Kormendy2013a, Thomas2014}. We fitted the core-Sersic model to Abell 1201's lens galaxy light during our initial analysis, however the model did not produce an improved fit to the data. Typical core sizes are $0.02$--$0.5$\,kpc \citep{Dullo2019}, therefore if Abell 1201 has a stellar core it may be we simply cannot resolve it, due to the data's resolution of $\sim 120$\,pc\,pixel$^{\rm -1}$. 

Aspects of the lens models which include a SMBH point towards a cored (or at least shallow) inner density. For example, the power-law mass model with a SMBH infers a slope $\gamma^{\rm mass} = 1.65^{+0.12}_{-0.12}$, which is much shallower than many massive elliptical strong lenses with near isothermal slopes of $\gamma^{\rm mass} \sim 2$ \citep{Koopmans2009}. Decomposed models including a SMBH give comparable inner densities. When fitting the core-Sersic model we only included it in the model for the lens galaxy's light. We did not fit it as part of a decomposed mass model and therefore did not try to constrain the stellar core via the ray-tracing and lensing analysis. Future studies hunting for SMBHs in strong lenses may benefit from doing this, because an improved model of the lens's central mass density could help break the degeneracy seen in this work with $M_{\rm BH}$.

\subsubsection{Outlook for Strong Lensing}

Abell 1201 is the second strong lens in which the central SMBH mass has been constrained. It is the first to do so without a central image, as well as the first to provide a measurement of $M_{\rm BH}$ as opposed to an upper limit. This raises a number of questions: what is so special about Abell 1201 that makes it sensitive to its SMBH? Can $M_{\rm BH}$ be measured in other known strong lenses? How common an occurrence will this be amongst the incoming samples of 100,000 strong lenses?

Abell 1201 is a unique strong lens in that its counter image is close to the lens centre and it is a cD galaxy in a galaxy cluster. The cluster potential exerts a large external shear (which is seen in our lens models) that brings the counter-image even closer to the lens centre \citep{Smith2017a}, an effect that is not present in most known galaxy-scale strong lenses, which are typically field galaxies. Thus, a very specific set of circumstances may make Abell 1201 sensitive to its SMBH, and a strategy to finding more systems is to target cD / BCG galaxies with instruments like Multi Unit Spectroscopic Explorer (MUSE). 

On the other hand, some known strong lenses in surveys like the Sloan Lens ACS Survey \citep{Bolton2008a} and Strong Lensing in the Legacy Survey \citep{Sonnenfeld2013b} may be sensitive to their central SMBH and appropriate lens modeling has simply not been performed. Certainly, every strong lens will provide an upper limit on $M_{\rm BH}$, the question is whether any are low enough to be informative for models of galaxy evolution. Whilst the multiple images of strong lenses are predominantly observed at radii well beyond Abell 1201's $1\,$kpc counter image, there are examples of strong lenses where the extended emission of the lensed source goes this close. For example, SLACS1250+0523, which was modeled by \citet{Nightingale2019}. In many surveys, for a candidate strong lens to be worthy of following up with higher resolution imaging, a visible counter image clearly distinct from the lens's emission is typically required. Systems like Abell 1201 may therefore be common in nature but rarely selected for followup. We leave it to future work to investigate what constraints known strong lenses can place on $M_{\rm BH}$. 

% Recent work by \citet{Shajib2022} shows the analysis of a strongly lensed quasar which, when fitted with a decomposed mass model, appears to show qualitatively similar behaviour to Abell 1201. The model appears to have missing mass in its centre which either a radial gradient in the 

It has long been expected that strong lensing can constrain SMBH masses when a central third or fifth image is observed \citep{Rusin2000, Mao2000, Keeton2003, Hezaveh2015}. Such a system was presented by \citet{Winn2003}, who placed an upper limit of $M_{\rm BH} < 2 \times 10^8$\,M$_{\rm \odot}$. These systems require the inner density profile of the lens galaxy to be sufficiently cored that the central image is not demagnifed below the observing instrument's detection limit. Given that no other such observation has been made despite numerous attempts \citep{Jackson2015, Wong2017} this appears to be a rare occurrence. Lower limits on $M_{\rm BH}$ have been placed in systems where a central image is not detected \citep{Quinn2016}.

Abell 1201 demonstrates that a SMBH mass measurement is possible even when the lens's inner density is not cored. This offers hope that large samples of strong lenses can one day constrain the $M_{\rm BH}$--$\sigma_{\rm e}$ relation. This would enable the masses of non-active black holes to be measured at high redshifts, and would provide measurements on the high $\sigma_{\rm e}$ end of the relation where few ETGs are observed in the local Universe. With over 100000 strong lenses set to be observed in the next decade \citep{Collett2015} it is inevitable that more SMBH measurements via strong lensing will be made, however more work is necessary to determine how common an occurrence this will be, and in what types of strong lenses and at how high of a redshift such constraints are feasible. If the detectability of a strong lens's SMBH depends on a specific set of circumstances like Abell 1201, there will also be unavoidable selection effects that must be accounted for.
\ificcvfinal
% !TeX spellcheck = en_US
%\section*{Code Availability Statement}
%A MATLAB implementation of the methods and simulations presented in this paper are openly available in an open-source repository available at {\small\texttt{\url{https://fish-tue.github.io/single-origin-destination-routing}}}.


\section*{Acknowledgment}
We thank Dr.\ I.\ New and F.\ Paparella for proofreading the~paper.


\fi

% \clearpage
%%%%%%%%% REFERENCES

{\small
\bibliographystyle{ieee_fullname}
\bibliography{egbib}
}
%%%%%%%%% Appendix
\appendix

\begin{center}
\textbf{\large Supplemental Materials}
\end{center}
%%%%%%%%%% Merge with supplemental materials %%%%%%%%%%
%%%%%%%%%% Prefix a "S" to all equations, figures, tables and reset the counter %%%%%%%%%%
\renewcommand{\theequation}{S.\arabic{equation}}
\renewcommand\thefigure{S.\arabic{figure}}
\renewcommand\thetable{S.\arabic{table}}
\setcounter{equation}{0}
\setcounter{figure}{0}
\setcounter{table}{0}

\documentclass[%
 preprint,
superscriptaddress,
 amsmath,amssymb,
 aps,
]{revtex4-2}

\usepackage{graphicx}%
\usepackage{dcolumn}%
\usepackage{bm}%
\usepackage{amsmath}
\usepackage{amssymb}
\DeclareMathOperator\erf{erf}
\DeclareMathOperator\erfc{erfc}
\usepackage{algorithm}
\usepackage{algpseudocode}

\renewcommand{\thesection}{S\arabic{section}}  
\renewcommand{\thetable}{S\arabic{table}}  
\renewcommand{\thefigure}{S\arabic{figure}}
\renewcommand{\theequation}{S\arabic{equation}}
\renewcommand{\figurename}{Supplemental Material, Figure} 

\begin{document}


\title{Supplemental Material:\\ Biological rhythms generated by a single activator-repressor loop with heterogeneity and diffusion.}

\author{Pablo Rojas}
\affiliation{ Theoretical Physics and Center for Interdisciplinary Nanostructure Science and Technology (CINSaT), University of Kassel, Kassel, Germany}
\author{Oreste Piro}%
\affiliation{Department of Ecology and Marine Resources, Institut Mediterrani d’Estudis Avançats, IMEDEA (CSIC–UIB), Balearic Islands, Spain}
\affiliation{Departament de F{\'i}sica, Universitat de les Illes Balears, Ctra. de Valldemossa, km 7.5, Palma de Mallorca E-07122, Spain}
\author{Martin E. Garcia}
\affiliation{ Theoretical Physics and Center for Interdisciplinary Nanostructure Science and Technology (CINSaT), University of Kassel, Kassel, Germany}%

\maketitle

\tableofcontents



\section{\label{sec:goodwin} Oscillations in Goodwin model }


The Goodwin model consists of a set of ordinary differential equations (ODEs), that represents a negative feedback loop in a chemical reaction network \cite{goodwin65,griffith68}. In this network, the protein at the end of the loop represses its own transcription. For each substance in the loop, the temporal evolution of its amount is described by production and degradation terms. In its simplest version, the only nonlinearity appears in the repression as a Hill function. The model assumes a well-mixed or compartmental condition. Eq. 1 and Eq. 2 from the main text are special cases of the more general $n$ reaction steps version, that reads as in Eq.\ref{eq:goodwin_n}. 

\begin{equation}
\label{eq:goodwin_n}
\begin{aligned}
%
	\frac{dx_1}{dt} & = \omega_{1} \frac{1}{1 + x_n^h} - \gamma_{1}x_1 \\
%
	\frac{dx_2}{dt} & = \omega_{2}x_1 - \gamma_{2}x_2 \\
	: &   \\
	\frac{dx_n}{dt} & = \omega_{n}x_{n-1} - \gamma_{n}x_n \\
\end{aligned}
\end{equation}

The secant condition, derived in \cite{tyson1978dynamics}, is the necessary condition that the exponent $h$ of the Hill function must fulfill to enable oscillatory solutions (Eq.\ref{eq:secant_condition}, Fig.1 in main text).
\begin{equation}
\label{eq:secant_condition}
h \geq \sec^n \left(\frac{\pi}{n} \right)  
\end{equation}

\section{\label{sec:deterministic} Deterministic model}


The modeled system is an extension of the Goodwin model to account for spatial heterogeneity, which transforms the set of ODEs into a set of partial differential equations (PDEs). Cell is represented by a symmetric spatial domain. For the sake of simplicity, no distinction between nucleus and cytoplasm transport properties is considered, i.e. homogenous diffusion coefficient. The membrane of the cell at the borders is impermeable.

The model first considers 2 species, mRNA and protein, whose spatial concentration are represented by  $m(\vec{r},t)$ and $p(\vec{r},t)$, respectively. Transcription of mRNA molecules and translation of proteins similarly to Goodwin model, but spatially restricted with functions $f_{GEN}(\vec{r})$ and $f_{RIB}(\vec{r})$, that represent the gene and ribosome locations, respectively. Linear degradation is considered to occur across the whole domain for both species, therefore it is equivalent to Goodwin model. We represented the molecular transport of mRNA and protein with Fickian diffusion. A parameter $p_{thresh}$ is used in the repression to control the scale of concentration of protein in the Hill function. The equations for the system are:
\begin{eqnarray}
\label{eq:deterministic2elem}
\dot{m}(\vec{r},t)   & = & \frac{\omega_m}{1+ \left(\dfrac{p(\vec{r},t) }{p_{thresh}} \right)^h} \, f_{GEN}(\vec{r}) -\gamma_m \, m(\vec{r},t)   + D_m \, \nabla^2 m(\vec{r},t)    \\
\dot{p}(\vec{r},t) & =& \omega_p \; f_{RIB}(\vec{r}) \, m(\vec{r},t)- \gamma_p \, p(\vec{r},t)   + D_p \nabla^2 \, p(\vec{r},t)    \nonumber 
\end{eqnarray}

This model represents our problem in the three dimensional space. To simplify our analysis we present our result with the reduction to one dimensional domain. Therefore, the equations read as:

\begin{eqnarray}
\label{eq:deterministic2elem1D}
\dot{m}(x,t)   & = & \frac{\omega_m}{1+    \left(\dfrac{p(x,t)  }{p_{thresh}} \right)^h} \, f_{GEN}(x) -\gamma_m \, m(x,t)  + D_m \, \nabla^2 m(x,t)   \nonumber \\
\dot{p}(x,t)   & = & \omega_p \; f_{RIB}(x) \, m(x,t) - \gamma_p \, p(x,t)  + D_p \nabla^2 \, p(x,t)      
\end{eqnarray}

The localization of gene and ribosomes is set to boxcar functions  $f_{GEN}(x)$ and $f_{RIB}(x)$ centered at $x_m$ and $x_p$, respectively, with radii $R_{m}$ and $R_{p}$.

\begin{equation}
\label{eq:fgen}
f_{GEN}(x) = 
    \begin{cases}       
        1 & \text{if } \; x \in [x_m - R_{m},x_m + R_{m} ]    \\
        0 & \text{else }
    \end{cases}
\end{equation}


\begin{equation}
\label{eq:frib}
f_{RIB}(x) = 
    \begin{cases}       
        1 & \text{if } \; x \in [x_p - R_{p},x_p + R_{p} ]    \\
        1 & \text{if } \; x \in [-x_p - R_{p},-x_p + R_{p} ]    \\
        0 & \text{else }
    \end{cases}
\end{equation}

The model as in Eq. \ref{eq:deterministic2elem1D} applies repression to the transcription of mRNA by comparing the values of $p(x,t)$ at each point $x$ with the respective parameter $p_{thresh}$. Due to the nonlinear nature of the Hill function, this differs from the well-mixed formulation in which the total amount is first computed and then compared to the respective parameter. We found these differences to lead to slight quantitative differences, only affecting the exact values for the onset of oscillations but preserving the structure of the solutions. Therefore, unless otherwise stated, the study shows results using Eq. \ref{eq:deterministic2elem1D}. However, for comparison with existing models we resorted to a formulation that compares the value of the repressing species, computed as the amount that is located within the gene location. We called this \emph{volumetric repression} (Eq. \ref{eq:deterministic2elem1Dvolum}).


\begin{eqnarray}
\label{eq:deterministic2elem1Dvolum}
\dot{m}(x,t)   & = & \frac{\omega_m}{1+    \left(\dfrac{P_{GEN}(t)  }{P_{thresh}} \right)^h} \, f_{GEN}(x) -\gamma_m \, m(x,t)  + D_m \, \nabla^2 m(x,t)   \nonumber \\
\dot{p}(x,t)   & = & \omega_p \; f_{RIB}(x) \, m(x,t) - \gamma_p \, p(x,t)  + D_p \nabla^2 \, p(x,t)      ,
\end{eqnarray}

where $ P_{GEN}(t) $ is the amount of $ p(x,t) $ integrated in the gene location (Eq. \ref{eq:Pgen_integral_value}). 

\begin{equation}
\label{eq:Pgen_integral_value}
	P_{GEN}(t)  =  \int_\Omega p(x,t) f_{GEN}(x) dx, 
\end{equation}
where $\Omega$ is the cell volume. 

The solution of the PDEs were obtained numerically using second order finite differences in the spatial domain, and Runge-Kutta-4 in the time domain. 


\section{\label{sec:three_molecules} Feedback loop with three species }

The extension of our model to 3 participating species is done similarly to 2 species. The equations for 3 participating species that is the corresponding to Eq. \ref{eq:deterministic2elem1D} for 2 species, are shown in Eq. \ref{eq:3elem}.


\begin{eqnarray}
\label{eq:3elem}
\dot{m}(x,t)   & = & \frac{\omega_m}{1+   \left(\dfrac{r(x,t)  }{r_{thresh}} \right)^h} \, f_{GEN}(x) -\gamma_m \, m(x,t)  + D_m \, \nabla^2 m(x,t)   \nonumber \\
\dot{p}(x,t)   & = & \omega_p \; f_{RIB}(x) \, m(x,t)- \gamma_p \, p(x,t)  + D_p \nabla^2 \, p(x,t)        \\
\dot{r}(x,t)   & = & \omega_r \; f_{REP}(x) \, p(x,t)- \gamma_p \, r(x,t)   + D_r \nabla^2 \, r(x,t)    \nonumber 
\end{eqnarray}

The function $f_{REP}(x)$ represents the site where the production of the repressor $r$ takes place. To simplify the analysis, $f_{REP}(x)$ is chosen to be identical to $f_{RIB}(x)$ (Eq.\ref{eq:frep}), so that only one distance remains as a parameter of the problem.
 
\begin{equation}
\label{eq:frep}
f_{REP}(x) = f_{RIB}(x) = 
    \begin{cases}       
        1 & \text{if } \; x \in [x_p - R_{p},x_p + R_{p} ]    \\
        1 & \text{if } \; x \in [-x_p - R_{p},-x_p + R_{p} ]    \\        
        0 & \text{else }
    \end{cases}
\end{equation}

The model with 3 species have a better comparison with ODE models when the formulation with volumetric repression is used. Under this framework, Eq. \ref{eq:3elem} is modified as:

\begin{eqnarray}
\label{eq:3elem_volum}
\dot{m}(x,t)   & = & \frac{\omega_m}{1+   \left(\dfrac{R_{GEN}(t)  }{R_{thresh}} \right)^h} \, f_{GEN}(x) -\gamma_m \, m(x,t)  + D_m \, \nabla^2 m(x,t)   \nonumber \\
\dot{p}(x,t)   & = & \omega_p \; f_{RIB}(x) \, m(x,t) - \gamma_p \, p(x,t)  + D_p \nabla^2 \, p(x,t)        \\
\dot{r}(x,t)   & = & \omega_r \; f_{REP}(x) \, p(x,t) - \gamma_p \, r(x,t)   + D_r \nabla^2 \, r(x,t)    \nonumber 
\end{eqnarray}

where 

\begin{equation}
\label{eq:Rgen_integral_value}
	R_{GEN}(t)  =  \int_\Omega r(x,t) f_{GEN}(x) dx, 
\end{equation}

In the examples shown for 3 species, we used volumetric repression.

\begin{figure}[h]
    \includegraphics[width=6.6in]{panel_3_elem_2_parameter_sets.pdf}
    \caption{ Amplitude of oscillations in an extension of the model for 3 species. Left panel: Parameter Set 1 described in the main text. Right panel: Parameter Set 2. Lines indicate minimum and maximum of the trajectories in the long term approximation. Shaded areas between minimum and maximum indicate amplitude of oscillations. Parameter Set 1 only displays oscillations when there is a separation between the sources. Parameter Set 2 shows oscillations when sources are co-localized (as predicted by the well-mixed model) and when sources are separated. Notice the intermediate region when no oscillation is present. Oscillations in the separation case present higher amplitude.  }
    \label{fig:3elem_loop}
\end{figure}

The problem is solved for 2 sets of parameters. In the Parameter Set 1, parameters for the transport of substances are selected according to the ones reviewed and used in \citep{fonkeu2019mrna,sturrock2011spatio}. These parameters represent experimental measurement of diffusion coefficients and kinetic rates for mRNA and protein. Since in this work reviews other sources of experimental data and reports a wide variability on the measurements, we simplified the choice of parameters by using the corresponding order of magnitude. In the Parameter Set 2, we selected kinetic parameters that would generate oscillations in the well-mixed Goodwin Model, inspired in Ref. \citep{gonze2013goodwin}, in order to induce an equivalence to the corresponding well-mixed version under low values of $x_p$. Full set of parameters in Table \ref{tab:param3elem}.


\section{\label{sec:stochastic} Stochastic model }

We constructed a stochastic model of the coupled reaction and diffusion of the systems, by treating each of the molecules as an individual agent, as shown in Fig. \ref{fig:stochastic_scheme}. The Eq. \ref{eq:SSA} represents the sets of equations that represent the creation, degradation and displacements of the molecules. In these equations, $\phi$ denotes chemical species that are of no interest, and is used in order to represent degradation and creation of molecules. To solve this system, we used the stochastic simulation algorithm (SSA) described in Ref. \cite{erban2020}. 
\begin{eqnarray}
\label{eq:SSA}
mRNA     & \rightarrow   & \phi        \nonumber  \\
protein  & \rightarrow   & \phi         \nonumber \\
\phi     & \rightarrow  & mRNA           \\
\phi     & \rightarrow    & protein     \nonumber \\
x_i(t+\Delta t) & =&  x_i(t ) + \sqrt{2D \Delta t} \xi , \nonumber 
\end{eqnarray}
In our SSA, we used a time step $\Delta t$, small enough so that the probability of a certain reaction ocurring during this interval of time is $k \Delta t$, $k$ being its reaction rate. We therefore generate random numbers at each time step to determine if an individual molecule is degraded. We also determine whether a new molecule of a certain species must be created, in which case it will be positioned at a random location within its own creation zone. During the same interval, an existing molecule can update its position by determining its stochastic displacement $\sqrt{2D \Delta t} \xi$, where $ \xi $ is a number drawn from a Gaussian distribution. We therefore determine, at each time step, whether an individual molecule is degraded, created and where its new position is, if corresponding (Alg. \ref{alg:SSA_steps}). The reaction rates are determined using the equivalent expresions from the deterministic model (Eq. \ref{eq:reaction_rates_SSA}).
\begin{eqnarray}
\label{eq:reaction_rates_SSA}
k_{degradation \, mRNA}     & =   &  \gamma_m        \nonumber  \\
k_{degradation \, protein}  & =   &  \gamma_p         \nonumber \\
k_{creation \, mRNA}     & =   & \frac{\omega_m}{1+    \left(\frac{protein_{transcription \; zone}  }{protein_{thresh}} \right)^h}       \\
k_{creation \, protein}  & =   &  \omega_p \quad {mRNA}_{translation \, zone} \,        \nonumber 
\end{eqnarray}


\begin{figure}
\begin{algorithm}[H]
\caption{Stochastic model steps. }\label{alg:SSA_steps}
\begin{algorithmic}
\State set initial condition and parameters
\While{$t \leq t_{max}$} 
	\State determine mRNA (protein) molecules in translation (transcription) zone
	\State compute reaction rates and probabilities
	\For{each molecule}
		\State generate random numbers	
		\State compute displacements
		\State update positions
	\EndFor
	\For{each molecule}
		\State generate random number
		\If{$random \quad number \leq probability \quad degradation$}
		    \State eliminate molecule
    		\EndIf
	\EndFor
	\For{each species (mRNA and Protein)}
		\State generate random number
		\If{$random \quad number \leq probability \quad creation$}
		    \State insert new molecule in random position within production zone
    		\EndIf
	\EndFor
\EndWhile
\end{algorithmic}
\end{algorithm}
\end{figure}



\begin{figure}[h!]
    \includegraphics[width=6.0in]{diagram_stoch_scheme_borders.png}
    \caption{ Scheme of the stochastic model. Molecules are modeled as individual particles. They are created either at gene (transcription zone) or ribosome (translation zone), similarly to deterministic model. To create molecules, the stochastic simulation algorithm (SSA) is implemented with a propensity function based on the number of molecules of the other species in their own production zone. Molecules travel following the Langevin equation, and are reflected at the borders of the cell. The results shown in Fig. 4 in the main text where obtained with a 1D version of this model. }
    \label{fig:stochastic_scheme}
\end{figure}

\section{\label{sec:relaxation} Derivation of relaxation times }

\subsection*{Reduced problem}
We first consider a symmetric one dimensional domain, in which for $t\geq0$ the production of a substance whose concentration is $c(x,t)$ is done at a constant rate $\omega$ in a punctual source centered in the interval $[-L,L]$, represented by $\delta(x)$. At $t=0$, $c(x,t=0)=0$ in the whole spatial domain. This substance is subjected to Fickian diffusion (with a constant diffusion constant $D$), and degradation (with $\gamma$ as degradation constant). The walls of the domain $[-L,L]$ are impermeable. For $t<0$ the system is at its equilibrium state $c(x)=0$ since no source is available, and this equilibrium is perturbed by the sudden appearance of the punctual source. This scenario is an idealized cartoon of the instant in which hypothetical mRNA (protein) initiates transcription (translation) inside a cell, without prior concentration. The new concentration as approaches equilibrium at long times. We will derived a characteristic time that describes the relaxation of the concentration at a certain spatial point, from its initial condition $c=0$ to the new established profile $c=c_{t \rightarrow \infty}$. The equation for the reduced problem reads:

\begin{equation}
\label{eq:finite_problem}
%
%
\left\{
\begin{aligned}
	\frac{\partial c}{\partial t} 			&= D  \nabla^2 c - \gamma c + \omega  \delta (x) 
													& \quad \forall x \mbox{ in } [-L,L]  \\
	\left.\frac{\partial c}{\partial x} \right|_L 	&= \left.\frac{\partial c}{\partial x} \right|_{-L} = 0  \\
	c(x,0) &= 0 & \forall x \mbox{ in } [-L,L]\\
\end{aligned} 
\right.
%
\end{equation}

Since the problem is symmetric, we can rewrite the system for the positive sub-interval, by transforming the production term into a boundary condition $\left( \left.\frac{\partial c}{\partial x} \right|_0 = -\frac{\omega}{2D}\right)$.

\begin{equation}
\left\{
\begin{aligned}
	\frac{\partial c}{\partial t} 			&= D  \nabla^2 c - \gamma c  
													& \quad \forall x \mbox{ in }[0,L]  \\
	\left.\frac{\partial c}{\partial x} \right|_0 	&= -\frac{\omega}{2D}  \\
	\left.\frac{\partial c}{\partial x} \right|_L 	&= 0  \\
	c(x,0) &= 0 & \forall x \mbox{ in }[0,L]\\
\end{aligned} 
\right.
\end{equation}

The solution of the problem is simpler if we assume the limit $L \rightarrow \infty$. Since our purpose is to derive estimates analytically, this assumption proves to be practical and does not affect the conclusions. For $L \rightarrow \infty$ the problem reads:


\begin{equation}
\label{eq:infinite_symm_problem}
\left\{
\begin{aligned}
	\frac{\partial c}{\partial t} 			&= D  \nabla^2 c - \gamma c  
													& \quad \forall x \mbox{ in }[0,\infty)  \\
	\left.\frac{\partial c}{\partial x} \right|_0 	&= -\frac{\omega}{2D}  \\
	\left.\frac{\partial c}{\partial x} \right|_{x \rightarrow \infty} 	&= 0  \\
	c(x,0) &= 0 & \forall x \mbox{ in }[0,\infty)\\
\end{aligned} 
\right.
\end{equation}


The solution to the steady state problem, i.e. $ \frac{\partial c}{\partial t} = 0$ is asymptotically approached by the transient solution at long times $t \rightarrow \infty$. The steady state solution becomes:
\begin{equation}
\label{eq:steady_state_solution}
c_{t \rightarrow \infty}= \frac{\omega}{2 \sqrt{D \gamma}} e^{-\sqrt{\frac{\gamma}{D}} x} 
\end{equation}

To obtain the transient solution, we will work with the Laplace transform $u(x,s) = \mathcal{L}[c(x,t)]$. The resulting differential equation is:

\begin{equation}
\label{eq:laplace_ODE}
	\frac{\partial^2 u}{\partial x^2} - \left( \frac{s+\gamma}{D} \right) u = 0 
\end{equation}
and we can propose a solution of the form:
\begin{equation*}
	u(x,s)= C_1 e^{-\sqrt{\frac{s +\gamma}{D}} x} 
\end{equation*}
 
By applying the boundary condition at $x=0$:

\begin{equation*}
	\frac{\partial u}{\partial x}=   \mathcal{L} \left[ \frac{\partial c}{\partial x}\right]  
\end{equation*}
\begin{equation*}
	C_1 \left( -\sqrt{\frac{s +\gamma}{D}} \right) = -\frac{\omega}{2D \, s}
\end{equation*}
\begin{equation*}
	\Rightarrow C_1  =  \frac{\omega}{2 \sqrt{D} \, s \, \sqrt{s +\gamma}} 
\end{equation*}

Then, the solution for $u$ is: 
\begin{equation}
\label{eq:solution_u}
	u(x,s)= \frac{\omega}{2 \sqrt{D} \, s \, \sqrt{s +\gamma}} e^{-\sqrt{\frac{s +\gamma}{D}} x} 
\end{equation}

To obtain $c(x,t)$, it is necessary to anti-transform the solution by $c(x,t) = \mathcal{L}^{-1}[ u(x,s)]$. 

\begin{equation}
\label{eq:antitransform_c}
	c(x,t)= \mathcal{L}^{-1} \left[ \frac{\omega}{2 \sqrt{D} \, s \, \sqrt{s +\gamma}} e^{-\sqrt{\frac{s +\gamma}{D}} x} \right]
\end{equation}

It is possible to antitransform first the factors $\mathcal{L}^{-1} \left[ \frac{1}{s} \right] $ and $\mathcal{L}^{-1} \left[ \frac{e^{-\sqrt{\frac{s +\gamma}{D}} x}}{\sqrt{s +\gamma}} \right] $, and use these results to arrive to the solution of $c(x,t)$.

\begin{equation}
\label{eq:solution_analytic_with_integral}
	c(x,t)=  \frac{\omega}{2 \sqrt{D \pi } } \int_{0}^{t} \frac{e^{- \left( \gamma y + \frac{x^2}{4Dy} \right)}}{\sqrt{y}} dy
\end{equation}

Solving the integral:

\begin{equation}
c(x,t) = \frac{\omega}{4\sqrt{D \gamma}} \left[ e^{-\sqrt{\frac{\gamma}{D}} x}  \erfc\left( \frac{x}{2\sqrt{D t}} - \sqrt{\gamma t} \right)   - e^{\sqrt{\frac{\gamma}{D}} x}  \erfc\left( \frac{x}{2\sqrt{D t}} + \sqrt{\gamma t} \right) \right] 
\label{solution_analytic}
\end{equation}

This solution is displayed in Fig.\ref{fig:delays_analytic}a for different times. 

\subsection*{Relaxation time}
The time at which $\frac{\partial c}{\partial t}$ maximizes can be obtained by the solution of:

\begin{equation}
\label{eq:maximal_c_rate}
	\frac{\partial }{\partial t} \left( \frac{\omega}{2 \sqrt{D \pi } }  \frac{e^{- \left( \gamma t + \frac{x^2}{4Dt} \right)}}{\sqrt{t}} \right) = 0 
\end{equation}

From which we can obtain:

\begin{equation*}
	\frac{\partial }{\partial t} \left(  \frac{e^{- \left( \gamma t + \frac{x^2}{4Dt} \right)}}{\sqrt{t}} \right) = \left( - \frac{e^{- \left( \gamma t + \frac{x^2}{4Dt} \right)}}{\sqrt{t}} \right) \frac{1 }{t^{3/2} } \left(      \gamma t - \frac{x^2}{4D t} + \frac{ 1 }{2} \right) 	 = 0
\end{equation*}

Therefore, the \emph{relaxation time} $t_{R}$ results as a solution of:
\begin{equation}
\label{eq:equation_relax_time}
t_{R}^2 +\frac{t_{R}}{2\gamma} - \frac{x^2}{4D\gamma} = 0
\end{equation}

Since we are interested in the interval $t\geq 0$, one of the solutions is ruled out and we obtain:
\begin{equation}
\label{eq:solution_relax_time}
t_{R} = \frac{1}{4\gamma}   \left(  \sqrt{1 + 4  \left(\frac{x}{\lambda}\right)^2 } -1  \right)
\end{equation}

where $ \lambda  = D/\gamma$ is the characteristic length. For values of $x/\gamma$ large enough, $t_{R}$ approaches a linear dependence on $x$ (Fig.\ref{fig:delays_analytic}b-c) (notice the difference with the $x^2/D$ dependence of time-scales in purely diffusive systems).


\subsection*{Dimensionless version of the reduced problem}

A change of variables that render the reduced problem as dimensionless can help the relevant parameters. The change of variables:

\begin{center}
$
\begin{array}{rllcrl}
	x^\ast & = \frac{x}{\lambda} & = \frac{x}{\sqrt{\frac{D}{\gamma}}} & \Rightarrow & \; x & =  x^* \sqrt{\frac{D}{\gamma}} \\
	t^* & = \frac{2 t \sqrt{D \gamma}}{x}  & = \frac{2 t }{ \frac{x}{\lambda \gamma  } } & \Rightarrow & \; t & =  \frac{t^* x^*}{2 \gamma}
\end{array}
$
\end{center}

yields the steady state solution ($t\rightarrow \infty$)  re-written as:

\begin{center}
$
\begin{array}{rlcrl}
	c^* & = \frac{c}{\frac{\omega}{2\sqrt{D \gamma}}} &  &  &  \\
	c_{t\rightarrow \infty}(x) & =  \frac{\omega}{2\sqrt{D \gamma}} e^{-\sqrt{\frac{\gamma}{D}} x} & \Rightarrow & \; c_{t\rightarrow \infty}^*(x^*) & =  e^{-x^*}
\end{array}
$
\end{center}

For the time-varying solution the dimensionless version yields:
\begin{equation}
\label{eq:dimensionless_solution}
c^*(x^*,t^*) = \frac{1}{2} \left[ e^{-x^*}  \erfc\left( \frac{1}{\sqrt{2}} \left[ \sqrt{\frac{x^*}{t^*}} - \sqrt{x^* t^* } \right] \right)   - e^{x^*}  \erfc\left( \frac{1}{\sqrt{2}} \left[ \sqrt{\frac{x^*}{t^*}} + \sqrt{x^* t^* } \right] \right) \right] 
\end{equation}

therefore, the problem scales with $x$ proportional to lambda $\lambda$, but in time $t$ it scales with $x/\lambda$ and inversely to $\gamma$.


\subsection*{Linear dependence of delay on distance in  oscillatory solutions}

The spatio-temporal distribution of $m(x,t)$ and $p(x,t)$ in oscillatory solutions (Fig.\ref{fig:heatmap_space_time}) described in the main text, shows an almost constant speed of propagation , in agreement with the derived dependence on $x$. 
For sufficiently small or sufficiently large separation of the sources, the spatial distribution of $m(x,t)$ and $p(x,t)$ remain constant over time (Fig.\ref{fig:heatmap_space_time}a,d).
However, for intermediate separations, these distributions display oscillatory behavior in time (Fig. \ref{fig:heatmap_space_time}b,d).
The period of oscillations increases as the distance between sources increases. Propagation in the spatial distribution of species are recognizable as the slanted brighter regions. The slope of the regions denotes the constant propagation speed. 


\begin{figure}[h]
    \includegraphics[width=6.6in]{delays_in_analytic.pdf}
    \caption{ Relaxation times in the reduced problem.  (a) Concentration of the substance $c$ as a function of $x$ at different times. The spatial profile converges asymptotically to the steady state. (b) Normalized concentration as a function of re-scaled time. Dots represent relaxation times, defined as the times in which the rate of increase of concentration is maximal. (c) Analytical solution for relaxation times show asymptotically linear dependence with distance to the source. }
    \label{fig:delays_analytic}
\end{figure}

\begin{figure}[h]
    \includegraphics[width=6.6in]{delays_in_PDE.pdf}
    \caption{ Propagation of concentration gradients arising from spatial separation of the sources. (a) Distribution of the species $m(x,t)$ and $p(x,t)$ in space and time, between center and borders of the cell, for different separation of sources. Concentrations are shown in normalized logarithmic scale. The two displayed oscillatory solutions ($x_p = 1.5 \lambda$ and  $x_p = 3 \lambda$) show different periods of oscillation, but the same dependence of diffusion-induced delays with distance (slope of the bright regions). The parameters for this figure correspond to the ones in Fig.4 in main text. $h= 10 $, $\lambda = \lambda_m = \lambda_p =1$, $ D_m = D_p = 0.1$ , $ \gamma_m = \gamma_p = 0.1 $, $ R_{m} = R_{p} = 0.5$, $ R_{cell} = 10$ , $\omega_m = \omega_p = 10 $  } 
    \label{fig:heatmap_space_time}
\end{figure}

\section{\label{sec:params} Parameters }

Parameters used in the models with two species are given in arbitrary units (Table \ref{tab:param2elem}), with the exception that the amounts can be interpreted in molecules. Parameters in models for three species are selected based on previous work with their corresponding units (Table \ref{tab:param3elem}). Nevertheless, parameters in models with two species could be interpreted as having the same units as the models for three species, i.e. $\mu m$ for sizes and distances, $h$ for time and their corresponding derived units. 

\begin{table}[h!]
\caption{\label{tab:param2elem}
Parameters used in the models with 2 species. Values are given in arbitrary units. }
\begin{ruledtabular}
\begin{tabular}{lcccccccccc}
 &$ D_{m}$  & $D_{p}$ &  $ \gamma_{m}$  & $ \gamma_{p}$& $ \omega_{m}$  & $ \omega_{p}$&$R_{cell}$ &$ R_{m}$  & $R_{p}$& $p_{thresh}$ \\
\hline
Fig. 2-3& 0.1 & 0.1 & 0.1 & 0.1 & 10.0 & 20.0 & 10.0 & 0.5 & 0.5 & 0.1\\
Fig. 4  & 0.1 & 0.1 & 0.1 & 0.1 & 10.0 & 10.0 & 10.0 & 0.5 & 0.5 & 0.1\\
\end{tabular}
\end{ruledtabular}
\end{table}



\begin{table}[h!]
\caption{\label{tab:param3elem}
Parameters used in the models with 3 species. }
\begin{ruledtabular}
\begin{tabular}{lcccccccccccccc}
 &$ D_{m}$  & $D_{p}$ &$ D_{r}$ &  $ \gamma_{m}$  & $ \gamma_{p}$ &  $ \gamma_{r}$ & $ \omega_{m}$  & $ \omega_{p}$ & $ \omega_{r}$  &$R_{cell}$ &$ R_{m}$  & $R_{p}$  &$ R_{r}$ & $r_{thresh}$ \\
 &$  \left(\frac{\mu m ^2}{h}\right)$  & $  \left(\frac{\mu m ^2}{h}\right)$ &$  \left(\frac{\mu m ^2}{h}\right)$ &  $  \left(\frac{1}{h}\right)$   & $  \left(\frac{1}{h}\right)$  &  $  \left(\frac{1}{h}\right)$  & $  \left(\frac{mol.}{h\,\mu m}\right)$   & $  \left(\frac{mol.}{h\,\mu m}\right)$ & $  \left(\frac{mol.}{h\,\mu m}\right)$  &$ (\mu m) $ & $ (\mu m) $   & $ (\mu m) $   & $ (\mu m) $ & $ (mol.) $  \\
\hline
Param. Set 1 & 10.0 & 10.0 & 10.0 & 1.0 & 1.0 & 1.0 & 50.0 & 100.0 & 100.0 & 10.0 & 0.5 & 0.5 & 0.5 & 2.0\\
Param. Set 2 & 0.01 & 0.01 & 0.01 & 0.1 & 0.1 & 0.1 & 5.0  & 10.0  & 10.0  & 4.0  & 0.5 & 0.5 & 0.5 & 1.0\\
\end{tabular}
\end{ruledtabular}
\end{table}

\bibliography{supp_text}

\end{document}

\clearpage


\end{document}
