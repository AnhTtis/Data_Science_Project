\section{Discussion}
\label{sec:discussion}
In this paper, we address challenges to apply NeRFs in hazy scenes.
Our method jointly learns the 3D structure and the plausible clear-view appearance of the scene, as well as the haze properties (airlight and scattering coefficient) from hazy observations.
Our key technical contributions are:
1. incorporating the scattering phenomena in NeRF's rendering equation;
2. deploying physically inspired inductive biases and haze-specific regularizers to disambiguate haze and clear-view components.
The proposed techniques take a significant step towards adopting NeRF-related approaches in real-life scenarios where adverse weather conditions are detrimental to 3D reconstruction.

\paragraph{Relation to the Koschmieder Law.}
Koschmieder model is a 2D image formation model that describes the relationship between the clear-view and the hazy images.
Assuming homogeneous scattering, our formulation can be simplified to Koschmieder's model, which reduces to supervising with \cref{eq:koschmieder} and \(\loss_{\textrm{dcp}}\).
Our \(\loss_{\textrm{2D}}\) exploits this relation to improve the convergence as shown in \cref{sec:ablation}.
In \cref{tab:koschmieder}, we examine this relation by comparing this simplified method (denoted as Koschm) with a variant of \moniker{} that uses a learnable scalar instead of the proposed MLP to model the scattering coefficient (denoted as Scalar).
For both synthetic and real data, the Scalar and Koschm solutions are comparable, which validates the relation to Koschmieder law; our full model shows a considerable advantage in both scenarios thanks to its more general formulation.
\begin{table}[htbp]
\scriptsize
\scalebox{0.92}{
\begin{tabular}{@{\hspace{-0.02em}}c*{7}
{P{0.07\linewidth}}}
\toprule
\multirow{2}{*}{Method} & \multicolumn{4}{c}{Synthetic Data (Scene24)} & \multicolumn{3}{c}{Real Data (``Elephant'' Scene)} \\\cmidrule(lr){2-5} \cmidrule(lr){6-8}
 & PSNR & SSIM & LPIPS & Chamfer & PSNR & SSIM & LPIPS \\\midrule
Koschm & 18.62 & 0.79 & 0.13 & 1.44 & 16.84 & 0.69 & 0.18 \\
Scalar & 18.50 & 0.79 & 0.13 & 1.46 & 16.80 & 0.69 & 0.18 \\
\moniker{} & \textbf{22.78} & \textbf{0.82} & \textbf{0.09} & \textbf{1.22} & \textbf{17.87} & \textbf{0.73} & \textbf{0.15} \\
\bottomrule
\end{tabular}}
    \caption{\textbf{Relation to the 2D Koschmieder Law.} When using a scalar to model the scattering coefficient (Scalar), our method can simplify to supervising with the 2D Koschmieder law (Koschm). Our full model using an MLP to model the spatially varying scattering coefficient has a clear advantage for heterogeneous haze.}
    \label{tab:koschmieder}
    \vspace{-0.5cm}
\end{table}

\paragraph{Limitations and future work.}
While we have demonstrated the effectiveness of our method to capture data, certain scenarios are not addressed in the scope of this paper and are research directions for future work.
These include severe haze scenes, where camera registration fails due to the lack of discriminative image features, and dynamic haze, where the haze distribution change not only spatially but also temporally.
Nonetheless, we believe that our method is an important step towards anticipated application scenarios, such as autonomous driving or underwater imaging.
The principle of incorporating physics into the neural rendering is also applicable to other ill-posed low-level vision tasks, such as image denoising, image brightening, and super-resolution.