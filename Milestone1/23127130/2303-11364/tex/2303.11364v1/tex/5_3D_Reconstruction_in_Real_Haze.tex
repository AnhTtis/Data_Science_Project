\section{Experimentally Captured Results}
\subsection{Data Collection}
We captured three indoor hazy scenes using two professional haze machines and an iPhone 12 Pro.
The scenes contain 38, 82, and 68 hazy images and 79, 47, and 47 clear images for testing respectively.
To ensure temporal consistency (dynamic haze is out of scope for this paper), we capture the data after the haze has settled and appears steady.
After the capturing, we merge hazy images and clear images under the same scene and adopt COLMAP to register camera positions, which yields a consistent registration for the hazy and clear view for ease of evaluation.
We scale and translate the object such that it lies inside the unit sphere and roughly at the center of the world coordinates.
More details on the data capture are described in the supplement.

\subsection{Implementation details.}
Several changes are applied to the model to facilitate the optimization for captured data.
First, we remove the mask loss (~\cref{eq:mask_loss}).
Second, similar to NeuS~\cite{wang2021neus}, the region outside uses the parameterization of NeRF++\cite{zhang2020nerf++}, for which we also applied the haze attenuation following \cref{eq:C_surface,eq:alpha,eq:C_haze}.

\subsection{Comparison}
For evaluation, we adopt the same baselines described in \cref{sec:comparisons}.
The 2D photometric quality is evaluated using the same metrics, and geometry reconstruction quality is omitted due to the lack of ground truth.

\paragraph{Quantitative Evaluation.}
The results are reported in \cref{tab:real_quantitative}.
One can observe that our method outperforms other baselines in all metrics, demonstrating the robustness of the proposed method in the real-world scenario.

\paragraph{Qualitative Evaluation.}
As shown in \cref{fig:real_qualitative},
vanilla NeuS includes haze in the synthesized views.
On the other hand, adopting 2D image dehazing or video dehazing methods as pre-processing also cannot generate desired clear results since these learning-based models have limited generalization ability in real-world data.
Although using COLMAP as pre-processing can suppress residual haze, the dehazed results tend to have the color distortion problem and the limited structure (see the background at the $3^{rd}$ column of \cref{fig:real_qualitative}).
Our method achieves significantly better visual quality compared to other baselines.
