\documentclass[11pt, 
%  onecolumn,
 reprint,
 %preprint,
 notitlepage,
 amsmath,amssymb,
 aps,pra,superscriptaddress
]{revtex4-2}

\usepackage[T1]{fontenc}
\usepackage[utf8]{inputenc}
\usepackage[colorlinks,citecolor=blue,urlcolor=blue]{hyperref} %pra criar índice remissivo
\usepackage{graphics}
\usepackage{subfigure}
\usepackage{natbib}
\usepackage{physics}
\usepackage{siunitx}
\usepackage{chemformula}

\renewcommand{\theequation}{S\arabic{equation}}
\renewcommand{\thefigure}{S\arabic{figure}}
\renewcommand{\bibnumfmt}[1]{[S#1]}
\renewcommand{\citenumfont}[1]{S#1}


%%%%%%%%%%%%%%%%%%%%%%%%%%%%%%%%%%%%%%%%%
%%%%%%%%%%%%%%%%%%%%%%%%%%%%%%%%%%%%%%%%%
\begin{document}

\title{Supporting Information to\\ Classical Density Functional Theory Reveals Structural Information of \ch{H2} and \ch{CH4} Fluids Adsorbed in MOF-5}
\author{Elvis do A. Soares}%
\email{elvis.asoares@gmail.com}
\affiliation{Engenharia de Processos Químicos e Bioquímicos (EPQB), Escola de Química, Universidade Federal do Rio de Janeiro, 21941-909, Rio de Janeiro, RJ, Brazil}%
\author{Amaro G. Barreto Jr.}%
\affiliation{Engenharia de Processos Químicos e Bioquímicos (EPQB), Escola de Química, Universidade Federal do Rio de Janeiro, 21941-909, Rio de Janeiro, RJ, Brazil}%
\author{Frederico W. Tavares}%
\email{tavares@eq.ufrj.br}
\affiliation{Engenharia de Processos Químicos e Bioquímicos (EPQB), Escola de Química, Universidade Federal do Rio de Janeiro, 21941-909, Rio de Janeiro, RJ, Brazil}%
\affiliation{Programa de Engenharia Química, COPPE, Universidade Federal do Rio de Janeiro, 21941-909, Rio de Janeiro, RJ, Brazil}%

\maketitle

\section{The Two-Yukawa representation of LJ interaction}

The attractive contribution of the LJ potential in this work is defined as
\begin{align}
    u_\text{att}(r) = \begin{cases}
        0, \quad & r<\sigma \\
        4\epsilon\left[ \left( \frac{\sigma}{r} \right)^{12}-\left( \frac{\sigma}{r} \right)^{6}\right] , & r>\sigma,
    \end{cases}
    \label{eq:attractive_contribution}
\end{align}

This attractive contribution of the LJ potential can be mapped into the sum of two Yukawa potentials as 
\begin{align}
    u_\text{ty}(r) = -\epsilon_1 \frac{e^{-\lambda_1(r/\sigma-1)}}{r/\sigma} -\epsilon_2 \frac{e^{-\lambda_2(r/\sigma-1)}}{r/\sigma},
\end{align}
and our set of parameters $\{\epsilon_i, \lambda_i\}$ was obtained by the following constraints
\begin{enumerate}
    \item LJ zero: $u_\text{ty}(\sigma) =  u_\text{lj}(\sigma) = 0$;
    \item Derivative at zero: $\left. \dv{u_\text{ty}(r)}{r}\right|_{r \to \sigma} = \left. \dv{u_\text{lj}(r)}{r}\right|_{r \to \sigma} = -\frac{24 \epsilon }{\sigma }$;
    \item Integral of potential: $\int_\sigma^\infty u_\text{ty}(r) 4 \pi r^2 \dd{r} = \int_\sigma^\infty u_\text{lj}(r) 4 \pi r^2 \dd{r} = -\frac{32}{9} \pi  \sigma ^3 \epsilon$;
    \item Integral of squared potential: $\int_\sigma^\infty [u_\text{ty}(r)]^2 4 \pi r^2 \dd{r} = \int_\sigma^\infty [u_\text{lj}(r)]^2 4 \pi r^2 \dd{r} = \frac{512}{315} \pi  \sigma ^3 \epsilon ^2$.
\end{enumerate} 
These constraints are derived from the necessary condition to accurately reproduce the thermodynamic properties of a LJ fluid. Therefore, our parameters are $\epsilon_1 = -\epsilon_2 = 1.8577 \epsilon$, $\lambda_1 = 2.5449$, and $\lambda_2 = 15.4641$. 

The set of parameters reported by \citet{Kalyuzhnyi1996}($\epsilon_1 = 2.03\epsilon$, $\epsilon_2 = 1.6438\epsilon$,  $\lambda_1 = 2.69$ and $\lambda_2 = 14.7$) and from Tang \emph{et al.}~\cite{Tang1997,Tang2001}($\epsilon_1 = -\epsilon_2 = 2.1714\epsilon$,  $\lambda_1 = 2.9637$ and $\lambda_2 = 14.0167$) are quite different from our parameters. This is due to the different constraints used and mainly the choice of the potential length in Eq.~\eqref{eq:attractive_contribution}. Here we used $\sigma$ as the potential length in the attractive potential, and the previously cited works used the hard-core Barker-Henderson diameter.

\section{Methane Fluid parameters}

In this work, we used the JZG EoS to represent the thermodynamics of LJ fluids. 

For methane molecule can be approximated by a LJ particle as reported by the TraPPE force-field~\cite{Martin1998}, with the following parameters ($\sigma_{\ch{CH4}} = 3.73$ \si{\angstrom} and $\epsilon_{\ch{CH4}}/k_B = 148.0$ K. The PC-SAFT~\cite{Gross2001} also represents the methane molecule by a LJ particle with appropriate parameters ($m_{\ch{CH4}} = 1.0000$, $\sigma_{\ch{CH4}} = 3.7039$ \si{\angstrom}, $\epsilon_{\ch{CH4}}/k_B = 150.03$ K). However, the PC-SAFT dispersive contribution was calculated using a 2-order Baker-Henderson perturbation theory.

Figure~\ref{fig:vle_CH4} presents the Vapor-Liquid equibrium diagram and the vapor pressure of \ch{CH4} calculated from the JZG EoS in comparison with the experimental data. The dashed ans solid lines represent the JZG EoS with the PC-SAFT and TraPPE parameters, respectively. The VLE curve, as determined by the JZG EoS using the TraPPE parameters, shows a strong alignment with the experimental data from NIST~\cite{CH4NIST}. However, the VLE curve derived using the PC-SAFT parameters exhibits a minor discrepancy, specifically in computing the density of the liquid phase. The PC-SAFT results also overestimates the critical temperature when compared to the TraPPE results.

\begin{figure}[htbp]
    \centering
    \includegraphics[width=\linewidth]{vle-methane}
    \caption{Vapor-Liquid Equilibria (VLE) diagram of methane fluid.  Open symbols: Experimental Data. Lines: JZG EoS with PC-SAFT parameters (\emph{dashed line}) and TraPPE parameters (\emph{solid line}). }
    \label{fig:vle_CH4}
\end{figure} 

The Figure~\ref{fig:pressure_CH4} presents the vapor saturated pressure of methane fluid. In this case, the two set of parameters can represent well the saturation pressure. 

\begin{figure}[htbp]
    \centering
    \includegraphics[width=\linewidth]{pressure-methane}
    \caption{Vapor pressure of methane fluid. Open symbols: Experimental Data. Lines: JZG EoS with PC-SAFT parameters (\emph{dashed line}) and TraPPE parameters (\emph{solid line}). }
    \label{fig:pressure_CH4}
\end{figure} 

\section{The fundamental measure theory}

The functional $F_\text{hs}$ represents the excess Helmholtz free-energy functional due to hard-sphere exclusion volume, which have been described by the modified fundamental measure theory (FMT)~\cite{Rosenfeld1989a}. In this work, we apply the White-Bear functional \cite{Yu2002a,Roth2002} for the hard-sphere Helmholtz free-energy contribution as
\begin{align}
    \beta F_\text{hs}[\rho (\vb*{r})] = \int_{\mathcal{V}} \dd{\vb*{r}} \Phi_\text{FMT}(\{ n_\alpha(\vb*{r})\}),
\end{align}
where the reduced free energy density of a mixture of hard-spheres, $\Phi_\text{FMT}$, is a function of the set of weighted densities, $n_\alpha(\vb*{r})$.

The free energy density based on the antisymmetrized version of the Rosenfeld's functional~\cite{Rosenfeld1997} is
\begin{align}
    \Phi_\text{FMT}(\{n_\alpha(\vb*{r})\}) &= \phi_1(n_3)n_0 + \phi_2(n_3)(n_1 n_2 - \vb*{n}_{v1} \cdot \vb*{n}_{v2}) \nonumber \\
    &+\phi_3(n_3)n_2^3(1 - \vb*{n}_{v2} \cdot \vb*{n}_{v2}/n_2^2)^3,
\end{align}
where the first two terms of the binomial expansion of $(1 - \vb*{n}_{v2} \cdot \vb*{n}_{v2}/n_2^2)^3$ recover the usual White Bear functional. As reported by \citet{Kessler2021} the resulting antisymmetrized functional yields accurate results for hard spheres in narrow cylindrical while retaining the full three-dimensional properties and the bulk behavior of the original White Bear functional. 

The White-Bear mark I functions are defined as
\begin{align}
    \phi_1(n_3) &=  -\ln{(1-n_3)}, \\
    \phi_2(n_3) &=  \frac{1}{1-n_3}, \\
    \phi_3(n_3) &= \frac{2n_3 +2(1-n_3)^2\ln(1-n_3)}{72 \pi n_3^2 (1-n_3)^2},
\end{align}
with the weigthed densities defined as
\begin{align}
    n_\alpha(\vb*{r}) \equiv \int_{\mathcal{V}} \dd{\vb*{r}'} \rho(\vb*{r}') \omega^{(\alpha)}(\vb*{r}-\vb*{r}'),
    \label{eq:weighted_densities}
\end{align}
whose the linearly independent weights are defined as
\begin{align}
    \omega^{(3)}(\vb*{r}) &= \Theta(d/2-|\vb*{r}|), \\
    \omega^{(2)}(\vb*{r}) &= |\grad{\Theta(d/2-|\vb*{r}|)}| = \delta(d/2-|\vb*{r}|), \\
    \vb*{\omega}^{(v2)}(\vb*{r}) &= \grad{\Theta(d/2-|\vb*{r}|)} = \frac{\vb*{r}}{r}\delta(d/2-|\vb*{r}|),     
\end{align}
and the dependent weights given by $\omega^{(0)}(\vb*{r}) = \omega^{(2)}(\vb*{r})/\pi d^2 $, $\omega^{(1)}(\vb*{r}) = \omega^{(2)}(\vb*{r})/2\pi d $, and $\vb*{\omega}^{(v1)}(\vb*{r}) = \vb*{\omega}^{(v2)}(\vb*{r})/2\pi d $. Here, $\Theta(r)$ is the Heaviside step function, $\delta(r)$ is the Dirac delta function, and $d$ is the hard-sphere diameter. We can note that 
\begin{align}
    \int_{\mathcal{V}} \omega^{(3)}(\vb*{r}) \dd{\vb*{r}} &= 4 \pi \int_0^\infty \theta(d/2-|\vb*{r}|) r^2 \dd{r}= \frac{\pi}{6}\pi d^3, \\
    \int_{\mathcal{V}} \omega^{(2)}(\vb*{r}) \dd{\vb*{r}} &= 4 \pi \int \delta(d/2-|\vb*{r}|) r^2 \dd{r} =  \pi d^2, \\
    \int_{\mathcal{V}} \omega^{(v2)}(\vb*{r}) \dd{\vb*{r}} &= 4 \pi \int \frac{\vb*{r}}{r}\delta(d/2-|\vb*{r}|) r^2 \dd{r} =  0.
\end{align} 

The Fourier transforms of these weight functions are given in the form 
\begin{align}
    \widetilde{\omega}^{(3)}(\vb*{k}) &= \int \Theta(d/2-|\vb*{r}|) e^{-i\vb*{k}\cdot \vb*{r}} \dd[3]{r} \nonumber \\
    &= \int_0^{2\pi}\dd{\phi} \int_0^{d/2}\dd{r}r^2 \int_0^{\pi}\dd{\theta} \sin{\theta}e^{-ikr\cos{\theta}} \nonumber \\
    &= \frac{2\pi}{i k} \int_0^{d/2}\dd{r}r \left[ e^{ikr}-e^{-ikr} \right] \nonumber \\
    &= \frac{\pi d^2}{k} \left[ \frac{\sin(k d/2)}{(kd/2)^2}- \frac{\cos(k d/2)}{kd/2}\right] = \frac{\pi d^2}{k}  j_1(kd/2)
    \label{eq:omega3}
\end{align}
and
\begin{align}
    \widetilde{\omega}^{(2)}(\vb*{k}) &= \int \delta(d/2-|\vb*{r}|) e^{-i\vb*{k}\cdot \vb*{r}} \dd[3]{r} \nonumber \\
    &= \int_0^{2\pi}\dd{\phi} \int_0^{\infty}\dd{r}r^2 \delta(d/2-r)\int_0^{\pi}\dd{\theta} \sin{\theta}e^{-ikr\cos{\theta}} \nonumber \\
    &= \frac{2\pi}{i k} \int_0^{\infty}\dd{r}\delta(d/2-r) r \left[ e^{ikr}-e^{-ikr} \right] \nonumber \\
    &= \pi d^2\frac{\sin(k d/2)}{kd/2} = \pi d^2 j_0(kd/2)
\end{align}
and
\begin{align}
    \widetilde{\vb*{\omega}}^{(v2)}(\vb*{k}) &= -\frac{1}{i d/2} \dv{}{\vb*{k}}\left[\tilde{\omega}^{(2)}(\vb*{k}) \right] =  - i \vb*{k} \tilde{\omega}^{(3)}(\vb*{k}) 
\end{align}

The chemical potential is obtained by the bulk composition as
\begin{align}
    &\beta \mu^{\text{hs}} = \pdv{\Phi_\text{FMT}^{(b)}}{\rho^{(b)}}.
    \label{eq:hs_chemical_potential}
\end{align}

\section{The Weigthed Density Functional Theory}
\label{app:wdft}

The term $F_\text{att}$ represents the excess Helmholtz free-energy contribution due to the particle-particle attractive interaction, which can be described by the novel weighted density functional theory (WDFT) \cite{Yu2009} as the sum of a mean-field term and a correlation contribution, respectively, in the form 
\begin{align}
    F_\text{att}[\rho (\vb*{r})] =\ & \frac{1}{2} \int_{V} \dd{\vb*{r}} \int_{V} \dd{\vb*{r}'} \rho(\vb*{r}) u_\text{att}(|\vb*{r}-\vb*{r}'|) \rho(\vb*{r}') \nonumber \\
    &+ k_B T \int_{V} \dd{\vb*{r}} \Phi_\text{corr}(\bar{\rho}(\vb*{r})),
\end{align}
where the weighted density field is given by 
\begin{align}
    \bar{\rho}(\vb*{r})= \int_{V} \dd{\vb*{r}'} \rho(\vb*{r}')\bar{\omega}(\vb*{r}-\vb*{r}')
\end{align}
where $\bar{\omega}(\vb*{r})= \Theta(d-|\vb*{r}|)/(4\pi d^3/3)$, and $\Theta(x)$ is the Heaviside function. The Fourier transform of this weight function can be calculated using Eq.~\eqref{eq:omega3} but changing $d/2$ to $d$. The correlation free-energy density is defined as
\begin{align}
    \Phi_\text{corr}(\rho) = \beta \frac{F_\text{JZG}(\rho)}{V}-\beta \frac{F_\text{hs}(\rho)}{V}-\beta \rho^2 a_\text{mft},
\end{align}
such that the excess Helmholtz free-energy of the bulk fluid coincides with the appropriated free-energy for the LJ fluid. The MFT parameter can be identified as the integral $a_\text{mft} = 2 \pi \int_0^\infty u_\text{att}(r) r^2 \dd{r} = -(16/9) \pi \epsilon \sigma^3$. 

The chemical potential is obtained by the bulk composition as
\begin{align}
    & \mu^{\text{att}} = \pdv{(F_\text{JZG}(\rho_b)/V)}{\rho^{(b)}}.
    \label{eq:att_chemical_potential}
\end{align}

\section{The Grand-Potential Functional Derivatives}

For a LJ fluid enclosed by a volume $\mathcal{V}$, with temperature $T$, and chemical potential $\mu$ specified, under the action of an external potential $\phi^\text{ext}(\vb*{r})$, the grand potential, $\Omega$, is written as
\begin{align}
    &\Omega[\rho(\vb*{r})] = F^\text{id}[\rho(\vb*{r})]  + F^\text{exc}[\rho(\vb*{r})] + \int_{\mathcal{V}} \dd{\vb*{r}} \qty(\phi^\text{ext}(\vb*{r})-\mu ) \rho(\vb*{r}) , 
\end{align}
where 
\begin{align}
    F^\text{id}[\rho (\vb*{r})] = k_B T \int_{\mathcal{V}} \dd{\vb*{r}} \rho(\vb*{r})[\ln(\rho (\vb*{r})\Lambda^3)-1],
\end{align}
and 
\begin{align}
    F^\text{exc}[\rho(\vb*{r})] =\ & F^\text{hs}[\rho(\vb*{r})] + F^\text{att}[\rho(\vb*{r})]
\end{align}

The variations are obtained keeping constant the chemical potentials $\mu$, the external potential $\phi^\text{ext}(\vb*{r})$, the temperature $T$ and the volume $V$.
\begin{align}
    \Omega[\rho + \delta \rho] =\ & F^\text{id}[\rho + \delta \rho]  + F^\text{hs}[\rho + \delta \rho] + F^\text{att}[\rho + \delta \rho] \nonumber \\
    &  + \int_{\mathcal{V}} \dd{\vb*{r}} \qty(\phi^\text{ext}(\vb*{r})-\mu) [\rho(\vb*{r})+ \delta \rho(\vb*{r})], 
\end{align}
where the ideal gas term is given by 
\begin{align}
    &F^\text{id}[\rho + \delta \rho]  = k_B T  \int_{\mathcal{V}} \dd{\vb*{r}} [\rho(\vb*{r}) + \delta \rho(\vb*{r})]\{\ln[(\rho (\vb*{r}) + \delta \rho(\vb*{r}))\Lambda^3]-1\} \nonumber \\
    &= F^\text{id}[\rho] + k_B T  \int_{\mathcal{V}} \dd{\vb*{r}} \{ \delta \rho(\vb*{r})\ln(\rho(\vb*{r})\Lambda^3) + \mathcal{O}(\delta \rho(\vb*{r})^2) \} ,
\end{align}
the HS term is given by 
\begin{align}
    &F^\text{hs}[\rho + \delta \rho]  = F^\text{hs}[\rho] \nonumber \\
    & + k_B T \sum_\alpha \iint_{\mathcal{V}} \dd{\vb*{r}} \dd{\vb*{r}'} \delta \rho(\vb*{r}) \pdv{n_\alpha(\vb*{r}')}{\rho(\vb*{r})} \pdv{ \Phi^\text{hs}}{n_\alpha(\vb*{r}')} + \mathcal{O}(\delta \rho(\vb*{r})^2) ,
\end{align}
the WDFT term is given by 
\begin{align}
    &F^\text{att}[\rho + \delta \rho]  = F^\text{corr}[\rho] \nonumber + k_B T \iint_{\mathcal{V}} \dd{\vb*{r}} \dd{\vb*{r}'} \delta \rho(\vb*{r}) \pdv{\bar{\rho}(\vb*{r}')}{\rho(\vb*{r})} \pdv{ \Phi^\text{corr}}{\bar{\rho}(\vb*{r})} \\
    &+\frac{1}{2}\iint_{\mathcal{V}} \dd{\vb*{r}}\dd{\vb*{r}'}[\rho(\vb*{r})+\delta \rho(\vb*{r})] u_\text{att}(|\vb*{r}-\vb*{r}'|)[\rho(\vb*{r}')+\delta \rho(\vb*{r}')] \nonumber \\
    =\ & F^\text{corr}[\rho] + \frac{1}{2}\iint_{\mathcal{V}} \dd{\vb*{r}}\dd{\vb*{r}'}\rho(\vb*{r}) u_\text{att}(|\vb*{r}-\vb*{r}'|)\rho(\vb*{r}') \nonumber \\
    & +k_B T \iint_{\mathcal{V}} \dd{\vb*{r}} \dd{\vb*{r}'} \delta \rho(\vb*{r}) \pdv{\bar{\rho}(\vb*{r}')}{\rho(\vb*{r})} \pdv{ \Phi^\text{corr}}{\bar{\rho}(\vb*{r})} \nonumber \\
    &+ \iint_{\mathcal{V}} \dd{\vb*{r}}\dd{\vb*{r}'}\delta \rho(\vb*{r}) u_\text{att}(|\vb*{r}-\vb*{r}'|)\rho(\vb*{r}')+ \mathcal{O}(\delta \rho(\vb*{r})^2),
\end{align} 
and the weighted densities functional derivatives being 
\begin{align}
    \pdv{n_\alpha(\vb*{r}')}{\rho(\vb*{r})} &=  \int_{\mathcal{V}} \dd{\vb*{r}''} \delta(\vb*{r}''-\vb*{r}) \omega^{(\alpha)}(\vb*{r}'-\vb*{r}'') = \omega^{(\alpha)}(\vb*{r}'-\vb*{r}) \\
    \pdv{\bar{\rho}(\vb*{r}')}{\rho(\vb*{r})} &= \int_{\mathcal{V}} \dd{\vb*{r}''} \delta(\vb*{r}''-\vb*{r}) \bar{\omega}(\vb*{r}'-\vb*{r}'') =  \bar{\omega}(\vb*{r}'-\vb*{r}) 
\end{align}

Therefore, 
\begin{widetext}
\begin{align}
    \delta \Omega &= \Omega[\rho + \delta \rho] - \Omega[\rho] \nonumber \\
    &=  k_B T  \int_{\mathcal{V}} \dd{\vb*{r}} \delta \rho(\vb*{r})\ln(\rho(\vb*{r})\Lambda^3) +  k_B T \sum_\alpha \iint_{\mathcal{V}} \dd{\vb*{r}} \dd{\vb*{r}'} \delta \rho(\vb*{r}) \omega^{(\alpha)}(\vb*{r}'-\vb*{r}) \pdv{ \Phi^\text{hs}}{n_\alpha(\vb*{r}')} \nonumber \\
    & + \iint_{\mathcal{V}} \dd{\vb*{r}}\dd{\vb*{r}'}\delta \rho(\vb*{r}) u_\text{att}(|\vb*{r}-\vb*{r}'|)\rho(\vb*{r}') + k_B T \iint_{\mathcal{V}} \dd{\vb*{r}}\dd{\vb*{r}'} \delta \rho(\vb*{r}) \bar{\omega}(\vb*{r}'-\vb*{r}) \fdv{ F^\text{corr}}{\bar{\rho}(\vb*{r})} +  \int_{\mathcal{V}} \dd{\vb*{r}} \qty(\phi^\text{ext}(\vb*{r})-\mu ) \delta \rho(\vb*{r}).
    \label{eq:delta_omega}
\end{align}
\end{widetext}

Grouping the variational terms of $\delta \rho(\vb*{r})$,we get 
\begin{widetext}
    \begin{align}
    \delta \Omega &= \int_{\mathcal{V}} \dd{\vb*{r}} \delta \rho(\vb*{r})\left[k_B T \ln(\rho(\vb*{r})\Lambda^3) + k_B T \sum_\alpha \int_{\mathcal{V}} \dd{\vb*{r}'} \omega^{(\alpha)}(\vb*{r}'-\vb*{r}) \pdv{ \Phi^\text{hs}}{n_\alpha(\vb*{r}')} + \int_{\mathcal{V}}\dd{\vb*{r}'}u_\text{att}(|\vb*{r}-\vb*{r}'|)\rho(\vb*{r}') \right. \nonumber \\
    &\left. + k_B T \int_{\mathcal{V}}\dd{\vb*{r}'} \bar{\omega}(\vb*{r}'-\vb*{r}) \fdv{ F^\text{corr}}{\bar{\rho}(\vb*{r})} + \phi^\text{ext}(\vb*{r})-\mu \right], 
    \label{eq:delta_omega_better}
    \end{align}
\end{widetext}

Finally, the functional derivative is given as follows
\begin{align}
    \fdv{\Omega}{\rho(\vb*{r})} = k_B T \ln(\rho(\vb*{r})\Lambda^3) - k_B T c^{(1)}(\vb*{r}) + \phi^\text{ext}(\vb*{r})-\mu,
\end{align}
where we define the 1st direct correlation function in the form 
\begin{align}
    c^{(1)}(\vb*{r}) =\ & -\sum_\alpha \int_{\mathcal{V}} \dd{\vb*{r}'} \omega^{(\alpha)}(\vb*{r}'-\vb*{r}) \pdv{ \Phi^\text{hs}}{n_\alpha(\vb*{r}')}\nonumber \\
    & -\beta \int_{\mathcal{V}}\dd{\vb*{r}'}u_\text{att}(|\vb*{r}-\vb*{r}'|)\rho(\vb*{r}') \nonumber \\
    &- \int_{\mathcal{V}}\dd{\vb*{r}'} \bar{\omega}(\vb*{r}'-\vb*{r}) \fdv{ F^\text{corr}}{\bar{\rho}(\vb*{r})}.
\end{align} 

The thermodynamic equilibrium condition is obtained by making
\begin{align}
    \left. \fdv{\Omega}{\rho(\vb*{r})}\right|_{\mu,V,T} &= k_B T \ln(\rho(\vb*{r})\Lambda^3) - k_B T c^{(1)}(\vb*{r}) \nonumber \\
    &+ \phi^\text{ext}(\vb*{r})-\mu= 0,
    \label{eq:equilibrium_conditions}
\end{align}
where the equilibrium value of the grand potential is a minimum with respect to the density profile. Using the definition of the ideal chemical potential such that the total chemical potential is $\mu = k_B T \ln(\rho_b\Lambda^3) + \mu_\text{exc}$, we can rewrite the Eq.~\eqref{eq:equilibrium_conditions} in the form 
\begin{align}
    \rho(\vb*{r}) = \rho_b e^{-\beta\phi^\text{ext}(\vb*{r}) + c^{(1)}(\vb*{r}) + \beta \mu_\text{exc}}.
\end{align}


\bibliography{biblio.bib}

\end{document}
%%%%%%%%%%%%%%%%%%%%%%%%%%%%%%%%%%%%%%%%%
%%%%%%%%%%%%%%%%%%%%%%%%%%%%%%%%%%%%%%%%%