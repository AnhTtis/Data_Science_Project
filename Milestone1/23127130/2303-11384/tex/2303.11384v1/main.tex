\documentclass[11pt, 
 %onecolumn,
 reprint,%preprint
 notitlepage,
 amsmath,amssymb,
 aps,pra,superscriptaddress
]{revtex4-2}

\usepackage[T1]{fontenc}
\usepackage[colorlinks,citecolor=blue,urlcolor=blue]{hyperref} %pra criar índice remissivo
\usepackage{graphics}
\usepackage{subfigure}
\usepackage{natbib}
\usepackage{physics}
\usepackage{siunitx}
\usepackage{chemformula}

%%%%%%%%%%%%%%%%%%%%%%%%%%%%%%%%%%%%%%%%%
%%%%%%%%%%%%%%%%%%%%%%%%%%%%%%%%%%%%%%%%%
\begin{document}

\title{Classical Density Functional Theory Reveals Structural Information of \ch{H2} and \ch{CH4} Fluids Adsorbed in MOF-5}
\author{Elvis do A. Soares}%
\email{elvis.asoares@gmail.com}
\affiliation{Engenharia de Processos Químicos e Bioquímicos (EPQB), Escola de Química, Universidade Federal do Rio de Janeiro, 21941-909, Rio de Janeiro, RJ, Brazil}%
\author{Amaro G. Barreto Jr.}%
\affiliation{Engenharia de Processos Químicos e Bioquímicos (EPQB), Escola de Química, Universidade Federal do Rio de Janeiro, 21941-909, Rio de Janeiro, RJ, Brazil}%
\author{Frederico W. Tavares}%
\email{tavares@eq.ufrj.br}
\affiliation{Engenharia de Processos Químicos e Bioquímicos (EPQB), Escola de Química, Universidade Federal do Rio de Janeiro, 21941-909, Rio de Janeiro, RJ, Brazil}%
\affiliation{Programa de Engenharia Química, COPPE, Universidade Federal do Rio de Janeiro, 21941-909, Rio de Janeiro, RJ, Brazil}%

\date{\today}

\begin{abstract}
    This study employs classical Density Functional Theory (cDFT) to investigate the adsorption isotherms and structural information of \ch{H2} and \ch{CH4} fluids inside MOF-5. The results indicate that the adsorption of both fluids is highly dependent on the fluid temperature and the shape of the MOF-5 structure. Specifically, the \ch{CH4} molecules exhibit stronger interactions with the MOF-5 framework, resulting in a greater adsorbed quantity compared to \ch{H2}. Additionally, the cDFT calculations reveal that the adsorption process is influenced by the fluid-fluid spatial correlations between the fluid molecules and the external potential produced by the MOF-5 solid atoms. These findings are supported by comparison with experimental data and the structure factor of the adsorbed fluid inside the MOF-5. Overall, this work provides valuable insights into the adsorption mechanism of \ch{H2} and \ch{CH4} in MOF-5, emphasizing the importance of considering the structural properties of the adsorbed fluids in MOFs for predicting and designing their gas storage capacity.
\end{abstract} 

\maketitle

%%%%%%%%%%%%%%%%%%%%%%%%%%%%%%%%%%%%%%%%%
\section{Introduction}

The emergent global energy demand has resulted in a pressing need for the efficient storage of fuel gases such as hydrogen (\ch{H2}) and methane (\ch{CH4}). In response to this challenge, one of the most promising solutions is the use of nanoporous materials for gas adsorption.~\cite{Broom2016,Anstine2022} These materials are characterized by their pore sizes that fall within the nanoscale range. To achieve effective gas storage, a detailed understanding of the adsorption processes of these gases on nanoporous materials is essential. 

Metal-organic frameworks (MOFs) are a class of porous materials that emerged as promising alternatives for gas storage~\cite{Li2018,Ding2019} and separation applications~\cite{Qian2020,Lin2020,Fan2021} due to their high surface areas~\cite{Qiu2014,Bavykina2020}, tailorable topologies, tunable functional groups, and chemical, mechanical and thermal stabilities \cite{Howarth2016,Yuan2018}. MOFs consist of metal nodes and organic linkers combined to form porous structures with a wide range of geometries, sizes, and functionalities, making them versatile materials for other applications, such as catalysis, sensing, and drug delivery. 

Numerous studies have reported on the ability of MOFs to store \ch{H2} \cite{Suh2012,Zhao2022,Kwon2022} and \ch{CH4} \cite{Duren2004,Li2016}, making them attractive materials for gas storage applications. For efficient gas storage, it is essential to model the adsorption processes of these gases on MOFs. This involves understanding the interaction between the gas molecules and the material surface, and the factors that affect adsorption, such as temperature and pressure. By modeling these processes, we can design MOFs with optimized properties for gas storage.  
 
The most widely used techniques to study gas adsorption in MOFs are the Monte Carlo (MC) method~\cite{Yang2006,Keskin2009,Gallo2009,Getman2012} and classical Density Functional Theory (cDFT) calculations~\cite{Liu2009a,Fu2015a,Fu2015,Liu2016,Sang2021}. These methods allow for  modeling the adsorption process at the molecular level, taking into account the complex interaction between the gas molecules and the MOF structure.  

In this work, we use the cDFT of fluids to accurately model the intricate fluid-solid interactions and fluid-fluid correlations within MOF-5. Specifically, we investigate the behavior of \ch{H2} and \ch{CH4} fluids within the MOF-5 nanoporous material. According to the cDFT, the grand thermodynamic potential and Helmholtz free energy are functionals of the local density distribution, $\rho(\vb*{r})$. The interactions between the fluid particles can be taken into account by the use of the excess free-energy functional. To achieve accurate predictions, we incorporate the novel weighted density functional (WDFT) proposed by Yu~\cite{Yu2009} as the excess free-energy functional. The WDFT involves a corrected mean-field theory with a more accurate equation of state for the Lennard-Jones (LJ) fluid~\cite{Johnson1993}, which improves the description of fluid-fluid spatial correlations and bulk thermodynamic properties. 

The fluid-fluid correlations play a crucial role in the behavior of fluids, particularly in confined spaces or near interfaces. By considering the effects of fluid-fluid correlations, we better understanding of the spatial fluid structures inside the MOF-5 porous material. This improved understanding enables us to predict better the adsorption response to varying thermodynamic conditions, which has important implications for a wide range of  technological applications. 

Following this section, Section~\ref{sec:theory} presents a brief background regarding the implementation of the cDFT and its application for describing gas adsorption in MOFs. The discussion of the results from the proposed strategy is performed in Section~\ref{sec:results} and compared with the experimental data of \ch{H2} and \ch{CH4} adsorption in MOF-5 from the literature. Finally, Section~\ref{sec:conclusions} presents the conclusions of this work.

%%%%%%%%%%%%%%%%%%%%%%%%%%%%%%%%%%%%%%%%%%%%
\section{Theory and Methods}
\label{sec:theory}

\subsection{Classical Density Functional Theory}

According to the cDFT of fluids, the grand thermodynamic potential, $\Omega[\rho(\vb*{r})]$, and the Helmholtz free-energy, $F[\rho(\vb*{r})]$, are functionals of the local density distribution $\rho(\vb*{r})$. The grand potential functional $\Omega[\rho(\vb*{r})]$ is related to the free-energy functional $F[\rho(\vb*{r})]$ by a thermodynamic relation given as
\begin{align}
    \Omega[\rho(\vb*{r})] = F[\rho(\vb*{r})] -\mu  \int_V \dd{\vb*{r}}\rho(\vb*{r})
    \label{eq:dft_grandpotential}
\end{align}
where $\mu$ is the equilibrium chemical potential and $\phi_\text{ext}(\vb*{r})$ is an external potential acting on the fluid. The Helmholtz free-energy functional is determined by the sum of three terms, in the form
\begin{align}
    F[\rho(\vb*{r})] = F_\text{id}[\rho(\vb*{r})] + F_\text{exc}[\rho(\vb*{r})] + \int_V \dd{\vb*{r}}\phi_\text{ext}(\vb*{r})\rho(\vb*{r}),
    \label{eq:dft_freenergy}
\end{align}
where the first term is the ideal gas contribution, the second term is the excess free-energy parcel (excess of ideal gas), and last term is the external potential free-energy parcel. The ideal-gas contribution $ F^\text{id}$ is given by the exact expression
\begin{align}
    F^\text{id}[\rho (\vb*{r})] = k_B T \int_{V} \dd{\vb*{r}} \rho(\vb*{r})[\ln(\rho (\vb*{r})\Lambda^3)-1],
\end{align}
where $k_B$ is the Boltzmann constant, $T$ is the absolute temperature, and $\Lambda$ is the well-known thermal de Broglie wavelength. The grand potential $\Omega[\rho(\vb*{r})]$ has  a minimum value when $\rho(\vb*{r})$ is the equilibrium density distribution, \emph{i.e.}, the minimum value of $\Omega[\rho(\vb*{r})]$ is the equilibrium grand potential of the system. Then, the equilibrium density profile should be calculated extremizing the grand canonical potential, such that
\begin{align}
    \left. \fdv{\Omega[\rho(\vb*{r})]}{\rho(\vb*{r})}\right|_{\mu,V,T} =\ & k_B T\ln(\rho (\vb*{r})\Lambda^3) + \fdv{F_\text{exc}[\rho(\vb*{r})]}{\rho(\vb*{r})} \nonumber \\
    & + \phi_\text{ext}(\vb*{r})- \mu= 0.
    \label{eq:dft_equilibrium_condition}
\end{align}
Using the definition of the chemical potential in the form $\mu = k_B T\ln(\rho_b\Lambda^3) + \mu_\text{exc}$, where $\rho_b$ is the uniform fluid bulk density, we can write the simplified form of Eq.~\eqref{eq:dft_equilibrium_condition} as 
\begin{align}
    \rho(\vb*{r}) = \rho_b \exp[-\beta \phi_\text{ext}(\vb*{r}) + \Delta c^{(1)}(\vb*{r}) ]
    \label{eq:dft_equilibrium_condition_simplified}
\end{align}
where $\beta = (k_B T)^{-1}$ is the inverse of the thermal energy, and the term $\Delta c^{(1)}(\vb*{r})$ is related to the first-order direct correlation function $c^{(1)}(\vb*{r})$ through the relation $\Delta c^{(1)}(\vb*{r}) = - \beta \fdv*{F_\text{exc}[\rho (\vb*{r})]}{\rho(\vb*{r})} + \beta \mu^\text{exc} $, which acts as a correction for the external potential $\phi_\text{ext}(\vb*{r})$ due to the fluid-fluid correlation.

The excess free-energy functional $F_\text{exc}[\rho(\vb*{r})]$ contains all the information about the interaction between particles given by the pair potential $u(\vb*{r}-\vb*{r}')$. In our problem, the molecule-molecule interactions of the fluid can be well described by the Lennard-Jones potential in the form
\begin{align}
    u_\text{lj}(r) = 4\epsilon\left[ \left( \frac{\sigma}{r} \right)^{12}-\left( \frac{\sigma}{r} \right)^{6}\right].
\end{align}
To describe both the repulsive short-range and the attractive long-range interactions of the LJ potential, Barker and Henderson \cite{Barker1967a,Barker1967} suggested that the intermolecular potential can be decomposed as the sum $u_\text{lj}(r) = u_\text{hs}(r) + u_\text{att}(r)$ with the terms written as
\begin{align}
    u_\text{hs}(r) = \begin{cases}
        \infty, \quad & r<d \\
        0 , & r>d,
    \end{cases}
\end{align}
and
\begin{align}
    u_\text{att}(r) = \begin{cases}
        0, \quad & r<\sigma \\
        u_\text{ty}(r) , & r>\sigma,
    \end{cases}
\end{align}
where the hard-sphere potential $u_\text{hs}(r)$ is characterized by an effective  diameter given by the approximated formula \cite{Cotterman1986}
\begin{align}
    \frac{d}{\sigma} =\frac{1+0.2977 T^*}{1+0.33163 T^* + 1.0477\times 10^{-3} T^{*2}}
\end{align}
where $T^* = k_B T/\epsilon$ is the reduced temperature. The attractive potential $u_\text{att}(r) $ has been mapped onto a Two-Yukawa potential, $u_\text{ty}(r)$, to make the Fourier Transforms analytical, as discussed in Appendix~\ref{app:TYK}.

Therewith, the excess free-energy functional can be written in the form 
\begin{align}
    F_\text{exc}[\rho(\vb*{r})] =\ & F_\text{hs}[\rho(\vb*{r})] + F_\text{att}[\rho(\vb*{r})].
    \label{eq:excess_free-energy}
\end{align}
The term $F^\text{hs}$ represents the excess Helmholtz free-energy functional due to hard-sphere exclusion volume, described by the modified fundamental measure theory (FMT),\cite{Rosenfeld1989a} which provides an accurate description of the hard-sphere fluid structures. In this work, we have applied the White-Bear functional \cite{Yu2002a,Roth2002} for the hard-sphere Helmholtz free-energy contribution as
\begin{align}
    F_\text{hs}[\rho (\vb*{r})] = k_B T \int_{V} \dd{\vb*{r}} \Phi_\text{FMT}(\{ n^{(\alpha)}(\vb*{r})\}),
    \label{eq:hardsphere_free-energy}
\end{align}
where $\Phi_\text{FMT}$ is the local reduced free-energy density of a mixture of hard-spheres, which is a function of the set of weighted densities, $n^{(\alpha)}(\vb*{r})$. The term $F_\text{att}$ in Eq.~\eqref{eq:excess_free-energy} represents the excess Helmholtz free-energy contribution due to the particle-particle attractive interaction, which can be described by the novel weighted density functional theory (WDFT) \cite{Yu2009} as the sum of a mean-field term and a correlation contribution, respectively, in the form 
\begin{align}
    F_\text{att}[\rho (\vb*{r})] =\ & \frac{1}{2} \int_{V} \dd{\vb*{r}} \int_{V} \dd{\vb*{r}'} \rho(\vb*{r}) u_\text{att}(|\vb*{r}-\vb*{r}'|) \rho(\vb*{r}') \nonumber \\
    &+ k_B T \int_{V} \dd{\vb*{r}} \Phi_\text{corr}(\bar{\rho}(\vb*{r})),
\end{align}
where $\bar{\rho}$ is the weighted density field given by $\bar{\rho}(\vb*{r})= n_3(\vb*{r})/(\pi d^3/6)$, and the correlation free-energy density is given by 
\begin{align}
    \Phi_\text{corr}(\rho_b) = \beta \frac{F_\text{JZG}(\rho_b)}{V}-\beta \frac{F_\text{hs}(\rho_b)}{V}-\beta \rho_b^2 a_\text{mft}
\end{align}
such that the excess Helmholtz free-energy of the bulk fluid coincides with the apropriated JZG free-energy for the LJ fluid. The MFT parameter can be indentified as the integral $a_\text{mft} = 2 \pi \int_0^\infty u_\text{att}(r) r^2 \dd{r} = -(16/9) \pi \epsilon \sigma^3$.

Finally, the absolute adsorption quantity can be calculated by the definition as 
\begin{align}
    N = \int_{V _\text{uc}} \rho(\vb*{r}) \dd{\vb*{r}},
    \label{eq:adsorption_quantity}
\end{align}
where the local density distribution $\rho(\vb*{r})$ is given by Eq.~\eqref{eq:dft_equilibrium_condition_simplified}. The cDFT also possesses a notable capability of predicting the structural characteristics of the fluid confined within the porous material by utilizing the local density field, $\rho(\vb*{r})$. In our specific case, the local density field is a 3D field, which necessitates the utilization of isosurfaces to visualize our cDFT results. 

\subsection{Equation of state for the fluid phase}

In this study, the Johnson, Zollweg, and Gubbins (JZG) equation of state for LJ fluids was employed~\cite{Johnson1993}. While this equation has been shown to be highly accurate, it is important to note that it is a semi-empirical relation with 33 parameters. This EoS is written as  
\begin{align}
    \frac{F_\text{JZG}}{V} = \rho \epsilon\sum_{i=1}^8 \frac{a_i (\rho \sigma^3)^i}{i} + \rho \epsilon \sum_{i=1}^6 b_i G_i
\end{align}
where $a_i$ and $b_i$ are coefficients functions of temperature only. As reported in the original paper, the $G_i$ functions contain exponentials of the density and the one nonlinear parameter.

\begin{table}
    \caption{\label{tab:table_fluid}LJ parameters of the fluid molecules.}
    \begin{ruledtabular}
    \begin{tabular}{cccc}
    Molecule & $\epsilon/k_B$ (K) & $\sigma\ (\si{\angstrom})$  & Model \\ \hline
    \ch{CH4} & 150.03 & 3.704 & PC-SAFT\footnotemark\\
    \ch{H2} & 34.20 & 2.96 & Buch\footnotemark \\
    \end{tabular}
    \end{ruledtabular}
    \footnotetext[1]{Ref.~\cite{Gross2001}}
    \footnotetext[2]{Ref.~\cite{Buch1993}}
\end{table}

The \ch{H2} and \ch{CH4} molecules are described with the LJ potential, which the parameters are given in Table~\ref{tab:table_fluid}. The mapping of the equation of state is presented by the solid lines in Figure~\ref{fig:eos_H2} for \ch{H2} fluid and Figure~\ref{fig:eos_CH4} for \ch{CH4} fluid, where the open symbols are experimental data from Refs.~\cite{Leachman2009,Setzmann1991}. 

\begin{figure}[htbp]
    \centering
    \includegraphics[width=\linewidth]{h2-pressure.pdf}
    \caption{Equation of state of bulk \ch{H2} fluid over a broad pressure and temperature range. Open symbols: experimental data~\cite{Leachman2009,Setzmann1991}. Solid lines: JZG EoS with LJ parameters from Table~\ref{tab:table_fluid}.}
    \label{fig:eos_H2}
\end{figure}

Although the LJ approximation for the fluid made of \ch{H2} molecules is sufficient at high temperatures, it is noted that this approximation fails for temperatures below 100 K, which is well above the critical temperature, as shown in Figure~\ref{fig:eos_H2}. On the other hand, for the \ch{CH4} fluid, the LJ approximation is satisfactory even for the region below the critical temperature.

\begin{figure}[htbp]
    \centering
    \includegraphics[width=\linewidth]{ch4-pressure.pdf}
    \caption{Equation of state of bulk \ch{CH4} fluid over a broad pressure and temperature range. The caption is similar to Fig.\ref{fig:eos_H2}.}
    \label{fig:eos_CH4}
\end{figure}

Also relevant for adsorption calculations, the molar mass of \ch{H2} is $m_{\ch{H2}} = 2.016$ g/mol and the molar mass of \ch{CH4} is $m_{\ch{CH4}} = 16.043$ g/mol. 

\subsection{External Potential produced by the solid atoms}

The total external potential produced by the solid atoms on the fluid molecules is represented as a sum of the Lennard-Jones interaction between the solid atoms and the fluid molecules as defined following
\begin{align}
    \phi_\text{ext}(\vb*{r}) = \sum_{i \in\ \text{solid}} u^{(\text{lj})}_{if}(|\vb*{r}-\vb*{r}_i|) 
\end{align}
with the mixed parameters obtained by the Lorentz-Berthelot mixing rules given by  $\sigma_{if} = \frac{1}{2}(\sigma_{ii}+\sigma_{ff})$ and $\epsilon_{if} = (\epsilon_{ii}\epsilon_{ff})^{1/2}$. The  LJ parameters for the MOF-5 atoms are given in Table~\ref{tab:table_solid}.

\begin{table}
    \caption{\label{tab:table_solid}LJ parameters of the MOF-5 atoms.}
    \begin{ruledtabular}
    \begin{tabular}{cccc}
    Model & Atom & $\epsilon/k_B$ (K) & $\sigma\ (\si{\angstrom})$ \\ \hline
    DREIDING\footnotemark  & H  & 7.649 & 2.846 \\
    & C  & 47.856 & 3.473  \\
    & O & 48.151  &  3.033  \\
    & Zn & 27.677 & 4.045 \\
    \end{tabular}
    \end{ruledtabular}
    \footnotetext[1]{Ref.~\cite{Mayo1990}}
\end{table}

The MOF-5 has a lattice length of $L = 25.866\ \si{\angstrom}$ with unit cell volume of $V _\text{uc} = L^3 = 17305.325\ \si{\angstrom^3}$. The crystal density is $\rho_\text{cr} = 0.593\ \si{g/\centi\meter^3}$. 


\subsection{Numerical Methods}

To speed up the numerical calculations, we used of the Fast Fourier Transform (FFT) to calculate all the density convolutions. The analytical Fourier transform of the density profiles is computed as
\begin{align}
    \tilde{\rho}_{\vb*{k}} = \mathcal{F} \{ \rho(\vb*{r})\} = \int_V \dd{\vb*{r}} \rho(\vb*{r}) e^{-i\vb*{k}\cdot\vb*{r}}
\end{align}
where $\mathcal{F}$ represents the Fourier transform operator and $\mathcal{F}^{-1}$ represents the inverse Fourier transform operator given by 
\begin{align}
    \rho(\vb*{r}) \equiv \mathcal{F}^{-1}\{  \tilde{\rho}_{\vb*{k}}\}= \frac{1}{N_x N_y N_z}\sum_{\vb*{k}} \tilde{\rho}_{\vb*{k}}e^{i\vb*{k}\cdot\vb*{r}}
\end{align}
where $k_x = \{0,2\pi/L_x, 4\pi/L_x,\ldots,(N_x-1)2\pi/L_x\}$, $L_x$ is the length of the box in the $x$-direction, and $N_x=L_x/\Delta x$ is the number of gridpoints in the $x$-direction. And the same approach is used in other directions. We implemented the Fourier transform of the weight distribution $\tilde{\omega}_\alpha(\vb*{k})$ analytically in the reciprocal space. The weighted densities are then calculated 
\begin{align}
    n^{(\alpha)}(\vb*{r}) &= \int \dd{\vb*{r}'} \rho(\vb*{r}') \omega^{(\alpha)}(\vb*{r}-\vb*{r}') \nonumber \\
    &= \mathcal{F}^{-1} \{ \mathcal{F} \{ \rho(\vb*{r})\} \mathcal{F} \{ \omega^{(\alpha)}(\vb*{r})\}\}\nonumber \\
    &= \mathcal{F}^{-1} \{ \tilde{\rho}_{\vb*{k}} \tilde{\omega}^{(\alpha)}_{\vb*{k}}\}
    \label{eq:convolution-weighteddensity}
\end{align}

All these convolutions, i.e., Eqs.~\eqref{eq:convolution-weighteddensity} were solved using the FFT functions from the \emph{Scipy} package.~\cite{2020SciPy} Our group implements, the FMT and WDFT functionals in \emph{Python} code.~\cite{Elvis2023} The Gibbs phenomenon was reduced by multiplicating the Fourier transform $\tilde{\omega}_\alpha(\vb*{k})$ by the Lanczos $\sigma$-factor, $\sigma(k) = \sin(k/k_\text{max})/(k/k_\text{max})$, where $k_\text{max}$ is the maximum wavenumber from the FFT procedure.

The equilibrium condition for the cDFT, Eq.~\eqref{eq:dft_equilibrium_condition}, is solved using a fast inertial relaxation engine (FIRE) \cite{Bitzek2006,Guenole2020} also implemented in \emph{Python} by our group.~\cite{Elvis2020,Sermoud2021} The algorithm convergence criterion is $\norm{\beta \fdv*{\Omega}{\ln \rho(\vb*{r})} } \leq atol = 10^{-6}$.

\begin{figure}[htbp]
    \centering
    \includegraphics[width=\linewidth]{radialdistribution_lennardjones-rhob=0.84-T=0.75.pdf}
    \caption{Radial distribution function of LJ fluid over two different densities and temperatures. Open symbols: MD data.~\cite{Reatto1986} Solid line: our symmetrical DFT-1D result. Dashed line: our DFT-3D results with different grid sizes.}
    \label{fig:rdf}
\end{figure}

Figure~\ref{fig:rdf} illustrates an example of the Radial Distribution Function (RDF) calculation using our algorithm. We compared the 3D calculation with various numbers of grid points against the 1D symmetrical calculation with fine grid size. It is evident that an adequate number of grid points is required to reproduce the fluid's structural information inside the calculation box. Consequently, we discretized the unit cell of MOF-5 into $128^3$ gridpoints.

The initial density was considered uniform and equal to $\rho_b$. In the  highly repulsive region, where $\phi_\text{ext}(\vb*{r})/k_B > 1.6 \times 10^4$ K, the initial density $\rho(\vb*{r})$ was assumed to be zero. 

%%%%%%%%%%%%%%%%%%%%%%%%%%%%%%%%%%%%%%%%%%%
\section{Results and Discussion}
\label{sec:results}

The total gravimetric uptake, $N_a$ (in wt\%), is usually reported in experimental measurements.~\cite{Ahmed2017}  It can be calculated from the absolute adsorption quantity, $N$ from Eq.~\eqref{eq:adsorption_quantity}, by the relation
\begin{align}
    N_a = \frac{m_f N}{\rho_\text{cr}V _\text{uc} +m_f N} \times 100,
\end{align}
where $m_f$ is the fluid molecule mass, and $\rho_\text{cr} V _\text{uc}$ is the mass of the MOF unit cell. 

Fig.~\ref{fig:H2_absolute_adsorption_MOF5} shows the absolute isotherms of \ch{H2} adsorption in MOF-5 at 100 K, 125 K, 200 K, and 300 K in a wide range of pressure values. The DFT predictions align well with the experimental data from Ref.~\cite{Zhou2007}. These isotherms demonstrate that the amount of \ch{H2} adsorbed is significantly impacted by temperature. This can be attributed to the local density dependence on the external potential and temperature in Eq.~\eqref{eq:dft_equilibrium_condition_simplified}. As a result, when the temperature increases, the effect of the external potential on the local density decreases exponentially. 

\begin{figure}[htbp]
    \centering
    \includegraphics[width=\linewidth]{isotherm-absolute-H2-MOF5-Zhou2007.pdf}
    \caption{Absolute adsorption isotherms of \ch{H2} in MOF-5 over broad pressure and temperature ranges. Open symbols: experimental data~\cite{Zhou2007}. Solid lines: our DFT results.}
    \label{fig:H2_absolute_adsorption_MOF5}
\end{figure}

Similarly, Figure~\ref{fig:CH4_absolute_adsorption_MOF5} shows the absolute isotherms of \ch{CH4} adsorption in MOF-5 at 240 K, 270 K, and 300 K and the same pressure range. Overall,  the \ch{CH4} isotherms demonstrate higher adsorbed amount values than the \ch{H2} isotherms in MOF-5. This can be associated with the higher values of the LJ interaction parameter of \ch{CH4}. For instance, at 300 K and 60 bar, the amount of \ch{H2} adsorbed in MOF-5 is approximately 0.6\%, while the amount of \ch{CH4} adsorbed is 20\%. 

\begin{figure}[htbp]
    \centering
    \includegraphics[width=\linewidth]{isotherm-absolute-CH4-MOF5-Zhou2007.pdf}
    \caption{Absolute adsorption isotherms of \ch{CH4} in MOF-5 over a broad pressure and temperature ranges. Open symbols: experimental data~\cite{Zhou2007}. Solid lines: our DFT results.}
    \label{fig:CH4_absolute_adsorption_MOF5}
\end{figure}

An isosurface is a surface that represents points with the same value of the field within a volume of space. Figure~\ref{fig:profile_H2_MOF5} presents the local density isosurface of \ch{H2} inside the MOF-5 at the temperature of 300 K and three different pressure values: (a) 10 bar; (b) 30 bar; and (c) 60 bar. The blue isosurface represents the region with density values of 0.1 mol/L, the green isosurface with values of 1 mol/L, and the purple isosurface with values of 10 mol/L. We can observe that for any pressure, the blue region encloses all the atoms of MOF-5, indicating the adsorption of \ch{H2} molecules around those atoms. The green isosurface has a smaller area at 10 bar but increases in the area at higher pressures, indicating an increasing concentration of \ch{H2} in the vicinity of MOF-5 atoms. Finally, the purple isosurface starts to appear at 30 bar and increases in the area at 60 bar, and appears mainly in the region around the Zn atoms. 

The concentration of \ch{H2} molecules near the Zn atoms, as outlined in Fig.~\ref{fig:profile_H2_MOF5}c, can be attributed to the stronger interaction between \ch{H2} molecules and Zn atoms, as determined by their respective LJ parameters. Additionally, the high-concentration isosurface grows due to the volume exclusion effects that arise in the vicinity of the Zn atoms when \ch{H2} is present in high concentration. The excluded volume correlations near the \ch{Zn} atoms and the organic linkers limit the physical space available to accommodate additional \ch{H2} molecules in those regions.

\begin{figure*}[htbp]
    \centering
    \subfigure[][10 bar]{\includegraphics[scale=0.23]{H2-MOF5-densityprofile-10bar}}%
    \subfigure[][30 bar]{\includegraphics[scale=0.23]{H2-MOF5-densityprofile-30bar}}%
    \subfigure[][60 bar]{\includegraphics[scale=0.23]{H2-MOF5-densityprofile-60bar}}
    \caption{Density isosurfaces of \ch{H2} in MOF-5 with local density values of 0.1 mol/L (blue), 1 mol/L (green) and 10 mol/L (purple) at the temperature of 300 K and three different pressures of (a) 10 bar, (b) 30 bar and (c) 60 bar. }
    \label{fig:profile_H2_MOF5}
\end{figure*}

\begin{figure*}[htbp]
    \centering
    \subfigure[][10 bar]{\includegraphics[scale=0.23]{CH4-MOF5-densityprofile-10bar}}%
    \subfigure[][30 bar]{\includegraphics[scale=0.23]{CH4-MOF5-densityprofile-30bar}}%
    \subfigure[][60 bar]{\includegraphics[scale=0.23]{CH4-MOF5-densityprofile-60bar}}
    \caption{Density isosurfaces of \ch{CH4} in MOF-5 with local density values of 10 mol/L (yellow), 20 mol/L (red) and 40 mol/L (purple) at the temperature of 300 K and three different pressures of (a) 10 bar, (b) 30 bar and (c) 60 bar. }
    \label{fig:profile_CH4_MOF5}
\end{figure*} 

Figure~\ref{fig:profile_CH4_MOF5} illustrates the local density isosurfaces of \ch{CH4} within MOF-5 at 300 K and varying pressure values. The yellow isosurface represents the region with a density of 10 mol/L, the red isosurface with a density of 20 mol/L, and the purple isosurface with a density of 40 mol/L. At 10 bar, the yellow and red isosurfaces are just observed around the Zn atoms. At 30 bar, the yellow and red isosurfaces spread, and the purple isosurface begins to appear in the vicinity of the Zn atoms. At 60 bar, the yellow and red isosurfaces have spread throughout the whole box region, but the purple isosurface still surrounds the Zn atoms, indicating the highest \ch{CH4} concentration in this region.

The preferential adsorption of \ch{CH4} molecules at low and intermediate concentrations can be attributed to their strong interaction with the \ch{Zn} atoms, along with the volume exclusion effects in the proximity of these atoms, even at moderate pressures. Consequently, the concentration of \ch{CH4} molecules tends to be higher in other regions, such as in the vicinity of the organic linkers, as indicated by the negligible modification of the red isosurfaces from 30 bar to 60 bar. The highest concentration isosurface of \ch{CH4} is larger than that of \ch{H2} at 60 bar, owing to the differences in the molecular sizes of these two species.

To examine the spatial structure information of the adsorbed fluids within MOF-5, we compare the structure factor derived from the calculated density distributions with experimental data obtained from the X-ray diffraction pattern of MOF-5. The structure factor $S(q)$ is related to the power spectral density by   
\begin{align}
    S (q) \propto |\tilde{\rho}_{\vb*{k}}|^2,
\end{align}
where $q= (k_x^2+k_y^2+k_z^2)^{1/2}$ is the magnitude of the wavenumber. The results obtained from Figure~\ref{fig:structure_H2_MOF5} reveal the structure factor of \ch{H2} in MOF-5 at 300 K and different pressures, namely 10 bar, 30 bar, and 60 bar. Notably, the observed valleys in the structure factor are significantly correlated to the peaks observed in the experimental X-ray diffraction (XRD) pattern of MOF-5. In fact, there is no \ch{H2} molecule in the position of the MOF-5 atoms during the adsorption, but they are close. 
 
\begin{figure}[htbp]
    \centering
    \includegraphics[width=\linewidth]{structurefactor-H2-MOF5.pdf}
    \caption{Structure factor of \ch{H2} in MOF-5 at 300 K and three different pressure values. The black line represents the experimental XRD patterns of MOF-5 from Ref.~\cite{Zhao2009}. }
    \label{fig:structure_H2_MOF5}
\end{figure}

Figure~\ref{fig:structure_CH4_MOF5} presents the structure factor of \ch{CH4} fluid in MOF-5. We can note the higher values of the structure factor of \ch{CH4} than those values for \ch{H2}. In fact, these intensity values represent the adsorbed amount for \ch{CH4} inside the MOF-5 on each inverse length scale ($q=2\pi/l$ with $l$ being the length scale) as predicted by the results presented in Fig.~\ref{fig:CH4_absolute_adsorption_MOF5}. 

These findings are consistent with the fact that the fluid adsorption process takes place near the framework atoms and cannot accurately represent it as a spherical pore.

\begin{figure}[htbp]
    \centering
    \includegraphics[width=\linewidth]{structurefactor-CH4-MOF5.pdf}
    \caption{Structure factor of \ch{CH4} in MOF-5 at 300 K and three different pressure values. The caption is similar to Fig.~\ref{fig:structure_H2_MOF5}.}
    \label{fig:structure_CH4_MOF5}
\end{figure} 

%%%%%%%%%%%%%%%%%%%%%%%%%%%%%%%%%%%%%%%%%%%
\section{Conclusions}
\label{sec:conclusions}

The present study highlights the importance of temperature and pressure in the gas adsorption of MOF-5 for storage applications. For MOF-5, our results indicate that \ch{CH4} adsorption is more favorable than \ch{H2} due to the higher LJ interaction parameter of \ch{CH4}. At high pressures, \ch{CH4} saturates at the Zn atoms and occupies the region near organic linkers, while \ch{H2} concentration increases near the Zn atoms. The results suggest that the interaction parameters, volume exclusion effects, and the size of the gas molecules influence the adsorption behavior.

By implementing of cDFT, the local density distribution of adsorbed fluids within MOF-5 can be precisely computed, enabling insights into the adsorption mechanisms of \ch{H2} and \ch{CH4} molecules in this material. Notably, the adsorption mechanism is not directly dependent on the pore description of MOF-5 as a spherical pore or its pore size. Our calculated density isosurfaces suggested that the adsorption behavior is strongly linked to the spatial distribution of the constituent atoms within the framework.

The fluid structure factor can be derived from the local density distribution of adsorbed fluids within MOF-5. The structure factor curves can be experimentally measured using X-ray scattering techniques and provide valuable information about the spatial distribution of fluid molecules within MOF-5. The locations and concentrations of the adsorbed molecules can be determined by comparing the peaks in the structure factor curves with the positions of atoms in the MOF-5 framework.

The results presented here are significant as they provide insight into \ch{H2} and \ch{CH4} adsorption behavior in MOF-5 under different thermodynamic conditions for storage applications. Therefore, the combination of calculated fluid structure factor and X-ray scattering experiments can provide a comprehensive understanding of the adsorption behavior of fluid molecules inside the MOF-5, which is essential for developing efficient gas storage and separation technologies.


%%%%%%%%%%%%%%%%%%%%%%%%%%%%%%%%%%%%%%%%%%%%%%%%%%%%%%%%%%%%%%%%%%%%%%%%%%%%%%%%%%%%%%%%%
\section*{Acknowledgments}

The authors wish to thank Petrobras and Shell which provided financial support through the Research, Development, and Innovation Investment Clause, in collaboration with the Brazilian National Agency of Petroleum, Natural Gas, and Biofuels (ANP, Brazil). Additionally, this research was partially funded by CNPq, CAPES, and FAPERJ.

\section*{The data and code availability}
The data and code that support the findings of this study are available from the corresponding author in the repository: \url{https://github.com/elvissoares/PyDFTlj}.

%%%%%%%%%%%%%%%%%%%%%%%%%%%%%%%%%%%%%%%%%%%
\appendix

%%%%%%%%%%%%%%%%%%%%%%%%%%
\section{Two-Yukawa mapping of the LJ potential}
\label{app:TYK}

The attractive contribution of the LJ potential can be mapped into the two Yukawa potential as 
\begin{align}
    u_\text{ty}(r) = -\epsilon_1 \frac{e^{-\lambda_1(r/\sigma-1)}}{r/\sigma} -\epsilon_2 \frac{e^{-\lambda_2(r/\sigma-1)}}{r/\sigma}
\end{align}
where $\epsilon_1 = -\epsilon_2 = 1.8577 \epsilon$, $\lambda_1 = 2.5449$, and $\lambda_2 = 15.4641$. These parameters are derived from the necessary condition to accurately reproduce the thermodynamic properties of a LJ fluid, as determined by the first-order BH perturbation calculations.



\bibliography{biblio.bib}

\end{document}
%%%%%%%%%%%%%%%%%%%%%%%%%%%%%%%%%%%%%%%%%
%%%%%%%%%%%%%%%%%%%%%%%%%%%%%%%%%%%%%%%%%