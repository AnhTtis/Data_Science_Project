
\documentclass[pdflatex,sn-mathphys]{sn-jnl}
\usepackage{xcolor}
\usepackage{pdflscape}
\usepackage{caption}
\jyear{2023}%
%%
\newcommand\aap{A\&A}                % Astronomy and Astrophysics
\let\astap=\aap                          % alternative shortcut
\newcommand\aapr{A\&ARv}             % Astronomy and Astrophysics Review (the)
\newcommand\aaps{A\&AS}              % Astronomy and Astrophysics Supplement Series
\newcommand\actaa{Acta Astron.}      % Acta Astronomica
\newcommand\afz{Afz}                 % Astrofizika
\newcommand\aj{AJ}                   % Astronomical Journal (the)
\newcommand\ao{Appl. Opt.}           % Applied Optics
\let\applopt=\ao                         % alternative shortcut
\newcommand\aplett{Astrophys.~Lett.} % Astrophysics Letters
\newcommand\apj{ApJ}                 % Astrophysical Journal
\newcommand\apjl{ApJ}                % Astrophysical Journal, Letters
\let\apjlett=\apjl                       % alternative shortcut
\newcommand\apjs{ApJS}               % Astrophysical Journal, Supplement
\let\apjsupp=\apjs                       % alternative shortcut
% The following journal does not appear to exist! Disabled.
%\newcommand\apspr{Astrophys.~Space~Phys.~Res.} % Astrophysics Space Physics Research
\newcommand\apss{Ap\&SS}             % Astrophysics and Space Science
\newcommand\araa{ARA\&A}             % Annual Review of Astronomy and Astrophysics
\newcommand\arep{Astron. Rep.}       % Astronomy Reports
\newcommand\aspc{ASP Conf. Ser.}     % ASP Conference Series
\newcommand\azh{Azh}                 % Astronomicheskii Zhurnal
\newcommand\baas{BAAS}               % Bulletin of the American Astronomical Society
\newcommand\bac{Bull. Astron. Inst. Czechoslovakia} % Bulletin of the Astronomical Institutes of Czechoslovakia 
\newcommand\bain{Bull. Astron. Inst. Netherlands} % Bulletin Astronomical Institute of the Netherlands
\newcommand\caa{Chinese Astron. Astrophys.} % Chinese Astronomy and Astrophysics
\newcommand\cjaa{Chinese J.~Astron. Astrophys.} % Chinese Journal of Astronomy and Astrophysics
\newcommand\fcp{Fundamentals Cosmic Phys.}  % Fundamentals of Cosmic Physics
\newcommand\gca{Geochimica Cosmochimica Acta}   % Geochimica Cosmochimica Acta
\newcommand\grl{Geophys. Res. Lett.} % Geophysics Research Letters
\newcommand\iaucirc{IAU~Circ.}       % IAU Cirulars
\newcommand\icarus{Icarus}           % Icarus
\newcommand\japa{J.~Astrophys. Astron.} % Journal of Astrophysics and Astronomy
\newcommand\jcap{J.~Cosmology Astropart. Phys.} % Journal of Cosmology and Astroparticle Physics
\newcommand\jcp{J.~Chem.~Phys.}      % Journal of Chemical Physics
\newcommand\jgr{J.~Geophys.~Res.}    % Journal of Geophysics Research
\newcommand\jqsrt{J.~Quant. Spectrosc. Radiative Transfer} % Journal of Quantitiative Spectroscopy and Radiative Transfer
\newcommand\jrasc{J.~R.~Astron. Soc. Canada} % Journal of the RAS of Canada
\newcommand\memras{Mem.~RAS}         % Memoirs of the RAS
\newcommand\memsai{Mem. Soc. Astron. Italiana} % Memoire della Societa Astronomica Italiana
\newcommand\mnassa{MNASSA}           % Monthly Notes of the Astronomical Society of Southern Africa
\newcommand\mnras{MNRAS}             % Monthly Notices of the Royal Astronomical Society
\newcommand\na{New~Astron.}          % New Astronomy
\newcommand\nar{New~Astron.~Rev.}    % New Astronomy Review
\newcommand\nat{Nature}              % Nature
\newcommand\nphysa{Nuclear Phys.~A}  % Nuclear Physics A
\newcommand\pra{Phys. Rev.~A}        % Physical Review A: General Physics
\newcommand\prb{Phys. Rev.~B}        % Physical Review B: Solid State
\newcommand\prc{Phys. Rev.~C}        % Physical Review C
\newcommand\prd{Phys. Rev.~D}        % Physical Review D
\newcommand\pre{Phys. Rev.~E}        % Physical Review E
\newcommand\prl{Phys. Rev.~Lett.}    % Physical Review Letters
\newcommand\pasa{Publ. Astron. Soc. Australia}  % Publications of the Astronomical Society of Australia
\newcommand\pasp{PASP}               % Publications of the Astronomical Society of the Pacific
\newcommand\pasj{PASJ}               % Publications of the Astronomical Society of Japan
\newcommand\physrep{Phys.~Rep.}      % Physics Reports
\newcommand\physscr{Phys.~Scr.}      % Physica Scripta
\newcommand\planss{Planet. Space~Sci.} % Planetary Space Science
\newcommand\procspie{Proc.~SPIE}     % Proceedings of the Society of Photo-Optical Instrumentation Engineers
\newcommand\rmxaa{Rev. Mex. Astron. Astrofis.} % Revista Mexicana de Astronomia y Astrofisica
\newcommand\qjras{QJRAS}             % Quarterly Journal of the RAS
\newcommand\sci{Science}             % Science
\newcommand\skytel{Sky \& Telesc.}   % Sky and Telescope
\newcommand\solphys{Sol.~Phys.}      % Solar Physics
\newcommand\sovast{Soviet~Ast.}      % Soviet Astronomy (aka Astronomy Reports)
\newcommand\ssr{Space Sci. Rev.}     % Space Science Reviews
\newcommand\zap{Z.~Astrophys.}       % Zeitschrift fuer Astrophysik

%% as per the requirement new theorem styles can be included as shown below

\newcommand{\Lya}{Ly-$\rm{\alpha}$}
\newcommand{\fesc}{$f_{esc}(\rm{Ly\alpha})$ }
\newcommand{\fescC}{$f_{esc}(\rm{LyC})$ }
\newcommand{\Hb}{H$\rm{\beta}$ }
\newcommand{\Ha}{H$\rm{\alpha}$ }
\newcommand{\OIII}{[OIII]$_{4959,5007}$ }

\newcommand{\simname}{{\it Azahar}}

\theoremstyle{thmstyleone}%
\newtheorem{theorem}{Theorem}%  meant for continuous numbers
\newtheorem{proposition}[theorem]{Proposition}% 

\theoremstyle{thmstyletwo}%
\newtheorem{example}{Example}%
\newtheorem{remark}{Remark}%

\theoremstyle{thmstylethree}%
\newtheorem{definition}{Definition}%

\usepackage[superscript, biblabel, nomove]{cite}

\raggedbottom
%%\unnumbered% uncomment this for unnumbered level heads

\begin{document}

\title[Deciphering Lyman-$\alpha$ Emission Deep into the Epoch of Reionisation]{Deciphering Lyman-$\alpha$ Emission Deep into the Epoch of Reionisation}


\author*[1,2]{\fnm{Callum} \sur{Witten}}\email{cw795@cam.ac.uk}
\author[2,3]{\fnm{Nicolas} \sur{Laporte}}
\author[4]{\fnm{Sergio} \sur{Martin-Alvarez}}
\author[1,2]{\fnm{Debora} \sur{Sijacki}}
\author[1,2]{\fnm{Yuxuan} \sur{Yuan}}
\author[1,2]{\fnm{Martin G.} \sur{Haehnelt}}
\author[2,3]{\fnm{William} \sur{Baker}}
\author[5]{\fnm{James S.} \sur{Dunlop}}
\author[6]{\fnm{Richard S.} \sur{Ellis}}
\author[7]{\fnm{Norman} \sur{Grogin}}
\author[8]{\fnm{Garth} \sur{Illingworth}}
\author[9]{\fnm{Harley} \sur{Katz}}
\author[7]{\fnm{Anton M.} \sur{Koekemoer}}
\author[10]{\fnm{Daniel} \sur{Magee}}
\author[2,3,6]{\fnm{Roberto} \sur{Maiolino}}
\author[2,3]{\fnm{William} \sur{McClymont}}
\author[11]{\fnm{Pablo G.} \sur{P\'erez-Gonz\'alez}}
\author[2,3]{\fnm{David} \sur{Puskas}}
\author[12]{\fnm{Guido} \sur{Roberts-Borsani}}
\author[13]{\fnm{Paola} \sur{Santini}}
\author[2,3]{\fnm{Charlotte} \sur{Simmonds}}

\affil[1]{\tiny \orgdiv{Institute of Astronomy}, \orgname{University of Cambridge}, \orgaddress{\street{Madingley Road}, \city{Cambridge}, \postcode{CB3 0HA}, \country{UK}}}

\affil[2]{\tiny \orgdiv{Kavli Institute for Cosmology}, \orgname{University of Cambridge}, \orgaddress{\street{Madingley Road}, \city{Cambridge}, \postcode{CB3 0HA}, \country{UK}}}

\affil[3]{\tiny \orgdiv{Cavendish Labratory}, \orgname{University of Cambridge}, \orgaddress{\street{Madingley Road}, \city{Cambridge}, \postcode{CB3 9BB}, \country{UK}}}

\affil[4]{\tiny \orgdiv{Kavli Institute for Particle Astrophysics and Cosmology}, \orgname{Stanford University}, \orgaddress{\city{Stanford}, \postcode{CA 94305}, \country{USA}}}

\affil[5]{\tiny \orgdiv{Institute for Astronomy}, \orgname{University of Edinburgh, Royal Observatory}, \orgaddress{\city{Edinburgh}, \postcode{EH9 3HJ}, \country{UK}}}

\affil[6]{\tiny \orgdiv{Department of Physics and Astronomy}, \orgname{University College London}, \orgaddress{\street{Gower Street}, \city{London}, \postcode{WC1E 6BT}, \country{UK}}}

\affil[7]{\tiny \orgdiv{Space Telescope Science Institute}, \orgaddress{\street{3700 San Martin Drive}, \city{Baltimore}, \postcode{MD 21218}, \country{USA}}}

\affil[8]{\tiny \orgdiv{Department of Astronomy and Astrophysics}, \orgname{University of California}, \orgaddress{\city{Santa Cruz}, \postcode{CA 95064}, \country{USA}}}

\affil[9]{\tiny \orgdiv{Department of Physics}, \orgname{University of Oxford}, \orgaddress{\street{Denys Wilkinson Building, Keble Road}, \city{Oxford}, \postcode{OX1 3RH}, \country{UK}}}

\affil[10]{\tiny \orgdiv{UCO/Lick Observatory}, \orgname{University of California}, \orgaddress{\city{Santa Cruz}, \postcode{CA 95064}, \country{USA}}}

\affil[11]{\tiny \orgdiv{Centro de Astrobiolog\'{\i}a}, \orgname{CSIC-INTA}, \orgaddress{\street{Ctra. de Ajalvir km 4, Torrej\'on de Ardoz}, \city{Madrid}, \postcode{E-28850}, \country{Spain}}}

\affil[12]{\tiny \orgdiv{Department of Physics and Astronomy}, \orgname{University of California}, \orgaddress{\city{Los Angeles}, \postcode{CA 90095}, \country{USA}}}

\affil[13]{\tiny \orgdiv{INAF - Osservatorio Astronomico di Roma}, \orgaddress{\street{via di Frascati 33, 00078 Monte Porzio Catone}, \country{Italy}}}

%\keywords{distant galaxies, Lyman-$\alpha$ emission, merging galaxies, early Universe}

\maketitle

{\bf
A major event in cosmic history is the genesis of the first starlight in our Universe, ending the ``Dark Ages''. During this epoch, the earliest luminous sources were enshrouded in neutral and pristine gas, which was gradually ionised in a process called ``reionisation''. Hence, one of the brightest emission lines in star-forming galaxies, Lyman-$\alpha$ (\Lya), was predicted to emerge only towards the end of the epoch of reionisation, about one billion years after the Big Bang \cite{2014PASA...31...40D}. However, this picture has been challenged over the past decade by the surprising detection of \Lya\ in galaxies less than 500 million years old \cite{Bunker+23}. Here we show, by taking advantage of both high-resolution and high-sensitivity images from the James Webb Space Telescope programs PRIMER, CEERS and FRESCO, that all galaxies in our sample of \Lya\ emitters deep in the epoch of reionisation have {\it close companions}. To understand the physical processes that lead to the observed \Lya\ emission in our sample, we take advantage of novel on-the-fly radiative transfer magnetohydrodynamical simulations with cosmic ray feedback. We find that in the early Universe, the rapid build up of mass through frequent galactic mergers leads to very bursty star formation which in turn drives episodes of high intrinsic \Lya\ emission and facilitates the escape of \Lya\ photons along channels cleared of neutral gas. These merging galaxies reside in clustered environments thus creating sufficiently large ionised bubbles. This presents a solution to the long-standing puzzle of the detection of \Lya\ emission deep into the epoch of reionisation. 
}


\definecolor{orange}{rgb}{0.989988, 0.556646, 0.06421}
One of the main quests of modern extragalactic astronomy is to identify the first generation of stars and galaxies which formed a few hundred million years after the Big Bang, and to determine their physical properties. Combining imaging and spectroscopy is the ideal route to obtain robust estimates of their redshift, stellar mass, star-formation and dust content. In this context, the brightest intrinsic emission line in star-forming galaxies is Lyman-$\alpha$ (\Lya) at $\lambda$=1215.67$\text{\AA}$ \cite{2014PASA...31...40D}. However, within the first billion years of our Universe, galaxies are surrounded by neutral hydrogen formed during the recombination epoch. The neutrality of the intergalactic medium (IGM) deep into the epoch of reionisation ($z > 7$) leads to the decreasing fraction of detected \Lya~emission with increasing redshift \cite{Stark+17, deBarros+2019}, as it is absorbed by neutral hydrogen. However, the adoption of a strategy which targets the most luminous galaxies that show an excess in the 4.5$\mu$m band of \textit{Spitzer}, suggesting strong [OIII]+H$\beta$ fluxes, at $z\geq 7$ for spectroscopic follow up yields a 100\% success rate in the detection of \Lya~emission \cite{RB+16}. Two main hypotheses are used to explain the detection of \Lya~in the presence of an increasingly neutral Universe: a hard-radiation field driven by an active galactic nucleus (AGN) \cite{Laporte+17} or an enhanced radiation field produced by an overdensity of associated objects \cite{Castellano+16,Tilvi+20, Leonova+22, Morishita+22}. While it has been suggested that these most luminous galaxies hosting AGN could drive reionisation \cite{Naidu+20}, constraints on the escape fraction of such massive, luminous galaxies indicate otherwise \cite{Rosdahl+2022, Witten+23}. Moreover, numerical simulations where the interstellar medium (ISM) of galaxies is resolved, find low mass galaxies ($M_* \lesssim 10^7 M_\odot$) to be the most important contributor to reionisation \cite{Rosdahl+18, Trebitsch+21}. However, any conclusion regarding the impact on reionisation of the abundance of significant overdensities (protoclusters) in the early Universe \cite{Laporte+22, Morishita+22, Tang+23} is complicated by the presence of AGN within many of these early protoclusters \cite{Brinchmann+22}(Witten et al., in prep.). Given this continued significant uncertainty regarding what is facilitating the escape of \Lya~photons and hence the detection of \Lya~emitters (LAEs) deep into the epoch of reionisation (up to $z \sim 10.6$ \cite{Bunker+23}), we make use of the unparalleled sensitivity and resolution of \textit{JWST}/NIRCam \cite{Rieke+23} in order to accurately determine the properties of LAEs at $z > 7$, such as their stellar masses, star-formation rates and most notably their morphology. We compare these to the new \simname~suite (Martin-Alvarez et al., in prep.) of cosmological zoom-in simulations of high-redshift galaxies, which self-consistently model on-the-fly radiative transfer, magnetic fields and cosmic ray feedback, to shed light on the driving mechanism behind the observations of these LAEs for the very first time. 


We study all eight \textit{Hubble Space Telescope} (HST) photometrically selected galaxies that have been spectroscopically confirmed with the detection of \Lya~emission and which have been recently observed with \textit{JWST}/NIRCam as part of three different programs: PRIMER (PI: Dunlop), FRESCO (PI: Oesch), and CEERS (PI: Finkelstein). For the purpose of this paper, we hereby define an LAE at $z >7$ to be any galaxy exhibiting \Lya~emission with an equivalent-width greater than 10\AA, making it sufficiently bright as to be detected by current instruments. These eight galaxies range from $z = 7.1$ to $z = 8.7$ and fall within the GOODS-North, GOODS-South, EGS and COSMOS fields (see Table~\ref{tab:Galaxy_Properties} for the summary of their properties). Six of our sample are known to lie within overdensities (CEERS-44 and CEERS-1027 \cite{Tang+23}, z7-13433 and z7-20237 \cite{Jung+2022}, EGSY-8p68 \cite{Leonova+22} and COSY [Witten et al., in prep.]), five of which reside within known ionised bubbles. However, given that we observe \Lya\ emission, in an epoch where the IGM is largely neutral, in all of our sample a process must facilitate the escape of \Lya\ photons from the IGM as well as from the galaxy itself. 

NIRCam images of these galaxies indicate the presence of multiple components to each LAE, as shown in Figure~\ref{fig:stamps}. We estimate the redshift of each companion by carefully extracting the Spectral Energy Distribution (SED) of both the main component and its companions. We then fit each SED by assuming several parametric star formation histories  (SFH - burst, constant, delayed, exponential), allowing a large range for all parameters \cite{Carnall2018}. In all cases, the companions favour the same redshift as the spectroscopy of the main component. The stellar masses of our objects range from 3.9$\times$10$^7$ to 4.2$\times$10$^9$ M$_{\odot}$ and the star formation rates from $\sim 0.4$ to 42~M$_{\odot}$/yr. 

While we have strong constraints on the photometric redshift of the satellite galaxies, spectroscopy is clearly desirable for redshift confirmation and to improve constraints on galaxy properties. We therefore exploit existing XSHOOTER/VLT \cite{Vernet2011}, MOSFIRE/Keck \cite{MOSFIRE1,MOSFIRE2}, Near-Infrared Spectrograph (NIRSpec) on-board \textit{JWST} \cite{Jakobsen22} and NIRCam wide-field slitless spectroscopy (WFSS) observations \cite{Rieke2005} of these galaxies in order to spectroscopically confirm the presence of companion galaxies. NIRCam WFSS observations of two systems (z7-GSD-3811 and GSDY-2209651370, hereafter GSDY) clearly indicate extended \OIII emission covering both the main and satellite galaxies with small velocity offsets, spectroscopically confirming these systems as interacting (as seen in Figure~\ref{fig:FRESCO_spectra} and \ref{fig:Lya_spectra}). Moreover, the spectral range of the WFSS NIRCam observations allow us to place constraints on the \Hb flux from the main central galaxy and hence, assuming an intrinisic \Lya~to \Hb ratio of 24.9 (given case B recombination -- which assumes an escape fraction of Lyman-continuum photons of zero -- and $T_{e} = 10^4$~K) \cite{Osterbrock&Ferland06}, we can estimate the escape fraction of \Lya\ emission in these galaxies (reported in Table~\ref{tab:Galaxy_Properties}).

The spatial resolution of XSHOOTER and MOSFIRE (0.158'' and 0.1798'' per pixel, respectively) means that we are unable to resolve the \Lya~profile of the central and satellite galaxies for two of our systems (EGSY-8p68 and z7-20237). However, due to both the orientation of the slits and a large enough spatial and/or spectral offset between galaxies in two further systems (COSY and z7-13433) we gain tentative spectroscopic confirmation of \Lya~emission in the satellite galaxy. Finally, in the two systems that have only NIRSpec observations (CEERS-44 and CEERS-1027), the small arrays used do not cover the satellite galaxy and hence we are unable to spectroscopically confirm its presence. We therefore conclude that for all systems for which we have resolved spectroscopic observations of the satellite (50\%), we photometrically confirm the presence of the candidate companion galaxies. Moreover, we re-reduce and analyse the MOSFIRE spectra in \cite{Tang+23} and find clear evidence of multiple components to the \Lya~emission in two further galaxies (z7-30645 and z8-32350). Unfortunately, the lack of NIRCam imaging and the relatively poor sensitivity and resolution of \textit{HST} imply that we cannot, at this time, photometrically confirm the presence of companions associated with these two galaxies. Regardless of the lack of \textit{JWST} photometry, we clearly see further evidence supporting our findings that LAEs appear to be interacting with companion galaxies. Throughout the paper we provide further rationale for the conclusion that our sample are interacting with their companion. This includes the close spatial and velocity offsets of the components and the clear indications from the simulations that we exploit that similar systems are interacting and are ultimately merging.

In order to confirm that the presence of a companion is not a feature that is common for all $z = 7-9$ galaxies, we establish the companion rate for such galaxies. We use a sample of 30 galaxies with stellar masses from $7.4 < \rm{log(M}_{*} [$M$_{\odot}]) < 9.3$, matching those of our LAE sample. These galaxies are taken from \cite{Laporte+22, Tang+23,Harikane+22,Bouwens+22} and we find that 47\% of these galaxies have photometric-candidate companions within 5'' of the central galaxy. Many of these galaxies have not been spectroscopically followed-up and as such, we do not have constraints on their \Lya\ emission. Therefore, a fraction of this sample may be LAEs. However, it is clear that the 100\% rate of companions for our sample of LAEs is not typical of a mass-matched sample of galaxies that are not selected for \Lya\ emission. 

To reinforce the observational evidence supporting the idea that ongoing interactions aid to a larger detectability of LAEs during the epoch of reionisation, we explore comparable galaxy mergers using the novel \simname~simulation suite (Martin-Alvarez et al., in prep.). This simulation suite was performed before we obtained the NIRCam data, but remarkably, we find many simulated galaxies that match the photometry and the spatial geometry of our objects. The \simname~suite features a cosmological, high-resolution, zoom-in simulation of all progenitor galaxies that will form a massive spiral galaxy at $z = 1$ ($M_\text{halo} \sim 2 \cdot 10^{12}\,M_\odot$) and follows the same set of physical processes presented by the pathfinder Pandora project \cite{Martin-Alvarez+2023}, which focused on a single dwarf galaxy. Namely, \simname~incorporates a magneto-thermo-turbulent star formation \cite{Kimm2017, Martin-Alvarez2020} prescription, mechanical supernova (SN) feedback \cite{Kimm2014} and astrophysically-seeded magnetic fields through magnetised SN feedback \cite{Martin-Alvarez2021}. In the aftermath of SN explosions, 10\% of the available SN energy is in the form of energetic cosmic ray particles, which are allowed to anisotropically diffuse using an implicit scheme \cite{Dubois2016}. Most importantly for this work, \simname~also features on-the-fly radiative transfer \cite{Rosdahl2015}, employing a configuration similar to that of the {\sc SPHINX} simulations \cite{Rosdahl2018, Rosdahl2022}, capable of self-consistently reproducing reionisation while fully modelling the ISM of galaxies (with a maximum spatial resolution of $10$~pc), and crucially resolving the propagation and escape of ionising radiation \cite{Kimm2014, Garel2021}. To calculate the \Lya\ emission of each galaxy, we post-process our simulations with the publicly available RASCAS code \cite{Michel-Dansac20}. RASCAS accounts for both the production and resonant scattering of \Lya\ photons, and features a ``peeling'' algorithm for generating mock \Lya\ observations. We model the dust distribution according to the dust model described in \cite{Laursen09b}, with the dust either scattering or absorbing \Lya\ photons.

In our high resolution patch of the Universe ($\lesssim 8$~Mpc on a side), centred on the main progenitor of our $z \sim 1$ galaxy, the first ionised bubbles emerge at $z \sim 20$. By $z \sim 15$ several ionised bubbles exist that are a few (physical) kpc in radius, and rapidly begin merging as redshift approaches $z \sim 10$. This drastically increases the mean free path of ionising photons, with the coalesced bubble reaching scales of $100$~kpc. The resulting main bubble continues to grow, reaching a radius $\sim 0.5$~Mpc at $z \sim 7.3$. Due to their high spatial resolution, our zoom-in simulation is well-suited to study the propagation of radiation on small ISM scales. Conversely, as the simulation does not consider star formation outside the zoom region, the radii of the simulated bubbles represents only a lower limit for their true sizes, particularly when accounting for large-scale clustering \cite{Weinberger2019}.

For intrinsically bright LAEs, as studied here, a $\lesssim 0.5$~Mpc local ionised bubbles may be sufficient for these sources to be detectable \cite{Tang+23}. We further note that due to ongoing mergers our simulated galaxies have considerable relative peculiar velocities of up to $230$~km~s$^{-1}$. Sufficiently large velocities relative to the intervening neutral IGM gas may reduce damping wing suppression for particular lines of sight, hence requiring smaller local ionised bubbles to facilitate the detection of \Lya. Nevertheless, knowing from our observations that the LAEs sit in sufficiently large ionised bubbles, here we primarily focus on the probability of \Lya\ photons escaping the host galaxy and its intermediate surroundings. We find that a large amount of gas remains neutral within the local ionised bubble, actively feeding the galaxies along the cosmic web. Due to the moderate overdensity of our simulated volume, the main progenitor undergoes repeated mergers with multiple other galaxies brought in by the cosmic filaments, often involving several companions or mergers in rapid succession. 

To highlight one such merger occurring at $z \sim 7.3$, the four rightmost panels of Figure~\ref{fig:ObsVsSim} show different projections of \simname. This particular interaction features three merging galaxies that will constitute the main progenitor of the spiral galaxy formed by $z \sim 1$. These are very good analogues to the EGSY-8p68 observations shown on the left, with the three simulated galaxies having stellar masses of $M_{*,1} = 2 \cdot 10^{8}\,M_\odot$, $M_{*,2} = 3 \cdot 10^{8}\,M_\odot$, and $M_{*,3} = 8 \cdot 10^{7}\,M_\odot$ at this redshift. The observations shown on the left clearly indicate the superior resolution and sensitivity of \textit{JWST} (top panel) over \textit{HST} (bottom panel). The tri-component nature of EGSY-8p68 is entirely unresolved in existing \textit{HST} observations, but clearly identifiable in the F150W NIRCam imaging. Panel (\textit{a}) shows the large-scale view (150 kpc across) of the three filaments encompassing the merging system, where large amounts of HI gas (blue) along the filaments are feeding the star formation in these systems, resulting in significant NIRCam F150W emission (yellow to white). This active star formation is driving ionised hydrogen bubbles (grey) away from the filaments and into the low density regions. Panel (\textit{b}) shows the simulated \textit{JWST}-like 150W observation (with dust extinction modelled as an absorption screen along the line-of-sight), where we select the line-of-sight to illustrate the resemblance with EGSY8p68. Note that the simulated merging system is displayed at a slightly earlier stage of the merger with respect to EGSY8p68, with our simulated galaxies approaching the likely physical separation of the observed galaxies at $z \sim 7$. Interestingly, the mock F150W emission reveals compact galactic cores reminiscent of these \textit{JWST} observations, with diffuse emission (blending with the background) emerging from the extended, low density stellar discs. Panel (\textit{c}) shows a synthetic colour-composite observation using the \textit{JWST} filters at the full resolution of our simulation. While the two satellite galaxies are more diffuse, with stars actively forming in their discs, the main galaxy has a much more compact core due to the preceding rapid growth through repeated mergers. Panel (\textit{d}) is a colour composite of various simulated properties relevant for \Lya\ detectability. The emission from LyC ionising photons (purple) and HeII ionising photons (orange) is distributed on a scale of a few (physical) kpc. Importantly, there is a clear separation of the stellar emission and the HI gas during the merger event, with the ionising radiation escaping from star forming regions. 

To provide a quantitative confirmation and in-depth physical understanding of this effect, we explore in Figure~\ref{fig:TimeEvolution} the time evolution of the three merging galaxies in detail. In the top row the colours encode the same quantities as in panel (\textit{d}) of Figure~\ref{fig:ObsVsSim}, while in the second row we show the intrinsic \Lya\ emission (blue) and resonantly scattered \Lya\ emission (orange), produced by post-processing our simulations with RASCAS. The first column shows close up views of our simulated galaxies 1 and 2 at $z \sim 8.1$ undergoing minor mergers, with a complex topology of neutral gas and channels through which \Lya\ photons are escaping. The second column shows a larger scale view of these systems at $z \sim 8.1$. The third column shows the ongoing merger of our galaxies 1, 2 and 3 at $z \sim 7.3$ (with a different orientation with respect to Figure~\ref{fig:ObsVsSim}), where bursty star formation drives ionised channels through the ISM, with a significant amount of gas also tidally stripped from the galaxies. The fourth column shows a post-merger view of the newly formed disc-dominated galaxy at $z \sim 6.1$. Here, the HI is confined to the merger remnant galaxy, and the radiation can only escape through the channels opened by collimated, cosmic ray-driven bi-conical outflows. Focusing now on the \Lya\ radiation, all three galaxies have significant intrinsic \Lya\ emission due to their ongoing star formation. However, at this particular viewing angle, due to absorption by dust, \Lya\ scattered emission is only present around merging galaxy 2 at $z \sim 8.1$ and around galaxies 1 and 3 at $z \sim 7.3$. At $z \sim 7.3$, the remaining emission, not absorbed by dust, is very diffuse on large scales and hence not observable. 

To understand what drives the observable \Lya\ emission, in the third row we show the SFR evolution for our three merging galaxies, together with the SFR history of the merger remnant at $z \sim 6.1$. SFRs of all three galaxies undergo repeated cycles of bursts, driven by numerous mergers (`major' mergers (filled dots) and `minor' mergers (empty dots)) and a high SFR `plateau' during the final merger episode from $z \sim 7$ to $6$. This is mirrored in the evolution of the intrinsic \Lya\ emission of each galaxy as well as the total emission integrated across our three simulated galaxies, as shown in the fourth row, where there is a clear correspondence between the SFR peaks and peaks of intrinsic \Lya. {\it Our simulations hence demonstrate that high SFR triggered by gas-rich mergers results in brighter \Lya\ emission, which enhances the probability of these systems being detected.} 

To constrain this more robustly, we select an observable \Lya\ flux threshold of $10^{-18} \rm{erg}\,\rm{s}^{-1}\,\rm{cm}^{-2}$ consistent with our observations, and compute the ratio of scattered to intrinsic \Lya\ emission over the spatial extent of our mock observations of the galaxies. These have larger image sizes when the systems are further separated prior to the merger (up to 30 kpc at $z \sim 8.1$), and 10 kpc per side afterwards. This is shown in the fifth row, where the dark green curve denotes the median of the ratio 
of scattered to intrinsic \Lya\ flux and the shaded bands (quartiles and minimum-maximum values) show its distribution computed over 12 different lines-of-sight. As the fraction of escaping \Lya\ emission has a number of high peaks we conclude that the \Lya\ emission should be detectable for a number of the redshifts studied here. Interestingly, during close galactic encounters (as shown by the distance trajectories of three main galaxies shown in the same panel by coloured lines) and mergers, {\it significant gas tidal stripping and bursty star formation feedback leads to the opening of more low-column density sight-lines, hence facilitating the escape of \Lya\ radiation}. This effect can be clearly seen when comparing the most important merger instances to the peaks (high-tails) in the distribution of \Lya\ escape fraction. While our results are sensitive to the details of dust modelling (see supplementary material for a detailed discussion), our findings regarding higher intrinsic \Lya\ emission and enhanced \Lya\ escape fraction in mergers should be robust to the assumed dust model. 

The rightmost panels of Figure~\ref{fig:TimeEvolution} show various physical relations for the observations (black points) and the simulations (pink `violin' symbols, denoting the distribution of \Lya\ escape fraction over 12 lines-of-sight), using the same flux filter as above, demonstrating a very good correspondence between our simulations and our observed LAEs. Interestingly, both for observations and simulations we do not find any clear correlations between the \Lya\ escape fraction and any other galaxy properties (distance, SFR, stellar mass), further corroborating our finding that `favourable' lines-of-sights with evacuated neutral hydrogen channels lead to directional \Lya\ photon leakage. 

In this paper, we introduce a new interpretation to explain the unexpected detection of \Lya\ in $z\geq 7$ galaxies in an epoch where the IGM is mostly neutral. Combining the high-resolution and high-sensitivity of \textit{JWST} data with state-of-the-art radiative transfer on-the-fly magnetohydrodynamics simulations, we demonstrate that {\it three ingredients} are key to making \Lya\ emission detectable from our sample of galaxies deep in the epoch of reionisation: \textit{galactic mergers} driving high intrinsic \Lya\ emission in the host galaxy; \textit{a `favourable' line-of-sight} cleared of neutral hydrogen in the host galaxy by tidal interactions with companions and by star-formation feedback; \textit{an overdensity} of galaxies in order to drive a sufficiently large ionised bubble facilitating the escape of \Lya\ emission through the IGM. 


\begin{figure}
    \includegraphics[width=1.\columnwidth]{Images/LAE_stamps.png}\\
    \caption{Cutouts of \textit{JWST} images (F182M for GSDY and z7-GSD-3811, F200W for EGSY-8p68, F115W for CEERS-1027 and COSY and F150W for the remaining systems) showing the multiple components of each system. All images are smoothed with a Gaussian of FWHM $\sim 0.7$ kpc to enhance the visibility of the companion, except for EGSY-8p68 and z7-20237 where the two components become unresolved if we smooth the images. We note that the companion is always visible without smoothing, we smooth merely for aesthetic purposes. The position of the main target (*A in the text) is displayed by an orange arrow, the position of the first satellite (*B) is indicated by a blue arrow and the third satellite, if any, by a green arrow. The name of each candidate is indicated in the bottom right corner, and in the bottom left of each panel is a black scale showing 0.5''.}
    \label{fig:stamps}
\end{figure}

\begin{figure}
    \centering
    \includegraphics[width=\columnwidth]{Images/Fig1_simulation_v3.png}\\
    \caption{{\bf (Left panels)} NIRCam F150W (above) and \textit{HST} F160W (below) imaging of the LAE EGSY-8p68. The NIRCam imaging reveals three components to the system that were previously unresolved by \textit{HST}. {\bf (Right panels)} Analogue galaxy merger from the \simname~simulation: {\it a)} large-scale view of the filaments encompassing the galaxy merger (blue: HI density, grey to black: HII density, yellow to white: F150W intensity); {\it b)} simulated NIRCam F150W observation of the \simname~merger; {\it c)} fully resolved simulated observation in the NIRCam filters; and {\it d)} hydrogen gas and ionising radiation properties (blue: HI density, grey and black: HII density, yellow to white: F150W intensity, purple: LyC radiation energy density, orange: HeII ionising radiation energy density). The appearance of the observed system is very well reproduced by an ongoing merger of three galaxies, with an extended escaping LyC radiation field.}
    \label{fig:ObsVsSim}
\end{figure}


\begin{figure}
    \centering
    \includegraphics[width=0.95\columnwidth]{Images/Fig2_time_evolution.png}\\
    \includegraphics[width=1.0\columnwidth]{Images/Fig2_prop_evolution.pdf}\\
    \caption{{\bf (Rows 1-2)} The time evolution of our simulated merger system at $z = 8.1$ (close up and large scale view), $7.3$ and $6.1$. First row: maps of HI density (blue), HII density (grey to black), NIRCam F150W intensity (yellow to white), LyC radiation energy density (purple) and HeII ionising radiation energy density (orange). Second row: the RASCAS intrinsic \Lya\ flux (blue) and scattered \Lya\ flux (orange). {\bf (Rows 3-5)} The SFR for each of the three galaxies and the final system (black line). Dot symbols show mergers with companions (filled markers) and with other secondary systems (open markers); the intrinsic \Lya\ luminosity of the three galaxies and the entire system; the fraction of escaped \Lya\ luminosity filtered for pixels with fluxes higher than a given limit (median (dark green curve), quartiles (darker band), and min-max of the distribution across 12 lines-of-sight (lighter band)). Distance between the main progenitor and its companions is shown as well. The rightmost panels show key physical relations for the observations (black points) and the simulations (pink symbols).}
    \label{fig:TimeEvolution}
\end{figure}

\newpage 

\begin{landscape}
\begin{table}
    \centering
    \begin{tabular}{ l | ccccccc }
    \hline
    \hline
ID 	&	z	&	log($\rm{M_{*}[M_{\odot}]}$)	&   SFR $\rm{[M_{\odot}/yr]}$	& Separation [kpc] & L$_{Ly-\alpha}$ [10$^{43}$ erg/s] & \fesc & Reference \\ 
\hline
COSY-A & \textit{7.142} & 8.77 $^{+ 0.02 }_{- 0.02 }$ & 5.91 $^{+ 0.27 }_{- 0.25 }$ &$-$ & 1.37 & $< 0.1$ &\cite{RB+16},\cite{Witten+23} \\
COSY-B & \textit{7.142} & 7.59 $^{+ 0.13 }_{- 0.11 }$ & 0.39 $^{+ 0.14 }_{- 0.08 }$ &2.8 & $-$ & $-$ \\ %7.26 $^{+ 0.14 }_{- 0.12 }$ 
z7-GSD-3811-A & \textit{7.661} & 8.96 $^{+ 0.12 }_{- 0.13 }$ & 9.25 $^{+ 2.79 }_{- 2.39 }$ &$-$ & 0.386 &$ 0.22 \pm 0.08$ &  \cite{Song+2016}, this work \\
z7-GSD-3811-B & \textit{7.658} & 8.77 $^{+ 0.21 }_{- 0.21 }$ & 5.88 $^{+ 3.73 }_{- 2.28 }$ & 1.0 & $-$ & $-$ \\
EGSY-8p68-A & \textit{8.683} & 8.88 $^{+ 0.10 }_{- 0.18 }$ & 7.62 $^{+ 1.87 }_{- 2.54 }$ & $-$& 1.59 & $< 0.1$ & \cite{Zitrin+2015},\cite{Witten+23}\\
EGSY-8p68-B & 8.40$^{+0.37}_{-0.27}$ & 8.34 $^{+ 0.28 }_{- 0.36 }$ & 2.22 $^{+ 1.99 }_{- 1.25 }$ & 0.5 & $-$ & $-$ & \\
EGSY-8p68-C & 8.74$^{+ 0.50 }_{- 0.46 }$ & 8.63 $^{+ 0.26 }_{- 0.40 }$ & 4.26 $^{+ 3.49 }_{- 2.55 }$ & 0.6 &$-$ & $-$ &\\
CEERS-44-A & \textit{7.10} & 7.59 $^{+ 0.06 }_{- 0.06 }$ & 0.39 $^{+ 0.06 }_{- 0.05 }$ & $-$ & 0.348 &  $0.339 \pm 0.044$ & \cite{Tang+23}, \cite{Tang+23}\\
CEERS-44-B & 6.40 $^{+ 0.25 }_{- 0.27 }$ & 7.95 $^{+ 0.16 }_{- 0.18 }$ & 0.57 $^{+ 0.10 }_{- 0.08 }$ & 5.8 &$-$ & $-$ & \\
CEERS-1027-A & \textit{7.819} & 7.94 $^{+ 0.05 }_{- 0.04 }$ & 0.87 $^{+ 0.10 }_{- 0.08 }$  & $-$ &0.432& $0.085 \pm 0.018$ & \cite{Tang+23}, \cite{Tang+23} \\
CEERS-1027-B & 8.07 $^{+ 0.12 }_{- 0.14 }$ & 7.90 $^{+ 0.28 }_{- 0.21 }$ & 0.80 $^{+ 0.74 }_{- 0.30 }$ & 8.6 &$-$& $-$ \\
z7-13433-A & \textit{7.482} & 9.62 $^{+ 0.01 }_{- 0.01 }$ & 42.19 $^{+ 1.09 }_{- 1.00 }$  & $-$ & 1.00 & $-$ & \cite{Jung+2022} \\
z7-13433-B & \textit{7.478} & 8.67 $^{+ 0.03 }_{- 0.04 }$ & 4.65 $^{+ 0.33 }_{- 0.36 }$ & 2.6 &$-$& $-$ \\
z7-20237-A & \textit{7.6228} & 9.25 $^{+ 0.12 }_{- 0.15 }$ & 8.86 $^{+ 2.04 }_{- 1.63 }$ & $-$ & 0.317 & $-$ & \cite{Jung+2022} \\
z7-20237-B & 7.75 $^{+ 0.16 }_{- 0.59 }$ & 8.71 $^{+ 0.18 }_{- 0.25 }$ & 2.51 $^{+ 0.88 }_{- 0.64 }$  & 0.6 & $-$& $-$  \\
GSDY-A & \textit{7.957} & 9.42 $^{+ 0.21 }_{- 0.27 }$ & 17.80 $^{+ 4.56 }_{- 5.02 }$& $-$ & $0.19$ &  $>0.11$ & \cite{RB+22}, this work \\
GSDY-B & \textit{7.958} &8.73 $^{+ 0.21 }_{- 0.28 }$ &  2.73 $^{+ 1.02 }_{- 0.73 }$ & 2.2 &$-$& $-$&  \\
\hline
JADES-GS-z7-LA-A & \textit{7.278} &   7.15 $^{+ 0.13 }_{- 0.12 }$ & * & $-$ & 0.15 & $0.96 \pm 0.22$ &\cite{Saxena+23}\\
JADES-GS-z7-LA-B & 7.278 &   7.47 $^{+ 0.19 }_{- 0.28}$  & * & 1.2 &$-$&$-$&\\
GN-z11-A & \textit{10.603} & 9.1 $^{+ 0.3 }_{- 0.4 }$ & 21 $^{+ 22 }_{- 10 }$ &$-$&0.324& 0.03 $^{+ 0.05 }_{- 0.02 }$ & \cite{Bunker+23} \\
GN-z11-B & 10.603 & * & * & 2.5/13.1 & $-$ & $-$ \\

 \hline
    \end{tabular}
    \caption{Physical properties of all the systems studied in this paper. The columns are : (1) spectroscopic (italic) or photometric redshift of the candidate, (2) stellar mass, (3) Star Formation Rate averaged over at most the last 100 Myr, (4) separation between the main component (A) and each companion, (5) luminosity of observed \Lya\ emission (6) escape fraction of Ly-$\alpha$ photons and (7) reference for the Ly-$\alpha$ emission and the Ly-$\alpha$ escape fraction measurement. The table is split into two sections, our newly identified merging systems and those already identified in the literature. Both of these literature examples use data that is not public and hence we take the galaxy properties reported in the literature, * indicates where these properties have not been reported in the literature. Two candidate companions to GN-z11 have been discussed in the literature \cite{Tacchella+23}, but no galaxy properties have been reported for either.}
    \label{tab:Galaxy_Properties}
\end{table}
\end{landscape}
\newpage

\bmhead{Acknowledgments}

The authors would like to acknowledge the work of the PRIMER core team in obtaining and reducing the NIRCam data for the COSMOS field used in this work (PI: Dunlop, ID: 1837). The authors would also like to acknowledge the work of the CEERS team in obtaining and reducing the NIRCam data of the EGS field used in this work (PI: Finkelstein, ID: 1345). The authors would like to acknowledge the FRESCO team's work in obtaining observations of the GOODS-S and GOODS-N fields (PI: Oesch, ID: 1895). The authors would also like to thank Joki Rosdahl for their support with the generation of radiative transfer RAMSES simulations, and Thibault Garel for their support with the usage of RASCAS.

Support for this work was provided by NASA through grant JWST-GO-01837 awarded by the Space Telescope Science Institute, which is operated by the Association of Universities for Research in Astronomy, Inc., under NASA contract NAS 5-26555. This research has made use of the Keck Observatory Archive (KOA), which is operated by the W. M. Keck Observatory and the NASA Exoplanet Science Institute (NExScI), under contract with the National Aeronautics and Space Administration. Based on observations collected at the European Southern Observatory. This work used the DiRAC@Durham facility managed by the Institute for Computational Cosmology on behalf of the STFC DiRAC HPC Facility (www.dirac.ac.uk). The equipment was funded by BEIS capital funding via STFC capital grants ST/P002293/1, ST/R002371/1 and ST/S002502/1, Durham University and STFC operations grant ST/R000832/1. DiRAC is part of the National e-Infrastructure. CW thanks the Science and Technology Facilities Council (STFC) for a PhD studentship, funded by UKRI grant 2602262. NL acknowledges support from the Kavli foundation. DS, MGH and JSD acknowledge STFC support. PS acknowledges INAF Mini Grant 2022 “The evolution of passive galaxies through cosmic time”. RSE acknowledges financial support from ERC Advanced Grant FP7/669253. PGP-G acknowledges support  from  Spanish  Ministerio  de  Ciencia e Innovaci\'on MCIN/AEI/10.13039/501100011033 through grant PGC2018-093499-B-I00. DP acknowledges support by the Huo Family Foundation through a P.C. Ho PhD Studentship. RM, WB and WM acknowledge support by the Science and Technology Facilities Council (STFC) and ERC Advanced Grant 695671 "QUENCH". RM also acknowledges funding from a research professorship from the Royal Society. This research was supported in part by the National Science Foundation under Grant No. NSF PHY-1748958.


\bmhead{Author Contributions} CW reduced and analysed NIRCam imaging and WFSS, MOSFIRE and X-shooter data. NL extracted the photometry of each candidate and performed the SED-fitting analysis. SMA developed the \textit{Azahar} simulations, performed their basic analysis and extracted the Lyman-$\alpha$ luminosity for all the merging systems. YY ran the RASCAS simulations and showed the influence of line of sights. DS an MGH coordinated the simulations part of this paper. CW, NL, SMA, DS, YY and MGH wrote the paper, and developed the main interpretation of the results. RM, GRB, RSE, WM, WB, DP, CS and HK were key in interpreting the observational results. JSD, RSE, NG, GI, AMK, DM, PGP-G and PS were involved in obtaining and reducing the data for the PRIMER program which was used in this study. All authors discussed the results and commented on the manuscript.

\newpage
%\backmatter

\bmhead{Methods}

\bmhead{Data} 

The emergence of \textit{JWST} with its significant technical advancements over both ground- and space-based telescopes, has the potential to reveal new hints as to the driving forces behind the observed \Lya~emission discussed in this paper. We therefore decided to target all spectroscopically-confirmed LAEs at $z > 7$, with \textit{JWST}/NIRCam imaging. This search reveals eleven such LAEs, however, further constraints on the SED of these galaxies from NIRCam imaging reveals the \Lya~emission line detected in one of these LAEs is in fact not \Lya~(discussed in more detail below). Therefore, the sample becomes ten LAEs (seen in Table~\ref{tab:Galaxy_Properties}), eight of which are studied in this paper as they have publicly available NIRCam imaging. These galaxies are found in the GOODS-North and GOODS-South, EGS and COSMOS fields which have been observed by FRESCO (PI: Oesch, ID: 1895), CEERS (PI: Finkelstein, ID: 1345) and PRIMER (PI: Dunlop, ID: 1837) respectively.

The PRIMER survey covers $\sim 150 \rm{arcmin}^{2}$ of the COSMOS field with the F090W, F115W, F150W, F200W, F277W, F356W, F410M and F444W filters all at a $5 \sigma$ limiting depth of $m_{\rm AB} > 28$. The CEERS survey took imaging of $100 \rm{arcmin}^{2}$ of the EGS field in the F115W, F150W, F200W, F277W, F356W, F410M and F444W filters, reaching a $5 \sigma$ limiting depth of $m_{\rm AB} > 29$ in the short-wavelength or $m_{\rm AB} > 28.4$ in the long-wavelength filters \cite{Finkelstein+22}. 

The FRESCO survey's primary aim is to obtain NIRCam Long-Wavelength (LW) Grism spectra over the F444W filter for galaxies across GOODS-S. However, the simultaneous imaging in the short-wavelength channel provides imaging in both F182M and F210M filters, and direct imaging also obtained in the F444W filter additionally provides us with a total of 3 filters for our galaxies in GOODS-S/GOODS-N. FRESCO's LW WFSS brings significant advantages as the F444W filter covers the wavelength range of $3.9-5 \mu$m, therefore observing both \OIII and \Hb at $z \sim 7-9$. These emission lines ordinarily contaminate imaging in the F444W filter and hence constraints on the fluxes of these emission lines allow us to reduce the uncertainty in galaxy properties such as stellar mass, star-formation rate and age. We use these constraints on the emission lines present in F444W to create a mock F430M filter by calculating the flux in F444W from the continuum and then recombining this with the measured emission line fluxes to produce the expected flux in the F430M filter. Furthermore, when we fit the SED (discussed further below) we check the fluxes of the \OIII and \Hb emission lines in the best-fit SED model and confirm that these are consistent with the measured fluxes from the WFSS spectrum. 

Throughout our analysis we use the final reduced data products created by the PRIMER team for the COSMOS field, the publicly available reduced data products from the CEERS team for the EGS field and our own reduction of the GOODS-S imaging. In order to produce the GOODS-S images, we take all available NIRCam imaging data of GOODS-S, available from the FRESCO survey, and reduce them using version 1.8.5 of the \textit{JWST} pipeline. We first download the uncalibrated files from the MAST archive and follow the steps of the \textit{JWST} pipeline: (Stage 1) detector-level corrections and ramp fitting, (Stage 2) instrument-level and observing-mode corrections producing fully calibrated exposures, (Stage 3) producing the final mosaic. We take additional care after Stage 1 to ensure the removal of horizontal and vertical striping that can be present in the rate files. 

We extract the photometry of our objects using \textsc{SEXtractor} \cite{SEXtractor} on psf-matched images extracting all objects with more than 4 pixels (DETECT\_MINAREA 4) detected at more than 1.5$\sigma$ (DETECT\_THRESH and ANALYSIS\_THRESH 1.5). We used large deblending parameters to allow efficient separation of each component (DEBLEND\_NTHRESH 64 and DEBLEND\_MINCOUNT 0.000001), however, the small separation between the components of the EGSY-8p68 and z7-20237 systems makes the extraction of each component's photometry impossible with \textsc{SEXtractor}. For these two systems, we model the shape of each component using \textsc{Galfit} \cite{Galfit} assuming a Sersic profile and NIRCam PSFs, simulated using \textsc{WebbPSF} \footnote{https://www.stsci.edu/jwst/science-planning/proposal-planning-toolbox/psf-simulation-tool}. We first run \textsc{Galfit} on one of the NIRCam images where all components are clearly resolved (F115W), and we then fix the resultant position, Sersic index, axis ratio and angle for each component when extracting the photometry for the remaining images. Figure~\ref{fig:galfit} shows the modelling of the components for these two systems -- EGSY-8p68 and z7-20237.

 \captionsetup[figure]{labelfont={bf},name={Extended Figure},labelsep=period}
 \setcounter{figure}{0}    
\begin{figure}
    \centering
    \includegraphics[width=1.0\columnwidth]{Images/galfit.pdf}%
    \caption{Example of \textsc{Galfit} modelling for the two systems (EGSY8p68 and z7\_20237) for which the separation between each component is too small for the photometry to be extracted by \textsc{SEXtractor}. }
    \label{fig:galfit}
\end{figure}





{\bf SED fitting} 
We initially take care to ensure that, for galaxies that only have \Lya~emission detected in their spectra, the emission line detected is indeed \Lya. There are initially five galaxies for which we have only one observed emission line in the literature (z7-GSD-3811, z7-13433, z7-20237, z8-32350 and GSDY). In order to confirm that this emission line is \Lya, we use the SED-fitting code \textsc{Bagpipes} \cite{Carnall2018} on the available NIRCam data to produce an improved photometric redshift of each LAE. One of the key diagnostics in this fitting becomes the bright \OIII emission lines, expected from high redshift objects, that fall into either the medium-band filter F410M/F430M or a non-detection in this filter but a boost in the F444W filter. This photometric-redshift fitting results in a redshift that is consistent with the emission lines observed in all of the galaxies in our sample. We note that we have not included one LAE, that has existing NIRCam imaging, in our sample -- z8-19326 \cite{Jung+2022}. The photometric redshift, constrained by a significant boost in F410M ($z \sim 7.1$), for z8-19326 becomes inconsistent with the emission line detected being \Lya~and therefore this galaxy is not included in our sample. All of the galaxies in our sample show a clear \OIII excess in NIRCam imaging except one, z7-20237. It is therefore not possible to be certain that the emission line detected is indeed \Lya, but we move forward assuming that this is the case.

Following this previous step, in order to confirm that all components of each system are at a similar redshift, we again employ \textsc{Bagpipes} to estimate their photometric redshifts. We then compare these with the spectroscopic redshift of the main component. This fitting confirms that all the companion galaxies preferentially have high photometric redshifts, many of which are consistent with the spectroscopic redshift of the massive galaxy. In combination with the prior of the companion being closely located to a confirmed high-redshift galaxy, these become strong-photometric candidate companion galaxies.

We then use \textsc{Bagpipes}, with the redshift fixed to that of the spectroscopic redshift of the system, using different SFHs (constant, delayed, burst and burst+constant)) to fit the SED of all massive and companion galaxies. The best fit model is then obtained by taking the SFH that leads to the smallest Bayesian Information Criterion (BIC) as described in \cite{Laporte21}. We then report the galaxy properties returned from this method in Table~\ref{tab:Galaxy_Properties}. 


\begin{figure}
    \centering
    \includegraphics[width=1.0\columnwidth]{Images/SEDs.pdf} \\
    \caption{Best SED-fits for all systems with strong Ly-$\alpha$. The yellow dots are the observed photometry, the grey line shows the best fit. The physical parameters of the best fit are also indicated (see text for details)}
    \label{fig:SED}
\end{figure}

\begin{figure}
    \centering
    \includegraphics[width=0.66\columnwidth]{Images/SEDs_2.pdf} \\
    \caption{Same as Fig.~\ref{fig:SED}}
    \label{fig:SED2}
\end{figure}


{\bf Spectroscopy} 

\begin{figure}
    \centering
    \includegraphics[width=1.0\columnwidth]{Images/FRESCO_Spectra.png}%
    \caption{NIRCam WFSS Grism spectra of z7-GSD-3811 (above) and GSDY (below). Within each panel, the continuum subtracted 2D spectrum (above) and 1D spectrum (below) of the central galaxy are shown. The upper left sub-panel shows the F444W image of the system where two components can be identified. The horizontal dashed lines indicate the position of the central galaxy in both the image and 2D spectrum. The three dashed lines in the 1D spectrum denote the expected positions of the \OIII doublet (central and right dashed lines) and \Hb (left dashed line) given the objects' spectroscopic redshift, while the grey shaded region denotes the one-sigma noise level.}
    \label{fig:FRESCO_spectra}
\end{figure}

\begin{figure}
    \centering
    \includegraphics[width=1.0\columnwidth]{Images/Lya_spectra.png}%
    \caption{The X-shooter (above) and MOSFIRE (below) Gaussian-smoothed 2D spectra of COSY and z7-13433, respectively. These cover the wavelength range of the previously spectroscopically confirmed \Lya\ emission of the system. White circular apertures indicate the proposed two components of the \Lya\ emission that are coincident with the expected positions of the main and companion galaxies in the slit. The positive trace of the emission is indicated by white pixels, the negative trace by black pixels, this effect of positive and negative traces is caused due to the ABBA nodding procedure employed by both instruments.}
    \label{fig:Lya_spectra}
\end{figure}


\begin{table}
    \centering
    \begin{tabular}{ | l | ccccc | }
    \hline
ID 	&	$z$	& $\rm{[OIII]_{5007}}$ & $\rm{[OIII]_{4959}}$ & \Hb & SFR$_{\rm{inst}} \rm{[M_{\odot}/yr]}$	\\  \hline
z7-GSD-3811-A & $7.663$ & $8.3 \pm 0.4$ & $2.7 \pm 0.4$ & $1.0 \pm 0.3$ & $16.5 \pm 4.9$\\
z7-GSD-3811-B & $7.658$ & $4.8 \pm 0.4$ & $1.5 \pm 0.3$ & $<0.5$ & $<8.2$\\
GSDY-A & $7.956$ & $5.8 \pm 0.7$ & $1.0 \pm 0.4$ & $<0.5$ & $<9.0$\\
GSDY-B & $7.957$ & $3.1 \pm 0.6$ & $1.5 \pm 0.6$ & $0.8 \pm 0.3$ & $14.4 \pm 5.4$\\

 \hline
    \end{tabular}
    \caption{The observed redshift and fluxes of emission lines (in units of $10^{-18}$ erg/s/cm$^2$) detected in the NIRCam/WFSS spectra. The instantaneous star-formation rate (SFR) is not dust-corrected given the lack of constraints on dust in these galaxies.}
    \label{tab:ELfluxes}
\end{table}

As discussed earlier, it is desirable to spectroscopically confirm as many satellite galaxies as possible. The use of the Mutli-Shutter Array with NIRSpec to observe two of the systems means that, given the small shutter size, the satellite is not coincident with the shutter position. Instead, we make use of all existing ground-based observations of our sample of LAEs, of which five of the sample have been observed by MOSFIRE (z7-13433 and z7-20237 - ID: N199, PI: Jung - , EGSY-8p68- ID: C228M, PI: Zitrin-, GSDY, z7-GSD-3811 - ID: C182M, PI: Scoville ) and one with X-shooter (COSY - ID: 097.A-0043, PI: Ellis). 

We reduce the spectroscopic data using the standard MOSFIRE data reduction pipeline\footnote{\url{https://keck-datareductionpipelines.github.io/MosfireDRP/}} and EsoReflex\footnote{\url{https://www.eso.org/sci/software/esoreflex/}}. These pipelines perform wavelength callibration, sky subtraction and flat fielding as well as further standard data reduction steps in order to produce two-dimensional (2D) spectra. The resultant 2D spectra span the length of the original slit used for the observations and have spectral and spatial resolutions of $R \sim 3380$, 0.1798 "/px and $R \sim 8900$, 0.158 "/px for MOSFIRE and X-shooter respectively. 

For four of these galaxies, the existing MOSFIRE observations fail to resolve the \Lya~emission from the companions given their close proximity to the central galaxies (z7-20237, EGSY-8p68, GSDY, z7-GSD-3811). However, for two of these systems, COSY and z7-13433, there is a spatial offset between the central and companion galaxy in the slit direction of $\sim 3$ pixels and a slight difference in redshift that creates an offset in the spectral direction of the 2D spectra allowing us to disentangle the \Lya~emission originating from the two components in the 2D spectra.

The two components of each object are shown in Figure~\ref{fig:Lya_spectra}, where we can identify both the positive (white) and negative (black) traces of the object, caused by the ABBA nodding pattern of the MOSFIRE and X-shooter observations. We find that, for z7-13433, two emission lines are offset by $\sim 3$ pixels, which is consistent with the $\sim 0.5$" offset between the two components in the slit. The \Lya~emission that is coincident with the position of the satellite is detected at $2.8 \sigma$ corresponding to $z = 7.479$, while the \Lya~flux associated with the main galaxy is at $4 \sigma$ corresponding to $z = 7.482$. Therefore, these two components are offset by $\sim 125$ km/s. In the case of COSY, we expect a similar spatial offset between the two components, but the $\sim 0.7$" seeing during the observations makes distinguishing between the two components very challenging. However, there is clearly a second, previously unidentified component to the \Lya~emission of COSY at $\sim 0.9898 \mu$m. The systemic redshift of COSY is now very well constrained at $z = 7.1416$ thanks to the observation of several emission lines (NV and HeII \cite{Laporte+17} and [CII] \cite{Pentericci+16}). The main component of \Lya~emission is offset from the systemic redshift by $ 286.8$ km/s \cite{Laporte+17}. A large offset of the \Lya~profile from line-centre is expected for COSY, thanks to its low \fesc ($<10\%$ \cite{Witten+23}), however, the detection of \Lya~emission from COSY at $\sim 0.9898 \mu$m (an offset from line centre of just 8 km/s) would be inexplicable. This emission would fall directly on line-centre for COSY and an observation of \Lya~with both a 8.2 km/s and 286.8 km/s offset in the high-redshift Universe is so unlikely we consider this possibility to be negligible. Instead a far more probable solution is that the additional emission line detected in the X-shooter observations is in fact associated with the companion galaxy, given it is within the constraints we have on the spatial offset of the satellite and said satellite is constrained to being at a similar photometric redshift to COSY. As such, we attribute this $2.7 \sigma$ emission line to COSY-B. While these \Lya~detections are clearly tentative, given they are coincident with the expected position in the 2D spectra of strong-photometric candidates and given their proximity to confirmed LAEs this boosts their likelihood of being true detections. 

Two of the systems that cannot be resolved in their MOSFIRE 2D spectra are located in the GOODS-S field (z7-GSD-3811 and GSDY). This field was the target of WFSS observations by the NIRCam Long-wavelength (LW) Grism as part of the FRESCO observing program. FRESCO's LW WFSS brings significant advantages when observing $z \sim 7-9$ galaxies as the F444W filter used in this program covers the wavelength range of $3.9-5 \mu$m, therefore observing both \OIII and \Hb, which are expected to be bright in high-redshift galaxies. These lines have previously been observed in $z > 7$ galaxies using the publicly available FRESCO data \cite{Laporte+23}. Moreover, the significantly superior spatial resolution of NIRCam LW WFSS (60 milli-arcesconds) allows us to resolve emission lines where MOSFIRE and X-shooter would fail to do so. 

We reduced the Grism data following the steps described in \cite{Sun+22} including flat fielding, background subtraction, 1/f noise subtraction and WCS assignment. We also perform a careful astrometric analysis to avoid any offsets between the LW direct imaging in F444W and the LW channel Grism spectra in F444W. This allows us to extract the 2D Grism spectra associated with the astrometric position of any source, these spectra are then stacked and the 1D spectrum at the position of the source with an aperture of 3 pixels is extracted. 

The use of only one Grism direction (Grism R) in the observations means we cannot disentangle the target's continuum emission from the accumulation of contaminant object's continua emission along the direction of the Grism. However, observing emission lines when the redshift of the source is already confirmed, either from photometry or in this case from previously observed \Lya~emission, is considerably easier. As such, we perform a continuum subtraction from the 1D spectrum to remove any observed continuum and we focus on measuring \OIII and \Hb emission line fluxes. 

The final 2D WFSS spectra, shown in Figure~\ref{fig:FRESCO_spectra} show emission lines for both z7-GSD-3811 and GSDY. For z7-GSD-3811 both components of the \OIII emission are present and there is a clear, faint detection of H$\rm{\beta}$. Morevover, \OIII emission is also detected in the satellite galaxy. This emission is offset from the central galaxy by $170 \pm 70$ km/s. The 2D spectrum of GSDY shows a clear detection of [OIII]$_{5007}$ in both the central and satellite galaxies, and there is a very faint indication of [OIII]$_{4959}$ and \Hb in the satellite galaxy, but at a very low confidence. The emission detected in the central and satellite galaxy are offset by just $\sim 26$ km/s, but given the low spectral resolution of the NIRCam WFSS, this is significantly smaller than the $\sim 70$ km/s uncertainty on this measurement. 

We discuss in the main text the use of \Hb to estimate the intrinsic \Lya\ flux and hence the escape fraction of \Lya\ photons, however, it can also be used as a probe of the instantaneous SFR. We assume a H$\alpha$ to \Hb ratio of 2.85 and Case B recombination at $T_{\rm e} = 10,000$ K following \cite{Kennicutt+98} we estimate the instantaneous SFR, reported in Table~\ref{tab:ELfluxes}. Given the lack of constraints on the redenning of these galaxies, we do not dust-correct the \Hb flux and as such these measured instantaneous SFRs are lower bounds. 

The observations of multiple components to each emission line are taken as spectroscopic confirmation of the satellite galaxies. Moreover, the velocity offset observed in all of these galaxies is indicative of either a line-of-sight separation, or more likely, a difference in the line-of-sight velocities of the two objects as they interact with each other. This therefore is considered as further evidence that the systems are indeed interacting, as opposed to being two components of the same stable system. 

We include within Table~\ref{tab:Galaxy_Properties} two LAEs with companion galaxies identified within the literature (GNz11 \cite{Tacchella+23} and JADES-GS-z7-LA \cite{Saxena+23}). Both of these galaxies were identified in JADES data of GOODS-S and GOODS-N. Unfortunately this data is not yet publicly available, and the FRESCO data covering these galaxies is not deep enough to place strong photometric constraints on galaxy properties. Galaxy properties for the central and companion galaxies are reported for JADES-GS-z7-LA and as such we include the estimate of stellar mass in Table~\ref{tab:Galaxy_Properties}. For GNz11, while two candidate companion galaxies exist -- the "haze" and a potential, faint companion galaxy $\sim 3$" away \cite{Tacchella+23} -- the galaxy properties are not reported and as such we cannot give them in Table~\ref{tab:Galaxy_Properties}. 

We additionally note the recently reported $z = 9.3$ merging system \cite{Boyett+23} with an absence of detected \Lya\ emission. The reported specific-star-formation rate of this galaxy (sSFR $\sim -8$) is consistent with the high sSFR seen in our sample of merging LAEs, and therefore, one may posit that this is a contradiction to our results. We argue that non-detection of \Lya\ emission from a merging system is entirely reasonable -- while we see significant boosts in the SFR and hence intrinsic \Lya\ emission from merging systems, the escape of \Lya\ emission is only possible when the escape fraction of \Lya\ emission out of both the host galaxy and the surrounding IGM is non-zero. We explore the effect of viewing angle on the observed \Lya\ emission escaping our simulated merging systems and find that while the majority of viewing angles facilitate the escape of \Lya, several viewing angles exist for which the only \Lya\ emission escaping the galaxy is diffuse and is therefore unobservable. Moreover, without the presence of a large ionised bubble, the neutral IGM will not facilitate the escape and observation of \Lya\ emission. We postulate therefore that the non-detection of \Lya\ in this merging system is due to either the chance orientation in which neutral hydrogen is located in the line-of-sight causing an escape fraction that is close to zero or the lack of a large ionised bubble surrounding this system, or both.


{\bf Azahar simulations} 

\begin{figure}
    \includegraphics[width=0.95\columnwidth]{Images/eff_dust.pdf}\\
    \caption{Dust opacities superimposed on synthetic \Lya\ images at $z = 7.3$. The top row shows the intrinsic \Lya\ emission, the middle row shows the scattered \Lya\ emission generated with a default dust model, and the bottom row shows the scattered \Lya\ emission  with practically no dust  (dust opacity reduced by a factor of $10^6$). The three different columns show the galaxy from three different directions. Red and orange iso-contours correspond to a dust optical depth of 0.2 at the \Lya\ wavelength and H$\alpha$ wavelength, correspondingly. Note that the dust does not act as a screen for the \Lya\ emission because of the resonant nature of \Lya\ scattering.}
    \label{fig:eff_dust}
\end{figure}

The new \simname~simulations are a suite of multiple models spanning various combinations of canonical hydrodynamics, magnetic fields, radiative transfer, and cosmic rays physics with the aim of understanding their effects on galaxy formation as well as their complex interplay. The simulations are generated using the magneto-hydrodynamical code {\sc ramses} \cite{Teyssier2002}, which employs a constrained transport method for divergence-less evolution of the magnetic field \cite{Teyssier2006}. The main target of study for \simname~is a massive spiral galaxy with $M_\text{halo}\,(z = 1) \sim 2 \cdot 10^{12}\,M_\odot$ in a relatively large zoom region, approximately $8$~cMpc across along its largest axis. \simname~follows the set of physics presented by the pathfinder Pandora \cite{Martin-Alvarez+2023}. We provide here a brief summary, and refer the reader to that reference and its presenting publication (Martin-Alvarez et al., in prep.) for further details. In this work, we investigate one of the most complete simulations in the \simname~suite, the RTnsCRiMHD model. This model features a magneto-thermo-turbulent prescription for star formation \cite{Kimm2017, Martin-Alvarez2020}, mechanical supernova (SN) feedback \cite{Kimm2014} and astrophysically-seeded magnetic fields through magnetised SN feedback \cite{Martin-Alvarez2021}. This feedback injects 1\% of the SN energy as magnetic energy, and intends to reproduce the approximate magnetisation of SN remnants. This particular simulation of \simname~also includes cosmic rays modelled as an energy density allowed to anisotropically diffuse, evolved with an implicit solver \cite{Dubois2016, Dubois2019}. In \simname, cosmic rays are exclusively sourced by SN feedback, with each event injecting 10\% of their total energy as cosmic rays. We assume a constant diffusion coefficient for cosmic rays of $\kappa_\text{CR} = 3 \cdot 10^{28}\,\text{cm}^2\,\text{s}^{-1}$, and do not account for cosmic ray streaming in this model. This model also includes on-the-fly radiative transfer \cite{Rosdahl+15}, with a configuration similar to that of the {\sc SPHINX} simulations \cite{Rosdahl+18}, featuring three energy bins for radiation spanning the ionisation energy intervals from $13.6$~eV to $24.59$~eV (HI ionisation), $24.59$~eV to $54.42$~eV (HeI ionisation), and above $54.42$~eV (HeII ionisation). \simname~has a maximum spatial resolution in cells with a full cell-width of $\Delta x \sim 20\,\text{pc}$ (or equivalently, an approximate cell half-size or radius of $10\,\text{pc}$), and refinement is triggered whenever its size is larger than a quarter of its Jeans length, or it contains a total mass larger than $8\, m_\text{DM}$, where $m_\text{DM} \sim 4.5 \cdot 10^{5} M_\odot$ is the mass of dark matter particles in the zoom region. The stellar mass particle is $m_* \sim 4 \cdot 10^{4} M_\odot$. \simname~models cooling above and below $10^4 \text{K}$ \cite{Ferland1998, Rosen1995}. Finally, recent work has shown that when expanding simulation models to account for cosmic ray feedback physics at comparable resolutions, the escape fractions of LyC photons from galaxies is lower than in their absence \cite{Farcy2022}.

Therefore, \simname~features a realistic (yet computationally taxing) ISM model that considerably approaches the resolution required to converge in the propagation and escape of ionising photons from galaxies \cite{Kimm2014}. We note that, while close, our resolution still falls short from that regime and that even more sophisticated ISM models \cite{Katz2022} may be important for a complete understanding of photon propagation in the ISM. Opportunely, a considerable part of our results relies on gas being ejected from galaxies during merger events, which implies that our estimate of LAE detectability during these events will be more resilient to these caveats.

In this manuscript, we refer to the RTnsCRiMHD simulation simply as \simname.

{\bf Galaxy selection and measurements in the simulations}

In order to select galaxies in our simulation, we employ the {\sc halomaker} software \cite{Tweed2009} to detect and characterise dark matter halos. We identify the three progenitor galaxies of interest (as well as the galaxy merging with the main progenitor at $z \sim 8.1$) and follow them through time by tracking their innermost stellar particles. Their centres are determined using a shrinking spheres algorithm applied to their stellar component \cite{Power2003}, and they are assigned their corresponding halo as obtained by {\sc halomaker}. In order to select the system for study, we broadly reviewed the evolution of all galaxies with stellar masses $M_* > 10^{7} M_\odot$ in the redshift interval $z \in [9.0, 6.0]$. The most promising candidate, the main progenitor of the main \simname~galaxy was selected for further investigation. 

In order to assign measurements to individual galaxies, we measure values within their galactic region, defined by the radius $r_\text{gal} < 0.2\; r_\text{rvir}$. During the merging stages, we employ $r = \text{min}[r_\text{gal}, 0.45 D_{ij}]$, where $D_{ij}$ is the distance between the $i$ and $j$ progenitors, down to a minimum distance of $1.5$~kpc. 

{\bf RASCAS post-processing}

We post-process the simulation with the publicly available, massively parallel code RASCAS \cite{Michel-Dansac20}, for modelling the \Lya\ emission and resonant scattering in our simulations. RASCAS accounts for two sources for the \Lya\ emission, recombination and collisional excitation. For recombination, RASCAS adopts the case B recombination coefficient from \cite{Hui&Gnedin97}, and the fraction of recombinations producing \Lya\ photons from \cite{Cantalupo08}. For collisional excitation, RASCAS uses the fitting function for the collisional excitation rate from level 1s to 2p from \cite{Goerdt10}. We cast $N_{\rm MC} = 10^6$ Monte Carlo photon packets to sample the real \Lya\ photon distributions, from recombination and collisional excitation, respectively \footnote{We sample $N_{\rm MC} = 10^7$ photon packets for the images along the LOS in \autoref{fig:TimeEvolution}.}.  

After each scattering event, the \Lya\ photon changes its frequency and direction according to a phase function as implemented in RASCAS. RASCAS adopts the phase function in \cite{Hamilton40, Dijkstra&Loeb08} for the scattering of \Lya\ photons around line centre and Rayleigh scattering for \Lya\ photons in the line wing. At high H~\textsc{i} column density, \Lya\ photons will scatter many times locally until they shift in frequency large enough so that they can have a long mean free path. To reduce the associated computation cost, RASCAS adopts a core-skipping mechanism \cite{Smith15} to transit the photons to line wings while avoiding local scattering in space. RASCAS also implements the recoil effect and the transition due to deuterium with an abundance of D/H = $3 \times 10^{-5}$. The dust distribution according to the dust model described in \cite{Laursen09b}. Dust can either scatter or absorb \Lya\ photons. The probability of scattering is given by the dust albedo $a_{\rm dust} = 0.32$ following \cite{Li&Draine01}. The dust scattering with the Henyey-Greenstein phase function \cite{Henyey&Greenstein41} and the asymmetry parameter is set to $g=0.73$ following \cite{Li&Draine01}. We generate synthetic images and spectra with a peeling algorithm described in \cite{Yusef-Zadeh84, Wood&Reynold99, Costa22}, along 12 LOSs uniformly sampled using healpix algorithm \cite{Gorski05}.

We last note that the dust distribution has a large effect on the \Lya\ radiative transfer as well as the \OIII and \Hb emission that is used to infer the intrinsic \Lya\ emission from observations. In Figure~\ref{fig:eff_dust}, we show how reducing the amount of dust changes the observed \Lya\ images. We see the dust is able to completely mask the \Lya\ emission of some individual galaxies as resonant scattering of H~\textsc{i} increases the probability of \Lya\ photons encountering dust grains. The dust can suppress the \Ha emission by up to 20\% around the center of the three merging galaxies, if we assume intrinsic \Lya\ can be converted directly to intrinsic \Ha emission.

\begin{appendices}


\end{appendices}


\bibliography{sn-bibliography}% common bib file
%% if required, the content of .bbl file can be included here once bbl is generated
%%\input sn-article.bbl

%% Default %%
%%\input sn-sample-bib.tex%

\end{document}
