


\section{Introduction}

While 5G has entered the commercialization phase, the research community has started the exploration of future 6G networks. Compared with previous generations, 6G networks are expected to present more stringent performance requirements, i.e., terabits per second (Tbps) data rates for virtual reality, and more than $10^7/km^2$ connection densities with significantly lower latencies than 5G networks \cite{zhang20196g}. One of the main obstacles to wireless network evolution is the uncontrollable radio environment with reflections, diffractions, and scattering.   
Recently, reconfigurable intelligent surfaces (RISs) have emerged as promising techniques to enhance wireless signal propagation \cite{ertu}.   
In particular, the core feature of RISs is to manipulate the signal propagation path by intelligently configuring numerous small elements. Each RIS element can independently tune the phase of the incident signal, creating a smart radio environment\cite{emil}. RISs not only are technically attractive but also require low energy consumption and hardware cost, making it promising to enhance the spectral efficiency for real-world deployments. 
Given these advantages, RISs can be combined with other emerging techniques, including multiple-input multiple-output (MIMO),
millimeter-wave (mmWave) communications, unmanned aerial vehicle (UAV) networks, non-orthogonal multiple access (NOMA), vehicle-to-everything (V2X) networks, and so on\cite{mu2021simultaneously,mu2021intelligent}. Many existing studies and implementations have demonstrated RIS's capability of improving network capacity, coverage, energy efficiency, and security. 

Despite their potential, integrating RISs into wireless networks will significantly increase the complexity of network management \cite{marco}. For example, each RIS element requires independent phase-shift configurations, leading to large solutions spaces for optimization algorithms. The RIS configuration is more complicated when other control variables are jointly involved, such as beamforming, spectrum allocation, NOMA decoding order, or UAV trajectory design. Therefore, advanced optimization techniques are of paramount importance to handle such complexity and take full advantage of RISs. 
Motivated by the importance of optimization techniques, this work provides a comprehensive overview of optimization techniques of RIS-aided wireless communications, including model-based, heuristic, and machine learning (ML) approaches. There are several surveys devoted to the theory, design, analyses, and applications of RISs \cite{Almo,gong,moha,mohadz,cunh,rawa}. However, this work is different from existing surveys and tutorials by systematically summarizing and analyzing the optimization techniques for RIS-aided wireless networks, providing detailed comparisons, as well as including more state-of-the-art ML techniques. Specifically, as shown in Fig. \ref{fig1}, we focus on the following aspects:


\begin{table*}[!t]
\caption{Comparison of this work with existing surveys }
\centering
\small
\setstretch{1.5}
\begin{threeparttable} 
\resizebox{1\textwidth}{!}{%
\begin{tabular}{|m{0.65cm}<{\centering}|m{0.3cm}<{\centering}|m{0.4cm}<{\centering}|m{0.5cm}<{\centering}|m{0.5cm}<{\centering}|m{0.5cm}<{\centering}|m{0.65cm}<{\centering}|m{0.25cm}<{\centering}|m{0.55cm}<{\centering}||m{0.45cm}<{\centering}|m{0.9cm}<{\centering}|m{0.75cm}<{\centering}|m{1.1cm}<{\centering}||m{1.3cm}<{\centering}|m{1.55cm}<{\centering}|m{1.7cm}<{\centering}|m{1.05cm}<{\centering}|m{1.05cm}<{\centering}|m{1.05cm}<{\centering}|m{1.4cm}<{\centering}|m{1.05cm}<{\centering}|}
\hline \rule{0pt}{8pt} 
\multirow{5}*{\makecell{Ref.}} & \multicolumn{20}{c|}{\large RIS control and optimization-related contributions\tnote{1}}\\
\cline{2-21} \rule{0pt}{15pt} 
  &\multicolumn{8}{c||}{ Model-based approaches}&\multicolumn{4}{c||}{ Heuristic algorithms} & \multicolumn{8}{c|}{ Machine learning-based methods} \\
\cline{2-21} 
 & AO & MM & SCA & BCD & SDR & SOCP & FP & BnB & CCP & Meta-heuristic & Greedy method & Matching theory& Supervised learning & Unsupervised learning & Reinforcement learning & Federated learning  & Graph learning & Transfer learning  & Hierarchical learning & Meta- learning\\
\hline
\cite{Almo} & \checkmark & \checkmark & \checkmark & \checkmark &  &  &  &   &   & &  &   & \checkmark  &   &   &   &  &   &  &\\
\hline
\cite{gong} & \checkmark & \checkmark & \checkmark  & \checkmark  & \checkmark &  &  &   &     &     &   &    & \checkmark   &    &  \checkmark  &   &  &   & & \\
\hline
\cite{moha} & \checkmark &  \checkmark &   &    & \checkmark &    &   &    &   &   & &   &   &   & \checkmark  &   &  &   & & \\
\hline
\cite{mohadz} & \checkmark & \checkmark & \checkmark & & \checkmark & \checkmark &  & &  &   &   &  &   &   & \checkmark  &   &  &   & &\\
\hline
\cite{cunh} & \checkmark & \checkmark & \checkmark & \checkmark & \checkmark &  &  \checkmark &   & \checkmark  &  \checkmark &   & &  &   & \checkmark  &  &  &   & &\\
\hline
\cite{rawa} & \checkmark  &   &   &  &  \checkmark &   &   &   &   &  &   &   &  &  &  &   &  &   & & \\ %7
\hline
\cite{kfai} &  &  &  &  &   &  &  & &  &  &  &  & \checkmark & \checkmark &  \checkmark & \checkmark  &  &   & & \\ %8
\hline
\cite{yliu} & \checkmark &  & \checkmark  &  & \checkmark &   &   & \checkmark &  &  & \checkmark &  \checkmark &  \checkmark &  \checkmark  & \checkmark  & \checkmark  &  &   & & \\  %9
\hline
\cite{zheng2022survey} & \checkmark &  \checkmark  & \checkmark  &   &   &   &   &   &  &   &   &  \checkmark & \checkmark  &  &   &   &  &   & & \\
\hline
This work &  \checkmark  &  \checkmark  &  \checkmark  &  \checkmark  &  \checkmark  &  \checkmark    & \checkmark   &  \checkmark  &  \checkmark   &  \checkmark   &   \checkmark  &  \checkmark  &  \checkmark  & \checkmark &  \checkmark   &  \checkmark   &  \checkmark  &  \checkmark   &  \checkmark & \checkmark\\
\hline
\end{tabular}}

 \begin{tablenotes}    
        \footnotesize       
        \item[1] There are many surveys and tutorials on RISs recently, but Table \ref{tab1} focuses on studies that include control and optimization sections.   
\end{tablenotes} 
      
\end{threeparttable}  
\label{tab1}
\vspace{5pt}
\end{table*}


1) Problem formulations: \blue{We first introduce the fundamental theories of RIS technology, and then provide an overview of the problem formulations for optimizing RIS-aided wireless networks,} including maximization of sum-rate/capacity, energy efficiency, user fairness, and secrecy rate, and minimization of power consumption. In addition, we consider discrete RIS phase shifts and resource management problems that include integer control variables, and imperfect channel state information (CSI) with different error model constraints.
    
2) Model-based methods: In this work, model-based methods refer to algorithms that rely on specific optimization models with full knowledge of the defined problem\footnote{Note that some machine learning algorithms are also model-based, but here we use “model-based” to best describe the common features of a type of optimization algorithms.}. Model-based algorithms usually have demanding requirements for the properties and forms of problem formulations, e.g., convexity, continuity, and differentiability. We include the following model-based algorithms for optimizing RIS-aided wireless networks: alternating optimization (AO), the majorization-minimization (MM) method, successive convex optimization (SCA), block coordinate descent (BCD), semidefinite relaxation (SDR), second-order cone programming (SOCP), fractional programming (FP) and branch-and-bound (BnB). 
    
3) Heuristic algorithms: These algorithms apply heuristic rules for problem-solving. They provide more efficient alternatives to conventional model-based methods by sacrificing optimality and accuracy for low complexity and fast solutions. Heuristic algorithms can be used to solve NP-hard problems or serve as baselines and supplements for other algorithms. In this survey, we review the convex-concave procedure (CCP) algorithm, meta-heuristic algorithms, greedy algorithms, and matching theory for optimizing RIS-aided wireless networks. 

    
4) ML algorithms: ML algorithms are recognized as promising solutions for wireless network optimization\cite{eldar2022machine}. ML techniques do not need full knowledge of the defined problem, and they learn from data or interact with environments to find hidden patterns. We present state-of-the-art ML techniques for optimizing RIS-aided wireless networks, including supervised and unsupervised learning, reinforcement learning (RL), federated learning (FL), graph learning, transfer learning, hierarchical learning, and meta-learning. We provide in-depth analyses for algorithm features and applications towards RISs, i.e., the dataset acquisition of neural networks for RIS phase-shift optimization, and loss function definitions of unsupervised neural networks for data rate maximization. In addition, we compare model-based, heuristic, and ML approaches in terms of optimality, robustness, stability, and so on.

5) Applications and challenges towards 6G networks: We give an overview of RIS-assisted applications towards envisioned 6G networks, including NOMA, simultaneous wireless information and power transfer (SWIPT), mmWave and THz communications, nonterrestrial networks (NTNs), V2X communications, and integrated sensing and communication (ISAC). Moreover, we identify research challenges for the control and optimization of RISs.


In summary, the main contribution of this work is that we systematically survey the optimization techniques for RIS-aided wireless networks, ranging from problem formulations to the features and applications of various approaches. Our work aims to be a roadmap for researchers to optimize RIS-aided wireless networks. 
The rest of this work is organized as follows. Section \ref{sec-relat} reviews related work, while Section \ref{sec-bac} presents the problem formulations. Section \ref{sec-model}, \ref{sec-heu} and \ref{sec-ml} introduce model-based, heuristic, and ML optimization approaches, respectively, and we compare these three approaches in Section \ref{sec-compa}.
Section \ref{sec-futu} includes RIS-aided applications towards 6G networks and identifies future challenges. Finally, Section \ref{sec-con} concludes this survey.


