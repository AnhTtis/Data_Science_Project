

\section{\blue{RIS-assisted 6G Applications: Optimization Analyses and Challenges}}
\label{sec-futu}

%\begin{figure}[!t]
%\centering
%\includegraphics[width=0.6\linewidth]{Image/fig-noma.jpg}
%\caption{ Illustration of cluster-based RIS-aided NOMA.}
%\label{fig-noma}
%\setlength{\abovecaptionskip}{-2pt} 
%\vspace{-10pt}
%\end{figure}

\blue{This section analyzes control and optimization techniques for RIS-assisted 6G applications, e.g., potential optimization difficulties and algorithm selections.} In addition, we identify several research challenges for the optimization of RIS-aided wireless networks.   


\begin{table*}[!t]
\caption{\blue{\textbf{Control and optimization analyses for RIS-assisted 6G applications}} }
\centering
\small
\setstretch{1.05}
\resizebox{1\textwidth}{!}{%
\begin{tabular}{|m{1.3cm}<{\centering}|m{3.0cm}<{\centering}|m{4cm}<{\centering}|m{4.3cm}<{\centering}|m{4cm}<{\centering}|}
\hline 
\blue{6G applications}  &  \blue{\quad Key features}   &     \blue{Motivations for integrating RISs}    &  \blue{Potential optimization difficulties}  &   \blue{Analyses of optimization algorithm selections}  \\
\hline
\blue{RIS-NOMA} & \blue{NOMA enables spectrum sharing among users, e.g., multiple users can use the same time and frequency resource blocks, improving user fairness and spectral efficiency. }   & \blue{Due to the resource-sharing nature, the NOMA system is more vulnerable to security issues. Then RISs can be applied to reshape the signal propagation environment for security services against eavesdroppers.}   & \blue{RIS-NOMA integration increases the overall optimization complexity. Specifically, the decoding order may be frequently changed due to the dynamic RIS configuration, increasing the difficulty of applying conventional model-based algorithms.}  & \blue{ML techniques can be applied for intelligent decision-making in RIS-NOMA systems to handle the complexity and meanwhile improve long-term network performance. AO may be used to decouple the RIS control with NOMA optimization.}  \\
\hline
\blue{RIS-SWIPT} & \blue{SWIPT is an attractive solution to transmit electricity without using physical wire links, increasing the mobility, reliability, and safety of electronic devices.}   & \blue{The low energy efficiency at the energy receiver is one of the main issues for practical SWIPT deployment, and RISs become a promising solution, e.g., increasing sum-rate, reducing transmit power, and maximizing the minimum received power.}   & \blue{Existing studies mainly apply model-based methods for joint optimization of RISs and wireless power transfer, requiring dedicated model design. Although some low-complexity algorithms are proposed, i.e., MM \cite{chu2021intelligent} and bi-section search\cite{pan2020intelligent}, they still require full knowledge of the defined problem. }  & \blue{ ML approaches can be promising alternatives to handle the complexity of joint optimizing RISs and SWIPT. For instance, the RL agent can explore the complicated IoT environment without any prior knowledge, and intelligently optimize energy and information transfer and RIS control.  }  \\
\hline
\blue{RIS-mmWave and THz} & \blue{The increasing traffic demand and scarce bandwidth resources make mmWave and THz communications become appealing techniques.}   & \blue{mmWave and THz communications are vulnerable to signal blockages and attenuation, leading to severe path loss and reduced cover range. RISs can be applied to manipulate the signal propagation environment when the direct transmission is blocked.}   & \blue{ The optimization of RIS-mmWave and RIS-THz systems rely on accurate and practical channel estimation to identify the performance limit. Therefore, robust optimization techniques should be developed to handle the uncertainty in channel estimation. 
}  & \blue{ Compared with heuristic or ML algorithms, model-based algorithms are more reliable in guaranteeing the worst-case performance for RIS-mmWave and THz systems, e.g., maximizing worst-case network performance, or considering outage probability constraints. }  \\
\hline
\blue{RIS-NTN} & \blue{The NTN complements the limitations of terrestrial networks, providing flexible and reliable support for remote areas by UAVs, high-altitude platforms (HAPs) and low earth orbit satellites.}   & \blue{Frequent repositioning will increase the UAV power consumption, especially considering that UAVs are powered by a battery\cite{liu2020machine}. In this case, RISs can be applied to overcome this challenge, in which one can configure the RIS phase shifts instead of moving UAVs to save energy.}   & \blue{UAV communications are highly dynamic due to their mobility, and ML algorithms can be used to handle such uncertainty. However, ML model training requires many iterations, which may prevent UAVs from making real-time responses to network dynamics. }  & \blue{To overcome the tedious training iterations of conventional ML algorithms, transfer learning and meta-learning may be used to improve the training efficiency and make rapid responses to UAV dynamics, which is still an open issue.}  \\
\hline
\blue{RIS-V2X} & \blue{V2X is a key paradigm for envisioned 6G networks, enabling intelligent transportation with higher road safety and traffic efficiency.}   & \blue{Latency and reliability are the most critical requirements of V2X networks for road safety. However, the V2X transmission can be unstable due to fast-moving vehicles and the dynamic nature of wireless communications. Therefore, RISs can be exploited to improve channel capacity, coverage, signal strength and reliability.}   & \blue{A critical feature of V2X communications is the stringent requirement for reliability and safety, which means the proposed algorithm should guarantee the worst-case network performance. In addition, this indicates that the control and optimization algorithms should be efficient, robust and reliable.}  & \blue{ML algorithms can quickly adapt to dynamic environments, but the model training is time-consuming. Meanwhile, model-based algorithms can produce stable results, but they require full knowledge of the environment. Therefore, developing efficient optimization algorithms for V2X networks is still challenging for current studies.}  \\
\hline
\blue{RIS-ISAC} & \blue{ISAC is recently emerging as a key technology to support ubiquitous wireless connectivity and accurate sensing\cite{liu2022survey22}.}   & \blue{Target detection and parameter estimation are two primary tasks in radar sensing\cite{yao2022joint}, and RISs can be deployed to provide virtual LoS signal transmission, enabling the radar to sense targets in blocked areas\cite{10050406}. }   & \blue{ISAC offers significant potential by combining sensing and communication, but it also leads to extra complexity for network management, and such complexity further increases by involving RISs.}  & \blue{It is crucial to develop efficient optimization techniques to realize the full potential of the RIS-ISAC system, e.g., decoupling joint optimization into multiple sub-problems using AO, and applying ML algorithms for joint control.}  \\
\hline
\end{tabular}}
\label{tab-6gapp}
\vspace{-10pt}
\end{table*}



\subsection{ \blue{ Control and Optimization Analyses of RIS-assisted 6G applications}}

%\orange{{[Note: Here we have removed 6 separated subsections in the previous manuscript, e.g., RIS-NOMA, RIS-SWIPT, and RIS-V2X, and use Table.\ref{tab-6gapp} to summarize various RIS-aided 6G application scenarios.]}} 

\blue{Table \ref{tab-6gapp} summarizes RIS-assisted 6G applications, including NOMA, SWIPT, mmWave and THz communications, NTNs, V2X communications, and ISAC.}
\blue{For example, due to the resource-sharing nature, the NOMA system is more vulnerable to security issues. Then RISs can be applied to reshape the signal propagation environment for security services against eavesdroppers. The low energy efficiency at the energy receiver is one of the main issues for practical SWIPT deployment, and RISs become a promising solution to increase sum-rate, reduce transmit power, and maximize the minimum received power. In addition, Table \ref{tab-6gapp} also summarizes the motivations for integrating RISs with other 6G applications such as RIS-NTN, RIS-V2X, and RIS-ISAC.}

\blue{However, integrating RISs with 6G techniques also increases the difficulties for network management. In RIS-NOMA systems, the decoding order may be frequently changed due to the dynamic RIS configuration, increasing the difficulty of applying conventional model-based algorithms. For V2X networks, a critical feature is the stringent requirement for reliability and safety, which means the proposed algorithm should guarantee the worst-case network performance. Such a requirement means that the control and optimization algorithms should be efficient, robust and reliable. The potential optimization difficulties of other RIS-assisted 6G applications are also reviewed in Table \ref{tab-6gapp}. }


\blue{Finally, we analyze optimization algorithm selections for various RIS-assisted 6G applications. Integrating RISs will substantially increase the network management complexity, since RIS phase-shift control is highly coupled with other control variables such as decoding order in NOMA, beam selection in mmWave networks, and UAV altitude control. ML algorithms become promising solutions to handle such complexity, such as DDQN\cite{liu2020ris} and DDPG\cite{yang2020deep,yang2021machine}. In particular, these studies apply unified schemes to optimize network performance, overcoming the difficulties of reformulation and transformation for convexity. However, conventional ML algorithms require many iterations for model training, which may prevent the application to highly dynamic environments such as RIS-UAV. To this end, transfer learning and meta-learning may be used to improve training efficiency and make rapid responses.   
On the other hand, other applications such as V2X have more stringent service requirements to guarantee worst-case performance. In this case, model-based methods can usually provide more stable performance than ML or heuristic approaches, providing detailed proofs and explanations for the algorithm output.   }






%1) \textbf{RISs and NOMA}:
%NOMA is a key technique for 5G beyond and 6G networks, enabling spectrum sharing among users, e.g., multiple users can use the same time and frequency resource blocks. Compared with conventional orthogonal multiple access, NOMA improves user fairness and spectral efficiency and increases reliability. 
%Resource allocation is a critical part of RIS-NOMA systems to maximize data rate\cite{mu2021capacity} or minimize power consumption\cite{li2020joint,xie2021joint} by power and subchannels allocation and RISs configuration. For example, in \cite{mu2021joint}, Mu \textit{et al.} compare NOMA with TDMA and FDMA in RIS-aided wireless communications, demonstrating that RIS-NOMA can achieve a higher sum-rate. An AO-based method is proposed in \cite{xie2021joint} for transmit power minimization by jointly optimizing the beamforming, power allocation and RIS phase shifts. 
%In addition, due to the resource-sharing nature, the NOMA system is more vulnerable to security issues. Then RISs may be applied to reshape the signal propagation environment for security services against eavesdroppers, including secrecy rate maximization\cite{miao,zheng,dong2020enhancing}, artificial jamming signals\cite{guan2020intelligent,wang2020intelligent}, and creating signal or constellation overlapping\cite{lv2020secure}. 

%Although existing studies have demonstrated the potential of RIS-NOMA system\cite{mu2021capacity,li2020joint,liu2022reconfigurable,mu2021intelligent2}, such integration also increases the control and optimization complexity. In the RIS-NOMA systems, the decoding order may be frequently changed due to the dynamic RIS configuration, increasing the difficulty of applying conventional model-based algorithms.
%Therefore, ML techniques are applied for intelligent decision-making in the RIS-NOMA system, i.e., DDQN\cite{liu2020ris}, DDPG\cite{yang2020deep,yang2021machine}. In these studies, the main motivation for using ML algorithms is to handle the environment complexity and meanwhile improve long-term network performance. 
%For example, Liu \textit{et al.} optimize the phase and position of RISs and active beamforming of the BS in \cite{liu2020ris}, and these control variables are defined as a joint action, avoiding the complexity of decoupling and optimizing these variables one by one.
%RIS-NOMA systems have shown great potential, but the increased management complexity should be well considered.                 


%2) \textbf{RIS-aided SWIPT}:
%The future IoT networks will connect a huge number of devices for information exchange. One of the main challenges is the short battery life of IoT devices\cite{perera2017simultaneous}. 
%SWIPT is an attractive solution to transmit electricity without using physical wire links, increasing the mobility, reliability, and safety of electronic devices. 
%However, the low energy efficiency at the energy receiver is one of the main issues for practical SWIPT deployment, and RISs become a promising solution, e.g., increasing sum-rate\cite{lyu2020intelligent,chu2021intelligent,pan2020intelligent,zheng2020intelligent}, reducing transmit power \cite{wu2020joint22,wu2019weighted22}, and maximizing the minimum received power \cite{tang2020joint}.
%A sum-rate maximization problem is formulated in \cite{lyu2020intelligent} by applying RISs to improve the efficiency of energy and information transmission, achieving a much higher sum-rate than baseline schemes. Similarly, In \cite{chu2021intelligent},  Chu \textit{et al.} investigate the throughput maximization problem for a wireless-powered sensor network, jointly optimizing the RIS phase shifts and transmission time allocation of TDMA. RIS-aided SWIPT is investigated in \cite{pan2020intelligent,wu2020joint22,wu2019weighted22} by joint active and passive beamforming, and various model-based methods are applied for optimization, including BCD, SCA, QCQP \cite{pan2020intelligent}, AO and SDR \cite{wu2020joint22,wu2019weighted22}. Different from former studies, the authors in \cite{zheng2020intelligent} and \cite{tang2020joint} considered max-min problems to guarantee the system performance, in which \cite{zheng2020intelligent} maximized the minimum throughput, and \cite{tang2020joint} improved the minimum received power. 

%Note that the above studies mainly apply model-based methods for joint optimization of RISs and wireless power transfer, requiring dedicated model design. Although some low-complexity algorithms are proposed in these works, i.e., MM \cite{chu2021intelligent} and bi-section search\cite{pan2020intelligent}, they still require full knowledge of the defined problem. By contrast, ML approaches can be promising alternatives to handle the complexity of joint optimizing RISs and SWIPT.  
%However, there are few works that investigate RIS-aided SWIPT by using ML techniques, and developing an intelligent RIS-SWIPT system is still an open challenge.   





%3) \textbf{RIS-empowered mmWave and THz Communications}:
%The increasing traffic demand and scarce bandwidth resources make mmWave and THz communications become appealing techniques. %However, mmWave and THz communications are very vulnerable to signal blockages and attenuation, leading to severe path loss and reduced cover range. To this end, RISs can be applied to manipulate the signal propagation environment when the direct transmission is blocked. 
%Indoor RIS-mmWave is investigated in \cite{tan2018enabling} and \cite{perovic2020channel}. Specifically, the authors in \cite{tan2018enabling} introduced the design, modelling and optimization of reconfigurable 60 GHz reflect-array, while two optimization schemes are proposed in \cite{perovic2020channel} to maximize the channel capacity for indoor mmWave.
%For RIS-aided THz systems, the coverage of THz communications is analyzed in \cite{boulogeorgos2021coverage}, revealing a minimum transmission power that guarantees 100\% coverage probability. In addition, various optimization schemes are proposed to configure the THz system performance, such as data rate maximization, sum-rate maximization for indoor THz\cite{pan2022sum}, and energy efficiency maximization \cite{wu2021energy}. 

%These above studies demonstrate that RISs have great potential to enhance the performance of mmWave and THz communications, increasing the channel capacity\cite{perovic2020channel}, received power, data rate\cite{pan2022sum}, and energy efficiency\cite{wu2021energy}. While it is promising to apply RISs for mmWave and THz communications, there are still some open challenges. 
%One issue is the accurate and practical channel estimation for RIS-mmWave or RIS-THz communications, which is fundamental to identifying the performance limit and following optimizations. 
% In addition, robust optimization techniques should be developed to handle the uncertainty in channel estimation. Specifically, the problem modelling should be carefully designed, e.g., maximizing worst-case network performance, or considering outage probability constraints. 




%4) \textbf{RISs and Nonterrestrial Communications}:
%The NTN is an important part of future wireless networks, complementing the limitations of conventional terrestrial networks. NTNs can provide flexible and reliable support for communications in remote areas by UAVs, high-altitude platforms (HAPs) and low earth orbit satellites.  

%Applying UAVs as flying BSs or relays for wireless communications has been widely studied, and the primary motivation is that UAVs with high mobility can change their positions dynamically to adapt to diverse wireless environments\cite{zeng2016wireless}. 
%However, frequent repositioning will increase the UAV power consumption, especially considering that UAVs are powered by a battery (usually under 30 minutes)\cite{liu2020machine}. In this case, RISs can be applied to overcome this challenge, in which one can configure the RIS phase shifts instead of moving UAVs to save energy.
%In \cite{ma2020enhancing}, Ma \textit{et al.} analyze the performance of RIS-aided cellular communications with UAVs, and the simulations reveal that RISs can provide a 21dB gain for UAVs. %UAV communications are highly dynamic due to their mobility, and ML algorithms can be used to handle such uncertainty. However, ML model training requires many iterations, which may prevent UAVs from making real-time responses to network dynamics. To this end, transfer learning and meta-learning may be used to improve training efficiency, which is still an open issue. 

%Furthermore, HAPs and low earth orbit satellites have attracted the interest of academia and industry. RISs can perfectly match the requirements of these platforms due to the small size, low power consumption and cost features\cite{tekbiyik2022reconfigurable}. Meanwhile, integrating RISs into these systems can increase the channel capacity and received signal power and reduce the bit error rate. For example, the simulations in \cite{tekbiyik2022reconfigurable} demonstrate that RISs can diminish solar scintillation with a 945.4 km inter-satellite distance. In \cite{tekbiyik2021energy},  Tekbiyik \textit{et al.} apply RISs for the communications between IoT devices and satellites, achieving up to $10^5$ times transmission rates. Nevertheless, RIS-aided HAPs and satellites are in the very early stage, and many fundamental problems, such as channel estimation and resource allocation, are still open issues.   


%5) \textbf{RISs and V2X Communications}: 
%V2X is a key paradigm for the envisioned 6G networks, enabling higher road safety, traffic efficiency, and intelligent transportation. 
%Latency and reliability are the most critical requirements for V2X communications due to the importance of road safety. 
%However, the V2X transmission can be unstable due to fast-moving vehicles and the dynamic nature of wireless communications. Therefore, RISs can be exploited to improve channel capacity, coverage, signal strength and reliability.
%In particular, the authors in \cite{wang2020outage} obtained the outage probability expression by using series expansion and central limit theorem, showing that RISs can significantly reduce the outage probability for vehicle communications. The physical layer security of vehicle networks is studied in \cite{makarfi2020physical}, and the simulations demonstrate that RISs can improve the secrecy capacity. 
%Considering the high mobility of vehicles, RIS placement is important to guarantee the coverage of vehicle communications. An optimum RIS placement scheme is proposed in \cite{ozcan2021reconfigurable} for highway mmWave and THz vehicle communications, combating the high path loss in high frequency bands communications. Resource allocation is a crucial part of V2X communications\cite{chen2020resource,al2022reconfigurable}. In \cite{chen2020resource},  Chen \textit{et al.} jointly consider power and spectrum allocation and RIS configuration to maximize the V2X link capacity, and simulations show that RISs can increase 42.36\% higher channel capacity. 

%A critical feature of V2X communications is the stringent requirement for reliability and safety, which means the proposed algorithm should guarantee the worst-case network performance. In addition, this indicates that the control and optimization algorithms should be efficient, robust and reliable. 
%ML algorithms can quickly adapt to dynamic environments, but the model training can be time-consuming. Meanwhile, model-based algorithms can produce stable results, but they requires full knowledge of the environment. Therefore, developing efficient optimization algorithms for V2X networks is still a challenge for current studies.    



%6) \textbf{RIS-aided ISAC}:
%ISAC is recently emerging as a key technology to support ubiquitous wireless connectivity and accurate sensing\cite{liu2022survey22}. 
%Target detection and parameter estimation are two primary tasks in radar sensing\cite{yao2022joint}, and RISs can be deployed to provide virtual LoS signal transmission, enabling the radar to sense targets in blocked areas\cite{10050406}. 
%Radar communication coexistence (RCC) and dual-functional radar communication (DFRC) are two typical ISAC applications. The RCC system indicates that sharing the spectrum between co-existing radar and communication transmissions, and hence the interference between radar and communication transmitters should be carefully managed. 
%RISs provide an attractive solution to suppress the interference by manipulating the phase shifts. For instance, the authors in \cite{he2022ris} deploy one RIS near the BS to suppress the interference on radar, and another RIS on the user side to control the interference from the radar.
%On the other hand, the DFRC system applies fully-shared hardware and a unified transmit waveform to simultaneously implement sensing and communication functions. RISs can provide additional spatial degrees of freedom (DoFs) for the dual-functional waveform design, improving the sensing and communication performance simultaneously\cite{liu2022integrated}.

%ISAC has shown significant promise by combining sensing with communication, but it also leads to extra complexity for network management, and such complexity further increases by involving RISs. To this end, it is crucial to developing efficient optimization techniques to realize the full potential of the RIS-ISAC system, e.g., decoupling joint optimization into multiple sub-problems using AO, and applying ML algorithms for joint control.







\subsection{Challenges and Future Directions}
This subsection identifies research challenges and possible future directions. 


1) \textbf{Practical RIS Phase-shift Design}:  
Most existing RIS optimization studies rely on perfect CSI acquisition and static user conditions, which are impractical assumptions in the real world. Specifically, the wireless environment is highly dynamic due to various channel conditions and diverse user demands. Therefore, developing robust and practical algorithms for RIS control is of great importance for the real-world deployment of RISs, e.g., imperfect CSI acquisition and UEs with high mobility, which requires more research efforts.   

2) \textbf{Low-overhead Control}:
Many advanced control and optimization techniques have been proposed for RISs, but the communication and control overhead is neglected in most works. For example, frequent parameter exchange between the BS and RISs may lead to high overhead, and the model training overhead of ML algorithms can hamper the system efficiency. These issues are still open challenges, and agile optimization algorithms with low complexity and overhead are yet to be developed.    

3) \textbf{ML-enabled Intelligent RIS Beamforming}:
ML is one of the most promising techniques to facilitate future 6G networks, and integrating ML with RISs can bring intelligent prediction, clustering, and decision-making for RIS-aided wireless networks.   
Despite the significant potential, some critical questions, e.g., algorithm deployment, offline or online training, and training cost, are neglected in many existing studies.
Addressing these problems can further enable an intelligent future wireless network.


4) \textbf{Practical RIS Location Optimization}: In many existing studies, RIS location is considered as a predefined parameter for simulation. However, RIS location can considerably affect the system performance and therefore should be very carefully handled. Moreover, the real-world environment is more complicated when considering dense buildings and other obstacles. RIS location and scale should be jointly optimized by considering the wireless environment, user distribution, and service requirements, which still require research effort.

5) \textbf{Flexible Control Framework}:
The former analyses have shown that each optimization approach has its advantages and difficulties.  Model-based methods have higher stability and optimality, and heuristic methods have lower complexity, while ML techniques are more robust. 
One intuitive direction is to combine these methods to form a flexible optimization framework that can make the most of each approach's advantages and complement the difficulties. However, many existing studies stick with one type of optimization technique, and flexible control schemes are considered future challenges.     