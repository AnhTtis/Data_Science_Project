

\section{Heuristic Algorithms for RIS-aided Wireless Networks }
\label{sec-heu}


As presented in Section \ref{sec-model}, model-based algorithms have specific requirements for problem formulations, especially for convexity and continuity. 
Meanwhile, the large number of RIS elements, dynamic channel conditions, and various CSI levels further contribute to the overall complexity.  
Therefore, transformations and relaxations are required to convert the original problem into specific forms. 
Moreover, these transformations are usually problem-specific, requiring case-by-case analyses and dedicated design. 

By contrast, heuristic algorithms have fewer requirements for objective functions and constraints, which will significantly reduce the complexity. 
Compared with model-based methods, heuristic algorithms are usually considered low-complexity solutions. In the following, we will introduce four heuristic algorithms, including CCP, meta-heuristic algorithms, greedy algorithms, and matching-based algorithms. 


\subsection{Convex-concave Procedure} 



The CCP algorithm uses the local heuristic to solve difference of convex (DC) problems, which is considered a low-complexity solution for complicated wireless network optimization\footnote{The main reason that CCP is considered a heuristic algorithm is that it applies a simple heuristic rule for optimization, which is iteratively finding two points with the same tangent vectors\cite{lipp}. Hence, CCP fits well with our defined classifications of heuristic algorithms.}. 

DC problems are frequently formulated in many fields, representing many scenarios that cannot be solved in polynomial time. The DC problem is defined as
\begin{equation}\label{eq-ccp}
\begin{aligned}
\max\limits_{x\in \mathscr{X}}  &  \quad f_{0}(x)-g_{0}(x) \\
 \text{s.t.}  \quad & f_{i}(x)-g_{i}(x) \leq 0; \ i=1,2,3,,,I,
\end{aligned}
\end{equation}
where $f(x)$ and $g(x)$ are both convex. DC problems are usually non-convex unless $g_{i}(x)$ are affine, which is generally hard to solve. 


As shown in Fig. \ref{fig-ccp}, the core idea of CCP is to find $x^{l+1}$ in the $l+1$ iteration that satisfies $\nabla_x f(x^{l+1})=\nabla_x g(x^{l})$, indicating a point on $f(x)$ that has the same tangent with $g(x^{l})$\cite{yulie}. 
The CCP algorithm will first form 
    \begin{equation} \label{eq-ccp112}
        \overline{g}_{i}(x|x^l)=g_{i}(x^{l})+ \nabla_x g(x)(x-x^{l}),\ i=0,1,2,3,,,I. 
    \end{equation}
Then it solves the following problem to get $x^{l+1}$
    \begin{equation}\label{eq-ccp2}
    \begin{aligned}
    \max\limits_{x\in \mathscr{X}}  &  \quad f_{0}(x)-\overline{g}_{0}(x|x^l)\\
    \text{s.t.}  \quad & f_{i}(x)-\overline{g}_{i}(x|x^l) \leq 0; \ i=1,2,3,,,I.
    \end{aligned}
    \end{equation}
Equation (\ref{eq-ccp2}) is equivalent to  $\nabla_x f(x^{l+1})=\nabla_x g(x^{l})$ by deriving the objective function, and equations (\ref{eq-ccp112}) and (\ref{eq-ccp2}) are iteratively repeated until reaching the stop criteria. 
CCP algorithm does not require a dedicated step size design, and the main reason is that the estimator $f_{0}(x)-\overline{g}_{0}(x|x^l)$ is global. It retains all the information from the convex component $f(x)$ and only linearizes the concave portion $g(x)$.



\begin{figure}[!t]
\centering
\includegraphics[width=1\linewidth]{Image/fig-ccp.jpg}
\caption{Convex concave procedure for RIS-related optimization.}
\label{fig-ccp}
\setlength{\abovecaptionskip}{-2pt} 
\vspace{-10pt}
\end{figure}



Moreover, there are multiple extensions of the CCP algorithm.  For instance, the penalty CCP includes a penalty term for violations, which removes the requirements for feasible initial points. The RIS phase shift optimization problem is reformulated as \cite{cunh}
\begin{equation}\label{eq-ccp11}
\begin{aligned}
\max\limits_{\theta, \tau >0}  &  \quad f(\theta)-\tau^{l} \sum_{i=1}^{2I} v_{i} \\
\text{s.t.}  \quad & {g}(\theta) \geq  D,\\
\quad & |\theta_i^l|^2 - 2 Re\{\theta^*_i\theta_i^l\} \leq v_i-1,\\
\quad & |\theta_i|^2 \leq 1+v_{i+I}, \ i=1,2,3,,,I,
\end{aligned}
\end{equation}  
where $\tau^{l} \sum_{i=1}^{2I} v_{i}$ is the penalty term, $v_i$ are slack variables, and $\tau^{l}$ is a coefficient that will decline in each iteration for convergence. After some transformations, problem (\ref{eq-ccp11}) can be solved by using the CVX toolbox, and the detailed procedure is included in \cite{cunh}. 

CCP has been applied in \cite{guiz3,yuanbin,chen2021qos,niu2021simultaneous,xu2022reconfigurable} for controlling RIS phase shifts. In these works, the joint optimization problem is first decoupled into multiple sub-problems using BCD or AO, then penalty CCP is used to solve the RIS phase shifts sub-problem. The main motivation is the high complexity of solving non-convex RIS control problems. For example, 
the sum-rate maximization problem in \cite{chen2021qos} is converted into three sub-problems: joint optimization of the transmit power and spectrum sharing, SDR-based multi-user detection, and CCP-based RIS phase shifts. However, note that CCP is a heuristic algorithm that will find a locally optimum solution, and the initial point $x^0$ may affect the final output. In particular, there may exist multiple locally optimal solutions, and CCP can easily get stuck in a sub-optimal one.





\subsection{Meta-heuristic Algorithms}

One of the main difficulties of controlling RISs is the large number of RIS elements, leading to huge solution spaces. Therefore, it is hard to achieve exact solutions by finding a closed-form expression, and hence model-based approximation algorithms such as SCA and MM are applied. 
However, these methods have stringent requirements for objectives and constraints, especially for convexity, continuity, and differentiability. By contrast, meta-heuristic algorithms can search significantly large solution spaces with few or no additional requirements on problem forms \cite{bozorg2017meta}. 
It usually contains intelligent policies to guide the heuristic exploration, producing high-quality solutions efficiently. 
Meta-heuristic algorithms have been extensively developed, e.g., genetic algorithm (GA), particle swarm algorithm (PSO), ant colony optimization, simulated annealing, and tabu search \cite{beheshti2013review}. 

Fig. \ref{fig-pso} shows the steps of using population-based meta-heuristic algorithms for RIS phase-shift design. The first step is to initialize the algorithm parameters such as population numbers and crossover rate in a genetic algorithm.
Then, the algorithm will produce initial individuals, which indicates various RIS phase-shift designs. The objective function is converted into a fitness function, e.g., the sum-rate or energy efficiency. After that, the algorithm will constantly search for better solutions using heuristic rules iteratively, such as evolution strategy in a genetic algorithm, and particle movement for PSO. Finally, the heuristic exploration will stop if the fitness function values converge or reach maximum iteration numbers.

Compared with model-based methods, the main advantage is that meta-heuristic algorithms can easily adapt to both continuous and discrete RIS phase shifts without relaxation and transformation. PSO and GA are used for RIS phase shifts in \cite{dai2021reconfigurable} and \cite{zhi2022power} to maximize the data rate. Statistical CSI is investigated in \cite{zhi2021statistical} to obtain a closed-form expression of the uplink ergodic data rate, then GA is deployed for phase control to maximize the data rate. In addition, Tabu search is applied to irregular RIS to decide the element design in \cite{ruoc}. 

The simulations in \cite{dai2021reconfigurable,zhi2022power,zhi2021statistical,ruoc} show that meta-heuristic algorithms can significantly reduce the optimization complexity, especially for MINLP problems. 
However, it may be trapped in local optima, and the algorithm performance relies on the parameter settings. For example, the phase shifts of hundreds of RIS elements require a large number of populations in GA, leading to high exploration costs. By contrast, reducing the population numbers may lower the probability of finding optimal solutions.         

\begin{figure}[!t]
\centering
\includegraphics[width=0.7\linewidth]{Image/fig-pso.jpg}
\caption{Population-based meta-heuristic algorithms for RIS phase-shift control.}
\label{fig-pso}
\setlength{\abovecaptionskip}{-2pt} 
\vspace{-10pt}
\end{figure}






\subsection{Greedy Algorithms} 
Most former algorithms are designed to find global optima of the objective function. However, many problems are NP-hard and non-convex, and the solutions are usually problem-specific with a series of transformations. To this end, greedy algorithms are proposed as low-complexity alternatives. In particular, greedy algorithms refer to the problem-solving heuristic that makes locally optimal decisions at each stage regardless of global optima\cite{martins2006metaheuristics}.
Consider an minimization problem $\min f(\Vec{x})$, and the control variables $\Vec{x}$ include $\Vec{x}=\{x_1,x_2,...,x_i,...,X_I\}$. 
At each stage, the greedy algorithm will optimize only one control variable $x_i$ by $\hat{x_i}=\argmin_{x_i\in X_i} f(\Vec{x})$, while holding the rest of variables unchanged. Then it moves to the next stage until $i=I$.

RIS elements' on/off control is a non-convex problem with discrete constraints. It may be solved by relaxing the integer constraint $ \chi \in \{0,1\}$ into $0\leq \chi \leq 1$, but the reformulated problem can still be complicated. 
A low-complexity solution is a greedy element-by-element control. Specifically, it evaluates the on/off decision of one RIS element at each stage by observing the changes in objective functions. If the performance is improved by achieving a higher sum-rate and lower power consumption, then the on/off status will be updated\cite{zhaohui}. Similarly, as shown in Fig. \ref{fig-greedy}, this greedy scheme can also be applied to control RIS phase shifts, indicating that one element is optimized at a time by observing the improvement of objective functions, e.g., achieving higher data rate or energy efficiency. Then, the next RIS element is optimized sequentially\cite{rivera,Atapattu,tewes}. 

In addition, greedy algorithms are used to relax the constraints. For example, the RIS phase shift is allowed to violate the stringent constraints in \cite{angli}, then the achieved objective values are compared with the theoretical optimal results to find a feasible solution. Greedy schemes may be combined with AO to handle problems with multiple sub-objectives. For instance, a greedy scheme is applied in \cite{muham} to maximize the served users by controlling RIS phase shifts, then it schedules the users to minimize the age of information.  
The main advantage of the greedy algorithm is the low complexity by decoupling the joint optimization into multiple stages. However, instead of global optima, it can only achieve local optima. The simple greedy policy means that there is no guarantee for the algorithm's performance, which may lead to poor output in some cases.

\begin{figure}[!t]
\centering
\includegraphics[width=0.85\linewidth]{Image/fig-greedy.jpg}
\caption{Greedy method for RIS phase shift control.}
\label{fig-greedy}
\setlength{\abovecaptionskip}{-2pt} 
\vspace{-10pt}
\end{figure}





\subsection{Matching Theory-based Methods} 
The matching theory is useful for optimizing resource allocation problems, i.e., subcarrier assignment, user-BS and user-RIS association, and mode selection. These problems are formulated as MINLPs, and a possible solution is to relax the zero-one constraints and reformulate the problem. A linear conic relaxation method is proposed in \cite{zhang2021joint} for the user-RIS association, but it further requires SDP to solve the relaxed formulation, leading to high computational complexity. In addition, the problem becomes even more complicated when RIS on/off control and phase shifts are involved. Consequently, the primary motivation for applying matching theory is to achieve low-complexity solutions efficiently. 

Consider the most widely applied many-to-one matching problem with two finite and disjoint sets of players $\mathcal{U}$ and $\mathcal{B}$. $\mathcal{U}$ represents users, and $\mathcal{B}$ may be BS, RIS or subchannels.
\begin{definition}
The considered many-to-one matching problem is defined by\\
(a) Matching relationship function $f:=\mathcal{U}\times\mathcal{B}$ with $u \in \mathcal{U}$ and $b \in \mathcal{B}$, e.g., the many-to-one association relationship between multiple users and one BS;\\
(b) $|f(u)|=1$ with $\forall u \in \mathcal{U}$, indicating that one $u$ can only be matched with at most one $b$ in many-to-one matching problem, e.g., one user can be associated with at most one BS. \\
(c) $|f^{-1}(b)|\geq K$ with $\forall b \in \mathcal{B}$, which means that $b$ has a capacity limit for the connection with $u$. For example, one BS has a maximum service capability for users.\\ 
(d) $b=f(u)$ $\leftrightarrow$ $u=f^{-1}(b)$. This means the matching is bidirectional and mutual. 
\end{definition}
To describe the exchange operation between different matching, the swap matching is considered 
\begin{definition}
Given $u\in f^{-1}(b)$ and $u' \in f^{-1}(b')$ with $u,u' \in \mathcal{U} $ and  $b,b' \in \mathcal{B} $, the swap matching is defined by $f_{u,b,u',b'}=\{f \backslash \{ (u, b)(u',b') \} \} \cup  \{ (u, b')(u',b) \}$. 
\end{definition} 
Swap matching allows $u$ and $u'$ to exchange their matched $b$ and $b'$, while other players remain unchanged. For instance, two users can exchange their associated BSs without changing other association pairs.
\begin{definition} \label{defi-block}
$u$ and $u'$ become a swap blocking pair if and only if\\
(a) For all players in $\{ u,b,u',b' \}$, $F(f_{u,b,u',b'}) \geq F(f)$, where $F$ is the utility function of players. This means that the utility functions of all involved players will not decrease. \\
(b) At least one player in $\{ u,b,u',b' \}$ has $F(f_{u,b,u',b'}) > $ $ F(f)$, indicating that at least one player's utility is improved, e.g., at least one user achieves higher channel capacity or data rate by switching pairs.
\end{definition}
Definition \ref{defi-block} shows that the overall utility can be improved by finding swap matching pairs. Then the stable matching is defined by
\begin{definition} \label{defi-stable}
The matching relationship between two sets $\mathcal{U}$ and $\mathcal{B}$ is two-sided exchange-stable if there is no swap blocking pairs. This means that the overall utility such as channel capacity or sum-rate cannot be improved by switching user-BS associations.      
\end{definition}


\begin{table*}[!t]
\caption{Summary of heuristic algorithms for RIS-aided wireless networks}
\centering
\small
\setstretch{1.1}
\resizebox{1\textwidth}{!}{%
\begin{tabular}{|m{1.1cm}<{\centering}|m{3cm}<{\centering}|m{4.1cm}<{\centering}|m{3cm}<{\centering}|m{3.8cm}<{\centering}|m{4.4cm}<{\centering}|}
\hline 
Methods &  Main features   &     Advantage    &  Drawbacks    &  Difficulties   &  Application \qquad \qquad \qquad scenarios  \\
\hline
CCP & CCP aims to obtain local optima of DC problems by iteratively finding points with the same tangent values.  &  CCP algorithm does not require a dedicated step size design, and it retains all of the information from the convex
component and only linearizes the concave portion.  & The selection of initial points may affect the final results, and hence it requires an initialization method. &  The problem has to be reformulated into the DC form; initial points may need to be selected several times due to local optima.  & Penalty CPP is used for imperfect CSI in \cite{guiz3}, and statistical CSI in \cite{yuanbin}. Other applications include non-convex RIS control problems in \cite{chen2021qos,niu2021simultaneous,xu2022reconfigurable}. \\
\hline
Meta-heuristic algorithms & Meta-heuristic algorithms apply intelligent policies to guide the heuristic exploration iteratively, producing high-quality solutions efficiently. &  Meta-heuristic algorithms can search huge solution spaces with few or no additional assumptions.  & It can be trapped in local optima, and these algorithms require many iterations. The algorithm performance is sensitive to parameter selections.  & Selecting the best parameters is complicated in meta-heuristic algorithms. For instance, crossover probability in GA and inertia weight in PSO may decide the algorithm performance.   & Meta-heuristic algorithms are mainly deployed for RIS phase-shift control, including GA for rate maximization in \cite{zhi2022power}, PSO for phase-shift optimization in \cite{dai2021reconfigurable, zhi2021statistical}, tabu search for irregular RIS in \cite{ruoc}.     \\
\hline
Greedy algorithms &  Instead of global optima, greedy algorithms make locally optimal decisions at each stage of solving the problem.    & Greedy algorithms can greatly reduce the complexity by decoupling the original 
 problem into multiple stages.   &  It may present poor global performance, since the local optima in one stage may lead to bad results for the next stage.  &  Finding the trade-off between low complexity and good algorithm performance is critical to using the greedy heuristic.   & Greedy algorithms are used for RIS phase control in \cite{rivera,Atapattu,tewes} and on/off control in \cite{zhaohui} as low-complexity solutions, providing low-complexity alternative solutions for NP-hard problems.  \\
 \hline
Matching theory-based method &  Matching-based method is designed to solve matching or association problems with two sides of players.  &  Compared with direct optimization methods, matching-based methods have lower complexity for large-scale problems.  &  Matching-based method may require exhaustive searches to find matching pairs.  &  Defining the utility function for two sides of players with peer effects is difficult.  & Matching theory is mainly applied to resource allocation and association problems in RIS-aided networks, e.g., channel assignment and user-BS-RIS association\cite{wu2022resource, zuo2020resource}.    \\
 \hline
\end{tabular}}
\label{tab-heuristic}
\vspace{0pt}
\end{table*}



\begin{figure}[!t]
\centering
\includegraphics[width=0.9\linewidth]{Image/fig-match.jpg}
\caption{Matching theory applications in RIS-aided wireless networks.}
\label{fig-match}
\setlength{\abovecaptionskip}{-2pt} 
\vspace{-10pt}
\end{figure}

%The existence of stable matchings has been proved in \cite{bodine2011peer}. 
Definition \ref{defi-stable} is very useful in matching theory, since it provides locally optimal criteria, and it is easily achieved by searching and eliminating all the swap blocking pairs. Fig. \ref{fig-match} presents the applications of matching theory in RIS-aided wireless networks, including D2D-user pairing, user-BS-RIS association, channel assignment, etc. It shows that matching theory provides an efficient solution for overcoming these NP-hard problems. For instance, a RIS-aided maritime communication system is investigated in \cite{cao2022joint}, in which many-to-one matching was applied for the joint mode selection and power control of BSs. In RIS-assisted NOMA system, many-to-one matching is used for channel assignment \cite{wu2022resource, zuo2020resource} and user clustering \cite{zhang2020joint}, while many-to-one and many-to-many matching are jointly considered in \cite{ni2021resource} for the UE association and channel assignment. Moreover, matching theory is applied in \cite{el2022latency} for edge computation offloading in RIS-aided networks, and a deferred acceptance matching game is formulated in \cite{taghavi2021user} for user association in mmWave networks with RISs.
The simulations in \cite{cao2022joint,wu2022resource, zuo2020resource,zhang2020joint, ni2021resource, el2022latency, taghavi2021user} demonstrate that matching theory is a low-complexity solution for resource allocation and association problems in RIS-aided communication systems. 
However, matching theory relies on iterative searching to eliminate swap blocking pairs, and the searching cost may increase exponentially with more players. In addition, the wireless network players will affect each other, changing the overall interference level. Such peer effects may increase the complexity of applying matching theory.


\begin{figure}[!t]
\centering
\subfigure[Sum-rate comparison under various numbers of RIS elements. The greedy algorithm applies element-by-element RIS phase-shift control. It decides the phase-shift of one element at each time by observing the improvement in sum-rate, and then moves to the next element. The genetic algorithm considers different phase shift designs as individuals, and uses evolutionary strategies to find near-optimal solutions.]{ \label{fig_result3}
\includegraphics[width=7.2cm,height=5.2cm]{Image/fig-results4.jpg}
}
\,
\subfigure[Convergence performance of the genetic algorithm ]{ \label{fig_result4}
\includegraphics[width=7.2cm,height=5.2cm]{Image/fig-results3.jpg}
}
\setlength{\abovecaptionskip}{0pt} 
\caption{\blue{Simulation results of greedy and genetic algorithms. We consider a MISO system with one BS and multiple UEs, and detailed simulation parameters and algorithms can be found in \cite{zhou2023heuristic}}}.
\vspace{-15pt}
\label{fig-re-heu}
\end{figure}



\subsection{Discussions and Numerical Results}

Table \ref{tab-heuristic} compares heuristic algorithms in terms of main features, advantages, drawbacks, difficulties, and applications.  
Compared with model-based algorithms, a common advantage of heuristic algorithms is their low complexity. 

RIS-related optimization problems may involve summation, logarithm, fractional terms, and discrete constraints in problem formulations, which are non-convex and highly non-linear. Applying model-based algorithms generally require a series of transformation and relaxation to achieve a convex or a concave reformulation, but this complexity is avoided in heuristic algorithms. For instance, many problems are easily converted into DC forms, and the CCP algorithm can be applied by iteratively finding two points with the same tangent vectors. Compared with other estimation-based methods such as MM or SCA, the CCP method has much lower complexity, since no extra surrogate function is required. However, the initial point selection may affect the solution quality of the CCP algorithm.  

Greedy algorithms employ a simple greedy policy for decision-making. They aim to maximize the current benefit, disregarding the effect on future stages. Greedy algorithms can efficiently solve problems in near-linear time complexity. However, greedy algorithms can only generate locally optimal results, and the increasing number of control variables may lead to poor performance.

By contrast, meta-heuristic algorithms apply more advanced heuristic rules for iterative exploration, e.g., genetic algorithm, tabu search, and PSO. 
Similar to greedy algorithms, meta-heuristic algorithms have no requirements for problem formulations and constraints, and objective functions can be easily converted into fitness functions. 
However, compared with greedy algorithms, meta-heuristic algorithms can better guarantee the solution quality by using heuristic rules for iterative optimization.


Different from previous approaches, matching theory specializes in solving resource allocation and association problems.
In matching-based methods, the control variables are considered as matching operations, and the objective function is improved by searching swap matching pairs. Therefore, when handling these allocation problems, matching theory is more efficient than other heuristic algorithms due to its dedicated design. 

  

Note that heuristic algorithms may be combined with model-based algorithms. For instance, the energy-efficiency maximization problem in \cite{zhaohui} is decoupled into the beamforming optimization, phase control, and RIS on/off optimization, in which beamforming and phase control are solved by SCA, and RIS on/off is optimized by the greedy algorithm. Other combinations can be found in \cite{Atapattu} by combining SDR with greedy heuristic, and in \cite{ni2021resource} by combining SCA and SDR with matching methods. 

Finally, Fig. \ref{fig-re-heu} shows an example with greedy and genetic algorithms. In particular, the greedy algorithm applies element-by-element RIS phase-shift control. It decides the phase-shift of one element at each time by observing the improvement in sum-rate, and then moves to the next element. Meanwhile, genetic algorithm considers different phase shift combinations as individuals, and uses evolutionary strategies to find near-optimal solutions. Fig. \ref{fig_result3} provides the sum-rate under various numbers of RIS elements. It shows that heuristic algorithms can achieve satisfactory performance with a limited number of RIS elements. However, when the number of RIS elements increases, both greedy and genetic algorithms present sub-optimal results. In addition, Fig. \ref{fig_result4} illustrates the convergence performance of the genetic algorithm. It shows that the average values of individuals increase with iterations, and finally the optimal objective value converges. The main reason is that the genetic algorithm applies evolutionary policies, which will select elite individuals to produce new solutions, and therefore the solution quality is constantly improved.    













