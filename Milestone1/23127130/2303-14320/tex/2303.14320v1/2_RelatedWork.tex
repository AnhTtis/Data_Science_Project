
\section{Related Surveys}
\label{sec-relat}

There are many research directions relating to RISs, including channel modelling and estimation, signal processing, performance analysis, passive beamforming, and hardware designs. This work focuses on optimization techniques due to their paramount importance, and Table \ref{tab1} compares this work with existing surveys in terms of control and optimization-related contributions. 

Table \ref{tab1} shows that most existing works focus on model-based approaches, including AO, MM, SCA, and SDR. The main reason is that these techniques have been widely applied, e.g., using AO to decouple joint active and passive beamforming, and applying MM and SCA to approximate non-convex objectives. Then, heuristic algorithms are usually considered as low-complexity alternatives and supplements. For example, greedy algorithms are used for element-by-element RIS phase-shift control, and matching theory is applied for resource allocation. However, despite their importance, heuristic approaches are omitted in many existing surveys. Meanwhile, ML algorithms have been widely used for wireless network management, but existing surveys are limited in supervised learning and RL. In addition, some newly emerging techniques, such as graph learning and hierarchical learning, are not mentioned in existing surveys.       

More specifically, in many existing studies \cite{Almo,gong,moha,mohadz}, optimization techniques are very briefly discussed by introducing the algorithm titles that have been used in the literature, but the motivations and algorithm features are not included.  
Alghamdi \textit{et al.} overviewed optimization and performance analysis techniques of RISs, but it is limited in analyzing problem formulations \cite{rawa}. 
In \cite{kfai}, Faisal and Choi specialized in ML approaches for RIS-aided wireless networks, but model-based and heuristic approaches are not included. Besides, some state-of-the-art ML techniques, including graph learning and hierarchical learning, are not included in \cite{kfai}.  
By contrast, multiple model-based approaches are introduced in \cite{cunh} for signal processing of RISs, but many heuristic and ML techniques are not covered.    
Liu \textit{et al.} presented RIS beamforming, resource management and ML for RIS-aided wireless networks, but only RL is presented in detail \cite{yliu}. Supervised learning, unsupervised learning, and FL are briefly discussed in \cite{yliu}, while newer techniques, such as graph learning, transfer learning, and hierarchical learning, are not covered. In \cite{zheng2022survey}, Zheng \textit{et al.} surveyed the channel estimation and practical RIS control under imperfect/statistical/hybrid CSI, but some optimization techniques are not included. 



\begin{table*}[!t]
\caption{Summary of RIS-aided sum-rate/capacity maximization works under various constraints}
\centering
\small
\setstretch{1.15}
 \begin{threeparttable}  
\resizebox{1\textwidth}{!}{%
\begin{tabular}{|m{0.5cm}<{\centering}|m{2.1cm}<{\centering}|m{1.6cm}<{\centering}|m{1.8cm}<{\centering}|m{1.8cm}<{\centering}|m{4.8cm}<{\centering}|m{4.5cm}<{\centering}|m{3.8cm}<{\centering}|}
\hline 
Ref. & Scenario\tnote{1}  & Phase-shift resolution & Channel settings  & CSI & Control variables & Constraints & Algorithms  \\
\hline
\cite{ruoc} & MIMO-DL-MU irregular RISs &  Continuous & Rayleigh fading  & Perfect & BS beamforming matrix, RISs selection and phase shifts  &  Total transmit power, RISs selection and phase constraints  & AO, Tabu search, extraction based cross-entropy methods  \\
\hline
\cite{cunhua} & MIMO-DL-MU Multicell  &  Continuous & Rayleigh and Rician fading  & Perfect & BS beamforming vectors  and RIS phase shifts & Total transmit power and RISs phase constraints  & BCD, MM, Complex circle manifold  \\
\hline
\cite{shuowen}  &  MIMO-DL OFDM  & Continuous & Rayleigh and Rician fading &  Perfect   & Transmit covariance matrix and RIS phase shifts  & Total transmit power and RISs phase constraints  &   AO   \\
\hline
\cite{yu}  &  MIMO-UP/DL & Continuous/ Discrete &  Rayleigh fading  & Perfect &  Source precoders and RIS phase shifts  &  Total transmit power and RISs phase constraints  &  AO    \\
 \hline
\cite{ming} &  MISO-DL-MU  &  Discrete  & Correlated Rician fading  & Statistical/ Instantaneous  &  BS beamforming matrix and RIS phase shifts  &  Total transmit power and RISs phase constraints  & Penalty dual decomposition, Stochastic SCA   \\
 \hline
\cite{jiey} & MISO-DL-SU Cognitive Radio  & Continuous & Rician fading  & Perfect/ Imperfect  &  BS beamforming vectors  and RIS phase shifts   & Interference level, total transmit power, and RISs phase constraints  & BCD, SDP, SOCP \\
 \hline 
\cite{boya}  & MISO-DL-MU  & Discrete & Reflection-dominated  &  Imperfect &   BS beamforming vectors  and RIS phase shifts  & Total transmit power and RISs phase constraints  &  AO, B\&B, SDR\\
 \hline
\cite{huayan}  & MISO-DL-MU & Continuous & Rayleigh and Rician fading  & Perfect/ Imperfect  & BS beamforming vectors  and RIS phase shifts  &  Total transmit power and RISs phase constraints & FP, BCD, complex circle manifold  \\
 \hline
 \cite{chang}  &  SISO-UL-SU   & Discrete  & Rayleigh fading  & Estimated &  RIS phase shifts   &  RISs phase constraint  & Successive refinement algorithm    \\
\hline
\cite{xidong} &  MISO-DL-MU NOMA   & Continuous/ Discrete & Rayleigh and Rician fading  &  Perfect  & BS beamforming vector, RIS phase shifts, and user decoding order  & Successive interference cancellation, total transmit power, and RISs phase constraints  & AO, SCA, sequential rank-one constraint relaxation    \\
 \hline
\cite{yuanbin}  &  MISO-UL-MU mmWave V2X  & Continuous &  Saleh-Valenzuela   & Imperfect &  Transmit power, multi-user detection matrix, and RIS phase shifts  &  Target SINR, total transmit power, and RISs phase constraints  &  AO, SCA, penalty CCP \\
\hline
\cite{dai2021reconfigurable}  &  MIMO-DL-MU  &  Continuous/ Discrete &  Rician fading  & Perfect &  RIS phase shifts  &  RISs phase constraint  &  Particle swarm optimization (PSO) \\
\hline
\cite{zhi2021statistical}  &  MIMO-DL-MU  & Discrete & Rayleigh and Rician fading  & Statistical &  RIS phase shifts  &  RISs phase constraint  & Genetic algorithm   \\
\hline
\cite{rivera}  & Point to point communication &  Discrete & Rician fading  & Imperfect & RIS phase shifts  & RISs phase constraint  & Greedy algorithm  \\
\hline
\cite{ni2021resource}  &  Multi-cell NOMA   & Continuous & Rayleigh and Rician fading  & Perfect  & User association, subchannel assignment, reflection matrix, power allocation, and decoding order.  & Target data rate, total transit power, association, decoding, and RISs phase constraints. &  Matching method, convex upper bound substitution, SCA, and SDR. \\
\hline
\cite{chen2021qos}  &  V2X communications    &  Continuous & Rayleigh and Rician fading    &  Large-scale 
 and slowly varying   &  Transmit power, multi-user detection matrix, spectrum sharing, and RIS phase shifts    &  Maximum transmit power, RISs phase, QoS requirement, and spectrum sharing protocol    &  BCD, SDR, CCP \\
\hline
\end{tabular}}

 \begin{tablenotes}    
        \footnotesize       
        \item[1]MISO: multiple input single output; DL/UL: downlink/uplink; SU/MU: single user/ multiple users.  
\end{tablenotes} 
      
\end{threeparttable}  
\label{tab-rate}
\vspace{-15pt}
\end{table*}


This work is different from existing studies in the following aspects: 
\begin{itemize}
    \item  Control and optimization have been included in many surveys, but this work is the first to systematically investigate optimization techniques of RIS-aided wireless networks, ranging from problem formulations to steps, features, advantages, and difficulties of nearly 20 techniques.    
    \item We present in-depth analyses to apply these optimization techniques to RISs. For example, deep neural network (DNN) and deep reinforcement learning (DRL) are included in many existing surveys, but some important questions are not discussed, i.e., dataset acquisition for neural network training in RIS-aided environments, and customizing the state, action, and reward function definitions for RL-enabled RIS control. The answers to these questions are critical to taking full advantage of RISs.
    \item Finally, we present the most state-of-the-art ML techniques for optimizing RIS-aided wireless networks, e.g., graph learning, transfer learning, and hierarchical learning, which are not included in existing surveys, to the best of our knowledge. These novel techniques may bring new research directions.  
\end{itemize}
To summarize, this survey answers the following: what are the state-of-the-art techniques for optimizing RIS-aided wireless networks, and how do they cover different aspects with respect to each other? 

