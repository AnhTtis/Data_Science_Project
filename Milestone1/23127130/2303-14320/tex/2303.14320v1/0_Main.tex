\documentclass[conference]{IEEEtran}
\IEEEoverridecommandlockouts
% The preceding line is only needed to identify funding in the first footnote. If that is unneeded, please comment it out.
\usepackage{cite}
\usepackage{amsmath,amssymb,amsfonts}
\usepackage{algorithmic}
\usepackage{graphicx}
\usepackage{textcomp}
\usepackage{amsthm}
\usepackage{xcolor}
\usepackage{bbding}
\usepackage{makecell}
\usepackage{algorithm}  
\usepackage{color}
\usepackage{multirow}
\usepackage{bm}
\usepackage{array}
\usepackage{booktabs}
\usepackage{tabularx,booktabs}
\usepackage{multicol}
\usepackage{cleveref}
\usepackage{balance}
\usepackage{lastpage}
\usepackage{tabulary}
\usepackage{subfigure}
\usepackage{etoolbox}
\usepackage{bbm}
\usepackage{enumitem}
\usepackage{mathrsfs}
\setlength{\columnwidth}{6.5in}
\usepackage{multicol}
\usepackage{amsfonts,amssymb}
\usepackage{ulem}
\usepackage{cancel}
\usepackage{setspace}
\usepackage{stackengine}
\usepackage{threeparttable}
\newcommand\xrowht[2][0]{\addstackgap[.5\dimexpr#2\relax]{\vphantom{#1}}}
\DeclareMathOperator*{\argmax}{arg\,max}
\DeclareMathOperator*{\argmin}{arg\,min}

\usepackage{amsthm}
\theoremstyle{definition}
\newtheorem{definition}{Definition}
\newtheorem{theo}{Theorem}
\renewcommand{\proofname}{\textbf{Proof}}

\def\BibTeX{{\rm B\kern-.05em{\sc i\kern-.025em b}\kern-.08em
    T\kern-.1667em\lower.7ex\hbox{E}\kern-.125emX}}
    \newcommand{\notered}[1]{\textcolor{red}{[{\bf #1}]}}
        \newcommand{\red}[1]{\textcolor{red}{[{#1}}]}
\newcommand{\noteorange}[1]{\textcolor{orange}{[{\bf #1}]}}
\newcommand{\noteblue}[1]{\textcolor{blue}{[{\bf #1}]}}
\newcommand{\blue}[1]{\textcolor{blue}{#1}}
\newcommand{\orange}[1]{\textcolor{orange}{ #1}}
\newcommand{\notegreen}[1]{\textcolor{green}{[{\bf #1}]}}
\newcommand{\cyan}[1]{\textcolor{cyan}{{ #1}}}

\newcommand{\tabincell}[2]{\begin{tabular}{@{}#1@{}}#2\end{tabular}}

%\pagestyle{plain}

\begin{document}

\bstctlcite{IEEEexample:BSTcontrol}

\title{%A Comprehensive Tutorial on Optimization Techniques for RIS-aided Future Wireless Communications: Model-based, Heuristic and Machine Learning Approaches\\
A Survey on Model-based, Heuristic, and Machine Learning Optimization Approaches in RIS-aided Wireless Networks\\
\thanks{
This work was supported in part by the Natural Sciences and Engineering Research Council of Canada (NSERC) Canadian Collaborative Research and Training Experience Program (CREATE) under Grant 497981, the Canada Research Chairs Program, and the U.S National Science Foundation under Grants CCF-1908308 and CNS-2128448.

H. Zhou and M. Erol-Kantarci are with the School of Electrical Engineering and Computer Science, University of Ottawa, Ottawa, ON K1N 6N5, Canada. (emails:\{hzhou098, melike.erolkantarci\}@uottawa.ca).

Yuanwei Liu are with the School of Electronic Engineering and Computer Science, Queen Mary University of London, London E1 4NS, U.K. (email:yuanwei.liu@qmul.ac.uk).

H. Vincent Poor is with the Department of Electrical and Computer Engineering,
Princeton University, Princeton, NJ 08544 USA (e-mail: poor@princeton.edu).} 

}

\author{\IEEEauthorblockN{Hao Zhou, Melike Erol-Kantarci, \IEEEmembership{Senior Member, IEEE}, Yuanwei Liu, \IEEEmembership{Senior Member, IEEE},\\ and H. Vincent Poor, \IEEEmembership{Life Fellow, IEEE}} }

\maketitle

\begin{abstract}
Reconfigurable intelligent surfaces (RISs) have received considerable attention as a key enabler for envisioned 6G networks, for the purpose of improving the network capacity, coverage, efficiency, and security with low energy consumption and low hardware cost. However, integrating RISs into the existing infrastructure greatly increases the network management complexity, especially for controlling a significant number of RIS elements. To unleash the full potential of RISs, efficient optimization approaches are of great importance. 
This work provides a comprehensive survey on optimization techniques for RIS-aided wireless communications, including model-based, heuristic, and machine learning (ML) algorithms. 
In particular, we first summarize the problem formulations in the literature with diverse objectives and constraints, e.g., sum-rate maximization, power minimization, and imperfect channel state information constraints. 
Then, we introduce model-based algorithms that have been used in the literature, such as alternating optimization, the majorization-minimization method, and successive convex approximation. Next, heuristic optimization is discussed, which applies heuristic rules for obtaining low-complexity solutions. Moreover, we present state-of-the-art ML algorithms and applications towards RISs, i.e., supervised and unsupervised learning, reinforcement learning, federated learning, graph learning, transfer learning, and hierarchical learning-based approaches. Model-based, heuristic, and ML approaches are compared in terms of stability, robustness, optimality and so on, providing a systematic understanding of these techniques. 
Finally, we highlight RIS-aided applications towards 6G networks and identify future challenges.   
\end{abstract}

\begin{IEEEkeywords}
Reconfigurable intelligent surfaces, optimization, model-based methods, heuristics, machine learning.
\end{IEEEkeywords}

\vspace{-10pt}
%%%%%%%%% BODY TEXT

\section{Introduction}
\label{section:introduction}
%% 1. why should someone care?

%The advent of advanced interactive computer vision systems~\cite{hololens} and recent progress in vision-language and multi-modal models~\cite{} opens doors for such next generation of assistive agents. 
% We envision that the future assistive agents would build up on these visual and language reasoning capabilities of today and empower users to achieve goals in their everyday lives. In particular, such agents would be able to reason about \emph{unseen} human goals... 
% We posit that such agents would require the ability to understand user goals described in natural language at high-level i.e., without complete details about as well as unseen user goals. 

%Recent progress in augmented reality systems~\cite{hololens, magicleap}, as well as vision-language and multi-modal models~\cite{}, opens doors for the next generation of assistive agents. 
Inspired by recent progress in visual systems~\cite{MagicLeap, ungureanu2020hololens}, we consider an assistive egocentric agent capable of reasoning about daily activities. When invoked via natural language commands, for e.g., while baking a cake, the agent understands the steps involved in baking, tracks progress through the various stages of the task, detects and proactively prevents mistakes by making suggestions. Such an agent would empower users to learn new skills and accomplish tasks efficiently.
% One could envision invoking such an agent merely through natural language descriptions of tasks similar to how present day assistants such as Alexa, Siri etc.~\cite{voice_assistants} are invoked. 
%We envision such agents to empower users in daily life by  invoking them naturally through 

%% 2. Why is it challenging? 
%While recent progress in vision-language and multi-modal models~\cite{} opens doors for such next generation of assistive agents, various challenges remain in making such agents a reality. 
%To make such agents a reality, 

Developing such an egocentric agent capable of tracking and verifying everyday tasks based on their natural language specification is challenging for multiple reasons. First, such an agent must reason about various ways of doing a \emph{multi-step} task specified in natural language. This entails decomposing the task into relevant actions, state changes, object interactions as well as any necessary causal and temporal relationships between these entities. Secondly, the agent must ground these entities in egocentric observations to track progress and detect mistakes. Lastly, to truly be useful, such an agent must support tracking and verification for a combination of tasks and, ideally, even unseen tasks. These three challenges -- causal and temporal reasoning about task structure from natural language, visual grounding of sub-tasks, and compositional generalization -- form the core goals of our work.

% %% 3. What are we doing? What is our approach?
% \aks{I think this is a matter of preference, but I personally don't like related work in intro. I would make this paragraph be about EgoTV and NSG. Starting with something like - "To this end, we propose...", ie, your next paragraph.}
% \nk{+1, we should move parts of this para to lit review and delete the rest.}
% Recent research on language modeling enables decomposing tasks into multiple steps from natural language descriptions~\cite{llm_zero_shot_planning,proscript}. However, such \emph{task decompositions} cannot directly be leveraged for task tracking in egocentric agents because of lack of grounding into the visual observations or context. In parallel, the computer vision community has advanced action recognition~\cite{}, object detection and tracking~\cite{}, hand object interaction and object state change detection~\cite{ego_4d,change_it,}, step classification in procedural tasks~\cite{}, and even vision language reasoning~\cite{nsvqa,nscl,star_situated_reasoning,clevrer}, which may help with the grounding challenge. However, majority of current research on identifying actions, objects, steps, or state changes does not account for the overall task structure. Likewise, predominant research on vision language understanding~\cite{} and multi-modal grounding~\cite{} does not consider the temporal and causal constraints that emerge in task tracking and verification. We therefore focus on the order-aware visual grounding problem in our work, with an eye towards compositional generalization to scale usability of these agents. In particular, we aim to achieve visual grounding of the actions and objects corresponding to each step or sub-task obtained from the task description decomposition in an order-aware manner.

%% 4. What are our results/contributions?
As our first contribution, we propose a benchmark -- \emph{\textbf{Ego}centric \textbf{T}ask \textbf{V}erification} (\etv \inlineimg{figures/TV}) -- and a corresponding dataset in the AI2-THOR~\cite{ai2thor} simulator. % \emoji{tv}
Given a natural language (NL) task description and a corresponding egocentric video of an agent, the goal of \etv is to verify whether the task was successfully completed in the video or not.
\etv contains multi-step tasks with \emph{ordering} constraints on the steps and \emph{abstracted} NL task descriptions with omitted low-level task details inspired by the needs of real-world assistants. We also provide splits of the dataset focused on different generalization aspects, e.g., unseen visual contexts, compositions of steps, and tasks (see Figure~\ref{figure:dataset}).
% Next, we create splits of the dataset focused on different aspects of generalization, ranging from generalization to unseen visual context to unseen compositions of steps and tasks. Figure~\ref{figure:dataset} shows an example task and overview of generalization splits from \etv. Succeeding at \etv tasks requires decomposing tasks into partially-ordered steps from the NL description and order-aware visual grounding of these steps into the video. 

Our second contribution is a novel approach for order-aware visual grounding~--~\emph{\textbf{N}euro-\textbf{S}ymbolic \textbf{G}rounding} (NSG), capable of compositional reasoning and generalizing to unseen tasks owing to its ability to leverage abstract NL descriptions and compositional structure of tasks (task decomposition, ordering).~In contrast, state-of-the-art vision-language models~\cite{coca,clip,videoclip,clip_hitchiker} struggle to ground NL descriptions in egocentric videos, and do not generalize to unseen tasks.~NSG outperforms these models by~$\mathbf{33.8}\%$~on compositional generalization and~$\mathbf{32.8}\%$~on abstractly described task verification. Finally, to evaluate \nsg on real-world data, we instantiate \etv on the CrossTask~\cite{cross_task} instructional video dataset. %Specifically, we synthetically create videos with mistakes in CrossTask. 
We find that it also outperforms state-of-the-art models at task verification on CrossTask. We hope that the \etv~benchmark and dataset will enable future research on egocentric agents capable of aiding in everyday tasks.

% We experiment with many for the \etv tasks. We find that while these models generalize well to unseen visual context, they struggle to perform grounding from abstracted task descriptions and to generalize to new compositions of tasks. To deal with these challenges, we take inspiration from recent research on and develop . ~\rd{unclear why neurosymbolic models would do well on abstraction.} 

% To summarize, our main contributions are:~1)~\etv: a benchmark and synthetic dataset to systematically study egocentric task verification.
% 2)~\nsg: a novel neuro-symbolic approach to enable the core reasoning capability for \etv -- order-aware visual grounding. We demonstrate \nsg's capability on our synthetic \etv dataset as well as a real-world dataset derived from CrossTask. We will release both of these datasets and our models for future research on egocentric task tracking and verification. 


% Assistive agents require the ability to track actions and state changes from an egocentric perspective for effective assistance in day-to-day tasks. For example, an agent helping a user prepare a recipe would need to both generate the steps of the recipe (\textit{plan generation}) and track the user's actions to ensure the plan is executed correctly (\textit{plan verification}). We formulate this as a Video Entailment task~\cite{violin_dataset,9710490} \rd{should we call our task video-based goal entailment?}, wherein, given an egocentric video of an agent (or human) performing a task (\textit{premise}) and a NL task description (\textit{hypothesis}), the objective is to learn a model to track whether the given task was successfully executed in the video. 
% An ideal model should also be able to seamlessly generalize to novel compositions (of actions and objects) unseen during training. \rd{add a line about what we mean by abstraction and why is it important.} To this end, we generate a novel Vision-Language dataset on the AI2-THOR simulator~\cite{ai2thor} to study compositional and abstraction-based generalization. Our dataset provides effective evaluation measures in a controlled setting, while closely reflecting the diversity of real-world events. We implement and train a variety of end-to-end models based on existing state-of-the-art approaches. We empirically demonstrate that neural models suffer from overfitting and cannot effectively generalize to novel compositions of actions, objects, and scenes. 
% To address this problem, we propose an end-to-end Neuro-Symbolic (NeSy) framework that performs plan generation and verification. At the heart of our approach is the hypothesis that symbolic reasoning models are good at generalization and capturing compositional substructure, while neural models are good at learning representations from sensory data~\cite{10.5555/3326943.3327039,nscl,clevrer}. \rd{summarize contributions in a bulleted list.} \rd{also add a line about the main result e.g., x\% improvement as compared to end-to-end models}. 

% \rd{we also evaluate NeSy with real-world data: add briefly about CrossTask experiments.}

% % \fbox{\begin{minipage}{\linewidth}
% % \textbf{Problem Statement}

% % Given:
% % (i) Premise: Egocentric video of an agent performing a task.
% % (ii) Hypothesis: NL description of the task.

% % Learn: A model to track whether the premise entails the hypothesis. The output of the model is True if the given task is executed successfully in the video.
% % \end{minipage}}

% \textbf{Contributions:} 
% \begin{itemize}
%     \item We generate a benchmark video-language dataset to study compositional and abstraction-based generalization.
%     \item We evaluate the performance of a variety of state-of-the-art models and show that these (baseline) models cannot effectively generalize to novel compositions of actions.
%     \item We propose a novel end-to-end NeSy approach that significantly outperforms the baselines on some compositional generalization splits while performing on par with them on the rest.
%     \item We also evaluate our NeSy approach with real-world data showing similar performance improvements.
% \end{itemize}


\section{Related Surveys}
\label{sec-relat}

There are many research directions relating to RISs, including channel modelling and estimation, signal processing, performance analysis, passive beamforming, and hardware designs. This work focuses on optimization techniques due to their paramount importance, and Table \ref{tab1} compares this work with existing surveys in terms of control and optimization-related contributions. 

Table \ref{tab1} shows that most existing works focus on model-based approaches, including AO, MM, SCA, and SDR. The main reason is that these techniques have been widely applied, e.g., using AO to decouple joint active and passive beamforming, and applying MM and SCA to approximate non-convex objectives. Then, heuristic algorithms are usually considered as low-complexity alternatives and supplements. For example, greedy algorithms are used for element-by-element RIS phase-shift control, and matching theory is applied for resource allocation. However, despite their importance, heuristic approaches are omitted in many existing surveys. Meanwhile, ML algorithms have been widely used for wireless network management, but existing surveys are limited in supervised learning and RL. In addition, some newly emerging techniques, such as graph learning and hierarchical learning, are not mentioned in existing surveys.       

More specifically, in many existing studies \cite{Almo,gong,moha,mohadz}, optimization techniques are very briefly discussed by introducing the algorithm titles that have been used in the literature, but the motivations and algorithm features are not included.  
Alghamdi \textit{et al.} overviewed optimization and performance analysis techniques of RISs, but it is limited in analyzing problem formulations \cite{rawa}. 
In \cite{kfai}, Faisal and Choi specialized in ML approaches for RIS-aided wireless networks, but model-based and heuristic approaches are not included. Besides, some state-of-the-art ML techniques, including graph learning and hierarchical learning, are not included in \cite{kfai}.  
By contrast, multiple model-based approaches are introduced in \cite{cunh} for signal processing of RISs, but many heuristic and ML techniques are not covered.    
Liu \textit{et al.} presented RIS beamforming, resource management and ML for RIS-aided wireless networks, but only RL is presented in detail \cite{yliu}. Supervised learning, unsupervised learning, and FL are briefly discussed in \cite{yliu}, while newer techniques, such as graph learning, transfer learning, and hierarchical learning, are not covered. In \cite{zheng2022survey}, Zheng \textit{et al.} surveyed the channel estimation and practical RIS control under imperfect/statistical/hybrid CSI, but some optimization techniques are not included. 




This work is different from existing studies in the following aspects: 
\begin{itemize}
    \item  Control and optimization have been included in many surveys, but this work is the first to systematically investigate optimization techniques of RIS-aided wireless networks, ranging from problem formulations to steps, features, advantages, and difficulties of nearly 20 techniques.    
    \item We present in-depth analyses to apply these optimization techniques to RISs. For example, deep neural network (DNN) and deep reinforcement learning (DRL) are included in many existing surveys, but some important questions are not discussed, i.e., dataset acquisition for neural network training in RIS-aided environments, and customizing the state, action, and reward function definitions for RL-enabled RIS control. The answers to these questions are critical to taking full advantage of RISs.
    \item Finally, we present the most state-of-the-art ML techniques for optimizing RIS-aided wireless networks, e.g., graph learning, transfer learning, and hierarchical learning, which are not included in existing surveys, to the best of our knowledge. These novel techniques may bring new research directions.  
\end{itemize}
To summarize, this survey answers the following: what are the state-of-the-art techniques for optimizing RIS-aided wireless networks, and how do they cover different aspects with respect to each other? 


\vspace{-5pt}
\section{Diffusion Models and Part-Level Shape Representation}
\label{sec:background}

\subsection{Background on Diffusion Models}
\label{sec:background_ddpm}
We first briefly overview the technical background of diffusion models.
Diffusion models~\cite{Ho:2019DDPM} are latent variable models that approximate a data distribution $q (\mathbf{x}^{(0)})$ with a Markov chain, which is also called a \emph{reverse process}:
\vspace{-2pt}
\begin{align}
    p_{\theta} (\mathbf{x}^{(0)}) \coloneqq \int p_{\theta} (\mathbf{x}^{(0:T)}) d \mathbf{x}^{(1:T)},
\end{align}
where $p_{\theta} (\mathbf{x}^{(0:T)}) = p (\mathbf{x}^{(T)}) \Pi_{t=1}^{T} p_{\theta} (\mathbf{x}^{(t-1)} \vert \mathbf{x}^{(t)})$.
Here, $p (\mathbf{x}^{(T)}) = \mathcal{N} (\mathbf{x}^{(T)}; \mathbf{0}, \mathbf{I})$ is the standard normal prior
enabling tractable sampling.

The conditional probabilities $\{ p_{\theta} (\mathbf{x}^{(t-1)} \vert \mathbf{x}^{(t)}) \}_{t=1}^{T}$ are parameterized by a neural network whose weights are denoted by $\theta$.
The weights are optimized through the \emph{forward} diffusion process $q (\mathbf{x}^{(1:t)} \vert \mathbf{x}^{(0)})$ that sequentially adds Gaussian noises to the data $\mathbf{x}^{(0)} \sim q (\mathbf{x}^{(0)})$:
\begin{align}
\begin{gathered}
    q (\mathbf{x}^{(1:t)} \vert \mathbf{x}^{(0)}) \coloneqq \Pi_{s=1}^{t} q (\mathbf{x}^{(s)} \vert \mathbf{x}^{(s-1)}), \\
    \text{where} \,\, q (\mathbf{x}^{(s)} \vert \mathbf{x}^{(s-1)}) \coloneqq \mathcal{N} \left(\mathbf{x}^{(s)}; \sqrt{1 - \beta^{(s)}} \mathbf{x}^{(s-1)}, \beta^{(s)} \mathbf{I} \right),
    \raisetag{35pt}
\end{gathered}
\end{align}
and $\beta^{(s)}$ is an element of a monotonically increasing sequence $\beta^{(1:T)} \in (0, 1]^{T}$.
By choosing Gaussians as forward diffusion kernels, the conditional densities $q (\mathbf{x}^{(t)} \vert \mathbf{x}^{(0)})$ at $t=1,\dots,T$ can be expressed in the closed form:
\begin{align}
    q (\mathbf{x}^{(t)} \vert \mathbf{x}^{(0)}) = \mathcal{N} (\mathbf{x}^{(t)}; \sqrt{\bar{\alpha}^{(t)}} \mathbf{x}^{(0)}, (1 - \bar{\alpha}^{(t)}) \mathbf{I}),
\end{align}
where $\alpha^{(t)} \coloneqq 1 - \beta^{(t)}$ and $\bar{\alpha}^{(t)} \coloneqq \Pi_{s=1}^{t} \alpha^{(s)}$.
Over the forward process dissipating a sample $\mathbf{x}^{(0)} \sim q (\mathbf{x}^{(0)})$ toward $q(\mathbf{x}^{(T)}) = \mathcal{N} (\mathbf{0}, \mathbf{I})$, the weights $\theta$ parameterizing the reverse process $p_{\theta} (\mathbf{x}^{(0)})$ are learned by optimizing the following variational bound on negative log likelihood:
\begin{equation}
\begin{aligned}
    \mathbb{E}_{q (\mathbf{x}^{(0)})} & [-\log p_{\theta} (\mathbf{x}^{(0)})] \leq \\
    &\mathbb{E}_{q (\mathbf{x}^{(0)}, \dots, \mathbf{x}^{(T)})} \left[-\log \frac{p_{\theta} (\mathbf{x}^{(0:T)})}{q (\mathbf{x}^{(1:T)} \vert \mathbf{x}^{(0)})}\right].
\end{aligned}
\end{equation}
Following Ho \etal~\cite{Ho:2019DDPM}, we parameterize our reverse process $p_{\theta} (\mathbf{x}^{(t-1)} \vert \mathbf{x}^{(t)})$ as:
\begin{equation}
\begin{aligned}
    p_{\theta} (\mathbf{x}^{(t-1)} \vert \mathbf{x}^{(t)}) \coloneqq \mathcal{N} (\mathbf{x}^{(t-1)}; \boldsymbol{\mu}_{\theta} (\mathbf{x}^{(t)}, t), \beta^{(t)} \mathbf{I}).
\end{aligned}
\end{equation}
In particular, we use the parameterization $\boldsymbol{\mu}_{\theta} (\mathbf{x}^{(t)}, t) = \sfrac{1}{\sqrt{\alpha^{(t)}}} (\mathbf{x}^{(t)} - \sfrac{\beta^{(t)}}{\sqrt{1 - \bar{\alpha}^{(t)}}} \boldsymbol{\epsilon}_{\theta} (\mathbf{x}^{(t)}, t))$ and optimize its parameters $\theta$ with a training objective that encourages a network $\boldsymbol{\epsilon}_{\theta}$ to predict the noise $\boldsymbol{\epsilon} \sim \mathcal{N}(\mathbf{0}, \mathbf{I})$ present in the given data:
\begin{align}
    \mathcal{L}(\theta) \coloneqq \mathbb{E}_{t, \mathbf{x}^{(0)}, \boldsymbol{\epsilon}} \left[\left\lVert \boldsymbol{\epsilon} - \boldsymbol{\epsilon}_{\theta} \left(\sqrt{\bar{\alpha}^{(t)}} \mathbf{x}^{(0)} + \sqrt{1 - \bar{\alpha}^{(t)}}\boldsymbol{\epsilon}, t\right) \right\rVert^{2}\right].
    \label{eq:ddpm_loss}
\end{align}







\begin{figure}[t!]
\label{fig:spaghetti_overview}
\includegraphics[width=\linewidth]{figures/pipeline2_draft.pdf}
\caption{\textbf{Part-Level implicit representation by Hertz~\etal~\cite{Hertz:2022Spaghetti}.} A latent vector $\mathbf{z}$ encoding global geometry is first mapped to a set of part latents $\{\mathbf{p}_i\}_{i=1}^N$, each of which is decomposed into extrinsic parameters $\{\mathbf{e}_i\}_{i=1}^N$ and intrinsic latents $\{\mathbf{s}_i\}_{i=1}^N$. The decoder, conditioned on $\{(\mathbf{e}_i, \mathbf{s}_i)\}_{i=1}$, outputs an occupancy value given a query point $\mathbf{x}$.}
\vspace{-10pt}
\end{figure}

\vspace{-15pt}
\subsection{Part-Level Shape Representation}
\label{sec:background_part_representation}
Neural implicit representations~\cite{Chen:2019ImNet, Park:2019Deepsdf, Mescheder:2019OccNet} have been widely exploited in 3D shape generation and reconstruction due to their advantages in capturing fine details without limitation in resolutions even with a small memory footprint. However, their disadvantage of not supporting intuitive editing and manipulation has been a hindrance to increasing their utilization. To remedy the drawback, recent works~\cite{Genova:2019LearningShapeTemplates,Genova:2020LDIF,Hao:2020Dualsdf,Hui:2022NeuralTemplate,Hertz:2022Spaghetti} introduced \emph{dual} representations combining explicit and implicit representations, taking advantage of both of them. Among them, Hertz~\etal~\cite{Hertz:2022Spaghetti}, which our work is based on, was the first introducing a hybrid representation integrating two types of disentanglements simultaneously into an implicit representation: 1) part-level disentanglement, representing each local region separately, and 2) extrinsic-intrinsic disentanglement, describing extrinsic properties (\ie~the approximate shape and transformations) with parameters in the 3D space while encoding intrinsic properties (\ie~geometric details) using a latent code. This novel representation, called SPAGHETTI~\cite{Hertz:2022Spaghetti}, is learned in an auto-decoding setup without any supervision of the part decomposition.


In SPAGHETTI, a 3D shape is first mapped to a global latent $\mathbf{z}$ and then further encoded into a set of part embedding vectors $\{\Vp_i\}_{i=1}^{N}$, where $N$ denotes the number of parts. Each part embedding vector $\Vp_i$ is again mapped into both a set of extrinsic parameters $\Ve_i$ and an intrinsic latent $\Vs_i$ through an MLP. 
The set of extrinsic parameters $\Ve_i = \{\mathbf{c}_i, \mathbf{\Sigma}_i, \mathbf{\pi}_i \}$ of each part represents a Gaussian in the 3D space with mean $\mathbf{c}_i\in\mathbb{R}^3$ and covariance $\mathbf{\Sigma}_i \in \mathbb{R}^{3 \times 3}$, depicting an approximate shape of a part.
$\pi_i \in \mathbb{R}$ is the blending weight for the Gaussian mixture representation of the entire shape: $\sum_{i}\pi_i \mathcal{N}(\B{x} | \B{c}_i, \mathbf{\Sigma}_i)$, describing the volume of the shape as a probability distribution. Since $\{\Ve_i\}_{i=1}^{N}$ can only encode the part-level structural information, the intrinsic latents $\{\Vs_i\}_{i=1}^{N}$ supplement the detailed geometry information so that the pairs of the extrinsic parameters and intrinsic latents can be decoded back to the original shape in an implicit form. Specifically, an implicit decoder $\mathcal{D}$ is trained to predict an occupancy value at point $\mathbf{x}$:
\vspace{-0.5\baselineskip}
\begin{equation}
\begin{aligned}
\label{eq:decoder}
    o = \mathcal{D}\left(\mathbf{x}\, \Big\vert \, \{\Ve_i\}_{i=1}^{N},\,\{\Vs_i\}_{i=1}^{N}\right),
\end{aligned}
\end{equation}
where occupancy value $o \in [0,1]$ is 1 when the query point is inside the shape, and 0 otherwise. 
The keys to achieving both the part-level and extrinsic-intrinsic disentanglements in the training of decoder $\mathcal{D}$ are the regularizations forcing a single pair $(\Ve_i, \Vs_i)$ of a part to determine the occupancy of each point, and the Gaussian parameters in $\Ve_i$ to transform the corresponding local region. See the original paper~\cite{Hertz:2022Spaghetti} for the details of the decoder training.

The extrinsic vector $\mathbf{e}_i$ is precisely represented as a $16$-dimensional vector $\{\mathbf{c}_i, \lambda_i^1, \lambda_i^2, \lambda_i^3, \mathbf{u}_i^1, \mathbf{u}_i^2, \mathbf{u}_i^3, \pi_i\}$, where $\lambda_i^j \in \mathbb{R}$ and $\mathbf{u}_i^j \in \mathbb{R}^3$ are eigenvalues and eigenvectors of the covariance matrix $\mathbf{\Sigma}_i$, while the intrinsic vector $\mathbf{s}_i$ is a 512-dimensional vector. Note that the much smaller extrinsic vector contains the approximate shape information of the part; we leverage this fact in our effective cascaded diffusion model.

Also, note that SPAGHETTI is trained in an auto-decoding setup while regularizing the global latent code $\mathbf{z} \in \mathbb{R}^{512}$ to follow the unit Gaussian. Thus, the shapes can be simply generated by sampling a latent code $\mathbf{z}$ from the unit Gaussian in the $\mathbf{z}$ space, although we demonstrate that diffusion in the extrinsic and intrinsic embedding spaces can produce much more plausible shapes (Section~\ref{sec:shape_generation}).


 
%\input{4_Channel}




\section{Model-based Optimization Algorithms for RIS-aided Wireless Networks}
\label{sec-model}

This section introduces model-based algorithms and applications for optimizing RIS-aided wireless networks, including AO, MM, SCA, BCD, SDR, SOCP, FP, and BnB. In addition, we summarize the features, advantages, drawbacks, difficulties, and applications of these techniques.   

\subsection{Alternating Optimization} 



AO has been widely applied for RIS-related control and optimizations. The main reason is the high complexity of joint optimization problems that include multiple control variables such as BS beamforming matrix and RIS phase shifts. Fig. \ref{fig-AO} shows the AO approach to decouple the RIS passive beamforming with BS active beamforming, and the circles in the figures indicate objective functions with the same value. $x^{l}_{1}$ is first optimized from $x^{l}_{1}$ to $x^{l+1}_{1}$ and holding $x^{l}_{2}$ unchanged, where $l$ indicates the iteration numbers. Then $x^{l}_{2}$ is optimized to $x^{l+1}_{2}$ alternatively. AO can significantly reduce the computational complexity by optimizing $x^{l}$ to $x^{l+1}$ in an iterative manner, since sub-problems with fewer control variables are easier to solve than original problemsx. 

For an optimization problem with $I$ control variables $\Vec{x} = (x_1,x_2,...,x_i,...,x_I) $ and  $x_i \in X_{i}$, to minimize objective function $f(\Vec{x})$, AO method is described by
 \begin{equation} \label{eq-ao}
    \resizebox{0.8\hsize}{!}{$ x^{l}_i \leftarrow \argmin\limits_{x_{i} \in X_i} f(\underbrace{x^{l}_1,x^{l}_2,...,x^{l}_{i-1}}_{\textbf{done}}\underbrace{,x_{i},}_{\textbf{current}}\underbrace{x^{l-1}_{i+1},...,x^{l-1}_{I}}_{\textbf{todo}})$},    
\end{equation}
where $x^{l}_i$ is optimized while holding all the other control variables $(x^{l}_1,x^{l}_2,...,x^{l}_{i-1},x^{l-1}_{i+1},...,x^{l-1}_{I})$ constant. Then, equation (\ref{eq-ao}) is repeated by $i=i+1$ until meeting the stop criteria.  
In equation (\ref{eq-ao}), AO simplifies joint optimization problems by optimizing single control variables alternatively while holding other variables unchanged\cite{jcbe}. Each iteration is time-efficient by optimizing one individual variable, which is easily implemented. In addition, it does not require step size parameter tuning and extra storage vectors. However, AO may slow down in the near optimum, and single-variable monotonically decreasing in each iteration can not guarantee the global minimum.

The RIS is often combined with other techniques for joint optimization, such as joint active and passive beamforming, RIS-related resource allocation, RIS-NOMA, and RIS-MEC, leading to coupled control variables and large solution spaces.    
AO is particularly useful in solving such joint optimization problems. For example, 
the RIS-MEC system can be decoupled into RIS phase-shift control sub-problem and task offloading sub-problem, and these two sub-problems will be iteratively optimized to reduce the overall complexity. Joint active and passive beamforming is another example that has been widely investigated, which applies AO to generate BS active beamforming and RIS passive beamforming sub-problems\cite{qingqing,qingqing2,qingqing3}. 

\begin{figure}[!t]
\centering
\includegraphics[width=0.9\linewidth]{Image/fig-AO.jpg}
\caption{The alternating optimization procedure for joint active and passive beamforming.}
\label{fig-AO}
\setlength{\abovecaptionskip}{-2pt} 
\vspace{-10pt}
\end{figure}

\subsection{Block Coordinate Descent}  
Coordinate descent is a very useful method to solve large-scale optimization problems, and BCD is considered a generalized version to improve computation efficiency.
Compared with AO method, each block in the BCD algorithm may include several control variables, enabling dynamic block generation, selection and updating. Therefore, BCD method is more suitable than AO for optimizing a large number of control variables simultaneously, which has been widely applied to RIS-related optimization problems. 

BCD method sequentially minimizes the objective function $F(\Vec{x})$ in each block $x_{i}$ while the other blocks are held fixed. Specifically, it minimizes $ x^{l}_{i} \leftarrow \argmin\limits_{x_i \in \mathscr{X_{i}}} (f(x_{i})+f_{i}(x_i))$ while holding other blocks $x_{1},x_{2},...,x_{i-1},x_{i+1},...x_{I}$ fixed. However, it is worth noting that each block consists of multiple control variables, and the block selection and updating method will affect the BCD performance. An ideal block selection method is expected to maximize the improvement by choosing the blocks that decrease $F(\Vec{x})$ by the largest amount\cite{Julie}. On the other hand, there are many alternatives for the block updating method such as block proximal updating
     \begin{equation} \label{eq-bcd2}
        x^{l}_{i} \leftarrow \argmin\limits_{x_i \in \mathscr{X_{i}}} (f(x_{i})+f_{i}(x_i)+\frac{L_{i}^{l-1}}{2}\|x_{i}-x^{l-1}_{i}\|^2),
    \end{equation}
where $L_{i}^{l-1}>0$. Equation (\ref{eq-bcd2}) is more stable than conventional BCD by including $\frac{L_{i}^{l-1}}{2}\|x_{i}-x^{l-1}_{i}\|^2$\. 
The BCD algorithm is easily deployed with low memory requirements and iteration costs, allowing parallel or distributed implementations. But the block selection may affect the algorithm performance, and block updating is difficult in some cases. 

Similar to AO, BCD is considered as an iteration-based scheme to reduce problem-solving complexity. BCD has been applied to sum-rate maximization \cite{cunhua,jiey,huayan}, user fairness maximization \cite{gang,xianghao2}, and power minimization \cite{zhiyang}. 
As an example, a two-block BCD is used to maximize the sum-rate in \cite{huayan}, in which the first block is for BS active beamforming and the second is for RIS passive beamforming, then these blocks are iteratively optimized. 






\subsection{Majorization-Minimization Method} 

MM is an iterative optimization method that has been applied for RIS control and optimizations. Consider an optimization problem with $\min\limits_{x}  f(x) $ and $x \in \mathscr{X}$, and $f(x)$ is a continuous objective function and $\mathscr{X}$ is a convex closed set. In RIS-related control problems, the $f(x)$ is usually complicated to solve directly due to fractional and logarithmic terms.
As shown in Fig. \ref{fig-mm}, the main idea of the MM algorithm is to construct a surrogate function $g(x)$ that can locally approximate the objective function $f(x)$, e.g., power minimization or sum-rate maximization. $g(x)$ is considered an upper bound of $f(x)$, which is easier to be optimized. Therefore, optimizing $g(x)$ can either improve the objective function value or leave it unchanged with $g(x) \geq f(x)$\cite{ying}.   

Constructing a surrogate function $g(x)$ is the first step of applying the MM algorithm, since $g(x)$ will be optimized directly instead of the original objective $f(x)$. The $g(x)$ construction rules include:\\
    A1): $g(x^{l-1}|x^{l-1})=f(x)$; \quad A2): $g(x|x^{l-1}) \geq f(x)$;\\
    A3): $ g'(x|x^{l-1};d)|_{x=x^{l-1}}=f'(x^{l-1};d)|_{x=x^{l-1}}$;\\
    A4): $g(x|x^{l-1})$ is continuous in $x$ and $x^{l-1}$, \\
where $x^{l-1}$ is the produced point at iteration $l-1$, $g(x|x^{l-1})$ is an approximation function of $f(x)$ at the iteration $l$, $d$ indicates the distance from a point $x$ to a set $\mathscr{X}$ and $d=\inf\limits_{x'\in \mathscr{X}}||(x-x')||$. $f'(x;d)$ is the directional derivative of $f(x)$ in direction $d$.
Assumptions (A1) and (A2) indicate that $g(x|x^{l-1})$ is a tight upper bound of the original objective $f(x)$. It guarantees that optimizing $g(x|x^{l-1})$ can meanwhile find an improved objective value for $f(x)$. 
Note that surrogate function may be defined in various ways, e.g. Jensen's inequality, Convexity inequality, Cauchy–Schwarz inequality.     
Then, the surrogate function $g(x|x^{l-1})$ is iteratively minimized and updated by
$ x^{l} \leftarrow \argmin\limits_{x \in \mathscr{X}} g(x|x^{l-1})$ until convergence. 

MM applies surrogate functions to avoid the complexity of optimizing the objective function directly, transforming non-differentiable problems into smooth optimizations. However, it requires that the surrogate function $g(x)$ must be a global upper bound for $f(x)$, which is very demanding in some cases. As an estimation-based method, MM is considered a low-complexity solution for many RIS-related optimizations, including sum-rate maximization\cite{cunhua}, fairness maximization\cite{menghua,guiz}, secure transmission\cite{xianghao2,huiming2} and so on. For example, in \cite{cunhua}, the RIS phase-shift optimization problem is first converted into a non-convex quadratically constrained quadratic program (QCQP) problem\footnote{A QCQP problem example is given by equation (\ref{eq-qcqp}) in Section \ref{section_sdr}, which is frequently formulated in wireless networks.}, then the MM method is applied to obtain local optimal solutions.  

\begin{figure}[!t]
\centering
\setlength{\abovecaptionskip}{-2pt} 
\includegraphics[width=0.95\linewidth]{Image/fig-MM.jpg}
\caption{MM method for RIS-related optimizations.}
\label{fig-mm}
\vspace{-10pt}
\end{figure}


\subsection{Successive Convex Approximation} 

\begin{figure}[!t]
\centering
\setlength{\abovecaptionskip}{-2pt} 
\includegraphics[width=1.02\linewidth]{Image/fig-SCA.jpg}
\caption{ Using SCA algorithm for RIS-related optimization.}
\label{fig-SCA}
\vspace{-10pt}
\end{figure}

Similar to the MM algorithm, SCA applies a surrogate function $g(x)$ to approximate the original objective function $f(x)$, which is shown in Fig. \ref{fig-SCA}. However, the $g(x)$ in the SCA algorithm does not have to be a tight upper bound for $f(x)$, reducing the complexity of the surrogate function design \cite{palomar}. Therefore, SCA is more flexible and easier to be implemented for RIS-related optimization problems.

The SCA method first constructs a surrogate function $g(x)$, and the assumptions are similar to the MM algorithm:  \\ 
A1): $g(x|x^{l-1})$ is continuous in $\mathscr{X}$;\\
A2): $g(x^{l-1}|x^{l-1})=f(x)$;\\
A3): $g(x)$ is differentiable with $\nabla_{x} g(x|x^{l-1})|_{x=x^{l-1}}=\nabla_{x} f(x)|_{x=x^{l-1}}$. \\
SCA relaxes the upper bound condition for the surrogate function, but $g(x|x^{l-1})$ must be strongly convex in $\mathscr{X}$.
Then, solving the constructed surrogate problem 
$\mathbf{\hat{x}}(x^{l}) \leftarrow \argmin\limits_{x \in \mathscr{X}} g(x|x^{l-1})$,
and smoothing the next point by 
\begin{equation}
    x^{l}=x^{l-1}+\beta^{l-1}(\mathbf{\hat{x}}(x^{l})-x^{l-1}),
\end{equation} 
where $\beta^{l-1}$ is the step size for value updating.
Finally,  $g(x)$ construction and solving are repeated until meeting the convergence criteria.   

The surrogate function $g(x)$ does not have to be a tight upper bound for $f(x)$. Therefore, the step size in each iteration requires dedicated designs to guarantee an accurate approximation. The factor $\beta^{l-1}$ is used to control the $x^{l}$ updating step size. Meanwhile, the MM algorithm updates the whole control variable $x$ at each iteration, but SCA can be naturally implemented in a distributed manner when the constraints are separable. 

Compared with MM, SCA is more frequently applied in RIS-related optimizations due to the relaxed upper bound, e.g., sum-rate maximization in \cite{ming,yuanbin,xidong} and power minimization in \cite{huimei,jianyue}. For instance, the non-convex BS transmit power constraint in \cite{huimei} is replaced by a first order Taylor approximation to apply SCA algorithm, and then the obtained BS beamforming vector and RIS phase shifts are alternatively optimized for transmit power minimization.




\subsection{Semidefinite Relaxation}  
\label{section_sdr}

Many RIS-related signal processing problems can be described by QCQP formulations, and SDR is an efficient solution to solve QCQP problems\cite{shuzhong}. The QCQP problem is defined by
\begin{equation}\label{eq-qcqp}
\begin{aligned}
\min\limits_{x\in \mathscr{X}}  \qquad & x^{T}C x  \\
 \text{s.t.}  \qquad & x^{T}D_{i} x \geq b_{i}, i=1,2,3,,,n,  
\end{aligned}
\end{equation} 
where the "$\geq$" in the constraint can also be replaced by "$\leq$". Note that $x^{T}C x$ produces an $1\times 1$ matrix, and therefore $x^{T}Cx=Cx^{T}x=Tr(Cx^{T}x)$. Similarly, $x^{T}D_{i}x=D_{i}x^{T}x=Tr(D_{i}x^{T}x)$ is achieved. By introducing $X=xx^T$, then  
\begin{equation}\label{eq-qcqp2}
\begin{aligned}
\min\limits_{x\in \mathscr{X}}  \qquad & Tr(CX)  \\
 \text{s.t.}  \qquad & Tr(DX_{i}) \geq b_{i}, i=1,2,3,,,I,  \\
 \qquad & X\succeq 0,     \\
  \qquad & rank(X)=1,   
\end{aligned}
\end{equation}
where $X\succeq 0$ indicates that $X$ is positive semidefinite with $X=xx^T$. Then, the non-convex constraint $rank(X)=1$ is relaxed and achieve 
\begin{equation}\label{eq-qcqp3}
\begin{aligned}
\min\limits_{x\in \mathscr{X}}  \qquad & Tr(CX)    \\
 \text{s.t.}  \qquad & Tr(DX_{i}) \geq b_{i}, i=1,2,3,,,I,   \\
 \qquad & X\succeq 0.  
\end{aligned}
\end{equation}

Equation (\ref{eq-qcqp3}) is a SDR of (\ref{eq-qcqp2}), which can be efficiently solved by semidefinite programming (SDP)\cite{zhiquan}. However, the following issue is how to transform a global optimal solution $\hat{X}$ in equation (\ref{eq-qcqp3}) into a feasible solution $\hat{x}$ in equation (\ref{eq-qcqp}). An ideal solution is that $\hat{X}$ is of rank-one, and $\hat{x}$ is easily obtained by solving $\hat{X}=\hat{x}\hat{x}^T$. Otherwise, if $rank(\hat{x})>1$, rank-one approximation may be used to obtain a suboptimal solution $\widetilde{x}$ for the problem (\ref{eq-qcqp}). 

In addition, there are multiple methods to find a feasible $\widetilde{x}$ from $\hat{X}$, leading to various solution qualities. For instance, Mu \textit{et al.} proposed a penalty-based method to relax the rank-one constraint, finding a sub-optimal solution by introducing penalties if $rank(\hat{x})>1$ \cite{mu2021simultaneously}. SDR has been very generally applied to RIS-related optimization problems, since the $rank(x)=1$ is frequently formulated for phase control. Specifically, the RIS phase shift constraint $|\theta_n|=1$ is non-convex, and then $|\theta_n|=1$ is transformed and relaxed as equation (\ref{eq-qcqp2}) to (\ref{eq-qcqp3}), which can be efficiently solved by standard optimization tools. SDR has been used for sum-rate maximization \cite{boya, peilan, ni2021resource}, power minimization \cite{guizhou2,huimei,guiz3,jianyue}, fairness maximization \cite{gang,hailiang,zaid,menghua}, and secure transmission \cite{zheng,wei,xianghao,biqian,zijie}. 



\subsection{Second-order Cone Programming}  
SOCP is another method that is used to efficiently solve optimization problems in wireless networks, especially for QCQP and fractional problems. It enables affine combinations of variables to be constrained inside a second-order cone
\begin{equation}\label{eq-scop}
\begin{aligned}
\min\limits_{x\in \mathscr{X}}  \qquad & C^{T} x   \\
 \text{s.t.}  \qquad & ||A_{i}x+b_{i}||\leq c_{i}^{T}x+d_{i}, i=1,2,3,,,I, 
\end{aligned}
\end{equation} 
where $A \in \mathbb{R}^{n_i\times n}$, $b_i \in \mathbb{R}^{n_i}$, $c_i \in \mathbb{R}^n$, and $d_i \in \mathbb{R}$. The $x$ in equation (\ref{eq-scop}) may be RIS phase shifts, BS beamforming vectors, and so on, which depends on specific application scenarios. Consider the inverse image of the unit second-order cone with an affine mapping
\begin{equation} \label{eq-scop3}
 ||A_{i}x+b_{i}||\leq c_{i}^{T}x+d_{i} \leftrightarrow 
 \left[ \begin{array}{c}  
    A_{i} \\  
    c_{i}^{T} \\ 
  \end{array} \right] x +  
\left[ \begin{array}{c}  
    b_{i} \\  
    d_{i} \\ 
  \end{array}
\right] \in \mathscr{C}_{n_i+1}.       
\end{equation}
Therefore, SOCP is a convex optimization problem with a convex objective function and convex constraints. Equations (\ref{eq-scop3}) indicate the core properties of SOCP problems, and hence many problems are converted into SOCPs and solved efficiently\cite{miguso}. 

For instance, sum and fractional problems are frequently defined in RIS-related problems to maximize the sum-rate or total throughput regarding the SINR
\begin{equation}\label{eq-scop7}
\begin{aligned}
\min\limits_{x\in \mathscr{X}}  \qquad & \sum_{i=1}^{I}\frac{||C_{i}^T+D_{i}||^2}{A_{i}^Tx+B_{i}}  \\
 \text{s.t.}  \qquad & A_{i}^Tx+B_{i} \geq 0, i=1,2,3,,,I, 
\end{aligned}
\end{equation} 
which is converted into a SOCP by
\begin{equation}\label{eq-scop8}
\begin{aligned}
\min\limits_{x\in \mathscr{X}}  \qquad & \sum_{i=1}^{I}t_{i} \\
 \text{s.t.}  \qquad & (C_{i}^T+D_{i})^T(C_{i}^T+D_{i}) \leq t_{i}(A_{i}^Tx+B_{i}),    \\
  \qquad & A_{i}^Tx+B_{i} > 0, i=1,2,3,,,I.  
\end{aligned}
\end{equation} 

SOCP can be efficiently solved by the interior point method. Meanwhile, SOCP is less general than SDP since equation (\ref{eq-scop}) may be transformed into an SDP problem. However, the complexity of solving SOCP is $O(n^2\sum_{i}{n_i})$, while the complexity for SDP is $O(n^2\sum_{i}{n_i}^2)$\cite{nest}. Such complexity difference is crucial for large-dimension problems. 

Finally, to apply SOCP for RIS-aided optimizations, the first step is to utilize AO or BCD scheme to decouple the control variables into multiple sub-problems, e.g., BS precoding matrix and RIS passive beamforming\cite{guiz,yiqing,menghua}, coordinated transmit beamforming and RIS passive beamforming\cite{hailiang}. For example, the max-min data rate problem in \cite{menghua} is decoupled into SOCP-based BS beamforming and SDR-based RIS phase-shift control, and the data rate maximization problem in \cite{jiey} is converted into a SOCP-based BS active beamforming and SDR-based RIS passive beamforming.         



\subsection{Fractional Programming} 
FP refers to optimization problems involving ratios or fractional terms.
FP is particularly useful for wireless network optimizations due to the fractional terms in communication systems, especially for SINR and energy efficiency\cite{zappone}. 

Consider a single-ratio FP problem to maximize the SINR of single UE by $\max\limits_{x\in \mathscr{X}} {f(x)}/{g(x)}$, where $f(x)$ is the signal strength and $g(x)$ is the interference and noise. There are many classic methods to solve FP problems, such as Charnes-Cooper transform and Dinkelbach’s transform\cite{Dinke}. Dinkelbach’s method reformulates the problem into $\max\limits_{x\in \mathscr{X},y\in \mathbb{R}}  f(x)-yg(x)$, where $y$ is the auxiliary variable that is updated iteratively $y^{(l+1)}={f(x)^l}/{g(x)^l}$, 
and $l$ is the iteration number. Then, alternatively updating $y$ and $x$ will lead to a converged solution with non-decreasing $y^l$. However, instead of the single-ratio problem, sum-ratio FP problems are more frequently involved in wireless networks, i.e., maximizing sum-rate or total channel capacity as 
$\max\limits_{x\in \mathscr{X}}  \ \sum_{i=1}^{I} f_{i}(x)/g_{i}(x) $.

However, classic methods can not be directly generalized to sum-ratio cases, since maximizing single ratios cannot guarantee the convergence and maximization for sum-ratio cases. An equivalent transform proposed by \cite{shenk} is 
\begin{equation}\label{eq-fp5}
\max\limits_{x\in \mathscr{X},y\in \mathbb{R}}  \quad 2yf(x)^{0.5}-y^{2}g(x),  
\end{equation} 
which can be readily converted into sum-ratio problems.
In addition, equation (\ref{eq-fp5}) is further generalized to sum-ratio problems as
\begin{equation}\label{eq-fp7}
\max\limits_{x\in \mathscr{X},y\in \mathbb{R}}  \quad \sum_{i=1}^{I}F_{i}(2y_{i}C_{i}(x)^{0.5}-y_{i}^{2}D_{i}(x)),  
\end{equation}
where $F_{i}$ is a non-decreasing function. Equation (\ref{eq-fp7}) is particularly useful given the frequently used term $\sum log(1+SINR)$ in wireless communications. 

Moreover, RIS-related max-min fairness problems are formulated as
$\max\limits_{x\in \mathscr{X}}  \  \min\limits_{1\leq i \leq I} \ {f_{i}(x)}/{g_{i}(x)}$, 
which is reformulated by
\begin{equation}\label{eq-fp9}
\begin{aligned}
\max\limits_{x\in \mathscr{X}, y,z\in \mathbb{R}}  &  \quad z \\
 \text{s.t.}  \quad & 2y_{i}f_{i}(x)^{0.5}-y_{i}^{2}g_{i}(x) \geq z; i=1,2,3,,,I.
\end{aligned}
\end{equation}

FP method significantly reduces the optimization complexity by decoupling the fractional terms. Therefore, it has been widely used in wireless network optimizations, including power control, beamforming, energy efficiency and so on. 
FP method can significantly lower the problem-solving complexity by eliminating fractional items. This transformation is very useful for RIS-related optimization problems, especially considering that RIS phase shifts will affect the received signal strength and interference simultaneously.
For instance, FP is applied to maximize the sum-rate of RIS-aided wireless networks in \cite{huayan,shuaiqi}. In particular, consider $f(x)$ is the received signal strength with RIS phase shifts and $g(x)$ represents interference and noise, then FP can decouple $f(x)$ and $g(x)$ in the SINR terms as indicated by equation (\ref{eq-fp7}).




\begin{figure}[!t]
\centering
\setlength{\abovecaptionskip}{-3pt} 
\includegraphics[width=0.8\linewidth]{Image/fig-bb.jpg}
\caption{Using BnB for RIS control with discrete phase shifts.}
\label{fig-bb}
\vspace{-15pt}
\end{figure}








\begin{table*}[!t]
\caption{Summary of model-based optimization algorithms for RIS-aided wireless networks }
\centering
\setstretch{1.05}
\small
\resizebox{1\textwidth}{!}{%
\begin{tabular}{|m{1cm}<{\centering}|m{3.7cm}<{\centering}|m{3.6cm}<{\centering}|m{3cm}<{\centering}|m{3cm}<{\centering}|m{4.6cm}<{\centering}|}
\hline 
Methods &  Main features   &     Advantage    &  Drawbacks    &  Difficulties   &  Application \qquad \qquad \qquad scenarios  \\
\hline
AO & Decoupling the joint optimization into multiple sub-problems, and alternatively optimizing each sub-problem.   &  The problem-solving complexity is greatly reduced. Each sub-problem may be easier to solve.  & Iterative optimization may lead to sub-optimal results; the convergence must be proved. &  The complexity is high when each sub-problem is still complicated.  & AO is the most widely applied optimization scheme for RIS-related optimization; it applies other algorithms to solve the sub-problems\cite{chongwen}.  \\
\hline
BCD &  The control variables are divided into multiple blocks. Minimizing one block in each iteration and keeping other blocks fixed.  & Cheap iteration costs; low memory requirements; potential for parallel implementation   &   Block selection may affect the BCD performance, and block updating is difficult in some cases.  &  Block selection and updating methods are complicated.  & BCD is deployed as an alternative optimization scheme to reduce joint optimization complexity, e.g., sum-rate maximization \cite{cunhua,jiey,huayan}, power minimization \cite{zhiyang}, user fairness \cite{gang,xianghao2}     \\
\hline
MM & Iteratively constructing and optimizing an upper bound surrogate function that can locally approximate objective functions. &  Avoiding the complexity of optimizing non-convex objective functions directly.  &  The surrogate function must be a strict tight upper bound for objective functions, which is hard to achieve in practice.   &  The surrogate function must follow the shape of objective functions and meanwhile be easy to optimize.  &  A low-complexity solution for many RIS-aided optimizations, i.e., sum-rate maximization\cite{cunhua}, fairness maximization\cite{menghua,guiz}, secure transmission\cite{xianghao2,huiming2}.    \\
\hline
SCA & Constructing and optimizing surrogate functions iteratively to estimate the objective function.  & Low computational complexity; the tight upper bound is not required for the surrogate function; naturally implemented in a distributed manner.  &  The step size selection is critical for an accurate approximation.  &  Surrogate function and step size selection.  &  SCA is more frequently used in RIS-related optimizations than MM, e.g., sum-rate maximization\cite{ming,yuanbin,xidong}, power minimization\cite{huimei,jianyue}.     \\
\hline
SDR & SDR is used to solve QCQP problems by relaxing the rank constraint. Then the reformulated problem is efficiently solved by SDP. & Given the objective problem, SDR is easily implemented without extra parameters or settings. & Approximation is required if the relaxed solution is not rank one.  &  The reformulated problem is complicated if the achieved solution is not rank one. & SDR is particularly useful in RIS-related optimization problems, since the $rank(x)=1$ is frequently formulated for phase control \cite{boya} \cite{peilan,qingqing,guizhou2}.         \\
\hline
SOCP & SOCP utilizes the property of the second-order cone, and many problems are reformulated into SOCP, which is much easier to be solved.  & SOCP can be efficiently solved by many existing algorithms. It has a lower complexity $O(n^2\sum_{i}{n_i})$ than SDR.  &  Problem reformulation into SOCP is complicated.  &  The main difficulty lies in how to reformulate the original problem into SOCP.  & SOCP has been used for power minimization \cite{yiqing} and user fairness \cite{guiz,hailiang,menghua}.     \\
\hline
FP & FP refers to optimization problems that involve fractional terms, which is very useful for wireless communications considering the form of SINR and energy efficiency.  & FP is easily implemented without extra parameters or problem formulation requirements.  &  The reformulated problems generally require iterative optimization to approximate the solution of original FP problems.   & Compared with single-ratio problems, wireless communications are more related to sum-ratio problems, which are more complicated to solve.   & FP is widely applied in RIS-aided wireless network optimizations due to the form of SINR and energy efficiency \cite{huayan,jianyue,chongwen,linsong}.     \\
\hline
BnB & BnB is mainly designed for combinational optimization problems. It applies a tree to enumerate all possible subsets and sub-problems.  &  Lower complexity compared with direct optimizations. The solution quality is controlled by deploying customized search, branching, and pruning rules.  &  The algorithm is slow when constantly searching or branching in the worst case.   &  The algorithm performance relies on the searching and pruning method, which is hard to select in some cases.  &  BnB is mainly applied for discrete and combinational optimization problems, i.e., RIS on/off and discrete phase shift\cite{shiqi,qingqing2,boya}.      \\
\hline
\end{tabular}}
\label{tab-comparison}
\vspace{-13pt}
\end{table*}



\subsection{Branch-and-Bound } 

BnB is a classic scheme for combinatorial and discrete optimization problems\cite{clau}. To minimize $f(x)$ with $x\in \mathscr{X}$, BnB applies a tree scheme to enumerate all possible subsets $X_{i} \subseteq \mathscr{X}$, and each subset $X_{i}$ indicates a sub-problem $f_{i}(x)$. Solving sub-problems $f_{i}(x)$ will generate and prune branches based on the estimated lower and upper bounds.  

BnB algorithm consists of three basic operations: branching, bounding, and pruning.
Consider a non-linear integer programming problem, and BnB scheme is summarized as Fig. \ref{fig-bb}. It produces a series of sub-problems $f_{i}(x)$ that are equivalent to the original $f(x)$, which is much more efficient than brute-force enumeration.         
BnB algorithm provides an alternative solution for challenging problems that cannot be solved directly. An important advantage is that the quality of the solution is controlled by customized searching, branching, and pruning rules. On the other hand, it can be slow if constantly searching and branching, learning to exponential worst-case performance. 

BnB is mainly applied for RIS control with discrete phase shift, including sum-rate maximization \cite{boya}, power minimization \cite{qingqing2}, and max-min SINR \cite{shiqi}. The main reason is that the problem formulations are usually MINLP problems, which are NP-hard and intractable. As shown in Fig. \ref{fig-bb}, the MINLP is converted into an 0-1 integer linear programming using the special ordered set of type 1 (SOS1) transformation \cite{qingqing2} and reformulation-linearization \cite{shiqi}. 



\subsection{Analyses and Discussion}

Table \ref{tab-comparison} summarizes model-based algorithms for RIS-aided wireless networks, including main features, advantages, disadvantages, difficulties, and application scenarios. 

Firstly, considering the high complexity of RIS-related optimization, AO is regarded as the primary scheme to decouple the joint optimization problem into several sub-problems. Then, each sub-problem is alternatively solved by using different algorithms, e.g., SCA, SDR, and BnB. BCD algorithm applies the same alternative optimization scheme, but one block may include multiple variables, which is more efficient for problems with a large number of control variables. 

MM and SCA are two estimation-based algorithms that avoid the complexity of direct optimizations. Compared with MM, SCA relaxes the upper bound requirement for surrogate functions, but the step size selection may affect the solution quality. MM and SCA are usually considered low-complexity solutions for RIS-aided wireless network optimizations. 

SDR and FP are usually combined with other techniques for optimizations. SDR is mainly used to relax the RIS phases constraint, while FP can decouple the numerator and denominator for SINR and energy efficiency. These two techniques reformulate the original problems into low-complexity or even convex forms, then other techniques can be applied. SOCP takes advantage of the property of the second-order cone, which is efficiently solved by many existing methods. But the main difficulty is how to transform the problem with logarithm and fractional terms into a second-order cone. BnB is mainly designed for combinational and discrete optimization problems, e.g., RIS control with discrete phase shifts and elements on/off.

Finally, it is worth noting that these algorithms are not independent, and multiple algorithms are usually combined for transformation and optimizations. The main objective of Table \ref{tab-comparison} is to analyze the feasibility of these problems for various RIS-related optimizations, and the most efficient solution for specific scenarios requires case-by-case analyses. 




\section{Heuristic Algorithms for RIS-aided Wireless Networks }
\label{sec-heu}


As presented in Section \ref{sec-model}, model-based algorithms have specific requirements for problem formulations, especially for convexity and continuity. 
Meanwhile, the large number of RIS elements, dynamic channel conditions, and various CSI levels further contribute to the overall complexity.  
Therefore, transformations and relaxations are required to convert the original problem into specific forms. 
Moreover, these transformations are usually problem-specific, requiring case-by-case analyses and dedicated design. 

By contrast, heuristic algorithms have fewer requirements for objective functions and constraints, which will significantly reduce the complexity. 
Compared with model-based methods, heuristic algorithms are usually considered low-complexity solutions. In the following, we will introduce four heuristic algorithms, including CCP, meta-heuristic algorithms, greedy algorithms, and matching-based algorithms. 


\subsection{Convex-concave Procedure} 



The CCP algorithm uses the local heuristic to solve difference of convex (DC) problems, which is considered a low-complexity solution for complicated wireless network optimization\footnote{The main reason that CCP is considered a heuristic algorithm is that it applies a simple heuristic rule for optimization, which is iteratively finding two points with the same tangent vectors\cite{lipp}. Hence, CCP fits well with our defined classifications of heuristic algorithms.}. 

DC problems are frequently formulated in many fields, representing many scenarios that cannot be solved in polynomial time. The DC problem is defined as
\begin{equation}\label{eq-ccp}
\begin{aligned}
\max\limits_{x\in \mathscr{X}}  &  \quad f_{0}(x)-g_{0}(x) \\
 \text{s.t.}  \quad & f_{i}(x)-g_{i}(x) \leq 0; \ i=1,2,3,,,I,
\end{aligned}
\end{equation}
where $f(x)$ and $g(x)$ are both convex. DC problems are usually non-convex unless $g_{i}(x)$ are affine, which is generally hard to solve. 


As shown in Fig. \ref{fig-ccp}, the core idea of CCP is to find $x^{l+1}$ in the $l+1$ iteration that satisfies $\nabla_x f(x^{l+1})=\nabla_x g(x^{l})$, indicating a point on $f(x)$ that has the same tangent with $g(x^{l})$\cite{yulie}. 
The CCP algorithm will first form 
    \begin{equation} \label{eq-ccp112}
        \overline{g}_{i}(x|x^l)=g_{i}(x^{l})+ \nabla_x g(x)(x-x^{l}),\ i=0,1,2,3,,,I. 
    \end{equation}
Then it solves the following problem to get $x^{l+1}$
    \begin{equation}\label{eq-ccp2}
    \begin{aligned}
    \max\limits_{x\in \mathscr{X}}  &  \quad f_{0}(x)-\overline{g}_{0}(x|x^l)\\
    \text{s.t.}  \quad & f_{i}(x)-\overline{g}_{i}(x|x^l) \leq 0; \ i=1,2,3,,,I.
    \end{aligned}
    \end{equation}
Equation (\ref{eq-ccp2}) is equivalent to  $\nabla_x f(x^{l+1})=\nabla_x g(x^{l})$ by deriving the objective function, and equations (\ref{eq-ccp112}) and (\ref{eq-ccp2}) are iteratively repeated until reaching the stop criteria. 
CCP algorithm does not require a dedicated step size design, and the main reason is that the estimator $f_{0}(x)-\overline{g}_{0}(x|x^l)$ is global. It retains all the information from the convex component $f(x)$ and only linearizes the concave portion $g(x)$.



\begin{figure}[!t]
\centering
\includegraphics[width=1\linewidth]{Image/fig-ccp.jpg}
\caption{Convex concave procedure for RIS-related optimization.}
\label{fig-ccp}
\setlength{\abovecaptionskip}{-2pt} 
\vspace{-10pt}
\end{figure}



Moreover, there are multiple extensions of the CCP algorithm.  For instance, the penalty CCP includes a penalty term for violations, which removes the requirements for feasible initial points. The RIS phase shift optimization problem is reformulated as \cite{cunh}
\begin{equation}\label{eq-ccp11}
\begin{aligned}
\max\limits_{\theta, \tau >0}  &  \quad f(\theta)-\tau^{l} \sum_{i=1}^{2I} v_{i} \\
\text{s.t.}  \quad & {g}(\theta) \geq  D,\\
\quad & |\theta_i^l|^2 - 2 Re\{\theta^*_i\theta_i^l\} \leq v_i-1,\\
\quad & |\theta_i|^2 \leq 1+v_{i+I}, \ i=1,2,3,,,I,
\end{aligned}
\end{equation}  
where $\tau^{l} \sum_{i=1}^{2I} v_{i}$ is the penalty term, $v_i$ are slack variables, and $\tau^{l}$ is a coefficient that will decline in each iteration for convergence. After some transformations, problem (\ref{eq-ccp11}) can be solved by using the CVX toolbox, and the detailed procedure is included in \cite{cunh}. 

CCP has been applied in \cite{guiz3,yuanbin,chen2021qos,niu2021simultaneous,xu2022reconfigurable} for controlling RIS phase shifts. In these works, the joint optimization problem is first decoupled into multiple sub-problems using BCD or AO, then penalty CCP is used to solve the RIS phase shifts sub-problem. The main motivation is the high complexity of solving non-convex RIS control problems. For example, 
the sum-rate maximization problem in \cite{chen2021qos} is converted into three sub-problems: joint optimization of the transmit power and spectrum sharing, SDR-based multi-user detection, and CCP-based RIS phase shifts. However, note that CCP is a heuristic algorithm that will find a locally optimum solution, and the initial point $x^0$ may affect the final output. In particular, there may exist multiple locally optimal solutions, and CCP can easily get stuck in a sub-optimal one.





\subsection{Meta-heuristic Algorithms}

One of the main difficulties of controlling RISs is the large number of RIS elements, leading to huge solution spaces. Therefore, it is hard to achieve exact solutions by finding a closed-form expression, and hence model-based approximation algorithms such as SCA and MM are applied. 
However, these methods have stringent requirements for objectives and constraints, especially for convexity, continuity, and differentiability. By contrast, meta-heuristic algorithms can search significantly large solution spaces with few or no additional requirements on problem forms \cite{bozorg2017meta}. 
It usually contains intelligent policies to guide the heuristic exploration, producing high-quality solutions efficiently. 
Meta-heuristic algorithms have been extensively developed, e.g., genetic algorithm (GA), particle swarm algorithm (PSO), ant colony optimization, simulated annealing, and tabu search \cite{beheshti2013review}. 

Fig. \ref{fig-pso} shows the steps of using population-based meta-heuristic algorithms for RIS phase-shift design. The first step is to initialize the algorithm parameters such as population numbers and crossover rate in a genetic algorithm.
Then, the algorithm will produce initial individuals, which indicates various RIS phase-shift designs. The objective function is converted into a fitness function, e.g., the sum-rate or energy efficiency. After that, the algorithm will constantly search for better solutions using heuristic rules iteratively, such as evolution strategy in a genetic algorithm, and particle movement for PSO. Finally, the heuristic exploration will stop if the fitness function values converge or reach maximum iteration numbers.

Compared with model-based methods, the main advantage is that meta-heuristic algorithms can easily adapt to both continuous and discrete RIS phase shifts without relaxation and transformation. PSO and GA are used for RIS phase shifts in \cite{dai2021reconfigurable} and \cite{zhi2022power} to maximize the data rate. Statistical CSI is investigated in \cite{zhi2021statistical} to obtain a closed-form expression of the uplink ergodic data rate, then GA is deployed for phase control to maximize the data rate. In addition, Tabu search is applied to irregular RIS to decide the element design in \cite{ruoc}. 

The simulations in \cite{dai2021reconfigurable,zhi2022power,zhi2021statistical,ruoc} show that meta-heuristic algorithms can significantly reduce the optimization complexity, especially for MINLP problems. 
However, it may be trapped in local optima, and the algorithm performance relies on the parameter settings. For example, the phase shifts of hundreds of RIS elements require a large number of populations in GA, leading to high exploration costs. By contrast, reducing the population numbers may lower the probability of finding optimal solutions.         

\begin{figure}[!t]
\centering
\includegraphics[width=0.7\linewidth]{Image/fig-pso.jpg}
\caption{Population-based meta-heuristic algorithms for RIS phase-shift control.}
\label{fig-pso}
\setlength{\abovecaptionskip}{-2pt} 
\vspace{-10pt}
\end{figure}






\subsection{Greedy Algorithms} 
Most former algorithms are designed to find global optima of the objective function. However, many problems are NP-hard and non-convex, and the solutions are usually problem-specific with a series of transformations. To this end, greedy algorithms are proposed as low-complexity alternatives. In particular, greedy algorithms refer to the problem-solving heuristic that makes locally optimal decisions at each stage regardless of global optima\cite{martins2006metaheuristics}.
Consider an minimization problem $\min f(\Vec{x})$, and the control variables $\Vec{x}$ include $\Vec{x}=\{x_1,x_2,...,x_i,...,X_I\}$. 
At each stage, the greedy algorithm will optimize only one control variable $x_i$ by $\hat{x_i}=\argmin_{x_i\in X_i} f(\Vec{x})$, while holding the rest of variables unchanged. Then it moves to the next stage until $i=I$.

RIS elements' on/off control is a non-convex problem with discrete constraints. It may be solved by relaxing the integer constraint $ \chi \in \{0,1\}$ into $0\leq \chi \leq 1$, but the reformulated problem can still be complicated. 
A low-complexity solution is a greedy element-by-element control. Specifically, it evaluates the on/off decision of one RIS element at each stage by observing the changes in objective functions. If the performance is improved by achieving a higher sum-rate and lower power consumption, then the on/off status will be updated\cite{zhaohui}. Similarly, as shown in Fig. \ref{fig-greedy}, this greedy scheme can also be applied to control RIS phase shifts, indicating that one element is optimized at a time by observing the improvement of objective functions, e.g., achieving higher data rate or energy efficiency. Then, the next RIS element is optimized sequentially\cite{rivera,Atapattu,tewes}. 

In addition, greedy algorithms are used to relax the constraints. For example, the RIS phase shift is allowed to violate the stringent constraints in \cite{angli}, then the achieved objective values are compared with the theoretical optimal results to find a feasible solution. Greedy schemes may be combined with AO to handle problems with multiple sub-objectives. For instance, a greedy scheme is applied in \cite{muham} to maximize the served users by controlling RIS phase shifts, then it schedules the users to minimize the age of information.  
The main advantage of the greedy algorithm is the low complexity by decoupling the joint optimization into multiple stages. However, instead of global optima, it can only achieve local optima. The simple greedy policy means that there is no guarantee for the algorithm's performance, which may lead to poor output in some cases.

\begin{figure}[!t]
\centering
\includegraphics[width=0.85\linewidth]{Image/fig-greedy.jpg}
\caption{Greedy method for RIS phase shift control.}
\label{fig-greedy}
\setlength{\abovecaptionskip}{-2pt} 
\vspace{-10pt}
\end{figure}





\subsection{Matching Theory-based Methods} 
The matching theory is useful for optimizing resource allocation problems, i.e., subcarrier assignment, user-BS and user-RIS association, and mode selection. These problems are formulated as MINLPs, and a possible solution is to relax the zero-one constraints and reformulate the problem. A linear conic relaxation method is proposed in \cite{zhang2021joint} for the user-RIS association, but it further requires SDP to solve the relaxed formulation, leading to high computational complexity. In addition, the problem becomes even more complicated when RIS on/off control and phase shifts are involved. Consequently, the primary motivation for applying matching theory is to achieve low-complexity solutions efficiently. 

Consider the most widely applied many-to-one matching problem with two finite and disjoint sets of players $\mathcal{U}$ and $\mathcal{B}$. $\mathcal{U}$ represents users, and $\mathcal{B}$ may be BS, RIS or subchannels.
\begin{definition}
The considered many-to-one matching problem is defined by\\
(a) Matching relationship function $f:=\mathcal{U}\times\mathcal{B}$ with $u \in \mathcal{U}$ and $b \in \mathcal{B}$, e.g., the many-to-one association relationship between multiple users and one BS;\\
(b) $|f(u)|=1$ with $\forall u \in \mathcal{U}$, indicating that one $u$ can only be matched with at most one $b$ in many-to-one matching problem, e.g., one user can be associated with at most one BS. \\
(c) $|f^{-1}(b)|\geq K$ with $\forall b \in \mathcal{B}$, which means that $b$ has a capacity limit for the connection with $u$. For example, one BS has a maximum service capability for users.\\ 
(d) $b=f(u)$ $\leftrightarrow$ $u=f^{-1}(b)$. This means the matching is bidirectional and mutual. 
\end{definition}
To describe the exchange operation between different matching, the swap matching is considered 
\begin{definition}
Given $u\in f^{-1}(b)$ and $u' \in f^{-1}(b')$ with $u,u' \in \mathcal{U} $ and  $b,b' \in \mathcal{B} $, the swap matching is defined by $f_{u,b,u',b'}=\{f \backslash \{ (u, b)(u',b') \} \} \cup  \{ (u, b')(u',b) \}$. 
\end{definition} 
Swap matching allows $u$ and $u'$ to exchange their matched $b$ and $b'$, while other players remain unchanged. For instance, two users can exchange their associated BSs without changing other association pairs.
\begin{definition} \label{defi-block}
$u$ and $u'$ become a swap blocking pair if and only if\\
(a) For all players in $\{ u,b,u',b' \}$, $F(f_{u,b,u',b'}) \geq F(f)$, where $F$ is the utility function of players. This means that the utility functions of all involved players will not decrease. \\
(b) At least one player in $\{ u,b,u',b' \}$ has $F(f_{u,b,u',b'}) > $ $ F(f)$, indicating that at least one player's utility is improved, e.g., at least one user achieves higher channel capacity or data rate by switching pairs.
\end{definition}
Definition \ref{defi-block} shows that the overall utility can be improved by finding swap matching pairs. Then the stable matching is defined by
\begin{definition} \label{defi-stable}
The matching relationship between two sets $\mathcal{U}$ and $\mathcal{B}$ is two-sided exchange-stable if there is no swap blocking pairs. This means that the overall utility such as channel capacity or sum-rate cannot be improved by switching user-BS associations.      
\end{definition}


\begin{table*}[!t]
\caption{Summary of heuristic algorithms for RIS-aided wireless networks}
\centering
\small
\setstretch{1.1}
\resizebox{1\textwidth}{!}{%
\begin{tabular}{|m{1.1cm}<{\centering}|m{3cm}<{\centering}|m{4.1cm}<{\centering}|m{3cm}<{\centering}|m{3.8cm}<{\centering}|m{4.4cm}<{\centering}|}
\hline 
Methods &  Main features   &     Advantage    &  Drawbacks    &  Difficulties   &  Application \qquad \qquad \qquad scenarios  \\
\hline
CCP & CCP aims to obtain local optima of DC problems by iteratively finding points with the same tangent values.  &  CCP algorithm does not require a dedicated step size design, and it retains all of the information from the convex
component and only linearizes the concave portion.  & The selection of initial points may affect the final results, and hence it requires an initialization method. &  The problem has to be reformulated into the DC form; initial points may need to be selected several times due to local optima.  & Penalty CPP is used for imperfect CSI in \cite{guiz3}, and statistical CSI in \cite{yuanbin}. Other applications include non-convex RIS control problems in \cite{chen2021qos,niu2021simultaneous,xu2022reconfigurable}. \\
\hline
Meta-heuristic algorithms & Meta-heuristic algorithms apply intelligent policies to guide the heuristic exploration iteratively, producing high-quality solutions efficiently. &  Meta-heuristic algorithms can search huge solution spaces with few or no additional assumptions.  & It can be trapped in local optima, and these algorithms require many iterations. The algorithm performance is sensitive to parameter selections.  & Selecting the best parameters is complicated in meta-heuristic algorithms. For instance, crossover probability in GA and inertia weight in PSO may decide the algorithm performance.   & Meta-heuristic algorithms are mainly deployed for RIS phase-shift control, including GA for rate maximization in \cite{zhi2022power}, PSO for phase-shift optimization in \cite{dai2021reconfigurable, zhi2021statistical}, tabu search for irregular RIS in \cite{ruoc}.     \\
\hline
Greedy algorithms &  Instead of global optima, greedy algorithms make locally optimal decisions at each stage of solving the problem.    & Greedy algorithms can greatly reduce the complexity by decoupling the original 
 problem into multiple stages.   &  It may present poor global performance, since the local optima in one stage may lead to bad results for the next stage.  &  Finding the trade-off between low complexity and good algorithm performance is critical to using the greedy heuristic.   & Greedy algorithms are used for RIS phase control in \cite{rivera,Atapattu,tewes} and on/off control in \cite{zhaohui} as low-complexity solutions, providing low-complexity alternative solutions for NP-hard problems.  \\
 \hline
Matching theory-based method &  Matching-based method is designed to solve matching or association problems with two sides of players.  &  Compared with direct optimization methods, matching-based methods have lower complexity for large-scale problems.  &  Matching-based method may require exhaustive searches to find matching pairs.  &  Defining the utility function for two sides of players with peer effects is difficult.  & Matching theory is mainly applied to resource allocation and association problems in RIS-aided networks, e.g., channel assignment and user-BS-RIS association\cite{wu2022resource, zuo2020resource}.    \\
 \hline
\end{tabular}}
\label{tab-heuristic}
\vspace{0pt}
\end{table*}



\begin{figure}[!t]
\centering
\includegraphics[width=0.9\linewidth]{Image/fig-match.jpg}
\caption{Matching theory applications in RIS-aided wireless networks.}
\label{fig-match}
\setlength{\abovecaptionskip}{-2pt} 
\vspace{-10pt}
\end{figure}

%The existence of stable matchings has been proved in \cite{bodine2011peer}. 
Definition \ref{defi-stable} is very useful in matching theory, since it provides locally optimal criteria, and it is easily achieved by searching and eliminating all the swap blocking pairs. Fig. \ref{fig-match} presents the applications of matching theory in RIS-aided wireless networks, including D2D-user pairing, user-BS-RIS association, channel assignment, etc. It shows that matching theory provides an efficient solution for overcoming these NP-hard problems. For instance, a RIS-aided maritime communication system is investigated in \cite{cao2022joint}, in which many-to-one matching was applied for the joint mode selection and power control of BSs. In RIS-assisted NOMA system, many-to-one matching is used for channel assignment \cite{wu2022resource, zuo2020resource} and user clustering \cite{zhang2020joint}, while many-to-one and many-to-many matching are jointly considered in \cite{ni2021resource} for the UE association and channel assignment. Moreover, matching theory is applied in \cite{el2022latency} for edge computation offloading in RIS-aided networks, and a deferred acceptance matching game is formulated in \cite{taghavi2021user} for user association in mmWave networks with RISs.
The simulations in \cite{cao2022joint,wu2022resource, zuo2020resource,zhang2020joint, ni2021resource, el2022latency, taghavi2021user} demonstrate that matching theory is a low-complexity solution for resource allocation and association problems in RIS-aided communication systems. 
However, matching theory relies on iterative searching to eliminate swap blocking pairs, and the searching cost may increase exponentially with more players. In addition, the wireless network players will affect each other, changing the overall interference level. Such peer effects may increase the complexity of applying matching theory.


\begin{figure}[!t]
\centering
\subfigure[Sum-rate comparison under various numbers of RIS elements. The greedy algorithm applies element-by-element RIS phase-shift control. It decides the phase-shift of one element at each time by observing the improvement in sum-rate, and then moves to the next element. The genetic algorithm considers different phase shift designs as individuals, and uses evolutionary strategies to find near-optimal solutions.]{ \label{fig_result3}
\includegraphics[width=7.2cm,height=5.2cm]{Image/fig-results4.jpg}
}
\,
\subfigure[Convergence performance of the genetic algorithm ]{ \label{fig_result4}
\includegraphics[width=7.2cm,height=5.2cm]{Image/fig-results3.jpg}
}
\setlength{\abovecaptionskip}{0pt} 
\caption{\blue{Simulation results of greedy and genetic algorithms. We consider a MISO system with one BS and multiple UEs, and detailed simulation parameters and algorithms can be found in \cite{zhou2023heuristic}}}.
\vspace{-15pt}
\label{fig-re-heu}
\end{figure}



\subsection{Discussions and Numerical Results}

Table \ref{tab-heuristic} compares heuristic algorithms in terms of main features, advantages, drawbacks, difficulties, and applications.  
Compared with model-based algorithms, a common advantage of heuristic algorithms is their low complexity. 

RIS-related optimization problems may involve summation, logarithm, fractional terms, and discrete constraints in problem formulations, which are non-convex and highly non-linear. Applying model-based algorithms generally require a series of transformation and relaxation to achieve a convex or a concave reformulation, but this complexity is avoided in heuristic algorithms. For instance, many problems are easily converted into DC forms, and the CCP algorithm can be applied by iteratively finding two points with the same tangent vectors. Compared with other estimation-based methods such as MM or SCA, the CCP method has much lower complexity, since no extra surrogate function is required. However, the initial point selection may affect the solution quality of the CCP algorithm.  

Greedy algorithms employ a simple greedy policy for decision-making. They aim to maximize the current benefit, disregarding the effect on future stages. Greedy algorithms can efficiently solve problems in near-linear time complexity. However, greedy algorithms can only generate locally optimal results, and the increasing number of control variables may lead to poor performance.

By contrast, meta-heuristic algorithms apply more advanced heuristic rules for iterative exploration, e.g., genetic algorithm, tabu search, and PSO. 
Similar to greedy algorithms, meta-heuristic algorithms have no requirements for problem formulations and constraints, and objective functions can be easily converted into fitness functions. 
However, compared with greedy algorithms, meta-heuristic algorithms can better guarantee the solution quality by using heuristic rules for iterative optimization.


Different from previous approaches, matching theory specializes in solving resource allocation and association problems.
In matching-based methods, the control variables are considered as matching operations, and the objective function is improved by searching swap matching pairs. Therefore, when handling these allocation problems, matching theory is more efficient than other heuristic algorithms due to its dedicated design. 

  

Note that heuristic algorithms may be combined with model-based algorithms. For instance, the energy-efficiency maximization problem in \cite{zhaohui} is decoupled into the beamforming optimization, phase control, and RIS on/off optimization, in which beamforming and phase control are solved by SCA, and RIS on/off is optimized by the greedy algorithm. Other combinations can be found in \cite{Atapattu} by combining SDR with greedy heuristic, and in \cite{ni2021resource} by combining SCA and SDR with matching methods. 

Finally, Fig. \ref{fig-re-heu} shows an example with greedy and genetic algorithms. In particular, the greedy algorithm applies element-by-element RIS phase-shift control. It decides the phase-shift of one element at each time by observing the improvement in sum-rate, and then moves to the next element. Meanwhile, genetic algorithm considers different phase shift combinations as individuals, and uses evolutionary strategies to find near-optimal solutions. Fig. \ref{fig_result3} provides the sum-rate under various numbers of RIS elements. It shows that heuristic algorithms can achieve satisfactory performance with a limited number of RIS elements. However, when the number of RIS elements increases, both greedy and genetic algorithms present sub-optimal results. In addition, Fig. \ref{fig_result4} illustrates the convergence performance of the genetic algorithm. It shows that the average values of individuals increase with iterations, and finally the optimal objective value converges. The main reason is that the genetic algorithm applies evolutionary policies, which will select elite individuals to produce new solutions, and therefore the solution quality is constantly improved.    
























\section{ML-enabled Optimization for RIS-aided Wireless Networks }
\label{sec-ml}

ML has achieved great success in various fields, and this section investigates ML applications for the control and optimization of RIS-aided wireless networks, including supervised learning, unsupervised learning, RL, FL, graph learning, transfer learning, and hierarchical learning.

A variety of algorithms have been developed to optimize RIS-aided wireless networks. Early studies mainly considered model-based methods, and some heuristic algorithms are deployed as low-complexity solutions. However, there are several challenges for these conventional optimization techniques:

1) \textbf{Highly dynamic wireless environment}: Wireless networks are highly dynamic due to frequently changing channel conditions, traffic demands, and user conditions. These dynamics lead to great difficulty for conventional optimization schemes. As an example,  model-based methods need full knowledge of the formulated problem, but some sensitive information, e.g., real-time user locations, may be unknown in practice.

2) \textbf{Evolving network architecture}: The wireless network architecture is constantly evolving from RAN to cloud RAN, virtual RAN, and Open RAN. Consequently, these new architectures increase the complexity of network management, and conventional algorithms may have difficulty modelling and optimizing such complicated systems.

3) \textbf{Diverse user requirements}: Wireless network user types are not limited to enhanced Mobile Broad Band, Ultra Reliable Low Latency Communications, and massive Machine Type Communications. Some newly emerged applications, such as virtual and augmented reality, have more stringent requirements on network metrics, leading to a great burden for conventional optimization methods. 

Given these challenges, ML-enabled control and optimization techniques have become appealing approaches for wireless communications in general, as well as for  RIS-aided wireless networks. In the following, we will introduce the fundamentals and applications of various ML techniques.   





\begin{figure*}[!t]
\centering
\setlength{\abovecaptionskip}{-5pt} 
\includegraphics[width=0.95\linewidth]{Image/fig-nn.jpg}
\caption{Supervised learning for RIS-aided wireless networks.}
\label{fig-nn}
\vspace{0pt}
\end{figure*}


\subsection{Supervised Learning-enabled Optimization}

Supervised learning is designed to find the hidden relationships between inputs and labelled outputs. Supervised learning algorithms adjust their parameters to map the input to the expected output, and this relationship is used for the prediction and classification of unseen data. Table \ref{tab-supervised} summarizes supervised learning-based control and optimizations for RIS-aided wireless networks. 
It shows that most studies consider partial CSI or pilot signals as input to predict full CSI, RIS phase shifts, and data rate, and then utilize the prediction results to maximize the data rate.
This subsection will discuss how to apply supervised learning for optimizing RIS-aided wireless networks, including data acquisition, neural network architecture, loss functions and algorithm training. 

\subsubsection{Dataset Acquisition in RIS-aided Environments}
A fine-grained dataset is the prerequisite for deploying supervised learning, since it relies on the labelled output for validation. Table \ref{tab-supervised} indicates that the dataset is generated in various ways: simulators, exhaustive search, codebook, model-based optimization algorithms or live networks. For example, the exhaustive search method means trying different solutions randomly and then collecting the corresponding output to form labelled datasets\cite{taha2021enabling}\cite{aygul2021deep}. By contrast, a more efficient method is to reuse the data produced by AO\cite{song2021truly} and BCD\cite{hu2021reconfigurable} as model-based optimization algorithms. 

In addition, the algorithm performance also depends on the dataset size, ranging from 5000 \cite{alexandropoulos2020phase} and 30000\cite{aygul2021deep,taha2021enabling} to 200000 \cite{hu2021reconfigurable} in several studies. The simulation results in \cite{aygul2021deep,taha2021enabling} demonstrate that the achievable data rate is significantly improved when the number of training samples increases from 5000 to 30000. Note that the complex entry of the input data, especially the channel coefficient, is usually split into real and imaginary parts, increasing the dimension of the neural network input. 
Although there are several ways to generate the data for supervised learning, most existing datasets are simulation-based. Realistic datasets that are produced in real-world RIS-aided environments are still very rare. 

\subsubsection{Loss Functions and Algorithm Training}
Given the huge number of training samples, supervised learning models are trained to produce the expected output. Suppose that the prediction output is the RIS phase shifts \cite{zhang2021deep,aygul2021deep,alexandropoulos2020phase}, and   
the loss function is defined to minimize the mean square error (MSE) of algorithm training
\begin{equation}\label{ml-nnloss}
Loss (\omega)=\frac{1}{N}\sum^N_{i=1}(\theta_i-\hat{\theta}_{i}(\omega))^2,    
\end{equation}
where $N$ is the total number of outputs, i.e., the number of RIS elements, $\theta_i$ is the desired phase shift given by the dataset, $\omega$ is the neural network weight, and $\hat{\theta}_{i}(w)$ indicates the RIS phase shift predicted by neural networks. Meanwhile, note that the dataset must be divided into training and validation samples, since the objective of algorithm training is to predict unseen data. For example, the authors in \cite{song2021truly} includes 10000 samples to predict the RIS phase shifts, of which 90\% is used for training and the remaining 10\% for testing purposes. 





\subsubsection{Neural Network Architecture and Overfitting}
Table \ref{tab-supervised} shows that DNN is used in most studies to predict CSI or RIS phase shifts, and the network architecture ranges from 4 to 9 layers. It is known that more hidden layers may provide a better performance, but the computational complexity and training time will increase. Hence, the network architecture selection should consider the trade-off between performance and training costs.  

Overfitting is another important issue for neural network training. It means that the algorithm fits exactly to the current training data, but cannot achieve satisfying prediction for unseen data, which should be carefully prevented. One solution is to add a random dropout layer with probabilities, ignoring the contribution of some neurons \cite{hu2021reconfigurable}. Multiple methods are provided by \cite{song2021truly} to suppress overfitting in predicting RIS phase shifts, including larger datasets (CSI and RIS phase shift pairs), decreasing hidden layers, and early stopping. 

Finally, Fig. \ref{fig-nn} summarizes how to apply supervised learning for RIS-aided wireless networks. The datasets are produced by simulators, exhaustive searching, testbed, and model-based methods. Then, one specific model will be selected, i.e., FNN, CNN, and RNN. Note that each neural network model has unique features and advantages, e.g., RNNs are suitable for handling sequential data, and CNNs can better handle spatial data. The selection of neural network models require case-by-case analyses on the dataset size, quality, and data-processing demands.  The number of nodes and hidden layers of neural networks should be carefully designed, which will affect the network training time and accuracy. Finally, selected models are trained and implemented, and the algorithm output includes RIS phase shifts, achieved data rate, BS beamforming vectors and so on, which are further used to optimize network performance.




\begin{table*}[!t]
\caption{Summary of unsupervised learning for RIS-aided wireless networks }
\centering
\small
\setstretch{1.05}
\resizebox{1\textwidth}{!}{%
\begin{tabular}{|m{0.7cm}<{\centering}|m{2cm}<{\centering}|m{1.5cm}<{\centering}|m{1.9cm}<{\centering}|m{1.3cm}<{\centering}|m{1.7cm}<{\centering}|m{1cm}<{\centering}|m{1.5cm}<{\centering}|m{2.7cm}<{\centering}|m{2.4cm}<{\centering}|m{2.2cm}<{\centering}|}
\hline 
Ref. &  Scenario  & Phase-shift resolution & Channel settings & CSI & Objectives  &  Model &  Layer  &  Data generation & Input data  & Output data \\
\hline
\multirow{3}*{\cite{song2020unsupervised}} & \multirow{3}*{\makecell{MISO-DL-MU}} &   \multirow{3}*{Continuous}  & \multirow{3}*{\makecell{Rician \\ fading}}  &\multirow{3}*{Perfect} &  \multirow{3}*{\makecell{Maximizing \\ sum-rate}}  &  CNN  &  6 layers &  \multirow{3}*{\makecell{Generated by \cite{huayan}}}  &  CSI  &  RIS phase shifts  \\
\cline{7-8} \cline{10-11}
 &   &  & & &  &  FNN & 5 layers  &   & Effective channel matrix   &  BS beamforming vector  \\
\hline
\cite{nguyen2021machine} & MIMO-DL-SU  &  Continuous &   Rician fading   &  Perfect   &  Maximizing spectral efficiency  & DNN  &  4 layers  & Generated by random exploration  & CSI   & RIS phase shifts   \\
\hline
\cite{dinh2022unsupervised} & Broadcasting
Communications for IoTs  &  Continuous &  Rician fading   & Statistical  &  Maximizing spectral efficiency & DNN  & 4 layers   &  Generated by random exploration  &  CSI  & RIS phase shifts  \\
\hline
\cite{gao2020unsupervised} & MISO-DL-SU  &  Continuous &  Rayleigh fading   &  Perfect & Maximizing data rate  & FNN   &   7 layers    & Generated as \cite{huang2019indoor}   &  CSI   & RIS phase shifts  \\
\hline
\multirow{4}*{\cite{lopez2022deep}} & \multirow{4}*{\makecell{MISO-DL-MU}}  &  \multirow{4}*{\makecell{Continuous/  \\ discrete}} &  \multirow{4}*{\makecell{ Geometry-based \\ clustered\\ delay line}} &  \multirow{4}*{Estimated} &  \multirow{4}*{\makecell{Maximizing \\ sum-rate}} &  FNN &  6 layers  &  Obtained from \cite{bjornson2021configuring}  &  CSI  &  RIS phase shifts \\  
\cline{7-11} 
&   &  &  &  & &  \multicolumn{3}{c|}{ \makecell{K-means is used to cluster RIS elements \\ based on their estimated cascaded channel \\ coefficient without dataset.}}  & Estimated cascaded channel of RIS elements.  &  RIS element clusters.  \\
\hline
\end{tabular}}
\label{tab-unsuper}
\vspace{-10pt}
\end{table*}


\subsection{Unsupervised Learning-based Optimization}

Supervised learning is data-demanding, and fine-grained datasets may be inaccessible in practice and labeling might not be straightforward, preventing the application of supervised learning algorithms. On the contrary, unsupervised learning can find hidden patterns of unlabelled data without predefined targets or human intervention. Table \ref{tab-unsuper} summarizes unsupervised learning algorithms for RIS-aided wireless networks, including unsupervised DNN, FNN, CNN and K-means. This subsection will introduce unsupervised neural networks and clustering algorithms.

\subsubsection{Algorithm Training and Network Architecture of Unsupervised Neural Networks} \label{s-ml-un} 
Supervised neural networks aim to minimize the loss between predicted results and desired target, i.e., predicted and target data rate in the dataset. However, in unsupervised neural networks, the loss function is directly related to optimization objectives. For instance, to maximize the sum-rate, the loss function is defined by
\begin{equation} \label{eq-unsup}
Loss (w)=-\frac{1}{\mathcal{T}}\sum^{\mathcal{T}}_{i=1}(\sum_{k=1}^{K}w_{k} \log(1+\gamma_{k})),     
\end{equation}
where $\sum_{k=1}^{K}w_{k} \log(1+\gamma_{k})$ is the sum-rate of $K$ users, and $\mathcal{T}$ is the minibatch size. Given the loss function as equation (\ref{eq-unsup}), the weight and bias of neural networks will be optimized to minimize the loss function and meanwhile maximize the data rate. 
Table \ref{tab-unsuper} shows that most existing works apply 2 to 5 hidden layers. In particular, the hidden layer numbers are related to the problem's complexity. The authors in \cite{nguyen2021machine} used 1 hidden layer with 40 nodes for $8 \times 2$ MIMO, and 2 hidden layers for $16 \times 2$ MIMO, achieving satisfying simulation results without overfitting or underfitting. In addition, similar to supervised learning, early stop is applied in \cite{song2020unsupervised} to prevent overfitting.   

\subsubsection{Clustering Algorithms} 
Clustering algorithms are usually unsupervised ML algorithms, i.e., K-means and Density-based spatial clustering of applications with noise (DBSCAN). These algorithms are designed to partition objects into multiple sets to minimize the within-cluster sum of squares. Specifically, it aggregates objects with the same hidden patterns. For instance, K-means is used in \cite{lopez2022deep} to group RIS elements according to estimated channel coefficients, and then each group has the same RIS configurations to reduce the computational complexity. 



\begin{table*}[!t]
\caption{Summary of reinforcement learning for RIS-aided wireless networks }
\centering
\small
\setstretch{1.05}
\resizebox{1\textwidth}{!}{%
\begin{tabular}{|m{0.6cm}<{\centering}|m{2.1cm}<{\centering}|m{1.5cm}<{\centering}|m{1.7cm}<{\centering}|m{1.3cm}<{\centering}|m{1.7cm}<{\centering}|m{1.8cm}<{\centering}|m{2.9cm}<{\centering}|m{3.1cm}<{\centering}|m{3.3cm}<{\centering}|}
\hline 
Ref. &  Scenario  & Phase-shift resolution & Channel settings   & CSI & Objectives &  Algorithm  &   State definition  &  Action definition  & Reward function  \\
\hline
\cite{yang2020deep} & MISO-DL-MU NOMA  & Discrete &  Rayleigh fading  & Perfect  & Maximizing sum-rate  &  DDPG  & Current RIS phases &  RIS phase shifts    &  Sum-rate    \\
\hline
\cite{taha2020deep} & Point-to-point communications  &  Discrete   & Wideband geometric   & Estimated   &  Maximizing data-rate &  DRL & CSI  &  RIS phase shifts   & Data rate   \\
\hline
\cite{guo2021learning} &  MISO-DL-MU UAV  &  Continuous  & Saleh-Valenzuela  &  Imperfect  & Maximizing secrecy rate  &  DDPG  &  CSI  & RIS phase shifts and BS beamforming vector  &  Secrecy rate with penalty   \\
\hline
\cite{yang2020intelligent} & MISO-DL-MU with jammer & Continuous  &  Quasi-static flat-fading &  Perfect    & Maximizing sum-rate  &  Fast-policy hill-climbing learning  &  Previous jammer power and SINR, current CSI  & BS transmit power and RIS phase shifts   &   Maximizing data-rate, decreasing BS power and SINR penalty\\
\hline
\cite{yang2020deep2} & MISO-DL-MU  &  Continuous   & Rayleigh fading  &  Delayed &  Maximizing secrecy rate  &  DRL  &  CSI, previous secrecy rate and transmission rate, QoS level  & BS beamforming vector and RIS phase shifts   &  Maximize the system secrecy rate, guaranteeing QoS requirements.  \\
\hline
\cite{feng2020deep} & MISO-DL-SU  & Continuous  &  Rayleigh fading  &  Perfect  & Maximizing SNR  &  DDPG  &  SNR and current RIS phases   &  RIS phase shifts     &  Received SNR  \\
\hline
\cite{huang2020hybrid} & MISO-DL-MU THz &  Continuous  &  Rayleigh distribution  &  Perfect     &    Maximizing sum-rate  &  DDPG  &  Current BS and RIS beamforming vectors, CSI   &  BS beamforming vectors and RIS phase shifts   &  Throughput and the penalty of adjusting the beamforming direction.  \\
\hline
\cite{lee2020deep} & MISO-DL-MU  & Continuous  & Quasi-static flat-fading  &  Perfect  & Maximizing energy efficiency &  DRL  & CSI and energy level of RIS   & BS beamforming vector, RIS phase shifts and on/off   &  Energy efficiency  \\
\hline
\cite{lin2020deep} & MISO-DL-SU  &  Continuous  &  Quasi-static flat-fading   &  Perfect  & Power minimization  & DDPG   & CSI, previous outage events  & BS beamforming vector and RIS configurations  &  Energy efficiency  \\
\hline
\cite{huang2020reconfigurable} & MISO-DL-MU  & Continuous  &  Frequency flat fading   & Perfect & Maximizing channel capacity & DDPG & Transmit and received power, previous action, and CSI    & BS beamforming vector and RIS phase shifts  & Channel capacity    \\
\hline
\cite{zhang2021millimeter} & MISO-DL-MU mmWave &  Continuous  & 3GPP model  & Perfect/ Imperfect  &   Maximizing sum-rate   &   Distributed RL   &  CSI   &  RIS phase shifts  &  Data rate   \\
\hline
\cite{liu2020ris} &  MISO-DL-MU NOMA  & Continous & Rayleigh fading   &  Perfect  &  Maximizing energy efficiency  & Decaying DDQN   & RIS phases and positions, UE positions, and current BS power allocations & RIS phase and position and BS beamforming changes   &   Energy efficiency with penalty  \\
\hline
\cite{kim2021multi} &  Multi-cell communications  & Continuous  & Rayleigh fading   & Imperfect  &  Maximizing sum-rate &  Multi-agent DRL  &  Local and neighbor CSI, local sum-rate   & RIS phase shifts, BS beamforming vector, and UE power changes     & Sum-rate with interference penalties \\
\hline
\cite{samir2021optimizing} & MISO-UL-MU IoT UAV  &  Continuous  & Rician fading   & Perfect  & Minimizing sum AoI   &   DRL    & SNR and UAV height  & UAV  altitude changes  & Negative summation of age of information   \\
\hline
\multirow{2}*{\cite{yang2021machine}} & \multirow{2}*{\makecell{MISO-DL-MU \\ NOMA}}  &  \multirow{2}*{Continuous}  & \multirow{2}*{ \makecell{Rayleigh \\ fading}}  & \multirow{2}*{Perfect} &  \multirow{2}*{\makecell{Maximizing \\ sum-rate}}  & Object migration automation   &  Current RIS phase  &  Power allocation coefficient  &  Sum-rate  \\
\cline{7-10}
 &   &  &  & &   & DDPG  & Current RIS phase  &  RIS Phase changes  & Sum-rate difference \\
\hline
\end{tabular}}
\label{tab-rl}
\vspace{0pt}
\end{table*}



\subsection{Reinforcement Learning-based Optimization}


\begin{figure*}[!t]
\centering
\includegraphics[width=0.75\linewidth]{Image/fig-rl.jpg}
\caption{DRL-empowered RIS-aided wireless networks}
\label{fig-rl}
\setlength{\abovecaptionskip}{-2pt} 
\vspace{-10pt}
\end{figure*}

RL is the most widely applied ML technique for optimization. The RL agent interacts with the environment under the Markov decision process (MDP) to learn the best long-term policy. In particular, given the current system state $s$, the agent selects an action $a$ for implementation and receives a reward $r$, and then the environment will move to the next state $s'$.
Table \ref{tab-rl} summarizes existing studies that apply RL to RIS-aided wireless networks. This subsection will first analyze the state, action, and reward function definitions of these existing studies, and then present the algorithm architecture and training methods. 

\subsubsection{State Definition}
Table \ref{tab-rl} shows that the state may be defined in various ways, e.g., CSI\cite{taha2020deep,guo2021learning,yang2020intelligent}, current RIS phase\cite{yang2020deep,huang2020hybrid,liu2020ris}, position\cite{liu2020ris,samir2021optimizing}, energy level\cite{lee2020deep}, previous transmission rate\cite{yang2020deep2}.    
Specifically, the state refers to the environment status that should be considered for decision-making. For example, the CSI has a great effect on the RIS phase shifts, and therefore CSI is involved in the state definition of many studies \cite{taha2020deep,guo2021learning,yang2020intelligent}. Similarly, RIS-aided UAVs are investigated in \cite{samir2021optimizing}, and the UAV altitude is included in the state definitions because the height will directly affect the channel conditions.


\subsubsection{Action Definition}
In the context of MDP, the action indicates control variables that will change the state, such as RIS phase shifts\cite{yang2020deep}-\cite{yang2021machine}, 
BS beamforming \cite{guo2021learning,yang2020intelligent,yang2020deep2,huang2020hybrid,lee2020deep}, RIS positions\cite{liu2020ris} and elements on/off\cite{lee2020deep}. The control variables in problem formulations are easily converted into actions. However, note that many RL algorithms require discrete action spaces, but the control variables in problem formulations are usually continuous as shown in Section \ref{sec-bac}. The first solution is to quantize the control variables. For instance, the BS transmit power is quantized with an interval of 1 W\cite{lee2020deep}, and the RIS phase changes $\bigtriangleup \theta \in \{-\frac{\pi}{10},0,\frac{\pi}{10}\}$ in \cite{liu2020machine}. Another solution is to apply the DDPG algorithm, which can handle continuous action-space problems\cite{yang2020deep,guo2021learning,huang2020hybrid,lin2020deep,yang2021machine}.         




\subsubsection{Reward Functions}
The reward function is a crucial part of RL. As shown in Table \ref{tab-rl}, the reward function definition mainly depends on the optimization objectives, including data rate\cite{yang2020deep,taha2020deep,zhang2021millimeter,kim2021multi}, energy efficiency \cite{lee2020deep,lin2020deep,liu2020ris}, channel capacity\cite{huang2020reconfigurable}, and SNR\cite{feng2020deep}. Moreover, the reward function can include multiple objectives and constraints to balance the overall performance. As an example, the reward function in \cite{yang2020intelligent} has data rate as a positive term to maximize the data rate, while BS power consumption is a negative term to reduce power consumption. RL focuses on the long-term accumulated reward, which means it can better adapt to highly dynamic wireless environments without requiring full knowledge of the defined problem. 







\subsubsection{Algorithm Architecture and Training}
RL aims to maximize the long-term expected reward by
\begin{equation} \label{eq-rlmax}
V(s) =\mathbb{E}(\sum_{l=0}^{\infty}\gamma^{l} r(s_{l},a_{l})|s=s_{0}),
\end{equation}
where $V(s)$ indicates the expected accumulated reward at state $s$, $s_{0}$ is the initial state, $r(s_{l},a_{l})$ is the reward of choosing action $a_{l}$ at state $s_{l}$ in $l^{th}$ episode, and $\gamma$ is the discount factor $(0<\gamma<1)$ to balance the instant and future reward.

Then the state-action values are defined to represent the potential reward of selecting action $a$ under state $s$. The state-action value is updated by
\begin{equation} \label{eq-qvalue}
\resizebox{0.87\hsize}{!}{$Q^{new}(s,a)= Q^{old}(s,a)+\alpha(r+\gamma \max\limits_{a} Q(s',a) -Q^{old}(s,a))$},
\end{equation}
where $Q^{old}(s,a)$ and $Q^{new}(s,a)$ are old and new Q-values, respectively. $\alpha$ is the learning rate ($0< \alpha < 1 $). Equation (\ref{eq-qvalue}) indicates that a Q-table is used to record all the state-action values, leading to slow convergence for problems with large state-action space. To this end, DQN is proposed to use neural networks for Q-value estimation, and the loss function is defined by
\begin{equation} \label{eq-dqn}
\resizebox{0.87\hsize}{!}{$Loss(\omega)=Er(r+\gamma \max\limits_{a} Q(s',a,\omega')-Q(s,a,\omega))$},
\end{equation}
where $Er$ represents the error between the predicted Q-value $Q(s_,a,\omega)$ and target Q-value $r+\gamma \max\limits_{a} Q(s',a,\omega')$. $\omega$ and $\omega'$ are the weight of the main and target networks, respectively. 

In DQN, $\max\limits_{a} Q(s',a,\omega')$ indicates that the target network will select the action and meanwhile evaluate the action, and the maximizing operator will result in over-optimistic Q-value estimation. Then DDQN is proposed to mitigate Q-value over-estimation by
\begin{equation} \label{eq-ddqn}
\resizebox{0.88\hsize}{!}{$Loss(w)=Er(r+\gamma Q(s',\arg \max\limits_{a}Q(s',a,\omega),\omega') - Q(s,a,\omega))$},
\end{equation}
where $\arg \max\limits_{a}Q(s',a,\omega)$ means action selection of the main network, and $Q(s',\arg \max\limits_{a}Q(s',a,\omega),\omega')$ indicates the action evaluation of the target network.
 
DRL has been used for RIS phase-shift optimization in \cite{taha2020deep,feng2020deep, yang2020deep2}. In these studies, continuous phase shifts are quantized to form discrete action spaces for DQN or DDQN. On the contrary, DDPG can handle continuous action spaces directly without quantization, which has been used for continuous RIS phase-shift control in \cite{yang2020deep,guo2021learning,huang2020hybrid}. 





DDPG is considered a combination of actor-critic learning and DQN, in which the actor network selects actions, and the critic network evaluates the state-action values. The loss function of the critic network is defined as
\begin{equation} \label{eq-ddpg1}
\resizebox{0.88\hsize}{!}{$Loss(w)=Er(r+\gamma Q(s',a(s',\omega^{A'}),\omega^{C'}) - Q(s,a,\omega^{C}))$},
\end{equation}
where $a(s',\omega^{A'})$ indicates that action $a$ is selected by the target actor network with weight $\omega^{A'}$, and $Q(s',a(s',\omega^{A'}),\omega^{C'})$ means the state-action value is evaluated by the target critic network with weight $\omega^{C'}$. For the actor network, the policy gradient is 
\begin{equation} \label{eq-ddpg2}
\begin{aligned}
\nabla_{\omega^{A}} J \approx \frac{1}{\mathcal{T}}\sum^{\mathcal{T}}_{i=1} (\nabla_{a}Q(s,a,\omega^{C})&|_{s=s_{i},a=a(s_i,\omega^{A})} \\ 
& \cdot\nabla_{\omega^{A}}a(s_i,\omega^{A})|_{s=s_i}),
\end{aligned}
\end{equation}
In addition, similar to DQN, DDPG uses the experience replay technique to produce a minibatch for training. DDPG also uses the soft update method for target network updating.  

Finally, Fig. \ref{fig-rl} shows DRL-empowered RIS-aided wireless networks. Based on the current state $s$, the agent selects BS beamforming vector and RIS phase shifts as action $a$. Then the action $a$ is implemented and rewards $r$ are collected, e.g., sum-rate, energy efficiency, and power consumption. The system will arrive at a new state $s'$ that is indicated by CSI, user positions, and SNR. The experience tuple $<s,a,r,s'>$ is saved in the experience pool, and a mini-batch is sampled to train the main network. Finally, the target network will copy the weight of the main network, providing a stable reference for the main network training. The framework in Fig. \ref{fig-rl} may be easily generalized to many other RL algorithms without loss of generality.  





\begin{figure*}[!t]
\centering
\includegraphics[width=0.95\linewidth]{Image/fig-fl.jpg}
\caption{Federated learning and RIS-aided wireless networks}
\label{fig-fl}
\setlength{\abovecaptionskip}{-2pt} 
\vspace{-10pt}
\end{figure*}


\begin{table*}[!t]
\caption{Summary of federated learning and RIS-aided wireless networks }
\centering
\small
\setstretch{1}
\begin{threeparttable} 
\resizebox{1\textwidth}{!}{%
\begin{tabular}{|m{0.8cm}<{\centering}|m{1.7cm}<{\centering}|m{1.5cm}<{\centering}|m{1.8cm}<{\centering}|m{1cm}<{\centering}|m{3.3cm}<{\centering}|m{3.4cm}<{\centering}|m{3.8cm}<{\centering}|m{2.2cm}<{\centering}|}
\hline 
Ref. &  Scenario  & Phase-shift resolution  &  Channel settings & CSI &  FL-related objectives &   Control variables  &  Constraints & Algorithms   \\
\hline
\cite{liu2021reconfigurable} &  AirFL with NOMA  &  Continuous & Rician channel   & Perfect   & Minimizing the gap between converged and optimal  training loss.     & FL device selection, receiver beamforming and RIS phase shifts.  & Device selection, receiver beamform, and phase shifts constraints  & AO, Gibbs sampling, SCA  \\
\hline
\cite{yang2021reconfigurable} & AirFL with RISs  & Continuous  & Rician fading  &   Perfect &   Minimizing global loss  &  Transmit power and RIS phase shifts  & Power and dual constraints  & AO, QCQP, SDP   \\
\hline
\cite{ni2022star} & AirFL with NOMA  & Continuous & Rayleigh fading/  Rician fading     & Perfect   & Minimizing FL training gap  &  BS transmit power and RIS configurations    &  Transmit power, target rate, MSE tolerance, and RIS configuration constraints. & AO, SCA, SDR   \\
\hline
\cite{battiloro2022dynamic} & RIS-enhanced FL   & Discrete  & Generated by simulator   & Perfect   & Minimizing average system power consumption  &  RIS phase shifts and bits, bandwidth, and CPU frequency & Training latency, convergence rate, and learning performance constraints  & Stochastic Lyapunov optimization, greedy algorithm   \\
\hline
\cite{zheng2022balancing} & AirFL with RISs  & Continuous &  Empirical channel fading  & Perfect/ Imperfect   & Minimizing the MSE of the aggregated AirFL model   & Receive and transmit beamformer, RIS phase shifts   &  Total transmit power, phase shifts, and target rate constraints  &  AO, SCA  \\
\hline
\cite{ni2021over} &  AirFL with RISs and NOMA  &  Continuous & Rayleigh fading  &  Perfect & Maximizing the achievable hybrid rate of FL and NOMA   & User transmit power, BS receive scalar, and RIS phase shifts  & Target rate and MSE, phase configuration and total transmit power constraints  &   AO, SCA, SDR   \\
\hline
\cite{ni2021federated} &  AirFL with RISs and NOMA  & Continuous & Rayleigh fading   &  Perfect  &  Minimizing the FL MSE and cardinality  &  Transmit power, receive scalar, reflection coefficients, and learning participants  & Total transmit power, phase configuration, target MSE, and the number of learning devices   &  AO, SDR, SCA  \\
\hline
\cite{liu2021joint} &  AirFL with RISs   & Continuous  & Obtained from \cite{tang2020wireless}  & Perfect   &   Minimizing the effect of device selection and the communication error on the convergence rate  & Device selection, over-the-air transceivers, and RIS phase shifts   & Device selection, receiver beamforming, and phase configurations    &  Gibbs-sampling, SCA   \\
\hline
\cite{yang2022federated} & AirFL with RISs   &  Continuous  & Obtained from testbed   & Perfect     &  Maximizing FL utility  &  RIS phase shifts, user-RIS association, and bandwidth allocation        & Bandwidth allocation, RIS phase configurations and association, target SNR constraints  & Matching game, bisection search      \\
\hline
\cite{zhang2021energy} &  AirFL with RISs     &    Continuous & Rayleigh fading   & Perfect   &  Power minimization   & CPU frequency, power and bandwidth allocation, RIS configurations and accuracy design   &  Task completion time, maximum transmit power, phase configuration, total bandwidth       & AO, SDP, MM     \\
\hline
\cite{li2020enhanced} & FL-aided RIS optimization   &   Continuous  
& Wideband geometric/ Rayleigh fading  &  Predicted    &   Average rate maximization    &\multicolumn{3}{c|}{ \makecell{Local models: local devices train local DNNs to predict channel \\ rate using sampled channel vectors; \\Global model: edge server aggregates local DNN models and average.\tnote{1}\\
}} \\
\hline
\cite{zhong2022mobile} & FL-aided mobile RIS optimization   &   Continuous & Rician fading   &  Predicted    &   Sum-rate maximization    &  \multicolumn{3}{c|}{ \makecell{FL-DDPG is applied. Neural networks are trained at local agents\\ and then aggregated to predict Q-values. Control variables \\ include RIS positions and phase shifts, and AP power allocation .  }} \\
\hline
\end{tabular}}
 \begin{tablenotes}    
        \footnotesize       
        \item[1] The columns are combined because \cite{li2020enhanced} \cite{zhong2022mobile} are different than other studies by using FL as an optimization approach, while FL in other works \\ of Table \ref{tab-fl} is part of the optimization objectives. Therefore, instead of showing control variables and constraints, it is essential to present the local \\ and global models of FL-based optimization algorithms.   
\end{tablenotes} 
      
\end{threeparttable} 
\label{tab-fl}
\vspace{-12pt}
\end{table*}





\begin{figure*}[!t]
\centering
\includegraphics[width=1\linewidth]{Image/fig-gl.jpg}
\caption{Graph learning for RIS-aided wireless networks}
\label{fig-gl}
\setlength{\abovecaptionskip}{-2pt} 
\vspace{-10pt}
\end{figure*}


\subsection{ Federated Learning and RISs}

Different with conventional centralized ML algorithms, FL trains the model across multiple decentralized edge devices or servers that hold local datasets without exchanging data. In FL, each edge device will train a local model using local samples, and then a global model is formed by aggregating local model parameters. Afterwards, edge devices download the global model to update local models. 
Table \ref{tab-fl} summarizes existing works focusing on FL and RIS-aided wireless communications. This subsection first discusses RIS-enhanced over-the-air FL (AirFL), then introduces how to use FL optimization in RIS-aided environments. In particular, AirFL implements FL in wireless networks, using edge devices for local model training and edge server for model aggregation.

\subsubsection{RIS-enhanced Over-the-air FL}
The main advantage of FL is that it preserves data security and privacy, and the distributed property makes wireless networks an ideal platform for FL training. Therefore AirFL is proposed to combine FL with wireless communications. Specifically, it investigates how to enhance FL performance in RIS-aided wireless networks by minimizing global training loss\cite{liu2021reconfigurable, yang2021reconfigurable},  MSE\cite{ni2022star,ni2021federated}, power consumption\cite{zhang2021energy,battiloro2022dynamic}, maximizing the FL utility\cite{yang2022federated}. In \cite{yang2021reconfigurable}, Yang \textit{et al.} aim to minimize the global training loss of FL by controlling transmit power and RIS phase shifts, and the optimization problem is solved by AO-based QCQP and SDP. 
\cite{yang2022federated} proves that RISs can improve more than 30\% prediction accuracy of AirFL, and a 10 times lower AirFL test error is reported in \cite{ni2021federated} by using multi-RIS. In these works, the FL performance is improved by optimizing the resource allocation and user-RIS association, and then edge users can efficiently upload the local models.
Meanwhile, it is worth noting that these works still rely on model-based optimization algorithms, such as AO, QCQP\cite{yang2021reconfigurable}, SCA\cite{ni2022star,zheng2022balancing,ni2021over,ni2021federated}, and MM\cite{zhang2021energy}. 





\subsubsection{ FL for RIS-aided Wireless Communications} 
On the other hand, FL can also be used to optimize the performance of RIS-aided wireless communications. For example, FL is used in \cite{li2020enhanced} and \cite{zhong2022mobile} for average rate maximization, in which local models are deployed in user devices and the global model is aggregated by edge servers. In \cite{li2020enhanced}, federated neural networks consider sampled channel vectors as input to predict achievable rates. FL and DDPG are combined in \cite{zhong2022mobile}, and the local neural networks used in DDPG will be aggregated and updated.   

Fig. \ref{fig-fl} shows the interaction between FL and RIS-aided wireless communications. On the one hand, RIS can enhance the AirFL performance, i.e., increasing prediction accuracy and minimizing training loss. On the other hand, FL can improve the RIS-aided wireless network performance, e.g., maximizing data rate and minimizing power consumption. 








\subsection{Graph Learning }
Graph learning refers to a group of ML techniques in the graph domain, including graph neural networks (GNN), graph attention networks (GAN) and graph convolution networks (GCN). Compared with CNN, which operates on regular Euclidean data like images (2D grid) and text (1D sequence), graph learning is more efficient in describing graphs and structures. Graph learning aims to transform nodes, edges, and their features into low-dimension vector spaces by preserving properties such as graph structure\cite{cui2018survey}. 

 Wireless networks are highly dynamic, and the wireless data may be collected from non-Euclidean domains, which is represented by graph structure with high dependency on network topology.
The conventional approach of data processing is to convert the data with graph structure into Euclidean domain, but such transformation leads to high complexity and extra overhead. 
By contrast, graph learning enables the graph-structured data to be processed effectively, and transforming the wireless network topology into graphs can better describe the association and interference between network devices\cite{he2021overview}.
Therefore, graph learning has been applied to power control and interference management \cite{naderializadeh2020wireless}, resource allocation \cite{eisen2020optimal,jiang2020dynamic}, network slicing \cite{wang2020graph}, and so on. In the following, GNN is used as an example to introduce graph learning fundamentals, and then we explain how to apply graph learning for RIS control and optimizations. 






\subsubsection{GNN Fundamentals} 
The primary motivation for developing GNN is to extend the existing neural network architecture into graph-related data processing capabilities\cite{zhou2020graph}. In a graph, each node is described by its features and related nodes. Suppose that $z_{v}$ is a state vector to describe the features of node $v$, and it is defined by
\begin{equation}\label{eq-gnn1}
 z_v=f(y_v,y_{v}^{ed},y_{v}^{ne},z_{v}^{ne}),    
\end{equation}
where $y_v$ and $y_{v}^{ed}$ are the features of node $v$ and its edge, and $z_{v}^{ne}$ and $y_{v}^{ne}$ are the state and features of neighbour nodes, respectively.
Then, $z_{v}$ and $y_v$ are used to produce an output $o_v$ by
\begin{equation} \label{eq-gnn2}
o_v=g(z_v,y_v),    
\end{equation}
where $g$ is the output function to map the relationship between states, features, and outputs.  

Similarly, by collecting all the states and features, we have
\begin{equation} \label{eq-gnn3}
Z^{l+1}=f(Z^{l+1},Y),    
\end{equation}
\begin{equation}
O=g(Z,Y_N),    
\end{equation}
where $Z^{l+1}$ indicates all the states at $l^{th}$ iteration, $Y$ indicates all the features, $Y_N$ means the node features, and $O$ is the overall output. Equation (\ref{eq-gnn3}) shows that the system state is updated in an iterative manner, which is inspired by Banach’s fixed point theorem \cite{khamsi2011introduction}. Finally, similar to conventional neural networks, GNN aims to minimize the loss function.



\subsubsection{Graph Learning for RIS Control and Optimizations} 
Interference control is an important technique for multi-user environments to maximize the system sum-rate, and the interactions between RISs and UEs are easily described by a graph. The graph in Fig. \ref{fig-gl} includes $K+1$ nodes, in which one node represents the RIS and the rest are $K$ UEs. Given this scheme, GNN is applied to user scheduling and RIS configurations in \cite{zhang2022learning} and \cite{zhang2022user}. In particular, the GNN is trained in an unsupervised manner, and the inputs are user weights and pilot sub-frames of the scheduled users, and the outputs are RIS configurations and beamformers. Similarly, unsupervised GNN is applied in \cite{jiang2021learning} for network utility maximization, which takes pilot signals as input to optimize the BS beamforming and RIS configurations.     

In \cite{zhang2022learning,zhang2022user,jiang2021learning}, a useful feature of GNN is used to reduce the interference between users. Specifically, when updating one node in the GNN, all the neighbour nodes will be included in the updating function, which means GNN can better capture the mutual interference between users.  
In addition, RIS node updating is a function of all the user nodes, enabling GNN to configure RIS elements to improve the channel capacity of all users.   








\subsection{Transfer Learning}
Long training time and slow convergence are common issues of most ML algorithms, and one of the main reasons is that the model must explore the task from scratch. Fast decision-making is critical in wireless communications, but the low sampling efficiency may prevent applying ML to RIS-aided wireless networks. This subsection will introduce transfer learning fundamentals and explain how transfer learning can improve ML-enabled wireless networks with RISs.

\subsubsection{Transfer Learning Fundamentals}
Transfer learning can be combined with many ML algorithms, and here we consider transfer reinforcement learning (TRL) as an example\cite{zhou2022learning}. In conventional RL, the decision-making $\mathcal{D}_{RL}$ of one agent is described by 
\begin{equation} \label{eq-trl}
\mathcal{D}_{RL}:s \times \mathscr{K}\rightarrow a, r ,
\end{equation}
where $\mathscr{K}$ represents the agent's knowledge, $s$, $a$, and $r$ are the current state, selected action, and received reward, respectively. In equation (\ref{eq-trl}), the agent utilizes the collected knowledge $\mathscr{K}$  for decision-making and action selection.  

By contrast, the decision-making in TRL is
\begin{equation} \label{eq-trl2}
\mathcal{D}_{TRL}:s \times \mathcal{M}(\mathscr{K}_{expert}) \times \mathscr{K}_{learner} \rightarrow a,r,
\end{equation}
where $\mathscr{K}_{expert}$ and $\mathscr{K}_{learner}$ are the knowledge of the expert and learner agents, respectively. The learner is designed to solve the target task, and the expert has some existing knowledge of related source tasks.
Considering the similarities between the source and target tasks, the expert's experience may be reused by the learner as prior knowledge.
The $\mathcal{M}$ in equation (\ref{eq-trl2}) defines a mapping function. $\mathcal{M}(\mathscr{K}_{expert})$ indicates that the expert's experience will be transformed into digestible knowledge, boosting the learning process of the learner. With existing prior knowledge, the learner can achieve a jump-start at the exploration phase, achieving a higher exploration efficiency and average reward with faster convergence\cite{zhou2022knowledge}.     





\begin{figure}[!t]
\centering
\includegraphics[width=0.85\linewidth]{Image/fig-trl.jpg}
\caption{Transfer reinforcement learning for RIS-aided wireless networks}
\label{fig-trl}
\setlength{\abovecaptionskip}{-2pt} 
\vspace{-10pt}
\end{figure}



\subsubsection{Transfer Learning-boosted Wireless Networks with RIS} 
Fig. \ref{fig-trl} illustrates how to use TRL for RIS-aided wireless networks. We assume that the expert agent has existing knowledge of the source task, BS beamforming, and the learner agent is designed for the target task, joint active and passive beamforming. Then the task similarities mean that the learner can reuse the expert's experience to better handle target tasks. 
Specifically, the expert's knowledge may exist in various ways, e.g., state-action values, action selections, and reward definitions. Then the mapping function may be defined in different manners, and we present a Q-value-based mapping function as an example
\begin{equation} \label{eq-map}
\resizebox{0.89\hsize}{!}{$\begin{aligned}
Q^{new}(s^{L},a^{L})=  &Q^{E}(\mathcal{M}(s^{L}),\mathcal{M'}(a^{L}))+Q^{old}(s^{L},a^{L})+\\
&\alpha(r+\gamma \max\limits_{a} Q(s',a)-Q^{old}(s^{L},a^{L})),
\end{aligned}$}
\end{equation}
where $s^{L}$ and $a^{L}$ are the learner's state and action, $\mathcal{M}$ and $\mathcal{M'}$ are the state and action map functions, respectively, and $Q^{E}$ indicates the state-action value of the expert. 
Compared with conventional RL, the main difference is that $Q^{E}(\mathcal{M}(s^{L}),\mathcal{M'}(a^{L}))$ is involved as an extra reward for selecting $a^{L}$ under $s^{L}$. $\mathcal{M}$ and $\mathcal{M'}$ aim to obtain $s^E=\mathcal{M}(s^{L})$ and $a^E=\mathcal{M'}(a^{L})$, where $s^{E}$ and $a^{E}$ indicate the expert's state and action that are the closest to $s^{L}$ and $a^{L}$, respectively. By finding these similar states and actions,  good actions with high Q-values in the expert could also bring satisfying rewards for the learner, which will further guide the exploration of the learner. Note that both the control variables of the expert and learner include BS beamformer, which are used to define the action mapping function $\mathcal{M'}$.  

With transfer learning, the RL agent can achieve higher exploration efficiency and faster convergence, enhancing the efficiency of RIS-aided wireless networks. Transfer learning has been used in \cite{zhou2022learning,zhou2022knowledge2} for joint resource allocation of network slicing, and \cite{elsayed2020transfer} for mmWave networks, achieving faster convergence and better network performance. Similarly, transfer learning can be applied to ML-enabled RIS optimization for faster convergence and achieving prompt phase-shift responses. 



\subsection{Hierarchical Learning }

\begin{figure}[!t]
\centering
\includegraphics[width=1\linewidth]{Image/fig-hrl.jpg}
\caption{Hierarchical reinforcement learning for RIS-aided wireless networks}
\label{fig-hrl}
\setlength{\abovecaptionskip}{-2pt} 
\vspace{-10pt}
\end{figure}



\begin{table*}[!t]
\caption{Summary of ML-based control and optimization algorithms for RIS-aided wireless networks. }
\centering
\small
\setstretch{1.05}
\resizebox{1\textwidth}{!}{%
\begin{tabular}{|m{2cm}<{\centering}|m{2.5cm}<{\centering}|m{5.5cm}<{\centering}|m{4.2cm}<{\centering}|m{4.5cm}<{\centering}|}
\hline
ML \quad techniques & Typical algorithms & Main features & Difficulties & RIS-related applications\\
\hline
\multirow{1}*{\makecell{Supervised\\ learning}} & Supervised DNN, CNN, decision trees, and support vector machine. & The algorithm is trained to map the relationship between the given input and labelled output for classification and prediction. The input data is fed into the model, and then the model parameters are adjusted until the output is properly fitted.  
& 1) Supervised learning relies on fine-grained datasets to train the algorithm; 2) The algorithm training may be time-consuming; 3) The model is easy to be overfitted.   & Given the dataset input, DNN is applied to predict full channel states \cite{zhang2021deep} or optimal RIS phase shifts\cite{alexandropoulos2020phase}- \cite{hu2021reconfigurable}.  \\
\hline
Unsupervised learning  & K-means, DBSCAN, and unsupervised neural networks.  & Unsupervised learning algorithms aim to unveil hidden patterns of unlabelled datasets.  & The result performance is hard to be testified or explain. & Unsupervised DNN is used for RIS phase-shift control by defining objectives as loss functions\cite{nguyen2021machine}-\cite{gao2020unsupervised}. \\
\hline
\multirow{20}*{\makecell{Reinforcement \\ learning}}  & Q-learning & The agent interacts with the environment under the MDP framework, recording experience by a Q-table.  & 1)Long convergence time for large state-action problems; 2) Discrete states and actions only.  &\multirow{18}*{\makecell{RL is the most widely \\ applied ML technique for \\ the control and optimization \\  of RIS-aided wireless networks, \\ e.g., power minimization \cite{lin2020deep}, \\ sum-rate\cite{yang2020deep,taha2020deep,samir2021optimizing,yang2021machine},\\ secrecy rate \cite{guo2021learning,yang2020deep2},\\ and energy efficiency\cite{lee2020deep,liu2020ris}.\\ DDPG is especially useful \\ considering the continuous \\ RIS phase-shift control \\ requirements\\ \cite{yang2020deep, guo2021learning, huang2020hybrid,lin2020deep}. }}\\
\cline{2-4}
& Actor-critic learning & The actor is defined to select actions, while the critic evaluates the actions. & 1) Long convergence iterations; 2) Unstable performance due to the interaction between actor and critic. & \\
\cline{2-4}
& Deep reinforcement learning & DRL applies neural networks to predict state-action values, solving the large state-action issue of tabular Q-learning. 
& \multirow{3}*{\makecell{1) Hyperparameter tuning can \\ be difficult when lacking \\ experience. 2) The sampling \\ efficiency is low. }}  &\\
\cline{2-3}
& Double deep Q-learning & DDQN provides a more accurate Q-value estimation by decoupling the action selection and evaluation.  & &\\
\cline{2-4}
& Multi-agent reinforcement learning & Each agent applies RL or DRL independently to optimize its performance or achieve an overall goal. & The coordination mechanism of multiple agents must be carefully designed.  &\\
\cline{2-4}
& DDPG & DDPG combines actor-critic with policy gradients, optimizing problems with continuous action space.  &  1) Unstable and heavily dependent on appropriate hyperparameters; 2) overestimation in critic network.   &\\
\hline 
Federated learning & Federated deep learning, federated DRL & Local models are first trained using local datasets, and then the parameters are aggregated to form a global model. Local devices will download the global model to update local models. User privacy is well protected in FL.
& 1) High communication overhead due to parameter exchange; 2) the local device heterogeneity will affect the system performance. & On one hand, RISs can improve the AirFL performance by minimizing global loss and training gap; on the other hand, FL is used to optimize network performance \cite{li2020enhanced, zhong2022mobile}.    \\
\hline
Graph learning & Graph neural networks, and graph attention networks. & Graph learning refers to ML on graphs. It maps the graph features to vectors with the same dimensions in the embedding space, which is used for link prediction, matching and classification.  &  1) Dynamic and generative changing graph; 2)Interpretability of Graph Learning.  & GNN is used for user schedule and RIS configurations in \cite{zhang2022learning, zhang2022user, jiang2021learning} to maximize network utility and sum-rate.    \\
\hline
Transfer learning & Transfer reinforcement learning, transfer supervised learning  & Transfer learning aims to reuse the existing knowledge of experts to accelerate the learning process on target tasks, achieving faster convergence and less training efforts. & 1) The mapping function is hard to design, changing with different algorithms; 2) Transfer learning is vulnerable to adversarial attacks.  & Transfer learning may be used to accelerate the ML algorithm training for optimizing RIS-aided wireless networks.  \\
\hline
Hierarchical learning & Hierarchical reinforcement learning, hierarchical deep learning  &   Hierarchical learning decouples the task into multiple sub-tasks and goals, increasing the task exploration efficiency.  &  1) The goal and sub-task selection require case-by-case analyses; 2) The relationship between the meta-controller and the sub-controller may be unstable.  & It is used for optimizing RIS-aided networks with control variables that have different time scales.     \\
\hline
\end{tabular}}
\label{tab-mlsummary}
\vspace{-10pt}
\end{table*}


\begin{table*}[!t]
\caption{Summary of control and optimization techniques for RIS-aided wireless networks }
\centering
\small
\setstretch{1}
\resizebox{1\textwidth}{!}{%
\begin{tabular}{|m{1.5cm}<{\centering}|m{3.5cm}<{\centering}|m{3.3cm}<{\centering}|m{3.5cm}<{\centering}|m{3cm}<{\centering}|m{4cm}<{\centering}|}
\hline 
Optimization approaches  &  Main features   &     Advantage    &  Drawbacks    &  Difficulties   &  Application scenarios for RISs  \\
\hline
Model-based algorithms &  Model-based algorithms aim to find global optimal or at least sub-optimal results for target problems. They usually require full knowledge of the problem to find near-optimal solutions by using transformation, relaxation, and approximation.  &  Model-based algorithms, i.e., SCA, MM, can provide detailed proofs and explanations for the optimality. Target problems are efficiently solved with guaranteed optimality once the closed-form solution is achieved.  & Model-based solutions are usually problem-specific with certain requirements such as convexity and continuity, indicating case-by-case analyses and design. It has difficulty adapting to dynamically changing environments.  &  It has to apply transformations, division, and relaxation to convert the problem to specific forms. These transformations need a dedicated design for each problem.  & Numerous algorithms have been developed to solve RIS-aided optimization problems, e.g., AO to decouple the active and passive beamforming, SDR to relax the rank constraints, and SCA to estimate the sub-optimal results.    \\
\hline
Heuristic algorithms & It applies heuristic rules to find a trade-off between optimality and computational complexity. Heuristic algorithms focus on local optima and low-complexity solutions.  & Heuristic algorithms have much lower computational complexity. It has few requirements for the properties of target problems.  &  It only presents local optima in the current stage, indicating a bad performance in some cases.   & Heuristic rules should be carefully selected and designed, directly affecting the algorithm performance.  &  Considering the high complexity of RIS control problems, heuristic algorithms can provide low-complexity alternatives, i.e., sequential phase shift and on/off control using greedy rule, phase-shift optimization using GA.     \\
\hline
ML techniques & ML techniques are usually data-driven, providing unified control and optimization algorithms for certain types of problems. Most algorithms are easily applied without requiring dedicated design.    &  Data-driven approaches avoid the complexity of building dedicated optimization models. It can better adapt to the dynamic wireless environment given the learning capability.  & It may require long iteration numbers for the algorithm training. ML optimization techniques do not guarantee optimality.   &  Algorithm training is the main difficulty of applying ML, which is data and computation-demanding.  & Various ML techniques have been applied for RIS-related optimizations, e.g., neural networks for CSI prediction and RIS phase control, and DDPG for continuous RIS phase-shift optimization.   \\
\hline
\end{tabular}}
\label{tab-overallcom}
\vspace{0pt}
\end{table*}


Hierarchical learning is another technique that can be used for optimizing RIS-aided wireless networks. The main idea of hierarchical learning is to decouple the long-term task into multiple achievable goals to increase exploration efficiency\cite{pateria2021hierarchical}. In particular, it defines a meta-controller to select goals and a sub-controller to achieve these goals.
Based on the short-term performance of the sub-controller, the meta-controller can adjust the goal dynamically to guarantee the long-term performance of the whole system. Hierarchical learning can also be applied to optimization problems that include multiple control variables with different time scales\cite{zhou2023hierarchical}.  
For instance, in \cite{zhou2023hierarchical}, Zhou \textit{et al.} considered a meta-controller for sleep control, and sub-controllers for transmission power and RIS control, enabling control variables with different time scales. 

Fig. \ref{fig-hrl} shows how to apply hierarchical reinforcement learning to RIS-aided wireless networks. The agent consists of a meta-controller and a sub-controller. 
The sub-controller can produce long-term policy instructions for the sub-controller, such as the maximum number of active RIS elements that is available. Then, given high-level goals, the sub-controllers can select short-term decisions for RIS phase shifts. Meanwhile, the meta-controller focuses on average power consumption in a period as long-term network performance, and the sub-controller accounts for delay or data rate as instant metrics. This scheme can coordinate control variables with different time scales, balancing instant and long-term network metrics. More specifically, the state-action value of the meta-controller is updated by:
\begin{equation} \label{eq-hrl1}
\resizebox{0.89\hsize}{!}{$\begin{aligned}
&Q_{meta}^{new}(s_{meta},g_{meta}) = Q_{meta}^{old}(s_{meta},g_{meta})+\\
&\alpha(r_{ex}+\gamma \max\limits_{g} Q_{meta}(s_{meta}',g)-Q_{meta}^{old}(s_{meta},g_{meta})),
\end{aligned}$}
\end{equation}
where $s_{meta}$ and $s_{meta}'$ is the current and next meta-states, $g_{meta}$ is the goal, and $r_{ex}$ is the extrinsic reward, respectively. $Q^{old}_{meta}$ and $Q^{new}_{meta}$ are old and new state-action values for the meta-controller, indicating the accumulated reward by selecting $g_{meta}$ under state $s_{meta}$.  



Similarly, the Q-value of the sub-controller is updated by
\begin{equation} \label{eq-hrl2}
\resizebox{0.89\hsize}{!}{$\begin{aligned}
Q&_{sub}^{new}(s_{sub},g_{meta},a_{sub}) = Q_{sub}^{old}(s_{sub},g_{meta},a_{sub})+\\
&\alpha(r_{in}+\gamma \max\limits_{a} Q_{sub}(s_{sub}',g_{meta},a)-Q_{sub}^{old}(s_{sub},g_{meta},a_{sub})),
\end{aligned}$}
\end{equation}
where $s_{sub}$ and $s_{sub}'$ are current and the next sub-states, $a_{sub}$ is the action, and $r_{in}$ is the intrinsic reward. $Q^{new}_{sub}$ and $Q^{old}_{sub}$ are defined similarly as the meta-controller, indicating the expected reward of selecting $a_{sub}$ under state $s_{sub}$ and goal $g_{meta}$. Equation (\ref{eq-hrl2}) shows that the sub-controller is under the policy control of the meta-controller.





\subsection{Analyses and Discussion }

ML offers promising opportunities for optimizing RIS-aided wireless communications. Table \ref{tab-mlsummary} overviews various ML techniques \footnote{Note that there are many ML algorithms applied to wireless communications. Instead of collecting all the existing ML algorithms, Table \ref{tab-mlsummary} provides a compressed taxonomy to understand the feature of each technique along with RIS control applications.}.  
Supervised learning applies neural networks for prediction-based control, utilizing CSI to predict achievable data rate, but it relies on fine-grained labelled datasets\cite{alexandropoulos2020phase}- \cite{hu2021reconfigurable}. Meanwhile, neural networks may be used in an unsupervised manner by involving the objective function in the loss function\cite{nguyen2021machine}-\cite{gao2020unsupervised}. Such an unsupervised learning approach can reduce the dependence on labelled datasets. 

RL is the most widely applied ML technique for optimization problems, and each RL algorithm has its own features and difficulties. For example, DDPG can handle continuous action space of RIS but can be unstable\cite{yang2020deep, guo2021learning, huang2020hybrid,lin2020deep}, and DDQN can prevent overestimation but sampling efficiency is low \cite{liu2020ris}.
FL and graph learning are newly emerging ML techniques. Most existing works consider RIS-enhanced AirFL, demonstrating that RISs can improve the training efficiency and performance of FL \cite{liu2021reconfigurable, yang2021reconfigurable, ni2022star,ni2021federated,zhang2021energy,battiloro2022dynamic}. 
Graph learning has shown great potential in many other fields, and wireless network applications include power control and interference management \cite{naderializadeh2020wireless}, resource allocation \cite{eisen2020optimal,jiang2020dynamic}, and network slicing \cite{wang2020graph}.
Finally, transfer learning and hierarchical learning are promising ML techniques for RIS-aided wireless networks. Transfer learning can reduce the model training efforts, while hierarchical learning provides a novel architecture for applying ML to wireless communications, especially when optimization parameters have different timescales. However, more research is needed on both techniques as they are used for RIS-aided wireless networks.    



\begin{figure*}[!t]
\centering
\includegraphics[width=1\linewidth]{Image/fig-compar.jpg}
\caption{Comparison between model-based algorithms, heuristic methods, and ML-based algorithms.}
\label{fig-compar}
\setlength{\abovecaptionskip}{-2pt} 
\vspace{-10pt}
\end{figure*}



\section{Comparison between Model-based, Heuristic and ML Approaches}
\label{sec-compa}


This work has introduced three types of optimization techniques: model-based, heuristic, and ML approaches. One intuitive question is how to evaluate the advantages and difficulties of these techniques, which can help to select the most efficient algorithm for specific problems. To answer this question, we compare these approaches in Table \ref{tab-overallcom}, including main features, advantages, drawbacks, difficulties, and applications for RIS.

1) \textbf{Model-based method:} Table \ref{tab-overallcom} shows that model-based methods usually require full knowledge of the defined problem for transformation and approximation. Model-based approaches can provide efficient and stable solutions once the problem is properly reformulated. However, model-based algorithms are usually complicated to design, indicating a series of transformations, e.g., decoupling the denominator and numerator in SINR terms and relaxing integer constraints.

2) \textbf{Heuristic algorithms:} The primary benefit of the heuristic method is the low implementation complexity, decoupling the problem into multiple stages and solving sequentially, i.e., optimizing RIS control in an element-by-element manner. Heuristic algorithms achieve a trade-off between optimality and low computational complexity. 

3) \textbf{ML algorithms:} Data-driven ML algorithms present unified optimization schemes, applying to diverse problems without dedicated design. Most optimization problems can be converted into unified MDPs that include state, action, transition probability and rewards, and then RL is utilized to maximize the reward for a higher sum-rate or energy efficiency.  





Additionally, Fig. \ref{fig-compar} compares the advantage and disadvantages of these methods, including stability, optimality, generalization capability, robustness for the environment uncertainty, algorithm design/tuning difficulty, and computational resources requirements\footnote{Note that the objective of Fig. \ref{fig-compar} is not to demonstrate which technique is the best, but to provide insights into the features of diverse optimization techniques.}.

1) \textbf{Optimality:} Optimality indicates the quality of solutions, which is one of the most important metrics for optimization problems. Given full knowledge of the problem, model-based methods can usually achieve the best solution with detailed proofs. Meanwhile, ML algorithms can explore the solution space efficiently for optimization. RL takes $\epsilon$-greedy policy for action selection, and neural networks can predict the performance for unseen data. Finally, heuristic methods may find local optima instead of global optimality. 

2) \textbf{Generalization capability:} Generalization capability indicates how the optimization model can adapt to new problems. ML algorithms apply unified control and optimization models with good generalization capability. For instance, given labelled or unlabelled datasets, neural networks are easily deployed in supervised or unsupervised learning manners. In contrast, model-based methods have low generalization capability, since the optimization model requires case-by-case design. Some heuristic methods also show a high generalization capability. GA and PSO are considered examples by applying unified fitness functions to represent the optimization objective. 

3) \textbf{Robustness for the environment uncertainty:} Wireless networks are highly dynamic, and hence optimization techniques must be robust to environment uncertainties. With the learning capability, ML algorithms can adapt well to dynamic environments. Meanwhile, some heuristic algorithms can also adapt to environment changes by applying heuristic rules. However, such uncertainty may greatly affect the performance of model-based algorithms, since they require full knowledge of the optimization problem. One possible solution is to assume environment changes follow some specific distributions, but the optimization over distributions will further increase the complexity. Another solution is to use Monte Carlo sampling and repeat the optimization to achieve average results, which is time-consuming.   





4) \textbf{Algorithm design/tuning difficulty:} This metric refers to the difficulty of designing or tuning algorithms. ML methods apply unified frameworks with low design difficulty. However, some key parameters, such as the number of neural network layers and learning rates, increase the difficulty of algorithm tuning. Similarly, meta-heuristic algorithms are also sensitive to key parameters, e.g., population numbers and inertia weight. But other heuristic methods, especially greedy algorithms and matching theory, can be easily designed with little tuning effort.  
Model-based methods are usually complicated to design because it has stringent requirements for problem forms. Therefore, diverse transformation, relaxation and approximation techniques must be applied, increasing the algorithm design difficulty. 

5) \textbf{Computational resources requirements:} This metric evaluates the number of computing resources (time and space) that are consumed when running a particular algorithm. ML algorithms are usually computation-demanding, requiring a large number of computational resources for model training. For instance, iterative exploration of RL and backpropagation for neural network training are time-consuming. Many model-based methods need iterations to solve the problem, and the matrix operations also contribute to storage space requirements. Heuristic methods have low computational resource requirements because heuristic rules already simplify the difficulty of problem-solving. 

6) \textbf{Stability:} Stability is another critical metric to evaluate whether the optimization algorithm can provide a stable output. With full knowledge of the problem, model-based methods can present a very stable result, especially when closed-form expressions are obtained. ML algorithms can usually achieve good performance after long exploration, but the tedious training iterations will undermine the stability of the results. Most heuristic algorithms have low stability because heuristic rules do not guarantee a stable output.

In summary, model-based solutions have high optimality and stability, but their generalization capability is low. Such algorithms are hard to design and vulnerable to environment uncertainties. Heuristic algorithms focus on feasible solutions with low design difficulty and computational resource requirements, but optimality and stability cannot be guaranteed. Finally, ML algorithms present high optimality, generalization capability, and robustness for dynamic environments. But ML techniques are computation-demanding with moderate algorithm training and tuning challenges.      




\section{RIS-assisted Future 6G Applications and Challenges}
\label{sec-futu}

%\begin{figure}[!t]
%\centering
%\includegraphics[width=0.6\linewidth]{Image/fig-noma.jpg}
%\caption{ Illustration of cluster-based RIS-aided NOMA.}
%\label{fig-noma}
%\setlength{\abovecaptionskip}{-2pt} 
%\vspace{-10pt}
%\end{figure}

This section will introduce RIS-assisted applications towards 6G networks, including NOMA, SWIPT, mmWave and THz communications, NTNs, V2X communications, and ISAC. In addition, we identify several research challenges for the optimization of RIS-aided wireless communications.   

\subsection{RISs and NOMA}
NOMA is a key technique for 5G beyond and 6G networks, enabling spectrum sharing among users, e.g., multiple users can use the same time and frequency resource blocks. Compared with conventional orthogonal multiple access, NOMA improves user fairness and spectral efficiency and increases reliability. 
Resource allocation is a crucial part of RIS-NOMA systems to maximize data rate\cite{mu2021capacity} or minimize power consumption\cite{li2020joint,xie2021joint} by power and subchannels allocation and RISs configuration. For example, in \cite{mu2021joint}, Mu \textit{et al.} compares NOMA with TDMA and FDMA in RIS-aided wireless communications, demonstrating that RIS-NOMA can achieve a higher sum-rate. An AO-based method is proposed in \cite{xie2021joint} for transmit power minimization by jointly optimizing the beamforming, power allocation and RIS phase shifts. 
In addition, due to the resource-sharing nature, the NOMA system is more vulnerable to security issues. Then RISs may be applied to reshape the signal propagation environment for security services against eavesdroppers, including secrecy rate maximization\cite{miao,zheng,dong2020enhancing}, artificial jamming signals\cite{guan2020intelligent,wang2020intelligent}, and creating signal or constellation overlapping\cite{lv2020secure}. 

Although existing studies have demonstrated the potential of RIS-NOMA system\cite{mu2021capacity,li2020joint,liu2022reconfigurable,mu2021intelligent2}, such integration also increases the control and optimization complexity. In RIS-NOMA systems, the decoding order may be frequently changed due to the dynamic RIS configuration, increasing the difficulty of applying conventional model-based algorithms. Therefore, ML techniques are used to overcome the high complexity, i.e., DDQN\cite{liu2020ris}, DDPG\cite{yang2020deep,yang2021machine}. RIS-NOMA systems have shown great potential, but the increased management complexity should be well considered.                 


\subsection{RIS-aided SWIPT}
The future IoT networks will connect a huge number of devices for information exchange. One of the main challenges is the short battery life of IoT devices\cite{perera2017simultaneous}. SWIPT is an attractive solution to transmit electricity without using physical wire links, increasing the mobility, reliability, and safety of electronic devices. However, the low energy efficiency at the energy receiver is one of the main issues for practical SWIPT deployment, and RISs become a promising solution, e.g., increasing sum-rate\cite{lyu2020intelligent,chu2021intelligent,pan2020intelligent,zheng2020intelligent}, reducing transmit power \cite{wu2020joint22,wu2019weighted22}, and maximizing the minimum received power \cite{tang2020joint}. A sum-rate maximization problem is formulated in \cite{lyu2020intelligent} by applying RISs to improve the efficiency of energy and information transmission, achieving a much higher sum-rate than baseline schemes. Similarly, In \cite{chu2021intelligent},  Chu \textit{et al.} investigated the throughput maximization problem for a wireless-powered sensor network, jointly optimizing the RIS phase shifts and transmission time allocation of TDMA. RIS-aided SWIPT is investigated in \cite{pan2020intelligent,wu2020joint22,wu2019weighted22} by joint active and passive beamforming, and various model-based methods are applied for optimization, including BCD, SCA, QCQP \cite{pan2020intelligent}, AO and SDR \cite{wu2020joint22,wu2019weighted22}. Different from former studies, the authors in \cite{zheng2020intelligent} and \cite{tang2020joint} considered max-min problems to guarantee the system performance, in which \cite{zheng2020intelligent} maximized the minimum throughput, and \cite{tang2020joint} improved the minimum received power. 

Note that the above studies mainly apply model-based methods for joint optimization of RISs and wireless power transfer, requiring dedicated model design. Although some low-complexity algorithms are proposed in these works, i.e., MM \cite{chu2021intelligent} and bi-section search\cite{pan2020intelligent}, they still require full knowledge of the defined problem. By contrast, ML approaches can be promising alternatives to handle the complexity of joint optimizing RISs and SWIPT, which is still an open issue.   


\subsection{RIS-empowered mmWave and THz Communications}
The increasing traffic demand and scarce bandwidth resources make mmWave and THz communications become appealing techniques. However, mmWave and THz communications are very vulnerable to signal blockages and attenuation, leading to severe path loss and reduced cover range. To this end, RISs can be applied to manipulate the signal propagation environment when the direct transmission is blocked. Indoor RIS-mmWave is investigated in \cite{tan2018enabling} and \cite{perovic2020channel}. Specifically, the authors in \cite{tan2018enabling} introduced the design, modelling and optimization of reconfigurable 60 GHz reflect-array, while two optimization schemes are proposed in \cite{perovic2020channel} to maximize the channel capacity for indoor mmWave.
For RIS-aided THz systems, the coverage of THz communications is analyzed in \cite{boulogeorgos2021coverage}, revealing a minimum transmission power that guarantees 100\% coverage probability. In addition, various optimization schemes are proposed to configure the THz system performance, such as data rate maximization, sum-rate maximization for indoor THz\cite{pan2022sum}, and energy efficiency maximization \cite{wu2021energy}. 

These above studies demonstrate that RISs have great potential to enhance the performance of mmWave and THz communications, increasing the channel capacity\cite{perovic2020channel}, received power, data rate\cite{pan2022sum}, and energy efficiency\cite{wu2021energy}. While it is promising to apply RISs for mmWave and THz communications, there are still some open challenges. One issue is the accurate and practical channel estimation for RIS-mmWave or RIS-THz communications, which is the fundamental of identifying the performance limit and following optimizations.     


\subsection{RISs and Nonterrestrial Communications}
The NTN is an important part of future wireless networks, complementing the limitations of conventional terrestrial networks. NTNs can provide flexible and reliable support for communications in remote areas by UAVs, high-altitude platforms (HAPs) and low earth orbit satellites.  

Applying UAVs as flying BSs or relays for wireless communications has been widely studied, and the primary motivation is that UAVs with high mobility can change their positions dynamically to adapt to diverse wireless environments\cite{zeng2016wireless}. However, frequent repositioning will increase the UAV power consumption, especially considering that UAVs are powered by a battery (usually under 30 minutes)\cite{liu2020machine}. In this case, RISs can be applied to overcome this challenge, in which one can configure the RIS phase shifts instead of moving UAVs to save energy. In \cite{ma2020enhancing}, Ma \textit{et al.} analyzed the performance of RIS-aided cellular communications with UAVs, and the simulations reveal that RISs can provide a 21dB gain for UAVs. Joint UAV trajectory design and RIS passive beamforming are studied in \cite{li2020reconfigurable, liu2020machine}. UAV communications are highly dynamic due to the mobility, and ML algorithms may be applied to handle such uncertainty. %Although existing studies have shown the advantage of ML algorithms, the joint control of UAVs and RISs still requires more advanced ML techniques for fast and efficient optimization.  

Furthermore, HAPs and low earth orbit satellites have attracted the interest of academia and industry.   
RISs can perfectly match the requirements of these platforms due to the small size, low power consumption and cost features\cite{tekbiyik2022reconfigurable}. Meanwhile, integrating RISs into these systems can increase the channel capacity and received signal power and reduce the bit error rate. For example, the simulations in \cite{tekbiyik2022reconfigurable} demonstrate that RISs can diminish solar scintillation with a 945.4 km inter-satellite distance. In \cite{tekbiyik2021energy},  Tekbiyik \textit{et al.} applied RISs for the communications between IoT devices and satellites, achieving up to $10^5$ times transmission rates. Nevertheless, RIS-aided HAPs and satellites are in the very early stage, and many fundamental problems, such as channel estimation and resource allocation, are still open issues.   


\subsection{RISs and V2X Communications} 
V2X is a key paradigm for the envisioned 6G networks, enabling higher road safety, traffic efficiency, and intelligent transportation. 
Latency and reliability are the most critical requirements for V2X communications due to the importance of road safety. 
However, the V2X transmission can be unstable due to fast-moving vehicles and the dynamic nature of wireless communications. Therefore, RISs can be exploited to improve channel capacity, coverage, signal strength and reliability. In particular, 
the authors in \cite{wang2020outage} obtained the outage probability expression by using series expansion and central limit theorem, showing that RISs can significantly reduce the outage probability for vehicle communications. The physical layer security of vehicle networks is studied in \cite{makarfi2020physical}, and the simulations demonstrate that RISs can improve the secrecy capacity. 
Considering the high mobility of vehicles, RIS placement is important to guarantee the coverage of vehicle communications. An optimum RIS placement scheme is proposed in \cite{ozcan2021reconfigurable} for highway mmWave and THz vehicle communications, combating the high path loss in high frequency bands communications. Resource allocation is a crucial part of V2X communications\cite{chen2020resource,al2022reconfigurable}. In \cite{chen2020resource},  Chen \textit{et al.} jointly considered power and spectrum allocation and RIS configuration to maximize the V2X link capacity, and simulations show that RISs can increase 42.36\% higher channel capacity. 

A critical feature of V2X communications is the stringent requirement for reliability and safety, which means the proposed algorithm should guarantee the worst-case network performance. In addition, it indicates that the control and optimization algorithms should be robust, reliable and stable, which is still a challenge for current studies.    



\subsection{RIS-aided ISAC}
ISAC is recently emerging as a key technology to support ubiquitous wireless connectivity and accurate sensing\cite{liu2022survey22}. Target detection and parameter estimation are two primary tasks in radar sensing\cite{yao2022joint}, and RISs can be deployed to provide virtual LoS signal transmission, enabling the radar to sense targets in blocked areas\cite{10050406}. Radar communication coexistence (RCC) and dual-functional radar communication (DFRC) are two typical ISAC applications. The RCC system indicates that sharing the spectrum between co-existing radar and communication transmissions, and hence the interference between radar and communication transmitters should be carefully managed. RISs provide an attractive solution to suppress the interference by manipulating the phase shifts. For instance, the authors in \cite{he2022ris} deploy one RIS near the BS to suppress the interference on radar, and another RIS on the user side to control the interference from the radar.
On the other hand, the DFRC system applies fully-shared hardware and a unified transmit waveform to simultaneously implement sensing and communication functions. RISs can provide additional spatial degrees of freedom (DoFs) for the dual-functional waveform design, improving the sensing and communication performance simultaneously\cite{liu2022integrated}.





\subsection{Challenges and Future Directions}
This subsection identifies research challenges and possible future directions. 

1) Robust and Practical Control Algorithms:  
Uncertainty is one of the main features of wireless communications due to dynamic channel conditions and diverse user demands. Most existing studies rely on ideal assumptions, such as perfect CSI acquisition and static user conditions, which are hard to achieve in practice. ML algorithms are more robust to environment uncertainties than conventional optimization techniques, but some important questions, e.g., algorithm deployment, offline or online training, and training cost, are neglected in many existing studies. Robust and practical algorithms are of great importance for the real-world deployment of RISs, which requires more research efforts.   

2) Low-complexity and Low-overhead Optimization:
The joint optimization problems of RISs are usually first decoupled into multiple sub-problems, and then each sub-problem is handled by model-based, heuristic or ML algorithms. However, one important problem is the complexity and overhead of proposed algorithms. Some studies analyzed the algorithm complexity in terms of big $O$ notations, but the communication and control overhead is not investigated. For example, frequent parameter exchange between the BS and RISs may lead to high overhead, and the model training overhead of ML algorithms can hamper the system efficiency. These issues are still open challenges, and agile optimization algorithms with low complexity and overhead are yet to be developed.    

3) Graph Learning-based Control and Optimization:
Graph learning is an attractive ML algorithm to utilize the graph nature of wireless networks, but existing studies are limited to GNNs. The unique capability of graph learning can unveil the structural relations among wireless network elements, thus capturing more insights than analyzing data in an isolated grid manner.
There are more advanced graph learning techniques, such as graph generative networks and graph convolutional networks, that have not been investigated for RIS-aided wireless communications. For example, graph convolutional networks have shown great power to learn graph representations and achieved satisfying performance in a wide range of real-world applications, but the applications to wireless communications are still an open issue.


4) Flexible Control and Optimization Framework:
The former analyses have shown that each optimization approach has its advantages and difficulties.  Model-based methods have higher stability and optimality, and heuristic methods have lower complexity, while ML techniques are more robust. 
One intuitive direction is to combine these methods to form a flexible optimization framework that can make the most of each approach's advantages and complement the difficulties. For example, the joint optimization problem can be first decoupled into multiple sub-problems, and each sub-problem is solved by diverse optimization approaches that best fit the requirements. However, many existing studies stick with one type of optimization technique, and flexible control schemes are considered future challenges.     





\section{Conclusion}
\label{sec-con}
RIS technology is a key enabler for 6G networks, and control and optimization techniques are critical to exploiting the full potential of RISs. In this work, we have surveyed various approaches for optimizing RIS-aided wireless networks, including model-based, heuristic, and ML approaches. We have provided in-depth analyses of the algorithms' features, difficulties, and applications towards RIS, and we have further compared the advantages and disadvantages of nearly 20 techniques. Our analyses reveal that model-based methods exhibit satisfying performance and stability, but the algorithm design is complicated with low generalization capability. Heuristic algorithms can obtain low-complexity sub-optimal solutions, which are usually considered as baselines or supplements for other techniques. ML techniques have high generalization capability and optimality, but ML model training is computationally demanding and requires experience. Finally, the algorithm selection depends on specific optimization requirements, which should be jointly considered based on application scenarios.
It is hoped that this survey will serve as a roadmap for researchers to investigate advanced optimization techniques for RIS-aided wireless networks.     



%\red{The ref papers have been checked and revised, including "Proc. " for conference papers, abbreviations for journal names, capitalization in titles, and so on.}


\normalem
\bibliographystyle{IEEEtran}
\bibliography{Reference}

\end{document}
