

\section{RIS-assisted Future 6G Applications and Challenges}
\label{sec-futu}

%\begin{figure}[!t]
%\centering
%\includegraphics[width=0.6\linewidth]{Image/fig-noma.jpg}
%\caption{ Illustration of cluster-based RIS-aided NOMA.}
%\label{fig-noma}
%\setlength{\abovecaptionskip}{-2pt} 
%\vspace{-10pt}
%\end{figure}

This section will introduce RIS-assisted applications towards 6G networks, including NOMA, SWIPT, mmWave and THz communications, NTNs, V2X communications, and ISAC. In addition, we identify several research challenges for the optimization of RIS-aided wireless communications.   

\subsection{RISs and NOMA}
NOMA is a key technique for 5G beyond and 6G networks, enabling spectrum sharing among users, e.g., multiple users can use the same time and frequency resource blocks. Compared with conventional orthogonal multiple access, NOMA improves user fairness and spectral efficiency and increases reliability. 
Resource allocation is a crucial part of RIS-NOMA systems to maximize data rate\cite{mu2021capacity} or minimize power consumption\cite{li2020joint,xie2021joint} by power and subchannels allocation and RISs configuration. For example, in \cite{mu2021joint}, Mu \textit{et al.} compares NOMA with TDMA and FDMA in RIS-aided wireless communications, demonstrating that RIS-NOMA can achieve a higher sum-rate. An AO-based method is proposed in \cite{xie2021joint} for transmit power minimization by jointly optimizing the beamforming, power allocation and RIS phase shifts. 
In addition, due to the resource-sharing nature, the NOMA system is more vulnerable to security issues. Then RISs may be applied to reshape the signal propagation environment for security services against eavesdroppers, including secrecy rate maximization\cite{miao,zheng,dong2020enhancing}, artificial jamming signals\cite{guan2020intelligent,wang2020intelligent}, and creating signal or constellation overlapping\cite{lv2020secure}. 

Although existing studies have demonstrated the potential of RIS-NOMA system\cite{mu2021capacity,li2020joint,liu2022reconfigurable,mu2021intelligent2}, such integration also increases the control and optimization complexity. In RIS-NOMA systems, the decoding order may be frequently changed due to the dynamic RIS configuration, increasing the difficulty of applying conventional model-based algorithms. Therefore, ML techniques are used to overcome the high complexity, i.e., DDQN\cite{liu2020ris}, DDPG\cite{yang2020deep,yang2021machine}. RIS-NOMA systems have shown great potential, but the increased management complexity should be well considered.                 


\subsection{RIS-aided SWIPT}
The future IoT networks will connect a huge number of devices for information exchange. One of the main challenges is the short battery life of IoT devices\cite{perera2017simultaneous}. SWIPT is an attractive solution to transmit electricity without using physical wire links, increasing the mobility, reliability, and safety of electronic devices. However, the low energy efficiency at the energy receiver is one of the main issues for practical SWIPT deployment, and RISs become a promising solution, e.g., increasing sum-rate\cite{lyu2020intelligent,chu2021intelligent,pan2020intelligent,zheng2020intelligent}, reducing transmit power \cite{wu2020joint22,wu2019weighted22}, and maximizing the minimum received power \cite{tang2020joint}. A sum-rate maximization problem is formulated in \cite{lyu2020intelligent} by applying RISs to improve the efficiency of energy and information transmission, achieving a much higher sum-rate than baseline schemes. Similarly, In \cite{chu2021intelligent},  Chu \textit{et al.} investigated the throughput maximization problem for a wireless-powered sensor network, jointly optimizing the RIS phase shifts and transmission time allocation of TDMA. RIS-aided SWIPT is investigated in \cite{pan2020intelligent,wu2020joint22,wu2019weighted22} by joint active and passive beamforming, and various model-based methods are applied for optimization, including BCD, SCA, QCQP \cite{pan2020intelligent}, AO and SDR \cite{wu2020joint22,wu2019weighted22}. Different from former studies, the authors in \cite{zheng2020intelligent} and \cite{tang2020joint} considered max-min problems to guarantee the system performance, in which \cite{zheng2020intelligent} maximized the minimum throughput, and \cite{tang2020joint} improved the minimum received power. 

Note that the above studies mainly apply model-based methods for joint optimization of RISs and wireless power transfer, requiring dedicated model design. Although some low-complexity algorithms are proposed in these works, i.e., MM \cite{chu2021intelligent} and bi-section search\cite{pan2020intelligent}, they still require full knowledge of the defined problem. By contrast, ML approaches can be promising alternatives to handle the complexity of joint optimizing RISs and SWIPT, which is still an open issue.   


\subsection{RIS-empowered mmWave and THz Communications}
The increasing traffic demand and scarce bandwidth resources make mmWave and THz communications become appealing techniques. However, mmWave and THz communications are very vulnerable to signal blockages and attenuation, leading to severe path loss and reduced cover range. To this end, RISs can be applied to manipulate the signal propagation environment when the direct transmission is blocked. Indoor RIS-mmWave is investigated in \cite{tan2018enabling} and \cite{perovic2020channel}. Specifically, the authors in \cite{tan2018enabling} introduced the design, modelling and optimization of reconfigurable 60 GHz reflect-array, while two optimization schemes are proposed in \cite{perovic2020channel} to maximize the channel capacity for indoor mmWave.
For RIS-aided THz systems, the coverage of THz communications is analyzed in \cite{boulogeorgos2021coverage}, revealing a minimum transmission power that guarantees 100\% coverage probability. In addition, various optimization schemes are proposed to configure the THz system performance, such as data rate maximization, sum-rate maximization for indoor THz\cite{pan2022sum}, and energy efficiency maximization \cite{wu2021energy}. 

These above studies demonstrate that RISs have great potential to enhance the performance of mmWave and THz communications, increasing the channel capacity\cite{perovic2020channel}, received power, data rate\cite{pan2022sum}, and energy efficiency\cite{wu2021energy}. While it is promising to apply RISs for mmWave and THz communications, there are still some open challenges. One issue is the accurate and practical channel estimation for RIS-mmWave or RIS-THz communications, which is the fundamental of identifying the performance limit and following optimizations.     


\subsection{RISs and Nonterrestrial Communications}
The NTN is an important part of future wireless networks, complementing the limitations of conventional terrestrial networks. NTNs can provide flexible and reliable support for communications in remote areas by UAVs, high-altitude platforms (HAPs) and low earth orbit satellites.  

Applying UAVs as flying BSs or relays for wireless communications has been widely studied, and the primary motivation is that UAVs with high mobility can change their positions dynamically to adapt to diverse wireless environments\cite{zeng2016wireless}. However, frequent repositioning will increase the UAV power consumption, especially considering that UAVs are powered by a battery (usually under 30 minutes)\cite{liu2020machine}. In this case, RISs can be applied to overcome this challenge, in which one can configure the RIS phase shifts instead of moving UAVs to save energy. In \cite{ma2020enhancing}, Ma \textit{et al.} analyzed the performance of RIS-aided cellular communications with UAVs, and the simulations reveal that RISs can provide a 21dB gain for UAVs. Joint UAV trajectory design and RIS passive beamforming are studied in \cite{li2020reconfigurable, liu2020machine}. UAV communications are highly dynamic due to the mobility, and ML algorithms may be applied to handle such uncertainty. %Although existing studies have shown the advantage of ML algorithms, the joint control of UAVs and RISs still requires more advanced ML techniques for fast and efficient optimization.  

Furthermore, HAPs and low earth orbit satellites have attracted the interest of academia and industry.   
RISs can perfectly match the requirements of these platforms due to the small size, low power consumption and cost features\cite{tekbiyik2022reconfigurable}. Meanwhile, integrating RISs into these systems can increase the channel capacity and received signal power and reduce the bit error rate. For example, the simulations in \cite{tekbiyik2022reconfigurable} demonstrate that RISs can diminish solar scintillation with a 945.4 km inter-satellite distance. In \cite{tekbiyik2021energy},  Tekbiyik \textit{et al.} applied RISs for the communications between IoT devices and satellites, achieving up to $10^5$ times transmission rates. Nevertheless, RIS-aided HAPs and satellites are in the very early stage, and many fundamental problems, such as channel estimation and resource allocation, are still open issues.   


\subsection{RISs and V2X Communications} 
V2X is a key paradigm for the envisioned 6G networks, enabling higher road safety, traffic efficiency, and intelligent transportation. 
Latency and reliability are the most critical requirements for V2X communications due to the importance of road safety. 
However, the V2X transmission can be unstable due to fast-moving vehicles and the dynamic nature of wireless communications. Therefore, RISs can be exploited to improve channel capacity, coverage, signal strength and reliability. In particular, 
the authors in \cite{wang2020outage} obtained the outage probability expression by using series expansion and central limit theorem, showing that RISs can significantly reduce the outage probability for vehicle communications. The physical layer security of vehicle networks is studied in \cite{makarfi2020physical}, and the simulations demonstrate that RISs can improve the secrecy capacity. 
Considering the high mobility of vehicles, RIS placement is important to guarantee the coverage of vehicle communications. An optimum RIS placement scheme is proposed in \cite{ozcan2021reconfigurable} for highway mmWave and THz vehicle communications, combating the high path loss in high frequency bands communications. Resource allocation is a crucial part of V2X communications\cite{chen2020resource,al2022reconfigurable}. In \cite{chen2020resource},  Chen \textit{et al.} jointly considered power and spectrum allocation and RIS configuration to maximize the V2X link capacity, and simulations show that RISs can increase 42.36\% higher channel capacity. 

A critical feature of V2X communications is the stringent requirement for reliability and safety, which means the proposed algorithm should guarantee the worst-case network performance. In addition, it indicates that the control and optimization algorithms should be robust, reliable and stable, which is still a challenge for current studies.    



\subsection{RIS-aided ISAC}
ISAC is recently emerging as a key technology to support ubiquitous wireless connectivity and accurate sensing\cite{liu2022survey22}. Target detection and parameter estimation are two primary tasks in radar sensing\cite{yao2022joint}, and RISs can be deployed to provide virtual LoS signal transmission, enabling the radar to sense targets in blocked areas\cite{10050406}. Radar communication coexistence (RCC) and dual-functional radar communication (DFRC) are two typical ISAC applications. The RCC system indicates that sharing the spectrum between co-existing radar and communication transmissions, and hence the interference between radar and communication transmitters should be carefully managed. RISs provide an attractive solution to suppress the interference by manipulating the phase shifts. For instance, the authors in \cite{he2022ris} deploy one RIS near the BS to suppress the interference on radar, and another RIS on the user side to control the interference from the radar.
On the other hand, the DFRC system applies fully-shared hardware and a unified transmit waveform to simultaneously implement sensing and communication functions. RISs can provide additional spatial degrees of freedom (DoFs) for the dual-functional waveform design, improving the sensing and communication performance simultaneously\cite{liu2022integrated}.





\subsection{Challenges and Future Directions}
This subsection identifies research challenges and possible future directions. 

1) Robust and Practical Control Algorithms:  
Uncertainty is one of the main features of wireless communications due to dynamic channel conditions and diverse user demands. Most existing studies rely on ideal assumptions, such as perfect CSI acquisition and static user conditions, which are hard to achieve in practice. ML algorithms are more robust to environment uncertainties than conventional optimization techniques, but some important questions, e.g., algorithm deployment, offline or online training, and training cost, are neglected in many existing studies. Robust and practical algorithms are of great importance for the real-world deployment of RISs, which requires more research efforts.   

2) Low-complexity and Low-overhead Optimization:
The joint optimization problems of RISs are usually first decoupled into multiple sub-problems, and then each sub-problem is handled by model-based, heuristic or ML algorithms. However, one important problem is the complexity and overhead of proposed algorithms. Some studies analyzed the algorithm complexity in terms of big $O$ notations, but the communication and control overhead is not investigated. For example, frequent parameter exchange between the BS and RISs may lead to high overhead, and the model training overhead of ML algorithms can hamper the system efficiency. These issues are still open challenges, and agile optimization algorithms with low complexity and overhead are yet to be developed.    

3) Graph Learning-based Control and Optimization:
Graph learning is an attractive ML algorithm to utilize the graph nature of wireless networks, but existing studies are limited to GNNs. The unique capability of graph learning can unveil the structural relations among wireless network elements, thus capturing more insights than analyzing data in an isolated grid manner.
There are more advanced graph learning techniques, such as graph generative networks and graph convolutional networks, that have not been investigated for RIS-aided wireless communications. For example, graph convolutional networks have shown great power to learn graph representations and achieved satisfying performance in a wide range of real-world applications, but the applications to wireless communications are still an open issue.


4) Flexible Control and Optimization Framework:
The former analyses have shown that each optimization approach has its advantages and difficulties.  Model-based methods have higher stability and optimality, and heuristic methods have lower complexity, while ML techniques are more robust. 
One intuitive direction is to combine these methods to form a flexible optimization framework that can make the most of each approach's advantages and complement the difficulties. For example, the joint optimization problem can be first decoupled into multiple sub-problems, and each sub-problem is solved by diverse optimization approaches that best fit the requirements. However, many existing studies stick with one type of optimization technique, and flexible control schemes are considered future challenges.     