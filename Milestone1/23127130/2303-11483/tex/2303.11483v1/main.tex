%%
%% This is file `sample-sigconf.tex',
%% generated with the docstrip utility.
%%
%% The original source files were:
%%
%% samples.dtx  (with options: `sigconf')
%% 
%% IMPORTANT NOTICE:
%% 
%% For the copyright see the source file.
%% 
%% Any modified versions of this file must be renamed
%% with new filenames distinct from sample-sigconf.tex.
%% 
%% For distribution of the original source see the terms
%% for copying and modification in the file samples.dtx.
%% 
%% This generated file may be distributed as long as the
%% original source files, as listed above, are part of the
%% same distribution. (The sources need not necessarily be
%% in the same archive or directory.)
%%
%%
%% Commands for TeXCount
%TC:macro \cite [option:text,text]
%TC:macro \citep [option:text,text]
%TC:macro \citet [option:text,text]
%TC:envir table 0 1
%TC:envir table* 0 1
%TC:envir tabular [ignore] word
%TC:envir displaymath 0 word
%TC:envir math 0 word
%TC:envir comment 0 0
%%
%%
%% The first command in your LaTeX source must be the \documentclass
%% command.
%%
%% For submission and review of your manuscript please change the
%% command to \documentclass[manuscript, screen, review]{acmart}.
%%
%% When submitting camera ready or to TAPS, please change the command
%% to \documentclass[sigconf]{acmart} or whichever template is required
%% for your publication.
%%
%%
% IF ANONYMOUS
% \documentclass[sigconf, screen, anonymous, review]{acmart}
% ELSE IF NOT ANONYMOUS BUT REVIEW FRIENDLY
% \documentclass[sigconf, screen, review]{acmart}
% ELSE CAMERA-READY (DEFAULT)
\documentclass[sigconf]{acmart}

%%
%% \BibTeX command to typeset BibTeX logo in the docs
\AtBeginDocument{%
  \providecommand\BibTeX{{%
    Bib\TeX}}}

%% Rights management information.  This information is sent to you
%% when you complete the rights form.  These commands have SAMPLE
%% values in them; it is your responsibility as an author to replace
%% the commands and values with those provided to you when you
%% complete the rights form.
% \setcopyright{acmcopyright}
% \copyrightyear{2023}
% \acmYear{2023}
% \acmDOI{XXXXXXX.XXXXXXX}

% %% These commands are for a PROCEEDINGS abstract or paper.
% \acmConference[DESTION '23]{5th Workshop on Design Automation for CPS and IoT}{May 09, 2023}{San Antonio, TX}
% %%
% %%  Uncomment \acmBooktitle if the title of the proceedings is different
% %%  from ``Proceedings of ...''!
% %%
% %%\acmBooktitle{Woodstock '18: ACM Symposium on Neural Gaze Detection,
% %%  June 03--05, 2018, Woodstock, NY}
% \acmPrice{15.00}
% \acmISBN{978-1-4503-XXXX-X/18/06}


%%
%% Submission ID.
%% Use this when submitting an article to a sponsored event. You'll
%% receive a unique submission ID from the organizers
%% of the event, and this ID should be used as the parameter to this command.
%%\acmSubmissionID{123-A56-BU3}

%%
%% For managing citations, it is recommended to use bibliography
%% files in BibTeX format.
%%
%% You can then either use BibTeX with the ACM-Reference-Format style,
%% or BibLaTeX with the acmnumeric or acmauthoryear sytles, that include
%% support for advanced citation of software artefact from the
%% biblatex-software package, also separately available on CTAN.
%%
%% Look at the sample-*-biblatex.tex files for templates showcasing
%% the biblatex styles.
%%

%%
%% The majority of ACM publications use numbered citations and
%% references.  The command \citestyle{authoryear} switches to the
%% "author year" style.
%%
%% If you are preparing content for an event
%% sponsored by ACM SIGGRAPH, you must use the "author year" style of
%% citations and references.
%% Uncommenting
%% the next command will enable that style.
%%\citestyle{acmauthoryear}

\usepackage[utf8]{inputenc} % allow utf-8 input
\usepackage{hyperref}       % hyperlinks
\usepackage{url}            % simple URL typesetting
\usepackage{booktabs}       % professional-quality tables
\usepackage{amsfonts}       % blackboard math symbols
\usepackage{nicefrac}       % compact symbols for 1/2, etc.
\usepackage{microtype}      % microtypography
\usepackage{xcolor}         % colors
\usepackage{graphicx}
\usepackage{import}
% \usepackage{enumitem, amssymb}
% \usepackage{minted}
% \usepackage{caption}
% \usepackage{subcaption}
% \usepackage{comment}
\usepackage{xspace}
% \usepackage[colorinlistoftodos]{todonotes}
\usepackage{fancyhdr}

% user-defined macros in macros.sty file
\usepackage{macros}

\settopmatter{printacmref=false, printccs=true, printfolios=true} % We want page numbers on submissions
\renewcommand\footnotetextcopyrightpermission[1]{}

% Custom math stuff from Jesse
\DeclareMathOperator*{\argmax}{arg\,max}
\DeclareMathOperator*{\argmin}{arg\,min}
\newcommand{\V}[1]{\mathbf{#1}}

%%
%% end of the preamble, start of the body of the document source.
\begin{document}

%%
%% The "title" command has an optional parameter,
%% allowing the author to define a "short title" to be used in page headers.
\title{Automatic Measures for Evaluating\\ Generative Design Methods for Architects}

%%
%% The "author" command and its associated commands are used to define
%% the authors and their affiliations.
%% Of note is the shared affiliation of the first two authors, and the
%% "authornote" and "authornotemark" commands
%% used to denote shared contribution to the research.

\author{Eric Yeh}
\affiliation{
    \institution{Artificial Intelligence Center\\SRI International}
    \city{Menlo Park}
    \state{CA}
    \country{USA}
}
\email{eric.yeh@sri.com}

\author{Briland Hitaj}
\affiliation{
    \institution{Computer Science Laboratory\\SRI International}
    \city{New York}
    \state{NY}
    \country{USA}
}
\email{briland.hitaj@sri.com}

\author{Vidyasagar Sadhu}
\affiliation{
    \institution{Artificial Intelligence Center\\SRI International}
    \city{Menlo Park}
    \state{CA}
    \country{USA}
}
\email{srikanthvidyasagar.sadhu@sri.com}

\author{Anirban Roy}
\affiliation{
    \institution{Computer Science Laboratory\\SRI International}
    \city{Menlo Park}
    \state{CA}
    \country{USA}
}
\email{anirban.roy@sri.com}

\author{Takuma Nakabayashi}
\affiliation{
    \institution{Obayashi Corporation}
    \city{Menlo Park}
    \state{CA}
    \country{USA}
}
\email{takuma.nakabayashi@obayashi-usa.com}

\author{Yoshito Tsuji}
\affiliation{
    \institution{Obayashi Corporation}
    \city{Tokyo}
    \country{Japan}
}
\email{tsuji.yoshito@obayashi.co.jp}

%%
%% By default, the full list of authors will be used in the page
%% headers. Often, this list is too long, and will overlap
%% other information printed in the page headers. This command allows
%% the author to define a more concise list
%% of authors' names for this purpose.
\renewcommand{\shortauthors}{Yeh et al.}

%%
%% The abstract is a short summary of the work to be presented in the
%% article.
\import{\sectiondir}{abstract.tex}

%%
%% The code below is generated by the tool at http://dl.acm.org/ccs.cfm.
%% Please copy and paste the code instead of the example below.
%%
\begin{CCSXML}
<ccs2012>
<concept>
<concept_id>10010405.10010469.10010472.10010440</concept_id>
<concept_desc>Applied computing~Computer-aided design</concept_desc>
<concept_significance>500</concept_significance>
</concept>
</ccs2012>
\end{CCSXML}

\ccsdesc[500]{Applied computing~Computer-aided design}
\ccsdesc[500]{Applied computing~Computer-aided design}


%%
%% Keywords. The author(s) should pick words that accurately describe
%% the work being presented. Separate the keywords with commas.
\keywords{generative design, architectural design}
%% A "teaser" image appears between the author and affiliation
%% information and the body of the document, and typically spans the
%% page.
%% Excised for now
\begin{teaserfigure}
  \centering
  \includegraphics[width=0.75\textwidth]{overview.pdf}
  \caption{Examples of generated building candidates that meet our identified architectural criteria of realism, matching intent, and structural diversity. Our proposed generative models take conceptual sketches (left) and create a diverse set of realistic architectural designs (right) that match the content and the overall structural intent of the sketch, while exploring a variety of lower-level structural and material differences. }
  %\caption{Our aim is to identify generative models that can take conceptual sketches (left) and create a diverse set of realistic architectural designs (right) that match the content, the overall structural intent of the sketch, while exploring a variety of lower-level structural and material differences.}
  \Description{Examples of generated building candidates that meet our identified architectural criteria of realism, matching intent, and structural diversity. Our proposed generative models take conceptual sketches (left) and create a diverse set of realistic architectural designs (right) that match the content and the overall structural intent of the sketch, while exploring a variety of lower-level structural and material differences.  }
  \label{fig:teaser}
\end{teaserfigure}

%

%%
%% This command processes the author and affiliation and title
%% information and builds the first part of the formatted document.
\maketitle
\pagestyle{fancy}
\fancyhead{}
\fancyfoot{}
\fancyhead[L]{\footnotesize Automatic Measures for Evaluating Generative Design Methods for Architects}
\fancyhead[R]{\footnotesize Yeh et al.}
\fancyfoot[C]{\thepage}

\import{\sectiondir}{introduction.tex}
\import{\sectiondir}{approach.tex}
\import{\sectiondir}{experiments.tex}
\import{\sectiondir}{conclusions.tex}
\import{\sectiondir}{acknowledgments.tex}

%%
%% The next two lines define the bibliography style to be used, and
%% the bibliography file.
\bibliographystyle{ACM-Reference-Format}
\bibliography{main}

%%

\end{document}
\endinput
%%
%% End of file `sample-sigconf.tex'.
