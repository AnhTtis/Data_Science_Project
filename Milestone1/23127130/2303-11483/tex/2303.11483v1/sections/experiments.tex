% !TEX root = ../main.tex
\section{Experiments and Results}
\label{sec:experiements}
%% 

As part of our evaluation,
we elicited $24$ conceptual sketches from a team of architects.  
We then sampled $10$ designs from each method.  
For comparison, we also ran the conceptual sketches ``as-is'' across each of the measures.
Table \ref{tab:experiment} shows performance of different methods against each of the criteria.  Unsurprisingly, using conceptual sketches gives the best match with content distance, with a poorer FID (as sketches do not resemble real images).  Because there is no variation in these sketches, the structural diversity is poor as well.

MUNIT exhibited the best content distance, significantly outperforming the other approaches \footnote{Using a two tail t-test with $\alpha=0.05$} while also having a slight loss in realism  (lower FID).  ControlNet does exhibit a higher structural diversity score, albeit at a cost in content distance.

 \begin{figure}[]
\begin{center}
\includegraphics[width=1.\columnwidth]{figures/pix2pix_averaging.pdf}
\end{center}
\caption{Averaging effects can be observed from a sample of renders produced by Pix2pix.}
\label{fig:pix2pix_averaging}
\end{figure}

Pix2pix is a deterministic mapping that only generates one render from a given sketch, and thus has a zero structural diversity score.  Its realism is also poorer as well.  While one could argue this implementation does not have the advanced decoding capabilities seen in the newer networks, a visual inspection of the validation renders showed an averaging effect, such as  colors being muted to a handful of similar palettes (Figure \ref{fig:pix2pix_averaging}).




%%
