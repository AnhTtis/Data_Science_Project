% !TEX root = ../main.tex
\begin{abstract}
%% What are the key ideas and takeaways readers should know in this abstract?
The recent explosion of high-quality image-to-image methods has prompted interest in applying image-to-image methods towards artistic and design tasks.
Of interest for architects is to use these methods to generate design proposals from conceptual sketches, usually hand-drawn sketches that are quickly developed and can embody a design intent.
More specifically, instantiating a sketch into a visual that can be used to elicit client feedback is typically a time consuming task, and being able to speed up this iteration time is important.
While the body of work in generative methods has been impressive, there has been a mismatch between the quality measures used to evaluate the outputs of these systems and the actual expectations of architects.
In particular, most recent image-based works place an emphasis on realism of generated images.
While important, this is one of several criteria architects look for.
In this work, we describe the expectations architects have for design proposals from conceptual sketches, and identify corresponding automated metrics from the literature.
We then evaluate several image-to-image generative methods that may address these criteria and examine their performance across these metrics.
From these results, we identify certain challenges with hand-drawn conceptual sketches and describe possible future avenues of investigation to address them.
%%
\end{abstract}