% !TEX root = ../main.tex
\section{Introduction}
\label{sec:intro}

%% What are we doing and why is it important?
%% What are our primary scientific claims?
%% How do answering our claims contribute to the importance of the tool?

With the rise of autoregressive and iterative generative models and large available datasets, the quality of generated images has gone up dramatically along with interest in applying these models to real tasks.  
In this work, we study how current generative methods can facilitate exploration of the architectural design space.
While the initial design process varies from person to person, we focus on the common strategy of creating realistic and usable architectural renderings from conceptual sketches.


 \begin{figure}[h]
\begin{center}
\includegraphics[width=.55\columnwidth]{figures/conceptual_sketch1.png}
\end{center}
\caption{A sample conceptual sketch.}
\label{fig:conceptual_sketch}
\end{figure}

Figure \ref{fig:conceptual_sketch} gives an example of a conceptual sketch used in the initial design phase of a project.
Sketches are meant to quickly explore the design space, and as a result they exhibit certain characteristics. 
These sketches are hand-drawn, often sparsely populated, as they are neither intended to be exacting nor definitive.
Indeed, lines and markings made are often not intended to be followed strictly, as with edge maps, and instead use drawn motifs to convey the essence, the broad intents about what should be present.
For example, the horizontal line markings on the left-side facade may be interpreted as a series of wide windows. 
How this will be realized, whether these will be met with large windows with small insets, and the type of material used (e.g., steel, concrete, or other materials), are left unspecified. 

%% 
From our interviews with the architectural design team at Obayashi Corporation, we have identified three main criteria for renders: 

\begin{itemize}
    \item \textbf{Overall Intent}: Generated renders should contain the same overall structure as communicated by the sketch.
    \item \textbf{Realism}: Renders should be appear to be realistic buildings, with no visual artifacts.
    \item \textbf{Structural Variety}: Renders should contain a variety of structural and material elements while still respecting intended structure in the sketch.
\end{itemize}

For the rest of this work, we first describe our automated proxies for these criteria.  We then detail the image-to-image generative methods used to generate candidate designs from a corpus of conceptual sketches, with training details where relevant.  An evaluation against the automated measures is performed.  We follow with a discussion of these results and highlight issues with using conceptual sketches as inputs and describe future work to address these issues.

%% 

