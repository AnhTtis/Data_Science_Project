\section{General theorem}

\james{I'm sure notation below could be made more elegant}

Let $\{\cX_\secp\}_{\secp \in \bbN}, \{\cY_\secp\}_{\secp \in \bbN}$ be families of finite sets, $\{H_\secp = \{h : \cX_\secp \to \cY_\secp\}\}_{\secp \in \bbN}$ be a family of hash functions, $\{f_{\secp,0},f_{\secp,1} : \cX_\secp \to \{e^{i \theta} : \theta \in [0,2\pi]\}\}_{\secp \in \bbN}$ be two families of phase functions, $\{D_\secp: \cX_\secp \to [0,1]\}_{\secp \in \bbN}$ be a family of distributions, and $\{M_{\secp,h}\}_{\secp \in \bbN, h \in H_\secp}$ be a family of projective measurements in the $\{\cX_\secp\}_{\secp \in \bbN}$-basis. We will now drop the indexing by $\secp$ for convenience. 

For $h \in H$, define \[\ket{\psi_{h,0}}_{X,Y} \coloneqq \sum_{x \in \cX} f_0(x)\sqrt{D(x)}\ket{x}_X\ket{h(x)}_Y, ~~ \ket{\psi_{h,1}}_{X,Y} \coloneqq \sum_{x \in \cX} f_1(x)\sqrt{D(x)}\ket{x}_X\ket{h(x)}_Y,\] and for $h \in H, y \in \cY$, define the following normalized states \[\ket{\psi_{h,y,0}}_X \propto \sum_{x \in \cX : h(x)=y} f_0(x)\sqrt{D(x)}\ket{x}_X, ~~ \ket{\psi_{h,y,1}}_X \propto \sum_{x \in \cX: h(x) = y} f_1(x)\sqrt{D(x)}\ket{x}_X.\] Then, for $h \in H, y \in \cY, b \in \{0,1\}$, let \[\left\{\ket{\psi^m_{h,y,b}}_X\right\}_m\] be the set of possible states that result from applying measurement $M_h$ to $\ket{\psi_{h,y,b}}$, indexed by measurement outcomes $m$.

Now, we require three properties from the hash function $H$.

\begin{itemize}
    \item \textbf{$(f_0,f_1,M)$-phase hiding.} For all $h,y,b$, and $m$, $f_b(\cdot)$ is constant on $x \in \mathsf{Sup}(\ket{\psi_{h,y,b}^m})$.
    \item \textbf{$(D,M)$-targeted collapsing.} For $b \in \{0,1\}$, let \[\sigma_b \coloneqq \E_{h,y}\left[(h,y,\ket{\psi_{h,y,b}})\right], ~~ \widehat{\sigma}_b \coloneqq \E_{h,y,m}\left[(h,y,\ket{\psi_{h,y,b}^m})\right],\] where the first expectation is over the sampling of $h \gets H$ and measuring register $Y$ in the $\cY$-basis to obtain $y$, and the second expectation is additionally over measuring register $X$ with $M_h$. Then for any QPT adversary $\{\cA_\secp\}_{\secp \in \bbN}$ and any $b \in \{0,1\}$, \[\left|\Pr\left[\cA_\secp(\sigma_b) \to 1\right] - \Pr\left[\cA_\secp(\widehat{\sigma}_b) \to 1\right]\right| = \negl(\secp).\]
    \item \textbf{$(D,M)$-second preimage resistance.} For any QPT adversary $\{\cA_\secp\}_{\secp \in \bbN}$ and any $b \in \{0,1\}$,
    \[\Pr_{\cA_\secp\left(h,y,\ket{\psi^m_{h,y,b}}\right) \to x'}\left[x' \notin \mathsf{Sup}\left(\ket{\psi_{h,y,b}^m}\right) \wedge h(x') = y \right] = \negl(\secp),\] where the probability is additionally over the sampling of $h \gets H$, measuring register $Y$ in the $\cY$-basis to obtain $y$, and measuring register $X$ with $M_h$ to obtain left-over state $\ket{\psi_{h,y,b}^m}$.
\end{itemize}

Finally, consider the following experiment $\Exp_{H,f_0,f_1,D,M,\cA}(b)$.

\begin{enumerate}
    \item The challenger samples $h \gets H$, prepares $\ket{\psi_{h,b}}_{X,Y}$, and measures $Y$ in the $\cY$-basis to obtain $y$ and left-over state $\ket{\psi_{h,y,b}}_X$.
    \item The challenger sends $h,y,\ket{\psi_{h,y,b}}$ to $\cA$, who outputs a classical deletion certificate $\pi$ and a left-over quantum state $\rho$.
    \item If $h(\pi) = y$, output $\rho$, and otherwise output $\bot$.
\end{enumerate}

\begin{theorem}
If $H$ satisfies the above three properties with respect to $(f_0,f_1,D,M)$, then for any QPT $\{\cA_\secp\}_{\secp \in \bbN}$, \[\TD\left(\Exp_{H,f_0,f_1,D,M,\cA}(0),\Exp_{H,f_0,f_1,D,M,\cA}(1)\right) = \negl(\secp).\]
\end{theorem}

