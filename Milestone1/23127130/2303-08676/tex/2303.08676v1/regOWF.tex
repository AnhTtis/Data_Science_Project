\section{Publicly-Verifiable Deletion from Balanced Binary-Measurement TCR}\label{sec:regularOWF}

In this section, we show how to build a variety of cryptographic primitives with $\PVD$ from a specific type of hash function that we call \emph{balanced binary-measurement target-collision-resistant}.

\begin{definition}[Balanced Binary-Measurement TCR Hash]\label{def:BBMhash}
A hash function family $\cH = \{H_\secp : \{0,1\}^{m(\secp)} \to \{0,1\}^{n(\secp)}\}_{\secp \in \bbN}$ is \emph{balanced binary-measurement target-collision-resistant} if:
\begin{enumerate}
    \item There exists a family of efficiently computable \emph{single-output-bit} measurement functions $\cM = \{\{M[h] : \{0,1\}^{m(\secp)} \to \{0,1\}\}_{h \in H_\secp}\}_{\secp \in \bbN}$ such that $\cH$ is $\cM$-target-collision-resistant (\cref{def:target-CR}).
    \item There exists a constant $\delta > 0$ such that\footnote{It is also straightforward to generalize our results to any $\delta(\secp) = 1/\poly(\secp)$.} \[\Pr_{h \gets H_\secp, x \gets \{0,1\}^{m(\secp)}}\left[\bigg|\frac{A_{h,x,0} - A_{h,x,1}}{A_{h,x,0} + A_{h,x,1}}\bigg| \leq 1-\delta\right] = 1-\negl(\secp),\]
    
    where $A_{h,x,b} \coloneqq |\{x' \in h^{-1}(h(x)) : M[h](x') = b\}|.$
\end{enumerate}
\end{definition}

\begin{remark}
By \cref{thm:TC-from-TCR} and \cref{thm:CETC-generalization}, any balanced binary-measurement TCR $\cH$ with associated measurement function $\cM$ is also $\cM$-target-collapsing and certified everlasting $\cM$-target-collapsing.
\end{remark}


\subsection{Commitments}
\label{sec:com}

A \emph{canonical quantum bit commitment} \cite{Yan} consists of a family of pairs of unitaries $\{(Q_{\secp,0},Q_{\secp,1})\}_{\secp \in \bbN}$. To commit to a bit $b$, the committer applies $Q_{\secp,b}$ to the all-zeros state $\ket{0}$ to obtain a state on registers $C$ and $R$, and sends register $C$ to the receiver. To open, the committer sends the bit $b$ and the remaining state on register $R$. The receiver applies $Q_{\secp,b}^\dagger$ to registers $(C,R)$, measures the result in the standard basis, and accepts if all zeros are observed.

\begin{definition}[Computational Hiding]
A canonical quantum bit commitment $\{(Q_{\secp,0},Q_{\secp,1})\}_{\secp \in \bbN}$ satisfies \emph{computational hiding} if for any QPT adversary $\{\cA_\secp\}_{\secp \in \bbN}$,

\[\left|\Pr\left[\cA_\secp(\Tr_R\left(Q_{\secp,0}\ket{0}\right)) = 1\right] - \Pr\left[\cA_\secp(\Tr_R\left(Q_{\secp,1}\ket{0}\right)) = 1\right] \right| = \negl(\secp).\]
\end{definition}

\begin{definition}[Honest Binding]
A canonical quantum bit commitment $\{(Q_{\secp,0},Q_{\secp,1})\}_{\secp \in \bbN}$ satisfies \emph{honest binding} if for any auxiliary family of states $\{\ket{\psi_\secp}\}_{\secp \in \bbN}$ on register $Z$ and any family of physically realizable unitaries $\{U_\secp\}_{\secp \in \bbN}$ on registers $R,Z$, 

\[\left\| \left(Q_{\secp,1}\dyad{0}{0}Q_{\secp,1}^\dagger\right)U\left(Q_{\secp,0}\ket{0}\ket{\psi}\right)\right\| = \negl(\secp).\]
\end{definition}

\begin{definition}[Publicly-Verifiable Deletion]
A canonical quantum bit commitment $\{(Q_{\secp,0},Q_{\secp,1})\}_{\secp \in \bbN}$ has \emph{publicly-verifiable deletion} if there exists a measurement $\{V_{\secp}\}_{\secp \in \bbN}$ on register $R$, a measurement $\{D_{\secp}\}_{\secp \in \bbN}$ on register $C$, and a classical predicate $\Ver(\cdot,\cdot) \to \{\top,\bot\}$ that satisfy the following properties.

\begin{itemize}
    \item \textbf{Correctness of deletion.} For any $b \in \{0,1\}$, it holds that
    \[\Pr\left[\Ver(\vk,\pi) = \top : (\vk,\pi) \gets (V_\secp \otimes D_\secp)Q_{\secp,b}\ket{0}\right] = 1-\negl(\secp).\]
    \item \textbf{Certified everlasting hiding.} For any QPT adversary $\cA = \{\cA_\secp\}_{\secp \in \bbN}$, it holds that 
    \[\TD\left(\mathsf{EvExp}_{\cA}(\secp,0), \mathsf{EvExp}_{\cA}(\secp,1)\right) = \negl(\secp),\] where $\mathsf{EvExp}_{\cA}(\secp,b)$ is the following experiment.
    \begin{itemize}
        \item Prepare $Q_{\secp,b}\ket{0}$, measure register $R$ with $V_\secp$ to obtain $\vk$, and send $(\vk,C)$ to $\cA_\secp$.
        \item Parse $\cA_{\secp}$'s output as a deletion certificate $\pi$ and a left-over state $\rho$. If $\Ver(\vk,\pi) = \bot$, output $\bot$, and otherwise output $\rho$.
    \end{itemize}
\end{itemize}
\end{definition}

%\james{Define zero-knowledge proof with certified everlasting zero-knowledge}

\paragraph{Construction.} We construct a quantum canonical bit commitment with $\PVD$ as follows. Let $\cH = \{H_\secp : \{0,1\}^{m(\secp)} \to \{0,1\}^{n(\secp)}\}_{\secp \in \bbN}$ be a balanced binary-measurement TCR hash with associated measurement function $\cM = \{\{M[h]\}_{h \in H_\secp}\}_{\secp \in \bbN}$, and let $m = m(\secp)$, $n = n(\secp)$. For any $h \in H_\secp,y \in \{0,1\}^n$, and $b \in \{0,1\}$, we will define the state 

\[\ket{\psi_{h,y,b}} \coloneqq \frac{1}{\sqrt{|h^{-1}(y)|}}\sum_{x : h(x)= y}(-1)^{M[h](x)}\ket{x}.\]

\begin{itemize}
    \item Consider the following procedure $S_{\secp,b}$. Sample $h \gets H_\secp$ and for $i \in [\secp]$, prepare the state \[\frac{1}{\sqrt{2^m}}\sum_{x \in \{0,1\}^m}(-1)^{b \cdot M[h](x)}\ket{x}\ket{h(x)},\] and measure the second register to obtain $y_i$ and left-over state $\ket{\psi_{h,y_i,b}}$. Then, output \[(h,y_1,\dots,y_\secp),
    \bigotimes_{i \in [\secp]}\ket{\psi_{h,y_i,b}}.\]
    
    
    Now, $Q_{\secp,b}$ will be the purification of $S_{\secp,b}$, where the output register is $C$ and the auxiliary register is $R$. That is, $Q_{\secp, b}$ prepares the state
    \[\frac{1}{\sqrt{|H_\secp|2^{\secp m}}}\sum_{h,x_1,\dots,x_\secp}(-1)^{b \cdot \bigoplus_{i \in [\secp]}M[h](x_i)}\ket{h,h(x_1),\dots,h(x_\secp)}_R\ket{h,h(x_1),\dots,h(x_\secp),x_1,\dots,x_\secp}_C.\]
    \item $V_\secp$ measures register $R$ in the standard basis to obtain $\vk = (h,y_1,\dots,y_\secp)$. $D_\secp$ measures register $C$ in the standard basis to obtain $(h,y_1,\dots,y_\secp,x_1,\dots,x_\secp)$, and outputs $\pi = (x_1,\dots,x_\secp)$.
    \item $\Ver((h,y_1,\dots,y_\secp),(x_1,\dots,x_\secp))$ outputs $\top$ iff $h(x_i) = y_i$ for all $i \in [\secp]$.
\end{itemize}

\begin{theorem}\label{thm:commitment}
The above construction satisfies computational hiding, honest binding, and publicly-verifiable deletion. Thus, assuming the existence of a balanced binary-measurement TCR hash, there exists a quantum canonical bit commitment with $\PVD$.
\end{theorem}

\begin{proof}
%First, we note that computational hiding and certified everlasting hiding follow from \cref{corollary:target-collapsing} and a straightforward hybrid argument, so it suffices to show honest binding.\james{TODO: make the hybrid argument explicit}

First we argue computational hiding. On a commitment to $b$, the receiver sees the mixed state 

\[\E_{h,y_1,\dots,y_\secp}\left[\bigotimes_{i \in [\secp]}\ket{\psi_{h,y_i,b}}\right],\] where the expectation is over sampling $h \gets \cH$ and measuring random $y_1,\dots,y_\secp$. Note the following two facts.

\begin{enumerate}
	\item Given any state $\ket{\psi_{h,y,b}}$, let $M[h]\left(\ket{\psi_{h,y,b}}\right)$ be the mixed state that results from measuring the bit $M[h](\cdot)$ on $\ket{\psi_{h,y,b}}$. By the $\cM$-target-collapsing of $\cH$, we have that for any $b \in \{0,1\}$, \[\E_{h,y}\left[\ket{\psi_{h,y,b}}\right] \approx_c \E_{h,y}\left[M[h]\left(\ket{\psi_{h,y,b}}\right)\right],\] where $\approx_c$ denotes computational indistinguishability. The case of $b=0$ follows directly by definition of $\cM$-target-collapsing and the case of $b=1$ follows because a reduction can efficently map $\ket{\psi_{h,y,0}}$ to $\ket{\psi_{h,y,1}}$ using the fact that $M[h]$ is efficiently computable.
	\item For any $h,y$, $M[h]\left(\ket{\psi_{h,y,0}}\right)$ and $M[h]\left(\ket{\psi_{h,y,1}}\right)$ are equivalent states, which follows by definition.
\end{enumerate}

Thus, we can run the following hybrid argument.

\begin{itemize}
	\item $\Hyb_0$: The receiver is given a commitment to 0.
	\item $\Hyb_1\dots\Hyb_\secp$: In $\Hyb_i$, we switch $\ket{\psi_{h,y,0}}$ to $M[h]\left(\ket{\psi_{h,y,0}}\right)$. This is computationally indistinguishable from $\Hyb_{i-1}$ by the first fact above.
	\item $\Hyb_{\secp+1}$: Switch $M[h]\left(\ket{\psi_{h,y_i,0}}\right)$ to $M[h]\left(\ket{\psi_{h,y_i,1}}\right)$ for all $i \in [\secp]$. This is perfectly indistinguishable from $\Hyb_\secp$ by the second fact above.
	\item $\Hyb_{\secp+2}\dots\Hyb_{2\secp + 1}$: In $\Hyb_{i + \secp + 1}$, we switch $M[h]\left(\ket{\psi_{h,y_i,1}}\right)$ to $\ket{\psi_{h,y_i,1}}$. This is computationally indistinguishable from $\Hyb_{i+\secp}$ by the first fact above.
\end{itemize}

This completes the proof of computational hiding. Next, since $\cH$ satisfies \emph{certified everlasting} $\cM$-target-collapsing, we see that each hybrid is \emph{statistically close} when the receiver outputs a valid deletion certificate. Thus, the same proof establishes publicly-verifiable deletion.


%For this, it suffices to show (by Uhlmann's theorem) that the trace distance between the outputs of $S_{\secp,0}$ and $S_{\secp,1}$ is $1-\negl(\secp)$. We lower bound the trace distance by demonstrating a measurement that distinguishes with probability $1-\negl(\secp)$. 

%\james{TODO: Instead of citing Uhlmann's theorem, we can say that it suffices to demonstrate a measurement on just the $R$ register that accepts $Q_{\secp,0}\ket{0}$ with probability 1 and rejects $Q_{\secp,0}\ket{0}$ with probability $1-\negl(\secp)$, and note that any $U$ in the definition of honest-binding cannot change the success probability of this measurement}

Finally, we show honest binding. For this, it suffices to demonstrate a measurement on register $C$ that accepts with probability 1 on the output of $Q_{\secp,0}$ and with probability $\negl(\secp)$ on the output of $Q_{\secp,1}$. This suffices because any $U$ that breaks honest binding must then necessarily affect the result of this measurement by a $\nonnegl(\secp)$ amount, which is impossible since $U$ does not operate on $C$.

The measurement takes the classical part of the output $(h,y_1,\dots,y_\secp)$ and attempts to project the quantum part onto \[\dyad{\psi_{h,y_1,0}}{\psi_{h,y_1,0}} \otimes \dots \otimes \dyad{\psi_{h,y_\secp,0}}{\psi_{h,y_\secp,0}}.\] Clearly this accepts the output of $S_{\secp,0}$ with probability 1, so it suffices to show that the output of $S_{\secp,1}$ is accepted with probability $\negl(\secp)$. To see this, we bound

\begin{align*}
    &\E_{h,y_1,\dots,y_\secp}\left[\prod_{i \in [\secp]}|\braket{\psi_{h,y_i,1}|\psi_{h,y_i,0}}|^2\right] \\
    &\E_{h,y_1,\dots,y_\secp}\left[\prod_{i \in [\secp]}\left(\frac{1}{|h^{-1}(y_i)|}\left(\sum_{x:h(x)=y_i}(-1)^{M[h]}\bra{x}\right)\left(\sum_{x:h(x)=y_i}\ket{x}\right)\right)^2\right] \\
    &= \E_{h,y_1,\dots,y_\secp}\left[\prod_{i \in [\secp]}\left(\frac{1}{|h^{-1}(y_i)|}\left(|\{x : h(x)=y_i,M[h](x)=0\}|-|\{x : h(x)=y_i,M[h](x)=1\}|\right)\right)^2\right] \\
    %&=\E_{h,y_1,\dots,y_\secp}\left[\prod_{i \in [\secp]}\left(\frac{p_{y_i}-p_{y_i \oplus \Delta}}{p_{y_i}+p_{y_i \oplus \Delta}}\right)^2\right]\\
    &\leq \left(1-\delta\right)^{2\secp} + \negl(\secp) \\
    &=\negl(\secp),
\end{align*}

where the inequality follows from property (2) of \cref{def:BBMhash}.
  
%\begin{itemize}
    %\item \james{computational hiding is straightforward from target-collapsing}
    %\item \james{to show honest binding, we can establish that the fidelity between the states on register $C$ when $b = 0$ vs $b=1$ is negligible, which follows from $\delta$-balanced and $\secp$ repetition}
    %\item \james{certified everlasting hiding is straightforward from certified everlasting target-collapsing}
%\end{itemize}
\end{proof}



\subsection{Public-Key Encryption}
\label{sec:pke}

\begin{definition}[Trapdoor Phase-Recoverability]
We say that a balanced binary-measurement TCR hash has \emph{trapdoor phase-recoverability} if there exist algorithms $\Samp,\Recover$ with the following properties.
\begin{itemize}
    \item $\Samp(1^\secp)$: The sampling algorithm samples a uniformly random function $h \in H_\secp$ along with a trapdoor $\td$.
    \item $\Recover(\td,y,X)$: There exist constants $c,\epsilon$ such that with probability $1-\negl(\secp)$ over $(h,\td) \gets \Samp(1^\secp)$, 
    
    \begin{align*}
        &\Pr_{x \gets \{0,1\}^m}\left[\Recover(\td,h(x),\ket{\psi_{h,h(x),0}}) \to 0\right] \geq c + \epsilon,\\
        &\Pr_{x \gets \{0,1\}^m}\left[\Recover(\td,h(x),\ket{\psi_{h,h(x),1}}) \to 0\right] \leq c - \epsilon,
    \end{align*}
    
    %\begin{align*}
    %\Pr_{x \gets \{0,1\}^m}\left[\Recover(\td,h(x),\ket{\psi_{h,h(x),0}}) \to 0\right] \geq c+\epsilon
    
    %&\Pr_{x \gets \{0,1\}^m}\left[\Recover(\td,h(x),\ket{\psi_{h,h(x),1}}) \to 0\right] \leq c-\epsilon,
    %\end{align*}
    
    where \[\ket{\psi_{h,y,b}} \coloneqq \frac{1}{\sqrt{|h^{-1}(y)|}}\sum_{x:h(x) = y}(-1)^{M[h](x)}\ket{x}.\]
\end{itemize}
\end{definition}



\begin{theorem}\label{thm:PKE}
Assuming the existence of a binary-measurement TCR hash $\cH$ with trapdoor phase-recoverability, there exists public-key encryption with $\PVD$.
\end{theorem}

\begin{proof}
This follows from essentially the same construction as commitments. Let $\cM$ be the measurement function associated with $\cH$ and let $(\Samp,\Invert)$ be the associated trapdoor algorithms. Then, the PKE with $\PVD$ is defined as follows.
\begin{itemize}
    \item $\Gen(1^\secp)$: Sample $(h,\td) \gets \Samp(1^\secp)$ and set $\pk \coloneqq h, \sk \coloneqq \td$.
    \item $\Enc(\pk,b)$: For $i \in [\secp]$, prepare the state \[\frac{1}{\sqrt{2^m}}\sum_{x \in \{0,1\}^m}(-1)^{b \cdot M[h](x)}\ket{x}\ket{h(x)},\] and measure the second register to obtain $y_i$ and left-over state $\ket{\psi_{h,y_i,b}}$. Then, set  \[\ket{\ct} \coloneqq \left(y_1,\dots,y_\secp,
    \bigotimes_{i \in [\secp]}\ket{\psi_{h,y_i,b}}\right), ~~ \vk \coloneqq (h,y_1,\dots,y_\secp).\]
    \item $\Dec(\sk,\ket{\ct})$: Parse $\ket{\ct}$ as $(y_1,\dots,y_\secp,X_1,\dots,X_\secp)$, for $i \in [\secp]$ run \[b_i \gets \Recover(\td,y_i,X_i),\]
    and output 0 if $|\{i : b_i = 0\}|/\secp > c$, and output 1 otherwise.
    
    
    %perform the measurement \[\left\{\dyad{\psi_{h,y_1,0}}{\psi_{h,y_1,0}} \otimes \dots \otimes \dyad{\psi_{h,y_\secp,0}}{\psi_{h,y_\secp,0}}, \bbI - \dyad{\psi_{h,y_1,0}}{\psi_{h,y_1,0}} \otimes \dots \otimes \dyad{\psi_{h,y_\secp,0}}{\psi_{h,y_\secp,0}}\right\}\] on the second part. Output 0 if the first outcome is observed and 1 otherwise. Note that since $\ket{\psi_{h,y_i,0}}$ can be efficiently prepared (to within $\negl(\secp)$ trace distance) given $\td$ and $y_i$, this measurement can be efficiently implemented.
    
    \item $\Del(\ket{\ct})$: Parse $\ket{\ct}$ as $(y_1,\dots,y_\secp,X_1,\dots,X_\secp)$ and measure $X_i$ in the standard basis to obtain $\pi \coloneqq (x_1,\dots,x_\secp)$.
    \item $\Vrfy(\vk,\pi)$: Output $\top$ iff $h(x_i) = y_i$ for all $i \in [\secp]$.
\end{itemize}

Correctness follows from a standard Hoeffding inequality and correctness of deletion (\cref{def:correctness-deletion}) is immediate. Certified deletion security (\cref{def:security-deletion}) follows from the $\cM$-target-collapsing and certified everlasting $\cM$-target-collapsing of $\cH$, using the same hybrid argument as in the proof of \cref{thm:commitment}.
\end{proof}

\subsection{A Generic Compiler}
\label{sec:generic}
Let $(\Gen,\Enc,\Dec,\Del,\Vrfy)$ be the encryption scheme defined last section, let $\cA = \{\cA_\secp\}_{\secp \in \bbN}$ be an adversary, let $p(\secp)$ be a polynomial, and let $\cZ = \{Z_\secp(\aux)\}_{\aux \in \{0,1\}^{p(\secp)}, \secp \in \bbN}$ be a (static or interactive) family of distributions that is semantically-secure against $\cA$ with respect to $\aux$. That is, in the static case, it holds that for any $\aux \in \{0,1\}^{p(\secp)}$,

\[\bigg| \Pr\left[\cA_\secp(Z_\secp(\aux)) = 1\right] - \Pr\left[\cA_\secp(Z_\secp(0^{p(\secp)})) = 1\right]\bigg| \leq \negl(\secp),\] and in the interactive case,

\[\bigg| \Pr\left[\cA_\secp^{Z_\secp(\aux)} = 1\right] - \Pr\left[\cA_\secp^{Z_\secp(0^{p(\secp)})} = 1\right]\bigg| \leq \negl(\secp),\] where $\cA_\secp^{Z_\secp(\aux)}$ indicates that $\cA_\secp$ can interact with $Z_\secp(\aux)$, which is the description of an interactive machine initialized with $\aux$.





%$\cZ = \{\cZ_\secp(\cdot,\cdot)\}_{\secp \in \bbN}$ be any (potentially quantum) operation such that for any distribution $\cD = \{\cD_\secp\}_{\secp \in \bbN}$ over pairs $(x,\ket{\psi})$, where $x$ is a classical string and $\ket{\psi}$ is a quantum state, it holds that
%\[\bigg|\Pr_{(x,\ket{\psi}) \gets \cD_\secp}\left[\cA_\secp(\cZ_\secp(x,\ket{\psi})) = 1\right] -  \Pr_{(x,\ket{\psi})\gets \cD_\secp}\left[\cA_\secp(\cZ_\secp(0,\ket{\psi})) = 1\right] \bigg| = \negl(\secp).\] That is, $\cZ$ is semantically-secure in its first input (with respect to $\cA$). 



\begin{lemma}\label{lemma:compiler}
Given any $\cA,\cZ$ as described above, define the experiment $\mathsf{EvEnc}_{\cA,\cZ,\secp}(b)$ as follows.

\begin{itemize}
    \item Sample $(h,\td) \gets \Gen(1^\secp)$ and $(\ket{\ct},\vk) \gets \Enc(h,b)$.
    \item Run $\cA_\secp(h,\vk,\ket{\ct},Z_\secp(\td))$, and parse their output as a deletion certificate $\pi$ and a left-over quantum state $\rho$.
    \item If $\Vrfy(\vk,\pi) = \top$, output $\rho$, and otherwise output $\bot$.
\end{itemize}
Then it holds that
\[\TD\left(\mathsf{EvEnc}_{\cA,\cZ,\secp}(0),\mathsf{EvEnc}_{\cA,\cZ,\secp}(1)\right) = \negl(\secp).\]
\end{lemma}

\begin{proof}

First, we confirm that $\cH$ is $(\cM,\cZ)$-target-collision-resistant.  To see this, we first use the semantic security of $\cZ$ to switch to a hybrid where $\cA_\secp$ receives $Z_\secp(0)$ rather than $Z_\secp(\td)$, and then appeal directly to the fact that $\cH$ is $\cM$-target-collision-resistant (\cref{thm:target-CR}). Then by \cref{thm:TC-from-TCR} and \cref{thm:CETC-generalization}, we have that $\cH$ is certified everlasting $(\cM,\cZ)$-target-collapsing. Using the same hybrid argument as in the proof of \cref{thm:commitment} then completes the proof.



%Given $\widetilde{\cF}$, define $\widetilde{\cF}'$ to be the same family except that $\cZ_\secp(\td)$ is included in the description of the hash function. Note that if $\widetilde{\cF}$ is $(\cU,\cP)$-target-collision-resistant, then so is  $\widetilde{\cF}'$. This follows because we can first use the semantic security of $\cZ$ to switch $\cZ_\secp(\td)$ to $\cZ_\secp(0)$, and then appeal to $(\cU,\cP)$-target-collision-resistance of $\widetilde{\cF}$. Then, by \cref{thm:TC-from-TCR} and \cref{thm:CETC-generalization}, we have that $\widetilde{\cF}'$ is certified everlasting $(\cU,\cP)$-target-collapsing, which immediately implies the theorem.
\end{proof}

By instantiating $\cZ$ with various crytographic primitives, we immediately gives the following applications. We do not write formal definitions of each of these primitives, and instead refer the reader to \cite{cryptoeprint:2022/1178} for these.


\begin{corollary}\label{cor:compiler}
Assuming the existence of a balanced binary-measurement TCR hash with trapdoor phase-recoverability, and post-quantum \[X \in \left\{\begin{array}{r}\text{quantum fully-homormophic encryption, attribute-based encryption}, \\ \text{witness encryption, timed-release encryption}\end{array}\right\},\] there exists $X$ with $\PVD$.
\end{corollary}

The implications to witness encryption and timed-release encryption follow immediately by encrypting $\td$ with the appropriate encryption scheme (and in the case of timed-release encryption, considering the class of parallel-time-bounded adversaries). We briefly remark on the other two implications.

\begin{itemize}
    \item \textbf{Fully-homomorphic encryption.} If we encrypt $\td$ using a \emph{quantum} fully-homomorphic encryption (QFHE) scheme, then we obtain (Q)FHE with publicly-verifiable deletion. The reason we need QFHE for the compiler is for evaluation correctness: we need to decrypt $\ket{\ct}$ homomorphically under the QFHE (using $\td$) in order to obtain a (Q)FHE encryption of the plaintext, which can then be operated on.
    \item \textbf{Attribute-based encryption.} If we encrypt $\td$ using an attribute-based encryption (ABE) scheme, we immediately obtain a correct ABE scheme with certified deletion. In order to argue that this scheme has certified deletion security, we appeal to \cref{lemma:compiler} with an interactive $Z_\secp$ that runs the ABE security game, encrypting its input $\td$ into the challenge ciphertext.
\end{itemize}



\subsection{Balanced Binary-Measurement TCR from Almost-Regular OWFs}
\label{sec:almostreg}



\begin{definition}[Almost-Regular Function]\label{def:almost-regular}
A function $\cF = \{f_\secp : \{0,1\}^{m(\secp)} \to \{0,1\}^{n(\secp)}\}$ is almost-regular if there exists efficiently computable polynomials $r(\secp)$ and $p(\secp)$ such that for all $\secp \in \bbN$ and $x \in \{0,1\}^{m(\secp)}$, \[\frac{1}{p(\secp)} \cdot 2^{r(\secp)} \leq \big| \{x' \in \{0,1\}^{n(\secp)} : f_\secp(x') = f_\secp(x)\}\big| \leq p(\secp) \cdot 2^{r(\secp)}.\]
\end{definition}

Note that we assume $r(\secp)$ is efficiently computable, which means that the regularity of $\cF$ is \emph{known}. This is often contrasted with the more general class of functions that are \emph{unknown} regular. Throughout this work, we always means \emph{known} regular.

\begin{definition}[Balanced Function]\label{def:balanced}
A function $\cF = \{f_\secp : \{0,1\}^{m(\secp)} \to \{0,1\}^{n(\secp)}\}_{\secp \in \bbN}$ is $\delta$-\emph{balanced} for some constant $\delta \in [0,1)$ if there exists a family of sets $\{\mathsf{BAD}_\secp \subset \{0,1\}^{n(\secp)}\}_{\secp \in \bbN}$ such that
\begin{enumerate}
    \item $|\mathsf{BAD}_\secp|/2^{n(\secp)} = \negl(\secp)$.
    \item $\Pr_{x \gets \{0,1\}^{m(\secp)}}[f_\secp(x) \in \mathsf{BAD}_\secp] = \negl(\secp)$.
    \item For every $z \notin \mathsf{BAD}_\secp$, $\Pr_{x \gets \{0,1\}^{m(\secp)}}[f_\secp(x) = z] \cdot 2^{n(\secp)} \in [1-\delta,1+\delta]$.
\end{enumerate}
\end{definition}

\begin{definition}[One-Way Function]\label{def:OWF}
A function $\cF = \{f_\secp : \{0,1\}^{m(\secp)} \to \{0,1\}^{\ell(\secp)}\}$ is one-way if for any QPT adversary $\cA = \{\cA_\secp\}_{\secp \in \bbN}$,

\[\Pr\left[f_\secp(x')=f(x) : \begin{array}{r}x \gets \{0,1\}^{m(\secp)} \\ x' \gets \cA_\secp(f(x))\end{array}\right] = \negl(\secp).\]

We say that $\cF$ is one-way \emph{over its range} if for any QPT adversary $\cA = \{\cA_\secp\}_{\secp \in \bbN}$,
\[\Pr\left[f_\secp(x) = y : \begin{array}{r}y \gets \{0,1\}^{\ell(\secp)} \\ x \gets \cA_\secp(y)\end{array}\right] = \negl(\secp).\]
\end{definition}

\begin{definition}[Universal Hash]
A hash function family $\cH = \{H_\secp : \{0,1\}^{m(\secp)} \to \{0,1\}^{n(\secp)}\}_{\secp \in \bbN}$ is called $t(\secp)$-universal if for each distinct $x_1,\dots,x_{t(\secp)} \in \{0,1\}^{m(\secp)}$ and $y_1,\dots,y_{t(\secp)} \in \{0,1\}^{n(\secp)}$, it holds that 
\[\Pr_{h \gets H_\secp}\left[h(x_1) = y_1 \wedge \dots \wedge h(x_{t(\secp)}) = y_{t(\secp)}\right] = 2^{-n(\secp) \cdot t(\secp)}.\]
\end{definition}

\begin{importedtheorem}[\cite{balancedOWF}]\label{impthm:balancedOWF}
Let $\cF = \{f_\secp : \{0,1\}^{m(\secp)} \to \{0,1\}^{\ell(\secp)}\}$ be an almost-regular one-way function. Then there exists $n(\secp) < m(\secp)$ and $\delta \in [0,1)$ such that for any $3\secp$-universal hash family $\cH = \{H_\secp : \{0,1\}^{\ell(\secp)} \to \{0,1\}^{n(\secp)}\}_{\secp \in \bbN}$ where each $h \in H_\secp$ can be described by $s(\secp)$ bits, the function \[\cF' = \left\{f'_\secp : \{0,1\}^{s(\secp) + m(\secp)} \to \{0,1\}^{s(\secp) + n(\secp)}\right\}_{\secp \in \bbN}, ~~ \text{where} ~~ f'_\secp(h,x) \coloneqq (h,h(f_\secp(x))),\] is $\delta$-balanced and one-way over its range.
\end{importedtheorem}

Now consider any balanced function $\cF = \{f_\secp : \{0,1\}^{m(\secp)} \to \{0,1\}^{n(\secp)}\}_{\secp \in \bbN}$ that is one-way over its range, and define the family of hash functions \[\cH^\cF = \left\{H_\secp : \{0,1\}^{m(\secp)} \to \{0,1\}^{n(\secp)}\right\}_{\secp \in \bbN}\] as follows. For each $\Delta \in \{0,1\}^{n(\secp)}$, define $f_\Delta : \{0,1\}^{n(\secp)} \to \{0,1\}^{n(\secp)}$ to, on input $z$, output the lexicographically first element of $\{z,z \oplus \Delta\}$.\footnote{We don't need to worry about the case when $\Delta = 0^{n(\secp)}$ since we'll be sampling $\Delta$ uniformly, but one could define $f_\Delta$ to be the identity in that case.} Then we define

\[H_\secp \coloneqq \left\{h_{\secp,\Delta} \coloneqq f_\Delta \circ f_\secp\right\}_{\Delta \in \{0,1\}^{n(\secp)}}.\]

We will also define the family of measurement functions  \[\cM = \left\{\left\{M[h_{\secp,\Delta}]\right\}_{h_{\secp,\Delta} \in H_\secp}\right\}_{\secp \in \bbN}\] as follows. The predicate $M[h_{\secp,\Delta}] : \{0,1\}^m \to \{0,1\}$ takes $x$ as input, computes $z \coloneqq f_\secp(x)$, and outputs $0$ if $z < z \oplus \Delta$ and $1$ if $z > z \oplus \Delta$ (where ordering is lexicographical).

\begin{theorem}\label{thm:target-CR}
Let $\delta \in [0,1)$ be a constant and $\cF = \{f_\secp : \{0,1\}^{m(\secp)} \to \{0,1\}^{n(\secp)}\}_{\secp \in \bbN}$ be a $\delta$-balanced function that is one-way over its range. Let $\cH^\cF$ and $\cM$ be as defined above. Then, $\cH^\cF$ is a balanced binary-measurement TCR hash with associated measurement function $\cM$.
\end{theorem}

\cref{impthm:balancedOWF} and \cref{thm:commitment} immediately give the following corollary.

\begin{corollary}
Assuming almost-regular one-way functions, there exists a quantum canonical bit commitment with $\PVD$.
\end{corollary}

%\cref{thm:TC-from-TCR} and \cref{thm:CETC-generalization} immediately give the following corollary.

%\begin{corollary}\label{corollary:target-collapsing}
%$\cH^\cF$ is $(\cU,\cM)$-target-collapsing and certified everlasting $(\cU,\cM)$-target-collapsing.
%\end{corollary}



\begin{proof}(Of \cref{thm:target-CR})
First, we check property (2) of \cref{def:BBMhash}. By properties (2) and (3) of \cref{def:balanced}, it holds that with $1-\negl(\secp)$ probability over the sampling of $h \gets H_\secp$ and $x \gets \{0,1\}^m$, \[\bigg|\frac{|\{x' \in h^{-1}(h(x))\} : M[h] = 0| - |\{x' \in h^{-1}(h(x))\} : M[h] = 1|}{|\{x' \in h^{-1}(h(x))\} : M[h] = 0| + |\{x' \in h^{-1}(h(x))\} : M[h] = 1|}\bigg| \leq \delta.\]


Next, we check property (1). Throughout this proof, we will drop indexing by $\secp$ for convenience. Suppose there exists a QPT adversary $\cA$ that breaks the $\cM$-target-collision-resistance of $\cH$. That is, the following experiment outputs 1 with $\nonnegl(\secp)$ probability.\\

\noindent\underline{$\Exp_{\mathsf{TCR}}$}
\begin{itemize}
    \item The challenger samples $\Delta \gets \{0,1\}^n$ and prepares the state $1/\sqrt{2^m}\sum_{x \in \{0,1\}^m}\ket{x}$ on register $X$. It applies $h_\Delta$ on $X$ to a fresh register $Y$ and measures $y \in \{0,1\}^n$, and then measures $P[h_\Delta]$ on $X$ to obtain a bit $b$ and left-over state on register $X$. The challenger sends $(\Delta,y,b)$ and register $X$ to $\cA$.
    \item $\cA$ outputs a string $x' \in \{0,1\}^n$.
    \item Output 1 if $h_\Delta(x') = y$ and $M[h_\Delta](x') = 1-b$.
\end{itemize}



We now define an adversary $\cA'$ that breaks the one-wayness of $\cF$ over its range.\\

\noindent\underline{$\Exp_{\mathsf{OW}}$}

\begin{itemize}
    \item The challenger samples $z \gets \{0,1\}^n$ and sends $z$ to $\cA'$.
    \item $\cA'$ prepares the state $1/\sqrt{2^m}\sum_{x \in \{0,1\}^m}\ket{x}$ on register $X$, applies $f$ on $X$ to a fresh register $Z$, and measures $z' \in \{0,1\}^n$. If $z' = z$, then measure register $X$ to obtain $x'$, and return $x'$. Otherwise, set $\Delta \coloneqq z \oplus z'$, set $b = 0$ if $z' < z$ and $b = 1$ otherwise, and set $y = f_\Delta(z)$. Then, initialize $\cA$ with $(\Delta,y,b)$ and register $X$. Run $\cA$ and forward its output $x'$ to the challenger.
    \item Output 1 if $f(x') = z$.
\end{itemize}

It suffices to show that $\cA$'s input comes from the same distribution over $(X,\Delta,y,b)$ in both experiments. To see this, we describe an alternative but identical way to sample $(X,\Delta,y,b)$ in the experiment $\Exp_{\mathsf{TCR}}$. Recalling that $h_\Delta = f_\Delta \circ f$, the challenger could (1) apply $f$ on $X$ to a fresh register $Z$, (2) sample $\Delta \gets \{0,1\}^n$, (3) apply $f_\Delta$ on $Z$ to a fresh register $Y$, and (4) measure $Y$ to obtain $y$ and measure $M[h_\Delta]$ on $X$ to obtain $b$. Note that step (4) is equivalent to instead just measuring the $Z$ register to obtain $z$, defining $b = 0$ if $z < z \oplus \Delta$ and $b=1$ if $z > z \oplus \Delta$, and defining $y = f_\Delta(z)$. Thus, we can imagine first applying $f$ on $X$ to a fresh register $Z$, measuring $Z$ to obtain $z$, sampling $\Delta \gets \{0,1\}^n$, and defining $y = f_\Delta(z)$. Defining $z' = z \oplus \Delta$ and using the fact that $\Delta$ was sampled uniformly at random, we see that this is exactly the same distribution that is sampled in $\Exp_{\mathsf{OW}}$, except that $\cA$ is not initialized if $\Delta = 0^{n(\secp)}$ (in which case $\cA'$ wins the experiment anyway). 

\end{proof}

Now, we generalize the notion of almost-regularity (\cref{def:almost-regular}), balanced (\cref{def:balanced}), and one-wayness (\cref{def:OWF}) to function \emph{families}, where there is a set of of $f \in F_\secp$ associated with each security parameter. All previous definitions generalize to this setting with the requirement that they hold with $1-\negl(\secp)$ probability over $f \gets F_\secp$, and all previous claims follow. We consider families of functions with \emph{trapdoors} that allow us to invert the function and obtain public-key encryption along with other cryptographic primitives.

\begin{definition}[Superposition-invertible trapdoor function]
We say that a function family $\cF = \{F_\secp\}_{\secp \in \bbN}$ is a superposition-invertible trapdoor function if there exist algorithms $\Samp,\Invert$ with the following properties.
\begin{itemize}
    \item $\Samp(1^\secp)$: The sampling algorithm samples a uniformly random function $f \in F_\secp$ along with a trapdoor $\td$.
    \item $\Invert(\td,y)$: Given the trapdoor $\td$ and an image $y$, $\Invert$ outputs a state within negligible trace distance of \[\frac{1}{\sqrt{|f^{-1}(y)|}}\sum_{x:f(x)=y}\ket{x}.\]
\end{itemize}
\end{definition}

\begin{remark}
For the case of injective function families $\cF$, the notion of superposition-invertible trapdoor is equivalent to the standard notion of trapdoor, since there is only one preimage per image.
\end{remark}

\begin{claim}\label{claim:phase-recoverability}
Assuming injective trapdoor one-way functions (or more generally, superposition-invertible trapdoor almost-regular one-way functions), there exists a balanced binary-measurement TCR hash with trapdoor phase-recoverability.
\end{claim}

By \cref{thm:PKE} and \cref{cor:compiler}, we obtain the following corollary.

\begin{corollary}
Assuming the existence of injective trapdoor one-way functions (or more generally, superposition-invertible trapdoor almost-regular one-way functions), there exists PKE with $\PVD$. Additionally assuming post-quantum \[X \in \left\{\begin{array}{r}\text{quantum fully-homormophic encryption, attribute-based encryption}, \\ \text{witness encryption, timed-release encryption}\end{array}\right\},\] there exists $X$ with $\PVD$.
\end{corollary}

\begin{proof}(Of \cref{claim:phase-recoverability}) Given a superposition-invertible almost-regular one-way function, then we know from \cref{impthm:balancedOWF} that we can compose it with a $3\secp$-universal hash function to obtain a $\delta$-balanced function $\cF$ that is one-way over its range, and \cref{thm:target-CR} tells us that we can then obtain a balanced binary-measurement TCR hash $\cH^\cF = \{H_\secp\}_{\secp \in \bbN}$. It remains to check that the resulting hash has trapdoor phase-recoverability.

To see this, we observe that for any polynomials $m(\secp),n(\secp),t(\secp)$, there exists a superposition-invertible $t(\secp)$-universal hash function family $\{U_\secp : \{0,1\}^{m(\secp)} \to \{0,1\}^{n(\secp)}\}_{\secp \in \bbN}$ (without the need for a trapdoor). For example, we can use the Chor-Goldreich construction \cite{Chor-Goldreich}, where each hash in the family is defined by coefficients of a degree-$(t(\secp)-1)$ univariate polynomial over a finite field, and evaluation is polynomial evaluation. To invert, use a root-finding algorithm (e.g. \cite{Cantor1981ANA}) to recover the (at most polynomial) roots, and then arrange these in superposition. Note that for a compressing universal hash from $\{0,1\}^m \to \{0,1\}^n$, one would use a finite field of size at least $2^m$ and define the hash output to consist of (say) the first $n$ bits of the description of the finite field element that results from polynomial evaluation. In this case, the quantum inverter would first prepare a uniform superposition over all of the remaining $m-n$ bits of the field element, and run the above procedure in superposition.

Thus, given $h \in H_\secp$, where $h = f_\Delta \circ f$ for $\Delta \neq 0^n$, along with a trapdoor $\td$ for $f$, we can efficiently prepare the state \[\ket{\psi_{h,y,0}} = \frac{1}{\sqrt{|h^{-1}(y)|}}\sum_{x:h(x)=y}\ket{x}.\]

Then, the procedure $\Recover(\td,y,X)$ would measure register $X$ in the $\{\dyad{\psi_{h,y,0}}{\psi_{h,y,0}}, \bbI - \dyad{\psi_{h,y,0}}{\psi_{h,y,0}}\}$ basis, and output 0 if the first outcome is observed. We have that with probability $1-\negl(\secp)$ over the sampling of $h$,

\begin{align*}
    &\Pr_{x \gets \{0,1\}^m}\left[\Recover(\td,h(x),\ket{h,h(x),0}) \to 0\right] = 1,\\
    &\Pr_{x \gets \{0,1\}^m}\left[\Recover(\td,h(x),\ket{h,h(x),1}) \to 0\right] \leq (1-\delta)^2,
\end{align*}

by the proof of binding in \cref{thm:commitment}. This completes the proof.
%Given any superposition-invertible trapdoor almost-regular one-way function, by \cref{remark:invertible-hash} and \cref{impthm:balancedOWF} we can construct a superposition-invertible $\delta$-balanced function family $\cF$ that is one-way over its range. In turn, we can define superposition-invertible function family $\cH^\cF = \{H_\secp\}_{\secp \in \bbN}$ with associated algorithms $(\Samp,\Invert)$, and define predicate family $\cM = \{\{M[h]\}_{h \in H_\secp}\}_{\secp \in \bbN}$ as above.

\end{proof}

%Now, we let $\cD_{\mathsf{TCR}}$ denote the distribution over $(\Delta,y,b)$ seen by $\cA$ in the TCR experiment, and we let $\cD_{\mathsf{OW}}$ denote the distribution over $(\Delta,y,b)$ seen by $\cA$ in the one-wayness experiment. We will then use the following three claims to complete the proof. In what follows, for any $z \in \{0,1\}^n$, we define $p_z = \Pr_{x \gets \{0,1\}^m}[f(x) = z]$.

%\begin{claim}
%\[\Pr_{(\Delta,y,b) \gets \cD_{\mathsf{TCR}}}\left[z_0 \in \mathsf{BAD} \vee z_1 \in \mathsf{BAD} : \{z_0,z_1\} \coloneqq h_\Delta^{-1}(y)\right] = \negl(\secp).\]
%\end{claim}

%\begin{proof}
%For any fixed $\Delta$, which partitions $\{0,1\}^n$ into sets of two, we have that the probability of sampling a random pair with at least one element in $\mathsf{BAD}$ is at most
%\[|\mathsf{BAD}| \cdot \left(\frac{1+\delta}{2^n} + \negl(\secp)\right) = \frac{k}{2^n} + \negl(\secp) = \negl(\secp),\] by properties (1), (2), and (3) of \cref{def:balanced}.
%\end{proof}

%\begin{claim}
%\[\Pr_{(\Delta,y,b) \gets \cD_{\mathsf{OW}}}\left[z_0 \in \mathsf{BAD} \vee z_1 \in \mathsf{BAD} : \{z_0,z_1\} \coloneqq h_\Delta^{-1}(y)\right] = \negl(\secp).\]
%\end{claim}

%\begin{proof}
%This follows directly from property (2) of \cref{def:balanced} and a union bound.
%\end{proof}

%For any $(\Delta,y,b)$, define \[p_{(\Delta,y,b)}^{\mathsf{OW}} \coloneqq \Pr_{(\Delta^*,y^*,b^*) \gets \cD_{\mathsf{OW}}}[(\Delta,y,b) = (\Delta^*,y^*,b^*)], ~~~ p_{(\Delta,y,b)}^{\mathsf{TCR}} \coloneqq \Pr_{(\Delta^*,y^*,b^*) \gets \cD_{\mathsf{TCR}}}[(\Delta,y,b) = (\Delta^*,y^*,b^*)].\]

%\begin{claim}
%For each fixed $(\Delta,y,b)$ such that $h_{\Delta}^{-1}(y) \cap \mathsf{BAD} = \emptyset$, \[p_{(\Delta,y,b)}^{\mathsf{OW}} \geq (1-\delta) \cdot p_{(\Delta,y,b)}^{\mathsf{TCR}}.\]
%\end{claim}

%\begin{proof}
%We will calculate each of the two probabilities in the claim separately. Note first that each $(\Delta,y)$ defines a fixed $\{z_0,z_1\} \coloneqq h_{\Delta}^{-1}(y)$. Then we can write 
%\begin{align*}
%    p_{(\Delta,y,b)}^{\mathsf{OW}} = p_{z_{1-b}} \cdot p_{z_b},
%\end{align*}
%where $p_{z_{1-b}}$ is the probability that the challenger samples $z_{1-b}$ and $p_{z_b}$ is the probability that the reduction $\cA'$ samples $z_b$. Next, we can write 
%\begin{align*}
%    p_{(\Delta,y,b)}^{\mathsf{TCR}} = \frac{1}{2^n-1} \cdot (p_{z_0} + p_{z_1}) \cdot \frac{p_{z_b}}{p_{z_0} + p_{z_1}} = \frac{p_{z_b}}{2^n-1},
%\end{align*}
%which models sampling $\Delta$ uniformly at random from $\{0,1\}^n \setminus \{0^n\}$, measuring $y = h_\Delta(f(\cdot))$, and then measuring the bit $b$. Thus,

%\[p_{(\Delta,y,b)}^{\mathsf{OW}} /
%p_{(\Delta,y,b)}^{\mathsf{TCR}} \geq p_{1-z_b} \cdot (2^n-1) \geq 1-\delta,\]

%by property (3) of \cref{def:balanced}, which competes the proof of the claim.

%\end{proof}

%Now we complete the proof of the theorem by showing that $\cA'$ violates the one-wayness of $\cF$.

%\begin{align*}
    %\Pr[\Exp_{\mathsf{OW}} = 1] &\geq \sum_{\Delta,y,b}p_{(\Delta,y,b)}^{\mathsf{OW}} \cdot \Pr[f(x') = z : x' \gets \cA(\Delta,y,b,X)] \\
    %&\geq \sum_{\Delta,y,b : h_\Delta^{-1}(y) \cap \mathsf{BAD} = \emptyset}p_{(\Delta,y,b)}^{\mathsf{OW}} \cdot \Pr[f(x') = z : x' \gets \cA(\Delta,y,b,X)] - \negl(\secp) \\
    %&\geq \sum_{\Delta,y,b : h_\Delta^{-1}(y) \cap \mathsf{BAD} = \emptyset}(1-\delta) \cdot p_{(\Delta,y,b)}^{\mathsf{TCR}} \cdot \Pr[f(x') = z : x' \gets \cA(\Delta,y,b,X)] - \negl(\secp) \\
    %&\geq (1-\delta)\Pr[\Exp_{\mathsf{TCR}} = 1] - \negl(\secp) \\
    %&\geq \nonnegl(\secp).
%\end{align*}


\subsection{Balanced Binary-Measurement TCR from Pseudorandom Group Actions}
\label{sec:hmy}
Finally, we show that the recent public-key encryption scheme of \cite{HMY} based on pseudorandom group actions has publicly-verifiable deletion, which follows fairly immediately from our framework. First, we need some preliminaries from \cite{JQSY,HMY}.

%\dakshita{Here, can we instead just rely on the fact that there is a trapdoor that helps distinguish the superposition from the mixture? and relax the trapdoor property from Section 7.3 to only ask for distinguishing given a trapdoor (we can say that it is implied by superposition-invertibility..). This way we will just prove that the group action based scheme is an instance of the appropriate trapdoored collapsing function.}

\begin{definition}[Group Action] 
Let $G$ be a (not necessarily abelian) group, $S$ be a set, and $\star: G \times S \to S$ be a function where we write $g \star s$ to mean $\star(g,s)$. We say that $(G,S,\star)$ is a group action if it satisfies the following:
\begin{itemize}
    \item For the identity element $e \in G$ and any $s \in S$, we have $e \star s = s$.
    \item For any $g,h \in G$ and any $s \in S$, we have $(gh) \star s = g \star (h \star s)$.
\end{itemize}
\end{definition}

\cite{JQSY,HMY} also require a number of efficiency properties from the group action, and we refer the reader to their papers for these specifications.

\begin{definition}[Pseudorandom Group Action]
A group action $(G,S,\star)$ is \emph{pseudorandom} if it satisfies the following:
\begin{itemize}
    \item We have that \[\Pr_{s,t \gets S}[\exists g \in G \text{ s.t. } g \star s = t] = \negl(\secp).\]
    \item For any QPT adversary $\{\cA_\secp\}_{\secp \in \bbN}$,
    \[\big| \Pr_{s \gets S, g \gets S}[\cA_\secp(s,g \star s) = 1] - \Pr_{s,t \gets S}[\cA_\secp(s,t) = 1]\big| = \negl(\secp).\]
\end{itemize}
\end{definition}

Given a pseudorandom group action $(G,S,\star)$, \cite{HMY} consider the following hash family $\cH^{(G,S,\star)} = \{H_h\}_{h \in S_G}$, where $S_G = \{(s_0,s_1) \in S^2: \exists g \in G \text{ s.t. } s_1 = g \star s_0\}$.
\begin{itemize}
    \item The algorithm $\Samp(1^\secp)$ samples $s_0 \gets S,g \gets G$ and outputs $h = (s_0,s_1)$ as the description of the hash and $\td = g$ as the trapdoor.
    \item For an input $(b,k)$ where $b \in \{0,1\}$ and $k \in G$, define $h(b,k) \coloneqq k \star s_b$.
\end{itemize}



\begin{claim}
$\cH^{(G,S,\star)}$ is a balanced binary-measurement TCR hash with trapdoor phase-recoverability.
\end{claim}

\begin{proof}
Define predicate family $\cM$ as $M[h](b,k) = b$. That is, it does not depend on $h$, and simply outputs the first bit of its input. Then, this claim actually follows immediately from what is already proven in \cite{HMY}. First, \cite[Theorem 4.10]{HMY} shows that given $\td$ and $y \in S$, it is possible to perfectly distinguish \[\frac{1}{\sqrt{2}}\ket{0,h_0^{-1}(y)} + \frac{1}{\sqrt{2}}\ket{1,h_1^{-1}(y)} ~~ \text{and} ~~ \frac{1}{\sqrt{2}}\ket{0,h_0^{-1}(y)} - \frac{1}{\sqrt{2}}\ket{1,h_1^{-1}(y)},\] where $h_b \coloneqq h(b,\cdot)$, which establishes trapdoor phase-recoverability. Next, \cite[Theorem 4.19]{HMY} shows that $\cH^{(G,S,\star)}$ satisfies \emph{conversion hardness}, which is equivalent to our notion of $\cM$-target-collision-resistance. 
\end{proof}

By \cref{thm:PKE} and \cref{cor:compiler}, we obtain the following corollary.

\begin{corollary}
Assuming the existence pseudorandom group actions, there exists PKE with $\PVD$. Additionally assuming post-quantum \[X \in \left\{\begin{array}{r}\text{quantum fully-homormophic encryption, attribute-based encryption}, \\ \text{witness encryption, timed-release encryption}\end{array}\right\},\] there exists $X$ with $\PVD$.
\end{corollary}

%Now consider the following public-key encryption scheme from \cite{HMY}, where we have added the $\Del$ and $\Vrfy$ algorithms.

%\begin{itemize}
    %\item $\Gen(1^\secp)$: Sample $(h,\td) \gets \Samp(1^\secp)$ and set $\pk \coloneqq h, \sk \coloneqq \td$.
    %\item $\Enc(\pk,b)$: Prepare the state 
    %\[\frac{1}{\sqrt{2|G|}}\sum_{k \in G}\ket{0,k}_X\ket{h(0,k)}_Y + (-1)^b\ket{1,k}_X\ket{h(1,k)}_Y\] and measure the $Y$ register to obtain $y$ and a left-over state on register $X$. Then, set \[\ket{\ct} \coloneqq (y,X), ~~ \vk = (h,y).\]
    %\item $\Dec(\sk,\ket{\ct})$: The details of this algorithm are not important to us, so we refer the reader to \cite{HMY}. They demonstrate and prove a decryption algorithm that is correct (\cite[Theorem 4.10]{HMY}).
    %\item $\Del(\ket{\ct})$: Measure $\ket{\ct}$ in the standard basis to obtain $\pi \coloneqq x$.
    %\item $\Vrfy(\vk,\pi)$: Output $\top$ if $h(x) = y$.
%\end{itemize}

%\begin{theorem}
%If $(G,S,\star)$ is a pseudorandom group action, then the above scheme defined based on $\cH^{(G,S,\star)}$ is a PKE scheme with $\PVD$.
%\end{theorem}

%\begin{proof}
%Since \cite{HMY} showed standard semantic security, it suffices to show that $\cH^{(G,S,\star)}$ is certified everlasting $(\cU,\cM)$-target-collapsing, which will immediately imply certified deletion security. Now, it was shown in \cite[Theorem 4.19]{HMY} that $\cH^{(G,S,\star)}$ satisfies \emph{conversion hardness}, which is equivalent to our notion of $(\cU,\cM)$-target-collision-resistance. Thus, by \cref{thm:TC-from-TCR} and \cref{thm:CETC-generalization}, $\cH^{(G,S,\star)}$ is certified everlasting $(\cU,\cM)$-target-collapsing, which completes the proof.
%\end{proof}













