\section{Main Theorem: Certified Everlasting Target-Collapsing}

Collapsing can be considered a quantum analogue of classical collision-resistance. In the classical setting, a weaker security notion has also been studied, called \emph{target}-collision-resistance \cite{10.1145/73007.73011}, which requires that for any \emph{fixed} input $x$, no polynomial-time adversary can find a collision $x' \neq x$ such that $h(x) = h(x')$. We now formalize this definition, and then introduce a quantum analogue which we call \emph{target-collapsing}.

\begin{definition}[Target-Collision-Resistant Hash Function] A hash function family $\algo H = \{H_\lambda\}_{\lambda \in \N}$ is target-collision-resistant if, for every $\QPT$ adversary $\algo A$,
$$
|
\Pr[ \mathsf{TargetCollRes}_{\algo H,\algo A,\lambda}=1]| \leq \negl(\lambda).
$$
Here, the experiment $\mathsf{TargetCollRes}_{\algo H,\algo A,\lambda}$ is defined as follows:

\begin{enumerate}
    \item $\algo A$ sends an input $x \in \{0,1\}^{m(\secp)}$ to the challenger.
    \item The challenger samples a random hash function $h \rand H_\secp$ and sends $h$ to $\algo A$.
    \item $\algo A$ responds with another input $x' \in \{0,1\}^{m(\secp)}$.
    \item The experiment outputs 1 if $x \neq x'$ and $h(x) = h(x')$.
\end{enumerate}

\end{definition}

\begin{definition}[Target-Collapsing Hash Function]\label{def:targeted-collapsing} A hash function family $\algo H = \{H_\lambda\}_{\lambda \in \N}$ is target-collapsing if, for every $\QPT$ adversary $\algo A$,
$$
|
\Pr[ \mathsf{TargetCollapseExp}_{\algo H,\algo A,\lambda}(0)=1] - \Pr[ \mathsf{TargetCollapseExp}_{\algo H,\algo A,\lambda}(1)=1]
| \leq \negl(\lambda).
$$
Here, the experiment $\mathsf{TargetCollapseExp}_{\algo H,\algo A,\lambda}(b)$ is defined as follows:
\begin{enumerate}

    \item $\algo A$ sends an $m(\lambda)$-qubit quantum state
    in system $X$ to the challenger.
    
    \item The challenger samples a random hash function $h \rand H_\lambda$
    and coherently computes $h$ (into an auxiliary system $Y$) given the state in system $X$. The challenger then measures system $Y$ in the computational basis, which results in an outcome $y \in \bit^{n(\lambda)}$.
\item If $b=0$, the challenger does nothing. Else, if $b=1$, the challenger additionally measures system $X$ in the computational basis. Finally, the challenger sends the outcome state in systems $X$ to $\algo A$, together with the string $y \in \bit^{n(\lambda)}$ and a classical description of $h$.

\item $\algo A$ returns a bit $b'$, which we define as the output of the experiment.
\end{enumerate}
\end{definition}



\begin{definition}[Certified Everlasting Target-Collapsing] A hash family $\algo H = \{H_\lambda\}_{\lambda \in \N}$ is certified everlasting target-collapsing if, for any
$\algo A = (\algo A_0,\algo A_1)$ consisting of a computationally bounded algorithm $\algo A_0$ and a computationally unbounded algorithm $\algo A_1$, it holds that
$$
|
\Pr[ \mathsf{EvTargetCollapseExp}_{\algo H,\algo A,\lambda}(0)=1] - \Pr[ \mathsf{EvTargetCollapseExp}_{\algo H,\algo A,\lambda}(1)=1]
| \leq \negl(\lambda).
$$
Here, the experiment $\mathsf{EvTargetCollapseExp}_{\algo H,\algo A,\lambda}(b)$ is defined as follows:
\begin{enumerate}
   \item The adversary sends an $m(\lambda)$-qubit quantum state
    in system $X$ to the challenger.
    
    \item The challenger samples a random hash function $h \rand H_\lambda$
    and coherently computes $h$ (into an auxiliary system $Y$) given the state in system $X$. The challenger then measures system $Y$ in the computational basis, which results in an outcome $y \in \bit^{n(\lambda)}$.
\item If $b=0$, the challenger does nothing. Else, if $b=1$, the challenger additionally measures system $X$ in the computational basis. Finally, the challenger sends the outcome state in system $X$ to $\algo A$, together with the string $y \in \bit^{n(\lambda)}$ and a classical description of $h$.

\item $\algo A_0$ sends a classical certificate $\pi \in \bit^{m(\lambda)}$ to the challenger, and 
then forwards a polynomial-sized quantum state to $\algo A_1$.

\item The challenger checks if $h(\pi)=y$. If true, the challenger replies with $\top$ and the game continues; else, the challenger outputs $\bot$ and $\algo A$ loses.

\item $\algo A_1$ returns a bit $b'$, which we define as the output of the experiment.
\end{enumerate}
\end{definition}


%\section{Target-Collapsing Hashes and their Applications}

%\james{we should probably combine sections 4 and 5 into one section titled Certified Everlasting Target-Collapsing}

Our main theorem is the following: 

\begin{theorem}\label{thm:targeted-collapsing-CD} 

Let $\algo H = \{H_\lambda\}_{\lambda \in \N}$ be a hash function family which is both target-collapsing and target-collision-resistant. Then, $\algo H$ is certified everlasting target-collapsing, i.e. for any adversary
$\algo A = (\algo A_0,\algo A_1)$, where $\algo A_0$ is computationally bounded and $\algo A_1$ is computationally unbounded,
$$
|\Pr[\mathsf{EvTargetCollapseExp}_{\algo H,\algo A,\lambda}(0)] - \Pr[\mathsf{EvTargetCollapseExp}_{\algo H,\algo A,\lambda}(1)]| \leq \negl(\lambda). 
$$
\end{theorem}

%\subsection{Proof of certified everlasting collapsing}

%To prove \Cref{thm:targeted-collapsing-CD}, 


\begin{proof}

We consider the following hybrids. 

\begin{itemize}
    \item $\mathsf{Exp}_0(b)$:
\begin{enumerate}
   \item The adversary sends an $m(\lambda)$-qubit quantum state $\rho$ in a system $X$ to the challenger.
    
    \item The challenger samples a random hash function $h \rand H_\lambda$
    and coherently computes $h$ into an auxiliary system $Y$ initialized to $\ket{0^n}$.
    The challenger then measures $Y$ in the computational basis, which results in $y \in \bit^{n}$ and a partially measured state $\rho_y$.
\item If $b=0$, the challenger does nothing. Else, if $b=1$, the challenger additionally measures system $X$ in the computational basis. Finally, the challenger sends the outcome state in system $X$ to $\algo A$, together with the string $y \in \bit^{n(\lambda)}$ and a classical description of $h$.

\item $\algo A_0$ sends a classical certificate $\pi \in \bit^{m(\lambda)}$ to the challenger, and 
then forwards a polynomial-sized quantum state to $\algo A_1$.

\item The challenger checks if $h(\pi)=y$. If true, the challenger replies with $\top$ and the game continues; else, the challenger outputs $\bot$ and $\algo A$ loses.

\item $\algo A_1$ returns a bit $b'$, and wins if $b=b'$.
\end{enumerate}

    \item $\mathsf{Exp}_1(b)$:
    \begin{enumerate}
   \item The adversary sends an $m(\lambda)$-qubit quantum state $\rho$ in a system $X$ to the challenger.
    
    \item The challenger samples a random hash function $h \rand H_\lambda$
    and coherently computes $h$ into an auxiliary system $Y$ initialized to $\ket{0^n}$.
    The challenger then measures $Y$ in the computational basis, which results in $y \in \bit^{n}$ and a partially measured state $\rho_y$.
    
        \item The challenger samples a random string, $z \rand \bit^m$, prepares a $\ket{+}$ state in system $C$ and applies a controlled-$\mathsf{Z}^{z}$ operation, resulting in the state $\sigma$ given by
$$
\sigma_{CX} = \frac{1}{2} \sum_{c,c' \in \bit} \ketbra{c}{c'}_C \otimes \mathsf{Z}^{c \cdot z}{\rho_y}_X \left(\mathsf{Z}^{c' \cdot z}\right)^\dag.
$$
and sends system $X$ to $\algo A_0$, together with $y \in \bit^n$ and a classical description of $h$.

\item $\algo A_0$ replies with a certificate $\pi \in \bit^m$, and sends a quantum state $\rho_{\aux,c}$ to $\algo A_1$.
\item 
The challenger checks if $h(\pi)=y$. If yes, the game continues; else, $\algo A$ loses. Then, the challenger measures system $C$ with outcome $c'$. If $c'=b$, the game continues; else the challenger aborts and $\algo A$ wins.
\item $\algo A_1$ sends $b' \in \bit$ to the challenger.
\item The challenger checks if $b'=b$. If true, $\algo A$ wins.
    \end{enumerate}

\item $\mathsf{Exp}_2(b)$:
    \begin{enumerate}
  \item The adversary sends an $m(\lambda)$-qubit quantum state $\rho$ in a system $X$ to the challenger.
    
    \item The challenger samples a random hash function $h \rand H_\lambda$
    and coherently computes $h$ into an auxiliary system $Y$ initialized to $\ket{0^n}$.
    The challenger then measures $Y$ in the computational basis, which results in $y \in \bit^{n}$ and a partially measured state $\rho_y$.
    
        \item The challenger samples a random string, $z \rand \bit^m$, prepares a $\ket{+}$ state in system $C$ and applies a controlled-$\mathsf{Z}^{z}$ operation, resulting in the state $\sigma$ given by
$$
\sigma_{CX} = \frac{1}{2} \sum_{c,c' \in \bit} \ketbra{c}{c'}_C \otimes \mathsf{Z}^{c \cdot z}{\rho_y}_X \left(\mathsf{Z}^{c' \cdot z}\right)^\dag.
$$
and sends system $X$ to $\algo A_0$, together with $y \in \bit^n$ and a classical description of $h$.

\item $\algo A_0$ replies with a certificate $\pi \in \bit^m$, and sends a quantum state $\rho_{\aux,c}$ to $\algo A_1$.
\item 
The challenger checks if $h(\pi)=y$. If yes, the game continues; else, $\algo A$ loses. Then, the challenger applies the following projective measurement to system $C$:
$$
\Big\{\proj{\phi_\pi^{ z}},I - \proj{\phi_\pi^{ z}}\Big\} \,\quad \text{ with } \quad
\ket{\phi_\pi^{ z}} = \frac{1}{\sqrt{2}}  ( \ket{0} +  (-1)^{\langle \pi, z \rangle } \ket{1}).
$$
If the measurement succeeds, the game continues; else the challenger aborts. Finally, the challenger measures system $C$ with outcome $c'$. If $c'=b$, the game continues; else the challenger aborts.
\item $\algo A_1$ sends $b' \in \bit$ to the challenger.
\item The challenger checks if $b'=b$. If true, $\algo A$ wins.
    \end{enumerate}

Finally, we also use the following hybrid which is convenient for the sake of the proof.

\item $\mathsf{Exp}_3(b)$:
     \begin{enumerate}
   \item The adversary sends an $m(\lambda)$-qubit quantum state $\rho$ in a system $X$ to the challenger.
    
    \item The challenger samples a random hash function $h \rand H_\lambda$
    and coherently computes $h$ into an auxiliary system $Y$ initialized to $\ket{0^n}$.
    The challenger then measures systems $XY$ in the computational basis, which results in outcomes $x_0 \in \bit^m$ and $y=h(x_0) \in \bit^{n}$.
    
        \item The challenger samples a random string, $z \rand \bit^m$, prepares a $\ket{+}$ state in system $C$ and applies a controlled-$\mathsf{Z}^{z}$ operation, resulting in the state $\sigma$ given by
$$
\sigma_{CX} = \frac{1}{2} \sum_{c,c' \in \bit} \ketbra{c}{c'}_C \otimes \mathsf{Z}^{c \cdot z}{\proj{x_0}}_X \left(\mathsf{Z}^{c' \cdot z}\right)^\dag.
$$
and sends system $X$ to $\algo A_0$, together with $y \in \bit^n$ and a classical description of $h$.

\item $\algo A_0$ replies with a certificate $\pi \in \bit^m$, and sends a quantum state $\rho_{\aux,c}$ to $\algo A_1$.
\item 
The challenger checks if $h(\pi)=y$. If yes, the game continues; else, $\algo A$ loses. Then, the challenger applies the following projective measurement to system $C$:
$$
\Big\{\proj{\phi_\pi^{ z}},I - \proj{\phi_\pi^{ z}}\Big\} \,\quad \text{ with } \quad
\ket{\phi_\pi^{ z}} = \frac{1}{\sqrt{2}}  ( \ket{0} +  (-1)^{\langle \pi, z \rangle } \ket{1}).
$$
If the measurement succeeds, the game continues; else the challenger aborts. Finally, the challenger measures system $C$ with outcome $c'$. If $c'=b$, the game continues; else the challenger aborts.
\item $\algo A_1$ sends $b' \in \bit$ to the challenger.
\item The challenger checks if $b'=b$. If true, $\algo A$ wins.
    \end{enumerate}
\end{itemize}

Before we analyze the probability of distinguishing between the consecutive hybrids, we first show that the following statements hold for the final experiment $\mathsf{Exp}_3$:

\begin{claim}\label{claim:identical-certificate}
With overwhelming probability, the certificate $\pi$ returned by the adversary in
$\mathsf{Exp}_3$ is identical to the pre-image $x_0$ obtained by the challenger. In other words,\james{something that is not explicitly captured in the statement below is the fact that the adversary may have some state from earlier that is entangled with $\sigma$}
$$
\Pr \left[
 h(\pi) = y
\,\,\, \wedge \,\,\,
\pi \,\neq\, x_0
 \,\, \vline \,\, \substack{
 z \rand \bit^m\\
(\pi, \rho_{\aux}) \leftarrow \algo A_0(\sigma_X^{z})  }\right] \leq \negl(\lambda),
$$
where $y = h(x_0) \in \bit^{n}$ is the image and where $\sigma_X^{z}$ is the reduced state %\james{Isn't the reduced state just $\ket{\vec x_0}$? If so, we wouldn't have to sample $\vec z$ in the above game, and we can just give $\cA_0$ the vector $\vec x_0$ (which directly corresponds to targeted collision-resistance)} 
with respect to
$$
\sigma_{CX} = \frac{1}{2} \sum_{c,c' \in \bit} \ketbra{c}{c'}_C \otimes \mathsf{Z}^{c \cdot z}{\proj{x_0}}_X \left(\mathsf{Z}^{c' \cdot z}\right)^\dag.
$$
\end{claim}
\begin{proof}
Suppose for the sake of contraction that the probability is at least $1/\poly(\lambda)$. We now show that we can use $\algo A_0$ to break the target-collision-resistance of the hash family $\algo H = \{H_\lambda\}_{\lambda \in \N}$. 

Our reduction proceeds as follows:
\begin{enumerate}
 \item Run $\algo A_0$ to obtain an $m(\lambda)$-qubit quantum state $\rho$ in system $X$.

\item Measure system $X$ and send the outcome $x_0$ to the challenger.
    
\item Once the challenger replies with a description of a hash function $h \rand H_\lambda$, send the register $\ket{x_0}$, the image $y = h(x_0)$ as well as a description of $h$ to $\algo A_0$.

\item When $\algo A_0$ outputs $(\pi, \rho_{\aux})$, discard $\rho_{\aux}$ and output $(x_0,\pi)$.
\end{enumerate}
Notice that the state $\ket{x_0}$ which is sent to $\algo A_0$ is identical to the reduced state $\sigma_X$ with respect to $\sigma_{CX}$.
By assumption, $\algo A_0$ outputs a valid certificate $\pi \neq x_0$ such that $h(\pi) = h(x_0)$ with probability at least $1/\poly(\lambda)$. Thus, we have broken the target-collision-resistance of $\algo H$.
\end{proof}

\begin{claim}\label{claim:measurement-succeeds-wp-1}
If the challenger accepts the deletion certificate $\pi$ in Step $5$ of $\mathsf{Exp}_3$, then 
the subsequent projective measurement performed by the challenger succeeds with overwhelming probability.
\end{claim}
\begin{proof}
This follows from \Cref{claim:identical-certificate}, which states that the certificate $\pi$ returned by the adversary in
$\mathsf{Exp}_3$ is identical to the pre-image produced by the challenger with all but negligible probability.
Therefore, the projective measurement must also succeed with overwhelming probability.
\end{proof}

We now show the following:

\begin{claim}
$$
\mathsf{Adv}(\mathsf{Exp}_2) = 0.
$$
\end{claim}
\begin{proof}
We can now directly claim that the advantage is zero as follows. First, in the case that the challenger aborts (either because the deletion certificate is invalid or their projection fails), the experiment does not involve $b$, and thus the advantage of the adversary is $0$. Second, in the case that the challenger's projection succeeds, the register $C$ is either in the state
$$
\ket{\phi_\pi^{\vec z}} = \frac{1}{\sqrt{2}}  ( \ket{0} +  (-1)^{\langle \pi, \vec z \rangle } \ket{1}) \quad\,\, \text{ or } \quad\,\, (\ket{\phi_\pi^{\vec z}})^\perp = \frac{1}{\sqrt{2}}  ( \ket{0} -  (-1)^{\langle \pi, \vec z \rangle } \ket{1}) 
$$
for some $z \in \bit^m$, and thereby completely
unentangled from system $A$. Notice that the challenger's measurement of system $C$ with outcome $c'$ results in a uniformly random bit, which completely masks $b$. Therefore, the adversary's overall advantage in $\mathsf{Exp}_2$ is identical to $0$. 
\end{proof}
Next, we argue the following:

\begin{claim}
$$
|\mathsf{Adv}(\mathsf{Exp}_2) - \mathsf{Adv}(\mathsf{Exp}_1) | \,\leq \, \negl(\lambda).
$$    
\end{claim}
\begin{proof}
Recall that \Cref{claim:measurement-succeeds-wp-1} shows that the projective measurement performed by the challenger in Step $5$ of $\mathsf{Exp}_3$ succeeds with overwhelming probability.
We now argue that the same is also true for $\mathsf{Exp}_2$. 
Suppose for the sake of contradiction that there is a non-negligible difference between the success probabilities of the measurement. 
We now show that this implies the existence of an efficient distinguisher $\algo D$ that breaks the target-collapsing property of the hash family $\algo H = \{H_\lambda\}_{\lambda \in \N}$. 

Our reduction proceeds as follows:
The distinguisher $\algo D$ runs $\algo A_0$ to obtain an $m$-qubit state $\rho$ in system $X$,
and forwards it to the challenger who responds
with a description of a hash function $h \rand H_\lambda$, an image $y \in \bit^m$ and a state $\rho_y$ in system $X$
which is either a partially measured state (consisting of a superposition of pre-images)
or a single measured pre-image $\proj{x_0}$ such that $x_0 \in \bit^m$ and $h(x_0)=y$.
Next, $\algo D$ samples a random string $z \rand \bit^m$ and runs $\algo A_0$ given as input system $X$ of the state
$$
\sigma_{CX} = \frac{1}{2} \sum_{c,c' \in \bit} \ketbra{c}{c'}_C \otimes \mathsf{Z}^{c \cdot z}{\rho_y}_X \left(\mathsf{Z}^{c' \cdot z}\right)^\dag.
$$
Once $\algo A_0$ replies with a certificate $\pi$, then $\algo D$ applies the projective measurement given by
$$
\Big\{\proj{\phi_\pi^{z}},I - \proj{\phi_\pi^{z}}\Big\} \,\quad \text{ with } \quad
\ket{\phi_\pi^{z}} = \frac{1}{\sqrt{2}}  ( \ket{0} +  (-1)^{\langle \pi, z \rangle } \ket{1}).
$$
$\algo D$ outputs $1$, if the measurement succeeds, and $0$ otherwise.
If there is a non-negligible difference in success probabilities between the case when $\rho_y$ is a superposition of pre-images, or $\rho_y = \proj{x_0}$ is a single pre-image of $y$, this immediately breaks the targeted-collapsing property of $\algo H$. Therefore, the projective measurement in Step $5$ of $\mathsf{Exp}_2$ must also succeed with overwhelming probability.

Recall that $\mathsf{Exp}_2$ is identical to $\mathsf{Exp}_1$, except that the challenger applies an additional a measurement in Step $5$. Because the measurement succeeds with overwhelming probability, the advantage of the adversary must remain the same up to a negligible amount. This proves the claim.
\end{proof}

Suppose that $\algo A = (\algo A_0, \algo A_1)$ wins at $\mathsf{Exp}_0$ with probability $\epsilon >0$. We now show the following:

\begin{claim}\label{claim:exp_0-exp_1}
$$
|\mathsf{Adv}(\mathsf{Exp}_1) - \mathsf{Adv}(\mathsf{Exp}_0) | = \frac{\epsilon}{2}.
$$    
\end{claim}
\begin{proof}
Notice that the reduced state $\sigma_X$ in $\mathsf{Exp}_1$ is an equal mixture
$$
\sigma_X = \frac{1}{2} {\rho_y} + \frac{1}{2} \,  \underset{z}{\mathbb{E}} \left[\mathsf{Z}^{z}{\rho_y} \left(\mathsf{Z}^{ z}\right)^\dag \right].
$$
From \Cref{lem:random-Z} it follows that a random Pauli-$Z$ twirl induces a measurement in the computational basis. In other words, on average over the choice of $z \in \bit^m$, we have
\begin{align*}
\underset{z}{\mathbb{E}} \left[\mathsf{Z}^{z}{\rho_y} \left(\mathsf{Z}^{ z}\right)^\dag \right] & = \sum_{x \in \bit^m} \Tr[\ketbra{x}{x} \rho_y] \,\ketbra{x}{x}
\end{align*}
Therefore, the reduced state $\sigma_X$ in $\mathsf{Exp}_1$ precisely matches the state in $\mathsf{Exp}_0$:
with probability $1/2$, $\rho_y$ corresponds to a superposition of pre-images of $y$ and, with probability $1/2$, $\rho_y$ is equal to $\proj{x_0}$, i.e. it corresponds to a single (measured) pre-image $x_0$ such that $h(x_0) = y$.

Finally, we observe that $c'$ in $\mathsf{Exp}_1$ equals $b$ with probability $\frac{1}{2}$. Letting $\Pr[\mathsf{Exp}_0=1] = \epsilon$, we get
\begin{align*}
\big|\mathsf{Adv}(\mathsf{Exp}_1) - \mathsf{Adv}(\mathsf{Exp}_0) \big|
=\big| \frac{\epsilon}{2} - \epsilon \big| = \frac{\epsilon}{2}. 
\end{align*}
\end{proof}
Putting everything together, we get that $\frac{\epsilon}{2} \leq \negl(\lambda)$, and thus
\begin{claim}
$$
\epsilon = \mathsf{Adv}(\mathsf{Exp}_0) \leq \negl(\lambda).
$$    
\end{claim}

This completes the proof.

\end{proof}

\subsection{Distributional and Partial Target-Collapsing}

We now consider two ways to relax the notion of target-collapsing. First, we consider target-collapsing with respect to a particular distribution over initial inputs. That is, rather than allowing the adversary to submit an arbitrary state on register $X$ at the beginning of the experiment, we fix a state $\sum_x \sqrt{D(x)}\ket{x}$ specified by some distribution $D$ over the domain. We also consider target-collapsing where the measurement made by the challenger (in the case $b=1$) is not necessarily a full standard basis measurement. In particular, we consider a binary outcome measurement\footnote{One can easily generalize this notion to consider measurements with arbitrarily many outcomes, although we don't need that for this work.} specified by some classical predicate $P$ on the domain (which may depend on the choice of hash function $h$). That is, instead of measuring register $X$ in the standard basis, the challenger would measure a single bit obtained by evaluating $P(\cdot)$ on register $X$. The formal definition follows.

%We also generalize the notion of a hash function family to allow for sampling $h \in H_\secp$ along with some auxiliary information $\aux$, which we denote by $(h,\aux) \gets H_\secp$. We further parameterize the definition below by some function $\cZ_\secp$ 

\begin{definition}[$(\cD,\cP)$-Target-Collapsing Hash Function]
A hash function family $\cH = \{H_\secp : \{0,1\}^{m(\secp)} \to \{0,1\}^{n(\secp)}\}_{\secp \in \bbN}$ is $(\cD,\cP)$-target-collapsing for some distribution $\cD = \{D_\secp\}_{\secp \in \bbN}$ over $\{\{0,1\}^{m(\secp)}\}_{\secp \in \bbN}$ and family of predicates $\cP = \{\{P[h] : \{0,1\}^{m(\secp)} \to \{0,1\}\}_{h \in H_\secp}\}_{\secp \in \bbN}$ if, for every QPT adversary $\cA = \{\cA_\secp\}_{\secp \in \bbN}$,
$$
|
\Pr[ \mathsf{TargetCollapseExp}_{\algo H,\algo A,\algo D,\algo P,\lambda}(0)=1] - \Pr[ \mathsf{TargetCollapseExp}_{\algo H,\algo A,\algo D,\algo P,\lambda}(1)=1]
| \leq \negl(\lambda).
$$
Here, the experiment $\mathsf{TargetCollapseExp}_{\algo H,\algo A,\algo D,\algo P,\lambda}(b)$ is defined as follows:

\begin{enumerate}

    \item The challenger prepares the state \[\sum_{x \in \{0,1\}^{m(\secp)}}\sqrt{D_\secp(x)}\ket{x}\] on register $X$, and samples a random hash function $h \rand H_\lambda$. Then, it coherently computes $h$ on $X$ (into an auxiliary $n(\secp)$-qubit system $Y$) and measures system $Y$ in the computational basis, which results in an outcome $y \in \bit^{n(\lambda)}$.
    \item If $b=0$, the challenger does nothing. Else, if $b=1$, the challenger coherently computes $P[h]$ on $X$ (into an auxiliary one-qubit system $P$) and measures system $P$ in the computational basis. Finally, the challenger sends the outcome state in system $X$ to $\algo A$, together with the string $y \in \bit^{n(\lambda)}$ and a classical description of $h$.
    \item $\algo A$ returns a bit $b'$, which we define as the output of the experiment.
\end{enumerate}

We also say that the hash function family is \emph{certified everlasting} $(\cD,\cP)$-target-collapsing if for any $\algo A = (\algo A_0,\algo A_1)$ consisting of a computationally bounded algorithm $\algo A_0$ and a computationally unbounded algorithm $\algo A_1$, it holds that
$$
|
\Pr[ \mathsf{EvTargetCollapseExp}_{\algo H,\algo A,\algo D,\algo P,\lambda}(0)=1] - \Pr[ \mathsf{EvTargetCollapseExp}_{\algo H,\algo A,\algo D,\algo P,\lambda}(1)=1]
| \leq \negl(\lambda).
$$
Here, the experiment $\mathsf{EvTargetCollapseExp}_{\algo H,\algo A,\algo D,\algo P,\lambda}(b)$ is defined as follows:

\begin{enumerate}

    \item The challenger prepares the state \[\sum_{x \in \{0,1\}^{m(\secp)}}\sqrt{D_\secp(x)}\ket{x}\] on register $X$, and samples a random hash function $h \rand H_\lambda$. Then, it coherently computes $h$ on $X$ (into an auxiliary $n(\secp)$-qubit system $Y$) and measures system $Y$ in the computational basis, which results in an outcome $y \in \bit^{n(\lambda)}$.
    \item If $b=0$, the challenger does nothing. Else, if $b=1$, the challenger coherently computes $P_h$ on $X$ (into an auxiliary one-qubit system $P$) and measures system $P$ in the computational basis. Finally, the challenger sends the outcome state in system $X$ to $\algo A_0$, together with the string $y \in \bit^{n(\lambda)}$ and a classical description of $h$.
    \item $\algo A_0$ sends a classical certificate $\pi \in \bit^{m(\lambda)}$ to the challenger, and 
    then forwards a polynomial-sized quantum state to $\algo A_1$.
    \item The challenger checks if $h(\pi)=y$. If true, the challenger replies with $\top$ and the game continues; else, the challenger outputs $\bot$ and $\algo A$ loses.
    \item $\algo A_1$ returns a bit $b'$, which we define as the output of the experiment.
    
\end{enumerate}
\end{definition}

We will also define an analogous notion of $(\cD,\cP)$-target-collision-resistance, as follows.

\begin{definition}[$(\cD,\cP)$-Target-Collision-Resistant Hash Function]
A hash function family $\cH = \{H_\secp : \{0,1\}^{m(\secp)} \to \{0,1\}^{n(\secp)}\}_{\secp \in \bbN}$ is $(\cD,\cP)$-target-collision-resistant for some distribution $\cD = \{D_\secp\}_{\secp \in \bbN}$ over $\{\{0,1\}^{m(\secp)}\}_{\secp \in \bbN}$ and family of predicates $\cP = \{\{P[h] : \{0,1\}^{m(\secp)} \to \{0,1\}\}_{h \in H_\secp}\}_{\secp \in \bbN}$ if, for every QPT adversary $\cA = \{\cA_\secp\}_{\secp \in \bbN}$,
$$
|
\Pr[ \mathsf{TargetCollRes}_{\algo H,\algo A,\algo D,\algo P,\lambda}=1]| \leq \negl(\lambda).
$$
Here, the experiment $\mathsf{TargetCollRes}_{\algo H,\algo A,\algo D,\algo P,\lambda}$ is defined as follows:
\begin{enumerate}
    \item The challenger prepares the state \[\sum_{x \in \{0,1\}^{m(\secp)}}\sqrt{D_\secp(x)}\ket{x}\] on register $X$, and samples a random hash function $h \rand H_\lambda$. Next, it coherently computes $h$ on $X$ (into an auxiliary $n(\secp)$-qubit system $Y$) and measures system $Y$ in the computational basis, which results in an outcome $y \in \bit^{n(\lambda)}$. Next, it coherently computes $P[h]$ on $X$ (into an auxiliary one-qubit system $P$) and measures system $P$ in the computational basis, which results in a bit $b$. Finally, its sends the outcome state in system $X$ to $\algo A$, together with the string $y \in \{0,1\}^{n(\secp)}$ and a classical description of $h$.
    \item $\algo A$ responds with a string $x \in \{0,1\}^{m(\secp)}$.
    \item The experiment outputs 1 if $h(x) = y$ and $P_h(x) = 1-b$.
\end{enumerate}

\end{definition}


Next, we have the following theorems.

\begin{theorem}\label{thm:CETC-generalization}
Let $\cH = \{H_\secp\}_{\secp \in \bbN}$ be a hash function family that is both $(\cD,\cP)$-target-collapsing and $(\cD,\cP)$-target-collision-resistant, for some distribution $\cD$ and family of predicates $\cP$. Then, $\cH$ is certified everlasting $(\cD,\cP)$-target-collapsing.
\end{theorem}

\begin{proof}
This follows from the proof of \cref{thm:targeted-collapsing-CD} with the following tweaks:
\begin{itemize}
    \item Each experiment will start with the challenger preparing the state \[\sum_{x \in \{0,1\}^{m(\secp)}}\sqrt{D_\secp(x)}\ket{x}_X\] rather that the adversary sending an arbitrary state on register $X$.
    \item In Step 3 of $\Exp_1(b)$ and $\Exp_2(b)$, we sample $z \gets \{0,1\}$, coherently apply $P[h](\cdot)$ from the $X$ register into a fresh register $P$, and then apply a controlled-$\mathsf{Z}^{z}$ to the new register $P$, resulting in a state
    \[\propto \frac{1}{\sqrt{2}}\sum_{c \in \{0,1\}} \ket{c}_C \otimes \sum_{x : h(x)=y}\sqrt{D(x)}(-1)^{c \cdot z \cdot P[h](x)}\ket{x}_X\ket{P[h](x)}_P.\] 
    \item In Step 2 of $\Exp_3(b)$, rather than performing a full measurement in the computational basis, we measure $P[h](\cdot)$ on register $X$ to obtain a bit $d$. In Step 3, the resulting state is
    \[\propto \frac{1}{\sqrt{2}}\sum_{c \in \{0,1\}}\ket{c}_C \otimes \sum_{x:h(x)=y,P[h](x)=d}\sqrt{D(x)}(-1)^{c \cdot z \cdot d}\ket{x}_X.\]
    \item In Step 5 of $\Exp_2(b)$ and $\Exp_3(b)$, the challenger applies the following projective measurement to system $C$: 
    \[\Big\{\proj{\phi_\pi^{ z}},I - \proj{\phi_\pi^{ z}}\Big\} \,\quad \text{ with } \quad\ket{\phi_\pi^{ z}} = \frac{1}{\sqrt{2}}  ( \ket{0} +  (-1)^{z \cdot P[h](\pi)} \ket{1}).\]
    \item We prove \cref{claim:measurement-succeeds-wp-1} analogously, by reduction to the $(\cD,\cP)$-target-collision-resistance of $\cH$.
    \item To prove \cref{claim:exp_0-exp_1}, we use the fact that a random Pauli-Z twirl on register $P$ induced a measurement of the bit $P[h](\cdot)$ in the computational basis.
\end{itemize}


%\james{Follows from the proof of \cref{thm:targeted-collapsing-CD} with a few minor tweaks}
\end{proof}

\begin{theorem}\label{thm:TC-from-TCR}
Let $\cH = \{H_\secp : \{0,1\}^{m(\secp)}\to \{0,1\}^{n(\secp)}\}_{\secp \in \bbN}$ be a hash function family that is $(\cD,\cP)$-target-collision-resistant, for some distribution $\cD$ and family of predicates $\cP$. Then, $\cH$ is $(\cD,\cP)$-target-collapsing.
\end{theorem}

\begin{proof}
We will make use of the following fact (a special case of \cite[Claim 3.5]{cryptoeprint:2022/786}).

\begin{fact}\label{fact:distinguish-map}
Let $D$ be a projector, $\Pi_0,\Pi_1$ be  orthogonal projectors, and $\ket{\psi}$ be any state such that $\ket{\psi} \in \mathsf{span}(\Pi_0 + \Pi_1)$. Then,
    \[\big\| \Pi_1D\Pi_0\ket{\psi}\big\|^2 + \big\| \Pi_0D\Pi_1\ket{\psi}\big\|^2 \geq \frac{1}{2}\left(\|D\ket{\psi}\|^2-\left(\|D\Pi_0\ket{\psi}\|^2 + \|D\Pi_1\ket{\psi}\|^2\right)\right)^2.\]
\end{fact}

Now, suppose there exists an adversary $\{\cA_\secp\}_{\secp \in \bbN}$ that breaks the $(\cD,\cP)$-target-collapsing of $\cH$. Dropping parameterization by $\secp$ for convenience, we can write such an adversary as a binary outcome projective measurement $(D,I-D)$ applied to a state received from the challenger. For any $h \in H_\secp, y \in \{0,1\}^n$, let $\ket{\psi_{h,y}}$ be the normalized state such that \[\ket{\psi_{h,y}} \propto \ket{h,y}\otimes\sum_{x \in \{0,1\}^m: h(x)=y}\sqrt{D(x)}\ket{x},\] and let \[\Pi_{0,h} \coloneqq \sum_{x \in \{0,1\}^m : P[h](x) = 0}\dyad{x}{x}, ~~ \Pi_{1,h} \coloneqq \sum_{x \in \{0,1\}^m : P[h](x) = 1}\dyad{x}{x}.\]

Then, the adversary's advantage in the $(\cD,\cP)$-target-collapsing game can be written as 

\[\E_{h,y}\left[\|D\ket{\psi_{h,y}}\|^2 - \left(\|D\Pi_{0,h}\ket{\psi_{h,y}}\|^2 + \|D\Pi_{1,h}\ket{\psi_{h,y}}\|^2
\right)\right] = \nonnegl(\secp),\] where the expectation is over the sampling of $h \gets H_\secp$ and the challenger's measurement of $y$.

Thus, by \cref{fact:distinguish-map}, it follows that

\[\E_{h,y}\left[\|\Pi_{1,h}D\Pi_{0,h}\ket{\psi_{h,y}}\|^2 + \|\Pi_{0,h}D\Pi_{1,h}\ket{\psi_{h,y}}\|^2\right] = \nonnegl(\secp).\]

This completes the proof, as this expression exactly corresponds to the adversary's probability of winning the $(\cD,\cP)$-target-collision-resistance game by applying $D$ and then measuring in the computational basis.

\end{proof}

%Combined, these theorems directly imply the following.

%\begin{corollary}
%Let $\cH = \{H_\secp\}_{\secp \in \bbN}$ be a hash function family that is $(\cD,\cP)$-target-collision-resistant, for some distribution $\cD$ and family of predicates $\cP$. Then, $\cH$ is certified everlasting $(\cD,\cP)$-target-collapsing.
%\end{corollary}
