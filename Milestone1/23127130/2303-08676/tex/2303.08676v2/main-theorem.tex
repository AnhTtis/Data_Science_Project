\section{Main Theorem: Certified Everlasting Target-Collapsing}

%The notion of a \emph{collapsing} hash function was introduced by Unruh~\cite{cryptoeprint:2015/361} as a quantum strengthening of collision resistance. In the classical setting, 

%Collapsing can be considered a quantum analogue of classical collision-resistance. In the classical setting, a weaker security notion has also been studied, called \emph{target}-collision-resistance \cite{10.1145/73007.73011}, which requires that for any \emph{fixed} input $x$, no polynomial-time adversary can find a collision $x' \neq x$ such that $h(x) = h(x')$. We now formalize this definition, and then introduce a quantum analogue which we call \emph{target-collapsing}.

\subsection{Definitions}

In this section, we present our definitions of target-collapsing and (generalized) target-collision-resistance. We parameterize our definitions by a distribution $\cD$ over preimages and a measurement function $\cM$. Note that when $\cM$ is the identity function, the notion of $(\cD,\cM)$-target-collapsing corresponds to a notion where the entire preimage register is measured in the computational basis. In this case we drop parameterization by $\cM$ and just say $\cD$-target-collapsing. Also, when $\cD$ is the uniform distribution, we drop parameterization by $\cD$ and just say $\cM$-target-collapsing.

\begin{definition}[$(\cD,\cM)$-Target-Collapsing Hash Function]\label{def:target-collapsing}
Let $\lambda \in \N$ be the security parameter.
A hash function family given by $\cH = \{H_\secp : \{0,1\}^{m(\secp)} \to \{0,1\}^{n(\secp)}\}_{\secp \in \bbN}$ is $(\cD,\cM)$-target-collapsing for some distribution $\cD = \{D_\secp\}_{\secp \in \bbN}$ over $\{\{0,1\}^{m(\secp)}\}_{\secp \in \bbN}$ and family of functions $\cM = \{\{M[h] : \{0,1\}^{m(\secp)} \to \{0,1\}^{k(\secp)}\}_{h \in H_\secp}\}_{\secp \in \bbN}$ if, for every QPT adversary $\cA = \{\cA_\secp\}_{\secp \in \bbN}$,
$$
|
\Pr[ \mathsf{TargetCollapseExp}_{\algo H,\algo A,\algo D,\algo M,\lambda}(0)=1] - \Pr[ \mathsf{TargetCollapseExp}_{\algo H,\algo A,\algo D,\algo M,\lambda}(1)=1]
| \leq \negl(\lambda).
$$
Here, the experiment $\mathsf{TargetCollapseExp}_{\algo H,\algo A,\algo D,\algo M,\lambda}(b)$ is defined as follows:

\begin{enumerate}

    \item The challenger prepares the state \[\sum_{x \in \{0,1\}^{m(\secp)}}\sqrt{D_\secp(x)}\ket{x}\] on register $X$, and samples a random hash function $h \rand H_\lambda$. Then, it coherently computes $h$ on $X$ (into a fresh $n(\secp)$-qubit register $Y$) and measures system $Y$ in the computational basis, which results in an outcome $y \in \bit^{n(\lambda)}$.
    \item If $b=0$, the challenger does nothing. Else, if $b=1$, the challenger coherently computes $M[h]$ on $X$ (into a fresh $k(\secp)$-qubit register $V$) and measures system $V$ in the computational basis. Finally, the challenger sends the outcome state in system $X$ to $\algo A_\secp$, together with the string $y \in \bit^{n(\lambda)}$ and a description of the hash function $h$.
    \item $\algo A_\secp$ returns a bit $b'$, which we define as the output of the experiment.
\end{enumerate}
\end{definition}

We also define an analogous notion of $(\cD,\cM)$-target-collision-resistance, as follows. Similarly to above, we drop the parameterization by $\cM$ in the case that it is the identity function, and we drop the parameterization by $\cD$ in the case that it is the uniform distribution. Notice that target-collision-resistance (without parameterization) then coincides with the classical notion where a uniformly random input is sampled, and the adversary must find a collision with respect to this input (this is also sometimes called second-preimage resistance, or weak collision-resistance).

%\dakshita{check that this does not contradict classical definitions}\james{how does the above sound?}

\begin{definition}[$(\cD,\cM)$-Target-Collision-Resistant Hash Function]\label{def:target-CR}
A hash function family $\cH = \{H_\secp : \{0,1\}^{m(\secp)} \to \{0,1\}^{n(\secp)}\}_{\secp \in \bbN}$ is $(\cD,\cM)$-target-collision-resistant for some distribution $\cD = \{D_\secp\}_{\secp \in \bbN}$ over $\{\{0,1\}^{m(\secp)}\}_{\secp \in \bbN}$ and family of functions $\cM = \{\{M[h] : \{0,1\}^{m(\secp)} \to \{0,1\}^{k(\secp)}\}_{h \in H_\secp}\}_{\secp \in \bbN}$ if, for every QPT adversary $\cA = \{\cA_\secp\}_{\secp \in \bbN}$,
$$
|
\Pr[ \mathsf{TargetCollRes}_{\algo H,\algo A,\algo D,\algo M,\lambda}=1]| \leq \negl(\lambda).
$$
Here, the experiment $\mathsf{TargetCollRes}_{\algo H,\algo A,\algo D,\algo M,\lambda}$ is defined as follows:
\begin{enumerate}
    \item The challenger prepares the state \[\sum_{x \in \{0,1\}^{m(\secp)}}\sqrt{D_\secp(x)}\ket{x}\] on register $X$, and samples a random hash function $h \rand H_\lambda$. Next, it coherently computes $h$ on $X$ (into a fresh $n(\secp)$-qubit system $Y$) and measures system $Y$ in the computational basis, which results in an outcome $y \in \bit^{n(\lambda)}$. Next, it coherently computes $M[h]$ on $X$ (into a fresh $k(\secp)$-qubit register $V$) and measures system $V$ in the computational basis, which results in an outcome $v$. Finally, its sends the outcome state in system $X$ to $\algo A_\secp$, together with the string $y \in \{0,1\}^{n(\secp)}$ and a description of the hash function $h$.
    \item $\algo A_\secp$ responds with a string $x \in \{0,1\}^{m(\secp)}$.
    \item The experiment outputs 1 if $h(x) = y$ and $M[h](x) \neq v$.
\end{enumerate}

\end{definition}

Finally, we define the notion of a \emph{certified everlasting} target-collapsing hash.

\begin{definition}\label{def:ev-target-collapsing}
A hash function family $\cH = \{H_\secp : \{0,1\}^{m(\secp)} \to \{0,1\}^{n(\secp)}\}_{\secp \in \bbN}$ is certified everlasting $(\cD,\cM)$-target-collapsing for some distribution $\cD = \{D_\secp\}_{\secp \in \bbN}$ over $\{\{0,1\}^{m(\secp)}\}_{\secp \in \bbN}$ and family of functions $\cM = \{\{M[h] : \{0,1\}^{m(\secp)} \to \{0,1\}^{k(\secp)}\}_{h \in H_\secp}\}_{\secp \in \bbN}$ if for every two-part adversary $\algo A = \{\algo A_{0,\secp},\algo A_{1,\secp}\}_{\secp \in \bbN}$, where $\{\algo A_{0,\secp}\}_{\secp \in \bbN}$ is QPT and $\{\algo A_{1,\secp}\}_{\secp \in \bbN}$ is unbounded, it holds that 

%adversary $\algo A = \{(\algo A_{\secp,0},\algo A_{\secp,1})\}_{\secp \in \bbN}$ consisting of a  and an unbounded algorithm $\algo A_{\secp,1}$, and any bit $b' \in \{0,1\}$, it holds that
$$
|\Pr\left[\mathsf{EvTargetCollapseExp}_{\algo H,\algo A,\algo D,\algo M,\lambda}(0) = 1 \right] - \Pr\left[\mathsf{EvTargetCollapseExp}_{\algo H,\algo A,\algo D,\algo M,\lambda}(1) = 1\right]| \leq \negl(\lambda).
$$
Here, the experiment $\mathsf{EvTargetCollapseExp}_{\algo H,\algo A,\algo D,\algo M,\lambda}(b)$ is defined as follows:

\begin{enumerate}

    \item The challenger prepares the state \[\sum_{x \in \{0,1\}^{m(\secp)}}\sqrt{D_\secp(x)}\ket{x}\] on register $X$, and samples a random hash function $h \rand H_\lambda$. Then, it coherently computes $h$ on $X$ (into a fresh $n(\secp)$-qubit system $Y$) and measures system $Y$ in the computational basis, which results in an outcome $y \in \bit^{n(\lambda)}$.
    \item If $b=0$, the challenger does nothing. Else, if $b=1$, the challenger coherently computes $M[h]$ on $X$ (into an auxiliary $k(\secp)$-qubit system $V$) and measures system $V$ in the computational basis. Finally, the challenger sends the outcome state in system $X$ to $\algo A_{0,\secp}$, together with the string $y \in \bit^{n(\lambda)}$ and a description of the hash function $h$.
    \item $\algo A_{0,\secp}$ sends a classical certificate $\pi \in \bit^{m(\lambda)}$ to the challenger and initializes $\algo A_{1,\secp}$ with its residual state.
    \item The challenger checks if $h(\pi)=y$. If true, $\algo A_{1,\secp}$ is run until it outputs a bit $b'$. Otherwise, $b' \gets \{0,1\}$ is sampled uniformly at random. The output of the experiment is $b'$.
    
    
    %the challenger sends $\top$ to $\cA_{\secp,1}$ and the game continues; else, the game ends and the output of the experiment is $\bot$.
    %\item $\algo A_{\secp,1}$ outputs a bit $b'$, which we define as the output of the experiment.
    
\end{enumerate}
\end{definition}

\subsection{Main Theorem}
\label{sec:maintheorem}
Our main theorem is the following.

\begin{theorem}\label{thm:CETC-generalization}
Let $\cH = \{H_\secp\}_{\secp \in \bbN}$ be a hash function family that is both $(\cD,\cM)$-target-collapsing and $(\cD,\cM)$-target-collision-resistant, for some distribution $\cD$ and efficiently computable family of functions $\cM$. Then, $\cH$ is certified everlasting $(\cD,\cM)$-target-collapsing.
\end{theorem}

\begin{proof}

Throughout the proof, we will leave the security parameter implicit, defining $H \coloneqq H_\secp, D \coloneqq D_\secp, m \coloneqq m(\secp), n \coloneqq n(\secp)$, $k \coloneqq k(\secp)$, $\cA_0 \coloneqq \cA_{0,\secp}$, and $\cA_1 \coloneqq \cA_{1,\secp}$. Next, we define

\[\ket{\psi}_X \coloneqq \sum_{x \in \{0,1\}^m}\sqrt{D(x)}\ket{x}.\] For $h \in H, y \in \{0,1\}^m$, we define a unit vector \[\ket{\psi_{h,y}}_X \propto  \sum_{x \in \{0,1\}^m : h(x)=y}\sqrt{D(x)}\ket{x}.\] Finally, for $h \in H, y \in \{0,1\}^m, v \in \{0,1\}^k$ we define a unit vector \[\ket{\psi_{h,y,v}}_X \propto  \sum_{x \in \{0,1\}^m : h(x)=y, M[h](x)=v}\sqrt{D(x)}\ket{x}.\]

% ~~ \ket{\psi_{h,y}'}_{X,V} \propto \sum_{x \in \{0,1\}^m : h(x)=y}\sqrt{D(x)}\ket{x}\ket{F[h](x)},

We consider the following hybrids. 

\begin{itemize}
    \item $\mathsf{Exp}_0(b)$:
    \begin{enumerate}
    %\item The adversary sends an $m(\lambda)$-qubit quantum state $\rho$ in a system $X$ to the challenger.
    \item The challenger prepares $\ket{\psi}_X$, samples a random hash function $h \rand H_\lambda$, coherently computes $h$ on $X$ into a fresh $n$-qubit register $Y$, and measures $Y$ in the computational basis to obtain $y \in \bit^{n}$ and a left-over state $\ket{\psi_{h,y}}_X$.
    \item If $b=0$, the challenger does nothing. Else, if $b=1$, the challenger computes $M[h]$ on $X$ into a fresh $k$-qubit register $V$, and measures $V$ in the computational basis. Finally, the challenger sends the left-over state in system $X$ to $\algo A_0$, together with the string $y \in \bit^{n}$ and a classical description of $h$.

    \item $\algo A_0$ sends a classical certificate $\pi \in \bit^m$ to the challenger and initializes $\algo A_1$ with its residual state.

    \item The challenger checks if $h(\pi)=y$. If true, $\algo A_1$ is run until it outputs a bit $b'$. Otherwise, $b' \gets \{0,1\}$ is sampled uniformly at random. The output of the experiment is $b'$.
    
    
    %If true, the challenger sends $\top$ to $\cA_1$ and the game continues; else, the game ends and the output of the experiment is $\bot$.

    %\item $\algo A_1$ outputs a bit $b'$, which we define as the output of the experiment.
\end{enumerate}

    \item $\mathsf{Exp}_1(b)$:
    \begin{enumerate}
    \item The challenger prepares $\ket{\psi}_X$, samples a random hash function $h \rand H_\lambda$, coherently computes $h$ on $X$ into a fresh $n$-qubit register $Y$, and measures $Y$ in the computational basis to obtain $y \in \bit^{n}$ and a left-over state $\ket{\psi_{h,y}}_X$.
    \item The challenger computes $M[h]$ on $X$ into a fresh $k$-qubit register $V$ to obtain a state
    
    \[\propto \sum_{x \in \{0,1\}^m: h(x)=y} \sqrt{D(x)}\ket{x}_X\ket{M[h](x)}_V.\]
    
    Then, the challenger samples a random string $z \rand \bit^k$, prepares a $\ket{+}$ state in system $C$, and applies a controlled-$\mathsf{Z}^{z}$ operation from $C$ to $V$, which results in a state
    
    \begin{align*}
        &\propto \sum_{c \in \{0,1\}} \ket{c}_C \otimes \sum_{x \in \{0,1\}^m: h(x)=y} \sqrt{D(x)}\ket{x}_X\mathsf{Z}^{c \cdot z}\ket{M[h](x)}_V\\ &= \sum_{c \in \{0,1\}} \ket{c}_C \otimes \sum_{x \in \{0,1\}^m: h(x)=y} \sqrt{D(x)}(-1)^{c \cdot \langle M[h](x),z\rangle}\ket{x}_X\ket{M[h](x)}_V.
    \end{align*}
    
      %\[\propto \sum_{c \in \{0,1\}}\ket{c}_C \otimes \sum_{x \in \{0,1\}^m : h(x)=y}(-1)^{c \cdot \langle F[h](x),z\rangle}\ket{x}_X.\]
    
    
    Finally, the challenger uncomputes the $V$ register by again computing $M[h]$ from $X$ to $V$, and sends system $X$ to $\algo A_0$, together with $y \in \bit^n$ and a classical description of $h$.

    \item $\algo A_0$ sends a classical certificate $\pi \in \bit^{m}$ to the challenger and initializes $\algo A_1$ with its residual state.
    \item The challenger checks if $h(\pi)=y$. Then, the challenger measures system $C$ to obtain $c' \in \{0,1\}$ and checks that $c' = b$. If both checks are true, $\algo A_1$ is run until it outputs a bit $b'$. Otherwise, $b' \gets \{0,1\}$ is sampled uniformly at random. The output of the experiment is $b'$.
    
    
    %If false, the game ends and the output of the experiment is $\bot$. Then, the challenger measures system $C$ to obtain $c' \in \{0,1\}$. If $c'\neq b$, the game ends and the output of the experiment is $\bot$. Otherwise, if both checks passed, the output of the experiment is $\rho$.
    %\item $\algo A_1$ outputs a bit $b'$, which we define as the output of the experiment.
    \end{enumerate}

\item $\mathsf{Exp}_2(b)$:
    \begin{enumerate}
    \item The challenger prepares $\ket{\psi}_X$, samples a random hash function $h \rand H_\lambda$, coherently computes $h$ on $X$ into a fresh $n$-qubit register $Y$, and measures $Y$ in the computational basis to obtain $y \in \bit^{n}$ and a left-over state $\ket{\psi_{h,y}}_X$.
    \item The challenger computes $M[h]$ on $X$ into a fresh $k$-qubit register $V$. Then, the challenger samples a random string $z \rand \bit^k$, prepares a $\ket{+}$ state in system $C$, applies a controlled-$\mathsf{Z}^{z}$ operation from $C$ to $V$, and finally uncomputes the $V$ register by again computing $M[h]$ from $X$ to $V$. Note that this results in a state
    
    \[\propto \sum_{c \in \{0,1\}}\ket{c}_C \otimes \sum_{x \in \{0,1\}^m : h(x)=y}(-1)^{c \cdot \langle M[h](x),z\rangle}\ket{x}_X.\]
    
    Finally, it sends system $X$ to $\algo A_0$, together with $y \in \bit^n$ and a classical description of $h$.

    \item $\algo A_0$ sends a classical certificate $\pi \in \bit^{m}$ and initializes $\algo A_1$ with its residual state.


    \item The challenger checks if $h(\pi)=y$. Then, the challenger applies the following projective measurement to system $C$:
    \[\Big\{\proj{\phi_\pi^{ z}},I - \proj{\phi_\pi^{ z}}\Big\} \,\quad \text{ where } \quad \ket{\phi_\pi^{ z}} \coloneqq \frac{1}{\sqrt{2}}  \left( \ket{0} +  (-1)^{\langle M[h](\pi), z \rangle } \ket{1}\right),\] and checks that the first outcome is observed. Finally, the challenger measures system $C$ to obtain $c' \in \{0,1\}$ and checks that $c'=b$. If all three checks are true, $\algo A_1$ is run until it outputs a bit $b'$. Otherwise, $b' \gets \{0,1\}$ is sampled uniformly at random. The output of the experiment is $b'$.
    
    %Finally, the challenger measures system $C$ to obtain $c' \in \{0,1\}$. If $c'\neq b$, the game ends and the output of the experiment is $\bot$. Otherwise, if both checks passed, the challenger sends $\top$ to $\cA_1$ and the game continues.
    %\item $\algo A_1$ outputs a bit $b'$, which we define as the output of the experiment.
    \end{enumerate}

Finally, we also use the following hybrid which is convenient for the sake of the proof.

\item $\mathsf{Exp}_3(b)$:
     \begin{enumerate}
     \item The challenger prepares $\ket{\psi}_X$, samples a random hash function $h \rand H_\lambda$, coherently computes $h$ on $X$ into a fresh $n$-qubit register $Y$, and measures $Y$ in the computational basis to obtain $y \in \bit^{n}$ and a left-over state $\ket{\psi_{h,y}}_X$.
     \item The challenger computes $M[h]$ on $X$ into a fresh $k$-qubit register $V$. Then, the challenger measures $V$ in the computational basis to obtain $v \in \{0,1\}^k$. Next, the challenger samples a random string $z \rand \bit^k$, prepares a $\ket{+}$ state in system $C$, applies a controlled-$\mathsf{Z}^{z}$ operation from $C$ to $V$, and finally uncomputes the $V$ register by again computing $M[h]$ from $X$ to $V$. Note that this results in the state
    
    \[\frac{1}{\sqrt{2}}\left(\ket{0}_C + (-1)^{\langle v,z\rangle}\ket{1}_C\right) \otimes \ket{\psi_{h,y,v}}_X.\]
    
    Finally, the challenger sends system $X$ to $\algo A_0$, together with $y \in \bit^n$ and a classical description of $h$.
    
    \item $\algo A_0$ sends a classical certificate $\pi \in \bit^{m}$ to the challenger and initializes $\algo A_1$ with its residual state.

    \item The challenger checks if $h(\pi)=y$. Then, the challenger applies the following projective measurement to system $C$:
    \[\Big\{\proj{\phi_\pi^{ z}},I - \proj{\phi_\pi^{ z}}\Big\} \,\quad \text{ where } \quad \ket{\phi_\pi^{ z}} \coloneqq \frac{1}{\sqrt{2}}  \left( \ket{0} +  (-1)^{\langle M[h](\pi), z \rangle } \ket{1}\right),\] and checks that the first outcome is observed. Finally, the challenger measures system $C$ to obtain $c' \in \{0,1\}$ and checks that $c'=b$. If all three checks are true, $\algo A_1$ is run until it outputs a bit $b'$. Otherwise, $b' \gets \{0,1\}$ is sampled uniformly at random. The output of the experiment is $b'$.
    \end{enumerate}
\end{itemize}


Before we analyze the probability of distinguishing between the consecutive hybrids, we first show that the following statements hold for the final experiment $\mathsf{Exp}_3(b)$.

\begin{claim}\label{claim:identical-certificate}
The probability that the challenger accepts the deletion certificate $\pi$ in Step 4 of $\mathsf{Exp}_3(b)$ and $M[h](\pi) \neq v$ is negligible. That is,


\[\Pr_{h,y,v} \left[
 h(\pi) = y
\,\,\, \wedge \,\,\,
M[h](\pi) \,\neq\, v
 \,\, : \,\, \pi \gets \algo A_0(h,y,\ket{\psi_{h,y,v}})\right] \leq \negl(\lambda),\] where the probability is over the challenger preparing $\ket{\psi}$, sampling $h$, and measuring $y$ and $v$ as described in $\Exp_3(b)$ to produce the left-over state $\ket{\psi_{h,y,v}}$.

%where $y = h(x_0) \in \bit^{n}$ is the image and where $\sigma_X^{z}$ is the reduced state %\james{Isn't the reduced state just $\ket{\vec x_0}$? If so, we wouldn't have to sample $\vec z$ in the above game, and we can just give $\cA_0$ the vector $\vec x_0$ (which directly corresponds to targeted collision-resistance)} 
%with respect to
%$$
%\sigma_{CX} = \frac{1}{2} \sum_{c,c' \in \bit} \ketbra{c}{c'}_C \otimes \mathsf{Z}^{c \cdot z}{\proj{x_0}}_X \left(\mathsf{Z}^{c' \cdot z}\right)^\dag.
%$$
\end{claim}


\begin{proof}
This follows directly from the assumed $(\cD,\cM)$-target-collision resistance of $\cH$, since the above probability is exactly $\Pr[\mathsf{TargetCollRes}_{\algo H,\algo A,\algo D,\algo M,\lambda}=1]$.
\end{proof}

%\begin{proof}
%Suppose for the sake of contraction that the probability is at least $1/\poly(\lambda)$. We now show that we can use $\algo A_0$ to break the target-collision-resistance of the hash family $\algo H = \{H_\lambda\}_{\lambda \in \N}$. 

%Our reduction proceeds as follows:
%\begin{enumerate}
 %\item Run $\algo A_0$ to obtain an $m(\lambda)$-qubit quantum state $\rho$ in system $X$.

%\item Measure system $X$ and send the outcome $x_0$ to the challenger.
    
%\item Once the challenger replies with a description of a hash function $h \rand H_\lambda$, send the register $\ket{x_0}$, the image $y = h(x_0)$ as well as a description of $h$ to $\algo A_0$.

%\item When $\algo A_0$ outputs $(\pi, \rho_{\aux})$, discard $\rho_{\aux}$ and output $(x_0,\pi)$.
%\end{enumerate}
%Notice that the state $\ket{x_0}$ which is sent to $\algo A_0$ is identical to the reduced state $\sigma_X$ with respect to $\sigma_{CX}$.
%By assumption, $\algo A_0$ outputs a valid certificate $\pi \neq x_0$ such that $h(\pi) = h(x_0)$ with probability at least $1/\poly(\lambda)$. Thus, we have broken the target-collision-resistance of $\algo H$.
%\end{proof}

\begin{claim}\label{claim:measurement-succeeds-wp-1}
The probability that the challenger accepts the deletion certificate $\pi$ in Step $4$ of $\mathsf{Exp}_3(b)$ and
the subsequent projective measurement on system $C$ fails (returns the second outcome) is negligible.
\end{claim}
\begin{proof}
This follows directly from \Cref{claim:identical-certificate}, which implies that except with negligible probability, the register $C$ is in the state
\[\frac{1}{\sqrt{2}}\left(\ket{0} + (-1)^{\langle v,z\rangle}\ket{1}\right)\] at the time the challenger applies the projective measurement.

%which implies that in the case the certificate $\pi$ returned by the adversary in
%$\mathsf{Exp}_3$ is identical to the pre-image produced by the challenger with all but negligible probability.
%Therefore, the projective measurement must also succeed with overwhelming probability.
\end{proof}

For any experiment $\Exp_i(b)$, we define the advantage \[\mathsf{Adv}(\mathsf{\Exp}_i) \coloneqq |\Pr\left[\Exp_i(0) = 1\right]-\Pr\left[\Exp_i(1)=1\right]|.\]

\begin{claim}
$$
\mathsf{Adv}(\mathsf{Exp}_2) = 0.
$$
\end{claim}
\begin{proof}
First note that in the case that the challenger rejects because either the deletion certificate is invalid or their projection fails, the experiment does not involve $b$, and thus the advantage of the adversary is $0$. Second, in the case that the challenger's projection succeeds, the register $C$ is either in the state
$$
\frac{1}{\sqrt{2}}  ( \ket{0} +  (-1)^{\langle \pi, z \rangle } \ket{1}) \quad\,\, \text{ or } \quad\,\,  \frac{1}{\sqrt{2}}  ( \ket{0} -  (-1)^{\langle \pi, z \rangle } \ket{1}) 
$$
for some $z \in \bit^k$, and thereby completely
unentangled from the rest of the system. Notice that the challenger's measurement of system $C$ with outcome $c'$ results in a uniformly random bit, which completely masks $b$. Therefore, the experiment is also independent of $b$ in this case, and thus the adversary's overall advantage in $\mathsf{Exp}_2$ is $0$. 
\end{proof}
Next, we argue the following.

\begin{claim}
$$
|\mathsf{Adv}(\mathsf{Exp}_2) - \mathsf{Adv}(\mathsf{Exp}_1) | \,\leq \, \negl(\lambda).
$$    
\end{claim}
\begin{proof}
Recall that \Cref{claim:measurement-succeeds-wp-1} shows that the projective measurement performed by the challenger in Step $4$ of $\mathsf{Exp}_3$ succeeds with overwhelming probability. We now argue that the same is also true in $\mathsf{Exp}_2$. Suppose for the sake of contradiction that there is a non-negligible difference between the success probabilities of the measurement. 
We now show that this implies the existence of an efficient distinguisher $\algo A'$ that breaks the $(\cD,\cM)$-target-collapsing property of the hash family $\algo H = \{H_\lambda\}_{\lambda \in \N}$. 

$\algo A'$ receives $(y,h)$ and a state on register $X$ from its challenger. Next, it computes $M[h]$ on $X$ into a fresh $k$-qubit register $V$, samples a random string $z \rand \{0,1\}^k$, prepares a $\ket{+}$ state in system $C$, applies a controlled-$\mathsf{Z}^z$ operation from $C$ to $V$, and then uncomputes register $V$ by again applying $M[h]$ from $X$ to $V$. Then, it runs $\algo A$ on $(y,h,X)$, which outputs a certificate $\pi$. 



%Our reduction proceeds as follows:
%The distinguisher $\algo A'$ runs $\algo A_0$ to obtain an $m$-qubit state $\rho$ in system $X$,
%and forwards it to the challenger who responds
%with a description of a hash function $h \rand H_\lambda$, an image $y \in \bit^m$ and a state $\rho_y$ in system $X$
%which is either a partially measured state (consisting of a superposition of pre-images)
%or a single measured pre-image $\proj{x_0}$ such that $x_0 \in \bit^m$ and $h(x_0)=y$.
%Next, $\algo D$ samples a random string $z \rand \bit^m$ and runs $\algo A_0$ given as input system $X$ of the state
%$$
%\sigma_{CX} = \frac{1}{2} \sum_{c,c' \in \bit} \ketbra{c}{c'}_C \otimes \mathsf{Z}^{c \cdot z}{\rho_y}_X \left(\mathsf{Z}^{c' \cdot z}\right)^\dag.
%$$

Finally, $\algo A'$ applies the following projective measurement to system $C$:
$$
\Big\{\proj{\phi_\pi^{z}},I - \proj{\phi_\pi^{z}}\Big\} \,\quad \text{ where } \quad
\ket{\phi_\pi^{z}} \coloneqq \frac{1}{\sqrt{2}}  \left( \ket{0} +  (-1)^{\langle \pi, z \rangle } \ket{1}\right),
$$
and outputs $1$ if the measurement succeeds and $0$ otherwise.
If there is a non-negligible difference in success probabilities of this measurement between $\Exp_3(b)$ and $\Exp_2(b)$ (for any $b \in \{0,1\}$), then $\cA'$ breaks $(\cD,\cM)$-target-collapsing of $\cH$.


%between the case when $\rho_y$ is a superposition of pre-images, or $\rho_y = \proj{x_0}$ is a single pre-image of $y$, this immediately breaks the targeted-collapsing property of $\algo H$. Therefore, the projective measurement in Step $5$ of $\mathsf{Exp}_2$ must also succeed with overwhelming probability.

Now, recall that $\mathsf{Exp}_2(b)$ is identical to $\mathsf{Exp}_1(b)$, except that the challenger applies an additional a measurement in Step 4. Because the measurement succeeds with overwhelming probability, it follows from Gentle Measurement that the advantage of the adversary must remain the same up to a negligible amount. This proves the claim.
\end{proof}

%Suppose that $\algo A = (\algo A_0, \algo A_1)$ wins at $\mathsf{Exp}_0$ with probability $\epsilon >0$. We now show the following:

%\james{need to update this claim still}

\begin{claim}\label{claim:exp_0-exp_1}
\[\mathsf{Adv}(\Exp_1) = \mathsf{Adv}(\Exp_0)/2.\]   
\end{claim}
\begin{proof}


First note that in $\Exp_1(b)$, we can imagine measuring register $C$ to obtain $c'$ and aborting if $c' \neq b$ before the challenger sends any information to the adversary. This follows because register $C$ is disjoint from the adversary's registers. Next, by \Cref{lem:random-Z}, we have the following guarantees about the state on system $X$ given to the adversary in $\Exp_1(b)$. 

\begin{itemize}
    \item In the case $c' = b = 0$, the reduced state on register $X$ is $\ket{\psi_{h,y}}$.
    \item In the case that $c' = b = 1$, the reduced state on register $X$ is a mixture over $\ket{\psi_{h,y,v}}$ where $v$ is the result of measuring register $V$ in the computational basis.
\end{itemize}

Thus, this experiment is identical to $\Exp_0(b)$, except that we decide to abort and output a uniformly random bit $b'$ with probability 1/2 at the beginning of the experiment.

\end{proof}


%in the case $c' = b = 0$, the reduced state on register $X$ is $\ket{\psi_{}}$


%Note that the reduced state on register $X$ of \[\sum_{c \in \{0,1\}} \ket{c}_C \otimes \sum_{x \in \{0,1\}^m: h(x)=y} \sqrt{D(x)}\ket{x}_X\mathsf{Z}^{c \cdot z}\ket{M[h](x)}_V\] in $\Exp_1(b)$ is an equal mixture of a state where $V$ was unmeasured and a state where $V$ was measured in the computational basis, which follows from \Cref{lem:random-Z}.

%Notice that the reduced state $\sigma_X$ in $\mathsf{Exp}_1$ is an equal mixture
%$$
%\sigma_X = \frac{1}{2} {\rho_y} + \frac{1}{2} \,  \underset{z}{\mathbb{E}} \left[\mathsf{Z}^{z}{\rho_y} \left(\mathsf{Z}^{ z}\right)^\dag \right].
%$$
%From \Cref{lem:random-Z} it follows that a random Pauli-$Z$ twirl induces a measurement in the computational basis. In other words, on average over the choice of $z \in \bit^m$, we have
%\begin{align*}
%\underset{z}{\mathbb{E}} \left[\mathsf{Z}^{z}{\rho_y} \left(\mathsf{Z}^{ z}\right)^\dag \right] & = \sum_{x \in \bit^m} \Tr[\ketbra{x}{x} \rho_y] \,\ketbra{x}{x}
%\end{align*}
%Therefore, the reduced state $\sigma_X$ in $\mathsf{Exp}_1$ precisely matches the state in $\mathsf{Exp}_0$:
%with probability $1/2$, $\rho_y$ corresponds to a superposition of pre-images of $y$ and, with probability $1/2$, $\rho_y$ is equal to $\proj{x_0}$, i.e. it corresponds to a single (measured) pre-image $x_0$ such that $h(x_0) = y$.

%Then, we observe that $c'$ in $\mathsf{Exp}_1(b)$ equals $b$ with probability $\frac{1}{2}$. Letting $\Pr[\mathsf{Exp}_0(b)=1] = \epsilon$, we get
%\begin{align*}
%\big|\mathsf{Adv}(\mathsf{Exp}_1) - \mathsf{Adv}(\mathsf{Exp}_0) \big|
%=\big| \frac{\epsilon}{2} - \epsilon \big| = \frac{\epsilon}{2}. 
%\end{align*}
%\end{proof}
%Putting everything together, we get that %$\frac{\epsilon}{2} \leq \negl(\lambda)$, %and thus $\epsilon = %\mathsf{Adv}(\mathsf{Exp}_0) \leq %\negl(\lambda)$, which completes the proof.

Putting everything together, we have that $\mathsf{Adv}(\Exp_0) \leq \negl(\secp)$, which completes the proof.

\end{proof}

\subsection{Auxiliary Information}
\label{sec:mainaux}


Next, we generalize the above theorem statement to handle hash functions that are sampled with some auxiliary information. That is, there is an algorithm $(h,\aux) \gets \Samp(1^\secp)$ that samples the description of a hash function $h$ along with some auxiliary information $\aux$. We will want to allow the adversary to potentially see information about $\aux$ (but not necessarily all of it), so we define a family $\cZ = \{Z_\secp(\aux)\}_{\secp \in \bbN}$ that specifies what information the adversary sees about $\aux$. In the most straightforward case, $\cZ$ could be some distribution over classical or quantum states, parameterized by $\aux$. However, we also consider an \emph{interactive} $Z_\secp(\aux)$. That is, $Z_\secp$ is the description of an interactive machine that is initialized with $\aux$ and interacts with the adversary $\cA_\secp$.

%Now, we consider a variant on our definitions of target-collapsing and target-collision-resistance that will be useful for our hybrid-encryption based generic compiler in \cref{sec:regularOWF}. Instead of measuring the $Y$ register to obtain a single image $y$, the challenger will ``coherently'' measure $Y$ by copying it into a new register $Y'$. Then, it will \emph{encrypt} the register $Y'$ using some encryption scheme $\cZ$, and send this encrypted register to the adversary as part of its challenge. 

%We will consider families of machines $\cZ = \{Z_\secp(Y')\}_{\secp \in \bbN}$ that operate on a register $Y'$. In the most straightforward case, $Z_\secp(Y')$ will output some state on register $Y''$ that is sent to the adversary. However, we will also consider \emph{interactive} $Z_\secp(Y')$. That is, $Z_\secp$ may be the description of an interactive machine that is initialized with a state on register $Y'$, and interacts with the adversary $\cA$. Our formal definitions follow.

%\begin{definition}
%A hash function family $\cH = \{H_\secp : \{0,1\}^{m(\secp)} \to \{0,1\}^{n(\secp)}\}_{\secp \in \bbN}$ is $(\cD,\cM,\cZ)$-target-collapsing for some distribution $\cD = \{D_\secp\}_{\secp \in \bbN}$ over $\{\{0,1\}^{m(\secp)}\}_{\secp \in \bbN}$, family of functions $\cM = \{\{M[h] : \{0,1\}^{m(\secp)} \to \{0,1\}^{k(\secp)}\}_{h \in H_\secp}\}_{\secp \in \bbN}$, and family of (static or interactive) machines $\cZ = \{Z_\secp\}_{\secp \in \bbN}$ if, for every QPT adversary $\cA = \{\cA_\secp\}_{\secp \in \bbN}$,
%$$
%|
%\Pr[ \mathsf{TargetCollapseExp}_{\algo H,\algo A,\algo D,\algo M,\algo Z,\lambda}(0)=1] - \Pr[ \mathsf{TargetCollapseExp}_{\algo H,\algo A,\algo D,\algo M,\algo Z\lambda}(1)=1]
%| \leq \negl(\lambda),
%$$
%where the experiment $\mathsf{TargetCollapseExp}_{\algo H,\algo A,\algo D,\algo F,\algo Z,\lambda}(b)$ is defined as follows.

%\begin{enumerate}

    %\item The challenger prepares the state \[\sum_{x \in \{0,1\}^{m(\secp)}}\sqrt{D_\secp(x)}\ket{x}_X\] on register $X$, and samples a random hash function $h \rand H_\lambda$. Then, it coherently computes $h$ on $X$ twice into fresh $n(\secp)$-qubit registers $Y$ and $Y'$, producing the state
    
    %\[\sum_{x \in \{0,1\}^{m(\secp)}}\sqrt{D_\secp(x)}\ket{x}_X\ket{h(x)}_Y\ket{h(x)}_{Y'}.\]
    
    %and measures system $Y$ in the computational basis, which results in an outcome $y \in \bit^{n(\lambda)}$.
    %\item If $b=0$, the challenger does nothing. Else, if $b=1$, the challenger coherently computes $M[h]$ on $X$ into a fresh $k(\secp)$-qubit register $V$ and measures system $V$ in the computational basis. Then, the challenger sends the state on systems $(X,Y)$ to $\algo A$, together with a description of the hash function $h$.
    %\item The challenger initializes $Z_\secp(Y')$. In the case that $Z_\secp(Y')$ is static, the challenger additionally sends its output register $Y''$ to $\algo A$, and in the case that $\cZ_\secp(Y')$ is interactive, the adversary $\algo A$ additionally gets to interact with $\cZ_\secp(Y')$.
    %\item $\algo A$ returns a bit $b'$, which we define as the output of the experiment.
%\end{enumerate}


%in \cref{def:target-collapsing} except that $h$ is sampled by $(h,\aux) \gets \Samp(1^\secp)$, and the adversary is given (or interacts with) $Z_\secp(\aux)$ along with $(X,y,h)$.

%\end{definition}

\begin{definition}
A hash function family $\cH = \{H_\secp : \{0,1\}^{m(\secp)} \to \{0,1\}^{n(\secp)}\}_{\secp \in \bbN}$ with an associated sampling algorithm $\Samp$ is $(\cD,\cM,\cZ)$-target-collapsing for some distribution $\cD = \{D_\secp\}_{\secp \in \bbN}$ over $\{\{0,1\}^{m(\secp)}\}_{\secp \in \bbN}$, family of functions $\cM = \{\{M[h] : \{0,1\}^{m(\secp)} \to \{0,1\}^{k(\secp)}\}_{h \in H_\secp}\}_{\secp \in \bbN}$, and family of (static or interactive) distributions $\cZ = \{Z_\secp(\aux)\}_{(\cdot,\aux) \in \Samp(1^\secp), \secp \in \bbN}$ if, for every QPT adversary $\cA = \{\cA_\secp\}_{\secp \in \bbN}$,
$$
|
\Pr[ \mathsf{TargetCollapseExp}_{\algo H,\algo A,\algo D,\algo M,\algo Z,\lambda}(0)=1] - \Pr[ \mathsf{TargetCollapseExp}_{\algo H,\algo A,\algo D,\algo M,\algo Z,\lambda}(1)=1]
| \leq \negl(\lambda),
$$
where the experiment $\mathsf{TargetCollapseExp}_{\algo H,\algo A,\algo D,\algo M,\algo Z,\lambda}(b)$ is defined as in \cref{def:target-collapsing} except that $h$ is sampled by $(h,\aux) \gets \Samp(1^\secp)$, and the adversary is given (or interacts with) $Z_\secp(\aux)$ along with $(X,y,h)$.

\end{definition}

\begin{definition}
A hash function family $\cH = \{H_\secp : \{0,1\}^{m(\secp)} \to \{0,1\}^{n(\secp)}\}_{\secp \in \bbN}$ with an associated sampling algorithm $\Samp$ is $(\cD,\cM,\cZ)$-target-collision-resistant for some distribution $\cD = \{D_\secp\}_{\secp \in \bbN}$ over $\{\{0,1\}^{m(\secp)}\}_{\secp \in \bbN}$, family of functions $\cM = \{\{M[h] : \{0,1\}^{m(\secp)} \to \{0,1\}^{k(\secp)}\}_{h \in H_\secp}\}_{\secp \in \bbN}$, and family of (static or interactive) distributions $\cZ = \{Z_\secp(\aux)\}_{(\cdot,\aux) \in \Samp(1^\secp), \secp \in \bbN}$ if, for every QPT adversary $\cA = \{\cA_\secp\}_{\secp \in \bbN}$,
$$
\Pr[ \mathsf{TargetCollRes}_{\algo H,\algo A,\algo D,\algo M,\algo Z,\lambda}(0)=1] \leq \negl(\secp),
$$
where the experiment $\mathsf{TargetCollRes}_{\algo H,\algo A,\algo D,\algo M,\algo Z,\lambda}(b)$ is defined as in \cref{def:target-CR} except that $h$ is sampled by $(h,\aux) \gets \Samp(1^\secp)$, and the adversary is given (or interacts with) $Z_\secp(\aux)$ along with $(X,y,h)$.

\end{definition}

\begin{definition}
A hash function family $\cH = \{H_\secp : \{0,1\}^{m(\secp)} \to \{0,1\}^{n(\secp)}\}_{\secp \in \bbN}$ with an associated sampling algorithm $\Samp$ is certified everlasting $(\cD,\cM,\cZ)$-target-collapsing for some distribution $\cD = \{D_\secp\}_{\secp \in \bbN}$ over $\{\{0,1\}^{m(\secp)}\}_{\secp \in \bbN}$, family of functions $\cM = \{\{M[h] : \{0,1\}^{m(\secp)} \to \{0,1\}^{k(\secp)}\}_{h \in H_\secp}\}_{\secp \in \bbN}$, and family of (static or interactive) distributions $\cZ = \{Z_\secp(\aux)\}_{(\cdot,\aux) \in \Samp(1^\secp), \secp \in \bbN}$ if, for every two-part adversary $\algo A = \{\algo A_{0,\secp},\algo A_{1,\secp}\}_{\secp \in \bbN}$, where $\{\algo A_{0,\secp}\}_{\secp \in \bbN}$ is QPT and $\{\algo A_{1,\secp}\}_{\secp \in \bbN}$ is unbounded, it holds that 
$$
|\Pr[ \mathsf{EvTargetCollapseExp}_{\algo H,\algo A,\algo D,\algo M,\algo Z,\lambda}(0)=1]-\Pr[ \mathsf{EvTargetCollapseExp}_{\algo H,\algo A,\algo D,\algo M,\algo Z,\lambda}(1)=1]| \leq \negl(\secp),
$$
where the experiment $\mathsf{EvTargetCollapseExp}_{\algo H,\algo A,\algo D,\algo M,\algo Z,\lambda}(b)$ is defined as in \cref{def:target-CR} except that $h$ is sampled by $(h,\aux) \gets \Samp(1^\secp)$, and the first part of the adversary $\algo A_{0,\secp}$ is given (or interacts with) $Z_\secp(\aux)$ along with $(X,y,h)$.

\end{definition}


Now, the following generalization of \cref{thm:CETC-generalization} follows immediately from the proof of \cref{thm:CETC-generalization}, by additionally giving $Z_\secp(\aux)$ to the adversary in each of the experiments.

\begin{theorem}
Let $\cH = \{H_\secp\}_{\secp \in \bbN}$ be a hash function family that is both $(\cD,\cM,\cZ)$-target-collapsing and $(\cD,\cM,\cZ)$-target-collision-resistant, for some distribution $\cD$, efficiently computable family of functions $\cM$, and (static or interactive) distribution $\cZ$. Then, $\cH$ is certified everlasting $(\cD,\cM,\cZ)$-target-collapsing.
\end{theorem}


\subsection{Target-Collision-Resistance implies Target-Collapsing for Polynomial-Outcome Measurements}
\label{sec:tcr-implies}
%\dakshita{should we say that the two notions are equivalent? is that true?}\james{I thought we had some potential counterexamples that were (partial) target-collapsing but not (partial) target-collision-resistant. In any case, I don't know of a proof in the reverse direction.}

%\begin{lemma}\label{thm:TCR-from-TC}
%Let $\cH = \{H_\secp : \{0,1\}^{m(\secp)}\to \{0,1\}^{n(\secp)}\}_{\secp \in \bbN}$ be a hash function family that is $(\cD,\cM,\cZ)$-target-collapsing for some distribution $\cD = \{D_\secp\}_{\secp \in \bbN}$ over $\{\{0,1\}^{m(\secp)}\}_{\secp \in \bbN}$, family of \emph{single-bit output }functions $\cM = \{\{M[h] : \{0,1\}^{m(\secp)} \to \{0,1\}\}_{h \in H_\secp}\}_{\secp \in \bbN}$, and family of (static or interactive) distributions $\cZ = \{Z_\secp(\aux)\}_{(\cdot,\aux) \in \Samp(1^\secp), \secp \in \bbN}$. Then, $\cH$ is $(\cD,\cM,\cZ)$-target-collision-resistant.
%\end{lemma}

%\begin{proof}
%We will make use of the following claim.

%\begin{claim}
%Let $\Pi_0,\Pi_1$ be orthogonal projectors and let $\ket{\psi}$ be any state such that $\ket{\psi} \in \mathsf{im}(\Pi_0 + \Pi_1)$. Then for any unitary $U$, there exists a projector $D$ such that 

%\[\big| \|D\ket{\psi}\|^2 - \left(\|D\Pi_0\ket{\psi}\|^2 + \|D\Pi_1\ket{\psi}\|^2\right)\big| \geq \frac{1}{2}\left(\|\Pi_1 U \Pi_0 \ket{\psi}\|^2 + \| \Pi_0 U \Pi_1 \ket{\psi} \|^2\right).\]
%\end{claim}

%\begin{proof}
%Define $D$ to take as input register $X$, prepare a $\ket{+}$ state on register $P$, apply controlled $U$ from $P$ to $X$, and then measure $P$ in the $\{\ket{+},\ket{-}\}$ basis and accept if $\ket{+}$ is observed. First, define \[\ket{+_\psi} \coloneqq \ket{\psi} = \Pi_0\ket{\psi} + \Pi_1\ket{\psi}, ~~ \ket{-_\psi} \coloneqq \Pi_0\ket{\psi} - \Pi_1\ket{\psi},\] and note that

%\begin{align*}
    %\big|\|&D\ket{\psi}\|^2 - \left(\|D\Pi_0\ket{\psi}\|^2 + \|D\Pi_1\ket{\psi}\|^2\right)\big| \\
    %&= \big| \bra{+_\psi}D\ket{+_\psi} - \frac{1}{4}\left(\left(\bra{+_\psi} + \bra{-_\psi}\right)D\left(\ket{+_\psi} + \ket{-_\psi}\right)\right) - \frac{1}{4}\left(\left(\bra{+_\psi} - \bra{-_\psi}\right)D\left(\ket{+_\psi} - \ket{-_\psi}\right)\right)\big| \\
    %&= \frac{1}{2}\big|\bra{+_\psi}D\ket{+_\psi} - \bra{-_\psi}D\ket{-_\psi}\big|
%\end{align*}




%Then,

%\begin{align*}
    %&\bra{+_\psi}D\ket{+_\psi} - \bra{-_\psi}D\ket{-_\psi} \\
    %&= \bigg\| \dyad{+}{+}\left(\frac{1}{\sqrt{2}}\ket{0}\left(\Pi_0\ket{\psi} + \Pi_1\ket{\psi}\right) + \frac{1}{\sqrt{2}}\ket{1}\left(U\Pi_0\ket{\psi} + U\Pi_1\ket{\psi}\right)\right) \bigg\|^2 \\
    %& ~~ - \bigg\| \dyad{+}{+}\left(\frac{1}{\sqrt{2}}\ket{0}\left(\Pi_0\ket{\psi} - \Pi_1\ket{\psi}\right) + \frac{1}{\sqrt{2}}\ket{1}\left(U\Pi_0\ket{\psi} - U\Pi_1\ket{\psi}\right)\right) \bigg\|^2 \\
    %&= \frac{1}{4}\left(\left(\bra{\psi}\Pi_0 + \bra{\psi}\Pi_1\right)\left(U\Pi_0\ket{\psi} + U\Pi_1\ket{\psi}\right) + \left(\bra{\psi}\Pi_0U^\dagger + \bra{\psi}\Pi_1U^\dagger\right)\left(\Pi_0\ket{\psi} + \Pi_1\ket{\psi}\right)\right) \\
    %& ~~ - \frac{1}{4}\left(\left(\bra{\psi}\Pi_0 - \bra{\psi}\Pi_1\right)\left(U\Pi_0\ket{\psi} - U\Pi_1\ket{\psi}\right) + \left(\bra{\psi}\Pi_0U^\dagger - \bra{\psi}\Pi_1U^\dagger\right)\left(\Pi_0\ket{\psi} - \Pi_1\ket{\psi}\right)\right) \\
    %&= \frac{1}{2}\left(\bra{\psi}\Pi_0U\Pi_1\ket{\psi} + \bra{\psi}\Pi_1U\Pi_0\ket{\psi} + \bra{\psi}\Pi_0U^\dagger\Pi_1\ket{\psi} + \bra{\psi}\Pi_1U^\dagger\Pi_0\ket{\psi} \right)
%\end{align*}

%\james{stuck here}


%\end{proof}

%\end{proof}

In this section, we show that recent techniques from the collapsing hash function / collapsing commitment literature \cite{cryptoeprint:2022/786,crypto-2022-32202,crypto-2022-32124} imply that when $\cM$ is a function with polynomial number of outcomes, then $(\cD,\cM,\cZ)$-target-collision-resistance implies $(\cD,\cM,\cZ)$-target-collapsing. In this paper, we will only need to use this claim for \emph{two-outcome} measurements, but we show it for the more general case of polynomial-outcome measurements.

\begin{lemma}\label{thm:TC-from-TCR}
Let $\cH = \{H_\secp : \{0,1\}^{m(\secp)}\to \{0,1\}^{n(\secp)}\}_{\secp \in \bbN}$ be a hash function family that is $(\cD,\cM,\cZ)$-target-collision-resistant for some distribution $\cD = \{D_\secp\}_{\secp \in \bbN}$ over $\{\{0,1\}^{m(\secp)}\}_{\secp \in \bbN}$, family of functions $\cM = \{\{M[h] : \{0,1\}^{m(\secp)} \to \{0,1\}^{k(\secp)}\}_{h \in H_\secp}\}_{\secp \in \bbN}$ for $k(\secp)=O(\log \secp)$, and family of (static or interactive) distributions $\cZ = \{Z_\secp(\aux)\}_{(\cdot,\aux) \in \Samp(1^\secp), \secp \in \bbN}$. Then, $\cH$ is $(\cD,\cM,\cZ)$-target-collapsing.
\end{lemma}

\begin{proof}
We will make use of the following fact \cite[Claim 3.5]{cryptoeprint:2022/786}.

\begin{fact}\label{fact:distinguish-map}
Let $D$ be a projector, $\{\Pi_i\}_{i \in [N]}$ be pairwise orthogonal projectors, and $\ket{\psi}$ be any state such that $\ket{\psi} \in \mathsf{im}(\sum_{i \in [N]}\Pi_i)$. Then,
    \[\sum_{i \in [N]}\bigg\| \left(\sum_{j \neq i}\Pi_j\right)D\Pi_i\ket{\psi}\bigg\|^2  \geq \frac{1}{N}\left(\|D\ket{\psi}\|^2-\left(\sum_{i \in [N]}\|D\Pi_i\ket{\psi}\|^2\right)\right)^2.\]
\end{fact}

Now, suppose there exists an adversary $\{\cA_\secp\}_{\secp \in \bbN}$ that breaks the $(\cD,\cM,\cZ)$-target-collapsing of $\cH$. Dropping parameterization by $\secp$ for convenience, we can write such an adversary as a binary outcome projective measurement $(D,I-D)$ applied to a state received from the challenger. For any $h \in H_\secp, y \in \{0,1\}^n$, let $\ket{\psi_{h,y}}$ be the normalized state such that \[\ket{\psi_{h,y}} \propto \ket{h,y}\otimes\sum_{x \in \{0,1\}^m: h(x)=y}\sqrt{D(x)}\ket{x},\] and for $i \in \{0,1\}^k$, let \[\Pi_{i,h} \coloneqq \sum_{x \in \{0,1\}^m : M[h](x) = i}\dyad{x}{x}.\]

Then, the adversary's advantage in the $(\cD,\cM,\cZ)$-target-collapsing game can be written as 

\[\E_{h,y}\left[\|D\ket{\psi_{h,y}}\|^2 - \sum_{i \in \{0,1\}^k}\|D\Pi_{i,h}\ket{\psi_{h,y}}\|^2 \right] = \nonnegl(\secp),\] where the expectation is over the sampling of $h \gets H_\secp$ and the challenger's measurement of $y$.

Thus, by \cref{fact:distinguish-map}, it follows that

\[\E_{h,y}\left[\sum_{i \in \{0,1\}^k}\bigg\|\left(\sum_{j \neq i}\Pi_{j,h}\right)D\Pi_{i,h}\ket{\psi_{h,y}}\bigg\|^2\right] = \nonnegl(\secp),\] since $2^k = 2^{O(\log \secp)} = \poly(\secp)$. This completes the proof, as this expression exactly corresponds to the adversary's probability of winning the $(\cD,\cM,\cZ)$-target-collision-resistance game by applying $D$ and then measuring in the computational basis.

\end{proof}


%\ifsubmission
%\paragraph{Conclusion.}
%We refer the reader to Section \ref{sec:overreq} for an overview of how target-collapsing implies publicly-verifiable deletion.
%Due to lack of space, we defer the formal proof of this fact, and also our instantiations of target-collapsing functions from LWE/SIS following~\cite{Poremba22}, from almost-regular one-way functions as well as from pseudorandom group actions~\cite{HMY} to the appendices. 
%\else \fi

