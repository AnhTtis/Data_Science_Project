\section{OWF-based certified deletion}

Let $f = \{f_\secp : \{0,1\}^{m(\secp)} \to \{0,1\}^{n(\secp)}\}_{\secp \in \bbN}$ be a regular (\james{hopefully we can relax the regularity requirement somewhat}) one-way function and let $\cA = \{\cA_\secp\}_{\secp \in \bbN}$ be any QPT adversary. \alex{How is regularity defined here? Is it a similar definition as in Mark Zhandry's paper on collapsing hashes?}\james{For now I mean exactly regular, so each element of the image has the same number of preimages. But we can try to relax to either Mark's definition of semi-regularity, or perhaps almost-regularity (the relative size of each set of preimages is within a fixed polynomial ratio).}\james{An additional property we may need is that the range of the OWF (as a subset of the codomain) is somehow efficiently describable, so we can restrict the linear map to be only applied to elements in the range. Otherwise, it may be the case that for many pairs of vectors in the domain of the linear map, only 0 or 1 of them are valid images of the OWF. But maybe this can be generically obtained by compressing the image with a universal hash function.  }

Consider the following experiment $\Exp_{f,\cA,\secp}(b)$.

\begin{enumerate}
    \item The challenger samples a uniformly random full rank linear map $h: \bbF_2^{n} \to \bbF_2^{n-1}$ and prepares the state
    \[\frac{1}{\sqrt{2^m}}\sum_{x \in \{0,1\}^m}(-1)^{b \cdot p_{f,h}(x)}\ket{x}_X\ket{h(f(x))}_Z,\]
    where $p_{f,h}(x) = 0$ iff, letting $y \coloneqq f(x) \in \{0,1\}^n, z \coloneqq h(y) \in \{0,1\}^{n-1}$, it holds that $y$ is the lexicographially first string such that $h(y) = z$. The challenger then measures the $Z$ register in the standard basis to obtain $z$, and the remaining state collapses to 
    \[\frac{1}{\sqrt{2^{m-n+1}}}\left(\sum_{x : f(x) = y_{z,0}}\ket{x}_X + (-1)^b\sum_{x : f(x) = y_{z,1}}\ket{x}_X\right),\] where $y_{z,0},y_{z,1} \in \{0,1\}^n$ are the two strings such that $h(y_{z,0}) = h(y_{z,1}) = z$, ordered lexicographically.
    \item The challenger sends $h,z$, and the $X$ register to $\cA_\secp$, who outputs a classical deletion certificate $\pi$ and a left-over quantum state $\rho$.
    \item If $h(f(\pi)) = z$, output $\rho$, and otherwise output $\bot$.
\end{enumerate}

\begin{theorem}
\[\TD\left(\Exp_{f,\cA,\secp}(0),\Exp_{f,\cA,\secp}(1)\right) = \negl(\secp).\]
\end{theorem}

\begin{proof}
\james{Sketch} The proof will basically follow the same strategy used above in the collapsing-based certified deletion game.
\begin{enumerate}
	\item Hybrid 1: Entangle the phase bit with a challenger's register $C$ and measure this register at the end of experiment to obtain $b$.
	\item Hybrid 2: If the adversary outputs a valid deletion certificate $\pi$, project register $C$ onto $\frac{1}{\sqrt{2}}\left(\ket{0} + (-1)^{p_{f,h}(\pi)}\ket{1}\right).$ In this hybrid, $b$ is then a uniformly random bit.
\end{enumerate}
To show indistinguishability between Hybrid 1 and 2, we need to define an experiment where the bit $p_{f,h}(x)$ is measured on register $X$ before the state is given to the adversary. Let \[\ket{0_{f,h,z}} \coloneqq \frac{1}{\sqrt{2^{m-n}}}\sum_{x:f(x)=y_{z,0}}\ket{x}, ~~ \ket{1_{f,h,z}} \coloneqq \frac{1}{\sqrt{2^{m-n}}}\sum_{x:f(x)=y_{z,1}}\ket{x},\] and $\ket{+_{f,h,z}} \coloneqq \frac{1}{\sqrt{2}}(\ket{0_{f,h,z}} + \ket{1_{f,h,z}}), \ket{-_{f,h,z}} \coloneqq \frac{1}{\sqrt{2}}(\ket{0_{f,h,z}} - \ket{1_{f,h,z}})$. Then, measuring this bit on register $X$ collapses the state as follows: \[\frac{1}{\sqrt{2}}\ket{0}_C\ket{+_{f,h,z}}_X + \frac{1}{\sqrt{2}}\ket{1}_C\ket{-_{f,h,z}}_X \to \frac{1}{2}\ket{+}\bra{+}_C \otimes \ket{0_{f,h,z}}\bra{0_{f,h,z}}_X + \frac{1}{2}\ket{-}\bra{-}_C \otimes \ket{1_{f,h,z}}\bra{1_{f,h,z}}_X\] 
	
	
By [AAS, HMY] and the fact that all $\ket{+_{f,h,z}}$ and $\ket{-_{f,h,z}}$ (for all $h,z$) are orthogonal states, which follows from the regularity of $f$, any $\cA$ that can distinguish this switch can be used to map the state $\ket{0_{f,h,z}} \to \ket{1_{f,h,z}}$ (on average over $h,z$). This is impossible due to the one-wayness of $f$ and the fact that $f$ is regular (which allows the reduction to go through). Moreover, in this experiment, there is a negligible probability that the adversary outputs a valid $\pi$ but projecting register $C$ onto $\frac{1}{\sqrt{2}}\left(\ket{0} + (-1)^{p_{f,h}(\pi)}\ket{1}\right)$ fails, which again follows from the one-wayness and regularity of $f$. Thus, adding this projection to Hybrid 2 must only have a negligible affect on the experiment.

\end{proof}

\paragraph{Applications}
\begin{itemize}
    \item By purifying, this theorem can be used to obtain bit commitments with publicly-verifiable certified deletion. This will imply ZK proofs with publicly-verifiable everlasting zero-knowledge, and secure computation with publicly-verifiable EST.
    \item If $f$ admits a trapdoor that allows to prepare the uniform superposition over preimages for any image (a special case of this is the standard notion of an injective trapdoor function), then this implies public-key encryption with publicly-verifiable certified deletion.
\end{itemize}