\documentclass[11pt,pdfa]{article}
\usepackage[in]{fullpage}
\usepackage{palatino}

% Unicode compatibility
%\usepackage{iftex}
%\ifPDFTeX
%  \usepackage[utf8]{inputenc}
%  \usepackage[noTeX]{mmap}
%  \usepackage[T1]{fontenc}
%\fi
% XeTeX does not support mmap
%\ifLuaTeX
%  \usepackage{luatex85}
%  \usepackage[noTeX]{mmap}
%\fi

\usepackage[style=alphabetic,minalphanames=3,maxalphanames=4,maxnames=99,backref=true]{biblatex}
  \renewcommand*{\labelalphaothers}{\textsuperscript{+}}
  \renewcommand*{\multicitedelim}{\addcomma\space}
  \addbibresource{crypto.bib}\addbibresource{custom.bib}

\usepackage[normalem]{ulem}
\usepackage{comment}
\usepackage[shortlabels]{enumitem}
\usepackage{bm}
\usepackage{amsfonts}
\usepackage{xspace}
\usepackage{amsmath}
\usepackage{amssymb}
\usepackage{amsthm}
\usepackage{xcolor}
\usepackage{graphicx}
\usepackage{qtree}
\usepackage{tree-dvips}
\usepackage{float}
\usepackage{hyperref}
\usepackage{physics}
\usepackage{breakurl}
\usepackage{braket}
\usepackage{mathtools}
\usepackage{mathrsfs}
\usepackage{tikz}
\usepackage[linesnumbered,ruled,vlined]{algorithm2e}
\usepackage{qcircuit}
\usepackage[nameinlink,capitalize]{cleveref}

\newcommand{\fullversion}[1]{#1}
\newcommand{\subversion}[1]{}

  \hypersetup{colorlinks={true},linkcolor={blue},citecolor=magenta}

  \theoremstyle{plain}
  \newtheorem{theorem}{Theorem}[section]
  \newtheorem{lemma}[theorem]{Lemma}
  %\newtheorem{claim}{Claim}
  %\Crefname{claim}{Claim}{Claims}
  %\newtheorem*{lemma*}{Lemma}
  \newtheorem{corollary}[theorem]{Corollary}
  %\newtheorem{proposition}{Proposition}
  %\newtheorem{observation}{Observation}
  \newtheorem{definition}[theorem]{Definition}
  \newtheorem{remark}[theorem]{Remark}
  \newtheorem{claim}[theorem]{Claim}
  \newtheorem{importedtheorem}[theorem]{Imported Theorem}
 \newtheorem{construction}{Construction}
 \newtheorem{conjecture}{Conjecture}
  
  %\usepackage[numbers,sort&compress]{natbib}
  %\def\bibfont{\footnotesize}
  %\def\bibsep{\smallskipamount}
  %\def\bibhang{24pt}
  
  %\bibliographystyle{alpha}
  
  \usepackage[style=alphabetic,minalphanames=3,maxalphanames=4,maxnames=99,backref=true]{biblatex}
  \renewcommand*{\labelalphaothers}{\textsuperscript{+}}
  \renewcommand*{\multicitedelim}{\addcomma\space}
  
  \newcommand{\email}[1]{\href{mailto:#1}{\texttt{#1}}}
\newtheorem{fact}[theorem]{Fact}
\newcommand{\samptd}{\mathsf{SampTD}}
\newcommand{\GenTrap}{\mathsf{GenTrap}}

\newcommand{\algo}{\mathcal}
\newcommand{\Zq}{\mathbb{Z}_q}
\newcommand{\mxA}{\mathbf{A}}
\newcommand{\bfA}{\mxA}
\newcommand{\mxtd}{\mathbf{T}}
%\newcommand{\Tr}{\mathrm{Tr}}
\newcommand{\bfs}{{\bf s}}
\newcommand{\sk}{\mathsf{sk}\xspace}
\newcommand{\pk}{\mathsf{pk}\xspace}
\newcommand{\mpk}{\mathsf{mpk}\xspace}
\newcommand{\msk}{\mathsf{msk}\xspace}
\newcommand{\vk}{\mathsf{vk}\xspace}
\newcommand{\ct}{\mathsf{CT}\xspace}
\renewcommand{\vec}[1]{\mathbf{#1}}
\newcommand{\setup}{\mathsf{KeyGen}}
\newcommand{\bit}{\{0,1\}}
\newcommand{\Eval}{\mathsf{Eval}}
\newcommand{\SIS}{\mathsf{SIS}}
\newcommand{\ISIS}{\mathsf{ISIS}}
\newcommand{\LWE}{\mathsf{LWE}}
\newcommand{\enc}{\mathsf{Enc}}
\newcommand{\PKE}{\ensuremath{\mathsf{PKE}}\xspace}
\newcommand{\Revoke}{\mathsf{Revoke}}
\DeclareMathAlphabet\mathbfcal{OMS}{cmsy}{b}{n}
\newcommand{\FHE}{\ensuremath{\mathsf{FHE}}\xspace}
\newcommand{\proj}[1]{\ensuremath{|#1\rangle \langle #1|}}
\newcommand{\regR}{\mathbf{R}}
\newcommand{\regX}{\mathbf{X}}
\newcommand{\extractor}{{\cal E}}
\newcommand{\setS}{{\cal S}}

\newcommand{\keygen}{\mathsf{KeyGen}}
\newcommand{\KeyGen}{\mathsf{KeyGen}}
\newcommand{\Del}{\mathsf{Del}}
\newcommand{\Vrfy}{\mathsf{Vrfy}}
\newcommand{\PPT}{\mathsf{PPT}}
\newcommand{\POVM}{\mathsf{POVM}}
\newcommand{\Invert}{\mathsf{Invert}}
\newcommand{\QPT}{\mathsf{QPT}}
\newcommand{\CPTP}{\mathsf{CPTP}}
\newcommand{\aux}{\mathsf{aux}}
\newcommand{\FT}{\mathsf{FT}}
\newcommand{\dec}{\mathsf{Dec}}
\newcommand{\Mod}[1]{\ (\mathrm{mod}\ #1)}
\newcommand{\rand}{\raisebox{-1pt}{\ensuremath{\,\xleftarrow{\raisebox{-1pt}{$\scriptscriptstyle\$$}}\,}}}
\newcommand{\expt}{\mathsf{Expt}}
\newcommand{\ch}{\mathsf{Ch}}

\newcommand{\R}{\mathbb{R}}
\newcommand{\N}{\mathbb{N}}
\newcommand{\NN}{\mathbb{N}}
\newcommand{\C}{\mathbb{C}}
\newcommand{\Z}{\mathbb{Z}}
\newcommand{\valid}{\mathsf{Valid}}
\newcommand{\invalid}{\mathsf{Invalid}}
%\newcommand{\bfX}{\mathbf{X}}
%\newcommand{\tr}{\mathsf{Tr}}
%\newcommand{\ketbra}[2]{\left|#1\right\rangle\!\!\left\langle #2\right|}
\newcommand{\negl}{\mathsf{negl}}
\newcommand{\bfP}{\mathbf{P}}
\newcommand{\bfv}{\mathbf{v}}
\newcommand{\bfR}{\mathbf{R}}
\newcommand{\bfB}{\mathbf{B}}

\newcommand{\bfT}{\mathbf{T}}
\newcommand{\bfx}{\mathbf{x}}
\newcommand{\bft}{\mathbf{t}}
\newcommand{\bfG}{\mathbf{G}}
\newcommand{\distr}{\mathcal{D}}
% Bold vectors instead of arrow vectors

\newcommand{\alex}[1]{{\color{blue} Alex: #1 }}
\newcommand{\james}[1]{{\color{red} James: #1 }}
\newcommand{\dakshita}[1]{{\color{orange} Dakshita: #1 }}

\title{Publicly-Verifiable Deletion via Target-Collapsing Functions}
%or\\
%Cryptography with Publicly-Verifiable Deletion\\
%or\\
%\title{
%Target-Collapsing Hashes and Publicly-Verifiable Deletion}
\author{James Bartusek\footnote{bartusek.james@gmail.com}\\UC Berkeley\and 
Dakshita Khurana\footnote{dakshita@illinois.edu}\\UIUC \and 
Alexander Poremba\footnote{{aporemba@caltech.edu}}\\Caltech
}
\date{}


\newcommand{\poly}{\mathrm{poly}}

%\input{headers}
\definecolor{purple}{rgb}{1, 0, 1}

\newcommand{\ie}{\emph{i.e.,}\xspace}
\newcommand{\eg}{\emph{e.g.,}\xspace}
\newcommand{\abr}{\emph{abbr.}\xspace}
\newcommand{\ea}{\emph{et al.}\xspace}
\newcommand{\gensync}{\emph{GenSync}\xspace}
\newcommand{\colosseum}{\emph{Colosseum}\xspace}
\newcommand{\srep}{\emph{SREP}\xspace} % Set Reconciliation Enhances
\newcommand{\srepsim}{\emph{SREPSim}\xspace}
% Propagation
\newcommand{\esrep}{\emph{E-SREP}\xspace}
\newcommand{\epsrep}{\emph{EP-SREP}\xspace}
\newcommand{\mesrep}{\emph{ME-SREP}\xspace}
\newcommand{\mempoolsync}{\emph{MempoolSync}}

\newcommand{\fref}[1]{Fig.~\ref{#1}}
\newcommand{\tref}[1]{Table~\ref{#1}}
\newcommand{\aref}[1]{Algorithm~\ref{#1}}
\newcommand{\procref}[1]{Procedure~\ref{#1}}
\newcommand{\sref}[1]{Section~\ref{#1}}
\newcommand{\lineref}[1]{line~\ref{#1}}
\newcommand{\appref}[1]{Appendix~\ref{#1}}

% Change \eqref
\LetLtxMacro{\originaleqref}{\eqref}
\renewcommand{\eqref}{Eq.~\originaleqref}

% Theorems and corollaries
\newcounter{theoremcount}
\setcounter{theoremcount}{0}
\DeclareRobustCommand{\theorem}[1]{%
  \refstepcounter{theoremcount}%
  \noindent\textit{\textbf{Theorem \thetheoremcount\label{theorem:#1}: }}%
}
\DeclareRobustCommand{\theoremref}[1]{Theorem~\ref{theorem:#1}}

\DeclareRobustCommand{\proof}{\emph{Proof:}\xspace}
\DeclareRobustCommand{\qqed}{\hfill$\blacksquare$}

\newcounter{corollcount}
\setcounter{corollcount}{0}
\DeclareRobustCommand{\coroll}[1]{%
  \refstepcounter{corollcount}%
  \noindent\textit{\textbf{Corollary \thecorollcount\label{coroll:#1}: }}%
}
\DeclareRobustCommand{\corollref}[1]{Corollary~\ref{coroll:#1}}

\newcounter{lemmacount}
\setcounter{lemmacount}{0}
\DeclareRobustCommand{\lemma}[1]{%
  \refstepcounter{lemmacount}%
  \noindent\textit{\textbf{Lemma \thelemmacount\label{lemma:#1}: }}%
}
\DeclareRobustCommand{\lemmaref}[1]{Lemma~\ref{lemma:#1}}

\newcounter{definitioncount}
\setcounter{definitioncount}{0}
\DeclareRobustCommand{\definition}[1]{%
  \refstepcounter{definitioncount}%
  \noindent\textit{\textbf{Definition \thedefinitioncount\label{definition:#1}: }}%
}
\DeclareRobustCommand{\defref}[1]{Definition~\ref{definition:#1}}

%notes of different authors
\newif\ifnotes
\notestrue
\notesfalse

\newif\ifdiff
\difftrue
\difffalse

\newcommand{\anote}[1]{\ifnotes $\ll$\textsf{\textcolor{purple}{Ari: {#1}}}$\gg$ \fi}
\newcommand{\nnote}[1]{\ifnotes $\ll$\textsf{\textcolor{orange}{Novak: {#1}}}$\gg$ \fi}
\newcommand{\diff}[1]{\ifdiff\textcolor{orange}{#1}\else#1\fi}

%%% Local Variables:
%%% mode: latex
%%% TeX-master: "main"
%%% End:

\begin{document}

\maketitle

\begin{abstract}

We build quantum cryptosystems that support publicly-verifiable deletion from standard cryptographic assumptions. We introduce target-collapsing as a weakening of collapsing for hash functions, analogous to how second preimage resistance weakens collision resistance; that is, target-collapsing requires indistinguishability between superpositions and mixtures of preimages of an honestly sampled image. 

We show that target-collapsing hashes enable publicly-verifiable deletion ($\PVD$), proving 
conjectures from [Poremba, ITCS'23] and demonstrating that the Dual-Regev encryption (and corresponding fully homomorphic encryption) schemes support $\PVD$ under the LWE assumption. 
We further build on this framework to obtain a variety of primitives supporting publicly-verifiable deletion from weak cryptographic assumptions, including:
\begin{itemize}
    \item Commitments with $\PVD$ assuming the existence of injective one-way functions, or more generally, {\em almost-regular} one-way functions. Along the way, we demonstrate that (variants of) target-collapsing hashes can be built from almost-regular one-way functions.
    \item Public-key encryption with $\PVD$ assuming trapdoored variants of injective (or almost-regular) one-way functions. We also
    demonstrate that the encryption scheme of [Hhan, Morimae, and Yamakawa, Eurocrypt'23] based on pseudorandom group actions
    has $\PVD$.
    \item $X$ with $\PVD$ for $X \in \{$attribute-based encryption, quantum fully-homomorphic encryption, witness encryption, time-revocable encryption$\}$,
    assuming $X$ and trapdoored variants of injective (or almost-regular) one-way functions.  
\end{itemize}
\end{abstract}

\newpage
\tableofcontents

\newpage
\section{Introduction}

The increasing complexity of source code poses a key challenge to the reliability of large-scale software systems. Software bugs in these systems can lead to safety issues~\cite{bug_safety} for users around the world as well as cause non-negligible financial losses~\cite{bug_loss}. As such, developers have to spend a large amount of time and effort on bug fixing. Consequently, \aprfull (\apr), designed to automatically generate patches to fix software bugs, has attracted wide attention from both academia and industry~\cite{long2016prophet, legoues2012genprog, long2015spr, lou2020can, tufano2018empstudy}. 


To achieve \apr, one popular approach is known as Generate-and-Validate (G\&V)~\cite{qi2015gv, ghanbari2019prapr, lou2020can, le2016hdrepair, legoues2012genprog, wen2018capgen, hua2018sketchfix, martinez2016astor, koyuncu2020fixminder, liu2019tbar, liu2019avatar}, which is typically based on the following pipeline: First, fault localization techniques~\cite{wong2016fl, abreu2007ochiai, zhang2013injecting, papadakis2015metallaxis, li2019deepfl, li2017transforming} are applied to determine the suspicious locations in programs where bugs are likely to exist. Then, the buggy locations are used by the \apr tools to generate a list of patches that replace buggy lines with correct lines. Afterward, each patch is validated against the original test suite to identify any \emph{plausible patches} (i.e., passing all tests in the test suite). Finally, to determine the \emph{correct patches}, developers examine the list of plausible patches to see if any of them can correctly fix the bug. 

Traditional \apr tools can mainly be categorized into heuristic-based~\cite{legoues2012genprog, le2016hdrepair, wen2018capgen}, constraint-based~\cite{mechtaev2016angelix, le2017s3, demacro2014nopol, long2015spr} and \template~\cite{ghanbari2019prapr, hua2018sketchfix, martinez2016astor, liu2019tbar, liu2019avatar}. Among these traditional tools, \template \apr tools~\cite{ghanbari2019prapr, liu2019tbar, benton2020effectiveness} have been able to achieve state-of-the-art results. \Template \apr tools typically leverage pre-defined templates (e.g., adding a nullness check) for bug fixing. However, since these fix templates are typically handcrafted, the number and types of bugs they are able to fix can be limited. 



To address the limitations of traditional \apr, researchers have proposed various \learning \apr tools~\cite{li2020dlfix, chen2018sequencer, jiang2021cure, lutellier2020coconut, zhu2021recoder, ye2022rewardrepair} based on the \nmtfull (\nmt) architecture~\cite{sutskever2014mt} where the input is the buggy code snippets and the goal is to translate the buggy code snippets into a fixed version. To accomplish this, \learning \apr tools require supervised training datasets with pairs of both buggy and fixed code snippets in order to learn how to perform this translation step. These training data are usually obtained by mining historical bug fixes using heuristics/keywords~\cite{dallmeier2007benchmark}, which can be imprecise for identifying bug-fixing commits; even the actual bug-fixing commits can include irrelevant code changes, leading to further pollution in the dataset~\cite{xia2022alpharepair}.
% 
Moreover, it can be hard for such \apr tools to generalize and fix bug types unseen during training. 



To better leverage recent advances in \plmfull{s} (\plm{s}), researchers~\cite{xia2022alpharepair, xia2023repairstudy, kolak2022patch, prenner2021codexws} have directly applied \plm{s} to generate patches without bug-fixing datasets. These \llm-based \apr tools work by either directly generating a complete code function~\cite{prenner2021codexws, xia2023repairstudy} or predict/infill the correct code snippet given its surrounding context~\cite{xia2022alpharepair, xia2023repairstudy}. By directly using \llm{s} that are pre-trained on billions of open-source code snippets, \llm-based \apr tools can achieve state-of-the-art performance on many repair datasets~\cite{xia2022alpharepair}. 


% 
%
%

Traditional \apr tools have long used the insight of the \emph{plastic surgery hypothesis}~\cite{barr2014plastic} where it states that the code ingredients to fix a bug already exist within the same project. Traditional \apr tools have manually designed pattern-~\cite{ghanbari2019prapr, saha2017elixir} or heuristic-based~\cite{jiang2018simfix, legoues2012genprog} approaches to finding and using such relevant code ingredients to generate fixes for bugs. However, the plastic surgery hypothesis has been largely ignored in \llm-based \apr. In fact, \llm provides a unique opportunity to fully automate the plastic surgery hypothesis idea via fine-tuning (learning project-specific information via model updates from the buggy project) and prompting (directly providing relevant code ingredients to the model), and make it directly applicable to different languages (since the \llm{s} are typically multi-lingual).%
Moreover, despite the intensive manual efforts involved, traditional \apr tools still cannot fully leverage project-specific information due to large search space for leveraging/composing existing code ingredients. In contrast, the project-specific information can effectively leveraged by \llm{s} due to their power in code understanding/vectorization, e.g., even partial/imprecise information may still guide \llm{s} in correct patch generation!
 To this end, we ask the question: \emph{How useful is the plastic surgery hypothesis in the era of \plm{s}}?








\mypara{Our Work.} To answer the question, we present \ourtech{\xspace} -- a \llm-based approach that automatically utilizes the plastic surgery hypothesis by systematically combining multiple fine-tuning and prompting strategies for \apr. \ourtech fine-tunes \plm{s} using two novel domain-specific training strategies: \textbf{\epfinetune} -- we fine-tune using the original buggy project by aggressively masking out a high percentage of tokens, which allows \plm to learn project-specific code tokens and programming styles; and \textbf{\rofinetune} -- which only masks out a single continuous code sequence per training sample, allowing the model to get used to the final \csapr task of predicting a single continuous code sequence. Furthermore, we directly leverage the ability for \plm{s} to understand natural language instructions and introduce a novel prompting strategy, \textbf{\idprompting}, which uses information retrieval and static analysis to obtain a list of relevant identifiers for the buggy lines. While such relevant identifiers are critical for fixing some difficult bugs, they may not be seen by the \llm during inference due to limited context window size. Through the use of prompting, we directly tell the model to use these extracted identifiers (relevant code ingredients) to generate the correct code. Finally, to perform repair, we combine all four model variants (including the base model, both fine-tuned models and the base model with prompting) for the final repair.





While our insight of leveraging the plastic surgery hypothesis for \llm-based \apr is generalizable across different types of \plm{s}, to implement \ourtech, we choose a recent \plm{\xspace}, \ctfive~\cite{wang2021codet5}, which is pre-trained on millions of open-source code snippets. \ctfive is an encoder-decoder model trained using \mspfull (\msp) objective where a percentage of tokens are masked out and each continuous masked token sequence is referred to as a masked span. Also, although we only extract relevant identifiers from the current buggy project (since this paper focuses on the plastic surgery hypothesis), our work can be easily extended to obtain other code information (such as relevant statements or functions) from other sources, such as  the massive pre-training corpora~\cite{husain2020codesearchnet} or historical bug-fixing datasets~\cite{jiang2019infer}, which can provide more coding knowledge for \llm{s}. Besides, although we mainly focus on using traditional string comparison algorithms for information retrieval in this paper, these techniques can be easily replaced by other frequency-based retrieval~\cite{robertson2009probabilistic} and neural search (or embedding-based search)~\cite{reimers2019sentence}.
  In summary, this paper makes the following contributions:


%


\begin{itemize}[noitemsep, leftmargin=*, topsep=0pt]
    \item \textbf{Dimension.} This paper is the first to revisit the important plastic surgery hypothesis in the era of \llm{s}. It opens up a new dimension for \llm-based \apr to incorporate previously neglected information from the buggy project itself to boost \apr performance. Furthermore, it demonstrates the promising future of retrieval-based prompting for modern \llm-based \apr.
    \item \textbf{Implementation.} We implement \ourtech based on the recent \ctfive model. We augment the model using two novel fine-tuning strategies: \epfinetune and \rofinetune, along with a novel prompting strategy based on information retrieval and static analysis: \idprompting. We combine the patches generated by all four models together and perform patch ranking to speed up \apr.% 
    \item \textbf{Evaluation Study.} We conduct an extensive evaluation against state-of-the-art \apr tools. On the widely studied \dfj 1.2 and 2.0 datasets~\cite{just2014dfj}, \ourtech is able to achieve the new state-of-the-art results of 89 and 44 correct bug fixes (15 and 8 more than best baseline) respectively.  Furthermore, we perform a broad ablation study to justify our design. \ourtech demonstrates for the first time that the plastic surgery hypothesis can substantially boost \llm-based \apr and advance state-of-the-art \apr, while being fully automated and general. Moreover, even partial/imprecise code ingredients may still effectively guide \llm{s} for \apr!
\end{itemize}



\section{Preliminaries}

In this section, we review basic concepts from quantum computing and cryptography.

\subsection{Quantum Computing}

We refer to \cite{NielsenChuang11,Wilde13} for a comprehensive background on quantum computation. 

A finite-dimensional complex Hilbert space is denoted by $\mathcal{H}$, and we use subscripts to distinguish between different systems (or registers); for example, we let $\mathcal{H}_{\mathsf{A}}$ be the Hilbert space corresponding to a system $\mathsf{A}$. 
The tensor product of two Hilbert spaces $\algo H_{\mathsf{A}}$ and $\algo H_{\mathsf{B}}$ is another Hilbert space denoted by $\algo H_{\mathsf{AB}} = \algo H_{\mathsf{A}} \otimes \algo H_{\mathsf{B}}$.  We let $\algo L(\algo H)$
denote the set of linear operators over $\algo H$.
A quantum system over the $2$-dimensional Hilbert space $\mathcal{H} = \mathbb{C}^2$ is called a \emph{qubit}. For $n \in \mathbb{N}$, we refer to quantum registers over the Hilbert space $\mathcal{H} = \big(\mathbb{C}^2\big)^{\otimes n}$ as $n$-qubit states. We use the word \emph{quantum state} to refer to both pure states (unit vectors $\ket{\psi} \in \mathcal{H}$) and density matrices $\rho \in \mathcal{D}(\mathcal{H)}$, where we use the notation $\mathcal{D}(\mathcal{H)}$ to refer to the space of positive semidefinite linear operators of unit trace acting on $\algo H$. 
Occasionally, we consider \emph{subnormalized states}, i.e. states in the space of positive semidefinite operators over $\algo H$ with trace norm not exceeding $1$.

The \emph{trace distance} of two density matrices $\rho,\sigma \in \mathcal{D}(\mathcal{H)}$ is given by
$$
\TD(\rho,\sigma) = \frac{1}{2} \Tr\left[ \sqrt{ (\rho - \sigma)^\dag (\rho - \sigma)}\right].
$$

A quantum channel $\Phi:  \algo L(\algo H_{\mathsf{A}}) \rightarrow \algo L(\algo H_{\mathsf{B}})$ is a linear map between linear operators over the Hilbert spaces $\algo H_{\mathsf{A}}$ and $\algo H_{\mathsf{B}}$. 
We say that a channel $\Phi$ is \emph{completely positive} if, for a reference system $R$ of arbitrary size, the induced map $I_R \otimes \Phi$ is positive, and we call it \emph{trace-preserving} if $\Tr[\Phi(X)] = \Tr[X]$, for all $X\in \algo L(\algo H)$. A quantum channel that is both completely positive and trace-preserving is called a quantum $\CPTP$ channel. 

A polynomial-time \emph{uniform} quantum algorithm (or $\QPT$ algorithm) is a polynomial-time family of quantum circuits given by $\algo C = \{C_\lambda\}_{\lambda \in \N}$, where each circuit $C \in \algo C$ is described by a sequence of unitary gates and measurements; moreover, for each $\lambda \in \N$, there exists a deterministic polynomial-time Turing machine that, on input $1^\lambda$, outputs a circuit description of $C_\lambda$. Similarly, we also define (classical) probabilistic polynomial-time $(\PPT)$ algorithms. A quantum algorithm may, in general, receive (mixed) quantum states as inputs and produce (mixed) quantum states as outputs. Occasionally, we restrict $\QPT$ algorithms implicitly; for example, if we write $\Pr[\mathcal{A}(1^{\lambda}) = 1]$ for a $\QPT$ algorithm $\mathcal{A}$, it is implicit that $\mathcal{A}$ is a $\QPT$ algorithm that outputs a single classical bit.

\paragraph{Quantum Fourier transform.} Let $q \geq 2$ be a modulus and $n \in \N$ and let $\omega_q = e^{ \frac{2 \pi i}{q}} \in \mathbb{C}$ denote the primitive $q$-th root of unity.
The $m$-qudit \emph{$q$-ary quantum Fourier transform} over the ring $\Z_q^m$ is defined by the operation,
$$
\FT_q : \quad \ket{\vec x} \quad \mapsto \quad \sqrt{q^{-m}} \displaystyle\sum_{\vec y \in \Z_q^m} \omega_q^{\langle \vec x,\vec y\rangle} \ket{\vec y}, \quad\quad \forall \vec x \in \Z_q^m.
$$
The $q$-ary quantum Fourier transform is \emph{unitary} and can be efficiently implemented on a quantum computer for any integer modulus $q \geq 2$~\cite{892139}.


\paragraph{Pauli Twirling.}

We use the following unitary operators:
\begin{itemize}
    \item Pauli-$\mathsf{Z}$ operator:

    $$
\mathsf{Z}^z = \sum_{x \in \bit} (-1)^{x \cdot z} \proj{x}, \quad \text{ for } z \in \bit.
$$

\item Multi-qubit Pauli-$\mathsf{Z}$ operator:

    $$
\mathsf{Z}^z = \mathsf{Z}^{z_1} \otimes \dots \otimes \mathsf{Z}^{z_m}, \quad \text{ for } z \in \bit^m.
$$

\item Controlled-$\mathsf{Z}$ operator:


    $$
\mathsf{C}\mathsf{Z}^z = \sum_{c \in \bit} \proj{c} \otimes \mathsf{Z}^{c \cdot z}, \quad \text{ for } z \in \bit^m.
$$

\end{itemize}
Here, we use the notation $\mathsf{Z}^0 = I$ and $\mathsf{Z}^1 = \mathsf{Z}$, as well as $c \cdot z = (c \cdot z_1,\dots,c \cdot z_m)$ for $z \in \bit^m$.

We use the following well-known property of the Pauli-$\mathsf{Z}$ dephasing channel which says that, on average, a random Pauli-Z twirl induces a measurement in the computational basis.

\begin{lemma}[Pauli-$\mathsf{Z}$ Twirl]\label{lem:random-Z}
The Pauli-$\mathsf{Z}$ dephasing channel applied to an $m$-qubit state $\rho$ satsifies
$$
\algo Z(\rho) \,\overset{\mathrm{def}}{=} \,\ 2^{-m} \sum_{z \in \bit^m}  \mathsf{Z}^{z} \rho \left(\mathsf{Z}^{z}\right)^\dag= \sum_{x \in \bit^m} \Tr[\ketbra{x}{x} \rho] \,\ketbra{x}{x}.
$$
\end{lemma}



\subsection{Cryptography}

Throughout this work, wet $\lambda\in \N$ denote the security parameter. We assume that the reader is familiar with the fundamental cryptographic concepts. 

\paragraph{The Short Integer Solution problem.}
%\label{sec:sis}

The (inhomogenous) $\SIS$ problem was introduced by Ajtai~\cite{DBLP:conf/stoc/Ajtai96} in his seminal work on average-case lattice problems. The problem is defined as follows. 

\begin{definition}[Inhomogenous SIS problem,\cite{DBLP:conf/stoc/Ajtai96}]\label{def:ISIS} Let $n,m \in \N$ be integers, let $q\geq 2$ be a modulus and let $\beta >0$ be a parameter. The Inhomogenous Short Integer Solution problem $(\ISIS)$ problem is to find a short solution $\vec x \in \Z^m$ with $\|\vec x\| \leq \beta$ such that $\vec A \cdot \vec x = \vec y \Mod{q}$ given as input a tuple $(\vec A \rand \Z_q^{n \times m},\vec y \rand \Z_q^n)$.
The Short Integer Solution $(\SIS)$ problem is a homogenous variant of the $\ISIS$ problem with input $(\vec A \rand \Z_q^{n \times m},\vec 0 \in\Z_q^n)$.
\end{definition}


Micciancio and Regev~\cite{DBLP:journals/siamcomp/MicciancioR07} showed that the $\SIS$ problem is, on the average, as hard as approximating worst-case lattice problems to within small factors. Subsequently, Gentry, Peikert and Vaikuntanathan~\cite{cryptoeprint:2007:432} gave an improved reduction showing that, for parameters $m=\poly(n)$, $\beta=\poly(n)$ and prime $q \geq \beta \cdot \omega(\sqrt{n \log q})$, the average-case $\SIS_{n,q,\beta}^m$ problem is as hard as approximating the shortest independent vector problem $(\mathsf{SIVP})$ problem in the
worst case to within a factor $\gamma = \beta \cdot \tilde{O}(\sqrt{n})$.
We assume that $\SIS_{n,q,\beta}^m$, for $m=\Omega(n \log q)$, $\beta = 2^{o(n)}$ and $q=2^{o(n)}$, is hard against polynomial-time quantum adversaries.

\paragraph{The Learning with Errors problem.}
%\label{sec:lwe}

The \emph{Learning with Errors} problem serves as the primary basis of hardness of post-quantum cryptosystems and was introduced by Regev~\cite{Regev05}. The problem is defined as follows. 


\begin{definition}[Learning with Errors problem, \cite{Regev05}]\label{def:decisional-lwe} Let $n,m \in \N$ be integers, let $q\geq 2$ be a modulus and let $\alpha \in (0,1)$ be a noise ratio parameter. The (decisional) Learning with Errors $(\LWE_{n,q,\alpha q}^m)$ problem is to distinguish between the following samples
$$
(\vec A \rand \Z_q^{n \times m},\vec s^\intercal \vec A+ \vec e^\intercal \Mod{q}) \quad \text{ and } \quad (\vec A \rand \Z_q^{n \times m},\vec u \rand \Z_q^m),\,\,
$$
where $\vec s \rand  \Z_q^n$ is a uniformly random vector and where $\vec e \sim D_{\Z^m,\sigma}$ is a discrete Gaussian error vector, where $D_{\Z^m,\sigma}$
assigns probability proportional to
$\rho_\sigma(\vec x) = \exp(-\pi \|\vec x \|^2/ \sigma^2)$ to each $\vec x \in \Z^m$, for $\sigma = \alpha q>0$.

We rely on the quantum $\LWE_{n,q,\alpha q}^m$ assumption which states that the samples above are computationally indistinguishable for any $\QPT$ algorithm.

\end{definition}

It was shown in~\cite{Regev05,cryptoeprint:2017/258} that the $\LWE_{n,q,\alpha q}^m$ problem with parameter $\alpha q \geq 2 \sqrt{n}$ is at least as hard as approximating the shortest independent vector problem $(\mathsf{SIVP})$ to within a factor of $\gamma = \widetilde{O}(n / \alpha)$ in worst case lattices of dimension $n$. In this work we assume the subexponential hardness of $\LWE_{n,q,\alpha q}^m$ which relies on the worst case hardness of approximating short vector problems in
lattices to within a subexponential factor. 
We assume that $\LWE_{n,q,\alpha q}^m$, for $m=\Omega(n \log q)$, $q=2^{o(n)}$, $\alpha=1/2^{o(n)}$,  is hard against polynomial-time quantum adversaries. 
%We note that this assumption implies  $\SIS_{n,q,\beta}^m$ for the parameters described in~\Cref{sec:sis}. 

\section{Main Theorem: Certified Everlasting Target-Collapsing}

%The notion of a \emph{collapsing} hash function was introduced by Unruh~\cite{cryptoeprint:2015/361} as a quantum strengthening of collision resistance. In the classical setting, 

%Collapsing can be considered a quantum analogue of classical collision-resistance. In the classical setting, a weaker security notion has also been studied, called \emph{target}-collision-resistance \cite{10.1145/73007.73011}, which requires that for any \emph{fixed} input $x$, no polynomial-time adversary can find a collision $x' \neq x$ such that $h(x) = h(x')$. We now formalize this definition, and then introduce a quantum analogue which we call \emph{target-collapsing}.

\subsection{Definitions}

In this section, we present our definitions of target-collapsing and (generalized) target-collision-resistance. We parameterize our definitions by a distribution $\cD$ over preimages and a measurement function $\cM$. Note that when $\cM$ is the identity function, the notion of $(\cD,\cM)$-target-collapsing corresponds to a notion where the entire preimage register is measured in the computational basis. In this case we drop parameterization by $\cM$ and just say $\cD$-target-collapsing. Also, when $\cD$ is the uniform distribution, we drop parameterization by $\cD$ and just say $\cM$-target-collapsing.

\begin{definition}[$(\cD,\cM)$-Target-Collapsing Hash Function]\label{def:target-collapsing}
Let $\lambda \in \N$ be the security parameter.
A hash function family given by $\cH = \{H_\secp : \{0,1\}^{m(\secp)} \to \{0,1\}^{n(\secp)}\}_{\secp \in \bbN}$ is $(\cD,\cM)$-target-collapsing for some distribution $\cD = \{D_\secp\}_{\secp \in \bbN}$ over $\{\{0,1\}^{m(\secp)}\}_{\secp \in \bbN}$ and family of functions $\cM = \{\{M[h] : \{0,1\}^{m(\secp)} \to \{0,1\}^{k(\secp)}\}_{h \in H_\secp}\}_{\secp \in \bbN}$ if, for every QPT adversary $\cA = \{\cA_\secp\}_{\secp \in \bbN}$,
$$
|
\Pr[ \mathsf{TargetCollapseExp}_{\algo H,\algo A,\algo D,\algo M,\lambda}(0)=1] - \Pr[ \mathsf{TargetCollapseExp}_{\algo H,\algo A,\algo D,\algo M,\lambda}(1)=1]
| \leq \negl(\lambda).
$$
Here, the experiment $\mathsf{TargetCollapseExp}_{\algo H,\algo A,\algo D,\algo M,\lambda}(b)$ is defined as follows:

\begin{enumerate}

    \item The challenger prepares the state \[\sum_{x \in \{0,1\}^{m(\secp)}}\sqrt{D_\secp(x)}\ket{x}\] on register $X$, and samples a random hash function $h \rand H_\lambda$. Then, it coherently computes $h$ on $X$ (into a fresh $n(\secp)$-qubit register $Y$) and measures system $Y$ in the computational basis, which results in an outcome $y \in \bit^{n(\lambda)}$.
    \item If $b=0$, the challenger does nothing. Else, if $b=1$, the challenger coherently computes $M[h]$ on $X$ (into a fresh $k(\secp)$-qubit register $V$) and measures system $V$ in the computational basis. Finally, the challenger sends the outcome state in system $X$ to $\algo A_\secp$, together with the string $y \in \bit^{n(\lambda)}$ and a description of the hash function $h$.
    \item $\algo A_\secp$ returns a bit $b'$, which we define as the output of the experiment.
\end{enumerate}
\end{definition}

We also define an analogous notion of $(\cD,\cM)$-target-collision-resistance, as follows. Similarly to above, we drop the parameterization by $\cM$ in the case that it is the identity function, and we drop the parameterization by $\cD$ in the case that it is the uniform distribution. Notice that target-collision-resistance (without parameterization) then coincides with the classical notion where a uniformly random input is sampled, and the adversary must find a collision with respect to this input (this is also sometimes called second-preimage resistance, or weak collision-resistance).

%\dakshita{check that this does not contradict classical definitions}\james{how does the above sound?}

\begin{definition}[$(\cD,\cM)$-Target-Collision-Resistant Hash Function]\label{def:target-CR}
A hash function family $\cH = \{H_\secp : \{0,1\}^{m(\secp)} \to \{0,1\}^{n(\secp)}\}_{\secp \in \bbN}$ is $(\cD,\cM)$-target-collision-resistant for some distribution $\cD = \{D_\secp\}_{\secp \in \bbN}$ over $\{\{0,1\}^{m(\secp)}\}_{\secp \in \bbN}$ and family of functions $\cM = \{\{M[h] : \{0,1\}^{m(\secp)} \to \{0,1\}^{k(\secp)}\}_{h \in H_\secp}\}_{\secp \in \bbN}$ if, for every QPT adversary $\cA = \{\cA_\secp\}_{\secp \in \bbN}$,
$$
|
\Pr[ \mathsf{TargetCollRes}_{\algo H,\algo A,\algo D,\algo M,\lambda}=1]| \leq \negl(\lambda).
$$
Here, the experiment $\mathsf{TargetCollRes}_{\algo H,\algo A,\algo D,\algo M,\lambda}$ is defined as follows:
\begin{enumerate}
    \item The challenger prepares the state \[\sum_{x \in \{0,1\}^{m(\secp)}}\sqrt{D_\secp(x)}\ket{x}\] on register $X$, and samples a random hash function $h \rand H_\lambda$. Next, it coherently computes $h$ on $X$ (into a fresh $n(\secp)$-qubit system $Y$) and measures system $Y$ in the computational basis, which results in an outcome $y \in \bit^{n(\lambda)}$. Next, it coherently computes $M[h]$ on $X$ (into a fresh $k(\secp)$-qubit register $V$) and measures system $V$ in the computational basis, which results in an outcome $v$. Finally, its sends the outcome state in system $X$ to $\algo A_\secp$, together with the string $y \in \{0,1\}^{n(\secp)}$ and a description of the hash function $h$.
    \item $\algo A_\secp$ responds with a string $x \in \{0,1\}^{m(\secp)}$.
    \item The experiment outputs 1 if $h(x) = y$ and $M[h](x) \neq v$.
\end{enumerate}

\end{definition}

Finally, we define the notion of a \emph{certified everlasting} target-collapsing hash.

\begin{definition}\label{def:ev-target-collapsing}
A hash function family $\cH = \{H_\secp : \{0,1\}^{m(\secp)} \to \{0,1\}^{n(\secp)}\}_{\secp \in \bbN}$ is certified everlasting $(\cD,\cM)$-target-collapsing for some distribution $\cD = \{D_\secp\}_{\secp \in \bbN}$ over $\{\{0,1\}^{m(\secp)}\}_{\secp \in \bbN}$ and family of functions $\cM = \{\{M[h] : \{0,1\}^{m(\secp)} \to \{0,1\}^{k(\secp)}\}_{h \in H_\secp}\}_{\secp \in \bbN}$ if for every two-part adversary $\algo A = \{\algo A_{0,\secp},\algo A_{1,\secp}\}_{\secp \in \bbN}$, where $\{\algo A_{0,\secp}\}_{\secp \in \bbN}$ is QPT and $\{\algo A_{1,\secp}\}_{\secp \in \bbN}$ is unbounded, it holds that 

%adversary $\algo A = \{(\algo A_{\secp,0},\algo A_{\secp,1})\}_{\secp \in \bbN}$ consisting of a  and an unbounded algorithm $\algo A_{\secp,1}$, and any bit $b' \in \{0,1\}$, it holds that
$$
|\Pr\left[\mathsf{EvTargetCollapseExp}_{\algo H,\algo A,\algo D,\algo M,\lambda}(0) = 1 \right] - \Pr\left[\mathsf{EvTargetCollapseExp}_{\algo H,\algo A,\algo D,\algo M,\lambda}(1) = 1\right]| \leq \negl(\lambda).
$$
Here, the experiment $\mathsf{EvTargetCollapseExp}_{\algo H,\algo A,\algo D,\algo M,\lambda}(b)$ is defined as follows:

\begin{enumerate}

    \item The challenger prepares the state \[\sum_{x \in \{0,1\}^{m(\secp)}}\sqrt{D_\secp(x)}\ket{x}\] on register $X$, and samples a random hash function $h \rand H_\lambda$. Then, it coherently computes $h$ on $X$ (into a fresh $n(\secp)$-qubit system $Y$) and measures system $Y$ in the computational basis, which results in an outcome $y \in \bit^{n(\lambda)}$.
    \item If $b=0$, the challenger does nothing. Else, if $b=1$, the challenger coherently computes $M[h]$ on $X$ (into an auxiliary $k(\secp)$-qubit system $V$) and measures system $V$ in the computational basis. Finally, the challenger sends the outcome state in system $X$ to $\algo A_{0,\secp}$, together with the string $y \in \bit^{n(\lambda)}$ and a description of the hash function $h$.
    \item $\algo A_{0,\secp}$ sends a classical certificate $\pi \in \bit^{m(\lambda)}$ to the challenger and initializes $\algo A_{1,\secp}$ with its residual state.
    \item The challenger checks if $h(\pi)=y$. If true, $\algo A_{1,\secp}$ is run until it outputs a bit $b'$. Otherwise, $b' \gets \{0,1\}$ is sampled uniformly at random. The output of the experiment is $b'$.
    
    
    %the challenger sends $\top$ to $\cA_{\secp,1}$ and the game continues; else, the game ends and the output of the experiment is $\bot$.
    %\item $\algo A_{\secp,1}$ outputs a bit $b'$, which we define as the output of the experiment.
    
\end{enumerate}
\end{definition}

\subsection{Main Theorem}
\label{sec:maintheorem}
Our main theorem is the following.

\begin{theorem}\label{thm:CETC-generalization}
Let $\cH = \{H_\secp\}_{\secp \in \bbN}$ be a hash function family that is both $(\cD,\cM)$-target-collapsing and $(\cD,\cM)$-target-collision-resistant, for some distribution $\cD$ and efficiently computable family of functions $\cM$. Then, $\cH$ is certified everlasting $(\cD,\cM)$-target-collapsing.
\end{theorem}

\begin{proof}

Throughout the proof, we will leave the security parameter implicit, defining $H \coloneqq H_\secp, D \coloneqq D_\secp, m \coloneqq m(\secp), n \coloneqq n(\secp)$, $k \coloneqq k(\secp)$, $\cA_0 \coloneqq \cA_{0,\secp}$, and $\cA_1 \coloneqq \cA_{1,\secp}$. Next, we define

\[\ket{\psi}_X \coloneqq \sum_{x \in \{0,1\}^m}\sqrt{D(x)}\ket{x}.\] For $h \in H, y \in \{0,1\}^m$, we define a unit vector \[\ket{\psi_{h,y}}_X \propto  \sum_{x \in \{0,1\}^m : h(x)=y}\sqrt{D(x)}\ket{x}.\] Finally, for $h \in H, y \in \{0,1\}^m, v \in \{0,1\}^k$ we define a unit vector \[\ket{\psi_{h,y,v}}_X \propto  \sum_{x \in \{0,1\}^m : h(x)=y, M[h](x)=v}\sqrt{D(x)}\ket{x}.\]

% ~~ \ket{\psi_{h,y}'}_{X,V} \propto \sum_{x \in \{0,1\}^m : h(x)=y}\sqrt{D(x)}\ket{x}\ket{F[h](x)},

We consider the following hybrids. 

\begin{itemize}
    \item $\mathsf{Exp}_0(b)$:
    \begin{enumerate}
    %\item The adversary sends an $m(\lambda)$-qubit quantum state $\rho$ in a system $X$ to the challenger.
    \item The challenger prepares $\ket{\psi}_X$, samples a random hash function $h \rand H_\lambda$, coherently computes $h$ on $X$ into a fresh $n$-qubit register $Y$, and measures $Y$ in the computational basis to obtain $y \in \bit^{n}$ and a left-over state $\ket{\psi_{h,y}}_X$.
    \item If $b=0$, the challenger does nothing. Else, if $b=1$, the challenger computes $M[h]$ on $X$ into a fresh $k$-qubit register $V$, and measures $V$ in the computational basis. Finally, the challenger sends the left-over state in system $X$ to $\algo A_0$, together with the string $y \in \bit^{n}$ and a classical description of $h$.

    \item $\algo A_0$ sends a classical certificate $\pi \in \bit^m$ to the challenger and initializes $\algo A_1$ with its residual state.

    \item The challenger checks if $h(\pi)=y$. If true, $\algo A_1$ is run until it outputs a bit $b'$. Otherwise, $b' \gets \{0,1\}$ is sampled uniformly at random. The output of the experiment is $b'$.
    
    
    %If true, the challenger sends $\top$ to $\cA_1$ and the game continues; else, the game ends and the output of the experiment is $\bot$.

    %\item $\algo A_1$ outputs a bit $b'$, which we define as the output of the experiment.
\end{enumerate}

    \item $\mathsf{Exp}_1(b)$:
    \begin{enumerate}
    \item The challenger prepares $\ket{\psi}_X$, samples a random hash function $h \rand H_\lambda$, coherently computes $h$ on $X$ into a fresh $n$-qubit register $Y$, and measures $Y$ in the computational basis to obtain $y \in \bit^{n}$ and a left-over state $\ket{\psi_{h,y}}_X$.
    \item The challenger computes $M[h]$ on $X$ into a fresh $k$-qubit register $V$ to obtain a state
    
    \[\propto \sum_{x \in \{0,1\}^m: h(x)=y} \sqrt{D(x)}\ket{x}_X\ket{M[h](x)}_V.\]
    
    Then, the challenger samples a random string $z \rand \bit^k$, prepares a $\ket{+}$ state in system $C$, and applies a controlled-$\mathsf{Z}^{z}$ operation from $C$ to $V$, which results in a state
    
    \begin{align*}
        &\propto \sum_{c \in \{0,1\}} \ket{c}_C \otimes \sum_{x \in \{0,1\}^m: h(x)=y} \sqrt{D(x)}\ket{x}_X\mathsf{Z}^{c \cdot z}\ket{M[h](x)}_V\\ &= \sum_{c \in \{0,1\}} \ket{c}_C \otimes \sum_{x \in \{0,1\}^m: h(x)=y} \sqrt{D(x)}(-1)^{c \cdot \langle M[h](x),z\rangle}\ket{x}_X\ket{M[h](x)}_V.
    \end{align*}
    
      %\[\propto \sum_{c \in \{0,1\}}\ket{c}_C \otimes \sum_{x \in \{0,1\}^m : h(x)=y}(-1)^{c \cdot \langle F[h](x),z\rangle}\ket{x}_X.\]
    
    
    Finally, the challenger uncomputes the $V$ register by again computing $M[h]$ from $X$ to $V$, and sends system $X$ to $\algo A_0$, together with $y \in \bit^n$ and a classical description of $h$.

    \item $\algo A_0$ sends a classical certificate $\pi \in \bit^{m}$ to the challenger and initializes $\algo A_1$ with its residual state.
    \item The challenger checks if $h(\pi)=y$. Then, the challenger measures system $C$ to obtain $c' \in \{0,1\}$ and checks that $c' = b$. If both checks are true, $\algo A_1$ is run until it outputs a bit $b'$. Otherwise, $b' \gets \{0,1\}$ is sampled uniformly at random. The output of the experiment is $b'$.
    
    
    %If false, the game ends and the output of the experiment is $\bot$. Then, the challenger measures system $C$ to obtain $c' \in \{0,1\}$. If $c'\neq b$, the game ends and the output of the experiment is $\bot$. Otherwise, if both checks passed, the output of the experiment is $\rho$.
    %\item $\algo A_1$ outputs a bit $b'$, which we define as the output of the experiment.
    \end{enumerate}

\item $\mathsf{Exp}_2(b)$:
    \begin{enumerate}
    \item The challenger prepares $\ket{\psi}_X$, samples a random hash function $h \rand H_\lambda$, coherently computes $h$ on $X$ into a fresh $n$-qubit register $Y$, and measures $Y$ in the computational basis to obtain $y \in \bit^{n}$ and a left-over state $\ket{\psi_{h,y}}_X$.
    \item The challenger computes $M[h]$ on $X$ into a fresh $k$-qubit register $V$. Then, the challenger samples a random string $z \rand \bit^k$, prepares a $\ket{+}$ state in system $C$, applies a controlled-$\mathsf{Z}^{z}$ operation from $C$ to $V$, and finally uncomputes the $V$ register by again computing $M[h]$ from $X$ to $V$. Note that this results in a state
    
    \[\propto \sum_{c \in \{0,1\}}\ket{c}_C \otimes \sum_{x \in \{0,1\}^m : h(x)=y}(-1)^{c \cdot \langle M[h](x),z\rangle}\ket{x}_X.\]
    
    Finally, it sends system $X$ to $\algo A_0$, together with $y \in \bit^n$ and a classical description of $h$.

    \item $\algo A_0$ sends a classical certificate $\pi \in \bit^{m}$ and initializes $\algo A_1$ with its residual state.


    \item The challenger checks if $h(\pi)=y$. Then, the challenger applies the following projective measurement to system $C$:
    \[\Big\{\proj{\phi_\pi^{ z}},I - \proj{\phi_\pi^{ z}}\Big\} \,\quad \text{ where } \quad \ket{\phi_\pi^{ z}} \coloneqq \frac{1}{\sqrt{2}}  \left( \ket{0} +  (-1)^{\langle M[h](\pi), z \rangle } \ket{1}\right),\] and checks that the first outcome is observed. Finally, the challenger measures system $C$ to obtain $c' \in \{0,1\}$ and checks that $c'=b$. If all three checks are true, $\algo A_1$ is run until it outputs a bit $b'$. Otherwise, $b' \gets \{0,1\}$ is sampled uniformly at random. The output of the experiment is $b'$.
    
    %Finally, the challenger measures system $C$ to obtain $c' \in \{0,1\}$. If $c'\neq b$, the game ends and the output of the experiment is $\bot$. Otherwise, if both checks passed, the challenger sends $\top$ to $\cA_1$ and the game continues.
    %\item $\algo A_1$ outputs a bit $b'$, which we define as the output of the experiment.
    \end{enumerate}

Finally, we also use the following hybrid which is convenient for the sake of the proof.

\item $\mathsf{Exp}_3(b)$:
     \begin{enumerate}
     \item The challenger prepares $\ket{\psi}_X$, samples a random hash function $h \rand H_\lambda$, coherently computes $h$ on $X$ into a fresh $n$-qubit register $Y$, and measures $Y$ in the computational basis to obtain $y \in \bit^{n}$ and a left-over state $\ket{\psi_{h,y}}_X$.
     \item The challenger computes $M[h]$ on $X$ into a fresh $k$-qubit register $V$. Then, the challenger measures $V$ in the computational basis to obtain $v \in \{0,1\}^k$. Next, the challenger samples a random string $z \rand \bit^k$, prepares a $\ket{+}$ state in system $C$, applies a controlled-$\mathsf{Z}^{z}$ operation from $C$ to $V$, and finally uncomputes the $V$ register by again computing $M[h]$ from $X$ to $V$. Note that this results in the state
    
    \[\frac{1}{\sqrt{2}}\left(\ket{0}_C + (-1)^{\langle v,z\rangle}\ket{1}_C\right) \otimes \ket{\psi_{h,y,v}}_X.\]
    
    Finally, the challenger sends system $X$ to $\algo A_0$, together with $y \in \bit^n$ and a classical description of $h$.
    
    \item $\algo A_0$ sends a classical certificate $\pi \in \bit^{m}$ to the challenger and initializes $\algo A_1$ with its residual state.

    \item The challenger checks if $h(\pi)=y$. Then, the challenger applies the following projective measurement to system $C$:
    \[\Big\{\proj{\phi_\pi^{ z}},I - \proj{\phi_\pi^{ z}}\Big\} \,\quad \text{ where } \quad \ket{\phi_\pi^{ z}} \coloneqq \frac{1}{\sqrt{2}}  \left( \ket{0} +  (-1)^{\langle M[h](\pi), z \rangle } \ket{1}\right),\] and checks that the first outcome is observed. Finally, the challenger measures system $C$ to obtain $c' \in \{0,1\}$ and checks that $c'=b$. If all three checks are true, $\algo A_1$ is run until it outputs a bit $b'$. Otherwise, $b' \gets \{0,1\}$ is sampled uniformly at random. The output of the experiment is $b'$.
    \end{enumerate}
\end{itemize}


Before we analyze the probability of distinguishing between the consecutive hybrids, we first show that the following statements hold for the final experiment $\mathsf{Exp}_3(b)$.

\begin{claim}\label{claim:identical-certificate}
The probability that the challenger accepts the deletion certificate $\pi$ in Step 4 of $\mathsf{Exp}_3(b)$ and $M[h](\pi) \neq v$ is negligible. That is,


\[\Pr_{h,y,v} \left[
 h(\pi) = y
\,\,\, \wedge \,\,\,
M[h](\pi) \,\neq\, v
 \,\, : \,\, \pi \gets \algo A_0(h,y,\ket{\psi_{h,y,v}})\right] \leq \negl(\lambda),\] where the probability is over the challenger preparing $\ket{\psi}$, sampling $h$, and measuring $y$ and $v$ as described in $\Exp_3(b)$ to produce the left-over state $\ket{\psi_{h,y,v}}$.

%where $y = h(x_0) \in \bit^{n}$ is the image and where $\sigma_X^{z}$ is the reduced state %\james{Isn't the reduced state just $\ket{\vec x_0}$? If so, we wouldn't have to sample $\vec z$ in the above game, and we can just give $\cA_0$ the vector $\vec x_0$ (which directly corresponds to targeted collision-resistance)} 
%with respect to
%$$
%\sigma_{CX} = \frac{1}{2} \sum_{c,c' \in \bit} \ketbra{c}{c'}_C \otimes \mathsf{Z}^{c \cdot z}{\proj{x_0}}_X \left(\mathsf{Z}^{c' \cdot z}\right)^\dag.
%$$
\end{claim}


\begin{proof}
This follows directly from the assumed $(\cD,\cM)$-target-collision resistance of $\cH$, since the above probability is exactly $\Pr[\mathsf{TargetCollRes}_{\algo H,\algo A,\algo D,\algo M,\lambda}=1]$.
\end{proof}

%\begin{proof}
%Suppose for the sake of contraction that the probability is at least $1/\poly(\lambda)$. We now show that we can use $\algo A_0$ to break the target-collision-resistance of the hash family $\algo H = \{H_\lambda\}_{\lambda \in \N}$. 

%Our reduction proceeds as follows:
%\begin{enumerate}
 %\item Run $\algo A_0$ to obtain an $m(\lambda)$-qubit quantum state $\rho$ in system $X$.

%\item Measure system $X$ and send the outcome $x_0$ to the challenger.
    
%\item Once the challenger replies with a description of a hash function $h \rand H_\lambda$, send the register $\ket{x_0}$, the image $y = h(x_0)$ as well as a description of $h$ to $\algo A_0$.

%\item When $\algo A_0$ outputs $(\pi, \rho_{\aux})$, discard $\rho_{\aux}$ and output $(x_0,\pi)$.
%\end{enumerate}
%Notice that the state $\ket{x_0}$ which is sent to $\algo A_0$ is identical to the reduced state $\sigma_X$ with respect to $\sigma_{CX}$.
%By assumption, $\algo A_0$ outputs a valid certificate $\pi \neq x_0$ such that $h(\pi) = h(x_0)$ with probability at least $1/\poly(\lambda)$. Thus, we have broken the target-collision-resistance of $\algo H$.
%\end{proof}

\begin{claim}\label{claim:measurement-succeeds-wp-1}
The probability that the challenger accepts the deletion certificate $\pi$ in Step $4$ of $\mathsf{Exp}_3(b)$ and
the subsequent projective measurement on system $C$ fails (returns the second outcome) is negligible.
\end{claim}
\begin{proof}
This follows directly from \Cref{claim:identical-certificate}, which implies that except with negligible probability, the register $C$ is in the state
\[\frac{1}{\sqrt{2}}\left(\ket{0} + (-1)^{\langle v,z\rangle}\ket{1}\right)\] at the time the challenger applies the projective measurement.

%which implies that in the case the certificate $\pi$ returned by the adversary in
%$\mathsf{Exp}_3$ is identical to the pre-image produced by the challenger with all but negligible probability.
%Therefore, the projective measurement must also succeed with overwhelming probability.
\end{proof}

For any experiment $\Exp_i(b)$, we define the advantage \[\mathsf{Adv}(\mathsf{\Exp}_i) \coloneqq |\Pr\left[\Exp_i(0) = 1\right]-\Pr\left[\Exp_i(1)=1\right]|.\]

\begin{claim}
$$
\mathsf{Adv}(\mathsf{Exp}_2) = 0.
$$
\end{claim}
\begin{proof}
First note that in the case that the challenger rejects because either the deletion certificate is invalid or their projection fails, the experiment does not involve $b$, and thus the advantage of the adversary is $0$. Second, in the case that the challenger's projection succeeds, the register $C$ is either in the state
$$
\frac{1}{\sqrt{2}}  ( \ket{0} +  (-1)^{\langle \pi, z \rangle } \ket{1}) \quad\,\, \text{ or } \quad\,\,  \frac{1}{\sqrt{2}}  ( \ket{0} -  (-1)^{\langle \pi, z \rangle } \ket{1}) 
$$
for some $z \in \bit^k$, and thereby completely
unentangled from the rest of the system. Notice that the challenger's measurement of system $C$ with outcome $c'$ results in a uniformly random bit, which completely masks $b$. Therefore, the experiment is also independent of $b$ in this case, and thus the adversary's overall advantage in $\mathsf{Exp}_2$ is $0$. 
\end{proof}
Next, we argue the following.

\begin{claim}
$$
|\mathsf{Adv}(\mathsf{Exp}_2) - \mathsf{Adv}(\mathsf{Exp}_1) | \,\leq \, \negl(\lambda).
$$    
\end{claim}
\begin{proof}
Recall that \Cref{claim:measurement-succeeds-wp-1} shows that the projective measurement performed by the challenger in Step $4$ of $\mathsf{Exp}_3$ succeeds with overwhelming probability. We now argue that the same is also true in $\mathsf{Exp}_2$. Suppose for the sake of contradiction that there is a non-negligible difference between the success probabilities of the measurement. 
We now show that this implies the existence of an efficient distinguisher $\algo A'$ that breaks the $(\cD,\cM)$-target-collapsing property of the hash family $\algo H = \{H_\lambda\}_{\lambda \in \N}$. 

$\algo A'$ receives $(y,h)$ and a state on register $X$ from its challenger. Next, it computes $M[h]$ on $X$ into a fresh $k$-qubit register $V$, samples a random string $z \rand \{0,1\}^k$, prepares a $\ket{+}$ state in system $C$, applies a controlled-$\mathsf{Z}^z$ operation from $C$ to $V$, and then uncomputes register $V$ by again applying $M[h]$ from $X$ to $V$. Then, it runs $\algo A$ on $(y,h,X)$, which outputs a certificate $\pi$. 



%Our reduction proceeds as follows:
%The distinguisher $\algo A'$ runs $\algo A_0$ to obtain an $m$-qubit state $\rho$ in system $X$,
%and forwards it to the challenger who responds
%with a description of a hash function $h \rand H_\lambda$, an image $y \in \bit^m$ and a state $\rho_y$ in system $X$
%which is either a partially measured state (consisting of a superposition of pre-images)
%or a single measured pre-image $\proj{x_0}$ such that $x_0 \in \bit^m$ and $h(x_0)=y$.
%Next, $\algo D$ samples a random string $z \rand \bit^m$ and runs $\algo A_0$ given as input system $X$ of the state
%$$
%\sigma_{CX} = \frac{1}{2} \sum_{c,c' \in \bit} \ketbra{c}{c'}_C \otimes \mathsf{Z}^{c \cdot z}{\rho_y}_X \left(\mathsf{Z}^{c' \cdot z}\right)^\dag.
%$$

Finally, $\algo A'$ applies the following projective measurement to system $C$:
$$
\Big\{\proj{\phi_\pi^{z}},I - \proj{\phi_\pi^{z}}\Big\} \,\quad \text{ where } \quad
\ket{\phi_\pi^{z}} \coloneqq \frac{1}{\sqrt{2}}  \left( \ket{0} +  (-1)^{\langle \pi, z \rangle } \ket{1}\right),
$$
and outputs $1$ if the measurement succeeds and $0$ otherwise.
If there is a non-negligible difference in success probabilities of this measurement between $\Exp_3(b)$ and $\Exp_2(b)$ (for any $b \in \{0,1\}$), then $\cA'$ breaks $(\cD,\cM)$-target-collapsing of $\cH$.


%between the case when $\rho_y$ is a superposition of pre-images, or $\rho_y = \proj{x_0}$ is a single pre-image of $y$, this immediately breaks the targeted-collapsing property of $\algo H$. Therefore, the projective measurement in Step $5$ of $\mathsf{Exp}_2$ must also succeed with overwhelming probability.

Now, recall that $\mathsf{Exp}_2(b)$ is identical to $\mathsf{Exp}_1(b)$, except that the challenger applies an additional a measurement in Step 4. Because the measurement succeeds with overwhelming probability, it follows from Gentle Measurement that the advantage of the adversary must remain the same up to a negligible amount. This proves the claim.
\end{proof}

%Suppose that $\algo A = (\algo A_0, \algo A_1)$ wins at $\mathsf{Exp}_0$ with probability $\epsilon >0$. We now show the following:

%\james{need to update this claim still}

\begin{claim}\label{claim:exp_0-exp_1}
\[\mathsf{Adv}(\Exp_1) = \mathsf{Adv}(\Exp_0)/2.\]   
\end{claim}
\begin{proof}


First note that in $\Exp_1(b)$, we can imagine measuring register $C$ to obtain $c'$ and aborting if $c' \neq b$ before the challenger sends any information to the adversary. This follows because register $C$ is disjoint from the adversary's registers. Next, by \Cref{lem:random-Z}, we have the following guarantees about the state on system $X$ given to the adversary in $\Exp_1(b)$. 

\begin{itemize}
    \item In the case $c' = b = 0$, the reduced state on register $X$ is $\ket{\psi_{h,y}}$.
    \item In the case that $c' = b = 1$, the reduced state on register $X$ is a mixture over $\ket{\psi_{h,y,v}}$ where $v$ is the result of measuring register $V$ in the computational basis.
\end{itemize}

Thus, this experiment is identical to $\Exp_0(b)$, except that we decide to abort and output a uniformly random bit $b'$ with probability 1/2 at the beginning of the experiment.

\end{proof}


%in the case $c' = b = 0$, the reduced state on register $X$ is $\ket{\psi_{}}$


%Note that the reduced state on register $X$ of \[\sum_{c \in \{0,1\}} \ket{c}_C \otimes \sum_{x \in \{0,1\}^m: h(x)=y} \sqrt{D(x)}\ket{x}_X\mathsf{Z}^{c \cdot z}\ket{M[h](x)}_V\] in $\Exp_1(b)$ is an equal mixture of a state where $V$ was unmeasured and a state where $V$ was measured in the computational basis, which follows from \Cref{lem:random-Z}.

%Notice that the reduced state $\sigma_X$ in $\mathsf{Exp}_1$ is an equal mixture
%$$
%\sigma_X = \frac{1}{2} {\rho_y} + \frac{1}{2} \,  \underset{z}{\mathbb{E}} \left[\mathsf{Z}^{z}{\rho_y} \left(\mathsf{Z}^{ z}\right)^\dag \right].
%$$
%From \Cref{lem:random-Z} it follows that a random Pauli-$Z$ twirl induces a measurement in the computational basis. In other words, on average over the choice of $z \in \bit^m$, we have
%\begin{align*}
%\underset{z}{\mathbb{E}} \left[\mathsf{Z}^{z}{\rho_y} \left(\mathsf{Z}^{ z}\right)^\dag \right] & = \sum_{x \in \bit^m} \Tr[\ketbra{x}{x} \rho_y] \,\ketbra{x}{x}
%\end{align*}
%Therefore, the reduced state $\sigma_X$ in $\mathsf{Exp}_1$ precisely matches the state in $\mathsf{Exp}_0$:
%with probability $1/2$, $\rho_y$ corresponds to a superposition of pre-images of $y$ and, with probability $1/2$, $\rho_y$ is equal to $\proj{x_0}$, i.e. it corresponds to a single (measured) pre-image $x_0$ such that $h(x_0) = y$.

%Then, we observe that $c'$ in $\mathsf{Exp}_1(b)$ equals $b$ with probability $\frac{1}{2}$. Letting $\Pr[\mathsf{Exp}_0(b)=1] = \epsilon$, we get
%\begin{align*}
%\big|\mathsf{Adv}(\mathsf{Exp}_1) - \mathsf{Adv}(\mathsf{Exp}_0) \big|
%=\big| \frac{\epsilon}{2} - \epsilon \big| = \frac{\epsilon}{2}. 
%\end{align*}
%\end{proof}
%Putting everything together, we get that %$\frac{\epsilon}{2} \leq \negl(\lambda)$, %and thus $\epsilon = %\mathsf{Adv}(\mathsf{Exp}_0) \leq %\negl(\lambda)$, which completes the proof.

Putting everything together, we have that $\mathsf{Adv}(\Exp_0) \leq \negl(\secp)$, which completes the proof.

\end{proof}

\subsection{Auxiliary Information}
\label{sec:mainaux}


Next, we generalize the above theorem statement to handle hash functions that are sampled with some auxiliary information. That is, there is an algorithm $(h,\aux) \gets \Samp(1^\secp)$ that samples the description of a hash function $h$ along with some auxiliary information $\aux$. We will want to allow the adversary to potentially see information about $\aux$ (but not necessarily all of it), so we define a family $\cZ = \{Z_\secp(\aux)\}_{\secp \in \bbN}$ that specifies what information the adversary sees about $\aux$. In the most straightforward case, $\cZ$ could be some distribution over classical or quantum states, parameterized by $\aux$. However, we also consider an \emph{interactive} $Z_\secp(\aux)$. That is, $Z_\secp$ is the description of an interactive machine that is initialized with $\aux$ and interacts with the adversary $\cA_\secp$.

%Now, we consider a variant on our definitions of target-collapsing and target-collision-resistance that will be useful for our hybrid-encryption based generic compiler in \cref{sec:regularOWF}. Instead of measuring the $Y$ register to obtain a single image $y$, the challenger will ``coherently'' measure $Y$ by copying it into a new register $Y'$. Then, it will \emph{encrypt} the register $Y'$ using some encryption scheme $\cZ$, and send this encrypted register to the adversary as part of its challenge. 

%We will consider families of machines $\cZ = \{Z_\secp(Y')\}_{\secp \in \bbN}$ that operate on a register $Y'$. In the most straightforward case, $Z_\secp(Y')$ will output some state on register $Y''$ that is sent to the adversary. However, we will also consider \emph{interactive} $Z_\secp(Y')$. That is, $Z_\secp$ may be the description of an interactive machine that is initialized with a state on register $Y'$, and interacts with the adversary $\cA$. Our formal definitions follow.

%\begin{definition}
%A hash function family $\cH = \{H_\secp : \{0,1\}^{m(\secp)} \to \{0,1\}^{n(\secp)}\}_{\secp \in \bbN}$ is $(\cD,\cM,\cZ)$-target-collapsing for some distribution $\cD = \{D_\secp\}_{\secp \in \bbN}$ over $\{\{0,1\}^{m(\secp)}\}_{\secp \in \bbN}$, family of functions $\cM = \{\{M[h] : \{0,1\}^{m(\secp)} \to \{0,1\}^{k(\secp)}\}_{h \in H_\secp}\}_{\secp \in \bbN}$, and family of (static or interactive) machines $\cZ = \{Z_\secp\}_{\secp \in \bbN}$ if, for every QPT adversary $\cA = \{\cA_\secp\}_{\secp \in \bbN}$,
%$$
%|
%\Pr[ \mathsf{TargetCollapseExp}_{\algo H,\algo A,\algo D,\algo M,\algo Z,\lambda}(0)=1] - \Pr[ \mathsf{TargetCollapseExp}_{\algo H,\algo A,\algo D,\algo M,\algo Z\lambda}(1)=1]
%| \leq \negl(\lambda),
%$$
%where the experiment $\mathsf{TargetCollapseExp}_{\algo H,\algo A,\algo D,\algo F,\algo Z,\lambda}(b)$ is defined as follows.

%\begin{enumerate}

    %\item The challenger prepares the state \[\sum_{x \in \{0,1\}^{m(\secp)}}\sqrt{D_\secp(x)}\ket{x}_X\] on register $X$, and samples a random hash function $h \rand H_\lambda$. Then, it coherently computes $h$ on $X$ twice into fresh $n(\secp)$-qubit registers $Y$ and $Y'$, producing the state
    
    %\[\sum_{x \in \{0,1\}^{m(\secp)}}\sqrt{D_\secp(x)}\ket{x}_X\ket{h(x)}_Y\ket{h(x)}_{Y'}.\]
    
    %and measures system $Y$ in the computational basis, which results in an outcome $y \in \bit^{n(\lambda)}$.
    %\item If $b=0$, the challenger does nothing. Else, if $b=1$, the challenger coherently computes $M[h]$ on $X$ into a fresh $k(\secp)$-qubit register $V$ and measures system $V$ in the computational basis. Then, the challenger sends the state on systems $(X,Y)$ to $\algo A$, together with a description of the hash function $h$.
    %\item The challenger initializes $Z_\secp(Y')$. In the case that $Z_\secp(Y')$ is static, the challenger additionally sends its output register $Y''$ to $\algo A$, and in the case that $\cZ_\secp(Y')$ is interactive, the adversary $\algo A$ additionally gets to interact with $\cZ_\secp(Y')$.
    %\item $\algo A$ returns a bit $b'$, which we define as the output of the experiment.
%\end{enumerate}


%in \cref{def:target-collapsing} except that $h$ is sampled by $(h,\aux) \gets \Samp(1^\secp)$, and the adversary is given (or interacts with) $Z_\secp(\aux)$ along with $(X,y,h)$.

%\end{definition}

\begin{definition}
A hash function family $\cH = \{H_\secp : \{0,1\}^{m(\secp)} \to \{0,1\}^{n(\secp)}\}_{\secp \in \bbN}$ with an associated sampling algorithm $\Samp$ is $(\cD,\cM,\cZ)$-target-collapsing for some distribution $\cD = \{D_\secp\}_{\secp \in \bbN}$ over $\{\{0,1\}^{m(\secp)}\}_{\secp \in \bbN}$, family of functions $\cM = \{\{M[h] : \{0,1\}^{m(\secp)} \to \{0,1\}^{k(\secp)}\}_{h \in H_\secp}\}_{\secp \in \bbN}$, and family of (static or interactive) distributions $\cZ = \{Z_\secp(\aux)\}_{(\cdot,\aux) \in \Samp(1^\secp), \secp \in \bbN}$ if, for every QPT adversary $\cA = \{\cA_\secp\}_{\secp \in \bbN}$,
$$
|
\Pr[ \mathsf{TargetCollapseExp}_{\algo H,\algo A,\algo D,\algo M,\algo Z,\lambda}(0)=1] - \Pr[ \mathsf{TargetCollapseExp}_{\algo H,\algo A,\algo D,\algo M,\algo Z,\lambda}(1)=1]
| \leq \negl(\lambda),
$$
where the experiment $\mathsf{TargetCollapseExp}_{\algo H,\algo A,\algo D,\algo M,\algo Z,\lambda}(b)$ is defined as in \cref{def:target-collapsing} except that $h$ is sampled by $(h,\aux) \gets \Samp(1^\secp)$, and the adversary is given (or interacts with) $Z_\secp(\aux)$ along with $(X,y,h)$.

\end{definition}

\begin{definition}
A hash function family $\cH = \{H_\secp : \{0,1\}^{m(\secp)} \to \{0,1\}^{n(\secp)}\}_{\secp \in \bbN}$ with an associated sampling algorithm $\Samp$ is $(\cD,\cM,\cZ)$-target-collision-resistant for some distribution $\cD = \{D_\secp\}_{\secp \in \bbN}$ over $\{\{0,1\}^{m(\secp)}\}_{\secp \in \bbN}$, family of functions $\cM = \{\{M[h] : \{0,1\}^{m(\secp)} \to \{0,1\}^{k(\secp)}\}_{h \in H_\secp}\}_{\secp \in \bbN}$, and family of (static or interactive) distributions $\cZ = \{Z_\secp(\aux)\}_{(\cdot,\aux) \in \Samp(1^\secp), \secp \in \bbN}$ if, for every QPT adversary $\cA = \{\cA_\secp\}_{\secp \in \bbN}$,
$$
\Pr[ \mathsf{TargetCollRes}_{\algo H,\algo A,\algo D,\algo M,\algo Z,\lambda}(0)=1] \leq \negl(\secp),
$$
where the experiment $\mathsf{TargetCollRes}_{\algo H,\algo A,\algo D,\algo M,\algo Z,\lambda}(b)$ is defined as in \cref{def:target-CR} except that $h$ is sampled by $(h,\aux) \gets \Samp(1^\secp)$, and the adversary is given (or interacts with) $Z_\secp(\aux)$ along with $(X,y,h)$.

\end{definition}

\begin{definition}
A hash function family $\cH = \{H_\secp : \{0,1\}^{m(\secp)} \to \{0,1\}^{n(\secp)}\}_{\secp \in \bbN}$ with an associated sampling algorithm $\Samp$ is certified everlasting $(\cD,\cM,\cZ)$-target-collapsing for some distribution $\cD = \{D_\secp\}_{\secp \in \bbN}$ over $\{\{0,1\}^{m(\secp)}\}_{\secp \in \bbN}$, family of functions $\cM = \{\{M[h] : \{0,1\}^{m(\secp)} \to \{0,1\}^{k(\secp)}\}_{h \in H_\secp}\}_{\secp \in \bbN}$, and family of (static or interactive) distributions $\cZ = \{Z_\secp(\aux)\}_{(\cdot,\aux) \in \Samp(1^\secp), \secp \in \bbN}$ if, for every two-part adversary $\algo A = \{\algo A_{0,\secp},\algo A_{1,\secp}\}_{\secp \in \bbN}$, where $\{\algo A_{0,\secp}\}_{\secp \in \bbN}$ is QPT and $\{\algo A_{1,\secp}\}_{\secp \in \bbN}$ is unbounded, it holds that 
$$
|\Pr[ \mathsf{EvTargetCollapseExp}_{\algo H,\algo A,\algo D,\algo M,\algo Z,\lambda}(0)=1]-\Pr[ \mathsf{EvTargetCollapseExp}_{\algo H,\algo A,\algo D,\algo M,\algo Z,\lambda}(1)=1]| \leq \negl(\secp),
$$
where the experiment $\mathsf{EvTargetCollapseExp}_{\algo H,\algo A,\algo D,\algo M,\algo Z,\lambda}(b)$ is defined as in \cref{def:target-CR} except that $h$ is sampled by $(h,\aux) \gets \Samp(1^\secp)$, and the first part of the adversary $\algo A_{0,\secp}$ is given (or interacts with) $Z_\secp(\aux)$ along with $(X,y,h)$.

\end{definition}


Now, the following generalization of \cref{thm:CETC-generalization} follows immediately from the proof of \cref{thm:CETC-generalization}, by additionally giving $Z_\secp(\aux)$ to the adversary in each of the experiments.

\begin{theorem}
Let $\cH = \{H_\secp\}_{\secp \in \bbN}$ be a hash function family that is both $(\cD,\cM,\cZ)$-target-collapsing and $(\cD,\cM,\cZ)$-target-collision-resistant, for some distribution $\cD$, efficiently computable family of functions $\cM$, and (static or interactive) distribution $\cZ$. Then, $\cH$ is certified everlasting $(\cD,\cM,\cZ)$-target-collapsing.
\end{theorem}


\subsection{Target-Collision-Resistance implies Target-Collapsing for Polynomial-Outcome Measurements}
\label{sec:tcr-implies}
%\dakshita{should we say that the two notions are equivalent? is that true?}\james{I thought we had some potential counterexamples that were (partial) target-collapsing but not (partial) target-collision-resistant. In any case, I don't know of a proof in the reverse direction.}

%\begin{lemma}\label{thm:TCR-from-TC}
%Let $\cH = \{H_\secp : \{0,1\}^{m(\secp)}\to \{0,1\}^{n(\secp)}\}_{\secp \in \bbN}$ be a hash function family that is $(\cD,\cM,\cZ)$-target-collapsing for some distribution $\cD = \{D_\secp\}_{\secp \in \bbN}$ over $\{\{0,1\}^{m(\secp)}\}_{\secp \in \bbN}$, family of \emph{single-bit output }functions $\cM = \{\{M[h] : \{0,1\}^{m(\secp)} \to \{0,1\}\}_{h \in H_\secp}\}_{\secp \in \bbN}$, and family of (static or interactive) distributions $\cZ = \{Z_\secp(\aux)\}_{(\cdot,\aux) \in \Samp(1^\secp), \secp \in \bbN}$. Then, $\cH$ is $(\cD,\cM,\cZ)$-target-collision-resistant.
%\end{lemma}

%\begin{proof}
%We will make use of the following claim.

%\begin{claim}
%Let $\Pi_0,\Pi_1$ be orthogonal projectors and let $\ket{\psi}$ be any state such that $\ket{\psi} \in \mathsf{im}(\Pi_0 + \Pi_1)$. Then for any unitary $U$, there exists a projector $D$ such that 

%\[\big| \|D\ket{\psi}\|^2 - \left(\|D\Pi_0\ket{\psi}\|^2 + \|D\Pi_1\ket{\psi}\|^2\right)\big| \geq \frac{1}{2}\left(\|\Pi_1 U \Pi_0 \ket{\psi}\|^2 + \| \Pi_0 U \Pi_1 \ket{\psi} \|^2\right).\]
%\end{claim}

%\begin{proof}
%Define $D$ to take as input register $X$, prepare a $\ket{+}$ state on register $P$, apply controlled $U$ from $P$ to $X$, and then measure $P$ in the $\{\ket{+},\ket{-}\}$ basis and accept if $\ket{+}$ is observed. First, define \[\ket{+_\psi} \coloneqq \ket{\psi} = \Pi_0\ket{\psi} + \Pi_1\ket{\psi}, ~~ \ket{-_\psi} \coloneqq \Pi_0\ket{\psi} - \Pi_1\ket{\psi},\] and note that

%\begin{align*}
    %\big|\|&D\ket{\psi}\|^2 - \left(\|D\Pi_0\ket{\psi}\|^2 + \|D\Pi_1\ket{\psi}\|^2\right)\big| \\
    %&= \big| \bra{+_\psi}D\ket{+_\psi} - \frac{1}{4}\left(\left(\bra{+_\psi} + \bra{-_\psi}\right)D\left(\ket{+_\psi} + \ket{-_\psi}\right)\right) - \frac{1}{4}\left(\left(\bra{+_\psi} - \bra{-_\psi}\right)D\left(\ket{+_\psi} - \ket{-_\psi}\right)\right)\big| \\
    %&= \frac{1}{2}\big|\bra{+_\psi}D\ket{+_\psi} - \bra{-_\psi}D\ket{-_\psi}\big|
%\end{align*}




%Then,

%\begin{align*}
    %&\bra{+_\psi}D\ket{+_\psi} - \bra{-_\psi}D\ket{-_\psi} \\
    %&= \bigg\| \dyad{+}{+}\left(\frac{1}{\sqrt{2}}\ket{0}\left(\Pi_0\ket{\psi} + \Pi_1\ket{\psi}\right) + \frac{1}{\sqrt{2}}\ket{1}\left(U\Pi_0\ket{\psi} + U\Pi_1\ket{\psi}\right)\right) \bigg\|^2 \\
    %& ~~ - \bigg\| \dyad{+}{+}\left(\frac{1}{\sqrt{2}}\ket{0}\left(\Pi_0\ket{\psi} - \Pi_1\ket{\psi}\right) + \frac{1}{\sqrt{2}}\ket{1}\left(U\Pi_0\ket{\psi} - U\Pi_1\ket{\psi}\right)\right) \bigg\|^2 \\
    %&= \frac{1}{4}\left(\left(\bra{\psi}\Pi_0 + \bra{\psi}\Pi_1\right)\left(U\Pi_0\ket{\psi} + U\Pi_1\ket{\psi}\right) + \left(\bra{\psi}\Pi_0U^\dagger + \bra{\psi}\Pi_1U^\dagger\right)\left(\Pi_0\ket{\psi} + \Pi_1\ket{\psi}\right)\right) \\
    %& ~~ - \frac{1}{4}\left(\left(\bra{\psi}\Pi_0 - \bra{\psi}\Pi_1\right)\left(U\Pi_0\ket{\psi} - U\Pi_1\ket{\psi}\right) + \left(\bra{\psi}\Pi_0U^\dagger - \bra{\psi}\Pi_1U^\dagger\right)\left(\Pi_0\ket{\psi} - \Pi_1\ket{\psi}\right)\right) \\
    %&= \frac{1}{2}\left(\bra{\psi}\Pi_0U\Pi_1\ket{\psi} + \bra{\psi}\Pi_1U\Pi_0\ket{\psi} + \bra{\psi}\Pi_0U^\dagger\Pi_1\ket{\psi} + \bra{\psi}\Pi_1U^\dagger\Pi_0\ket{\psi} \right)
%\end{align*}

%\james{stuck here}


%\end{proof}

%\end{proof}

In this section, we show that recent techniques from the collapsing hash function / collapsing commitment literature \cite{cryptoeprint:2022/786,crypto-2022-32202,crypto-2022-32124} imply that when $\cM$ is a function with polynomial number of outcomes, then $(\cD,\cM,\cZ)$-target-collision-resistance implies $(\cD,\cM,\cZ)$-target-collapsing. In this paper, we will only need to use this claim for \emph{two-outcome} measurements, but we show it for the more general case of polynomial-outcome measurements.

\begin{lemma}\label{thm:TC-from-TCR}
Let $\cH = \{H_\secp : \{0,1\}^{m(\secp)}\to \{0,1\}^{n(\secp)}\}_{\secp \in \bbN}$ be a hash function family that is $(\cD,\cM,\cZ)$-target-collision-resistant for some distribution $\cD = \{D_\secp\}_{\secp \in \bbN}$ over $\{\{0,1\}^{m(\secp)}\}_{\secp \in \bbN}$, family of functions $\cM = \{\{M[h] : \{0,1\}^{m(\secp)} \to \{0,1\}^{k(\secp)}\}_{h \in H_\secp}\}_{\secp \in \bbN}$ for $k(\secp)=O(\log \secp)$, and family of (static or interactive) distributions $\cZ = \{Z_\secp(\aux)\}_{(\cdot,\aux) \in \Samp(1^\secp), \secp \in \bbN}$. Then, $\cH$ is $(\cD,\cM,\cZ)$-target-collapsing.
\end{lemma}

\begin{proof}
We will make use of the following fact \cite[Claim 3.5]{cryptoeprint:2022/786}.

\begin{fact}\label{fact:distinguish-map}
Let $D$ be a projector, $\{\Pi_i\}_{i \in [N]}$ be pairwise orthogonal projectors, and $\ket{\psi}$ be any state such that $\ket{\psi} \in \mathsf{im}(\sum_{i \in [N]}\Pi_i)$. Then,
    \[\sum_{i \in [N]}\bigg\| \left(\sum_{j \neq i}\Pi_j\right)D\Pi_i\ket{\psi}\bigg\|^2  \geq \frac{1}{N}\left(\|D\ket{\psi}\|^2-\left(\sum_{i \in [N]}\|D\Pi_i\ket{\psi}\|^2\right)\right)^2.\]
\end{fact}

Now, suppose there exists an adversary $\{\cA_\secp\}_{\secp \in \bbN}$ that breaks the $(\cD,\cM,\cZ)$-target-collapsing of $\cH$. Dropping parameterization by $\secp$ for convenience, we can write such an adversary as a binary outcome projective measurement $(D,I-D)$ applied to a state received from the challenger. For any $h \in H_\secp, y \in \{0,1\}^n$, let $\ket{\psi_{h,y}}$ be the normalized state such that \[\ket{\psi_{h,y}} \propto \ket{h,y}\otimes\sum_{x \in \{0,1\}^m: h(x)=y}\sqrt{D(x)}\ket{x},\] and for $i \in \{0,1\}^k$, let \[\Pi_{i,h} \coloneqq \sum_{x \in \{0,1\}^m : M[h](x) = i}\dyad{x}{x}.\]

Then, the adversary's advantage in the $(\cD,\cM,\cZ)$-target-collapsing game can be written as 

\[\E_{h,y}\left[\|D\ket{\psi_{h,y}}\|^2 - \sum_{i \in \{0,1\}^k}\|D\Pi_{i,h}\ket{\psi_{h,y}}\|^2 \right] = \nonnegl(\secp),\] where the expectation is over the sampling of $h \gets H_\secp$ and the challenger's measurement of $y$.

Thus, by \cref{fact:distinguish-map}, it follows that

\[\E_{h,y}\left[\sum_{i \in \{0,1\}^k}\bigg\|\left(\sum_{j \neq i}\Pi_{j,h}\right)D\Pi_{i,h}\ket{\psi_{h,y}}\bigg\|^2\right] = \nonnegl(\secp),\] since $2^k = 2^{O(\log \secp)} = \poly(\secp)$. This completes the proof, as this expression exactly corresponds to the adversary's probability of winning the $(\cD,\cM,\cZ)$-target-collision-resistance game by applying $D$ and then measuring in the computational basis.

\end{proof}


%\ifsubmission
%\paragraph{Conclusion.}
%We refer the reader to Section \ref{sec:overreq} for an overview of how target-collapsing implies publicly-verifiable deletion.
%Due to lack of space, we defer the formal proof of this fact, and also our instantiations of target-collapsing functions from LWE/SIS following~\cite{Poremba22}, from almost-regular one-way functions as well as from pseudorandom group actions~\cite{HMY} to the appendices. 
%\else \fi






\section{Publicly-Verifiable Deletion from Dual-Regev Encryption}\label{sec:Dual-Regev}

In this section, we recall the constructions of Dual-Regev public-key encryption as well as fully homomorphic encryption with publicly-verifiable deletion introduced by Poremba~\cite{Poremba22}. Using our main result on certified-everlasting target-collapsing hashes in \Cref{thm:CETC-generalization}, we prove the \emph{strong Gaussian-collapsing conjecture} in~\cite{Poremba22}, and then conclude that the aforementioned constructions achieve certified-everlasting security assuming the quantum hardness of $\LWE$ and $\SIS$.

First, let us recall the definition of public-key encryption with publicly-verifiable deletion. 

\subsection{Definition: Encryption with Publicly-Verifiable Deletion}

A public-key encryption (PKE) scheme with publicly-verifiable deletion (PVD) has the following syntax.

\begin{itemize}
    \item $\KeyGen(1^\secp) \to (\pk,\sk)$: the key generation algorithm takes as input the security parameter $\secp$ and outputs a public key $\pk$ and secret key $\sk$.
    \item $\Enc(\pk,m) \to (\vk,\ket{\ct})$: the encryption algorithm takes as input the public key $\pk$ and a plaintext $m$, and outputs a (public) verification key $\vk$ and a ciphertext $\ket{\ct}$.
    \item $\Dec(\sk,\ket{\ct}) \to m$: the decryption algorithm takes as input the secret key $\sk$ and a ciphertext $\ket{\ct}$ and outputs a plaintext $m$.
    \item $\Del(\ket{\ct}) \to \pi$: the deletion algorithm takes as input a ciphertext $\ket{\ct}$ and outputs a deletion certificate $\pi$.
    \item $\Vrfy(\vk,\pi) \to \{\top,\bot\}$: the verify algorithm takes as input a (public) verification key $\vk$ and a proof $\pi$, and outputs $\top$ or $\bot$.
\end{itemize}

\begin{definition}[Correctness of deletion]\label{def:correctness-deletion}
A PKE scheme with PVD satisfies \emph{correctness of deletion} if for any $m$, it holds with $1-\negl(\secp)$ probability over $(\pk,\sk) \gets \Gen(1^\secp), (\vk,\ket{\ct}) \gets \Enc(\pk,m),\pi \gets \Del(\ket{\ct}),\mu \gets \Vrfy(\vk,\pi)$ that $\mu = \top$.
\end{definition}

\begin{definition}[Certified deletion security]\label{def:security-deletion}
A PKE scheme with PVD satisfies \emph{certified deletion security} if it satisfies standard semantic security, and moreover, for any QPT adversary $\{\cA_\secp\}_{\secp \in \bbN}$, it holds that 
\[\TD\left(\mathsf{EvPKE}_{\cA,\secp}(0),\mathsf{EvPKE}_{\cA,\secp}(1)\right) = \negl(\secp),\] where the experiment $\mathsf{EvPKE}_{\cA,\secp}(b)$ is defined as follows.
\begin{itemize}
    \item Sample $(\pk,\sk) \gets \Gen(1^\secp)$ and $(\vk,\ket{\ct}) \gets \Enc(\pk,b)$.
    \item Run $\cA_\secp(\pk,\vk,\ket{\ct})$, and parse their output as a deletion certificate $\pi$ and a left-over quantum state $\rho$.
    \item If $\Vrfy(\vk,\pi) = \top$, output $\rho$, and otherwise output $\bot$.
\end{itemize}
\end{definition}

\ \\
Before we introduce the
Dual-Regev public-key schemes proposed by Poremba~\cite{Poremba22}, let us first recall some basic facts about
Gaussian superpositions.

\subsection{Gaussian 
Superpositions}

Let $m \in \N$. The \emph{Gaussian measure} $\rho_\sigma$ with parameter $\sigma > 0$ is defined as
\begin{align*}
\rho_\sigma(\vec x) = \exp(-\pi \|\vec x \|^2/ \sigma^2), \quad \,\, \forall \vec x \in \mathbb{R}^m.   
\end{align*}
Given a modulus $q \in \N$ and $\sigma \in (\sqrt{2m},q/\sqrt{2m})$, the \emph{truncated} discrete Gaussian distribution $D_{\Z_q^m,\sigma}$ over the finite set $\Z^m \cap (-\frac{q}{2},\frac{q}{2}]^m$ with support $\{\vec x \in \Z_q^m : \|\vec x\| \leq \sigma \sqrt{m}\}$ is defined as
$$
D_{\Z_q^m,\sigma}(\vec x) = \frac{\rho_\sigma(\vec x)}{\displaystyle\sum_{\vec z \in \Z_q^m,\|\vec z\| \leq \sigma\sqrt{m} } \rho_\sigma(\vec z)}.
$$
In this section, we consider
Gaussian superposition states over $\Z^m \cap (-\frac{q}{2},\frac{q}{2}]^m$ of the form
    $$
 \ket{\psi} =    \sum_{\vec x \in \Z_q^m} \rho_\sigma(\vec x) \ket{\vec x}.
    $$
The state $\ket{\psi}$ is not normalized for convenience. A standard tail bound~\cite[Lemma 1.5 (ii)]{Banaszczyk1993} implies that (the normalized variant of) $\ket{\psi}$ is within negligible trace distance of a \emph{truncated} discrete Gaussian superposition $ \ket{\tilde{\psi}}$
with support $\{\vec x \in \Z_q^m : \|\vec x\| \leq \sigma \sqrt{\frac{m}{2}}\}$, where
$$
\ket{\tilde{\psi}}
= 
\left(\sum_{\vec z \in \Z_q^m,\|\vec z\| \leq \sigma \sqrt{\frac{m}{2}} } \rho_{\frac{\sigma}{\sqrt{2}}}(\vec z) \right)^{-\frac{1}{2}}\sum_{\vec x \in \Z_q^m : \|\vec x\| \leq \sigma \sqrt{\frac{m}{2}}}
\rho_\sigma(\vec x) \ket{\vec x}.
$$
Note that a measurement of $\ket{\tilde{\psi}}$ results in a sample from the truncated discrete Gaussian distribution $D_{\Z_q^m,\frac{\sigma}{\sqrt{2}}}$.
We remark that Gaussian superpositions with parameter $\sigma = \Omega(\sqrt{m})$ can be efficiently implemented using standard quantum state preparation techniques; for example using \emph{quantum rejection sampling} and the \emph{Grover-Rudolph algorithm}~\cite{Grover2002CreatingST,Regev05,Brakerski18,brakerski2021cryptographic}.

Let $\vec A \in \Z_q^{n \times m}$. We use the following algorithm, denoted by $\mathsf{GenGauss}(\vec A,\sigma)$ which prepares a partially measured Gaussian superposition of pre-images of a randomly generated image.
\begin{enumerate}
\item Prepare a Gaussian superposition in system $X$ with parameter $\sigma > 0$:
    $$
 \ket{\psi} =    \sum_{\vec x \in \Z_q^m} \rho_\sigma(\vec x) \ket{\vec x} \otimes \ket{\vec 0}.
    $$
\item Apply the unitary $U_{\vec A}: \ket{\vec x}\ket{\vec 0} \rightarrow \ket{\vec x} \ket{\vec A \cdot \vec x \Mod{q}}$, which results in the state
$$
 \ket{ \psi} =   \sum_{\vec x \in \Z_q^m} \rho_\sigma(\vec x) \ket{\vec x} \otimes \ket{\vec A \cdot \vec x \Mod{q}}.
  $$
  \item Measure the second register in the computational basis, which results in $\vec y \in \Z_q^n$ and a state
    $$
    \ket{\psi_{\vec y}} = \sum_{\substack{\vec x \in \Z_q^m:\\ \vec A \vec x= \vec y \Mod{q}}} \rho_\sigma(\vec x) \ket{\vec x}.
    $$
\end{enumerate}

Finally, we use the following lemma which characterizes the Fourier transform of a partially measured Gaussian superposition.

\begin{lemma}[\cite{Poremba22}, Lemma 16]\label{lem:duality}
Let $m \in \N$, $q \geq 2$ be a prime and $\sigma \in (\sqrt{8m},q/\sqrt{8m})$.
Let $\vec A \in \Z_q^{n \times m}$ be a matrix whose columns generate $\Z_q^n$ and let $\vec y \in \Z_q^n$ be arbitrary. Then, the $q$-ary quantum Fourier transform of the (normalized variant of the) Gaussian coset state
$$
 \ket{\psi_{\vec y}} = \sum_{\substack{\vec x \in \Z_q^m\\ \vec A \vec x = \vec y \Mod{q}}}\rho_{\sigma}(\vec x) \ket{\vec x}
$$
is within negligible (in $m \in \N$) trace distance of the (normalized variant of the) Gaussian state
$$
 \ket{\hat\psi_{\vec y}} = \sum_{\vec s \in \Z_q^n} \sum_{\vec e \in \Z_q^m} \rho_{q/\sigma}(\vec e) \, \omega_q^{-\langle\vec s,\vec y \rangle} \ket{\vec s^\intercal \vec A + \vec e^\intercal \Mod{q}}.
$$
\end{lemma}


\subsection{(Strong) Gaussian-Collapsing Property.}

We use the following result which says that the Ajtai hash function is target-collapsing with respect to the truncated discrete Gaussian distribution.

\begin{theorem}[Gaussian-collapsing property, \cite{Poremba22}, Theorem 4]\label{thm:Gauss-collapsing}
Let $n\in \N$ and $q$ be a prime with $m \geq 2n \log q$, each parameterized by $\lambda \in \N$. Let  $\sigma \in (\sqrt{8m},q/\sqrt{8m})$.
Then,
the following samples are computationallyindistinguishable assuming the quantum hardness of decisional $\mathsf{LWE}_{n,q,\alpha q}^m$, for any noise ratio $\alpha \in (0,1)$ with relative noise magnitude $1/\alpha= \sigma \cdot 2^{o(n)}:$
$$
\Bigg(\vec  A \rand \Z_q^{n \times m},\,\, \ket{\psi_{\vec y}}=\sum_{\substack{\vec x \in \Z_q^m\\ \vec A \vec x = \vec y}}\rho_{\sigma}(\vec x) \,\ket{\vec x}, \,\,\vec y\in \Z_q^n \Bigg)\,\, \approx_c \,\,\,\, \Bigg(\vec  A \rand \Z_q^{n \times m}, \,\,\ket{\vec x_0},\,\, \vec A \cdot \vec x_0 \,\in \Z_q^n\Bigg)
$$
where $(\ket{\psi_{\vec y}},\vec y) \leftarrow \mathsf{GenGauss}(\vec A,\sigma)$ and where $\vec x_0 \sim D_{\Z_q^m,\frac{\sigma}{\sqrt{2}}}$ is a discrete Gaussian error.
\end{theorem}

Using our main theorem on certified-everlasting target-collapsing hashes in \Cref{thm:CETC-generalization}, we can now prove a stronger variant of \Cref{thm:Gauss-collapsing}. We show the following:

\begin{theorem}\label{thm:ajtai-certified-everlasting}
Let $\lambda \in \N$ be the security parameter, $n(\lambda) \in \N$, $q(\lambda) \in \N$ be a modulus, $m \geq 2n \log q$ and $\sigma \in (\sqrt{2m},q/\sqrt{2m})$. Then, the Ajtai hash function family
$\algo H = \{H_\lambda\}_{\lambda \in \N}$ with
$$
H_\lambda = \left\{ h_{\vec A}: \big\{\vec x \in \Z_q^m : \|\vec x\| \leq \sigma \sqrt{m/2}\big\} \rightarrow \Z_q^n \, \text{ s.t. } \, h_{\vec A}(\vec x) = \vec A \cdot \vec x \Mod{q}; \, \vec A \in \Z_q^{n \times m} \right\}.
$$
is certified everlasting $D_{\Z_q^m,\frac{\sigma}{\sqrt{2}}}$-target-collapsing assuming the quantum hardness of $\SIS_{n,q,\sigma\sqrt{2m}}^m$ and $\mathsf{LWE}_{n,q,\alpha q}^m$, for any noise ratio $\alpha \in (0,1)$ with relative noise magnitude $1/\alpha= \sigma \cdot 2^{o(n)}$.
\end{theorem}
\begin{proof}
By the Gaussian-collapsing property in \Cref{thm:Gauss-collapsing}, it follows that $\algo H$ is $D_{\Z_q^m,\frac{\sigma}{\sqrt{2}}}$-target-collapsing assuming the quantum hardness of $\mathsf{LWE}_{n,q,\alpha q}^m$, for any noise ratio $\alpha \in (0,1)$ with relative noise magnitude $1/\alpha= \sigma \cdot 2^{o(n)}$. Moreover, from the quantum hardness of $\SIS_{n,q,\sigma\sqrt{2m}}^m$ it follows that $\algo H$ is $D_{\Z_q^m,\frac{\sigma}{\sqrt{2}}}$-target-collision-resistant. Therefore, the claim follows from \Cref{thm:CETC-generalization}.
\end{proof}

As a corollary, we immediately recover the so-called strong Gaussian-collapsing property of the Ajtai hash function which was previously stated as a conjecture by Poremba~\cite{Poremba22}.

\begin{corollary}[Strong Gaussian-collapsing property]\label{SGC}\ \\
Let $\lambda \in \N$ be the security parameter, $n(\lambda) \in \N$, $q(\lambda) \in \N$ be a modulus and $m > 2n \log q$. Let $\sigma = \Omega(\sqrt{m})$ be a parameter. Then, the Ajtai hash function satisfies the strong Gaussian-collapsing property assuming the quantum hardness of $\SIS_{n,q,\sigma\sqrt{2m}}^m$ and $\mathsf{LWE}_{n,q,\alpha q}^m$, for any noise ratio $\alpha \in (0,1)$ with relative noise magnitude $1/\alpha= \sigma \cdot 2^{o(n)}$. In other words, for every $\QPT$ adversary $\algo A$,
$$
|\Pr[\mathsf{StrongGaussCollapseExp}_{\algo A,n,m,q,\sigma}(0)=1] - \Pr[\mathsf{StrongGaussCollapseExp}_{\algo A,n,m,q,\sigma}(1)=1] \leq \negl(\lambda)
$$
where $\mathsf{StrongGaussCollapseExp}_{\algo A,n,m,q,\sigma}(b)$ is the experiment from \Cref{fig:SGC}.
\end{corollary}
\begin{proof} 
To prove the statement, we can simply reduce the certified everlasting $D_{\Z_q^m,\frac{\sigma}{\sqrt{2}}}$-target-collapsing security of the Ajtai hash $\vec A = [\bar{\vec A} \, \| \, \bar{\vec A} \cdot \bar{\vec x} \Mod{q}] \in \Z_q^{n \times m}$ with $\bar{\vec x} \rand \bit^{m-1}$ to the
strong Gaussian-collapsing security, and invoke \Cref{thm:ajtai-certified-everlasting}. Here we rely on the fact that the distribution of $\vec A$ is statistically close to uniform by the leftover hash lemma whenever $m > 2n \log q$. Now consider the unbounded reduction that given $\vec A \in \Z_q^{n \times m}$, samples a uniformly random vector $\vec t = (\vec x,-1) \in \Z^{m}$ with $\vec x \in \bit^{m-1}$ such that $\bar{\vec A} \vec x = \bar{\vec A} \bar{\vec x} \Mod{q}$, and then runs the second part of the strong Gaussian-collapsing adversary on input $\vec t$ in order to predict the challenger's bit. Note that such vectors $\vec t$ exist because of how the matrix $\vec A$ is constructed in the experiment. If the strong Gaussian-collapsing adversary has noticeable advantage, then so does the reduction, which would break the certified everlasting $D_{\Z_q^m,\frac{\sigma}{\sqrt{2}}}$-target-collapsing security of the Ajtai hash. 
\end{proof}


\subsection{Dual-Regev Public-Key Encryption with Publicly-Verifiable Deletion}

We now consider the following Dual-Regev encryption scheme introduced by Poremba~\cite{Poremba22}.

\begin{construction}[Dual-Regev $\mathsf{PKE}$ with Publicly-Verifiable Deletion]\label{cons:dual-regev-cd}
Let $n \in \N$ be the security parameter, $m \in \N$ and $q$ be a prime. Let $\alpha \in (0,1)$ and $\sigma = 1/\alpha$ be parameters.
The Dual-Regev $\mathsf{PKE}$ scheme $\mathsf{DualPKECD} = (\KeyGen,\Enc,\Dec,\Del,\Vrfy)$ with certified deletion is defined as follows:
\begin{description}
\item $\KeyGen(1^\lambda) \rightarrow (\pk,\sk):$ sample a random matrix $\bar{\vec A} \rand \Z_q^{n\times m}$ and a vector $\bar{\vec x} \rand \bit^{m}$
and choose $\vec A = [\bar{\vec A} \| \bar{\vec A} \cdot \bar{\vec x} \Mod{q}]$.
Output $(\pk,\sk)$, where $\pk=\vec A \in \Z_q^{n \times (m+1)}$ and $\sk = (-\bar{\vec x}, 1) \in \Z_q^{m+1}$.
\item $\Enc(\pk,b) \rightarrow (\vk,\ket{\ct})$: parse the public key as $\vec A \leftarrow \pk$. To encrypt a single bit $b \in \bit$, generate the following pair for a random $\vec y \in \Z_q^n$:
$$
\vk \leftarrow (\vec A, \vec y), \quad \ket{\ct} \leftarrow \sum_{\vec s \in \Z_q^n} \sum_{\vec e \in \Z_q^{m+1}} \rho_{q/\sigma}(\vec e) \, \omega_q^{-\langle\vec s,\vec y \rangle}\ket{\vec s^\intercal \vec A + \vec e^\intercal +b \cdot (0,\dots,0, \lfloor\frac{q}{2} \rfloor)},
$$
where $\vk$ is the public verification key and $\ket{\ct}$ is an $(m+1)$-qudit quantum ciphertext.


\item $\Dec(\sk,\ket{\ct}) \rightarrow \bit:$ to decrypt, measure $\ket{\ct}$ in the computational basis with outcome $\vec c \in \Z_q^m$. Compute $\vec c^\intercal \cdot \sk \in \Z_q$ and output $0$, if it is closer to $0$ than to $\lfloor\frac{q}{2}\rfloor$, and output $1$, otherwise.

\item $\Del(\ket{\ct}) \rightarrow \pi:$ Measure $\ket{\ct}$ in the Fourier basis and output the measurement outcome $\pi \in \Z_q^{m+1}$.
\item $\Vrfy(\vk,\pi) \rightarrow \{\top,\bot\}:$ to verify a deletion certificate $\pi \in \Z_q^{m+1}$, parse $(\vec A,\vec y) \leftarrow \vk$ and output $\top$, if $\vec A \cdot \pi = \vec y \Mod{q}$ and $\| \pi \| \leq \sqrt{m+1}/\sqrt{2}\alpha$, and output $\bot$, otherwise.
\end{description}
\end{construction}


%%
Let us now illustrate how the deletion procedure takes place. Recall from \Cref{lem:duality} that the Fourier transform of the ciphertext $\ket{\ct}$ results in the \emph{dual} quantum state
\begin{align}\label{eq:dual-with-phase}
\ket{\widehat{\ct}}=\sum_{\substack{\vec x \in \Z_q^{m+1}:\\ \vec A \vec x = \vec y \Mod{q}}}\rho_{\sigma}(\vec x) \, \omega_q^{\langle\vec x,b \cdot (0,\dots,0,  \lfloor\frac{q}{2} \rfloor)\rangle} \,\ket{\vec x}.
\end{align}
In other words, a Fourier basis measurement of $\ket{\ct}$ necessarily erases all information about the plaintext $b \in \bit$ and results in a \emph{short} vector $\pi \in \Z_q^{m+1}$ such that $\vec A \cdot \pi = \vec y \Mod{q}$. Hence, to publicly verify a deletion certificate we can simply check whether it is a solution to the $\ISIS$ problem specified by the verification key $\vk=(\vec A,\vec y)$. Using \Cref{thm:ajtai-certified-everlasting}, we obtain the following:

\begin{theorem}
Let $n\in \N$ and let $q \geq 2$ be a prime modulus such that $q=2^{o(n)}$ and $m \geq 2n \log q$. Let $\sigma \in (\sqrt{8m},q/\sqrt{8m})$ and $\alpha \in (0,1)$ be a noise ratio with $1/\alpha= 2^{o(n)} \cdot \sigma$.
Then, the Dual-Regev public-key encryption scheme in \Cref{cons:dual-regev-cd} has everlasting certified deletion security assuming the quantum (subexponential) hardness of $\mathsf{LWE}_{n,q,\alpha q}^m$ and $\SIS_{n,q,\sigma\sqrt{2m}}^m$.
\end{theorem}
\begin{proof}
The proof is identical to the template used in~\cite[Theorem 7]{Poremba22}, except that the adversary is allowed to be computationally unbounded once the deletion certificate is submitted. This is in contrast with the original proof who considered forwarding the \emph{secret key} during the security experiment. We remark that we do not invoke the strong Gaussian-collapsing property to prove the indistinguishability of the hybrids; instead we use the (stronger) notion of certified everlasting $D_{\Z_q^m,\frac{\sigma}{\sqrt{2}}}$-target-collapsing property of the Ajtai hash shown in \Cref{thm:ajtai-certified-everlasting}. This results in the stronger notion of everlasting certified
deletion security.
\end{proof}



\subsection{Dual-Regev (Leveled) Fully Homomorphic Encryption with Publicly-Verifiable Deletion}\label{sec:dualfhe-pvd}

A homomorphic encryption scheme with certified deletion~\cite{Broadbent_2020,Poremba22,BBK22} is a scheme that supports both homomorphic operations as well as certified deletion of quantum ciphertexts. Here, the two properties are thought of as \emph{separate} features that may or may not be mutually exclusive. Several works~\cite{Poremba22,BBK22,BGGKMRR} have also considered the possibility of realizing both homomorphic evaluation and certified deletion \emph{simultaneously} within a single (possibly interactive) protocol. For example, Poremba~\cite{Poremba22} proposed a four-message protocol that allows a client to learn the outcome of a homomorphic evaluation performed by an untrusted quantum server, while simultaneously certifying that the server has subsequently deleted all data. Bartusek and Khurana~\cite{BBK22} subsequently defined the notion of a four-message protocol for \emph{blind delegation with certified deletion}, which can be instantiated using any $\FHE$ scheme with certified deletion. Crucially, both of the aforementioned four-message protocols require that the server is \emph{honest} during the evaluation phase of the protocol. Finally, in a subsequent follow-up work, Bartusek et al.~\cite{BGGKMRR} constructed a \emph{maliciously} secure bind delegation protocol with certified deletion which relied on succinct non-interactive arguments (SNARGs) for polynomial-time computation.

In this section, we recall the Dual-Regev (leveled) fully homomorphic encryption scheme with publicly-verifiable deletion introduced by Poremba~\cite{Poremba22}. The scheme is based on the \emph{dual variant} of of the (leveled) $\FHE$ scheme by Gentry, Sahai and Waters~\cite{GSW2013,mahadev2018classical}. 
Using our main result on certified-everlasting target-collapsing hashes in \Cref{thm:CETC-generalization}, we then prove the scheme achieves certified-everlasting security assuming the quantum hardness of $\LWE$ and $\SIS$.
Contrary to related works~\cite{Poremba22,BBK22,BGGKMRR}, we only consider the basic notion of $\FHE$ with publicly-verifiable deletion which treats both properties as separate features.

\paragraph{Parameters.} Let $\lambda \in \N$ be the security parameter and let $n \in \N$. Let $L$ be an upper bound on the depth of the polynomial-sized Boolean circuit which is to be evaluated. We choose the following set of parameters for the Dual-Regev leveled $\FHE$ scheme (each parameterized by $\lambda$).
\begin{itemize}
    \item a prime modulus $q \geq 2$.
    \item an integer $m \geq 2n \log q$.
    \item an integer $N = (m+1) \cdot \lceil \log q \rceil$.
     
     \item a noise ratio $\alpha\in (0,1)$ such that
$$
\sqrt{8(m+1)}\leq \alpha q \leq \frac{q}{\sqrt{8}(m+1)\cdot (N+1)^L}.
$$
\end{itemize}


\begin{construction}[Dual-Regev leveled $\mathsf{FHE}$ scheme with certified deletion]\label{cons:FHE-cd}
Let $\lambda \in \N$ be the security parameter.
The Dual-Regev (leveled) $\mathsf{FHE}$ scheme $\mathsf{DualFHECD} = (\KeyGen,\Enc,\Dec,\Eval,\Del,\Vrfy)$ with certified deletion consists of the following algorithms.
\begin{description}
\item $\KeyGen(1^\lambda,1^L) \rightarrow (\pk,\sk):$ sample $\bar{\vec A} \rand \Z_q^{n\times m}$ and vector $\bar{\vec x} \rand \bit^{m}$
and let $\vec A = [\bar{\vec A} \| \bar{\vec A} \cdot \bar{\vec x} \Mod{q}]^\intercal$.
Output $(\pk,\sk)$, where $\pk=\vec A \in \Z_q^{(m+1) \times n}$ and $\sk = (-\bar{\vec x}, 1) \in \Z_q^{m+1}$.
\item $\Enc(\pk,x) \rightarrow (\vk,\ket{\ct}):$ to encrypt a bit $x\in \bit$, parse the public key as $\vec A \in \Z_q^{(m+1) \times n} \leftarrow \pk$ and generate the following pair consisting of a verification key and ciphertext for a random $\vec Y \in \Z_q^{n \times N}$ with columns $\vec y_1,\dots,\vec y_N \in \Z_q^{n}$:
$$
\vk \leftarrow (\vec A,\vec Y), \quad\,\,
\ket{\ct} \leftarrow \sum_{\vec S \in \Z_q^{n \times N}} \sum_{\vec E \in \Z_q^{(m+1)\times N}} \rho_{q/\sigma}(\vec E) \, \omega_q^{-\Tr[\vec S^\intercal \vec Y]} \ket{\vec A\cdot \vec S + \vec E + x \cdot \vec G},
$$
where $\vec G = [\vec I \, \| \, 2 \vec I \, \| \, \dots \, \| \, 2^{\lceil \log q \rceil -1} \vec I] \in \Z_q^{(m+1) \times N}$  denotes the \emph{gadget matrix} and where $\sigma = 1/\alpha$.

\item $\Eval(\mathsf{C}_0,\mathsf{C}_1) \rightarrow \mathsf{C}_0 \mathsf{C}_1 \mathsf{C}$: to apply a $\mathsf{NAND}$ gate onto two registers $\mathsf{C}_0$ and $\mathsf{C}_1$ (possibly part of a larger ciphertext), append an ancilla system $\ket{\vec 0}_{\mathsf{C}}$, and apply the unitary $U_{\mathsf{NAND}}$, defined by
$$
U_{\mathsf{NAND}}: \quad \ket{\vec X}_{\mathsf{C}_0} \otimes \ket{\vec Y}_{\mathsf{C}_1} \otimes \ket{\vec Z}_\mathsf{C} \quad \rightarrow \quad  \ket{\vec X}_{\mathsf{C}_0} \otimes \ket{\vec Y}_{\mathsf{C}_1} \otimes \ket{\vec Z + \vec G - \vec X \cdot \vec G^{-1}(\vec Y) \Mod{q}}_{\mathsf{C}},
$$
where $\vec X,\vec Y,\vec Z \in \Z_q^{(m+1)\times N}$
and $\vec G^{-1}$ is the (non-linear) inverse operation such that $\vec G \circ \vec G^{-1} = \vec I$.
Output the resulting registers $\mathsf{C}_0 \mathsf{C}_1 \mathsf{C}$.


\item $\Dec(\sk,\mathsf{C}) \rightarrow \bit \, \mathbf{or} \, \bot:$ measure the register $\mathsf{C}$ in the computational basis to obtain $\vec C \in \Z_q^{(m+1)\times N}$ and compute $c = \sk^\intercal \cdot \vec c_N \in \Z \cap (-\frac{q}{2},\frac{q}{2}]$, where $\vec c_N \in \Z_q^{m+1}$ is the $N$-th column of $\vec C$; output $0$, if $c$
is closer to $0$ than to $\lfloor\frac{q}{2}\rfloor$,
and output $1$, otherwise.

\item $\Del(\ket{\ct}) \rightarrow \pi:$ measure $\ket{\ct}$ in the Fourier basis with outcomes $\pi = (\pi_1|\dots|\pi_N) \in \Z_q^{(m+1)\times N}$.

\item $\Vrfy(\vk,\pk,\pi) \rightarrow \bit:$ to verify the deletion certificate $\pi = (\pi_1\|\dots\|\pi_N) \in \Z_q^{(m+1)\times N}$, parse $(\vec A \in \Z_q^{(m+1) \times n},(\vec y_1 \|\dots \|\vec y_N)  \in \Z_q^{n \times N}) \leftarrow \vk$ and output $\top$, if both $\vec A^\intercal \cdot \pi_i = \vec y_i \Mod{q}$ and $\| \pi_i \| \leq \sqrt{m+1}/\sqrt{2}\alpha$ for every $i \in [N]$, and output $\bot$, otherwise.
\end{description}
\end{construction}

For additional details on the correctness of the scheme, we refer to Section $9$ of \cite{Poremba22}.

\begin{theorem}\label{thm:FHE-CD-security} Let $\lambda \in \N$ be the security parameter and let $L$ be an upper bound on the size of the Boolean circuit which is to be evaluated. Let $n \in \N$, let $q\geq 2$ be a prime modulus and let $m \geq 2 n \log q$. Let $N = (m+1) \cdot \lceil \log q \rceil$. Let $\alpha \in (0,1)$ be a noise ratio such that$$
\sqrt{8(m+1)N}\leq \alpha q \leq \frac{q}{\sqrt{8}(m+1)\cdot (N+1)^L}.
$$
Then, $\mathsf{DualFHECD}$ in \Cref{cons:FHE-cd} has everlasting certified deletion security assuming the quantum (subexponential) hardness of $\SIS_{n,q,\sigma\sqrt{2m}}^m$ and $\mathsf{LWE}_{n,q,\alpha q}^m$.
\end{theorem}
\begin{proof}
The proof is identical to the template in~\cite[Theorem 10]{Poremba22}, except that the adversary is allowed to be computationally unbounded once the deletion certificate is submitted. This is in contrast with the original proof who considered forwarding the \emph{secret key} during the security experiment. We remark that we do not invoke the strong Gaussian-collapsing property to prove the indistinguishability of the hybrids; instead we use the (stronger) notion of certified everlasting $D_{\Z_q^m,\frac{\sigma}{\sqrt{2}}}$-target-collapsing property of the Ajtai hash function shown in \Cref{thm:ajtai-certified-everlasting}. This results in the stronger notion of everlasting certified
deletion security.
\end{proof}





%\section{OWF-based certified deletion}

Let $f = \{f_\secp : \{0,1\}^{m(\secp)} \to \{0,1\}^{n(\secp)}\}_{\secp \in \bbN}$ be a regular (\james{hopefully we can relax the regularity requirement somewhat}) one-way function and let $\cA = \{\cA_\secp\}_{\secp \in \bbN}$ be any QPT adversary. \alex{How is regularity defined here? Is it a similar definition as in Mark Zhandry's paper on collapsing hashes?}\james{For now I mean exactly regular, so each element of the image has the same number of preimages. But we can try to relax to either Mark's definition of semi-regularity, or perhaps almost-regularity (the relative size of each set of preimages is within a fixed polynomial ratio).}\james{An additional property we may need is that the range of the OWF (as a subset of the codomain) is somehow efficiently describable, so we can restrict the linear map to be only applied to elements in the range. Otherwise, it may be the case that for many pairs of vectors in the domain of the linear map, only 0 or 1 of them are valid images of the OWF. But maybe this can be generically obtained by compressing the image with a universal hash function.  }

Consider the following experiment $\Exp_{f,\cA,\secp}(b)$.

\begin{enumerate}
    \item The challenger samples a uniformly random full rank linear map $h: \bbF_2^{n} \to \bbF_2^{n-1}$ and prepares the state
    \[\frac{1}{\sqrt{2^m}}\sum_{x \in \{0,1\}^m}(-1)^{b \cdot p_{f,h}(x)}\ket{x}_X\ket{h(f(x))}_Z,\]
    where $p_{f,h}(x) = 0$ iff, letting $y \coloneqq f(x) \in \{0,1\}^n, z \coloneqq h(y) \in \{0,1\}^{n-1}$, it holds that $y$ is the lexicographially first string such that $h(y) = z$. The challenger then measures the $Z$ register in the standard basis to obtain $z$, and the remaining state collapses to 
    \[\frac{1}{\sqrt{2^{m-n+1}}}\left(\sum_{x : f(x) = y_{z,0}}\ket{x}_X + (-1)^b\sum_{x : f(x) = y_{z,1}}\ket{x}_X\right),\] where $y_{z,0},y_{z,1} \in \{0,1\}^n$ are the two strings such that $h(y_{z,0}) = h(y_{z,1}) = z$, ordered lexicographically.
    \item The challenger sends $h,z$, and the $X$ register to $\cA_\secp$, who outputs a classical deletion certificate $\pi$ and a left-over quantum state $\rho$.
    \item If $h(f(\pi)) = z$, output $\rho$, and otherwise output $\bot$.
\end{enumerate}

\begin{theorem}
\[\TD\left(\Exp_{f,\cA,\secp}(0),\Exp_{f,\cA,\secp}(1)\right) = \negl(\secp).\]
\end{theorem}

\begin{proof}
\james{Sketch} The proof will basically follow the same strategy used above in the collapsing-based certified deletion game.
\begin{enumerate}
	\item Hybrid 1: Entangle the phase bit with a challenger's register $C$ and measure this register at the end of experiment to obtain $b$.
	\item Hybrid 2: If the adversary outputs a valid deletion certificate $\pi$, project register $C$ onto $\frac{1}{\sqrt{2}}\left(\ket{0} + (-1)^{p_{f,h}(\pi)}\ket{1}\right).$ In this hybrid, $b$ is then a uniformly random bit.
\end{enumerate}
To show indistinguishability between Hybrid 1 and 2, we need to define an experiment where the bit $p_{f,h}(x)$ is measured on register $X$ before the state is given to the adversary. Let \[\ket{0_{f,h,z}} \coloneqq \frac{1}{\sqrt{2^{m-n}}}\sum_{x:f(x)=y_{z,0}}\ket{x}, ~~ \ket{1_{f,h,z}} \coloneqq \frac{1}{\sqrt{2^{m-n}}}\sum_{x:f(x)=y_{z,1}}\ket{x},\] and $\ket{+_{f,h,z}} \coloneqq \frac{1}{\sqrt{2}}(\ket{0_{f,h,z}} + \ket{1_{f,h,z}}), \ket{-_{f,h,z}} \coloneqq \frac{1}{\sqrt{2}}(\ket{0_{f,h,z}} - \ket{1_{f,h,z}})$. Then, measuring this bit on register $X$ collapses the state as follows: \[\frac{1}{\sqrt{2}}\ket{0}_C\ket{+_{f,h,z}}_X + \frac{1}{\sqrt{2}}\ket{1}_C\ket{-_{f,h,z}}_X \to \frac{1}{2}\ket{+}\bra{+}_C \otimes \ket{0_{f,h,z}}\bra{0_{f,h,z}}_X + \frac{1}{2}\ket{-}\bra{-}_C \otimes \ket{1_{f,h,z}}\bra{1_{f,h,z}}_X\] 
	
	
By [AAS, HMY] and the fact that all $\ket{+_{f,h,z}}$ and $\ket{-_{f,h,z}}$ (for all $h,z$) are orthogonal states, which follows from the regularity of $f$, any $\cA$ that can distinguish this switch can be used to map the state $\ket{0_{f,h,z}} \to \ket{1_{f,h,z}}$ (on average over $h,z$). This is impossible due to the one-wayness of $f$ and the fact that $f$ is regular (which allows the reduction to go through). Moreover, in this experiment, there is a negligible probability that the adversary outputs a valid $\pi$ but projecting register $C$ onto $\frac{1}{\sqrt{2}}\left(\ket{0} + (-1)^{p_{f,h}(\pi)}\ket{1}\right)$ fails, which again follows from the one-wayness and regularity of $f$. Thus, adding this projection to Hybrid 2 must only have a negligible affect on the experiment.

\end{proof}

\paragraph{Applications}
\begin{itemize}
    \item By purifying, this theorem can be used to obtain bit commitments with publicly-verifiable certified deletion. This will imply ZK proofs with publicly-verifiable everlasting zero-knowledge, and secure computation with publicly-verifiable EST.
    \item If $f$ admits a trapdoor that allows to prepare the uniform superposition over preimages for any image (a special case of this is the standard notion of an injective trapdoor function), then this implies public-key encryption with publicly-verifiable certified deletion.
\end{itemize}
%\begin{table}[]
\centering
\begin{tabular}{l|c}
\toprule
                & Agreement (\%) \\
\midrule
Human vs. Human & $63.5 \pm 4.3$  \\
CLIP vs. Human  & $56.8 \pm 1.7$  \\
\midrule
HPS vs. Human   & $61.5 \pm 1.1$  \\
\bottomrule
\end{tabular}
\vspace{0.4cm}
\caption{Agreement on comparing images generated by  Stable Diffusion and DALL·E.}
\label{tab:agreement}
\end{table}

\section{Publicly-Verifiable Deletion from Balanced Binary-Measurement TCR}\label{sec:regularOWF}

In this section, we show how to build a variety of cryptographic primitives with $\PVD$ from a specific type of hash function that we call \emph{balanced binary-measurement target-collision-resistant}.

\begin{definition}[Balanced Binary-Measurement TCR Hash]\label{def:BBMhash}
A hash function family $\cH = \{H_\secp : \{0,1\}^{m(\secp)} \to \{0,1\}^{n(\secp)}\}_{\secp \in \bbN}$ is \emph{balanced binary-measurement target-collision-resistant} if:
\begin{enumerate}
    \item There exists a family of efficiently computable \emph{single-output-bit} measurement functions $\cM = \{\{M[h] : \{0,1\}^{m(\secp)} \to \{0,1\}\}_{h \in H_\secp}\}_{\secp \in \bbN}$ such that $\cH$ is $\cM$-target-collision-resistant (\cref{def:target-CR}).
    \item There exists a constant $\delta > 0$ such that\footnote{It is also straightforward to generalize our results to any $\delta(\secp) = 1/\poly(\secp)$.} \[\Pr_{h \gets H_\secp, x \gets \{0,1\}^{m(\secp)}}\left[\bigg|\frac{A_{h,x,0} - A_{h,x,1}}{A_{h,x,0} + A_{h,x,1}}\bigg| \leq 1-\delta\right] = 1-\negl(\secp),\]
    
    where $A_{h,x,b} \coloneqq |\{x' \in h^{-1}(h(x)) : M[h](x') = b\}|.$
\end{enumerate}
\end{definition}

\begin{remark}
By \cref{thm:TC-from-TCR} and \cref{thm:CETC-generalization}, any balanced binary-measurement TCR $\cH$ with associated measurement function $\cM$ is also $\cM$-target-collapsing and certified everlasting $\cM$-target-collapsing.
\end{remark}


\subsection{Commitments}
\label{sec:com}

A \emph{canonical quantum bit commitment} \cite{Yan} consists of a family of pairs of unitaries $\{(Q_{\secp,0},Q_{\secp,1})\}_{\secp \in \bbN}$. To commit to a bit $b$, the committer applies $Q_{\secp,b}$ to the all-zeros state $\ket{0}$ to obtain a state on registers $C$ and $R$, and sends register $C$ to the receiver. To open, the committer sends the bit $b$ and the remaining state on register $R$. The receiver applies $Q_{\secp,b}^\dagger$ to registers $(C,R)$, measures the result in the standard basis, and accepts if all zeros are observed.

\begin{definition}[Computational Hiding]
A canonical quantum bit commitment $\{(Q_{\secp,0},Q_{\secp,1})\}_{\secp \in \bbN}$ satisfies \emph{computational hiding} if for any QPT adversary $\{\cA_\secp\}_{\secp \in \bbN}$,

\[\left|\Pr\left[\cA_\secp(\Tr_R\left(Q_{\secp,0}\ket{0}\right)) = 1\right] - \Pr\left[\cA_\secp(\Tr_R\left(Q_{\secp,1}\ket{0}\right)) = 1\right] \right| = \negl(\secp).\]
\end{definition}

\begin{definition}[Honest Binding]
A canonical quantum bit commitment $\{(Q_{\secp,0},Q_{\secp,1})\}_{\secp \in \bbN}$ satisfies \emph{honest binding} if for any auxiliary family of states $\{\ket{\psi_\secp}\}_{\secp \in \bbN}$ on register $Z$ and any family of physically realizable unitaries $\{U_\secp\}_{\secp \in \bbN}$ on registers $R,Z$, 

\[\left\| \left(Q_{\secp,1}\dyad{0}{0}Q_{\secp,1}^\dagger\right)U\left(Q_{\secp,0}\ket{0}\ket{\psi}\right)\right\| = \negl(\secp).\]
\end{definition}

\begin{definition}[Publicly-Verifiable Deletion]
A canonical quantum bit commitment $\{(Q_{\secp,0},Q_{\secp,1})\}_{\secp \in \bbN}$ has \emph{publicly-verifiable deletion} if there exists a measurement $\{V_{\secp}\}_{\secp \in \bbN}$ on register $R$, a measurement $\{D_{\secp}\}_{\secp \in \bbN}$ on register $C$, and a classical predicate $\Ver(\cdot,\cdot) \to \{\top,\bot\}$ that satisfy the following properties.

\begin{itemize}
    \item \textbf{Correctness of deletion.} For any $b \in \{0,1\}$, it holds that
    \[\Pr\left[\Ver(\vk,\pi) = \top : (\vk,\pi) \gets (V_\secp \otimes D_\secp)Q_{\secp,b}\ket{0}\right] = 1-\negl(\secp).\]
    \item \textbf{Certified everlasting hiding.} For any QPT adversary $\cA = \{\cA_\secp\}_{\secp \in \bbN}$, it holds that 
    \[\TD\left(\mathsf{EvExp}_{\cA}(\secp,0), \mathsf{EvExp}_{\cA}(\secp,1)\right) = \negl(\secp),\] where $\mathsf{EvExp}_{\cA}(\secp,b)$ is the following experiment.
    \begin{itemize}
        \item Prepare $Q_{\secp,b}\ket{0}$, measure register $R$ with $V_\secp$ to obtain $\vk$, and send $(\vk,C)$ to $\cA_\secp$.
        \item Parse $\cA_{\secp}$'s output as a deletion certificate $\pi$ and a left-over state $\rho$. If $\Ver(\vk,\pi) = \bot$, output $\bot$, and otherwise output $\rho$.
    \end{itemize}
\end{itemize}
\end{definition}

%\james{Define zero-knowledge proof with certified everlasting zero-knowledge}

\paragraph{Construction.} We construct a quantum canonical bit commitment with $\PVD$ as follows. Let $\cH = \{H_\secp : \{0,1\}^{m(\secp)} \to \{0,1\}^{n(\secp)}\}_{\secp \in \bbN}$ be a balanced binary-measurement TCR hash with associated measurement function $\cM = \{\{M[h]\}_{h \in H_\secp}\}_{\secp \in \bbN}$, and let $m = m(\secp)$, $n = n(\secp)$. For any $h \in H_\secp,y \in \{0,1\}^n$, and $b \in \{0,1\}$, we will define the state 

\[\ket{\psi_{h,y,b}} \coloneqq \frac{1}{\sqrt{|h^{-1}(y)|}}\sum_{x : h(x)= y}(-1)^{M[h](x)}\ket{x}.\]

\begin{itemize}
    \item Consider the following procedure $S_{\secp,b}$. Sample $h \gets H_\secp$ and for $i \in [\secp]$, prepare the state \[\frac{1}{\sqrt{2^m}}\sum_{x \in \{0,1\}^m}(-1)^{b \cdot M[h](x)}\ket{x}\ket{h(x)},\] and measure the second register to obtain $y_i$ and left-over state $\ket{\psi_{h,y_i,b}}$. Then, output \[(h,y_1,\dots,y_\secp),
    \bigotimes_{i \in [\secp]}\ket{\psi_{h,y_i,b}}.\]
    
    
    Now, $Q_{\secp,b}$ will be the purification of $S_{\secp,b}$, where the output register is $C$ and the auxiliary register is $R$. That is, $Q_{\secp, b}$ prepares the state
    \[\frac{1}{\sqrt{|H_\secp|2^{\secp m}}}\sum_{h,x_1,\dots,x_\secp}(-1)^{b \cdot \bigoplus_{i \in [\secp]}M[h](x_i)}\ket{h,h(x_1),\dots,h(x_\secp)}_R\ket{h,h(x_1),\dots,h(x_\secp),x_1,\dots,x_\secp}_C.\]
    \item $V_\secp$ measures register $R$ in the standard basis to obtain $\vk = (h,y_1,\dots,y_\secp)$. $D_\secp$ measures register $C$ in the standard basis to obtain $(h,y_1,\dots,y_\secp,x_1,\dots,x_\secp)$, and outputs $\pi = (x_1,\dots,x_\secp)$.
    \item $\Ver((h,y_1,\dots,y_\secp),(x_1,\dots,x_\secp))$ outputs $\top$ iff $h(x_i) = y_i$ for all $i \in [\secp]$.
\end{itemize}

\begin{theorem}\label{thm:commitment}
The above construction satisfies computational hiding, honest binding, and publicly-verifiable deletion. Thus, assuming the existence of a balanced binary-measurement TCR hash, there exists a quantum canonical bit commitment with $\PVD$.
\end{theorem}

\begin{proof}
%First, we note that computational hiding and certified everlasting hiding follow from \cref{corollary:target-collapsing} and a straightforward hybrid argument, so it suffices to show honest binding.\james{TODO: make the hybrid argument explicit}

First we argue computational hiding. On a commitment to $b$, the receiver sees the mixed state 

\[\E_{h,y_1,\dots,y_\secp}\left[\bigotimes_{i \in [\secp]}\ket{\psi_{h,y_i,b}}\right],\] where the expectation is over sampling $h \gets \cH$ and measuring random $y_1,\dots,y_\secp$. Note the following two facts.

\begin{enumerate}
	\item Given any state $\ket{\psi_{h,y,b}}$, let $M[h]\left(\ket{\psi_{h,y,b}}\right)$ be the mixed state that results from measuring the bit $M[h](\cdot)$ on $\ket{\psi_{h,y,b}}$. By the $\cM$-target-collapsing of $\cH$, we have that for any $b \in \{0,1\}$, \[\E_{h,y}\left[\ket{\psi_{h,y,b}}\right] \approx_c \E_{h,y}\left[M[h]\left(\ket{\psi_{h,y,b}}\right)\right],\] where $\approx_c$ denotes computational indistinguishability. The case of $b=0$ follows directly by definition of $\cM$-target-collapsing and the case of $b=1$ follows because a reduction can efficently map $\ket{\psi_{h,y,0}}$ to $\ket{\psi_{h,y,1}}$ using the fact that $M[h]$ is efficiently computable.
	\item For any $h,y$, $M[h]\left(\ket{\psi_{h,y,0}}\right)$ and $M[h]\left(\ket{\psi_{h,y,1}}\right)$ are equivalent states, which follows by definition.
\end{enumerate}

Thus, we can run the following hybrid argument.

\begin{itemize}
	\item $\Hyb_0$: The receiver is given a commitment to 0.
	\item $\Hyb_1\dots\Hyb_\secp$: In $\Hyb_i$, we switch $\ket{\psi_{h,y,0}}$ to $M[h]\left(\ket{\psi_{h,y,0}}\right)$. This is computationally indistinguishable from $\Hyb_{i-1}$ by the first fact above.
	\item $\Hyb_{\secp+1}$: Switch $M[h]\left(\ket{\psi_{h,y_i,0}}\right)$ to $M[h]\left(\ket{\psi_{h,y_i,1}}\right)$ for all $i \in [\secp]$. This is perfectly indistinguishable from $\Hyb_\secp$ by the second fact above.
	\item $\Hyb_{\secp+2}\dots\Hyb_{2\secp + 1}$: In $\Hyb_{i + \secp + 1}$, we switch $M[h]\left(\ket{\psi_{h,y_i,1}}\right)$ to $\ket{\psi_{h,y_i,1}}$. This is computationally indistinguishable from $\Hyb_{i+\secp}$ by the first fact above.
\end{itemize}

This completes the proof of computational hiding. Next, since $\cH$ satisfies \emph{certified everlasting} $\cM$-target-collapsing, we see that each hybrid is \emph{statistically close} when the receiver outputs a valid deletion certificate. Thus, the same proof establishes publicly-verifiable deletion.


%For this, it suffices to show (by Uhlmann's theorem) that the trace distance between the outputs of $S_{\secp,0}$ and $S_{\secp,1}$ is $1-\negl(\secp)$. We lower bound the trace distance by demonstrating a measurement that distinguishes with probability $1-\negl(\secp)$. 

%\james{TODO: Instead of citing Uhlmann's theorem, we can say that it suffices to demonstrate a measurement on just the $R$ register that accepts $Q_{\secp,0}\ket{0}$ with probability 1 and rejects $Q_{\secp,0}\ket{0}$ with probability $1-\negl(\secp)$, and note that any $U$ in the definition of honest-binding cannot change the success probability of this measurement}

Finally, we show honest binding. For this, it suffices to demonstrate a measurement on register $C$ that accepts with probability 1 on the output of $Q_{\secp,0}$ and with probability $\negl(\secp)$ on the output of $Q_{\secp,1}$. This suffices because any $U$ that breaks honest binding must then necessarily affect the result of this measurement by a $\nonnegl(\secp)$ amount, which is impossible since $U$ does not operate on $C$.

The measurement takes the classical part of the output $(h,y_1,\dots,y_\secp)$ and attempts to project the quantum part onto \[\dyad{\psi_{h,y_1,0}}{\psi_{h,y_1,0}} \otimes \dots \otimes \dyad{\psi_{h,y_\secp,0}}{\psi_{h,y_\secp,0}}.\] Clearly this accepts the output of $S_{\secp,0}$ with probability 1, so it suffices to show that the output of $S_{\secp,1}$ is accepted with probability $\negl(\secp)$. To see this, we bound

\begin{align*}
    &\E_{h,y_1,\dots,y_\secp}\left[\prod_{i \in [\secp]}|\braket{\psi_{h,y_i,1}|\psi_{h,y_i,0}}|^2\right] \\
    &\E_{h,y_1,\dots,y_\secp}\left[\prod_{i \in [\secp]}\left(\frac{1}{|h^{-1}(y_i)|}\left(\sum_{x:h(x)=y_i}(-1)^{M[h]}\bra{x}\right)\left(\sum_{x:h(x)=y_i}\ket{x}\right)\right)^2\right] \\
    &= \E_{h,y_1,\dots,y_\secp}\left[\prod_{i \in [\secp]}\left(\frac{1}{|h^{-1}(y_i)|}\left(|\{x : h(x)=y_i,M[h](x)=0\}|-|\{x : h(x)=y_i,M[h](x)=1\}|\right)\right)^2\right] \\
    %&=\E_{h,y_1,\dots,y_\secp}\left[\prod_{i \in [\secp]}\left(\frac{p_{y_i}-p_{y_i \oplus \Delta}}{p_{y_i}+p_{y_i \oplus \Delta}}\right)^2\right]\\
    &\leq \left(1-\delta\right)^{2\secp} + \negl(\secp) \\
    &=\negl(\secp),
\end{align*}

where the inequality follows from property (2) of \cref{def:BBMhash}.
  
%\begin{itemize}
    %\item \james{computational hiding is straightforward from target-collapsing}
    %\item \james{to show honest binding, we can establish that the fidelity between the states on register $C$ when $b = 0$ vs $b=1$ is negligible, which follows from $\delta$-balanced and $\secp$ repetition}
    %\item \james{certified everlasting hiding is straightforward from certified everlasting target-collapsing}
%\end{itemize}
\end{proof}



\subsection{Public-Key Encryption}
\label{sec:pke}

\begin{definition}[Trapdoor Phase-Recoverability]
We say that a balanced binary-measurement TCR hash has \emph{trapdoor phase-recoverability} if there exist algorithms $\Samp,\Recover$ with the following properties.
\begin{itemize}
    \item $\Samp(1^\secp)$: The sampling algorithm samples a uniformly random function $h \in H_\secp$ along with a trapdoor $\td$.
    \item $\Recover(\td,y,X)$: There exist constants $c,\epsilon$ such that with probability $1-\negl(\secp)$ over $(h,\td) \gets \Samp(1^\secp)$, 
    
    \begin{align*}
        &\Pr_{x \gets \{0,1\}^m}\left[\Recover(\td,h(x),\ket{\psi_{h,h(x),0}}) \to 0\right] \geq c + \epsilon,\\
        &\Pr_{x \gets \{0,1\}^m}\left[\Recover(\td,h(x),\ket{\psi_{h,h(x),1}}) \to 0\right] \leq c - \epsilon,
    \end{align*}
    
    %\begin{align*}
    %\Pr_{x \gets \{0,1\}^m}\left[\Recover(\td,h(x),\ket{\psi_{h,h(x),0}}) \to 0\right] \geq c+\epsilon
    
    %&\Pr_{x \gets \{0,1\}^m}\left[\Recover(\td,h(x),\ket{\psi_{h,h(x),1}}) \to 0\right] \leq c-\epsilon,
    %\end{align*}
    
    where \[\ket{\psi_{h,y,b}} \coloneqq \frac{1}{\sqrt{|h^{-1}(y)|}}\sum_{x:h(x) = y}(-1)^{M[h](x)}\ket{x}.\]
\end{itemize}
\end{definition}



\begin{theorem}\label{thm:PKE}
Assuming the existence of a binary-measurement TCR hash $\cH$ with trapdoor phase-recoverability, there exists public-key encryption with $\PVD$.
\end{theorem}

\begin{proof}
This follows from essentially the same construction as commitments. Let $\cM$ be the measurement function associated with $\cH$ and let $(\Samp,\Invert)$ be the associated trapdoor algorithms. Then, the PKE with $\PVD$ is defined as follows.
\begin{itemize}
    \item $\Gen(1^\secp)$: Sample $(h,\td) \gets \Samp(1^\secp)$ and set $\pk \coloneqq h, \sk \coloneqq \td$.
    \item $\Enc(\pk,b)$: For $i \in [\secp]$, prepare the state \[\frac{1}{\sqrt{2^m}}\sum_{x \in \{0,1\}^m}(-1)^{b \cdot M[h](x)}\ket{x}\ket{h(x)},\] and measure the second register to obtain $y_i$ and left-over state $\ket{\psi_{h,y_i,b}}$. Then, set  \[\ket{\ct} \coloneqq \left(y_1,\dots,y_\secp,
    \bigotimes_{i \in [\secp]}\ket{\psi_{h,y_i,b}}\right), ~~ \vk \coloneqq (h,y_1,\dots,y_\secp).\]
    \item $\Dec(\sk,\ket{\ct})$: Parse $\ket{\ct}$ as $(y_1,\dots,y_\secp,X_1,\dots,X_\secp)$, for $i \in [\secp]$ run \[b_i \gets \Recover(\td,y_i,X_i),\]
    and output 0 if $|\{i : b_i = 0\}|/\secp > c$, and output 1 otherwise.
    
    
    %perform the measurement \[\left\{\dyad{\psi_{h,y_1,0}}{\psi_{h,y_1,0}} \otimes \dots \otimes \dyad{\psi_{h,y_\secp,0}}{\psi_{h,y_\secp,0}}, \bbI - \dyad{\psi_{h,y_1,0}}{\psi_{h,y_1,0}} \otimes \dots \otimes \dyad{\psi_{h,y_\secp,0}}{\psi_{h,y_\secp,0}}\right\}\] on the second part. Output 0 if the first outcome is observed and 1 otherwise. Note that since $\ket{\psi_{h,y_i,0}}$ can be efficiently prepared (to within $\negl(\secp)$ trace distance) given $\td$ and $y_i$, this measurement can be efficiently implemented.
    
    \item $\Del(\ket{\ct})$: Parse $\ket{\ct}$ as $(y_1,\dots,y_\secp,X_1,\dots,X_\secp)$ and measure $X_i$ in the standard basis to obtain $\pi \coloneqq (x_1,\dots,x_\secp)$.
    \item $\Vrfy(\vk,\pi)$: Output $\top$ iff $h(x_i) = y_i$ for all $i \in [\secp]$.
\end{itemize}

Correctness follows from a standard Hoeffding inequality and correctness of deletion (\cref{def:correctness-deletion}) is immediate. Certified deletion security (\cref{def:security-deletion}) follows from the $\cM$-target-collapsing and certified everlasting $\cM$-target-collapsing of $\cH$, using the same hybrid argument as in the proof of \cref{thm:commitment}.
\end{proof}

\subsection{A Generic Compiler}
\label{sec:generic}
Let $(\Gen,\Enc,\Dec,\Del,\Vrfy)$ be the encryption scheme defined last section, let $\cA = \{\cA_\secp\}_{\secp \in \bbN}$ be an adversary, let $p(\secp)$ be a polynomial, and let $\cZ = \{Z_\secp(\aux)\}_{\aux \in \{0,1\}^{p(\secp)}, \secp \in \bbN}$ be a (static or interactive) family of distributions that is semantically-secure against $\cA$ with respect to $\aux$. That is, in the static case, it holds that for any $\aux \in \{0,1\}^{p(\secp)}$,

\[\bigg| \Pr\left[\cA_\secp(Z_\secp(\aux)) = 1\right] - \Pr\left[\cA_\secp(Z_\secp(0^{p(\secp)})) = 1\right]\bigg| \leq \negl(\secp),\] and in the interactive case,

\[\bigg| \Pr\left[\cA_\secp^{Z_\secp(\aux)} = 1\right] - \Pr\left[\cA_\secp^{Z_\secp(0^{p(\secp)})} = 1\right]\bigg| \leq \negl(\secp),\] where $\cA_\secp^{Z_\secp(\aux)}$ indicates that $\cA_\secp$ can interact with $Z_\secp(\aux)$, which is the description of an interactive machine initialized with $\aux$.





%$\cZ = \{\cZ_\secp(\cdot,\cdot)\}_{\secp \in \bbN}$ be any (potentially quantum) operation such that for any distribution $\cD = \{\cD_\secp\}_{\secp \in \bbN}$ over pairs $(x,\ket{\psi})$, where $x$ is a classical string and $\ket{\psi}$ is a quantum state, it holds that
%\[\bigg|\Pr_{(x,\ket{\psi}) \gets \cD_\secp}\left[\cA_\secp(\cZ_\secp(x,\ket{\psi})) = 1\right] -  \Pr_{(x,\ket{\psi})\gets \cD_\secp}\left[\cA_\secp(\cZ_\secp(0,\ket{\psi})) = 1\right] \bigg| = \negl(\secp).\] That is, $\cZ$ is semantically-secure in its first input (with respect to $\cA$). 



\begin{lemma}\label{lemma:compiler}
Given any $\cA,\cZ$ as described above, define the experiment $\mathsf{EvEnc}_{\cA,\cZ,\secp}(b)$ as follows.

\begin{itemize}
    \item Sample $(h,\td) \gets \Gen(1^\secp)$ and $(\ket{\ct},\vk) \gets \Enc(h,b)$.
    \item Run $\cA_\secp(h,\vk,\ket{\ct},Z_\secp(\td))$, and parse their output as a deletion certificate $\pi$ and a left-over quantum state $\rho$.
    \item If $\Vrfy(\vk,\pi) = \top$, output $\rho$, and otherwise output $\bot$.
\end{itemize}
Then it holds that
\[\TD\left(\mathsf{EvEnc}_{\cA,\cZ,\secp}(0),\mathsf{EvEnc}_{\cA,\cZ,\secp}(1)\right) = \negl(\secp).\]
\end{lemma}

\begin{proof}

First, we confirm that $\cH$ is $(\cM,\cZ)$-target-collision-resistant.  To see this, we first use the semantic security of $\cZ$ to switch to a hybrid where $\cA_\secp$ receives $Z_\secp(0)$ rather than $Z_\secp(\td)$, and then appeal directly to the fact that $\cH$ is $\cM$-target-collision-resistant (\cref{thm:target-CR}). Then by \cref{thm:TC-from-TCR} and \cref{thm:CETC-generalization}, we have that $\cH$ is certified everlasting $(\cM,\cZ)$-target-collapsing. Using the same hybrid argument as in the proof of \cref{thm:commitment} then completes the proof.



%Given $\widetilde{\cF}$, define $\widetilde{\cF}'$ to be the same family except that $\cZ_\secp(\td)$ is included in the description of the hash function. Note that if $\widetilde{\cF}$ is $(\cU,\cP)$-target-collision-resistant, then so is  $\widetilde{\cF}'$. This follows because we can first use the semantic security of $\cZ$ to switch $\cZ_\secp(\td)$ to $\cZ_\secp(0)$, and then appeal to $(\cU,\cP)$-target-collision-resistance of $\widetilde{\cF}$. Then, by \cref{thm:TC-from-TCR} and \cref{thm:CETC-generalization}, we have that $\widetilde{\cF}'$ is certified everlasting $(\cU,\cP)$-target-collapsing, which immediately implies the theorem.
\end{proof}

By instantiating $\cZ$ with various crytographic primitives, we immediately gives the following applications. We do not write formal definitions of each of these primitives, and instead refer the reader to \cite{cryptoeprint:2022/1178} for these.


\begin{corollary}\label{cor:compiler}
Assuming the existence of a balanced binary-measurement TCR hash with trapdoor phase-recoverability, and post-quantum \[X \in \left\{\begin{array}{r}\text{quantum fully-homormophic encryption, attribute-based encryption}, \\ \text{witness encryption, timed-release encryption}\end{array}\right\},\] there exists $X$ with $\PVD$.
\end{corollary}

The implications to witness encryption and timed-release encryption follow immediately by encrypting $\td$ with the appropriate encryption scheme (and in the case of timed-release encryption, considering the class of parallel-time-bounded adversaries). We briefly remark on the other two implications.

\begin{itemize}
    \item \textbf{Fully-homomorphic encryption.} If we encrypt $\td$ using a \emph{quantum} fully-homomorphic encryption (QFHE) scheme, then we obtain (Q)FHE with publicly-verifiable deletion. The reason we need QFHE for the compiler is for evaluation correctness: we need to decrypt $\ket{\ct}$ homomorphically under the QFHE (using $\td$) in order to obtain a (Q)FHE encryption of the plaintext, which can then be operated on.
    \item \textbf{Attribute-based encryption.} If we encrypt $\td$ using an attribute-based encryption (ABE) scheme, we immediately obtain a correct ABE scheme with certified deletion. In order to argue that this scheme has certified deletion security, we appeal to \cref{lemma:compiler} with an interactive $Z_\secp$ that runs the ABE security game, encrypting its input $\td$ into the challenge ciphertext.
\end{itemize}



\subsection{Balanced Binary-Measurement TCR from Almost-Regular OWFs}
\label{sec:almostreg}



\begin{definition}[Almost-Regular Function]\label{def:almost-regular}
A function $\cF = \{f_\secp : \{0,1\}^{m(\secp)} \to \{0,1\}^{n(\secp)}\}$ is almost-regular if there exists efficiently computable polynomials $r(\secp)$ and $p(\secp)$ such that for all $\secp \in \bbN$ and $x \in \{0,1\}^{m(\secp)}$, \[\frac{1}{p(\secp)} \cdot 2^{r(\secp)} \leq \big| \{x' \in \{0,1\}^{n(\secp)} : f_\secp(x') = f_\secp(x)\}\big| \leq p(\secp) \cdot 2^{r(\secp)}.\]
\end{definition}

Note that we assume $r(\secp)$ is efficiently computable, which means that the regularity of $\cF$ is \emph{known}. This is often contrasted with the more general class of functions that are \emph{unknown} regular. Throughout this work, we always means \emph{known} regular.

\begin{definition}[Balanced Function]\label{def:balanced}
A function $\cF = \{f_\secp : \{0,1\}^{m(\secp)} \to \{0,1\}^{n(\secp)}\}_{\secp \in \bbN}$ is $\delta$-\emph{balanced} for some constant $\delta \in [0,1)$ if there exists a family of sets $\{\mathsf{BAD}_\secp \subset \{0,1\}^{n(\secp)}\}_{\secp \in \bbN}$ such that
\begin{enumerate}
    \item $|\mathsf{BAD}_\secp|/2^{n(\secp)} = \negl(\secp)$.
    \item $\Pr_{x \gets \{0,1\}^{m(\secp)}}[f_\secp(x) \in \mathsf{BAD}_\secp] = \negl(\secp)$.
    \item For every $z \notin \mathsf{BAD}_\secp$, $\Pr_{x \gets \{0,1\}^{m(\secp)}}[f_\secp(x) = z] \cdot 2^{n(\secp)} \in [1-\delta,1+\delta]$.
\end{enumerate}
\end{definition}

\begin{definition}[One-Way Function]\label{def:OWF}
A function $\cF = \{f_\secp : \{0,1\}^{m(\secp)} \to \{0,1\}^{\ell(\secp)}\}$ is one-way if for any QPT adversary $\cA = \{\cA_\secp\}_{\secp \in \bbN}$,

\[\Pr\left[f_\secp(x')=f(x) : \begin{array}{r}x \gets \{0,1\}^{m(\secp)} \\ x' \gets \cA_\secp(f(x))\end{array}\right] = \negl(\secp).\]

We say that $\cF$ is one-way \emph{over its range} if for any QPT adversary $\cA = \{\cA_\secp\}_{\secp \in \bbN}$,
\[\Pr\left[f_\secp(x) = y : \begin{array}{r}y \gets \{0,1\}^{\ell(\secp)} \\ x \gets \cA_\secp(y)\end{array}\right] = \negl(\secp).\]
\end{definition}

\begin{definition}[Universal Hash]
A hash function family $\cH = \{H_\secp : \{0,1\}^{m(\secp)} \to \{0,1\}^{n(\secp)}\}_{\secp \in \bbN}$ is called $t(\secp)$-universal if for each distinct $x_1,\dots,x_{t(\secp)} \in \{0,1\}^{m(\secp)}$ and $y_1,\dots,y_{t(\secp)} \in \{0,1\}^{n(\secp)}$, it holds that 
\[\Pr_{h \gets H_\secp}\left[h(x_1) = y_1 \wedge \dots \wedge h(x_{t(\secp)}) = y_{t(\secp)}\right] = 2^{-n(\secp) \cdot t(\secp)}.\]
\end{definition}

\begin{importedtheorem}[\cite{balancedOWF}]\label{impthm:balancedOWF}
Let $\cF = \{f_\secp : \{0,1\}^{m(\secp)} \to \{0,1\}^{\ell(\secp)}\}$ be an almost-regular one-way function. Then there exists $n(\secp) < m(\secp)$ and $\delta \in [0,1)$ such that for any $3\secp$-universal hash family $\cH = \{H_\secp : \{0,1\}^{\ell(\secp)} \to \{0,1\}^{n(\secp)}\}_{\secp \in \bbN}$ where each $h \in H_\secp$ can be described by $s(\secp)$ bits, the function \[\cF' = \left\{f'_\secp : \{0,1\}^{s(\secp) + m(\secp)} \to \{0,1\}^{s(\secp) + n(\secp)}\right\}_{\secp \in \bbN}, ~~ \text{where} ~~ f'_\secp(h,x) \coloneqq (h,h(f_\secp(x))),\] is $\delta$-balanced and one-way over its range.
\end{importedtheorem}

Now consider any balanced function $\cF = \{f_\secp : \{0,1\}^{m(\secp)} \to \{0,1\}^{n(\secp)}\}_{\secp \in \bbN}$ that is one-way over its range, and define the family of hash functions \[\cH^\cF = \left\{H_\secp : \{0,1\}^{m(\secp)} \to \{0,1\}^{n(\secp)}\right\}_{\secp \in \bbN}\] as follows. For each $\Delta \in \{0,1\}^{n(\secp)}$, define $f_\Delta : \{0,1\}^{n(\secp)} \to \{0,1\}^{n(\secp)}$ to, on input $z$, output the lexicographically first element of $\{z,z \oplus \Delta\}$.\footnote{We don't need to worry about the case when $\Delta = 0^{n(\secp)}$ since we'll be sampling $\Delta$ uniformly, but one could define $f_\Delta$ to be the identity in that case.} Then we define

\[H_\secp \coloneqq \left\{h_{\secp,\Delta} \coloneqq f_\Delta \circ f_\secp\right\}_{\Delta \in \{0,1\}^{n(\secp)}}.\]

We will also define the family of measurement functions  \[\cM = \left\{\left\{M[h_{\secp,\Delta}]\right\}_{h_{\secp,\Delta} \in H_\secp}\right\}_{\secp \in \bbN}\] as follows. The predicate $M[h_{\secp,\Delta}] : \{0,1\}^m \to \{0,1\}$ takes $x$ as input, computes $z \coloneqq f_\secp(x)$, and outputs $0$ if $z < z \oplus \Delta$ and $1$ if $z > z \oplus \Delta$ (where ordering is lexicographical).

\begin{theorem}\label{thm:target-CR}
Let $\delta \in [0,1)$ be a constant and $\cF = \{f_\secp : \{0,1\}^{m(\secp)} \to \{0,1\}^{n(\secp)}\}_{\secp \in \bbN}$ be a $\delta$-balanced function that is one-way over its range. Let $\cH^\cF$ and $\cM$ be as defined above. Then, $\cH^\cF$ is a balanced binary-measurement TCR hash with associated measurement function $\cM$.
\end{theorem}

\cref{impthm:balancedOWF} and \cref{thm:commitment} immediately give the following corollary.

\begin{corollary}
Assuming almost-regular one-way functions, there exists a quantum canonical bit commitment with $\PVD$.
\end{corollary}

%\cref{thm:TC-from-TCR} and \cref{thm:CETC-generalization} immediately give the following corollary.

%\begin{corollary}\label{corollary:target-collapsing}
%$\cH^\cF$ is $(\cU,\cM)$-target-collapsing and certified everlasting $(\cU,\cM)$-target-collapsing.
%\end{corollary}



\begin{proof}(Of \cref{thm:target-CR})
First, we check property (2) of \cref{def:BBMhash}. By properties (2) and (3) of \cref{def:balanced}, it holds that with $1-\negl(\secp)$ probability over the sampling of $h \gets H_\secp$ and $x \gets \{0,1\}^m$, \[\bigg|\frac{|\{x' \in h^{-1}(h(x))\} : M[h] = 0| - |\{x' \in h^{-1}(h(x))\} : M[h] = 1|}{|\{x' \in h^{-1}(h(x))\} : M[h] = 0| + |\{x' \in h^{-1}(h(x))\} : M[h] = 1|}\bigg| \leq \delta.\]


Next, we check property (1). Throughout this proof, we will drop indexing by $\secp$ for convenience. Suppose there exists a QPT adversary $\cA$ that breaks the $\cM$-target-collision-resistance of $\cH$. That is, the following experiment outputs 1 with $\nonnegl(\secp)$ probability.\\

\noindent\underline{$\Exp_{\mathsf{TCR}}$}
\begin{itemize}
    \item The challenger samples $\Delta \gets \{0,1\}^n$ and prepares the state $1/\sqrt{2^m}\sum_{x \in \{0,1\}^m}\ket{x}$ on register $X$. It applies $h_\Delta$ on $X$ to a fresh register $Y$ and measures $y \in \{0,1\}^n$, and then measures $P[h_\Delta]$ on $X$ to obtain a bit $b$ and left-over state on register $X$. The challenger sends $(\Delta,y,b)$ and register $X$ to $\cA$.
    \item $\cA$ outputs a string $x' \in \{0,1\}^n$.
    \item Output 1 if $h_\Delta(x') = y$ and $M[h_\Delta](x') = 1-b$.
\end{itemize}



We now define an adversary $\cA'$ that breaks the one-wayness of $\cF$ over its range.\\

\noindent\underline{$\Exp_{\mathsf{OW}}$}

\begin{itemize}
    \item The challenger samples $z \gets \{0,1\}^n$ and sends $z$ to $\cA'$.
    \item $\cA'$ prepares the state $1/\sqrt{2^m}\sum_{x \in \{0,1\}^m}\ket{x}$ on register $X$, applies $f$ on $X$ to a fresh register $Z$, and measures $z' \in \{0,1\}^n$. If $z' = z$, then measure register $X$ to obtain $x'$, and return $x'$. Otherwise, set $\Delta \coloneqq z \oplus z'$, set $b = 0$ if $z' < z$ and $b = 1$ otherwise, and set $y = f_\Delta(z)$. Then, initialize $\cA$ with $(\Delta,y,b)$ and register $X$. Run $\cA$ and forward its output $x'$ to the challenger.
    \item Output 1 if $f(x') = z$.
\end{itemize}

It suffices to show that $\cA$'s input comes from the same distribution over $(X,\Delta,y,b)$ in both experiments. To see this, we describe an alternative but identical way to sample $(X,\Delta,y,b)$ in the experiment $\Exp_{\mathsf{TCR}}$. Recalling that $h_\Delta = f_\Delta \circ f$, the challenger could (1) apply $f$ on $X$ to a fresh register $Z$, (2) sample $\Delta \gets \{0,1\}^n$, (3) apply $f_\Delta$ on $Z$ to a fresh register $Y$, and (4) measure $Y$ to obtain $y$ and measure $M[h_\Delta]$ on $X$ to obtain $b$. Note that step (4) is equivalent to instead just measuring the $Z$ register to obtain $z$, defining $b = 0$ if $z < z \oplus \Delta$ and $b=1$ if $z > z \oplus \Delta$, and defining $y = f_\Delta(z)$. Thus, we can imagine first applying $f$ on $X$ to a fresh register $Z$, measuring $Z$ to obtain $z$, sampling $\Delta \gets \{0,1\}^n$, and defining $y = f_\Delta(z)$. Defining $z' = z \oplus \Delta$ and using the fact that $\Delta$ was sampled uniformly at random, we see that this is exactly the same distribution that is sampled in $\Exp_{\mathsf{OW}}$, except that $\cA$ is not initialized if $\Delta = 0^{n(\secp)}$ (in which case $\cA'$ wins the experiment anyway). 

\end{proof}

Now, we generalize the notion of almost-regularity (\cref{def:almost-regular}), balanced (\cref{def:balanced}), and one-wayness (\cref{def:OWF}) to function \emph{families}, where there is a set of of $f \in F_\secp$ associated with each security parameter. All previous definitions generalize to this setting with the requirement that they hold with $1-\negl(\secp)$ probability over $f \gets F_\secp$, and all previous claims follow. We consider families of functions with \emph{trapdoors} that allow us to invert the function and obtain public-key encryption along with other cryptographic primitives.

\begin{definition}[Superposition-invertible trapdoor function]
We say that a function family $\cF = \{F_\secp\}_{\secp \in \bbN}$ is a superposition-invertible trapdoor function if there exist algorithms $\Samp,\Invert$ with the following properties.
\begin{itemize}
    \item $\Samp(1^\secp)$: The sampling algorithm samples a uniformly random function $f \in F_\secp$ along with a trapdoor $\td$.
    \item $\Invert(\td,y)$: Given the trapdoor $\td$ and an image $y$, $\Invert$ outputs a state within negligible trace distance of \[\frac{1}{\sqrt{|f^{-1}(y)|}}\sum_{x:f(x)=y}\ket{x}.\]
\end{itemize}
\end{definition}

\begin{remark}
For the case of injective function families $\cF$, the notion of superposition-invertible trapdoor is equivalent to the standard notion of trapdoor, since there is only one preimage per image.
\end{remark}

\begin{claim}\label{claim:phase-recoverability}
Assuming injective trapdoor one-way functions (or more generally, superposition-invertible trapdoor almost-regular one-way functions), there exists a balanced binary-measurement TCR hash with trapdoor phase-recoverability.
\end{claim}

By \cref{thm:PKE} and \cref{cor:compiler}, we obtain the following corollary.

\begin{corollary}
Assuming the existence of injective trapdoor one-way functions (or more generally, superposition-invertible trapdoor almost-regular one-way functions), there exists PKE with $\PVD$. Additionally assuming post-quantum \[X \in \left\{\begin{array}{r}\text{quantum fully-homormophic encryption, attribute-based encryption}, \\ \text{witness encryption, timed-release encryption}\end{array}\right\},\] there exists $X$ with $\PVD$.
\end{corollary}

\begin{proof}(Of \cref{claim:phase-recoverability}) Given a superposition-invertible almost-regular one-way function, then we know from \cref{impthm:balancedOWF} that we can compose it with a $3\secp$-universal hash function to obtain a $\delta$-balanced function $\cF$ that is one-way over its range, and \cref{thm:target-CR} tells us that we can then obtain a balanced binary-measurement TCR hash $\cH^\cF = \{H_\secp\}_{\secp \in \bbN}$. It remains to check that the resulting hash has trapdoor phase-recoverability.

To see this, we observe that for any polynomials $m(\secp),n(\secp),t(\secp)$, there exists a superposition-invertible $t(\secp)$-universal hash function family $\{U_\secp : \{0,1\}^{m(\secp)} \to \{0,1\}^{n(\secp)}\}_{\secp \in \bbN}$ (without the need for a trapdoor). For example, we can use the Chor-Goldreich construction \cite{Chor-Goldreich}, where each hash in the family is defined by coefficients of a degree-$(t(\secp)-1)$ univariate polynomial over a finite field, and evaluation is polynomial evaluation. To invert, use a root-finding algorithm (e.g. \cite{Cantor1981ANA}) to recover the (at most polynomial) roots, and then arrange these in superposition. Note that for a compressing universal hash from $\{0,1\}^m \to \{0,1\}^n$, one would use a finite field of size at least $2^m$ and define the hash output to consist of (say) the first $n$ bits of the description of the finite field element that results from polynomial evaluation. In this case, the quantum inverter would first prepare a uniform superposition over all of the remaining $m-n$ bits of the field element, and run the above procedure in superposition.

Thus, given $h \in H_\secp$, where $h = f_\Delta \circ f$ for $\Delta \neq 0^n$, along with a trapdoor $\td$ for $f$, we can efficiently prepare the state \[\ket{\psi_{h,y,0}} = \frac{1}{\sqrt{|h^{-1}(y)|}}\sum_{x:h(x)=y}\ket{x}.\]

Then, the procedure $\Recover(\td,y,X)$ would measure register $X$ in the $\{\dyad{\psi_{h,y,0}}{\psi_{h,y,0}}, \bbI - \dyad{\psi_{h,y,0}}{\psi_{h,y,0}}\}$ basis, and output 0 if the first outcome is observed. We have that with probability $1-\negl(\secp)$ over the sampling of $h$,

\begin{align*}
    &\Pr_{x \gets \{0,1\}^m}\left[\Recover(\td,h(x),\ket{h,h(x),0}) \to 0\right] = 1,\\
    &\Pr_{x \gets \{0,1\}^m}\left[\Recover(\td,h(x),\ket{h,h(x),1}) \to 0\right] \leq (1-\delta)^2,
\end{align*}

by the proof of binding in \cref{thm:commitment}. This completes the proof.
%Given any superposition-invertible trapdoor almost-regular one-way function, by \cref{remark:invertible-hash} and \cref{impthm:balancedOWF} we can construct a superposition-invertible $\delta$-balanced function family $\cF$ that is one-way over its range. In turn, we can define superposition-invertible function family $\cH^\cF = \{H_\secp\}_{\secp \in \bbN}$ with associated algorithms $(\Samp,\Invert)$, and define predicate family $\cM = \{\{M[h]\}_{h \in H_\secp}\}_{\secp \in \bbN}$ as above.

\end{proof}

%Now, we let $\cD_{\mathsf{TCR}}$ denote the distribution over $(\Delta,y,b)$ seen by $\cA$ in the TCR experiment, and we let $\cD_{\mathsf{OW}}$ denote the distribution over $(\Delta,y,b)$ seen by $\cA$ in the one-wayness experiment. We will then use the following three claims to complete the proof. In what follows, for any $z \in \{0,1\}^n$, we define $p_z = \Pr_{x \gets \{0,1\}^m}[f(x) = z]$.

%\begin{claim}
%\[\Pr_{(\Delta,y,b) \gets \cD_{\mathsf{TCR}}}\left[z_0 \in \mathsf{BAD} \vee z_1 \in \mathsf{BAD} : \{z_0,z_1\} \coloneqq h_\Delta^{-1}(y)\right] = \negl(\secp).\]
%\end{claim}

%\begin{proof}
%For any fixed $\Delta$, which partitions $\{0,1\}^n$ into sets of two, we have that the probability of sampling a random pair with at least one element in $\mathsf{BAD}$ is at most
%\[|\mathsf{BAD}| \cdot \left(\frac{1+\delta}{2^n} + \negl(\secp)\right) = \frac{k}{2^n} + \negl(\secp) = \negl(\secp),\] by properties (1), (2), and (3) of \cref{def:balanced}.
%\end{proof}

%\begin{claim}
%\[\Pr_{(\Delta,y,b) \gets \cD_{\mathsf{OW}}}\left[z_0 \in \mathsf{BAD} \vee z_1 \in \mathsf{BAD} : \{z_0,z_1\} \coloneqq h_\Delta^{-1}(y)\right] = \negl(\secp).\]
%\end{claim}

%\begin{proof}
%This follows directly from property (2) of \cref{def:balanced} and a union bound.
%\end{proof}

%For any $(\Delta,y,b)$, define \[p_{(\Delta,y,b)}^{\mathsf{OW}} \coloneqq \Pr_{(\Delta^*,y^*,b^*) \gets \cD_{\mathsf{OW}}}[(\Delta,y,b) = (\Delta^*,y^*,b^*)], ~~~ p_{(\Delta,y,b)}^{\mathsf{TCR}} \coloneqq \Pr_{(\Delta^*,y^*,b^*) \gets \cD_{\mathsf{TCR}}}[(\Delta,y,b) = (\Delta^*,y^*,b^*)].\]

%\begin{claim}
%For each fixed $(\Delta,y,b)$ such that $h_{\Delta}^{-1}(y) \cap \mathsf{BAD} = \emptyset$, \[p_{(\Delta,y,b)}^{\mathsf{OW}} \geq (1-\delta) \cdot p_{(\Delta,y,b)}^{\mathsf{TCR}}.\]
%\end{claim}

%\begin{proof}
%We will calculate each of the two probabilities in the claim separately. Note first that each $(\Delta,y)$ defines a fixed $\{z_0,z_1\} \coloneqq h_{\Delta}^{-1}(y)$. Then we can write 
%\begin{align*}
%    p_{(\Delta,y,b)}^{\mathsf{OW}} = p_{z_{1-b}} \cdot p_{z_b},
%\end{align*}
%where $p_{z_{1-b}}$ is the probability that the challenger samples $z_{1-b}$ and $p_{z_b}$ is the probability that the reduction $\cA'$ samples $z_b$. Next, we can write 
%\begin{align*}
%    p_{(\Delta,y,b)}^{\mathsf{TCR}} = \frac{1}{2^n-1} \cdot (p_{z_0} + p_{z_1}) \cdot \frac{p_{z_b}}{p_{z_0} + p_{z_1}} = \frac{p_{z_b}}{2^n-1},
%\end{align*}
%which models sampling $\Delta$ uniformly at random from $\{0,1\}^n \setminus \{0^n\}$, measuring $y = h_\Delta(f(\cdot))$, and then measuring the bit $b$. Thus,

%\[p_{(\Delta,y,b)}^{\mathsf{OW}} /
%p_{(\Delta,y,b)}^{\mathsf{TCR}} \geq p_{1-z_b} \cdot (2^n-1) \geq 1-\delta,\]

%by property (3) of \cref{def:balanced}, which competes the proof of the claim.

%\end{proof}

%Now we complete the proof of the theorem by showing that $\cA'$ violates the one-wayness of $\cF$.

%\begin{align*}
    %\Pr[\Exp_{\mathsf{OW}} = 1] &\geq \sum_{\Delta,y,b}p_{(\Delta,y,b)}^{\mathsf{OW}} \cdot \Pr[f(x') = z : x' \gets \cA(\Delta,y,b,X)] \\
    %&\geq \sum_{\Delta,y,b : h_\Delta^{-1}(y) \cap \mathsf{BAD} = \emptyset}p_{(\Delta,y,b)}^{\mathsf{OW}} \cdot \Pr[f(x') = z : x' \gets \cA(\Delta,y,b,X)] - \negl(\secp) \\
    %&\geq \sum_{\Delta,y,b : h_\Delta^{-1}(y) \cap \mathsf{BAD} = \emptyset}(1-\delta) \cdot p_{(\Delta,y,b)}^{\mathsf{TCR}} \cdot \Pr[f(x') = z : x' \gets \cA(\Delta,y,b,X)] - \negl(\secp) \\
    %&\geq (1-\delta)\Pr[\Exp_{\mathsf{TCR}} = 1] - \negl(\secp) \\
    %&\geq \nonnegl(\secp).
%\end{align*}


\subsection{Balanced Binary-Measurement TCR from Pseudorandom Group Actions}
\label{sec:hmy}
Finally, we show that the recent public-key encryption scheme of \cite{HMY} based on pseudorandom group actions has publicly-verifiable deletion, which follows fairly immediately from our framework. First, we need some preliminaries from \cite{JQSY,HMY}.

%\dakshita{Here, can we instead just rely on the fact that there is a trapdoor that helps distinguish the superposition from the mixture? and relax the trapdoor property from Section 7.3 to only ask for distinguishing given a trapdoor (we can say that it is implied by superposition-invertibility..). This way we will just prove that the group action based scheme is an instance of the appropriate trapdoored collapsing function.}

\begin{definition}[Group Action] 
Let $G$ be a (not necessarily abelian) group, $S$ be a set, and $\star: G \times S \to S$ be a function where we write $g \star s$ to mean $\star(g,s)$. We say that $(G,S,\star)$ is a group action if it satisfies the following:
\begin{itemize}
    \item For the identity element $e \in G$ and any $s \in S$, we have $e \star s = s$.
    \item For any $g,h \in G$ and any $s \in S$, we have $(gh) \star s = g \star (h \star s)$.
\end{itemize}
\end{definition}

\cite{JQSY,HMY} also require a number of efficiency properties from the group action, and we refer the reader to their papers for these specifications.

\begin{definition}[Pseudorandom Group Action]
A group action $(G,S,\star)$ is \emph{pseudorandom} if it satisfies the following:
\begin{itemize}
    \item We have that \[\Pr_{s,t \gets S}[\exists g \in G \text{ s.t. } g \star s = t] = \negl(\secp).\]
    \item For any QPT adversary $\{\cA_\secp\}_{\secp \in \bbN}$,
    \[\big| \Pr_{s \gets S, g \gets S}[\cA_\secp(s,g \star s) = 1] - \Pr_{s,t \gets S}[\cA_\secp(s,t) = 1]\big| = \negl(\secp).\]
\end{itemize}
\end{definition}

Given a pseudorandom group action $(G,S,\star)$, \cite{HMY} consider the following hash family $\cH^{(G,S,\star)} = \{H_h\}_{h \in S_G}$, where $S_G = \{(s_0,s_1) \in S^2: \exists g \in G \text{ s.t. } s_1 = g \star s_0\}$.
\begin{itemize}
    \item The algorithm $\Samp(1^\secp)$ samples $s_0 \gets S,g \gets G$ and outputs $h = (s_0,s_1)$ as the description of the hash and $\td = g$ as the trapdoor.
    \item For an input $(b,k)$ where $b \in \{0,1\}$ and $k \in G$, define $h(b,k) \coloneqq k \star s_b$.
\end{itemize}



\begin{claim}
$\cH^{(G,S,\star)}$ is a balanced binary-measurement TCR hash with trapdoor phase-recoverability.
\end{claim}

\begin{proof}
Define predicate family $\cM$ as $M[h](b,k) = b$. That is, it does not depend on $h$, and simply outputs the first bit of its input. Then, this claim actually follows immediately from what is already proven in \cite{HMY}. First, \cite[Theorem 4.10]{HMY} shows that given $\td$ and $y \in S$, it is possible to perfectly distinguish \[\frac{1}{\sqrt{2}}\ket{0,h_0^{-1}(y)} + \frac{1}{\sqrt{2}}\ket{1,h_1^{-1}(y)} ~~ \text{and} ~~ \frac{1}{\sqrt{2}}\ket{0,h_0^{-1}(y)} - \frac{1}{\sqrt{2}}\ket{1,h_1^{-1}(y)},\] where $h_b \coloneqq h(b,\cdot)$, which establishes trapdoor phase-recoverability. Next, \cite[Theorem 4.19]{HMY} shows that $\cH^{(G,S,\star)}$ satisfies \emph{conversion hardness}, which is equivalent to our notion of $\cM$-target-collision-resistance. 
\end{proof}

By \cref{thm:PKE} and \cref{cor:compiler}, we obtain the following corollary.

\begin{corollary}
Assuming the existence pseudorandom group actions, there exists PKE with $\PVD$. Additionally assuming post-quantum \[X \in \left\{\begin{array}{r}\text{quantum fully-homormophic encryption, attribute-based encryption}, \\ \text{witness encryption, timed-release encryption}\end{array}\right\},\] there exists $X$ with $\PVD$.
\end{corollary}

%Now consider the following public-key encryption scheme from \cite{HMY}, where we have added the $\Del$ and $\Vrfy$ algorithms.

%\begin{itemize}
    %\item $\Gen(1^\secp)$: Sample $(h,\td) \gets \Samp(1^\secp)$ and set $\pk \coloneqq h, \sk \coloneqq \td$.
    %\item $\Enc(\pk,b)$: Prepare the state 
    %\[\frac{1}{\sqrt{2|G|}}\sum_{k \in G}\ket{0,k}_X\ket{h(0,k)}_Y + (-1)^b\ket{1,k}_X\ket{h(1,k)}_Y\] and measure the $Y$ register to obtain $y$ and a left-over state on register $X$. Then, set \[\ket{\ct} \coloneqq (y,X), ~~ \vk = (h,y).\]
    %\item $\Dec(\sk,\ket{\ct})$: The details of this algorithm are not important to us, so we refer the reader to \cite{HMY}. They demonstrate and prove a decryption algorithm that is correct (\cite[Theorem 4.10]{HMY}).
    %\item $\Del(\ket{\ct})$: Measure $\ket{\ct}$ in the standard basis to obtain $\pi \coloneqq x$.
    %\item $\Vrfy(\vk,\pi)$: Output $\top$ if $h(x) = y$.
%\end{itemize}

%\begin{theorem}
%If $(G,S,\star)$ is a pseudorandom group action, then the above scheme defined based on $\cH^{(G,S,\star)}$ is a PKE scheme with $\PVD$.
%\end{theorem}

%\begin{proof}
%Since \cite{HMY} showed standard semantic security, it suffices to show that $\cH^{(G,S,\star)}$ is certified everlasting $(\cU,\cM)$-target-collapsing, which will immediately imply certified deletion security. Now, it was shown in \cite[Theorem 4.19]{HMY} that $\cH^{(G,S,\star)}$ satisfies \emph{conversion hardness}, which is equivalent to our notion of $(\cU,\cM)$-target-collision-resistance. Thus, by \cref{thm:TC-from-TCR} and \cref{thm:CETC-generalization}, $\cH^{(G,S,\star)}$ is certified everlasting $(\cU,\cM)$-target-collapsing, which completes the proof.
%\end{proof}














%\input{group-actions}
%\chapter{Supporting Results}\label{chap:alg}


In this chapter we collect results which are called upon in the main body of the work, but which differ enough in character from our study of decompositions that they would distract from the central narrative of this thesis. With the potential exception of \Cref{thm:specialodds}, the results of this chapter are far from new -- we choose not to confine them to an appendix solely because we furnish proofs which could not be found in any references. There is no unified theme to this chapter, and accordingly the reader may wish to visit these sections as specific results are called upon elsewhere in the work.


\section{An Algebraic Result}


The first result we give is required in the proof of \Cref{thm:cauchylike}, and it ensures that for a given set of non-negative numbers, we can choose a set of weights whose odd powers enjoy the vanishing property of equation \eqref{eq:wowza}. This result is used to eliminate selective odd-order terms in the Taylor expansions of non-negative functions. 


\begin{thm}\label{thm:specialodds}
Let \(\ell\) be any odd integer and set \(s=\frac{\ell+1}{2}\). There exist numbers \(\eta_1,\dots,\eta_s\in\mathbb{R}\) and \(x_1,\dots,x_s\geq 0\) such that for each odd \(j\leq \ell\),
\begin{equation}\label{eq:wowza}
    \sum_{i=1}^sx_i\eta_i^j=\begin{cases}
    0 & j<\ell,\\
    1 & j=\ell.
    \end{cases}
\end{equation}
\end{thm}

\begin{proof}
It suffices to show that \(Mx=e_s\) has a non-negative solution \(x\), where \(e_s\) is the last standard basis vector in \(\mathbb{R}^s\) and the matrix \(M\) is given by
\[
    M=\begin{bmatrix}
    \eta_1&\eta_2&\cdots &\eta_s\\
    \eta_1^3&\eta_2^3&\cdots&\eta_s^{3}\\
    \vdots&\vdots&\ddots&\vdots\\
    \eta_1^\ell &\eta_2^\ell &\cdots&\eta_s^\ell 
    \end{bmatrix}.
\]
Our aim is to select \(\eta_1,\dots,\eta_s\) so that \(M\) is invertible and the entries of \(M^{-1}e_s\) are non-negative. Denoting by \(x_k\) the \(k^\mathrm{th}\) entry in column \(s\) of \(M^{-1}\), and assuming for now that \(M\) is nonsingular, we can write
\[
    x_k=\frac{(-1)^{s+k}\mathrm{det}(M_{sk})}{\mathrm{det}(M)},
\]
where \(M_{sk}\) is the matrix minor of \(M\) obtained by deleting row \(s\) and column \(k\) of \(M\). 
    

Using the properties of \(M\), we establish a formula for \(\mathrm{det}(M)\). Note that this is a homogeneous polynomial in the variables \(\eta_1,\dots,\eta_s\) of degree \(s^2\). If \(\eta_i=0\) for any \(i\) then \(M\) is singular, meaning that \(\eta_i\mid \mathrm{det}(M)\). Similarly, if \(i\neq j\) and \(\eta_i=\pm\eta_j\) then \(M\) has two linearly dependent columns, meaning once again that \(\mathrm{det}(M)=0\) and \((\eta_i^2-\eta_j^2)\mid \mathrm{det}(M)\) whenever \(i\neq j\). It follows that \(\mathrm{det}(M)=PQ\) where \(P\) and \(Q\) are polynomials and \(P\) is given by
\[
    P=\bigg(\prod_{i=1}^s\eta_i\bigg)\bigg(\prod_{i=1}^s\prod_{j=1}^{i-1}(\eta_i^2-\eta_j^2)\bigg).
\]
This polynomial has degree \(s^2\), and it follows from the Fundamental Theorem of Algebra that \(Q\) has degree zero and is constant. Moreover, the first term in the expansion of \(P\) equals the product down the diagonal of \(M\). Comparing coefficients, we see that \(Q=1\) and \(\mathrm{det}(M)=P\).


We can compute \(\mathrm{det}(M_{sk})\) with the formula above, since this minor assumes the same form as \(M\). Omitting appropriate terms from the determinant formula for \(M\), we get
\[
    \mathrm{det}(M_{sk})=\bigg(\prod_{\substack{1\leq i\leq s\\
    i\neq k}}\eta_i\bigg)\bigg(\prod_{\substack{1\leq j< i\leq s\\i,j\neq k}}(\eta_i^2-\eta_j^2)\bigg)
\]
Equipped with this identity and the formula for \(x_k\) above, we are now able to write
\[
    x_k=(-1)^{s+k}\bigg(\eta_k\prod_{j=1}^{k-1}(\eta_k^2-\eta_j^2)\prod_{i=k+1}^s(\eta_i^2-\eta_k^2)\bigg)^{-1}=\bigg(\eta_k\prod_{\substack{1\leq i\leq s\\i\neq k}}(\eta_k^2-\eta_i^2)\bigg)^{-1}
\]
To ensure that each \(x_k\) is positive, it suffices to choose \(\eta_1,\dots,\eta_s\) so that \(|\eta_1|<\cdots<|\eta_s|\) and \(\mathrm{sgn}(\eta_k)=(-1)^{s+k}\). Therefore taking \(\eta_k=(-1)^{s+k}k\) gives \eqref{eq:wowza}.
\end{proof}


Our choice of \(\eta_k\) above suffices for the purpose of proving \Cref{thm:cauchylike}, but other choices may be better suited to other applications.


As one example, the result above might be useful for understanding the coefficients of non-negative polynomials over \(\mathbb{R}\) via a similar argument to that employed in proving \Cref{thm:cauchylike}. In particular, let
\[
    P(x)=\sum_{j=0}^dc_jx^j
\]
be a non-negative polynomial on \(\mathbb{R}\) of even degree \(d\), fix \(\ell=d-1\), and choose constants \(x_1,\dots,x_s\) and \(\eta_1,\dots,\eta_s\) as in \eqref{eq:wowza}. Since \(x_iP(-\eta_ix)\geq 0\) for each \(i\) we have for all \(x\in\mathbb{R}\) that
\[
    0\leq \sum_{i=1}^sx_iP(-\eta_ix)=\sum_{i=1}^sx_i\sum_{j=0}^dc_j(-\eta_i)^jx^j=\sum_{\substack{0\leq j\leq d,\\j\textrm{even}}}^dc_jx^j\bigg(\sum_{i=1}^sx_i\eta_i^j\bigg)-c_\ell x^\ell
\]
Replacing \(x_iP(-\eta_ix)\) with \(x_iP(\eta_ix)\) and repeating this argument, we find after rearranging and taking a maximum that
\[
    |c_\ell|\leq \sum_{\substack{0\leq j\leq d,\\j\textrm{even}}}^dc_jx^{j-\ell}\bigg(\sum_{i=1}^sx_i\eta_i^j\bigg).
\]


By bounding the \(s\)-dependant terms in the inequality above and optimizing over \(x\), it is possible to obtain a bound on \(c_\ell\). Iterating this argument also affords bounds on all other odd-order coefficients of \(P\). Moreover, by using a complete Vandermonde matrix in \Cref{thm:specialodds} instead of one omitting even rows, the interested reader might also glean information about the sums in \eqref{eq:wowza} for even values of \(j\), and thereby recover explicit bounds in the preceding application and \Cref{thm:cauchylike}.


\section{Useful Estimates}


In this short section we establish some straightforward results from elementary real analysis relating to properties of the maximum, the supremum, and the triangle inequality. Their proofs are not difficult, but the results warrant justification which we include here. Our first lemma states a simple property of the maximum.


\begin{lem}\label{lem:maxproplem}
For \(m\in\mathbb{N}\) let \(a_1,\dots,a_m\) and \(b_1,\dots,b_m\) be real numbers. Then
\begin{equation}\label{eq:maxprop}
    |\max_{j\leq m}a_j-\max_{j\leq m}b_j|\leq \max_{j\leq m}|a_j-b_j|.
\end{equation}
\end{lem}


\begin{proof}
We use induction on \(m\), proving the non-trivial case \(m=2\) as our base case. Note that
\[
    \max\{a_1,b_1\}=\max\{a_1-a_2+a_2,b_1-b_2+b_2\}\leq \max\{a_1-a_2,b_1-b_2\}+\max\{a_2,b_2\}.
\]
Rearranging this, we see that \(\max\{a_1,b_1\}-\max\{a_2,b_2\}\leq \max\{a_1-a_2,b_1-b_2\}\), and the same inequality clearly holds when \(a_1\) is swapped with \(a_2\) and \(b_1\) with \(b_2\). Consequently,
\begin{align*}
    |\max\{a_1,b_1\}-\max\{a_2,b_2\}|&=\max\{\max\{a_1,b_1\}-\max\{a_2,b_2\},\max\{a_2,b_2\}-\max\{a_1,b_1\}\}\\
    &\leq \max\{\max\{a_1-a_2,b_1-b_2\},\max\{a_2-a_1,b_2-b_1\}\}\\
    &=\max\{a_1-a_2,b_1-b_2,a_2-a_1,b_2-b_1\}\\
    &=\max\{|a_1-a_2|,|b_1-b_2|\}.
\end{align*}
This establishes the base case. Suppose next that inequality \eqref{eq:maxprop} holds for \(m\) terms. Using the preceding argument we obtain the bound
\begin{align*}
    |\max_{j\leq m+1}a_j-\max_{j\leq m+1}b_j|&=|\max\{\max_{j\leq m}a_j,a_{m+1}\}-\max\{\max_{j\leq m}b_j,b_{m+1}\}|\\
    &\leq \max\{|\max_{j\leq m}a_j-\max_{j\leq m}b_j|,|a_{m+1}-b_{m+1}|\},
\end{align*}
and with this it follows from our inductive hypothesis that
\begin{align*}
    |\max_{j\leq m+1}a_j-\max_{j\leq m+1}b_j|\leq \max\{ \max_{j\leq m}|a_j-b_j|,|a_{m+1}-b_{m+1}|\}= \max_{j\leq m+1}|a_j-b_j|  .  
\end{align*}
Hence inequality \eqref{eq:maxprop} holds with \(m+1\) terms, and the claimed estimates follow by induction.
\end{proof}


Now we establish a similar property for the supremum.


\begin{lem}\label{lem:supprop}
For any non-negative function \(g:\mathbb{R}^n\times\mathbb{R}^n\rightarrow\mathbb{R}\) and for any two points \(x,y\in\mathbb{R}^n\),
\[
    |\sup_{\xi\in\mathbb{R}^n} g(x,\xi)-\sup_{\xi\in\mathbb{R}^n} g(y,\xi)|\leq \sup_{\xi\in\mathbb{R}^n} |g(x,\xi)-g(y,\xi)|.
\]
\end{lem}

\begin{proof}
For any non-negative function \(g\) we can use subadditivity of the supremum to get
\[
    \sup_{\xi\in\mathbb{R}^n} g(x,\xi)=\sup_{\xi\in\mathbb{R}^n} (g(x,\xi)-g(y,\xi)+g(y,\xi))\leq \sup_{\xi\in\mathbb{R}^n} |g(x,\xi)-g(y,\xi)|+\sup_{\xi\in\mathbb{R}^n} g(y,\xi).
\]
Therefore we can rearrange to get
\[
    \sup_{\xi\in\mathbb{R}^n} g(x,\xi)-\sup_{\xi\in\mathbb{R}^n} g(y,\xi)\leq \sup_{\xi\in\mathbb{R}^n} |g(x,\xi)-g(y,\xi)|.
\]  
An identical argument interchanging \(y\) and \(x\) shows that 
\[
    \sup_{\xi\in\mathbb{R}^n} g(y,\xi)-\sup_{\xi\in\mathbb{R}^n} g(x,\xi)\leq \sup_{\xi\in\mathbb{R}^n} |g(x,\xi)-g(y,\xi)|
\]
The claimed estimate then follows from taking a maximum.
\end{proof}


Finally, we establish a variant of the triangle inequality for products of functions.


\begin{lem}\label{lem:proddiff}
Let \(f_1,\dots,f_k\) be bounded functions on \(\mathbb{R}^n\), none of which is identically zero. Then for any two points \(x,y\in\mathbb{R}^n\),
\[
    \bigg|\prod_{j=1}^kf_j(x)-\prod_{j=1}^kf_j(y)\bigg|\leq \bigg(\prod_{j=1}^k\sup_{\mathbb{R}^n}|f_j|\bigg)\sum_{j=1}^k\frac{|f_j(x)-f_j(y)|}{\sup_{\mathbb{R}^n}|f_j|}.
\]
\end{lem}


\begin{proof}
We argue by induction on \(k\). There is nothing to show when \(k=1\), and when \(k=2\) we use the triangle inequality to write
\[
    |f_1(x)f_2(x)-f_1(y)f_2(y)|\leq |f_1(x)||f_2(x)-f_2(y)|+|f_2(y)||f_1(x)-f_1(y)|.
\]
Upon taking a supremum of \(|f_1(x)|\) and \(|f_2(y)|\), we get the desired estimate when \(k=2\). Suppose next that the estimate holds for \(k\) functions. Using the triangle inequality, we have
\begin{align*}
    \bigg|\prod_{j=1}^{k+1}f_j(x)-&\prod_{j=1}^{k+1}f_j(y)\bigg|=\bigg|f_{k+1}(x)\prod_{j=1}^{k}f_j(x)-f_{k+1}(y)\prod_{j=1}^{k}f_j(y)\bigg|\\
    &=\bigg|f_{k+1}(x)\prod_{j=1}^{k}f_j(x)-f_{k+1}(y)\prod_{j=1}^{k}f_j(x)+f_{k+1}(y)\prod_{j=1}^{k}f_j(x)-f_{k+1}(y)\prod_{j=1}^{k}f_j(y)\bigg|\\
    &\leq \bigg(\prod_{j=1}^{k}|f_j(x)|\bigg)|f_{k+1}(x)-f_{k+1}(y)|+|f_{k+1}(y)|\bigg|\prod_{j=1}^{k}f_j(x)-\prod_{j=1}^{k}f_j(y)\bigg|.
\end{align*}
Applying the inductive hypothesis, we see now that
\[
    \bigg|\prod_{j=1}^{k+1}f_j(x)-\prod_{j=1}^{k+1}f_j(y)\bigg|\leq \bigg(\prod_{j=1}^{k}|f_j(x)|\bigg)|f_{k+1}(x)-f_{k+1}(y)|+\bigg(\prod_{j=1}^{k+1}\sup_{\mathbb{R}^n}|f_j|\bigg)\sum_{j=1}^k\frac{|f_j(x)-f_j(y)|}{\sup_{\mathbb{R}^n}|f_j|}.
\]
The required estimate then follows by taking a supremum in the product on the right, completing the inductive step and showing that the claimed estimate holds for all \(k\) as claimed. 
\end{proof}

It is easy to see that the preceding results still hold if \(\mathbb{R}^n\) is replaced with any domain \(\Omega\subseteq\mathbb{R}^n\).


\section{H\"older Continuous Functions}\label{sec:holdexamples}


Our main results for H\"older spaces would not be very useful if there were no interesting functions to which they could apply. In this section we provide some concrete (and not completely trivial) examples of \(\alpha\)-H\"older continuous functions for various values of \(\alpha\). In doing so, we are also able to illustrate various techniques useful in proving H\"older continuity.


We begin with a straightforward and well-known example of an \(\alpha\)-H\"older continuous function.

\begin{clm}
If \(\alpha\in (0,1]\) then \(f(x)=|x|^\alpha\) defined for \(x\in\mathbb{R}^n\) belongs to \(C^\alpha(\mathbb{R}^n)\), and \([f]_{\alpha,\mathbb{R}^n}=1\).
\end{clm}

\begin{proof}
To verify this, we assume without loss of generality that \(x,y\in\mathbb{R}^n\) and \(|x|>|y|\), so that 
\[
    0<1-\bigg(\frac{|y|}{|x|}\bigg)^\alpha\leq 1-\frac{|y|}{|x|}\leq \bigg(1-\frac{|y|}{|x|}\bigg)^\alpha
\] 
since \(\alpha\leq 1\) and \(\frac{|y|}{|x|}<1\) by assumption. From the estimates above it follows now that
\[
    \frac{|f(x)-f(y)|}{|x-y|^\alpha}\leq  \frac{||x|^\alpha-|y|^\alpha|}{||x|-|y||^\alpha}=\frac{1-(\frac{|y|}{|x|})^\alpha}{(1-\frac{|y|}{|x|})^\alpha}\leq \frac{1-\frac{|y|}{|x|}}{1-\frac{|y|}{|x|}}=1.
\]
The same estimate clearly holds if we interchange \(x\) and \(y\). Since \(x\) and \(y\) were arbitrary, it follows from the estimate above that \([f]_{\alpha,\mathbb{R}^n}\leq 1\). Moreover this holds with equality, for if we assume that \([f]_\alpha<1\), then we would have \(|x|^\alpha=|f(x)|\leq [f]_{\alpha,\mathbb{R}^n}|x|^\alpha<|x|^\alpha\), a contradiction for \(x\neq 0\). It follows that \([f]_{\alpha,\mathbb{R}^n}=1\), as claimed.
\end{proof}


Next we move on to a more interesting example: Cantor's `Middle Thirds' function, sometimes colloquially referred to as the `Devil's Staircase.' This is famously a continuous bijection on \([0,1]\) whose derivative is zero except on a set of measure zero; as such, it serves as a pathological counterexample to the intuitive but incorrect idea that a continuous increasing function should be increasing on some interval. 


The Cantor function, which we denote by \(F\), is a fractal curve which turns out to be H\"older continuous. Additionally, it is proved in \cite[Chapter 9]{Teschl} that by taking \(f_0(x)=x\) on \([0,1]\) and defining \(f_{n+1}(x)=Tf_n(x)\), where \(T\) is the transformation
\[
    Tf(x)=\begin{cases}
    \hfil\frac{1}{2}f(3x) & 0\leq x\leq \frac{1}{3},\\
    \hfil\frac{1}{2} & \frac{1}{3}<x<\frac{2}{3},\\
    \frac{1}{2}(1+f(3x-2)) & \frac{2}{3}\leq x\leq 1,
    \end{cases}
\]
then we obtain a sequence of functions \(\{f_n\}\) which converge uniformly to \(F\) on \([0,1]\). By solving an exercise in \cite{Teschl}, we furnish a proof to the following well-known property of \(F\).


\begin{clm}
The Cantor function \(F\) belongs to \(C^{\log_32}([0,1])\) and \([F]_{\log_32,[0,1]}\leq1\).
\end{clm}

\begin{proof} Given a continuous bijection \(f\) of \([0,1]\) such that \(f(0)=0\), \(f(1)=1\) and \(f\in C^{\alpha}([0,1])\) for \(0<\alpha\leq 1\), we first examine the H\"older continuity of \(Tf\). In particular, we show that
\[
    [Tf]_\alpha\leq \frac{3^\alpha[f]_\alpha}{2}.
\]
This is done by considering six exhaustive cases which depend on the locations of \(x,y\in[0,1]\).

\noindent\underline{\textit{Case 1}}: If \(0\leq x\leq \frac{1}{3}\) and \(0\leq y\leq \frac{1}{3}\) then
\[
    |Tf(x)-Tf(y)|=\frac{1}{2}|f(3x)-f(3y)|\leq \frac{[f]_\alpha}{2}|3x-3y|^\alpha=\frac{3^\alpha[f]_\alpha}{2}|x-y|^\alpha.
\]
\noindent\underline{\textit{Case 2}}: If \(0\leq x\leq \frac{1}{3}\) and \(\frac{1}{3}< y<\frac{2}{3}\) then
\[
    |Tf(x)-Tf(y)|=\frac{1}{2}|f(3x)-1|=\frac{1}{2}|f(3x)-f(1)|\leq \frac{[f]_\alpha}{2}|3x-1|^\alpha\leq \frac{3^\alpha[f]_\alpha}{2}|x-y|^\alpha,
\]
where the last inequality holds since \(|3x-1|=3|x-\frac{1}{3}|\leq 3|x-y| \).


\noindent\underline{\textit{Case 3}}: If \(0\leq x\leq \frac{1}{3}\) and \(\frac{2}{3}\leq y\leq 1\) then 
\[
    |Tf(x)-Tf(y)|=\frac{1}{2}|f(3x)-f(1)+f(0)-f(3y-2)|\leq \frac{3^\alpha[f]_\alpha}{2}\bigg|x-\frac{1}{3}\bigg|^\alpha+\frac{3^\alpha[f]_\alpha}{2}\bigg|y-\frac{2}{3}\bigg|^\alpha.
\]
To get the required inequality, we must show that \(|x-\frac{1}{3}|^\alpha+|y-\frac{2}{3}|^\alpha\leq |x-y|^\alpha\) for \(\alpha\) sufficiently large. To this end we observe that for \(\alpha\leq 1\) the function \(Q:[0,\frac{1}{3}]\times[\frac{2}{3},1]\rightarrow\mathbb{R}\) defined by
\[
    Q(x,y)=\bigg|\frac{x-\frac{1}{3}}{x-y}\bigg|^\alpha+\bigg|\frac{y-\frac{2}{3}}{x-y}\bigg|^\alpha
\]
achieves a unique global maximum at \((0,1)\). It follows that on \([0,\frac{1}{3}]\times[\frac{2}{3},1]\) we have the bound \( Q(x,y)\leq Q(0,1)=\frac{2}{3^\alpha}\). If \(\alpha\geq\log_32\) then we see that \(Q(x,y)\leq \frac{2}{3^\alpha}\leq 1\) in the domain of interest, meaning that again,
\[
    |Tf(x)-Tf(y)|\leq \frac{3^\alpha[f]_\alpha}{2}|x-y|^\alpha.
\]
\underline{\textit{Case 4}}: If \(\frac{1}{3}<x<\frac{2}{3}\) and \(\frac{1}{3}<y<\frac{2}{3}\) then
\[
    |Tf(x)-Tf(y)|=\bigg|\frac{1}{2}-\frac{1}{2}\bigg|=0\leq \frac{3^\alpha[f]_\alpha}{2}|x-y|^\alpha.
\]
\noindent\underline{\textit{Case 5}}: If \(\frac{1}{3}<x<\frac{2}{3}\) and \(\frac{2}{3}\leq y\leq 1\) then
\[
    |Tf(x)-Tf(y)|=\frac{1}{2}|1-(1+f(3y-2))|=\frac{1}{2}|f(3y-2)-f(0)|\leq \frac{[f]_\alpha}{2}|3y-2|^\alpha\leq\frac{3^\alpha[f]_\alpha}{2}|x-y|^\alpha.
\]
\noindent\underline{\textit{Case 6}}: If \(\frac{2}{3}\leq y\leq 1\) and \(\frac{2}{3}\leq y\leq 1\) then
\[
    |Tf(x)-Tf(y)|=\frac{1}{2}|f(3x-2)-f(3y-2)|\leq \frac{[f]_\alpha}{2}|3x-3y|^\alpha=\frac{3^\alpha[f]_\alpha}{2}|x-y|^\alpha.
\]


Thus, if \(\log_32\leq\alpha\leq 1\) then for \(x,y\in[0,1]\) we have the following H\"older estimate on \(Tf\),
\[
    |Tf(x)-Tf(y)|\leq \frac{3^\alpha[f]_\alpha}{2}|x-y|^\alpha.
\]
Equipped with this property of \(T\), we now observe that if we fix \(\alpha=\log_32\) and if \([f]_\alpha\leq1\), then \(|Tf(x)-Tf(y)|\leq |x-y|^\alpha\) and \([Tf]_\alpha\leq 1\). The function \(f_0(x)=x\) satisfies \([f_0]_\alpha\leq 1\) since for \(x,y\in[0,1]\) we have \(|x-y|\leq 1\) and 
\[
    |f_0(x)-f_0(y)|=|x-y|=|x-y|^{1-\alpha}|x-y|^\alpha\leq |x-y|^\alpha.
\]
Consequently with \(\alpha=\log_32\) we have \([f_1]_\alpha=[Tf_0]_\alpha\leq 1\), and it follows by induction that \([f_n]_\alpha\leq 1\) for every \(n\). Using this estimate, we see that for every \(n\) the Cantor function \(F\) satisfies
\[
    |F(x)-F(y)|\leq |F(x)-f_n(x)|+|F(y)-f_n(y)|+|x-y|^\alpha.
\]
The left-hand side is independent of \(n\), and since the functions \(f_n\) converge to \(F\) pointwise we have that \(|F(x)-f_n(x)|\rightarrow0\) as \(n\rightarrow\infty\) for each \(x\in[0,1]\). It follows in the limit that
\[
    |F(x)-F(y)|\leq|x-y|^\alpha
\]
for every \(x,y\in[0,1]\), meaning that \(F\in C^{\log_32}([0,1])\) as we wished to show.
\end{proof}


This approach to computing H\"older continuity for fractal curves can be applied more generally. In particular, it suffices to find an appropriate map \(T\) corresponding to a given fractal curve, and to employ the technique used above to show that \([Tf]_\alpha\leq [f]_\alpha\) for some \(\alpha\). The map in question is most easily deduced from symmetries of the fractal curve in question. 

One such fractal curve is the Minkowski `Question Mark' function. This is often denoted `\(?\)' in the literature, however to avoid confusion (no pun intended) we denote this function by \(Q\). It has fixed points at \(0\) and \(1\) and it enjoys the following self-similarity relations for \(x\leq \frac{1}{2}\),
\begin{align*}
    &Q\bigg(\frac{x}{1+x}\bigg)=\frac{1}{2}Q(x),\\
    &Q(x)+Q(1-x)=1,
\end{align*}
see \cite{ALKAUSKAS_2009}. Salem shows in \cite{salem} that \(Q\in C^{\alpha}([0,1])\), with the H\"older exponent \(\alpha\) given by
\[
    \alpha=\frac{\log 2}{2\log(\frac{1+\sqrt{5}}{2})}.
\]
The proof of this result in \cite{salem} uses an altogether different technique from that employed above, but we note that owing to the symmetries listed above, \(Q\) is a fixed point of the map
\[
    Tf(x)=\begin{cases}
    \hfil\frac{1}{2}f(\frac{x}{1-x}) & x\leq\frac{1}{2}\\
    \frac{1}{2}+\frac{1}{2}f(\frac{2x-1}{x}) & x>\frac{1}{2}
    \end{cases}
\]
which preserves bijections on \([0,1]\). By taking \(f_0(x)=x\) and defining \(f_{n+1}=Tf_n\) as we did for the Cantor function, we obtain a sequence of functions which converge uniformly to \(Q\) since \(T\) is a contraction mapping -- we direct the interested reader to \cite{bezier} for a thorough investigation of this idea. Further, we close by noting that a similar approach can be used to study (and for our purposes, compute the H\"older continuity of) other fractals like the Blancmange curve.




\printbibliography

\newpage 
\appendix




\end{document}
