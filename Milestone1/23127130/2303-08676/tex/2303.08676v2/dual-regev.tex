\section{Publicly-Verifiable Deletion from Dual-Regev Encryption}\label{sec:Dual-Regev}

In this section, we recall the constructions of Dual-Regev public-key encryption as well as fully homomorphic encryption with publicly-verifiable deletion introduced by Poremba~\cite{Poremba22}. Using our main result on certified-everlasting target-collapsing hashes in \Cref{thm:CETC-generalization}, we prove the \emph{strong Gaussian-collapsing conjecture} in~\cite{Poremba22}, and then conclude that the aforementioned constructions achieve certified-everlasting security assuming the quantum hardness of $\LWE$ and $\SIS$.

First, let us recall the definition of public-key encryption with publicly-verifiable deletion. 

\subsection{Definition: Encryption with Publicly-Verifiable Deletion}

A public-key encryption (PKE) scheme with publicly-verifiable deletion (PVD) has the following syntax.

\begin{itemize}
    \item $\KeyGen(1^\secp) \to (\pk,\sk)$: the key generation algorithm takes as input the security parameter $\secp$ and outputs a public key $\pk$ and secret key $\sk$.
    \item $\Enc(\pk,m) \to (\vk,\ket{\ct})$: the encryption algorithm takes as input the public key $\pk$ and a plaintext $m$, and outputs a (public) verification key $\vk$ and a ciphertext $\ket{\ct}$.
    \item $\Dec(\sk,\ket{\ct}) \to m$: the decryption algorithm takes as input the secret key $\sk$ and a ciphertext $\ket{\ct}$ and outputs a plaintext $m$.
    \item $\Del(\ket{\ct}) \to \pi$: the deletion algorithm takes as input a ciphertext $\ket{\ct}$ and outputs a deletion certificate $\pi$.
    \item $\Vrfy(\vk,\pi) \to \{\top,\bot\}$: the verify algorithm takes as input a (public) verification key $\vk$ and a proof $\pi$, and outputs $\top$ or $\bot$.
\end{itemize}

\begin{definition}[Correctness of deletion]\label{def:correctness-deletion}
A PKE scheme with PVD satisfies \emph{correctness of deletion} if for any $m$, it holds with $1-\negl(\secp)$ probability over $(\pk,\sk) \gets \Gen(1^\secp), (\vk,\ket{\ct}) \gets \Enc(\pk,m),\pi \gets \Del(\ket{\ct}),\mu \gets \Vrfy(\vk,\pi)$ that $\mu = \top$.
\end{definition}

\begin{definition}[Certified deletion security]\label{def:security-deletion}
A PKE scheme with PVD satisfies \emph{certified deletion security} if it satisfies standard semantic security, and moreover, for any QPT adversary $\{\cA_\secp\}_{\secp \in \bbN}$, it holds that 
\[\TD\left(\mathsf{EvPKE}_{\cA,\secp}(0),\mathsf{EvPKE}_{\cA,\secp}(1)\right) = \negl(\secp),\] where the experiment $\mathsf{EvPKE}_{\cA,\secp}(b)$ is defined as follows.
\begin{itemize}
    \item Sample $(\pk,\sk) \gets \Gen(1^\secp)$ and $(\vk,\ket{\ct}) \gets \Enc(\pk,b)$.
    \item Run $\cA_\secp(\pk,\vk,\ket{\ct})$, and parse their output as a deletion certificate $\pi$ and a left-over quantum state $\rho$.
    \item If $\Vrfy(\vk,\pi) = \top$, output $\rho$, and otherwise output $\bot$.
\end{itemize}
\end{definition}

\ \\
Before we introduce the
Dual-Regev public-key schemes proposed by Poremba~\cite{Poremba22}, let us first recall some basic facts about
Gaussian superpositions.

\subsection{Gaussian 
Superpositions}

Given a modulus $q \in \N$, and integer $m \in \N$ and a parameter $\sigma \in (\sqrt{2m},q/\sqrt{2m})$, we consider
Gaussian superposition states over the finite set $\Z^m \cap (-\frac{q}{2},\frac{q}{2}]^m$ of the form
    $$
 \ket{\psi} =    \sum_{\vec x \in \Z_q^m} \rho_\sigma(\vec x) \ket{\vec x}.
    $$
The state $\ket{\psi}$ is not normalized for convenience. A standard tail bound~\cite[Lemma 1.5 (ii)]{Banaszczyk1993} implies that (the normalized variant of) $\ket{\psi}$ is within negligible trace distance of a \emph{truncated} discrete Gaussian superposition $ \ket{\tilde{\psi}}$
with support $\{\vec x \in \Z_q^m : \|\vec x\| \leq \sigma \sqrt{\frac{m}{2}}\}$, where
$$
\ket{\tilde{\psi}}
= 
\left(\sum_{\vec z \in \Z_q^m,\|\vec z\| \leq \sigma \sqrt{\frac{m}{2}} } \rho_{\frac{\sigma}{\sqrt{2}}}(\vec z) \right)^{-\frac{1}{2}}\sum_{\vec x \in \Z_q^m : \|\vec x\| \leq \sigma \sqrt{\frac{m}{2}}}
\rho_\sigma(\vec x) \ket{\vec x}.
$$
Note that a measurement of $\ket{\tilde{\psi}}$ results in a sample from the truncated discrete Gaussian distribution $D_{\Z_q^m,\frac{\sigma}{\sqrt{2}}}$.
We remark that Gaussian superpositions with parameter $\sigma = \Omega(\sqrt{m})$ can be efficiently implemented using standard quantum state preparation techniques; for example using \emph{quantum rejection sampling} and the \emph{Grover-Rudolph algorithm}~\cite{Grover2002CreatingST,Regev05,Brakerski18,brakerski2021cryptographic}.

Let $\vec A \in \Z_q^{n \times m}$ be a matrix. We use the following algorithm, denoted by $\mathsf{GenGauss}(\vec A,\sigma)$ which prepares a partially measured Gaussian superposition of pre-images of a randomly generated image.
\begin{enumerate}
\item Prepare a Gaussian superposition in system $X$ with parameter $\sigma > 0$:
    $$
 \ket{\psi} =    \sum_{\vec x \in \Z_q^m} \rho_\sigma(\vec x) \ket{\vec x} \otimes \ket{\vec 0}.
    $$
\item Apply the unitary $U_{\vec A}: \ket{\vec x}\ket{\vec 0} \rightarrow \ket{\vec x} \ket{\vec A \cdot \vec x \Mod{q}}$, which results in the state
$$
 \ket{ \psi} =   \sum_{\vec x \in \Z_q^m} \rho_\sigma(\vec x) \ket{\vec x} \otimes \ket{\vec A \cdot \vec x \Mod{q}}.
  $$
  \item Measure the second register in the computational basis, which results in $\vec y \in \Z_q^n$ and a state
    $$
    \ket{\psi_{\vec y}} = \sum_{\substack{\vec x \in \Z_q^m:\\ \vec A \vec x= \vec y \Mod{q}}} \rho_\sigma(\vec x) \ket{\vec x}.
    $$
\end{enumerate}

Finally, we use the following lemma which characterizes the Fourier transform of a partially measured Gaussian superposition.

\begin{lemma}[\cite{Poremba22}, Lemma 16]\label{lem:duality}
Let $m \in \N$, $q \geq 2$ be a prime and $\sigma \in (\sqrt{2m},q/\sqrt{2m})$.
Let $\vec A \in \Z_q^{n \times m}$ be a matrix whose columns generate $\Z_q^n$ and let $\vec y \in \Z_q^n$ be arbitrary. Then, the $q$-ary quantum Fourier transform of the (normalized variant of the) Gaussian coset state
$$
 \ket{\psi_{\vec y}} = \sum_{\substack{\vec x \in \Z_q^m\\ \vec A \vec x = \vec y \Mod{q}}}\rho_{\sigma}(\vec x) \ket{\vec x}
$$
is within negligible (in $m \in \N$) trace distance of the (normalized variant of the) Gaussian state
$$
 \ket{\hat\psi_{\vec y}} = \sum_{\vec s \in \Z_q^n} \sum_{\vec e \in \Z_q^m} \rho_{q/\sigma}(\vec e) \, \omega_q^{-\langle\vec s,\vec y \rangle} \ket{\vec s^\intercal \vec A + \vec e^\intercal \Mod{q}}.
$$
\end{lemma}


\subsection{(Strong) Gaussian-Collapsing Property.}

We use the following result which says that the Ajtai hash function is target-collapsing with respect to the truncated discrete Gaussian distribution.

\begin{theorem}[Gaussian-collapsing property, \cite{Poremba22}, Theorem 4]\label{thm:Gauss-collapsing}
Let $n\in \N$ and $q$ be a prime with $m \geq 2n \log q$, each parameterized by $\lambda \in \N$. Let  $\sigma \in (\sqrt{2m},q/\sqrt{2m})$.
Then,
the following samples are computationallyindistinguishable assuming the quantum hardness of decisional $\mathsf{LWE}_{n,q,\alpha q}^m$, for any noise ratio $\alpha \in (0,1)$ with relative noise magnitude $1/\alpha= \sigma \cdot 2^{o(n)}:$
$$
\Bigg(\vec  A \rand \Z_q^{n \times m},\,\, \ket{\psi_{\vec y}}=\sum_{\substack{\vec x \in \Z_q^m\\ \vec A \vec x = \vec y}}\rho_{\sigma}(\vec x) \,\ket{\vec x}, \,\,\vec y\in \Z_q^n \Bigg)\,\, \approx_c \,\,\,\, \Bigg(\vec  A \rand \Z_q^{n \times m}, \,\,\ket{\vec x_0},\,\, \vec A \cdot \vec x_0 \,\in \Z_q^n\Bigg)
$$
where $(\ket{\psi_{\vec y}},\vec y) \leftarrow \mathsf{GenGauss}(\vec A,\sigma)$ and where $\vec x_0 \sim D_{\Z_q^m,\frac{\sigma}{\sqrt{2}}}$ is a discrete Gaussian error.
\end{theorem}

Using our main theorem on certified-everlasting target-collapsing hashes in \Cref{thm:CETC-generalization}, we can now prove a stronger variant of \Cref{thm:Gauss-collapsing}. We show the following:

\begin{theorem}\label{thm:ajtai-certified-everlasting}
Let $\lambda \in \N$ be the security parameter, $n(\lambda) \in \N$, $q(\lambda) \in \N$ be a modulus, $m \geq 2n \log q$ and $\sigma \in (\sqrt{2m},q/\sqrt{2m})$. Then, the Ajtai hash function family
$\algo H = \{H_\lambda\}_{\lambda \in \N}$ with
$$
H_\lambda = \left\{ h_{\vec A}: \big\{\vec x \in \Z_q^m : \|\vec x\| \leq \sigma \sqrt{m/2}\big\} \rightarrow \Z_q^n \, \text{ s.t. } \, h_{\vec A}(\vec x) = \vec A \cdot \vec x \Mod{q}; \, \vec A \in \Z_q^{n \times m} \right\}.
$$
is certified everlasting $D_{\Z_q^m,\frac{\sigma}{\sqrt{2}}}$-target-collapsing assuming the quantum hardness of $\SIS_{n,q,\sigma\sqrt{2m}}^m$ and $\mathsf{LWE}_{n,q,\alpha q}^m$, for any noise ratio $\alpha \in (0,1)$ with relative noise magnitude $1/\alpha= \sigma \cdot 2^{o(n)}$.
\end{theorem}
\begin{proof}
By the Gaussian-collapsing property in \Cref{thm:Gauss-collapsing}, it follows that $\algo H$ is $D_{\Z_q^m,\frac{\sigma}{\sqrt{2}}}$-target-collapsing assuming the quantum hardness of $\mathsf{LWE}_{n,q,\alpha q}^m$, for any noise ratio $\alpha \in (0,1)$ with relative noise magnitude $1/\alpha= \sigma \cdot 2^{o(n)}$. Moreover, from the quantum hardness of $\SIS_{n,q,\sigma\sqrt{2m}}^m$ it follows that $\algo H$ is $D_{\Z_q^m,\frac{\sigma}{\sqrt{2}}}$-target-collision-resistant. Therefore, the claim follows from \Cref{thm:CETC-generalization}.
\end{proof}

As a corollary, we immediately recover the so-called strong Gaussian-collapsing property of the Ajtai hash function which was previously stated as a conjecture by Poremba~\cite{Poremba22}.

\begin{corollary}[Strong Gaussian-collapsing property]\label{SGC}\ \\
Let $\lambda \in \N$ be the security parameter, $n(\lambda) \in \N$, $q(\lambda) \in \N$ be a modulus and $m > 2n \log q$. Let $\sigma = \Omega(\sqrt{m})$ be a parameter. Then, the Ajtai hash function satisfies the strong Gaussian-collapsing property assuming the quantum hardness of $\SIS_{n,q,\sigma\sqrt{2m}}^m$ and $\mathsf{LWE}_{n,q,\alpha q}^m$, for any noise ratio $\alpha \in (0,1)$ with relative noise magnitude $1/\alpha= \sigma \cdot 2^{o(n)}$. In other words, for every $\QPT$ adversary $\algo A$,
$$
\mathsf{StrongGaussCollapseExpt}_{\algo A,n,m,q,\sigma}(0) \,\approx_c \, \mathsf{StrongGaussCollapseExpt}_{\algo A,n,m,q,\sigma}(1),
$$
where $\mathsf{StrongGaussCollapseExpt}_{\algo A,n,m,q,\sigma}(B)$ is the experiment from \Cref{fig:SGC}.
\end{corollary}
\begin{proof} 
To prove the statement, we can simply reduce the certified everlasting $D_{\Z_q^m,\frac{\sigma}{\sqrt{2}}}$-target-collapsing security of the Ajtai hash $\vec A = [\bar{\vec A} \, \| \, \bar{\vec A} \cdot \bar{\vec x} \Mod{q}] \in \Z_q^{n \times m}$ with $\bar{\vec x} \rand \bit^{m-1}$ to the
strong Gaussian-collapsing security, and invoke \Cref{thm:ajtai-certified-everlasting}. Here we rely on the fact that the distribution of $\vec A$ is statistically close to uniform by the leftover hash lemma whenever $m > 2n \log q$.
Recall that the adversary $\algo A = (\algo A_0,\algo A_1)$ in the certified-everlasting target-collapsing security experiment in \Cref{def:ev-target-collapsing} consists of a polynomially bounded algorithm $\algo A_0$ and an unbounded algorithm $\algo A_1$. Therefore, during the reduction, $\algo A_1$ can simply brute-force search for a trapdoor solution $\vec t = (\vec x,-1) \in \Z^{m}$ with $\vec x \in \bit^{m-1}$ such that $\bar{\vec A} \vec x = \bar{\vec A} \bar{\vec x} \Mod{q}$ given $\vec A \in \Z_q^{n \times m}$ as input. Note that such a vector exists because of how the matrix $\vec A$ is constructed in the experiment.
\end{proof}


\subsection{Dual-Regev Public-Key Encryption with Publicly-Verifiable Deletion}

We now consider the following Dual-Regev encryption scheme introduced by Poremba~\cite{Poremba22}.

\begin{construction}[Dual-Regev $\mathsf{PKE}$ with Publicly-Verifiable Deletion]\label{cons:dual-regev-cd}
Let $n \in \N$ be the security parameter, $m \in \N$ and $q$ be a prime. Let $\alpha \in (0,1)$ and $\sigma = 1/\alpha$ be parameters.
The Dual-Regev $\mathsf{PKE}$ scheme $\mathsf{DualPKECD} = (\KeyGen,\Enc,\Dec,\Del,\Vrfy)$ with certified deletion is defined as follows:
\begin{description}
\item $\KeyGen(1^\lambda) \rightarrow (\pk,\sk):$ sample a random matrix $\bar{\vec A} \rand \Z_q^{n\times m}$ and a vector $\bar{\vec x} \rand \bit^{m}$
and choose $\vec A = [\bar{\vec A} \| \bar{\vec A} \cdot \bar{\vec x} \Mod{q}]$.
Output $(\pk,\sk)$, where $\pk=\vec A \in \Z_q^{n \times (m+1)}$ and $\sk = (-\bar{\vec x}, 1) \in \Z_q^{m+1}$.
\item $\Enc(\pk,b) \rightarrow (\vk,\ket{\ct})$: parse the public key as $\vec A \leftarrow \pk$. To encrypt a single bit $b \in \bit$, generate the following pair for a random $\vec y \in \Z_q^n$:
$$
\vk \leftarrow (\vec A, \vec y), \quad \ket{\ct} \leftarrow \sum_{\vec s \in \Z_q^n} \sum_{\vec e \in \Z_q^{m+1}} \rho_{q/\sigma}(\vec e) \, \omega_q^{-\langle\vec s,\vec y \rangle}\ket{\vec s^\intercal \vec A + \vec e^\intercal +b \cdot (0,\dots,0, \lfloor\frac{q}{2} \rfloor)},
$$
where $\vk$ is the public verification key and $\ket{\ct}$ is an $(m+1)$-qudit quantum ciphertext.


\item $\Dec(\sk,\ket{\ct}) \rightarrow \bit:$ to decrypt, measure $\ket{\ct}$ in the computational basis with outcome $\vec c \in \Z_q^m$. Compute $\vec c^\intercal \cdot \sk \in \Z_q$ and output $0$, if it is closer to $0$ than to $\lfloor\frac{q}{2}\rfloor$, and output $1$, otherwise.

\item $\Del(\ket{\ct}) \rightarrow \pi:$ Measure $\ket{\ct}$ in the Fourier basis and output the measurement outcome $\pi \in \Z_q^{m+1}$.
\item $\Vrfy(\vk,\pi) \rightarrow \{\top,\bot\}:$ to verify a deletion certificate $\pi \in \Z_q^{m+1}$, parse $(\vec A,\vec y) \leftarrow \vk$ and output $\top$, if $\vec A \cdot \pi = \vec y \Mod{q}$ and $\| \pi \| \leq \sqrt{m+1}/\sqrt{2}\alpha$, and output $\bot$, otherwise.
\end{description}
\end{construction}


%%
Let us now illustrate how the deletion procedure takes place. Recall from \Cref{lem:duality} that the Fourier transform of the ciphertext $\ket{\ct}$ results in the \emph{dual} quantum state
\begin{align}\label{eq:dual-with-phase}
\ket{\widehat{\ct}}=\sum_{\substack{\vec x \in \Z_q^{m+1}:\\ \vec A \vec x = \vec y \Mod{q}}}\rho_{\sigma}(\vec x) \, \omega_q^{\langle\vec x,b \cdot (0,\dots,0,  \lfloor\frac{q}{2} \rfloor)\rangle} \,\ket{\vec x}.
\end{align}
In other words, a Fourier basis measurement of $\ket{\ct}$ necessarily erases all information about the plaintext $b \in \bit$ and results in a \emph{short} vector $\pi \in \Z_q^{m+1}$ such that $\vec A \cdot \pi = \vec y \Mod{q}$. Hence, to publicly verify a deletion certificate we can simply check whether it is a solution to the $\ISIS$ problem specified by the verification key $\vk=(\vec A,\vec y)$. Using \Cref{thm:ajtai-certified-everlasting}, we obtain the following:

\begin{theorem}
Let $n\in \N$ and let $q \geq 2$ be a prime modulus such that $q=2^{o(n)}$ and $m \geq 2n \log q$. Let $\sigma \in (\sqrt{2m},q/\sqrt{2m})$ and $\alpha \in (0,1)$ be a noise ratio with $1/\alpha= 2^{o(n)} \cdot \sigma$.
Then, the Dual-Regev public-key encryption scheme in \Cref{cons:dual-regev-cd} has everlasting certified deletion security assuming the quantum (subexponential) hardness of $\mathsf{LWE}_{n,q,\alpha q}^m$ and $\SIS_{n,q,\sigma\sqrt{2m}}^m$.
\end{theorem}
\begin{proof}
The proof is identical to the template used in~\cite[Theorem 7]{Poremba22}, except that the adversary is allowed to be computationally unbounded once the deletion certificate is submitted. This is in contrast with the original proof who considered forwarding the \emph{secret key} during the security experiment. We remark that we do not invoke the strong Gaussian-collapsing property to prove the indistinguishability of the hybrids; instead we use the (stronger) notion of certified everlasting $D_{\Z_q^m,\frac{\sigma}{\sqrt{2}}}$-target-collapsing property of the Ajtai hash shown in \Cref{thm:ajtai-certified-everlasting}. This results in the stronger notion of everlasting certified
deletion security.
\end{proof}



\subsection{Dual-Regev (Leveled) Fully Homomorphic Encryption with Publicly-Verifiable Deletion}

In this section, we recall the Dual-Regev (leveled) fully homomorphic encryption scheme with publicly-verifiable deletion introduced by Poremba~\cite{Poremba22}. The scheme is based on the \emph{dual variant} of of the (leveled) $\FHE$ scheme by Gentry, Sahai and Waters~\cite{GSW2013,mahadev2018classical}.

Let $\lambda \in \N$ be the security parameter. Suppose we would like to evaluate $L$-depth circuits consisting of $\mathsf{NAND}$ gates. We choose $n(\lambda,L) \gg L$ and a prime $q=2^{o(n)}$. Then, for integer parameters $m \geq 2 n \log q$ and $N = (m+1) \cdot \lceil \log q \rceil$, we let $\vec I$ be the $(m+1) \times (m+1)$ identity matrix and let
$\vec G = [\vec I \, \| \, 2 \vec I \, \| \, \dots \, \| \, 2^{\lceil \log q \rceil -1} \vec I] \in \Z_q^{(m+1) \times N}$ denote the so-called \emph{gadget matrix} which converts a binary representation of a vector back to its original vector representation over the field $\Z_q$. 
Note that the associated (non-linear) inverse operation $\vec G^{-1}$ converts vectors in $\Z_q^{m+1}$ to their binary representation in $\bit^N$. In other words, we have that $\vec G^{-1} \cdot \vec G = \vec I$ acts as the identity.


\begin{construction}[Dual-Regev leveled $\mathsf{FHE}$ scheme with certified deletion]\label{cons:FHE-cd}
Let $\lambda \in \N$ be the security parameter.
The Dual-Regev (leveled) $\mathsf{FHE}$ scheme $\mathsf{DualFHECD} = (\KeyGen,\Enc,\Dec,\Eval,\Del,\Vrfy)$ with certified deletion consists of the following algorithms.
\begin{description}
\item $\KeyGen(1^\lambda) \rightarrow (\pk,\sk):$ sample $\bar{\vec A} \rand \Z_q^{n\times m}$ and vector $\bar{\vec x} \rand \bit^{m}$
and let $\vec A = [\bar{\vec A} \| \bar{\vec A} \cdot \bar{\vec x} \Mod{q}]^\intercal$.
Output $(\pk,\sk)$, where $\pk=\vec A \in \Z_q^{(m+1) \times n}$ and $\sk = (-\bar{\vec x}, 1) \in \Z_q^{m+1}$.
\item $\Enc(\pk,x) \rightarrow (\vk,\ket{\ct}):$ to encrypt a bit $x\in \bit$, parse the public key as $\vec A \in \Z_q^{(m+1) \times n} \leftarrow \pk$ and generate the following pair consisting of a verification key and ciphertext for a random $\vec Y \in \Z_q^{n \times N}$ with columns $\vec y_1,\dots,\vec y_N \in \Z_q^{n}$:
$$
\vk \leftarrow (\vec A,\vec Y), \quad\,\,
\ket{\ct} \leftarrow \sum_{\vec S \in \Z_q^{n \times N}} \sum_{\vec E \in \Z_q^{(m+1)\times N}} \rho_{q/\sigma}(\vec E) \, \omega_q^{-\Tr[\vec S^\intercal \vec Y]} \ket{\vec A\cdot \vec S + \vec E + x \cdot \vec G},
$$
where $\vec G \in \Z_q^{(m+1)\times N}$ denotes the \emph{gadget matrix} and where $\sigma = 1/\alpha$.

\item $\Eval(\mathsf{C}_0,\mathsf{C}_1) \rightarrow \mathsf{C}_0 \mathsf{C}_1 \mathsf{C}$: to apply a $\mathsf{NAND}$ gate onto two registers $\mathsf{C}_0$ and $\mathsf{C}_1$ (possibly part of a larger ciphertext), append an ancilla system $\ket{\vec 0}_{\mathsf{C}}$, and apply the unitary $U_{\mathsf{NAND}}$, defined by
$$
U_{\mathsf{NAND}}: \quad \ket{\vec X}_{\mathsf{C}_0} \otimes \ket{\vec Y}_{\mathsf{C}_1} \otimes \ket{\vec Z}_\mathsf{C} \quad \rightarrow \quad  \ket{\vec X}_{\mathsf{C}_0} \otimes \ket{\vec Y}_{\mathsf{C}_1} \otimes \ket{\vec Z + \vec G - \vec X \cdot \vec G^{-1}(\vec Y) \Mod{q}}_{\mathsf{C}},
$$
where $\vec X,\vec Y,\vec Z \in \Z_q^{(m+1)\times N}$.
Output the resulting registers $\mathsf{C}_0 \mathsf{C}_1 \mathsf{C}$.


\item $\Dec(\sk,\mathsf{C}) \rightarrow \bit \, \mathbf{or} \, \bot:$ measure the register $\mathsf{C}$ in the computational basis to obtain an outcome $\vec C \in \Z_q^{(m+1)\times N}$ and compute $c = \sk^\intercal \cdot \vec c_N \in \Z \cap (-\frac{q}{2},\frac{q}{2}]$, where $\vec c_N \in \Z_q^{m+1}$ is the $N$-th column of $\vec C$; output $0$, if $c$
is closer to $0$ than to $\lfloor\frac{q}{2}\rfloor$,
and output $1$, otherwise.

\item $\Del(\ket{\ct}) \rightarrow \pi:$ measure $\ket{\ct}$ in the Fourier basis with outcomes $\pi = (\pi_1|\dots|\pi_N) \in \Z_q^{(m+1)\times N}$.

\item $\Vrfy(\vk,\pk,\pi) \rightarrow \bit:$ to verify the deletion certificate $\pi = (\pi_1\|\dots\|\pi_N) \in \Z_q^{(m+1)\times N}$, parse $(\vec A \in \Z_q^{(m+1) \times n},(\vec y_1 \|\dots \|\vec y_N)  \in \Z_q^{n \times N}) \leftarrow \vk$ and output $\top$, if both $\vec A^\intercal \cdot \pi_i = \vec y_i \Mod{q}$ and $\| \pi_i \| \leq \sqrt{m+1}/\sqrt{2}\alpha$ for every $i \in [N]$, and output $\bot$, otherwise.
\end{description}
\end{construction}

For additional details on the correctness of the scheme, we refer to Section $9$ of \cite{Poremba22}.

\begin{theorem}\label{thm:FHE-CD-security} Let $L$ be an upper bound on the $\mathsf{NAND}$-depth of the circuit which is to be evaluated. Let $n\in \N$ and $q$ be a prime modulus with $n=n(\lambda,L) \gg L$, $q=2^{o(n)}$ and $m \geq 2n \log q$, each parameterized by $\lambda \in \N$. Let $N = (m+1) \cdot \lceil \log q \rceil$ be an integer. Let $\sigma \in (\sqrt{2m},q/\sqrt{2m})$ and let $\alpha \in (0,1)$ such that $1/\alpha= 2^{o(n)} \cdot \sigma$.
Then, $\mathsf{DualFHECD}$ in \Cref{cons:FHE-cd} has everlasting certified deletion security assuming the quantum (subexponential) hardness of $\SIS_{n,q,\sigma\sqrt{2m}}^m$ and $\mathsf{LWE}_{n,q,\alpha q}^m$.
\end{theorem}
\begin{proof}
The proof is identical to the template in~\cite[Theorem 10]{Poremba22}, except that the adversary is allowed to be computationally unbounded once the deletion certificate is submitted. This is in contrast with the original proof who considered forwarding the \emph{secret key} during the security experiment. We remark that we do not invoke the strong Gaussian-collapsing property to prove the indistinguishability of the hybrids; instead we use the (stronger) notion of certified everlasting $D_{\Z_q^m,\frac{\sigma}{\sqrt{2}}}$-target-collapsing property of the Ajtai hash function shown in \Cref{thm:ajtai-certified-everlasting}. This results in the stronger notion of everlasting certified
deletion security.
\end{proof}