\section{Definitions: Target-Collapsing and Target-Collision-Resistance}

Unruh~\cite{cryptoeprint:2015/361} introduced the notion of collapsing hash functions in his seminal work on computationally binding quantum commitments. This property is captured by the following definition.

\begin{definition}[Collapsing Hash Function \cite{cryptoeprint:2015/361}] A hash function family $\algo H = \{H_\lambda\}_{\lambda \in \N}$ is called collapsing if, for every $\QPT$ adversary $\algo A$,
$$
|
\Pr[ \mathsf{CollapseExp}_{\algo H,\algo A,\lambda}(0)=1] - \Pr[ \mathsf{CollapseExp}_{\algo H,\algo A,\lambda}(1)=1]
| \leq \negl(\lambda).
$$
Here, the experiment $\mathsf{CollapseExp}_{\algo H,\algo A,\lambda}(b)$ is defined as follows:
\begin{enumerate}
    \item The challenger samples a random hash function $h \rand H_\lambda$, and sends a description of $h$ to $\algo A$.
    \item $\algo A$ responds with a (classical) string $y \in \bit^{n(\lambda)}$ and $m(\lambda)$-qubit quantum state in system $X$.
    \item The challenger coherently computes $h$ (into an auxiliary system $Y$) given the state in system $X$, and then performs a two-outcome measurement on $Y$ indicating whether the output of $h$ equals $y$. If $h$ does not equal $y$ the challenger aborts and
outputs $\bot$.

\item If $b=0$, the challenger does nothing. Else, if $b=1$, the challenger measures the $m(\lambda)$-qubit system $X$ in the computational basis. The challenger returns the state in system $X$ to $\algo A$.

\item $\algo A$ returns a bit $b'$, which we define as the output of the experiment.
\end{enumerate}
\end{definition}
