%\documentclass[aps,twocolumn,preprintnumbers,amsmath,amssymb]{revtex4}
%%showpacs
%\usepackage{dcolumn}
%\usepackage{bm}
%\usepackage{graphicx,xcolor,subfigure}

%\usepackage{comment}
%\usepackage{bibunits}
%\usepackage{xcolor}
%\DeclareMathOperator{\Tr}{Tr}
%\DeclareMathOperator{\Immag}{Im}
%
\documentclass[%
 reprint,
%superscriptaddress,
%groupedaddress,
%unsortedaddress,
%runinaddress,
%frontmatterverbose,
%preprint,
%showpacs,preprintnumbers,
%nofootinbib,
%nobibnotes,
%bibnotes,
 amsmath,amssymb,
aps
pra,
onecolumn
%twocolumn,
%prb,
%rmp,
%prstab,
%prstper,
%floatfix,
]{revtex4-2}
\usepackage{braket}
\usepackage{graphicx}% Include figure files
%\usepackage{dcolumn}% Align table columns on decimal point
\usepackage{bm}% bold math
%\usepackage{subfigure}
%\usepackage{subcaption}
\usepackage{float}
\usepackage{color}
%\usepackage{soul}
%\usepackage{caption}
%\usepackage{subcaption}
%\usepackage{widetext}
%\usepackage[caption = false]{subfig}
%\usepackage{romannum}
%\usepackage{textgreek}
\usepackage{hyperref}% add hypertext capabilities
%\usepackage[mathlines]{lineno}% Enable numbering of text and display math
%\linenumbers\relax % Commence numbering lines
\usepackage{placeins}
%\usepackage[showframe,%Uncomment any one of the following lines to test
%%scale=0.7, marginratio={1:1, 2:3}, ignoreall,% default settings
%%text={7in,10in},centering,
%%margin=1.5in,
%%total={6.5in,8.75in}, top=1.2in, left=0.9in, includefoot,
%%height=10in,a5paper,hmargin={3cm,0.8in},
%]{geometry}

%\newcommand{\ad}{a^{\dag}}
%\newcommand{\bd}{b^{\dag}}

\usepackage[normalem]{ulem}

\newcommand{\tcr}[1]{{\color{red} #1}}

\begin{document}

\title{Supplemental Material: Non-Pauli errors can be efficiently sampled in qudit surface codes}
\author{Yue Ma$^1$}

\author{Michael Hanks$^1$}

\author{M. S. Kim$^1$}

\affiliation{$^1$QOLS, Blackett Laboratory, Imperial College London, London SW7 2AZ, United Kingdom\\
}

%\begin{abstract}

%In the Supplemental Information, we first show explicit examples on why a forest error subgraph indicates an exact Pauli twirling approximation, and why a loop in the error subgraph implies correlated errors. Then we show a way of setting an upper bound on the logical error rate based on the logical error rate obtained after the Pauli twirling approximation. This is achieved by estimating the probability of having error patterns that contain loops in the primal lattice or the dual lattice. We then investigate the asymmetry of the lattice as opposed to the symmetric cases considered in the main text, and another way of quantifying the maximal amount of correlation.

%\end{abstract}

%\date{\today}

\maketitle

\section{Explicit examples of discretizing the errors by stabilizer measurements}

\begin{figure}[h]
\centering
\includegraphics[width=0.5\textwidth]{LatticeV2.pdf}
\caption{An example of a 2D surface code with $n_h=4$ and $n_v=5$, after contracting the nodes on the rough boundaries to one node. Both the primal lattice and the dual lattice are shown. On the primal lattice, the horizontal size $n_h$ refers to the number of edges in each row, while the vertical size $n_v$ refers to the number of nodes in each column. Qudits that have been subject to the error $\hat{F}$ are marked in red and labelled by blue letters.}\label{fig:supp_lattice}
\end{figure}

As shown in Fig.~\ref{fig:supp_lattice}, we consider the error pattern where the edges that correspond to the qudits with error $\hat{F}$ are colored red and labelled a to h. On the primal lattice, the error subgraph consists of three disconnected components, none of which containing loops. The initial state is $\hat{F}_a\hat{F}_b\hat{F}_c\hat{F}_d\hat{F}_e\hat{F}_f\hat{F}_g\hat{F}_h|\psi\rangle$, where subscripts a to h label the qudits, and the state $|\psi\rangle$ is in the code space. We start from the isolated qudit with error, qudit a. We measure the vertex operator centered at node $(1,1)$, obtaining a measurement result $x_a$, which can take a value $0,1,\cdots$ or $d-1$, and correspondingly project the state to $\sum_{i=0}^{d-1}f_{i,x_a}\hat{X}_a^i\hat{Z}_a^{x_a}\hat{F}_b\hat{F}_c\hat{F}_d\hat{F}_e\hat{F}_f\hat{F}_g\hat{F}_h|\psi\rangle$. Note that the Pauli-$Z$ operator for qudit a now only has exponential $x_a$. We then measure the vertex operator centered at node $(2,1)$ to get the measurement result $-x_a$ and no change in the state. This additional stabilizer measurement with definite result and without changing the state indicates that the measurements are well-defined at the rough boundaries, which will be shown later. Next we focus on the tree of qudits with errors, labelled as b, c, d, e. The general procedure is to start measuring the vertex operator centered at one end node, and proceed along the chain until for the next vertex measurement there are more than one edge that have not been projected yet.
We stop there and start from another end node. Finally we can resume the measurement of vertices that were left before as more projections of edges have been implemented. The order of vertex operator measurements is: $(0,2)\rightarrow(1,2)\rightarrow(3,2)\rightarrow(2,3)\rightarrow(2,2)$. The measurement results are denoted as $x_c$, $x_b$, $x_d$, $x_e$, and $-x_c-x_b-x_e-x_d$, respectively, where $x_b,x_c,x_d,x_e$ can take a value from $0,1,\cdots,d-1$. The state is projected to $\sum_{i=0}^{d-1}f_{i,x_a}\hat{X}_a^i$ $\sum_{i=0}^{d-1}f_{i,x_c+x_b}\hat{X}_b^i$ $\sum_{i=0}^{d-1}f_{i,x_c}\hat{X}_c^i$ $\sum_{i=0}^{d-1}f_{i,-x_d}\hat{X}_d^i$ $\sum_{i=0}^{d-1}f_{i,x_e}\hat{X}_e^i$ $\hat{Z}_a^{x_a}$ $\hat{Z}_b^{x_c+x_b}$ $\hat{Z}_c^{x_c}$ $\hat{Z}_d^{-x_d}$ $\hat{Z}_e^{x_e}$ $\hat{F}_f$ $\hat{F}_g$ $\hat{F}_h$ $|\psi\rangle$ after the final measurement. To deal with the boundary qudits f, g and h which are only covered by one vertex measurement each, we can invert the contraction process such that they are described by $((0,3),(0,4))$, $((1,3),(1,4))$, $((4,3),(4,4))$, respectively. Measuring the vertex operators $(0,3)$, $(1,3)$ and $(4,3)$ correspondingly discretize the qudits f, g and h, and the state is projected to $\sum_{i=0}^{d-1}f_{i,x_a}\hat{X}_a^i$ $\sum_{i=0}^{d-1}f_{i,x_c+x_b}\hat{X}_b^i$ $\sum_{i=0}^{d-1}f_{i,x_c}\hat{X}_c^i$ $\sum_{i=0}^{d-1}f_{i,-x_d}\hat{X}_d^i$ $\sum_{i=0}^{d-1}f_{i,x_e}\hat{X}_e^i$ $\sum_{i=0}^{d-1}f_{i,-x_f}\hat{X}_f^i$ $\sum_{i=0}^{d-1}f_{i,-x_g}\hat{X}_g^i$ $\sum_{i=0}^{d-1}f_{i,-x_h}\hat{X}_h^i$ $\hat{Z}_a^{x_a}$ $\hat{Z}_b^{x_c+x_b}$ $\hat{Z}_c^{x_c}$ $\hat{Z}_d^{-x_d}$ $\hat{Z}_e^{x_e}$ $\hat{Z}_f^{-x_f}$ $\hat{Z}_g^{-x_g}$ 
 $\hat{Z}_h^{-x_h}$ $|\psi\rangle$, where all the $Z$ errors are fully discretized. Similarly, measurements of the plaquette operators discretize $X$ errors. As the error subgraph on the dual lattice in Fig.~\ref{fig:supp_lattice} also does not contain loops, the $X$ errors are fully discretized as well. We have therefore demonstrated the complete error discretization by syndrome measurements for a forest error subgraph,
\begin{align}           &\hat{P}^{(\cdots,x_a,\cdots,x_h,z_a,\cdots,z_h,\cdots)}\hat{F}_a\cdots\hat{F}_h|\psi\rangle\nonumber\\
&=f_{-z_a,x_a}f_{-z_a-z_b,x_c+x_b}f_{-z_c,x_c}f_{z_a+z_b+z_d+z_e,-x_d}\nonumber\\
&~~~~f_{-z_a-z_b-z_e,x_e}f_{z_f+z_g,-x_f}f_{z_g,-x_g}f_{-z_h,-x_h}\nonumber\\
&~~~~\hat{X}_a^{-z_a}\hat{Z}_a^{x_a}\hat{X}_b^{-z_a-z_b}\hat{Z}_b^{x_c+x_b}\hat{X}_c^{-z_c}\hat{Z}_c^{x_c}\hat{X}_d^{z_a+z_b+z_d+z_e}\hat{Z}_d^{-x_d}\nonumber\\
&~~~~\hat{X}_e^{-z_a-z_b-z_e}\hat{Z}_e^{x_e}\hat{X}_f^{z_f+z_g}\hat{Z}_f^{-x_f}\hat{X}_g^{z_g}\hat{Z}_g^{-x_g}\hat{X}_h^{-z_h}\hat{Z}_h^{-x_h}|\psi\rangle,
\end{align}
where $\hat{P}^{(\cdots)}$ is the projector corresponding to a specific set of syndromes. 

For each forest error subgraph, different Pauli errors correspond to distinct syndrome measurement outcomes. Upon relabelling the superscripts we can arrive at the equivalence with the Pauli twirled error channel, \begin{equation}\label{eq:supp_channelTwirl}
\mathcal{F}(\rho)=\sum_{i,j=0}^{d-1}|f_{i,j}|^2\hat{X}^i\hat{Z}^j\rho\hat{Z}^{-j}\hat{X}^{-i},
\end{equation}
as
\begin{align}
    &\hat{P}^{(\Gamma_{a,b,\cdots})}\cdots\hat{F}_b\hat{F}_a|\psi\rangle\langle\psi|\hat{F}_a^{\dagger}\hat{F}_b^{\dagger}\cdots\hat{P}^{(\Gamma_{a,b,\cdots})}=\nonumber\\
    &\hat{P}^{(\Gamma_{a,b,\cdots})}\cdots\mathcal{F}_b(\mathcal{F}_a(|\psi\rangle\langle\psi|))\cdots\hat{P}^{(\Gamma_{a,b,\cdots})}
\end{align}
for a forest error subgraph and $\Gamma_{a,b,\cdots}$ represents any compatible syndrome measurement outcome. 

\begin{figure}[t]
\centering
\includegraphics[width=0.5\textwidth]{CountMain1V2.pdf}
\caption{The probability of not equivalent to the Pauli-twirled error model as a function of the lattice size characterized by $n_h+n_v$, for almost symmetric lattices $n_h=n_v\pm1$. Different colors correspond to different single qudit physical error rates. Solid lines follow the analytical expression Eq.~\eqref{eq:count_ana}. Markers are from the Monte-Carlo simulations, where for each data point $10000$ samples are taken. Error bars represent the standard deviation.}\label{fig:count_sym}
\end{figure}


The equivalence shown above does not hold if the error subgraph contains loops. As an example, we consider an error subgraph on the primal lattice formed by the nodes $(3,1)$, $(3,2)$, $(2,2)$ and $(2,1)$. We label the corresponding qudits with error as $A: ((3,1),(3,2))$, $B: ((3,2),(2,2))$, $C: ((2,2),(2,1))$ and $D: ((2,1),(3,1))$. Suppose the Pauli-$X$ errors have been completely discretized by the measurements of plaquette operators and the state is 
\begin{align}
&\hat{X}_A^{z_A}\hat{X}_B^{z_B}\hat{X}_C^{z_C}\hat{X}_D^{z_D}\sum_{j_A,j_B,j_C,j_D=0}^{d-1}f_{z_A,j_A}f_{z_B,j_B}f_{z_C,j_C}f_{z_D,j_D}\nonumber\\
&\hat{Z}_A^{j_A}\hat{Z}_B^{j_B}\hat{Z}_C^{j_C}\hat{Z}_D^{j_D}|\psi\rangle.\nonumber
\end{align}
The vertex operators are measured in the order of $(3,1)\rightarrow(3,2)\rightarrow(2,2)\rightarrow(2,1)$, and the outcomes are $x_A,x_B,x_C,-x_A-x_B-x_C$, respectively. The state is projected to
\begin{align}
    &\hat{X}_A^{z_A}\hat{X}_B^{z_B}\hat{X}_C^{z_C}\hat{X}_D^{z_D}\hat{Z}_B^{-x_B}\hat{Z}_C^{x_C+x_B}\hat{Z}_D^{-x_A}\nonumber\\
    &\sum_{j_A=0}^{d-1}f_{z_A,j_A}f_{z_B,-x_B+j_A}f_{z_C,x_C+x_B-j_A}f_{z_D,-x_A-j_A}|\psi\rangle.\nonumber
\end{align}
The remaining sum of the products of $f_{ij}$ coefficients is due to the fact that there is one less stabilizer measurement for a loop compared with a tree subgraph. It can also be explained as coming from the superposition of equivalent paths that are different by the application of a stabilizer operator, therefore corresponding to the same syndrome. It indicates that due to the correlations this situation cannot be represented by the Pauli-twirled channel Eq.~\eqref{eq:supp_channelTwirl}. We have contracted all the nodes on the rough boundaries to one node, as illustrated in Fig.~\ref{fig:supp_lattice}, to cover the situations of logical operators and boundary stabilizers as loops. For instance, on the primal lattice, the loop made of nodes $(4,0),(1,1),(1,2),(1,3)$ is a logical operator, and the loop made of nodes $(4,0),(0,1),(1,1)$ corresponds to a deformed stabilizer generator.


\section{Probability of not equivalent to the Pauli-twirled channel}

\begin{figure}[t]
\centering
\includegraphics[width=0.5\textwidth]{CountMain2and3V2.pdf}
\caption{The probability of not being equivalent to the Pauli-twirled error model as a function of the aspect ratio of the lattice, fixing $n_h+n_v=41$ (blue) or $n_h+n_v=113$ (red). We fix the single qudit physical error rate as (a) $p=0.03$, (b) $p=0.05$. Solid lines follow the analytical expression Eq.~\eqref{eq:count_ana}. Markers are from the Monte-Carlo simulations, and  error bars represent the standard deviation.}\label{fig:count_ana}
\end{figure}

We can estimate the probability of having error patterns that do not contain loops in both the primal lattice and the dual lattice. To achieve this, we assume each qudit is subject to the original error channel $\mathcal{L}_s$ [see main text Eq.~(1)] and sample the error patterns via the Monte-Carlo simulation. The probability that the result is not equivalent to the Pauli-twirled channel is estimated by the number of sampled error patterns that contains loops divided by the total number of samples.

We can also derive an approximate analytical expression to estimate the probability of having loops. As discussed in the main text, by taking into account only the smallest loops (three-edge on the rough boundaries and four-edge in the bulk), we have
\begin{equation}\label{eq:count_ana}
    p_{\mathrm{not-Pauli-twirled}}\approx n_hn_v\cdot 2p^4+(n_h+n_v)(2p^3-3p^4).
\end{equation}

In Fig.~\ref{fig:count_sym}, we illustrate $p_{\mathrm{not-Pauli-twirled}}$ as a function of the lattice size $n_h+n_v$, for almost symmetric lattice shapes $n_h=n_v\pm1$. The analytical formula Eq.~\eqref{eq:count_ana} fits well with the numerical results. For a fixed physical error rate $p$, a larger lattice corresponds to a larger probability of deviation from the Pauli-twirled error channel. This is because a larger number of qudits also indicate a larger number of error edges, therefore forming a loop in one of the error subgraphs is more likely. Similarly, increasing the single qudit error rate results in a larger probability of the error channel not being exactly equivalent to the Pauli-twirled version. As shown in Fig.~\ref{fig:count_sym}, for a single-qudit error rate $p=0.04$, up to $n_h\approx n_v\approx 70$, the probability of not being equivalent to the Pauli-twirled error model is bounded by $p_{\mathrm{not-Pauli-twirled}}\leq 0.04$. For a larger single qudit error rate $p=0.05$, if we require $p_{\mathrm{not-Pauli-twirled}}\leq 0.05$, the lattice size has to satisfy $n_h+n_v<100$.

In Fig.~\ref{fig:count_ana}, we demonstrate how the aspect ratio of the surface code, defined as the ratio between the horizontal size $n_h$ and the vertical size $n_v$, changes the probability $p_{\mathrm{not-Pauli-twirled}}$. For fixed $p$ and $n_h+n_v$, a more symmetric lattice corresponds to a larger probability of not being exactly equivalent to the Pauli-twirled error model. This is because the total number of physical qudits in the surface code is $2n_hn_v-n_h-n_v+1$. If $n_h+n_v$ is fixed, $n_h/n_v$ closer to $1$ indicates a larger product of $n_hn_v$. This means more errors for a fixed physical error rate $p$, therefore the error edges are more likely to form a loop. As seen in the red data points in Fig.~\ref{fig:count_ana} (b), for the most asymmetric lattices considered, namely, $(n_h,n_v)=(16,97)$, $(23,90)$, $(94,19)$, $(99,14)$, $p_{\mathrm{not-Pauli-twirled}}<0.05$ is satisfied, even though for more symmetric lattices $p_{\mathrm{not-Pauli-twirled}}>0.05$.

In Fig.~\ref{fig:count_aspect}, we show how the probability $p_{\mathrm{not-Pauli-twirled}}$ depends on the single qudit physical error rate $p$. For each pair of $(n_h,n_v)$, which corresponds to a certain value of $n_h+n_v$ and $n_h/n_v$, $p_{\mathrm{not-Pauli-twirled}}$ increases with $p$. This can be expected from the analytical expression Eq.~\eqref{eq:count_ana}, which implies that the dependence contains both $p^3$ scaling coming from the boundaries and $p^4$ scaling coming from the inside of the lattices. In fact, for the cases where $p_{\mathrm{not-Pauli-twirled}}$ is much smaller than the saturation value $1$, the size of the lattice is relatively small such that the boundary contributions are comparable to the non-boundary contributions, or even dominate over the latter ones. In Fig.~\ref{fig:count_aspect} (b), the more symmetric lattices result in larger $p_{\mathrm{not-Pauli-twirled}}$, as discussed before, but in Fig.~\ref{fig:count_aspect} (a), the most asymmetric lattices turn out to lead to larger values of $p_{\mathrm{not-Pauli-twirled}}$. This might be a result of the small size of the lattice, such that logical errors occur at a comparable probability to stabilizers.
%{\color{red}the loops on the contracted lattice that correspond to a logical operator might contribute to $p_{\mathrm{not-Pauli-twirled}}$ on the same order as the terms in Eq.~\eqref{eq:count_ana} that correspond to a stabilizer.}

\begin{figure}[t]
\centering
\includegraphics[width=0.5\textwidth]{CountMain4V2.pdf}
\caption{Monte-Carlo simulation results of the probability of not being equivalent to the Pauli-twirled error model as a function of the single qudit physical error rate. We fix the aspect ratios of the lattice to different values, each one corresponding to a different color. We consider two fixed values of the perimeter of the lattice, (a) $n_h+n_v=23$, (b) $n_h+n_v=137$. Error bars are small, thus not shown for clarity.}\label{fig:count_aspect}
\end{figure}

\begin{figure}[t]
\centering
\includegraphics[width=0.48\textwidth]{RatioMain2and3V2.pdf}
\caption{The proportion of erroneous qudits that are in loops as a function of the aspect ratio of the lattice, fixing $n_h+n_v=53$ (red) or $n_h+n_v=191$ (blue). Markers are from the Monte-Carlo simulations, and  error bars represent the standard deviation. (a) $p=0.1$, solid lines follow the analytical expression  Eq. (3) in the main text. (b) $p=0.25$, solid lines follow  Eq. (3) in the main text with $4p^3$ replaced by the correction term $4.745p^{3.057}$.}\label{fig:Ratio_ana}
\end{figure}


The value of $p_{\mathrm{not-Pauli-twirled}}$ is useful in setting the upper bound of the logical error rate. To be specific, suppose we have taken $N$ samples, of which $(1-p_{\mathrm{not-Pauli-twirled}})N$ samples have forest error subgraphs and $p_{\mathrm{not-Pauli-twirled}}N$ samples have non-forest error subgraphs. Suppose the logical error rate for forest graphs, $p_l$, can be straightforwardly simulated. Then the largest possible number of samples that cannot be error-corrected is $(1-p_{\mathrm{not-Pauli-twirled}})p_lN+p_{\mathrm{not-Pauli-twirled}}N$, where we have assumed that all the samples with non-forest error subgraphs cannot be error-corrected. The logical error rate without the Pauli twirling approximation is thus upper bounded by  $p_l-p_{\mathrm{not-Pauli-twirled}}p_l+p_{\mathrm{not-Pauli-twirled}}$.



\section{Dependence of the proportion of erroneous qudits that are in loops on the lattice asymmetry}

In the main text, we have shown how $p_{\mathrm{loop-edge}}$ changes with $n_h$ for the symmetric lattices (see main text Fig. 2). Here we demonstrate the dependence of $p_{\mathrm{loop-edge}}$ on the aspect ratio $n_h/n_v$. This is important for small lattices where the contributions from the boundaries are comparable to the contributions from the bulk. Examples are plotted in Fig.~\ref{fig:Ratio_ana}. For a small $p=0.1$, the analytical formula in Eq. (3) in the main text agrees well with the numerical simulations. For a larger $p=0.25$, including the correction as shown in Fig. 3 in the main text leads to a better fit to the simulation data. As the product $n_hn_v$ is in the denominator in Eq. (3) in the main text, a more symmetric lattice results in a smaller value of $p_{\mathrm{loop-edge}}$.



\section{maximal one-dimensional span of a loop for asymmetric lattices}

In the main text Fig. 4, we have demonstrated how the maximal one-dimensional span of a loop for a symmetric lattice increases with the lattice size. Here we show how it depends on the asymmetry of the lattice in Fig.~\ref{fig:maxLoopAsym}. For the considered large values of $n_h+n_v$, we find that increasing $p$ leads to a larger $L_{\mathrm{max,loop}}$, and increasing $n_h+n_v$ leads to a larger $L_{\mathrm{max,loop}}$ as well, but $L_{\mathrm{max,loop}}$ is relatively insensitive to the value of $\log(n_h/n_v)$ if both $p$ and $n_h+n_v$ are fixed. This is due to the slow logarithmic growth of $L_{\mathrm{max,loop}}$ with respect to the lattice size as illustrated in the main text Fig. 4, effectively suppressing the difference in $n_hn_v$ for fixed $n_h+n_v$ and changing $n_h/n_v$.


\begin{figure}[t]
\centering
\includegraphics[width=\textwidth]{LoopAsym_v2.pdf}
\caption{The maximal one-dimensional span of a loop as a function of the lattice aspect ratio $n_h/n_v$, fixing the single qudit physical error rate $p$ and the perimeter $n_h+n_v$. Data points are from Monte-Carlo simulations,
where each data point consists of 200 samples for the asymmetric cases and 2000 samples for the symmetric cases. Error bars
represent the standard deviation.}\label{fig:maxLoopAsym}
\end{figure}

\begin{figure}[t]
\centering
\includegraphics[width=0.5\textwidth]{maxNodeNumAll_v2.pdf}
\caption{The maximal total number of nodes in a loop as a function of the lattice size $n_h$ for almost symmetric lattices $n_h = n_v \pm 1$, obtained from the Monte-Carlo simulations,
where each data point consists of 2000 samples and error bars
represent the standard deviation. Different colors correspond
to different single qudit physical error rates.}\label{fig:maxLoopNum}
\end{figure}

\section{maximal total number of nodes in a loop}

Similar to the one-dimensional span of a loop in the main text, we can define another quantity to describe the size of a loop, namely, the total number of nodes in a loop, $N_{\mathrm{max,loop}}$. The results for symmetric lattices are shown in Fig.~\ref{fig:maxLoopNum}. The slow increase of $N_{\mathrm{max,loop}}$ as a function of $n_h$ follows the same trend as in Fig. 4(a) in the main text. This is because for symmetric lattices the loops are expected to be symmetric in the horizontal and vertical directions, resulting in a fixed relation between  $N_{\mathrm{max,loop}}$ and $L_{\mathrm{max,loop}}$.




\end{document}