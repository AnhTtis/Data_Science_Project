
\section{Introduction and Related Work}
\label{section:intro}


\begin{figure}[]
	\centering
	\includegraphics[width=\linewidth]{images/teaser_squared.pdf}
	\caption{Compared to keypoint approaches, e.g. OpenPose~\cite{openpose_full,openpose_hand}, ShaRPy estimates the 3D hand pose and shape, and indicates uncertainty by detecting unobserved and error-prone regions (both visualized in red on the hand surface).}
	\label{fig:intro}
\end{figure}
\begin{figure*}[t]
	\centering
	\includegraphics[width=\linewidth]{images/overview_short.pdf}
	\caption{Overview of ShaRPy. First, a network based on Yolact~\cite{yolact} detects hands and regresses features in a correspondence space. The outputs and the depth map are used in a subsequent energy minimization framework for pose and shape estimation. Lastly, we detect uncertainties with respect to the pose parameters. As the network weights are learned, the first part of the pipeline is fixed at inference time. However, the remaining parts are still adaptable at inference time, i.e., the weights can be individually tuned for different hand scenes in clinical setups to improve performance.}
	\label{fig:overview}
\end{figure*}





Hand function is affected by musculoskeletal rheumatic diseases. 
Rheumatoid Arthritis (RA) and Psoriatic Arthritis (PsA) are both common chronic inflammatory diseases, characterized by joint pain and swelling that can result in joint destruction~\cite{merola_RA}. 
In view of improved treatment options a more detailed, objective assessment of hand function is desirable, as it can potentially serve as a biomarker for changes in disease activity and patient quality of life~\cite{liphardt_obj}. 
This would allow for early therapy adjustment and potentially improve the prediction of increased risk of joint destruction.
In clinical practice, functional assessments are mainly based on subjective questionnaires~\cite{salaffi} or manual tests~\cite{higgins} that can discriminate between healthy individuals and patients, but lack sensitivity for disease monitoring over time~\cite{rydholm}. 
The gold standard for objective hand motion assessment is marker-based motion capturing~\cite{metcalf}. 
Other methods use gloves and inertial measurement units ~\cite{henderson,salchow} to record or monitor hand motion. 
A major drawback of these technologies is that they are contact-based, time-consuming to set up, and do not provide direct and intuitive visual feedback options. Furthermore, they are not suitable for patient monitoring at home.
Hence, simple markerless hand movement assessments based on consumer-friendly sensor systems such as RGB-D cameras are desirable and show promising potential to be applied in the future~\cite{phutane_liphardt_braeunig}. 

In the computer vision community, camera-based hand reconstruction has a rich history~\cite{survey_nonrigid}.
Hand pose estimation algorithms usually reconstruct hands as a set of keypoints~\cite{rgb_zimmermann,rgb_hampali}.
However, the visual interpretability of keypoints is limited (cf.~\autoref{fig:intro}) as they do not reflect shape and shape-dependent pose.
For example, the neutral posture with all fingers closed of a thick hand is identical to a thin hand with a slight abduction in the Metacarpophalangeal (MCP) joints.
Another line of work focuses on estimating the pose of parametric hand models including shape~\cite{meshtransformer,cpf,honnotate,depth_mueller}.
Commonly, a neural network~\cite{meshtransformer,cpf} is trained, which is fixed at inference time and restricted in its generalization capability with respect to unseen shapes, poses, and viewpoints.
Alternative approaches are based on energy optimization~\cite{honnotate,depth_mueller}, which can be adjusted to individual video sequences and extended to fit clinical requirements, e.g., including anthropometric hand constraints. 
Furthermore, in setups with only a single RGB~\cite{meshtransformer,rgb_zimmermann,rgb_hampali} camera, it is challenging to estimate the parameters in a metrically accurate 3D space because the depth of a hand can only be estimated up to a certain scale. To avoid complex setups with multiple cameras, we prefer to use additional depth information of a single RGB-D~\cite{honnotate,mueller_2017,sridhar_2016} sensor. 
The common goal of all the above approaches is to estimate the most plausible pose of the hand and its skeleton. However, in difficult cases (cf. \autoref{fig:intro}), this means that the finger segments can be mislabelled, point into the wrong direction, or are speculated at positions that are not visible. In clinical setups, besides accuracy, it is important to identify and discard unreliable measurements and avoid false positives in the assessment of hand functions.
\newline
To tackle all these limitations, we propose, to the best of our knowledge, the first markerless hand tracking method, which provides accurate hand pose \textit{and} shape parameters \textit{and} estimates the uncertainty that remains in those in order to discard unreliable predictions, e.g., when a finger is hidden or its estimate is inconsistent with the observations in the input.
Our approach requires only a single RGB-D camera, which makes it easily applicable and allows us to determine a metrically accurate hand shape and pose. ShaRPy makes the following contributions:
\begin{itemize}
\item We present the first framework that utilizes dense correspondence predictions to estimate uncertainty through unobserved and error-prone regions of a parametric hand model after shape and pose optimization.
\item We introduce a novel correspondence space with semantic encodings, which can be directly transformed into a hand part segmentation. The transformation enables a consistent coarse-to-fine mapping between hand segments and their respective features within each segment, and is utilized for precise correspondence matching and uncertainty estimation.
\item We demonstrate the benefits of our approach in the context of markerless hand function assessments as a method to monitor the activity of musculoskeletal rheumatic diseases as well as through a state-of-the-art pose estimation benchmark.
\end{itemize}







