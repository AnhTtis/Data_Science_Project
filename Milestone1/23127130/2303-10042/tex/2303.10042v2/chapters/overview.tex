



\section{Overview}

An overview of ShaRPy is shown in \autoref{fig:overview}. First, a pre-trained multi-task network~\cite{yolact} predicts for each hand in an RGB image $\boldsymbol I$ its bounding box, a label indicating whether it is the left or right hand, a segmentation mask $\boldsymbol M$, and a correspondence image $\boldsymbol F$. The correspondence image assigns each pixel of the hand to a unique feature in a novel correspondence space with semantic encodings (\autoref{section:dense}). Subsequently, the optimal pose and shape parameters of a parametric hand model are found in a two-stage energy minimization framework using the additional depth image $\boldsymbol D$ (\autoref{section:optim}). Lastly, we estimate the uncertainty through the calculation of unobserved and error-prone regions on the surface of the hand model and visualize the results accordingly (\autoref{sec:uncertainty}). Our tracking approach leverages the advantages of video data and reuses the network output, e.g. the Region-Of-Interest (ROI) defined by the bounding boxes, and hand model predictions of the previous frame at timestep $k-1$ to improve the predictions in the current frame $k$. 


\paragraph{Hand Model.} We employ the widely adopted MANO model~\cite{mano} as a parametric representation of the hand. The model is represented by a set of vertices $\mathcal{V} \subseteq \mathbb{R}^3$, deformed by a kinematic tree of 15 finger joints $\mathcal{J} \subseteq \mathbb{R}^3$, and a root wrist joint. The rigid motion of the wrist is described by the translation vector $\mathbf{t} \in \mathbb{R}^3$ and rotation $\mathbf{R}\in \mathbb{R}^3$ in axis-angle notation. Similarly, the per-joint rotations are denoted as the pose $\boldsymbol {\theta} \in \mathbb{R}^{3\lvert\mathcal{J}\rvert}$, and the hand shape is parameterized by $\boldsymbol \beta \in \mathbb{R}^{10}$. A linear function maps the pose and shape parameters to joints and, subsequently, to vertices.
As the model is not anatomically constrained, the orientation of the joints is not aligned with the natural bone structure. Together with the high number of 3 Degrees-of-Freedom (DoF) per joint, the parametrization can lead to unnatural poses. Inspired by~\cite{cpf}, we rephrase the orientation of a per-joint pose such that the respective joint moves within the sagittal, coronal, and transverse plane. Furthermore, we propose to limit the DoF per joint with respect to anatomy considering the special case of the thumb.  In total, we reduce the number of optimizable pose parameters from $3 \cdot \lvert \mathcal{J} \rvert = 45$ to $23$. The optimized MANO model is shown in \autoref{fig:mano} and enables an anatomically correct pose parametrization. Please note that, in the following sections, we use $\theta \in \mathbb{R}^{23}$ to denote the \textit{anatomically optimal} pose. 

\begin{figure}[t]
	\includegraphics[width=\linewidth]{images/mano.pdf}
	\caption{Left: The anatomical MANO model with exemplary movements of joint $i$ in the sagittal ($\theta^i_x$) and coronal plane ($\theta^i_y$). Middle: Dense correspondence encoding. Right: Segmentation sets $S^i_{3d}$ computed from correspondence space.}\label{Overview}
	\label{fig:mano}
\end{figure}





