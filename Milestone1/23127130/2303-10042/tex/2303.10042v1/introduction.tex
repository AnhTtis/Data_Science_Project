\begin{figure}[t]
\includegraphics[width=\linewidth]{teaser_arxiv.eps}
\caption{Compared to keypoint approaches, e.g. OpenPose~\cite{openpose_full,openpose_hand}, ShaRPy estimates the 3D hand pose and shape, and indicates uncertainty by detecting unobserved and error-prone regions (both visualized in red on the hand surface).}
\label{fig:intro}
\end{figure}
\section{Introduction and Related Work}
\label{section:intro}



Hand function is affected by musculoskeletal rheumatic diseases. 
Rheumatoid Arthritis (RA) and Psoriatic Arthritis (PsA) are both common chronic inflammatory diseases, characterized by joint pain and swelling that can result in joint destruction~\cite{merola_RA}. 
In view of improved treatment options a more detailed, objective assessment of hand function is desirable, as it can potentially serve as a biomarker for changes in disease activity and patient quality of life~\cite{liphardt_obj}. 
This would allow for early therapy adjustment and potentially improve the prediction of increased risk of joint destruction.
In clinical practice, functional assessments are mainly based on subjective questionnaires~\cite{salaffi} or manual tests~\cite{higgins} that can discriminate between healthy individuals and patients, but lack sensitivity for disease monitoring over time~\cite{rydholm}. 
The gold standard for objective hand motion assessment is marker-based motion capturing~\cite{metcalf}. 
Other methods use gloves and inertial measurement units ~\cite{henderson,salchow} to record or monitor hand motion. 
A major drawback of these technologies is that they are contact-based, time-consuming to set up, and do not provide direct and intuitive visual feedback options.
Hence, simple markerless camera-based hand movement assessments are desirable and show promising potential to be applied in the future~\cite{phutane_liphardt_braeunig}. 
\newline
In the computer vision community, camera-based hand reconstruction has a rich history~\cite{survey_nonrigid}.
Hand pose estimation algorithms usually reconstruct hands as a set of keypoints~\cite{rgb_zimmermann,rgb_hampali}.
However, the visual interpretability of keypoints is limited (cf.~\autoref{fig:intro}) as they do not reflect shape and shape-dependent pose.
For example, the neutral posture with all fingers closed of a thick hand is identical to a thin hand with a slight abduction in the Metacarpophalangeal (MCP) joints.
Another line of work focuses on estimating the pose of parametric hand models including shape~\cite{meshtransformer,cpf,honnotate,depth_mueller}.
Commonly, a neural network~\cite{meshtransformer,cpf} is trained, which is fixed at inference time and restricted in its generalization capability with respect to unseen shapes, poses, and viewpoints.
Alternative approaches are based on energy optimization, e.g.~\cite{honnotate,depth_mueller}, which can be adjusted to individual video sequences and extended to fit clinical requirements, e.g. including anthropometric hand constraints. 
The common goal of all the above approaches is to estimate the most plausible pose of the hand and its skeleton. However, in difficult cases (cf. \autoref{fig:intro}), this means that the finger segments can be mislabelled, point into the wrong direction, or are speculated at positions that are not visible. In clinical setups, besides accuracy, it is important to identify and discard unreliable measurements and avoid false positives in the assessment of hand functions.
\newline
To tackle all these limitations, we propose, to the best of our knowledge, the first markerless hand tracking method, which provides accurate hand pose \textit{and} shape parameters \textit{and} estimates the uncertainty that remains in those in order to discard unreliable predictions.
Our approach requires only a single RGB-D camera, which makes it easily applicable and allows us to determine a metrically accurate hand shape and pose.






