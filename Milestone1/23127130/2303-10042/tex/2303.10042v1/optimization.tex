\subsection{Pose and Shape Tracking with Uncertainty Estimation}
\label{section:optim}
In this stage, we solve an energy-minimization problem $E$ to obtain the optimal MANO parameter set $\Omega^k = (R^k,t^k,\theta^k,\beta^k)$ at timestep $k$:
\begin{equation*}
 \arg \min_{\Omega^k} E = \arg \min_{\Omega^k} \left[   \omega_{3d} \lambda E_{3d}(\mathcal{C}_{3d}) + \omega_{2d} E_{2d}(\mathcal{C}_{2d}) + E_{reg}(\Omega^k, \Omega^{k-1}) \right]
\end{equation*}
We denote the respective weights of a term $E_*$ as $\omega_*$ and define $\lambda = \exp{(J+1)}$, where $J$ is the Jaccard index of the predicted mask $\boldsymbol M$ and the mask $\boldsymbol M_v$ of the rasterized MANO model (by using Nvdiffrast~\cite{nvdiffrast}). $E_{3d}$ and $E_{reg}$ are similar to Mueller et al.~\cite{depth_mueller}: The data term $E_{3d}$ consists of a point-to-point and point-to-plane error. The regularization term $E_{reg}$ enforces plausible poses and shapes, as well as temporal smoothness, and consists of $E_{shape}$, $E_{pose}$, and $E_{temp}$~\cite{depth_mueller}. We introduce the term $E_{2d}$ defined on the set of valid pixels $\mathcal{C}_{2d}$ within $\boldsymbol M$ and $\boldsymbol M_v$. For each pixel $(x,y)\in C_{2d}$, the term penalizes the squared L2 norm between $\boldsymbol F(x,y)$ and $\boldsymbol F_v(x,y)$, where $F_v$ is the correspondence image of the rasterized MANO model. %
In other words, $E_{2d}$ enforces MANO to lie within the predicted hand silhouette and provides a more accurate estimation of $\boldsymbol \beta$ compared to $E_{3d}$. 
In our energy minimization framework, we distinguish between the \textit{Initialization} phase, which is only executed in the first frame or when the tracking is lost, and the \textit{Refinement} phase, in which we iteratively minimize $E$. During initialization, we first solve the orthogonal Procrustes problem to obtain the initial wrist parameters $\boldsymbol R$ and $\boldsymbol t$. Secondly, we make use of an implicit pose prior to initialize $\boldsymbol \theta$ with plausible parameters. For this purpose, we transform $\boldsymbol \theta$ into a PCA space pre-computed from annotated RGB(-D) datasets~\cite{interhand,freihand,h2o,honnotate,rgb_hampali}. Then, we solve $E$ with respect to the PCA pose parameters.
\newline
\newline
\textit{\textbf{Uncertainty Estimation.}} At last, we compute an uncertainty value $u_i$ for each segment $i$ on the surface of the MANO model given by $S^i_{3d}$ such that:
\begin{equation*}
     u_i =  \begin{cases}
       1 & \text{if segment $i$ unobserved or error-prone} \\
       0 & \text{else}
    \end{cases}
\end{equation*}
Since a segment relates to the set of vertices deformed by a particular joint, we can directly infer uncertainty with respect to its respective pose parameter.
We consider a segment $i$ as unobserved if:
\begin{equation*}
 \frac{\lvert \mathcal{V}^i_{vis} \rvert}{\lvert S^i_{3d} \rvert} < \tau_{v},\quad\text{with}\quad\mathcal{V}^i_{vis} = \{v \in \mathcal{S}^i_{3d} \mid (*,v) \in \mathcal{C}_{3d}\}
\end{equation*}
Further, we consider a segment $i$ as error-prone if:
\begin{equation*}
 \frac{\lvert \mathcal{P}_{2d} \rvert}{\lvert S^i_{2d} \rvert} > \tau_{2d}\quad\text{or}\quad\frac{\lvert \mathcal{P}_{3d} \rvert}{\lvert S^i_{3d} \rvert} > \tau_{3d} 
\end{equation*}
We define $\mathcal{P}^i_{2d} = \{ (x,y) \in \mathcal{S}^i_{2d} \mid (x,y) \in \mathcal{C}_{2d}\wedge E_{2d}(x,y) > \varepsilon_{2d}\}$ as the set of error-prone pixels and, analogously, $\mathcal{P}^i_{3d} = \{ \boldsymbol v \in S^i_{3d} \mid (*,\boldsymbol v) \in C_{3d}\wedge E_{z}(\boldsymbol v) > \varepsilon_{3d}\}$ as the set of error-prone vertices. The term $E_{z}(\boldsymbol v)$ is defined as the average L1 loss between the z-axis values of all pairs in $\mathcal{C}_{3d}$, in which $\boldsymbol v$ is included.








