% This is samplepaper.tex, a sample chapter demonstrating the
% LLNCS macro package for Springer Computer Science proceedings;
% Version 2.20 of 2017/10/04
%
\RequirePackage{amsmath}
\documentclass[runningheads]{llncs}
%
\usepackage[T1]{fontenc}
\usepackage{graphicx}
\usepackage{multirow}
\usepackage{capt-of}
\usepackage{amsfonts}
\usepackage{amsmath}
\usepackage{booktabs}
\usepackage{tabularx}
\usepackage{pifont}
\usepackage{xspace}
\usepackage{bm}
\usepackage{paralist}


% Used for displaying a sample figure. If possible, figure files should
% be included in EPS format.
%
% If you use the hyperref package, please uncomment the following line
% to display URLs in blue roman font according to Springer's eBook style:
% \renewcommand\UrlFont{\color{blue}\rmfamily}

%
% Comment
%
\usepackage{fancyvrb}
\usepackage{color}
\definecolor{cb_orange}{rgb}{1.0,0.51,0.0}
\definecolor{cb_blue}{rgb}{0.22,0.49,0.72}
\definecolor{cb_green}{rgb}{0.3,0.67,0.29}
\definecolor{cb_red}{rgb}{0.89,0.1,0.11}
\definecolor{cb_purple}{rgb}{0.6, 0.31, 0.64}
\definecolor{cadetgrey}{rgb}{0.57, 0.64, 0.69}
\newcommand{\TODO}[1]{\textbf{\textcolor{cb_red}{#1}}}
\newcommand{\MORE}[1]{\textcolor{cb_blue}{#1}}
\newcommand{\blar}[1]{\textcolor{cb_blue}{\st{#1}}}
\newcommand{\delete}[1]{}

\newcommand{\prior}[0]{PRIME\xspace}
\newcommand{\ours}[0]{PRIME\texttt{+}\xspace}

\begin{document}
%
\title{Enhancing Breast Cancer Risk Prediction by Incorporating Prior Images}
%
\titlerunning{Enhanced Risk Prediction With Prior Images}
% If the paper title is too long for the running head, you can set
% an abbreviated paper title here
%
% \author{Anonymous}
% \author{First Author\inst{1}\orcidID{0000-1111-2222-3333} \and
% Second Author\inst{2,3}\orcidID{1111-2222-3333-4444} \and
% Third Author\inst{3}\orcidID{2222--3333-4444-5555}}

 % TODO: \orcidID{0000-1111-2222-3333}
\author{Hyeonsoo Lee\inst{1} \and
Junha Kim\inst{1} \and
Eunkyung Park\inst{1} \and
Minjeong Kim\inst{1} \and
Taesoo Kim\inst{1} \and
Thijs Kooi\inst{1}
}

\authorrunning{H. Lee et al.}
%\authorrunning{Anonymous}
% First names are abbreviated in the running head.
% If there are more than two authors, 'et al.' is used.
%
% \institute{Anonymous Organization \\
% \email{**@******.***}
% }
\institute{Lunit Inc, Seoul, Republic of Korea \\
\email{\{hslee, junha.kim, ekpark, mjkim0918, taesoo.kim, tkooi@lunit.io\}}}
%
\maketitle              % typeset the headßer of the contribution
%
\begin{abstract}
%

Recently, deep learning models have shown the potential to predict breast cancer risk and enable targeted screening strategies, but current models do not consider the change in the breast over time. In this paper, we present a new method, \ours, for breast cancer risk prediction that leverages prior mammograms using a transformer decoder, outperforming a state-of-the-art risk prediction method that only uses mammograms from a single time point. We validate our approach on a dataset with 16,113 exams and further demonstrate that it effectively captures patterns of changes from prior mammograms, such as changes in breast density, resulting in improved short-term and long-term breast cancer risk prediction. Experimental results show
% that our model achieves a higher C-index 0.73 than the baseline method 0.68 (p < 0.05) on held-out test sets.
that our model achieves a statistically significant improvement in performance over the state-of-the-art based model, with a C-index increase from 0.68 to 0.73 (p < 0.05) on held-out test sets.

\keywords{Breast Cancer  \and Mammogram \and Risk Prediction.}
\end{abstract}
%
%
%

\section{Introduction}
\label{sec:introduction}
% Importance and appeal of children's drawings
Children's depictions of the human figure are highly expressive and varied.
As one of the very first subjects children attempt to draw, the representation begins as an almost unintelligible cloud of scribbles. 
As the child grows, their representation of the human figure becomes more developed and is extended to graphically represent many different types of characters: people, animals, and even personified objects (see Figure 1).

Who among us has not wished, either as a child or as an adult, to see such figures come to life and move around on the page?
Sadly, while it is relatively fast to produce a single drawing, creating the sequence of images necessary for animation is a much more tedious endeavor, requiring discipline, skill, patience, and sometimes complicated software.
As a result, most of these figures remain static upon the page.

% We built a system to animate them.
Inspired by the importance and appeal of the drawn human figure, we design and build a system to automatically animate it given an in-the-wild photograph of a child's drawing. 
Our system is fast, intuitive, and robust to much of the variation present in these types of drawings, making it well-suited to allow our target audience--children--to see their own characters coming to life.
The system is comprised of four stages: figure detection, segmentation masking, pose estimation/rigging, and animation. 
We describe each stage and identify common causes of failure in each. 
For object detection and pose estimation, we make use of existing computer vision models designed to detect human figures and joints in photographs; we fine-tune these models for use with children's drawings.
For segmentation, we present a straightforward, image processing-based method that, for animation purposes, is more useful and accurate than segmentation masks obtained from a fine-tuned object detection model.
During the animation step, we take advantage of the \textit{twisted perspective} commonly seen in children’s drawings to retarget motion capture data onto the character in a novel and appealing way.

% We use existing machine learning models. However, given the wide domain gap it's not clear how much fine-tuning data was needed. So we ran some experiments to find out and report it.
While our system leverages existing models and techniques, most are not directly applicable to the task due to the many differences between photographic images and simple pen and paper representations. 
To this end, we couple the presentation of our system with a set of experiments exploring the relationship between fine-tuning training set size and success rates.
We also include a perceptual study validating viewer preference for incorporating \textit{twisted perspective} into the motion retargeting step.

We validate the desirability and appeal of our system by building and publicly releasing a version of it as the \AD Demo \,\cite{animateddrawings}.
Launched in December 2021, this demo has been used by millions of people around the world to animate their children's drawings.
Inspired by this reception, our second contribution is The Amateur Drawings Dataset: \hjs{180,000 drawings and user-accepted annotations collected, with consent, through the demo. See Section \ref{sec:UI} for a description of how the annotations were generated.}
We believe this dataset will be a resource to researchers from various fields seeking to better understand the space of amateur drawings, evaluate new algorithms in this domain, or develop new drawing-based tools in general.

To summarize, our contributions are as follows:
\begin{enumerate}
    \item 
    We explore the problem of automatic sketch-to-animation for children's drawings of human figures and present a framework that achieves this effect. We also present a set of experiments determining the amount of training data necessary to achieve high levels of success and a perceptual study validating the usefulness of our motion retargeting technique.
    \item To encourage additional research in the domain of amateur drawings, we present a first-of-its-kind dataset of 180,000 user-submitted amateur drawings, along with user-accepted bounding box, segmentation mask, and joint location annotations.
\end{enumerate}

Upon acceptance of this paper, we plan to publicly release the Amateur Drawings Dataset, project code, and fine-tuned model weights.


\section{Method}
\label{sec:methods}
\section{Testing for anisotropy}\label{sec:TestingAnisotropy}
The specific hypothesis to be tested is whether, above some energy threshold, $E_{\rm th}$, the mean composition of UHECRs coming from directions near to the galactic plane is significantly higher in mass than those arriving further from it. This is to be tested using \xmax{} as a mass sensitive parameter. Typically, \xmax{} based composition analyses leverage the first two moments of \xmax{} distributions binned in energy, to comment on primary mass. This approach, however, does not lend itself well to quantifying the significance of a result testing the above statement. Instead, a test statistic, $TS$, which quantifies the degree of dissimilarity between the \xmax{} distributions in the two regions in a single value is preferred. For this, the returned value from the Anderson-Darling two-sample homogeneity test \cite{andersondarling}, \textit{AD-test}, has been selected as it scales with the dissimilarity of the tested distributions. The AD-test has good sensitivity to the full width of a distribution \cite{scholz1987k}, and has more power than the Kolmorogov-Smirnov test while remaining robust against false positives \cite{engmann2011comparing}.

To use the AD-test and \xmax{} for this purpose, two modifications are required. First, a single $TS$ comparing all events in each region above $E_{\rm th}$ is desired. So, all events with $E\geq E_{\rm th}$ in the on- and off-plane samples separately need to be collected into a common on-plane distribution and a common off-plane distribution. To do this, the natural evolution of \xmax{} with energy needs to be removed so that spectral features in the flux do not influence the result. Therefore, we define an energy-normalized \xmax{} value
\begin{equation}\label{eq:XmaxNorm}
X_{\text{max}}^{'} =  X_{\text{max}} -  \underbrace{\left(649 + 63.1 \, Z_{18} + 1.97 \, Z_{18}^{2}\right)}_\text{EPOS-LHC elongation rate for iron},
\end{equation}
where $Z_{18}=\log_{10}\left(E_\text{rec}/\,\text{EeV}\right)$. The last term in \autoref{eq:XmaxNorm} is the natural energy evolution of mean \xmax{} for iron primaries as predicted by EPOS-LHC~\cite{Pierog:2013ria}\footnote{Choice of hadronic interaction model varies result by $\sim0.02$\,\gcm{}.}. Second, the \xmaxnorm{} distribution of an on-plane sample populated with primaries which are on average heavier than those in the off-plane sample will display a lower mean and a narrower width than that of the off-plane \xmaxnorm{} distribution. Since the null hypothesis is that there is either no composition difference or a heavier off-plane sample, a $TS$ sensitive to the ordering of the \xmaxnorm{} distributions is required\footnote{Modifying the test to also require $\sigma( X_{\text{max}}^\prime)^{\rm on} < \sigma( X_{\text{max}}^\prime)^{\rm off}$ would be more restrictive, but conservatively has not been applied.}. The AD-test is insensitive to ordering, so it is modified to
\begin{equation}
TS =
\begin{cases}
    AD: \langle X_{\text{max}}^\prime \rangle^{\rm on} < \langle X_{\text{max}}^\prime \rangle^{\rm off} \\
    -3\hspace{1mm}: \text{else}
\end{cases},
\end{equation}
where $AD$ is the result of the AD-test comparing the on- and off-plane distributions, and $-3$ is selected as it is well below the minimum of the AD-test.

\vspace{-.1cm}
\myparagraph{Scan for energy and galactic latitude thresholds}
\vspace{-.1cm}
A scan has been used to select the optimal on/off splitting latitude, $b_{\rm split}$, and minimum energy, $E_{\rm th}$, as uncertainties in GMF models and source distributions make other approaches impractical. In this scan, each trial [$E_{\rm th}$, $b_{\rm split}$] pair is used to form on- and off-plane subsets and the $TS$ is extracted. To preserve the statistical strength of the sparse FD data set, a coarse scan of $5^\circ$ steps in $\abs{\,b\,}$ from $20^\circ$ to $35^\circ$ and 0.1\,\lge{} steps in energy from $18.4$ to $19.4$\,\lge{} is used. The scan is performed on the data set from~\cite{Aab:2014kda}, which includes events through Dec 31\textsuperscript{st} 2012. At the time of writing, this \textit{scan data set} represents $54\,\%$ of the analyzed events. The remaining $46\,\%$ of events, the \textit{post-scan data set}, is reserved as blind. 

\begin{figure}[!htb]
    % \vspace{-.4cm}
    \centering
    \includegraphics[width=0.45\textwidth]{Figures/ScanResults.pdf}
    \vspace{-2mm}
    \caption{Parameter scan over 54\% of the data.}\label{fig:PRDScan}
    % \vspace{-7mm}
\end{figure}

Interestingly, as shown in \autoref{fig:PRDScan}, all tested pairs result in $\langle X_{\text{max}}^\prime \rangle^{\rm on} < \langle X_{\text{max}}^\prime \rangle^{\rm off}$. An optimal [$E_{\rm th}$, $b_{\rm split}$] of [$10^{18.7}$\,eV,$30^\circ$] was found with a $TS = 8.4$. The selected [$E_{\rm th}$, $b_{\rm split}$] is applied as a prescription to the post-scan data set, which independently confirms the result with a $TS = 12.6$, for a total $TS=21.0$ for the full data set. 

\vspace{-.1cm}
\myparagraph{Statistical significance}
\vspace{-.1cm}
The chance probability of the observed TS occurring with in an isotropic sky is tested using Monte Carlo methods on randomized skies derived from the real data. To form each randomized sky, the arrival direction is first decoupled from the energy and \xmaxnorm{} values of each event. These are then randomly re-paired to create a new sky which maintains the real \xmax{}, energy, and sky exposure distributions, but has a scrambled arrival direction/composition pairing. The above analysis is then used to extract a $TS$ from each sky which is compared to the result in data. Skies which display more extreme on-/off-plane differences than those observed in data are tallied and used to calculate the probability of an isotropic sky generating the observed $TS$. The results of this procedure are shown in  \autoref{fig:TStoSignificanceConversionNew}.

\begin{figure}[!htb]
    \centering
    \includegraphics[width=.8\columnwidth]{Figures/MCADSig.pdf}
    % \vspace{-3mm}
    \caption{The Monte Carlo determination of the post-scan (red) and all-data (blue) significance with 1 and 10 billion randomized skies, respectively.}\label{fig:TStoSignificanceConversionNew}
    % \vspace{-3mm}
\end{figure}

For the blind, post-scan data set, the prescribed [$E_{\rm th},b_{\rm split}$] pair is used to split each randomized sky into on- and off-plane samples and a $TS$ is extracted. In one billion random skies, only 5865 resulted in a more extreme $TS$ than the 12.6 observed in data. This indicates a chance probability of $5.87\times10^{-6}$ which corresponds to 4.4\,$\sigma$. 

To calculate the significance of the result when the scan- and post-scan data sets are combined, the entire analysis chain, including the scan, is duplicated. In each random sky, 54\,\% of the data is used to scan for the [$E_{\rm th},b_{\rm split}$] pair which results in the most extreme result, fully penalizing for the scan. These values are then used to split all data in the random sky into on- and off-plane subsamples and the $TS$ for the sky is extracted. From 10 billion random skies, only 5964 resulted in a more extreme $TS$ than the 21.0 observed in data. This indicates to a chance probability of $5.96\times10^{-7}$ which corresponds to 4.9\,$\sigma$. The strong penalization of the scanned data is evident as the additional 54\,\% of the data (with \Dxmaxmunorm{} $= 8.5$\,\gcm{}) only resulted in an 11\,\% increase of the significance of the observation. 

\myparagraph{\xmax{} moments and trends}

To illustrate the difference in composition on and off the plane, the first two moments of the \xmax{} distribution in each 0.1\,\lge{} energy bin has been plotted in \autoref{fig:CompositionPlots} for both regions. Above $10^{18.7}$\,eV there is a clear separation in \xmaxmu{} for all energy bins. Most energy bins also display a separation in \xmaxsigma{}. Heavier primaries are expected to, on average, have a shallower \xmax{} and lower shower-to-shower fluctuations. Therefore the correlated difference seen here indicates that, for this data sample, primaries from the on-plane region have a higher mean mass than that of the off-plane region above $10^{18.7}$\,eV.

To evaluate the degree to which fluctuation plays a role in the observed result, the growth of the $TS$ over time has been plotted in \autoref{fig:TimeEvolution}. The time evolution of the signal is consistent with linear growth at a rate of 1.3\,$TS$\,yr$^{-1}$. This behavior is in line with expectations for a real difference in mean mass between the subsamples. The shaded region of \autoref{fig:TimeEvolution} shows preliminary data from 2019. These reconstructions were not subject to a validated reconstruction chain and may change. Still, when added, a 3.7/4.4\,$\sigma$ (post-scan/all data) statistical significance is expected. The best fit rate of growth of 1.3\,$TS$\,yr$^{-1}$ remains unchanged.

\begin{figure*}[!hbt]
\centering
    \begin{minipage}{.63\textwidth}
      \centering
      \captionsetup{width=.9\linewidth}
      \includegraphics[width=.49\textwidth,valign=t]{Figures/Mean-crop.pdf}
      \includegraphics[width=.49\textwidth,valign=t]{Figures/Sigma-crop.pdf}
      \vspace{-1mm}
      \caption{The first (left) and second (right) moments of the \xmax{} distributions from on- and off-plane regions.}
      \label{fig:CompositionPlots} 
    \end{minipage}%
    \hfill
    \begin{minipage}{.35\textwidth}
      \centering
      \vspace{-1mm}
      \includegraphics[width=\textwidth,valign=t]{Figures/TimePredict.pdf}
      \vspace{1.5mm}
      \captionof{figure}{The time evolution of the TS with significance indicated on the right. The shaded region is preliminary data.}
      \label{fig:TimeEvolution}
    \end{minipage}
\end{figure*}

\section{Experiments}
\label{sec:experiments}
\section{Experiments}
\label{sec:Exp}
\subsection{Datasets and Evaluation Metrics}
We conducted extensive experiments and ablations on two standard WSVAD evaluation datasets~\cite{sultani2018real,lv2021localizing}. As per standard in WSVAD, the training videos only have video-level labels, and the test videos have frame-level labels. Other details are given below:

\noindent\textbf{UCF-Crime}~\cite{sultani2018real} is a large-scale dataset that contains 1,900 untrimmed real-world outdoor and indoor surveillance videos. The total length of the videos is 128 hours, which contains 13 classes of anomalous events.
We follow the standard split: the training set contains 1,610 videos, and the test set contains 290 videos.

\noindent\textbf{TAD} dataset~\cite{lv2021localizing} contains real-world videos of traffic scenes with a total length of 25 hours and 1,075 average frames per video.
The videos contain more than 7 categories of anomalies that are common on roads.
The dataset is partitioned as a training set with 400 videos, and a test set with 100 videos.

\noindent\textbf{Evaluation Metrics}. Following previous works~\cite{sultani2018real,zhong2019graph}, we used the Area Under the Curve (AUC) of the frame-level ROC (Receiver Operating Characteristic) as the main evaluation metric for TAD and UCF-Crime. Intuitively, a larger AUC means a larger margin between the normal and abnormal snippet predictions, hence indicating a better anomaly classifier.
Inspired by Lv~\etal~\cite{lv2021localizing}, besides evaluating AUC on the overall test set with normal and abnormal videos, denoted as $\mathrm{AUC}_O$, we also computed the AUC on abnormal ones alone, denoted as $\mathrm{AUC}_A$.
The rationale is to remove normal videos where all snippets are normal (label 0), and keep only the abnormal ones with both kinds of snippets (label 0,1), which truly challenges a classifier's capability of localizing anomalies.
 
\subsection{Implementation Details}
\label{sec:ID}

\noindent\textbf{Video Sequence Partition}. Existing works partition each video into multiple coarse snippets, and use the \emph{average feature} in each one as the input to their classifiers (Figure~\ref{fig:pip} left). However, we find that the subtle anomaly feature is often diluted by averaging features over the coarse snippets (see Appendix).
This has less impact on the traditional MIL compared to our UMIL, as MIL only leverages the confident snippets with apparent anomalies.
Therefore, in UMIL training, we used fine-grained snippets with one-second lengths. In testing, to generate the prediction for a coarse snippet, we used the \emph{average predictions} over the fine snippets inside the coarse one (Figure~\ref{fig:pip} right).
\begin{figure}[t]
	\centering
	\includegraphics[width=\linewidth]{figure/pipeline.pdf}
    \vspace{-8mm}
	\caption{Average feature versus average prediction testing. $\theta,f$: the feature backbone and anomaly classifier, respectively.}
	\label{fig:pip}
    \vspace{-6mm}
\end{figure}

\noindent\textbf{Baseline}. We built a baseline to validate that the improvements of UMIL are indeed from the unbiased training scheme (Section~\ref{sec:Abla}), rather than the above testing scheme based on average predictions. Specifically, the baseline has exactly the same model design as UMIL, and we trained it with the MIL objective in Eq.~\eqref{eq:1} on fine snippets and tested it by averaging predictions. Hence the only difference between the baseline and UMIL is the training objective.

\noindent\textbf{Model Training}. We implemented the backbone $\theta$ with the X-CLIP-B/32 model~\cite{xclip} fine-tuned on Kinectics-400~\cite{carreira2017quo} to improve its capabilities in action recognition. We used the fully connected layer to implement the anomaly classifier $f$ and the cluster head $g$.
We trained our model with the AdamW optimizer~\cite{loshchilov2019decoupled} using an initial learning rate of $8$e-$6$, weight decay of $0.001$, and batch size of $8$.
We utilized the cosine annealing scheduler and warmed up the learning rate for 5 epochs.
Our UMIL model was pre-trained with MIL for 30 epochs, followed by 10 epochs of UMIL training.
We conducted all experiments on $4$ TITAN RTX GPUs.
We implement the max value scores as well as max margin scores~\cite{lv2021localizing} in $\mathcal{C}$ supervision of Eq~\ref{eq:4}.
We also incorporated entropy minimization as a standard auxiliary objective~\cite{liu2021cycle,long2018conditional}, and added the self-training loss, which leverages the learned unbiased anomaly classifier $f$ to generate accurate pseudo-labels on samples in the ambiguous set $\mathcal{A}$ for additional supervision. Details in Appendix.

\subsection{Main Results}
\label{sec:4.3}
\noindent\textbf{UCF-Crime and TAD}.
In Table~\ref{tab:ucf-crime}, we compared our UMIL with other state-of-the-art (SOTA) methods in both Unsupervised VAD (UVAD) and WSVAD. On UCF-Crime~\cite{sultani2018real}, UMIL achieves the best $\mathrm{AUC}_O$ and $\mathrm{AUC}_A$ among all the methods, with an improvement of +1.37\% and +1.3\%, respectively. UMIL also significantly outperforms all methods in TAD~\cite{lv2021localizing} by +3.3\% on $\mathrm{AUC}_O$ and +4.2\% on $\mathrm{AUC}_A$.

\begin{table}[t]
    \centering
    \scalebox{0.8}{
    % \resizebox{\linewidth}{!}{%
    \begin{tabular}{@{}c|c|c|c}
      \toprule\hline
        Category & Method         & $\mathrm{AUC}_O$ (\%) & $\mathrm{AUC}_A$ (\%)\\ 
      \hline\hline
      \multirow{6}{*}{\rotatebox{90}{UVAD}}
      & SVM Baseline   & 50.00  & 50.00     \\
      & Conv-AE~\cite{hasan2016learning}   & 50.60    & -   \\
      & Sohrab et al.~\cite{sohrab2018subspace}  & 58.50  & -  \\
      & Lu et al.~\cite{lu2013abnormal}  & 65.51  & -  \\
      & BODS~\cite{wang2019gods}           & 68.26  & -  \\
      & GODS~\cite{wang2019gods}           & 70.46  & -  \\ \hline
      \multirow{9}{*}{\rotatebox{90}{WSVAD}}
      & Sultani et al.~\cite{sultani2018real} & 75.41 &54.25    \\
      & Zhang et al.~\cite{zhang2019temporal}            & 78.66  & -    \\
      & Motion-Aware~\cite{zhu2019motion} & 79.10   & 62.18    \\
      & GCN-Anomaly~\cite{zhong2019graph} & 82.12  & 59.02    \\
      & Wu et al.~\cite{Wu2020not} & 82.44  & -    \\
      & RTFM~\cite{tian2021weakly}          & 84.30 & -   \\ 
      & WSAL~\cite{lv2021localizing}          & 85.38  & 67.38\\ 
      & \cellcolor{mygray}Baseline & \cellcolor{mygray}80.67  & \cellcolor{mygray}60.57 \\ 
      & \cellcolor{mygray}\textbf{UMIL}  & \cellcolor{mygray}\textbf{86.75}  & \cellcolor{mygray}\textbf{68.68} \\ \hline\bottomrule
    \end{tabular}%
    % }
    }
    \vspace{-2mm}
    \caption{Frame-level AUC performance on UCF-Crime. Best results in bold. $\mathrm{AUC}_O$ and $\mathrm{AUC}_A$ denote that the AUC computed on the overall test set and only abnormal test videos, respectively. ``UVAD'' and ``WSVAD'' under category denote Unsupervised VAD and Weakly-Supervised VAD, respectively.} 
    \label{tab:ucf-crime}
    \vspace{-4mm}
\end{table}

\noindent\textbf{Overall Observations}.
1) Notice that our baseline performs similarly (\eg, $\mathrm{AUC}_O$ on TAD) or even worse (\eg, 60.57\% versus 67.38\% on UCF-Crime $\mathrm{AUC}_O$) compared to existing MIL-based methods. This validates that the improvements from UMIL are not from the test scheme of averaging predictions. 
2) In particular, our improvement in $\mathrm{AUC}_A$ indicates that the superior performance of UMIL on $\mathrm{AUC}_O$ is not merely from easy normal videos, but also from improved capabilities to identify anomalous snippets in abnormal videos.
3) Moreover, on both datasets, WSVAD significantly improves over UVAD on $\mathrm{AUC}_O$, which empirically validates that detecting open-set anomalies in UVAD is ill-posed (Section~\ref{sec:intro}). However, the improvements in $\mathrm{AUC}_A$ are much smaller (\eg, 54.25\% over 50.00\% on UCF-Crime). This shows that the existing WSVAD methods are still biased toward the apparent normal/abnormal, causing many false positives and negatives on ambiguous snippets from the abnormal videos.
4) Our UMIL significantly improves the $\mathrm{AUC}_A$ over MIL (\eg, +4.2\% on TAD), which demonstrates the effectiveness of using ambiguous snippets in UMIL to learn an unbiased invariant classifier.
5) Interestingly, TAD tends to have larger $\mathrm{AUC}_O$ but lower $\mathrm{AUC}_A$, \eg, from UCF-Crime to TAD, UMIL's $\mathrm{AUC}_O$ is 6.2\% higher, but $\mathrm{AUC}_A$ is 2.8\% lower. The improved overall performance suggests that TAD has stronger context bias in the confident set, \ie, more apparent normal/abnormal snippets, and the dropped $\mathrm{AUC}_A$ indicates that it contains more subtle anomalies in the ambiguous snippets that are hard to detect and localize.
This also explains why our UMIL improves $\mathrm{AUC}_A$ more on TAD than UCF-Crime by incorporating ambiguous snippets to remove the context bias from the confident set.

\begin{table}[t]
    \centering
    \scalebox{0.95}{
    % \resizebox{\linewidth}{!}{%
    \begin{tabular}{@{}c|c|c|c}
      \toprule\hline
      Category       & Method         & $\mathrm{AUC}_O$ (\%) & $\mathrm{AUC}_A$ (\%)\\ \hline\hline
      \multirow{3}{*}{\rotatebox{90}{UVAD}}
      & SVM Baseline   & 50.00  & 50.00     \\
      & Luo~\etal~\cite{luo2017revisit}  & 57.89  & 55.84  \\
      & Liu~\etal~\cite{liu2018future}           & 69.13 & 55.38   \\ \hline
      \multirow{6}{*}{\rotatebox{90}{WSVAD}}
      & Sultani~\etal~\cite{sultani2018real} & 81.42 &55.97    \\
      & Motion-Aware~\cite{zhu2019motion} & 83.08  & 56.89    \\
      & GIG~\cite{lv2020global}          & 85.64 & 58.65   \\ 
      & WSAL~\cite{lv2021localizing}          & 89.64  & 61.66\\ 
      & \cellcolor{mygray}Baseline & \cellcolor{mygray}89.10 & \cellcolor{mygray}56.47 \\ 
      %& MIL baseline + RTFM\cite{tian2021weakly}        &  91.28 & 57.65 \\ 
      & \cellcolor{mygray}Ours & \cellcolor{mygray}{\textbf{92.93}} & \cellcolor{mygray}{\textbf{65.82}} \\ \hline\bottomrule
    \end{tabular}%
    % }
    }
    \vspace{-3mm}
    \caption{Frame-level AUC performance on TAD benchmark.} 
    \vspace{-3mm}
    \label{tab:tad}
\end{table}

\begin{table}[t]
\centering
\scalebox{0.75}{
% \resizebox{\linewidth}{!}{%
\begin{tabular}{cccc|cc}
\toprule\hline
Baseline & ST & RTFM* & UMIL &  $\mathrm{AUC}_O$ (\%) - UCF  &  $\mathrm{AUC}_O$ (\%) - TAD\\ \hline \hline
\checkmark & & & & 80.67 & 89.10 \\
\checkmark & \checkmark & & & 82.01 & 90.80\\ \hline
\checkmark & \checkmark & \checkmark & & 83.45 & 91.28 \\ 
\checkmark & & & \checkmark & 83.66 & 91.74 \\ 
\cellcolor{mygray}\checkmark & \cellcolor{mygray}\checkmark & \cellcolor{mygray} & \cellcolor{mygray}\checkmark & \cellcolor{mygray}\textbf{86.75} & \cellcolor{mygray}\textbf{92.93}  \\ \hline
\bottomrule
\end{tabular}%
}
\vspace{-3mm}
\caption{Ablation studies of the components in UMIL on UCF-Crime and TAD. *: we re-implemented RTFM with our backbone and average-prediction-based testing scheme for fair comparison.}
\vspace{-3mm}
\label{tab:ablation}
\end{table}

\begin{table}[t]
\centering
\scalebox{1.0}{
% \resizebox{\linewidth}{!}{%
\begin{tabular}{cccccc}
\toprule\hline
Threshold(\%) & 10 & \cellcolor{mygray}\textbf{30} & 50 & 70 & 90  \\ \hline \hline
$\mathrm{AUC}_O$ (\%) - UCF & 86.8 & \cellcolor{mygray}\textbf{86.8} & 85.9 & 84.3 & 83.1 \\ 
$\mathrm{AUC}_O$ (\%) - TAD & 92.7 & \cellcolor{mygray}\textbf{93.0} & 92.8 & 91.5 & 91.1 \\ \hline
\bottomrule
\end{tabular}%
}
\vspace{-3mm}
\caption{Ablation on the threshold to divide the confident/ambiguous snippet set on UCF-Crime and TAD.}
\label{tab:thre}
\vspace{-5mm}
\end{table}
\subsection{Ablations}
\label{sec:Abla}

\noindent\textbf{Components}. Our approach has 2 main components: 1) the self-training objective; 2) the UMIL objective in Eq.~\eqref{eq:4}. We validate the effectiveness of each component in Table~\ref{tab:ablation} with $\mathrm{AUC}_O$. All ablations in the table are on the equal ground---using average prediction instead of average feature for anomaly detection (\ie, Baseline). By comparing the first two lines, we observe that self-training can improve $\mathrm{AUC}_O$ from $80.67\%$ to $82.01\%$ on UCF-crime and $89.10\%$ to $90.80\%$ on TAD. To independently evaluate the effectiveness of UMIL objective, we re-implement the SOTA RTFM~\cite{tian2021weakly} using our backbone and add the self-training objective, namely RTFM*. The result is listed in line 3. Our UMIL in line 4 still significantly outperforms RTFM* (+$3.3\%$ on UCF-crime and +$1.7\%$ on TAD), hence validating the effectiveness of our unbiased learning objectives.
\begin{figure}[t]
	\centering
	\includegraphics[width=0.4\textwidth]{figure/cm.pdf}
    \vspace{-4mm}
	\caption{Ablations on the trade-off parameters.}
	\label{fig:cm}
    \vspace{-4mm}
\end{figure}

\noindent\textbf{Confident Threshold}. We then conducted experiments to analyze the effects of the variance threshold for dividing confident and ambiguous snippets as in Section~\ref{sec:step1}.
Specifically, we selected $k$ (\%) training snippets with the minimum variance on their prediction history with varying $k$ as in Table~\ref{tab:thre}. Overall the threshold is easy to determine, \ie, 10-50\% is a reasonable range with 30\% being the best.

\noindent\textbf{Trade-off Parameters}. Recall that we use $\alpha$ and $\beta$ in Eq.~\eqref{eq:4} as the trade-off for the supervision from the ambiguous set $\mathcal{A}$ and clustering, respectively. We empirically find in Figure~\ref{fig:cm} that $\alpha,\beta=0.1$ are suitable across the two datasets, hence we used this setting in the experiments by default. In general, the choice of $\alpha$ depends on the strength of the context bias in the confident set, \eg, TAD has strong bias as analyzed in Section~\ref{sec:4.3}, which cannot be overcome with a small $\alpha$ (\eg, $\alpha$=0.01 has low performance).

\begin{figure}
    \centering
    \footnotesize
    \scalebox{1.05}{
    \begin{subfigure}[t]{0.23\textwidth}
         \includegraphics[width=\textwidth]{figure/rocs_u.pdf}
         \phantomcaption
         \label{fig:roca}
    \end{subfigure}
    \begin{subfigure}[t]{0.23\textwidth} % the hidden unwanted image
         \includegraphics[width=\textwidth]{figure/rocs_t.pdf}
         \phantomcaption
         \label{fig:rocb}   
    \end{subfigure}}
    \vspace{-8mm}
    \caption{ROC curves on UCF and TAD. Note that we only show part of the curves for visual clarity, as the other part of the methods have a large overlap when the true positive rate approaches 100\%.}
    \label{fig:roc}
    \vspace{-6mm}
\end{figure}

\noindent\textbf{Class-wise AUC}. On UCF-Crime dataset, the class of anomaly in each test video is given. This allows us to plot the class-wise $\mathrm{AUC}_A$ to examine models' capabilities to detect subtle abnormal events. In Figure~\ref{fig:hist}, we compared UMIL with baseline and RTFM*, where ``Average'' shows the overall $\mathrm{AUC}_A$ and the rest shows the class-wise one.
We have the following observation:
1) Both of the two MIL-based methods perform well on human-centric anomaly classes with drastic motions, \eg, ``Assault'' and ``Burglary''. These classes correspond to apparent anomalies as the backbone expresses the human action feature well (fine-tuned on the action recognition Kinetics400 dataset\cite{carreira2017quo}).
2) However, we notice that they easily fail to distinguish anomalies with subtle motions, \eg, ``Arson'' and ``Vandalism'', as well as non-human-centric anomalies, \eg, ``Explosion''. These classes correspond to ambiguous anomalies discarded by the biased training in MIL.
3) Our UMIL performs similarly on the above apparent anomaly classes and much better on the other subtle anomalies, which largely contributes to the superior anomaly detection and localization performance.
Overall, observation 1 and 2 empirically verifies the biased prediction situation of MIL in Figure~\ref{fig:abstract} and Figure~\ref{fig:2}. In contrast, our UMIL convincingly improves the performance on ambiguous anomalies with almost no sacrifice on the confident ones, which validates the effectiveness of our approach, \ie, identifying the invariance between the two types of anomalies to remove the bias in MIL.
\begin{figure}[t]
	\centering
	\includegraphics[width=\linewidth]{figure/subtle.pdf}
    \vspace{-6mm}
	\caption{Class-wise $\mathrm{AUC}_A$ of three methods on UCF-Crime.}
	\label{fig:hist}
    \vspace{-7mm}
\end{figure}

\begin{figure*}[t!]
	\centering
    \vspace{-4mm}
	\includegraphics[width=1\textwidth]{figure/cases_new.pdf}
    \vspace{-10mm}
	\caption{Visualization cases of ground-truth and anomaly score curves of various approaches. The white and black triangles denote the location of the normal and abnormal frame displayed on the left, respectively. The green curves represent the anomaly predictions of various methods. The pink background corresponds to the ground-truth abnormal regions.}
	\label{fig:cases}
    \vspace{-4mm}
\end{figure*}

\noindent\textbf{ROC Curve}. In Figure~\ref{fig:roc}, we draw the ROC Curve on the overall test set for our baseline, the re-implemented RTFM* and UMIL, which shows the true and false positive rate for detecting anomaly on a sweeping threshold over the predictions. VAD is evaluated using the area under this curve to demonstrate the overall separation of normal and abnormal snippet predictions. However, when applying a detector for real-world usage, we need to choose a specific threshold (\eg, with a maximum tolerable false positive rate). We observe from Figure~\ref{fig:roc} that our UMIL outperforms the two MIL baselines in every inch, which further shows the strength of our proposed unbiased training.

\noindent\textbf{Qualitative Analysis}. In Figure~\ref{fig:cases}, we show the continuous predictions of anomaly probabilities from our baseline, RTFM*, and our UMIL on 4 test videos on UCF-crime. We summarize the observations:
1) For the MIL baseline (2nd column), we observe that it assigns a larger probability on the pre-explosion snippets from B1 and B2 (top two videos), \eg, workers performing maintenance and snippets with smoke, yet the actual explosion may have a lower prediction (\eg, comparing the height of the green lines on the white and black triangle locations). Similarly, on B3, the running person (white triangle) triggers a larger anomaly prediction than the actual vandalism (black triangle). This further illustrates the biased prediction problem in MIL.
2) RTFM (3rd column) uses feature magnitudes to assist anomaly detection by assuming anomalous snippets have larger magnitudes, which indeed improves over the baseline sometimes, \eg, R2 is no longer biased to smoke. However, its assumption has no guarantee to hold and hence the failure on subtle anomalies persists, \eg, false alarm in R1 white triangle location and low prediction in R3 black triangle location.
3) In contrast, our UMIL localizes the anomalies accurately in U1-U3, \eg, having consistently high scores in the pink areas, which holds its ground on the name ``unbiased''.
4) In the 4th video, however, RTFM's prediction in the pink area is more consistent than ours. By inspecting the frames on the left, we realize that the two peaks in the pink area of U4 correspond to the burning fire and the running suspect caught on fire. Hence UMIL's prediction is reasonable and sufficient for triggering the alarm on the first peak.

\noindent\textbf{Computational Efficiency}. Lastly, we investigated the speed of the proposed model.
For inference, our method processes a 5-frame clip in $0.003$ seconds on a Nvidia 2080Ti GPU.
Notably, this is almost $80 \times$ faster than the SOTA RTFM~\cite{tian2021weakly}, which spends 0.76 seconds to process a 16-frame clip on Nvidia 2080Ti.
Thanks to our unbiased training scheme, we can fine-tune the backbone to learn a WSVAD-tailored representation, which achieves even better performance than existing SOTA.
This also shows the promising future of UMIL in real-time applications.

\section{Conclusion}
\label{sec:conclusion}
%\section{}
%\label{sec:resDir}


\section{Conclusion}
\label{sec:conclusion}
% <>
Since its advent in 1931, Koopman operator theory \cite{koopman:1931} has only recently been actively utilized for solving practical problems, thanks to the introduction of the DMD algorithm in 2008 \cite{schmid:2008}. Since then, a multitude of DMD algorithm variations have risen to prominence and found utility across various fields. A notable feature of our survey paper was reviewing and categorizing the results of over 100 research papers based on both application and algorithm type in smart mobility and vehicle engineering  (see Table~\ref{tab1} and Section~\ref{sec:vehicApp}).  Additionally, this survey paper identified potential research gaps in smart mobility and vehicular engineering applications (Remarks~\ref{remGap1}--\ref{remGap6}). Finally, this review paper discussed theoretical aspects of Koopman operator theory that have been largely neglected by the smart mobility and vehicle engineering community and yet have large potential for contributing to solving open problems in these areas (see Section~\ref{subsec:theorIssue}).

\noindent{\textbf{Future Research Directions.}}	Given the emergence of cyber-threats against connected and autonomous vehicles as well as robotic systems (see, e.g.,~\cite{nekouei2021randomized,mohammadi2022generation}), a future research direction might include utilizing Koopman operator-based algorithms for designing cyber-resilient vehicular and smart mobility applications (see, e.g.,~\cite{taheri2022data} for a related line of research). Another potential research direction is using Koopman operator-based algorithms for predicting the motion of vulnerable road users (VRUs), e.g., pedestrians and cyclists (see, e.g.,~\cite{pool2019context,scholler2020constant}). Finally, rehabilitation robotics and robotic exoskeletons can be the benefactors of the predictive capabilities of Koopman operator-based algorithms for detecting tripping events and/or system  identification in various modes of locomotion (see, e.g.,~\cite{kumar2019extremum,aprigliano2019pre}).



%Fig. 1 depicts the accumulation of such algorithms since 2014, which are particular to vehicle engineering and smart mobility, i.e., the focus of this review. Table 1 summarizes the varieties of relevant algorithms developed in those studies. Furthermore, we have highlighted theoretical issues, whose expansion will have potential applications to the wide research area of smart mobility and vehicle engineering.  

%Although fairly comprehensive, we have found several gaps in this research area. In particular, we could not find any studies related to elevators, robots/vehicles employing crawling, slithering, hopping or peristaltic locomotion, arctic or special-terrain vehicles such as those employing screws or tracks, hovercraft and other amphibious vehicles or subsystems which tolerate flexible environments, classification or guidance systems related to vehicles for drilling or agriculture, or for current-ripple, power-split, battery health monitoring, nuclear propulsion, exoskeletons/prosthetics, personal mobility, motorsports, specialized rovers or similar open problems in emerging areas.  These examples are, of course, not exhaustive.  
%
%The purely data-driven nature of Koopman operators holds the promise of capturing unknown and complex dynamics for reduced-order model generation and system identification, through which the rich machinery of linear control techniques can be utilized. The emergent nature of the smart mobility and vehicular-related applications, where  the Koopman operator  in each particular application needs to be approximated, implies that the development of various Koopman operator approximation  algorithms is expected to grow along with the vehicular problems they aim to solve.  Given the ongoing development of this research area and the many existing open problems in the fields of smart mobility and vehicle engineering, a survey of techniques and open challenges of applying Koopman operator theory to this vibrant area is warranted.  To the best of our knowledge, this survey paper is the \emph{first of its kind} reviewing the applications of Koopman operator theory within a focused research area, namely, smart mobility and vehicle engineering applications. A \emph{notable feature} of our survey paper is reviewing and categorizing the results of over 100 research papers based on both application and algorithm type  (see Tables~\ref{tab1}--~\ref{tab4} and Section~\ref{sec:vehicApp}) that are concerned with the applications of Koopman operator theory to the field of smart mobility and vehicular engineering. Such a \emph{comprehensive and  detailed categorization} will be beneficial to the research practitioners working in the field.  Furthermore, this review paper discusses theoretical aspects of Koopman operator theory that have been largely neglected by the smart mobility and vehicle engineering community and yet have large potential for contributing to solving open problems in these areas. Additionally, our survey paper seeks to \emph{identify gaps} in the smart mobility and vehicle engineering research where new and existing Koopman operator-based methods have the potential to further develop and address unsolved problems  potentially benefiting from the perspectives of nonlinear system identification, control, global linearization, and the predictive powers that Koopman operator theory has to offer (see, e.g., Remarks~\ref{remGap1}--\ref{remGap6}). 


%
% ---- Bibliography ----
%
% BibTeX users should specify bibliography style 'splncs04'.
% References will then be sorted and formatted in the correct style.
%
\bibliographystyle{splncs04}
\bibliography{refs}

\end{document}
