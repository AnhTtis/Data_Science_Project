% 1. Introduces the importance of early detection of breast cancer and the limited effectiveness of mammographic screening

% 1.1 Broader Impact
% Breast cancer + cancer screening

% 1.2 Set up and the bridge
% Cancer screening reduces mortality.
% 1.3 The Problem (However .., the BUT)
% Studies have shown low sensitivity unless paired with highly sensitivity methods (MRI, US) are too costly to become a part of routine practice in nation-wide screening settings.  Hence the need for personalized screening programs based on accurate assessment of patient's risk is desired.

%2.1 Review of current practice (Gail, TC) and some recent DL based MMG/DBT risk models to address the problem above.
%2.2 Limitation of current SOTA -- lack of prior knowledge in the learning process

%3 Our contribution: transformer decoder framework to incorporate patient's prior information etc. etc.

Breast cancer impacts women globally~\cite{BreastCancerStatistics} and mammographic screening for women over a certain age has been shown to reduce mortality~\cite{hakama2008cancer,paci2012summary,duffy2020effect}.
%Many developed countries have therefore adopted mammographic screening for women over a certain age, as it has been shown to reduce mortality~\cite{duffy2002impact}. 
However, studies suggest that mammography alone has limited sensitivity~\cite{health2007screening}. To mitigate this, supplemental screening like MRI or 
a tailored screening interval have been explored to add to the screening protocol~\cite{bakker2019supplemental,hussein2023supplemental}. However, these imaging techniques are expensive and add additional burdens for the patient. Recently, several studies~\cite{yala2021toward,yala2022multi,eriksson2022risk} revealed the potential of artificial intelligence (AI) to develop a better risk assessment model to identify women who may benefit from supplemental screening or a personalized screening interval and these may lead to improved screening outcomes.
% Additional screening with more sensitive methods such as MRI or ultrasound may improve sensitivity ~\cite{liu2020decoupling} but a population wide implementation with such methods is infeasible due to the additional cost.  
%However, studies have shown that mammographic screening has limited sensitivity for some women~\cite{health2007screening}. 
%Cancers that could potentially be found with more sensitive screening methods are routinely missed~\cite{liu2020decoupling}. For example, adding MRI or ultrasound screening would improve early detection, but are too costly to offer to the whole screening population.
% Ideally, more sensitive yet costly imaging should be targeted to patients with higher risk of developing breast cancer to minimize the cost of screening, while simultaneously minimizing mortality. In other words, a reliable and accurate method to estimate breast cancer risk would enable more personalized screening. 

% 2. Discuss current questionnaire-based models, such as Gail and Tyrer-Cuzick, and their limitations.
In clinical practice, breast density and traditional statistical methods for predicting breast cancer risks such as the Gail~\cite{BCRiskTool2011} and the Tyrer-Cuzick models~\cite{tyrer2004breast} have been used to estimate an individual's risk of developing breast cancer. However, these models do not perform well enough to be utilized in practical screening settings~\cite{brentnall2020risk} and require the collection of data that is not always available. Recently, deep neural network based models that predict a patient's risk score directly from mammograms have shown promising results~\cite{brentnall2020risk,eriksson2022risk,yala2021toward,liu2020decoupling,gastounioti2022artificial}. These models do not require additional patient information and have been shown to outperform traditional statistical models. 
%However, these models have demonstrated low performance, possibly due to incomplete incorporation of factors affecting breast cancer risk~\cite{brentnall2020risk}. Additionally, collecting the required data for these models can be expensive and time-consuming.
% 3. Recent advances in the use of deep neural networks trained on mammograms for breast cancer risk prediction are described, highlighting their improved performance over previous models.
%Recently, risk models based on deep neural networks that use mammograms alone to predict a patient's risk have been introduced. 

%This approach reduces the need to collect patient information and has shown promising results in terms of improved performance~\cite{eriksson2022risk,yala2021toward,liu2020decoupling,gastounioti2022artificial,wanders2022interval,lehman2022deep}. 
%Previous research has demonstrated that these models exhibit superior performance when compared to models that rely on questionnaire-based methods~\cite{yala2021toward}. 

% 4. The possibility of using historical images for breast cancer risk prediction is mentioned, noting that conventional radiologists also utilize historical images.
When prior mammograms are available, radiologists compare prior exams to the current mammogram to aid in the detection of breast cancer. Several studies have shown that utilizing past mammograms can improve the classification performance of radiologists in the classification of benign and malignant masses \cite{hayward2016improving,varela2005use,roelofs2007importance,sumkin2003optimal}, especially for the detection of subtle abnormalities \cite{roelofs2007importance}. More recently, deep learning models trained on both prior and current mammograms have shown improved performance in breast cancer classification tasks~\cite{park2019screening}. Integrating prior mammograms into deep learning models for breast cancer risk prediction can provide a more comprehensive evaluation of a patient's breast health.

% 5. The limitation of not utilizing past images in existing deep learning models is explained, and it is proposed that this paper's model can utilize past images to predict breast cancer risk, achieving significant performance improvement over existing models.

In this paper we introduce a deep neural network that makes use of prior mammograms, to assess a patient's risk of developing breast cancer, dubbed \ours (PRIor Mammogram Enabled risk prediction). We hypothesize that mammographic parechnymal pattern changes between current and prior allow the model to better assess a patient's risk. Our method is based on a transformer model that uses attention~\cite{vaswani2017attention}, similar to how radiologists would compare current and prior mammograms. 

The method is trained and evaluated on a large and diverse dataset of over 9,000 patients and shown to outperform a model based on state-of-the art risk prediction techniques for mammography \cite{yala2021toward}. To the best of our knowledge, this is the first breast cancer risk prediction model which effectively leverages the information from both prior and current mammograms.

% Therefore, in this work, we demonstrate the value of prior mammograms in improving the risk prediction performance and present a method for learning with both current and prior exams which leads to large gains in risk prediction performance.
% here we need to explain a bit about the network architecture
% Specifically, we present a transformer decoder based architecture that learns to compare spatial context across current and prior mammograms using the attention mechanism.
% Here we need to add one sentence about the dataset
% Our approach outperforms previous models that only use current mammograms, as demonstrated on a large and diverse in-house mammography dataset of over 9000 patients.
% Our work demonstrates the effectiveness of incorporating past mammograms into a deep learning model for breast cancer risk prediction. 
% The effectiveness of integrating past mammograms into breast cancer risk prediction model, as demonstrated in this study, has the potential to enable the model to catch the patterns of change from past (e.g. change in breast density), which improves long term risk prediction.
% here we need to explain a bit about why and in what data it works best (for example best on long term risk, best for high density breast?)



%%% Previous draft
% % 4. Mention any potential limitations or challenges with current models and the need for further research in this area.
% Despite the advances of deep neural networks in breast cancer risk prediction, their performance and generalizability in certain important subpopulations, such as women with dense breasts or women aged 45 to 54, remain largely unknown~\cite{AmericanCancerSociety2022,gastounioti2022artificial}. This limits the effectiveness of these models in real-world settings and calls for further research to understand the limitations of current models and to develop more effective models for breast cancer risk prediction. Moreover, it is important to ensure that these models are robust, accurate, and generalizable across different ethnicities and populations. Additionally, there is a lack of understanding of the models' ability to predict breast cancer risk and detect highly visible cancers, which might have different parenchymal patterns~\cite{liu2020decoupling}. To improve the performance and generalizability of breast cancer risk prediction models and ensure their responsible deployment in the clinic, addressing these limitations is of utmost importance.

% % 5. Discuss the proposed method and contribution, strengths of the proposed method, potential challenges or limitations, potential applications, and future directions.
% % TODO