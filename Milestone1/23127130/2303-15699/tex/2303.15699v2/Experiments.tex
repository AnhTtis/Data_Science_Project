\subsection{Dataset}
We compiled an in-house mammography dataset comprising 16,113 exams (64,452 images) from 9,113 patients across institutions from the United States, gathered between 2010 and 2021. Each mammogram includes at least one prior mammogram. The dataset has 3,625 biopsy-proven cancer exams, 5,394 biopsy-proven benign exams, and 7,094 normal exams. Mammograms were captured using Hologic (72.3\%) and Siemens (27.7\%) devices. We partitioned the dataset by patient to create training, validation, and test sets. The validation set contains 800 exams (198 cancer, 210 benign, 392 normal) from 400 patients, and the test set contains 1,200 exams (302 cancer, 290 benign, 608 normal) from 600 patients.
All data was de-identified according to the the U.S HHS Safe Harbor Method. Therefore, the data has no  PHI (Protected Health Information) and IRB (Institutional Review Board) approval is not required.
% We compiled an in-house mammography dataset comprising 38,106 four-view full-field digital mammograms from 21,993 patients across 11 US institutions, collected between 2010 and 2021. The dataset includes 2,999 biopsy-confirmed cancer-positive cases, 5,000 benign cases confirmed by biopsy, and 13,994 (7,000 follow-up benign and 6,994 normal) cases with at least one year of follow-up. The mammograms were acquired using devices from two manufacturers, Hologic (73.8\%) and Siemens (26.2\%). For model evaluation, we sampled a validation set of 1,000 patients (250 cancer, 250 benign, and 500 normal) with three exams spaced approximately 2 years apart per patient.

\subsection{Evaluation}

We make use of Uno's C-index~\cite{uno2011c} and the time-dependent AUC~\cite{kamarudin2017time}. The C-index measures the performance of a model by evaluating how well it correctly predicts the relative order of survival times for pairs of individuals in the dataset. The C-index ranges from 0 to 1, with a value of 0.5 indicating random predictions and a value of 1 indicating that the model is perfect. Time-dependent ROC analysis generates an ROC curve and the area under the curve (AUC) for each specific time point in the follow-up period, enabling evaluation of the model's performance over time.
To compare the C-index of two models, we employ the compareC~\cite{kang2015comparing} test, and make use of the DeLong test \cite{delong1988comparing} to compare the time-dependent AUC values. Confidence bounds are generated using bootstrapping with 1,000 bootstraps. 

We evaluate the effectiveness of \ours  by comparing it with two other models: (1) baseline based on MIRAI, a state-of-the art risk prediction method from~\cite{yala2021toward}, and (2) \prior, a model that uses prior images by simply summing $x_{curr}$ and $x_{prior}$ without the use of the transformer decoder. 

\subsection{Implementation Details}

Our model is implemented in Pytorch and trained on four V100 GPUs. We trained the model using stochastic gradient descent (SGD) for 20K iterations with a learning rate of 0.005, weight decay of 0.0001, and momentum of 0.9. We use a cosine annealing learning rate scheduling strategy~\cite{loshchilov2016sgdr}.

We resize the images to 960 $\times$ 640 pixels and use a batch size of 96. To augment the training data, we apply geometric transformations such as vertical flipping, rotation and photometric transformations such as brightness/contrast adjustment, Gaussian noise, sharpen, CLAHE, and solarize. Empirically, we find that strong photometric augmentations improved the risk prediction model's performance, while strong geometric transformations had a negative impact. This is consistent with prior work \cite{liu2020decoupling} showing that risk prediction models focus on overall parenchymal pattern.

\subsection{Results}

\textbf{Ablation Study.}
\begin{figure}
       \centering
        \setlength{\tabcolsep}{1pt}
        {\scriptsize
        \begin{tabular}{c c c c c c c }
            { Original } &
            \multicolumn{2}{c}{  } &
            \multicolumn{4}{c}{$\longleftarrow$ Object level variations $\longrightarrow$} \\
            \includegraphics[width=0.185\linewidth]{images/ablation/chair.jpg} &
            \multicolumn{2}{c}{  } &
            \includegraphics[width=0.185\linewidth]{images/ablation/1_only_prompt_mixing/bench.jpg} &
            \includegraphics[width=0.185\linewidth]{images/ablation/1_only_prompt_mixing/stool.jpg} &
            \includegraphics[width=0.185\linewidth]{images/ablation/1_only_prompt_mixing/armchair.jpg} &
            \includegraphics[width=0.185\linewidth]{images/ablation/1_only_prompt_mixing/saddle.jpg} \\
            \multicolumn{3}{c}{  } &
            \multicolumn{4}{c}{ Only Prompt Mixing } \\
            \multicolumn{3}{c}{ } &
            \includegraphics[width=0.185\linewidth]{images/ablation/2_with_self_attn_injection/bench.jpg} &
            \includegraphics[width=0.185\linewidth]{images/ablation/2_with_self_attn_injection/stool.jpg} &
            \includegraphics[width=0.185\linewidth]{images/ablation/2_with_self_attn_injection/armchair.jpg} &
            \includegraphics[width=0.185\linewidth]{images/ablation/2_with_self_attn_injection/saddle.jpg} \\
            \multicolumn{3}{c}{  } &
            \multicolumn{4}{c}{ + Attention-Based Shape Localization } \\
            \multicolumn{3}{c}{ } &
            \includegraphics[width=0.185\linewidth]{images/ablation/3_background_blending/bench.jpg} &
            \includegraphics[width=0.185\linewidth]{images/ablation/3_background_blending/stool.jpg} &
            \includegraphics[width=0.185\linewidth]{images/ablation/3_background_blending/armchair.jpg} &
            \includegraphics[width=0.185\linewidth]{images/ablation/3_background_blending/saddle.jpg} \\
            \multicolumn{3}{c}{  } &
            \multicolumn{4}{c}{ + Controllable Background Preservation } \\
        \end{tabular}
        }
    \vspace{1mm}
    \captionof{figure}{
    Ablating our full object variations pipeline. Original image was crated using the prompt ``A \emph{chair} with a dog on it''. 
    }
    \vspace{-10pt}
    \label{fig:ablation}
\end{figure}

To better understand the merit of the transformer decoder, we first performed an ablation study on the architecture. Our findings, summarized in Table~\ref{table:ablation}, include two sets of results: one for all exams in the test set and the other by excluding cancer exams within 180 days of cancer diagnosis which are likely to have visible symptoms of cancer, by following a previous study~\cite{yala2021toward}. This latter set of results is particularly relevant as risk prediction aims to predict unseen risks beyond visible cancer patterns. We also compare our method to two other models, the state-of-the-art baseline and \prior models.

As shown in top rows in Table~\ref{table:ablation}, the baseline obtained a C-index of 0.68 (0.65 to 0.71). By using the transformer decoder to jointly model prior images, we observed improved C-index from 0.70 (0.67 to 0.73) to 0.73 (0.70 to 0.76).
%When adding the prior image, the C-index increased to 0.70 (0.67 to 0.73) when adding the prior image and to 0.73 (0.70 to 0.76) using the transformed decoder. 
The C-index as well as all AUC differences between the baseline and the \ours are all statistically significant (p < 0.05) except the 4-year AUC where we had limited number of test cases.
%except the 4-year AUC between the baseline and \ours are all statistically significant (p < 0.05). 
%The 1, 2, and 3-year AUC were found to be significantly higher than the baseline (p < 0.05) and marginally significant for the 4-year AUC, where we had limited number of test cases.

We observe similar performance improvements when evaluating using cases with at least 180 days to cancer diagnosis.
%The results of the model on the dataset that excluded 180 days of cancer diagnosis are shown in the bottom rows of Table~\ref{table:ablation}.
%Both \prior and \ours had significantly higher C-index than the baseline, with \prior's C-index at 0.68 (0.63 to 0.73) and \ours's C-index at 0.70 (0.66 to 0.74), all with a p-value < 0.05.
%and the baseline's C-index at 0.63 (0.59 to 0.67), all with a p-value < 0.05.
%The 2-year and 3-year AUC of \ours were significantly higher (p < 0.05), and the 4-year AUC was marginally higher (p = 0.08) than that of the baseline. 
%Similarly, for \prior, the 3-year AUC was significantly higher (p < 0.05), and the 2-year and 4-year AUC were marginally higher than that of the baseline (p = 0.16 and 0.05, respectively).
Interestingly, the C-index as well as time-dependent AUCs of all three methods decreased compared to when evaluating using all cases. 
The intuition behind this result is that mammograms taken near the cancer diagnosis ( < 180 days) likely contain visible signs of cancer and thus the task of risk prediction is easier.
The model must learn patterns of risk, not visible signs of cancer, in order to perform well under this evaluation setting. Our results support this intuition as the performance improvements over the baseline are much more pronounced for longer term risk (3, 4-year AUC) than short term risk (1 year). 
%This suggests that cancer cases diagnosed in early stage are relatively easier to predict.
%The difference in performance between the baseline method and the others became more pronounced compared to the results of all cases, except for the 1-year AUC.
The \prior and \ours models, which incorporate prior mammograms, show high performance for long-term risk prediction (3, 4-year AUC), indicating that considering changes in breast over time contain useful information for breast cancer risk prediction.

% However, the \prior model exhibited lower performance than the baseline for short-term risk prediction (1, 2-year AUC), as detailed blobs are more critical in short-term risk prediction. This finding is consistent with previous work~\cite{liu2020decoupling}. It suggests that modeling without considering regional differences between prior and current mammograms leads to poorer performance.

Lastly, we empirically confirm that a transformer decoder effectively models spatial relations between prior and current mammograms by demonstrating consistent performance improvements of \ours across both short-term and long-term risk prediction settings.
%In contrast, \ours leverages a transformer decoder to consider the spatial relations between prior and current mammograms, resulting in high performance for both short-term and long-term risk prediction. 
Our results suggest that incorporating changes in patients using prior mammograms and a transformer decoder improves the performance of breast cancer risk prediction models.

\begin{table}[t]
\centering
\caption{To better understand why the addition of prior images works, we split our test set into two groups based on the mammographic density: change and no change. The first and second row corresponds to performance of the baseline and RP+ model, respectively. Empty cell indicates an insufficient number of cases available for evaluation.}
\label{table:density_change}
\begin{tabularx}{\linewidth}{|>{\centering\arraybackslash}X*{6}{|>{\centering\arraybackslash}X}|}
\hline
\multirow{2}{*}{\textbf{Density chg}} & \multirow{2}{*}{\textbf{C-index}} & \multicolumn{4}{c|}{\textbf{Time-dependent AUC}} \\
\cline{3-6}
& & \textbf{1-year} & \textbf{2-year} & \textbf{3-year} & \textbf{4-year} \\
\hline
\multirow{2}{*}{\textbf{Change}} 
& 0.63$_{\pm0.14}$ & 0.74$_{\pm0.17}$ & 0.66$_{\pm0.18}$ & 0.56$_{\pm0.16}$ & - \\
& 0.75$_{\pm0.10}$ & 0.82$_{\pm0.13}$ & 0.76$_{\pm0.14}$ & 0.74$_{\pm0.14}$ & - \\
\hline
\multirow{2}{*}{\textbf{No change}} 
& 0.69$_{\pm0.03}$ & 0.70$_{\pm0.05}$ & 0.72$_{\pm0.05}$ & 0.72$_{\pm0.04}$ & 0.71$_{\pm0.09}$ \\
& 0.73$_{\pm0.03}$ & 0.74$_{\pm0.05}$ & 0.75$_{\pm0.05}$ & 0.77$_{\pm0.04}$ & 0.76$_{\pm0.08}$ \\
\hline
\end{tabularx}
\end{table}


\textbf{Analysis based on density.}
To better understand why adding prior images improves performance, we divided our test set into subgroups to examine the performance of the baseline model and the \ours model on each of these groups.
Mammographic breast density is one of the most important risk factor to predict breast cancer~\cite{veronesi2005goldhirsch,lee2017risk}.
Women with dense breasts have a four-to six-fold higher risk of breast cancer~\cite{boyd2013mammographic}.
The addition of mammographic breast density has improved the performance traditional breast cancer risk models~\cite{brentnall2015mammographic} and can therefore help us understand why the addition of prior images works. 

Mammographic breast density was determined using the Breast Imaging Reporting and Data System (BI-RADS) composition classification. BI-RADS category A,B are defined as fatty breasts and BI-RADS category C, D are classified as dense breasts.
To determine the density category, we employed an internally developed density prediction model, as most exams lack BI-RADS ground truth. This model achieved an accuracy of 0.81 on the internal density validation set.

% Our hypothesis is that incorporating prior density information would improve the cancer risk prediction.
We categorized the exams into two groups based on changes in density: "change" and "no change".
Density change was defined according to whether the BI-RADS category is changed in the current image as compared to the prior image.
As shown in Table~\ref{table:density_change}, the baseline model performs poorly for "change", with a C-index of 0.63 (0.49 to 0.77), especially for long-term risk prediction, with 3-year AUC of 0.56 (0.40 to 0.72). This suggests that the baseline model have limitations in accurately predict long-term risk when there is a density change from the prior exam.
However, \ours is able to predict long-term risk accurately even when a density change has occurred (3-year AUC = 0.74 (0.60 to 0.88)), by learning to refer previous exams properly. This demonstrates the potential usefulness of incorporating past mammogram information into breast cancer risk prediction models.
Thus, we believe that incorporating prior exams is important to identify changes in texture which are important for long term risk prediction.

% our proposed model shows a larger improvement in C-index for both "Increase" (0.09) and "Decrease" (0.15) categories compared to the "No change" category (0.04). These results suggest that incorporating density change information into our model improves its performance for breast cancer risk prediction.


\section{Density method}

In this section we discuss the numerical implementation of the density method, which follows the same lines as \cite{BMO_2022} with some new technical difficulties working on the sphere. The sphere is assumed to be discretized by a mesh that remains the same during the  optimization process. Let $V_h= \Span(\phi_1,...,\phi_n) \subset \H^1(\S^N)$ be a finite element space. If $v = \sum_i v_i \phi_i$ we denote by $\bar v = (v_1,...,v_n)^T$ its coordinates on the basis $(\phi_i)_i$. Even if we made the choice here to discretize both the density and the eigenfunctions on the same FE space $V_h$, we could have considered different ones as we did in our previously cited work. Let $\rho \in V_h$ be a density (i.e. $\rho : \S^n \to [0,1]$). We denote by $\bar \mu_k^\eps(\rho)$ the eigenvalue of the finite-dimensional eigenvalue problem :
\begin{equation}
    \label{eqn:fe_eigenvalue}
    \bm{M}^\eps(\rho) \bar u_k^\eps(\rho) = \bar \mu_k^\eps(\rho) \bm{K}^\eps(\rho) \bar u_k^\eps(\rho)
\end{equation}

where
\[
    \bm{M}^\eps(\rho) = \left( \int_{\S^n} (\rho+\eps) \nabla \phi_i \nabla \phi_j \right)_{i,j}
\]
and
\[
    \bm{K}^\eps(\rho) = \left( \int_{\S^n} (\rho+\eps^2) \phi_i \phi_j \right)_{i,j}.
\]
Since we use a gradient-based optimization method, we need to differentiate $\bar \mu_k^\eps(\rho)$ with respect to its coordinates $(\rho_1,...,\rho_n)$ in the basis $(\phi_i)_i$. Assuming that $ \bar \mu_k^\eps(\rho)$ is simple, we can differentiate the equation \eqref{eqn:fe_eigenvalue} and multiply on the left by $(\bar u_k^\eps(\rho))^T$ to get
\begin{equation}
    \partial_l \bar \mu_k^\eps = \frac{(\bar u_k^\eps)^T\left( \partial_l \bm{M}^\eps  - \bar \mu_k^\eps \partial_l \bm{K}^\eps \right) \bar u_k^\eps}{(\bar u_k^\eps)^T  \bm{K}^\eps(\rho) \bar u_k^\eps}
\end{equation}
where the derivatives of the matrices are respectively
\[
    \partial_l \bm{M}^\eps(\rho) = \left( \int_{\S^n} \phi_l \nabla \phi_i \nabla \phi_j \right)_{i,j}
\]
and
\[
    \partial_l \bm{K}^\eps(\rho) = \left( \int_{\S^n} \phi_l \phi_i \phi_j \right)_{i,j}.
\]

One may remember that since we are on the sphere, we don't have scale-homogeneity of the eigenvalue as it is the case in $\R^n$. Hence we have to enforce the condition $\int_{\S^n} \rho = m$, which in the discrete case becomes
$\bar \rho \cdot \bm g = m$ where
\[
  \bm g = \left(\int_{\S^n} \phi_l\right)_l.
\]

\subsection{Multiple eigenvalues.}

One of the main hurdles in spectral shape optimization is to handle the multiplicity of the eigenvalues. Indeed, in practice, the optimal density is expected to have high multiplicity. The method presented in \cite{BMO_2022} consisted in adding a constraint forcing the eigenvalues that were close (depending on a certain threshold $\sigma$) to get them even closer. While it yielded good results in the planar case, it suffered one peculiar issue here, especially in the case of $\mu_1$. Indeed, even if the optimal density seems to be of multiplicity $2$, the third eigenvalue is observed to be close to $\mu_2$ leading the previous method either to fall into a local maximum of multiplicity $3$ if $\sigma$ was too high, or to be too unstable to converge if $\sigma$ was too small.
One way to overcome this issue is to compute a better direction than the one given by the gradient of $\mu_k$. Several numerical methods have been investigated in this direction, such as finding the direction $h$ as a solution to the problem
\[
    \max_{h \in V_h, \| h \| = 1} \min \left\{ \dd \mu_k^\eps(\rho). h, ..., \dd \mu_{k+m-1}^\eps(\rho).h \right\}
\]
where $m$ is the "guessed" multiplicity, chosen such that $\mu_{k+m-1}^\eps(\rho) - \mu_k^\eps(\rho) \leq \sigma$ and $\mu_{k+m}^\eps(\rho) - \mu_k^\eps(\rho) > \sigma$. Such method has been used for instance in\cite{antunes_numerical_2017}. Heuristically, it produces a direction that increases all the selected eigenvalues by a maximal amount. Another, more rigorous method would be to consider the true directional derivative of our multiple eigenvalue in the direction $h$ (which always exists), namely
\[
    (\mu^\eps_k)'(\rho)(h) := \lim_{t \to 0^+} \frac{\mu^\eps_k(\rho+th) - \mu_k^\eps(\rho)}{t}
\]
and in the same way as before, to search the direction that maximizes this variation :
\[
    \max_{h \in V_h, \| h \| = 1} (\mu^\eps_k)'(\rho)(h).
\]
This has been brilliantly studied in \cite{de_gournay_velocity_2006} where the author shows that the search of the direction can be efficiently solved by semi-definite programming methods.

However, in our case, these methods did not seem to perform better that our previous one. It turned out that a simple modification of our problem leads to a very good convergence. Instead of searching for a direction based on directional derivatives, we regularize our problem to make it differentiable. It can then be handled better from the interior point optimizer. The idea is the following : suppose that $\mu_k^\eps(\rho)$ stays away from $\mu_{k-1}^\eps(\rho)$ during the whole optimization process (which is always the case in practice) and is part of a cluster $\mu_{k}^\eps(\rho), ..., \mu_{k+m-1}^\eps(\rho)$ of eigenvalues closer than $\sigma$. Then obviously

\begin{equation}
    \mu_{k}^\eps(\rho) = \min \left\{\mu_{k}^\eps(\rho), ..., \mu_{k+m-1}^\eps(\rho)\right\}.
\end{equation}

But for $x_0,...,x_{m-1} > 0$, we have that
\begin{equation}
    \min \left\{ x_0,..., x_{m-1}\right\} = \lim_{p\to +\infty} \left( \sum_{i} x_i^{-p}\right)^{-1/p}
\end{equation}

so that by taking $p$ large, we can approximately write
\begin{equation}
    \min \left\{ x_0,..., x_{m-1}\right\} \approx \left( \sum_{i} x_i^{-p}\right)^{-1/p}.
\end{equation}

In this spirit, we instead optimize the functional
\begin{equation}
    \rho \mapsto \left( \sum_{i} \mu_{k+i}^\eps(\rho)^{-p}\right)^{-1/p}.
\end{equation}

This function being a symmetric function of the eigenvalues, it is expected to be smooth near $\rho$ if $m$ is the actual multiplicity of $\mu_k^\eps(\rho)$ \cite{kato2013perturbation}. If it is not, the largest eigenvalues of the cluster are mostly ignored.

\subsection{Dimension 1}

After running a series of simulations on the sphere, it appeared that the optimal density seems to be axially symmetric in the case of $\mu_1$. In order to get even more insight on the density problem, we subsequently ran simulations only in 1D, considering the density $\rho$ as a real function of the latitude $\theta \in [0,\pi]$, i.e. the angle from one pole to a point on the sphere. By separation of variables, if $\rho : \S^2 \to [0,1]$ is axially symmetric and not degenerated, then $\mu_1(\rho)$ is the least non-zero eigenvalue of the following two eigenvalue problems

\begin{equation*}
    \begin{cases}
        -\frac{1}{\sin(\theta)} \frac{\dd}{\dd \theta} \left( \rho(\theta) \sin(\theta) \frac{\dd y}{\dd \theta} \right) + \frac{\rho(\theta)}{\sin^2(\theta)} y = \rho(\theta) \bar \mu y \mbox{ on } (0, \pi) \\
        y(0) \mbox{ and } y(\pi) \mbox{ are finite}
    \end{cases}
\end{equation*}

\begin{equation*}
    \begin{cases}
        -\frac{1}{\sin(\theta)} \frac{\dd}{\dd \theta} \left( \rho(\theta) \sin(\theta) \frac{\dd y}{\dd \theta} \right) = \rho(\theta) \tilde \mu y \mbox{ on } (0, \pi) \\
        y(0) \mbox{ and } y(\pi) \mbox{ are finite}
    \end{cases}.
\end{equation*}
See \cite{ashbaugh_sharp_1995} for more details. Formally, by developping the derivative in the first differential equation, we can see that the condition $y(0)=y\pi)=0$ is forced by the term in $\frac{1}{\sin(\theta)}$, which penalizes large values of $y$ in $0$ and $\pi$. In the same way, we can see that in the second differential equation, the boundary conditions needs to be $y'(0)=y'(\pi)=0$.

As before, since we allow $\rho$ to vanish, we need to regularize the problem to make it elliptic. The problems that are actually solved are

\begin{equation*}
    \begin{cases}
        -\frac{\dd}{\dd \theta} \left( \left(\rho(\theta) \sin(\theta) + \eps \right)\frac{\dd y}{\dd \theta} \right) + \frac{\rho(\theta)+\eps}{\sin(\theta)} y = (\rho(\theta)\sin(\theta)+\eps^2) \bar \mu y \mbox{ on } (0, \pi) \\
        y(0)=y(\pi) =0
    \end{cases}
\end{equation*}

\begin{equation*}
    \begin{cases}
        -\frac{\dd}{\dd \theta} \left( \left(\rho(\theta) \sin(\theta) + \eps \right) \frac{\dd y}{\dd \theta} \right) = (\rho(\theta)\sin(\theta)+\eps^2) \tilde \mu y \mbox{ on } (0, \pi) \\
        y'(0)=y'(\pi)=0
    \end{cases}
\end{equation*}
where $\eps$ is supposed to be small.


\subsection{Numerical considerations.}

Our optimization procedure is carried out by IPOPT \cite{wachter2006implementation} while the finite element computations is perfomed in GetFEM \cite{MR4199501}. A first optimization is carried on a coarse mesh of $2246$ vertices with $\P_1$ finite elements. A second optimization is then performed with the result of the previous one as initilization on a mesh consisting in $35401$ elements (the meshes that are used can be vizualized Figure ~\ref{fig:meshes_density}). For each $m$, the optimization is performed multiple times with different initialization and the density giving
the best value is finally kept.

\begin{figure}
    \centering
    \includegraphics[width=0.33\textwidth]{density_mesh_coarse.png}
    \includegraphics[width=0.33\textwidth]{density_mesh_fine.png}
    \caption{The meshes used for the density method.}
    \label{fig:meshes_density}
\end{figure}

For both optimizations we take $p=20$ and $\eps = 10^{-4}$.



\section{Results : density method.}

In this section we discuss the results obtained by the density method described above. We focus only on the first three eigenvalues, since they already shows a rich behaviour. In each graph, the value of the optimized eigenvalue is plotted in green as a function of the total mass $m$ and the corresponding density is denoted by $\rho^m$. In order to give a comparison, for each $\mu_k$ we plot in red the corresponding eigenvalue for a union of $k$ disjoint geodesic balls of surface area $m/k$ (denoted $\UB^m_k$). This values have been approximated by a finite element (FE) decomposition of the following 1D eigenvalue problem :
\begin{equation}
    \begin{cases}
        -\frac{1}{\sin(\theta)} \frac{\dd}{\dd \theta} \left( \sin(\theta) \frac{\dd y}{\dd \theta} \right) + \frac{1}{\sin^2(\theta)} y = \mu y \mbox{ on } (0, \theta_m) \\
        \frac{\dd y}{\dd \theta}(\theta_m) = 0 \\
        y(0) \mbox{ is finite}
    \end{cases}
\end{equation}
where $\theta_m = \arccos{(1-\frac{m}{2\pi})}$ is the geodesic radius of the ball of surface area $m$ on $\S^2$. In practice, the solution is approximated using $\P_1$ FE with $10000$ degrees of freedom.

\subsection{Validation : $\mu_1$ with constraint.}

In order to validate our optimization process, we rely on the following result of \cite{bucur_sharp_2022}, which states that if we run the optimization outside of a ball of the right area then the optimum is a ball :

\begin{theorem}Let $m \in (0, |\S^n|/2)$ and let $\B^m$ be a geodesic ball of measure $m$ in $\S^n$. Let
$\Omega \subset \S^n \setminus \B^m$ be an open Lipschitz set such that $|\Omega| = m$. Then
$\mu_1(\Omega) \leq \mu_1(\B^m)$.
\end{theorem}

This theorem, proved by some mass transplantation technique "à la Weinberger", is hence also valid for densities and could be reformulated as follows :

\begin{theorem}Let $m \in (0, |\S^n|/2)$ and let $\B^m$ be a geodesic ball of measure $m$ in $\S^n$. Let
$\rho : \S^n \to [0,1]$ such that $\rho = 0$ on $\B^m$ and $ \int_{\S^n} \rho = m$. Then
$\mu_1(\rho) \leq \mu_1(\B^m)$.
\end{theorem}

In practice, we run the optimization process in the whole sphere but add the constraint that all degrees of freedom of $\rho$ that lies inside a certain ball $\B^m$ stay equal to $0$. This constraint is easily handled by IPOPT. The plots in Figures \ref{fig:mu_1_constraint_density_examples} and \ref{fig:mu_1_density_constraint} show that the optimal density is indeed the characteristic function of a ball.

\begin{figure}
    \centering
    \includegraphics[width=0.24\textwidth]{density_mu_1_constraint_2.17.png}
    \includegraphics[width=0.24\textwidth]{density_mu_1_constraint_5.0.png}
    \caption{Example of optimal densities for $\mu_1$ for $m \in \{2.17, 5.0\}$ when $\rho$ is supported outside of a ball.}
    \label{fig:mu_1_constraint_density_examples}
\end{figure}


\begin{figure}
    \centering
    \includegraphics[width=0.49\textwidth]{density_chart_mu_1_constraint_left.png}
    \caption{Optimal value of $\mu_1$ obtained by the density method, with the constraint that the support of $\rho$ is located outside of a ball.}
    \label{fig:mu_1_density_constraint}
\end{figure}



\subsection{An interesting case : $\mu_1$ in the whole sphere.}

We now consider the case where $\rho$ is allowed to fill the whole sphere. One first observation is that the optimal eigenvalue is expected to be never less that $n$. Indeed, this eigenvalue is associated to constant densities hence we can chose the density $\rho = \frac{m}{|\S^n|}$ which leads to $\mu_1(\rho) = \mu_1(\S^n) = n$. The simulations actually suggest that this value is only attained near $m=|\S^n|$ as shown in Figure \ref{fig:mu_1_density}.

\begin{figure}
    \centering
    \includegraphics[width=0.49\textwidth]{density_chart_mu_1_left.png}
    \includegraphics[width=0.49\textwidth]{density_chart_mu_1_right.png}
    \caption{Optimal value of $\mu_1$ obtained by the density method.}
    \label{fig:mu_1_density}
\end{figure}

Since the eigenvalue goes to infinity as $m$ goes to $0$, the graph is displayed on two different scales for a better readability. The reader should pay a particular attention to the range of the different axes. Something interesting happens : the spherical cap seems not to be optimal for values of $m$ greater than $m \approx 4.5$. This is allowed by the fact that $\rho$ can fill the whole sphere, which wasn't allowed with the ball constraint. A zoom on the range $m \in [3.5,6.5]$ is displayed Figure \ref{fig:mu_1_density_split}.

\begin{figure}
    \centering
    \includegraphics[width=0.49\textwidth]{density_biffurcation.png}
    \caption{Optimal value of $\mu_1$ near $m\approx 5.0$.}
    \label{fig:mu_1_density_split}
\end{figure}

We illustrate the behaviour of the optimal $\rho$ in Figure \ref{fig:mu_1_density_examples} for different values of $m$. Deep blue color corresponds to $\rho = 0$ while red color corresponds to $\rho = 1$.

\begin{remark}
Apart from $m=2.0$ which seems to be the characteristic function of a geodesic ball, the optimal density seems to be some "homogenized" spherical cap. This surprising result has an important theoretical implication : even if the ball $B^m$ were optimal for $\mu_1$ among shapes $\Omega$ such that $|\Omega| = m \leq |\S^n|/2$,  it would be impossible to prove it using the standard mass transplantation technique of Weinberger \cite{weinberger_isoperimetric_1956}. Indeed, the proof would also hold for densities, but the numerical results strongly indicates that it is false for $m>0.5$. To go further, it would be interesting to investigate if this kind of "homogenization" of the sphere could be attained by some sequence of actual domains. This could for instance suggest the non-existence of optimal domains for a large enough $m$.
\end{remark}

The inspection of these results leads to the following conjecture :

\begin{conjecture}Let $m\in(0,|\S^n|)$. Then the optimal density $\rho^m$ of the problem
\begin{equation*}
  \max \left\{ \mu_1(\rho) :  \rho : \S^n \rightarrow [0,1], \int_{\S^n}\rho =m\right\}.
\end{equation*}
is axially symmetric.
\end{conjecture}

In the light of this conjecture, we illustrate on the same Figure \ref{fig:mu_1_density_examples} the density $\rho$ as the result of the 1D optimization procedure.

\begin{figure}
    \centering
    \includegraphics[width=0.24\textwidth]{density_mu_1_2.0.png}
    \includegraphics[width=0.24\textwidth]{density_mu_1_4.98.png}
    \includegraphics[width=0.24\textwidth]{density_mu_1_8.05.png}
    \includegraphics[width=0.24\textwidth]{density_mu_1_11.2.png}

    \includegraphics[width=0.24\textwidth]{density_1d_2.0.png}
    \includegraphics[width=0.24\textwidth]{density_1d_4.98.png}
    \includegraphics[width=0.24\textwidth]{density_1d_8.05.png}
    \includegraphics[width=0.24\textwidth]{density_1d_11.2.png}
    \caption{Example of optimal densities for $\mu_1$ for $m \in \{2.0, 4.98, 8.05, 11.2\}$ (top) along with their latitudinal profile (bottom).}
    \label{fig:mu_1_density_examples}
\end{figure}

Note that it would be interesting to get more information on the behaviour of the density near $m=4.5$ where it seems to start to homogenize. A good indication that the optimal density is an actual domain would be that the size of the set $\rho^m \notin \{0,1\}$ is always proportionnal to the size of an element under mesh refinement. On the contrary, if $\rho^m$ is a density, then the size of this set should be independant of the size of one element. Let $N$ be the number of elements of the segment $[0,\pi]$. For $N \in \{100,200,400,800\}$, we compute the quantity
\[
  h_m(N) = \frac{N}{\pi} \int_0^\pi  \rho^m(1-\rho^m).
\]
Since $\pi/N$ is the size of one element, this quantity should be constant in $N$ if $\rho^m$ is a domain for the reasons stated previously. If $\rho^m$ is not a domain, then we would expect $h_m(N)$ to double when doubling the number of points in the mesh.
To compare this quantity for different $N$ we normalize this quantity at $N=100$ and define
\[
  \mbox{Dispertion}_m(N) = \frac{h_m(N)}{h_m(100)}
\]
We plot the graph of $\mbox{Dispertion}_m$ for different values of $m$ in Figure \ref{fig:1d_refinement}

\begin{figure}
    \centering
    \includegraphics[width=0.8\textwidth]{density_1D_refinement.png}
    \caption{Mesh-refinement procedure une 1D for different values of $m$.}
    \label{fig:1d_refinement}
\end{figure}

A few things can be deduced from this graph. For $m$ large enough (approximately $m>4.6$), the behaviour is the one we would observe for a density which is not a domain, as the dispertion grows with the number of points. On the other hand, it appears that for $m$ small enough $\rho^m$ seems to be a domain since its dispertion is constant. It is the subject of the following conjecture.

\begin{conjecture}There exists $\delta > 0$ such that for all $m \in (0,\delta)$, $\rho^m = \1_{\B^m}$ i.e. the optimal density is the one of a geodesic ball.
\end{conjecture}

Following the numerical observations above, the value of $\delta$ would lie between $3.5$ and $4.6$.

\subsection{Results for $\mu_2$}

\begin{figure}
    \centering
    \includegraphics[width=0.49\textwidth]{density_chart_mu_2_left.png}
    \includegraphics[width=0.49\textwidth]{density_chart_mu_2_right.png}
    \caption{Optimal value of $\mu_2$ obtained by the density method.}
    \label{fig:mu_2_density}
\end{figure}


The results for $\mu_2$ are the most stable ones. Indeed, no matter the value of $m$, the corresponding optimal density is always attained by the characteristic function of the union of two balls of the same measure as can be seen in Figure ~\ref{fig:mu_2_density}. This numerical experiment gives strong insights of the validity of Theorem 2 of \cite{bucur_sharp_2022} which asserts that $\mu_2$ is maximal for the disjoint union of two balls $B^\frac{m}{2} \sqcup B^\frac{m}{2}$. Figure \ref{fig:mu_2_density_examples} shows the optimal densities that are obtained for some values of $m$.

\begin{figure}
    \centering
    \includegraphics[width=0.24\textwidth]{density_mu_2_2.31.png}
    \includegraphics[width=0.24\textwidth]{density_mu_2_5.46.png}
    \includegraphics[width=0.24\textwidth]{density_mu_2_8.23.png}
    \includegraphics[width=0.24\textwidth]{density_mu_2_11.01.png}
    \caption{Example of optimal densities for $\mu_2$ for $m \in \{2.31, 5.46, 8.23, 11.01\}$.}
    \label{fig:mu_2_density_examples}
\end{figure}


\subsection{Results for $\mu_3$}

\begin{figure}
    \centering
    \includegraphics[width=0.49\textwidth]{density_chart_mu_3_left.png}
    \includegraphics[width=0.49\textwidth]{density_chart_mu_3_right.png}
    \caption{Optimal value of $\mu_3$ obtained by the density method.}
    \label{fig:mu_3_density}
\end{figure}


As for the case of $\mu_1$, the optimization procedure for $\mu_3$ exhibits various different behaviours depending on the value of $m$. Note that the union of three balls of the same surface area seems to never be optimal, as shown in Figure \ref{fig:mu_3_density}.

\begin{figure}
    \centering
    \includegraphics[width=0.24\textwidth]{density_mu_3_2.0.png}
    \includegraphics[width=0.24\textwidth]{density_mu_3_5.0.png}
    \includegraphics[width=0.24\textwidth]{density_mu_3_8.03.png}
    \includegraphics[width=0.24\textwidth]{density_mu_3_11.0.png}
    \caption{Example of optimal densities for $\mu_3$ for $m \in \{2.0, 5.0, 8.03, 11.0\}$.}
    \label{fig:mu_3_density_examples}
\end{figure}

In Figure \ref{fig:mu_3_density_examples} are displayed the different types of densities that can be obtained with the density method.  As it could be expected, for small $m$ we get the same type of result than in the plane \cite{BMO_2022}. Only the last, for large $m$, seems to be an actual characteristic function of some "napkin ring"-shaped domain. For $m \approx 8.0$, we get some homogenized geodesic annulus. On the two-dimensional sphere it is expected to see that for large $m$ the eigenvalue goes to $2$ since $\mu_1(\S^2)$ is of multiplicity $3$ and $\mu_1(\S^2)=2$.


\subsection{Data.}

All the final solutions are available in MEDIT format at \url{https://github.com/EloiMartinet/Neumann_Sphere/}. A FreeFem++ \cite{MR3043640} script allowing to read the solutions and compute the eigenvalues is also provided for replicability purposes. See the README file for more information.

Lastly, we divided the exams based on the level of breast density, with a fatty group consisting of density A and B, and a dense group consisting of density C and D. Both the baseline and \ours performs better in fatty group than dense group. We suspect this is because deep neural networks generally work better on low density images given that visual cues of cancer in images with lower breast density are more clearly visible.