
The performance and efficiency of ACO algorithm's processing and computation depend directly on the parameters of the problem. Typically, this problem involves 6 primary variables:
\begin{itemize}
    \item \textbf{Generation:} This parameter in the ACO algorithm determines the number of ant generations. In this study, the number of generations is regarded as the stopping condition for the algorithm, which will halt after N generations. Increasing the number of generations can enhance the accuracy of the algorithm by enabling more ants to address the problem and emit pheromones. However, the trade-off is that the algorithm may run slower as the number of generations increases.

    \item \textbf{Ant Count:} The number of ants in each generation is controlled by this parameter. If the number of ants is too small, the algorithm may quickly converge to a suboptimal solution. Conversely, if the number of ants is too large, the algorithm may become slower.

    \item \textbf{$\alpha$:} the $\alpha$ parameter is a weight that balances the importance between the amount of pheromone left on the paths and the heuristic information, which provides a prior knowledge of the problem. A higher alpha value places more emphasis on the heuristic information, while a lower $\alpha$ value places more emphasis on the pheromone trails. In other words, $\alpha$ controls the degree to which the ants rely on the previous knowledge of the problem (heuristic information) or the information gained from their exploration (pheromone trail). The optimal value of $\alpha$ depends on the problem being solved and can be determined through experimentation. A higher $\alpha$ value is generally preferred for problems where prior knowledge plays an important role, while a lower $\alpha$ value may be better suited for problems where exploration is necessary.

    \item \textbf{$\beta$:} controls the weight given to the heuristic information about the problem being solved. This parameter determines the effect of the problem's domain knowledge on the movement of the ants. $\beta$ controls the relative influence of the pheromone trail and the heuristic information in the decision-making process of the ants. A higher value of $\beta$ gives more weight to the heuristic information, while a lower value of $\beta$ gives more weight to the pheromone trail. The optimal value of $\beta$ depends on the specific problem being solved and may require experimentation to find.

    \item \textbf{$\rho$:} the pheromone evaporation rate is controlled by this parameter. It determines how quickly the pheromone trail will evaporate over time. A high value of $\rho$ means that the pheromone evaporates quickly, while a low value of $\rho$ means that the pheromone trail will persist longer. By setting an appropriate value for $\rho$, the algorithm can balance the importance of the current and past solutions. If $\rho$ is set too high, the ants will tend to converge on a suboptimal solution too quickly, while setting $\rho$ too low may cause the ants to explore too much and take a longer time to find a good solution. Therefore, the value of $\rho$ must be chosen carefully based on the characteristics of the problem being solved.

    \item \textbf{Pheromone Quantity:} The pheromone quantity parameter in ACO algorithm has a dual effect on the algorithm. First, it helps ants to find potential solutions by creating a stronger and longer-lasting pheromone trail, but a very high quantity can trap the algorithm in local optima. Second, a high pheromone quantity can cause fast convergence of the algorithm, leading to suboptimal solutions. To prevent this, it is essential to balance the pheromone quantity with other parameters, including the exploration-exploitation trade-off and the pheromone evaporation rate.

    The experimental results indicate that if the pheromone does not evaporate, it can adversely affect the performance of the algorithm for POS-tagging. Furthermore, the number of ants and generations should be fine-tuned to achieve both acceptable accuracy and desirable execution time. Table (4) presents the optimal combination of these parameters obtained through numerous experiments. To determine this combination, all feasible parameter combinations were tested within the designated intervals.
\end{itemize}


\begin{table}
	\caption{Input parameters}
	\centering
	\begin{tabular}{cccccc}
		\toprule
		Generation   &Ants    &$\alpha$   &$\beta$   &$\rho$  &quantity \\
            \midrule
		3 &20 &0.9    &0.9    &0.95   &10     \\
		\bottomrule
	\end{tabular}
	\label{tab:table}
\end{table}

