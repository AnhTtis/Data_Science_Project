The experiments for this method utilized the Bijankhan corpus, which is a vast collection of text in the Persian language consisting of 2.6 million individual words categorized into 32 different tags, covering 4,300 diverse topics \cite{19}. During the experiments, the training data for the method was composed of 80\% of the sentences in the corpus, which were used to create probability matrices. The remaining 20\% of the data was used as the test data.\\
As a means of comparing the ACO-tagger to the Viterbi algorithm in a more fair and accurate manner, both algorithms have been tested concurrently. Both algorithms were used to tag each sentence in the test data, and the final accuracy of both algorithms was calculated on the same dataset. The results of the comparison showed that the ACO-tagger achieved an accuracy of 96.867\%, which is higher than the Viterbi algorithm's accuracy of 96.361\%. Additionally, the ACO-tagger demonstrated superior performance in tagging lengthy sentences compared to Viterbi, which accuracy decreases as the sentence length increases.\\
The findings from the conducted tests suggest that the ACO-tagger is a promising method for sequence labeling tasks such as POS-tagging. Compared to previous methods, ACO-tagger exhibited superior performance in terms of accuracy, especially for lengthy sentences. This indicates that the algorithm is robust and effective in solving complex optimization problems. Moreover, ACO-tagger's ability to generalize across different datasets makes it a versatile and adaptable tool for various applications. Its potential is not limited to language processing but can also be extended to other domains, such as speech recognition, image segmentation, and bioinformatics. Overall, the results indicate that ACO-tagger is a powerful and promising method for optimizing sequence labeling tasks, and it opens up new possibilities for future research in this field.\\
In addition to ACO, other swarm intelligence algorithms can also be applied for optimization problems in various fields. Among these algorithms, the Firefly Algorithm (FA) \cite{20} and Artificial Bee Colony Algorithm (ABC) \cite{21} are promising approaches. The FA algorithm is inspired by the flashing patterns of fireflies and has shown success in solving optimization problems such as feature selection, image segmentation, and clustering. Similarly, the ABC algorithm is based on the foraging behavior of honey bees and has demonstrated strong performance in optimizing various functions, including feature selection, data classification, and image processing. Overall, swarm intelligence algorithms such as ACO, FA, and ABC present powerful and versatile tools for solving optimization problems in diverse applications.