Different methods have been used for tagging words, and the variety of methods is evident in this case.
The Brill Tagger method, a rule-based approach, was one of the first methods introduced for this work. In a study by Karbasian et al., the Brill Tagger method achieved a tagging accuracy of 94.27\% for the Persian language using the Bijankhan corpus \cite{11}. However, the challenge with rule-based approaches is that the rules for determining tags need to be defined and created, typically requiring an expert's assistance. As a result, different sets of rules may be developed based on the expert's perspective. \\
A different approach to tagging utilizes neural networks, specifically recurrent neural networks like LSTM. This method applies the effect of a word's category to the subsequent words in the recurrent layer, resulting in more accurate tagging. A study by Kochari and al. explored the Bijankhan corpus using LSTM, incorporating the previous word's tag (bigram), and achieved 88.94\% accuracy in correctly tagging words. In this method, by increasing the number of previous words, a higher percentage of accuracy can be achieved. As far as in the test with increasing words to 5 (5-gram), this algorithm has reached 95.6\% accuracy \cite{12}.\\
In a study on Bijankhan's corpus, Hosseini et al. utilized the artificial neural network (ANN) approach for POS tagging and achieved a high level of accuracy. Their research found that the ANN-based POS tagging system had an accuracy rate of 95.7\% \cite{13}.\\
One of the most famous and widely used methods used for POS tagging is HMM (Hidden Markov Model). This model chooses the most probable tag sequence as output by calculating the tag probability of each word. For this purpose, the words are considered as a chain of different states, and at each stage, the probability of each tag is calculated for the corresponding word. These calculations are based on the Markov chain method. Okhovvat et al. reported an accuracy of 98.1\% for the Hidden Markov Model in their study on the RCIS corpus \cite{14}. Azimizadeh et al. employed the HMM approach for POS tagging on the Bijan Khan corpus, which resulted in a 95.11\% accuracy rate in their research \cite{15}.\\
Besharati et al. carried out a research that integrated LSTM and HMM techniques to boost POS tagging. Their study accomplished remarkable outcomes, as they were able to attain a high level of accuracy of 97.29\% in tagging words \cite{16}.
In a different study, Bokaei et al. emphasized improving the Viterbi model in POS tagging. Through the introduction of a new method, they achieved an enhanced accuracy rate for the model, which demonstrates the potential for further development in this field of research \cite{17}.