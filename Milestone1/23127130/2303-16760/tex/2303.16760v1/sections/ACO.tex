Ant Colony Optimization (ACO) is a powerful metaheuristic method used in solving hard combinatorial optimization problems with discrete structures \cite{4}. ACO is inspired by the foraging behavior of ants, which involves a collective decision-making process. The algorithm was first presented by Dorigo et al. in 1999 and has since been widely used in various fields \cite{5}. ACO is known for its ability to find optimal solutions in complex problems by utilizing a stochastic process that involves simulating the foraging behavior of ants. In ACO, a colony of virtual ants is used to explore the search space, with each ant leaving pheromone trails that guide the search towards promising areas. Through this process, ACO is able to efficiently find near-optimal solutions to complex problems that are difficult to solve using traditional methods \cite{6}.\\
In nature, ants use a chemical substance called pheromone, specifically formic acid, to leave trails as they navigate. In the following, ants of the next generations can sense the amount of pheromone and use that information to follow the previously travelled paths. This process, known as marking, enables ants to achieve self-organization and find the optimal search path while foraging for food \cite{7}. By using this marking behavior, ants can explore a large area of their surroundings and converge toward the richest food sources. The algorithm uses pheromone trails to guide the search process and iteratively improves the solution by adjusting the pheromone levels on each candidate solution. The use of pheromone trails allows the ACO algorithm to efficiently explore the search space and converge to an optimal solution.\\
The algorithm operates by simulating the foraging behavior of ants, with each ant representing a potential solution to the optimization problem. In the beginning, each ant randomly navigates the search space, generating a path that represents its proposed solution. ACO incorporates pheromone trails into its search mechanism by assigning values to the edges of a graph to represent the strength of the pheromone trails. As ants follow the pheromone trails, the paths with stronger pheromone levels become more attractive and more likely to be selected by subsequent ants.\\
Similarly, the ACO algorithm assigns a probability to each edge based on the pheromone trail strength, and ants are more likely to choose edges with higher pheromone levels. The selected paths represent random variables, which are used to define a fitness function for each ant. The fitness function represents the objective function of the optimization problem, with a better fitness value indicating a shorter path and a better solution.\\
In the ACO algorithm, after the initial random paths are generated by the ants, the quality of the solutions provided by each ant is evaluated. Based on this evaluation, the amount of pheromone is updated for the paths taken by each ant. In the next iteration, the ants choose their paths randomly based on the available pheromone. The probability of choosing a particular path is calculated based on the amount of pheromone deposited on that path. The higher the amount of pheromone, the higher the probability of choosing that path. By iteratively updating the pheromone levels and re-generating the paths, the algorithm searches the solution space to find the optimal solution. This approach allows the algorithm to balance between exploration and exploitation of the solution space, which can result in finding better solutions. The amount of pheromone left by the ants on the path is updated according to the path that ant has passed. This means that the better solutions have a higher amount of pheromone, and as a result, in the next generations, more ants will choose the path with the higher amount of pheromone. This behavior leads to a positive feedback loop in which good solutions attract more ants, and as a result, the pheromone on that path increases even more \cite{8}.\\
In an experiment, Deneubourg was able to prove that ants will finally converge toward the shorter path. This experiment demonstrated that ants were capable of solving complex problems and finding the shortest path between their nest and a food source. The experiment also showed that ants are capable of self-organization and coordination, which is a fundamental principle of swarm intelligence and the basis for the development of the ACO algorithm \cite{9}. The effectiveness of this approach has been demonstrated in various applications, including job scheduling, network routing, and machine learning. In these applications, the ACO algorithm has been shown to be competitive with other state-of-the-art methods in terms of solution quality and computational efficiency.