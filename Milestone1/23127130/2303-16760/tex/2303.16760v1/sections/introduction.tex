In natural language processing, Part-of-Speech (POS) tagging is a task that involves identifying and labeling the parts of speech of words in a sentence, such as nouns, verbs, adjectives, adverbs, etc \cite{1}. This task is critical for disambiguating the meaning of words by understanding the structure and grammar of sentences. For example, the word "\FR{کاوش} " can refer to a name or an infinitive that means "exploration", and POS tagging can help differentiate which meaning is intended in a given context.
The purpose of POS tagging is to provide a standardized representation of a sentence, which can then be used as input for various NLP tasks, such as named entity recognition, syntactic parsing, and sentiment analysis \cite{2}. The goal of this task is to accurately assign a tag to each word in a sentence, which can be challenging due to the ambiguity and complexity of natural language \cite{3}.\\
POS tagging can also be used to extract more detailed information about a text, such as identifying named entities (e.g., people, places, and organizations) and detecting sentiment. It can also help improve the accuracy of language models and other NLP algorithms. Overall, the importance of POS tagging in NLP lies in its ability to provide a foundation for further language analysis and processing, enabling more sophisticated and accurate models to be built for a range of applications \cite{1}.\\
This article presented a new method for solving the POS-tagging task in NLP using the Ant Colony Optimization (ACO) algorithm, a novel approach that utilizes swarm intelligence and evolutionary computing algorithms. The ACO algorithm is employed as an optimizer to improve the tags on words, with the goal of finding the best tag sequence and providing the most optimal possible tag sequence as the final answer. The proposed method has shown promising results in improving the accuracy and efficiency of POS-tagging tasks. The research and experiments in this paper are based on the Persian language, using the Bijankhan linguistic corpus which is one of the most comprehensive linguistic corpora available in Persian. However, it should be noted that the algorithm is not dependent on the Persian language and can be adapted to work with other languages as well.\\
In order to assess the performance of ACO-tagger, we have compared it with the Viterbi algorithm. The Viterbi algorithm is a dynamic algorithm that uses Hidden Markov Models to solve problems, particularly in POS-tagging, by employing probabilistic models. This algorithm works by computing the probability of each possible sequence of tags and selecting the sequence with the highest probability as the output.
% To evaluate the performance of ACO-tagger, the algorithm was compared with the widely-used Viterbi algorithm, which is a dynamic programming algorithm based on the Hidden Markov Model (HMM) that is often used for POS-tagging tasks.
On the other hand, the ACO algorithm is a stochastic optimization method inspired by the behaviour of ants in nature that can solve non-linear problems. ACO works by simulating the behaviour of ants as they search for the shortest path between their nest and a food source. In the context of POS-tagging, the ACO algorithm is used to iteratively search for the best tag sequence for each word in a given sentence. By comparing the results of the ACO-tagger with those of the Viterbi algorithm, it is possible to determine whether the ACO approach is a viable alternative for POS-tagging tasks and to assess its strengths and limitations in comparison to other methods.