
\documentclass{article} % For LaTeX2e
\usepackage{iclr2023_conference,times}

\usepackage{diagbox}
\usepackage{amssymb}
\usepackage{floatrow}
\usepackage{wrapfig}

\usepackage{dsfont}
\usepackage{xspace}
\usepackage{listings}
\usepackage{numprint}
\usepackage{multirow}
\usepackage{setspace}
\usepackage{enumitem}
\usepackage{lipsum}
\usepackage{caption}
\captionsetup{font=small}
\usepackage{booktabs}
\usepackage{algorithm}
\usepackage{algpseudocode}

\usepackage{microtype}
\usepackage{graphicx}
\graphicspath{{figures/}}
\usepackage{sidecap}
\usepackage{subfigure}

\usepackage{balance}
\usepackage{pifont}
\usepackage{caption}
\usepackage{comment}
\usepackage{soul}
\usepackage{bbm}

% Optional math commands from https://github.com/goodfeli/dlbook_notation.
\newcommand{\bbox}{\text{bbox}}
\newcommand{\alphapck}{\alpha_\bbox}
\newcommand{\kcycle}{\text{k-CyPCK}}
\newcommand{\cycle}{\text{-CyPCK}}

\newcommand{\I}{\mathbf{I}}
\newcommand{\Ia}{\I^\text{a}}
\newcommand{\Ib}{\I^\text{b}}
\newcommand{\Iatob}{\I^\text{a $\rightarrow$ b}}
\newcommand{\F}{\mathbf{F}}
\newcommand{\Fa}{\F^\text{a}}
\newcommand{\Fb}{\F^\text{b}}
\newcommand{\f}{\mathbf{f}}
\newcommand{\fa}{\f^\text{a}}
\newcommand{\fb}{\f^\text{b}}
\newcommand{\p}{\mathbf{p}}
\newcommand{\pa}{\p^\text{a}}
\newcommand{\pb}{\p^\text{b}}
\newcommand{\A}{\boldsymbol{\Phi}_\text{align}}
\newcommand{\G}{\mathbf{G}}
\newcommand{\C}{\mathbf{C}}
\newcommand{\Ca}{\C^\text{a}}
\newcommand{\Cb}{\C^\text{b}}
\newcommand{\cc}{\mathbf{c}}
\newcommand{\cca}{\cc^\text{a}}
\newcommand{\ccb}{\cc^\text{b}}
\newcommand{\Irec}{\I_\text{Recon}}
\newcommand{\M}{\mathbf{M}}
\newcommand{\Mrec}{\M_\text{Recon}}
\newcommand{\loss}{\mathcal{L}}
\newcommand{\T}{\mathcal{T}}
\newcommand{\W}{\mathcal{W}}
\newcommand{\Id}{\mathcal{I}}


\usepackage{hyperref}
\definecolor{BlueBlack}{RGB}{36, 113, 163}
\hypersetup{
    colorlinks,
    linkcolor={red},
    citecolor={BlueBlack},
    urlcolor={blue}
}
\usepackage{cleveref}[capitalise]
\crefname{section}{Sec.}{Secs.}
\crefname{figure}{Fig.}{Figs.}
\crefname{table}{Tab.}{Tabs.}
\crefname{equation}{Eq.}{Eqs.}
\Crefname{section}{Section}{Sections}
\usepackage{url}

\title{\textbf{T}emperature \textbf{S}chedules for self-super\-vised contrastive methods on long-tail data}




\author{
Anna Kukleva\thanks{equal contribution. Code available at:  \href{https://github.com/Annusha/temperature\_schedules}{github.com/annusha/temperature\_schedules}}    $^{\,1}$, Moritz Böhle$^{*1}$, Bernt Schiele$^{1}$, Hilde Kuehne$^{2,3}$, Christian Rupprecht$^{4}$ \\
% \thanks{denotes equal contribution. Code is available: }\
\small $^{1}$ MPI for Informatics, Saarland Informatics Campus, \small $^{2}$ Goethe University Frankfurt, \\ \small $^{3}$ MIT-IBM Watson AI Lab,$^{4}$ University of Oxford  \;$\Vert$\; \texttt{\{akukleva,mboehle\}@mpi-inf.mpg.de}\\
}


% Math operators with less space usage 
\newcommand\myeq{\mkern1.25mu{=}\mkern1.25mu}
\newcommand\myneq{\mkern1.25mu{\neq}\mkern1.25mu}
\newcommand\myrightarrow{\mkern1.25mu{\rightarrow}\mkern1.25mu}
\newcommand\myminus{\mkern1.25mu{-}\mkern1.25mu}
\newcommand\myplus{\mkern1.25mu{+}\mkern1.25mu}
\newcommand\mygreater{\mkern1.25mu{>}\mkern1.25mu}
\newcommand\myin{\mkern1.25mu{\in}\mkern1.25mu}
\newcommand\mytimes{\mkern1.25mu{\times}\mkern1.25mu}


\newcommand{\fix}{\marginpar{FIX}}
\newcommand{\new}{\marginpar{NEW}}

\newcommand{\bernt}[1]{\textcolor[rgb]{0.82, 0.1, 0.26}{\textbf{Bernt: #1}}}
\newcommand{\anna}[1]{\textcolor[rgb]{1, 0.3, 0.1}{\textbf{A: #1}}}
\newcommand{\headclass}[1]{\textcolor[rgb]{1, 0, 0}{#1}}
\newcommand{\tailclass}[1]{\textcolor[rgb]{0, 0, 1}{#1}}
\newcommand{\hkc}[1]{\textcolor[rgb]{0, 0, 1}{\textbf{Hilde: #1}}}
\newcommand{\moritz}[1]{\textcolor[rgb]{0, .5, .5}{\textbf{Moritz: #1}}}
\newcommand{\cmark}{\ding{51}}

% TODO: disable this for the final version!
%\usepackage{ulem}
\newcommand{\REM}[1]{\textcolor[rgb]{1, 0, 1}{\textbf{#1}}}
\newcommand{\ADD}[1]{\textcolor[rgb]{0.25, 0.75, 0}{\textbf{#1}}}
\newcommand{\rebuttal}[1]{\textcolor[rgb]{0, 0, 0}{#1}}
\newcommand{\green}[1]{\textcolor[rgb]{0, 1, 0}{\textbf{#1}}}
\newcommand{\red}[1]{\textcolor[rgb]{1, 0, 0}{\textbf{#1}}}
\newcommand{\greendark}[1]{\textcolor[HTML]{40A940}{\textbf{#1}}}
\newcommand{\reddark}[1]{\textcolor[HTML]{DA3C3D}{\textbf{#1}}}
\newcommand{\yellow}[1]{\textcolor[HTML]{F08722}{\textbf{#1}}}
\newcommand{\purple}[1]{\textcolor[HTML]{800080}{\textbf{#1}}}
\newcommand{\cyan}[1]{\textcolor[HTML]{19BFCF}{\textbf{#1}}}
\newcommand{\TODO}[1]{\textcolor[rgb]{1, 0.5, 0}{\textbf{TODO: #1}}}

\newfloatcommand{capbtabbox}{table}[][\FBwidth]

\makeatletter
\newcommand{\ie}{\textit{i}.\textit{e}.\@ifnextchar{,}{}{~}}
\makeatother
\makeatletter
\newcommand{\eg}{\textit{e}.\textit{g}.\@ifnextchar{,}{}{~}}
\makeatother

\newcommand{\myparagraph}[1]{\vspace{2pt}\noindent{\bf #1}}

\iclrfinalcopy % Uncomment for camera-ready version, but NOT for submission.
\begin{document}


\maketitle

\vspace{-3mm}
\begin{abstract}


Most approaches for self-supervised learning (SSL) are optimised on curated balanced datasets, \eg ImageNet, despite the fact that natural data usually exhibits long-tail distributions. In this paper, we analyse the behaviour of one of the most popular variants of SSL, \ie contrastive methods, on long-tail data. In particular, we investigate the role of the temperature parameter $\tau$ in the contrastive loss, by analysing the loss through the lens of average distance maximisation, and find that a large $\tau$ emphasises {group-wise} discrimination, 
whereas a small $\tau$ leads to a higher degree of {instance} discrimination.
While $\tau$ has thus far been treated exclusively as a \emph{constant} hyperparameter, in this work, we propose to employ a \emph{dynamic} $\tau$ and show that a simple cosine schedule can yield significant improvements in the learnt representations. 
Such a schedule results in a constant `task switching' between an emphasis on {instance} discrimination and {group-wise} discrimination and thereby ensures that the model learns both group-wise features, as well as instance-specific details. 
Since frequent classes benefit from the former, while infrequent classes require the latter, we find this method to consistently improve separation between the classes in long-tail data without any additional computational cost. 


\end{abstract}


%%%%%%%%% BODY TEXT
% \section{Introduction}
\label{sec:intro}
\begin{figure}[t]
\begin{center}
    \includegraphics[width=1\linewidth]{figures/teaser.pdf}
\end{center}
\vspace{-0.1in}
\caption{\textbf{{\em Foggy} vs {\em Clear} NeRF.} Our \ournerf gets rid of reconstruction errors manifested as foggy ``floaters" in the density volume without additional input or significant computational overhead. 
%
Below are density profiles along a given ray before and after our geometry correction procedure, where we discard density peaks corresponding to floaters.
}
\label{fig:teaser}
\vspace{-0.2in}
\end{figure}



%The emergence of 
Neural Radiance Fields (NeRFs)~\cite{mildenhall2020nerf}  %and its variants 
have made revolutionary contributions in %photo-realistic 
novel view synthesis~\cite{barron2021mip,barron2022mip}, 
autonomous driving~\cite{rematas2022urban,tancik2022block}, digital human~\cite{hong2022headnerf,zhao2022humannerf}, and 3D content generation~\cite{eg3d,poole2022dreamfusion,lin2022magic3d}.
%by leveraging a multi-layer perceptron (MLP) to implicitly model the mapping from input 5D coordinates (i.e., 3D coordinates $\mathbf{x} = (x,y,z)$ and 2D viewing directions $\mathbf{d}=(\theta,\phi)$) to volume density $\sigma$ and view-dependent emitted radiance color $\mathbf{c} = (r,g,b)$. 
%
%They then use traditional volume rendering mechanisms on the obtained continuous 5D function (i.e., MLP) to generate novel views. 
To date, unfortunately, most NeRF-based methods encounter challenges when tackling large-scale cluttered scenes (e.g., Fig.~\ref{fig:teaser}):
\begin{enumerate}[leftmargin=0.16in, topsep=2pt,itemsep=-1ex,partopsep=1ex,parsep=1ex]
\item Input observations used for NeRF are often too sparse  compared to forward-facing or synthetic looking-inward scenes;
%\item Recovering fine-grained objects within a large volume is challenging for NeRF; %in capturing details accurately.
\item View-dependent visual effects give rise to ambiguity, resulting in a ``foggy" density field as shown in Fig.~\ref{fig:teaser}. 
%
Such artifacts are particularly pronounced in indoor scenes strewn with view-dependent appearances, such as specular highlights, glossy surface reflections from man-made objects. 
\end{enumerate}

Despite attempts to enhance NeRF's rendering quality given suboptimal input, such as using 3D conical frustums~\cite{barron2021mip,barron2022mip}, physically-grounded augmentations~\cite{chen2022aug}, and misalignment correction~\cite{jiang2022alignerf},  these challenges have yet to be fully resolved.
%
Depth supervision~\cite{deng2022depth, wei2021nerfingmvs} or proxy geometry~\cite{xu2021scalable,wu2022scalable} images can help alleviate the challenges in handling large-scale with sparse input, at the expense of %but they come at the cost of requiring 
expensive pre-processing or additional input.
%
Another line of work~\cite{wang2021neus, oechsle2021unisurf, wang2022neuris} achieves better reconstruction of surface geometry by using signed distances instead of volume density as scene representation. However, they sacrifice the ability to synthesize photo-realistic novel views.

%We observe that NeRF has been suffering from foggy ``floater" artifacts in large-scale cluttered scenes.
%
%Such artifacts are particularly pronounced in indoor scenes strewn with view-dependent appearances from man-made objects. 
%
To address the above issues, we propose an extension to NeRF, dubbed as {\bf \ournerf}, which enforces effective {\em appearance} and {\em geometry} constraints conducive to accurate colors and 3D densities estimation. We believe \ournerf can contribute beyond novel view synthesis, such as NeRF object detection~\cite{hu2022nerf}, NeRF object segmentation~\cite{zhi2021place, liu2022unsupervised, fan2022nerf,ren2022neural}, and NeRF registration~\cite{goli2022nerf2nerf}, where the rooms for improvement are substantial if more accurate color and density estimation are available.

Correspondingly, there are two steps in \ournerf. First, for appearance correction, the view-independent and view-dependent color components are predicted from the underlying 3D scene, which is combined to produce the final color estimation (Fig.~\ref{fig:toaster}).
%
The view-independent component (diffuse color and shading) captures the overall scene color, while the view-dependent component (highlights or reflections) captures color variations due to changes in viewing angle.
%
\ournerf then discards these view-dependent appearances in the training views to prevent them from interfering with the density estimation.
%
Second, a simple and effective geometry correction procedure will be performed to further eliminate the foggy ``floaters" or density errors. This geometry correction procedure is based on an assumption in line with traditional ray tracing in computer graphics.
\begin{comment}
% xh: basically copying method
On the other hand, ClearNeRF performs a geometric correction procedure performed on each traced ray during inference to refine the density estimation and better tackle the floater artifacts. 
%
The geometry correction procedure assumes that there should only be one salient peak along each traced ray during NeRF inference. 
Only the salient peak closest to the ray origin (the camera center) corresponds to  true geometry while the others will be manifested as foggy floaters hovering in the density volume. 
%
This assumption is in line with traditional ray tracing in computer graphics where in the absence of noise, only one intersection per ray should be returned to indicate the closest ray-object intersection.
%
\end{comment}
%%%%%%%%%%%
%As shown in Fig.~\ref{fig:teaser}, when reconstructing an indoor scene with sparse input and highly view-dependent objects, NeRF produces severe floating artifacts due to its attempt to explain view-dependent appearances.
%
Experiments verify that our proposed \ournerf can effectively get rid of floater artifacts without additional input.% or significant computational overhead. 


In summary, our contributions include the following:
\begin{itemize}[leftmargin=0.16in, topsep=2pt,itemsep=-1ex,partopsep=1ex,parsep=1ex]
    \item We propose a concise method for decomposing view-independent and view-dependent appearance during NeRF training and eliminate the interference of view-dependent appearance.
    \item We propose a geometric correction procedure performed on each traced ray during inference to refine the density estimation and better tackle the floater artifacts.
    \item Extensive experiments and ablations verify the effectiveness of our core designs and results in improvements over the vanilla NeRF and other state-of-the-art alternatives.
    %without additional computational resources or other inputs.
\end{itemize}





\section{Introduction}
\label{sec:intro}


Deep Neural Networks have shown remarkable capabilities at learning %abstract 
representations of their inputs that are useful for %solving 
a variety of %complex 
tasks. Especially since the advent of recent self-supervised learning (SSL) techniques, rapid progress towards learning universally useful representations has been made. 

Currently, however, SSL on images is mainly carried out on benchmark datasets that have been constructed and curated for supervised learning (\eg ImageNet \citep{deng2009imagenet}, CIFAR \citep{krizhevsky2009learning}, etc.).
Although the labels of curated datasets are not \emph{explicitly} used in SSL, the \textit{structure} of the data still follows the predefined set of classes.
In particular, the class-balanced nature of curated datasets could result in a learning signal for unsupervised methods.
As such, these methods are often not evaluated in the settings they were designed for, \ie learning from truly unlabelled data.
Moreover, some methods (\eg \citep{asano2019self,caron2020unsupervised}) even explicitly enforce a uniform prior over the embedding or label space, which cannot be expected to %enforcing priors that do not 
hold for uncurated datasets.

In particular, {uncurated}, real-world data tends to follow long-tail distributions \citep{REED200115}, in this paper, we analyse SSL methods on long-tailed data. Specifically, we analyse the behaviour of contrastive learning (CL) methods, which are among the most popular learning paradigms for SSL. 

In CL, the models are trained such that embeddings of different samples are repelled, while embeddings of different `views' (\ie augmentations) of the same sample are attracted. The strength of those attractive and repelling forces between samples is controlled by a temperature parameter $\tau$, which has been shown to play a crucial role in learning good representations~\citep{chen2020mocov2, chen2020simple}. To the best of our knowledge, $\tau$ has thus far almost exclusively been treated as a \textit{constant} hyper-parameter. % in prior work.


In contrast, %in this work,
we employ a \emph{dynamic} $\tau$ during training and show that this has a strong effect on the learned embedding space for long-tail distributions. 
In particular, by introducing a simple schedule for $\tau$ we consistently improve the representation quality across 
a wide range of settings.
Crucially, %this comes without any additional computational cost
these gains are obtained without additional costs
%\emph{for free}, 
%or loss terms 
and only require oscillating $\tau$ with a cosine schedule.  


This mechanism is grounded in our novel understanding of the effect of temperature on the contrastive loss. 
In particular, we analyse the contrastive loss from an average distance maximisation perspective, which gives intuitive insights as to why a large temperature emphasises 
\emph{group-wise discrimination}, whereas a small temperature 
leads to a higher degree of 
\emph{instance discrimination}
and more uniform distributions over the embedding space. 
Varying $\tau$ during training ensures that the model learns both %easily separable 
group-wise and
%features as well as hard 
instance-specific
features,
resulting in better separation between head and tail classes.

Overall, our contributions are summarised as follows:
$\bullet$ we carry out an extensive %theoretical and practical
analysis of the effect of $\tau$ on imbalanced data; $\bullet$ we analyse the contrastive loss from an average distance perspective to understand the emergence of semantic structure; $\bullet$ we propose a simple yet effective temperature schedule that improves the performance across different settings;
$\bullet$ we show that the proposed $\tau$ scheduling is robust and consistently improves the performance for different %contrastive learning methods and 
hyperparameter choices.


% \section{Related work}

\label{sec:related_work}
% \rict{
% In this section we cover prior art closely related to our study. First, we discuss the state of the art in semantic segmentation. Next, we provide an overview of works tackling the robustness of neural networks. We follow by covering the state of the art in calibration techniques---namely methods to reduce the overconfidence of neural networks. We conclude with a paragraph summarizing other studies that share similarities with ours.
% }
We study robustness and uncertainty in 
% image 
semantic
segmentation. In doing so, we touch several fields, which we cover in the following. We further discuss related studies.

% \rict{Many different problem formulations fall under the large hood of ``uncertainty''. For what concerns \textit{model calibration},}

\myparagraph{Segmentation models} 
% \rict{Provided with an image, the segmentation task is that of predicting the semantic class each pixel belongs to.}
Modern segmentation pipelines typically consist of encoder-decoder architectures~\cite{CsurkaNow22SemanticImageSegmentation2DecasesOfResearch, BadrinarayananPAMI17SegnetDeepConvEncoderDecoder, NohICCV15LearningDeconvolutionNNSegmentation, RonnebergerMICCAI15UNetSegmentation}. Decoders are usually designed \textit{ad hoc} for segmentation, with DeepLab~\cite{ChenPAMI17DeeplabSemanticImgSegmentationDeepFullyConnectedCRF, ChenX17RethinkingAtrousConvolutionSemSegm, ChenECCV18EncoderDecoderAtrousSeparableConvSemSegm} and UPerNet~\cite{XiaoECCV18UnifiedPerceptualParsingSceneUnderstanding} being two of the most prominent. On the other hand, the evolution of 
% backbone models 
encoders
has been closely related to that of classification 
% models.
% ResNet~\cite{he2016deep} has been the most popular for years;
models, with ResNet~\cite{he2016deep} being one of the most popular for years.
%
The rise of transformers in computer vision~\cite{DosovitskiyICLR21AnImageIsWorth16x16WordsTransformersAtScale} has led to a flurry of works leveraging self-attention for 
% different tasks \cite{DosovitskiyICLR21AnImageIsWorth16x16WordsTransformersAtScale, touvron2021training, carion2020end, ZhengCVPR21RethinkingSiSSeq2SeqPerspTransformers, XieNIPS21SegFormerSemSegmTransformers, StrudelICCV21SegmenterTransformerForSemSegm, zhou2022understanding}. 
segmentation~\cite{ZhengCVPR21RethinkingSiSSeq2SeqPerspTransformers,StrudelICCV21SegmenterTransformerForSemSegm,XieNIPS21SegFormerSemSegmTransformers}.
% This has also lead to 
Novel convolutional architectures inspired by transformers have also risen~\cite{LiuCVPR22AConvNet4The2020s}. We compare several recent segmentation models 
% to 
against
ResNet baselines, in terms of both robustness and uncertainty.

\myparagraph{Robustness} The brittleness of neural networks to changes in the input domain is a well-studied problem and many sub-formulations exist~\cite{TaoriNeurIPS20MeasuringRobustnessToNaturalDistributionShifts}. 
Robustness against synthetic shifts takes into account 
% domains generated
samples crafted
by artificially altering images, for example injecting noise or blur (corruption robustness~\cite{HendrycksICLR19BenchmarkingNNRobustnessCommonCorruptionsPerturbations,KamannCVPR20BenchmarkingRobustnessSemSegmModels}), or crafting imperceptible perturbations to induce model failure (adversarial robustness~\cite{GoodfellowICLR15ExplainingHarnessingAdvExamples}).
% and many solutions have been devised 
% \cite{CsurkaBC17AComprehensiveSurveyDAForVisualApplications, VolpiNIPS19GeneralizingUnseenDomainsAdvDataAugm, ToldoX20UnsupervisedDASSReview}. 
Robustness against \textit{natural} shifts focuses on changes that may arise naturally, without human intervention~\cite{RechtICLR19DuImageNetClassifiersGeneralizeToImageNet,hendrycks2021natural}.

In this work we are interested in comparing the robustness of different off-the-shelf segmentation models under \textit{natural} domain shifts, since these are particularly relevant in real-world applications.
% \rict{since we have real-world applications in mind.}
In particular, we focus on semantic segmentation of urban scenes, hence, 
we evaluate models on samples from unseen geographical locations~\cite{VarmaWACV19IDDDatasetExploringADUnconstrainedEnvironments} and weather conditions~\cite{SakaridisICCV21ACDCAdverseConditionsDatasetSIS}.
% For a discussion of recent robustness studies see our analyses paragraph.
Segmentation robustness against natural shifts has been studied before~\cite{YueICCV19DomainRandomizationPyramidConsistencySimulationToReal,VolpiCVPR22OnRoadOnlineAdaptSiS}, yet not in tandem with uncertainty and within a large-scale study taking 
% in
into
consideration several recent models.\looseness=-1

% from rel work
% While it is relatively extended practice to measure robustness against simulated image corruptions~\cite{HendrycksICLR19BenchmarkingNNRobustnessCommonCorruptionsPerturbations, zhou2022understanding,KamannCVPR20BenchmarkingRobustnessSemSegmModels} (\eg blur, noise, etc) such synthetic interventions may not be representative of natural shifts \cite{taori2020measuring}. Thus, we argue that our setting is closer to real-world deployment scenarios. \todo{move to rel work}



% \myparagraph{Reliability}
\myparagraph{Uncertainty}
% \myparagraph{\rict{Calibration}}
Guo \etal \cite{GuoICML17OnCalibrationOfModernNNs} have shown that deep 
% learning 
models 
% tend to be 
are
overconfident. They have proposed a simple, yet effective solution known as temperature scaling (\ts) where the output logits are divided by a temperature parameter before 
% applying 
the softmax layer. Other calibration methods have been proposed (\eg, \cite{naeini2015obtaining, kull2019beyond, gupta2020calibration}), but \ts is still
% the most 
very
popular due to its simplicity and the fact that it does not alter predictions. 

Calibration with \ts is effective
% works reasonably well 
in \id settings; yet, Ovadia \etal \cite{ovadia2019can} have shown that model calibration degrades significantly out of domain. Some methods have been proposed that address this problem~\cite{pampari2020unsupervised, park2020calibrated, wang2020transferable}, by assuming access to unlabeled \ood images beforehand.
%---a constraint for real-world deployment. 
% To the best of our knowledge, only 
On the other hand, Gong~\etal~\cite{gong2021confidence} have proposed methods that improve \ood calibration without any data from the target domain. 
%\rict{(domain generalization settings)}. 
They propose to cluster the calibration set in different ``domains'' and find a different temperature value for each. At test time, images are calibrated using the temperature from the closest cluster. 
% On the other hand, while originally devised for \id calibration, 
%
\textit{Ad hoc} for semantic segmentation, Ding~\etal~\cite{ding2021local} propose a content-dependent calibration strategy that learns a small calibration network to predict a temperature for each pixel in an image.

In our study we test \id and \ood performance of several calibration methods, focusing on techniques that do not require access to \ood samples~\cite{GuoICML17OnCalibrationOfModernNNs,gong2021confidence,ding2021local}---as generally robustness is evaluated on unseen domains~\cite{HendrycksICLR19BenchmarkingNNRobustnessCommonCorruptionsPerturbations,KamannCVPR20BenchmarkingRobustnessSemSegmModels,TaoriNeurIPS20MeasuringRobustnessToNaturalDistributionShifts,pinto2022impartial}

% \looseness=-1

\myparagraph{Previous analyses}
In~\cref{tab:comparison} we 
compare
related studies 
% focused 
on different aspects of reliability. Several works have suggested that transformer-based \textit{classifiers} are more robust than CNNs \cite{BhojanapalliICCV21UnderstandingRobustnessTransformersImageClass, naseer2021intriguing, bai2021transformers, mao2022towards, paul2022vision}. Yet, the recent ConvNeXt \cite{LiuCVPR22AConvNet4The2020s} has challenged this result and later work have suggested that further investigation is needed~\cite{pinto2022impartial}. 
Minderer \etal \cite{minderer2021revisiting} have compared calibration of several classifiers, concluding that convolution-free models are more robust \textit{and} better calibrated. In contrast, Pinto \etal \cite{pinto2022impartial} have compared recent transformers and CNNs, arguing there is ``no clear winner''. 
Some works have compared the robustness of transformers and CNNs 
% in
for
segmentation~\cite{XieNIPS21SegFormerSemSegmTransformers, zhou2022understanding}---but only against \textit{synthetic} domain shifts.
% (\eg noise, blur).
We broadly study robustness \textit{and} uncertainty in segmentation under \textit{natural} domain shifts. Similarly to \cite{pinto2022impartial}, we do not observe a single model family which is better calibrated in all scenarios. In contrast with \cite{minderer2021revisiting} though, we observe that robustness and calibration \textit{do not} go hand in hand. 
This shows that not all trends observed in classification transfer to segmentation, confirming the importance of task-specific studies like ours.





% \myparagraph{Uncertainty estimation} To be safely deployed in the real world, deep learning models should be reliable. We evaluate the reliability of segmentation models on three main aspects, model calibration, misclassification detection and out of domain detection.

% \myparagraph{Calibration} Ideally, a well calibrated model should be more confident for correct predictions than incorrect ones,
% %this is often measured by the Expected Calibration Error (ECE).
% Guo \etal \cite{GuoICML17OnCalibrationOfModernNNs} observed that deep learning models tend to be overconfident and propose a simple, yet effective solution known as \ts where the output logits are divided by a ``temperature'' parameter before applying the softmax layer. Although other calibration methods have been proposed \cite{naeini2015obtaining, kull2019beyond, gupta2020calibration} \ts is still the most popular due to its simplicity and that it does not 
% % change 
% alter
% the predicted labels. 

% Ovadia \etal \cite{ovadia2019can} showed that calibration of models degrades under domain shifts. Some methods have been proposed to improve calibration under covariate shift \cite{pampari2020unsupervised, park2020calibrated, wang2020transferable}, but they assume access to the out of domain images (without labels) beforehand, which is a strong constraint for real-world deployment. To the best of our knowledge, only one technique has been proposed to improve calibration without any data from the target domain \cite{gong2021confidence} which relies on clustering the calibration set to find different \ts parameters depending on the image features. On the other hand, while originally devised to improve calibration \id, Ding \etal propose a content-dependent calibration strategy, specifically designed for segmentation, that learns a small calibration network to predict a temperature for each pixel in an image \cite{ding2021local}. In our experiments we study these two techniques with several segmentation models to improve calibration out of domain.

% Recently, Minderer \etal \cite{minderer2021revisiting} compared the calibration of several classification models and concluded transformers are better calibrated than CNNs, however, the more recent ConvNeXt \cite{LiuCVPR22AConvNet4The2020s} was not in the analysis and Pinto \etal \cite{pinto2022impartial} later observed there was ``no clear winner''. Unlike previous analyses, our work is centered on segmentation with special emphasis on domain shifts and we explore methods to improve calibration out of domain. 

% \myparagraph{Misclassification detection} Although calibration is an intuitive and reasonable metric, it has some blind spots, for instance, a random binary classifier that always predicts with $50\%$ confidence will be perfectly calibrated in a balanced set but its confidence is completely uninformative. The idea of the misclassification detection task is to predict the samples in which the prediction is wrong. Following \cite{malinin2019ensemble} we use the Prediction Rejection Ration (PRR) since they showed other metrics can be biased by the model performance.

% \myparagraph{Out of domain detection} Aside from correctly assessing the uncertainty in their predictions, perhaps an even more fundamental task for reliable models should be to detect when a certain input does not fall into the training domain, we use the Area Under the Receiver Operating Characteristic curve (AUROC) \cite{murphy2012machine} since it is agnostic to unbalanced classes and we have different number of in and \ood images \cite{pinto2022impartial}.

\section{Related Work}
\label{sec:rel_work}

Self-supervised representation learning (SSL) from visual data is a quickly evolving field. 
Recent methods are based on various forms of comparing embeddings between transformations of input images. 
We divide current methods into two categories: contrastive learning~\citep{he2019moco, chen2020mocov2, chen2020simple, oord2018representation}, and non-contrastive learning \citep{grill2020bootstrap,zbontar2021barlow,chen2021exploring,bardes2022vicreg,wei2022masked,Gidaris_2021_CVPR, asano2019self, caron2020unsupervised, he2022masked}.
Our analysis concerns the structure and the properties of the embedding space of contrastive methods when training on imbalanced data. 
Consequently, this section focuses on contrastive learning methods, their analysis and application to imbalanced training datasets. 

\myparagraph{Contrastive Learning}
employs instance discrimination \citep{wu2018unsupervised} to learn representations by forming positive pairs of images through augmentations and a loss formulation that maximises their similarity while simultaneously minimising the similarity to other samples. 
Methods such as MoCo~\citep{he2019moco, chen2020mocov2}, SimCLR~\citep{chen2020simple, chen2020big}, SwAV~\citep{caron2020unsupervised}, CPC~\citep{oord2018representation}, CMC~\cite{tian2020contrastive}, and Whitening~\citep{ermolov2021whitening} have shown impressive representation quality and down-stream performance using this learning paradigm.
CL has also found applications beyond SSL pre-training, such as multi-modal learning \citep{shvetsova2022everything}, domain generalisation \citep{yao2022pcl}, semantic segmentation \citep{van2021unsupervised}, 3D point cloud understanding \citep{afham2022crosspoint}, and 3D face generation \citep{deng2020disentangled}.

% To further improve the representations, 
\myparagraph{Negatives.} The importance of negatives for contrastive learning is remarkable and noticed in many prior works~\citep{wang2021solving, yeh2021decoupled, zhang2022dual, iscen2018mining, kalantidis2020hard, robinson2020contrastive, khaertdinov2022dynamic}. \citet{yeh2021decoupled} propose decoupled learning by removing the positive term from the denominator, \citet{robinson2020contrastive} develop an unsupervised hard-negative sampling technique, \citet{wang2021solving} propose to employ a triplet loss, \rebuttal{and \citet{zhang2022dual, khaertdinov2022dynamic} propose to improve negative mining with the help of different temperatures for positive and negative samples that can be defined as input-independent or input-dependent functions, respectively.}
% discussed in depth in recent work.. 
In contrast to \emph{explicitly} choosing a specific subset of negatives, we discuss the Info-NCE loss~\citep{oord2018representation} through the lens of an average distance perspective with respect to {all} negatives and show that the temperature parameter can be used to \emph{implicitly} control the effective number of negatives.

\myparagraph{Imbalanced Self-Supervised Learning.} Learning on imbalanced data instead of curated balanced datasets is an important application since natural data commonly follows long-tailed distributions~\citep{REED200115, liu2019large, wang2017learning}. 
In recent work, ~\citet{kang2020exploring},~\citet{yang2020rethinking},~\citet{liu2021selfsupervised},~\citet{zhong2022self},~\citet{gwilliam2022beyond} discover that self-supervised learning generally allows to learn a more robust embedding space than a supervised counterpart.  
\citet{tian2021divide} explore the down-stream performance of contrastive learning on standard benchmarks based on large-scale uncurated pre-training and propose a multi-stage distillation framework to overcome the shift in the distribution of image classes.   
\citet{jiang2021self,zhou2022contrastive} propose to address the data imbalance by identifying and then emphasising tail samples during training in an unsupervised manner. For this, \citet{jiang2021self} compare the outputs of the trained model before and after pruning, assuming that tail samples are more easily `forgotten' by the pruned model and can thus be identified. \citet{zhou2022contrastive}, use the loss value for each input to identify tail samples and then use stronger augmentations for those. %To alleviate this problem, they propose to modify the contrastive framework SimCLR~\citep{chen2020simple} by contrasting an online model with a pruned version of the same model to identify difficult samples. 
Instead of modifying the architecture or the training data of the underlying frameworks, we show that a simple approach---\ie oscillating the temperature of the Info-NCE loss~\citep{oord2018representation} to alternate between instance and group discrimination---can achieve similar performance improvements at a low cost.


\myparagraph{Analysis of Contrastive Learning (CL).} 
Given the success of CL in representation learning, it is essential to understand its properties.
While some work analyses the interpretability of embedding spaces \citep{bau2017network,fong2018net2vec,laina2020quantifying,laina2021measuring}, here the focus lies on understanding the structure and learning dynamics of the objective function such as in \cite{saunshi2019theoretical, tsai2020self, chen2021intriguing}.
E.g., \citet{chen2021intriguing} study the role of the projection head, the impact of multi-object images, and a feature suppression phenomenon. \citet{wen2021toward} analyse the feature learning process to understand the role of augmentations in CL. \citet{robinson2021can} find that an emphasis on instance discrimination can improve representation of some features at the cost of suppressing otherwise well-learned features. \cite{wang2020understanding, wang2021understanding} analyse the uniformity of the representations learned with CL. In particular, \citet{wang2021understanding} focus on the impact of individual negatives and describe a uniformity-tolerance dilemma when choosing the temperature parameter. In this work, we rely on the previous findings, expand them to long-tailed data distributions and complement them with an understanding of the emergence of semantic structure. 



% \section{Method}
\label{sec: method}
% This section introduces the rendering pipeline of our proposed hierarchical compositional scene. 
% our pipeline consists of three processes, including decomposing the text into editable 3D layout, rendering the compositional views with local (object) NeRFs and global (scene) NeRF and the joint optimization on these hierarchical 3D representations.

% Note that the transformation between the object and the scene frame is defined by ${p}_o$ and ${D}_o$. 
%
% Next, we build a residual connection to add ${\sigma}_o$ and the referenced global color, and the rendering result will be used to calculate the SDS loss based on the global text.  
% Fig.~\ref{fig:framework} illustrates our pipeline, which consists of three main components, including the editable 3D scene layout based on multi-object text (Sec.~\ref{ssec:layout}), the scene rendering pipeline that composites the predictions from all local NeRFs (Sec.~\ref{ssec:render}), and the joint optimization on both local and global representation models (Sec.~\ref{sec:optimization}).
% To elaborate, our editable 3D scene layout represents a global frame of the scene by decomposing it into a set of local frames, where each is parameterized by a local NeRF, a 3D bounding box, and a corresponding local text prompt.
% For instance, the text prompt `A teddy bear and a stuffed monkey sit side by side' is interpreted as a 3D scene layout, as shown in Fig.~\ref{fig:framework}.  
% The whole 3D layout, \ie, scene frame, consists of two 3D bounding boxes, \ie local frames \#1 and \#2, with specific local text prompts, \ie, `a teddy bear' and `a stuffed monkey'. 
% %
% To render the scene view, we first calculate the ray-box intersections between the boxes and rays $({\boldsymbol{r}}_o, \boldsymbol{\phi}_d, {\boldsymbol{\theta}}_d)$, where the ${\boldsymbol{r}}_o$ is the ray origin and the $({\boldsymbol{r}}_o, \boldsymbol{\phi}_d)$ is its direction.
% Then, to infer each object's properties in local NeRFs, we sample the global points $({\boldsymbol{x}}_g, {\boldsymbol{y}}_g, {\boldsymbol{z}}_g)$ in the global frame within the ray-box intersection intervals and project them into the normalized local location $({\boldsymbol{x}}_l, {\boldsymbol{y}}_l, {\boldsymbol{z}}_l)$ in the local frame.
% %
% Given the local sampling points $({\boldsymbol{x}}_l, {\boldsymbol{y}}_l, {\boldsymbol{z}}_l)$, the implicit local NeRF ${\boldsymbol{\theta}}_l$ outputs four pseudo-color channels ${\boldsymbol{C}}_l$ and density $\boldsymbol{\sigma}$, which can be used to render a local view of the local frame to match its local text prompt.
% %
% We further calibrate the predicted pseudo-color $\boldsymbol{C}_l$ from local frames by adding the global embeddings ${\boldsymbol{emb}}_g$ to improve the global view consistency.
% Then, the calibrated predictions after composition are used to reconstruct the scene view by volumetric rendering along the rays.
% %
% Lastly, the rendered views based on local and global frames are guided by score distillation sampling loss $\nabla \mathcal{L}_{\text{SDS}}$~\cite{poole2022dreamfusion} to optimize all the learnable parameters. 
To resolve the issue of guidance collapse, our principal strategy is to \textit{decompose the scene into reusable components and compose/recompose them into a unified and consistent one}.
This enables flexible control over the generated content with direct use of prompts and box layouts, as illustrated in \cref{fig:teaser}.
%
Our proposed CompoNeRF confers several key benefits:
1) \textbf{Semantic Coherence}: It reliably creates 3D objects with detailed textures and global consistency, exemplified by authentic light interactions, such as reflections on the bed surface.
2) \textbf{Modularity and Reusability}: CompoNeRF functions as an ensemble of independently trained NeRF models. These can be efficiently stored and later retrieved from a cached dataset, enabling their reuse in various cases.
3) \textbf{Editability}: Our approach allows for flexible scene modification, such as interchanging the lamp for a vase filled with sunflowers or altering its scale, by simply adjusting the box dimensions for later finetuning. This feature enhances flexibility and creative possibilities. 


% Furthermore, the usage of layout boxes enables more flexible control over the generated content compared with the intricate sketch shape in Latent-NeRF\cite{metzer2022latent}. 
\begin{figure*}[t]
    \centering
    \includegraphics[width=0.9\linewidth]{figures/method.pdf}
    % \vspace{-12pt}
    \caption{\textbf{Framework Overview}.
The CompoNeRF model unfolds in three stages: 1) Editing 3D scene, which initiates the process by structuring the scene with 3D boxes and textual prompts; 2) Scene rendering, which encapsulates the composition/recomposition process, facilitating the transformation of NeRFs to a global frame, ensuring cohesive scene construction. Here, we specify design choices between density-based or color-based(without refining density) composition; 3) Joint Optimization, which leverages textual directives to amplify the rendering quality of both global and local views, while also integrating revised text prompts and NeRFs for refined scene depiction.
  % The model is structured into three components: Composition, Decomposition, and Recomposition. Composition deals with the foundational setup, detailed with choices for density-based and color-based composition. Decomposition utilizes the modularity of the CompoNeRF feature, caching each NeRF module offline for efficient recalibration. Recomposition reuses these cached NeRFs and adjusts the semantic context, providing a revised output with the inclusion of the offline NeRF enhancements.
    % Our model consists of two branches where the upper part is individual NeRFs, and the lower part denotes global calibration with our tailored composition model. The specific designs for density-based and color-based composition modules are highlighted. 
    % CompoNeRF consists of three parts: 1). The editable 3D scene layout configures the scene representations with 3D boxes and text prompts; 2).  The scene rendering includes the global calibration and the compositional process; 3). The joint optimization applies global and local text guidance on global and local render views.
    % The global frame (scene space) contains a set of local frames. Each is  represented by a local NeRF associated with a 3D box and text prompt defined by the editable 3D layout.
    % The scene view is volumetric rendered by sampling the points $({\boldsymbol{x}}_g, \boldsymbol{y}_g, \boldsymbol{z}_g)$ intersected with any local frame along the ray $(\boldsymbol{r}_o, {\boldsymbol{\phi}}_d, \boldsymbol{\theta}_d)$.
    % The sampling points are first inferred through the local NeRF with the local frame locations $({\boldsymbol{x}}_l, \boldsymbol{y}_l, \boldsymbol{z}_l)$ projected from the global location $({\boldsymbol{x}}_g, \boldsymbol{y}_g, \boldsymbol{z}_g)$.
    % And then, all the local predictions are calibrated by a global MLP with conditional input to render the scene view.
    % During the optimization, the text guidance is applied to both local views predicted by local frames only and global views predicted by the composition of all local frame predictions.
    }
    \label{fig:framework}
    % \vspace{-8pt}
\end{figure*}

\subsection{Preliminaries}
Defining individual object bounding boxes as \textit{local frames} and the overall scene coordinate system as the \textit{global frame}, we build the foundation of NeRF and diffusion processes.

\label{sec:background}
\noindent \textbf{3D Representation in Latent Space.}
Our methodology capitalizes on the state-of-the-art text-to-image generative model—Stable Diffusion as described by Rombach et al\cite{rombach2022high}.
We build upon the Latent-NeRF framework~\cite{metzer2022latent}, which computes latent colors for individual objects by considering their sample positions within a localized frame. Specifically, it maps a three-dimensional point in local coordinates \(\boldsymbol{x}_l = (x_l, y_l, z_l)\) to a volumetric density \(\boldsymbol{\sigma}_l\) and an associated color \(\boldsymbol{C}_l\), expressed as \((\boldsymbol{C}_l, \boldsymbol{\sigma}_l) = f_{\boldsymbol{\theta}_l}(x_l, y_l, z_l)\). Here, \(f\) represents a Multi-Layer Perceptron (MLP) characterized by parameters \(\boldsymbol{\theta}_l\).
 This NeRF-generated color is then assessed in the context of the Stable Diffusion model, using text prompts to guide NeRF toward spatially coherent inference with intricate context.
% to infer pseudo-color for each object using local NeRF.
% Specifically, the representation maps a point $\boldsymbol{x}_l = \left({x}_l, {y}_l, {z}_l\right)\in [-1, 1]$ in the local frame to its corresponding volumetric density $\boldsymbol{\sigma}_l$ and emitted color $\boldsymbol{C}_l$, \ie,  $\left(\boldsymbol{C}_l, {\boldsymbol{\sigma}_l}\right)=\boldsymbol{\theta}_{_l}\left({x_l}, {y}_l, {z}_l\right)$.
% The predicted pseudo-color is fed forward into the decoder of the Stable Diffusion model to obtain the final rendering result.

\noindent \textbf{Volume Rendering with Multiple Objects.}
% For each local frame $j$ with NeRF parameterized as $\theta_j$, we follow original NeRF design\cite{nerf} to integrate $(\boldsymbol{C}_l, \boldsymbol{\sigma}_l)$ of   sampled points from any hit ray $r_l=(\boldsymbol{o}_l, \boldsymbol{d}_l)$ by,
% For consistent scene rendering, object transmittance $T_k$ must be recalculated in the global frame based on independent properties inferred from local NeRFs. Hence, we sort predictions according to their distance to $\boldsymbol{o}_g$. 
% Similar to \cref{eq:volrend}, global color $\hat{\boldsymbol{C}}_g$ of ray $\boldsymbol{r}_g=(\boldsymbol{o}_g, \boldsymbol{d}_g)$ is predicted by the volumetric rendering integrating over $m$ objects,
We extend the volume rendering process to accommodate multiple objects by assigning each a local frame, denoted as $j$, with NeRF parameters $\boldsymbol{\theta}_{l, j}$. Drawing from the foundational NeRF approach \cite{nerf}, in each local frame, we integrate the color $\boldsymbol{C}_l$ and density $\boldsymbol{\sigma}_l$ for points $\boldsymbol{x}_l$ sampled along a ray $\boldsymbol{r}_l$, emanates from the camera origin $\boldsymbol{o}_l$ in direction $\boldsymbol{d}_l$. This is formalized in the predicted color integration for $\hat{\boldsymbol{C}}_l$ as:
{\setlength\abovedisplayskip{2pt}
\setlength\belowdisplayskip{2pt}
\begin{equation}
\label{eq:volrend}
{\hat{\boldsymbol{C}}_l}({\boldsymbol{r}_l})=\sum_{k=1}^{N} T_{l, k} \left(1-\exp \left(-\sigma_{l, k} \delta_k\right) \right) {\boldsymbol{C}}_{l,k},
\end{equation}}where $T_{l, k}=\exp \left(-\sum_{j=1}^{k-1} \sigma_{l,j} \delta_j\right)$ represents the transmittance to the $k$-th of total $N$ sample, calculated exponentially over the cumulative density along $\boldsymbol{r}_l$, and $\delta_k$ is the interval between adjacent samples.
%
To synthesize a coherent scene, we transition from processing individual local frames to a collective global frame. Within this global context, we reconcile object attributes inferred from their individual local NeRFs for refined $\boldsymbol{\sigma}_g, \boldsymbol{C}_g$ along with $T_{g, k}$. The samples $\boldsymbol{x}_g$ are ordered based on their spatial distances from the origin $\boldsymbol{o}_g$ following the coordinate transformation. We then express the volumetric rendering of a ray $\boldsymbol{r}_g$ integrating $m$ objects within the global frame as follows:
{
\setlength\abovedisplayskip{2pt}
\setlength\belowdisplayskip{2pt}
\begin{equation}
\label{eq:multi_volrend}
{\hat{\boldsymbol{C}}_g}({\boldsymbol{r}_g})=\sum_{k=1}^{m*N} T_{g, k} \left(1-\exp \left(-\sigma_{g, k} \delta_k\right) \right) {\boldsymbol{C}}_{g,k}. 
\end{equation}}

\noindent \textbf{Score Distillation Sampling.}
% During the SDS process, a noise image $\boldsymbol{X}_t$ is first generated by adding a sampled noise $\epsilon \sim \mathcal{N}(0, I)$ in noise level $t$ into a rendered view $\boldsymbol{X}$ from a NeRF.
To facilitate the conversion from text descriptions to 3D models, DreamFusion~\cite{poole2022dreamfusion} utilizes Score Distillation Sampling (SDS), leveraging the generative capabilities of a diffusion model, denoted as $\phi$, to guide the optimization of NeRF parameters, symbolized as $\boldsymbol{\theta}$.
%
Initially, SDS creates a noisy image $\boldsymbol{X}_t$ by infusing a randomly sampled noise $\epsilon$, which follows a normal distribution $\mathcal{N}(0, I)$, into a NeRF-rendered image $\boldsymbol{X}$ at a given noise level $t$.
The diffusion model $\phi$ then estimates the noise $\epsilon_\phi\left(\boldsymbol{X}_t, t, T\right)$ from this noisy image, conditioned by the noise level $t$ and an optional text prompt $T$. 
The key step in SDS involves calculating the gradient of the loss function, which measures the discrepancy between the estimated noise and the originally added noise:
{\setlength\abovedisplayskip{2pt}
\setlength\belowdisplayskip{2pt}
\begin{equation}
\label{eq:sds_loss}
\nabla_\theta \mathcal{L}_{\text{SDS}}(\boldsymbol{X}_t, T)=  w(t)\left(\epsilon_\phi\left(\boldsymbol{X}_t, t, T\right)-\epsilon\right),
\end{equation}}where $w(t)$ is a weighting function that adjusts the influence of the gradient based on the noise level. 
The gradients across all rendered views direct the update of $\boldsymbol{\theta}$, ensuring that the NeRF-generated images align with the text descriptions. Additionally, we incorporate the 'perturb and average' technique from SJC for more robust $\mathcal{L}_{\text{SDS}}$. For a comprehensive understanding of these methods, the reader is directed to the detailed explanations provided in \cite{poole2022dreamfusion,wang2022score}.

%
%
% \subsection{Editable 3D Scene Layout}
% \label{ssec:layout}
% The 3D scene layout explicitly combines language structures with 3D layouts in an editable way.
% Given the input text prompt $T$, the attribute-object pairs can be easily obtained based on user control.
% Note that the text prompt indicates the multi-object text prompt by default.
% % available for free in many structured representations, such as the constituency tree.
% As shown in Fig.~\ref{fig:framework}, we can extract multiple noun phrases with their binding attributes and map these local text prompts into corresponding regions.
% Specifically, we define the scene structure with $m$ local frames, each employs a local NeRF $\boldsymbol{\theta}_l$ as representation, the local text prompt $T_{l} \subseteq{T}$ and its spatial layout with 3D boxes $\mathbf{b} = \{\mathbf{p}, \mathbf{s}\} \in  \mathbb{R}^6$ of each object entity, where $\mathbf{p}=\{p_x, p_y, p_z\}$ refers to the center point and $\mathbf{s}=\{s_x, s_y, s_z\}$ denotes the box scale. 
% \textit{Our editable 3D layout is easy to be collected and edited with its simplicity, allowing for versatile and interactive user control by modifying the box's or text's properties to define a new scene}.
% Moreover, as depicted in Fig.~\ref{fig:teaser}, each component in a 3D scene layout can be replaced or re-composited with other trained local NeRFs, which is more friendly for flexible user editions compared with using only text prompts.
% We fine-tuned the new layout by global rendering, which enables scalable re-editing.
% Each relationship $r_k \in R$ is a triplet in a <subject-predictive object> format, where a subject node is. After we generate the scene graph from the complex prompts, we can sample the closest relationship with the 2d spatial layout as the initial 3D position. fine-tuned the new layout by global rendering, which enables scalable re-editing
%
% \subsection{Scene Rendering Pipeline}
% \label{ssec:render}
% In CompoNeRF, the scene images are rendered by a ray-casting approach following the design of NeRF.
% % Each ray to be cast is generated based on the camera pose, intrinsic, and transformation.
% The camera is defined by a pinhole camera model, casting a set of rays $(\boldsymbol{r}_o, \boldsymbol{\phi}_d, {\boldsymbol{\theta}}_d)=\boldsymbol{o}+t\boldsymbol{d}$ through each pixel on the frame of size $H \times W$, where the $\boldsymbol{r}_o \in  \mathbb{R}^3$ is the origin and the $(\boldsymbol{\phi}_d, \boldsymbol{\theta}_d)$ is the viewing direction.
% Along this ray, we sample all the points intersected with any layout box of local frames.
% For each hit sampled point, the color and volumetric density are computed through the local NeRF of the hit local frame.
% The ray color perdition is calculated by the differentiable integration applied on all the point-predicted colors and volumetric density along the ray.
%
% \noindent \textbf{Ray-box Intersection with Local Frames.}
% Given a ray $\boldsymbol{r}_i$, each box $\boldsymbol{b}_j$ of the local frame is applied with the AABB ray intersection test algorithm to check the intersections.
% When the ray $r_i$ is hit with a box $\boldsymbol{b}_j$ of the local frame, we use the entrance and exit points as near $\boldsymbol{t}_{in}$ and far $\boldsymbol{t}_{out}$ bounds to sample $N$ equidistant quadrature points, $
% \boldsymbol{t}_{i,j,n}=\frac{n-1}{N-1}\left(\boldsymbol{t}_{out}-\boldsymbol{t}_{in}\right)+\boldsymbol{t}_{in} , n \in \left[1, N\right]$
% % Despite each local frame only having a small number of hit rays compared to the scene, we observe that it is enough to represent each object accurately while maintaining short rendering times.
% Note that the coordinates of sampled points are first projected into normalized coordinates using the box scale of local frames to enable each local NeRF to learn the scale-independent representation.
% The bounding box $\mathbf{b}$ of the local frame in global coordinate can be transformed into a canonical bounding box by ${(\mathbf{b}} - \boldsymbol{p}) / \mathbf{s}$.
% Considering the rendering efficiency, we only calculate the valid points, interacted with the boxes, and set all the empty points with a constant background color.
%
% The appearance of a set object representations depends on its interaction with the scene and illumination which should be decided by the local frame location.
% To ensure the volumetric consistency, we only calibrate the emitted color with scene location, while the gradient still can be propagated.
% Since the overall color depends on both the global  positions $({x}_w, {y}_w, {z}_w)$ and ray directions $({\phi}_d, {\theta}_d)$, the global color embedding is learned based on both the positions and ray directions.
% Since the overall color depends on both the global  positions $({x}_w, {y}_w, {z}_w)$ and ray directions $({\phi}_d, {\theta}_d)$, the global color embedding is learned based on both the positions and ray directions.
% \subsection{The Proposed CompoNeRF}
% \subsubsection{Composition Module}
% CompoNeRF aims to composite multiple NeRFs to reconstruct multi-object scenes with both box and prompt guidance.
% %
% Our framework, as shown in \cref{fig:framework}, applies the AABB ray intersection test algorithm to check for intersections on each box in the global frame. We then samples $\boldsymbol{x}_g$ within the ray box intervals, and project them to $\boldsymbol{x}_l$ to infer  $\left(\boldsymbol{C}_l, {\boldsymbol{\sigma}_l}\right)$ in separate NeRF models. 
% %
% We then utilize volume rendering to obtain rendered views for each local frame respectively. 
% %
% After that, they would be passed on to our tailored composition Module to infer 
% $\left(\boldsymbol{C}_g, {\boldsymbol{\sigma}_g}\right)$
% for global rendering. 
% Next, we match local and global texts with their corresponding image outputs by SDS losses. 
% We also support recomposition by passing samples from cached models into $\boldsymbol{x}_l$ to continue the above process.
\begin{figure}[t!]
    \centering
    \includegraphics[width=\linewidth]{figures/abls.pdf}
    % \vspace{-22pt}
    % \caption{Ablation study on text guidance. (a) without local SDS losses. (b) without global SDS losses. (c) vanilla SDS losses without perturb and average scoring~\cite{wang2022score}. (d) full model.}
    \caption{\textbf{Design Impact Comparison: Density vs. Color-based Methods.} The top row illustrates the density-based approach's detailed rendering and quick convergence in the 'table wine' scene. The bottom row highlights the color-based method's enhancements and its drawbacks, such as geometric and shadow inaccuracies, particularly in close-up views and slow convergence.
    % \textbf{(a)} global text guidance(integrating local frames by \cref{eq:multi_volrend}) and global calibration(integrating local frames, then aligning the rendering result directly with the full text). 
    }
    \label{fig:abls}
    % \vspace{-20pt}
\end{figure}
\subsection{The Proposed CompoNeRF}
\subsubsection{Composition Module}
CompoNeRF is designed to composite multiple NeRFs to reconstruct scenes featuring multiple objects, utilizing guidance from both bounding boxes and textual prompts. Within our framework, depicted in \cref{fig:framework}, the Axis-Aligned Bounding Box (AABB) ray intersection test algorithm is applied to ascertain intersections across each box in the global frame. Subsequently, we sample points \(\boldsymbol{x}_g\) within the intervals of the ray-box and project them to \(\boldsymbol{x}_l\) to deduce the corresponding color \(\boldsymbol{C}_l\) and density \(\boldsymbol{\sigma}_l\) within individual NeRF models.
%
These properties are processed through our composition module to infer the global color \(\boldsymbol{C}_g\) and density \(\boldsymbol{\sigma}_g\), crucial for the global rendering.
%
Volume rendering techniques~\cite{kajiya1984ray} are then employed to procure the rendered views for both local and global frames. We propose dual SDS losses to ensure coherence between the image outputs and their corresponding textual descriptions. Additionally, our approach facilitates recomposition by channeling samples from cached models back into local frames along with the text revision, thereby streamlining the integration.

% As shown in \cref{fig:abls}(a), we verify its necessity by dropping $\nabla \mathcal{L}_{\text{SDS}_g}$. 
% %
% Compared with our full model, its layout does not fit our shared sense of a room, \ie, \emph{nightstand} is usually lower than \emph{bed}; \emph{lamp} needs a base to support it. Additionally,  it lacks global consistency, such as light reflection, to make it more realistic. 
% %
% Therefore, we leverage the full text semantics to ensure consistent global rendering across local frames. 
% %
% Instead of conditioning the global rendering view with the full prompt directly, we note that global calibration is necessary for geometry and color to be learned sufficiently.
% For example, we observe that geometric completeness and texture of \emph{nightstand} are not ideal. Although reflection appears around \emph{nightstand}, \emph{bed} is stripped of the light. 
% %
% Therefore, we opt to leverage the correlation between the rendering output of the combined NeRFs and the overall semantics to perform multi-object scene reconstruction.  
%

\noindent\textbf{Global Composition.}
The independent optimization of each local frame may inadvertently result in a lack of global coherence within the scene. To address this, our scene composition process is designed to integrate these frames, thereby achieving a more consistent result.
%
Before exploring the specifics of the module, it is imperative to discuss two critical design decisions within the composition module, as depicted in \cref{fig:framework}.
%
Upon integrating the properties inferred from \(\boldsymbol{x}_g\) into the composition module, they are fine-tuned through gradients derived from the global SDS loss.  This process leads to a critical consideration: the necessity and implications of refining the global density \(\boldsymbol{\sigma}_g\). This can be divided into two approaches: \textbf{1) Density-based:} The advantage of adjusting \(\boldsymbol{\sigma}_g\) is that it can adjust geometry, thus yielding a scene more congruent with the global text prompt. 
However, this comes at the cost of potentially compromising the optimal color \(\boldsymbol{C}_g\), as calibrating \(\boldsymbol{\sigma}_g\) introduces more uncertainty for subsequent color refinement as it requires prior density features $\boldsymbol{h}$ as shown at \cref{fig:compo}. 
\textbf{2) Color-based:} Conversely, directly employing \(\boldsymbol{\sigma}_l\) mitigates this uncertainty but at the expense of reduced geometric control, presenting a challenging balance to strike in the pursuit of precise scene composition.
% , which may lead to suboptimal outcomes.
%
After thorough experiments, exemplified in \cref{fig:abls}, we have opted for the density-based approach to refine \(\boldsymbol{\sigma}_g\)  prioritizing both \textbf{accuracy and efficiency}. The test revealed that it excels in rendering intricate details, such as enhanced wood grain textures and more naturally contoured 'salad', as accentuated by boxes. This method also demonstrated a swifter convergence rate. Conversely, while the color-based improved reflections and reduced flickering on the 'wine cup', it was plagued by issues such as sparse density, which adversely brings holes at the base of the 'cup' and the corner of the 'table'.
Furthermore, upon close examination, it becomes evident that shadow artifacts of 'wine' on the 'table' are pronounced, suggesting that its disadvantages outweigh its advantages.
%  in this context
% \textbf{Global Composition.}
% Each local frame is optimized independently, causing a lack of global connections for scene composition.
% Before delving into module details, there are two choices (see \cref{fig:framework}) on the composition module design we need to elaborate on first. 
% %
% In \cref{fig:framework}, by taking $\boldsymbol{x}_g$ into the composition module, their inferred properties are calibrated with gradients propagated from the global SDS loss. 
% However, it remains unclear whether $\boldsymbol{\sigma}_g$ should be refined or not. 
% %
% The trade-off on its usage is the density adjustment bringing a more reasonable layout and more geometric details that fit the global text prompt. While its potential downside is that $\boldsymbol{C}_g$ may not be optimal as $\boldsymbol{\sigma}_g$ has more uncertainty compared to $\boldsymbol{\sigma}_l$, bringing sub-optimal rendering results. 

% We choose the density-based method after comparing them with the experiment shown in \cref{fig:abls}. 
% %
% Specifically, we test both designs on the scene \emph{table wine} and discover that the density-based design provides more intrinsic details(as indicated by green boxes), \eg, enriched wood grains, and a more natural shape for \emph{salad} and has much faster convergence speed. In contrast, the color-based method enhances the reflection and smooths flickering on \emph{wine cup}, (as indicated by red boxes), but it suffers from 1) sparse density, resulting in poorly generated geometry at the base of  \emph{cup} and the wood \emph{table} corner. Additionally, shadow artifacts appeared on \emph{table} when viewed up close, outweighing benefits of the color-based method.

\begin{figure}[t!]
    \centering
    \includegraphics[width=\linewidth]{figures/compo_module.pdf}
    % \vspace{-24pt}
    % \caption{Ablation study on text guidance. (a) without local SDS losses. (b) without global SDS losses. (c) vanilla SDS losses without perturb and average scoring~\cite{wang2022score}. (d) full model.}
    \caption{\textbf{Detail of Composition module}: density-based design. 
    }
    \label{fig:compo}
    % \vspace{-18pt}
\end{figure}
\noindent\textbf{Network Design.}
The compositional framework of our network, as delineated in \cref{fig:compo}, is predicated on an architecture that employs a suite of MLPs, represented as \(\{\boldsymbol{\theta}_l\}_{l=1}^{m}\),  each dedicated to a distinct local frame. To harmonize \(\boldsymbol{\sigma}_l\) and \(\boldsymbol{C}_l\), we incorporate global MLPs, including density calibrator $f_{\boldsymbol{\theta}_{g_d}}$ and color calibrator $f_{\boldsymbol{\theta}_{g_c}}$.
%
A transformation module complements this system, tasked with maintaining the spatial coherence between the global and local frames. It governs the transformation of sampling points $\boldsymbol{x}$, ray directions $\boldsymbol{d}$, and adjacent sampling distances $\delta$. This module also orders the points $\{\boldsymbol{x}_{g,j}\}_j$ by their distance to the global camera origin $\boldsymbol{o}_g$, ensuring that each local point $\boldsymbol{x}_l$ is accurately matched with its corresponding global point $\boldsymbol{x}_g$ for subsequent volume rendering. 
%
The network design is:
{
\setlength\abovedisplayskip{4.5pt}
\setlength\belowdisplayskip{4.5pt}
\begin{align}
\label{eq:g_c_d}
{\boldsymbol{\sigma}_g}  &= \alpha_d f_{\boldsymbol{\theta}_{g_d}}({\boldsymbol{x}_g}) + \boldsymbol{\sigma}_l, \\
{\boldsymbol{C}_g}  &= \alpha_c f_{\boldsymbol{\theta}_{g_c}}(\boldsymbol{h}, {\boldsymbol{d}_g}) + \boldsymbol{C}_l. 
\end{align}}In contrast to the local frames, the global frame's color output $\boldsymbol{C}_g$ is inferred based on $\boldsymbol{h}$ and conditional on $\boldsymbol{d}_g$ to enable a view-dependent lighting effect.
% Denote the density features as $\boldsymbol{h}$. 
%
%
Residual learning is leveraged here, where \(\boldsymbol{\sigma}_l, \boldsymbol{C}_l\) serve as foundational elements that support the learning of global density \(\boldsymbol{\sigma}_g\) and color \(\boldsymbol{C}_g\). The parameters \(\alpha_d, \alpha_c\) are adjustable, allowing fine-tuning of the influence that local components exert on the global outputs.
%
It is imperative to acknowledge that in our color-based method, density calibration is intentionally excluded to concentrate solely on the refinement of color dynamics as shown at \cref{fig:framework}. This is achieved by conditioning the process on both spatial and directional global inputs \((\boldsymbol{x}_g, \boldsymbol{d}_g)\), as demonstrated in the following equations:
\begin{align}
\setlength\abovedisplayskip{4.5pt}
\setlength\belowdisplayskip{4.5pt}
\label{eq:g_c_c}
\boldsymbol{\sigma}_g = \boldsymbol{\sigma}_l, \quad
{\boldsymbol{C}_g} = \alpha_c f_{\boldsymbol{\theta}_{g_c}}({\boldsymbol{x}_g}, {\boldsymbol{d}_g}) + \boldsymbol{C}_l.
\end{align}
The integration of extra $\boldsymbol{x}_g$ aims to facilitate a fair comparison under same inputs with the density-based. It enhances the visual appeal of effects like the wine cup's reflection, as demonstrated in \cref{fig:abls}. However, this method is not without its compromises. It tends to produce artifacts and is characterized by a slower convergence rate. Additionally, this approach limits the ability to precisely control density, subsequently impacting the intricate geometric details.


\begin{figure*}[t!]
    \centering
    \includegraphics[width=\linewidth]{figures/sota.pdf}
    % \vspace{-24pt}
    \caption{\textbf{Qualitative comparison with other text-to-3D methods using multi-object text prompts}. Cases 1-3 demonstrate simpler settings characterized by compositions involving two objects. In contrast, Cases 4-8 delve into more intricate scenarios featuring compositions with more than two objects. Smaller images are presented to illustrate the generated local NeRFs(partially shown in Cases 4-8).}
    \label{fig:sota}
    % \vspace{-5pt}
\end{figure*}
%
% \begin{table*}[t!]
% \centering
% \resizebox{\textwidth}{!}
% {
% \begin{tabular}{cccccccc}
% \toprule
% Method            & \rotatebox{60}{table wine}  & \rotatebox{60}{teddy monkey} & \rotatebox{60}{computer mouse} & \rotatebox{60}{bed room}  & \rotatebox{60}{chess} & \rotatebox{60}{pisa tower} & \rotatebox{60}{astronaut} & \rotatebox{60}{tesla}  \\ \midrule
% LatentNeRF  & 21.55 & 27.38 & 17.13 & 21.86 & 31.19 & 24.31 & 27.07 & 25.16 \\
% SJC & 23.33 & 27.37 & 18.00 & 22.54 & 30.53 & \textbf{26.18 }& 27.84 & 23.55 \\
% CompoNeRF & \textbf{32.68} & \textbf{28.57}	 &\textbf{ 22.34} &\textbf{ 28.65} & \textbf{31.45} & \textbf{28.96} & 25.82 & 25.95 & 24.42 & \textbf{32.71} & \textbf{26.13 }& \textbf{26.38} & \textbf{30.98} & \textbf{33.37} \\
% \bottomrule
% \end{tabular}
% }
% \vspace{-10pt}
% \caption{Performance of our CompoNeRF in different 3D scenes. We use CLIP score \cite{parmar2023zero,zhang2023sine,wang2023imagen} as our evaluation metric, which is a common evaluation metric in text-to-image generation tasks to evaluate the similarity of the generated image to the text prompt. }
% \label{perclass}
% \end{table*}
%
\begin{table*}[t!]
% \scalebox{0.8}
\renewcommand{\arraystretch}{1.2}
\fontsize{4pt}{4pt}
\selectfont 
\centering
% \vspace{-8pt}
\resizebox{\textwidth}{!}
{
% \begin{tabular}{lcccccccc}
% \hline
% Method     & table\_wine    & tesla          & pyramid        & chess          & apple and banana      & astronaut      & glass\_balls   & Eiffel\_tower    \\ \hline
% LatentNeRF & 21.55          & 25.16          & 27.43          & 31.19          & 27.69          & 27.07          & 29.51          & 26.32          \\
% SJC        & 23.33          & 23.55          & 25.62          & 30.53          & 28.21          & 27.84          & 28.76          &27.41 \\
% \textbf{CompoNeRF(Ours)}     & \textbf{32.68} & \textbf{26.13} & \textbf{28.96} & \textbf{31.45} & \textbf{33.37} & \textbf{32.71} & \textbf{30.98} & \textbf{28.44}          \\ \hline
% \end{tabular}
\begin{tabular}{lcccccccc}
\hline
Method                   & Case 1         & Case 2         & Case 3         & Case 4         & Case 5         & Case 6         & Case 7         & Case 8         \\ 
\hlineB{1.1}
LatentNeRF               & 25.16          & 27.07          & 27.69          & 31.19          & 21.55          & 26.32          & 27.43          & 29.51          \\
SJC                      & 23.55          & 27.84          & 28.21          & 30.53          & 23.33          & 27.41          & 25.62          & 28.76          \\
\textbf{CompoNeRF (Ours)} & \textbf{26.13} & \textbf{32.71} & \textbf{33.37} & \textbf{31.45} & \textbf{36.06} & \textbf{28.44} & \textbf{28.96} & \textbf{30.98} \\ \hlineB{1.1}
\end{tabular}
}

% \vspace{-6pt}
\caption{\textbf{Performance comparison of our CompoNeRF in different 3D scenes}. For our evaluation metric, we utilize the average of CLIP scores~\cite{parmar2023zero,zhang2023sine,wang2023imagen} across different views, which serve to assess the similarity between the generated images and the global text prompt. }
\label{tb:perclass}
\end{table*}
% \cref{fig:framework} depicts the network architecture of the composition module. Denote $m$ as local MLP $\{\boldsymbol{\theta}_l\}_{l=1}^{m}$ for each local frame. Then, we introduce the global MLPs including density $\boldsymbol{\theta}_{g_d}$ and $\boldsymbol{\theta}_{g_c}$ calibrators to refine $\boldsymbol{\sigma}_l$ and $\boldsymbol{C}_l$. 
% %
% In detail, the network design is, 
% {
% % \setlength\abovedisplayskip{4.5pt}
% % \setlength\belowdisplayskip{4.5pt}
% \begin{align}
% \label{eq:g_c_d}
% {\boldsymbol{\sigma}_g}  &= \alpha_d \boldsymbol{\theta}_{g_d}({\boldsymbol{\sigma}_l}) + \boldsymbol{\sigma}_l, \\  
% {\boldsymbol{C}_g}  &= \alpha_c \boldsymbol{\theta}_{g_c}({\boldsymbol{C}_l},  {\boldsymbol{d}_g}) + \boldsymbol{C}_l, 
% \end{align}}
% %
% where residual $\boldsymbol{\sigma}_l, \boldsymbol{C}_l$ assist in learning $\boldsymbol{\sigma}_g$ and $\boldsymbol{C}_g$, while $\alpha_d, \alpha_c$ balance their contribution as learnable parameters.
% %
% Note that the color-based omits density calibration, and simply uses the shared color refinement.



% The 3D boxes are only used for the spatial configuration of local NeRFs, while the implicit representation of local NeRFs is inferred by the canonical samples inside the local frame without considering the global relationship across different objects.
% To relieve such location-dependent effects, we further calibrate the output color and density from the local NeRF with global coordinates $({\boldsymbol{x}}_g, {\boldsymbol{y}}_g, {\boldsymbol{z}}_g)$ and ray directions $\left({\boldsymbol{\phi}}_{d}, {\boldsymbol{\theta}}_{d}\right)$ as the conditional input.
% % to inject the global visual clues.
% %
% %
% Specifically, we adopt a shared MLP $\boldsymbol{\theta}_{g}$ to calibrate all the predicted object colors, that is,
% {\setlength\abovedisplayskip{4.5pt}
% \setlength\belowdisplayskip{4.5pt}
% \begin{align}
% \label{eq:MLP_dyn_2}
% {\boldsymbol{C}_g} = {\boldsymbol{C}_l} + \boldsymbol{emb}_{g} &= {\boldsymbol{C}_l} + \boldsymbol{\theta}_{g}({\boldsymbol{x}}_g, {\boldsymbol{y}}_g, {\boldsymbol{z}}_g, {\boldsymbol{\phi}}_{d}, {\boldsymbol{\theta}}_{d}),
% \end{align}}
% where ${\boldsymbol{C}_l}$ is the color predicted by the local NeRF.
% Therefore, the scene color can preserve the view-consistent behavior from the original architecture and add consistency across poses for the volumetric density.
% Since the color and density values share the same latent expression in $({\boldsymbol{x}}_l, {\boldsymbol{y}}_l, {\boldsymbol{z}}_l)$, we only calibrate the emitted scene color explicitly with the scene location, as the densities of local NeRFs also are implicitly adjusted during optimization.

% \noindent \textbf{Global and Local Volumetric Rendering.}
% After compositing all the interacted points, each ray $\boldsymbol{r}_i$ collects a set sampling points by $\{\boldsymbol{t}_{i,j,n} \}_{j=1, n=1}^{m_j, N}$, where $m_j$ is the number of the hit object.
% For each sampling point, the inference results with the respective 3D representations are the local color $\boldsymbol{c}_{l}$, global color $\boldsymbol{c}_{g}$, and density $\sigma$.

% In fact, the local view $\hat{C}_{l,j}$ of single object $j$ also can be rendered by the sampled points  belongs to the same local frames as shown at Fig.~\ref{fig:framework}.

\subsubsection{Recomposition}
Our architecture advances scene reconstruction by providing an intuitive interface for layout manipulation.  This capability is crucial for the reconfiguration of scene elements into novel scenes, as depicted in \cref{fig:framework}. Here, the input panel allows for adjustments in the attributes of bounding boxes, such as modifying the position and scale of the 'apple' bounding box prior to composition. The refinement process further involves sampling ray-box intervals from the global frame, leading to transformed coordinates with the corresponding ray samples that are then incorporated into the pipeline, as demonstrated in \cref{fig:compo}.
%
Each bounding box represents an individual NeRF, providing the flexibility to move, scale, or remove elements as needed. CompoNeRF's capabilities also extend to textual edits, exemplified by the transformation of 'wine' into 'juice'.
%
Since NeRFs have been well trained, we only finetune \(\theta_g, \theta_l\) to align text prompts to promote consistency of both local and global views.
%
Moreover, the NeRFs once retrained within the edited scene, are also structured to be decomposable and cacheable in future scene compositions.
% Our CompoNeRF architecture facilitates the seamless reconstruction of scenes leveraging existing models. It enables precise editing of bounding boxes parameterized by \(\{\boldsymbol{\theta}_l\}_{l=1}^{m}\), allowing for their reconfiguration into new layouts. Refer to \cref{fig:framework}, the input panel permits the modification of attributes such as the position and scale of the 'apple' node's bounding box prior to composition. The process is further refined by sampling from the updated ray-box intervals within the global frame, which are then projected onto \(\boldsymbol{x}_l\), ensuring a streamlined reconstruction that integrates the 'apple' effectively. This addition is executed with careful attention to color consistency, positioning the 'apple' adjacent to the 'French bread' to complement the scene's overall palette. Each bounding box represents an individual NeRF, which means they can be manipulated through moving, scaling, and removal operations. CompoNeRF also extends its editing prowess to textual modifications, as evidenced by the 'wine cup' now appearing filled with juice—a change propagated through both subtexts and the global test. 
% %
% Since NeRFs have been well trained, we only finetune $\theta_g, \theta_l$ to align text prompts to promote consistency of both local and global views . 
% %
% Moreover, the NeRFs, once retrained within the reimagined scene, are also structured to be decomposable and cacheable for subsequent scene compositions.

% , as shown in Fig.~\ref{fig:framework}.
% For each scene described by the multi-object text prompt $T$, we
% To enhance the guidance of local representations, we use the local text prompt $T_l \subseteq T$ of a single object to optimize the local NeRFs in local views.
% The scene views $\hat{\boldsymbol{X}}_g=\{\hat{\boldsymbol{C}}_{g,i}\}_{i=1}^{H\times W}$ is obtained from the predicted pixel values of $H \times W$ rays by compositing all the ray-box interaction values.
% Similarly, the rendered view $\hat{\boldsymbol{X}}_{l,j}$ of the local frame $\boldsymbol{\theta}_j$ without compositing other objects can be calculated by $\hat{\boldsymbol{C}}_{l,j}$, as depicted in Sec.~\ref{ssec:render}.
% We use the local color instead of the globally calibrated color to obtain a local view because the local NeRF should learn the object identity unrelated to its placed position, as the position can be different during user edition.
% % Compared to cropping the local region from a global view for training, separate rendering can avoid the undesired information from other objects brought by the occlusion and resolution adjustments.
% Formally, we employ the following loss as the learning objective,
\begin{figure*}[t!]
    \centering
    \includegraphics[width=\linewidth]{figures/editing.pdf}
    % \vspace{-23pt}
    \caption{\textbf{Scene Editing Outcome:} Demonstrated here are the stages of our recomposition, utilizing cached source scenes. Each NeRF is individually identified by colorful labels. These decomposed nodes are then positioned in the initial layout and subsequently calibrated to form the final composition. The detailed description of the ambient environment is underscored, enhancing the scene's realism.}
    \label{fig:app}
    % \vspace{-12pt}
\end{figure*} 
\subsubsection{Optimization}
\label{sec:optimization}
During optimization, our method employs dual text guidance to align rendering results with both global and local textual descriptions. The optimization objective is:
{
\small
\setlength\abovedisplayskip{2pt}
\setlength\belowdisplayskip{2pt}
\begin{equation}
\label{eqn:loss_f}
\mathcal{L}= {\alpha_g}\nabla\mathcal{L}_{\text{SDS}}(\hat{\boldsymbol{X}}_{g}, T) + {\alpha_l}\sum_{j=1}^{m} \nabla\mathcal{L}_{\text{SDS}}(\hat{\boldsymbol{X}}_{l,j}, T_{l,j}) + \beta\mathcal{L}_{\text{sparse}},\nonumber
\end{equation}
}where $T$ signifies the global text prompt, while $T_{l}$ pertains to a specific object within the global context. The hyperparameters $\alpha_{g}, \alpha_{l}$, and $\beta$ modulate the respective loss weights. 
% $\nabla \mathcal{L}_{\text{SDS}}$ is the score distillation sampling loss, as described in Sec.~\ref{sec:background}.
As suggested in~\cite{metzer2022latent}, we use $L_{\text{sparse}}$ included to penalize the binary entropy of local NeRFs' densities, thereby mitigating the issue of extraneous floating radiance.
Additionally, incorporating directional cues such as "front view" or "side view" into the input text, as suggested by \cite{poole2022dreamfusion,metzer2022latent} proves beneficial in specifying camera poses during the training phase, further enhancing the alignment of our generated scenes with the intended perspectives.
% Note that the global calibration in the scene frame can adaptively revise both $({C}_l, {\sigma})$ in local NeRF with $\nabla \mathcal{L}_{SDS}$ along with the back-propagating gradient.



% \clearpage
\section{Method}
\label{sec:method}
In the following, we describe our approach and analysis of contrastive learning on long-tailed data. For this, we will first review the core principles of contrastive learning for the case of uniform data (\cref{subsec:contrastive_learning}). In \cref{subsec:max_margin}, we then place a particular focus on the temperature parameter $\tau$ in the contrastive loss and its impact on the learnt representations. 
Based on our analysis, in \cref{subsec:CL_on_LT} we discuss how the choice of $\tau$ might negatively affect the learnt representation of rare classes in the case of long-tailed distributions. Following this, we describe a simple proof-of-concept based on additional coarse supervision to test our hypothesis. We then further develop 
\underline{t}emperature \underline{s}chedules (TS)
that yield significant gains with respect to the separability of the learnt representations in \cref{sec:results}.


\subsection{Contrastive Learning}
\label{subsec:contrastive_learning}
% \textbf{Definitions}
% Anchor, positive, negative, hard-negatives, easy-negatives, tolerance

\textbf{The Info-NCE loss} is a popular objective for contrastive learning (CL) and has lead to impressive results for learning useful representations from unlabelled data \citep{oord2018representation, wu2018unsupervised, he2019moco, chen2020simple}. Given a set of inputs $\{x_1, \dots, x_N\}$, and the cosine similarities $s_{ij}$ between learnt representations $u_i\myeq f(\mathcal A(x_i))$ and $v_j\myeq g(\mathcal A (x_j))$ of the inputs, the loss is defined by:
\begin{align}
    \mathcal L_\text{c} = \sum_{i = 1}^N-\log\dfrac{\exp\left(s_{ii} / \tau\right)}{\exp\left(s_{ii} / \tau\right) + \sum_{j\neq i} \exp\left(s_{ij} / \tau\right)}.
    \label{eq:info-nce}
\end{align}
Here, 
 $\mathcal A(\cdot)$ applies a random augmentation to its input and $f$ and $g$ are deep neural networks. For a given $x_i$, we will refer to $u_i$ as the \emph{anchor} and to $v_j$ as a \emph{positive} sample if $i\myeq j$ and as a \emph{negative} if $i\myneq j$.
Last, $\tau$ denotes the \emph{temperature} of the Info-NCE loss and has been found to crucially impact the learnt representations of the model~\citep{wang2020understanding,wang2021understanding,robinson2021can}.


\begin{figure}[!t]
\begin{center}
%\framebox[4.0in]{$\;$}
\includegraphics[width=\textwidth]{iclr2023/evolution4.pdf}
\end{center}
\vspace{-1em}
\caption{\textbf{Coverage of the embedding space during training.} To measure coverage we uniformly sample $500$ bins on the unit sphere. Each training sample is assigned to the closest bin and we plot a histogram of the assignments. X-axis: bins. Y-axis: number of training samples in a bin. Colors denotes epochs: light is the 1st epoch of training, dark is the last. For small $\tau$ (a) the representations are more uniformly distributed (cf.~\cref{sec:method}). 
}
\label{fig:tau_distances}
\end{figure}

\myparagraph{Uniformity.} Specifically, a small $\tau$ has been tied to more uniformly distributed representations, see \cref{fig:tau_distances}. For example, \cite{wang2021understanding} show that the loss is `hardness-aware', \ie negative samples closest to the anchor receive the highest gradient. In particular, for a given anchor, the gradient with respect to the negative sample $v_j$ is scaled by its relative contribution to the denominator in \cref{eq:info-nce}:
\begin{align}
    \frac{\partial \mathcal L_c}{\partial v_j} = \frac{\partial \mathcal L_c}{\partial s_{ij}} \times \frac{\partial s_{ij}}{\partial v_j}
    = \frac1\tau\times [\text{softmax}_{k}(s_{ik}/\tau)]_j \times \frac{\partial s_{ij}}{ \partial v_j}\quad .
    \label{eq:gradient}
\end{align}
As a result, for sufficiently small $\tau$, the model minimises the cosine similarity to the nearest negatives in the embedding space, as softmax approaches an indicator function that selects the largest gradient. The optimum of this objective, in turn, is to distribute the embeddings as uniformly as possible over the sphere, as this reduces the average similarity between nearest neighbours, see also \cref{fig:tau_distances,,fig:spheres}. 

\myparagraph{Semantic structure.} 
In contrast, a large $\tau$ has been observed to induce more semantic structure in the representation space.
However, while the effect of small $\tau$ has an intuitive explanation, the phenomenon that larger $\tau$ induce semantic structure is much more poorly understood and has mostly been described empirically~\citep{wang2021understanding,robinson2021can}. Specifically, note that for any given positive sample, all negatives are repelled from the anchor, with close-by samples receiving exponentially higher gradients. Nonetheless, for large $\tau$, tightly packed semantic clusters emerge. However, if close-by negatives are heavily repelled, how can this be? Should the loss not be dominated by the hard-negative samples and thus break the semantic structure?
%, see \citet{wang2021understanding}?   --- CR: I think we cite them enough in this part

To better understand both phenomena, we propose to view the contrastive loss through the lens of \emph{average distance} maximisation, which we describe in the following section.


\subsection{Contrastive learning as average distance maximisation}
\label{subsec:max_margin}
As discussed in the previous section, the parameter $\tau$ plays a crucial role in shaping the learning dynamics of contrastive learning. To understand this role better, in this section, we present a novel viewpoint on the mechanics of the contrastive loss that explain the observed model behaviour. In particular, and in contrast to \cite{wang2021understanding} who focused on the impact of \emph{individual} negatives, for this we discuss the \emph{cumulative} impact that all negative samples have on the loss.

To do so, we express the %individual
summands $\mathcal L_c^i$ of the loss in terms of distances $d_{ij}$ instead of similarities $s_{ij}$:
%
\begin{equation}
    0 \;\leq\; d_{ij} \;=\; \frac{1-s_{ij}}{\tau} \;\leq\; \frac2\tau\quad \text{and}\quad c_{ii} = \exp(d_{ii}).
    \label{eq:margin_def}
\end{equation}
This allows us to rewrite the loss $\mathcal L_c^i$ as
\begin{equation}
    \label{eq:margin_loss}
    \mathcal L^i_\text{c} 
    = -\log\left(\frac{\exp\left(-d_{ii}\right)}{\exp\left(-d_{ii}\right)+ \sum_{j\neq i}\exp\left(-d_{ij}\right)}\right)
    = \log\left(1+c_{ii}{\sum_{j\neq i}\exp\left(-d_{ij}\right)}\right) \, .
\end{equation}
%
%
As the effect $c_{ii}$ of the positive sample for a given anchor is the same for all negatives, in the following we place a particular focus on %the role of 
the negatives and their relative influence on the loss in \cref{eq:margin_loss}; for a discussion of the influence of positive samples, please see \cref{appendix:positives}.


%
To understand the impact of the temperature $\tau$, first note that the loss monotonically increases with the sum $S_i=\sum_{j\neq i}\exp(-d_{ij})$ of exponential distances in \cref{eq:margin_loss}.
As $\log$ is a continuous, monotonic function, % alternatively: diminishing returns?
we base the following discussion on the impact of $\tau$ on the sum $S_i$.
%As such, the contrastive objective can be understood as a margin maximisation 


\begin{figure}[!t]
\begin{center}
%\framebox[4.0in]{$\;$}
\includegraphics[scale=0.1125]{iclr2023/fig1.pdf}
\end{center}
\caption{\textbf{Loss contribution by similarity.} X-axis: cosine similarity between anchor and negative. All curves are normalised such that their max y-value is 1. %For more results see supplement. 
\textbf{a)}: influence of an individual negative sample to the loss depending on its similarity to anchor for different $\tau$; \textbf{b)}: average histogram of distribution of negatives over the hypersphere with respect to their similarity to the anchor; \textbf{c)}: cumulative impact that negative samples have on the loss. The \emph{cumulative} contribution of negatives shifts left, towards less similar samples, in contrast to individual contributions of negatives. As $\tau\rightarrow\infty$, the cumulative distribution coincides with the histogram b). 
% 
}
\vspace{-1.5em}
\label{fig:negative_impact}
\end{figure}



\myparagraph{For small $\tau$}, the nearest neighbours of the anchor point dominate $S_i$, as differences in similarity are amplified. As a result, the contrastive objective maximises the average distance to nearest neighbours, leading to a uniform distribution over the hypersphere, see \cref{fig:spheres}. Since individual negatives dominate the loss, this argument is consistent with existing interpretations, \eg \cite{wang2021understanding}, as described in the previous section. 

\myparagraph{For large $\tau$}, (\eg $\tau\geq1$), on the other hand, the contributions to the loss from a given negative are on the same order of magnitude for a wide range of cosine similarities.  
Hence, the constrastive objective can be thought of as maximising the average distance over a wider range of neighbours.
Interestingly, since distant negatives will typically outnumber close negatives, the strongest \emph{cumulative} contribution to the contrastive loss will come from more distant samples, despite the fact that \emph{individually} the strongest contributions will come from the closest samples. To visualise this, in \cref{fig:negative_impact}a, we plot the contributions of \emph{individual} samples depending on their distance, as well as the distribution of similarities $s_{ij}$ to negatives over the entire dataset in \cref{fig:negative_impact}b. Since the number of negatives at larger distances (\eg $s_{ij}\approx 0.1$) significantly outnumber close negatives ($s_{ij}>0.9$), the peak of the cumulative contributions%
\footnote{To obtain the cumulative contributions, we group the negatives into 100 non-overlapping bins of size 0.02 depending on their distance to the anchor and report the sum of contributions of a given bin.}
shifts towards lower similarities for larger $\tau$, as can be seen in \cref{fig:negative_impact}c; in fact, for $\tau\myrightarrow\infty$, the distribution of cumulative contributions approaches the distribution of negatives.


Hence, the model can significantly decrease the loss by increasing the distance to relatively `easy negatives' for much longer during training, \ie to samples that are easily distinguishable from the anchor by simple patterns. Instead of learning `hard' features that allow for better \emph{instance discrimination} between hard negatives, the model will be biased to learn easy patterns that allow for \emph{group-wise discrimination} and thereby increase the margin between clusters of samples. Note that since the clusters as a whole mutually repel each other, the model is optimised to find a trade-off between the expanding forces between hard negatives (\ie within a cluster) and the compressing forces that arise due to the margin maximisation between easy negatives (\ie between clusters).

Importantly, such a bias towards easy features can prevent the models from learning hard features---\ie by focusing on \emph{group-wise discrimination}, the model becomes agnostic to instance-specific features that would allow for a better \emph{instance discrimination} (cf.~\cite{robinson2021can}). In the following, we discuss how this might negatively impact rare classes in long-tailed distributions.


\subsection{Temperature schedules for contrastive learning on long-tail data}
\label{subsec:CL_on_LT}
% In the last section, 
% 
As discussed in \cref{sec:intro}, naturally occurring data typically exhibit long-tail distributions, with some classes occurring much more frequently than others; across the dataset, \textit{head} classes appear frequently, whereas \textit{tail} classes contain fewest number of samples. Since self-supervised learning methods are designed to learn representations from unlabelled data, it is important to investigate their performance on imbalanced datasets.


\begin{figure}[!t]
\begin{center}
%\framebox[4.0in]{$\;$}
\includegraphics[scale=0.35]{iclr2023/spheres2.pdf}
\end{center}
\caption{%\textbf{Head and tail classes in embedding space.} 
\rebuttal{\textbf{Representations of a head and a tail class.} }
Visualisation of the influence of $\tau$ on representations of two semantically close classes (trained with all 10 classes). Red: \headclass{single head class} and blue: \tailclass{single tail class} from CIFAR10-LT. Small $\tau\myeq0.1$ promotes uniformity, while large $\tau\myeq1.0$ creates dense clusters. 
With $\tau_{\{head/tail\}}$ we refer to coarse supervision described in \cref{subsec:CL_on_LT} which separates tail from head classes. 
% 
In black / \headclass{red} / \tailclass{blue}, we respectively show the average kNN accuracy over all classes / \headclass{the head class} / \tailclass{the tail class}.%,
}
\label{fig:spheres}
\end{figure}

\myparagraph{Claim: Tail classes benefit from instance discrimination.}
As discussed in \cref{subsec:max_margin}, sufficiently large $\tau$ are required for semantic groups to emerge during contrastive learning as this emphasises group-wise discrimination. However, as shown by \cite{robinson2021can}, this can come at the cost of encoding instance-specific features and thus hurt the models' instance discrimination capabilities. 

We hypothesise that this disproportionately affects tail classes, as tail classes consist of only relatively few instances to begin with. Their representations should thus \emph{remain distinguishable} from most of their neighbours and not be grouped with other instances, which are likely of a different class. In contrast, since head classes are represented by many samples, grouping those will be advantageous.

To test this hypothesis, we propose to explicitly train head and tail classes with different $\tau$,
to emphasise group discrimination for the former while ensuring instance discrimination for the latter.

% 
\myparagraph{Experiment: Controlling $\tau$ with coarse supervision.}
We experiment on CIFAR10-LT (a long-tail variant of CIFAR10 - see \cref{subsec:implementation}) in which we select a different $\tau$ depending on whether the anchor $u_i$ is from a head or a tail class, \ie of the 5 \emph{most} or \emph{least} common classes. We chose a relatively large $\tau$ ($\tau_\text{head}\myeq1.0$) for the 5 head classes to emphasise group-wise discrimination and a relatively small $\tau$ ($\tau_\text{tail}\myeq0.1$) for the 5 tail classes to encourage the model to learn instance-discriminating features.

As can be seen in \cref{fig:spheres}, this simple manipulation of the contrastive loss indeed provides a significant benefit with respect to the semantic structure of the embedding space, despite only weakly supervising the learning by adjusting $\tau$ according to a coarse (frequent/infrequent) measure of class frequency. 

\rebuttal{In particular, in \cref{fig:spheres}, we show the projections of a single head class and a single tail class onto the three leading PCA dimensions and the corresponding kNN accuracies. %(red/blue) for those two classes, as well as the kNN performance averaged over \emph{all} classes (black). 
We would like to highlight the following results.} First, without any supervision, we indeed find that the head class consistently performs better for larger values of $\tau$ (\eg $1.0$), whereas the tail class consistently benefits from smaller values for $\tau$ (\eg $0.1$). Second, when training the model according to the coarse $\tau$ supervision as described above, we are not only able to maintain the benefits of large $\tau$ values for the head class, but significantly outperform all constant $\tau$ versions for the tail class, which improves the overall model performance on all classes; detailed results for all classes are provided in the appendix.

\myparagraph{\underline{T}emperature \underline{S}chedules (TS) without supervision.}
%
Such supervision with respect to the class frequency is, of course, generally not available when training on unlabelled data and these experiments are only designed to test the above claim and provide an intuition about the learning dynamics on long-tail data. However, we would like to point out that the supervision in these experiments is very coarse and only separates the unlabelled data into \emph{frequent} and \emph{infrequent} classes. Nonetheless, while the results are encouraging, they are, of course, based on additional, albeit coarse, labels.
Therefore, in what follows, we present an unsupervised method that yields similar benefits.

% \myparagraph{\underline Unsupervised $\tau$ \underline Adjustment (U$\tau$A).} 
In detail, we propose to modify $\tau$ according to a cosine schedule, such that it alternates between an upper ($\tau_{+}$) and a lower ($\tau_{-}$) bound  at a fixed period length $T$:
\begin{align}
    \label{eq:cos_tau}
    \tau_{\cos}(t) &= (\tau_+-\tau_-) \times (1+\cos(2\pi\, t/T)) / 2 + \tau_-\; ;
    % 
\end{align}
here, $t$ denotes training epochs. This method is motivated by the observation that $\tau$ controls the trade-off between learning easily separable features and learning instance-specific features.
%

Arguably, however, the models should learn both types of features: \ie the representation space should be structured according to easily separable features that (optimally) represent semantically meaningful group-wise patterns, whilst still allowing for instance discrimination within those groups.

Therefore, we propose to \emph{alternate} between both objectives as in ~\cref{eq:cos_tau}, to ensure that throughout training the model learns to encode instance-specific patterns, whilst also structuring the representation space along semantically meaningful features.
Note that while we find a cosine schedule to work best and to be robust with respect to the choice for $T$ (\cref{subsec:ablations}), we also evaluate alternatives. Even randomly sampling $\tau$ from the interval $[\tau_-,\tau_+]$ improves the model performance.
This indicates that the \emph{task switching} between group-wise discrimination (large $\tau$) and instance discrimination (small $\tau$) is indeed the driving factor behind the performance improvements we observe.
% 

% %%%%%%%%%%%%%%%%%%%%%%%%%%%%%%%%%%%%%%%%%%%%%%%
%%%%%%%        4. Results         %%%%%%%
%%%%%%%%%%%%%%%%%%%%%%%%%%%%%%%%%%%%%%%%%%%%%%%

\section{Results}
\label{sec:results}

\subsection{MOS prediction results}
\label{subsec:mos_results}
We first evaluate our MOS-prediction performance in comparison with other approaches. In particular, we compare against NISQA~\cite{mittag2019non}, which we modified to estimate human-accessed MOS. Originally, they estimate perceptual objective listening quality assessment (POLQA)~\cite{beerends2013perceptual} scores using a CNN and BLSTM architecture. We also compare against the PMOS model proposed in~\cite{dong2020pyramid}, which is identical in structure to our PMOS model. Finally, we include our proposed SE+PMOS approach~\cite{nayem2021incorporating} (no joint training), where our PMOS model is held fixed while the SE model is training using the embeddings from the PMOS encoder. 

We use four metrics to evaluate MOS-estimation performance: mean absolute error (MAE), epsilon insensitive root mean squared error (RMSE)~\cite{rec2012p}, Pearson’s correlation coefficient $\gamma$ (PCC), and Spearman’s rank correlation coefficient $\rho$ (SRCC). 

%    Later, both models are jointly-trained for fine tuning. Our proposed PMOS model is similar of \cite{nayem2021incorporating}, however, SE models are different in structure.

%%%%%%%%%%%%%%%%%%%%%%%%%%%%%%%%%%%%%%%%%%%%%%%%%%%%
% Table 1, MOS results
%%%%%%%%%%%%%%%%%%%%%%%%%%%%%%%%%%%%%%%%%%%%%%%%%%%%
\begin{table}[t!]

\centering
\caption{Performance comparison with MOS prediction models {comparing against the ground truth MOS obtained from human subjects}. Best results are shown in \textbf{bold}.}
\label{tab:mos_results}
% \vspace{-0.5em}
\resizebox{\columnwidth}{!}{%
\begin{tabular}{| l | c c c c | }
\cline{2-5}
   \multicolumn{1}{c|}{}         & {MAE}$\downarrow$ & {RMSE}$\downarrow$ & {PCC ($\gamma$)}$\downarrow$ & {SRCC ($\rho$)}$\downarrow$ \\ \hline
   
NISQA~\cite{mittag2019non}    & 0.62 ($\pm$0.18)        & 0.7 ($\pm$0.16)      & 0.71 ($\pm$0.14)           & 0.79 ($\pm$0.15)            \\
PMOS~\cite{dong2020pyramid}                      & 0.51 ($\pm$0.15)         & 0.57 ($\pm$0.12)          & 0.88 ($\pm$0.17)           & 0.88 ($\pm$0.14)           \\
SE+PMOS~\cite{nayem2021incorporating}                     & \textbf{0.45} ($\pm$0.08) & \textbf{0.52} ($\pm$0.09) & \textbf{0.9} ($\pm$0.12) & \textbf{0.91} ($\pm$0.1)           \\
Proposed                     & \textbf{0.45} ($\pm$0.08) & \textbf{0.52} ($\pm$0.09) & \textbf{0.9} ($\pm$0.12) & \textbf{0.91} ($\pm$0.1)         \\
\hline
\end{tabular}
}
% \vspace{-2em}
\end{table}

Table~\ref{tab:mos_results} shows the results, where our proposed approach and SE+PMOS clearly outperform the other MOS prediction models according to all metrics. MAE is minimized by $0.6$ compared to the original PMOS~\cite{dong2020pyramid} approach. There is also a $0.05$ reduction in RMSE. This justifies our proposed approach that combines MOS estimation and speech enhancement tasks. Note, however, that similar results are obtained for our proposed approach and the SE+PMOS approach, which suggests that joint training (e.g., fine tuning) may help speech enhancement more than MOS prediction.  




\subsection{Speech enhancement model}
\label{subsec:se_results}
%%%%%%%%%%%%%%%%%%%%%%%%%%%%%%%%%%%%%%%%%%%%%%%%%%%%
% Table 2, SE comparison results on COSINE & VOiCES
%%%%%%%%%%%%%%%%%%%%%%%%%%%%%%%%%%%%%%%%%%%%%%%%%%%%

% Please add the following required packages to your document preamble:
% \usepackage{multirow}
% \usepackage[table,xcdraw]{xcolor}
% If you use beamer only pass "xcolor=table" option, i.e. \documentclass[xcolor=table]{beamer}
\begin{table*}[t!]
\centering
\caption{Average results of the speech enhancement models in different performance metrics. Best results are shown in \textbf{bold}.}
\label{tab:results_cosineVoices}
\resizebox{\linewidth}{!}{%
\begin{tabular}{ | l | l | c c c c | c c c c | }
\cline{3-10}
\multicolumn{1}{l}{\multirow{2}{*}{}} &                    & \multicolumn{4}{ c |}{{COSINE}}                             & \multicolumn{4}{ c |}{{VOiCES}} 
\\ \hline
\multicolumn{1}{|l|}{{models}}                 & \multicolumn{1}{c|}{{loss func.}} & {PESQ}$\uparrow$ & {SI-SDR}$\uparrow$ & {ESTOI}$\uparrow$ & {MOS-LQO}$\uparrow$ & {PESQ}$\uparrow$ & {SI-SDR}$\uparrow$ & {ESTOI}$\uparrow$ & {MOS-LQO}$\uparrow$ \\ \hline
\multicolumn{1}{|l|}{{Mixture}}                                        & {-}                                                       & {1.46} & {0.53}   & {0.62}  & {4.04}    & {1.26} & {-1.3}   & {0.48}  & {2.74}    \\ \hline
\multicolumn{1}{|l|}{}                                                        & mse                                                              & 2.68          & 2.8             & 0.8            & 3.2              & 2.3           & 1.2             & 0.69           & 3.5              \\ 
\multicolumn{1}{|l|}{}                                                        & mos~\cite{fu2019learning}                                                              & 2.8           & 3.8             & 0.82           & 4.2              & 2.37          & 1.66            & 0.74           & 5.3              \\ 
\multicolumn{1}{|l|}{}                                                        & mse+sa                                                           & 2.72          & 3.1             & 0.82           & 4                & 2.35          & 1.6             & 0.7            & 3.8              \\ 
\multicolumn{1}{|l|}{}                                                        & mos+sa                                                           & 2.89          & 4.1             & 0.85           & 4.4              & 2.42          & 1.72            & 0.77           & 5.7              \\ 
\multicolumn{1}{|l|}{\multirow{-5}{*}{SE}}                                    & sdr~\cite{kawanaka2020stable}                                                              & 2.7           & 4.5             & 0.82           & 3.4                & 2.32          & 2.01            & 0.72           & 3              \\ \hline
\multicolumn{1}{|l|}{{ }}                                 & mse                                                              & 3.1           & 4               & 0.85           & 4.2              & 2.48          & 1.8             & 0.8            & 6                \\ 
\multicolumn{1}{|l|}{{}}                                 & mse+sa                                                           & 3.19          & 4.6             & 0.93           & 4.8              & 2.54          & 2.08            & 0.86           & 6.3              \\  
\multicolumn{1}{|l|}{\multirow{-3}{*}{SE+PMOS~\cite{nayem2021incorporating}}}        & mse+sa+mos                                                       & 3.19          & 4.5             & 0.92           & \textbf{5.1}     & 2.53          & 2.06            & 0.84           & \textbf{6.5}     \\ \hline
\multicolumn{1}{|l|}{}                                                        & pesq                                                             & \textbf{3.28} & 4.4             & 0.9            & 5                & \textbf{2.67} & 2.01            & 0.83           & 6.1              \\ 
\multicolumn{1}{|l|}{\multirow{-2}{*}{MetricGAN~\cite{fu2019metricGAN}} }                             & stoi                                                             & 3.19          & 4.3             & \textbf{0.94}  & 4.8              & 2.5           & 2               & \textbf{0.87}  & 5.8              \\ \hline
\multicolumn{1}{|l|}{SSEMS~\cite{zezario2019specialized}}                                                   & qnet ($\phi=0dB$)                                                       & 2.85          & 2.99            & 0.83           & 3                & 2.4           & 1.8             & 0.7            & 2.8              \\ \hline
\multicolumn{1}{|l|}{{Chi++\textsubscript{fQSM,bS}~\cite{nayem2021towards}}}                     &    dc+cls+sa                                                              & 2.9           & 3.3             & 0.84           & 3.4              & 2.44          & 1.78            & 0.7            & 3                \\ \hline
\multicolumn{1}{|l|}{}                                 & mse+sa                                                           & 3.25          & 4.8             & \textbf{0.94}  & 4.75             & 2.64          & 2.1             & \textbf{0.87}  & 6.2              \\ 
\multicolumn{1}{|l|}{\multirow{-2}{*}{Proposed}} & mse+sa+mos                                                       & 3.25          & \textbf{4.82}   & \textbf{0.94}  & 5.04             & 2.64          & \textbf{2.13}   & \textbf{0.87}  & 6.47             \\ \hline
\end{tabular}
}
\end{table*}
For speech enhancement, we compare against a baseline approach without an attention mechanism \cite{graves2013speech}. We denote this baseline approach as SE. Five separate loss functions are applied to optimize this approach, and they are MSE, MSE plus signal approximation, MOS, signal approximation with MOS, and SDR. To compute the MOS loss function, we utilize the SE loss function from \cite{fu2019learning} which leverages objective-MOS (oMOS) ratings learned from a speech assessment model~\cite{fu2018quality}. SDR~\cite{kawanaka2020stable} loss functions are proposed in literature previously with different enhancement architectures. For the SDR loss function, the SE model is optimized using the following cost function:
\begin{align}
    \mathcal{L}_{SDR} = \sum_{n=1}^N \mathcal{K}_{\theta}  \Big( 10 \log \frac{\Vert s^n\Vert^2}{\Vert s^n-\hat{s}^n\Vert^2} \Big)
\end{align}
where $\mathcal{K}_\theta(a)=\theta\cdot \tanh(\frac{a}{\theta})$, $\theta$ is a clipping parameter, $N$ is the mini-batch size, and $s^n$ and $\hat{s}^n$ are the n\textsuperscript{th} sample of the clean and estimated speech signal in time. We use $\theta=20$ in our training. We also compare against a generative adversarial network (GAN) approach that individually optimizes with PESQ and STOI~\cite{fu2019metricGAN}. We denote this model as MetricGAN. 
% They estimate the IRM conditioned on continuous space of the discriminator label based on either PESQ or STOI target label. 
They estimate the IRM for a speech mixture conditioned on a GAN discriminator that outputs evaluation scores in continuous space (i.e. scores between 0 and 1) based on either normalized PESQ or STOI target metrics. 
We compare our model with the ensemble-based Specialized Speech Enhancement Model Selection (SSEMS) approach~\cite{zezario2019specialized} that uses Quality-Net~\cite{fu2018quality} as its objective function in a black-box manner. Quality-Net is an oMOS approach that estimates the Perceptual Evaluation of Speech Quality (PESQ) score. The SSEMS approach uses an ensemble of enhancement models, each trained on audio at specific SNRs and speaker genders. During inference, it selects the output with the highest PESQ score. SSEMS uses a SNR threshold of $20$ dB, while we use a threshold of $0$ dB for balanced training and better performance. Additionally, we conduct a comparison with our initial approach that integrates MOS embeddings in speech enhancement, as presented in \cite{nayem2021incorporating}. This model is referred to as SE+PMOS, and it does not involve joint training or the QSM language model. We evaluate SE+PMOS with varying combinations of loss functions. %We compare against a quantized speech enhancement model which utilizes a spectral language model~\cite{nayem2021towards}. This model is motivated from chimera++~\cite{wang2018alternative} in structure with BLSTM layers and deep clustering (dc) loss.
%Traditional chimera++ model estimates a phase-sensitive mask which has been applied in the task of speech enhancement in non-speech noisy conditions with multi-talker speech~\cite{wichern2019wham, yang2019improved}. However, in \cite{nayem2021incorporating}, they estimate quantized speech signal, not mask; they use cross-entropy classification (cls) loss, and signal approximation loss altogether. They report best results using per-frequency quantized spectral model (fQSM) as language model for beam search (bS) with beam size $100$. We use this model as our comparison model denoting as Chi++\textsubscript{fQSM,bS}. 
All models are trained using the experimental setup that is previously mentioned. We modify the comparison models using the code provided by the original authors.

We assess speech enhancement performance using PESQ~\cite{rix2001perceptual}, scale-invariant SDR (SI-SDR)~\cite{le2019sdr}, and extended STOI (ESTOI)~\cite{jensen2016algorithm}. In the absence of actual human quality objective, we measure the predicted MOS score of the enhanced speech, using our proposed PMOS model, since we aim to improve human-assessed speech quality. We denote this metric as MOS listener quality objective (MOS-LQO). Table~\ref{tab:results_cosineVoices} shows the average results of the different enhancement models, according to each of the performance metrics on COSINE and VOiCES dataset. As the scores of the unprocessed mixtures show, the VOiCES corpus is  more challenging than the COSINE corpus. 
With the baseline SE model, we experiment with 5 different combination of loss functions. Using the MSE loss only in SE:mse, we see improvements in objective scores, except with MOS-LQO for the COSINE data. Then we apply a MOS loss $\mathcal{L}_{mos}$ as the sole objective criterion, as proposed in \cite{fu2019learning}. Our experimental results show that this approach results in an overall improvement of $1.4$ in MOS-LQO compared to SE:mse. %We apply MOS-LQO scores of enhanced speech to calculate MOS loss $\mathcal{L}_{mos}$ as the only objective criteria as proposed in \cite{fu2019learning}, which gives improves MOS-LQO by $1.4$ overall compared with SE:mse. 
Then we separately combine the signal approximation loss with the mse loss and MOS loss (e.g., mse+sa and mos+sa). In PESQ, we gain an average of $\ge0.05$ and $\ge0.07$ compared to the models that use only the MSE loss and only the MOS loss, respectively. Furthermore, the model trained with the mos+sa loss function achieves the highest MOS-LQO score of $4.4$ and $5.7$ among all five loss functions tested with the SE model in COSINE and VOiCES dataset, respectively. This result is on average $1.15$ MOS-LQO higher than that obtained with the mse+sa loss function. These scores suggest that $\mathcal{L}_{mse}$ and $\mathcal{L}_{sa}$ maximize the overall speech intelligibility, whereas $\mathcal{L}_{mos}$ guides the model towards perceptual speech quality. Note that in all these $\mathcal{L}_{mos}$ calculations, we use a separately trained PMOS model's output without joint learning.
Lastly, we apply the SDR loss function as proposed in \cite{kawanaka2020stable}, which is used as the pre-training stage for model training. We observe an average gain of $0.9$ in SI-SDR, however, it yields a poor score according to other metrics, especially a $0.7$ loss in MOS-LQO compared to SE with mse and sa loss terms. 

SE+PMOS is separately investigated with 3 combinations of loss functions, i.e. mse, mse+sa, and mse+sa+mos. Compared with SE models, SE+PMOS with mse loss achieves $0.9$ SI-SDR and $1.75$ MOS-LQO improvements on average, which shows the benefit of incorporating the PMOS model. The SE+PMOS:mse+sa model improves the performance further with an average of $0.14$ ESTOI gain over the SE:mse+sa model. The inclusion of the mos loss gives the best MOS-LQO scores of $5.1$ and $6.5$ over all the comparison models in noisy and reverberant conditions, respectively.

%%%%%%%%%%%%%%%%%%%%%%%%%%%%%%%%%%%%%%%%%%%%%%%%%%%%
% Table 3, SE comparison test results on CHiME 5+4 
%%%%%%%%%%%%%%%%%%%%%%%%%%%%%%%%%%%%%%%%%%%%%%%%%%%%

% Please add the following required packages to your document preamble:
% \usepackage{multirow}
\begin{table*}[t!]
\centering
\caption{Average testing results of the speech enhancement models on CHiME-5 and CHiME-4 datasets. Best results are shown in \textbf{bold}.}
\label{tab:results_chime}
\resizebox{\linewidth}{!}{%
\begin{tabular}{| l | l | c c c c c | c c c c c |}
\cline{3-12}
\multicolumn{1}{l}{\multirow{2}{*}{}} &                    & \multicolumn{5}{ c |}{{CHiME-5}}                             & \multicolumn{5}{ c |}{{CHiME-4}} 
\\ \hline
\multicolumn{1}{|l|}{models}                          & \multicolumn{1}{c|}{loss func.} & PESQ$\uparrow$          & SI-SDR$\uparrow$       & ESTOI$\uparrow$         & MOS-LQO$\uparrow$      & WER\%$\downarrow$         & PESQ$\uparrow$          & SI-SDR$\uparrow$        & ESTOI$\uparrow$         & MOS-LQO$\uparrow$      & WER\%$\downarrow$         \\ \hline
\multicolumn{1}{|l|}{Mixture}                & -                               & 1.7           & 2.4          & 0.52          & 3.8          & 152.1         & 1.96          & 2.86          & 0.6           & {4.6} & {33.7} \\ \hline
\multicolumn{1}{|l|}{SE}                              & mos+sa                          & {2.25} & {3.9} & {0.62} & {4}   & {96.4} & {2.32} & {5.22} & {0.63} & {5}   & {25.6} \\ \hline
\multicolumn{1}{|l|}{SE+PMOS}                         & mse+sa+mos                      & 2.37          & 6.1          & 0.67          & 4.4          & 84.5          & 2.45          & 7.6           & 0.7           & 5.8          & 22.6          \\ \hline
\multicolumn{1}{|l|}{\multirow{2}{*}{MetricGAN}}      & pesq                            & \textbf{2.44} & {6.3} & {0.65} & {4.1} & {94.8} & \textbf{2.51} & {7}    & {0.68} & {5.3} & {19.7} \\ 
\multicolumn{1}{|l|}{}                                & stoi                            & 2.39          & 6.2          & \textbf{0.71} & 4.1          & 91.3          & 2.45          & {6.45} & \textbf{0.73} & 5.6          & 21.5          \\ \hline
\multicolumn{1}{|l|}{\multirow{2}{*}{Proposed}} & mse+sa                          & 2.41          & 7.1          & {0.68} & 4.7          & \textbf{78.3} & 2.5           & {7.9}  & 0.72          & 5.76         & \textbf{18.1} \\ 
\multicolumn{1}{|l|}{}                                & mse+sa+mos                      & 2.41          & \textbf{7.3} & {0.68} & \textbf{4.9} & 79.4          & {2.5}  & \textbf{8.61} & \textbf{0.73} & \textbf{6}   & 18.9          \\ \hline
\end{tabular}}
\end{table*}
MetricGAN optimizes PESQ or STOI, therefore, it outperforms other comparison models in terms of PESQ and ESTOI, although the scores for the SE+PMOS approaches are higher according to the other evaluation metrics even though these metrics are not leveraged during training. 
SSEMS yields the lowest scores across all metrics compared with SE+PMOS and MetricGAN approaches, though we do parameter tuning for this model.
Chi++\textsubscript{fQSM,bS} estimates quantized speech, and the results show that it affects the traditional objective functions. This performs poorly compared with the SE+PMOS and MetricGAN approaches, however, on average, it outperforms SSEMS in all criteria, and the SE models in terms of PESQ. With the MOS-LQO criteria, it fails to produce good scores. This points out the importance of incorporating perceptual features during enhancement, which Chi++\textsubscript{fQSM,bS} clearly lacks.

We calculate the performance of our proposed model using two combinations of loss functions. 
Using only mse and sa loss terms, we achieve the highest ESTOI scores for both corpora, though these results are nearly identical to the model trained with all three loss terms. Using $\mathcal{L}$ (eq:\ref{eq:loss}) in our proposed model, we obtain the highest SI-SDR scores while maintaining similar PESQ and ESTOI performance as compared to the best-performing model. Specifically, our proposed model achieves the highest ESTOI score and an average PESQ score that is only $0.03$ less than that of the best performing MetricGAN:pesq model.
Contrasting with the Chi++\textsubscript{fQSM,bS} model, which uses spectral language model to estimate quantized speech, our proposed approach outperforms the quantized model according to all metrics, which proves the significance of joint learning.% to direct speech enhancement model towards perceptually better speech using a speech quality assessment model.
When comparing MOS-LQO scores, our proposed:mse+sa+mos model achieves better scores than the other models except the SE+PMOS:mse+sa+mos model with an average of only $0.05$ declination. Thus, the inclusion of a spectral language model helps the model proposed (e.g., mse+sa+mos) to estimate better quality speech according to the overall evaluation criteria. 
It is important to note that our proposed approach performs best according to SI-SDR in both noisy and reverberant environments, where this metric is not used by any of the approaches during optimization.  

We further examine our approaches using completely unseen corpora. We test models with the CHiME-5 and CHiME-4 corpora where the models are trained from the COSINE dataset according to the system setup mentioned in section~\ref{subsec:setup}. Table~\ref{tab:results_chime} shows the performance evaluated according to PESQ, SI-SDR, ESTOI, MOS-LQO, and word error rate (WER). To calculate WER, we use the conventional ASR baseline that is provided with CHiME-5 and CHiME-4 dataset. We investigate WER with both GMM based ASR and end-to-end ASR, however, we find that the end-to-end approach results in a higher error compared to the GMM baseline. This might happen due to larger data requirements of the end-to-end ASR system as mentioned in \cite{barker2018fifth}. Therefore, we use the GMM ASR approach to compare the WER performance of the enhancement models.
From the scores of mixtures, we find that CHiME-5 is more challenging than CHiME-4 with a $118.8\%$ higher WER and a $0.46$ lower SI-SDR. Our proposed approach yields the best MOS-LQO scores with $4.9$ with CHiME-5 and $6$ with CHiME-4 data. The proposed mse+sa model results in the lowest WER of $78.3$ and $18.1$ using CHiME-5 and CHiME-4, respectively. Note that the WER of the GMM baseline ASR for the CHiME-5 challenge is $72.8$ in binaural and $91.7$ in single array conditions. Here our approaches enhance monaural speech, a more challenging condition. Our proposed approach outperforms other comparison models in terms of SI-SDR with a $5.29$ average improvement compared to others. According to PESQ and ESTOI metrics, MetricGAN variants give the best performace, however, proposed model's performance is $0.02$  and $ 0.015$ lower according to PESQ and ESTOI, respectively, for the best performing MetricGAN models. Hence, our proposed approach is effective on out-of-vocabulary scenario trained by a comparable dataset.


% \nayem{*** Possibly add graphs of evaluation metrics vs SNRs.}

%%%%%%%%%%%%%%%%%%%%%%%%%%%%%%%%%%%%%%%%%%%%%%%%%%%%
% Table 3, DNSMOS results
%%%%%%%%%%%%%%%%%%%%%%%%%%%%%%%%%%%%%%%%%%%%%%%%%%%%
% \begin{table}[thb!]

% \centering
% \caption{Average MOS ratings of the speech enhancement modes on CHiME-4 and CHiME-5 datasets using DNSMOS P.835~\cite{reddy2022dnsmos}. Best results are shown in \textbf{bold}.}
% \label{tab:dnsmos_results}
% % \vspace{-0.5em}
% \resizebox{\columnwidth}{!}{%
% \begin{tabular}{| l | c c | }
% \cline{2-3}
%   \multicolumn{1}{c|}{}         & {CHiME-4} & {CHiME-5} \\ \hline
   
% Mixture   & 1.54 ($\pm$0.85)         & 1.3 ($\pm$1.1)                \\
% PMOS+SE                      & 4.28 ($\pm$0.9)       & 3.67 ($\pm$1.3)\\
% MetricGAN                    & 4.26 ($\pm$0.87) & 3.5 ($\pm$1.34)          \\
% Proposed                     & \textbf{4.32} ($\pm$0.8)& \textbf{3.8} ($\pm$1.41)            \\ \hline
% Clean                     & 4.67 ($\pm$1.2) & -      \\
% \hline
% \end{tabular}
% }
% % \vspace{-2em}
% \end{table}

%%%%%%%%%%%%%%%%%%%%%%%%%%%%%%%%%%%%%%%%%%%%%%%%%%%%
% Fig 3, DNSMOS results plot
%%%%%%%%%%%%%%%%%%%%%%%%%%%%%%%%%%%%%%%%%%%%%%%%%%%%

\begin{figure}[b!]
    \centering
\begin{tikzpicture}
	\begin{axis}[
	    cycle list/Dark2-4,
		boxplot/draw direction = y,
		boxplot/box extend=0.8,
% 		x=3em,
% 		x axis line style = {opacity=0.6},
		axis x line* = bottom,
		axis y line = left,
		enlarge y limits,
		ymajorgrids,
		xtick = {1, 2, 3, 4, 5, 6, 7, 8},
		xticklabel style = {align=center, font=\small, rotate=60, alias={xtick-\ticknum}},
		xticklabels = {Mixture, SE+PMOS, MetricGAN, Proposed, Mixture, SE+PMOS, MetricGAN, Proposed},
% 		xtick style = {draw=none}, % Hide tick line
		ylabel = {MOS},
		ytick = {1, 2, 3, 4, 5},
	]
	
	\addplot+[
        boxplot prepared={
        lower whisker=1, lower quartile=1.45,
        median=1.74,
        upper quartile=2.5, upper whisker=4.05, }, fill, draw=black]
        coordinates {}
        node[above, color=black] at
        (boxplot box cs: \boxplotvalue{median},.5)
        {\scriptsize \pgfmathprintnumber{\boxplotvalue{median}}};
    \addplot+[
        boxplot prepared={
        lower whisker=1.38, lower quartile=1.84,
        median=2.28,
        upper quartile=3.1, upper whisker=4.3, }, fill, draw=black]
        coordinates {}
        node[above, color=black] at
        (boxplot box cs: \boxplotvalue{median},.5)
        {\scriptsize \pgfmathprintnumber{\boxplotvalue{median}}};
    \addplot+[
        boxplot prepared={
        lower whisker=1.3, lower quartile=1.75,
        median=2.13,
        upper quartile=3.2, upper whisker=4.1, }, fill, draw=black]
        coordinates {}
        node[above, color=black] at
        (boxplot box cs: \boxplotvalue{median},.5)
        {\scriptsize \pgfmathprintnumber{\boxplotvalue{median}}};
    \addplot+[
        boxplot prepared={
        lower whisker=1.4, lower quartile=1.9,
        median=2.46,
        upper quartile=3.16, upper whisker=4.34, }, fill, draw=black]
        coordinates {}
        node[above, color=black] at
        (boxplot box cs: \boxplotvalue{median},.5)
        {\scriptsize \pgfmathprintnumber{\boxplotvalue{median}}};
        
    \addplot+[
        boxplot prepared={
        lower whisker=1.0, lower quartile=1.35,
        median=1.64,
        upper quartile=2.39, upper whisker=4.18, }, fill, draw=black]
        coordinates {}
        node[above, color=black] at
        (boxplot box cs: \boxplotvalue{median},.5)
        {\scriptsize \pgfmathprintnumber{\boxplotvalue{median}}};
    \addplot+[
        boxplot prepared={
        lower whisker=1.31, lower quartile=1.8,
        median=2.18,
        upper quartile=2.76, upper whisker=4.24, }, fill, draw=black]
        coordinates {}
        node[above, color=black] at
        (boxplot box cs: \boxplotvalue{median},.5)
        {\scriptsize \pgfmathprintnumber{\boxplotvalue{median}}};
    \addplot+[
        boxplot prepared={
        lower whisker=1.26, lower quartile=1.71,
        median=2.06,
        upper quartile=3.17, upper whisker=4.32, }, fill, draw=black]
        coordinates {}
        node[above, color=black] at
        (boxplot box cs: \boxplotvalue{median},.5)
        {\scriptsize \pgfmathprintnumber{\boxplotvalue{median}}};
    \addplot+[
        boxplot prepared={
        lower whisker=1.34, lower quartile=1.85,
        median=2.25,
        upper quartile=3.07, upper whisker=4.48, }, fill, draw=black]
        coordinates {}
        node[above, color=black] at
        (boxplot box cs: \boxplotvalue{median},.5)
        {\scriptsize \pgfmathprintnumber{\boxplotvalue{median}}};
        
	\end{axis}
	
	\path (0,0) coordinate (P);
    \draw [thick,decoration={brace,mirror,raise=5em},decorate] (xtick-0|-P) -- (xtick-3.5|-P) 
        node[midway,yshift=-6em]{CHiME-4};
    \draw [thick,decoration={brace,mirror,raise=5em},decorate] (xtick-4|-P) -- (xtick-7.5|-P) 
        node[midway,yshift=-6em]{CHiME-5};

    % \node[text width=3cm] at (1.54,0.5) 
    % {\scriptsize 1.54};

\end{tikzpicture}

\caption{MOS ratings of the speech enhancement modes on CHiME-4 and CHiME-5 datasets using DNSMOS P.835.}
    % \vspace{-2em}
\label{fig:dnsmos_results}
    % \vspace{-0.4cm}
\end{figure}

\subsection{Perceptual quality evaluation}
\label{subsec:dnsmos}

We finally evaluate our model using P.835 metric~\cite{reddy2022dnsmos} to measure perceptual quality. We calculate the DNSMOS score on a scale of $[1-5]$ ($1$ = worst, $5$ = best) for the mixture, PMOS+SE, MetricGAN, and our proposed models using the CHiME-4~\cite{vincent2017analysis} and CHiME-5~\cite{barker2018fifth} datasets (simulated and real-recording). Figure~\ref{fig:dnsmos_results} shows the scores. With CHiME-4, the original mixture scores range from $1.45$ to $2.5$ with a median of $1.74$. Our proposed model achieves a median MOS of $2.46$, which is higher than the others. Fon CHiME-5, the original mixture scores range from $1.0$ to $4.18$. Our proposed model outperforms the others with a median of $2.25$. Our proposed model and PMOS+SE have smaller standard deviations compared to MetricGAN. Overall, our proposed model improves noisy speech in both the acoustic and perceptual aspects. 




% \subsection{Listening results}
% \label{subsec:listening_results}

% We conduct an IRB-approved listening study using Amazon Mechanical Turk to conceive the perceptual quality of enhanced speech assessed by normal-hearing listeners. 

% This study follows the design structure of \cite{nayem2021towards} and figure~\ref{fig:survey} shows the actual listener study interface of a single question. The study is conducted as follows, the participant will listen to two audio signals, one is enhanced and the other is clean audio as reference.  Then they provide a preference score using a Likert scale. The scale ranges from $-3$ to $+3$, where $-3$ refers to a strong preference towards the first signal, $+3$ refers to a strong preference towards the second signal, and $0$ refers to no preference. Before providing a score, the participant can listen to the signals as many as times they like, where the scores are not limited to integer values. The two signals are randomly selected, and the participant listens to different audio clips in each question. The audio clips are chosen from the CHiME-5 and CHiME-4 corpus spoken by both males and females in equal proportion. Prior to actual survey questions, each participants has to pass eligibility test and make themselves familiar with the upcoming study session by going through a practice session. The structure of this practice session is similar to the actual study, however, speakers' voice and audio clips which participants hear in practice session are not used in the actual study. A tentative feedback is provided in the practice session to give a guideline to the participants, however, to avoid biases and leading answers, the feedback is provided in a form of range where the expected answer should reside.



%  \begin{figure}[thb!]
%     \centering
%     \includegraphics[width = 0.5\linewidth]{IEEEtran/figs/survey.png}
%     % \vspace{-2em}
%     \caption{A question of actual listener study interface conducted on MTurk.}
%     \label{fig:survey}
%     % \vspace{-2em}
% \end{figure}

% \nayem{***One paragraph on the statistics of the conducted study.}
% The study session contains total 30 questions, which is preceded by a practice session of 7 questions. Ten participants (9 male, 1 female) who are native English speakers over the age of 18 participated, where a headset/headphone was required to be worn. On average, participants took 14 minutes to complete the study, they were given $\$3$ monetary incentive.


\section{Discussion}
\label{sec:discuss}

Our proposed model outperforms all comparison models on SI-SDR metrics for both seen and unseen datasets, without optimization of any of the models (Table \ref{tab:results_cosineVoices}, \ref{tab:results_chime}). This means that our approach improves speech quality by minimizing the distortion ratio when separated from the noise component. Additionally, our models yield the best MOS-LQO ratings on real-world captured audios (CHiME datasets, Table \ref{tab:results_chime}). These results are consistent with the findings of \cite{zezario2022deep, nayem2021incorporating} that incorporating embeddings from a speech assessment model improves SE performance, and the results of \cite{braun2022effect} that using MOS loss during model optimization leads to higher MOS-LQO scores. Our proposed approach achieves PESQ and ESTOI scores that are only slightly lower than those of the best-performing model, with a difference of only $0.03$ and $0.01$, respectively. This indicates that speech quality and intelligibility metrics are closely related to the subjective speech quality metric (MOS-LQO), and that these metrics can be improved without explicit optimization. Furthermore, our proposed model achieves the best average DNSMOS scores with low standard deviations on CHiME datasets (Figure \ref{fig:dnsmos_results}), indicating that it is effective in a wide range of real-world noise levels. This is a desirable quality for an effective SE model to be effective not only in high SNRs and limited noisy environments, but also in large SNR ranges and real-world conditions such as those offered by the CHiME dataset.

When comparing our proposed model that uses mse+sa+mos loss to the PMOS+SE model (as shown in Table \ref{tab:results_chime}), we can observe significant improvements in all performance metrics. As both models use the same loss function, the improvements are attributed to the incorporation of LM and the joint learning method. Moreover, we found that these two models exhibit similar performance on the MOS prediction (Table \ref{tab:mos_results}), indicating that the benefits of joint learning mostly impact the enhancement part of the model.

An intriguing finding is that our proposed model shows a decline in WER\% when MOS loss is incorporated, especially for larger real-world recordings such as CHiME-5, with degradation up to $1.1$. Although our study is not primarily concerned with ASR performance, this suggests a potential trade-off between ASR accuracy and subjective speech quality scores. Further investigation is needed to comprehend this relationship.

Our proposed method demonstrates that training a speech enhancement (SE) model and a MOS-based speech assessment model jointly can lead to better speech quality measured by objective metrics such as perceptual quality, intelligibility, and MOS ratings. However, we acknowledge that our study's use of subjective MOS (sMOS) estimation instead of actual human listeners may introduce discrepancies between MOS-LQO and human-rated MOS, which could impact our findings. To address this limitation, we plan to conduct sMOS evaluation by human listeners in future work. Although we used the same MOS prediction model for all comparison models, we believe that incorporating human-rated sMOS evaluations will provide more robust insights into our proposed method's effectiveness.
For computing loss terms, we opt for the MSE loss function along with a bi-gram language model that considers only time-along transitions. Our aim is to keep the model simple and focus on the effectiveness of our approach. However, we acknowledge that using different loss functions for different loss components and employing a more complex language model that considers both temporal and spectral transition levels can be beneficial. We plan to explore these possibilities in our future work.




\section{Experimental Results}
\label{sec:results}

In this section, we validate our hypothesis that simple manipulations of the temperature parameter in~\cref{eq:info-nce} lead to better performance for long-tailed data. First, we introduce our experimental setup in \cref{subsec:implementation}, then in \cref{subsec:results} we discuss the results across three imbalanced datasets and, finally, we analyse different design choices of the framework through extensive ablation studies in \cref{subsec:ablations}. 



\subsection{Implementation Details}
\label{subsec:implementation}
\myparagraph{Datasets.}
We consider long-tailed (LT) versions of the following three popular datasets for the experiments: CIFAR10-LT, CIFAR100-LT, and ImageNet100-LT. For most of the experiments, we follow the setting from SDCLR~\citep{jiang2021self}. In case of \textbf{CIFAR10-LT/CIFAR100-LT}, the original datasets~\citep{krizhevsky2009learning} consist of 60000  32x32 images sampled uniformly from 10 and 100 semantic classes, respectively, where 50000 images correspond to the training set and 10000 to a test set.
% 
Long-tail versions of the datasets are introduced by~\citet{cui2019class} and consist of a subset of the original datasets with an exponential decay in the number of images per class. The imbalance ratio controls the uniformity of the dataset and is calculated as the ratio of the sizes of the biggest and the smallest classes. By default, we use an imbalance ratio 100 if not stated otherwise. Experiments in \cref{table:cifar10_cifar100_imb100}, \cref{table:simclr-sdclr} are the average over three runs with different permutations of classes.
\textbf{ImageNet100-LT} is a subset of the original ImageNet-100~\citep{tian2020contrastive} consisting of 100 classes for a total of 12.21k 256x256 images. The number of images per class varies from 1280 to 25. 


\myparagraph{Training.}
We use an SGD optimizer for all experiments with a weight decay of 1e-4. As for the learning rate, we utilize linear warm-up for 10 epochs that is followed by a cosine annealing schedule starting from 0.5. We train for 2000 epochs for CIFAR10-LT and CIFAR100-LT and 800 epochs for ImageNet100-LT. For CIFAR10-LT and CIFAR100-LT we use a ResNet18~\citep{he2016deep} backbone.
For ImageNet100-LT we use a ResNet50~\citep{he2016deep} backbone. For both the MoCo~\citep{he2019moco} and the SimCLR~\citep{chen2020simple} experiments, we follow \citet{jiang2021self} and use the following augmentations: resized crop, color jitters, grey scale and horizontal flip. MoCo details: we use a dictionary of size 10000, a projection dimensionality of 128 and a projection head with one linear layer. SimCLR details: we train with a batch size of 512 and a projection head that has two layers with an output size of 128. For evaluation, we discard the projection head and apply l2-normalisation. Regarding the proposed \underline temperature \underline schedules (TS), we use a period length of $T\myeq400$ with $\tau_+\myeq1.0$ and $\tau_-\myeq0.1$ if not stated otherwise; for more details, see \cref{appendix:implementation_details}.


\myparagraph{Evaluation}
We use k nearest neighbours (kNN) and linear classifiers to assess the learned features. For kNN, we compute $l2$-normalised  distances between LT
%
samples from the train set and the class-balanced test set. For each test image, we assign it to the majority class among the top-k closest train images. We report accuracy for kNN with $k\myeq1$ (kNN@1) and with $k\myeq10$ (kNN@10). Compared to fine-tuning or linear probing, kNN directly evaluates the learned embedding since it relies on the learned metric and local structure of the space. 
%
We also evaluate the linear separability and generalisation of the space with a linear classifier that we train on the top of frozen backbone. For this, we consider two setups: balanced few-shot linear probing (FS LP) and long-tailed linear probing (LT LP). For FS LP, the few-shot train set is a direct subset of the original long-tailed train set with the shot number equal to the minimum class size in the original LT train set. For LT LP, we use the original LT training set. For extended tables, see \cref{sec:extend_results}.

\subsection{Effectiveness of Temperature Schedules}
\label{subsec:results}
\myparagraph{Contrastive learning with TS.}
In \cref{table:cifar10_cifar100_imb100} we present the efficacy of temperature schedules (TS) for two well-known contrastive learning frameworks MoCo~\citep{he2019moco} and SimCLR~\citep{chen2020simple}. We find that both frameworks benefit from varying the temperature and we observe consistent improvements
%
over all evaluation metrics for CIFAR10-LT and CIFAR100-LT,
% 
\ie
the local structure of the embedding space (kNN) and the global structure (linear probe) %that is captured by the linear probe 
are both improved. 
% 
Moreover, we show in \cref{table:simclr-sdclr} that our finding also transfers to ImageNet100-LT.
%
Furthermore, in \cref{table:moco_imb150} we evaluate the performance of the proposed method on the CIFAR10 and CIFAR100 datasets with different imbalance ratios. An imbalance ratio of 50 (imb50) reflects less pronounced imbalance, and imb150 corresponds to the datasets with only 30 (CIFAR10) and 3 (CIFAR100) samples for the smallest class. Varying $\tau$ during training improves the performance for different long-tailed data\rebuttal{; for a discussion on the dependence of the improvement on the imbalance ratio, please see the appendix}.
%
% 
\begin{table*}[hb!]
\begin{center}
\vspace{0pt}
\caption{
    \textbf{CIFAR10 and CIFAR100 Results} The performance of test-time methods over two datasets: CIFAR10 and CIFAR100.  
    The results are grouped per AT model, evaluating different test-time methods: `Base' in which we do not use any test-time method, followed by other test-time methods.
    % \vspace{-30pt}
}
\resizebox{\textwidth}{!}{%

\begin{tabular}{lllllcccccc}
\hline\noalign{\smallskip}\hline

\rowcolor{gray!5}  &  &  &  &  &  & \multicolumn{4}{c}{Attack}\\
\rowcolor{gray!5} & & & & & & \multicolumn{2}{c}{$L_{\infty}$} & \multicolumn{2}{c}{$L_{2}$}\\
\rowcolor{gray!5} \multirow{-3}{*}{Dataset} & \multirow{-3}{*}{AT Method} & \multirow{-3}{*}{\makecell{Trained Threat \\ Model}} & \multirow{-3}{*}{Architecture}  & \multirow{-3}{*}{\makecell{Test-Time \\ Method}}  & \multirow{-3}{*}{Clean} & $8/255$ & $16/255$ & $0.5$ & $1.0$\\

\hline\noalign{\smallskip}\hline\noalign{\smallskip}

\multirow{17}{*}{CIFAR10} & PAT \cite{laidlaw2020perceptual} & & RN50 & & $71.60\%$ & $28.70\%$ & $-$ & $33.30\%$ & $-$\\


\cline{2-10}\noalign{\smallskip}

& \multirow{5}{*}{AT \cite{madry2017towards}} & \multirow{5}{*}{$L_{2}, \epsilon=0.5$} & \multirow{5}{*}{RN50} & Base & $90.83\%$ & $29.04\%$ & $00.93\%$ & $69.24\%$ & $36.21\%$\\

& & & &  RSmooth \cite{cohen2019certified} & $89.43\%$ &  $30.84\%$ & $01.23\%$ & $68.94\%$ & $38.53\%$\\

& & & &  TTE \cite{perez2021enhancing} & $\textbf{90.99}\%$ &  $36.41\%$ & $02.40\%$ & $71.90\%$ & $41.18\%$\\

& & & & DRQ \cite{schwinn2022improving} & $88.79\%$ & $45.37\%$ & $07.09\%$ & $77.56\%$ & $51.28\%$\\

& & & & \cellcolor{golden!10} \AlgoName  & \cellcolor{golden!10} $87.40\%$ & \cellcolor{golden!10} $\textbf{51.66\%}$ & \cellcolor{golden!10} $\textbf{14.96\%}$ & \cellcolor{golden!10} $\textbf{78.66\%}$ & \cellcolor{golden!10} $\textbf{59.82\%}$\\



\cline{2-10}\noalign{\smallskip}


& \multirow{4}{*}{Rebuffi \emph{et al.} \cite{rebuffi2021fixing}} & \multirow{4}{*}{$L_{2}, \epsilon=0.5$} & \multirow{4}{*}{WRN28-10} & Base & $\textbf{91.79\%}$ & $47.85\%$ & $05.00\%$ & $78.80\%$ & $54.73\%$\\

& & & &  TTE \cite{perez2021enhancing} & $91.59\%$ & $50.49\%$ & $06.79\%$ & $79.18\%$ & $55.38\%$\\


& & & &  DRQ \cite{schwinn2022improving} & $90.99\%$ & $58.66\%$ & $\textbf{13.69\%}$ & $84.12\%$ & $64.69\%$\\

& & & &  \cellcolor{golden!10} \AlgoName & \cellcolor{golden!10} $88.23\%$ & \cellcolor{golden!10} $\textbf{59.99\%}$ & \cellcolor{golden!10} $11.45\%$ & \cellcolor{golden!10} $\textbf{85.56\%}$ & \cellcolor{golden!10} $\textbf{66.15\%}$\\


\cline{2-10}\noalign{\smallskip}

& \multirow{4}{*}{Rebuffi \emph{et al.} \cite{rebuffi2021fixing}} & \multirow{4}{*}{$L_{\infty}, \epsilon=8/255$} & \multirow{4}{*}{WRN28-10} & Base & $\textbf{87.33\%}$ & $60.77\%$ & $25.44\%$ & $66.72\%$ & $35.01\%$\\

 & & & & TTE \cite{perez2021enhancing} & $87.30\%$ & $61.52\%$ & $27.50\%$ & $66.88\%$ & $36.07\%$\\

& & & & DRQ \cite{schwinn2022improving} & $87.17\%$ & $66.23\%$ & $33.62\%$ & $72.24\%$ & $44.56\%$\\

& & & & \cellcolor{golden!10} \AlgoName & \cellcolor{golden!10} $85.00\%$ & \cellcolor{golden!10} $\textbf{66.86\%}$ & \cellcolor{golden!10} $\textbf{34.88\%}$ & \cellcolor{golden!10} $\textbf{74.24\%}$ & \cellcolor{golden!10} $\textbf{53.02\%}$\\


\cline{2-10}\noalign{\smallskip}

& \multirow{3}{*}{Gowal \emph{et al.} \cite{gowal2020uncovering}} & \multirow{3}{*}{$L_{\infty}, \epsilon=8/255$}  & \multirow{3}{*}{WRN70-16} & Base & $\textbf{91.09\%}$ & $65.88\%$ & $25.95\%$ & $66.43\%$ & $27.21\%$\\

& & & & DRQ \cite{schwinn2022improving} & $90.77\%$ & $71.00\%$ & $35.89\%$ & $72.87\%$ & $39.51\%$\\

 & & & & \cellcolor{golden!10} \AlgoName  & \cellcolor{golden!10} $88.18\%$ & \cellcolor{golden!10} $\textbf{72.02}\%$ & \cellcolor{golden!10} $\textbf{40.30\%}$ & \cellcolor{golden!10} $\textbf{75.90\%}$ & \cellcolor{golden!10} $\textbf{49.21\%}$\\

\hline\noalign{\smallskip}



\multirow{9}{*}{CIFAR100} & \multirow{5}{*}{Rebuffi \emph{et al.} \cite{rebuffi2021fixing}} & \multirow{5}{*}{$L_{\infty}, \epsilon=8/255$} & \multirow{5}{*}{WRN28-10}  & Base & $\textbf{62.40}\%$ & $32.06\%$ & $12.47\%$ & $38.32\%$ & $18.86\%$\\

& & & & TTE \cite{perez2021enhancing} &  $62.35\%$ &  $33.25\%$ & $13.84\%$ & $39.14\%$ & $20.22\%$\\

& & & & DRQ \cite{schwinn2022improving} & $61.32\%$ & $38.22\%$ & $19.41\%$ & $44.58\%$ & $26.78\%$\\

& & & & \cellcolor{golden!10} \AlgoName & \cellcolor{golden!10} $55.18\%$ & \cellcolor{golden!10} $37.61\%$ & \cellcolor{golden!10} $19.72\%$ & \cellcolor{golden!10} $45.86\%$ & \cellcolor{golden!10} $32.79\%$\\

& & & & \cellcolor{golden!10} \AlgoNameTop  & \cellcolor{golden!10} $57.95\%$ & \cellcolor{golden!10} $\textbf{38.73\%}$ & \cellcolor{golden!10} $\textbf{19.87\%}$ & \cellcolor{golden!10} $\textbf{47.56\%}$ & \cellcolor{golden!10} $\textbf{33.01\%}$\\


\cline{2-10}\noalign{\smallskip}


& & & & Base & $\textbf{69.15\%}$ & $36.90\%$ & $13.64\%$ & $40.86\%$ & $17.20\%$\\


& & & & DRQ \cite{schwinn2022improving} & $69.12\%$ & $43.96\%$ & $20.25\%$ & $48.95\%$ & $25.43\%$\\

& & & & \cellcolor{golden!10} \AlgoName & $59.83\%$ & \cellcolor{golden!10} $44.66\%$ & \cellcolor{golden!10} $\textbf{23.81\%}$ & \cellcolor{golden!10} $51.32\%$ & \cellcolor{golden!10} $\textbf{37.02\%}$\\

& \multirow{-4}{*}{Gowal \emph{et al.} \cite{gowal2020uncovering}}  & \multirow{-4}{*}{$L_{\infty}, \epsilon=8/255$} & \multirow{-4}{*}{WRN70-16}  & \cellcolor{golden!10} \AlgoNameTop  & \cellcolor{golden!10} $62.47\%$ & \cellcolor{golden!10} $\textbf{46.09\%}$ & \cellcolor{golden!10} $23.48\%$ & \cellcolor{golden!10} $\textbf{53.06\%}$ & \cellcolor{golden!10} $36.19\%$ \\



\hline\noalign{\smallskip} \hline\noalign{\smallskip}



\end{tabular}
}
% \vspace{-20pt}
\label{table:cifar10_and_cifar100}
\end{center}
\end{table*}


\begin{table}[!ht]
\centering
\small
\tabcolsep=0.15cm
\begin{tabular}{c|cc|cc|cc|cc}  % 4 = method, model, gn(5/69), gb(64/69), hm
% \toprule
& \multicolumn{4}{c}{CIFAR10-LT}  & \multicolumn{4}{c}{CIFAR100-LT}  \\
method  & kNN@1 & kNN@10 & FS LP & LT LP & kNN@1 & kNN@10 & FS LP & LT LP  \\
\midrule
MoCo  & 63.54 & 64.56  & 69.31  & 65.11 & 28.69 & 28.75  & 26.86  & 30.41    \\
% MoCo + S$\tau$A \\
MoCo + TS   &  \textbf{64.99} &  \textbf{65.01} &  \textbf{72.87} &  \textbf{66.86}  &  \textbf{30.31} &  \textbf{29.75} &  \textbf{28.97} &  \textbf{32.05}  \\
\midrule
SimCLR  & 59.84 & 60.19 & 68.29 & 61.86 & 28.81 & 28.12  & 25.70  &  31.20  \\
SimCLR + TS  & \textbf{63.09} & \textbf{62.91} & \textbf{71.86} & \textbf{65.03} &  \textbf{31.06} & \textbf{30.06}  & \textbf{28.89}  & \textbf{33.28}  \\


% \bottomrule
\end{tabular}
\vspace{.25em}
\caption{\textbf{Effect of temperature scheduling.} Comparison of MoCo vs MoCo+TS and SimCLR vs SimCLR+TS on CIFAR10-LT and CIFAR100-LT with kNN, few-shot and long-tail linear probe (FS LP and LT LP). 
}
\label{table:cifar10_cifar100_imb100}
\vspace{-1em}
\end{table}

\begin{table}[!ht]
\centering
\small
\tabcolsep=0.15cm
\begin{tabular}{c|cc|cc|cc|cc}  % 4 = method, model, gn(5/69), gb(64/69), hm
% \toprule
& \multicolumn{4}{c|}{CIFAR-10-LT} &  \multicolumn{4}{c}{CIFAR-100-LT}   \\
% \midrule
& \multicolumn{2}{c|}{imb 50} & \multicolumn{2}{c|}{imb 150} &\multicolumn{2}{c|}{imb 50} & \multicolumn{2}{c}{imb 150}  \\
method  & kNN@10 & FS LP & kNN@10 & FS LP & kNN@10 & FS LP & kNN@10 & FS LP  \\
\midrule
MoCo & 69.12 & 74.16 & 59.13 & 65.76 & 32.22 & 33.53  & 25.36 & 22.73  \\
MoCo + TS & \textbf{71.49} & \textbf{76.37} & \textbf{60.83} & \textbf{68.59} & \textbf{33.24} & \textbf{35.03}  & \textbf{26.75} &  \textbf{22.78} \\


% \bottomrule
\end{tabular}
\vspace{.25em}
\caption{\textbf{Effect of imbalance ratio.} MoCo vs MoCo+TS on CIFAR10-LT and CIFAR100-LT for imbalance ratio 50 (imb50) and 150 (imb150). Evaluation metrics: kNN classifier and few-shot linear probe (FS LP). }
\label{table:moco_imb150}
\vspace{-1em}
\end{table}




% \input{iclr2023/tables/simclr_sdclr.tex}

\begin{table}[!h]
\centering
\small
\tabcolsep=0.15cm
\begin{tabular}{c|ccc|ccc|ccc}  % 4 = method, model, gn(5/69), gb(64/69), hm
% \toprule
& \multicolumn{3}{c}{CIFAR-10-LT} &  \multicolumn{3}{c}{CIFAR-100-LT} &  \multicolumn{3}{c}{ImageNet-100-LT}  \\
method  & kNN@10 & FS LP & LS LP & kNN@10 & FS LP & LT LP & kNN@10 & FS LP & LT LP \\
\midrule
SimCLR  & 60.19 & 68.29 & 61.68 & 28.12 & 25.70 & 31.20  & 38.00 &  42.64 & 44.82 \\
SDCLR  & 60.74 & 71.03 & 64.99 & 29.22 & 27.28 &  \textbf{34.23} & 37.36 &  42.74  &  46.40\\
SimCLR + TS & \textbf{62.91} & \textbf{71.86} & \textbf{65.03} & \textbf{30.06} & \textbf{28.89} & 33.28 & \textbf{38.86} & \textbf{45.18} & \textbf{47.26} \\

% \bottomrule
\end{tabular}

\caption{\textbf{Comparison with SDCLR.} SimCLR vs SDCLR vs SimCLR+TS on CIFAR10-LT, CIFAR100-LT, and ImageNet100-LT. Evaluation: kNN classifier, few-shot (FS LP) and long-tail linear probe (LT LP).}
\label{table:simclr-sdclr}
% \vspace{-mm}
\end{table}

\myparagraph{TS vs SDCLR.} 
%
Further, we compare our method with SDCLR~\citep{jiang2021self}. In SDCLR, %the authors propose to modify
SimCLR is modified s.t.~the embeddings of the online model are contrasted with those of a pruned version of the same model, which is updated after every epoch.
%
Since the pruning is done by simply masking the pruned weights of the original model, SDCLR requires twice as much memory compared to the original SimCLR and extra computational time to prune the model every epoch. %, see supplement for a comparison. 
In contrast, our method does not require any changes in the architecture or training. In \cref{table:simclr-sdclr} we show that 
%our simple method
this simple approach improves not only over the original SimCLR, but also over SDCLR in most metrics. 
% 

\subsection{Ablations}
\label{subsec:ablations}
In this section, we evaluate how the hyperparameters in~\cref{eq:cos_tau} can influence the model behaviour.

\myparagraph{Cosine Boundaries.}
%
First, we vary the lower $\tau_-$ and upper $\tau_+$ bounds of $\tau$ for the cosine schedule. In \cref{table:min_max_tau_cosine} we assess the performance of MoCo+TS with different $\tau_-$ and $\tau_+$ on CIFAR10 with FS LP. We observe a clear trend that with a wider range of $\tau$ values the performance increases. We attribute this to the ability of the model to learn better `hard' features with low $\tau$ and improve semantic structure for high $\tau$. 
Note that $0.07$ is the value for $\tau$ in many current contrastive learning methods. 
%
% \input{iclr2023/tables/ablations_cos_boundary.tex}

\begin{minipage}[b]{0.55\textwidth}
\centering
\small
\tabcolsep=0.15cm
\begin{tabular}{c|ccccc}  % 4 = method, model, gn(5/69), gb(64/69), hm
\toprule
\backslashbox{$\tau_-$}{$\tau_+$} & 0.2 & 0.3 & 0.4 & 0.5 & 1.0  \\
\midrule
0.07 & 69.46 & 68.86 & 71.29 & 71.83 & \textbf{73.26}  \\
0.1 & 68.17 & 70.34 & 71.25 & 72.31 &  72.87 \\
0.2 & 68.89 & 69.37 & 70.12 & 69.65 &  71.42 \\

\bottomrule
\end{tabular}
\captionof{table}{\textbf{Influence of cosine boundaries.} Best performance with the largest difference between $\tau_-$ and $\tau_+$. CIFAR10 with MoCo+TS, evaluating few-shot linear probes (FS LP).}
\label{table:min_max_tau_cosine}
\end{minipage}
\hfill
\begin{minipage}[b]{0.4\textwidth} % 0.35
\begin{minipage}[b]{0.45\textwidth}
% \kern0pt
    \small
    \centering
    \begin{tabular}{l|c}  % 4 = method, model, gn(5/69), gb(64/69), hm
    
    \toprule
    % & \multicolumn{1}{c}{CIFAR-10-LT}   \\
    
    \,\,\,\,\,\,\,TS  & FS LP\\
    \midrule
    \reddark{$\blacksquare$} fixed & 68.89 \\
    \greendark{$\blacksquare$} step &  70.18  \\
    \cyan{$\blacksquare$} rand &  70.26  \\
    \purple{$\blacksquare$} oscil &  71.50  \\
    \yellow{$\blacksquare$} cos &   \textbf{72.31}  \\
    
    \bottomrule
    \end{tabular}
    \par\kern0pt
\end{minipage}
\hfill
\begin{minipage}[b]{0.45\textwidth}
\kern0pt
    \includegraphics[width=0.95\textwidth, height=2.3cm]{iclr2023/schedules3.pdf}
\vspace{-.075cm}
    % \par\kern0pt
\end{minipage}

\captionof{table}{\textbf{Alternative Schedules}. Constant, step function, and random sampling. All functions are bounded by $0.1$ and $0.5$.  
}
\label{table:alternatives}
% \vspace{-mm}
\end{minipage}



\myparagraph{Cosine Period.}
Further, we investigate if the length of the period $T$ in~\cref{eq:cos_tau} impacts the performance of the model. In \cref{table:periods}, we show that modifying the temperature $\tau$ based on the cosine schedule is beneficial during training independently of the period $T$. The %estimated 
performance varies insignificantly depending on $T$ and consistently improves over standard fixed $\tau\myeq0.2$, whereas the best performance we achieve with $T\myeq400$. Even though the performance is stable with respect to the length of the period, it changes within one period as we show in \cref{fig:periods}. 
Here, we average the accuracy of one last full period over different models trained with different $T$ and find that the models reach the best performance around $0.7\,T$. %For details, see supplement.
%Our recommendation for the hyperparameters includes a lower bound $\tau_-\myeq0.1$, upper bound $\tau_+\myin[0.5, 2.0]$, period length $T\myeq200$ epochs and 
Based on this observation, we recommend to stop training after $(n-0.3)\,T$ epochs, where $n$ is the number of full periods.
% \moritz{Is it not n-0.3 full periods?}



\begin{figure}[!h]
\begin{floatrow}
\capbtabbox[0.35\textwidth]{%
  \small
  \begin{tabular}{c|c|c} \toprule
 T &  T / $\#$epochs & FS LP \\ \midrule
 no & fixed $\tau$ & 68.89 \\
  200 & 0.1 & 71.86 \\
  400 & 0.2 & 72.87 \\
  1000 & 0.5  & 72.47 \\
  2000 & 1.0  & 72.22 \\
  4000 & 2.0  & 72.10 \\ \bottomrule
  \end{tabular}
}{%
% \vspace{1mm}
  \caption{\textbf{Influence of the period length $T$.} Few-shot linear probe accuracy (FS LP) of MoCo+TS on CIFAR10-LT.}%
  \label{table:periods}
}
\ffigbox[0.63\textwidth]{%
  \includegraphics[scale=0.3]{iclr2023/period.pdf}
}{%
  \caption{\textbf{Dependence on relative time of one period.} Blue: Average FS LP of last period of the models trained with $T=200,400, 1000, 2000$. Light blue: variance. Yellow: Relative cosine value over relative time. CIFAR10-LT trained with  MoCo+TS. }%
  \label{fig:periods}
}

\end{floatrow}
\end{figure}



\myparagraph{Alternatives to Cosine Schedule.}
Additionally, we test different methods of varying the temperature parameter $\tau$ and report the results in \cref{table:alternatives}: %As an alternative, 
we examine \rebuttal{a linearly oscillating (oscil) function}, a step function, and random sampling. % that we briefly discuss in \cref{subsec:CL_on_LT}. 
\rebuttal{For the linear oscillations, we follow the same schedule as for the cosine version, as shown on the right of \cref{table:alternatives}. } For the step function, we change $\tau$ from a low (0.1) to a high (0.5) value and back every 200 epochs. For random, we uniformly sample values for $\tau$ from the range [0.1, 0.5]. In \cref{table:alternatives} we observe that both those methods for varying the $\tau$ value also improve the performance over the fixed temperature, while with the cosine schedule the model achieves the best performance. These results indicate
%It confirms our hypothesis 
that it is indeed the \emph{task switching} between group-wise and instance-wise discrimination %changing the task
during training which is the driving factor for the observed improvements  %is beneficial 
for unsupervised long-tail representation learning. 
We assume the reason why slow oscillation of the temperature performs better than fast (\ie random) temperature changes is grounded in learning dynamics and the slow evolution of the embedding space during training. 
%



% \section{Discussion and Limitations}

Although we can ablate concepts efficiently for a wide range of object instances, styles, and memorized images, our method is still limited in several ways. First, while our method overwrites a target concept, this does not guarantee that the target concept cannot be generated through a different, distant text prompt. We show an example in \reffig{limitation} (a), where after ablating {\menlo Van Gogh}, the model can still generate {\menlo starry night painting}. However, upon discovery, one can resolve this by explicitly ablating the target concept {\menlo starry night painting}. Secondly, when ablating a target concept, we still sometimes observe slight degradation in its surrounding concepts, as shown in \reffig{limitation} (c). 

\nupur{Our method does not prevent a downstream user with full access to model weights from re-introducing the ablated concept~\cite{ruiz2022dreambooth,kumari2022multi,gal2022image}. Even without access to the model weights, one may be able to iteratively optimize for a text prompt with a particular target concept. Though that may be much more difficult than optimizing the model weights, our work does not guarantee that this is impossible.}

Nevertheless, we believe every creator should have an ``opt-out'' capability. We take a small step towards this goal, creating a computational tool to remove copyrighted images and artworks from large-scale image generative models.


\section{Conclusion}
\label{sec:conclusion}

In this work, we discover the surprising effectiveness of temperature schedules for self-supervised contrastive representation learning on
%in application to 
imbalanced datasets. % for self-supervised contrastive learning.
In particular, we find that a simple cosine schedule for $\tau$ consistently improves two state-of-the-art contrastive methods over several datasets and different imbalance ratios, without introducing any additional cost.

Importantly, our approach is based on a novel perspective on the contrastive loss, in which the average distance maximisation aspect is emphasised.
This perspective sheds light on which samples dominate the contrastive loss and explains why large values for $\tau$ can lead to the emergence of tight clusters in the embedding space, despite the fact that individual instance \emph{always} repel each other. 

Specifically, we find that while a large $\tau$ is thus necessary to induce semantic structure, the concomitant focus on \emph{group-wise} discrimination biases the model to encode easily separable features rather than instance-specific details. However, in long-tailed distributions, this can be particularly harmful to the most infrequent classes, as those require a higher degree of instance discrimination to remain distinguishable from the prevalent semantic categories. % \moritz{semantic categories maybe a bit too strong?}
The proposed cosine schedule for $\tau$ overcomes this tension, by alternating between an emphasis on instance discrimination (small $\tau$) and group-wise discrimination (large $\tau$). As a result of this constant `task switching', the model is trained to both structure the embedding space according to semantically meaningful features, whilst also encoding instance-specific details such that rare classes remain distinguishable from dominant ones.


 
\clearpage
\section*{Ethics Statement}
The paper proposes an analysis and a method to improve the performance of self-supervised representation learning methods based on the contrastive loss. The method and investigation in this paper do not introduce any ethical issues to the field of representation learning, as it is decoupled from the training data. Nonetheless, we would like to point out that representation learning does not automatically prevent models from learning harmful biases from the training data and should not be used outside of research applications without thorough evaluation for fairness and bias.

\section*{Acknowledgements}
C.\ R.\ is supported by VisualAI EP/T028572/1 and ERC-UNION-CoG-101001212.


\bibliography{iclr2023_conference}
\bibliographystyle{iclr2023_conference}

\clearpage
% \newpage
\appendix

\section*{\LARGE Appendix}


\section{Dataset Details}
\label{sec:dataset_details}

This section describes the details about the dataset we used in experiments (Section~\ref{sec:experiment}).

We use "tiny" version of Taskonomy dataset provided by \citep{taskonomy2018}, which consists of images and labels collected from 35 different buildings.
We use the train and val split for training and early-stopping, respectively, and use the "muleshoe" building included in the test split for evaluation.

To demonstrate our universal few-shot learner, we use ten dense prediction tasks in Taskonomy dataset~\citep{taskonomy2018}, which are semantic segmentation (SS), surface normal (SN), Euclidean distance (ED), Z-buffer depth (ZD), texture edge (TE), occlusion edge (OE), 2D keypoints (K2), 3D keypoints (K3), reshading (RS), and principal curvature (PC).
All labels are normalized into $[0, 1]$ with task-specific pre-processing.
For details on the pre-processing, we refer readers to \cite{taskonomy2018}.
Based on the annotations provided by Taskonomy, we preprocess some tasks to increase the diversity of tasks.
Specifically, we modify three single-channel tasks that can be easily augmented: Euclidean distance, texture edge, and occlusion edge.
\begin{enumerate}[leftmargin=0.5cm]
    \item 
    \textbf{Texture edge} (TE) labels are generated by applying Sobel edge detector~\citep{kanopoulos1988design} to RGB images, which consists of a Gaussian filter and image gradient computation.
    The Gaussian filter has two hyper-parameters, namely kernel size and the standard deviation, where adjusting those hyper-parameters yield different \emph{thickness} of detected edges.
    We use three different sets of hyper-parameters -- $(3, 1), (11, 2), (19, 3)$ -- to produce $3$-channel labels.
    We give an example of each channel of TE task in Figure~\ref{fig:texture_edge_augmentation}.
    
    \item
    \textbf{Euclidean distance} (ED) labels consists of pixel-wise depth map, where the depth is computed by the Euclidean distance from each image pixel to the camera's optical center.
    As this task is very similar to the Z-buffer depth prediction (ZD) whose label pixels are the distance from each image pixel to the camera plane, we augment the ED task by segmenting the depth range and re-normalizing within each segment.
    Specifically, we compute the $5$-quantiles of the pixel-wise depth labels in the whole dataset, then use each quantile as different channels after re-noramlization into $[0, 1]$.
    Thus the objective of each channel of the augmented ED task is to predict Euclidean distance within a specific range, where the ranges are disjoint for different channels.
    We give an example of each channel of ED task in Figure~\ref{fig:euclidean_distance_augmentation}.
    To visualize 5-channel labels, we average the first and the second channels as "R"-channel, the third and the fourth channels as "G"-channel, and use the fifth channel as "B"-channel.
    
    \item
    \textbf{Occlusion edge} (OE) labels are similar to texture edge, but they are constructed to depend on only the 3D geometry rather than color or lighting~\citep{taskonomy2018}.
    We observe that the channel augmentation by quantiles (that we apply to Euclidean distance task) can fairly diversify the labels.
    Therefore, we augment the OE labels into 5-channel labels, where we visualize them similar to the ED labels.
    We give an example of each channel of OE task in Figure~\ref{fig:occlusion_edge_augmentation}.
\end{enumerate}

Also, for semantic segmentation, we exclude three classes ("bottle", "toilet", "book"), as little images of the classes are included in the Taskonomy dataset.
The 12 classes we used in experiments are: "chair", "couch", "plant", "bed", "dining table", "tv", "mircrowave", "oven", "sink", "fridge", "clock", and "base".

\begin{figure}[ht!]
    \centering
    \includegraphics[width=0.8\textwidth]{figure_files/Texture_Edge_Augmentation.pdf}
    \caption{Channel augmentation on texture edge prediction (TE) task. We apply three different sets of hyper-parameters (kernel size, standard deviation) in Sobel edge detector to generate a 3-channel edge task.
    Second to Fourth columns show the augmented channel with different kernel size and standard deviation, where the last column shows the 3-channel label visualized as RGB.}
    \label{fig:texture_edge_augmentation}
\end{figure}
\begin{figure}[ht!]
    \centering
    \includegraphics[width=\textwidth]{figure_files/Euclidean_Distance_Augmentation.pdf}
    \caption{Channel augmentation on Euclidean distance prediction (ED) task. We compute 5-quantiles of the pixel-wise label distribution, and use each $p$-th 5-quantile as each channel after re-normalizing into $[0, 1]$.
    Second to Fifth columns show the augmented channel with different quantile, where the last column shows the 5-channel label visualized as RGB.}
    \label{fig:euclidean_distance_augmentation}
\end{figure}
\begin{figure}[ht!]
    \vspace{-0.2cm}
    \centering
    \includegraphics[width=\textwidth]{figure_files/Occlusion_Edge_Augmentation.pdf}
    \caption{Channel augmentation on occlusion edge prediction (OE) task. We compute 5-quantiles of the pixel-wise label distribution, and use each $p$-th 5-quantile as each channel after re-normalizing into $[0, 1]$.
    Second to Fifth columns show the augmented channel with different quantile, where the last column shows the 5-channel label visualized as RGB.}
    \label{fig:occlusion_edge_augmentation}
\end{figure}


\clearpage
\section{Implementation Details}
\label{sec:implementation_details}

This section describes the implementation details in our experiments (Section~\ref{sec:experiment}).

\subsection{Architecture Details of VTM}
\label{sec:arch-vtm}
\paragraph{Encoders and Decoders}
We employ BEiT-B architecture~\citep{bao2021beit} pretrained on Imagenet-22k dataset~\citep{deng2009imagenet} with $224 \times 224$ resolution as our image encoder.
For our label encoder and decoder, we follow the DPT-B architecture~\citep{ranftl2021vision}.
Specifically, we use a randomly initialized ViT-B~\citep{dosovitskiy2020image} as label encoder $g$ and extract features from $3, 6, 9, 12$-th layers of the encoder to form multi-level label features (label tokens).
Similarly, we extract multi-level image features (image tokens) from $3, 6, 9, 12$-th layers of the image encoder (BEiT).
As the DPT-B architecture decodes four-level features using RefineNet-based decoder~\citep{lin2017refinenet}, we pass the predicted query label features from matching module at each layer to the decoder.
As the label values of tasks in Taskonomy are normalized to $[0, 1]$, we use a sigmoid activation function at the head of the decoder to produce values in $[0, 1]$.
To predict semantic segmentation task whose label values are discrete (either $0$ or $1$), we discretize the predicted label with threshold $0.1$.

\paragraph{Matching Modules}
In the implementation of the matching module with multihead attention, we adopt three conventions in vision transformer~\citep{dosovitskiy2020image} which slightly modifies the equations described in Section~\ref{sec:architecture}.
Recall that the matching module is computed on three input matrices $\mathbf{q}\in\mathbb{R}^{M\times d}$ and $\mathbf{k},\mathbf{v}\in\mathbb{R}^{NM\times d}$ as follows:
\begin{align}
    \text{MHA}(\mathbf{q},\mathbf{k},\mathbf{v}) &= \text{Concat}(\mathbf{o}_1, ..., \mathbf{o}_H)w^O, \\
    \text{where }\mathbf{o}_h &= \text{Softmax}\left(\frac{\mathbf{q}w_h^Q(\mathbf{k}w_h^K)^\top}{\sqrt{d_H}}\right)\mathbf{v}w_h^V,
\end{align}
where $H$ is number of heads, $d_H$ is head size, and $w_h^Q,w_h^K,w_h^V\in\mathbb{R}^{d\times d_H}$, $w^O\in\mathbb{R}^{Hd_H\times d}$.
First, we perform layer normalization~\citep{ba2016layer} before each input projection matrices $w_h^Q, w_h^K, w_h^V$ and after the output projection matrix $w^O$, where we share the layer normalization parameters for $w_h^Q$ and $w_h^K$.
Second, we add a residual connection with GELU non-linearity~\citep{hendrycks2016gaussian} after gathering the outputs from multiple heads as follows:
\begin{align}
    \text{MHA}(\mathbf{q}, \mathbf{k}, \mathbf{v}) &= \mathbf{o} + \text{GELU}(\mathbf{o}w^O), \\
    \text{where}~\mathbf{o} &= \text{Concat}(\mathbf{o}_1, \mathbf{o}_2, \cdots, \mathbf{o}_H).
\end{align}
Finally, we apply Dropout~\citep{srivastava2014dropout} with rate 0.1 in the attention scores.

\subsection{Architecture Details of Baselines}
\label{sec:arch-baseline}
\paragraph{Encoders and Decoders}
For the supervised learning baselines based on transformer encoder (DPT and InvPT), we use the same encoder backbone with ours (BEiT pretrained on ImageNet-22k).
We use the decoder of DPT-B configuration in \cite{ranftl2021vision} for DPT as ours, and use the original multi-task decoder implementation provided by \cite{ye2022inverted} for InvPT.
For few-shot learning baselines (HSNet, VAT, DGPNet), we use ResNet-101~\citep{he2016deep} pretrained on ImageNet-1k~\citep{deng2009imagenet} as their encoder backbones, which is their best configuration.
For the other architectural details, we follow the original implementation of each method provided by \cite{min2021hypercorrelation} (HSNet), \cite{hong2022cost} (VAT), and \cite{johnander2021dense} (DGPNet).

\paragraph{Modification on Few-shot Baselines}
As HSNet and VAT are designed for semantic segmentation, we slightly modify their architectures to train them on general dense prediction tasks.
Specifically, both models involve a binary masking operation to filter out support image features using their labels (which are assumed to be binary), before computing 4D correlation tensor between support and query feature pixels.
For continuous labels of general dense prediction tasks, the binary masking becomes pixel-wise multiplication with labels.
However, as the correlation is computed by cosine similarity between feature pixels that is norm-invariant, all non-zero feature pixels with the same direction are treated in the same manner.
This make them unable to discriminate different non-zero label values, \emph{e.g.}, correlation between query and support feature pixels would be the same regardless of the assigned support label values. 
Therefore, we move the masking operation to after computing the cosine-similarity, so that the models can recognize different non-zero label values through different norms of the masked features by (non-binary) labels.

We use the DGPNet without modification as it is based on a regression method (Gaussian Processes) which is inherently applicable to general dense prediction tasks with continuous labels.


\subsection{Training Details}

\paragraph{Training}
We train all models with 300,000 iterations using the Adam optimizer~\citep{kingma2015adam}, and use \emph{poly} learning rate schedule~\citep{liu2015parsenet} with base learning rates $10^{-5}$ for pre-trained parameters and $10^{-4}$ for parameters trained from scratch.
The models are early-stopped based on the validation metric.
At each episodic training of iteration, we sample a batch of episodes with size 8.
In each episode, we construct a 5-channel task from the training tasks $\mathcal{T}_\text{train}$ by first splitting all channels of training tasks and randomly sample 5 channels among them.
Then support and query sets are sampled for the selected channels, where we use support and query size of 4 for Ours and DGP, while using 1 for HSNet and VAT as they only supports 1-shot training.
To train DPT, we construct a batch of each target task $\mathcal{T}_\text{test}$, whose channels are given at once, with batch size $64$.
To train InvPT, we construct a batch of all ten tasks, whose channels are all given at once, while using batch size $16$ due to its large memory consumption.

\paragraph{Data Augmentation}
We apply random crop (from $256 \times 256$ resolution to $224 \times 224$) and random horizontal flip to images, where the random horizontal flip is applied except for surface normal labels as their values are sensitive to the horizontal direction (flipping images and labels together changes the semantics of the task).
As we apply random crop during training, the resolution of test images ($256 \times 256$) differs from the training images.
To evaluate the models with consistent resolution, we perform five-crop (cropping the four corners and center of an image) to test query images so that the model also predicts five-cropped labels, then aggregate them by averaging the overlapping regions to produce final prediction for evaluation of resolution $(256 \times 256)$.
For few-shot models, we apply center crop to support images at test-time.

\paragraph{Task Augmentation}
For episodic training of few-shot models, we further apply two kinds of task augmentation.
First, for each channel of $C$-channel labels sampled at each episode ($C=5$ in our experiments), we apply random jittering and gaussian blur on each channel independently.
Then we apply MixUp~\citep{zhang2018mixup} on the augmented channels and auxiliary channels which are additionally sampled from the training tasks $\mathcal{T}_\text{train}$, to create a linearly interpolated label of two channels.
We apply the task augmentation consistently in each episode to preserve the task identity.


\clearpage
\section{Additional Results}
\label{sec:additional_results}

This section provides additional results on our experiments (Section~\ref{sec:experiment}).


\subsection{Additional Results on Ablation Study}
\label{sec:additional_results_on_ablation_study}

\subsubsection{Sensitivity to the Choice of Support Set}
\label{sec:support_set_sensitivity}
As discussed in Section~\ref{sec:experiment}, we evaluate the $10$-shot performance of our VTM with four different support sets that are disjointly sampled from the training data $\mathcal{D}_\text{train}$.
We report the results in Table~\ref{tab:support_set_choice}, which shows that our model is robust to the choice of support set.
We use the first support set ($\#1$) in Table~\ref{tab:support_set_choice} for comparison with other baselines or ablated variants in Section~\ref{sec:experiment}, due to the huge computational cost for evaluating few-shot baselines HSNet and VAT.

\begin{table}[ht]
\caption{Ablation study on the choice of support set. We disjointly sample four different support sets and report the $10$-shot performance on each set, with the mean and standard deviation.}
\label{tab:support_set_choice}
\begin{center}
    \renewcommand{\arraystretch}{1.5}
    \renewcommand{\aboverulesep}{0pt}
    \renewcommand{\belowrulesep}{0pt}
    \setlength\tabcolsep{2pt}
    \small
    \begin{tabular}{c|cc|cc|cc|cc|cc}
        \toprule
        \multirow{4}{*}{Support set} &
        \multicolumn{10}{c}{Tasks} \\
        
        \cmidrule{2-11}
        &
        \multicolumn{2}{c|}{Fold 1} & \multicolumn{2}{c|}{Fold 2} & \multicolumn{2}{c|}{Fold 3} & 
        \multicolumn{2}{c|}{Fold 4} & \multicolumn{2}{c}{Fold 5} \\
        
        \cmidrule{2-11}
        &
        SS & SN & ED & ZD & TE & OE & K2 & K3 & RS & PC \\
        &
        mIoU ↑ & mErr ↓ & RMSE ↓ & RMSE ↓ & RMSE ↓ & RMSE ↓ & RMSE ↓ & RMSE ↓ & RMSE ↓ & RMSE ↓ \\
        
        \midrule
        \# 1 &
        0.4097 & 11.4391 & 0.0741 & 0.0316 & 0.0791 & 
		0.0912 & 0.0639 & 0.0519 & 0.1089 & 0.0420 \\

        \# 2 &
        0.4190 & 11.8860 & 0.0845 & 0.0338 & 0.0839 & 
		0.0926 & 0.0629 & 0.0497 & 0.1131 & 0.0437 \\

        \# 3 &
        0.3781 & 11.7418 & 0.0776 & 0.0343 & 0.0807 & 
		0.0944 & 0.0656 & 0.0494 & 0.1101 & 0.0425 \\

        \# 4 &
        0.4017 & 11.6203 & 0.0794 & 0.0362 & 0.0799 & 
		0.0908 & 0.0672 & 0.0502 & 0.1158 & 0.0424 \\
		
	\midrule
        Mean &
        0.4021 & 11.6718 & 0.0789 & 0.0340 & 0.0809 & 
		0.0922 & 0.0649 & 0.0503 & 0.1120 & 0.0427 \\

        Std. &
        0.0152 & 0.1640 & 0.0038 & 0.0016 & 0.0018 & 
		0.0014 & 0.0016 & 0.0010 & 0.0027 & 0.0006 \\
		
        \bottomrule
        
    \end{tabular}
\end{center}
\end{table}


% \paragraph{Ablation Study on Training Procedure}
\subsubsection{Ablation Study on Training Procedure}
\label{sec:training_procedure}
To understand the source of the generalization performance of our method more clearly, we conduct an ablation study on training procedure.
We compare four models based on DPT architecture with different training procedures as follows.
\begin{itemize}[leftmargin=0.5cm]
    \item \textbf{M1}: Randomly initialized DPT, 10-shot trained.
    \item \textbf{M2}: DPT with BEiT pre-trained encoder, 10-shot fine-tuned.
    \item \textbf{M3} (Ours w/o Matching): DPT with BEiT pre-trained encoder, multi-task trained with task-specific bias tuning, and then 10-shot fine-tuned.
    \item \textbf{M4} (Ours): DPT with BEiT pre-trained encoder, meta-trained with task-specific bias tuning, and then 10-shot fine-tuned.

\end{itemize}

\begin{table}[ht]
\vspace{-0.2cm}
\caption{10-shot learning performance of ablated variants of DPT and Ours.}
\vspace{-0.2cm}
\label{tab:training_procedure_ablation}
\begin{center}
    \renewcommand{\arraystretch}{1.5}
    \renewcommand{\aboverulesep}{0pt}
    \renewcommand{\belowrulesep}{0pt}
    \setlength\tabcolsep{2pt}
    \small
    \begin{tabular}{c|cc|cc|cc|cc|cc}
        \toprule
        \multirow{4}{*}{Model} &
        \multicolumn{10}{c}{Tasks} \\
        
        \cmidrule{2-11}
        &
        \multicolumn{2}{c|}{Fold 1} & \multicolumn{2}{c|}{Fold 2} & \multicolumn{2}{c|}{Fold 3} & 
        \multicolumn{2}{c|}{Fold 4} & \multicolumn{2}{c}{Fold 5} \\
        
        \cmidrule{2-11}
        &
        SS & SN & ED & ZD & TE & OE & K2 & K3 & RS & PC \\
        &
        mIoU ↑ & mErr ↓ & RMSE ↓ & RMSE ↓ & RMSE ↓ & RMSE ↓ & RMSE ↓ & RMSE ↓ & RMSE ↓ & RMSE ↓ \\
        
        \midrule
        M1 &
        0.0644 & 21.0976 & 0.1959 & 0.0711 & 0.0995 & 
		0.1842 & 0.0670 & 0.0600 & 0.2335 & 0.0431 \\
		
        M2 &
        0.0582 & 15.8135 & 0.1615 & 0.0530 & 0.1136 & 
		0.1480 & 0.0948 & 0.0606 & 0.1858 & 0.0431 \\
        
        M3 &
        0.2681 & 13.0704 & 0.1111 & 0.0404 & \textbf{0.0778} & 
		0.1061 & \textbf{0.0613} & 0.0537 & 0.1559 & 0.0445 \\
        
        M4 &
        \textbf{0.4097} & \textbf{11.4391} & \textbf{0.0741} & \textbf{0.0316} & 0.0791 & 
		\textbf{0.0912} & 0.0639 & \textbf{0.0519} & \textbf{0.1089} & \textbf{0.0420} \\
        
        \bottomrule
        
    \end{tabular}
\end{center}
\vspace{-0.1cm}
\end{table}
\begin{figure}[ht]
    \centering
    \vspace{-0.3cm}
    \includegraphics[width=\textwidth]{figure_files/Fine-Tuning_Visualization.pdf}
    \caption{Qualitative comparison of Ours and its ablated variants in training procedure. All models use 10 labeled examples for each target task, where M3 and M4 observe additional labeled examples of training tasks (different from the target task) in each fold.
    }
    \label{fig:training_procedure_qualitative}
    \vspace{-0.3cm}
\end{figure}

We summarize the quantitative result in Table~\ref{tab:training_procedure_ablation} and qualitative comparison in Figure~\ref{fig:training_procedure_qualitative}.
First, as expected, we observe that DPT with naive 10-shot training (M1) fails to generalize to the test examples in most of the tasks, except for two 2D texture-related tasks (TE, K2). We conjecture that TE and K2 are “easy” cases in terms of few-shot learning, as they are defined as low-level computational algorithms on RGB images, while other high-level tasks require knowledge about semantics (SS) or 3D space (SN, ED, ZD, OE, K3, RS, PC).
Second, we note that BEiT pretraining (M2) largely improves the few-shot generalization performance, allowing the model to produce coarse predictions of the dense labels. However, it still cannot capture object-level fine-grained details in many tasks.
Third, we observe that multi-task training and few-shot adaptation, combined with an efficient parameter-sharing strategy of bias tuning (M3, M4), further improves the performance with a clear gap with M2 where the predictions are also qualitatively finer than M2’s.
Finally, as discussed in Section~\ref{sec:ablation_study}, M4 still further improves over M3 with a clear gap. This shows that in a few-shot learning setting, our matching framework and episodic training are more effective than simple multi-task pretraining employed in M3.
In summary, we may conclude that the fast generalization of Ours is benefitted from episodic training of various tasks followed by parameter-efficient few-shot adaptation as well as powerful pre-training of the encoder (BEiT).



\subsubsection{Fine-tuning with Full Supervision}
To further explore how our method scales well when a large labeled dataset is given, we also fine-tuned our VTM with full supervision of test tasks.
For the fine-tuning, we used the same training dataset as the fully-supervised DPT and employed the episodic fine-tuning objective (Section 3.3). For evaluation, since providing the entire training data as the support set for the matching module is infeasible, we provide a random subset of the training data as the support set to the model.
We summarize the result in Figure~\ref{fig:performance_on_shots_with_full}, which extends Figure~\ref{fig:performance_on_shots} in Section~\ref{sec:experiment}.
In most tasks, our model consistently improves when more supervision is given.
With full supervision at test tasks, our model performs slightly worse than the DPT baseline in seven tasks and performs better or similarly in the other three tasks.
We conjecture that the performance degradation comes from two aspects: (1) the absence of direct input-output connection, \emph{i.e.}, the matching module serves as a bottleneck, and (2) negative transfer from meta-training tasks to test tasks.

\begin{figure}[ht!]
    \centering
    \vspace{-0.3cm}
    \includegraphics[width=\textwidth]{figure_files/Performance_on_Shots_v3.pdf}
    \caption{Performance of VTM on various shots.
    In general, VTM consistently improves performance as more supervision is given, and even surpasses fully supervised baselines on many tasks.
    }
    \label{fig:performance_on_shots_with_full}
    \vspace{-0.3cm}
\end{figure}

\subsubsection{Effect of Number of Training Tasks}
\label{sec:number_of_training_tasks}
The amount of meta-training tasks is an important factor that can affect the performance of the universal few-shot learner.
To verify this, we fixed two test tasks (SS, SN) and trained our VTM on five different subsets of the original eight training tasks (three different subsets with two tasks and two different subsets with five tasks).
We summarize the results in the Table~\ref{tab:training_tasks_ablation}.
As expected, the performance consistently improves as we increase the number of training tasks.
We also note that the few-shot performance becomes sensitive to the choice of training tasks when their number is small (two), presumably as the model becomes reliant on training tasks more correlated to test tasks, while the variance decreases substantially when more training tasks are added.
In addition, the experiment with incomplete training data (Appendix~\ref{sec:incomplete_experiment}) shows the potential ability of our methods in more realistic settings where the training dataset is formed by a combination of different task-specific datasets.
From these results, we expect that our model can further enhance its universality on few-shot learning by utilizing a combined training dataset of much more diverse tasks, which we leave as future work.

\begin{table}[ht]
\caption{10-shot learning performance of Ours with various number of training tasks.}
\vspace{-0.2cm}
\label{tab:training_tasks_ablation}
\begin{center}
    \renewcommand{\arraystretch}{1.5}
    \renewcommand{\aboverulesep}{0pt}
    \renewcommand{\belowrulesep}{0pt}
    \setlength\tabcolsep{6pt}
    \small
    \begin{tabular}{c|cc}
        \toprule
        \multirow{4}{*}{Number of Training Tasks} &
        \multicolumn{2}{c}{Tasks} \\
        
        \cmidrule{2-3}
        &
        \multicolumn{2}{c}{Fold 1} \\
        
        \cmidrule{2-3}
        &
        SS & SN \\
        &
        mIoU ↑ & mErr ↓ \\
        
        \midrule
        2 &
        0.2878 ± 0.0565 & 17.3947 ± 4.8742 \\
		
        5 &
        0.3919 ± 0.0132 & 12.6769 ± 0.1235 \\
        
        8 &
        \textbf{0.4097} & \textbf{11.4391} \\
        
        \bottomrule
        
    \end{tabular}
\vspace{-0.2cm}
\end{center}
\end{table}


\subsubsection{Episodic Training with Incomplete Dataset}
\label{sec:incomplete_experiment}
It would make our method more practical if the model could learn from an incomplete dataset where images are not associated with whole training task labels.
To see how our framework extends to such incomplete settings, we conducted an additional experiment.
We simulate the extreme case of incomplete data by partitioning the training images, such that each image is associated with only a single task out of 8 training tasks.
Specifically, we partitioned the buildings in Taskonomy into eight groups – each corresponds to a different training task.
As this reduces the effective size of training data by the number of training tasks (1/8 in our case), we also train a baseline where we use complete data but use only 1/8 of the training images (for each building, we discard 7/8 of the images).
The results are summarized in Table~\ref{tab:incomplete_dataset}.
We can see that the performance degradation is marginal when we give incomplete data, which implies that our method can be promising in handling realistic scenarios where the training data is a collection of heterogeneous datasets with different label annotations.

\begin{table}[ht]
\caption{10-shot learning performance of Ours trained with incomplete and complete multi-task dataset.}
\vspace{-0.2cm}
\label{tab:incomplete_dataset}
\begin{center}
    \renewcommand{\arraystretch}{1.5}
    \renewcommand{\aboverulesep}{0pt}
    \renewcommand{\belowrulesep}{0pt}
    \setlength\tabcolsep{6pt}
    \small
    \begin{tabular}{c|cc}
        \toprule
        \multirow{4}{*}{Training Data} &
        \multicolumn{2}{c}{Tasks} \\
        
        \cmidrule{2-3}
        &
        \multicolumn{2}{c}{Fold 1} \\
        
        \cmidrule{2-3}
        &
        SS & SN \\
        &
        mIoU ↑ & mErr ↓ \\
        
        \midrule
        Incomplete (one task per building) &
        0.3559 & 13.6207 \\

        Complete (1/8 training data) &
        0.3980 & 12.1633 \\
        
        Complete (whole training data) &
        0.4097 & 11.4391 \\
        
        \bottomrule
        
    \end{tabular}
\vspace{-0.2cm}
\end{center}
\end{table}


\subsection{Further Analysis}

\subsubsection{Parameter-Efficiency Analysis}
We report the number of task-specific and shared parameters of our VTM and two supervised baselines, DPT and InvPT, to compare how our task adaptation is parameter-efficient.
As DPT is a single-task learning model, no parameters are shared across tasks and the whole network should be trained independently for every new task.
InvPT, which is a multi-task learning model, shares a large portion of its parameters across tasks (\emph{e.g.}, encoder backbone), still consumes many parameters for each task in the decoder.

Due to the extensive amount of parameter-sharing, our method is also promising in continual learning setting.
As all task-specific knowledge is included in the bias parameters of the image encoder, the knowledge acquired from past tasks can be recalled without forgetting by keeping the corresponding bias parameters and switching to them whenever a past model is needed.
We especially note that the size of bias parameters is fairly small (288 KB, which amounts to keeping about 3 labeled images of 256x256 resolution for each task).
This allows our model to retain past knowledge very efficiently by keeping the tuned bias parameters plus a few-shot support set, whose external memory requirement is far less compared to memory-based approaches in continual learning that keep hundreds of images~\citep{bang2021rainbow,wang2022continual}.
While the continual learning setting is not our main focus, applying our method to a continual learning setting would be an interesting future direction.

\begin{table}[ht]
\caption{Number of task-specific and shared parameters for a single-channel task (in million).}
\vspace{-0.2cm}
\label{tab:number_of_parameters}
\begin{center}
    \renewcommand{\arraystretch}{1.5}
    \small
    \begin{tabular}{cccc}
        \toprule
        Model & Task-Specific & Shared \\
        \midrule
        DPT (supervised learning) & 110.55 & 0 \\
        InvPT (multi-task learning) & 24.57 & 106.75 \\
        Ours (few-shot learning) & 0.0703 & 202.95 \\
        \bottomrule
        
    \end{tabular}
\vspace{-0.2cm}
\end{center}
\end{table}

\subsubsection{Computation Cost Analysis}
To analyze how our method is computationally efficient compared to supervised DPT, we measured the MACs (multiply–accumulate operations) of our model and DPT using an open-source python library thop~\footnote{https://github.com/Lyken17/pytorch-OpCounter}.
We report the results in Table~\ref{tab:computation_cost}.
Having encoded the support set (e.g., 10-shot), we can see that the computational cost of our model’s inference on a single query image is about 30\% larger than the cost of DPT’s, due to the Matching part.

\begin{table}[ht]
\caption{MACs of Ours and DPT on a single-query inference for a single-channel task.}
\vspace{-0.2cm}
\label{tab:computation_cost}
\begin{center}
    \renewcommand{\arraystretch}{1.5}
    \small
    \begin{tabular}{ccc}
        \toprule
        Model & MACs (G) \\
        \midrule
        DPT & 30.15 \\
        Ours after encoding support (10-shot) & 38.79 \\
        \bottomrule
        
    \end{tabular}
\vspace{-0.2cm}
\end{center}
\end{table}

\subsubsection{Role of Attention Heads}
To analyze the role of attention heads, in Figure~\ref{fig:multihead_attention}, we visualized the attention maps for each head over support images for a given query patch, feature level (3rd level in this example), and task (RS in this example).
The figure shows that each head attends to different regions of the support images.
Moreover, we can find some patterns in heads; for example, the first head tends to attend to flat areas of the scene, such as the floor or ceiling (low-frequency features), while the third head tends to attend to objects, such as couch or plant (high-frequency features).
To further verify the benefit of multi-head attention in the matching module, we also trained our VTM with single head in the matching modules.
The result is summarized in the table below and Table~\ref{tab:attention_heads}.
We can see the performance drop in both SS and SN tasks, which supports that exploiting multiple heads benefits our matching framework.

\begin{figure}[ht]
    \centering
    \includegraphics[width=\textwidth]{figure_files/Multihead_Attention_Visualization.pdf}
    \caption{Visualization of multi-head attention maps of VTM. Here we visualize the matching module at 3rd level for reshading (RS) task.
    }
    \label{fig:multihead_attention}
\end{figure}
\begin{table}[ht]
\caption{10-shot learning performance of Ours with different number of attention heads in Matching module.}
\label{tab:attention_heads}
\begin{center}
    \renewcommand{\arraystretch}{1.5}
    \renewcommand{\aboverulesep}{0pt}
    \renewcommand{\belowrulesep}{0pt}
    \setlength\tabcolsep{6pt}
    \small
    \begin{tabular}{c|cc}
        \toprule
        \multirow{4}{*}{Number of Attention Heads} &
        \multicolumn{2}{c}{Tasks} \\
        
        \cmidrule{2-3}
        &
        \multicolumn{2}{c}{Fold 1} \\
        
        \cmidrule{2-3}
        &
        SS & SN \\
        &
        mIoU ↑ & mErr ↓ \\
        
        \midrule
        1 &
        0.3702 & 12.5936 \\
        
        4 &
        \textbf{0.4097} & \textbf{11.4391} \\
        
        \bottomrule
        
    \end{tabular}
\end{center}
\end{table}


\clearpage
\subsection{Additional Qualitative Comparison with Baselines}
\label{sec:additional_qualitative_comparison_with_baselines}

We provide additional results on the qualitative evaluation of our model and the baselines.
Figure~\ref{fig:appendix_comparison_1}-\ref{fig:appendix_comparison_4} show visualizations on different query image and support set, where we vary the class of semantic segmentation task included in each support.
The result shows consistent trends of that we discussed in Section~\ref{sec:experiment}.
Ours is competitive to the fully supervised baselines (DPT and InvPT), while the other few-shot baselines (HSNet, VAT, DGPNet) fail to learn different dense prediction tasks.


In Figure~\ref{fig:appendix_comparison_2}, even the GT label for semantic segmentation ("couch" class) is noisy as it is a pseudo-label generated by a pre-trained segmentation model~\citep{taskonomy2018}, our model successfully segments two couches present in the figure.
This can be attributed to the task-agnostic architecture of VTM based on non-parametric matching.

\begin{figure}[ht]
    \centering
    \includegraphics[width=\textwidth]{figure_files/Appendix_Comparison_1.pdf}
    \caption{Additional results of qualitative comparison between Ours and the baselines.
    }
    \label{fig:appendix_comparison_1}
\end{figure}

\begin{figure}[ht]
    \centering
    \includegraphics[width=\textwidth]{figure_files/Appendix_comparison_2.pdf}
    \caption{Additional results of qualitative comparison between Ours and the baselines.
    }
    \label{fig:appendix_comparison_2}
\end{figure}

\begin{figure}[ht]
    \centering
    \includegraphics[width=\textwidth]{figure_files/Appendix_comparison_3.pdf}
    \caption{Additional results of qualitative comparison between Ours and the baselines.
    }
    \label{fig:appendix_comparison_3}
\end{figure}

\begin{figure}[ht]
    \centering
    \includegraphics[width=\textwidth]{figure_files/Appendix_comparison_4.pdf}
    \caption{Additional results of qualitative comparison between Ours and the baselines.
    }
    \label{fig:appendix_comparison_4}
\end{figure}


\clearpage
\subsection{Additional Qualitative Comparison with Our Variants}
\label{sec:additional_qualitative_comparison_with_our_variants}

We also provide additional results on the qualitative evaluation of our model and our ablated variants, Ours w/o Matching and Ours w/o Adaptation.
Figure~\ref{fig:appendix_ablation_1}-\ref{fig:appendix_ablation_4} show visualizations on different query image and support set. 
The results show a consistent trend with the quantitative results in Table~\ref{tab:main_table}.
Interestingly, our method without adaptation already exhibits some degree of adaptation to the unseen tasks even without fine-tuning and task-specific components, showing that the non-parametric architecture of our model and the parameter sharing derived from is appropriate to learn generalizable knowledge to understand the novel tasks.
On the other hand, adding a task-specific component and adaptation mechanism to the model allows more dramatic improvement in understanding novel tasks from few-shot examples, showing the importance of the adaptation mechanism in our task.
Finally, we observe that equipping the matching mechanism with the adaptation module provides much sharper and fast adaptation to the unseen tasks, which verifies our claims.


\begin{figure}[ht]
    \centering
    \includegraphics[width=\textwidth]{figure_files/Appendix_abaltion_1.pdf}
    \caption{Additional results of qualitative comparison between Ours and its ablated variants.
    }
    \label{fig:appendix_ablation_1}
\end{figure}

\begin{figure}[ht]
    \centering
    \includegraphics[width=\textwidth]{figure_files/Appendix_abaltion_2.pdf}
    \caption{Additional results of qualitative comparison between Ours and its ablated variants.
    }
    \label{fig:appendix_ablation_2}
\end{figure}

\begin{figure}[ht]
    \centering
    \includegraphics[width=\textwidth]{figure_files/Appendix_abaltion_3.pdf}
    \caption{Additional results of qualitative comparison between Ours and its ablated variants.
    }
    \label{fig:appendix_ablation_3}
\end{figure}

\begin{figure}[ht]
    \centering
    \includegraphics[width=\textwidth]{figure_files/Appendix_abaltion_4.pdf}
    \caption{Additional results of qualitative comparison between Ours and its ablated variants.
    }
    \label{fig:appendix_ablation_4}
\end{figure}
\appendix
\section{Appendix}
\label{sec:appendix}

\subsection{Pseudo-Code for reproducibility of cosine schedule}
\begin{algorithm}
\caption{Cosine Schedule}\label{alg:cap}
\begin{algorithmic}
\Require period $T \geq 0$, $\tau_- = 0.1, \tau_+ = 1.0$
\State $ep \gets $ current epoch 
\State $tau \gets (\tau_+-\tau_-) \times (1+$np.cos$(2\times$np.pi$\,\times ep/T)) / 2 + \tau_-$
\end{algorithmic}
\end{algorithm}


Insert \cref{alg:cap} into your favourite contrastive learning framework to check it out!

\subsection{Implementation Details}
\label{appendix:implementation_details}

\myparagraph{Evaluation details.}
Following~\citet{jiang2021self}, we separate 5000 images for CIFAR10/100-LT as a validation set for each split. As we discussed in the main paper,
% \cref{subsec:ablations}, 
the performance of the model depends on the relative position within a period $T$. Therefore we utilise the validation split to choose a checkpoint for further testing on the standard test splits for CIFAR10/100-LT. Precisely, for each dataset, we select the evaluation epoch for the checkpoint based only on the validation set of the first random split; the other splits of the same dataset are evaluated using the same number of epochs. Note that for ImageNet100-LT there is no validation split and we select the last checkpoint as in~\citet{jiang2021self}. For a fair comparison, we also reproduce the numbers from~\cite{jiang2021self} in the same way. 


\myparagraph{Division into head, mid, and tail classes.}
Following~\citet{jiang2021self}, we divide all the classes into three categories: head classes are with the most number of samples, tail classes are with the least number of samples and mid are the rest. In particular, for CIFAR10-LT for each split there are 4 head classes, 3 mid classes, and 3 tail classes; for CIFAR100-LT there are 34 head classes, 33 mid classes, 33 tail classes; for ImageNet100-LT head classes are classes with more than 100 instances, tail classes have less than 20 instances per class, and mid are the rest. 

\subsection{Extended results}
\label{sec:extend_results}


\myparagraph{Extension of \cref{fig:spheres}}
In \cref{fig:supervised_tau} we provide full results of kNN accuracy on CIFAR10 when the model is trained with different fixed $\tau$ values and with coarse binary supervision. Especially tail classes are improved by instance discrimination (small $\tau_\mathrm{tail}$). 
%

\begin{figure}[!h]
\begin{center}
%\framebox[4.0in]{$\;$}
\includegraphics[scale=0.3]{iclr2023/fig7.pdf}
\end{center}
\caption{kNN accuracy for CIFAR10-LT trained with MoCo. Comparison between $\tau=0.1$, $\tau=0.5$, $\tau=1.0$. [0.1, 1.0] denotes coarse binary supervision with $\tau_\mathrm{head}=1.0$ and $\tau_\mathrm{tail}=0.1$. MAA: mean average accuracy over all classes.}
\label{fig:supervised_tau}
\end{figure}

% 

\myparagraph{Head-mid-tail classes evaluation.}
In the following, we present a detailed comparison of SimCLR and SimCLR+TS on head, mid, and tail classes on CIFAR10-LT in \cref{table:cifar10-head-mid-tail}, on CIFAR100-LT in \cref{table:cifar100-head-mid-tail} and on ImageNet100-LT in \cref{table:imagenet100-head-mid-tail}. We observe consistent improvement for all evaluation metrics for all types of classes over the three datasets. 

% \input{iclr2023/tables/appendix_cifar10.tex}

\begin{table}[!h]
\centering
\small
\tabcolsep=0.15cm
\begin{tabular}{c|cccccc|cccccc}  % 4 = method, model, gn(5/69), gb(64/69), hm
\toprule
& \multicolumn{12}{c}{CIFAR-10-LT}  \\
& \multicolumn{6}{c|}{kNN@1} &  \multicolumn{6}{c}{kNN@10}  \\
method  & Head & & Mid & & Tail & &  Head & & Mid & & Tail & \\
% \multirow{2}{*}{$\lambda$}  & $N_{/J}$  & $B_{/J}$  & \multirow{2}{*}{$hm_{/J}$} \\
% &  (5/69) & (64/69) &  \\
\midrule
SimCLR & 84.93 & \scriptsize{$\pm$ 3.44} &  54.08 & \scriptsize{$\pm$ 4.24} &  32.14 & \scriptsize{$\pm$ 7.44} &  88.03  & \scriptsize{$\pm$ 3.32} &  53.76  & \scriptsize{$\pm$ 4.80}  &   29.52  & \scriptsize{$\pm$ 9.44} \\
SimCLR + TS & 87.24 & \scriptsize{$\pm$ 3.05} & 58.96  & \scriptsize{$\pm$ 5.21} & 35.02  & \scriptsize{$\pm$ 8.27} &  89.92  & \scriptsize{$\pm$ 2.97} &  59.31  & \scriptsize{$\pm$ 4.69}  &   30.51  & \scriptsize{$\pm$ 12.38} \\
\midrule
& \multicolumn{6}{c|}{FS LP} &  \multicolumn{6}{c}{LT LP}  \\
method  & Head & & Mid & & Tail & &  Head & & Mid & & Tail & \\
\midrule
SimCLR  &  76.38 & \scriptsize{$\pm$ 5.24} & 63.20	& \scriptsize{$\pm$ 2.95} & 62.60  & \scriptsize{$\pm$ 3.63} &  89.52 & \scriptsize{$\pm$ 3.15} & 56.98  & \scriptsize{$\pm$ 4.74}  &   29.88  & \scriptsize{$\pm$ 8.11} \\
SimCLR + TS &  80.54  & \scriptsize{$\pm$ 5.02} & 66.50   & \scriptsize{$\pm$ 4.38} & 65.67  & \scriptsize{$\pm$ 4.07} &  91.73  & \scriptsize{$\pm$ 2.49} &  62.09  & \scriptsize{$\pm$ 4.21}  &  32.38   & \scriptsize{$\pm$ 9.23} \\

\bottomrule
\end{tabular}

\caption{Detailed evaluation on CIFAR10-LT. Evaluation metrics: kNN@{1,10}, FS LP states for few-shot linear probe, and LT LP states for long-tail linear probe. We report the average performance with the standard deviation over three different random splits for different sets of classes: head, mid, and tail.  }
\label{table:cifar10-head-mid-tail}
% \vspace{-mm}
\end{table}


% \input{iclr2023/tables/appendix_cifar100.tex}

\begin{table}[!h]
\centering
\small
\tabcolsep=0.15cm
\begin{tabular}{c|cccccc|cccccc}  % 4 = method, model, gn(5/69), gb(64/69), hm
\toprule
& \multicolumn{12}{c}{CIFAR-100-LT}  \\
& \multicolumn{6}{c|}{kNN@1} &  \multicolumn{6}{c}{kNN@10}  \\
method  & Head & & Mid & & Tail & &  Head & & Mid & & Tail & \\
% \multirow{2}{*}{$\lambda$}  & $N_{/J}$  & $B_{/J}$  & \multirow{2}{*}{$hm_{/J}$} \\
% &  (5/69) & (64/69) &  \\
\midrule
SimCLR & 53.87 & \scriptsize{$\pm$  2.12} &  24.56 & \scriptsize{$\pm$ 1.51} &  7.26 & \scriptsize{$\pm$ 0.39} &  58.46  & \scriptsize{$\pm$ 1.79} &  22.15  & \scriptsize{$\pm$ 1.47}  &   2.83  & \scriptsize{$\pm$ 0.61} \\
SimCLR + TS & 57.14 & \scriptsize{$\pm$ 1.95} &  26.00 & \scriptsize{$\pm$ 1.20} &  8.31 & \scriptsize{$\pm$ 0.57} &  61.93  & \scriptsize{$\pm$ 1.88} &  24.22  & \scriptsize{$\pm$ 2.23}  &  3.05   & \scriptsize{$\pm$ 0.54} \\
\midrule
& \multicolumn{6}{c|}{FS LP} &  \multicolumn{6}{c}{LT LP}  \\
method  & Head & & Mid & & Tail & &  Head & & Mid & & Tail & \\
\midrule
SimCLR  & 33.48 & \scriptsize{$\pm$ 1.24} &  24.25 & \scriptsize{$\pm$ 2.12} &  19.12 & \scriptsize{$\pm$ 1.35} &  62.19  & \scriptsize{$\pm$ 1.80} &  26.56  & \scriptsize{$\pm$ 1.46}  &  3.92   & \scriptsize{$\pm$ 0.46} \\
SimCLR + TS & 37.5 & \scriptsize{$\pm$ 1.33} & 27.64  & \scriptsize{$\pm$ 1.95} & 21.26  & \scriptsize{$\pm$ 0.66} &  65.24  & \scriptsize{$\pm$ 2.04} &  29.20  & \scriptsize{$\pm$ 1.48}  &   4.42  & \scriptsize{$\pm$ 0.26} \\

\bottomrule
\end{tabular}

\caption{Detailed evaluation on CIFAR100-LT. Evaluation metrics: kNN@{1,10}, FS LP states for few-shot linear probe, and LT LP states for long-tail linear probe. We report the average performance with the standard deviation over three different random splits for different sets of classes: head, mid, and tail.}
\label{table:cifar100-head-mid-tail}
% \vspace{-mm}
\end{table}



% \input{iclr2023/tables/appendix_imagenet100.tex}

\begin{table}[!h]
\centering
\small
\tabcolsep=0.15cm
\begin{tabular}{c|cccccc|cccccc}  % 4 = method, model, gn(5/69), gb(64/69), hm
\toprule
& \multicolumn{12}{c}{ImageNet100-LT}  \\
& \multicolumn{6}{c|}{kNN@1} &  \multicolumn{6}{c}{kNN@10}  \\
method  & Head & & Mid & & Tail & &  Head & & Mid & & Tail & \\
% \multirow{2}{*}{$\lambda$}  & $N_{/J}$  & $B_{/J}$  & \multirow{2}{*}{$hm_{/J}$} \\
% &  (5/69) & (64/69) &  \\
\midrule
SimCLR      & 55.13 & \scriptsize{} & 30.00  & \scriptsize{} & 10.71  & \scriptsize{} &  58.51  & \scriptsize{} &  29.70  & \scriptsize{}  &  8.71   & \scriptsize{} \\
SimCLR + TS & 57.23 & \scriptsize{} &  30.26 & \scriptsize{} & 13.14  & \scriptsize{} &  60.41  & \scriptsize{} &  29.53  & \scriptsize{}  &  10.14   & \scriptsize{} \\
\midrule
& \multicolumn{6}{c|}{FS LP} &  \multicolumn{6}{c}{LT LP}  \\
method  & Head & & Mid & & Tail & &  Head & & Mid & & Tail & \\
\midrule
SimCLR      & 51.79 & \scriptsize{} &  36.77 & \scriptsize{} & 30.29 & \scriptsize{} &  67.59  & \scriptsize{} &  36.47  & \scriptsize{}  &   9.43  & \scriptsize{} \\
SimCLR + TS & 60.41 & \scriptsize{} &  40.38 & \scriptsize{} &  33.57 & \scriptsize{} &  70.67  & \scriptsize{} &   38.85 & \scriptsize{}  &   10.29  & \scriptsize{} \\

\bottomrule
\end{tabular}

\caption{Detailed evaluation on ImageNet100-LT. Evaluation metrics: kNN@{1,10}, FS LP states for few-shot linear probe, and LT LP states for long-tail linear probe. We report the average performance for different sets of classes: head, mid, and tail.}
\label{table:imagenet100-head-mid-tail}
% \vspace{-mm}
\end{table}

% \input{iclr2023/tables/uniform_vs_long-tail.tex}

\begin{table}[!ht]
\centering
\small
\tabcolsep=0.15cm
\begin{tabular}{c|cc|cc|cc|cc}  % 4 = method, model, gn(5/69), gb(64/69), hm
\toprule
& \multicolumn{4}{c}{CIFAR10-Uniform}  & \multicolumn{4}{c}{CIFAR10-LT}  \\
method  & kNN@1 & kNN@10 & FS LP & LT LP & kNN@1 & kNN@10 & FS LP & LT LP  \\
% \multirow{2}{*}{$\lambda$}  & $N_{/J}$  & $B_{/J}$  & \multirow{2}{*}{$hm_{/J}$} \\
% &  (5/69) & (64/69) &  \\
\midrule
MoCo  & 83.47 & 84.87  & \textbf{90.19} & \textbf{87.70} & 63.00 & 64.10  & 68.89  & 63.99    \\

MoCo + TS   &  \textbf{83.78} &  \textbf{85.85} &  90.02 &  87.40  &  \textbf{65.68} &  \textbf{65.91} &  \textbf{72.31} &  \textbf{66.64}  \\

\bottomrule
\end{tabular}

\caption{\textbf{Influence of TS on uniform vs long-tailed distribution.} Comparison of MoCo vs MoCo+TS  on CIFAR10-Uniform and CIFAR-LT-imb100, one split.  Evaluation metrics: kNN classifier, FS LP denotes few-shot linear probe, LT LP denotes long-tail linear probe. 
}
\label{table:uniform-vs-ong-tail}
\end{table}


\rebuttal{\myparagraph{Influence of TS on Uniform vs Long-Tailed Distributions.}
To further corroborate that TS particularly helpful for imbalanced data, we apply TS for the uniformly distributed data. 
In \cref{table:uniform-vs-ong-tail}, we can observe that the cosine schedule yields significant and consistent gains for the long-tailed version of CIFAR10 (CIFAR10-LT), but not for the uniform one (CIFAR10-Uniform). 
We assume that both head classes and tail classes for long-tail distribution should be expected to benefit from a better separation between the two: on the one hand, the tail classes form better clusters and are thus easier to classify based on their neighbours, on the other hand, the clusters of the head classes are 'purified', which should similarly improve performance. Weather, for the uniform distribution, we do not observe such influence of TS and the performance changes only marginally. 
}




% \input{iclr2023/6_appendix_positives}
% 
\subsection{Influence of the positive samples on contrastive learning}
\label{appendix:positives}
In \cref{subsec:max_margin}, we particularly focused on the impact of the \emph{negative samples} on the learning dynamics under the contrastive objective, as they likely are the driving factor with respect to the semantic structure. In fact, we find that the positive samples should have an inverse relation with the temperature $\tau$ and thus cannot explain the observed learning dynamics, as we discuss in the following.
%. In the following, we additionally discuss how the positive samples affect the representation learning via the factor $c_{ii}$ in \cref{eq:margin_loss}.

To understand the impact of the \emph{positive samples}, first note their role in the loss (same as \cref{eq:margin_loss}): 
\begin{equation}
    \label{eq:margin_loss_pos}
    \mathcal L^i_\text{c} 
    = \log\left(1+c_{ii}S_i\right) \, .
\end{equation}
In particular, $c_{ii}$ scales the entire sum $S_i\myeq\sum_{j\neq i}\exp (-d_{ij})$. As such, encoding two augmentations of the same instance at a large distance is much more `costly' for the model than encoding two different samples close to each other, as each and every summand $S_i$ is amplified by the corresponding $c_{ii}$. As a result, the model will be biased to `err on the safe side' and become invariant to the augmentations, which has been one of the main motivations for introducing augmentations in contrastive learning in the first place, cf.~\cite{tian2020makes, chen2020simple, caron2020unsupervised}.


Consequently, the positive samples, of course, also influence the forming of clusters in the embedding space as they induce invariance with respect to augmentations. Note, however, that this does not contradict our analysis regarding the impact of negative samples, but rather corroborates it. 

In particular, $c_{ii}$ biases the model to become invariant to the applied augmentations for all values of $\tau$; in fact, for small $\tau$, this invariance is even emphasised as $c_{ii}$ increases for small $\tau$ and the influence of the negatives is diminished.
Hence, if the augmentations were the main factor in inducing semantic structure in the embedding space, $\tau$ should have the opposite effect of the one we and many others \citep{wang2021understanding, zhang2022dual, zhang2021temperature} observe. 

Thus, instead of inducing semantic structure on their own, we believe the positive samples to rather play a critical role in influencing which features the model can rely on for grouping samples in the embedding space; for a detailed discussion of this phenomenon, see also \cite{chen2021intriguing}.



\end{document}



