%% 
%% Copyright 2007-2020 Elsevier Ltd
%% 
%% This file is part of the 'Elsarticle Bundle'.
%% ---------------------------------------------
%% 
%% It may be distributed under the conditions of the LaTeX Project Public
%% License, either version 1.2 of this license or (at your option) any
%% later version.  The latest version of this license is in
%%    http://www.latex-project.org/lppl.txt
%% and version 1.2 or later is part of all distributions of LaTeX
%% version 1999/12/01 or later.
%% 
%% The list of all files belonging to the 'Elsarticle Bundle' is
%% given in the file `manifest.txt'.
%% 

%% Template article for Elsevier's document class `elsarticle'
%% with numbered style bibliographic references
%% SP 2008/03/01
%%
%% 
%%
%% $Id: elsarticle-template-num.tex 190 2020-11-23 11:12:32Z rishi $
%%
%%
\documentclass[preprint,12pt]{elsarticle}

%% Use the option review to obtain double line spacing
%% \documentclass[authoryear,preprint,review,12pt]{elsarticle}

%% Use the options 1p,twocolumn; 3p; 3p,twocolumn; 5p; or 5p,twocolumn
%% for a journal layout:
%% \documentclass[final,1p,times]{elsarticle}
%% \documentclass[final,1p,times,twocolumn]{elsarticle}
%% \documentclass[final,3p,times]{elsarticle}
%% \documentclass[final,3p,times,twocolumn]{elsarticle}
%% \documentclass[final,5p,times]{elsarticle}
%% \documentclass[final,5p,times,twocolumn]{elsarticle}

%% For including figures, graphicx.sty has been loaded in
%% elsarticle.cls. If you prefer to use the old commands
%% please give \usepackage{epsfig}

%% The amssymb package provides various useful mathematical symbols
\usepackage{amssymb}
%% The amsthm package provides extended theorem environments
%% \usepackage{amsthm}
\usepackage{svg}
\usepackage{float}
\usepackage{subfig}
\usepackage[hyphens]{url}
\usepackage{bibunits}

\defaultbibliography{references}

%% The lineno packages adds line numbers. Start line numbering with
%% \begin{linenumbers}, end it with \end{linenumbers}. Or switch it on
%% for the whole article with \linenumbers.
%% \usepackage{lineno}

\usepackage{comment}

\journal{arXiv}

\begin{document}

\begin{frontmatter}

%% Title, authors and addresses

%% use the tnoteref command within \title for footnotes;
%% use the tnotetext command for theassociated footnote;
%% use the fnref command within \author or \address for footnotes;
%% use the fntext command for theassociated footnote;
%% use the corref command within \author for corresponding author footnotes;
%% use the cortext command for theassociated footnote;
%% use the ead command for the email address,
%% and the form \ead[url] for the home page:https://www.overleaf.com/project/6183e5b5102788748dd902d3
%% \title{Title\tnoteref{label1}}
%% \tnotetext[label1]{}
%% \author{Name\corref{cor1}\fnref{label2}}
%% \ead{email address}
%% \ead[url]{home page}
%% \fntext[label2]{}
%% \cortext[cor1]{}
%% \affiliation{organization={},
%%             addressline={},
%%             city={},
%%             postcode={},
%%             state={},
%%             country={}}
%% \fntext[label3]{}

\title{Power sector effects of alternative options for electrifying heavy-duty vehicles: go electric, and charge smartly}

\author[inst1]{Carlos Gaete-Morales}
\author[inst2]{Julius Jöhrens}
\author[inst2]{Florian Heining}
\author[inst1]{Wolf-Peter Schill\corref{cor}}

\cortext[cor]{Corresponding author \ead{wschill@diw.de}}

\affiliation[DIW]{organization={German Institute for Economic Research (DIW Berlin)},
            addressline={Mohrenstraße 58}, 
            city={Berlin},
            postcode={10117}, 
            country={Germany}}

\affiliation[ifeu]{organization={ifeu - Institute for Energy and Environmental Research},%Department and Organization
            addressline={Wilckensstraße 3}, 
            city={Heidelberg},
            postcode={69120}, 
            country={Germany}}

\begin{abstract}
In the passenger car segment, battery-electric vehicles (BEV) have emerged as the most promising option to decarbonize transportation. For heavy-duty vehicles (HDV), the technology space still appears to be more open. Aside from BEV, electric road systems (ERS) for dynamic power transfer are discussed, as well as indirect electrification with trucks that use hydrogen fuel cells or e-fuels. Here we investigate the power sector implications of these alternative options. We apply an open-source capacity expansion model to future scenarios of Germany with high renewable energy shares, drawing on detailed route-based truck traffic data. Results show that power sector costs are lowest for flexibly charged BEV that also carry out vehicle-to-grid operations, and highest for HDV using  e-fuels. If BEV and ERS-BEV are not charged in an optimized way, power sector costs increase, but are still substantially lower than in scenarios with hydrogen or e-fuels. This is a consequence of the relatively poor energy efficiency of indirect HDV electrification, which outweighs its temporal flexibility benefits. We further find a higher use of solar PV for BEV and ERS-BEV, and a higher use of wind power and, to some extent, fossil generators for hydrogen and e-fuels.
\end{abstract}

%%Graphical abstract
%\begin{graphicalabstract}
%\includegraphics{grabs}
%\end{graphicalabstract}

%%Research highlights
%\begin{highlights}
%\item Research highlight 1
%\item Research highlight 2
%\end{highlights}

\begin{keyword}
%% keywords here, in the form: keyword \sep keyword
Heavy-duty vehicles \sep battery-electric vehicles \sep catenary \sep hydrogen \sep power sector modeling
\end{keyword}

\end{frontmatter}

%% \linenumbers

% \begin{comment}
% Terminology: we need consistent abbreviations for "BEV, ERS-BEV, PtL and H2"
% frame everything as ERS (Electric Road Systems)
% - BEV: Battery electric vehicles
% - ERS-BEV: electric vehicles with dynamic charging option over an ERS
% - PtL
% - FCEV
% - "trucks" vs. "HDV" --> generally use HDV; possible use "trucks" in text as well
% - power sector vs. power system (costs) --> go for power sector
% \end{comment}

\begin{bibunit}

% Joule: Good introductions are succinct, presenting only the background information needed for readers to understand the motivation for the study and the results. No subheadings, please.
\section{Introduction}\label{sec: introduction}
% Background: ghg emissions reduction by renewable electricity and sector coupling, also in transportation
Making energy consumption climate neutral in all end-use sectors is of paramount importance for mitigating climate change \cite{de_coninck_strengthening_2018}. A key strategy for achieving this is to substitute fossil fuels by renewable electricity, facilitated by direct or indirect electrification of end uses in mobility, heating, and industrial applications \cite{shukla_contribution_2022}. In the transportation sector, battery-electric vehicles (BEV) have emerged as the most promising option for the passenger car segment. Already today, BEV can lead to sizeable greenhouse gas emission reductions compared to internal combustion engines \cite{hoekstra_underestimated_2019}, which will further increase when the electricity mix becomes cleaner. In many countries, markets for electric passenger cars have been soaring in the past years, and are expected to continue to grow strongly in the near future \cite{iea_global_2022}. For heavy-duty vehicles (HDV), however, the technology space still appears to be more open. While the feasibility of pure battery-electric HDV has been assessed to be increasing \cite{nykvist_feasibility_2021}, they compete with other options. This includes electric road systems (ERS), which allow for dynamic power transfer to electric vehicles on the road \citep{boltze_insights_2020, speth_comparing_2021}; trucks with hydrogen fuel cells; or conventional HDV with internal combustion engines that use power-to-liquid (PtL), also referred to as e-fuels, which are produced with renewable electricity \cite{hannula_nearterm_2019, lajevardi_simulating_2022, plotz_hydrogen_2022,li_transition_2022}. 

% Different energy efficiency and flexibility characteristics of different electrification options
These options of direct or indirect electrification of HDV have different properties concerning, on the one hand, energy efficiency, and, on the other hand, temporal flexibility of electricity use. For example, direct electrification via BEV is more energy efficient compared to indirect electrification via electrolysis-based hydrogen or e-fuels \cite{ueckerdt_2021,lajevardi_simulating_2022}. Yet, the temporal flexibility of BEV may be constrained by charging availability and limited battery capacities, as vehicle batteries are costly and heavy. In contrast, indirect electrification via hydrogen or e-fuels may entail large-scale and low-cost storage options \cite{taljegard_2017, stoeckl_2021}, but the overall energy efficiency of these supply chains is lower compared to BEV. Temporal power sector flexibility becomes increasingly important with growing shares of renewables, as the potential for firm renewable generation such as hydropower, bioenergy, or geothermal power is limited in many countries. In contrast, wind and solar power potentials are often abundant, but they have variable generation profiles that depend on weather conditions and daily and seasonal cycles \citep{lopezprol_economics_2021}. Integrating growing shares of such variable renewables thus requires an increasing use of flexibility options in the power sector \citep{kondziella_flexibility_2016}.

Against this background, we investigate the power sector implications of different options for (in-)directly electrifying HDV, particularly focusing on the trade-off between energy efficiency and temporal flexibility. To do so, we apply an open-source capacity expansion model \cite{zerrahn_long-run_2017,gaete_2021} to 2030 scenarios of the Central European power sector with high renewable energy shares. We focus on the domestic traffic of HDV in Germany with a gross vehicle weight above 26 tonnes, drawing on a detailed data set of truck trips on inner-German routes. We include stationary-charged BEV trucks as well as hybrid battery-catenary trucks as a particular example of an electric road system technology (ERS-BEV), fuel-cell hydrogen electric trucks (FCEV), and such with internal diesel combustion engines powered by e-fuels (ICEV PtL), see Figure~\ref{fig_00}. We compare the power sector costs of these options, as well as their repercussions on the optimal power plant fleet and its use.

\begin{figure}[ht!]
    \centering
    \textcolor{white}{\frame{\includegraphics[width =0.98\textwidth]{figures/diagram_svg-raw.pdf}}}
    \caption{Overview of direct and indirect HDV electrification options covered in this analysis. For direct electrification via BEV and ERS-BEV we differentiate scenarios with flexible charging without and with V2G, as well as scenarios with inflexible, i.e.,~non-optimized charging. Hydrogen and PtL options are assumed to be always operated flexibly, constrained by hydrogen of e-fuel storage capacities, and do not allow reconversion to electricity.}
    \label{fig_00}
\end{figure}

While there is a broad literature on the potential power sector impacts of battery-electric passenger cars \cite{richardson_electric_2013,muratori_shape_2020,mangipinto_impact_2022}, according research for electric HDV is sparse. A common finding of analyses focusing on passenger cars is that controlled charging can help to avoid problematic load peaks, to integrate renewable electricity generation, and to lower overall system costs \cite{schill_power_2015,gnann_2018,sadeghian_a_2022}. In particular, feeding back electricity from car batteries to the system (vehicle-to-grid, V2G) can help to balance the diurnal variability of solar PV \cite{brown_2018}.

% Brinkel et al 2020: multi-objective optimization, costs and emissions; case study NL; no ESM, but historic day ahead market prices and emissions

How electrified HDV interact with the power sector is hardly explored so far. Some analyses investigate grid or peak load impacts without using power sector models. For example, an analysis of the distribution grid impacts of BEV-HDV with depot charging finds that many existing substations in the US would allow for substantial HDV charging without grid upgrades \cite{borlaug_heavyduty_2021}. Earlier studies conclude that local grid impacts of electrified HDV near logistics centres and along major roads may be substantial in Switzerland and Finland \citep{liimatainen_2019}, or that ERS-HDV could strongly increase the peak load in Norway \cite{taljegard_2017}. Using capacity expansion power sector models, other analyses come to somewhat deviating conclusions on the impacts of electric HDV on the overall load profile and the wider power sector; in a Scandinavian-German case study that includes ERS-HDV in a subset of scenarios, effects are relatively substantial \cite{taljegard_impacts_2019}, while in a European model study the impacts are more moderate \cite{plotz_impact_2019}. Importantly, both analyses assume exogenous, i.e.,~inflexible, charging profiles of HDV and do not temporally optimize their (dis-)charging operations.
Beyond power sector effects, other studies focus on the impacts of alternative HDV options on energy efficiency and its consequences for the total cost of ownership (TCO) \cite{noll_analyzing_2022}, or explore potential health and climate impacts of electric HDV \cite{lin_2021}.

We contribute to the literature with a dedicated analysis of the power sector impacts of different HDV electrification options that co-optimizes their charging and discharging operations (including V2G) with the capacity and dispatch decisions in the power sector. We do so for a wide range of HDV technologies, including dynamic power supply via electric road systems. We use a power sector model that fully captures the hourly variability of load and renewable generation over all hours of a full year, and apply it to a future scenario with high shares of variable renewables. The model code and all input data, including detailed hourly HDV mission profiles for domestic transport in Germany, are provided open source for transparency and reproducibility.


\section{Results}\label{sec: results}
% more details

% color for groups
%As described above, we aim to assess the effect of HDV fleet electrification on the German power sector over costs, investment decisions and dispatch, 
In the main part of this paper, we show results for a setting where Germany is interconnected with its neighbors, i.e.,~electric HDV in Germany benefit from the flexibility provided by the European interconnection.\footnote{In the Supplemental Information, we also show results for a case where the German power sector is modeled in isolation, see Section~\ref{sec: sensitivities}.} We compare nine HDV scenarios against a reference scenario without any electrification of HDV (Ref). The first three scenarios cover pure BEV trucks with varying degrees of flexibility: these are either charged as flexibly as possible (BEV~Flex), additionally have the option of feeding back electricity to the grid (BEV~Flex~V2G), or are charged in a way that is not co-optimised with the power sector (BEV~Inflex). There are three respective scenarios for ERS-BEV, which have smaller batteries than pure BEV. Three additional scenarios cover indirectly electrified HDV powered by hydrogen or hydorgen-derived e-fuels. As for fuel-cell trucks, we differentiate two domestic hydrogen supply chains, either on-site electrolysis at filling stations, which is temporally inflexible (FCEV Distributed), or centralized electrolysis and transport via liquefied hydrogen, which also comes with low-cost storage opportunities and thus higher temporal flexibility (FCEV Centralized). The final scenario represents conventional trucks with internal combustion engines powered by e-fuels (ICEV PtL). To separate effects, we counterfactually assume that the whole domestic fleet of HDV (318,700 vehicles~\textgreater{}~26 t) in the year 2030 consists of the particular vehicle type.

\subsection{Lowest power sector costs and electricity prices for BEV with V2G}

Compared to the reference case without electrified trucks, yearly power sector costs\footnote{Power sector costs comprise fixed and variable costs of all electricity generation and storage technologies over a full year. They do not include the costs of HDV electrification infrastructure, e.g.,~overhead lines, charging stations, or filling stations for hydrogen or PtL.} increase in all scenarios with electrified HDV (Figure~\ref{fig_01}, upper panel). That is, the cost of the additional electricity demand induced by HDVs always outweighs their potential flexibility benefits. Cost effects, however, vary considerably between different options. Flexible BEV with V2G incur the lowest additional power sector costs of 1.8~bn~Euros/year (around 5,600~Euros/year per vehicle), followed by optimally charged BEV without V2G with 2.3~bn~Euros/year (around 7,200~Euros/year per vehicle). If BEV charging is not optimized, system costs are markedly higher with 3.8~bn~Euros/year (11,900~Euros/year per vehicle). Results are qualitatively similar for ERS-BEV, but on a slightly higher cost level. The differences between the three ERS-BEV cases are less pronounced than for pure BEV, as their temporal flexibility potential is much smaller. The battery capacity of an ERS-BEV is only around a quarter of that of a pure BEV (655~kWh usable capacity per truck vs. 181~kWh). In contrast, power sector cost are substantially higher for FCEV (12.6 or 12.7~bn~Euros/year, for decentralized or centralized hydrogen provision, i.e.~around 39,700~Euros/year per vehicle) and even more so for PtL (16.8~bn~Euros/year, or 52,700~Euros/year per vehicle). This is a direct consequence of high conversion losses of hydrogen and PtL supply chains and vehicles drive trains. Because of these losses, the two hydrogen supply chains increase the electricity demand more than twice as much as the battery-electric options. The electricity demand of PtL-HDV is nearly four times as high as in the case of BEV. Notably, the cost differences between BEV and ERS-BEV scenarios are much smaller than the differences between these direct-electric options and indirect electrification via hydrogen or PtL.

\begin{figure}[ht!]
    \centering
    \textcolor{white}{\frame{\includegraphics[width =0.6\textwidth]{figures/Fig_01_svg-raw.pdf}}}
    \caption{Changes in yearly power sector costs and electricity demand induced by different HDV options (upper panel), and average wholesale market prices of charging electricity (lower panel).}
    \label{fig_01}
\end{figure}

Complementary to power sector costs, we also evaluate average yearly wholesale electricity prices for HDV electricity (Figure~\ref{fig_01}, lower panel). This allows to largely separate the differences in overall electricity consumption of the various HDV options from their ability to make use of low-cost electricity. Average prices are calculated by multiplying hourly wholesale prices of electricity consumed by the different HDV options or fed back to the grid with respective hourly quantities, summing up over the whole year, and dividing by the overall electricity consumption of the fleet. That is, the numbers also account for revenues of electricity sold via V2G.\footnote{Here we assume that HDV operators receive the respective hourly wholesale price whenever they feed electricity back to the grid.} BEV with V2G accordingly face the lowest average electricity prices, as these also benefit from revenues of feeding back to the grid, followed by BEV without V2G. Average electricity prices paid by ERS-BEV are somewhat higher, as their smaller batteries limit the ability for temporally optimizing their charging and V2G decisions. In contrast, pure BEV can leverage their larger batteries to make better use of hours with low electricity prices. In case of inflexible charging, results change: average electricity prices faced by inflexibly charged BEV are high, and even slightly above those of inflexible ERS-BEV. This is because inflexibly charged BEV benefit less from cheap electricity prices around midday related to abundant PV feed-in than inflexible ERS-BEV, which on average have a better (catenary) grid connection during these hours.

Prices for electricity used in hydrogen and PtL supply chains are in the same range as those of ERS-BEV options, but below those of inflexibly charged BEV or ERS-BEV. Further, centralized hydrogen and PtL supply can make use of lower prices than decentralized supply, because their low-cost storage options allow for higher temporal flexibility. Electricity prices of centralized hydrogen supply chains and PtL are also cheaper than those faced by inflexibly operated BEV or ERS-BEV. In terms of overall costs, these temporal flexibility benefits are, however, by far outweighed by the higher overall energy consumption of the FCEV and PtL options (Figure~\ref{fig_01}, upper panel).

\subsection{Capacity and dispatch effects}\label{sec: capacity and dispatch effects}

\begin{figure}[ht!]
    \centering
    \textcolor{white}{\frame{\includegraphics[width =0.9\textwidth]{figures/Fig_02_svg-raw.pdf}}}
    \caption{Effects of different HDV scenarios on optimal generation capacity (upper panel) and on yearly generation (lower panel) in Germany}
    \label{fig_02}
\end{figure}

The upper panel of Figure~\ref{fig_02} shows optimal generation capacities in the reference (on the left) and the changes induced by HDV (on the right) in Germany, where a minimum renewable energy share of 80\% applies. In the reference, variable solar PV and onshore wind power dominate the capacity mix.\footnote{In the Supplemental Information, we discuss the effects of including an upper bound for wind power investments by the year 2030, see Section~\ref{sec: sensitivities}.} These are complemented by smaller firm capacities of natural gas and bioenergy. Optimal capacity additions related to the electrification of HDVs are predominantly a mix of solar PV and onshore wind power, as these are the lowest-cost options that are available. Flexible BEV and ERS-BEV lead to the highest PV shares in the capacity additions, especially if combined with V2G. BEV with V2G essentially serve as short-duration grid storage, which favors the expansion of solar PV. HDV options that are temporally less flexible or that, overall, require more electricity favor higher onshore wind power capacities. If, alternatively, more PV was built, this would lead to increasing amounts of unused renewable surplus energy. Offshore wind power is not added here because of relatively unfavourable cost assumptions compared to onshore wind power.\footnote{Given current tenders for offshore wind power capacity in Germany, it may play a larger role than assumed here, while the real-world potentials to expand onshore wind power may in fact be limited.} FCEV and PtL options have the highest capacity needs because of substantial conversion losses. Decentralized electrolysis further requires a substantial addition of long-duration electricity storage capacity (P2G2P, power-to-gas-to-power) of 9.4~GW and 6.0~TWh (for more details, see Section~\ref{sec: si storage}). This is necessary to compensate for the temporal inflexibility of on-site electrolyzers without large-scale hydrogen storage, which are forced to operate with a similar time profile as hydrogen demand. No hard coal or lignite generation capacities are added, partly due to a CO$_{2}$ price of 100~Euros/ton which discourages such investments, but some natural gas (OCGT) and oil-fired generators. Especially in the scenarios with low temporal flexibility (BEV Inflex, ERS-BEV Inflex and FCEV Distributed), investments into natural gas and oil generation capacity are at the assumed maxima (compare Section~\ref{sec: exp proc}). 

Yearly electricity generation in the reference, as well as the changes induced by different HDV scenarios, are shown in the lower panel of Figure~\ref{fig_02}. Here, the picture is generally similar to optimal capacity changes, but the share of wind power in additional electricity generation tends to be higher because of higher full-load hours as compared to solar PV. Flexible BEV, particularly if combined with V2G, cause the highest increase in PV generation, which also drives up the overall renewable share to above 81\%. Indirect HDV electrification options that use centralized hydrogen or PtL supply chains, which require much more electricity than BEV and ERS-BEV, use their temporal flexibility to also increase electricity imports and the use of the most energy-efficient natural gas-fired power plants (CCGT). In the case of temporally inflexible hydrogen production at filling stations (FECV On-Site), this is different: as imports and CCGT are less available in the hours of decentralized hydrogen production, more wind and solar generation in combination with long-duration electricity storage are used, which also increase the share of renewables in the overall system to nearly 83\%. A higher utilization of natural gas power plants also translates into increasing direct carbon emissions, as discussed in Section~\ref{sec: CO2}.

\subsection{Time series reveal differences in flexibility characteristics}

\begin{figure}[htp!]
    \centering
    \textcolor{white}{\frame{\includegraphics[width =0.96\textwidth]{figures/Fig_03_svg-raw.pdf}}}
    \caption{Sample of electricity time series five exemplary days: ``HDV electricity demand'' illustrates electricity flowing to BEV or ERS-BEV batteries, including amounts later charged back via V2G, as well as electricity needs of hydrogen and PtL supply chains; ``HGV discharging to the grid" are V2G power flows from BEV or ERS-BEV to the grid; and ``Residual load'' is the net load of the remaining electricity system after considering the feed-in potential of variable renewables. The samples start on a Saturday. BEV Flex and ERS-BEV Flex without V2G are omitted in the graph, as charging patterns are very similar to the shown V2G cases. Additional time series are shown in the SI.}
    \label{fig_03}
\end{figure}

Figure~\ref{fig_03} illustrates that flexibly operated BEV trucks (BEV Flex V2G) are able to charge their batteries in hours of low residual load, which generally goes along with low wholesale market prices (compare Figure~\ref{fig_03b}). This enables them to utilize renewable surplus energy (i.e., negative residual load) to a substantial extent. They also make use of the V2G option and feed back renewable surplus energy to the grid, whenever the battery capacity is not needed for driving. In the exemplary illustration, this is visible particularly in summer during weekends, when trucks are not driving, but assumed to have a grid connection (top right panel in Figure~\ref{fig_03}).\footnote{Note that the time series shown begin on a Saturday. For simplification, we assume that both Saturday and Sunday are completely truck-free.} Here, BEV with V2G charge their batteries with solar PV surplus energy on Saturday afternoon, and feed it back to the grid in the night between Saturday and Sunday. This is different in the following days, as HDV are used between Monday and Friday, and much less battery capacity is available for V2G. If BEV charging is not optimized, but follows an inflexible, pre-determined pattern (BEV Inflex), charging profiles are less peaky and much more balanced. BEV are not able to make particular use of cheap renewable surplus electricity in this case, but carry out a substantial part of their charging in hours with positive residual load. 

ERS-BEV generally follow similar patterns as pure (non-catenary) BEV. Yet, their smaller batteries make ERS-BEV temporally less flexible, so they can make less use of renewable surplus energy and have to draw more electricity from the grid during hours with positive residual load. For the same reason, their potential for feeding electricity back to the grid is also much smaller than in the case of pure BEV. ERS-BEV sometimes also charge their batteries during driving (compare Figure~\ref{fig_08}). 

Hydrogen and PtL supply chains show very different patterns of electricity use. Centralized hydrogen supply chains (FCEV Centralized), which come with large-scale hydrogen storage capacities, generally have a flat consumption profile in many hours, as their high fixed costs make it optimal to use them with relatively high full load hours. This limits their ability to make use of renewable surplus energy. Yet, they can use the temporal flexibility provided by large-scale hydrogen storage to reduce electricity consumption in hours of high positive residual load, i.e.,~high prices. Decentralized hydrogen supply chains (FCEV On-Site) have an electricity use profile that largely follows the hydrogen demand profile, driven by limited on-site hydrogen storage capacity. Accordingly, they are less able to avoid electricity consumption in hours of high residual load centralized electrolyzers. The PtL supply chain (ICEV PtL) also comes with large-scale storage and thus has a relatively similar pattern as the centralized hydrogen option. The load, however, is generally higher because of higher overall energy consumption of PtL versus hydrogen. This goes along with larger investments in electrolyzers, which are 24.4 GW in the ICEV PtL scenario, as compared to 13.5~GW for FCEV Centralized (see Section~\ref{sec: electrolysis capacity} for more details).

% Joule: The discussion should explain the significance of the results and place them into a broader context. It is often helpful to the reader to indicate the directions in which the work might be built on going forward. It should not be redundant with the results. The discussion may contain subheadings and can be combined with the results section.
\section{Discussion}\label{sec: discussion}
% Summary
\subsection{Differences in temporal flexibility and energy efficiency drive effects}
We analyze the power sector effects of alternative options for directly or indirectly electrifying heavy-duty vehicles in Germany, focusing on power sector costs as well as optimal investment and dispatch decisions of electricity generation and storage technologies. We find that differences in temporal flexibility and energy efficiency are important drivers of results.

Fleets of BEV or ERS-BEV trucks that are charged in an optimized way cause the lowest increase in power sector costs. This is because they require much less electricity than HDVs powered by green hydrogen or e-fuels. At the same time, they can offer temporal flexibility to the power sector, which lowers the costs compared to non-optimized charging. This is increasingly valuable with growing shares of variable renewable energy sources. Flexibility benefits are particularly large if BEV also carry out vehicle-to-grid operations, i.e.,~if the truck batteries can be used as a distributed grid storage option to some extent. In contrast, indirect electrification of the HDV fleet via FCEV or PtL incurs much higher power sector costs, even if compared to inflexibly charged BEV or ERS-BEV. That is, the temporal flexibility benefits of centralized hydrogen or PtL supply chains, which come with low-cost storage options, do not outweigh their energy efficiency drawbacks in terms of power sector costs.

Alternative options of (in-)directly electrifying HDV fleets further entail different optimal generation capacity mixes. The additional electricity for battery-electric HDV is largely supplied by solar PV in the lowest-cost solution, in particular if V2G is also available. In contrast, centralized hydrogen and PtL supply chains go along with higher optimal shares of wind power, and also lead to a higher use of non-renewable electricity generators. The latter translates into higher direct carbon emissions of these technologies. However, emission outcomes depend on the assumed boundary conditions, such as renewable shares and CO$_{2}$ prices.

Additional sensitivity analyses show that results are generally robust if Germany is modeled as an electric island, i.e.,~if the flexibility benefits of the European interconnection are neglected (Section~\ref{sec: sensitivity island}). Here, flexible BEV with V2G are even more beneficial, and inflexible BEV operations lead to slightly higher cost increases, because an isolated power sector is more flexibility-constrained. Further, results are robust against the assumption that the potential for wind power expansion is limited, and solar PV accordingly plays a larger role (Section~\ref{sec: sensitivity wind}).

% Conclusions
\subsection{Direct electrification and smart charging preferable}
From a pure power sector perspective, direct electrification of the truck fleet would thus be clearly preferable. Aside from the cost benefits determined here, this appears to be even more important as the utilization of renewable energy sources is unlikely to grow exponentially \citep{hansen_2017}, and renewable growth rates in fact have been smaller than what would be required in $1.5^{\circ}C$-compatible scenarios\cite{cherp_2021}. A similar argument has been made for the global scale-up of electrolysis \citep{odenweller_2022}. All of this calls for energy-efficient, direct electrification options of HDV wherever possible.
% From a pure power sector cost perspective, we further find that BEV are slightly better than ERS-BEV; it is not clear if this finding would still hold when the benefits of reduced battery sizes and differences in fast-charging and catenray infrastructure costs were considered. This has to be analyzed with dedicated analöyses in the future.

Moreover, smart charging of BEV or ERS-BEV is highly desirable, as it provides temporal flexibility which leads to lower power sector costs, and also lower carbon emissions, as compared to non-optimized charging. This corroborates the findings of Pickering at al., who also highlight the benefits of controlled vehicle charging in decarbonized and largely renewable future European energy systems \cite{pickering_diversity_2022}. If, additionally, feeding back electricity to the grid via V2G is possible, this further lowers the costs and also increases the optimal shares of solar PV.

% Limitations
\subsection{Limitations and avenues for future research}
Like any model-based study, our analysis has some limitations. We briefly discuss how these may qualitatively impact results. First, we deliberately focus on a detailed analysis of the power sector implications of alternative HDV electrification options and do not aim to compare the overall system costs of different options. The latter would require also considering infrastructure costs related to catenary lines, chargers, or filling stations, as well as differences in vehicle costs. Note that such future cost estimates are highly uncertain, so that Monte Carlo approaches or other methods to deal with parameter uncertainty would have to be used. Differences in the costs of vehicles and related charging infrastructure may in fact outweigh some of the differences in power sector costs identified here, especially between pure BEV and ERS-BEV. Yet, the very large power sector cost differences between (ERS-)BEV on the one hand, and FCEV or PtL trucks on the other, would require the latter to be substantially cheaper in terms of vehicle and infrastructure costs in order to break even, which appears unlikely.

While focusing on the power sector, we further we assume extensive charging and overhead line infrastructure to be available for BEV and ERS-BEV, both in depots and en route, and that vehicles are always connected to the grid when idling. In reality, the availability of grid connections is likely to be lower for a variety of reasons, which would decrease the flexibility potentials of optimized charging. The flexibility benefits of realizing very high levels of grid connection may not necessarily outweigh related infrastructure costs \citep{brinkel_2020}.

Another simplification of our numerical analysis is the assumption of perfect foresight, which results in an upper limit of the flexibility benefits that can be realized by optimized (dis-)charging of BEV and ERS-BEV. We also abstract from including electric passenger vehicles or other future sector coupling options, which may provide flexibility to the power sector and compete with electric trucks for electricity use in times of high renewable availability. In particular, power-to-heat technologies may be able to efficiently exploit hours of high renewable generation and low energy prices, as heat storage is relatively cheap compared to electricity storage \citep{bloess_2018}.

To facilitate numerically efficient solutions and keep results tractable, we further assume electricity generation capacities to be fixed in Germany's neighbor countries. At the same time, the modeled HDV fleets only cover domestic truck transport in Germany. This may lead to an overestimation of the power sector flexibility supplied by Germany's neighbors for integrating electric trucks in Germany.

\begin{comment}
\begin{itemize}
    \item The truck operating profiles are synthetic. They are fitted to the known macroscopic statistics of the freight transport system in Germany, but are not validated against empirical data of actual mission profiles. 
    \item The vehicle configuration is generic. In practice, a profile-specific differentiation of the vehicle configuration is to be expected. This should, in some cases, tend to lead to lower battery capacities and thus lower vehicle ranges. It would also somewhat reduce the flexibility in the optimized charging of the vehicles.
    \item The number of operating profiles had to be strongly limited in order not to exceed the computing capacities. The small number of discrete profiles is likely to result in stronger fluctuations in the time-dependent results than we would expect with a larger number of more evenly distributed operating profiles.
    \item If battery prices should not continue to fall as expected or short charging cycles on long trips should be acceptable for truck operators (given sufficient public infrastructure), future battery trucks could have significantly smaller batteries than assumed here. This would decrease flexibility potential accordingly.
    \item We model long-duration electricity storage in a stylized way by using plausible cost and roundtrip efficiency parameters from the literature. Yet, long-duration electricity storage is likely to make use of hydrogen technologies, and may thus interact with the hydrogen infrastructure modeled here for mobility uses. How this impacts results is left for future research.
    \item V2G: would be less used if we assumed higher battery degradation costs than 15 Euros / MWh
\end{itemize}
\end{comment}

% Outlook
Future research may address some of these limitations. While our analysis focuses on the power sector, it also appears desirable to investigate complementary cost measures. This includes detailed total cost of ownership analyses from a HDV operator perspective, as well as investigations of overall system cost effects, which would also consider cost differences of charging and ERS infrastructures, as well as purchase cost differences of different types of HDV. This requires a whole set of additional and detailed parameter assumptions, which we leave for future research. Further, it would be desirable to carry out similar analyses for other countries and world regions. While we focus on Germany, we expect that general results should also hold for comparable non-hydro electricity sectors in other countries in temperate climates.

\section*{Acknowledgements} 

We thank Hinrich Helms for valuable comments on eralier drafts. This work has benefited from research grants by the Federal Ministry for the Environment, Nature Conservation and Nuclear Safety (BMU) and by the Federal Ministry for Economic Affairs and Climate Action via the projects ``My eRoads'' (Fkz 16EM4006-1) and ``enERSyn'' (Fkz 01MV22004B).

\section{Experimental procedures}\label{sec: exp proc}

\subsection{The power sector model DIETER}
We use the open-source power sector model \textit{Dispatch and Investment Evaluation Tool with Endogenous Renewables} (DIETER). It is a linear program that minimizes power sector costs by optimizing capacity and dispatch decisions for a full year in an hourly resolution \cite{zerrahn_long-run_2017,gaete_2021}. Its objective function includes fixed an variable costs of all electricity generation and storage technologies, electrolysis and PtL plants, as well as hydrogen or e-fuel transportation. It does not include the costs of charging or catenary infrastructure, hydrogen filling stations, or PtL filling stations. Accordingly, the power sector cost figures provided above do not include the costs of HDV electrification infrastructure. We further do not consider the option of hydrogen imports, as these are likely to be unavailable at scale by 2030. In general, the global scaling up of green hydrogen supply remains uncertain \cite{odenweller_2022}. 

Endogenous model variables include power sector costs, optimal generation and electricity storage capacities (Germany) and their hourly use (all countries), hourly decisions for HDV charging and discharging, as well as the capacity and operational decisions of electrolysis and PtL generation and storage infrastructure. In addition, we interpret the marginals of the hourly energy balance as wholesale prices \citep{brown_2018}. 

Exogenous model inputs include fixed and variable costs of all electricity generation and storage technologies, efficiency parameters, as well as time-series variable renewable energy availability profiles and electric load. In the case of inflexible HDV charging (BEV Inflex, ERS-BEV Inflex), we assume that the vehicles always start charging as soon as an opportunity arises, and that vehicles batteries are fully charged by the time the next trip starts, if possible. That is, the charging power is lower, the longer a vehicle is connected to the grid. This resembles the ``balanced'' charging profile defined in \cite{gaete_2021_emobpy}.

The geographic scope of the model version used here includes Germany and its neighboring countries plus Italy. In order to reduce numerical complexity and improve tractability, we allow for endogenous generation capacity investment only in Germany, and fix the power plant portfolio for the other countries to values derived from ENTSO-E's Ten Year Network Development Plans \cite{tyndp_2018,tyndp_2020}. We further assume upper limits for investments into fossil generation technologies according to the main scenario for 2030 from the German Federal Grid Development plan (NEP, \cite{nep_2018}). The model is required to satisfy at least 80 percent of the load in Germany with domestic renewable electricity generation. This includes the additional load related to directly or indirectly electrifying HDV. This reflects the current German government's target for 2030 that has also been set out in the Renewable Energy Sources Act.

Different versions of the model have been used in various earlier studies to investigate various aspects of renewable energy integration, electricity storage, and sector coupling \cite[e.g.,~][]{schill_2018,schill_joule_2020,stoeckl_2021,gils_2022}. The model code and all input data are available open source.\footnote{General DIETER repo: \url{https://gitlab.com/diw-evu/dieter_public}. Particular model version used for the present study: \url{https://gitlab.com/diw-evu/projects/my_eroads}. Model documentation: \url{https://diw-evu.gitlab.io/dieter_public/dieterpy/}.}

%\subsection{Power sector input data}
%generation capacity and cost assumptions, renewable time series

\subsection{Mobility data of heavy-duty vehicles}

% for FCEV and PtL, different approach

% + energy consumption of vehicles

% Note: no cross-border traffic included

% Synthesis of usage patterns based on PTV Validate and KBA statistics.

We generate synthetic truck usage patterns that are intended to approximate the German fleet of HDV larger than 26 tons. The main data source for the usage patterns is the traffic model PTV Validate, from which we extract a database of daily truck trips in domestic German road freight transport. In order to derive typical daily driving profiles, the following steps are performed (for more detailed information, see~SI):

\begin{enumerate}
    \item Calculation of annual mileage for transport relations. For this step, we build on statistics of typical annual mileages for different distance classes and derive a steady function. 
    \item Calculation of daily idle time. This is calculated based on the number of daily truck trips (and the number of stops in between) as well as typical times for loading and unloading, depending on the goods type. In addition, a mandatory driver’s break of 45~minutes is assumed for daily driving profiles \textgreater{} 4.5 hours.
    \item The daily operating time of a vehicle on a particular route is determined from the above assumptions using the formula ``operating time = journey time + idle time + driver’s break''.
\end{enumerate}

Repeating steps 1-3, synthetic daily travel profiles with a certain daily operating time are obtained for all transport relations in the model. In order to limit these to a manageable number of daily driving profiles for the power sector model, all relations that have approximately the same daily operating time are combined. We derive a number of 19 daily profiles which are distributed over the course of the day such that the empirically observed course of the daily mileage on the German road network is reproduced. For these profiles, the time series of charging availability (in the depot, during idle and driver’s resting times) and of the electricity demand of pure BEV-HDV (500 km battery range) and ERS-HDV (150 km battery range) are calculated. The resulting electricity demands and charging availabilities are used as inputs for the DIETER model.



\clearpage

% \bibliographystyle{elsarticle-num} 
% \bibliography{references}
\defaultbibliographystyle{elsarticle-num}
\biboptions{sort&compress}
\putbib
\end{bibunit}

\clearpage

\begin{bibunit}

\renewcommand{\thesection}{SI}
\renewcommand{\thepage}{SI}
\global\long\def\thefigure{SI.\arabic{figure}}
\global\long\def\thetable{SI.\arabic{table}}
\global\long\def\thepage{SI.\arabic{page}}
\setcounter{figure}{0}
\setcounter{table}{0}
\setcounter{page}{1}
\newpage

\section{Supplemental Information}

%\subsection{Key input parameters for the power sector model}
%cost assumptions, renewable and load time series data (weather year)

\subsection{Additional information on HDV profiles}
% maybe 2 pages, including graphs
Vehicle load profiles are key input parameters to this analysis. In case of non-optimized, inflexible charging, respective time series are provided to the power sector model as exogenous parameters. In case of optimized charging, the vehicles' charging and, if V2G is available, discharging profiles are determined as endogenous variables. In this case, mission-profiles become important model inputs. Data sets for truck trips are available for Germany and also Europe \citep{speth_synthetic_2022}. However, these trips are not connected to vehicle mission profiles, e.g. information on daily starting and charging times, in these data sets. Therefore, some studies choose to simply refer to the observed traffic volume on typical roads in order to approximate the energy demand profile of the vehicle fleet \citep{taljegard_impacts_2019}. However, it is impossible to determine flexibility potentials in the load profile when using this approach.

This study is based on synthetic truck usage patterns that are intended to roughly approximate the German HDV fleet (\textgreater{}~26 t) in 2030, when a broad electrification of heavy-duty road transport is assumed. The main data source for the usage patterns is the traffic model PTV Validate which in turn is based on the official German governmental forecast for goods flow matrices for the year 2030 \citep{schubert_2014}. Within PTV Validate, the freight flows are allocated to vehicle trips between approximately 10,000 origin and destination districts in Germany, i.e.,~to individual transport relations. For the present study, we use aggregated source and destination districts at county level (approx.~400 counties in Germany). PTV Validate also determines the vehicle class of trucks used for the transport of a certain goods type on a given transport relation. Only the journeys of trucks in the size class \textgreater{}~26 t are considered here, as these account for a large proportion of the energy consumption of the truck fleet and also represent the primary intended field of application for electricity consumption by overhead lines.

The result is a database of daily trips by HDVs in domestic German traffic. International traffic is not considered here to reduce complexity. Note that the purpose of this analysis is mainly to illustrate the general power sector implications of different HDV technologies, rather than forecasting their absolute impacts. 

In order to derive typical daily driving profiles from the aforementioned trip database, the following steps are performed:

\begin{enumerate}
    \item \textbf{Calculation of annual mileage for transport relations:} The distance of a transport relation is assumed as characteristic for the radius of operation that a vehicle performing transports on the given transport relation will typically have. This allows us to utilize official statistics from the German Federal Road Administration which give typical annual mileages per vehicle class for the operating ranges 0-50 km, 50-150 km and \textgreater{}~150 km. Using a regression approach, a dependency between the distance of each transport relation and the typical annual mileage of vehicles operating there is obtained.
    \item \textbf{Calculation of daily idle time:} First, the number of daily trips made by a truck is estimated. For example, trucks on short shuttle routes often make multiple trips a day, whereas trucks on long-distance routes often only make one trip a day. The number of trips entails a corresponding number of stops, which can in principle be used to recharge the traction battery if the stop is long enough. Second, depending on the type of goods transported, typical durations for loading and unloading were estimated. These determine the duration of the assumed idle times in the daily driving profile. In addition, a mandatory driver’s break of 45 minutes is assumed for daily driving profiles of \textgreater{}~4.5 hours.
    \item \textbf{Calculation of daily operating time:} The daily operation time of a vehicle on a particular route is therefore defined as 
    \begin{equation}
        \textrm{operating time = journey time + idle time + driver’s break.}
    \end{equation}

\end{enumerate}

For step 1, we use the regression function

\begin{equation} 
M= a \cdot R^b + c
\end{equation}

where M is the annual mileage, R the distance of a relation and a, b and c optimization coefficients. The objective function is constructed such that the average annual mileages of each distance class from Table~\ref{Table_a1} are reproduced when iterating over all transport relations.

The regression result for the parameters is a = 21,180, b = 0.2979 and c = 0. The resulting dependency is shown in Figure~\ref{fig_a1}.

\begin{table}[ht!]
    \caption{Annual mileage of HDV \textgreater{}~26~t per distance class. Source: \citep{bast_2017}}
    \label{Table_a1}
    \centering
    \begin{tabular}{|c|c|}
    \hline
    Distance class      & \begin{tabular}[c]{@{}c@{}}Average annual mileage \\ {[}km{]}\end{tabular} \\ \hline
    \textless 50 km     & 45,684                                       \\
    51-150 km           & 78,190                                       \\
    \textgreater 150 km & 117,121                                      \\
    All classes         & 94,950                                       \\ \hline
    \end{tabular}
\end{table}

For step 2, we use the ratio of the daily mileage\footnote{Annual mileage (cf. step 1) divided by 240 working days.}  and the distance of the transport relation to obtain the number of trips per day and thus the number of stops on a daily mission. The idle time per stop is determined based on an expert guess for each goods class. By multiplying the number of stops with the idle time per stop, we obtain the total daily idle time for the vehicle operation on a given transport relation, considering the type of transported goods.

\begin{figure}[ht!]
    \centering
    \includegraphics[width=12cm]{figures/Fig_SI01.png}
    \caption{Annual mileage as a function of distance class.}
    \label{fig_a1}
\end{figure}

Repeating steps 1-3, synthetic daily travel profiles with a certain daily operating time are obtained for all transport relations in the model. In order to limit these to a manageable number daily driving profiles for the energy system modelling, all relations with a similar daily operating time are combined. For each daily operating time, averaged values for the idle times are used. The essential parameters for the operational profiles are summarized in Table~\ref{Table_a2}.

\begin{table}[ht!]
\caption{Key data of daily driving profiles}
\label{Table_a2}
\centering
\resizebox{\columnwidth}{!}{%
\begin{tabular}{|c|c|c|c|c|c|}
\hline
\begin{tabular}[c]{@{}c@{}}Daily operating \\ time {[}h{]}\end{tabular} & \begin{tabular}[c]{@{}c@{}}Average distance of \\ relation {[}km{]}\end{tabular} & \begin{tabular}[c]{@{}c@{}}Average daily \\ mileage {[}km{]}\end{tabular} & \begin{tabular}[c]{@{}c@{}}Average daily \\ idle time {[}h{]}\end{tabular} & \begin{tabular}[c]{@{}c@{}}Number of \\ vehicles\end{tabular}  & \begin{tabular}[c]{@{}c@{}}Total annual \\ mileage {[}bn. km{]}
\end{tabular} \\ \hline
3  & 23  & 196  & 0.11 & 21,073 & 0.99   \\
4  & 74  & 281  & 0.06 & 29,007 & 1.96  \\
5  & 136 & 354  & 0.04 & 7,069  & 0.6   \\
6  & 175 & 383  & 0.78 & 34,772 & 3.2   \\
7  & 237 & 415  & 1.43 & 43,544 & 4.34   \\
8  & 219 & 399  & 2.43 & 86,826 & 8.31   \\
9  & 204 & 351  & 3.97 & 77,117 & 6.49   \\
10 & 448 & 527  & 2.89 & 19,166 & 2.42  \\ \hline
\end{tabular}%
}
\end{table}

The assumptions regarding the vehicles and their infrastructure are made independently of the daily operating time for the sake of simplicity and are summarized in Table~\ref{Table_a3}. The design of the vehicles is oriented towards the needs of long-haul transport with a range of 500~km for BEV. For profiles in local and regional transport (with mostly lower daily mileage), smaller batteries could generally also be sufficient, so the capacity tends to be overestimated in this analysis.

\begin{table}[ht!]
\caption{Assumptions on BEV and ERS-BEV HDV and infrastructure}
\label{Table_a3}
\centering
\resizebox{\columnwidth}{!}{%
\begin{tabular}{c|cc|}
\cline{2-3}
    & \multicolumn{1}{c|}{\textbf{BEV}} & \textbf{ERS-BEV} \\ \hline
\multicolumn{1}{|c|}{Vehicle size class} & \multicolumn{2}{c|}{\textgreater 26 t GVW} \\ \hline
\multicolumn{1}{|c|}{Battery range}   & \multicolumn{1}{c|}{500 km} & 150 km        \\ \hline
\multicolumn{1}{|c|}{Effective battery capacity} & \multicolumn{1}{c|}{655 kWh} & 181 kWh \\ \hline
\multicolumn{1}{|c|}{Average speed}  & \multicolumn{2}{c|}{79 km/h} \\ \hline
\multicolumn{1}{|c|}{Energy consumption}  & \multicolumn{1}{c|}{1.31 kWh/km} & \multicolumn{1}{l|}{\begin{tabular}[c]{@{}l@{}}Overhead line connected:      1.42 kWh/km \\ Overhead line disconnected: 1.25 kWh/km\end{tabular}} \\ \hline
\multicolumn{1}{|c|}{\begin{tabular}[c]{@{}c@{}}Intermediate charging during \\ (un-)loading stop\end{tabular}}   & \multicolumn{1}{c|}{\begin{tabular}[c]{@{}c@{}}200 kW nominal\\ (166 kW effective )\end{tabular}} & \textbf{-} \\ \hline
\multicolumn{1}{|c|}{\begin{tabular}[c]{@{}c@{}}Intermediate charging during \\ driving time break\end{tabular}} & \multicolumn{1}{c|}{\begin{tabular}[c]{@{}c@{}}500 kW\\ (415 kW effective)\end{tabular}}   & \textbf{-}  \\ \hline
\multicolumn{1}{|c|}{Power supply from overhead line}  & \multicolumn{1}{c|}{-} & 400 kW \\ \hline
\end{tabular}%
}
\end{table}

The structure of the synthetic daily driving profiles is illustrated below using the example of a profile with an operating time of 8~hours (Figure~\ref{fig_a2}, panel~a). The profile starts with a fully charged battery in the depot. For the case of ERS trucks, the share of electrified roads is assumed according to the motorway share of the transport relations on which the profile is based (assuming that all motorways are equipped with overhead lines). At loading / unloading stops, a charging opportunity with a power of 200~kW is assumed for battery trucks, and a high-power charging option with 500~kW is assumed during the mandatory driver’s break. For ERS trucks, on the other hand, a stationary charging opportunity is assumed only at the depot.

The number of idle stops during loading / unloading is in the range between 2 and 3 for average daily mileages above 300~km, which makes up the bulk of the total vehicle fleet. Only for short transport relations, it is higher. For the sake of simplicity, we assume 2 idling stops for all synthetic driving profiles. However, the calculation of absolute idle time per day (which is more relevant for the energy system modelling) is based on the respective number of idle stops for all contributing transport relations.

\begin{figure}[ht!]
  \centering
  \subfloat{\includegraphics[width=\textwidth]{figures/Fig_SI02a.PNG}} \\ \centering
  \subfloat{(a)} \\
  \subfloat{\includegraphics[width=0.938\textwidth]{figures/Fig_SI02b.PNG}} \\ \centering
  \subfloat{(b)} \\
  \subfloat{\includegraphics[width=0.938\textwidth]{figures/Fig_SI02c.PNG}} \\ \centering
  \subfloat{(c)}
  \caption{Exemplary daily driving profile with an operating time of 8 hours}
  \label{fig_a2}
\end{figure}

The resulting time-resolved electricity demand and charging availability are shown in Figure~\ref{fig_a2} (for BEV in panel~b and ERS-BEV in panel~c).

The traffic model data does not include information on the starting time of a driving profile. Therefore, in the next step, the profiles are distributed over the day according to the empirically observed course of the daily mileage \citep{bast_2022}, taking into account their relative frequency (i.e. the mileage represented by them). For each daily operating period, two different starting times are defined. The operating durations of eight and nine hours, which are particularly strongly represented, are given three and four different start times respectively. The resulting temporal distribution of traffic volumes is shown in Figure~\ref{fig_a3}. The contribution of the example profile shown above (eight hours operating time, with three different starting times) corresponds to the dark blue coloured areas in Figure~\ref{fig_a3}. The resulting electricity demands and charging availabilities of the profiles are fed into the DIETER model.

\begin{figure}[ht!]
    \centering
    \includegraphics[width=\textwidth]{figures/Fig_SI03.PNG}
    \caption{Distribution of daily driving patterns over the course of the day}
    \label{fig_a3}
\end{figure}

With the procedure described, we have generated an ensemble of operational profiles for domestic German heavy road freight transport that reflect well both the macroscopic parameters of the German transport sector (mileage, distribution of daily driving distances, daily pattern of traffic volume) and operational boundary conditions (loading / unloading times, mandatory driver’s breaks). 

\clearpage

\subsection{Additional results for the baseline model specification}

In the following, we provide additional results for the baseline model specification discussed in the main part of the paper. This includes the effects of electrified HDV on capacity and yearly use of electricity storage, electrolysis capacity, the shares of renewable energy sources in overall demand, and carbon emissions, as well as additional time series results.

\subsubsection{Electricity storage capacity and use}\label{sec: si storage}

\begin{figure}[ht!]
    \centering
    \textcolor{white}{\frame{\includegraphics[width =0.65\textwidth]{figures/Fig_SI04_svg-raw.pdf}}}
    \caption{Effects of different HDV options on optimal electricity storage energy capacity (top panel), and on storage output power (bottom panel)}
    \label{fig_04}
\end{figure}

Figure~\ref{fig_04} shows optimal electricity storage energy capacities in the reference and the changes induced by HDV (upper panel), as well as optimal storage output power rating in the reference, and corresponding effects of HDV (lower panel). Effects are differentiated by technology, i.e.,~for lithium-ion batteries and power-to-gas-to-power (P2G2P) storage. Batteries come with relatively high energy-specific investment costs and relatively low power-specific costs, which makes them a typical short-duration storage technology with a low energy-to power ratio and many yearly cycles; P2G2P, conversely, comes with very low energy-specific higher power-specific costs, and is thus optimally used as a long-duration storage technology \cite{sepulveda_2021}. Pumped hydro storage is assumed to be fixed to TYNDP scenario assumptions because of limited expansion opportunities in Germany.

As for storage energy capacity, long-duration electricity storage clearly dominates lithium-ion batteries because of lower energy-specific investments costs. We further find the largest impact of HDV in the case of on-site, i.e.,~inflexible hydrogen provision. Here, the installed energy capacity is nearly four times as high as in the reference. As hydrogen has to be generated largely at the time of demand, and the potential to expand fossil generators is limited, this means that long-duration storage is used to balance variable renewable electricity generation with electricity demand for hydrogen production. For inflexible BEV or ERS-BEV, as well as centralized hydrogen supply, long-duration electricity storage capacity is expanded to a much smaller extent; and it is nearly zero for centralized PtL despite the highest additional electricity demand, as liquid e-fuels come with even cheaper bulk storage options.

When it comes to optimal storage output power, (in-)directly electrified HDV have a noticeable effect on both batteries and long-duration storage technologies (lower panel of Figure~\ref{fig_04}). Overall, BEV or ERS-BEV with V2G substantially reduce the required storage power, as truck batteries substitute stationary batteries, which would otherwise be needed to balance diurnal fluctuations of solar PV. Inflexible charging of BEV or ERS-BEV conversely increases optimal storage power. For long-duration storage, impacts of HDV on power output capacity are largely similar as those for energy capacity, with by far the largest effect for inflexible on-site hydrogen generation.


\subsubsection{Electrolysis capacity}\label{sec: electrolysis capacity}
\begin{figure}[ht!]
    \centering
    \textcolor{white}{\frame{\includegraphics[width =0.3\textwidth]{figures/Fig_SI05_svg-raw.pdf}}}
    \caption{Optimal electrolysis capacity}
    \label{fig_05}
\end{figure}

Optimal investments into electrolyzers are shown in Figure~\ref{fig_05}. It can be seen that only investments into proton exchange membrane (PEM) electrolyzers are made, which have lower conversion losses, but higher specific investment costs than chlor-alkali electrolyzers (ALK). This is because renewable energy is relatively scarce and costly in our scenario, and higher energy efficiency accordingly matters more than lower investment costs. This changes if Germany is modeled in isolation, as this leads to larger renewable surplus generation, which in turn makes lower-cost ALK technology a part of the optimal electrolysis technology mix (compare also~\cite{stoeckl_2021}).

The electrolysis capacity required to supply the German HDV fleet is highest in the PtL scenario with 24.4~GW, as e-fuels require more hydrogen than the two FCEV cases. Among the hydrogen scenarios, inflexible on-site generation further leads to a substantially higher optimal electrolysis capacity of 19.1~GW compared to 13.5~GW for centralized electrolysis. This is, again, because on-site electrolyzers have to produce hydrogen with a similar time profile as the hydrogen demand at filling stations. In contrast, centralized electrolyzers can use low-cost production-site hydrogen storage to balance fluctuations of renewable supply and hydrogen demand.

\subsubsection{Renewable shares}
\begin{figure}[ht!]
    \centering
    \textcolor{white}{\frame{\includegraphics[width =0.7\textwidth]{figures/Fig_SI06_svg-raw.pdf}}}
    \caption{Share of renewable electricity generation in electricity demand (Germany only)}
    \label{fig_06}
\end{figure}

Figure~\ref{fig_06} shows the shares of renewable electricity generation in electricity demand for Germany for alternative HDV electrification scenarios. It turns out that the minimum renewable share constraint of 80\% is only binding in the reference and the FCEV Centralized scenario. Flexible BEV and ERS-BEV, especially if combined with V2G, increase the renewable share by almost up to four percentage points in the BEV Flex V2G scenario. This is because vehicle batteries can foster the integration of additional solar PV generation, even beyond the electricity demand related to the vehicles themselves. This is, in contrast, not the case for inflexibly charged BEV or ERS-BEV.

Centralized hydrogen and PtL supply chains hardly increase the optimal renewable share, despite relatively high flexibility potentials related to their low-cost centralized storage options. This is because these flexibility benefits are outweighed by the sheer amount of additional renewable electricity needed for HDV, which is substantially higher than in BEV or ERS-BEV cases (compare Figure~\ref{fig_01}) and thus causes increasing integration efforts. On-site electrolysis, in turn, leads to a relatively strong increase of the renewable share, but only because fossil generators are often already at their assumed expansion limits during the hours of hydrogen production.

\subsubsection{Direct carbon emissions}\label{sec: CO2}

\begin{figure}[ht!]
  \centering
  \textcolor{white}{\frame{\includegraphics[width =0.86\textwidth]{figures/Fig_SI10_svg-raw.pdf}}}
  \caption{Direct CO$_{2}$ emissions from electricity generation. The left panel shows annual emissions from Germany, and the right panel shows emissions from Germany's neighboring countries. Each panel contains, on the left, the overall emissions of the reference without electrified HDV; the right-hand side shows differences between HDV scenarios and the reference.}
  \label{fig_CO2}
\end{figure}

Among all scenarios, CO$_{2}$ emissions increase the most if the HDV fleet uses e-fuels or hydrogen (Figure~\ref{fig_CO2}). This is a direct consequence of additional electricity generation from natural gas in these scenarios. Among the two hydrogen cases, on-site electrolysis at filling stations leads to lower emission impacts, as its temporal flexibility restrictions limit the possibility to increase imports or generation from natural gas plants; instead, additional renewable energy, combined with long-duration electricity storage, is used (compare Section~\ref{sec: capacity and dispatch effects}). Emission effects are smaller for BEV and ERS-BEV, and even negative for flexibly charged BEV, especially if combined with V2G. The latter is driven by an additional expansion of solar PV facilitated by V2G. For neighboring countries, relative emission effects are smaller, as by assumption they have lower renewable energy shares and, in turn, higher emissions, as well as no electrified truck fleets.

\subsubsection{Time series results}
% maybe RLDC: 6 overall? one for BEV, one for BEV-V2G, one for ERS-BEV, and one each for PtL and the two H2 options?

Figure~\ref{fig_07} shows additional electricity time series for the BEV~Flex and ERS-BEV~Flex scenarios without V2G, which are omitted in Figure~\ref{fig_03} because of space restrictions. The Figure also differentiates the grid electricity demand of HDV for electricity directly used for driving in ERS-BEV (Grid2Wheel) and battery charging (Grid2Bat). It can be seen that ERS-BEV are partly charging their batteries while being connected to the catenary.

\begin{figure}[ht!]
    \centering
    \textcolor{white}{\frame{\includegraphics[width =0.96\textwidth]{figures/Fig_SI07_svg-raw.pdf}}}
    \caption{Additional time series of electricity demand, complementary to Figure~\ref{fig_03}}
    \label{fig_07}
\end{figure}

Figure~\ref{fig_03b} is complementary to Figure~\ref{fig_03} as it does not show residual load, but the dual variables of the energy balance, which are interpreted as wholesale electricity market prices. These largely follow the residual load: prices are generally lower when the residual load is lower.

\begin{figure}[htp!]
    \centering
    \textcolor{white}{\frame{\includegraphics[width =0.96\textwidth]{figures/Fig_03b_svg-raw.pdf}}}
    \caption{Sample of electricity time series five exemplary days: ``HDV electricity demand'' illustrates electricity flowing to BEV or ERS-BEV batteries, including amounts later charged back via V2G, as well as electricity needs of hydrogen and PtL supply chains; ``HGV discharging to the grid'' are V2G power flows from BEV or ERS-BEV to the grid; and ``Electricity price'' is the wholesale electricity market price in the respective hour. The samples start on a Saturday.}
    \label{fig_03b}
\end{figure}

Figures~\ref{fig_09} and \ref{fig_09} show additional exemplary time series not for the aggregate HDV fleet, but for two specific vehicle profiles. It can be seen that these profiles are ``peakier'' than aggregate fleet profiles. Again, it can be seen that ERS-BEV are partly charging their batteries while connected to the catenary. Also, the sequence of Grid2Bat and V2G operations is more clearly visible than in the aggregate fleet time series.

\begin{figure}[htp!]
    \centering
    \textcolor{white}{\frame{\includegraphics[width =0.96\textwidth]{figures/Fig_SI08_svg-raw.pdf}}}
    \caption{Time series of electricity demand for vehicle profile nr. 05}
    \label{fig_08}
\end{figure}

\begin{figure}[htp!]
    \centering
    \textcolor{white}{\frame{\includegraphics[width =0.96\textwidth]{figures/Fig_SI09_svg-raw.pdf}}}
    \caption{Time series of electricity demand for vehicle profile nr. 16}
    \label{fig_09}
\end{figure}

\clearpage

\subsection{Sensitivity analyses}\label{sec: sensitivities}

\subsubsection{Effects of the European interconnection}\label{sec: sensitivity island}

In the main part of the paper we show results for a central European interconnection where HDVs in Germany may benefit from the flexibility provided by geographical balancing. In the following, we compare results for alternative model runs where the German power sector is modeled in isolation. This helps to separate the effects of the European interconnection on results. It also gives a qualitative indication of how results are potentially distorted in model analyses that focus on single countries.

% 4-panel graph: system costs and electricity prices paid by HDV, for interconnected case and DE only
% CG remove 'system' word in the plot
\begin{figure}[ht!]
    \centering
    \textcolor{white}{\frame{\includegraphics[width =\textwidth]{figures/Fig_SI11_svg-raw.pdf}}}
    \caption{Changes in yearly power sector costs and electricity demand induced by different HDV options (left panels), and average wholesale market prices of charging electricity (right panels).}
    \label{fig_11}
\end{figure}

Results are qualitatively similar for a case where the German power sector is modeled in isolation from its neighboring countries (lower left panel). Here, flexible BEV with V2G are even slightly more beneficial than in the setting where Germany is interconnected with its neighbor countries (1.6~bn Euros vs.~1.8~bn Euros additional costs), and inflexible BEV operations are slightly worse (3.9~bn Euros vs.~3.8~bn Euros). This is because the German power system in isolation is much more flexibility-constrained as compared to an interconnected one.

If Germany is modeled as an electric island, average electricity prices for HDV options that are particularly flexible (BEV~Flex, BEV~Flex~V2G, ICEV~PtL) are lower than in the interconnected case (lower right panel of Figure~\ref{fig_11}). The reason is that additional V2G flexibility is more beneficial in such a flexibility-constrained setting. In contrast, temporally less flexible options are forced to pay higher average prices for charging electricity.

% 4-panel graph: 2 on capacity effects and 2 on yearly dispatch effects
\begin{figure}[ht!]
    \centering
    \textcolor{white}{\frame{\includegraphics[width =\textwidth]{figures/Fig_SI12_svg-raw.pdf}}}
    \caption{Effects of different HDV options on optimal electricity generation capacity (left panels), and on yearly electricity generation (right panels)}
    \label{fig_12}
\end{figure}

If Germany is modeled as an island, the missing European interconnection no longer allows for a notable import capacity, so both overall and firm capacity needs to increase (lower left panel of Figure~\ref{fig_12})). Further, HDV-induced effects on optimal generation capacities are generally smaller than if the interconnection with neighboring countries is considered. This is because there is already a larger power plant fleet in the respective baseline, with more under-utilized capacity available. This particularly benefits the temporally flexible BEV and ERS-BEV options (especially with V2G), as these allow making use of substantial parts of renewable surplus energy present in the reference. In the hydrogen and PtL cases, this effect vanishes or even reverses, as the additional load induced by HDV is much higher than the renewable surplus energy available in the reference

\begin{comment}
\begin{itemize}
    \item in DE isolated case, not only gas, but also hard coal, 20 percent non-renewable electricity ...
\end{itemize}

\begin{itemize}
    \item storage effects much lower in DE isolated setting because of much higher capacity in baseline; BEV V2G even absolute reduction, HDV substitute grid storage
\end{itemize}
\end{comment}

\begin{figure}[ht!]
    \centering
    \textcolor{white}{\frame{\includegraphics[width =\textwidth]{figures/Fig_SI13_svg-raw.pdf}}}
    \caption{Renewable share with neighboring countries (left panels), and on DE isolated (right panels)}
    \label{fig_13}
\end{figure}

\subsubsection{Effects of a capacity constraint for onshore wind power}\label{sec: sensitivity wind}

In the main part, we assume that investments into solar PV and wind onshore are possible without limits. We consider this to be a meaningful assumption in order to highlight general effects, which are not overly driven by exogenous constraints, and should thus be generally applicable also to other world regions. Yet in the specific German case, especially the capacity expansion of onshore wind power is in fact likely to be constrained in a 2030 perspective. Because of long lead times of planning and admission processes, it could be assumed that wind power expansion may not exceed 110 GW by 2030. Figures~\ref{fig_14}, \ref{fig_15} and \ref{fig_16} show how such an onshore wind power cap would influence results. It turns out that results are largely robust.

%Figure: cost and price effects (4 panels)
\begin{figure}[ht!]
    \centering
    \textcolor{white}{\frame{\includegraphics[width =\textwidth]{figures/Fig_SI14_svg-raw.pdf}}}
    \caption{Changes in yearly power sector costs and electricity demand induced by different HDV options (left panels), and average wholesale market prices of charging electricity (right panels).}
    \label{fig_14}
\end{figure}

%Figure: capacity and yearly energy effects (4 panels)
\begin{figure}[ht!]
    \centering
    \textcolor{white}{\frame{\includegraphics[width =\textwidth]{figures/Fig_SI15_svg-raw.pdf}}}
    \caption{Effects of different HDV options on optimal electricity generation capacity (top panels), and on yearly electricity generation (bottom panels)}
    \label{fig_15}
\end{figure}

%Figure: renewable shares (1 panel)
\begin{figure}[ht!]
    \centering
    \textcolor{white}{\frame{\includegraphics[width =\textwidth]{figures/Fig_SI16_svg-raw.pdf}}}
    \caption{Renewable share with neighboring countries (left panels), and on DE isolated (right panels)}
    \label{fig_16}
\end{figure}


% maybe later on: sensitivities with other weather years (pv, wind on & wind off time series)

\clearpage

\putbib
\end{bibunit}

\end{document}
\endinput
%%
%% End of file `elsarticle-template-num.tex'.
