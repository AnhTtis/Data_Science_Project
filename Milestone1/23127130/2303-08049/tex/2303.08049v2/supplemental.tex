\documentclass[ superscriptaddress, aps, pre, twocolumn, floatfix  ]{revtex4-2}

\pdfoutput=1 
\usepackage[ usenames, dvipsnames ]{xcolor}
\usepackage{amsmath, amsfonts, amssymb, amsthm, amstext, bm, bbm,  multirow}
\usepackage[colorlinks=true, linkcolor=blue,citecolor=red, urlcolor=blue, hypertexnames=true]{hyperref}
%\usepackage{graphicx, subcaption}
%\usepackage{ragged2e}
%\DeclareCaptionJustification{justified}{\justifying}
%\captionsetup{justification=justified,singlelinecheck=false,labelfont=large}



\newcommand{\brac}[1]{\left( #1 \right) }


\usepackage{graphicx}
\usepackage{hyperref}
\usepackage{color}
\graphicspath{{figures/}}

\newcommand{\cami}[1] {\textcolor{red}{**- #1 -**}}

%\newcommand{\sectionprl}[1]{{\em #1}\/.\,---\,}

\newcommand{\rme}{{\rm e}}
\newcommand{\rmd}{{\rm d}}
\newcommand{\rmi}{{\rm i}}


\setcounter{equation}{0}

\setcounter{figure}{0}

\renewcommand{\theequation}{S\arabic{equation}}

\renewcommand{\thefigure}{S\arabic{figure}}

% -- short commands --
\newcommand{\figref}[1]{Fig.~\ref{fig:#1}} 



\begin{document}
\newcommand{\titlename}{\underline{\textsc{Supplemental material}}\\ \bigskip Out-of-time-ordered correlator in the one-dimensional Kuramoto-Sivashinsky and Kardar-Parisi-Zhang equations}


\title[]{\titlename}

\author{Dipankar Roy}
\email[ ]{dipankar.roy@icts.res.in}
\affiliation{International Centre for Theoretical Sciences, Tata Institute of Fundamental Research,
	Bangalore 560089, India}


\author{David A. Huse}
\email[ ]{huse@princeton.edu}
\affiliation{Physics Department, Princeton University, Princeton, NJ, 08544, USA}



\author{Manas Kulkarni}
\email[ ]{manas.kulkarni@icts.res.in}
\affiliation{International Centre for Theoretical Sciences, Tata Institute of Fundamental Research,
	Bangalore 560089, India}
\date{\today}



%\onecolumngrid
%\begin{center} 
%	\textbf{ \textit{\textcolor{red}{\huge -- Incomplete --} } }
%\end{center}
%\twocolumngrid



\maketitle




\tableofcontents 

%\vspace{0.8cm}

% -------------------------
% simulation protocol
% -------------------------

\section{Numerical methods for the KPZ equation \label{sec:app-nmkpz}}
Here we discuss the numerical method we employ for solving the KPZ equation. The Lam-Shin method \cite{1998-lam-shin} is a finite-difference technique where central difference is used for the second-derivative term and the nonlinear term is handled with a modified difference term adapted for the 1D KPZ equation. The height profile $h_n$ at the $n$-th grid point (assuming periodic boundary conditions) satisfies
\begin{equation}
\frac{ \textrm{d} h_{n} }{ \textrm{d} t} = C_{n} + g N_n + \xi_n ,
\label{eq:ls-kpz}
\end{equation}
where 
\begin{equation}
\begin{aligned}
C_n  & = h_{n+1} + h_{n-1} - 2 h_{n}, \\
N_n  & = \frac{1}{3} \big[ \brac{ h_{n+1} - h_{n} }^{2} + \brac{ h_{n+1} - h_{n} }\brac{ h_{n} - h_{n-1} } \\
& \quad \  + \brac{ h_{n} - h_{n-1} }^{2} \big] .
\end{aligned}
\end{equation}
Note that here we set $\Delta x = L/ N $ to $1$ \cite{1998-lam-shin}. Thus the height $h_{n}$ in the Lam-Shin numerical scheme is directly coupled only to the nearest neighbours. With this discretization shown in Eq.~\eqref{eq:ls-kpz}, we use Euler-Maruyama method for time-marching \cite{1998-lam-shin, 1992-kloeden-platen}.



\bibliographystyle{apsrev4-1}
\bibliography{ks-otoc-refs.bib}


	

\end{document}



