\documentclass[
superscriptaddress,
preprint,amsmath,amssymb,aps,prl]{revtex4-2}
\usepackage{verbatim}
\usepackage{multirow}
\usepackage{caption}
\usepackage{graphicx}% Include figure files
\usepackage{dcolumn}% Align table columns on decimal point
\usepackage{bm}
\usepackage{multirow}
\usepackage{braket} % for braket notation
\renewcommand{\arraystretch}{1.4} % pmatrix vertical spacing (Rotation transformations of the sigma^{flat})

\begin{document}

\title{Supplemental Material for Optical transitions of a single nodal ring in SrAs$_3$: radially and axially resolved characterization}

\author{Jiwon Jeon}
 \thanks{These authors contributed equally to this work.}
\affiliation{%
 Natural Science Research Institute, University of Seoul, Siripdaero 163, Seoul 02504, Korea
 }%
 \affiliation{%
 Physics Department, University of Seoul, Siripdaero 163, Seoul 02504, Korea
 }%

\author{Jiho Jang}%
 \thanks{These authors contributed equally to this work.}
 \affiliation{%
 Department of Physics and Astronomy, Seoul National University, Seoul 08826, Korea
 }%

\author{Hoil Kim}%
 \affiliation{%
 Center for Artificial Low Dimensional Electronic Systems, Institute for Basic Science (IBS), Pohang 37673, Korea
 }%
  \affiliation{%
 Department of Physics, Pohang University of Science and Technology (POSTECH), Pohang 37673, Korea
 }%

\author{Taesu Park}%
 \affiliation{%
 Department of Chemistry, Pohang University of Science and Technology (POSTECH), Pohang 37673, Korea
 }%

 \author{DongWook Kim}%
 \affiliation{%
 Department of Physics, Hanyang University, Seoul 04763, Korea
 }%

 \author{Soonjae Moon}%
 \affiliation{%
 Department of Physics, Hanyang University, Seoul 04763, Korea
 }%

\author{Jun Sung Kim}%
 \affiliation{%
 Center for Artificial Low Dimensional Electronic Systems, Institute for Basic Science (IBS), Pohang 37673, Korea
 }%
  \affiliation{%
 Department of Physics, Pohang University of Science and Technology (POSTECH), Pohang 37673, Korea
 }%

\author{Ji Hoon Shim}%
 \affiliation{%
 Department of Chemistry, Pohang University of Science and Technology (POSTECH), Pohang 37673, Korea
 }%

\author{Hongki Min}%
 \email{hmin@snu.ac.kr}
 \affiliation{%
 Department of Physics and Astronomy, Seoul National University, Seoul 08826, Korea
 }%

\author{Eunjip Choi}%
\email{echoi@uos.ac.kr}
 \affiliation{%
 Physics Department, University of Seoul, Siripdaero 163, Seoul 02504, Korea
 }%

\date{\today}% It is always \today, today,
             %  but any date may be explicitly specified
\newcommand{\beginsupplement}{%
        \setcounter{table}{0}
        \renewcommand{\thetable}{S\arabic{table}}%
        \setcounter{figure}{0}
        \renewcommand{\thefigure}{S\arabic{figure}}%
     }


%---------------------------------------
%\end{comment}

%\keywords{Suggested keywords}%Use showkeys class option if keyword
                              %display desired

\maketitle
\beginsupplement
\tableofcontents

\newpage
\section {\uppercase\expandafter{\romannumeral1}. Temperature-dependent optical conductivity}
\subsection{\uppercase\expandafter{\romannumeral1-1} Fitting analysis of the Drude, phonon, and interband contributions}

%이곳에 SrAs3의 제작방법, 특성 등을 서술하고 e, h, QO(T), $E_F$ 를 정리한 표가 들어가길 희망합니다.
This section presents $T$-dependent optical conductivity $\sigma_{1}(\omega)$ and fitting analysis result. 
\newline

Figure \ref{fig:sm:sigma} shows the optical conductivity $\sigma_{1}(\omega)$ for $E \parallel k_x$ and $E \parallel k_z$. They were calculated by fitting reflectivity $R_{1}(\omega)$ using Kramers-Kronig (KK) constrained variational dielectric functions of RefFit \cite{doi:10.1063/1.1979470}.
\newline

Figure \ref{fig:sm:fitcomponent} shows the Drude, phonon, and interband contributions to the optical conductivity: $\sigma_1(\omega) = \sigma^{\mathrm{D}}(\omega) + \sigma^{\mathrm{Ph}}(\omega) + \sigma^{\mathrm{IB}}(\omega)$, where $\sigma^{\mathrm{D}}(\omega) = \frac{\omega^2_{p}}{4\pi}\frac{\gamma}{(\gamma^2+\omega^2)}$ is the Drude conductivity, $\sigma^{\mathrm{Ph}}(\omega) = \displaystyle\sum_{j=1}^{2}\frac{\omega^2_{p,j}\omega^2\gamma_j}{(\omega^2_{0,j}-\omega^2)^2+\omega^2\gamma_{j}^2}$ is the phonon contribution described by Lorentzian peaks.
Here, the fitting parameters $\omega_{p}$ and $\gamma$ are the plasma frequency and the width of the Drude peak. $\omega_{p,j}$,$ \omega_{0,j}$ and $\gamma_{j}$ are the plasma frequency, peak position, and peak width of the two phonon peaks, respectively, for $j=1$ and $2$.
For the interband conductivity, $\sigma^{\mathrm{IB}}(\omega)$ is obtained by subtracting $\sigma^{\mathrm{D}}(\omega)$ and $\sigma^{\mathrm{Ph}}(\omega)$ from $\sigma_{1}(\omega)$, $\sigma^{\mathrm{IB}}(\omega)$ = $\sigma_{1}$$(\omega)$ $-$ [$\sigma^{\mathrm{Ph}}(\omega)$ $+$ $\sigma^{\mathrm{D}}(\omega)$] 
\newline

Figure \ref{fig:sm:seperate} displays the fitting curves $\sigma^{\mathrm{D}}(\omega)$, $\sigma^{\mathrm{Ph}}(\omega)$ and $\sigma^{\mathrm{IB}}(\omega)$ for 5 K $\leq$ $T$ $\leq$ 300 K for $E \parallel k_x$ and $E \parallel k_z$. 
\newline

Figure \ref{fig:sm:drude} shows the fitting parameters $\omega_{p}$ and $\gamma$ of the Drude peak as a function of $T$. 
\newline

Table \ref{table:stable1} summarizes the Drude and phonon fitting parameters at $T$ = 5 K for the two light polarizations. 
\newline


\newpage

\begin{figure}[h]
\caption{\label{fig:sm:sigma} Temperature dependent  $\sigma_1(\omega)$ for $E \parallel k_x$ and $E \parallel k_z$.
}
\includegraphics[width=1\columnwidth]{sm_sigma}% Here is how to import EPS art
%\captionsetup{justification=raggedright,singlelinecheck=false}
\end{figure}

\newpage

\begin{figure}[h]
\caption{\label{fig:sm:fitcomponent} Decomposing  $\sigma_1(\omega)$ into the Drude, phonon, and interband conductivities. }
\includegraphics[width=1\columnwidth]{sm_fitcomponent}% Here is how to import EPS art
%\captionsetup{justification=raggedright,singlelinecheck=false}

\end{figure}

\newpage

\begin{figure}[h]
\caption{\label{fig:sm:seperate} Fitting curves for the Drude, phonon, and interband conductivities for 5 K$\leq$ $T$ $\leq$ 300 K.}
\includegraphics[width=1\columnwidth]{sm_seperate}% Here is how to import EPS art
%\captionsetup{justification=raggedright,singlelinecheck=false}
\end{figure}


\newpage
%\subsection {\uppercase\expandafter{\romannumeral1}-2. Table S1 : Fitting parameter at $T = 5$ K}

%\subsection {\uppercase\expandafter{\romannumeral1}-3. Figure S2 : The Drude fitting paramters for $E \parallel k_x$ and $E \parallel k_z$}
\begin{figure}[h]
%\vspace{15mm}
\caption{\label{fig:sm:drude}
$T$-dependence of the  Drude fitting parameter $\omega^2_{p}$ and $\gamma$.
}
\includegraphics[width=1\columnwidth]{sm_drude}% Here is how to import EPS art
%\captionsetup{justification=raggedright,singlelinecheck=false}
\end{figure}
%\newpage

\begin{table}[h]
\centering
\caption{\label{table:stable1}Drude and Phonon fitting parameters for $T$ = 5 K}
\setlength{\tabcolsep}{10pt}
%\newcommand{\arraystretch}{1.5}
\label{t4}
\begin{tabular}{c|cc|ccc|ccc}
\noalign{\smallskip}\noalign{\smallskip}\hline\hline
\multirow{2}{*}{(cm$^{-1}$)} & \multicolumn{2}{c|}{Drude} & \multicolumn{3}{c|}{Phonon 1} & \multicolumn{3}{c}{Phonon 2}\\
\cline{2-9}
      & $\omega_{p}$  & $\gamma$ & $\omega_0$ & $\omega_p$ & $\gamma$ & $\omega_0$ & $\omega_p$ & $\gamma$\\
\hline
 $E \parallel k_x$ & 500.00 & 16.00 & 117.99 & 200.00 & 1.15 & 171.98 & 300.32 & 1.22 \\
 $E \parallel k_z$ & 1540.40 & 25.24 & 153.07 & 168.63 & 3.68 & 240.72 & 229.92 & 4.69 \\
\hline
\hline
\end{tabular}
\end{table}
\newpage

\subsection{\uppercase\expandafter{\romannumeral1-2} Theoretical phonon frequency calculation}

\begin{table}[h]
\centering
\caption{\label{table:sm:phonon}Calculated phonon frequencies}
\setlength{\tabcolsep}{15pt}
%\newcommand{\arraystretch}{1.5}
\label{t4}
\begin{tabular}{c|cc|cc}
\noalign{\smallskip}\noalign{\smallskip}\hline\hline
\multirow{2}{*}{(cm$^{-1}$)} & \multicolumn{2}{c|}{Phonon 1} & \multicolumn{2}{c|}{Phonon 2}\\
\cline{2-5}
      & $\omega_{calc}$ & Mode & $\omega_{calc}$ & Mode\\
\hline
 $E \parallel k_x$ & 98.74 & $B_u$ & 154.44 & $B_u$\\
 $E \parallel k_z$ & 138.76 & $A_u$ & 219.82 & $A_u$\\
\hline
\hline
\end{tabular}
\end{table}



First-principles calculation of phonon modes in $\mathrm{SrAs_3}$ is performed within the density functional perturbation theory \cite{togo2015first} routine using Vienna ab-initio simulation package \cite{PhysRevB.54.11169}.
After obtaining the dynamical matrix of the relaxed structure, post-processing is performed with Phonopy code to obtain phonon frequencies and modes at the zone center.
Phonon peaks shown in the IR conductivity measurement are matched with calculated IR active phonon frequencies considering group theory analysis.

\newpage
\section{\uppercase\expandafter{\romannumeral2}. Model and methods}
\subsection{\uppercase\expandafter{\romannumeral2}-1. Band structures and first-principles calculations details} \label{sec:dft}
The electronic structure of the bulk $\mathrm{SrAs_3}$ is calculated using the density functional theory (DFT) calculation implemented in WIEN2K \cite{doi:10.1063/1.5143061}, using a full-potential linearized augmented plane wave method. The exchange-correlation functional of the modified Becke-Johnson (mBJ) potential is used for an accurate description of the valence and conduction bands in semimetallic $\mathrm{SrAs_3}$, as well as to avoid the bandgap underestimation of the Perdew-Berke-Ernzerhof (PBE) functional \cite{PhysRevLett.102.226401}. The core separation energy is set to $-6.0$ Ry, and $\mathrm{R_{MT}K_{MAX}}$ is chosen to be 7. For the self-consistent field (SCF) calculation, 7 $\times$ 8 $\times$ 7 $k$-mesh of $\mathrm{SrAs_3}$ is used.

\begin{figure}[h]
\includegraphics[scale=1]{sm_figure_band}% Here is how to import EPS art
\captionsetup{justification=raggedright,singlelinecheck=false}
\caption{Band structures of $\mathrm{SrAs_3}$ from first-principles calculations.}
\label{fig:sm:band}
\end{figure}

In Fig. \ref{fig:sm:band}, the band structure of $\mathrm{SrAs_3}$ is plotted along the high-symmetry points of the Brillouin zone, revealing a simple electronic structure with only a single nodal ring crossing the Fermi level near the $\mathrm{Y}$ point. This suggests that $\mathrm{SrAs_3}$ is an ideal candidate for studying nodal-ring semimetals.

\subsection{\uppercase\expandafter{\romannumeral2}-2. Model Hamiltonian}
The low-energy electronic structure of $\mathrm{SrAs_3}$ near the Y point can be described by the following minimal model Hamiltonian \cite{PhysRevB.95.045136, PhysRevLett.118.176402}:

\begin{widetext}
\begin{align} \label{eq:sm:ham}
H(\bm{k}) &= f_0(\bm{k})\sigma_0 s_0 + f_1(\bm{k})\sigma_1 s_0 + f_2(\bm{k})\sigma_2 s_0 + \Delta_{\rm SOC}\sigma_3 s_3, \\
f_0(\bm{k}) &= a_0 + a_x k_x^2 + a_{xy} k_x k_y + a_y k_y^2 + a_z k_z^2, \\
f_1(\bm{k}) &= b_0 + b_x k_x^2 + b_{xy} k_x k_y + b_y k_y^2 + b_z k_z^2, \label{eq:sm:ham_f1} \\
f_2(\bm{k}) &= \hbar v_z k_z,
\end{align}
\end{widetext}
where $\bm{k}$ is the wave vector measured from the Y point, and $\bm{\sigma}$ and $\bm{s}$ are Pauli matrices acting on the pseudospin and spin degrees of freedom, respectively, and $\Delta_{\rm SOC}$ is the strength of the spin-orbit coupling, which, as a leading approximation, we take as constant. Here $a_i$, $b_i$ ($i=0$, $x$, $xy$, $y$, $z$), and $v_z$ are material-dependent parameters determined from first-principles calculations and the experiments (see Sec. II-3)

The eigenvalues of the Hamiltonian are given by
\begin{equation}
E_{\pm}(\bm{k}) = f_0(\bm{k}) \pm \sqrt{|f_1(\bm{k})|^2 + |f_2(\bm{k})|^2 + \Delta_{\rm SOC}^2},
\end{equation}
where the conduction and valence bands are each doubly degenerate, and $f_0(\bm{k})$ is an energy tilt term. The band structure exhibits a gapped nodal ring defined by $f_1(\bm{k})=0$ and $f_2(\bm{k})=0$, that is,
\begin{equation} \label{eq:sm:nodal ring equation}
    b_x k_x^2 + b_{xy} k_x k_y + b_y k_y^2 = |b_0|, \ k_z=0.
\end{equation}
Note that this is an ellipse in the $k_z=0$ plane rotated by an angle $\phi_0 = 23.5^{\circ}$ clockwise with respect to the $k_x$ axis, and the lengths of the semi-major and semi-minor axes are given by $k_l = 0.073 \ \mathrm{\AA^{-1}}$ and $k_s = 0.063 \ \mathrm{\AA^{-1}}$, respectively, as shown in Fig. 1(d) in the main text with the average radius $k_0=\sqrt{k_l k_s}=0.068 \ \mathrm{\AA^{-1}}$ and $k_l/k_s = 1.16$.

\subsection{\uppercase\expandafter{\romannumeral2}-3. Determination of the model parameters} \label{sec:sm:params}
Overall, the band structure of $\mathrm{SrAs_3}$ calculated using the mBJ functional well describes the nodal-ring semimetal feature. However, there is a discrepancy in the optical conductivity between experimental results and calculations, such as the frequency where $\sigma_{zz}$ ($\sigma_{xx}$) starts to decrease (increase) and the position of the SOC-induced peak. These are related to the band overlap energy $2\varepsilon_0$ at the Y point and the strength of the SOC $\Delta_{\mathrm{SOC}}$, respectively. Moreover, the choice of the exchange-correlation functionals has a significant impact on the band overlap energy $2\varepsilon_0$, as the low-energy band structure is extremely sensitive to the exchange-correlation functional used in DFT calculations \cite{PhysRevB.95.045136,Kim2022}. 

Despite these limitations, the nodal-ring feature and the energy dispersion near the nodal point are retained, regardless of the choice of the exchange-correlation functional, except for their relative position. Therefore, we extract $2\varepsilon_0$ and $\Delta_{\mathrm{SOC}}$ from the experiment and use them as the model parameters. The other parameters in the model Hamiltonian in Eq. (\ref{eq:sm:ham}) are chosen to fit the band structure obtained from first-principles calculations.


\begin{table}[h]
\renewcommand{\arraystretch}{1.0}
\begin{tabular*}{0.9\linewidth}{@{\extracolsep{\fill}} cccccc }
\hline \hline
$a_{0}$ & $a_{x}$ & $a_{xy}$ & $a_{y}$ & $a_{z}$ & $\Delta_{\mathrm{SOC}}$ \\
 ($\mathrm{eV}$) & ($\mathrm{eV \cdot \AA^{-2}}$) & ($\mathrm{eV \cdot \AA^{-2}}$) & ($\mathrm{eV \cdot \AA^{-2}}$) & ($\mathrm{eV \cdot \AA^{-2}}$)  &  (eV) \\
  -0.0448 & 2.744 & -0.319 & 12.736  & -7.230 & 0.015 \\ \hline 
$m_{0}$ & $m_{x}$ & $m_{xy}$ & $m_{y}$ & $m_{z}$ & $\hbar v_z$ \\
 ($\mathrm{eV}$) & ($\mathrm{eV \cdot \AA^{-2}}$) & ($\mathrm{eV \cdot \AA^{-2}}$) & ($\mathrm{eV \cdot \AA^{-2}}$) & ($\mathrm{eV \cdot \AA^{-2}}$)  &  ($\mathrm{eV \cdot \AA^{-1}}$) \\
  0.0645 & -12.849 & -3.091 & -15.741 & 9.215 & 2.292 \\ \hline \hline
\end{tabular*}
\caption{The parameters for the Hamiltonian in Eq. (\ref{eq:sm:ham}).}
\end{table}

To calculate the optical conductivity of $\mathrm{SrAs_3}$, as shown in Fig. 3(b) in the main text, we need to set the Fermi energy. According to field-dependent Hall resistivity and quantum oscillation measurements, the hole density $n_h$ is one or two orders of magnitude higher than the electron density $n_e$ \cite{Kim2022}. Using this information, we can obtain the corresponding Fermi energy $E_{\mathrm{F}}=-15$ meV through the density calculation based on the Hamiltonian in Eq. (\ref{eq:sm:ham}). The negative value of the Fermi energy implies that the sample is hole-doped, as the maximum energy of the valence band is set to zero.


\subsection{\uppercase\expandafter{\romannumeral2}-4. Methods for calculations of optical conductivity}
The optical conductivity is calculated within the Kubo formalism \cite{mahan2000many}:
\begin{align} \label{eq:sm:kubo}
    \sigma_{ij}(\omega) &= -\frac{ie^2}{\hbar} \sum_{s, s'} \int \frac{d^dk}{(2\pi)^d}\frac{f_{s,\bm{k}} - f_{s',\bm{k}}}{\varepsilon_{s, \bm{k}} - \varepsilon_{s', \bm{k}}} \frac{M_i^{ss'}(\bm{k}) M_j^{s's}(\bm{k})}{\hbar\omega+\varepsilon_{s, \bm{k}} - \varepsilon_{s', \bm{k}} + i\gamma},
\end{align}
where $i, j = x, y, z$ are spatial directions, $\varepsilon_{s, \bm{k}}$ and $f_{s, \bm{k}} = 1/[1+e^{(\varepsilon_{s, \bm{k}}-\mu)/(k_\mathrm{B} T)}]$ are the eigenenergy and the Fermi-Dirac distribution function for the band index $s$ and wave vector $\bm{k}$, respectively, $\mu$ is the chemical potential, $\gamma$ is a phenomenological broadening term proportional to the inverse lifetime which takes $\gamma \rightarrow 0^{+}$ for the clean limit, and $M_i^{ss'}(\bm{k})=\braket{s,\bm{k}|\hbar\hat{v}_i|s',\bm{k}'}$ with the velocity operator $\hat{v}_i$ obtained from the relation $\hat{v}_i = \frac{1}{\hbar}\frac{\partial \hat{H}}{\partial k_i}$. From now on, we only consider the real part of the longitudinal optical conductivity.

% \subsection{Details of the optical conductivity calculations}
\section{\uppercase\expandafter{\romannumeral3}. Low-frequency asymptotic forms of the optical conductivity}
\subsection{\uppercase\expandafter{\romannumeral3}-1. Low-energy effective Hamiltonian}
Interestingly, the Hamiltonian in Eq. (\ref{eq:sm:ham}) can be rewritten as a form of a collection of gapped anisotropic graphene sheets at low energies, as we show explicitly below.
We introduce the polar coordinates $\bm{k} = (k_{\rho} \cos{\phi}, k_{\rho}\sin{\phi})$, which transforms $f_1(\bm{k})$ into the following form:
\begin{align}
    f_1(\bm{k}) &= \frac{\hbar^2}{2m_0}\left[
    \Phi(\phi)^2 k_{\rho}^2 - k_0^2 \right] + b_z k_z^2,
\end{align}
where $k_0 = \sqrt{k_l k_s}$ is the characteristic scale of the nodal ring size, which is nothing but the geometric mean of the lengths of the semi-major and semi-minor axes of the ellipse, and $m_0$ is the effective mass defined from the relation $b_0=\frac{\hbar^2 k_0^2}{2m_0}$. Here, $\Phi(\phi)$ describes the elliptical nature of the nodal ring defined by
\begin{align}
    \Phi(\phi) &= \sqrt{\frac{k_s}{k_l}\cos^2{(\phi+\phi_0)} + \frac{k_l}{k_s}\sin^2{(\phi+\phi_0)}},
\end{align}
which reduces to 1 for the circular case $(k_l = k_s)$. Note that the nodal ring in polar coordinates is given by $k_{\rho}(\phi) = k_0/\Phi(\phi)$. Near the nodal ring, by keeping only linear terms in $\bm{k}$, we can approximate Hamiltonian in Eq. (\ref{eq:sm:ham}) as 
\begin{align} \label{eq:sm:ham_gr}
    H(\bm{k}) = \Delta_{\mathrm{tilt}}(\bm{k})\sigma_0 s_0 + \hbar v_{\rho}(\phi) \delta k_{\rho} \sigma_1 s_0 + \hbar v_z k_z \sigma_2 s_0 + \Delta_{\mathrm {SOC}} \sigma_3 s_3,
\end{align}
where $v_{\rho}(\phi)$ is the angle-dependent velocity given by
\begin{equation} \label{eq:sm:velocity}
    v_{\rho}(\phi) =  \frac{\hbar k_0}{m_0} \Phi(\phi),
\end{equation}
and $\delta k_{\rho}$ is a wave vector measured from the nodal ring. Since the approximate Hamiltonian in Eq. (\ref{eq:sm:ham_gr}) contains only diagonal components $s_0$ and $s_3$ in spin space, it can be considered as two decoupled copies of $H_+$ and $H_-$ given by
\begin{align} \label{eq:sm:ham_gr_2by2}
    H_{\pm}(\bm{k}) = \Delta_{\mathrm{tilt}}(\bm{k})\sigma_0 + \hbar v_{\rho}(\phi) \delta k_{\rho} \sigma_1 + \hbar v_z k_z \sigma_2 \pm \Delta_{\mathrm {SOC}} \sigma_3.
\end{align}
Note that $H_+$ and $H_-$ have the same energy dispersion, leading to the same optical conductivity. This means that the total optical conductivity is $g=2$ times of the optical conductivity calculated from $H_+$ (or $H_-$), where $g$ denotes the band degeneracy factor. From now on, we will focus on $H_+$.


The Hamiltonian in Eq. (\ref{eq:sm:ham_gr_2by2}) takes the same form as that of graphene with different velocities ($v_{\rho}$ and $v_z$), an energy gap (2$\Delta_{\mathrm{SOC}}$) and an energy tilt ($\Delta_{\mathrm{tilt}}$).
% low-energy graphene with an energy gap induced by the spin-orbit coupling, energy tilt, and anisotropy (different velocities along the $k_{\rho}$ and $k_z$ direction). 
We emphasize that the graphene approximation is valid only at low energies near the nodal ring where the phase space for allowed interband transitions for a given frequency forms a toroidal shape enclosing the nodal ring \cite{PhysRevLett.119.147402}.

\subsection{\uppercase\expandafter{\romannumeral3}-2. Optical conductivity as a sum of gapped anisotropic graphene sheets}

At low frequencies, the interband contribution to the optical conductivity can be obtained by summing all the contributions from each gapped anisotropic graphene sheet along the nodal ring (see Eq. (3) in the main text). The optical conductivity of a single Dirac cone in a gapped anisotropic graphene sheet [Eq. (\ref{eq:sm:ham_gr_2by2})] is given by \cite{doi:10.1073/pnas.1809631115}
\begin{align} \label{eq:sm:opcd_gr}
    \sigma_{ii}^{\mathrm{gr}}(\omega, \phi) &= \frac{e^2}{16\hbar} \frac{v_{i}(\phi)^2}{v_{\rho}(\phi) v_z} \left[1+\left(\frac{2\Delta_{\mathrm{SOC}}}{\hbar\omega}\right)^2\right] \Theta(\hbar\omega - 2 \mathrm{max}[\Delta_{\mathrm{SOC}}, |\varepsilon_{\mathrm{F}}^{\mathrm{gr}}(\phi)|]).
\end{align}
Here $\varepsilon_{\mathrm{F}}^{\mathrm{gr}}(\phi)$ is the angle-dependent Fermi energy of a graphene sheet measured from the middle of the gap. The angle dependence arises from the energy tilt term ($\Delta_{\mathrm{tilt}}$). Note that $\varepsilon_{\mathrm{F}}^{\mathrm{gr}}(\phi)$ differs from the Fermi energy of the material ($E_{\mathrm{F}}$), as shown in Fig. \ref{fig:sm:sfig_th1}. The optical conductivity vanishes in the low-frequency range due to the Pauli blocking for $\hbar\omega < 2|\varepsilon_{\mathrm{F}}^{\mathrm{gr}}(\phi)|$ or the SOC-induced optical gap for $\hbar\omega < 2\Delta_{\mathrm{SOC}}$. As the frequency increases, the optical conductivity approaches that without SOC, showing flat behavior.

Plugging Eq. (\ref{eq:sm:opcd_gr}) into Eq. (3) in the main text, we get
\begin{align}
    \label{eq:sm:opcd_int}
    \sigma_{ii}(\omega) &= g \frac{e^2}{16\hbar}
    \left[
    \int_0^{2\pi} \frac{d\phi}{2\pi} \frac{v_{i}(\phi)^2}{v_{\rho}(\phi) v_z} \mathcal{F}_{ii}(\phi) \Theta(\hbar\omega - 2 \mathrm{max}[\Delta_{\mathrm{SOC}}, |\varepsilon_{\mathrm{F}}^{\mathrm{gr}}(\phi)|])
    \right]
    \left[
    1+\left(\frac{2\Delta_{\mathrm{SOC}}}{\hbar\omega}\right)^2
    \right].
\end{align}
The step function $\Theta(x)$ term in Eq. (\ref{eq:sm:opcd_int}) inside the integral restricts the range of angle $\phi$ where the interband transitions are allowed for a given frequency $\omega$, making the calculation of the integral non-trivial. In order to understand the role of the Fermi energy $\varepsilon_{\mathrm F}^{\mathrm{gr}}(\phi)$ and $\Delta_{\mathrm{SOC}}$ in optical conductivity, we present plots of the Fermi surface and the corresponding band structures of graphene sheets for selected values of $\phi$ in Fig. \ref{fig:sm:sfig_th1}. In the current experiment, the Fermi energy $E_{\mathrm{F}}$ of $\mathrm{SrAs_3}$ lies within the SOC-induced gap for at least certain range of angle, as shown in Fig. \ref{fig:sm:sfig_th1}(b).

\begin{figure}
\includegraphics[scale=1.0]{sm_figure_ring}
\captionsetup{justification=raggedright,singlelinecheck=false}
\caption{(a) Schematic illustration of the Fermi surface in the $k_z=0$ plane, with the electron (hole) pocket indicated in red (blue) and the nodal ring depicted by the black solid line. (b) The corresponding band structures of graphene sheets are plotted for a selected value of $\phi$ with the conduction (valence) bands colored red (blue). The black curves represent the conduction band minimum and valence band maximum and the black dashed line represents the Fermi energy $E_{\mathrm{F}}$. The orange, green, purple arrows indicate the onset energies of the interband transitions at different values of $\phi$. The allowed ranges of angles for the corresponding interband transitions are presented with the same color in both panels (a) and (b).}
\label{fig:sm:sfig_th1}
\end{figure}
For $\hbar\omega < 2\Delta_{\mathrm{SOC}}$, we find that the $\Theta(x)$ term in Eq. (\ref{eq:sm:opcd_int}) vanishes for all $\phi$ as can be found in Fig. \ref{fig:sm:sfig_th1}(b), indicating that no interband transitions are allowed. At $\hbar\omega = 2\Delta_{\mathrm{SOC}}$, the step function becomes nonzero in a certain range of angles (orange lines in Fig. \ref{fig:sm:sfig_th1}) and the interband transitions set in. As the frequency increases, the range of angles for which interband transitions are allowed expands (green lines in Fig. \ref{fig:sm:sfig_th1}) and eventually reaches $0<\phi<2\pi$ (purple lines in Fig. \ref{fig:sm:sfig_th1}).

The important thing to note is that the threshold frequency for interband transitions is fixed at $\hbar\omega = 2\Delta_{\mathrm{SOC}}$ as long as the Fermi energy $E_{\mathrm{F}}$ lies within the SOC-induced gap for at least certain range of angle. As a result, the position of the SOC-induced optical peak is $\hbar\omega = 2\Delta_{\mathrm{SOC}}$, while the amplitude of the optical peak could vary depending on $E_{\mathrm{F}}$.

The angular integral in Eq. (\ref{eq:sm:opcd_int}) cannot be solved analytically. However, for frequencies above 2$\mathrm{max}[\varepsilon_{\mathrm{F}}^{\mathrm{gr}}(\phi)]$ where interband transitions are allowed for the entire range of angles ($0<\phi<2\pi$), we can analytically obtain the integral as follows:
\begin{align}
    \label{eq:sm:opcd_analytic}
    \sigma_{ii}(\omega) &=
        \sigma_{ii}^{\mathrm{flat}} 
        \left[
        1+\left(\frac{2\Delta_{\mathrm{SOC}}}{\hbar\omega}\right)^2
        \right] \Theta(\hbar\omega - 2 \Delta_{\mathrm{SOC}}),
\end{align}
where
\begin{align}
    \label{eq:sm:opcd_flat_int}
    \sigma_{ii}^{\mathrm{flat}} =
    g \frac{e^2 k_0}{16\hbar}
    \biggl(
    \int_0^{2\pi} \frac{d\phi}{2\pi} \frac{v_{i}(\phi)^2}{v_{\rho}(\phi) v_z} \mathcal{F}_{ii}(\phi) \biggr).
\end{align}
The term inside the square bracket in Eq. (\ref{eq:sm:opcd_analytic}) represents the effect of SOC and determines the amplitude of the SOC-induced optical peak. When interband transitions are allowed only within a limited range of angles, the square bracketed term is modified and can only be obtained numerically.

\subsection{\uppercase\expandafter{\romannumeral3}-3. Calculating the flat optical conductivity}
In this subsection, we present methods to calculate the flat conductivity $\sigma_{ii}^{\mathrm{flat}}$ analytically. Let us consider the following Hamiltonian describing the
circular nodal ring with radius $r_0$ and velocity $v$ for both the radial and $k_z$ directions:
\begin{equation} \label{eq:sm:ham_flat_circle}
    H_0(\bm{k}) = \hbar v\left[\sqrt{\left(\frac{k_x}{r_0}\right)^2 + \left(\frac{k_y}{r_0}\right)^2} - 1\right]\sigma_1 + \hbar v k_z \sigma_2,
\end{equation}
and denote the corresponding flat optical conductivity as $\sigma^{\mathrm{flat, 0}}_{ii}$. Then the corresponding optical conductivity is given by $\sigma^{\mathrm{flat, 0}}_{xx} = \sigma^{\mathrm{flat, 0}}_{yy} = \sigma^{\mathrm{flat, 0}}_{zz}/2 = r_0 e^2/(32\hbar)$ \cite{PhysRevLett.119.147402}. Next, consider the optical conductivity of the elliptical ring. After the following coordinate scaling, $k_x \rightarrow \frac{r_0}{k_l} k_x$, $k_y \rightarrow \frac{r_0}{k_s} k_y$, $k_z \rightarrow \frac{v_z}{v} k_z$, the Hamiltonian $H_0(\bm{k})$ in Eq. (\ref{eq:sm:ham_flat_circle}) is transformed into
\begin{equation} \label{eq:sm:ham_flat_ellipse}
    \tilde{H}(\bm{k}) = \hbar v\left[\sqrt{\left(\frac{k_x}{k_l}\right)^2 + \left(\frac{k_y}{k_s}\right)^2} - 1\right]\sigma_1 + \hbar v_z k_z \sigma_2,
\end{equation}
where the nodal ring is no longer a circle but an ellipse with the lengths of the semi-major and semi-minor axes $k_l$ and $k_s$, respectively, and the Fermi velocities along the in-plane and out-of-plane directions are $v$ and $v_z$, respectively. 


Denoting $\tilde{\sigma}^{\mathrm{flat}}_{ii} = \tilde{\sigma}^{\mathrm{flat}}_{ii}(k_l,k_s,v/v_z)$ as the flat conductivity obtained from $\tilde{H}(\bm{k})$ in Eq. (\ref{eq:sm:ham_flat_ellipse}), we get the following scaling relation of the flat optical conductivity from the Kubo formula in Eq. (\ref{eq:sm:kubo}):
\begin{align}
    \tilde{\sigma}^{\mathrm{flat}}_{xx} &= \left(\frac{v}{v_z}\right)\left(\frac{k_s}{k_l}\frac{\sqrt{k_l k_s}}{r_0}\right)\sigma^{\mathrm{flat, 0}}_{xx}, \label{eq:sm:sigma_xx}\\
    \tilde{\sigma}^{\mathrm{flat}}_{yy} &= \left(\frac{v}{v_z}\right)\left(\frac{k_l}{k_s}\frac{\sqrt{k_l k_s}}{r_0}\right)\sigma^{\mathrm{flat, 0}}_{yy}, \\
    \tilde{\sigma}^{\mathrm{flat}}_{zz} &= \left(\frac{v_z}{v}\right)\left(\frac{\sqrt{k_l k_s}}{r_0}\right)\sigma^{\mathrm{flat, 0}}_{zz} \label{eq:sm:sigma_zz}.
\end{align}

Finally, we rotate the $k_x$ and $k_y$ axes by $\phi_0$ clockwise. The rotated Hamiltonian corresponds to the model Hamiltonian in Eq. (\ref{eq:sm:ham}). Then the flat optical conductivity matrix transforms to $\sigma^{\mathrm{flat}}(k_l, k_s, v/v_z, \phi_0) = R^{-1} \tilde{\sigma}^{\mathrm{flat}}(k_l, k_s, v/v_z) R$, where $R$ is the relevant rotation matrix. The longitudinal components are then given as
\begin{align}
\sigma^{\mathrm{flat}}_{xx}(k_l, k_s, v/v_z, \phi_0) &= \frac{1}{2} \left[\tilde{\sigma}^{\mathrm{flat}}_{xx} + \tilde{\sigma}^{\mathrm{flat}}_{yy} + \left(\tilde{\sigma}^{\mathrm{flat}}_{xx} - \tilde{\sigma}^{\mathrm{flat}}_{yy}\right)\cos2\phi_0 \right], \label{eq:sm:sigma_xx_rot} \\
\sigma^{\mathrm{flat}}_{yy}(k_l, k_s, v/v_z, \phi_0) &= \frac{1}{2} \left[\tilde{\sigma}^{\mathrm{flat}}_{xx} + \tilde{\sigma}^{\mathrm{flat}}_{yy} - \left(\tilde{\sigma}^{\mathrm{flat}}_{xx} - \tilde{\sigma}^{\mathrm{flat}}_{yy}\right)\cos2\phi_0 \right],\\
\sigma^{\mathrm{flat}}_{zz}(k_l, k_s, v/v_z, \phi_0) &= \tilde{\sigma}^{\mathrm{flat}}_{zz}. \label{eq:sm:sigma_zz_rot}
\end{align}
Plugging Eqs. (\ref{eq:sm:sigma_xx}-\ref{eq:sm:sigma_zz}) into Eqs. (\ref{eq:sm:sigma_xx_rot}-\ref{eq:sm:sigma_zz_rot}), we find the analytic form of the flat optical conductivity (Eq. (5) in the main text) given by
\begin{align}
\sigma^{\mathrm{flat}}_{xx}(k_l, k_s, v/v_z, \phi_0) &= \frac{1}{2}\frac{v}{v_z} \frac{\sqrt{k_l k_s}}{r_0}
\left[\frac{k_s}{k_l} + \frac{k_l}{k_s} + \left(\frac{k_s}{k_l} - \frac{k_l}{k_s}\right)\cos2\phi_0 \right] \sigma^{\mathrm{flat},0}_{xx} \nonumber \\
&= k_0 \frac{e^2}{16\hbar} \frac{g}{4} \frac{v}{v_z} \left[\frac{k_s}{k_l}+\frac{k_l}{k_s} + \left(\frac{k_s}{k_l}-\frac{k_l}{k_s}\right) \cos{2\phi_0} \right], \\
\sigma^{\mathrm{flat}}_{yy}(k_l, k_s, v/v_z, \phi_0) &= \frac{1}{2}\frac{v}{v_z} \frac{\sqrt{k_l k_s}}{r_0}
\left[\frac{k_s}{k_l} + \frac{k_l}{k_s} - \left(\frac{k_s}{k_l} - \frac{k_l}{k_s}\right)\cos2\phi_0 \right] \sigma^{\mathrm{flat},0}_{yy} \nonumber \\
&= k_0 \frac{e^2}{16\hbar} \frac{g}{4} \frac{v}{v_z} \left[\frac{k_s}{k_l}+\frac{k_l}{k_s} - \left(\frac{k_s}{k_l}-\frac{k_l}{k_s}\right) \cos{2\phi_0} \right], \\
\sigma^{\mathrm{flat}}_{zz}(k_l, k_s, v/v_z, \phi_0) &= \frac{v_z}{v}\frac{\sqrt{k_l k_s}}{r_0}\sigma^{\mathrm{flat},0}_{zz} \nonumber \\
&= k_0 \frac{e^2}{16\hbar} g \frac{v_z}{v},
\end{align}
where $k_0 = \sqrt{k_l k_s}$ and $g=2$ is the band degeneracy factor.

The analytic form of the flat conductivity can be directly calculated from the integral in Eq. (\ref{eq:sm:opcd_flat_int}) with the geometric factors $\mathcal{F}_{ii}(\phi)$ that describe the projection of the polarization direction on the graphene sheet and that of the graphene velocity on the current direction. The geometric factors can be obtained from the Kubo formula in Eq. (\ref{eq:sm:kubo}) and are given by
\begin{align}
    \mathcal{F}_{xx}(\phi) &= k_0 \cos^2{\left(\phi+\phi_0\right)} \left[\frac{k_s^4}{\left(\sqrt{(k_s\cos{\left(\phi+\phi_0\right)})^2+(k_l\sin{\left(\phi+\phi_0\right)})^2}\right)^{5}} \right], \\
    \mathcal{F}_{yy}(\phi) &= k_0 \sin^2{\left(\phi+\phi_0\right)} \left[\frac{k_l^4}{\left(\sqrt{(k_s\cos{\left(\phi+\phi_0\right)})^2+(k_l\sin{\left(\phi+\phi_0\right)})^2}\right)^{5}} \right], \\
    \mathcal{F}_{zz}(\phi) &=  \frac{1}{k_0}\sqrt{(k_s\cos{\left(\phi+\phi_0\right)})^2+(k_l\sin{\left(\phi+\phi_0\right)})^2}.
\end{align}
Note that for a circular nodal ring with $v_{\rho}=v_z$, $\mathcal{F}_{ii}(\phi)$ reduce to $\mathcal{F}_{xx}(\phi)=\cos^2{\left(\phi+\phi_0\right)}$, $\mathcal{F}_{yy}(\phi)=\sin^2{\left(\phi+\phi_0\right)}$, and $\mathcal{F}_{zz}(\phi)=1$, respectively.


\section {\uppercase\expandafter{\romannumeral4}. Temperature dependence of the chemical potential}

\begin{figure}[!h]
\includegraphics[width=0.5\columnwidth]{sm_figure_muvsT}
% \captionsetup{justification=raggedright,singlelinecheck=false}
\caption{Calculated chemical potential as a function of temperature.}
\label{fig:sm:chemical}
\end{figure}

The carrier density is given by
\begin{align} \label{eq:sm:density}
    n = \int_{-\infty}^{\infty} d\varepsilon \ D(\varepsilon) f(\varepsilon, \mu(T), T)
    = \int_{-\infty}^{\infty} d\varepsilon \ D(\varepsilon) f(\varepsilon, E_{\mathrm{F}}, 0)
\end{align}
where $D(\varepsilon)$ is the density of states, $f(\varepsilon, \mu, T)=1/[1+e^{(\varepsilon-\mu)/(k_\mathrm{B} T)}]$ is the Fermi-Dirac distribution function at the chemical potential $\mu$ for a given temperature $T$, and $E_{\mathrm{F}}=\mu(T=0)$ is the Fermi energy.
In calculating the carrier density, we use the model parameters in Table S3.
Since the number of carriers remains constant with respect to temperature changes, the variation of the chemical potential as a function of temperature can be given by solving Eq. (\ref{eq:sm:density}) for $\mu(T)$. Figure \ref{fig:sm:chemical} shows the numerical solution of the chemical potential as a function of temperature with the Fermi energy $E_{\mathrm{F}}=-15$ meV. As temperature increases, the chemical potential is shifted towards regions of low density of states.


\bibliographystyle{apsrev4-2}
\bibliography{sm}
\end{document}