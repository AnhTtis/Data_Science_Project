\documentclass[aps, prl, reprint, superscriptaddress, amsmath, amssymb]{revtex4-2}
\usepackage{graphicx}
\usepackage{dcolumn}
\usepackage{bm}
\usepackage{color}
\usepackage{braket}
\usepackage{minitoc} % table of contents for a specific section
\begin{document}
\title{
Optical transitions of a single nodal ring in SrAs$_3$: radially and axially resolved characterization
} 

\author{Jiwon Jeon}
 \thanks{These authors contributed equally to this work.}
\affiliation{%
 Natural Science Research Institute, University of Seoul, Seoul 02504, Korea
 }%
 \affiliation{%
 Physics Department, University of Seoul, Seoul 02504, Korea
 }%

\author{Jiho Jang}%
 \thanks{These authors contributed equally to this work.}
 \affiliation{%
 Department of Physics and Astronomy, Seoul National University, Seoul 08826, Korea
 }%

\author{Hoil Kim}%
 \affiliation{%
 Center for Artificial Low Dimensional Electronic Systems, Institute for Basic Science (IBS), Pohang 37673, Korea
 }%
  \affiliation{%
 Department of Physics, Pohang University of Science and Technology (POSTECH), Pohang 37673, Korea
 }%

\author{Taesu Park}%
 \affiliation{%
 Department of Chemistry, Pohang University of Science and Technology (POSTECH), Pohang 37673, Korea
 }%

 \author{DongWook Kim}%
 \affiliation{%
 Department of Physics, Hanyang University, Seoul 04763, Korea
 }%

 \author{Soonjae Moon}%
 \affiliation{%
 Department of Physics, Hanyang University, Seoul 04763, Korea
 }%

\author{Jun Sung Kim}%
 \affiliation{%
 Center for Artificial Low Dimensional Electronic Systems, Institute for Basic Science (IBS), Pohang 37673, Korea
 }%
  \affiliation{%
 Department of Physics, Pohang University of Science and Technology (POSTECH), Pohang 37673, Korea
 }%

\author{Ji Hoon Shim}%
 \affiliation{%
 Department of Chemistry, Pohang University of Science and Technology (POSTECH), Pohang 37673, Korea
 }%

\author{Hongki Min}%
 \email{hmin@snu.ac.kr}
 \affiliation{%
 Department of Physics and Astronomy, Seoul National University, Seoul 08826, Korea
 }%

\author{Eunjip Choi}%
\email{echoi@uos.ac.kr}
 \affiliation{%
 Physics Department, University of Seoul, Seoul 02504, Korea
 }%
 
\date{\today}

\begin{abstract}
We perform polarized optical reflection measurements on a single nodal-ring semimetal $\rm{SrAs_3}$.
For the radial and axial directions of the ring, the optical conductivity $\sigma_1(\omega)$ exhibits a flat absorption $\sigma^{\mathrm{flat}}$ over a certain frequency range.
In addition, a prominent optical peak appears at 2$\Delta_{\mathrm{SOC}}$ = 30 meV. 
For comparison, we theoretically calculate $\sigma_1(\omega)$ using an effective model Hamiltonian and first-principles calculations, which successfully reproduces the data for both directions.
The $\sigma^{\mathrm{flat}}$ establishes that the universal power-law of optical conductivity holds robustly in the nodal ring. 
Furthermore, key quantities of the nodal ring such as the band overlap energy, average ring radius, ring ellipticity, and the spin-orbit coupling (SOC) gap are determined from this comparative study.  
As temperature increases, $\sigma_1(\omega)$ shows a  substantial change, suggesting that a $T$-driven evolution occurs in the nodal ring. 
\end{abstract}

\maketitle

Topological semimetals, novel quantum materials characterized by symmetry-protected band contact or crossing, have received significant attention in recent years due to their unique fundamental properties and potential for device applications.
Nodal-line semimetals (NLSMs) are a type of topological semimetals that have two energy bands crossing over a continuous line in the $k$-space \cite{PhysRevB.84.235126, Fang_2016}.
Optically, NLSMs are predicted to exhibit universal power-law behavior as they absorb light: When the two crossing bands forming the nodal line disperse as $E(k) \sim k^{z}$ in the effective dimensions $d$, the interband transition between them leads to optical conductivity $\sigma_{1}(\omega) \sim \omega^{(d-2)/z}$ \cite{PhysRevB.87.125425}.
For $d$ = 2, $\sigma_{1}(\omega)$ becomes frequency-independent irrespective of $z$. 
The measurement of optical conductivity and investigation of the power-law have been reported in real NLSMs such as, among others, ZrSiS, ZrSiSe, BaNiS$_2$, YbMnSb$_2$, TbMn$_6$Sn$_6$, and NbAs$_2$ \cite{PhysRevLett.119.187401, Shao2020, PhysRevB.104.L201115, PhysRevB.100.125136, PhysRevB.107.045115, doi:10.1073/pnas.1809631115}.

A nodal ring is a particular type of nodal-line structure created when two energy bands cross, forming a closed nodal loop such as a circle or ellipse in the Brillouin zone (BZ).
For a circular nodal ring, theory predicts that flat optical conductivity $\sigma^{\mathrm{flat}}$ emerges as a result of the power-law, and furthermore, important features of the ring such as its size and energy-dispersion can be determined through optical conductivity \cite{PhysRevLett.119.147402, PhysRevB.96.155150}.
However, on the experimental side, optical study of such a standard nodal line, nodal ring, is still lacking due to the lack of suitable materials:
Most previous optical works on NLSMs were performed on open (unclosed) nodal lines \cite{PhysRevB.104.L201115, PhysRevB.107.045115, doi:10.1073/pnas.1809631115}, closed nodal lines but with complex structure composed of multiple rectangular loops \cite{PhysRevLett.119.187401, Shao2020} or irregular-shaped loops \cite{PhysRevB.103.125131, PhysRevB.106.245145} entangled together.
Quantitative optical data analysis for such materials presents a significant challenge. 
A real material possessing a simple-shaped nodal ring, circular or elliptical,  especially only a single ring present in the BZ, would be highly beneficial for optical studies.

One critical issue in the optical experiments of NLSMs is that, in most real materials, topologically trivial bands coexist at the Fermi energy $E_{\rm F}$ with the non-trivial bands.
The trivial bands create their own intra- and interband transitions that seriously obscure the optical signal from the non-trivial band \cite{PhysRevB.104.L201115, doi:10.1073/pnas.1809631115, C9TC03464A, PhysRevB.93.121113, Feng2017, PhysRevB.103.125131}.
To overcome this problem, a material possessing only the nodal ring without any trivial bands crossing $E_{\mathrm{F}}$ is strongly desired.

In this work, we conduct an optical measurement on $\rm{SrAs_3}$.
Recent studies have revealed that $\rm{SrAs_3}$ is a rare example featuring only a single (doubly degenerate) nodal ring in the BZ \cite{PhysRevLett.124.056402, Hosen2020, Kim2022}.
Furthermore, no trivial bands intersect $E_{\mathrm{F}}$ \cite{Hosen2020, https://doi.org/10.48550/arxiv.2211.05978}, making $\rm{SrAs_3}$ an excellent candidate for optical studies.
For an ideal nodal ring, theory predicts that $\sigma_{1}(\omega)$ becomes a constant in frequency, $\sigma^{\mathrm{flat}} = \frac{e^2}{16\hbar}k_0$ (per degeneracy) \cite{PhysRevLett.119.147402}, which is determined by the radius of the nodal ring, $k_0$.
In $\rm{SrAs_3}$, the nodal ring has a slightly elliptical shape, which may cause $\sigma_{1}(\omega)$ to differ from the predicted constant, where the actual $\sigma_{1}(\omega)$ is an interesting open question. 
Here we measure $\sigma_{1}(\omega)$ of $\rm{SrAs_3}$ over a wide range of frequencies.
We further perform theoretical calculations of $\sigma_{1}(\omega)$ and compare the result with the data. This comparative study firmly establishes the power-law in a nodal ring. 

A single crystal $\rm{SrAs_3}$ was synthesized and thoroughly characterized \cite{Kim2022}.
An axis-resolved optical measurement was performed using polarized light.
The experimental details are described in the Supplemental Material (SM) Sec. I \cite{SM}.

\nocite{Homes:93}
\nocite{togo2015first}
\nocite{PhysRevB.54.11169}
\nocite{doi:10.1063/1.5143061}
\nocite{PhysRevLett.102.226401}
\nocite{PhysRevLett.118.176402}
\nocite{mahan2000many}

%\cite{Kim2022,Homes:93,doi:10.1063/1.1979470,togo2015first,PhysRevB.54.11169,doi:10.1063/1.5143061,PhysRevLett.102.226401,PhysRevB.95.045136,PhysRevLett.118.176402,mahan2000many,PhysRevLett.119.147402,doi:10.1073/pnas.1809631115}

\begin{figure}
\includegraphics[width=1\columnwidth]{fig1}
\caption{\label{fig:fig1}
Schematic drawing of (a) crystal structure and (b) band structure of $\rm{SrAs_3}$.
(c) A simplified view of (b) where $2\varepsilon_0$ and $E_{\mathrm{F}}$ are the band overlap energy and Fermi energy, respectively.
(d) The nodal ring is elliptically elongated where $k_{l}$, $k_{s}$, and $\phi_{0}$ are the lengths of the semi-major and semi-minor axes and the rotation angle, respectively.
The blue and red pockets represent the hole and electron carriers, respectively.
The yellow curves in (b-d) highlight the nodal ring.
The SOC gap is omitted in the figures for clarity.
In the optical measurement, incident light is polarized as specified in (a) and (d).}
\end{figure}

Figure 1 presents the crystal structure and electronic band structure of $\rm{SrAs_3}$.
The lattice of $\rm{SrAs_3}$, which belongs to the $C2/m$ space group, consists of As-planes separated by Sr atoms that are stacked along the $c$-axis.
Figure 1(b) displays the conduction and valence bands, which are both mainly from As $4p$ orbitals, that intersect each other leading to the formation of a nodal ring.
Figure 1(c), a simplified view of Fig. 1(b), shows that the two bands overlap by $2\varepsilon_0$. They disperse asymmetrically along the $k_x$ and $k_y$ directions, causing the nodal ring to tilt.
Figure 1(d) displays the nodal ring in $k$-space. The ring is elliptically elongated and rotated as shown by ARPES and DFT study \cite{PhysRevLett.124.056402, PhysRevB.95.045136}.  
%The parameters $k_l$, $k_s$, and $\phi_0$ denote the lengths of the semi-major and semi-minor axes of the nodal ring, and the rotation angle, respectively.
In $\rm{SrAs_3}$, the SOC creates an energy gap along the nodal ring.
This gap is omitted intentionally for clarity in Fig. 1.
The Fermi energy and carrier density of our sample were precisely determined from quantum oscillation measurements \cite{Kim2022}.
The $E_{\mathrm{F}}$, indicated in Fig. 1(c), shows that the hole is dominant over the electron, where the actual carrier densities are  $n_h$ = 4.2$\times 10^{17}$ cm$^{-3}$ for the hole and $n_e$ = 6.9$\times 10^{15}$ cm$^{-3}$ for the electron at $T$ = 5 K, respectively, which is schematically represented in the sizes of the blue (hole) and red (electron) pockets as well in Fig. 1(d).

\begin{figure}
\includegraphics[width=1\columnwidth]{fig2} 
\caption{\label{fig:fig2}
Reflectivity of $\rm{SrAs_3}$ measured for two polarization directions (a) $E$ $\parallel$ $k_x$ and (b) $E$ $\parallel$ $k_z$.
Inset shows the wide-range reflectivity up to 4.8 eV at $T$ = 300 K.}
\end{figure}

The optical reflectivity was measured with incident light that is polarized along $E$ $\parallel$ $a$ and $E$ $\parallel$ $b$.
In the $k$-space, they correspond to $E$ $\parallel$ $k_x$ and $E$ $\parallel$ $k_z$, which will probe the radial and axial directions of the nodal ring, respectively. 
Figure 2 shows that, for the $E$ $\parallel$ $k_x$ polarization, reflectivity rises for $\hbar\omega$ $<$ 50 meV ($T$ = 300 K) due to a Drude response. 
As $T$ decreases, the Drude peak becomes sharper.
In addition, a new hump develops at 30 meV for $T$ $<$ 140 K.
For $E$ $\parallel$ $k_z$, the Drude reflectivity is broader and its level is higher than $E$ $\parallel$ $k_x$.
The 30 meV-hump appears in $E$ $\parallel$ $k_z$ as well.   
Notably, a shoulder-like feature is present at $\sim$ 90 meV which is absent for $E$ $\parallel$ $k_x$.
As $T$ decreases, the shoulder feature shifts to higher energy reaching $\sim$ 140 meV at $T$ = 5 K.

\begin{figure}
\includegraphics[width=1\columnwidth]{fig3}
\caption{\label{fig:fig3}
Optical conductivity obtained from (a) reflectivity measurement at $T$ = 5 K and (b) theoretical calculation for interband transitions at $T=0$ K.
The A, B, and C represent the SOC-induced optical peak, the flat conductivity region, and the interband transition above the band overlap energy, respectively.
The $\sigma^{\rm{flat}}_{xx}$ and $\sigma^{\rm{flat}}_{zz}$ marked by red dashed lines in panel (b) show  the flat conductivity levels.
(c) Schematic band structure near the Fermi energy $E_{\mathrm{F}}$ along the radial direction, where SOC opens a gap of 2$\Delta_{\mathrm{SOC}}$ along the nodal ring. The black solid line indicates the band overlap energy 2$\varepsilon_0$.
The arrows A, B, and C represent the corresponding interband transitions in (a). 
(d) Schematic diagram illustrating the nodal ring as a collection of gapped anisotropic graphene sheets.
}
\end{figure}

To analyze the optical features quantitatively, we fit $R(\omega)$ using the Kramers-Kronig (KK) constrained variational dielectric functions of RefFit \cite{doi:10.1063/1.1979470} and obtain the optical conductivity $\sigma_{1}(\omega)$.
Figure 3(a) shows $\sigma_{1}(\omega)$ for the lowest measured temperature $T = 5$ K.
We present the full $T$-dependent $\sigma_{1}(\omega)$ data in the SM Fig. S1. 
For $E \parallel k_x$, $\sigma_{1}(\omega)$ displays an absorption peak (A) at 30 meV and a flat conductivity (B) over the 70 - 129 meV range.
At higher energies, $\sigma_{1}(\omega)$ gradually increases (C) with energy.
For $E \parallel k_z$, Peak-A becomes more pronounced and the flat optical conductivity $\sigma_{zz}^{\mathrm{flat}} \approx 239 \ \mathrm{cm^{-1} \thinspace \mathrm{\Omega^{-1}}}$ is significantly enhanced over $\sigma_{xx}^{\mathrm{flat}} \approx 83 \ \mathrm{cm^{-1} \thinspace \mathrm{\Omega^{-1}}}$ for $E \parallel k_x$. 
In the region C, $\sigma_{1}(\omega)$ drops markedly for $E \parallel k_z$ around $\hbar\omega = 129$ meV in contrast to the increase for $E \parallel k_x$.
Along with the interband transitions A, B, and C, $\sigma_1(\omega)$ exhibits an intraband transition (Drude) peak at low energies. 
The two sharp peaks between the Drude peak and Peak-A are optical phonons of $A_u$ mode (for $E \parallel k_z$) and $B_u$ mode (for $E \parallel k_x$), respectively (see SM Sec. II-2 \cite{SM}).
For quantitative analysis, we decompose $\sigma_{1}(\omega)$ into three components: Drude, phonon, and interband conductivities as $\sigma_1(\omega) = \sigma^{\mathrm{D}}(\omega) + \sigma^{\mathrm{Ph}}(\omega) + \sigma^{\mathrm{IB}}(\omega)$.
Each component is obtained separately by fitting the data (see SM Sec. II for details \cite{SM}). In the following discussion, we will focus on the Drude peak and interband transitions.

The Drude peak is described by $\sigma^{\mathrm{D}}(\omega) = \frac{\omega^2_{p}}{4\pi}\frac{\gamma}{(\gamma^2+\omega^2)}$, where $\omega_{p}$ and $\gamma$ = 1/$\tau$ are the plasma frequency and the carrier scattering rate, respectively.
$\sigma^{\mathrm{D}}(\omega)$ arises overwhelmingly from the hole carrier
due to its dominance over the electron.
Using the fitting results for $\omega_{p,x}^{2}$ ($E \parallel k_x$) and $\omega_{p,z}^{2}$ ($E \parallel k_z$), summarized in Table S1, we obtain $\omega_{p,z}^{2}$/$\omega_{p,x}^{2}$ = 9.5 which, from the definition of $\omega_{p}^{2}$ = $\frac{4\pi ne^{2}}{m^*}$ ($n$ = hole density,  $m^{*}$ = effective mass), is equivalent to  $m^*_{x}$/$m^*_{z}$.
It shows that the effective mass of the hole moving along the Sr-chain ($b$-axis) is substantially lighter than that along the Sr-zigzag direction ($a$-axis).
A similar result was reported in the chain- and zigzag-directions of black phosphorus \cite{19833544}.
We can calculate $m^*$ by inserting the hole density $n$ = 4.2$\times 10^{17}$ cm$^{-3}$ into $\omega_{p,z}^{2}$, which yields $m^*_z$ = 0.016$m_0$.
We further calculate the carrier mobility $\mu = \frac{e\tau}{{m^*}}$ using $\tau$ = 1/$\gamma$ and $m^{*}$ for $E$ $\parallel$ $k_x$ and $E$ $\parallel$ $k_z$.
This leads to $\mu_{x}$ = 3.9$\times 10^{3}$ $\rm{cm}^2 \thinspace V^{-1} \thinspace s^{-1}$ and $\mu_{z}$ = 2.3$\times 10^{4}$ $\rm{cm}^2 \thinspace V^{-1} \thinspace s^{-1}$, respectively, which are similar to previous transport results \cite{PhysRevB.50.5180, PhysRevB.32.1183}.  
The high $\mu$'s support the suppressed back-scattering of the topological bands of $\rm{SrAs_3}$.

Next, we investigate the interband conductivity $\sigma^{\mathrm{IB}}(\omega)$.
To understand the behaviors of A, B, and C both qualitatively and quantitatively, we perform a theoretical calculation using a four-band model Hamiltonian that takes into account the two crossing bands of the nodal ring including their spin degrees of freedom:
\begin{align} \label{eq:ham_full}
    H(\bm{k}) = f_0(\bm{k})\sigma_0 s_0 + f_1(\bm{k})\sigma_1 s_0
    + f_2(\bm{k})\sigma_2 s_0 + \Delta_{\rm SOC}\sigma_3 s_3,
\end{align}
where $f_0(\bm{k}) = a_0 + a_x k_x^2 + a_{xy} k_x k_y + a_y k_y^2 + a_z k_z^2$, $f_1(\bm{k}) = b_0 + b_x k_x^2 + b_{xy} k_x k_y + b_y k_y^2 + b_z k_z^2$, $f_2(\bm{k}) = \hbar v_z k_z$, and $\bm{k}$ is the wave vector measured from the Y point in the BZ, and $\bm{\sigma}$ and $\bm{s}$ are Pauli matrices acting on the pseudospin and spin degrees of freedom, respectively.
$\Delta_{\rm SOC}$ is the strength of the spin-orbit coupling, which, as a leading approximation, we take as constant. Here the coefficients $a_0$, $a_i$, $b_i$ ($i=x$, $xy$, $y$, $z$), and $v_z$ are obtained from first-principles calculations. For the band overlap energy 2$\varepsilon_0 = 2b_0$ and the spin-orbit coupling $\Delta_{\mathrm{SOC}}$, which are difficult to estimate correctly from first-principles calculations, we determine them by comparing with the optical data (see SM Sec. III \cite{SM}). Then we calculate the interband conductivity $\sigma_1(\omega)$ using the Kubo formula. The full numerical results of $\sigma_1(\omega)$ are presented in Fig. 3(b).

To gain some insight into the optical features, we adopt a simple picture that the nodal ring is considered as a collection of gapped anisotropic graphene sheets, as shown in Fig. 3(d). In this picture, the Hamiltonian in Eq. (1) at low frequencies can be transformed into \cite{SM}
\begin{align} \label{eq:gr_ham}
    H(\bm{k}) = \Delta_{\mathrm{tilt}}(\bm{k})\sigma_0 + \hbar v_{\rho}(\phi) \delta k_{\rho} \sigma_1 + \hbar v_z k_z \sigma_2 + \Delta_{\mathrm{SOC}} \sigma_3. \nonumber \\
\end{align}
Here $(k_{\rho}, \phi)$ are the polar coordinates of $(k_x, k_y)$, $\delta k_{\rho}$ and $\delta k_z$ are wave vectors measured from the nodal ring along the radial and $k_z$ directions, respectively, and $v_{\rho}(\phi)$ and $v_z$ are the corresponding Fermi velocities along the two directions. Note that $v_{\rho}(\phi)$ has an angular dependence, while $v_z$ is constant. $\Delta_{\mathrm{tilt}}(\bm k)$ is the energy tilt term and $\Delta_{\mathrm{SOC}}$ acts as a mass term opening an energy gap.

Then the optical conductivity of the nodal ring can be obtained by summing up all the contributions from each gapped anisotropic graphene sheet along the nodal ring as
\begin{equation} \label{eq:opcd_as_sum}
    \sigma_{ii} (\omega) = k_0 \int_0^{2\pi} \frac{d\phi}{2\pi} \ \sigma^{\mathrm{ gr}} (\omega, \phi) \mathcal{F}_{ii} (\phi),
\end{equation}
where $i = x$, $y$, $z$ are spatial directions, $\sigma_{ii}^{\mathrm{gr}}(\omega, \phi)$ is the optical conductivity of the gapped anisotropic graphene, $k_0$ is the characteristic scale of the nodal ring size which will be defined later, and $\mathcal{F}_{ii}(\phi)$ are geometric factors related to the shape of the nodal ring (see SM Sec. IV \cite{SM}).

The optical conductivity for the Hamiltonian in Eq. (\ref{eq:gr_ham}) at zero temperature in the clean limit is then given by \cite{SM}
\begin{align} \label{eq:opcd}
    \sigma_{ii}(\omega) &\approx \sigma^{\mathrm{flat}}_{ii} \left[1+\left(\frac{2\Delta_{\mathrm{SOC}}}{\hbar\omega}\right)^2\right] \Theta(\hbar\omega - 2 \Delta_{\mathrm{SOC}}).
\end{align}
Here, $\sigma_{ii}^{\mathrm{flat}}$ is the flat optical conductivity which will be discussed in Eq. (\ref{eq:opcd_flat_analytic}), the term inside the square bracket represents the effect of SOC, and $\Theta(x)$ is the Heaviside step function. The result in Eq. (\ref{eq:opcd}) shows that, at the onset of interband transitions, the SOC-induced optical peak appears at $\hbar\omega = 2 \Delta_{\mathrm{SOC}}$ in both $\sigma_{xx}$ and $\sigma_{zz}$, and therefore the peak position defines the magnitude of the SOC gap.

In obtaining Eq. (\ref{eq:opcd}), we use the fact that the Fermi energy of $\mathrm{SrAs_3}$ lies within the SOC gap for at least a certain range of angle, as shown in Fig. 3(c), giving a peak corresponding to the gap size. Note that Eq. (\ref{eq:opcd}) is an approximate form neglecting the effect of the energy tilt term. For details, see SM Sec. IV \cite{SM}.


Based on the experimental data in Fig. 3(a), we estimate 2$\Delta_{\mathrm{SOC}} \cong 30$ meV. The SOC gap is one of the key factors that determines the band structure of NLSMs but has seldom been measured precisely due to its small magnitude. We emphasize that the high energy resolution of IR spectroscopy, in combination with the successful theoretical analysis, allows for the reliable determination of $\Delta_{\mathrm{SOC}}$.

As frequency increases, the optical conductivity approaches $\sigma^{\mathrm{flat}}_{ii}$ in Eq. (\ref{eq:opcd}). This can be obtained analytically as \cite{SM}
\begin{subequations} \label{eq:opcd_flat_analytic}
    \begin{align} 
        \sigma_{xx}^{\mathrm{flat}} &= k_0 \frac{e^2}{16\hbar} \frac{g}{4} \frac{v}{v_z} \left[\frac{k_s}{k_l}+\frac{k_l}{k_s} + \left(\frac{k_s}{k_l}-\frac{k_l}{k_s}\right) \cos{\left(2\phi_0\right)} \right], \\
        \sigma_{zz}^{\mathrm{flat}} &= k_0 \frac{e^2}{16\hbar} g \frac{v_z}{v},
    \end{align}
\end{subequations}
where $k_l$ and $k_s$ are the lengths of the semi-major and semi-minor axes of the nodal ring, $k_0 = \sqrt{k_l k_s}$, $v = \hbar k_0 / m_0$, $\phi_0$ is the angle between the semi-major axis of the nodal ring and the $k_x$ axis, $g=2$ is the band degeneracy factor. From $k_0=0.068 \ \mathrm{\AA^{-1}}$, $k_l/k_s=1.16$, $v=2.88 \times 10^5 \ \mathrm{m/s}$, $v_z=3.48 \times 10^5 \ \mathrm{m/s}$, and $\phi_0 = 23.5^{\circ}$ \cite{SM}, $\sigma_{xx}^{\mathrm{flat}}$ and $\sigma_{zz}^{\mathrm{flat}}$ are estimated to be 78 $\mathrm{cm}^{-1}\mathrm{\Omega}^{-1}$ and 247 $\mathrm{cm}^{-1}\mathrm{\Omega}^{-1}$, respectively, showing good agreement with the experimental result.

It should be noted that for a given nodal ring size ($k_0 = \sqrt{k_l k_s}$), $\sigma_{xx}^{\mathrm{flat}}$ depends on how much the nodal ring is elongated, while $\sigma_{zz}^{\mathrm{flat}}$ does not depend on the nodal ring shape. For a circular nodal ring ($k_l=k_s$) with equal velocity $v = v_z$, Eq. (\ref{eq:opcd_flat_analytic}) reduces to $\sigma_{xx}^{\mathrm{flat}} = \sigma_{zz}^{\mathrm{flat}}/2 =  ge^2 k_0/(32\hbar)$ consistent with the previous result \cite{PhysRevLett.119.147402}.

Having investigated the SOC-induced peak (A) and the flat conductivity (B), we move to the region C. For $\hbar \omega$ $>$ $2\varepsilon_0 \approx$ 129 meV, the flat conductivity is no longer maintained and $\sigma_{xx}$ increases gradually while $\sigma_{zz}$ shows a sudden drop, indicating that the picture we employed in Eq. (2) does not hold any more at high frequencies \cite{PhysRevLett.119.147402, PhysRevB.96.155150}. At $\hbar \omega = 2\varepsilon_0$, the joint density of states for interband transitions shows an abrupt change. Above this frequency, the optical conductivity exhibits a non-trivial frequency-dependence depending on the direction due to the difference in the velocity matrix elements along the $k_x$ and $k_z$ directions.
Remarkably, our analytic estimation for the flat conductivity using Eq. (\ref{eq:opcd_flat_analytic}) and the full numerical calculation using the Kubo formula over a wide frequency range are in excellent agreement with the experiment not only qualitatively but also quantitatively.

\begin{figure}[!b]
\includegraphics[width=1.0\columnwidth]{fig4} 
\caption{\label{fig:fig4}
Interband transition optical conductivity $\sigma^{\mathrm{IB}}(\omega)$ for $E$ $\parallel$ $k_z$.
(a) Experimental results taken over 5 K $\leq$ $T$ $\leq$ 300 K range.
(b) Theoretical results calculated for the same temperatures.
In regions B and C, apparent discrepancies exist between (a) and (b) as emphasized by the arrow and the dashed curve.
}
\end{figure}

Having discussed $\sigma_{1}(\omega)$ at $T$ = 5 K, we now move to the $T$-dependent behavior of $\sigma_{1}(\omega)$.
For intraband $\sigma^{\mathrm{D}}(\omega)$, the Drude peak grows and broadens with increasing $T$ (see Fig. S3). 
The former is attributed to the increase in carrier density due to the thermal population, while the latter is caused by the increase in carrier scattering.

The interband optical conductivity, $\sigma^{\mathrm{IB}}(\omega)$ for $E$ $\parallel k_z$ is presented in Fig. 4(a).
As $T$ increases, Peak-A weakens and the flat conductivity $\sigma_{zz}^{\mathrm{flat}}$ decreases.
In addition, the frequency at which $\sigma_{zz}$ drops shifts to a lower energy.
For comparison, we numerically calculate $\sigma^{\mathrm{IB}}(\omega)$ at finite $T$ by taking the thermal effect into account, i.e., by incorporating the Fermi-Dirac distribution and $T$-driven chemical potential $\mu(T)$ shift in the Kubo formula (see SM Sec. V \cite{SM}).
The calculated $\sigma^{\mathrm{IB}}(\omega)$ in Fig. 4(b) shows that Peak-A is strongly suppressed with $T$ similar to the experimental data.
This suppression occurs because thermally excited electrons and holes occupy the CB-bottom and VB-top, respectively, and Pauli blocking forbids the transition across the SOC gap. 
In regions B and C, the $T$-dependent change of the calculated $\sigma^{\mathrm{IB}}(\omega)$ becomes smaller due to reduced thermal population at higher energies.
However, unlike this prediction, the experimental $\sigma^{\mathrm{IB}}(\omega)$ exhibits substantial changes in B and C: $\sigma_{zz}^{\mathrm{flat}}$ is reduced more than expected by the calculation and 2$\varepsilon_0$ redshifts by $\sim$ 40 meV. 
The latter is in sharp contrast to the theory, where 2$\varepsilon_0$ is fixed with respect to $T$.
Such discrepancies between the data and theory suggest that some mechanism other than the thermal effect is at work.
For example, if we assume that CB and VB move away from each other with $T$, then
2$\varepsilon_0$ will redshift, as inferred from Fig. 1(c). Simultaneously, the nodal ring radius $k_0$ decreases, reducing $\sigma_{zz}^{\mathrm{flat}}$ according to Eq. 5(b).
The CB- and VB-movement is thus one possible scenario for the peculiar $T$-dependence of $\sigma^{\mathrm{IB}}(\omega)$. 
However, at this point, we cannot provide additional supporting evidence or underlying cause for the bands' movements.
Nevertheless, it is clear that the thermal effect alone cannot account for the $T$-dependent changes in regions B and C.
Further studies are necessary to unveil the origin of this intriguing $T$-driven $\sigma^{\mathrm{IB}}(\omega)$ changes.

To conclude, we performed the first measurement of optical transitions on a nodal ring semimetal by adopting $\rm{SrAs_3}$, a unique material that possesses only a single nodal ring in the BZ without any interrupting trivial bands. 
For comparison, we theoretically calculated the optical conductivity $\sigma_{1}(\omega)$, which successfully reproduced the characteristic optical transitions of the nodal ring such as the SOC-induced Peak-A and the flat conductivity $\sigma^{\mathrm{flat}}$.
Our study demonstrates that $\sigma_{1}(\omega)$ originates exclusively from the single nodal ring, establishing $\rm{SrAs_3}$ as an ideal platform for investigating nodal rings in topological materials.

\begin{acknowledgments}
This work was supported by the NRF grant funded by the Korea government (NRF-2021R1A2C1009073).
J.J. was supported by Basic Science Research Program of Korea (NRF-2018R1A6A1A06024977).
J.J. and H.M. were supported by the NRF grant funded by the Korea government (NRF-2018R1A2B6007837 and NRF-2023R1A2C1005996), the Creative-Pioneering Researchers Program through Seoul National University (SNU), and the Center for Theoretical Physics.
H. K. and J. S. K. were supported by the Institute for Basic Science (IBS) through the Center for Artificial Low Dimensional Electronic Systems (no. IBS-R014-D1), and by the NRF through the Basic Science Research Program (2022R1A2C3009731), and the Max Planck POSTECH/Korea Research Initiative (2022M3H4A1A04074153).
T.P. and J.S. were supported by the NRF grant funded by the Korea government (NRF-2021R1A2C2010972).
The work at HYU was supported by the NRF grant funded by the Korean government (MSIT) (2022R1F1A1072865 and RS-2022-00143178) and BrainLink program funded by the Ministry of Science and ICT through the NRF (2022H1D3A3A01077468). 
\end{acknowledgments}

\bibliographystyle{apsrev4-2}
\bibliography{main}
\end{document}