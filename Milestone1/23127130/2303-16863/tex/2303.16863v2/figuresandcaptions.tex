\documentclass{article}
\usepackage{graphicx}
\usepackage{float}
\usepackage[subsection]{placeins}
\begin{document}
\makeatletter
\setlength{\@fptop}{0pt}
\makeatother

\title{\vspace{-2.0cm}Figures and text}
\maketitle

\begin{figure}[htp!]
    \centering
    \includegraphics[width=16cm]{ross458c_89slabnest_profile1_SAMPLING_profile.png}
    \caption{Pressure-temperature profile of winning model. The dashed lines are condensation curves. The right part of the figure is the placement of the cloud in the atmosphere, the grey parts are the 1$\sigma$ errors.}
    \label{profile}
\end{figure}


\FloatBarrier
In Figure~\ref{profile} we see the thermal profile for the winning model. This is the black line. The pink is the 1 and 2$\sigma$ errors. The condensation curve for soot/graphite is missing, because it doesn't have a normal condensation curve, the way that the other condensates do. It is stable over a range of temperatures. The condensation curves do not mean that the cloud will condense, it means that it can. The right part of the figure is where the cloud is placed, it is placed right in the bottom of the atmosphere. The clouds that could condense at that location are the MnS and MgSiO3 clouds. KCl and Na2S condense at cooler temperatures. 
\FloatBarrier
\clearpage 

\begin{figure}[t!]
    \centering
    \includegraphics[width=16cm]{ross458c_89slabnest_profile1_SAMPLING_SPAG_SPECGREENTEST.png}
    \caption{Spectrum of Ross458C in the 1-2.4 $\mu$m with retrieved spetrum in green.}
    \label{spectrum}
\end{figure}


In Figure~\ref{spectrum} we see the spectrum of Ross-458C. The areas where the max likehood don't match well with the spectrum are telluric areas. Manually removing those wavelengths affected by the telluric absorption, and then rerunning the retrieval, does not change the C/O ratio. 
\\
(The water absorption just before 1.5 $\mu$m is not as low as would be expected, confirming the lack of water in the spectrum, which is seen in the [O/H] ratio. [C/H] matches well with the metallicity, but the [O/H] ratio is low, which means that the carbon abundance is what is expected, but the low oxygen abundance is driving the high C/O ratio.) 
\FloatBarrier
\clearpage 

\begin{figure}[t!]
    \centering
    \includegraphics[width=16cm]{ross458c_89slabnest_profile1_SAMPLING_POST_corner.png}
    \caption{Corner plot of winning model.}
    \label{cornerplot}
\end{figure}

In Figure~\ref{cornerplot} we see the corner plot. This shows the probability distribution of the different parameters and how they are correlated. Some, such as the CO and CO2 are not well constrained. CO2 shows a bimodal distribution. H2O and CH4 are well constrained.  
\FloatBarrier
\clearpage


\begin{figure}[t!]
    \centering
    \includegraphics[width=16cm]{ross458c_89slabnest_profile1_SAMPLING_contribution2.png}
    \caption{Contribution function of winning model}
    \label{contribution function2}
\end{figure}
In Figure~\ref{contribution function2} we see the contribution function. This shows where in the atmosphere the flux is coming from. The darker the area, the more flux from that area. The blue line is the gas opacity where  $\tau_{gas}$ = 1.0. The cloud is the purple line (at $\tau_{cloud}$ = 1.0), and is placed in the bottom of the atmosphere, below the photosphere. It overlaps with the gas opacity at the J-band, at around 1.3 $\mu$m. 

\FloatBarrier
\clearpage



\begin{figure}[h!]
    \centering
    \includegraphics[width=16cm]{ross458c_89slabnest_profile1_SAMPLING_abundances.png}
    \caption{abundance plot}
    \label{abundance plot}
\end{figure}
In Figure~\ref{abundance plot} we see the abundance plot. It shows the retrieved gas mixing ratios as a thick full line, along with equilibrium predictions, which are the dashed lines. The gas fractions are vertically constant. The pink shade is the 16th and 84th percentile. The H2O and CH4 matches up pretty well with the dashed line.
The dashed lines are from thermochemical grid models.

\FloatBarrier
\clearpage

\begin{figure}[t!]
    \centering
    \includegraphics[width=16cm]{ross458c_89slabnest_profile1_SAMPLING_SPAG_SPECNEW.png}
    \caption{Spectrum with Saumon \& Marley models overplotted.}
    \label{spectrumnew}
\end{figure}
In Figure~\ref{spectrumnew} we see the spectrum with Saumon and Marley models overplotted. There are four different models overplotted. $f_{sed}$ is the efficiency of sedimention of condensate particles. Larger values mean faster particle growth and faster grain sizes. Smaller values mean thicker clouds, as the large particles will have rained out of the atmosphere (REF Burningham 2011). $K_{zz}$ is the eddy diffusivity. $f_{sed}$ is related to $K_{zz}$. The models are all "normalised" to the J-band, meaning that they all overlap there. 
\FloatBarrier
\clearpage

\begin{figure}[t!]
    \centering
    \includegraphics[width=16cm]{ross458c_89slabnest_profile1_SAMPLING_profilemodels.png}
    \caption{Spectrum with Saumon \& Marley models overplotted.}
    \label{profilemodels}
\end{figure}

\FloatBarrier
\clearpage

In figure \ref{profilemodels} we see the thermal profile with Sonora models overplotted. The Sonora models are cloudless grid models with solar metallicity and C/O ratio. The models best match the retrieved profile at lower pressures.  

\FloatBarrier
\clearpage

Model selection. For nested sampling, the top ranked model is a cloudy model with a slab cloud and power law opacity. The evidence difference between the first and second ranked model, which is the same model, but with different treatment of the line wings, is not strong, whereas the strength of the difference to the other models is substantial.
\\ 
\linebreak
The results from the EMCEE runs rank the same two models in top, with a very strong preference for the top ranked model compared to third ranked and below. 

\FloatBarrier
\clearpage

\begin{figure}[t!]
    \centering
    \includegraphics[width=16cm]{spectrumwithmodels2105.png}
    \caption{Spectrum with Saumon \& Marley models overplotted.}
    \label{spectrummodels}
\end{figure}


\end{document}