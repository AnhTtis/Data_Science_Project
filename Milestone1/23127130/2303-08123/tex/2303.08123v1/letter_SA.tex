\documentclass[12pt,a4paper]{letter}
\usepackage{amsmath}
\usepackage{txfonts}
\usepackage[x11names]{xcolor}

\usepackage[margin=2.5cm, vmargin=2.6cm]{geometry}


\begin{document}

\name{Simon D. Angus} 
\address{Simon D. Angus\\
 Department of Economics and SoDa Laboratories,\\ Monash Business School\\ Monash University\\ Wellington Rd, Clayton, 3800, Australia}
%\name{Paweł Gora} 
%\address{Paweł Gora\\
%         Faculty of Mathematics, Informatics and Mechanics,\\
%         University of Warsaw,\\
%         Banacha 2, 02-097 Warsaw,\\ Poland}



\signature{Simon D. Angus} 

\def\today{17.03.2022}


\begin{letter}{Prof. Jason Papin\\
Editor-in-Chief\\
PLOS Computational Biology}
  
\opening{Dear Professor Jason Papin,}

Please find with this submission to \emph{PLOS Computational Biology} our manuscript, ``{\em Identifying Promising Candidate Radiotherapy Protocols via GPU-GA in-silico}'' prepared by Wojciech Ozimek, Rafa\l{} Bana{\'s}, Pawe\l{} Gora, Simon D. Angus and Monika J. Piotrowska.

Our paper introduces and demonstrates a powerful new coupled numerical simulation and search methodology, `GPU-GA' (Graphical Processor Unit -- Genetic Algorithm), which effectively brings \emph{high throughput} thinking to the previously overwhelming problem of finding quasi-optimal tumour irradiation protocols amidst a vast solution space.

By leveraging an existing, high fidelity numerical simulation model of EMT6/Ro spheroids, we demonstrate significant gains in tumour suppression are found with GPU-GA over the state-of-the-art search approach, and, by virtue of the highly scaled setting, extend these results to coupled variation in fractional dose and timing.

It is our view that the GPU-GA methodology opens up a new frontier in tumour treatment discovery \emph{in silico} and should lead to major advances in the traditionally slow-moving clinical setting where experimentation is highly costly, and so, pre-clinical solution screening of the kind we introduce is all the more valuable.

We declare that none of the submitted materials have been published or is under consideration for publication elsewhere. We also declare no competing interests exist.

We commend our paper to your review at {\em PLOS Computational Biology}.

Many thanks for your consideration. 
 
\closing{Yours faithfully, \\ On behalf of the authors,}
\end{letter}

\end{document}


Cover letter (PDF or Word) which includes (TO DO list):
\begin{itemize}
    \item Article title {\bf [done]}
    \item Describe briefly why your work is appropriate for Science Advances: What is new and how does it advance the field
Any information needed to ensure a fair review process, including related manuscripts submitted to other journals
    \item A statement confirming that none of the material has been published or is under consideration for publication elsewhere {\bf [done]}
    \item A statement noting potential conflicts of interest {\bf [done]}
    \item For investigations on humans, a statement indicating that informed consent was obtained after the nature and possible consequences of the studies were explained {\bf [not relevant]}
    \item For authors using laboratory animals, a statement that the animals' care was in accordance with institutional guidelines {\bf [not relevant]}
    \item Specification of where all data underlying the study are available, or will be deposited, and whether there are any restrictions on data availability such as a materials transfer agreement (MTA)
    \item Information on any reference material or additional data files uploaded to the Auxiliary files section (see below)
\end{itemize}



% !TeX spellcheck = en_GB


%%% Local Variables: 
%%% mode: latex
%%% TeX-master: t
%%% End: 
