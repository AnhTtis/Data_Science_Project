\documentclass[10pt,letterpaper]{article}
\usepackage[top=0.85in,left=0.75in,footskip=0.75in]{geometry}
%\usepackage[top=0.85in,left=1.25in,footskip=0.75in]{geometry}

% amsmath and amssymb packages, useful for mathematical formulas and symbols
\usepackage{amsmath,amssymb,multirow}

% Use adjustwidth environment to exceed column width (see example table in text)
\usepackage{changepage}

% Use Unicode characters when possible
\usepackage[utf8x]{inputenc}

% textcomp package and marvosym package for additional characters
\usepackage{textcomp,marvosym}

% cite package, to clean up citations in the main text. Do not remove.
\usepackage{cite}

% Use nameref to cite supporting information files (see Supporting Information section for more info)
\usepackage{nameref,hyperref}

% line numbers
\usepackage[right]{lineno}

% ligatures disabled
\usepackage{microtype}
\DisableLigatures[f]{encoding = *, family = * }

% color can be used to apply background shading to table cells only
\usepackage[table]{xcolor}

% array package and thick rules for tables
\usepackage{array}

% create "+" rule type for thick vertical lines
\newcolumntype{+}{!{\vrule width 2pt}}

% create \thickcline for thick horizontal lines of variable length
\newlength\savedwidth
\newcommand\thickcline[1]{%
  \noalign{\global\savedwidth\arrayrulewidth\global\arrayrulewidth 2pt}%
  \cline{#1}%
  \noalign{\vskip\arrayrulewidth}%
  \noalign{\global\arrayrulewidth\savedwidth}%
}

% \thickhline command for thick horizontal lines that span the table
\newcommand\thickhline{\noalign{\global\savedwidth\arrayrulewidth\global\arrayrulewidth 2pt}%
\hline
\noalign{\global\arrayrulewidth\savedwidth}}


% Remove comment for double spacing
%\usepackage{setspace} 
%\doublespacing

% Text layout
\raggedright
\setlength{\parindent}{0.5cm}
\textwidth 6.9in %5.25
\textheight 8.75in

% Bold the 'Figure #' in the caption and separate it from the title/caption with a period
% Captions will be left justified
\usepackage[aboveskip=1pt,labelfont=bf,labelsep=period,justification=raggedright,singlelinecheck=off]{caption}
\renewcommand{\figurename}{Fig}

% Use the PLoS provided BiBTeX style
\bibliographystyle{plos2015}

% Remove brackets from numbering in List of References
\makeatletter
\renewcommand{\@biblabel}[1]{\quad#1.}
\makeatother



% Header and Footer with logo
\usepackage{lastpage,fancyhdr,graphicx}
\usepackage{epstopdf}
%\pagestyle{myheadings}
\pagestyle{fancy}
\fancyhf{}
%\setlength{\headheight}{27.023pt}
%\lhead{\includegraphics[width=2.0in]{PLOS-submission.eps}}
\rfoot{\thepage/\pageref{LastPage}}
\renewcommand{\headrulewidth}{0pt}
\renewcommand{\footrule}{\hrule height 2pt \vspace{2mm}}
\fancyheadoffset[L]{2.25in}
\fancyfootoffset[L]{0.0in}
%\fancyfootoffset[L]{2.25in}
%\lfoot{\today}

%% Include all macros below

\newcommand{\lorem}{{\bf LOREM}}
\newcommand{\ipsum}{{\bf IPSUM}}

%% END MACROS SECTION


\begin{document}
\vspace*{0.2in}

% Title must be 250 characters or less.
%\begin{flushleft}

{\Large
\textbf\newline{S1 Appendix for identifying promising candidate radiotherapy protocols via GPU-GA~\textit{in-silico} } % Please use "sentence case" for title and headings (capitalize only the first word in a title (or heading), the first word in a subtitle (or subheading), and any proper nouns).
}
\newline
% Insert author names, affiliations and corresponding author email (do not include titles, positions, or degrees).
\\
Wojciech Ozimek\textsuperscript{1},
Rafa\l{} Bana{\'s}\textsuperscript{2},
Pawe\l{} Gora\textsuperscript{3},
Simon D. Angus\textsuperscript{4,*},
Monika J. Piotrowska\textsuperscript{5},
%with the Lorem Ipsum Consortium\textsuperscript{\textpilcrow}
\\
\bigskip
\textbf{1} Ardigen SA, 76 Podole Street, 30-394 Krakow, Poland
\\
\textbf{2} NVIDIA Corporation, Al. Chmielna 73, 00-801, 00-001 Warsaw, Poland
\\
\textbf{3} Institute of Informatics, University of Warsaw,
Banacha 2, 02-097 Warsaw, Poland
\\
\textbf{4} Department of Economics and SoDa Laboratories, Monash Business School, Monash University,\\ Wellington Rd, Clayton, 3800, Australia
\\
\textbf{5} Institute of Applied Mathematics and Mechanics, University of Warsaw, Banacha 2, 02-097 Warsaw, Poland
\bigskip

% Insert additional author notes using the symbols described below. Insert symbol callouts after author names as necessary.
% 
% Remove or comment out the author notes below if they aren't used.
%
% Primary Equal Contribution Note
%\Yinyang These authors contributed equally to this work.

% Additional Equal Contribution Note
% Also use this double-dagger symbol for special authorship notes, such as senior authorship.
%\ddag These authors also contributed equally to this work.

% Current address notes
%\textcurrency Current Address: Dept/Program/Center, Institution Name, City, State, Country % change symbol to "\textcurrency a" if more than one current address note
% \textcurrency b Insert second current address 
% \textcurrency c Insert third current address

% Deceased author note
%\dag Deceased

% Group/Consortium Author Note
%\textpilcrow Membership list can be found in the Acknowledgments section.

% Use the asterisk to denote corresponding authorship and provide email address in note below.
* simon.angus@monash.edu.au
%\end{flushleft}


% ------------------------------------------------------------------------- %


\section*{Infrastructure used in experiments}
All the experiments with programs prepared in C++ were run using a computational infrastructure provided by the Interdisciplinary Centre for Mathematical and Computational Modelling of the University of Warsaw (ICM) (\textit{39}). Experiments requiring CPUs were run on $36$ cores of a Rysy cluster, with Intel(R) Xeon(R) Gold 6154/6252 CPU processors (3.7~GHz) and 380 GB RAM. Experiments requiring GPUs were run on 4~GPUs of graphical processing units NVIDIA V100 on the same cluster - Rysy. At early stage of the project, some preliminary tests were also run on the Topola cluster.


\section*{GA Operators: additional details}

\subsection*{Selection}
Selection procedure is an important step for GA to obtain the convergence and desired results. During the process, a portion of the current (parent) population is selected and used as a parental population for a new generation. It ensures diversity of protocols, enhances exploration of the search space, helps in preventing overfitting to relatively good protocols, and leads to higher possibility for finding better results. To empower results, we use 3~different selection algorithms, commonly applied in GA problems: \textit{simple selection}, \textit{roulette wheel selection} and \textit{tournament selection}, c.f. Fig.~4 (panel B) in the main article. 

The first method, \textit{simple selection}, sorts protocols by associated fitness values in descending order and selects 40\%  of them. This algorithm is very simple, however, it might limit the searching space for new possible treatment protocols and result in being stuck in local minima. 

To prevent such situation, we use two alternative algorithms, which might mitigate the aforementioned risk. In the \textit{roulette wheel selection} algorithm, called also fitness proportionate selection, we associate fitness values with the probability of selection. The probability $p_i$ for each protocol is calculated as 
\[
p_i = \frac{f_i}{\sum^N_{j=1}f_j},
\]
where $f_i$ denotes a fitness value for $i$-th protocol and $N$ is the number of treatment protocols in the considered generation ($40$). Next, we calculate the cumulative sum of those probabilities, summing them all to 1. To illustrate \textit{roulette wheel selection}, we can use a roulette wheel, where the proportions of the wheel segments are represented by probability $p_i$. We order the segments according to their order on the wheel and for each segment, we calculate sum of probabilities of all previous segments. For choosing a single treatment protocol, we take a random value $q$ between 0 and 1 and select a protocol associated with the first segment for which the partial sum of probabilities is higher or equal to $q$. We run the \textit{roulette wheel selection} algorithm in the loop to obtain treatment protocols as parents for the next generation (40\% i.e. 16 protocols). This method promotes protocols with higher probabilities, however, it allows to select weaker protocols with a small chance. This non-zero chance for weak protocols is an advantage over the simple selection which may help in escaping from the local minima.

For a similar purpose, we use the third selection algorithm, the \textit{tournament selection}, where we randomly select $k$~protocols and run tournaments among them, so that the best from the selected $k$~protocols wins the tournament and is eventually selected for the further consideration (crossover and mutation). For each protocol, the probability of being selected to a given tournament is associated with its fitness value - first, all protocols are sorted according to the fitness value, then the probability of selecting the $k$-th best protocol to the tournament is $p_i = q*(1-q)^{(k-1)}$ with a chosen probability $q$.

%\begin{itemize}
%    \item probability of the best protocol: $p_1 = q$
%    \item probability of the second best protocol: $p_2 = q*(1-q)$
%    \item probability of the $k$-th best protocol: $p_i = q*(1-q)^{(k-1)}$
%\end{itemize}

In our case, we repeat selection of $k=16$~protocols for a tournament $16$ times to obtain the desired number of parents protocols for crossover. Similarly to the \textit{roulette wheel selection}, this algorithm also allows weaker protocols to be selected. Comparing to the roulette selection, we can adjust the probability of selection by modifying the value of the parameter $q$. In this work we use the $q=0.9$ as it enables the \textit{roulette wheel selection} to select some worse performing protocols and reduces the risk of getting stuck in local minima.

%For all three selection operators, we retain 10\% of the best protocols (according to their fitness score) using the simple selection. These retained protocols are not taken into account in crossover and mutation operations to prevent the loss of the best protocols while GA is being proceeded.

In addition, for each GA generation we use simple selection to retain 10\% of the best protocols (according to their fitness score). These protocols are not taken into account in crossover and mutation operations, to prevent the loss of the best protocols, and copied directly to the next GA generation.

\subsection*{Crossover}
\label{crossover}
The goal of the crossover operation is to produce the new generation from the selected parents. In our experiments, the newly generated child inherits genes (doses at time-steps) from two randomly selected parents. This operation is repeated until the size of new population is appropriate. We introduce three different types of crossover algorithms: \textit{single-point crossover}, \textit{two-point crossover} and \textit{uniform crossover}, see Fig.~4 panel C in the main article. For crossover operator, we benefit from storing protocols as sparse vectors (with zeroes corresponding to the lack of dose and non-zeroes to a dose given at the particular time-step). This makes algorithms more conceptually straightforward and very similar to those presented in the literature (\textit{37}, \textit{38}).

In the \textit{single-point crossover}, we randomly choose two parent protocols (from the pool of previously selected patents) and select the same crossover point for both of them. Next, we swap the parts of protocols after the crossing points and obtain two children protocols. Crossing point is selected randomly, although with restricted range of choice between $25\%$ and $75\%$ of the length of a protocol. This limitation is important in the case of sparse representation, because in a situation when only a few first or last positions are swapped, the protocols will remain unchanged.

Second algorithm, \textit{two-point crossover}, is a variation of the aforementioned operator. Instead of selecting a single crossing point, we randomly selects two points. For one child protocol, the first and the last parts come from the first parent while the middle from the second parent. Analogously, the second child protocol has the middle part from the first parent and the first and the last parts from the second parent. In this case, we also introduce limitations for the location of crossover points positions. The range for the first point is set to between $25\%$ and $50\%$ of the length of a protocol, while for the second: between $50\%$ and $75\%$. Again the procedure is repeated as long as we get the required number of children protocols. 

The last introduced crossover operation is the so-called \textit{uniform crossover}, which also takes two parent protocols and returns two children protocols. However, the child protocol is created by selecting single positions, one-by-one, from both parent protocols with equal probability. If one child receives a gene from the first parent, then the other child receives a corresponding gene from the second parent on this position.

Clearly, protocols generated by \textit{crossover operators} might do not meet the constraints regarding the minimal time interval between doses and maximal single dose value and the total dose. To prevent the former issue, we designed  \textit{guard algorithms}. One of them identifies doses that violate time constraint and tries to assign them at another time-steps. If all the non-zero doses are too close to each other to allocate a dose between them, the algorithm skips dose allocation. The latter problem with exceeding the total acceptable dose  (10~Gy)  was resolved by another guard algorithm -- if the total dose exceeds 10~Gy then the algorithm starts to subtract the value of 0.25~Gy from the consecutive largest doses in a loop, until the sum of all doses for a protocol is acceptable.

\subsection*{Mutation}
Mutation is an essential genetic operator to sustain diversity of a population and to avoid a stuck of the GA in local minima. Mutation operators are applied with a low probability to reduce the possibility of random search and loss of properties of good protocols. On the other hand, usage of too low mutation probability  lower exploratory abilities of GA. In the presented study, we implement four different mutation operators: \textit{swap mutation}, \textit{split mutation}, \textit{dose time mutation} and \textit{dose value mutation}. For simplicity, in the case of mutation operators we use sparse vector representation of protocols, similarly as in the case of crossover operators.

\textit{Swap mutation} operator swaps two doses. We start by iterating over protocols and selecting (with a probability $p$) a random position on the chromosome to swap. Such operation might break the constraints regarding minimal time interval between two doses and thus we apply the same \textit{guard algorithm} %as described in Sec.~\ref{crossover}
to overcome that issue.

\textit{Split mutation} operator randomly (with probability $p$) selects non-zero (larger than 0.25~Gy) dose and splits it into two smaller ones.
%If the selected dose is larger than the minimal value of 0.25~Gy, it is split into two smaller doses. 
Next, the algorithm removes the split dose and allocates the smaller doses at randomly selected positions. It does not replace doses but appends them to existing values, only if the new, combined dose, is not larger than a maximal allowed single dose value (chosen for the particular GA experiment). If the new combined dose is larger than maximal allowed single dose, algorithm skips the operation and tries to find a better position for a dose. Similarly as in the case of the swap mutation, we used the \textit{guard algorithms} to meet the constraints of both minimal time interval between doses and the maximal value of a total dose.

The third operator, \textit{dose time mutation}, alters the position of doses in a protocol. With a probability of $p$, it randomly selects the dose at a source position from a protocol and the new destination position for it. Next,  source dose is added to the dose at the destination position and the dose at the source position is removed (zeroed). To meet the constraints, we clamp the dose at the destination position to maximal allowed single dose value. The remained dose is allocated back at the source position. Using this algorithm, we avoid losing dose values from a protocol keeping the sum of doses for the protocol at the same level.

The last introduced operator, \textit{dose value mutation}, changes the value of selected dose. With a probability $p$ we allocate a new dose to the protocol. First, we calculate the difference between a sum of dose values and the maximum total dose (10~Gy). If it is smaller or equal to zero we do nothing, otherwise (which may happen only in the case of applying more than $1$ mutation operator to a given protocol), we randomly select an existing dose in the protocol and replace it with a larger dose, which does not break the maximum allowed sum limit. However, this might break a time interval constraints, so again we use the \textit{guard algorithm}. The GA algorithm allows for multiple types of mutation and also repeated use of a given type of mutation to be applied for a single protocol. Such a combination of mutations induces the possibility for a loss in a total dose. The benefit of this mutation operation is the increase of the sum of doses for a protocol, alleviating the possible loss of dose values from split and swap mutations.

\section*{GA metrics}
For every experiment we collected data to further evaluate the performance of protocols. For each GA loop iteration we stored the best fitness function values and associate the best protocols. Next, we use them to analyse and compare the best protocols from different iterations and experiments. In order to check whether our GA model does not overfit, for each iteration we store the average value of a fitness function for all protocols.

\section*{GA experiments}
Due to the limited resources and relatively long computation time, we organised the experiments in batches. Each batch was focusing on particular assumptions and hypotheses from previous runs, so results of experiments from previous batches influenced our decisions on how to configure the next series of experiments. 

In total, we ran $8$ series of experiments, their settings are presented in %\ref{tab:exp_settings}~
S1~Table. Below, we focus on the last batch among all the performed experiments since others served for the improvement of used approaches by testing various settings and configurations of GA algorithms.

Based on the results of previous batches, in the eighth batch we decided to run the best performing configuration of operators for all available single maximum doses and time-delay intervals. We conducted experiments with the three introduced selection operators, \textit{roulette wheel selection}, \textit{tournament selection} and \textit{simple selection}. We used \emph{two-point crossover} operator and a combination of three mutation operators, namely: \textit{swap mutation}, \textit{split mutation}, and \textit{dose time mutation}. To capture all of the possible time intervals and maximal single dose configurations, we finally decided to use intervals of 6, 12, 24, and 48~h and single admission doses limited to 1.25, 2.0, 2.5, 3.0, 3.5, 4.0 and 4.5~Gy. This gave us a total of 84 configurations ($3$ selection operators $\times$ $4$ time delay minimal intervals $\times$ $7$ maximal single dose values), increasing a searching space compared to the previous study (\textit{7}). Each individual configuration was run three times to ensure robustness of the results. Because our GA algorithm usually converged after 30 iterations we decide to use 50 iterations of GA per single run. During evaluation of results, we decided to conduct additional 3 runs up to 100 iterations of GA for \textit{roulette wheel selection}, because for some experiments, the variability of fitness function was higher. However, it did not affect the results. Each protocol was evaluated by simulations 80 times (8 times for each of 10 tumours in library, see Section \textit{The fitness function \& experimental setup} in the main article), the average and standard deviations of evaluation values were calculated. For each configuration, we collected values for 3 best protocols from all iterations obtained during those 10 GA runs.

The results of these experiments for 3 best protocols obtained during GA runs are summarised in %\ref{tab:roulette-selection}
S2~Table. The superiority of \emph{simple selection} over \emph{tournament} and \emph{roulette wheel} is visible. Clearly, the results for \emph{simple selection} are dominant for the shortest time interval of 6~h.


% ----------------------------------------------------------------------- %
\end{document}
% ----------------------------------------------------------------------- %


\begin{table}[htp]
    \centering
    \caption{Configurations of batches in GA experiments. \textit{Number of runs} represents the number of GA runs for each configuration of settings. Starting from batch 6, we introduced the retaining best 10\% of protocols. Final evaluation on all configurations was conducted in the batch 8 (for robustness, as least 3 times for each configuration).}
    \begin{tabular}{|c|c|c|c|c|c|c|}
    \hline
        Series & Number & \multirow{3}{*}{Selection} & \multirow{3}{*}{Crossover} & \multirow{3}{*}{Mutation} & Max & Time \\
        No. & of runs & & & & single & interval \\
        & & & & & dose [Gy] & [h] \\
        \hline
        \multirow{2}{*}{1} & \multirow{2}{*}{8} & simple, & single-point & dose time, & \multirow{2}{*}{2} & \multirow{2}{*}{3} \\
         & & tournament & uniform & dose value & & \\
        \hline
        \multirow{2}{*}{2} & \multirow{2}{*}{24} & simple, & \multirow{2}{*}{single-point} & dose time, & \multirow{2}{*}{2.5} & 6, \\
         & & tournament & & dose value & & 24 \\
        \hline
        \multirow{4}{*}{3} & \multirow{4}{*}{12} & \multirow{4}{*}{simple} & single-point, & dose time, & \multirow{4}{*}{2.5} & \multirow{4}{*}{24} \\
        & & & two-point, & dose value, & & \\
        & & & uniform & swap, & & \\
        & & & & split & & \\
        \hline
        \multirow{3}{*}{4} & \multirow{3}{*}{22} & \multirow{3}{*}{simple} & single-point, & dose time, & 3, 3.5, & \multirow{3}{*}{24} \\
        & & & two-point, & dose value & 4, 4.5 & \\
        & & & uniform & & & \\
        \hline
        \multirow{2}{*}{5} & \multirow{2}{*}{25} & \multirow{2}{*}{simple} & two-point, & dose time, & 2, 2.5, 3, & \multirow{2}{*}{48} \\
        & & & uniform & dose value & 3.5, 4, 4.5 & \\
        \hline
        \multirow{2}{*}{6} & \multirow{2}{*}{25} & \multirow{2}{*}{simple} & \multirow{2}{*}{two-point} & dose time, & 2, 2.5, 3, & 24, \\
        & & & & dose value & 3.5, 4, 4.5 & 48 \\
        \hline
        \multirow{4}{*}{7} & \multirow{4}{*}{30} & \multirow{4}{*}{simple} & \multirow{4}{*}{two-point} & dose time, & 1.25, & 6, \\
        & & & & dose value & 2, 2.5, & 12, \\
        & & & & dose value & 3, 3.5, & 24, \\
        & & & & dose value & 4, 4.5 & 48 \\
        \hline
        \multirow{4}{*}{8} & \multirow{4}{*}{272} & simple & single-point & dose time, & 1.25, & 6, \\
        & & tournament & two-point & dose value, & 2, 2.5 & 12, \\
        & & roulette wheel & uniform & swap, & 3, 3.5, & 24, \\
        & & & & split & 4, 4.5 & 48 \\
        \hline
    \end{tabular}
    \label{tab:exp_settings}
\end{table}


\begin{table}[ht]
\centering
\caption{
Summary of the results for selection operators: \emph{simple selection}, \emph{roulette wheel selection} and \emph{tournament selection}. Reported values are means and standard deviations (for 3 best protocols). Bold values indicate the highest fitness function values for a particular time interval and maximum dose value out of available selection operators.}\label{tab:roulette-selection}
\begin{tabular}{|c|c|r|r|r|r|}
    \hline
    \multicolumn{2}{|c|}{\multirow{2}{*}{\textbf{Simple selection}}} & \multicolumn{4}{c|}{Min time interval} \\
    \cline{3-6}
    \multicolumn{2}{|c|}{} & 6~h & 12~h & 24~h & 48~h \\
    \hline
    \multirow{7}{*}{Max single dose} & 1.25~Gy & \textbf{1142.55 (4.84)} & \textbf{1160.79 (3.85)} & \textbf{1046.52 (4.09)} & 982.75 (1.39) \\\cline{2-6}
    & 2.0~Gy & \textbf{1179.78 (6.49)} & \textbf{1181.75 (3.29)} & \textbf{1156.23 (2.7)} & 1034.06 (0.21) \\\cline{2-6}
    & 2.5~Gy & \textbf{1197.31 (16.73)} & 1198.58 (4.57) & \textbf{1179.41 (17.76)} & \textbf{1074.56 (4.55)} \\\cline{2-6}
    & 3.0~Gy & \textbf{1213.46 (21.42)} & \textbf{1220.67 (4.25)} & \textbf{1193.23 (11.04)} & \textbf{1116.95 (0.53)} \\\cline{2-6}
    & 3.5~Gy & \textbf{1234.54 (7.56)} & \textbf{1235.49 (14.59)} & 1240.3 (22.22) & \textbf{1153.63 (1.73)} \\\cline{2-6}
    & 4.0~Gy & \textbf{1248.37 (44.21)} & \textbf{1271.51 (26.07)} & \textbf{1259.14 (6.38)} & 1153.24 (6.83) \\\cline{2-6}
    & 4.5~Gy & \textbf{1268.23 (17.44)} & \textbf{1279.39 (7.78)} & 1271.97 (10.3) & \textbf{1189.35 (15.55)} \\\cline{2-6}
    \hline\hline
    \multicolumn{2}{|c|}{\multirow{2}{*}{\textbf{Tournament selection}}} & \multicolumn{4}{c|}{Min time interval} \\
    \cline{3-6}
    \multicolumn{2}{|c|}{} & 6~h & 12~h & 24~h & 48~h \\
    \hline
    \multirow{7}{*}{Max single dose} & 1.25~Gy & 1132.24 (6.39) & 1154.73 (1.48) & 1045.35 (0.88) & 988.55 (8.61) \\\cline{2-6}
    & 2.0~Gy & 1167.22 (1.93) & 1180.82 (5.39) & 1132.48 (22.7) & \textbf{1034.82 (1.95)} \\\cline{2-6}
    & 2.5~Gy & 1187.57 (8.01) & \textbf{1198.64 (3.99)} & 1163.35 (10.12) & 1074.41 (1.26) \\\cline{2-6}
    & 3.0~Gy & 1192.37 (4.47) & 1207.65 (22.55) & 1190.1 (6.96) & 1116.0 (4.31) \\\cline{2-6}
    & 3.5~Gy & 1210.81 (47.82) & 1226.08 (24.22) & \textbf{1249.1 (5.83)} & 1149.67 (4.89) \\\cline{2-6}
    & 4.0~Gy & 1233.58 (12.3) & 1250.71 (11.65) & 1253.17 (12.68) & \textbf{1158.9 (2.91)} \\\cline{2-6}
    & 4.5~Gy & 1251.98 (27.07) & 1277.99 (11.41) & \textbf{1275.11 (1.1)} & 1172.56 (13.56) \\\cline{2-6}
    \hline\hline
    \multicolumn{2}{|c|}{\multirow{2}{*}{\textbf{Roulette selection}}} & \multicolumn{4}{c|}{Min time interval} \\
    \cline{3-6}
    \multicolumn{2}{|c|}{} & 6~h & 12~h & 24~h & 48~h \\
    \hline
    \multirow{7}{*}{Max single dose} & 1.25~Gy & 1130.29 (5.84) & 1155.53 (3.01) & 1040.01 (1.67) & \textbf{991.99 (8.55)} \\\cline{2-6}
    & 2.0~Gy & 1161.11 (3.14) & 1178.66 (5.11) & 1138.92 (15.9) & 1031.1 (2.26) \\\cline{2-6}
    & 2.5~Gy & 1176.81 (4.16) & 1193.78 (15.9) & 1166.56 (14.12) & 1052.09 (15.31) \\\cline{2-6}
    & 3.0~Gy & 1202.3 (21.66) & 1208.35 (17.35) & 1189.1 (12.34) & 1107.09 (12.64) \\\cline{2-6}
    & 3.5~Gy & 1224.58 (20.92) & 1232.26 (1.42) & 1217.83 (23.16) & 1132.03 (11.45) \\\cline{2-6}
    & 4.0~Gy & 1227.4 (29.44) & 1227.82 (7.33) & 1238.54 (20.19) & 1158.25 (6.18) \\\cline{2-6}
    & 4.5~Gy & 1246.75 (25.58) & 1261.55 (11.44) & 1255.3 (13.21) & 1168.19 (21.0) \\\cline{2-6}
    \hline
\end{tabular} 
\medskip
\raggedright
\label{tab:results-table}
\end{table}



%{\color{red}{\bf\Large I guess at some point we decided not to add a detail description of batches, am I right?\\}}



{\bf Description of Batches}\\
In this section, we describe the most important batches of GA experiments we conducted.

%\woj{Batch no.1, between 23-25.08.2020}
{\bf Batch No. 1}

In the first batch of experiments we used the following fitness function
\begin{equation*}
   f(p_{i,j}) = 1 600 - n_{i,j}, 
\end{equation*}
where $p_{i,j}$ denotes considered protocol $i$ evaluated on a single tumour $j$, 1\,600 is the maximal number of possibly occupied sites, $n_{i,j}$ denotes the number of the occupied sites after testing protocol $i$ on tumour $j$.

Comparing to other batches in which formula ~\eqref{eq:final_fit_fun} was used to calculate the fitness function, this formula leads to increase of all fitness by $100$.

Based on our preliminary experiments, in this batch we decided to use the \textit{dose time mutation} and \textit{dose value mutation} operators. Our goal was to scrutinise two types of crossover operators, namely \textit{single-point crossover} and \textit{uniform crossover}, with selection operators: \textit{simple selection} and \textit{tournament selection}. For tournament selection, we tested three different probability values: 0.099, 0.5 and 0.9. The maximum single dose was 2.0Gy, minimal time interval was equal to 3~h and the experiments were run till reached 300 iterations of GA.

The results showed a superiority of \textit{simple selection}, with a stable increase of fitness value. The results for \textit{tournament selection} were lower by ~about 20 points than for the \textit{simple selection}, with the significant drop in fitness value, especially for the lower probability. There was no notable difference between \textit{uniform crossover} and \textit{single-point crossover} operators. 

%\woj{Batch no.2, between 07-16.09.2020}
{\bf Batch No. 2}\\
For the second batch of experiments we used the final fitness function formula \eqref{eq:final_fit_fun}. We decided to increase a maximum single dose to 2.5~Gy, use only \emph{single-point crossover} operator and test more realistic time interval between doses, i.e., 6~h and 24~h. Moreover, we evaluated \textit{dose time mutation} and \textit{dose value mutation} operators, making it to a total number of 24 different experiment configurations.

The results indicated the key importance of time between doses interval, with the average superiority of 25 points in fitness function value for 6~h interval over 24~h interval. As in batch no. $1$, the results for \emph{tournament selection} were lower than for a \emph{simple selection} and the level of variance was significantly larger. Thus, in the further analysis, we focused on \emph{simple selection} to reduce the time required for every batch of experiments.


%\woj{Batch no.3, between 17-21.09.2020}
{\bf Batch No. 3}\\
The main contribution of this batch of experiments was the introduction of annealing probability of mutation in comparison to the prior fixed value. Our annealing formula started with the initial value of 0.1 and decreased towards 0.001 using a quartic function:
\begin{equation*}
   f(i, n) = max((1 - (i - 1) / n) ^ 4 * m_{p}), eps), 
\end{equation*}
where $i$ denotes iteration of GA algorithm, $n$ total number of GA iterations, $m_{p}$ denotes initial value of mutation probability 0.1 and $eps$ a minimal value of mutation probability of 0.001. We tested two sets of mutation operators: in the first, we mutated both \textit{dose time mutation} and \textit{dose value mutation}, while in the second, we additionally used \emph{swap mutation} and \emph{split mutation} operators. Together with this improvement, we evaluated three types of \textit{crossover operators}. To simulate mimic the realistic therapy and reduce the searching space, in this batch we fixed the time interval between dose to 24~h. The rest of the experiment settings was the same as in batch no. 1, with \emph{simple selection}, and maximum single dose of 2.5~Gy.

Fitness function values for the experiments with all four \emph{mutation operators} showed variability at the beginning of the experiments, finally reaching almost the same results as with \textit{dose time mutation} and \textit{dose value mutation} only. There was no significant difference between experiments with annealing mutation probability and experiments with fixed probability values. Finally, the only noteworthy difference occurred for the \emph{uniform crossover}, over the \emph{single-point crossover} and \emph{two-point crossover} operators, achieving up to 20 points increase.

%\woj{Batch no.4, between 26.09-04.10.2020}
{\bf Batch No. 4}\\
We decided to extend the experiments with the different maximum single doses of 3.0~Gy, 3.5~Gy, 4.0~Gy and 4.5~Gy. We applied \emph{mutation operators} to \textit{dose time mutation} and \textit{dose value mutation}, and all other settings remained the same as in the batch no.~3.

The results confirmed the negligible difference between \emph{single-point crossover} and \emph{two-point crossover} operators in all runs. Moreover, the superiority of the \emph{uniform crossover} operator, observed in batch no.~4, was no longer maintained for larger doses, and in consequence, the results for all \emph{crossover operators} remain almost identical. It is worth to underline that the huge improvement of the fitness function value was observed with the change of maximum dose value, with the lowest fitness function value of 1180 for 3.0~Gy and up to 1260 for 4.5~Gy.

%\woj{Batch no.5, between 01-10.10.2020}

{\bf Batch No. 5}\\
Having established the importance of the maximum single dose, we increased the time interval between doses from 24~h to 48~h. To capture larger number of possible maximum doses, we tested doses between 2.0~Gy up to 4.5~Gy, with 0.5~Gy steps. Finally, to limit the computational time, we investigated only \emph{two-point crossover} and \emph{uniform crossover} operators.

Experiment outcomes showed an interesting fact, namely for time interval of 48~h, the \emph{two-point crossover} operator results were higher by 10 points on average than the results for \emph{uniform crossover} operator, which is the opposite to the batch no.3 results, with time interval of 24~h.
Increasing the time interval to 48~h we considerably lowered the results, even up to 100 points, and the highest difference was observed for 2.5~Gy and 3.5~Gy maximum doses.

%\woj{Batch no.6, between 10-21.10.2020 and batch 6.1 between 01-11.11.2020}
{\bf Batch No. 6}\\
To achieve our goal of reaching the highest value of fitness function, we introduced retaining of the best protocols from the last iteration of the GA. Ten percent of the best parent protocols were passed unchanged to the next iteration, allowing us to maintain at lest the best results from the previous iteration. We evaluated this setup using only \emph{two-point crossover} operator, mutation operators acting on dose time and value with annealing probability and \emph{simple selection}. As in batch no.~5, we tested maximum doses between 2.0~Gy up to 4.5~Gy, with 0.5~Gy steps, but this time two time intervals of 24~h and 48~h.

The results obtained with the improved algorithm showed an increase in results stability, but the final value remained akin. The important conclusion from this batch is that we can reduce the number of iterations because the GA convergence is achievable earlier than 50\% of all the iterations.

For comparison with results from \textit{(7)}, we run additional experiments with time interval of 6~h and 12~h, and maximum single dose of 1.25~Gy. The results for 6~h time interval were lower by 10 points comparing to 12~h interval, which is in contrary to our previous experiments.
Overall, the results for additional experiments were similar to results for experiments with 2.0~Gy maximum dose \& 24~h time interval or 3.5~Gy \& 47~h time interval.

%\woj{Batch no.7, between 24-29.11.2020 and batch 7.1, between 06-14.12.2020}
{\bf Batch No. 7}\\
To establish the difference between the best and worst protocols, we introduced inversion of the GA. Our intention was to obtain the lower fitness function value and to analyse the protocols, searching for possible patterns. The experiments set up was similar to batch {no.~6}, testing both 24~h and 48~h time intervals, and maximum single doses between 2.0~Gy and 4.5~Gy. Similarly as in batch {no.~6}, we tested additional two configurations with maximum dose of 1.25~Gy and time intervals of 6~h and 12~h.

The GA algorithm was converging fast, however for obtained protocols we saw a significant variance. Most of the experiments were able to achieve the lowest fitness function value of about 910.


\end{document}