\section{Our Method: DS-MPEPC}
In this section, we describe our algorithm for robot navigation. This is based on our new collision probability function $\tilde{p}_{c_i}$ (Section~\ref{ref:coll_prob}) which is used to define the survivability term $\tilde{p}_{s_i}$ (as in Equation~\ref{eqn:psi}). Moreover, we define an additional terminal state cost $J_{terminal}$ (Section~\ref{ref:term_cost}) based on time-to-goal and time-to-collision values.
The modified cost function is given below:
\begin{multline}\label{eqn:mod_cost}
\tilde{J}(q_{z*}) = \\ \sum^N \big( \tilde{p}_{s_i} * J_{progress_i} + J_{action_i} + (1 - \tilde{p}_{s_i}) * J_{collision_i} \big) \\+ J_{terminal}(x_N)    
\end{multline}


\subsection{Collision Probability ($\tilde{p}_{c_i}$)}\label{ref:coll_prob}
As seen in Equation~\ref{eqn:pci}, the collision probability is defined as a function of the robot's distance to its closest obstacle ($d_o$). The collision probability captures the risk associated with a particular configuration and is a bell-shaped curve that raises to $1$ when the $d_o = 0$, and $0$ when $d_o = \infty$. Our proposed collision probability formulation aims to be less conservative in terms of assigning a collision probability to a trajectory segment. 

A robot almost in collision with a wall or an obstacle has a survivability of zero. From Equation~\ref{eqn:orig_cost}, we observe the progress cost is scaled down to zero by the survivability term. Consider a goal located at a short distance (e.g., $2$m) in front of the robot. Since the progress cost is scaled down to zero, the planner does not select a trajectory that leads the robot to the goal even when it increases the distance to the wall and is safe. A purely distance-based collision probability does not capture the ability of the robot to reduce the {\em{risk}} by moving away from the obstacle.

We modify the collision probability function based on the intuition that, for a robot with close obstacle proximity, it is safer to move away from the obstacle rather than toward it. We consider the following behavior for the modified $p_c$ function.
\begin{itemize}
    \item $p_c$ increases as the robot comes close to an obstacle.
    \item $p_c$ values reduce along motion directions with higher time to collision. 
\end{itemize}

Hence, our modified collision probability is a function of distance to the closest obstacle (a reactive term) and time to collision (an anticipatory term). Time to collision (TTC) is defined as the number of seconds in the future before a collision if the agent and obstacle continue their direction of motion. A TTC value of $0$ indicates an agent is already in collision, while a TTC value of $\infty$ indicates the agent and the obstacle will not collide if they follow their current velocities forever.

The modified collision probability has two terms as  represented below:
\begin{equation*}
\tilde{p}_c = \texttt{Reactive term} * \texttt{Anticipatory term}
\end{equation*}

The reactive component captures the {\em{risk}} associated with being at a particular configuration given by the 2-D position and orientation. The reactive term is distance-based and physically represents the localization uncertainty. Thus this terms is same as $p_{c}$.  

The anticipatory component is a function of time-to-collision (TTC), which provides a look ahead at the time available to the robot to avoid collision based on the velocity.

\begin{equation}\label{eqn:mod_pc}
\tilde{p}_c = \exp{\bigg(-\frac{d_o^2}{\sigma_{d}^2}\bigg)} * \bigg(1 - a\exp{\bigg( \frac{(1/TTC)^2}{\sigma_{1/TTC}^2} \bigg)} \bigg)
\end{equation}

From Equation~\ref{eqn:mod_pc}, we see that the anticipatory component only reduces the effect of the distance-based term provided the robot moves away from the obstacle based on the velocity direction. Thus, the anticipatory component cannot increase the collision probability.
\\\\
\subsubsection{Weight parameter $a$}
A robot in a narrow passage moving parallel to the walls has a time-to-collision value of $\infty$. The robot is not heading towards the obstacle, but it could be unsafe to regard the collision probability as $0$ in tight spaces. We introduce a weight parameter $a$ that limits the effect of the time-to-collision term. With a weight of $a < 1$, the time-to-collision component merely reduces the effect of the distance-based term but does not nullify it. Thus, the following Lemma holds:

\begin{lemma}\label{lemma:1}
The proposed collision probability ($\tilde{p}_{c_i}$) results in a survivability ($\tilde{p}_{s_i}$), which is at least the original survivability ($p_{s_i}$). Thus, it is less conservative in terms of regarding a trajectory segment as survivable.
\end{lemma}
\begin{proof}
From Equation~\ref{eqn:mod_pc}, the maximum and minimum values of $\tilde{p}_{c_i}$ in relation to $p_{c_i}$ is given by,
$$
\max \tilde{p}_{c_i} = p_{c_i}, \quad \min \tilde{p}_{c_i} = (1 - a) \cdot p_{c_i}.
$$
From the definition of survivability, $\tilde{p}_{s_i} = \Pi_1^i (1 - {p}_{c_k})$.
From the maximum values of $\tilde{p}_{c_i}$, we know $\min (1 - \tilde{p}_{c_i}) = (1 - p_{c_i})$. Thus, the minimum values of $\tilde{p}_{s_i}$ are bounded by
$$\min \tilde{p}_{s_i} = \Pi_1^i (1 - {p}_{c_k}) = p_{s_i}.$$
\end{proof}

\begin{lemma}\label{lemma:2}
For a robot in collision state $\tilde{p}_{s_i} = 0, \forall i$
\end{lemma}
\begin{proof}
    When the robot is in collision state (i.e., $d_o = 0$), the TTC value is $0$ for any non-zero relative velocity between the robot and the obstacle. Hence, in Equation~\ref{eqn:mod_pc} the term
    $\big(1 - a\exp{\big( \frac{(1/TTC)^2}{\sigma_{1/TTC}^2} \big)} \big) = 1$. Thus, $\tilde{p}_{c_1} = p_{c_1} = 1 \implies \tilde{p}_{s_1} = 0$. From Equation~\ref{eqn:psi}, we get $\tilde{p}_{s_i} = 0, \forall i$.
\end{proof}

\begin{figure*}[h!]
\centering
\begin{subfigure}{0.24\textwidth}
  \centering
  \includegraphics[trim={5cm 5cm 0 2cm},clip, width=.95\linewidth]{Figures/Evaluation/sample.png}
  \caption{Sampled Trajectories}
  \label{fig:sample}
\end{subfigure}
\begin{subfigure}{0.24\textwidth}
  \centering
  \includegraphics[trim={5cm 5cm 0 2cm},clip,width=.95\linewidth]{Figures/Evaluation/currentcost.png}
  \caption{MPEPC Cost}
  \label{fig:cost}
\end{subfigure}
\begin{subfigure}{0.24\textwidth}
  \centering
  \includegraphics[trim={5cm 5cm 0 2cm},clip,width=.95\linewidth]{Figures/Evaluation/ttg.png}
  \caption{Time-to-Goal}
  \label{fig:ttg}
\end{subfigure}
\begin{subfigure}{0.24\textwidth}
  \centering
  \includegraphics[trim={5cm 5cm 0 2cm},clip,width=.95\linewidth]{Figures/Evaluation/ttc.png}
  \caption{Time-to-Collision}
  \label{fig:ttc}
\end{subfigure}
\caption{We illustrate a scenario with two robots in a T-shaped corridor. One robot is stationary, and the other robot is moving toward the goal by turning into the corridor. (a) We show a few sampled trajectories for the robot moving toward the goal. (b) We illustrate 50 trajectories with the lowest cost as computed with the MPEPC cost function. We observe the trajectory distribution is directed toward the other robot, and would eventually lead to a deadlock before reaching the goal. In (c), we show a distribution of 50 trajectories with the lowest TTG values. In (d), we show a distribution of 50 trajectories with the highest TTC value. From (c) and (d) we notice the distribution of trajectories allows the robot to move around the stationary robot and TTC and TTG values provide a metric to reduce deadlocks.}
\label{fig:ttcttg_Eval}
\end{figure*}

\subsection{Terminal Cost ($J_{terminal}$) }\label{ref:term_cost}
The cost formulation optimizes for maximum progress made toward the goal while maintaining safety. The MPEPC cost may cause the robot to deadlock in front of an obstacle because a given local trajectory makes the maximum progress towards the goal while being safe. In some scenarios, moving away from the goal can eventually steer the robot towards the goal. We incorporate this behavior into the cost formulation by introducing a terminal state cost. We define the terminal cost as a function of the agent's expected time-to-goal and time-to-collision as computed at the terminal state of the local trajectory.
In figure~\ref{fig:ttcttg_Eval}, we illustrate a scenario to show time-to-goal and time-to-collision values are suitable for deadlock reduction.
\\
{\subsubsection{Expected Time-to-Goal:}}
The expected time-to-goal is computed as the ratio of the euclidean distance to the goal from the terminal trajectory state and the component of the terminal state velocity in the direction of the goal vector. Hence, if the terminal velocity tends to zero or if the terminal velocity has no components in the goal direction, the expected time-to-goal tends to infinity. Consequently, when the distance to the goal tends to zero, the time-to-goal tends to zero.
When a robot is deadlocked (or frozen), its distance to the goal is non-zero, its velocity is zero, and the time-to-goal is $\infty$. Hence, time-to-goal provides a relevant metric for deadlock detection. 
\\
{\subsubsection{Expected Time-to-Collision:}}
The time-to-collision value depends on the agent's and obstacle's velocities. An agent halting before a static robot would have a time-to-collision of $\infty$, indicating the agent is safe. 
A robot halting and facing an obstacle may have all its future trajectories passing through the obstacle due to its orientation, while the robot looking into free space may have more collision-free trajectories to optimize.
To incorporate the idea, we compute the TTC at the terminal state, assuming the robot travels by maintaining its terminal state orientation with its max velocity. This approximate computation of the time-to-goal is suitable, as it captures the terminal state orientation of the robot.
\\
{\subsubsection{Computing $J_{terminal}$:}}

We define a novel terminal state cost based on the time-to-goal and time-to-collision values, which aid in reducing deadlocks and facilitate the robot reaching the goal. $J_{terminal}$ has three main terms. First, the terminal state survivability $p_{s_N}$ ensures the $J_{terminal}$ shrinks to zero when the trajectory collides. Second, the $C_{TTG}$ term is such that $C_{TTG} \rightarrow 1$ as $TTG \rightarrow \infty$. The third term $C_{TTC} \rightarrow 1$ as $TTC \rightarrow \infty$. Thus the product $C_{TTG} * C_{TTC}$ acts on trajectories with $TTG \rightarrow \infty$, and it prefers trajectories with longer time-to-collision values. As mentioned above, a configuration with a higher time-to-collision may have more trajectories to choose from in the subsequent timestep, thus aiding in resolving a deadlock. Moreover, $J_{terminal}$ has negative values only on trajectories with non-zero survivability to ensure it works only on non-colliding trajectories. We define the terminal state cost as follows:
$$
C_{TTG} = \exp \bigg(-\frac{{1}/{TTG}^2}{\sigma_{1/TTG}^2}\bigg),
$$
$$
C_{TTC} = \exp \bigg(- \frac{1/TTC^2}{\sigma_{1/TTC}^2} \bigg),
$$
\begin{equation}
    J_{terminal} = - p_{s_N} * \bigg( C_{TTG} * C_{TTC} \bigg).
\end{equation}
As can be observed, the terminal state cost ($J_{Terminal}$) is bounded $J_{terminal} = [-1, 0]$.
\begin{lemma}\label{lemma:3}
 The modified cost function maintains the MPEPC property that avoids moving when robot is already in a collision state.
\end{lemma}
\begin{proof}
    When the robot is in collision state ($d_o$), the survivability $\tilde{p}_{s_i} = 0, \; \forall i$ from Lemma~\ref{lemma:2}. Under this survivability condition, the modified cost definition (Equation~\ref{eqn:mod_cost}) reduces to the MPEPC's cost definition (Equation~\ref{eqn:orig_cost}).
\end{proof}

\subsection{Trajectory Optimization}
The navigation optimization problem selects a suitable trajectory and control input by optimizing with our modified cost function. From Section~\ref{ref:cost_mpepc}, we know the navigation problem is framed as an unconstrained finite horizon optimization problem to select a trajectory parameter $z*$. Given a $z*$ the trajectory is completely defined (Section~\ref{sec:smoothlaw}). Thus, for a $z*$, the trajectory is simulated and the cost is computed with our proposed cost function. The optimization minimizes $\tilde{J}(q_{z^*})$ (Equation~\ref{eqn:mod_cost}) to compute an optimal $z*$ having the minimum trajectory cost.

\begin{equation*}
\underset{z^*}{\text{minimize}} \quad \tilde{J}(q_{z^*})   
\end{equation*}

The robot's control inputs in terms of linear and angular velocity is computed from the optimal $z*$ as described in Section~\ref{sec:smoothlaw}.


% \begin{lemma}
% $J_{terminal}$ does not affect the safety offered by the MPEPC cost formulation. 
% \end{lemma}

% For trajectories colliding with an obstacle $p_s = 0$. Thus, $J_{terminal} = 0$. Thus, $J_{terminal} < 0$ only for safe trajectories.

% As the $v \rightarrow 0$, $C_{TTG} \rightarrow 1$. Thus, $J_{terminal}$ is dominated by the product $C_{TTG} * C_{TTC}$. Since, $C_TTC \rightarrow 0$ as the orientation is safe. The terminal cost tends to zero. 
