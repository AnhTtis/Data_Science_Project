\section{Evaluations}
In this section we describe our evaluation setup and highlight the performance on challenging benchmarks.

\begin{figure*}
\centering
\begin{subfigure}{0.19\textwidth}
  \centering
  \includegraphics[width=.9\linewidth]{Figures/Hall2/t_5.png}
  \caption{t = 5s}
\end{subfigure}%
\begin{subfigure}{0.19\textwidth}
  \centering
  \includegraphics[width=.9\linewidth]{Figures/Hall2/t_10.png}
  \caption{t = 10s}
\end{subfigure}
\begin{subfigure}{0.19\textwidth}
  \centering
  \includegraphics[width=.9\linewidth]{Figures/Hall2/t_15.png}
  \caption{t = 15s}
\end{subfigure}
\begin{subfigure}{0.19\textwidth}
  \centering
  \includegraphics[width=.9\linewidth]{Figures/Hall2/t_20.png}
  \caption{t = 20s}
\end{subfigure}
\begin{subfigure}{0.19\textwidth}
  \centering
  \includegraphics[width=.9\linewidth]{Figures/Hall2/t_25.png}
  \caption{t = 25s}
\end{subfigure}
\caption{{\bf Evaluation in real-world environments: Hall scenario}: The figure illustrates a hallway scenario with multiple pedestrians denoted by red and green disks.  This noisy data is captured using sensors The static obstacles in the environment are represented in black by the occupancy map. The robot (blue) follows a pedestrian denoted by a green disk while navigating around static (black) and other dynamic obstacles (red). The red and green traces show the trajectory followed by different pedestrian over the past few timesteps. The black line shows the predicted future trajectory of the pedestrian. The blue trace shows the trajectory of the robot during the navigation. The figure from left to right show the navigation simulation at regular time intervals. We observe DS-MPEPC is able to compute a smooth path around the obstacles.}
\label{fig:hall}
\end{figure*}


\begin{figure*}
\centering
\begin{subfigure}{0.32\textwidth}
  \centering
  \includegraphics[width=.95\linewidth]{Figures/Corridor/3.png}
  \caption{A subfigure}
\end{subfigure}%
\begin{subfigure}{0.32\textwidth}
  \centering
  \includegraphics[width=.95\linewidth]{Figures/Corridor/5.png}
  \caption{A subfigure}
\end{subfigure}
\begin{subfigure}{0.32\textwidth}
  \centering
  \includegraphics[width=.95\linewidth]{Figures/Corridor/6.png}
  \caption{A subfigure}
\end{subfigure}
\caption{{\bf Evaluation in real-world environments: L-shaped corridor}: We highlight the performance in another real-world scene captured using sensors. The figure illustrates a scenario with the robot (blue) following a pedestrian (green disk) into a narrow corridor. Figures (a) and (b) show the actual trajectory followed by the robot during the simulation. Figure (c) shows the evaluated trajectories (gray) by the planner during the simulation, while the optimal trajectory at each planning cycle is denoted in blue. DS-MPEPC is able to compute a smooth trajectory in this challenging scenario.}
\label{fig:lcorridor}
\end{figure*}


\begin{figure*}
\centering
\begin{subfigure}{0.22\textwidth}
  \centering
  \includegraphics[width=.95\linewidth]{Figures/T-corridor/mpepc/1.png}
  \caption{Actual robot trajectory}
\end{subfigure}%
\begin{subfigure}{0.22\textwidth}
  \centering
  \includegraphics[width=.95\linewidth]{Figures/T-corridor/mpepc/2.png}
  \caption{Evaluated trajectories}
\end{subfigure}
\quad \quad \quad
\begin{subfigure}{0.22\textwidth}
  \centering
  \includegraphics[width=.95\linewidth]{Figures/T-corridor/modified/1.png}
  \caption{Actual robot trajectory}
\end{subfigure}
\begin{subfigure}{0.22\textwidth}
  \centering
  \includegraphics[width=.95\linewidth]{Figures/T-corridor/modified/2.png}
  \caption{Evaluated trajectories}
\end{subfigure}
\caption{{\bf T-Corridor}: This scenario considers two robots, one moving from the bottom and turning into the corridor on the right. The other robot is stationary and obstructs this corridor at its entrance. Figures (a) and (b) show the robot trajectory (blue) and evaluated trajectories for the moving robot while using the original MPEPC cost function. We can observe the robot deadlocks. Figures (c) and (d) show the robot trajectory while using our proposed cost modification. In this case, the robot navigates successfully without a deadlock with DS-MPEPC.}
\label{fig:tcorridor}
\end{figure*}



\begin{figure*}
\centering
\begin{subfigure}{0.15\textwidth}
  \centering
  \includegraphics[height=5cm,width=.75\linewidth]{Figures/Corridor_2agent/mpepc/t_2.png}
  \caption{t=2s}
\end{subfigure}%
\begin{subfigure}{0.15\textwidth}
  \centering
  \includegraphics[height=5cm,width=.75\linewidth]{Figures/Corridor_2agent/mpepc/t_4.png}
  \caption{t=4s}
\end{subfigure}
\begin{subfigure}{0.15\textwidth}
  \centering
  \includegraphics[height=5cm,width=.75\linewidth]{Figures/Corridor_2agent/mpepc/t_6.png}
  \caption{t=6s}
\end{subfigure}
\quad \quad \quad
\begin{subfigure}{0.15\textwidth}
  \centering
  \includegraphics[height=5cm,width=.75\linewidth]{Figures/Corridor_2agent/modified/t_2.png}
  \caption{t=2s}
\end{subfigure}%
\begin{subfigure}{0.15\textwidth}
  \centering
  \includegraphics[height=5cm,width=.75\linewidth]{Figures/Corridor_2agent/modified/t_4.png}
  \caption{t=4s}
\end{subfigure}
\begin{subfigure}{0.15\textwidth}
  \centering
  \includegraphics[height=5cm,width=.75\linewidth]{Figures/Corridor_2agent/modified/t_6.png}
  \caption{t=6s}
\end{subfigure}
\caption{{\bf Narrow Corridor}: We consider two robots navigating a narrow corridor in opposing directions. Figure (a)-(c) illustrates the case with two robots using MPEPC's cost function. We observe the robots eventually deadlock in this complex case. Figure (d)-(f) shows the trajectories followed by the robots for the modified cost function by DS-MPEPC. In this case, the robot navigates without colliding and deadlocking and reaches the other side of the corridor. This demonstrates the improved navigation behavior of DS-MPEPC with non-circular agents.}
\label{fig:narrowcorridor}
\end{figure*}

\begin{figure*}
\centering
\begin{subfigure}{0.24\textwidth}
  \centering
  \includegraphics[width=.85\linewidth]{Figures/Circle/modified/t_4.png}
  \caption{t = 4s}
\end{subfigure}
\begin{subfigure}{0.24\textwidth}
  \centering
  \includegraphics[width=.85\linewidth]{Figures/Circle/modified/t_6.png}
  \caption{t = 6s}
\end{subfigure}%
\begin{subfigure}{0.24\textwidth}
  \centering
  \includegraphics[width=.85\linewidth]{Figures/Circle/modified/t_8.png}
  \caption{t = 8s}
\end{subfigure}
\begin{subfigure}{0.24\textwidth}
  \centering
  \includegraphics[width=.85\linewidth]{Figures/Circle/modified/t_10.png}
  \caption{t = 10s}
\end{subfigure}
\caption{ {\bf Multi-Agent Benchmark}: We illustrate a scenarios with four non-circular agents arranged on the perimeter of the circle moving to their diagonally opposite position. The proposed cost function aids in navigation the robot safely in this multi-agent scenario. DS-MPEPC can generate smooth and deadlock-free paths in this scenarios.}
\label{fig:circlescenario}
\end{figure*}

\begin{figure*}
\centering
\begin{subfigure}{0.19\textwidth}
  \centering
  \includegraphics[width=.95\linewidth]{Figures/10agent/modified/t_5.png}
  \caption{t = 5s}
\end{subfigure}
\begin{subfigure}{0.19\textwidth}
  \centering
  \includegraphics[width=.95\linewidth]{Figures/10agent/modified/t_10.png}
  \caption{t = 10s}
\end{subfigure}
\begin{subfigure}{0.19\textwidth}
  \centering
  \includegraphics[width=.95\linewidth]{Figures/10agent/modified/t_15.png}
  \caption{t = 15s}
\end{subfigure}
\begin{subfigure}{0.19\textwidth}
  \centering
  \includegraphics[width=.95\linewidth]{Figures/10agent/modified/t_20.png}
  \caption{t = 20s}
\end{subfigure}
\begin{subfigure}{0.19\textwidth}
  \centering
  \includegraphics[width=.95\linewidth]{Figures/10agent/modified/t_25.png}
  \caption{t = 25s}
\end{subfigure}
\caption{ {\bf Multi-Agent Benchmark}: We illustrate a scenarios with ten robots arranged on the perimeter of the circle moving to their diagonally opposite position. The proposed cost function aids in navigation the robot safely in this multi-agent scenario. DS-MPEPC can generate smooth and deadlock-free paths in this scenarios.}
\label{fig:circlescenario10}
\end{figure*}


\subsection{Evaluation Setup}
Our proposed method is implemented over the MPEPC~\cite{park_mpepc} navigation framework. Our evaluations are run in a MATLAB simulation a laptop running a $2.7$ GHz Quad-Core Intel i7 processor. Moreover, we also test the method by dynamically simulating them on environments generated from real data as in~\cite{park_mpepc}. For our evaluations, the planner uses a receding horizon of $T = 5$s with a timestep of $0.2$s. The weight parameter $a$ in collision probability ($\tilde{p}_{c_i}$) is set to $0.7$ for the evaluation. For $J_{terminal}$ computation, the $\sigma_{1/TTG} = 10^{-3}$ and $\sigma_{1/TTC} = 0.5$ are used. 

\subsection{Navigation Behavior}
We evaluate our cost function in multiple complex scenarios in simulation. In particular, we consider three indoor environments: First, a hall environment with multiple pedestrians and static obstacles, an L-shaped corridor, and a T-shaped corridor. The hall and L-shaped environments are based on real data traces, which are used to create the static obstacles and pedestrian trajectories in the simulation scenario. In the hall scenario, we increase the pedestrians from the four available pedestrian trajectories by spatially moving the pedestrian trajectories to a different portion of the environment to create additional pedestrians.

The hall environment consists of multiple pedestrians (denoted by red disks) and static obstacles (black regions). The navigation scenario involves the robot following a selected pedestrian (green disk) while avoiding collisions with other pedestrians and static obstacles. Figure~\ref{fig:hall} shows the resulting trajectory generated by our cost function.

The L-shaped corridor involves the robot following a pedestrian (green disk) into a narrow corridor. Figure~\ref{fig:lcorridor} shows the resulting trajectory generated by our cost function. We observe the robot successfully maneuvering and entering the narrow passage to follow the moving target.

The T-shaped corridor environment involves a robot turning into a corridor with the other robot staying stationary and obstructing the path. This scenario has a stationary robot blocking the moving robot, and the MPEPC cost formulation deadlocks the agent. The deadlock occurs as the halting trajectory makes the most progress while remaining safe. Our cost modification helps the agent to detour and move around the obstruction to reach the goal. Figure~\ref{fig:tcorridor} shows the resulting trajectory generated by MPEPC and our modified cost function.

\subsection{Multi-Agent Scenario}
We observe our proposed cost function was able to generate multi-agent navigation behavior in cluttered scenarios using non-circular agents. In this subsection, we evaluate our cost function in two multi-agent scenarios. In the first scenario, we consider two agents navigating a narrow corridor in opposing directions. In this particular test case (Figure~\ref{fig:narrowcorridor}), we observe the MPEPC cost to lead the robot to a deadlock, and our modified cost function navigates the robots safely.

Second, we consider a circle scenario with four-agent and ten-agents. The robots are initially on the boundary and move towards the diagonally opposite location. Figures~\ref{fig:circlescenario} and~\ref{fig:circlescenario10} illustrate the resulting trajectories in this scenario. We observe the planner navigates the robot safely and maintains a safe distance between the agents.

\subsection{Performance}
The proposed cost function involves computing the time-to-collision and time-to-goal values which are fast to compute. The optimization problem is similar to MPEPC formulation and is suitable for real-time navigation performance.