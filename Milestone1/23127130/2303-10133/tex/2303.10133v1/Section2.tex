\section{Related Works}

%\nocite{trautman2013robot}.

\subsection{Navigation in Dynamic Environments}
Potential field methods~\cite{pfm} navigate robots by adding a repulsion field around obstacles and attracting them towards the goal but can suffer from local minima issues, oscillations, etc. There is considerable work on developing model-based methods based on velocity obstacles~\cite{vo} and their variants~\cite{rvo,orca}. VO~\cite{vo} presents an efficient method that defines a set of relative velocities between pairs of agents that cause collisions. At each time step, a velocity outside the VO is chosen for collision avoidance. RVO~\cite{rvo} extends the VO concept by assuming agents share equal responsibility for collision avoidance. In the ORCA algorithm~\cite{orca}, the RVO constraints are linearized to reduce the collision avoidance problem to that of linear programming. VO and their variants have shown good behavior in dynamic scenarios and are widely used. However, they generally consider simple velocity-controlled disk-shaped or elliptical~\cite{best2016real} agents, and constructing VO for arbitrarily shaped obstacles is non-trivial or results in very conservative behavior. In addition, planning a velocity at each time step makes generating smooth, acceleration-limited paths difficult. Furthermore, they can be overly conservative in environments in cluttered scenes~\cite{he2017efficient}. With ORCA, the linear constraints regard a half-space as invalid, and the linear program can be infeasible even for a small number of nearby obstacles~\cite{orca}. 

The VO concept was extended to a variety of agent dynamics, including double integrators~\cite{avo}, linear agents~\cite{lqr,lqg}, differential drive agents~\cite{orca_dd}, etc. However, these modifications either rely on augmenting the bounding geometry of the agents or linearizing the non-convex VO shape causing the formulation to be overly conservative. DWA~\cite{dwa} plans to avoid collisions in the velocity space. They are shown to be successful in low-speed scenarios but produce highly reactive behavior~\cite{brito_mpcc}. Buffered Voronoi cell (BVC)~\cite{bvc} presents an efficient method of computing collision-free trajectories by reasoning in the position space. Other techniques for collision avoidance evaluate trajectories against a cost function to optimize for a navigation plan between multiple agents based on a time-to-collision model~\cite{nh-ttc,cglr}. Inevitable collision states (ICS)~\cite{ics} computes a set of states that have no collision-free trajectories for an infinite time horizon. Though ICS provides a theoretical guarantee on collision avoidance, they are very conservative and could regard the entire workspace as forbidden. Trautman et al.~\cite{trautman} develop a probabilistic predictive model of cooperation and its importance for safe and efficient navigation in human crowds.

Model predictive control (MPC) has been used to generate smooth collision-free trajectories that show predictive navigation behavior. Cheng et al.~\cite{mpc_orca} employ ORCA constraints in an MPC framework, which reduces the velocity vibrations compared to ORCA. The feasible velocity set is defined by the ORCA constraints and can be infeasible due to ORCA's conservativeness. Brito et al.~\cite{brito_mpcc} present a model predictive contouring control (MPCC) frame that assumes a reference path and computes convex constraints on free space with predicted agent behaviors for generating local trajectories. Computing convex free space for arbitrary environments can be non-trivial and also conservative. Park et al.~\cite{park_mpepc} propose an MPEPC navigation framework that solves a finite horizon, unconstrained optimization to generate a smooth local trajectory.  %MOVE THESE DETAILS OF MPEC TO SECTION III(C)The MPEPC framework uses a probability term to evaluate whether a trajectory {\em{survives}} without collision. However, a distance-based probability can be conservative in some cases as it does not account for the direction of motion. In this paper, we define a new collision probability to reduce this conservative behavior. 
Optimizing for maximum progress at each timestep may not eventually lead the robot to its goal as the path can be obstructed by an obstacle like a piece of wall, and the robot gets deadlocked. The robot may need to move away or make limited progress toward the goal at certain timesteps to reach the goal eventually.

\subsection{Learning-based Navigation}
Recently, learning-based methods~\cite{cadrl,long} have been used for navigation in real-world dynamic scenes. In practice, they can handle sensor uncertainty in terms of better time-to-goal and success rate. DenseCAvoid~\cite{densecavoid} uses reinforcement learning and trajectory predictions to generate smooth, collision-free trajectories. Arpino et al.~\cite{rl_pednav} propose an RL-based method for robot navigation among pedestrians in real-life indoor environments. One major challenge is the lack of explainability and collision-free guarantees with such learning techniques. 

\subsection{Deadlock Resolution}
Deadlocks happen when the robot halts before reaching its goal and can be caused by local minimum decisions in the navigation problem due to a finite planning horizon. Frequently deadlocks are resolved using some heuristic rule. In BVC~\cite{bvc}, the robot detours along the edges of the Voronoi cell to resolve deadlocks. In V-RVO~\cite{vrvo}, the method proposed a simple communication-based strategy for deadlock resolution. In~\cite{wallfollowing}, the planner detects a deadlock and performs wall following to resolve deadlocks. In this paper, we define a terminal state cost function based on the time-to-goal and time-to-collision values, which shows deadlock-resolving behavior in our test scenarios.

% They plan in the velocity space and choose a collision-free velocity at each timestep based on the current observation of their surrounding. Consequently, the acceleration between consecutive timesteps can be arbitrary. Besides, VO computation for non-circular static obstacles could be non-trivial.

% In this paper, we consider a continuous decision-making process, where an agent with short-term predictive capability reasons makes an informed decision at each time step to navigate towards their goal safely. We utilize the idea of model predictive equilibrium point control (MPEPC) which is utilized to optimize over a parameterized trajectory space. We build upon a model predictive control architecture where trade-offs between progress toward the goal, quality of motion, and probability of collision along a trajectory are fully considered. 







