\section{Introduction}

% {\color{red}{
% \begin{itemize}
%     \item DEFINE WHAT IS THE PROBLEM OF ROBOT NAVIGATION IN CLUTTERED DYNAMIC SCENES. WHAT ARE THE METRICS OF CLUTTER, AND DYNAMISM? GIVE MOTIVATION 

%     \item YOUR GOALS ARE SAFETY AND DEADLOCK: DEFINE WHAT THEY MEAN (Collision Avoidance). WHY DO DEADLOCKS OCCUR? WHAT MAKES IT HARD

%     \item WHAT ARE THE PREVIOUS METHODS (Dynamic planning, Velocity Obstcles, optimization based, learning methods (e.g. DenseCAvoid)... WHY ARE THESE METHODS NOT GOOD ENOUGH

%     \item WHY DO  YOU USE AN OPTIMIZATION BASED PLANNER; HOW DOES MODEL PREDICTIVE CONTROL AND MORE SPECIFICALLY MPEPC  FAMILY OF METHODS (BETTER COST FUNCTION TO IMPROVE SAFETY AND SMOOTHNESS) COME IN PICTURE; WHAT ARE THEIR BENEFITS AND DRAWBACKS (IN TERMS OF SAFETY AND DEADLOCKS)

%     \item YOUr MODIFICATION IS A DIFFERENT COST FUNCTION WHICH RESULTS IN DEADLOCK FREE, LESS CONSERVATIVE IN DECISION MAKING AND MAINTAINS SAFETY

%     \item 1. HOW TO SHOW DEADLOCK FREE CLAIM FORMALLY? QUANTIFY THE SCENARIOS AND MAKE QUALITATIVE CLAIMS? USE LOCAL MINIMA ANALYSIS! GIVE PROPERTIES OF THE METRIC AND HOW IT LEADS TO LESS DEADLOCK BEHAVIORS

%     \item 2. YOUR COST FUNCTION PROVIDES SIMILAR SAFETY CRITERIA AS MPEPC

%     \item 3. HOW DO YOU SHOW LESS CONVERSATIVE BEHAVIOUR? PRESENT SOME PROPERTIES FORMALLY, AND ARGUE WHY THOSE PROPERTIES RESULTS IN LESS CONVERSATIVE

%     \item 4. DO YOU HAVE A MULTI-AGENT EXTENSION BASED ON YOUR NEW COST FUNCTION?

%     \item COMPARISONS: MPEPC, ORCA (HANDLING DYNAMIC CONSTRAINTS), NH-TTC (MULTIPLE ROBOTS), DYNAMIC WINDOWS METHOD?
% \end{itemize}
% }
% }

Robots are gaining widespread use in everyday applications in the form of autonomous ride-sharing vehicles, robot vacuums, home monitoring robots, delivery robots, and urban surveillance drones. A key problem in all these applications is for the robot to move between different locations by navigating safely around static and dynamic obstacles. 

In this paper, we address the problem of computing safe paths for one or more robots in cluttered dense scenes. These include indoor pedestrian-rich environments such as homes, public places, and shopping malls, which can be particularly challenging scenes to navigate due to their highly dynamic, cluttered, and uncertain nature. First, pedestrian motion is unpredictable, and second, the robots have imperfect knowledge of their surroundings. Moreover, the robot may operate in spatially constrained regions such as narrow corridors, doorways, and spaces with multiple obstructions such as furniture, large objects, or pedestrians in the scene. Despite these challenges, the navigation algorithm must steer the robot safely, smoothly, and cooperatively around pedestrians and other obstacles.

The problem of autonomous navigation in large static or dynamic scenes has been well-studied. The main challenge is to generate trajectories that progress towards the goal, remain collision-free (safe) around obstacles, and are smooth. Many geometric or model-based methods~\cite{vo,rvo,orca,bvc} have been proposed for complex dynamic scenes consisting of one or more robots. Some model-based approaches optimize between these potentially conflicting navigation objectives, which can result in local minima or deadlocks. Thus, the robot may remain collision-free but may be deadlocked (or frozen) and unable to reach its goal successfully, even when a safe trajectory to the goal exists. Another class of methods is based on model predictive control (MPC)~\cite{mpc_orca,brito_mpcc}, which 
%uses a model of the robot and 
optimizes over a finite horizon to generate smooth, collision-free trajectories by incorporating predictions of agent and obstacle future behaviors. Its variant includes model predictive equilibrium point control (MPEPC) navigation~\cite{park_mpepc}, which formulates local navigation as a continuous, unconstrained, finite-horizon optimization problem by augmenting MPC with equilibrium point control (EPC~\cite{epc}). The underlying navigation formulation considers non-holonomic (e.g., wheelchair) dynamics and can handle static and moving obstacles to generate safe and smooth trajectories in dynamic environments. However, due to conflicting performance objectives, these methods can also result in a deadlock in certain scenarios.

\begin{figure}
    \centering
    \frame{\includegraphics[width=0.4\linewidth]{Figures/title.png}}
    \quad
    \frame{\includegraphics[width=0.42\linewidth]{Figures/title2.png}}
    \caption{An illustrative scenario showing a robot (blue) navigating a space with 12 pedestrians (red disks) with our modified cost function in real-world scenarios captured using sensors. Black regions denote static obstacles/ (Left) The red traces show the trajectory followed by the pedestrians over the past few timesteps computed using DS-MPEPC, and the blue trace shows the actual path followed by the agent. (Right) The figure illustrates the evaluated trajectories in this scenario, which are represented in gray. The blue trajectories are the optimal trajectory at each timestep.}
    \label{fig:my_label}
\end{figure}

Recently, learning-based methods have been used for navigation in dynamic environments. They can handle noisy sensor data and can work well in some environments.
%improved performance in terms of higher success rate and lower time-to-goal compared to prior model-based methods~\cite{rvo,orca}. 
Despite these advantages, learning-based methods lack safety guarantees and are non-explainable. %In addition, learning-based methods suffer from sim-to-real gap. 
Therefore, model-based approaches, owing to their safety and explainability, are still desirable for real-life navigation applications, though we need better methods to navigate in challenging scenarios. 
%In this paper, we focus on improving model-based formulation to generate intelligent navigation behavior by reducing conservativeness and deadlock behavior in the planner by proposing novel cost formulations.

\subsection{Main Contributions}
In this paper, we present DS-MPEPC, an improved MPEPC-based ~\cite{park_mpepc} navigation algorithm. Our approach is designed to improve the performance in terms of reducing conservativeness and avoiding deadlocks (or to unfreeze), thereby resulting in improved navigation behavior. We present a modified trajectory cost formulation that improves the navigation in terms of reducing deadlocks, allowing agents to navigate a narrow passages, and can be used for multi-agent navigation in dynamic environments. The novel components of our work include:  

\begin{enumerate}
    \item A new collision probability formulation that captures the risk associated with a configuration state and the time available for the robot to avoid an impending collision. Our formulation is less conservative in terms of assigning a collision probability to a trajectory segment and results in improved navigation performance.
    %and is upper bounded by MPEPC's collision probability. 

    \item A novel terminal cost term based on expected time-to-goal value. The terminal cost term preferentially chooses safe trajectories that reduce deadlocking behavior.

    \item We prove that optimizing against the modified cost function maintains the safety conditions that prevent the robot in a collision state from moving (Lemma~\ref{lemma:3}).
\end{enumerate}

We evaluate the proposed formulation in a variety of simulated scenarios with both static, dynamic, and multi-agent cases. We also simulate the robot motion in an environment generated from real-world sensor data as in MPEPC~\cite{park_mpepc} and demonstrate the benefits. The overall approach is fast and works well on challenging scenarios.
%We compare our modified formulation with the MPEPC to highlight the improvements.


%They have largely been explored for simple agent dynamics like velocity-controlled vehicles, and their safety guarantees may not trivially extend to complex dynamics. Extension of the VO concept has been proposed for other agent dynamics~\cite{ORCA-DD} by enlarging the bounding geometry of the agents, which consequently results in a larger portion of the velocity space being invalid and resulting in conservative behavior. 

%andtrajectories evaulation the The red disk evaluated trajectories by the planner in a typical freezing scenario. The trajectories in gray indicate the various trajectory samples, and the trajectory in green indicates the trajectories with lower cost. Our cost formulations can select trajectories that avoid freezing behavior.