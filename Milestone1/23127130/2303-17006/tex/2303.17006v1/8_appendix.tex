\begin{table}[h]
\begin{tabular}[0.5\textwidth]{cc}
\hline
\textbf{Generation Type} &\textbf{Pearson's \textit{r}} \\
\hline
Reference Text  & -.69      \\
Greedy          & -.23      \\
\textit{p} = 0.3         & -.43   \\
\textit{p} = 0.5         & -.50      \\
\textit{p} = 0.6         & -.56  \\
\textit{p} = 0.8         & -.65     \\
\textit{p} = 0.9         & -.68      \\
\textit{k} = 10          & -.40   \\
\textit{k} = 20          & -.45      \\
\textit{k} = 50          & -.56     \\
\textit{k} = 100         & -.63     \\
\textit{k} = 200         & -.65     \\
\textit{k} = 500         & -.69     \\
Vanilla         & -.74     \\
\hline
\end{tabular}
\caption{Pearson's correlation coefficient (\textit{r}) \textbf{between UID score and average sentence surprisal} (all \textit{p} < 0.01)}
\label{tab:suprisal_uid}
\end{table}

\begin{figure}[h]
%\centering
\includegraphics[width=0.5\textwidth]{human_data_dist.png}
\caption{Frequency of responses (Yes/Somewhat/No) for each qualitative measure in our human annotated dataset.}
\label{fig:human_data_dist}
\end{figure}
 

\begin{table}[h]
\centering
\begin{tabular}{c|cc}
& \multicolumn{2}{c}{\textbf{Pearson's \textit{r}}} \\
\textbf{Quality} & \textbf{UID Score} & \textbf{Surprisal}   \\
\hline
Related     & .01  & \textbf{-.13$^{\ast}$}    \\
Furthering      & .03  & \textbf{-.10$^{\ast}$} \\
Interesting & -.04 & -.01 \\
\hline
\end{tabular}
\caption{Pearson's correlation coefficient (\textit{r}) of \textbf{UID score and surprisal with human judgments of qualitative metrics} ($^{\ast}$\textit{p}<0.01)}
\label{tab:judgments}
\end{table}
\begin{figure*}[h!]
 \centering
  \includegraphics[width=\textwidth]{surprisal.png}
  \caption{Histograms of \textbf{average sentence surprisal} for responses generated using different decoding settings and human-generated reference text (left-top).}
  \label{fig:surp}
\end{figure*}

\begin{table}[h]
%\centering
%\small
\resizebox{0.5\textwidth}{!}{
\begin{tabular}[0.5\textwidth]{cc|ccc}
& &   \multicolumn{3}{c}{\textbf{Pearson's \textit{r}}} \\
\textbf{Surprisal  interval} & \textbf{n}   & \textbf{Related}  & \textbf{Furthering}  & \textbf{Interesting}   \\
\hline
\\
(0.8,1.2)       & 24  & -.03     & -.04      & -.00    \\
(1.2,1.6)       & 64  & -.10     & -.16     & .08    \\
(1.6,2.0)       & 91  & .05     & .14    & .10   \\
(2.0,2.4)       & 109 & -.14    & -.08     & \textbf{-.27$^{\ast}$}\\
(2.4,2.8)       & 111 & -.12    & .05      & .09     \\
(2.8,3.2)       & 105 & -.02   & .06      & -.00    \\
(3.2,3.6)       & 99  & -.13   & .12       & .01  \\
(3.6,4.0)       & 66  & .02     & -.06    & .06     \\
(4.0,4.4)       & 42  & -.01    & -.00     & .06     \\
(4.4,4.8)       & 24  & .20    & .34     & .23    \\
(4.8,5.2)       & 12  & -.13      & -.37     & -.12     \\
(5.2,5.6)       & 13  & .60         & .83      & .76 \\
\hline
\end{tabular}}
\caption{Pearson's \textit{r} between \textbf{surprisal and human judgments} of qualitative measures for dialog responses bucketed by surprisal [Surprisal interval = the ranges of surprisal values used for bucketing responses, n = number of responses in each surprisal interval, $^{\ast}$p-value < .05]}
\label{tab:corr_human_surp}
\end{table}



\section{Human evaluation study details} \label{sec:mturk}
Raters were selected based on the criteria that they be located in the US, and had attempted a minimum of 500 HITS at an accepted work rate greater than 97\% on MTurk.
 We asked raters on MTurk to answer if a candidate response satisfied each of the qualitative measures (interesting, furthering and related) and gave them three response options: "Yes", "Somewhat" and "No". In a pilot study of $360$ responses, we also included a measure for fluency. All of the responses were rated ``Yes" by majority vote and we removed this measure from further analysis as all the generations in this study were fluent as indicated by the pilot study and from our observation. For correlation calculations, we assign integer score values to each of the three response options as $3$ for "Yes", $2$ for "Somewhat" and $1$ for "No". Thus, the higher the score, the better the response is rated. Following the pilot study, for 194 dialogue histories, we showed the raters 4 candidate dialogue responses (total of 776 dialogue responses) and collected ratings on all *3* measures from *3* raters per dialogue history. In all, we obtained a total of 776*3, i.e., 2328 total response-rating pairs. To calculate the score for each response along every measure, we take the mean of all ratings as the score. For cases where at least 2 out of 3 raters agree, we take majority vote  as the final score. This constituted (2018 out of 2328) 86.68\% of all the ratings collected. We show the overall distribution of qualitative scores for all the response-rating pairs in Figure \ref{fig:human_data_dist}. We verified the rater responses by checking if they were rating human-generated responses highly as those came from a trusted source (Persona-Chat). We also manually inspected a random subset of dialog history-candidate response sets and found the results to be in accordance with our intuitions.

\begin{figure*}
\centering
    \begin{subfigure}{\textwidth}
        \includegraphics[width=\textwidth]{mturk_1.png}
    \end{subfigure}
    \\
    \begin{subfigure}{\textwidth}
        \includegraphics[width=\textwidth]{mturk_2.png}
    \end{subfigure}
\caption{Screenshots of our MTurk study interface for collecting human judgments on 4 candidate responses per dialogue history, along 3 quality measures.}
\label{fig:mturk}
\end{figure*}




\begin{figure*}
 \begin{subfigure}{\textwidth}
        \includegraphics[width=\textwidth]{detailed_instructions.png}
        \caption{Detailed instructions that MTurk raters could expand at any time.}
    \end{subfigure}
    \\
    \\
    \begin{subfigure}{\textwidth}
        \includegraphics[width=\textwidth]{mturk_examples.png}
        \caption{Examples responses for each measure and rating category shown to MTurk raters.}
    \end{subfigure}
\caption{Instructions and examples from MTurk study.}
\label{fig:mturkex}
\end{figure*}    

\section{Numerical Example}\label{sec_examples}
Consider a second order SISO flat system of the form in \eqref{BINF}, with $v_k = -\sin(x_{1,k}) + x_{1,k}x_{2,k}^2 - x_{1,k}^3x_{2,k} + u_k.$ In this example, we compare the performance of three nonlinear controllers: (i) An exact linearizing and stabilizing controller designed using basis functions that include $v$ in their span \cite[Cor. 2]{DePersis22}, and two locally stabilizing controllers (ii and iii) designed using the following choice of basis functions\footnote{The method described in \cite[Cor. 2]{DePersis22} requires that the unknown map \eqref{eqn_expressionforv} is linear in $u$, which is why we use the basis functions \eqref{eqn_ex_basisfunctions}. Although the choice of the basis functions in \eqref{eqn_ex_basisfunctions} is different from that in \eqref{eqn_specificchoice}, one can easily see from the proof of Theorem \ref{thm_aprioriFL} that using inputs of the form \eqref{eqn_PEinputSISOflat} also guarantees collective PE of \eqref{eqn_ex_basisfunctions}.} which do not contain $v$ in their span \cite[Cor. 2 and Sec.~III.B]{DePersis22}
\begin{equation}
	\Theta(\xi_k,u_k) = \begin{bmatrix}
		u_k & \xi_k^\top & (\xi_k^2)^\top & (\xi_k^3)^\top
	\end{bmatrix}^{\hspace{-0.5mm}\top}.\label{eqn_ex_basisfunctions}
\end{equation}
For all three controllers, PE of the basis functions of order one is a necessary and sufficient condition for the feasibility of the convex program that is solved to obtain the control gains (cf. \cite[Cor. 2, Thm. 2, and Thm. 5]{DePersis22}). For controllers (i) and (ii), PE is enforced by sampling the input randomly. For controller (iii), PE is enforced \textit{a priori} using the results of Theorem~\ref{thm_aprioriFL}. In this case, we used a straightforward extension of \cite[Cor. 2]{DePersis22} such that collected data from multiple experiments (i.e., collective PE) can be used to design the controller.

Since the system is unstable, the input data (of length $N=21$) for controllers (i) and (ii) had to be sampled from the uniform distribution $U(-0.25,0.25)$, whereas using multiple experiments as in Theorem~\ref{thm_aprioriFL} allowed us to use inputs (each of length $N_j=3$) with larger magnitudes (sampled from $U(-1,1)$). In \cite{vanWaarde20}, a similar observation was made for linear systems. As a result, a larger quantitative level of PE was attained (cf. Remark \ref{remark_qPE} and Table~\ref{table_comparison}).

The performance of the closed-loop system (over $T=20$ time instants) was compared starting from the same initial conditions (randomly sampled from $U(-1,1)\times U(-1,1)$). Table~\ref{table_comparison} shows the average cumulative stabilization errors (defined as $\sum_{k=0}^{T-1}\frac{1}{T}|x_{i,k}|$, for $i=1,2,\,T=20$) for all three controllers over 100 experiments, excluding 5 (respectively 4) unstable experiments for controllers (ii) and (iii). Controller~(i) is the best performing one since it enforces exact nonlinearity cancellation. Controller (iii) is shown to outperform controller (ii), although the same basis functions \eqref{eqn_ex_basisfunctions} were used, potentially suggesting that the region of attraction of controller (iii) is larger compared to (ii). This can be attributed to the fact that larger levels of PE were attained using multiple experiments.


