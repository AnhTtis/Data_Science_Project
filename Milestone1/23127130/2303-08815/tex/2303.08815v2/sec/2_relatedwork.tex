\section{Related Work}
\label{sec:relatedwork}

\boldparagraph{Lane detection.}

Lane detection only considers predicting and evaluating the lane divider lines without spatial relation (merging and fork). 
Since most lane detection datasets only provide front-view images, previous lane detection methods~\cite{tabelini2021keep,wang2022keypoint,garnett20193d,lstr,guo2020gen,liu2022learning} were stuck in predicting lines with a small curvature in a limited horizontal FOV. BezierLaneNet~\cite{feng2022rethinking} uses a fully convolutional network to predict Bezier lanes defined with 4 Bezier control points. PersFormer~\cite{chen2022persformer} 
proposes a Transformer-based architecture for spatial transformation and unifies 2D and 3D lane detection. 

\boldparagraph{Online HD map construction.}
Online HD map construction can be seen as an advanced setting of lane detection, consisting of  lines and polygons with various semantics in the local 360$^\circ$ FOV  perception range of ego-vehicle.
With advanced 2D-to-BEV modules~\cite{Ma2022VisionCentricBP}, previous online HD map construction methods cast it into semantic segmentation task on the transformed BEV features~\cite{polarbev,cvt,bevformer,liu2022bevfusion,liu2022petrv2,lu2022ego3rt}. Building vectorized semantic HD map online achieves increasing interests nowadays~\cite{hdmapnet,maptr,vectormapnet,instagram,bemapnet,pivotnet}, HDMapNet~\cite{hdmapnet} follows a segmentation-then-vectorization paradigm.
To achieve end-to-end learning~\cite{detr,deformdetr,yolos}, VectorMapNet~\cite{vectormapnet} 
adopts a coarse-to-fine two-stage pipeline for vectorized HD map learning.
MapTR~\cite{maptr} proposes unified permutation-equivalent modeling to exploit the undirected nature of semantic HD map and designs a parallel end-to-end framework. BeMapNet~\cite{bemapnet} and PivotNet~\cite{pivotnet} propose a Bezier-based representation and pivot-based representation for modeling the map geometry. While the above works focus on the map elements without physical directions (\ie, lane divider, pedstrian crossing and road boundary) and lane graph topology, we aims to fill the gap in this work.


\boldparagraph{Road graph construction.}
There is a long history of extracting the road graph from remote sensor data (\eg, aerial imagery and satellite imagery). 
Many works~\cite{Mattyus_2017_ICCV, zhou2018d, batra2019improved,He2020Sat2GraphRG,buslaev2018fully}
frame the road graph as a pixel-wise segmentation problem  
and utilizes morphological post-processing methods to extract the road graph. RoadTracer~\cite{bastani2018roadtracer} uses an iterative search process to extract graph topology step by step. Some works~\cite{chu2019neural,Tan_2020_CVPR,xu2021icurb,li2019topological,mi2021hdmapgen} follow this sequential generation paradigm. Different from the above works, we focus on the online, ego-centric setting with vehicle-mounted sensors to produce more fine-grained lane-level graph.


\boldparagraph{Lane graph construction.}
Lane graph is traditionally constructed with an offline pipeline~\cite{centerlinedet,laneextract, buchner2023learning}. \cite{laneextract} proposes a mutli-step training pipeline to construct the lane graph of aerial images based on pixel-wise modeling. \cite{buchner2023learning} proposes a bottom-up approach to aggregate multiple local aerial lane graphs into a global consistent graph. CenterlineDet~\cite{centerlinedet} proposes a DETR-like decision-making transformer network to iteratively update the global lane graph with vehicle-mounted sensors.
Recently, STSU~\cite{stsu} shift the offline lane graph construction to the online, ego-centric setting with vehicle-mounted sensors. It models the lane graph as a set of disjoint pieces split by junction points and a set of connections among those pieces. A DETR-like Transformer decoder is proposed to detect those Beizer lane pieces and  a successive MLP head is used to predict inter-piece connectivity. Based on STSU, \cite{can2022topology} designs a network and utilizes minimal circle extracted by time-consuming offline processing to further supervise the network to produce lane graph.  Different from the above graph modelings, we regard the path as the primitive of the lane graph, and model the lane graph in a novel path-wise manner.
