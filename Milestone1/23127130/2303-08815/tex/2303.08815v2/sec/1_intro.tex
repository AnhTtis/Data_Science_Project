\section{Introduction}
\label{sec:intro}

The lane graph  contains detailed lane-level traffic information,
and serves to provide path-specific guidance for trajectory planning, \ie, an automated vehicle can trace a path from the lane graph as reliable planning prior.

Lane graph is traditionally constructed with an offline map generation pipeline.  
However, autonomous driving demands a high degree of freshness in lane topology. Hence, online lane graph construction with vehicle-mounted sensors (\eg, cameras and LiDAR) is of great application value.



An intuitive solution is to model the lane graph in a pixel-wise manner and adopt a segmentation-then-vectorization paradigm.
For example,  HDMapNet~\cite{hdmapnet} predicts  a  segmentation map and a direction map on dense bird's-eye-view (BEV) features.
It then extracts the skeleton from the coarse segmentation map with a morphological thinning algorithm and extracts the graph topology  by greedily tracing the single-pixel-width skeleton with the predicted direction map. 
Pixel-wise modeling incurs heuristic and time-consuming post-processing and often fails in complicated topology (see Fig.~\ref{fig:qualitative_comparison} row 3).

Lane graph is also modeled in a piece-wise manner~\cite{stsu}, which splits the lane graph into lane pieces at junction points (\ie, merging points and fork points) and predicts an inter-piece connectivity matrix.  Based on the connectivity, pieces are linked and merged into the lane graph. 
However, piece-wise modeling breaks down the continuity of  the lane.  Aligning the pieces is a challenging problem, especially at complicated road intersections.
And piece-wise modeling often breaks down the lane graph into fragmented short pieces which are hard for a neural network to learn to perceive (\eg, $V_1^{\text{piece}}$ in Fig.~\ref{fig:topo_model} (b)).



We argue that the path is the primitive of the lane graph.
Human drivers focus on and drive  along the continuous 
 and complete paths instead of considering lane pieces. 
Autonomous vehicles also require path-specific guidance from 
 lane graph for trajectory planning.
Continuous paths play an important role in indicating the driving flow, while pixel-wise and piece-wise modeling often fail to merge pixels and pieces into continuous paths (see Fig.~\ref{fig:qualitative_comparison}).

Motivated by this, we propose to model the lane graph in an alternative path-wise manner. 
We decouple the lane graph into a set of continuous paths with a proposed Graph2Path algorithm,  perform path detection through set prediction~\cite{detr, maptr}, 
and extract a fine-grained lane graph with a Path2Graph algorithm.  
Based on this path-wise modeling, we propose an online lane graph construction framework, termed LaneGAP, which  feeds onboard sensor data into an end-to-end network for path detection, and further transforms detected paths into a lane graph.


\begin{figure*}[thbp]
\centering
\includegraphics[width=\linewidth]{figures/topo_model.pdf}
% \vspace{-58pt}
\caption{\textbf{Modeling comparison.} \textbf{(a)} Pixel-wise modeling~\cite{hdmapnet} utilizes a predefined Graph2Pixel algorithm to rasterize the lane graph into a segmentation map and a direction map on dense BEV pixels, and heuristic Pixel2Graph post-processing is needed to recover the lane graph from the predicted segmentation map $V_{\text{pixel}}$ and direction map $D_{\text{pixel}}$ (direction map is not drawn here for simplicity). \textbf{(b)} Piece-wise modeling~\cite{stsu} utilizes a predefined Graph2Piece algorithm to  split the lane graph into a set of pieces and  the connectivity matrix among pieces, and then it merges the predicted pieces $\mathcal{V}_{\text{piece}}$ to the graph with the Piece2Graph algorithm based on predicted connectivity $E_{\text{piece}} $. \textbf{(c)} The proposed path-wise modeling translates the lane graph into continuous and complete paths with a simple predefined Graph2Path algorithm to traverse the graph. 
We perform path detection
and adopt a simple Path2Graph algorithm  to recover the lane graph.}
\label{fig:topo_model}
    % \vspace{-4pt}
\end{figure*}


We compare LaneGAP with the pixel-wise modeling method HDMapNet~\cite{hdmapnet} and piece-wise modeling methods STSU~\cite{stsu}, MapTR~\cite{maptr} under strictly fair conditions (encoder, model size, training schedule, \etc) on the challenging nuScenes~\cite{nuscenes} and Argoverse2~\cite{av2} datasets, which covers diverse graph topology and traffic conditions. With only camera input, LaneGAP achieves the best graph construction quality both quantitatively (see Tab.\ref{tab:modeling}) and qualitatively (see Fig.~\ref{fig:qualitative_comparison}), while running at the fastest inference speed. 
We further push forward the performance of lane graph construction, by introducing the multi-modality input. 
We believe  modeling the lane graph at the path level is  reasonable and promising. 
We hope LaneGAP can serve as a fundamental module of the self-driving system and boost the development of downstream motion planning.

Our contributions can be summarized as follows:
\begin{itemize}
    \item  
    We propose to model the lane graph in a novel path-wise manner,  which well preserves the continuity of the lane and encodes traffic information for planning.
    \item  Based on our path-wise modeling, we present an online lane graph construction method, termed LaneGAP. LaneGAP end-to-end learns the  path and constructs the lane graph via the designed Path2Graph algorithm.
    \item 
    We  qualitatively and quantitatively demonstrate the superiority of LaneGAP over pixel-based and piece-based methods. 
    LaneGAP can cope with diverse traffic conditions, especially for road intersections with complicated lane topology.
\end{itemize}




\begin{figure*}[thbp]
\centering
\includegraphics[width=0.9\linewidth]{figures/qualitative_comparison.pdf}
% \vspace{-58pt}
\caption{\textbf{Qualitative comparison of pixel-wise HDMapNet, piece-wise MapTR, and path-wise LaneGAP on the nuScenes and Argoverse2 val splits.} Given the 360$^\circ$ horizontal FOV images as input, the proposed path-wise modeling well preserves the continuity across different scenarios, especially for complicated lane topology with more than 4 junction points. The top 2 rows are from the nuScenes val split, and the bottom 2 rows are from the Argoverse2 val split. More visualizations are available in the supplementary material.}
\label{fig:qualitative_comparison}
\vspace{-4pt}
\end{figure*}