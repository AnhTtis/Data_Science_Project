% \begin{table*}[t]\centering
%     \begin{tabular}{|c|c||c|c|c||c|c|c|}
%     \hline
    
%     & & \multicolumn{3}{|c||}{\textbf{GraphSage}}  &  \multicolumn{3}{|c|}{\textbf{GAT}}\\ \hline
%     \textbf{Dataset} & \textbf{System} & \textbf{0\%} & \textbf{10\%} & \textbf{25\%} & \textbf{0\%} & \textbf{10\%} & \textbf{25\%} \\ \hline \hline
    
% \multirow{2}{*}{PR}& Quiver & 92.64 & 45.40 & 26.03 & 79.06 & 28.77 & 26.33 \\
% & \name & 19.68 & 18.60 & 19.24 & 20.89 & 18.63 & 17.79 \\ \hline
% \multirow{2}{*}{PA} & Quiver & 130.58 & 67.60 & 67.19 & 127.66 & 67.90 & 67.40 \\
% & \name & 35.56 & 28.53 & 25.47 & 50.41 & 34.16 & 29.91 \\ \hline
% \multirow{2}{*}{AM} & Quiver & 231.31 & 78.38 & 78.46 & 212.07 & 78.75 & 78.63 \\
%  & \name & 38.45 & 34.22 & 31.95 & 46.92 & 34.53 & 32.63 \\ \hline
% \end{tabular}

%     \caption{Total epoch time (in seconds) using the \texttt{NVLink} host as a function of the per-GPU cache size.}
%     %\Sandeep{Quiver sorts nodes by degree and stores them across GPU. I assume that between 40 - 100 percent , the hot nodes would be in the 40 percent and the remaining percentages dont add too much benefit. Also the node with 25 percent might act like a hot gpu and start to bottleneck everything,}
%     \label{tab:cache-NVLink}
    
% \end{table*}

\begin{figure*}[t]
\centering
  \begin{subfigure}[b]{\columnwidth}
    \centering
    \includegraphics[width=\columnwidth]{figures/quiver_cache_sage.pdf}
    \caption{GraphSage}
  \end{subfigure}
  \begin{subfigure}[b]{\columnwidth}
    \centering
    \includegraphics[width=\columnwidth]{figures/quiver_cache_gat.pdf}
        \caption{GAT}
  \end{subfigure}
    \caption{Epoch time (in seconds) on the \texttt{NVLink} host for different GPU cache sizes. Speedups are over DGL.}
    \label{fig:cache-nvlink}
\end{figure*}  