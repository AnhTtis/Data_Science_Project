% Todo: sync current layout 
% 1/2 side column is 50 lines.
% If 4 paragraphs, approximately 12 lines per paragraph. 
% with expected targets.
% Page Layout.
% 2 + 1 + 1 + 3 + 3 + 1 = 11 
% Intro 2 page.
% Background 1 page
% Motivation 1 page.
%     Challenge:
%         1 Figure/ Challenge.
% Design 3 page.
% Components 
%     Split :2  
%     Co-Operative: 1
%     Cache: 0.5
% ==== Split page. 
% Evaluation 3 pages (.5 page)
% Settings
% Main 
% Components
% Related .5 - 1 page



%%
%% The first command in your LaTeX source must be the \documentclass command.
\documentclass[letterpaper,twocolumn,10pt]{article}
\usepackage{usenix}

% Recommended, but optional, packages for figures and better typesetting:
\usepackage{multirow}
\usepackage{algorithm}
%\usepackage{algorithmic}
\usepackage{algpseudocode}
\newcommand{\ptitle}[1]{{\color{blue}\{#1\}}}
% \newcommand{\TODO}[1]{}
\usepackage{subcaption}
\usepackage{xspace}
\usepackage{tikz}
\usepackage{verbatim}
\usepackage{graphicx}
\usepackage{hyperref}
\hypersetup{
    colorlinks=true,
    linkcolor=blue,
    filecolor=magenta,      
    urlcolor=cyan,
    pdftitle={Overleaf Example},
    pdfpagemode=FullScreen,
    }




\newcommand{\name}{GSplit\xspace}

\newcommand{\Tname}{Split parallelism\xspace}
\newcommand{\TName}{Split Parallelism\xspace}
\newcommand{\tname}{split parallelism\xspace}
\newcommand{\Tadj}{Split-parallel\xspace}
\newcommand{\TAdj}{Split-Parallel\xspace}
\newcommand{\tadj}{split-parallel\xspace}

\newcommand{\mypar}[1]{\vspace*{0.05in}\noindent\textbf{#1.}}
\newcommand{\red}[1]{{\color{red} #1}}
\newcommand{\circled}[1]{\tikz[baseline=(char.base)]{\node[shape=circle,fill,inner sep=1pt] (char) {\textcolor{white}{\footnotesize #1}};}}


\newcommand{\Sandeep}[1]{{\color{violet} \{Sandeep: #1\}}}
\newcommand{\hui}[1]{{\color{blue}\bfseries\textit{ \{Hui: #1\}}}}
\newcommand{\ms}[1]{{\color{red}\bfseries\textit{ \{Marco: #1\}}}}
\newcommand{\TODO}[1]{{\color{blue}\bfseries\textit{ \{TODO: #1\}}}}

\usepackage{authblk}

\begin{document}
\date{}

% show page numbers
%\settopmatter{printfolios=true}
\title{\Large \bf \name: Scaling Graph Neural Network Training on Large Graphs via Split-Parallelism}

\author[1]{Sandeep Polisetty}
\author[1]{Juelin Liu}
\author[1]{Kobi Falus}
\author[3]{Yi Ren Fung} 
\author[2]{\\Seung-Hwan Lim}
\author[1]{Hui Guan}
\author[1]{Marco Serafini}

\affil[1]{University of Massachusetts, Amherst}
\affil[2]{Oak Ridge National Laboratory, Tennessee}
\affil[3]{University of Illinois, Urbana-Champaign}

\maketitle


\begin{abstract}
  Large-scale graphs with billions of edges are ubiquitous in many industries, science, and engineering fields such as recommendation systems, social graph analysis, knowledge base, material science, and biology. Graph neural networks (GNN), an emerging class of machine learning models, are increasingly adopted to learn on these graphs due to their superior performance in various graph analytics tasks. Mini-batch training is commonly adopted to train on large graphs, and data parallelism is the standard approach to scale mini-batch training to multiple GPUs.
  In this paper, we argue that several fundamental performance bottlenecks of GNN training systems have to do with inherent limitations of the data parallel approach.
  We then propose split parallelism, a novel parallel mini-batch training paradigm. 
  We implement split parallelism in a novel system called \name and show that it outperforms state-of-the-art systems such as DGL, Quiver, and PaGraph.
  %Combined with in-GPU data caching, split parallelism speeds up large-scale GNN training by moving computation to the data, which maximizes data access locality and scales peak memory utilization. 
\end{abstract}

\section{Introduction}
\label{sec:intro}
\section{Introduction}

The increasing complexity of source code poses a key challenge to the reliability of large-scale software systems. Software bugs in these systems can lead to safety issues~\cite{bug_safety} for users around the world as well as cause non-negligible financial losses~\cite{bug_loss}. As such, developers have to spend a large amount of time and effort on bug fixing. Consequently, \aprfull (\apr), designed to automatically generate patches to fix software bugs, has attracted wide attention from both academia and industry~\cite{long2016prophet, legoues2012genprog, long2015spr, lou2020can, tufano2018empstudy}. 


To achieve \apr, one popular approach is known as Generate-and-Validate (G\&V)~\cite{qi2015gv, ghanbari2019prapr, lou2020can, le2016hdrepair, legoues2012genprog, wen2018capgen, hua2018sketchfix, martinez2016astor, koyuncu2020fixminder, liu2019tbar, liu2019avatar}, which is typically based on the following pipeline: First, fault localization techniques~\cite{wong2016fl, abreu2007ochiai, zhang2013injecting, papadakis2015metallaxis, li2019deepfl, li2017transforming} are applied to determine the suspicious locations in programs where bugs are likely to exist. Then, the buggy locations are used by the \apr tools to generate a list of patches that replace buggy lines with correct lines. Afterward, each patch is validated against the original test suite to identify any \emph{plausible patches} (i.e., passing all tests in the test suite). Finally, to determine the \emph{correct patches}, developers examine the list of plausible patches to see if any of them can correctly fix the bug. 

Traditional \apr tools can mainly be categorized into heuristic-based~\cite{legoues2012genprog, le2016hdrepair, wen2018capgen}, constraint-based~\cite{mechtaev2016angelix, le2017s3, demacro2014nopol, long2015spr} and \template~\cite{ghanbari2019prapr, hua2018sketchfix, martinez2016astor, liu2019tbar, liu2019avatar}. Among these traditional tools, \template \apr tools~\cite{ghanbari2019prapr, liu2019tbar, benton2020effectiveness} have been able to achieve state-of-the-art results. \Template \apr tools typically leverage pre-defined templates (e.g., adding a nullness check) for bug fixing. However, since these fix templates are typically handcrafted, the number and types of bugs they are able to fix can be limited. 



To address the limitations of traditional \apr, researchers have proposed various \learning \apr tools~\cite{li2020dlfix, chen2018sequencer, jiang2021cure, lutellier2020coconut, zhu2021recoder, ye2022rewardrepair} based on the \nmtfull (\nmt) architecture~\cite{sutskever2014mt} where the input is the buggy code snippets and the goal is to translate the buggy code snippets into a fixed version. To accomplish this, \learning \apr tools require supervised training datasets with pairs of both buggy and fixed code snippets in order to learn how to perform this translation step. These training data are usually obtained by mining historical bug fixes using heuristics/keywords~\cite{dallmeier2007benchmark}, which can be imprecise for identifying bug-fixing commits; even the actual bug-fixing commits can include irrelevant code changes, leading to further pollution in the dataset~\cite{xia2022alpharepair}.
% 
Moreover, it can be hard for such \apr tools to generalize and fix bug types unseen during training. 



To better leverage recent advances in \plmfull{s} (\plm{s}), researchers~\cite{xia2022alpharepair, xia2023repairstudy, kolak2022patch, prenner2021codexws} have directly applied \plm{s} to generate patches without bug-fixing datasets. These \llm-based \apr tools work by either directly generating a complete code function~\cite{prenner2021codexws, xia2023repairstudy} or predict/infill the correct code snippet given its surrounding context~\cite{xia2022alpharepair, xia2023repairstudy}. By directly using \llm{s} that are pre-trained on billions of open-source code snippets, \llm-based \apr tools can achieve state-of-the-art performance on many repair datasets~\cite{xia2022alpharepair}. 


% 
%
%

Traditional \apr tools have long used the insight of the \emph{plastic surgery hypothesis}~\cite{barr2014plastic} where it states that the code ingredients to fix a bug already exist within the same project. Traditional \apr tools have manually designed pattern-~\cite{ghanbari2019prapr, saha2017elixir} or heuristic-based~\cite{jiang2018simfix, legoues2012genprog} approaches to finding and using such relevant code ingredients to generate fixes for bugs. However, the plastic surgery hypothesis has been largely ignored in \llm-based \apr. In fact, \llm provides a unique opportunity to fully automate the plastic surgery hypothesis idea via fine-tuning (learning project-specific information via model updates from the buggy project) and prompting (directly providing relevant code ingredients to the model), and make it directly applicable to different languages (since the \llm{s} are typically multi-lingual).%
Moreover, despite the intensive manual efforts involved, traditional \apr tools still cannot fully leverage project-specific information due to large search space for leveraging/composing existing code ingredients. In contrast, the project-specific information can effectively leveraged by \llm{s} due to their power in code understanding/vectorization, e.g., even partial/imprecise information may still guide \llm{s} in correct patch generation!
 To this end, we ask the question: \emph{How useful is the plastic surgery hypothesis in the era of \plm{s}}?








\mypara{Our Work.} To answer the question, we present \ourtech{\xspace} -- a \llm-based approach that automatically utilizes the plastic surgery hypothesis by systematically combining multiple fine-tuning and prompting strategies for \apr. \ourtech fine-tunes \plm{s} using two novel domain-specific training strategies: \textbf{\epfinetune} -- we fine-tune using the original buggy project by aggressively masking out a high percentage of tokens, which allows \plm to learn project-specific code tokens and programming styles; and \textbf{\rofinetune} -- which only masks out a single continuous code sequence per training sample, allowing the model to get used to the final \csapr task of predicting a single continuous code sequence. Furthermore, we directly leverage the ability for \plm{s} to understand natural language instructions and introduce a novel prompting strategy, \textbf{\idprompting}, which uses information retrieval and static analysis to obtain a list of relevant identifiers for the buggy lines. While such relevant identifiers are critical for fixing some difficult bugs, they may not be seen by the \llm during inference due to limited context window size. Through the use of prompting, we directly tell the model to use these extracted identifiers (relevant code ingredients) to generate the correct code. Finally, to perform repair, we combine all four model variants (including the base model, both fine-tuned models and the base model with prompting) for the final repair.





While our insight of leveraging the plastic surgery hypothesis for \llm-based \apr is generalizable across different types of \plm{s}, to implement \ourtech, we choose a recent \plm{\xspace}, \ctfive~\cite{wang2021codet5}, which is pre-trained on millions of open-source code snippets. \ctfive is an encoder-decoder model trained using \mspfull (\msp) objective where a percentage of tokens are masked out and each continuous masked token sequence is referred to as a masked span. Also, although we only extract relevant identifiers from the current buggy project (since this paper focuses on the plastic surgery hypothesis), our work can be easily extended to obtain other code information (such as relevant statements or functions) from other sources, such as  the massive pre-training corpora~\cite{husain2020codesearchnet} or historical bug-fixing datasets~\cite{jiang2019infer}, which can provide more coding knowledge for \llm{s}. Besides, although we mainly focus on using traditional string comparison algorithms for information retrieval in this paper, these techniques can be easily replaced by other frequency-based retrieval~\cite{robertson2009probabilistic} and neural search (or embedding-based search)~\cite{reimers2019sentence}.
  In summary, this paper makes the following contributions:


%


\begin{itemize}[noitemsep, leftmargin=*, topsep=0pt]
    \item \textbf{Dimension.} This paper is the first to revisit the important plastic surgery hypothesis in the era of \llm{s}. It opens up a new dimension for \llm-based \apr to incorporate previously neglected information from the buggy project itself to boost \apr performance. Furthermore, it demonstrates the promising future of retrieval-based prompting for modern \llm-based \apr.
    \item \textbf{Implementation.} We implement \ourtech based on the recent \ctfive model. We augment the model using two novel fine-tuning strategies: \epfinetune and \rofinetune, along with a novel prompting strategy based on information retrieval and static analysis: \idprompting. We combine the patches generated by all four models together and perform patch ranking to speed up \apr.% 
    \item \textbf{Evaluation Study.} We conduct an extensive evaluation against state-of-the-art \apr tools. On the widely studied \dfj 1.2 and 2.0 datasets~\cite{just2014dfj}, \ourtech is able to achieve the new state-of-the-art results of 89 and 44 correct bug fixes (15 and 8 more than best baseline) respectively.  Furthermore, we perform a broad ablation study to justify our design. \ourtech demonstrates for the first time that the plastic surgery hypothesis can substantially boost \llm-based \apr and advance state-of-the-art \apr, while being fully automated and general. Moreover, even partial/imprecise code ingredients may still effectively guide \llm{s} for \apr!
\end{itemize}




\section{Background and Motivations}
\section{Background on Network Calculus}
\label{sec: background}


\begin{figure*}[tbh]
\centering
\begin{subfigure}[b]{0.3\textwidth}
    \centering
    \includegraphics[width=\linewidth]{images/in-out.png}
    \caption{Arrival and departure data and their relation with delay $d(t)$ and backlog $b(t)$. For a FIFO system, the delay is the horizontal distance between $R(t)$ and $R^*(t)$ but some other multiplexing techniques may shift the data to a later priority, causing a longer delay.}
    \label{fig: data in-out}
\end{subfigure}
\hfill
\begin{subfigure}[b]{0.35\textwidth}
    \centering
    \includegraphics[width=\linewidth]{images/arrival-service.png}
    \caption{Characteristics of an arrival curve and a service curve. From any point of observation, the arriving data never exceeds its arrival curve; the departure data is also never less than the service curve with respect to the data arrival.}
    \label{fig: arrival-service curves}
\end{subfigure}
\hfill
\begin{subfigure}[b]{0.33\textwidth}
    \centering
    \includegraphics[width=\linewidth]{images/bound.png}
    \caption{Delay and backlog bounds of a system. Backlog is the maximum vertical distance between $\alpha(t)$ and $\beta(t)$; FIFO delay is their maximum horizontal distance; but for arbitrary multiplexing, the delay guarantee is when the system clears its buffer, thus it's the intersection of $\alpha(t)$ and $\beta(t)$.}
    \label{fig: system bounds}
\end{subfigure}
\caption{Network calculus framework. We let $R(t)$ and $R^*(t)$ be the arrival and departure data flow of a system; $\alpha(t)$ be the piecewise linear concave arrival curve and $\beta(t)$ be the piecewise linear convex service curve of a system.}
% \hossein{Better to show piece-wise linear concave arrival curve and piece-wise linear convex service curve instead of token-bucket and rate-latency.}}
\end{figure*}

We recall some of the network calculus essentials for a better understanding of the framework used in Saihu. In the following context, we use the following notation: $\mbb{R}^+$ is the set of non-negative real numbers; $[x]_+$ denotes $\max(0, x)$

The data flow is by convention modeled as a left-continuous wide-sense increasing function $R(t): \mbb{R}^+ \mapsto \mbb{R}^+$ with respect to time $t$~\cite{ncbook2001leboudec}. 

A system $\mcal{S}$ receives arrival data described as a cumulative function $R(t)$ and delivers departure data as another cumulative function $R^*(t)$. Figure~\ref{fig: data in-out} illustrates such a system $\mcal{S}$. The benefit of representing a system like this is that we can observe system backlog and delay with such a model. 

\begin{definition}[Backlog and Delay~\cite{ncbook2001leboudec}]
    The backlog of a system at time~$t$ is
    \begin{equation}
        b(t) = R(t) - R^*(t)
    \end{equation}
    
    The virtual delay of a FIFO system at time $t$ is
    \begin{equation}
        d_{FIFO}(t) = \inf \lbp \tau \geq 0 : R(t) \leq R^*(t+\tau) \rbp
    \end{equation}
\end{definition}



The backlog of a system can be viewed as the vertical distance between $R$ and $R^*$. The FIFO (\textit{First-in First-out}) delay is the horizontal distance between $R$ and $R^*$. One may obtain other delay values if the multiplexing technique is not FIFO.

% \begin{figure}
%     \centering
%     \includegraphics[width=0.9\linewidth]{images/in-out.png}
%     \caption{In/out data flow; delay and backlog}
%     \label{fig: data in-out}
% \end{figure}

Since we are interested in the system guarantee instead of a single instance of data flow, we would like to have general bounds to the arrival and departure data flows. Therefore, we define \textit{arrival curve} and \textit{service curve} as the bounds of arrival and departure data flows.

\begin{definition}[Arrival Curve~\cite{ncbook2001leboudec}]
    Given a wide-sense increasing function $\alpha: \mbb{R}^+ \mapsto \mbb{R}^+$, we say that a flow $R(t)$ is $\alpha$-constrained if and only if for all $s \leq t$:
    \begin{equation}
        R(t) - R(s) \leq \alpha(t-s)
    \end{equation}
    We say $R(t)$ has $\alpha$ as an arrival curve.
\end{definition}

\begin{definition}[Service Curve~\cite{ncbook2001leboudec}]
    Given a wide-sense increasing function $\beta: \mbb{R}^+ \mapsto \mbb{R}^+$ and $\beta(0) = 0$. A system $\mcal{S}$ having $R(t)$ and $R^*(t)$ as its arrival and departure flows. We say $\mcal{S}$ offers a service curve $\beta$ if and only if
    \begin{equation}
        R^*(t) \geq (R \otimes \beta)(t) =: \inf_{s \leq t} \lbp R(s) + \beta(t-s) \rbp
    \end{equation}
    where $\otimes$ denotes the min-plus convolution
\end{definition}

Figure~\ref{fig: arrival-service curves} illustrates the arrival and service curves. Any segment of arrival flow $R(t)$ is constrained by arrival curve $\alpha$ and the output curve $R^*(t)$ is always no less than the curve $R\otimes\beta$. As a result, an arrival curve upper bounds the incoming traffic, and a service curve lower bounds the outgoing traffic.

% \begin{figure}
%     \centering
%     \includegraphics[width=\linewidth]{images/arrival-service.png}
%     \caption{Arrival/Service curve}
%     \label{fig: arrival-service curves}
% \end{figure}

We consider 2 special types of curves throughout this paper, \textit{token-bucket} (or sometimes called \textit{leaky-bucket}) curve and \textit{rate-Latency} curve.

\begin{definition}[Token-bucket and Rate-latency~\cite{ncbook2001leboudec}]
    A token-bucket curve $\gamma_{r,b}$ with arrival rate $r$ and burst $b$ is defined as
    \begin{equation}
        \gamma_{r,b}(t) = b + rt
    \end{equation}

    A rate-latency curve $\beta_{R,T}$ with service rate $R$ and latency $T$ is defined as
    \begin{equation}
        \beta_{R,T}(t) = R \lb t - T \rb_+
    \end{equation}
\end{definition}

A token-bucket curve is determined by a burst $b$ and an arrival rate~$r$. Burst represents the maximum possible data volume that can arrive simultaneously, and arrival rate represents the maximum long-term data rate~\cite{bouillard2022tradeoff}.
A rate-latency curve is determined by a latency~$T$ and a service rate~$R$. Latency represents the time a server needs before starting to process the incoming data, and service rate represents the minimum rate to process data after the initial latency.

With the help of arrival and service curves, we can derive delay and backlog bounds for a system $\mcal{S}$ illustrated in Figure~\ref{fig: system bounds}. Suppose a system $\mcal{S}$ has arrival curve $\alpha$ and service curve~$\beta$, its worst-case backlog $b^*$ is the maximum vertical distance between~$\alpha$ and~$\beta$. Similarly, depending on the multiplexing technique applied to the system, its worst-case delay bound $d^*$ is the maximum horizontal distance between $\alpha$ and $\beta$ if $\mcal{S}$ is a FIFO system. If we don't have any information about its multiplexing technique, referred to as arbitrary multiplexing, the best we can say is that when $\alpha$ and $\beta$ intersect each other, where all data has been delivered out of the system. Consequently, the worst-case delay bound for arbitrary multiplexing is the time required for $\mcal{S}$ to clear its buffer.

% \begin{figure}
%     \centering
%     \includegraphics[width=\linewidth]{images/bound.png}
%     \caption{System delay/backlog bounds}
%     \label{fig: system bounds}
% \end{figure}

While a service curve captures the slowest possible output speed of a system, a link's transmission capacity limits the speed as well. Hence, we model this phenomenon using a \textit{greedy shaper} with a sub-additive function $\sigma: \mbb{R}^+ \mapsto \mbb{R}^+$ concatenated with a server. We consider a concatenation as shown in Figure \ref{fig: system}. By convention we assume $\sigma(0) = 0$ and $\beta(t) \leq \sigma(t), \forall t \in \mbb{R}^+$, meaning that the buffer is cleared at the beginning and the service never exceed its physical limitation. With the above definition, such greedy shaper conserves the service provided by the system due to theorem \ref{thm: shaping}.

\begin{figure}[thb]
    \centering
    \includegraphics[width=0.7\linewidth]{images/system.png}
    \caption{Shaping of departure data. A flow that has an arrival curve $\alpha$ feeds into a server with an arrival data flow $R(t)$. The server having service curve $\beta$ takes $R(t)$ and gives a departure data flow $R^*(t)$ to a shaper with shaping function $\sigma$. The shaper takes $R^*(t)$ and shape the data flow as another departure $D(t)$.}
    \label{fig: system}
\end{figure}


\begin{theorem}[Shaping conserves service \cite{ncbook2001leboudec}]
\label{thm: shaping}
Following the system shown in Figure \ref{fig: system}, we have
\begin{equation}
     D = R^* \otimes \sigma \geq \lp R \otimes \beta \rp \otimes \sigma = R \otimes \lp \beta \otimes \sigma \rp = R \otimes \beta
\end{equation}
\end{theorem}

In the following context, we model the shaping function $\sigma$ as a token-bucket curve $\gamma_{C,L}$ with transmission capacity $C$ and the packet size $L$ to capture the link capacity and packetization~\cite{bouillard2022tradeoff}.

\section{Threat Model and Advantages of Our Hardware-based Adversarial Detector} \label{sec: motivation}
\ry{In this part, I want to highlight the comparison between hardware and software attacks}
%Normally, software-based adversarial detectors are easier to implement, cheaper to develop and more well-studied than those based on hardware computational signals.
% We would like to stress that our goal for investigating hardware-based adversarial detectors is not to achieve better performance in detection than the conventional white-box software based methods.  
\subsection{Threat Model} \label{sec: threat model}
\ry{This section is threat model: attack is `white-box', detector is `black-box'}
The victim is a DNN classifier, which is pre-trained with a public dataset. The testing dataset may be kept private.
We assume the strongest `white-box' attack model, where the attacker has full knowledge of the victim model and training dataset in order to generate adversarial samples with minimum perturbations. 
On the contrary, the detection system assumes the most limited scenario, under a `black-box' view of the victim, without access to the victim's inputs, parameters, and intermediate outputs or execution details. 
The only information available to the detector to distinguish adversarial samples is the EM side-channel measurement and the victim model's prediction class.
For training the adversarial detector with EM traces, a public benign dataset is used. 

\if false 
\ry{In this part, we discuss more settings of the detector especially the data used in two phases.}
In general, the detecting process can be summed up into two phases, training phase and detecting phase.
To begin with, we train an Out-of-Distribution(OOD) detector on a public benign dataset of the same classification task, which should be distinct from the victim's training dataset.
For each query, the detector will obtain the classification result and an EM trace along with the model execution to fit its EM classifiers and anomaly detectors.  
During the detection phase, the victim model is in operation and under attack when the pre-trained detector decides whether the current input is adversarial or not, only based on the victim model output and its EM trace.
\fi 

\subsection{Advantages}
Compared to software-based adversarial detection methods, our hardware-based detector, EMShepherd, has three distinct advantages: privacy-preserving, portability, and robustness.

\begin{itemize}[leftmargin=*]
    \item \ry{Add a new motivation here. The motivation is that using \name can help the user protect their privacy.} 
    \name protects the DNN model user's data privacy as it is agnostic to the model's inputs, which instead are always required by prior reconstruction-based detection methods~\cite{meng2017magnet, yang2022you}. 
    %Most model users are benign whose inputs may be sensitive and should not be shared with \textit{third-party detectors}. 
    The sensitive inputs should not be shared with \textit{third-party detectors}. 
    Our design only requires the output class labels and the EM signals, which are passively leaked to common acquisition equipment. 
    %    Our design is suitable for such cases as it only requires the EM signals and the inference outputs during the model execution. Generally speaking, EM signals and labels have less private information leakage.
    \item \ry{The second motivation is still related to privacy. This time we consider model privacy when the model structure or parameters should be kept private.}
   \name also protects the model confidentiality.  No model information, including %Using hardware-based detectors can prevent the third-party defender from accessing some confidential model information such as  
   hyper-parameters, parameters, and logits, is needed, in stark contrast to the previous software-based detection methods~\cite{ma2019nic,feinman2017detecting}.
    %Our \name only acquires the EM traces during model inference in a passive and noninvasive manner, 
    The EM data processing and the adversarial detector training process are both victim model-agnostic. 
    Therefore, our method has more general usage, applicable to closed-source DNN applications, which are pervasive in edge devices where the user only queries the models for the final prediction output. 
    \item \ry{The third motivation is portability.}  
    Owing to the model-agnostic feature, EMShepherd can be easily ported for wide-range hardware devices with different DNN implementations for diverse applications. It can be used as a `plug and play' (PnP) device, aside from the target system, to work automatically without user intervention or contact with the victim system. 
    \item \ry{The last motivation is about adaptive attacks, we should propose that EM signal is hard to imitate, so it is hard for adaptive attacks to generate sample fraud both detector and victim.} 
    Adaptive attack~\cite{adaptive} is a threat to most software defense methods where the attacker adjusts the adversarial perturbations to mislead both the victim models and defense systems.
   %  The hardware-based detection method can provide a double protection on top of most software defense methods such as adversarial training.
   %  Although the adptive adversarial example fools the robust model, its computation patterns during the DNN model execution are still well kept in the EM traces and our EMShepherd framework still works well for detecting the new type of adversarial examples.  
   %  Meanwhile, due to the high complexity of EM signals and non-explicit dependency of the EM signals on computations, it is extremely hard to have an adaptive attack on our detection method, i.e., adversarial examples whose EM signals are deliberately controlled to evade the EM-based detector.
   However, due to the high complexity and non-explicit dependency of the EM signals on computations and data, 
   it is extremely hard to have an adaptive attack on our detection method, 
   i.e., adversarial examples whose EM signals are deliberately controlled to evade the EM-based detector. 
\end{itemize}






\section{Split-Parallel Training} 
\section{Applications}
\label{sec:apps}
To demonstrate the wide range of usagages of our model, we implement a series of applications:
\begin{enumerate}
	\item Incremental surface \& color reconstruction
	\item 3D saliency detection
	\item Open vocabulary scene understanding
	\item Surface infrared field
	\item 3D style transfer
\end{enumerate}
Originating from our motivation in inspection and service robotics, we implement 1) Incremental surface \& color reconstruction for visualization of robot surroundings.
For robot exploration, we implement 2) 3D saliency detection to indicate the salient regions in maps.
For recovering object-level semantic information in environments, we implement 3) open vocabulary scene understanding to yield the regions containing the objects..
Furthermore, to demonstrate the flexibility, we implement 4) surface infrared fields and 5) 3D style transfer for artistic purposes. 

In~\cref{fig:latent_diff}, we classify those 3 applications into 3 categories: (a) directly obtaining the properties from sensor observation, such as application 1) and 4). (b) processing on sensor data and predict properties, such as application 2), 5). (c) extending (b) to operating beyond latent features, such as application 3).
%Thus, in the following, we discuss about those categories of applications.
% we mainly describe the application 1) (\cref{sec:incremental_reconstruction}) and 3) (\cref{sec:openvoc}).

Application 1) and 4) are in the first one category. Thus, we mainly describe 1) incremental surface \& color reconstruction (\cref{sec:incremental_reconstruction}), while for 4) we can easily exchange color with infrared.
%
For the second with 2) and 5) in~\cref{sec:fabircated_prop}, we mainly describe the usage of fabricated properties.
As the mapping part is redundant to previous category, it will not be detailed.
%
The third category is the application 3) that maps a LIM for high dimensional latent fields.
We demonstrate that this application provides a flexible inference in \cref{sec:openvoc}.


%Afterwards, we evaluate application 1) and 3) in~\cref{sec:exp} and extensively show demonstration for all application in~\cref{sec:exp:extensive_app}.

%\documentclass[a4paper]{amsart}%[a4paper]
%%%%%% GENERAL MATH COMMANDS
% Reals
\newcommand{\R}{{\mathbb R}}
% Integers
\newcommand{\Z}{{\mathbb Z}}
% Naturals
\newcommand{\N}{{\mathbb N}}
% Expectation
\DeclareMathOperator*{\E}{\mathbb{E}}
% ^th notation
\newcommand{\tth}{^{\text{th}}}
% Small dots for integer range [a .. b]
\newcommand{\sdots}{\,..\,}
% Vectorized version of matrix
\newcommand{\matvec}{\mbox{vec}}

% := sign
\newcommand{\defeq}{\vcentcolon=}
% Zero function
\newcommand{\zf}{\mathbf{0}}
% Vector of ones
\newcommand{\ones}{\mathbf{1}}

% Argmin and argmax definitions
\DeclareMathOperator*{\argmax}{arg\,max}
\DeclareMathOperator*{\argmin}{arg\,min}


%%%%% PROBLEM STATEMENT NOTATION 
% \newcommandtwoopt{\St}[2][t][]{{S_{#1}^{#2}}} % State
\newcommand{\task}[1][i]{{\mathcal{T}_{#1}}} % Task, optionally takes index
\newcommand{\tasks}{\{ \task \}_{i=1}^N}
\newcommand{\losst}[1][i]{{l_{#1}}}
\newcommand{\lossv}[1][i]{{l_{#1}^{\textrm{val}}}}
\newcommand{\tasktarget}{{\mathcal{T}_{\textrm{target}}}}
\newcommand{\lossttarget}{l_{\textrm{target}}}
\newcommand{\lossvtarget}{l_{\textrm{target}}^{\textrm{val}}}
\newcommand{\lossttargetit}{l_{\textrm{target}}^{(k)}}
\newcommand{\losstotal}{l^{\textrm{total}}}
\newcommand{\lossopt}{l^*}

\newcommand{\thetait}[2]{\theta_{#1}^{(#2)}}
\newcommand{\phit}[1]{\phi^{(#1)}}
\newcommand{\hist}[2]{S_{#1}^{(#2)}}
\newcommand{\grad}[2]{G_{#1}^{(#2)}}

\newcommand{\Alg}{\textup{\textbf{Opt}}}
\newcommand{\MetaAlg}{\textup{\textbf{MetaOpt}}}

%%%%% Theorems
\newtheoremstyle{mytheoremstyle} % name
    {\topsep}                    % Space above
    {\topsep}                    % Space below
    {\itshape}                   % Body font
    {}                           % Indent amount
    {\scshape}                   % Theorem head font
    {.}                          % Punctuation after theorem head
    {.5em}                       % Space after theorem head
    {}  % Theorem head spec (can be left empty, meaning ‘normal’)
\theoremstyle{mytheoremstyle}
\theoremstyle{plain}
\newtheorem{theorem}{Theorem}
\newtheorem{proposition}{Proposition}
\newtheorem{assumption}{Assumption}
\newtheorem{definition}{Definition}
\newtheorem{lemma}{Lemma}
\theoremstyle{remark}
\newtheorem{remark}{Remark}

%\begin{document}
%\section{Bipartite case and connection to known structures}



In this section we study a class of structures that can be defined on bipartite $d$-maps, when $d=2b$ is an even integer. We will call such structures \emph{$b$-bipartite-grand-Schnyder structures}, or \emph{$b$-BGS structures} for short. These structures 
%OB removed:are greatly reminiscent of the $2b$-GS structures, and in fact they 
can be identified with a subclass of $2b$-GS structures called \emph{even} (even $2b$-GS structures form a proper, non-empty, subclass of $2b$-GS structures for bipartite $2b$-maps). As we will see, specific classes of $b$-BGS structures can be identified with previously known structures.
%For this subclass definitions take a simplified form, and We emphasize that this is indeed a proper subclass: after providing the precise definitions, it will be clear that not every $2b$-GS structure on a bipartite $2b$-map corresponds to a $b$-BGS structure. 
%The importance of this subclass lies in the fact that for specific $b$ and bipartite $2b$-maps, $b$-BGS structures can be identified with previously known structures.

We start this section by giving the four incarnations of $b$-BGS structures and then state their equivalence. 
%This equivalence can be thought of as a specialization of the general bijections, which are stated in Section~\ref{sec:statements}. 
In the second part we will state the existence condition for $b$-BGS structures. 
%OB removed: while assuming the existence theorem~\ref{thm:main} for (ordinary) GS structures. 
In the third part we explain the connection between $b$-BGS structures and previously known structures.
%for specific $b$ and bipartite $2b$-maps to previously known structures.

\subsection{Incarnations of bipartite grand-Schnyder structures}\label{sec:incar_bipartite}\hfill\\
In this section, we fix an integer $b\geq 2$ and a bipartite $2b$-map $G$. 
We also fix the bicoloring of the vertices of $G$ in black and white by requiring the outer vertex $v_1$ to be black (and hence the outer vertex $v_i$ is black if and only if $i$ is odd). 

We now define 4 incarnations of $b$-BGS structures. They are represented in Figure~\ref{fig:bipartite_compiled2}. Our first incarnation is in terms of corner labelings.
\fig{width=\linewidth}{bipartite_compiled2}{Four incarnations of a 3-bipartite-grand-Schnyder structure.}

\begin{definition}\label{def:BGS-labeling}
A \emph{$b$-bipartite-grand-Schnyder corner labeling}, or \emph{$b$-BGS labeling}, of $G$ is a $2b$-GS labeling of $G$ such that the corners incident to black vertices have odd labels, while the corners incident to white vertices have even labels.
\end{definition}

%% \begin{remark}\label{reduced_labeling}
%% Observe that the parity condition implies we can replace each label $i \in \{1,...,2b\}$ by $\lfloor (i-1)/2 \rfloor$ with no loss of information. The result of this operation was studied on some special subclasses of adapted maps \cite{OB-EF:Schnyder, Barriere-Huemer:4-Labelings-quadrangulation, FeHuKa}. See subsection~\ref{sec:bipartite_angulations} for more details.
%% \end{remark}

The parity condition is equivalent to requiring the \emph{label jumps} (as defined in Section~\ref{sec:incarnations}) between consecutive corners in clockwise order around a vertex to be even, and the label jumps between consecutive corners in clockwise order around an inner face to be odd. This observation underlies the other incarnations of BGS structures.

First, recall the mapping $\Phi$ defined in Section~\ref{sec:statements} between corner labelings and marked orientations. Observe that by specializing $\Phi$, the set of $b$-BGS labelings is in bijection with the subclass of $2b$-GS marked orientations of $G$ such that the weight of each inner arc is even, and the number of marks in each inner corner is even. Let us call this subclass the \emph{even $2b$-GS marked orientations} of $G$. For an even $2b$-GS marked orientation, we can divide the weight of each arc and the number of marks in each corner by 2 without loss of information. This leads to the following definition.

\begin{definition}\label{def:BGS-marked}
A \emph{$b$-bipartite-grand-Schnyder marked orientation}, or \emph{$b$-BGS marked orientation}, of $G$ is a weighted orientation of $G$ together with a corner marking satisfying the following conditions. 
\begin{itemize}
  %\item[(BM0)] The weight of outer arcs are 0. For any inner arc $a$ whose origin is an outer vertex $v_i$, the weight of $a$ is equal to the number of marks in the corner at $v_i$ on the left of $a$. the weight of $a$ and the number of marks in the corner of $v_i$ on the left of $a$ are both equal to $b-\text{deg}(f)/2$, where $f$ is the face on the left of $a$.
  \item[(BM0)] The weight of every outer arc is 0. For any inner arc $a$ whose initial vertex is an outer vertex $v_i$, the weight of $a$ and the number of marks in the corner of $v_i$ on the left of $a$ are both equal to $b-\mathrm{deg}(f)/2$, where $f$ is the face on the left of $a$.
  \item[(BM1)] For any inner face $f$, the total number of marks in the corners of $f$ is $b - \mathrm{deg}(f)/2$.
  \item[(BM2)] The weight of every inner edge is $b-1$, and the outgoing weight of every inner vertex $v$ is $b+m$, where $m$ is the number of marks in the corners incident to $v$.
  \item[(BM3)] The weight of every inner arc $a$ is at least $b-\mathrm{deg}(f)/2$, where $f$ is the face on the left of $a$.
\end{itemize}
\end{definition}

% Note that the conditions (BM0), (BM1) and (BM3) combined implies that for an inner arc $a$ whose origin is an outer vertex $v_i$, the weight of $a$ and the number of marks in the corner of $v_i$ on the left of $a$ are both equal to $b-\text{deg}(f)/2$, where $f$ is the face on the left of $a$. 

It is clear that $b$-BGS marked orientations are in bijection with even $2b$-GS marked orientations. Hence, from the above discussion we get:

\begin{lem}\label{lem:bij-BGS-beta}
The set $\bBL_G$ of $b$-BGS labelings of $G$ and the set $\bBM_G$ of $b$-BGS marked orientations of $G$ are in bijection. 
\end{lem}

The next incarnation is in terms of angular orientations. Recall the bijection $\Psi$ between marked orientations and angular orientations. Under $\Psi$, the even $2b$-GS marked orientations of $G$ correspond to the $2b$-GS angular orientations of $G$ such that the weight of every arc of the angular map $G^+$ is even. We shall call this subclass of $2b$-GS angular orientations \emph{even}. As before, we can divide every weight by 2, which leads to the following definition.
%in the previous case, we do not use this description as the definition of $b$-BGS angular orientations, but use the version which divides the weight of every arc by 2 instead. 

\begin{definition}\label{def:BGS-angular}
A \emph{$b$-bipartite-grand-Schnyder angular orientation}, or \emph{$b$-BGS angular orientation}, of $G$ is a weighted orientation of the angular map $G^+$ satisfying the following conditions: 
\begin{itemize}
  \item[(BA0)] The weights of outer arcs are 0. Any inner arc $a$ of $G^+$ whose initial vertex is an outer vertex $v_i$ has weight 0 unless $a$ is the arc preceding the outer edge ($v_i,v_{i-1}$) in clockwise order around $v_i$. %For this arc, the weight is $b-\text{deg}(f_i)/2$.
  \item[(BA1)] The outgoing weight of any star vertex $v_f$ is $b-\mathrm{deg}(f)/2$, and the weight of every star edge incident to $v_f$ is $b-\mathrm{deg}(f)/2$.
  \item[(BA2)] The outgoing weight of any inner original vertex is $b$. The weight of any original inner edge is $\mathrm{deg}(f)/2+\mathrm{deg}(f')/2-b-1$, where $f,f'$ are the faces of $G$ incident to $e$.
\end{itemize}
\end{definition}


It is clear that $b$-BGS angular orientations are in bijection with even $2b$-GS angular orientations. Hence, from the above discussion we get:

\begin{lem}\label{lem:bij-BGS-gamma}
The set $\bBM_G$ of $b$-BGS marked orientations of $G$ and the set $\bBA_G$ of $b$-BGS angular orientations of $G$ are in bijection.
\end{lem}


\begin{remark} \label{rk:weight-frozen-bip}
Note that Conditions (BA0) and (BA1) together imply that the inner arc $a$ whose initial vertex is an outer vertex $v_i$ and is pointing toward the star vertex $v_{f_i}$, where $f_i$ is the inner face incident to the outer edge $\{v_i,v_{i-1}\}$, has weight $b-\mathrm{deg}(f_i)/2$.
\end{remark}

Now we give the final incarnation of bipartite $2b$-GS structures as a $b$-tuple of trees. Let us recall the bijection $\Theta$ defined in Section~\ref{sec:statements} between GS corner labelings and GS woods. Observe that under $\Theta$ the set of $b$-BGS labelings of $G$ is in bijection with the subclass of $2b$-GS woods satisfying the following condition: 
\begin{itemize}
%  \item[($\dagger$)] \emph{For every $i \in \{1,\ldots,b\}$, and each black (resp. white) inner vertex $v$, the arcs leading $v$ to its parent in $W_{2i}$ and in $W_{2i-1}$ (resp. in $W_{2i}$ and $W_{2i+1}$) are the same.}
\item[($\dagger$)] \emph{For every $i \in \{1,\ldots,b\}$, and each black (resp. white) inner vertex $v$, the incident outgoing arcs in $W_{2i}$ and in $W_{2i-1}$ (resp. in $W_{2i}$ and $W_{2i+1}$) are the same.}
\end{itemize}

%Observe that under $\Theta$ the $b$-BGS labelings of $G$ are in bijection with the subclass of $2b$-GS woods satisfying the condition that
%
% for every $i \in \{1,\ldots,b\}$, and each black (resp. white) inner vertex $v$, the arcs leading $v$ to its parent in $W_{2i}$ and in $W_{2i-1}$ (resp. in $W_{2i}$ and $W_{2i+1}$) are the same. 
 
 Let us call this subclass the \emph{even $2b$-GS woods}. Note that Condition ($\dagger$) implies that there are redundancies in considering both the odd and the even colors. Focusing on the even colors leads to the following definition.

%% \begin{definition}\label{def:BGS-woods}
%% A \emph{$b$-bipartite-grand-Schnyder wood}, or \emph{$b$-BGS wood}, of $G$ is a $b$-tuple $(W_1',\ldots,W_b')$ of subsets of arcs satisfying:\begin{itemize}
%%   % \item[(BW0)] For all $i \in [b]$, $W_i'$ is a spanning tree of $G$ rooted at $v_{2i}$ and oriented toward its root. It contains all the outer edges except $\{v_{2i}, v_{2i+1}\}$ and does not contain any inner edge incident to $v_{2i}$ or $v_{2i+1}$.
%%   \item[(BW0)] For all $i \in [b]$, every vertex $v \neq v_{2i}$ has exactly one outgoing arc in $W_i'$, while $v_{2i}$ has no outgoing arc in $W_i'$. For $k \neq 2i$, the arc in $W_i'$ going out of the outer vertex $v_k$ is the outer arc oriented from $v_k$ to $v_{k+1}$. Lastly, $W_i'$ does not contain any inner arc oriented toward $v_{2i}$ or $v_{2i+1}$.
%%   \item[(BW1)] For every inner vertex $v$, the incident outgoing arcs $a_1',\ldots,a_b'$ in $W_1',\ldots,W_b'$ are not all the same, and they appear in clockwise order around $v$.
%%   \item[(BW2)] Let $v$ be a black (resp. white) vertex with outgoing arcs $a_1',\ldots,a_b'$ in $W_1',\ldots,W_b'$ respectively (for $v = v_{2i}$ we adopt the convention $a_i' = (v_{2i}, v_{2i-1})$). If $a$ is an inner arc oriented toward $v$ that belongs to $W_i'$, then $a$ appears strictly between $a_{i+1}'$ and $a_i'$ (resp. $a_i'$ and $a_{i-1}'$) in clockwise order around $v$.
%%   \item[(BW3)] \SL{Updated according to Olivier's corrections:}\\
%% \sl{Let $v$ be a black (resp. white) vertex with outgoing arcs $a_1',\ldots,a_b'$ in $W_1',\ldots,W_b'$ respectively (for $v = v_{2i}$ we adopt the convention $a_i' = (v_{2i}, v_{2i-1})$). Let $a$ be an inner arc oriented toward $v$, let $f$ be the face on its right, and let $\epsilon$ be the number of sets in $W_1',...,W_b'$ containing the opposite arc $-a$.
  
%%   If $a$ belongs to $W_i'$, and $b-\deg(f)/2-\eps \geq 0$, then $a$ appears strictly between $a_{i+1+b-\deg(f)/2-\eps}'$ and $a_i'$ (resp. $a_{i+b-\deg(f)/2-\eps}'$ and $a_{i-1}'$) in clockwise order around $v$.
  
%%   If $a$ does not belong to any tree but is between the outgoing arcs in $W_i'$ and $W_{i+1}'$ in clockwise order around the initial vertex of $a$, and $b-\deg(f)/2-\eps \OB{>} 0$, then $a$ appears strictly between $a_{i+1+b-\deg(f)/2-\eps}'$ and $a_{i+1}'$ (resp. $a_{i+b-\deg(f)/2-\eps}'$ and $a_{i}'$) in clockwise order around $v$.}
%%    \OB{My suggestion was slightly different actually, although it does not change the math:}
%% \ob{Suppose $a$ is an inner arc oriented toward $v$ that belongs to $W_i$. Let $f$ be the face on the right of $a$ and $\epsilon$ be the number of sets in $W_1',...,W_b'$ containing the opposite arc $-a$. If $b-\text{deg}(f)/2-\eps \geq 0$, then $a$ appears strictly between $a_{i+1+b-\text{deg}(f)/2-\eps}'$ and $a_{i+1}'$ (resp. $a_{i+b-\text{deg}(f)/2-\eps}'$ and $a_{i}'$) in clockwise order around $v$. The same holds if $a$ does not belong to any tree but is between the outgoing arcs in $W_i'$ and $W_{i+1}'$ in clockwise order around the origin of $a$.}
 
%%  \SL{Maybe we should explain, when $b - \deg(f)/2 - \eps = 0$, what "$a$ appears strictly between $a_{i+1}'$ and $a_{i+1}'$ " means: I think this only says the opposite arc cannot have color $i+1$.}
%% \OB{Yes I }

%% \end{itemize}
%% \end{definition}

%\SL{I wonder if we should combine (BW2) and (BW3) into one condition (similar to (BW2')) since now they look more alike (than those for ordinary GS woods). Such as:} 
%\OB{Do you say they look more alike because we need to add the convention for $v = v_{2i}$ for BW2 as well? I am ok with either formulation.}

\begin{definition}\label{def:BGS-woods}
A \emph{$b$-bipartite-grand-Schnyder wood}, or \emph{$b$-BGS wood}, of $G$ is a $b$-tuple $(W_1',\ldots,W_b')$ of subsets of arcs satisfying:\begin{itemize}
  \item[(BW0)] For all $i \in [b]$, every vertex $v \neq v_{2i}$ has exactly one outgoing arc in $W_i'$, while $v_{2i}$ has no outgoing arc in $W_i'$. For $k \neq 2i$, the arc in $W_i'$ going out of the outer vertex $v_k$ is the outer arc oriented from $v_k$ to $v_{k+1}$. Lastly, $W_i'$ does not contain any inner arc oriented toward $v_{2i}$ or $v_{2i+1}$.
  \item[(BW1)] For every inner vertex $v$, the incident outgoing arcs $a_1',\ldots,a_b'$ in $W_1',\ldots,W_b'$ are not all the same, and they appear in clockwise order around $v$.
  \item[(BW2)] Let $v$ be a black (resp. white) vertex with outgoing arcs $a_1',\ldots,a_b'$ in $W_1',\ldots,W_b'$ respectively (for $v = v_{2i}$ we adopt the convention $a_i' = (v_{2i}, v_{2i-1})$). Let $a$ be an inner arc oriented toward $v$, let $f$ be the face on its right, let $\epsilon$ be the number of sets in $W_1',...,W_b'$ containing the opposite arc $-a$ and let $m = \max(0,b-\deg(f)/2-\eps)$.

 If $a$ belongs to $W_i'$, then $a$ appears strictly between $a_{i+1+m}'$ and $a_i'$ (resp. $a_{i+m}'$ and $a_{i-1}'$) in clockwise order around $v$. 
  
 If $a$ belongs to none of the sets $W_1',\ldots,W_b'$ but is between the outgoing arcs in $W_i'$ and $W_{i+1}'$ in clockwise order around the initial vertex of $a$, then $a$ appears strictly between $a_{i+1+m}'$ and $a_{i+1}'$ (resp. $a_{i+m}'$ and $a_i'$) in clockwise order around $v$. If $m=0$ this condition means that $-a\neq a_{i+1}'$ (resp. $-a\neq a_{i}'$).
\end{itemize}
\end{definition}



\fig{width=\linewidth}{def-bip-woods}{Condition (BW2)  of $b$-BGS woods, where $m := \max(0,b-\deg(f)/2-\eps)$.}


Conditions (BW2)  are represented in Figure~\ref{fig:def-bip-woods}. We now mention a few facts about $b$-BGS woods, but omit their proofs because they are either easy or similar to the one provided for ordinary GS woods in Section~\ref{sec:remaining-proofs}.
%to see or similar results are claimed in Section~\ref{sec:incarnations} for ordinary GS woods and will be proved in Section~\ref{sec:remaining-proofs}.

\begin{remark}\label{rem:reduced-wood}\hfill
\begin{compactitem}
  % \item \SL{(To be confirmed!)} (BW2) and (BW3) imply that for $i \in [b]$, and $a$ an inner arc in $W_i'$ oriented toward an outer vertex, the terminal vertex of $a$ cannot be in $\{v_{2i},...,v_{2i+1+d-\mathrm{deg}(f)}\}$, where $f$ is the face on the right of $a$. Hence the last statement of $(BW0)$ is redundant.
  \item For each $i \in [b]$, $W_i'$ is a spanning tree of $G$ oriented toward its root $v_{2i}$.
  \item  Similar to the case of ordinary GS woods, if $G$ is $2b$-adapted, then the last statement in Conditions (BW2) (about arcs which are in none of the trees $W_1',\ldots,W_b'$) can be dropped because it is redundant with the other conditions.
%%   \begin{itemize} \item[(BW2')] Let $v$ be a black (resp. white) vertex with outgoing arcs $a_1',\ldots,a_b'$ in $W_1',\ldots,W_b'$ respectively (for $v = v_{2i}$ we adopt the convention $a_i' = (v_{2i}, v_{2i-1})$). Suppose $a$ is an inner arc oriented toward $v$ that belongs to $W_i$, then $a$ appears strictly between $a_{i+1+m}'$ and $a_i'$ (resp. $a_{i+m}'$ and $a_{i-1}'$) in clockwise order around $v$, where $m = \max(0,b-\deg(f)/2-\eps)$, $f$ is the face on the right of $a$, and $\eps$ is the number of trees in $W_1',\ldots,W_b'$ containing the opposite arc $-a$.
%% \OB{I agree with this.}. 
%% \SL{If we adopt the new Condition (NEW BW2), then we'll just say the last statement is redundant.}
\end{compactitem}
\end{remark}

Next, we claim that $b$-BGS woods are in bijection with even $2b$-GS woods. This is not obvious and will be elaborated in the following lemma.

\begin{lemma}\label{lem:reduced-wood}
Let $G$ be a bipartite $2b$-map. Let $\cW=(W_1,...,W_{2b})$ be an even $2b$-GS wood and define the mapping $\lambda$ as $\lambda(W_1,...,W_{2b}) = (W_2,...,W_{2b}) = (W_1',...,W_b')$. Then $\lambda$ is a bijection between even $2b$-GS woods and $b$-BGS woods.
As a result, the set $\bBL_G$ of $b$-BGS labelings of $G$ and the set $\bBW_G$ of $b$-BGS woods of $G$ are in bijection.
\end{lemma}

\begin{proof}
%\SL{To be updated.} First we show that $(W_1',...,W_b') = \lambda(W_1,...,W_{2b})$ is indeed a $b$-BGS wood. Conditions (BW0) and (BW1) are immediate from the conditions (W0) and (W1) of oridinary $2b$-GS woods. 
%Conditions (BW2) and (BW3) can also be easily deduced from the conditions (W2), (W3) and the evenness of $(W_1,...,W_{2b})$. 
%Consider for instance the condition (BW3) for a black vertex $v$. Let $a$ be an arc of $W_{2i}$ oriented toward $v$, and let $f$ be the face on the right of $a$, and let $\delta$ be number of trees in $W_1,\ldots,W_{2b}$ containing the opposite arc $-a$. Suppose $2b-\text{deg}(f) - \delta \geq 0$. By (W3), $b$ is strictly between the arcs $a_{2i+1+d-\text{deg}(f)-\delta}$ and $a_{2i-1}$, where $a_1,\ldots,a_{2b}$ are the arcs in $W_1,...,W_{2b}$ having initial vertex $v$. Since $\cW$ is even, we have $\delta = 2\eps$, $a_{2i-1}=a_{2i}=a_i'$ and $e_{2i+1+d-\text{deg}(f)-\delta}=e_{2i+2+d-\text{deg}(f) - \delta} = e'_{i+1+b-\text{deg}(f)/2-\eps}$. The argument for (BW2) and for white vertices is similar.\OB{The above may need to be adjusted, lumping (BW2) and (BW3) together, if (BW3) is updated according to the above.}
 First we show that $(W_1',...,W_b') = \lambda(W_1,...,W_{2b})$ is indeed a $b$-BGS wood. Conditions (BW0) and (BW1) are immediate from Conditions (W0) and (W1) of ordinary $2b$-GS woods.
Condition (BW2) can also be easily deduced from Conditions (W2), (W3) and the evenness of $(W_1,...,W_{2b})$ as follows.
%Consider the case where $v$ is black. 

Let $a = (u,v)$ be an inner arc of $G$. We consider the case where $v$ is black.
Let $f$ be the face on the right of $a$, and let $2\eps$ be number of sets in $(W_1,\ldots,W_{2b})$ containing the opposite arc $-a$. Let us first assume  $a$ is in $W_i'$. By definition, $a$ is in $W_{2i}$, and since $u$ is white, $a$ is also in $W_{2i+1}$. By applying Conditions (W2) and (W3) to the color $2i$ and $2i+1$, we conclude that $a$ appears strictly between the outgoing arcs in $W_{2i+2k+2}$ and $W_{2i}$ in clockwise order around $v$, where $k = \max(0, 2b-\deg(f)-2\eps)$. This translates to $a$ being strictly between $a_{i+1+m}'$ and $a_i'$, where $m = k/2$. 

Now, suppose $a$ is not in any of the sets $W_1',...,W_b'$, but between the outgoing arcs  $W_i'$ and $W_{i+1}'$ in clockwise order around $u$. Since $u$ is white, $a$ is between the outgoing arcs in $W_{2i+1}$ and $W_{2i+2}$ in clockwise order around $u$.
%$W_1,...,W_{2b}$ but is between the outgoing arcs in $W_{2i+1}$ and $W_{2i+2}$ in clockwise order around $u$ (i.e. $a$ is between the outgoing arcs in $W_i'$ and $W_{i+1}'$). 
The last statement of (W3) implies that $a$ appears strictly between the outgoing arcs in $W_{2i+2k+2}$ and $W_{2i+1}$ in clockwise order around $v$, where $k = \max(0, 2b-\deg(f)-2\eps)$. This translates into $a$ being strictly between $a_{i+1+m}'$ and $a_{i+1}'$. The case where $v$ is white is similar.

Next we prove that $\lambda$ is a bijection. Injectivity follows directly from definition~\ref{def:BGS-woods} as the odd colors $(W_1,W_3,...,W_{2b-1})$ can be recovered from the even ones: the outgoing edge of $W_{i}' = W_{2i}$ at a black (resp. white) inner vertex now also belongs to $W_{2i-1}$ (resp. $W_{2i+1}$). The assignment of odd colors to outer edges is different: for each $i \in [b]$, we simply force that the outer arc $(v_{2i},v_{2i+1})$ to have all the odd colors, and the outer arc $(v_{2i-1},v_{2i})$ to have all the odd colors except for color $2i-1$.

To prove surjectivity, we need to show for any $b$-tuple $\cW'=(W_1',...,W_b')$ satisfying Conditions (BW0-BW2), the tuple $\cW=(W_1,W_2,...,W_{2b})$ obtained by the recovery rule outlined above is an even $2b$-GS wood. As before, it is easy to check that Conditions (BW0-BW2) for $\cW'$ imply Conditions (W0-W3) for $\cW$.
%\OB{I think that the last statement of (W0) is not harder to check than (W2) or (W3) so I deleted the specific discussion about it.}
Hence $\cW$ is a $2b$-GS wood. Moreover it is clear from the definition that $\cW$ is even, which complete the proof of the surjectivity of $\lambda$.
%% Moreover, it is clear that if $\cW$ is a $2b$-GS wood, then it is even. It remains to verfy (W0).
%% The only nontrivial point of (W0) is its last statement. For $i \in [b]$, the last statement of (BW0) guarantees that $W_{2i} = W_i'$ does not contain any inner arc oriented toward $v_{2i}$ or $v_{2i+1}$. To complete the proof, we need to show $W_{2i-1}$ does not contain any inner arc oriented toward $v_{2i-1}$ or $v_{2i}$. Note that by construction an inner arc in $W_{2i_1}$ is either an inner arc in $W_{2i}$ whose initial vertex is a black inner vertex or an inner arc in $W_{2i-2}$ whose initial vertex is a white inner vertex. However, the terminal vertex of the former kind cannot be $v_{2i}$ by (BW0) and also $v_{2i-1}$ since $v_{2i-1}$ is black, and the terminal vertex of the later kind avoids $v_{2i-1}$ by (BW0) and also $v_{2i}$ since $v_{2i}$ is white. Therefore we have proved that for each $k$, $W_k$ does not contain any inner arc oriented toward $v_i$ or $v_{i+1}$ and hence proved the whole lemma.
%% \SL{Or: by the third bullet point of Remark~\ref{rem:reduced-wood}, after (W2) and (W3) are verified, the last statement of (W0) is just a consequence.}
%% \OB{That would not work since Remark~\ref{rem:reduced-wood} is for bipartite woods, and the corresponding statement for ordinry GS has been deleted.}

The argument above immediately implies that the set $\bBL_G$ of $b$-BGS labelings of $G$ and the set $\bBW_G$ of $b$-BGS woods of $G$ are in bijection.  
\end{proof}

%% Lemma~\ref{lem:reduced-wood} immediately implies the following result.

%% \begin{prop}\label{prop:bij-BGS-theta}
%% The set $\bBL_G$ of $b$-BGS labelings of $G$ and the set $\bBW_G$ of $b$-BGS woods of $G$ are in bijection.
%% \end{prop}

%Lemma~\ref{lem:reduced-wood} immediately implies that the set $\bBL_G$ of $b$-BGS labelings of $G$ and the set $\bBW_G$ of $b$-BGS woods of $G$ are in bijection.  
%We conclude this part by summarizing our results:
%Proposition~\ref{prop:bij-BGS-beta}, Proposition~\ref{prop:bij-BGS-gamma} and Proposition~\ref{prop:bij-BGS-theta} into one theorem.

We conclude this subsection with following Theorem, which summarizes Lemma \ref{lem:bij-BGS-beta},  Lemma \ref{lem:bij-BGS-gamma} and Lemma \ref{lem:reduced-wood}:

\begin{thm}\label{thm:bij-BGS}
Given a bipartite $2b$-map $G$, the sets $\bBL_G$ of $b$-BGS labelings, $\bBW_G$ of $b$-BGS woods, $\bBM_G$ of $b$-BGS marked orientations and $\bBA_G$ of $b$-BGS angular orientations are in bijections.
\end{thm}

\subsection{Existence Result}
In this subsection, we state the existence theorem for $b$-BGS structures, and show that it is a consequence of Theorem~\ref{thm:main} for non-bipartite GS structures.

\begin{thm}\label{thm:BGS-main}
Let $b \geq 2$, and let $G$ be a bipartite $2b$-map. There exists a $b$-BGS wood (resp. labeling, marked orientation, angular orientation) for $G$ if and only if $G$ is $2b$-adapted. 

Moreover for any fixed $b$, there is an algorithm which takes as input a $2b$-adapted map, and outputs a $b$-BGS wood (resp. labeling, marked orientation, angular orientation) in linear time in the number of vertices. 
\end{thm}

\begin{proof}
We assume Theorem~\ref{thm:main}, which will be proved in Section~\ref{sec:proof-existence}. Note that $b$-BGS woods can be identified as a subclass of $2b$-GS wood (which is especially clear from the $b$-BGS labeling incarnation), so the necessity of $2b$-adaptedness is clear from Theorem~\ref{thm:main}. 

It remains to prove the sufficiency of $2b$-adaptedness. Let $G$ be a $2b$-adapted bipartite map. By Theorem~\ref{thm:bij-BGS} the different incarnations of $b$-BGS structures are in bijection. Hence it suffices to prove the existence of a $b$-BGS angular orientation. In turn this reduces to proving the existence of an even $2b$-GS angular orientation.

%After establishing the equivalence among different incarnations and their reduced forms, it suffices to prove the existence of one of the incarnations. Here we choose the angular orientation incarnation. 

By Theorem~\ref{thm:main}, $G$ admits a $2b$-GS angular orientation $\cA$. Let us call \emph{odd} the arcs of $G^+$ having an odd weight in $\cA$. If there is no odd arc, then $\cA$ is even and we are done. Otherwise, let us explain how to produce an angular orientation $\cA'$ with fewer odd arcs. 
%If $\cA$ is even, then we can obtain a $b$-BGS angular orientation by dividing the weight of each arc by 2. The rest of the proof is devoted to showing how to convert an uneven $2b$-GS angular orientation to an even one.
Since $G$ is bipartite, the degree of every face is even. Hence the outgoing weight of every vertex of $G^+$ is even and the total weight of every edge of $G^+$ is even. Now, suppose that the arc $a_1 = (u_0, u_1)$ is odd, then the opposite arc $-a = (u_1, u_0)$ is also odd. Since the total outgoing weight of $u_1$ is even, $u_1$ must have another outgoing odd arc $a_2 = (u_1,u_2) $. By repeating this process we get a path of odd arcs $a_1 = (u_0, u_1), a_2 = (u_1,u_2),\ldots$. Note that this path cannot reach the outer vertices, because every arc incident to an outer vertex has even weight. Therefore, the sequence of odd arcs must contain a directed cycle. Note that if we subtract 1 from the weight of every arc in this cycle and add 1 to their opposite arcs, the resulting weighted orientation $\cA'$ is still a $2b$-GS angular orientation of $G$. Moreover, $\cA'$ has fewer odd arcs as promised, and repeating this process leads to an even $2b$-GS angular orientation. 
%In fact, this procedure will become intuitive after we discuss the lattice property of $2b$-GS structures in Section~\ref{sec:lattice}. The operation of subtracting 1 from every arc in a directed cycle while adding 1 to their opposite arcs is closely related to the ordering of the lattice of $2b$-GS structures on $G$.

For the runtime, if $G$ has $n$ vertices, once we obtain some $2b$-GS angular orientation $\cA$, which takes linear time by Theorem~\ref{thm:main}, testing whether $\cA$ is even, or finding all the odd arcs if it is not, takes only linear time since the total number of edges is linear in $n$. The runtime of eliminating all odd arcs is linear in the total number of odd arcs. Hence the total runtime to construct an even $2b$-GS angular orientation is linear in $n$. The conversion from an even $2b$-GS structure to a $b$-BGS structure takes linear time, and the bijections between the different incarnations are also linear, which concludes the proof.
\end{proof}

\begin{Remark}
In the above proof we showed that Theorem~\ref{thm:main} implies Theorem~\ref{thm:BGS-main}. We mention that, conversely, Theorem~\ref{thm:BGS-main} implies Theorem~\ref{thm:main}. Indeed, let us assume Theorem~\ref{thm:BGS-main}, and prove the existence of $d$-GS marked orientations for $d$-adapted maps. Given a $d$-adapted map $G$ we can draw an ``edge vertex'' at the center of every edge to obtain a bipartite $2d$-adapted $2d$-map $\overline{G}$. By Theorem~\ref{thm:BGS-main}, $\overline{G}$ admits a $2d$-GS marked orientation. We can then delete the edge vertices and obtain a $d$-GS marked orientation of $G$ in the way illustrated in Figure~\ref{fig:bgs_to_gs}.
\end{Remark}

\fig{width=\linewidth}{bgs_to_gs}{Merge an edge of $\overline{G}$.}

\subsection{Connections to Known Structures}\label{sec:bip-classical}

%%\documentclass{amsart}
%%%%%% GENERAL MATH COMMANDS
% Reals
\newcommand{\R}{{\mathbb R}}
% Integers
\newcommand{\Z}{{\mathbb Z}}
% Naturals
\newcommand{\N}{{\mathbb N}}
% Expectation
\DeclareMathOperator*{\E}{\mathbb{E}}
% ^th notation
\newcommand{\tth}{^{\text{th}}}
% Small dots for integer range [a .. b]
\newcommand{\sdots}{\,..\,}
% Vectorized version of matrix
\newcommand{\matvec}{\mbox{vec}}

% := sign
\newcommand{\defeq}{\vcentcolon=}
% Zero function
\newcommand{\zf}{\mathbf{0}}
% Vector of ones
\newcommand{\ones}{\mathbf{1}}

% Argmin and argmax definitions
\DeclareMathOperator*{\argmax}{arg\,max}
\DeclareMathOperator*{\argmin}{arg\,min}


%%%%% PROBLEM STATEMENT NOTATION 
% \newcommandtwoopt{\St}[2][t][]{{S_{#1}^{#2}}} % State
\newcommand{\task}[1][i]{{\mathcal{T}_{#1}}} % Task, optionally takes index
\newcommand{\tasks}{\{ \task \}_{i=1}^N}
\newcommand{\losst}[1][i]{{l_{#1}}}
\newcommand{\lossv}[1][i]{{l_{#1}^{\textrm{val}}}}
\newcommand{\tasktarget}{{\mathcal{T}_{\textrm{target}}}}
\newcommand{\lossttarget}{l_{\textrm{target}}}
\newcommand{\lossvtarget}{l_{\textrm{target}}^{\textrm{val}}}
\newcommand{\lossttargetit}{l_{\textrm{target}}^{(k)}}
\newcommand{\losstotal}{l^{\textrm{total}}}
\newcommand{\lossopt}{l^*}

\newcommand{\thetait}[2]{\theta_{#1}^{(#2)}}
\newcommand{\phit}[1]{\phi^{(#1)}}
\newcommand{\hist}[2]{S_{#1}^{(#2)}}
\newcommand{\grad}[2]{G_{#1}^{(#2)}}

\newcommand{\Alg}{\textup{\textbf{Opt}}}
\newcommand{\MetaAlg}{\textup{\textbf{MetaOpt}}}

%%%%% Theorems
\newtheoremstyle{mytheoremstyle} % name
    {\topsep}                    % Space above
    {\topsep}                    % Space below
    {\itshape}                   % Body font
    {}                           % Indent amount
    {\scshape}                   % Theorem head font
    {.}                          % Punctuation after theorem head
    {.5em}                       % Space after theorem head
    {}  % Theorem head spec (can be left empty, meaning ‘normal’)
\theoremstyle{mytheoremstyle}
\theoremstyle{plain}
\newtheorem{theorem}{Theorem}
\newtheorem{proposition}{Proposition}
\newtheorem{assumption}{Assumption}
\newtheorem{definition}{Definition}
\newtheorem{lemma}{Lemma}
\theoremstyle{remark}
\newtheorem{remark}{Remark}

%\begin{document}
%\subsection{Connections to previously known structures}

\subsubsection{Bipartite grand-Schnyder structures on $2b$-angulations of girth $2b$, and their relation to 2-orientations and bipolar orientations}\label{sec:bipartite_angulations}
%\input{other_structures_bipartite_1}
In this subsection we relate $b$-BGS structures to the \emph{even Schnyder decompositions} defined in~\cite{OB-EF:Schnyder} for $2b$-angulations. The special case $b=2$ is the most classical, as $2$-BGS structures are in bijection with plane bipolar orientations.


Recall from Section~\ref{sec:classical} that a $d$-angulation $G$ is $d$-adapted if and only if it has girth $d$, and that all four incarnations of $d$-GS structures can be simplified in this case. These structures where studied in~\cite{OB-EF:Schnyder} under the name of \emph{$d$-Schnyder structures}. 
%These structures were studied by the first and the second author~\cite{OB-EF:Schnyder} under the name of $d$-Schnyder structures. 
In particular, the $d$-GS marked orientation incarnation and the $d$-GS angular orientation incarnation both simplify into the same type of weighted orientations of $G$ (with no marks) called \emph{$d/(d-2)$-orientations}. In~\cite{OB-EF:Schnyder} the $d$-GS corner labelings and the $d$-GS woods of $d$-angulations were called \emph{$d$-Schnyder labelings} and \emph{$d$-Schnyder decompositions}, respectively.

When $d=2b$ is an even integer, a (nonempty) subclass of Schnyder structures on $2b$-angulations of girth $2b$ called \emph{even} was studied in~\cite{OB-EF:Schnyder}. %Again we fix the bicoloring of the vertices so that the outer vertex $v_1$ is black. 
The class of even $d$-Schnyder structures can easily be identified with the class of $b$-BGS structures of $2b$-angulations:% defined in Section~\ref{sec:incar_bipartite} was characterized in a similar way as in the first subsection~\ref{sec:incar_bipartite}, and is closely related to the $b$-BGS structures on $2b$-angulations. More precisely,
\begin{compactitem}
\item A $2b$-Schnyder labeling is \emph{even} if all the corners incident to black (resp. white) vertices have odd (resp. even) labels. This characterization exactly coincides with the definition of $b$-BGS labelings on $2b$-angulations.
\item A $2b/(2b-2)$-orientation is \emph{even} if the weight of every inner arc is even. Dividing every weight by 2 gives a structure called \emph{$b/(b-1)$-orientation} in~\cite{OB-EF:Schnyder} (these are weighted orientations of the inner edges such that edges have weight $b-1$ and vertices have weight $b$). For a $2b$-angulation the $b/(b-1)$-orientations exactly coincide with the $b$-BGS marked orientations (no mark) and $b$-BGS angular orientations (weight 0 on star edges).
\item A $2b$-Schnyder decomposition is \emph{even} if for every $i \in \{1,...,d\}$, and each black (resp. white) inner vertex $v$, the arcs leading $v$ to its parent in $W_{2i}$ and in $W_{2i-1}$ (resp. in $W_{2i}$ and $W_{2i+1}$) are the same. It was shown in~\cite{OB-EF:Schnyder} that keeping only the trees of even color does not result in any loss of information. This simplified structures, called \emph{reduced Schnyder decompositions} in~\cite{OB-EF:Schnyder}, coincide with the $b$-BGS woods of $2b$-angulations.
%It can be shown, by an similar argument as in Lemma~\ref{lem:reduced-wood}, that taking only the trees of even colors will not incur any loss of information, and by doing so we retrieve exactly the $b$-BGS woods on $2b$-angulations. 
\end{compactitem}


The case $b=2$ (of $b$-BGS structures on $2b$-angulations) is classical and precedes~\cite{OB-EF:Schnyder}. Let $G$ be a quadrangulation. By definition, a $2/1$-orientation of $G$ is simply an (unweighted) orientation of the inner edges of $G$ such that every inner vertex has outdegree 2. These are simply called \emph{2-orientations} of $G$, and $G$ admit such an orientation if and only if it is simple (that is, has no double edge, which is equivalent to having girth 4 in this case). 
Next consider the corner labeling incarnation: because of the parity condition in BGS corner labeling, there is no loss of information in replacing each label $i$ by $\lfloor (i-1)/2 \rfloor$. 
%As mentioned in Section~\ref{sec:incar_bipartite}, an even $2b$-Schnyder labeling on a $2b$-angulation can be modified (without loss of information) by replacing each label $i$ by $\lfloor (i-1)/2 \rfloor$. In particular, for plane quadrangulations (case $b=2$), the labels are 0,0,1,1 around each inner face. 
This incarnation of 2-orientations was studied by Felsner et al. in~\cite{FeHuKa}.
%\OB{The preceding citation needs to be more explicit (if there is a generalization beyond $b=2$).}


As explained in the introduction, $2$-oriented quadrangulations are in bijection with plane bipolar orientations. The bijection is given by Figure~\ref{fig:2-orientations}. 
%\OB{We could move the figure and discussion from the introduction to this place.}



%In terms of orientations, the 2-BGS structures of quadrangulations correspond to the so-called 2-orientations

%\begin{remark}\label{rem:bipartite_4} 
Let us finally mention that 2-orientations were used by Barri\'ere and Huemer~\cite{Barriere-Huemer:4-Labelings-quadrangulation} to design a straight-line drawing algorithm for quadrangulations. These structures (in the form of, dual, even $4$-Schnyder structures) were also used in~\cite{OB-EF:Schnyder} to design a drawing algorithm for 4-regular plane maps. In a forthcoming article~\cite{OB-EF-SL:4-GS-drawing}, we will present extensions of these two algorithms (and of the drawing algorithm of He~\cite{He93:reg-edge-labeling}, which is based on transversal structures).

%$2$-BGS structures for plane quadrangulations (i.e. even $4$-Schnyder structures) are known to have many drawing applications. Barri\'ere and Huemer~\cite{Barriere-Huemer:4-Labelings-quadrangulation} used such structures, though in a different incarnation, to design a straight-line drawing algorithm for quadrangulations. The first and the second author used the \emph{dual} of such structures, which we will define in Section~\ref{sec:dual}, to design an \emph{orthogonal} drawing algorithm for 4-regular plane maps. In a forthcoming article~\cite{OB-EF-SL:4-GS-drawing}, we will present extensions of both algorithms. In particular, the evenness condition can be dropped in both cases. 
%\end{remark}
\subsubsection{BGS structures for quadrangulations of the hexagon, and their relation to Felsner woods}\label{sec:Felsner} 
In this subsection, we consider the $3$-BGS structures for \emph{quadrangulations of the hexagon} (6-maps where inner faces have degree 4). This case bears a strong analogy to the case of transversal structures discussed in Section~\ref{sec:transversal}. For these maps (which are edge-tight) the BGS structures can be identified with certain edge colorings, and they are related to the Felsner woods of 3-connected plane maps. 


\begin{figure}
\begin{center}
\includegraphics[width=\linewidth]{Felsner_woods}
\end{center}
\caption{On the left, a Felsner edge-coloring of an quadrangulation of the hexagon, and the 3 associated bipolar orientations (blue-green, red-blue, and green-red). 
On the right, the corresponding coloring of corners of the diagonal-map, and the 3 associated spanning trees.}
\label{fig:Felsner_woods}
\end{figure}


Let $G$ be a quadrangulation of the hexagon. Clearly, such a map is 6-adapted if it is simple and every 4-cycle bounds a face.\footnote{The 6-adapted quadrangulations of the hexagon are sometimes called \emph{irreducible quadrangulations of the hexagon} in the literature~\cite{FuPoScL,bouttier2014irreducible}.}
%We now consider the case of $6$-adapted maps with inner faces of degree $4$, also called \emph{irreducible quadrangulations of the hexagon} in the literature~\cite{FuPoScL,bouttier2014irreducible} (irreducible means that every 4-cycle bounds a face). It bears a strong analogy to the case of transversal structures discussed in Section~\ref{sec:transversal}. 
A \emph{Felsner edge-coloring} of $G$ is a coloring of the 
inner edges of $G$ in red, blue, green with the following properties (see the top-left part of 
Figure~\ref{fig:Felsner_woods}):
\begin{itemize}
\item[(C0)]
All inner edges incident to $v_1$ and $v_4$ are blue, all inner edges incident to $v_2$ and $v_5$ are green, and all inner edges incident to $v_3$ and $v_6$ are red.
\item[(C1)]
Around every inner vertex, the incident edges form 3 non-empty groups in clockwise order: red edges, green edges, and blue edges. 
\end{itemize} 
Felsner edge-colorings are closely related to extensions of Schnyder structures developed by Felsner~\cite{F01,Felsner:geodesic-embedbings,Felsner:lattice} for $3$-connected maps. 
Precisely, with the bi-partition of the vertices of $G$ into black and white vertices (where $v_1$ is black), one can classically associate a plane map $M$ to $G$, called the \emph{diagonal-map} of $G$, where the vertices of $M$ are 
the black vertices of $G$, and there is one edge of $M$ for each inner face $f$ of $G$, which connects the two diagonally opposed black vertices around $f$. 
The obtained map is actually a \emph{suspended map}, that is, a map with $3$ distinguished vertices ($v_1,v_3,v_5$) incident to the outer face, whose marking is indicated by a dangling half-edge incident to the outer face; the dangling half-edges at $v_1,v_3,v_5$ are colored blue, red, and green respectively. 
Let $M^{\infty}$ be the map obtained from $M$ by joining the dangling half-edges to an additional vertex $v_{\infty}$ in the outer face. 
The map $M$ is called \emph{quasi-3-connected} (case considered by Felsner) if $M^{\infty}$ is $3$-connected, which is equivalent to the fact that $G$ is 6-adapted and has at least one inner edge incident to each of $v_1,v_3,v_5$. Since each edge of $G$ corresponds to a corner of $M$, a Felsner edge-coloring is equivalent (see the top-right part of Figure~\ref{fig:Felsner_woods}) to a coloring of the 
corners of $M$ in red, blue or green such that:
\begin{itemize}
\item[(C0')]
For each color $c\in\{$red, blue, green$\}$, the corners of label $c$ in the outer face are those in the interval delimited by the dangling half-edges of the two other colors; 
and all inner corners incident to the distinguished outer vertex carrying the dangling half-edge of color $c$ have color~$c$.
\item[(C1')]
Around every non-distinguished vertex and every inner face, the incident corners in clockwise order form 3 non-empty groups: red corners, green corners, and blue corners. 
\end{itemize} 
These are exactly the corner colorings defined by Felsner in~\cite{Felsner:geodesic-embedbings}. 
%He also shows that in such a structure, around each edge the $3$ colors appear among the 4 incident corners, and the color appearing twice is at two corners that are not diagonally opposed around the edge. 
%In the Felsner coloring of $G$, upon coloring the outer edges in clockwise order as blue/green/red/blue/green/red starting with $(v_1,v_2)$, this condition translates to the property that every face has the 3 colors on its boundary, and has a unique unicolored corner. 
Felsner also shows that such a coloring yields $3$ spanning trees of $M$ (thus giving an extension of Schnyder woods to 3-connected plane maps). The red (resp. blue, green) tree is rooted at $v_3$ (resp. $v_1$, $v_5$), with the parent edge of each non-root vertex of $M$ being the unique incident edge marking the separation between the groups of green/blue edges (resp. of red/green edges, of blue/red edges). These trees are represented in the bottom-right part of Figure~\ref{fig:Felsner_woods}.  


%Similarly as for transversal structures, one can associate some plane bipolar orientations (three here) to a Felsner edge-coloring. 
As we now explain, one can associate some plane bipolar orientations to a Felsner edge-coloring, in a way that is similar to the case of transversal structures treated in Section~\ref{sec:transversal}.
%Precisely, 
For a quadrangulation of the hexagon $G$ endowed with a Felsner edge-coloring, we define the \emph{red-blue bipolar orientation} as the oriented map obtained by deleting the green edges, orienting the red edges from white to black, orienting the blue edges from black to white, and orienting the $6$ outer edges in the flow-direction from $v_6$ to $v_3$. Similarly, the green-red (resp. blue-green) bipolar orientation is obtained by erasing the blue (resp. red) edges, orienting the red (resp. green) edges from black to white, orienting the green (resp. blue) edges from white to black, and orienting the outer edges in the flow-direction from $v_2$ to $v_5$ (resp. $v_4$ to $v_1$), see the bottom-left part of Figure~\ref{fig:Felsner_woods}. 

\begin{remark}
The three bipolar orientations defined above are also natural in the context of orthogonal surface representations (which are specific 3D representations) associated with the Felsner structures~\cite{Felsner:geodesic-embedbings,felsner2008schnyder}. Then the faces of the red-blue (resp. green-red, blue-green) bipolar orientation correspond to the \emph{flats} of the orthogonal surface in the direction orthogonal to the $y$-axis (resp. $x$-axis, $z$-axis), and the dual bipolar orientation indicates order constraints on the $y$-coordinates (resp. $x$-coordinates, $z$-coordinates) of those flats so as to have a valid \emph{rigid} orthogonal surface representation of the Felsner structure. The red-blue bipolar orientation has also been recently used to obtain enumerative results on Felsner structures~\cite{enumerationFelsnerColorings}. 
\end{remark}

\begin{figure}
\begin{center}
\includegraphics[width=12cm]{corner_labeling_felsner}
\end{center}
\caption{On the left, a $3$-BGS arc labeling on a quadrangulation of the hexagon.
 On the right, the corresponding Felsner edge-coloring (upon coloring blue/green/red the outer edges $(v_i,v_{i+1})$ for $i=1/2/3$ modulo $3$). 
The two bottom rows show the local conditions at inner vertices and inner faces (of 6 possible types) when superimposing both structures. The top row shows the associated $3$-BGS angular orientation. 
}
\label{fig:corresp_BGS_Felsner_coloring}
\end{figure}

We now discuss the link with bipartite grand-Schnyder structures. First note that quadrangulations of the hexagon are edge-tight in the sense of Section~\ref{sec:arc_labeling}. 
%\OB{I have removed the terminology ``irreducible'' from most of the paper. Sometimes I replaced it with ``6-adapted'', but sometimes I just deleted it, which slightly changes the statement like in the following sentence (or, with lesser importance, the previous sentence).}
As illustrated in Figure~\ref{fig:corresp_BGS_Felsner_coloring}, for $G$ a quadrangulation of the hexagon, there is a direct bijection between the Felsner edge-colorings of $G$ and the $3$-BGS structures of $G$. 
Let $\cAL$ be a $6$-GS arc labeling of $G$. It corresponds to a 3-BGS corner labeling if and only if the inner arcs with black (resp. white) initial vertex have odd (resp. even) labels. We call such an arc labeling a \emph{$3$-BGS arc labeling} of $G$. 
Condition (AL2) and the parity property imply that in a $3$-BGS arc labeling $\cAL$, around any black (resp. white) inner vertex there are 3 non-empty groups of outgoing arcs of label 1,3,5 (resp. 2,4,6). Hence, to a $3$-BGS arc labeling $\cAL$, one can associate a Felsner edge-coloring $\eta(\cAL)$ by coloring blue (resp. green, red) the inner edges with arc labels $\{1,4\}$ (resp. $\{2,5\}$, $\{3,6\}$). Conversely, to a Felsner edge-coloring $\cF$ of $G$, one associates a $3$-BGS arc labeling $\beeta(\cF)$ by giving the label 1 (resp. 3,5) to the arcs of color blue (resp. red, green) with black initial vertex, and the label 2 (resp. 4,6) to the arcs of color green (resp. blue, red) with white initial vertex. It is clear that $\eta$ and $\beeta$ are inverse mappings, hence bijections, between the set of $3$-BGS arc labelings and the set of Felsner edge-colorings of $G$.



%As illustrated in Figure~\ref{fig:corresp_BGS_Felsner_coloring}, for $G$ a quadrangulation of the hexagon, there is a direct bijection between the Felsner edge-colorings of $G$ and the $3$-BGS arc labelings of $G$: for $\cAL$ a $3$-BGS arc labeling of $G$, each inner edge with labels $\{1,4\}$ (resp. $\{2,5\}$, $\{3,6\}$) is turned into a blue (resp. green, red) edge. 

Note that the red-blue (resp. green-red, blue-green) bipolar orientation of the Felsner edge-coloring $\eta(\cAL)$ is exactly the plane bipolar orientation $B_5$ (resp. $B_1$, $B_3$) associated to the $3$-BGS arc labeling $\cAL$ by the mapping $\beta$ (see Section~\ref{sec:arc_labeling}). Hence, according to Remark~\ref{rk:bipolarBi}, the even grand-Schnyder wood $\cW=(W_1,W_2,W_3,W_4,W_5,W_6)$ associated to $\cAL$ can be easily obtained from these bipolar orientations. Precisely, up to changing the tree-root of $W_i$ from $v_i$ to $v_{i-2}$ for all $i$, 
the tree $W_5$ (resp. $W_2$) is the leftmost outgoing tree (resp. rightmost ingoing tree) of the red-blue bipolar orientation, 
the tree $W_1$ (resp. $W_4$) is the leftmost outgoing tree (resp. rightmost ingoing tree) of the green-red bipolar orientation, and the tree $W_3$ (resp. $W_6$) is the leftmost outgoing tree (resp. rightmost ingoing tree) of the blue-green bipolar orientation. This correspondence is represented in Figure~\ref{fig:corresp_BGS_Felsner_coloring}. (Note that the trees $W_1,\ldots,W_6$ of the grand-Schnyder wood are closely related to the three bipolar orientations rather than to the three spanning trees of the Felsner wood.) 


Let us finally consider the angular orientations incarnation.
In~\cite{Felsner:lattice} Felsner shows that (when $G$ has at least one inner edge incident to each of $v_1,v_3,v_5$) his corner labelings of $M$ correspond to orientations of the star edges of $G^+$ (these edges are those of the superimposition of $M$ with its dual, upon considering that there are 3 outer faces separated by the dangling half-edges) such that $v_1,v_3,v_5$ have outdegree~$2$, $v_2,v_4,v_6$ have outdegree $0$, all inner vertices of $G$ have outdegree $3$, and the star vertices have outdegree $1$. Letting $s_i$ be the star vertex in the inner face containing the outer edge $(v_{i-1},v_i)$, these orientations of $G^+$ defined in~\cite{Felsner:lattice} coincide with the $3$-BGS angular orientations of $G$, upon returning the edges $(v_2,s_2), (v_1,s_2), (v_4,s_4), (v_3,s_4), (v_6,s_6), (v_5,s_6)$. 
Moreover, as shown in the top-part of Figure~\ref{fig:corresp_BGS_Felsner_coloring}, the correspondence in~\cite{Felsner:lattice} commutes with our correspondence $\Gamma$ between $3$-BGS arc labelings and $3$-BGS angular orientations. 
 

 
%\bibliographystyle{plain}
%\bibliography{biblio-Schnyder}

%\end{document}




%\end{document}
 
% \subsection{Cooperative Training} 
\TODO{1st paragraph: explain the goal of cooperative training; basic idea to achieve the goal}
The key challenge of co-operative training with gpu local graph the input features of a layer are not self-sufficient. 
The result of a gather operation might be incomplete as some neighbours might be resident in another gpu, similarly 
the destinations nodes hidden value needed for the scatter operation might be not be available on the same gpu. 
To facilitate co-operative training, we must exchange hidden feature data during training. 
We define two forms of exchange. 
\begin{itemize}
    \item Push to owner: partial data from each remote node on a gpu is pushed to its owner gpu.
    \item Push from owner: data from owner node on a gpu is pushed to all gpus which have this node as a remote node. 
\end{itemize}
Using the above two communication patterns allow us to construct the standard gather and scatter layers. 
The forward pass of the gather function first performs a local gather and then synchronization with remote. 
The backward pass proceeds in the reverse faction where data is pushed from owner and then further pushed to src. 
We similary, design the forward and backward pass of scatter. 
We note that communication happens through cross edges.
The receiving gpu should have necessary information on how much data is expected from each sending gpu. 
Note, that this communication is blocking till all gpus finish the exchange. 

\begin{algorithm}[t]
\caption{SAGE for distributed graphs}
\begin{algorithmic}[1]
  \Function{push\_to\_owner}{$g:device$, $G^l_g$, $H_g$}
    \For{ $ i \neq g \land i \in [0,4]$}
        \State $ send(H^{g}[G^l_g.to[i]], dest = i)$
    \EndFor
    \For{$ i  \neq g \land i \in [0,4]$}
       \State $ H^{g}[G^l_g.from[i]] += recv(src = i)$
    \EndFor
    \Return $H^{g}[O^l_g]$
  \EndFunction
  \Function{push\_from\_owner}{$g:device$, $G^l_g$, $H_g$}
    \For  {$ i  \neq g \land i \in [0,4]$}
       \State $ send(H^{g}[G^l_g.from[i]], dest = i)$
    \EndFor
    \For  {$ i  \neq g \land i \in [0,4]$}
       \State $ H^{g}[G^l_g.to[i]]  = recv(src = i)$
    \EndFor
  \EndFunction
  \Function{GatherLayer}{}
    \Function{forward}{$H,G^l_g,g:device$}
        \State $H1 \gets gather(H,G^l_g)$
        \State $H2 \gets push\_to\_owner(H1, G^l_g)$
        \State \S\Return $H2[G[O^l_g]]$
    \EndFunction
    \Function{backward}{$H_g,G^l_g,g:device$}
        \State $H \gets empty $
        \State $H[G[O^l_g] \gets H_g $
        \State $H[G[R^l_g] \gets push\_from\_owner(H) $    
        \State $H1 \gets gather\_backward(H,G^l_g)$
        \State \Return $H1$
    \EndFunction
  \EndFunction
  \Function{ScatterLayer}{}
   \Function{forward}{$H,G^l_g,g:device$}
        \State $H1 \gets pull\_from\_self(H,G^l_g) $
        \State $H2 \gets push\_from\_owner(H1 $
        \State \Return $H2$
    \EndFunction
    \Function{backward}{$H_g,G^l_g,g:device$}
        \State $H2 \gets pull\_from\_owner $
        \State $H1 \gets (H,G^l_g) $
        \State \Return $ H2[G[O^l_g]] $
    \EndFunction
  \EndFunction
\end{algorithmic}
\end{algorithm}

\TODO{how to do the forward and backward during training using GCN /GAT example} 
We show the computation of the following function using our API.
\TODO{Is this example required. The algorithm seems to be straightforward}
\begin{equation*}
     H^{l+1} = \sum(H^{l}(v) * H^{l}(u))
\end{equation*}

% %     \begin{verbatim}    
% %     pull_from_self
% %     global_scatter
% %     edge_wise_local_computation
% %     node_wise_local_gather
% %     global_gather
% % \end{verbatim} &

\subsection{Execution Model}
\TODO{Caching and other details}

We use the partitioning function for execution management and input data management. 
Every node in $V^0$ is assigned to a gpu partition. 
In case of cache miss $V^0$ is moved only to the cache partition.
After loading the cache, each layer proceeds to exchange data such that the final value of node, lies on the owner GPU. 
The training proceeds as such till the training node vectors are predicted. 

 
%\TODO{Sandeep: This an attempt to proof why metis works. If this is unconvincing we can skip it}
\section{Split-Parallel Optimal Scheduling}

\TODO{1st paragraph should summarize this subsection in high-level: 1 sentences: What is goal of split-parallel online  scheduling; 2nd sentence: what is the basic idea/insight we use in the split-parallel online scheduling}
We attempt to justify our use of partitioning the training graph. 
We rely on the insight that sampling in graphs even with a few hops quickly span multiple partitions and that we can use properties of the training graph to generate efficient partitions of the sample graph. 

\TODO{Define Cost Function}
\TODO{1st paragraph explains the goal of this assignment: why we do this one different from the input vertices: two principles - fast; reduce edge cuts (reduce GPU-GPU communication) --> offline ensures fast; metis algorithm works pretty well.} 
Given a sampled global graph $G_s(V,E)$, with $L$ layer  we can construct the split graphs using a partitioning function $P(v^l)$ which maps for every layer, a vertex to a gpu. 
In our approach we incur 2 costs, the cost of cache refresh and the cost of inter layer communication. 
To achieve load balancing in both computation and communication with caching, we would like to have each gpu to have approximately the same number of vertices mapped to it per layer. 
While we efficiently manage our cache by ensuring that the vertices in $v^0$ are equally distributed, we optimize cost for inner nodes in our approach due to intermediate shuffling through the partitioning function. 
As we have have local gather and scatter which aggregate messages, the total cost of shuffling for a node $v^l$ is defined as the number of partitions which contain atleast one source vertex which has an incident edge to it. 
Formally, We define a cost function for a vertex as below:
\begin{equation*}
    C[v^l] = |\{p: E(u,v) \in G_s \land P(u) = p \land p \neq P(v)\}|
\end{equation*}
This is due to the local gathering and scattering in our system.
We also want the partitioning function to be computationally tractable as this function must be run online during training. 
Since we do not constrain $P(v^m_i) \neq P(v^n_j)$, we can express a wide array of schedules. 
To make our cost function, cache percentage agnostic, we achieve optimal cache utilization (i.e communication) in the first layer by ensuring that all the vertices are equally partitioned.

\TODO{flattening}

\textit{lemma:} The cost for a sampled graph does not change when a vertex v mapped to partition $i$ is mapped to $j$,if there exists atleast one vertex $u$ in both partitions i,j with an edge to v. 
\textit{Proof:} While moving $v$ to $j$, might at decrease one partition cross edge to j, we will also add a cross edge to $i$. \TODO{remove this if its trivial}
Thus since every vertex has a self edge with its lower layer, we can move the move the vertex to same gpu partition as its lower layer such that $P(v^{l-1}) = P(v^l)$.
This is a form of graph coarsening commonly used in graph partitioning. \cite{karypis1997metis}

\TODO{union upper bound}

\textit{lemma:} The optimal cost of the sampled graph is upper bounded by the cost of the training graph partitioned to minimize communication costs while balancing vertices. 

\textit{proof by contradiction: } We assume that the partitioned training algorithm has a  lower cost has a lower cost than the sampled graph. 
We can construct induced sampled graph from the training graph using the vertices and edges in the sample, while keeping its original partitioning.
From our experiments in motivation we show that a random partition of training vertices can easily cover the training graph with 3 hops even with a highly balanced partitioning such as pagraph. 
Thus this induced graph can be assumed to be balanced. 
If the cost of this induced graph is lower, implies that sampled graph function is not optimal.

Thus a partitioning function which optimizes communication in the training graph allows us to 
 to generate the optimal partitioning of the sampled graph.
Thus we therefore use Metis to split partition the training graph while minimizing the communication costs. 
As the partitioning function through metis can be performed as a preprocessing function, the split partitioning can be quickly performed online. 


 

\section{Implementation} 
\section{Implementation}
\label{sec:impl}

At \company, we have deployed \sysname in our internal clusters to serve daily DL workloads.
The internal clusters consist of heterogeneous GPUs, including NVIDIA T4 GPU and NVIDIA A10 GPU.
Integrated with Kubernetes~\cite{k8s}, \sysname manages thousands of GPUs in each cluster and more than 20,000 GPUs in all.

\parabf{Service manager.}
For online workloads, we use the existing service manager at \company which deploys containers, discovers service, and autoscales horizontal pods.

\parabf{Global manager.}
We modify the Kubernetes scheduler to schedule offline workloads.
The workload profiler takes several dedicated GPUs, whose number is negligible to the total number of GPUs.
When a new offline workload comes, the workload profiler performs a few dry runs of the workload and utilizes the NVIDIA Data Center GPU Manager (DCGM) tools~\cite{dcgm} and NVIDIA Management Library (NVML)~\cite{nvml} libraries to collect GPU metrics.
We collect about 2,000 data for each GPU type to train the speed predictor.
The MLPs of the speed predictor have four layers with hidden size $64\times 64$.
The MLPs are trained with momentum SGD optimizer~\cite{ruder2016overview} in PyTorch v1.8.0~\cite{paszke2019pytorch} until they converge.
\sysname invokes the scheduler periodically to schedule all offline workloads.
When moving workloads, we record checkpoints of offline workloads and restart the workloads after transmitting the models and checkpoints.
As the datasets are usually colossal, we store the datasets in a remote file system and fetch data during the execution.
We implement the scheduler as a third-party plugin to the Kubernetes scheduler.


\parabf{Local executor.}
Each local executor executes online workloads according to the service manager and offline workloads according to the global manager.
DL workloads are executed in Docker containers with our customized components.
We add Best-Effort GPU DevicePlugin in Kubernetes and relevant control paths with Kubelet and \sysprobe for offline workloads.
To control SM percentage, we leverage the environment variable $CUDA\_MPS\_ACTIVE\_THREAD\_PERCENTAGE$ provided by MPS.
The GPU monitor collects resource metrics through DCGM~\cite{dcgm} and NVML~\cite{nvml} for NVIDIA GPU.
The \sysprobe updates the state machine with the collected resource metrics and empirically-set thresholds.
When the state is unhealthy, the \sysprobe will ask the NodeManager in Kubernetes to evict offline workloads.
\bytecuda intercepts nearly 800 CUDA driver APIs for GPU memory allocation and kernel launch.
The GPU memory quota of offline workloads is fixed to $40\%$ as Figure~\ref{fig:motiv_gpu_resource} reports that most online workloads use less than $60\%$ GPU memory.
We adopt the cpuset of Cgroup for CPU isolation.
For memory, \sysname will evict offline workloads if memory usage is higher than a threshold or the kernel swap daemon is busy for a long time.
The parameters to calculate GPU load in Equation~\ref{equ:gpu_load}$\&$\ref{equ:clock_factor} are empirically selected through trial-and-error.


\section{Evaluation} 
We present in section~\ref{ssec:faces} an application of PnP-HVAE on face images, using a pretrained state-of-the-art hierarchical VAE. 
Next, we study the application of our framework to natural images. To that end, we introduce  in section~\ref{ssec:patchVDVAE}  a patch hierachical VAE architecture, that is able to model natural images of different resolutions. In section~\ref{ssec:app_nat}, we provide deblurring, super-resolution and inpainting experiments to demonstrate the relevance of the proposed method.

Additional results are presented in Appendix~\ref{app:add}. All experiments can be reproduced using the code available at \url{https://github.com/jprost76/PnP-HVAE}.



\subsection{Face Image restoration (FFHQ)}\label{ssec:faces}
We first demonstrate the effectiveness of PnP-HVAE on highly structured data, by performing face image restoration.
Latent variable generative models can accurately model structured images such as face images \cite{karras2019style,vahdat2020nvae,child2021very,kingma2018glow}, and then be used to produce high quality restoration of such data. 
In our experiments, we use the VDVAE model of~\cite{child2021very}, pre-trained on the FFHQ dataset~\cite{karras2019style}, as our hierarchical VAE prior.
VDVAE has $L=66$ latent variable groups in its hierarchy and generates images at resolution $256\times256$.

We compare PnP-HVAE with the intermediate layer optimization algorithm (ILO)~\cite{daras2021intermediate} that is based on a different class of generative models than HVAE. ILO is a GAN inversion method which optimizes the image latent code along with the intermediate layer representation of a StyleGAN to generate an image consistent with a degraded observation.
We use the official implementation of ILO, along with a StyleGAN2 model~\cite{karras2020analyzing, stylegan2pytorch}, that was trained for 550k iterations on images of resolution $256\times256$ from FFHQ.  
As VDVAE and StyleGAN models are not trained on the same train-test split of FFHQ, we chose to evaluate the methods on a subset of 100 images from the CelebA dataset~\cite{liu2018large}. 
For super-resolution, the degradation model corresponds to the application of a gaussian low-pass filter followed by a $\times 4$ sub-sampling, and the addition of a gaussian white noise with $\sigma=3$.
For the deblurring, we considered motion blur and  gaussian kernels, both with a noise level $\sigma=8$. %

We provide quantitative comparisons in table~\ref{table:comp_ILO}, along with a visual comparison of the results in figure~\ref{fig:face_restoration}.
PnP-HVAE has the best  PSNR and SSIM results for all the considered restoration tasks, while ILO provides better results  for the perceptual distance.
By jointly optimizing the image and its latent variable, PnP-HVAE provides  results that are both realistic and consistent with the degraded observation.
On the other hand,  ILO  only optimizes on an extended latent space. This method generates  sharp and realistic images with better LPIPS scores,   
but the results lack  of consistency with respect to the observation, which explains the overall lower PSNR performance. 






\subsection{PatchVDVAE: a HVAE for natural images}\label{ssec:patchVDVAE}
Available generative models in the literature operate on images of  fixed resolutions and
are either restrained to datasets of limited diversity, or even to registered face images~\cite{kingma2018glow,child2021very, vahdat2020nvae, karras2019style}, or requiring additional class information~\cite{brock2018large, dhariwal2021diffusion, song2020score, luhman2022optimizing}.
Fitting an unconditional model on natural images appears to be a more difficult task, as their resolution can change, and their content is highly diverse.
The complexity of the problem can be reduced by learning a prior model on patches of reduced dimension. 
For image restoration problems, the patch model can be reused on images of higher dimensions~\cite{zoran2011learning,prost2021learning,altekruger2022patchnr}. When the model is a full CNN, the prior on the set of the  patches can  be computed efficiently by applying the network on the full image~\cite{prost2021learning}.

We thus introduce  patchVDVAE, a fully convolutional hierarchical VAE.
Contrary to existing HVAE models whose resolution is constrained by the constant tensor at the input of the top-down block, patchVDVAE can generate images of different resolutions by controlling the dimension of the input latent. 
This amounts to defining a prior on patches whose dimension corresponds to the receptive field of the VAE. A similar model is used for image denoising in~\cite{prakash2021interpretable}.

 
For PatchVDVAE architecture, we use the same bottom-up and top-down blocks as VDVAE~\cite{child2021very}, and replace the constant trainable input in the first top-down block by a latent variable, to make the model fully convolutional (details on the  architecture are given in Appendix~\ref{app:details}). 
The training dataset is composed of $128\times 128$ patches extracted from a combination of DIV2K~\cite{agustsson2017ntire} and Flickr2K~\cite{Lim_2017_CVPR_workshops} datasets.
We perform data augmentation by extracting  patches at $3$ resolutions: HR-images and $\times 2$ and $\times 4$ downscaled images. 
The model is trained for $7.10^5$ iterations with a batch size of $64$. Following the recommendation of~\cite{hazami2022efficient}, we use Adamax optimizer with an exponential moving average and gradient smoothing of the variance.
We set the decoder model to be a gaussian with diagonal covariance, as in~\cite{luhman2022optimizing}.
PatchVDVAE is fully convolutional and can generate images of dimension that are multiples of $64$ as illustrated by
figure~\ref{fig:vdvae}.

\newlength{\patchwidth}
\setlength{\patchwidth}{0.135\columnwidth}
\begin{figure}[!ht]
    \centering
    \begin{subfigure}[t]{.34\columnwidth}\hspace{0.1cm}
        \setlength{\tabcolsep}{0.02pt}
\renewcommand{\arraystretch}{0}
        \begin{tabular}{*{2}{p{1.03\patchwidth}}}
            \includegraphics[width=\patchwidth]{figures_arxiv/patchVDVAE/samples/generated/64x64/setup-5-image-0018.png} &
            \includegraphics[width=\patchwidth]{figures_arxiv/patchVDVAE/samples/generated/64x64/setup-5-image-0016.png} \\
            \includegraphics[width=\patchwidth]{figures_arxiv/patchVDVAE/samples/generated/64x64/setup-5-image-0008.png} &
            \includegraphics[width=\patchwidth]{figures_arxiv/patchVDVAE/samples/generated/64x64/setup-5-image-0019.png}   
        \end{tabular}
    \end{subfigure}\hspace{-0.15cm}
    \begin{subfigure}[t]{.64\columnwidth}
\begin{tabular}{cc}\vspace{-0.1cm}
\includegraphics[width=2\patchwidth]{figures_arxiv/patchVDVAE/samples/generated/256x256/setup-2-image-0009.png}&
        \includegraphics[width=2\patchwidth]{figures_arxiv/patchVDVAE/samples/generated/256x256/setup-2-image-0002.png}\end{tabular}

    \end{subfigure}
    \caption{\label{fig:vdvae} Left: $64\times64$ patches samples from our patchVDVAE model trained on patches from natural images.
    Right: PatchVDVAE is fully convolutional and it can generate images of higher resolution (here: $128\times128$).\vspace{-0.2cm}}
\end{figure}

\subsection{Natural images restoration}\label{ssec:app_nat}
We  evaluate PnP-HVAE on natural image restoration.
For each task, we report the average value of the PSNR, the SSIM, and the LPIPS metrics on $20$ images from the test set of the BSD dataset~\cite{MartinFTM01}.\\


\noindent
{\bf Image deblurring.}
In the experiments, we consider $2$ gaussian kernels and $2$ motion blur kernels from~\cite{levin2009understanding}, with $3$ different noise levels 
$\sigma \in \{2.55, 7.65, 12.75\}$.
As a baseline we consider  EPLL~\cite{zoran2011learning}, which learns a prior on image patches with a gaussian mixture model.
We also compare PnP-HVAE  with PnP-MMO and GS-PnP, $2$ competing convergent Plug-and-Play methods based on CNN denoisers.
PnP-MMO~\cite{pesquet2021learning} restricts the denoiser to be contraction in order to guarantee the convergence of the PnP forward-backard algorithm. GS-PnP~\cite{hurault2022gradient} considers a gradient step denoiser and reaches state-of-the-art performances of non converging methods~\cite{zhang2021plug}.
We set the temperature $\tau$  in our method as $0.95$, $0.8$ and $0.6$ for noise levels $2.55$, $7.65$ and $12.75$ respectively, and we let it run for a maximum of $50$ iterations. 
For the three compared methods we use the official implementations and pre-trained models provided by the respective authors. 
Details on the choice of hyperparameters for the concurrent methods are provided in the Appendix~\ref{app:details}
Figure~\ref{fig:deblurring_bsd} illustrates that our method provides correct deblurring results. 

According to table~\ref{tab:deb}, the performance of PnP-HVAE is between those of EPLL and GS-PnP and it outperforms PnP-MMO for large noise levels.\\

\begin{table}
\begin{center}\footnotesize
    \begin{tabular}{>{\centering}m{.3cm}*{5}{c}}
    $\sigma$ &Method & PSNR$\uparrow$ & SSIM$\uparrow$ & LPIPS$\downarrow$  \\ 
    \hline
    \multirow{4}{*}{\vcell{$2.55$}}
    & PnP-HVAE & $27.75$ & $0.79$ & $0.31$\\
    & GS-PNP \cite{hurault2022gradient} & $\mathbf{29.59}$ & $\mathbf{0.84}$ & $\mathbf{0.22}$\\
    & EPLL \cite{zoran2011learning} & $26.49$ & $0.71$ & $0.36$\\ 
    & PnP-MMO \cite{pesquet2021learning} & $\underbar{29.50}$ & $\underbar{0.83}$ & $\underbar{0.20}$ \\ \hline
    \multirow{4}{*}{\vcell{$7.65$}}
    & PnP-HVAE & $\underbar{26.36}$ & $\underbar{0.72}$ & $\underbar{0.40}$\\
    & GS-PNP \cite{hurault2022gradient} & $\mathbf{27.33}$ & $\mathbf{0.77}$ & $\mathbf{0.31}$\\
    & EPLL \cite{zoran2011learning} & $24.04$ & $0.66$ & $0.45$ \\ 
    & PnP-MMO \cite{pesquet2021learning} & $25.34$ & $0.69$ & $0.34$\\
    \hline
    \multirow{4}{*}{\vcell{$12.75$}}
    & PnP-HVAE & $\underbar{25.12}$ & $\mathbf{0.73}$ & $\underbar{0.47}$\\
    & GS-PNP \cite{hurault2022gradient} & $\mathbf{26.32}$ & $\mathbf{0.73}$ & $\mathbf{0.37}$\\
    & EPLL \cite{zoran2011learning} & $23.28$ & $0.61$ & $0.51$ \\ 
    & PnP-MMO \cite{pesquet2021learning} & $22.42$ & $0.53$& $0.54$ \\
    \hline
    &\vspace*{-.3cm}\\
            \multicolumn{2}{c}{Blur and motion kernels}& \multicolumn{3}{c}{
        \includegraphics*[scale=1]{figures_arxiv/kernels/4.png}\;\includegraphics*[scale=1]{figures_arxiv/kernels/7.png}\;\includegraphics*[scale=1]{figures_arxiv/kernels/9.png}\;\includegraphics*[scale=1]{figures_arxiv/kernels/11.png}} 
    \end{tabular}
        \caption{\label{tab:deb}Comparison  of PnP-HVAE  and other restoration methods on deblurring. Results are averaged on $4$ kernels.\vspace{-0.2cm}}% on image deblurring.}
    \end{center}
\end{table}

\begin{figure}
    
    \begin{subfigure}[h]{\linewidth}
        \centering
        \includegraphics*[width=\columnwidth]{figures_arxiv/deb_s255_k7.pdf}\vspace{-0.1cm}
        \caption{Gaussian blur, $\sigma=2.55$}
    \end{subfigure}
    \begin{subfigure}[h]{\linewidth}
        \centering
        \includegraphics*[width=\columnwidth]{figures_arxiv/deb_s765_k11.pdf}\vspace{-0.1cm}
        \caption{Motion blur, $\sigma=7.65$}
    \end{subfigure}\vspace*{-0.1cm}
    \caption{\label{fig:deblurring_bsd} Natural image deblurring\vspace{-0.1cm}}
\end{figure}

\noindent {\bf Effect of the temperature.}
PnP-HVAE gives control on the temperature of the prior over the latent space.
In figure~\ref{fig:temp_effect}, we illustrate that reducing the temperature increases the strength of the regularization prior. In this example the tuning $\tau=0.7$ produces the best performance.\\
\begin{figure}[!ht]
   
    \includegraphics[width=\columnwidth]{figures_arxiv/demo_temp.pdf}\vspace{-0.15cm}
    \caption{ \label{fig:temp_effect} Effect of the temperature in PnP-VAE on a deblurring problem, with $\sigma=7.65$.\vspace{-0.15cm}}
\end{figure}


\noindent
{\bf Image inpainting.}
Next we consider the task of noisy image inpainting. 
We compose a test-set of 10 images from the validation set of BSD~\cite{MartinFTM01} and we create masks
  by occluding diverse objects of small size in the images. 
A gaussian white noise with $\sigma=3$ is added to the images.
As a comparaison, we still consider GS-PnP and EPLL.
For PnP-HVAE, the temperature is set to $\tau=0.6$, and the algorithm is run for a maximum of $200$ iterations, unless the residual $||\x_{k+1}-\x_k||$ is on a plateau.
We provide on Table~\ref{tab:inpainting_bsd} the distortion metrics with the ground truth, as well as a visual
\begin{table}



\begin{center}
    \begin{tabular}{cccc}
        & PSNR$\uparrow$ & SSIM$\uparrow$ &LPIPS$\downarrow$ \\\hline
        PnP-HVAE  & $\mathbf{29.54}$ & $\mathbf{0.93}$ & $\mathbf{0.06}$\\
        GS-PNP & $28.52$ & $\mathbf{0.93}$ & $0.09$\\
        EPLL & $\underline{29.16}$ & $\mathbf{0.93}$ & $\mathbf{0.06}$\\
    \end{tabular}
    \caption{\label{tab:inpainting_bsd}Quantitative evaluation for inpainting on BSD.}
    \end{center}
\end{table}
comparison on figure~\ref{fig:inpainting_bsd}. 
With its hierarchical structure,  PnP-HVAE outperforms the compared methods. \vspace{0.05cm}



\begin{figure}[!h]
    \includegraphics[width=\columnwidth]{figures_arxiv/demo_inp_bsd2.pdf}\vspace{-0.1cm}
    \caption{\label{fig:inpainting_bsd}Natural image inpainting\vspace{-0.3cm}}
\end{figure}












\section{Related Work} 
\section{Related work}
\noindent \textbf{Video foundation models.}
With sufficient computational power and an abundant source of data, there have been attempts to build a single large-scale foundation model that can be adapted to diverse downstream tasks.
Along with the success of foundations models in the natural language processing domain~\cite{brown2020language,chen2021evaluating,devlin2019bert} and in computer vision~\cite{bertasius2021space,jia2021scaling,radford2021learning}, video data has become another data type of interest, as it has grown in scale due to numerous internet video-sharing platforms.
Accordingly, several methods to train a video foundation model have been proposed.
Due to the innate multi-modality of video data, \textit{i.e.}, a combination of visual $\cdot$ vocal $\cdot$ textual context, most works have centered around the variations of the cross-modal attention mechanism \cite{akbari2021vatt,bertasius2021space,gabeur2020multi,luo2020univl,neimark2021video,tan2021look,wei2020multi,yang2021taco}.
In addition, as most video data lack proper labels or descriptions, contrastive learning methods were studied to learn meaningful feature representations or enhance video-text alignment in a self-supervised manner \cite{akbari2021vatt,kuang2021video,luo2020univl,yang2021taco}.

More specifically, MERLOT \cite{zellers2021merlot} proposed a multi-modal representation learning method for visual commonsense reasoning, which also performed well in twelve video reasoning tasks.
VATT \cite{akbari2021vatt} introduced a multi-modal learning method via contrastive learning. 
The pre-trained model performed well in a variety of vision tasks from image classification to video action recognition and zero-shot video retrieval.
Another representative work, UniVL \cite{luo2020univl} proposed a straightforward pre-training method with auxiliary loss functions. 
After fine-tuning on a specific task, the pre-trained model performed outstandingly in a wide range of tasks of text-to-video retrieval, action segmentation, action step localization, video sentiment analysis, and video captioning.
Other foundation models for multiple video tasks include \cite{li2020hero,sun2019learning,sun2019videobert,zhu2020actbert,fu2021violet,wang2022all}. 

\noindent \textbf{Auxiliary learning.}
In order to enhance the performance of one or a multitude of primary tasks, auxiliary learning methods can be incorporated.
\cite{ruder2017overview} introduced Multi-task learning (MTL) to the deep neural networks by training a single model with multiple task losses to assist learning on the main task.
Such a method is generally adapted to pre-train the foundation models in the self-supervised manner~\cite{li2020hero,sun2019learning,sun2019videobert,zhu2020actbert,fu2021violet,wang2022all}.
However, these various pretext task losses used in the pre-training phase are ignored in the fine-tuning phase, and only the primary task loss is minimized.

Recently, meta-learning methods have been introduced for auxiliary learning.
\cite{liu2019self,navon2020auxiliary,shu2019meta} proposed a meta-learning method in which the model learns auxiliary tasks to generalize well to unseen data. 
In these settings, a separate subset of data is held out as the primary task, while the others are used as auxiliary tasks that aid the primary task's performance.
Similar methods were adopted for computer vision tasks such as semantic segmentation \cite{xu2021leveraging}.
Other domain applications include navigation tasks with reinforcement learning \cite{ye2021auxiliary}, or self-supervised learning methods on graph data \cite{hwang2020self}.

\section{Conclusions} 
\section{Conclusions}
We consider the phase-extraction problem, and we showed that, given a unitary $U = e^{i\pi H}$ and its inverse $U^{\dag}$, we could implement a block-encoding of $\phi(H)$ for some smooth function $\phi(x)$. The word `smooth' here means existence and continuity of the derivatives: the higher the number of continuous derivatives that a function has, the faster its Fourier sum (and thus the Laurent polynomial on the eigenphases) uniformly converges to that function. We are confident this can have many more applications beyond what is shown in this work. It is also worth remarking that Jackson showed that the convergence rate of a Fourier series is almost-optimal, in the sense that no trigonometric (or, equivalently, complex exponential) series can approximate the desired function faster, up to that $\log d$ factor~\cite[p.\ 21]{jacksonTheoryApproximation1930a}. Also remember that `smoothing' a function, i.e., replacing its derivative with a continuous function, does not give faster convergence for free in general, as its derivative will become steep in the points where we smooth out discontinuities, and this translates to a high Lipschitz constant: a~clear example is given by Eq.~\ref{eq:lipschitz-constant-recurrence-solution}, but in that case, fortunately, nothing depends on the size of the input $N$, and thus does not influence the asymptotic query complexity of Algorithm~\ref{alg:prop-sampling-qsp}, although the constant factor can become large even for $p = 20$. From a theoretical point of view, this work shows that, for any $\eta > 0$, there is an algorithm with query complexity 
$$\Tilde{\bigO}\left(\frac{1}{\bar{c}^{\frac{1}{2} + \eta}} \frac{1}{\epsilon^\eta} \right)$$
solving the proportional-sampling problem. This statement seems to suggest there exists an algorithm which directly solves the problem with $\eta = 0$, and an open question would be to find such algorithm.


It is also interesting to remark that Theorems~\ref{thm:haah-construction},~\ref{thm:haah-completion} indeed allow the construction for any $\phi$, even complex-valued, provided that its absolute value is reciprocal.

One could think that, in Section~\ref{sec:prop-sampling}, instead of using the linear function in the phase-extraction subroutine, we could approximate the square root and then apply the transformation directly on $e^{i \pi c(x)}$. However, in the case of proportional sampling this would be inconvenient, as the derivative of the square root function has a discontinuity with an infinite jump around 0, and we could not choose a constant $\delta$ if we had values of the oracle that are too close to $0$.

\bibliographystyle{plain}
\bibliography{ref}
\end{document}