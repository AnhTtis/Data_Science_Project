% \documentclass[sigconf, nonacm]{acmart}
\documentclass[sigplan,nonacm]{acmart}
%% Fonts used in the template cannot be substituted; margin 
%% adjustments are not allowed.
%%
%% \BibTeX command to typeset BibTeX logo in the docs
\AtBeginDocument{%
  \providecommand\BibTeX{{%
    \normalfont B\kern-0.5em{\scshape i\kern-0.25em b}\kern-0.8em\TeX}}}

%% Rights management information.  This information is sent to you
%% when you complete the rights form.  These commands have SAMPLE
%% values in them; it is your responsibility as an author to replace
%% the commands and values with those provided to you when you
%% complete the rights form.
\setcopyright{acmcopyright}
\copyrightyear{2018}
\acmYear{2018}
\acmDOI{XXXXXXX.XXXXXXX}

%% These commands are for a PROCEEDINGS abstract or paper.
\acmConference[Conference acronym 'XX]{Make sure to enter the correct
  conference title from your rights confirmation emai}{June 03--05,
  2018}{Woodstock, NY}
%
%  Uncomment \acmBooktitle if th title of the proceedings is different
%  from ``Proceedings of ...''!
%
\acmBooktitle{Woodstock '18: ACM Symposium on Neural Gaze Detection,
 June 03--05, 2018, Woodstock, NY} 
\acmPrice{15.00}
\acmISBN{978-1-4503-XXXX-X/18/06}
% enable page numbers
\settopmatter{printfolios=true}

\usepackage{cite}
\usepackage{amsmath,amsfonts}
\usepackage{algorithmic}
\usepackage{graphicx}
\usepackage{textcomp}
\usepackage{xcolor}
\usepackage{mathtools}
\usepackage{xspace}

\usepackage{microtype}
\usepackage{graphicx}
\usepackage{subcaption}
\usepackage{booktabs} 
\usepackage{hyperref}
\hypersetup{ 
    colorlinks=true, 
    linkcolor=blue,
    anchorcolor=black,
    citecolor=purple,
    filecolor=cyan,
    menucolor=red,
    runcolor=cyan,
    urlcolor=magenta
} 
\usepackage{multirow}
\usepackage{bm}
\usepackage{comment}
\usepackage{apptools}
% \usepackage{subfig}
\usepackage{xspace}
\usepackage[ruled,noline,noend,linesnumbered]{algorithm2e}
\usepackage{amsmath}

% \usepackage[T1]{fontenc}
% \usepackage[utf8]{inputenc}
% \usepackage{authblk}
\DeclareMathOperator*{\argmin}{arg\,min}

\newcommand{\name}{GSplit\xspace}
\newcommand{\TName}{Split Parallelism\xspace}
\newcommand{\Tname}{Split parallelism\xspace}
\newcommand{\tname}{split parallelism\xspace}
\newcommand{\Tadj}{Split-parallel\xspace}
\newcommand{\TAdj}{Split-Parallel\xspace}
\newcommand{\tadj}{split-parallel\xspace}
\newcommand{\ds}{mixed frontier\xspace}
\newcommand{\dss}{mixed frontiers\xspace}
\newcommand{\DS}{Mixed Frontier\xspace}
\newcommand{\DSs}{Mixed Frontiers\xspace}
\newcommand{\Ds}{Mixed frontier\xspace}
\newcommand{\Dss}{Mixed frontiers\xspace}

\newcommand{\mypar}[1]{\vspace*{0.05in}\noindent\textbf{#1.}}
\newcommand{\outline}[1]{{\color{blue}\bfseries\textit{\{#1\}}}}

\newcommand{\hui}[1]{{\color{blue}\bfseries\textit{\{Hui: #1 \}}}}
\newcommand{\ms}[1]{{\color{red}\bfseries\textit{\{Marco: #1\}}}}
\newcommand{\sand}[1]{{\color{teal}\bfseries\textit{\{Sandeep: #1 \}}}}
\newcommand{\juelin}[1]{{\color{olive}\bfseries\textit{\{Juelin: #1 \}}}}
\newcommand{\todo}[1]{{\color{red}\bfseries\textit{\{TODO: #1 \}}}}
\newcommand{\TODO}[1]{{\color{red}\bfseries\textit{\{TODO: #1\}}}}

\DeclareMathOperator*{\E}{\mathbb{E}} % expected value \E{}

\def\BibTeX{{\rm B\kern-.05em{\sc i\kern-.025em b}\kern-.08em
    T\kern-.1667em\lower.7ex\hbox{E}\kern-.125emX}}
% make references clickable 
\usepackage[]{hyperref}

\begin{document}
\title{\name: Scaling Graph Neural Network Training on Large Graphs via Split-Parallelism}
\author{Sandeep Polisetty, Juelin Liu, Jakob Falus}
\affiliation{%
  \institution{University of Massachusetts Amherst}
  % \city{Amherst}
  % \state{Massachusetts}
  % \postcode{01002}
}
\email{{spolisetty, juelinliu, kobi}@cs.umass.edu}

\author{Yi Ren Fung}
\affiliation{%
  \institution{University of Illinois Urbana-Champaign }
  % \city{Urbana}
  % \state{Illinois}
  % \postcode{61820 }
}
\email{yifung2@illinois.edu}

\author{Seung-Hwan Lim}
\affiliation{%
  \institution{Oak Ridge National Laboratory}
  % \city{TODO: update}
  % \state{Tennessee}
  % \postcode{TODO: update}
}
\email{lims1@ornl.gov}


\author{Hui Guan, Marco Serafini}
\affiliation{%
  \institution{University of Massachusetts Amherst}
  % \city{Amherst}
  % \state{Massachusetts}
  % \postcode{01002}
}
\email{{huiguan, marco}@cs.umass.edu}


\begin{abstract}
    Graph neural networks (GNNs), an emerging class of machine learning models for graphs, have gained popularity for their superior performance in various graph analytical tasks. Mini-batch training is commonly used to train GNNs on large graphs, and data parallelism is the standard approach to scale mini-batch training across multiple GPUs. One of the major performance costs in GNN training is the loading of input features, which prevents GPUs from being fully utilized.
    In this paper, we argue that this problem is exacerbated by redundancies that are inherent to the data parallel approach. To address this issue, we introduce a hybrid parallel mini-batch training paradigm called \tname. \Tname avoids redundant data loads and splits the sampling and training of each mini-batch across multiple GPUs online, at each iteration, using a lightweight splitting algorithm. We implement \tname in \name and show that it outperforms state-of-the-art mini-batch training systems like DGL, Quiver, and $P^3$.
\end{abstract}

\maketitle
\pagestyle{plain} % should come right after \maketitle

The advance of Pre-trained Language Models (PLMs) like GPT-3 \cite{brown2020language} and LLaMA \cite{DBLP:journals/corr/abs-2302-13971} has substantially improved the performance of deep neural networks across a variety of Natural Language Processing (NLP) tasks. Various language models, based on the Transformer \cite{vaswani2017attention} architecture,  have been proposed, leading to state-of-the-art (SOTA) performance on the fundamental discrimination tasks. These models are first trained with self-supervised training objectives (e.g., predicting masked tokens according to surrounding tokens) on massive unlabeled text data, then fine-tuned on annotated data to adapt to downstream tasks of interest.  However, annotated data is usually limited to a wide range of downstream tasks, which results in overfitting and a lack of generalization to unseen data.

One straightforward way to deal with this data scarcity problem is data augmentation , and incorporating generative models to perform data augmentation has been widely adopted recently . Despite its popularity, the generated text can easily deviate from the real data distribution without exploiting any of the signals passed back from the discrimination task. In previous studies, generative data augmentation and discrimination have been well studied as separate problems, but it is less clear how these two can be leveraged in one framework and how their performances can be improved simultaneously. \looseness=-1

Generative Adversarial Networks (GANs) \cite{https://doi.org/10.48550/arxiv.1406.2661} are good attempts to couple generative and discriminative models in an adversarial manner, where a two-player minimax game between learners is carefully crafted. GANs have achieved tremendous success in domains such as image generation , and related studies have also shown their effectiveness in semi-supervised learning. However,  in the text field, GANs are difficult to train, most training objectives work well for only one model, either the discriminator or the generator, so rarely both learners can be optimal at the same time. This essentially arises from the adversarial nature of GANs, that during the process, optimizing one learner can easily destroy the learning ability of the other, making GANs fail to converge.

Another limitation of simultaneously optimizing the generator and the discriminator comes from the discrete nature of text in NLP, as no gradient propagation can be done from discriminators to generators. One theoretically sound attempt is to use reinforcement learning (RL), but the sparsity and the high variance of the rewards in NLP make the training particularly unstable \cite{caccia2019language}. 

To address these shortcomings, we novelly introduce a self-consistent learning framework based on one generator and one discriminator: the generator and the discriminator are alternately trained by way of cooperation instead of competition, and the selected samples are used as the medium to pass the feedback signal from the discriminator. Specifically, in each round of training, the samples generated by the generator are synthetically labeled by the discriminator, and then only part of them would be selected based on dynamic thresholds and used for the training of the discriminator and the generator in the next round. Several benefits can be discovered from this cooperative training process. First, a closed-loop form of cooperation can be established so that we can get the optimal generator and discriminator at the same time. Second, this framework helps improve the generation quality while ensuring the domain specificity of generator, which in turn contributes to training. Third, a steady stream of diverse synthetic samples can be added to the training in each round and lead to continuous improvement of the performance of all learners. Finally, we can start the training with only domain-related corpus and obtain strong results, while these data can be easily sampled with little cost or supervision. Also, the performance on labeled datasets can be further boosted based on the strong baselines. As an example to demonstrate the effectiveness of our framework in the text field, we examine it on four downstream text generation benchmarks, including AFQMC, CHIP-STS, QQP, and MRPC. The experiments show that our method significantly improves over standalone state-of-the-art discriminative models on zero-shot and full-data settings.

Our contributions are summarized as follows,

$\bullet$ We propose a self-consistent learning framework in the text field that incorporates the generator and the discriminator, in which both achieve remarkable performance gains simultaneously.

$\bullet$ We propose a dynamic selection mechanism such that cooperation between the generator and the discriminator drives the convergence to reach their scoring consensus.

$\bullet$ Experimental results show that the generator in our framework can continuously adjust its generation samples based on the performance of downstream tasks, while the discriminator can outperform the strong baselines.


\section{Background and Motivation}
This section first introduces mini-batch GNN training, 
% in Section~\ref{sec:background}.
and then 
% discusses the data loading-induced performance bottleneck and 
elaborates on the limitations of existing optimizations. 
% , such as caching and hybrid parallelism, 
% in Section~\ref{sec:problem}.

\begin{figure}[ht]
    \centering
    \includegraphics[width=\linewidth]{figures/mini-batch.pdf}
    \caption{Example of a mini-batch.}
    \label{fig:minibatch}
\end{figure}

\subsection{Mini-Batch GNN Training and Data Parallelism} 
A GNN model is defined as a sequence of \emph{GNN layers}.\footnote{In the following, the term ``layer'' will refer to GNN layers, not to neural network layers unless otherwise stated.}
During each mini-batch training iteration, there are three phases: \textit{sampling}, \textit{loading}, and \textit{training}.
The sampling phase randomly selects a mini-batch starting from the target vertices.
A mini-batch with two target vertices is shown in Figure~\ref{fig:comparison}(a). 
In the loading phase, the input features of the vertices in the bottom layer of the mini-batch are loaded into the GPUs.
During forward propagation, each GNN layer $l > 0$ aggregates and transforms the features of the vertices in the layer $l-1$ of the sample and produces the features of the vertices in the layer $l$ (see Figure~\ref{fig:minibatch}).
The last GNN layer computes the features of the target vertices, which are then used to calculate the loss. 
During backward propagation, the layers are executed in reverse order to compute gradients. 
Finally, all GPUs aggregate and apply the computed gradients.


\begin{figure}[ht]
    \centering
    \begin{subfigure}[t]{0.45\linewidth}
        \centering
        \includegraphics[width=\textwidth]{results/breakdown/orkut_epoch_breakdown.pdf}
        \caption{The fraction of sampling, loading, and training time per epoch of DGL, P3* and Quiver on Orkut with the GAT model.}
        \label{fig:orkut_epoch_breakdown}
    \end{subfigure}
    \hfill
    \begin{subfigure}[t]{0.45\linewidth}
        \centering
        \includegraphics[width=\textwidth]{results/breakdown/quiver_epoch_breakdown.pdf}
        \caption{The percentage of sampling, loading, and training time per epoch of Quiver on Orkut and Papers100M with the GraphSage model. }
        \label{fig:quiver_epoch_breakdown}
    \end{subfigure}
    \caption{Epoch time breakdown of existing systems on a four-GPU host with NVLink.}
    \label{fig:epoch_breakdown}
\end{figure}

% \begin{figure}[h!]
%     \centering
%     \advance\leftskip-1cm
% % \advance\rightskip-3cm
% \includegraphics[keepaspectratio=true,width=.5\textwidth]{results/epoch_breakdown/sage_epoch_breakdown_arxiv.png}
%     % \includegraphics[width=0.85\columnwidth]{results/epoch_breakdown/sage_epoch_breakdown.png}
%      \caption{The fraction of sampling, loading, and training time per epoch of existing systems when training the GraphSage model on a four GPU host with NVLink. \ms{New figures. This figure is out of bounds. Larger fonts. Edit text as needed}
%      %DGL and P3 cache no feature data in all experiments. Quiver caches 44\% for Papers100M, and 12\% percent for Friendster.  All systems use the GAT model with fanouts [15, 15, 15] and batch size 1024. 
%      }
%     \label{fig:epoch_breakdown}
% \end{figure}

Data parallelism is the most commonly used training strategy for mini-batch GNN training. 
In data parallel training, the target vertices are partitioned among GPUs, where each partition corresponds to a separate \textit{micro-batch} (see Figure~\ref{fig:comparison}(a)). 
Each GPU independently loads the input features of all the vertices in the bottom layer of its micro-batch and trains on it.
This approach has two limitations: a high cost of data loading and a high degree of computational and data loading redundancy.

% Instead of creating multiple independent and overlapping micro-batches as done by data parallelism, \tname generates a single mini-batch for all the target vertices and splits it without overlaps, avoiding redundant computation and loads.

\mypar{Data loading bottleneck}
Input feature loading is a major overhead in data-parallel GNN training, which contributes to a large fraction of the total training time and prevents a good utilization of the GPUs.
% When training a GNN using a mini-batch with N target vertices, the GPUs need to load the feature vectors of the vertices in the k-hop neighborhood of these target vertices into their memory. 
% 
Figure \ref{fig:orkut_epoch_breakdown} shows the time breakdown of sampling, feature loading, and forward/backward pass per epoch with three GNN systems: DGL, Quiver, and P3*.
We will initially focus on DGL, which is a standard data-parallel baseline, and discuss optimizations shortly.
% We observe that data loading can take up to 69\% of the epoch time for DGL, 48\% for P3*, and 79\% for Quiver. 
We observe that data loading can take more than 60\% of the epoch time for DGL. 
This data loading-induced performance bottleneck is also reported in other GNN training literature~\citep{quiver, pagraph, wholegraph,ugache}. 

\mypar{Redundant loading and computation}
Table~\ref{tab:redundancy} further reports the degree of computational and data loading redundancy in data-parallel training.
With 4 GPUs, data parallelism creates 4 separate micro-batches (``Micro''), causing up to $1.2\times$ compute and $2.5\times$ feature loading compared to having only a single mini-batch (``Mini'').

% \begin{table}[t]
% \centering 
% \tabcolsep=0.08cm
% \begin{tabular}{|l||c|c|c|}
% \hline 
% \textbf{Dataset} & \textbf{4x Micro} & \textbf{1x Mini} & \textbf{\% redundancy} \\ \hline \hline
% % ogbn-products & 195M& 155M & 25.5\% \\
% orkut & 926M & 751M & 23\% \\
% % papers100M & 327M & 131M & 148.9\% \\
% papers100M & 452M & 389M & 16\% \\
% % friendster & 452M & 389M & 16\% \\
% % amazon & 473M & 263M & 79.2\% \\
% \hline
% \end{tabular}
% \caption{Redundant computation. The total number of edges computed over one epoch when each mini-batch is sampled as 4 micro-batches of size 1024 (\texttt{4x Micro}) vs. 1 mini-batch of size 4096 (\texttt{1x Mini}). \ms{Numbers for the other graphs?}}
% \label{tab:overlap-edges} 
% \end{table}

\begin{table}[ht]
\centering
% \resizebox{\columnwidth}{!}{%
\small 
\tabcolsep=0.02cm
\begin{tabular}{|c|ccc|ccc|}
\hline
\multirow{2}{*}{\textbf{Graph}} & \multicolumn{3}{c|}{\textbf{\# Edges Computed}}                                                 & \multicolumn{3}{c|}{\textbf{\# Feature Vectors Loaded}}                                                \\ \cline{2-7} 
                                & \multicolumn{1}{c|}{\textbf{ Micro}} & \multicolumn{1}{c|}{\textbf{ Mini}} & \textbf{Ratio} & \multicolumn{1}{c|}{\textbf{ Micro}} & \multicolumn{1}{c|}{\textbf{ Mini}} & \textbf{Ratio} \\ \hline
\textbf{Orkut}                  & \multicolumn{1}{c|}{926M}              & \multicolumn{1}{c|}{751M}             & 1.2x           & \multicolumn{1}{c|}{422M}              & \multicolumn{1}{c|}{169M}             & 2.5x          \\ \hline
\textbf{Papers100M}             & \multicolumn{1}{c|}{452M}              & \multicolumn{1}{c|}{389M}             & 1.2x          & \multicolumn{1}{c|}{231M}              & \multicolumn{1}{c|}{154M}             & 1.5x           \\ \hline
\textbf{Friendster}             & \multicolumn{1}{c|}{13.4B}             & \multicolumn{1}{c|}{13.1B}            & 1.0x          & \multicolumn{1}{c|}{11.4B}             & \multicolumn{1}{c|}{9.4B}             & 1.2x           \\ \hline
\end{tabular}%
% }
\caption{Redundant computation and data loading. The total number of edges computed and feature data loaded over one epoch when each mini-batch is sampled as 4 micro-batches of size 1024 (Micro) vs. 1 mini-batch of size 4096 (Mini).}
\vspace{-0.2cm}
\label{tab:redundancy}
\end{table}




% For example, in Figure~\ref{fig:comparison}(a), although there are only two target vertices in that mini-batch, a total of eight input features need to be loaded into the GPU memory for forward and backward propagation on a 2-layer GNN. 

% The data loading phase can take up a significant portion of GNN training time and even become the performance bottleneck, depending on the system, the graph characteristics, and the GNN model.


\subsection{Limitations of Existing Optimizations} \label{sec:problem}

\begin{comment}
Table~\ref{tab:loading-bottleneck} shows the time spent on the three phases, sampling, data loading, and training, when training two GraphSAGE and GAT models on two graph datasets, Papers100M and Orkut, using DGL~\citep{dgl} on a four-GPU server with NVLink.
DGL is one of the well-established GNN training frameworks that support mini-batch data parallel training. 
Details on the experiment settings are in Section~\ref{sec:settings}.
% DGL doesn't support caching only a part of a graph in GPU memory and both graphs we consider are too big to fit. 
From the table, we observe that data loading can take up to XX\% of the epoch time. \ms{update number}
This data loading-induced performance bottleneck is also reported in other GNN training literature~\citep{quiver, pagraph}. 

\begin{table*}[t]
\small 
\begin{tabular}{|c||c|c||c|c|c|c||c|c|c|c|}
\hline 
\multirow{2}{*}{ Graph }  & \multirow{2}{*}{ System}  & \multirow{2}{*}{ Cache \%} &\multicolumn{4}{|c||}{SAGE} & \multicolumn{4}{|c|}{GAT} \\
\cline{4-11}
 & & & S  & L  & FB & Total & S  & L  & FB  & Total  \\  
\hline \hline  
 & DGL & no cache & 6.57 & 14.46 & 11.89 & 32.92 & 6.36 & 14.42 & 31.36 & 52.14\\
Papers100M & Quiver & SAGE: 51\% - GAT: 44\%  &11.17 & 17.23 & 8.99 & 37.38 & 11.19 & 22.09 & 25.06 & 58.33 \\
& $P^{3*}$ & OOM & OOM & OOM & OOM & OOM & OOM & OOM & OOM & OOM \\
\hline
& DGL & no cache &0.96 & 96.59 & 13.29 & 110.84 & 0.98 & 96.32 & 17.55 & 114.85 \\
Orkut & Quiver & 100\% & 4.08 & 6.85 & 12.81 & 23.74 & 3.83 & 6.91 & 17.44 & 28.18\\
& $P^{3*}$ & 100\% & 0.97 & 2 & 15.72 & 18.68 & 1.01 & 1.83 & 25.89 & 28.73 \\
\hline
\end{tabular}
\caption{Data loading bottleneck on a system with NVLink. 
\ms{Explain S/L/T once finalized. make it look less similar to the evaluation results (since they are the same numbers). }
}
\label{tab:loading-bottleneck}
\end{table*}
% 
\end{comment}

% \begin{figure}
%     \centering
%     \includegraphics[width=0.49\columnwidth]{results/epoch_breakdown/Papers100M_epoch_breakdown.png}
%     \includegraphics[width=0.49\columnwidth]{results/epoch_breakdown/Friendster_epoch_breakdown.png}

%      \caption{The fraction of sampling, loading, and training time per epoch of existing systems when training GAT on a 4 GPU host with NVLink. 
%      %DGL and P3 cache no feature data in all experiments. Quiver caches 44\% for Papers100M, and 12\% percent for Friendster.  All systems use the GAT model with fanouts [15, 15, 15] and batch size 1024. 
%      }
%     \label{fig:epoch_breakdown}
% \end{figure}


% \subsection{Existing Optimizations} \label{sec:limitations}
% \mypar{Existing optimizations}
Many approaches have been proposed to address the data loading-induced performance bottleneck of mini-batch training. 
These approaches fall broadly into three categories: caching~\citep{pagraph, gnnlab, quiver, wholegraph}, hybrid parallelism~\citep{gandhi2021p3}, and algorithmic optimizations~\citep{dong2021global,twolevel, ramezani2020gcn,liu2023bgl}. 
In this work, 
% we do not consider algorithmic optimizations that impose approximation choices onto the users and impact model accuracy. 
we focus on optimizations that can be used in general-purpose GNN training systems that scale to multiple GPUs without imposing modeling choices that can impact the model accuracy or semantics, such as using specific sampling algorithms or relaxing synchrony.

We now discuss caching and hybrid parallelism approaches 
% for single-host mini-batch training 
and show that they still suffer from high data loading costs.
Besides caching, none of these solutions addresses the fundamental problem of redundant loading and computation that is inherent in data parallelism.

\mypar{Limitations of data-parallel caching}
To reduce data loading time, several systems maintain a static cache in the main memory of the GPUs. 
This cache is populated offline with frequently-accessed input features~\citep{pagraph, gnnlab}.
The latest systems use a distributed shared memory to enable GPUs to fetch features from other GPUs' memory using fast GPU-to-GPU interconnects like NVLink \citep{quiver, dsp, ugache}.
As shown in Figure~\ref{fig:orkut_epoch_breakdown}, the distributed shared-memory caching mechanism in Quiver~\citep{quiver} can reduce loading time for the Orkut graph, whose features are too large to fit in a single GPU memory but can be fully cached across multiple GPUs. 

Distributed caching, however, does not fully address the performance issue of data loading, which can still take a significant fraction of the epoch time as shown in Figure~\ref{fig:quiver_epoch_breakdown}.
For the GraphSage model, data loading over NVLink can be relatively expensive for Quiver with a graph that can be fully cached like Orkut.
For larger graphs such as Papers100M, only a part of input features can be cached on GPUs. 
Data loading can still stress the PCIe bus between the host and devices, resulting in unsatisfactory training performance. 
With Papers100M, only up to 60\% of the input features can be cached and the data loading time remains high.
%We fixed this bug of Quiver and termed the improved system DistCache. DGL can still outperform DistCache}
%Even with Orkut, which can be fully cached in the distributed GPU memory, Quiver still needs to spend up to XX\% of the time loading data.\ms{update numbers}

\begin{comment}
Table~\ref{tab:loading-bottleneck} reports the maximum percentage of features that can be cached on a four-GPU server with NVLink interconnect. 
With Friendster, only up to XX\% of the input features can be cached and the data loading time remains high (up to 14\% of the epoch time).  
Even with Orkut, which can be fully cached in the distributed GPU memory, Quiver still needs to spend up to XX\% of the time loading data.\ms{update numbers}
\end{comment}


\begin{comment}    
These experiments consider a system equipped with NVLink.
Without NVLink, data loads from the host memory or other GPUs always need to occur over PCIe.
This makes distributed GPU caching not effective, as we will show in our experiments.
\ms{Review this later. PCIe numbers are still inconclusive.}
\end{comment}

% % \begin{table}[t]
% \centering 
% \tabcolsep=0.08cm
% \begin{tabular}{|l||c|c|c|}
% \hline 
% \textbf{Dataset} & \textbf{4x Micro} & \textbf{1x Mini} & \textbf{\% redundancy} \\ \hline \hline
% % ogbn-products & 195M& 155M & 25.5\% \\
% orkut & 926M & 751M & 23\% \\
% % papers100M & 327M & 131M & 148.9\% \\
% papers100M & 452M & 389M & 16\% \\
% % friendster & 452M & 389M & 16\% \\
% % amazon & 473M & 263M & 79.2\% \\
% \hline
% \end{tabular}
% \caption{Redundant computation. The total number of edges computed over one epoch when each mini-batch is sampled as 4 micro-batches of size 1024 (\texttt{4x Micro}) vs. 1 mini-batch of size 4096 (\texttt{1x Mini}). \ms{Numbers for the other graphs?}}
% \label{tab:overlap-edges} 
% \end{table}

\begin{table}[ht]
\centering
% \resizebox{\columnwidth}{!}{%
\small 
\tabcolsep=0.02cm
\begin{tabular}{|c|ccc|ccc|}
\hline
\multirow{2}{*}{\textbf{Graph}} & \multicolumn{3}{c|}{\textbf{\# Edges Computed}}                                                 & \multicolumn{3}{c|}{\textbf{\# Feature Vectors Loaded}}                                                \\ \cline{2-7} 
                                & \multicolumn{1}{c|}{\textbf{ Micro}} & \multicolumn{1}{c|}{\textbf{ Mini}} & \textbf{Ratio} & \multicolumn{1}{c|}{\textbf{ Micro}} & \multicolumn{1}{c|}{\textbf{ Mini}} & \textbf{Ratio} \\ \hline
\textbf{Orkut}                  & \multicolumn{1}{c|}{926M}              & \multicolumn{1}{c|}{751M}             & 1.2x           & \multicolumn{1}{c|}{422M}              & \multicolumn{1}{c|}{169M}             & 2.5x          \\ \hline
\textbf{Papers100M}             & \multicolumn{1}{c|}{452M}              & \multicolumn{1}{c|}{389M}             & 1.2x          & \multicolumn{1}{c|}{231M}              & \multicolumn{1}{c|}{154M}             & 1.5x           \\ \hline
\textbf{Friendster}             & \multicolumn{1}{c|}{13.4B}             & \multicolumn{1}{c|}{13.1B}            & 1.0x          & \multicolumn{1}{c|}{11.4B}             & \multicolumn{1}{c|}{9.4B}             & 1.2x           \\ \hline
\end{tabular}%
% }
\caption{Redundant computation and data loading. The total number of edges computed and feature data loaded over one epoch when each mini-batch is sampled as 4 micro-batches of size 1024 (Micro) vs. 1 mini-batch of size 4096 (Mini).}
\vspace{-0.2cm}
\label{tab:redundancy}
\end{table}

\begin{figure*}[ht]
    \centering
    \includegraphics[width=\textwidth]{figures/overview.drawio.pdf}
    \vspace{-5mm}
    \caption{Overview of the \name training pipeline.}
    \label{fig:overview}
\end{figure*}

\mypar{Limitations of existing hybrid parallelism} 
The alternative to data-parallelism for mini-batch GNN training is called \emph{push-pull parallelism} proposed by the P3 system~\citep{gandhi2021p3}.
P3 targets distributed multi-host systems and aims to avoid transferring input features among hosts.
Each host keeps a slice of the feature vector of each vertex in host memory.
Like in data-parallel training, each GPU is associated with a micro-batch.
However, in push-pull parallelism, each GPU computes the input layer of \emph{all} micro-batches on its feature slice.
GPUs then exchange partial activations and continue the iteration in a data-parallel fashion.

$P^3$ is a multi-host system that does not use GPUs for caching, so it was not previously evaluated in a single-host multi-GPU setting.
To fill this gap, we have implemented its push-pull parallelism approach in a single-host multi-GPU setting.
We call this implementation P3*.
Figure~\ref{fig:comparison}(b) illustrates how P3* applies hybrid parallelism. 
Like the original $P^3$, this implementation partitions the cached features among GPUs at the cost of a cross-GPU push-pull shuffle.
For the Orkut graph, which can be fully cached, P3* does not exchange features among GPUs during the loading time, as Quiver does.
Instead, it pushes the bottom layer of all micro-batches to all GPUs, which induces a much lower loading cost.
The additional overhead of push-pull shuffling outweighs the gains in terms of loading time during the forward and backward pass, which results in higher overall training costs (see Figure~\ref{fig:orkut_epoch_breakdown}).
For other graphs that cannot be fully cached, P3* loads all the features in the mini-batch.
The training time still is higher than the other two systems due to the cost of shuffling.

\begin{comment}
\mypar{Implications for Training Cost}
One of the most important reasons for optimizing the training time of ML models is to reduce the dollar cost of training.
This depends not only on the training time but also on the type of server that is used for training.
GNN models are very small and they can be easily replicated at each GPU, unlike for example large language models that need to be partitioned among multiple GPUs.
For example, the models we consider range around XXX \ms{report size}.
The computation in GNN training is also relatively lightweight.
Therefore, the benefit of using high-end servers with NVLink mainly comes from mitigating the data loading bottleneck.
However, are these speedups sufficient to justify the additional cost of using high-end GPU servers for GNN training?

To answer this question, we consider the cost of the two AWS instance types we used for our evaluation and evaluate the cost per epoch of each system.
We consider a p3.8xlarge instance with NVLink, which currently costs \$12.24 per hour and a g4dn.12xlarge without NVLink, which costs \$3.912 per hour.
The cost per epoch is a good metric to measure the overall training cost because all the systems we consider run the models unmodified using synchronous training and have similar convergence rates per epoch, as we validated experimentally.

Figure XXX shows the cost per epoch achieved by different systems. 
For the Papers100M graph, which cannot be entirely cached in the distributed GPU cache, running DGL on an instance \emph{without} NVLink results in a much lower cost per epoch than using a caching-based system on NVLink.
Using a cheaper instance is preferable unless minimizing the absolute training time is the main goal.

The Orkut graph can be entirely cached in the distributed GPU cache.
In this case, using NVLink has a similar cost per epoch as PCIe, so using a high-end server is preferable since it can speed up training at no extra cost.
\end{comment}

% \mypar{Summary and motivation}
% Data-parallel training suffers from a fundamental problem, which is \emph{redundant data loading}. 
% In each training iteration of data-parallel training, two micro-batches (prepared for two GPUs separately) can share the same input vertices due to the interconnected nature of graphs. 
% The input features of these overlapped vertices need to be loaded twice, one for each GPU. 
% Caching speeds up some of these data loads but it does not fundamentally solve the redundant loading problem.
% The push-pull parallelism approach proposed by P3 eliminates redundant data loads but it introduces an expensive shuffle during training that adds a significant cost to the end-to-end training time.
% This cost often outweighs its benefits compared to data parallelism with caching.

\section{\name: Overview} \label{sec:opportunities}
Our work is based on the observation that it is possible to eliminate redundant data loads \emph{and} reduce end-to-end training times by using the right type of hybrid parallelism.
Instead of sampling micro-batches that have overlapping input vertices, one micro-batch for each GPU, the sampling phase can prepare non-overlapping \textit{splits} and assign each split to a specific GPU.
The splits are obtained running an online \emph{splitting algorithm} while sampling each mini-batch at each iteration.
Each GPU loads only the features of the vertices within its assigned splits.
Splits must be consistent with the location of cached features to benefit from them.
During training, the hidden features of a vertex are computed only by one GPU and then shuffled to other GPUs.
We refer to this parallelism technique as \textit{\tname}.
% The key challenge in making \tname effective is to devise a lightweight splitting algorithm that minimizes shuffling cost and balances load among GPUs.


Figure~\ref{fig:comparison}(c) illustrates how \tadj training reduces input feature loads. 
Data parallelism requires loading eight feature vectors into the GPUs before the training phase, four for each GPU, of which 3 are loaded redundantly.
\Tname eliminates the redundant loads and requires loading only five feature vectors (three for GPU 1 and two for GPU 2). 
Push-pull parallelism, illustrated in Figure~\ref{fig:comparison}(b), requires no data loads if the graph is fully cached.
However, it shuffles partial activations across GPUs for eight edges (the number of edges in the bottom layer of the GNN), whereas \tname shuffles vertex features over three edges (the number of vertices that lie in the boundaries between splits). 

\Tname also avoids the redundant computation of hidden features that occurs in data parallelism.
This reduction in computational cost helps compensate for the cost of shuffling and sometimes offsets it.
For example the hidden features of vertex $c$ in Figure~\ref{fig:comparison}(a) are computed by both GPUs with data parallelism, but with split parallelism, they are computed only by GPU 1 (see Figure~\ref{fig:comparison}(c)).
When used for sampling, \tname also avoids redundant computation in the sampling step.

In the following, we describe how we embodied \tname into the \name system (Section~\ref{sec:overview}).
We describe \name's novel lightweight online splitting algorithm, which provides probabilistic performance guarantees, in Section~\ref{sec:workload}.
\name also offers a programming API that supports efficient single-GPU kernels for sampling and training, as discussed in Section~\ref{sec:api}. 

% Next, we illustrate how we faced the technical challenges involved in implementing \name. 
% The first challenge is to devise a lightweight online splitting algorithm that can provide probabilistic performance guarantees (Section~\ref{sec:workload}).
% The second is to find an appropriate programming API to support efficient single-GPU kernels for sampling and training (Section~\ref{sec:api}).

\begin{comment}
Materializing split parallelism, however, faces two research challenges. 
\begin{itemize}
    \item[RQ1] 
    Recent research has developed optimized single-GPU kernels for GNN sampling \ms{refs} and training \ms{refs}.
    Data-parallelism runs single-GPU code and kernels on multiple GPUs as-is, in an embarrassingly parallel manner.
    Split parallelism introduces coordination across multiple GPUs.
    An appropriate API to abstract away coordination can facilitate running existing optimized kernels with split parallelism. 
    \item[RQ2] 
    The second challenge is the efficient implementation of cross-GPU coordination, which requires the design of new data structures to enable effective mapping between vertices IDs in splits assigned to different GPUs. 
    % The second challenge is to minimize the run-time overhead caused by split parallelism during sampling and training.  
    % \hui{fill in the research challenge, add 1-2 sentences to elaborate on why this is important to address}
\end{itemize}
This work addresses the two challenges and developed a GNN training system called \name{} that implements split parallelism. We next provide an overview of the system \name{} in Section~\ref{sec:overview} and then elaborate on proposed solutions to address the two challenges in Section~\ref{sec:api}. 
\end{comment}

\section{The \name{} Training Pipeline}
\label{sec:overview}

We now present a running example to describe one training iteration in \name (see Figure~\ref{fig:overview}). 
The example shows how \name uses the splitting algorithm to obtain splits on-the-fly, as described in detail in Section~\ref{sec:workload}, and introduces the abstractions used by \name's API, such as the local and mixed frontier and the split index, which are further described in Section~\ref{sec:api}.
Note that although \name can be combined with distributed GPU caching, the example assumes no caching to simplify the illustration and ease the description.
%\ms{Say that same communication pattern during sampling and training, so we can build index}
% \ms{Discuss that sampling does not need to be split-parallel in order to get the benefits of reduced loading. We could sample centrally, but we use this implementation for convenience.}

\mypar{Sampling}
The sampling step prepares the splits for the training phase, with each split corresponding to one GPU. 
\name pushes sampling to the GPU for performance reasons, in line with recent work~\citep{nextdoor,c-saw,wang2021skywalker,gsampler}.
It uses a \tadj implementation of sampling, where all GPUs sample and split the same mini-batch cooperatively.
% However, it is not necessary to make sampling \tadj or to push it to the GPU to achieve our goal of eliminating redundant loads, since using \tadj training is sufficient.
%For example, our preliminary implementation of \name used CPU-based sampling~\citep{gsplit-arxiv}.

\Tadj sampling proceeds top-down and layer-by-layer.
At each layer, a GPU has a set of vertices whose neighbors should be sampled.
This set is called the \emph{frontier}.
% This frontier-based approach is followed by the sampling algorithms included in DGL and by existing APIs for GNN sampling such as NextDoor~\citep{nextdoor} and gSampler~\citep{gsampler}.
%\Tadj sampling follows the same approach, but now 
\name uses an embarrassingly parallel splitting algorithm to separate local and remote vertices by mapping each vertex to its split in constant time, as discussed in Section~\ref{sec:workload}.

Each GPU samples its local split of the same mini-batch, rather than a separate micro-batch.
At each layer, a GPU starts from a frontier, samples its neighbors, and obtains what we call the \emph{\ds}.
Unlike the regular frontier, a \ds can include remote vertices.
For example, in Figure~\ref{fig:overview}, GPU 1 starts from its layer-2 frontier $\{a,b\}$ and samples the \ds $\{e,h,f\}$, which includes the remote vertex $h$. 
Similarly, GPU 2 starts from the frontier $\{c,d\}$ and samples the \ds $\{f,g,h,i\}$, which includes the two remote vertices $h$ and $i$.
To continue sampling in a \tadj manner, it is necessary to shuffle them to their partitions.
GPUs then build the \emph{local frontier} for the next layer based on the vertices they receive in the messages.
For example, the new local frontier of GPU 2 at layer-1 is $\{h,i\}$, where $h$ was received from GPU 1.
Analogously, the local frontier of GPU 1 is $\{e,f,g\}$, where $\{g\}$ was sampled by GPU 2.
The sampling of the next layer starts from the new local frontier.
The next \emph{layer} is the union of mixed and local frontier: $\{e, f, g, h\}$ for GPU 1 and $\{h, i, f, g\}$ for GPU 2.
The splitting step creates an auxiliary data structure, called the \emph{shuffle index}, that consists of gather and scatter indexes to efficiently send and receive sparse vertex data during the shuffle rounds at each layer.
% We describe the splitting process and the shuffle index in detail in Section~\ref{sec:api}.
%The shuffle index is reused to perform \tadj training over the same layer in the same iteration, as we now discuss.

\mypar{Loading}
% Sampling and splitting iterate until they sample all required layers.
After sampling finishes, the loading step loads the input vertex features from the host memory into the GPUs.
When \tname is combined with caching, a GPU can skip loading an input feature if it is already cached locally.
% Unlike data-parallel systems using caching, \tname does not require loading \sand{cache-missed} input features among GPUs in the load phase.
Unlike data-parallel systems that could load redundant input features, \tname eliminates the redundant CPU-GPU data loading as splits do not have overlapping vertices. 
In the example of Figure~\ref{fig:overview}, we consider no caching, so GPU 1 loads the input features of the vertices $\{j,k,l\}$ from the host memory and GPU 2 of the vertices $\{m,p\}$.

\mypar{Training}
The training phase begins after loading and proceeds bottom-up.
At each layer, a GPU is responsible for computing the hidden features of the vertices in its local frontier by aggregating the features of the neighbors in the layer below.
These vertices are the same vertices that were in the \ds for that layer during sampling, so we can reuse the shuffle index generated at that step.
In the example, GPU 1 must compute the layer-2 hidden features of vertices $a$ and $b$.
To do that, it needs the layer-1 hidden features of vertices $\{e,f,g,h\}$, which constituted the \ds during sampling and include the remote vertex $h$.

GPUs receive the features of remote neighbors by performing an all-to-all shuffle.
The shuffle rebuilds the \ds, which now holds hidden features.
The backward pass then proceeds top-down, layer-by-layer, in the reverse direction.
Data flows along the same edges as in the sampling step, allowing us to reuse the shuffle index.
\section{The Splitting Algorithm} \label{sec:workload}
% Balancing load among GPUs is important in multi-GPU training.
% Data-parallel systems partition a mini-batch and balance load between GPUs simply by assigning a similar number of target vertices to each micro-batch before sampling starts.
% The hybrid push-pull parallelism proposed by $P^3$ pushes each micro-batch to each GPU at each iteration.
% It balances the load by partitioning the input features along the feature dimension so that each partition includes a slice of the features of all vertices.

\Tadj training must split each sampled mini-batch among multiple GPUs in a way that minimizes communication costs during shuffles and balances load across GPUs.
A na\"{i}ve approach would be to run a min-edge-cut graph partitioning algorithm on each mini-batch we sample.
However, this would be too computationally expensive, since splitting must be executed during the sampling step of each iteration, and is hard to parallelize across multiple GPUs.
\name uses an embarrassingly parallel online splitting algorithm that maps each vertex to a split independently from each other in constant time. 
It does so by providing \emph{probabilistic} guarantees, rather than deterministic ones: given a random mini-batch sample, it minimizes the \emph{expected} communication cost while balancing the \emph{expected} load per split.
Formally, the splitting algorithm solves the following problem.


% This problem is analogous to classic min-cut partitioning, but two aspects make the mini-batch splitting peculiar. 
% First, splitting must be fast as it is executed on each mini-batch.
% Running a graph partitioning algorithm online at each training iteration can introduce a performance bottleneck. 
% On the other hand, partitioning the entire input graph offline does not necessarily result in good splits for the mini-batches, which are randomly sampled online.
% Second, mini-batch splitting should be cache-aware to reduce data loading time. 
% Vertices at the bottom layer of the current mini-batch should be placed where their features are cached, which is determined offline before training.


\noindent\textbf{Problem definition.} (Mini-batch splitting problem)
\emph{
Let $M(V_M,E_M)$ be a sampled mini-batch of a graph $G(V_G,E_G)$, $S$ a set of splits, and $f_M:\: V_M \rightarrow S$ a splitting function that assigns each vertex in $V_M$ to a split.
Let $S_i$ be the set of vertices assigned to split $i$ by $f_M$, $X_i$ a random variable expressing the number of vertices at layer $l > 0$ in $S_i$, $Y$ a random variable expressing the number of edges in $E_M$ having endpoints in two different splits, and $\epsilon \geq 0$ a tunable constant.
The mini-batch splitting problem involves finding the splitting function $f_M$ that solves the following minimization problem:}

\begin{equation}
\begin{aligned}
\min_{f_M }\quad & \E[Y]\\
\textrm{s.t.} \quad & \forall i:\: \E[X_i] \leq (1 + \epsilon) \cdot \sum_{i \in [1,|S|]} \E[X_i] /|S|
\end{aligned}
\label{eqn:prob}
\end{equation}

The problem is to find a \emph{splitting function} $f_M$ that can be used online by the splitting algorithm to map each sampled vertex to a split, each corresponding to a different GPU device.
The random variables $X_i$ and $Y$ represent the computation and communication cost of the splits, respectively.
When the algorithm assigns a vertex at layer $l>0$ to a split, the corresponding GPU must sample its neighbors during the sampling phase and compute its hidden feature during the training phase.
Edges connecting vertices in different splits induce communication costs during the shuffle phases of both sampling and training. 
This problem is NP-hard since it can be reduced to min-edge-cut graph partitioning by selecting an appropriate sampling function.

\mypar{Splitting algorithm}
To avoid solving this NP-hard problem online, our approach is to reduce it to a problem that can be solved offline.
Therefore, we propose using a splitting algorithm that has an offline and an online part.
Offline, the algorithm finds a \emph{global partitioning function} $f_G:\: V_G \rightarrow D$ that statically maps each vertex in the whole input graph to a GPU device.
Then, online, the splitting algorithm uses $f_G$ as a substitute to $f_M$ to map each vertex to a device and thus to its corresponding split at each iteration.
Online splitting is embarrassingly parallel since $f_G$ maps each vertex to a split independently and does it in constant time.

\name uses the global partitioning $f_G$ also to determine the GPU where it statically caches input features of a vertex.
This ensures that the caches are consistent with the splits.

\mypar{Finding the global partitioning function}
We now describe the details of the offline part of the splitting algorithm, which finds the global partitioning function.
The first stage of the offline algorithm is pre-sampling, which weighs the vertices and edges of the input graph.
Weights represent the computational and communication costs incurred by GPUs during \tadj sampling and training.
The second stage uses a weighted min-edge-cut graph partitioning algorithm to find the global partitioning function.

Given an input graph $G$, the pre-sampling stage assigns weights to the vertices and edges of $G$ to obtain a weighted graph $G_w$.
It runs the same sampling algorithm used during training for a fixed number of epochs.
At each iteration, the algorithm samples the k-hop neighborhood of the training (target) vertices in the mini-batch. 
It then assigns to each vertex $v$ a weight $w_V(v) = k_v/N$, where $k_v$ is the total number of times $v$ is sampled at a layer $l > 0$ across all samples and $N$ is the number of samples.
The weight of each edge $e$ is $w_E(e) = k_e/N$, where $k_e$ is the total number of times $e$ is sampled across all samples.

After completing the pre-sampling stage and obtaining the weighted graph $G_w$, the offline algorithm runs a weighted min-edge-cut graph partitioning algorithm on $G_w$ to obtain partitions.
The algorithm outputs partitions that minimize the sum of the weights of the edges in the cut and ensure that the load per partition, which is the sum of the weights of their vertices, is balanced. 
% Any vertex feature that needs to be cached is now assigned to a GPU based on the partitions.
Formally, given a graph $G_w(V,E,w_V,w_E)$ and the number of partitions $d = |D|$, where $D$ is the set of GPU devices, a weighted min-edge-cut graph partition algorithm outputs the set of partitions $P=\{P_1,\ldots,P_d\}$ of $V$ that solves the following problem:

\begin{equation}
\begin{aligned}
\min_P \quad & \sum_{e \in C} w_E(e)\\
\textrm{s.t.} \quad & \forall i:\: L_i \leq (1 + \epsilon) \cdot L /d
\end{aligned}
\label{eqn:optim}
\end{equation}
where $w_E(e) = k_e/N$ is the edge weight of $e$, $C = \{\langle u, v \rangle \in E: u \in P_i, v \in P_j, i \neq j\}$ is the edge cut set, $L_i = \sum_{v \in P_i} w_V(v) = \sum_{v \in P_i} k_v/N$ is the load of each partition $P_i$, $L = \sum_{v \in V} w_V(v)$ is the total load across all partitions, and $\epsilon \geq 0$ is a tunable constant.
% The first equation defines the minimization function as the sum of the weights of the edges in the cut.
% The second equation defines the optimization constraint, which balances the load of each partition $P_i$ defined as the sum of the weights of its vertices 
Note that since the minimization problem of Eq.~\ref{eqn:optim} is NP-hard, we use heuristics to solve it in practice, for example using Metis~\citep{karypis1997metis}.
The partitions $P$ found by the graph partitioning algorithm determine the global partitioning function $f_G$ from vertices to GPUs. %, which in turns determines the splitting function $f_M$ used by the online splitting algorithm.

\begin{comment}
Let $X_i$ be a random variable expressing the load on the $i$-th split of $S$ as the number of vertices \ms{number of incoming vertices} at layer $l>0$ in the split and let $Y$ be a random variable expressing the communication cost of $S$ as the number of edges across different splits.
In the following, we show that $\E[X_i] = L_i$ and $\E[Y] = \sum_{e \in C} k_e/N$.
After doing that, we can conclude based on Eq.~\ref{eqn:optim} that $\E[Y]$ is minimized and $\E[X_i]$ across different splits is balanced.
\end{comment}

\mypar{Analysis}
We now show that our splitting algorithm finds a solution to the mini-batch splitting problem of Eq.~\ref{eqn:prob} by reducing it to the optimization problem of Eq.~\ref{eqn:optim} with $d=|S|$, which we can solve with a heuristic.
We show that $\E[X_i] = L_i$ and $\E[Y] = \sum_{e \in C} k_e/N$.
After doing that, we can conclude that Eq.~\ref{eqn:optim} minimizes $\E[Y]$ and constraints $\E[X_i]$ as in Eq.~\ref{eqn:prob}.

We start by showing that $\E[X_i] = L_i$.
% Given a vertex $v \in V$, let the random variable $Z_v$ be 1 if the vertex appears at layer $l>0$ in the sample $S$ and 0 otherwise. 
Given a vertex $v \in V$, let the random variable $Z_v$ be the number of layers $l > 0$ where the vertex appears in the sample $S$. 
% \ms{$Z_v$ is the number of incoming edges in the split}
We have that $X_i = \sum_{v \in P_i} Z_v$, which implies that $\E[X_i]= \sum_{v \in P_i} \E[{Z_v}]$.
If we assign vertex weights using a sufficiently large number of samples $N$, according to the law of large numbers, we have $\E[{Z_v}] = k_v/N$, so $\E[X_i] = L_i$.

We use a similar argument to show that $\E[Y] = \sum_{e \in C} k_e/N$.
Given an edge $e \in E$, let the random variable $Z_e$ be the number of times $e$ is sampled (at different layers) in $S$. 
The set of cross-split edges in the sample $S$ is a subset of the cross-partition edges $C$ in the whole graph, so $Y = \sum_{e \in C} Z_e$.
The expected value of $Y$ is given by $\E[Y] = \sum_{e \in C} \E[Z_e]$.
If we use a sufficiently large number of samples $N$ to assign edge weights, according to the law of large numbers, we have that $\E[Z_e] = k_e/N$, so $\E[Y] = \sum_{e \in C} k_e/N$.

\begin{comment}
\mypar{Analysis}
To minimize communication overhead during training, we assign a score to each edge and aim to minimize the weighted sum of edge cuts. This score represents the probability of an edge being sampled in a mini-batch. Our analysis demonstrates that by minimizing the weighted sum of edge cuts, we effectively reduce the communication overhead in a sample. Although our focus in this analysis is on a single layer, the results apply to all layers within the sample.

Let $e$ be an edge, and define the random variable $Z_e$ as 1 if the edge appears in a sampled layer, and 0 otherwise. To estimate the expected value $\E[Z_e]$, we compute the score of each edge using the same k-hop sampling algorithm used in the training. According to the law of large numbers, the expected mean of a random variable will converge as we average its outcomes a sufficiently large number of times. \juelin{We can also use k-hop neighbor gathering (including all k-hop neighbors) to compute the exact probability. Yet this approach is not scalable as it goes out-of-memory on many graphs}

Consider a set $S$ of edges across all partitions, and let $X$ be the count of edges from $S$ that appear in a sampled layer. We denote this as $X = \sum_{e \in S} Z_e$. Assuming independence among all $Z_e$ values, the expected value of $X$ is given by $\E[X] = \sum_{e \in S} \E[Z_e]$. The edge-cut partitioner minimizes the sum of edge weights, which is equivalent to minimizing $\sum_{e \in S} \E[Z_e]$. Hence, it reduces the expected number of cross-edges in a sample.
\ms{We have never described how we assign edge weights}

A similar argument can be made for the load-balancing constraint. In this case, we consider the computation cost as the number of destination vertices in the partition (or the number of edges to aggregate in the graph convolution network). Let $v$ be a vertex, and let $Z_v$ represent the number of times the vertex appears in the layer.
\ms{Why number of times? Isn't this binary, 1 if it appears and 0 if it does not?}
Let $T$ denote the set of vertices in a given partition, and $Y$ be the total count of vertices from $T$ that appear in a sample. We express this as $Y = \sum_{v \in T} Z_v$. Assuming independence among all $Z_v$ values, the expected value of $Y$ is given by $\E[Y] = \sum_{v \in T} \E[Z_v]$. In this case, we aim to balance the sum of vertex weights within each partition, which is $\sum_{v \in T} \E[Z_v]$, to achieve an evenly distributed workload in a sample.
\end{comment}
\section{\name's API} \label{sec:api}
\name offers a \emph{layer-centric} programming API that simplifies the development of GNN sampling and training code and kernels by hiding cross-GPU coordination.
Besides simplicity, this approach supports the reuse of optimized single-GPU kernels for GNN sampling and training, including the ones proposed by recent research~\citep{adaptgear, fastkernel, wu2021seastar, nextdoor, gsampler, fu2022tlpgnn, ye2023sparsetir}. 
% 
% The API introduces two new operators, \textit{split} and \textit{shuffle}, that give each GPU the abstraction of operating only on its local frontiers.
% These two functions reduce the cost of communication by building a shuffle index data structure and reusing it for every shuffle of the same iteration.
% We now describe how to implement split-parallel sampling and training with \name's API and how to use the split and shuffle operators. 

\mypar{Motivation for a layer-centric API}
% Its \emph{scatter} function produces messages for each edge separately based on the features of the source and destination vertices.
% The \emph{gather} function aggregates these messages incrementally using a commutative and associative function, and the \emph{transform} function computes the partial activations that are accumulated using a \emph{sync} function and then given to the \emph{apply} function to compute the hidden features of the destination vertex.
Single-GPU kernels used by data-parallel systems are layer-centric: they assume that at each GNN layer, the source vertex, destination vertex, and edge features for all incoming edges incident to the same destination vertex are locally available.
\name's API supports this property even after it splits destination vertices of the same mini-batch among different GPUs.
The alternative hybrid mini-batch parallelism approach, which is $P^3$, breaks this assumption to offer an \emph{edge-centric} API to enable pushing part of the computation of each micro-batch to multiple hosts and GPUs in a fine-grained manner.
For example, implementing state-of-the-art attention-based models such as Graph Attention Networks (GAT) requires a custom implementation to ensure correctness~\citep{gandhi2021p3}.
% \name's split/shuffle API uses a layer-centric approach and ensures that all incoming edges for the same destination vertex in a layer are available locally at each GPU before starting the layer computation.
% This supports existing single-GPU kernels and modern attention-based GNN models such as GAT.

\mypar{Sampling} \label{sec:sampling}
% In split-parallel sampling, all GPUs sample the same mini-batch.
% Each GPU is responsible for sampling a split of the mini-batch and two splits do not have overlapping vertices.
Algorithm~\ref{algo:sp-sample} shows the pseudocode of sampling using \name's API. 
The \texttt{sp\_sample} function executed by each GPU takes as input the subset of target vertices in the mini-batch that are local to the GPU  according to the splitting algorithm (\texttt{local\_targets}) and produces as output a \emph{split}, which consists of the set of \texttt{edges} at each layer that are used by the local training kernels to aggregate features.
%In the example of Figure~\ref{fig:overview}, for GPU 1, the sampler starts with the local target vertices $\{a, b\}$ and outputs the split consisting of the edges colored in blue.
The function also outputs a \textit{shuffle index} (\texttt{shuffle\_idx}), which consists of gather and scatter indexes to efficiently send and receive sparse vertex data in the frontiers during the all-to-all shuffle rounds at each layer.
%The index will be described more in detail in Section~\ref{sec:implementation}.
% 
% Split-parallel sampling works from top to bottom and layer by layer. 
% It starts from target vertices at layer $L$ and samples the neighbors of these vertices as layer $L-1$. 
% At each layer, $l$, sampling starts from a local frontier of vertices in the GPU's local partition and produces a mixed frontier, which could include remote vertices.
% In order to continue sampling vertices for Layer $l-1$, GPUs must shuffle data to send locally-sampled \emph{remote} vertices and receive remotely-sampled \emph{local} vertices.
% 
In the pseudocode, we abstract away the single-GPU kernels that sample one layer by using the \texttt{sample\_layer} function, which takes the local frontier from the previous layer as input (Line~\ref{ln:sample}).
The function outputs a new mixed frontier (\texttt{mixed\_front}) and the edges from the input local frontier to the output mixed frontier (\texttt{edges[l]}).
The mixed frontier includes both local and remote vertices.
The sampling code must invoke the \texttt{split} function to perform a shuffle and obtain a local frontier (Line~\ref{ln:split}).
The function also produces the shuffle index for the layer (\texttt{shuffle\_idx}).

% In Figure~\ref{fig:overview}, at layer 1, GPU 1 runs \texttt{sample\_layer} on the local frontier $\{a,b\}$ to sample the mixed frontier $\{e,h,f\}$, which includes the remote vertex $h$. 
% GPU 1 then calls the \texttt{split} function to obtain the local frontier for layer 1, which is $\{e,f,g\}$.

% Sampling algorithms can be categorized into individual and collective ones~\citep{nextdoor}.
% The previous discussion considers the case of individual sampling algorithms, which sample the neighbors of a vertex based only on the adjacency list of that vertex. 
% To run collective sampling algorithms such as LADIES~\citep{ladies}, GPUs must broadcast all the vertices they sampled locally so that each GPU gets the entire global frontier.
%Each GPU can then sample a fixed number of neighbors by accessing the adjacency lists stored on the host memory in pinned memory.
% \ms{check} \juelin{Ladies is too lightweight to use GPU, original paper only uses CPU for sampling. So it doesn't make too much sense here to compare it.}

%\mypar{Comparison with data-parallel sampling}
The pseudocode of Algorithm~\ref{algo:sp-sample} is almost exactly the same as the one that would be used in data parallelism.
%In data parallelism, each GPU samples micro-batches independent from each other using single-GPU code.
The most common sampling implementations in DGL~\citep{dgl}, research prototypes, and single-GPU sampling APIs~\citep{nextdoor,c-saw,wang2021skywalker,gsampler}, operate layer-by-layer.
Their behavior can be encapsulated by the same \texttt{sample\_layer} function used in Algorithm~\ref{algo:sp-sample}.
The main difference in data-parallel training is that the function now directly returns the \texttt{local\_front} for the next layer at Line~\ref{ln:sample}.
There is no notion of mixed frontier and no need to invoke the \texttt{split} function in Line~\ref{ln:split}.
\begin{algorithm}[t]
    \caption{Split-parallel sampling.}
    \label{algo:sp-sample}
    \let\textnormal\ttfamily
    \footnotesize

    \DontPrintSemicolon
    \SetStartEndCondition{ }{}{}%
    \SetKwProg{Fn}{def}{\string:}{}
    \SetKwFunction{Range}{range}%%
    \SetKw{KwTo}{in}\SetKwFor{For}{for}{\string:}{}% \SetKwIF{If}{ElseIf}{Else}{if}{:}{elif}{else:}{}% \SetKwFor{While}{while}{:}{fintq}%
    \newcommand{\forcond}{\em l \KwTo [L-1, 0]}
\AlgoDontDisplayBlockMarkers\SetAlgoNoEnd\SetAlgoNoLine%
    \SetKwFunction{dpsample}{dp\_sample}
    \SetKwFunction{splitsample}{sp\_sample}
    \SetKwFunction{sneigh}{\textbf{\color{teal}sample\_layer}}
    \SetKwFunction{unique}{unique}
    \SetKwFunction{union}{union}
    \SetKwFunction{local}{local\_unique}
    \SetKwFunction{concat}{concat}
    \SetKwFunction{csplit}{\textbf{\color{purple} add}}
    \SetKwFunction{split}{\textbf{\color{purple} split}}
    \SetKwFunction{localfrontier}{local\_frontier}
    \SetKwFunction{dst}{dst}
    \SetKwFunction{src}{src}
    \SetKwFunction{add}{add}
    \SetKwFunction{remap}{remap}    
    \SetKwFunction{getall}{getAll}    
    \SetKwFunction{bsplit}{\textbf{create\_sp\_frontier}}
    \SetKwFunction{shuffleidx}{shuffle\_idx}


    % \Fn{\dpsample{\em local\_targets}}{
    %     front[L] = local\_targets\;
    %     local\_ids.\add{front[L]}\;
    %     \For{\forcond}{
    %         samp\_edges[l] = \\ $\quad$ \sneigh{front[l+1]}\;
    %         local\_ids.\add{\em samp\_edges[l].\dst{}}\;
    %         front[l] = local\_ids.\getall{}\;
    %         edges[l] = \\ $\quad$ local\_ids.\remap{samp\_edges[l]}\;
    %     }
    %     \Return front, edges\;
    % }    
    % \;
    \Fn{\splitsample{\em local\_targets}}{
        local\_front[L] = local\_targets\;
        \For{\forcond}{
            mixed\_front, edges[l] = \sneigh{local\_front[l+1]}\label{ln:sample}\;
            local\_front[l], edges[l], shuffle\_idx[l] = \split{mixed\_front, edges[l]}\label{ln:split}\;
        }
        \Return edges, \shuffleidx\;
    }    
\end{algorithm}

\begin{algorithm}[t]
    \caption{Forward pass of split-parallel training.}
    \label{algo:sp-forward}
    \let\textnormal\ttfamily
    \footnotesize
    \DontPrintSemicolon
    \SetStartEndCondition{ }{}{}%
    \SetKwProg{Fn}{def}{\string:}{}
    \SetKwFunction{Range}{range}%%
    \SetKw{KwTo}{in}\SetKwFor{For}{for}{\string:}{}% \SetKwIF{If}{ElseIf}{Else}{if}{:}{elif}{else:}{}% \SetKwFor{While}{while}{:}{fintq}%
    \newcommand{\forcond}{\em l \KwTo [0, L-1]}
\AlgoDontDisplayBlockMarkers\SetAlgoNoEnd\SetAlgoNoLine%
    \SetKwFunction{dpfwpass}{dp\_forward}
    \SetKwFunction{splitfwpass}{sp\_forward}
    \SetKwFunction{local}{local\_unique}
    \SetKwFunction{csplit}{create\_upper\_split}
    \SetKwFunction{split}{{\color{turquoise} split}}
    \SetKwFunction{dst}{dst}
    \SetKwFunction{src}{src}
    \SetKwFunction{add}{add}
    \SetKwFunction{remap}{remap}    
    \SetKwFunction{getall}{getAll}    
    \SetKwFunction{shuffle}{\textbf{\color{purple}shuffle}}
    \SetKwFunction{splitidx}{shuffle\_idx}

    % \Fn{\dpfwpass{\em model, in\_feat, edges}}{
    %     hidden = in\_feat \;
    %     \For{\forcond}{
    %         gnn\_layer = model.layer[l]\;
    %         hidden = gnn\_layer(edges[l], hidden)\;
    %     }
    %     \Return hidden\;
    % }    
    % \;
    \Fn{\splitfwpass{\em model, feat, edges, \splitidx}}{
        hidden = feat\; \label{ln:input}
        \For{\forcond}{
            \textbf{\color{teal} gnn\_layer} = model.layer(l)\;
            mixed\_hidden = \shuffle{\splitidx[l], local\_hidden} \label{ln:shuffle}\;
            local\_hidden = \textbf{\color{teal} gnn\_layer}(edges[l], mixed\_hidden)\label{ln:layer}\;
        }
        \Return hidden;
    }    
\end{algorithm}


\begin{comment}
 ==== LADIES discussion

The main goal of \name is to speed up loading, which only requires training to be \tadj as discussed in Section~\ref{sec:overview}.
This gives the \name runtime some flexibility on how to run sampling code written with its API.
Our current implementation based on \tadj sampling supports \emph{individual} sampling algorithms, such as the common neighborhood sampling~\citep{graphsage}, where the neighbors of each vertex in the frontier are sampled independently.
The \name runtime can also run the sampling code of Algorithm~\ref{algo:sp-sample} in a centralized fashion, where each GPU samples a different mini-batch independently and produces splits and shuffle indexes for all GPUs.
The split function does not perform a shuffle.
A centralized implementation supports running \emph{collective} sampling algorithms such as LADIES~\citep{ladies}, which require accessing the entire frontier to sample the next layer.
We leave this extension for future work.
    
\end{comment}



\mypar{Training}
\label{sec:training}
We now explain how to implement cooperative split-parallel training.
Algorithm~\ref{algo:sp-forward} shows the pseudocode for the forward propagation. 
The backward propagation works similarly except that the computation happens from the top layer to the bottom layer. 

The \texttt{sp\_forward} function takes as input the GNN model (\texttt{model}) the input features of the vertices in the local split (\texttt{feat}), the structure of the split (\texttt{edges}), and the shuffle index (\texttt{shuffle\_idx}).
The latter two inputs are produced by the sampling code of Algorithm~\ref{algo:sp-sample}.
It produces as output the hidden features of the target vertices.
% The pseudocode iterates over all GNN layers.
% The GNN layer $l$ at one GPU computes the hidden features for layer $l+1$ locally by using single-GPU code, which we abstract away with the \texttt{gnn\_layer} function.
% Split-parallel training interleaves feature shuffling with the computation of a GNN layer. 
At each layer $l$, each GPU starts by shuffling the features/activations of its local vertices to other GPUs using the \texttt{shuffle} function provided by \name
(Line~\ref{ln:shuffle}). 
% At layer $0$, these features are the input features of the local vertices (Line~\ref{ln:input}).
The output of this function is the \texttt{mixed\_hidden} tensor, which contains all the features required to compute the hidden features at layer $l+1$.
These could include the features of remote vertices.
The GNN layer then computes the next-layer hidden features for the vertices in the local partition of the GPU (Line~\ref{ln:layer}).

% We use the example in Figure~\ref{fig:overview} to illustrate the forward pass. 
% To compute layer 1's features for the vertices in their partitions, GPUs 1 and 2 require layer 0's features as input. 
% GPU 1 requires the input features of the local vertices $\{j, k, l\}$ and the remote vertex $\{m\}$. 
% Similarly, GPU 2 requires the local vertices $\{m, p\}$ and the remote vertices $\{j, k\}$. 
% The \texttt{shuffle} function exchanges the input features of the remote vertices, $\{m\}$ and $\{j, k\}$, between the two GPUs so that the GPUs can compute the hidden features of layer 1's vertices. 
% These features are added to the local features to form the \texttt{mixed\_hidden} tensor.
%%%After layer 1's computation finishes, GPU 1 and GPU 2 move to the next layer and again shuffle the features of remote vertices, $\{f, g\}$ and $\{h\}$ from layer 1, and then proceed to compute layer 2's features for the target vertices. 

%\mypar{Comparison with data-parallel training}
Data-parallel training code is similar to the pseudocode in Algorithm~\ref{algo:sp-forward}.
%The implementation of the \texttt{model} is the same as the \tadj implementation.
The main difference is that there is only a local frontier, so there is no need to use the \texttt{shuffle} function to get a mixed frontier and we can skip Line~\ref{ln:shuffle}.
Each GNN layer directly takes as input the local frontier (\texttt{local\_hidden}) from the previous layer at Line~\ref{ln:layer} and produces the local frontier for the next layer.
Optimized single-GPU training kernels can be directly used as \texttt{gnn\_layer} implementations since they operate on one layer at a time~\citep{adaptgear, fastkernel, wu2021seastar, fu2022tlpgnn, ye2023sparsetir}.


\begin{comment}
==== Implementation section 
\begin{figure}[tp]
    \centering
    \includegraphics[width=\columnwidth]{figures/index2.drawio.pdf}
     \caption{Execution of the \texttt{split} function at GPU 1 for layer 1 in the example of Figure~\ref{fig:overview}. Vertices assigned to GPU 1 (resp. GPU 2) are colored in blue (resp. red).}
    \label{fig:shuffle-index}
\end{figure}

\section{\name Implementation}
\label{sec:implementation}
\name is implemented based on the DGL and consists of approximately 5600 lines of C++ and CUDA code and 200 lines of Python code.
% The \name API gives each GPU the abstraction of operating on a local split.
% To make each split fit into a single GPU and reduce memory and computation costs in the presence of arbitrarily large graphs, \name uses a compact representation of the split subgraph that considers only local vertices and edges by remapping each vertex ID to a local vertex ID.
% The \texttt{split} and \texttt{shuffle} functions must map between different local representations when serializing and deserializing messages.
We now discuss how the split and shuffle functions construct the shuffle index and use it to perform the mapping.

\mypar{Split function}
Figure~\ref{fig:shuffle-index} illustrates the execution of the \texttt{split} function at GPU 1 for layer 1 in the example of Figure~\ref{fig:overview}.
The function (Alg.~\ref{algo:sp-sample}, Line~\ref{ln:split}) takes a \ds as input and outputs the next layer of the split,
% (\texttt{edges[l]} in Line 5 of Algorithm~\ref{algo:sp-sample}), 
% the shuffle index, (\texttt{shuffle\_idx}), 
which is depicted in Figure~\ref{fig:shuffle-index}, and the next local frontier,
% (\texttt{local\_front[l]}), 
which is not depicted.
The shuffle index built by the \texttt{split} function contains four data structures: the mixed index, the mixed offsets, the layer index, and the layer offsets.

To shuffle messages using the NCCL all-to-all primitive, a sender GPU must gather vertex IDs from a sparse tensor (the \ds) into a dense tensor (the message). 
NCCL requires a message tensor as input, so all vertices in the \ds that belong to the same partition must be placed in contiguous memory locations.
The mixed index is used as a gather index that indicates the corresponding position in the \ds for each position in the message array. 
NCCL also requires the offsets of the boundaries of each message, which are contained in the mixed offsets array.

After the all-to-all shuffle, each GPU receives messages containing only the vertices that are local to its partition. 
The \texttt{split} function adds these vertices to the local layer using a scatter operation.
This is done by creating the layer index as a scatter index, which is an array that maps each position in the message to the local ID assumed by the vertex stored at that position in the message.

\mypar{Shuffle function}
The \texttt{shuffle} function reuses the shuffle index to make the hidden representation and the gradients flow between different splits.
The index can be reused because the shuffle at a layer always exchanges information about the same vertices across the sampling, forward pass, and backward pass phases.
For example, consider the shuffle for layer 1 in Figure~\ref{fig:overview}.
During the sampling phase, GPU 2 sends the IDs of vertices $\{f, g\}$ to GPU 1.
During the forward pass for the same layer, GPU 1 sends the features of $\{f, g\}$ back to GPU 2.

During the forward pass, data flows bottom-up from lower to higher layers.
Consider the execution of layer 1 at GPU 1.
GPU 1 must send the features of vertices $\{f, g\}$ to GPU 2.
Vertex features flow bottom-up through the data structures depicted in Figure~\ref{fig:shuffle-index}.
The \texttt{shuffle} function gets as input the hidden features of the vertices in layer 1 (\texttt{local\_hidden}), which were computed locally (Alg.~\ref{algo:sp-forward}, Line~\ref{ln:shuffle}).
It then uses the layer index as a gather index, rather than a scatter index, to create output messages from these features.
The forward pass sends hidden feature vectors instead of vertex IDs, so all offsets need to be multiplied accordingly.
After the all-to-all operation, the \texttt{shuffle} function at the receivers uses the mixed index as a scatter index to scatter the feature vectors of the input message into the \texttt{mixed\_hidden} tensor.

In the backward pass, information flows top-down like during sampling.
The mixed index is again used as a gather index to build messages and the lower index is used as a scatter index to build arrays of gradients.
If a GPU receives gradients for the same vertex in multiple messages, as in the case of vertex $f$ in Figure~\ref{fig:shuffle-index}, the scatter operation uses an atomic operation to aggregate the values of the gradients.

\begin{figure}[tp]
    \centering
    \includegraphics[width=0.65\columnwidth]{figures/kernel.drawio.pdf}
     \caption{Building the mixed index and offsets at GPU 1 for layer 1 in the example of Figure~\ref{fig:shuffle-index}.}
    \label{fig:kernel}
\end{figure}

\mypar{Dedicated kernels}
\name uses dedicated kernels to build the shuffle index efficiently. 
We now discuss how the mixed index and offset arrays are constructed during sampling. 
The \texttt{split} function builds the mixed index based on an arbitrary partition function that assigns vertices to GPUs. 
\name constructs the mixed index for all messages in a single pass using a fused dedicated kernel.
Figure~\ref{fig:kernel} shows how the kernel builds the mixed index for GPU 1 and layer 1 shown in Figure~\ref{fig:shuffle-index}. 

The fused kernel initially creates a binary \emph{partition array} of size $s \cdot p$, where $s$ is the size of the \ds and $p$ is the number of partitions/GPUs. 
Then, the kernel sets the binary array element at position $p \cdot i$ to 1 if and only if the $i^{th}$ element of the \ds belongs to partition $p$ according to the partitioning function. 
Subsequently, the kernel performs a prefix sum on the binary array and decrements all elements by one. 
If the bit at position $p \cdot i$ is set to 1 in the binary array, the corresponding element in the prefix sum array contains the position of the $i^{th}$ element of the \ds in the upper index. 
The prefix sum array also includes the boundary offset of the $k^{th}$ message at position $s \cdot k - 1$, which is stored in the mixed offset.

\end{comment}
\section{Evaluation}
\label{sec:eval}
In this section, we evaluate \name by answering the following questions: 
What are the end-to-end speedups that can be achieved by \name relative to the baselines (§~\ref{sec:speedups})?
What is the impact of using \name's splitting algorithm, which provides probabilistic performance guarantees (§~\ref{sec:eval-split})?
How does \name scale to a larger number of GPUs within one host and across hosts (§~\ref{sec:eval-scalability})? 
How does performance vary when we vary the hyperparameters (\S~\ref{sec:ablation})?

% We first describe the experiment settings (machines, baselines, etc.) in Section~\ref{sec:settings} and then report our experiment
% results in Sections~\ref{sec:results}-\ref{} to answer the questions. \hui{fill in}

\subsection{Experiment Settings}
\label{sec:settings}

\mypar{GNN models}
We consider two popular and diverse GNN models: GraphSage~\citep{graphsage} and GAT~\citep{velivckovic2017graph}.
We use the standard neighborhood sampling algorithm.
Its low computational complexity makes it less likely to hide the cost of shuffling during \tadj sampling.
By default, we use a sampling fanout of 15, 3 GNN layers, a default hidden size of 256 as used in \citep{graphsage}, and a batch size of 1024. 
% \ms{Describe implementations a bit, including P3*}
% We implemented two GNN models
% , GraphSage and GAT, on top of \name. 
\name's \tadj implementations use the same sampling and training kernels as DGL's data-parallel one.


\mypar{Datasets}
We use three large datasets listed in Table~\ref{tab:dataset}.
The Papers100M dataset is the largest homogeneous dataset from the Open Graph Benchmark (OGB), a standard benchmark for GNN training ~\citep{ogb-node-dataset}.
We use two more large synthetic graphs from the SNAP repository \citep{snapnets} which are commonly used for GNN training evaluation.
\begin{table}[ht]
    \centering
    \begin{tabular}{|l||c|c|c|} \hline
         \textbf{Dataset} & \textbf{\# Nodes} & \textbf{\# Edges} & \textbf{\# Feat} \\ \hline \hline
         %  ogbn-arxiv & .16M & 1.1M & 128 \\ \hline
         % Products & 2.4M & 62M & 100 \\ 
         Orkut & 3.1M & 120M & 512 \\ 
         Papers100M & 111M & 1.6B & 128 \\ 
         Friendster &  65M & 1.9B & 128\\ \hline
    \end{tabular}
    \caption{Datasets used for the evaluation}
    \label{tab:dataset}
    \vspace{-.4cm}
\end{table}

\mypar{Hardware setup} 
By default, our experiments use an AWS EC2 p3.8xlarge instance with 4 NVIDIA V100 GPUs (16GB memory) and Xeon E5-2686 v4  @ 2.70GHz,  with 32 CPU cores and 244 GB RAM. 
GPUs are connected to the CPU with a PCIe 3.0  16 bus and with each other via NVLink.
For experiments with 8 GPUs, we use a similar p3.16xlarge instance having 64 CPU cores and 488 GB RAM.

\begin{table*}[]
\centering
\scalebox{1}{
\begin{tabular}{|c|c|c|c|c|c|c|c|c|c|c|c|}
\hline
\multirow{2}{*}{ Graph } & \multirow{2}{*}{ System } & \multicolumn{5}{c|}{\textbf{GraphSAGE}} & \multicolumn{5}{c|}{\textbf{GAT}} \\
\cline{3-12}
& & S & L & FB & Total(s) & Speedup & S & L & FB & Total(s) & Speedup \\ \hline
\multirow{5}{*}{ Orkut } & DGL & 1.5 & 62.7 & 9.2 & 73.4 & 4.4$\times$& 1.5 & 62.8 & 17.1 & 81.4 & 3.6$\times$\\
& P3* & 4.0 & 1.5 & 8.5 & 14.1 & 0.8$\times$& 4.0 & 1.7 & 37.6 & 43.3 & 1.9$\times$\\
& Quiver & 4.9 & 4.3 & 8.7 & 17.8 & 1.1$\times$& 4.7 & 4.2 & 16.4 & 25.5 & 1.1$\times$\\
%& DistCache & 4.6 & 4.3 & 8.6 & 17.7 & 1.1$\times$& 4.5 & 4.3 & 16.4 & 25.2 & 1.1$\times$\\ 
& Edge & 1.9 & 1.3 & 25.1 & 28.3 & 1.7$\times$& 1.9 & 1.3 & 33.3 & 36.5 & 1.6$\times$\\ 
& \textbf{\name} & 1.9 & 0.1 & 14.8 & 16.7 & & 1.9 & 0.1 & 20.5 & 22.5 & \\  \hline
\multirow{5}{*}{ Papers100M } & DGL & 4.6 & 9.5 & 11.3 & 25.4 & 1.4$\times$& 4.7 & 9.0 & 31.7 & 45.4 & 1.2$\times$\\
& P3* & 3.3 & 11.5 & 25.8 & 40.6 & 2.2$\times$& 3.3 & 11.3 & 65.7 & 80.4 & 2.2$\times$\\
& Quiver & 11.8 & 10.4 & 11.7 & 34.7 & 1.9$\times$& 11.0 & 11.0 & 30.5 & 53.5 & 1.4$\times$\\
%& DistCache & 11.1 & 6.0 & 10.9 & 28.1 & 1.5$\times$& 10.8 & 6.4 & 30.4 & 47.8 & 1.3$\times$\\
& Edge & 11.7 & 0.1 & 16.3 & 28.1 & 1.5$\times$& 11.5 & 0.1 & 40.1 & 51.7 & 1.4$\times$\\
& \textbf{\name} & 3.9 & 2.6 & 11.8 & 18.3 & & 3.8 & 2.3 & 31.1 & 37.2 & \\ \hline
\multirow{5}{*}{ Friendster } & DGL & 62.7 & 283.4 & 61.1 & 407.2 & 2.9$\times$& 62.6 & 284.8 & 245.9 & 593.3 & 1.7$\times$\\
& P3* & 85.9 & 350.8 & 151.5 & 588.1 & 4.1$\times$& 76.5 & 351.4 & 613.8 & 1041 & 3.0$\times$\\
& Quiver & 132.5 & 24.9 & 63.6 & 223.9 & 1.6$\times$& 135.5 & 24.8 & 243.5 & 404.2 & 1.2$\times$\\
%& DistCache & 128.4 & 25.1 & 62.8 & 218.0 & 1.5$\times$& 133 & 24.6 & 241.1 & 399.0 & 1.1$\times$\\
& Edge & 65.7 & 1.0 & 121.8 & 188.5 & 1.3$\times$& 106.1 & 0.7 & 368.2 & 475.0 & 1.4$\times$\\
& \textbf{\name} & 41.2 & 2.2 & 98.9 & 142.3 & & 62.1 & 2.2 & 283.5 & 347.8 & \\ \hline
\end{tabular}
}
\caption{Epoch time (in seconds).  S = Sampling, L = Loading, FB = Forward and backward pass. The speedups are the total epoch time of other systems relative to \name. 
}
\vspace{-0.5cm}
\label{tab:main_nvlink}
\end{table*}


\mypar{Baselines}
We consider the systems described in Section~\ref{sec:problem} as baselines.
All systems perform synchronous training to avoid biasing model accuracy and use GPU-based sampling.

\begin{itemize}
\item \textbf{DGL} is a standard production library for data-parallel GNN training~\citep{dgl}.
We use DGL version 1.1.3, the same one we use as a component of \name. %, which was released in January 2024.
DGL only supports caching input features and the graph topology when they fully fit into one GPU.

\item \textbf{Quiver} is a recent data-parallel GNN training system that uses distributed caches and leverages fast direct GPU-GPU buses like NVLink~\citep{quiver}.
We use version 0.1.1.
Quiver supports distributed and partial caching across multiple GPUs.

\item \bm{$P^3$} is a distributed GNN training system that uses hybrid push-pull parallelism.
Its source code is not publicly available, so we adapt the push-pull parallelism approach to a single-host multi-GPU system and refer to our implementation as \textbf{P3*}. 

%\item \textbf{DistCache} is a baseline we implemented to further evaluate distributed caching with data parallelism. It is a minor improvement of Quiver that optimizes feature loading from host memory as done in DGL.

\item \textbf{Edge} is a variant of \name used to investigate the impact of using a na\"ive offline splitting algorithm that does not weigh vertices and edges using pre-sampling (see Section~\ref{sec:eval-split}). It uses min-cut partitioning and balances the number of edges and target vertices in each partition, as commonly done in data-parallel GNN training systems~\citep{distdgl}, while minimizing the number of edges across partitions.

\end{itemize}

We configure all systems to maximize the memory available for caching the graph structure and input features while allocating sufficient memory to sample and train without going out of memory.
We configure Quiver
and \name to use the same sampling frequency criterion to rank the input features to cache as proposed in~\citep{gnnlab}.
%\name supports distributed caching to store the graph structure, unlike the other systems.
P3* cannot cache input features for only a subset of the vertices, so it only uses caching for the Orkut graph.

% \mypar{Cache configuration}
% All systems cache the graph structure and input features in the GPU if feasible, otherwise they store it on the host-pinned memory, using UVA memory access.
% DGL cannot use input feature caching for the graphs in Table~\ref{tab:dataset} since they exceed the memory capacity of a single GPU.
% Quiver, DistCache, and \name cache as many feature vectors as possible while allocating sufficient memory to sample and train without going out of memory.
% We configure Quiver, DistCache, and \name to use the same sampling frequency criterion to rank the input features to cache as proposed in~\citep{gnnlab}.
% P3* cannot cache input features for only a subset of the vertices, so it only uses caching for the Orkut graph.
% \name supports distributed caching to store the structure of all the graphs, unlike the other systems.

\subsection{End-to-End Performance}
\label{sec:speedups}
\mypar{Overview}
We now compare the performance of \name with existing work: DGL, Quiver, and P3*. 
The comparison is based on epoch time only because none of these systems biases the accuracy of the GNN models they train.
We measure the total epoch time and break it down for the three steps of the mini-batch training iterations: sampling a subgraph, which also includes splitting for \name, loading the input features into each GPU, and performing the forward and backward pass.

We report the results in Table~\ref{tab:main_nvlink}.
Overall, \name outperforms DGL by up to 4.4x (2.5x on average), P3* by up to 4.1x (2.4x on average), and Quiver by up to 1.9x (1.4x on average). 
%It also outperforms our DistCache baseline by up to 1.7x (1.3x on average).
\name consistently reduces the loading time compared to the other systems.
By avoiding redundant computation, \name can reduce its sampling and training costs, mitigate the additional cost of shuffling, and in some cases be even faster than some data-parallel systems in those steps.
Overall, the speedups tend to be higher for larger graphs such as Papers100M and Friendster, which cannot be fully cached on GPU and have higher loading costs.

\mypar{Sampling time comparison}
The sampling step in \name entails not only sampling the mini-batch, as in the other systems, but also splitting the mini-batch, constructing the shuffle indexes, and shuffling vertices.
The evaluation shows that these additional costs are balanced, and sometimes offset, by the elimination of redundant work and by the use of distributed caching for the graph structure.
\name's online splitting is not a performance bottleneck because it is embarrassingly parallel and fast.

% Compared to DGL and P3*, 
% \name has a higher sampling time in most cases. \ms{Update}
% Still, its sampling is faster than Quiver, which 
% has a less efficient sampler implementation.
% The Orkut graph is interesting since its entire topology is cached at each GPU.
% This minimizes the cost of sampling and magnifies \name's additional overheads, such as splitting and shuffling.
% Nonetheless, even in this case, the performance of \name is comparable with the baselines.
% In summary, sampling is more costly in \name than in some data-parallel systems, due to the extra splitting and shuffling cost, but cheaper than in others.
% The additional sampling overhead of \name is outweighed by its gains in terms of loading time, as we will show. 

\mypar{Loading time comparison}
Reducing the input feature loading time by avoiding redundant loads is one of the main goals of \name.
The results show that \name achieves this goal and consistently has the lowest loading time among all systems, graphs, and models.
This is because \name avoids redundant input feature loads from host memory or other GPUs.

DGL has a high data loading costs overall since it does not support distributed caching and none of the graphs we consider fits in a single GPU's memory.

P3* and Quiver perform much better than DGL in the Orkut graph because its input features can be fully cached across GPUs.
The input features for the Papers100M and Friendster graphs cannot be fully cached, even when using a distributed cache.
Quiver has lower loading times than DGL and P3*  for Friendster because it supports caching only a subset of the features in the GPUs, but it cannot leverage its cache effectively for Papers100M because of the high cost of loading cache misses from the host memory.
%DistCache is faster than Quiver on Papers100M thanks to our optimizations, but it still does not outperform \name.
% For Friendster, using a more complex model like GAT impacts the performance of caching.
% GAT requires more memory per mini-batch, which reduces the size of the caches and increases the data loading time due to cache misses.
%This can be seen by comparing the loading times of the GraphSage and GAT models in Quiver and DistCache.

% P3 partitions the the graph feature data along the feature dimension.
% Since P3 does not have in built caching, we can either put the partitioned graph features on GPU or fetch the from host using UVA. 
% For orkut the graph features are partitioned and stored on GPU whereas for the other graphs, P3 fetches these features stored on host using UVA.
% P3 has similar data loading costs to DGL, but for orkut the graph can be cached using distributed caching across all GPUs. 
% For the systems with \textit{NVLINK} the gpus have fast interconnects which allow Quiver to store a global cache fully distributed across all GPUs, on systems with \textit{PCI-E} the each GPU has a replicated cache. 
% To configure the cache size we store the maximum which allows training to complete without out of memory.
% All cache misses are accessed using zero-copy memory similar to DGL.
% Quiver feature loading time is better than DGL and P3, when the graph features are stored on the CPU, as it caches a portion of the features on the GPU while accessing them using the same UVA.
% P3 feature loading is faster than Quiver as P3 loads the partitioned feature data from the GPU where as Quiver data is loaded from distributed GPU memory.
% \name loading time is lower than quiver but as remote gpu memory access is faster than uva based host memory access, this reduction contributes lower to overall speed up.
% Across all the graphs we have lower loading time as vertex data only has to be loaded only one gpu. 
% We see a variance in loading times across models SAGE and GAT for quiver, as GAT consumes higher memory therefore has lesser amount of data available to be cached, causing more cache misses from the distributed gpu cache. 
% \input{tables/partitioning}

\mypar{Forward/backward pass time comparison}
Compared to the data-parallel systems, 
%each GPU performs the forward and backward (FB) passes for each iteration independently, without coordinating with others except to synchronize model parameters.
hybrid parallel systems such as P3* and \name introduce shuffles during training, which typically result in larger FB times.
When training the GraphSage model, P3* can partially compensate for its push-pull shuffle overhead by pushing some of the computation of each micro-batch to all GPUs.
Thanks to this optimization, P3* achieves the lowest training time of all systems on the Orkut graph.
In all the other cases, however, P3* has the largest FB time.
This is because the GPUs must shuffle all the partial activations for all micro-batches.
In particular, more complex models like GAT tend to have large partial activations.

\name mitigates the shuffle overhead by avoiding computing hidden features redundantly.
Its FB times, though, are still larger than in the other data-parallel systems.
The gap is smaller for GAT, which is a more computationally complex model because \name benefits more from avoiding redundant computation.
Compared to P3*, \name always has lower training times except in the case of the Orkut graph with the GraphSage model.
This is thanks to the avoidance of redundant computation and its less expensive shuffles.

\subsection{Evaluation of the Splitting Algorithm}
\label{sec:eval-split}
\mypar{End-to-end performance impact}
\name relies on an offline graph partitioning algorithm to provide probabilistic performance guarantees: balancing the expected load across splits and minimizing the expected communication costs.
We evaluate the impact of using this algorithm by combining \name's online splitting with three alternative offline partitioning algorithms that do not provide these guarantees. 

The \emph{\name} is the pre-sampling-based algorithm with probabilistic guarantees described in Section~\ref{sec:workload}.
The \emph{Node} algorithm partitions the graph using only the pre-sampled node weights. Comparing it to \emph{\name} shows the impact of using edge weights during graph partitioning.
The \emph{Edge} algorithm uses min-cut partitioning but it does not assign weights to vertices and edges using pre-sampling. It balances the number of edges and target vertices in each partition, as commonly done in data-parallel GNN training systems~\citep{distdgl} while minimizing the number of edge cuts across partitions.
Finally, \emph{Rand} partitioning algorithm randomly assigns each vertex to a partition. 

Table~\ref{tab:main_nvlink} shows the end-to-end performance benefit of using \name's splitting algorithm compared to the Edge baseline.
\name helps improve the end-to-end training performance by up to 1.5$\times$ on Orkut, 1.7$\times$ on Papers100M, and 1.4$\times$ on Friendster.

\begin{figure}[ht]
    \centering
    \begin{subfigure}[t]{0.47\linewidth}
        \centering
        \includegraphics[width=\textwidth]{results/simulation/papers100M_edge_epoch10_gsplit_vs_base.pdf}
        % \caption{Workload imbalance of \name vs. others}
        \label{fig:gsplit_partition_vs_others_workload}
    \end{subfigure}
    \hfill
    \begin{subfigure}[t]{0.5\linewidth}
        \centering
        \includegraphics[width=\textwidth]{results/simulation/papers100M_crs_epoch10_gsplit_vs_base.pdf}
        % \caption{Communication overhead of \name vs. other partitioning algorithms}
        \label{fig:gsplit_partition_vs_others_comm}
    \end{subfigure}
    \vspace{-20pt}
    \caption{\name vs. other offline partitioning algorithms.}
    \label{fig:gsplit_partition_vs_others}
\end{figure}

We analyze the reasons for these speedups more in-depth in Figure~\ref{fig:gsplit_partition_vs_others}, which compares the workload imbalance and communication costs of different partition strategies using the Papers100M graph. 
We quantify the \textit{workload imbalance} among the splits in each iteration as the maximum number of edges at layer $l >0 $ per split divided by the average, and the \textit{communication cost} as the percentage of cross-edges among splits over all the edges in the mini-batch.

As shown in Figure~\ref{fig:gsplit_partition_vs_others}, the \emph{Rand} baseline leads to the most evenly distributed computation cost across partitions. 
Yet, it results in a high communication overhead with 75\% of the edges crossing two partitions in most iterations. 
The \emph{Edge} baseline achieves a much lower edge cut and reduces the communication overhead. However, balancing the target vertices alone does not guarantee that the splits of the sampled mini-batches will be balanced.
%Even though we constraint the maximum load imbalance among graph partitions to be up to $1.05$, the actual imbalance among the splits of the sampled mini-batches is notably higher than that, as shown in Figure~\ref{fig:friendster_bal}.

The splitting algorithm of \name achieves the benefit of both approaches. 
It has a lower communication overhead compared to the random partition algorithm and a more balanced workload than simply balancing the number of target vertices in each partition. 
This is thanks to its offline pre-sampling approach.
% In particular, we can see that the splitting algorithm benefits from the probabilistic load balancing guarantees mandated in Eq.~\ref{eqn:prob}, since the load imbalance among splits is now around $1.05$ as expected.
% This lower load imbalance is key to speeding up end-to-end training, even though it constraints the solution space and results in a higher communication cost than the Edge baseline.

In addition, we observe that assigning weights to edges using \name effectively reduces the communication overhead in a mini-batch.
Compared to Node, which does not weigh edges, the average ratio of cross edges over total edges is reduced from 9\% to 5\% for Papers100M as shown in Figure~\ref{fig:gsplit_partition_vs_others}. Better yet, the reduction in communication costs does not significantly impact workload imbalance. We observe a similar trend in Orkut and Friendster. 

\mypar{Cost of the splitting algorithm}
The splitting algorithm has two offline steps: pre-sampling and graph partitioning.
Empirically, we found that running ten epochs of sampling during the pre-sampling stage is sufficient.
Using a larger number of sampling epochs has little impact on load balancing and communication costs. 
When using $30$ and $100$ pre-sampling epochs, the difference in average load imbalance per mini-batch remains within 2\% for all the graphs, while the percentage of cross edges over the total number of edges per mini-batch remains within 7\% for Orkut, 2\% for Papers100M and Friendster. 
Pre-sampling is fast relative to the overall training time.
Using a machine with four RTX 3090 GPUs, it is 19s for Orkut, 20s for Papers100M, and 288s for Friendster. 

The final offline step of \name is graph partitioning, which is commonly used in many distributed mini-batch GNN training systems.
We use METIS~\citep{mtmetis2013ipdps} to partition the graphs on an AWS r7a.x24large instance, which has 48 cores (96 threads) and 768GB of memory. The partitioning time is 14s for Orkut, 78s for Papers100M, and 534s for Friendster.
Both pre-sampling and partitioning are one-time costs that can be amortized by training over the same dataset multiple times. 

% \begin{figure*}[!h]
%     \centering
%     \includegraphics[width=0.85 \textwidth]{results/abalations.pdf}
%     \caption{Scalability and ablation studies on the Papers100M graph. }
%     \label{fig:ablation}
%     \vspace{-.2cm}
% \end{figure*}


% \begin{figure*}[!h]
%     \centering
%     \includegraphics[width= \textwidth]{results/world_size.pdf}
%     \caption{varying number of gpus}
%     \label{fig:varygpus}
% \end{figure*}

% \hui{add scalability study here}


% When varying the number of GPUs from 2 to 8, \name's forward/backward (FB) time scales well because each GPU must process smaller splits. 
% In our 8 GPU host, GPUs are organized in two NVLink cliques of 4 GPUs each.
% The two cliques are connected by an NVSwitch. 
% In this configuration, DistCache (and Quiver) replicates the content of the cache to avoid communication over the NVSwitch.
% This explains why its loading time does not change much when going from 4 to 8 GPUs.



% \begin{figure}[t]
% %    \advance\leftskip-2.2cm
% % \advance\rightskip2cm
% % \includegraphics[keepaspectratio=true,width=.5\textwidth]{results/abalations_scalability.pdf}
% \hspace*{-1cm}
% \includegraphics[keepaspectratio=true,width=.5\textwidth]{results/abalations_FR_scalability.pdf}

%   \caption{Scalability on the Friendster graph 
%   \juelin{How about merge this into Fig 7}
%   \sand{We cant merge as they use a different hidden size,}
%   \ms{use "(a) \# GPUs, single host" and "(b) \# Hosts, 4 GPUs each" as x axis titles. }
%   }
%   \label{fig:friendster_scalabiity}
% \end{figure}

\begin{figure*}[t!]
    \centering
  \includegraphics[width=\textwidth]{results/arxiv_abalations_friendster_combined.pdf}
  \caption{Scalability and ablation study. The reported speedups are the epoch time of other systems relative to \name.}
\label{fig:ablation}
\end{figure*}
\subsection{Scalablity}
\label{sec:eval-scalability}
We show how \name scales to a varying number of GPUs and hosts in Figure~\ref{fig:ablation}.
\textbf{Single-host.} 
We first evaluate using a single host and varying the number of GPUs in Figure~\ref{fig:ablation}(a). 
\name scales better than the other systems with a larger number of GPUs because it has more opportunities to avoid redundant loads and computation.
It can also make more efficient use of the GPU caches thanks to its use of collective communication primitives.
Quiver relies on direct remote memory access to transfer cached input features across GPUs efficiently.
This, however, is only possible between GPUs that have direct NVLink connections.
In our 8 GPU host, not all GPUs are directly connected. Quiver circumvents this problem by replicating cached features across GPUs that have no direct links.
\name, by contrast, does not need to cache features redundantly.
\textbf{Multi-hosts.} We also run distributed multi-host experiments where each host has 4 GPUs and show the results in Figure~\ref{fig:ablation}(b). 
\name uses a hybrid approach to scale to multiple hosts, which uses data parallelism across hosts and \tname within each host. 
All hosts cache the same input features on their GPUs.
% We observe the hybrid approach outperforms pure split parallelism as there is less coordination between 8 GPUs. 
We observe that \name shows consistent speedups in all configurations and models. 



% \begin{figure*}[t]
%     \centering
%     \includegraphics[width=0.85 \textwidth]{results/abalations_new.pdf}
%     \caption{Ablation study on the Friendster graph.}
%     \label{fig:ablation}
% %    \vspace{-.2cm}
% \end{figure*}
% \begin{figure*}[t]
%     \centering
%   \includegraphics[width=0.85\textwidth]{figures/abalations_papers100M.pdf}
%   \caption{Ablation study on the papers100M graph. \sand{The training time is slightly more for quiver then dgl, speedups need to be made consistent with main table.These numbers look too inconsistent for us to use. }}
%     \label{fig:ablation}
% %    \vspace{-.2cm}
% \end{figure*}
\subsection{Ablation Study}
\label{sec:ablation}

We now estimate how consistent \name's speedups are when 
% the number of GPUs in the system or 
the model and training hyperparameters change.
The results for the Friendster graph are reported in Figure~   \ref{fig:ablation}.
\textbf{Hidden size.} Increasing the hidden feature size impacts the FB time of \name negatively, increasing the overall volume of data shuffled. However, it also increases the gains of avoiding redundant computation, especially for complex models such as GAT. 
The two factors balance out and \name shows consistent speedups over the baselines, as shown in Figure~\ref{fig:ablation}(c).
\textbf{Batch size.}
We vary the mini-batch size while keeping a hidden size of 128 to avoid going out of memory. 
Larger mini-batches increase the relative cost of shuffling during the FB phase but also offer more opportunities to save on redundant data loading.
Overall, \name always outperforms the data-parallel baselines, as shown in Figure~\ref{fig:ablation}(d).
\textbf{Number of GNN Layers.}
In this experiment, we use a hidden size of 128 and pick the largest sampling fanout that avoids going out of memory for each number of layers.
The results are reported in Figure~\ref{fig:ablation}(e).
The GNNs are most commonly trained with 2 or 3 layers, i.e., 2 or 3 hops from the target vertices.
\name consistently outperforms the baselines in these settings.
Eventually, adding more layers increases the number of shuffles, making \tname more expensive.
We validate this intuition by evaluating a very deep GNN with 4 layers (4 hops) and observe that the \name is only slightly faster than Quiver for GAT and slower for GraphSage, which is a simpler GNN model where the relative cost of shuffling over computation is higher.
These results suggest that for very deep GNNs a hybrid approach that uses \tname only for the bottom layers and data parallelism for the higher layers would be a promising avenue for future research.

% \begin{figure*}[!h]
%     \centering
%     \includegraphics[width=\textwidth]{results/depth.pdf}
%     \caption{Varying depth}
%     \label{fig:depth}
% \end{figure*}
\input{arxiv-sections/8-relwork}
\section{Conclusion}
This paper presented \tname, a new hybrid parallelism approach for mini-batch training.
Compared to data parallelism and push-pull parallelism, \tname achieves the dual goals of reducing the cost of data loading, by avoiding redundant loads, and keeping the sampling and training cost low, by avoiding redundant computation. 
%Our \name system outperforms baselines using other forms of mini-batch GNN training parallelism by up to 1.9-4.4$\times$. 
%Future work includes extending \tname to optimize its hybrid parallelism strategy given any GNN training program and heterogeneous hardware platform.  
%We embody these ideas in \name, introduce novel splitting algorithms, and a programming API that supports the reuse of efficient sampling and training kernels.
% use the plain bibliography style
\bibliographystyle{plain}
\bibliography{ref}

\end{document}