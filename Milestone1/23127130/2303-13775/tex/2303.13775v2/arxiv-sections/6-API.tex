\section{\name's API} \label{sec:api}
\name offers a \emph{layer-centric} programming API that simplifies the development of GNN sampling and training code and kernels by hiding cross-GPU coordination.
Besides simplicity, this approach supports the reuse of optimized single-GPU kernels for GNN sampling and training, including the ones proposed by recent research~\citep{adaptgear, fastkernel, wu2021seastar, nextdoor, gsampler, fu2022tlpgnn, ye2023sparsetir}. 
% 
% The API introduces two new operators, \textit{split} and \textit{shuffle}, that give each GPU the abstraction of operating only on its local frontiers.
% These two functions reduce the cost of communication by building a shuffle index data structure and reusing it for every shuffle of the same iteration.
% We now describe how to implement split-parallel sampling and training with \name's API and how to use the split and shuffle operators. 

\mypar{Motivation for a layer-centric API}
% Its \emph{scatter} function produces messages for each edge separately based on the features of the source and destination vertices.
% The \emph{gather} function aggregates these messages incrementally using a commutative and associative function, and the \emph{transform} function computes the partial activations that are accumulated using a \emph{sync} function and then given to the \emph{apply} function to compute the hidden features of the destination vertex.
Single-GPU kernels used by data-parallel systems are layer-centric: they assume that at each GNN layer, the source vertex, destination vertex, and edge features for all incoming edges incident to the same destination vertex are locally available.
\name's API supports this property even after it splits destination vertices of the same mini-batch among different GPUs.
The alternative hybrid mini-batch parallelism approach, which is $P^3$, breaks this assumption to offer an \emph{edge-centric} API to enable pushing part of the computation of each micro-batch to multiple hosts and GPUs in a fine-grained manner.
For example, implementing state-of-the-art attention-based models such as Graph Attention Networks (GAT) requires a custom implementation to ensure correctness~\citep{gandhi2021p3}.
% \name's split/shuffle API uses a layer-centric approach and ensures that all incoming edges for the same destination vertex in a layer are available locally at each GPU before starting the layer computation.
% This supports existing single-GPU kernels and modern attention-based GNN models such as GAT.

\mypar{Sampling} \label{sec:sampling}
% In split-parallel sampling, all GPUs sample the same mini-batch.
% Each GPU is responsible for sampling a split of the mini-batch and two splits do not have overlapping vertices.
Algorithm~\ref{algo:sp-sample} shows the pseudocode of sampling using \name's API. 
The \texttt{sp\_sample} function executed by each GPU takes as input the subset of target vertices in the mini-batch that are local to the GPU  according to the splitting algorithm (\texttt{local\_targets}) and produces as output a \emph{split}, which consists of the set of \texttt{edges} at each layer that are used by the local training kernels to aggregate features.
%In the example of Figure~\ref{fig:overview}, for GPU 1, the sampler starts with the local target vertices $\{a, b\}$ and outputs the split consisting of the edges colored in blue.
The function also outputs a \textit{shuffle index} (\texttt{shuffle\_idx}), which consists of gather and scatter indexes to efficiently send and receive sparse vertex data in the frontiers during the all-to-all shuffle rounds at each layer.
%The index will be described more in detail in Section~\ref{sec:implementation}.
% 
% Split-parallel sampling works from top to bottom and layer by layer. 
% It starts from target vertices at layer $L$ and samples the neighbors of these vertices as layer $L-1$. 
% At each layer, $l$, sampling starts from a local frontier of vertices in the GPU's local partition and produces a mixed frontier, which could include remote vertices.
% In order to continue sampling vertices for Layer $l-1$, GPUs must shuffle data to send locally-sampled \emph{remote} vertices and receive remotely-sampled \emph{local} vertices.
% 
In the pseudocode, we abstract away the single-GPU kernels that sample one layer by using the \texttt{sample\_layer} function, which takes the local frontier from the previous layer as input (Line~\ref{ln:sample}).
The function outputs a new mixed frontier (\texttt{mixed\_front}) and the edges from the input local frontier to the output mixed frontier (\texttt{edges[l]}).
The mixed frontier includes both local and remote vertices.
The sampling code must invoke the \texttt{split} function to perform a shuffle and obtain a local frontier (Line~\ref{ln:split}).
The function also produces the shuffle index for the layer (\texttt{shuffle\_idx}).

% In Figure~\ref{fig:overview}, at layer 1, GPU 1 runs \texttt{sample\_layer} on the local frontier $\{a,b\}$ to sample the mixed frontier $\{e,h,f\}$, which includes the remote vertex $h$. 
% GPU 1 then calls the \texttt{split} function to obtain the local frontier for layer 1, which is $\{e,f,g\}$.

% Sampling algorithms can be categorized into individual and collective ones~\citep{nextdoor}.
% The previous discussion considers the case of individual sampling algorithms, which sample the neighbors of a vertex based only on the adjacency list of that vertex. 
% To run collective sampling algorithms such as LADIES~\citep{ladies}, GPUs must broadcast all the vertices they sampled locally so that each GPU gets the entire global frontier.
%Each GPU can then sample a fixed number of neighbors by accessing the adjacency lists stored on the host memory in pinned memory.
% \ms{check} \juelin{Ladies is too lightweight to use GPU, original paper only uses CPU for sampling. So it doesn't make too much sense here to compare it.}

%\mypar{Comparison with data-parallel sampling}
The pseudocode of Algorithm~\ref{algo:sp-sample} is almost exactly the same as the one that would be used in data parallelism.
%In data parallelism, each GPU samples micro-batches independent from each other using single-GPU code.
The most common sampling implementations in DGL~\citep{dgl}, research prototypes, and single-GPU sampling APIs~\citep{nextdoor,c-saw,wang2021skywalker,gsampler}, operate layer-by-layer.
Their behavior can be encapsulated by the same \texttt{sample\_layer} function used in Algorithm~\ref{algo:sp-sample}.
The main difference in data-parallel training is that the function now directly returns the \texttt{local\_front} for the next layer at Line~\ref{ln:sample}.
There is no notion of mixed frontier and no need to invoke the \texttt{split} function in Line~\ref{ln:split}.
\begin{algorithm}[t]
    \caption{Split-parallel sampling.}
    \label{algo:sp-sample}
    \let\textnormal\ttfamily
    \footnotesize

    \DontPrintSemicolon
    \SetStartEndCondition{ }{}{}%
    \SetKwProg{Fn}{def}{\string:}{}
    \SetKwFunction{Range}{range}%%
    \SetKw{KwTo}{in}\SetKwFor{For}{for}{\string:}{}% \SetKwIF{If}{ElseIf}{Else}{if}{:}{elif}{else:}{}% \SetKwFor{While}{while}{:}{fintq}%
    \newcommand{\forcond}{\em l \KwTo [L-1, 0]}
\AlgoDontDisplayBlockMarkers\SetAlgoNoEnd\SetAlgoNoLine%
    \SetKwFunction{dpsample}{dp\_sample}
    \SetKwFunction{splitsample}{sp\_sample}
    \SetKwFunction{sneigh}{\textbf{\color{teal}sample\_layer}}
    \SetKwFunction{unique}{unique}
    \SetKwFunction{union}{union}
    \SetKwFunction{local}{local\_unique}
    \SetKwFunction{concat}{concat}
    \SetKwFunction{csplit}{\textbf{\color{purple} add}}
    \SetKwFunction{split}{\textbf{\color{purple} split}}
    \SetKwFunction{localfrontier}{local\_frontier}
    \SetKwFunction{dst}{dst}
    \SetKwFunction{src}{src}
    \SetKwFunction{add}{add}
    \SetKwFunction{remap}{remap}    
    \SetKwFunction{getall}{getAll}    
    \SetKwFunction{bsplit}{\textbf{create\_sp\_frontier}}
    \SetKwFunction{shuffleidx}{shuffle\_idx}


    % \Fn{\dpsample{\em local\_targets}}{
    %     front[L] = local\_targets\;
    %     local\_ids.\add{front[L]}\;
    %     \For{\forcond}{
    %         samp\_edges[l] = \\ $\quad$ \sneigh{front[l+1]}\;
    %         local\_ids.\add{\em samp\_edges[l].\dst{}}\;
    %         front[l] = local\_ids.\getall{}\;
    %         edges[l] = \\ $\quad$ local\_ids.\remap{samp\_edges[l]}\;
    %     }
    %     \Return front, edges\;
    % }    
    % \;
    \Fn{\splitsample{\em local\_targets}}{
        local\_front[L] = local\_targets\;
        \For{\forcond}{
            mixed\_front, edges[l] = \sneigh{local\_front[l+1]}\label{ln:sample}\;
            local\_front[l], edges[l], shuffle\_idx[l] = \split{mixed\_front, edges[l]}\label{ln:split}\;
        }
        \Return edges, \shuffleidx\;
    }    
\end{algorithm}

\begin{algorithm}[t]
    \caption{Forward pass of split-parallel training.}
    \label{algo:sp-forward}
    \let\textnormal\ttfamily
    \footnotesize
    \DontPrintSemicolon
    \SetStartEndCondition{ }{}{}%
    \SetKwProg{Fn}{def}{\string:}{}
    \SetKwFunction{Range}{range}%%
    \SetKw{KwTo}{in}\SetKwFor{For}{for}{\string:}{}% \SetKwIF{If}{ElseIf}{Else}{if}{:}{elif}{else:}{}% \SetKwFor{While}{while}{:}{fintq}%
    \newcommand{\forcond}{\em l \KwTo [0, L-1]}
\AlgoDontDisplayBlockMarkers\SetAlgoNoEnd\SetAlgoNoLine%
    \SetKwFunction{dpfwpass}{dp\_forward}
    \SetKwFunction{splitfwpass}{sp\_forward}
    \SetKwFunction{local}{local\_unique}
    \SetKwFunction{csplit}{create\_upper\_split}
    \SetKwFunction{split}{{\color{turquoise} split}}
    \SetKwFunction{dst}{dst}
    \SetKwFunction{src}{src}
    \SetKwFunction{add}{add}
    \SetKwFunction{remap}{remap}    
    \SetKwFunction{getall}{getAll}    
    \SetKwFunction{shuffle}{\textbf{\color{purple}shuffle}}
    \SetKwFunction{splitidx}{shuffle\_idx}

    % \Fn{\dpfwpass{\em model, in\_feat, edges}}{
    %     hidden = in\_feat \;
    %     \For{\forcond}{
    %         gnn\_layer = model.layer[l]\;
    %         hidden = gnn\_layer(edges[l], hidden)\;
    %     }
    %     \Return hidden\;
    % }    
    % \;
    \Fn{\splitfwpass{\em model, feat, edges, \splitidx}}{
        hidden = feat\; \label{ln:input}
        \For{\forcond}{
            \textbf{\color{teal} gnn\_layer} = model.layer(l)\;
            mixed\_hidden = \shuffle{\splitidx[l], local\_hidden} \label{ln:shuffle}\;
            local\_hidden = \textbf{\color{teal} gnn\_layer}(edges[l], mixed\_hidden)\label{ln:layer}\;
        }
        \Return hidden;
    }    
\end{algorithm}


\begin{comment}
 ==== LADIES discussion

The main goal of \name is to speed up loading, which only requires training to be \tadj as discussed in Section~\ref{sec:overview}.
This gives the \name runtime some flexibility on how to run sampling code written with its API.
Our current implementation based on \tadj sampling supports \emph{individual} sampling algorithms, such as the common neighborhood sampling~\citep{graphsage}, where the neighbors of each vertex in the frontier are sampled independently.
The \name runtime can also run the sampling code of Algorithm~\ref{algo:sp-sample} in a centralized fashion, where each GPU samples a different mini-batch independently and produces splits and shuffle indexes for all GPUs.
The split function does not perform a shuffle.
A centralized implementation supports running \emph{collective} sampling algorithms such as LADIES~\citep{ladies}, which require accessing the entire frontier to sample the next layer.
We leave this extension for future work.
    
\end{comment}



\mypar{Training}
\label{sec:training}
We now explain how to implement cooperative split-parallel training.
Algorithm~\ref{algo:sp-forward} shows the pseudocode for the forward propagation. 
The backward propagation works similarly except that the computation happens from the top layer to the bottom layer. 

The \texttt{sp\_forward} function takes as input the GNN model (\texttt{model}) the input features of the vertices in the local split (\texttt{feat}), the structure of the split (\texttt{edges}), and the shuffle index (\texttt{shuffle\_idx}).
The latter two inputs are produced by the sampling code of Algorithm~\ref{algo:sp-sample}.
It produces as output the hidden features of the target vertices.
% The pseudocode iterates over all GNN layers.
% The GNN layer $l$ at one GPU computes the hidden features for layer $l+1$ locally by using single-GPU code, which we abstract away with the \texttt{gnn\_layer} function.
% Split-parallel training interleaves feature shuffling with the computation of a GNN layer. 
At each layer $l$, each GPU starts by shuffling the features/activations of its local vertices to other GPUs using the \texttt{shuffle} function provided by \name
(Line~\ref{ln:shuffle}). 
% At layer $0$, these features are the input features of the local vertices (Line~\ref{ln:input}).
The output of this function is the \texttt{mixed\_hidden} tensor, which contains all the features required to compute the hidden features at layer $l+1$.
These could include the features of remote vertices.
The GNN layer then computes the next-layer hidden features for the vertices in the local partition of the GPU (Line~\ref{ln:layer}).

% We use the example in Figure~\ref{fig:overview} to illustrate the forward pass. 
% To compute layer 1's features for the vertices in their partitions, GPUs 1 and 2 require layer 0's features as input. 
% GPU 1 requires the input features of the local vertices $\{j, k, l\}$ and the remote vertex $\{m\}$. 
% Similarly, GPU 2 requires the local vertices $\{m, p\}$ and the remote vertices $\{j, k\}$. 
% The \texttt{shuffle} function exchanges the input features of the remote vertices, $\{m\}$ and $\{j, k\}$, between the two GPUs so that the GPUs can compute the hidden features of layer 1's vertices. 
% These features are added to the local features to form the \texttt{mixed\_hidden} tensor.
%%%After layer 1's computation finishes, GPU 1 and GPU 2 move to the next layer and again shuffle the features of remote vertices, $\{f, g\}$ and $\{h\}$ from layer 1, and then proceed to compute layer 2's features for the target vertices. 

%\mypar{Comparison with data-parallel training}
Data-parallel training code is similar to the pseudocode in Algorithm~\ref{algo:sp-forward}.
%The implementation of the \texttt{model} is the same as the \tadj implementation.
The main difference is that there is only a local frontier, so there is no need to use the \texttt{shuffle} function to get a mixed frontier and we can skip Line~\ref{ln:shuffle}.
Each GNN layer directly takes as input the local frontier (\texttt{local\_hidden}) from the previous layer at Line~\ref{ln:layer} and produces the local frontier for the next layer.
Optimized single-GPU training kernels can be directly used as \texttt{gnn\_layer} implementations since they operate on one layer at a time~\citep{adaptgear, fastkernel, wu2021seastar, fu2022tlpgnn, ye2023sparsetir}.


\begin{comment}
==== Implementation section 
\begin{figure}[tp]
    \centering
    \includegraphics[width=\columnwidth]{figures/index2.drawio.pdf}
     \caption{Execution of the \texttt{split} function at GPU 1 for layer 1 in the example of Figure~\ref{fig:overview}. Vertices assigned to GPU 1 (resp. GPU 2) are colored in blue (resp. red).}
    \label{fig:shuffle-index}
\end{figure}

\section{\name Implementation}
\label{sec:implementation}
\name is implemented based on the DGL and consists of approximately 5600 lines of C++ and CUDA code and 200 lines of Python code.
% The \name API gives each GPU the abstraction of operating on a local split.
% To make each split fit into a single GPU and reduce memory and computation costs in the presence of arbitrarily large graphs, \name uses a compact representation of the split subgraph that considers only local vertices and edges by remapping each vertex ID to a local vertex ID.
% The \texttt{split} and \texttt{shuffle} functions must map between different local representations when serializing and deserializing messages.
We now discuss how the split and shuffle functions construct the shuffle index and use it to perform the mapping.

\mypar{Split function}
Figure~\ref{fig:shuffle-index} illustrates the execution of the \texttt{split} function at GPU 1 for layer 1 in the example of Figure~\ref{fig:overview}.
The function (Alg.~\ref{algo:sp-sample}, Line~\ref{ln:split}) takes a \ds as input and outputs the next layer of the split,
% (\texttt{edges[l]} in Line 5 of Algorithm~\ref{algo:sp-sample}), 
% the shuffle index, (\texttt{shuffle\_idx}), 
which is depicted in Figure~\ref{fig:shuffle-index}, and the next local frontier,
% (\texttt{local\_front[l]}), 
which is not depicted.
The shuffle index built by the \texttt{split} function contains four data structures: the mixed index, the mixed offsets, the layer index, and the layer offsets.

To shuffle messages using the NCCL all-to-all primitive, a sender GPU must gather vertex IDs from a sparse tensor (the \ds) into a dense tensor (the message). 
NCCL requires a message tensor as input, so all vertices in the \ds that belong to the same partition must be placed in contiguous memory locations.
The mixed index is used as a gather index that indicates the corresponding position in the \ds for each position in the message array. 
NCCL also requires the offsets of the boundaries of each message, which are contained in the mixed offsets array.

After the all-to-all shuffle, each GPU receives messages containing only the vertices that are local to its partition. 
The \texttt{split} function adds these vertices to the local layer using a scatter operation.
This is done by creating the layer index as a scatter index, which is an array that maps each position in the message to the local ID assumed by the vertex stored at that position in the message.

\mypar{Shuffle function}
The \texttt{shuffle} function reuses the shuffle index to make the hidden representation and the gradients flow between different splits.
The index can be reused because the shuffle at a layer always exchanges information about the same vertices across the sampling, forward pass, and backward pass phases.
For example, consider the shuffle for layer 1 in Figure~\ref{fig:overview}.
During the sampling phase, GPU 2 sends the IDs of vertices $\{f, g\}$ to GPU 1.
During the forward pass for the same layer, GPU 1 sends the features of $\{f, g\}$ back to GPU 2.

During the forward pass, data flows bottom-up from lower to higher layers.
Consider the execution of layer 1 at GPU 1.
GPU 1 must send the features of vertices $\{f, g\}$ to GPU 2.
Vertex features flow bottom-up through the data structures depicted in Figure~\ref{fig:shuffle-index}.
The \texttt{shuffle} function gets as input the hidden features of the vertices in layer 1 (\texttt{local\_hidden}), which were computed locally (Alg.~\ref{algo:sp-forward}, Line~\ref{ln:shuffle}).
It then uses the layer index as a gather index, rather than a scatter index, to create output messages from these features.
The forward pass sends hidden feature vectors instead of vertex IDs, so all offsets need to be multiplied accordingly.
After the all-to-all operation, the \texttt{shuffle} function at the receivers uses the mixed index as a scatter index to scatter the feature vectors of the input message into the \texttt{mixed\_hidden} tensor.

In the backward pass, information flows top-down like during sampling.
The mixed index is again used as a gather index to build messages and the lower index is used as a scatter index to build arrays of gradients.
If a GPU receives gradients for the same vertex in multiple messages, as in the case of vertex $f$ in Figure~\ref{fig:shuffle-index}, the scatter operation uses an atomic operation to aggregate the values of the gradients.

\begin{figure}[tp]
    \centering
    \includegraphics[width=0.65\columnwidth]{figures/kernel.drawio.pdf}
     \caption{Building the mixed index and offsets at GPU 1 for layer 1 in the example of Figure~\ref{fig:shuffle-index}.}
    \label{fig:kernel}
\end{figure}

\mypar{Dedicated kernels}
\name uses dedicated kernels to build the shuffle index efficiently. 
We now discuss how the mixed index and offset arrays are constructed during sampling. 
The \texttt{split} function builds the mixed index based on an arbitrary partition function that assigns vertices to GPUs. 
\name constructs the mixed index for all messages in a single pass using a fused dedicated kernel.
Figure~\ref{fig:kernel} shows how the kernel builds the mixed index for GPU 1 and layer 1 shown in Figure~\ref{fig:shuffle-index}. 

The fused kernel initially creates a binary \emph{partition array} of size $s \cdot p$, where $s$ is the size of the \ds and $p$ is the number of partitions/GPUs. 
Then, the kernel sets the binary array element at position $p \cdot i$ to 1 if and only if the $i^{th}$ element of the \ds belongs to partition $p$ according to the partitioning function. 
Subsequently, the kernel performs a prefix sum on the binary array and decrements all elements by one. 
If the bit at position $p \cdot i$ is set to 1 in the binary array, the corresponding element in the prefix sum array contains the position of the $i^{th}$ element of the \ds in the upper index. 
The prefix sum array also includes the boundary offset of the $k^{th}$ message at position $s \cdot k - 1$, which is stored in the mixed offset.

\end{comment}