\section{Conclusions}\label{sec:conclusions}
%

%We have studied the data complexity of the why-provenance problem for Datalog queries and central subclasses thereof (linear and non-recursive). 

The takeaway of our work is that for recursive queries the why-provenance problem is, in general, intractable, whereas for non-recursive queries it is highly tractable in data complexity. 
%
With the aim of overcoming the conceptual limitations of arbitrary proof trees, we considered the new class of unambiguous proof trees and showed that it does not affect the data complexity of the why-provenance problem.
%
Interestingly, we have experimentally confirmed that unambiguous proof trees help to exploit off-the-shelf SAT solvers towards an efficient computation of the why-provenance.
%
Note that we have performed a preliminary comparison with~\cite{ElKM22} by focusing on a setting that both approaches can deal with. In particular, we used the scenarios $\mathsf{Doctors}\text{-}i$, for $i \in [7]$, and measured the end-to-end runtime of our approach (not the delays). For the simple scenarios, the two approaches are comparable in the order of a second. For the demanding scenarios ($\mathsf{Doctors}\text{-}i$ for $i \in \{1,5,7\}$), our approach is generally faster.
%and that for some of the most demanding cases, the approach of~\cite{ElKM22} runs out of memory. 
%For further details on this comparative evaluation we refer the reader to the extended version of the paper.

%There are still interesting technical problems that are waiting to be studied. 

It would be extremely useful to provide a complete classification of the data complexity of the why-provenance problem in the form of a dichotomy result. It would also provide further insights to pinpoint the combined complexity of the problem, where the Datalog query is part of the input. Finally, it is crucial to perform a more thorough experimental evaluation of our SAT-based machinery in order to understand better whether it can be applied in practice.