\documentclass{article}
\pdfpagewidth=8.5in
\pdfpageheight=11in

\usepackage{kr}

\usepackage{times}
\usepackage{soul}
\usepackage{url}
\usepackage[hidelinks]{hyperref}
\usepackage[utf8]{inputenc}
\usepackage[small]{caption}
\usepackage{graphicx}
\usepackage{amsmath}
\usepackage{amsthm}
\usepackage{booktabs}
\usepackage{algorithm}
\usepackage{algorithmic}
\urlstyle{same}
\usepackage{comment}
\usepackage{amssymb}

% the following package is optional:
%\usepackage{latexsym}

\definecolor{purple}{rgb}{1, 0, 1}

\newcommand{\ie}{\emph{i.e.,}\xspace}
\newcommand{\eg}{\emph{e.g.,}\xspace}
\newcommand{\abr}{\emph{abbr.}\xspace}
\newcommand{\ea}{\emph{et al.}\xspace}
\newcommand{\gensync}{\emph{GenSync}\xspace}
\newcommand{\colosseum}{\emph{Colosseum}\xspace}
\newcommand{\srep}{\emph{SREP}\xspace} % Set Reconciliation Enhances
\newcommand{\srepsim}{\emph{SREPSim}\xspace}
% Propagation
\newcommand{\esrep}{\emph{E-SREP}\xspace}
\newcommand{\epsrep}{\emph{EP-SREP}\xspace}
\newcommand{\mesrep}{\emph{ME-SREP}\xspace}
\newcommand{\mempoolsync}{\emph{MempoolSync}}

\newcommand{\fref}[1]{Fig.~\ref{#1}}
\newcommand{\tref}[1]{Table~\ref{#1}}
\newcommand{\aref}[1]{Algorithm~\ref{#1}}
\newcommand{\procref}[1]{Procedure~\ref{#1}}
\newcommand{\sref}[1]{Section~\ref{#1}}
\newcommand{\lineref}[1]{line~\ref{#1}}
\newcommand{\appref}[1]{Appendix~\ref{#1}}

% Change \eqref
\LetLtxMacro{\originaleqref}{\eqref}
\renewcommand{\eqref}{Eq.~\originaleqref}

% Theorems and corollaries
\newcounter{theoremcount}
\setcounter{theoremcount}{0}
\DeclareRobustCommand{\theorem}[1]{%
  \refstepcounter{theoremcount}%
  \noindent\textit{\textbf{Theorem \thetheoremcount\label{theorem:#1}: }}%
}
\DeclareRobustCommand{\theoremref}[1]{Theorem~\ref{theorem:#1}}

\DeclareRobustCommand{\proof}{\emph{Proof:}\xspace}
\DeclareRobustCommand{\qqed}{\hfill$\blacksquare$}

\newcounter{corollcount}
\setcounter{corollcount}{0}
\DeclareRobustCommand{\coroll}[1]{%
  \refstepcounter{corollcount}%
  \noindent\textit{\textbf{Corollary \thecorollcount\label{coroll:#1}: }}%
}
\DeclareRobustCommand{\corollref}[1]{Corollary~\ref{coroll:#1}}

\newcounter{lemmacount}
\setcounter{lemmacount}{0}
\DeclareRobustCommand{\lemma}[1]{%
  \refstepcounter{lemmacount}%
  \noindent\textit{\textbf{Lemma \thelemmacount\label{lemma:#1}: }}%
}
\DeclareRobustCommand{\lemmaref}[1]{Lemma~\ref{lemma:#1}}

\newcounter{definitioncount}
\setcounter{definitioncount}{0}
\DeclareRobustCommand{\definition}[1]{%
  \refstepcounter{definitioncount}%
  \noindent\textit{\textbf{Definition \thedefinitioncount\label{definition:#1}: }}%
}
\DeclareRobustCommand{\defref}[1]{Definition~\ref{definition:#1}}

%notes of different authors
\newif\ifnotes
\notestrue
\notesfalse

\newif\ifdiff
\difftrue
\difffalse

\newcommand{\anote}[1]{\ifnotes $\ll$\textsf{\textcolor{purple}{Ari: {#1}}}$\gg$ \fi}
\newcommand{\nnote}[1]{\ifnotes $\ll$\textsf{\textcolor{orange}{Novak: {#1}}}$\gg$ \fi}
\newcommand{\diff}[1]{\ifdiff\textcolor{orange}{#1}\else#1\fi}

%%% Local Variables:
%%% mode: latex
%%% TeX-master: "main"
%%% End:


\newtheorem{example}{Example}
\newtheorem{theorem}{Theorem}
\newtheorem{corollary}[theorem]{Corollary}
\newtheorem{proposition}[theorem]{Proposition}
\newtheorem{lemma}[theorem]{Lemma}
\newtheorem{definitionAux}[theorem]{Definition}
\newenvironment{definition}{\begin{definitionAux}
	}{\end{definitionAux}}


\pdfinfo{
/TemplateVersion (KR.2022.0, KR.2023.0)
}


\title{The Complexity of Why-Provenance for Datalog Queries}

\author{
	Marco Calautti$^1$\and
	Ester Livshits$^2$\and
	Andreas Pieris$^{2,3}$\and
	Markus Schneider$^2$\\
	\affiliations
	$^1$Department of Computer Science, University of Milan\\
	$^2$School of Informatics, University of Edinburgh\\
	$^3$Department of Computer Science, University of Cyprus\\[1mm]
	\emails 
	marco.calautti{@}unimi.it, ester.livshits{@}ed.ac.uk,
	apieris{@}inf.ed.ac.uk, m.schneider{@}ed.ac.uk
}

\begin{document}

\maketitle

%\begin{comment}
\begin{abstract}
Explaining why a database query result is obtained is an essential task towards the goal of Explainable AI, especially nowadays where expressive database query languages such as Datalog play a critical role in the development of ontology-based applications. A standard way of explaining a query result is the so-called why-provenance, which essentially provides information about the witnesses to a query result in the form of subsets of the input database that are sufficient to derive that result. To our surprise, despite the fact that the notion of why-provenance for Datalog queries has been around for decades and intensively studied, its computational complexity remains unexplored. The goal of this work is to fill this apparent gap in the why-provenance literature. Towards this end, we pinpoint the data complexity of why-provenance for Datalog queries and key subclasses thereof. The takeaway of our work is that why-provenance for recursive queries, even if the recursion is limited to be linear, is an intractable problem, whereas for non-recursive queries is highly tractable. Having said that, we experimentally confirm, by exploiting SAT solvers, that making why-provenance for (recursive) Datalog queries work in practice is not an unrealistic goal.
\end{abstract}
%\end{comment}

\section{Introduction}
\label{sec:introduction}
% \begin{itemize}
%     % Diffusion of FL
%     \item {\st{Diffusion of FL}}
%     % Security threats to FL
%     \item {\st{Security threats to FL with particular focus on model poisoning}}
%     % Limitations of existing countermeasures
%     \item {\st{Current countermeasures (e.g., KRUM) and their limitations}}
%     % Proposed method and its advantages
%     \item {\st{Intuitive description of the proposed method and its difference (i.e., advantages) w.r.t. state of the art}}
%     % Main contributions
%     \item {\st{Summary of the main contributions of this work}}
%     % Paper's structure and organization
%     \item {\st{Paper's structure and organization}}
% \end{itemize}

% Diffusion of FL
Recently, {\em federated learning} (FL) has emerged as the leading paradigm for training distributed, large-scale, and privacy-preserving machine learning (ML) systems~\cite{mcmahan2017googleai,mcmahan2017aistats}. 
The core idea of FL is to allow multiple edge clients to collaboratively train a shared, global model without disclosing their local private training data.
%Specifically, an FL system consists of a central server and many edge clients; 
A typical FL round involves the following steps: {\em(i)} the server randomly picks some clients and sends them the current, global model; {\em(ii)} each selected client locally trains its model with its own private data; then, it sends the resulting local model to the server;\footnote{Whenever we refer to global/local model, we mean global/local model {\em parameters}.} {\em(iii)} the server updates the global model by computing an \emph{aggregation function}, usually the average (FedAvg), on the local models received from clients.
% \begin{enumerate}
%     \item[{\em(i)}] the server sends the current, global model to the clients and appoints some of them for training;
%     \item[{\em(ii)}] each selected client locally trains its copy of the global model with its own private data; then, it sends the resulting local model back to the server;\footnote{Whenever we refer to global/local model, we mean global/local model {\em parameters}.}
%     \item[{\em(iii)}] the server updates the global model by computing an \emph{aggregation function} on the local models received from clients (by default, the average, also referred to as FedAvg~\cite{mcmahan2017aistats}).
% \end{enumerate}
This process goes on until the global model converges. %(e.g., after a certain number of rounds or other similar stopping criteria).
%\\
% The advantages of FL over the traditional, centralized learning paradigm are undoubtedly clear in terms of flexibility/scalability (clients can join/disconnect from the FL network dynamically), network communications (only model weights\footnote{We will use \textit{parameters} and \textit{weights} interchangeably.} are exchanged between clients and server), and privacy (each client's private training data is kept local at the client's end and not uploaded to the server).
\\
% Security threats to FL
%However, the growing adoption of FL also raises security concerns~\cite{costa2022covert}, particularly about its confidentiality, integrity, and availability.
Although its advantages over standard ML, FL also raises security concerns~\cite{costa2022covert}. %, particularly about its confidentiality, integrity, and availability~\cite{costa2022covert}.
% OLD, LONG VERSION
% Indeed, some work deals with privacy leakage that may expose the local data of some clients~\cite{melis2019sp}. 
% A large body of work, instead, investigates attacks that usually aim to detriment the predictive accuracy of the learned global model. For instance, \emph{data poisoning} attacks achieve this goal by letting an adversary pollute the training set of some corrupt FL clients with maliciously crafted examples~\cite{jagielski2018sp}.
% Similarly, in \emph{model poisoning} the attacker attempts to tweak the global model weights~\cite{bhagoji2019pmlr} by directly perturbing the local model's weights of some infected FL clients before these are sent to the central server for aggregation, usually via so-called Byzantine attacks. 
% It turns out that Byzantine model poisoning attacks severely impact standard FedAvg; therefore, more robust aggregation functions must be designed to make FL systems secure.
Here, we focus on \emph{untargeted model poisoning} attacks~\cite{bhagoji2019pmlr}, where an adversary attempts to tweak the global model weights %\footnote{We will use the terms \textit{parameters} and \textit{weights} interchangeably.} 
by directly perturbing the local model's parameters of some infected clients before these are sent to the central server for aggregation.
In doing so, the adversary aims to jeopardize the global model \textit{indiscriminately} at inference time.
Such model poisoning attacks severely impact standard FedAvg; therefore, more robust aggregation functions must be designed to secure FL systems.
\\
% In this paper, we focus on designing a novel robust aggregation scheme at the server's end to contrast the effect of Byzantine model poisoning attacks.
%
% Current countermeasures and their limitations
%Several countermeasures have been proposed in the literature to combat model poisoning attacks on FL systems.
% Some methods use simple statistics more robust than plain average to smooth the impact of malicious updates (e.g., Trimmed Mean and FedMedian~\cite{yin2018icml}). 
% Other defenses implement outlier detection techniques to discard malicious updates from the aggregation performed at the server's end. Those are either based on heuristics (e.g., Krum/Multi-Krum~\cite{blanchard2017nips} and Bulyan~\cite{mhamdi2018pmlr}) or data-driven approaches (e.g., K-means clustering~\cite{shen2016acm} or DnC via spectral analysis~\cite{shejwalkar2021ndss}). 
% Finally, some strategies rely on a centralized ``source of trust'' to spot potential malicious updates (e.g., FLTrust~\cite{cao2020fltrust}).
% Several countermeasures have been proposed in the literature to combat model poisoning attacks on FL systems, i.e., to discard possible malicious local updates from the aggregation performed at the server's end. 
% These techniques range from simple statistics more robust than plain average (e.g., Trimmed Mean and FedMedian~\cite{yin2018icml}) to outlier detection heuristics (e.g., Krum/Multi-Krum~\cite{blanchard2017nips} and Bulyan~\cite{mhamdi2018pmlr}) or data-driven approaches (e.g., spectral analysis via K-means clustering~\cite{shen2016acm} or spectral analysis), or methods based on ``source of trust'' (e.g., FLTrust~\cite{cao2020fltrust}).
% OLD, LONG VERSION
%Several countermeasures have been proposed in the literature to combat Byzantine model poisoning attacks on FL systems.
% Descriptive statistics
% For example, Trimmed Mean and FedMedian aggregate local model updates using more robust statistics than standard average~\cite{yin2018icml}.
%
% % Heuristics for outlier detection
% Many existing Byzantine-resilient strategies implement some outlier detection heuristics to discard the model updates sent by potentially malicious clients from the input of the aggregation function.
% One of the most popular heuristics is Krum~\cite{blanchard2017nips}.
% This strategy tries to mitigate the impact of Byzantine attacks by selecting as a global model the local model with the smallest sum of Euclidean distances to {\em all} the other local models.
% Although powerful, Krum requires the server to know (or, at least, estimate) the number of malicious FL clients upfront, which is generally impossible in a realistic attack scenario. %
% Moreover, Krum may become ineffective for complex, high-dimensional model parameter spaces due to the curse of dimensionality.
% Bulyan~\cite{mhamdi2018pmlr} tries to overcome this issue by combining Krum with a variant of Trimmed Mean.
% % Data-driven outlier detection
% Other strategies use data-driven outlier detection techniques -- e.g., via K-means clustering~\cite{shen2016acm} -- to spot potential malicious local model updates. 
% %For instance, Shen et al. propose to cluster local model updates with K-means and thus identify outliers.
%
% % Other techniques
% As far as the server is concerned, any local model received can be from a potential malicious client. 
% FLTrust~\cite{cao2020fltrust} assumes the server acts as a client, i.e., trains a local model on an additional {\em trustworthy} dataset at the server's end and compares it against all the local models from other clients. 
% This way, the server can rely on some ``source of trust'' when discarding potentially malicious clients.
%\\
% Limitations of existing Byzantine-resilient strategies
Unfortunately, existing defense mechanisms either rely on simple heuristics (e.g., Trimmed Mean and FedMedian by~\cite{yin2018icml}) or need strong and unrealistic assumptions to work effectively (e.g., foreknowledge or estimation of the number of malicious clients in the FL system, as for Krum/Multi-Krum~\cite{blanchard2017nips} and Bulyan~\cite{mhamdi2018pmlr}, which, however, cannot exceed a fixed threshold).
Furthermore, outlier detection methods using K-means clustering~\cite{shen2016acm} or spectral analysis like DnC~\cite{shejwalkar2021ndss} do not directly consider the temporal evolution of local model updates received.
Finally, strategies like FLTrust~\cite{cao2020fltrust} require the server to collect its own dataset and act as a proper client, thereby altering the standard FL protocol.
\\
% OLD, LONG VERSION
% Overall, existing Byzantine-resilient strategies are either simple heuristics (e.g., FedMedian) or, if they are more complex, they rely on strong and unrealistic assumptions to work effectively (e.g., knowing the number of malicious clients in the FL system in advance, as for Krum and alike).
% Furthermore, data-driven outlier detection methods do not consider the temporary evolution of local model updates received (e.g., K-means clustering). 
% Finally, strategies like FLTrust requires the server to collect its own dataset and act as a proper client, thereby altering the standard FL protocol.
%
% Description of the proposed method
This work introduces a novel pre-aggregation \textit{filter} robust to untargeted model poisoning attacks. Notably, this filter $(i)$ operates without requiring prior knowledge or constraints on the number of malicious clients and $(ii)$ inherently integrates temporal dependencies. 
The FL server can employ this filter as a preprocessing step before applying \textit{any} aggregation function, be it standard like FedAvg or robust like Krum or Bulyan.
Specifically, we formulate the problem of identifying corrupted updates as a multidimensional (i.e., matrix-valued) time series anomaly detection task. 
The key idea is that legitimate local updates, resulting from well-calibrated iterative procedures like stochastic gradient descent (SGD) with an appropriate learning rate, show \textit{higher predictability} compared to malicious updates. This hypothesis stems from the fact that the sequence of gradients (thus, model parameters) observed during legitimate training exhibit regular patterns, as validated in Section~\ref{subsec:intuition}. %until convergence. 
%This regularity may be more pronounced for smooth convex loss functions, but it can still be captured within an appropriate time window, even for more complex and convoluted loss surfaces. 
%We provide evidence of this claim in Appendix~B, where we show that the average mutual information (i.e., ``predictability''), calculated over pairs of legitimate model updates sent at different FL rounds, is significantly higher than the corresponding computation for a malicious client.
\\
Inspired by the matrix autoregressive (MAR) framework for multidimensional time series forecasting~\cite{chen2021je}, we propose the FLANDERS ({\em \textbf{F}ederated \textbf{L}earning meets \textbf{AN}omaly \textbf{DE}tection for a \textbf{R}obust and \textbf{S}ecure}) filter.
The main advantages of FLANDERS over existing strategies like FLDetector~\cite{zhao2020multivariate} are its resilience to large-scale attacks, where $50\%$ or more FL participants are hostile, and the capability of working under realistic non-iid scenarios.
We attribute such a capability to two key factors: $(i)$ FLANDERS works without knowing a priori the ratio of corrupted clients, and $(ii)$ it embodies temporal dependencies between intra- and inter-client updates, quickly recognizing local model drifts caused by evil players. Below, we summarize our main contributions:

\begin{itemize}
\item[{\em(i)}]
We provide empirical evidence that the sequence of models sent by legitimate clients is more predictable than those of malicious participants performing untargeted model poisoning attacks.
\\
\item[{\em(ii)}] 
We introduce FLANDERS, the first pre-aggregation filter for FL robust to untargeted model poisoning based on multidimensional time series anomaly detection.
\\
\item[{\em(iii)}] 
We integrate FLANDERS into Flower,\footnote{\scriptsize{\url{https://flower.dev/}}} a popular FL simulation framework for reproducibility.
\\
\item[{\em(iv)}] 
We show that FLANDERS improves the robustness of the existing aggregation methods under multiple settings: different datasets, client's data distribution (non-iid), models, and attack scenarios.
\\
\item[{\em(v)}] 
We publicly release all the implementation code of FLANDERS along with our experiments.\footnote{\scriptsize{\url{https://anonymous.4open.science/r/flanders_exp-7EEB}}}
\end{itemize}

% Paper's structure and organization
The remainder of the paper is structured as follows. %some related work and the current state-of-the-art solutions to security issues that FL entails. 
Section~\ref{sec:background} covers background and preliminaries. 
In Section~\ref{sec:related}, we discuss related work.
Section~\ref{sec:problem} and Section~\ref{sec:method} describe the problem formulation and the method proposed. % to tackle it. 
Section~\ref{sec:experiments} gathers experimental results. %, and Section~\ref{sec:limitations} discusses some limitations of this work.
Finally, we conclude in Section~\ref{sec:conclusion}.
 %discusses the limitations of this work and draws future research directions.
%reports conclusions and draws perspectives for future research directions.

%%%%%%% OLD %%%%%%%
%to overcome the resilience of Byzantine failures in distributed Stochastic Gradient Descent computations. 
% The strength of Krum is its time complexity, which is linear in the gradient dimension. 
% However, the robustness of the approach is guaranteed for gradient-based learning applications only when the majority of the clients are not compromised. 
% Besides, the aggregation mechanism of Krum, as well as that of similar methods, is robust from a coarse-grained perspective and does not provide solutions to errors and perturbations that may occur at inference time.
%A related approach to~\cite{blanchard2017nips} is the work of Su et al.~\cite{su2016dc}. Here, the authors propose an iterated approximate agreement to tackle a multi-layer scenario attacked by Byzantine agents. 
%However, the method works efficiently on the sole discrete context and it is inapplicable to continuous state environments.
%\gabri{Maybe, we should just talk about the main limitations of existing countermeasures without digging into their details (or, we can just mention Krum as this is the most popular one). I will move the description of all these methods to the Related Work section.}
\section{Notation and Preliminaries}\label{sec_prel}
Let $\mathbb{Z}_{>0}$ denote the set of positive integers and let $\mathbb{Z}_{[a,b]}$ denote the set of integers in the interval $[a,b]$. The $m\times m$ identity matrix is denoted by $I_m$ and its columns by $e_i$ for $i\in\mathbb{Z}_{[1,m]}$. We use $\mathbf{0}$ to denote a vector or a matrix of zeros of appropriate dimensions. For a sequence $\{z_k\}_{k=0}^{N-1}$ with $z_k\in\mathbb{R}^\eta$, we denote its stacked vector as $z = \begin{bmatrix}z_0^\top &z_1^\top & \dots & z_{N-1}^\top\end{bmatrix}^\top$ and a stacked window of it as $z_{[l,j]} = \begin{bmatrix}z_l^\top &z_{l+1}^\top & \dots & z_{j}^\top\end{bmatrix}^\top$ with $0\leq l<j$.\par
Persistence of excitation of a sequence and its extension to multiple sequences \cite{vanWaarde20} are defined as follows.
\begin{definition} The sequence \(\{z_k\}_{k=0}^{N-1}\), $z_k\in\mathbb{R}^{\eta}$, is said to be persistently exciting of order \(L\) if \(\textup{rank}(\mathscr{H}_{L}(z))=\eta L\), where $\mathscr{H}_L(z) = \begin{bmatrix}
		z_{[0,L-1]} & z_{[1,L]} & \cdots & z_{[N-L,N-1]}
	\end{bmatrix}$.
	\label{def_PE}
\end{definition}
\begin{definition}[\cite{vanWaarde20}]\label{def_cPE}
	The sequences $\{z_k^{(j)}\}_{k=0}^{N_j-1}$, with $z_k^{(j)}\in\mathbb{R}^\eta$ and $j\in\mathbb{Z}_{[1,r]}$, are said to be \textit{collectively persistently exciting} of order $L$ if rank$(\mathcal{H}_L(\mathscr{Z}))=\eta L$, where $\mathscr{Z} = \begin{bmatrix}
		(z^{(1)})^\top & \cdots & (z^{(r)})^\top
	\end{bmatrix}^\top,$ and
	\begin{equation*}
		\mathcal{H}_L(\mathscr{Z}) = \begin{bmatrix}
			\mathscr{H}_L(z^{(1)}) & \cdots & \mathscr{H}_L(z^{(r)})
		\end{bmatrix}.
	\end{equation*}
\end{definition}
\section{Why-Provenance for Datalog Queries}\label{sec:why-provenance}
%


As already discussed in the Introduction, why-provenance is a standard way of explaining why a query result is obtained. It essentially collects all the subsets of the database (without unnecessary atoms) that allow us to prove (or derive) a query result. We proceed to formalize this simple idea, and then introduce the main problem of interest.

Given a proof tree $T = (V,E,\lambda)$ (of some fact w.r.t.~some database and Datalog program), the {\em support} of $T$ is the set
\[
\support{T}\ =\ \left\{\lambda(v) \mid v \in V \text{ is a leaf of } T\right\},
\]
which is essentially the set of facts that label the leaves of the proof tree $T$. Note that $\support{T}$ is a subset of the underlying database since, by definition, the leaves of a proof tree are labeled with database atoms.
%
The formal definition of why-provenance for Datalog queries follows.


\begin{definition}[\textbf{Why-Provenance for Datalog}]\label{def:why-provenance}
	Consider a Datalog query  $Q = (\dep,R)$, a database $D$ over $\esch{\dep}$, and a tuple $\bar t \in \adom{D}^{\arity{R}}$. The {\em why-provenance of $\bar t$ w.r.t.~$D$ and $Q$} is defined as the family of sets of facts 
	\[
	\{\support{T} \mid T \text{ is a proof tree of } R(\bar t) \text{ w.r.t. } D \text{ and } \dep\}
	\]
	which we denote by $\why{\bar t}{D}{Q}$. \hfill\markfull
\end{definition}

\begin{comment}
\begin{definition}[\textbf{Why-Provenance}]\label{def:why-provenance}
	Consider a Datalog program $\dep$, a database $D$ over $\esch{\dep}$, and a fact $\alpha$ over $\sch{\dep}$. The {\em why-provenance of $\alpha$ w.r.t.~$D$ and $\dep$} is the set 
	\[
	\{\lfacts{T} \mid T \text{ is a proof tree of } \alpha \text{ w.r.t. } D \text{ and } \dep\}
	\]
	denoted as $\why{\alpha}{D}{\dep}$. \hfill\markfull
\end{definition}
\end{comment}
 
Intuitively speaking, a set of facts $D' \subseteq D$ that belongs to $\why{\bar t}{D}{Q}$ should be understood as a ``real'' reason why the tuple $\bar t$ is an answer to the query $Q$ over the database $D$, i.e., $D'$ explains why $\bar t \in Q(D)$. By ``real'' we mean that all the facts of $D'$ are really used in order to derive the tuple $\bar t$ as an answer. Here is a simple example of why-provenance.

\begin{example}\label{exa:why-provenance}
	Let $Q = (\dep,A)$, where $\dep$ is the program that encodes the path accessibility problem as in Example~\ref{exa:proof-tree}, and let $D$ be the database from Example~\ref{exa:proof-tree}. 
	%
	It can be verified that the why-provenance of the unary tuple $(d)$ w.r.t.~$D$ and $Q$ consists of $\{S(a), T(a,a,d)\}$ and the database $D$ itself. The former set is actually the support of the first proof tree given in Example~\ref{exa:proof-tree}, while $D$ is the support of the second proof tree. 
	%
	Recall that $A(d)$ has infinitely many proof trees w.r.t.~$D$ and $\dep$, whereas $\why{(d)}{D}{Q}$ contains only two sets. Thus, in general, there is no 1-1 correspondence between proof trees of a fact $R(\bar t)$ and members of the why-provenance of $\bar t$. \hfill\markfull
\end{example}


We would like to pinpoint the inherent complexity of the problem of computing the why-provenance of a tuple w.r.t.~a database and a Datalog query.
%
To this end, we need to study the complexity of  recognizing whether a certain subset of the database belongs to the why-provenance, that is, whether a candidate explanation is indeed an explanation.
%forms such an explanation. 
This leads to the following algorithmic problem parameterized by a class $\class{C}$ of Datalog queries; $\class{C}$ can be, e.g., $\DAT$, $\LDAT$, or $\NRDAT$:

\smallskip

\begin{center}
	\fbox{\begin{tabular}{ll}
			{\small PROBLEM} : & $\mathsf{Why\text {-}Provenance[C]}$
			\\[2mm]
			{\small INPUT} : & A Datalog query $Q = (\dep,R)$ from $\class{C}$,\\
			& a database $D$ over $\esch{\dep}$,\\
			& a tuple ${\bar t} \in \adom{D}^{\arity{R}}$, and $D' \subseteq D$.\\[2mm]
			{\small QUESTION} : &  Does $D' \in \why{\bar t}{D}{Q}$?
	\end{tabular}}
\end{center}


\smallskip

Our goal is to study the above problem and pinpoint its complexity. We are actually interested in the {\em data complexity} of $\mathsf{Why\text {-}Provenance[C]}$, where the query $Q$ is fixed, and only the database $D$, the tuple $\bar t$, and $D'$ are part of the input, i.e., for each $Q = (\dep,R)$ from $\class{C}$, we consider the problem:

\smallskip

\begin{center}
	\fbox{\begin{tabular}{ll}
			{\small PROBLEM} : & $\mathsf{Why\text {-}Provenance}[Q]$
			\\[2mm]
			{\small INPUT} : & A database $D$ over $\esch{\dep}$,\\
			& a tuple ${\bar t} \in \adom{D}^{\arity{R}}$, and $D' \subseteq D$.\\[2mm]
			{\small QUESTION} : &  Does $D' \in \why{\bar t}{D}{Q}$?
	\end{tabular}}
\end{center}


\smallskip

%dubbed $\mathsf{Why\text {-}Provenance[Q]}$, which takes as input a database $D$ over $\esch{\dep}$, a tuple ${\bar t} \in \adom{D}^{\arity{R}}$, and $D' \subseteq D$, and asks whether $D' \in \why{\bar t}{D}{Q}$.
%
\noindent By the typical convention, the problem $\mathsf{Why\text {-}Provenance[C]}$ is in a certain complexity class $C$ in data complexity if, for every query $Q$ from $\class{C}$, $\mathsf{Why\text {-}Provenance}[Q]$ is in $C$.
%
On the other hand, $\mathsf{Why\text {-}Provenance[C]}$ is hard for a certain complexity class $C$ in data complexity if there exists a query $Q$ from $\class{C}$ such that $\mathsf{Why\text {-}Provenance}[Q]$ is hard for $C$.


%This is a practically relevant setting as the really large object is the database, while the query is typically small.

\section{Complexity Analysis}
\label{sec:complexity_analysis}

{\bf Size bounds.} For a join query $Q$, its hypergraph $H(Q)$ has one node per variable in $Q$ and one hyperedge per relation in $Q$.  Figures~\ref{fig:example_intro_varorder} depicts a query hypergraph.

An edge cover is a subset of (hyper)edges of $H(Q)$ such that each node appears in at least one edge. Edge cover can be formulated as an integer programming problem by assigning to each edge $R_i$ a weight $w_{R_i}$ that can be $1$ if $R_i$ is part of the cover and $0$ otherwise. The size of an edge cover upper bounds the size of the query result, since the Cartesian product of the relations in the cover includes the
query result: $|Q(\db)| \leq |R_1|^{w_{R_1}}\cdot\ldots\cdot|R_n|^{w_{R_n}}$, where the database $\db$ is $(R_1,\ldots,R_n)$. By minimizing the size of the edge cover, we can obtain a lower upper bound on the size of the query result. This bound becomes tight for fractional weights~\cite{AGM:2013}.  Minimizing the sum of the weights thus becomes the objective of a linear program.

\begin{definition}[\cite{AGM:2013}]\label{def:agm}
Given a join query $Q$ over a database $(R_1,\ldots,R_n)$, the {\em fractional edge cover number} $\rho^*(Q)$ is the cost of an optimal solution to the linear program with variables $(w_{R_i})_{i\in[n]}$ representing weights of $(R_i)_{i\in[n]}$:
\begin{flalign*}
\textrm{minimize} &\prod_{i\in[n]} |R_i|^{w_{R_i}}\\
\textrm{subject to} &\sum_{R\textrm{ is relation of } X} w_R \geq 1~~\textrm{for each variable } X \\
&~~~\forall i\in[n]: \omega_{R_i}\geq 0.
\end{flalign*}
\end{definition}

\begin{example}
\em
Consider the triangle query:
\begin{align*}
Q_{\vartriangle} = R(A,B), S(B,C), T(C,A)
\end{align*}
Figure~\ref{fig:triangle_hypergraph_viewtree} gives the hypergraph of $Q_{\vartriangle}$. The linear program is: 
\begin{flalign*}
\textrm{\em minimize} & \quad |R|^{w_{R}} \cdot |S|^{w_{S}} \cdot |T|^{w_{T}} \\ 
\textrm{\em subject to} & \quad
\begin{tabular}[t]{@{}c@{\hspace*{.5em}}c@{\hspace*{.25em}}c@{\hspace*{.25em}}c@{\hspace*{.25em}}c@{\hspace*{.25em}}c@{\hspace*{.25em}}c}
$A:$ & $w_{R}$ & & & $+$& ${w_{T}}$& $\geq 1$ \\
$B:$ & $w_{R}$ & $+$ & ${w_{S}}$ & & & $\geq 1$ \\
$C:$ & & & $w_{S}$ & $+$ & ${w_{T}}$ & $\geq 1$
\end{tabular}
\end{flalign*}
% 
For $|R|=|S|=|T|=N$, setting $w_{R}=w_{S}=w_{T}=1/2$ gives the optimal solution $\rho^*(Q_{\vartriangle}) = N^{3/2}$. Consequently, the query result has $\bigO{N^{3/2}}$ tuples. This bound is tight in the sense that there exist classes of databases for which the result size is at least $\Omega(N^{3/2})$. For the acyclic query $Q$ in Section~\ref{sec:introduction}, setting the weights $1$ to each of the three relations gives $\rho^*(Q)=N^3$ if all relations have size $N$.
\punto
\end{example}

\nop{
Cardinality constraints can be used to lower the size bounds of query results. For instance, if the number of distinct $A$-values in $R(A,B)$ is $k \ll N$, then we can refine  $Q_\vartriangle$ as $R(A,B),S(B,C),T(C,A),U(A)$ with the new size bound $\rho^*(Q_\vartriangle) = N \cdot k$, where $w_{S}=1$ and $w_{U}=1$.

Join selectivities can also be incorporated to obtain a size {\em estimate} (in contrast to an upper bound). For instance, assume the selectivity of the join on $A$ between $R$ and $T$ is very low: $sel(A) = \frac{|R(A,B),T(C,A)|}{|R|\cdot|T|} = \frac{k}{N}$. Then, we consider a relation $U(A,B,C)=R(A,B),T(C,A)$ whose size estimate is $k \cdot N$ and use this as a cardinality constraint to obtain an estimate of $k \cdot N$ for $Q_\vartriangle$'s size since the size of the join of $S$ and $U$ cannot exceed the size of $U$.
}

Similarly to $\rho^*(Q)$, the {\em factorization width} $\fw(Q)$ governs the sizes of the factorized results of a join query $Q$~\cite{Olteanu:FactBounds:2015:TODS}. In a factorized join over a variable order $\omega$, the values of a variable $X$  depend on the tuples of values of its $\mathit{key}(X)$ variables and are independent of the values for other variables. A tight bound on this number is then given by the size of a join query that covers the variables in $\mathit{key}(X)\cup\{X\}$. We denote this restriction of $Q$ by $Q_{\mathit{key}(X)\cup\{X\}}$. An upper bound on the size of the factorization is then given by the maximum over all variables in $\omega$ of their number of values. This can be improved by going over all possible variable orders of $Q$ and taking the minimum upper bound. This is the factorization width of the query.

\begin{definition}\label{def:fw}
Given a join query $Q$, the {\em factorization width} of $Q$ is  $\fw(Q) = \min_{\omega\in\Omega(Q)} \max_{v\in\mathit{vars}(Q)} \rho^*(Q_{\mathit{key}(X)\cup\{X\}})$.
\end{definition}

\begin{example}\em
For acyclic queries $Q$ over relations $R_1,\ldots,R_n$, $\fw(Q)=\max_{i\in[n]}(|R_i|)$, while $\rho^*(Q)$ can be as much as $\prod_{i\in[n]}|R_i|$ as in our running example. Here are examples of restrictions of our natural join $Q$ in Section~\ref{sec:intro_example}: $\mathit{key}(D)\cup\{D\}=\{C,D\}$ is covered by the query restriction $Q_{\{C,D\}}$ that is the relation $T$; $\mathit{key}(C)\cup\{C\}=\{A,C\}$ is covered by the query restriction $Q_{\{A,C\}}$ that is the relation $S$. For the triangle query $Q_\vartriangle$ and variable order $A-B-C$: $\mathit{key}(C)\cup\{C\}=\{A,B,C\}$ is covered by $Q_\vartriangle$, while $\mathit{key}(B)\cup\{B\}=\{A,B\}$ is covered by relation $R$.
\punto
\end{example}

For any join query $Q$, its factorization width is the fractional hypertree width~\cite{Olteanu:FactBounds:2015:TODS}, a parameter that captures tracta\-bility for a host of computational problems~\cite{FAQ:PODS:2016}.

\begin{proposition}
\label{prop:factorization}
Given a join query $Q$, for every database $\db$, the result $Q(\db)$ admits:
\begin{itemize}
\item a flat representation of size $\bigO{\rho^*(Q)}$~{\em\cite{AGM:2013}};
\item a factorized representation of size $\bigO{\fw(Q)}$~{\em\cite{Olteanu:FactBounds:2015:TODS}}.
\end{itemize}

There are classes of databases $\db$ for which the above size bounds are tight. The flat and factorized representations of $Q(\db)$ can be computed worst-case optimally{\em~\cite{Ngo:SIGREC:2013,Olteanu:FactBounds:2015:TODS}}.
\end{proposition}


\subsection{Dynamic Factorization Width}
\label{sec:dynamic_width}

\milos{Doesn't consider other rings (only LR), indicator projections, and factorizable updates}

As in the non-incremental case, different variable orders may lead to wildly different performance of our IVM approach. In this section, we settle the question of which variable orders can best support IVM under updates to a given set of relations and thereby pinpoint the complexity of maintaining query results under updates. This is captured by a novel notion called {\em dynamic factorization width}, which is a refinement of the factorization width.

We first recall the complexities in the non-incremental case. There, we only materialize the root view of a view tree over a variable order with the smallest factorization width, and we thus have the time data complexity $\bigO{\fw(Q)}$ for computing factorized joins~\cite{Olteanu:FactBounds:2015:TODS} and aggregates over them~\cite{BKOZ:PVLDB:2013,FAQ:PODS:2016}; for cofactor matrices over factorized joins, there is an additional $\bigO{m^2}$ factor, since the sizes of these matrices can be quadratic in the number $m$ of variables (features)~\cite{SOC:SIGMOD:2016}. The space complexity is $\bigO{1}$ or $\bigO{m^2}$ to store the aggregate or cofactor matrix in addition to the database (modulo logarithmic factors in the data size for data iterators).

We next discuss the IVM case. Let $Q$ be any join query. For any variable order $\omega \in \Omega(Q)$, let $\tau(\omega)$ be the view tree inferred from $\omega$. This view tree has exactly one leaf for each relation symbol in $Q$.

We consider updates to relations whose relation symbols in $Q$ form a set ${\mathcal{U}}$; a relation may have several relation symbols  if it is involved in self-joins in $Q$, in which case all of them are in ${\mathcal{U}}$. For a relation symbol $R\in{\mathcal{U}}$, let $\Upsilon_{\tau(\omega)}(R)$ be the set of views that are ancestors of the leaf $R$ in $\tau(\omega)$, i.e., it consists of all the views (recursively) defined using $R$. 

The time needed to compute the delta for a view $\VIEW[keys]{V^{@X}_{rels}}$ is upper bounded by that of a join query $Q^{\sf rels}_{\sf keys \cup \{X\}-\sigma(R)}$ over relations in {\sf rels} that cover $X$ and the variables in {\sf keys} but excluding the variables in $R$. The reason for the exclusion is that a single-tuple update to $R$ binds the variables in $R$ to constants. The overall time to compute the deltas of all views in $\Upsilon_{\tau(\omega)}(R)$ is then
\begin{align*}
T(\omega,R) = \sum_{\VIEW[keys]{V^{@X}_{rels}}\in\Upsilon_{\tau(\omega)}(R)} \rho^*(Q^{\sf rels}_{{\sf keys\cup\{x\}}-\sigma(R)}).
\end{align*}

We are now ready to define the dynamic factorization width that captures the time complexity of incremental maintenance of $Q$ under updates to relations in ${\mathcal{U}}$.

\begin{definition}
Given a join query $Q$ and a set of relation symbols ${\mathcal{U}}$ in $Q$. Then, the {\em dynamic factorization width} of $Q$ and ${\mathcal{U}}$ is $\dfw(Q,{\mathcal{U}}) = \min_{\omega\in\Omega(Q)}\max_{R\in\mathcal{U}} T(\omega,R).$
\end{definition}

\begin{theorem}
Given a query $Q$ with $m$ variables, database $\db$, a payload ring $\RING$, and a set of relations ${\mathcal{U}}$ in $\db$. The time complexity of incrementally maintaining the result of $Q$ over the ring $\RING$ under single-tuple updates to relations in ${\mathcal{U}}$ is $\bigO{\dfw(Q,{\mathcal{U}})\cdot T_\RING}$, where $T_\RING$ is $\bigO{1}$ for rings of numbers and $\bigO{m^2}$ for the degree-$1$ matrix ring.
\end{theorem}

\begin{example}\label{ex:time-complexity}
\em
For our query $Q$ in Section~\ref{sec:intro_example} and database $\db$, the (static) factorization width is $\fw(Q)=O(|R|+|S|+|T|)$. Under single-tuple updates to relations in a set ${\mathcal{U}}_1\subseteq\{R,S\}$, the dynamic factorization width is $\dfw(Q,{\mathcal{U}}_1)=1$ since there are no free variables of the views over $R$ or $S$ in the variable order in Figure~\ref{fig:example_intro_varorder}. This means that we can maintain the result of a sum aggregate over $Q$ in $\bigO{1}$ time under ${\mathcal{U}}_1$ updates. The same holds for ${\mathcal{U}}_2\subseteq\{S,T\}$, i.e., $\dfw(Q,{\mathcal{U}}_2)=1$, as supported by the variable order $C-\{ D, A - \{ B, E \}\}$. However, $\dfw(Q,{\mathcal{U}}_3)=\bigO{|\db|}$ for ${\mathcal{U}}_3=\{R,S,T\}$ since there is no variable order without free variables above all three relations and some variable orders have one free variable above at least one of the three relations. Under the variable order in Figure~\ref{fig:example_intro_varorder}, $\dfw(Q,{\mathcal{U}}_3)=\min(|R|,|S|)$.

The triangle query $Q_\vartriangle$ has the (static) factorization width $\fw(Q_\vartriangle)= \rho^*(Q_\vartriangle)$. For any relation $U \in \{ R, S, T \}$, the dynamic factorization width is $\dfw(Q,\{ U \})=1$ as supported by a path variable order that has the variables in $U$ as prefix. We can thus maintain an aggregate over the triangle query in $\bigO{1}$ under single-tuple updates to exactly one of its three relations. For updates to at least two relations ${\mathcal{U}}_4$, $\dfw(Q,{\mathcal{U}}_4)=O(|\db|)$. For instance, assume a variable order $A-B-C$. We need to cover: no variable under updates to $R$; one of the variables $A$ or $B$ under updates to $S$ or $T$ respectively (the case for other permutations of this variable order is analog). Maintenance has thus lower time cost than recomputation.
\punto
\end{example}

We next analyze the space complexity $S(Q)$ of our approach. This is the sum of the sizes of the views in a view tree. The space needed by the keys of a view $\VIEW[keys]{V^{@X}_{rels}}$ is given by the fractional edge cover of a join query built using relation symbols {\sf rels} to cover the variables in {\sf keys}. To obtain the minimum size, we go over all variable orders of $Q$:
\begin{align*}
 S(Q) = \min_{\omega\in\Omega(Q)}\sum_{\VIEW[keys]{V^{@X}_{rels}}\in\tau(\omega)} \rho^*(Q^{\sf rels}_{\sf keys}).
\end{align*}

\begin{theorem}
Given a query $Q$ with $m$ variables, database $\db$, a payload ring $\RING$. The space complexity required by the materialization of a view tree for $Q$ over the ring $\RING$ is $\bigO{S(Q)\cdot T_\RING}$, where $T_\RING$ is $\bigO{1}$ for the sum ring and $\bigO{m^2}$ for the degree-$m$ matrix ring.
\end{theorem}

There are three differences between the formula $S(Q)$ and Definition~\ref{def:fw} of the factorization width $\fw(Q)$: (1) the use of summation vs. maximum, though the gap between them is linear in $m$ and thus independent of the database size; (2) the cover for $S(Q)$ can only use relation symbols of the view; (3) for $S(Q)$, we only need to cover $\sf keys$ and not also the variable at the view as in the case of $\fw(Q)$. The interplay of (2) and (3) can in fact make $S(Q)$ larger than $\fw(Q)$.
For acyclic queries, both complexities are linear if all relations have the same size and $S(Q)$ can be smaller than $\fw(Q)$ in case some relations are asymptotically smaller than others. 
For cyclic queries, however, $S(Q)$ can be larger than $\fw(Q)$. We show this for the triangle query $Q_\vartriangle$ and relations of the same size $N$. Under any variable order, there is a view of size $\bigO{N^2}$, whereas $\fw(Q_\vartriangle)=N^{3/2}$. For instance, for the variable order $A-B-C$, we materialize the view $\VIEW[A,B]{V^{@C}_{ST}} = \VSUM_{C} \VIEW[B,C]{S} \VPROD \VIEW[C,A]{T} \VPROD \VIEW[C]{\VLIFT_{C}}$, which may create $\bigO{N^2}$ pairs $(A,B)$ as we need both $S$ and $T$ to cover the variables $A$ and $B$. To avoid the large intermediate result, we join all three relations at the same time~\cite{Ngo:SIGREC:2013}, so as to cover $(A,B)$ using $R$. That would, however, require recomputation of this 3-way join for each update. This takes $\bigO{N}$ time since only two of the three variables are bound to constants. In contrast, our IVM approach trades off space for time: We need $\bigO{N^2}$ space but then support $\bigO{1}$ updates to one of the three relations (Example~\ref{ex:time-complexity}).

%%%%%%%%%%%%%%%%%%%%%%%%%%%%%%%%%%%%%%%%%%%%

%\input{refined-trees.tex}
\section{Unambiguous Proof Trees}\label{sec:unambiguous-trees}
%

Although non-recursive and minimal-depth proof trees form central classes that deserve our attention, there are still proof trees from those classes that can be classified as counterintuitive. More precisely, we can devise proof trees that are both non-recursive and minimal-depth, but they are ambiguous concerning the way some facts are derived.
%Here is an example that illustrates this phenomenon.

\begin{example}\label{exa:unambiguous-trees}
	Let $Q = (\dep,A)$, where $\dep$ is the Datalog program that encodes the path accessibility problem as in Example~\ref{exa:proof-tree}. Consider also the database
	\[
	D\ =\ \{S(a),S(b),T(a,a,c),T(b,b,c),T(c,c,d)\}.
	\]
	The following is a proof tree of the fact $A(d)$ w.r.t.~$D$ and $\dep$ that is both non-recursive and minimal-depth, but suffers from the ambiguity issue mentioned above:
	
	\centerline{
	\includegraphics[width=.48\textwidth]{proof-tree-3.pdf}}

	\noindent Indeed, there are two nodes labeled with the fact $A(c)$, but their subtrees differ, and thus, it is ambiguous how $A(c)$ is derived. Hence, the database $D$, which belongs to the why-provenance of $(d)$ w.r.t.~$D$ and $Q$ relative to non-recursive and minimal-depth proof trees
	%$\nrwhy{(d)}{D}{Q}$ and $\mdwhy{(d)}{D}{Q}$ 
	due to the above proof tree, might be classified as a counterintuitive explanation since it does not correspond to an intuitive derivation process where each fact is derived once due to an unambiguous  reason. \hfill\markfull
	%Observe also that in whatever way we try to convert the above proof tree into an unambiguous one, the resulting tree will have a smaller support. In fact, $\unwhy{(d)}{D}{Q}$ consists of the sets of facts $\{S(a),T(a,a,c),T(c,c,d)\}$ and $\{S(b),T(b,b,c),T(c,c,d)\}$, which is what one expects as conceptually meaningful explanations for the tuple $(d)$. \hfill\markfull
\end{example}


The above discussion leads to the novel class of unambiguous proof trees, where all occurrences of a fact in such a tree must be proved via the same derivation.


%Recall that, for two labeled rooted trees $T$ and $T'$, $T \eqtree T'$ means that $T$ and $T'$ are isomorphic (respecting also the node-labels). Recall also that, for a node $v$ in $T$, $T[v]$ is the subtree of $T$ rooted at $v$.


\begin{definition}[\textbf{Unambiguous Proof Tree}]\label{def:unambiguous-proof-tree}
	Consider a Datalog program $\dep$, a database $D$ over $\esch{\dep}$, and a fact $\alpha$ over $\sch{\dep}$. An {\em unambiguous proof tree of $\alpha$ w.r.t.~$D$ and $\dep$} is a proof tree $T = (V,E,\lambda)$ of $\alpha$ w.r.t.~$D$ and $\dep$ such that, for all $v,u \in V$, $\lambda(v) = \lambda(u)$ implies $T[v] \eqtree T[u]$. \hfill\markfull
\end{definition}


Considering again Example~\ref{exa:unambiguous-trees}, we can construct an unambiguous proof tree of $A(d)$ w.r.t.~$D$ and $\dep$ by simply replacing the subtree of the second child of $A(d)$ with the subtree of its first child (or vice versa).
%
Now, why-provenance relative to unambiguous proof trees is defined as expected: for a Datalog query $Q = (\dep,R)$, a database $D$ over $\esch{\dep}$, and a tuple $\bar t \in \adom{D}^{\arity{R}}$, the {\em why-provenance of $\bar t$ w.r.t.~$D$ and $Q$ relative to unambiguous proof trees} is the family
\begin{multline*}
\{\support{T} \mid T \text{ is an unambiguous proof tree of }\\
R(\bar t) \text{ w.r.t. } D \text{ and } \dep\}
\end{multline*}
denoted $\unwhy{\bar t}{D}{Q}$.
%
Considering again Example~\ref{exa:unambiguous-trees}, $\unwhy{(d)}{D}{Q}$ consists of $\{S(a),T(a,a,c),T(c,c,d)\}$ and $\{S(b),T(b,b,c),T(c,c,d)\}$, which is what one expects as conceptually intuitive explanations for the tuple $(d)$, unlike the whole database $D$.
%
The algorithmic problems
\[
\mathsf{Why\text {-}Provenance_{UN}[C]} \quad \text{and} \quad  \mathsf{Why\text {-}Provenance_{UN}}[Q]
\] 
are defined in the expected way. We can show that the data complexity of why-provenance remains unchanged.

\def\theunambiguouscomplexity{
	The following hold:
	\begin{enumerate}
		\item $\mathsf{Why\text {-}Provenance_{UN}[C]}$ is \NP-complete in data complexity, for each class $\class{C} \in \{\DAT,\LDAT\}$.
		\item $\mathsf{Why\text {-}Provenance_{UN}[\NRDAT]}$ is in $\ACZ$ in data compl.
	\end{enumerate}
}
\begin{theorem}\label{the:complexity-unambiguous-proof-trees}
	\theunambiguouscomplexity
\end{theorem}

%The proof of Theorem~\ref{the:complexity-unambiguous-proof-trees} mimics that of Theorem~\ref{the:complexity-non-recursive-proof-trees} for non-recursive proof trees. In fact, the \NP-hardness is inherited from the proof that $\mathsf{Why\text {-}Provenance_{NR}[\LDAT]}$ is \NP-hard as for linear Datalog programs, non-recursive and unambiguous proof trees coincide. 

For item (1), we show that $\mathsf{Why\text {-}Provenance_{UN}[\DAT]}$ is in \NP~and $\mathsf{Why\text {-}Provenance_{NR}[\LDAT]}$ is \NP-hard.
%
The latter is established via a reduction from the problem of deciding whether a directed graph has a Hamiltonian cycle.
%
The \NP~upper bound relies on a characterization of the existence of an unambiguous proof tree of a fact $\alpha$ w.r.t.~a database $D$ and a Datalog program $\dep$ with $\support{T} = D' \subseteq D$ via the existence of a so-called {\em unambiguous proof DAG} $G$ of $\alpha$ w.r.t.~$D$ and $\dep$ with $\support{G} = D'$ of polynomial size. 
%
Interestingly, unlike arbitrary
%, non-recursive, and minimal-depth 
proof trees, we can directly go from an unambiguous proof tree $T$ to a polynomially-sized unambiguous proof DAG with the same support as $T$, without applying any intermediate steps for reducing the depth or the subtree count of $T$. This is because an unambiguous proof tree has, by definition, ``small'' depth and subtree count (in fact, the subtree count is one).
%
The $\ACZ$ upper bound in item (2) is shown via FO rewritability. The target FO query is obtained as in the proof of Theorem~\ref{the:non-recursive-complexity}, but considering only unambiguous proof trees in the definition of $\cq{Q}$.


\subsection{Computing Why-Provenance via SAT Solvers}\label{sec:reduction-to-sat}
%


We proceed to discuss how off-the-shelf SAT solvers can be used to efficiently compute the why-provenance of a tuple relative to unambiguous proof trees. 
%In particular, the why-provenance of a tuple can be extracted from the satisfying truth assignments of a Boolean formula. 
We then discuss a proof-of-concept implementation and report encouraging results of a preliminary experimental evaluation. 
%
Let us stress that focusing on unambiguous proof trees was crucial towards these encouraging results as it is unclear how a SAT-based implementation can be made practical for proof trees that are not unambiguous. This is mainly because unambiguous proof trees, unlike other classes of proof trees, have always subtree count one, which is crucial for keeping the size of the Boolean formula manageable.



%; this is discussed further below.


%We proceed to discuss a data-efficient reduction of the problem $\mathsf{Why\text {-}Provenance_{UN}[DAT]}$ to $\mathsf{SAT}$, that is, the problem of deciding whether a Boolean formula is satisfiable, which opens up the possibility of employing efficient SAT solvers for explaining query answers to Datalog queries.
%
%We are also going to discuss in the next subsection a proof-of-concept implementation of this reduction, and report encouraging results of a preliminary experimental evaluation. Note that the choice of focusing on the class of unambiguous proof trees was crucial towards these promising results, as it is currently unclear how such a reduction can be made practical for arbitrary, non-recursive, or minimal-depth proof trees.


Consider a Datalog query $Q = (\dep,R)$, a database $D$ over $\esch{\dep}$, and a tuple $\bar t \in \adom{D}^{\arity{R}}$. We construct in polynomial time in $D$ a Boolean formula $\phi_{(\bar t,D,Q)}$ such that the why-provenance of $\bar t$ w.r.t.~$D$ and $Q$ relative to unambiguous proof trees can be computed from the truth assignments that make $\phi_{(\bar t,D,Q)}$ true.
This relies on the characterization mentioned above of the existence of an unambiguous proof tree of $R(\bar t)$ w.r.t.~$D$ and $\dep$ with $\support{T} = D' \subseteq D$ via the existence of an unambiguous proof DAG $G$ of $R(\bar t)$ w.r.t.~$D$ and $\dep$ with $\support{G} = D'$.
%
The formula $\phi_{(\bar t,D,Q)}$ is of the form
$\phi_{\mi{graph}} \wedge \phi_{\mi{acyclic}} \wedge \phi_{\mi{root}} \wedge \phi_{\mi{proof}}$, where $\phi_{\mi{graph}}$ verifies that a truth assignment corresponds to a syntactically correct labeled directed graph $G$, $\phi_{\mi{acyclic}}$ verifies that $G$ is acyclic, $\phi_{\mi{root}}$ verifies that $R(\bar t)$ is the unique root of $G$, and $\phi_{\mi{proof}}$ verifies that $G$ is an unambiguous proof DAG.
%
\begin{comment}
\begin{itemize}
	\item $\phi_{\mi{graph}}$ verifies that a truth assignment corresponds to a syntactically correct labeled directed graph $G$,
	
	\item $\phi_{\mi{acyclic}}$ verifies that $G$ is acyclic,
	
	\item $\phi_{\mi{root}}$ verifies that $R(\bar t)$ is the unique root of $G$, and
	
	\item $\phi_{\mi{proof}}$ verifies that $G$ is an unambiguous proof DAG.
\end{itemize}
\end{comment}

The key ingredient in the construction of $\phi_{(\bar t,D,Q)}$ is the so-called  {\em downward closure of $R(\bar t)$ w.r.t.~$D$ and $\dep$}, taken from~\cite{ElKM22}, which, intuitively speaking, is a hypergraph that encodes all possible proof DAGs of $R(\bar t)$ w.r.t.~$D$ and $\dep$. 
%
We first construct this hypergraph $H$, which can be done in polynomial time in the size of $D$, and then guided by $H$ we build the formula $\phi_{(\bar t,D,Q)}$, which essentially searches for an unambiguous proof DAG inside the hypergraph $H$.
%The formula $\phi_{(\bar t,D,Q)}$ essentially searches for a proof DAG inside this hypergraph.
%
Now, a truth assignment $\tau$ to the variables of $\phi_{(\bar t,D,Q)}$ naturally gives rise to a database denoted $\db{\tau}$. Let $\sem{\phi_{(\bar t,D,Q)}}$ be the family
\[
\left\{\db{\tau} \mid \tau \text{ is a satisfying assignment of } \phi_{(\bar t,D,Q)}\right\}.
\]
We can then show the next technical result:

\def\prowhyprovenancesat{
	Consider a Datalog query $Q = (\dep,R)$, a database $D$ over $\esch{\dep}$, and a tuple $\bar t \in \adom{D}^{\arity{R}}$. It holds that $\unwhy{\bar t}{D}{Q} = \sem{\phi_{(\bar t,D,Q)}}$.
}

\begin{proposition}\label{pro:why-provenance-sat}
\prowhyprovenancesat
\end{proposition}

The above proposition provides a way for computing the why-provenance of a tuple relative to unambiguous proof trees via off-the-shelf SAT solvers.
%: construct the Boolean formula, use a SAT solver to compute its satisfying assignments, and convert those assignments into databases. 
But how does this machinery behave when applied in a practical context? In particular, we are interested in the incremental computation of the why-provenance by enumerating its members instead of computing the whole set at once. The rest of the section is devoted to providing a preliminary answer to this question.


\subsection{Some Implementation Details}
%

Before presenting our experimental results, let us first briefly discuss some interesting aspects of the implementation. In what follows, fix a Datalog query $Q = (\dep,R)$, a database $D$ over $\esch{\dep}$, and a tuple $\bar t \in \adom{D}^{\arity{R}}$.

\medskip

\noindent \textbf{Constructing the Downward Closure.} Recall that the construction of $\phi_{(\bar t,D,Q)}$ relies on the downward closure of $R(\bar t)$ w.r.t.~$D$ and $\dep$. It turns out that the hyperedges of the downward closure can be computed by executing a slightly modified Datalog query $Q_{\downarrow}$ over a slightly modified database $D_{\downarrow}$. In other words, the answers to $Q_{\downarrow}$ over $D_{\downarrow}$ coincide with the hyperedges of the downward closure. Hence, to construct the downward closure we exploit a state-of-the-art Datalog engine, that is, version 2.1.1 of DLV~\cite{AACC+18}.
%
Note that our approach based on evaluating a Datalog query differs form the one in~\cite{ElKM22}, which uses an extension of Datalog with set terms.
%The latter can be simulated by terminating existential rules, which let the authors of~\cite{ElKM22} to use VLog

%a Datalog engine differs from the one used in~\cite{ElKM22}, which relies on an engine called VLog that supports a more expressive language than Datalog, in particualr, existentiallanguages

\medskip
\noindent \textbf{Constructing the Formula.} Recall that $\phi_{(\bar t,D,Q)}$ consists of four conjuncts, where each one is responsible for a certain task. As it might be expected, the heavy task is to verify that the graph in question is acyclic (performed by the formula $\phi_{\mi{acyclic}}$).
%
Checking the acyclicity of a directed graph via a Boolean formula is a well-studied problem in the SAT literature.
%with several different encodings. 
%
For our purposes, we employ the technique of {\em vertex elimination}~\cite{RankoohR22}.
%
The advantage of this approach is that the number of Boolean variables needed for the encoding of $\phi_{\mi{acyclic}}$ is of the order $O(n \cdot \delta)$, where $n$ is the number of nodes of the graph, and $\delta$ is the so-called \emph{elimination width} of the graph, which, intuitively speaking, is related to how connected the graph is.


\medskip
\noindent \textbf{Incrementally Constructing the Why-Provenance.} Recall that we are interested in the incremental computation of the why-provenance, which is more useful in practice than computing the whole set at once. To this end, we need a way to enumerate all the members of the why-provenance without repetitions. This is achieved by adapting a standard technique from the SAT literature for enumerating the satisfying assignments of a Boolean formula, called {\em blocking clause}.
%
We initially collect in a set $S$ all the facts of $D$ occurring in the downward closure of $R(\bar t)$ w.r.t.~$D$ and $\dep$. Then, after asking the SAT solver for an arbitrary satisfying assignment $\tau$ of $\phi_{(\bar t,D,Q)}$, we output the database $\db{\tau}$, and then construct the ``blocking'' clause
$
\vee_{\alpha \in S} \ell_\alpha,
$
where $\ell_\alpha = \neg x_\alpha$ if $\alpha \in \db{\tau}$, and $\ell_\alpha = x_\alpha$ otherwise. We then add this clause to the formula, which expresses that no other satisfying assignment $\tau'$ should give rise to the same member of the why-provenance.
%, i.e., $\db{\tau'}=\db{\tau}$. 
This will exclude the previously computed explanations from the computation. We keep adding such blocking clauses each time we get a new member of the why-provenance until the formula is unsatisfiable.
%, in which case the whole why-provenance has been computed.

\subsection{Experimental Evaluation}
%


We now proceed to experimentally evaluate the SAT-based approach discussed above. To this end, we consider a variety of scenarios from the literature consisting of a Datalog query $Q = (\dep,R)$ and a family of databases $\mathcal{D}$ over $\esch{\dep}$.


{\footnotesize 
	\begingroup
	\setlength{\tabcolsep}{5pt} % Default value: 6pt
	\renewcommand{\arraystretch}{1.3} % Default value: 1
	\begin{table*}[t]
		\centering
		\begin{tabular}{|c||c|c|c|}
			\hline
			\textbf{Scenario} & \textbf{Databases} & \textbf{Query Type}  & \textbf{Number of Rules}\\ \hline
			\hline
			$\mathsf{TransClosure}$ & $D_\mathsf{bitcoin}$ (235K), $D_\mathsf{facebook}$ (88.2K) & linear, recursive & 2 \\ \hline
			$\mathsf{Doctors\text{-}}i$, $i \in [7]$ & $D_1$ (100K) & linear, non-recursive & 6\\ \hline
			$\mathsf{Galen}$ & $D_1$ (26.5K), $D_2$ (30.5K), $D_3$ (67K), $D_4$ (82K) & non-linear, recursive & 14\\ \hline
			$\mathsf{Andersen}$ & $D_1$ (68K), $D_2$ (340K), $D_3$ (680K), $D_4$ (3.4M), $D_5$ (6.8M) & non-linear, recursive & 4 \\ \hline
			$\mathsf{CSDA}$ & $D_\mathsf{httpd}$ (10M), $D_\mathsf{postgresql}$ (34.8M), $D_\mathsf{linux}$ (44M) & linear, recursive & 2\\ \hline
		\end{tabular}
		\caption{Experimental scenarios.}
		\label{tab:scenarios}
	\end{table*}
	\endgroup
}


\medskip
\noindent \textbf{Experimental Scenarios.} All the considered scenarios are summarized in Table~\ref{tab:scenarios}.
%, where we give the name of the scenario, the various databases with their size (number of facts), the type of the query, and the number of Datalog rules occurring in the underlying program. 
Here is brief description:


\begin{description}
	\item[$\mathsf{TransClosure}$.] This scenario computes the transitive closure of a graph and asks for connected nodes. The database $D_\mathsf{bitcoin}$ stores a portion of the Bicoin network~\cite{Weber19}, whereas
	%with nodes representing transactions and directed edges representing flow of money from one transaction to the other~\cite{Weber19}. 
	$D_\mathsf{facebook}$ stores different ``social circles'' from Facebook~\cite{McAuley12}.
	%, where nodes represent users and an edge from $u$ to $v$ denotes that $u$ has chosen $v$ to be part of her social circle~\cite{McAuley12}.
	%
	%The scenarios $\mathsf{DOCTORS\text{-}1},\ldots,\mathsf{DOCTORS\text{-}7}$, and $\mathsf{GALEN}$ from~\cite{ElhalawatiKM22}, which have been used to experimentally evaluate the computation of ondemand why-provenance for standard proof trees, and the scenarios $\mathsf{ANDERSEN}$ and $\mathsf{CSDA}$ from~\cite{FanMK22}.
	
	
	\item[$\mathsf{Doctors}$.] The scenarios $\mathsf{Doctors\text{-}}i$, for $i \in [7]$, were used in~\cite{ElKM22} and represent queries obtained from a well-known data-exchange benchmark involving existential rules (the existential variables have been replaced with fresh constants). All such scenarios share the same database with 100K facts. 
	
	
	\item[$\mathsf{Galen}$.] This scenario used in~\cite{ElKM22} implements the ELK calculus~\cite{KazakovKS14} and asks for all pairs of concepts that are related with the $\mathsf{subClassOf}$ relation. The various databases contain different portions of the Galen ontology~\cite{Galen}.
	
	\item[$\mathsf{Andersen}$.] This scenario used in~\cite{FanMK22} implements the classical Andersen ``points-to'' algorithm for determining the flow of data in procedural programs and asks for all the pairs of a pointer $p$ and a variable $v$ such that $p$ points to $v$. The databases are encodings of program statements of different length.
	
	
	\item[$\mathsf{CSDA}$.] This scenario (Context-Sensitive Dataflow Analysis) used in~\cite{FanMK22} is similar to $\mathsf{Andersen}$ but asks for null references in a program. The databases $D_\mathsf{httpd}$, $D_{\mathsf{postgresql}}$, and $D_{\mathsf{linux}}$ store the statements of the httpd web server, the PostgreSQL DBMS, and the Linux kernel, respectively. 
\end{description}


\begin{comment}
\begin{description}
\item[$\mathsf{TransClosure}$.] This scenario computes the transitive closure of a graph and asks for all pairs of connected nodes. The database $D_\mathsf{bitcoin}$ stores a portion of the Bicoin network with nodes representing transactions and directed edges representing flow of money from one transaction to the other~\cite{Weber19}. $D_\mathsf{facebook}$ stores different ``social circles'' from facebook, where nodes represent users and an edge from $u$ to $v$ denotes that $u$ has chosen $v$ to be part of her social circle~\cite{McAuley12}.
%
%The scenarios $\mathsf{DOCTORS\text{-}1},\ldots,\mathsf{DOCTORS\text{-}7}$, and $\mathsf{GALEN}$ from~\cite{ElhalawatiKM22}, which have been used to experimentally evaluate the computation of ondemand why-provenance for standard proof trees, and the scenarios $\mathsf{ANDERSEN}$ and $\mathsf{CSDA}$ from~\cite{FanMK22}.


\item[$\mathsf{Doctors}$.] The scenarios $\mathsf{Doctors\text{-}}i$, where $i \in [7]$, used in~\cite{ElKM22} represent seven queries obtained from a well-known data-exchange benchmark involving existential rules (existential variables have been replaced with fresh constants). All such scenarios share the same database with 100K facts. 


\item[$\mathsf{Galen}$.] This scenario used in~\cite{ElKM22} implements the ELK calculus~\cite{KazakovKS14} and asks for all pairs of concepts that are related with the $\mathsf{subClassOf}$ relation. The various databases contain different portions of the Galen ontology~\cite{Galen}.

\item[$\mathsf{Andersen}$.] This scenario used in~\cite{FanMK22} implements the classical Andersen ``points-to'' algorithm for determining the flow of data in procedural programs and asks for all the pairs of a pointer $p$ and a variable $v$ such that $p$ points to $v$. The databases are encodings of program statements of different length.


\item[$\mathsf{CSDA}$.] This scenario (Context-Sensitive Dataflow Analysis) used in~\cite{FanMK22} is similar to $\mathsf{Andersen}$ but asks for null references in a program. The databases $D_\mathsf{httpd}$, $D_{\mathsf{postgresql}}$, and $D_{\mathsf{linux}}$ store the statements of the httpd web server, the PostgreSQL DBMS, and the Linux kernel, respectively. 
\end{description}
\end{comment}


%Regarding $\mathsf{ANDERSEN}$, and $\mathsf{CSDA}$, let us clarify that these scenarios were meant to stress test Datalog engines, and thus are quite demanding for our hardware setup. Indeed, the authors of~\cite{FanMK22} tested these scenarios on a server with 160GB of RAM and a Xeon CPU. For this reason, we had to exclude the two largest databases from the $\mathsf{ANDERSEN}$ scenario, as for these databases, even simply collecting the answers to the $\mathsf{ANDERSEN}$'s query, which is necessary in order to obtain tuples to explain, goes out of memory on our machine.


\medskip
\noindent \textbf{Experimental Setup.} For each scenario $s$ consisting of the query $Q = (\dep,R)$ and the family of databases $\mathcal{D}$, and for each $D \in \mathcal{D}$, we have computed $Q(D)$ using DLV, and then selected five tuples $\bar t^1_{s,D},\ldots,\bar t^5_{s,D}$ from $Q(D)$ uniformly at random. 
%
Then, for each $i \in [5]$, we constructed the downward closure of $R(\bar t^i_{s,D})$ w.r.t.~$D$ and $\dep$ by first computing the adapted query $Q_{\downarrow}$ and database $D_{\downarrow}$ via a Python 3 implementation and then using DLV for the actual computation of the downward closure, then
%
we constructed the Boolean formula $\phi_{(\bar t^i_{s,D},D,Q)}$ via a C++ implementation, and finally 
%
we ran the state-of-the-art SAT solver Glucose (see, e.g.,~\cite{Audemard18}), version 4.2.1, with input the above formula to enumerate the members of $\unwhy{\bar t^i_{s,D}}{D}{Q}$.
%
%Note that the size of the why-provenance in some cases can be extremely large (in the order of millions), and thus, we executed the enumeration until either 10K explanations have been constructed, or a timeout of 5 minutes has been reached (whatever comes first). We did not set any timeout for building the downward closure and the Boolean formula.
%
All the experiments have been conducted on a laptop with an Intel(R) Core(TM) i7-10750H CPU @ 2.60GHz, and 32GB of RAM, running Fedora Linux 37. The Python code is executed with Python 3.11.2, and the C++ code has been compiled with g++ 12.2.1, using the -O3 optimization flag.


\begin{figure}[t]
	\centering
	\includegraphics[width=.464\textwidth]{ground_formula_ANDERSEN.pdf}
	\caption{Building the downward closure and the Boolean formula.}
	\label{fig:andersen-task1}
\end{figure}

\begin{figure}[t]
	\centering
	\includegraphics[width=.47\textwidth]{exptimes_ANDERSEN.pdf}
	\caption{Incremental computation of the why-provenance.}
	\label{fig:andersen-task2}
\end{figure}


\medskip
\noindent \textbf{Experimental Results.} Due to space constraints, we are going to present only the results based on the $\mathsf{Andersen}$ scenario. Nevertheless, the final outcome is aligned with what we have observed based on all the other scenarios.
%
%Recall that we are dealing with two main tasks, which we are going to discuss separately, namely (1) the construction of the downward closure and the Boolean formula, and (2) the incremental computation of the why-provenance.


Concerning the construction of the downward closure and the Boolean formula, we report in Figure~\ref{fig:andersen-task1} the total running time for each database of the $\mathsf{Andersen}$ scenario (recall that there are five databases of varying size, and thus we have five plots). Furthermore, each plot consists of five bars that correspond to the five randomly chosen tuples. Each such bar shows the time for building the downward closure plus the time for constructing the Boolean formula. 
%
%The two contributions are presented with two different colors, although in some cases the contribution of building the formula is invisible since it is too small.
%
%It is evident 
We have observed that almost all the time is spent for computing the downward closure, whereas the time for building the formula is negligible. Hence, our efforts should concentrate on improving the computation of the downward closure.
%
Moreover, for the reasonably sized databases (68K, 340K, and 680K facts) the total time is in the order of seconds, which is quite encouraging. Now, for the very large databases that we consider (3.4M and 6.8M facts), the total time is between half a minute and a minute, which is also encouraging taking into account the complexity of the query, the large size of the databases, and the limited power of our machine.
%that has been used for performing the experiments.



For the incremental computation of the why-provenance, we give in Figure~\ref{fig:andersen-task2}, for each database of the $\mathsf{Andersen}$ scenario, the times required to build an explanation, that is, the time between the current member of the why-provenance and the next one (this time is also known as the delay).
%
Each of the five plots collects the delays of constructing the members of the why-provenance (up to a limit of 10K members or 5 minutes timeout) for each of the five randomly chosen tuples. We use box plots, where the bottom and the top borders of the box represent the first and third quartile, i.e., the delay under which 25\% and 75\% of all delays occur, respectively, and the orange line represents the median delay. Moreover, the bottom and the top whisker represent the minimum and maximum delay, respectively. All times are expressed in milliseconds and we use logarithmic scale.
%
As we can see, most of the delays are below 1 millisecond, with the median in the order of microseconds. Therefore, once we have the Boolean formula in place, incrementally computing the members of the why-provenance is extremely fast.


\begin{comment}
\medskip
\noindent \textbf{Comparative Evaluation.}
We conclude this section by performing a preliminary comparison with the approach of~\cite{ElKM22}. 
%
Let us first clarify that our work deals with a different problem. For a Datalog query $Q = (\dep,R)$, a database $D$ over $\esch{\dep}$, and a tuple $\bar t \in \adom{D}^{\arity{R}}$, the approach from~\cite{ElKM22} has been designed and evaluated for building the whole set $\why{\bar t}{D}{Q}$, whereas our approach has been designed and evaluated for incrementally computing $\unwhy{\bar t}{D}{Q}$. However, there is a setting where a reasonable comparison can be performed, which will provide some insights for the two approaches.
%
This is when the Datalog query $Q$ is both linear and non-recursive in which case the sets $\why{\bar t}{D}{Q}$ and $\unwhy{\bar t}{D}{Q}$ coincide since a proof tree of $R(\bar t)$ w.r.t.~$D$ and $\dep$ is trivially unambiguous.
%
Therefore, towards a fair comparison, we are going to consider the scenarios $\mathsf{Doctors}\text{-}i$, for $i \in [7]$, which consist of a Datalog query that is linear and non-recursive, and consider the end-to-end runtime of our approach (not the delays) without, of course, setting a limit on the number of members of why-provenance to build, or on the total runtime.




%We point out that providing a reasonable comparison between our approach and the one above is tricky, since they basically solve different problems. Indeed, while in our case we might have the advantage of needing to construct much less supports, since our proof trees are more refined, in the all-trees case there is no need to check for the unambiguity of the underlying trees. For these reasons, we believe the only reasonable comparison that can be made is over scenarios for which the two notions of trees coincide. This is only the case with the DOCTORS-based scenarios, which employ queries that are both linear and non-recursive. It is easy to verify that for such queries, unmabiguous trees and arbitrary trees coincide.\footnote{In any case, even if we wanted to make a comparison over all scenarios, this would not be possible, since the authors of~\cite{ElhalawatiKM22} did not provide the tools needed to construct the set of existential rules that are able to build the why-provenance. Hence, we have to rely on the pregenerated ones for the scenarios they also consider (i.e., $\mathsf{DOCTORS}$ and $\mathsf{GALEN}$).}

\begin{figure}[t]
	\centering
	\includegraphics[width=.477\textwidth]{comparison.pdf}
	\caption{Comparative Evaluation Using the Doctors Scenarios.}
	\label{fig:comparison-time}
\end{figure}

%Since the approach of~\cite{ElhalawatiKM22} can only construct the whole why-provenance, without enumeration, we compare the end-to-end running time of our apprach (i.e., by combining tasks \emph{1)} and \emph{2)}) with the running time of their approach. In this case, of course, we did not specify any limit on the number of supports to build, or on the total running time. Finally, although not strictly necessary for enumerating the supports, our implementation keeps in main memory each support that is produced, so to be fair with the approach of~\cite{ElhalawatiKM22} which keeps all supports in memory by design. 


The comparison is shown in Figure~\ref{fig:comparison-time}. For each scenario, we present the runtime for all five randomly chosen tuples for our approach (in blue) and the approach of~\cite{ElKM22} (in red); if a bar is missing for a certain tuple, then the execution ran out of memory.
%
We observe that for the simple scenarios the two approaches are comparable in the order of a second.
%
Now, concerning the demanding scenarios, i.e., $\mathsf{Doctors}\text{-}i$ for $i \in \{1,5,7\}$, we observe that our approach is, in general, faster. Observe also that for some of the most demanding cases, the approach of~\cite{ElKM22} runs out of memory.
%
We believe that the latter is due to the use of the rule engine VLog, which is intended for materialization-based reasoning with existential rules, whereas our approach relies on a Datalog engine (in particular, DLV), and thus, exploiting all the optimizations that are typically employed for evaluating a Datalog query. For example, the technique of {\em magic-set rewriting}, implemented by DLV, can greatly reduce the memory usage by building much fewer facts during the evaluation of the rules; see, e.g.,~\cite{LAAC+19}.
\end{comment}
\section{Conclusions}
We consider the phase-extraction problem, and we showed that, given a unitary $U = e^{i\pi H}$ and its inverse $U^{\dag}$, we could implement a block-encoding of $\phi(H)$ for some smooth function $\phi(x)$. The word `smooth' here means existence and continuity of the derivatives: the higher the number of continuous derivatives that a function has, the faster its Fourier sum (and thus the Laurent polynomial on the eigenphases) uniformly converges to that function. We are confident this can have many more applications beyond what is shown in this work. It is also worth remarking that Jackson showed that the convergence rate of a Fourier series is almost-optimal, in the sense that no trigonometric (or, equivalently, complex exponential) series can approximate the desired function faster, up to that $\log d$ factor~\cite[p.\ 21]{jacksonTheoryApproximation1930a}. Also remember that `smoothing' a function, i.e., replacing its derivative with a continuous function, does not give faster convergence for free in general, as its derivative will become steep in the points where we smooth out discontinuities, and this translates to a high Lipschitz constant: a~clear example is given by Eq.~\ref{eq:lipschitz-constant-recurrence-solution}, but in that case, fortunately, nothing depends on the size of the input $N$, and thus does not influence the asymptotic query complexity of Algorithm~\ref{alg:prop-sampling-qsp}, although the constant factor can become large even for $p = 20$. From a theoretical point of view, this work shows that, for any $\eta > 0$, there is an algorithm with query complexity 
$$\Tilde{\bigO}\left(\frac{1}{\bar{c}^{\frac{1}{2} + \eta}} \frac{1}{\epsilon^\eta} \right)$$
solving the proportional-sampling problem. This statement seems to suggest there exists an algorithm which directly solves the problem with $\eta = 0$, and an open question would be to find such algorithm.


It is also interesting to remark that Theorems~\ref{thm:haah-construction},~\ref{thm:haah-completion} indeed allow the construction for any $\phi$, even complex-valued, provided that its absolute value is reciprocal.

One could think that, in Section~\ref{sec:prop-sampling}, instead of using the linear function in the phase-extraction subroutine, we could approximate the square root and then apply the transformation directly on $e^{i \pi c(x)}$. However, in the case of proportional sampling this would be inconvenient, as the derivative of the square root function has a discontinuity with an infinite jump around 0, and we could not choose a constant $\delta$ if we had values of the oracle that are too close to $0$.

\bibliographystyle{kr}
%\bibliography{references}

\begin{thebibliography}{}
	\bibitem[\protect\citeauthoryear{Abiteboul, Hull, and Vianu}{1995}]{AbHV95}
	Abiteboul, S.; Hull, R.; and Vianu, V.
	\newblock 1995.
	\newblock {\em Foundations of Databases}.
	\newblock Addison-Wesley.
	
	\bibitem[\protect\citeauthoryear{Adrian \bgroup et al\mbox.\egroup
	}{2018}]{AACC+18}
	Adrian, W.~T.; Alviano, M.; Calimeri, F.; Cuteri, B.; Dodaro, C.; Faber, W.;
	Fusc{\`{a}}, D.; Leone, N.; Manna, M.; Perri, S.; Ricca, F.; Veltri, P.; and
	Zangari, J.
	\newblock 2018.
	\newblock The {ASP} system {DLV:} advancements and applications.
	\newblock {\em K{\"{u}}nstliche Intell.} 32(2-3):177--179.
	
	\bibitem[\protect\citeauthoryear{Audemard and Simon}{2018}]{Audemard18}
	Audemard, G., and Simon, L.
	\newblock 2018.
	\newblock On the glucose {SAT} solver.
	\newblock {\em Int. J. Artif. Intell. Tools} 27(1):1840001:1--1840001:25.
	
	\bibitem[\protect\citeauthoryear{Benedikt \bgroup et al\mbox.\egroup
	}{2022}]{BBGKM22}
	Benedikt, M.; Buron, M.; Germano, S.; Kappelmann, K.; and Motik, B.
	\newblock 2022.
	\newblock Rewriting the infinite chase.
	\newblock {\em PVLDB} 15(11):3045--3057.
	
	\bibitem[\protect\citeauthoryear{Bourgaux \bgroup et al\mbox.\egroup
	}{2022}]{BBPT22}
	Bourgaux, C.; Bourhis, P.; Peterfreund, L.; and Thomazo, M.
	\newblock 2022.
	\newblock Revisiting semiring provenance for datalog.
	\newblock In {\em KR}.
	
	\bibitem[\protect\citeauthoryear{Buneman, Khanna, and Tan}{2001}]{BuKT01}
	Buneman, P.; Khanna, S.; and Tan, W.~C.
	\newblock 2001.
	\newblock Why and where: {A} characterization of data provenance.
	\newblock In {\em ICDT},  316--330.
	
	\bibitem[\protect\citeauthoryear{Cook}{1974}]{Cook74}
	Cook, S.~A.
	\newblock 1974.
	\newblock An observation on time-storage trade off.
	\newblock {\em J. Comput. Syst. Sci.} 9(3):308--316.
	
	\bibitem[\protect\citeauthoryear{Dam{\'{a}}sio, Analyti, and
		Antoniou}{2013}]{DaAA13}
	Dam{\'{a}}sio, C.~V.; Analyti, A.; and Antoniou, G.
	\newblock 2013.
	\newblock Justifications for logic programming.
	\newblock In {\em LPNMR},  530--542.
	
	\bibitem[\protect\citeauthoryear{Dantsin \bgroup et al\mbox.\egroup
	}{2001}]{DEGV01}
	Dantsin, E.; Eiter, T.; Gottlob, G.; and Voronkov, A.
	\newblock 2001.
	\newblock Complexity and expressive power of logic programming.
	\newblock {\em {ACM} Comput. Surv.} 33(3):374--425.
	
	\bibitem[\protect\citeauthoryear{Deutch \bgroup et al\mbox.\egroup
	}{2014}]{DMRT14}
	Deutch, D.; Milo, T.; Roy, S.; and Tannen, V.
	\newblock 2014.
	\newblock Circuits for datalog provenance.
	\newblock In {\em ICDT},  201--212.
	
	\bibitem[\protect\citeauthoryear{Eiter \bgroup et al\mbox.\egroup
	}{2012}]{EOSTX12}
	Eiter, T.; Ortiz, M.; Simkus, M.; Tran, T.; and Xiao, G.
	\newblock 2012.
	\newblock Query rewriting for horn-shiq plus rules.
	\newblock In {\em AAAI}.
	
	\bibitem[\protect\citeauthoryear{Elhalawati, Kr{\"{o}}tzsch, and
		Mennicke}{2022}]{ElKM22}
	Elhalawati, A.; Kr{\"{o}}tzsch, M.; and Mennicke, S.
	\newblock 2022.
	\newblock An existential rule framework for computing why-provenance on-demand
	for datalog.
	\newblock In {\em RuleML+RR}.
	
	\bibitem[\protect\citeauthoryear{Esparza, Luttenberger, and
		Schlund}{2014}]{EsLS14}
	Esparza, J.; Luttenberger, M.; and Schlund, M.
	\newblock 2014.
	\newblock Fpsolve: {A} generic solver for fixpoint equations over semirings.
	\newblock In {\em CIAA},  1--15.
	
	\bibitem[\protect\citeauthoryear{Fan, Mallireddy, and Koutris}{2022}]{FanMK22}
	Fan, Z.; Mallireddy, S.; and Koutris, P.
	\newblock 2022.
	\newblock Towards better understanding of the performance and design of datalog
	systems.
	\newblock In {\em Datalog 2.0},  166--180.
	
	\bibitem[\protect\citeauthoryear{Gebser \bgroup et al\mbox.\egroup
	}{2016}]{GKKOSW16}
	Gebser, M.; Kaminski, R.; Kaufmann, B.; Ostrowski, M.; Schaub, T.; and Wanko,
	P.
	\newblock 2016.
	\newblock Theory solving made easy with clingo 5.
	\newblock In {\em ICLP},  2:1--2:15.
	
	\bibitem[\protect\citeauthoryear{Kazakov, Kr{\"{o}}tzsch, and
		Simancik}{2014}]{KazakovKS14}
	Kazakov, Y.; Kr{\"{o}}tzsch, M.; and Simancik, F.
	\newblock 2014.
	\newblock The incredible {ELK} - from polynomial procedures to efficient
	reasoning with {EL} ontologies.
	\newblock {\em J. Autom. Reason.} 53(1):1--61.
	
	\bibitem[\protect\citeauthoryear{Lee, Lud{\"{a}}scher, and
		Glavic}{2019}]{LeLG19}
	Lee, S.; Lud{\"{a}}scher, B.; and Glavic, B.
	\newblock 2019.
	\newblock {PUG:} a framework and practical implementation for why and why-not
	provenance.
	\newblock {\em {VLDB} J.} 28(1):47--71.
	
	\bibitem[\protect\citeauthoryear{Leone \bgroup et al\mbox.\egroup
	}{2006}]{LPFEGPS06}
	Leone, N.; Pfeifer, G.; Faber, W.; Eiter, T.; Gottlob, G.; Perri, S.; and
	Scarcello, F.
	\newblock 2006.
	\newblock The {DLV} system for knowledge representation and reasoning.
	\newblock {\em {ACM} Trans. Comput. Log.} 7(3):499--562.
	
	\bibitem[\protect\citeauthoryear{Leone \bgroup et al\mbox.\egroup
	}{2019}]{LAAC+19}
	Leone, N.; Allocca, C.; Alviano, M.; Calimeri, F.; Civili, C.; Costabile, R.;
	Fiorentino, A.; Fusc{\`{a}}, D.; Germano, S.; Laboccetta, G.; Cuteri, B.;
	Manna, M.; Perri, S.; Reale, K.; Ricca, F.; Veltri, P.; and Zangari, J.
	\newblock 2019.
	\newblock Enhancing {DLV} for large-scale reasoning.
	\newblock In {\em {LPNMR}},  312--325.
	
	\bibitem[\protect\citeauthoryear{McAuley and Leskovec}{2012}]{McAuley12}
	McAuley, J., and Leskovec, J.
	\newblock 2012.
	\newblock Learning to discover social circles in ego networks.
	\newblock In {\em {NIPS}},  539–547.
	
	\bibitem[\protect\citeauthoryear{Rankooh and Rintanen}{2022}]{RankoohR22}
	Rankooh, M.~F., and Rintanen, J.
	\newblock 2022.
	\newblock Propositional encodings of acyclicity and reachability by using
	vertex elimination.
	\newblock In {\em {AAAI}},  5861--5868.
	
	\bibitem[\protect\citeauthoryear{{The Oxford Library}}{2007}]{Galen}
	{The Oxford Library}.
	\newblock 2007.
	\newblock Galen ontology.
	
	\bibitem[\protect\citeauthoryear{Vardi}{1995}]{Vardi95}
	Vardi, M.~Y.
	\newblock 1995.
	\newblock On the complexity of bounded-variable queries.
	\newblock In {\em PODS},  266--276.
	
	\bibitem[\protect\citeauthoryear{Weber \bgroup et al\mbox.\egroup
	}{2019}]{Weber19}
	Weber, M.; Domeniconi, G.; Chen, J.; Weidele, D. K.~I.; Bellei, C.; Robinson,
	T.; and Leiserson, C.~E.
	\newblock 2019.
	\newblock Anti-money laundering in bitcoin: Experimenting with graph
	convolutional networks for financial forensics.
	\newblock {\em CoRR} abs/1908.02591.
	
	\bibitem[\protect\citeauthoryear{Zhao, Subotic, and Scholz}{2020}]{ZhSS20}
	Zhao, D.; Subotic, P.; and Scholz, B.
	\newblock 2020.
	\newblock Debugging large-scale datalog: {A} scalable provenance evaluation
	strategy.
	\newblock {\em {ACM} Trans. Program. Lang. Syst.} 42(2):7:1--7:35.
\end{thebibliography}

\newpage
\appendix

\section{Data Complexity of Why-Provenance}\label{appsec:all-trees}

In this section, we provide the missing details for Section~\ref{sec:complexity}.

\subsection{Recursive Queries}
%

We proceed to give the full proof of Theorem~\ref{the:recursive-complexity}, which we recall here for convenience:

\begin{manualtheorem}{\ref{the:recursive-complexity}}
	\therecursivecomplexity
\end{manualtheorem}


To prove the above result, it suffices to show that:
\begin{itemize}
	\item $\mathsf{Why\text {-}Provenance[\DAT]}$ is in \NP~in data complexity.
	\item $\mathsf{Why\text {-}Provenance[\LDAT]}$ is \NP-hard in data complexity.
\end{itemize}



\medskip
\noindent \underline{\textbf{Upper Bound}}
\smallskip

\noindent Our main task is to prove Proposition~\ref{pro:characterization-all-trees}, which we recall below, that will allow us to devise a guess-and-check procedure that runs in polynomial time in the size of the database.

\begin{manualproposition}{\ref{pro:characterization-all-trees}}
	\procharacterizationalltrees
\end{manualproposition}

The direction $(2)$ implies $(1)$ is shown by ``unravelling'' the proof DAG $G$ of $\alpha$ w.r.t.~$D$ and $\dep$ into a proof tree $T$ of $\alpha$ w.r.t.~$D$ and $\dep$ with $\support{T} = \support{G}$. 
More precisely, we go over the nodes of $G$ starting from its root and ending at its leaves using breadth-first search. Whenever we encounter a node $v$ that has $k$ incoming edges, we create $k$ copies of its subDAG. The subDAG of a node $v$ contains $v$ itself and every node reachable from $v$, and an edge $(u_1,u_2)$ if there is an edge $(w_1,w_2)$ in $G$, where $u_1$ is a copy of $w_1$ and $u_2$ is a copy of $w_2$. Note that these copies preserve the labels of the nodes. We then replace each incoming edge of $v$ with an edge to the root of a distinct copy of its subDAG. Note that since $G$ is acyclic, the above operation on $v$ has no impact on the nodes that have been processed before $v$.
It is rather straightforward that the result is a tree $T$ with a root $v$ that has the same label as the root of $G$, and where the leaves have the same labels as the leaves of $G$ (hence, for each leaf $v$ of the tree we have that $\lambda(v)\in D$, as the same holds for the labels of the leaves of the proof DAG). Moreover, it is easy to verify that Property~(3) of Definition~\ref{def:proof-tree} holds since $G$ satisfies the equivalent property~$(3)$ of Definition~\ref{def:proof-dag} and our copies preserve the labels of the nodes. Therefore, the resulting tree is a proof tree of $\alpha$ w.r.t.~$D$ and $\Sigma$. 

Concerning the direction $(1)$ implies $(2)$, as discussed in the main body of the paper, the proof proceeds in three main steps captured by Lemmas~\ref{lem:depth-reduction},~\ref{lem:scount-reduction}, and~\ref{lem:from-trees-to-dags}, which we prove next.

\begin{manuallemma}{\ref{lem:depth-reduction}}
	\lemmadepthreduction
\end{manuallemma}
\begin{proof}
	We prove the claim for $f(|D|)=|\base{D,\dep}|\times |D|$ by induction on $n = \depth{T}$.
	
	\medskip
	\noindent \textbf{Base Case.} For any $n\le |\base{D,\dep}|\times |D|$, the claim holds trivially.
	
	\medskip
	\noindent \textbf{Inductive Step.} We assume that the claim holds for $n \in \{|\base{D,\dep}|\times |D|,\dots,p\}$, and prove that it holds for $n=p+1$. Let $T$ be a proof tree of $\alpha$ w.r.t.~$D$ and $\Sigma$ with $\depth{T} = p+1$. Since $p+1>|\base{D,\dep}|\times |D|$, there exists a path $v_1\rightarrow v_2\dots\rightarrow v_{p+2}$ of length $p+1$ in $T$ and a label $\beta$, such that $\beta$ is the label of $k>|D|$ nodes $v_{i_1},\dots,v_{i_k}$ along the path. (Note that $|\base{D,\dep}|$ is an upper bound on the number of distinct labels in $T$.) We assume, without loss of generality, that $i_1<i_2<\dots<i_k$.
	We will show that for some $v_{i_j}$ and $v_{i_r}$ with $j< r$, it holds that
	\[
	\support{T[v_{i_j}]}\ =\ \support{T[v_{i_r}]}. 
	\]
	Recall that for a node $v$, $T[v]$ is the subtree of $T$ rooted at $v$.
	
	An easy observation is that for $T_1,T_2$ such that $T_2$ is a subtree of $T_1$, it holds that $\support{T_2}\subseteq\support{T_1}$.
	Hence, for all $j<r$, we have that: 
	\[
	\support{T[v_{i_r}]}\subseteq\support{T[v_{i_j}]}.
	\]
	Now assume, towards a contradiction, that for every $j<r$,
	\[\support{T[v_{i_r}]}\subsetneq\support{T[v_{i_j}]}.\] 
	We then conclude that
	\[
	\support{T[v_{i_k}]}\subsetneq\dots\subsetneq \support{T[v_{i_1}]}.
	\]
	This means that $\support{T[v_{i_1}]}$ contains $k>|D|$ distinct facts, which in turn means that $\support{T}$ contains at least $k>|D|$ distinct facts. This is a contradiction to the fact that $\support{T}=D'$ for some $D'\subseteq D$. 
	
	
	Therefore, for some $j<r$ it holds that 
	\[
	\support{T[v_{i_r}]}=\support{T[v_{i_j}]}.
	\]
	We can now shorten the path $v_1\rightarrow v_2\dots\rightarrow v_{p+2}$ in $T$ and obtain another proof tree $T_1$ with the same support $D'$, by replacing the subtree $T[v_{i_j}]$ with the subtree $T[v_{i_r}]$. An important observation here is that $T_1$ is still a proof tree of $\alpha$ w.r.t.~$D$ and $\Sigma$. Since we do not modify the root node $v$, it still holds that $\lambda(v)=\alpha$. Moreover, the set of leaves of $T_1$ is contained in the set of leaves of $T$; hence, for every leaf $v$ of $T_1$ it holds that $\lambda(v)\in D$. Finally, since $T[v_{i_r}]$ is a subtree of $T$, it satisfies property $(3)$ of Definition~\ref{def:proof-tree} (that is, if $v$ is a node with $n\ge1$ children $u_1,\ldots,u_n$, then there is a rule $R_0(\bar x_0)\ \assign\ R_1(\bar x_1),\ldots,R_n(\bar x_n) \in \dep$ and a function $h : \bigcup_{i \in [n]} \bar x_i \ra \ins{C}$ such that $\lambda(v) = R_0(h(\bar x_0))$, and $\lambda(u_i) = R_i(h(\bar x_i))$ for each $i \in [n]$). Therefore, this property also holds for every node $v$ of $T_1$ (for the parent of the node $v_{i_j}$ that we replace with the node $v_{i_r}$ the property holds because $\lambda(v_{i_j})=\lambda(v_{i_r})$).
	
	Clearly, when applying the above procedure, we eliminate at least one path of length $p+1$ and we do not introduce any new path of length $p+1$.
	If we repeat this process for every path of length $p+1$, we will eventually obtain a proof tree $T_2$ of $\alpha$ w.r.t.~$D$ and $\dep$ with $\depth{T_2} \le p$ and $\support{T_2} = D'$. The claim follows by the inductive hypothesis.
\end{proof}

Before proving Lemma~\ref{lem:scount-reduction}, we show the following result, where we do not consider the support of the proof tree. 

\begin{lemma}\label{lem:exists-unumbiguous-tree}
	For each Datalog program $\dep$, database $D$ over $\esch{\dep}$, and fact $\alpha$ over $\sch{\dep}$, if there exists a proof tree $T$ of $\alpha$ w.r.t.~$D$ and $\dep$, then there exists also such a proof tree $T'$ with $|\quot{T'[\beta]}| =1$ for every fact $\beta$ that occurs in $T'$.
\end{lemma}
\begin{proof}
	Let $T=(V,E,\lambda)$ be a proof tree of $\alpha$ w.r.t.~$D$ and $\dep$. We construct another proof tree $T'=(V',E',\lambda')$ of $\alpha$ w.r.t.~$D$ and $\dep$ with the desired property in the following way. Let $S$ be the set that contains, for every fact $\beta$ that occurs in $T$ (i.e., it is the label of some node in $V$), one subtree $T[v]$ of $T$ with $\lambda(v)=\beta$ of smallest depth among all such subtrees; if more than one such tree exists, we choose one arbitrarily. For every $1\le i\le \depth{T}$, let $S^i$ be the set that contains all the trees of $S$ of depth exactly $i$. We will now inductively construct a set of trees that will contain a single representative tree for every fact $\beta$ that occurs in $T$. Then, we will use the representative tree of $\alpha$ as the tree $T'$.
	
	We define:
	\begin{itemize}
		\item $Z_1 = S^1$;
		\item $Z_{i+1} = \mathsf{Op}^{i+1}(Z_i) \cup Z_i$, for $i\ge 1$.
	\end{itemize}
	where $\mathsf{Op}^{i+1}(Z_i)$ contains, for every tree $T''$ in $S^{i+1}$, the tree that is obtained from it using the following procedure. Let $u_1,\dots,u_n$ be the direct children of the root of $T''$. For every child $u_j$ with $\lambda(u_j)=\beta$, we replace the subtree $T''[u_j]$ with a tree of $Z_i$ whose root is labeled with $\beta$. Intuitively, the existence of such a tree is guaranteed because $S$ contains a smallest depth subtree for each fact, and since $T''[u_j]$ is of depth at most $i$, the set $S$ has a tree of depth at most $i$ rooted with a node labeled with $\beta$.
	Formally, we prove the following properties of the sets $Z_i$:
	\begin{enumerate}
		\item For every label $\beta$, if $S^i$ contains a tree with root $v$ such that $\lambda(v)=\beta$, then $Z_i$ contains a tree with root $u$ such that $\lambda(u)=\beta$.
		\item For every label $\beta$, if $Z_i$ contains a tree with root $u$ such that $\lambda(u)=\beta$, then there is a tree $T''$ with root $w$ such that $\lambda(w)=\beta$ and $T''\in S^{k}$ for some $k\le i$.
		\item For every label $\beta$, if $Z_i$ contains a tree with root $u$ such that $\lambda(u)=\beta$, then there is precisely one such tree.
		%\item If $T''\in S^i$, $v$ it the root of $T''$, and $u_1,\dots,u_n$ are the non-leaf children of $v$ in $T''$, then for every $u_j$ with $\lambda(u_j)=\beta$, there exists a tree in $Z_{i-1}$ with root $w$ such that $\lambda(w)=\beta$. 
		\item For every tree $T''$ of $Z_i$, if $u$ is a leaf node of $T''$, then $\lambda(u)\in D$.
		\item For every tree $T''$ of $Z_i$, if $u$ is a node of $T''$ with $n\ge1$ children $u_1,\ldots,u_n$, then there exists a rule $R_0(\bar x_0)\ \assign\ R_1(\bar x_1),\ldots,R_n(\bar x_n)$ in $\dep$ and a function $h : \bigcup_{i \in n} \bar x_i \ra \ins{C}$ such that $\lambda(u) = R_0(h(\bar x_0))$ and $\lambda(u_j) = R_j(h(\bar x_j))$, for $j \in [n]$.
		\item For every tree $T''$ of $Z_i$, $|\quot{T''[\beta]}| =1$, for each fact $\beta$ that occurs in $T''$.
	\end{enumerate}
	We prove all six properties by induction on $i$. 
	
	\medskip
	\noindent \textbf{Base Case.} For $i=1$, the first two properties trivially hold as $Z_1=S^1$ by definition. Since $S$ contains a single tree for each fact $\beta$ (i.e., a tree where the root is labeled with $\beta$), so does $S^1$, and the third property also holds. 
	The fourth and fifth properties hold because every tree of $S^1$ (and so every tree of $Z_1$) is a subtree of a proof tree, and these are properties of proof trees. The last property is satisfied since a tree of $Z_1$ contains one root node $v$ and its children $u_1,\dots,u_n$, and it cannot be the case that $\lambda(v)=\lambda(u_j)$ for some $j \in [n]$ (as the leaves correspond to extensional predicates, while the root corresponds to an intentional predicate).
	
	\medskip
	\noindent \textbf{Inductive Step.} We assume that the claim holds for $i=1,\dots,p$ and prove that it holds for $i=p+1$. 
	The first property holds by construction, since the set $\mathsf{Op}^{p+1}(Z_{p})$ contains, for every tree $T''$ of $S^{p+1}$, another tree with the same root (we only modify the subtrees of its children). Moreover, if the children of the root of $T''$ are $u_1,\dots,u_n$, then for every $r\in [n]$, the subtree $T''[u_r]$ (which is also a subtree of the original $T$) is of depth at most $p$. Hence, the smallest depth subtree for the label $\lambda(u_r)$ in $T$ occurs in $S_k$ for some $1\le k\le p$. By the inductive assumption, the set $Z_k$ contains a tree with root $v$ such that $\lambda(v)=\lambda(u_r)$, and since $Z_k\subseteq Z_{p}$, this tree also appears in $Z_{p}$; hence, our construction is well-defined.
	
	
	The second property is satisfied since every tree of $Z_{p+1}$ with root $u$ such that $\lambda(u)=\beta$ either occurs in $Z_p$ or is obtained from a tree of $S_{p+1}$. In the first case, the inductive assumption implies that there is a tree $T''$ with a root $w$ and $\lambda(w)=\beta$ in $S^k$ for some $k\le p$. In the second case, the definition of $\mathsf{Op}^{p+1}(Z_{p})$ implies that there is a tree $T''$ with a root $w$ and $\lambda(w)=\beta$ in $S^{p+1}$.
	
	The third property holds because $S^{p+1}$ contains a single tree per fact, and so the same holds for $\mathsf{Op}^{p+1}(Z_{p})$. Moreover, $Z_p$ contains a single tree per fact due to the inductive assumption. We will show that it cannot be the case that there is a label $\beta$ and two trees $T_1,T_2$ such that: \textit{(1)} $\lambda(v_1)=\beta$ for the root $v_1$ of $T_1$, \textit{(2)} $\lambda(v_2)=\beta$ for the root $v_2$ of $T_2$, \textit{(3)} $T_1\in Z_p$, and \textit{(4)} $T_2\in \mathsf{Op}^{p+1}(Z_{p})$. Assume, towards a contradiction, that such two trees exist. Then, $S^{p+1}$ contains a tree $T_3$ with root $v_3$ such that $\lambda(v_3)=\beta$ (this is the tree from which $T_2$ is obtained). Moreover, the inductive assumption and property $(2)$ imply that there is a tree $T_4$ with root $v_4$ in $S^k$ for some $k\le p$ such that  $\lambda(v_4)=\beta$. We conclude that $S$ contains two trees whose root is labeled with $\beta$ -- one with depth $p+1$ and one with depth $k\le p$. This is a contradiction to the fact that $S$ only contains one smallest depth subtree of $T$ whose root is labeled with $\beta$.
	
	As for the fourth property, as aforementioned, every tree $T''$ of $Z_{p+1}$ either occurs in $Z_p$ or is obtained from a tree of $S_{p+1}$. In the first case, the claim immediately follows from the inductive assumption. In the second case, let $T''$ be a tree of $\mathsf{Op}^{p+1}(Z_{p})$. Assume that the root of $T''$ is $u$ and its children are $u_1,\dots,u_n$. Then, every leaf node of $T''$ is also a leaf node of $T''[u_j]$ for some $j\in [n]$, and since $T''[u_j]$ is a tree of $Z_p$ by construction, we have that $\lambda(u_j)\in D$ by the inductive assumption.
	
	The fifth property holds for every tree of $Z_p$ by the inductive assumption. We will show that it also holds for every tree $T''$ of $\mathsf{Op}^{p+1}(Z_{p})$.
	Each tree of $S^{p+1}$ is a subtree of $T$; hence, it satisfies the desired property (which is a property of proof trees). In particular, the property is satisfied by the root node, and since we do not modify the label of the root node or the labels of its children, the root of the obtained tree $T''$ also satisfies this property. For every child of the root, its subtree is replaced with a tree from $Z_{p}$ that satisfies the desired property by the inductive assumption, and so every node of $T''$ satisfies this property.
	
	Finally, for a tree of $Z_{p+1}$ that also occurs in $Z_p$, the last property holds from the inductive assumption. For a tree $T''$ of $Z_{p+1}$ that comes from $\mathsf{Op}^{p+1}(Z_{p})$, the last property holds for every label $\delta$ that occurs in $T''$ and is not the label of the root, due to the inductive assumption (since we replace the subtrees under the children of the root with trees from $Z_p$). Note that the label $\beta$ of the root of $T''$ cannot occur in a tree of $Z_p$ (and, in particular, as the label of one of its children). This holds since $Z_p\subseteq Z_{p+1}$ and due to property $(3)$ that we have already proved. Thus, it also holds that $|\quot{T''[\beta]}| =1$.
	This concludes our proof for the six properties.
	
	Now, by definition, $S$ contains a tree $T_1$ whose root is labeled with $\alpha$. Assume that the depth of this tree is $k$, then $T_1\in S^k$. Property $(1)$ then implies that $Z_k$ contains a tree $T_2$ whose root is labeled with $\alpha$. It is only left to show that this tree is a proof tree of $\alpha$ w.r.t.~$D$ and $\dep$ that satisfies the desired properties, and then we will define $T'=T_2$, and that will conclude our proof. The first property of proof trees (Definition~\ref{def:proof-tree}) is clearly satisfied as $\lambda(v)=\alpha$ for the root node $v$ of $T_2$. Properties $(2)$ and $(3)$ of proof trees are satisfied due to properties $(4)$ and $(5)$, respectively, of the sets $Z_i$. Hence, $T_2$ is indeed a proof tree of $\alpha$ w.r.t.~$D$ and $\dep$. Property $(6)$ of the sets $Z_i$ implies that $|\quot{T_2[\beta]}| =1$, for each fact $\beta$ that occurs in $T_2$. Therefore, we can indeed define $T'=T_2$ and obtain the desired proof tree.
\end{proof}

We now proceed to prove Lemma~\ref{lem:scount-reduction}, which we recall here:

\begin{manuallemma}{\ref{lem:scount-reduction}}
	\lemmascountreduction
\end{manuallemma}
\begin{proof}
	Given a proof tree $T$ of $\alpha$ w.r.t.~$D$ and $\dep$ with $\depth{T}\le f(|D|)$ and $\support{T}=D'$, we construct another proof tree $T'$ of $\alpha$ w.r.t.~$D$ and $\dep$ with $\scount{T'} \leq g(|D|)$ and $\support{T'}=D'$ in two steps. First, for every fact of $D'$, we select one path in $T$ from the root to a leaf labeled with this fact. Then, we ``freeze'' those paths (i.e., we do not modify them in $T'$) in order to preserve the support. However, it is not sufficient to keep only these paths, but we also need to keep the siblings of each node along these paths to obtain a valid proof tree, and, in particular, to satisfy the last property of Definition~\ref{def:proof-tree}. The second step is then to reduce, for every sibling node $v$ and for every fact $\beta$ in $T[v]$, the number of equivalence classes in $\quot{T[v][\beta]}$.
	
	Let $v$ be a node of $T$. An easy observation is that if $\lambda(v)=\beta$ for some fact $\beta$, then $T[v]$ is a proof tree of $\beta$ w.r.t.~$D$ and $\dep$. Then, Lemma~\ref{lem:exists-unumbiguous-tree} implies that there exists another proof tree $T_v$ of $\beta$ w.r.t.~$D$ and $\dep$ whose root node is labeled with $\beta$, such that $|\quot{T_v[\delta]}| =1$, for every fact $\delta$ that occurs in $T_v$. We can then replace the subtree $T[v]$ of $T$ with the tree $T_v$. We do that for each sibling node. Clearly, the tree $T'$ that we obtain via this procedure remains a proof tree of $\alpha$ w.r.t.~$D$ and $\dep$. This holds since we do not modify the label of the root; hence, Property~$(1)$ of Definition~\ref{def:proof-tree} holds. Moreover, Property~$(2)$ of Definition~\ref{def:proof-tree} holds because the set of leaves of $T'$ is a subset of the set of leaves of $T$ (due to the construction in the proof of Lemma~\ref{lem:exists-unumbiguous-tree}). Finally, Property~$(3)$ of Definition~\ref{def:proof-tree} is satisfied for every node in a subtree $T_v$ since, as aforementioned, every $T_v$ is a proof tree of some fact w.r.t.~$D$ and $\dep'$. Moreover, this property holds for every node along the frozen paths because for each such path $v_1\ra\dots\ra v_n$, if $u_1,\dots,u_m$ are the children of $v_j$, then one of these children is $v_{j+1}$ and we do not modify its label or the labels of its siblings (we only replaces the subtrees underneath them).
	It is only left to show that $T'$ satisfies the desired property.
	
	To this end, we observe that: \textit{(1)} every fact $\beta$ in $T'$ occurs polynomialy many times on the frozen paths, and \textit{(2)} there are polynomialy many such sibling nodes. Property $(1)$ holds since $\depth{T} \le f(|D|)$ for some polynomial $f$; hence, there are polynomialy many nodes on a path (at most $f(|D|)+1$). Moreover, since we freeze one path per fact of $D'$, there are $|D'|$ paths. Hence, each fact occurs at most $[f(|D|)+1]\times |D'|$ times on the frozen paths. Property $(2)$ holds because each node on a frozen path has at most $b-1$ siblings, where $b$ is the maximal number of atoms occurring in the body of some rule in $\Sigma$. Therefore, the number of sibling nodes is bounded by $[f(|D|)+1]\times |D'|\times (b-1)$. Due to our construction, for every fact $\beta$ that occurs in a subtree $T''$ of some sibling node, it hold that $|\quot{T''[\beta]}|=1$. Therefore, 
	\[|\quot{T'[\beta]}|\ \le\ [f(|D|)+1]\times |D'| \times b \]
	(one equivalence class for each node on the frozen paths, and one equivalence class for each sibling node). The claim then follows with:
	\[g(|D|)\ =\ [f(|D|)+1]\times |D'| \times b.\]
	This concludes our proof.
\end{proof}

Finally, we prove Lemma~\ref{lem:from-trees-to-dags}, which we recall here:

\begin{manuallemma}{\ref{lem:from-trees-to-dags}}
	\lemmafromtreestodags
\end{manuallemma}
\begin{proof}
	Our goal here is to construct a DAG $G=(V,E,\lambda)$ that contains, for every fact $\beta$ that occurs in $T$, and every equivalence class of $\quot{T[\beta]}$, a single DAG representing this class. To this end, we first add, for every fact $\beta$ that occurs in $T$, and every equivalence class $C_{\beta}$ of $\quot{T[\beta]}$, $k$ nodes $v^{C_{\beta}}_1,\dots,v^{C_{\beta}}_k$ to $V$, where $k$ is the maximal number of occurrences of the trees of $C_{\beta}$ under a single node of $T$. Clearly, $k\le b$, where $b$ is the maximal number of atoms occurring in the body of some rule in $\Sigma$. We then define $\lambda(v^{C_{\beta}}_j)=\beta$ for every $j\in [k]$. Note that the tree $T$ itself belongs to some equivalence class $C\in\quot{T[\alpha]}$ (since $\lambda(v)=\alpha$ for the root node $v$ of $T$ by the definition of proof trees), and this tree does not appear under any node of $T$; hence, for this equivalence class we have that $k=0$. In this case, we add a single node $v^{C}$ to $V$.
	
	Next, we add the edges $(v^{C_{\beta}}_j,v^{C'_{\beta'}}_{1}),\dots,(v^{C_{\beta}}_j,v^{C'_{\beta'}}_{p})$ to $E$ if for every tree $T'$ of $C$ with root $v$, we have that $T'[u]\in C'$ for precisely $p$ children $u$ or $v$. (Observe that this either holds for all trees of $C$ or none of them, since $C$ is an equivalence class.) The number of nodes in $V$ is bounded by $|\base{D,\dep}|\times f(|D|)\times b$ as the number of facts that occur in $T$ is bounded by $|\base{D,\dep}|$, there are at most $f(|D|)$ equivalence classes for each fact, and we have at most $b$ nodes for each combination of a fact and its equivalence class; hence, by defining 
	\[
	g(|D|)\ =\ |\base{D,\dep}|\times f(|D|)\times b,
	\]
	$|V|\le g(|D|)$. It remains to show that $G$ is a proof DAG.
	
	Let $v$ be the root of the proof tree $T$. As aforementioned, the subtree of $v$ (which is $T$ itself) belongs to some equivalence class $C$ of $\quot{T[\alpha]}$, and we have a node $v^C$ in $V$ with $\lambda(v^C)=\alpha$. Clearly, there is no other node $u\neq v$ in $T$ such that $T[u]\in C$; hence, by the definition of $E$, the node $v^C$ has no incoming edges. Contrarily, for every other equivalence class $C_\beta$ of some fact $\beta$, if $V$ contains precisely $k$ nodes $v^{C_\beta}_1,\dots,v^{C_\beta}_k$ corresponding to this class, then there exists, by definition, a node $u$ in $T$ with $\lambda(u)=\delta$ for some $\delta$ that has $k$ children $u_1,\dots,u_k$ whose subtrees all belong to $C_\beta$. In this case, $E$ contains the edges  $(v^{C_{\delta}}_1,v^{C_{\beta}}_{1}),\dots,(v^{C_{\delta}}_1,v^{C_{\beta}}_{k})$, where $C_{\delta}$ is the equivalence class of the subtree $T[u]$. We conclude that $G$ has a single node with no incoming edges, and the label of this node is $\alpha$; hence, property $(1)$ of Definition~\ref{def:proof-dag} holds.
	
	We can similarly show that every leaf $v\in V$ corresponds to a leaf $v'$ of $T$ labelled with the same fact. That is, if $v$ is a leaf of $G$, then it is of the form $v^{C_\beta}_j$, where $\beta$ is the label of some leaf of $T$ and $C_\beta$ is an equivalence class that contains trees with a single node labeled with $\beta$. Since $T$ is a proof tree, we have that $\lambda(v')\in D$ for every leaf $v'$ of $T$ and so $\lambda(v)\in D$ for every leaf $v$ of $G$, and property $(2)$ of Definition~\ref{def:proof-dag} holds. 
	
	Finally, we show that property $(3)$ of Definition~\ref{def:proof-dag} is satisfied by $G$. Let $v^{C_\beta}_j$ be a node in $V$ with $n\ge 1$ outgoing edges $(v^{C_\beta}_j,v^{C_{\delta_1}}_{j_1}),\dots,(v^{C_\beta}_j,v^{C_{\delta_n}}_{j_n})$. By the definition of $V$, there exists a node $u$ in $T$ with $\lambda(u)=\beta$ such that $T[u]\in C_\beta$. 
	By the definition of $E$, if among the equivalence classes $C_{\delta_1},\dots,C_{\delta_n}$ there are precisely $p$ occurrences of some class $C$ (corresponding to some fact $\gamma$), then for every tree of $C_\beta$ (in particular, for $T[u]$), the root node of the tree (in particular, the node $u$) has precisely $p$ children $u_1,\dots,u_p$ such that $T[u_\ell]\in C$ for every $\ell\in[p]$. This also means that $\lambda(u_\ell)=\gamma$ for every $\ell\in[p]$ by the definition of $V$. We conclude that there is a node $u$ of $T$ with children $u_1,\dots,u_n$ such that $\lambda(u)=\beta$ and $\lambda(u_i)=\delta_i$ for every $i\in[n]$. 
	Since $T$ is a proof tree, this means that there exists a rule $R_0(\bar x_0)\ \assign\ R_1(\bar x_1),\ldots,R_n(\bar x_n) \in \dep$ and a function $h : \bigcup_{i \in [n]} \bar x_i \ra \ins{C}$ such that $\lambda(v) = R_0(h(\bar x_0))$, and $\lambda(u_i) = R_i(h(\bar x_i))$ for $i \in [n]$.
	Therefore, we also have that $\lambda(v^{C_\beta}_j)=R_0(h(\bar x_0))$ and $\lambda(v^{C_{\delta_i}}_{j_i})=R_i(h(\bar x_i))$ for every $i\in [n]$, and property $(3)$ indeed holds. We conclude that $G$ is a proof DAG of $\alpha$ w.r.t.~$D$ and $\dep$ with $|V|\le g(|D|)$.
\end{proof}

\medskip
\noindent \textbf{Finalize the Proof.} With the above technical lemmas in place, it is now easy to show the NP upper bound. Fix a Datalog query $Q = (\dep,R)$. Given a database $D$ over $\esch{\dep}$, a tuple $\bar t \in \adom{D}^{\arity{R}}$, and a subset $D'$ of $D$, to decide whether $D' \in \why{\bar t}{D}{Q}$ we simply need to check for the existence of a proof tree $T$ of $R(\bar t)$ w.r.t.~$D$ and $\dep$ such that $\support{T} = D'$. By Proposition~\ref{pro:characterization-all-trees}, this is tantamount to the existence of a compact proof DAG $G$ of $R(\bar t)$ w.r.t.~$D$ and $\dep$ with $\support{G} = D'$.
%
It is clear that the existence of such a proof DAG can be checked by simply guessing a polynomially-sized (w.r.t.~$|D|$) labeled directed graph $G = (V,E,
\lambda)$, and then checking whether $G$ is acyclic, rooted, and a proof DAG of $R(\bar t)$ w.r.t.~$D$ and $\dep$ with $\support{G} = D'$. Since both steps can be carried out in polynomial time, $\mathsf{Why\text {-}Provenance}[Q]$ is in \NP, and thus, $\mathsf{Why\text {-}Provenance[\DAT]}$ is in \NP~in data complexity.


%%%%%%%%%%%%%%%%%%%%%%%%%%%%%%%%%%%%%%

\medskip

\noindent \underline{\textbf{Lower Bound}}
\smallskip

\noindent We proceed to establish that $\mathsf{Why\text {-}Provenance[\LDAT]}$ is \NP-hard in data complexity. To this ends, we need to show that there exists a linear Datalog query $Q$ such that the problem $\mathsf{Why\text {-}Provenance}[Q]$ is \NP-hard. The proof is via a reduction from $\mathsf{3SAT}$, which takes as input a Boolean formula $\varphi = C_1 \wedge \ldots \wedge C_m$ in 3CNF, where each clause has exactly 3 literals (a Boolean variable $v$ or its negation $\neg v$), and asks whether $\varphi$ is satisfiable.


\medskip
\noindent \textbf{The Linear Datalog Query.}
We start by defining the linear Datalog query $Q = (\dep,R)$. If the name of a variable is not important, then we use $\_$ for a fresh variable occurring only once in $\dep$. By abuse of notation, we use semicolons instead of commas in a tuple expression in order to separate terms with a different semantic meaning. The program $\dep$ follows:
\begin{eqnarray*}
\sigma_1 &:& R(x)\,\, \assign\,\, {\rm Var}(x;z,\_),{\rm Assign}(x,z), \\
\sigma_2 &:& R(x)\,\, \assign\,\, {\rm Var}(x;\_,z), {\rm Assign}(x,z),\\
%\\
\sigma_3 &:& {\rm Assign}(x,y)\,\, \assign\,\, C(x,y;\_,\_;\_,\_),{\rm Assign}(x,y), \\
\sigma_4 &:& {\rm Assign}(x,y)\,\, \assign\,\, C(\_,\_;x,y;\_,\_),{\rm Assign}(x,y), \\
\sigma_5 &:& {\rm Assign}(x,y)\,\, \assign\,\, C(\_,\_;\_,\_;x,y),{\rm Assign}(x,y), \\
%\\
\sigma_6 &:& {\rm Assign}(x,z)\,\, \assign\,\, {\rm Next}(x,y;z,\_),R(y), \\
\sigma_7 &:& {\rm Assign}(x,z)\,\, \assign\,\, {\rm Next}(x,y;\_,z),R(y), \\
%\\
\sigma_8 &:& R(x)\,\, \assign\,\, {\rm Last}(x).
\end{eqnarray*}	
It is easy to verify that $\dep$ is indeed a linear Datalog program.
%
The high-level idea underlying the program $\dep$ is, for each variable $v$ occurring in a given Boolean formula $\varphi$, to non-deterministically assign a value ($0$ or $1$) to $v$, and then check whether the global assignment makes $\varphi$ true.
%
The rules $\sigma_1$ and $\sigma_2$ are responsible for assigning $0$ or $1$ to a variable $v$; the last two positions of the relation ${\rm Var}$ always store the values $0$ and $1$, respectively.
%
The rules $\sigma_3$, $\sigma_4$, and $\sigma_5$ are responsible for checking whether an assignment for a certain variable $v$ makes a literal that mentions $v$ in some clause $C$ (and thus, $C$ itself) true.
%
The rules $\sigma_6$ and $\sigma_7$ are responsible, once we are done with a certain variable $v$, to consider the variable $u$ that comes after $v$; the relation ${\rm Next}$ provides an ordering of the variables in the given 3CNF Boolean formula.
%
Finally, once all the variables of the formula have been considered, $\sigma_8$ brings us to the last variable, which is a dummy one, that indicates the end of the above process.




\medskip
\noindent \textbf{From $\mathsf{3SAT}$ to  $\mathsf{Why\text {-}Provenance}[Q]$.} We now establish that $\mathsf{Why\text {-}Provenance}[Q]$ is \NP-hard by reducing from $\mathsf{3SAT}$.
%
Consider a 3CNF Boolean formula $\varphi = C_1 \wedge \cdots \wedge C_m$ with $n$ Boolean variables $v_1,\ldots,v_n$. For a literal $\ell$, we write $\lvar{\ell}$ for the variable occurring in $\ell$, and $\lsign{\ell}$ for the number $1$ (resp., $0$) if $\ell$ is a variable (resp., the negation of a variable).
%
We define $D_\varphi$ as the database over $\esch{\dep}$
\begin{eqnarray*}
	&& \{{\rm Var}(v_i;0,1) \mid i \in [n]\}\\
	&\cup& \{{\rm Next}(v_i,v_{i+1};0,1) \mid i \in [n-1]\}\\
	&\cup& \{{\rm Next}(v_n,\bullet;0,1), {\rm Last}(\bullet)\}\\
	&\cup& \{C(\lvar{\ell_1},\lsign{\ell_1};\lvar{\ell_2},\lsign{\ell_2};\lvar{\ell_3},\lsign{\ell_3}) \mid \\
	&& \hspace{28mm} (\ell_1 \vee \ell_2 \vee \ell_3) \text{ is a clause of } \varphi \},
\end{eqnarray*}
which essentially stores the clauses of $\varphi$ and provides an ordering of the variables occurring in $\varphi$, with $\bullet$ being a dummy one.
%
We can show the next lemma, which essentially states that the above construction leads to a correct polynomial-time reduction from $\mathsf{3SAT}$ to $\mathsf{Why\text {-}Provenance}[Q]$:

\begin{lemma}\label{lem:reduction-from-3sat}
	%The following hold:
	%\begin{enumerate}
		%\item 
		$D_\varphi$ can be constructed in polynomial time in $\varphi$. Furthermore,
		%\item 
		$\varphi$ is satisfiable iff $D_\varphi \in \why{(v_1)}{D_\varphi}{Q}$.
	%\end{enumerate}
\end{lemma}
\begin{proof}
Clearly, $D_{\varphi}$ is over $\esch{\dep}$, and $D_{\varphi},(v_1)$ can be constructed in polynomial time w.r.t.~$\varphi$. We now show that $\varphi$ is satisfiable if and only if there exists a proof tree $T$ of $R(v_1)$ w.r.t.~$D_{\varphi}$ and $\Sigma$ such that $\support{T} = D_{\varphi}$.


We start with the $(\Rightarrow)$ direction. Assume $\varphi$ is satisfiable via the truth assignment $\mu$. For every variable $v_i$, we denote by $S_{v_i}$ the set of facts of the form $C(v_i,\mu(v_i);\_,\_;\_,\_)$, $C(\_,\_;v_i,\mu(v_i);\_,\_)$, and $C(\_,\_;\_,\_;v_i,\mu(v_i))$. We define the labeled rooted tree $T=(V,E,\lambda)$ where the root $v \in V$ is labeled with $\lambda(v) = R(v_1)$, and inductively:
	\begin{enumerate}
		\item if $v \in V$ is labeled with $\lambda(v) = R(v_i)$, for $i \in [n]$, then $v$ has two children $u_1,u_2$, where 
		$\lambda(u_1)$ is the (only) fact in $D_{\varphi}$ of the form $Var(v_i;0,1)$ and  $\lambda(u_2) = {\rm Assign}(v_i,\mu(v_i))$.
	\item if $v \in V$ is labeled with $\lambda(v) = {\rm Assign}(v_i,\mu(v_i))$ for $i \in [n]$, and $S_{v_i}$ is not empty, then $v$ 
has two children $u_1,u_2$, where $\lambda(u_1)=f$ for some fact $f\in S_{v_i}$ and $\lambda(u_2) = {\rm Assign}(v_i,\mu(v_i))$. We then remove $f$ from $S_{v_i}$. 
\item if $v \in V$ is labeled with $\lambda(v) = {\rm Assign}(v_i,\mu(v_i))$ for $i \in [n-1]$, and $S_{v_i}$ is empty, then $v$ 
has two children $u_1,u_2$, where $\lambda(u_1)={\rm Next}(v_i,v_{i+1};0,1)$ and $\lambda(u_2) = R(v_{i+1})$.
\item if $v \in V$ is labeled with $\lambda(v) = {\rm Assign}(v_n,\mu(v_n))$ and $S_{v_n}$ is empty, then $v$
has two children $u_1,u_2$, where $\lambda(u_1)={\rm Next}(v_n,\bullet;0,1)$ and and $\lambda(u_2) = R(\bullet)$.
\item if $v \in V$ is labeled with $\lambda(v) = R(\bullet)$, then $v$ has one child $u_1$, where $\lambda(u_1)={\rm Last}(\bullet)$.
	\end{enumerate}
	One can verify that $T$ is indeed a proof tree of $R(v_1)$ w.r.t.~$D_\varphi$ and $\Sigma$. In particular, the root is labeled with $R(v_1)$ by construction, and it is easy to see that the labels of the leaves all occur in $D_\varphi$. Furthermore, since $\dep$ is linear, at each level of $T$ there exists at most one non-leaf node. The edges from this node are defined in items~$(1)-(5)$ above. The edges defined in item~(1) are obtained by considering rules $\sigma_1,\sigma_2$, the edges defined in item~(2) are obtained by considering rules $\sigma_3,\sigma_4,\sigma_5$, and the edges defined in items~(3) and~(4) are obtained by considering rules $\sigma_6,\sigma_7$. Finally, the edge defined in item~(5) is obtained by considering rule $\sigma_8$. Hence, Property~$(3)$ of Definition~\ref{def:proof-tree} holds.
 
 Regarding the support of $T$, since $\mu$ is a satisfying assignment, every clause of $\varphi$ contains at least one literal $\ell_j$ such that $\mu(\lvar{\ell_j})=\lsign{\ell_j}$. Item~(1) ensures that we follow this satisfying assignment (i.e., add a node labeled with ${\rm Assign}(v_i,\mu(v_i))$ under a node $R(v_i)$ corresponding to the variable $v_i$). Item~(2) ensures that whenever we consider a variable $v_i$, we touch every atom over the predicate $C$ corresponding to a clause that contains the variable $v_i$ with the correct sign (positive if $\mu(v_i)=1$ or negated if $\mu(v_i)=0$); that is, every clause satisfied by the assignment to this variable. Items~(3) and~(4) ensure that we go over all the variables, as the atoms over the predicate ${\rm Next}$ only allow us to move from a variable $v_i$ to the next variable $v_{i+1}$. This, in turn, ensures that we touch every atom over the predicate ${\rm Var}$ by item~(1), every atom over the predicate $C$ by item~(2) (since, as aforementioned, each such clause contains a literal with a sign that is consistent with the assignment to the corresponding variable), and every atom over the predicate ${\rm Next}$ by items~(3) and~(4). Finally, item~(5) ensures that we touch the atom ${\rm Last}(\bullet)$. Hence, we conclude that $\support{T} = D_{\varphi}$.
	

Next, we prove the $(\Leftarrow)$ direction. Assume that $T=(V,E,\lambda)$ is a proof tree for $R(v_1)$ w.r.t.~$D_\varphi$ and $\Sigma$ such that $\support{T}=D_{\varphi}$. 
	Note that by linearity of $\dep$, at each level of $T$, besides the last one, there exists precisely one non-leaf node. These nodes thus form a path $u_0,u_1,\ldots$ in $T$. Note that $u_0$, i.e., the root, is necessarily labeled with the fact $R(v_1)$ by the definition of $Q$. Then, the children of the root are labeled with ${\rm Var}(v_1;0,1)$ and ${\rm Assign}(v_1,b_1)$ (this is the node $u_1$) for some $b_1\in\{0,1\}$, based on rule $\sigma_1$ or $\sigma_2$, as these are the only options. At this point, either the node $u_2$ is labeled with $R(v_2)$ based on rule $\sigma_6$ or $\sigma_7$, or $u_2,\dots,u_j$ for some $j\ge 2$ are all labeled with ${\rm Assign}(v_1,b_1)$, and then $u_{j+1}$ is labeled with $R(v_2)$. Then, the same reasoning applies to $v_2$ and all the variables that come next.

    Since the atoms over the predicate ${\rm Next}$ only consider consecutive variables, it is only possible to move from $v_i$ to $v_{i+1}$ along the path, and since $\support{T}=D_\varphi$, the set $\support{T}$ contains all the atoms over ${\rm Next}$, and we conclude that we go over all variables. Moreover, for every variable $v_i$, once we select the label ${\rm Assign}(v_i,b_i)$ for the child of the node labeled with $R(v_i)$, there is no rule that allows us to obtain the label ${\rm Assign}(v_i,1-b_i)$, and so the labels over the predicate ${\rm Assign}$ correspond to a truth assignment $\mu$ to the variables $v_1,\dots,v_n$. Finally, since all the atoms over the predicate $C$ appear in the support, the rules $\sigma_3,\sigma_4,\sigma_5$ imply that for every clause $(\ell_1\vee\ell_2\vee\ell_3)$, either ${\rm Assign}(\lvar{\ell_1},\lsign{\ell_1})$, or ${\rm Assign}(\lvar{\ell_2},\lsign{\ell_2})$, or ${\rm ssign}(\lvar{\ell_3},\lsign{\ell_3})$ appears as a label along the path. Hence, $\mu$ is a satisfying truth assignment.
\end{proof}

By Lemma~\ref{lem:reduction-from-3sat}, $\mathsf{Why\text {-}Provenance}[Q]$ is \NP-hard, and thus, $\mathsf{Why\text {-}Provenance[\LDAT]}$ is \NP-hard in data complexity.


\subsection{Non-Recursive Queries}

We now focus on non-recursive Datalog queries, and give the full proof of Theorem~\ref{the:non-recursive-complexity}, which we recall here:

\begin{manualtheorem}{\ref{the:non-recursive-complexity}}
	\thenonrecursivecomplexity
\end{manualtheorem}

Given a non-recursive Datalog query $Q$, we have already explained in the main body how the FO query $Q_{\mi{FO}}$ is constructed. Our main task here is to establish the correctness of this construction, i.e., Lemma~\ref{lem:fo-tree-equiv}, which we recall here:


\begin{manuallemma}{\ref{lem:fo-tree-equiv}}
	\lemmaforewriting
\end{manuallemma}

\begin{proof}
	We first discuss the $(\Rightarrow)$ direction. There is a proof tree $T$ of $R(\bar t)$ w.r.t.~$D$ and $\dep$ with $\support{T} = D'$. Thus, $T$ is a $Q$-tree, the set $\cq{Q}$ contains the CQ $\cq{T}$, that is, the CQ induced by $T$, and the set $\cqeq{Q}$ contains a CQ $\varphi(\bar y)$ that is the same as $\cq{T}$ up to variable renaming. It is then an easy exercise to show that that $D'$ satisfies the sentence $\psi_{\varphi(\bar y)}[\bar x / \bar t]$. This in turn implies that $\bar t \in Q_{\mi{FO}}(D')$.
	
	We now discuss the $(\Leftarrow)$ direction. There is a CQ $\varphi(\bar y) \in \cqeq{Q}$ such that $D'$ satisfies the sentence $\psi_{\varphi(\bar y)}[\bar x / \bar t]$. It is then an easy exercise to show that there exists a proof tree of $R(\bar t)$ w.r.t.~$D$ and $\dep$ with $\support{T} = D'$. This in turn implies that $D \in \why{R(\bar t)}{D}{Q}$.
\end{proof}
\section{Non-Recursive Proof Trees}\label{appsec:refined-trees}
%


As discussed in the main body of the paper (see Section~\ref{sec:refined-trees}), the standard notion of why-provenance, which was defined in Section~\ref{sec:why-provenance} and thoroughly analyzed in Section~\ref{sec:complexity}, relies on arbitrary proof trees without any restriction. Indeed, a subset of the input database $D$ belongs to the why-provenance of a tuple $\bar t$ w.r.t.~$D$ and a Datalog query $Q = (\dep,R)$ as long as it is the support of {\em any} proof tree of $R(\bar t)$ w.r.t.~$D$ and $\dep$.
%
However, as already discussed in the literature (see, e.g., the recent work~\cite{BBPT22}), there are proof trees that are counterintuitive. Such a proof tree is the second one in Example~\ref{exa:proof-tree} as the fact $A(a)$ is derived from itself.
%
Now, a member $D'$ of $\why{\bar t}{D}{Q}$, witnessed via such an unnatural proof tree, might be classified as a counterintuitive explanation of $\bar t$ as it does not correspond to an intuitive derivation process, which can be extracted from the proof tree, that derives from $D'$ the fact $R(\bar t)$.
%
This leads to the need of considering refined classes of proof trees that overcome the conceptual limitations of arbitrary proof trees, which in turn lead to conceptually intuitive explanations.
%
In this section, we focus on the class of non-recursive proof trees. Roughly, a non-recursive proof tree is a proof
tree that does not contain two nodes labeled with the same fact and such that one is the descendant of the other, which reflects the above discussion that using a fact to derive itself is a counterintuitive phenomenon. The formal definition follows:

\begin{definition}[\textbf{Non-Recursive Proof Tree}]\label{def:non-recursove-proof-tree}
	Consider a Datalog program $\dep$, a database $D$ over $\esch{\dep}$, and a fact $\alpha$ over $\sch{\dep}$. A {\em non-recursive proof tree of $\alpha$ w.r.t.~$D$ and $\dep$} is a proof tree $T = (V,E,\lambda)$ of $\alpha$ w.r.t.~$D$ and $\dep$ such that, for every two nodes $v,u \in V$, if there is a path from $v$ to $u$ in $T$, then $\lambda(v) \neq \lambda(u)$. \hfill\markfull
\end{definition}



We now define why-provenance relative to non-recursive proof trees. Given a Datalog query $Q = (\dep,R)$, a database $D$ over $\esch{\dep}$, and a tuple $\bar t \in \adom{D}^{\arity{R}}$, the {\em why-provenance of $\bar t$ w.r.t.~$D$ and $Q$ relative to non-recursive proof trees} is defined as the family of sets of facts
\begin{multline*}
\{\support{T} \mid T \text{ is a non-recursive proof tree of }\\
R(\bar t) \text{ w.r.t. } D \text{ and } \dep\}
\end{multline*}
denoted $\nrwhy{\bar t}{D}{Q}$.
%
Then, the algorithmic problems
\[
\mathsf{Why\text {-}Provenance_{NR}[C]} \quad \text{and} \quad  \mathsf{Why\text {-}Provenance_{NR}}[Q]
\] 
are defined in the exact same way as those in Section~\ref{sec:why-provenance} with the key difference that $\nrwhy{\bar t}{D}{Q}$ is used instead of $\why{\bar t}{D}{Q}$, i.e., the question is whether the given subset of the database belongs to $\nrwhy{\bar t}{D}{Q}$. 
%
We proceed to study the data complexity of $\mathsf{Why\text {-}Provenance_{NR}[C]}$ for each class $\class{C} \in \{\DAT,\LDAT,\NRDAT\}$. As shown in the case of arbitrary proof trees, for recursive queries, even if the recursion is restricted to be linear, the problem is in general intractable, whereas for non-recursive queries it is highly tractable. We first focus on recursive queries.



\subsection{Recursive Queries}
%

We show the following complexity result:

\begin{theorem}\label{the:complexity-non-recursive-proof-trees-np}
	$\mathsf{Why\text {-}Provenance_{NR}[C]}$ is \NP-complete in data complexity, for each class $\class{C} \in \{\DAT,\LDAT\}$.
\end{theorem}


To prove Theorem~\ref{the:complexity-non-recursive-proof-trees-np}, it suffices to show that:
\begin{itemize}
	\item $\mathsf{Why\text {-}Provenance_{NR}[\DAT]}$ is in \NP~in data complexity.
	\item $\mathsf{Why\text {-}Provenance_{NR}[\LDAT]}$ is \NP-hard in data complexity.
\end{itemize}
Let us first focus on the upper bound.


\medskip

\noindent \underline{\textbf{Upper Bound}}
\smallskip

\noindent The proof is similar to the proof of the analogous result for $\mathsf{Why\text {-}Provenance[\DAT]}$ established in Section~\ref{sec:why-provenance}.
%
Given a Datalog program $\dep$, a database $D$ over $\esch{\dep}$, and a fact $\alpha$ over $\sch{\dep}$, we first define the notion of {\em non-recursive proof DAG of $\alpha$ w.r.t.~$D$ and $\dep$}.
%
We then proceed to establish a result analogous to Proposition~\ref{pro:characterization-all-trees}: the existence of a non-recursive proof tree of $\alpha$ w.r.t.~$D$ and $\dep$ with $\support{T} = D' \subseteq D$ is equivalent to the existence of a polynomially-sized non-recursive proof DAG $G$ of $\alpha$ w.r.t.~$D$ and $\dep$ with $\support{G} = D'$. This in turn leads to a guess-and-check algorithm that runs in polynomial time.
%
Let us formalize the above high-level description.



\begin{definition}[\textbf{Non-Recursive Proof DAG}]\label{def:non-recursive-proof-dag}
	Consider a Datalog program $\dep$, a database $D$ over $\esch{\dep}$, and a fact $\alpha$ over $\sch{\dep}$. 
	%
	A {\em non-recursive proof DAG of $\alpha$ w.r.t.~$D$ and $\dep$} is a proof DAG $G=(V,E,\lambda)$ of $\alpha$ w.r.t.~$D$ and $\dep$ such that, for every two nodes $v,u \in V$, if there is a path from $v$ to $u$ in $G$, then $\lambda(v) \neq \lambda(u)$. \hfill\markfull
\end{definition}


The analogous result to Proposition~\ref{pro:characterization-all-trees} follows:



\begin{proposition}\label{pro:characterization-nr-trees}
		For a Datalog program $\dep$, there is a polynomial $f$ such that, for every database $D$ over $\esch{\dep}$, fact $\alpha$ over $\sch{\dep}$, and $D' \subseteq D$, the following are equivalent:
	\begin{enumerate}
		\item There is a non-recursive proof tree $T$ of $\alpha$ w.r.t.~$D$ and $\dep$ such that $\support{T} = D'$.
		\item There is a non-recursive proof DAG $G = (V,E,\lambda)$ of $\alpha$ w.r.t.~$D$ and $\dep$ with $\support{G} = D'$ and $|V| \leq f(|D|)$.
	\end{enumerate}
\end{proposition}



The direction (2) implies (1) is shown by ``unravelling'' the non-recursive proof DAG $G$ into a non-recursive proof $T$ with $\support{G} = \support{T}$. We use the same ``unravelling'' construction as in the proof of direction (2) implies (1) of Proposition~\ref{pro:characterization-all-trees}, which {\em preserves non-recursiveness}.


We now proceed with (1) implies (2). The underlying construction proceeds in two main steps captured by Lemmas~\ref{lem:scount-reduction-nr} and~\ref{lem:from-trees-to-dags-nr} given below.

%whose essence is captured by three technical lemmas (Lemma~\ref{lem:depth-reduction},~\ref{lem:scount-reduction}, and~\ref{lem:from-trees-to-dags}).

\medskip 

$\bullet$ The \textbf{\textit{first step}} is to show that a non-recursive proof tree $T$ of $\alpha$ w.r.t.~$D$ and $\dep$ with $\support{T} = D'$ can be converted into a non-recursive proof tree $T'$ of $\alpha$ w.r.t.~$D$ and $\dep$ with $\support{T'} = D'$ that has ``small'' subtree count.


\begin{lemma}\label{lem:scount-reduction-nr}
		For each Datalog program $\dep$, there is a polynomial $f$ such that, for every database $D$ over $\esch{\dep}$, fact $\alpha$ over $\sch{\dep}$, and $D' \subseteq D$, if there is a non-recursive proof tree $T$ of $\alpha$ w.r.t.~$D$ and $\dep$ with $\support{T} = D'$, then there is also such a proof tree $T'$ with $\scount{T'} \leq f(|D|)$.
\end{lemma}

\begin{proof}
	We first observe that the non-recursive proof tree $T$, by definition, has ``small'' depth. In particular, since no two nodes on a path of $T$ have the same label, the length of a path is bounded by the number of labels, that is, $|\base{D,\dep}|$, which is clearly polynomial in the size of the database $D$.
	%
	The other crucial observation is that the construction underlying Lemma~\ref{lem:scount-reduction}, which converts a proof tree of ``small'' depth into a proof tree of ``small`` subtree count with the same support {\em preserves non-recursiveness}.
	%
	Consequently, we can apply the construction underlying Lemma~\ref{lem:scount-reduction} to the non-recursive proof tree $T$ and get a non-recursive proof tree $T'$ with $\support{T} = \support{T'}$ such that $\scount{T'} \leq f(|D|)$, where $f$ is the polynomial provided by Lemma~\ref{lem:scount-reduction}.
\end{proof}


\smallskip

$\bullet$ The \textbf{\textit{second step}} shows that a non-recursive proof tree $T$ of $\alpha$ w.r.t.~$D$ and $\dep$ with $\support{T} = D'$ of ``small'' subtree count can be converted into a compact non-recursive proof DAG $G$ of $\alpha$ w.r.t.~$D$ and $\dep$ with $\support{G} = D'$.

\begin{lemma}\label{lem:from-trees-to-dags-nr}
	For each Datalog program $\dep$ and a polynomial $f$, there is a polynomial $g$ such that, for every database $D$ over $\esch{\dep}$, fact $\alpha$, and $D' \subseteq D$, if there is a non-recursive proof tree $T$ of $\alpha$ w.r.t.~$D$ and $\dep$ with $\support{T} = D'$ and $\scount{T} \leq f(|D|)$, then there is a non-recursive proof DAG $G = (V,E,\lambda)$ of $\alpha$ w.r.t.~$D$ and $\dep$ with $\support{G} = D'$ and $|V| \leq g(|D|)$.
\end{lemma}

\begin{proof}
	We employ the construction underlying Lemma~\ref{lem:from-trees-to-dags}, which converts a proof tree of ``small'' subtree count into a non-recursive proof DAG of polynomial size with the same support, since it {\em preserves non-recursiveness}. The latter holds since, for each path of the proof tree, there is a path in the proof DAG with the same labels, and vice versa. 
\end{proof}

It is now clear that the direction (1) implies (2) of Proposition~\ref{pro:characterization-nr-trees} is an immediate consequence of Lemmas~\ref{lem:scount-reduction-nr} and~\ref{lem:from-trees-to-dags-nr}.


\medskip
\noindent
\textbf{Finalize the Proof.} We can now finalize the proof of the claim that $\mathsf{Why\text {-}Provenance_{NR}[\DAT]}$ is in \NP~in data complexity.
%
Fix a Datalog query $Q = (\dep,R)$. Given a database $D$ over $\esch{\dep}$, a tuple $\bar t \in \adom{D}^{\arity{R}}$, and a subset $D'$ of $D$, to decide whether $D' \in \nrwhy{\bar t}{D}{Q}$ we simply need to check for the existence of a non-recursive proof tree $T$ of $R(\bar t)$ w.r.t.~$D$ and $\dep$ such that $\support{T} = D'$. By Proposition~\ref{pro:characterization-nr-trees}, this is tantamount to the existence of a polynomially-sized non-recursive proof DAG $G$ of $R(\bar t)$ w.r.t.~$D$ and $\dep$ with $\support{G} = D'$.
%
The existence of such a non-recursive proof DAG can be checked via a non-deterministic algorithm that runs in polynomial time in the size of the database as it was done for proving that $\mathsf{Why\text {-}Provenance[\DAT]}$ is in \NP~in data complexity (Theorem~\ref{the:recursive-complexity}). The only difference is that now we need to additionally check that the guessed DAG is also non-recursive.
%
This can be done by going over the nodes of the
DAG using depth-first search, and remembering all the labels that we have seen on the current path. Whenever we encounter a new node, we need to verify that its label does not belong to the set of labels that we have already seen on this path. If it does, then we reject; otherwise, we add its label to the set of labels and continue. Whenever we go ``back'' in the DAG (move from a node $u$ to a node $v$ when there is an edge $(v,u)$), we remove the label of the child node from the set. This requires $O(|V| \cdot |E|)$ time, where $V$ and $E$ are the sets of nodes and edges of the DAG, respectively.
%
Consequently, $\mathsf{Why\text {-}Provenance}[Q]$ is in \NP, and thus, $\mathsf{Why\text {-}Provenance[\DAT]}$ is in \NP~in data complexity.



\medskip

\noindent \underline{\textbf{Lower Bound}}
\smallskip

\noindent We proceed to establish that $\mathsf{Why\text {-}Provenance_{NR}[\LDAT]}$ is \NP-hard in data complexity. To this end, we need to show that there exists a linear Datalog query $Q$ such that the problem $\mathsf{Why\text {-}Provenance_{NR}}[Q]$ is \NP-hard. The proof is via a reduction from the problem $\mathsf{Ham\text{-}Cycle}$, which takes as input a directed graph $G = (V,E)$ and asks whether $G$ has a Hamiltonian cycle, i.e., whether there exists a cycle $v_1,\ldots,v_n,v_1$ in $G$ such that, for distinct integers $i,j \in [n]$, $v_i \neq v_j$, and $V = \{v_1,\ldots,v_n\}$.



\medskip
\noindent \textbf{The Linear Datalog Query.}
We start by defining the linear Datalog query $Q = (\dep,{\rm Path})$. If the name of a variable is not important, then we use $\_$ for a fresh variable occurring only once in $\dep$. By abuse of notation, we use semicolons instead of commas in a tuple expression in order to separate terms with a different meaning. The program $\dep$ follows:
\begin{eqnarray*}
\sigma_1 &:& {\rm MarkedE}(x)\,\, \assign \,\, {\rm First}(x)\\
\sigma_2 &:& {\rm MarkedE}(y)\,\, \assign \,\, E(\_,\_;x,y;\_), {\rm MarkedE}(x),\\
\sigma_3 &:& {\rm Path}(y)\,\, \assign \,\, E(x,y;\_,\_;z), {\rm MarkedE}(z), N(x), \\
\sigma_4 &:& {\rm Path}(y)\,\, \assign \,\, E(x,y;\_,\_;\_), {\rm Path}(x), N(x).
\end{eqnarray*}
It is easy to verify that $\dep$ is indeed a linear Datalog program.
\begin{comment}
The high-level idea underlying the program $\dep$ is, for each variable $v$ occurring in a given Boolean formula $\varphi$, to non-deterministically assign a value ($0$ or $1$) to $v$, and then check whether the global assignment makes $\varphi$ true.
%
The rules $\sigma_1$ and $\sigma_2$ are responsible for assigning $0$ or $1$ to a variable $v$; the last two positions of the relation ${\rm Var}$ always store the values $0$ and $1$, respectively.
%
The rules $\sigma_3$, $\sigma_4$, and $\sigma_5$ are responsible for checking whether an assignment for a certain variable $v$ makes a literal that mentions $v$ in some clause $C$ (and thus, $C$ itself) true.
%
The rules $\sigma_6$ and $\sigma_7$ are responsible, once we are done with a certain variable $v$, to consider the variable $u$ that comes after $v$; the relation ${\rm Next}$ provides an ordering of the variables in the given 3CNF Boolean formula.
%
Finally, once all the variables of the formula have been considered, $\sigma_8$ brings us to the last variable, which is a dummy one, that indicates the end of the above process.
\end{comment}



\medskip
\noindent \textbf{From $\mathsf{Ham\text{-}Cycle}$ to  $\mathsf{Why\text {-}Provenance_{NR}}[Q]$.} We now show that $\mathsf{Why\text {-}Provenance}[Q]$ is \NP-hard by reducing from $\mathsf{Ham\text{-}Cycle}$.
%
Consider a directed graph $G = (V,E)$ with $E = \{e_1,\ldots,e_m\}$.
%
Let $D_G$ be the database over $\esch{\dep}$
\begin{multline*}
\{{\rm First}(1)\} \cup \{N(v) \mid v \in V\}\ \cup \\
\{E(u,v;i,i+1;m+1) \mid i \in [m] \text{ and } e_i = (u,v)\},
\end{multline*}
which essentially stores the graph $G$ and provides an ordering of its edges.
%
We now show the next lemma, which states that the above construction leads to a correct polynomial-time reduction from $\mathsf{Ham\text{-}Cycle}$  to $\mathsf{Why\text {-}Provenance_{NR}}[Q]$.

\begin{lemma}\label{lem:reduction-from-hasm-cycle}
	The following hold:
	\begin{enumerate}
	\item $D_G$ can be constructed in polynomial time w.r.t.~$G$.
	\item $G$ has a Hamiltonian cycle iff $D_G \in \nrwhy{(v^*)}{D_G}{Q}$ for some arbitrary node $v^* \in V$.
	\end{enumerate}
\end{lemma}

\begin{proof}
	It is straightforward to see that $D_G$ can be constructed in polynomial time in the size of $G$. We proceed to establish item (2). We start with the direction $(\Rightarrow )$.
	%
	%\medskip
	%\noindent \underline{Direction $(\Rightarrow)$}
	%\smallskip
	%
	%\noindent 
	Assume that $G$ has a Hamiltonian cycle $v_1,\ldots,v_n,v_{n+1}$, and w.l.o.g.~let $v_1 = v_{n+1} = v^*$. Hence, there exist edges $(v_1,v_2),\ldots,(v_{n-1},v_n),(v_n,v_{n+1})$ in $G$. We define the labeled rooted tree $T=(V',E',\lambda)$: for the root $v \in V'$, let $\lambda(v) = {\rm Path}(v_{n+1}) = {\rm Path}(v^*)$, and inductively:
	\begin{enumerate}
		\item If $v \in V'$ is such that $\lambda(v) = {\rm Path}(v_{i+1})$, for $i \in \{2,\ldots,n\}$, then $v$ has 3 children $u_1,u_2,u_3$, where 
		$\lambda(u_1)$ is the (only) fact in $D_{G}$ of the form $E(v_i,v_{i+1};\cdot,\cdot;\cdot)$, $\lambda(u_2) = {\rm Path}(v_i)$, and $\lambda(u_3) = N(v_i)$.
		
		\item if $v \in V'$ is such that $\lambda(v) = {\rm Path}(v_2)$, $v$ has 3 children $u_1,u_2,u_3$, where $\lambda(u_1)$ is the (only) fact in $D_G$ of the form $E(v_1,v_2;\cdot,\cdot;m+1)$, $\lambda(u_2) = {\rm MarkedE}(m+1)$, and $\lambda(u_3) = N(v_1)$.
		
		\item if $v \in V'$ is such that $\lambda(v) = {\rm MarkedE}(i+1)$, for $i \in \{1,\ldots,m\}$, then $v$ has 2 children $u_1,u_2$, where $\lambda(u_1)$ is the (only) fact in $D_G$ of the form $E(\cdot,\cdot;i,i+1;\cdot)$, and $\lambda(u_2) = {\rm MarkedE}(i)$.
		
		\item if $v \in V'$ is such that $\lambda(v) = {\rm MarkedE}(1)$, then $v$ has only one child $u$, where $\lambda(u) = {\rm First}(1)$.
	\end{enumerate}
	This completes the construction of $T$.
	%
	We proceed to show that $T$ is a non-recursive proof tree of ${\rm Path}(v^*)$ w.r.t.~$D_G$ and $Q$ such that $\support{T} = D_G$, which in turn implies that $D_G \in \nrwhy{(v^*)}{D_G}{Q}$, as needed.
	%
	Since $\dep$ is linear, at each level of $T$ there exists at most one non-leaf node. Moreover, the edges in $T$ going from level $i$ to level $i+1$, with $i \in \{0,\ldots,n-1\}$, and the labels of the corresponding nodes, as defined in item~(1), are valid since they are obtained by considering rule $\sigma_4$ and there exist edges $(v_i,v_{i+1}) \in E$, for $i \in \{2,\ldots,n\}$. The edges in $T$ from level $n-1$ to $n$ are obtained considering $\sigma_3$, and the labels are again valid since $(v_1,v_2) \in E$. Then, for all the remaining levels, except the last one, we consider $\sigma_2$ in item~(3), and for the last level we consider $\sigma_1$, in item~(4). In these last two cases, it is easy to verify that the labels are valid.
	Crucially, by construction of $T$, for every non-leaf node there is no other node in $T$ with the same label, and thus, the proof tree is trivially non-recursive. Regarding the support, by item~(1), $N(v_i) \in \support{T}$, for $i \in \{2,\ldots,n\}$. By item~(2), $N(v_1) \in \support{T}$. Moreover, by item~(3), $E(u,v;i,i+1;m+1) \in \support{T}$, where $(u,v) = e_i$, for $i \in [m]$. Finally, by item~(4), ${\rm First}(1) \in \support{T}$. Therefore, $\support{T}=D_G$, and the claim follows.
	
	%\medskip
	%\noindent \underline{Direction $(\Leftarrow)$}
	%\smallskip
	
	%\noindent 
	We now proceed with the direction $(\Leftarrow)$. Assume that $D_G \in \nrwhy{(v^*)}{D_G}{Q}$, which in turn implies that there is a non-recursive proof tree $T=(V,E,\lambda)$ of $(v^*)$ w.r.t.~$D_G$ and $Q$ such that $\support{T}=D_{G}$. Let $n$ be the number of nodes in $G$, and assume w.l.o.g.\ that $n \ge 2$.
	Note that by linearity of $\dep$, at each level of $T$, besides the last one, there exists precisely one non-leaf node. Let $u_i$ be the non-leaf node at level $i$. Clearly, $u_0,u_1,\ldots$ is a path in $T$. Note that, by the definition of $Q$, $u_0$, i.e., the root, is necessarily labeled with a fact using the predicate ${\rm Path}$; this is also the case for $u_i$, at level $i \in \{1,\ldots,n-1\}$. Indeed, assume, towards a contradiction, that $i \in \{1,\ldots,n-1\}$ is the first level in $T$ where $u_i$ is \emph{not} labeled with a fact using the predicate ${\rm Path}$. Since $\sigma_3$ and $\sigma_4$ are the only rules in $\dep$ with a body-atom using the predicate $N$, and since these are the only rules where ${\rm Path}$ appears in the head, $\support{T}$ contains no more than $i < n$ facts using the predicate $N$, and thus, $\support{T} \neq D_G$, which is a contradiction.
	
	Since $u_0,\ldots,u_{n-1}$ is a path in $T$, and such nodes are all labeled with a fact using ${\rm Path}$, we conclude that such labels are obtained by using $\sigma_4$, as it is the only rule having the predicate ${\rm Path}$ both in its body and its head. Hence, because of the atom $E(x,y;\_,\_;\_)$ in $\body{\sigma_4}$, which uses an extensional predicate of $\dep$, and from the fact that $T$ is a proof tree, $\lambda(u_i) = {\rm Path}(v_i)$, where $v_i$ is some node of the graph $G$. Therefore, since $T$ is non-recursive, ${\rm Path}(v_0),{\rm Path}(v_1),\ldots,{\rm Path}(v_{n-1})$ are all distinct, and thus, $v_0,v_1,\ldots,v_{n-1}$ are distinct. Hence, again from the fact that the atom $E(x,y;\_,\_;\_)$ appears in $\body{\sigma_4}$ with an extensional predicate, the fact that $T$ is a proof tree, and by the construction of $D_G$, we conclude that $(v_{n-1},v_{n-2}),\ldots,(v_1,v_0)$ are all the edges of the graph $G$. Therefore, $v_0,\ldots,v_{n-1}$ is the reverse of a Hamiltonian path in $G$. 
	%
	Moreover, let $S_i \subseteq \support{T}$ be the set of facts using the predicate $N$ that label nodes of $T$ up to level $i$. Because of the atom $N(x)$ in the body of $\sigma_4$, we conclude that $S_{n-1} = \{N(v_1),\ldots,N(v_{n-1})\}$, that is, $S_{n-1}$ contains all nodes of the graph $G$, except for $v_0$.
	
	Let us focus now on the node $u_{n-1}$ of $T$. Recall that $\lambda(u_{n-1}) = {\rm Path}(v_{n-1})$. Note that the children of $u_{n-1}$ cannot be labeled using $\sigma_4$ anymore, since it contains the two body atoms ${\rm Path}(x),N(x)$, where $N$ is an extensional predicate. Hence, since $v_0,\ldots,v_{n-1}$ are precisely all the nodes of $G$, one of the children of $u_{n-1}$ would necessarily be labeled with a fact ${\rm Path}(v)$, where $v$ will necessarily coincide with some of the nodes in $v_0,\ldots,v_{n-1}$, and thus, $T$ would not be non-recursive. Hence, the only rule left is $\sigma_3$. As already discussed, up to level $n-1$, the set of facts in $\support{T}$ with predicate $N$ is $S_{n-1}=\{N(v_1),\ldots,N(v_{n-1})\}$. Hence, for $\support{T}$ to also contain ${\rm Path}(v_0) \in D_G$, there must be at least one more node below level $n-1$ in $T$ that is labeled with $N(v_0)$. Since $\sigma_3$ has no body atom with predicate ${\rm Path}$, necessarily one of the children of $u_{n-1}$ is labeled with $N(v_0)$. Hence, thanks to the atom $E(x,y;\_,\_;z)$ in the body of $\sigma_4$, we conclude that there is also an edge $(v_0,v_{n-1})$ in $G$, and thus, $v_0,v_1,\ldots,v_{n-1},v_0$ is the reverse of a Hamiltonian cycle of $G$, and the claim follows.
\end{proof}

By Lemma~\ref{lem:reduction-from-hasm-cycle}, $\mathsf{Why\text {-}Provenance_{NR}}[Q]$ is \NP-hard. Thus, $\mathsf{Why\text {-}Provenance_{NR}[\LDAT]}$ is \NP-hard in data complexity.




\subsection{Non-Recursive Queries}
%

We now focus on non-recursive Datalog queries, and show the following about the data complexity of why-provenance relative to non-recursive proof trees:

\begin{theorem}\label{the:non-recursive-complexity-nr}
	$\mathsf{Why\text {-}Provenance_{NR}[\NRDAT]}$ is in $\ACZ$ in data complexity.
\end{theorem}

\begin{proof}
This is shown via first-order rewritability as done for Theorem~\ref{the:non-recursive-complexity}. In fact, the construction of the target FO query is exactly the same as in the proof of Theorem~\ref{the:non-recursive-complexity} with the key difference that, for a Datalog query $Q$, the set of CQs $\cq{Q}$ is defined by considering only non-recursive proof trees, i.e., is the set $\{\cq{T} \mid T \text{ is a {\em non-recursive} $Q$-tree}\}$.
\end{proof}


%%%%%%%%%%%%%%%%%%%%%%%%%%%%%%%%%%%%%%%


\section{Minimal-Depth Proof Trees}\label{sec:minimal-depth-proof-trees}
%

We now focus on another refined class of proof trees that has been considered in the literature. Recall that the depth of a rooted tree $T$, denoted $\depth{T}$, is the length of the longest path from its root to a leaf node. Given a Datalog program $\dep$, a database $D$ over $\esch{\dep}$, and a fact $\alpha \in \dep(D)$, let $\mtd{\alpha}{D}{\dep}$ be the integer
\[
\min\{\depth{T} \mid T \text{ is a proof tree of } \alpha \text{ w.r.t. } D \text{ and } \dep\},
\]
i.e., the minimal depth over all proof trees of $\alpha$ w.r.t.~$D$ and $\dep$. The notion of minimal-depth proof tree follows:

\begin{definition}[\textbf{Minimal-Depth Proof Tree}]\label{def:min-depth-proof-tree}
	Consider a Datalog program $\dep$, a database $D$ over $\esch{\dep}$, and a fact $\alpha$ over $\sch{\dep}$. A {\em minimal-depth proof tree of $\alpha$ w.r.t.~$D$ and $\dep$} is a proof tree $T$ of $\alpha$ w.r.t.~$D$ and $\dep$ such that $\depth{T}$ coincides with $\mtd{\alpha}{D}{\dep}$. \hfill\markfull
\end{definition}


Why-provenance relative to minimal-depth proof trees is defined as expected. Given a Datalog query $Q = (\dep,R)$, a database $D$ over $\esch{\dep}$, and a tuple $\bar t \in \adom{D}^{\arity{R}}$, the {\em why-provenance of $\bar t$ w.r.t.~$D$ and $Q$ relative to minimal-depth proof trees} is defined as the family of sets of facts
\begin{multline*}
\{\support{T} \mid T \text{ is a minimal-depth proof tree of }\\
R(\bar t) \text{ w.r.t. } D \text{ and } \dep\}
\end{multline*}
denoted $\mdwhy{\bar t}{D}{Q}$.
%
Then, the algorithmic problems
\[
\mathsf{Why\text {-}Provenance_{MD}[C]} \quad \text{and} \quad  \mathsf{Why\text {-}Provenance_{MD}}[Q]
\] 
are defined as expected. We proceed to study the data complexity of $\mathsf{Why\text {-}Provenance_{MD}[C]}$ for each class $\class{C} \in \{\DAT,\LDAT,\NRDAT\}$. As shown in the case of arbitrary and non-recursive proof trees, for recursive queries, even if the recursion is restricted to be linear, the problem is in general intractable, whereas for non-recursive queries it is highly tractable. We first focus on recursive queries.



\subsection{Recursive Queries}
%

We show the following complexity result:

\begin{theorem}\label{the:complexity-minimal-depth-proof-trees-np}
	$\mathsf{Why\text {-}Provenance_{MD}[C]}$ is \NP-complete in data complexity, for each class $\class{C} \in \{\DAT,\LDAT\}$.
\end{theorem}



To prove Theorem~\ref{the:complexity-non-recursive-proof-trees-np}, it suffices to show that:
\begin{itemize}
	\item $\mathsf{Why\text {-}Provenance_{MD}[\DAT]}$ is in \NP~in data complexity.
	\item $\mathsf{Why\text {-}Provenance_{MD}[\LDAT]}$ is \NP-hard in data compl.
\end{itemize}
Let us first focus on the upper bound.


\medskip

\noindent \underline{\textbf{Upper Bound}}
\smallskip

\noindent The proof is similar to the proof of the analogous result for $\mathsf{Why\text {-}Provenance_{NR}[\DAT]}$ (see Theorem~\ref{the:complexity-non-recursive-proof-trees-np}). 
%
Given a Datalog program $\dep$, a database $D$ over $\esch{\dep}$, and a fact $\alpha$ over $\sch{\dep}$, we first define the notion of {\em minimal-depth proof DAG of $\alpha$ w.r.t.~$D$ and $\dep$}.
%
We then establish a result analogous to Proposition~\ref{pro:characterization-nr-trees}: the existence of a minimal-depth proof tree of $\alpha$ w.r.t.~$D$ and $\dep$ with $\support{T} = D' \subseteq D$ is equivalent to the existence of a polynomially-sized minimal-depth proof DAG $G$ of $\alpha$ w.r.t.~$D$ and $\dep$ with $\support{G} = D'$. This in turn allows us to devise a guess-and-check algorithm that runs in polynomial time. We proceed to formalize this high-level description.


The notion of depth can be naturally transferred to rooted DAGs. In particular, for a rooted DAG $G$, the {\em depth} of $G$, denoted $\depth{G}$, is defined as the length of the longest path from the root of $G$ to a leaf node of $G$.
%
Given a Datalog program $\dep$, a database $D$ over $\esch{\dep}$, and a fact $\alpha \in \dep(D)$, let $\mgd{\alpha}{D}{\dep}$ be the integer
\[
\min\{\depth{G} \mid G \text{ is a proof DAG of } \alpha \text{ w.r.t. } D \text{ and } \dep\},
\]
i.e., the minimal depth over all proof DAGs of $\alpha$ w.r.t.~$D$ and $\dep$. Before introducing minimal-depth proof DAGs, let us establish a key property of $\mgd{\alpha}{D}{\dep}$, which will play a crucial role in our complexity analysis. 




\begin{proposition}\label{pro:depth-ptime}
	Consider a Datalog program $\dep$, a database $D$ over $\esch{\dep}$, and a fact $\alpha \in \dep(D)$. The integer $\mgd{\alpha}{D}{\dep}$ can be computed in polynomial time in $|D|$.
\end{proposition}

\begin{proof}
	The proof relies on the well-known immediate consequence operator for Datalog. Roughly, the operator constructs in different steps all the facts that can be derived starting from $D$ and applying the rules of $\dep$. In particular, each fact is constructed ``as early as possible'', and we are going to show that the step at which a fact $\alpha$ is first obtained coincides with $\mgd{\alpha}{D}{\dep}$.
	
	A fact $R(\bar t)$ is an {\em immediate consequence of $D$ and $\dep$} if
	\begin{itemize}
		\item $R(\bar t) \in D$, or
		\item there exists a Datalog rule $R(\bar x)\,\assign\, R_1(\bar x_1),\ldots,R_n(\bar x_n)$ in $\dep$ and a function $h : \bigcup_{i \in [n]} \bar x_i \ra \ins{C}$ such that $\{R_1(h(\bar x_1)),\ldots,R_n(h(\bar x_n))\} \subseteq D$ and $h(\bar x) = \bar t$.
	\end{itemize}
	The immediate consequence operator for $\dep$ is defined as the function $T_\dep$ from the set $S$ of databases over $\sch{\dep}$ to $S$
	\begin{multline*}
	T_\dep(D)\ =\ \{R(\bar t) \mid R(\bar t) \text{ is an immediate}\\
	\text{consequence of } D \text{ and } \dep\}.
	\end{multline*}
	We then define
	\[
	T_\dep^0(D)\ =\ D, 
	\]
	and for each $i>0$, 
	\[
	T_\dep^i(D)\ =\ T_\dep( T_\dep^{i-1}(D)).
	\]
	Finally, we define
	\[
	T_\dep^\infty(D)\ =\ \bigcup_{i \ge 0} T_\dep^i(D).
	\]
	%It is well-known that for every $k$-ary query $Q=(\dep,P)$, database $D$ over $\esch{\dep}$, and tuple $\bar t \in \adom{D}^k$, $\bar t \in Q(D)$ iff $P(\bar t) \in T_{\dep}^\infty(D)$.
	%
	It is not difficult to see that 
	\[
	T_\dep^\infty(D)\ =\ T_\dep^{|\base{D,\dep}|}(D),
	\]
	which in turn implies that $T_\dep(D)$ can be computed in polynomial time in the size of $D$; see, e.g.,~\cite{DEGV01}. Hence, for each $i \ge 0$, $T_\dep^i(D)$ can be computed in polynomial time in the size of $D$.
	%
	Note that $T_\dep^\infty(D) = \dep(D)$~\cite{AbHV95}.
	%
	%Moreover, for every program $\dep$ and database $D$ over $\esch{\dep}$, there exists $n \ge 0$ that is a polynomial w.r.t.\ $D$, and it is such that $T_\dep^\infty(D) = T_\dep^n(D)$, since the number of facts in $\base{D,\dep}$ is at most polynomial w.r.t.\ $D$. Hence, for each $i \ge 0$, $T_\dep^i(D)$ can be computed in polynomial time w.r.t.\ $D$.
	
	The above discussion, together with the following auxiliary lemmas, will prove our claim. For a fact $\alpha \in T_\dep^\infty(D)$, we write $\rank{\alpha}{D,\dep}$ for the integer $\min\{i \mid \alpha \in T_\dep^i(D)\}$.
	
	\begin{lemma}\label{lem:rank-leq}
		%Consider a program $\dep$, and a database $D$ over $\esch{\dep}$. For every $n \ge 0$, and every $R(\bar t) \in T_\dep^\infty(D)$, $\mingdepth{D}{(\dep,R)}{\bar t}=n$ iff $\rank{R(\bar t)}{D,\dep} = n$.
		For every fact $\alpha \in T_\dep^\infty(D)$, $\rank{\alpha}{D,\dep} = \mgd{\alpha}{D}{\dep}$.
	\end{lemma}

	\begin{proof}
		Consider an arbitrary fact $\alpha \in T_\dep^\infty(D)$ and let $n = \mgd{\alpha}{D}{\dep}$. We proceed by induction on $n$.
		
		\medskip
		\noindent \textbf{Base Case.} For $n = 0$, the claim follows immediately by the definition of proof DAG and of $T_\dep$.
		
		\medskip
		\noindent \textbf{Inductive Step.} There is a proof DAG $G=(V,E,\lambda)$ of $\alpha$ w.r.t.~$D$ and $\dep$ with $\depth{G} = n$. Let $u_1,\ldots,u_k$ be the out-neighbours of the root node $v$, and let $\lambda(u_i) = R_i(\bar u_i)$, for each $i \in [k]$. Since $\depth{G} =n$, for each $i \in [k]$, $\mgd{R_i(\bar u_i)}{D}{\dep} \le \depth{G_i} < n$, with $G_i$ being the subDAG of $G$ rooted at $u_i$. By inductive hypothesis, $\rank{R_i(\bar u_i)}{D,\dep} \leq \mgd{R_i(\bar u_i)}{D}{\dep}$, for each $i \in [k]$, and thus, $\{R_1(\bar u_1),\ldots,R_k(\bar u_k)\} \subseteq T_\dep^{n-1}(D)$. Moreover, by the definition of proof DAG, there exist a rule $R_0(\bar x_0) \assign R_1(\bar x_1),\ldots,R_k(\bar x_k)$ in $\dep$ and a function $h : \bigcup_{i \in [n]} \bar x_i \ra \ins{C}$ such that $\alpha = R_0(h(\bar x_0))$ and $h(\bar x_i) = \bar u_i$ for each $i \in [k]$. Consequently, $\rank{\alpha}{D,\dep} \leq \max_{i \in [k]} \{\rank{R_i(\bar u_i)}{D,\dep}\} + 1 \leq n$.
		%
		We further observe that $\alpha \not\in T_{\dep}^{n-1}(D)$ since, otherwise, by induction hypothesis we can conclude that $\mgd{\alpha}{D}{\dep} < n$, which is a contradiction. Hence, $\rank{\alpha}{D,\dep} = n$.
		%
		%$\alpha \in T_\dep^n(D)$. Observe that $\alpha \not\in T_\dep^{n-1}(D)$; otherwise, $\mgd{\alpha}{D}{\dep} < n$, which is a contradiction. Therefore, $\rank{\alpha}{D,\dep} \leq n$, as needed.
	\end{proof}
	
	\begin{comment}
	\begin{lemma}\label{lem:rank-geq}
		For every fact $\alpha \in T_\dep^\infty(D)$, $\rank{\alpha}{D,\dep} \geq \mgd{\alpha}{D}{\dep}$.
	\end{lemma}

	\begin{proof}
		Consider an arbitrary fact $\alpha \in T_\dep^\infty(D)$ and let $n = \rank{\alpha}{D,\dep}$. We proceed by induction on $n$.
		
		\medskip
		\noindent \textbf{Base Case.} For $n = 0$, the claim follows immediately by the definition of proof DAG and of $T_\dep$.
		
		\medskip
		\noindent \textbf{Inductive Step.} Assume that $\rank{\alpha}{D,\dep} = n$. There exist a rule $R_0(\bar x_0) \assign R_1(\bar x_1),\ldots,R_k(\bar x_k)$ in $\dep$ and a function $h : \bigcup_{i \in [n] \bar x_i} \ra \ins{C}$ such that $\alpha = R_0(h(\bar x_0))$ and $\{R_1(h(\bar x_1)),\ldots,R_k(h(\bar x_k))\} \subseteq T_\dep^{n-1}(D)$. Hence, $\rank{R_i(h(\bar x_i))}{D,\dep} < n$, for each $i \in [k]$.
		%
		Clearly, by inductive hypothesis, $\mgd{R_i(h(\bar x_i))}{D}{\dep} \leq \rank{R_i(h(\bar x_i))}{D,\dep}$, for each $i \in [k]$.
		%
		Let $G_i$ be a proof DAG of $R_i(h(\bar x_i))$ w.r.t.~$D$ and $\dep$ with $\depth{G_i} = \mgd{R_i(h(\bar x_i))}{D}{\dep}$, for each $i \in [k]$. Consider now the DAG $G$ obtained by adding to the joint union of $G_1,\ldots,G_k$ a node $v$, which is the root of $G$, labeled by $\alpha$ and its only out-neighbours are the roots of $G_1,\ldots,G_k$. It is easy to see that $G$ is a proof DAG of $\alpha$ w.r.t.~$D$ and $\dep$ with $\depth{G} = \max_{i \in [k]} \{\depth{G_i}\} + 1 \leq n$. Therefore, $\mgd{\alpha}{D}{\dep} \leq n$, as needed.
	\end{proof}
	\end{comment}
	
	The claim follows by Lemma~\ref{lem:rank-leq}.
\end{proof}


The central notion of minimal-depth proof DAG follows:

\begin{definition}[\textbf{Minimal-Depth Proof DAG}]\label{def:minimal-depth-proof-dag}
	Consider a Datalog program $\dep$, a database $D$ over $\esch{\dep}$, and a fact $\alpha$ over $\sch{\dep}$. 
	%
	A {\em minimal-depth proof DAG of $\alpha$ w.r.t.~$D$ and $\dep$} is a proof DAG $G$ of $\alpha$ w.r.t.~$D$ and $\dep$ such that $\depth{G}$ coincides with $\mgd{\alpha}{D}{\dep}$. \hfill\markfull
\end{definition}


The analogous result to Proposition~\ref{pro:characterization-nr-trees} follows:



\begin{proposition}\label{pro:characterization-md-trees}
	For a Datalog program $\dep$, there is a polynomial $f$ such that, for every database $D$ over $\esch{\dep}$, fact $\alpha$ over $\sch{\dep}$, and $D' \subseteq D$, the following are equivalent:
	\begin{enumerate}
		\item There is a minimal-depth proof tree $T$ of $\alpha$ w.r.t.~$D$ and $\dep$ such that $\support{T} = D'$.
		\item There is a minimal-depth proof DAG $G = (V,E,\lambda)$ of $\alpha$ w.r.t.~$D$ and $\dep$ with $\support{G} = D'$ and $|V| \leq f(|D|)$.
	\end{enumerate}
\end{proposition}



The direction (2) implies (1) is shown by ``unravelling'' the minimal-depth proof DAG $G$ into a minimal-depth proof tree $T$ such that $\support{G} = \support{T}$.
We use the same ``unravelling'' construction as in the proof of direction (2) implies (1) of Proposition~\ref{pro:characterization-all-trees}, which {\em preserves the minimality of the depth}.
%
We now proceed with the direction (1) implies (2). The underlying construction proceeds in two main steps captured by Lemmas~\ref{lem:scount-reduction-md} and~\ref{lem:from-trees-to-dags-md} given below.



\medskip 

$\bullet$ The \textbf{\textit{first step}} is to show that a minimal-depth proof tree $T$ of $\alpha$ w.r.t.~$D$ and $\dep$ with $\support{T} = D'$ can be converted into a minimal-depth proof tree $T'$ of $\alpha$ w.r.t.~$D$ and $\dep$ with $\support{T'} = D'$ that has ``small'' subtree count.


\begin{lemma}\label{lem:scount-reduction-md}
	For each Datalog program $\dep$, there is a polynomial $f$ such that, for every database $D$ over $\esch{\dep}$, fact $\alpha$ over $\sch{\dep}$, and $D' \subseteq D$, if there is a minimal-depth proof tree $T$ of $\alpha$ w.r.t.~$D$ and $\dep$ with $\support{T} = D'$, then there is also such a proof tree $T'$ with $\scount{T'} \leq f(|D|)$.
\end{lemma}

\begin{proof}
	We first argue that the proof tree $T$ has ``small'' depth. In particular, by Lemma~\ref{lem:depth-reduction}, there exists a polynomial $f$ and a proof tree $T'$ of $\alpha$ w.r.t.~$D$ and $\dep$ with $\depth{T'} \leq f(|D|)$. Since, by hypothesis, $T$ is of minimal-depth, we conclude that $\depth{T} \leq f(|D|)$.
	%
	The other crucial observation is that the construction underlying Lemma~\ref{lem:scount-reduction}, which converts a proof tree of ``small'' depth into a proof tree of ``small`` subtree count with the same support {\em preserves the minimality of the depth}. It can be verified that the proof tree $T'$ obtained by applying on $T$ the construction underlying Lemma~\ref{lem:scount-reduction} is such that $\depth{T'} \leq \depth{T}$. Hence, since $T$ is a minimal-depth proof tree, then so is $T'$.
	%
	Consequently, we can apply the construction of Lemma~\ref{lem:scount-reduction} to the minimal-depth proof tree $T$ and get a minimal-depth proof tree $T'$ with $\support{T} = \support{T'}$ such that $\scount{T'} \leq f(|D|)$, where $f$ is the polynomial provided by Lemma~\ref{lem:scount-reduction}.
\end{proof}


\smallskip

$\bullet$ The \textbf{\textit{second step}} shows that a minimal-depth proof tree $T$ of $\alpha$ w.r.t.~$D$ and $\dep$ with $\support{T} = D'$ of ``small'' subtree count can be converted into a compact minimal-depth proof DAG $G$ of $\alpha$ w.r.t.~$D$ and $\dep$ with $\support{G} = D'$.

\begin{lemma}\label{lem:from-trees-to-dags-md}
	For each Datalog program $\dep$ and a polynomial $f$, there is a polynomial $g$ such that, for every database $D$ over $\esch{\dep}$, fact $\alpha$, and $D' \subseteq D$, if there is a minimal-depth proof tree $T$ of $\alpha$ w.r.t.~$D$ and $\dep$ with $\support{T} = D'$ and $\scount{T} \leq f(|D|)$, then there is a minimal-depth proof DAG $G = (V,E,\lambda)$ of $\alpha$ w.r.t.~$D$ and $\dep$ with $\support{G} = D'$ and $|V| \leq g(|D|)$.
\end{lemma}

\begin{proof}
	We employ the construction underlying Lemma~\ref{lem:from-trees-to-dags}, which converts a proof tree of ``small'' subtree count into a non-recursive proof DAG of polynomial size with the same support, since it {\em preserves the minimality of the depth}. 
	%The latter holds since, for each path of the proof tree, there is a path in the proof DAG with the same labels, and vice versa. 
\end{proof}


It is now clear that the direction (1) implies (2) of Proposition~\ref{pro:characterization-md-trees} is an immediate consequence of Lemmas~\ref{lem:scount-reduction-md} and~\ref{lem:from-trees-to-dags-md}.


\medskip
\noindent
\textbf{Finalize the Proof.} We can now finalize the proof of the claim that $\mathsf{Why\text {-}Provenance_{MD}[\DAT]}$ is in \NP~in data complexity.
%
Fix a Datalog query $Q = (\dep,R)$. Given a database $D$ over $\esch{\dep}$, a tuple $\bar t \in \adom{D}^{\arity{R}}$, and a subset $D'$ of $D$, to decide whether $D' \in \mdwhy{\bar t}{D}{Q}$ we simply need to check for the existence of a minimal-depth proof tree $T$ of $R(\bar t)$ w.r.t.~$D$ and $\dep$ such that $\support{T} = D'$. By Proposition~\ref{pro:characterization-md-trees}, this is tantamount to the existence of a polynomially-sized minimal-depth proof DAG $G$ of $R(\bar t)$ w.r.t.~$D$ and $\dep$ with $\support{G} = D'$.
%
The existence of such a non-recursive proof DAG can be checked via a non-deterministic algorithm that runs in polynomial time in the size of the database as it was done for proving that $\mathsf{Why\text {-}Provenance[\DAT]}$ is in \NP~in data complexity (Theorem~\ref{the:recursive-complexity}). The only difference is that now we need to additionally check that the guessed DAG $G$ is of minimal-depth, i.e., that $\depth{G}$ coincides with $\mgd{R(\bar t)}{D}{\dep}$.
%
It remains to argue that the latter can be done in polynomial time. The fact that $\mgd{R(\bar t)}{D}{\dep}$ can be computed in polynomial time follows from Proposition~\ref{pro:depth-ptime}. Now, $\depth{G}$ can be computed by converting $G$ into an edge-weighted graph $G'$ by assigning weight $-1$ to each edge, and then running Dijkstra's polynomial-time algorithm for finding the smallest weight of a path between two nodes.\footnote{We recall that Dijkstra's algorithm is correct on graphs with \emph{arbitrary} integer edge labels, only when the input graph is a DAG. Indeed, for general graphs, computing the length of the longest path between two nodes is $\NP$-hard.}
%
Consequently, $\mathsf{Why\text {-}Provenance_{MD}}[Q]$ is in \NP, and thus, $\mathsf{Why\text {-}Provenance_{MD}[\DAT]}$ is in \NP~in data complexity.


\medskip

\noindent \underline{\textbf{Lower Bound}}
\smallskip

\noindent We proceed to establish that $\mathsf{Why\text {-}Provenance_{MD}[\LDAT]}$ is \NP-hard in data complexity. To this ends, we need to show that there exists a linear Datalog query $Q$ such that the problem $\mathsf{Why\text {-}Provenance_{MD}}[Q]$ is \NP-hard. The proof is via a reduction from $\mathsf{3SAT}$, which takes as input a Boolean formula $\varphi = C_1 \wedge \ldots \wedge C_m$ in 3CNF, where each clause has exactly 3 literals (a Boolean variable $v$ or its negation $\neg v$), and asks whether $\varphi$ is satisfiable.


\medskip
\noindent \textbf{The Linear Datalog Query.}
We start by defining the linear Datalog query $Q = (\dep,R)$. We actually adapt the query $Q$ used in the proof of the fact that $\mathsf{Why\text {-}Provenance[\LDAT]}$ is \NP-hard in data complexity (see Theorem~\ref{the:recursive-complexity}) in such a way that every proof tree has the same depth.
%
As usual, we use $\_$ if the name of a variable is not important, and semicolons in a tuple expression in order to separate terms with a different semantic meaning. The program $\dep$ follows:
\begin{eqnarray*}
\sigma_1 &:& R(x)\,\, \assign \,\,{\rm Var}(x;y,\_;z),{\rm Assign}(x,y,z), \\
\sigma_2 &:& R(x)\,\, \assign \,\, {\rm Var}(x;\_,y;z),{\rm Assign}(x,y,z),\\
\sigma_3 &:& {\rm Assign}(x,y,z)\,\, \assign \,\, {\rm NextC}(x;z,w;k,\ell),\\
&& \hspace{10mm}  C(x,y;\_,\_;\_,\_;z,w;k,\ell), {\rm Assign}(x,y,w),\\
\sigma_4 &:& {\rm Assign}(x,y,z) \,\, \assign \,\, {\rm NextC}(x;z,w;k,\ell),\\
&& \hspace{10mm} C(\_,\_;x,y;\_,\_;z,w;k,\ell),{\rm Assign}(x,y,w), 
\end{eqnarray*}
\begin{eqnarray*}
\sigma_5 &:& {\rm Assign}(x,y,z) \,\, \assign \,\, {\rm NextC}(x;z,w;k,\ell),\\
&& \hspace{10mm} C(\_,\_;\_,\_;x,y;z,w;k,\ell), {\rm Assign}(x,y,w), \\
\sigma' &:& {\rm Assign}(x,y,z) \,\, \assign \,\, {\rm NextC}(x;z,w;y,\_),\\
&& \hspace{45mm}{\rm Assign}(x,y,w), \\
\sigma'' &:& {\rm Assign}(x,y,z) \,\, \assign \,\, {\rm NextC}(x;z,w;\_,y),\\
&& \hspace{45mm}{\rm Assign}(x,y,w), \\
\sigma_6 &:& {\rm Assign}(x,z,w) \,\, \assign \,\, {\rm Next}(x,y;z,\_;w),R(y), \\
\sigma_7 &:& {\rm Assign}(x,z,w) \,\, \assign \,\, {\rm Next}(x,y;\_,z;w),P(y), \\
\sigma_8 &:& R(x) \,\, \assign \,\, {\rm Last}(x).
\end{eqnarray*}
The key difference compared to the Datalog program used in the proof of Theorem~\ref{the:recursive-complexity} is that now, roughly speaking, the rules $\sigma_1, \sigma_2$ also attach the id of the first clause to the variable assignment; the last position of the relation ${\rm Var}$ stores the id of the first clause. The rules $\sigma_3,\sigma_4,\sigma_5,\sigma',\sigma''$, where $\sigma_3$, $\sigma_4$ and $\sigma_5$ are adapted from the previous proof, whereas $\sigma'$, $\sigma''$ are new, force a subtree to always to perform $m$ ``steps'', when going through those rules; {\rm NextC} provides an ordering of the clauses of the formula.
%
It is easy to verify that $\dep$ is indeed a linear Datalog program.
%



\medskip
\noindent \textbf{From $\mathsf{3SAT}$ to  $\mathsf{Why\text {-}Provenance_{MD}}[Q]$.} We now establish that $\mathsf{Why\text {-}Provenance_{MD}}[Q]$ is \NP-hard by reducing from $\mathsf{3SAT}$.
%
Consider a 3CNF Boolean formula $\varphi = C_1 \wedge \cdots \wedge C_m$ with $n$ Boolean variables $v_1,\ldots,v_n$. For a literal $\ell$, we write $\lvar{\ell}$ for the variable occurring in $\ell$, and $\lsign{\ell}$ for the number $1$ (resp., $0$) if $\ell$ is a variable (resp., the negation of a variable).
%
We define $D_\varphi$ as the database over $\esch{\dep}$
\begin{eqnarray*}
&& \{{\rm Var}(v_i;0,1;1) \mid i \in [n]\}\\
&\cup& \{{\rm Next}(v_i,v_{i+1};0,1;m+1) \mid i \in [n-1]\}\\
&\cup& \{{\rm Next}(v_n,\bullet;0,1;m+1), {\rm Last}(\bullet)\}\\
&\cup& \{C(\lvar{\ell_1},\lsign{\ell_1};\lvar{\ell_2},\lsign{\ell_2};\lvar{\ell_3},\lsign{\ell_3};i,i+1;0,1) \mid \\
&& \hspace{20mm} i \in [m] \text{ with } C_i = (\ell_1 \vee \ell_2 \vee \ell_3)\}\\
&\cup& \{{\rm NextC}(v_i;j,j+1;0,1) \mid i \in [n]\ \text{ and } j \in [m]\}.
\end{eqnarray*}
%
We can show the next lemma, which essentially states that the above construction leads to a correct polynomial-time reduction from $\mathsf{3SAT}$ to $\mathsf{Why\text {-}Provenance}[Q]$:

\begin{lemma}\label{lem:reduction-from-3sat-md}
	The following hold:
	\begin{enumerate}
		\item $D_\varphi$ can be constructed in polynomial time in $\varphi$. 
		\item $\varphi$ is satisfiable iff $D_\varphi \in \why{(v_1)}{D_\varphi}{Q}$.
	\end{enumerate}
\end{lemma}

\begin{proof}
	It is straightforward to see that $D_\varphi$ can be constructed in polynomial time in the size of $\varphi$. We proceed to establish item (2). But let us first state and prove an auxiliary technical lemma, which essentially states that all proof trees of $R(v_1)$ w.r.t.~$D_\varphi$ and $\dep$ have the same depth. 
	
	\begin{lemma}\label{lem:same-depth}
		For every proof tree $T$ of $R(v_1)$ w.r.t.~$D_\varphi$ and $\dep$, $\depth{T} = n \cdot (m+2) + 1$.
	\end{lemma}

	\begin{proof}
		A proof tree of $R(v_1)$ w.r.t.~$D_\varphi$ and $\dep$ has a root node $v$ labeled with $R(v_1)$, and thus, its children are necessarily labeled with facts obtained by means of either rule $\sigma_1$ or $\sigma_2$. From these children, only one, say $u$, is labeled with an intensional fact, which is of the form ${\rm Assign}(v_1,\cdot,1)$ (the constant $1$ is due to the variable $z$ in the body of rules $\sigma_1,\sigma_2$). Then, the children of $u$ are necessarily labeled with facts obtained via one of $\sigma_3,\sigma_4,\sigma_5,\sigma',\sigma''$ (this is because of the constant $1$ in the fact labeling $u$). Moreover, due to the presence of the atom ${\rm NextC}(x;z,w;k,\ell)$ in the body of each such rule, only one child of $u$, say $u'$, is labeled with an intensional fact of the form ${\rm Assign}(v_1,\cdot,2)$. Applying the same reasoning, from this node with label ${\rm Assign}(v_1,\cdot,2)$, one of the rules $\sigma_3,\sigma_4,\sigma_5,\sigma',\sigma''$ will be used, and the id on the third position will be increased again, until it will coincide with $m+1$. Up to this point, the longest path in the tree from the root is of length $m$+1. It is not difficult to see now that the only rule that can be applied is either $\sigma_6$ or $\sigma_7$ due to the variable $w$ appearing in the atom ${\rm Next}(x,y;z,\_;w)$. After one of such rules is applied the only node labeled with an intensional fact has as label the fact $R(v_2)$, if $n>1$, and $R(\bullet)$ otherwise, and the total depth is $m+1+1 = m+2$. Thus, either $\sigma_1$ will then be used again, or $\sigma_8$. By applying the above reasoning for all the variables of $\varphi$, the total depth of a proof tree of $R(v_1)$ w.r.t.~$D_\varphi$ and $\dep$ is precisely $n \cdot (m+2) + 1$, as needed.
	\end{proof}

	With Lemma~\ref{lem:same-depth} in place, to prove our claim it suffices to show that $\varphi$ is satisfiable iff there is a proof tree $T$ (regardless of its depth) of $R(v_1)$ w.r.t.~$D_\varphi$ and $\dep$ such that $\support{T} = D_\varphi$. To prove this last claim, it is not difficult to see how one can adapt the proof given for Theorem~\ref{the:recursive-complexity}. The main difference is that we additionally need to argue that when a proof tree of $R(v_1)$ uses the rules $\sigma_3,\sigma_4,\sigma_5,\sigma',\sigma''$, and thus, we have a node in the proof tree labeled with a fact of the form ${\rm Assign}(v_i,\mathsf{val},j)$, if the truth value $\mathsf{val}$ chosen for $v_i$ does not make the clause $C_j$ true, then the proof tree will follow $\sigma'$ or $\sigma''$, depending on the value of $\mathsf{val}$. Hence, in any case, through rules $\sigma_3,\sigma_4,\sigma_5,\sigma',\sigma''$, any proof tree will be able to touch all facts with predicate ${\rm NextC}$ in the database. Since these are the only new facts w.r.t.~the ones used in the proof of Theorem~\ref{the:recursive-complexity} (all other facts have been only slightly modified), if $\varphi$ is satisfiable, then we have a proof tree $T$ of $R(v_1)$ w.r.t.~$D_\varphi$ and $\dep$ with the property that $\support{T} = D_\varphi$, and vice versa.
\end{proof}


By Lemma~\ref{lem:reduction-from-3sat-md}, $\mathsf{Why\text {-}Provenance_{MD}}[Q]$ is \NP-hard. Thus, $\mathsf{Why\text {-}Provenance_{MD}[\LDAT]}$ is \NP-hard in data complexity.


\subsection{Non-Recursive Queries}
%

We now focus on non-recursive Datalog queries, and show the following about the data complexity of why-provenance relative to minimal-depth proof trees:

\begin{theorem}\label{the:non-recursive-complexity-md}
	$\mathsf{Why\text {-}Provenance_{MD}[\NRDAT]}$ is in $\ACZ$ in data complexity.
\end{theorem}



This is shown via FO rewritability as done for Theorem~\ref{the:non-recursive-complexity}. Let us stress, however, that the target FO query cannot be obtained by simply refining the space of proof trees underlying the definition of $\cq{Q}$, for a Datalog query $Q$, as done in the case of non-recursive proof trees (see Theorem~\ref{the:non-recursive-complexity-nr}). The key reason is that the notion of minimal-depth is not over all proof trees, but over the proof trees of a certain fact. This dependency on the
fact (and indirectly, on the database) does not allow us to simply consider a refined family of proof trees as we did
for non-recursive proof-trees, and a slightly more involved construction is needed. Actually, we are going to adapt the FO query underlying Theorem~\ref{the:non-recursive-complexity}.

\medskip

\noindent \textbf{First-Order Rewriting.} Consider a non-recursive Datalog query $Q = (\dep,R)$. Recall that in the proof of Theorem~\ref{the:non-recursive-complexity}, for a CQ $\varphi(\bar y) \in \cqeq{Q}$, we defined an FO query $Q_{\varphi(\bar y)} = \psi_{\varphi(\bar y)}(x_1,\ldots,x_{\arity{R}})$, where $x_1,\ldots,x_{\arity{R}}$ are distinct variables that do not occur in any of the CQs of $\cqeq{Q}$, with the following property: for every database $D$ and tuple $\bar t \in \adom{D}^{\arity{R}}$, $\bar t \in Q_{\varphi(\bar y)}(D)$ iff $\bar t$ is an answer to $\varphi(\bar y)$ over $D$, and, in addition, {\em all} the atoms of $D$ are used in order to entail the sentence $\varphi[\bar y/\bar t]$, i.e., there are no other facts in $D$ besides the ones that have been used as witnesses for the atoms occurring in $\varphi[\bar y/\bar t]$.
%
We are going to extend $Q_{\varphi(\bar y)}$ into $Q^+_{\varphi(\bar y)} = \psi^+_{\varphi(\bar y)}(x_1,\ldots,x_{\arity{R}})$ that, in addition, performs the minimal depth check.
%
Recall that, with $\varphi$ being of the form $\exists \bar z \, (R_1(\bar w_1) \wedge \cdots \wedge R_n(\bar w_n))$, the formula $\psi_{\varphi(\bar y)}$, with free variables $\bar x = (x_1,\ldots,x_{\arity{R}})$, is of the form
\[
\exists \bar y \exists \bar z \left(\varphi_1\ \wedge\ \varphi_2\ \wedge\ \varphi_3\right),
\]
where $\varphi_1$ is defined as
\[
\bigwedge\limits_{i \in [n]}\, R_i(\bar w_i)\ \wedge\ (\bar x = \bar y)\ \wedge\ \bigwedge\limits_{\substack{u,v \in \var{\varphi}, \\ u \neq v}} \neg (u = v)
\]
$\varphi_2$ is defined as
\[
\bigwedge\limits_{P \in \{R_1,\ldots,R_n\}} \neg \left(\exists \bar u_{P} \left(P(\bar u_P)\ \wedge\ \bigwedge\limits_{\substack{i \in [n], \\ R_i = P}}\, \neg(\bar w_i = \bar u_P)\right)\right)
\]
and $\varphi_3$ is defined as
\[
\bigwedge\limits_{P \in \esch{\dep} \setminus \{R_1,\ldots,R_n\}} \neg \left(\exists \bar u_P \, P(\bar u_P)\right).
\]
Now, the formula $\psi^+_{\varphi(\bar y)}$ is
\[
\exists \bar y \exists \bar z \left(\varphi_1\ \wedge\ \varphi_2\ \wedge\ \varphi_3\ \wedge\ \varphi_4\right),
\]
where the additional conjunct $\varphi_4$ is defined as follows. For a CQ $\chi(\bar s) \in \cqeq{Q}$, we write $\depth{\chi(\bar s),Q}$ for the integer
\[
\min \{\depth{T} \mid T \text{ is a } Q\text{-tree with } \chi(\bar s) \eqtree \cq{T}\}
\]
that is, the smallest depth among all $Q$-trees whose induced CQ is isomorphic to $\chi(\bar s)$. Now, the formula $\varphi_4$ is
\begin{multline*}
	\bigwedge\limits_{\substack{\xi(\bar s) \in \cqeq{Q} \\ \text{with } \xi = \exists \bar u (P_1(\bar v_1),\ldots,P_m(\bar v_m)), \\ \depth{\xi(\bar s),Q} < \depth{\varphi(\bar y),Q}}} \neg \exists \bar s \exists \bar u \bigg(
	\bigwedge\limits_{i \in [m]}\, P_i(\bar v_i)\ \wedge\\
	(\bar x = \bar s)\ \wedge\ \bigwedge\limits_{\substack{u,v \in \var{\varphi}, \\ u \neq v}} \neg (u = v)\bigg)
\end{multline*}
which states that there is no other CQ $\xi(\bar s) \in \cqeq{Q}$ whose depth $d$ is smaller than the one of $\varphi(\bar y)$, and it witnesses the existence of a proof tree of depth $d$ of a fact $R(\bar t)$ w.r.t.~$D$ and $\dep$, where $\bar t$ is the tuple witnessed by $(x_1,\ldots,x_{\arity{R}})$.

\begin{comment}
\begin{multline*}
\exists \bar y  \exists \bar z\bigg(\bigwedge\limits_{i \in [n]}\, R_i(\bar w_i)\ \wedge\ (\bar x = \bar y)\ \wedge\ \bigwedge\limits_{\substack{u,v \in \var{\varphi}, \\ u \neq v}} \neg (u = v)\ \wedge\\
%
\bigwedge\limits_{P \in \{R_1,\ldots,R_n\}} \neg \left(\exists \bar u_{P} \bigwedge\limits_{\substack{i \in [n], \\ R_i = P}}\, \left(\neg(\bar w_i = \bar u_P)\ \wedge\ P(\bar u_P)\right)\right)\ \wedge \\
\bigwedge\limits_{P \in \esch{\dep} \setminus \{R_1,\ldots,R_n\}} \neg \left(\exists \bar u_P \, P(\bar u_P)\right)\bigg),
\end{multline*}
where, for two tuples of variables $\bar u = (u_1,\ldots,u_k)$ and $\bar v = (v_1,\ldots,v_k)$, $(\bar u = \bar v)$ is $\bigwedge_{i = 1}^{k} (u_i = v_i)$.
\end{comment}


With the FO query $Q^+_{\varphi(\bar y)}$ for each CQ $\varphi(\bar y) \in \cqeq{Q}$ in place, it should be clear that the desired FO query $Q^+_{\mi{FO}}$ is defined as $\Phi^+(x_1\ldots,x_{\arity{R}})$, where
\[
\Phi^+\ =\ \bigvee_{\varphi(\bar y) \in \cqeq{Q}} \psi^+_{\varphi(\bar y)}.
\]
We proceed to show the correctness of the construction.



\begin{lemma}\label{lem:fo-tree-equiv-md}
	Given a non-recursive Datalog query $Q=(\dep,R)$, a database $D$ over $\esch{\dep}$, $\bar t \in \adom{D}^{\arity{R}}$, and $D' \subseteq D$, $D' \in \mdwhy{\bar t}{D}{Q}$ iff $\bar t \in Q^+_{\mi{FO}}(D')$.
\end{lemma}

\begin{proof}
	We start with the $(\Rightarrow)$ direction. By hypothesis, there is a minimal-depth proof tree $T$ of $R(\bar t)$ w.r.t.~$D$ and $\dep$ such that $\support{T} = D'$.
	%
	By Lemma~\ref{lem:fo-tree-equiv}, we get that the CQ $\varphi(\bar y) \in \cqeq{Q}$ induced by $T$ is such that $\bar t \in Q_{\varphi(\bar y)}(D')$. Thus, to prove that $\bar t \in Q^+_{\varphi(\bar y)}(D')$, it suffices to show that $\bar t \in Q'(D')$, where $Q' = \varphi_4(x_1,\ldots,x_{\arity{R}})$ with $\varphi_4$ being
	\begin{multline*}
	\bigwedge\limits_{\substack{\xi(\bar s) \in \cqeq{Q} \\ \text{with } \xi = \exists \bar u (P_1(\bar v_1),\ldots,P_m(\bar v_m)), \\ \depth{\xi(\bar s),Q} < \depth{\varphi(\bar y),Q}}} \neg \exists \bar s \exists \bar u \bigg(
	\bigwedge\limits_{i \in [m]}\, P_i(\bar v_i)\ \wedge\\
	(\bar x = \bar s)\ \wedge\ \bigwedge\limits_{\substack{u,v \in \var{\varphi}, \\ u \neq v}} \neg (u = v)\bigg).
	\end{multline*}
	Towards a contradiction, assume that $\bar t \not\in Q'(D')$. This in turn implies that there exists a CQ $\xi(\bar s) \in \cqeq{Q}$ of the form $\exists \bar u (P_1(\bar v_1),\ldots,P_m(\bar v_m))$ with $\depth{\xi(\bar s),Q} < \depth{\varphi(\bar y),Q}$ such that the sentence
	\[
	\exists \bar s \exists \bar u \bigg(
	\bigwedge\limits_{i \in [m]}\, P_i(\bar v_i)\ \wedge\\
	(\bar t = \bar s)\ \wedge\ \bigwedge\limits_{\substack{u,v \in \var{\varphi}, \\ u \neq v}} \neg (u = v)\bigg)
	\]
	is satisfied by $D'$. This implies that there is an assignment $h$ to the variables in $\xi$ with $h(\bar s) = \bar t$, $h(u) \neq h(v)$, for each $u,v \in \var{\xi}$ with $u \neq v$, and, for each $i \in [m]$, $P_i(h(\bar v_i)) \in D'$.
	% 
	Since $\xi(\bar s)$ is induced by a $Q$-tree $T'$ with depth $d=\depth{\xi(\bar s),Q}$, $T'$ is a proof tree of $R(\bar t)$ w.r.t.~$D$ and $\dep$ of depth $d$ such that 
	\[
	\support{T'} = \{P_1(h(\bar v_1)),\ldots,P_m(h(\bar v_m))\}\ \subseteq\ D'.
	\]
	However, since
	\[
	\depth{T'}\ =\ \depth{\xi(\bar s),Q}\ <\ \depth{\varphi(\bar y),Q}
	\]
	and
	\[
	\depth{\varphi(\bar y),Q}\ \leq\ \depth{T}, 
	\]
	we conclude that $\depth{T'} < \depth{T}$, which contradicts the fact that $T$ is a minimal-depth proof tree.
	
	We now proceed with direction $(\Leftarrow)$. By hypothesis, $\bar t \in Q^+_{\mi{FO}}(D')$. Therefore, there exists a CQ $\varphi(\bar y) \in \cqeq{Q}$ such that $\bar t \in Q^+_{\varphi(\bar y)}(D')$. It is clear that $\bar t \in Q_{\mi{FO}}(D')$, and form the proof of Lemma~\ref{lem:fo-tree-equiv} we get that $\varphi(\bar y)$ is induced by a $Q$-tree $T$ that is also a proof tree of $R(\bar t)$ w.r.t.~$D$ and $\dep$ with $\support{T} = D'$. We assume w.l.o.g. that $\depth{T} = \depth{\varphi(\bar y),Q}$, i.e., $T$ is the proof tree of the smallest depth among those that induce $\varphi(\bar y)$. It remains to show that $T$ is a minimal-depth proof tree. Towards a contradiction, assume that $T$ is not a minimal-depth proof tree. Thus, there exists another proof tree $T'$ for $R(\bar t)$ w.r.t.~$D$ and $\dep$ with $\depth{T'} < \depth{T}$. Let $\xi(\bar s) \in \cqeq{Q}$ be the CQ induced by $T'$. It is clear that
	\[
	\depth{\xi(\bar s),Q}\ \leq\ \depth{T'}\ <\ \depth{\varphi(\bar y),Q}.
	\]
	This allows us to conclude that $\bar t \not\in Q^+_{\varphi(\bar y)}(D')$, which is a contradiction. Consequently, $T$ is a minimal-depth proof tree of $R(\bar t)$ w.r.t.~$D$ and $\dep$. Since $\support{T} = D'$, we get that $D' \in \mdwhy{R(\bar t)}{D}{Q}$, as needed.
\end{proof}
\newcommand{\nodes}[1]{\mathsf{nodes}(#1)}
\newcommand{\edges}[1]{\mathsf{edges}(#1)}
\newcommand{\atomtotuple}[1]{\langle #1 \rangle}
\def\curnode{\mathsf{CurNode}}
\def\hedge{\mathsf{HEdge}}

\section{Unambiguous Proof Trees}\label{appsec:unambiguous-trees}
In this section, we provide proofs for all claims of Section~\ref{sec:unambiguous-trees}, and provide further details on our experimental evaluation.

\subsection{Proof of Theorem~\ref{the:complexity-unambiguous-proof-trees}}
We start by proving Theorem~\ref{the:complexity-unambiguous-proof-trees}, which we recall here for convenience:

\begin{manualtheorem}{\ref{the:complexity-unambiguous-proof-trees}}
	\theunambiguouscomplexity
\end{manualtheorem}
We prove item~(1) and item~(2) of Theorem~\ref{the:complexity-unambiguous-proof-trees} separately. We start by focusing on item~(1).


\medskip
\noindent
\underline{\textbf{Proof of Item~(1)}}
\smallskip

\noindent Our main task is to show that $\mathsf{Why\text {-}Provenance_{UN}[\DAT]}$ is in \NP. The \NP-hardness of $\mathsf{Why\text {-}Provenance_{UN}[\LDAT]}$ follows from the $\NP$-hardness of $\mathsf{Why\text {-}Provenance_{NR}[\LDAT]}$, which we have already shown in Section~\ref{appsec:refined-trees}. The latter follows from the observation that, in the case of linear Datalog programs, non-recursive proof trees and unambiguous proof trees coincide.
%
We now show the \NP~upper bound. 

This result relies on a characterization of the existence of an unambiguous proof tree of a fact $\alpha$ w.r.t.~a database $D$ and a Datalog program $\dep$ with $\support{T} = D' \subseteq D$ via the existence of a so-called {\em unambiguous proof DAG} $G$ of $\alpha$ w.r.t.~$D$ and $\dep$ with $\support{G} = D'$ of polynomial size. This in turn allows us to devise a guess-and-check algorithm that runs in polynomial time We proceed to formalize the above  high-level description.


For a rooted DAG $G=(V,E,\lambda)$ and a node $v \in V$, we use $G[v]$ to denote the subDAG of $G$ rooted at $v$. Moreover, two rooted DAGs $G=(V,E,\lambda)$ and $G'=(V',E',\lambda')$ are \emph{isomorphic}, denoted $G \eqtree G'$, if there is a bijection $h : V \ra V'$ such that , for each node $v \in V$, $\lambda(v) = \lambda(h(v))$, and for each two nodes $u,v \in V$,  $(u,v) \in E$ iff $(h(u),h(v)) \in E'$. With the above definitions  in place, we can now introduce the key notion of unambiguous proof DAG.

\begin{definition}[\textbf{Unambiguous Proof DAG}]\label{def:u-proof-dag}
	Consider a Datalog program $\dep$, a database $D$ over $\esch{\dep}$, and a fact $\alpha$ over $\sch{\dep}$. An \emph{unambiguous proof DAG of $\alpha$ w.r.t.\ $D$ and $\dep$} is a proof DAG $G=(V,E,\lambda)$ of $\alpha$ w.r.t.\ $D$ and $\dep$ such that, for all $v,u \in V$, $\lambda(u)=\lambda(v)$ implies $G[u] \eqtree G[v]$.\hfill\markfull
\end{definition}

We are now ready to present our characterization.

\begin{proposition}\label{pro:characterization-u-trees}
	For a Datalog program $\dep$, there is a polynomial $f$ such that, for every database $D$ over $\esch{\dep}$, fact $\alpha$ over $\sch{\dep}$, and $D' \subseteq D$, the following are equivalent:
	\begin{enumerate}
		\item There exists an unambiguous proof tree $T$ of $\alpha$ w.r.t.\ $D$ and $\dep$ such that $\support{T} = D'$.
		\item There is an unambiguous proof DAG $G=(V,E,\lambda)$ of $\alpha$ w.r.t.\ $D$ and $\dep$ with $\support{G}=D'$ and $|V| \le f(|D|)$.
	\end{enumerate}
\end{proposition}
\begin{proof}
	We first prove (1) implies (2). Let $T$ be an unambiguous proof tree of $\alpha$ w.r.t.~$D$ and $\dep$ with $\support{T} = D'$. 
	%We first observe that the depth of $T$ is polynomial in $|D|$. This holds since no two nodes on a path have the same label $\alpha$ (otherwise, their subtrees belong to two different equivalence classes of $\quot{T[\alpha]}$); hence, the length of a path in $T$ is bounded by the number of labels, that is, by $|\base{D,\dep}|$. 
	By definition, the subtree count of $T$ is ``small''; in fact, for every label $\alpha$ of $T$, $|\quot{T[\alpha]}|=1$. We then employ the construction underlying Lemma~\ref{lem:from-trees-to-dags}, which converts a proof tree of ``small'' subtree count into proof DAG of polynomial size with the same support, since it {\em preserves unambiguity}.
	
	For (2) implies (1), we employ the ``unravelling'' construction used to prove that (2) implies (1) in Proposition~\ref{pro:characterization-all-trees} since it also {\em preserves unambiguity}. 
	%In fact, consider an unambiguous proof DAG $G=(V,E,\lambda)$ of $\alpha$ w.r.t.\ $D$ and $\dep$, such that $\support{G}=D'$. The proof tree $T' =(V',E',\lambda')$ of $\alpha$ w.r.t.\ $D$ and $\dep$, with $\support{T'}=D'$, obtained as the result of unravelling $G$ is such that, for every edge $(u',v') \in E'$, there is an edge $(u,v) \in E$ with $\lambda'(u')=\lambda(u)$ and $\lambda'(v')=\lambda(v)$, and for every edge $(u,v) \in E$, there is an edge $(u',v') \in E'$ such that $\lambda(u) = \lambda'(u')$ and $\lambda(v)=\lambda'(v')$. Hence, if $G$ is unambiguous, $T'$ must necessarily be unambiguous as well.
	%As in the case of arbitrary trees, the fact that (1) implies (2) follows from the combination of these two results (see the proof of Lemma~\ref{lem:from-trees-to-dags}). Indeed, it is easy to verify that the procedure for converting a proof tree into a proof DAG shown in the proof of Lemma~\ref{lem:from-trees-to-dags} preserves unambiguity.
	%This holds since for every $\alpha$, there is a single equivalence class $C$ in $\quot{T[\alpha]}$; hence, by the construction of the DAG, there is a single node $v_C^\alpha$ in the DAG labeled with $\alpha$. Moreover, if for every label $\alpha$, there is a single node in the DAG labeled with $\alpha$, then there is a single subDAG with a root labeled with $\alpha$, and all the subtrees that we obtain from it (and will appear under the nodes labeled with $\alpha$ in the proof tree) will have the same structure and same label (hence, will belong to the same equivalence class).
	%\textcolor{red}{Proof strategy: Here it is actually just the one used all over again in Theorems~\ref{thm:characterization-all-trees},\ref{thm:characterization-nr-trees}, we do not need any additional lemma, since unambiguous proof trees are already of the shape we need.}
\end{proof}


\noindent
\textbf{Finalize the Proof.}
With Proposition~\ref{pro:characterization-u-trees} in place, proving item~(1) of Theorem~\ref{the:complexity-unambiguous-proof-trees} is straightforward. Indeed, we can employ a guess-and-check algorithm similar in spirit to the one employed to prove the $\NP$ upper bound of Theorem~\ref{the:recursive-complexity}. The key difference is that here we also need to verify that the guessed DAG $G$ is unambiguous. This can be easily done by guessing, together with the graph $G$, for every pair $u,v$ of nodes of $G$ with the same label, a bijection $h_{(u,v)}$ from the nodes of $G[u]$ to the nodes of $G[v]$. The number of nodes of $G$ is polynomial w.r.t.\ $|D|$, by Proposition~\ref{pro:characterization-u-trees}, and thus the number of bijections to guess is polynomial w.r.t.\ $|D|$. With the above bijections in place, it is enough to verify that each bijection $h_{(u,v)}$ witnesses that $G[u] \eqtree G[v]$. The latter check can be easily performed in polynomial time.

\medskip
\noindent
\underline{\textbf{Proof of Item~(2)}}
\smallskip

\noindent 
This is shown via first-order rewritability as done for Theorem~\ref{the:non-recursive-complexity}. In fact, the construction of the target FO query is exactly the same as in the proof of Theorem~\ref{the:non-recursive-complexity} with the key difference that, for a Datalog query $Q$, the set of CQs $\cq{Q}$ is defined by considering only unambiguous proof trees, i.e., is the set $\{\cq{T} \mid T \text{ is a {\em unambiguous} $Q$-tree}\}$.

%To prove this item, it is enough to observe that the the same machinery we employed for arbitrary trees can be applied to unambiguous proof trees. The key difference is in the family of trees we consider while collecting the CQs in the set $\cq{Q}$, for a Datalog query $Q$. That is, we define $\cq{Q}$ as the set $\{\cq{T} \mid T \text{ is an unmabiguous } Q\text{-tree}\}$. With the above definition in place, the same arguments used to prove Theorem~\ref{the:non-recursive-complexity} apply.

\subsection{Proof of Proposition~\ref{pro:why-provenance-sat}}
The goal is to prove Propostion~\ref{pro:why-provenance-sat}. But first we need to introduce some auxiliary notions and results, which will then allow us to formally define the Boolean formula $\phi_{(\bar t,D,Q)}$. We will then proceed with the proof of Propostion~\ref{pro:why-provenance-sat}




\medskip
\noindent
\textbf{A More Refined Characterization.} The Boolean formula in question relines on a more refined characterization than the one provided by Proposition~\ref{pro:characterization-u-trees}. For this, we need to define a new kind of graph that witnesses the existence of an unambiguous proof tree.

\begin{definition}[\textbf{Compressed DAG}]\label{def:compressed-dag}
Consider a Datalog program $\dep$, a database $D$ over $\esch{\dep}$, and a fact $\alpha$ over $\sch{\dep}$. A \emph{compressed DAG} of $\alpha$ w.r.t.\ $D$ and $\dep$ is a rooted DAG $G=(V,E)$, with $V \subseteq \base{D,\dep}$, such that:
\begin{enumerate}
	\item The root of $G$ is $\alpha$.
	\item If $\beta \in V$ is a leaf node, then $\beta \in D$.
	\item If $\beta \in V$ has $n \geq 1$ outgoing edges $(\beta,\gamma_1),\ldots,(\beta,\gamma_n)$, then there is a rule $R_0(\bar x_0)\ \assign\ R_1(\bar x_1),\ldots,R_m(\bar x_m) \in \dep$ and a function $h : \bigcup_{i \in [m]} \bar x_i \ra \ins{C}$ such that $\beta = R_0(h(\bar x_0))$, and $\{\gamma_i\}_{i \in [n]} = \{R_i(h(\bar x_i)) \mid i \in [m]\}$. \hfill\markfull
\end{enumerate}
\end{definition}

A compressed DAG can be seen as a proof DAG-like structure where no more than one node is labeled with the same fact.
The above definition allows us to refine the characterization given in Proposition~\ref{pro:characterization-u-trees} as follows; for a non-labeled DAG $G=(V,E)$, with a slight abuse of notation, we denote $\support{G} = \{v \in V \mid v \text{ is a leaf of } G\}$.

%\begin{lemma}\label{lem:from-udags-to-cdags}
%	For a Datalog program $\dep$, there is a polynomial $f$ such that, for every database $D$ over $\esch{\dep}$, fact $\alpha$ over $\sch{\dep}$, and $D' \subseteq D$, the following are equivalent:
%	\begin{enumerate}
%		\item There exists an unambiguous proof DAG $G$ of $\alpha$ w.r.t.\ $D$ and $\dep$ such that $\support{G} = D'$.
%		\item There exists a compressed DAG $G'$ of $\alpha$ w.r.t.\ $D$ and $\dep$, such that $\support{G'}=D'$.
%	\end{enumerate}
%\end{lemma}



\begin{proposition}\label{pro:characterization-u-trees-improved}
	For a Datalog program $\dep$, database $D$ over $\esch{\dep}$, fact $\alpha$ over $\sch{\dep}$, and database $D' \subseteq D$, the following are equivalent:
	\begin{enumerate}
		\item There exists an unambiguous proof tree $T$ of $\alpha$ w.r.t.\ $D$ and $\dep$ such that $\support{T} = D'$.
		\item There exists a compressed DAG $G$ of $\alpha$ w.r.t.\ $D$ and $\dep$, such that $\support{G}=D'$.
	\end{enumerate}
\end{proposition}

\begin{proof}
	We first prove that (1) implies (2). Due to Proposition~\ref{pro:characterization-u-trees}, it suffices to show that if there exists an unambiguous proof DAG $G' = (V',E',\lambda')$ of $\alpha$ w.r.t.~$D$ and $\dep$ such that $\support{G'}=D'$, then there exists a compressed DAG $G=(V,E)$ of $\alpha$ w.r.t.\ $D$ and $\dep$ with $\support{G}=D'$.
	%
	Assume that $G'=(V',E',\lambda')$ is an unambiguous proof DAG of $\alpha$ w.r.t.\ $D$ and $\dep$ such that $\support{G'}=D'$. Since $G'$ is unambiguous, for every two non-leaf nodes $u,v \in V'$ with $\beta = \lambda'(u)=\lambda'(v)$, we must have that $S= \{\lambda'(u_1),\ldots,\lambda'(u_n)\} = \{\lambda'(v_1),\ldots,\lambda'(v_m)\}$, where $u_1,\ldots,u_n$ and $v_1,\ldots,v_m$ are the children of $u$ and $v$ in $G'$, respectively. So, for each fact $\beta$ labeling a non-leaf node in $G'$, let us call the above (unique) set $S$ the \emph{justification of $\beta$ in $G'$}. Hence, constructing a compressed DAG $G=(V,E)$ for $\alpha$ w.r.t.\ $D$ and $\dep$, with $\support{G}=D'$ is straightforward. That is, the root of $G$ is $\alpha$, and if a node $\beta \in V$, letting $S=\{\gamma_1,\ldots,\gamma_n\}$ be its justification in $G'$, $G$ has nodes $\gamma_1,\ldots,\gamma_n$, and edges $(\beta,\gamma_1),\ldots,(\beta,\gamma_n)$.
	
	For proving (2) implies (1), we use an ``unravelling'' construction, similar to the one employed in the proof of Proposition~\ref{pro:characterization-all-trees} to convert a proof DAG to a proof tree. In particular, consider a compressed DAG $G=(V,E)$ of $\alpha$ w.r.t.~$D$ and $\dep$ with $\support{G}=D'$. By definition of $G$, for each non-leaf node $\beta$ of $G$, its children $\gamma_1,\ldots,\gamma_n$ in $G$ are such that there exists a rule $R_0(\bar x_0)\ \assign\ R_1(\bar x_1),\ldots,R_m(\bar x_m) \in \dep$ and a function $h : \bigcup_{i \in [m]} \bar x_i \ra \ins{C}$ such that $\beta = R_0(h(\bar x_0))$, and $\{\gamma_i\}_{i \in [n]} = \{R_i(h(\bar x_i)) \mid i \in [m]\}$; we call $(\sigma,h)$ the \emph{trigger of $\beta$ in $G$}, for some arbitrarily chosen pair $(\sigma,h)$ as described above.
	%
	We unravel $G$ into an unambiguous proof tree $T=(V',E',\lambda')$ of $\alpha$ w.r.t.~$D$ and $\dep$ as follows. We add a node $v$ to $T$ with label $\lambda'(v) = \alpha$. Then, if $v$ is a node of $T$ with some label $\lambda'(v) = \beta$, letting $(\sigma,h)$ be the trigger of $\beta$ in $G$, where $\sigma = R_0(\bar x_0)\ \assign\ R_1(\bar x_1),\ldots,R_m(\bar x_m)$, we add $m$ fresh new nodes $u_1,\ldots,u_m$ to $T$, where $u_i$ has label $\lambda'(u_i)=R_i(h(\bar x_i))$, for $i \in [m]$, and we add edges $(v,u_1),\ldots,(v,u_m)$ to $T$.
	%
	The fact that $T$ is a proof tree of $\alpha$ w.r.t.\ $D$ and $\dep$ follows by construction. To see that $T$ is unambiguous, observe that by the definition of compressed DAG, and by the construction of $T$, for every two non-leaf nodes $u,v$ of $T$ with the same label $\beta$, $u$ and $v$ have the same number of children $u_1,\ldots,u_n$, and $v_1,\ldots,v_n$, with $\lambda'(u_i)=\lambda'(v_i)$, for $i \in [n]$. Clearly, $\support{T}=D'$.
\end{proof}

With the above characterization in place, we are now ready to discuss how we construct our Boolean formula.

\medskip
\noindent
\textbf{Graph of Rule Instances and Downward Closure.} For our purposes, a \emph{(directed) hypergraph} $\mathcal{H}$ is a pair $(V,E)$, where $V$ is the set of \emph{nodes} of $\mathcal{H}$, and $E$ is the set of its \emph{hyperedges}, i.e., pairs of the form $(\alpha,T)$, where $\alpha \in V$, and $\emptyset \subsetneq T \subseteq V$. For two nodes $u,v$ of $\mathcal{H}$, we say that \emph{$u$ reaches $v$ in $\mathcal{H}$}, if either $u=v$, or there exists a sequence of hyperedges of the form $(u_1,T_1),\ldots,(u_n,T_n)$, with $u_1=u$, $v \in T_n$, and $u_i \in T_{i-1}$ for $1 < i \le n$. For $u \in V$, we write $\downof{\mathcal{H}}{u}$ for the hypergraph $(V',E')$ obtained from $\mathcal{H}$, where $V'$ contains $u$ and all nodes reachable from $u$, and the hyperedges are all the hyperedges of $\mathcal{H}$ mentioning a node of $V'$.
%
We can now introduce the notion of graph of rule instances.

\begin{definition}[\textbf{Graph of Rule Instances}]
Consider a Datalog program $\dep$, and a database $D$ over $\esch{\dep}$. 
The \emph{graph of rule instances (GRI) of $D$ and $\dep$} is the hypergraph $\gri{D}{\dep}=(V,E)$, with $V \subseteq \base{D,\dep}$, such that

\begin{enumerate}
	\item For each $\alpha \in D$, $\alpha \in V$.
	\item If there exists a rule $R_0(\bar x_0)\ \assign\ R_1(\bar x_1),\ldots,R_n(\bar x_n)$ in $\dep$ and a function $h : \bigcup_{i \in [n]} \bar x_i \ra \ins{C}$ such that $\alpha_i = R_i(h(\bar x_i)) \in V$, for $i \in [n]$, then $\alpha_0 = R_0(h(\bar x_0)) \in V$, and $(\alpha_0,\{\alpha_1,\ldots,\alpha_n\}) \in E$.
\end{enumerate}
\end{definition}


Roughly, $\gri{D}{\dep}$ is a structure that ``contains'' all possible compressed DAGs of $\alpha$ w.r.t.~$D$ and $\dep$.
%
Since we are interested in finding only compressed DAGs of a specific fact $\alpha$, we do not need to consider $\gri{D}{\dep}$ in its entirety, but we only need the sub-hypergraph of $\gri{D}{\dep}$ containing $\alpha$, and all nodes reachable from it.
%
Formally, for a Datalog program $\dep$, a database $D$ over $\esch{\dep}$, and a node (i.e., a fact) $\alpha$ of $\gri{D}{\dep}$, the \emph{downward closure} of $\alpha$ w.r.t.~$D$ and $\dep$ is the hypergraph $\downc{D}{\dep}{\alpha} = \downof{\gri{D}{\dep}}{\alpha}$.
%
In other words, the downward closure keeps from $\gri{D}{\dep}$ only the part that is relevant to derive the fact $\alpha$. It is easy to verify that $\downc{D}{\dep}{\alpha}$ ``contains'' all compressed DAGs of $\alpha$ w.r.t.~$D$ and $\dep$, and the next technical result follows:
%In particular, consider a $k$-ary query $Q=(\dep,P)$, a database $D$ over $\esch{\dep}$, a tuple $\bar t \in \adom{D}^k$, and a proof DAG $G=(V,E,\lambda)$ of $\bar t$ w.r.t.\ $D$ and $\dep$. We say that $G$ is a \emph{subgraph induced by} $\downc{D}{Q}{\bar t}$ if for every node $v \in V$ with outgoing edges $(u,v_1),\ldots,(u,v_n)$ in $G$, we have that $(\lambda(u),\{\lambda(v_1),\ldots,\lambda(v_n)\})$ is a hyperedge of $\downc{D}{Q}{\bar t}$.
%The following lemma is straightforward.

\begin{lemma}\label{lem:dag-in-down}
	Consider a Datalog program $\dep$, a database $D$ over $\esch{\dep}$, a fact $\alpha$ over $\sch{\dep}$, and a compressed DAG $G=(V,E)$ of $\alpha$ w.r.t.\ $D$ and $\dep$. Then, for every node $\beta \in V$ with outgoing edges $(\beta,\gamma_1),\ldots,(\beta,\gamma_n)$ in $G$, we have that $(\beta,\{\gamma_1,\ldots,\gamma_n\})$ is a hyperedge of $\downc{D}{\dep}{\alpha}$.
\end{lemma}

%\begin{proof}
%	The claim follows by definition of compressed DAG, the definition  of GRI, and the definition of downward closure.
%\end{proof}

%The main idea of our reduction is thus to first construct the downward closure of the given tuple $\bar t$ w.r.t.\ $D$ and $Q$, (this is clearly doable in polynomial time w.r.t.\ $D$), and then construct, starting from $\downc{D}{Q}{\bar t}$, a Boolean formula in CNF $\varphi$ whose satisfying truth assignments correspond to the unambiguous proof DAGs of $\bar t$ w.r.t.\ $D$ and $Q$, having $D$ as their support, that can be extracted from $\downc{D}{Q}{\bar t}$.
%%The latter are obtained by letting the formula have access to $\downc{D}{Q}{\bar t}$. 
%
%We point out that if $\bar t \not \in Q(D)$, or not all facts in $D$ occur as nodes in $\downc{D}{Q}{\bar t}$, then, no proof DAG $G$ for $\bar t$ w.r.t.\ $D$ and $Q$ exists that has $\support{G} = D$. So, the interesting case in our reduction is when $\bar t \in Q(D)$, and all facts of $D$ occur in $\downc{D}{Q}{\bar t}$. We will focus on this case for the rest of this section.

%We are now ready to discuss the construction of the Boolean formula.

\smallskip
\noindent
\textbf{The Boolean Formula.} We are now ready to introduce the desired Boolean formula. For a Datalog query $Q = (\dep,R)$, a database $D$ over $\esch{\dep}$, and a tuple $\bar t \in \adom{D}^{\arity{R}}$, we construct in polynomial time in $|D|$ the formula $\phi_{(\bar t,D,Q)}$ such that the why-provenance of $\bar t$ w.r.t.~$D$ and $Q$ relative to unambiguous proof trees can be computed from the truth assignments that make $\phi_{(\bar t,D,Q)}$ true. 
%In particular, $\phi_{(\bar t,D,Q)}$ has the following properties; let $\downc{D}{Q}{R(\bar t)}=(V,E)$.

Let $\downc{D}{Q}{R(\bar t)}=(V,E)$. The set of Boolean variables of $\phi_{(\bar t,D,Q)}$ is composed of four disjoint sets $V_N$, $V_H$, $V_E$, and $V_C$ of variables. Each variable in $V_N$ corresponds to a node of $\downc{D}{Q}{R(\bar t)}$, i.e., $V_N = \{x_\alpha \mid \alpha \in V\}$, each variable in $V_H$ corresponds to a hyperedge of $\downc{D}{Q}{R(\bar t)}$, i.e., $V_H = \{y_e \mid e \in E\}$, and each variable in $V_E$ corresponds to a "standard edge" that can be extracted from a hyperedge of $\downc{D}{Q}{R(\bar t)}$, i.e., $V_E = \{z_{(\alpha,\beta)} \mid (\alpha,T) \in E \text{ with } \beta \in T \}$. The set $V_C$ will be discussed later. 
%
Roughly, the variables in $V_N$ and $V_E$ that will be true via a satisfying assignment of $\phi_{(\bar t,D,Q)}$ will induce the nodes and the edges of a compressed DAG $G$ for $R(\bar t)$ w.r.t.\ $D$ and $Q$, which, by Proposition~\ref{pro:characterization-u-trees-improved}, will imply that $\support{G} \in \unwhy{\bar t}{D}{Q}$.

%The clauses of the formula will force a satisfying truth assignment to assign true to the variables in $V_N \cup V_E$ so that they encoderepresent an unambiguous proof DAG of $\bar t$ w.r.t.\ $D$ and $Q$, and false all the other variables in $V_E$. In such a truth assignment, the value given to the variables in $V_N \cup V_H$ will be uniquely determined by the values assigned to the variables in $V_E$.

The formula $\phi_{(\bar t,D,Q)}$ is of the form
%a conjunction of different formulas:
\[
\phi_{\mi{graph}} \wedge \phi_{\mi{root}} \wedge \phi_{\mi{proof}} \wedge \phi_{\mi{acyclic}}.
\]
We proceed to discuss each of the above formulas. In the following, we use $A \Rightarrow B$ as a shorthand for $\neg A \vee B$.
The first formula $\phi_{\mi{graph}}$ is in charge of guaranteeing consistency between the truth assignments of the variables in $V_N$ and the variables in $V_E$, i.e., if an edge between two nodes is stated to be part of $G$, then the two nodes must belong to G as well:
$$ \phi_{\mi{graph}} = \bigwedge\limits_{z_{(\alpha,\beta)} \in V_E} (z_{(\alpha,\beta)} \Rightarrow x_\alpha) \wedge (z_{(\alpha,\beta)} \Rightarrow x_\beta).$$
%
The second formula $\phi_{\mi{root}}$ guarantees that the atom $R(\bar t)$ is indeed a node of $G$, it is the root of $G$, and no other atom that is a node of $G$ can be the root (i.e., it must always have at least one incoming edge):
\begin{multline*}
	\phi_{\mi{root}} = x_{R(\bar t)} \wedge
	\left(\bigwedge\limits_{z_{(\alpha,R(\bar t))} \in V_E} \neg z_{(\alpha,R(\bar t))} \right)\wedge \\
	\bigwedge\limits_{\substack{x_\alpha \in V_N \\ \text{ with } \alpha \neq R(\bar t)}} \left( x_\alpha \Rightarrow \bigvee\limits_{z_{(\beta,\alpha)} \in V_E} z_{(\beta,\alpha)}\right).
\end{multline*}
%
We now move to the next formula $\phi_{\mi{proof}}$. Roughly, this formula is in charge of ensuring that, whenever an intensional atom $\alpha$ is a node of $G$, then it must have the correct children in $G$. That is, its children are the ones coming from some hyperedge of $\downc{D}{Q}{R(\bar t)}$, and no other nodes are its children (this is needed to guarantee that $G$ is a \emph{compressed} DAG). This is achieved with two sub-formulas. The first part is in charge of choosing some hyperedge $(\alpha,T)$ of $\downc{D}{Q}{R(\bar t)}$, for each intensional atom $\alpha$, while the second guarantees that for each selected hyperedge $(\alpha,T)$ (one per intensional atom $\alpha$), \emph{all and only} the nodes in $T$ are children of $\alpha$ in $G$:
\begin{multline*}
	\phi_{\mi{proof}} = 
	\bigwedge\limits_{\substack{x_\alpha \in V_N \text{ with } \\ \alpha \text{ intensional }}} 
	\left( x_\alpha \Rightarrow \bigvee\limits_{y_{(\alpha,T)} \in V_H} y_{(\alpha,T)} \right) \wedge \\
	\bigwedge\limits_{\substack{y_{e} \in V_H \\ \text{with } e=(\alpha,T)}} \left( \bigwedge\limits_{z_{(\alpha,\beta)} \in V_E} y_{e} \Rightarrow 
	\ell_{e,\beta}
	\right),
\end{multline*}
where, for a hyperedge $e=(\alpha,T)$, $\ell_{e,\beta}$ denotes $z_{(\alpha,\beta)}$ if $\beta \in T$, and $\neg z_{(\alpha,\beta)}$ otherwise.

\medskip
\noindent
\textbf{Remark.} Although we are interested in choosing \emph{exactly one} hyperedge $(\alpha,T)$ for each intensional atom $\alpha$, the above formula uses a simple disjunction rather an exclusive or. This is fine as any truth assignment that makes two variables $y_{(\alpha,T_1)}$ and $y_{(\alpha,T_2)}$ true cannot be a satisfying assignment, since the second subformula in $\phi_{\mi{proof}}$, e.g., requires that the variables $z_{(\alpha,\beta)}$ with $\beta \in T_1$ are true, while all others must be false. Hence, since $T_1 \neq T_2$, when considering the hyperedge $(\alpha,T_2)$, this subformula will not be satisfied.

%\smallskip
%We now discuss the formula $\phi_{\mi{Support}}$. This formula is in charge of verifying that the set of nodes of $G$ corresponding to extensional atoms coincides with the database $D$, i.e., $\support{G}=D$.\footnote{Recall that we assume all the database facts appear in $\downc{D}{Q}{R(\bar t)}$, and so all the variables we use in $\phi_{\mi{Support}}$ are well defined.}
%
%$$ \phi_{\mi{Support}} = \bigwedge\limits_{\alpha \in D} x_\alpha.$$

\medskip
\noindent The remaining formula to discuss is $\phi_{\mi{acyclic}}$. This last formula is in charge of checking that $G$, i.e., the graph whose edges correspond to the true variables in $V_E$, is acyclic. Checking acyclicity of a graph encoded via Boolean variables in a Boolean formula is a well-studied problem in the SAT literature, and thus different encodings exist. For the sake of our construction, it is enough to use the simplest (yet, not very efficient in practice) encoding, which just encodes the transitive closure of the graph. However, for our experimental evaluation, we will implement this formula using a more efficient encoding based on so-called vertex elimination, which reduces by orders of magnitude the size of the formula $\phi_{\mi{acyclic}}$~\cite{RankoohR22}.

To encode the transitive closure, we now need to employ the set of Boolean variables $V_C$, having a variable of the form $t_{(\alpha,\beta)}$, for every two nodes $\alpha,\beta$ of $\downc{D}{Q}{R(\bar t)}$. Intuitively, $t_{(\alpha,\beta)}$ denotes whether a path exists from $\alpha$ to $\beta$ in $G$.
With these variables in place, writing $\phi_{\mi{acyclic}}$ is straightforward: it just encodes the transitive closure of the underlying graph, and then checks whether no cycle exists:
\begin{multline*}
	\phi_{\mi{acyclic}} = \left(\bigwedge\limits_{z_{(\alpha,\beta)} \in V_E} z_{(\alpha,\beta)} \Rightarrow t_{(\alpha,\beta)}\right) \wedge \\
	%
	\left(\bigwedge\limits_{\substack{z_{(\alpha,\beta)} \in V_E, t_{(\beta,\gamma)} \in V_C}} z_{(\alpha,\beta)} \wedge t_{(\beta,\gamma)} \Rightarrow t_{(\alpha,\gamma)}\right) \wedge \\
	\left( \bigwedge\limits_{t_{(\alpha,\alpha)} \in V_C} \neg t_{(\alpha,\alpha)} \right).
\end{multline*}

This completes the construction of our Boolean formula. One can easily verify that the formula is in CNF, and that can be constructed in polynomial time. 
%
We proceed to show its correctness, i.e., Proposition~\ref{pro:why-provenance-sat}.
%Consider a Datalog query $Q = (\dep,R)$, a database $D$ over $\esch{\dep}$, and a tuple $\bar t \in \adom{D}^{\arity{R}}$. 
For a truth assignment $\tau$ from the variables of $\phi_{(\bar t,D,Q)}$ to $\{0,1\}$, we write $\db{\tau}$ for the database $\{\alpha \in D \mid x_\alpha \in V_N \text{ and } \tau(x_\alpha)  =1\}$, i.e., the database collecting all facts having a corresponding variable in $\phi_{(\bar t,D,Q)}$ that is true w.r.t.~$\tau$. Finally, we let $\sem{\phi_{(\bar t,D,Q)}}$ as
\[
\left\{\db{\tau} \mid \tau \text{ is a satisfying assignment of } \phi_{(\bar t,D,Q)}\right\}.
\]

We are now ready to prove Proposition~\ref{pro:why-provenance-sat}, which we report here for convenience:

\begin{manualproposition}{\ref{pro:why-provenance-sat}}
	\prowhyprovenancesat
\end{manualproposition}

\begin{proof}
%Consider a Datalog query $Q = (\dep,R)$, a database $D$ over $\esch{\dep}$, and a tuple $\bar t \in \adom{D}^{\arity{R}}$. 
Due to Proposition~\ref{pro:characterization-u-trees-improved}, it suffices to show that:

\begin{lemma}\label{lem:sat-solutions}
	For every database $S \subseteq D$, the following are equivalent:
	\begin{enumerate}
		\item There exists a compressed DAG $G$ of $R(\bar t)$ w.r.t.~$D$ and $\dep$ with $\support{G}=S$.
		\item There exists a satisfying truth assignment $\tau$ of $\phi_{(\bar t,D,Q)}$ such that $\db{\tau}=S$.
	\end{enumerate}
\end{lemma}

\begin{proof}
	We start with the implication $(1) \Rightarrow (2)$. Assume that $G=(V,E)$ is a compressed DAG for $R(\bar t)$ w.r.t.~$D$ and $\dep$ with $\support{G}=S$. We construct the following truth assignment $\tau$ of $\phi_{(\bar t,D,Q)}$:
	\begin{itemize} 
		\item For each $x_\alpha \in V_N$, $\tau(x_\alpha)=1$ if $\alpha$ is a node of $G$, otherwise $\tau(x_\alpha)=0$. 
		\item For each $z_{(\alpha,\beta)} \in V_E$, $\tau(z_{(\alpha,\beta)})=1$ if there is an edge $(\alpha,\beta)$ in $G$, otherwise $\tau(z_{(\alpha,\beta)})=0$. 
		\item For each $y_{(\alpha,T)} \in V_H$, $\tau(y_{(\alpha,T)}) = 1$ if $\alpha$ is a node in $G$ with outgoing edges $(\alpha,\beta_1),\ldots,(\alpha,\beta_n)$ such that $T=\{\beta_1,\ldots,\beta_n\}$, otherwise $\tau(y_{(\alpha,T)})=0$.
		\item For each $t_{(\alpha,\beta)} \in V_C$, $\tau(t_{(\alpha,\beta)}) = 1$ if there is a path from $\alpha$ to $\beta$ in $G$, otherwise $\tau(t_{(\alpha,\beta)})=0$.
	\end{itemize}
	We now claim that $\tau$ makes $\phi_{(\bar t,D,Q)}$ true and $\db{\tau}=S$.
	
	\medskip
	\noindent
	\textbf{Observation 1.} By Lemma~\ref{lem:dag-in-down}, every node $\alpha$ of $G$ has a variable $x_\alpha$ in $\phi_{(\bar t,D,Q)}$. Similarly, if a node $\alpha$ has outgoing edges $(\alpha,\beta_1),\ldots,(\alpha,\beta_n)$ in $G$, then $z_{(\alpha,\beta_i)}$ for $i \in [n]$, and $y_{(\alpha,T)}$, with $T=\{\beta_1,\ldots,\beta_n\}$, are all variables of $\phi_{(\bar t,D,Q)}$.
	
	\medskip
	
	From Observation 1, each fact $\alpha \in \support{G}=S$ has a corresponding variable $x_\alpha$ in $\phi_{(\bar t,D,Q)}$. Moreover, by construction of $\tau$, $\tau(x_\alpha)=1$ for each $\alpha \in S$. Furthermore, for all facts $\beta$ of $D$ not in $S$ it means there is no node in $G$ labeled with $\beta$, and thus $\tau(x_\beta)=0$. Hence $\db{\tau}=S$. 
	
	We now show that $\tau$ makes $\phi_{(\bar t,D,Q)}$ true. We proceed by considering its subformulas separately.
	\begin{description}
	%\smallskip
	%\noindent
	\item[\underline{$\phi_\mi{graph}$}.] It is clear that $\tau$ makes this formula true since when $\tau(z_{(\alpha,\beta)}) = 1$, it means that $\alpha$ and $\beta$ are nodes of $G$, and thus, by construction of $\tau$, $\tau(x_\alpha) = \tau(x_\beta)=1$.
	
	%\smallskip
	%\noindent
	\item[\underline{$\phi_\mi{root}$}.] Since $R(\bar t)$ is the root of $G$, by Observation 1 and by construction of $\tau$, $\tau(x_{R(\bar t)}) = 1$. To see why the formula
	\[
	\left(\bigwedge\limits_{z_{(\alpha,R(\bar t))} \in V_E} \neg z_{(\alpha,R(\bar t))} \right)
	\]
	is true, since $R(\bar t)$ is the root, it does not have any incoming edges, and thus, by construction of $\tau$, $\tau(z_{(\alpha,R(\bar t))})=0$, for each $z_{(\alpha,R(\bar t))} \in V_E$. 
	Finally, regarding the formula
	$$\bigwedge\limits_{\substack{x_\alpha \in V_N \\ \text{ with } \alpha \neq R(\bar t)}} \left( x_\alpha \Rightarrow \bigvee\limits_{z_{(\beta,\alpha)} \in V_E} z_{(\beta,\alpha)}\right),$$
	if $\tau(x_\alpha)=1$, for some $\alpha \neq R(\bar t)$, by construction of $\tau$, $\alpha$ is a node of $G$. By Observation 1, for every edge $(\beta,\alpha)$ in $G$, we have that $z_{(\beta,\alpha)}$ is a variable of $\phi_{(\bar t,D,Q)}$, and $\tau(z_{(\beta,\alpha)}) = 1$, hence the disjunction is true.
	
	%\smallskip
	%\noindent
	\item[\underline{$\phi_\mi{proof}$}.] We start by considering the formula
	\[
	\bigwedge\limits_{\substack{x_\alpha \in V_N \text{ with } \\ \alpha \text{ intensional }}} 
	\left( x_\alpha \Rightarrow \bigvee\limits_{y_{(\alpha,T)} \in V_H} y_{(\alpha,T)} \right).
	\]
	If $\tau(x_\alpha)=1$, for some intensional fact $\alpha$, it means that $\alpha$ is a node of $G$. Let $(\alpha,\beta_1),\ldots,(\alpha,\beta_n)$ be the outgoind edges of $\alpha$ in $G$. By Observation 1, $y_{(\alpha,T)}$, with $T=\{\beta_1,\ldots,\beta_n\}$, is a variable of $\phi_{(\bar t,D,Q)}$, and by construction of $\tau$, $\tau(y_{(\alpha,T)}) = 1$, hence the disjunction is true.
	%
	We now consider the formula
	\[
	\bigwedge\limits_{\substack{y_{e} \in V_H \\ \text{with } e=(\alpha,T)}} \left( \bigwedge\limits_{z_{(\alpha,\beta)} \in V_E} y_{e} \Rightarrow 
	\ell_{e,\beta}
	\right).
	\]
	If $\tau(y_e)=1$, for $e=(\alpha,T)$, then $\alpha$ is a node of $G$ with outgoing edges $(\alpha,\beta_1),\ldots,(\alpha,\beta_n)$, where $T=\{\beta_1,\ldots,\beta_n\}$. By Observation 1, $z_{(\alpha,\beta_i)} \in V_E$, for $i \in [n]$, and by construction of $\tau$, $\tau(z_{(\alpha,\beta_i)})=1$, for $i \in [n]$. Hence, all implications of the form $y_e \Rightarrow \ell_{e,\beta}$, where $\beta \in \{\beta_1,\ldots,\beta_n\}$ are true.
	Regarding the other implications, i.e., when $\beta \not \in \{\beta_1,\ldots,\beta_n\}$, since $\alpha$ has no other outgoing edges in $G$, by construction of $\tau$, $\tau(z_{(\alpha,\beta)})=0$, for all other facts $\beta \not \in \{\beta_1,\ldots,\beta_n\}$. Hence, the whole formula is satisfied.
	
	%\smallskip
	%\noindent
	\item[\underline{$\phi_{\mi{acyclic}}$}.] The fact that the formula is true follows from the acyclicity of $G$, Observation 1, and the construction of $\tau$.
	\end{description}


	%\medskip
	We now proceed with $(2) \Rightarrow (1)$. By hypothesis, there is a truth assignment $\tau$ that makes $\phi_{(\bar t,D,Q)}$ true and $\db{\tau} = S$. We define the auxiliary sets
	\begin{eqnarray*}
		\nodes{\tau} &=& \{\alpha \mid x_\alpha \in V_N \text{ and } \tau(x_{\alpha}) = 1\}\\
		%
		\edges{\tau} &=& \{(\alpha,\beta) \mid z_{(\alpha,\beta)} \in V_E \text{ and } \tau(z_{(\alpha,\beta)})=1\}. 
	\end{eqnarray*}
	%Let us first make some useful observations.
	
	%\medskip
	%\noindent
	%\textbf{Observation 2.} 
	
	\noindent Since $\tau$ makes $\phi_\mi{graph}$ true, every fact occurring in $\edges{\tau}$ also occurrs in $\nodes{\tau}$; hence, $G=(\nodes{\tau},\edges{\tau})$ is a well-defined directed graph. Moreover, since $\db{\tau}=S$, all the nodes without outgoing edges in $G$ are precisely the ones in $S$. Finally, since $\tau$ makes $\phi_{\mi{acyclic}}$ true, $G$ is acyclic, and since $\tau$ satisfies $\phi_\mi{root}$, $R(\bar t)$ is the only node of $G$ without incoming edges. Thus, $G$ is a DAG, its root is $R(\bar t)$, and its leaves is exactly the set $S$. It remains to argue that $G$ is a compressed DAG of $R(\bar t)$ w.r.t.~$D$ and $\dep$.
	
	%\smallskip
	%\noindent
	%\textbf{Observation 3.} 
	Consider a node $\alpha$ of $G$ which is an intensional fact. It is clear that $\tau(x_\alpha)=1$, and thus, the dijunction in $\phi_\mi{proof}$ must be true. Hence, $\tau(y_{(\alpha,T)})=1$, for some hyperedge $(\alpha,T)$ of $\downc{D}{Q}{R(\bar t)}$. In particular, since $\tau$ makes
	\[
	\bigwedge\limits_{\substack{y_{e} \in V_H \\ \text{with } e=(\alpha,T)}} \left( \bigwedge\limits_{z_{(\alpha,\beta)} \in V_E} y_{e} \Rightarrow 
	\ell_{e,\beta}
	\right)\]
	true in $\phi_{\mi{proof}}$, and $\tau(y_{(\alpha,T)})=1$, we must have that $\tau$ assigns $1$ to all variables of the form $z_{(\alpha,\beta)}$ with $\beta \in T$, and $0$ to all other variables of the form $z_{(\alpha,\beta)}$ with $\beta \not \in T$. This means that there is no other variable of the form $y_{(\alpha,T')}$ with $T'=T$ that is assigned $1$ by $\tau$. Hence, 
	for each node $\alpha$ of $G$ which is intensional, $\alpha$ has outgoing edges $(\alpha,\beta_1),\ldots,(\alpha,\beta_n)$, and these are such that there exists a hyperedge of $\downc{D}{Q}{R(\bar t)}$ of the form $(\alpha,T)$, with $T=\{\beta_1,\ldots,\beta_n\}$. The latter, by definition of downward closure, implies that for each node $\alpha$ of $G$ which is intensional, $\alpha$ has outgoing edges $(\alpha,\beta_1),\ldots,(\alpha,\beta_n)$, and these are such that there exists a rule $\sigma \in \dep$ of the form $R_0(\bar x_0)\ \assign\ R_1(\bar x_1),\ldots,R_m(\bar x_m)$, with $m \ge n$, and a function $h : \bigcup_{i \in [n]} \bar x_i \rightarrow \ins{C}$, such that $R_0(h(\bar x_0))=\alpha$ and $\{R_1(h(\bar x_1)),\ldots,R_m(h(\bar x_m))\} = \{\beta_1,\ldots,\beta_n\}$. Hence, $G$ is a compressed DAG of $R(\bar t)$ w.r.t.~$D$ and $\dep$, as needed.
	%
%	\smallskip
%	We now show how to construct an unambiguous proof DAG $G$ for $\bar t$ w.r.t.\ $D$ and $Q$ with $\support{G}=S$, starting from $G'$.
%	We inductively construct $G=(V,E,\lambda)$ as the following node-labeled graph. There is a node $u \in V$ labeled with $\lambda(u)=g$. Moreover, for every node $u \in V$ labeled with some intensional fact $\alpha = \lambda(u)$ that is a node of $G'$, letting $(\sigma,h)$ be the trigger induced by $\alpha$, with $\sigma = R_1(\bar x_1),\ldots,R_m(\bar x_m) \rightarrow R(\bar x)$, $V$ contains $m$ additional distinct nodes $v_1,\ldots,v_m$ with $\lambda(v_i)=R_i(h(\bar x_i))$, for $i \in [m]$. We claim that $G$ is an unambiguous proof DAG for $\bar t$ w.r.t.\ $D$ and $Q$ with $\support{G}=S$.
%	
%	Indeed, $G$ is acyclic by construction (as at each inductive step, we consider fresh nodes). Moreover, by Observation 1, since only the node $g$ in $G'$ has no incoming edges, There is only one node in $G$ without incoming edges, and this node is labeled with $g$. Hence, $G$ is a node-labeled DAG rooted at a node $u$ with label $g$. We now prove it is actually a proof DAG.
%	For this, note that for every node $u$ of $G$ labeled with an intensional fact $\alpha$, $u$ has $m$ children $v_1,\ldots,v_m$ that are obtained by means of a pair $(\sigma,h)$ of a rule of the form $R_1(\bar x_1),\ldots,R_m(\bar x_m) \rightarrow R(\bar x)$, and function, where the label of $u$ coincides with $R(h(\bar x))$, and $\lambda(v_i)=R_i(h(\bar x_i))$, for $i \in [m]$. This is the very definition of a proof DAG.
%	
%	To see that $G$ is unambiguous, assume, towards a contradiction, that $G$ is not unambiguous. So, there are two nodes $u,v$ in $G$ with $\lambda(u)=\lambda(v)$, but $G[u] \eqtree G[v]$ does not hold. The latter means that there must be two nodes $u'$ and $v'$ in $G[u]$ and $G[v]$ respectively, such that $\alpha=\lambda(u')=\lambda(v')$, but their immediate children are labeled differently. However, this cannot be the case, since for any node in $G$ labeled with an intensional fact $\alpha$, its children are always labeled with facts in $T$, where $(\alpha,T)$ is the (unique) trigger induced by $\alpha$.
%	
%	Finally, the fact that $\support{G}=S$ follows from the fact that the leaves of $G'$ are precisely the ones in $S$, by Observation 2. This concludes our proof.
\end{proof}

Proposition~\ref{pro:why-provenance-sat} immediately follows from Lemma~\ref{lem:sat-solutions}.
\end{proof}

\subsection{Implementation Details}
In this section, we expand on the discussion of our implementation presented in the main body of the paper. In what follows, fix a Datalog query $Q = (\dep,R)$, a database $D$ over $\esch{\dep}$, and a tuple $\bar t \in \adom{D}^{\arity{R}}$.

\medskip
\noindent \textbf{Constructing the Downward Closure.}
Recall that the construction of $\phi_{(\bar t,D,Q)}$ relies on the downward closure of $R(\bar t)$ w.r.t.~$D$ and $\dep$. It turns out that the hyperedges of the downward closure can be computed by executing a slightly modified Datalog query $Q_{\downarrow}$ over a slightly modified database $D_{\downarrow}$. In other words, the answers to $Q_{\downarrow}$ over $D_{\downarrow}$ coincide with the hyperedges of the downward closure.
For this, we are going to adopt a slight modification of an existing approach presented in~\cite{ElKM22}. In that paper, the authors were studying the problem of computing the why-provenance of Datalog queries w.r.t.\ \emph{standard} proof trees. However, except for the construction of $\downc{D}{Q}{R(\bar t)}$, their approach to compute the supports for general trees is fundamentally different from ours, since they employ existential rule-based engines rather than SAT solvers; it is not clear how their approach could be adapted for our purposes, as we require checking whether the underlying trees are unambiguous. Moreover, the approach of~\cite{ElKM22} computes the \emph{whole set} of supports all at once, while our approach based on SAT solvers allows to \emph{enumerate} supports, and thus allows the incremental construction of the why-provenance.
%
Nonetheless, the construction of $\downc{D}{Q}{R(\bar t)}$ is common to both approaches, and thus we borrow the techniques of~\cite{ElKM22} for this task, which we briefly recall in the following.

The main idea is to employ an existing Datalog engine to compute the answers of a query $Q_{\downarrow}$ obtained from $Q$ over a slight modification $D_{\downarrow}$ of $D$; the answers in $Q_{\downarrow}(D_{\downarrow})$ will coincide with all the hyperedges of $\downc{D}{Q}{R(\bar t)}$. 
%
To this end, the rules of $Q_{\downarrow}$ contain all the rules in $\dep$, which will be in charge of deriving all nodes of $\gri{D}{\dep}$, plus an additional set of rules that will be in charge of using such nodes to construct all the hyperedges of $\downc{D}{Q}{R(\bar t)}$. Formally, let $\omega$ be the maximum arity of predicates in $\dep$, and $b$ the maximum number of atoms in the body of a rule of $\dep$. We define two new predicates: 
\begin{itemize}
	\item $\curnode$ of arity $\omega+1$ that stores the current node of $\downc{D}{Q}{R(\bar t)}$ being processed during the evaluation of $Q_{\downarrow}$.
	\item $\hedge$ of arity $(\omega+1)+b\times(\omega+1)$, which stores the hyperedges being constructed during the evaluation of $Q_{\downarrow}$.
\end{itemize}
%
Furthermore, for an atom $\alpha = P(\bar u)$, we denote $\atomtotuple{\alpha}$ as the tuple of length $\omega+1$ of the form $c_P,\bar u,\star,\ldots,\star$, where $c_P$ and $\star$ are constants not in $D$. Intuitively, $\atomtotuple{\alpha}$ encodes the atom $\alpha$ as a tuple of fixed length. Finally, for an atom $\alpha$ and a sequence of atoms $\beta_1,\ldots,\beta_n$, with $n \le b$, we use $\tup{\alpha,\beta_1,\ldots,\beta_n}$ to denote the tuple of length $(\omega+1)+b \times (\omega+1)$ of the form $\tup{\alpha},\tup{\beta_1},\ldots,\tup{\beta_n},\star,\ldots,\star$.
%Similarly, forn an atom $\alpha = R(\bar t)$, and a set of atoms $T= \beta_1,\ldots,\beta_n$, with $\beta_i = R_i(\bar t_i)$, for $i \in [n]$, we use $\edgeatom{\alpha}{T}$ to denote the atom of the form
%$$ hedge()$$  

We define the Datalog query $Q_{\downarrow} = (\dep',\hedge)$, where $\dep' = \dep \cup \dep''$, where, for each rule $\sigma \in \dep$ of the form $R_0(\bar x_0)\ \assign\ R_1(\bar x_1),\ldots,R_n(\bar x_n)$, $\dep''$ contains the rules
\begin{multline*}
\sigma_1\ =\ \hedge(\atomtotuple{R_0(\bar x_0),R_1(\bar x_1),\ldots,R_n(\bar x_n)})\ \assign\ \\ \curnode(\atomtotuple{R_0(\bar x_0)}), R_1(\bar x_1),\ldots,R_n(\bar x_n)
\end{multline*}
and, for each $i \in [n]$,
\begin{multline*}
\sigma^{(i)}_2\ =\ \curnode(\atomtotuple{R_i(\bar x_i)})\ \assign\ \\ \curnode(\atomtotuple{R_0(\bar x_0)}),R_1(\bar x_1),\ldots,R_n(\bar x_n).
\end{multline*}
\begin{comment}
\[
\begin{array}{ll}
	\sigma_1: & \hedge(\atomtotuple{R_0(\bar x_0),R_1(\bar x_1),\ldots,R_n(\bar x_n)})\ \assign\ \\
	& \curnode(\atomtotuple{R_0(\bar x_0)}), R_1(\bar x_1),\ldots,R_n(\bar x_n), \\\ \\
	
	\sigma^{(i)}_2: & \curnode(\atomtotuple{R_i(\bar x_i)})\ \assign\ \curnode(\atomtotuple{R_0(\bar x_0)}),\\
	&  R_1(\bar x_1),\ldots,R_n(\bar x_n) \hfill \forall i \in [n].
\end{array}
\]
\end{comment}
Essentially, the rule $\sigma_1$ will construct a hyperedge $(\alpha,T)$ of $\downc{D}{Q}{R(\bar t)}$ whenever it is known that $\alpha$ is a node of $\downc{D}{Q}{R(\bar t)}$, and all the atoms in $T$ are used in $\gri{D}{\dep}$ to generate $\alpha$.
The rules of the form $\sigma^{(i)}_2$ are marking the new nodes as being part of $\downc{D}{Q}{\bar t}$.
%
%\medskip
%\noindent
%\textbf{Remark.} 
Note that the rules of $\dep''$ contain constants, whereas, according to our definition, rules are constant-free. Nevertheless, all existing Datalog engines support rules with constants, and for the sake of keeping the discussion simple, we slightly abuse our definition of rules in this section. It is easy to adapt the above set of rules to a set of rules without constants, by adding some auxiliary facts to the database.

%\medskip

The database $D_{\downarrow}$ is $D \cup \{\curnode(\tup{R(\bar t)})\}$, which simply states that $R(\bar t)$ must be a node of $\downc{D}{Q}{R(\bar t)}$.
One can easily see that each tuple in $Q_{\downarrow}(D_{\downarrow})$ encodes a hyperedge of $\downc{D}{Q}{R(\bar t)}$, and thus, we can construct $\downc{D}{Q}{R(\bar t)}$ by simply asking $Q_{\downarrow}$ over $D_{\downarrow}$.


Let us note that the main differences between our definition of $Q_{\downarrow}$ and $D_{\downarrow}$ w.r.t.\ the ones of~\cite{ElKM22} is that we encode nodes and hyperedges of $\downc{D}{Q}{R(\bar t)}$ as tuples, which allows us to employ the same Datalog engine that is used to answer the original query $Q$, rather than using external engines supporting more expressive languages such as existential rules.

\medskip
\noindent \textbf{Constructing the Formula.} 
Regarding the construction of the formula $\phi_{(\bar t,D,Q)}$, as already discussed before, for efficiency reasons, we consider a different encoding of the subformula $\phi_\mi{acyclic}$. Rather than using the transitive closure, we employ the technique of vertex elimination~\cite{RankoohR22}. The advantage of this approach is that it requires a number of Boolean variables for the encoding of $\phi_\mi{acyclic}$ which is of the order of $O(n \times \delta)$, where $n$ is the number of nodes of the underlying graph, and $\delta$ is the so-called \emph{elimination width} of the graph, which, roughly, is related to how connected the underlying graph is. Hence, we can avoid the costly construction of quadratically many variables whenever the elimination width is low.

\medskip
\noindent \textbf{Incrementally Constructing the Why-Provenance.}
Recall that we are interested in the incremental computation of the why-provenance, which is more useful in practice than computing the whole set at once. To this end, we need a way to enumerate all the members of the why-provenance without repetitions. This is achieved by adapting a standard technique from the SAT literature for enumerating the satisfying assignments of a Boolean formula, called {\em blocking clause}.
%
We initially collect in a set $S$ all the facts of $D$ occurring in the downward closure of $R(\bar t)$ w.r.t.~$D$ and $\dep$. Then, after asking the SAT solver for an arbitrary satisfying assignment $\tau$ of $\phi_{(\bar t,D,Q)}$, we output the database $\db{\tau}$, and then construct the ``blocking'' clause
$
\vee_{\alpha \in S} \ell_\alpha,
$
where $\ell_\alpha = \neg x_\alpha$ if $\alpha \in \db{\tau}$, and $\ell_\alpha = x_\alpha$ otherwise. We then add this clause to the formula, which expresses that no other satisfying assignment $\tau'$ should give rise to the same member of the why-provenance.
%, i.e., $\db{\tau'}=\db{\tau}$. 
This will exclude the previously computed explanations from the computation. We keep adding such blocking clauses each time we get a new member of the why-provenance until the formula is unsatisfiable.


{\footnotesize 
	\begingroup
	\setlength{\tabcolsep}{5pt} % Default value: 6pt
	\renewcommand{\arraystretch}{1.6} % Default value: 1
	\begin{figure*}[t]
		\centering
		\begin{tabular}{cc}
			\multicolumn{2}{c}{\includegraphics[width=135mm]{ground_formula_Doctors}} \\
			\multicolumn{2}{c}{(a) $\mathsf{Doctors}$} \\[6pt]
			\includegraphics[width=65mm, height=53.2mm]{ground_formula_TransClosure} &   \includegraphics[width=65mm]{ground_formula_Galen} \\
			(b) $\mathsf{TransClosure}$ & (c) $\mathsf{Galen}$ \\[6pt]
			\includegraphics[width=65mm]{ground_formula_Andersen_uniform_size} &   \includegraphics[width=65mm]{ground_formula_CSDA} \\
			(d) $\mathsf{Andersen}$ & (e) $\mathsf{CSDA}$
		\end{tabular}
		\caption{Building the downward closure and the Boolean formula (all scenarios).}
		\label{fig:all-task1}
	\end{figure*}
	\endgroup
}

\subsection{Experimental Evaluation}
In this section, we provide further details on the performance of our SAT-based approach, by presenting the results of the experimental evaluation over all the scenarios we considered in the paper; we report again the $\mathsf{Andersen}$ scenario for the sake of completeness. Recall that in our experimental analysis we consider two main tasks separately: (1) construct the downward closure and the Boolean formula, and (2) incrementally compute the why-provenance using the SAT solver.


Concerning task 1, we report in Figure~\ref{fig:all-task1} one plot for each scenario we consider, where in each plot we report the total running time for each database of that scenario. Furthermore, for each plot, and each database considered therein, we have five bars, that correspond to the five randomly chosen tuples. Each such bar shows the time for building the downward closure plus the time for constructing the Boolean formula. To ease the presentation, we grouped the $\mathsf{Doctors}$-based scenarios in one plot (recall that all such scenarios share a single database).

We can see that in most of the scenarios, the running time is in the order of some seconds. This is especially true for the $\mathsf{TransClosure}$ and $\mathsf{Doctors}$-based scenarios, having the simplest queries, while for the $\mathsf{Galen}$ scenario, where the query is more complex, as it involves non-linear recursion, the time is slightly higher for the largest database $D_{4}$. The $\mathsf{Andersen}$ and $\mathsf{CSDA}$ scenarios are the most challenging, since they both contain very large databases. Moreover, although the databases in $\mathsf{Andersen}$ are smaller than those of $\mathsf{CSDA}$, the complexity of its query, which involves non-linear recursion, makes the running time of $\mathsf{Andersen}$ over its largest database (6.8M facts) comparable to $\mathsf{CSDA}$ with the larger database $D_\mathsf{httpd}$ (10M facts). Of course, for the much larger databases $D_\mathsf{postgresql}$ and $D_\mathsf{linux}$, the running time is much higher, going up to 6-7 minutes for some tuples.
Considering the size of the databases at hand, we believe the running times for these last scenarios are quite reasonable.
%
As already discussed in the main body of the paper, we observed that most of the time is spent in building the downward closure.


Concerning task 2, that is, the incremental computation of the why-provenance, we present in Figure~\ref{fig:all-task2} one plot for each scenario we consider, where in each plot of scenario $s$ we report, for each database of $s$, the times required to build an explanation, that is, the time between the current member of the why-provenance and the next one (this time is also known as the delay).
%
Each plot collects the delays of constructing the members of the why-provenance (up to a limit of 10K members or 5 minutes timeout) for each of the five randomly chosen tuples. We use box plots, where the bottom and the top borders of the box represent the first and third quartile, i.e., the delay under which 25\% and 75\% of all delays occur, respectively, and the orange line represents the median delay. Moreover, the bottom and the top whisker represent the minimum and maximum delay, respectively. All times are expressed in milliseconds and we use logarithmic scale. 
%
As we did for task 1, we grouped the $\mathsf{Doctors}$-based scenarios in one plot.

{\footnotesize 
	\begingroup
	\setlength{\tabcolsep}{5pt} % Default value: 6pt
	\renewcommand{\arraystretch}{1.6} % Default value: 1
	\begin{figure*}[t]
		\centering
		\begin{tabular}{cc}
			\multicolumn{2}{c}{\includegraphics[width=135mm]{exptimes_Doctors}} \\
			\multicolumn{2}{c}{(a) $\mathsf{Doctors}$} \\[6pt]
			\includegraphics[width=65mm, height=53.2mm]{exptimes_TransClosure} &   \includegraphics[width=65mm]{exptimes_Galen} \\
			(b) $\mathsf{TransClosure}$ & (c) $\mathsf{Galen}$ \\[6pt]
			\includegraphics[width=65mm]{exptimes_Andersen_uniform_size} &   \includegraphics[width=65mm]{exptimes_CSDA} \\
			(d) $\mathsf{Andersen}$ & (e) $\mathsf{CSDA}$
		\end{tabular}
		\caption{Incremental computation of the why-provenance (all scenarios).}
		\label{fig:all-task2}
	\end{figure*}
	\endgroup
}

As observed in the main body of the paper, most of the delays are even lower than 1 millisecond, with the median in the order of microseconds. Therefore, once we have the Boolean formula in place, incrementally computing the members of the why-provenance is extremely fast. The worst case occurs in the $\mathsf{TransClosure}$ scenario, when considering the Facebook database, where we also have the only two cases where the construction of the supports exceeds the 5 minutes mark before being able to construct 10K supports, i.e., for the third and fifth tuple. Overall, in the $\mathsf{TransClosure}$ scenario w.r.t.~the Facebook database, the average delay is higher, and some supports require up to 10 seconds to be constructed. We believe that this has to do with the fact that the database $D_\text{facebook}$ encodes a graph which is highly connected, and thus the CNF formula, and in particular the formula $\phi_{\mi{acyclic}}$ encoding the acyclicity check, becomes quite large, and thus is much more demanding for the SAT solver. This was somehow expected since the vertex-elimination technique for checking acyclicity performs better the less connected the underlying graph is. We have confirmed this by running some other experiments with databases taken from~\cite{FanMK22}, which contain highly connected, synthetic graphs. In this case, although constructing the downward closure is very efficient (in the order of seconds), the construction of the formula $\phi_\mi{acyclic}$ goes out of memory. Hence, we expect that in applications where highly connected input graphs are common, a different approach for checking acyclicity in a CNF formula would be required. Nonetheless, we can safely conclude that in most cases, computing the members of the why-provenance can be done very efficiently.

\subsection{Comparative Evaluation}
As mentioned in Section~\ref{sec:conclusions}, we have performed a preliminary comparison with~\cite{ElKM22}. We conclude this section by discussing  the details of this comparative evaluation.
%
Let us first clarify that our implementation deals with a different problem. For a Datalog query $Q = (\dep,R)$, a database $D$ over $\esch{\dep}$, and a tuple $\bar t \in \adom{D}^{\arity{R}}$, the approach from~\cite{ElKM22} has been designed and evaluated for building the whole set $\why{\bar t}{D}{Q}$, whereas our approach has been designed and evaluated for incrementally computing $\unwhy{\bar t}{D}{Q}$. However, there is a setting where a reasonable comparison can be performed, which will provide some insights for the two approaches.
%
This is when the Datalog query $Q$ is both linear and non-recursive in which case the sets $\why{\bar t}{D}{Q}$ and $\unwhy{\bar t}{D}{Q}$ coincide since a proof tree of $R(\bar t)$ w.r.t.~$D$ and $\dep$ is trivially unambiguous.
%
Therefore, towards a fair comparison, we are going to consider the scenarios $\mathsf{Doctors}\text{-}i$, for $i \in [7]$, which consist of a Datalog query that is linear and non-recursive, and consider the end-to-end runtime of our approach (not the delays) without, of course, setting a limit on the number of members of why-provenance to build, or on the total runtime.




%We point out that providing a reasonable comparison between our approach and the one above is tricky, since they basically solve different problems. Indeed, while in our case we might have the advantage of needing to construct much less supports, since our proof trees are more refined, in the all-trees case there is no need to check for the unambiguity of the underlying trees. For these reasons, we believe the only reasonable comparison that can be made is over scenarios for which the two notions of trees coincide. This is only the case with the DOCTORS-based scenarios, which employ queries that are both linear and non-recursive. It is easy to verify that for such queries, unmabiguous trees and arbitrary trees coincide.\footnote{In any case, even if we wanted to make a comparison over all scenarios, this would not be possible, since the authors of~\cite{ElhalawatiKM22} did not provide the tools needed to construct the set of existential rules that are able to build the why-provenance. Hence, we have to rely on the pregenerated ones for the scenarios they also consider (i.e., $\mathsf{DOCTORS}$ and $\mathsf{GALEN}$).}

%Since the approach of~\cite{ElhalawatiKM22} can only construct the whole why-provenance, without enumeration, we compare the end-to-end running time of our apprach (i.e., by combining tasks \emph{1)} and \emph{2)}) with the running time of their approach. In this case, of course, we did not specify any limit on the number of supports to build, or on the total running time. Finally, although not strictly necessary for enumerating the supports, our implementation keeps in main memory each support that is produced, so to be fair with the approach of~\cite{ElhalawatiKM22} which keeps all supports in memory by design. 


The comparison is shown in Figure~\ref{fig:comparison-time}. For each scenario, we present the runtime for all five randomly chosen tuples for our approach (in blue) and the approach of~\cite{ElKM22} (in red); if a bar is missing for a certain tuple, then the execution ran out of memory.
%
We observe that for the simple scenarios the two approaches are comparable in the order of a second.
%
Now, concerning the demanding scenarios, i.e., $\mathsf{Doctors}\text{-}i$ for $i \in \{1,5,7\}$, we observe that our approach is, in general, faster. Observe also that for some of the most demanding cases, the approach of~\cite{ElKM22} runs out of memory.
%
We believe that the latter is due to the use of the rule engine VLog, which is intended for materialization-based reasoning with existential rules, whereas our approach relies on a Datalog engine (in particular, DLV), and thus, exploiting all the optimizations that are typically employed for evaluating a Datalog query. For example, the technique of {\em magic-set rewriting}, implemented by DLV, can greatly reduce the memory usage by building much fewer facts during the evaluation of the rules; see, e.g.,~\cite{LAAC+19}.

{\footnotesize 
	\begingroup
	\setlength{\tabcolsep}{5pt} % Default value: 6pt
	\renewcommand{\arraystretch}{1.6} % Default value: 1
\begin{figure*}[t]
	\centering
	\includegraphics[width=135mm]{comparison}
	\caption{Comparison of the end-to-end computation of the why-provenance.}
	\label{fig:comparison-time}
\end{figure*}
\endgroup
}





\end{document}

