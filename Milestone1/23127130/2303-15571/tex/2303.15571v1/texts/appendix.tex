
\section{\name GradCAM}
\label{sec: gradCAM}

The GradCAM uses the gradient information flowing into the last convolutional layer of the classifier to assign importance values to each neuron for a particular decision of interest.
\yf{exaplain GradCAM.}
%We use the GradCAM~\cite{selvaraju2016grad} to illustrate the actual performance of the EM classifiers.\yf{exaplain GradCAM.}
Fig.~\ref{fig: gradCAM} (a)-(c) present the average spectrogram of the first EM segment for Class 0, 1, and 5, respectively, \yf{which segment?}
and Figure~\ref{fig: gradCAM} (d)-(f) show the corresponding coarse GradCAM localization (red heatmaps).
The GradCAM heatmaps illustrate the sensitivity (magnitude of gradient) of the EM classification on the input neurons, 
which are the pixels on the spectrogram here.
The most sensitive (bright) parts are where the EM classifier focuses on when making a decision.
%The GradCAM can visualize the parts of an image that the classification model has to look to make a particular decision (the red parts).
%It explains that our classifiers focus on the entire spectrogram (features dispersed along the time)  to distinguish different classes.
For example, the primary signal of Class 1 is from the middle part of the spectrogram as shown in Fig.~\ref{fig: gradCAM} (b), and correspondingly GradCAM in Fig.~\ref{fig: gradCAM} (e) shows that our classifier also emphasizes the central part.
%It demonstrates that the additional perturbations cause some part of EM emanation leakage to change when inferring an adversarial example.
%Spectrograms of different classes (inputs) may activate different neurons of the classifiers, leading to different logits outputs, which can be fed to the anomaly detector for detecting adversarial examples.
The GradCAM heatmaps also guide the defender to improve the EM segments' pre-processing by selecting the significant frequency bands.

\begin{figure}[htb]
    \centering
    \subcaptionbox{Spectrogram0}[.315\linewidth][c]{%
    \includegraphics[width=\linewidth]{imgs/spectrogram0.png}}\quad
    \subcaptionbox{Spectrogram1}[.295\linewidth][c]{%
    \includegraphics[width=\linewidth]{imgs/spectrogram1.png}}\quad
    \subcaptionbox{Spectrogram5}[.3\linewidth][c]{%
    \includegraphics[width=\linewidth]{imgs/spectrogram5.png}}\quad
    \subcaptionbox{GradCAM0}[.31\linewidth][c]{%
    \includegraphics[width=\linewidth]{imgs/spectrogramCAM0.png}}\quad
    \subcaptionbox{GradCAM1}[.28\linewidth][c]{%
    \includegraphics[width=\linewidth]{imgs/spectrogramCAM1.png}}\quad
    \subcaptionbox{GradCAM5}[.32\linewidth][c]{%
    \includegraphics[width=\linewidth]{imgs/spectrogramCAM5.png}}\quad
    \caption{The average spectrogram and GradCAM results of the first EM segment from Class 0, 1, 5}
        % The figure(a)(b)(c) are  of samples from corresponding classes.
    % The figure(d)(e)(f) are those desired classes' GradCAM results, indicating which part of image the model is looking to for make the decision.
    % The model focus on the high light envelope of spectrograms to make its decision.}
    \Description{The sample spectrogram and GradCAM reuslts of the first batch from Class 0, 1, 5.
    The figure(a)(b)(c) are the average spectrogram of samples from corresponding classes.
    The figure(d)(e)(f) are those desired classes' GradCAM results, indicating which part of image the model is looking to for make the decision.
    The model focus on the high light envelope of spectrograms to make its decision.}
    \label{fig: gradCAM}
    \vspace{-0.3cm}
\end{figure}
