% ****** Start of file apssamp.tex ******
%
%   This file is part of the APS files in the REVTeX 4.2 distribution.
%   Version 4.2a of REVTeX, December 2014
%
%   Copyright (c) 2014 The American Physical Society.
%
%   See the REVTeX 4 README file for restrictions and more information.
%
% TeX'ing this file requires that you have AMS-LaTeX 2.0 installed
% as well as the rest of the prerequisites for REVTeX 4.2
%
% See the REVTeX 4 README file
% It also requires running BibTeX. The commands are as follows:
%
%  1)  latex apssamp.tex
%  2)  bibtex apssamp
%  3)  latex apssamp.tex
%  4)  latex apssamp.tex
%
\documentclass[%
 reprint,
%superscriptaddress,
%groupedaddress,
%unsortedaddress,
%runinaddress,
%frontmatterverbose, 
%preprint,
%preprintnumbers,
%nofootinbib,
%nobibnotes,
%bibnotes,
 amsmath,
 amssymb,
 aps,
 prl,
%pra,
%prb,
%rmp,
%prstab,
%prstper,
floatfix,
]{revtex4-2}
\usepackage{graphicx}% Include figure files
\usepackage{dcolumn}% Align table columns on decimal point
\usepackage{bm}% bold math

\usepackage{hyperref}% add hypertext capabilities
%\usepackage[mathlines]{lineno}% Enable numbering of text and display math
%\linenumbers\relax % Commence numbering lines

%\usepackage[showframe,%Uncomment any one of the following lines to test 
%%scale=0.7, marginratio={1:1, 2:3}, ignoreall,% default settings
%%text={7in,10in},centering,
%%margin=1.5in,
%%total={6.5in,8.75in}, top=1.2in, left=0.9in, includefoot,
%%height=10in,a5paper,hmargin={3cm,0.8in},
%]{geometry}

\begin{document}

\preprint{APS/123-QED}

\title{Turbulence in Magnetic Reconnection Jets from Injection to Sub-Ion Scales}

\author{Louis Richard}
 \email{louis.richard@irfu.se}
\affiliation{%
Swedish Institute of Space Physics, Uppsala, Sweden}
\affiliation{Department of Physics and Astronomy, Space and Plasma Physics, Uppsala University, Sweden 
}%

\author{Luca Sorriso-Valvo}
\affiliation{
CNR/ISTP – Istituto per la Scienza e la Tecnologia dei Plasmi, Bari, Italy
}
\affiliation{
Swedish Institute of Space Physics, Uppsala, Sweden
}%
\affiliation{
Space and Plasma Physics, School of Electrical Engineering and Computer Science, KTH Royal Institute of Technology, Stockholm, Sweden
}%

\author{Emiliya Yordanova}
\affiliation{
Swedish Institute of Space Physics, Uppsala, Sweden
}%

\author{Daniel B. Graham}
\affiliation{
Swedish Institute of Space Physics, Uppsala, Sweden
}%

\author{Yuri V. Khotyaintsev}
\affiliation{
Swedish Institute of Space Physics, Uppsala, Sweden
}%

\date{\today}

\begin{abstract}
We investigate turbulence in magnetic reconnection jets in Earth’s magnetotail using data from the Magnetospheric Multiscale spacecraft. We show that fully developed turbulence is observed in many reconnection jets. The observed turbulence develops on the time scale of a few ion gyroperiods, resulting in intermittent multifractal energy cascade from the characteristic scale of the jet down to the ion scales. We show that at ion scales, the turbulence transitions to a kinetic-Alfvén wave turbulence with decreased intermittency. The observed energy transfer rate across the inertial range is the largest reported for space plasmas so far. 
\end{abstract}

\maketitle

The interplay between the two ubiquitous phenomena, magnetic reconnection, and turbulence, is a long-standing problem in collisionless plasmas~\cite{lazarian_turbulent_2015}. Magnetic reconnection is a process that provides energization and acceleration of plasma through explosive topological reconfiguration of the magnetic field~\cite{biskamp_magnetic_2000,yamada_magnetic_2010}. It is responsible for the generation of fast plasma flows (jets) as observed, for example, in solar flares~\cite{masuda_loop-top_1994}, black hole flares~\cite{ripperda_black_2022} and planetary magnetotails~\cite{yamada_magnetic_2010}. In the reconnection region, turbulence and wave growth due to kinetic processeses~\cite{khotyaintsev_collisionless_2019} can in turn affect the dynamics of the magnetic reconnection through anomalous resistivity~\cite{drake_formation_2003,graham_direct_2022}. On the other hand, turbulence is a universal process that transfers kinetic and magnetic energy from large injection scales to small scales through an energy cascade produced by non-linear interactions among fluctuations~\cite{biskamp_magnetohydrodynamic_2003,frisch_turbulence_1995}. If the turbulence is fully developed, such energy transfer is globally scale invariant over a range of scales, the inertial range, where large-scale forcing and small-scale dissipation can be neglected. This produces power-law scaling of statistical quantities, such as the power spectral density and the moments of the scale-dependent fluctuations~\cite{frisch_turbulence_1995}. However, inhomogeneity of the energy transfer results in intermittency, i.e., formation of spatially concentrated structures such as current sheets and vortices~\cite{karimabadi_coherent_2013} where dissipation occurs~\cite{matthaeus_turbulence_2021}.

Numerical simulations show that turbulence develops in reconnection jets, resulting in the formation of secondary reconnection sites~\cite{lapenta_secondary_2015}. \textit{In-situ} spacecraft observations in reconnection jets suggest development of turbulence~\cite{voros_bursty_2006,huang_observations_2012,osman_multi-spacecraft_2015,jin_characteristics_2022}, forming current sheets where energy is dissipated~\cite{osman_multi-spacecraft_2015,fu_intermittent_2017,pucci_properties_2017}. However, the transient nature of reconnection jets observed by spacecraft yields short samples of field fluctuations, and thus, it is difficult to achieve statistical convergence. As a result, the complex interplay between magnetic reconnection and turbulence in reconnection jets remains unclear. In addition, in collisionless plasmas, the interaction of the particles with spatially concentrated structures provides efficient particle heating and acceleration~\cite{dalena_test-particle_2014,zhdankin_electron_2019,lemoine_first-principles_2022}, and results in deviations from the thermodynamic equilibrium Maxwellian velocity distribution functions ~\cite{servidio_local_2012,del_sarto_pressure_2016,zhdankin_generalized_2022,sorriso-valvo_turbulence-driven_2019}. Hence, the description of the non-linear energy transfer from injection to sub-ion scales is crucial to understand the energization of the content of collisionless plasma jets.

In this Letter, we use data from the Magnetospheric Multiscale (MMS) spacecraft \cite{burch_magnetospheric_2016} in the terrestrial magnetosphere to investigate turbulence in reconnection jets. In particular, we study 330 plasma jets in the plasma sheet of Earth's magnetotail ($\beta_i \geq 0.5$, where $\beta_i = n_i k_B T_i / P_{mag}$, $n_i$ is the ion number density, $T_i$ the ion temperature and $P_{mag} = B^2/2\mu_0$ is the magnetic pressure)~\cite{richard_are_2022}.  The jets are observed relatively close ($\approx 3.5~R_E$, with $R_E\approx 6371~\mathrm{km}$ the Earth's radius) to the statistical location of the reconnection X-line in the magnetotail at $X_{GSM}\approx -25~R_E$ in geocentric solar magnetospheric (GSM) coordinates~\cite{nagai_solar_2005}, indicating that they are likely to be reconnection outflows rather than generated by other mechanisms (e.g., kinetic ballooning/interchange~\cite{pritchett_kinetic_2010}). We use magnetic field measurements from MMS's FGM instrument \cite{russell_magnetospheric_2016} and electric field measurements from the EDP instrument \cite{lindqvist_spin-plane_2016,ergun_axial_2016}. The moments of the ion and electron velocity distributions are measured by the FPI instrument \cite{pollock_fast_2016} with corrections removing low-energy photo-electrons and a background ion population to account for penetrating radiation \cite{gershman_systematic_2019}.

Figure \ref{fig:example} presents an example of a fast ($|\mathbf{V}_i|\geq300~\mathrm{km}~\mathrm{s}^{-1}$ with $\mathbf{V}_i$ the ion bulk velocity) Earthward jet [Fig.~\ref{fig:example}c] observed on May 28, 2017. We observe enhanced fluctuations in the magnetic field $\mathbf{B}$ and electric field $\mathbf{E}$ [Fig.~\ref{fig:example}a-b]. Their power spectra [Fig.~\ref{fig:example}d] exhibit a Kolmogorov-like power-law scaling, $|\mathbf{k}_\perp|^{-5/3}$~\cite{kolmogorov_local_1941} in a range spanning from the injection scale, estimated as the correlation scale, $l_c=38\rho_i=3.7~R_E$ (see Supplemental Material~\cite{supplemental_material}), down to the ion gyroradius, $\rho_i=v_{Ti}/\omega_{ci}\approx 619~\mathrm{km}$. Here, we use temporal-to-spatial scale equivalence $l_\perp=\langle V \rangle \tau$ with $V=|\mathbf{V}_i|$, after verifying the Taylor hypothesis's validity and the assumption of anisotropic electromagnetic fluctuations (see Supplemental Material~\cite{supplemental_material}). At sub-ion scales, the magnetic field spectrum steepens to $|\mathbf{k}_\perp|^{-2.8}$ while the electric field spectrum rises to $|\mathbf{k}_\perp|^{-0.8}$, reminiscent of sub-ion scale turbulence in the solar wind~\cite{alexandrova_universality_2009} and the Earth's magnetosheath~\cite{matteini_electric_2017}. The observed Kolmogorov spectrum suggests global scale-invariant energy transfer across the inertial range.

In fully developed turbulence, the power spectrum (equivalent to the second-order moment of the fluctuations) cannot describe the fluctuations fully~\cite{paladin_anomalous_1987}. Indeed, the formation of spatially concentrated structures results in scale-dependent, non-Gaussian deviations from the large-scale distribution of the field fluctuations, which can only be described using higher-order moments. Hence, we compute the structure functions of the magnetic field $S_{m}(\tau)=\langle |\Delta \mathbf{B}(\tau)|^m \rangle = \langle | \mathbf{B}(t+\tau) - \mathbf{B}(t)|^m \rangle$ with $\tau$ the time scale and $\langle \cdot \rangle$ the ensemble time average, having verified ergodicity and statistical convergence (see Supplemental Material~\cite{supplemental_material}). We observe a clear $S_m(\tau)\propto \tau^{\zeta(m)}$ scaling at large scales $l>d_i$ [Fig.~\ref{fig:example}e], which confirms the global scale-invariant nature of the fluctuations of the field. In addition, the flatness $\mathcal{F}(\tau)=S_4(\tau)/S_2^2(\tau)$, which measures the wings of the distribution of $\Delta \mathbf{B}(\tau)$, i.e., the occurrence of large gradients, is monotonically increasing as the scale decreases [Fig.~\ref{fig:example}f] indicating intermittency \cite{frisch_turbulence_1995}. This means that the energy is transferred inhomogeneously across the inertial range. 

\begin{figure}
\includegraphics[width=\linewidth]{figure_1.pdf}% Here is how to import EPS art
\caption{\label{fig:example}Example of a reconnection jet with fully developed turbulence. (a) Magnetic field, (b) electric field, 
and (c) ion bulk velocity in GSM coordinates; (d) power spectral density of the electromagnetic fields normalized to $P_0=P(f_{d_i})$, where $f_{d_i}=\langle V \rangle/2\pi d_i$; (e) structure functions and (f) flatness of the magnetic field; (g) mixed third-order moment $-Y(\tau)$, Eq.~\ref{eq:pp98}. The dotted lines in panels (d)-(g) indicate the correlation scale $l_c$. The dashed-dotted lines in panels (d) and (e)-(g) indicate $|k_\perp|\rho_i=1$ and $\rho_i$, respectively. The dashed lines are reference slopes in panel (d), and fit in panels (f)-(g).}
\end{figure}


\begin{figure*}
\includegraphics[width=\linewidth]{figure_2.pdf}% Here is how to import EPS art
\caption{\label{fig:sea}Superposed analysis of the 24 reconnection jets. (a) Magnetic field, electric field, and electron number density power spectra; (b) inertial range scaling exponents of the structure functions $S_m(\tau)\propto \tau^{\zeta(m)}$; (c) flatness $\mathcal{F}(\tau)=S_4(\tau)/S_2^2(\tau)$; (d) phase speed in the plasma frame $V_{ph}^{\mathrm{jet}} = |\delta \mathbf{E}_\perp|/|\delta \mathbf{B}_\perp| - \langle V \rangle$ normalized to the Alfv\'en speed; (e) sub-ion range scaling exponents of the structure functions $S_m(\tau)\propto \tau^{\zeta(m)}$; (f) energy cascade rate. The thin transparent lines in panels (a)-(f) show the individual rescaled cases, and the solid lines show the ensemble averages. The thick green lines in panels (b) and (e) are fitted $p$ model in the inertial and sub-ion ranges, respectively. The thick green line in panel (d) is the prediction for KAW from~\cite{stasiewicz_small_2000}. The dashed orange lines in panel (f) indicate the standard deviation over the 24 cases. The dotted and dashed-dotted lines indicate the correlation scale $l_c$ and the ion gyroradius, respectively.}
\end{figure*}

Knowledge of the energy transfer rate by the turbulence cascade across the scales is crucial to understand the energy budget in the reconnection jets. Here, we estimate the energy cascade rate in the incompressible single-fluid magnetohydrodynamic (MHD) framework using the Politano-Pouquet law~\cite{politano_von_1998} of linear scaling of the mixed third-order moments $Y^{\pm}(\tau)$ under the assumption of homogeneity, scale separation, isotropy, and time stationarity,

\begin{equation}
    \label{eq:pp98}
    Y^{\pm} = \langle |\Delta \mathbf{Z}^\pm (t, \tau)|^2 \Delta \mathbf{Z}_l^\mp(t, \tau)\rangle = -\frac{4}{3} \varepsilon^\pm \langle V\rangle \tau \, ,
\end{equation}

where $\mathbf{Z}^\pm = \mathbf{V}_i \pm \mathbf{B}/\sqrt{\mu_0n_im_i}$ are the Elsasser fields and $\varepsilon^\pm$ is the energy cascade rate. Since Eq.~\ref{eq:pp98} is obtained from first-principle MHD equations, the linear scaling of the energy flux is the only formal definition of fully developed turbulence~\cite{marino_scaling_2023}. In addition, the sign of the total energy flux $Y=\left ( Y^+ + Y^-\right )/2$ indicates a direct energy cascade from large to small scales if $Y < 0$ or inverse energy cascade if $Y>0$~\cite{marino_scaling_2023}. The total energy flux $Y$ [Fig.~\ref{fig:example}g] shows a linear scaling in $\tau$ with $Y<0$, indicating that the energy cascades across the inertial range at a constant rate $\varepsilon=\left ( \varepsilon^+ + \varepsilon^-\right )/2$. In particular, the total energy transfer rate is $\varepsilon= 1.5_{-0.4}^{+1.9}\times 10^9~\mathrm{J}~\mathrm{kg}^{-1}~\mathrm{s}^{-1}$. To our knowledge, this is the largest energy transfer rate ever measured \textit{in-situ}~\cite{osman_anisotropic_2011,hadid_compressible_2018,sorriso-valvo_turbulence-driven_2019,bandyopadhyay_observation_2020}.

Our analysis of the statistical properties of turbulence in this example reconnection jet indicates that the energy is inhomogeneously transferred across a well-defined inertial range. In particular, the linear scaling of the mixed third-order moment is compelling evidence of fully developed turbulence. To our knowledge, this is the first observation of the Politano-Pouquet law in magnetized ($\beta_i\approx2.6$) reconnection jets. 

To provide a complete systematic description of the turbulence in magnetotail reconnection jets, we form an ensemble of 24 cases that show clear signatures of fully developed turbulence in the statistical sense introduced in the example. In the other 306 out of 330 cases, it is unclear if turbulence is fully developed, ongoing, absent/suppressed, or statistical convergence is not achieved. We analyze the properties of the ensemble average of the 24 reconnection jets [Fig.~\ref{fig:sea}] after normalizing the spatial scales to the ion inertial length $d_i$ to account for the variability of the plasma conditions. This procedure results in a robust reference sample of turbulence in reconnection jets [Fig.~\ref{fig:sea}]. As in the example described in Fig.~\ref{fig:example}, the magnetic field $\delta \mathbf{B}$, electric field $\delta \mathbf{E}$, and electron number density $\delta n_e$ power spectra [Fig.~\ref{fig:sea}a] show a clear inertial range from the injection scale $l_c$ to the ion scales with a spectral exponent $-1.72\pm 0.03$ close to the Kolmogorov value. The injection scale is $l_c\sim 10\rho_i\approx 3~R_E$ [Fig.~\ref{fig:histograms}a] consistent with the typical dimension of the reconnection jet across the flow~\cite{nakamura_spatial_2004}. This suggests that turbulence in the reconnection jet is generated by its relative motion with respect to the ambient plasma.

We estimate the energy injection and cascade rates to examine the energy balance in turbulent reconnection jets. Assuming that the energy injection rate in the system corresponds to the decay rate of the energy-containing eddies~\cite{matthaeus_evolution_1994}, the former can be estimated using the von K\'arm\'an energy decay law~\cite{politano_von_1998,wan_von_2012} $\varepsilon_{vK}^\pm = -\mathrm{d} |\mathbf{Z}^\pm|^2/\mathrm{d}t = \alpha^\pm |\mathbf{Z}^\pm|^2|\mathbf{Z}^\mp| / l_c^\pm$, with $l_c^\pm$ the correlation length  of $\mathbf{Z}^\pm$ and $\alpha^\pm=4C_\epsilon^\pm /9\sqrt{3}\approx 0.03$~\cite{linkmann_nonuniversality_2015}. On the other hand, using the ensemble signed average of $\varepsilon$ from Eq.~\ref{eq:pp98}, we estimate the average energy cascade rate $\langle \varepsilon \rangle \approx 2.1_{-0.5}^{+0.9}\times 10^8~\mathrm{J}~\mathrm{kg}^{-1}~\mathrm{s}^{-1}$, positive and nearly constant over the inertial range, which is a compelling signature of fully developed turbulence. The energy cascade rate is consistent with the ensemble average total von K\'arm\'an energy decay rate $\langle\varepsilon_{vK}\rangle=(\langle\varepsilon_{vK}^+\rangle + \langle\varepsilon_{vK}^-\rangle)/2=1.1_{-0.3}^{+1.9}\times 10^8~\mathrm{J}~\mathrm{kg}^{-1}~\mathrm{s}^{-1}$ indicating that the turbulent energy transfer to smaller scales balances the energy injection. This indicates that the energy released by magnetic reconnection in the form of strongly driven plasma jets is efficiently transferred to sub-ion scales through non-linear interactions. As a result, as seen in the power spectra [Fig.~\ref{fig:sea}a], there is no energy accumulation across the inertial range.

\begin{figure}
\includegraphics[width=\linewidth]{figure_3.pdf}% Here is how to import EPS art
\caption{\label{fig:histograms}Histograms of (a) the correlation scales and (b) the lifetime of the jet. Blue triangles correspond to the 24 reconnection jets studied here, and the red diamonds to the entire dataset.}
\end{figure}

To understand how the energy is spatially distributed across the cascade and obtain a quantitative description of intermittency, we analyze the high-order moments of the fluctuations using the ensemble of reconnection jets. We observe that the flatness [Fig.~\ref{fig:sea}c] monotonically increases as the scale decreases from $l_c$ to $\rho_i$ indicating intermittent spatially inhomogeneous energy cascade in the inertial range. Using the structures functions $S_m(\tau)$ up to the order $m=6$, we find that the scaling exponents $\zeta(m)/\zeta(3)$ [Fig.~\ref{fig:sea}b] show a clear non-linear monotonic increase, indicating a multifractal energy cascade~\cite{frisch_turbulence_1995}. Here, we normalize the scaling exponents to $\zeta(3)=1.1 \pm 0.3~\mathrm{nT}^3~\mathrm{s}^{-1}$ to account for deviations from Kolmogorov's prediction $\zeta(3)=1 ~\mathrm{nT}^3~\mathrm{s}^{-1}$~\cite{benzi_extended_1993}. It can be shown~\cite{frisch_singularity_1985} that $\zeta(m)$ is equivalent to the singularity spectrum of the Hausdorff dimension $D(h)$ of the set of singularities by the Legendre transformation $D(h)=\min_m(mh-\zeta(m) + 1)$ with $h$ the Hölder exponent [Fig.~\ref{fig:sea}b inset]. To quantitatively estimate the spatial inhomogeneity of the energy cascade, we fit the observed $\zeta(m)$, with $1\leq m\leq 4$, to the multifractal $p$-model~\cite{meneveau_simple_1987}

\begin{equation}
    \label{eq:p-model}
    \frac{\zeta(m)}{\zeta(3)} = 1 - \log_2\left [p^{mh} + (1-p)^{mh}\right ]\, ,
\end{equation}

where $p\in [0.5, 1]$ is the intermittency parameter and $h=(\alpha - 1) / 2$ with $\alpha = \zeta(2) / \zeta(3) + 1$ the spectral slope [Fig.~\ref{fig:sea}a]. We observe that the scaling exponents $\zeta(m)$ [Fig.~\ref{fig:sea}b] and the singularity spectrum [Fig.~\ref{fig:sea}b inset] are well described by the $p$-model, with the intermittency parameter $p=0.756 \pm 0.001$ larger than values reported in the solar wind~\cite{carbone_cascade_1993} and hydrodynamic laboratory experiments~\cite{meneveau_simple_1987}. This indicates that, in the reconnection jets, the energy cascades from the injection scale to the ion scales in a multifractal, strongly inhomogeneous manner.

The statistical results described above provide strong evidence that an inertial range energy cascade is developed in this ensemble of reconnection jets. However, linear micro-instabilities~\cite{gary_theory_1993} can also play an important role in the energy budget in reconnection jets~\cite{hietala_ion_2015}. Hence, one may ask how fast the non-linear interactions develop compared with linear instabilities. Assuming Alfv\'enic outflow and statistical location of the X-line at $X_{GSM}\approx -25~R_E$~\cite{nagai_solar_2005}, the lifetime $\tau_l$ of the reconnection jet reads $\tau_l f_{ci0}=(2\pi)^{-1}\delta x_l / d_i$, where $\delta x_l$ is the distance from X-line to spacecraft and $f_{ci0}=e B_0/2\pi m_i$ is the ion gyrofrequency in the background field $B_0=|\mathbf{B}_{ext}|/2$~\cite{sergeev_current_2003}, with $\mathbf{B}_{ext}=\mathbf{B}\sqrt{1 + \beta_i}$ obtained from the pressure balance assumption. From Fig.~\ref{fig:histograms}b, it is clear that the distribution of our ensemble of reconnection jets mirrors that of the entire dataset of 330 cases. However, none of the 31 jets observed within $\tau_lf_{ci0} \leq 1$ from the X-line showed signatures of fully developed turbulence. In particular, the median lifetime of the turbulent jets is $\tau_lf_{ci}\approx7.2_{-3.2}^{7.8}$ so that $\delta x_l/d_i\approx 45_{-20}^{49}$. Development of turbulent fluctuations at similar distances ($>30 d_i$)  from the reconnection X-line has been observed in simulations~\cite{higashimori_ion_2015}. This suggests that the non-linear energy cascade can reach a fully developed state already after a few ion gyroperiods, which is much faster than the growth rate ($\gamma = 10^{-2}-10^{-1}\omega_{ci0}$) of the micro-instabilities~\cite{hietala_ion_2015}. Thus, in the reconnection jets, the energy transfer across the inertial range is dominated by non-linear interactions rather than micro-instabilities~\cite{bandyopadhyay_interplay_2022}. 

At scales $l\ll \rho_i$, kinetic processes will affect the energy transfer so that the turbulence can transition to a kinetic Alfv\'en wave (KAW) cascade~\cite{schekochihin_astrophysical_2009}, a whistler cascade~\cite{stawicki_solar_2001}, or a Hall MHD cascade~\cite{galtier_wave_2006}. Kinetic-scale waves in the magnetotail plasma jets have been suggested to be consistent with KAW~\cite{chaston_correction_2012}. To investigate the sub-ion scales energy transfer, we estimate the phase speed of the electromagnetic fluctuations in the plasma frame $V_{ph}^{\mathrm{jet}} = |\delta \mathbf{E}_\perp|/ |\delta \mathbf{B}_\perp| - \langle V \rangle$ and compare with the prediction for KAW~\cite{stasiewicz_small_2000} with $|\mathbf{k}_\perp |= |\mathbf{k}|$ having checked the anisotropy of the electromagnetic field fluctuations (see Supplemental Material~\cite{supplemental_material}). The phase speed of the electromagnetic fluctuations shows a clear dispersive behavior, $V_{ph}^{\mathrm{jet}}/V_A\propto |\mathbf{k}_\perp |$, in excellent agreement with the prediction for KAW, providing evidence that at sub-ion scales the turbulence transitions to a KAW energy cascade.

At sub-ion scales, we also observe that the scaling exponents $\zeta(m)$ show a weakly non-linear monotonic increase with $m$ [Fig.~\ref{fig:sea}e]. In addition, the flatness monotonically increases as the scale decreases [Fig.~\ref{fig:sea}c], indicating intermittent energy transfer at sub-ion scales. However, using the multifractal $p$-model of energy cascade, Eq.~\ref{eq:p-model}, with $\alpha=2.8$ yields an intermittency parameter $p=0.568 \pm 0.005$ that is close to mono-fractal ($p=0.5$), similar to what is found in solar wind observations~\cite{kiyani_global_2009}. The decreased intermittency at the sub-ion scales suggests that the micro-instabilities substantially contribute to the energy transfer in this range.

We have presented a complete systematic statistical description of fully developed turbulence in reconnection jets. We find that the energy is injected at the characteristic scale of the jet and transferred across the inertial range through non-linear interactions. In particular, the average energy transfer rate is $\langle \varepsilon \rangle = 2.1_{-0.5}^{+0.9}\times 10^8~\mathrm{J}~\mathrm{kg}^{-1}~\mathrm{s}^{-1}$, which makes reconnection jets the strongest driver of turbulence observed so far in space plasmas \cite{marino_scaling_2023}. We showed that the turbulence transitions to a KAW cascade at sub-ion scales, where the energy can be dissipated into plasma heating ~\cite{osman_multi-spacecraft_2015,fu_intermittent_2017}  through, e.g., stochastic heating and 
Landau damping~\cite{gershman_wave-particle_2017,liang_ion_2017}. 
As a result of the plasma heating, the gyroradii of the particles increase so that they can interact with the inertial-range fluctuations at progressively larger scales~\cite{dalena_test-particle_2014}. Eventually, the supra-thermal particles are accelerated by the large-scale electric field of the jet~\cite{richard_proton_2022}. Thus, the jet-generated turbulence is a staircase for seed particles to climb in energy. This scenario could explain the observation of supra-thermal ion gamma-ray flares at nebula~\cite{tavani_discovery_2011} and active galactic nuclei~\cite{shukla_gamma-ray_2020}. 
Our results also provide new insights into the interplay between turbulence and magnetic reconnection. We show that reconnection outflows drive a strong turbulent cascade, which is an essential part of the fast turbulent MHD reconnection model~\cite{lazarian_turbulent_2015} and can be relevant to the generation of solar wind turbulence~\cite{zhao_turbulent_2022} by reconnection in the solar corona~\cite{drake_switchbacks_2021,raouafi_magnetic_2023}.
 
MMS data are available at the MMS Science Data Center; see Ref. \footnote{See \url{https://lasp.colorado.edu/mms/sdc/public}.}. Data analysis was performed using the \verb+pyrfu+ analysis package \footnote{See \url{https://pypi.org/project/pyrfu/}.}. 

We thank the MMS team and instrument PIs for data access and support. This work was supported by the Swedish National Space Agency (SNSA) Grants 139/18 and 145/18, and by the Swedish Research Council (VR) Research Grant 2022-03352.

\bibliographystyle{apsrev4-2}
\bibliography{main}% Produces the bibliography via BibTeX.

\end{document}
%
% ****** End of file apssamp.tex ******
