% ****** Start of file apssamp.tex ******
%
%   This file is part of the APS files in the REVTeX 4.2 distribution.
%   Version 4.2a of REVTeX, December 2014
%
%   Copyright (c) 2014 The American Physical Society.
%
%   See the REVTeX 4 README file for restrictions and more information.
%
% TeX'ing this file requires that you have AMS-LaTeX 2.0 installed
% as well as the rest of the prerequisites for REVTeX 4.2
%
% See the REVTeX 4 README file
% It also requires running BibTeX. The commands are as follows:
%
%  1)  latex apssamp.tex
%  2)  bibtex apssamp
%  3)  latex apssamp.tex
%  4)  latex apssamp.tex
%
\documentclass[%
 reprint,
%superscriptaddress,
%groupedaddress,
%unsortedaddress,
%runinaddress,
%frontmatterverbose, 
%preprint,
%preprintnumbers,
%nofootinbib,
%nobibnotes,
%bibnotes,
 amsmath,amssymb,
 aps,
 prl,
%pra,
%prb,
%rmp,
%prstab,
%prstper,
%floatfix,
]{revtex4-2}
\usepackage{mathtools}
\usepackage{graphicx}% Include figure files
\usepackage{dcolumn}% Align table columns on decimal point
\usepackage{bm}% bold math
%\usepackage{hyperref}% add hypertext capabilities
%\usepackage[mathlines]{lineno}% Enable numbering of text and display math
%\linenumbers\relax % Commence numbering lines

%\usepackage[showframe,%Uncomment any one of the following lines to test 
%%scale=0.7, marginratio={1:1, 2:3}, ignoreall,% default settings
%%text={7in,10in},centering,
%%margin=1.5in,
%%total={6.5in,8.75in}, top=1.2in, left=0.9in, includefoot,
%%height=10in,a5paper,hmargin={3cm,0.8in},
%]{geometry}

\begin{document}

\preprint{APS/123-QED}

\title{Supplemental Material for ``Turbulence in Magnetic Reconnection Jets from Injection to Sub-Ion Scales''}

\author{Louis Richard}
 \email{louis.richard@irfu.se}
\affiliation{%
Swedish Institute of Space Physics, Uppsala, Sweden}
\affiliation{Department of Physics and Astronomy, Space and Plasma Physics, Uppsala University, Sweden 
}%

\author{Luca Sorriso-Valvo}
\affiliation{
CNR/ISTP – Istituto per la Scienza e la Tecnologia dei Plasmi, Bari, Italy
}
\affiliation{
Swedish Institute of Space Physics, Uppsala, Sweden
}%
\affiliation{
Space and Plasma Physics, School of Electrical Engineering and Computer Science, KTH Royal Institute of Technology, Stockholm, Sweden
}%

\author{Emiliya Yordanova}
\affiliation{
Swedish Institute of Space Physics, Uppsala, Sweden
}%

\author{Daniel B. Graham}
\affiliation{
Swedish Institute of Space Physics, Uppsala, Sweden
}%

\author{Yuri V. Khotyaintsev}
\affiliation{
Swedish Institute of Space Physics, Uppsala, Sweden
}%

\date{\today}

\maketitle

%\tableofcontents
\section{Introduction}
We present a detailed analysis of the validity of the various assumptions used throughout the Letter. In particular, we focus here on the validity of the Taylor hypothesis, the assumption of anisotropic electromagnetic fluctuations, the ergodicity theorem, and the statistical convergence. This analysis, presented here for the example case in the Letter, was performed on all of the 24 cases selected (see text in Letter).

\section{Taylor hypothesis and wavevector anisotropy}
\subsection{Taylor hypothesis}
\label{ssec:taylor}
The transformation from temporal to spatial scales using the Taylor hypothesis is crucial to compare our observations with models of turbulence. In space plasmas, the Taylor hypothesis is satisfied if $V_f> V_A$ where $V_f=\langle |\mathbf{V}_i|\rangle$ is the flow velocity, and $V_A$ is the Alfv\'en velocity, i.e., the plasma convection is much faster than the propagation of the electromagnetic field fluctuations which are effectively frozen in to the flow. In particular, in the solar wind and magnetosheath, the Taylor hypothesis is largely known to hold. On the other hand, in the reconnection outflow, $V_f\leq V_A$~\cite{parker_sweets_1957,haggerty_reduction_2018}. However, observations reported that even for sub-Alfv\'enic flows, the Taylor hypothesis appears to hold~\cite{voros_bursty_2006,stawarz_observations_2016,bandyopadhyay_observation_2020}. Here, in order to verify that the Taylor hypothesis is valid, we apply the multi-spacecraft interferometry~\cite{graham_electrostatic_2016,graham_universality_2019} to the magnetic field $\mathrm{B}$ ($f < 64~\mathrm{Hz}$) with a spacecraft separation $|\Delta \mathbf{r}|=66~\mathrm{km}=0.15d_i$.



We plot, for the example [Fig.~1 in the Letter], the constructed dispersion relation $f/f_{d_i}$ versus $|\mathbf{k}|d_i$ in Figure~\ref{fig:multispaceraft-fk}a where $f$ is the frequency measured in the spacecraft frame and $f_{d_i}=\langle V_i\rangle / 2\pi d_i$. We observe that across all scales, the maximum wave power is along $|\mathbf{k}|d_i=f/f_{d_i}$, so that $|\mathbf{k}| = 2\pi f/\langle V_i \rangle$ or $\lambda = \langle V_i\rangle \tau$ meaning that the Taylor hypothesis is valid. 

\begin{figure}[!h]
    \centering
    \includegraphics[width=\linewidth]{figure_si-1.pdf}
    \caption{Joint frequency-wavenumber spectrum in the spacecraft frame. The wavenumber is normalized to the ion inertial length $d_i=c/\omega_{pi}$ and the frequency is normalized to the Taylor transformed ion inertial length $f_{d_i}=\langle V_i\rangle / 2\pi d_i$.}
    \label{fig:multispaceraft-fk}
\end{figure}

\subsection{Wavevector anisotropy}
In order to compare observations with predictions from the inertial range energy cascade ~\cite{schekochihin_astrophysical_2009} and kinetic Alfv\'en wave turbulence in the sub-ion range~\cite{stasiewicz_small_2000}, we used the assumption of $|\mathbf{k}|=|\mathbf{k}_\perp| \gg k_\parallel$. Turbulence is known to be strongly anisotropic in the solar wind~\cite{sahraoui_three_2010} and in the magnetosheath~\cite{sahraoui_anisotropic_2006}. The $k$-filtering technique~\cite{pincon_local_1991} applied to Cluster data in a reconnection outflow showed that in the ion diffusion region $k_\parallel\gg |\mathbf{k}_\perp|$~\cite{eastwood_observations_2009}, while further from the reconnection region $k_\parallel\ll |\mathbf{k}_\perp|$~\cite{huang_observations_2012}. In order to verify that the assumption is valid, we apply the multi-spacecraft interferometry~\cite{graham_electrostatic_2016,graham_universality_2019} to the magnetic field $\mathrm{B}$ ($f < 64~\mathrm{Hz}$) with a spacecraft separation $|\Delta \mathbf{r}|=66~\mathrm{km}=0.15d_i$.

\begin{figure}[!h]
    \centering
    \includegraphics[width=\linewidth]{figure_si-2.pdf}
    \caption{Magnetic field wave power in the ($|\mathbf{k}_\perp|d_i$, $k_\parallel d_i$) space. The dashed lines indicates $k_\parallel=|\mathbf{k}_\perp|^{2/3}$.}
    \label{fig:multispaceraft-kk}
\end{figure}

Fig.~\ref{fig:multispaceraft-kk} presents the magnetic field wave power in the $(|\mathbf{k}_\perp|d_i, k_\parallel d_i)$ binned space. We see that for the example presented in Fig.~1 in the Letter, the wave-power in the $(|\mathbf{k}_\perp|\rho_i,k_\parallel\rho_i)$ space peaks for $k_\parallel\ll |\mathbf{k}_\perp|$ meaning that the assumption of anisotropic electromagnetic fluctuations is valid.

\section{Correlation scale and ergodicity}
In order to estimate the correlation scale, under the Taylor frozen-in hypothesis, we use $l_c=\langle V\rangle \tau_c$ where $l_c$ is the correlation scale, $\langle V \rangle$ the average flow velocity and the correlation time $\tau_c=(\tau_c^+ + \tau_c^-) / 2$ where $\tau_c^\pm$ is the $e$-folding time~\cite{smith_heating_2001} of the trace of the auto-correlation function of the Elsasser variables $R^\pm (\tau) = \langle \mathbf{Z}^\pm(t). \mathbf{Z}^\pm(t+\tau)\rangle_T$, with $\langle \cdot\rangle_T$ denoting the ensemble time average and $\mathbf{Z}^\pm = \mathbf{V_i}\pm \mathbf{B}^2/2\mu_0$ with $\mathbf{V}_i$ the ion bulk velocity. 

\begin{figure}[!h]
    \centering
    \includegraphics[width=\linewidth]{figure_si-3.pdf}
    \caption{Auto-correlation function of (a) $\mathbf{Z}^+$ and (b) $\mathbf{Z}^-$. The dashed lines indicate the $e$-folding time and the orange line the corresponding decaying exponential.}
    \label{fig:auto-corr}
\end{figure}

Fig.~\ref{fig:auto-corr} presents the autocorrelation function of the Elsasser variables $\mathbf{Z}^\pm$ for the example [Fig.~1 in the Letter]. We observe that the auto-correlation function is well-fitted by a decaying exponential. In particular, $\tau_c=44~\mathrm{s}$, with $\tau_c^+=22~\mathrm{s}$ and $\tau_c^-=66~\mathrm{s}$, so that the reconnection jet interval 2017-05-28T00:35:26.553 - 2017-05-28T00:38:58.054 UT contains $4.8~\tau_c$, hence the ergodicity theorem is satisfied.

\section{Statistical convergence}
In order to provide a reliable statistical description of the turbulence, we must first ensure convergence of the moments of the probability distribution function (PDF) of the magnetic field and velocity increments $\Delta \mathbf{Z}^\pm (\tau) = \mathbf{Z}^\pm (t+\tau) - \mathbf{Z}^\pm (t)$ using Taylor frozen-in hypothesis. We tested the convergence of the moments of the PDF against several tests~\cite{dudok_de_wit_can_2004,kiyani_extracting_2006}. Here, we present the more restrictive test we used. 

 One can show that for a finite sample size, the $m$th order moment of the increments diverges if $m\gamma>1$ with $\gamma$ the scaling index of the ranked distribution of $\Delta \mathbf{Z}^\pm$~\cite{dudok_de_wit_can_2004}. Hence, moments of the PDF of the increments of the Elsasser variables are only meaningful up to the order~\cite{dudok_de_wit_can_2004}
 
\begin{equation}
    m_{max}=\left \lfloor\frac{1}{\gamma}\right \rfloor - 1 \, .
\end{equation}

\begin{figure}[!h]
    \centering
    \includegraphics[width=\linewidth]{figure_si-4.pdf}
    \caption{Statistical convergence test following~\cite{dudok_de_wit_can_2004}. Ranked increments (a) $\Delta \mathbf{Z}^-_l$ and (b) $\Delta \mathbf{Z}^+_l$.}
    \label{fig:convergence}
\end{figure}

We compute $\Delta \mathbf{Z}^\pm (\tau)$ for $\tau=\langle V_i\rangle /d_i$. We see that the ranked distribution of  $\Delta \mathbf{Z}^\pm (\tau)$ behaves as a power law up to $10^2$. In particular, a fit with Levenberg-Marquart least square fitting method yields $\gamma=0.125\pm 0.002$ for  $\Delta \mathbf{Z}^-$ and $\gamma=0.132\pm 0.002$ for  $\Delta \mathbf{Z}^+$. Hence, $m_{max} = 6$, which is the maximum order of moments that can be meaningfully assessed (e.g., to estimate $p$ from the multi-fractal $p$-model~\cite{meneveau_simple_1987}). 

\bibliographystyle{apsrev4-2}
\bibliography{main}% Produces the bibliography via BibTeX.

\end{document}