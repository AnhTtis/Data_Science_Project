\documentclass[preprint,
superscriptaddress,
 amsmath,amssymb,
 aps,
 prb,
]{revtex4-1}

\usepackage{graphics}
\usepackage{hyperref}
\usepackage{amssymb}
\usepackage{amsmath,mathtools}
\usepackage{xcolor,colortbl}
\usepackage[framemethod=TikZ]{mdframed}

\newcommand{\angstrom}{\textup{\AA}}
\newcommand{\ve}[1]{\boldsymbol{#1}} % Uncomment for BOLD vectors.
\newcommand{\ud}{\mathrm{d}}
% \pdfsuppresswarningpagegroup=1

\begin{document}
\title{Photoemission Orbital Tomography for Excitons in Organic Molecules}


\author{C. S. Kern}
\author{A. Windischbacher}
\author{P. Puschnig}
\email[E-mail address: ]{peter.puschnig@uni-graz.at}
\affiliation{Institute of Physics, NAWI Graz, University of Graz, 8010 Graz, Austria}

\begin{abstract}
Driven by recent developments in time-resolved photoemission spectroscopy, we extend the successful method of photoemission orbital tomography (POT) to excited states. Our theory retains the intuitive orbital picture of POT, while respecting both the entangled character of the exciton wave function and the energy conservation in the process. Analyzing results from three organic molecules, we classify generic exciton structures and give a simple interpretation in terms of natural transition orbitals. We validate our findings by directly simulating pump-probe experiments with time-dependent density functional theory.
\end{abstract}

\date{\today}
\maketitle 

\emph{Introduction} ---
In the past decade, photoemission orbital tomography (POT)~\cite{Puschnig2009a,Dauth2011,Nguyen2015,Woodruff2016,Puschnig2017,Kliuiev2019} has emerged as a powerful technique that relates the measured photoemission angular distribution (PAD) from oriented films of organic molecules with the molecular orbitals from which the electron has been emitted. This direct connection arises from modeling photoelectrons by plane waves. While this plane-wave assumption has been debated~\cite{Bradshaw2015,Egger2018} and, in fact, demonstrated to be insufficient in certain circumstances~\cite{Dauth2016a,Metzger2020}, POT has led to valuable insights, for instance, into the hybridization between organic layers and the substrate~\cite{Ziroff2010,Zamborlini2017,Yang2022}, the geometry of adsorbate layers~\cite{Graus2016,Kliuiev2019,Hurdax2022}, the nature of reaction products~\cite{Haags2020} or real space images of orbitals~\cite{Luftner2013,Weiss2015,Kliuiev2016a,Graus2019,Jansen2020}. Particularly the latter aspect has also stimulated discussions on how to build a formal bridge between quantum mechanical wave functions and the experimentally observed momentum space distributions~\cite{Truhlar2019,Krylov2020}.

Despite these numerous achievements, until very recently, POT could only be applied to study occupied molecular orbitals by photoexciting electrons from the ground state. With the advent of laser high-harmonic generation and free-electron lasers, it has become possible to study also the dynamics of excited states in time- and angle-resolved photoemission spectroscopy (trARPES) experiments. On the one hand, band structure movies of crystalline solids have shown the temporal evolution of the electron system over the complete Brillouin zone~\cite{Rohwer2011,Eich2017,Nicholson2018}. 
On the other hand, for molecular systems, optically excited states, involving transitions from HOMO to LUMO, have recently been observed with trARPES on their intrinsic temporal and spatial scales~\cite{Wallauer2020,Baumgartner2022,Neef2022}.
In general, however, an exciton is an entangled state composed of multiple electron-hole transitions, for which a theoretical foundation of POT is still lacking. 


The aim of this Letter is to fill this gap and establish a consistent framework that allows us to interpret measured PAD maps from excited states in terms of exciton wave functions. Specifically, we assume that the exciton wave function is represented in a product basis of valence and conduction states, as typically done when solving the electron-hole Bethe-Salpeter equation (BSE)~\cite{Rohlfing2000} or Casida's equation in time-dependent density functional theory (TDDFT)~\cite{Casida1995,Onida2002}. Expanding the concept of Dyson orbitals~\cite{Pickup1977,Krylov2020,Ortiz2020} to excited states, we arrive at the result that the PAD can be interpreted as the Fourier transform of a coherent sum of the electronic part of the exciton wave function. These relations, as well as the unexpected consequences of the hole for the measured kinetic energy spectrum, are illustrated for generic cases of exciton compositions in a series of organic molecules in the gas phase. We further show how exciton photoemission can be interpreted in terms of the well-known concept of natural transition orbitals (NTOs)~\cite{Martin2003} and, underpinning our findings, the PAD is also simulated directly by means of a TDDFT approach where no assumptions on the final state are made whatsoever.

\emph{Theory} ---
Excitons are the fundamental opto-electronical excitations in organic molecules and non-metallic solids. For such bound electron-hole pairs, 
we assume that the
wave function of the $m$-th exciton, with excitation energy $\Omega_m$, can be expanded in the single-particle electron $\{\chi_c(\ve{r}_e)\}$ and hole basis $\{\phi_v(\ve{r}_h)\}$ as
\begin{equation}
\label{eq:quasiparticle}
\psi_m(\ve{r}_h, \ve{r}_e) = \sum_{v,c} X^{(m)}_{vc} \phi_v^*(\ve{r}_h) \chi_c(\ve{r}_e).
\end{equation}
Here, the sum runs over all pairs of valence and conduction states $\{v,c\}$, respectively, and $X^{(m)}_{v c}$ is the transition density matrix that describes the character of the exciton.
Note that here and in the following derivations, we use the Tamm-Dancoff approximation~\cite{Benedict1989} for better readability. In the general case and in our calculations, however, we also consider de-excitations.

Our goal is to find a consistent expression that connects the exciton wave function as defined in Eq.~\ref{eq:quasiparticle} with measured photoemission momentum maps. Following Ref.~\cite{Dauth2014}, we describe the photoelectron probability with Fermi's golden rule as the transition from an initial $N$-particle state $\Psi_\mathrm{i}^N$ to a final state $\Psi_\mathrm{f}^N$, triggered by the photon field $\ve{A}$ with energy $\omega$. We couple this classical field to the electrons' momenta $\ve{P}$ in the dipole approximation and use the Coulomb gauge as well as Hartree atomic units unless stated otherwise.
Denoting the energy of $\Psi_\mathrm{i}^N$ and $\Psi_\mathrm{f}^N$ with $E_i$ and $E_f$ respectively, the photoelectron probability is
\begin{equation}
\label{eq:goldenrule}
W_{\mathrm{i} \rightarrow \mathrm{f}} = 2 \pi \left| \left\langle \Psi_\mathrm{f}^N \right| \ve{A} \ve{P} \left| \Psi_\mathrm{i}^N \right\rangle \right|^2
 \delta\left(\omega + E_\mathrm{i} - E_\mathrm{f} \right).
\end{equation}
In contrast to our earlier work on photoemission from the electronic ground state  $\Psi_\mathrm{i,0}^N$~\cite{Dauth2014}, now the initial state is given by the  the $m$-th exciton which can also be expressed as
\begin{equation}
\label{eq:initialstate}
\left| \Psi_{\mathrm{i},m}^N \right\rangle = \sum_{v,c} X^{(m)}_{vc}
a^\dagger_c a_v 
\left|\Psi_\mathrm{i,0}^N \right\rangle.
\end{equation}
Here, $a_v$ and $a^\dagger_c$ create a hole and an electron in state $v$ and $c$, respectively. Its energy $E_{\mathrm{i},m}^N$ is the sum of the ground state energy $E_{\mathrm{i},0}^N$ and the excitation energy $\Omega_m$.

For the final state $\Psi_\mathrm{f}^N$, one commonly assumes the sudden approximation, in which the correlation between the emitted electron $\gamma_{\ve{k}}$ and the remaining system can be neglected~\cite{Damascelli2004}. 
We thus identify the total energy of this final state  as the sum of $E_{\mathrm{f},j}^{N-1}$ and the photoelectron's kinetic energy, $E_\mathrm{kin} = k^2 / 2$. Then the energy conservation from Eq.~\ref{eq:goldenrule} demands
\begin{eqnarray}
E_\mathrm{kin} = \omega - (E_{\mathrm{f},j}^{N-1} - E_{\mathrm{i},0}^N) + \Omega_m = \omega - \varepsilon_j + \Omega_m,
\end{eqnarray}
where we have introduced the $j$-th ionization potential $\varepsilon_j$ as the energy difference between the $N$ electron ground state and the $j$-th excited state of the $(N-1)$-electron system. 
In taking the overlap between the latter two wave functions, we utilize the Dyson orbital for electron detachment, $D_{j,m}$, in the usual way~\cite{Pickup1977}, see Appendix for details. This reduces the matrix element of Eq.~\ref{eq:goldenrule} to an integral over a single coordinate only:
\begin{eqnarray}
\label{eq:me2}
% \mathcal{M}_{m, j}(\ve{k}) & = &   
\left\langle \Psi_{\mathrm{f},j}^N \right| \ve{A} \ve{P} \left| \Psi_{\mathrm{i},m}^N \right\rangle & = &
\ve{A} \int \ud^3 r \, \overline{\gamma}_{\ve{k}}(\ve{r}) \, \ve{p} \, D_{j,m}(\ve{r}) \nonumber \\
& \approx & (\ve{A} \ve{k}) \, \mathcal{F}\left[ D_{j,m} \right] (\ve{k}).
\end{eqnarray}
In the second line, we make use of the plane-wave approximation, $\gamma_{\ve{k}}(\ve{r}) \propto \mathrm{e}^{\mathrm i \ve{k} \ve{r}}$, that is inherent to POT~\cite{Puschnig2009a,Puschnig2017} and that naturally introduces
the Fourier transform of the Dyson orbital, modulated by a weakly angle-dependent polarization factor. It is generally accepted that Dyson orbitals represent the most appropriate way to describe photoemission in a single-orbital picture~\cite{Oana2007,Dauth2014,Gozem2015,Truhlar2019,Krylov2020}, however, their computation as correlated wave functions from multi-reference methods~\cite{Oana2009,Vidal2020} is often not feasible. Therefore -- and to preserve the single-particle picture that has been so fruitful in POT for later -- we approximate $\Psi_\mathrm{f,j}^{N-1}$ and $\Psi_\mathrm{i,0}^N$ by single Slater determinants each. With this assumption, and the derivation shown in Sec.~A of the appendix, we obtain the simple result that the Dyson orbital reduces to a coherent sum over all unoccupied orbitals $\chi_c$, 
\begin{equation}
\label{eq:dyson2}
D_{j,m}(\ve{r}) = \sum_{c} X^{(m)}_{j c}  \chi_{c}(\ve{r}).
\end{equation}
Importantly, only the $j$-th row of the transition density matrix $X^{(m)}_{vc}$ contributes, thereby fixing the final hole position in the orbital $\phi_j$. Finally, the photoemission angular distribution arising from the $m$-th exciton is obtained by summing over all possible final state hole configurations as follows
\begin{eqnarray}
\label{eq:finalintensity}
I_m(\ve{k}) & \propto  & 
\left| \ve{A} \ve{k} \right|^2
\sum_{j} \left| \sum_{c} X^{(m)}_{j c}  \mathcal{F}\left[\chi_{c} \right] (\ve k)\right|^2 \nonumber \\
& \times &  \delta\left(\omega - E_\mathrm{kin} - \varepsilon_j + \Omega_m \right).
\end{eqnarray}
From this expression, which we refer to as ''exPOT'' (exciton POT) in the remainder of this Letter, we expect the photoemission signal from a general exciton to have contributions at multiple kinetic energies that are in concordance with the energy conservation and thus depend on the hole's position after electron detachment described by the ionization energy $\varepsilon_j$. At each allowed kinetic energy, momentum maps take the form of a Fourier transform of the \emph{coherent} sum over unoccupied states, weighted by the corresponding transition density matrix elements. While this quantity can be readily computed from a BSE or Casida calculation, we want to stress that physical intuition about the character of the exciton can be enhanced by introducing NTOs~\cite{Martin2003,Krylov2020}, which can also be used in our formalism, as derived in Sec.~B of the appendix, and which will be demonstrated in the following.

\emph{Simple cases} ---
Before presenting our numerical results, we explain the implications of Eq.~\ref{eq:finalintensity} on the example of four prototypical exciton structures  schematically depicted in Figure~\ref{figure1}. 
In case (i), the exciton involves only a single transition from the highest occupied orbital $\phi_1$ to the lowest unoccupied orbital $\chi_1$, which is, in fact, a common case for the lowest exciton in some organic molecules~\cite{Wallauer2020}. Evidently, exPOT predicts that the observed PAD is given by the Fourier transform of $\chi_1$ appearing at the kinetic energy $E_\mathrm{kin}  = \omega - \varepsilon_1 + \Omega_1$, where $\omega$ is the probe photon energy, $\varepsilon_1$ the ionization potential corresponding to $\phi_1$, and $\Omega_1$ denotes the exciton energy, i.e. the pump photon energy. This is also illustrated in the bottom part of Figure~\ref{figure1}.
\begin{figure}
\begin{center}
\includegraphics[width=0.6\columnwidth]{fig1.pdf}
\caption{Four prototypical exciton structures and the corresponding predictions of exPOT for the observed PAD maps as detailed in the text.}
\label{figure1}
\end{center}
\end{figure}

For case (ii), we assume the exciton wave function as $\psi = \frac{1}{\sqrt{2}}(\phi_2 \chi_1 + \phi_1 \chi_2)$.
Here, Eq.~\ref{eq:finalintensity} leads to PAD maps of both $\chi_1$ and $\chi_2$, however, appearing at kinetic energies corresponding to the ionization levels of $\phi_2$ and $\phi_1$, respectively, as also illustrated in Figure~\ref{figure1}. 
Note that this exciton represents a truly entangled state~\cite{Plasser2016} which can also be seen after transformation to the NTO basis (see table in Sec.~B of the appendix).
The situation is somewhat different for case (iii), where $\psi = \frac{1}{\sqrt{2}}(\phi_2 \chi_1 + \phi_1 \chi_1)$. Here, we expect to observe two identical PADs appearing at two different kinetic energies, depending on whether, after the electron has been emitted, the hole resides in state $\phi_1$ or $\phi_2$. Finally in case (iv), the exciton is described by $\psi = \frac{1}{\sqrt{2}}(\phi_1 \chi_1 + \phi_1 \chi_2)$ and Eq.~\ref{eq:finalintensity} suggests that the PAD is proportional to the Fourier transform of a \emph{coherent} sum of the unoccupied orbitals $\chi_1$ and $\chi_2$, which appears at $E_\mathrm{kin}  = \omega - \varepsilon_1 + \Omega_1$. In the following, we want to give examples for the non-trivial cases (ii)--(iv) by actual numerical simulations.

\emph{Numerical results} ---
Let us now compare the predictions of our exPOT approach for organic molecules with computationally more demanding, but accurate TDDFT calculations as implemented in the real-space code OCTOPUS~\cite{Andrade2015,Tancogne-Dejean2020}. Here, photoemission spectra and PAD maps are obtained by recording the flux of photoelectron density through a detector surface (t-SURFF)~\cite{Wopperer2017,DeGiovannini2017}, which seamlessly allows for pump-probe setups and where no assumptions on the final state need to be made. 

For a better comparability of the two theoretical approaches, exPOT vs. t-SURFF, we take several precautions. 
First, we focus on planar molecules for which the plane wave approximation (PWA) has already been well tested~\cite{Lueftner2014,Kliuiev2019}.
Second, we choose the probe field in $z$-direction, that is perpendicular to the molecular plane, which is also known to minimize  possible deficiencies of the PWA~\cite{Dauth2016}. 
Third, we ensure that pump pulses are long enough to only excite the specific exciton in question, since for ultrashort pulses considerable energy broadening needs to be taken into account~\cite{Popova2016,Reuner2023}. Equivalently, we keep our probe pulses long enough for a resonable kinetic energy resolution in the spectra and choose probe energies in the XUV regime for the sake of the sudden approximation~\cite{Hammon2021}.
Fourth, we limit ourselves to the adiabatic local density approximation (ALDA) since more advanced functionals, such as hybrids, would be computationally too demanding for the real-time propagation utilized for the t-SURFF method. We emphasize, however, that for the exPOT method in general the latter restriction is not necessary.

\begin{figure}
\begin{center}
\includegraphics[width=0.6\columnwidth]{fig2.pdf}
\caption{\textbf{Comparison of exPOT with results from t-SURFF for TCNQ.} (a) total angle-integrated photoelectron intensity from t-SURFF (grey) and its projection onto the HOMO ($v$=1, green), HOMO-1 ($v$=2, orange) and HOMO-2 ($v$=3, blue) states, with corresponding kinetic energy positions $\omega - \varepsilon_j$ indicated by the vertical dashed lines in the same colors. Red arrows mark the energy of the pump pulse $\omega_{\mathrm{pump}}$, full vertical lines $\omega - \varepsilon_j + \omega_{\mathrm{pump}}$ respectively. (b) PAD maps from t-SURFF at the kinetic energies indicated by the full vertical lines in panel (a). 
(c) PAD maps  obtained from the exPOT approach with the first three NTOs.}
\label{figure2}
\end{center}
\end{figure}
With the aim to find real-life examples for the cases (ii)--(iv) outlined above, we have selected three prototypical $\pi$-conjugated molecules, namely tetracyanoquinodimethane (TCNQ), porphine and perylenetetracarboxylic dianhydride (PTCDA), and perform linear-response TDDFT calculations within the Casida formalism in OCTOPUS. 
For TCNQ, the solution reveals an exciton with $\Omega_m=6.76$~eV 
which is strongly allowed for $y$-polarization (molecular geometry and choice of axis are depicted in Sec.~C of the appendix). Its exciton wave function has major contributions from $\phi_3 \chi_2$ (0.44), $\phi_2 \chi_3$ (0.35) and $\phi_1 \chi_6$ (0.07) (see Appendix for more details). Thus it represents an entangled state as in case (ii).
In the t-SURFF calculations, we set the pump energy $\omega_\mathrm{pump} = \Omega_m$ and employ a probe energy of $\omega=35$~eV (details in Appendix~D). The resulting kinetic energy spectrum of the emitted electrons is depicted in panel (a) of Figure~\ref{figure2}. It is dominated by emissions from the three highest occupied orbitals $\phi_1$, $\phi_2$ and $\phi_3$ indicated by the green, orange and blue dashed vertical lines, respectively. Importantly, however, we also observe three emission peaks at kinetic energies larger by precisely $\omega_\mathrm{pump}$. This behavior, already qualitatively illustrated in the second column of Figure~\ref{figure1}, is in perfect accordance with the energy conservation of Eq.~\ref{eq:finalintensity}.
Despite the orders of magnitude smaller peak heights for the exciton emission, we obtain three distinct PAD maps (at the kinetic energies marked by vertical full lines), which are displayed in panel (b). Comparing with our exPOT theory, indeed, the Fourier transforms of the first three NTOs of this entangled exciton, as depicted in panel (c), are in very good agreement with the PAD maps from t-SURFF.


Next, we present our results for the optical excitation in porphin at $\Omega_m=3.94$~eV in $x$-direction, which serves as an example for case (iii) defined in Figure~\ref{figure1}. 
From the t-SURFF calculation, we obtain two identical momentum maps at the kinetic energies corresponding to the hole in state $\phi_1$ and $\phi_4$ (left and middle column of panel (a) in Fig.~\ref{figure3}). Note that here, in contrast to the above PADs from TCNQ, we have projected the t-SURFF ARPES intensities on the respective ground-state orbitals, since the total photoelectron yield is also affected by other contributions which are not relevant for our case (Appendix~D). 
Casida's approach leads to almost equal contributions of $\phi_1 \chi_2$ (0.27) and $\phi_4 \chi_2$ (0.25) to the exciton wave function, which can be written as a single NTO $\widetilde{\chi}_1$ (see SM) resulting in the PAD depicted in the rightmost column of panel (a) in Fig.~\ref{figure3}.
The excellent agreement with the corresponding t-SURFF maps further validates the exPOT predictions. Remarkably, while a single NTO might be enough to explain photoemission from an excited state of such character, it can be comprised of contributions from different valance states, which then leads to photoemission signatures of the same conduction state at different kinetic energies. 
\begin{figure}
\begin{center}
\includegraphics[width=0.6\columnwidth]{fig3.pdf}
\caption{\textbf{Comparison of exPOT with results from t-SURFF for porphin and PTCDA.} (a) PADs for porphin from t-SURFF at different kinetic energies (left and middle column) compared to the exPOT map of the first NTO (right column). (b) Different methods for PTCDA, showing contributions from $v=4$ (top row) and $v=8$ (bottom row), see text for details.}
\label{figure3}
\end{center}
\end{figure}

Conversely, in case (iv), we consider an exciton structure with transitions involving only a single hole state $v$ but multiple conduction states $c$.
For PTCDA at an excitation energy of $\Omega_m=4.45$ eV (polarized in $y$-direction), we encounter even two such transitions: $\phi_8 \chi_4$ (0.29), $\phi_8 \chi_3$ (0.03) and $\phi_4 \chi_8$ (0.06), $\phi_4 \chi_2$ (0.06). 
The top row of panel (b) of Fig.~\ref{figure3} is devoted to the contribution from $v=8$, with the state-projected result from t-SURFF in the left column agreeing very well with the exPOT result in the middle column, evaluated with the contribution from $\widetilde{\chi}_2$ only. Importantly, the computation of the latter implicitly involves a \emph{coherent} sum over the unoccupied states $\chi_4$ and $\chi_3$, while wrongly performing an \emph{incoherent} summation clearly worsens the agreement with the t-SURFF reference (see right panel labeled I. S.).
The second major set of contributions to this exciton, $\phi_4 \chi_8$ and $\phi_4 \chi_2$, leads to a PAD at the kinetic energy corresponding to $\varepsilon_4$ and is shown in the bottom row of panel (b). Again, the t-SURRF result (left column) agrees well with exPOT (middle column). This time however, we need to take into account a sum over multiple NTOs ($\widetilde{\chi}_{\lambda}$) while the PAD from a single NTO ($\widetilde{\chi}_1$) is not sufficiently accurate. This is due to the fact that, in general, the electron or hole contributions can contribute to different NTOs and only the coherent sum over $\lambda$ is equivalent to the coherent sum of Eq.~\ref{eq:finalintensity} (see also comparison of PADs in the SM, Sec. S4). In summary, we have not only proven excellent agreement of the exPOT theory with \emph{ab-initio} simulations for case (iv), but could also emphasize the necessity of the \emph{coherent} superposition of the electron orbitals for such a case.

\emph{Conclusions} ---
We demonstrate an extension of photoemission orbital tomography to excited states, termed exPOT, and thereby provide the theoretical foundations to interpret photoemission angular distributions maps as measured in pump-probe ARPES experiments of organic molecular layers in terms of exciton wave functions. 
We illustrate the consequences of exPOT on the example of three  organic molecules, covering a range of prototypical exciton structures, and validate our findings by TDDFT calculations that directly incorporate the pump and probe fields. 
In our method, the simplicity of the orbital interpretation can be retained by identifying Fourier-transformed NTOs as the observables in photoemission of excitons. The evaluation of the ARPES intensity, however, demands a coherent sum over electron contributions to reflect the entangled character of an exciton wave function, as well as an incoherent sum over hole contributions to fulfill energy conservation.
While in this Letter we have restricted ourselves to organic molecules in the gas phase and, moreover, to calculations from linear-response TDDFT, we remark that the extension of exPOT to periodic systems and to a more sophisticated treatment of electron-hole correlations using the GW/BSE approach is straight-forward. 

\begin{acknowledgments}
The authors want to thank Wiebke Bennecke, G. S. Matthijs Jansen and Stefan Mathias for helpful discussions. This work was supported by the Austrian  Science Fund project I 4145 and the Doctoral Academy NanoGraz. We further acknowledge computational recources at the Vienna Scientific Cluster.
\end{acknowledgments}


\appendix


\section{Approximations for the Dyson orbital}
\label{sec:dyson}
% \subsection{\label{subsec:dyson}}
In this section, we derive the expression for the Dyson orbitals in terms of single-particle orbitals, which ultimately lead to Eq.~(7) of the main text. In addition, a brief discussion on the implications of the necessary approximations is given. 
In a first step, we assume the many-body initial ground-state $\Psi_{\mathrm i, 0}^N$ to be represented by a \textit{single} Slater determinant with a suitable one-electron basis for the occupied orbitals, $\left\{\phi_v\right\}$:
\begin{align}
 \left\langle \mathbf r \middle| 
 \Psi_{\mathrm i, 0}^N \right\rangle = 
\frac{1}{\sqrt{N!}}
\left|
\begin{array}{cccc}
\phi_{N}(\ve{r}_1)  &  \phi_{N}(\ve{r}_2)  & \cdots & \phi_{N}(\ve{r}_{N}) \\
\vdots            &     \vdots         &        &  \vdots        \\
\phi_1(\ve{r}_1)  &  \phi_1(\ve{r}_2)  & \cdots & \phi_1(\ve{r}_{N})
\end{array}
\right|.
% =: \left| \phi_1 \phi_2 \dots \phi_N \right|.
\end{align} 
Labeling the unoccupied part of our one-electron basis by $\{\chi_c\}$ and assuming the exciton wave function to be representable in our basis as in Eq.~(1) of the main text, we can write the the $m$-th excited initial state, $\Psi_{\mathrm{i},m}^N$, as 
\begin{align}
\label{eq:m_initial_slater}
\left\langle \mathbf r \middle|
\Psi_{\mathrm{i},m}^{N}\right\rangle = 
\frac{1}{\sqrt{N!}} \sum_{v,c} X^{(m)}_{vc}
\left|
\begin{array}{cccc}
\chi_{c}(\ve{r}_1)  &  \chi_{c}(\ve{r}_2)  & \cdots & \chi_{c}(\ve{r}_{N}) \\
\phi_{N}(\ve{r}_1)  &  \phi_{N}(\ve{r}_2)  & \cdots & \phi_{N}(\ve{r}_{N}) \\
\vdots            &     \vdots         &        &  \vdots        \\
\phi_{v+1}(\ve{r}_1)  &  \phi_{v+1}(\ve{r}_2)  & \cdots & \phi_{v+1}(\ve{r}_{N}) \\
\phi_{v-1}(\ve{r}_1)  &  \phi_{v-1}(\ve{r}_2)  & \cdots & \phi_{v-1}(\ve{r}_{N}) \\
\vdots            &     \vdots         &        &  \vdots        \\
\phi_1(\ve{r}_1)  &  \phi_1(\ve{r}_2)  & \cdots & \phi_1(\ve{r}_{N})
\end{array}
\right|. 
%= \sum_{\{v,c\}} X^{(m)}_{vc}
%\left| \phi_1 \phi_2 \dots \phi_{v-1} \phi_{v+1} \dots \phi_N \chi_c\right|.
\end{align}
Here, an electron has been removed from the state $v$ and promoted to state $c$, and the transition density matrix $X_{v c}^{(m)}$ provides the weights of a given electron-hole pair for the $m$-the exciton.
Using the language of second quantization, this state can also be written with the help of fermionic creation- and annihilation operators for electrons and holes, acting on the ground state $\Psi_\mathrm{i,0}^N$:
\begin{equation}
\label{eq:initialstate}
\left| \Psi_{\mathrm{i},m}^N \right\rangle = \sum_{v,c} X^{(m)}_{vc}
a^\dagger_c a_v 
\left|\Psi_\mathrm{i,0}^N \right\rangle.
\end{equation}

For the final state of the photoemission process, we neglect the correlation between the outgoing photoelectron with wave function $\gamma_{\ve{k}}$ and the remaining $N-1$ electron wave function $\Psi_{\mathrm{f},j}^{N-1}$ and assume orthogonality between the two. 
With the help of the operator $\mathcal{A}$, which properly anti-symmetrizes the total $N$-electron wave function, $\Psi_{\mathrm{f},j,\ve{k}}^{N}$ can be written as:
\begin{equation}
\label{eq:finalstate}
\Psi_{\mathrm{f},j,\ve{k}}^{N} = \mathcal{A} \Psi_{\mathrm{f},j}^{N-1} \gamma_{\ve{k}}.
\end{equation}
Like before, we write $\Psi_{\mathrm f, j}^{N-1}$ as a single slater determinant
\begin{align}
\label{eq:final_slater}
\left\langle \mathbf r \middle|
\Psi_{\mathrm f,j}^{N-1} \right\rangle= 
\frac{1}{\sqrt{(N-1)!}}
\left|
\begin{array}{cccc}
\phi_{N}(\ve{r}_1)  &  \phi_{N}(\ve{r}_2)  & \cdots & \phi_{N}(\ve{r}_{N-1}) \\
\vdots            &     \vdots         &        &  \vdots        \\
\phi_{j+1}(\ve{r}_1)  &  \phi_{j+1}(\ve{r}_2)  & \cdots & \phi_{j+1}(\ve{r}_{N-1}) \\
\phi_{j-1}(\ve{r}_1)  &  \phi_{j-1}(\ve{r}_2)  & \cdots & \phi_{j-1}(\ve{r}_{N-1}) \\
\vdots            &     \vdots         &        &  \vdots        \\
\phi_1(\ve{r}_1)  &  \phi_1(\ve{r}_2)  & \cdots & \phi_1(\ve{r}_{N-1})
\end{array}
\right|, 
% = \left| \phi_1 \phi_2 \dots \phi_{j-1} \phi_{j+1} \dots \phi_N \right|,
\end{align}
where an electron has been removed from the $j$-th state. Its is important to note that here we have assumed that the one-particle basis for this $(N-1)$-electron state, $\left\{\phi_v\right\}$, is the same as the one used for the $N$-electron initial state. Thus, electronic relaxation effects upon photoionization are neglected.  In terms of an annihilation operator acting on the initial ground state, this $N-1$-electron part of the final state can then be denoted as
\begin{equation}
\label{eq:finalstate2}
\left| \Psi_{\mathrm{f},j}^{N-1} \right\rangle = a_j
\left|\Psi_{\mathrm i,0}^N \right\rangle.
\end{equation}

Next, we introduce the Dyson orbital~\cite{Ortiz2020, Krylov2020} as the overlap between the $N$-electron wave function of the initial state and the $N-1$-electron wave function of the final state:
\begin{align}
\label{eq:dyson1}
 D_{j,m}(\mathbf r) = 
 \sqrt{N} \int \mathrm d^3 r_2 \cdots \mathrm d^3 r_N \;
 \overline{\Psi}_{\mathrm{i},m}^N \left(\ve{r},\ve{r}_2, \dots \ve{r}_N \right)
 \Psi_{\mathrm{f},j}^{N-1} \left( \ve{r}_2, \dots \ve{r}_N\right)
\end{align}
For the sake of a more compact mathematical derivation to be shown below, we rewrite this expression for the Dyson orbital with the help of creation operators, 
\begin{align}
\label{eq:dyson2}
 D_{j,m}(\mathbf r) =
 \sum_{v'} \left\langle \Psi_{\mathrm{i},m}^N
 \middle| a_{v'}^{\dagger} \middle|
 \Psi_{\mathrm{f},j}^{N-1} \right\rangle \phi_{v'} (\mathbf r)
 +\sum_{c'} \left\langle \Psi_{\mathrm{i},m}^N
 \middle| a_{c'}^{\dagger} \middle|
 \Psi_{\mathrm{f},j}^{N-1} \right\rangle \chi_{c'} (\mathbf r)
 ,
\end{align}
where, owing to the conjunction of the single particle basis set in $\{\phi_v'\}$ and $\{\chi_c'\}$, we have to sum over both for the expansion of the $j$-th Dyson orbital of the $m$-th exciton. Inserting Eqs.~\ref{eq:initialstate} and~\ref{eq:finalstate2} into Eq.~\ref{eq:dyson2}, we get
\begin{align}
 D_{j,m}(\mathbf r) =
 \sum_{v'} \sum_{v,c} X_{v c}^{(m)} \left\langle \Psi_\mathrm{i,0}^N 
 \middle| a_v^{\dagger} a_c a_{v'}^{\dagger} a_j \middle|
 \Psi_{\mathrm i,0}^N \right\rangle \phi_{v'} (\mathbf r)
 +\sum_{c'} \sum_{v,c} X_{v c}^{(m)} \left\langle \Psi_\mathrm{i,0}^N 
 \middle| a_v^{\dagger} a_c a_{c'}^{\dagger} a_j \middle|
 \Psi_{\mathrm i,0}^N \right\rangle \chi_{c'} (\mathbf r),
\end{align}
of which all integrals in the sum over $v'$ vanish due to orthogonality. In the sum over $c'$, we get no contributions for $c \ne c'$ by the same argument, thus we arrive at our final result:
\begin{align}
\label{eq:coherent_sum}
 D_{j,m}(\mathbf r) &=
 \sum_{v,c} X_{v c}^{(m)} \left\langle \Psi_\mathrm{i,0}^N 
 \middle| a_v^{\dagger} a_c a_{c}^{\dagger} a_j \middle|
 \Psi_{\mathrm i,0}^N \right\rangle \chi_{c} (\mathbf r) = \nonumber \\
 &=
 \sum_{c} X_{j c}^{(m)} \left\langle \Psi_\mathrm{i,0}^N 
 \middle| a_j^{\dagger} a_j a_c a_{c}^{\dagger} \middle|
 \Psi_{\mathrm i,0}^N \right\rangle \chi_{c} (\mathbf r) = \nonumber \\
 &=
 \sum_{c} X_{j c}^{(m)} \chi_{c} (\mathbf r).
\end{align}
Here, the index $j$ labels the valence state from which the photoelectron has been removed. This involves the approximation that the many-body wave function can be expressed as a \textit{single} Slater-determinant. While this simplification is necessary to retain both the orbital picture utilized in POT and the computational feasibility for organic molecules, it should be stressed that it is possible to compute Dyson orbitals from higher-order wave function methods, \cite{Ortiz2020, Pomogaev2021} although differences to e.g. Kohn-Sham orbitals seem to be rather minor. On the other hand, for small molecules, photoemission spectra from Dyson orbitals in conjunction with a Coulomb wave final state have proven to be in better agreement with photonenergy-dependent experimental data.\cite{Gozem2015}

\section{Natural transition orbitals}
\label{sec:NTO}

In this section, we give a short summary of the concept of natural transition orbitals (NTOs) and argue that they can in fact replace the coherent sum expression in Eq.~\ref{eq:coherent_sum}.
As outlined in the main part of this paper, a given exciton wave function $\psi(\ve{r}_h, \ve{r}_e)$ is conveniently expanded in a single-particle electron $\chi_c(\ve{r}_e)$ and hole basis  $\phi_v(\ve{r}_h)$ as
\begin{equation}
\label{eq:quasiparticle}
\psi(\ve{r}_h, \ve{r}_e) = \sum_{v, c} X_{vc} \phi_v^*(\ve{r}_h) \chi_c(\ve{r}_e),
\end{equation}
where the sum extends over all pairs of valence states $\{v\}$ and conduction states $\{c\}$. The coupling coefficients $X^{(m)}_{v c}$, also described as the transition density matrix, can, for instance, be obtained by treating the electron-hole correlation within the Bethe-Salpeter approach\cite{Rohlfing2000} or by solving Casida's equations in the framework of linear-response time-dependent density functional theory (TDDFT).\cite{Casida1995}
Either way, the coefficients $X_{vc}$ describe the contribution that each electron-hole pair makes to the excited state. Frequently, there is no dominant configuration, thereby making an intuitive interpretation of the excited state in terms of orbitals difficult.

To overcome these deficiency in the description of the exciton wave function, one seeks a more compact orbital representation for the electronic transition density matrix. By applying separate unitary transformations to the occupied orbitals $\phi_v$ and the virtual orbitals $\chi_c$, the correspondence between the excited electron and the empty hole can be obtained, without changing the physically relevant quantity, the transition density. As summarized below, this is accomplished by an orbital transformation leading to a set of natural transition orbitals,\cite{Martin2003,Krylov2020} which we denote in the following by $\widetilde{\phi}_\lambda$ and $\widetilde{\chi}_\lambda$.

Let us assume that there are $N_v$ active occupied orbitals  $\phi_v$, and a number of $N_c$ unoccupied (or virtual) orbitals $\chi_c$. Then, the transition density matrix $X_{vc}$ is a matrix with $N_v$ rows and $N_c$ columns, whose singular value decomposition can be written in the following way
\begin{equation}
\label{eq:svd}
X = V \, \Lambda \, C^T.
\end{equation}
Here, $V$ and $C$ are quadratic matrices of sizes $N_v \times N_v$ and $N_c \times N_c$, respectively, and the rectangular $N_v \times N_c$-matrix $\Lambda$ has only non-vanishing elements $\lambda_1, \lambda_2,\dots \lambda_{N_v}$ in the diagonal. These singular values are ordered according to their size, thus $\lambda_1 > \lambda_2 > \dots > \lambda_{N_v}$, and fulfill the following normalization condition
\begin{equation}
\sum_{i=1}^{N_v} \lambda_i^2 = 1.
\end{equation}
Note that we have assumed that $N_v < N_c$ as is typically the case in the calculation of optically excited states. By making use of the transformation
\begin{eqnarray}
\widetilde{\phi}_\lambda & = & \sum_{v=1}^{N_v} V_{\lambda v}^T \phi_v \\
\widetilde{\chi}_\lambda & = & \sum_{c=1}^{N_c} C_{\lambda c}^T \chi_c,
\end{eqnarray}
we obtain a new set of orbitals, the \emph{natural transition orbitals} $\widetilde{\phi}_\lambda$ and $\widetilde{\chi}_\lambda$, respectively, which can be used to express  the exciton wave function in an alternative and often more compact way:
\begin{equation}
\label{eq:NTOexciton}
\psi(\ve{r}_h, \ve{r}_e) = \sum_{\lambda=1}^{N_v}  \Lambda_{\lambda} \widetilde{\phi}_{\lambda}^*(\ve{r}_h) \widetilde{\chi}_{\lambda}(\ve{r}_e).
\end{equation}
The advantage of this representation of the exciton wave function over the one from Eq.~\ref{eq:quasiparticle}, where the sum runs over all $N_v \times N_c$ electron-hole pairs, is that the sum in Eq.~\ref{eq:NTOexciton}  extends only over the $N_v$ singular values. Moreover, typically only very few singular values contribute significantly to the summation allowing for a compact interpretation of the exciton in terms of only just a few transitions between the new set of NTOs  $\tilde{\phi}_{\lambda_i} \rightarrow \tilde{\chi}_{\lambda_i}$. 
Let us repeat here the final expression for the photoemission of an exciton (Eq. 7 in the main text) for convenience:
\begin{equation}
\label{eq:finalintensity}
I_m(\ve{k}) \propto  
\left| \ve{A} \ve{k} \right|^2
\sum_{j} \left| \sum_{c} X^{(m)}_{j c}  \mathcal{F}\left[\chi_{c} \right] (\ve k)\right|^2 
\cdot  \delta\left(\omega - E_\mathrm{kin} - \varepsilon_j + \Omega_m \right).
\end{equation}
We can now insert Eq.~\ref{eq:svd} into Eq.~\ref{eq:finalintensity} and -- by making use of the fact that the Fourier transform $\mathcal{F}$ is a linear operator -- we get
\begin{equation}
\label{eq:nto_intensity}
I_m(\ve{k}) \propto 
\left| \ve{A} \ve{k} \right|^2
\sum_{j} \left| \sum_{\lambda} V_{j \lambda} \Lambda_{\lambda} \mathcal{F}\left[\widetilde{\chi}_{\lambda} \right] (\ve k)\right|^2  
\cdot \delta\left(\omega + E_\mathrm{kin} - \varepsilon_j + \Omega_m \right).
\end{equation}
At first sight, it seems that we have not gained much: we have just replaced the summation over $c$ with the summation over $\lambda$ and replaced the prefactors. But, as stated above, a given exciton $m$ is often characterized by just a few NTOs and one can easily control the accuracy of its representation in terms of NTOs by introducing a threshold for the $\Lambda_{\lambda}$. Moreover, it is our believe that NTOs are useful when dealing with excitons, since the character of the transition is encoded in just a few single-particle orbitals and with introducing Eq.~\ref{eq:nto_intensity}, we can assign physical meaning to these orbitals as actual observables of the excited-state photoemission experiment.

\begin{table}
\caption{Transition density matrices $X_{vc}$ as well as their singular value decompositions $X = V \Lambda C^T$ for the four simple exciton structures defined in Fig.~1 of the main part. Additionally, the exciton wave functions $\psi$ are given in terms of the NTOs $\widetilde{\phi}$ and $\widetilde{\chi}$, respectively. {\label{tab:fourcases}}}
\begin{tabular}{ccccc}
\hline \hline
\\[-1em]
   &  (i)  &  (ii)  & (iii)    &  (iv) \\
\\[-1em]
\hline 
\\[-1em]
$X_{vc}=$ & 
$\left( \begin{array}{cc} 1 & 0 \\ 0 & 0 \end{array} \right)$ &
$\left( \begin{array}{cc} 0 & \frac{1}{\sqrt{2}} \\ \frac{1}{\sqrt{2}} & 0 \end{array} \right)$ &
$\left( \begin{array}{cc} \frac{1}{\sqrt{2}}  & 0 \\ \frac{1}{\sqrt{2}}  & 0 \end{array} \right)$ &
$\left( \begin{array}{cc} \frac{1}{\sqrt{2}}  & \frac{1}{\sqrt{2}}  \\ 0 & 0 \end{array} \right)$ \\
\\[-1em]
$\Lambda=$ & 
$\left( \begin{array}{cc} 1 & 0 \\ 0 & 0 \end{array} \right)$ &
$\left( \begin{array}{cc} \frac{1}{\sqrt{2}} & 0 \\ 0 & \frac{1}{\sqrt{2}} \end{array} \right)$ &
$\left( \begin{array}{cc} 1 & 0 \\ 0 & 0 \end{array} \right)$ &
$\left( \begin{array}{cc} 1 & 0 \\ 0 & 0 \end{array} \right)$  \\
$V=$ & 
$\left( \begin{array}{cc} 1 & 0 \\ 0 & 1 \end{array} \right)$ &
$\left( \begin{array}{cc} 0 & 1 \\ 1 & 0 \end{array} \right)$ &
$\left( \begin{array}{cc} \frac{1}{\sqrt{2}} & -\frac{1}{\sqrt{2}} \\ \frac{1}{\sqrt{2}} & \frac{1}{\sqrt{2}} \end{array} \right)$ &
$\left( \begin{array}{cc} 1 & 0 \\ 0 & 1 \end{array} \right)$  \\
$C=$ & 
$\left( \begin{array}{cc} 1 & 0 \\ 0 & 1 \end{array} \right)$ &
$\left( \begin{array}{cc} 1 & 0 \\ 0 & 1 \end{array} \right)$ &
$\left( \begin{array}{cc} 1 & 0 \\ 0 & 1 \end{array} \right)$ &
$\left( \begin{array}{cc} \frac{1}{\sqrt{2}} & -\frac{1}{\sqrt{2}} \\ \frac{1}{\sqrt{2}} & \frac{1}{\sqrt{2}} \end{array} \right)$  
\\
\hline
\\[-1em]
$\psi =$ & 
$\widetilde{\phi}_1 \widetilde{\chi}_1$ &
$ \frac{1}{\sqrt{2}}\left( \widetilde{\phi}_1 \widetilde{\chi}_1 + \widetilde{\phi}_2 \widetilde{\chi}_2 \right)$ &
$\widetilde{\phi}_1 \widetilde{\chi}_1$ &
$\widetilde{\phi}_1 \widetilde{\chi}_1$ \\
\\[-1em]
\hline
\\[-1em]
$\widetilde{\phi}_1= $ & 
$\phi_1$  & 
$\phi_2$  & 
$\frac{1}{\sqrt{2}}\left( \phi_1  + \phi_2  \right)$ &
$\phi_1$ \\
\\[-1em]
\hline
\\[-1em]
$\widetilde{\phi}_2= $ & 
---  & 
$\phi_1$  & 
--- &
--- \\
\hline
\\[-1em]
$\widetilde{\chi}_1= $ & 
$\chi_1$  & 
$\chi_1$  & 
$\chi_1$  & 
$\frac{1}{\sqrt{2}}\left( \chi_1  + \chi_2  \right)$  \\
\\[-1em]
\hline
\\[-1em]
$\widetilde{\chi}_2= $ & 
---  & 
$\chi_2$  & 
--- &
--- \\
\hline \hline
\end{tabular}
\end{table}
Before we give numerical examples for the transition density matrix resulting from actual linear response TDDFT calculations in the Appendix~\ref{sec:casida}, we illustrate the transformation of Eq.~\ref{eq:svd} for the four simple exciton cases defined in Fig.~1 of the main part in Table~\ref{tab:fourcases}. Here, only $N_v=2$ occupied and $N_c=2$ unoccupied orbitals are taken into account for setting up the transition density matrix. So, all matrices are simple $2 \times 2$ matrices. The consequences of the resulting set of NTOs $\widetilde{\phi}$ and $\widetilde{\chi}$ are explained in the main part of the paper.





\section{Ground State and Linear Response Calculations}
\label{sec:casida}

The structures of the three molecules TCNQ (C$_{12}$H$_{4}$N$_{4}$), porphin (C$_{20}$H$_{14}$N$_{4}$) and PTCDA (C$_{24}$H$_{8}$O$_{6}$) were optimized using the real-space mode of GPAW~\cite{Mortensen2005,Enkovaara2010} in conjunction with the BGFS minimizer from the Atomic Simulation Environment (ASE).\cite{Larsen2017} We used a simulation box with $0.2$~$\mathrm{\AA}$ spacing, $8$~$\mathrm{\AA}$ vacuum around each molecule and set the maximum force criterion to $0.02$~eV$/\mathrm{\AA}$. These relaxed geometries were then used in all further calculations and are depicted in Fig.~\ref{figure_SI} together with the Cartesian coordinate system and the direction of the pump field incidence.
\begin{figure}
\begin{center}
\includegraphics[width=15cm]{SM_fig1.png}
\caption{Geometries of the three molecules used in our investigation (TCNQ, porphin and PTCDA), arrows mark the direction of the pump pulse in the photoemission simulation}
\label{figure_SI}
\end{center}
\end{figure}

To solve Casida's equation and perform the NTO analysis, we employed the linear-response TDDFT (LR-TDDFT) implementation of the real-space code OCTOPUS.\cite{Andrade2015,Tancogne-Dejean2020}
For the three molecules, we used a simulation domain with spheres of radius $8$~$\mathrm \AA$ around each atom and a spacing of $0.2$~$\mathrm\AA$. While the latter value for the spacing may not lead to fully converged results for the geometry optimization described before, as well as for the optical spectra in the following, we choose $0.2$~$\mathrm \AA$ none the less for all calculations to be consistent with the numerically very demanding ARPES simulations. For the same reason, we used the local density approximation (LDA)~\cite{Dirac1930} for LR-TDDFT calculations with the Perdew-Zunger implementation of correlation~\cite{Perdew1981} and norm-conserving Troullier-Martins pseudopotentials.\cite{Troullier1991} Having computed the respective groundstate of the three molecules this way, we solved Casida's equation with the same numerical parameter and took an energy window of $32$~eV, $28$~eV and $30$~eV for TCNQ, Porphin and PTCDA respectively. In this range combinations of occupied and unoccupied states were considered, thereby obtaining the transition density matrices $X_{v c}^{(m)}$ for the excitons of interest. Note that our calculations also include de-excitations beyond the Tamm-Dancoff approximation. For the NTOs, we computed the singular value decomposition of Eq.~\ref{eq:svd} with python's numpy package.\cite{Harris2020}
The results of the TR-TDDFT calculations are shown in Table~\ref{tab:casida_states} and optical spectra are shown in Fig~\ref{figure_spectra} for comparison with the real-time TDDFT calculations of the next section.
\begin{table}
\caption{Casida excitation energies, $\Omega_m$, their corresponding single particle contributions in terms of the inital Kohn-Sham molecular orbitals, $X_{v c}^2$, and the eigenvalues (magnitudes) of the natural transition orbitals, $\Lambda_{\lambda}^2$, for the three molecules presented in the main text. All contributions greater than $0.01$ are shown, those referenced in our investigation are highlighted. {\label{tab:casida_states}}}
\begin{tabular}{cccccccc}
\hline \hline
\multicolumn{2}{c}{TCNQ} & & \multicolumn{2}{c}{porphin} & & \multicolumn{2}{c}{PTCDA} \\
\multicolumn{2}{c}{$\Omega_m=6.76~\mathrm{eV}$} & & \multicolumn{2}{c}{$\Omega_m=3.52~\mathrm{eV}$} & & \multicolumn{2}{c}{$\Omega_m=5.51~\mathrm{eV}$} \\
\hline
\\[-1em]
$ \phi_v \rightarrow \chi_c $ & $X_{vc}^2$ & & $ \phi_v \rightarrow \chi_c $ & $X_{vc}^2$ & & $ \phi_v \rightarrow \chi_c $ & $X_{vc}^2$ \\
\cellcolor{orange!75}$\boldsymbol{3 \rightarrow 2}$ & $\boldsymbol{0.44}$ & & 
$2 \rightarrow 1$ & $0.36$ & & \cellcolor{violet!50}$\boldsymbol{8 \rightarrow 4}$ & $\boldsymbol{0.29}$ \\
\cellcolor{orange!50}$\boldsymbol{2 \rightarrow 3}$ & $\boldsymbol{0.35}$ & & 
\cellcolor{teal!50}$\boldsymbol{1 \rightarrow 2}$ & $\boldsymbol{0.27}$ & & \cellcolor{violet!25}$ \boldsymbol{11 \rightarrow 2}$ & $\boldsymbol{0.23}$ \\
\cellcolor{orange!25}$\boldsymbol{1 \rightarrow 6}$ & $\boldsymbol{0.07}$ & & 
\cellcolor{teal!50}$\boldsymbol{4 \rightarrow 2}$ & $\boldsymbol{0.25}$ & & $ 7 \rightarrow 7 $ & $ 0.17 $ \\
$ 5 \rightarrow 4 $ & $ 0.03 $ & & \cellcolor{teal!50}$\boldsymbol{8 \rightarrow 2}$ & $\boldsymbol{0.05}$ & & $ 4 \rightarrow 8 $ & $ 0.06 $ \\
${17 \rightarrow 2}$ & $0.02$         & & $ 3 \rightarrow 3 $ & $ 0.04 $ & & $ 4 \rightarrow 2 $ & $ 0.06 $ \\
${11 \rightarrow 3}$ & $0.02$ & &                     &          & & $ 1 \rightarrow 5 $ & $ 0.05 $ \\
                    &          & &                     &          & & \cellcolor{violet!50}$\boldsymbol{8 \rightarrow 3}$ & $\boldsymbol{0.03}$ \\
                    &          & &                     &          & & ${16 \rightarrow 1}$ & $0.02$ \\
                    &          & &                     &          & & ${7 \rightarrow 1}$ & $0.02$ \\
                    &          & &                     &          & & ${9 \rightarrow 2}$ & $0.02$ \\
\hline 
\\[-1em]
$ \widetilde{\phi}_{\lambda} \rightarrow \widetilde{\chi}_{\lambda} $ & $\Lambda_{{\lambda}}^2$ & & $ \widetilde{\phi}_{\lambda} \rightarrow \widetilde{\chi}_{\lambda} $ & $\Lambda_{{\lambda}}^2$ & & $ \widetilde{\phi}_{\lambda} \rightarrow \widetilde{\chi}_{\lambda} $ & $\Lambda_{\lambda}^2$ \\
\cellcolor{orange!75}$\boldsymbol{1 \rightarrow 1}$ & $\boldsymbol{0.46}$ & & 
\cellcolor{teal!50}$\boldsymbol{1 \rightarrow 1}$ & $\boldsymbol{0.61}$ & & 
\cellcolor{violet!50}$\boldsymbol{1 \rightarrow 1}$ & $\boldsymbol{0.32}$ \\
\cellcolor{orange!50}$\boldsymbol{2 \rightarrow 2}$ & $\boldsymbol{0.39}$ & & 
$ 2 \rightarrow 2 $ & $ 0.40 $ & & \cellcolor{violet!25}$\boldsymbol{2 \rightarrow 2}$ & $\boldsymbol{0.32}$ \\
\cellcolor{orange!25}$\boldsymbol{3 \rightarrow 3}$ & $\boldsymbol{0.07}$ & & 
$ 3 \rightarrow 3 $ & $ 0.05 $ & & $ 3 \rightarrow 3 $ & $ 0.20 $ \\
$ 4 \rightarrow 4 $ & $ 0.04 $ & &                     &          & & $ 4 \rightarrow 4 $ & $ 0.06 $ \\
$ 5 \rightarrow 5 $ & $0.01$ & &                     &          & & $ 5 \rightarrow 5 $ & $ 0.05 $ \\
                    &          & &                     &          & & $ 6 \rightarrow 6 $ & $ 0.03 $ \\
\hline \hline
\end{tabular}
\end{table}

\section{Real-time TDDFT Calculations}
\label{sec:tddft}
In this section, we describe the methods to obtain the ab-initio simulations of photoemission from real-time TDDFT (RT-TDDFT) with OCTOPUS.
While in the last section, the results for linear-response calculations already delivered the desired excitation energies, we also employed a RT-TDDFT method for optical spectra.\cite{Yabana1996} Using the ground state calculations with the same parameter as described in the previous section, we perturbed the system at initial time $t=0$ with a Dirac-$\delta$ pulse (pulse strength: 0.01~$\mathrm{\AA}^{-1}$) that equally excites all optically allowed transitions. We evolved the system for further $30$~fs, with a time steps of 2~as, and then Fourier transformed the time-dependent dipole-moment to get the optical spectrum.\cite{Yabana2006} In Fig.~\ref{figure_spectra}, we compare the optical spectra from RT-TDDFT with those from the Casida calculations of the previous section. For all three molecules, we find very good agreement, thus assuring the convergence of our methods. Since we also use TDDFT in the real-time fashion for the ARPES simulations, we use the excitation energies (marked by $\star$ symbols) from RT-TDDFT.
\begin{figure}
\begin{center}
\includegraphics[width=11cm]{SM_fig2.pdf}
\caption{Absorption spectra of the molecules TCNQ, porphin and PTCDA calculated with OCTOPUS in RT-TDDFT (full curves) and within the linear-response Casida formalism (dashed curves). Excitation energies used in the pump-probe photoemission simulations are marked with a star.}
\label{figure_spectra}
\end{center}
\end{figure}

Having obtained the excitation energies of interest, we now describe the method used for the pump-probe ARPES simulations with t-SURFF.\cite{Wopperer2017,DeGiovannini2017} For all three molecules, we first computed the ground state as described above, with the only difference that we used a spherical simulation box around the center of the molecules with $R = 35$~$\mathrm \AA$ radius. Then, the systems were subjected to pump pulses with respective energies $\Omega_m$ for $t_{\mathrm{pump}}=20$~fs, followed by $t_{\mathrm{probe}}=15$~fs of propagation time with the probe pulse. While the energy and direction of the pump pulses were varied according to the excitations within the different molecules, we always probed with $z$-polarized fields and a photon energy of $\omega = 35$~eV. For both types of pulses, we used a $\cos(\omega t)$ function, shaped by a hull function of $\sin^2$-type to ensure gradual on- and off-switching of the fields, thereby avoiding non-resonant excitations. The field amplitudes were varied such that the radiation would correspond to a laser with intensity $10^8$~W$/$cm$^2$. In order to avoid spurious effects of reflected electron density at the border of our simulation region, we inserted a complex absorbing potential (CAP)~\cite{DeGiovannini2015} described by $\mathrm i \xi \sin^2(\frac{\Theta(r-R_0) \pi}{2 R})$, with magnitude $\xi=-0.2$~a.u. and onset at $R_0=20$~$\mathrm \AA$. Over all times, we recorded the flux of electron density through a spherical surface~\cite{Wopperer2017,DeGiovannini2017} at $R_0$ and thus obtained energy- and angle-resolved photoemission intesities in an \emph{ab-inito} way as a direct numerical simulation of the experiment.

In the following, we give additional results that complement those of the main text for all three molecules. For each molecule in Fig.~\ref{fig:TCNQ}--\ref{fig:PTCDA}, we show the kinetic energy spectra from t-SURFF (Panels (a)) in conjunction with momentum maps from the different methods presented for a series of orbitals that are relevant for the respective excitons (Panels (b)). For TCNQ in Fig.~\ref{fig:TCNQ}, all results between the different theoretical descriptions agree well, with the exception of maps for $v=11$, where the results from t-SURFF are different to exPOT. Interestingly, it seems that the t-SURFF map for $v=11$ depicts what seems to be missing for the exPOT map for $v=2$, i.e. the accentuation of the main feature at $k_x=0$ $\mathrm{\AA^{-1}}$, $k_y\ge2$ $\mathrm{\AA^{-1}}$.
The additional results for porphin in Fig.~\ref{fig:Por} show very good agreement as well, with the one exception of $v=8$, which does not agree at all. For the two pathological cases, $v=11$ in TCNQ and $v=8$ in porphin, we wish to remark that for both cases the contributions to the transition matrix are alread quite small (1-2 \%) such that better converged LR-TDDFT calculations might give other results. The same argument is valid for the t-SURFF calculations, where it can be seen in the kinetic energy-resolved spectra that the peaks stemming from these two transitions are by approximately an order of magnitude smaller than those of the main contributions. 

%\subsection{TCNQ}
\begin{figure}
\begin{center}
\includegraphics[width=10.cm]{SM_fig3.pdf}
\caption{\textbf{Summary of results for TCNQ excited with 6.7 eV in $\boldsymbol y$-direction.} Kinetic energy spectrum from t-SURFF is shown in panel (a) with $I(|\boldsymbol k|)$ in grey, as well as the projection on states $v=\{3,2,1,5,11\}$. In the same colors, we show $E_v$ in dashed lines and $E_v + \omega_{\mathrm{pump}}$ in full lines. In panel (b), the corresponding momentum maps of the state-projected photoemission intensities from t-SURFF are shown in each line of the leftmost column. In the left-middle column, we show the results from exPOT for the sum over NTOs and the equal results from exPOT with the coherent sum over $X_{v c} \chi_c$ in the middle-right column. For comparison, the results with an incoherent sum are shown in the rightmost column (see text for details).}
\label{fig:TCNQ}
\end{center}
\end{figure}

%\subsection{Porphin}
\begin{figure}
\begin{center}
\includegraphics[width=10.cm]{SM_fig4.pdf}
\caption{\textbf{Summary of results for porphin excited with 3.5 eV in $\boldsymbol x$-direction.} Kinetic energy spectrum from t-SURFF is shown in panel (a) with $I(|\boldsymbol k|)$ in grey, as well as the projection on states $v=\{2,1,4,8,3\}$. In the same colors, we show $E_v$ in dashed lines and $E_v + \omega_{\mathrm{pump}}$ in full lines. In panel (b), the corresponding momentum maps of the state-projected photoemission intensities from t-SURFF are shown in each line of the leftmost column. In the left-middle column, we show the results from exPOT for the sum over NTOs and the equal results from exPOT with the coherent sum over $X_{v c} \chi_c$ in the middle-right column. For comparison, the results with an incoherent sum are shown in the rightmost column (see text for details).}
\label{fig:Por}
\end{center}
\end{figure}

%\subsection{PTCDA}
\begin{figure}
\begin{center}
\includegraphics[width=10.cm]{SM_fig5.pdf}
\caption{\textbf{Summary of results for PTCDA excited with 5.45 eV in $\boldsymbol y$-direction.} Kinetic energy spectrum from t-SURFF is shown in panel (a) with $I(|\boldsymbol k|)$ in grey, as well as the projection on states $v=\{8,7,4,11,1\}$. In the same colors, we show $E_v$ in dashed lines and $E_v + \omega_{\mathrm{pump}}$ in full lines. In panel (b), the corresponding momentum maps of the state-projected photoemission intensities from t-SURFF are shown in each line of the leftmost column. In the left-middle column, we show the results from exPOT for the sum over NTOs and the equal results from exPOT with the coherent sum over $X_{v c} \chi_c$ in the middle-right column. For comparison, the \emph{wrong} results with an incoherent sum are shown in the rightmost column (see text for details).}
\label{fig:PTCDA}
\end{center}
\end{figure}


%\bibliographystyle{apsrev4-1}
%\bibliography{references,SM_references}

%merlin.mbs apsrev4-1.bst 2010-07-25 4.21a (PWD, AO, DPC) hacked
%Control: key (0)
%Control: author (72) initials jnrlst
%Control: editor formatted (1) identically to author
%Control: production of article title (-1) disabled
%Control: page (0) single
%Control: year (1) truncated
%Control: production of eprint (0) enabled
\begin{thebibliography}{63}%
\makeatletter
\providecommand \@ifxundefined [1]{%
 \@ifx{#1\undefined}
}%
\providecommand \@ifnum [1]{%
 \ifnum #1\expandafter \@firstoftwo
 \else \expandafter \@secondoftwo
 \fi
}%
\providecommand \@ifx [1]{%
 \ifx #1\expandafter \@firstoftwo
 \else \expandafter \@secondoftwo
 \fi
}%
\providecommand \natexlab [1]{#1}%
\providecommand \enquote  [1]{``#1''}%
\providecommand \bibnamefont  [1]{#1}%
\providecommand \bibfnamefont [1]{#1}%
\providecommand \citenamefont [1]{#1}%
\providecommand \href@noop [0]{\@secondoftwo}%
\providecommand \href [0]{\begingroup \@sanitize@url \@href}%
\providecommand \@href[1]{\@@startlink{#1}\@@href}%
\providecommand \@@href[1]{\endgroup#1\@@endlink}%
\providecommand \@sanitize@url [0]{\catcode `\\12\catcode `\$12\catcode
  `\&12\catcode `\#12\catcode `\^12\catcode `\_12\catcode `\%12\relax}%
\providecommand \@@startlink[1]{}%
\providecommand \@@endlink[0]{}%
\providecommand \url  [0]{\begingroup\@sanitize@url \@url }%
\providecommand \@url [1]{\endgroup\@href {#1}{\urlprefix }}%
\providecommand \urlprefix  [0]{URL }%
\providecommand \Eprint [0]{\href }%
\providecommand \doibase [0]{http://dx.doi.org/}%
\providecommand \selectlanguage [0]{\@gobble}%
\providecommand \bibinfo  [0]{\@secondoftwo}%
\providecommand \bibfield  [0]{\@secondoftwo}%
\providecommand \translation [1]{[#1]}%
\providecommand \BibitemOpen [0]{}%
\providecommand \bibitemStop [0]{}%
\providecommand \bibitemNoStop [0]{.\EOS\space}%
\providecommand \EOS [0]{\spacefactor3000\relax}%
\providecommand \BibitemShut  [1]{\csname bibitem#1\endcsname}%
\let\auto@bib@innerbib\@empty
%</preamble>
\bibitem [{\citenamefont {Puschnig}\ \emph {et~al.}(2009)\citenamefont
  {Puschnig}, \citenamefont {Berkebile}, \citenamefont {Fleming}, \citenamefont
  {Koller}, \citenamefont {Emtsev}, \citenamefont {Seyller}, \citenamefont
  {Riley}, \citenamefont {Ambrosch-Draxl}, \citenamefont {Netzer},\ and\
  \citenamefont {Ramsey}}]{Puschnig2009a}%
  \BibitemOpen
  \bibfield  {author} {\bibinfo {author} {\bibfnamefont {P.}~\bibnamefont
  {Puschnig}}, \bibinfo {author} {\bibfnamefont {S.}~\bibnamefont {Berkebile}},
  \bibinfo {author} {\bibfnamefont {A.~J.}\ \bibnamefont {Fleming}}, \bibinfo
  {author} {\bibfnamefont {G.}~\bibnamefont {Koller}}, \bibinfo {author}
  {\bibfnamefont {K.}~\bibnamefont {Emtsev}}, \bibinfo {author} {\bibfnamefont
  {T.}~\bibnamefont {Seyller}}, \bibinfo {author} {\bibfnamefont {J.~D.}\
  \bibnamefont {Riley}}, \bibinfo {author} {\bibfnamefont {C.}~\bibnamefont
  {Ambrosch-Draxl}}, \bibinfo {author} {\bibfnamefont {F.~P.}\ \bibnamefont
  {Netzer}}, \ and\ \bibinfo {author} {\bibfnamefont {M.~G.}\ \bibnamefont
  {Ramsey}},\ }\href {\doibase 10.1126/science.1176105} {\bibfield  {journal}
  {\bibinfo  {journal} {Science}\ }\textbf {\bibinfo {volume} {326}},\ \bibinfo
  {pages} {702} (\bibinfo {year} {2009})}\BibitemShut {NoStop}%
\bibitem [{\citenamefont {Dauth}\ \emph {et~al.}(2011)\citenamefont {Dauth},
  \citenamefont {K\"orzd\"orfer}, \citenamefont {K\"ummel}, \citenamefont
  {Ziroff}, \citenamefont {Wiessner}, \citenamefont {Sch\"oll}, \citenamefont
  {Reinert}, \citenamefont {Arita},\ and\ \citenamefont {Shimada}}]{Dauth2011}%
  \BibitemOpen
  \bibfield  {author} {\bibinfo {author} {\bibfnamefont {M.}~\bibnamefont
  {Dauth}}, \bibinfo {author} {\bibfnamefont {T.}~\bibnamefont
  {K\"orzd\"orfer}}, \bibinfo {author} {\bibfnamefont {S.}~\bibnamefont
  {K\"ummel}}, \bibinfo {author} {\bibfnamefont {J.}~\bibnamefont {Ziroff}},
  \bibinfo {author} {\bibfnamefont {M.}~\bibnamefont {Wiessner}}, \bibinfo
  {author} {\bibfnamefont {A.}~\bibnamefont {Sch\"oll}}, \bibinfo {author}
  {\bibfnamefont {F.}~\bibnamefont {Reinert}}, \bibinfo {author} {\bibfnamefont
  {M.}~\bibnamefont {Arita}}, \ and\ \bibinfo {author} {\bibfnamefont
  {K.}~\bibnamefont {Shimada}},\ }\href {\doibase
  10.1103/PhysRevLett.107.193002} {\bibfield  {journal} {\bibinfo  {journal}
  {Phys. Rev. Lett.}\ }\textbf {\bibinfo {volume} {107}},\ \bibinfo {pages}
  {193002} (\bibinfo {year} {2011})}\BibitemShut {NoStop}%
\bibitem [{\citenamefont {Nguyen}\ \emph {et~al.}(2015)\citenamefont {Nguyen},
  \citenamefont {Borghi}, \citenamefont {Ferretti}, \citenamefont {Dabo},\ and\
  \citenamefont {Marzari}}]{Nguyen2015}%
  \BibitemOpen
  \bibfield  {author} {\bibinfo {author} {\bibfnamefont {N.~L.}\ \bibnamefont
  {Nguyen}}, \bibinfo {author} {\bibfnamefont {G.}~\bibnamefont {Borghi}},
  \bibinfo {author} {\bibfnamefont {A.}~\bibnamefont {Ferretti}}, \bibinfo
  {author} {\bibfnamefont {I.}~\bibnamefont {Dabo}}, \ and\ \bibinfo {author}
  {\bibfnamefont {N.}~\bibnamefont {Marzari}},\ }\href {\doibase
  10.1103/PhysRevLett.114.166405} {\bibfield  {journal} {\bibinfo  {journal}
  {Phys. Rev. Lett.}\ }\textbf {\bibinfo {volume} {114}},\ \bibinfo {pages}
  {166405} (\bibinfo {year} {2015})}\BibitemShut {NoStop}%
\bibitem [{\citenamefont {Woodruff}(2016)}]{Woodruff2016}%
  \BibitemOpen
  \bibfield  {author} {\bibinfo {author} {\bibfnamefont {P.}~\bibnamefont
  {Woodruff}},\ }\href {\doibase 10.1017/CBO9781139149716} {\emph {\bibinfo
  {title} {Modern Techniques of Surface Science}}}\ (\bibinfo  {publisher}
  {Cambridge University Press},\ \bibinfo {year} {2016})\BibitemShut {NoStop}%
\bibitem [{\citenamefont {Puschnig}\ and\ \citenamefont
  {Ramsey}(2018)}]{Puschnig2017}%
  \BibitemOpen
  \bibfield  {author} {\bibinfo {author} {\bibfnamefont {P.}~\bibnamefont
  {Puschnig}}\ and\ \bibinfo {author} {\bibfnamefont {M.}~\bibnamefont
  {Ramsey}},\ }in\ \href {\doibase
  https://doi.org/10.1016/B978-0-12-409547-2.13782-5} {\emph {\bibinfo
  {booktitle} {Encyclopedia of Interfacial Chemistry}}},\ \bibinfo {editor}
  {edited by\ \bibinfo {editor} {\bibfnamefont {K.}~\bibnamefont {Wandelt}}}\
  (\bibinfo  {publisher} {Elsevier},\ \bibinfo {address} {Oxford},\ \bibinfo
  {year} {2018})\ pp.\ \bibinfo {pages} {380 -- 391}\BibitemShut {NoStop}%
\bibitem [{\citenamefont {Kliuiev}\ \emph {et~al.}(2019)\citenamefont
  {Kliuiev}, \citenamefont {Zamborlini}, \citenamefont {Jugovac}, \citenamefont
  {Gurdal}, \citenamefont {von Arx}, \citenamefont {Waltar}, \citenamefont
  {Schnidrig}, \citenamefont {Alberto}, \citenamefont {Iannuzzi}, \citenamefont
  {Feyer}, \citenamefont {Hengsberger},\ and\ \citenamefont
  {Castiglioni}}]{Kliuiev2019}%
  \BibitemOpen
  \bibfield  {author} {\bibinfo {author} {\bibfnamefont {P.}~\bibnamefont
  {Kliuiev}}, \bibinfo {author} {\bibfnamefont {G.}~\bibnamefont {Zamborlini}},
  \bibinfo {author} {\bibfnamefont {M.}~\bibnamefont {Jugovac}}, \bibinfo
  {author} {\bibfnamefont {Y.}~\bibnamefont {Gurdal}}, \bibinfo {author}
  {\bibfnamefont {K.}~\bibnamefont {von Arx}}, \bibinfo {author} {\bibfnamefont
  {K.}~\bibnamefont {Waltar}}, \bibinfo {author} {\bibfnamefont
  {S.}~\bibnamefont {Schnidrig}}, \bibinfo {author} {\bibfnamefont
  {R.}~\bibnamefont {Alberto}}, \bibinfo {author} {\bibfnamefont
  {M.}~\bibnamefont {Iannuzzi}}, \bibinfo {author} {\bibfnamefont
  {V.}~\bibnamefont {Feyer}}, \bibinfo {author} {\bibfnamefont
  {M.}~\bibnamefont {Hengsberger}}, \ and\ \bibinfo {author} {\bibfnamefont
  {J.~O.~L.}\ \bibnamefont {Castiglioni}},\ }\href {\doibase
  10.1038/s41467-019-13254-7} {\bibfield  {journal} {\bibinfo  {journal}
  {Nature Communications}\ }\textbf {\bibinfo {volume} {10}},\ \bibinfo {pages}
  {5255} (\bibinfo {year} {2019})}\BibitemShut {NoStop}%
\bibitem [{\citenamefont {Bradshaw}\ and\ \citenamefont
  {Woodruff}(2015)}]{Bradshaw2015}%
  \BibitemOpen
  \bibfield  {author} {\bibinfo {author} {\bibfnamefont {A.~M.}\ \bibnamefont
  {Bradshaw}}\ and\ \bibinfo {author} {\bibfnamefont {D.~P.}\ \bibnamefont
  {Woodruff}},\ }\href {\doibase 10.1088/1367-2630/17/1/013033} {\bibfield
  {journal} {\bibinfo  {journal} {New J. Phys.}\ }\textbf {\bibinfo {volume}
  {17}},\ \bibinfo {pages} {013033} (\bibinfo {year} {2015})}\BibitemShut
  {NoStop}%
\bibitem [{\citenamefont {Egger}\ \emph {et~al.}(2019)\citenamefont {Egger},
  \citenamefont {Kollmann}, \citenamefont {Hurdax}, \citenamefont {L\"uftner},
  \citenamefont {Yang}, \citenamefont {Wei{\ss}}, \citenamefont {Gottwald},
  \citenamefont {Richter}, \citenamefont {Koller}, \citenamefont {Soubatch},
  \citenamefont {Tautz}, \citenamefont {Puschnig},\ and\ \citenamefont
  {Ramsey}}]{Egger2018}%
  \BibitemOpen
  \bibfield  {author} {\bibinfo {author} {\bibfnamefont {L.}~\bibnamefont
  {Egger}}, \bibinfo {author} {\bibfnamefont {B.}~\bibnamefont {Kollmann}},
  \bibinfo {author} {\bibfnamefont {P.}~\bibnamefont {Hurdax}}, \bibinfo
  {author} {\bibfnamefont {D.}~\bibnamefont {L\"uftner}}, \bibinfo {author}
  {\bibfnamefont {X.}~\bibnamefont {Yang}}, \bibinfo {author} {\bibfnamefont
  {S.}~\bibnamefont {Wei{\ss}}}, \bibinfo {author} {\bibfnamefont
  {A.}~\bibnamefont {Gottwald}}, \bibinfo {author} {\bibfnamefont
  {M.}~\bibnamefont {Richter}}, \bibinfo {author} {\bibfnamefont
  {G.}~\bibnamefont {Koller}}, \bibinfo {author} {\bibfnamefont
  {S.}~\bibnamefont {Soubatch}}, \bibinfo {author} {\bibfnamefont {F.~S.}\
  \bibnamefont {Tautz}}, \bibinfo {author} {\bibfnamefont {P.}~\bibnamefont
  {Puschnig}}, \ and\ \bibinfo {author} {\bibfnamefont {M.~G.}\ \bibnamefont
  {Ramsey}},\ }\href {\doibase 10.1088/1367-2630/ab0781} {\bibfield  {journal}
  {\bibinfo  {journal} {New J. Phys.}\ }\textbf {\bibinfo {volume} {21}},\
  \bibinfo {pages} {043003} (\bibinfo {year} {2019})}\BibitemShut {NoStop}%
\bibitem [{\citenamefont {Dauth}\ \emph
  {et~al.}(2016{\natexlab{a}})\citenamefont {Dauth}, \citenamefont {Graus},
  \citenamefont {Schelter}, \citenamefont {Wie\ss{}ner}, \citenamefont
  {Sch\"oll}, \citenamefont {Reinert},\ and\ \citenamefont
  {K\"ummel}}]{Dauth2016a}%
  \BibitemOpen
  \bibfield  {author} {\bibinfo {author} {\bibfnamefont {M.}~\bibnamefont
  {Dauth}}, \bibinfo {author} {\bibfnamefont {M.}~\bibnamefont {Graus}},
  \bibinfo {author} {\bibfnamefont {I.}~\bibnamefont {Schelter}}, \bibinfo
  {author} {\bibfnamefont {M.}~\bibnamefont {Wie\ss{}ner}}, \bibinfo {author}
  {\bibfnamefont {A.}~\bibnamefont {Sch\"oll}}, \bibinfo {author}
  {\bibfnamefont {F.}~\bibnamefont {Reinert}}, \ and\ \bibinfo {author}
  {\bibfnamefont {S.}~\bibnamefont {K\"ummel}},\ }\href {\doibase
  10.1103/PhysRevLett.117.183001} {\bibfield  {journal} {\bibinfo  {journal}
  {Phys. Rev. Lett.}\ }\textbf {\bibinfo {volume} {117}},\ \bibinfo {pages}
  {183001} (\bibinfo {year} {2016}{\natexlab{a}})}\BibitemShut {NoStop}%
\bibitem [{\citenamefont {Metzger}\ \emph {et~al.}(2020)\citenamefont
  {Metzger}, \citenamefont {Graus}, \citenamefont {Grimm}, \citenamefont
  {Zamborlini}, \citenamefont {Feyer}, \citenamefont {Schwendt}, \citenamefont
  {L\"uftner}, \citenamefont {Puschnig}, \citenamefont {Sch\"oll},\ and\
  \citenamefont {Reinert}}]{Metzger2020}%
  \BibitemOpen
  \bibfield  {author} {\bibinfo {author} {\bibfnamefont {C.}~\bibnamefont
  {Metzger}}, \bibinfo {author} {\bibfnamefont {M.}~\bibnamefont {Graus}},
  \bibinfo {author} {\bibfnamefont {M.}~\bibnamefont {Grimm}}, \bibinfo
  {author} {\bibfnamefont {G.}~\bibnamefont {Zamborlini}}, \bibinfo {author}
  {\bibfnamefont {V.}~\bibnamefont {Feyer}}, \bibinfo {author} {\bibfnamefont
  {M.}~\bibnamefont {Schwendt}}, \bibinfo {author} {\bibfnamefont
  {D.}~\bibnamefont {L\"uftner}}, \bibinfo {author} {\bibfnamefont
  {P.}~\bibnamefont {Puschnig}}, \bibinfo {author} {\bibfnamefont
  {A.}~\bibnamefont {Sch\"oll}}, \ and\ \bibinfo {author} {\bibfnamefont
  {F.}~\bibnamefont {Reinert}},\ }\href {\doibase 10.1103/PhysRevB.101.165421}
  {\bibfield  {journal} {\bibinfo  {journal} {Phys. Rev. B}\ }\textbf {\bibinfo
  {volume} {101}},\ \bibinfo {pages} {165421} (\bibinfo {year}
  {2020})}\BibitemShut {NoStop}%
\bibitem [{\citenamefont {Ziroff}\ \emph {et~al.}(2010)\citenamefont {Ziroff},
  \citenamefont {Forster}, \citenamefont {Sch\"oll}, \citenamefont {Puschnig},\
  and\ \citenamefont {Reinert}}]{Ziroff2010}%
  \BibitemOpen
  \bibfield  {author} {\bibinfo {author} {\bibfnamefont {J.}~\bibnamefont
  {Ziroff}}, \bibinfo {author} {\bibfnamefont {F.}~\bibnamefont {Forster}},
  \bibinfo {author} {\bibfnamefont {A.}~\bibnamefont {Sch\"oll}}, \bibinfo
  {author} {\bibfnamefont {P.}~\bibnamefont {Puschnig}}, \ and\ \bibinfo
  {author} {\bibfnamefont {F.}~\bibnamefont {Reinert}},\ }\href {\doibase
  10.1103/PhysRevLett.104.233004} {\bibfield  {journal} {\bibinfo  {journal}
  {Phys. Rev. Lett.}\ }\textbf {\bibinfo {volume} {104}},\ \bibinfo {pages}
  {233004} (\bibinfo {year} {2010})}\BibitemShut {NoStop}%
\bibitem [{\citenamefont {Zamborlini}\ \emph {et~al.}(2017)\citenamefont
  {Zamborlini}, \citenamefont {L\"uftner}, \citenamefont {Feng}, \citenamefont
  {Kollmann}, \citenamefont {Puschnig}, \citenamefont {Dri}, \citenamefont
  {Panighel}, \citenamefont {Santo}, \citenamefont {Goldoni}, \citenamefont
  {Comelli}, \citenamefont {Jugovac}, \citenamefont {Feyer},\ and\
  \citenamefont {Schneider}}]{Zamborlini2017}%
  \BibitemOpen
  \bibfield  {author} {\bibinfo {author} {\bibfnamefont {G.}~\bibnamefont
  {Zamborlini}}, \bibinfo {author} {\bibfnamefont {D.}~\bibnamefont
  {L\"uftner}}, \bibinfo {author} {\bibfnamefont {Z.}~\bibnamefont {Feng}},
  \bibinfo {author} {\bibfnamefont {B.}~\bibnamefont {Kollmann}}, \bibinfo
  {author} {\bibfnamefont {P.}~\bibnamefont {Puschnig}}, \bibinfo {author}
  {\bibfnamefont {C.}~\bibnamefont {Dri}}, \bibinfo {author} {\bibfnamefont
  {M.}~\bibnamefont {Panighel}}, \bibinfo {author} {\bibfnamefont {G.~D.}\
  \bibnamefont {Santo}}, \bibinfo {author} {\bibfnamefont {A.}~\bibnamefont
  {Goldoni}}, \bibinfo {author} {\bibfnamefont {G.}~\bibnamefont {Comelli}},
  \bibinfo {author} {\bibfnamefont {M.}~\bibnamefont {Jugovac}}, \bibinfo
  {author} {\bibfnamefont {V.}~\bibnamefont {Feyer}}, \ and\ \bibinfo {author}
  {\bibfnamefont {C.~M.}\ \bibnamefont {Schneider}},\ }\href {\doibase
  10.1038/s41467-017-00402-0} {\bibfield  {journal} {\bibinfo  {journal}
  {Nature Communications}\ }\textbf {\bibinfo {volume} {8}},\ \bibinfo {pages}
  {335} (\bibinfo {year} {2017})}\BibitemShut {NoStop}%
\bibitem [{\citenamefont {Yang}\ \emph {et~al.}(2022)\citenamefont {Yang},
  \citenamefont {Jugovac}, \citenamefont {Zamborlini}, \citenamefont {Feyer},
  \citenamefont {Koller}, \citenamefont {Puschnig}, \citenamefont {Soubatch},
  \citenamefont {Ramsey},\ and\ \citenamefont {Tautz}}]{Yang2022}%
  \BibitemOpen
  \bibfield  {author} {\bibinfo {author} {\bibfnamefont {X.}~\bibnamefont
  {Yang}}, \bibinfo {author} {\bibfnamefont {M.}~\bibnamefont {Jugovac}},
  \bibinfo {author} {\bibfnamefont {G.}~\bibnamefont {Zamborlini}}, \bibinfo
  {author} {\bibfnamefont {V.}~\bibnamefont {Feyer}}, \bibinfo {author}
  {\bibfnamefont {G.}~\bibnamefont {Koller}}, \bibinfo {author} {\bibfnamefont
  {P.}~\bibnamefont {Puschnig}}, \bibinfo {author} {\bibfnamefont
  {S.}~\bibnamefont {Soubatch}}, \bibinfo {author} {\bibfnamefont {M.~G.}\
  \bibnamefont {Ramsey}}, \ and\ \bibinfo {author} {\bibfnamefont {F.~S.}\
  \bibnamefont {Tautz}},\ }\href {\doibase 10.1038/s41467-022-32643-z}
  {\bibfield  {journal} {\bibinfo  {journal} {Nature Communications}\ }\textbf
  {\bibinfo {volume} {13}},\ \bibinfo {pages} {5148} (\bibinfo {year}
  {2022})}\BibitemShut {NoStop}%
\bibitem [{\citenamefont {Graus}\ \emph {et~al.}(2016)\citenamefont {Graus},
  \citenamefont {Grimm}, \citenamefont {Metzger}, \citenamefont {Dauth},
  \citenamefont {Tusche}, \citenamefont {Kirschner}, \citenamefont {K\"ummel},
  \citenamefont {Sch\"oll},\ and\ \citenamefont {Reinert}}]{Graus2016}%
  \BibitemOpen
  \bibfield  {author} {\bibinfo {author} {\bibfnamefont {M.}~\bibnamefont
  {Graus}}, \bibinfo {author} {\bibfnamefont {M.}~\bibnamefont {Grimm}},
  \bibinfo {author} {\bibfnamefont {C.}~\bibnamefont {Metzger}}, \bibinfo
  {author} {\bibfnamefont {M.}~\bibnamefont {Dauth}}, \bibinfo {author}
  {\bibfnamefont {C.}~\bibnamefont {Tusche}}, \bibinfo {author} {\bibfnamefont
  {J.}~\bibnamefont {Kirschner}}, \bibinfo {author} {\bibfnamefont
  {S.}~\bibnamefont {K\"ummel}}, \bibinfo {author} {\bibfnamefont
  {A.}~\bibnamefont {Sch\"oll}}, \ and\ \bibinfo {author} {\bibfnamefont
  {F.}~\bibnamefont {Reinert}},\ }\href {\doibase
  10.1103/PhysRevLett.116.147601} {\bibfield  {journal} {\bibinfo  {journal}
  {Phys. Rev. Lett.}\ }\textbf {\bibinfo {volume} {116}},\ \bibinfo {pages}
  {147601} (\bibinfo {year} {2016})}\BibitemShut {NoStop}%
\bibitem [{\citenamefont {Hurdax}\ \emph {et~al.}(2022)\citenamefont {Hurdax},
  \citenamefont {Kern}, \citenamefont {Bon\'e}, \citenamefont {Haags},
  \citenamefont {Egger}, \citenamefont {Yang}, \citenamefont {Kirschner},
  \citenamefont {Richter}, \citenamefont {Soubatch}, \citenamefont {Koller},
  \citenamefont {Tautz}, \citenamefont {Sterrer}, \citenamefont {Puschnig},\
  and\ \citenamefont {Ramsey}}]{Hurdax2022}%
  \BibitemOpen
  \bibfield  {author} {\bibinfo {author} {\bibfnamefont {P.}~\bibnamefont
  {Hurdax}}, \bibinfo {author} {\bibfnamefont {C.~S.}\ \bibnamefont {Kern}},
  \bibinfo {author} {\bibfnamefont {T.}~\bibnamefont {Bon\'e}}, \bibinfo
  {author} {\bibfnamefont {A.}~\bibnamefont {Haags}}, \bibinfo {author}
  {\bibfnamefont {L.}~\bibnamefont {Egger}}, \bibinfo {author} {\bibfnamefont
  {X.}~\bibnamefont {Yang}}, \bibinfo {author} {\bibfnamefont {H.}~\bibnamefont
  {Kirschner}}, \bibinfo {author} {\bibfnamefont {M.}~\bibnamefont {Richter}},
  \bibinfo {author} {\bibfnamefont {S.}~\bibnamefont {Soubatch}}, \bibinfo
  {author} {\bibfnamefont {G.}~\bibnamefont {Koller}}, \bibinfo {author}
  {\bibfnamefont {F.~S.}\ \bibnamefont {Tautz}}, \bibinfo {author}
  {\bibfnamefont {M.}~\bibnamefont {Sterrer}}, \bibinfo {author} {\bibfnamefont
  {P.}~\bibnamefont {Puschnig}}, \ and\ \bibinfo {author} {\bibfnamefont
  {M.}~\bibnamefont {Ramsey}},\ }\href {\doibase 10.1021/acsnano.2c08631}
  {\bibfield  {journal} {\bibinfo  {journal} {ACS Nano}\ }\textbf {\bibinfo
  {volume} {16}},\ \bibinfo {pages} {17435} (\bibinfo {year}
  {2022})}\BibitemShut {NoStop}%
\bibitem [{\citenamefont {Haags}\ \emph {et~al.}(2020)\citenamefont {Haags},
  \citenamefont {Reichmann}, \citenamefont {Fan}, \citenamefont {Egger},
  \citenamefont {Kirschner}, \citenamefont {Naumann}, \citenamefont {Werner},
  \citenamefont {Vollgraff}, \citenamefont {Sundermeyer}, \citenamefont
  {Eschmann}, \citenamefont {Yang}, \citenamefont {Brandstetter}, \citenamefont
  {Bocquet}, \citenamefont {Koller}, \citenamefont {Gottwald}, \citenamefont
  {Richter}, \citenamefont {Ramsey}, \citenamefont {Rohlfing}, \citenamefont
  {Puschnig}, \citenamefont {Gottfried}, \citenamefont {Soubatch},\ and\
  \citenamefont {Tautz}}]{Haags2020}%
  \BibitemOpen
  \bibfield  {author} {\bibinfo {author} {\bibfnamefont {A.}~\bibnamefont
  {Haags}}, \bibinfo {author} {\bibfnamefont {A.}~\bibnamefont {Reichmann}},
  \bibinfo {author} {\bibfnamefont {Q.}~\bibnamefont {Fan}}, \bibinfo {author}
  {\bibfnamefont {L.}~\bibnamefont {Egger}}, \bibinfo {author} {\bibfnamefont
  {H.}~\bibnamefont {Kirschner}}, \bibinfo {author} {\bibfnamefont
  {T.}~\bibnamefont {Naumann}}, \bibinfo {author} {\bibfnamefont
  {S.}~\bibnamefont {Werner}}, \bibinfo {author} {\bibfnamefont
  {T.}~\bibnamefont {Vollgraff}}, \bibinfo {author} {\bibfnamefont
  {J.}~\bibnamefont {Sundermeyer}}, \bibinfo {author} {\bibfnamefont
  {L.}~\bibnamefont {Eschmann}}, \bibinfo {author} {\bibfnamefont
  {X.}~\bibnamefont {Yang}}, \bibinfo {author} {\bibfnamefont {D.}~\bibnamefont
  {Brandstetter}}, \bibinfo {author} {\bibfnamefont {F.~C.}\ \bibnamefont
  {Bocquet}}, \bibinfo {author} {\bibfnamefont {G.}~\bibnamefont {Koller}},
  \bibinfo {author} {\bibfnamefont {A.}~\bibnamefont {Gottwald}}, \bibinfo
  {author} {\bibfnamefont {M.}~\bibnamefont {Richter}}, \bibinfo {author}
  {\bibfnamefont {M.~G.}\ \bibnamefont {Ramsey}}, \bibinfo {author}
  {\bibfnamefont {M.}~\bibnamefont {Rohlfing}}, \bibinfo {author}
  {\bibfnamefont {P.}~\bibnamefont {Puschnig}}, \bibinfo {author}
  {\bibfnamefont {M.}~\bibnamefont {Gottfried}}, \bibinfo {author}
  {\bibfnamefont {S.}~\bibnamefont {Soubatch}}, \ and\ \bibinfo {author}
  {\bibfnamefont {F.~S.}\ \bibnamefont {Tautz}},\ }\href {\doibase
  10.1021/acsnano.0c06798} {\bibfield  {journal} {\bibinfo  {journal} {ACS
  Nano}\ }\textbf {\bibinfo {volume} {14}},\ \bibinfo {pages} {15766} (\bibinfo
  {year} {2020})}\BibitemShut {NoStop}%
\bibitem [{\citenamefont {L\"uftner}\ \emph
  {et~al.}(2014{\natexlab{a}})\citenamefont {L\"uftner}, \citenamefont {Ules},
  \citenamefont {Reinisch}, \citenamefont {Koller}, \citenamefont {Soubatch},
  \citenamefont {Tautz}, \citenamefont {Ramsey},\ and\ \citenamefont
  {Puschnig}}]{Luftner2013}%
  \BibitemOpen
  \bibfield  {author} {\bibinfo {author} {\bibfnamefont {D.}~\bibnamefont
  {L\"uftner}}, \bibinfo {author} {\bibfnamefont {T.}~\bibnamefont {Ules}},
  \bibinfo {author} {\bibfnamefont {E.~M.}\ \bibnamefont {Reinisch}}, \bibinfo
  {author} {\bibfnamefont {G.}~\bibnamefont {Koller}}, \bibinfo {author}
  {\bibfnamefont {S.}~\bibnamefont {Soubatch}}, \bibinfo {author}
  {\bibfnamefont {F.~S.}\ \bibnamefont {Tautz}}, \bibinfo {author}
  {\bibfnamefont {M.~G.}\ \bibnamefont {Ramsey}}, \ and\ \bibinfo {author}
  {\bibfnamefont {P.}~\bibnamefont {Puschnig}},\ }\href {\doibase
  10.1073/pnas.1315716110} {\bibfield  {journal} {\bibinfo  {journal} {Proc.
  Nat. Acad. Sci. U. S. A.}\ }\textbf {\bibinfo {volume} {111}},\ \bibinfo
  {pages} {605} (\bibinfo {year} {2014}{\natexlab{a}})}\BibitemShut {NoStop}%
\bibitem [{\citenamefont {Wei{\ss}}\ \emph {et~al.}(2015)\citenamefont
  {Wei{\ss}}, \citenamefont {L\"uftner}, \citenamefont {Ules}, \citenamefont
  {Reinisch}, \citenamefont {Kaser}, \citenamefont {Gottwald}, \citenamefont
  {Richter}, \citenamefont {Soubatch}, \citenamefont {Koller}, \citenamefont
  {Ramsey}, \citenamefont {Tautz},\ and\ \citenamefont {Puschnig}}]{Weiss2015}%
  \BibitemOpen
  \bibfield  {author} {\bibinfo {author} {\bibfnamefont {S.}~\bibnamefont
  {Wei{\ss}}}, \bibinfo {author} {\bibfnamefont {D.}~\bibnamefont {L\"uftner}},
  \bibinfo {author} {\bibfnamefont {T.}~\bibnamefont {Ules}}, \bibinfo {author}
  {\bibfnamefont {E.~M.}\ \bibnamefont {Reinisch}}, \bibinfo {author}
  {\bibfnamefont {H.}~\bibnamefont {Kaser}}, \bibinfo {author} {\bibfnamefont
  {A.}~\bibnamefont {Gottwald}}, \bibinfo {author} {\bibfnamefont
  {M.}~\bibnamefont {Richter}}, \bibinfo {author} {\bibfnamefont
  {S.}~\bibnamefont {Soubatch}}, \bibinfo {author} {\bibfnamefont
  {G.}~\bibnamefont {Koller}}, \bibinfo {author} {\bibfnamefont {M.~G.}\
  \bibnamefont {Ramsey}}, \bibinfo {author} {\bibfnamefont {F.~S.}\
  \bibnamefont {Tautz}}, \ and\ \bibinfo {author} {\bibfnamefont
  {P.}~\bibnamefont {Puschnig}},\ }\href {\doibase 10.1038/ncomms9287}
  {\bibfield  {journal} {\bibinfo  {journal} {Nature Communications}\ }\textbf
  {\bibinfo {volume} {6}},\ \bibinfo {pages} {8287} (\bibinfo {year}
  {2015})}\BibitemShut {NoStop}%
\bibitem [{\citenamefont {Kliuiev}\ \emph {et~al.}(2016)\citenamefont
  {Kliuiev}, \citenamefont {Latychevskaia}, \citenamefont {Osterwalder},
  \citenamefont {Hengsberger},\ and\ \citenamefont
  {Castiglioni}}]{Kliuiev2016a}%
  \BibitemOpen
  \bibfield  {author} {\bibinfo {author} {\bibfnamefont {P.}~\bibnamefont
  {Kliuiev}}, \bibinfo {author} {\bibfnamefont {T.}~\bibnamefont
  {Latychevskaia}}, \bibinfo {author} {\bibfnamefont {J.}~\bibnamefont
  {Osterwalder}}, \bibinfo {author} {\bibfnamefont {M.}~\bibnamefont
  {Hengsberger}}, \ and\ \bibinfo {author} {\bibfnamefont {L.}~\bibnamefont
  {Castiglioni}},\ }\href {\doibase 10.1088/1367-2630/18/9/093041} {\bibfield
  {journal} {\bibinfo  {journal} {New J. Phys.}\ }\textbf {\bibinfo {volume}
  {18}},\ \bibinfo {pages} {093041} (\bibinfo {year} {2016})}\BibitemShut
  {NoStop}%
\bibitem [{\citenamefont {Graus}\ \emph {et~al.}(2019)\citenamefont {Graus},
  \citenamefont {Metzger}, \citenamefont {Grimm}, \citenamefont {Nigge},
  \citenamefont {Feyer}, \citenamefont {Sch\"oll},\ and\ \citenamefont
  {Reinert}}]{Graus2019}%
  \BibitemOpen
  \bibfield  {author} {\bibinfo {author} {\bibfnamefont {M.}~\bibnamefont
  {Graus}}, \bibinfo {author} {\bibfnamefont {C.}~\bibnamefont {Metzger}},
  \bibinfo {author} {\bibfnamefont {M.}~\bibnamefont {Grimm}}, \bibinfo
  {author} {\bibfnamefont {P.}~\bibnamefont {Nigge}}, \bibinfo {author}
  {\bibfnamefont {V.}~\bibnamefont {Feyer}}, \bibinfo {author} {\bibfnamefont
  {A.}~\bibnamefont {Sch\"oll}}, \ and\ \bibinfo {author} {\bibfnamefont
  {F.}~\bibnamefont {Reinert}},\ }\href {\doibase 10.1140/epjb/e2019-100015-x}
  {\bibfield  {journal} {\bibinfo  {journal} {Eur. Phys. J. B}\ }\textbf
  {\bibinfo {volume} {92}},\ \bibinfo {pages} {80} (\bibinfo {year}
  {2019})}\BibitemShut {NoStop}%
\bibitem [{\citenamefont {Jansen}\ \emph {et~al.}(2020)\citenamefont {Jansen},
  \citenamefont {Keunecke}, \citenamefont {D\"uvel}, \citenamefont {M\"oller},
  \citenamefont {Schmitt}, \citenamefont {Bennecke}, \citenamefont {Kappert},
  \citenamefont {Steil}, \citenamefont {Luke}, \citenamefont {Steil},\ and\
  \citenamefont {Mathias}}]{Jansen2020}%
  \BibitemOpen
  \bibfield  {author} {\bibinfo {author} {\bibfnamefont {M.}~\bibnamefont
  {Jansen}}, \bibinfo {author} {\bibfnamefont {M.}~\bibnamefont {Keunecke}},
  \bibinfo {author} {\bibfnamefont {M.}~\bibnamefont {D\"uvel}}, \bibinfo
  {author} {\bibfnamefont {C.}~\bibnamefont {M\"oller}}, \bibinfo {author}
  {\bibfnamefont {D.}~\bibnamefont {Schmitt}}, \bibinfo {author} {\bibfnamefont
  {W.}~\bibnamefont {Bennecke}}, \bibinfo {author} {\bibfnamefont
  {J.}~\bibnamefont {Kappert}}, \bibinfo {author} {\bibfnamefont
  {D.}~\bibnamefont {Steil}}, \bibinfo {author} {\bibfnamefont {D.~R.}\
  \bibnamefont {Luke}}, \bibinfo {author} {\bibfnamefont {S.}~\bibnamefont
  {Steil}}, \ and\ \bibinfo {author} {\bibfnamefont {S.}~\bibnamefont
  {Mathias}},\ }\href {\doibase 10.1088/1367-2630/ab8aae} {\bibfield  {journal}
  {\bibinfo  {journal} {New Journal of Physics}\ }\textbf {\bibinfo {volume}
  {22}},\ \bibinfo {pages} {063012} (\bibinfo {year} {2020})}\BibitemShut
  {NoStop}%
\bibitem [{\citenamefont {Truhlar}\ \emph {et~al.}(2019)\citenamefont
  {Truhlar}, \citenamefont {Hiberty}, \citenamefont {Shaik}, \citenamefont
  {Gordon},\ and\ \citenamefont {Danovich}}]{Truhlar2019}%
  \BibitemOpen
  \bibfield  {author} {\bibinfo {author} {\bibfnamefont {D.~G.}\ \bibnamefont
  {Truhlar}}, \bibinfo {author} {\bibfnamefont {P.~C.}\ \bibnamefont
  {Hiberty}}, \bibinfo {author} {\bibfnamefont {S.}~\bibnamefont {Shaik}},
  \bibinfo {author} {\bibfnamefont {M.~S.}\ \bibnamefont {Gordon}}, \ and\
  \bibinfo {author} {\bibfnamefont {D.}~\bibnamefont {Danovich}},\ }\href
  {\doibase 10.1002/ange.201904609} {\bibfield  {journal} {\bibinfo  {journal}
  {Angewandte Chemie}\ }\textbf {\bibinfo {volume} {131}},\ \bibinfo {pages}
  {12460} (\bibinfo {year} {2019})}\BibitemShut {NoStop}%
\bibitem [{\citenamefont {Krylov}(2020)}]{Krylov2020}%
  \BibitemOpen
  \bibfield  {author} {\bibinfo {author} {\bibfnamefont {A.~I.}\ \bibnamefont
  {Krylov}},\ }\href {\doibase 10.1063/5.0018597} {\bibfield  {journal}
  {\bibinfo  {journal} {J. Chem. Phys.}\ }\textbf {\bibinfo {volume} {153}},\
  \bibinfo {pages} {080901} (\bibinfo {year} {2020})}\BibitemShut {NoStop}%
\bibitem [{\citenamefont {Rohwer}\ \emph {et~al.}(2011)\citenamefont {Rohwer},
  \citenamefont {Hellmann}, \citenamefont {Wiesenmayer}, \citenamefont {Sohrt},
  \citenamefont {Stange}, \citenamefont {Slomski}, \citenamefont {Carr},
  \citenamefont {Liu}, \citenamefont {Avila}, \citenamefont {Kalläne},
  \citenamefont {Mathias}, \citenamefont {Kipp}, \citenamefont {Rossnagel},\
  and\ \citenamefont {Bauer}}]{Rohwer2011}%
  \BibitemOpen
  \bibfield  {author} {\bibinfo {author} {\bibfnamefont {T.}~\bibnamefont
  {Rohwer}}, \bibinfo {author} {\bibfnamefont {S.}~\bibnamefont {Hellmann}},
  \bibinfo {author} {\bibfnamefont {M.}~\bibnamefont {Wiesenmayer}}, \bibinfo
  {author} {\bibfnamefont {C.}~\bibnamefont {Sohrt}}, \bibinfo {author}
  {\bibfnamefont {A.}~\bibnamefont {Stange}}, \bibinfo {author} {\bibfnamefont
  {B.}~\bibnamefont {Slomski}}, \bibinfo {author} {\bibfnamefont
  {A.}~\bibnamefont {Carr}}, \bibinfo {author} {\bibfnamefont {Y.}~\bibnamefont
  {Liu}}, \bibinfo {author} {\bibfnamefont {L.~M.}\ \bibnamefont {Avila}},
  \bibinfo {author} {\bibfnamefont {M.}~\bibnamefont {Kalläne}}, \bibinfo
  {author} {\bibfnamefont {S.}~\bibnamefont {Mathias}}, \bibinfo {author}
  {\bibfnamefont {L.}~\bibnamefont {Kipp}}, \bibinfo {author} {\bibfnamefont
  {K.}~\bibnamefont {Rossnagel}}, \ and\ \bibinfo {author} {\bibfnamefont
  {M.}~\bibnamefont {Bauer}},\ }\href {\doibase 10.1038/nature09829} {\bibfield
   {journal} {\bibinfo  {journal} {Nature}\ }\textbf {\bibinfo {volume}
  {471}},\ \bibinfo {pages} {490} (\bibinfo {year} {2011})}\BibitemShut
  {NoStop}%
\bibitem [{\citenamefont {Eich}\ \emph {et~al.}(2017)\citenamefont {Eich},
  \citenamefont {Plötzing}, \citenamefont {Rollinger}, \citenamefont
  {Emmerich}, \citenamefont {Adam}, \citenamefont {Chen}, \citenamefont
  {Kapteyn}, \citenamefont {Murnane}, \citenamefont {Plucinski}, \citenamefont
  {Steil}, \citenamefont {Stadtmüller}, \citenamefont {Cinchetti},
  \citenamefont {Aeschlimann}, \citenamefont {Schneider},\ and\ \citenamefont
  {Mathias}}]{Eich2017}%
  \BibitemOpen
  \bibfield  {author} {\bibinfo {author} {\bibfnamefont {S.}~\bibnamefont
  {Eich}}, \bibinfo {author} {\bibfnamefont {M.}~\bibnamefont {Plötzing}},
  \bibinfo {author} {\bibfnamefont {M.}~\bibnamefont {Rollinger}}, \bibinfo
  {author} {\bibfnamefont {S.}~\bibnamefont {Emmerich}}, \bibinfo {author}
  {\bibfnamefont {R.}~\bibnamefont {Adam}}, \bibinfo {author} {\bibfnamefont
  {C.}~\bibnamefont {Chen}}, \bibinfo {author} {\bibfnamefont {H.~C.}\
  \bibnamefont {Kapteyn}}, \bibinfo {author} {\bibfnamefont {M.~M.}\
  \bibnamefont {Murnane}}, \bibinfo {author} {\bibfnamefont {L.}~\bibnamefont
  {Plucinski}}, \bibinfo {author} {\bibfnamefont {D.}~\bibnamefont {Steil}},
  \bibinfo {author} {\bibfnamefont {B.}~\bibnamefont {Stadtmüller}}, \bibinfo
  {author} {\bibfnamefont {M.}~\bibnamefont {Cinchetti}}, \bibinfo {author}
  {\bibfnamefont {M.}~\bibnamefont {Aeschlimann}}, \bibinfo {author}
  {\bibfnamefont {C.~M.}\ \bibnamefont {Schneider}}, \ and\ \bibinfo {author}
  {\bibfnamefont {S.}~\bibnamefont {Mathias}},\ }\href {\doibase
  10.1126/sciadv.1602094} {\bibfield  {journal} {\bibinfo  {journal} {Science
  Advances}\ }\textbf {\bibinfo {volume} {3}},\ \bibinfo {pages} {e1602094}
  (\bibinfo {year} {2017})}\BibitemShut {NoStop}%
\bibitem [{\citenamefont {Nicholson}\ \emph {et~al.}(2018)\citenamefont
  {Nicholson}, \citenamefont {Lücke}, \citenamefont {Schmidt}, \citenamefont
  {Puppin}, \citenamefont {Rettig}, \citenamefont {Ernstorfer},\ and\
  \citenamefont {Wolf}}]{Nicholson2018}%
  \BibitemOpen
  \bibfield  {author} {\bibinfo {author} {\bibfnamefont {C.~W.}\ \bibnamefont
  {Nicholson}}, \bibinfo {author} {\bibfnamefont {A.}~\bibnamefont {Lücke}},
  \bibinfo {author} {\bibfnamefont {W.~G.}\ \bibnamefont {Schmidt}}, \bibinfo
  {author} {\bibfnamefont {M.}~\bibnamefont {Puppin}}, \bibinfo {author}
  {\bibfnamefont {L.}~\bibnamefont {Rettig}}, \bibinfo {author} {\bibfnamefont
  {R.}~\bibnamefont {Ernstorfer}}, \ and\ \bibinfo {author} {\bibfnamefont
  {M.}~\bibnamefont {Wolf}},\ }\href {\doibase 10.1126/science.aar4183}
  {\bibfield  {journal} {\bibinfo  {journal} {Science}\ }\textbf {\bibinfo
  {volume} {362}},\ \bibinfo {pages} {821} (\bibinfo {year}
  {2018})}\BibitemShut {NoStop}%
\bibitem [{\citenamefont {Wallauer}\ \emph {et~al.}(2021)\citenamefont
  {Wallauer}, \citenamefont {Raths}, \citenamefont {Stallberg}, \citenamefont
  {M\"unster}, \citenamefont {Brandstetter}, \citenamefont {Yang},
  \citenamefont {G\"udde}, \citenamefont {Puschnig}, \citenamefont {Soubatch},
  \citenamefont {Kumpf}, \citenamefont {Bocquet}, \citenamefont {Tautz},\ and\
  \citenamefont {H\"ofer}}]{Wallauer2020}%
  \BibitemOpen
  \bibfield  {author} {\bibinfo {author} {\bibfnamefont {R.}~\bibnamefont
  {Wallauer}}, \bibinfo {author} {\bibfnamefont {M.}~\bibnamefont {Raths}},
  \bibinfo {author} {\bibfnamefont {K.}~\bibnamefont {Stallberg}}, \bibinfo
  {author} {\bibfnamefont {L.}~\bibnamefont {M\"unster}}, \bibinfo {author}
  {\bibfnamefont {D.}~\bibnamefont {Brandstetter}}, \bibinfo {author}
  {\bibfnamefont {X.}~\bibnamefont {Yang}}, \bibinfo {author} {\bibfnamefont
  {J.}~\bibnamefont {G\"udde}}, \bibinfo {author} {\bibfnamefont
  {P.}~\bibnamefont {Puschnig}}, \bibinfo {author} {\bibfnamefont
  {S.}~\bibnamefont {Soubatch}}, \bibinfo {author} {\bibfnamefont
  {C.}~\bibnamefont {Kumpf}}, \bibinfo {author} {\bibfnamefont {F.~C.}\
  \bibnamefont {Bocquet}}, \bibinfo {author} {\bibfnamefont {F.~S.}\
  \bibnamefont {Tautz}}, \ and\ \bibinfo {author} {\bibfnamefont
  {U.}~\bibnamefont {H\"ofer}},\ }\href {\doibase 10.1126/science.abf3286}
  {\bibfield  {journal} {\bibinfo  {journal} {Science}\ }\textbf {\bibinfo
  {volume} {371}},\ \bibinfo {pages} {1056} (\bibinfo {year}
  {2021})}\BibitemShut {NoStop}%
\bibitem [{\citenamefont {Baumg\"artner}\ \emph {et~al.}(2022)\citenamefont
  {Baumg\"artner}, \citenamefont {Reuner}, \citenamefont {Metzger},
  \citenamefont {Kutnyakhov}, \citenamefont {Heber}, \citenamefont {Pressacco},
  \citenamefont {Min}, \citenamefont {Peixoto}, \citenamefont {Reiser},
  \citenamefont {Kim}, \citenamefont {Lu}, \citenamefont {Shayduk},
  \citenamefont {Izquierdo}, \citenamefont {Brenner}, \citenamefont {Roth},
  \citenamefont {Sch\"oll}, \citenamefont {Molodtsov}, \citenamefont {Wurth},
  \citenamefont {Reinert}, \citenamefont {Madsen}, \citenamefont
  {Popova-Gorelova},\ and\ \citenamefont {Scholz}}]{Baumgartner2022}%
  \BibitemOpen
  \bibfield  {author} {\bibinfo {author} {\bibfnamefont {K.}~\bibnamefont
  {Baumg\"artner}}, \bibinfo {author} {\bibfnamefont {M.}~\bibnamefont
  {Reuner}}, \bibinfo {author} {\bibfnamefont {C.}~\bibnamefont {Metzger}},
  \bibinfo {author} {\bibfnamefont {D.}~\bibnamefont {Kutnyakhov}}, \bibinfo
  {author} {\bibfnamefont {M.}~\bibnamefont {Heber}}, \bibinfo {author}
  {\bibfnamefont {F.}~\bibnamefont {Pressacco}}, \bibinfo {author}
  {\bibfnamefont {C.-H.}\ \bibnamefont {Min}}, \bibinfo {author} {\bibfnamefont
  {T.~R.~F.}\ \bibnamefont {Peixoto}}, \bibinfo {author} {\bibfnamefont
  {M.}~\bibnamefont {Reiser}}, \bibinfo {author} {\bibfnamefont
  {C.}~\bibnamefont {Kim}}, \bibinfo {author} {\bibfnamefont {W.}~\bibnamefont
  {Lu}}, \bibinfo {author} {\bibfnamefont {R.}~\bibnamefont {Shayduk}},
  \bibinfo {author} {\bibfnamefont {M.}~\bibnamefont {Izquierdo}}, \bibinfo
  {author} {\bibfnamefont {G.}~\bibnamefont {Brenner}}, \bibinfo {author}
  {\bibfnamefont {F.}~\bibnamefont {Roth}}, \bibinfo {author} {\bibfnamefont
  {A.}~\bibnamefont {Sch\"oll}}, \bibinfo {author} {\bibfnamefont
  {S.}~\bibnamefont {Molodtsov}}, \bibinfo {author} {\bibfnamefont
  {W.}~\bibnamefont {Wurth}}, \bibinfo {author} {\bibfnamefont
  {F.}~\bibnamefont {Reinert}}, \bibinfo {author} {\bibfnamefont
  {A.}~\bibnamefont {Madsen}}, \bibinfo {author} {\bibfnamefont
  {D.}~\bibnamefont {Popova-Gorelova}}, \ and\ \bibinfo {author} {\bibfnamefont
  {M.}~\bibnamefont {Scholz}},\ }\href {\doibase 10.1038/s41467-022-30404-6}
  {\bibfield  {journal} {\bibinfo  {journal} {Nature Communications}\ }\textbf
  {\bibinfo {volume} {13}},\ \bibinfo {pages} {2741} (\bibinfo {year}
  {2022})}\BibitemShut {NoStop}%
\bibitem [{\citenamefont {Neef}\ \emph {et~al.}(2022)\citenamefont {Neef},
  \citenamefont {Beaulieu}, \citenamefont {Hammer}, \citenamefont {Dong},
  \citenamefont {Maklar}, \citenamefont {Pincelli}, \citenamefont {Xian},
  \citenamefont {Wolf}, \citenamefont {Rettig}, \citenamefont {Pflaum},\ and\
  \citenamefont {Ernstorfer}}]{Neef2022}%
  \BibitemOpen
  \bibfield  {author} {\bibinfo {author} {\bibfnamefont {A.}~\bibnamefont
  {Neef}}, \bibinfo {author} {\bibfnamefont {S.}~\bibnamefont {Beaulieu}},
  \bibinfo {author} {\bibfnamefont {S.}~\bibnamefont {Hammer}}, \bibinfo
  {author} {\bibfnamefont {S.}~\bibnamefont {Dong}}, \bibinfo {author}
  {\bibfnamefont {J.}~\bibnamefont {Maklar}}, \bibinfo {author} {\bibfnamefont
  {T.}~\bibnamefont {Pincelli}}, \bibinfo {author} {\bibfnamefont {R.~P.}\
  \bibnamefont {Xian}}, \bibinfo {author} {\bibfnamefont {M.}~\bibnamefont
  {Wolf}}, \bibinfo {author} {\bibfnamefont {L.}~\bibnamefont {Rettig}},
  \bibinfo {author} {\bibfnamefont {J.}~\bibnamefont {Pflaum}}, \ and\ \bibinfo
  {author} {\bibfnamefont {R.}~\bibnamefont {Ernstorfer}},\ }\href {\doibase
  10.48550/arXiv.2204.06824} {\bibfield  {journal} {\bibinfo  {journal}
  {arXiv}\ } (\bibinfo {year} {2022}),\ 10.48550/arXiv.2204.06824}\BibitemShut
  {NoStop}%
\bibitem [{\citenamefont {Rohlfing}\ and\ \citenamefont
  {Louie}(2000)}]{Rohlfing2000}%
  \BibitemOpen
  \bibfield  {author} {\bibinfo {author} {\bibfnamefont {M.}~\bibnamefont
  {Rohlfing}}\ and\ \bibinfo {author} {\bibfnamefont {S.~G.}\ \bibnamefont
  {Louie}},\ }\href {\doibase 10.1103/PhysRevB.62.4927} {\bibfield  {journal}
  {\bibinfo  {journal} {Phys. Rev. B}\ }\textbf {\bibinfo {volume} {62}},\
  \bibinfo {pages} {4927} (\bibinfo {year} {2000})}\BibitemShut {NoStop}%
\bibitem [{\citenamefont {Casida}()}]{Casida1995}%
  \BibitemOpen
  \bibfield  {author} {\bibinfo {author} {\bibfnamefont {M.~E.}\ \bibnamefont
  {Casida}},\ }in\ \href {\doibase 10.1142/9789812830586_0005} {\emph {\bibinfo
  {booktitle} {Recent Advances in Density Functional Methods}}},\ pp.\ \bibinfo
  {pages} {155--192}\BibitemShut {NoStop}%
\bibitem [{\citenamefont {Onida}\ \emph {et~al.}(2002)\citenamefont {Onida},
  \citenamefont {Reining},\ and\ \citenamefont {Rubio}}]{Onida2002}%
  \BibitemOpen
  \bibfield  {author} {\bibinfo {author} {\bibfnamefont {G.}~\bibnamefont
  {Onida}}, \bibinfo {author} {\bibfnamefont {L.}~\bibnamefont {Reining}}, \
  and\ \bibinfo {author} {\bibfnamefont {A.}~\bibnamefont {Rubio}},\ }\href
  {\doibase 10.1103/RevModPhys.74.601} {\bibfield  {journal} {\bibinfo
  {journal} {Rev. Mod. Phys.}\ }\textbf {\bibinfo {volume} {74}},\ \bibinfo
  {pages} {601} (\bibinfo {year} {2002})}\BibitemShut {NoStop}%
\bibitem [{\citenamefont {Pickup}(1977)}]{Pickup1977}%
  \BibitemOpen
  \bibfield  {author} {\bibinfo {author} {\bibfnamefont {B.~T.}\ \bibnamefont
  {Pickup}},\ }\href {\doibase https://doi.org/10.1016/0301-0104(77)85131-8}
  {\bibfield  {journal} {\bibinfo  {journal} {Chemical Physics}\ }\textbf
  {\bibinfo {volume} {19}},\ \bibinfo {pages} {193} (\bibinfo {year}
  {1977})}\BibitemShut {NoStop}%
\bibitem [{\citenamefont {Ortiz}(2020)}]{Ortiz2020}%
  \BibitemOpen
  \bibfield  {author} {\bibinfo {author} {\bibfnamefont {J.~V.}\ \bibnamefont
  {Ortiz}},\ }\href {\doibase 10.1063/5.0016472} {\bibfield  {journal}
  {\bibinfo  {journal} {The Journal of Chemical Physics}\ }\textbf {\bibinfo
  {volume} {153}},\ \bibinfo {pages} {070902} (\bibinfo {year} {2020})},\
  \Eprint {http://arxiv.org/abs/https://doi.org/10.1063/5.0016472}
  {https://doi.org/10.1063/5.0016472} \BibitemShut {NoStop}%
\bibitem [{\citenamefont {Martin}(2003)}]{Martin2003}%
  \BibitemOpen
  \bibfield  {author} {\bibinfo {author} {\bibfnamefont {R.~L.}\ \bibnamefont
  {Martin}},\ }\href {\doibase 10.1063/1.1558471} {\bibfield  {journal}
  {\bibinfo  {journal} {The Journal of Chemical Physics}\ }\textbf {\bibinfo
  {volume} {118}},\ \bibinfo {pages} {4775} (\bibinfo {year}
  {2003})}\BibitemShut {NoStop}%
\bibitem [{\citenamefont {Benedict}\ \emph {et~al.}(1998)\citenamefont
  {Benedict}, \citenamefont {Shirley},\ and\ \citenamefont
  {Bohn}}]{Benedict1989}%
  \BibitemOpen
  \bibfield  {author} {\bibinfo {author} {\bibfnamefont {L.~X.}\ \bibnamefont
  {Benedict}}, \bibinfo {author} {\bibfnamefont {E.~L.}\ \bibnamefont
  {Shirley}}, \ and\ \bibinfo {author} {\bibfnamefont {R.~B.}\ \bibnamefont
  {Bohn}},\ }\href {\doibase 10.1103/PhysRevLett.80.4514} {\bibfield  {journal}
  {\bibinfo  {journal} {Phys. Rev. Lett.}\ }\textbf {\bibinfo {volume} {80}},\
  \bibinfo {pages} {4514} (\bibinfo {year} {1998})}\BibitemShut {NoStop}%
\bibitem [{\citenamefont {Dauth}\ \emph {et~al.}(2014)\citenamefont {Dauth},
  \citenamefont {Wiessner}, \citenamefont {Feyer}, \citenamefont {Sch\"oll},
  \citenamefont {Puschnig}, \citenamefont {Reinert},\ and\ \citenamefont
  {K\"ummel}}]{Dauth2014}%
  \BibitemOpen
  \bibfield  {author} {\bibinfo {author} {\bibfnamefont {M.}~\bibnamefont
  {Dauth}}, \bibinfo {author} {\bibfnamefont {M.}~\bibnamefont {Wiessner}},
  \bibinfo {author} {\bibfnamefont {V.}~\bibnamefont {Feyer}}, \bibinfo
  {author} {\bibfnamefont {A.}~\bibnamefont {Sch\"oll}}, \bibinfo {author}
  {\bibfnamefont {P.}~\bibnamefont {Puschnig}}, \bibinfo {author}
  {\bibfnamefont {F.}~\bibnamefont {Reinert}}, \ and\ \bibinfo {author}
  {\bibfnamefont {S.}~\bibnamefont {K\"ummel}},\ }\href {\doibase
  10.1088/1367-2630/16/10/103005} {\bibfield  {journal} {\bibinfo  {journal}
  {New J. Phys.}\ }\textbf {\bibinfo {volume} {16}},\ \bibinfo {pages} {103005}
  (\bibinfo {year} {2014})}\BibitemShut {NoStop}%
\bibitem [{\citenamefont {Damascelli}(2004)}]{Damascelli2004}%
  \BibitemOpen
  \bibfield  {author} {\bibinfo {author} {\bibfnamefont {A.}~\bibnamefont
  {Damascelli}},\ }\href@noop {} {\bibfield  {journal} {\bibinfo  {journal}
  {Phys. Scr.}\ }\textbf {\bibinfo {volume} {T109}},\ \bibinfo {pages} {61}
  (\bibinfo {year} {2004})}\BibitemShut {NoStop}%
\bibitem [{\citenamefont {Melania~Oana}\ and\ \citenamefont
  {Krylov}(2007)}]{Oana2007}%
  \BibitemOpen
  \bibfield  {author} {\bibinfo {author} {\bibfnamefont {C.}~\bibnamefont
  {Melania~Oana}}\ and\ \bibinfo {author} {\bibfnamefont {A.~I.}\ \bibnamefont
  {Krylov}},\ }\href {\doibase 10.1063/1.2805393} {\bibfield  {journal}
  {\bibinfo  {journal} {The Journal of Chemical Physics}\ }\textbf {\bibinfo
  {volume} {127}},\ \bibinfo {pages} {234106} (\bibinfo {year} {2007})},\
  \Eprint {http://arxiv.org/abs/https://doi.org/10.1063/1.2805393}
  {https://doi.org/10.1063/1.2805393} \BibitemShut {NoStop}%
\bibitem [{\citenamefont {Gozem}\ \emph {et~al.}(2015)\citenamefont {Gozem},
  \citenamefont {Gunina}, \citenamefont {Ichino}, \citenamefont {Osborn},
  \citenamefont {Stanton},\ and\ \citenamefont {Krylov}}]{Gozem2015}%
  \BibitemOpen
  \bibfield  {author} {\bibinfo {author} {\bibfnamefont {S.}~\bibnamefont
  {Gozem}}, \bibinfo {author} {\bibfnamefont {A.~O.}\ \bibnamefont {Gunina}},
  \bibinfo {author} {\bibfnamefont {T.}~\bibnamefont {Ichino}}, \bibinfo
  {author} {\bibfnamefont {D.~L.}\ \bibnamefont {Osborn}}, \bibinfo {author}
  {\bibfnamefont {J.~F.}\ \bibnamefont {Stanton}}, \ and\ \bibinfo {author}
  {\bibfnamefont {A.~I.}\ \bibnamefont {Krylov}},\ }\href {\doibase
  10.1021/acs.jpclett.5b01891} {\bibfield  {journal} {\bibinfo  {journal} {The
  Journal of Physical Chemistry Letters}\ }\textbf {\bibinfo {volume} {6}},\
  \bibinfo {pages} {4532} (\bibinfo {year} {2015})},\ \bibinfo {note} {pMID:
  26509428},\ \Eprint
  {http://arxiv.org/abs/https://doi.org/10.1021/acs.jpclett.5b01891}
  {https://doi.org/10.1021/acs.jpclett.5b01891} \BibitemShut {NoStop}%
\bibitem [{\citenamefont {Oana}\ and\ \citenamefont {Krylov}(2009)}]{Oana2009}%
  \BibitemOpen
  \bibfield  {author} {\bibinfo {author} {\bibfnamefont {C.~M.}\ \bibnamefont
  {Oana}}\ and\ \bibinfo {author} {\bibfnamefont {A.~I.}\ \bibnamefont
  {Krylov}},\ }\href {\doibase 10.1063/1.3231143} {\bibfield  {journal}
  {\bibinfo  {journal} {The Journal of Chemical Physics}\ }\textbf {\bibinfo
  {volume} {131}},\ \bibinfo {pages} {124114} (\bibinfo {year} {2009})},\
  \Eprint {http://arxiv.org/abs/https://doi.org/10.1063/1.3231143}
  {https://doi.org/10.1063/1.3231143} \BibitemShut {NoStop}%
\bibitem [{\citenamefont {Vidal}\ \emph {et~al.}(2020)\citenamefont {Vidal},
  \citenamefont {Krylov},\ and\ \citenamefont {Coriani}}]{Vidal2020}%
  \BibitemOpen
  \bibfield  {author} {\bibinfo {author} {\bibfnamefont {M.~L.}\ \bibnamefont
  {Vidal}}, \bibinfo {author} {\bibfnamefont {A.~I.}\ \bibnamefont {Krylov}}, \
  and\ \bibinfo {author} {\bibfnamefont {S.}~\bibnamefont {Coriani}},\ }\href
  {\doibase 10.1039/C9CP03695D} {\bibfield  {journal} {\bibinfo  {journal}
  {Phys. Chem. Chem. Phys.}\ }\textbf {\bibinfo {volume} {22}},\ \bibinfo
  {pages} {2693} (\bibinfo {year} {2020})}\BibitemShut {NoStop}%
\bibitem [{\citenamefont {Plasser}(2016)}]{Plasser2016}%
  \BibitemOpen
  \bibfield  {author} {\bibinfo {author} {\bibfnamefont {F.}~\bibnamefont
  {Plasser}},\ }\href {\doibase 10.1063/1.4949535} {\bibfield  {journal}
  {\bibinfo  {journal} {The Journal of Chemical Physics}\ }\textbf {\bibinfo
  {volume} {144}},\ \bibinfo {pages} {194107} (\bibinfo {year}
  {2016})}\BibitemShut {NoStop}%
\bibitem [{\citenamefont {Andrade}\ \emph {et~al.}(2015)\citenamefont
  {Andrade}, \citenamefont {Strubbe}, \citenamefont {{De Giovannini}},
  \citenamefont {Larsen}, \citenamefont {Oliveira}, \citenamefont
  {Alberdi-Rodriguez}, \citenamefont {Varas}, \citenamefont {Theophilou},
  \citenamefont {Helbig}, \citenamefont {Verstraete}, \citenamefont {Stella},
  \citenamefont {Nogueira}, \citenamefont {Aspuru-Guzik}, \citenamefont
  {Castro}, \citenamefont {Marques},\ and\ \citenamefont
  {Rubio}}]{Andrade2015}%
  \BibitemOpen
  \bibfield  {author} {\bibinfo {author} {\bibfnamefont {X.}~\bibnamefont
  {Andrade}}, \bibinfo {author} {\bibfnamefont {D.}~\bibnamefont {Strubbe}},
  \bibinfo {author} {\bibfnamefont {U.}~\bibnamefont {{De Giovannini}}},
  \bibinfo {author} {\bibfnamefont {A.~H.}\ \bibnamefont {Larsen}}, \bibinfo
  {author} {\bibfnamefont {M.~J.~T.}\ \bibnamefont {Oliveira}}, \bibinfo
  {author} {\bibfnamefont {J.}~\bibnamefont {Alberdi-Rodriguez}}, \bibinfo
  {author} {\bibfnamefont {A.}~\bibnamefont {Varas}}, \bibinfo {author}
  {\bibfnamefont {I.}~\bibnamefont {Theophilou}}, \bibinfo {author}
  {\bibfnamefont {N.}~\bibnamefont {Helbig}}, \bibinfo {author} {\bibfnamefont
  {M.~J.}\ \bibnamefont {Verstraete}}, \bibinfo {author} {\bibfnamefont
  {L.}~\bibnamefont {Stella}}, \bibinfo {author} {\bibfnamefont
  {F.}~\bibnamefont {Nogueira}}, \bibinfo {author} {\bibfnamefont
  {A.}~\bibnamefont {Aspuru-Guzik}}, \bibinfo {author} {\bibfnamefont
  {A.}~\bibnamefont {Castro}}, \bibinfo {author} {\bibfnamefont {M.~A.~L.}\
  \bibnamefont {Marques}}, \ and\ \bibinfo {author} {\bibfnamefont
  {A.}~\bibnamefont {Rubio}},\ }\href {\doibase 10.1039/C5CP00351B} {\bibfield
  {journal} {\bibinfo  {journal} {Physical Chemistry Chemical Physics}\
  }\textbf {\bibinfo {volume} {17}},\ \bibinfo {pages} {31371} (\bibinfo {year}
  {2015})}\BibitemShut {NoStop}%
\bibitem [{\citenamefont {Tancogne-Dejean}\ \emph {et~al.}(2020)\citenamefont
  {Tancogne-Dejean}, \citenamefont {Oliveira}, \citenamefont {Andrade},
  \citenamefont {Appel}, \citenamefont {Borca}, \citenamefont {{Le Breton}},
  \citenamefont {Buchholz}, \citenamefont {Castro}, \citenamefont {Corni},
  \citenamefont {Correa}, \citenamefont {{De Giovannini}}, \citenamefont
  {Delgado}, \citenamefont {Eich}, \citenamefont {Flick}, \citenamefont {Gil},
  \citenamefont {Gomez}, \citenamefont {Helbig}, \citenamefont {H{\"{u}}bener},
  \citenamefont {Jest{\"{a}}dt}, \citenamefont {Jornet-Somoza}, \citenamefont
  {Larsen}, \citenamefont {Lebedeva}, \citenamefont {L{\"{u}}ders},
  \citenamefont {Marques}, \citenamefont {Ohlmann}, \citenamefont {Pipolo},
  \citenamefont {Rampp}, \citenamefont {Rozzi}, \citenamefont {Strubbe},
  \citenamefont {Sato}, \citenamefont {Sch{\"{a}}fer}, \citenamefont
  {Theophilou}, \citenamefont {Welden},\ and\ \citenamefont
  {Rubio}}]{Tancogne-Dejean2020}%
  \BibitemOpen
  \bibfield  {author} {\bibinfo {author} {\bibfnamefont {N.}~\bibnamefont
  {Tancogne-Dejean}}, \bibinfo {author} {\bibfnamefont {M.~J.~T.}\ \bibnamefont
  {Oliveira}}, \bibinfo {author} {\bibfnamefont {X.}~\bibnamefont {Andrade}},
  \bibinfo {author} {\bibfnamefont {H.}~\bibnamefont {Appel}}, \bibinfo
  {author} {\bibfnamefont {C.~H.}\ \bibnamefont {Borca}}, \bibinfo {author}
  {\bibfnamefont {G.}~\bibnamefont {{Le Breton}}}, \bibinfo {author}
  {\bibfnamefont {F.}~\bibnamefont {Buchholz}}, \bibinfo {author}
  {\bibfnamefont {A.}~\bibnamefont {Castro}}, \bibinfo {author} {\bibfnamefont
  {S.}~\bibnamefont {Corni}}, \bibinfo {author} {\bibfnamefont {A.~A.}\
  \bibnamefont {Correa}}, \bibinfo {author} {\bibfnamefont {U.}~\bibnamefont
  {{De Giovannini}}}, \bibinfo {author} {\bibfnamefont {A.}~\bibnamefont
  {Delgado}}, \bibinfo {author} {\bibfnamefont {F.~G.}\ \bibnamefont {Eich}},
  \bibinfo {author} {\bibfnamefont {J.}~\bibnamefont {Flick}}, \bibinfo
  {author} {\bibfnamefont {G.}~\bibnamefont {Gil}}, \bibinfo {author}
  {\bibfnamefont {A.}~\bibnamefont {Gomez}}, \bibinfo {author} {\bibfnamefont
  {N.}~\bibnamefont {Helbig}}, \bibinfo {author} {\bibfnamefont
  {H.}~\bibnamefont {H{\"{u}}bener}}, \bibinfo {author} {\bibfnamefont
  {R.}~\bibnamefont {Jest{\"{a}}dt}}, \bibinfo {author} {\bibfnamefont
  {J.}~\bibnamefont {Jornet-Somoza}}, \bibinfo {author} {\bibfnamefont {A.~H.}\
  \bibnamefont {Larsen}}, \bibinfo {author} {\bibfnamefont {I.~V.}\
  \bibnamefont {Lebedeva}}, \bibinfo {author} {\bibfnamefont {M.}~\bibnamefont
  {L{\"{u}}ders}}, \bibinfo {author} {\bibfnamefont {M.~A.~L.}\ \bibnamefont
  {Marques}}, \bibinfo {author} {\bibfnamefont {S.~T.}\ \bibnamefont
  {Ohlmann}}, \bibinfo {author} {\bibfnamefont {S.}~\bibnamefont {Pipolo}},
  \bibinfo {author} {\bibfnamefont {M.}~\bibnamefont {Rampp}}, \bibinfo
  {author} {\bibfnamefont {C.~A.}\ \bibnamefont {Rozzi}}, \bibinfo {author}
  {\bibfnamefont {D.~A.}\ \bibnamefont {Strubbe}}, \bibinfo {author}
  {\bibfnamefont {S.~A.}\ \bibnamefont {Sato}}, \bibinfo {author}
  {\bibfnamefont {C.}~\bibnamefont {Sch{\"{a}}fer}}, \bibinfo {author}
  {\bibfnamefont {I.}~\bibnamefont {Theophilou}}, \bibinfo {author}
  {\bibfnamefont {A.}~\bibnamefont {Welden}}, \ and\ \bibinfo {author}
  {\bibfnamefont {A.}~\bibnamefont {Rubio}},\ }\href {\doibase
  10.1063/1.5142502} {\bibfield  {journal} {\bibinfo  {journal} {The Journal of
  Chemical Physics}\ }\textbf {\bibinfo {volume} {152}},\ \bibinfo {pages}
  {124119} (\bibinfo {year} {2020})}\BibitemShut {NoStop}%
\bibitem [{\citenamefont {Wopperer}\ \emph {et~al.}(2017)\citenamefont
  {Wopperer}, \citenamefont {{De Giovannini}},\ and\ \citenamefont
  {Rubio}}]{Wopperer2017}%
  \BibitemOpen
  \bibfield  {author} {\bibinfo {author} {\bibfnamefont {P.}~\bibnamefont
  {Wopperer}}, \bibinfo {author} {\bibfnamefont {U.}~\bibnamefont {{De
  Giovannini}}}, \ and\ \bibinfo {author} {\bibfnamefont {A.}~\bibnamefont
  {Rubio}},\ }\href {\doibase 10.1140/epjb/e2017-70548-3} {\bibfield  {journal}
  {\bibinfo  {journal} {The European Physical Journal B}\ }\textbf {\bibinfo
  {volume} {90}},\ \bibinfo {pages} {51} (\bibinfo {year} {2017})}\BibitemShut
  {NoStop}%
\bibitem [{\citenamefont {{De Giovannini}}\ \emph {et~al.}(2017)\citenamefont
  {{De Giovannini}}, \citenamefont {H{\"{u}}bener},\ and\ \citenamefont
  {Rubio}}]{DeGiovannini2017}%
  \BibitemOpen
  \bibfield  {author} {\bibinfo {author} {\bibfnamefont {U.}~\bibnamefont {{De
  Giovannini}}}, \bibinfo {author} {\bibfnamefont {H.}~\bibnamefont
  {H{\"{u}}bener}}, \ and\ \bibinfo {author} {\bibfnamefont {A.}~\bibnamefont
  {Rubio}},\ }\href {\doibase 10.1021/acs.jctc.6b00897} {\bibfield  {journal}
  {\bibinfo  {journal} {Journal of Chemical Theory and Computation}\ }\textbf
  {\bibinfo {volume} {13}},\ \bibinfo {pages} {265} (\bibinfo {year}
  {2017})}\BibitemShut {NoStop}%
\bibitem [{\citenamefont {L\"uftner}\ \emph
  {et~al.}(2014{\natexlab{b}})\citenamefont {L\"uftner}, \citenamefont {Ules},
  \citenamefont {Reinisch}, \citenamefont {Koller}, \citenamefont {Soubatch},
  \citenamefont {Tautz}, \citenamefont {Ramsey},\ and\ \citenamefont
  {Puschnig}}]{Lueftner2014}%
  \BibitemOpen
  \bibfield  {author} {\bibinfo {author} {\bibfnamefont {D.}~\bibnamefont
  {L\"uftner}}, \bibinfo {author} {\bibfnamefont {T.}~\bibnamefont {Ules}},
  \bibinfo {author} {\bibfnamefont {E.~M.}\ \bibnamefont {Reinisch}}, \bibinfo
  {author} {\bibfnamefont {G.}~\bibnamefont {Koller}}, \bibinfo {author}
  {\bibfnamefont {S.}~\bibnamefont {Soubatch}}, \bibinfo {author}
  {\bibfnamefont {F.~S.}\ \bibnamefont {Tautz}}, \bibinfo {author}
  {\bibfnamefont {M.~G.}\ \bibnamefont {Ramsey}}, \ and\ \bibinfo {author}
  {\bibfnamefont {P.}~\bibnamefont {Puschnig}},\ }\href {\doibase
  10.1073/pnas.1315716110} {\bibfield  {journal} {\bibinfo  {journal} {Proc.
  Nat. Acad. Sci. U. S. A.}\ }\textbf {\bibinfo {volume} {111}},\ \bibinfo
  {pages} {605} (\bibinfo {year} {2014}{\natexlab{b}})}\BibitemShut {NoStop}%
\bibitem [{\citenamefont {Dauth}\ \emph
  {et~al.}(2016{\natexlab{b}})\citenamefont {Dauth}, \citenamefont {Graus},
  \citenamefont {Schelter}, \citenamefont {Wie\ss{}ner}, \citenamefont
  {Sch\"oll}, \citenamefont {Reinert},\ and\ \citenamefont
  {K\"ummel}}]{Dauth2016}%
  \BibitemOpen
  \bibfield  {author} {\bibinfo {author} {\bibfnamefont {M.}~\bibnamefont
  {Dauth}}, \bibinfo {author} {\bibfnamefont {M.}~\bibnamefont {Graus}},
  \bibinfo {author} {\bibfnamefont {I.}~\bibnamefont {Schelter}}, \bibinfo
  {author} {\bibfnamefont {M.}~\bibnamefont {Wie\ss{}ner}}, \bibinfo {author}
  {\bibfnamefont {A.}~\bibnamefont {Sch\"oll}}, \bibinfo {author}
  {\bibfnamefont {F.}~\bibnamefont {Reinert}}, \ and\ \bibinfo {author}
  {\bibfnamefont {S.}~\bibnamefont {K\"ummel}},\ }\href {\doibase
  10.1103/PhysRevLett.117.183001} {\bibfield  {journal} {\bibinfo  {journal}
  {Phys. Rev. Lett.}\ }\textbf {\bibinfo {volume} {117}},\ \bibinfo {pages}
  {183001} (\bibinfo {year} {2016}{\natexlab{b}})}\BibitemShut {NoStop}%
\bibitem [{\citenamefont {Popova-Gorelova}\ \emph {et~al.}(2016)\citenamefont
  {Popova-Gorelova}, \citenamefont {K\"upper},\ and\ \citenamefont
  {Santra}}]{Popova2016}%
  \BibitemOpen
  \bibfield  {author} {\bibinfo {author} {\bibfnamefont {D.}~\bibnamefont
  {Popova-Gorelova}}, \bibinfo {author} {\bibfnamefont {J.}~\bibnamefont
  {K\"upper}}, \ and\ \bibinfo {author} {\bibfnamefont {R.}~\bibnamefont
  {Santra}},\ }\href {\doibase 10.1103/PhysRevA.94.013412} {\bibfield
  {journal} {\bibinfo  {journal} {Phys. Rev. A}\ }\textbf {\bibinfo {volume}
  {94}},\ \bibinfo {pages} {013412} (\bibinfo {year} {2016})}\BibitemShut
  {NoStop}%
\bibitem [{\citenamefont {Reuner}\ and\ \citenamefont
  {Popova-Gorelova}(2023)}]{Reuner2023}%
  \BibitemOpen
  \bibfield  {author} {\bibinfo {author} {\bibfnamefont {M.}~\bibnamefont
  {Reuner}}\ and\ \bibinfo {author} {\bibfnamefont {D.}~\bibnamefont
  {Popova-Gorelova}},\ }\href {\doibase 10.1103/PhysRevA.107.023101} {\bibfield
   {journal} {\bibinfo  {journal} {Phys. Rev. A}\ }\textbf {\bibinfo {volume}
  {107}},\ \bibinfo {pages} {023101} (\bibinfo {year} {2023})}\BibitemShut
  {NoStop}%
\bibitem [{\citenamefont {Hammon}\ and\ \citenamefont
  {K\"ummel}(2021)}]{Hammon2021}%
  \BibitemOpen
  \bibfield  {author} {\bibinfo {author} {\bibfnamefont {S.}~\bibnamefont
  {Hammon}}\ and\ \bibinfo {author} {\bibfnamefont {S.}~\bibnamefont
  {K\"ummel}},\ }\href {\doibase 10.1103/PhysRevA.104.012815} {\bibfield
  {journal} {\bibinfo  {journal} {Phys. Rev. A}\ }\textbf {\bibinfo {volume}
  {104}},\ \bibinfo {pages} {012815} (\bibinfo {year} {2021})}\BibitemShut
  {NoStop}%
\bibitem [{\citenamefont {Pomogaev}\ \emph {et~al.}(2021)\citenamefont
  {Pomogaev}, \citenamefont {Lee}, \citenamefont {Shaik}, \citenamefont
  {Filatov},\ and\ \citenamefont {Choi}}]{Pomogaev2021}%
  \BibitemOpen
  \bibfield  {author} {\bibinfo {author} {\bibfnamefont {V.}~\bibnamefont
  {Pomogaev}}, \bibinfo {author} {\bibfnamefont {S.}~\bibnamefont {Lee}},
  \bibinfo {author} {\bibfnamefont {S.}~\bibnamefont {Shaik}}, \bibinfo
  {author} {\bibfnamefont {M.}~\bibnamefont {Filatov}}, \ and\ \bibinfo
  {author} {\bibfnamefont {C.~H.}\ \bibnamefont {Choi}},\ }\href {\doibase
  10.1021/acs.jpclett.1c02494} {\bibfield  {journal} {\bibinfo  {journal} {The
  Journal of Physical Chemistry Letters}\ }\textbf {\bibinfo {volume} {12}},\
  \bibinfo {pages} {9963} (\bibinfo {year} {2021})},\ \bibinfo {note} {pMID:
  34617764},\ \Eprint
  {http://arxiv.org/abs/https://doi.org/10.1021/acs.jpclett.1c02494}
  {https://doi.org/10.1021/acs.jpclett.1c02494} \BibitemShut {NoStop}%
\bibitem [{\citenamefont {Mortensen}\ \emph {et~al.}(2005)\citenamefont
  {Mortensen}, \citenamefont {Hansen},\ and\ \citenamefont
  {Jacobsen}}]{Mortensen2005}%
  \BibitemOpen
  \bibfield  {author} {\bibinfo {author} {\bibfnamefont {J.~J.}\ \bibnamefont
  {Mortensen}}, \bibinfo {author} {\bibfnamefont {L.~B.}\ \bibnamefont
  {Hansen}}, \ and\ \bibinfo {author} {\bibfnamefont {K.~W.}\ \bibnamefont
  {Jacobsen}},\ }\href {\doibase 10.1103/PhysRevB.71.035109} {\bibfield
  {journal} {\bibinfo  {journal} {Phys. Rev. B}\ }\textbf {\bibinfo {volume}
  {71}},\ \bibinfo {pages} {035109} (\bibinfo {year} {2005})}\BibitemShut
  {NoStop}%
\bibitem [{\citenamefont {Enkovaara}\ \emph {et~al.}(2010)\citenamefont
  {Enkovaara}, \citenamefont {Rostgaard}, \citenamefont {Mortensen},
  \citenamefont {Chen}, \citenamefont {Dulak}, \citenamefont {Ferrighi},
  \citenamefont {Gavnholt}, \citenamefont {Glinsvad}, \citenamefont {Haikola},
  \citenamefont {Hansen}, \citenamefont {Kristoffersen}, \citenamefont
  {Kuisma}, \citenamefont {Larsen}, \citenamefont {Lehtovaara}, \citenamefont
  {Ljungberg}, \citenamefont {Lopez-Acevedo}, \citenamefont {Moses},
  \citenamefont {Ojanen}, \citenamefont {Olsen}, \citenamefont {Petzold},
  \citenamefont {Romero}, \citenamefont {Stausholm-Moller}, \citenamefont
  {Strange}, \citenamefont {Tritsaris}, \citenamefont {Vanin}, \citenamefont
  {Walter}, \citenamefont {Hammer}, \citenamefont {H\"akkinen}, \citenamefont
  {Madsen}, \citenamefont {Nieminen}, \citenamefont {Norskov}, \citenamefont
  {Puska}, \citenamefont {Rantala}, \citenamefont {Schiotz}, \citenamefont
  {Thygesen},\ and\ \citenamefont {Jacobsen}}]{Enkovaara2010}%
  \BibitemOpen
  \bibfield  {author} {\bibinfo {author} {\bibfnamefont {J.}~\bibnamefont
  {Enkovaara}}, \bibinfo {author} {\bibfnamefont {C.}~\bibnamefont
  {Rostgaard}}, \bibinfo {author} {\bibfnamefont {J.~J.}\ \bibnamefont
  {Mortensen}}, \bibinfo {author} {\bibfnamefont {J.}~\bibnamefont {Chen}},
  \bibinfo {author} {\bibfnamefont {M.}~\bibnamefont {Dulak}}, \bibinfo
  {author} {\bibfnamefont {L.}~\bibnamefont {Ferrighi}}, \bibinfo {author}
  {\bibfnamefont {J.}~\bibnamefont {Gavnholt}}, \bibinfo {author}
  {\bibfnamefont {C.}~\bibnamefont {Glinsvad}}, \bibinfo {author}
  {\bibfnamefont {V.}~\bibnamefont {Haikola}}, \bibinfo {author} {\bibfnamefont
  {H.~A.}\ \bibnamefont {Hansen}}, \bibinfo {author} {\bibfnamefont {H.~H.}\
  \bibnamefont {Kristoffersen}}, \bibinfo {author} {\bibfnamefont
  {M.}~\bibnamefont {Kuisma}}, \bibinfo {author} {\bibfnamefont {A.~H.}\
  \bibnamefont {Larsen}}, \bibinfo {author} {\bibfnamefont {L.}~\bibnamefont
  {Lehtovaara}}, \bibinfo {author} {\bibfnamefont {M.}~\bibnamefont
  {Ljungberg}}, \bibinfo {author} {\bibfnamefont {O.}~\bibnamefont
  {Lopez-Acevedo}}, \bibinfo {author} {\bibfnamefont {P.~G.}\ \bibnamefont
  {Moses}}, \bibinfo {author} {\bibfnamefont {J.}~\bibnamefont {Ojanen}},
  \bibinfo {author} {\bibfnamefont {T.}~\bibnamefont {Olsen}}, \bibinfo
  {author} {\bibfnamefont {V.}~\bibnamefont {Petzold}}, \bibinfo {author}
  {\bibfnamefont {N.~A.}\ \bibnamefont {Romero}}, \bibinfo {author}
  {\bibfnamefont {J.}~\bibnamefont {Stausholm-Moller}}, \bibinfo {author}
  {\bibfnamefont {M.}~\bibnamefont {Strange}}, \bibinfo {author} {\bibfnamefont
  {G.~A.}\ \bibnamefont {Tritsaris}}, \bibinfo {author} {\bibfnamefont
  {M.}~\bibnamefont {Vanin}}, \bibinfo {author} {\bibfnamefont
  {M.}~\bibnamefont {Walter}}, \bibinfo {author} {\bibfnamefont
  {B.}~\bibnamefont {Hammer}}, \bibinfo {author} {\bibfnamefont
  {H.}~\bibnamefont {H\"akkinen}}, \bibinfo {author} {\bibfnamefont {G.~K.~H.}\
  \bibnamefont {Madsen}}, \bibinfo {author} {\bibfnamefont {R.~M.}\
  \bibnamefont {Nieminen}}, \bibinfo {author} {\bibfnamefont {J.~K.}\
  \bibnamefont {Norskov}}, \bibinfo {author} {\bibfnamefont {M.}~\bibnamefont
  {Puska}}, \bibinfo {author} {\bibfnamefont {T.~T.}\ \bibnamefont {Rantala}},
  \bibinfo {author} {\bibfnamefont {J.}~\bibnamefont {Schiotz}}, \bibinfo
  {author} {\bibfnamefont {K.~S.}\ \bibnamefont {Thygesen}}, \ and\ \bibinfo
  {author} {\bibfnamefont {K.~W.}\ \bibnamefont {Jacobsen}},\ }\href {\doibase
  10.1088/0953-8984/22/25/253202} {\bibfield  {journal} {\bibinfo  {journal}
  {Journal of Physics: Condensed Matter}\ }\textbf {\bibinfo {volume} {22}},\
  \bibinfo {pages} {253202} (\bibinfo {year} {2010})}\BibitemShut {NoStop}%
\bibitem [{\citenamefont {Larsen}\ \emph {et~al.}(2017)\citenamefont {Larsen},
  \citenamefont {Mortensen}, \citenamefont {Blomqvist}, \citenamefont
  {Castelli}, \citenamefont {Christensen}, \citenamefont {Dulak}, \citenamefont
  {Friis}, \citenamefont {Groves}, \citenamefont {Hammer}, \citenamefont
  {Hargus}, \citenamefont {Hermes}, \citenamefont {Jennings}, \citenamefont
  {Jensen}, \citenamefont {Kermode}, \citenamefont {Kitchin}, \citenamefont
  {Kolsbjerg}, \citenamefont {Kubal}, \citenamefont {Kaasbjerg}, \citenamefont
  {Lysgaard}, \citenamefont {Maronsson}, \citenamefont {Maxson}, \citenamefont
  {Olsen}, \citenamefont {Pastewka}, \citenamefont {Peterson}, \citenamefont
  {Rostgaard}, \citenamefont {Schiotz}, \citenamefont {Sch\"utt}, \citenamefont
  {Strange}, \citenamefont {Thygesen}, \citenamefont {Vegge}, \citenamefont
  {Vilhelmsen}, \citenamefont {Walter}, \citenamefont {Zeng},\ and\
  \citenamefont {Jacobsen}}]{Larsen2017}%
  \BibitemOpen
  \bibfield  {author} {\bibinfo {author} {\bibfnamefont {A.~H.}\ \bibnamefont
  {Larsen}}, \bibinfo {author} {\bibfnamefont {J.~J.}\ \bibnamefont
  {Mortensen}}, \bibinfo {author} {\bibfnamefont {J.}~\bibnamefont
  {Blomqvist}}, \bibinfo {author} {\bibfnamefont {I.~E.}\ \bibnamefont
  {Castelli}}, \bibinfo {author} {\bibfnamefont {R.}~\bibnamefont
  {Christensen}}, \bibinfo {author} {\bibfnamefont {M.}~\bibnamefont {Dulak}},
  \bibinfo {author} {\bibfnamefont {J.}~\bibnamefont {Friis}}, \bibinfo
  {author} {\bibfnamefont {M.~N.}\ \bibnamefont {Groves}}, \bibinfo {author}
  {\bibfnamefont {B.}~\bibnamefont {Hammer}}, \bibinfo {author} {\bibfnamefont
  {C.}~\bibnamefont {Hargus}}, \bibinfo {author} {\bibfnamefont {E.~D.}\
  \bibnamefont {Hermes}}, \bibinfo {author} {\bibfnamefont {P.~C.}\
  \bibnamefont {Jennings}}, \bibinfo {author} {\bibfnamefont {P.~B.}\
  \bibnamefont {Jensen}}, \bibinfo {author} {\bibfnamefont {J.}~\bibnamefont
  {Kermode}}, \bibinfo {author} {\bibfnamefont {J.~R.}\ \bibnamefont
  {Kitchin}}, \bibinfo {author} {\bibfnamefont {E.~L.}\ \bibnamefont
  {Kolsbjerg}}, \bibinfo {author} {\bibfnamefont {J.}~\bibnamefont {Kubal}},
  \bibinfo {author} {\bibfnamefont {K.}~\bibnamefont {Kaasbjerg}}, \bibinfo
  {author} {\bibfnamefont {S.}~\bibnamefont {Lysgaard}}, \bibinfo {author}
  {\bibfnamefont {J.~B.}\ \bibnamefont {Maronsson}}, \bibinfo {author}
  {\bibfnamefont {T.}~\bibnamefont {Maxson}}, \bibinfo {author} {\bibfnamefont
  {T.}~\bibnamefont {Olsen}}, \bibinfo {author} {\bibfnamefont
  {L.}~\bibnamefont {Pastewka}}, \bibinfo {author} {\bibfnamefont
  {A.}~\bibnamefont {Peterson}}, \bibinfo {author} {\bibfnamefont
  {C.}~\bibnamefont {Rostgaard}}, \bibinfo {author} {\bibfnamefont
  {J.}~\bibnamefont {Schiotz}}, \bibinfo {author} {\bibfnamefont
  {O.}~\bibnamefont {Sch\"utt}}, \bibinfo {author} {\bibfnamefont
  {M.}~\bibnamefont {Strange}}, \bibinfo {author} {\bibfnamefont {K.~S.}\
  \bibnamefont {Thygesen}}, \bibinfo {author} {\bibfnamefont {T.}~\bibnamefont
  {Vegge}}, \bibinfo {author} {\bibfnamefont {L.}~\bibnamefont {Vilhelmsen}},
  \bibinfo {author} {\bibfnamefont {M.}~\bibnamefont {Walter}}, \bibinfo
  {author} {\bibfnamefont {Z.}~\bibnamefont {Zeng}}, \ and\ \bibinfo {author}
  {\bibfnamefont {K.~W.}\ \bibnamefont {Jacobsen}},\ }\href
  {http://stacks.iop.org/0953-8984/29/i=27/a=273002} {\bibfield  {journal}
  {\bibinfo  {journal} {Journal of Physics: Condensed Matter}\ }\textbf
  {\bibinfo {volume} {29}},\ \bibinfo {pages} {273002} (\bibinfo {year}
  {2017})}\BibitemShut {NoStop}%
\bibitem [{\citenamefont {Dirac}(1930)}]{Dirac1930}%
  \BibitemOpen
  \bibfield  {author} {\bibinfo {author} {\bibfnamefont {P.~A.~M.}\
  \bibnamefont {Dirac}},\ }\href {\doibase 10.1017/S0305004100016108}
  {\bibfield  {journal} {\bibinfo  {journal} {Mathematical Proceedings of the
  Cambridge Philosophical Society}\ }\textbf {\bibinfo {volume} {26}},\
  \bibinfo {pages} {376} (\bibinfo {year} {1930})}\BibitemShut {NoStop}%
\bibitem [{\citenamefont {Perdew}\ and\ \citenamefont
  {Zunger}(1981)}]{Perdew1981}%
  \BibitemOpen
  \bibfield  {author} {\bibinfo {author} {\bibfnamefont {J.~P.}\ \bibnamefont
  {Perdew}}\ and\ \bibinfo {author} {\bibfnamefont {A.}~\bibnamefont
  {Zunger}},\ }\href {\doibase 10.1103/PhysRevB.23.5048} {\bibfield  {journal}
  {\bibinfo  {journal} {Phys. Rev. B}\ }\textbf {\bibinfo {volume} {23}},\
  \bibinfo {pages} {5048} (\bibinfo {year} {1981})}\BibitemShut {NoStop}%
\bibitem [{\citenamefont {Troullier}\ and\ \citenamefont
  {Martins}(1991)}]{Troullier1991}%
  \BibitemOpen
  \bibfield  {author} {\bibinfo {author} {\bibfnamefont {N.}~\bibnamefont
  {Troullier}}\ and\ \bibinfo {author} {\bibfnamefont {J.~L.}\ \bibnamefont
  {Martins}},\ }\href {\doibase 10.1103/PhysRevB.43.1993} {\bibfield  {journal}
  {\bibinfo  {journal} {Phys. Rev. B}\ }\textbf {\bibinfo {volume} {43}},\
  \bibinfo {pages} {1993} (\bibinfo {year} {1991})}\BibitemShut {NoStop}%
\bibitem [{\citenamefont {Harris}\ \emph {et~al.}(2020)\citenamefont {Harris},
  \citenamefont {Millman}, \citenamefont {van~der Walt}, \citenamefont
  {Gommers}, \citenamefont {Virtanen}, \citenamefont {Cournapeau},
  \citenamefont {Wieser}, \citenamefont {Taylor}, \citenamefont {Berg},
  \citenamefont {Smith}, \citenamefont {Kern}, \citenamefont {Picus},
  \citenamefont {Hoyer}, \citenamefont {van Kerkwijk}, \citenamefont {Brett},
  \citenamefont {Haldane}, \citenamefont {del R{'{\i}}o}, \citenamefont
  {Wiebe}, \citenamefont {Peterson}, \citenamefont {G{'{e}}rard-Marchant},
  \citenamefont {Sheppard}, \citenamefont {Reddy}, \citenamefont {Weckesser},
  \citenamefont {Abbasi}, \citenamefont {Gohlke},\ and\ \citenamefont
  {Oliphant}}]{Harris2020}%
  \BibitemOpen
  \bibfield  {author} {\bibinfo {author} {\bibfnamefont {C.~R.}\ \bibnamefont
  {Harris}}, \bibinfo {author} {\bibfnamefont {K.~J.}\ \bibnamefont {Millman}},
  \bibinfo {author} {\bibfnamefont {S.~J.}\ \bibnamefont {van~der Walt}},
  \bibinfo {author} {\bibfnamefont {R.}~\bibnamefont {Gommers}}, \bibinfo
  {author} {\bibfnamefont {P.}~\bibnamefont {Virtanen}}, \bibinfo {author}
  {\bibfnamefont {D.}~\bibnamefont {Cournapeau}}, \bibinfo {author}
  {\bibfnamefont {E.}~\bibnamefont {Wieser}}, \bibinfo {author} {\bibfnamefont
  {J.}~\bibnamefont {Taylor}}, \bibinfo {author} {\bibfnamefont
  {S.}~\bibnamefont {Berg}}, \bibinfo {author} {\bibfnamefont {N.~J.}\
  \bibnamefont {Smith}}, \bibinfo {author} {\bibfnamefont {R.}~\bibnamefont
  {Kern}}, \bibinfo {author} {\bibfnamefont {M.}~\bibnamefont {Picus}},
  \bibinfo {author} {\bibfnamefont {S.}~\bibnamefont {Hoyer}}, \bibinfo
  {author} {\bibfnamefont {M.~H.}\ \bibnamefont {van Kerkwijk}}, \bibinfo
  {author} {\bibfnamefont {M.}~\bibnamefont {Brett}}, \bibinfo {author}
  {\bibfnamefont {A.}~\bibnamefont {Haldane}}, \bibinfo {author} {\bibfnamefont
  {J.~F.}\ \bibnamefont {del R{'{\i}}o}}, \bibinfo {author} {\bibfnamefont
  {M.}~\bibnamefont {Wiebe}}, \bibinfo {author} {\bibfnamefont
  {P.}~\bibnamefont {Peterson}}, \bibinfo {author} {\bibfnamefont
  {P.}~\bibnamefont {G{'{e}}rard-Marchant}}, \bibinfo {author} {\bibfnamefont
  {K.}~\bibnamefont {Sheppard}}, \bibinfo {author} {\bibfnamefont
  {T.}~\bibnamefont {Reddy}}, \bibinfo {author} {\bibfnamefont
  {W.}~\bibnamefont {Weckesser}}, \bibinfo {author} {\bibfnamefont
  {H.}~\bibnamefont {Abbasi}}, \bibinfo {author} {\bibfnamefont
  {C.}~\bibnamefont {Gohlke}}, \ and\ \bibinfo {author} {\bibfnamefont {T.~E.}\
  \bibnamefont {Oliphant}},\ }\href {\doibase 10.1038/s41586-020-2649-2}
  {\bibfield  {journal} {\bibinfo  {journal} {Nature}\ }\textbf {\bibinfo
  {volume} {585}},\ \bibinfo {pages} {357} (\bibinfo {year}
  {2020})}\BibitemShut {NoStop}%
\bibitem [{\citenamefont {Yabana}\ and\ \citenamefont
  {Bertsch}(1996)}]{Yabana1996}%
  \BibitemOpen
  \bibfield  {author} {\bibinfo {author} {\bibfnamefont {K.}~\bibnamefont
  {Yabana}}\ and\ \bibinfo {author} {\bibfnamefont {G.~F.}\ \bibnamefont
  {Bertsch}},\ }\href {\doibase 10.1103/PhysRevB.54.4484} {\bibfield  {journal}
  {\bibinfo  {journal} {Phys. Rev. B}\ }\textbf {\bibinfo {volume} {54}},\
  \bibinfo {pages} {4484} (\bibinfo {year} {1996})}\BibitemShut {NoStop}%
\bibitem [{\citenamefont {Yabana}\ \emph {et~al.}(2006)\citenamefont {Yabana},
  \citenamefont {Nakatsukasa}, \citenamefont {Iwata},\ and\ \citenamefont
  {Bertsch}}]{Yabana2006}%
  \BibitemOpen
  \bibfield  {author} {\bibinfo {author} {\bibfnamefont {K.}~\bibnamefont
  {Yabana}}, \bibinfo {author} {\bibfnamefont {T.}~\bibnamefont {Nakatsukasa}},
  \bibinfo {author} {\bibfnamefont {J.-I.}\ \bibnamefont {Iwata}}, \ and\
  \bibinfo {author} {\bibfnamefont {G.~F.}\ \bibnamefont {Bertsch}},\ }\href
  {\doibase https://doi.org/10.1002/pssb.200642005} {\bibfield  {journal}
  {\bibinfo  {journal} {physica status solidi (b)}\ }\textbf {\bibinfo {volume}
  {243}},\ \bibinfo {pages} {1121} (\bibinfo {year} {2006})},\ \Eprint
  {http://arxiv.org/abs/https://onlinelibrary.wiley.com/doi/pdf/10.1002/pssb.200642005}
  {https://onlinelibrary.wiley.com/doi/pdf/10.1002/pssb.200642005} \BibitemShut
  {NoStop}%
\bibitem [{\citenamefont {{De Giovannini}}\ \emph {et~al.}(2015)\citenamefont
  {{De Giovannini}}, \citenamefont {Larsen},\ and\ \citenamefont
  {Rubio}}]{DeGiovannini2015}%
  \BibitemOpen
  \bibfield  {author} {\bibinfo {author} {\bibfnamefont {U.}~\bibnamefont {{De
  Giovannini}}}, \bibinfo {author} {\bibfnamefont {A.~H.}\ \bibnamefont
  {Larsen}}, \ and\ \bibinfo {author} {\bibfnamefont {A.}~\bibnamefont
  {Rubio}},\ }\href {https://doi.org/10.1140/epjb/e2015-50808-0} {\bibfield
  {journal} {\bibinfo  {journal} {The European Physical Journal B}\ }\textbf
  {\bibinfo {volume} {88}},\ \bibinfo {pages} {56} (\bibinfo {year}
  {2015})}\BibitemShut {NoStop}%
\end{thebibliography}%


\end{document}