\section{Related Work}\label{sec:related}



\paragraph{Flag manifolds}
Besides being mathematically interesting objects~\cite{wiggerman1998fundamental,donagi2008glsms,alekseevsky1997flag}, flags and flag manifolds have been explored by a series of works from Nishimori~\etal addressing subspace independent component analysis (ICA) via Riemannian optimization~\cite{nishimori2006riemannian0, nishimori2006riemannian1, nishimori2007flag, nishimori2008natural,nishimori2007flag,nishimori2006riemannian}. Nested sequences of subspaces (e.g. flags) appear in 
the weights in principal component analysis (PCA) \cite{ye2022optimization} and the result of a wavelet transform \cite{kirby2001geometric}.



\paragraph{Flag manifolds in computer vision} The utilization of flag manifolds in computer vision is a recent development. Ma~\etal~\cite{ma2021flag} employ nested subspace methods to compare large datasets. Additionally, they port self-organizing mappings to work on flag manifolds, enabling parameterization of a set of flags of a fixed type. This method was applied to hyper-spectral image data analysis~\cite{ma2022self}. Ye~\etal~\cite{ye2022optimization} derive closed-form analytic expressions for the set of operators required for Riemannian optimization algorithms on the flag manifold, while Nguyen~\cite{nguyen2022closed} provides closed-form expressions for logarithmic maps and geodesics on flag manifolds. Marrinan~\etal~\cite{marrinan2014finding} investigate the averaging of Grassmanians into flags, demonstrating that flag means behave more like medians and are therefore more robust to the presence of outliers among the subspaces being averaged. Building on this work, they utilize flag averages to improve the detection of chemical plumes in hyperspectral videos~\cite{marrinan2016flag}. Finally, Mankovich~\etal~\cite{mankovich2022flag} also average Grassmannians into flags by providing the median as a flag and an algorithm to compute it. 


