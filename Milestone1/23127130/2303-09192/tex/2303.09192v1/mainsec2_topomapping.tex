\subsection{VPR for Loop Closing}

The topological map $\mathcal{G} = (\mathcal{I}, \mathcal{A})$ initialized by the active exploration experience in Sec.~\ref{active_explore} is unidirectionally connected in the temporal axis. Each node~(a panoramic RGB image observation) is just connected with its preceding node and next node, failing to reflect the nodes' spatial adjacency. We propose to further complete the initial map $\mathcal{G}$ by adding edges to any two unconnected nodes if they possess a high visual similarity. In this work, we adopt VLAD-based visual place recognition~(VPR)~\cite{VLAD, netVLAD} to measure the ``visual similarity'' between two nodes.

Specifically, given $N$ unidirectionally connected image nodes collected during exploration in a room scene, we extract the local SIFT~\cite{SIFT} feature for each image. Then we get the global VLAD~\cite{VLAD} descriptor for each image by first clustering all SIFT features with K-Means~\cite{K-means} into $k$ centroids~(in our case $k=16$), and then stacking the residuals between the local SIFT features and centroids. After VLAD descriptors construction, we store all VLAD features into a ball tree~\cite{liuBallTree,bansal2019-lb-wayptnav} with leaf size 60. Then we can query each image's top-N ``most visually similar'' images from the corresponding ball tree, the node pairs whose similarity score is below a threshold~(in our case 1.15) are added edges. 

\begin{table}[t]
    \centering
        \caption{Total Loop Closing Time for Topological Mapping on Gibson 14 Room Scenes. Hardware: 10 cores of Intel Xeon Platinum 8268, 32GB RAM, and an SSD. SPTM requires an Nvidia A100 GPU.}
    \begin{tabular}{c|c}
    \toprule
      Methods   & Total Time Spent \\
      \hline
      SPTM~\cite{savinov2018semiparametric}  & $\approx 2.1$ hrs \\
      \textbf{VLAD-Based VPR (Ours)} & $\approx 0.2$ hrs\\
      \bottomrule
    \end{tabular}
    \label{tab:vpr_time}
\end{table}

It is worth noting that our VLAD-based VPR is more efficient for loop closing than SPTM~\cite{savinov2018semiparametric} which uses a binary classification network that requires exhaustive pairwise checking to detect loops. We report the average loop closing time of the two methods on all the 14 Gibson test rooms in Table~\ref{tab:vpr_time}, showing VLAD-based VPR's speed advantage.

Apart from the VPR, we train a model named \textit{ActionAssigner} to assign an action list to each new edge. The architecture of \textit{ActionAssigner} is similar to \textit{MotionPlanner}, except that \textit{ActionAssigner} predicts a sequence of actions with two node features as input, while \textit{MotionPlanner} is a one-step action predictor~(predict just one action).

After topological mapping, the completed topological map represents a room scene through the edges between nodes and the actions corresponding to each edge. It reflects both spatial adjacency and traversability of the room scene so that it can be used for navigation tasks. Given the image observations for the start and goal positions, we localize them on a topological map via the same VPR procedure. Once localized, we apply Dijkstra's algorithm~\cite{dijkstra} to find the shortest path between the two nodes. We can then navigate the agent from the start position to the goal without metric information.