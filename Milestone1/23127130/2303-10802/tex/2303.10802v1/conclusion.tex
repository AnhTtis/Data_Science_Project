\section{Conclusion}
\label{sec:conclusion}
% \noindent
This paper proposed a new peer-agreement based sample selection technique, PASS, for noisy-label learning 
% which aims to enhance the training process of robust classifiers for noisy labels. 
to improve the performance of robust classifiers.
We show that PASS can be easily integrated into SOTA noisy-label learning methods~\cite{li2020dividemix, Garg_2023_WACV, xu2021faster, feng2021ssr, zheltonozhskii2022contrast, nishi2021augmentation}
to improve their classification accuracy results on several noisy-label learning benchmarks, including CIFAR-100~\cite{krizhevsky2009learning}, Red Mini-ImageNet from CNWL~\cite{xu2021faster}, Animal-10N~\cite{song2019selfie}, CIFAR-N~\cite{wei2022learning}, Clothing1M~\cite{xiao2015learning}, Mini-Webvision~\cite{li2017webvision}, and Imagenet~\cite{deng2009imagenet}. 
We plan to work on a theoretical analysis of PASS to improve our understanding of the proposed method. 
%applicability and effectiveness in real-world scenarios
%Furthermore, an automated thresholding technique (Otsu~\cite{Otsu1979threshold}) is employed to replace local-clustering models, facilitating the separation of clean and noisy training samples. Extensive testing and empirical analysis are conducted to evaluate the proposed technique 
%on several IDN benchmarks, including CIFAR-100~\cite{krizhevsky2009learning}, Red Mini-ImageNet from Controlled Noisy Web Labels (CNWL)~\cite{xu2021faster}, ANIMAL-10N~\cite{song2019selfie}, CIFAR-N~\cite{wei2022learning}, CLOTHING-1M~\cite{xiao2015learning}, Mini-Webvision~\cite{li2017webvision}, and ImageNet~\cite{deng2009imagenet}. 
%Whilst, theoretical analysis will be a crucial research for future work, as it will help to ensure its applicability and effectiveness in real-world scenarios
% The collected findings and empirical analysis demonstrate that the suggested method outperforms state-of-the-art methods, particularly in the presence of real-world noise, and is comparable to CLOTHING-1M and Mini-Webvision. The ablation study clearly shows the superiority of Otsu with probability agreement.
% \gustavo{Future work: theoretical analysis?}