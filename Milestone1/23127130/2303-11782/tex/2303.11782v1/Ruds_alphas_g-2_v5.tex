
\documentclass[aps,prl,twocolumn,showpacs,nofootinbib,superscriptaddress]{revtex4-1}

\usepackage{graphicx}

\usepackage[colorlinks=true,linkcolor=blue,citecolor=blue,urlcolor=blue]{hyperref}

\begin{document}

\title{Novel Method to Reliably Determine the QCD Coupling from $R_{\rm uds}$ Measurements and its effects to Muon $g-2$ and $\alpha(M_Z^2)$ within the Tau-Charm Energy Region}

\author{Jian-Ming Shen}\email{shenjm@hnu.edu.cn}
\affiliation{School of Physics and Electronics, Hunan Provincial Key Laboratory of High-Energy Scale Physics and Applications, Hunan University, Changsha 410082, People's Republic of China}

\author{Bing-Hai Qin}
\affiliation{School of Physics and Electronics, Hunan Provincial Key Laboratory of High-Energy Scale Physics and Applications, Hunan University, Changsha 410082, People's Republic of China}

\author{Jiang Yan}
\affiliation{Department of Physics, Chongqing Key Laboratory for Strongly Coupled Physics, Chongqing University, Chongqing 401331, People's Republic of China}

\author{Sheng-Quan Wang}
\affiliation{Department of Physics, Guizhou Minzu University, Guiyang 550025, People's Republic of China}

\author{Xing-Gang Wu}\email{wuxg@cqu.edu.cn}
\affiliation{Department of Physics, Chongqing Key Laboratory for Strongly Coupled Physics, Chongqing University, Chongqing 401331, People's Republic of China}

\date{\today}

\begin{abstract}

We present a novel method for precisely determining the QCD running coupling from $R_{\rm uds}(s)$ measurements in electron-positron annihilation. When calculating the fixed-order perturbative QCD (pQCD) approximant of $R_{\rm uds}$, its effective coupling constant $\alpha_s(Q_*^2)$ is determined by using the principle of maximum conformality, a rigorous scale-setting method for gauge theories, whose resultant pQCD series satisfies all the requirements of renormalization group. Contribution due to the uncalculated higher-order terms is given by the Bayesian analysis. Using $R_{\rm uds}$ data measured by the KEDR detector at $22$ centre-of-mass energies between $1.84$ and $3.72$ GeV, we obtain $\alpha_s(M_Z^2)=0.1227^{+0.0117}_{-0.0132}({\rm exp.})\pm0.0016({\rm the.})$, where the theoretical uncertainty (the.) is negligible compared to the experimental one (exp.). Numerical analyses confirm that the new method for calculating $R_{\rm uds}(s)$ leads to a negligible renormalization scale dependence, a significant stabilization of the perturbative series, and a significant reduction of theoretical uncertainty. It thus provides a reliable theoretical basis for precise determination of the QCD running coupling from $R_{\rm uds}$ Measurements at future Tau-Charm Facility. It can also be applied for the precise determination of the hadronic contribution to muon $g-2$ and QED coupling $\alpha(M_Z^2)$ within the tau-charm energy range.

%\pacs{14.80.Bn, 12.38.Bx, 12.38.Cy}

\end{abstract}

\maketitle

\emph{Introduction.}---Quantum chromodynamics (QCD) is the fundamental non-Abelian gauge theory of strong interactions. Its running coupling ($\alpha_s$) sets the strength of the strong interaction among quarks and gluons, which is crucial and deserves the best possible precision. The strong running couplings become weak at short distances due to the property of asymptotic freedom, allowing perturbative calculation of physical observables involving large momentum transfer~\cite{Gross:1973id, Politzer:1973fx}. The strong running coupling in itself is not a physical observable, but rather a quantity defined in the context of perturbation theory, enters into perturbative QCD (pQCD) predictions for experimentally measurable observables. Its value must be inferred from such measurements and is subject to experimental and theoretical uncertainties \cite{Workman:2022ynf}.

Total hadronic $e^+e^-$ annihilation rate $R$ is a fundamental observable in QCD, which provides one of the cleanest platforms for determining $\alpha_s$~\cite{Chetyrkin:1994js}. The $R$ value also contributes to the standard model (SM) prediction for the muon anomalous magnetic moment $a_\mu={(g-2)_\mu}/2$ and the QED running coupling evaluated at the $Z$ pole, $\alpha(M_Z^2)$, i.e., for a review see Ref.\cite{Aoyama:2020ynm}. Many experiments have measured the $R$ value. The recent data \cite{BESIII:2021wib, Anashin:2018vdo} were given by the BES III detector at BEPC II~\cite{BESIII:2009fln} and the KEDR detector at the VEPP-4M $e^+e^-$ collider \cite{Anashin:2013twa}. A collection of all available $R$ data is given in Ref.\cite{Workman:2022ynf}. Theoretically, the $R$ value is evaluated in massless pQCD~\cite{Chetyrkin:1996ela, Davier:2005xq}, and its QCD corrections have now been calculated in the $\overline{\rm MS}$-scheme to order $\alpha_s^4$~\cite{Chetyrkin:1979bj, Gorishnii:1990vf, Surguladze:1990tg, Baikov:2008jh, Baikov:2010je, Baikov:2012er, Baikov:2012zm, Baikov:2012zn}. It has been found the power suppressed finite-quark-mass effects are well under control~\cite{Chetyrkin:2000zk, Harlander:2002ur, Kiyo:2009gb} and the same applies to mixed QCD and electroweak corrections \cite{Czarnecki:1996ei, Harlander:1998cmq}.

The $R$ for the continuum light hadron (containing $u$, $d$ and $s$ quarks) production, denoted by $R_{\rm uds}$, is usually adopted to test the validity of pQCD calculation in relatively low energy region~\cite{Kuhn:1998ze, Martin:1999bp}. The measured $R_{\rm uds}(s)$ excludes the contribution from resonances and reflects the lowest order cross section for the inclusive light hadronic event production through one photon annihilation of $e^+e^-$. So it can be directly compared with the pQCD prediction and be directly used to extract $\alpha_s$, or equivalently the QCD scale parameter $\Lambda$.

At present the pQCD calculations for $R_{\rm uds}$ are usually analyzed by using conventional scale-setting method, i.e., one calculates the central value by simply setting the renormalization scale $\mu_r$ equal to the center-of-mass energy $\mu_r=\sqrt{s}$; the theory uncertainties are estimated by varying the renormalization scale over an arbitrary range such as $\sqrt{s}/2<\mu_r<2\sqrt{s}$. This leads to conventional renormalization scheme and scale ambiguities, and makes the scale uncertainty be one of the most important systematic errors for pQCD prediction.

The principle of maximum conformality (PMC)~\cite{Brodsky:2011ta, Brodsky:2011ig, Mojaza:2012mf, Brodsky:2012rj, Brodsky:2013vpa} has been proposed to eliminate the conventional renormalization scale-and-scheme ambiguities. The $\alpha_s$-running behavior is governed by the renormalization group equation (RGE), and then the $\{\beta_i\}$-terms emerged in perturbative series can be inversely adopted for fixing the correct value of $\alpha_s$. The PMC single-scale-setting approach (PMCs)~\cite{Shen:2017pdu, Yan:2022foz} determines an overall effective $\alpha_s$ (its argument is called as the PMC scale) for any fixed order prediction with the help of RGE. The PMC scale can be treated as the effective momentum flow of the process. It has been demonstrated that the PMC prediction is free of renormalization scale-and-scheme ambiguities up to any fixed order~\cite{Wu:2018cmb, Wu:2019mky}, being consistent with the fundamental renormalization group approaches~\cite{Stueckelberg:1953dz, Peterman:1978tb, rge4, Wu:2014iba} and the self-consistency requirements of the renormalization group~\cite{Brodsky:2012ms, Wu:2013ei}. The PMC reduces in the Abelian limit to the Gell-Mann-Low method~\cite{GellMann:1954fq} and it provides a solid way to extend the well-known Brodsky-Lepage-Mackenzie (BLM) method~\cite{Brodsky:1982gc} to all orders.

In this work, we will first adopt the PMCs approach to deal with the perturbatives series of $R_{\rm uds}(s)$. Contributions due to uncalculated higher-order will be estimated by using the Bayesian analysis. Using the predicted $R_{\rm uds}(s)$ as the basic input, we will extract the exact value of $\alpha_s$ from the KEDR data on $R_{\rm uds}(s)$ and then calculate its effect to Muon $g-2$ and $\alpha(M_Z^2)$.

\emph{Calculation technology.}---Total hadronic $e^+e^-$ annihilation rate $R(s)$ is related to the theoretically calculable Adler function $D$ as follows~\cite{Adler:1974gd},
\begin{eqnarray}\label{eq:dispersion}
D(Q^2)=-12\pi^2Q^2\frac{d}{d Q^2}\Pi(Q^2)=\int_{4m_\pi^2}^{\infty}\frac{Q^2R(s)ds}{(s+Q^2)^2}.
\end{eqnarray}
Here the Adler function $D$ is defined as the logarithmic derivative of the hadronic vacuum polarization function $\Pi$, which can be written in terms of $\Pi$ and the photon field anomalous dimension, $\gamma$, e.g.~\cite{Baikov:2012zm}
\begin{equation}
D(\alpha_s) = 12\pi^2 \left[\gamma(\alpha_s) - \beta(\alpha_s)\frac{\partial} {\partial{\alpha_s}}{\Pi}(Q^2,{\alpha_s})\right]. \label{eq:Dexpression}
\end{equation}
$\gamma$ and $\Pi$ are given by the perturbative expansions,
\begin{eqnarray}\label{eq:gammaPi}
\gamma=\frac{d_R}{16\pi^2}\sum\limits_{i\geq 0} {\gamma_i}{\left(\frac{\alpha_s}{\pi}\right)^i}, \;\;
\Pi=\frac{d_R}{16\pi^2}\sum\limits_{i\geq 0}{\Pi_i}{\left(\frac{\alpha_s}{\pi}\right)^i}, \nonumber
\end{eqnarray}
where $d_R=N_c$ is the dimension of the quark representation of the colour gauge group. The coefficients $\gamma_i=(\sum_f q_f^2)\gamma_i^{\rm ns}+(\sum_f q_f)^2 \gamma_i^{\rm si}$ and $\Pi_i=(\sum_f q_f^2)\Pi_i^{\rm ns}+(\sum_f q_f)^2 \Pi_i^{\rm si}$, where the superscripts ``ns'' and ``si'' denote the non-singlet and the singlet components, respectively. The singlet contribution starts from order-$\alpha_s^3$, i.e., $\Pi_0^{\rm si}=\Pi_1^{\rm si}=\Pi_2^{\rm si}=0$, $\gamma_0^{\rm si}=\gamma_1^{\rm si}=\gamma_2^{\rm si}=0$. All these perturbative coefficients $\gamma^{\rm ns}_i$, $\gamma^{\rm si}_i$, $\Pi^{\rm ns}_i$ and $\Pi^{\rm si}_i$ up to four-loop level can be found in Ref.\cite{Baikov:2012zm}.

Using the perturbative expansions of $\gamma$ and $\Pi$, one can then obtain the perturbative expansion for $R(s)$. As for $R_{\rm uds}(s)$, the perturbative expression reads
\begin{eqnarray}\label{eq:Rudsconv}
R_{\rm uds}^{(\ell)}(s) = 2\left[1+\sum_{i=1}^{\ell}r_i \left(\frac{\alpha_s(s)}{\pi}\right)^i\right],
\end{eqnarray}
where $\ell$ specifies the known loop level of the QCD correction, the renormalization scale is set to $\mu_r^2=s$. The results for generic values of $\mu_r$ can be easily recovered by using the standard RGE evolution. The perturbative coefficients $r_i$ can be divided into conformal parts ($r_{i,0}$) and non-conformal parts (proportional to $\beta_i$), i.e. $r_i=r_{i,0}+\mathcal{O}(\{\beta_i\})$. The $\{\beta_i\}$-pattern at different orders exist special degeneracies~\cite{Brodsky:2013vpa, Mojaza:2012mf, Bi:2015wea}, which lead to
\begin{eqnarray}
r_1 &=& r_{1,0}, \\
r_2 &=& r_{2,0} + \beta_0 r_{2,1}, \\
r_3 &=& r_{3,0} + \beta_1 r_{2,1} + 2{\beta_0}r_{3,1} + \beta_0^2 r_{3,2}, \\
r_4 &=& r_{4,0} + {\beta_2}{r_{2,1}} + 2{\beta_1}{r_{3,1}} + \frac{5}{2}{\beta_1}{\beta_0}{r_{3,2}} \nonumber\\
&& + 3{\beta_0}{r_{4,1}}+ 3\beta_0^2{r_{4,2}} + \beta_0^3{r_{4,3}}, \\
&& \hspace{-4.5mm} \cdots  \nonumber
\end{eqnarray}
where
\begin{eqnarray}\label{eq:rij}
r_{i(\geq 1),0} &=& {3\over{4}}\gamma^{\rm ns}_i, \;\;\; r_{i(\geq 2),1} = {3\over{4}}\Pi^{\rm ns}_{i-1}, \nonumber\\
r_{i(\geq 3),2} &=& -{\pi^2 \over {4}}\gamma^{\rm ns}_{i-2}, \;\;\; r_{i(\geq 4),3} = -{3\pi^2\over{4}}\Pi^{\rm ns}_{i-3}.
\end{eqnarray}
It is noted that for $R_{\rm uds}$, only $u$, $d$ and $s$ quarks are produced, thus the number of active flavours is $n_f=3$. Since $\sum_{f=u,d,s} q_f=0$, the singlet contribution vanishes in the present considered three-flavor case.
The $\gamma$ describes the QCD-induced corrections to the running of QED coupling constant $\alpha$ in $\overline{\rm MS}$ scheme \cite{Baikov:2012zm}, thus its coefficients $\gamma^{\rm ns}_i$ are kept as conformal coefficients, and represents the intrinsic perturbative nature of $R(s)$. Starting from $r_3$, terms proportional to $\pi^2$ arise due to continuation of the spacelike perturbative results into the timelike domain. These ``$\pi^2$-terms'' also called ``kinematical terms'', and can be predicted from those of lower order. It is necessary to emphasize that, Eq. (\ref{eq:Rudsconv}) only partially retains the effects due to continuation of the spacelike perturbative results into the timelike domain, and has certain shortcomings (see, e.g., \cite{Kataev:1995vh, Shirkov:2000qv, Prosperi:2006hx, Nesterenko:2017wpb}). As shown in Eq.(\ref{eq:rij}), all ``$\pi^2$-terms'' are nonconformal, thus will be ressummed to certain level in the PMCs scale-setting procedure.

Following the standard procedure of the PMCs approach~\cite{Shen:2017pdu, Wu:2019mky}, the overall renormalization scale can be determined by requiring all the nonconformal $\{\beta_i\}$-terms vanish, the pQCD approximant (\ref{eq:Rudsconv}) then changes to the following conformal series,
\begin{equation}\label{eq:Rudspmc}
R_{\rm uds}^{(\ell)}(s)|_{\rm PMCs}=2\left[1+\sum_{i=1}^{\ell}r_{i,0}\left(\frac{\alpha_s(Q_*^2)}{\pi}\right)^i\right],
\end{equation}
where the PMC scale $Q_{*}$ can be fixed up to N$^{(\ell-2)}$LL-accuracy, i.e. $\ln Q^2_*/s$ can be expanded as a power series over $\alpha_s(Q_*^2)$,
\begin{equation}\label{eq:PMCscale}
\ln\frac{Q_*^2}{s} = \sum^{\ell-2}_{i=0} S_i \left(\frac{\alpha_s(Q_*^2)}{\pi}\right)^i,
\end{equation}
where the coefficients $S_i\;(i=0,1,2)$ reads,
\begin{eqnarray}
S_0 &=& -\frac{\Pi^{\rm ns}_1}{\gamma^{\rm ns}_1}, \label{eq:s0}\\
S_1 &=& \frac{2\gamma^{\rm ns}_2 \Pi^{\rm ns}_1}{{\gamma^{\rm ns}_1}^2}-\frac{2 \Pi^{\rm ns}_2}{\gamma^{\rm ns}_1} +\beta_0 \left(\frac{{\Pi^{\rm ns}_1}^2}{{\gamma^{\rm ns}_1}^2}+\frac{\pi^2}{3}\right), \label{eq:s1}\\
S_2 &=& -\frac{4 {\gamma^{\rm ns}_2}^2 \Pi^{\rm ns}_1}{{\gamma^{\rm ns}_1}^3}+\frac{3 \gamma^{\rm ns}_3 \Pi^{\rm ns}_1}{{\gamma^{\rm ns}_1}^2}
   +\frac{4 \gamma^{\rm ns}_2 \Pi^{\rm ns}_2}{{\gamma^{\rm ns}_1}^2}-\frac{3 \Pi^{\rm ns}_3}{\gamma^{\rm ns}_1} \nonumber\\
&& +\beta_0 \left(\frac{\pi^2 \gamma^{\rm ns}_2}{3 \gamma^{\rm ns}_1}-\frac{5 \gamma^{\rm ns}_2 {\Pi^{\rm ns}_1}^2}{{\gamma^{\rm ns}_1}^3}+\frac{6 \Pi^{\rm ns}_1 \Pi^{\rm ns}_2}{{\gamma^{\rm ns}_1}^2}\right) \nonumber\\
&& +\beta_1\left(\frac{3{\Pi^{\rm ns}_1}^2}{2{\gamma^{\rm ns}_1}^2}+\frac{\pi^2}{2}\right) -\beta_0^2\frac{2{\Pi^{\rm ns}_1}^3}{{\gamma^{\rm ns}_1}^3}.
\end{eqnarray}
Eq.(\ref{eq:PMCscale}) shows that the logarithmic form $\ln Q^2_*/s$ is a power series in $\alpha_s$, which resums all the known $\{\beta_i\}$-terms via the RGE, and is independent of $\mu_r$ at any fixed order. Together with the $\mu_r$-independent conformal coefficients, the resulting pQCD series is exactly scheme-and-scale independent~\cite{Wu:2018cmb} and the conventional renormalization scale ambiguity is eliminated.

The resulting conformal series (\ref{eq:Rudspmc}) with an overall $\alpha_s(Q_*)$ provides not only precise pQCD predictions for the known fixed-order, but also a reliable basis for estimating the contributions from the uncalculated higher-order (UHO) terms. As an estimation of the UHO terms of the perturbative series, we adopt a Bayesian-based approach (BA) \cite{Cacciari:2011ze, Shen:2022nyr} to quantify it in terms of a probability distribution. The conditional probability density function (p.d.f.) $f_c(c_{n}|c_l,c_{l+1},\dots,c_k)$ for a generic (uncalculated) coefficient $c_{n}$ ($n>k$) of any possible perturbative series $\rho_k=\sum_{i=l}^{k}c_i\alpha_s^i$ with given coefficients $\{c_l,c_{l+1},\dots,c_k\}$ is given by
\begin{eqnarray}
\label{eq:conditionalpdf1}
f_c(c_n|c_l,\cdots,c_k)=\int h_0(c_n|\bar c)f_{\bar c}(\bar c|c_l,\cdots,c_k) {\rm d}{\bar c},\;\;
\end{eqnarray}
where ${\bar c}$ ($>0$) is a common boundary for the absolute values of all the known coefficients $\{c_l,\dots,c_k\}$ and the unknown coefficient $c_n$ one want to evaluate. $h_0(c_n|\bar c)$ is the conditional p.d.f. of $c_n$ given ${\bar c}$. The conditional p.d.f. of ${\bar c}$ given coefficients $\{c_l,\cdots,c_k\}$, $f_{\bar c}({\bar c}|c_l,\cdots,c_k)$, can be determined by applying the Bayes' theorem,
\begin{eqnarray}
\label{eq:bayes}
f_{\bar c}(\bar c|c_l,\cdots,c_k)=\frac{h(c_l,\cdots,c_k|\bar c)g_0(\bar c)}{\int h(c_l,\cdots,c_k|\bar c)g_0(\bar c) {\rm d}{\bar c}}\;,
\end{eqnarray}
where $h(c_l,\cdots,c_k|\bar c)$ is the \emph{likelihood function} for $\bar{c}$ ; i.e., the joint p.d.f. for the coefficients viewed as a function of $\bar{c}$, evaluated with coefficients actually obtained in the calculation. The function $g_0(\bar c)$ is the prior p.d.f. for ${\bar c}$. Both $g_0(\bar c)$ and $h_0(c_i|\bar c)$ depend on the model assumption. Here we use the CH model \cite{Cacciari:2011ze}, which suggests: both $\ln{\bar c}$ and $c_i$ are equally probable for all their possible values; all the coefficients that we know and that we want to evaluate are mutually independent with the exception for the common bound (${\bar c}$), which results in $h(c_l,\cdots,c_k|{\bar c})=\prod_{i=l}^k h_0(c_i|{\bar c})$. Using the CH model, we obtain a symmetric posterior distribution for negative and positive $c_n$: a central plateau with suppressed tails \cite{Cacciari:2011ze, Shen:2022nyr}. The knowledge of p.d.f. $f_c(c_{n}|c_l,c_{l+1},\dots,c_k)$ allows one to calculate the degree-of-belief (DoB) that the value of $c_{n}$ belongs to some credible interval (CI). The symmetric smallest CI of fixed $p\%$ DoB for $c_{n}$ is,
\begin{eqnarray}\label{eq:CI}
c_n\in[-c_n^{(p)},c_n^{(p)}]\;,
\end{eqnarray}
where the boundary $c_n^{(p)}$ is defined implicitly by
\begin{eqnarray}\label{eq:DoB}
p\% = \int_{-c_n^{(p)}}^{c_n^{(p)}} f_c(c_n |c_l,\dots,c_k) {\rm d} c_n.
\end{eqnarray}
We take $p\%=95.5\%$ in the following calculation.

\emph{Predictions of $R_{\rm uds}^{(4)}$.}---The PMC prediction for $R_{\rm uds}(s)$ up to order $\alpha_s^4$ reads,
\begin{eqnarray}\label{eq:pmcs}
&&\hspace{-7mm} \frac{1}{2}R_{\rm uds}^{(4)}(s)|_{\rm PMCs}=1+\frac{\alpha_s(Q_*^2)}{\pi}+0.2174\alpha_s^2(Q_*^2) \nonumber\\
&&\hspace{20mm} +0.1108\alpha_s^3(Q_*^2)+0.0698\alpha_s^4(Q_*^2),
\end{eqnarray}
where $Q_*$ can be fixed up to N$^2$LL accuracy,
\begin{eqnarray}\label{eq:pmcscale}
\ln\frac{Q_*^2}{s}=0.2249+1.5427\alpha_s(Q_*^2)+2.4933\alpha_s^2(Q_*^2).
\end{eqnarray}
Both the PMC conformal series (\ref{eq:pmcs}) and the PMC scale (\ref{eq:pmcscale}) are scale-independent, which will have residual scale dependence due to uncalculated terms~\cite{Zheng:2013uja}. The UHO coefficients predicted by using BA are $r_{5,0}/\pi^5\in[-0.4622,0.4622]$ for the pQCD approximant (\ref{eq:pmcs}) and $S_3/\pi^3\in[-4.4159,4.4159]$ for the PMC scale (\ref{eq:pmcscale}).

As a comparison, we present the conventional prediction for $R_{\rm uds}(s)$, i.e., taking $\mu_r=\sqrt{s}$,
\begin{eqnarray}\label{eq:conv}
&&\hspace{-5mm} \frac{1}{2}R_{\rm uds}^{(4)}(s)|_{\rm Conv.}=1+\frac{\alpha_s(s)}{\pi}+0.1661\alpha_s^2(s) \nonumber\\
&&\hspace{22mm} -0.3317\alpha_s^3(s)-1.0972\alpha_s^4(s).
\end{eqnarray}
Because the known coefficients of the conventional pQCD series are scale-dependent at every order, the BA can only be applied after one specifies the choices for the renormalization scale, thus introducing extra uncertainties for the BA. Such extra uncertainty can be evaluated by varing the renormalization scale $\mu_r$ in the range $\sqrt{s}/2<\mu_r<2\sqrt{s}$. The next UHO coefficient of the pQCD approximant (\ref{eq:conv}) predicted by using BA is $r_5(\mu_r=\sqrt{s})/\pi^5\in[-1.5931,1.5931]$.

Eq.(\ref{eq:pmcs}) and Eq.(\ref{eq:conv}) give the relative importance of different perturbative terms in the pQCD series, which indicate that the much-improved convergence for the pQCD approximant can be obtained by using the PMCs approach. This conclusion can be illustrated more clearly by table \ref{tab:Ruds}, where numerical results of the total and individual-order contributions for $R_{\rm uds}^{(4)}$ with fixed $\sqrt{s}/\Lambda=8$ are presented. Here and following, unless otherwise specified, $\Lambda=\Lambda_{\overline{\rm MS}}^{(3)}$, represents the $\Lambda$ parameter of $\overline{\rm MS}$ scheme in three-flavor QCD.

\begin{table}[htbp]
\centering
\caption{Total and individual-order contributions for $R_{\rm uds}^{(4)}(s)$ with fixed $\sqrt{s}/\Lambda=8$ under conventional (Conv.) and PMCs scale-setting approaches. The conventional predictions are for $\mu_{r}=\sqrt{s}$.} \label{tab:Ruds}
\begin{tabular}{lcccccc}
\hline
& ~LO~ & ~NLO~ & ~N$^{2}$LO~ & ~N$^{3}$LO~ & ~N$^{4}$LO~ & ~Total~ \\  \hline
~Conv.~ & $2$ & $0.1643$ & $0.0221$ & $-0.0114$ & $-0.0097$ & $2.1653$ \\  \hline
~PMCs~ & $2$ & $0.1426$ & $0.0218$ & $0.0025$ & $0.0003$ & $2.1672$\\  \hline
\end{tabular}
\end{table}

\begin{figure}[htbp]
\centering
\includegraphics[width=0.45\textwidth]{RudsConv8}\vspace{2mm}
\includegraphics[width=0.45\textwidth]{RudsPMCs8}
\caption{$R_{\rm uds}(s)$ versus the renormalization scale $\mu_r$ at fixed $\sqrt{s}/\Lambda=8$ using conventional (upper) and PMCs (lower) scale-setting approaches, respectively. The dotted, dashed, dash-doted, and solid lines represent $R_{\rm uds}(s)$ at order $\alpha_s$, order $\alpha_s^2$, order $\alpha_s^3$, and order $\alpha_s^4$, respectively.}
\label{fig:RudsVSscale}
\end{figure}

We present the pQCD approximant of $R_{\rm uds}(s)$ versus the renormalization scale $\mu_r$ before and after applying the PMCs with fixed $\sqrt{s}/\Lambda=8$ in Fig.~\ref{fig:RudsVSscale}. The renormalization scale dependence under conventional scale-setting is moderate when more-and-more loop terms have been added, and the scale variation in $\sqrt{s}/2<\mu_r<2\sqrt{s}$ becomes $\sim0.8\%$ for the four-loop prediction $R_{\rm uds}^{(4)}(s)$. The PMC series is scale invariant at any fixed order, which also shows a good convergent behavior and can be treated as the intrinsic perturbative nature of the series.

\begin{figure}[htb]
\includegraphics[width=0.45\textwidth]{Ruds.pdf}
\caption{$R_{\rm uds}(s)$ as a function of $\sqrt{s}/\Lambda$ with different QCD corrections. The dashed curves from up to down are for conventional predictions by taking $\mu_r\equiv \sqrt{s}$ at order $\alpha_s^2$, order $\alpha_s^3$, and order $\alpha_s^4$, respectively. The thin curves from down to up are for PMCs predictions at order $\alpha_s^2$, order $\alpha_s^3$, and order $\alpha_s^4$, respectively.}
\label{fig:Ruds}
\end{figure}

Moreover, we present the perturbative behavior of $R^{(4)}_{\rm uds}(s)$ in a larger energy range by Fig. \ref{fig:Ruds}, where the dashed curves from up to down are for conventional predictions (\ref{eq:conv}) at order $\alpha_s^2$, order $\alpha_s^3$, and order $\alpha_s^4$, respectively, and the thin curves from down to up are for PMC predictions (\ref{eq:pmcs}) at order $\alpha_s^2$, order $\alpha_s^3$, and order $\alpha_s^4$, respectively.
Fig. \ref{fig:Ruds} indicates that the loop convergence of $R_{\rm uds}^{(4)}(s)$ has been markedly improved after applying the PMCs scale-setting procedure. The curves corresponding to the PMC prediction (\ref{eq:pmcs}) at order-$\alpha_s^3$ and order-$\alpha_s^4$ are nearly indistinguishable from each other. Whereas, the conventional prediction (\ref{eq:conv}), whose validity range has been demonstrated to be strictly limited to $\sqrt{s}/\Lambda>{\rm exp}(\pi/2)\simeq4.81$ \cite{Nesterenko:2017wpb},  converges rather slowly when the center-of-mass energy $\sqrt{s}$ approaches this value, and the corresponding curves start to swerve quite above the boundary of its convergence range.

\emph{Determination of $\alpha_s$.}---We adopt the PMC prediction (\ref{eq:pmcs}) as the input to fit the $R_{\rm uds}$ data in the energy range $1.84\;{\rm GeV}\sim 3.72\;{\rm GeV}$ measured by KEDR Collaboration~\cite{Anashin:2018vdo}. All the data summarized in Table $16$ of Ref.\cite{Anashin:2018vdo} are not independent but rather have point-by-point correlated effects, then the least squares (LS) estimators are determined by the minimum of $\chi^2$ function~\cite{Workman:2022ynf},
\begin{eqnarray}
\chi^2(\Lambda) = \left(\textbf{e}-\textbf{t}\right)^T V^{-1}\left(\textbf{e}-\textbf{t}\right),
\end{eqnarray}
where $\textbf{e}=(R_{\rm uds}^{\rm exp.}(s_1),R_{\rm uds}^{\rm exp.}(s_2),\cdots,R_{\rm uds}^{\rm exp.}(s_{N}))$ is the column vector composed of $N=22$ experimental data, and $\textbf{t}=(R_{\rm uds}^{\rm the.}(s_1),R_{\rm uds}^{\rm the.}(s_2),\cdots,R_{\rm uds}^{\rm the.}(s_N))$ is the corresponding column vector composed of theoretical predictions. The superscript $T$ denotes the transpose. $V^{-1}$ is the inverse covariance matrix which is derived from statistical errors and systematic uncertainties taking into account the correlation matrix presented in Table $18$ of Ref.~\cite{Anashin:2018vdo}. The experimental uncertainty of the fitted parameter $\Lambda$ is determined by requiring~\cite{Workman:2022ynf}
\begin{eqnarray}
\chi^2(\Lambda)=\chi^2_{\rm min}+1.
\end{eqnarray}
The fitting results are presented in Table \ref{tab:PMCfit}, where the $1$st and $2$nd errors in $3$rd and $4$th columns are experimental and theoretical uncertainties, respectively. The theoretical uncertainty is caused by the UHO of the pQCD approximant for $R_{\rm uds}(s)$ (\ref{eq:pmcs}) and the residual scale dependence (the UHO of the PMC scale (\ref{eq:pmcscale})), and is estimated by using the Bayesian-based approach. We also present the value of $\chi^2_{\rm min}$, which can be used for assessing the goodness-of-fit. For the computation of $\alpha_s(M_Z^2)$ based on $\Lambda=\Lambda_{\overline{\rm MS}}^{(3)}$, we use the RunDec routine to firstly computing $\Lambda_{\overline{\rm MS}}^{(4)}$ and $\Lambda_{\overline{\rm MS}}^{(5)}$ and finally extract $\alpha_s(M_Z^2)$, as suggested by Ref. \cite{Herren:2017osy}. For the $4$-loop pQCD correction $R^{4}_{\rm uds}(s)$, its $\chi^2_{\rm min}/n_{\rm d.o.f.}=10.5706/21\simeq 0.50$, which corresponds to $p>95\%$, indicating a good goodness-of-fit and the reasonableness of the fitted parameter $\Lambda$. The resultant $\alpha_s(M_Z^2)=0.1227^{+0.0117+0.0016}_{-0.0132-0.0016}$ is consistent with the world average $\alpha_s(M_Z^2)=0.1179\pm0.0009$~\cite{Workman:2022ynf}. Theoretical uncertainty is $\sim1.3\%$, and is negligible compared to the experimental one ($\sim10\%$). Thus the accurate theoretical prediction (\ref{eq:pmcs}) for $R_{\rm uds}(s)$ allows to extract $\alpha_s$ with high precision at the future Tau-Charm facility, such as the Super Tau-Charm Facility in China \cite{Huang:2017wbc, Peng:2020orp}. It is necessary to emphasize that when using the conventional prediction to fit $R_{\rm uds}$ data, the fitted $\Lambda$ should satisfy the self-consistent requirement $\sqrt{s}/\Lambda>{\rm exp}(\pi/2)\simeq4.81$.

\begin{table}[tbh]
\caption{\label{tab:PMCfit}{The fitted $\Lambda$ from $R_{\rm uds}$ data below the $D\bar D$ threshold measured by KEDR collaboration~\cite{Anashin:2018vdo}. Results for different QCD corrections $(\ell=2,3,4)$ are presented. The 1st error is experimental and the 2nd theoretical.}}
\begin{center}
\begin{tabular}{cccc}
\hline
~$R_{\rm uds}^{\rm the.}$~ & ~$\chi^2_{\rm min}/n_{\rm d.o.f.}$~ & ~$\Lambda$\,[MeV]~ & ~$\alpha_s(M_Z^2)$~  \\ \hline
$R_{\rm uds}^{(2)}|_{\rm PMCs}$ & $11.0717/21$ & $333^{+138+162}_{-143-111}$ & $0.1170^{+0.0082+0.0094}_{-0.0109-0.0081}$ \\ \hline
$R_{\rm uds}^{(3)}|_{\rm PMCs}$ & $10.6954/21$ & $370^{+177+81}_{-166-66}$ & $0.1206^{+0.0104+0.0050}_{-0.0125-0.0045}$ \\ \hline
$R_{\rm uds}^{(4)}|_{\rm PMCs}$ & $10.5706/21$ & $406^{+207+27}_{-186-25}$ & $0.1227^{+0.0117+0.0016}_{-0.0132-0.0016}$ \\
\hline
\end{tabular}
\end{center}
\end{table}

\emph{Contribution to $(g-2)_\mu$ and $\alpha(M_Z^2)$}.---Hadronic vacuum polarization (HVP) is not only a critical part of the Standard Model (SM) prediction for the anomalous magnetic moment of the muon, $a_\mu={(g-2)_\mu}/2$, but also a crucial ingredient for global fits to electroweak (EW) precision observables due to its contribution to the running of the fine-structure constant encoded in $\Delta\alpha_{\rm had}(q^2)$. Traditionally, the leading order HVP contribution to $a_\mu$ can be determined via the dispersion relation~\cite{Brodsky:1967sr,Lautrup:1968tdb}
\begin{equation}\label{amu_HVP}
a_\mu^{\rm HVP,LO}=\frac{\alpha^2}{3\pi^2}\int_{s_{\rm th}}^{\infty} \frac{K(s)}{s}R(s) {\rm d}s,
\end{equation}
where $\alpha=\alpha(0)$, $s_{\rm th}=m_{\pi}^2$, the kernel function $K(s)$ can be expressed analytically \cite{Brodsky:1967sr,Lautrup:1968tdb}.

The running QED coupling, $\alpha(q^2)$, is determined via, $\alpha(q^2)=\alpha/\big(1-\Delta\alpha_{\rm had}(q^2)-\Delta\alpha_{\rm lep}(q^2)\big)$, where the contributions to the running are separated into hadronic (had) and leptonic (lep) components, accordingly. The effective QED coupling at the $Z$ boson mass, $\alpha(M_{Z}^2)$, is the least precisely known of the three fundamental electro-weak (EW) parameters of the SM (the Fermi constant $G_F$, $M_Z$ and $\alpha(M_{Z}^2)$), and its uncertainty from hadronic contributions hinders the accuracy of EW precision fits. The hadronic contributions to the QED coupling are determined from the dispersion relation
\begin{equation}\label{eq:alpha_had}
\Delta\alpha_{\rm had}(q^2) = -\frac{\alpha q^2}{3\pi}{\rm P} \int^{\infty}_{s_{th}} \frac{R(s)}{s(s-q^2)} {\rm d}s\,,
\end{equation}
where P indicates the principal value of the integral.

\begin{figure}[htb]
\includegraphics[width=0.45\textwidth]{g-2.pdf}
\caption{$a_\mu^{\rm HVP,LO}[1.8\leq\sqrt{s}\leq3.7{\rm GeV}]$ as a function of $\Lambda$. The band shows the total uncertainty, including effects from the residual scale dependence and the UHO ontribution.}
\label{fig:g2}
\end{figure}

Using the PMC prediction (\ref{eq:pmcs}), we evaluate the contribution of $R_{\rm uds}(s)$ to $a_\mu^{\rm HVP,LO}$ in energy region $1.8-3.7$ GeV, and present it as a function of $\Lambda$ in Fig.\ref{fig:g2}, where the band represents the theoretical uncertainty, including contributions from the residual scale dependence and the UHO of the pQCD approximant for $R_{\rm uds}(s)$ (\ref{eq:pmcs}). Such theoretical uncertainty is $\sim 0.02\%$ at $\Lambda=0.1\,{\rm GeV}$ and increases to $\sim 0.5\%$ at $\Lambda=0.7\,{\rm GeV}$. As for numerical results, if taking the same input $\alpha_s(M_Z^2)=0.1182\pm0.0012$ as the KNT18 \cite{Keshavarzi:2018mgv}, we obtain $a_\mu^{\rm HVP,LO}[1.841\leq\sqrt{s}\leq2.00{\rm GeV}]\times 10^{10}=6.38\pm 0.02$, where the total uncertainty includes effects from the $\alpha_s$ uncertainty, the residual scale dependence, and the UHO contribution. This result is in good agreement with the one reported by KNT18 \cite{Keshavarzi:2018mgv}, $a_\mu^{\rm HVP,LO}[1.841\leq\sqrt{s}\leq2.00{\rm GeV}]\times 10^{10}=6.38\pm 0.11$, but with a decreased error, whose error is dominated by the variation of the renormalization scale $\mu_r$ in the range $\sqrt{s}/2<\mu_r<2\sqrt{s}$. When taking the same input as the DHMZ19 \cite{Davier:2019can}, i.e. $\alpha_s(M_Z^2)=0.1193\pm0.0028$ from the fit to $Z$ precision data \cite{Baak:2014ora}, we obtain,
\begin{equation}\label{eq:g-2}
\hspace{-1mm}a_\mu^{\rm HVP,LO}[1.8\leq\sqrt{s}\leq3.7{\rm GeV}]\times 10^{10}=33.49\pm0.15,
\end{equation}
where the error is the squared average of those from the $\alpha_s$ uncertainty $(\pm0.14)$, the residual scale dependence $(\pm0.04)$, and the UHO contribution $(\pm0.01)$. As for the running electromagnetic coupling at $M_Z^2$, our prediction for the hadronic contribution from $1.8-3.7$ GeV range to the running of $\alpha(M_Z^2)$,
\begin{equation}\label{eq:alpha}
\Delta\alpha_{\rm had}(M_Z^2)[1.8\leq\sqrt{s}\leq3.7{\rm GeV}]\times10^{4}=24.28\pm0.10,
\end{equation}
whose uncertainty is dominated by the $\alpha_s$ uncertainty $(\pm0.10)$, and both the residual scale dependence $(\pm0.02)$ and the UHO contribution $(\pm 0.00)$ are quite small. Our present predictions (\ref{eq:g-2}) and (\ref{eq:alpha}) are in good agreement with the results reported by DHMZ19 \cite{Davier:2019can} but with decreased errors.

\emph{Summary.}---The hadronic $e^+e^-$ annihilation rate $R(s)$ is one of the most precise and theoretically safe observables involving strong interactions. The PMC provides a rigorous method for solving the conventional renormalization scheme-and-scale ambiguities, and its PMC scale reflects the virtuality of the underlying QCD subprocess. By applying the PMCs, we have shown that a reliable and self-consistent analysis for $R_{\rm uds}(s)$ can be achieved. Our new calculation for $R_{\rm uds}(s)$ leads to a scale-invariant prediction, a significant stabilization of the perturbative series, and a reduction of theoretical uncertainty. It thus can provide a reliable and competitive determination for the QCD running coupling at future high-precision measurement on $R_{\rm uds}(s)$, and will help to improve the accuracy of the SM predictions for the muon magnetic anomaly $a_\mu$ as well as the QED coupling $\alpha(M_Z^2)$.

\begin{acknowledgments}
\emph{Acknowledgments.}---We are grateful to Prof. Guang-Shun Huang, Hai-Ming Hu and Shu-Lei Zhang for fruitful discussions. This work was supported in part by the Natural Science Foundation of China under Grant No.11905056, No.12147102, No.12175025 and No.12265011.
\end{acknowledgments}

\begin{thebibliography}{99}

\bibitem{Gross:1973id}
D.~J.~Gross and F.~Wilczek,
%``Ultraviolet Behavior of Nonabelian Gauge Theories,''
Phys.\ Rev.\ Lett.\  {\bf 30}, 1343 (1973).

%\cite{Politzer:1973fx}
\bibitem{Politzer:1973fx}
H.~D.~Politzer,
%``Reliable Perturbative Results for Strong Interactions?,''
Phys.\ Rev.\ Lett.\  {\bf 30}, 1346 (1973).

%\cite{Workman:2022ynf}
\bibitem{Workman:2022ynf}
R.~L.~Workman \textit{et al.} [Particle Data Group],
%``Review of Particle Physics,''
PTEP \textbf{2022}, 083C01 (2022).

%\cite{Chetyrkin:1994js}
\bibitem{Chetyrkin:1994js}
K.~G.~Chetyrkin, J.~H.~Kuhn and A.~Kwiatkowski,
%``QCD corrections to the $e^{+} e^{-}$ cross-section and the $Z$ boson decay rate,''
Phys.\ Rept.\ {\bf 277}, 189 (1996).

%\cite{Aoyama:2020ynm}
\bibitem{Aoyama:2020ynm}
T.~Aoyama, \textit{et al.},
%``The anomalous magnetic moment of the muon in the Standard Model,''
Phys. Rept. \textbf{887}, 1 (2020).

%\cite{BESIII:2021wib}
\bibitem{BESIII:2021wib}
M.~Ablikim \textit{et al.} [BESIII Collaboration],
%``Measurement of the Cross Section for $e^{+}e^{-}\to$Hadrons at Energies from 2.2324 to 3.6710~GeV,''
Phys. Rev. Lett. \textbf{128}, 062004 (2022).

% %\cite{Anashin:2015woa}
% \bibitem{Anashin:2015woa}
% V.~V.~Anashin {\it et al.} [KEDR Collaboration],
% %``Measurement of $R_{\text{uds}}$ and $R$ between 3.12 and 3.72 GeV at the KEDR detector,''
% Phys.\ Lett.\ B {\bf 753}, 533 (2016).

% %\cite{Anashin:2016hmv}
% \bibitem{Anashin:2016hmv}
% V.~V.~Anashin {\it et al.} [KEDR Collaboration],
% %``Measurement of $R$ between 1.84 and 3.05 GeV at the KEDR detector,''
% Phys.\ Lett.\ B {\bf 770}, 174 (2017).

%\cite{Anashin:2018vdo}
\bibitem{Anashin:2018vdo}
V.~V.~Anashin {\it et al.} [KEDR Collaboration],
%``Precise measurement of $R_{\text{uds}}$ and $R$ between 1.84 and 3.72 GeV at the KEDR detector,''
Phys.\ Lett.\ B {\bf 788}, 42 (2019).

%\cite{BESIII:2009fln}
\bibitem{BESIII:2009fln}
M.~Ablikim \textit{et al.} [BESIII Collaboration],
%``Design and Construction of the BESIII Detector,''
Nucl. Instrum. Meth. A \textbf{614}, 345 (2010).

%\cite{Anashin:2013twa}
\bibitem{Anashin:2013twa}
V.~V.~Anashin \textit{et al.},
%``The KEDR detector,''
Phys. Part. Nucl. \textbf{44}, 657 (2013).

%\cite{Chetyrkin:1996ela}
\bibitem{Chetyrkin:1996ela}
K.~G.~Chetyrkin, J.~H.~Kuhn and A.~Kwiatkowski,
%``QCD corrections to the $e^{+} e^{-}$ cross-section and the $Z$ boson decay rate,''
Phys. Rept. \textbf{277}, 189 (1996).

%\cite{Davier:2005xq}
\bibitem{Davier:2005xq}
M.~Davier, A.~Hocker and Z.~Zhang,
%``The Physics of Hadronic Tau Decays,''
Rev. Mod. Phys. \textbf{78}, 1043 (2006).

%\cite{Chetyrkin:1979bj}
\bibitem{Chetyrkin:1979bj}
K.~G.~Chetyrkin, A.~L.~Kataev and F.~V.~Tkachov,
%``Higher Order Corrections to Sigma-t (e+ e- ---\ensuremath{>} Hadrons) in Quantum Chromodynamics,''
Phys. Lett. B \textbf{85}, 277 (1979).

%\cite{Surguladze:1990tg}
\bibitem{Surguladze:1990tg}
L.~R.~Surguladze and M.~A.~Samuel,
%``Total hadronic cross-section in e+ e- annihilation at the four loop level of perturbative QCD,''
Phys. Rev. Lett. \textbf{66}, 560 (1991)
[erratum: Phys. Rev. Lett. \textbf{66}, 2416 (1991)].

%\cite{Gorishnii:1990vf}
\bibitem{Gorishnii:1990vf}
S.~G.~Gorishnii, A.~L.~Kataev and S.~A.~Larin,
%``The $O(\alpha^{3}_{s})$-corrections to $\sigma_{tot}(e^{+}e^{-}\rightarrow hadrons)$ and $\Gamma(\tau^{-} \rightarrow \nu_{\tau} + hadrons)$ in QCD,''
Phys. Lett. B \textbf{259}, 144 (1991).

%\cite{Baikov:2008jh} {Ree4loop}
\bibitem{Baikov:2008jh}
P.~A.~Baikov, K.~G.~Chetyrkin and J.~H.~Kuhn,
%``Order $\alpha^4(s)$ QCD Corrections to $Z$ and tau Decays,''
Phys.\ Rev.\ Lett.\  {\bf 101}, 012002 (2008).

%\cite{Baikov:2010je}
\bibitem{Baikov:2010je}
P.~A.~Baikov, K.~G.~Chetyrkin and J.~H.~Kuhn,
%``Adler Function, Bjorken Sum Rule, and the Crewther Relation to Order $\alpha_s^4$ in a General Gauge Theory,''
Phys.\ Rev.\ Lett.\  {\bf 104}, 132004 (2010).

%\cite{Baikov:2012er}
\bibitem{Baikov:2012er}
P.~A.~Baikov, K.~G.~Chetyrkin, J.~H.~Kuhn and J.~Rittinger,
%``Complete ${\cal O}(\alpha_s^4)$ QCD Corrections to Hadronic $Z$-Decays,''
Phys. Rev. Lett. \textbf{108}, 222003 (2012).

%\cite{Baikov:2012zn}
\bibitem{Baikov:2012zn}
P.~A.~Baikov, K.~G.~Chetyrkin, J.~H.~Kuhn and J.~Rittinger,
%``Adler Function, Sum Rules and Crewther Relation of Order ${\cal O}(\alpha_s^4)$: the Singlet Case,''
Phys.\ Lett.\ B {\bf 714}, 62 (2012).

%\cite{Baikov:2012zm}
\bibitem{Baikov:2012zm}
P.~A.~Baikov, K.~G.~Chetyrkin, J.~H.~Kuhn and J.~Rittinger,
%``Vector Correlator in Massless QCD at Order ${\cal O}(\alpha_s^4)$ and the QED beta-function at Five Loop,"
JHEP {\bf 1207}, 017 (2012).

% %\cite{Chetyrkin:1995ii}
% \bibitem{Chetyrkin:1995ii}
% K.~G.~Chetyrkin, J.~H.~Kuhn and M.~Steinhauser,
% %``Heavy quark vacuum polarization to three loops,''
% Phys. Lett. B \textbf{371}, 93 (1996).

% %\cite{Chetyrkin:1996cf}
% \bibitem{Chetyrkin:1996cf}
% K.~G.~Chetyrkin, J.~H.~Kuhn and M.~Steinhauser,
% %``Three loop polarization function and O (alpha-s**2) corrections to the production of heavy quarks,''
% Nucl. Phys. B \textbf{482}, 213 (1996).

% %\cite{Chetyrkin:1990kr}
% \bibitem{Chetyrkin:1990kr}
% K.~G.~Chetyrkin and J.~H.~Kuhn,
% %``Mass corrections to the Z decay rate,''
% Phys. Lett. B \textbf{248}, 359 (1990).

% %\cite{Chetyrkin:1994ex}
% \bibitem{Chetyrkin:1994ex}
% K.~G.~Chetyrkin and J.~H.~Kuhn,
% %``Quartic mass corrections to R(had),''
% Nucl. Phys. B \textbf{432}, 337 (1994).

% %\cite{Baikov:2004ku}
% \bibitem{Baikov:2004ku}
% P.~A.~Baikov, K.~G.~Chetyrkin and J.~H.~Kuhn,
% %``Vacuum polarization in pQCD: First complete O(alpha(s)**4) result,''
% Nucl. Phys. B Proc. Suppl. \textbf{135}, 243 (2004).

%\cite{Chetyrkin:2000zk}
\bibitem{Chetyrkin:2000zk}
K.~G.~Chetyrkin, R.~V.~Harlander and J.~H.~Kuhn,
%``Quartic mass corrections to $R_{had}$ at $\mathcal O(\alpha^3_s)$,''
Nucl. Phys. B \textbf{586}, 56 (2000)
[erratum: Nucl. Phys. B \textbf{634}, 413 (2002)].

%\cite{Harlander:2002ur}
\bibitem{Harlander:2002ur}
R.~V.~Harlander and M.~Steinhauser,
%``rhad: A Program for the evaluation of the hadronic $R$-ratio in the perturbative regime of QCD,''
Comput.\ Phys.\ Commun.\  {\bf 153}, 244 (2003).

%\cite{Kiyo:2009gb}
\bibitem{Kiyo:2009gb}
Y.~Kiyo, A.~Maier, P.~Maierhofer and P.~Marquard,
%``Reconstruction of heavy quark current correlators at O(alpha(s)**3),''
Nucl. Phys. B \textbf{823}, 269 (2009).

%\cite{Czarnecki:1996ei}
\bibitem{Czarnecki:1996ei}
A.~Czarnecki and J.~H.~Kuhn,
%``Nonfactorizable QCD and electroweak corrections to the hadronic Z boson decay rate,''
Phys. Rev. Lett. \textbf{77}, 3955 (1996).

%\cite{Harlander:1998cmq}
\bibitem{Harlander:1998cmq}
R.~Harlander, T.~Seidensticker and M.~Steinhauser,
%``Complete corrections of Order alpha alpha-s to the decay of the Z boson into bottom quarks,''
Phys. Lett. B \textbf{426}, 125 (1998).

%\cite{Kuhn:1998ze}
\bibitem{Kuhn:1998ze}
J.~H.~Kuhn and M.~Steinhauser,
%``A Theory driven analysis of the effective QED coupling at $M_Z$,''
Phys.\ Lett.\ B {\bf 437}, 425 (1998).

%\cite{Martin:1999bp}
\bibitem{Martin:1999bp}
A.~D.~Martin, J.~Outhwaite and M.~G.~Ryskin,
%``The $R$ ratio in $e^+ e^-$, the determination of $\alpha(M^2_Z)$ and a possible nonperturbative gluonic contribution,''
J.\ Phys.\ G {\bf 26}, 600 (2000).

\bibitem{Brodsky:2011ta}
S.~J.~Brodsky and X.~G.~Wu,
%``Scale Setting Using the Extended Renormalization Group and the Principle of Maximum Conformality: the QCD Coupling Constant at Four Loops,''
Phys.\ Rev.\ D {\bf 85}, 034038 (2012).

\bibitem{Brodsky:2011ig}
S.~J.~Brodsky and L.~Di Giustino,
%``Setting the Renormalization Scale in QCD: The Principle of Maximum Conformality,''
Phys. Rev. D {\bf 86}, 085026 (2012).

\bibitem{Brodsky:2012rj}
S.~J.~Brodsky and X.~G.~Wu,
%``Eliminating the Renormalization Scale Ambiguity for Top-Pair Production Using the Principle of Maximum Conformality,
Phys.\ Rev.\ Lett.\ {\bf 109}, 042002 (2012).

\bibitem{Mojaza:2012mf}
M.~Mojaza, S.~J.~Brodsky, and X.~G.~Wu,
%``Systematic All-Orders Method to Eliminate Renormalization-Scale and Scheme Ambiguities in Perturbative QCD,''
Phys.\ Rev.\ Lett.\ {\bf 110}, 192001 (2013).

\bibitem{Brodsky:2013vpa}
S.~J.~Brodsky, M.~Mojaza, and X.~G.~Wu,
%``Systematic Scale-Setting to All Orders: The Principle of Maximum Conformality and Commensurate Scale Relations,''
Phys.\ Rev.\ D {\bf 89}, 014027 (2014).

%\cite{Shen:2017pdu}
\bibitem{Shen:2017pdu}
J.~M.~Shen, X.~G.~Wu, B.~L.~Du and S.~J.~Brodsky,
%``Novel All-Orders Single-Scale Approach to QCD Renormalization Scale-Setting,''
Phys.\ Rev.\ D {\bf 95}, 094006 (2017).

%\cite{Yan:2022foz}
\bibitem{Yan:2022foz}
J.~Yan, Z.~F.~Wu, J.~M.~Shen and X.~G.~Wu,
%``Precise perturbative predictions from fixed-order calculations,''
J. Phys. G \textbf{50}, 045001 (2023).

%\cite{Wu:2018cmb}
\bibitem{Wu:2018cmb}
X.~G.~Wu, J.~M.~Shen, B.~L.~Du and S.~J.~Brodsky,
%``Novel demonstration of the renormalization group invariance of the fixed-order predictions using the principle of maximum conformality and the $C$-scheme coupling,''
Phys.\ Rev.\ D {\bf 97}, 094030 (2018).

%\cite{Wu:2019mky}
\bibitem{Wu:2019mky}
X.~G.~Wu, J.~M.~Shen, B.~L.~Du, X.~D.~Huang, S.~Q.~Wang and S.~J.~Brodsky,
%``The QCD renormalization group equation and the elimination of fixed-order scheme-and-scale ambiguities using the principle of maximum conformality,''
Prog. Part. Nucl. Phys. \textbf{108}, 103706 (2019).

%\cite{Stueckelberg:1953dz}
\bibitem{Stueckelberg:1953dz}
E.~C.~G.~Stueckelberg and A.~Petermann,
%``Normalization of constants in the quanta theory,''
Helv.\ Phys.\ Acta {\bf 26}, 499 (1953).
%CITATION = HPACA,26,499;%%

%\cite{Peterman:1978tb}
\bibitem{Peterman:1978tb}
A.~Peterman,
%``Renormalization Group and the Deep Structure of the Proton,''
Phys.\ Rept.\ {\bf 53}, 157 (1979).
%%CITATION = PRPLC,53,157;%%

\bibitem{rge4}
N.N. Bogoliubov and D.V. Shirkov,
%``Application of the renormalization group to improve the formulae of perturbation theory,"
Dok. Akad. Nauk SSSR {\bf 103}, 391 (1955).

\bibitem{Wu:2014iba}
X.~G.~Wu, Y.~Ma, S.~Q.~Wang, H.~B.~Fu, H.~H.~Ma, S.~J.~Brodsky and M.~Mojaza,
%``Renormalization Group Invariance and Optimal QCD Renormalization Scale-Setting,''
Rept. Prog. Phys. \textbf{78}, 126201 (2015).

%\cite{Brodsky:2012ms}
\bibitem{Brodsky:2012ms}
S.~J.~Brodsky and X.~G.~Wu,
%``Self-Consistency Requirements of the Renormalization Group for Setting the Renormalization Scale,''
Phys.\ Rev.\ D {\bf 86}, 054018 (2012).

%\cite{Wu:2013ei}
\bibitem{Wu:2013ei}
X.~G.~Wu, S.~J.~Brodsky and M.~Mojaza,
%``The Renormalization Scale-Setting Problem in QCD,''
Prog. Part. Nucl. Phys. \textbf{72}, 44 (2013).

%\cite{GellMann:1954fq}
\bibitem{GellMann:1954fq}
M.~Gell-Mann and F.~E.~Low,
%Quantum electrodynamics at small distances,
Phys.\ Rev.\ {\bf 95}, 1300 (1954).

\bibitem{Brodsky:1982gc}
S.~J.~Brodsky, G.~P.~Lepage and P.~B.~Mackenzie,
%``On the Elimination of Scale Ambiguities in Perturbative Quantum Chromodynamics,''
Phys.\ Rev.\ D {\bf 28}, 228 (1983).

%\cite{Adler:1974gd}
\bibitem{Adler:1974gd}
S.~L.~Adler,
%``Some Simple Vacuum Polarization Phenomenology: $e^+ e^- \to$ Hadrons: The $\mu$ - Mesic Atom x-Ray Discrepancy and $g_{\mu}^{-2}$,''
Phys. Rev. D \textbf{10}, 3714 (1974).

%\cite{Bi:2015wea}
\bibitem{Bi:2015wea}
H.~Y.~Bi, X.~G.~Wu, Y.~Ma, H.~H.~Ma, S.~J.~Brodsky and M.~Mojaza,
%``Degeneracy Relations in QCD and the Equivalence of Two Systematic All-Orders Methods for Setting the Renormalization Scale,''
Phys.\ Lett.\ B {\bf 748}, 13 (2015).

%\cite{Kataev:1995vh}
\bibitem{Kataev:1995vh}
A.~L.~Kataev and V.~V.~Starshenko,
%``Estimates of the higher order QCD corrections to R(s), R(tau) and deep inelastic scattering sum rules,''
Mod. Phys. Lett. A \textbf{10}, 235 (1995).

%\cite{Shirkov:2000qv}
\bibitem{Shirkov:2000qv}
D.~V.~Shirkov,
%``Analytic perturbation theory for QCD observables,''
Theor. Math. Phys. \textbf{127}, 409 (2001).

%\cite{Prosperi:2006hx}
\bibitem{Prosperi:2006hx}
G.~M.~Prosperi, M.~Raciti and C.~Simolo,
%``On the running coupling constant in QCD,''
Prog. Part. Nucl. Phys. \textbf{58}, 387 (2007).

%\cite{Nesterenko:2017wpb}
\bibitem{Nesterenko:2017wpb}
A.~V.~Nesterenko,
%``Electron\textendash{}positron annihilation into hadrons at the higher-loop levels,''
Eur. Phys. J. C \textbf{77}, 844 (2017).

% %\cite{Nesterenko:2019rag}
% \bibitem{Nesterenko:2019rag}
% A.~V.~Nesterenko,
% %``Explicit form of the R-ratio of electron\textendash{}positron annihilation into hadrons,''
% J. Phys. G \textbf{46}, 115006 (2019).

% %\cite{Nesterenko:2020nol}
% \bibitem{Nesterenko:2020nol}
% A.~V.~Nesterenko,
% %``Recurrent form of the renormalization group relations for the higher-order hadronic vacuum polarization function perturbative expansion coefficients,''
% J. Phys. G \textbf{47}, 105001 (2020).

%\cite{Cacciari:2011ze}
\bibitem{Cacciari:2011ze}
M.~Cacciari and N.~Houdeau,
%``Meaningful characterisation of perturbative theoretical uncertainties,''
JHEP \textbf{09}, 039 (2011).

%\cite{Shen:2022nyr}
\bibitem{Shen:2022nyr}
J.~M.~Shen, Z.~J.~Zhou, S.~Q.~Wang, J.~Yan, Z.~F.~Wu, X.~G.~Wu and S.~J.~Brodsky,
%``Extending the Predictive Power of Perturbative QCD Using the Principle of Maximum Conformality and Bayesian Analysis,''
[arXiv:2209.03546 [hep-ph]].

\bibitem{Zheng:2013uja}
X.~C.~Zheng, X.~G.~Wu, S.~Q.~Wang, J.~M.~Shen and Q.~L.~Zhang,
%``Reanalysis of the BFKL Pomeron at the next-to-leading logarithmic accuracy,''
JHEP \textbf{10}, 117 (2013).

%\cite{Herren:2017osy}
\bibitem{Herren:2017osy}
F.~Herren and M.~Steinhauser,
%``Version 3 of RunDec and CRunDec,''
Comput.\ Phys.\ Commun.\ {\bf 224}, 333 (2018).

%\cite{Huang:2017wbc}
\bibitem{Huang:2017wbc}
G.~S.~Huang, C.~Li, H.~B.~Li, \textit{et al.},
%``Physics on the high intensive electron position accelerator at 2~7 GeV (in Chinese),''
Chin. Sci. Bull. \textbf{62}, 1226 (2017).

%\cite{Peng:2020orp}
\bibitem{Peng:2020orp}
H.~P.~Peng, Y.~H.~Zheng and X.~R.~Zhou,
%``Super Tau-Charm Facility of China,''
Physics \textbf{49}, 513 (2020).

%\cite{Brodsky:1967sr}
\bibitem{Brodsky:1967sr}
S.~J.~Brodsky and E.~De Rafael,
%``SUGGESTED BOSON - LEPTON PAIR COUPLINGS AND THE ANOMALOUS MAGNETIC MOMENT OF THE MUON,''
Phys. Rev. \textbf{168}, 1620 (1968).

%\cite{Lautrup:1968tdb}
\bibitem{Lautrup:1968tdb}
B.~E.~Lautrup and E.~De Rafael,
%``Calculation of the sixth-order contribution from the fourth-order vacuum polarization to the difference of the anomalous magnetic moments of muon and electron,''
Phys. Rev. \textbf{174}, 1835 (1968).

%\cite{Keshavarzi:2018mgv}
\bibitem{Keshavarzi:2018mgv}
A.~Keshavarzi, D.~Nomura and T.~Teubner,
%``Muon $g-2$ and $\alpha(M_Z^2)$: a new data-based analysis,''
Phys. Rev. D \textbf{97}, 114025 (2018).

% %\cite{Davier:2017zfy}
% \bibitem{Davier:2017zfy}
% M.~Davier, A.~Hoecker, B.~Malaescu and Z.~Zhang,
% %``Reevaluation of the hadronic vacuum polarisation contributions to the Standard Model predictions of the muon $g-2$ and ${\alpha (m_Z^2)}$ using newest hadronic cross-section data,''
% Eur. Phys. J. C \textbf{77}, 827 (2017).

%\cite{Davier:2019can}
\bibitem{Davier:2019can}
M.~Davier, A.~Hoecker, B.~Malaescu and Z.~Zhang,
%``A new evaluation of the hadronic vacuum polarisation contributions to the muon anomalous magnetic moment and to $\mathbf{\boldsymbol\alpha(m_Z^2)}$,''
Eur. Phys. J. C \textbf{80}, 241 (2020).

%\cite{Baak:2014ora}
\bibitem{Baak:2014ora}
M.~Baak \textit{et al.} [Gfitter Group],
%``The global electroweak fit at NNLO and prospects for the LHC and ILC,''
Eur. Phys. J. C \textbf{74}, 3046 (2014).

\end{thebibliography}

\end{document}
