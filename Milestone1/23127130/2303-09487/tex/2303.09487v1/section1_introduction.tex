\section{Introduction}
\label{section_introduction}
\setcounter{equation}{0}

%% The paper in a nutshell
Given a $n$--dimensional lattice $\Weights \subseteq \R^n$,
a trigonometric polynomial is a function
\[
	f: \R^n \to \R ,\, u\mapsto f(u) := \sum\limits_{\weight\in \Weights} c_\weight \, \mexp{\weight,u},
\]
where 
$\sprod{\cdot,\cdot}$ denotes the Euclidean scalar product 
and the finitely many nonzero coefficients $c_\weight\in \C$ satisfy $ c_{-\weight} =\overline{c_{\weight}}$. 
Such functions are good $L^2$--aproximations for 
$\Lambda$--periodic functions, where $\Lambda$ is the dual lattice. 
This paper offers a new approach to optimizing such a 
trigometric function, over $\R^n$, when this latter is invariant under a crystalographic reflection group. 
We show how the problem can then be reduced to polynomial optimization on a semi--algebraic set 
and handled with a variation on 
Lasserre hierarchy. The resulting algorithm is applied to the exploration of the spectral bound on the chromatic numbers of set avoiding graphs.

%% What's on the market
In the literature of trigonometric optimization, one often regards the lattice simply as a free $\Z$--module, that is, $\Weights=\Z^n$, ignoring the geometry and only taking central symmetry into account. 
For the purpose of optimization, a hierarchy of Hermitian sums of squares reinforcements provides a numerical solution \cite{dumitrescu07,bach22}.
Alternatively, one can apply Lasserre's hierarchy with complex variables \cite{josz18}, where one has to restrict to the compact torus.

%% What we consider and why its  important
In this article, $\Weights$ is the weight lattice of a crystallographic root system in $\R^n$. 
Root  and weight lattices  provide optimal configurations for a variety of problems in  geometry and information theory, with incidence in  physics and chemistry. 
The $\RootA[2]$ root lattice (the hexagonal lattice) is classically known to be optimal for sampling, packing, covering, and quantization in the plane \cite{conway1988a,kunsch05}, 
but also proved, or conjectured, to be optimal for energy minimization problems \cite{Petrache20,faulhuber23}.  
More recently, the $\RootE[8]$ lattice was  proven to give an optimal solution for the sphere packing problem and a large class of energy minimization problems in dimension $8$
\cite{Petrache20,Viazovska17,Viazovska22}.
From an approximation point of view, weight lattices of root systems describe Gaussian cubature \cite{Xu09,Moody2011}, 
a rare occurence on multidimensional domains.
In a different direction, the triangulations associated with 
infinite families of root systems are relevant in graphics and computational geometry, see for instance \cite{Choudhary20} and references within.

%% Details on what we do
The distinguishing feature of the lattices associated to crystallographic root system is their intrisic symmetry.
This latter is given by the so called Weyl group $\weyl$, a finite group generated by orthogonal reflections w.r.t. $\sprod{\cdot,\cdot}$. 
It is this feature that we emphasize and offer to exploit in an optimization context.  
We present a new approach to numerically solve the 
trigonometric optimization problem 
\begin{equation}\label{OptiProblemExpo}
f^* := \min\limits_{u\in \R^n} f(u)
\end{equation}
under the assumption of crystallographic symmetry, that is, for $A\in\weyl$, we have $f(A\,u) = f(u)$, or equivalently $c_{A\,\weight} = c_\weight$.
The first step of our approach, 
in \Cref{section_trigonometric}, 
is a symmetry reduction that translates the trigonometric optimization above to the problem of optimizing a polynomial 
over a semi--algebraic set, a subject that 
ripened in the last two decades \cite{lasserre01,parrilo03,sturmfels03,parrilo05,klerk05,lasserre09,laurent09,blekherman12,lasserre21}. 
The second step of our approach, 
in \Cref{section_optimization}, is thus an adaptation of 
Lasserre's hierarchy of moment relaxations and sums of squares reinforcements.
We indeed modify the hierarchy introduced in \cite{holscherer05,holscherer06,lasserre06} to work directly in the basis of generalized Chebyshev polynomials. 
These  are not homogeneous but naturally 
filtered by a weighted degree, different from the usual degree. 

The simplest case of this symmetry reduction scheme, the univariate case, is obvious but maybe worth reviewing 
to get the initial idea.
The group is then $\weyl=\{1,-1\}$ 
and the invariance condition is thus
$f(-u)=f(u)$ for all $u\in\R$.
That implies that one can write 
$$f(u)= \sum_{k\in \N} \frac{c_k}{2}	
\left(\exp(2 \pi \mathrm{i}\, k u) + \exp(-2 \pi \mathrm{i} \,ku)\right) 
= \sum_{k\in \N} c_k \,\cos(2\pi\, k u)
= \sum_{k\in \N} c_k \, T_k(\cos(2\pi\, u)),$$ 
where $\{T_k\}_{k\in\N}$ 
are the Chebyshev polynomials of the first kind. 
We  thus have
\[
f^* := \min\limits_{u\in \R^n} f(u) = 
\min\limits_{z^2\leq 1} \sum_{k\in \N} c_k  T_k(z)
\]
the right hand side being a polynomial optimization problem with semi--algebraic constraints.

With $\Omega=\Z^n$ and $\weyl=\{1,-1\}^n$, 
one can use products of univariate Chebyshev polynomials to operate a similar symmetry reduction. 
This is the $\RootA[1]\times \ldots \times\RootA[1]$ case. 
We look at all the lattices associated to crystallographic root systems, 
offering a wider range of  domains of periodicity 
(hexagon, rhombic dodecahedron, icositetrachoron, \ldots) 
and simplices of any dimension, 
or cartesian products of these, as fundamental domains.
The key to the symmetry reduction then is the existence and properties of generalized Chebyshev polynomials. 
They allow to rewrite any invariant trigonometric 
polynomials as polynomials of the fundamental generalized cosines.
These generalized Chebyshev polynomials arose in different contexts, in particular in the search of multivariate 
orthogonal polynomials \cite{DunnLidl1980,EierLidl1982,HoffmanWithers,MacDonald1990,beerends91}.
A more recent development is the description of their domain of orthogonality, the image of the generalized cosines,
as a compact semi--algebraic set given by a
unified and explicit polynomial matrix inequality \cite{chromaticissac22,TOrbits,TobiasThesis}.
Such a description is necessary to proceed algorithmically with the obtained polynomial optimization problem.

In the algorithmic approach, we solve a primal/dual semi--definite program (SDP) that models a moment--relaxation/sums of squares reinforcement in terms of generalized Chebyshev polynomials. 
Our \textsc{Maple} package 
\textsc{GeneralizedChebyshev}\footnote{\href{https://github.com/TobiasMetzlaff/GeneralizedChebyshev}{https://github.com/TobiasMetzlaff/GeneralizedChebyshev}} 
allows to compute the parameters of the SDP, specifically the matrices which impose the semi--definite constraints. 
The user can then solve the problem with a SDP solver of their personal preference. 
Beyond that, the package offers a large variety of tools, including the matrices from \cite{TOrbits}, 
a function to rewrite invariants in terms of generalized Chebyshev polynomials and an implemented recurrence formula for their computation. 
We can thus compare our method with the one in \cite{dumitrescu07} in practice. 
We observe in several examples throughout \Cref{sec_casestudy} that the quality of the approximation is improved, while the computational complexity is reduced. 

As a second set of contributions, in \Cref{section_chromatic},
we apply  our method 
to the computation of spectral bounds 
for chromatic numbers of set avoiding graphs. 
The first such graph considered was the Euclidean distance graph 
\cite{Soifer09,BdCOV,BPT15,deGrey18}, where the vertices are the points of $\R^n$ and the set to be avoided is the sphere. 
As set of vertices we consider either $\R^n$, 
or a lattice thereof.
As for the set to be avoided we mostly consider the boundary of a polytope with crystallographic symmetry.
Choosing  appropriate discrete measures on the polytope, 
the  spectral bound from \cite{BdCOV} 
made specific to the chromatic number can be expressed 
as the solution of  a max--min  optimization problem on a trigonometric polynomial. 
Our symmetry reduction technique 
of \Cref{section_trigonometric} then allows us to retrieve,
with simple proofs, the chromatic number 
of the $\RootA$ lattice (\Cref{thm_LatticeChromatic}), 
of the graph avoiding the crosspolytope of radius $2$ in 
$\Z^n$ (\Cref{thm_ZnL1r2bound}), 
and of the graph avoiding the cube in $\R^n$  
(\Cref{thm_cubeRnbound}). 
In other cases, 
we apply the algorithm in \Cref{section_optimization} to
compute lower bounds numerically. We improve on \cite{furedi04}  
by $+2$ for the chromatic number of $\Z^4$ avoiding the crosspolytope of radius $4$ (\Cref{B4C4D4L1Table2}).
We also give further bounds 
for the rhombic dodecahedron (\Cref{A3RhombicTable}) 
as well as the icositetrachroron 
(\Cref{B4D4IcositetrachoronTable2}). 