\section{Spectral bounds for set avoiding graphs}
\label{section_chromatic}

\setcounter{equation}{0}

In this last section, we apply our method for trigonometric optimization problems with crystallographic symmetry to the computation of spectral bounds for chromatic numbers. The chromatic number of a graph gives the minimal number of colors needed to paint the vertices, so that no edge connects two vertices of the same color. When dealing with set avoiding graphs, \cite{BdCOV} provides a lower bound, which involves minimizing the Fourier transformation of a measure.

While this bound has been used and strengthened for the graph $\R^n$ avoiding Euclidean distance $1$ \cite{Soifer09,deGrey18,BdCOV,BPT15}, it has not been widely used as a tool for polytopes. Crystallographic symmetry in the trigonometric optimization problem arises, when the polytope has Weyl group symmetry. Then we can rewrite the spectral bound in terms of generalized Chebyshev polynomials and use the results of \Cref{section_trigonometric,section_optimization}.

An advantage of our approach is that rewriting the optimization problem in terms of polynomials allows in several cases to compute bounds with simple proofs and to recover many results. In other cases, we compute numerical bounds with the modified Lasserre hierarchy from \Cref{section_optimization}. Our approach allows to study the quality of the spectral bound and to estimate the optimal involved measure, see \Cref{B3L1NormCoefficients}.

\subsection{Computing spectral bounds with Chebyshev polynomials}

Let $V\leq\R^n$ be an Abelian group and $S\subseteq V$ be bounded, centrally--symmetric with $0\notin\overline{S}$. We consider the \textbf{set avoiding graph} $G(V,S)$, where $V$ is the set of vertices and two vertices $u,v\in V$ are connected by an edge if and only if $u - v\in S$. In this context, we call $S$ the \textbf{avoided set}.%\footnote{Sometimes $G(V,S)$ is called a Cayley graph with connection set $S$, see for example \cite{vallentin14}. Since we work specifically with Abelian groups and have a geometric interpretation, we choose the name set avoiding graph.}.

A set of vertices $I\subseteq V$ is called \textbf{independent} for $G(V,S)$, if no pair of vertices in $I$ are connected by an edge, that is, for all $u,v\in I$, we have $u-v\notin S$. A \textbf{measurable coloring} $X$ of $G(V,S)$ is a partition of $V$ in independent Lebesgue--measurable sets. The \textbf{measurable chromatic number} of $G(V,S)$ is
\[
	\chi_m(V,S)
:=	\inf\{ \nops{X} \,\vert\, X \mbox{ is a measurable coloring of } V \}.
\]

\subsubsection{The spectral bound}

In \cite{BdCOV}, Bachoc, Decorte, de Oliveira Filho and Vallentin generalized the Hoffman \cite{Hoffman1970} and Lovasz \cite{Lovasz1979} bounds for finite graphs to the case $V=\R^n$, using the framework of bounded self--adjoint operators. Showing that the result holds for any set avoiding graph $G(V,S)$ is a straightforward adaptation of \cite[\S 5.1]{DMMV}.

\begin{theorem}\label{thm_spectral_bound}
\emph{\cite[\S 3.1]{BdCOV}}
Let $\measure$ be a finite Borel measure supported on $S$ with Fourier transformation
\[
\widehat{\measure}(u)
=	\int_{S} \mexp{u,v} \, \mathrm{d}\measure (v).
\]
Then the measurable chromatic number of $G(V,S)$ satisfies
\[
\chi_m(V,S)
\geq	1-\frac{\sup\limits_{u\in\R^n}\widehat{\measure}(u)}{\inf\limits_{u\in\R^n}\widehat{\measure}(u)}.
\]
\end{theorem}

The problem of computing the measurable chromatic number of $G(V,S)$ gained fame after Hardwiger and Nelson in 1950 studied the case $V=\R^2$ and $S=\mathbb{S}^1$, the Euclidean unit sphere, which remains unsolved. Current bounds and the history of the problem can be found in \cite{Soifer09} and \cite{deGrey18}.

For $V=\R^n$ and $S=\mathbb{S}^{n-1}$ the Euclidean unit sphere, the bounds obtained from \Cref{thm_spectral_bound} have been computed for $\chi_m(\R^n,\mathbb{S}^{n-1})$, see for example \cite{BPT15}. In this case, the optimal measure is the surface measure on $\mathbb{S}^{n-1}$. Beyond the spectral bound, the computation of $\chi_m(\R^n,\mathbb{S}^{n-1})$ has been studied in \cite{BPS21,ambrus22,GM22}.

\subsubsection{Reformulation in terms of Chebyshev polynomials}

For a root system $\Roots$ in $\R^n$ with Weyl group $\weyl$ and weight lattice $\Weights$, we consider those $S\subseteq V$ with Weyl group symmetry, that is $\weyl\, S = S$. The $\weyl$--invariant trigonometric polynomials $\R[\Weights]^\weyl$ with support in $S$ are the Fourier transformations of atomic $\weyl$--invariant Borel measures supported on $\Weights\cap S$. We treat the optimization problem in \Cref{thm_spectral_bound} for this class of measures with the theory developped in \Cref{section_optimization}. In fact, by an averaging argument on all orbits, we see that an optimal measure for \Cref{thm_spectral_bound} is obtained from such a $\weyl$--invariant trigonometric polynomial. We denote by 
\[
\Image
=	\{ \gencos{}(u) \,\vert\, u\in\R^n \}
=	\{ z\in\R^n \,\vert\, \posmat(z) \succeq 0 \}
\]
the image of the generalized cosines and define
\begin{equation}
F(S)
:=	\begin{array}[t]{rl}
	\max\limits_{c} \min\limits_{z}	&	\sum\limits_{\weight \in S\cap\Weights^+} c_\weight\,T_\weight(z) \\
	\mbox{s.t.}						&	z\in\Image ,\, c\in\R^{S\cap\Weights^+}_{\geq 0} ,\, \sum\limits_{\weight\in S\cap\Weights^+} c_\weight = 1 
	\end{array}
.
\end{equation}

\begin{theorem}\label{thm_ChromaticChebyshevBound}
Let $\weyl\,S=S$ and $S\cap\Weights\neq\emptyset$. The measurable chromatic number of $G(V,S)$ satisfies
\[
		\chi_m(V,S)
\geq	1-\frac{1}{F(S)}.
\]
\end{theorem}
\begin{proof}
Since $S$ is bounded, the nonempty set $S\cap\Weights$ is finite. We consider the atomic Borel measure
\[
	\measure
=	\sum\limits_{\weight\in S\cap\Weights} \frac{c_\weight}{\nops{\weyl\weight}} \, \delta_\weight
\]
with $\delta_\weight$ Dirac and $0\leq c_\weight = c_{-\weight}\in\R$, such that, for all $A\in\weyl$, $c_{A\weight}=c_\weight$. Then the Fourier transformation is
\begin{align*}
	\widehat{\measure}(u)
	=	\int_{S} \mexp{u,v} \, \mathrm{d}\measure (v)
	=	\sum\limits_{\weight \in S\cap\Weights}
	\frac{c_\weight}{\nops{\weyl\weight}} \, \mexp{\weight,u} 
	=	\sum\limits_{\weight \in S\cap\Weights^+}
	c_\weight \, \gencos{\weight}(u)
	=	\sum\limits_{\weight \in S\cap\Weights^+}
	c_\weight \, T_\weight(\gencos{}(u)).
\end{align*}
In particular, we have
\begin{align*}
		\widehat{\measure}(u)
\leq	\sum\limits_{\weight \in S\cap\Weights} \frac{c_\weight}{\nops{\weyl\weight}}
=		\sum\limits_{\weight \in S\cap\Weights^+} c_\weight
\end{align*}
and equality holds for $u=0$. Optimizing over the coefficients $c$ under the condition $\sum_{\weight} c_\weight = 1$ and using \Cref{coro_OptiProbelmRewrite} with \Cref{thm_spectral_bound} gives the lower bound $1-1/F(S)$ for $\chi_m(V,S)$.
\end{proof}

In practice, the problem of computing $F(S)$ analytically is not always possible. Instead we can use the theory of \Cref{section_optimization} to lower bound it numerically. For $d \in \N$ sufficiently large, we consider the SDP
\begin{equation}
F(S,d)
:=	\begin{array}[t]{rl}
\sup		&	- \trace(\mathbf{A}_0\,\mathbf{X}) \\
\mbox{s.t.}	&	\mathbf{X} \in\mathrm{Sym}^{(d)}_{\succeq 0},\, \sum\limits_{\weight\in S\cap\Weights^+} \trace(\mathbf{A}_\weight\,\mathbf{X}) = 1,	\\
&	\trace(\mathbf{A}_\weight\,\mathbf{X})	\geq	0 \tbox{for} \weight	\in	S\cap\Weights^+,	\\
&	\trace(\mathbf{A}_\nu\,\mathbf{X})		=		0 \tbox{for} \nu		\in	\Weights^+\setminus (S\cup\{0\})
\end{array}
,
\end{equation}
where the semi--definite cone $\mathrm{Sym}^{(d)}_{\succeq 0}$ and the finitely many matrices $\mathbf{A}_0,\mathbf{A}_\weight,\mathbf{A}_\nu \in \mathrm{Sym}^{(d)}_{\succeq 0}$ are defined as in \Cref{MomentMatrixCoeff}.

\begin{corollary}\label{coro_ChromBoundLasserre}
\emph{[of \Cref{thm_MaxMinConvergence,thm_ChromaticChebyshevBound}]} Let $\weyl\,S=S$ and $S\cap\Weights\neq\emptyset$. The sequence $(F(S,d))_{d\in\N}$ is monotonously non--decreasing and we have
\[
\chi_m(V,S)
\geq	1-\frac{1}{F(S,d)}.
\]
Furthermore, if $\mathrm{QM}(\posmat)$ is Archimedean, then $\lim\limits_{d\to \infty} F(S,d) = F(S)$.
\end{corollary}

\subsection{The chromatic number of a coroot lattice}

For an $n$--dimensional lattice $V=\Corootlattice$ in $\R^n$, we call $\lambda \in \Corootlattice \setminus\{0\}$ a \textbf{strict Vorono\"{i} vector}, if the intersection $(\lambda + \Vor(\Corootlattice)) \cap \Vor(\Corootlattice)$ is a facet of $\Vor(\Corootlattice)$, that is, a face of dimension $n-1$ of the Vorono\"{i} cell. In this case, a natural choice for the avoided set $S$ is the set of all strict Vorono\"{i} vectors of $\Corootlattice$. The chromatic number $\chi(\Corootlattice)$ of the lattice $\Corootlattice$ is defined as the chromatic number of the graph $G(\Corootlattice) := G(\Corootlattice, S)$. 

\begin{figure}[H]
	\begin{center}
		\includegraphics[height=4cm]{hexagonchromlattice2.png}
	\end{center}
	\caption{The chromatic number of the $\RootA[2]$ coroot lattice is $\chi(\Corootlattice(\RootA[2])) = 3 $.}\label{fig_ChiLambda}
\end{figure}

The chromatic number of several instances of these graphs was computed in \cite{DMMV}, some of them through the spectral bound from \Cref{thm_spectral_bound}. In this subsection, we reprove the bounds for the case, where $\Corootlattice$ is the coroot lattice of an irreducible root system.

\begin{proposition}\label{prop_StrictVoronoiHighestRoot}
Assume that $\Corootlattice$ is the coroot lattice of an irreducible root system $\Roots$ with highest root $\highestroot$. Then the set of strict Vorono\"{i} vectors of $\Corootlattice$ is the orbit $S = \weyl \highestroot^\vee$.
\end{proposition}
\begin{proof}
By \cite[Chapitre VI, \S 1, Proposition 11 et 12]{bourbaki456}, there are at most two distinct root lengths and two roots have the same length if and only if they are in the same $\weyl$--orbit. If $\roots \in\Roots$, then $\sprod{ \highestroot , \highestroot } \geq \sprod{ \roots , \roots }$ and so
\[
\sprod{ \highestroot^\vee , \highestroot^\vee }
=		\frac{4}{\sprod{ \highestroot , \highestroot }}
\leq	\frac{4}{\sprod{ \roots , \roots }}
=		\sprod{ \roots^\vee , \roots^\vee }.
\]
Thus, $\highestroot^\vee$ is a short root of the coroot system $\Roots^\vee$. The lattice generated by $\Roots^\vee$ is $\Corootlattice$ and, by the discussion before \cite[Chapter 21, Theorem 8]{conway1988a}, the short roots $\weyl(\Roots^\vee)\highestroot^\vee$ are the strict Vorono\"{i} vectors. As $\weyl(\Roots)=\weyl(\Roots^\vee)$, the statement follows.
\end{proof}

If $\highestroot^\vee\in\Weights$, we obtain
\begin{equation}\label{eq_LatticeChebyshevBound}
		\chi(\Corootlattice)
\geq	1-\frac{1}{\min\limits_{z\in\Image} T_{\highestroot^\vee}(z)}.
\end{equation}
Indeed, since the strict Vorono\"{i} vectors form a single $\weyl$--orbit, there is no freedom for the coefficients in \Cref{thm_ChromaticChebyshevBound} and we are left with minimizing with respect to $z\in\Image$.

If $\highestroot^\vee\notin\Weights$, we can replace $T_{\highestroot^\vee}$ by $T_\weight$ with $\weight=\ell\highestroot^\vee\in\Weights$ for some $\ell>0$, because $\R^n$ is invariant under scaling. For example, this is the case for $\RootG[2]$, where $\highestroot^\vee = \highestroot/3 = \fweight{2}/3$ (and this is the only exception for the irreducible root systems). However, since the coroot lattice of $\RootG[2]$ is the hexagonal one from \Cref{fig_ChiLambda}, this case is covered by $\RootA[2]$.

We now reprove the bounds from \cite{DMMV}.

\begin{theorem}\label{thm_LatticeChromatic}
The following statements hold.
\begin{enumerate}
\item The spectral bound is sharp for $\chi(\Corootlattice(\RootC)) = 2$.
\item The spectral bound is sharp for $\chi(\Corootlattice(\RootA)) = n$.
\item We have $\chi(\Corootlattice(\RootB)) = \chi(\Corootlattice(\RootD)) \geq n$.
\end{enumerate}
\end{theorem}
\begin{proof}
\emph{1.} We have $\Corootlattice(\RootC)=\Z^n$. When we partition $\Z^n$ in elements with even and odd $\ell_1$--norm, then this gives an admissible coloring with $\chi(\Corootlattice(\RootC)) \leq 2$. To see that the spectral bound is sharp, note that $\highestroot^\vee = \highestroot/2 = \fweight{1}$ and consider the Chebyshev polynomial $T_{\highestroot^\vee}=T_{\fweight{1}}=z_1$. With \Cref{eq_LatticeChebyshevBound}, we obtain
\[
		\chi(\Corootlattice(\RootC))
\geq	1-\frac{1}{\min\limits_{z\in\Image} T_{\highestroot^\vee}(z)}
=		1-\frac{1}{\min\limits_{z\in\Image} z_1}
\geq	1-\frac{1}{-1}
=		2,
\]
because  $\Image\subseteq[-1,1]^n$.
	
\emph{2.} We have $\chi(\Corootlattice(\RootA)) = n$ \cite{DMMV} and $\highestroot^\vee = \highestroot = \fweight{1}+\fweight{n-1}$ with $-\fweight{1}\in\weyl\fweight{n-1}$. In \Cref{eq_LatticeChebyshevBound}, we consider
\[
	T_{\fweight{1}+\fweight{n-1}}
=	\nops{\weyl\,\fweight{1}}\,T_{\fweight{1}}\,T_{\fweight{n-1}} - \sum\limits_{\substack{\weight\in \weyl\,\fweight{1}\\ \weight\neq \fweight{1}}} T_{\weight+\fweight{n-1}}
=	n\,z_1\,z_{n-1} - (T_0+(n-2)\,T_{\fweight{1}+\fweight{n-1}}).
\]
The last equation follows from the fact that, if $\weight=-\fweight{n-1}$, then $\weight+\fweight{n-1}=0$, and, if $\weight \neq - \fweight{n-1}$, then $\weight+\fweight{n-1}\in \weyl(\fweight{1}+\fweight{n-1})$, see \Cref{equation_WeightsRootsA}. Since $-\fweight{1}\in\weyl\fweight{n-1}$, we also have $z_1\,z_{n-1} = z_1\,\overline{z_1} =\nops{z_1}^2$ for $z\in\Image$ (in the case of $\RootA$, $\Image$ is complex and can be embedded in $\R^{n-1}$ with \Cref{eq_realgencos}). Altogether, we obtain
\[
		\chi(\Corootlattice(\RootA))
\geq	1-\frac{1}{\min\limits_{z\in\Image} T_{\highestroot^\vee}(z)}
=		1-\frac{n-1}{\min\limits_{z\in\Image} n\,z_1\,z_{n-1}-1}
=		1-\frac{n-1}{\min\limits_{z\in\Image} n\,\nops{z_1}^2-1}
\geq	1-\frac{n-1}{-1}
=		n.
\]

\emph{3.} For $\Roots = \RootB[2]$, we are in the situation of \emph{1.} with $\chi(\Corootlattice(\RootB[2])) = 2$ (the cubic lattice). For $\Roots= \RootB[3]$, we are in the situation of \emph{2.} with $\chi(\Corootlattice(\RootB[3])) = 3$ (see \Cref{RhombicDodecahedron}). The root system $\RootD$ is not defined for $n\leq 3$. Thus, let $n\geq 4$ and $\Roots\in\{\RootB,\RootD\}$. For $1 \leq i \leq n-1$, we have $\roots_i^\vee(\RootB) = \roots_{i}^\vee(\RootD)$ and $\roots_{n}^\vee(\RootB) = \roots_{n}^\vee(\RootD) - \roots_{n-1}^\vee(\RootD)$ as well as $\roots_{n}^\vee(\RootD) = \roots_{n}^\vee(\RootB) + \roots_{n-1}^\vee(\RootB)$. Hence, we have $\Corootlattice(\RootB) = \Corootlattice(\RootD)$ with $\highestroot^\vee = \highestroot = \fweight{2}$. We consider $T_{\highestroot} = T_{\fweight{2}}(z) = z_2$ and minimize on $\Image$. By \Cref{thm_HermiteCharacterization}, we have $\Image = \{z\in \R^n \,\vert \, \posmat(z)\succeq 0 \}$ and the first entry of $\posmat$ is $4\,\posmat_{11}=T_0-T_{2\fweight{1}}$ with
\[
	T_{2\fweight{1}}
=	\nops{\weyl\,\fweight{1}}\,T_{\fweight{1}}^2 - \sum\limits_{\substack{\weight\in \weyl\,\fweight{1}\\ \weight\neq \fweight{1}}} T_{\weight+\fweight{1}}
=	2\,n\,z_1^2 - (1+2\,(n-1)\,z_2).
\]
The last equation follows from the fact that, if $\weight=-\fweight{1}$, then $\weight+\fweight{1}=0$, and, if $\weight \neq - \fweight{1}$, then $\weight+\fweight{1}\in \weyl(\fweight{2})$, see \Cref{equation_WeightsRootsB,equation_WeightsRootsD}. Thus, for $z\in\Image$, we have
\[
		0
\leq	4\,\posmat_{11}(z)
=		T_0(z) - T_{2\fweight{1}}(z)
=		1 - (2\,n\,z_1^2 - 1 - 2\,(n-1)\,z_2)
\Leftrightarrow
		z_2
\geq	\frac{n\,z_1^2-1}{n-1}
\geq	\frac{-1}{n-1}
\]
and obtain
\[
		\chi(\Corootlattice(\Roots))
\geq	1-\frac{1}{\min\limits_{z\in\Image} T_{\highestroot^\vee}(z)}
=		1-\frac{1}{\min\limits_{z\in\Image} T_{\fweight{2}}(z)}
=		1-\frac{1}{\min\limits_{z\in\Image} z_2}
\geq	1-\frac{n-1}{-1}
=		n.
\]
\end{proof}

\begin{remark}
Since, up to rescaling, two adjacent vertices in $G(\Corootlattice)$ are also adjacent in the graph $G(\Corootlattice, \Corootlattice \cap \partial\Vor(\Lambda))$, the value of $\chi(\Lambda)$ also gives a lower bound on $\chi_m(\R^n, \partial\Vor(\Lambda))$, even if the two numbers can be far from each other. For instance, we have $\chi(\Lambda(\RootA[n])) = n+1$, but $\chi(\R^n, \partial\Vor(\Lambda(\RootA[n]))) = 2^n$ \emph{\cite{BBMP}}.
\end{remark} 

\subsection{The chromatic number of $\Z^n$ for the crosspolytope}
\label{sssec_CP}

We consider the integer lattice $V=\Z^n$ together with the avoided set
\[
	\mathbb{B}^1_r
:=	\{ u\in\Z^n \, \vert \, \norm{ u }_1 = \nops{u_1}+\ldots+\nops{u_n}=r\}
\]
for $r\in\N$. Two vertices in the graph $G(\Z^n,\mathbb{B}^1_r)$ are adjacent, if the absolute values of the differences between their coordinates sums up to $r$. The convex hull of $\mathbb{B}^1_r$ is the ball of radius $r$ with respect to the $\ell_1$--norm, also known as the crosspolytope, see \Cref{pic_C3B3L1Norm}. Several bounds for the chromatic number $\chi(\Z^n,\mathbb{B}^1_r)$ were given in \cite{furedi04} without using spectral bounds, but through combinatorial arguments. If $\mathbb{B}^1_r \subseteq \Weights$ is contained in the weight lattice of some root system in $\R^n$, then we can compare by computing
\begin{equation}\label{eq_ZnChebyshevBound}
\chi(\Z^n , \mathbb{B}^1_r)
\geq	1-\frac{1}{F(r)},
\end{equation}
where $F(r):=F(\mathbb{B}^1_r)$ is defined as in \Cref{thm_ChromaticChebyshevBound}.

\begin{lemma}\label{prop_L1WeightsZ}
Let $0 < r\in\N$. If $\Roots$ is a root system of type $\RootB$, $\RootC$ or $\RootD$, then $\mathbb{B}^1_r \subseteq \Weights$ and the dominant weights are $\mathbb{B}^1_r\cap\Weights^+=$
\[
\begin{cases}
	\{	\alpha_1\,\fweight{1} + \ldots + \alpha_n\,\fweight{n} \, \vert \,
	\alpha\in\N^n,\,\sum\limits_{i=1}^n i\,\alpha_i = r \}
	,&	\tbox{if} \Roots = \RootC	\\
	\{	\alpha_1\,\fweight{1} + \ldots + \alpha_{n-1}\,\fweight{n-1} + 2\,\alpha_n\,\fweight{n} \, \vert \,
	\alpha\in\N^n,\,\sum\limits_{i=1}^n i\,\alpha_i = r \}
	,&	\tbox{if} \Roots = \RootB	\\
	\{	 \alpha_1\,\fweight{1} + \ldots + \alpha_{n-2}\,\fweight{n-2} + 2(\alpha_{n-1}\,\fweight{n-1} + \alpha_n\,\fweight{n}) \, \vert \,
	\alpha\in\N^n,\,\sum\limits_{i=1}^{n} i \alpha_i + \alpha_{n-1} = r \}
	,&	\tbox{if} \Roots = \RootD
\end{cases}	.
\]
\end{lemma}
\begin{proof}
This follows from \Cref{equation_WeightsRootsC,equation_WeightsRootsB,equation_WeightsRootsD}.
\end{proof}

\begin{figure}[H]
\begin{center}
	\begin{subfigure}{.3\textwidth}
		\centering
		\includegraphics[width=1.5cm]{B3C3L1NormLvl1.png}
		\caption{$r=1$}
		\label{L1Lvl1}
	\end{subfigure}
	\begin{subfigure}{.3\textwidth}
		\centering
		\includegraphics[width=3cm]{B3C3L1NormLvl2.png}
		\caption{$r=2$}
		\label{L1Lvl2}
	\end{subfigure}
	\quad\quad\quad
	\begin{subfigure}{.3\textwidth}
		\centering
		\includegraphics[width=4.5cm]{B3C3L1NormLvl3.png}
		\caption{$r=3$}
		\label{L1Lvl3}
	\end{subfigure}
	\caption{The crosspolytope of radius $r$ with respect to the $\ell_1$--norm and the points $\mathbb{B}^1_r$ with integer coordinates on the boundary.}
	\label{pic_C3B3L1Norm}
\end{center}
\end{figure}

\begin{remark}\label{remark_ChromZnRn}
Denote by $\mathcal{P}$ the crosspolytope from \emph{\Cref{pic_C3B3L1Norm}} for $r=1$, that is, $\mathcal{P} = \mathrm{ConvHull}(\mathbb{B}^1_1)$. Then $G(\Z^n,\mathbb{B}^1_r)$ is a discrete subgraph of $G(\R^n,\partial(r\mathcal{P}))$ and, since $\R^n$ is scaling invariant, we have
\[
		\chi_m(\R^n,\partial\mathcal{P})
=		\chi_m(\R^n,\partial(r\mathcal{P}))
\geq	\chi(\Z^n,\mathbb{B}^1_r).
\]
Hence, computing the spectral bound for the chromatic number of $\Z^n$ always yields a lower bound for the chromatic number of $\R^n$.
\end{remark}

\subsubsection{Analytical bounds}

We compute the spectral bound for $\chi(\Z^n,\mathbb{B}^1_r)$ first for the cases, where our rewriting technique allows for an analytical proof.

\begin{proposition}\label{coro_SpectralBoundZodd}
Let $r\in \N$ be odd. The spectral bound is sharp for $\chi (\Z^n,\,\mathbb{B}^1_r) = 2$.
\end{proposition}
\begin{proof}
Since $r$ is odd, partitioning the vertices of $G(\Z^n,\mathbb{B}^1_r)$ in those with even and those with odd $\ell_1$--norm yields two independent sets. Hence, $\chi (\Z^n,\,\mathbb{B}^1_r) = \chi (\Z^n,\,\mathbb{B}^1_1) = 2$. To see that the spectral bound is sharp, let $\Roots$ be a root system of type $\RootC$. By \Cref{prop_L1WeightsZ}, we have $\mathbb{B}^1_1 = \weyl \fweight{1}$ and so
\[
	\chi (\Z^n,\,\mathbb{B}^1_1) \geq 1 - \frac{1}{\min\limits_{z\in\Image} z_1} \geq 1-\frac{1}{-1} = 2 .
\]
\end{proof}

The chromatic number of $\Z^n$ for $\ell_1$--distance $r=2$ is $2\,n$. This was proven in \cite[Theorem 1]{furedi04} with a purely combinatorial argument by fixing a coloring and showing that it is admissible and minimal.

\begin{theorem}\label{thm_ZnL1r2bound}
The spectral bound is sharp for $\chi (\Z^n,\,\mathbb{B}^1_2) = 2\,n$.
\end{theorem}
\begin{proof}
Let $\Roots$ be a root system of type $\RootC$. Thanks to \Cref{prop_L1WeightsZ}, we have $\mathbb{B}^1_2 = \weyl (2\,\fweight{1}) \cup \weyl\fweight{2}$. We choose $c=1/(2\,n-1)$ and consider
\[
	c\,T_{2\fweight{1}} + (1-c)\,T_{\fweight{2}}
=	\frac{2\,n\,z_1^2 - 2(n-1)z_2 - 1}{2\,n-1} + \frac{2(n-1)z_2}{2\,n-1}
=	\frac{2\,n\,z_1^2 - 1}{2\,n-1} ,
\]
where the expression for $T_{2\fweight{1}}$ is obtained as in the proof of \Cref{thm_LatticeChromatic} (\emph{3.}). By \Cref{eq_ZnChebyshevBound}, we have
\[
		\chi (\Z^n,\,\mathbb{B}^1_2)
\geq	1-\frac{1}{\min\limits_{z\in\Image} c\,T_{2\fweight{1}}(z) + (1-c)\,T_{\fweight{2}}(z)}
\geq	1-\frac{1}{(2\,n\,z_1^2 - 1)/(2\,n-1)}
\geq	1-\frac{2\,n-1}{ - 1}
=		2\,n.
\]
\end{proof}

\begin{corollary}\label{coro_Z2L1rEven}
Let $0 < r \in \N$ be even. The spectral bound is sharp for $\chi (\Z^2,\,\mathbb{B}^1_r) = 4$.
\end{corollary}
\begin{proof}
For $r=2$, this is a special case of \Cref{thm_ZnL1r2bound}. Since $2$ divides $r$ whenever $r$ is even, the spectral bound gives at least $4$ for $\chi (\Z^2,\,\mathbb{B}^1_r)$. Let $\mathcal{P}=\mathrm{ConvHull}(\mathbb{B}^1_1)$ be the crosspolytope in $\R^2$, that is, a square. By \cite{BBMP} and \Cref{remark_ChromZnRn}, we have
\[
4 = \chi_m(\R^2,\,\partial\mathcal{P}) = \chi_m(\R^2,\,\partial(r\mathcal{P})) \geq \chi(\Z^2,\,\mathbb{B}^1_r) \geq \chi(\Z^2,\,\mathbb{B}^1_2) \geq 4.
\]
\end{proof}

\subsubsection{Numerical bounds}

We will now give spectral bounds for $\chi(\Z^n,\mathbb{B}^1_r)$ numerically for the dimensions $n=3$ and $n=4$. In order to do so, we apply \Cref{coro_ChromBoundLasserre} and compute $F(r,d):=F(\mathbb{B}^1_r,d)$ for $d\in \N$ sufficiently large.

\subsubsection*{Dimension $n = 3$}

\begin{table}[H]
\begin{center}
	\begin{tabular}{|c|c||c|c|c|c|c|c|c|}
		\hline
		$\Roots$	&	$d \backslash r$	&	$2$			&	$4$			&	$6$			&	$8$			&
		$10$		&	$12$		&	$14$		\\
		\hline
		\hline
		$\RootB[3]$	&	$3$					&	$6.00000$	&	$6.28148$	&	$6.01551$	&	$-		$	&
		$-		$	&	$-		$	&	$-		$	\\
		\hline
		&	$4$					&	$6.00000$	&	$6.28148$	&	$6.07717$	&	$6.28148$	&
		$-		$	&	$-		$	&	$-		$	\\
		\hline
		&	$5$					&	$6.00000$	&	$6.28148$	&	$6.29004$	&	$6.28183$	&
		$6.12543$	&	$-		$	&	$-		$	\\
		\hline
		&	$6$					&	$6.00000$	&	$6.28148$	&	$6.30244$	&	$6.29799$	&
		$6.27850$	&	$6.28234$	&	$-		$	\\
		\hline
		&	$7$					&	$6.00000$	&	$6.28148$	&	$6.30269$	&	$6.30435$	&
		$6.30031$	&	$6.29708$	&	$6.27830$	\\
		\hline
		&	$8$					&	$6.00000$	&	$6.28148$	&	$6.30269$	&	$6.30463$	&
		$6.30053$	&	$6.30088$	&	$6.29604$	\\
		\hline
		&	$9$					&	$6.00000$	&	$6.28148$	&	$6.30269$	&	$6.30501$	&
		$6.30502$	&	$6.30227$	&	$6.301858$	\\
		\hline
		\hline
		$\RootC[3]$	&	$3$					&	$6.00000$	&	$6.28148$	&	$6.02310$	&	$-		$	&
		$-		$	&	$-		$	&	$-		$	\\
		\hline
		&	$4$					&	$6.00000$	&	$6.28148$	&	$6.29021$	&	$6.28198$	&
		$-		$	&	$-		$	&	$-		$	\\
		\hline
		&	$5$					&	$6.00000$	&	$6.28148$	&	$6.30182$	&	$6.29951$	&
		$6.29810$	&	$-		$	&	$-		$	\\
		\hline
		&	$6$					&	$6.00000$	&	$6.28148$	&	$6.30269$	&	$6.30455$	&
		$6.30048$	&	$6.30069$	&	$-		$	\\
		\hline
		&	$7$					&	$6.00000$	&	$6.28148$	&	$6.30269$	&	$6.30494$	&
		$6.30057$	&	$6.30229$	&	$6.30156$	\\
		\hline
	\end{tabular}
\end{center}
\caption{The bound $\chi (\Z^3,\,\mathbb{B}^1_r) \geq 1-1/F(r,d)$.}
\label{B3C3L1Table2}
\end{table}

The value $\chi (\Z^3,\,\mathbb{B}^1_2)=6$ is obtained immediately with $F(2,1)$. The highest value in the table is given by $F(9,10)$ for $\RootB[3]$. Furthermore, $F(4,d)$ seems to be stable in the case of both root systems. We give the obtained optimal coefficients, which coincide for $\RootB[3]$ and $\RootC[3]$ in \Cref{B3L1NormCoefficients,C3B3L1NormTable} and \Cref{table_appendix}.

\begin{figure}[H]
	\begin{center}
		\begin{subfigure}{.4\textwidth}
			\centering
			\includegraphics[width=6cm]{B3L1SurfaceMeasure4.png}
			\caption{$r=4$}
		\end{subfigure}
		\quad
		\begin{subfigure}{.4\textwidth}
			\centering
			\includegraphics[width=6cm]{B3L1SurfaceMeasure6.png}
			\caption{$r=6$}
		\end{subfigure}
		\quad
		\begin{subfigure}{.4\textwidth}
			\centering
			\includegraphics[width=6cm]{B3L1SurfaceMeasure8.png}
			\caption{$r=8$}
		\end{subfigure}
		\quad
		\begin{subfigure}{.4\textwidth}
			\centering
			\includegraphics[width=6cm]{B3L1SurfaceMeasure10.png}
			\caption{$r=10$}
		\end{subfigure}
		\quad
		\begin{subfigure}{.4\textwidth}
			\centering
			\includegraphics[width=6cm]{B3L1SurfaceMeasure12.png}
			\caption{$r=12$}
		\end{subfigure}
		\quad
		\begin{subfigure}{.4\textwidth}
			\centering
			\includegraphics[width=6cm]{B3L1SurfaceMeasure14.png}
			\caption{$r=14$}
		\end{subfigure}
		\caption{The coefficients $c_\alpha$ for $F(r,9)$ in the case of $\RootB[3]$, indicated by the intensity of the color as $\mathrm{RGB}(1,1-(c_\alpha-c_{\min})/(c_{\max}-c_{\min}),1-(c_\alpha-c_{\min})/(c_{\max}-c_{\min}))$.}
		\label{B3L1NormCoefficients}
	\end{center}
\end{figure}

\begin{figure}[H]
	\begin{minipage}{0.2\textwidth}
		\begin{flushright}
			\begin{overpic}[width=1\textwidth,,tics=10]{B3L4Z1Coeff}
				\put (83, 90) {\large \textcolor{violet}{$\displaystyle 0.01754$}}
				\put (83, 73) {\large \textcolor{blue}{$\displaystyle 0.22680$}}
				\put (70, 60) {\large \textcolor{red}{$\displaystyle 0.59375$}}
				\put (83, 47) {\large \textcolor{OliveGreen}{$\displaystyle 0.16189$}}
			\end{overpic}
		\end{flushright}
	\end{minipage} \hfill
	\begin{minipage}{0.75\textwidth}
		\begin{table}[H]
			\begin{center}
				\begin{tabular}{|c|c||c|c|}
					\hline
					\multicolumn{2}{|c|}{$\RootC[3]$}								&	\multicolumn{2}{|c|}{$\RootB[3]$}								\\
					\hline
					\hline
					$1-1/F(4,7)$	&	$c_\alpha$									&	$1-1/F(4,9)$	&	$c_\alpha$									\\
					\hline
					$6.28148$		&	\textcolor{violet}{$c_{400}=0.01752	$}		&	$6.28148$		&	\textcolor{violet}{$c_{400}=0.01754$}		\\
					&	\textcolor{blue}{$c_{210}=0.22681	$}		&					&	\textcolor{blue}{$c_{210}=0.22680$}			\\
					&	\textcolor{red}{$c_{101}=0.59380	$}		&					&	\textcolor{red}{$c_{102}=0.59375$}			\\
					&	\textcolor{OliveGreen}{$c_{020}=0.16185$}	&					&	\textcolor{OliveGreen}{$c_{020}=0.16189$}	\\
					\hline
				\end{tabular}
			\end{center}
		\end{table}
	\end{minipage}
	\centering
	\caption{The crosspolytope with radius $r=4$ and the obtained optimal coefficients. Supporting points $\weight=\alpha_1\,\fweight{1}+\alpha_2\,\fweight{2}+\alpha_3\,\fweight{3}$ in the same Weyl group orbit have the same coefficients $c_\alpha$, denoted by red, blue, green and purple dots.}
	\label{C3B3L1NormTable}
\end{figure}

\begin{remark}
This computation confirms the lower bound $7$ from \emph{\cite[Prop. 9]{furedi04}}.
\end{remark}
	
\subsubsection*{Dimension $n = 4$}

\begin{table}[H]
\begin{center}
	\begin{tabular}{|c|c||c|c|c|c|c|c|c|c|}
		\hline
		$\Roots$	&	$d \backslash r$	&	$2$			&	$4$			&	$6$			&
		$8$			&	$10$		&	$12$		&	$14$		\\
		\hline
		\hline
		$\RootB[4]$	&	$4$					&	$8.00000$	&	$10.33968$	&	$9.09234$	&
		$10.33968$	&	$-		$	&	$-		$	&	$-		$	\\
		\hline
		&	$5$					&	$8.00000$	&	$10.33969$	&	$9.72339$	&
		$10.33969$	&	$9.17503$	&	$-		$	&	$-		$	\\
		\hline
		&	$6$					&	$8.00000$	&	$10.83655$	&	$10.18050$	&
		$10.33969$	&	$9.90514$	&	$10.33968$	&	$-		$	\\
		\hline
		&	$7$					&	$8.00000$	&	$10.86019$	&	$10.51696$	&
		$10.51282$	&	$10.16103$	&	$10.33968$	&	$10.03938$	\\
		\hline
		\hline
		$\RootC[4]$	&	$4$					&	$8.00000$	&	$10.33993$	&	$9.72014$	&
		$10.33968$	&	$-		$	&	$-		$	&	$-		$	\\
		\hline
		&	$5$					&	$8.00000$	&	$10.83902$	&	$10.07664$	&
		$10.33968$	&	$9.94864$	&	$-		$	&	$-		$	\\
		\hline
		\hline
		$\RootD[4]$	&	$4$					&	$8.00000$	&	$10.34750$	&	$9.08887$	&
		$10.33969$	&	$-		$	&	$-		$	&	$-		$	\\
		\hline
		&	$5$					&	$8.00000$	&	$10.39184$	&	$9.72430$	&
		$10.34011$	&	$9.52887$	&	$-		$	&	$-		$	\\
		\hline
		&	$6$					&	$8.00000$	&	$10.83844$	&	$10.34886$	&
		$10.35578$	&	$9.97888$	&	$10.33971$	&	$-		$	\\
		\hline
	\end{tabular}
\end{center}
\caption{The bound $\chi (\Z^4,\,\mathbb{B}^1_r) \geq 1-1/F(r,d)$.}
\label{B4C4D4L1Table2}
\end{table}

The value $\chi (\Z^4,\,\mathbb{B}^1_2)=8$ is obtained immediately with $F(2,1)$. The highest value is $F(4,7)$ for $\RootB[4]$. None of the computed bounds $F(r,d)$ is stable in $d$ and we are limited by the size of the semi--definite program, see \Cref{SDPMatrixNumberSizeTable}. Again, in the case of $\RootB[4]$ for example, we see that $F(4,7)\geq F(8,7)$, because we do not take the limit.

\begin{remark}
This computation improves the lower bound $9$ from \emph{\cite[Prop. 9]{furedi04}} by $+2$.
\end{remark}

\subsection{The chromatic number of $\R^n$ for Vorono\"{i} cells}

Finally we consider the case of the Euclidean space $V=\R^n$ as a set of vertices, where the avoided set $S=\partial\mathcal{P}$ is the boundary of a convex centrally--symmetric polytope $\mathcal{P}$. This setting was studied in \cite{BBMP}, giving bounds on $\chi_m(\R^n,\partial\mathcal{P})$ without using spectral bounds. There it was proven that $\chi_m(\R^n,\partial\mathcal{P}) \leq 2^n$ whenever $\mathcal{P}$ tiles $\R^n$ and equality is conjectured. We now investigate the strength of the spectral bound for certain instances of this graph.



\begin{figure}[H]
	\begin{center}
		\includegraphics[height=4cm]{Color_Rn}
	\end{center}
	\caption{The chromatic number of $\R^2$ for the hexagon is $2^2=4$ \cite{BBMP}.}\label{fig_chromaticnumberhexagon}
\end{figure}

Given a Weyl group $\weyl$ associated to a root systems in $\R^n$, the Vorono\"{i} cell of the coroot lattice $\Corootlattice$ is a convex centrally--symmetric polytope, invariant under $\weyl$ and tiles $\R^n$ by $\Corootlattice$--translation, see \Cref{eq_VoronoiTiles}. If the root system is irreducible with highest root $\highestroot$, then we have $\Vor( \Corootlattice ) = \weyl\,\fundom$, where
\[
\fundom
=	\{ u \in \R^n \,\vert\, \forall \, 1\leq i \leq n: \, \sprod{u,\roots_i} \geq 0 \mbox{ and } \sprod{u,\highestroot} \leq 1 \}
\]
is a fundamental domain of the affine Weyl group $\weyl\ltimes \Corootlattice$, see \Cref{prop_FundomAffineWeyl}. In particular, the part of the boundary $\partial\Vor( \Corootlattice ) \cap \overline{\PC}$, which is also contained in the fundamental Weyl chamber, lies on a hyperplane parallel to $\sprod{ \cdot , \highestroot^\vee } = 0$. Rescaling the polytope $\Vor(\Corootlattice)$ by a factor $\tilde{r}>0$ does not affect the chromatic number, that is, $\chi_m(\R^n,\partial\Vor( \Corootlattice )) = \chi_m(\R^n,\partial(\tilde{r}\,\Vor( \Corootlattice )))$. If we choose $\tilde{r}= r\,\sprod{\highestroot,\highestroot}/2$ for some $0 \neq r \in \N$, then $\partial(\tilde{r}\,\Vor( \Corootlattice )) \cap \Weights \neq \emptyset$ and we obtain a hierarchy of lower bounds
\begin{equation}\label{eq_VoronoiLowerBound}
		\chi_m ( \R^n , \partial \Vor ( \Corootlattice ) )
\geq	\ldots
\geq 	1-\frac{1}{F(4r)}
\geq 	1-\frac{1}{F(2r)}
\geq 	1-\frac{1}{F(r)}
\geq 	1-\frac{1}{F(1)},
\end{equation}
where $F(r):=F(S_r)$ is as in \Cref{thm_ChromaticChebyshevBound} with $S_r:=\weyl\{ u \in \overline{\PC} \,\vert\, \sprod{u,\highestroot^\vee} = r \}$. 

\begin{remark}
The quantity $1-1/F(r)$ is a lower bound for $\chi_m ( \R^n , \partial \Vor ( \Corootlattice ) )$. More precisely, we have
\[
		\chi_m ( \R^n , \partial \Vor ( \Corootlattice ) )
\geq	\chi ( \Weights , S_r )
\geq	1-\frac{1}{F(r)}
\]
and $F(r)$ is the minimum of the Fourier transformation of the optimal measure $\measure$ (with mass $1$) in \emph{\Cref{thm_spectral_bound}} for the graph $G(\Weights,S_r)$.
\end{remark}

To compute $F(r)$ numerically, we use \Cref{coro_ChromBoundLasserre} and write $F(r,d) := F(S_r,d)$. Note that, in this case, $F(r,d) \geq F(\ell\,r,d)$ is only certain for $\ell\in\N$ when $d\to \infty$.

\begin{figure}[H]
	\begin{center}
		\begin{overpic}[height=4cm,,tics=10]{hexagonlevels}
			\put (52, 48) {\small \textcolor{red}{$\displaystyle r=1$}}
			\put (58, 58) {\small \textcolor{red}{$\displaystyle r=2$}}
			\put (64, 68) {\small \textcolor{red}{$\displaystyle r=3$}}
			\put (70, 78) {\small \textcolor{red}{$\displaystyle r=4$}}
		\end{overpic}
	\end{center}
	\caption{Rescaling the hexagon increases the number of weights $S_r\cap\Weights$ on the boundary.}\label{fig_UnitDist}
\end{figure}

\subsubsection{The hexagon in $\R^2$}

The hexagon in $\R^2\cong \R^3/\langle [1,1,1]^t\rangle$, as it has appeared several times now in the article, is the Vorono\"{i} cell of the coroot lattice $\Corootlattice$ for $\RootA[2]$ and $\RootG[2]$. It has $6$ vertices and $6$ edges.

For $\RootA[2]$, the vertices of the hexagon are the orbits of the fundamental weights $\fweight{1}$ and $\fweight{2}$. The centers of the edges are the orbit of $(\fweight{1}+\fweight{2})/2$. We fix a hierarchy order $d\geq 3$ and consider $F(r,d)$ for $1 \leq r\leq 2d$.

For $\RootG[2]$, the vertices are the orbit of $\fweight{1}/3$. The centers of edges are the orbit of $\fweight{2}/6$. If $r\in\N$ is not a multiple of $3$, then $S_r=\emptyset$. Thus we consider $F(3r,d)$ for $1 \leq r\leq 2d$, but still write $F(r,d)$.

The first column indicates the root system, that is, $\RootA[2]$ or $\RootG[2]$. Then the rows are indexed by the relaxation order $d$ and the columns by the radius $r$.

\begin{table}[H]
	\begin{center}
		\scalebox{0.65}{
			\begin{tabular}{|c|c||c|c|c|c|c|c|c|c|c|c|c|c|c|c|}
				\hline
				$\Roots$	&	$d \backslash r$	&	$1$			&	$2$			&	$3$			&	$4$			&	$5$			&	$6$			&	$7$			&	$8$			&
				$9$			&	$10$		&	$11$		&	$12$		&	$13$		&	$14$		\\
				\hline
				\hline
				$\RootA[2]$	&	$3$					&	$2.99386$	&	$3.57143$	&	$3.52451$	&	$3.57143$	&	$3.37484$	&	$3.57143$	&	$-		$	&	$-		$	&
				$-		$	&	$-		$	&	$-		$	&	$-		$	&	$-		$	&	$-		$	\\
				\hline
				&	$4$					&	$3.00000$	&	$3.57143$	&	$3.52911$	&	$3.57143$	&	$3.54698$	&	$3.57143$	&	$3.47461$	&	$3.57143$	&
				$-		$	&	$-		$	&	$-		$	&	$-		$	&	$-		$	&	$-		$	\\
				\hline
				&	$5$					&	$3.00000$	&	$3.57143$	&	$3.52912$	&	$3.57143$	&	$3.54789$	&	$3.57143$	&	$3.54016$	&	$3.57143$	&
				$3.51384$	&	$3.57143$	&	$-		$	&	$-		$	&	$-		$	&	$-		$	\\
				\hline
				&	$6$					&	$3.00000$	&	$3.57143$	&	$3.52912$	&	$3.57143$	&	$3.54789$	&	$3.57143$	&	$3.54786$	&	$3.57143$	&
				$3.55920$	&	$3.57143$	&	$3.47623$	&	$3.57143$	&	$-		$	&	$-		$	\\
				\hline
				&	$7$					&	$3.00000$	&	$3.57143$	&	$3.52912$	&	$3.57143$	&	$3.54789$	&	$3.57143$	&	$3.55183$	&	$3.57143$	&
				$3.55921$	&	$3.57143$	&	$3.51433$	&	$3.57143$	&	$3.14739$	&	$3.57143$	\\
				\hline
				&	$8$					&	$3.00000$	&	$3.57143$	&	$3.52912$	&	$3.57143$	&	$3.54789$	&	$3.57143$	&	$3.55347$	&	$3.57143$	&
				$3.55921$	&	$3.57143$	&	$3.53571$	&	$3.57143$	&	$3.25411$	&	$3.57143$	\\
				\hline
				\hline
				$\RootG[2]$	&	$3$					&	$2.99732$	&	$3.57143$	&	$3.39930$	&	$3.57143$	&	$2.47997$	&	$3.57143$	&	$-		$	&	$-		$	&
				$-		$	&	$-		$	&	$-		$	&	$-		$	&	$-		$	&	$-		$	\\
				\hline
				&	$4$					&	$2.99962$	&	$3.57143$	&	$3.52821$	&	$3.57143$	&	$3.41805$	&	$3.57143$	&	$2.54024$	&	$3.57143$	&
				$-		$	&	$-		$	&	$-		$	&	$-		$	&	$-		$	&	$-		$	\\
				\hline
				&	$5$					&	$3.00000$	&	$3.57143$	&	$3.52908$	&	$3.57143$	&	$3.49102$	&	$3.57143$	&	$2.76603$	&	$3.57143$	&
				$2.45902$	&	$3.57143$	&	$-		$	&	$-		$	&	$-		$	&	$-		$	\\
				\hline
				&	$6$					&	$3.00000$	&	$3.57143$	&	$3.52912$	&	$3.57143$	&	$3.52318$	&	$3.57143$	&	$3.39290$	&	$3.57143$	&
				$2.70265$	&	$3.57143$	&	$2.98423$	&	$3.57143$	&	$-		$	&	$-		$	\\
				\hline
				&	$7$					&	$3.00000$	&	$3.57143$	&	$3.52912$	&	$3.57143$	&	$3.54301$	&	$3.57143$	&	$3.54780$	&	$3.57143$	&
				$3.53627$	&	$3.57143$	&	$3.28144$	&	$3.57143$	&	$2.50993$	&	$3.57143$	\\
				\hline
				&	$8$					&	$3.00000$	&	$3.57143$	&	$3.52912$	&	$3.57143$	&	$3.54656$	&	$3.57143$	&	$3.55294$	&	$3.57143$	&
				$3.54181$	&	$3.57143$	&	$3.54139$	&	$3.57143$	&	$3.13764$	&	$3.57143$	\\
				\hline
			\end{tabular}
		}
	\end{center}
	\caption{The bound $\chi_m (\R^2,\,\partial \Vor(\Corootlattice(\RootA[2]))) = \chi_m (\R^2,\,\partial \Vor(\Corootlattice(\RootG[2]))) \geq 1-1/F(r,d)$ for the hexagon.}
	\label{A2HexagonTable}
\end{table}

For $r=1$, there is no choice for the coefficients $c_\weight$, as $S_1$ only contains one element in both cases $\RootA[2]$ and $\RootG[2]$. The value $F(1)$ is $-1/2$. This gives spectral bound $3$ and is obtained from $F(r,d)$ for $d\geq 4$, respectively $d\geq 5$. Furthermore, this fits with the bound from \Cref{thm_LatticeChromatic}, where $\chi(\Corootlattice) \geq n$ for $\RootA$.

For $r\geq 2$, the best possible bound we obtained is already assumed at $r=2$ and $d=3$. We display the optimal coefficients for the corresponding measure below. This bound is assumed in all $F(r,d)$ with $r$ even at lowest possible order. For $r$ odd, the value converges but does not stabilize. 

Although we recover that the chromatic number of $\R^2$ for the hexagon is $4$, see \Cref{fig_UnitDist}, our computations indicate that the spectral bound is not sharp and never will be with $r,d\to\infty$.

\begin{figure}[H]
	\begin{minipage}{0.2\textwidth}
		\begin{flushright}
			\begin{overpic}[width=1\textwidth,,tics=10]{A2Level2Coeff}
				\put (58,95) {\textcolor{red}{$\displaystyle 0.33333$}}
				\put (65,75) {\textcolor{blue}{$\displaystyle 0.66667$}}
			\end{overpic}
		\end{flushright}
	\end{minipage} \hfill
	\begin{minipage}{0.75\textwidth}
		\begin{table}[H]
			\begin{center}
				\scalebox{1}{
					\begin{tabular}{|c||c|c||c|c|}
						\hline
						&	\multicolumn{2}{c||}{$\RootA[2]$}					&	\multicolumn{2}{c|}{$\RootG[2]$}			\\
						\hline
						\hline
						$r$		&	$1-1/F(r,8)$	&	$c_\alpha=c_{\conj{\alpha}}$
						&	$1-1/F(r,8)$	&	$c_\alpha$						\\
						\hline
						$1$		&	$3.00000$		&	$c_{10}=1.00000$				&	$3.00000$	&	$c_{10}=1.00000$	\\
						\hline
						$2$		&	$3.57143$		&	\textcolor{red}{$c_{20}=0.33333$}				&	$3.57143$	&	\textcolor{red}{$c_{20}=0.33333$}	\\
						&					&	\textcolor{blue}{$c_{11}=0.66667$}				&				&	\textcolor{blue}{$c_{01}=0.66667$}	\\
						\hline
					\end{tabular}
				}
			\end{center}
		\end{table}
	\end{minipage}
	\centering
	\caption{The scaled Vorono\"{i} cell and the optimal coefficients for $F(2,8)$. Supporting points $\weight=\alpha_1\,\fweight{1}+\alpha_2\,\fweight{2}$ in the same Weyl group orbit and their additive inverse $\conj{\weight}$ have the same coefficients $c_\alpha=c_{\conj{\alpha}}$, denoted by either red or blue dots.}
	\label{HexagonTable}
\end{figure}

From \Cref{HexagonTable}, we recover the coefficients $1/3$ for the vertices and $2/3$ for the centers of faces, indicating the best possible discrete measure. Indeed, for $r\in\N$, we have
\begin{equation}\label{remark_A2G2Min}
\begin{split}
	F(2r)
=&	\begin{cases}
	\min\limits_{z\in\Image} \frac{2}{3} \, T_{r\,r}(z) + \frac{1}{6} \, (T_{2r\,0}(z)+T_{0\,2r}(z)) = \min\limits_{z\in\Image} \frac{2}{3} \, T_{1\,1}(z) +  + \frac{1}{6} \, (T_{2\,0}(z)+T_{0\,2}(z)) , & \tbox{if} \Roots=\RootA[2]\\
	\min\limits_{z\in\Image} \frac{2}{3} \, T_{0\,r}(z) + \frac{1}{3} \, T_{2r\,0}(z) = \min\limits_{z\in\Image} \frac{2}{3} \, T_{0\,1}(z) + \frac{1}{3} \, T_{2\,0}(z) , & \tbox{if} \Roots=\RootG[2]
\end{cases}\\
=&	\min\limits_{z\in\Image} 2\,z_1^2 - 2/3\,z_1 - 1/3
=	-7/18
\end{split}
\end{equation}
(for $\RootA[2]$, we have to substitute $z_i=z_1\pm \mathrm{i}\,z_2$, so that $\Image\subseteq \R^2$). In both cases, $1-1/F(2r) = 25/7 \approx 3.57143$. Note that $F(2)$ corresponds to the trigonometric polynomial in \Cref{example_A2PolyRewrite} up to a factor $1/3$.

\begin{figure}[H]
\begin{center}		
	\begin{subfigure}{.4\textwidth}
		\centering
		\includegraphics[height=6.4cm]{A2DeltoidMin.png}
		\hspace{.1cm}
		\includegraphics[height=6cm]{A2FundomMin.png}
		\caption{$\RootA[2]$}
		\label{A2Min}
	\end{subfigure}
	\quad
	\begin{subfigure}{.4\textwidth}
		\centering
		\includegraphics[height=6.4cm]{G2DeltoidMin.png}
		\hspace{.1cm}
		\includegraphics[height=6cm]{G2FundomMin.png}
		\caption{$\RootG[2]$}
		\label{G2Min}
	\end{subfigure}
	\caption{The minimizers $z$ (lines, above) for $F(2r)$ in the image $\Image$ of the generalized cosines with preimages $u$ (ovals, below). In the coordinates $u$, we can observe the periodicity with respect to the coroot lattice $\Corootlattice$ as well as the $\weyl$--invariance, yielding the crystallographic symmetry on the alcove $\fundom$ of $\weyl\ltimes\Corootlattice$ (simplex).}
	\label{fig_A2G2Min}
\end{center}
\end{figure}

\subsubsection{The rhombic dodecahedron in $\R^3$}

The rhombic dodecahedron in $\R^3$ (\Cref{RhombicDodecahedron}) is the Vorono\"{i} cell of the coroot lattice $\Corootlattice$ for $\RootA[3]$ and $\RootB[3]$. It has $14$ vertices, $24$ edges and $12$ faces.

For $\RootA[3]$, the vertices are the orbits of $\fweight{1}$, $\fweight{2}$ and $\fweight{3}$. The centers of the edges are the orbits of $(\fweight{i}+\fweight{2})/2$ for $i=1,2$, and the centers of the facets are the orbit of $(\fweight{1}+\fweight{3})/2$.

For $\RootB[3]$, the vertices are the orbits of $\fweight{1}$ and $\fweight{3}$. The centers of the edges are the orbit of $(\fweight{1}+\fweight{3})/2$, and the centers of the facets are the orbit of $\fweight{2}/2$.

\begin{figure}[H]
	\begin{center}
		\begin{subfigure}{.3\textwidth}
			\centering
			\includegraphics[width=4cm, height=4cm]{A3VoronoiCell.png}
			\caption{$\RootA[3]$}
			\label{RhombicDodecahedronA}
		\end{subfigure}
		\quad
		\begin{subfigure}{.3\textwidth}
			\centering
			\includegraphics[width=4cm, height=4cm]{B3VoronoiCell.png}
			\caption{$\RootB[3]$}
			\label{RhombicDodecahedronB}
		\end{subfigure}
		\caption{The rhombic dodecahedron is the Vorono\"{i} cell of the coroot lattice for $\RootA[3]$ and $\RootB[3]$.}
	\label{RhombicDodecahedron}
\end{center}
\end{figure}

\begin{table}[H]
\begin{center}
	\scalebox{0.65}{
		\begin{tabular}{|c|c||c|c|c|c|c|c|c|c|c|c|c|c|c|c|}
			\hline
			$\Roots$	&	$d \backslash r$	&	$1$			&	$2$			&	$3$			&	$4$			&	$5$			&	$6$			&	$7$			&	$8$			&
			$9$			&	$10$		&	$11$		&	$12$		&	$13$		&	$14$		\\
			\hline
			\hline
			$\RootA[3]$	&	$4$					&	$3.99424$	&	$6.10767$	&	$5.86933$	&	$6.10766$	&	$5.81858$	&	$6.10766$	&	$4.77576$	&	$6.10766$	&
			$-		$	&	$-		$	&	$-		$	&	$-		$	&	$-		$	&	$-		$	\\
			\hline
			&	$5$					&	$3.99611$	&	$6.10767$	&	$5.86964$	&	$6.10766$	&	$5.90988$	&	$6.10767$	&	$5.85369$	&	$6.10766$	&
			$5.46888$	&	$6.10766$	&	$-		$	&	$-		$	&	$-		$	&	$-		$	\\
			\hline
			&	$6$					&	$3.99653$	&	$6.10767$	&	$5.86972$	&	$6.10767$	&	$5.93658$	&	$6.10767$	&	$5.85762$	&	$6.10766$	&
			$5.85825$	&	$6.10766$	&	$3.78978$	&	$6.10766$	&	$-		$	&	$-		$	\\
			\hline
			&	$7$					&	$3.99702$	&	$6.10767$	&	$5.86988$	&	$6.10767$	&	$5.94146$	&	$6.10766$	&	$5.96334$	&	$6.10767$	&
			$5.85986$	&	$6.10766$	&	$4.12186$	&	$6.10766$	&	$-		$	&	$6.10766$	\\
			\hline
			&	$8$					&	$3.99719$	&	$6.10767$	&	$5.86992$	&	$6.10767$	&	$5.94327$	&	$6.10767$	&	$6.05399$	&	$6.10767$	&
			$5.86357$	&	$6.10766$	&	$5.59839$	&	$6.10766$	&	$3.88490$	&	$6.10766$	\\
			\hline
			\hline
			$\RootB[3]$	&	$3$					&	$3.83791$	&	$6.10767$	&	$3.39918$	&	$6.10766$	&	$-		$	&	$6.10766$	&	$-		$	&	$-		$	&
			$-		$	&	$-		$	&	$-		$	&	$-		$	&	$-		$	&	$-		$	\\
			\hline
			&	$4$					&	$3.84571$	&	$6.10767$	&	$4.11626$	&	$6.10766$	&	$-		$	&	$6.10766$	&	$-		$	&	$6.10766$	&
			$-		$	&	$-		$	&	$-		$	&	$-		$	&	$-		$	&	$-		$	\\
			\hline
			&	$5$					&	$3.98454$	&	$6.10767$	&	$5.80542$	&	$6.10766$	&	$5.08174$	&	$6.10767$	&	$-		$	&	$6.10766$	&
			$-		$	&	$6.10766$	&	$-		$	&	$-		$	&	$-		$	&	$-		$	\\
			\hline
			&	$6$					&	$3.99667$	&	$6.10767$	&	$5.87057$	&	$6.10767$	&	$5.86644$	&	$6.10767$	&	$5.82630$	&	$6.10766$	&
			$-		$	&	$6.10766$	&	$-		$	&	$6.10766$	&	$-		$	&	$-		$	\\
			\hline
			&	$7$					&	$3.99872$	&	$6.10767$	&	$5.87057$	&	$6.10767$	&	$5.94578$	&	$6.10766$	&	$5.96989$	&	$6.10767$	&
			$5.88810$	&	$6.10766$	&	$-		$	&	$6.10766$	&	$-		$	&	$6.10766$	\\
			\hline
			&	$8$					&	$3.99925$	&	$6.10767$	&	$5.87057$	&	$6.10767$	&	$5.96374$	&	$6.10767$	&	$5.99825$	&	$6.10767$	&
			$5.94949$	&	$6.10766$	&	$5.92157$	&	$6.10766$	&	$5.31568$	&	$6.10766$	\\
			\hline
			&	$9$					&	$3.99972$	&	$6.10767$	&	$5.87057$	&	$6.10767$	&	$5.97050$	&	$6.10767$	&	$6.00193$	&	$6.10767$	&
			$5.98345$	&	$6.10767$	&	$5.98654$	&	$6.10766$	&	$5.93977$	&	$6.10766$	\\
			\hline
		\end{tabular}
	}
\end{center}
	\caption{The bound $\chi_m (\R^3,\,\partial \Vor(\Corootlattice(\RootA[3]))) = \chi_m (\R^3,\,\partial \Vor(\Corootlattice(\RootB[3]))) \geq 1-1/F(r,d)$ for the rhombic dodecahedron.}
\label{A3RhombicTable}
\end{table}

For $r=1$, the numerically computed bound seems to converge to $4$. For $r\geq 2$, the best possible bound we obtain is already assumed at $r=2$ and $d=3$, respectively $d=4$. We display the optimal coefficients for the corresponding measure below. This bound is approximately assumed in all $F(r,d)$ with $r$ even at lowest possible order $d$. For $r$ odd, the value does not stabilize with $r$ or $d$ growing. $\RootA[3]$ and $\RootB[3]$ give the same coefficients for the same supporting points.
As in the case of the hexagon, the gap between the spectral bound for such discrete measures and the actual chromatic number of $\R^3$ for the rhombic dodecahedron (known to be $8$ by \cite{BBMP}) seems quite large.

\begin{figure}[H]
\begin{minipage}{0.2\textwidth}
	\begin{flushright}
		\begin{overpic}[width=1\textwidth,,tics=10]{A3Level2Coeff}
			\put (47, 97) {\large \textcolor{violet}{$\displaystyle 0.10283$}}
			\put (67, 84) {\large \textcolor{red}{$\displaystyle 0.24388$}}
			\put (81, 73) {\large \textcolor{OliveGreen}{$\displaystyle 0.06050$}}
			\put (70, 61) {\large \textcolor{blue}{$\displaystyle 0.59279$}}
		\end{overpic}
	\end{flushright}
\end{minipage} \hfill
\begin{minipage}{0.75\textwidth}
	\begin{table}[H]
		\begin{center}
			\begin{tabular}{|c||c|c||c|c|}
				\hline
				&	\multicolumn{2}{c||}{$\RootA[3]$}												&	\multicolumn{2}{c|}{$\RootB[3]$}								\\
				\hline
				\hline
				$r$					&	$1-1/F(r,8)$	&	$c_\alpha=c_{\conj{\alpha}}$			&	$1-1/F(r,9)$	&	$c_\alpha$									\\
				\hline
				$1$					&	$3.99719$		&	$c_{010}=0.33298$						&	$3.99972$		&	$c_{100}=0.33332$							\\
				&					&	$c_{100}=0.66702$											&					&	$c_{001}=0.66668$							\\
				\hline
				$2$					&	$6.10767$		&	\textcolor{violet}{$c_{020}=0.10282$}	&	$6.10767$		&	\textcolor{violet}{$c_{200}=0.10283$}		\\
				&					&	\textcolor{red}{$c_{110}=0.24392$}							&					&	\textcolor{red}{$c_{101}=0.24388$}			\\
				&					&	\textcolor{OliveGreen}{$c_{200}=0.06050$}					&					&	\textcolor{OliveGreen}{$c_{002}=0.06050$}	\\
				&					&	\textcolor{blue}{$c_{101}=0.59276$}							&					&	\textcolor{blue}{$c_{010}=0.59279$}			\\
				\hline
			\end{tabular}
		\end{center}
	\end{table}
\end{minipage}
\centering
\caption{The scaled Vorono\"{i} cell and the obtained optimal coefficients. Supporting points $\weight=\alpha_1\,\fweight{1}+\alpha_2\,\fweight{2}+\alpha_3\,\fweight{3}$ in the same Weyl group orbit and their additive inverse $\conj{\weight}$ have the same coefficients $c_\alpha=c_{\conj{\alpha}}$, denoted by red, blue, green and purple dots.}
\label{RhombicDodecahedronTable}
\end{figure}

As we can observe, the most weight is put on the center of faces, then on the centers of edges and only small weight is put on the vertices. We investigate the minimizers of the associated sum of generalized Chebyshev polynomials. Similar to \Cref{remark_A2G2Min}, one finds the following.
\begin{enumerate}
	\item For $\Roots = \RootB[3]$, the minimizers for $F(2,8)$ are $z_{\mathrm{min}} \approx	(0.05927, z_2, 0.22212)$ with $z_2\in\R$ so that $z_{\mathrm{min}}\in\Image$.	
	\item For $\Roots = \RootA[3]$, the minimizers for $F(2,8)$ are $z_{\mathrm{min}} \approx	(0.22209, 0.05915, z_3)$ with $z_3\in\R$ so that $z_{\mathrm{min}}\in\Image$.
\end{enumerate}

\begin{figure}[H]
\begin{center}
	\begin{subfigure}{.3\textwidth}
		\centering
		\includegraphics[width=4cm, height=4cm]{A3FundomMin.png}
		\caption{\textcolor{red}{$u_{\mathrm{min}}$}}
		\label{A3MinFundom}
	\end{subfigure}
	\quad
	\begin{subfigure}{.3\textwidth}
		\centering
		\includegraphics[width=4cm, height=4cm]{A3DeltoidMin.png}
		\caption{\textcolor{red}{$z_{\mathrm{min}}$}}
		\label{A3MinDeltoid}
	\end{subfigure}
	\caption{In the case of $\RootA[3]$, there are two minimizers \textcolor{red}{$z_{\mathrm{min}} \approx (0.22209, 0.05915, \pm 0.23708)$} for $F(2,8)$ on the boundary of \textcolor{blue}{$\Image$}, the image of the gernalized cosines, with two preimages \textcolor{red}{$u_{\mathrm{min}}\approx (0.40432, \pm 0.15713, 0.17550)$} on the boundary of \textcolor{blue}{$\fundom$}, the fundamental domain of $\weyl\ltimes\Corootlattice$.}
	\label{A3Min}
\end{center}
\end{figure}

\subsubsection{The icositetrachoron in $\R^4$}

The icositetrachoron in $\R^4$ is the Vorono\"{i} cell of the coroot lattice $\Corootlattice$ for $\RootB[4]$ and $\RootD[4]$. It has $24$ vertices, $96$ edges, $96$ faces and $24$ facets. The facets are octahedral cells.

For $\RootB[4]$, the vertices are the orbits of $\fweight{1}$ and $\fweight{4}$. The centers of edges are the orbits of $(\fweight{1}+\fweight{4})/2$ and $\fweight{3}/2$. The centers of faces are the orbit of $(\fweight{1}+\fweight{3})/3$. The centers of facets are the orbit of $\fweight{2}/2$.

For $\RootD[4]$, the vertices are the orbits of $\fweight{1}$, $\fweight{3}$ and $\fweight{4}$. The centers of edges are the orbits of $(\fweight{1}+\fweight{3})/2$, $(\fweight{1}+\fweight{4})/2$ and $(\fweight{3}+\fweight{4})/2$. The centers of faces are the orbit of $(\fweight{1}+\fweight{3}+\fweight{4})/3$. The centers of facets are the orbit of $\fweight{2}/2$.

\begin{table}[H]
\begin{center}
	\scalebox{0.7}{
		\begin{tabular}{|c|c||c|c|c|c|c|c|c|c|c|c|c|c|}
			\hline
			$\Roots$	&	$d\backslash r$	&	$1$			&	$2$			&	$3$			&	$4$			&	$5$			&	$6$			&	$7$			&
			$8$			&	$9$			&	$10$		&	$11$		&	$12$		\\
			\hline
			\hline
			$\RootB[4]$	&	$4$				&	$3.01160$	&	$10.00001$	&	$-		$	&	$10.00000$	&	$-		$	&	$10.0000$	&	$-		$	&
			$10.00000$	&	$-		$	&	$-		$	&	$-		$	&	$-		$	\\
			\hline
			&	$5$				&	$3.77462$	&	$10.00035$	&	$-		$	&	$10.00000$	&	$-		$	&	$10.00000$	&	$-		$	&
			$10.00000$	&	$-		$	&	$10.00000$	&	$-		$	&	$-		$	\\
			\hline
			&	$6$				&	$3.99453$	&	$10.02433$	&	$9.10927$	&	$10.01295$	&	$8.91701$	&	$10.00001$	&	$4.69147$	&
			$10.00000$	&	$-		$	&	$10.00000$	&	$-		$	&	$10.00000$	\\
			\hline
			&	$7$				&	$3.99961$	&	$10.02434$	&	$9.12574$	&	$10.01902$	&	$9.26148$	&	$10.00819$	&	$9.32108$	&
			$10.00000$	&	$8.35442$	&	$10.00000$	&	$4.15681$	&	$10.00000$	\\
			\hline
			\hline
			$\RootD[4]$	&	$4$				&	$3.07035$	&	$10.00004$	&	$-		$	&	$10.00000$	&	$-		$	&	$10.00000$	&	$-		$	&
			$10.00000$	&	$-		$	&	$-		$	&	$-		$	&	$-		$	\\
			\hline
			&	$5$				&	$3.94031$	&	$10.00231$	&	$-		$	&	$10.00000$	&	$-		$	&	$10.00000$	&	$-		$	&
			$10.00000$	&	$-		$	&	$10.00000$	&	$-		$	&	$-		$	\\
			\hline
			&	$6$				&	$3.99496$	&	$10.02432$	&	$9.11312$	&	$10.01314$	&	$8.93873$	&	$10.00001$	&	$5.12215$	&
			$10.00000$	&	$-		$	&	$10.00000$	&	$-		$	&	$10.00000$	\\
			\hline
		\end{tabular}
	}
\end{center}
	\caption{The bound $\chi_m (\R^4,\,\partial \Vor(\Corootlattice(\RootB[4]))) = \chi_m (\R^4,\,\partial \Vor(\Corootlattice(\RootD[4]))) \geq 1-1/F(r,d)$ for the icositetrachoron.}
\label{B4D4IcositetrachoronTable2}
\end{table}

For $r=1$, the numerically computed bound seems to converge to $4$. For $r\geq 2$, the best possible bound we obtained is assumed at $r=2$ and $d=7$, respectively $d=6$. For $r$ odd, the value is always smaller than for $r$ even.

We observe that, for $\RootB[4]$, we have $F(2,7) \geq F(4,7)$ although $2$ divides $4$. This is because the monotonous growth in \Cref{thm_ChromaticChebyshevBound} only holds for $d\to\infty$. In the $\RootD[4]$ case, we have the same for $F(2,6) \geq F(4,6)$. We display the optimal coefficients for the corresponding measure below.

\begin{table}[H]
\begin{center}
	\begin{tabular}{|c||c|c||c|c|}
		\hline
		&	\multicolumn{2}{c||}{$\RootB[4]$}		&	\multicolumn{2}{c|}{$\RootD[4]$}			\\
		\hline
		\hline
		$r$		&	$1-1/F(r,7)$	&	$c_\alpha$			&	$1-1/F(r,6)$	&	$c_\alpha$			\\
		\hline
		$1$		&	$3.99961$		&	$c_{1000}=0.33303$	&	$3.99496$		&	$c_{1000}=0.33305$	\\
		&					&						&					&	$c_{0010}=0.33348$	\\
		&					&	$c_{0001}=0.66697$	&					&	$c_{0001}=0.33348$	\\
		\hline
		$2$		&	$10.02434$		&	$c_{0100}=0.40062$	&	$10.02432$		&	$c_{0100}=0.40188$	\\
		&					&	$c_{1001}=0.35491$	&					&	$c_{1001}=0.17692$	\\
		&					&						&					&	$c_{1010}=0.17692$	\\
		&					&	$c_{0010}=0.17769$	&					&	$c_{0011}=0.17726$	\\
		&					&	$c_{0002}=0.04444$	&					&	$c_{0002}=0.02228$	\\
		&					&						&					&	$c_{0020}=0.02228$	\\
		&					&	$c_{2000}=0.02234$	&					&	$c_{2000}=0.02245$	\\
		\hline
	\end{tabular}
\end{center}
\caption{The optimal coefficients for $F(r,7)$, respectively $F(r,6)$. The coefficients associated to $\weight=\alpha_1\,\fweight{1}+\ldots+\alpha_4\,\fweight{4}$ are denoted by $c_\alpha$.}
\label{icositetrachoronTable}
\end{table}

Recall from \Cref{equation_WeightsRootsB,equation_WeightsRootsD} that the fundamental weights satisfy $\fweight{i}(\RootB[4]) = \fweight{i}(\RootD[4])$ for $i=1,2,4$ and $\fweight{3}(\RootB[4]) = \fweight{3}(\RootD[4]) + \fweight{4}(\RootD[4])$. In the case of $r=2$ in \Cref{icositetrachoronTable}, we observe that
\begin{enumerate}
\item the centers of facets are weighted with $0.40062\approx 0.40188$,
\item the centers of faces are not weighted,
\item the centers of edges are weighted with $0.35491\approx 0.17692+0.17692$ and $0.17769\approx 0.17726$ and
\item the vertices are weighted with $0.02234\approx 0.02245$ and $0.04444\approx 0.02228+0.02228$.
\end{enumerate}
Further computations are limited by the size of the semi--definite program, see \Cref{SDPMatrixNumberSizeTable}. Note that the chromatic number of $\R^4$ for the icositetrachoron polytope is at least $15$ \cite[Theorem 5]{BBMP}, proven analytically via a discrete subgraph and its clique density.

\subsubsection{The cube in $\R^n$}

The cube $[-1/2,1/2]^n$ is the Vorono\"{i} cell of the coroot lattice for the root system $\RootC$, that is, for the cubic lattice $\Corootlattice(\RootC) = \Z^n$. In this case, the chromatic number is known to be $2^n$, see \cite{BBMP} for a counting argument that does not involve spectral bounds. We reprove this fact with the spectral bound by taking a $\weyl$--invariant measure, which is supported on the vertices and centers of edges, faces, etc. of $\Vor(\Corootlattice(\RootC))$.

\begin{proposition}\label{thm_cubeRnbound}
The spectral bound is sharp for $\chi_m (\R^n,\,\partial \Vor(\Corootlattice(\RootC))) = 2^n$.
\end{proposition}
\begin{proof}
The set of dominant weights $\weight\in\Weights^+$ of $\RootC$ with $\sprod{\weight,\highestroot^\vee}=1$ is $\{\fweight{1},\ldots,\fweight{n}\}$. 
We set $(2^n-1)\,c_i := \binom{n}{i}$. Then $c_1,\ldots,c_n \geq 0$ with $c_1 + \ldots + c_n = 1$ and the polynomial
\[
	\sum_{\sprod{\weight,\highestroot^\vee}=1} c_\weight \, T_\weight (z)
=	\sum_{i=1}^n c_i \, z_i .
\]
is an admissible choice for \Cref{eq_VoronoiLowerBound}. We show that it provides the optimal bound $2^n$. 
To do so, we rely on the formula for the fundamental weights from \Cref{equation_WeightsRootsC}, which gives us
\[
	(2^n-1)\,c_i\,\gencos{i}(u) = \sigma_i(\cos(2\pi u_1),\ldots,\cos(2\pi u_n)),
\]
where $\sigma_i$ is the $i$--th elementary symmetric function. When we substitute $z_i=\gencos{i}(u)$ for $u\in\R^n$, then
\[
		(2^n-1)\,\sum_{i=1}^n c_i \, z_i
=		\sum_{i=1}^n (2^n-1)\,c_i \, \gencos{i}(u)
=		\sum\limits_{i=1}^n \sigma_i(\mcos{u_1},\ldots,\mcos{u_n})
=		\prod\limits_{k=1}^n \underbrace{(1+\mcos{u_k})}_{\geq 0}-1
\geq	-1
\]
follows from Vieta's formula and equality holds for $u = 1/2 \, \fweight{j}$. Hence,
\[
		\chi_m (\R^n,\,\partial \Vor(\Corootlattice(\RootC)))
\geq	1-\frac{1}{\min\limits_{z\in\Image} \sum_{i=1}^n c_i \, z_i }
\geq	1-\frac{2^n-1}{- 1}
=		2^n.
\]
\end{proof}

\begin{remark}
For small $n$ $(2\leq n\leq 10)$, one can observe experimentally that the polynomial
\[
	p
:=	1 + \sum\limits_{i=1}^n \binom{n}{i} z_i \in \RX.
\]
is one of two linear factors in $\det(\posmat)$, $\posmat$ being the matrix from \emph{\Cref{thm_HermiteCharacterization}}, and $\Image$ is contained in the halfspace $\{z\in\R^n\,\vert\,p(z)\geq 0\}$. We conjecture that it is true in general. This would simplify the proof of \emph{\Cref{thm_cubeRnbound}}, giving it completely in terms of generalized Chebyshev polynomials and providing a new motivation for the choice of coefficients.
\end{remark}

\subsection{Discussion on the results}
\label{sec_chromaticdiscussion}

In addition to provide bounds on the chromatic number of the graphs that we consider, our method gives information on the discrete measures supported on lattice points up to scaling.

For example, in the case of the hexagon, even by increasing the number of support points, we did not get a discrete measure providing a better bound, see \Cref{A2HexagonTable}. Our experiments then suggest that the optimal measure supported on rational points is the one supported by two orbits: the vertices of the hexagon, with weight $1/3$, and the middle of the edges, with weight $2/3$. 

In the case of the cross--polytope from \Cref{sssec_CP}, we observe a different phenomenon: when increasing the number of possible support points, the optimal measure distribution does not appear to stabilize. It seems then reasonable to expect the bound to get better when increasing the number of points, even though it is hard to conjecture for an optimal discrete measure after our experiments, see \Cref{B3L1NormCoefficients}.
Moreover, we note that the larger the set of possible support points is, the higher we need to go in the order of the hierarchy to get a good bound. 
This can be explained by the fact that the weighted degrees of the involved Chebyshev polynomials get higher, making the semi--definite programs harder to solve. 

Even if we could prove that the spectral bound is sharp for several of our set avoiding graphs, sometimes the bounds that we obtain look far from the expected chromatic number of $\R^n$. 
This might happen for several reasons. 
First, when considering our discrete measures supported on lattices, we are always implicitly computing a bound for a discrete subgraph of $\R^n$, that might have a chromatic number smaller than $\R^n$. 
However, this is not the only reason: getting back to the hexagon, the measure supported on the vertices and the middles of edges gives a bound for a discrete graph.
However, it was proven in \cite{BBMP} that this graph has chromatic number $4$. 
In this case, it is likely that the spectral bound is exactly $25/7$, and does not give the chromatic number. 
Such a phenomenon was already observed in \cite{DMMV}, where, for the lattice $\mathrm{E}_7$, the optimal spectral bound was computed to be $10$, while the chromatic number of this lattice is $14$.
