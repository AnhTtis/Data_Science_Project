\section{Crystallographic symmetries}
\label{section_trigonometric}
\setcounter{equation}{0}

To rewrite the trigonometric optimization problem in \Cref{OptiProblemExpo} to a polynomial optimization problem, we require the lattice $\Weights$ to be full--dimensional and stable under some finite reflection group $\weyl$, that is, $\weyl \, \Weights =\Weights$. 
Then $\weyl$ must be the Weyl group of some crystallographic root system \cite[Chapter 9]{kane13} and $\Weights$ is the associated weight lattice. 
We need several facts from the theory of Lie algebras, root systems and lattices, which come from \cite{bourbaki456,humphreys12,conway1988a}. 
In particular, we need \Cref{theorem_BourbakiGenerators}, which states that any trigonometric polynomial with crystallographic symmetry can be written uniquely as a polynomial in fundamental invariants, also known as the generalized cosines. 
Subsequently, the feasible region of the so obtained polynomial optimization problem is the image of the fundamental invariants, a compact basic semi--algebraic set whose equations were given explicitely in \cite{chromaticissac22,TOrbits,TobiasThesis}.

\subsection{Root systems and Weyl groups}

Denote by $\sprod{\cdot , \cdot }$ the Euclidean scalar product. A subset $\Roots\subseteq \R^n$ is called a \textbf{root system} in $\R^n$, if the following conditions hold.
\begin{enumerate}
	\item[R1] $\Roots$ is finite, spans $\R^n$ and does not contain $0$.
	\item[R2] If $\rho, \tilde{\rho} \in \Roots$, then $\langle\tilde{\rho},\rho^\vee\rangle \in \Z$, where $\rho^\vee:=\frac{2\,\rho}{\langle\rho,\rho\rangle}$.
	\item[R3] If $\rho, \tilde{\rho} \in \Roots$, then $s_\rho(\tilde{\rho}) \in \Roots$, where $s_\rho$ is the  reflection defined by $ s_\rho(u) = u - \langle u,\rho^\vee\rangle \rho$ for $u \in  \R^n$.
\end{enumerate}
The elements of $\Roots$ are called \textbf{roots} and the \textbf{rank} of $\Roots$ is $\rank(\Roots):=\dim(\R^n)$. The elements $\rho^\vee$ are called the \textbf{coroots}. Furthermore, $\Roots$ is called \textbf{reduced}, if additionally the following condition holds.
\begin{enumerate}
	\item[R4] For $\rho \in \Roots$ and $c \in \R$, we have $c\rho \in \Roots$ if and only if $c = \pm 1$.$\phantom{\frac{1}{2}}$
\end{enumerate}

We assume that the ``reduced'' property R4 always holds when we speak of a ``root system''. 
Sometimes the ``crystallographic'' property R2 is emphasized as a seperate condition \cite{kane13}. For visualizations, see \Cref{example_2RootSys}.

\subsubsection{Weyl group and weights}

The \textbf{Weyl group} $\weyl$ of $\Roots$ is the group generated by the reflections $s_\roots$ for $\roots \in \Roots$. 
This is a finite subgroup of the orthogonal group on $\R^n$ with respect to the inner product $\sprod{\cdot , \cdot }$. 
The Weyl groups are the groups we consider in this article and now we define the lattices of interest.

A subset $\Base=\{\rho_1,\ldots,\rho_n\}\subseteq \Roots$ is called a \textbf{base}, if the following conditions hold.
\begin{enumerate}
	\item[B1] $\Base$ is a basis of $\R^n$.
	\item[B2] Every root $\rho \in \Roots$ can be written as $\rho = \alpha_1\,\rho_1 + \ldots +\alpha_n \, \rho_n$ or $\rho = -\alpha_1\,\rho_1 - \ldots -\alpha_n \,\rho_n$ for some $\alpha \in \N^n$.
\end{enumerate}
Every root system contains a base \cite[Chapitre VI, \S 1, Theorem 3]{bourbaki456}. A partial ordering $\succeq$ on $\R^n$ is defined by $u \succeq v$ if and only if $u-v  = \alpha_1\,\roots_1 + \ldots +\alpha_n\, \roots_n$ for some $\alpha \in \N^n$.

A \textbf{weight} of $\Roots$ is an element $\weight\in \R^n$, such that, for all $\roots\in\Roots$, we have $\sprod{\weight,\roots^\vee}  \in \Z$. The set of weights forms a lattice $\Weights$, called the \textbf{weight lattice}. By the condition R2, every root is a weight. For a base $\Base=\{\rho_1,\ldots,\rho_n\}$, the \textbf{fundamental weights} are the elements $\{ \fweight{1}, \ldots , \fweight{n}\}$, such that, for $1\leq i,j \leq n$, $\sprod{ \fweight{i}, \roots_j^\vee} = \delta_{i,j}$. The weight lattice is left invariant under the Weyl group, that is, $\weyl\Weights = \Weights$.

The \textbf{fundamental Weyl chamber} of $\weyl$ relative to $\Base$ is
\[
	\PC
:=	\{ u \in \R^n \,\vert\, \forall \, \roots\in\Base:\, \sprod{ u,\roots_i} > 0 \}.
\]
The closure $\overline{\PC}$ is a fundamental domain of $\weyl$ \cite[Chapitre V, \S 3, Th\'{e}or\`{e}me 2]{bourbaki456}. Hence, $\overline{\PC}$ contains exactly one element per $\weyl$--orbit and the weights in $\overline{\PC}$ are called \textbf{dominant}. We denote $\Weights^+ := \Weights\cap\overline{\PC}$.
 
\begin{proposition}\label{remark_PermutationOrb}
For $\weight\in\Weights^+$, there exists a unique $\conj{\weight}\in\Weights^+$ with $-\weight\in\weyl\conj{\weight}$. Furthermore, there exists a permutation $\sigma\in\mathfrak{S}_n$ of order at most $2$, such that, for all $1 \leq i \leq n$, we have $\conj{\fweight{i}} = \fweight{\sigma(i)}$.
\end{proposition}
\begin{proof}
Let $A$ be the longest element of $\weyl$ \cite[Chapitre VI, \S 1, Proposition 17 et Corollaire 3]{bourbaki456}. Then $A \PC = -\PC$ and so $\conj{\weight} = -A \, \weight\in\Weights^+$. We define the permutation $\sigma\in\mathfrak{S}_n$ via the property $\rho_i=-A\,\rho_{\sigma(i)}$ for $1\leq i\leq n$. Since $A$ is an involution and the inner product is $\weyl$--invariant, we obtain
\[
\conj{\fweight{i}}
=	-A\,\fweight{i}
=	\sum\limits_{j=1}^n \sprod{-A\,\fweight{i},\roots_{j}^\vee} \, \fweight{j}
=	\sum\limits_{j=1}^n \sprod{\fweight{i},-A\,\roots_{j}^\vee} \, \fweight{j}
=	\sum\limits_{j=1}^n \sprod{\fweight{i},\roots_{\sigma(j)}^\vee} \, \fweight{j}
=	\fweight{\sigma(i)}.
\]
\end{proof}

\subsubsection{The Vorono\"{i} cell}

The set of all coroots $\roots^\vee$ spans a lattice $\Corootlattice$ in $\R^n$, called the \textbf{coroot lattice}. This Abelian group acts on $\R^n$ by translation and is the dual lattice of the weight lattice, that is, $\Weights^*=\{u\in \R^n\,\vert\,\forall\,\weight\in\Weights:\,\sprod{\weight,u}\in\Z\}=\Corootlattice$.

Denote by $\norm{\cdot}$ the Euclidean norm. The \textbf{Vorono\"{i} cell} of $\Corootlattice$ is 
\[
\Vor(\Corootlattice)
:=	\{ u\in \R^n \,\vert\, \forall \lambda\in\Corootlattice:\,\norm{u}\leq \norm{u - \lambda} \}
\]
and tiles $\R^n$ by $\Corootlattice$--translation, that is,
\begin{equation}\label{eq_VoronoiTiles}
	\R^n
	=	\bigcup\limits_{\lambda\in\Corootlattice} (\Vor(\Corootlattice)+\lambda),
\end{equation}
where ``$+$'' denotes the Minkowski sum. The intersection of two distinct cells $\Vor(\Corootlattice)+\lambda$ and $\Vor(\Corootlattice)+\tilde{\lambda}$ is either empty or a common facet, that is a face of dimension $n-1$ \cite[Chapter 2, \S 1.2]{conway1988a}.

The \textbf{affine Weyl group} is the group generated by the reflections $s_{\roots,\ell}$ for $\roots\in\Roots$ and $\ell\in\Z$, where $s_{\roots,\ell}$ is defined via $s_{\roots,\ell}(u)=s_{\roots}(u)+\ell\,\roots^\vee$, see \cite[Chapitre VI, \S 2, D\'{e}finition 1]{bourbaki456}. It can also be seen as the semi--direct product $\weyl\ltimes \Corootlattice$ \cite[Chapitre VI, \S 2, Proposition 1]{bourbaki456}. We are interested in the chambers of this infinite reflection group, which are called \textbf{alcoves} to avoid confusion. In particular, the closure of any alcove is a fundamental domain for $\weyl\ltimes \Corootlattice$.

\begin{proposition}
\emph{\cite[Chapitre VI, \S2, Proposition 4]{bourbaki456} and \cite[Chapter 21, \S 3, Theorem 5]{conway1988a}} There is a unique alcove of $\weyl\ltimes \Corootlattice$ in $\PC$, which contains $0$ in its closure $\fundom$. We have $\Vor(\Corootlattice) = \weyl\,\fundom$.
\end{proposition}

The rest of this subsection is devoted to describe $\fundom$. Assume that $\R^n=V^{(1)}\oplus\ldots\oplus V^{(k)}$ is the direct sum of proper orthogonal subspaces and that, for each $1\leq i\leq k$, $\Roots^{(i)}$ is a root system in $V^{(i)}$. Then $\Roots:=\Roots^{(1)}\cup\ldots\cup\Roots^{(k)}$ is a root system in $\R^n$ and called the \textbf{direct sum} of the $\Roots^{(i)}$. If a root system is not the direct sum of at least two root systems, then it is called \textbf{irreducible}, see \cite[Chapitre VI, \S 1.2]{bourbaki456}.

The Weyl group $\weyl$ is the product of the Weyl groups corresponding to the irreducible components, see the discussion before \cite[Chapitre VI, \S 1, Proposition 5]{bourbaki456}. Furthermore, any alcove of the affine Weyl group is the product of alcoves corresponding to the irreducible components, see the discussion after \cite[Chapitre VI, \S 2, Proposition 2]{bourbaki456}. We are thus left to determine $\fundom$ for irreducible root systems. If $\Roots$ is irreducible with base $\Base$, then there is a unique positive root $\highestroot \in \Roots^+$, which is maximal with respect to the partial ordering $\succeq$ induced by $\Base$ \cite[Chapitre VI, \S 1, Proposition 25]{bourbaki456}. We call $\highestroot$ the \textbf{highest root}.

\begin{proposition}\label{prop_FundomAffineWeyl}
\emph{\cite[Chapitre VI, \S 2, Proposition 5 et Corollaire]{bourbaki456}} Let $\Roots$ be an irreducible root system and $\Base = \{\roots_1, \ldots , \roots_n\}$ be a base, such that $\highestroot = \alpha_1 \, \roots_1^\vee + \ldots + \alpha_n \, \roots_n^\vee$ is the highest root of $\Roots$ for some $\alpha\in\R^n$. Then
\[
	\fundom
=	\{ u \in \R^n \,\vert\, \forall \, 1\leq i \leq n: \, \sprod{u,\roots_i} \geq 0 \mbox{ and } \sprod{u,\highestroot} \leq 1 \}
\]
is a fundamental domain for $\weyl\ltimes \Corootlattice$. Furthermore, for $1\leq i\leq n$, we have $\alpha_i > 0$ and
\[
	\fundom
=	\mathrm{ConvHull} \left(0,\,\frac{\fweight{1}}{\alpha_1},\,\ldots,\,\frac{\fweight{n}}{\alpha_n}\right).
\]
\end{proposition}

In particular, if $\Roots$ is irreducible, then any closed alcove of the affine Weyl group is a simplex.

Every root system can be uniquely decomposed into irreducible components \cite[Chapitre VI, \S 1, Proposition 6]{bourbaki456} and there are only finitely many cases \cite[Chapitre VI, \S 4, Th\'{e}or\`{e}me 3]{bourbaki456} denoted by $\RootA$, $\RootB$, $\RootC$ $(n\geq 2)$, $\RootD$ $(n\geq 4)$, $\RootE[6,7,8]$, $\RootF[4]$ and $\RootG[2]$. Throughout this article, we shall focus on the four infinite families $\RootA$, $\RootB$, $\RootC$, $\RootD$ and the special case $\RootG[2]$. For those root systems, the base, fundamental weights and Weyl group are recalled in \Cref{Appendix_IrredRootSys}.

\begin{example}\label{example_2RootSys}
In the $2$--dimensional case, we can consider the following irreducible root systems.

\begin{minipage}{0.45\textwidth}
	\begin{figure}[H]
		\begin{minipage}{0.4\textwidth}
			\begin{flushright}
				\begin{overpic}[width=\textwidth,grid=false,tics=10]{A2fundom}
					\put (20,95) {\large $\displaystyle \rho_2$}
					\put (95,55) {\large $\displaystyle \rho_1$}
					\put (75,66) {\large $\displaystyle \fweight{1}$}
					\put (50,81) {\large $\displaystyle \fweight{2}$}
				\end{overpic}
			\end{flushright}
		\end{minipage} \hfill
		\begin{minipage}{0.5\textwidth}
			$\weyl(\RootA[2]) \cong \mathfrak{S}_3$\\
			$\fweight{1}=[ 2,-1,-1]^t /3$\\
			$\fweight{2}=[ 1, 1,-2]^t /3$\\
			$\rho_{1}=[1,-1,0]^t =\rho_{1}^\vee$\\
			$\rho_{2}=[0,1,-1]^t =\rho_{2}^\vee$\\
			$\highestroot = \rho_{1}^\vee + \rho_{2}^\vee$
		\end{minipage}
		\centering
		\caption{The root system $\RootA[2]$ in $\R^3/\langle [1,1,1]^t \rangle$.}\label{figA2}
		\label{example_rootsystemA2}
	\end{figure}
\end{minipage}\hfill
\begin{minipage}{0.45\textwidth}
	\begin{figure}[H]
		\begin{minipage}{0.4\textwidth}
			\begin{flushright}
				\begin{overpic}[width=\textwidth,,tics=10]{B2fundom}
					\put (53,100) {\large $\displaystyle \rho_2$}
					\put (99,  8) {\large $\displaystyle \rho_1$}
					\put (69, 83) {\large $\displaystyle \fweight{2}$}
					\put (95, 55) {\large $\displaystyle \fweight{1}$}
				\end{overpic}
			\end{flushright}
		\end{minipage} \hfill
		\begin{minipage}{0.5\textwidth}
			$\weyl(\RootB[2]) \cong \mathfrak{S}_2\ltimes\{\pm 1\}^2$\\
			$\fweight{1}=[ 1,0]^t$\\
			$\fweight{2}=[ 1,1]^t /2$\\
			$\rho_{1}=[1,-1]^t =\rho_{1}^\vee$\\
			$\rho_{2}=[0,1]^t =\rho_{2}^\vee/2$\\
			$\highestroot=\rho_{1}^\vee+\rho_{2}^\vee$
		\end{minipage}
		\centering
		\caption{The root system $\RootB[2]$ in $\R^2$.}\label{example_rootsystemB2}\label{figB2}
	\end{figure}
\end{minipage}

~\vspace{.2cm}

\begin{minipage}{0.45\textwidth}
	\begin{figure}[H]
		\begin{minipage}{0.4\textwidth}
			\begin{flushright}
				\begin{overpic}[width=\textwidth,grid=false,tics=10]{G2fundom}
					\put (15,75) {\large $\displaystyle \rho_2$}
					\put (80,55) {\large $\displaystyle \rho_1$}
					\put (52,95) {\large $\displaystyle \fweight{2}$}
					\put (65,75) {\large $\displaystyle \fweight{1}$}
				\end{overpic}
			\end{flushright}
		\end{minipage} \hfill
		\begin{minipage}{0.5\textwidth}
			$\weyl(\RootG[2])\cong\mathfrak{S}_3\ltimes \{\pm 1\}$\\
			$\fweight{1}=[0,-1,1,]^t$\\
			$\fweight{2}=[-1,-1,2]^t$\\
			$\rho_{1}=[1,-1,0]^t = \rho_{1}^\vee$\\
			$\rho_{2}=[-2,1,1]^t = 3\,\rho_{1}^\vee $\\
			$\highestroot = 3\,\rho_{1}^\vee + 6\,\rho_{2}^\vee$
		\end{minipage}
		\centering
		\caption{The root system $\RootG[2]$ in $\R^3/\langle [1,1,1]^t \rangle$.}\label{example_rootsystemG2}
	\end{figure}
\end{minipage}\hfill
\begin{minipage}{0.45\textwidth}
	\begin{figure}[H]
		\begin{minipage}{0.4\textwidth}
			\begin{flushright}
				\begin{overpic}[width=\textwidth,,tics=10]{C2fundom}
					\put (53, 95) {\large $\displaystyle \rho_2$}
					\put (77, 20) {\large $\displaystyle \rho_1$}
					\put (69, 80) {\large $\displaystyle \fweight{2}$}
					\put (77, 54) {\large $\displaystyle \fweight{1}$}
				\end{overpic}
			\end{flushright}
		\end{minipage} \hfill
		\begin{minipage}{0.5\textwidth}
			$\weyl(\RootC[2]) \cong \mathfrak{S}_2\ltimes\{\pm 1\}^2$\\
			$\fweight{1}=[ 1,0]^t $\\
			$\fweight{2}=[ 1,1]^t $\\
			$\rho_{1}=[1,-1]^t =\rho_{1}^\vee$\\
			$\rho_{2}=[0,2]^t = 2\,\rho_{2}^\vee$\\
			$\highestroot = 2\,\rho_{1}^\vee + 2\,\rho_{2}^\vee$
		\end{minipage}
		\centering
		\caption{The root system $\RootC[2]$ in $\R^2$.}\label{example_rootsystemC2}\label{figC2}
	\end{figure}
\end{minipage}

~\\
~\\
Here, the roots are depicted in green, the base in red and the fundamental weights in blue. The Vorono\"{i} cell of the coroot lattice $\Corootlattice$ is the gray shaded region, we have two squares $(\RootC[2]$ and $\RootB[2])$ and two hexagons $(\RootA[2]$ and $\RootG[2])$. The fundamental domain of the affine Weyl group is the blue shaded simplex.
\end{example}

\subsection{Trigonometric polynomials with Weyl group symmerty}

From now on, $\Roots$ is a root system in $\R^n$ with Weyl group $\weyl$, weight lattice $\Weights = \Z\,\fweight{1} \oplus \ldots \oplus \Z\,\fweight{n}$ and coroot lattice $\Corootlattice=\Weights^*$. For $\weight \in \Weights$, we define the function
\[
\mathfrak{e}^{\weight} :
\begin{array}[t]{ccl}
\R^n & \to & \C, \\
u & \mapsto & \mexp{\weight,u}.
\end{array}
\]
A $\C$--linear combination of these functions is a \textbf{trigonometric polynomial}. The set of all trigonometric polynomials forms an algebra  that we denote by $\C[\Weights]$.

The set $\{\mathfrak{e}^{\weight}\,\vert\,\weight\in\Weights\}$ is closed under multiplication $\mathfrak{e}^{\weight}\,\mathfrak{e}^{\tilde{\weight}} = \mathfrak{e}^{\weight + \tilde{\weight}}$ and thus a group with neutral element $\mathfrak{e}^0$ and inverse $(\mathfrak{e}^{\weight})^{-1}=\mathfrak{e}^{-\weight}$. Since $\Weights$ is the free $\Z$--module generated by the $\fweight{i}$, $\C[\Weights]$ is generated by $\{\mathfrak{e}^{\pm\fweight{1}},\ldots,\mathfrak{e}^{\pm\fweight{n}}\}$.

Since the coroot lattice $\Corootlattice$ is the dual lattice of $\Weights$, any element $f\in\C[\Weights]$ is $\Corootlattice$--periodic, that is, for all $u\in \R^n$ and $\lambda\in\Corootlattice$, we have $f(u+\lambda) = f(u)$.

\subsubsection{Generalized cosines and Chebyshev polynomials}

The Weyl group $\weyl$ acts linearly on $\C[\Weights]$ by the action described on its basis as
\[
	\cdot: \begin{array}[t]{ccl}
	\weyl	\times	\C[\Weights]	&	\to		&	\C[\Weights],\\
	(A,\mathfrak{e}^{\weight})		&	\mapsto	&	\mathfrak{e}^{A\weight}.
	\end{array}
\]
A trigonometric polynomial $f\in\C[\Weights]$ is called \textbf{$\weyl$--invariant}, if, for all $A\in\weyl$, we have $A \cdot f = f$. The \textbf{generalized cosine function} associated to $\weight\in\Weights$ is the $\weyl$--invariant trigonometric polynomial
\begin{equation}\label{eq_gencos}
\gencos{\weight} :
\begin{array}[t]{ccl}
\R^n & \to & \C,\\
u & \mapsto & \dfrac{1}{\nops{\weyl}} \sum\limits_{A\in\weyl} \mathfrak{e}^{A \weight}(u) .
\end{array}
\end{equation}

\begin{theorem}\label{theorem_BourbakiGenerators}
\emph{\cite[Chapitre VI, \S 3, Th\'eor\`eme 1]{bourbaki456}} The following statements hold.
\begin{enumerate}
\item The $\gencos{\fweight{1}}, \ldots, \gencos{\fweight{n}}$ are algebraically independent.
\item The set of $\weyl$--invariants is the polynomial $\C$--algebra generated by the $\gencos{\fweight{1}},\ldots,\gencos{\fweight{n}}$, that is, 
\[
	\C[\Weights]^\weyl=\C[\gencos{\fweight{1}},\ldots,\gencos{\fweight{n}}].
\]
\end{enumerate}
\end{theorem}

The above \Cref{theorem_BourbakiGenerators} states that, for every $f\in\C[\Weights]^\weyl$, there exists a unique polynomial $g\in\CX:=\C[z_1,\ldots,z_n]$ with the property $f(u) = g(\gencos{}(u))$, where $\gencos{}$ is the function
\[
\gencos{} :
\begin{array}[t]{ccl}
	\R^n & \to & \C^n,\\
	u & \mapsto & \left(\gencos{\fweight{1}}(u),\ldots,\gencos{\fweight{n}}(u)\right) .
\end{array}
\]
This property is exclusive for Weyl groups \cite{farkas86}.

\begin{definition}
The \textbf{generalized Chebyshev polynomials of the first kind} associated to $\weight\in\Weights$ is the unique $T_\weight\in\CX$, such that $T_\weight(\gencos{}(u)) = \gencos{\weight}(u)$. 
\end{definition}

The coefficients of the $T_\weight$ are real. We have $T_0 = 1$, $T_{\fweight{i}}=z_i$ and, for $\weight,\nu\in\Weights$,
\begin{equation}\label{eq_TPolyRecurrence}
\vert\weyl \vert\,T_{\weight}\,T_{\nu}=\sum\limits_{A \in\weyl} T_{\weight + A \nu}.
\end{equation}
The set $\{T_\weight\,\vert\,\weight\in\Weights^+\}$ forms a vector space basis of $\CX$ \cite{lorenz06}.

This definition is a generalization of the univariate Chebyshev polynomials of the first kind $T_\ell(\cos(u)) = \cos(\ell \, u)$ with $\ell \in \Z$, which correspond to the root system $\RootA[1]$. 

\subsubsection{Real cosines and Chebyshev polynomials}

For our approach in \Cref{section_optimization}, we need the generalized Chebyshev polynomials to be defined on a real domain. This is always true for $-I_n\in\weyl$ and what follows is only necessary for $-I_n\notin\weyl$. Let $\weight,\conj{\weight}\in\Weights^+$ with $-\weight\in\weyl\conj{\weight}$. The \textbf{real generalized cosines} associated to the pair $(\weight,\conj{\weight})$ are
\[
	\Re (\gencos{\weight})
=	\frac{\gencos{\weight}+\gencos{\conj{\weight}}}{2}
	\tbox{and}
	\Im (\gencos{\weight})
=	\frac{\gencos{\weight}-\gencos{\conj{\weight}}}{2\mathrm{i}}.
\]
By construction, those are real--valued $\weyl$--invariant trigonometric polynomials. We are interested in the pairs $(\weight,\conj{\weight})$ with $\weight=\fweight{i}$ a fundamental weight. Let $\sigma\in \mathfrak{S}_n$ be the permutation from \Cref{remark_PermutationOrb}. Then $\conj{\weight} = \fweight{\sigma(i)}$ is also a fundamental weight and we define the function
\begin{equation}\label{eq_realgencos}
\gencos{\R}:
\begin{array}[t]{ccl}
\R^n		& \to 		&	\R^{n}, \\
u 		& \mapsto 	&	\left( \gencos{\fweight{1},\R}( u),\ldots, \gencos{\fweight{n},\R}( u)\right),
\end{array}
\end{equation}
where $\gencos{\fweight{i},\R}:=\gencos{\fweight{i}}$ for $i=\sigma(i)$ and $\gencos{\fweight{i},\R} := \Re (\gencos{\fweight{i}}),\gencos{\fweight{\sigma(i)},\R} := \Im (\gencos{\fweight{i}})$ for $i<\sigma(i)$.

\begin{proposition}\label{LemmaComplexConjugatePoly}
Let $\weight,\conj{\weight} \in \Weights$ with $-\weight\in\weyl\conj{\weight}$. Then there exist unique $\TT_\weight , \TT_{\conj{\weight}} \in \RX$, such that
\[
	T_{{\weight}}	(\gencos{}(u))
=	\TT_{{\weight}}	(\gencos{\R}(u)) + \mathrm{i} \, \TT_{\conj{\weight}}	(\gencos{\R}(u))
\tbox{and}
	T_{\conj{\weight}}	(\gencos{}(u))
=	\TT_{{\weight}}	(\gencos{\R}(u)) - \mathrm{i}\, \TT_{\conj{\weight}}	(\gencos{\R}(u)) .
\]
\end{proposition}
\begin{proof}
Note that
\[
	(T_\weight + T_{\conj{\weight}}) (\gencos{}(u))
=	\frac{1}{\nops{\weyl \weight}} \sum\limits_{\tilde{\weight} \in \weyl \weight} \mathfrak{e}^{\tilde{\weight}}(u) + \mathfrak{e}^{-\tilde{\weight}}(u)
\]
is invariant under both $\weyl$ and $\{\pm I_n\}$. Let $\sigma\in \mathfrak{S}_n$ be the permutation from \Cref{remark_PermutationOrb}. Then the $\C$--algebra $(\C[\Weights]^{\weyl})^{\{\pm I_n\}}$ is generated by the $ \gencos{\fweight{i}}+\gencos{\fweight{\sigma(i)}} $ with $ 1 \leq i \leq \sigma(i) \leq n $. Thus, $(T_\weight + T_{\conj{\weight}})(\gencos{}(u))/2$ can be written as a polynomial $\TT_\weight$ in $\gencos{\R}(u)$. Similarly,
\[
	(T_\weight - T_{\conj{\weight}}) (\gencos{}(u))
=	\frac{1}{\nops{\weyl\weight}} \sum\limits_{\tilde{\weight} \in \weyl\weight} \mathfrak{e}^{\tilde{\weight}}(u) - \mathfrak{e}^{-\tilde{\weight}}(u)
\]
is invariant under $\weyl$, but anti--invariant under $\{\pm I_n\}$. The elements of $\C[\Weights]^{\weyl}$, which are anti--invariant under $\{\pm I_n\}$, are, as an $\C$--algebra, generated by the $ \gencos{\fweight{i}}-\gencos{\fweight{\sigma(i)}}$ with $ 1\leq \sigma(i) < i \leq n $. Hence, $(T_\weight - T_{\conj{\weight}})(\gencos{}(u))/(2\mathrm{i})$ can be written as a polynomial $\TT_{\conj{\weight}}$ in $\gencos{\R}(u)$. As polynomials, $\TT_{\weight}$ and $\TT_{\conj{\weight}}$ are analytical functions and $\Image_\R$ has nonempty interior. Hence, they are unique.
\end{proof}

\begin{convention}\label{convention_real}
From now on, we will write $T_\weight$ and $\gencos{}$ for $\TT_\weight$ and $\gencos{\R}$, even if $-I_n\notin\weyl$. As we have shown above, the reformulation follows from a permutation $\sigma$ and a substitution $z_i \mapsto z_i \pm \mathrm{i}\,z_{\sigma(i)}$. For our implementation, it is important to remember this caveat, but for the article itself, we shall simplify the notation.
\end{convention}

\subsection{The image of the generalized cosines as a basic semi--algebraic set}

We call $\Image:=\gencos{}(\R^n)$ the \textbf{image of the generalized cosines}. If $\fundom$ is a fundamental domain for the affine Weyl group $\weyl\ltimes\Corootlattice$, then $\Image=\gencos{}(\fundom)$ due to the $\weyl$--invariance and $\Corootlattice$--periodicity. In particular, $\Image$ is compact. With \Cref{convention_real}, $\Image$ is a real set and contained in the cube $[-1,1]^n$.

For the purpose of optimization, we need a polynomial description of $\Image$ as a basic semi--algebraic set. Recently, a closed formula was given via a polynomial matrix inequality. This formula is available in the standard monomial basis $z$ \cite{chromaticissac22,TOrbits}, and in the basis of generalized Chebyshev polynomials $T_{\weight}$ \cite{TobiasThesis}.

\begin{theorem}\label{thm_HermiteCharacterization}
\emph{\cite[Theorem 2.19]{TobiasThesis}} Let $\Roots$ be a root system of type $\RootA$, $\RootB$, $\RootC$, $\RootD$ or $\RootG[n-1]$ and define the symmetric matrix polynomial $\posmat\in\RX^{n\times n}$ via
\begin{align*}
	2^{i+j}\,\posmat(z)_{ij}
=&	- T_{(i+j)\, \fweight{1}}(z) + \sum\limits_{\ell=1}^{\lceil (i+j)/2 \rceil -1} \left( 4 \binom{i+j-2}{\ell-1} - \binom{i+j}{\ell} \right) T_{(i+j-2\,\ell)\, \fweight{1}}(z) \\
&+	\frac{1}{2} \begin{cases}
	4\binom{i+j-2}{(i+j)/2-1} - 	\binom{i+j}{(i+j)/2}	,&	\tbox{if} i+j \tbox{is even}	\\
	0														,&	\tbox{if} i+j \tbox{is odd}
	\end{cases}.
\end{align*}
Then $\Image = \{z\in \R^n \,\vert \, \posmat(z)\succeq 0 \}$.
\end{theorem}

The matrix polynomial $\posmat\in\RX^{n\times n}$ from \Cref{thm_HermiteCharacterization} follows the pattern
\[
\begin{bmatrix}
	\frac{T_{0}-T_{2\,\fweight{1}}}{4}& 
	\frac{T_{\fweight{1}} -T_{3\,\fweight{1}}}{8}& 
	\frac{T_{0}- T_{4\,\fweight{1}}}{16}&
	\frac{2\,T_{\fweight{1}}- T_{3\,\fweight{1}} - T_{5\,\fweight{1}}}{32}&
	\cdots\\
	
	\frac{T_{\fweight{1}} -T_{3\,\fweight{1}}}{8}& 
	\frac{T_{0}- T_{4\,\fweight{1}}}{16}&
	\frac{2\, T_{\fweight{1}}- T_{3\,\fweight{1}} - T_{5\,\fweight{1}}}{32}&
	\frac{2\, T_{0} +  T_{2\,\fweight{1}} - 2\, T_{4\,\fweight{1}} -  T_{6\,\fweight{1}}}{64}&
	\cdots\\
	
	\frac{T_{0}- T_{4\,\fweight{1}}}{16}& 
	\frac{2\,T_{\fweight{1}}- T_{3\,\fweight{1}} - T_{5\,\fweight{1}}}{32}& 
	\frac{2 \,T_{0} +  T_{2\,\fweight{1}}-2\, T_{4\,\fweight{1}} -  T_{6\,\fweight{1}}}{64}&
	\frac{5 \,T_{\fweight{1}} - T_{3\,\fweight{1}} - 3 \,T_{5\,\fweight{1}} - T_{7\,\fweight{1}}}{128}&
	\cdots\\
	
	\frac{2\,T_{\fweight{1}}- T_{3\,\fweight{1}} - T_{5\,\fweight{1}}}{32}&
	\frac{2 \,T_{0} +  T_{2\,\fweight{1}}-2\, T_{4\,\fweight{1}} -  T_{6\,\fweight{1}}}{64}&
	\frac{5 \,T_{\fweight{1}} - T_{3\,\fweight{1}} - 3\, T_{5\,\fweight{1}} - T_{7\,\fweight{1}}}{128}&
	\frac{5 \,T_{0} + 4 \,T_{2\,\fweight{1}} - 4 \,T_{4\,\fweight{1}} - 4\,T_{6\,\fweight{1}} - T_{8\,\fweight{1}}}{256}&
	\cdots\\
	
	\vdots & \vdots & \vdots & \vdots & \ddots
\end{bmatrix}.
\]
\begin{figure}[H]
	\begin{center}
		\begin{subfigure}{.2\textwidth}
			\centering
			\includegraphics[width=\textwidth]{cosinoidA2.png}
			\caption{$\RootA[2]$}
		\end{subfigure}\quad
		\begin{subfigure}{.2\textwidth}
			\centering
			\includegraphics[width=\textwidth]{cosinoidC2.png}
			\caption{$\RootC[2]$}
		\end{subfigure}\quad
		\begin{subfigure}{.2\textwidth}
			\centering
			\includegraphics[width=\textwidth]{cosinoidB2.png}
			\caption{$\RootB[2]$}
		\end{subfigure}\quad
		\begin{subfigure}{.2\textwidth}
			\centering
			\includegraphics[width=\textwidth]{cosinoidG2.png}
			\caption{$\RootG[2]$}
		\end{subfigure}\quad
		\begin{subfigure}{.3\textwidth}
			\centering
			\includegraphics[width=\textwidth]{A3Deltoid.png}
			\caption{$\RootA[3]$}
		\end{subfigure}\quad
		\begin{subfigure}{.3\textwidth}
			\centering
			\includegraphics[width=\textwidth]{C3Deltoid.png}
			\caption{$\RootC[3]$}
		\end{subfigure}\quad
		\begin{subfigure}{.3\textwidth}
			\centering
			\includegraphics[width=\textwidth]{B3Deltoid.png}
			\caption{$\RootB[3]$}
		\end{subfigure}
		\caption{The image of the generalized cosines for the irreducible root systems of rank $2$ and $3$.}
		\label{fig_OrbitSpaceIntro}
	\end{center}
\end{figure}

\begin{remark}~
\begin{enumerate}
\item If we are in one of the special cases $\RootE[6,7,8]$ or $\RootF[4]$, then such a polynomial description of $\Image$ can also be obtained with \emph{\cite[\S 4]{procesischwarz85}}. In this case, one obtains a Gram matrix of differentials and has to rewrite the entries in the coordinates $z$ of $\Image$.

\item The root system may not be irreducible, that is, $\Roots=\Roots^{(1)} \cup \ldots \cup \Roots^{(k)}$ for some $k\in \N$.
Hence, we can write the fundamental domain of the affine Weyl group as $\fundom = \fundom^{(1)} \times \ldots \times \fundom^{(k)}$ and thus $\Image=\gencos{\fundom}$ is the positivity locus of a block--diagonal matrix polynomial
\[
\posmat(z^{(1)},\ldots,z^{(k)})
=	\diag(\posmat^{(1)}(z^{(1)}),\ldots,\posmat^{(k)}(z^{(k)})),
\]
where the $\posmat^{(i)}$ are matrix polynomials corresponding to the irreducible $\Roots^{(i)}$.

As an example, take $k$ orthogonal copies of $\RootA[1]$.
Then $\Image=[-1,1]^k$ is the positivity locus of the matrix polynomial $\posmat=\diag(1-z_1^2,\ldots,1-z_k^2)$.
\end{enumerate}
\end{remark}

\subsection{Optimizing trigonometric polynomials with crystallographic symmetry}

We now address the trigonometric optimization problem from \Cref{OptiProblemExpo}. With the theory that was presented in the previous subsections, we can rewrite the objective function uniquely in terms of generalized Chebyshev polynomials using \Cref{theorem_BourbakiGenerators}. Indeed, with the generalized cosines from \Cref{eq_gencos} we can write any $f\in\C[\Weights]^\weyl$ uniquely as
\[
f
=	\sum\limits_{\weight\in S} c_\weight\,\gencos{\weight}
\]
for some finite set $S\subseteq\Weights^+$ of dominant weights. If $c_{\weight} = \overline{c_{\conj{\weight}}} \in \R$ whenever $-\weight\in\weyl\conj{\weight}$, then $f$ takes only real values and
\begin{equation}\label{coro_OptiProbelmRewrite}
	f^*
:=	\min\limits_{u \in \R^n} f(u)
=	\min\limits_{z\in\Image} \sum\limits_{\weight\in S} c_\weight \, T_\weight(z)
\end{equation}
is the global minimum of $f$ on $\R^n$. This transforms the region of optimization from $\R^n$ into the image $\Image$ of the generalized cosines. Thanks to \Cref{thm_HermiteCharacterization}, we can describe the latter explicitly as a compact basic semi--algebraic set with the Chebyshev basis. This makes it possible to solve the problem numerically with techniques from classical polynomial optimization, which is subject to \Cref{section_optimization}.

\begin{example}\label{example_A2PolyRewrite}
The symmetric group $\mathfrak{S}_3$ acts on $\R^3/\langle [1,1,1]^t\rangle$ by permutation of coordinates and leaves the lattice $\Weights:=\Z\,\fweight{1} + \Z\,\fweight{2}:=\Z\,[0,-1,-1]^t + \Z\,[-1,-1,2]^t$ invariant. This is the weight lattice of the root system $\RootG[2]$ with Weyl group $\weyl := \mathfrak{S}_3 \times \{\pm1\}$. We consider the $\weyl$--invariant trigonometric polynomial
\begin{align*}
	f(u)
:=&	\,\gencos{\textcolor{red}{2\,\fweight{1}}}(u)+2\,\gencos{\textcolor{blue}{\fweight{2}}}(u)\\
=&	\,(
	\mcos{\sprod{2\,\fweight{1},u}} + 
	\mcos{\sprod{2\,\fweight{1} - 2\,\fweight{2},u}} + 
	\mcos{\sprod{4\,\fweight{1} - 2\,\fweight{2},u}} \\
&	\,
	+ 2\,\mcos{\sprod{\fweight{2},u}} + 
	2\,\mcos{\sprod{3\,\fweight{1} - \fweight{2},u}} + 
	2\,\mcos{\sprod{3\,\fweight{1} - 2\,\fweight{2},u}} 
	)/3
\end{align*}
with $u = (u_1,u_2,-u_1-u_2)\in \R^3/\langle [1,1,1]^t\rangle$. In the coordinates $z=\gencos{}(u)=(\gencos{\fweight{1}}(u),\gencos{\fweight{2}}(u))\in\Image$, we have
\[
	f(z)
=	T_{\textcolor{red}{2\,\fweight{1}}}(z)+2\,T_{\textcolor{blue}{\fweight{2}}}(z)
=	(6\,z_1^2 - 2\,z_1 - 2\,z_2 - 1) + 2 \, (z_2)
=	6\,z_1^2 - 2\,z_1 - 1.
\]
This univariate polynomial is minimal in $z_1=1/6$ and $z=(1/6,z_2)\in\Image$ if and only if $z_2\in [-11/24,-1/3]$. Hence, the minimum of $f$ is
\[
	f^*
=	\min\limits_{u\in\R^2} f(u)
=	\min_{z\in \Image} 6\,z_1^2 - 2\,z_1 - 1
=	-\frac{7}{6}.
\]
\begin{figure}[H]
\begin{center}
	\begin{overpic}[height=4cm,grid=false,tics=10]{A2Level2Coeff.png}
	\put ( 83, 87) {\large $\displaystyle \textcolor{red}{2\,\fweight{1}}$}
	\put ( 53, 97) {\large $\displaystyle \textcolor{blue}{\fweight{2}}$}
	\end{overpic}
	\hspace{1cm}
	\includegraphics[height=4cm]{ToeplitzCirclesG2.png}
	\hspace{1cm}
	\includegraphics[height=4cm]{G2DeltoidMin.png}
\caption{The support of $f$ as a trigonometric polynomial on the left consists of the $\weyl$--orbits of $\textcolor{red}{2\,\fweight{1}}$ and $\textcolor{blue}{\fweight{2}}$. The graph of this $\weyl$--invariant periodic function is depicted in the middle. The image of the generalized cosines \textcolor{blue}{$\Image$} on the right is the new feasible region of the polynomial optimization problem and the set of minimizers for $f$ is \textcolor{red}{$\{1/6\} \times [-11/24,-1/3]$}.}
\label{A2level2}
\end{center}
\end{figure}
\end{example}