%           ---> !!! FIGURES IN THE MAIN FILE FOR EASY ORDERING !!! <----
%Multi-core or modular quantum architectures %are a new technology~\cite{Monroe_2014, https://doi.org/10.48550/arxiv.2201.08825, https://doi.org/10.48550/arxiv.2201.08861,https://doi.org/10.48550/arxiv.2210.10921} proposed to overcome the scalability problem encountered in current single-core or monolithic processors. This new technology
%introduce new challenges, such as the movement of qubits between cores. As stated in the last section, these movements are costly and should be avoided to the extent possible. Optimal compilation or mapping is required to reduce these inter-core movements. So far, only one mapping technique has been proposed for multi-core or modular quantum computing architectures, presented 

In~\cite{10.1145/3387902.3392617} a technique for mapping quantum algorithms on multi-core architectures based on graph partitioning has been proposed. The goal is to place qubits in the different quantum processors such that inter-core movements are minimized. An illustrative example of this mapping technique is shown in Figure~\ref{fig::mapp_exmple}. %For a more detailed explanation we refer the reader to ~\cite{10.1145/3387902.3392617}.  
Note that in the proposal presented in~\cite{10.1145/3387902.3392617} the following assumptions that simplify the quantum circuit mapping problem are made: i) all-to-all connectivity between cores and among physical qubits within the cores. This means that there is no need for qubit routing inside the core, nor for optimal initial placement. Regarding inter-core routing and qubit placement at the core level, all qubits are at a one-hop distance, and therefore when two qubits have to interact and cannot be placed from the beginning on the same quantum core it is enough to place one of them on any other core; ii) SWAP operations (i.e. exchange of quantum states) are used for inter-core communication that makes simpler the management of resources as it not required to check if there is space (i.e. qubits that do not have any information) for exchanging qubits between cores; iii) only a fixed modular architecture is considered consisting of 10 cores with 10 qubits per core, which is not enough for analyzing the performance of the quantum circuit mapper. In the following section, different architectures will be used to further analyze this proposed mapping procedure.





%The proposed mapping technique works as follows. First, the quantum circuit is partitioned in time. Each resulting time-slice consist of one or several CNOT gates and the corresponding qubit interaction graph in which physical qubits correspond to the nodes of the graph and edges are the interactions between qubits (i.e. two-qubit gates).  and the cores are the partitions.

%Their approach bases the initial placement procedure on a static partitioning graph problem where the total number of partitions is fixed and limits the total number of qubits in each partition. However, only algorithms~\cite{6771089, Fiduccia1982ALH} that find approximate solutions to this problem were available since heuristic solvers~\cite{HARIF2022789} cannot guarantee the fixed size of the partitions. 

%For the routing problem, posed as a qubit assignment problem, instead of finding a global assignment of the qubits to the cores, they used a partitioning over time graph problem, the relaxed Overall Extreme Exchange algorithm. The rOEE algorithm is a relaxed version of the Overall Extreme Exchange algorithm~\cite{PARK1995899}. For each time slice (i.e. set of two-qubit gates that can be performed in parallel), there will be a graph representing the interactions between qubits at that specific time in the quantum circuit. Edges will have a cost defined by a lookahead weight, which will consider future qubits interactions. For each time slice, the rOEE finds a valid assignment (i.e. qubits that interact in that time slice are placed in the same partition). When the algorithm finishes, it returns a path consisting of a sequence of valid assignments through the quantum algorithm. An example is presented in Figure~\ref{fig::mapp_exmple}.

%This mapping technique has only been simulated on a specific modular architecture (10 cores and 10 qubits per core); in the following section, different architectures will be used to analyze this proposed mapping procedure in depth.

