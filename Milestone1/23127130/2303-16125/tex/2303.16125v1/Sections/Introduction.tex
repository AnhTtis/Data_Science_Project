%           ---> !!! FIGURES IN THE MAIN FILE FOR EASY ORDERING !!! <----

\begin{comment}
\textcolor{red}{TODO: 1) Review the concept of distributed quantum computing. 2) Acknowledgement section: QUADRATURE? QIA?. 3) read again the paper and check that everything is clear and correct.}
\end{comment}

Quantum computers are a revolutionary technology that can outperform classical computing in areas such as scientific simulation~\cite{https://doi.org/10.48550/arxiv.1903.10541}, cryptography~\cite{Mavroeidis_2018}, machine learning~\cite{9274431}, search or optimization~\cite{Montanaro_2016}, thanks to the use of quantum mechanics phenomena like superposition and entanglement. Current quantum computing technologies, commonly called NISQ (Noisy Intermediate-Scale Quantum)~\cite{Preskill_2018} devices, are limited in the number of qubits and prone to errors. The most advanced quantum processors consist of a few tens of noisy qubits (e.g. IBM's 433-qubit Osprey processor \cite{Osprey}), meaning that their state can be easily modified due to the interaction with the external environment (decoherence) and that quantum gates and measurements are implemented with imperfect operations. Algorithms for NISQ devices have been developed to leverage their scarce and noisy resources such as Quantum Approximate Optimization Algorithm (QAOA) or Variational Quantum Eigensolver (VQE)~\cite{Bharti_2022}. However, to build a universal fault-tolerant quantum computer and achieve the full computational power these machines will provide, it is necessary to scale them up in a way that the number of qubits is increased without incurring much higher error rates. Therefore, the scalability of quantum computing systems is one of the main challenges the quantum community is currently facing.

Nowadays NISQ computers are implemented as single-chip processors, also referred as single-core quantum processors, in which all qubits are integrated within a single chip. This monolithic  architecture is hardly scalable due to challenges in the control electronics and wiring \cite{sebastiano2020cryo}, an increase of undesired interactions between qubits (i.e. crosstalk)~\cite{Ding_2020} and a decrease of the device uniformity and yield. To overcome these challenges and solve the scaling problem, modular quantum computing architectures have been already proposed for different qubit implementation technologies ~\cite{Monroe_2014, https://doi.org/10.48550/arxiv.2201.08825, https://doi.org/10.48550/arxiv.2201.08861,https://doi.org/10.48550/arxiv.2210.10921}. The main idea is to combine multiple quantum processors and connect them via single control systems, classical communication links and ultimately quantum communication technologies~\cite{QuantumIntranet, IBMRoadmap,IBMRoadmap2025}. We refer to the latter, in which both classical and quantum communication channels are incorporated as multi-core quantum computing architectures. They will allow performing distributed multi-core quantum computing in which a large algorithm consisting of more qubits than there are in a single processor, is partitioned into smaller instances and executed on several quantum chips.


% and they will allow to perform distributed quantum computations. 
 


%to the following main challenges: i) when the number of qubits increases, they may be closer together, increasing errors such as crosstalk~\cite{Ding_2020}. ii) In quantum architectures where qubits must operate at temperatures close to absolute zero, there is insufficient cooling power available for a large number of qubits~\cite{8036394}. iii) Higher power consumption. iv) More difficulty to keep qubits isolated from the environment. In summary, the number of errors increases with the number of qubits: the more qubits there are on a single chip, and the more gate operations are performed, the more error-prone the quantum device will become.




%Due to this error increase, IBM realized that this single-core approach is not scalable and has envisioned using a different architecture in which several processors would be connected planned to be launched in 2023~\cite{IBMRoadmap,IBMRoadmap2025}.

%The new technology to be used by IBM is called a multi-core quantum processor. The first approach will be to interconnect multiple NISQ quantum processors via classical links, using circuit cutting and knitting techniques. The final goal would be based on the interconnection of chips through quantum intranet~\cite{QuantumIntranet}, composed of quantum coherent links as well as classical links. Multi-core quantum processors have been proposed for several technologies, such as distributed quantum computers based on trapped ions~\cite{Monroe_2014} or distributed superconducting quantum architectures~\cite{https://doi.org/10.48550/arxiv.2201.08825}.


With this novel architectural approach, new challenges emerge as pointed out in \cite{rodrigo2021double} that include: i) the implementation of input/output communication ports for each core (processor) as well as the definition of the ratio of qubits devoted to computation and communication; ii) the development of the technology required for communicating quantum information between chips and corresponding communication protocols; and iii) compilation techniques, including placement and routing of qubits and scheduling of quantum operations, that allow for an efficient distributed multi-core quantum computation, which will be the central topic of this paper.


%in particular, the interconnection of NISQ processors. In quantum physics is not possible to copy the state of the qubits, requiring them to be moved from one core to another to transport data by means of, for instance, shuttling or teleportation. These movements are expensive, needing to be reduced as much as possible. Therefore, performing an optimal compilation~\cite{https://doi.org/10.48550/arxiv.2112.14139} and efficient inter-core communication~\cite{10.1109/ISCA.2006.24} are required.

%\textcolor{red}{Carmina: structure of this part: i) talk about what compilation in general (adapt quantum algorithms...) means and more specifically in the context of multi-core architecture (allocation of qubits in different cores and routing of qubits between core); ii) although compilation in multi-core have some similarities with the techniques used in single-core there are some fundamental differences such as...; iii) Whereas there are plenty of mapping/compilation techniques for single-core not so many for multi-core, just.... In this paper....  }

%Another challenge this technology presents is that quantum algorithms cannot be directly executed, as each NISQ quantum processor has a series of constraints. The main restriction is the limited connectivity between physical qubits. For instance, some quantum technologies, such as superconducting, have qubits arranged in a linear 2D array topology with limited connectivity between them where only nearest-neighbor (NN) interactions are allowed, requiring quantum information to be moved to adjacent qubits. Translating a hardware-agnostic algorithm to a new one that considers all physical constraints is called the compilation or mapping of a quantum algorithm.

Executing an algorithm on a NISQ processor, requires to perform some modifications on the corresponding quantum circuit such that all quantum hardware constraints are satisfied. This process of adapting the quantum circuit to the quantum processor restrictions is usually called mapping or transpiling. Whereas several quantum circuit mapping techniques have been proposed for single-core quantum architectures~\cite{https://doi.org/10.48550/arxiv.2007.01000,mappings, 8382253, 7059001, Venturelli_2018,10.1145/3297858.3304075} only recently, the first compilation techniques for mapping quantum algorithms onto connectivity-simplified multi-core quantum 
architectures have been proposed \cite{rodrigo2021double, 10.1145/3387902.3392617}. In \cite{10.1145/3387902.3392617}, the authors propose a method for mapping quantum programs to a modular
quantum architecture based on graph partitioning techniques. However, this approach is only tested on a relatively small and fixed quantum computing multi-core architecture in which the number of cores and qubits per core are both constant (i.e. 10 cores $\times$ 10 qubits per core) irrespective of the width (i.e number of qubits) of the circuit to be executed.  

This paper focuses on the very new field of compilation techniques for scalable multi-core quantum computer architectures with the aim of performing distributed quantum computing. To this purpose, the challenges of mapping quantum algorithms to these modular architectures are discussed in Section II, emphasizing the main differences with single-core mapping methods. In section III, we introduce one of the most recent and advanced works on mapping for modular architectures \cite{10.1145/3387902.3392617}. In Section IV, we further explore the performance of this mapping approach by performing an architectural scalability analysis. Finally, conclusions are presented in Section V. 





%The compilation or mapping of quantum algorithms translates hardware-agnostic algorithms into new ones where all physical constraints are considered. The main constraint is the limited connectivity between physical qubits. Qubits need to be allocated in the same core to execute two-qubit gates, needing to perform inter-core movements. Furthermore, in some quantum technologies, such as superconducting chips, qubits are arranged in a linear 2D array topology with limited connectivity between them where only nearest-neighbor (NN) interactions are allowed. This restriction requires quantum information to be moved to adjacent qubits after assigning qubits in the same core.

%Although compiling quantum algorithm techniques are similar between single-core and multi-core processors, several differences in the mapping steps can be noted. Some of the fundamental differences rely on allocating qubits in cores. For instance, it should be considered that not all movements will have the same cost since inter-core movements will be more expensive than intra-core movements. In addition, the time taken to perform inter-core movements is not deterministic, needed to be calculated at runtime, contrary to single-core processors, where the time is deterministic.

%There are several mapping techniques for single-core quantum processors~\cite{https://doi.org/10.48550/arxiv.2007.01000,mappings, 8382253, 7059001, Wille2016UsingD, Venturelli_2018}.

%However, only a mapping algorithm for multi-core quantum processors, based on the relaxed Overall Extreme Exchange (rOEE) algorithm~\cite{10.1145/3387902.3392617}, has been proposed so far. In this paper, we explore, stress, and analyze the application of the rOEE for mapping quantum algorithms in different multi-core architectures.

