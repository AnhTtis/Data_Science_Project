Multi-core or modular quantum computing architectures are a promising approach to overcome the scaling difficulties encountered in monolithic or single-core quantum processors. However, this new architectural design comes with a set of challenges such as qubit interactions across cores. Inter-core communications can be minimized through the process of mapping, as proposed  in~\cite{10.1145/3387902.3392617}. In this paper, we have further analyzed the performance of this quantum circuit mapping technique by performing several experiments in which different architectures with all-to-all connectivity are considered. The most important findings can be summarized as follows: i) Non-local communications and execution time increase with the circuit width. ii) The number of physical qubits is the most important factor regarding execution time and therefore, using a variable number of qubits per core is more efficient. iii) The higher the number of cores, the longer the execution time and the higher the non-local communications. Note also, that the performance highly depends on the quantum algorithm to be executed. 

%Although the proposed technique operates successfully, it has an important constraint: the number of physical qubits is limited due to the long execution time. An approach to overcome it would be adapting the number of physical qubits used to the logical qubits the quantum circuit has. However, further research will be necessary to overcome all challenges multi-core architectures present.