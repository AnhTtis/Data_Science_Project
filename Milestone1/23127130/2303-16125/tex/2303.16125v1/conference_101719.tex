\documentclass[conference]{IEEEtran}
\IEEEoverridecommandlockouts
% The preceding line is only needed to identify funding in the first footnote. If that is unneeded, please comment it out.
\usepackage{cite}
\usepackage{amsmath,amssymb,amsfonts}
\usepackage{algorithmic}
\usepackage{graphicx}
\usepackage{textcomp}
\usepackage{xcolor}
\usepackage{url}
\usepackage{tabularx}
\usepackage{subfig}
\usepackage{comment}


\def\tablename{Table}
\usepackage{etoolbox}
\makeatletter
\patchcmd{\@makecaption}
  {\scshape}
  {}
  {}
  {}
\makeatletter
\patchcmd{\@makecaption}
  {\\}
  {.\ }
  {}
  {}
\makeatother
\def\tablename{Table}
\def\BibTeX{{\rm B\kern-.05em{\sc i\kern-.025em b}\kern-.08em
    T\kern-.1667em\lower.7ex\hbox{E}\kern-.125emX}}
\begin{document}

% The mapping of quantum algorithms in multi-core quantum computing architectures

\title{Mapping quantum algorithms to multi-core quantum computing architectures\\
%{\footnotesize \textsuperscript{*}Note: Sub-titles are not captured in Xplore and
%should not be used}
%\thanks{Identify applicable funding agency here. If none, delete this.}
}


\author{\IEEEauthorblockN{Anabel Ovide\IEEEauthorrefmark{1},
 Santiago Rodrigo\IEEEauthorrefmark{2}, Medina Bandic\IEEEauthorrefmark{3}, Hans Van Someren\IEEEauthorrefmark{3},  Sebastian Feld\IEEEauthorrefmark{3}, Sergi Abadal\IEEEauthorrefmark{2}, \\
 Eduard Alarcon \IEEEauthorrefmark{2}, and Carmen G. Almudever\IEEEauthorrefmark{1}}
 
 
\IEEEauthorblockA{\IEEEauthorrefmark{1}\textit{Technical University of Valencia, Spain}}
\IEEEauthorblockA{\IEEEauthorrefmark{2}\textit{Technical University of Catalonia, BarcelonaTech, Spain}}
\IEEEauthorblockA{\IEEEauthorrefmark{3}\textit{Technical University of Delft, The Netherlands}}}





\begin{comment}
\author{\IEEEauthorblockN{1\textsuperscript{st} Given Name Surname}
\IEEEauthorblockA{\textit{dept. name of organization (of Aff.)} \\
\textit{name of organization (of Aff.)}\\
City, Country \\
email address or ORCID}
\and
\IEEEauthorblockN{2\textsuperscript{nd} Given Name Surname}
\IEEEauthorblockA{\textit{dept. name of organization (of Aff.)} \\
\textit{name of organization (of Aff.)}\\
City, Country \\
email address or ORCID}
\and
\IEEEauthorblockN{3\textsuperscript{rd} Given Name Surname}
\IEEEauthorblockA{\textit{dept. name of organization (of Aff.)} \\
\textit{name of organization (of Aff.)}\\
City, Country \\
email address or ORCID}
\and
\IEEEauthorblockN{4\textsuperscript{th} Given Name Surname}
\IEEEauthorblockA{\textit{dept. name of organization (of Aff.)} \\
\textit{name of organization (of Aff.)}\\
City, Country \\
email address or ORCID}
\and
\IEEEauthorblockN{5\textsuperscript{th} Given Name Surname}
\IEEEauthorblockA{\textit{dept. name of organization (of Aff.)} \\
\textit{name of organization (of Aff.)}\\
City, Country \\
email address or ORCID}
\and
\IEEEauthorblockN{6\textsuperscript{th} Given Name Surname}
\IEEEauthorblockA{\textit{dept. name of organization (of Aff.)} \\
\textit{name of organization (of Aff.)}\\
City, Country \\
email address or ORCID}
}
\end{comment}

\maketitle

\begin{abstract}
Current monolithic quantum computer architectures have limited scalability. One promising approach for scaling them up is to use a modular or multi-core architecture, in which different quantum processors (cores) are connected via quantum and classical links. This new architectural design poses new challenges such as the expensive inter-core communication. To reduce these movements when executing a quantum algorithm, an efficient mapping technique is required. In this paper, a detailed critical discussion of the quantum circuit mapping problem for multi-core quantum computing architectures is provided. In addition, we further explore the performance of a mapping method, which is formulated as a partitioning over time graph problem, by performing an architectural scalability analysis.

%So far, there is only one technique for mapping quantum algorithms in multi-core quantum architectures, based on a partitioning over time graph problem. In this paper, we explore its performance by performing an architectural scalability analysis.
\end{abstract}

\begin{IEEEkeywords}
scalability quantum computing systems, multi-core quantum computers, mapping of quantum algorithms. 
\end{IEEEkeywords}



 \begin{figure*}[!h]
\centering
 \includegraphics[width=\linewidth]{fig/opt_map.drawio.drawio.png}
\caption{
An illustrative example of the quantum circuit mapping procedure. We consider  a 1-D linear array quantum processor shown at the top-left, where only adjacent qubits (circles) can interact. Next, an optimal initial placement of qubits is performed based on the quantum circuit to be executed. Note that, the first two CNOT gates (CNOTS in time steps 1 and 2) can be directly performed as qubits 0 and 2 as well qubits 0 and 3 are adjacent. However, an extra SWAP gate has to be inserted for allowing the execution of the other two CNOT gates between qubit 2 and  qubit 3 and between quit 0 and qubit 1. A SWAP gate exchanges the state of the two involved qubits. In addition, note that scheduling the quantum gates saves one time step as the last two CNOT gates can be performed in parallel.}
\label{fig:map_example}
\end{figure*}

\section{Introduction}
% Importance and appeal of children's drawings
Children's depictions of the human figure are highly expressive and varied.
As one of the very first subjects children attempt to draw, the representation begins as an almost unintelligible cloud of scribbles. 
As the child grows, their representation of the human figure becomes more developed and is extended to graphically represent many different types of characters: people, animals, and even personified objects (see Figure 1).

Who among us has not wished, either as a child or as an adult, to see such figures come to life and move around on the page?
Sadly, while it is relatively fast to produce a single drawing, creating the sequence of images necessary for animation is a much more tedious endeavor, requiring discipline, skill, patience, and sometimes complicated software.
As a result, most of these figures remain static upon the page.

% We built a system to animate them.
Inspired by the importance and appeal of the drawn human figure, we design and build a system to automatically animate it given an in-the-wild photograph of a child's drawing. 
Our system is fast, intuitive, and robust to much of the variation present in these types of drawings, making it well-suited to allow our target audience--children--to see their own characters coming to life.
The system is comprised of four stages: figure detection, segmentation masking, pose estimation/rigging, and animation. 
We describe each stage and identify common causes of failure in each. 
For object detection and pose estimation, we make use of existing computer vision models designed to detect human figures and joints in photographs; we fine-tune these models for use with children's drawings.
For segmentation, we present a straightforward, image processing-based method that, for animation purposes, is more useful and accurate than segmentation masks obtained from a fine-tuned object detection model.
During the animation step, we take advantage of the \textit{twisted perspective} commonly seen in children’s drawings to retarget motion capture data onto the character in a novel and appealing way.

% We use existing machine learning models. However, given the wide domain gap it's not clear how much fine-tuning data was needed. So we ran some experiments to find out and report it.
While our system leverages existing models and techniques, most are not directly applicable to the task due to the many differences between photographic images and simple pen and paper representations. 
To this end, we couple the presentation of our system with a set of experiments exploring the relationship between fine-tuning training set size and success rates.
We also include a perceptual study validating viewer preference for incorporating \textit{twisted perspective} into the motion retargeting step.

We validate the desirability and appeal of our system by building and publicly releasing a version of it as the \AD Demo \,\cite{animateddrawings}.
Launched in December 2021, this demo has been used by millions of people around the world to animate their children's drawings.
Inspired by this reception, our second contribution is The Amateur Drawings Dataset: \hjs{180,000 drawings and user-accepted annotations collected, with consent, through the demo. See Section \ref{sec:UI} for a description of how the annotations were generated.}
We believe this dataset will be a resource to researchers from various fields seeking to better understand the space of amateur drawings, evaluate new algorithms in this domain, or develop new drawing-based tools in general.

To summarize, our contributions are as follows:
\begin{enumerate}
    \item 
    We explore the problem of automatic sketch-to-animation for children's drawings of human figures and present a framework that achieves this effect. We also present a set of experiments determining the amount of training data necessary to achieve high levels of success and a perceptual study validating the usefulness of our motion retargeting technique.
    \item To encourage additional research in the domain of amateur drawings, we present a first-of-its-kind dataset of 180,000 user-submitted amateur drawings, along with user-accepted bounding box, segmentation mask, and joint location annotations.
\end{enumerate}

Upon acceptance of this paper, we plan to publicly release the Amateur Drawings Dataset, project code, and fine-tuned model weights.


\begin{figure*}[!h]
    \centering
    \includegraphics[width=\linewidth]{fig/rOEE_ex.drawio.drawio.png}
    \caption{ Example of the quantum circuit mapping technique proposed in~\cite{10.1145/3387902.3392617}. A multi-core quantum architecture composed of 2 cores and 4 qubits per core with all-to-all connectivity is shown on the left. Next, the quantum circuit to be mapped with its respective time slices is presented. Note that each time slice can be represented by a qubit interaction graph in which virtual qubits correspond to the nodes and edges are the interactions between them (i.e. two-qubit gates). For each of the time slices, a valid assignment is returned by using a relaxed version of the so-called Overall Extreme Exchange (rOEE) algorithm. To achieve it, qubits are exchanged between cores by means of SWAP gates, until a valid assignment is found.}


    \label{fig::mapp_exmple}
\end{figure*}


\section{From single-core to multi-core mapping}

%\begin{table*}[!h]
%\centering
%\footnotesize\addtolength{\tabcolsep}{-60pt}
%\def\tabularxcolumn#1{m{#1}}
%\setlength{\extrarowheight}{3pt}
%\begin{tabularx}{1\textwidth} { 
%| >{\centering\arraybackslash}X
% |>{\centering\arraybackslash}X 
%   >{\centering\arraybackslash}X 
%   >{\centering\arraybackslash}X 
%   >{\centering\arraybackslash}X | }
% \hline
% \textbf{Benchmark} & \textbf{Qubits} & \textbf{Two-qubit gates} & \textbf{Average two-qubit gates} & \textbf{Slices}\\
% \hline
% \textbf{Cuccaro adder} &  50-100 & 120-240   & 1.657-1.662 &  70-140 \\
 
% \textbf{QUEKO} &   50-100  & 600-2000 & 10-20   & 50-100\\
% \textbf{Quantum Volume} & 50-100 & 10000-30000 & 4.5-8 & 1750-3750\\
% \textbf{QFT adder} & 50-100  & 10000-80000   & 1.016-1.028 & 10000-80000\\
% \hline
%\end{tabularx}\\
%\vspace{5pt}

%\caption{Characterization of the different quantum algorithms used as benchmarks, showing the minimum and maximum of different values.}
%\label{table:comp_bench}
%\end{table*}


%Nowadays single-core quantum processors are far from being general-purpose quantum computers and from the so-called quantum practicality. To achieve these goals, scalability is necessary, but going beyond the NISQ era and developing a scalable quantum computer is a challenge, as it must handle an increasing number of qubits without losing fidelity.

%Current single-core quantum computers have been scaled considerably in the past years, reaching a total of 127 qubits by the quantum processor Eagle developed by IBM~\cite{Eagle}, and planning to achieve the number of 1000 qubits in the following years~\cite{IBMRoadmap}. 

%However, scaling this kind of architecture entails several challenges due to the increasing amount of errors. Principal errors are caused by: i) qubit decoherence. ii) Gates and measurements are imperfect and prone to errors caused by, for instance, frequency collisions~\cite{8614500} or crosstalk~\cite{Ding_2020}. iii) Hardware complications such as wiring, system cooling or isolation from the environment to preserve qubits' states~\cite{sebastiano2020cryo}. Increasing the number of qubits on a single chip considerably increases the above-mentioned errors, decreasing execution fidelity and complicating scalability.


%\begin{figure}
%\centering
%\includegraphics[width=86mm]{fig/map_process.drawio.png}
%\vspace{-0.5cm}
%\caption{Mapping process procedure. The compiler receives as input the quantum hardware description, on the right, as well as the quantum algorithm, on the left. It outputs the scheduled operations, i.e compiled quantum algorithm, that can be executed by the quantum device and performs the initial placement and the routing process.}
%\vspace{-0.1cm}
%\label{fig:map_process}
%\end{figure}

%           ---> !!! FIGURES IN THE MAIN FILE FOR EASY ORDERING !!! <----

Quantum circuit mapping techniques have been developed for single-chip NISQ processors, as part of the compilation process, to deal with their constraints and allow to successfully execute quantum algorithms~\cite{https://doi.org/10.48550/arxiv.2007.01000,mappings, 8382253, 7059001, Venturelli_2018,10.1145/3168822,8980312,10.1145/3297858.3304075}. More precisely, quantum circuit mapping is about transforming hardware-agnostic quantum circuit descriptions into a hardware-compliant version that considers all physical restrictions of a given quantum processor. One of the main constraints in current quantum devices is the reduced connectivity between physical qubits, which usually limits their possible interactions to only nearest-neighbour requiring qubits to be moved to adjacent positions to execute the desired two-qubit operation (e.g. CNOT gate). The circuit mapping procedure consists of different steps: i) \textbf{gate decomposition}, in which gates of the circuit are decomposed into a series of native gates implementable in the quantum processor; ii) \textbf{initial placement} of qubits, where quantum circuit qubits, i.e. virtual qubits, are assigned to the physical qubits of the device. This process helps to minimize the (movement) operations needed in the routing stage; iii) \textbf{routing} of qubits, in which non-neighboring qubits that need to perform a two-quit gate are moved to adjacent physical qubits (which share a connection) by means of, for instance, SWAP gates; and iv) \textbf{scheduling} of operations to leverage parallelism while respecting their dependencies and quantum hardware constraints. An example of the quantum circuit mapping process is shown in Figure~\ref{fig:map_example}.

%The compilation or mapping of a quantum algorithm is proved to be an $NP$-completed problem~\cite{complexity} in which a hardware-agnostic quantum algorithm is translated into one that considers all physical constraints a quantum processor has. The main constraint is the limited connectivity between physical qubits, which restricts their possible interactions requiring qubits to be moved to execute the desired operations. The mapping process introduces several gates (commonly SWAPs gates) to place qubits in the required positions, which causes an increase in the circuit depth, decreasing the success probability of the algorithm.






%The gates the initial quantum circuit has may not be supported by the quantum processor; the \textbf{gate decomposition} process adapts them into a set of universal quantum gates that the quantum processor can recognize. The \textbf{scheduler} organizes gates through time, parallelizing operations while respecting quantum processor constraints and gate dependencies. In the \textbf{initial placement}, quantum circuit qubits, i.e. logical qubits, are assigned to the physical qubits. This process optimizes further steps by minimizing future operations needed in the routing stage. After the initial placement, the \textbf{routing} procedure checks whether all two-qubit gates can be performed. When the qubits involved in operations do not satisfy the connectivity restrictions, the routing process finds a path (e.g. shortest path) where the involved qubits are moved to the required positions. An example of the mapping process is in Figure~\ref{fig:map_example}.




As mentioned in the previous section, multi-core quantum computing architectures are a promising approach to scale up current single-core quantum computers.
%alleviating the increasing number of errors. 
Existing proposals agree on an architecture based on interconnecting multiple NISQ processors~\cite{QuantumIntranet, https://doi.org/10.48550/arxiv.2201.08861} consisting of tens to hundreds of qubits, increasing the total qubit count without losing that much fidelity and improving isolation. In this architectural design, NISQ processors will be ultimately interconnected through short-range quantum-coherent links and classical links in the form of a so-called `quantum intranet'~\cite{QuantumIntranet}. Quantum coherent links will be responsible for transporting qubits (or quantum states) from core to core, for instance, by means of shuttling o quantum teleportation. Several challenges arise with this new architecture, being the most relevant for this work the need for exchanging quantum information between cores. %Similar to the single-core mapping problem, qubits will need to find a way to interact across cores, but in this case from core to core, whenever they need to interact.
Note that these inter-core communications are more expensive and error prone than those performed in single-core architectures. Therefore, multi-core quantum computing architectures require the development of a new breed of compilation techniques that will have to consider the following fundamental different aspects:  
 
 

%This approach has been proposed for different quantum technologies, such as superconducting qubits~\cite{https://doi.org/10.48550/arxiv.2201.08861, https://doi.org/10.48550/arxiv.2201.08825}, and ion trap devices~\cite{Monroe_2014, IONQ}. The first multi-core quantum processors are planned to be launched by IBM in the following years~\cite{IBMRoadmap}. 
%Heron will first feature only classical links, followed by Flamingo, which will incorporate quantum links, and finally, Kookaburra, combining classical and quantum links~\cite{IBMRoadmap}.

%Several challenges arise with this new architecture, being the most relevant for this work the need for exchanging quantum information among cores. Similar to the single-core mapping problem, qubits will need to be moved, but in this case from core to core, whenever they need to interact. These communications between cores are more expensive and error prone than those performed in single-core architectures.

%being necessary to optimize the compilation, trying to avoid the inter-core movements by optimally planning and moving qubits among cores.
%performing two-qubit remote gates~\cite{https://doi.org/10.48550/arxiv.2201.08861} or an efficient compilation.




%\textcolor{red}{Go to the point. We do not need to show an example of trivial and optimal mapping. We just need to show that in order to make the circuit runnable we need to add some extra SWAP gates. Explain how mapping works in the caption of the example. We just need the left figure.}

%The mapping procedure consists of four different processes in no specific order i) gate decomposition, ii) initial placement, iii) routing, and iv) scheduling. The gates the initial quantum circuit has may not be supported by the quantum processor; the \textbf{gate decomposition} process adapts them into a set of universal quantum gates that the quantum processor can recognize. The \textbf{scheduler} organizes gates through time, parallelizing operations while respecting quantum processor constraints and gate dependencies. In the \textbf{initial placement}, quantum circuit qubits, i.e. logical qubits, are assigned to the physical qubits. This process optimizes further steps by minimizing future operations needed in the routing stage. After the initial placement, the \textbf{routing} procedure checks whether all two-qubit gates can be performed. When the qubits involved in operations do not satisfy the connectivity restrictions, the routing process finds a path (e.g. shortest path) where the involved qubits are moved to the required positions. An example of the mapping process is in Figure~\ref{fig:map_example}.

%\textcolor{red}{Once the previous point is made mention that several single-core mapping techniques have been proposed and add references and enumerate (bullets) and explain the main challenges and therefore differences. Also, talk about compilers for distributed quantum computers and try to derive some similarities and differences.}

%In multi-core quantum processors, the mapping of quantum algorithms can be more complicated due to the movements of qubits between cores via coherent quantum links. As mentioned before, these communications are more expensive than intra-core communications, being necessary to optimize the compilation trying to avoid the inter-core movements by optimally planning and moving qubits among cores. Note that the mapping of quantum algorithms on multi-core architectures has two different parts: first, an optimal mapping of qubits between cores must be performed, and second, qubits on single cores should be arranged to be nearest neighbors.

%Several differences can be observed in the mapping steps between single-cores and multi-cores quantum architectures. One main difference is the time consumption of qubit movements between cores. The time is not deterministic (e.g. teleportation) so it must be calculated at runtime, complicating the scheduling process. The operation time of inter-core communications is then unknown at compile time, meaning that the scheduler must be dynamic. In addition, the scheduler must have control of the available resources, i.e. which qubits are available and which are occupied with performing inter-core movements.

%The initial placement must consider different qubit movement costs: inter-core movements will be more expensive than intra-core movements. It should place the qubits considering movement costs and qubits interactions; i.e. qubits that will interact must be placed, if possible, in the same core. The gate decomposition and routing process are similar to single-core architectures. 


%There are several mapping techniques for single-core NISQ devices~\cite{https://doi.org/10.48550/arxiv.2007.01000,mappings, 8382253, 7059001, Wille2016UsingD, Venturelli_2018,10.1145/3168822}. 

%However, performing the mapping procedure in multi-core architectures entails new challenges:

\textbf{Inter-core communication}: Similar to the single-chip case in which qubits need to be adjacent for interacting, qubits placed in different cores cannot directly perform a two-qubit gate. To do so, they have to make use of entanglement-based quantum communication protocols that require the generation of the so-called \textit{Bell pairs} allowing to perform, for instance, remote CNOTs between distant qubits or to teleport quantum states from one core to another~\cite{9334411, rodrigo2022characterizing}. This comes with an overhead of resources needed for creating and distributing entangled pairs. In addition, the entanglement generation is a non-deterministic process making the scheduling task more complex. 


\textbf{Not all qubits have the same functionality}: In each of the quantum cores there will be qubits devoted to computation and storage and qubits used for communication. Communication qubits will handle inter-core communications, whereas storage qubits will perform local operations. Mapping techniques will have to include information about qubit `types' and which ones are being used as well as the resources available for communication. Note that the more qubits are dedicated to communication, the higher the number of inter-core communications that can be performed in parallel.

%\newpage
 
\textbf{ A two-step quantum circuit mapping process}: An initial qubit placement and routing should be done at the quantum core level, placing qubits that need to interact on the same core and use efficient routing techniques to reduce inter-communication operations, but also within the quantum processors to reduce the overhead created due to their limited qubit connectivity as in  the single-chip case. 

%by optimally planning and moving qubits among cores. Following it, mapping approaches for single-core processors should be applied separately to each core.
 
 
 
 
    %\item  The time consumption of qubit movements between cores is not deterministic, requiring a dynamic scheduler.
    
    %\item  The scheduler must have control of the available resources, i.e. which qubits are available and which are occupied performing inter-core movements.
    %\item The initial placement should place the qubits considering the different qubit movement costs and qubits interactions: qubits that will interact must be placed, if possible, on the same core.
    
   % \item The mapping must be performed in two steps: first, optimally planning and moving qubits among cores; second, mapping approaches for single-core processors should be applied separately to each core. 
%\end{itemize}

%There are several mapping solutions for single-core NISQ devices~\cite{https://doi.org/10.48550/arxiv.2007.01000,mappings, 8382253, 7059001, Wille2016UsingD, Venturelli_2018}. However, only a mapping approach for mapping quantum algorithms in multi-core quantum architectures~\cite{10.1145/3387902.3392617}, has been proposed so far, which will be discussed in the next section. 

Multi-core or modular architectures for scaling up quantum devices share a lot of similarities with the quantum networks that are being deployed for a future quantum internet~\cite{van2016path,https://doi.org/10.48550/arxiv.2201.08825,8910635}. The main difference resides in the fact that communications links, instead of being short-range, are long-range~\cite{Kozlowski_2019}, resulting in the need for a more complicated infrastructure to move qubits between quantum devices, i.e. quantum repeaters. Due to this quantum network infrastructure, moving qubits among devices would be more complex, needing to perform entanglement swaps~\cite{9334411}, which increases latency considerably as its duration grows exponentially with the distance between devices. One possible application of quantum networks is to perform distributed quantum computing, for which compilation techniques have been already proposed~\cite{9334411,https://doi.org/10.48550/arxiv.2112.14139,Cicconetti_2022}. However, not so much attention has been paid so far to the development of compilers for multi-core quantum computing architectures \cite{rodrigo2021double, 10.1145/3387902.3392617}. In the next sections, we  will focus on the mapping technique proposed in~\cite{10.1145/3387902.3392617}. % and further analyse its performance by performing an architectural scalability analysis. 




\section{Quantum circuit partitioning for distributed quantum computing}

%           ---> !!! FIGURES IN THE MAIN FILE FOR EASY ORDERING !!! <----
%Multi-core or modular quantum architectures %are a new technology~\cite{Monroe_2014, https://doi.org/10.48550/arxiv.2201.08825, https://doi.org/10.48550/arxiv.2201.08861,https://doi.org/10.48550/arxiv.2210.10921} proposed to overcome the scalability problem encountered in current single-core or monolithic processors. This new technology
%introduce new challenges, such as the movement of qubits between cores. As stated in the last section, these movements are costly and should be avoided to the extent possible. Optimal compilation or mapping is required to reduce these inter-core movements. So far, only one mapping technique has been proposed for multi-core or modular quantum computing architectures, presented 

In~\cite{10.1145/3387902.3392617} a technique for mapping quantum algorithms on multi-core architectures based on graph partitioning has been proposed. The goal is to place qubits in the different quantum processors such that inter-core movements are minimized. An illustrative example of this mapping technique is shown in Figure~\ref{fig::mapp_exmple}. %For a more detailed explanation we refer the reader to ~\cite{10.1145/3387902.3392617}.  
Note that in the proposal presented in~\cite{10.1145/3387902.3392617} the following assumptions that simplify the quantum circuit mapping problem are made: i) all-to-all connectivity between cores and among physical qubits within the cores. This means that there is no need for qubit routing inside the core, nor for optimal initial placement. Regarding inter-core routing and qubit placement at the core level, all qubits are at a one-hop distance, and therefore when two qubits have to interact and cannot be placed from the beginning on the same quantum core it is enough to place one of them on any other core; ii) SWAP operations (i.e. exchange of quantum states) are used for inter-core communication that makes simpler the management of resources as it not required to check if there is space (i.e. qubits that do not have any information) for exchanging qubits between cores; iii) only a fixed modular architecture is considered consisting of 10 cores with 10 qubits per core, which is not enough for analyzing the performance of the quantum circuit mapper. In the following section, different architectures will be used to further analyze this proposed mapping procedure.





%The proposed mapping technique works as follows. First, the quantum circuit is partitioned in time. Each resulting time-slice consist of one or several CNOT gates and the corresponding qubit interaction graph in which physical qubits correspond to the nodes of the graph and edges are the interactions between qubits (i.e. two-qubit gates).  and the cores are the partitions.

%Their approach bases the initial placement procedure on a static partitioning graph problem where the total number of partitions is fixed and limits the total number of qubits in each partition. However, only algorithms~\cite{6771089, Fiduccia1982ALH} that find approximate solutions to this problem were available since heuristic solvers~\cite{HARIF2022789} cannot guarantee the fixed size of the partitions. 

%For the routing problem, posed as a qubit assignment problem, instead of finding a global assignment of the qubits to the cores, they used a partitioning over time graph problem, the relaxed Overall Extreme Exchange algorithm. The rOEE algorithm is a relaxed version of the Overall Extreme Exchange algorithm~\cite{PARK1995899}. For each time slice (i.e. set of two-qubit gates that can be performed in parallel), there will be a graph representing the interactions between qubits at that specific time in the quantum circuit. Edges will have a cost defined by a lookahead weight, which will consider future qubits interactions. For each time slice, the rOEE finds a valid assignment (i.e. qubits that interact in that time slice are placed in the same partition). When the algorithm finishes, it returns a path consisting of a sequence of valid assignments through the quantum algorithm. An example is presented in Figure~\ref{fig::mapp_exmple}.

%This mapping technique has only been simulated on a specific modular architecture (10 cores and 10 qubits per core); in the following section, different architectures will be used to analyze this proposed mapping procedure in depth.



\section{Results}



%\begin{figure*}[!h] 
%     \centering
%    \subfloat[]{
 %           \includegraphics[width=0.3\textwidth]{fig/results/nCom_Strong_Comp_cucc.png}\label{fig:sub5}
 %   }
  %  \subfloat[]{
  %      \includegraphics[width=0.3\textwidth]{fig/results/time_fix_var_Strong_Comp_cucc.png}\label{fig:sub3}
   % }
    %\subfloat[]{
    %    \includegraphics[width=0.3\textwidth]{fig/results/weak_scaling.png}\label{fig:sub4}
    %}
    %\caption{Experiments results for different architectures. The comparison between a fixed number of qubits per core and a variable number of qubits per core is shown in the first and second rows, focusing on the non-local communications (SWAPs) and execution time, respectively. The weak and strong scaling results are shown in the last row.}
    %\label{fig:bench_comp}
%\end{figure*}

\begin{figure}[!h] 
     \centering
    \subfloat[]{
        \includegraphics[width=0.49\columnwidth]{fig/results/tight/non_local_communications.pdf}\label{fig:sub1}
        }
    \subfloat[]{
        \includegraphics[width=0.49\columnwidth]{fig/results/tight/execution_time.pdf}\label{fig:sub2}
    }\\
    \vspace{-9pt}
    \subfloat[]{
           \includegraphics[width=0.49\columnwidth]{fig/results/tight/nCom_Strong_Comp_cucc.pdf}\label{fig:sub5}
    }
    \subfloat[]{
      \includegraphics[width=0.49\columnwidth]{fig/results/tight/time_fix_var_Strong_Comp_cucc.pdf}\label{fig:sub3}
    }\\
    \vspace{-9pt}
    \subfloat[]{
       \includegraphics[width=0.49\columnwidth]{fig/results/tight/weak_scaling.pdf}\label{fig:sub4}
    }
    \caption{Non-local communications (SWAPs) and execution time for different multi-core architectures. (a) and (b) for a fixed and a variable number of qubits per core. (c) and (d) when a strong scaling of the architecture is performed. (e) Weak scaling of the multi-core architecture.}
    \label{fig:bench_comp}
\end{figure}


\section*{Results}
We started by assembling a dataset derived from public hikes. This process included an iterative data cleaning process to remove erroneous/false data, identify and remove breaks (e.g. Fig \ref{Fig2}) to give us a final usable dataset containing 7,636 GPS tracks, with over 1.4 million individual data points and covering almost 88,000 km of travel in the U.K. 

Our curated hike dataset allowed us to create a data-driven model which we can directly compare with existing walking speed algorithms. The model formulation was selected using a small-scale exploratory study which considered data from Scotland (see \nameref{S3_Appendix}). In this exploratory study, multiple different model types were explored which could fit the data, and which matched existing knowledge about walking speeds. Cross-validation methods showed that there was very little difference in performance of the best models, therefore the final model was a Generalised Linear Model (GLM), which was chosen as it was the simplest of those tested (we had no evidence that a more complex model would be superior). This choice also meant that our model was both easy to interpret, and simple to apply to future work.

This final GLM model included all three of the variables suggested by Arnet \cite{Arnet2009ArithmeticalJapan}:

\begin{equation}
    v = exp(a+b\phi+c\theta+d\theta^2)
\end{equation}
where
\begin{quote}
$v = \text{walking speed (km/h)}$\\
$\phi = \text{hill slope angle (degrees)}$\\
$\theta = \text{walking slope angle (degrees)}$
\end{quote}

Terrain obstruction level was included as a factor variable, while we considered the road types as both factor variables and interaction terms. Not all terms had a significant effect on all variables; we therefore created a model with all possible terms, and removed them one at a time (in order of least significance) until all remaining terms were significant to at least 95\% confidence  level (using Wald test). The final values for a, b, c and d are given in Table \ref{tab:2ROUK model variable values} for each of the terrain obstruction levels and road types. The critical gradient for this model is between 14 -- 16 degrees when walking uphill and -16 -- -18 degrees when walking downhill (depending on road and obstruction conditions), which is in line with previous findings. 

Fig \ref{Fig3} shows the predicted walking speeds under different conditions. The importance of including both the hill slope and terrain obstruction variables can be clearly seen when looking at the Off Road Light Obstruction speed predictions. When directly ascending or descending a slope, the walking speed is comparable to walking on a road. However, when traversing a slope while off road, the walking speed is comparable to traversing a slope of double the gradient while on a road or path. Similarly, comparing the walking speed predictions of Off Road Light Obstruction and Off Road Heavy Obstruction reveals that just 10 cm of vegetation (our cutoff point for heavy obstruction) can reduce the walking speed by more than 0.5 km/h.

\begin{table}[!ht]
\begin{adjustwidth}{-0.5in}{0in}
    \centering
    \caption{Final walking speed model variable coefficients}
    \begin{tabular}{|l+c|c|c|c|}
    \hline
    & $a$ & $b$ & $c$  & $d$ \\ 
    \thickhline
    Paved road & 1.580 & -0.00389 & -0.00726 & -0.00218 \\ 
    \hline
    Unpaved road & 1.580 & -0.00389 & -0.00965 & -0.00248 \\
    \hline
    Off-road (obstruction unknown) & 1.536 & -0.00731 & -0.00965 & -0.00187 \\
    \hline
    Off-road (light obstruction) & 1.580 & -0.00731 & -0.00965 & -0.00187 \\ 
    \hline
    Off-road (heavy obstruction) & 1.400 & -0.00731 & -0.00965 & -0.00187 \\ 
    \hline
    \end{tabular}
    \label{tab:2ROUK model variable values}
\end{adjustwidth}
\end{table}

\begin{figure}[!h]
\begin{adjustwidth}{-2.25in}{0in} 
    \includegraphics[width=\linewidth]{Images/Paper/Fig3.eps}
    \captionsetup{width=1\linewidth}
    \caption[width=\textwidth]{{\bf Walking speed predictions under different terrain conditions.}  When: (A) travelling directly up or down hills of varying slope, (B) traversing across hills of varying slope.}
    \label{Fig3}
    \end{adjustwidth}
\end{figure}

Fig \ref{Fig4} compares the Paved Road and Off Road Heavy Obstruction speed predictions from our model against the existing functions from Naismith, Tobler and Campbell et al. When looking at the walking slope, the largest areas of deviation between our model and Naismith's rule occurs when descending a slope, as Naismith's rule does not predict a reduced speed in this scenario. For both Tobler's and Campbell et al.'s functions, the shape of the walking slope component is relatively similar to our new model, with the main distinction being the peak predicted speed on flat ground. None of the existing functions account for the hill slope, which leads to large disparities when predicting the walking speed for slope traversals. A further example of this can be seen in \nameref{S6_Appendix}, which shows the walking speeds for a simulated off-road route which encounters the full range of hill and walking slopes.

\begin{figure}[!h]
\begin{adjustwidth}{-2.25in}{0in} 
    \includegraphics[width=\linewidth]{Images/Paper/Fig4.eps}
    \captionsetup{width=1\linewidth}
    \caption[width=\textwidth]{{\bf Comparison of new model and existing hiking functions.}  Predicted walking speeds of the new model, Naismith's rule, Tobler's function and Campbell et al.'s function when: (A, C, E) travelling directly up or down hills of varying slope, (B, D, F) traversing across hills of varying slope.}
    \label{Fig4}
\end{adjustwidth}
\end{figure}

When comparing the performances of each of the models (Table \ref{tab:2comparison}), the predicted speeds for individual 50 m sections had a lower RMSE and percentage error, and a higher R squared value using our new model than in the existing ones. To isolate the impact of each of the slope variables, we filtered the results to look at the data where a slope was being directly climbed or traversed. Figs \ref{Fig5}A, B and \ref{Fig6}A, B show the RMSE and mean residuals for each of the models, for data which was within 5 degrees of directly climbing (A) or traversing (B) hills of varying slope. From this we can clearly see that Naismith's rule consistently overestimates walking speeds when descending a slope, and underestimates speeds when climbing a slope. When ascending or descending a slope, the RMSE of our GLM is similar to that of Tobler's hiking function. However, one of the main areas where we see an improvement using our model is on slight declines. Tobler's hiking function suggests that walking speed increases on mild descents up to a maximum of 6 km/h. It is clear from Fig \ref{Fig5}A, that Tobler's function overestimates the walking speed in this region. Campbell et al.'s function has a slightly lower RMSE value than our new model on the steepest walking slopes, however it underestimates the walking speeds on flat ground and mild slopes. Previous research has found that most walking takes place on low walking slopes \cite{Proffitt1995PerceivingSlant}, and this is evidenced by our data ($\sim$98\% of our data was from walking slopes of under 10 degrees). Improved walking speed predictions in this region therefore have the greatest impact in real-world situations. Within this region our model consistently has a lower RMSE than the existing functions, and a mean residual error close to 0 km/h. 

\begin{table}[!ht]
\centering
\caption{Comparison of new model against existing methods to calculate walking speeds.}
\begin{tabular}{|l|c|c|c|c|}
\hline
& New Model & Naismith & Tobler & Campbell\\
\hline
Average \% error & 23.68 & 26.36 & 26.17 & 25.33\\
\hline
MSE & 1.20 & 1.61 & 1.53 & 1.58\\
\hline
RMSE & 1.10 & 1.27 & 1.24 & 1.26\\
\hline
R\textsuperscript{2}  & 0.09 & -0.22 & -0.16 & -0.19\\
\hline
\end{tabular}
\label{tab:2comparison}  
\end{table}

\begin{figure}[!h]    
\begin{adjustwidth}{-2.25in}{0in} 
    \includegraphics[width=\linewidth]{Images/Paper/Fig5.eps}
    \captionsetup{width=1\linewidth}
    \caption[width=\textwidth]{{\bf Comparing RMSE values for the new model, Naismith's rule, Tobler's function and Campbell et al.'s function.} When: (A) travelling directly up or down hills of varying slope (all data), (B) traversing across hills of varying slope (all data), (C) travelling directly up or down hills of varying slope (off-road data only), (D) traversing across hills of varying slope (off-road data only). Campbell et al.'s function does not provide off-road speed estimates, so was not included in the off-road data comparisons.}
    \label{Fig5}
\end{adjustwidth}
\end{figure}

\begin{figure}[!h]
    \begin{adjustwidth}{-2.25in}{0in} 
    \includegraphics[width=\linewidth]{Images/Paper/Fig6.eps}
    \captionsetup{width=1\linewidth}
    \caption[width=\textwidth]{{\bf Comparing mean residual values for the new model, Naismith's rule, Tobler's function and Campbell et al.'s function.} When: (A) travelling directly up or down hills of varying slope, (B) traversing across hills of varying slope, (C)  travelling directly up or down hills of varying slope (off-road data only), (D) traversing across hills of varying slope (off-road data only). Campbell et al.'s function does not provide off-road speed estimates, so was not included in the off-road data comparisons.}
    \label{Fig6}
\end{adjustwidth}
\end{figure}

 We also see an improvement in RMSE when using our model to predict speeds for hill traversals (Fig \ref{Fig5}B). We can note from Fig \ref{Fig6}B that both Naismith's rule and Tobler's hiking function consistently overestimate the walking speed when traversing a slope, as they do not take into account the impact that the hill slope has on reducing walking speeds. The performance of Campbell et al's model improves as the hill slope increases, although we suggest this is more due to it underestimating the speed on shallow slopes. We do see that the average error in our model increases as the hill slope increases, but we believe that this is due to limited volumes of data at high hill slopes ($\sim$0.5\% of our data occurs on hill slopes steeper than 40 degrees). 

As well as looking at the overall performance of our new model, we looked to explore how well our model performed in off-road conditions, compared to the off-road adjustments for the existing functions (Naismith's reduced base speed of 4 km/h, and Tobler's correction factor of 0.6). Figs \ref{Fig5}C, D and \ref{Fig6}C, D show the RMSE and mean residuals, only considering data which was recorded in off-road conditions. From Figs \ref{Fig5}C and \ref{Fig6}C it is clear that Tobler's function consistently underestimates the walking speed when off-road. The factor of 0.6 is a larger reduction in walking speed than is observed in practice. As we found when looking at our data as a whole, Naismith's rule underestimates the walking speed when climbing a slope and overestimates when descending a slope. Our new model does not suffer from these problems, with both a lower RMSE and lower absolute mean residual value across all walking slopes. Both of these existing models also consistently underestimate walking speeds when traversing a slope, unlike our new model which has a mean residual of less than 0.4 km/h on slopes of up to 35 degrees. The error in predictions of our new model does increase as the hill slope increases, though the RMSE is generally lower than seen in the existing models. On the steepest hill slopes our model appears to perform less well than the existing ones, though only 0.2\% of our off-road data occurred on a hill slope steeper than 40 degrees. 

Although we have shown an improvement in walking speed predictions over short sections of routes, this did not translate to similar results when looking at predicted walking times for routes as a whole. Our model and all of the existing models which we have explored here had an average percentage error of 13.5\% - 15.5\% when predicting the time taken for a complete route. However, based on the errors seen in Figs \ref{Fig5} and \ref{Fig6}, we believe that this is a result of errors cancelling out over the course of a hike. For example while ascending a hill, Naismith's rule will underestimate the walking speed (and thus overestimate the walking time), but it will then overestimate the walking speed on the subsequent descent, leading to a relatively accurate total time estimate. The results here suggest that Naismith's rule, and other existing functions, are still a good rule of thumb to calculate route times as a whole, but time estimates for individual sections of a route will be less accurate than when using the new model found here.




\section{Conclusions}
%\section{}
%\label{sec:resDir}


\section{Conclusion}
\label{sec:conclusion}
% <>
Since its advent in 1931, Koopman operator theory \cite{koopman:1931} has only recently been actively utilized for solving practical problems, thanks to the introduction of the DMD algorithm in 2008 \cite{schmid:2008}. Since then, a multitude of DMD algorithm variations have risen to prominence and found utility across various fields. A notable feature of our survey paper was reviewing and categorizing the results of over 100 research papers based on both application and algorithm type in smart mobility and vehicle engineering  (see Table~\ref{tab1} and Section~\ref{sec:vehicApp}).  Additionally, this survey paper identified potential research gaps in smart mobility and vehicular engineering applications (Remarks~\ref{remGap1}--\ref{remGap6}). Finally, this review paper discussed theoretical aspects of Koopman operator theory that have been largely neglected by the smart mobility and vehicle engineering community and yet have large potential for contributing to solving open problems in these areas (see Section~\ref{subsec:theorIssue}).

\noindent{\textbf{Future Research Directions.}}	Given the emergence of cyber-threats against connected and autonomous vehicles as well as robotic systems (see, e.g.,~\cite{nekouei2021randomized,mohammadi2022generation}), a future research direction might include utilizing Koopman operator-based algorithms for designing cyber-resilient vehicular and smart mobility applications (see, e.g.,~\cite{taheri2022data} for a related line of research). Another potential research direction is using Koopman operator-based algorithms for predicting the motion of vulnerable road users (VRUs), e.g., pedestrians and cyclists (see, e.g.,~\cite{pool2019context,scholler2020constant}). Finally, rehabilitation robotics and robotic exoskeletons can be the benefactors of the predictive capabilities of Koopman operator-based algorithms for detecting tripping events and/or system  identification in various modes of locomotion (see, e.g.,~\cite{kumar2019extremum,aprigliano2019pre}).



%Fig. 1 depicts the accumulation of such algorithms since 2014, which are particular to vehicle engineering and smart mobility, i.e., the focus of this review. Table 1 summarizes the varieties of relevant algorithms developed in those studies. Furthermore, we have highlighted theoretical issues, whose expansion will have potential applications to the wide research area of smart mobility and vehicle engineering.  

%Although fairly comprehensive, we have found several gaps in this research area. In particular, we could not find any studies related to elevators, robots/vehicles employing crawling, slithering, hopping or peristaltic locomotion, arctic or special-terrain vehicles such as those employing screws or tracks, hovercraft and other amphibious vehicles or subsystems which tolerate flexible environments, classification or guidance systems related to vehicles for drilling or agriculture, or for current-ripple, power-split, battery health monitoring, nuclear propulsion, exoskeletons/prosthetics, personal mobility, motorsports, specialized rovers or similar open problems in emerging areas.  These examples are, of course, not exhaustive.  
%
%The purely data-driven nature of Koopman operators holds the promise of capturing unknown and complex dynamics for reduced-order model generation and system identification, through which the rich machinery of linear control techniques can be utilized. The emergent nature of the smart mobility and vehicular-related applications, where  the Koopman operator  in each particular application needs to be approximated, implies that the development of various Koopman operator approximation  algorithms is expected to grow along with the vehicular problems they aim to solve.  Given the ongoing development of this research area and the many existing open problems in the fields of smart mobility and vehicle engineering, a survey of techniques and open challenges of applying Koopman operator theory to this vibrant area is warranted.  To the best of our knowledge, this survey paper is the \emph{first of its kind} reviewing the applications of Koopman operator theory within a focused research area, namely, smart mobility and vehicle engineering applications. A \emph{notable feature} of our survey paper is reviewing and categorizing the results of over 100 research papers based on both application and algorithm type  (see Tables~\ref{tab1}--~\ref{tab4} and Section~\ref{sec:vehicApp}) that are concerned with the applications of Koopman operator theory to the field of smart mobility and vehicular engineering. Such a \emph{comprehensive and  detailed categorization} will be beneficial to the research practitioners working in the field.  Furthermore, this review paper discusses theoretical aspects of Koopman operator theory that have been largely neglected by the smart mobility and vehicle engineering community and yet have large potential for contributing to solving open problems in these areas. Additionally, our survey paper seeks to \emph{identify gaps} in the smart mobility and vehicle engineering research where new and existing Koopman operator-based methods have the potential to further develop and address unsolved problems  potentially benefiting from the perspectives of nonlinear system identification, control, global linearization, and the predictive powers that Koopman operator theory has to offer (see, e.g., Remarks~\ref{remGap1}--\ref{remGap6}). 


\section*{Acknowledgments}

We acknowledge support from EU, grant HORIZON-ERC-101042080 (S.A.) and grant HORIZON-EIC-2022-PATHFINDEROPEN-01-101099697 (S.A., E.A and C.G.A.), from project PRG 946 funded by the Estonian Research Council (A.O.), from MICIIN and European ERDF under grant PID2021-123627OB-C51 (C.G.A.) and from MICIIN with funding from European Union NextGenerationEU(PRTR-C17.I1) and by Generalitat de Catalunya (E.A.)



\bibliographystyle{ieeetr}
\bibliography{bib/main}
%\printbibliography

\end{document}
