\pdfoutput=1
\documentclass{amsart}
\usepackage{a4wide}
\usepackage{amsmath}
\usepackage{amssymb}
\usepackage{graphicx,color}
\usepackage{hyperref}
\theoremstyle{definition}
\newtheorem{definition}{Definition}
\newtheorem{example}{Example}
\newtheorem{notation}{Notation}
\newtheorem{move}{Move}
\theoremstyle{theorem}
\newtheorem{theorem}{Theorem}
\newtheorem{proposition}{Proposition}
\newtheorem{lemma}{Lemma}
\newtheorem{claim}{Claim}
\newtheorem{question}{Question}
\newtheorem{corollary}{Corollary}
\newtheorem{fact}{Fact}
\newtheorem{observe}{Observation}
\theoremstyle{remark}
\newtheorem{remark}{Remark}

\newcommand{\ri}{\operatorname{RI}}
\newcommand{\B}{H}
\newcommand{\kk}{\operatorname{knot}}
\newcommand{\s}{\operatorname{split}}
\newcommand{\jo}{\operatorname{join}}
\newcommand{\V}{\operatorname{vol}}


\begin{document}
\title[Automatic computation of crosscap number]{Automatic computation of crosscap number of alternating knots}
\author{Kaito Yamada}
\address{866 Nakane, Hitachinaka, Ibaraki, 312-8508, Japan}
\email{nanigasi.py@gmail.com}
\author{Noboru Ito}
\address{866 Nakane, Hitachinaka, Ibaraki, 312-8508, Japan}
\email{nito@gm.ibaraki-ct.ac.jp}
\keywords{automatic computation; crosscap number; alternating knot}
\date{March 17, 2023}
\maketitle
\begin{abstract}
We specify the computational complexity of crosscap numbers  of alternating knots by introducing an automatic computation. 
%, which is nearly equals the sum of the second and penultimate absolute coefficients of Jones polynomial.
For an alternating knot $K$, let $\mathcal{E}$ be the number of edges of its diagram.  Then 
there exists a code such that the complexity of this computation of the crosscap number of $K$ is estimated by $O(\mathcal{E}^3)$.  
\end{abstract}
\section{Introduction}
The genus of a surface bounded by a knot $K$ is studied for about nine decades; remarkable progress has been made for the orientable case (the knot genus).  However, for non-orientable case (the crosscap number), there has not been a   \emph{general} approach to compute the crosscap number $C(K)$ of a knot $K$, whereas some pioneers, Hatcher-Thurston \cite{HatcherThurston1985}, Teragaito \cite{Teragaito2004}, Hirasawa-Teragaito \cite{HirasawaTeragaito2006}, and  Ichihara-Mizushima \cite{IchiharaMizushima2010}, give computations of crosscap numbers of certain families of knots.    

In \cite{ItoYamada2021}, the author KY introduces a new tool  as a Eulerian graph $G$ of a given alternating knot diagram $D$ to compute crosscap numbers automatically by computer aid.  The plan is as in the following list: 
\begin{enumerate}
\item If $G$ is obtained from  $D$, $G$ is uniquely determined.      \label{Y1}
\item Given $G$, the reduced graph is defined. \label{Y2}
\item For $n$, by inputting the initial data of the splice-unknotting number one, the data generates the list of alternating knot diagrams with splice-unknotting number $n$.   \label{Y3}
\end{enumerate}
The paper \cite{ItoYamada2021}  provides (\ref{Y1}) and (\ref{Y2}).  In this paper, the task (\ref{Y3}) is accomplished. 
\section{Crosscap number and splice-unknotting number}
\begin{definition}[non-orientable genus / crosscap number]
Let $K$ be a knot.  Then $C(K)$ is the minimum number among the first Betti numbers of non-orientable surfaces $\Sigma$ ($\subset \mathbb{R}^3$) with $\partial(\Sigma)$ $=$ $K$.  
\end{definition}
\begin{definition}[$S^-$]
Given a crossing on a link  diagram, there are the two possible ways to splice it.  One of them preserves orientations and the other does not; the symbol $S^-$ denotes the latter way.  
\end{definition}
\begin{definition}
Let $\ri^-$ be the first Reidemeister move resolving a single crossing.  
Let $D$ be a knot projection of a knot $K$.  The nonnegative number $u^-(D)$ is the minimum number of splices of type $S^-$ among any sequences of splices of types $S^-$ and $\ri^-$.   
\end{definition}
\begin{fact}[Ito-Takimura \cite{ItoTakimura2020}, Kindred \cite{ Kindred2020}]
Let $D$ be any prime alternating knot diagram of a knot $K$.  
\[
u^- (D) = C(K).  
\]  
\end{fact}
%Before setting definitions of graphs, we recall Fact~\ref{CorKD} by \cite{ItoTakimura2018}.  
\begin{fact}[\cite{ItoTakimura2018}, cf.~ {\cite[twisted $S^+$ move/bridging operation]{ItoTakimura2022}}]\label{CorKD}
Every prime alternating knot diagram is obtained by applying bridging operation successively some times.   

More precisely, every prime alternating knot diagram $D$ with $u^-(D)=n$ is obtained from the knot projection with no crossing by applying bridging operation successively $n$ times.   
\end{fact}
\section{Main Result}   
\begin{theorem}\label{MainResult}
For an alternating knot $K$, let $C(K)$ be the crosscap number.  
A knot Eulerian graph is given by a knot diagram as in Definition~\ref{KEG}.  
Let $E$ be the number of edges of a knot Eulerian graph given by a knot diagram.  Then the complexity of the computation of $C(K)$ is bounded by $O(E^3)$.  
%, which is sharper estimate than the above.  
\end{theorem}
The number $V$ of vertices of a knot Eulerian graph is the less than or equal to the crossing number $\mathcal{V}$ of a knot diagram, i.e., $O(V) \le O(\mathcal{V})$.  Let $\mathcal{E}$ be the number of edges.  
$O(E)=O(V) \le O(\mathcal{V}) = O(\mathcal{E})$.   
%By comparing the case of diagrams with Theorem~\ref{MainResult}, we understand the advantage of Theorem~\ref{MainResult}.  
\begin{corollary}
For an alternating knot $K$, let $C(K)$ be the crosscap number.  Let $\mathcal{E}$ be the number of edges of its diagram.    
Then there exists a code such that the complexity of the computation of $C(K)$ is bounded by $O(\mathcal{E}^3)$.
\end{corollary}
\begin{remark}
Even if $K$ is a non-alternating knot, the above estimation works in some case (cf.~\cite[Theorem~2, Fig.~1]{ItoTakimura2020}).  However,  the extent to which it will work is yet to be solved.  
\end{remark} 
\section{A short review of knot Eulerian graphs and update}\label{CAid}
\begin{definition}[twist region \cite{KalfagianniLee2016}]
A \emph{twist region} of a diagram consists of maximal collections of bigon regions arranged end to end; we suppose that the crossings in each twist region occur in an alternating fashion.    
\end{definition}
Note that a single crossing adjacent to no bigon is also a twist region.  
\begin{definition}[knot Eulerian graph]\label{KEG}
%Using Fact~\ref{CorKD}, 
Every twisted region of an alternating knot diagram $D$ is given by a bridge operation increasing odd/even crossings, and $\ri^+$ if necessary.   The twisted region with odd/even crossings 
is represented as in Figure~\ref{Vertex}.  We call it an odd/even \emph{vertex} or a vertex simply.  
Every vertex has four endpoints that are two inputs and two outputs.  
%EPV%For any edge, we can add/remove any \emph{empty-vertex} on it as in Figure~\ref{Vertex} (lower).   By definition, every empty-vertex has two endpoints that are an input and an output.  
Each vertex has an information as follows:
\begin{itemize}
\item By knot diagram, it is natural to suppose that every input corresponds to the unique output.  
\item  Valency of each vertex is exactly four.  
\end{itemize}
\begin{figure}[h!]
\includegraphics[width=12cm]{fig1.pdf}
\caption{Twisted regions of knot   diagrams and vertices of a knot Eulerian graph.}\label{Vertex}
\end{figure}
Further, we set the following rule for constructing our actual code.
\begin{itemize}
\item 
%We often consider twist regions, each of which consists of odd crossings, we call it a \emph{vertex}.  
We often decompose each twisted region with even crossings into a single $\ri^+$ and a twisted region consisting of odd crossings.  
\item For every vertex, orientations of inputs or outputs are induced by knot diagrams as in Figure~\ref{Vertex}.  Each twist region with at least two crossings naturally gives a symmetric axis of the vertex and the left and right sides are called \emph{poles} and they are denoted by $A$ and $B$ in arbitrary way.   The  symmetric axis is called an \emph{axis}.  There are variations of axes by orientations (Figure~\ref{VertexVariation}).  
\begin{figure}[htbp]    
\centering
\includegraphics[width=10cm]{VertexVariationA.pdf} 
   \caption{Variations of vertices}
   \label{VertexVariation}
\end{figure}  
If a twist region has exactly a single crossing, we define the axis in arbitrary way.      
%\item The orientation of an empty-vertex is induced by an input to an output (cf.~Figure~\ref{Vertex}, the second line).  Since an empty-vertex has no $A/B$; and thus we call its pole is called ``None".  Then the pole is denoted by ``$N$" in a code and by ``e" in figures.  
\end{itemize}
By the above definition, an alternating knot diagram implies vertexes; further the knot diagram has information of connections of the vertexes, which connections are canonically presented by edges. To code the edges, it is sufficient to use information of endpoints on poles.   Then a tuple $(u, v, P_u, P_v)$ denotes an edge, which satisfies conditions as follows.  
%Each edge has presented by a tuple $(u, v, T_u, T_v)$ with an information $u, v, T_u$, and $T_v$ as follows.  
%This information determines which side a vertex connect to.  
\begin{itemize}
\item (the first element of the tuple) $u$ denotes the vertex that is the starting point of the  edge.  
\item (the second element of the tuple) $v$ denotes the vertex that is the end point of the  edge.     
\item (the third element of the tuple) $P_u$ is the pole on which $u$ is.   
\item (the fourth element of the tuple) $P_v$ is the pole on which $v$ is.   
\end{itemize}
%To eliminate an exception, if $u=v$, we decompose this edge  by a single empty-vertex.  

By the above construction, every alternating knot diagram gives a graph with four valences.   
The resulting graph is a Eulerian graph; we shall call it a \emph{knot Eulerian graph}.  
\end{definition}
\begin{remark}
In the previous paper \cite{ItoYamada2021}, $T_u$ denotes the vertex whereas we use $P_u$ that denotes it; $A$ and $B$ are called types whereas we call them \emph{poles}.   Note also that we do not need any empty vertex in this paper.  
%since odd/even are also called types.    
\end{remark}
\begin{fact}[\cite{ItoYamada2021}]\label{PropKE}
An oriented knot projection gives a knot Eulerian graph.  
\end{fact}
%\begin{proof}
%Since any crossing of oriented alternating knot belong to a twisting, by replacing each twisting with a real vertex having odd/even-information, we have a $4$-valent graph, which is a knot Eulerian graph.    Here, one may add empty-vertices arbitrarily.  
%Since each edge of the knot Eulerian graph has an orientation and each real vertex has information of types (odd/even) and axises, we recover the oriented knot projection from the knot Eulerian graph (here, we ignore empty-vertices if necessary).  Here, the information of axises is essential because twisting type is determined by an axis (alike Conway notation).  
%Thus, there exists a bijection between the set of oriented knot projections and that of knot Eulerian graphs except for adding or removing empty-vertices.    
%\end{proof}
%\begin{definition}\label{ReKED}
%We give \emph{reduced} knot Eulerian graphs as an analogue of reduced alternating knot diagrams in the following. 
%\begin{itemize}
%\item Step~1: Remove the empty-vertices.   
%\item Step~2 ($\ri^-$): Find an edge $(u, u, A, B)$ ($(u, u, B, A)$,~resp.);  remove the vertex $u$ and the edge $(u, u, A, B)$ ($(u, u, B, A)$,~resp.); replace $(a, u, T_a, A)$ and $(u, b, B, T_b)$ ($(a, u, T_a, B)$ and $(u, b, A, T_b)$,~resp.) with $(a, b, T_a, T_b)$.  
%\item Step~3: If we find any extra vertex, we unify two vertexes.  To do this, we will check every pair (as in Figure~\ref{Eg}, upper) of two vertices $u, v$ belonging to a single twisting where $u, v$ will be unified.  
%In order to make our programming, we should list every case, but here we see an example only since reader can easily recover the list: 
%
%If we find two neighboring vertices $P_1$ and $P_2$ as in  Figure~\ref{Eg} (lower left), we remove $P_1$, $P_2$ and add a vertex $Q$ as in Figure~\ref{Eg} (lower right).
%\end{itemize}
%\end{definition}
%\begin{figure}
%\includegraphics[width=6cm]{7-2.pdf}
%\includegraphics[width=8cm]{Eg.pdf}
%\caption{}\label{Eg}
%\end{figure}
%Here, we recall Fact~\ref{ITLemma} by Khovanov \cite{Khovanov1997} and \cite{ItoTakimura2013},   
%\begin{fact}\label{ITLemma}  
%For a give knot projection $P$, we apply successive first flat Reidemeister moves decreasing crossings until we have a knot projection, say $P^{1r}$, with no monogon.   Then, $P^{1r}$ is uniquely determined.   
%\end{fact}
%Then we have:  
%\begin{corollary}\label{CorKE}
%For any oriented knot projection, a reduced knot Eulerian graph is determined uniquely.  
%\end{corollary}
%%\includegraphics[width=8cm]{EV.pdf}
%By Corollary~\ref{CorKE}, it is sufficient to make a computer program on knot Eulerian graphs to treat all the alternating knots.      
%\section{Result of computations}
%The implementation is given by: 
% 
%\texttt{https://github.com/nanigasi-san/knot}  
%
%This implementation realizes (\ref{Y1}) in Section~\ref{CAid} and the judgement of equivalence of any two knot Eulerian graphs. 
%For any knot Eulerian graph,  reduced knot Eulerian graph is  uniquely determined via  Steps~1--3 of Definition~\ref{ReKED}.   
In the rest of this paper, a vertex and an edge of a knot Eulerian graph are called a vertex and an edge, respectively.  However, to avoid any confusion, a vertex and an edge of a knot diagram is called a \emph{crossing} and a \emph{knot-edge}, respectively.  
\section{An estimation of computation complexity of bridge operations}\label{sec:Estimate}
Since it is possible to apply a bridge operation to \emph{every} knot-edge, we define two operations for knot Eulerian graph.  
\begin{itemize}
\item Decomposition of a vertex  (equivalently, a decomposition of a twist region of a knot diagram).  
\item $\ri^+$ (equivalently, the first Reidemeister move increasing a single crossing). 
\end{itemize} 
\subsection{Decomposition of a vertex}\label{sec:Deomp}
%Let $b$ be a link-edge which will be applied by a bridge operation.  We consider the case that $b$ is of a twist region.  
We define a decomposition of a vertex as in Figure~\ref{Decomp}.
\begin{figure}[htbp] %  figure placement: here, top, bottom, or page
   \centering
   \includegraphics[width=5cm]{DecompA.pdf} 
   \caption{An example of a  decomposition of a vertex}
   \label{Decomp}
\end{figure}
There are two cases.  Let Odd (Even,~resp.) be the vertex type corresponding to a twist region of odd (even,~resp.) crossings.  
\begin{itemize}
\item Odd $\to$ Even and Odd / Odd and Even.  
\item Even $\to$ Odd and Odd  / Even and Even.  
\end{itemize}
%If there is a bigon which will be applied by  an edge in a twist region of a diagram and 
\subsection{$\ri^+$}
\begin{itemize}
\item $(u, v, P_u, P_v)$ $\to$ $(u, O, P_u, A)$, $(O, O, B, A)$, and $(O, v, B, P_v)$
\end{itemize}
\subsection{Mixture of a decomposition of a vertex and applications of $\ri^+$}
\begin{figure}[htbp] %  figure placement: here, top, bottom, or page
   \centering
   \includegraphics[width=13cm]{DeRIA.pdf} 
   \caption{An example of a mixture of a decomposition of a vertex and applications of $\ri^+$}
   \label{DeRI}
\end{figure}
Figure~\ref{DeRI} shows an example.  To make a computer program, we give a priority to a decomposition to fix the place where we apply a bridge operation, i.e., we always decompose a vertex before a $\ri^+$ is applied.     
%\subsection{An estimation of computational complexity from $n$ to $n+1$}\label{EDecomp}
%\begin{itemize}
%\item Decompositions: $O(V)$.  
%\item $\ri^+$: $O(V)$. 
%\end{itemize}
\subsection{Selecting two edges which will be applied by a bridge operation}\label{Ecycle}
%In this section, we estimate  possibilities of bridge operations.  The number of pairs are at most $E(E-1)$ where $E$ is the number of edges of a given knot Eulerian graph.  
%Since $K$ is an alternating knot, it is sufficient to search pairs of edges which share a region of a link diagram and thus, so is a knot Eulerian graph.   
We estimate the complexity of computation of choice of the two places as follows: 
\begin{itemize}
\item We have a list of ordered edges $e_1, e_2, \dots, e_E$ that consists of a Euler cycle representing the knot Eulerian graph.   $\to$ $O(E)$.   
\item 
%Suppose that the selected face of exactly $m$ edges: 
Two edges are chosen $\to$  $O(E^2)$.  
%Here, we use $O(m^2)$ $\le$ $O(E^2)$ drived from $m \le E$.            
\end{itemize}
In the above process, these listing and selecting are totally estimated by $O(E^2)$.  
%In the above process, we have a list of ordered edges that consists of an Euler cycle representing the knot Eulerian graph.   Then 
Let $e_s$ $=$ $(a, b, P_a, P_b)$ and $e_g$ $=$ $(c, d, P_c, P_d)$ ($s<g$) for two edges.
%\begin{lemma}
%$O(m^2)$ $=$ $O(E)$.  
%\end{lemma}
%\begin{proof}
%Noting that $4V=2E$ and $V-E+F=2$, we claim: 
%\begin{align}\label{Ineq}
%m \le E - F+2 (= V).  
%\end{align}
%It is sufficient to show (\ref{Ineq}).   
%We show the claim by induction with respect to $F$ creating new faces from the circle with no vertex.  Suppose that we draw a knot Eulerian graph on the sphere.  By definition, \[
%V-E+F=2.  
%\] 
%There is no case $F=0$ or $1$.  If $F=2$, $m=1$ and $E=1$, which implies the claim.  Suppose that the case $F=n$, the claim holds and we denote the number of edges by $E(n)$.     
%$F=n+1$, we must add at least one $E$ since $V-E+F=2$ holds.  
%\end{proof}
%the upper bound  the complexity of computation is $O(F)=O(E)$.   Hence, $O(F \times E^2)$ $=$ $O(E^3)$.  
%By $E=2V$, $V-E+F=2$, $F-2=-V+E=\frac{1}{2}E$.  
%\item If we consider ``3D version" of bridge operations in $\mathbb{R}^3$, the complexity of computation is at most $O(E^2)$.  
%\end{itemize}
%\end{remark}
\subsection{Application of a bridge operation}\label{SecBridge}
%A single application of a bridge operation implies adding a twist region on a link diagram.   In this section, we realize the process in  a code.   Note that a single bridge operation fixes two places on edges that starts and ends twisting, which needs two other vertices than the vertex (\ref{SecAlg}, Steps~1, 2) corresponding to a twist region.     
%There are two cases since the vertex $P$ corresponding to a twist region is odd or even.     
\subsubsection{Algorithm}\label{SecAlg}
\begin{enumerate}
\item[Step~0:] We apply a decomposition as in Section~\ref{sec:Deomp} to vertices which connect to the selected above two edges.   
\item[Step~1:] We apply a single $\ri^+$ to  every edge.    
\item[Step~2:] Add a vertex $P$ with inputs $a, b$ and outputs $c, d$ as in Figure~\ref{Vertex}.  Define the four edges of the vertex $P$, i.e., $\alpha=(a, P, P_a, A)$, $\beta=(b, P, P_b, B)$, $\gamma=(P, c, B, P_c)$, $\delta=(P, d, A, P_d)$.  
\item[Step~3:] Add new edges $\{e'_{s+1}, e'_{s+2}, \dots, e'_{g-1}\}$ corresponding to $\{e_{s+1}, e_{s+2}, \dots, e_{g-1}\}$ where we give $e'_i$ by reversing the orientation of $e_i$.   
\item[Step~4:] Remove edges in $\{e_{s}, e_{s+1}, \dots, e_{g}\}$.     
\end{enumerate}
\begin{figure}[htbp]    \centering
\includegraphics[width=5cm]{fig5.pdf} 
   \caption{Vertex $P$ in a general case}
   \label{AddingPUnori}
\end{figure}
\subsection{Examples: a bridge operation on a crosscap number one knot, Cases I--I\!I\!I}
We give examples: Cases I--I\!I\!I as in Figures~\ref{CaseI}--\ref{CaseIII}.  In Figures~\ref{CaseI}--\ref{CaseIII}, note that applying $\ri^+$'s at Step~1 often be omitted if they are wasted operations; in our actual code, we do not omit them.  
%\subsubsection{Case~I}
%\begin{enumerate}
%\item[Step~0:] Skipped.   
%\item[Step~1:] We apply a single $\ri^+$ to  every edge.  (Fig.~\ref{CaseI})  
%\item[Step~2:] Add a vertex (Fig.~\ref{CaseI}, third arrow).  
%\item[Step~3:] Add new edges $e'_i$ ($s+1 \le i \le g-1$); we give $e'_i$ by reversing the orientation of $e_i$.   
%\item[Step~4:] Remove edges $e_i$ ($s+1 \le i \le g-1$).  
%\end{enumerate}
\begin{figure}[h!]    
\centering
   \includegraphics[width=12cm]{odd1_to_odd2.pdf} 
   \caption{Case~I}
   \label{CaseI}
\end{figure}
%\subsubsection{Case~I\!I}
\begin{figure}[htbp]    
\centering
   \includegraphics[width=12cm]{odd1_to_odd_even.pdf} 
   \caption{Case~I\!I}
   \label{CaseII}
\end{figure}
%\subsubsection{Case~I\!I\!I}
\begin{figure}[htbp]    
\centering
   \includegraphics[width=12cm]{odd_to_odd2_even.pdf} 
   \caption{Case~I\!I\!I}
   \label{CaseIII}
\end{figure}
\subsection{An estimation of computation complexity of bridge operations}\label{Ebridge}
After applying steps in Section~\ref{Ecycle}, we give an estimation of the complexity of the computation in Section~\ref{SecBridge}.  
%for choosing two edges to apply a bridge operation, the complexity of the computation $O(E)$ for \emph{alternating knots} \footnote{If not, $O(E(E-1))$.}.  Further 
\begin{itemize}
%\item We find an Euler cycle $C$  representing a knot by listing every Euler cycle; in the same time, we list the ordered edges including $C$ $\to$ $O(E)$.  
%\item Vertices are added $\to$ $O(1)$.  
\item We apply decompositions of four vertices that connect to the two selected edges $\to O(1)$.  
\item We apply $\ri^+$ for every edge $\to$ $O(E)$.  
\item We select two edges from decomposed edges including old/newborn edges by the above two steps (decompositions and $\ri^+$'s) $\to$ $O(1)$.  
\item We add a new vertex $P$  and new edges $\alpha$, $\beta$, $\gamma$, and $\delta$ $\to$ $O(1)$.   
\item We add new edges $e'_{s+1}$, \dots $e'_{g-1}$  $\to$ $O(E)$.   
%\item We reverse orientations of edges between $e_{s}$ and $e_g$ $\to$ $O(E)$. 
%\item Remove empty two vertices $\to$ $O(1)$.    
\item We remove the edges from $e_s$ to $e_g$ $\to$ $O(E)$.  
\end{itemize}
The above process is totally estimated by $O(E)$.
\section{Proof of Theorem~\ref{MainResult}}
\begin{proof}
By Sections~\ref{Ecycle} and \ref{Ebridge}, for each fixing a pair which will be applied by a bridge operation, we complete a bridge operation in a code.  
\begin{itemize}
\item To fix each pair which will be applied by a bridge operation, the complexity of the computation is bounded by $O(E^2)$.  
\item To apply a bridge operation in a code, the complexity of the computation is bounded by $O(E)$.  
\end{itemize}
Hence $O(E^2) \times O(E)$ $=$ $O(E^3)$.  
\end{proof}
\section{Actual code}
%In theory, every twisted region consists of odd crossings or even crossings; however, since every twist region consisting of even crossings is decomposed into a single $\ri^+$ and a twist region consisting of odd crossings.  For writing an actual code, we intend to reduce cases in the main step, i.e.  adding a twisted region.   
%In Section~\ref{sec:Estimate}, in order to simplify the proof, we avoid applying $\ri^+$.  However, in the actual code, every twist region consisting of even crossings is decomposed into $\ri^+$ and a twist region consisting of odd crossings.  Thus, before starting Step 0, we apply $\ri^+$ to each edges followed by a bridge operation with odd crossings.  This increases the computational complexity by $O(E)$; however, for the final step, which is the main operation of the algorithm, the number of cases is reduced.  In fact, the final step is just adding a vertex labeled by ``\emph{odd}".  It is more convenient for eliminating bugs.  
For the detail, please see the actual code: 

\texttt{https://github.com/nanigasi-san/splus/blob/main/algorithms/main.py}


\section{Acknowledgements}
The work is partially supported by JSPS KAKENHI Grant Numbers  20K03604 and 22K03603 and Toyohashi Tech Project of Collaboration with KOSEN Grant Number 2309.  
\bibliographystyle{plain}
\bibliography{Ref}
\end{document}
