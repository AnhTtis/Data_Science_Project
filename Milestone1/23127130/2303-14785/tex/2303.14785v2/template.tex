\documentclass[journal]{IEEEtran}

\usepackage{siunitx}
\usepackage{booktabs}
\usepackage{hyperref}
\usepackage{soul}
\usepackage{caption}
\usepackage{cite}
\usepackage{amsmath,amssymb,amsfonts}
\usepackage{algorithmic}
\usepackage{graphicx}
\usepackage{textcomp}
\usepackage{seqsplit}
\usepackage{xcolor}
\usepackage{makecell}
\usepackage{enumitem}
\usepackage{babel}
\usepackage{hyperref}
\def\BibTeX{{\rm B\kern-.05em{\sc i\kern-.025em b}\kern-.08em
    T\kern-.1667em\lower.7ex\hbox{E}\kern-.125emX}}
    
\usepackage{amsthm}
\newtheoremstyle{cited}{3pt}{3pt}{\itshape}{}{\bfseries}{.}{.5em}{\thmname{#1} \thmnumber{#2}\thmnote{\normalfont#3}}
    
\newtheorem{theorem}{Theorem}
\theoremstyle{definition}
\newtheorem{definition}{Definition}
\newtheorem{conjecture}{Conjecture}
\newtheorem{problem}{Problem}
\theoremstyle{cited}
\newtheorem{citedthm}[theorem]{Theorem}



% hyperref is always at the end of preamble
\usepackage{hyperref}



\begin{document}

\title{Review of the NIST Light-weight Cryptography Finalists}

\author{
    \IEEEauthorblockN{
    William J Buchanan\IEEEauthorrefmark{1}, Leandros Maglaras\IEEEauthorrefmark{1}}\\
    \IEEEauthorblockA{\IEEEauthorrefmark{1}Blockpass ID Lab, Edinburgh Napier University}}


\maketitle


\begin{abstract}
Since 2016, NIST has been assessing light-weight encryption methods, and, in 2022, NIST published the final 10: ASCON, Elephant, GIFT-COFB, Grain128-AEAD, ISAP, Photon-Beetle, Romulus, Sparkle, TinyJambu, and Xoodyak. The time
that the article was written, NIST announced ASCON as the chosen method that will be published as NIST’s lightweight
cryptography standard later in 2023. In this article we provide a comparison between these methods in
terms of energy efficiency, time for encryption and time for hashing.
    \end{abstract}
    
    \begin{IEEEkeywords}
Light-weight cryptography, NIST
    \end{IEEEkeywords}







\section{Introduction}

The ability to reason about plans is critical for performing long-horizon tasks \citep{erol1996hierarchical, sohn2018hierarchical, sharma-etal-2022-skill}, compositional generalization \citep{corona-etal-2021-modular} and generalization to unseen tasks and environments \citep{shridhar2020alfred}.
Consider a simple long-horizon planning scenario where a robot is tasked with preparing a meal and serving it on the table. 
This presents a non-trivial planning problem since the agent needs to understand the sequence of operations required to perform the task and search for the relevant objects in the unfamiliar environment by interacting with various objects. %



Large language models have been recently shown to possess commonsense knowledge about the world such as object affordances and physical dynamics \citep{ouyang2022training,chowdhery2022palm}.
Early approaches considered text based environments and fine-tuned PLMs to predict actions given the history of past observations and actions \citep{jansen-2020-visually,micheli-fleuret-2021-language,yao-etal-2020-keep}.
Recent work has used this ability to reason about plans from text instructions in simulated household environments with simplifying assumptions such as text-only environment observations or feedback \citep{huang2022language,ahn2022can,li2022pre,logeswaran-etal-2022-shot}.


We focus on \emph{visually grounded planning} with PLMs --- the ability to adapt plans based on interaction and visual feedback from the environment.
While PLMs have strong planning commonsense priors, predictions from a PLM may not be directly realizable in the environment since the observation and action spaces are unknown.
This requires \emph{grounding} the PLM in the environment and adapting it to observe visual feedback, which is highly non-trivial.
Some prior works assume the availability of a pre-trained affordance function \citep{ahn2022can} or a success detector \citep{mirchandani2021ella}.
Notably, SayCan \citep{ahn2022can} completely decouples the PLM from observation information by selecting actions that have both high affordability (through a pre-trained affordance model) and high PLM likelihood.
Although this partially addresses the grounding problem, the use of visual feedback for action affordance alone is limited.
Often an agent must choose one of many affordable actions using information from observations.
For example, a driving agent should re-navigate and possibly turn around when encountering a ``road closed'' sign, but both turning around and driving forward are indistinguishable to SayCan because they are both affordable and the PLM is blind to observations.

Another workaround explored in prior work is translating the information in the visual observations to text using a pre-trained captioning system \citep{shridhar2021alfworld,huang2022language}.
However, it can be difficult to faithfully describe an image in words and information is lost in this inherently noisy process, which limits the information available to the planner.



Recent work shows that PLMs can be adapted for various natural language tasks by inserting tunable embeddings or soft prompts at the input of the PLM (also called prompt tuning or prefix tuning)~\citep{li-liang-2021-prefix,lester-etal-2021-power}.
This approach also extends to multi-modal understanding tasks such as image captioning \citep{mokady2021clipcap} and VQA \citep{tsimpoukelli2021multimodal} where images are encoded as soft prompts and finetuned for the target task.
Transformer based architectures have also been successfully applied to offline Reinforcement Learning in recent work \citep{chen2021decision,janner2021offline,li2022pre,reid2022can}.

Taking inspiration from these works, we propose the simple approach of embedding visual observations (`visual prompts') and \textit{directly inserting them as PLM input embeddings}.
The visual encoder and PLM are jointly trained for the target task, an approach we call \textbf{\oursfull}~(\ours).
By teaching the PLM to use observations for planning in an end to end manner, we remove the dependency on external data such as captions and affordability information that was used in prior work.
We show that this simple approach performs better than prior PLM-based planning approaches on two embodied planning benchmarks based on ALFWorld~\citep{shridhar2021alfworld} and Virtualhome~\cite{puig2018virtualhome}.



\section{Related Work}

\subsection{Pattern discovery on systematic AI errors}

Systematic errors, sometimes coined as blind spots or unknown-unknowns \cite{BeatMachineChallengingHumans}, refer to model's failure over a group of instances that share similar semantics. There are various approaches for discovering such patterns, including algorithmic, human, or hybrid techniques.

A number of studies have shown that fully algorithmic techniques can help automatically discover unknown-unknowns \cite{lakkaraju2017identifying, coveragebasedutility}. Recently, several studies have also been proposed to advance the methods towards discovering automatic slices or subclasses that are semantically coherent \cite{DominoDiscoveringSystematicErrorsCrossModal, SpotlightGeneralMethodDiscoveringSystematic}, or to propose a framework for evaluating blindspot discovery methods in a unified manner \cite{EvaluatingSystemicErrorDetectionMethods}.

On the other hand, researchers have also explored how human intelligence can identify blind spots where automatic techniques alone do not work. Several studies \cite{BeatMachineChallengingHumans, ContradictMachineHybridApproach, HybridHumanAIWorkflowsUnknown, InvestigatingHumanMachineComplementarity} demonstrated that a well-designed crowdsourcing study can detect problematic instances. Hybrid workflows to leverage the abilities of both humans and machines \cite{HybridHumanAIWorkflowsUnknown, lakkaraju2017identifying, han2021iterative, chung2018unknownexamples} have also been explored throughout several studies in proposing collaborative human-AI workflow \cite{HybridHumanAIWorkflowsUnknown} or generating text descriptions \cite{han2021iterative} about spurious patterns.

While these studies demonstrate how human intelligence plays a significant role, tool support is still lacking to guide practitioners to inspect, identify, and mitigate systematic errors. In our study, we provide a workflow and systematic support for inspecting which systematic errors are attributed to interpretable concepts.

\subsection{Visual analytics for ML diagnostics}
Visual analytics tools in recent years have evolved to offer interactive ways for inspecting the machine learning process. In general, these tools aim to better visualize the predictive results in a model-agnostic manner or present the structure of the model in a model-specific way. Model-agnostic approaches propose to better visualize machine learning results regardless of model types. Many visualizations among them are largely designed on the grounds of confusion matrix as tree or flow diagram \cite{shen2020designing, VisualizingSurrogateDecisionTrees}, comparative visual design \cite{ManifoldModelAgnosticFrameworkInterpretation, ExplainExploreVisualExplorationMachine, olson2021contrastive, kaul2021improvingcounterfactuals, krause2017workflow}, radial \cite{VisualMethodsAnalyzingProbabilistic} or multi-axes based layout \cite{SquaresSupportingInteractivePerformance}. On the other hand, model-specific inspections also gained attention to support the inspection of a deep neural network inside its layers, neurons, or activations \cite{liu2017analyzingtraining, ShapeShopUnderstandingDeepLearning, TopoActVisuallyExploringShape, DeepVIDDeepVisualInterpretation}.

Visual analytic tools can also help inspect and explain the potential cause of systematic failures such as a shifted or skewed distribution of the training examples termed as out-of-distribution \cite{OoDAnalyzerInteractiveAnalysisOutofDistribution}, covariate or concept shift \cite{DiagnosingConceptDriftVisual} or machine biases \cite{FairVisVisualAnalyticsDiscovering, FairSightVisualAnalyticsFairness, WhatIfToolInteractiveProbing}. The OoD analyzer \cite{OoDAnalyzerInteractiveAnalysisOutofDistribution} presented a grid-based layout to visualize the distributional differences in training and test sets. The problem of concept drift was tackled and presented as visualizations in a 2D heatmap visualization \cite{DiagnosingConceptDriftVisual} or distribution-based scatterplot \cite{ConceptExplorerVisualAnalysisConcept}. Other interactive tools such as Deblinder \cite{DiscoveringValidatingAIErrorsCrowdsourced}, SEAL \cite{SEALInteractiveToolSystematicError}, or Error Analysis \cite{erroranalysis} have recently been proposed to mitigate systematic errors with subclass labeling or user-generated report. Compared to previous work, our study aims to promote a human-in-the-loop workflow consisting of tasks to identify biased patterns and their association/attribution aspects with the perspective of spurious associations.

% Recent visualization studies also proposed how to better explain them with counterfactuals [BF-1], or to present them in a form of report [BF-3]. 


\subsection{Understanding model with concept interpretability}

The XAI methods to explain the behavior of black box models \cite{InterpretabilityFeatureAttributionQuantitative, AutomaticConceptbasedExplanations, BayesianCaseModelGenerative, ConceptWhiteningInterpretable2020} have been recently expanded to a concept-level sensitivity. The method called TCAV (Testing Concept Activation Vector) \cite{InterpretabilityFeatureAttributionQuantitative} provides a post-hoc method to explain the global influence of a concept in a pre-trained model. ACE (Automatic Concept Extraction) \cite{AutomaticConceptbasedExplanations} was proposed to identify and filter interpretable concepts from the meaningful clusters of segments on the basis of TCAV. In \cite{ConceptWhiteningInterpretable2020}, Concept Whitening (CW) purposefully alters batch normalization layers to a concept whitening layer to learn an interpretable latent space. Especially, the whitening step in this method points out that the concept space needs to be preprocessed to better align concept vectors.

These concept-level interpretability methods, however, require the human ability to observe and extract semantically meaningful concepts \cite{AutomaticConceptbasedExplanations}. There are various ways to identify and extract concepts in collaboration with humans and systems \cite{AutomaticConceptbasedExplanations, NeuroCartographyScalableAutomaticVisualSummarization, zhao2021humanintheloopextraction, DASHVisualAnalyticsDebiasingImage,  ConceptExplorerVisualAnalysisConcept, ProtoSteerSteeringDeepSequence, AnchorVizFacilitatingSemanticData, ConceptVectorTextVisualAnalytics, VisualConceptProgrammingVisualAnalytics}. ConceptExtract \cite{zhao2021humanintheloopextraction} aimed to support concept extraction and classification in a human-in-the-loop workflow and visual tools. In DASH \cite{kwon2022dash}, problematic biases from irrelevant concepts can be identified through observations from users, which were proposed to be mitigated through random image generation using GAN techniques. ConceptExplainer \cite{ConceptExplorerVisualAnalysisConcept} was designed to explore the concept associations focusing on validating conceptual overlapping between classes, especially serving as a concept exploration tool for non-expert users. In \cite{VisualConceptProgrammingVisualAnalytics}, a self-supervised technique was proposed to automatically extract visual vocabulary to allow experts to refine the labeled data and understand the concepts.

Unlike existing work, our study proposes an interactive workflow of exploring concepts for the purpose of inspecting systematic errors and spurious concept associations behind them. Similar to \cite{WhatDidMyAILearn}, our human-in-the-loop workflow aims to promote the sensemaking of practitioners specifically in the problem of systematic errors where they can iteratively work on subsetting, contrasting patterns in instances, and hypothesizing spurious associations.


% All these methods including.. share the idea of defining a concept vector with a group of semantically coherent segments. While we take the approach of pre-processing steps on concept space in [] and sensitivity, we expand the utility of concept exploration towards inspecting model's false behaviors. In our study, we demonstrate that using concept interpretability can help not only interpreting the concept association towards misclassificaitons, and tracing back ... then removing the biases to further improve the quality of classification.






\section{Methods}

 In this work, we conducted 20 semi-structured interviews to identify and understand whether and how computer science researchers from diverse sub-disciplines currently approach the potential UCs of their research innovations, what barriers they may encounter, and what design opportunities may exist to support this process. 
 
 \paragraph{Sampling and Participants.} We used purposive sampling to select researchers in computer science who were affiliated with \rr{institutions in North America with very high research activities (R1).} \rr{We focused on North American R1 institutions to reduce potential confounds related to the difference in academic culture and structures, such as funding applications and opportunities, requirements for promotion, and the structure of Ph.D. programs.} Participants were required to work on applied research that has led, or could lead, to systems used by the general public. 
 
 We recruited participants via email after reading their webpages and publications to determine whether their research met our criteria and to ensure diversity in levels of research experience. The email briefly described the research goal of finding ways to support researchers in anticipating UCs. We recruited participants until we reached a sample that satisfied our goal of interviewing computer science researchers \rr{from diverse disciplines and seniority levels} and until the interviews reached saturation. Although we attempted to sample researchers from a variety of applied sub-disciplines, our findings may not encapsulate the thoughts and actions of all computer science researchers. 

\begin{table}[t]
\footnotesize
\centering
\caption{Overview of CS researchers in the study.}
\label{table:reserachers}
\centering
\begin{tabular}{p{0.01\columnwidth}p{0.115\columnwidth}p{0.24\columnwidth}p{0.21\columnwidth}p{0.116\columnwidth}p{0.06\columnwidth}}
    \textbf{\#} & \textbf{Institution} & \textbf{Position}& \textbf{Research Area} & \textbf{Released Public Products} & \textbf{Gender}\\ 
    \toprule
     P1 &  Public & PhD Student & NLP, HCI & No & Male\\
     P2 & Private & Assistant Professor & Social Computing &Yes & Male\\
     P3 & Public & Assistant Professor & ML & No & Male\\
     P4 & Private & Associate Professor & Security, Smartphones, AI & Yes & Male\\
     P5 & Public & PhD Student & NLP & No & Male \\
     P6 & Public & Full Professor & NLP & Yes & Female\\
     P7 & Public & Full Professor & Robotics & Yes & Male\\
     P8 & Public & PhD Student & Computer Vision, ML & No & Male\\
     P9 & Private & PhD Student & AI & No & Male\\
     P10 &Private & PhD Student & AR, VR & No & Female \\
     P11 & Private & Assistant Professor & Brain-Computer Interfaces & No & Female\\
     P12 & Private & Postdoc & Accessibility & Yes & Female \\
     P13 & Private & PhD Student & Fabrication, Sensing & Yes & Male\\
     P14 & Private & Associate Professor &  HCI & Yes & Female \\
     P15 & Public & Assistant Professor & AR, Accessibility & Yes & Male\\
     P16 & Public & PhD Student & CS Education, HCI & No & Female \\
     P17 & Public & PhD Student & CS Education & Yes & Male\\
     P18 & Public & Assistant Professor & AI, Robotics & No & Male\\
     P19 & Public & Assistant Professor & Security & Yes & Male\\
     P20 & Public & PhD Student & NLP & Yes & Female\\
    \bottomrule
\end{tabular}
\end{table}

 Our final sample included 20 computer science researchers (7 female, 13 male) from 10 different academic institutions across North America. All participants had built systems as part of their research, and 9 had released one or more systems as (part of) a public product. Participants held various academic positions in their respective computer science departments (see Table~\ref{table:reserachers}):  10 participants were Ph.D. students or postdocs, and the remaining 10 were assistant, associate, or full professors. Our participants worked in a variety of research areas: 9 participants described their work as being mainly in AI or related areas (e.g. CV, ML, NLP). The remaining participants worked on accessibility, AR/VR, CS education, hardware, social computing, robotics, and security, or a combination of the above. All participants had industry experience, meaning that they have either collaborated, interned, or obtained full-time positions in industry while working toward their research projects.

\paragraph{Interview Protocol.} 

\rr{We prefaced our interviews by loosely defining "unintended consequences" as both desirable and undesirable outcomes of one's research, allowing participants to further elaborate on the term's meaning.} 
%\rr{We prefaced our interviews with a definition of ``unintended consequences of technology,'' which we loosely defined as unforeseen, desirable and undesirable outcomes of digital technology innovations. }%one's research to allow participants to further elaborate on the meaning of the word over the course of the interview.}
 We then divided our interviews into five sections: (1) participant research experience (e.g., research areas, educational and professional background); (2) prior experiences with UCs (e.g., from community norms in their sub-disciplines about considering UCs and/or their own research products have resulted in UCs); (3) understanding whether, when, and how they consider UCs in the research process; (4) understanding barriers to considering UCs in the research process; (5) understanding where researchers perceive opportunities to augment and improve the process to consider UCs. 
 %
 To avoid response bias, we started by explaining that the topic of UCs is relatively new to computer science. In addition to their own experience, we asked questions about habits and norms in researchers' labs and communities to better understand the context for their opinions and avoid participants feeling accused or put on the spot. We used additional questions to probe three topics that came up repeatedly: understanding whether researchers actively anticipated UCs, understanding what may hinder them from doing so, and understanding the design opportunities to support them in anticipating UCs. See Supplementary Materials for the complete list of interview questions. 
 
 Nineteen interviews were conducted remotely over Zoom, and one was conducted in person. All participants consented to being audio-recorded. Each interview was between 30-45 minutes long. We made a financial donation of \$15 per participant to a COVID-19 relief fund to compensate each participant for their time. The study was determined as exempt by our Institutional Review Board (IRB). 

\paragraph{Analysis.} 
% development code book, coding process, affinity diagramming
Our research team used an inductive thematic analysis process~\cite{braun2006using} where two researchers individually reviewed and conducted open coding on two interviews. Next, three researchers met to create and discuss the first draft of the codebook. Two members of the research team then independently coded several more interviews and refined or added them to the codebook, which was discussed with the full research team. Once consensus was reached, all interviews were re-coded using the revised codebook. 

The final codebook contained 12 top-level codes relevant to how researchers have thought about, experienced, and responded to UCs; it also included codes related to attitudes towards UCs and support researchers needed to think about UCs. Finally, three researchers used affinity diagramming to develop themes based on our codes. %these codes to organize our findings into three open questions. 
\rr{Although we discussed both positive and negative UCs in our interviews, the nature of our research led us to focus on participants' reports of negative UCs.} We slightly edited some of the quotes in Section \ref{Results} for readability. 

\paragraph{Positionality.}

 We acknowledge that our academic and professional backgrounds shape our perspectives on this topic. One author teaches computer ethics at an R1 institution. Collectively, we are US-based researchers at two R1 universities and a large US-based multinational corporation. Our academic backgrounds are in Computer Science, primarily as HCI researchers. 
\section{Results}\label{sec:Background}

And, so, since 2016, researchers have been probing the submitted methods, and in 2022 NIST published the final 10: ASCON, Elephant, GIFT-COFB, Grain128-AEAD, ISAP, Photon-Beetle, Romulus, Sparkle, TinyJambu, and Xoodyak. A particular focus is on the security of the methods, along with their performance on low-cost FPGAs/embedded processes and their robustness against side-channel attacks.

The current set of benchmarks includes running on an Arduino Uno R3 (AVR ARmega 328P), Arduino Nano Every (AVR ARmega 4809), Arduino MKR Zero (ARM Cortex M10+) and Arduino Nano 33 BLE (ARM Cortex M4F). These are just 8-bit processors and fit into an Arduino board. Along with their processing limitations, they are also limited in their memory footprint (to run code and also to store it). The lightweight cryptography method must thus overcome these limitations, and still, be secure and provide a good performance level. Running AES in block modes on these devices is often not possible, as there is not enough resources. Overall we use a benchmark for encryption — with AEAD (Authenticated Encryption with Additional Data) and for hashing. With AEAD we add extra information — such as the session ID — into the encryption process. This type of method can bind the encryption to a specific stream.



\subsection{ARM Cortex M3}

In Table \ref{tab:table01} [1], we see a sample run using an Arduino Due with an ARM Cortex M3 running at 84MHz. The tests are taken in comparison with the ChaCha20 stream cipher and defined for AEAD, and where the higher the value the better the performance. We can see that Sparkle, Xoodyak and ASCON are the fastest of all. Sparkle has a 100\% improvement, and Xoodyak gives a 60\% increase in speed over ChaCha20. Elephant, ISAP and PHOTON-Beetle have the worst performance for encryption (with around 1/20th of the speed of ChaCha20).

\begin{table*}
\caption{\label{tab:table01} Arduino Due with an ARM Cortex M3 running at 84MHz for encryption against ChaCha20 \cite{light01}}
\centering
\begin{tabular}{|l|l|l|l|l|l|l|l|l|}
\hline
Algorithm&Key Bits&Nonce Bits&Tag Bits&Encrypt 128~B&Decrypt 128~B&Encrypt 16~B &Decrypt 16~B&Aver
\\ \hline \hline
Schwaemm128-128 (SPARKLE)	&128	&128&	128	&1.6	&1.58	&2.84	&2.39	&2.01\\
Xoodyak 	&128	&128	&128	&1.66	&1.51	&1.73	&1.6	&1.62\\
ASCON-128	&128	&128&	128	&1.54	&1.44	&1.78	&1.68	&1.61\\
TinyJAMBU-128 	&128	&96	&64	&0.93	&0.95	&1.63	&1.61	&1.21\\
GIFT-COFB	&128	&128	&128	&1.01	&1.01	&1.16	&1.15	&1.08\\
Grain-128AEAD	&128	&96	&64	&0.26	&0.26	&0.56	&0.56	&0.37\\
Romulus-M1	&128	&128	&128	&0.1	&0.11	&0.15	&0.16	&0.13\\
PHOTON-Beetle-AEAD-ENC-128	&128	&128	&128	&0.06	&0.07	&0.11	&0.12	&0.08\\
ISAP-A-128	&128	&128	&128	&0.08	&0.08	&0.03	&0.04	&0.05\\
Delirium (Elephant)	&128	&96	&128	&0.04	&0.05	&0.06	&0.07	&0.05\\
\hline
\end{tabular}
\end{table*}

Not all of the finalists can do hash functions. Table \ref{tab:table02} outlines these.

\begin{table*}
\caption{\label{tab:table02} Arduino Due with an ARM Cortex M3 running at 84MHz for hashing against BLAKE2s \cite{NISTgov}}
\centering
\begin{tabular}{|l|l|l|l|l|l|}
\hline
Algorithm	& Hash Bits	& 1024 bytes	& 128 bytes	& 16 bytes	& Average\\
\hline\hline
Esch256 (SPARKLE) 	&256	&0.89	&0.78	&1.5	&1.06\\
Xoodyak 	&256	&0.71	&0.65	&1.43	&0.93\\
GIMLI-24-HASH	&256	&0.54	&0.47	&0.86	&0.62\\
ASCON-HASH 	&256	&0.51	&0.41	&0.63	&0.52\\
PHOTON-Beetle-HASH	&256	&0.01	&0.01	&0.05	&0.02\\
\hline
\end{tabular}
\end{table*}


Again, we see Sparkle and Xoodyak in the lead, with Sparkle actually faster in the test than BLAKE2s, and Xoodyak just a little bit slower. ASCON has a weaker performance, and PHOTON-Beetle is relatively slow. For all the tests, the ranking for authenticated encryption is (and where the higher the rank, the better):

14 SPARKLE
12 Xoodyak
12 ASCON
10 TinyJAMBU
9 GIFT-COFB, Gimli
4 Grain-128AEAD,KNOT
0 Elephant, ISAP, PHOTON-Beetle

and for hashing SPARKLE and Xoodyak are ranked the same:

7 SPARKLE, Xoodyak 5 Gimli 3 ASCON 0 PHOTON-Beetle

\subsection{Uno Nano performance}

For AEAD on Uno Nano Every [2], the benchmark is against AES GCM. We can see in \ref{tab:table03} , that SPARKLE is 4.7 times faster than AES GCM for 128-bit data sizes, and Xoodyak comes in second with a 3.3 times improvement over AES GCM. When it comes to 8-bit data sizes TinyJambu actually is the fastest, but where Sparkle and Xoodyak still perform well. PHOTON-Beetle, Grain128 and ISAP do not do well, and only slightly improve on AES GCM. In fact, Grain128 and ISAP are actually slower than AES GCM.




\begin{table*}
\caption{\label{tab:table03} Uno Nano for AEAD against AES GCM and showing cycles (showing fastest of the method)}
\centering
\begin{tabular}{|l|l|l|l|l|l|l|l|l|l|l|}
\hline
Algorithm&Impl.&Primary&Flag&Size&Enc(0:8)&Dec(0:8)&Enc(128:129)&Dec(128:128)&Bench.(128)&Bench.(8)
\\ \hline \hline
sparkle       &rhys	          &yes&	   O3	&12290	&1276	&1316	&4648    &5072  &4.7  &3.3\\
Xoodyak       &XKCP-AVR8	  &yes&    O3	&4560	&2596	&2608	&7184    &7128  &3.3  &1.6\\
knot	      &$avr8_speed$   &no&	   Os	&1664	&2124	&2140	&8144    &8160  &2.9  &2\\
ascon 	      &rhys	          &no&     O3	&5180	&1240	&1284	&8056    &8488  &2.8  &3.3\\
GIFT-COFB     &rhys	          &yes&    O1	&23312	&1852	&1892	&8220    &8776  &2.7  &2.2\\
saeaes	      &ref	          &no&     O3	&17062	&1208	&1212	&8992    &9004  &2.6  &3.4\\
hyena	      &rhys           &yes&    O3	&293860	&1912	&1964	&8960    &9396  &2.5  &2.2\\
elephant      &rhys           &no&     O3	&13106	&1924	&1948	&9260    &9796  &2.4  &2.2\\
estate	      &ref            &yes&    O3	&9434 	&1424	&1448	&10276   &10292 &2.3  &2.9\\
romulus	      &rhys           &no&     O3	&19346 	&1632	&1676	&10152   &10568 &2.2  &2.5\\
spook	      &rhys           &no&     O3	&12942 	&2984	&2968	&10272   &10708 &2.2  &1.4\\
tinyjambu     &rhys           &yes&    O3	&9174 	&1232	&1288	&10364   &10888 &2.2  &3.4\\
subterranean  &rhys           &yes&    Os	&6042 	&3372	&3460	&10288   &10944 &2.2  &1.2\\
orange        &rhys           &yes&    O3	&12140 	&2500	&2536	&11200   &11620 &2    &1.7\\
gimli         &rhys           &yes&    O3	&21272 	&1920	&1956	&11944   &12360 &1.9  &2.2\\
skinny        &rhys           &no&     O1	&12452 	&1604	&1644	&12960   &14372 &1.7  &2.6\\
photon-beetle & $avr8_speed$  &yes&    Os	&3536 	&2444	&2472	&20076   &20092 &1.2  &1.7\\
{\bf reference}&rhys          &yes&    O2	&7874 	&4152	&4156	&23812   &23764 &1    &1\\
grain128aead  &rhys           &yes&    O2	&9532 	&3992	&3980	&30396   &30124 &0.8  &1\\
isap          &rhys           &no&     O2	&3824 	&20212	&20256	&42936   &43372 &0.5  &0.2\\
\hline
\end{tabular}
\end{table*}

And so for AEAD  (performance) the ordering is

1. Sparkle
2. Xoodyak
3. Ascon
4. GIFT-COFB.
5. Elephant.
6. Romulus.
7. Tiny Jambu.
8. PHOTON-Beetle.
9. Grain128
10. ISAP.

For hashing on an Uno Nano Every, Table \ref{tab:table04} shows a similar performance level as to the ARM Cortex M3 assessment. In this case, the benchmark hash is SHA-256, and we can see that it takes Sparkle twice as many cycles for a 128-bit hash, and 2.9 times for Xoodyak. PHOTON-Beetle is way behind with a 128-bit hash and which is 17.4 times slower than SHA-256. That said, though, PHOTON-Beetle could be more focused on reducing power consumption rather than speed. GIMLI and SKINNY are included to show a comparison with well-designed methods in lightweight hashing. It can be seen that every method beats SKINNY, but only SPARKLE and Xoodyak beat GIMLI.


\begin{table*}
\caption{\label{tab:table04}  Uno Nano for hashing against SHA-256 and showing cycles (showing fastest of the method for hashing)}
\centering
\begin{tabular}{|l|l|l|l|l|l|l|l|l|l|l|}
\hline
Algorithm&Impl.&Primary&Flag&Size&h(8)&h(16)&h(32)&h(64)&h(128)&Benchmark
\\ \hline \hline
{\bf reference}&$nacl_ref$    &yes&    O3	&18774 	&768	&768	&772     &1364  &1968  &1\\
sparkle       &rhys	          &yes&	   O1	&7912	&1036	&1036	&1468    &2272  &3884  &2\\
Xoodyak       &XKCP-AVR8	  &yes&    O3	&2604	&1284	&1288	&1924    &3192  &5732  &2.9\\
gimli         &rhys           &yes&    O3	&19554 	&1284	&1920	&2544    &3804  &6312  &3.2\\
ascon 	      &rhys	          &yes&    O3	&2178	&2972	&3552	&4736    &7088  &11784 &6\\
drygascon     &rhys           &no&	   O3	&15500	&4604	&4600	&6540    &10360 &17912 &9.1\\
photon-beetle & $avr8_speed$  &yes&    O3	&2948 	&2372	&2364	&6940    &16084 &34172 &17.4\\
skinny        &rhys           &yes&    O2	&9784 	&7048	&10556	&13976   &20952 &34896 &17.7\\
\hline
\end{tabular}
\end{table*}


And so for hashing (performance) the ordering is:
\begin{enumerate}
    \item Sparkle.
    \item Xoodyak.
    \item Ascon
    \item PHOTON-Beetle. 
\end{enumerate}
    
This paper presented a comprehensive analysis of the use of \acrfull{PINN} for power system dynamic simulations. We show that \glspl{PINN} (i) are 10 to 1'000 times faster than conventional solvers, (ii) do not face issues of numerical instability unlike conventional solvers, and, (iii) achieve a decoupling between the power system size and the required solution time. However, \glspl{PINN} are less flexible (i.e. they do not easily handle parameter changes), and require an up-front training cost. Overall, this makes \gls{PINN}-based solutions well-suited for repetitive tasks as well as task where run-time speed is crucial, such as for screening.

Besides the comparison between conventional and \gls{NN}-based methods, this paper conducts a deeper analysis on the parameters that affect the performance of the \gls{NN} solutions. In that respect, we introduce a new \gls{NN} regularisation, called dtNN, as a intermediate step between \glspl{NN} and \glspl{PINN}. We show that \glspl{PINN} achieve overall higher levels of accuracy, and more balanced error distributions thanks to the evaluation of the collocation points.


\bibliographystyle{IEEEtran}
\bibliography{template}

% \appendix


\end{document}
