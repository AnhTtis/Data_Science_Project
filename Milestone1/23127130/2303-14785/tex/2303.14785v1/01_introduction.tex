


\section{Introduction}
While AES and SHA work well together within computer systems, they struggle in an IoT/embedded world as they take up: too much processing power; too much physical space; and consume too much battery power. So NIST outlines a number of methods which can be used for light-weight cryptography, and which could be useful in IoT and RFID devices \cite{buchanan2017lightweight}. They define the device spectrum as:

\begin{itemize}
\item    Conventional cryptography. Servers and Desktops. Tablets and smart phones.
\item    Light-weight cryptography. Embedded Systems. RFID and Sensor Networks.
\end{itemize}


With embedded systems, we commonly see 8-bit, 16-bit and 32-bit microcontrollers, and which would struggle to cope with real-time demands for conventional cryptography methods. And in the 40+ years since the first 4-bit processor, there is even a strong market for 4-bit processors. RFID and sensor network devices, especially, have limited numbers of gates available for security, and are often highly constrained with the power drain on the device.

So AES is typically a non-starter for many embedded devices. In light-weight cryptography, we often see smaller block size (typically 64 bits or 80 bits), smaller keys (often less than 90 bits) and less complex rounds (and where the S-boxes often just have 4-bits).

For light-weight cryptography the main constraints that we have are typically related to ener requirements, gate equivalents (GEs), and timing. With passive RFID devices, we do not have an associated battery for the power supply, and where the chip must power itself from energy coupled from the radio wave. An RFID device is thus likely to be severely constrained in the power drain associated with any cryptography functions, along with being constrained for the timing requirements and for the number of gates used. Even if an RFID device has an associated battery (active RFID), it may be difficult to recharge the battery, so the drain on power must often be minimised.

There is thus often a compromise between the cryptography method used and the overall security of the method. Thus often light-weight cryptography methods balance performance (throughput) against power drain and GE, and do not perform as well as main-stream cryptography standards (such as AES and SHA-256). Along with this, the method must also have a low requirement for RAM (where the method requires the usage of running memory to perform its operation) and ROM (where the method is stored on the device). In order to assess the strengths of various methods we often define the area that the cryptography function will use on the device – and which is defined in $\mu m^2$.

And, so, since 2016, researchers have been probing the submitted methods, and in 2022 NIST published the final 10: ASCON, Elephant, GIFT-COFB, Grain128-AEAD, ISAP, Photon-Beetle, Romulus, Sparkle, TinyJambu, and Xoodyak. A particular focus is on the security of the methods, along with their performance on low-cost FPGAs/embedded processes and their robustness against side-channel attacks. The time that the artile was written, NIST announced ASCON as the chosen method that will be published as NIST's lightweight cryptography standard later in 2023. Nevertheless in this article we provide a comparison between these methods in terms of energy efficiency, time for encryption and time for hashing using an Arduino Due with an ARM Cortex M3 running at 84MHz, Arduino Nano Every (AVR ARmega 4809), Arduino MKR Zero (ARM Cortex
M10+) and Arduino Nano 33 BLE.

