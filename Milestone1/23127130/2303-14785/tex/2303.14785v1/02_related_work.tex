
\section{Related work}
\label{related_work}



Bellini et al \cite{bellini2022randomness} outlined a set of tests for the NIST finalists related to their randomness. These tests were similar to the tests performed for the AES and the SHA-3 standards. The tests include: Avalanche Plaintext; Avalanche Key; Plaintext-Ciphertext correlation; Cipher Block Chaining Mode; Random; Low-Density with Plaintext ;Low-Density with Key; High-Density with Plaintext; and High-Density with Key. The main conclusion is that the underlying primitives for most of the methods produce datasets that seem random.  in the first third of the total number of rounds. 

Elsadek et al \cite{elsadek2022hardware} performed an evaluation on the NIST finalists. In terms of the size of each implementation (Table \ref{tab:els}, we see that TinyJambu and Grain128 have the smallest footprint, while Sparkle has by far the largest footprint. In terms of gate equivalent (GE), TinyJambu requires 3,600 GEs, while Sparkle needs 39,500.

\begin{table*}

\caption{\label{tab:els} Average energy efficiency \cite{elsadek2022hardware}}
\centering
\begin{tabular}{|l|l|l|l|}
\hline
& Area ($\mu m^2$ - synthesis over 22nm)&Area (kGE)\\
\hline\hline
TinyJambu&716&3.6\\
Grain128-AEAD&861&4.3\\
GIFT-COFB&1,618&8.1\\
Romulus&1,961&9.8\\
ASCON&21,93&11\\
Xoodyak&2,387&11.9\\
Photon-Beetle&2,523&12.6\\
ISAP&3,080&15.4\\
Elephant&3,458&17.3\\
Sparkle&7,897&739.5\\
\hline
\end{tabular}
\end{table*}



%\begin{figure*}
%\begin{center}
%\includegraphics[width=0.95\linewidth]{figures/ransom.png}
%\caption{Ransomware on IPFS %\cite{karapapas2020ransomware}}
%\label{ransom}
%\end{center}
%\end{figure*}







