\section{Introduction}

\begin{figure}[t!]
  \centering
  
   \def\svgwidth{0.45\textwidth}{\input{Zeichnung-1.pdf_tex}}
  
  \caption{\textbf{The classic NeRF framework compared to our NeRFtrinsic Four.} Training a \acs{nerf} is usually limited to one type of camera and requires known camera parameters. We present \approach which jointly optimizes the camera parameters ($\Pi$) of multiple diverse cameras without the necessity of a preprocessing step. Our approach utilizes Gaussian Fourier features (GF) to learn the extrinsic camera parameters. Furthermore, we individually optimize the intrinsic camera parameters per given camera.
  }
\end{figure}

Generating novel views and producing rich, photo-realistic images requires multiple camera angles from different viewpoints to generate a detailed 3D scene representation. For the generation of novel views the knowledge of the intrinsic and extrinsic camera parameters of the training images is crucial~\cite{heigl_plenoptic_1999,levoy_light_1996,mildenhall_nerf_2020,szeliski_stereo_1998}. Intrinsic camera parameters that are dependent on camera properties like focal length or pixel dimensions impact the image that represents a captured part of a scene. In turn, individual camera parameters influence the projection of pixels into the 3D coordinate space. This projection into the 3D space is required by \ac{nerf}~\cite{mildenhall_nerf_2020, meng_gnerf_2021, wang_nerf_2021} which use the intrinsic parameters for the ray projection. Besides intrinsic camera parameters, the camera angle and position, denoted as extrinsic, are decisive.

Traditional approaches use rich \ac{sfm} algorithms like COLMAP~\cite{schonberger_structure--motion_2016} to determine  camera parameters in a preprocessing step and later utilize these estimations to train the \ac{nerf}~\cite{martin-brualla_nerf_2021,mildenhall_nerf_2020}. Although \ac{sfm} enables training, the preprocessing step is always necessary for new data. This prevents the \ac{nerf} from being end-to-end capable. Moreover, standard \ac{sfm} algorithms highly depend on texture to estimate accurate camera parameters. 

One use case is the representation of a 3D scene based on images from varying cameras. To easily generate a 3D representation of such a scenario, the network should be capable of processing the images directly without requiring any preprocessing. To avoid the preprocessing step, existing approaches investigate the joint optimization of camera parameters and the \ac{nerf}~\cite{lin_barf_2021,meng_gnerf_2021,yen-chen_inerf_2021,wang_nerf_2021,xia_sinerf_2022}. Concurrent joint optimization approaches either require given intrinsic camera parameters~\cite{lin_barf_2021,meng_gnerf_2021}, assume a pretrained \ac{nerf} as given~\cite{yen-chen_inerf_2021} or are restricted to one single camera~\cite{wang_nerf_2021,xia_sinerf_2022}.

We propose \approach, a novel approach that optimizes diverse intrinsic and extrinsic camera parameters along with \ac{nvs}. Unlike existing approaches, our method is not constrained to one single camera and does not require a preprocessing step to estimate the camera parameters. It also does not rely on a pretrained \ac{nerf}. We demonstrate the effectiveness of our approach on three benchmarks, namely \acs{llff}~\cite{mildenhall_local_2019}, \acs{bleff}~\cite{wang_nerf_2021} and our own \datasetname. Our evaluation shows that \approach outperforms state-of-the-art joint optimization methods in terms of image quality and camera parameter estimation on \acs{llff}, \acs{bleff} and \datasetname. Overall, our approach provides a more versatile solution for handling real-world scenes with varying cameras.

In summary, our approach contributes:

\begin{itemize}
    \item A dynamic joint end-to-end trainable optimization framework, capable of handling  diverse cameras.
    \item A pose-\ac{mlp}, using Gaussian Fourier features for the handling of challenging poses.   
    \item Our novel \datasetname dataset focusing on the challenge of diverse cameras, on which we demonstrate the advantages of \approach\footnote{\href{https://github.com/HannahHaensen/nerftrinsic_four}{GitHub-\approach}}.
\end{itemize}
 