\chapter{Conclusions}
In this chapter we will give a summary of the main results of this thesis, as well as give indications for future research. The main conclusions are as follows:
\begin{enumerate}
    \item We have used the maximally symmetric ansatz of global AdS spacetime to fix the coupling needed to achieve the renormalization of the bulk action of an arbitrary HCG theory in the context of the Kounterterm renormalization scheme. We determined the coupling needed for the renormalization of the CCG theory using also this method. Using the self-replication property of the Kounterterms in codimension-2 submanifolds,  we proposed a renormalized Holographic EE formula for a CFT dual to CCG theory.
    
    \item  To test our proposal we  choose  two different entangling shapes: a hypersphere region in arbitrary CFT dimension and a cylinder in $d=4$ dimensions. For the case of the sphere, due to the symmetries that it possesses, it was possible to obtain a non-perturbative solution of the embedding of the codimension-two surface. We replace the solution of the embedding in our proposal to extract the universal part of the entanglement entropy. We were able to see that in this scenario the divergences were successfully cancelled. From this expression we were able to read off the F-quantity and the type-A anomaly of this theory. We note that the discrepancy with the values for CFTs dual to Einstein gravity is only of an overall coupling dependent factor.  
    
    For the cylindrical entangling region, we work perturbatively to obtain an embedding near the boundary. Using this and the Kouterterm method, we were able to read of the coefficient of the type-B anomaly from the renormalized EE.  
    
    
    \item We observed that for an arbitrary entangling region, the extrinsic curvature (with one covariant and one contravariant index) scales in terms of the holographic coordinate as $O(z)$ so the $S_{K^4}$ term producing the difference between the splittings should scale as $O(z^4)$. With this in mind, we were able to show that for $d=2$ and $d=4$ it is impossible to distinguish between the minimal and non-minimal prescriptions from the universal part of the EE. This occurs because the part of the EE corresponding to $S_{K^4}$ scales as $O(z^{d-5})$, so it will not contribute to the universal EE.


    \item  Finally, we consider a deformed sphere as an entangling region in a $CFT_3$ with the deformation parameter being $\epsilon$. Using the CHM map it is possible to relate the EE of a ball-shaped region of flat space to the free energy of a thermal CFT in a spherical background. The perturbation around the $\epsilon$ parameter can be considered as a variation of the free energy. Therefore, it is possible to relate the correlation functions of the energy momentum tensor to the terms of the series expansion of the EE for the deformed sphere. 

    
    The coefficient of the two-point correlation function $C_T$ is related to the $S^{(2)}_{CCG}$ term in the expansion of the EE around $\epsilon$. The fourth-order term of the expansion is then related to $C_T$ and $t_4$ which are the coefficients appearing in the calculation of the fourth-point correlation function of the momentum energy-momentum tensor.
    
    To obtain the $C_T$ coefficient for a general CCG theory we use the result of Einstein gravity to fix the polynomial in $l$, and therefore we can then read $C_T$ off from the $S^{(2)}_{\epsilon}$ term. In Einstein gravity the $t_4$ is zero and therefore, in this case, we can not use it to fix the polynomial in $S^{(4)}_{CCG}$. 

    Thus, we use the massless limit of Ref.  \cite{Li:2019auk} to obtain the polynomial \eqref{P2} in the Fourier expansion. By tracking it in the entropy functionals of Eqs.\eqref{SCCGmin t4} and \eqref{SCCGnonmin t4} we were able to arrive at two different expressions for $t_4$ (one for the minimal and the other for the non-minimal). Our conclusion is that only the non-minimal prescription is consistent. As a consistency check we computed the $t_4$ for the Einsteinian cubic gravity case and saw that it matches with the results obtained in Ref.\cite{Bueno:2018xqc}. 

    \end{enumerate}

In the process of conducting this thesis we realized, based on a power counting argument, that the Kounterterm method also works to achieve the renormalization of the EE for any higher curvature gravity theory in $D=4$. This  has already been discussed in  Ref.\cite{Anastasiou:2022pzm} in which the author of this thesis was involved.


To conclude with this thesis, we will mention a possible avenue for future research.  As previously mentioned, the appearance of CFT coefficients of the stress tensor in the expansion of the HEE around the deformation parameter suggests the possibility of extending the CHM map. This means that the HEE calculated on the deformed spherical surface could be represented as the partition function on a similarly deformed thermal sphere. It would be intriguing to derive a comprehensive mapping between the distorted Euclidean-sphere manifold and the perturbed spherical entangling surface in terms of their respective deformation parameters. 






