\chapter{Introduction}

Just as the study of  black body radiation led us to the formulation of quantum mechanics, the study of the statistical properties of black holes (BH) could help us to elucidate aspects of a theory of quantum gravity. These are objects where the effect of gravity is very intense and present interesting characteristics such as temperature and  emission of radiation \cite{Hawking:1974rv,Hawking:1975vcx}. They also have an associated entropy that is proportional to the area of its event horizon \cite{Hawking:1974rv, Bekenstein:1972tm,Bekenstein:1973ur,Bekenstein:1974ax}. The fact that the BH entropy  serves as an upper bound to the entropy that can be contained in a given region of spacetime \cite{Bekenstein:1980jp} \footnote{The formulation of this statement, called the Bekenstein bound, only works under certain assumptions such as weak gravity, spherical symmetry, etc. These conditions are violated by real physical systems and therefore a better formulation of a strong bound on the information content of spacetime regions was sought. Its present formulation is described by the covariant entropy bound \cite{Bousso:1999xy}.} inspired the formulation of the holographic principle \cite{tHooft:1993dmi,Susskind:1994vu}. It tells us that the fundamental degrees of freedom of a physical system in a given volume are encoded by the area that delimits that region.  In Ref. \cite{Susskind:1994vu} it was explained how this idea could be implemented in the context of string theory and finally in Refs. \cite{Maldacena:1997re, Witten:1998qj,Gubser:1998bc,Aharony:1999ti} the first concrete realization of the holographic principle was obtained in the famous anti-de Sitter/conformal field theory (AdS/CFT) correspondence\footnote{Also known as gauge/gravity duality. }. It states that in the limit of large $N$, a $d$-dimensionsal $SU(N)$ gauge theory with conformal symmetry is equivalently described by a superstring theory in a $(d+1)$-dimensional background of anti-de Sitter type (for more details go to section (\ref{HEE})). 

Since its formulation, the correspondence  has been an important step towards the understanding of the relationship between string theory, quantum field theory and general relativity. Moreover, it encounters applications in the context of  QCD \cite{Kruczenski:2003uq, Sakai:2005yt,Erlich:2005qh}, nuclear physics \cite{Mateos:2007ay}, non-equilibrium physics \cite{Keranen:2014lna}, quantum information \cite{Chen:2021lnq} and condensed matter physics \cite{Hartnoll:2007ai,Hartnoll:2007ih}. One of the most important claims made by the conjecture is that strongly coupled gauge theories can be studied using classical gravity. This property, has been exploited in many situation as a computational tool to translate difficult problems into simpler ones.  For a \textit{real world} example we can see that it turns out that one prediction of AdS/CFT is indeed close to the experimental results of the real quark–gluon plasma \cite{Kovtun:2004de}.



 On the other side of the story we have quantum entanglement, a property of many-body quantum systems (and quantum field theories) that will be described in great detail in sections (\ref{EE in QM}) and (\ref{EE in QFT}). The entanglement entropy, which is a specific measure of it, has been utilized as an order parameter to describe phase transitions and critical points in various phases of matter\cite{Kitaev:2005dm,Klebanov:2007ws}. In the 1980s and 1990s, several authors notice that the leading divergent term of the EE of a region of interest is proportional to its surface area \cite{Bombelli:1986rw, Srednicki:1993im}. As a result of this, it was proposed that the entropy of black holes might arise, at least in part, from the entanglement of quantum fields across the horizon \cite{tHooft:1984kcu, Susskind:1993ws, Callan:1994py}. Furthermore, a remarkable success on this topic was achieved in \cite{Strominger:1996sh} where, in the context of string theory, a microscopic derivation of the BH entropy for BPS black holes was provided.



 
 

Despite the many similarities, a concrete gravitational interpretation of entanglement entropy had not yet been formulated in the early 2000s. It was not until the second half of that decade, that S. Ryu and T. Takayanagui presented a proposal, based on holographic arguments, for this missing link \cite{Ryu:2006bv,Ryu:2006ef}. The conjecture, simply and elegantly, states that the EE for a CFT that admits a holographic description can be computed by finding a minimal area of a codimension-two hypersurface in the dual gravitational theory. From its formulation to the present day, much has been written on the subject. Ideas on how bulk geometry can arise from entanglement have been developed in Refs.\cite{VanRaamsdonk:2009ar, VanRaamsdonk:2010pw}, and beyond that it has been used as an easy way to obtain the EE for situations that would otherwise be intractable.

In recent years, higher curvature gravity theories (HCG) have been studied in the context of the AdS/CFT correspondence. Usually, they  appear as stringy or quantum correction in effective (super)gravity theories. A property of holographic CFTs that have Einstein gravity as their dual is that their central charges are equal $a=c$ \cite{Henningson:1998gx}. If we want to describe more general field theories such as strongly coupled CFTs with $a\neq c$, higher curvature terms have to be incorporated in the gravitational side of the description. This correction to the Einstein-Hilbert (EH) case have had non trivial consequences in the description of holographic fluids \cite{Brigante:2007nu} and in conformal collider physics \cite{Hofman:2008ar}. In addition, due to the limitations of working in theories that are dual to EH, the computation of EE has also been explored in more spicy CFTs dual to higher order curvature theories. In this thesis we will pay special attention to the case of cubic curvature gravity theories (CCG). Recently, a description for the EE of the CFT dual to these cubic theories was found  \cite{Caceres:2020jrf, Bueno:2020uxs}. In this project, we will see their description and consequences, as well as provide new results on their characterization. 

In the next chapter we present an overview of all the topics mentioned so far in this introduction. We start by giving a definition of entanglement in quantum mechanics (section (\ref{EE in QM})) and then we measure it using the bedrock concept of statistical mechanics: entropy. In section (\ref{EE in QFT}),  we  extend these concepts to the domain of quantum field theory by providing the appropriate technology to obtain the EE in such scenarios. After that, we will give a brief introduction to the AdS/CFT correspondence that will be used to demonstrate the Ryu-Takayanagi (RT) formula (see section \ref{HEE}). The EE carries universal information related to fundamental properties of the CFT. In order for this information to be extracted a renormalization procedure should be implemented. For this purpose, we will use a scheme inherited from the renormalized bulk action based on the addition of extrinsic curvature counterterms \cite{Olea:2005gb,Olea:2006vd} (Kounterterms (\ref{Kounterterms section})). 


In chapter \ref{chapter 3}, we will start by determining  the renormalized CCG action using the Kounterterm method (section (\ref{Fixing the coupling in CCG})). In section (\ref{Renormalized HEE in CCG}) we obtain an expression for the renormalized EE in CCG.  On top of that, we give some remarks about the splitting problem \cite{Miao:2015iba,Miao:2014nxa} that arises in the determination of the EE. This is consequence of the several ways to do the regularization process (that give us different final results) of the conic singularities present in the computation of the EE in CCG\footnote{We will talk much more about this later.}. Then, we will test our proposal to  recover the universal part of the EE for spherical (section (\ref{First test: Hypersphere})) and cylindrical (section (\ref{Second test: Cylinder})) entangling regions. Finally, we consider harmonic deformations of spherical entangling surfaces to obtain the $C_{T}$ charge of the two-point function of the CFT stress-tensor, in terms of the lowest order term in the deformation of the entanglement density on the deformed region (section (\ref{Deformed sphere I})). We also explore the computation of higher correlators of the stress tensor from higher order terms in the deformation parameter (section (\ref{Deformed sphere II})).

In the last chapter we present the conclusions and final remarks. We will also discuss some open questions that could lead to future research work.






