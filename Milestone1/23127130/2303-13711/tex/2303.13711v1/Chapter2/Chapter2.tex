\chapter{Results on HEE in CCG} \label{chapter 3}

\section{Fixing the coupling in CCG} \label{Fixing the coupling in CCG}
 
In this section  we are going to follow the procedure described in section (\ref{Kounterterms section}) to fix the coupling for the Kounterterm that will render the generic cubic curvature gravitational action finite (without considering terms of covariant derivatives of the curvature tensor).  The prescription tells us that now we have to evaluate the Lagrangian of Eq.\eqref{CubicLagrangian} in the pure $AdS_{d+1}$ solution and then take the derivative with respect to the effective radius to find the coupling that will achieve the renormalization of the action. In this ansatz the cubic curvature invariants become:

\begin{minipage}{.47\textwidth}
 \begin{align}
&R_{\mu\:\:\nu}^{\:\:\rho\:\:\sigma}R_{\rho\:\:\sigma}^{\:\:\lambda\:\:\tau}R_{\lambda\:\:\tau}^{\:\:\mu\:\:\nu}=-\frac{d(d^2-1)}{\ell^6_{eff}}\label{maximal Riemmann}\\
 &R^{\mu\nu}_{\:\:\:\:\rho \sigma}R^{\rho \sigma}_{\:\:\:\:\lambda \tau}R^{\lambda\tau}_{\:\:\:\:\mu\nu}=-\frac{4d(d+1)}{\ell^6_{eff}}\\
 &R^{\mu\nu\rho}_{\:\:\:\:\:\:\:\sigma}R_{\mu\nu\rho\tau}R^{\sigma\tau}=-\frac{2d^2(d+1)}{l^6_{eff}}\\
 &R_{\mu\nu\rho\sigma}R^{\mu\nu\rho\sigma}R=-\frac{2d^2(d+1)^2}{\ell^6_{eff}}
\end{align}
\end{minipage}
\begin{minipage}{.47\textwidth}
\begin{align}
  &R^{\mu\rho}R^{\nu\sigma}R_{\mu\nu\rho\sigma}=-\frac{d^3(d+1)}{\ell^6_{eff}}\\
  & R R_{\mu}^{\:\:\nu}R_{\nu}^{\:\:\rho}R_{\rho}^{\:\:\mu}=-\frac{d^3(d+1)}{\ell^6_{eff}}\\
  &R_{\mu\nu}R^{\mu\nu}R=-\frac{d^3(d+1)^2}{l^6_{eff}}\\
  &R^3=-\frac{d^3(d+1)^3}{\ell^6_{eff}}.\label{maximal Ricci}
\end{align}
\end{minipage}\\

Notice that the cosmological constant does not depend explicitly on $\ell_{eff}$. However, after imposing the equation of motion it is possible to see that the cosmological constant is a polynomial on $1/\ell_{eff}$ with coupling dependent coefficients. After that one can replace the expression obtained in the Lagrangian to eliminate the $\ell_0$ dependence of it. This process is very messy and large so in section (\ref{Kounterterms section}) we provided a shortcut to computed it. We start by replacing Eq.\eqref{maximal Riemmann}-\eqref{maximal Ricci} in Eq.\eqref{CubicLagrangian}:
 \begin{align}
 \mathcal{L}=&-\frac{d(d+1)}{\ell^2_{eff}}-2\Lambda_0-\frac{d(d+1)}{\ell^6_{eff}}[\mu_1(d-1)+4\mu_2+2\mu_3d+2\mu_4d(d+1)\nonumber\\&+\mu_5d^2+\mu_6d^2+\mu_7d^2(d+1)+\mu_8d^2(d+1)^2].  \end{align}
Taking the derivative with respect to $\ell_{eff}$ the cosmological constant vanishes and therefore we do not have to use the EOMs explicitly to fix the coupling. Taking the derivative we get:
 \begin{align}
   \frac{d\mathcal{L}}{d\ell_{eff}}=&\frac{2d(d+1)}{\ell^3_{eff}}+\frac{6d(d+1)}{\ell^7_{eff}}[\mu_1(d-1)+4\mu_2+2\mu_3d+2\mu_4d(d+1)\nonumber\\&+\mu_5d^2+\mu_6d^2+\mu_7d^2(d+1)+\mu_8d^2(d+1)^2].
 \end{align}
After replacing Eq.\eqref{HigherCurvaturecoupling}, we obtain the expression for the coupling of this particular theory for even and odd  bulk dimension as:
\begin{equation}
  c_d^{CCG} =
    \begin{cases}
     a_{d} \frac{(-1)^{\frac{d+1}{2}}\ell_{eff}^{d-1}}{(\frac{d+1}{2})(d-1)!} & \mbox{if $d$ is odd};\\
     a_{d} \frac{(-1)^{\frac{d}{2}}\ell_{eff}^{d-2}}{2^{d-3}d((\frac{d}{2}-1)!)^2} & \mbox{if $d$ is even,}
    \end{cases}   \label{coupling HEE in CCG}  
\end{equation}
where
\begin{align}
 a_d=&1+\frac{3}{\ell^4_{eff}}\big(\mu_1(d-1)+4\mu_2+2\mu_3d+2\mu_4d(d+1)+\mu_5d^2\nonumber\\&+\mu_6d^2 +\mu_7d^2(d+1)
 +\mu_8d^2(d+1)^2\big).\label{ad}
\end{align}
A nontrivial test is that after fixing the Lagrangian couplings to correspond to Lovelock theory in $d=6$, the Eq.\eqref{ad} reproduce the result of Refs.\cite{Anastasiou:2021jcv, Kofinas:2007ns}. Now, we assume that the same coupling that achieve the renormalization of pure AdS also renormalize any other AlAdS action (in $D\leq 5$). Therefore, we proposed that the renormalized gravitational action is given by :
\begin{equation}
    I_{CCG}^{ren}=\frac{1}{16\pi G}\int_{\mathcal{M}}d^{d+1}x\sqrt{|G|}\mathcal{L}_{CCG}+\frac{c^{CCG}_d}{16\pi G}\int_{\partial \mathcal{M}}d^{d}x\sqrt{|h|} B_d.\label{RenormalizedCubicCurvatureAction}
\end{equation}
Now that we have our renormalized gravitational action, in the next section, we will exploit the self-replication property of the Kounterterms to obtain an expression for the $S_{KT}$, which will isolate the universal component of the EE.

\section{Renormalized HEE in CCG}\label{Renormalized HEE in CCG}

In section (\ref{renormalized EE}) we explore a procedure inherited from the bulk renormalization by Kounterterms to cancel the divergences appearing in the EE. This procedure have already been used in the context of higher curvature gravity theories. In Ref.\cite{Anastasiou:2021swo} a prescription was given for a generic quadratic curvature theory and in Ref.\cite{Anastasiou:2021jcv} the authors achieve the renormalization of the EE in Lovelock gravity. It should be noted that in these two instances, the splitting problem (outlined in section (\ref{HEE})), did not arise, implying that the minimal and non-minimal prescriptions yielded the same HEE functional. In contrast, for CCG, as stated in Ref. \cite{Caceres:2020jrf}, two distinct functionals would emerge. Based on what we have developed until now, we proposed that the universal part of the EE for the two splitting can be written as:

\textbf{Minimal Splitting}

\begin{equation}
   S_{CCG}^{Univ-min} =S_{CCG}^{min}+S_{KT}.\label{minimal Spliting}
\end{equation}

\textbf{Non-minimal Splitting}


\begin{equation}
  S_{CCG}^{Univ-non-min} =S_{CCG}^{non-min}+S_{KT},\label{non-minimal Spliting}
\end{equation}

where the expressions for $S_{CCG}^{min}$ and $S_{CCG}^{Univ-non-min}$ can be found in Eq.\eqref{HEE in CCG minimal} and Eq.\eqref{HEE in CCG nonminimal}. In this case the codimension-three Kouterterm is :
\begin{equation}
    S_{KT}=\frac{c_d^{CCG}}{4G}\bigg\lfloor \frac{d+1}{2}\bigg\rfloor\int_{\partial\Sigma}d^{d-2}x\sqrt{\tilde{\sigma}}B_{d-2}.\label{HEE in CCG Kounterterm}
\end{equation}

It is easy to see where this expression comes from. One can use the replica trick for the renormalized bulk action of Eq.\eqref{RenormalizedCubicCurvatureAction}. The divergent part can be evaluated using Dong's functional \cite{Dong:2013qoa}  (minimum) or Camp's functional  \cite{Camps:2013zua,Camps:2016gfs} (non-minimum) as was done in the paper mentioned above  (E. Caceres, et.al. \cite{Caceres:2020jrf}) and the contribution of the Kounterterms can be obtained as in section  (\ref{renormalized EE}) but with the CCG coupling of Eq.\eqref{coupling HEE in CCG}. As mentioned before, the differences between the two splittings come from the term $S_{K^4}$ that depends exclusively of contractions of four extrinsic curvature tensor as you can see in Eq.\eqref{SK4 minimal} and Eq.\eqref{SK4 nonminimal}. If we consider the Fefferman-Graham (FG) coordinates for the bulk metric  \cite{fefferman1985elie} where the conformal boundary is located at $z=0$ we can easily find that, near the boundary, the extrinsic curvature of the codimension-two surface scales as $O(z)$\footnote{This is the case for the extrinsic curvature expressed with one contravariant and one covariant index. The reason to be interested in this particular form of the extrinsic curvature is that any term in $S_{K^4}$ can be expressed as contraction of this object.} meaning that the  $S_{K^4}$ should scales as $O(z^4)$ for the two splittings. Considering that the squared root of the determinant of the induced metric of the cod-2 embedding scales as $O(z^{-(d-1)})$ we  notice that the term in the FG expansion that contributes to the logarithmic anomaly of the universal EE in even dimension is  $z^{d-2}$. We can see that for $d=2$ and $d=4$ the $S_{K^4}$ term will not contribute and therefore the universal part for the two splittings will be the same. In odd dimensional CFTs the universal part is finite, therefore the upper limit in the integration of $z$ of the $S_{K^4}$ will always contribute to it meaning that we can always find a difference between the the universal part of each splitting. As a side note, we should notice that we are not going to see a difference in the divergence structure unless we go to $d=7$ dimensions or higher due to the fact that $\sqrt{\sigma}S_{K^4}=O(z^{d-5})$.


Finally, it should be noted that this renormalization method allows us to determine the Weyl anomaly, present in even-dimensional CFTs, which is characterized by the coefficient that multiplies the logarithmic divergence of the EE. Considering an entanglement region $A$, whose width is described by $L$, we would obtain that the associated renormalized EE is $S^{\text{Univ}} \left[A\right]=c_0\ln(\frac{L}{\delta})$, where: 

\begin{equation}
    c_0=(-1)^{d/2+1}2A\chi[\partial \Sigma]-\sum_{i}C_i\left(\partial_n\int\limits_{\mathcal{M}_n}I_i\right) \,.\label{central charges}
\end{equation}

The first coefficient on the right-hand side of Eq.\eqref{central charges} is the type-$A$ central charge, which is multiplied by the Euler characteristic denoted by $\chi[\partial \Sigma]$. The second part is formed by the coefficients $C_i$, which represent the type-$B$ central charges, and the $I_i$, which are the conformal invariants in that dimension. The  $A$ charge will be the only relevant factor for universal EE when a ball-shaped entanglement region is chosen, while for a cylinder-shaped region, only the $B$ charge will survive. 



The upcoming analysis will involve extracting the universal component of the EE for highly-symmetric shapes in a CFT that has a CCG bulk dual. This will be achieved through the use of the renormalization scheme outlined in Eqs.\eqref{minimal Spliting} and \eqref{non-minimal Spliting}. With this, we determine the $A$ and $F$ charges \cite{Jafferis:2011zi,Klebanov:2011gs}. These two quantities are expected to characterize the degrees of freedom of the theory and to decrease monotonically along the renormalization group flow \cite{Myers:2010xs,Myers:2010tj}. Next, in section \ref{Second test: Cylinder}, we compute the EE for a cylindrical region in a four-dimensional CFT and, from it, extract the type-$B$ charge.




\section{First test: Hypersphere} \label{First test: Hypersphere}

We want to obtain the EE for the vacuum state for an entangling region $A$ that has the shape of an hypersphere. The corresponding bulk of this state is  pure $AdS_{d+1}$, and its metric in Poincaré coordinates is:

\begin{equation}
    ds^2=\frac{\ell^2_{eff}}{z^2}\left(d\tau^2+dz^2+dr^2+r^2d\Omega_{d-2}^2\right), \label{Poincare}
\end{equation}
 
 where $\Omega_{d-2}$ represents the angular directions of an $\mathbb{S}^{d-2}$ sphere. Now, the region of interest in a Cauchy slice $\tau=cte$ of the CFT is going to be described by:
 \begin{equation}
     A:\left[\tau=cte, z=\delta \:\:and\:\: r\leq R\right], \label{Entangling A}
 \end{equation}
with entangling surface located in $r=R$ and also we have introduce a cutoff at $z=\delta$. The bulk extremal surface could be found by assuming that the $r$ coordinate extends to the bulk and due to the radial symmetry of the entangling surface the embedding should have the form $r=f(z)$. Replacing in the EOMs of the theory one obtains a differential equation for $f(z)$ whose solutions reads  $f(z)=\sqrt{R^2-z^2}$. Therefore the embedding\footnote{Here we change the notation a bit and call $\Sigma$ to the hypersurface of codimension two in the bulk where the EE calculation is located. The induced metric would be $\sigma$ with indices $\bar{\mu},\bar{\nu}$. } is a spherical hemisphere of same radius: 
  \begin{equation}
     \Sigma:\left[\tau=cte,  r^2+z^2=R^2\right] \label{Entangling region bulk}
 \end{equation}

 where $\partial\Sigma= \partial A$ meaning that is anchored the entangling regions.
 In Poincare coordinates, the induced metric reads:
 \begin{equation}
    ds_{\Sigma}^2=\sigma_{\bar{\mu}\bar{\nu}}dy^{\bar{\mu}}dy^{\bar{\nu}}=\frac{\ell^2_{eff}}{z^2}\left(\frac{R^2}{R^2-z^2}dz^2+(R^2-z^2)\Omega_{d-2}^2\right) ,\label{induced metric}
\end{equation}
 with normal vectors
 \begin{equation}
     n^{(1)}_{\mu}=\left(\frac{\ell_{eff}}{z},0,0,...,0\right) \quad;\quad n^{(2)}_{\mu}=\left(0,\frac{\ell_{eff}}{\sqrt{r^2+z^2}},\frac{\ell_{eff}r}{z\sqrt{r^2+z^2}},...,0\right). \label{normal vector sphere}
 \end{equation}
Due to the fact that the extrinsic curvature can be formulated in terms of a Lie derivative with respect to the normal vectors, we can see that all components of $ K_{\bar{\mu}\bar{\nu}}^{(1)}$ and $ K_{\bar{\mu}\bar{\nu}}^{(2)}$ must vanish.  This is because the normal vectors to this hypersurface are  Killing vectors of the space. From this observation we realize that in this particular case the minimum and non-minimum splitting will be equal because $S_{K^4}^{min}=S_{K^4}^{non-min}=0$. Moreover the term $S_{K^2R}$ would also be zero and therefore the EE would be:
\begin{equation}
    S_{CCG}^{min} = S_{CCG}^{non-min}= S_{CCG} =\frac{\mathcal{A}[\Sigma]}{4 G}-\frac{1}{8 G}\int_{\Sigma}d^{d-1}x\sqrt{\sigma}S_{R^2}.
\end{equation}
To find the term $S_{R^2}$ we have to compute all the squared curvature invariants appearing in Eq.\eqref{SR2} using the metric of Eq.\eqref{Poincare} and the normal vectors of \eqref{normal vector sphere}:

\begin{minipage}{.47\textwidth}
 \begin{align}
& R^2=\frac{d^2(d+1)^2}{\ell^4_{eff}}\label{R2}\\
    &R_{\mu\nu}R^{\mu\nu}=\frac{d^2(d+1)}{\ell^4_{eff}}\\
    &R_{a}^{\:\:\:a}R=\frac{2d^2(d+1)}{\ell^4_{eff}}\\
    &R_{a\mu}R^{a\mu}=\frac{2d^2}{\ell^4_{eff}}\\
   & R_{\mu\nu}R_{a}^{\:\;\mu a\nu}=\frac{2d^2}{\ell^4_{eff}}\\
  & R_{ab}R^{ab}=\frac{2d^2}{\ell^4_{eff}}\\
   & R_{a}^{\:\:a}R_{b}^{\:\:b}=\frac{4d^2}{\ell^4_{eff}}
\end{align}
\end{minipage}
\begin{minipage}{.47\textwidth}
\begin{align}
  &R_{\mu\nu\rho\sigma}R^{\mu\nu\rho\sigma}=\frac{2d(d+1)}{\ell^4_{eff}}\\
    &RR_{ab}^{\:\:\:\:ab}=\frac{2d(d+1)}{\ell^4_{eff}}\\
   & R_{a\mu\nu\rho}R^{a\mu\nu\rho}=\frac{4d}{\ell^4_{eff}}\\
    &R_{a\mu}R_{b}^{\:\:a b\mu}=\frac{2d}{\ell^4_{eff}}\\
    &R_{a b \mu \nu}R^{a b \mu \nu}=\frac{4}{\ell^4_{eff}}\\
   & R_{a\mu b\nu}R^{a \nu b\mu}=\frac{2d}{\ell^4_{eff}}\\
   & R_{a\mu\:\:\nu}^{\:\:\:\:\:a}R_{b}^{\:\:\mu b\nu}=\frac{2(2d-1)}{\ell^4_{eff}}.\label{R last}
\end{align}
\end{minipage}\\


Replacing Eqs.\eqref{R2}-\eqref{R last} in Eq.\eqref{SR2} gives:
 \begin{align}
      S_{R^2}=&-\frac{6}{\ell^4_{eff}}\left(\mu_1(d-1)+4\mu_2+2\mu_3d+2\mu_4d(d+1)+\mu_5d^2+\mu_6d^2\right.\nonumber\\& \left. +\mu_7d^2(d+1)+\mu_8d^2(d+1)^2\right) .
 \end{align}
Notice that this term does not depend on the coordinates and therefore its contribution to the EE   would be proportional to the area of the codimension-two hypersurface. Therefore the whole EE would be just proportional to the area and we can see that the proportionality constant is related to $a_d$ of Eq.\eqref{coupling HEE in CCG}. Thus:
 \begin{equation}
    S_{CCG}^{Univ} =\frac{a_d\mathcal{A}[\Sigma]}{4 G}  .
 \end{equation}
If we add the EE Kounterterm of Eq.\eqref{HEE in CCG Kounterterm} and express $c_d^{CCG}=a_dc_d^{EH}$ we can obtain an equation for the Universal part of the EE, as shown below: 
\begin{equation}
     S_{CCG}=\frac{a_d}{4 G}\left(\mathcal{A}[\Sigma]+c_d^{EH}\bigg\lfloor \frac{d+1}{2}\bigg\rfloor\int_{\partial\Sigma}d^{d-2}x\sqrt{\tilde{\sigma}}B_{d-2}\right)=\frac{a_d}{4G}\mathcal{A}^{Univ}[\Sigma].\label{renormalized EE for a sphere}
\end{equation}
It can be observed from that the universal HEE for this region is proportional to the universal part of the minimum surface area. The explicit form of $\mathcal{A}^{Univ}[\Sigma]$ has already been obtained in \cite{Anastasiou:2017xjr, Anastasiou:2018rla} (Also see Appendix of \cite{Anastasiou:2021swo}) :
\begin{equation}
\mathcal{A}^{Univ}[\Sigma]=
    \begin{cases}
      (-1)^{\frac{d-1}{2}}\frac{2^{d-1}\pi^{\frac{d-1}{2}}\ell_{eff}^{d-1}}{(d-1)!} & \mbox{if $d$ is odd};\\
      
      
      (-1)^{\frac{d}{2}-1}\frac{2\pi^{\frac{d-1}{2}}\ell_{eff}^{d-1}}{(\frac{d}{2}-1)!}\log{\left(\frac{2R}{\delta}\right)} & \mbox{if $d$ is even}.\\
    \end{cases}   \label{Area universal}    
\end{equation}
Having successfully isolated the universal part of the EE, it is time to ask ourselves what information we can obtain from it. From \cite{Myers:2010xs,Myers:2010tj} we  note that the candidates to the $C$-function  for even and odd CFTs can be read from Eq.\eqref{renormalized EE for a sphere}. This quantities are important because they have been  conjectured to decrease along the renormalization group (RG) flow  \cite{Zamolodchikov:1986gt,Jafferis:2011zi,Klebanov:2011gs,Cardy:1988cwa,Komargodski:2011vj}. In this particular scenario we have that :
\begin{equation}
S_{CCG}^{Univ}[\Sigma]=
    \begin{cases}
      (-1)^{\frac{d-1}{2}}F & \mbox{if $d$ is odd};\\
     
      (-1)^{\frac{d}{2}-1}4 A \log{\left(\frac{2R}{\delta}\right)}& \mbox{if $d$ is even,}\\
    \end{cases}     
\end{equation}
where
\begin{equation}
    F=a_d\frac{2^{d}\pi^{\frac{d-1}{2}}\ell_{eff}^{d-1}}{8G(d-1)!}  \quad;\quad A=a_d\frac{\pi^{\frac{d-1}{2}}\ell_{eff}^{d-1}}{8G(\frac{d}{2}-1)!} .\label{anomaly sphere}
\end{equation}

The Eq.\eqref{anomaly sphere} are the corresponding $F$-quantity  and type-$A$ charge of this theory. It is worth noting that these results are identical to those of Einstein-AdS gravity, except for an overall factor $a_d$ which depend on the CCG couplings.

\section{Second test: Cylinder}\label{Second test: Cylinder}
Analogous to the methodology used in the previous section, let us begin by considering the Euclidean space $AdS_5$ expressed in polar coordinates as:
\begin{equation}
    ds^2=\frac{\ell^2_{eff}}{z^2}\left(d\tau^2+dz^2+dx_3^2+dr^2+r^2d\theta^2\right) \label{Poincare cylinder}
\end{equation}
We define the entanglement region at a cutoff  $z = \delta$ and at a fixed time $\tau=0$. We take $r \leq l$ and $x_3\in [0,H]$ to describe a cylinder of length $H$, radius $l$ and axis along coordinate $x_3$. One can parametrize  the bulk extremal surface as $r=f(z)$ and use the EOM of CCG to obtain a differential equation for it. We show that, near the boundary,  the ansatz given in  Ref.\cite{Anastasiou:2021swo} also solves perturbatively the differential equation coming from the CCG EOM and therefore  the embedding can be cast as:
  \begin{equation}
     \Sigma:\left[\tau=cte,  r=l\left(1-\frac{z^2}{4l^2}+O(z^4)\right)\right]. 
 \end{equation}
Using this embedding the induced metric is:
 \begin{equation}
    ds_{\Sigma}^2=\frac{\ell^2_{eff}}{z^2}\left(\left(1+\frac{z^2}{4l^2}+O(z^4)\right)dz^2+dx_3^2+\left(l-\frac{z^2}{4l}+O(z^4)\right)^2d\theta^2\right) ,\label{induced metric cylinder}
\end{equation}
 where the normal vectors to the hypersurface are given by:
 \begin{equation}
     n^{(1)}_{\mu}=\left(\frac{\ell_{eff}}{z},0,0,0,0\right)\quad;\quad n^{(2)}_{\mu}=\left(0,\frac{\ell_{eff}}{\sqrt{4l^2+z^2}},0,\frac{2\ell_{eff}l}{z\sqrt{4l^2+z^2}},0\right).\label{normal vectors cylinder}
 \end{equation}
 With  the induced metric and the normal coordinates we have all the ingredients to obtain all the terms appearing in the computation of the EE in CCG. We will present the results of the quantities involved (see Appendix \ref{Explicit calculation: EE for the cylinder} for detailed calculation) in the following:
 \begin{align}
     \frac{\mathcal{A}[\Sigma]}{4G} &=\frac{\pi H}{4l G}\ell^3_{eff}\left(\frac{l^2}{\delta^2}-\frac{1}{4}\ln\left(\frac{l}{\delta}\right)\right)\label{area cylinder}\\
      S_{R^2}=&-\frac{6}{\ell^4_{eff}}\left(3\mu_1+4\mu_2+8\mu_3+40\mu_4+16\mu_5+16\mu_6 +80\mu_7+400\mu_8\right) \\
      S_{K^2R}=& -\frac{z^2}{l^2\ell^4_{eff}}(40\mu_4+8\mu_3+12\mu_2-3\mu_1)+O(z^4)\\
      S_{K^4}^{min}=&S_{K^4}^{non-min}=O(z^4). \label{SK4 for the cylinder}
 \end{align}
 As you can see again, the term $S_{R^2}$ is a constant and can come out of the integration. Moreover, if we put it together with the contribution of the usual Einstein part we will obtain and overall factor $a_4$ multiplying the area. Note also that the terms $S_{\text{K}^4}$ which, in principle, should produce two different functionals, coincide in the normalizable order. If we replace Eqs.\eqref{area cylinder}-\eqref{SK4 for the cylinder} into the HEE functional for CCG we obtain:
\begin{equation}
  S_{CCG}=\frac{\pi H}{4l G}\ell^3_{eff}\left(\frac{a_4l^2}{\delta^2}-\frac{1}{4}b_4\ln\left(\frac{l}{\delta}\right)\right)+O(\delta),\label{HEE cylinder}\\
\end{equation}
where:
\begin{align}
 a_4&=1+\frac{3}{\ell^4_{eff}}\left(3\mu_1+4\mu_2+8\mu_3+40\mu_4+16\mu_5+16\mu_6 +80\mu_7+400\mu_8\right)\label{a4}\\ 
    b_4&=a_4-\frac{4}{\ell^2_{eff}}\left(40\mu_4+8\mu_3+12\mu_2-3\mu_1\right).\label{b4}
\end{align}
Note that in Eq.\eqref{HEE cylinder} we obtain the anomalous logarithmic term, typical of the even dimensional CFTs but in addition appears another divergent term proportional to $a_4$. To subtract this last term we have to add the corresponding counterterm $S_{KT}$ for this theory. From Eq.\eqref{HEE in CCG Kounterterm} we can see that:
 \begin{align}
  S_{KT}&=\frac{c_4}{4G}\bigg\lfloor\frac{4+1}{2}\bigg\rfloor\int_{\partial\Sigma}dx^2\sqrt{\tilde{\sigma}}B_2=-\frac{2c_4\pi H }{G}\frac{\ell_{eff}l}{\delta^{2}}+O(\delta).\nonumber
 \end{align}
 We can express $c_4=a_dc_4^{EH}$ and from Eq.\eqref{coupling Einstein Hilbert} we know tat $c_4^{EH}=\frac{\ell^2_{eff}}{8}$. Therefore:
 \begin{equation}
  S_{KT}=  -\frac{a_4\pi H }{4G}\frac{\ell^3_{eff}l}{\delta^{2}}+O(\delta).
 \end{equation}
 This term is exactly the same (with negative sign) as the first part of the right-hand side of Eq.\eqref{HEE cylinder}. Therefore, it does the job of recovering the universal part:
 \begin{equation}
     S_{CCG}^{Univ}=S_{CCG}+S_{KT}=-b_4\frac{\pi H\ell^3_{eff}}{16l G}\ln\left(\frac{l}{\delta}\right).
 \end{equation}
The type-B anomaly coefficient of a holographic CFT dual to CCG can be inferred from this result. In $d=4$ the universal term  of the EE is associated to c (the type-B anomaly) by the following equation \cite{Hung:2011xb,Bhattacharyya:2014yga}:
 \begin{equation}
      S_{CCG}^{Univ}=-\frac{cH}{2l}log\left(\frac{l}{\delta}\right).
 \end{equation}
Then, we can see in our result that the type-B anomaly, for this particular theory, is $c=b_4\frac{\pi\ell^3_{eff}}{8G}$. 
 

 

 
  \section{Deformed sphere I: $C_T$} \label{Deformed sphere I}


The coefficients of the contact-term expansion of stress-tensor correlators are crucial properties of CFTs. These coefficients are inherent to the definition of the theory and have diverse interpretations. For example, the coefficient $C_{\text{T}}$, which determines the unitarity of the theory, is associated with the two-point function and is subject to a positivity constraint \cite{Osborn:1993cr}. Moreover, the coefficients $C_{\text{T}}$, $t_\text{2}$, and $t_\text{4}$, which are related to the three-point function, control the energy flux that an observer located in a specific direction at null infinity receives \cite{Osborn:1993cr,Erdmenger:1996yc,Hofman:2008ar}. In this and the next section we will give a recipe to obtain the mentioned quantities from the universal part of the  EE. We will first review the method for the Einstein Hilbert case and then we will move on to the CCG case.


 
Lets start, by mentioning some results concerning the EE of a spherical entangling region. In Ref.\cite{Casini:2011kv} it was shown that by making a sequence of conformal transformations it was possible to relate the vacuum state of the original geometry to a thermal state placed on an $\mathbb{S}_d$ background\footnote{This sequence of transformations was first performed by Casini, Huerta and Myers, and is therefore called the CHM map.}. Using that map, the EE of that region was related to the free energy $F=-ln(Z[\mathbb{S}_d])$ of a CFT residing in $\mathbb{S}_{d}$. Then,  the AdS/CFT dictionary was used to translate this problem to the one of finding the horizon entropy of a certain topological black hole. 
 

On the other hand, in  Ref.\cite{Allais:2014ata}  it was conjectured that the sphere minimizes the universal contribution to the entanglement entropy, among all possible entanglement regions with sphere topology, for the ground state of a CFT. To notice this, the authors make an analysis  in $d=3$ dimensions of a deformed spherical entangling region with deformation parameter $\epsilon$ and then they generalized their results to any dimension. Furthermore, in Ref.\cite{Mezei:2014zla}, Mezei shows that the $C_T$ charge appears at order $O(\epsilon^2)$ in the expansion of the renormalized EE around the deformation parameter. Therefore, up to second order in $\epsilon$, the EE  reads as:



\begin{equation}
    S^{Univ}_{EH}(\mathbb{S}_{\epsilon})= S^{Univ,(0)}_{EH}(\mathbb{S})+\epsilon^2 S^{Univ,(2)}_{EH}(\mathbb{S})+O(\epsilon^4).
\end{equation}

where $S^{Univ,(0)}_{EH}(\mathbb{S})$  is the renormalized EE of the unperturbed sphere and the term $S^{Univ,(2)}_{EH}(\mathbb{S})\propto C_T$. The heuristic argument that can be used to understand this proportionality, is that the $S^{Univ,(2)}$ term can be though as obtained by making a variation of the free energy (considering the relationship with the EE given by the CHM map). This variation, as you can see, give us the two correlation function of the stress energy momentum tensor and therefore it should be proportional to $C_T$.

Now that we have made this remarks its time to obtain the $C_T$ for a $CFT_3$ dual to CCG, however it would be useful to review first the computation for the EH case. For that we will follow the Ref.\cite{Anastasiou:2020smm}.

The Poincaré-AdS4 spacetime can be expressed in polar coordinates as follows:

\begin{equation}
    ds^2=\frac{\ell^2}{z^2}\left(d\tau^2+dz^2+dr^2+r^2d\phi^2\right). \label{Poincare CT}
\end{equation}

We want to compute the ground state EE in the dual $CFT_3$ for a region that is a deformed sphere with a parameter deformation $\epsilon$.  If the corresponding theory dual to the CFT is Einstein-AdS gravity, we can obtain the entanglement entropy using the RT formula:
\begin{equation}
    S_{EH}=\frac{\mathcal{A}[\Sigma_{RT}]}{4G}.\label{RT formula}
\end{equation}

The results of Refs.\cite{Allais:2014ata,Anastasiou:2020smm} have shown that the embedding function for the RT surface is expressed as follows:
\begin{equation}
    \Sigma_{RT}: \:r=\sqrt{1-z^2}\left(1+\epsilon \sum_{l}\left(\frac{1-z}{1+z}\right)^{l/2}\frac{1+lz}{1-z^2}(a_l\cos{(l\phi)}+b_l\sin{(l\phi}))+O(\epsilon^2)\right).\label{embedding deformed sphere}
\end{equation}

We can use this embedding to obtain the induced metric and with it compute the minimal surface to obtain the EE. Following these steps and after renormalization, we obtain that the universal part of the EE is:
\begin{equation}
    S_{EH}^{Univ}=-\frac{\pi \ell^2}{2G}\left(1+\epsilon^2\sum_{l}\frac{l(l^2-1)}{4}(a_l^2+b_l^2)+O(\epsilon^4)\right).\label{EE deformed sphere for EH}
\end{equation}
The quadratic term in the perturbation of the HEE,  indicates the susceptibility of entanglement to changes in the shape of the entangling region. This term holds universal information, thanks to the coefficient $C_T$ that appears in the two-point correlation function of the stress tensor. Indeed, the subleading term of Eq.\eqref{EE deformed sphere for EH}, can equivalently be written as:
\begin{equation}
     S_{EH}^{Univ,(2)}=\frac{\pi^4 C_T}{24}\sum_{l}l(l^2-1)(a_l^2+b_l^2)\rightarrow  C_T=\frac{3\ell^2}{\pi^3G} .\label{SEE eps}
\end{equation}
Now we want to compute the EE for a CFT with a general Cubic Curvature Gravity as a dual theory. We are going to fix the  cod-2 brane at the position of the RT surface. This is analog as assuming the Einstein-Hilbert EOMs for determining the adapted coordinates for the foliation with respect to the cod-2 brane. Then, we introduce the corresponding embedding into the correct entropy functional for the theory. At least, for small values in the CCG theory couplings, we know that this prescription should work at the perturbative level.

Using the embedding given by Eq.\eqref{embedding deformed sphere}, we can compute the relevant terms appearing in  the HEE in CCG functional, these are:
\begin{align}
    S_{R^2}&=-\frac{6 \left(2 \mu_\mathrm{1}+4 \mu_\mathrm{2}+6 \mu_\mathrm{3}+24 \mu_\mathrm{4}+9 \mu_\mathrm{5}+9 \mu_\mathrm{6}+36 \mu_\mathrm{7}+144 \mu_\mathrm{8}\right)}{\ell_{eff}^{4}}\label{SR2 deformed CT}\\
    S_{K^2R}&= \frac{12z^4 (\mu_1 -4 \mu_2 -2 \mu_3 -8 \mu_4)}{\ell_{eff}^4(1-z^2)^2}\sum_{l}\left(\frac{1-z}{z +1}\right)^{l}(a_l^{2}+b_l^{2}) l^{2} (l^2 -1)^{2} \epsilon^{2}\\
    &+\mathrm{O}\! \left(\epsilon^{3}\right)\nonumber\\
    S_{K^4}^{min}&=S_{K^4}^{non-min}=\mathrm{O}\! \left(\epsilon^{4}\right).\label{SK4 deformed CT}
\end{align}
We can see from here that at least at order $\epsilon^2$ the two splittings coincide, that is $S_{CCG}^{min}=S_{CCG}^{non-min}=S_{CCG}$.  By replacing in the functional Eq.\eqref{HEE in CCG minimal} or Eq.\eqref{HEE in CCG nonminimal} and after integration (See Appendix \ref{Explicit calculation: EE for the Deformed sphere}) we obtain, up to order $\epsilon^2$ :
\begin{align}
   S_{CCG}=&-\frac{a_3\pi \ell_{eff}^2}{2G}\left(1-\frac{1}{\delta}\right)-\frac{\pi \ell_{eff}^2}{8G}\sum_l\bigg[(a_l^2+b_l^2)l(l^2-1)\bigg(1\label{HEE in CCG deformed sphere for CT}\\&+\frac{3}{\ell_{eff}^4}(4\mu_1-4\mu_2+2\mu_3+8\mu_4
   +9\mu_5+9\mu_6+36\mu_7+144\mu_8)\bigg)\nonumber\\
   &-\frac{a_3}{\delta}(a^2+b^2)l^2\bigg]\epsilon^2\nonumber+O(\epsilon^3).\nonumber 
\end{align}
The Kounterterm is constructed by setting  $z=\delta$ and then use it to compute the chern-form in one dimension $B_1=-\frac{2}{\ell_{eff}}+O(\epsilon^3,\delta)$. Then, we replace it in the formula for the Kounterterm \eqref{HEE in CCG Kounterterm} (with the identification $c_4=\frac{a_3\ell_{eff}^2}{4}$):
 \begin{equation}
  S_{KT}=  -\frac{\pi \ell_{eff}^2 a_3}{2G}\frac{1}{\delta}\left(1+\sum_l\frac{l^2}{4}(a_l^2+b_l^2)\epsilon^2\right)+O(\epsilon^3,\delta) . 
 \end{equation}

 The previous expression, with an overall minus sign, is equal to the divergent part of Eq.\eqref{HEE in CCG deformed sphere for CT}. Thus, when we sum $S_{CCG}$ and $S_{KT}$, the divergence cancels out successfully, and we recover the universal part:
 \begin{align}
     S_{CCG}^{Univ}&=-\frac{\pi \ell_{eff}^2}{2G}\bigg[a_3+\sum_l\frac{1}{4}(a_l^2+b_l^2)l^2(l^2-1)\bigg(1+\frac{3}{\ell_{eff}^4}(4\mu_1-4\mu_2+2\mu_3\nonumber\\&+8\mu_4+9\mu_5+9\mu_6+36\mu_7+144\mu_8)\bigg)\epsilon^2\bigg] +O(\epsilon^4).\label{EE in CCG universal for deformed sphere}
 \end{align}
 By examining the $\epsilon^2$ term of the expression given above, we can determine the $C_T^{CCG}$ for the CCG theory. We assume that the polynomial in $l$ is the same for EH and CCG, as it should be independent of the specific theory and only depend on the geometry of the chosen entanglement region. We compare the order $\epsilon^2$ term of Eq.\eqref{EE in CCG universal for deformed sphere} with Eq.\eqref{SEE eps} and observe that: 
\begin{equation}
     S_{CCG}^{Univ,(2)}=\frac{\pi^4 C_T^{CCG}}{24}\sum_{l}l(l^2-1)(a_l^2+b_l^2), \label{univ deformed sphere CT}
\end{equation}
where:
\begin{equation}
    C_T^{CCG}=\bigg(1+\frac{3}{\ell_{eff}^4}(4\mu_1-4\mu_2+2\mu_3+8\mu_4+9\mu_5+9\mu_6+36\mu_7+144\mu_8)\bigg)C_T.\label{Ct CCG}
\end{equation}
Therefore, we find that the $C_T^{CCG}$ is equivalent to the $C_T$ of the Einstein-AdS, but multiplied by a coefficient that depends on the couplings of the CCG theory. The Eq.\eqref{Ct CCG} is consistent with what was found in Ref.\cite{Bueno:2020uxs}, albeit using a different computational method.

\section{Deformed sphere II: $t_4$} \label{Deformed sphere II}

From the previous section, we can observe that Eq.\eqref{univ deformed sphere CT} does not have contributions from the splitting-dependent terms. Nevertheless, as per Eq.\eqref{SK4 deformed CT}, these terms should appear at the next subleading order, i.e. in $S_{\text{CCG}}^{Univ,\text{(4)}}$. To calculate this, we need to consider higher orders in the $\epsilon$ expansion of the embedding function in Eq.\eqref{embedding deformed sphere}. Using the heuristic interpretation described at the beginning of the previous section, we can relate the coefficients of this higher powers in the $\epsilon$ expansion with higher point functions of the stress tensor and therefore use it to compute the coefficients ($C_T,t_2, t_4$, etc) that determine them.
 
Again we use the embedding obtained for the case of Einstein gravity. For computational purposes, we will suppress the solution of type $sin(l\phi)$ in Eq.\eqref{embedding deformed sphere}. Then, up to second order in the expansion around the deformation parameter, we obtain:
 \begin{align}
    \Sigma_{RT}: \:\:r&=\sqrt{1-z^2}\bigg[1+\epsilon \sum_{l}\left(\frac{1-z}{1+z}\right)^{l/2}\left(\frac{1+lz}{1-z^2}\right)\cos{(l\phi)}\nonumber\\
    &+\epsilon^2\sum_{l}\left(\frac{1-z}{1+z}\right)^{l}\left(\frac{1}{4\left(1-z^{2}\right)^{2}}\right)\bigg(\left(1+2lz+\left(3l^{2}-2\right)z^{2}+2l\left(l^{2}-1\right)z^{3}\right)&\nonumber\\
    &+\left(2l\left(l^{2}-4\right)z^{3}+\left(3l^{2}-5\right)z^{2}+8lz+4\right)\cos\left(2l\phi\right)\bigg)+O(\epsilon^4)\bigg].\label{embedding deformed sphere eps2}
\end{align}
As far as we are aware, the explicit expression for the $R_{22}$ term in Ref.\cite{Allais:2014ata}, which we are providing here, constitutes a new and original result. To ensure consistency, we will now compute the universal HEE for the case of Einstein gravity. We will use the embedding \eqref{embedding deformed sphere eps2} in the RT formula \eqref{RT formula}, resulting in:
 \begin{equation}
    S_{EH}^{Univ}=-\frac{\pi \ell^{2}}{2G}\left(1+\sum_{l}\frac{1}{4} l\left(l^{2}-1\right) \epsilon^{2}-\sum_{l}\frac{ l\left(23 l^{6}-246l^{4}+63 l^{2}-2\right)}{4\left(64l^{2}-16\right)} \epsilon^{4}+O(\epsilon^6)\right).\label{EE deformed sphere for EH epsilon4}
\end{equation}
 We know that for CFTs in d=3 dual to  Einstein Gravity, $t_4=0$ and $t_2=0$ \cite{Mezei:2014zla} so the term appearing in $\epsilon^4$ should be proportional just to  $C_T^{EH}$. We can expressed Eq.\eqref{EE deformed sphere for EH epsilon4} as:
 \begin{equation}
    S_{EH}^{Univ}=-\frac{\pi \ell^{2}}{2G}-\frac{\pi^{4}C_T^{EH}}{24}\sum_{l}l\left(l^{2}-1\right) \epsilon^{2}+\frac{\pi^{4}C_T^{EH}}{24}\sum_{l}\left(\frac{ l\left(23 l^{6}-246l^{4}+63 l^{2}-2\right)}{16\left(4l^{2}-1\right)}\right) \epsilon^{4} .\label{EE deformed sphere for EH epsilon4 in terms of CT}
\end{equation}
For reasons that will be described below, let's assign a name to the polynomial on $l$ that multiplies $C_T$ at the order $\epsilon^4$:
\begin{equation}
    P_1(l)=\sum_{l}\left(\frac{\pi^4 l\left(23 l^{6}-246l^{4}+63 l^{2}-2\right)}{384\left(4l^{2}-1\right)}\right)\label{P1}
\end{equation}
Now, to compute the EE in CCG, we employ the same RT surface embedding. As stated in Ref.\cite{Bueno:2020uxs}, the use of this surface for evaluating the HEE functional for the deformed sphere is valid up to the leading order in the higher curvature couplings.  We proceed to find all the quantities involved up to order $\epsilon^4$:
 \begin{align}
     S_{R^2}&=-\frac{6 \left(2 \mu_\mathrm{1}+4 \mu_\mathrm{2}+6 \mu_\mathrm{3}+24 \mu_\mathrm{4}+9 \mu_\mathrm{5}+9 \mu_\mathrm{6}+36 \mu_\mathrm{7}+144 \mu_\mathrm{8}\right)}{\ell_{eff}^{4}}\\
     S_{K^2R}&=\frac{12z^{4}\left(\mu_\mathrm{1}-4 \mu_\mathrm{2}-2 \mu_\mathrm{3}-8 \mu_\mathrm{4}\right)}{\ell_{eff}^4\left(1-z^{2}\right)}\sum_{l}\left(\frac{1-z}{1+z}\right)^{l} l^{2} \left(l^{2}-1\right)^{2} \epsilon^{2}\nonumber\\
     &+\frac{12z^{4}\left(\mu_\mathrm{1}-4 \mu_\mathrm{2}-2 \mu_\mathrm{3}-8 \mu_\mathrm{4}\right)}{\ell_{eff}^{4}\left(1-z^{2}\right)^{3}}\sum_{l}\left(\frac{1-z}{1+z}\right)^{\frac{3 l}{2}}l^{2}\left(l^{2}-1\right) \left(2 l^{4} z^{2}-2 l^{3} z-14 l^{2} z^{2}\right.\nonumber\\
     &\left.+8 l^{2}+2 l z+3 z^{2}+1\right) \cos\left(l \phi\right)\epsilon^3\nonumber\\
     &-\frac{12\mathit{lz}^{3}\left(\mu_\mathrm{1}-4 \mu_\mathrm{2}-2 \mu_\mathrm{3}-8 \mu_\mathrm{4}\right)}{\ell_{eff}^{4}\left(1-z^{2}\right)^{4} }\sum_{l}\left(\frac{1-z}{1+z}\right)^{2 l}\bigg(\left(l^{2}-1\right)\big(l \left(6 l^{4}-23 l^{2}-1\right) z^{5}\nonumber\\
     &+3 \left(11 l^{4}-34 l^{2}+5\right) z^{4}+2 l \left(15 l^{4}-29 l^{2}-4\right) z^{3}+2l^{2}\left(14 l^{2}+4\right) z^{2}\\
     &+l\left(53l^{2}-3\right) z-l^{4}+22 l^{2}-3\big) \cos\left(l \phi\right)^{4}+\frac{1}{2}\left(l^{2}-1\right)\big(2 l \left(14 l^{6}-88 l^{4}+111 l^{2}-16\right) z^{5}\nonumber\\
     &-\left(8 l^{6}+56 l^{4}-139 l^{2}+9\right) z^{4}+2 l \left(19 l^{4}-34 l^{2}+18\right) z^{3}-2 \left(5 l^{6}-5 l^{4}+24 l^{2}-3\right) z^{2}\nonumber\\
     &-2 l \left(2 l^{4}+17 l^{2}+5\right) z+2 l^{4}-29 l^{2}+3\big)\cos\! \left(l \phi\right)^{2}+\frac{l\left(1-z^{2}\right)}{4}\big(\big(8 l^{8}-112 l^{6}\nonumber\\
     &+332 l^{4}-168 l^{2}+21\big) z^{2}+4 l \left(4 l^{6}-17 l^{4}+22 l^{2}-9\right) z\nonumber\\
     &+8 l^{6}-142 l^{4}+68 l^{2}-15\big)\bigg)+O(\epsilon^5)\nonumber\\
     S_{K^4}^{min}&=\frac{3 z^{8}\left(-4 \mu_\mathrm{2}+\mu_\mathrm{1}-2 \mu_\mathrm{3}-8 \mu_\mathrm{4}\right)}{\ell_{eff}^{4}\left(1-z^{2}\right)^{4}} \sum_{l}\left(\frac{1-z}{1+z}\right)^{2 l} l^{4} \left(l^2-1\right)^{4} \epsilon^{4}+O(\epsilon^5)\label{SK4min t4}\\
      S_{K^4}^{non-min}&=\frac{6 z^{8}\left(\mu_\mathrm{1}+2 \mu_\mathrm{2}\right)}{\ell_{eff}^{4}\left(1-z^{2}\right)^{4}} \sum_{l}\left(\frac{1-z}{1+z}\right)^{2 l} l^{4} \left(l^2-1\right)^{4} \epsilon^{4}+O(\epsilon^5).\label{SK4nonmin t4}
 \end{align}
 We can see that the last two equation for $S_{K^4}^{min}$ and the $S_{K^4}^{non-min}$  are not the same at order $\epsilon^4$ and therefore we will have a difference between the splittings. Notice that this is the first scenario where we can find a difference between the minimal and non-minimal prescription. Now, we replace this terms in the functional of the EE. We already know that the leading and subleading order will be the same of Eq.\eqref{EE in CCG universal for deformed sphere}. On the other hand, by seeking the polynomial $P_1(l)$ given by Eq.\eqref{P1}, we can notice that there is a term proportional to $C_T^{CCG}$ at order  $\epsilon^4$ as well as other term that is supposed to be proportional to $t_4$ (the $t_2$ always vanishes in this dimension): 
\begin{align}
   S_{\text{CCG}}^{\text{Univ,min}}&=-\frac{\pi \ell_{\text{eff}}^{2}}{2G}a_3-\epsilon^2C_{\text{T}}^{\text{CCG}}\sum_l\frac{\pi^{4}}{24}l\left(l^{2}-1\right) \nonumber\\
   &+\epsilon^4\bigg(C_{\text{T}}^{\text{CCG}}  P_{1}\left(l\right)-\sum_l\frac{135 \pi l^{3} \left(l^{2}-1\right)^{3} \left(\mu_\mathrm{1}-4 \mu_\mathrm{2}-2 \mu_\mathrm{3}-8 \mu_\mathrm{4}\right)}{256 G \ell_{\text{eff}}^{2} \left(4 l^{2}-1\right) \left(4 l^{2}-9\right)}\bigg) +O(\epsilon^6) \,, \label{SCCGmin t4}\\
    S_{\text{CCG}}^{\text{Univ,non-min}}&=-\frac{\pi \ell_{\text{eff}}^{2}}{2G}a_3-\epsilon^2C_{\text{T}}^{\text{CCG}}\sum_l\frac{\pi^{4}}{24}l\left(l^{2}-1\right)  \nonumber\\
   &+\epsilon^4\bigg(C_{\text{T}}^{\text{CCG}} P_{1}\left(l\right)-\sum_l\frac{135 \pi l^{3} \left(l^{2}-1\right)^{3} \left(\mu_\mathrm{1}+2 \mu_\mathrm{2}\right)}{128 G \ell_{\text{eff}}^{2} \left(4 l^{2}-1\right) \left(4 l^{2}-9\right)}\bigg) +O(\epsilon^6) \,.
   \label{SCCGnonmin t4}
\end{align}

It is not surprising to find that the polynomial $P_1(l)$ appears in the last two equations, as it relies solely on geometric factors and is independent of the specific theory at hand. However, it is important to note that this only holds true for CFTs duals to EH gravity with perturbative corrections, as we are employing the RT surface for the computation of the entanglement entropy. For more generic CFTs, the aforementioned assertion may not hold and $P_1(l)$ could no longer be theory independent.



The discrepancy between Eq.\eqref{SCCGmin t4} and Eq.\eqref{SCCGnonmin t4}, at order $\epsilon^4$, can be traced back to the coupling dependent parts of Eq.\eqref{SK4min t4} and Eq.\eqref{SK4nonmin t4}, respectively. Surprisingly, the coupling combination $(\mu_\mathrm{1}+2 \mu_\mathrm{2})$ that appears in the non-minimal prescription agrees with the coupling dependent part of the $t_4$ charge obtained for a CFT that is dual to a CCG theory (see Ref.\cite{Sen:2014nfa,Chu:2016tps}). Thus, the order $O(\epsilon^4)$ in Eq.\eqref{SCCGnonmin t4} can be expressed as a linear combination of the $C_{T}$ and $t_{4}$ charges of CCG \footnote{In Ref.\cite{Bueno:2015ofa}, it was shown that in generic CFTs, the term $O(\epsilon^4)$ of the HEE cannot be completely determined by $C_{T}$ and $t_{4}$ when corner contributions are taken into account.}.



To establish the normalization and obtain the $t_4$, we consider the massless limit of CCG, which was analyzed in Ref.\cite{Li:2019auk}. When demanding that the massive modes are decoupled from the particle spectrum of the theory, we arrive at the equations (in $d=3$):
\begin{align}
    &12\mu_{7}+9\mu_{6}+5\mu_{5}+48\mu_{4}+16\mu_{3}+24\mu_{2}-3\mu_{1}=0 \,,\label{1st constranint}\\
    &432\mu_{8}+120\mu_{7}+36\mu_6+32\mu_{5}+16\mu_{4}+28\mu_{3}+6\mu_{1}+24\mu_{2}=0 \,.\label{second constraint}
\end{align}
By solving this set of equations, we can represent the coupling constants $\mu_1$ and $\mu_2$ in relation to the remaining couplings:
\begin{align}
    \mu_1&=-48\mu_8-12\mu_7-3\mu_6-3\mu_5-\frac{16}{3}\mu_{4}-\frac{4}{3}\mu_3 \,,\label{mu1}\\
    \mu_2&=-6\mu_8-2\mu_7-\frac{3}{4}\mu_6-\frac{7}{12}\mu_5-\frac{8}{3}\mu_4-\frac{5}{6}\mu_3 \,.\label{mu2}
\end{align}
Replacing these relations in $(\mu_\mathrm{1}+2 \mu_\mathrm{2})$  we  recover the coupling dependence of the $t_4$  which was previously determined in Ref.\cite{Li:2019auk}. This is\footnote{This result was obtained without the need to deal with two different splittings.}: 
\begin{equation}
    t_4^{\text{CCG}-massless}=
   - \frac{120(360\mu_8+96\mu_7+27\mu_6+25\mu_5+64\mu_4+18\mu_3)}{\ell_{\text{eff}}^4}+\mathcal{O}(\mu^2) \,.
    \label{masslesst4}
\end{equation}
In contrast to the non-minimal counterpart, substituting Eq. \eqref{mu1} and Eq. \eqref{mu2} into the expression $\left(\mu_\mathrm{1}-4 \mu_\mathrm{2}-2 \mu_\mathrm{3}-8 \mu_\mathrm{4}\right)$, that appears in the minimal entropy functional given in Eq. \eqref{SCCGmin t4}, will not yield Eq. \eqref{masslesst4} . This is further evidence of its inconsistency.

For the massless case, it is easy to see that the polynomial of the multipole momenta accompanying the $t_4$ is:
\begin{equation}
     P_{2}(l)=\sum_l \frac{\pi^4}{2048}\frac{l^{3} \left(l^{2}-1\right)^{3}}{ \left(4 l^{2}-1\right) \left(4 l^{2}-9\right)} \,.\label{P2}
\end{equation}
In the generic CCG theory, $P_{2}(l)$ should also appear multiplying the $t_4$. Therefore, by keeping track of it in the Eqs.\eqref{SCCGmin t4} and \eqref{SCCGnonmin t4}, we obtain:
\begin{align}
   S_{\text{CCG}}^{\text{Univ, min}}=&-\frac{\pi \ell_{\text{eff}}^{2}}{2G}a_3-\epsilon^2C_{\text{T}}^{\text{CCG}}\sum_l\frac{\pi^{4}}{24}l\left(l^{2}-1\right)\nonumber \\
   &+\epsilon^{4}C_{\text{T}}^{\text{CCG}}\left[P_{1}\left(l\right)-t_4^{CCG-min}P_{2}\left(l\right)\right] +\mathcal{O} \left(\epsilon^6 \right)\,,\\
    S_{\text{CCG}}^{ \text{Univ, non-min}}=&-\frac{\pi \ell_{\text{eff}}^{2}}{2G}a_3-\epsilon^2C_T^{CCG}\sum_l\frac{\pi^{4}}{24}l\left(l^{2}-1\right) \nonumber
    \\&+\epsilon^{4}C_{\text{T}}^{\text{CCG}}\left[P_{1}\left(l\right)-t_4^{CCG-non-min}P_{2}\left(l\right)\right] +\mathcal{O}\left(\epsilon^6 \right)\,,
    \label{SCCGfinal}
\end{align}
\normalsize
where we can identify the $t_4$ charges for each of the splittings. These are:
\begin{align}
    t_4^{\text{CCG-min}}&=\frac{360\left(\mu_\mathrm{1}-4 \mu_\mathrm{2}-2 \mu_\mathrm{3}-8 \mu_\mathrm{4}\right)}{\ell_{\text{eff}}^{4}\left(1+\frac{3}{\ell_{\text{eff}}^4}(4\mu_1-4\mu_2+2\mu_3+8\mu_4+9\mu_5+9\mu_6+36\mu_7+144\mu_8)\right)}\,,\label{t4 min}\\
   t_4^{\text{CCG-non-min}}&=
    \frac{720(\mu_1+2\mu_2)}{\ell_{\text{eff}}^4\left(1+\frac{3}{\ell_{\text{eff}}^4}(4\mu_1-4\mu_2+2\mu_3+8\mu_4+9\mu_5+9\mu_6+36\mu_7+144\mu_8)\right)}\,.\label{t4 nonmin}
\end{align}
Therefore, the non-minimal prescription is the one that matches the $t_4$ value derived by Sen and Sinha in Ref.\cite{Sen:2014nfa} for graviton perturbations on a shockwave background. Moreover, we can contrast the non-minimal splitting of $t_4$ in Eq.\eqref{t4 nonmin} with its equivalent expression for Einsteinian Cubic Gravity (ECG), as presented in Ref.\cite{Bueno:2018xqc}. To define this theory the CCG couplings should be taken as:
\begin{equation}
  \mu_1=-\frac{3}{2}\mu  \:\: ; \:\: \mu_2=-\frac{\mu}{8} \:\:  ;\:\:\mu_3= 0 \:\: ; \:\: \mu_4= 0 \:\: ; \:\:\mu_5= \frac{3}{2}\mu  \:\: ;\:\:  \mu_6= -\mu \:\: ;\:\:\mu_7= 0  \:\:;\:\:\mu_8=0\,.
\label{ECG_couplings}
\end{equation}
Thus, we obtain that the value of $t_4$ for this theory,
\begin{equation}
     t_4^{\text{ECG-non min}}=-\frac{1260\mu}{\ell_{\text{eff}}^{4}\bigg(1-\frac{3}{\ell_{\text{eff}}^4}\mu\bigg)}\,,
\end{equation}
is in full agreement with the given reference.



