\thispagestyle{plain}
\begin{center}
    \Large
    \textbf{Aspects of Holographic
Entanglement Entropy in Cubic
Curvature Gravity}
        
    \vspace{0.4cm}
    \large
    by Andrés Argandoña
        
    \vspace{0.9cm}
    \textbf{Abstract}
\end{center}


In this thesis we explore general aspects of the entanglement entropy (EE) for Conformal Field Theories (CFTs) dual to Cubic Curvature Gravity. We derived a covariant expression for the EE by using a scheme inherited from the bulk renormalization method through extrinsic counterterms. We evaluate this functional in  different entangling regions to calculate CFT data. In particular, we compute the $t_4$ coefficient of the 3-point function of the stress-tensor correlator by considering a deformed entangling region. We observe that there is a discrepancy between the outcomes attained through the employment of the EE functional for minimal and non-minimal splittings. We find  that only the $t_4$ obtained from the non-minimal functional agrees with previous results in the literature that were computed by splitting-independent CFT methods for specific theories such as the massless graviton case.



