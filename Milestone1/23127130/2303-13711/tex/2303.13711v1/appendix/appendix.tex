\appendix
\chapter{Explicit calculation: EE for the cylinder}\label{Explicit calculation: EE for the cylinder}

In this Appendix we will approach the calculation of the Entanglement Entropy for the vacuum state of a cylindrical region of a CFT which, on the gravitational side, is characterized by a Cubic Curvature theory. We start by considering the metric of Eq.\eqref{Poincare cylinder}:
\begin{equation}
    ds^2=\frac{\ell^2_{eff}}{z^2}\left(d\tau^2+dz^2+dx_3^2+dr^2+r^2d\theta^2\right) \label{Poincare for cylinder appendix}
\end{equation}
A cylindrical region in the CFT in a Cauchy slice is defined in the following way:
 \begin{equation}
     A:\left[\tau=cte, z=\delta \:\:and\:\: r\leq l\right] 
 \end{equation}
Therefore, the entangling surface is obtain at constant radius $l$:
  \begin{equation}
     \partial A:\left[\tau=cte, z=\delta \:\:and\:\: r=l\right] 
 \end{equation}
In the bulk, the extremal surface that is anchored to $\partial A$, in the near-boundary region, is described by the embedding:
 \begin{equation}
     \Sigma:\left[\tau=cte,  r=l\left(1-\frac{z^2}{4l^2}+O(z^4)\right)\right]
 \end{equation}
 With this we compute the induced metric $\sigma$ for the codimension-two hypersurface:
 \begin{equation}
    ds_{\Sigma}^2=\frac{\ell^2_{eff}}{z^2}\left(\left(1+\frac{z^2}{4l^2}+O(z^4)\right)dz^2+dx_3^2+\left(l-\frac{z^2}{4l}+O(z^4)\right)^2d\theta^2\right) 
\end{equation}
 therefore the square root of the determinant of the induced metric is :
 \begin{equation}
     \sqrt{\sigma}=\frac{\ell^3_{eff}l}{z^3}-\frac{1}{8}\frac{\ell^3_{eff}}{lz}+O(z) \label{detsigma}
 \end{equation}
and using the Eq.\eqref{normal vectors cylinder} of the normal vectors  we can compute the extrinsic curvatures as:
 \begin{align}
     K_{\:\:\:\bar{\mu}}^{(1)\:\bar{\nu}}&=0\label{intrinsic curvature 1}\\
      K_{\:\:\:\bar{z}}^{(2)\:\bar{z}}&=-\frac{z^3}{8\ell^3_{eff}l}+O(z^5)\\
       K_{\:\:\:\bar{x_3}}^{(2)\:\bar{x_3}}&=-\frac{z}{2\ell_{eff}l}+\frac{z^3}{16\ell^3_{eff}l}+O(z^5)\\
        K_{\:\:\:\bar{\theta}}^{(2)\:\bar{\theta}}&=\frac{z}{2\ell_{eff}l}+\frac{3z^3}{16\ell^3_{eff}l}+O(z^5)
      \label{intrinsic curvature 2}
 \end{align}
Now using \eqref{detsigma} we can obtain the Area of $\Sigma$:
\begin{align}
   \frac{\mathcal{A}[\Sigma]}{4G}&=\frac{1}{4G}\int_{\Sigma}d^3x\left(\frac{\ell^3_{eff}l}{z^3}-\frac{1}{8}\frac{\ell^3_{eff}}{lz}+O(z)\right)\nonumber\\
   &=\frac{2\pi H}{4G}\int_{\delta}^{z_{max}}dz\left(\frac{\ell^3_{eff}l}{z^3}-\frac{1}{8}\frac{\ell^3_{eff}}{lz}+O(z)\right)\nonumber\\
   &=\frac{\pi Hl}{2G}\int_{\delta/l}^{z_{max}/l}d\rho\left(\frac{\ell^3_{eff}}{l^2\rho^3}-\frac{1}{8}\frac{\ell^3_{eff}}{l^2\rho}+O(z)\right)\nonumber \\
   &=\frac{\pi H}{2G}\frac{\ell^3_{eff}}{l}\left(-\frac{1}{2\rho^2}-\frac{1}{8}\ln(\rho)\right)_{\delta/l}^{z_{max}/l} \nonumber\\
   &=\frac{\pi H}{4l G}\ell^3_{eff}\left(\frac{l^2}{\delta^2}-\frac{1}{4}\ln\left(\frac{l}{\delta}\right)\right)
\label{Area cylinder}
\end{align}
where the upper limit has been neglected because it only contributes to the finite part, which in $d=4$ is not universal. Now, for the cubic part of Eq.\eqref{minimal Spliting} and Eq.\eqref{non-minimal Spliting} and using the results for the extrinsic curvature we obtain:
 \begin{align}
      S_{R^2}=&-\frac{6}{\ell^4_{eff}}\left(3\mu_1+4\mu_2+8\mu_3+40\mu_4+16\mu_5+16\mu_6 +80\mu_7+400\mu_8\right) \\
      S_{K^2R}=& -\frac{z^2}{l^2\ell^4_{eff}}(40\mu_4+8\mu_3+12\mu_2-3\mu_1)+O(z^4)\\
      S_{K^4}^{min}=&S_{K^4}^{non-min}=O(z^4)
 \end{align}

 Replacing all this results in the entropy for CCG \eqref{minimal Spliting} or \eqref{non-minimal Spliting} we get:
 \begin{align}
    S_{CCG} =&\frac{\mathcal{A}[\Sigma]}{4 G}-\frac{1}{8 G}\int_{\Sigma}d{3}x\sqrt{\sigma}\bigg[\frac{6}{\ell^4_{eff}}\left(3\mu_1+4\mu_2+8\mu_3+40\mu_4+16\mu_5\right.\nonumber\\
    &\left.+16\mu_6 +80\mu_7+400\mu_8\right)-\frac{z^2}{l^2 \ell^4_{eff}}(40\mu_4+8\mu_3+12\mu_2-3\mu_1)+O(z^4)\bigg]\nonumber \\
    &=\frac{a_4\mathcal{A}[\Sigma]}{4 G}+\frac{1}{8G}\int_{\Sigma}dx^{3}\left(\frac{\ell^3_{eff}l}{z^3}+\frac{1}{8}\frac{\ell^3_{eff}}{lz}\right)\nonumber\\
    &\times\left(\frac{z^2}{l^2 \ell^4_{eff}}(40\mu_4+8\mu_3+12\mu_2-3\mu_1)\right)+O(z)\nonumber\\
    &=\frac{a_4\mathcal{A}[\Sigma]}{4 G}+\frac{1}{8G}\int_{\Sigma}dx^{3}\left(\frac{\ell^3_{eff}l}{z^3}+\frac{1}{8}\frac{\ell^3_{eff}}{lz}+O(z)\right)\nonumber\\
    &\times\left(\frac{z^2}{l^2 \ell^4_{eff}}(40\mu_4+8\mu_3+12\mu_2-3\mu_1)+O(z^4)\right)\nonumber\\
    &=\frac{a_4\mathcal{A}[\Sigma]}{4 G}+\frac{\pi H\left(40\mu_4+8\mu_3+12\mu_2-3\mu_1\right)}{4Gl\: \ell_{eff}}\int_{\delta}^{z_{max}}dz \left(\frac{1}{z}+O(z)\right)\nonumber\\
    &=\frac{a_4\mathcal{A}[\Sigma]}{4G}+\frac{\pi H(40\mu_4+8\mu_3+12\mu_2-3\mu_1)}{4Gl\: \ell_{eff}}\ln{\left(\frac{l}{\delta}\right)}+O(\delta)
 \end{align}
Now we replace the area formula  \eqref{Area cylinder} in this last equation:
\begin{align}
  S_{CCG}&=\frac{a_4\pi H}{4l G}\ell^3_{eff}\left(\frac{l^2}{\delta^2}-\frac{1}{4}\ln\left(\frac{l}{\delta}\right)\right)\nonumber\\
  &+\frac{\pi H\left(40\mu_4+8\mu_3+12\mu_2-3\mu_1\right)}{4Gl\: \ell_{eff}}\ln{\left(\frac{l}{\delta}\right)}+O(\delta)\nonumber\\
  &=\frac{\pi H}{4l G}\ell^3_{eff}\bigg(\frac{a_4l^2}{\delta^2}-\frac{1}{4}\bigg(a_4-\frac{4}{\ell^2_{eff}}\left(40\mu_4+8\mu_3+12\mu_2-3\mu_1\bigg)\bigg)\ln\left(\frac{l}{\delta}\right)\right)\nonumber\\
  &=\frac{\pi H}{4l G}\ell^3_{eff}\left(\frac{a_4l^2}{\delta^2}-\frac{1}{4}b_4\ln\left(\frac{l}{\delta}\right)\right)
\end{align}
where $b_4$ and $a_4$ is defined in Eq.\eqref{a4} and Eq.\eqref{b4}. As we have mention before, this last expression has a divergent term that is not part of the logarithmic divergences and therefore we need to use the Kounterterms to get rid of it. For that we start with  \eqref{Poincare for cylinder appendix} and take  $z=\delta$ because the Kounterterms are  defined at the boundary:
  \begin{equation}
    ds_{\partial\Sigma}^2=\tilde{\sigma}_{\bar{a}\bar{b}}dY^{\bar{a}}dY^{\bar{b}}=\frac{\ell^2_{eff}}{\delta^2}\left(dx_3^2+\left(l-\frac{\delta^2}{4l}\right)^2d\theta^2\right) \label{induced metric Kounterterm}
\end{equation}
 with normal vector:
 \begin{equation}
     n^{(3)}_z=-\frac{\ell_{eff}}{z}\sqrt{1+\frac{z^2}{4l^2}}
 \end{equation}
Now we compute the determinant of the metric of cod-3 and the extrinsic curvature needed to obtain the Kounterterm $B_2=tr(\kappa)$:
\begin{equation}
    \sqrt{\tilde{\sigma}}=\frac{\ell^2_{eff} l}{\delta^2}-\frac{\ell^2_{eff}}{4l}+O(\delta^2)  \:\:\:\:;\:\:\:\:Tr(\kappa)=\frac{2}{\ell_{eff}}+\frac{1}{4}\frac{\delta^2}{\ell_{eff}l^2}+O(\delta^4)\label{sqrt root and trace cylinder}\end{equation}
therefore:
\begin{equation}
 B_2=-2\int_{0}^{1}ds\int_{0}^{s}dtTr(\kappa)=-2\left(\frac{2}{\ell_{eff}}+\frac{1}{4}\frac{\delta^2}{\ell_{eff}l^2}+O(\delta^4)\right)
\end{equation}

Now we replace \eqref{sqrt root and trace cylinder} into the definition of what we obtained in Eq.\eqref{HEE in CCG Kounterterm}:
 \begin{align}
  S_{KT}&=\frac{c_4}{4G}\bigg\lfloor\frac{4+1}{2}\bigg\rfloor\int_{\partial\Sigma}d^2x\sqrt{\tilde{\sigma}}B_2\nonumber\\
  &=\frac{c_4}{2G}\int_0^Hdx_3\int_{0}^{2\pi}d\theta\left(\frac{\ell^2_{eff} l}{\delta^2}-\frac{\ell^2_{eff}}{4l}+O(\delta^2) \right)\nonumber\\&\times\left[-2\left(\frac{2}{\ell_{eff}}+\frac{1}{4}\frac{\delta^2}{\ell_{eff}l^2}+O(\delta^4)\right)\right]   \nonumber  \\
  &=-\frac{2c_4\pi H }{G}\frac{\ell_{eff}l}{\delta^{2}}+O(\delta),
 \end{align}
 and replacing $c_4=\frac{a_4\ell^2_{eff}}{8}$, we get
 \begin{equation}
  S_{KT}=  -\frac{a_4\pi H }{4G}\frac{\ell^3_{eff}l}{\delta^{2}}+O(\delta).
 \end{equation}
Finally, we can use this result to obtain the universal part of the EE in a cylindrical entangling region as:
 \begin{equation}
     S_{CCG}^{Univ}=S_{CCG}+S_{KT}=-b_4\frac{\pi H\ell^3_{eff}}{16l G}\ln\left(\frac{l}{\delta}\right).
 \end{equation}
 

\chapter{Explicit calculation: EE for the Deformed sphere} \label{Explicit calculation: EE for the Deformed sphere}

In this Appendix we will show the explicit calculation of the $C_T$ in a CCG theory by considering a deformed spherical entanglement region. We start using the embedding of Eq.\eqref{embedding deformed sphere} to compute the squared root of the determinant of the metric:
\begin{align}
   \sqrt{\sigma}&=\frac{\ell_{eff}^{2}}{z^{2}}\left[1+\sum_{l}\bigg(\frac{1}{(1-z^2)}\left(\frac{1-z}{z+1}\right)^{\frac{l}{2}} \left((l^{2}-1) z^{2}+l z+1\right) \left(a_l \cos\! \left(l \phi\right)+b_l \sin\! \left(l \phi\right)\right)  \epsilon\right.\nonumber\\ 
   &-\frac{1}{2 \left(1-z^2\right)^{2}} \left(\frac{1-z}{z+1}\right)^{l}\bigg(  \left(a_l^2-b_l^2\right)  \left(z^{2}+2lz+1\right) \left((1-2l^2)z^2+l^2\right) \cos\! \left(l \phi\right)^{2}\nonumber\\
   &+2 \left(z^2+2lz+1\right)  \left((1-2l^2)z^2+l^2\right) a_l\sin\! \left(l \phi\right) b_l \cos\! \left(l \phi\right)\\
   &+z^{2} \left(z^2-1\right) \left(a_l^{2}+b_l^{2}\right) l^{4}+2\left(\left( a_l^{2}- b_l^{2}\right) z^{3}-a_l^{2} z\right) l^{3}\nonumber\\&+\left(-2 b_l^{2} z^{4}+a_l^{2} z^{2}-a_l^{2}\right) l^{2}+2 b_l^{2} l \,z^{3}+b_l^{2} z^{2} \left(z^{2}+1\right)\bigg)\epsilon^2+ \mathrm{O}\! \left(\epsilon^{3}\right)\bigg)\bigg]\nonumber
\end{align}

We use the same embedding to compute the quantities of interest in the computation of the EE in CCG and the results are  the Eq. \eqref{SR2 deformed CT}-\eqref{SK4 deformed CT}. We will show them here for continuity: 
\begin{align}
    S_{R^2}&=-\frac{6 \left(2 \mu_\mathrm{1}+4 \mu_\mathrm{2}+6 \mu_\mathrm{3}+24 \mu_\mathrm{4}+9 \mu_\mathrm{5}+9 \mu_\mathrm{6}+36 \mu_\mathrm{7}+144 \mu_\mathrm{8}\right)}{\ell_{eff}^{4}}\\
    S_{K^2R}&= \frac{12z^4 (\mu_1 -4 \mu_2 -2 \mu_3 -8 \mu_4)}{\ell_{eff}^4(1-z^2)^2}\sum_{l}\left(\frac{1-z}{z +1}\right)^{l}(a_l^{2}+b_l^{2}) l^{2} (l^2 -1)^{2} \epsilon^{2}\\
    &+\mathrm{O}\! \left(\epsilon^{3}\right)\nonumber\\
    S_{K^4}^{min}&=\mathrm{O}\! \left(\epsilon^{4}\right)\\
     S_{K^4}^{non-min}&=\mathrm{O}\! \left(\epsilon^{4}\right)
\end{align}
We can see from here that at least at order $\epsilon^2$ the two splittings coincide, that is $S_{CCG}^{min}=S_{CCG}^{non-min}=S_{CCG}$. Replacing all this results in the computation for the EE:

\begin{align}
   &S_{CCG}=\frac{\mathcal{A}[\Sigma]}{4G}-\frac{1}{8G}\int_{\Sigma}dx^2\sqrt{\sigma}\left(S_{R2}+S_{K^2R}+S_{K^4}\right)\nonumber \\
   &=\frac{1}{4G}\int_{\Sigma}dx^2\sqrt{\sigma}\left(1-\frac{1}{2}\left(S_{R2}+S_{K^2R}+S_{K^4}\right)\right)\nonumber\\
   &=\frac{1}{4G}\int_{\Sigma}dx^2\left(\frac{\ell_{eff}^{2}}{z^{2}}\bigg[1+\sum_{l}\bigg(\frac{1}{(1-z^2)}\left(\frac{1-z}{z+1}\right)^{\frac{l}{2}} \left((l^{2}-1) z^{2}+l z+1\right) \left(a_l \cos\! \left(l \phi\right)\right.\right.\nonumber\\
   &\left.+b_l \sin\! \left(l \phi\right)\right)  \epsilon-\frac{1}{2 \left(1-z^2\right)^{2}} \left(\frac{1-z}{z+1}\right)^{l}\bigg(  \left(a_l^2-b_l^2\right)  \left(z^{2}+2lz+1\right) \left((1-2l^2)z^2\right.\nonumber\\
   &\left.+l^2\right) \cos\! \left(l \phi\right)^{2}+2 \left(z^2+2lz+1\right)  \left((1-2l^2)z^2+l^2\right) a_l\sin\! \left(l \phi\right) b_l \cos\! \left(l \phi\right)\nonumber\\
   &+z^{2} \left(z^2-1\right) \left(a_l^{2}+b_l^{2}\right) l^{4}+2\left(\left( a_l^{2}- b_l^{2}\right) z^{3}-a_l^{2} z\right) l^{3}+\left(-2 b_l^{2} z^{4}+a_l^{2} z^{2}-a_l^{2}\right) l^{2}+2 b_l^{2} l \,z^{3}\nonumber\\
   &+b_l^{2} z^{2} \left(z^{2}+1\right)\bigg)\epsilon^2\left.+ \mathrm{O}\! \left(\epsilon^{3}\right)\bigg)\bigg]\right)\bigg(a_3\nonumber\\
   &-\frac{6z^4(\mu_1 -4 \mu_2 -2 \mu_3 -8 \mu_4) }{\ell_{eff}^4(1-z^2)^2}\sum_{l}\left(\frac{1-z}{z +1}\right)^{l} (a_l^{2}+b_l^{2})l^{2} (l^2 -1)^{2} \epsilon^{2}+\mathrm{O}\! \left(\epsilon^{3}\right)\bigg)\nonumber\\
   &=\frac{\ell_{eff}^2}{4G}\int_0^{2\pi} d\phi\int_\delta^1dz\bigg[\frac{1}{z^2}\bigg(a_3+\frac{a_3}{(1-z^2)}\sum_{l}\left(\frac{1-z}{z+1}\right)^{\frac{l}{2}} \left((l^{2}-1) z^{2}+l z+1\right) \left(a_l \cos\! \left(l \phi\right)\right.\nonumber\\
   &+\left.b_l \sin\! \left(l \phi\right)\right)  \epsilon -\frac{1}{2 \left(1-z^2\right)^{2}} \sum_{l}\left(\frac{1-z}{z+1}\right)^{l}\bigg(a_3\bigg(  \left(a_l^2-b_l^2\right)  \left(z^{2}+2lz+1\right) \left((1-2l^2)z^2\right.\nonumber\\
   &\left.+l^2\right) \cos\! \left(l \phi\right)^{2}+2 \left(z^2+2lz+1\right)  \left((1-2l^2)z^2+l^2\right) a_l\sin\! \left(l \phi\right) b_l \cos\! \left(l \phi\right)\nonumber\\
   &+z^{2} \left(z^2-1\right) \left(a_l^{2}+b_l^{2}\right) l^{4} 
   +2\left(\left( a_l^{2}- b_l^{2}\right) z^{3}-a_l^{2} z\right) l^{3}+\left(-2 b_l^{2} z^{4}+a_l^{2} z^{2}-a_l^{2}\right) l^{2}+2 b_l^{2} l \,z^{3}\nonumber\\
   &+b_l^{2} z^{2} \left(z^{2}+1\right)\bigg)+\frac{12z^4(a_l^{2}+b_l^{2}) }{L^4}l^{2} (l^2 -1)^{2}(\mu_1 -4 \mu_2 -2 \mu_3 -8 \mu_4)\bigg)\epsilon^2\bigg)\bigg] \nonumber\\
   &=\frac{\pi \ell_{eff}^2}{2G}\int_\delta^1dz\bigg[\frac{1}{z^2}\bigg(a_3-\frac{1}{2 \left(1-z^2\right)^{2}} \sum_{l}\left(\frac{1-z}{z+1}\right)^{l}\bigg(a_3\bigg(  \frac{1}{2}\left(a_l^2-b_l^2\right)  \left(z^{2}+2lz+1\right)\nonumber\\
   &\times\left((1-2l^2)z^2+l^2\right)+z^{2} \left(z^2-1\right) \left(a_l^{2}+b_l^{2}\right) l^{4}+2\left(\left( a_l^{2}- b_l^{2}\right) z^{3}-a_l^{2} z\right) l^{3}\nonumber\\
   &+\left(-2 b_l^{2} z^{4}+a_l^{2} z^{2}-a_l^{2}\right) l^{2} +2 b_l^{2} l \,z^{3}+b_l^{2} z^{2} \left(z^{2}+1\right)\bigg)\nonumber\\
   &+\frac{12z^4(a_l^{2}+b_l^{2}) }{\ell_{eff}^4}l^{2} (l^2 -1)^{2}(\mu_1 -4 \mu_2 -2 \mu_3 -8 \mu_4)\bigg)\epsilon^2\bigg)
   +O(\epsilon^3)\bigg] \nonumber\\
   &=-\frac{a_3\pi \ell_{eff}^2}{2G}\left(1-\frac{1}{\delta}\right)-\frac{\pi \ell_{eff}^2}{8G}\sum_l\bigg[(a_l^2+b_l^2)l(l^2-1)\bigg(1+\frac{3}{\ell_{eff}^4}(4\mu_1-4\mu_2\nonumber\\
   &+2\mu_3+8\mu_4+9\mu_5+9\mu_6+36\mu_7+144\mu_8)\bigg)-\frac{a_3}{\delta}(a^2+b^2)l^2\bigg]\epsilon^2+O(\epsilon^3)
\end{align}
Now we are going to construct the Kouterterm. First we set $z=delta$ to obtain the metric for the entangling region in the boundary:
  \begin{align}
   ds_{\partial\Sigma}^2&=\frac{\ell_{eff}^2}{\delta^2}\bigg((1-\delta^2+O(\delta^3))+\sum_l2\left(a_l \cos\! \left(l \phi\right)+b_l \sin\! \left(l \phi\right)\right)\left(1-\frac{1}{2} l^{2} \delta^{2}+\mathrm{O}\! \left(\delta^{3}\right)\right) \epsilon\nonumber\\
  &+\sum_l\left(\left(a_l^{2}-b_l^{2}\right) \left(l^{2}-1\right) \cos\! \left(l \phi\right)^{2}+2 a_l b_l\left(l^2-1\right) \sin\! \left(l \phi\right)   \cos\! \left(l \phi\right)\right.\nonumber\\
  &\left.-a_l^{2} l^{2}-b_l^{2}\right)\left(-1+\left(l^{2}-1\right) \delta^{2}+\mathrm{O}\! \left(\delta^{3}\right)\right)\epsilon^2 +O(\epsilon^3)\bigg)d\phi^2  
\end{align}
We can use this result to obtain the squared root of the determinant of the codimension-3 metric (we need it to obtain the Kounterterm):
\begin{align}
\sqrt{\tilde{\sigma}}&=\frac{\ell_{eff}}{\delta}\bigg(1-\sum_l\left(a_l \cos\! \left(l \phi\right)+b_l \sin\! \left(l \phi\right)\right)\epsilon\nonumber\\
&-\sum_l\frac{l^2}{2} \left(2 \cos\! \left(l \phi\right) \sin\! \left(l \phi\right) a_l b_l+(a_l^{2}-b_l^{2}) \cos\! \left(l \phi\right)^{2}-a_l^{2}\right)\epsilon^2\bigg)+O(\epsilon^3,\delta)    \label{metric Kounterterm deformed CT}
    \end{align}
On the other hand for $D=4$ the $S_{KT}$ is defined in codimension-3 and therefore we will need to compute $B_1$:
\begin{equation}
 B_1=-2\int_{0}^{1}dsTr(\kappa)=-\frac{2}{\ell_{eff}}+O(\epsilon^3,\delta)  \label{B1 deformed CT}
\end{equation}
Now we can replace \eqref{metric Kounterterm deformed CT} and \eqref{B1 deformed CT} in Eq.\eqref{HEE in CCG Kounterterm}
 \begin{align}
  S_{KT}&=\frac{c_3}{4G}\bigg\lfloor\frac{3+1}{2}\bigg\rfloor\int_{\partial\Sigma}dx^2\sqrt{\tilde{\sigma}}B_1\\
  &=\frac{c_3}{2G}\int_{0}^{2\pi}d\phi\bigg[\frac{\ell_{eff}}{\delta}\bigg(1-\sum_l\left(a_l \cos\! \left(l \phi\right)+b_l \sin\! \left(l \phi\right)\right)\epsilon\\
  &-\sum_l\frac{l^2}{2} \left(2 \cos\! \left(l \phi\right) \sin\! \left(l \phi\right) a_l b_l+(a_l^{2}-b_l^{2}) \cos\! \left(l \phi\right)^{2}-a_l^{2}\right)\epsilon^2\bigg)\left(-\frac{2}{\ell_{eff}}+O(\delta^2)\right)\bigg]\\
  &=-\frac{2\pi c_3}{G}\frac{1}{\delta}\left(1+\sum_l\frac{l^2}{4}(a_l^2+b_l^2)\epsilon^2\right)+O(\epsilon^3,\delta)  
 \end{align}
 we can express $c_4=\frac{a_3\ell_{eff}^2}{4}$, therefore:
 \begin{equation}
  S_{KT}=  -\frac{\pi \ell_{eff}^2 a_3}{2G}\frac{1}{\delta}\left(1+\sum_l\frac{l^2}{4}(a_l^2+b_l^2)\epsilon^2\right)+O(\epsilon^3,\delta)  
 \end{equation}
Notice that this result is the same (with a negative sign) as the divergent part of the EE, therefore:
 \begin{align}
     S_{CCG}^{Univ}=S_{CCG}+S_{KT}&=-\frac{\pi \ell_{eff}^2}{2G}\bigg[a_3+\sum_l\frac{1}{4}(a_l^2+b_l^2)l^2(l^2-1)\bigg(1+\frac{3}{\ell_{eff}^2}(4\mu_1-4\mu_2+2\mu_3\nonumber\\&+8\mu_4+9\mu_5+9\mu_6+36\mu_7+144\mu_8)\bigg)\epsilon^2\bigg] +O(\epsilon^4)
 \end{align}
If we compare with Eq.\eqref{SEE eps} we can express the subleading term ($\epsilon^2$ term) as:
\begin{equation}
     S_{EE}^{Univ,CCG,(2)}=\frac{\pi^4 C_T^{CCG}}{24}\sum_{l}l(l^2-1)(a_l^2+b_l^2)
\end{equation}
where $C_T^{CCG}$ is given in  Eq.\eqref{Ct CCG}. Now, as an example, let us calculate the $C_T$ for Einstein cubic gravity. We set the couplings of the general CCG theory to be :
\begin{equation}
  \mu_1=-\frac{3}{2}\mu  \:\: ; \:\: \mu_2=-\frac{\mu}{8} \:\:  ;\:\:\mu_3= 0 \:\: ; \:\: \mu_4= 0 \:\: ; \:\:\mu_5= \frac{3}{2}\mu  \:\: ;\:\:  \mu_6= -\mu \:\: ;\:\:\mu_7= 0  \:\:;\:\:\mu_8=0
\end{equation}
replacing in  Eq.\eqref{Ct CCG} we obtain:
\begin{equation}
    C_T^{ECG}=\bigg(1-\frac{3}{\ell_{eff}^4}\mu\bigg)C_T\label{Ct ECG}
\end{equation}

