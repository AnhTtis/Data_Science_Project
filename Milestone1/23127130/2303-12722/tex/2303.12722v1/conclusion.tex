\section{Conclusion, Limitations, and Future Work}
\label{con}


\begin{figure}
\centering
\vskip -5pt
\includegraphics[width=0.30\linewidth]{figures/case_studies_on_real_images.pdf}
\vskip -5pt
\caption{\small Inversion on complex (fractal) images.}  
\label{fig:real}
\vskip -15pt
\end{figure}


We present a gradient descent-based method to find the fractal parameters for a given image. 
Our approach is stable, effective, and can be easily implemented using popular deep learning frameworks. 
It can also be extended to finding fractal parameters for other purposes, \eg, to facilitate downstream vision tasks, by plugging in the corresponding losses.  


We see some limitations yet to be resolved. For example, IFS itself cannot handle color images. Our current approach is hard to recover very detailed structures (\eg,~\autoref{fig:real}). We leave them for our future work. 
We also plan to further investigate our approach on 3D point clouds. 
