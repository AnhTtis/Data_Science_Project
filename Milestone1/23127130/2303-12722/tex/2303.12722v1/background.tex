\section{Background}
\label{background}

\begin{figure*}[t]
\begin{minipage}{.46\linewidth}
\begin{algorithm}[H]
\footnotesize
\SetAlgoLined
\caption{IFS generation process.\\ See~\autoref{background} for details.}
\label{alg:ifs}
\SetKwInOut{Input}{Input}
\SetKwInOut{Output}{Output}
\Input{Fractal parameter $\sS$, \# of iterations $T$, an initial point $\vv^{(0)}\in\R^2$}
\For{$t\leftarrow 1$ \KwTo $T$}{
{\textbf{Sample} $(\mA^{(t)}, \vb^{(t)})$ from $\sS$}\\
{$\vv^{(t)} = \mA^{(t)} \vv^{(t-1)} +\vb^{(t)}$}
}
\Output{$\{\vv^{(0)},\cdots,\vv^{(T)}\}$}
\end{algorithm}
\end{minipage}
\hfill
\begin{minipage}{.46\linewidth}
\begin{algorithm}[H]
\footnotesize
\SetAlgoLined
\caption{IFS generation process via the \FE layer.\\ See~\autoref{ss_deep} for details.}
\label{alg:deep_ifs}
\SetKwInOut{Input}{Input}
\SetKwInOut{Output}{Output}
\Input{Fractal embedding $\{\mS_\text{A}, \mS_\text{b}\}$, \# of iterations $T$, an initial point $\vv^{(0)}\in\R^2$}
{\textbf{Sample} index sequence $\vz = [z^{(1)}, \cdots, z^{(t)}]$;}\\
\For{$t\leftarrow 1$ \KwTo $T$}{
{$\vv^{(t)}=\texttt{mat}(\mS_\text{A}[:, z^{(t)}]) \vv^{(t-1)} + \mS_\text{b}[:,
z^{(t)}]$;}
}
\Output{$\{\vv^{(0)},\cdots,\vv^{(T)}\}$}
\end{algorithm}
\end{minipage}
\vskip -15pt
\end{figure*}



We first provide the background about fractal generation via Iterated Function Systems (IFS)~\cite{barnsley2014fractals}. 
As briefly mentioned in~\autoref{intro}, an IFS can be thought of as ``drawing'' points iteratively on a canvas. The point transition is governed by a small set of $2\times 2$ affine transformations $\sS$: 
\begin{align}
    \label{eq:fractal_code}
	\sS = \left\{\left(\mA_n\in\R^{2\times 2}, \vb_n\in\R^2, p_n\in\R \right)\right\}_{n=1}^N. 
\end{align}
Here, an $(\mA_n, \vb_n)$ pair specifies an affine transformation:
\begin{align}
\label{eq:IFS}
\mA_n {\vv} +\vb_n,
\end{align}
which transforms a 2D position $\vv\in\R^2$ to a new position. The $p_n$ in \autoref{eq:fractal_code} is the corresponding probability of each transformation, \ie, $\sum_n p_n = 1$ and $p_n\geq 0, \forall n\in[N]$. $N$ is typically $2\sim10$. 

Given $\sS$, an IFS generates an image as follows. Starting from an initial point ${\color{red}\vv^{(0)}}\in\R^2$, an IFS repeatedly samples a transformation $(\mA^{(t)}, \vb^{(t)})$ from $\{(\mA_n, \vb_n )\}_{n=1}^N$ with replacement following the probability $\{p_n\}_{n=1}^N$, and applies 
\begin{align}
\label{eq:IFS}
{\color{red}\vv^{(t)}} = \mA^{(t)} {\color{red}\vv^{(t-1)}} +\vb^{(t)}
\end{align} 
to arrive at the next 2D point. This stochastic process can continue forever, but in practice we set a pre-defined number of iterations $T$.
The traveled points $\{\vv^{(0)},\cdots,\vv^{(T)}\}$ are then used to synthesize a fractal image. For example, one can quantize them into integer pixel coordinates and render each pixel as a binary (\ie, $1$) value on a black (\ie, $0$) canvas. We summarize the IFS fractal generation process in \autoref{alg:ifs}. Due to the randomness in sampling $(\mA^{(t)}, \vb^{(t)})$, one $\sS$ can create different but geometrically-similar fractal images. We call $\sS$ the fractal parameters. 


\paragraph{Simplified parameters.} We follow~\cite{kataoka2020pre,Anderson_2022_WACV} to set $p_n$ by the determinant of the corresponding $\mA_n$. That is, $p_n\propto|\texttt{det}(\mA_n)|$. In the following, we will ignore $p_n$ when defining $\sS$:
\begin{align}
    \label{eq:fractal_code_simple}
	\sS = \left\{\left(\mA_n, \vb_n\right)\right\}_{n=1}^N. 
\end{align}
