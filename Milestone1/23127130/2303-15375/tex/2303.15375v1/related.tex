% !TEX root = paper.tex
\section{Related Work}
\label{sec:related}
With the rapid development of memory technologies, many new types of memory other than the regular DDR-based DRAM have emerged in datacenters, each with distinct characteristics and trade-offs. These include but are not limited to persistent memory such as Intel Optane DIMM~\cite{hemem,fast20yang,eurosys22xiang}, RDMA-based remote/disaggregated memory~\cite{farm, infiniswap, socc20kalia}, and even byte-addressable SSD~\cite{10.1145/3297858.3304061,8416845}. While they have been extensively studied and profiled, CXL memory, as a new member in the memory tier, still has unclear performance characteristics and indications. 

Since the concept's inception in 2019, CXL has been discussed by a body of researchers. For instance, Meta envisions using CXL memory for memory tiering and swapping~\cite{tpp}; Microsoft built a CXL memory prototype system for memory disaggregation exploration~\cite{pond,10034802}. Most of them used NUMA machines to emulate the behavior of CXL memory. Gouk \etal built a CXL memory prototype on FPGA-based RISC-V CPU~\cite{directcxl}. Different from the prior studies, we are the first to conduct CXL memory research on the commodity CPU and CXL device with both micro-benchmarks and real applications. This makes our research more realistic and comprehensive. 

