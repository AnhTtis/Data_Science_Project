% !TEX root = paper.tex
\begin{abstract}


\footnote{This work has been accepted by a conference. The authoritative version of this work  will appear in the Proceedings of the IEEE/ACM International Symposium on Microarchitecture (MICRO), 2023. Please refer to \url{https://doi.org/10.1145/3613424.3614256} for the official version of this paper.}The ever-growing demands for memory with larger capacity and higher bandwidth %by datacenter servers
have driven recent innovations on memory expansion and disaggregation technologies based on Compute eXpress Link %\textsuperscript{TM} (CXL\textsuperscript{TM}).
(CXL).
%
Especially, CXL-based memory expansion technology has recently gained notable attention for its ability not only to economically expand memory capacity and bandwidth but also to decouple memory technologies from a specific memory interface of the CPU.
%
However, since \cxlmem devices %and systems that support them 
have not been widely available,
%due to the limited availability of \cxlmem devices, 
they have been emulated using DDR memory in a remote NUMA node. %a remote NUMA node in a multi-socket system.
%
%Recent studies on the performance characterizations and efficient uses of CXL memory commonly treat CXL-memory systems as conventional NUMA systems, where CXL memory is emulated using a remote NUMA node %(without CPU cores and caches) 
%in a multi-socket system. 
%
In this paper, for the first time, we comprehensively evaluate a true CXL-ready system based on the latest 4\textsuperscript{th}-generation Intel Xeon  CPU 
%(Sapphire Rapids) 
with three CXL memory devices from different manufacturers. 
%
Specifically, we run a set of microbenchmarks not only to compare the performance of true CXL memory with that of emulated CXL memory %(\ie, DDR memory in a remote NUMA node) 
but also to analyze the complex interplay between the CPU and CXL memory in depth. 
%
This reveals important differences between emulated \cxlmem and true \cxlmem, some of which will compel researchers to revisit the analyses and proposals from recent work.
%
Next, we identify opportunities for memory-bandwidth-intensive applications to benefit from the use of \cxlmem.
%
Lastly, we propose a CXL-memory-aware dynamic page allocation policy, \policy to more efficiently use CXL memory as a bandwidth expander.
%
We demonstrate that \policy can automatically converge to an empirically favorable percentage of pages allocated to \cxlmem, which improves the performance of memory-bandwidth-intensive applications by up to 24\% when compared to the default page allocation policy designed for traditional NUMA systems. 

\end{abstract}
