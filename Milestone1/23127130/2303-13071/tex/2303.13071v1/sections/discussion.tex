\section{Discussion} \label{sec:discussion}

% \paragraph{Limitations and Future Work.}
\noindent{\textbf{Limitations and Future Work.}}
While PanoHead exhibits excellent images and shapes quality from $360^\circ$, it still contains minor artifacts, e.g. in the teeth area.  
Similar to the original EG3D, flickering texture issue is also noticeable in our model. Switching to StyleGAN3~\cite{Karras2021} as the backbone would help preserve high-frequency details. In practice, we also observe more noticeable flickering artifacts with a higher swapping probability of the conditional camera pose. We set this value to 70\% as opposed to 50\% in EG3D since we empirically find it enhances $360^\circ$ rendering quality but at the minor cost of flickering texture artifacts. 
Another observation is that it lacks finer high-frequency geometric details, e.g. hair tips. We leave it as future work to quantitatively evaluate our geometric quality such as using depth maps.
Finally, although PanoHead is able to generate diverse images in terms of gender, races, and appearances, reliance on training with only several datasets combination still makes it suffer from data bias, to some extent. In spite of our data collection effort, large-scale full-head annotated training image dataset is one of the most critical directions to facilitate full-head synthesis research. We anticipate such datasets can resolve some of the limitations aforementioned. 

\begin{figure}[t]
    \centering
    \includegraphics[width=1\textwidth]{figures/pti2.pdf}
    \caption{Single-view reconstruction from different camera poses. The first column shows the target images, second column projected RGB images and reconstructed 3D shapes using GAN inversion, last two columns rendered images from any given camera poses. }
    \label{fig:recon}
\end{figure}
% \vspace{-1.2em}
% \paragraph{Ethical considerations.}

\noindent{\textbf{Ethical considerations.}}
PanoHead is not specifically designed for any malicious uses, yet we do realize that the single-view portrait reconstruction could be manipulated, which might pose a social threat. We do not encourage the method being used for violating others' rights in any forms.
% \vspace{-1.2em}
\section{Conclusion}
% \paragraph{Conclusion.}
We propose PanoHead, the first 3D GAN framework that synthesizes view-consistent full head images with only single-view images. With our novel design in foreground-ware tri-discrimination, 3D tri-grid scene representation, and self-adaptive image alignment, PanoHead enables authentic multiview-consistent full-head image synthesis in $360^\circ$ and demonstrates compelling qualitative and quantitative results compared with state-of-the-art 3D GANs. Furthermore, we present 360-degree photo-realistic reconstruction
 with highly detailed geometry from single-view real portraits.  
 We believe the proposed method presents an interesting direction for 3D portraits creation, which sheds light on many potential downstream tasks.
