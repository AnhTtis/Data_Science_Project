


\section{RQ3: Understanding User Expectations}
\label{sec:RQ3}
\begin{figure*}[t]
\centering
\includegraphics[width=0.5\textwidth]{fig/rank.pdf}
\caption{Overall payment preference of the four probes. The bars are color-coded dark blue, light blue, orange, and red to indicate the fraction of participants that selected each probe respectively to be their 1st, 2nd, 3rd, and last choice.}\label{fig:payment_pref}
\end{figure*}

After identifying how interaction experiences shape the security perception of VR authentication users, we turn our attention to participants' expectations of an ``ideal'' authentication  experience. Our subsequent discussion builds on our findings in \textbf{RQ1} and \textbf{RQ2} as well as qualitatively analyzing participant responses to the surveys. 

\subsection{Misalignment between the Perceived and Actual Security of VR Authentication}

User authentication in VR is a new concept for users, where \textit{a gap still exists between the perceived and actual security.} The technical properties of VR authentication mechanisms may differ between the real world and VR, despite the similar virtualization metaphors. For example, the technology to secure a VR token would be different than the EMV chip in a physical card even though they share a card metaphor. We observed that some participants, even those who were technologically savvy, tended to overestimate the security of VR. On the other hand, we found that making the authentication more oblivious and seamless to users may make them underestimate the actual security in VR (Section~\ref{sec:RQ2-1-1}). The observations are associated with users' lack of understanding of these novel technologies in VR to assess the actual security. For example, one participant expected the transparency for the VR authentication protocol to be \textit{``well-defined and open-sourced''} for properly evaluating its security (P3). 

Another reason behind this gap is that participants prioritize interaction experience over security, resulting in an incorrect perception of the security of the authentication method. Figure~\ref{fig:overall_perception} and~\ref{fig:payment_pref}  show participants selecting what they considered the less secure, yet the more usable, \tap as their preferred authentication method. Participants attempted to bridge this gap by envisioning new alternatives of interaction modalities for all different concepts of authentication, such as new ways to interact with a card and drawing 3D signature \textit{``instead of a flat 2D traditional signature''} (P5). Along the same lines, researchers and developers have been actively improving both the usability and security of VR authentication~\cite{mathis2021fast,watson-chi22, olade-icvars20,zhu2020blinkey,liebers-chi21}.


\subsection{Challenges in Being Aware of Threats Across VR and the Real World.} 
Participants enjoyed being present and immersed in the VR environment. Meanwhile, they desired awareness of security threats in VR and the physical world. For example, when immersed in VR, participants wanted the VR application \textit{``to reflect the surrounding environment''} both virtually and physically for them to notice bystanders (P21). Another instance is one participant's need to know about suspicious activities in VR by \textit{``sending me [them] an SMS/email/etc.)''} even when they are back in the physical world (P8). 

Unfortunately, offering such cross-contextual awareness with an enjoyable experience is still an open challenge, even when the users develop a proper understanding of VR authentication. There are several reasons. First, VR creates an immersion effect, and users have limited perceptual capacities~\cite{tseng-chi22}, which is more challenging compared to more conventional digital mediums, such as websites and mobile apps. Furthermore, it is challenging to interpret suspicious activities when users are present in VR. Because the VR threats can be made more imperceptible, e.g., invisible card-skimming attacks (Section~\ref{sec:RQ2-2}), and users miss the physical context to better understand bystander activities in VR. 


\subsection{Difficulties of Assessing Security Risks in Playful VR.}


Many users have high expectations for VR technology and applications, seeking creative and enjoyable experiences \cite{olsson-puc13, tang-chi23}. In our study, we found that participants also desired a playful authentication experience and proposed various ideas to achieve it. These ideas focused on improving usability and implementation, such as the interface to manage multiple virtual cards where users could \textit{``select the card from a pop up UI''} (P20) and informing payment success with \textit{``more fancy effects''} (P24). Participants also anticipated emotional appeals (e.g., enjoyment and playfulness) from the authentication interfaces, especially in the game context. 
\begin{displayquote} 
\textit{``I want the payment process to be cool and make me feel good. some visual effects could help and make me happy to make the payment.''} (P24) 
\end{displayquote} 
  
However, it is important to consider the trade-off associated with a playful authentication experience. As discussed in Section~\ref{sec:RQ2-3}, gamifying the VR experience can reduce participants' sensitivity to security risks and feedback. This reduction in awareness may have significant consequences and pose risks in security-sensitive scenarios, such as high-risk payments. This concern aligns with previous research conducted in the domain of digital games \cite{chen-chb10}.


\noindent
\colorbox{beaublue}{\textbf{Takeaway-RQ3.}} 
While participants prioritized interaction experience, the security of authentication is important to them. Participants expected both usability and security, but their expectations appeared conflicting. Our findings highlight three tensions. First, we noticed a gap between the perceived and actual security in VR authentication as some other contexts~\cite{khan-soups15}, and the VR interactions further exacerbate the discrepancy. Second, participants liked the immersion of VR while expressing concerns about threats in both the virtual and physical worlds. Third, although participants envisioned playful experiences, such experiences can diminish their sensitivity to security risks. 













