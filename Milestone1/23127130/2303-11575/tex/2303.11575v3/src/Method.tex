
\section{Study Method}
We use a ``technology probe'' approach to understanding how interaction experiences affect participants’ perception of authentication in VR and extract design guidelines~\cite{hutchinson-chi03}. The idea of the ``technology probe'' approach entails using a set of proof-of-concept interfaces. As these interfaces package basic interactions, researchers can reveal phenomena otherwise hidden from user interactions~\cite{chandrasekaran-soups21}
This approach is commonly used to evaluate emerging technologies, including VR and user authentication~\cite{mathis-vr22,tang-chi21}.
Our probes implement the core interaction patterns of user authentication to elicit user responses in regard to their interaction experience and perception of security. We deploy our probes for participants to make routine payments in a VR game, which provides a context that mimics VR payments in real-world scenarios.



\subsection{Authentication Probes}\label{sec:probe design}
Here, we describe the process that led to the design of four authentication probes in VR. 


\subsubsection{Design process.} 
Our experiment design primarily aims to create a realistic payment authentication scenario to capture authentic reactions from the participants. Toward that end, we iteratively developed our design by considering the usage context and probes together. Our research team conducted regular meetings and tests to discuss, test, and refine our concepts and implementations. 

To explore the probe design space, we drew insights from research literature and industry products, many of which are still in the early stage of commercialization. We identified three design dimensions of VR authentication: providing something you know, showing something you have, and proving who you are~\cite{ross-nist19}. These dimensions cover unique interaction patterns and inherent security properties. By including these variations, our probes helped us maximally capture different user interactions and identify common themes. We then narrowed the design scope down to authentication based on PIN, token, and behavior (gesture). 

Rather than replicating authentication interfaces in other digital contexts such as mobile apps, we designed the probes to establish a mapping of real-world experiences into VR. Such mapping has already become a popular design choice in other applications of VR for authentication~\cite{argelaguet-cg13}. To provoke participants, we opted not to optimize the usability of probes for everyone in our implementation. We, however, made our design \textit{flexible} by allowing varying gestures to interact with the interface and modification of inputs.

Our context for authentication initially involved payment at a vending machine that dispenses virtual objects. However, we found, in pilot studies, that this task was not ideal for engaging people in VR. As a result, we shifted our focus to integrating payment authentication within VR games, a more popular scenario. Building on this context, we further refined our probe design of payment authentication to align with the virtualization of this context. 

 

\subsubsection{The Four Probes.}

We designed four probes: (1) floating PIN pad (\pinf), (2) on-kiosk PIN pad (\pink), (3) tap-to-pay (\tap), and (4) signature (\sign). We now describe these four probes, which are shown in Figure~\ref{fig:Probe_design}.

\paragraph{$\bullet$ Floating PIN pad (\pinf).}
\pinf resembles the default virtual input interface of many VR platforms, where participants interact with a floating PIN pad. \pinf also conceptualizes the idea of giving participants a personal and isolated virtual experience in the authentication.  Participants enter PINs by pointing to a floating PIN pad that follows the participants’ viewport.  \pinf randomizes the PIN layout, which is a common security mechanism in digital PIN pads against observers~\cite{mathis-vr22}. Note that \pinf serves more like a baseline authentication probe.

\paragraph{$\bullet$ On-kiosk PIN pad (\pink).}
In contrast to \pinf, \pink presents a better mapping of physical-world experiences by rendering its PIN pad on the kiosk where participants initiate and confirm payment. As such, participants do not experience a gap in the transition between authentication and other payment tasks, e.g., selecting the items.  

\paragraph{$\bullet$ Tap-to-pay (\tap).} 
\tap represents how participants could own and use a personal virtual object as a unique token to authenticate. To pay, participants take out a virtual credit card from an inventory on their virtual body and tap the card on the kiosk display. The kiosk checks whether the card is in proximity and being held by the user. 

\paragraph{$\bullet$ Signature (\sign).} 
\sign represents how participants draw a unique signature to give consent and prove their identity. It virtualizes the real-world signing processes.
Participants grab a virtual pen from the kiosk and sign in the designated area using the hand-held controller; Only after participants sign, they can proceed to check out.\\


Our design achieves the goals of the technology probe method to collect in-context information about the usage, test out the technology, and inspire future design by upholding the principles as proposed by Hutchinson et al.~\cite{hutchinson-chi03}. As the functionality of technology probes should be simple, our designs only package the essential frontend interaction instead of the full backend of authentication mechanisms, e.g., verifying an encrypted token or comparing signatures. Instead, our study adopts the idea in ``Wizard-of-Oz'' studies~\cite{mecke-icmum18} in creating an impression that the system has full functionality through a mock registration process. Also, our probes also support logging of participants' interaction with the interface, including the fine-grained timing, to supplement the analysis. 


\begin{figure*}[h]
\centering
\includegraphics[width=\textwidth]{fig/Probe_design_updated.pdf}
\caption{Our four probes for VR authentication.}\label{fig:Probe_design}
\end{figure*}

\subsection{Experiment Context: VR Archery Game}
\label{sec:game_design}

We design a VR game -- an archery contest -- where participants trade in-game credits, using the probes to authenticate their payment. Participants earn credits by shooting virtual targets and consume these points when they refill arrows. 
To make the game and payment realistic to participants, we match their in-game credits with a physical compensation: the participant who scores the highest wins a ``grand prize'' (a small gift).
Such compensation, which is added to our base compensation for participating in the study, is standard in prior studies that include gamification~\cite{brewer-idc13, mccoy-18, tondello-19}. We determined the value of the prize (a 90 USD fitness tracker) based on our local wages. Below, we describe the main components of this game (see Figure~\ref{fig:game_scene}).  



\begin{figure*}[t]
\centering
\includegraphics[width=\textwidth]{fig/Scene_design_updated.pdf}
\caption{The environment of our archery game with key components marked and key scenes illustrated. }\label{fig:game_scene}
\end{figure*}



\paragraph{Environment.} We situate the contest in an indoor archery range. We assign the participant a chamber. In the chamber, the participant can find the bow set and a kiosk instructing them to shoot, refill arrows, and authenticate their payment. We place the targets towards the other end of the room at a distance. The participant can also find the information display that shows their current credits and remaining progress. 

\paragraph{Bow set.} The participant can interact with the bow using the two hand-held VR controllers. They need one hand to hold the bow and the other hand to draw. The participant shoots the arrow toward the target by releasing the drawing hand. The participant will need to refill after they use up three arrows. Each arrow consumes ten credits.

\paragraph{Targets.} We place eight targets in the range at two different distances. Taking down each closer target rewards 20 credits to the participants while 30 for the farther targets. 

\paragraph{Kiosk.} The kiosk displays game instructions and payment interfaces. To refill arrows, participants interact with the kiosk to select how many arrows to load. After the participant confirms the selection, the kiosk will display the authentication interface among one of the four probes. Once the participant completes authentication, the kiosk will display a waiting page that emulates the running backend processes. After that, the kiosk will accept the credit payment and refill arrows. 

\paragraph{Information display.} The display shows the current credit, the remaining arrows, and how many rounds the participant has completed. 



\paragraph{Implementation.}
We implemented our VR game using Unity, a mainstream VR engine, and C\#.
In the experiment, we ran the game on a commodity PC (CPU: Intel i7-12700F, GPU: Nvidia 3060Ti), which is connected to an Oculus Quest 2 VR headset.
Participants interacted using the headset and its hand-held controllers.












\subsection{Experiment Design}
Here we explain our experimental procedure, the instruments we used to collect participants' responses, and our recruitment.
\subsubsection{Experimental Procedure.}\label{sec:expreiment_procedure}


\begin{figure*}[t]
\centering
\includegraphics[width=\textwidth]{fig/Game_flow_updated.pdf}
\caption{Our experimental procedure.}\label{fig:game_flow}
\end{figure*}


We designed a with-in-subject experiment to evaluate the four probes.
The study consists of two phases--an enrollment phase and the main study phase.
The purpose of enrollment is to familiarize participants with our VR and authentication setups. In the main study, participants played the VR game and interacted with our probes during authentication.
Figure~\ref{fig:game_flow} illustrates our experimental procedure.

\paragraph{Enrollment.} The experimenter first obtained consent and asked the participant to complete a pre-study survey. After that, the experimenter explained the game context to the participants. We assigned participants a ``nickname’’ for the game. The experimenter then introduced the PIN, the virtual card, and the signature setup to the participant (we assigned each participant the same credential information for the statistics of authentication time). We described to the participants that they would enroll the information, including the signature, for the use of authentication in the second phase. 
The experimenter then walked participants through how to use our VR setup.
After the tutorial, the experimenter let the participants practice with the registration kiosk, similar to the one they will use for authentication, and present their credentials to it.
After registration, the participant will practice archery. The enrollment phase took about 20 minutes.

\paragraph{Main study.} The one-hour main study took place on another day to reduce participants’ fatigue.
We first reminded the participants of the study procedure, game context, and the authentication interaction.

Then, the participants entered the game with 400 points. They proceeded to finish four sessions; each included the game with an in-study survey. In each session, the participants paid using one of the probes (in a randomized order) to authenticate. Each session consisted of three rounds. Participants authenticated to pay for the arrows (min: 1, max: 3), except for the first round of every session where three arrows were given for free. At the end of each session, the participants completed an in-study survey, and following the second session, they were given a brief break outside of VR. Once all four sessions were completed, participants were instructed to fill out a post-study survey. After the completion of the survey, participants redeemed their credits to compete for the grand prize.

After the experiment, we disclosed our full study purpose to the participants. We clarified that our focus is to evaluate the interaction experiences and perceptions from the frontend, and we did not process their information, i.e., signature, in the backend.

\subsubsection{Instruments.}
\label{sec:Instruments}
Our study mainly relies on surveys that elicit participants' responses for our qualitative analysis. 
We chose to use surveys instead of more active methods such as think-aloud studies~\cite{charters-bej03} to minimize the interference with participants' game experiences.

\paragraph{Pre-study survey.} In the pre-study survey, we collected participants' demographic backgrounds, including their gender, age, education, and experience with VR. We also used the standard affinity for technology interaction (ATI) score (computed from 6-point Likert ratings) to understand participants' tech-savviness~\cite{franke-ijhci19}.


\paragraph{In-study survey.} The in-study survey assessed (1) participants' general experience in the VR context and (2) the perceived usability of authentication probes.   
For the former, we used the IPQ VR presence questionnaire, a standard measure of participants' sense of presence in VR. 
The sense of presence is a commonly adopted measure of VR experience. It is defined as participants' subjective perception of being and acting in the virtual world (though their body resides in the physical world)~\cite{schwind-chi19}.
This measure consists of four sub-scales (1) sense of being here (PRES), (2) spatial presence (SP), (3) involvement (INV), and (4) experienced realism (REAL).
For the perceived usability, we collected participants' open-ended comments on the usability issues of each authentication probe when participants exited VR. Moreover, participants completed the system usability scale (SUS), a standard measure to assess the overall perceived usability, for each authentication probe~\cite{brooke-jus13}.

\paragraph{Game log.} In addition, we logged participants' behaviors and timings in the game, including their archery performance as well as the time spent on authentication.
We used these objective behavior logs to support our qualitative findings.

\paragraph{Post-study survey.}
The post-study survey consists of three major components. 
First, to understand participants' engagement in the routine payment of the game, we asked them to explain how they decided on the number of arrows to pay. 
Second, we wanted to understand participants' security perception of payment authentication.
We designed questions to elicit participants' responses from different angles. 
From prior research, we identified the five aspects related to the security of payment authentication, namely \textit{consent}~\cite{herzberg-cacm03,lyastani-sp20}, \textit{security}~\cite{kim-ecra10},  \textit{privacy}~\cite{zimmermann-ijhci20}, \textit{being alerted}~\cite{reese-soups19,khan-soups15,wolf-chi19}, and \textit{in control}~\cite{nilsson-chi05}.
We asked the participants to evaluate and elaborate on their agreement on statements related to the five aspects. One example is: when the participant used \tap to pay for arrows, ``I [the participant] felt that my [the participant's] payment was secured''.
We collected both their Likert-scale rating and open-ended responses to explain perceptions. 
However, rather than relying on the quantitative ratings, we mainly study the relation between interaction experiences and perception from their open-ended responses.
Third, the post-study survey asked participants about their preferences among the four probes, their quality expectations that affect their preferences, and their suggestions to improve these probes.
We use these questions to further understand participants' expectations of VR authentication.




\subsubsection{Participants and Recruitment.}
Consistent with prior work in VR authentication~\cite{mathis-vr22}, we recruited 24 participants from our organization. We stopped recruiting when we observed data saturation from our qualitative coding~\cite{guest-fm06}. Each participant received compensation (a 30 USD gift) after they completed the study.

The demographics of participants are as follows. 18 out of 24 identified themselves as man (6 women). The average age of our participants is 29.6 years old (std: 4.8). 20 of them completed or were studying for a graduate degree. Most participants have a background in computer science. Participants' ATI score (mean: 4.35, std: 1.1) shows a high affinity for technology interaction)~\cite{franke-ijhci19}. 20 participants had used VR before (mainly for gaming). But none of the 20 participants frequently used it.


\subsubsection{Ethics and participant safety.} Our study and recruitment were approved by the IRB-equivalent body of our organization. While the study poses a low risk to participants, we took the following steps to protect their physical safety and data privacy. First, we communicated potential risks and their right to withdraw through our informed consent and disclosure processes. Second, we spread our study into two days to minimize fatigue. In addition, we regularly checked in with the participants and made sure they were comfortable to proceed during the study. Third, we ensured that our physical space was clear and safe for the use of VR. The VR application notified our participants when they got close to physical boundaries. Last, our data collection and analysis do not include sensitive personal information, and all data were anonymized and stored in a secure server in our organization.



\subsection{Data Analysis}
We analyzed survey and log data as described in Section~\ref{sec:Instruments}. To analyze such qualitative data, the first author started open coding and took memos while recruiting participants. Meanwhile, another researcher coded data independently. We coded each response the participant made corresponding to a question. The whole team discussed the memos, reconciled the codes, and refined the codebook iteratively. The two coders reached high inter-rater reliability (Cohen’s Kappa $\kappa$ = 0.81) using responses from 6 randomly sampled participants for each question (78 out of 312 responses in total), then we converged on a codebook to code the rest of the data.
Using Grounded Theory~\cite{walker2006grounded}, high-level themes emerge from our coding.
When we observed data saturation from our coding~\cite{saunders2018saturation}, we stopped recruiting. 
We make our codebook available via an anonymous link.\footnote{Our codebook and survey are available at: \url{https://osf.io/re9py/?view_only=e2e01010cf484bd8b476442a0c92398b}}


In Figure~\ref{fig:framework}, we show our analysis framework along with themes we identified from the analysis.
In addition, we report the quantitative data to support our qualitative analysis, including the IPQ questionnaire, SUS scales, and participants’ Likert-scale rating on security and privacy perception.




\subsection{Limitations}

Our study has the following limitations. First, our participant population has a demographic bias, e.g., most participants are tech-savvy. Our population and demographics are comparable with prior work applying similar methods~\cite{ohagan-chi23, yan-imwut20,liang-imwut20, mathis-vr22}. Using a technology probe, our study's objective is not to generalize but present a set of findings and recommendations that guide future design.
Nevertheless, our analysis reveals that even these tech-savvy participants faced challenges in interacting with VR and assessing security, let alone other users. 
Future work may study the probes with a more diverse population, generalize the findings, and quantitatively measure users' experiences and perceptions. 
Second, our study does not investigate how users use VR authentication longitudinally in the wild where users’ experience and perception of security may change over time. In addition, participants' self-reported responses may be biased due to social desirability~\cite{phillips-ajs72}. Last, the purpose of our study is to understand users with probes in an early design phase. We focus on the novel experience of fundamental authentication concepts in VR payment. As such, we did not explore all alternative implementations of the VR interaction and interface.
Despite these limitations, we believe that our work still presents a significant contribution. To the best of our knowledge, our exploratory study is the first to investigate the interplay between interaction experience and perception of security for VR authentication.













