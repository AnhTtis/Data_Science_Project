

\section{Introduction}




Virtual reality (VR) systems immerse individuals in a digital world, one that simulates real-world interactions with objects and characters~\cite{burdea-vrt03}.
In addition to specialized use cases (e.g., military training and healthcare~\cite{rizzo-jcpms11}), VR technology is seeing widespread adoption in everyday settings, such as gaming, social interactions, shopping, and commerce~\cite{freeman-cscw21,speicher-imwut17,hock-chi17}. Payment features in VR empowers these activities and contributes to the growth of VR economics~\cite{lau_2022}.

To enable payments, VR systems access sensitive user data and assets, raising the need to authenticate users. VR service providers deploy user authentication methods borrowed from traditional platforms to verify users' identities, such as using passwords and personal identification numbers (PINs). However, VR presents a unique context where users interact with digital objects to perform routine activities, such as payment, which were once limited to conventional platforms. Recent research has shown that the context in which authentication is used (e.g., where and for what purpose) affects how users perceive the security of authentication. For example, users feel insecure when using an ATM in a crowded space~\cite{mathis-vr22}. There is a critical need to understand \textbf{how the unique experience in VR contributes to users' security perception of payment authentication.} With such understanding, we can guide the future design of authentication methods that are both secure and usable in the growing VR commerce ecosystem. Our paper provides this understanding by investigating these related research questions.


\begin{itemize}[noitemsep]

\item \textbf{RQ1 -- Interaction Experiences:} What are the factors in users' interaction experiences that contribute to their security perception of authentication in VR?
\item \textbf{RQ2 -- Influences on Security Perception:} How do users' interaction experiences influence their security perceptions of authentication in VR?
\item \textbf{RQ3 -- Understanding User Expectations:} What are the tensions in users' expectations for VR authentication in relation to \textbf{RQ1} and \textbf{RQ2}?


\end{itemize}


To answer these questions, we leverage \textit{technology probes}~\cite{hutchinson-chi03}, where proof-of-concept interfaces uncover hidden phenomena in user interaction, to study user authentication in VR. We designed four probes pertaining to authentication interactions for payment authentication VR -- two variants of entering a PIN, tapping a virtual card, and signing a signature -- to evaluate the user interaction experience and perceived security. 
We embed these probes in a routine payment interaction for users when they play a VR archery game, which is an organic study context.
These probes follow three interaction paradigms of authentication using something you know (e.g., PIN), something you have (e.g., token), and something you are (e.g., biometrics)~\cite{ross-nist19}. The probes and the payment context of our VR game allow us to draw valuable insights into user perception of authentication in VR.


We conducted a user study with 24 participants and evaluated their experiences in the VR game with the probes. We qualitatively analyzed open-ended responses from the participants, which is also supported by quantitative ratings, e.g., the overall usability of our designs. Our analysis reveals these findings in response to the three research questions:
\begin{itemize} [noitemsep]
\item \textbf{RQ1:} The interaction experiences of participants were associated with the perceived usability of authentication and their experience in the gamified context of VR payment. Participants benefited from intuitive virtualization and seamless interactions in authentication. However, they faced unique challenges, such as motion control. Our participants exhibited feelings of high presence and engagement in the VR game and payment. At the same time, they were sensitive to the interruptions caused by payment authentication.  

\item \textbf{RQ2:} Participants found value in realistic interactions in VR authentication as they could transfer their real-world understanding to the virtual environment. However, usability challenges and limited knowledge about VR authentication jointly hindered this translation, such as losing the sense of ownership of their signature in VR. Additionally, the immersive nature of VR heightened participants' uncertainty about threats in both the virtual and physical realms. Moreover, the gamified VR context may decrease participants' sensitivity to security risks associated with payment authentication.  

\item \textbf{RQ3:} While participants prioritized usability, secure authentication remained a crucial consideration. However, they expressed contradictory expectations from a VR authentication method. These contradictions stemmed from a mismatch between perceived and actual security, the inability to detect threats in both VR and the physical worlds, and the difficulties in accurately assessing risks in playful VR experiences.



\end{itemize}

Based on our findings, we propose recommendations to meet participants' expectations for VR authentication. These include (1) exploring virtualized interaction metaphors to bridge the gap between perceived and actual security, (2) implementing system support to detect threats in both VR and the real world, and (3) enhancing security risk communication through multiplexed feedback channels. We also highlight the open challenges and research opportunities associated with our findings and study setup, such as the optimal approach to aid security decision-making for different user groups.


 


  

    










