\section{Conclusion}


We presented a technology probe study to investigate how interaction experiences in VR authentication affect users’ security perception. We designed four probes, using variants of PIN authentication, a virtual card, and a signature, that represent the paradigms of user authentication. We embedded these probes in the routine payments of a VR archery game. In our user study, we collected participants’ responses using surveys regarding interaction experiences, security perceptions, and expectations for authentication. We revealed how participants benefited from the virtualization of authentication in VR and faced unique challenges in interactions, e.g., motion control. Participants encountered difficulties and ambiguity due to VR interactions when transferring their prior authentication knowledge to the VR context. Participants' expectations centered around improving interaction factors with security remained a crucial but secondary factor. However, their expectations were conflicting. We identified tensions in their expectations, which drive our recommendations for future work.   

