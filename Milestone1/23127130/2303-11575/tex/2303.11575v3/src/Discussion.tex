\section{Discussion}
Our findings yield recommendations to guide the design of future VR authentication to calibrate users' security perceptions, enhance VR systems' awareness of threats, and provide flexible feedback on security risks.


\paragraph{Exploring virtualized metaphors for calibrating security perceptions.} 

As we discussed in Section~\ref{sec:RQ2-1}, users have a limited and varying understanding of VR's inherent security properties, and how authentication interaction is virtualized may lead to misalignment between users' perceptions and the actual security properties. Calibrating such perceptions is challenging in the two following aspects, for which we provide corresponding suggestions. In general, we can actively leverage interaction metaphors~\cite{raja-soups11, angus-sdvrs95} in helping VR users align their security perceptions in a usable way. 

First, the security model of VR authentication can be more involved than the traditional platforms, including the use of unconventional and multi-modal modalities, e.g., eye tracking input and biometrics~\cite{mathis2021fast, liebers-chi21}. As such, the designers of VR authentication services could convey the security properties by \textit{ \textbf{integrating metaphors apt to VR authentication modalities}}. For example, when using eye tracking for two-factor authentication (PIN and biometrics), symbolizing eye interaction may help inform users of the use of biometrics when comfortably engaging them~\cite{lee-chi22}.   

On the other hand, the designer could \textit{ \textbf{better connect the context with authentication metaphors}}. We observed the positive effect that \tap reminded some participants of the payment context (Section~\ref{sec:RQ2-3}). We can further strengthen this association with the context by naturally embedding metaphors in the VR interactions with the primary task~\cite{micallef-soups17}. For example, to indicate enhanced security, one VR game can award users persistent credits after users make an effort to complete the multi-factor authentication. Another opportunity is to communicate security by storytelling in VR~\cite{kumar-chi18}. 


\paragraph{System support for enhancing awareness of threats across VR and the real world}
As we found in Section~\ref{sec:RQ2-2}, VR users lack awareness and face difficulties in interpreting threats in both VR and the real world. These difficulties further prevent users from adequately assessing security. Though solutions exist to improve VR users' contextual awareness for safety reasons ~\cite{kudo-hci21}, it is not practical to rely on users only to comprehend all the security threats. Thus, we propose multiple improvements for the VR system to automatically detect threats across virtual and physical contexts and adapt security measures in authentication accordingly.  

First, we can design \textit{\textbf{intelligent systems to recognize and comprehend threats related to VR}}. Our observations in Section~\ref{sec:RQ2-2} and prior studies identified threats in users' virtual world (dark patterns~\cite{ruth-usenix19}, bystanders~\cite{stephenson-sp22}, etc.) and physical contexts (physical imposter~\cite{mathis2021fast}, unauthorized users~\cite{de-csur19}, etc.). The VR system may utilize machine learning to enhance their cross-contextual awareness~\cite{kudo-hci21}. It may comprehend the security implication of a virtual or physical entity based on the VR scene, physical environment, and multi-sensory inputs. For instance, the VR system can tell whether a bystander is actively observing the users during payment.  

Next, the VR system could employ \textit{\textbf{access control to safeguard users' virtual assets actively}}. Based on the contextual understanding, the VR platform could control other entities' access to one user's scene and asset, e.g., payment token, when they perform authentication. We noticed our participants' need for personalizing such access control for multi-user settings, e.g., to avoid \textit{"accident payment from the kids"} in a family with children (P20).  

Moreover, the VR system may \textit{\textbf{automatically adapt authentication methods according to threats in the context}} while balancing usability requirements. Though users prioritize usability over security, they still consider the security of authentication crucial (Section~\ref{sec:RQ3}). The VR system can adaptively enhance security measures to defend against perceived threats. For example, when detecting an active observer, the VR authentication may shuffle the PIN pad from the default. 

\paragraph{Multiplexing communication of security risk in gamified VR.} 
Current VR applications are predominately games, and users expect a playful experience (Section~\ref{sec:RQ3}). Section~\ref{sec:RQ2-3} explained how a gamified VR context might reduce people's sensitivity to risks and security feedback. Our findings in Section~\ref{sec:RQ3} echo prior work that users still demand transparency and control for payment authentication, e.g., properly understanding suspicious behavior. This complements the above recommendation to reduce users' effort in making security decisions. For security-sensitive applications, especially payment, risk communication is often necessary for users' assessment of security. Here we discuss the potential to improve the efficacy of risk communication by multiplexing the feedback in VR.   

VR applications deliver feedback over multiple modalities (e.g., visual, audio, and haptic) to their users. The payment authentication service may choose to \textbf{\textit{utilize orthogonal feedback modalities}} to the primary application context. As such, we avoid overwhelming users and causing security fatigue~\cite{harbach-chi14}. In addition, the application can interleave the timing of authentication feedback with the primary activities in VR. The choice of feedback modality also stands on an \textbf{\textit{understanding about VR users' perceptual capacity and sensitivity}}, which may also vary across different demographics, e.g., people with accessibility challenges~\cite{stephenson-sp22}.   

\paragraph{Future research direction.} Our work opens up several directions for future research, which stem from our findings, methodology, and limitations. 
First, our recommendations primarily focus on improving the perceived and actual security of VR authentication through effective security communication and infrastructural support. Furthermore, the VR ecosystem involves multiple stakeholders, such as platform providers, application developers, authentication service providers, and banks, each with distinct roles and responsibilities in payment authentication. Coordinating with these stakeholders to systematically enhance security poses an ongoing challenge.

Building upon the recommendations, future research could explore effective ways to assist users in making informed security decisions and implementing security measures, for example, making trade-offs between security and usability of authentication. More specifically, it would be valuable to investigate the optimal level of control to give users versus automating decision-making on their behalf~\cite{filipczuk-AAMAS22}.

In our user study's population, the participants primarily consisted of young and tech-savvy individuals who are often early adopters of VR technology. To generalize our findings, it would be valuable for researchers to explore how users from diverse backgrounds, including different age groups and levels of technology expertise, perceive VR authentication~\cite{ratakonda-idc19}.

In addition, future work may explore more possibilities of VR interaction for authentication with users involved in the co-design process~\cite{yao-chi19}. Through participatory design studies, experts and users can collaborate to design both the frontend interaction and backend infrastructure of VR authentication systems.

Next, we recommend conducting longitudinal research, such as diary studies~\cite{hayashi-chi11}, to examine users' long-term adoption and usage of VR authentication. Users' security attitudes and behaviors may evolve over time as they interact with the system~\cite{li-sp23}. These longitudinal studies will provide valuable insights into usability and security issues in real-world scenarios, including how users respond to suspicious activities and threats~\cite{downs-apwg07}.





    
    

