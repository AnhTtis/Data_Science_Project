
\section{Related Work}
\paragraph{User authentication in virtual/augmented reality (VR/AR).} Prevailing user authentication methods on smart devices, such as smartphones, use one or more of the following factors: (1) unique knowledge (e.g., PIN or unlock patterns~\cite{von2013patterns}), (2) tokens (e.g., a device with coded ID data~\cite{nguyen-sensys16}), and (3) behavioral and physical biometrics (e.g., gestures and iris~\cite{liu2017usability, kumar-pr10}). In the context of VR/AR, authentication schemes build upon these methods but offer unique security properties compared to real-world authentication. For knowledge-based authentication in VR/AR, virtual PINs displayed in the 3D space can make shoulder-surfing more difficult~\cite{mathis2021fast}. Meanwhile, multiple input modalities can be used to select PINs or draw unlock patterns, such as eye gaze, head pose, controller tapping, and foot movements~\cite{watson-chi22, olade-icvars20}. Biometric authentication, particularly behavioral biometrics, is a prominent area of research in VR/AR. It leverages the multi-modal input modalities to capture users’ biometric traits, such as motion trajectory~\cite{kupin-icmm19}, electromyography~\cite{chen2021user}, eye tracking~\cite{zhu2020blinkey}. Behavioral biometrics are often associated with particular tasks users perform in VR/AR, e.g., throwing a ball~\cite{liebers-chi21}. There are also preliminary efforts in exploring token-based authentication mainly for AR, e.g., QR codes. User authentication in VR/AR often requires active interaction with the VR/AR interfaces. This raises usability issues and presents a tradeoff between security and privacy~\cite{stephenson-sp22}. 




\paragraph{Security and privacy perception.} Prior research investigated how users perceive the security and privacy of user authentication mechanisms, with a focus on established methods like FIDO2 authentication. Lyastani et al.~\cite{lyastani2020fido2} discovered that users express concerns about security issues related to the loss of authentication tokens. Lassak et al.’s study~\cite{lassak2021s} identified misconceptions among users regarding the storage of biometric data in FIDO2 biometric authentication. These studies highlight the disconnect between users' security perception and the actual security provided by authentication methods.  

Recent research has also examined users’ security and privacy perception in VR. VR developers and users felt the lack of privacy due to opaque data collection policies~\cite{adams-soups18}. Many users center their concern around the threats from other users, e.g., as a bystander~\cite{de-csur19}. Additionally, users are worried about potential deception by digital content in VR~\cite{lebeck-sp18}. Users’ security and privacy perception of VR authentication received more attention recently. Stephenson et al. discovered that users often thought VR/AR authentication was as secure as other platforms in their online survey~\cite{stephenson-sp22}.  

Users’ perceptions of security and privacy are associated with multiple factors, including their interaction with the system. For instance, Distler et al.~\cite{distler2019security} studied how the user interface (UI) designs impact users’ perceived security of mobile e-voting apps. They discovered that inadequate UI feedback and contextual information reduce users’ sense of security. Users’ security and privacy perceptions also depend on other factors, such as personal experience. Jeong and Chiasson~\cite{jeong-chi20} found that children and adults have different interpretations and perceptions of security warnings, e.g., the symbolism of a police officer icon. Differing preconceptions are challenging for establishing trust with the system, even with visual security clues~\cite{stransky-soups21}. 


\paragraph{User authentication for payment.} 
User authentication plays a crucial role in payment services by preventing fraud and minimizing financial risks. The authentication requirements vary depending on the context, such as using chip cards for in-store transactions or requiring one-time passwords (OTPs) for online shopping~\cite{acharya2013two}. Users' perception of authentication security significantly influences their adoption and use of payment services. Mobile payments have gained popularity due to users associating perceived control and security with user authentication on their device~\cite{zhang-mdpi19}. Similarly, some users also desire enhanced security using biometrics in cryptocurrency wallets~\cite{voskobojnikov-chi21}. Trust is impacted by users' understanding of how payment services ensure authentication security, such as password confidentiality~\cite{zhou-id11}. Different authentication processes in various geographies can lead to differing security perceptions~\cite{george-ndss17, busse-eurospw20}. The payment environment also affects security perception, with users considering ATM authentication riskier than payments in a restaurant due to their unawareness of attacks~\cite{volkamer-soups18}. In addition to security, factors such as usability in using a user authentication method also impact the use of associated payment service~\cite{kujala2017role}. 

\paragraph{Contributions to the literature.} To the best of our knowledge, our work is the first to systematically study the interplay between users' interaction experiences and security perception of VR authentication. Prior studies indicated such interplay in other contexts. For example, Khan et al. studied how interruptions of implicit authentication affect users' sense of security~\cite{khan-soups15}. As an emerging interaction technology, VR brings novel interaction experiences and security issues to its users~\cite{adams-soups18, de-csur19}. Thus, it is natural for us to hypothesize that interactions with user authentication in VR play a distinct role in shaping users' security perceptions. However, prior work on user authentication in VR has primarily focused on different research questions, such as enhancing its security properties~\cite{mathis2021fast, kupin-icmm19,chen2021user,zhu2020blinkey,liebers-chi21} or studying the usability and security perception~\cite{george-ndss17, stephenson-sp22, abdelrahman-avi22} of authentication respectively. Prior work in VR showed preliminary findings, e.g., users' security behaviors changed regarding their virtual surroundings~\cite{mathis-vr22}. These findings motivate our in-depth investigation of this interplay. Unlike prior studies based solely on surveys or interviews~\cite{stephenson-sp22}, our study method helped us attain in-context insights by employing ``technology probes''~\cite{hutchinson-chi03} for authentication in a realistic VR payment use case. We will also discuss how our findings supplement prior studies.








