


\section{RQ1: Interaction Experiences}


We observed two overarching themes of participants' experiences that influence their security perception of authentication. The first is the \textbf{perceived usability of authentication}, which consists of two sub-themes: \textit{virtualization} and \textit{seamlessness} of the authentication interaction. The second theme is \textbf{experience in VR game context}, which is influenced by the two sub-themes: participant \textit{presence} and \textit{engagement}. In the following, we elaborate on these themes and discuss how our probes uncovered these experiences.

\subsection{Perceived Usability of Authentication}
The virtualization and seamlessness of the authentication interaction affect the perceived usability. Our probes entail additional aspects regarding \textit{authentication interface}, \textit{process of authentication}, and \textit{motion control} that are not typically present in real-world authentication interaction.  These aspects contribute to the virtualization and seamlessness of the VR authentication, leading to different perceptions of usability. 

\subsubsection{Virtualization of Authentication Interaction.}
Participants expressed positive feedback regarding the usability of \textit{intuitive} VR interactions and interfaces that resemble their real-life experiences. Virtualizing \textit{familiar} interfaces from the real world made them feel more \textit{immersed}. However, complex VR interactions introduced an additional learning curve, particularly with \sign, which involved more friction compared to PIN methods and \tap. For instance, one participant felt \textit{``signing in the virtual world was very different as compared to the real world''} (P13). Nevertheless, another participant found \sign was easy after practicing. %
\begin{displayquote}
\textit{``It's very easy to learn and use, and the functionality can be easily picked up.''} (P11)
\end{displayquote}

An intuitive and realistic \textit{authentication interface}, in terms of presentation and feedback, can enhance the virtualization of interaction, even though the authentication concept is new to participants. The token-based authentication \tap was favorable to participants, primarily due to its interface presentation:

\begin{displayquote}
\textit{``since it closely mimics the way I use card payments in my day-to-day life.''
} (P8)
\end{displayquote}

Participants exhibited different preferences over the interaction modality, often by comparing it with real-world counterparts. One participant mentioned, when describing the virtual PIN pad, that the \textit{``laser pointer is less similar to using keyboards in real world''} (P2). 
Moreover, participants expected more realistic interaction feedback like what they receive in the real world, especially for the highly interactive \sign approach. For example, they demanded a better sense of writing on a ``paper'' in VR.
\begin{displayquote}
\textit{``It is still usable and the pen writes like real writing but still did not get the feeling of writing on a paper.''} (P1)
\end{displayquote}
In addition, participants perceived virtualization differently due to their prior experience in the real world, i.e., how they performed payment. For example, one participant was not used to a shuffled PIN pad as \textit{``most terminals have a fixed layout (I am [they are] thinking of ATMs and gas stations)''} (P5).

Generally, the virtualization of real-world authentication interactions helped improve perceived usability. However, some participants expected VR interfaces to save them more effort compared to authentication interactions in the physical world, e.g., moving \textit{``to get a better look at the numpad''} on the kiosk (P13). 



\subsubsection{Seamlessness of Authentication Interaction.}
The seamlessness of authentication interaction also contributed to better-perceived usability. Interestingly, participants associated seamlessness with specific aspects such as the \textit{comfort} of interacting in VR, the \textit{physical, mental, and time efforts}, and the \textit{smoothness} of the interaction process.

As mentioned earlier, our VR authentication probes exhibit familiar properties of real-world authentication. They also inherit some of the difficulties from previous authentication experiences, including the transitions in the workflow, the memorability of PIN, and the additional usability challenge to use a shuffled PIN pad where \textit{``the numbers shifted locations between attempts''} (P13). 

The \textit{motion control} in VR introduced unique challenges that made authentication interaction less seamless. This aspect had a more noticeable impact on probes that required complex interactions (\sign, \pinf, \pink) compared to \tap.  The challenges with motion control included inconsistency in action control when interacting with digital components. This inconsistency affected the accuracy of translating participants' actions, such as how their virtual gestures aligned with their intentions. For example, one participant preferred \pink over \pinf due to the observation that \textit{``stable kiosk was easier to use as it allowed for better calibration''} in entering PINs (P7). 

Moreover, spatial awareness in VR, such as coordinating the movements of a virtual object and the avatar, contributed to the challenge of motion control. Some participants indicated that lacking spatial awareness hindered their ability to control a virtual object, e.g., stretching their arm to sign a signature:
\begin{displayquote}
\textit{``It takes a while to get used to the proper distance between the pen and the kiosk screen.''} (P7)
\end{displayquote}


These aspects became evident when participants commented on the seamlessness of the \sign probe. As mentioned before, some participants thought \sign was easy, while others did not. One participant felt that performing the VR gesture was easy conceptually but hard in practice.
\begin{displayquote}
\textit{``Signature: it was conceptually easy, but the execution was somewhat cumbersome and it required a complex gesture.''} (P6)
\end{displayquote}




\subsubsection{Quantitative Usability Analysis.} The above interaction experiences manifest in different usability ratings for the four probes. By comparing participants’ SUS rating (Figure~\ref{fig:sus}), a standard usability metric, we find that \tap received the highest SUS score (mean: 82.1, std: 14.6) and cost the least time (mean: 3.3s, std: 2.0s), indicating ``excellent'' usability~\cite{brooke-jus13}. It benefits from seamless and intuitive interaction.
\pinf (mean: 70.6, std: 19.9) and \pink (mean: 76.1, std: 11.7) closely follow \tap, demonstrating a ``good'' usability. They consumed comparable time to complete: \pinf (mean: 10.8s, std: 11.1s) and \pink(mean: 10.4s, std: 6.6s). As discussed, participants appreciated the usability of these methods due to their familiarity and relative ease, despite the effort required to memorize and enter the PINs.  Moreover, the differences in scores between \pinf and \pink are small. Participants commented on different usability issues for them, such as the inconvenience of walking towards the kiosk for \pink and the distraction caused by entering PINs on a moving panel for \pinf.
\sign (mean: 60.7, std: 14.7), which is also the most time-consuming probe (mean: 18.3s, std: 9.8s), received the lowest score among the four probes but still achieved ``OK'' usability. Although its interface appeared intuitive, signing with the virtual pen proved to be challenging for multiple participants in the VR environment.

\begin{figure*}[h]
    \begin{minipage}[t]{0.4\linewidth}
        \centering
        \includegraphics[width=\linewidth]{fig/sus.pdf}
        \caption{SUS scores (average and standard deviation) for each authentication probe.}
        \label{fig:sus}
    \end{minipage}%
    \hfill
    \begin{minipage}[t]{0.56\linewidth}
        \centering
        \includegraphics[width=\linewidth]{fig/ipq.pdf}
\caption{IPQ sense of presence scores for game contexts with the four authentication probes. A higher score indicates a higher level of presence. From 0 to 6, a score higher than 3 stands for neutral. All subscales have a positive mean score, except REAL. The IPQ scores of our setups are consistent with prior implementations for VR authentication~\cite{mathis-vr22}.}\label{fig:ipq}
    \end{minipage}
\end{figure*}






\subsection{Experience in the VR Game Context}

In addition to the interaction experience with the authentication probe, participants' experience with the VR game (as a proxy for the VR context) appeared to influence their perception of security. We observed two sub-themes related to how participants felt a sense of \textit{presence} and \textit{engagement} in the game.


\subsubsection{Feeling Present in the VR World.} 
Using the IPQ presence questionnaire (Figure~\ref{fig:ipq}), we observed that participants rated their presence in the VR world positively. However, we did not observe a significant difference in the IPQ scores between sessions with different authentication probes, which is consistent with previous findings about authentication is often perceived as a secondary task~\cite{de-soups10}. When participants felt present in this world, some of them expressed dissatisfaction when authentication disrupted their immersive experience through an ``unreal'' interface, e.g., \pinf:
\begin{displayquote}
\textit{``the floated pad makes it so unreal that I know it is in VR rather than real life. I do not like the experience.''} (P24)
\end{displayquote}

\subsubsection{High Engagement in the Game.} 
We observed the participants to be highly engaged in the archery game and the routine payment. When deciding which arrows to pay for, participants mentioned various factors, including  \textit{their strategy to compete} and \textit{enjoyment of the game}. Among the participants, 19 explicitly mentioned that they would like to shoot as many arrows as they could, and six people said it was fun to play, despite differences in their archery performances (highest score: 1470, lowest score: 190). This high level of engagement also led some participants to expect less effort during authentication compared to the time spent actively playing the game, as expressed by one participant: \textit{``the time I am [they are] actually playing the game''} (P18).\\




\noindent
{\colorbox{beaublue}{\textbf{Takeaway-RQ1.}}}
Participants' interaction experiences included their perceived usability of authentication and their experience in the VR game context. Our findings were consistent with prior observations that PIN authentication has acceptable usability ~\cite{george-ndss17} and that user authentication is perceived as a secondary task compared to the overall context~\cite{mathis-vr22}. We further observed that participants associated the usability of authentication with better virtualization and more seamless interactions. When fully engaged in the VR game, participants preferred less disruptive experiences to keep their immersion. Moreover, we found that unique usability challenges of VR interactions, such as motion control, influenced participants' authentication experience, particularly with behavior-based authentication. 

