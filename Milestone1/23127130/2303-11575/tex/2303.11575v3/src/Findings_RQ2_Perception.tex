
\section{RQ2: Influences on Security Perception}
\label{sec:RQ2}
Previously, we listed the interaction themes that influenced user perception of the security of authentication. In the following, we discuss \textit{how} these interaction themes influenced participants’ security risk assessment while authentication. Our qualitative analysis of participant responses revealed three themes of influences in Figure~\ref{fig:framework}. These themes include forming security perceptions, being aware of threats, and assigning risk to threats.

\subsection{Interactions Affect the Transfer of Prior Understanding About Authentication into VR}
\label{sec:RQ2-1}
Facing a limited understanding of novel VR authentication concepts, participants transferred their prior experience with real-world authentication to VR. This transfer of knowledge, however, did not result in a consistent security perception. The virtualization and seamlessness of authentication interactions, while enhancing usability, affected how participants formed their security perceptions of VR.


\subsubsection{Impact of Prior Knowledge and Experience.}\label{sec:RQ2-1-1}
Participants \textit{associated the virtual probes to authentication processes from the physical world with which they are familiar}. Examples of such processes include possessing secret knowledge (PIN) that is \textit{``only known by me [the participant]''} and using a shuffled PIN pad that \textit{``gave me [the participant] some sense of security''}  (P20, P17). 

However, participants' \textit{prior understanding and real-life experiences varied}, leading to differences in their perceptions. For example, the participants held different views on how consent works in VR payments, as shown in Figure~\ref{fig:overall_perception}. For example, one user viewed \sign as less familiar as they \textit{``rarely signed to pay (only recently in the US…)''} (P6). Meanwhile, others \textit{``naturally perceive it as giving my [their] consent''} as they would in the physical world (P12). Some participants associated consent with privacy implications. For instance, P20 raised a possible privacy concern with \sign, which \textit{``requires more information [signature]''} (P20) than PINs. 

Meanwhile, participants \textit{expressed varying levels of confidence about the security of authentication}. Some participants were not confident due to lacking information, for example, \textit{``without more information on how the payment actually works''} or without knowing \textit{``how the backend works''} (P13, P3). P3 also suspected that \tap would need additional verification or certification steps for the virtual card. On the contrary, some participants more confidently assumed that \textit{``the technology behind tapping makes me [them] believe it is safe''} (P14). This finding contributes to the polarized responses, e.g., security and privacy of \tap (Figure~\ref{fig:overall_perception}(a, b)). 

 
\subsubsection{Impact of Usability.}
\label{sec:RQ2-1-2}
We observed that, because of a limited understanding of VR authentication, the usability properties of the probe affected the participants' security perception. This was the case for the \sign and \tap probes. In particular, as VR interactions do not mimic real-world sensations very well, participants felt a loss of control over authentication. For example, some participants were not confident they were providing consent when using \sign and \tap, compared to \pinf and \pink (Figure~\ref{fig:overall_perception}(a)). When signing in VR, some participants did not feel that the VR signature belonged to them. One participant expressed: 

\begin{displayquote} 

\textit{``The sign-to-pay method was a bit hard to use, so I think I just tried to write something, and I felt less like providing my signature.''} (P10) 

\end{displayquote} 

Similarly, the same participant felt that \tap was not secure as they still thought \textit{``it's not my [their] real card but a card-like object''}. In addition, the lack of feedback associated with the seamlessness of the VR interaction contributed to the loss of control, particularly for \tap. For instance, P2 expressed concern that \tap appeared too \textit{`` no-brainer ''} without any warnings (P2). 

 
The \textit{usability characteristics of the interaction appear to undermine participants' confidence in security}. Some participants expressed concerns about the security of \sign, the authentication method with the lowest usability. They believed that this method compromised security by accepting inconsistent signatures, which could potentially make them more susceptible to impersonation.

\begin{displayquote} 

\textit{``The sign-to-pay felt the most insecure as people easily have access to my cheques and can probably fake in the VR world since the VR signatures were clearly less accurate than the real-world.''} (P5) 

\end{displayquote} 

The participants expressed greater confidence in the security of traditional PIN methods, which also offered more acceptable usability. In such cases, a higher level of interaction provided participants with a sense of control (Figure~\ref{fig:overall_perception}(e)). In a similar vein, a subgroup of participants expressed a willingness to accept additional interaction to prioritize security and transparency. This included incorporating extra warnings and utilizing a shuffled PIN pad to avoid \textit{``making accident payment''} (P15). 






\subsection{Presence in VR Alters the Awareness of Contextual Threats}
\label{sec:RQ2-2}
Participants engage in threat modeling as they analyze the security properties of the VR authentication. During this process, they identified several entities based on their understanding of real-world payment and the VR environment. These entities included payment and authentication service providers, physical and virtual bystanders, third-party apps, and malware. Participants also \textit{recognized software vulnerabilities} that could lead to compromising personal information, such as PINs and signatures:
\begin{displayquote}
\textit{``Similar reason that pins and signatures are just exposed to the game or malware in the gaming system.''} (P24)
\end{displayquote}
Another example is participants' \textit{awareness of virtual and physical bystanders}. For example, P6 considered \pink less secure than \pinf if bystanders in VR were able to see the kiosk.

Participants expressed concerns that their presence in the VR environment reduced their awareness of both the virtual and physical worlds. This lack of awareness potentially exposed their information (PIN, signature) and assets (cards and tokens) to adversaries. They worried that \textit{attacks in VR could be more imperceptible than in the physical world}, such as being deceived by an invisible terminal for phishing purposes:
\begin{displayquote}
\textit{``Tap-to-pay is the easiest, but I feel it not safe because I can easily touch it to an invisible system in the VR world.''} (P19)
\end{displayquote}
\textit{Noticing attacks depended on the type of authentication interaction in VR}.  For example, it might be easy for a person to notice an attacker \textit{``stealing a card,''} but harder to observe a physical or virtual shoulder surfing attack when signing or entering a PIN (P6). Participants were also concerned that malicious users could leverage prior knowledge about the victim in the physical world to launch attacks in VR:
\begin{displayquote}
\textit{``signing can be also copied by anyone who knows my signature in the real world.''} (P12)
\end{displayquote}

Last, some participants had \textit{a general lack of trust due to their ambiguity surrounding the VR technology and authentication}. P12, for instance, expressed skepticism about the security of \tap, which stems from their distrust of the VR technology:
\begin{displayquote}
\textit{``Tap-to-pay was the simplest, but it was way too simple to believe that the entire payment process behind the scene was dealt with as I wanted. I'm not sure if this is due to my distrust to the specific payment system, or just to the VR world, or both.''} (P12)
\end{displayquote}






\subsection{Gamified Context Reduces the Sensitivity to Risks}
\label{sec:RQ2-3}
Our context for user authentication revolves around participants' routine payment in a VR game. During these activities, participants were highly engaged in the game and the payment process. Some participants noted that their \textit{sensitivity to authentication security related to the relevance of virtualization to payment}. For instance, P9 felt a greater sense of control when using \tap, as it simulated an actual payment experience by physically grabbing the card and initiating the payment. On the other hand, \pink, \pinf, and \sign were perceived as less specific to payment.


However, the \textit{gamified context may reduce participants' sensitivity to security and privacy risks}. For instance, P7 argued that they did not feel a loss of privacy in a VR game compared to real-life transactions.  
\begin{displayquote}
\textit{``In the context of the game I didn't feel like giving away privacy. However if I were to imagine this with real transactions, then I'd feel like giving some of my privacy away, similar to every time I pay with something else than cash.''} (P7)
\end{displayquote}

Furthermore, the gamified interactions and interfaces of certain authentication probes made some participants feel less attentive to security. When signing a signature in VR, the gaming aspect overshadowed P7’s sense of giving consent.
\begin{displayquote}
\textit{``The sign to pay wasn't completely obvious you were actually paying for something, it could have been part of the game to have to write your name.''} (P7)
\end{displayquote}
Similarly, P10 thought \pinf \textit{``felt a bit too much like being in a game as well''} compared to \pink (P10).

We also observed a dichotomy in whether participants prefer to be alerted to the fact that a payment is taking place (Figure~\ref{fig:overall_perception}(d)). Some participants felt like they need to be alerted because the lack of feedback and their limited understanding make them feel less secure (as we explained in Section~\ref{sec:RQ2-1}). Other participants felt like the payment authentication was secure, and they did not see a need to be alerted if not prompted. For example, the more personal interface of \pinf captured a participant's attention because it \textit{``clearly wouldn't let me [them] proceed through the game''} (P16).\\



\begin{figure*}[h]
\centering
\includegraphics[width=\textwidth]{fig/likert.pdf}
\caption{The overall security perception of the four probes. We color-coded the bars that represent the percentages of participants (red: strongly disagree, orange: disagree, grey: neither agree nor disagree, light blue: agree, dark blue: strongly agree). }\label{fig:overall_perception}
\end{figure*}


\noindent
\colorbox{beaublue}{\textbf{Takeaway-RQ2.}} 
Our findings confirmed our hypothesis that authentication interactions have an impact on the security perception of VR authentication, similar to previous studies on applications like implicit authentication~\cite{khan-soups15, wiefling-acsac20}. Our findings in VR authentication further complement previous work as follows.
Stephenson et al.'s survey~\cite{stephenson-sp22} mainly discovered that the usability of knowledge-based authentication in VR is perceived as a tradeoff between security and usability. Participants in our study generalized this tradeoff to other forms of VR authentication, including behavioral biometrics. Moreover, we observed that the virtualization of authentication interactions in VR may bias participants when transferring prior understanding of authentication to VR, potentially leading to a false yet confident sense of security. Our participants also expressed uncertainties regarding threats in both the virtual and physical realms due to the immersive nature of VR. Additionally, the gamified context of VR had the potential to reduce participants' sensitivity to security risks. 