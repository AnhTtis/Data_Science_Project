%%%%%%%%%%%%%%%%%%%%%%%%%%%%%%%%%%%%%%%%%%%%%%%%%%%%%%%%%%%%%%%%%%%%%%%%%%%%%%%%
% Template for USENIX papers.
%
% History:
%
% - TEMPLATE for Usenix papers, specifically to meet requirements of
%   USENIX '05. originally a template for producing IEEE-format
%   articles using LaTeX. written by Matthew Ward, CS Department,
%   Worcester Polytechnic Institute. adapted by David Beazley for his
%   excellent SWIG paper in Proceedings, Tcl 96. turned into a
%   smartass generic template by De Clarke, with thanks to both the
%   above pioneers. Use at your own risk. Complaints to /dev/null.
%   Make it two column with no page numbering, default is 10 point.
%
% - Munged by Fred Douglis <douglis@research.att.com> 10/97 to
%   separate the .sty file from the LaTeX source template, so that
%   people can more easily include the .sty file into an existing
%   document. Also changed to more closely follow the style guidelines
%   as represented by the Word sample file.
%
% - Note that since 2010, USENIX does not require endnotes. If you
%   want foot of page notes, don't include the endnotes package in the
%   usepackage command, below.
% - This version uses the latex2e styles, not the very ancient 2.09
%   stuff.
%
% - Updated July 2018: Text block size changed from 6.5" to 7"
%
% - Updated Dec 2018 for ATC'19:
%
%   * Revised text to pass HotCRP's auto-formatting check, with
%     hotcrp.settings.submission_form.body_font_size=10pt, and
%     hotcrp.settings.submission_form.line_height=12pt
%
%   * Switched from \endnote-s to \footnote-s to match Usenix's policy.
%
%   * \section* => \begin{abstract} ... \end{abstract}
%
%   * Make template self-contained in terms of bibtex entires, to allow
%     this file to be compiled. (And changing refs style to 'plain'.)
%
%   * Make template self-contained in terms of figures, to
%     allow this file to be compiled. 
%
%   * Added packages for hyperref, embedding fonts, and improving
%     appearance.
%   
%   * Removed outdated text.
%
%%%%%%%%%%%%%%%%%%%%%%%%%%%%%%%%%%%%%%%%%%%%%%%%%%%%%%%%%%%%%%%%%%%%%%%%%%%%%%%%

\documentclass[letterpaper,twocolumn,10pt]{article}
\usepackage{usenix-2020-09}

% to be able to draw some self-contained figs
\usepackage{tikz}
\usepackage{amsmath}
\usepackage{csquotes}
\usepackage{multirow}
\usepackage{adjustbox}
\usepackage{xspace}
\usepackage{cleveref}
\usepackage[utf8]{inputenc}
\usepackage{enumitem}
\usepackage{hyperref}
\usepackage{csquotes}
\newenvironment*{innerquote}
  {\setlength{\leftmargini}{0.15cm}%
  \quote}
  {\endquote}

\SetBlockEnvironment{innerquote}
\usepackage{lipsum}


\newif\ifdraft
\drafttrue
%\draftfalse

%%%% COMMENTS AND NEW TEXT %%%
% Howto:
%   for a ToDo, just write \TODOS{my todo} 
%   for new text write \NEWS{my new text}
%   to suggest replacing text with other text write \REPLACES{old text I didn't like}{new text I like more}
%
% TODOS won't show up outside of draft mode, NEW text will show up (without color) and for REPLACED text, only the new text will show outside of draft mode

\ifdraft 
    \newcommand{\TODOS}[1]{{\color{purple} \textbf{TODO:} #1}}
    \newcommand{\NEWS}[1]{{\color{purple}#1}}
    \newcommand{\REPLACES}[2]{{\color{gray}\footnotesize[#1]}{\color{purple}#2}}
\else
    \newcommand{\TODOS}[1]{}
    \newcommand{\NEWS}[1]{#1}
    \newcommand{\REPLACES}[2]{#2}
\fi


\newcommand{\ttp}{Tap-to-Pay\xspace}
\newcommand{\stp}{Sign-to-Pay\xspace}
\newcommand{\ptpf}{PIN-to-Pay (Floating)\xspace}
\newcommand{\ptpk}{PIN-to-Pay (Kiosk)\xspace}

\newcommand{\tap}{\texttt{TAP}\xspace}
\newcommand{\sign}{\texttt{SIGN}\xspace}
\newcommand{\pink}{\texttt{PIN-K}\xspace}
\newcommand{\pinf}{\texttt{PIN-F}\xspace}

\usepackage{titlesec}
\usepackage{authblk}

\titlespacing{\paragraph}{%
  0pt}{%              left margin
  0.3\baselineskip}{% space before (vertical)
  1em}%               space after (horizontal)

\makeatletter
\def\@maketitle{%
\newpage%
\null%
%\vskip 2em % <================================================ space c)
\begin{center}%
    \let\footnote\thanks %
    {\LARGE \@title %
      \par % <================================================= space d)
    }
%   \vskip 1.5em % <=========================================== space d)
    {\large
     \lineskip 0
     \begin{tabular}[t]{c}
        \baselineskip=12pt
        \@author
     \end{tabular}
     \par% <=================================================== space e')
    }
%   \vskip 1em % <============================================= space e')
    {\large \@date}
\end{center}
\par % <======================================================= space f)
\vskip 1.5em} % <============================================== space f)
\makeatother


%-------------------------------------------------------------------------------



\begin{document}
%-------------------------------------------------------------------------------



%don't want date printed
\date{}

% make title bold and 14 pt font (Latex default is non-bold, 16 pt)
\title{\Large \bf ``I Want the Payment Process to be Cool'': Understanding How Interaction Factors into Security and Privacy Perception of Authentication in Virtual Reality}


%for single author (just remove % characters)
% \author{
% {\rm Your N.\ Here}\\
% Your Institution
% \and
% {\rm Second Name}\\
% Second Institution
% copy the following lines to add more authors
% \and
% {\rm Name}\\
%Name Institution
% } % end author
% \setlength{\affilsep}{0em}
\author[1]{\rm Jingjie Li*}
\newcommand\CoAuthorMark{\footnotemark[\arabic{footnote}]} % get the current value
\author[2]{\rm Sunpreet Singh Arora}
\author[1]{\rm Kassem Fawaz}
\author[1]{\rm Younghyun Kim} 
\affil[1]{University of Wisconsin--Madison, \textit{\{jingjie.li, kfawaz, younghyun.kim\}@wisc.edu}}
\affil[2]{Visa Research, \textit{\{sunarora, caliu, smeiser, mominaei, mshirvan, kwagner\}@visa.com}}
\author[2]{\\\rm Can Liu}
\author[3]{\rm Sebastian Meiser*}
\affil[3]{University of Lübeck, \textit{sebastian@meiser-web.de}}
\author[2]{\rm Mohsen Minaei} 
\author[4]{\rm Maliheh Shirvanian*} 
\affil[4]{Netflix Inc., \textit{maliheh21@gmail.com}}
\author[2]{\rm Kim Wagner} 

% \setlength{\affilsep}{0pt}




\maketitle
\begingroup\def\thefootnote{*}\footnotetext{The work was done while the authors were at Visa Research.}\endgroup


%-------------------------------------------------------------------------------
\begin{abstract}
%
Deformable image registration is a fundamental task in medical image analysis and plays a crucial role in a wide range of clinical applications. 
Recently, deep learning-based approaches have been widely studied for deformable medical image registration and achieved promising results. However, existing deep learning image registration techniques do not  theoretically guarantee %diffeomorphic 
topology-preserving transformations. This is a key property to preserve anatomical structures and achieve plausible transformations that can be used in real clinical settings.
%
We propose a novel framework for deformable image registration. Firstly, we introduce a  novel regulariser based on conformal-invariant properties in a nonlinear elasticity setting.
Our regulariser enforces  the deformation field  to be  smooth, invertible and orientation-preserving. %differentiable.  
More importantly, we strictly guarantee topology preservation yielding to a clinical meaningful registration.  Secondly, 
we boost the performance of our regulariser through coordinate MLPs, where one can view the to-be-registered images as continuously differentiable entities. 
We demonstrate, through numerical and visual experiments, that our framework is able to outperform current techniques for image registration.
%
\keywords{Homeomorphic image registration \and Lung CT \and Conformal invariant hyperelastic regularisation.}
%
\end{abstract}


% Importance and appeal of children's drawings
Children's depictions of the human figure are highly expressive and varied.
As one of the very first subjects children attempt to draw, the representation begins as an almost unintelligible cloud of scribbles. 
As the child grows, their representation of the human figure becomes more developed and is extended to graphically represent many different types of characters: people, animals, and even personified objects (see Figure 1).

Who among us has not wished, either as a child or as an adult, to see such figures come to life and move around on the page?
Sadly, while it is relatively fast to produce a single drawing, creating the sequence of images necessary for animation is a much more tedious endeavor, requiring discipline, skill, patience, and sometimes complicated software.
As a result, most of these figures remain static upon the page.

% We built a system to animate them.
Inspired by the importance and appeal of the drawn human figure, we design and build a system to automatically animate it given an in-the-wild photograph of a child's drawing. 
Our system is fast, intuitive, and robust to much of the variation present in these types of drawings, making it well-suited to allow our target audience--children--to see their own characters coming to life.
The system is comprised of four stages: figure detection, segmentation masking, pose estimation/rigging, and animation. 
We describe each stage and identify common causes of failure in each. 
For object detection and pose estimation, we make use of existing computer vision models designed to detect human figures and joints in photographs; we fine-tune these models for use with children's drawings.
For segmentation, we present a straightforward, image processing-based method that, for animation purposes, is more useful and accurate than segmentation masks obtained from a fine-tuned object detection model.
During the animation step, we take advantage of the \textit{twisted perspective} commonly seen in children’s drawings to retarget motion capture data onto the character in a novel and appealing way.

% We use existing machine learning models. However, given the wide domain gap it's not clear how much fine-tuning data was needed. So we ran some experiments to find out and report it.
While our system leverages existing models and techniques, most are not directly applicable to the task due to the many differences between photographic images and simple pen and paper representations. 
To this end, we couple the presentation of our system with a set of experiments exploring the relationship between fine-tuning training set size and success rates.
We also include a perceptual study validating viewer preference for incorporating \textit{twisted perspective} into the motion retargeting step.

We validate the desirability and appeal of our system by building and publicly releasing a version of it as the \AD Demo \,\cite{animateddrawings}.
Launched in December 2021, this demo has been used by millions of people around the world to animate their children's drawings.
Inspired by this reception, our second contribution is The Amateur Drawings Dataset: \hjs{180,000 drawings and user-accepted annotations collected, with consent, through the demo. See Section \ref{sec:UI} for a description of how the annotations were generated.}
We believe this dataset will be a resource to researchers from various fields seeking to better understand the space of amateur drawings, evaluate new algorithms in this domain, or develop new drawing-based tools in general.

To summarize, our contributions are as follows:
\begin{enumerate}
    \item 
    We explore the problem of automatic sketch-to-animation for children's drawings of human figures and present a framework that achieves this effect. We also present a set of experiments determining the amount of training data necessary to achieve high levels of success and a perceptual study validating the usefulness of our motion retargeting technique.
    \item To encourage additional research in the domain of amateur drawings, we present a first-of-its-kind dataset of 180,000 user-submitted amateur drawings, along with user-accepted bounding box, segmentation mask, and joint location annotations.
\end{enumerate}

Upon acceptance of this paper, we plan to publicly release the Amateur Drawings Dataset, project code, and fine-tuned model weights.

\section{Testing for anisotropy}\label{sec:TestingAnisotropy}
The specific hypothesis to be tested is whether, above some energy threshold, $E_{\rm th}$, the mean composition of UHECRs coming from directions near to the galactic plane is significantly higher in mass than those arriving further from it. This is to be tested using \xmax{} as a mass sensitive parameter. Typically, \xmax{} based composition analyses leverage the first two moments of \xmax{} distributions binned in energy, to comment on primary mass. This approach, however, does not lend itself well to quantifying the significance of a result testing the above statement. Instead, a test statistic, $TS$, which quantifies the degree of dissimilarity between the \xmax{} distributions in the two regions in a single value is preferred. For this, the returned value from the Anderson-Darling two-sample homogeneity test \cite{andersondarling}, \textit{AD-test}, has been selected as it scales with the dissimilarity of the tested distributions. The AD-test has good sensitivity to the full width of a distribution \cite{scholz1987k}, and has more power than the Kolmorogov-Smirnov test while remaining robust against false positives \cite{engmann2011comparing}.

To use the AD-test and \xmax{} for this purpose, two modifications are required. First, a single $TS$ comparing all events in each region above $E_{\rm th}$ is desired. So, all events with $E\geq E_{\rm th}$ in the on- and off-plane samples separately need to be collected into a common on-plane distribution and a common off-plane distribution. To do this, the natural evolution of \xmax{} with energy needs to be removed so that spectral features in the flux do not influence the result. Therefore, we define an energy-normalized \xmax{} value
\begin{equation}\label{eq:XmaxNorm}
X_{\text{max}}^{'} =  X_{\text{max}} -  \underbrace{\left(649 + 63.1 \, Z_{18} + 1.97 \, Z_{18}^{2}\right)}_\text{EPOS-LHC elongation rate for iron},
\end{equation}
where $Z_{18}=\log_{10}\left(E_\text{rec}/\,\text{EeV}\right)$. The last term in \autoref{eq:XmaxNorm} is the natural energy evolution of mean \xmax{} for iron primaries as predicted by EPOS-LHC~\cite{Pierog:2013ria}\footnote{Choice of hadronic interaction model varies result by $\sim0.02$\,\gcm{}.}. Second, the \xmaxnorm{} distribution of an on-plane sample populated with primaries which are on average heavier than those in the off-plane sample will display a lower mean and a narrower width than that of the off-plane \xmaxnorm{} distribution. Since the null hypothesis is that there is either no composition difference or a heavier off-plane sample, a $TS$ sensitive to the ordering of the \xmaxnorm{} distributions is required\footnote{Modifying the test to also require $\sigma( X_{\text{max}}^\prime)^{\rm on} < \sigma( X_{\text{max}}^\prime)^{\rm off}$ would be more restrictive, but conservatively has not been applied.}. The AD-test is insensitive to ordering, so it is modified to
\begin{equation}
TS =
\begin{cases}
    AD: \langle X_{\text{max}}^\prime \rangle^{\rm on} < \langle X_{\text{max}}^\prime \rangle^{\rm off} \\
    -3\hspace{1mm}: \text{else}
\end{cases},
\end{equation}
where $AD$ is the result of the AD-test comparing the on- and off-plane distributions, and $-3$ is selected as it is well below the minimum of the AD-test.

\vspace{-.1cm}
\myparagraph{Scan for energy and galactic latitude thresholds}
\vspace{-.1cm}
A scan has been used to select the optimal on/off splitting latitude, $b_{\rm split}$, and minimum energy, $E_{\rm th}$, as uncertainties in GMF models and source distributions make other approaches impractical. In this scan, each trial [$E_{\rm th}$, $b_{\rm split}$] pair is used to form on- and off-plane subsets and the $TS$ is extracted. To preserve the statistical strength of the sparse FD data set, a coarse scan of $5^\circ$ steps in $\abs{\,b\,}$ from $20^\circ$ to $35^\circ$ and 0.1\,\lge{} steps in energy from $18.4$ to $19.4$\,\lge{} is used. The scan is performed on the data set from~\cite{Aab:2014kda}, which includes events through Dec 31\textsuperscript{st} 2012. At the time of writing, this \textit{scan data set} represents $54\,\%$ of the analyzed events. The remaining $46\,\%$ of events, the \textit{post-scan data set}, is reserved as blind. 

\begin{figure}[!htb]
    % \vspace{-.4cm}
    \centering
    \includegraphics[width=0.45\textwidth]{Figures/ScanResults.pdf}
    \vspace{-2mm}
    \caption{Parameter scan over 54\% of the data.}\label{fig:PRDScan}
    % \vspace{-7mm}
\end{figure}

Interestingly, as shown in \autoref{fig:PRDScan}, all tested pairs result in $\langle X_{\text{max}}^\prime \rangle^{\rm on} < \langle X_{\text{max}}^\prime \rangle^{\rm off}$. An optimal [$E_{\rm th}$, $b_{\rm split}$] of [$10^{18.7}$\,eV,$30^\circ$] was found with a $TS = 8.4$. The selected [$E_{\rm th}$, $b_{\rm split}$] is applied as a prescription to the post-scan data set, which independently confirms the result with a $TS = 12.6$, for a total $TS=21.0$ for the full data set. 

\vspace{-.1cm}
\myparagraph{Statistical significance}
\vspace{-.1cm}
The chance probability of the observed TS occurring with in an isotropic sky is tested using Monte Carlo methods on randomized skies derived from the real data. To form each randomized sky, the arrival direction is first decoupled from the energy and \xmaxnorm{} values of each event. These are then randomly re-paired to create a new sky which maintains the real \xmax{}, energy, and sky exposure distributions, but has a scrambled arrival direction/composition pairing. The above analysis is then used to extract a $TS$ from each sky which is compared to the result in data. Skies which display more extreme on-/off-plane differences than those observed in data are tallied and used to calculate the probability of an isotropic sky generating the observed $TS$. The results of this procedure are shown in  \autoref{fig:TStoSignificanceConversionNew}.

\begin{figure}[!htb]
    \centering
    \includegraphics[width=.8\columnwidth]{Figures/MCADSig.pdf}
    % \vspace{-3mm}
    \caption{The Monte Carlo determination of the post-scan (red) and all-data (blue) significance with 1 and 10 billion randomized skies, respectively.}\label{fig:TStoSignificanceConversionNew}
    % \vspace{-3mm}
\end{figure}

For the blind, post-scan data set, the prescribed [$E_{\rm th},b_{\rm split}$] pair is used to split each randomized sky into on- and off-plane samples and a $TS$ is extracted. In one billion random skies, only 5865 resulted in a more extreme $TS$ than the 12.6 observed in data. This indicates a chance probability of $5.87\times10^{-6}$ which corresponds to 4.4\,$\sigma$. 

To calculate the significance of the result when the scan- and post-scan data sets are combined, the entire analysis chain, including the scan, is duplicated. In each random sky, 54\,\% of the data is used to scan for the [$E_{\rm th},b_{\rm split}$] pair which results in the most extreme result, fully penalizing for the scan. These values are then used to split all data in the random sky into on- and off-plane subsamples and the $TS$ for the sky is extracted. From 10 billion random skies, only 5964 resulted in a more extreme $TS$ than the 21.0 observed in data. This indicates to a chance probability of $5.96\times10^{-7}$ which corresponds to 4.9\,$\sigma$. The strong penalization of the scanned data is evident as the additional 54\,\% of the data (with \Dxmaxmunorm{} $= 8.5$\,\gcm{}) only resulted in an 11\,\% increase of the significance of the observation. 

\myparagraph{\xmax{} moments and trends}

To illustrate the difference in composition on and off the plane, the first two moments of the \xmax{} distribution in each 0.1\,\lge{} energy bin has been plotted in \autoref{fig:CompositionPlots} for both regions. Above $10^{18.7}$\,eV there is a clear separation in \xmaxmu{} for all energy bins. Most energy bins also display a separation in \xmaxsigma{}. Heavier primaries are expected to, on average, have a shallower \xmax{} and lower shower-to-shower fluctuations. Therefore the correlated difference seen here indicates that, for this data sample, primaries from the on-plane region have a higher mean mass than that of the off-plane region above $10^{18.7}$\,eV.

To evaluate the degree to which fluctuation plays a role in the observed result, the growth of the $TS$ over time has been plotted in \autoref{fig:TimeEvolution}. The time evolution of the signal is consistent with linear growth at a rate of 1.3\,$TS$\,yr$^{-1}$. This behavior is in line with expectations for a real difference in mean mass between the subsamples. The shaded region of \autoref{fig:TimeEvolution} shows preliminary data from 2019. These reconstructions were not subject to a validated reconstruction chain and may change. Still, when added, a 3.7/4.4\,$\sigma$ (post-scan/all data) statistical significance is expected. The best fit rate of growth of 1.3\,$TS$\,yr$^{-1}$ remains unchanged.

\begin{figure*}[!hbt]
\centering
    \begin{minipage}{.63\textwidth}
      \centering
      \captionsetup{width=.9\linewidth}
      \includegraphics[width=.49\textwidth,valign=t]{Figures/Mean-crop.pdf}
      \includegraphics[width=.49\textwidth,valign=t]{Figures/Sigma-crop.pdf}
      \vspace{-1mm}
      \caption{The first (left) and second (right) moments of the \xmax{} distributions from on- and off-plane regions.}
      \label{fig:CompositionPlots} 
    \end{minipage}%
    \hfill
    \begin{minipage}{.35\textwidth}
      \centering
      \vspace{-1mm}
      \includegraphics[width=\textwidth,valign=t]{Figures/TimePredict.pdf}
      \vspace{1.5mm}
      \captionof{figure}{The time evolution of the TS with significance indicated on the right. The shaded region is preliminary data.}
      \label{fig:TimeEvolution}
    \end{minipage}
\end{figure*}
% \section{Roadmap}
% \TODOS{May delete or merge with the method, NEED TO UPDATE ALONG WITH THE FRAMEWORK FIGURE}
% In the following sections, we report our findings that answer the research questions respectively. We first report the interaction experiences participants perceived in the task and the factors that contribute to such experiences. These factors include \TODOS{XXX}. Second, we analyze how these experiences influence users’ perception of authentication, including security and privacy. Our thematic analysis reveals that \TODOS{XXX}. Last, we summarize users’ expectations of adopting interactive authentication in VR. We center our results around the content analysis on participants’ responses, and we use statistics, e.g., users’ rating on perceived security, to support our analysis.

\begin{figure*}[h]
\centering
\includegraphics[width=\textwidth]{fig/Framework.pdf}
\caption{Our analysis framework and summary of results. We connect themes that are most related across research questions. }\label{fig:framework}
\end{figure*}

\section{RQ1: Interaction Experience}

Participants’ interaction experiences are dependent on factors around two main aspects: (1) the perceived usability of authentication using the four setups and (2) their experience in the VR game -- the context for payment authentication.







\subsection{Perceived Usability of Authentication}
We reveal the components in authentication interaction and their associated characteristics, which affect overall usability.

\subsubsection{Components of Authentication Interaction}
\label{sec:Components of Authentication Interaction}

We categorize the interaction components of VR authentication interaction into three themes: (1) motion control, (2) authentication interface, and (3) process of authentication.

\paragraph{Motion control.} Participants performed gestures and interacted with digital objects when they authenticated. The motion control in making these actions in VR impacted the usability of authentication regarding the three subthemes (1) spatial awareness in the VR space, (2) action control and consistency, and (3) interaction modality.

First, \textit{spatial awareness in the VR space} includes feeling a virtual object in relation to their avatar. Some participants noticed their lack of spatial awareness hindered them in controlling a virtual object, e.g., stretching their arm to sign a signature:
\begin{displayquote}
\textit{``It takes a while to get used to the proper distance between the pen and the kiosk screen.''} (P7)
\end{displayquote}


On the other hand, the \textit{action control and consistency} impacted the quality of participants' actions, e.g., how their gestures aligned with their intention, via the hand-held controllers.
Participants preferred to have better \textit{action control and consistency} when interacting with digital components. For example, one participant preferred \pink over \pinf due to such consistency in entering PINs. 
\begin{displayquote}
\textit{``The floating idea seemed okay, but I think the stable kiosk was easier to use as it allowed for better calibration of my pointer.''} (P10)
\end{displayquote}

Last, participants had different preferences over the \textit{interaction modality} used to control interface components, e.g., using the controller to sign:
\begin{displayquote}
\textit{``could be easier if I can just use my finger to sign because grabbing the pen is not [a] very good experience. ''} (P24)
\end{displayquote}
They also compared it to other interaction modalities in the real world, for example, \textit{``laser pointer is less similar to using keyboards in real world.''} (P2)



\paragraph{Authentication interface.} The design of the authentication interfaces affected participants’ perceived usability regarding (1) the interface presentation and (2) the interaction feedback.



The \textit{interface presentation} relates to how the interface components for authentication are displayed and visualized. 
Participants sometimes expect different presentations in VR than in the real world. For example, \pink displays the PIN pad on the kiosk, which requires participants to move their avatar compared to \pinf:
\begin{displayquote}
\textit{``This seemed like the real world kiosk, where the number-pad was at a fixed place. I had to move to get a better look at the numpad.''} (P13)
\end{displayquote}
In another example, one participant mentioned the aesthetic style of the virtual card:
\begin{displayquote}
\textit{``expect some more fancy effects than real life card experience, such as when success the card shows another color etc.''} (P24)
\end{displayquote}


The \textit{interaction feedback} refers to how the authentication system confirms participants' actions, e.g., delivering a confirm message as P24 described above.
Some participants were concerned about the lack of this feedback during authentication in VR. Such feedback includes visual and haptic cues. Sometimes, they expected the feedback they would receive in the real world, e.g., the sense of writing on a paper.
\begin{displayquote}
\textit{``It is still usable and the pen writes like real writing but still did not get the feeling of writing on a paper.''} (P1)
\end{displayquote}


\paragraph{Process of authentication.} The process of authentication affects perceived usability due to (1) its learning curve, (2) the transition in workflow, and (3) knowledge to memorize.
The \textit{learning curve} refers to how easy participants will get used to the interactions in VR. For example, one participant felt signing was easy after practicing.
\begin{displayquote}
\textit{``It's very easy to learn and use, and the functionality can be easily picked up.''} (P3)
\end{displayquote}

Meanwhile, authentication probes may require multiple steps of interaction, i.e. the \textit{transitions in the workflow}
participants perceived simpler transition in the workflow of authentication as merit, which relates to the difficulties of VR motion control:
\begin{displayquote}
\textit{``I do not need to accomplish complicated movements/actions in the virtual env.''} (P18)
\end{displayquote}
This aspect was also related to the security measures, i.e., PIN pad shuffling in our probes: \textit{``The challenging part was that the numbers shifted locations between attempts''} (P13).

In addition, how much \textit{knowledge to memorize} the participants should memorize is associated with their experiences. More than remembering every step to complete the authentication, participants mentioned the burden of memorizing the PIN: \textit{``It was hard for me to memorize the pin''} (P11).


\subsubsection{Usability Characteristics of Authentication}
Participants mentioned multiple usability characteristics of the authentication interactions related to the above components. These characteristics are in three themes: (1) easiness, (2) realism, and (3) intuitiveness. Then, we explain how the characteristics, along with the interaction components, affect the overall usability of each authentication probe.


\paragraph{Easiness.} Other than generally describing an interaction as easy or cumbersome, participants associate easiness with \textit{comfortability} of interacting in VR, the \textit{physical, mental, and time efforts}, and the \textit{smoothness} of the interaction process. Participants perceived the easiness of one authentication method differently. For example, as mentioned before, some participants thought signing was easy, while others did not. one participant felt it was easy conceptually but hard in practice when performing the VR gesture.
\begin{displayquote}
\textit{``Signature: it was conceptually easy, but the execution was somewhat cumbersome and it required a complex gesture.''} (P6)
\end{displayquote}
 
\paragraph{Realism.} Participants often liked the VR interaction and interfaces that are realistic as in real life, which made them feel \textit{familiar} and \textit{immersed}. However, the participants differed in such perception due to their prior experience. For example, one participant did not think a shuffled PIN pad is realistic for an ATM or gas station.
\begin{displayquote}
\textit{``I got confused by the randomization of the numbers on the pin-pad. I am not used to this since most terminals have a fixed layout (I am thinking of ATMs and gas stations).''} (P5)
\end{displayquote}
Meanwhile, when participants felt immersed in VR, some of them did not like authentication to break such experience through an ``unreal'' interface:
\begin{displayquote}
\textit{``the floated pad makes it so unreal that I know it is in VR rather than real life. I do not like the experience.''} (P24)
\end{displayquote}

\paragraph{Intuitiveness.} Last, participants were in favor of interfaces that are intuitive. The intuitiveness appears as a result of the interaction interface being \textit{simple} and \textit{clear}, and participants could rely on their prior authentication experience in real life: \textit{The interface is intuitive and somewhat matches what you have in real life''} (P7). However, as mentioned above, some participants got confused by the interaction components, e.g., the shuffled PIN pad.


\paragraph{Overall usability.} In Figure~\ref{fig:sus}, the overall usability for the four probes via participants’ SUS rating, a standard usability metric. In addition, we also compare the authentication time in Figure~\ref{fig:time}.
\tap received the highest SUS score (mean: 82.1, std: 14.6), indicating ``excellent'' usability~\cite{brooke-jus13}. It has the merit of easiness, realism, and intuitiveness. 
Nevertheless, some users still mentioned the burden of walking towards the kiosk to tap: \textit{``because in this game I [the participant] have no arms that I can extend my reach to a kiosk that is a few feet away''} (P16).

\pinf (mean: 70.6, std: 19.9) and \pink (mean: 76.1, std: 11.7) closely follow \tap, showing a ``good'' usability. Participants liked their usability because they were familiar with this scheme, and it was relatively easy, despite the effort in memorizing and entering the PINs. However, the margin between \pinf and \pink is small. Participants commented on different usability issues for them. For \pink, though it looked more real, some participants felt burdensome walking towards the kiosk to see the PIN pad clearly; others thought entering the keys on a moving PIN pad (\pinf) was distracting. 

\sign (mean: 60.7, std: 14.7) has the lowest score among the four but still has an ``OK'' usability. Though its interface seemed intuitive, signing with the virtual pen was not easy for multiple participants in VR, as \textit{``signing in the virtual world was very different as compared to the real world''} (P13).


\begin{figure}[h]
\centering
\includegraphics[width=\columnwidth]{charts/ipq.pdf}
\caption{IPQ sense of presence scores for the four authentication probes. A higher score indicates a higher level of presence. From 0 to 6, a score higher than 3 stands for neutral. All subscales have a positive mean score, except REAL. The IPQ scores of our setups are consistent with prior implementations for VR authentication~\cite{mathis-vr22}.}\label{fig:ipq}
\end{figure}

\subsection{Experience in the VR Game}
Participants’ interaction experiences also consist of their game experiences. We notice that participants perceived a high presence in the VR world and demonstrated high engagement in the game.


\paragraph{Feeling present in the VR world.} Using the IPQ presence questionnaire (Figure~\ref{fig:ipq}), we observe that participants had a positive rating on their presence in the VR world overall. At the same time, we do not see a noticeable difference in the IPQ scores between different authentication probes, which confirms that authentication is perceived as a secondary task~\cite{de-soups10}.

\paragraph{High engagement in the game.} We found that participants were highly engaged in the archery game and the routine payment. When describing their motivation in deciding on the arrows to pay for, participants mentioned the reasons for \textit{their strategy to compete} and \textit{enjoyment of the game}. Participants’ strategies are based on their confidence in the archery skill and a cost-benefit analysis. Among them, 19 explicitly mentioned that they would like to shoot as many arrows as they could, and six people said it was fun to play. However, participants’ archery performances differed a lot. They obtained a final score of 824.17 on average but had a large gap between the highest and the lowest (1470 and 190).



\section{RQ2: Security and Privacy Perception}



This section reveals how the interaction experiences we discussed in the previous section factor into participants’ security and privacy perception of authentication. 
We qualitatively analyze participants' responses and show five themes in Figure~\ref{fig:framework}. We then explain each aspect regarding participants' rating on the four authentication probes.




\subsection{Influences of Interaction on Perception}
\label{sec: Influences of Interaction on Perception}

We present five key insights that explain how the interaction factors influence participants' perception of authentication. We find that interaction with authentication mechanisms in VR serves as a bridge between participants’ perceptions of VR and their real-life experiences. However, contextual uncertainties in VR make understanding of threats and risks more difficult.

\paragraph{Realistic authentication interactions help users translate real-life perceptions into VR.} More than contributing to usability, realistic authentication interactions and interfaces bridge participants’ security and privacy perceptions. However, there is still a gap in fully mirroring users’ real-world sensations in VR, which reduces the perceived consent and security. For example, this gap reduces participants’ feeling of giving consent to pay when signing in VR.
\begin{displayquote}
\textit{``The sign-to-pay method was a bit hard to use, so I think I just tried to write something, and I felt less like providing my signature.''} (P10)
\end{displayquote}
Similarly, the same participant felt \tap not secure as they still thought \textit{``it's not my [their] real card but a card-like object''} (P10). 

Second, participants’ real-life experiences differed, manifesting in their perceptions in VR. 
For example, one user thought \sign is less reasonable as they \textit{``rarely signed to pay (only recently in the US…)’’} (P6). Meanwhile, others \textit{``naturally perceive it as giving my [their] consent.''} as they usually did in the physical world (P12).

In addition, participants sometimes evaluated the realism of authentication interaction based on how it is related to the context of payment. One participant noticed little difference in how the four authentication probes alert them for payments: 
\begin{displayquote}
\textit{``Similar to transactions in physical world i felt the need to be alerted for payments across all the methods.''} (P13)
\end{displayquote}
Another participant thought authentication by card-tapping in \tap is most specific to payment scenarios. 
\begin{displayquote}
\textit{``Grabbing the card and making the payment made it seem like an actual payment. The other 3 felt like they could have been anything.''} (P9)
\end{displayquote}

However, interactions that exactly mirror the physical world might not necessarily fulfill participants’ security and privacy needs. They expected \textit{``we could do more than real life with card''} in VR (P24) --  as mentioned in Section~\ref{sec:Components of Authentication Interaction}, P24 thought the virtual card could inform them of authentication results by changing its color. P10 felt more secure and alerted to pay when using the floating PIN pad of \pinf, which gives them a more personal view compared to the kiosk in \pink. 
\begin{displayquote}
\textit{``I think pin numbers are genuinely more secure, but I would vote more highly for the floating method because it seems a bit harder for others to see the numbers that I am putting in.''} (P10)
\end{displayquote}

\paragraph{Usability challenges and a lack of feedback in VR interactions make users feel insecure.} 
Usability challenges during interaction reduced realism of authentication mechanisms for participants. In addition, lack of proper feedback negatively impacts participants’ perception of consent in VR. Some participants assumed that the certain authentication mechanisms trade off security for usability %
e.g., accepting an inconsistent or even impersonated signature:
\begin{displayquote}
\textit{``The sign-to-pay felt the most insecure as people easily have access to my cheques and can probably fake in the VR world since the VR signatures were clearly less accurate than the real-world.''} (P5)
\end{displayquote}


Authentication mechanisms that involved interactions without apparent feedback made participants question their security despite good usability, e.g., unexpected behaviors in VR, e.g., \textit{``making accident payment''} (P15). In another example, P2 thought \tap seemed too easy without any warning: 
\begin{displayquote}
\textit{``Tap-to-pay seems more no-brainer, so it is better to give warnings.''} (P2)
\end{displayquote}
A subset of participants were willing to accept additional friction during interaction for security and transparency, including the shuffled PIN pad.


\paragraph{Users lack the understanding of VR authentication processes behind the interface.} 
Due to similar processes in VR, participants transferred their prior knowledge about physical-world authentication into understanding the security and privacy of VR authentication. Examples of the processes include possessing secret knowledge (PIN), that is \textit{``only known by me [the participant]''}, and using the shuffled PIN pad, which \textit{``gave me [the participant] some sense of security''} (P20, P17). Though participants' prior understanding could differ, for example, whether the PIN is personal compared to the signature:
\begin{displayquote}
\textit{``Sign-to-pay requires my signature, so it requires more information from me. The pin is not personal information, so the privacy concern is less.''} (P20)
\end{displayquote}

Meanwhile, participants also brought their knowledge in identifying the vulnerability of VR authentication. For example, PINs and signatures in VR are also prone to malware.
\begin{displayquote}
\textit{``Similar reason that pins and signatures are just exposed to the game or malware in the gaming system.''} (P24)
\end{displayquote}

Moreover, participants were aware of the differences in VR authentication’s backend compared to authentication in the physical world, although they shared similar interactions and interfaces. However, they felt difficult in assessing the inherent security and privacy of VR authentication, \textit{``without more information on how the payment actually works''} (P13). Some participants assumed VR used different mechanisms than the physical world. Some of them thought that there could be additional verification or certification steps for the virtual card in \tap.
\begin{displayquote}
\textit{``I don't know how the backend works, I assume it should quire a pre-certification process such as a cookie.''} (P3)
\end{displayquote}

As such, P3 also struggled to define a proper threat model:
\begin{displayquote}
\textit{``Ofc, to consider this deeply requires a solid definition about what adversary we are facing. E.g., a man-in-the-middle or ish.''} (P3)
\end{displayquote}


Last, some participants felt a lack of trust due to such ambiguity in VR authentication and the technology generally. For example, P12’s described how their negative perception of \tap’s security came from their distrust of VR.
\begin{displayquote}
\textit{``Tap-to-pay was the simplest, but it was way too simple to believe that the entire payment process behind the scene was dealt with as I wanted. I'm not sure if this is due to my distrust to the specific payment system, or just to the VR world, or both.''} (P12)
\end{displayquote}


\paragraph{Users associate uncertainties in the VR environment with potential threats.} 
Participants also lacked awareness of their VR environment. 
Some were not sure how the VR environment would cater to multiple users sharing the same space, and prevent users from performing malicious actions. P1 was afraid that he would expose his PINs to other users in the same VR space.
\begin{displayquote}
\textit{``If I was sharing my virtual space with others and they can see what I was typing then I would have gone for "Sign to pay" (although not so usable).''} (P1)
\end{displayquote}
Another concern with malicious users is that they could leverage their prior knowledge about a potential target from the physical world to launch attacks in VR:
\begin{displayquote}
\textit{``signing can be also copies by anyone who knows my signature in the real world.''} (P12)
\end{displayquote}

Participants were also concerned about attacks in VR that could be more imperceptible to them than the physical world, e.g., using an invisible terminal to phish users. 
\begin{displayquote}
\textit{``Tap-to-pay is the easiest, but I feel it not safe because I can easily touch it to an invisible system in the VR world.''} (P19)
\end{displayquote}
Attacks in VR may have different degrees of noticeability. For example, participants would not necessarily realize their PINs being observed \textit{``unlike [attackers] stealing a card''} (P6).

\paragraph{The gamified VR context alters users’ sensitivity to security and privacy.} 
Note that our context for user authentication is participants’ routine payment in a VR game. As described, participants were highly engaged in the game. However, such game context reduced some participants’ sensitivity to security and privacy. P7 argued that they did not feel losing privacy in a VR game compared to real-life transactions.  
\begin{displayquote}
\textit{``In the context of the game I didn't feel like giving away privacy. However if I were to imagine this with real transactions, then I'd feel like giving some of my privacy away, similar to every time I pay with something else than cash.''} (P7)
\end{displayquote}

The gamified interactions and interfaces of some authentication probes also made some participants feel less related to security and privacy.
For example, when signing a signature in VR, the feeling of playing a game overrode P7’s sense of giving consent.
\begin{displayquote}
\textit{``The sign to pay wasn't completely obvious you were actually paying for something, it could have been part of the game to have to write your name.''} (P7)
\end{displayquote}

Similarly, P10 thought \pinf \textit{``felt a bit too much like being in a game as well''} compared to \pink (P10).


\begin{figure}[h]
\centering
\includegraphics[width=\columnwidth]{charts/likert.pdf}
\caption{Overall security and privacy perception of the four probes. We color-coded the bars (red: strongly disagree, orange: disagree, grey: neither agree nor disagree, light blue: agree, dark blue: strongly agree)}\label{fig:overall_perception}
\end{figure}








\subsection{Overall Perception}
In Figure~\ref{fig:overall_perception}, we show participants’ overall rating of perception regarding the five aspects (consent, security, privacy, being alerted, and in control). Below we further explain their ratings.

\paragraph{Consent.} \pinf and \pink made more participants feel they gave consent compared with the \tap and \sign.
Compared to \tap, the higher level of user interaction in providing PINs contributed to this result. However, \tap still gave participants a decent feeling of giving consent, benefiting from the realistic interface. As mentioned before, the usability hurdle of signing in VR reduced participants’ feeling of providing consent using \sign.


\paragraph{Security.} Most participants associated PIN authentication, a familiar method, with security. 
Though participants were still concerned about shouldering-surfing attacks, some appreciated the shuffled PIN pad and the more personal PIN pad in \pinf. The response for \tap is polarized (many of them responded neither disagree nor agree). Participants lacked information to assess this method from the interaction. Some participants were afraid of unexpected behaviors; Others trusted \tap to have comparable security measures as the physical credit card. As mentioned for \sign, participants thought their VR signatures could be more easily forged as the system seemed to have a higher tolerance for accepting inconsistent signatures.

\paragraph{Privacy.} The participants were more in agreement than disagreement that \tap put their privacy less at risk. For those who felt they were losing privacy in \tap, they thought their name and card number rendered on the virtual card was a concern. In contrast, more participants associated drawing their signature with a privacy loss to either the payment authentication systems or bystanders. In the two PIN methods, participants were less concerned about their privacy when compared to \sign but not as confident as they are in \tap. This contradiction is due to whether the participants considered the secret PIN personal information.

\paragraph{Being alerted.} Participants’ responses to all four authentication setups are polarized for this aspect. In general, higher levels of interaction alerted users. Such polarized responses are due to how much attention the participants paid to compensate in matching their cautiousness on potential security risks. For example, some participants paid extra attention to \tap as it was too easy, while others did not.

\paragraph{In control.} Engagement in interaction smoothly and the sense of ownership made participants feel in control. For example, participants thought they were in control of \pinf and \pink the action of providing PIN is explicit and looks familiar. In contrast, several participants thought they did not feel in control with \tap, as the card did not seem too personal, and the interaction appeared frivolous. The barriers to drawing VR signatures made some participants feel out of control with \sign.




 









\section{RQ3: Meeting User Expectations}


In this section, we summarize how we can meet participants’ requirements and expectations for VR authentication from their preferences on our authentication probes, suggestions to improve the probes, and desired qualities of payment authentication. 

\begin{figure}[t]
\centering
\includegraphics[width=\columnwidth]{charts/rank.pdf}
\caption{Overall payment preference of the four probes. The bars are color coded dark blue, light blue, orange and red to indicate the fraction of users that selected each probe respectively to be their 1st, 2nd, 3rd and last choice.}\label{fig:payment_pref}
\end{figure}




We first show participants’ overall preference over the four probes. The majority of the participants selected \tap as their top priority and selected \sign as their last choice. \pinf and \pink, with little difference between themselves, follow \tap. This result directly reflects that participants prioritized the usability of authentication. Nevertheless, most participants also mentioned that they valued security and privacy, along with usability.

Next, we present four themes from users’ requirements for VR authentication. These themes mainly center around improvements in authentication interaction, which also improve participants’ perceived security and privacy.

\paragraph{Providing flexible interaction modalities to authenticate.} Participants expected interaction modalities that are more flexible and personalized to them. Such flexibility might make authentication in VR easier through better realism and accessibility. For example, some participants considered hand-gesture interaction an alternative to a controller. Moreover, the authentication system could support a variety of interactions inspired by real-world interaction.
For example, P14 mentioned the potential of swiping or inserting a card to pay other than tapping:
\begin{displayquote}
\textit{``There are also insert-card-to-pay and swipe-to-pay scenarios.''} (P14)
\end{displayquote}


\paragraph{Being creative in designing authentication interfaces for VR.} 
Participants were aware that mapping realistic authentication interactions into VR might reduce the learning curve in early adoption, especially for the elderly: 
\begin{displayquote}
\textit{``The users (especially elders) would hesitate to move onto this whole new experience if there is nothing that resembles their previous experience in the real world.''} (P12)
\end{displayquote}
Nevertheless, many of them expected the VR authentication interface to be creative instead of using an exact mapping. Such creativity may also improve usability, e.g., saving physical effort by \textit{``tap to pay on a floating panel''} (P23); or it may add functions regarding the usage in payment. For example, P20 desired a wallet interface to hold multiple virtual cards, and they could \textit{``select the card from a pop up UI''} (P20); Further, participants also anticipated emotional appeals (e.g., enjoyment) from the authentication interfaces, especially in the game context.
\begin{displayquote}
\textit{``I want the payment process to be cool and make me feel good. some visual effects could help and make me happy to make the payment.''} (P24)
\end{displayquote}
However, multiple participants expected such creativity and novelty in VR without brining specific ideas.
\begin{displayquote}
\textit{``However I do hope there are better ways to do than emulating real world payment system in the virtual world.''} (P7)
\end{displayquote}

\paragraph{Informing and offering feedback for VR authentication from multiple channels.} 
Many participants desired feedback and transparency in VR authentication for security and privacy. First, they expected feedback in the virtual world to increase their awareness of the VR surroundings and unexpected behaviors. For example, P21 suggests using a virtual mirror to inform them of their surroundings.
\begin{displayquote}
\textit{``For pin-to-pay, maybe add some mirrors to reflect the surrounding environment.''} (P21)
\end{displayquote}
Similarly, P24 wanted to \textit{``add a cancelation feature in case of an accidental grap-and-tap''} (P24). Also, VR authentication may leverage different modalities, e.g., visual effects and sound, to deliver feedback.

Moreover, participants demand transparency of VR authentication from the physical world. P8 wanted to be informed about suspicious activities through email even when they are in the physical world.
\begin{displayquote}
 \textit{``I could only become sure of its security depending on how the system notifies me w.r.t. suspicious activity (e.g., sending me an SMS/email/etc.)''} (P8)
\end{displayquote}
Meanwhile, service providers may open-source their protocol to authenticate users.
\begin{displayquote}
\textit{``Transparent: All the codes and methods behind should be well-defined and open-sourced.''} (P3)
\end{displayquote}


\paragraph{Adapting the security of authentication to the context of payment.} In Section~\ref{sec: Influences of Interaction on Perception}, we find that the VR game might reduce some participants’ sensitivity to security and privacy. Nevertheless, participants did expect security measures for different contexts of payment. One participant proposed that authentication can happen at launch time on device instead of a routine in the app -- \textit{``after authentication one can use any stored information to do payment (including card information).''} (P11)
Nevertheless, P8 stated that multi-factor authentication can be helpful when handling large payments.
\begin{displayquote}
\textit{``Tap-to-pay: ask the user to enter pin from time-to-time (or when the object in question is above a certain price) as an extra layer of authentication.''} (P8)
\end{displayquote}
Also, the system may offer additional security for shared usage e.g., to ensure \textit{``no accident payment from the kids''} (P20). 

In summary, we observe that most participants’ requirements centered around interaction experiences and the usability of authentication. Their requirements align with our findings in RQ2 -- how interaction experiences influence security and privacy perception. However, participants’ requirements could be non-specific, especially regarding security and privacy, due to their lack of understanding of VR. Sometimes, participants’ requirements might lead to conflicts, e.g., the need for feedback vs. the interaction effort, but participants rarely named options to resolve these conflicts.






\section{Discussion and New Perspectives}\label{sec:discuss}
% and Future Directions

In this section, we first discuss challenges and practical considerations, including non-stationarity, heterogeneity, unobserved confounders, subsampling, and expert knowledge.
Then, two new perspectives of temporal causal discovery are provided, which in our opinion will be a promising avenue for future research.

\subsection{Challenges and Practical Considerations}



\textbf{Non-stationarity of data:} We are often faced with non-stationarity in practical scenarios, where the probability distributions of temporal variables conditional on their causes or even the causal relations may change across time, especially for temporal data.
In this condition, causal discovery approaches presuming a fixed causal model may give misleading results. 
Whereas, several types of research have shown that non-stationarity contains information for causal discovery \cite{CD_from_change/conf/uai/TianP01, CD_from_change/peters2016causal, Discussion/Nonstation_hetero/ijcai_ZhangHZGS17, Discussion/Nonstation/state_space_icml_Huang0GG19}.
Thus, it's important to properly tackle the non-stationarity in applications.
Non-stationarity may result from the change of underlying systems and can be seen as a soft intervention \cite{soft_interv/korb2004varieties} done by nature. 
Following this idea, a line of work \cite{Discussion/Nonstation_hetero/ijcai_ZhangHZGS17, Discussion/Nonstation_hetero2/jmlr/Huang0ZRSGS20} leverages a surrogate such as time and domain index to account for nonstationarity where the causal relations are changed, and the CD-NOD framework is proposed. 
Instead of leveraging informative non-stationarity to causal structure learning, another set of research focuses on modeling time-varying relationships \cite{Discussion/Nonstation/pr_GaoY22}. 
Besides, the approach for slowly varying non-stationary process, such as evolutionary spectral and locally stationary processes, is proposed in \cite{Discussion/Nonstation/slowly_varying/du2020causal}.






\textbf{Heterogeneity of data:} In causal discovery for practical applications, the heterogeneity of data lies in two levels: (1) The interacting temporal processes are heterogeneous (having different distributions), for instance, causally related meteorological observations from different stations are influenced by several major weather systems separately \cite{Discussion/heterogeneous/pakdd_BehzadiHP19}. (2) The underlying generating process changes across data sets or different domains \cite{intro/nonts_surveys/glymour2019review}, for instance stock prices from different markets \cite{Discussion/Nonstation_hetero2/jmlr/Huang0ZRSGS20} or individual behaviours in different paradigms \cite{MTS/Attention/icdm_InGRA_ChuWMJZY20}.
For the first condition where the heterogeneity exists among temporal variables, the inferred relations of the traditional causal discovery approaches, which have been designed for specific homogeneous data types, may be inaccurate. As a remedy, several variants of Granger causality, based on methods such as generalized linear models and minimum message length, are proposed in \cite{Applications/anomaly/work2_icdm_BehzadiHP17, Discussion/heterogeneous/pakdd_BehzadiHP19, Discussion/heterogeneous/entropy/Hlavackova-Schindler20}.
For the second condition, a line of work \cite{Discussion/Nonstation_hetero/ijcai_ZhangHZGS17,  Discussion/Nonstation_hetero2/jmlr/Huang0ZRSGS20} leverages the distribution shift from heterogeneity as a soft intervention to assist causal structure learning, which is similar to that in non-stationary data.  
Whereas, another line of causal discovery approaches \cite{MTS/Attention/icdm_InGRA_ChuWMJZY20, Discussion/NewForm/ACD_LoweMSW22} in the second condition focuses on inductively modeling typical structure in heterogeneous data within an end-to-end framework. 



\textbf{Unobserved confounders:}
In practice, we are often met with cases where causal sufficiency is violated, \ie, there exist unobserved confounders. 
This challenging setting may lead to incorrect causal relations~\cite{MTS/FCM/VAR_LINGAM_extend2_icml_GeigerZSGJ15}.
As summarized in Table~\ref{tab:ts_category_overview}, most temporal causal discovery approaches cannot handle unobserved confounders in a straightforward way.
Several constraint-based approaches are designed without causal sufficiency and approaches
Besides, unobserved confounders are modeled by applying a structural bias in~\cite{Discussion/NewForm/ACD_LoweMSW22}.
Several recent studies termed as causal representation learning take a new perspective on unobserved confounders.
It will be detailed in subsection (\ref{subsection:causal_rep}).

\textbf{Subsampling:} In real-world applications, temporal data, especially time series, may be sampled at a rate lower than the rate of the underlying causal process due to the difficulties in data collection.
An ordinary causal discovery algorithm for sub-sampled time series may lead to spurious causal relations and missed ones. 
Several remarks and approaches~\cite{Discussion/subsample/work1, Discussion/subsample/work2_icml_GongZSTG15, Discussion/subsample/work3_nips_rateagnostic_PlisDFC15, Discussion/subsample/uai_subsample_aggr_GongZSGT17, Discussion/subsample/work5_pgm_constraintOPT_HyttinenPJED16, Discussion/subsample/biometrika/tank2019identifiability} are proposed for this issue.

\textbf{Expert knowledge: }Expert knowledge can help the causal discovery process in practice.  % 要强调practical issues.
The approaches of fusing expert knowledge can be categorized into three types~\cite{intro/nonts_surveys/BN21}: (1) \textit{Soft constraints}: the learning process can be influenced by the knowledge~\cite{Discussion/knowledge/ausai/ODonnellNHKAH06}. % (\ie, conditions given with a probability $0<p<1$).
(2) \textit{Hard constraints}: the learnt structure must conform to the enforced requirements (\ie, conditions given with a probability $p=0$ or $p=1$). 
In~\cite{Discussion/knowledge/artmed/AsvatourianLML20}, hard constraints are leveraged in structure learning with a time dependant exposure.
Studies in~\cite{MTS/SB/NTS_NOTEARS} add prior knowledge forbidding the existence of intra-slice dependencies, which is helpful to recover edges that are not explicitly encoded by the prior knowledge.
(3) \textit{Interactive learning}: the human input is leveraged in the learning process~\cite{Discussion/knowledge/ecsqaru/MessaoudLA09, Discussion/knowledge/kdd/MelkasSCMNMP21,https://doi.org/10.48550/arxiv.2206.05420, 9222294}.








\subsection{New perspectives}


\subsubsection{Extension in amortized and supervised paradigms}


In the traditional paradigms, causal discovery methods mostly either treat observational data separately or train a distinct model for each individual. 
These methods do not make full use of the common structure across different samples or supervised information from the datasets whose causal structures are clearly explored, thus suffering from several issues such as the small sample challenge and lack the inductive capability.
Recently, causal discovery is conducted in new paradigms to solve this problem. We can roughly categorize them into two groups: methods based on \textbf{amortized modeling} \cite{MTS/Attention/icdm_InGRA_ChuWMJZY20, Discussion/NewForm/ACD_LoweMSW22}, and methods based on \textbf{supervised learning} \cite{benozzo2017supervised, wang2022meta}.
We introduce them in this subsection, which we believe are a promising avenue for future research. 


In amortized modeling, a global causal discovery framework is trained for individuals with different causal structures. 
As for scenarios with temporal data, these approaches have been detailed in \ref{subsection:NN_Granger} as the deep learning extension of Granger causality with inductive modeling.
InGRA \cite{MTS/Attention/icdm_InGRA_ChuWMJZY20} leverages prototype learning to extract common causal structure while ACD \cite{Discussion/NewForm/ACD_LoweMSW22} proposes an encoder-decoder framework to conduct amortized causal discovery. These methods make full use of information from massive samples and are able to infer causal relations for newly arrived individuals, which are useful in real-world applications such as e-commerce, social network, and neuroimages.

Another line of work has predominately focused on treating the inference process as a black box and learning the mapping from sample data to causal graph structures via supervised learning. Here the label information is causal structure and can be easily accessed in synthetic datasets. 
Earlier work \cite{Discussion/NewForm/RCC/jmlr/Lopez-PazMR15, DBLP:conf/aaai/TonSF21} on learning causal relations by supervised learning is restricted to learning pairwise causal direction where the problem is cast into a classification task to distinguish between $X \to Y$ and $Y \to X$ by using observed samples.
It's later extended to discovery graph structure in \cite{Discussion/NewForm/DAG_EQ/corr/abs-2006-04697,petersen2022causal}.
As the labeled information for training is often originated from synthetic data or real-world datasets which have been explored, the requirement of a supervised approach, in which the distributions of training and test data match or highly overlap, is not guaranteed. In \cite{Discussion/NewForm/ML4S/kdd/00040DJWH022, Discussion/NewForm/CSIvA_DeepMind}, methods such as vicinal graph and meta-learning are leveraged in supervised causal discovery to tackle this `domain shift' issue.  
For the temporal setting, a supervised estimation of Granger causality between time series is proposed in \cite{benozzo2017supervised}. As a recent advance, a method for learning causal discovery is proposed in \cite{wang2022meta} where the learned from large datasets with known causal relations outperform the algorithm in the traditional paradigm when testing on temporal datasets such as fMRI. 
% It's also noted in \cite{wang2022meta} that the causal discovery algorithms in traditional paradigm depart from strong human assumptions about causality. In these approaches (such as constraint-based, score-based and Granger causality), human intuition is implemented in different form. 




% \subsubsection{Extension causal discovery towards causal representation learning (to edit)}
\subsubsection{Extension in causal representation learning}
\label{subsection:causal_rep}
% \subsection{Nonlinear ICA, causal representation learning...}

Extracting the causes of particular phenomena whether explicitly or implicitly from a deep learning black box can be beneficial to the downstream tasks.
The aforementioned causal discovery methods focus on inferring relations between observed variables, or start from the premise that the causal variables are given before hand.
Although some approaches learn causal relations under unobserved variables.
There exist real-world observations (e.g., sensor measurements, image pixels in video) which are not well structured to causal variables to begin with. 
As a generalization of causal discovery from observed variables, there has recently been a growing interest in \textbf{causal representation learning} \cite{CausalRepresentation/nontemp/icml/LocatelloPRSBT20, CausalRepresentation/nontemp/towardsCRL/ScholkopfLBKKGB21, CausalRepresentation/nontemp/CausalVAE/YangLCSHW21}, which aims at learning representation of causal factors in an underlying system.
It estimates latent causal variable graphs from observations.




A line of works in causal representation learning identifies independent factors of variations based on disentanglement and Independent Component Analysis (ICA).
At the heart of this methodology is the postulation of mutually independent latent factors.
It's hard to identify true latent variables, especially in general nonlinear cases.
As a remedy, recent approaches \cite{CausalRepresentation/nontemp/icml/LocatelloPRSBT20, CausalRepresentation/iVAE_nontemp/aistats/KhemakhemKMH20, DBLP:conf/aistats/HyvarinenM17, DBLP:conf/nips/HyvarinenM16} leverage additional information in multiple views, auxiliary variables, or temporal structure, combined with deep learning methods like VAEs and contrastive learning.
A connection between ICA and causality has been recently drawn in \cite{CausalRepresentation/IMA/nontemp/nips/GreseleKSSB21, DBLP:conf/uai/Monti0H19}.
In the context of temporal data, the identifiability of causal variables from temporal sequences is discussed in latent temporal causal process estimation (\textbf{LEAP}) \cite{Discussion/latent/iclr_LEAP_YaoSHS022}. It first provides causal identifiability conditions in a nonparametric, nonstationary setting, and a parametric setting. Then it proposes a learning framework to extract latent causal relations, which extends VAE with a learned causal process network by enforcing the assumed conditions.
The non-stationary noise, modeled by flow-based estimators, can be viewed as a soft intervention to aid identification.
In line with LEAP, subsequent works \cite{TDRL_DBLP:journals/corr/abs-2210-13647} extend the identification theory to a more general case.   % Change to NIPS form citation




Another line of work leverage intervention and data augmentation to help to identify latent causal relations. Under data augmentation, it's demonstrated in \cite{CausalRepresentation/line2/nips/KugelgenSGBSBL21} that common contrastive learning methods can block-identify causal variables that remain unchanged. 
For the temporal setting, \textbf{CITRIS} \cite{CausalRepresentation/CITRIS/icml/LippeMLACG22} is proposed. It's a VAE framework learning causal representation where latent causal factors have possibly been interved on.
By using intervention target information for identification, CITRIS is devoid of suffering from functional or distributional form constraints.
Besides, causal factors in CITRIS are considered as either scalars or potentially multidimensional vectors, which is more practical in complex scenarios. Along this line of work, instantaneous causal relations are extracted in iCITRIS \cite{CausalRepresentation/interv/iCITRIS/abs-2206-06169}.






















Our work builds on existing methods from several fields but is, to our knowledge, the first work focused specifically on fully automatic animation of children's drawings of human figures. 
To ground the work, we provide a summary of salient observations from the field of children's art analysis.
In addition, we briefly review related work on 2D image-to-animation and object and pose estimation for abstract images. 


\subsection{Analysis of Children's Drawings}

\hjs{
Children's drawings have long been of interest to the scientific community.
For well over a century, researchers from multiple fields have 
collected\,\cite{IndianaS55:online,kellogg1967rhoda,AWebbasedDatabaseforDrawingsofGods,geist2002they}
and analyzed them, seeking to understand and measure children's thought processes\,\cite{sully2021studies,barnes1892study,clark1897child,buhler2013mental}, 
intellectual development\,\cite{goodenough1926measurement},
and perceptions\,\cite{chambers1983stereotypic,doi:10.1080/01443410500344167}.
}
Particular attention has been given to drawings of human figures, one of the first and most frequently drawn subjects throughout childhood\,\cite{cox2013children}.

As the child develops, the schemas they employ to represent the human form become more complete (see Figure \ref{fig:tadpole-transitional-conventional}).
Even within these schemas, there is significant variation.
In addition to asymmetries and variation in color and proportion, many body parts appear optional to include; a study of drawings by 4-6 year old children showed that, while heads, legs, and eyes are almost universally present, other body parts (including torsos, arms, hands, and feet) were frequently absent\,\cite{cox2013children}.
Inversely, non-human body parts are frequently added in order to represent other subject classes\,\cite{kellogg1969analyzing}. With the addition of large ears, the figure may represent a cat or bear (Figures \ref{fig:maskrcnn_before_after}.m and \ref{fig:maskrcnn_before_after}.g); with the addition of a crown, it can represent a pineapple (Figure \ref{fig:maskrcnn_before_after}.n).
All of these sources of character variation make automatic character animation from drawings a non-trivial task.

\begin{figure}
\includegraphics[width=\linewidth]{images/tadpole-transition-conventional.png}
\caption{
As children learn to draw the human figure, the morphologies of the schemas they employ vary and evolve considerably\,\cite{cox2014drawings}.
Children frequently begin by drawing a \textit{tadpole figure}, a circular head region from which arms and legs extend. 
Some will progress to a \textit{transitional figure}, dropping the arms down so they extend from the legs. 
When a line is drawn between the legs, creating the separate torso region, the \textit{conventional figure} is formed.
Though these are small changes from the perspective of the drawer, they result in significantly different character morphologies when viewed through the lens of character animation.
A successful drawing-to-animation system must be robust to these types of variations.}
\label{fig:tadpole-transitional-conventional}
\end{figure}

Many researchers have focused closely on the unique style of children's drawings.
The psychologist and artist John Willats argues that, in order to understand the style of children's drawings, one must understand that the primary picture primitives employed by children are \textit{regions}, or 2D areas\,\cite{willats2006making}.
A squat volume, such as a head or torso, may be represented by a circular or ellipsoid region, whereas an elongated volume, such as a leg, may be represented by a long, thin region or even a single line.
These regions are not depictions of the object from any particular point of view. 
Rather, they are \textit{3D volumetric object-centered descriptions}\,\cite{marr1982vision},
2D areas with attributes perceptually similar to those of 3D object they are meant to represent.
%The regions begin as circles and lines, but later become modified to better reflect the perceptually impactful aspects of the objects they represent; a region representing a sugar cube or die may be given square corners, and a long region representing an arm may be given a bend to depict the elbow or split at the end to represent fingers (CITE Willats, 2005).

There are two stylistic outcomes of these \textit{object-centered descriptions} that bear mention.
First, the use of foreshortening is very rare in children's drawings \,\cite{piaget1956, willats1992representation}. 
This design choice is understandable; foreshortening a long region, such as a limb, results in a short region which does not adequately reflect the \textit{longness} of the object.
Second, the human figure may appear to have been drawn from many different perspectives, so as to make each part of the character maximally recognizable.
For example, the head and torso may face forward while the legs and feet are pointed to the side.
This technique, often referred to as \textit{twisted perspective}, is frequently seen and well-documented\,\cite{dziurawiec1992twisted}.
Both of these stylistic aspects are used to guide the design decisions of our system when applying human motion capture data onto the character.


\subsection{2D Image to Animation}

Previous researchers have proposed methods to animate drawings or photographs, many of which rely upon additional modes of user input.
Hornung et al. present a method to animate a 2D character in a photograph, given user-annotated joint locations\,\cite{Hornung2007anim2Dpicmotion}.
Pan and Zhang demonstrate a method to animate 2D characters with user-annotated joint locations via a variable-length needle model\,\cite{Pan2011}.
Jain et al. present an integrated approach to generate 3D proxies for animation given joint locations, segmentation masks, and per-part bounding boxes\,\cite{jain:2012}. 
Levi and Gotsman provide a method to create an articulated 3D object from a set of annotated 2D images and an initial 3D skeletal pose\,\cite{ArtiSketch}.
\textit{Live Sketch}\,\cite{su2018livesketch}
tracks control points from a video and applies their motion to user-specified control points upon a character.
Other approaches allow the user to specify character motions through a puppeteer interface, using RGB or RGB-D cameras\,\cite{held20123d,barnes2008video}.
\textit{ToonCap}\,\cite{Fan:2018:TAL} focuses on an inverse problem, capturing poses of a known cartoon character, given a previous image of the character annotated with layers, joints, and handles. 


\textit{Toonsynth}\,\cite{Dvoroznak18-SIG} and \textit{Neural Puppet}\,\cite{poursaeed2020neural} both present methods to synthesize animations of hand-drawn characters given a small set of drawings of the character in specified poses.
Hinz et al. train a network to generate new animation frames of a single character given 8-15 training images with user-specified keypoint annotations\,\cite{hinz2022charactergan}.

\textit{Monster Mash}\,\cite{Dvoroznak20-SA} presents an intuitive framework for sketch-based modeling and animation, and \textit{2.5D Cartoon Models}\,\cite{10.1145/1778765.1778796} presents a novel method of constructing 3D-like characters from a small number of 2D representations. 
Both of these are intuitive and well designed animation tools targeted towards amateur users.


\hjs{
Some animation methods are specifically tailored toward particular forms, such as faces\,\cite{elor2017bringingPortraits}, coloring book characters\,\cite{magnenat2015live}, or characters with human-like proportions. 
One notable work that is focused on the human form is \textit{Photo Wake Up}\,\cite{weng2019photo}. 
The authors show a method for creating a rigged and textured 3D mesh from a single image of a human-like figure.
Similar to us, their end goal is to allow users to seamlessly bring 2D characters to life; their work does an impressive job of accomplishing this.
Our method differs in two significant ways. 
First, while their work is focused on creating a 3D model for a mixed reality use case, 
ours is specifically focused on animating twisted perspective figures while staying within a 2D plane.
Second, children's drawings are much more abstract, incorrectly proportioned, and non human-like than the examples demonstrated in the paper.
We test our method upon the more abstract examples demonstrated in their paper and, with minor segmentation cleaning, they were successfully animated by our method.
}












\hjs{While the approaches listed here are wonderful tools to ease the burden of animation, none were perfectly suited to our use case.
Some require additional user input beyond the drawing itself, making the animation process more complex.
Others require the user to consistently draw the same character in multiple poses, which is beyond the skills of young children.
Others are focused on animating specific forms, precluding their use on children's drawings of the human figure.}


%Siarohin and colleagues propose a method for animating arbitrary classes of subjects,
%but require training videos of class members moving\,\cite{Siarohin_2019_NeurIPS}, making it unsuitable children's drawings.


\subsection{Detection, Segmentation, and Pose Estimation on Non-Photorealistic Images}

\hjs{
Aided by the the existence of large annotated datasets\,\cite{lin2014microsoft,6909866,6682899}, researchers have made considerable progress solving the problems of object detection, segmentation, and pose estimation from photographs. See, for example\,\cite{MaskRCNNhe2017mask,openpose19,guler2018densepose,alphapose,toshev2014deeppose}.
We explain the methods in this area that we leverage in Sections \ref{sec:character_detection} and \ref{sec:joint_detection}.

While traditional methods for detection, segmentation, and pose estimation of non-photorealistic images exist\,\cite{choi2012retrieval,bregler2002turning,davis2006sketching,eitz2012humans}, the lack of easily available datasets has resulted in slower adoption of deep learning models.
Some researchers are addressing this problem by developing methods and releasing datasets focused on the domain of anime characters\,\cite{chen2022bizarre,10.1145/3011549.3011552}, professional sketches\,\cite{brodt2022sketch2pose}, and mouse doodles\,\cite{ha2017neural}.
Other researchers have presented a non-deep learning method for inferring character poses from \textit{gesture drawings}\,\cite{Gesture3D}.
}
Because the Amateur Drawings Dataset is comprised of in-the-wild photographs of drawings created by the general public, we believe it will complement the value of existing datasets and allow for new dimensions of exploration and analysis.

%\section{}
%\label{sec:resDir}


\section{Conclusion}
\label{sec:conclusion}
% <>
Since its advent in 1931, Koopman operator theory \cite{koopman:1931} has only recently been actively utilized for solving practical problems, thanks to the introduction of the DMD algorithm in 2008 \cite{schmid:2008}. Since then, a multitude of DMD algorithm variations have risen to prominence and found utility across various fields. A notable feature of our survey paper was reviewing and categorizing the results of over 100 research papers based on both application and algorithm type in smart mobility and vehicle engineering  (see Table~\ref{tab1} and Section~\ref{sec:vehicApp}).  Additionally, this survey paper identified potential research gaps in smart mobility and vehicular engineering applications (Remarks~\ref{remGap1}--\ref{remGap6}). Finally, this review paper discussed theoretical aspects of Koopman operator theory that have been largely neglected by the smart mobility and vehicle engineering community and yet have large potential for contributing to solving open problems in these areas (see Section~\ref{subsec:theorIssue}).

\noindent{\textbf{Future Research Directions.}}	Given the emergence of cyber-threats against connected and autonomous vehicles as well as robotic systems (see, e.g.,~\cite{nekouei2021randomized,mohammadi2022generation}), a future research direction might include utilizing Koopman operator-based algorithms for designing cyber-resilient vehicular and smart mobility applications (see, e.g.,~\cite{taheri2022data} for a related line of research). Another potential research direction is using Koopman operator-based algorithms for predicting the motion of vulnerable road users (VRUs), e.g., pedestrians and cyclists (see, e.g.,~\cite{pool2019context,scholler2020constant}). Finally, rehabilitation robotics and robotic exoskeletons can be the benefactors of the predictive capabilities of Koopman operator-based algorithms for detecting tripping events and/or system  identification in various modes of locomotion (see, e.g.,~\cite{kumar2019extremum,aprigliano2019pre}).



%Fig. 1 depicts the accumulation of such algorithms since 2014, which are particular to vehicle engineering and smart mobility, i.e., the focus of this review. Table 1 summarizes the varieties of relevant algorithms developed in those studies. Furthermore, we have highlighted theoretical issues, whose expansion will have potential applications to the wide research area of smart mobility and vehicle engineering.  

%Although fairly comprehensive, we have found several gaps in this research area. In particular, we could not find any studies related to elevators, robots/vehicles employing crawling, slithering, hopping or peristaltic locomotion, arctic or special-terrain vehicles such as those employing screws or tracks, hovercraft and other amphibious vehicles or subsystems which tolerate flexible environments, classification or guidance systems related to vehicles for drilling or agriculture, or for current-ripple, power-split, battery health monitoring, nuclear propulsion, exoskeletons/prosthetics, personal mobility, motorsports, specialized rovers or similar open problems in emerging areas.  These examples are, of course, not exhaustive.  
%
%The purely data-driven nature of Koopman operators holds the promise of capturing unknown and complex dynamics for reduced-order model generation and system identification, through which the rich machinery of linear control techniques can be utilized. The emergent nature of the smart mobility and vehicular-related applications, where  the Koopman operator  in each particular application needs to be approximated, implies that the development of various Koopman operator approximation  algorithms is expected to grow along with the vehicular problems they aim to solve.  Given the ongoing development of this research area and the many existing open problems in the fields of smart mobility and vehicle engineering, a survey of techniques and open challenges of applying Koopman operator theory to this vibrant area is warranted.  To the best of our knowledge, this survey paper is the \emph{first of its kind} reviewing the applications of Koopman operator theory within a focused research area, namely, smart mobility and vehicle engineering applications. A \emph{notable feature} of our survey paper is reviewing and categorizing the results of over 100 research papers based on both application and algorithm type  (see Tables~\ref{tab1}--~\ref{tab4} and Section~\ref{sec:vehicApp}) that are concerned with the applications of Koopman operator theory to the field of smart mobility and vehicular engineering. Such a \emph{comprehensive and  detailed categorization} will be beneficial to the research practitioners working in the field.  Furthermore, this review paper discusses theoretical aspects of Koopman operator theory that have been largely neglected by the smart mobility and vehicle engineering community and yet have large potential for contributing to solving open problems in these areas. Additionally, our survey paper seeks to \emph{identify gaps} in the smart mobility and vehicle engineering research where new and existing Koopman operator-based methods have the potential to further develop and address unsolved problems  potentially benefiting from the perspectives of nonlinear system identification, control, global linearization, and the predictive powers that Koopman operator theory has to offer (see, e.g., Remarks~\ref{remGap1}--\ref{remGap6}). 


%-------------------------------------------------------------------------------
% \section*{Acknowledgments}
% %-------------------------------------------------------------------------------

% The USENIX latex style is old and very tired, which is why
% there's no \textbackslash{}acks command for you to use when
% acknowledging. Sorry.


%-------------------------------------------------------------------------------
\bibliographystyle{plain}
\bibliography{sample-base}
\appendix
\appendix
\section{Skew Equations}
We will justify and show the three equations used in Lemma \ref{skew rel} to narrow our search for these skew axial algebras. Although they do not provide much use to understanding how these algebras could be constructed, they do make the proof easier.

Suppose $v$ is an $\mu$-eigenvector of an axis, $x$, where $\mu\neq1$. Then the projection on that axis should be equal to 0; that is, $\lm_x(v)=0$. Coincidentally, nearly all of the eigenvectors in Lemma \ref{eigen a} and \ref{eigen b} satisfy that rule. However we have
\begin{equation*}
 0=\lm_b\left(-\frac{P}{\bt}a+Pb+c\right) = -\frac{P}{\bt}\lmf_1+P+\lmf_2.
\end{equation*}
Whence we get Equation (\ref{proof1}).

\begin{defn}
Let $x$ be a $\mon{\al,\bt}$-axis in $A$, $\lm\in \{1,0, \al, \bt\}$ and $v\in A$. We denote $[v]^x_\lm$ to be the component of $v$ in $ A_\lm(x)$. 
\end{defn}
\begin{lem}
Let $w:=\frac{1}{2}(b-c)$. We have $[a]^a_\bt=0$, $[b]^a_\bt=w$, $[c]^a_\bt=-w$, $[\sg]^a_\bt=0$. Further, $[ab]^a_\bt=\bt w$, $[ac]^a_\bt=-\bt w$, $[bc]^a_\bt=0$, $[a\sg]^a_\bt=0$, $[b\sg]^a_\bt=\dt^fw$, $[c\sg]^a_\bt=-\dt^fw$ and $[\sg^2]^a_\bt=0$.
\end{lem}
\proof
As $a\in A_1(a)$, it has no $\bt$-component in $A_\bt(a)$ and $[a]^a_\bt=0$. As $\sg\in A_{\{1,0,\al\}}(a)$, it has no $\bt$-component in $A_\bt(a)$ and $[\sg]^a_\bt=0$. We can express $b$ in terms of the eigenvectors of $\text{ad}_a$ in Lemma \ref{eigen a}. The reader can check
\[ b= \lm_1 a+ \frac{1}{\al}\left(\ep a+\frac{1}{2}(\al-\bt)(b+c)-\sg\right)+ \frac{1}{\al}\left(\gm a +\frac{1}{2}\bt(b+c)+\sg\right)+\frac{1}{2}(b-c).\]
Thus $[b]_\bt^a=w$. As $c=b^{\tu{a}}$, we get $[c]_\bt^a=-w$.

Let $x, y \in A_{\{0,1,\al\}}(a)$ and notice $x^2, xy\in A_{\{1,0,\al\}}(a)$ and so has no $\bt$-component in $A_\bt(a)$. Therefore $[\sg^2]^a_\bt=[a\sg]^a_\bt=0$. Also
\[ [bc]_\bt^a=P\left([a]_\bt^a+\frac{1}{\bt}[\sg]_\bt^a\right)=0.\]
Note that
\[ [ab]_\bt^a=[\sg]_\bt^a+\bt[a]_\bt^a+\bt[b]_\bt^a=\bt w\]
and 
\[ [b\sg]_\bt^a=(\al-\bt)[\sg]_\bt^a+\bt(\al-\bt)[a]_\bt^a+dt^f[b]_\bt^a=\dt^f w.\]
Applying $\tu{a}$, we get $[ac]_\bt^a$ and $[c\sg]_\bt^a$. \qed



Let $u:= (b -\al)a - \bt b=\sg -(\al-\bt)a$. As $A_\bt(b)=\{0\}$, we have that $u\in A_{\{1,0\}}(b)$. By Lemma \ref{Seress}, the following holds
\[b(au)=(ba)u.\]
Notice
\[ au = a(\sg -(\al-\bt)a)=(\dt -(\al-\bt))a+\frac{1}{2}\bt(\al-\bt)(b+c)+(\al-\bt)\sg\]
and so
\begin{eqnarray*}
[b(au)]_\bt^a &=& (\dt -(\al-\bt))[ab]_\bt^a+\frac{1}{2}\bt(\al-\bt)([b]_\bt^a+[bc]_\bt^a)+(\al-\bt)[b\sg]_\bt^a\\
& =& \left(\bt(\dt -(\al-\bt))+\frac{1}{2}\bt(\al-\bt)+(\al-\bt)\dt^f\right)w
\end{eqnarray*}
We also have 
\begin{eqnarray*}
[(ba)u]_\bt^a&=&[(\sg+\bt a +\bt b)(\sg -(\al-\bt)a)]_\bt^a\\
&=& [\sg^2]_\bt^a -(\al-2\bt)[a\sg]_\bt^a +\bt [b\sg]_\bt^a -\bt(\al-\bt)[a]_\bt^a - \bt(\al-\bt)[ab]_\bt^a\\
&=& (\bt\dt^f -\bt^2(\al-\bt)) w
\end{eqnarray*}
By Lemma \ref{Seress}, we have $0=(ba)u-b(au)$ moreover $0=[(ba)u]_\bt-[b(au)]_\bt$. Looking at the coefficient of $w$, we have
\begin{eqnarray*} 
0&=& (\bt\dt^f-\bt^2(\al-\bt))\\
& -& \left(\bt \dt -\bt(\al-\bt)+\frac{1}{2}\bt(\al-\bt)+(\al-\bt)\dt^f\right)\\
&=&-\bt^2(\al-\bt) -\bt\dt+\frac{1}{2}\bt(\al-\bt)-(\al-2\bt)\dt^f.
\end{eqnarray*}
Rearranging we get Equation (\ref{proof2}).

Let $v:=Pa+\frac{P}{\bt}\sg -\al c=c(b-\al)$. Notice that $v \in A_{\{1,0\}}(b)$. Again by Lemma \ref{Seress}, the following holds
\[b(av)=(ba)v.\]
We have
\begin{eqnarray*}
av &=& Pa +\frac{P}{\bt}\left(\dt a + \frac{1}{2}\bt(\al-\bt)(b+c) +(\al-\bt)\sg\right)\\
& -&\al(\bt a +\bt c +\sg)\\
&=&\left(P +\frac{P}{\bt}\dt -\al\bt\right)a+\left(\frac{1}{2}(\al-\bt)P\right)b\\
&+&\left(\frac{1}{2}(\al-\bt)P-\al\bt\right)c+\left(\frac{P}{\bt}(\al-\bt)-\al\right)\sg.
\end{eqnarray*}
Therefore
\begin{eqnarray*}
[b(av)]_\bt^a &=&\left(P +\frac{P}{\bt}\dt -\al\bt\right)[ab]_\bt^a+\left(\frac{1}{2}(\al-\bt)P\right)[b]_\bt^a\\
&+&\left(\frac{1}{2}(\al-\bt)P-\al\bt\right)[bc]_\bt^a+\left(\frac{P}{\bt}(\al-\bt)-\al\right)[b\sg]_\bt^a.\\
&=&\left(\bt \left(P +\frac{P}{\bt}\dt -\al\bt\right)+\dt^f\left(\frac{P}{\bt}(\al-\bt)-\al\right)\right)w
\end{eqnarray*}
We also have
\begin{eqnarray*}
[(ba)v]_\bt^a&=&\left[\left(\bt a +\bt b +\sg\right)\left(Pa+\frac{P}{\bt}\sg -\al c\right)\right]_\bt^a\\
&=&2P[a\sg]_\bt^a +\frac{P}{\bt}[\sg^2]_\bt^a -\al [c \sg]_\bt^a + \bt P [a]_\bt^a -\al\bt [ac]_\bt^a\\
&+&\bt P [ab]_\bt^a +P[b\sg]_\bt^a -\al\bt [bc]_\bt^a\\
&=&\left(\al \dt^f +\al\bt^2 +\bt^2 P  +P\dt^f\right)w
\end{eqnarray*}
By Lemma \ref{Seress}, $0=[b(av)]^a_\bt-[(ba)v]^a_\bt$ and looking at the coefficient of $w$, we get 
\begin{eqnarray*}
0&=&[b(av)]_\bt-[(ba)v]_\bt\\
&=&\left(\bt P +\dt P -\al\bt^2+\frac{1}{2}(\al-\bt)P+\frac{P}{\bt}(\al-\bt)\dt^f -\al\dt^f\right)\\
&-&\left(\bt^2P +P\dt^f+\al\dt^f +\al\bt^2 \right)\\ 
&=&\left(\frac{P}{\bt}\left[\bt^2 +\bt\dt+\frac{1}{2}\bt(\al-\bt)+(\al-2\bt)\dt^f-\bt^3\right]-2\al(\dt^f+\bt^2)\right).
\end{eqnarray*}
From Equation (\ref{proof2}), we get that
\begin{eqnarray*}
0&=&\frac{P}{\bt}\left[\bt^2 -\bt^2(\al-\bt) +\frac{1}{2}\bt(\al-\bt)-(\al-2\bt)\dt^f\right.\\
&+&\left.\frac{1}{2}\bt(\al-\bt)+(\al-2\bt)\dt^f-\bt^3\right]-2\al(\dt^f+\bt^2)\\
&=&\frac{P}{\bt}\left[\bt^2 -\bt^2(\al-\bt) +\bt(\al-\bt)-\bt^3\right]-2\al(\dt^f+\bt^2)\\
&=&\frac{P}{\bt}\al\bt\left[1-\bt\right]-2\al(\dt^f+\bt^2).
\end{eqnarray*}
Hence we get Equation (\ref{proof3}).

\section*{Acknowledgements}
I would like to thank Professor Sergey Shpectorov for his guidance throughout my PhD studies so far and pushing me to complete this paper. I would also like to thank my family for their continuing support. 
%\bibliography{\jobname}

%%%%%%%%%%%%%%%%%%%%%%%%%%%%%%%%%%%%%%%%%%%%%%%%%%%%%%%%%%%%%%%%%%%%%%%%%%%%%%%%
\end{document}
%%%%%%%%%%%%%%%%%%%%%%%%%%%%%%%%%%%%%%%%%%%%%%%%%%%%%%%%%%%%%%%%%%%%%%%%%%%%%%%%

%%  LocalWords:  endnotes includegraphics fread ptr nobj noindent
%%  LocalWords:  pdflatex acks
