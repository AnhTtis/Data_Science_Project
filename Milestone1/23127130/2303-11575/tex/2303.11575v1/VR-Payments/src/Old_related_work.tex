

Different types of user authentication methods were proposed for mobile devices. 
SwipePass~\cite{chen2022swipepass} implemented an acoustic-based authentication method based on the modulated audio which is emitted and received by smartphone's build-in modules. It extracted the audio during the user performing the Android pattern to unlock their devices. AirAuth~\cite{aumi2014airauth} utilize depth sensor to collect user's in-air gesture for authentication purpose. Burgbacher and Hinrichs~\cite{burgbacher2014implicit} proposed an passive continuous authentication method by tracking user's single continuous stroke on touchscreen device's virtual keyboard to verify users' identities. Magauth~\cite{zhang2021magauth} use the magnetic field changes from user's wrist wearables with magnetic strap bands as a second factor along with Android unlock patterns to authenticate users. Garda~\cite{liu2017usability} used free-form gestures as password to authenticate users and proofed the gesture passwords can resist brute-force, shoulder-surfing, and dictionary attacks.


As virtual reality and augment reality becomes popular, there are already several works designed authentication method for the VR/AR systems. RubikAuth~\cite{mathis2021fast} authenticates users by asking them selecting digits from a virtual 3D cube with different controlling methods, like eye gaze, head pose, and controller tapping. Oculock~\cite{luo2020oculock} utilized signals of electrooculography based on HVS sensing framework to authenticate users by measuring the signals similarity by viewing the same scenario. ElectricAuth~\cite{chen2021user} was designed based on Electrical Muscle Stimulation (EMS), which is a biometrics system for interactive applications. This system stimulates the users' forearm muscles with EMS-based challenge and measures the user's involuntary movements as sequential signal for authentication purpose. Zhang et al.~\cite{zhang2018continuous} proposed a continuous passive authentication system based on tracking users' eye movement when wearing VR headset. Blinkey~\cite{zhu2020blinkey} combined knowledge-based and biometric features for authentication. It designs a user's passcode as a set of his/her blinks' rhythms and the pupil size of the user. 