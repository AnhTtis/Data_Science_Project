\section{Discussion}
% \TODOS{~2 page, dont have to fill it for getting approval}
Our findings yield recommendations to guide the design of future VR authentication by resolving the uncertainties in VR environments, offering transparent authentication interfaces, and identifying the responsibility to implement such improvements.  

\paragraph{Summary of findings.} 
In response to \textbf{RQ1}, we observe participants' interaction experiences with VR authentication depended on multiple factors, including the components of authentication interaction, usability characteristics of authentication, and the context of authentication (e.g., routine payment in a VR game). These factors helped participants adapt their real-world authentication experiences into VR. However, VR posed unique usability challenges to participants' motion control and spatial awareness when they performed authentication.

In response to \textbf{RQ2}, we find that realistic interactions in VR authentication helped users translate their real-world perception of authentication into VR. However, due to usability issues in VR, participants faced difficulties assessing their security and privacy concerns. They lost the sense of signing their signatures, lacked  knowledge of VR authentication’s backend, and exhibited uncertainties about their virtual surroundings. In addition, the gamified VR context sometimes reduced participants' sensitivity to security and privacy.

In response to \textbf{RQ3}, we identify participants’ expectations and requirements for VR authentication, which centered around improving the interaction factors. For example, participants demanded flexible and engaging interfaces for authentication. However, their expectations can sometimes be nonspecific and even appear conflicting, e.g., the need for receiving feedback and a reduced interaction burden.



% \begin{itemize}
%     \item Authentication interaction as a proxy for users to understand security by engaging them stimulation
    
%     \item Interface can also be dececiving 
    
%     \item how do security services bridge vr and real world -- persistent trust, more than just interface, realism can be an illusion
% \end{itemize}

% \TODOS{added} Participants willing to learn how authentication works in the Metaverse to understand and evaluate the security of those authentication methods. It motivates authentication algorithm designers to illustrate the idea of novel authentication method to users so that users are move confident to adopt those authentication methods in Metaverse. It does not only apply to novel authentication methods like sign-to-pay, but also apply to matured authentication methods in physical world like tap-to-pay. 


\paragraph{Resolving the perceptual uncertainties and ambiguity for VR authentication.} 
The lack of awareness in a VR environment, due to usability challenges or deception, prevents users from adequately assessing security and privacy. Therefore, we provide the following recommendations to resolve the uncertainties from the virtual world, the physical environment, and the authentication interface itself.


First, the VR platform could employ \textit{\textbf{access control to actively safeguard users' virtual assets}}.
The VR platform could control others’ access to one user’s scene when they perform authentication. 
It can also monitor the VR scenes to detect and mitigate potential threats, e.g., an over-privileged VR recorder or a hidden phishing interface~\cite{ruth-usenix19}.

Next, the VR platform could \textit{ \textbf{make users aware of their physical environment, e.g., bystanders}}. The VR experiences immerse users into the virtual world so that users lack the awareness of physical surroundings, leading to safety and security issues, such as shoulder-surfing. While the VR platform could notify users of the bystanders~\cite{kudo-hci21}, VR designers should avoid breaking the immersion or inducing security fatigue. They may consider visualizing the bystanders with digital avatars that match the aesthetic style of the VR scene.

Third, the VR platform should \textit{ \textbf{not misrepresent the security properties of its authentication interfaces}}. Users might draw overestimate the security of VR authentication based on how much the interface look similar to their prior experiences. For example, while inserting a card into a reader looks similar, the technology to secure a VR token would be different than the EMV chip in a physical credit card~\cite{van-tr16}. Thus, the service provider should explain how the authentication’s backend processes work in an understandable manner.








\paragraph{Engaging users in understanding the security and privacy of VR through authentication interface designs.} 
Users have a limited and differing understanding of VR's inherent security and privacy properties. The security model of VR authentication can be more involved than the conventional platforms. For example, second factor authentication in VR can include virtual behavioral biometrics, e.g., how to perform tap a card, along virtual token-based authentication methods. As such, we provide suggestions to help users understand such properties through enjoyable VR interactions. We hope that the result is helping users better decide between usability, security, and privacy when using VR authentication. 

Authentication service providers could \textit{ \textbf{convey the security and privacy properties through engaging interfaces}}. Instead of displaying text messages, the service could convey these properties through the interface interaction of VR authentication in a more engaging way. Such designs can be enjoyable, thus motivating users to adopt them. For example, the interface could visualize the biometrics being used for authentication around users’ avatar~\cite{lee-chi22}. Or the designer could gamify the authentication experiences, e.g, showing a reward badge after users made effort in completing a multi-factor authentication~\cite{micallef-soups17}.

As users will use VR authentication for payments in contexts other than gaming, the interface \textit{ \textbf{design should match the context of payment}}, including its aesthetic style. Otherwise, the immersion experience might be broken leading to security fatigue. Moreover, the interface design of authentication could relay different levels of urgency according to its context, e.g., the amount of transaction, through color signals or haptic cues~\cite{alagarai-chi15}. 






%  To fulfill users’ security and privacy expectations, VR stakeholders should jointly establish trust.


\paragraph{Establishing persistent trust of authentication in VR via standardization.} 
Securing VR authentication involves multiple stakeholders, which might be problematic to the users. For example, users might not trust some stakeholders due to preconceptions. Here, we identify the open challenges and opportunities in establishing user trust in VR.


First, \textit{ \textbf{identifying the responsibility for various stakeholders in the VR and payment ecosystems}} is challenging. Multiple stakeholders, such as the hardware/software platform providers, app developers, and service providers for payment, will be involved to provide a secure authentication experience to users. It is an open challenge to identify the responsibility of each stakeholder, especially in decentralized VR~\cite{cannavo-cem20}.

Second, we see an opportunity in that established service providers could \textit{ \textbf{standardize VR authentication and provide transparency}}. For example, the Metaverse Standards Forum~\cite{msf-22} invites hundreds of service providers and academic partners to provide interoperability for VR in the metaverse. Such standardization can also include guidelines and open API for authentication in VR. Service providers may also leverage established infrastructure, e.g., social log-in and paired device, to provide persistent trust and transparency~\cite{stephenson-sp22}. 





\paragraph{Future research direction.} 
Our work opens up several directions for future research. First, researchers could study how users from different backgrounds, e.g., different age groups and technology expertise, perceive VR authentication~\cite{ratakonda-idc19}. Our study presents the findings from users that are tech-savvy and security-informed. 

Second, researchers may want to study VR authentication in different usage contexts. Besides routine payment in a VR game, VR payments could happen in different applications (e.g., shopping) and through various payment methods, including crypto-currencies. The infrastructure of the payment method, such as the blockchain, will affect users' perception of security and privacy. 

% Third, follow-up work could study how users react to different behaviors and interactions of the authentication system, e.g., in response to false negatives~\cite{missing}. Such investigation would allow the designers to optimize for deploying VR authentication in the wild.

Third, we encourage future research to investigate users’ longitudinal adoption and use of VR authentication through diary studies~\cite{hayashi-chi11}. Users’ security and privacy attitudes may vary over time as they interact with the system. Such longitudinal studies may allow researchers to capture more usability and security issues in the wild, e.g., how users react to suspicious activities and threats~\cite{downs-apwg07}. 

Last, we hope future work could engage users in the co-design process of VR authentication~\cite{yao-chi19}. In a participatory design study, experts and users could collaboratively design the frontend interaction and backend infrastructure of VR authentication systems.