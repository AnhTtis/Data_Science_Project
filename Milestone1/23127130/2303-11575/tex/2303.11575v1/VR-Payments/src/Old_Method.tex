\subsection{Participants and Recruitment}
We recruited 24 participants within a research department of our institution in the US.
Our study and recruitment received approval from the privacy and legal offices of our organization.
The study includes two sessions in different days.
The first is a training session to familiarize participants of our setup in VR; the second is the main study session.
The entire study takes about 60-80 minutes, and each participant received a 30 USD gift after their second session as compensation. 
The amount of compensation matches our local salary.
Our study is within-subject -- all participants experience the same designs but in a randomized order.

In total, 75\% (18 out of 24)  of the participants identified themselves as male indicating a male bias in our sample. Among the different age groups, the 25-29 year group dominated the total population with 75\% of the participants. Although many of our participants are in the younger age group, majority of our participants are well educated, where more than 80\% of them have a graduate degree. Majority of the participants have backgrounds in computer science, where they mainly focus on the AI and/or the Security field.
Three individuals indicated that they owned a VR and/or AR headset, while none of them reported that they frequently use them.
However, majority of the participants (18 out of 24) reported that they have used a VR headset in the past for gaming and entertainment. Some participants also reported that they have used VR for education, training, and socialization purposes as well. 


\begin{table}[]
\caption{Demographics Information}
\label{tab:demo}
\resizebox{\columnwidth}{!}{

\footnotesize{
\begin{tabular}{lll}
\hline
\multicolumn{1}{|l|}{\multirow{2}{*}{\textbf{Gender}}}      & \multicolumn{1}{l|}{{Male}}                                  & \multicolumn{1}{l|}{18} \\ %\cline{2-3} 
\multicolumn{1}{|l|}{}                                      & \multicolumn{1}{l|}{{Female}}                                & \multicolumn{1}{l|}{6}  \\ \hline
\multicolumn{1}{|l|}{\multirow{5}{*}{\textbf{Age}}}         & \multicolumn{1}{l|}{{25-29}}                                 & \multicolumn{1}{l|}{16} \\ %\cline{2-3} 
\multicolumn{1}{|l|}{}                                      & \multicolumn{1}{l|}{{30-34}}                                 & \multicolumn{1}{l|}{3}  \\ %\cline{2-3} 
\multicolumn{1}{|l|}{}                                      & \multicolumn{1}{l|}{{35-39}}                                 & \multicolumn{1}{l|}{3}  \\ %\cline{2-3} 
\multicolumn{1}{|l|}{}                                      & \multicolumn{1}{l|}{{40-45}}                                 & \multicolumn{1}{l|}{1}  \\ %\cline{2-3} 
\multicolumn{1}{|l|}{}                                      & \multicolumn{1}{l|}{{N/A}}                                   & \multicolumn{1}{l|}{1}  \\ \hline
\multicolumn{1}{|l|}{\multirow{3}{*}{\textbf{Education}}}   & \multicolumn{1}{l|}{{Bachlor's Degree}}                      & \multicolumn{1}{l|}{4}  \\ %\cline{2-3} 
\multicolumn{1}{|l|}{}                                      & \multicolumn{1}{l|}{{Graduate or professional degree}}       & \multicolumn{1}{l|}{6}  \\ %\cline{2-3} 
\multicolumn{1}{|l|}{}                                      & \multicolumn{1}{l|}{{PhD}}                                   & \multicolumn{1}{l|}{14} \\ \hline
\multicolumn{1}{|l|}{\multirow{3}{*}{\textbf{VR/AR Owned}}} & \multicolumn{1}{l|}{{Yes - VR}}                              & \multicolumn{1}{l|}{3}  \\ %\cline{2-3} 
\multicolumn{1}{|l|}{}                                      & \multicolumn{1}{l|}{{Yes - AR}}                              & \multicolumn{1}{l|}{1}  \\ %\cline{2-3} 
\multicolumn{1}{|l|}{}                                      & \multicolumn{1}{l|}{{No}}                                    & \multicolumn{1}{l|}{21} \\ \hline
\multicolumn{1}{|l|}{\multirow{6}{*}{\textbf{VR/AR Used}}}  & \multicolumn{1}{l|}{{Very Frequently}}                       & \multicolumn{1}{l|}{0/0}   \\ %\cline{2-3} 
\multicolumn{1}{|l|}{}                                      & \multicolumn{1}{l|}{{Frequently}}                            & \multicolumn{1}{l|}{0/0}   \\ %\cline{2-3} 
\multicolumn{1}{|l|}{}                                      & \multicolumn{1}{l|}{{Occasionally}}                          & \multicolumn{1}{l|}{0/0}   \\ %\cline{2-3} 
\multicolumn{1}{|l|}{}                                      & \multicolumn{1}{l|}{{Rarely}}                                & \multicolumn{1}{l|}{4/1}    \\ %\cline{2-3} 
\multicolumn{1}{|l|}{}                                      & \multicolumn{1}{l|}{{Very Rarely}}                           & \multicolumn{1}{l|}{16/10}    \\ %\cline{2-3} 
\multicolumn{1}{|l|}{}                                      & \multicolumn{1}{l|}{{Never}}                                 & \multicolumn{1}{l|}{6/13}    \\ \hline
\multicolumn{1}{|l|}{\multirow{7}{*}{\textbf{Apps used}}}   & \multicolumn{1}{l|}{{Gaming/Entertainment}}                  & \multicolumn{1}{l|}{17} \\ %\cline{2-3} 
\multicolumn{1}{|l|}{}                                      & \multicolumn{1}{l|}{{Socializing/Communication}}             & \multicolumn{1}{l|}{2}  \\ %\cline{2-3} 
\multicolumn{1}{|l|}{}                                      & \multicolumn{1}{l|}{{Education}}                             & \multicolumn{1}{l|}{2}  \\ %\cline{2-3} 
\multicolumn{1}{|l|}{}                                      & \multicolumn{1}{l|}{{Work related (Training, Productivity)}} & \multicolumn{1}{l|}{1}  \\ %\cline{2-3} 
\multicolumn{1}{|l|}{}                                      & \multicolumn{1}{l|}{{Finance}}                               & \multicolumn{1}{l|}{0}  \\ %\cline{2-3} 
\multicolumn{1}{|l|}{}                                      & \multicolumn{1}{l|}{{Other}}                                 & \multicolumn{1}{l|}{1}  \\ %\cline{2-3} 
\multicolumn{1}{|l|}{}                                      & \multicolumn{1}{l|}{{None}}                                  & \multicolumn{1}{l|}{2}  \\ \hline           
\end{tabular}}
}
\end{table}


To evaluate the tech-savviness of the participants we asked them to record their level of agreement with the nine statements shown in~\Cref{fig:tech_savvy}. As we can see, the participants clearly fall into the category of tech-savvy users. Majority of the users like to investigate new systems in more details and go through the technical details.


% \begin{figure}[h]
% \centering
% \includegraphics[width=\columnwidth]{charts/tech_savvy.png}
% \caption{Participants familiarity with technology}\label{fig:tech_savvy}
% \end{figure}


In~\Cref{tab:login} we see the self-reported methods that the participants have used for payment logins. Unsurprisingly, we see that all the participants report that they have used passwords as their mean of login for a payment. Furthermore, with the usages of face and fingerprint biometrics on the phones today we see that almost all the participants (except two participant) also report using biometrics as their login method. Other popular login methods for payments were Tokens, Paired Accounts, and Paired Devices.



\begin{table}[t!]
\caption{Login methods used}
\label{tab:login}

\resizebox{\columnwidth}{!}{
\footnotesize{
\begin{tabular}{|l|l|l|}
\hline
\textbf{Method}                                       & \textbf{\begin{tabular}[c]{@{}l@{}}Login for\\ Payment\end{tabular}} & \textbf{\begin{tabular}[c]{@{}l@{}}Login in \\ AR/VR\end{tabular}} \\ \hline
\textbf{Password}                                     & 24                                                                   & 4                                                                  \\ %\hline
\textbf{Unlock Pattern}                               & 9                                                                    & 0                                                                  \\ %\hline
\textbf{Signature}                                    & 7                                                                    & 0                                                                  \\ %\hline
\textbf{Biometric (e.g., Face, Iris, Fingerprint)}    & 22                                                                   & 1                                                                  \\ %\hline
\textbf{Token (e.g., Credit Card)}                    & 18                                                                   & 1                                                                  \\ %\hline
\textbf{Paired Device (e.g., Smartphone)}             & 14                                                                   & 2                                                                  \\ %\hline
\textbf{Paired Account (e.g., Chase bank with Venmo)} & 15                                                                   & 0                                                                  \\ %\hline
\textbf{Others}                                       & 0                                                                    & 0                                                                  \\ %\hline
\textbf{N/A}                                          & 0                                                                    & 22                                                                 \\ %\hline
\hline
\end{tabular}
}
}
\end{table}



\begin{figure}[t!]
\centering
\includegraphics[width=\columnwidth]{charts/payment_methods.png}
\caption{Payment Methods}\label{fig:payment_methods}
\end{figure}




% \begin{figure}[h]
% \centering
% \includegraphics[width=\columnwidth]{charts/security_background.png}
% \caption{Participants security background}\label{fig:security_background}
% \end{figure}

\TODOS{Can break them into separate sections}
\subsection{Protocol}\label{sec:studyprotocol}

\TODOS{The flow of the experiment}

\TODOS{disclosure of motivates?} \TODOS{Motivation?}

\begin{figure*}
\centering
\includegraphics[width=0.95\textwidth]{fig/Protocol.pdf}
\caption{Protocol}\label{fig:protocol}
\end{figure*}

\subsubsection{Study Procedure}
The study consists of two sessions--an enrollment session and the main study session.
The purpose of enrollment is to get participants familiar with our VR setup and interaction, as well as putting them into the game context.
During the enrollment, the experimenter first obtained the consent and asked the participant to complete a background survey about their demographics. 

After that, the experimenter leveraged story-telling to give participant a context.
The participants were asked to imagine that they owned a game account with the assigned name to participate in the archery competition in VR.
The experimented then describing to them about the purpose of enrollment is to get themselves registered in the system, so that they can participate in the game. 
To register, they needed to provide a PIN, sign their signature, and tap their virtual card on the kiosk in a private room inside VR.
We provided a PIN and an account name for them (for the consistency in our analysis of authentication time across different participants), and the experimenter showed them the virtual card they ``owned'' in VR.

Then, the experimenter walked participants through how to use the VR set up to interact in the game, including the basic body controls, such as moving, grasping, tapping, UI selection, etc.
After the introduction, the experimenter initiated the enrollment scene for the participants, where participants enrolled the information and practiced at least five times to get familiarized with the interaction. Such training phase is commonly used in authentication study to get participants familiar with the system, especially for emerging interfaces such as VR.

To reduce participants' fatigue, the second session of an one-hour main study took place in another day.
We first reminded the participants of the study procedure, game context, and the authentication interaction.
The main study consisted of four setups for each participant. 
In each setup, the participant paid for the arrows using one of the four authentication interfaces in a randomized order. 
At the end of each setup, we asked the participants to fill in the per-study question, including the IPQ questionnaire~\cite{schwind-chi19} for VR experience and the system usability scale (SUS)~\cite{bangor-ijhci08} of the payment authentication interfaces.
The participants took off the VR headset and had a rest after completing two setups.
And they also reported identified usability issues.
After completing all four setups, the participants proceeded to the post-study survey, which includes the question regarding security and privacy perception, adoption preferences, and security and privacy attitudes. 
After the experiment, we disclosed our full study purpose of evaluating users' security and privacy perception of authentication to users as well as our focus on the front-end interaction instead of the back-end system.



% \subsubsection{Data Collection}

\subsection{Data Collection and Analysis}

% \subsubsection{Statistical Analysis}
% \begin{itemize}
%     \item Game log (score, payed arrow, authentication time, payment selection time...
%     \item response for scale
% \end{itemize}
% \subsubsection{Qualitative Analysis}
% We conducted open coding for the open-ended survey...

In our experiment, we collected the following three categories of data.
First, we collected quantitative data in the survey, which include participants' demographic information, participant's security and privacy perception rating, the adoption preference ranking, and the VR experience and system usability survey using standard IPQ and SUS scale respectively~\cite{schwind-chi19, bangor-ijhci08}.
We analyzed such data by examining comparing them among the four design choices of authentication interaction.
When we compare them, we compute and report the statistical significance.

Second, we collected the log data during experiment.
The log records participants' interaction and timing of events when they played the game and performed authentication. 
Similarly, we compared the four design choices, in particular the time consumed in authentication and the error rates, by reporting their statistical significance. 

Third, we collected participants' responses to the structured survey and analyzed the content qualitatively. 
The structured survey consists of two parts. 
First, in-between sessions, participants are asked to comment on the usability issues they encountered for each design.
In the post-study structured survey, participants are asked to explain their choices regarding the security and privacy perception questions as well as the adoption preferences.
To analyze such qualitative data, the first author conducted open coding by going through all responses and took memo. 
Then, the whole team discussed and refined the codebook iteratively. 
We also observed data saturation from our code and memo, which we leveraged to terminate the recruitment.
Two researchers including the first author independently coded 25\% of the data and reached a high Inter-Rater Reliability.

\subsection{Limitations}
1. demographics diversity, we didnt correlate the demographic factors with their perception 2. longtidunal, 3. self-reported bias, 4. context diversity, 5. fully functional system? Real world system?