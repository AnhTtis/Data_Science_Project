
\section{Related Work}
% \TODOS{May move to the last after discussion}
% In this section, we cover the latest research on user authentication, user perception of security and privacy, and user experience of payment

\paragraph{User authentication in virtual/augmented reality (VR/AR).} Prevailing user authentication methods on smart devices such as smartphones and digital assistants use one or more of the following factors: (i) unique knowledge such as PIN and/or unlock pattern~\cite{von2013patterns}, (ii) physical biometrics, for example, iris~\cite{kumar-pr10}, (iii) behavioral biometrics, for instance, gestures~\cite{liu2017usability}, or (iv) tokens, for example, a device with coded ID data~\cite{nguyen-sensys16}. These authentication methods can be extended to VR. VR/AR devices can leverage user data captured from various sensory inputs (e.g., head and hand movement) and immersive displays to prevent impersonation and shoulder-surfing attacks~\cite{watson-chi22}. Users of VR/AR may input their secret knowledge, e.g., PIN, to the authentication service using non-traditional methods. For example, In RubikAuth~\cite{mathis2021fast}, users provide PINs by selecting digits from a virtual 3D cube via eye gaze, head pose, and controller tapping. In Olade et al.~\cite{olade-icvars20}, users use different methods to draw unlock patterns in VR. Researchers have also proposed using different biometric traits for authentication in VR/AR, such as motion trajectory~\cite{kupin-icmm19}, electromyography~\cite{chen2021user}, eye tracking~\cite{zhu2020blinkey}. Some proposed biometric authentication methods for VR/AR leverage not only users’ inherent physiological traits but also their unique behavioral traits when performing particular tasks in VR/AR, e.g., throwing a ball~\cite{liebers-chi21}. While users may use physical and digital tokens in authentication and access control for VR/AR~\cite{chan-icfcds15, pereira-ismar21}, these authentication methods often require active interaction with the VR/AR interfaces. This raises usability issues and also presents a tradeoff between security and privacy~\cite{stephenson-sp22}.
% \TODOS{The survey and the atm work, describe some real-life gesture authentication}







\paragraph{Security and privacy perception.} Users’ perceptions of security and privacy based on their interactions with a system impact how readily they adopt and use the system. %such as smart devices, and authentication~\cite{missing}. Such perceptions form from their interactions with the system. 
For example, Distler et al.~\cite{distler2019security} studied how user interface (UI) designs impact users’ perceived security of mobile e-voting apps. They discovered that the lack of appropriate UI feedback and contextual information reduces users’ sense of security. Users’ security and privacy perceptions also depend on other factors, such as personal experience. Jeong and Chiasson~\cite{jeong-chi20} found that children and adults have different interpretations and perceptions of security warnings, e.g., whether an icon of a police officer represents security. Differing preconceptions are challenging for establishing trust with the system, even with visual security clues~\cite{stransky-soups21}. Researchers also investigated how users perceive the security and privacy of user authentication mechanisms. Much work has focused on established authentication methods, e.g., the FIDO2 authentication. Lyastani et al.~\cite{lyastani2020fido2} discovered that users are concerned about the security issues in losing the authentication token. Lassak et al.’s study~\cite{lassak2021s} indicated that users have misconceptions about how biometric data is stored using FIDO2 biometric authentication. Both studies showed that users’ security perception of authentication does not necessarily match the inherent security. Recent research also focused on users’ security and privacy perception in VR. VR developers and users felt the lack of privacy due to opaque data collection policies~\cite{adams-soups18}. Many users center their concern around the threats from other users, e.g., as a bystander~\cite{de-csur19}. Users are also concerned about being deceived by the digital content in VR~\cite{lebeck-sp18}.


\paragraph{User authentication for payment.} 
% User authentication is crucial for securing payments in both physical and online settings. %including physical stores, online retailers, and cryptocurrency markets.
Authentication requirements can differ in different settings. For example, using a chip card may suffice in a physical store, whereas additional one-time passwords (OTPs) may be required while shopping online~\cite{acharya2013two}. The perceived security of authentication also impact users’ adoption and use of payment services. For example, there is a significant uptake of mobile payments because users associate perceived control and security with user authentication on their devices~\cite{zhang-mdpi19}. Voskobojnikov et al. also showed similar takeaways from their study on cryptocurrency wallets: some users desired enhanced security using biometrics when transferring funds~\cite{voskobojnikov-chi21}.  
Users’ understanding of how the payment services ensure the security of authentication, e.g., the confidentiality of the password, also impacts their trust~\cite{zhou-id11}.
Different authentication processes for conducting payments in different geographies can also result in differing perceptions of security~\cite{busse-eurospw20}.
The environment where users authenticate for payment factors into their perceived security. For example, several users indicated authenticating with an ATM is riskier than payments in a restaurant due to their unawareness of attacks~\cite{volkamer-soups18}.
In addition to security, factors such as usability in using a user authentication method, also impacts use of associated payment service~\cite{kujala2017role}.






% \subsection{Research Gaps}
% \TODOS{A summary of prior work and what's missing to motivate our research}
