\paragraph{Perception of Consent.} We asked the participant whether they felt they have given consent to the payment method. The response to this question is captured in \Cref{fig:consent}. The responses show that for the two PIN payment methods, namely \ptpf and \ptpk, the participants felt they gave consent to the payment method more when compared with \ttp and \stp methods. We believe that the higher level of user interaction in PIN methods might have contributed to this response. Since the users had to enter a number they might have had this perception that by entering the PIN they are giving consent to the payment transaction. Contrary to our expectation, signing did not have a similar affect. %%WHY?!

\begin{figure}[h]
\centering
\includegraphics[width=\columnwidth]{charts/alert.png}
\caption{Users' perception of being alerted to the payment method}\label{fig:alert}
\end{figure}

\paragraph{Perception of Being Alerted.}  We asked the participants whether they felt they need to be alerted for each of the tested payment methods. The response to this question is shown in \Cref{fig:alert}. The participants felt that they do not need to pay as much attention while entering the PIN. It seems that they felt as long as they enter the correct PIN there is not much need to pay attention to the payment method and there is not a need for a feedback from the system. Comment to this set of questions shows that the participant would have preferred to receive a feedback after tapping and signing to make sure they are tapping on the right spot or their signature has matched. 

\begin{figure}[h]
\centering
\includegraphics[width=\columnwidth]{charts/secured.png}
\caption{Users' perception of payment method being secured}\label{fig:secured}
\end{figure}

\paragraph{Perception of Being Secured.} We asked the participant whether they feel the payment method is secure. The responses are shown in \Cref{fig:secured}. Most of the participants associate PIN numbers with security and hence, felt that this method is more secure compared to others. Some participants had concerns that their signature may be forged and payment be made by others. With respect to tapping, the main concern seems to come from the fact that the system has low user friction and some users were concerned that they may mistakenly tap their card with no intention of making a payment. 


\begin{figure}[h]
\centering
\includegraphics[width=\columnwidth]{charts/control.png}
\caption{User's perception of being in control of their payment}\label{fig:control}
\end{figure}

\paragraph{Perception of Being in Control. } We asked the participant wether they feel they are in control of their payment. The responses are show in \Cref{fig:control}, which shows that majority of the participants agreed that PIN gives them the feeling of control. As opposed to PIN methods, tapping the card has less friction and therefore felt to the users that the system is in control as they do not need to enter any input (PIN or signature). 


\begin{figure}[h]
\centering
\includegraphics[width=\columnwidth]{charts/privacy.png}
\caption{Users' perception of their privacy being given away}\label{fig:privacy}
\end{figure}

\paragraph{Perception of Privacy. } With respect to privacy it seems that the participants were more in agreement than disagreement that in \ttp their privacy is less at risk. In contrast in \stp method, several of the participants agreed (N=10) or strongly agreed (N=10) that their privacy is given away. In the two PIN methods, participants were less concerned about their privacy when compared to \stp but not as confident as they are in \ttp. This result suggests that since the participants did not need to enter any PIN or draw any signature, they did not have as much concerns about their privacy, where as they directly associated drawing their signature to giving away their privacy. 



% \paragraph{Payment Preference.} The results to the payment preference method shows that the majority of the participants selected \ttp as their first priority and selected \stp as their last choice. None of the participants selected \stp as their first preference which might attributed to their perception of securityand usability of this method. 


%\subsubsection{Users' security perception rating}
% Finding: perception diverges regarding the threat model users first perceived: who is the adversary? Observer or payment service provider?
\TODOS{Please help put the bar charts for the five questions about consent, alert, control, security, privacy here and add some description texts just to compare them, e.g., describing which is better than which for a certain aspect and quote some examples--leading to the factors. The bar charts can just be taken from the ms form screenshot}

\subsubsection{Factors that influence security perception}
\TODOS{I have enough content for the approval version (need to put it down here). I need help in computing the irr, too.}

From users’ explanation on their rating for the questions related to security, we identified four themes of the factors that influence users’ perception regarding the security of the four authentication methods. The themes include (1) \textit{user experience and interaction in VR}, (2) \textit{users’ threat modeling for authentication in VR}, (3) \textit{users’ understanding and expectation of payment authentication in VR}, and (4) \textit{the context of payment}. Next, we describe how these factors affect users’ perception.


\paragraph{User experience and interaction in VR.} We observed users associate their security perception with their experience when interacting to authenticate in VR. 

First, users related their \textit{\textbf{feeling of realism and relatedness to real-life authentication}} to their security perception. Users mentioned that performing the actions they are familiar with in real-world payments contributed to the feeling of giving consent to pay in VR. For example, one user favored card-tapping for such feeling of payment consent, compared to other three interfaces. 
\begin{displayquote}
``Grabbing the card and making the payment made it seem like an actual payment.’’
\end{displayquote}
However, users’ feeling of realism or relatedness may differ, depending on their real-life experience. One user refers to signature as a natural way to give consent: 
\begin{displayquote}
``The act of signing is usually used as a form of giving consents in the physical world, and I naturally perceive it as giving my consent.’’
\end{displayquote}
However, another user thought sign-to-pay is less reasonable since it is not what they are familiar with when paying in their daily life.
\begin{displayquote}
``I've rarely signed to pay (only recently in the US, but even then the signature comes after the payment part)’’
\end{displayquote}
Also, the physicality affects users’ feeling of realism thus perception, too. The visual presentation of the VR credit card adds realism to the authentication experience, however, some users mentioned the gap in its physicality as a virtual object, e.g., the lack of feedback, reduced its realism.


Second, the \textbf{\textit{physical and mental efforts}} sometimes improve users' perception about the security of authentication, despite they hinder the usability.
The physical effort focuses on the interactions users perform, while the mental effort highlights the attention users pay, e.g., memorizing the PIN or ensuring the signature quality.
Users often compare tap-to-pay with pin-to-pay regarding such effort, as tap-to-pay is often perceived as the easiest while pin-to-pay requires effort in memorizing and entering the PIN. 
Interestingly, such effort makes users alerted and paying more attention.
Further, it helps pin-to-pay overweighs tap-to-pay regarding its security as one user describes.
\begin{displayquote}
``The tap-to-pay is easy so it feels less secure than the pin ones.''
\end{displayquote}

Third, users raised \textbf{\textit{usability issues when interacting in VR}}, which factor into security perception. 
Most usability issues centered around signing in VR. 
Users felt the barrier in signing using a VR controller, which is less natural in VR compared to the real world.
And users considered it hard to recreate consistent signatures.
Though some users mentioned it makes them feel more ``engaged'' while being ``cumbersome'', this usability gap reduces their feeling of signing their own signature:
\begin{displayquote}
``The sign-to-pay method was a bit hard to use, so I think I just tried to write something, and I felt less like providing my signature.''
\end{displayquote}

On the other hand, users are aware of the security-usability tradeoffs due to the usability issues.
When users' less-accurate signatures were accepted in VR, they were worried that it is a sacrifice of security.
\begin{displayquote}
``The sign-to-pay felt the most insecure as people easily have access to my cheques and can probably fake in the VR world since the VR signatures were clearly less accurate than the real-world.''
\end{displayquote}
Meanwhile, though the shuffled PIN pad impedes usability, users valued it for security.
\begin{displayquote}
``The shuffled numbers for the pin gave me some sense of security.''
\end{displayquote}

In addition, users mentioned the usability issues due to unexpected behaviors, which alerts users and reduces the feeling of security of authentication.
For example, one user showed concern about accidentally tap to pay when it further reduces the difficulties of payment.
\begin{displayquote}
``I think tap-to-pay would be the easiest method of the four if there was a mechanism to extend my "arm", but that would also introduce a much greater risk of accidentally paying for something when I may not have intended to.''
\end{displayquote}

Last, the design of users' perception of security is related to the design of \textbf{\textit{user interface components}} in authentication.
These components include both the visual presentation and the way of interaction in authentication.
For example, though users perceive the security of the two PIN pad interfaces (ego-centric and on-kiosk) similarly, some users brought up that the more personal view of ego-centric PIN pad would help improve security, as it seems less exposed to others.
\begin{displayquote}
``I think pin numbers are genuinely more secure, but I would vote more highly for the floating method because it seems a bit harder for others to see the numbers that I am putting in.''
\end{displayquote}
In addition, users felt more alerted to complete the payment when being surrounded by the ego-centric floating PIN pad.
\begin{displayquote}
``I frequently forgot that after purchasing arrows required payment on the kiosk. The floating pin-to-pay clearly wouldn't let me proceed through the game without doing so first.''
\end{displayquote}
Likewise, users also mentioned that the presence of the pen for signature and the card-tapping icon showing on the interface kept them alerted. 
However, some users may still perceive the difference between the VR interaction and the real world interactions, as mentioned in the previous section, which also decreases their perceived security.
\begin{displayquote}
``Tap-to-pay seemed less secure because it's not my real card but a card-like object that I am tapping.''
\end{displayquote}


\paragraph{Threat modeling of authentication in VR.}
The second theme we identified that contributes to users' security perception is their threat models. 
Users named privacy and security vulnerabilities, among which many are related to other potentially malicious entities.

Users worried that \textbf{\textit{other potentially malicious entities}} will open up vulnerabilities when authenticating to pay. 
First, they realize other malicious users may reside either in the VR space or the physical world.
Note that though our VR scene does not allow multiple-users' co-presence, users are aware of the potential risk.
Malicious users may observe users' behavior, including the PIN, signature, or card when sharing the VR space.
For instance, one user prefers sign-to-pay over pin-to-pay in multi-user scenarios as pin-to-pay may leak PIN information to others.
\begin{displayquote}
``If I was sharing my virtual space with others and they can see what I was typing then I would have gone for "Sign to pay" (although not so usable).''
\end{displayquote}
Moreover, threats could come from the physical space, where attackers leverage the knowledge of users' signatures in forging VR signatures.
\begin{displayquote}
``signing can be also copies by anyone who knows my signature in the real world.''
\end{displayquote}
Beyond other users, users recognized the threats due to other entities that handle the authentication and transaction such as a malicious app, who may have more privilege, e.g., accessing a signature digitally.
\begin{displayquote}
``Sign-to-pay seems the securest approach since it's hardest to be copied by some one else. However, this might not be true in a digital world since a signature will be as simple as copy-and-paste.''
\end{displayquote}
However, some mentioned that the definition of adversary of payment authentication in VR is not clear.
\begin{displayquote}
``Ofc, to consider this deeply requires a solid definition about what adversary we are facing. E.g., a man-in-the-middle or ish.''
\end{displayquote}

Users' awareness of \textbf{\textit{privacy vulnerabilities and attacks}} builds up their threat models.
Users are concerned about the private information from different angles. 
They are aware of the PIN code as a secret they know and other personal information, including their name, card information, and behavioral traits during signing. 
When comparing PIN codes with signature, some users felt it is not personal hence less privacy-related.
\begin{displayquote}
``Sign-to-pay requires my signature, so it requires more information from me. The pin is not personal information, so the privacy concern is less.'' 
\end{displayquote}
However, others mentioned that reusing PIN codes across different services is risky.
\begin{displayquote}
``I might reuse from other services, thus revealing some private information.''
\end{displayquote}
Similarly, users perceived the risk that their signatures in VR can be observed and forged.

The privacy of tap-to-pay seemed obscure to some users, as they did not have to volunteer information.
\begin{displayquote}
``Tap-to-pay required me to volunteer no information, so it felt secure enough.''
\end{displayquote}
Even though, some users noticed that faithfully rendering card information, e.g., cardholder name and card number, is also a privacy risk when sharing the VR scene.
\begin{displayquote}
``I feel unsafe if the full credit card number shows on the virtual credit card.''
\end{displayquote}

In parallel, users named multiple \textbf{\textit{security vulnerabilities and risks}} associated with the use of VR.
First, users showed fear of the unauthorized behaviors when sharing the same VR scene with others. 
For example, users were afraid that their digital credit cards can be misused by others. 
\begin{displayquote}
``The tap-to-pay mechanism could be any person holding my card (I might give the card temporarily to a friend/family).''
\end{displayquote}
Also, they still carried concern about malware that affects VR.
\begin{displayquote}
``Similar reason that pins and signatures are just exposed to the game or malware in the gaming system.''
\end{displayquote}

The other line of security factors attribute to VR's visual interface, which is prone to illusion and deception. 
For example, an attacker may exploit a hidden payment terminal to trick users to pay.
\begin{displayquote}
``Tap-to-pay is the easiest, but I feel it not safe because I can easily touch it to an invisible system in the VR world.''
\end{displayquote}
On the other hand, users thought the noticeability of these attacks differs in VR, e.g., stealing a card vs. shoulder-surfing PIN codes.
\begin{displayquote}
``...because an attacker might learn my PIN, it's for them to memorize and I would not necessarily know it (unlike stealing a card).''
\end{displayquote}

\paragraph{Understanding and expectation of the authentication system.}
Participants demonstrated different understanding and expectation of the security given the front-end interaction.
They demonstrated their understanding in the following domains -- (1) its security guarantee, (2) the sense of ownership in the assets, (3) security alert, and (4) trust. \TODOS{TO expand}

First, participants could be unclear about the \textbf{\textit{security guarantee}} given the interactions.
They feel the security of tap-to-pay is obscure, as they do not know how security could be integrated into such digital token without a physical card.
\begin{displayquote}
``I don't know how the backend works, I assume it should quire a pre-certification process such as a cookie or ish.''
\end{displayquote}
Some of them suspected that there should be additional verification step on the digital card.
Even for the PIN pad methods, some users did not realize that PIN shuffling's security purposes.
A lack of understanding could alert users.

Second, \textit{\textbf{the sense of ownership}} increases users' awareness of security.
Such ownership includes the ownership of digital assets and the virtual environment.
\begin{displayquote}
``The pin-to-pay feels more personal as the pin is something suppose only known by me.''
\end{displayquote}
For example, users considered owning a secure PIN as a key to protect the system; meanwhile, some of they did not feel they own their signature in VR due to the usability and interaction hassles in signing as discussed in the previous section.

Also, users mentioned that they would feel less secured when they are not the sole owner of the virtual space as previously discussed in their threat modelling.

Third, participants expected the presence of \textbf{\textit{security warnings}}, especially for tap-to-pay which seems the easiest authentication process to them.
Such warnings and alert may be delivered via channels outsides of VR:
\begin{displayquote}
``Even though all of the above payments closely mimic payments in real life, I could only become sure of its security depending on how the system notifies me w.r.t. suspicious activity (e.g., sending me an SMS/email/etc.)''
\end{displayquote}

Last, participants established trust based on their prior experience in the real world and their understanding of VR authentication.
Such trust of distrust may be associated with the VR technology in general:
\begin{displayquote}
``Tap-to-pay was the simplest, but it was way too simple to believe that the entire payment process behind the scene was dealt with as I wanted. I'm not sure if this is due to my distrust to the specific payment system, or just to the VR world, or both.''
\end{displayquote}



\paragraph{The context of payment.}
The context of payment also pays a role in participants' security perception.
Such contextual factor include (1) the entities in handling the payment, (2) the seriousness of payment, (3) and the sense of agency.  
% \TODOS{TO expand}
First, participants considered multiple \textbf{\textit{entities who handle the payment transaction}} in their threat modelling, which is not straightforward in VR.
\begin{displayquote}
``Even though it might seem that the system is storing my signature (with sign-to-pay) and my pin (with pin-to-pay), any fears of violation of privacy should also be valid with real-life payments. The reason why most (or at-least, I) continue to use such ways of payment in real-life is because of trust in the system (vendors, infrastructure, etc.)''
\end{displayquote}

Second, participants considered the \textbf{\textit{importance of the payment}} when they evaluating security.
Even PIN to pay made some participants feel not serious.
\begin{displayquote}
The pin-to-pay methods felt more like providing my info, but I think the floating one felt a bit too much like being in a game as well.
\end{displayquote}

Last, some participants mentioned that their \textbf{\textit{sense of agency}} in completing the payment adds to their perceived security.
\begin{displayquote}
``I guess the 4 methods don't make a difference, as I was the person who conducted the payment, I didn't feel any insecurity.''
\end{displayquote}