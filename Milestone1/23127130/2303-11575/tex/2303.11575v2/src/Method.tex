
\section{Methodology}
We use a ``technology probe'' approach to understand how interaction experiences affect users’ perception of authentication in VR and extract design guidelines~\cite{hutchinson-chi03}. The idea of the ``technology probe'' approach entails using a set of proof-of-concept interfaces. As these interfaces package basic interactions, researchers can reveal phenomena otherwise hidden from user interactions~\cite{chandrasekaran-soups21}
This approach is commonly used to evaluate emerging technologies, including VR and user authentication~\cite{mathis-vr22,tang-chi21}.
Our probes implement the core interaction patterns of user authentication to elicit users’ responses in regard to users’ interaction experience and perception of security and privacy.
We deploy our probes for users to make routine payments in a VR game, which is an ecologically valid context. 






\subsection{Probe Design}\label{sec:probe design}

\begin{figure}[h]
\centering
\includegraphics[width=\columnwidth]{fig/Probe_design.pdf}
\caption{Our four probes for VR authentication.}\label{fig:Probe_design}
\end{figure}




Our probe designs represent three major paradigms of interactive user authentication leveraging something you know (PIN), something you have (a virtual card), and something you are (signature)~\cite{ross-nist19}. 
We designed four probes: (1) floating PIN pad (\pinf), (2) on-kiosk PIN pad (\pink), (3) tap-to-pay (\tap), and (4) signature (\sign).

The following describes the design of the four probes, which are shown in Figure~\ref{fig:Probe_design}.


\paragraph{Floating PIN pad (\pinf).}
\pinf resembles the default virtual input interface of many VR platforms, where users interact with a floating PIN pad.
\pinf conceptualizes the idea of giving users a personal and isolated virtual experience in authentication. 
Users enter PINs via pointing on a floating PIN pad that follows users’ viewport. 
\pinf randomizes the PIN layout, which is a common security mechanism in digital PIN pad against observers~\cite{mathis-vr22}.

\paragraph{On-kiosk PIN pad (\pink).}
In contrast to \pinf, for \pink, we mapped real-world interfaces into VR. Such mapping is becoming a popular design choice to recreate real-world experiences in VR~\cite{argelaguet-cg13}.
\pink maps real-world experiences into the VR scene by rendering its PIN pad on the kiosk where users initiate and confirm payment. As such, users do not experience a gap in the transition between authentication and other payment tasks, e.g., selecting the items.  

\paragraph{Tap-to-pay (\tap).} 

\tap represents how users could use a personal virtual object as a unique token to authenticate. It establishes the realistic mapping by rendering a credit card as the authentication token. To pay, users take out a virtual credit card from an inventory on their virtual body and tap the card on the kiosk display. The kiosk checks whether the card is in proximity and being held by the user. In addition, the virtual card renders a name and a card number.


\paragraph{Signature (\sign).} 
\sign represents how users draw a unique signature to give consent and prove their identity. It maps the real-world signing processes.
Users grab a virtual pen from the kiosk and sign on the designated area using the hand-held controller; Only after users sign, they can proceed to check out.

Note that, as technology probes for interaction, our designs only package the essential frontend interaction instead of the full backend of authentication mechanisms, e.g., verifying an encrypted token or comparing signatures. Instead, our study adopts the idea in ``Wizard-of-Oz'' studies~\cite{mecke-icmum18} in creating an impression that the system has the full functionality through a mock registration process.



\subsection{Experiment Context: VR Archery Game}
\label{sec:game_design}

We design a VR game -- an archery contest -- where participants trade in-game credits, using the probes to authenticate their payment. Participants earn credits by shooting virtual targets and consume these points when they refill arrows. To make the game and payment realistic to participants, we match their in-game credits with a physical award: the participant who scores the highest wins a grand prize (a 90 USD smartwatch). In the below, we further describe the main components in this game (see Figure~\ref{fig:game_scene}).  



\begin{figure}[t]
\centering
\includegraphics[width=\columnwidth]{fig/game_scene.pdf}
\caption{The environment of our archery game with key components marked. }\label{fig:game_scene}
\end{figure}



\paragraph{Environment.} We situate the contest in an indoor archery range. We assign the participant a chamber. In the chamber, the participant can find the bow set and a kiosk instructing them to shoot, refill arrows, and authenticate their payment. We place the targets towards the other end of the room at a distance. The participant can also find the information display that shows their current credits and remaining progress. 

\paragraph{Bow set.} The participant can interact with the bow using the two hand-held VR controllers. They need one hand to hold the bow and the other hand to draw. The participant shoot the arrow toward the target by releasing the drawing hand. The participant will need to refill after they use up three arrows. Each arrow consumes ten credits.

\paragraph{Targets.} We place eight targets in the range at two different distances. Taking down each closer target rewards 20 credits to the participants while 30 for the farther targets. 

\paragraph{Kiosk.} The kiosk displays game instructions and payment interfaces. To refill arrows, participants interact with the kiosk to select how many arrows to load. After the participant confirms the selection, the kiosk will display the authentication interface among one of the four probes. Once the participant completes authentication, the kiosk will display a waiting page that emulates the running backend processes. After that, the kiosk will accept the credit payment and refill arrows. 

\paragraph{Information display.} The display shows the current credit, the remaining arrows, and how many rounds the participant has completed. 



\paragraph{Implementation.}
We implemented our VR game using Unity, a mainstream VR engine, and C\#.
In the experiment, we ran the game on a commodity PC (CPU: Intel i7-12700F, GPU: Nvidia 3060Ti), which is connected to an Oculus Quest 2 VR headset.
Participants interacted using the headset and its hand-held controllers.












\subsection{Experiment Design}
Here we explain our experimental procedure, the instruments we used to collect users' responses, and our recruitment.
\subsubsection{Experimental Procedure}\label{sec:expreiment_procedure}


\begin{figure}[t]
\centering
\includegraphics[width=\columnwidth]{fig/Game_flow.pdf}
\caption{Our experimental procedure.}\label{fig:game_flow}
\end{figure}


We designed a with-in subject experiment to evaluate the four probes.
The study consists of two phases--an enrollment phase and the main study phase.
The purpose of enrollment is to familiarize participants with our VR and authentication setups. In the main study, participants played the VR game and interacted with our probes during authentication.
Figure~\ref{fig:game_flow} illustrates our experimental procedure

\paragraph{Enrollment.} The experimenter first obtained consent and asked the participant to complete a pre-study survey. After that, the experimenter explained the game context to the participants. We assigned participants a ``nickname’’ for the game. The experimenter then introduced the PIN, the virtual card, and the signature setup to the participant (we assigned each participant the same credential information for the statistics of authentication time). We described to the participants that they would enroll the information, including the signature, for the use of authentication in the second phase. 
The experimenter then walked participants through how to use our VR setup.
After the tutorial, the experimenter let the participants practice with the registration kiosk, similar to the one they will use for authentication, and present their credentials to it.
After registration, the participant will practice archery. The enrollment phase took about 20 minutes.

\paragraph{Main study.} The one-hour main study took place on another day to reduce participants’ fatigue.
We first reminded the participants of the study procedure, game context, and the authentication interaction.

Then, the participant entered the game with 400 points. They proceeded to finish four sessions; each includes the game with an in-study survey. In each session, the participant will pay using one of the probes (in a randomized order) to authenticate. Each session consists of three rounds. Participants authenticated to pay for the arrows (min: 1, max: 3), except the first round of every session where three arrows were free. At the end of each session, the participant completed an in-study survey. And after the second session, the participant took a rest outside of VR. After four sessions, we instructed the participant to fill in the post-study survey. Then, they redeemed the credits, which we used to compete for the grand prize after we stopped recruitment.

After the experiment, we disclosed our full study purpose to the participant. We clarified that our focus is to evaluate the interaction experience from the frontend, and we did not process their information, i.e., signature, in the backend.

\subsubsection{Instruments}
Our study mainly relies on surveys that elicit users' responses for our qualitative analysis. 
We chose to use surveys instead of more active methods such as think-aloud studies~\cite{charters-bej03} to minimize the interference with users' game experiences.

\paragraph{Pre-study survey.} In the pre-study survey, we collected participants' demographic backgrounds, including their gender, age, education, and experience with VR. We also used the standard affinity for technology interaction (ATI) score (computed from 6-point Likert ratings) to understand participants' tech-savviness~\cite{franke-ijhci19}.


\paragraph{In-study survey.} The in-study survey assessed (1) users' general experience in the VR context and (2) the perceived usability of authentication probes.   
For the former, we used the IPQ VR presence questionnaire, a standard measure of users' sense of presence in VR. 
The sense of presence is a commonly adopted measure of VR experience. It is defined as users' subjective perception of being and acting in the virtual world (though their body resides in the physical world)~\cite{schwind-chi19}.
This measure consists of four sub-scales (1) sense of being here (PRES), (2) spatial presence (SP), (3) involvement (INV), and (4) experienced realism (REAL).
For the perceived usability, we collected participants' open-ended comments on the usability issues of each authentication probe when participants exited VR. Moreover, participants completed the system usability scale (SUS), a standard measure to assess the overall perceived usability, for each authentication probe~\cite{brooke-jus13}.

\paragraph{Game log.} In addition, we logged participants' behaviors and timings in the game, including their archery performance as well as the time spent on payment and authentication.
We used these objective behavior logs to support our qualitative findings.

\paragraph{Post-study survey.}
The post-study survey consists of three major components. 
First, to understand participants' engagement in the routine payment of the game, we asked them to explain how they decided on the number of arrows to pay. 
Second, we wanted to understand participants' security and privacy perception of payment authentication.
We designed questions to elicit participants' responses from different angles. 
From prior research, we identified the five aspects related to the security and privacy of payment authentication, namely \textit{consent}~\cite{herzberg-cacm03,lyastani-sp20}, \textit{security}~\cite{kim-ecra10},  \textit{privacy}~\cite{zimmermann-ijhci20}, \textit{being alerted}~\cite{reese-soups19,khan-soups15,wolf-chi19}, and \textit{in control}~\cite{nilsson-chi05}.
We asked the participants to evaluate and elaborate on their agreement on statements related to the five aspects. One example is: when the participant used \tap to pay for arrows, ``I [the participant] felt that my [the participant's] payment was secured''.
We collected both their Likert-scale rating and open-ended responses to explain perceptions. 
However, rather than relying on the quantitative ratings, we mainly study the relation between interaction experiences and perception from their open-ended responses.
Third, the post-study survey asked participants about their preferences among the four probes, their quality expectations that affect their preferences, and their suggestions to improve these probes.
We use these questions to understand users' expectations on VR authentication.




\subsubsection{Participants and Recruitment}
Our study and recruitment were approved by the IRB-equivalent body of our organization. Consistent with prior work in VR authentication~\cite{mathis-vr22}, we recruited 24 participants from our organization. We stopped recruiting when we observed data saturation from our qualitative coding~\cite{guest-fm06}. Each participant received compensation (a 30 USD gift) after they completed the study.

The demographics of participants are as follows. 18 out of 24 identified themselves as male (6 female). 16 of them are in the 25-29 age group. 20 of them completed or were studying for a graduate degree. Most participants have a background in computer science. Participants' ATI score (mean: 4.35, std: 1.1) shows a high affinity for technology interaction)~\cite{franke-ijhci19}. 20 participants had used VR before (mainly for gaming). But none of the 20 participants frequently used it.




\subsection{Data Analysis}

To analyze such qualitative data, the first author started open coding and took memos while recruiting participants. Meanwhile, another researcher coded data independently. The whole team discussed the memos, reconciled the codes, and refined the codebook iteratively. Using Grounded Theory~\cite{walker2006grounded}, high-level themes emerge from our coding.
And the two coders reached high inter-rater reliability (Cohen’s Kappa $\kappa$ = 0.81) using 25\% of the statements.
When we observed data saturation from our coding~\cite{saunders2018saturation}, we stopped recruiting. 
We make our codebook available via an anonymous link.\footnote{Our codebook and survey are available at: \url{https://osf.io/bcne7/?view_only=717805fc875847c19e5e1d5ef97a16e5}}
In Figure~\ref{fig:framework}, we show our analysis framework along with themes we identified from analysis.

In addition, we report the quantitative data to support our qualitative analysis, including the IPQ questionnaire, SUS scales, and participants’ Likert-scale rating on security and privacy perception.


\subsection{Limitations}

Our study has the following limitations. First, our participant population has a demographic bias, e.g., most participants are tech-savvy.
Using a technology probe, our study's objective is not to generalize but present a set of findings and recommendations that guide future design.
Nevertheless, our analysis reveals that even these tech-savvy participants faced challenges in interaction with VR and assessing security and privacy, let alone other users. 
Future work may study the probes with a more diverse population, generalize the findings, and quantitatively measure users' experiences and perceptions. 
Second, our study does not investigate how users use VR authentication longitudinally in the wild where users’ experience and perception of security and privacy may change over time. In addition, participants' self-reported responses may be biased due to social desirability~\cite{phillips-ajs72}. Despite these limitations, we believe that our work still presents a significant contribution. To the best of our knowledge, our exploratory study is the first to investigate the interplay between interaction experience and perception of security and privacy for VR authentication.

