
\section{RQ1: Interaction Experience}

Participants’ interaction experiences are dependent on factors around two main aspects: (1) the perceived usability of authentication using the four setups and (2) their experience in the VR game -- the context for payment authentication.







\subsection{Perceived Usability of Authentication}
We reveal the components in authentication interaction and their associated characteristics, which affect overall usability.

\subsubsection{Components of Authentication Interaction}
\label{sec:Components of Authentication Interaction}

We categorize the interaction components of VR authentication interaction into three themes: (1) motion control, (2) authentication interface, and (3) process of authentication.

\paragraph{Motion control.} Participants performed gestures and interacted with digital objects when they authenticated. The motion control in making these actions in VR impacted the usability of authentication regarding the three subthemes (1) spatial awareness in the VR space, (2) action control and consistency, and (3) interaction modality.

First, \textit{spatial awareness in the VR space} includes feeling a virtual object in relation to their avatar. Some participants noticed their lack of spatial awareness hindered them in controlling a virtual object, e.g., stretching their arm to sign a signature:
\begin{displayquote}
\textit{``It takes a while to get used to the proper distance between the pen and the kiosk screen.''} (P7)
\end{displayquote}


On the other hand, the \textit{action control and consistency} impacted the quality of participants' actions, e.g., how their gestures aligned with their intention, via the hand-held controllers.
Participants preferred to have better \textit{action control and consistency} when interacting with digital components. For example, one participant preferred \pink over \pinf due to such consistency in entering PINs. 
\begin{displayquote}
\textit{``The floating idea seemed okay, but I think the stable kiosk was easier to use as it allowed for better calibration of my pointer.''} (P10)
\end{displayquote}

Last, participants had different preferences over the \textit{interaction modality} used to control interface components, e.g., using the controller to sign:
\begin{displayquote}
\textit{``could be easier if I can just use my finger to sign because grabbing the pen is not [a] very good experience. ''} (P24)
\end{displayquote}
They also compared it to other interaction modalities in the real world, for example, \textit{``laser pointer is less similar to using keyboards in real world.''} (P2)



\paragraph{Authentication interface.} The design of the authentication interfaces affected participants’ perceived usability regarding (1) the interface presentation and (2) the interaction feedback.



The \textit{interface presentation} relates to how the interface components for authentication are displayed and visualized. 
Participants sometimes expect different presentations in VR than in the real world. For example, \pink displays the PIN pad on the kiosk, which requires participants to move their avatar compared to \pinf:
\begin{displayquote}
\textit{``This seemed like the real world kiosk, where the number-pad was at a fixed place. I had to move to get a better look at the numpad.''} (P13)
\end{displayquote}
In another example, one participant mentioned the aesthetic style of the virtual card:
\begin{displayquote}
\textit{``expect some more fancy effects than real life card experience, such as when success the card shows another color etc.''} (P24)
\end{displayquote}


The \textit{interaction feedback} refers to how the authentication system confirms participants' actions, e.g., delivering a confirm message as P24 described above.
Some participants were concerned about the lack of this feedback during authentication in VR. Such feedback includes visual and haptic cues. Sometimes, they expected the feedback they would receive in the real world, e.g., the sense of writing on a paper.
\begin{displayquote}
\textit{``It is still usable and the pen writes like real writing but still did not get the feeling of writing on a paper.''} (P1)
\end{displayquote}


\paragraph{Process of authentication.} The process of authentication affects perceived usability due to (1) its learning curve, (2) the transition in workflow, and (3) knowledge to memorize.
The \textit{learning curve} refers to how easy participants will get used to the interactions in VR. For example, one participant felt signing was easy after practicing.
\begin{displayquote}
\textit{``It's very easy to learn and use, and the functionality can be easily picked up.''} (P3)
\end{displayquote}

Meanwhile, authentication probes may require multiple steps of interaction, i.e. the \textit{transitions in the workflow}
participants perceived simpler transition in the workflow of authentication as merit, which relates to the difficulties of VR motion control:
\begin{displayquote}
\textit{``I do not need to accomplish complicated movements/actions in the virtual env.''} (P18)
\end{displayquote}
This aspect was also related to the security measures, i.e., PIN pad shuffling in our probes: \textit{``The challenging part was that the numbers shifted locations between attempts''} (P13).

In addition, how much \textit{knowledge to memorize} the participants should memorize is associated with their experiences. More than remembering every step to complete the authentication, participants mentioned the burden of memorizing the PIN: \textit{``It was hard for me to memorize the pin''} (P11).


\subsubsection{Usability Characteristics of Authentication}
Participants mentioned multiple usability characteristics of the authentication interactions related to the above components. These characteristics are in three themes: (1) easiness, (2) realism, and (3) intuitiveness. Then, we explain how the characteristics, along with the interaction components, affect the overall usability of each authentication probe.


\paragraph{Easiness.} Other than generally describing an interaction as easy or cumbersome, participants associate easiness with \textit{comfortability} of interacting in VR, the \textit{physical, mental, and time efforts}, and the \textit{smoothness} of the interaction process. Participants perceived the easiness of one authentication method differently. For example, as mentioned before, some participants thought signing was easy, while others did not. one participant felt it was easy conceptually but hard in practice when performing the VR gesture.
\begin{displayquote}
\textit{``Signature: it was conceptually easy, but the execution was somewhat cumbersome and it required a complex gesture.''} (P6)
\end{displayquote}
 
\paragraph{Realism.} Participants often liked the VR interaction and interfaces that are realistic as in real life, which made them feel \textit{familiar} and \textit{immersed}. However, the participants differed in such perception due to their prior experience. For example, one participant did not think a shuffled PIN pad is realistic for an ATM or gas station.
\begin{displayquote}
\textit{``I got confused by the randomization of the numbers on the pin-pad. I am not used to this since most terminals have a fixed layout (I am thinking of ATMs and gas stations).''} (P5)
\end{displayquote}
Meanwhile, when participants felt immersed in VR, some of them did not like authentication to break such experience through an ``unreal'' interface:
\begin{displayquote}
\textit{``the floated pad makes it so unreal that I know it is in VR rather than real life. I do not like the experience.''} (P24)
\end{displayquote}

\paragraph{Intuitiveness.} Last, participants were in favor of interfaces that are intuitive. The intuitiveness appears as a result of the interaction interface being \textit{simple} and \textit{clear}, and participants could rely on their prior authentication experience in real life: \textit{The interface is intuitive and somewhat matches what you have in real life''} (P7). However, as mentioned above, some participants got confused by the interaction components, e.g., the shuffled PIN pad.


\paragraph{Overall usability.} In Figure~\ref{fig:sus}, the overall usability for the four probes via participants’ SUS rating, a standard usability metric. In addition, we also compare the authentication time in Figure~\ref{fig:time}.
\tap received the highest SUS score (mean: 82.1, std: 14.6), indicating ``excellent'' usability~\cite{brooke-jus13}. It has the merit of easiness, realism, and intuitiveness. 
Nevertheless, some users still mentioned the burden of walking towards the kiosk to tap: \textit{``because in this game I [the participant] have no arms that I can extend my reach to a kiosk that is a few feet away''} (P16).

\pinf (mean: 70.6, std: 19.9) and \pink (mean: 76.1, std: 11.7) closely follow \tap, showing a ``good'' usability. Participants liked their usability because they were familiar with this scheme, and it was relatively easy, despite the effort in memorizing and entering the PINs. However, the margin between \pinf and \pink is small. Participants commented on different usability issues for them. For \pink, though it looked more real, some participants felt burdensome walking towards the kiosk to see the PIN pad clearly; others thought entering the keys on a moving PIN pad (\pinf) was distracting. 

\sign (mean: 60.7, std: 14.7) has the lowest score among the four but still has an ``OK'' usability. Though its interface seemed intuitive, signing with the virtual pen was not easy for multiple participants in VR, as \textit{``signing in the virtual world was very different as compared to the real world''} (P13).


\begin{figure}[h]
\centering
\includegraphics[width=\columnwidth]{charts/ipq.pdf}
\caption{IPQ sense of presence scores for the four authentication probes. A higher score indicates a higher level of presence. From 0 to 6, a score higher than 3 stands for neutral. All subscales have a positive mean score, except REAL. The IPQ scores of our setups are consistent with prior implementations for VR authentication~\cite{mathis-vr22}.}\label{fig:ipq}
\end{figure}

\subsection{Experience in the VR Game}
Participants’ interaction experiences also consist of their game experiences. We notice that participants perceived a high presence in the VR world and demonstrated high engagement in the game.


\paragraph{Feeling present in the VR world.} Using the IPQ presence questionnaire (Figure~\ref{fig:ipq}), we observe that participants had a positive rating on their presence in the VR world overall. At the same time, we do not see a noticeable difference in the IPQ scores between different authentication probes, which confirms that authentication is perceived as a secondary task~\cite{de-soups10}.

\paragraph{High engagement in the game.} We found that participants were highly engaged in the archery game and the routine payment. When describing their motivation in deciding on the arrows to pay for, participants mentioned the reasons for \textit{their strategy to compete} and \textit{enjoyment of the game}. Participants’ strategies are based on their confidence in the archery skill and a cost-benefit analysis. Among them, 19 explicitly mentioned that they would like to shoot as many arrows as they could, and six people said it was fun to play. However, participants’ archery performances differed a lot. They obtained a final score of 824.17 on average but had a large gap between the highest and the lowest (1470 and 190).



\section{RQ2: Security and Privacy Perception}



This section reveals how the interaction experiences we discussed in the previous section factor into participants’ security and privacy perception of authentication. 
We qualitatively analyze participants' responses and show five themes in Figure~\ref{fig:framework}. We then explain each aspect regarding participants' rating on the four authentication probes.




\subsection{Influences of Interaction on Perception}
\label{sec: Influences of Interaction on Perception}

We present five key insights that explain how the interaction factors influence participants' perception of authentication. We find that interaction with authentication mechanisms in VR serves as a bridge between participants’ perceptions of VR and their real-life experiences. However, contextual uncertainties in VR make understanding of threats and risks more difficult.

\paragraph{Realistic authentication interactions help users translate real-life perceptions into VR.} More than contributing to usability, realistic authentication interactions and interfaces bridge participants’ security and privacy perceptions. However, there is still a gap in fully mirroring users’ real-world sensations in VR, which reduces the perceived consent and security. For example, this gap reduces participants’ feeling of giving consent to pay when signing in VR.
\begin{displayquote}
\textit{``The sign-to-pay method was a bit hard to use, so I think I just tried to write something, and I felt less like providing my signature.''} (P10)
\end{displayquote}
Similarly, the same participant felt \tap not secure as they still thought \textit{``it's not my [their] real card but a card-like object''} (P10). 

Second, participants’ real-life experiences differed, manifesting in their perceptions in VR. 
For example, one user thought \sign is less reasonable as they \textit{``rarely signed to pay (only recently in the US…)’’} (P6). Meanwhile, others \textit{``naturally perceive it as giving my [their] consent.''} as they usually did in the physical world (P12).

In addition, participants sometimes evaluated the realism of authentication interaction based on how it is related to the context of payment. One participant noticed little difference in how the four authentication probes alert them for payments: 
\begin{displayquote}
\textit{``Similar to transactions in physical world i felt the need to be alerted for payments across all the methods.''} (P13)
\end{displayquote}
Another participant thought authentication by card-tapping in \tap is most specific to payment scenarios. 
\begin{displayquote}
\textit{``Grabbing the card and making the payment made it seem like an actual payment. The other 3 felt like they could have been anything.''} (P9)
\end{displayquote}

However, interactions that exactly mirror the physical world might not necessarily fulfill participants’ security and privacy needs. They expected \textit{``we could do more than real life with card''} in VR (P24) --  as mentioned in Section~\ref{sec:Components of Authentication Interaction}, P24 thought the virtual card could inform them of authentication results by changing its color. P10 felt more secure and alerted to pay when using the floating PIN pad of \pinf, which gives them a more personal view compared to the kiosk in \pink. 
\begin{displayquote}
\textit{``I think pin numbers are genuinely more secure, but I would vote more highly for the floating method because it seems a bit harder for others to see the numbers that I am putting in.''} (P10)
\end{displayquote}

\paragraph{Usability challenges and a lack of feedback in VR interactions make users feel insecure.} 
Usability challenges during interaction reduced realism of authentication mechanisms for participants. In addition, lack of proper feedback negatively impacts participants’ perception of consent in VR. Some participants assumed that the certain authentication mechanisms trade off security for usability %
e.g., accepting an inconsistent or even impersonated signature:
\begin{displayquote}
\textit{``The sign-to-pay felt the most insecure as people easily have access to my cheques and can probably fake in the VR world since the VR signatures were clearly less accurate than the real-world.''} (P5)
\end{displayquote}


Authentication mechanisms that involved interactions without apparent feedback made participants question their security despite good usability, e.g., unexpected behaviors in VR, e.g., \textit{``making accident payment''} (P15). In another example, P2 thought \tap seemed too easy without any warning: 
\begin{displayquote}
\textit{``Tap-to-pay seems more no-brainer, so it is better to give warnings.''} (P2)
\end{displayquote}
A subset of participants were willing to accept additional friction during interaction for security and transparency, including the shuffled PIN pad.


\paragraph{Users lack the understanding of VR authentication processes behind the interface.} 
Due to similar processes in VR, participants transferred their prior knowledge about physical-world authentication into understanding the security and privacy of VR authentication. Examples of the processes include possessing secret knowledge (PIN), that is \textit{``only known by me [the participant]''}, and using the shuffled PIN pad, which \textit{``gave me [the participant] some sense of security''} (P20, P17). Though participants' prior understanding could differ, for example, whether the PIN is personal compared to the signature:
\begin{displayquote}
\textit{``Sign-to-pay requires my signature, so it requires more information from me. The pin is not personal information, so the privacy concern is less.''} (P20)
\end{displayquote}

Meanwhile, participants also brought their knowledge in identifying the vulnerability of VR authentication. For example, PINs and signatures in VR are also prone to malware.
\begin{displayquote}
\textit{``Similar reason that pins and signatures are just exposed to the game or malware in the gaming system.''} (P24)
\end{displayquote}

Moreover, participants were aware of the differences in VR authentication’s backend compared to authentication in the physical world, although they shared similar interactions and interfaces. However, they felt difficult in assessing the inherent security and privacy of VR authentication, \textit{``without more information on how the payment actually works''} (P13). Some participants assumed VR used different mechanisms than the physical world. Some of them thought that there could be additional verification or certification steps for the virtual card in \tap.
\begin{displayquote}
\textit{``I don't know how the backend works, I assume it should quire a pre-certification process such as a cookie.''} (P3)
\end{displayquote}

As such, P3 also struggled to define a proper threat model:
\begin{displayquote}
\textit{``Ofc, to consider this deeply requires a solid definition about what adversary we are facing. E.g., a man-in-the-middle or ish.''} (P3)
\end{displayquote}


Last, some participants felt a lack of trust due to such ambiguity in VR authentication and the technology generally. For example, P12’s described how their negative perception of \tap’s security came from their distrust of VR.
\begin{displayquote}
\textit{``Tap-to-pay was the simplest, but it was way too simple to believe that the entire payment process behind the scene was dealt with as I wanted. I'm not sure if this is due to my distrust to the specific payment system, or just to the VR world, or both.''} (P12)
\end{displayquote}


\paragraph{Users associate uncertainties in the VR environment with potential threats.} 
Participants also lacked awareness of their VR environment. 
Some were not sure how the VR environment would cater to multiple users sharing the same space, and prevent users from performing malicious actions. P1 was afraid that he would expose his PINs to other users in the same VR space.
\begin{displayquote}
\textit{``If I was sharing my virtual space with others and they can see what I was typing then I would have gone for "Sign to pay" (although not so usable).''} (P1)
\end{displayquote}
Another concern with malicious users is that they could leverage their prior knowledge about a potential target from the physical world to launch attacks in VR:
\begin{displayquote}
\textit{``signing can be also copies by anyone who knows my signature in the real world.''} (P12)
\end{displayquote}

Participants were also concerned about attacks in VR that could be more imperceptible to them than the physical world, e.g., using an invisible terminal to phish users. 
\begin{displayquote}
\textit{``Tap-to-pay is the easiest, but I feel it not safe because I can easily touch it to an invisible system in the VR world.''} (P19)
\end{displayquote}
Attacks in VR may have different degrees of noticeability. For example, participants would not necessarily realize their PINs being observed \textit{``unlike [attackers] stealing a card''} (P6).

\paragraph{The gamified VR context alters users’ sensitivity to security and privacy.} 
Note that our context for user authentication is participants’ routine payment in a VR game. As described, participants were highly engaged in the game. However, such game context reduced some participants’ sensitivity to security and privacy. P7 argued that they did not feel losing privacy in a VR game compared to real-life transactions.  
\begin{displayquote}
\textit{``In the context of the game I didn't feel like giving away privacy. However if I were to imagine this with real transactions, then I'd feel like giving some of my privacy away, similar to every time I pay with something else than cash.''} (P7)
\end{displayquote}

The gamified interactions and interfaces of some authentication probes also made some participants feel less related to security and privacy.
For example, when signing a signature in VR, the feeling of playing a game overrode P7’s sense of giving consent.
\begin{displayquote}
\textit{``The sign to pay wasn't completely obvious you were actually paying for something, it could have been part of the game to have to write your name.''} (P7)
\end{displayquote}

Similarly, P10 thought \pinf \textit{``felt a bit too much like being in a game as well''} compared to \pink (P10).


\begin{figure}[h]
\centering
\includegraphics[width=\columnwidth]{charts/likert.pdf}
\caption{Overall security and privacy perception of the four probes. We color-coded the bars (red: strongly disagree, orange: disagree, grey: neither agree nor disagree, light blue: agree, dark blue: strongly agree)}\label{fig:overall_perception}
\end{figure}








\subsection{Overall Perception}
In Figure~\ref{fig:overall_perception}, we show participants’ overall rating of perception regarding the five aspects (consent, security, privacy, being alerted, and in control). Below we further explain their ratings.

\paragraph{Consent.} \pinf and \pink made more participants feel they gave consent compared with the \tap and \sign.
Compared to \tap, the higher level of user interaction in providing PINs contributed to this result. However, \tap still gave participants a decent feeling of giving consent, benefiting from the realistic interface. As mentioned before, the usability hurdle of signing in VR reduced participants’ feeling of providing consent using \sign.


\paragraph{Security.} Most participants associated PIN authentication, a familiar method, with security. 
Though participants were still concerned about shouldering-surfing attacks, some appreciated the shuffled PIN pad and the more personal PIN pad in \pinf. The response for \tap is polarized (many of them responded neither disagree nor agree). Participants lacked information to assess this method from the interaction. Some participants were afraid of unexpected behaviors; Others trusted \tap to have comparable security measures as the physical credit card. As mentioned for \sign, participants thought their VR signatures could be more easily forged as the system seemed to have a higher tolerance for accepting inconsistent signatures.

\paragraph{Privacy.} The participants were more in agreement than disagreement that \tap put their privacy less at risk. For those who felt they were losing privacy in \tap, they thought their name and card number rendered on the virtual card was a concern. In contrast, more participants associated drawing their signature with a privacy loss to either the payment authentication systems or bystanders. In the two PIN methods, participants were less concerned about their privacy when compared to \sign but not as confident as they are in \tap. This contradiction is due to whether the participants considered the secret PIN personal information.

\paragraph{Being alerted.} Participants’ responses to all four authentication setups are polarized for this aspect. In general, higher levels of interaction alerted users. Such polarized responses are due to how much attention the participants paid to compensate in matching their cautiousness on potential security risks. For example, some participants paid extra attention to \tap as it was too easy, while others did not.

\paragraph{In control.} Engagement in interaction smoothly and the sense of ownership made participants feel in control. For example, participants thought they were in control of \pinf and \pink the action of providing PIN is explicit and looks familiar. In contrast, several participants thought they did not feel in control with \tap, as the card did not seem too personal, and the interaction appeared frivolous. The barriers to drawing VR signatures made some participants feel out of control with \sign.




 









\section{RQ3: Meeting User Expectations}


In this section, we summarize how we can meet participants’ requirements and expectations for VR authentication from their preferences on our authentication probes, suggestions to improve the probes, and desired qualities of payment authentication. 

\begin{figure}[t]
\centering
\includegraphics[width=\columnwidth]{charts/rank.pdf}
\caption{Overall payment preference of the four probes. The bars are color coded dark blue, light blue, orange and red to indicate the fraction of users that selected each probe respectively to be their 1st, 2nd, 3rd and last choice.}\label{fig:payment_pref}
\end{figure}




We first show participants’ overall preference over the four probes. The majority of the participants selected \tap as their top priority and selected \sign as their last choice. \pinf and \pink, with little difference between themselves, follow \tap. This result directly reflects that participants prioritized the usability of authentication. Nevertheless, most participants also mentioned that they valued security and privacy, along with usability.

Next, we present four themes from users’ requirements for VR authentication. These themes mainly center around improvements in authentication interaction, which also improve participants’ perceived security and privacy.

\paragraph{Providing flexible interaction modalities to authenticate.} Participants expected interaction modalities that are more flexible and personalized to them. Such flexibility might make authentication in VR easier through better realism and accessibility. For example, some participants considered hand-gesture interaction an alternative to a controller. Moreover, the authentication system could support a variety of interactions inspired by real-world interaction.
For example, P14 mentioned the potential of swiping or inserting a card to pay other than tapping:
\begin{displayquote}
\textit{``There are also insert-card-to-pay and swipe-to-pay scenarios.''} (P14)
\end{displayquote}


\paragraph{Being creative in designing authentication interfaces for VR.} 
Participants were aware that mapping realistic authentication interactions into VR might reduce the learning curve in early adoption, especially for the elderly: 
\begin{displayquote}
\textit{``The users (especially elders) would hesitate to move onto this whole new experience if there is nothing that resembles their previous experience in the real world.''} (P12)
\end{displayquote}
Nevertheless, many of them expected the VR authentication interface to be creative instead of using an exact mapping. Such creativity may also improve usability, e.g., saving physical effort by \textit{``tap to pay on a floating panel''} (P23); or it may add functions regarding the usage in payment. For example, P20 desired a wallet interface to hold multiple virtual cards, and they could \textit{``select the card from a pop up UI''} (P20); Further, participants also anticipated emotional appeals (e.g., enjoyment) from the authentication interfaces, especially in the game context.
\begin{displayquote}
\textit{``I want the payment process to be cool and make me feel good. some visual effects could help and make me happy to make the payment.''} (P24)
\end{displayquote}
However, multiple participants expected such creativity and novelty in VR without brining specific ideas.
\begin{displayquote}
\textit{``However I do hope there are better ways to do than emulating real world payment system in the virtual world.''} (P7)
\end{displayquote}

\paragraph{Informing and offering feedback for VR authentication from multiple channels.} 
Many participants desired feedback and transparency in VR authentication for security and privacy. First, they expected feedback in the virtual world to increase their awareness of the VR surroundings and unexpected behaviors. For example, P21 suggests using a virtual mirror to inform them of their surroundings.
\begin{displayquote}
\textit{``For pin-to-pay, maybe add some mirrors to reflect the surrounding environment.''} (P21)
\end{displayquote}
Similarly, P24 wanted to \textit{``add a cancelation feature in case of an accidental grap-and-tap''} (P24). Also, VR authentication may leverage different modalities, e.g., visual effects and sound, to deliver feedback.

Moreover, participants demand transparency of VR authentication from the physical world. P8 wanted to be informed about suspicious activities through email even when they are in the physical world.
\begin{displayquote}
 \textit{``I could only become sure of its security depending on how the system notifies me w.r.t. suspicious activity (e.g., sending me an SMS/email/etc.)''} (P8)
\end{displayquote}
Meanwhile, service providers may open-source their protocol to authenticate users.
\begin{displayquote}
\textit{``Transparent: All the codes and methods behind should be well-defined and open-sourced.''} (P3)
\end{displayquote}


\paragraph{Adapting the security of authentication to the context of payment.} In Section~\ref{sec: Influences of Interaction on Perception}, we find that the VR game might reduce some participants’ sensitivity to security and privacy. Nevertheless, participants did expect security measures for different contexts of payment. One participant proposed that authentication can happen at launch time on device instead of a routine in the app -- \textit{``after authentication one can use any stored information to do payment (including card information).''} (P11)
Nevertheless, P8 stated that multi-factor authentication can be helpful when handling large payments.
\begin{displayquote}
\textit{``Tap-to-pay: ask the user to enter pin from time-to-time (or when the object in question is above a certain price) as an extra layer of authentication.''} (P8)
\end{displayquote}
Also, the system may offer additional security for shared usage e.g., to ensure \textit{``no accident payment from the kids''} (P20). 

In summary, we observe that most participants’ requirements centered around interaction experiences and the usability of authentication. Their requirements align with our findings in RQ2 -- how interaction experiences influence security and privacy perception. However, participants’ requirements could be non-specific, especially regarding security and privacy, due to their lack of understanding of VR. Sometimes, participants’ requirements might lead to conflicts, e.g., the need for feedback vs. the interaction effort, but participants rarely named options to resolve these conflicts.




