\begingroup\def\thefootnote{*}\footnotetext{The work was done while the authors were at Visa Research.}\endgroup

\section{Introduction}




Virtual reality (VR) systems immerse individuals in a digital world, one that simulates real-world interactions with objects and characters~\cite{burdea-vrt03}.
In addition to specialized use cases (e.g., military training and healthcare~\cite{rizzo-jcpms11}), VR technology is seeing widespread adoption in more routine settings, such as gaming, social interactions, shopping, and commerce~\cite{freeman-cscw21,speicher-imwut17,hock-chi17}. 
We are already in the early phase of a VR commerce boom, where service providers, such as retailers and banks, are creating virtual branches to serve users in popular VR worlds~\cite{lau_2022}.

VR systems access sensitive user data and assets, raising the need to authenticate users, especially for payment scenarios that empower the VR commerce boom~\cite{lau_2022}. VR service providers deploy user authentication methods borrowed from traditional platforms to verify users' identities, such as using passwords and personal identification numbers (PINs). However, there is limited understanding of how users perceive user authentication in VR, especially \textbf{how users’ interaction experience factors into their security and privacy perception of authentication in VR}. Recent research has shown that the context in which authentication is used (e.g., where and for what purpose) affects users' security and privacy perception. For example, users feel insecure when using an ATM in a crowded space~\cite{mathis-vr22}. VR presents a unique context where users interact with digital objects to perform routine activities, such as payment, which were once limited to conventional platforms. There is a need to understand how users perceive the security and privacy properties of payment authentication in the growing VR commerce ecosystem. With such understanding, we can guide the future design of authentication methods that are both secure and usable. Our paper provides this understanding by investigating the following research questions.


\begin{itemize}[noitemsep]

\item \textbf{RQ1 -- Interaction Experience:} What are the factors that contribute to users’ interaction experience of authentication in VR, and how?
\item \textbf{RQ2 -- Security and Privacy Perception:} How does users' interaction experience, in regards to the above factors, affect users’ security and privacy perceptions of authentication in VR?
\item \textbf{RQ3 -- Meeting User Expectations:} How can we meet users’ expectations on VR authentication that improve interaction experience, security, and privacy?

\end{itemize}


To answer these questions, we leverage \textit{technology probes}~\cite{hutchinson-chi03}, a method that leverages proof-of-concept interfaces to uncover hidden phenomena in user interaction, to study user authentication in VR. We designed four probes pertaining to authentication interactions in VR -- entering a PIN, tapping a virtual card, and signing a signature -- to evaluate of the user interaction experience and perceived security and privacy. 
These probes follow three interaction paradigms of authentication using something you know (e.g., PIN), something you have (e.g., token), and something you are (e.g., biometrics)~\cite{ross-nist19}. They resemble existing authentication mechanisms in both VR and the physical world~\cite{mathis-vr22}. 
We embed these probes in a routine payment interaction for users when they play a VR archery game, which is an ecologically valid context.


We conducted a user study with 24 participants and evaluated their experiences in the VR game with the probes. We qualitatively analyzed open-ended responses from the participants, which include 312 statements.  Our analysis is also supported by quantitative ratings, e.g., the overall usability of our designs.

Our analysis reveals these findings in response to the three research questions:
\begin{itemize} [noitemsep]
\item \textbf{RQ1:} Participants' interaction experiences with VR authentication were associated with multiple factors, including the usability characteristics of authentication. When participants adapted their real-world authentication experiences in performing VR authentication, some interaction factors in VR, such as participants' motion control and spatial awareness, posed unique usability challenges.

\item \textbf{RQ2:} Participants benefited from the realistic interactions in VR authentication in translating real-world perception of authentication into VR. However, participants faced the gap due to usability issues in VR, e.g., losing the sense of signing their signature. Also, the gamified VR context may reduce participants' sensitivity to security and privacy.  

\item \textbf{RQ3:} Participants’ expectations for VR authentication centered around improving the interaction factors, e.g., making the interface more engaging. Although their expectations were sometimes nonspecific and even appeared conflicting, e.g., the need for receiving feedback and a reduced interaction burden.


\end{itemize}


Based on our findings, we propose several recommendations to accommodate participants’ expectations for VR authentication, further resolving the gaps and conflicts. To reduce the uncertainties from the environment of VR authentication, we suggest service providers not only inform users but provide active access control to mitigate threats, e.g., an invisible phishing interface. Further, engaging interface designs can motivate users to understand the security properties of authentication. In addition, we discuss the open challenges and opportunities in implementing these recommendations, e.g., identifying the responsibility among stakeholders in VR and payment ecosystems. 


 


  

    










