\section{Conclusion}
We presented a technology probe study to investigate how interaction experiences in VR authentication affect users’ security and privacy perception. We designed four probes, using a PIN, a virtual card, and a signature, that represent the paradigms of user authentication. We embedded these probes in the routine payments of a VR archery game. In our user study, we collected participants’ responses using surveys regarding interaction experiences, security and privacy perceptions, and expectations for authentication. We revealed that VR interactions posed unique usability challenges, e.g., in controlling motion. These challenges also created difficulties when participants translated their real-life perceptions into VR. Participants' expectations centered around improving interaction factors. However, their expectations were sometimes nonspecific and conflicting. 

