%%%%%%%%%%%%%%%%%%%%%%%%%%%%%%%%%%%%%%%%%%%%%%%%%%%%%%%%%%%%%%%%%%%%%%%%%%%%%%%%
%2345678901234567890123456789012345678901234567890123456789012345678901234567890
%        1         2         3         4         5         6         7         8

\documentclass[journal,twoside,web]{ieeecolor}
%\documentclass[letterpaper, 10pt, conference]{ieeeconf}

%\IEEEoverridecommandlockouts
% This command is only needed if you want to use the \thanks command
%\overrideIEEEmargins
% See the \addtolength command later in the file to balance the column lengths
% on the last page of the document



% The following packages can be found on http:\\www.ctan.org
\usepackage{generic}
\usepackage{cite}
\usepackage{amsmath,amssymb,amsfonts}
%\usepackage[cal = pxtx, scr = dutchcal]{mathalpha}
\usepackage[scr = dutchcal]{mathalpha}
%\usepackage{bbold}
%\usepackage{algorithmic}
\usepackage{graphicx}
\usepackage{textcomp}
\usepackage{stmaryrd}
\usepackage{tabularx}
\usepackage{multirow}
\usepackage{float}
\usepackage{mathtools}

\usepackage{color}
\newcommand{\blue}[1]{\textcolor{blue}{#1}}

%\newcommand{\td}[1]{\tilde{#1}}
%\newcommand{\f}[2]{\frac{#1}{#2}}
%\newcommand{\LR}[1]{\left( #1 \right)}
%\newcommand{\LRs}[1]{\left[ #1 \right]}
\newcommand{\mbf}[1]{\mathbf{ #1}}
\newcommand{\tbf}[1]{\textbf{#1}}
\newcommand{\mbs}[1]{\boldsymbol{#1}}
\newcommand{\mbb}[1]{\mathbb{#1}}
\newcommand{\mcl}[1]{\mathcal{#1}}
\newcommand{\mscr}[1]{\mathscr{#1}}
\newcommand{\mfk}[1]{\mathfrak{#1}}
\newcommand{\R}{\mathbb{R}}
\newcommand{\N}{\mathbb{N}}
%\newcommand{\C}{\mathbb{C}}
\newcommand{\norm}[1]{\left\lVert{#1}\right\rVert}
\newcommand{\ip}[2]{\left\langle #1, #2 \right\rangle}


%\DeclareSymbolFont{bbsymbol}{U}{bbold}{m}{n}
%\DeclareMathSymbol{\PPi}{\mathalpha}{bbsymbol}{0213F}
%\newcommand{\PPi}{\text{\usefont{U}{bbold}{m}{n}$\mathbb{\Pi}$}}
%\MakeRobust{\PPi}

% Matrix structures
\newcommand{\bmat}[1]{\begin{bmatrix}#1\end{bmatrix}}
\newcommand{\pmat}[1]{\begin{pmatrix}#1\end{pmatrix}}
\newcommand{\smallbmat}[1]{\left[\scriptsize\begin{smallmatrix}
		#1\end{smallmatrix} \right]}
\newcommand{\mat}[1]{\begin{matrix}#1\end{matrix}}
\newcommand{\smallmat}[1]{\scriptsize\begin{smallmatrix}
		#1\end{smallmatrix}}
\newcommand{\lbmat}[1]{\left[\!\!\begin{array}{l}#1\end{array}\!\!\right]}
\newcommand{\rbmat}[1]{\left[\!\begin{array}{r}#1\end{array}\!\right]}
\newcommand{\srbmat}[1]{\small\left[\!\!\!\begin{array}{r}#1\end{array}\!\!\!\right]}
\newcommand{\slbmat}[1]{\small\left[\!\!\!\begin{array}{l}#1\end{array}\!\!\!\right]}
\newcommand{\srmat}[1]{\small\begin{array}{r}#1\end{array}}
\newcommand{\slmat}[1]{\small\begin{array}{l}#1\end{array}}

% Define theorem environments, and set counters
\newtheorem{thm}{Theorem}
\newtheorem{defn}[thm]{Definition}
\newtheorem{lem}[thm]{Lemma}
\newtheorem{prop}[thm]{Proposition}
\newtheorem{cor}[thm]{Corollary}
\newtheorem{remark}[thm]{Remark}
\newtheorem{example}[thm]{Example}

% Big brackets
\let\bl\bigl
\let\bbl\Bigl
\let\bbbl\biggl
\let\bbbbl\Biggl
\let\br\bigr
\let\bbr\Bigr
\let\bbbr\biggr
\let\bbbbr\Biggr

% Document specific commands
\newcommand{\dom}[2]{\text{I}_{#1}^{#2}}
\renewcommand{\vec}[1]{\bar{{#1}}}
\newcommand{\nvec}[1]{\bar{\text{#1}}}
\newcommand{\mvec}[1]{\mbf{\bar{#1}}}
\newcommand{\fvars}[3]{#1\rrbracket^{#2}_{#3}}
\newcommand{\fvarss}[3]{#1\rrbracket^{#2}_{#3}\thinspace}
\newcommand{\enn}[1]{\text{n}_{\text{#1}}}
\newcommand{\ttimes}[0]{\!\times\!}


% Define footnote without number
\newcommand\blfootnote[1]{%
	\begingroup
	\renewcommand\thefootnote{}\footnote{#1}%
	\addtocounter{footnote}{-1}%
	\endgroup
}

% Define block for equations
\floatstyle{ruled}
\newfloat{block}{thbp}{lop}
\floatname{block}{Block}

% Reduce size of equations...
%\newcommand*{\Scale}[2][4]{\scalebox{#1}{$#2$}}%
%\newcommand*{\Resize}[2]{\resizebox{#1}{!}{$#2$}}%
\newcommand\Scale[2]{\scalebox{#1}{\mbox{\ensuremath{\displaystyle #2}}}}
\newcommand\Resize[2]{\resizebox{#1}{!}{\mbox{\ensuremath{\displaystyle #2}}}}


\title{\LARGE \bf
	A Parameterization of Polynomials on Distributed States and a PIE Representation of Nonlinear PDEs
}



\author{Declan S. Jagt, Matthew M. Peet %
\thanks{\tbf{Acknowledgement:} This work was supported by National Science Foundation grant CMMI-1935453.} %
}% <-this % stops a space



\begin{document}
	
	
	
	\maketitle
	\thispagestyle{empty}
	\pagestyle{empty}
	
	
	%%%%%%%%%%%%%%%%%%%%%%%%%%%%%%%%%%%%%%%%%%%%%%%%%%%%%%%%%%%%%%%%%%%%%%%%%%%%%%%%
	\begin{abstract}		
		
		Partial Integral Equations (PIEs) have previously been used to represent systems of linear (1D) Partial Differential Equations (PDEs) with homogeneous Boundary Conditions (BCs), facilitating analysis and simulation of such distributed-state systems. In this paper, we extend these result to derive an equivalent PIE representation of scalar-valued, 1D, polynomial PDEs, with linear, homogoneous BCs. To derive this PIE representation of polynomial PDEs, we first propose a new definition of polynomials on distributed states $\mbf{u}\in L_2[a,b]$, that naturally generalizes the concept of polynomials on finite-dimensional states to infinite dimensions. We then define a subclass of distributed polynomials that is parameterized by Partial Integral (PI) operators. We prove that this subclass of polynomials is closed under addition and multiplication, providing formulae for computing the sums and products of such polynomials. Applying these results, we then show how a large class of polynomial PDEs can be represented in terms of distributed PI polynomials, proving equivalence of solutions of the resulting PIE representation to those of the original PDE. Finally, parameterizing quadratic Lyapunov functions by PI operators as well, we formulate a stability test for quadratic PDEs as a linear operator inequality optimization problem, which can be solved using the PIETOOLS software suite.
		We illustrate how this framework can be used to test stability of several common nonlinear PDEs.
		
		
%%%%%%%%%%%%%%%%%%%%%%%%%%%%%%%%%%%%%%%%%%%%%%%%%%%%%%%%%%%%%%%%%%%%%%%%%%%%%%%%%%%%%%%%%%%	
		
	\end{abstract}

%\blfootnote{\vspace*{-0.00cm}%
%\tbf{Acknowledgement:} This work was supported by National Science Foundation grant CMMI-1935453. \vspace*{-0.25cm}}
	
%%%%%%%%%%%%%%%%%%%%%%%%%%%%%%%%%%%%%%%%%%%%%%%%%%%%%%%%%%%%%%%%%%%%%%%%%%%%%%%%
	
\vspace*{-0.2cm}
\section{INTRODUCTION}

In this paper, we consider the problem of representation of polynomials on distributed-state systems and how this representation may be used to simplify analysis of nonlinear Partial Differential Equations (PDEs). Nonlinear PDEs are frequently used to model physical processes, including fluid dynamics (e.g. Navier-Stokes), population growth (e.g. Fisher's equation), and wave propagation (e.g. Korteweg-de Vries). However, we argue that the PDE representation of such nonlinear systems unnecessarily complicates analysis and simulation of solutions. For example, consider Burgers' equation, with Dirichlet Boundary Conditions (BCs):
\begin{align*}
	\dot{u}(t,s)&=u_{ss}(t,s)+ru(t,s) -u(t,s)u_{s}(t,s),	&	s&\in[0,1],	\\
	u(t,0)&=0,\qquad u(t,1)=0.
\end{align*}
To verify stability of this system, we can use the candidate Lyapunov Function (LF) $V(u)=\frac{1}{2}\|u\|_{L_2}^2\geq 0$, with the derivative satisfying $\dot{V}(u)=r\|u\|_{L_2}^2-\|u_{s}\|_{L_2}^2$ along solutions to the PDE. It can be proven that, for any $r<\pi^2$, the derivative of this LF satisfies $\dot{V}(u)\leq 0$, thus proving stability of the system~\cite{valmorbida2014semi}. However, verifying that $\dot{V}(u)\leq 0$ for $r\in (0,\pi^2)$ is not trivial, as it necessitates deriving some upper bound on the norm of $u$ in terms of the norm of its derivative $u_{s}$, invoking e.g. the Poincar\'e inequality. In this manner, the representation of the system dynamics as a (polynomial) function of $u$, $u_{s}$, and $u_{ss}$ complicates the task of verifying suitability of even a simple, fixed LF candidate $V(u)=\frac{1}{2}\|u\|^2_{L_2}$.

Similarly, suppose that we adjust the BCs, imposing e.g. a Neumann condition $u_{s}(t,1)=0$. In this case, despite the fact that neither the gradient $\nabla V(u)$ nor the expression for $\dot{u}(t)$ is changed, the derivative $\dot{V}(u)=\nabla V(u) \dot{u}$ no longer satisfies $\dot{V}(u)=r\|u\|^2_{L_2}-\|u_{s}\|_{L_2}^2$ along solutions to the system. More generally, it is unclear how the BCs affect stability properties of the system, and how we can account for them in testing fitness of any candidate LF.

Because of these difficulties associated with testing suitability of LF candidates for general PDEs, most prior work focuses only on limited classes of PDEs with specific BCs, proving results only for the system under consideration.
For example, extensive research has been done deriving stability conditions for Navier-Stokes equations~\cite{goulart2012SOS_stability_fluid,huang2015SOS_stability_fluid,ahmadi2019framework,fuentes2022SOS_stability_fluid}, commonly expanding solutions using e.g. a Galerkin basis, and proving decay of a LF using Sum-Of-Squares (SOS) techniques.
Similarly, stability of the Kuramoto-Sivanshinsky was studied in~\cite{goluskin2019Energy_Kuramoto_Sivashinsky}, assuming periodic BCs, and verifying negativity $\dot{V}(u)\leq 0$ of a quadratic LF using discretization.
However, these results for Navier-Stokes and Kuramoto-Sivanshisnky equations offer only limited insight into how to test stability of other nonlinear systems.

Prior work studying more general systems includes~\cite{fridman2016new}, deriving a Linear Matrix Inequality (LMI) stability test for a class of wave equations $u_{tt}=u_{ss}+f(u,s,t)$, assuming a bound $f_{u}(u,s,t)<g$ on the nonlinear term. % and using specific inequalities (e.g. Wirtinger's inequality) to prove negativity conditions for the derivative $\dot{V}$ of a quadratic LF.
Similarly, stability of classes of 2nd order, parabolic PDEs is analysed in~\cite{papachristodoulou2006SOS_stability_PDE,valmorbida2015stability,mironchenko2019ISS_nonlinear_PDE}, e.g. deriving polynomial positivity conditions for verifying stability of such systems. However, these results too are limited in their applications, and it is unclear how the proposed stability conditions may be tested in general.

%stability of the Kuramoto-Sivanshinsky ~\cite{goluskin2019Energy_Kuramoto_Sivashinsky}, focus only on Kuramoto–Sivashinsky equation with periodic BCs, is analysed using a quadratic LF, and verifying negativity $V(u)\leq 0$ using discretization
%~\cite{fridman2016new}, focus only on wave equations $u_{tt}=u_{ss}+f(u,s,t)$, with a priori bound $f_{u}(u,s,t)<g$ on function $f$. Then, use specific inequalities (e.g. Wirtinger) to derive LMI conditions for stability.
%~\cite{ahmadi2019framework}, focus only on Navier-Stokes Equation, using quadratic storage functionals  to derive polynomial positivity conditions for verifying input-output properties of a flow model.
%What about ~\cite{goulart2012SOS_stability_fluid,huang2015SOS_stability_fluid,fuentes2022SOS_stability_fluid}
%authors of ~\cite{papachristodoulou2006SOS_stability_PDE,mironchenko2019ISS_nonlinear_PDE} use integration by parts to derive stability conditions for a more expansive class of 2nd order, parabolic PDEs, but do not provide a framework for actually testing these conditions for general systems.

\textcolor{black}{
In order to derive a framework for testing LFs for more general nonlinear PDEs, we argue that the main obstacle currently prohibiting these efforts is the PDE representation itself.
In particular, as noted, the representation of the PDE dynamics in terms of a polynomial function of not only the state $u$ but also its derivatives $(u_{s},u_{ss},\hdots)$ makes it difficult to verify $\dot{V}(u)\leq 0$ for any candidate function $V(u)$, in general. Moreover, the presence of BCs in the PDE representation further complicates this analysis, as the LF need only satisfy $\dot{V}(u)\leq 0$ for $u$ satisfying these BCs. A more appropriate representation of distributed-state systems, then, satisfies the following two properties:\\
1. The system dynamics must be represented as a polynomial function of only a single distributed state $u$, and must parameterized in a manner that is sufficiently rich for most purposes, whilst still being amenable to theoretical analysis and numerical representation; 	\\
2. The BCs must be incorporated into the representation in a manner that eliminates the need for explicitly accounting for them in e.g. testing stability.
}

Regarding the second requirement, that of accounting for the BCs in the representation, a framework for doing so has already been developed in e.g.~\cite{shivakumar2022GPDE_Arxiv}, using Partial Integral (PI) operators. In particular, the authors of~\cite{shivakumar2022GPDE_Arxiv} show that, assuming the BCs to be suitably well-posed, there exists a unitary bijection between the state space $X$ of the PDE solutions, constrained by BCs (e.g. $u(t,0)=u(t,1)=0$) and continuity constraints (e.g. $u_{ss}\in L_2[0,1]$), and the \textit{fundamental state} space $L_{2}[0,1]$. More specifically, there exists a PI operator $\mcl{T}$, such that for any fundamental state $v\in L_2$, we can define an associated PDE state $u=\mcl{T}v\in X$. Using this relation, then, we can express the dynamics in terms of this fundamental state $v$, and analyse e.g. stability without having to explicitly account for any BCs.

%In particular, rather than considering the dynamics of the PDE state $u$, the authors of~\cite{shivakumar2022GPDE_Arxiv} consider the dynamics of a fundamental state $v$, which is free of the BCs and continuity constraints imposed upon the PDE state. Then, assuming the BCs to be sufficiently well-posed, there exists a PI operator $\mcl{T}$ such that $u=\mcl{T}v$. Substituting this relation into the PDE, the authors show that there exists an equivalent representation of the system as a Partial Integral Equation (PIE), that is free of the BCs and continuity constraints inherent to the PDE representation. Moreover, there exists a unitary map between solutions to the PDE and solutions to the PIE, allowing e.g. stability of the PDE to be verified in the PIE representation.

%However, this still leaves the challenge of actually parameterizing the nonlinear system dynamics, which is the focus of this paper. In particular, we are interested in systems with dynamics of the form $\dot{u}=f(u)$ wherein $f$ is ``polynomial'' function of the state. Here, we say ``polynomial'' in quotation marks, as in fact, we have no rigorous definition of what even constitutes a polynomial on distributed state $u\in L_2$.

The goal of this paper, then, will be to tackle the first challenge with representation of distributed-state systems: that of parameterizing the nonlinear system dynamics. For example, consider a 2nd order PDE with dynamics of the form $\dot{u}=f(u)$, where $f$ is polynomial of degree at most $d$ in $u$. How might we parameterize a class of polynomial functions $f$ on distributed states $u$? An obvious parameterization of $f$ might be in terms of the monomial vector $Z_{d}(u,u_{s},u_{ss})$, uniquely representing $f(u)=c^T Z_{d}(u,u_{s},u_{ss})$ by a vector of coefficients $c\in\R^{{3+d}\choose{d}}$. However, we argue that this representation is both insufficiently general, and unnecessarily complex. Indeed, the space of polynomials that can be represented in this manner is only of dimension ${3+d}\choose{d}$. Seeing as the distributed state $u$ is infinite-dimensional, this hardly seems like a sufficiently broad class of polynomials. On the other hand, the representation is also too complex in that the monomial basis $Z_{d}(u,u_{s},u_{ss})$ is defined in terms of not only the state $u$, but also its derivatives $u_{s}$ and $u_{ss}$ -- states that are clearly not independent. A more natural basis for the space of distributed polynomials should be defined by only a single distributed state $v$.

In order to parameterize polynomial PDEs, then, we first have to determine how we actually define polynomials on distributed states. Specifically, we have to define a monomial basis for this space of polynomials, extending the concept of monomials $Z_{d}(x)$ for finite-dimensional states $x\in\R^{n}$ to infinite-dimensional states $u\in L_2$.
Moreover, once we have such a basis, we have to define a suitable class of operators $\mcl{C}$, so that general polynomials $p$ on distributed states $u$ can be represented in the linear form $p(u)=\mcl{C}Z_{d}(u)$. This class of operators should be sufficiently general to allow common polynomial vector fields to be represented in the linear format, but sufficiently narrow so as to admit a reasonable parameterization.
Finally, once we have defined an appropriate class of polynomials on distributed states, we have to derive formulae for representing PDEs in terms of such distributed polynomials, using the PI framework to incorporate the BCs.

%For a finite-dimensional state $x\in\R^{n}$, we can parameterize polynomials $p(x)=c^T Z_{d}(x)$ by coefficients $c\in\R^{{n+d}\choose{d}}$, where $Z_{d}=\bmat{1&x_{1}&x_{2}&x_{1}x_{2}&\hdots&x_{n}^{d}}^T$ is a vector of monomials in $x$. An obvious representation of polynomials in $u\in L_2$ would then be $p(u)(s)=c^T Z_{d}(u)(s)$, where now $Z_{d}(u)(s)=\bmat{1&u(s)&u(s)^2&\hdots&u(s)^d}$. However, closer inspection quickly reveals that this representation is profoundly limiting, and ignores the distributed, high-dimensional nature of $u$. Specifically, this representation fails to include terms such as $u(s_1)u(s_2)$ for distinct $s_1,s_2$, which amounts to excluding cross terms such as $x_1x_2$ in the representation of finite-dimensional polynomials $p(x)$. In fact, the dimension of the set of polynomials that can be represented as $p(u)(s)=c^T Z_{d}(u)(s)$ is only $d+1$, whereas in practice, the set of polynomial functions of $u\in L_2$ should be uncountable. Clearly, then, we require a new method for parameterization of polynomials on distributed states.
%second, since the class of polynomials on $L_2$ is bound to be infinite-dimensional, how do we parameterize a class of polynomials in a way that is sufficiently general but not too complex
%Third, how may we use this representation

In the remainder of this paper, we address each of these challenges, proposing a standardized representation of polynomials on distributed states, and combining this representation with the PI framework to derive a new representation of polynomial PDEs. In particular, in Section~\ref{sec:polynomials}, we propose a linear representation of polynomials on distributed states, and parameterize a subclass of these polynomials by PI operators. In Section~\ref{sec:polynomial_operations}, we then show that this subclass of polynomials is closed under addition and multiplication, satisfying the standard properties we expect of polynomials. Using these properties, in Section~\ref{sec:PIE}, we then prove that we can represent a general class of nonlinear PDEs in terms of such polynomials, presenting formulae for converting these PDEs to equivalent Partial Integral Equations (PIEs). Finally, in Section~\ref{sec:stability}, we illustrate how this PIE representation facilitates the derivation of stability tests for nonlinear PDEs, proposing a stability test for quadratic systems, and applying this test to verify stability of several examples.


	
\section{Notation}\label{sec:notation}

For a given domain $\Omega\subset\R^d$, let $L_{0}^{n}[\Omega]$ and $L_2^n[\Omega]\subseteq L_{0}[\Omega]$ denote the sets of all $\R^n$-valued measurable and square-integrable functions on $\Omega$, respectively, where we omit the domain when clear from context. For $k\in\N$, define Sobolev subspaces $H_{k}^n[\Omega]$ of $L_2$ for $\Omega\subseteq\R^{d}$ as
\begin{align*}
	&H_{k}^{n}[\Omega]\!=\!\bl\{\mbf{v}\!\in\! L_2^{n}[\Omega] \mid \partial_s^{\alpha}\mbf{v}\!\in\! L_2^n[\Omega],\ \forall \alpha \in\N^{d}:\! \|\alpha\|_{\infty}\!\leq k\br\},
\end{align*}
where $\|\alpha\|_{\infty}=\max_{i\in\{1,\hdots,d\}}\|\alpha_{i}\|$ and where we write $\partial_{s}^{\alpha}\mbf{v}=\frac{\partial^{\alpha_1}}{\partial s_1^{\alpha_1}}\cdots\frac{\partial^{\alpha_d}}{\partial s_d^{\alpha_d}}\mbf{v}$.
For $d\in\N$, let $\mbs{\theta}_{(1:d)}:=(\theta_{1},\hdots,\theta_{d})$. Accordingly, for a sufficiently integrable function $\mbf{w}$ on a uniform, rectangular domain $[a,b]^{d}$, write
\begin{equation*}
	\int_{a}^{b}\mbf{w}(\mbs{\theta}_{(1:d)}) d\mbs{\theta}_{(d:1)}
	= \int_{a}^{b}\!\hdots\!\int_{a}^{b}\mbf{w}(\theta_{1},\hdots,\theta_{d})d\theta_{d}\hdots d\theta_{1}.
\end{equation*}




\section{A Linear Representation of Polynomials on a Distributed State}\label{sec:polynomials}

In this section, we propose a new, compact, operator-based, linear representation of polynomials on a distributed state. In constructing this representation, we attempt to mirror the linear representation of polynomials on a finite-dimensional state. In particular, we note that any polynomial $p(x)$ of degree $d\in\N$ on a state $x\in\R^{n}$ is uniquely defined by a coefficient vector $c$ and monomial vector $Z_{d}(x)$ as
\begin{equation*}
	p(x)=c^T Z_{d}(x).
\end{equation*}
Accordingly, to define a linear representation of polynomials $p(\mbf{u})$ on a distributed state $\mbf{u}\in L_2$,  we first propose a vector of monomials $Z_{\otimes d}(\mbf{u})$ in the following subsection. In the next subsection, we then define a set of bounded, linear operators $\mcl{C}$ acting on the monomials $Z_{\otimes d}(\mbf{u})$, so that we may represent a general class of polynomials on $\mbf{u}$ as
\begin{equation*}
	p(\mbf{u})(s)=\mcl{C} Z_{\otimes d}(\mbf{u}).
\end{equation*}
In Section~\ref{sec:polynomial_operations}, we show that this class of polynomials is indeed closed under addition and multiplication. %In Section~\ref{sec:polynomial_operations}, this representation of polynomials is then used to represent nonlinear PDEs.



\subsection{Monomials on a Distributed State}\label{sec:polynomials:monomials}

In this subsection, we propose a definition of monomial functions on distributed states $\mbf{u}\in L_2$, that will allow us to define a general class of polynomials on $\mbf{u}$ as linear combination of these functions. To motivate this definition, let us consider first a discretized state $x=\bmat{x_1&\hdots&x_{n}}^T=\bmat{\mbf{u}(s_1)&\hdots&\mbf{u}(s_n)}^T\in\R^{n}$. How do we define a monomial basis $Z_{d}(x)$ for polynomials of degree $d\in\N$ on the variables $x\in\R^{n}$? This monomial basis can be decomposed as $Z_{d}(x)=\bmat{Z^{h}_{0}(x)^T&\hdots &Z^{h}_{d}(x)^T}^T$, where each vector $Z^{h}_{k}(x)$ consists of the homogeneous monomials of degree exactly $k$ in $x$. Using the Kronecker product $(x\otimes x)_{ij}=x_{i}x_{j}$, we may express these monomials as
%\begin{align*}
%	Z^{h}_{0}(x)&=1,	\hspace*{1.5cm}
%	Z^{h}_{1}(x)=(x_{i})_{i\in\{1,\hdots,n\}} =x,	\\
%	Z^{h}_{2}(x)&=(x_{i}x_{j})_{i,j\in\{1,\hdots,n\}}=x\otimes x,\\
%	Z^{h}_{3}(x)&=x\otimes x\otimes x,\\
%&\vdots\\
%Z^{h}_{d}(x)&=x^{\otimes d},
%\end{align*}
%\begin{align*}
%	Z^{h}_{0}(x)&=1,	&
%	Z^{h}_{1}(x)&=x,	\\
%	Z^{h}_{2}(x)&=x\otimes x,	&
%	Z^{h}_{3}(x)&=x\otimes x\otimes x,\\
%	&\vdots\\
%	Z^{h}_{d}(x)&=x^{\otimes d},
%\end{align*}
\begin{equation*}
	Z_{d}(x)=\slbmat{Z^{h}_{0}(x)\\ Z^{h}_{1}(x)\\ Z^{h}_{2}(x)\\ Z^{h}_{3}(x)\\~\vdots\\ Z^{h}(x)}=\slbmat{1\\ x\\ x\otimes x \\ x\otimes x\otimes x\\~\vdots\\ x^{\otimes d}},
\end{equation*}
where $x^{\otimes d}$ denotes the $d$th power Kronecker product of $x$ with itself. Consider now how we may abstract these monomials to infinite dimensions. Extending the vector $x\in\R^{n}$ to a function $\mbf{u}\in L_2[a,b]$, we now have that $\mbf{u}(s_i)$ and $\mbf{u}(s_j)$ are independent variables for any $s_i,s_j\in [a,b]$. Thus whereas in finite dimensions, the independent variables are indexed by $i\in\{1,\hdots,n\}$, for functions on $\mbf{u}(s)$, every point $s\in[a,b]$ defines an independent variable. Accordingly, the degree one monomials in infinite dimensions would simply be the function $\mbf{u}(s)$ where we have made the continuum extension $x_i\mapsto \mbf{u}(s)$. Likewise, for the monomials of degree 2, the Kronecker product $(x\otimes x)_{ij}=x_{i}x_{j}$ has a natural equivalent in infinite-dimensions defined by the tensor product $(\mbf{u}\otimes\mbf{u})(s,\theta)=\mbf{u}(s)\mbf{u}(\theta)$. We denote this tensor product by $Z^{h}_{\otimes2}(\mbf{u}):=\mbf{u}\otimes \mbf{u} \in L_2[[a,b]^2]$ so that $(Z^{h}_{\otimes2}(\mbf{u}))(s,\theta)=(\mbf{u}\otimes \mbf{u})(s,\theta)=\mbf{u}(s)\mbf{u}(\theta)$ and $Z^{h}_{\otimes2}(\mbf{u})$ is the functional equivalent of the degree 2 monomial basis. More generally, we define the homogeneous degree $d$ \textit{distributed monomial} $Z^{h}_{\otimes d}(\mbf{u})\in L_2[[a,b]^{d}]$ as
\begin{equation*}
	Z^{h}_{\otimes d}(\mbf{u})(\mbs{\theta}_{(1:d)}):=\mbf{u}^{\otimes d}(\mbs{\theta}_{(1:d)})=\mbf{u}(\theta_{1})\cdots\mbf{u}(\theta_{d}),
\end{equation*}
%where $\mbf{u}^{\otimes d}=\overbrace{\mbf{u}\otimes\cdots\otimes\mbf{u}}^{d \text{ factors}}$, and 
where now $\mbs{\theta}_{(1:d)}=(\theta_1,\cdots,\theta_d) \in [a,b]^d$ becomes a set of $d$ continuum indices.
%\begin{align*}
%	{Z}^{h}_{\otimes 0}(\mbf{u})&=1,	\hspace*{1.5cm}
%	{Z}^{h}_{\otimes 1}(\mbf{u})=(\mbf{u}(s))_{s\in[a,b]}=\mbf{u},	\\
%	{Z}^{h}_{\otimes 2}(\mbf{u})&=(\mbf{u}(s)\mbf{u}(\theta))_{s,\theta\in[a,b]}=\mbf{u}\otimes\mbf{u}\\
%	{Z}^{h}_{\otimes 3}(\mbf{u})&=\mbf{u}\otimes\mbf{u} \otimes \mbf{u}\\
%&\vdots\\
%	{Z}^{h}_{\otimes d}(\mbf{u})&=\mbf{u}^{\otimes d}
%\end{align*}
%More generally, writing $\mbf{u}^{\otimes k}=\overbrace{\mbf{u}\otimes\cdots\otimes\mbf{u}}^{k \text{ factors}}$ for $k\in\N$, we define the \textit{distributed monomial} of power $k$ in $\mbf{u}$ as
%\begin{align*}
%	Z^{h}_{\otimes k}(\mbf{u})(\mbs{\theta}_{(1:k)})&\!:=\mbf{u}^{\otimes k}(\mbs{\theta}_{(1:k)})=\mbf{u}(\theta_{1})\cdots\mbf{u}(\theta_{k}),	&
%	\theta_{i}\!\in\![a,b].
%\end{align*}
Finally, the distributed monomial basis for polynomials of degree at most $d\in\N$ in $\mbf{u}$ becomes
\begin{equation*}\Resize{\linewidth}{
	Z_{\otimes d}(\mbf{u})(\mbs{\theta}_{(1:d)})\!:=\!\slbmat{Z^{h}_{\otimes 0}(\mbf{u})\\Z^{h}_{\otimes 1}(\mbf{u})(\theta_{1})\\Z^{h}_{\otimes 2}(\mbf{u})(\theta_{1},\theta_{2})\\Z^{h}_{\otimes 3}(\mbf{u})(\theta_{1},\theta_{2},\theta_{3})\\~\vdots\\Z^{h}_{\otimes d}(\mbf{u})(\mbs{\theta}_{(1:d)})}
	\!=\!\slbmat{1\\\mbf{u}(\theta_{1})\\(\mbf{u}\otimes\mbf{u})(\theta_{1},\theta_{2})\\(\mbf{u}\otimes\mbf{u}\otimes\mbf{u})(\theta_{1},\theta_{2},\theta_{3})\\~\vdots\\\mbf{u}^{\otimes d}(\mbs{\theta}_{(1:d)})}.}
\end{equation*}
We define an associated function space as
\begin{equation*}
	L_{2}[[a,b]^{(0:d)}]:=\R\times L_2[a,b]\times\cdots\times L_2[[a,b]^{d}],
\end{equation*} 
so that $Z_{\otimes d}(\mbf{u})\in L_{2}[[a,b]^{(0:d)}]$ for any $\mbf{u}\in L_2[a,b]$.

\subsection{Polynomials on a Distributed State}
Having defined homogeneous and non-homogeneous monomials on a distributed state $\mbf{u}\in L_2$, we may now define a polynomial on $L_2$ as any function that can be represented in terms of a linear operator acting on such monomials.
\begin{defn}[Distributed Polynomial]\label{defn:distributed_polynomial}
For given $d\in\N$, we say that a function $p:L_2[a,b] \rightarrow \R$ is a distributed homogeneous polynomial of degree $d$ if there exists a bounded linear operator $\mcl{C}: L_2[[a,b]^{(0:d)}] \rightarrow \R$ such that
\begin{equation*}
	p(\mbf{u})=\mcl{C}Z^{h}_{\otimes d}(\mbf{u}).
\end{equation*}
More generally, we say that $p$ is a (non-homogeneous) distributed polynomial of degree $d$ if
$p(\mbf{u})=\sum_{k=0}^d p_k(\mbf{u}),$
where each $p_k$ is a homogeneous polynomial of degree $k$. We may extend this definition to function-valued polynomials on a domain $\Omega\subseteq\R^{n}$, where in this case $p:\Omega\times L_2[a,b] \rightarrow \R$ is a polynomial of degree $d$ if for every $s\in\Omega$, $p(s):L_2[a,b] \rightarrow \R$ is a distributed polynomial of degree $d$.
\end{defn}

\paragraph*{\tbf{Example}} Suppose $\mbf{u}\in H_{1}[0,1]$. Then, the functions \\[-1.6em]
\begin{align*}
	p_{1}(\mbf{u})(s)&=\mbf{u}(s)^2,	&%\hspace*{0.5cm}
	p_{2}(\mbf{u})(s)&=\int_{s}^{1}\!\!\int_{0}^{\theta} \mbf{u}(\theta)\mbf{u}(\eta)\thinspace d\eta d\theta, \notag\\
	p_{3}(\mbf{u})(s)&=\mbf{u}_{s}(s)\mbf{u}(s),	&%	\notag\\
	p_{4}(\mbf{u})(s)&=\mbf{u}(s)\mbf{u}(1-s),
\end{align*}
are all polynomials of degree $2$ in $\mbf{u}$. Indeed, for each $i$ we may express $p_{i}(\mbf{u})=\mcl{C}_{i}(\mbf{u}\otimes\mbf{u})$, where for $\mbf{w}\in H_{1}[[0,1]^2]$
{\small
\begin{align*}
	(\mcl{C}_{1}\mbf{w})(s)&=\Delta_{\theta}^{s}\mbf{w}(s,\theta),	&
	(\mcl{C}_{2}\mbf{w})(s)&=\int_{s}^{1}\!\!\int_{0}^{\theta}\mbf{w}(\theta,\eta)d\eta d\theta,	\\
	(\mcl{C}_{3}\mbf{w})(s)&=\Delta_{\theta}^{s}\partial_{s}\mbf{w}(s,\theta),	&
	(\mcl{C}_{4}\mbf{w})(s)&=\Delta_{\theta}^{(1-s)}\mbf{w}(s,\theta).
\end{align*}}
where we define $\Delta_{\theta}^{s}\mbf{u}(\theta)=\mbf{u}(s)$ for $\mbf{u}\in H_1$.


\subsection{A Class of Operators on Distributed Monomials}\label{sec:polynomials:coefficients}

In the previous subsections, we proposed a monomial vector $Z_{\otimes d}(\mbf{u})$ on distributed states $\mbf{u}$, and defined a polynomial in $\mbf{u}$ as any function that can be expressed as $p(\mbf{u})=\mcl{C}Z_{\otimes d}(\mbf{u})$ for some bounded linear operator $\mcl{C}$. In this subsection, we paramterize a subspace of such polynomials by integral operators $\mcl{C}$. To motivate this parameterization, consider first how we define polynomials on finite-dimensional states $x\in\R^n$. In the linear representation, a homogeneous polynomial of degree $d$ in $x$ is defined by the homogeneous monomial basis $Z^{h}_{d}(x)$ and a vector of coefficients $c$ as
\begin{equation*}
	p(x)=c^TZ^{h}_{d}(x) = \sum_{j_1=1}^{n}\cdots\sum_{j_d=1}^{n} c_{j_1\hdots j_d} \thinspace x_{j_1}\cdots x_{j_d}
\end{equation*}
{Abstracting this to infinite-dimensions, the indices $j_1,\hdots,j_{d}\in\{1,\hdots,n\}$ become variables $\theta_{1},\hdots,\theta_{d}\in[a,b]$, and the vector of coefficients becomes a function $c_{j_{1}\hdots j_{d}}\mapsto C(\theta_{1},\hdots,\theta_{d})$. Similarly, the monomials map to $x_{j_1}\cdots x_{j_{d}}\mapsto \mbf{u}(\theta_{1})\cdots \mbf{u}(\theta_{d})=\mbf{u}^{\otimes d}(\mbs{\theta}_{(1:d)})=Z^{h}_{\otimes d}(\mbs{\theta}_{(1:d)})$. Finally, replacing the sums in $j$ with integrals with respect to $\theta$, we obtain a linear parameterization of homogeneous polynomials on a distributed state $\mbf{u}\in L_2[a,b]$ as}
\begin{equation*}\Resize{\linewidth}{
	p(\mbf{u})\!=\!\mcl{C}Z^{h}_{\otimes d}(\mbf{u})	
	\!=\!\int_{a}^{b}\!\!\!\cdots\!\int_{a}^{b}\!\! C(\theta_{1},...,\theta_{d}) \mbf{u}(\theta_{1})\cdots\mbf{u}(\theta_{d})d\theta_{d}\;...\; d\theta_{1}, }
\end{equation*}
where we define the integral operator $\mcl{C}:L_2[[a,b]^{d}]\to \R$ as
\begin{equation*}\Scale{0.9}{
	\mcl{C}\mbf{w}=\int_{a}^{b}\!\cdots\!\int_{a}^{b}C(\mbs{\theta}_{(1:d)}) \mbf{w}(\mbs{\theta}_{(1:d)}) d\mbs{\theta}_{(d:1)}. }
\end{equation*}
Of course, any non-homogeneous polynomial is simply a sum of homogeneous polynomials and hence, to parameterize non-homogeneous polynomials on a distributed state, we may use the form
\begin{equation*}\Resize{\linewidth}{
	p(\mbf{u})=\sum_{k=0}^{d}\mcl{C}_{k}Z^{h}_{\otimes k}(\mbf{u})
%	&=C_{0}+\sum_{k=1}^d\int_{a}^{b}\!\cdots\!\int_{a}^{b}C_{k}(\theta_{1},\hdots,\theta_{k})\mbf{u}(\theta_{1})\cdots\mbf{u}(\theta_{k}) d\theta_{k}\cdots d\theta_{1}
	\!=\!C_{0}\!+\!\sum_{k=1}^{d}\int_{a}^{b}\!\! C_{k}(\mbs{\theta}_{(1:k)}) \mbf{u}^{\otimes k}(\mbs{\theta}_{(1:k)}) d\mbs{\theta}_{(k:1)}. }
\end{equation*}
%\begin{defn}
%	For given $d\in\N$, we say that a function $p:L_2[a,b] \rightarrow \R$ is a distributed homogeneous PI polynomial of degree $d$ if there exists a function $C\in L_2[[a,b]^d]$ such that
%	\begin{equation*}
	%		p(\mbf{u})=\int_{a}^{b}\cdots\int_{a}^{b} C(\theta_{1},\hdots,\theta_{d})\mbf{u}^{\otimes d}(\theta_{1},\hdots,\theta_{d})d\theta_{d} \cdots d\theta_{1}.
	%	\end{equation*}
%	More generally, we say that $p$ is a (non-homogeneous) PI polynomial of degree $d$ if
%	$p(\mbf{u})=\sum_{k=0}^d p_k(\mbf{u}),$
%	where each $p_k$ is a homogeneous PI polynomial of degree $k$. 
%\end{defn}
In this manner, we can parameterize scalar-valued polynomials $p:L_2[a,b]\to\R$ on distributed states by $d$ kernel functions $C_{k}\in L_2\bl[[a,b]^k\br]$, and a constant $C_{0}\in\R$. To parameterize function-valued polynomials $p:[a,b]\times L_2[a,b]\to\R$, then, we introduce an additional variable $s\in[a,b]$ to the kernels, defining operators $\mcl{C}_{d}$ on $\mbf{w}\in L_2[[a,b]^{d}]$ as
\begin{align*}
	(\mcl{C}_{d}\mbf{w})(s)=\int_{a}^{b}\!\cdots\!\int_{a}^{b}C_{d}(s,\mbs{\theta}_{(1:d)}) \mbf{w}(\mbs{\theta}_{(1:d)}) d\mbs{\theta}_{(d:1)}.
\end{align*}
%Using these operators, we may parameterize a class of function-valued polynomials in a linear format as
%\begin{align*}
%	p(\mbf{u})(s)&=\sum_{k=0}^d\bl(\mcl{C}_{k}Z^{h}_{\otimes k}(\mbf{u})\br)(s)
%	=C_{0}(s) +\sum_{k=1}^{d}\bl(\mcl{C}_{k}\mbf{u}^{\otimes k}\br)(s).
%	%	&=C_{0}(s)+\sum_{k=1}^d\int_{a}^{b}\!\cdots\!\int_{a}^{b}C_{k}(s,\theta_{1},\hdots,\theta_{k})\mbf{u}(\theta_{1})\cdots\mbf{u}(\theta_{k}) d\theta_{k}\cdots d\theta_{1}
%%	&=C_{0}(s)+\sum_{k=1}^{d}\int_{a}^{b}C_{k}(s,\mbs{\theta}_{(1:k)}) \mbf{u}^{\otimes k}(\mbs{\theta}_{(1:k)})\thinspace d\mbs{\theta}_{(k:1)}
%\end{align*}
%Here, we require the kernel functions $C_{k}(s,\mbs{\theta}_{(1:k)})$
%This leaves the task of fixing a class of kernel functions $C_{k}(s,\mbs{\theta}_{(1:k)})$ to parameterize the integral operators $\mcl{C}_{k}$. Naturally, we require these functions to be square-integrable with respect to $\mbs{\theta}_{(1:k)}$, imposing $C_{k}\in L_{0,2}\bl[[a,b]^{k+1}\br]$ where
Here, we require the kernel function $C_{d}(s,\mbs{\theta}_{(1:d)})$ to be square-integrable with respect to $\mbs{\theta}_{(1:d)}\in[a,b]^{d}$, imposing $C_{d}\in L_{0,2}\bl[[a,b]^{d+1}\br]$ where
\begin{equation*}\Scale{0.9}{
	L_{0,2}\bl[[a,b]^{d+1}\br]
	\!:=\!\bbl\{C\!\in\! L_{0}[[a,b]^{d+1}] \thinspace\bbr\rvert\thinspace C(s)\!\in\! L_2[[a,b]^{d}],\thinspace\forall s\in[a,b]\bbr\}.}
\end{equation*}
%{\small%
%\begin{align*}
%	&L_{0,2}\bl[[a,b]^{k+1}\br]
%	\!:=\!\bbl\{C\in L_0\bl[[a,b]^{k+1}\br] ~\bbr\rvert~\\
%	&\hspace*{2.5cm} \smallint_{a}^{b} |C(s,\mbs{\theta}_{(1:k)})|^2 d\theta_{j}<\infty,\enspace \forall j\in\{1,\hdots,k\} \bbr\},
%\end{align*}}
In addition, in order for us to incorporate the polynomials defined by the operators $\mcl{C}_{k}$ into the representation of Partial Integral Equations (PIEs) in Section~\ref{sec:PIE}, we allow the kernel functions $C_{k}\in L_{0,2}$ to define Partial Integral (PI) operators $\mcl{C}_{k}$. To represent such kernels, we first define indicator functions on $s,\theta\in[a,b]$ as
\begin{align*}
	\mbf{I}_{1}(s,\theta)&:=\begin{cases}
		1,	&	\theta< s,	\\
		0,	&	\theta\geq s,
	\end{cases}	&
	\mbf{I}_{2}(s,\theta)&:=\begin{cases}
		0,	&	\theta\leq s,	\\
		1,	&	\theta> s,
	\end{cases}
\end{align*}
so that $R(s,\theta)=R(s,\theta)\mbf{I}_{1}(s,\theta)+R(s,\theta)\mbf{I}_{2}(s,\theta)$ for any $R\in L_{0,2}[a,b]$. We also use $\mbf{I}_{0}(s,\theta)=\delta(s-\theta)$ to denote the Dirac delta, so that for $\mbf{u}\in H_{1}[a,b]$,
\begin{align*}
	\int_{a}^{b}\mbf{I}_{0}(s,\theta)\mbf{u}(\theta)d\theta=\int_{a}^{b}\delta(s-\theta)\mbf{u}(\theta)d\theta = \mbf{u}(s).
\end{align*}
Using the functions $\mbf{I}_{j}$ on $[a,b]$, we can then define indicator functions on $[a,b]^{d}$ as
\begin{align*}
	\mbf{I}_{\mbs{j}}(s,\mbs{\theta}_{(1:d)})&:=\mbf{I}_{j_1}(s,\theta_{1})\cdots\mbf{I}_{j_d}(s,\theta_{d}),	&
	\mbs{j}&\in\{0,1,2\}^{d}.
\end{align*}
Finally, we define a class of PI operators as follows.
\begin{defn}[$\Pi_{3^{d}}$]
	For $d\in\N$, let
	{\small
	\begin{align*}
		&\mcl{N}_{3^{d}}[a,b]:=\bbl\{R ~\bbr\rvert~ R(s,\mbs{\theta}_{(1:d)})=\!\!\!\!\!\!\sum_{\mbs{j}\in\{0,1,2\}^{d}}\!\!\!\!\!\mbf{I}_{\mbs{j}}(s,\mbs{\theta}_{(1:d)})R_{\mbs{j}}(s,\mbs{\theta}_{(1:d)}),\\
		%&\exists\thinspace R_{\mbs{j}}\in L_{0,2}\bl[[a,b]^{d+1}\br]:\\
		&\hspace*{3.5cm} R_{\mbs{j}}\in L_{0,2}\bl[[a,b]^{d+1}\br],\enspace \mbs{j}\in\{0,1,2\}^{d} %R(s,\mbs{\theta}_{(1:d)})=\!\!\!\!\!\sum_{\mbs{j}\in\{0,1,2\}^{d}}\!\!\!\!\mbf{I}_{\mbs{j}}(s,\mbs{\theta}_{(1:d)})R_{\mbs{j}}(s,\mbs{\theta}_{(1:d)})
		\bbr\}.
	\end{align*}}
	Then, for given $R\in\mcl{N}_{3^{d}}[a,b]$, we define the PI operator $\mcl{R}:=\mcl{P}[R]$ for $\mbf{w}\in L_{2}[[a,b]^{d}]$ as
	{%\small
	\begin{align*}
		\bl(\mcl{R}\mbf{w}\br)(s)=\int_{a}^{b}R(s,\mbs{\theta}_{(1:d)}) \mbf{w}(\mbs{\theta}_{(1:d)})\thinspace d\mbs{\theta}_{(d:1)},
	\end{align*}}
	We define $\Pi_{3^{d}}[a,b]$ as the set of operators of this form, so that $\mcl{R}\in\Pi_{3^{d}}$ if and only if $\mcl{R}=\mcl{P}[R]$ for some $R\in\mcl{N}_{3^{d}}$.
\end{defn}
%For completeness, we also define $\mcl{N}_{3^{0}}[a,b]=L_{0}[a,b]$, and write $\mcl{R}\in\Pi_{3^{0}}$ if there exists some $R\in\mcl{N}_{3^{0}}$ such that $(\mcl{R}w)(s)=R(s)w$ for all $w\in\R$ and $s\in[a,b]$.
Defining a set of PI operators on $L_2[[a,b]^{d}]$ in this manner, we finally define a subclass of distributed polynomials parameterized by PI operators as follows
\begin{defn}
	For given $d\in\N$, we say that a function $p:[a,b]\times L_2[a,b] \rightarrow \R$ is a distributed homogeneous PI polynomial of degree $d$ if there exists a PI operator $\mcl{C}=\mcl{P}[C]\in\Pi_{3^{d}}[a,b]$ such that
	\begin{equation*}\Resize{\linewidth}{
		p(s)(\mbf{u})=\bl(\mcl{C}Z^{h}_{\otimes d}(\mbf{u})\br)(s)=\int_{a}^{b} C(s,\mbs{\theta}_{(1:d)})\mbf{u}^{\otimes d}(\mbs{\theta}_{(1:d)})d\mbs{\theta}_{(d:1)}.}
	\end{equation*}
%	We say that $p$ is a distributed PI polynomial of degree $d$ if $p(\mbf{u})=\sum_{k=0}^d p_k(\mbf{u})$, where each $p_k$ is a homogeneous PI polynomial of degree $k$.	
%	
	We say that $p$ is a distributed PI polynomial of degree $d$ if it can be expressed as a sum of homogeneous PI polynomials of degree at most $d$, i.e. if there exists $\mcl{C}=[C_{0},\mcl{C}_{1},\hdots,\mcl{C}_{d}]$ with $C_{0}\in L_0[a,b]$ and $\mcl{C}_{k}\in\Pi_{3^{k}}[a,b]$ such that
	\begin{align*}
		p(s)(\mbf{u})&=\bl(\mcl{C}Z_{\otimes d}(\mbf{u})\br)(s)
		%=\sum_{k=0}^d\bl(\mcl{C}_{k}Z^{h}_{\otimes k}(\mbf{u})\br)(s)
		=C_{0}(s) +\sum_{k=1}^{d}\bl(\mcl{C}_{k}Z^{h}_{\otimes k}(\mbf{u})\br)(s).
		%	&=C_{0}(s)+\sum_{k=1}^d\int_{a}^{b}\!\cdots\!\int_{a}^{b}C_{k}(s,\theta_{1},\hdots,\theta_{k})\mbf{u}(\theta_{1})\cdots\mbf{u}(\theta_{k}) d\theta_{k}\cdots d\theta_{1}
	%	&=C_{0}(s)+\sum_{k=1}^{d}\int_{a}^{b}C_{k}(s,\mbs{\theta}_{(1:k)}) \mbf{u}^{\otimes k}(\mbs{\theta}_{(1:k)})\thinspace d\mbs{\theta}_{(k:1)}
	\end{align*}
%	$p(\mbf{u})=\sum_{k=0}^d p_k(\mbf{u}),$
%	where each $p_k$ is a homogeneous PI polynomial of degree $k$.	
%	For given $d\in\N$ and $\mbf{u}\in L_2[a,b]$, we say that a function $p(\mbf{u})\in L_{0}[a,b]$ is a distributed PI polynomial of degree at most $d$ in $\mbf{u}$ if there exists an operator $\mcl{C}=\bmat{\mcl{C}_{0}&\hdots&\mcl{C}_{d}}$ with $\mcl{C}_{k}\in\Pi_{3^{k}}$ for $k\in\{0,\hdots,d\}$ such that\\[-1.8em]
%	\begin{align*}
%		p(\mbf{u})=\mcl{C}Z_{\otimes d}(\mbf{u})
%		=\sum_{k=0}^{d}\mcl{C}_{k}\mbf{u}^{\otimes k}.
%	\end{align*}
\end{defn}
\smallskip
In the next section, we show that this subclass of polynomials indeed behaves as we would expect, in that it is closed under addition and multiplication.

\paragraph*{\tbf{Example}}
For $\mbf{u}\in L_2[0,1]$, consider the polynomial
\begin{align*}
	p(\mbf{u})&=\int_{0}^{s}[s-1]\theta\mbf{u}(\theta) +\int_{s}^{1}s[\theta-1]\mbf{u}(\theta)d\theta		\\
	&+\int_{0}^{s}\eta^2 \mbf{u}(s)\mbf{u}(\eta)d\eta +\int_{s}^{1}\int_{s}^{1}s[\theta-\eta]\mbf{u}(\theta)\mbf{u}(\eta)d\eta d\theta.
\end{align*}
We can represent this polynomial in terms of PI operators $\mcl{T}:=\mcl{P}[T]\in\Pi_{3}$ and $\mcl{R}:=\mcl{P}[R]\in\Pi_{3^2}$ as
\begin{align*}
	&p(\mbf{u})=\bmat{\mcl{T}&\mcl{R}}\slbmat{\mbf{u}\\\mbf{u}\otimes\mbf{u}}	\\
	&\quad=\int_{0}^{1}T(s,\theta)\mbf{u}(\theta)d\theta +\int_{0}^{1}\!\!\int_{0}^{1}R(s,\theta,\eta)\mbf{u}(\theta)\mbf{u}(\eta)d\eta d\theta,
\end{align*}
where, for $s,\theta,\eta\in[0,1]$,
\begin{align*}
	T(s,\theta)&:=\mbf{I}_{1}(s,\theta)T_{1}(s,\theta)+\mbf{I}_{2}(s,\theta)T_{2}(s,\theta),	\\
	R(s,\theta,\eta)&:=\mbf{I}_{01}(s,\theta,\eta)R_{01}(s,\eta)+\mbf{I}_{22}(s,\theta,\eta)R_{22}(s,\theta,\eta),
\end{align*}
\begin{align*}
	\text{with}\qquad &T_{1}(s,\theta):=[s-1]\theta,	&	&T_{2}(s,\theta):=s[\theta-1],	\\
	&R_{01}(s,\eta):=\eta^2,	&	&R_{22}(s,\theta,\eta)=s[\theta-\eta].
\end{align*}
%Then $T\in\mcl{N}_{3}[0,1]$ and $R\in\mcl{N}_{3^2}[0,1]$, and, defining $\mcl{T}:=\mcl{P}[T]\in\Pi_{3}$ and $\mcl{R}:=\mcl{P}[R]\in\Pi_{3^2}$, we may express



\section{Addition and Multiplication of Distributed Polynomials}\label{sec:polynomial_operations}

A crucial property of polynomials on finite-dimensional states $x\in\R^{n}$ is that they are closed under addition and multiplication, so that
\begin{equation*}
	p_1(x)+p_2(x)=c^T Z_{d}(x) +b^T Z_{d}(x) = (c+b)^T Z_{d}(x),
\end{equation*}
and
\begin{equation*}
	p_1(x)p_2(x)=c^T Z_{d}(x)b^T Z_{q}(x)=(c\!\otimes\! b)^T (Z_{d}(x)\!\otimes\! Z_{q}(x)).
\end{equation*}

In this section, we show how these results naturally extend to the class of distributed polynomials parameterized by PI operators. In particular, in the following subsection, we first define an addition operation for the PI operators $\mcl{B},\mcl{C}\in\Pi_{3^{d}}$, and use this operation to define a sum of distributed polynomials. In the next subsection, we abstract the Kronecker product $b\otimes c$ to a tensor product $\mcl{B}\otimes\mcl{C}$ of PI operators, and use this to derive an expression for the product of distributed polynomials. These results will allow us to derive a PIE representation of nonlinear PDEs in Section~\ref{sec:PIE}.


\subsection{Formulae for Addition of Distributed Polynomials}

Addition rules for distributed polynomials parameterized by PI operators follow trivially from the linear nature of PI operators. In particular we have the following result.
\begin{prop}\label{prop:PI_addition}
	For given $d\in\N$, let $Q,R\in \mcl{N}_{3^{d}}[a,b]$, and let $\alpha,\beta\in L_{0}[a,b]$. Then, $B:=\alpha Q+\beta R\in\mcl{N}_{3^{d}}[a,b]$, and, for any $\mbf{w}\in L_2[[a,b]^{d}]$
	\begin{equation*}
		(\mcl{P}[B]\mbf{w})(s)\!=\!\alpha(s)(\mcl{P}[Q]\mbf{w})(s)+\beta(s)(\mcl{P}[R]\mbf{w})(s),	\enspace	s\in[a,b].
	\end{equation*}
	We define the addition operation $+:\Pi_{2^{d}}\times\Pi_{3^{d}}\to\Pi_{3^{d}}$ accordingly, so that $\mcl{P}[Q]+\mcl{P}[R]=\mcl{P}[B]$.
\end{prop}
\begin{proof}
	Since $Q,R\in\mcl{N}_{3^{d}}$, by definition, there exist functions $Q_{\mbs{j}},R_{\mbs{j}}\in L_{2,0}\bl[[a,b]^{d+1}\br]$ for $\mbs{j}\in\{0,1,2\}^{k}$ such that
	\begin{align*}
		Q(s,\mbs{\theta}_{(1:d)})=\sum_{\mbs{j}\in\{0,1,2\}^{d}}\mbf{I}_{\mbs{j}}(s,\mbs{\theta}_{(1:d)})Q_{\mbs{j}}(s,\mbs{\theta}_{(1:d)}),	\\
		R(s,\mbs{\theta}_{(1:d)})=\sum_{\mbs{j}\in\{0,1,2\}^{d}}\mbf{I}_{\mbs{j}}(s,\mbs{\theta}_{(1:d)})R_{\mbs{j}}(s,\mbs{\theta}_{(1:d)}),
	\end{align*}
	Defining $B:=\alpha Q+\beta R$, it follows that we can expand $B$ as
	{\small
	\begin{align*}
		&B(s,\mbs{\theta}_{(1:d)})	\\
		&=\!\!\!\!\!\sum_{\mbs{j}\in\{0,1,2\}^{d}}\!\!\!\!\mbf{I}_{\mbs{j}}(s,\mbs{\theta}_{(1:d)})\bl[\alpha(s) Q_{\mbs{j}}(s,\mbs{\theta}_{(1:d)})+\beta(s) R_{\mbs{j}}(s,\mbs{\theta}_{(1:d)})\br].
	\end{align*}}
	Since $\alpha Q_{\mbs{j}}+\beta R_{\mbs{j}}\in L_{0,2}[[a,b]^{d+1}]$ for each $\mbs{j}\in\{0,1,2\}^{2}$, by definition, $B\in\mcl{N}_{3^{d}}[a,b]$. By linearity of the integral, it immediately follows that $\alpha(\mcl{P}[Q]\mbf{w})+\beta(\mcl{P}[R]\mbf{w})=\mcl{P}[B]\mbf{w}$ for any $\mbf{w}\in L_2[[a,b]^{d}]$.
\end{proof}
Using this result, it is clear that the sum of two PI polynomials is itself a PI polynomial as well.
\begin{cor}\label{cor:polynomial_addition}
	For given $d\in\N$, let $\mcl{B}:=\bmat{\mcl{B}_{0}&\hdots&\mcl{B}_{d}}$ and $\mcl{C}:=\bmat{\mcl{C}_{0}&\hdots&\mcl{C}_{d}}$ for $\mcl{B}_{k},\mcl{C}_{k}\in\Pi_{3^{k}}$ define distributed polynomials $p_1(\mbf{u})=\mcl{B}Z_{\otimes d}(\mbf{u})$ and $p_2(\mbf{u})=\mcl{C}Z_{\otimes d}(\mbf{u})$. Then, for any $\mbf{u}\in L_2$,
	\begin{align*}
		p_1(\mbf{u})+p_2(\mbf{u})=[\mcl{B}+\mcl{C}]Z_{\otimes d}(\mbf{u}),
	\end{align*}
	where $\mcl{B}+\mcl{C}:=\bmat{\mcl{B}_{0}+\mcl{C}_{0}&\hdots&\mcl{B}_{d}+\mcl{C}_{d}}$.
\end{cor}
\begin{proof}
	By Proposition~\ref{prop:PI_addition}, for any $s\in[a,b]$.
	\begin{align*}
		&p_1(\mbf{u})(s)+p_2(\mbf{u})(s)=\bl(\mcl{B}Z_{\otimes d}(\mbf{u})\br)(s) +\bl(\mcl{C}Z_{\otimes d}(\mbf{u})\br)(s)	\\
		&=\sum_{k=0}^{d}(\mcl{B}_{k}\mbf{u}^{\otimes k})(s) +\sum_{k=0}^{d}(\mcl{C}_{k}\mbf{u}^{\otimes k})(s)	\\
		&=\sum_{k=0}^{d}([\mcl{B}_{k}+\mcl{C}_{k}]\mbf{u}^{\otimes k})(s)
		=\bl([\mcl{B}+\mcl{C}]Z_{\otimes d}(\mbf{u})\br)(s).
	\end{align*}
\end{proof}


\subsection{Formulae for Multiplication of Distributed Polynomials}

In order to compute a product of two PI polynomials, we first define a tensor product of PI operators as follows
\begin{prop}\label{prop:PI_multiplication}
	For given $d,q\in\N$, let $Q\in \mcl{N}_{3^{d}}[a,b]$ and $R\in \mcl{N}_{3^{q}}[a,b]$. Define
	\begin{align*}
		B(s,\mbs{\theta}_{(1:d+q)})&:=Q(s,\mbs{\theta}_{(1:d)})R(s,\mbs{\theta}_{(d+1:d+q)}).
	\end{align*}
	Then, $B\in \mcl{N}_{3^{d+q}}[a,b]$. Moreover, for any $\mbf{u}\in L_{2}\bl[[a,b]^{d}\br]$ and $\mbf{v}\in L_{2}\bl[[a,b]^{q}\br]$
	\begin{align*}
		(\mcl{P}[Q]\mbf{u})(s)(\mcl{P}[R]\mbf{v})(s)&=\bl(\mcl{P}[B](\mbf{u}\otimes \mbf{v})\br)(s),	&	s&\in[a,b].
	\end{align*}
	We define the tensor product $\otimes:\Pi_{3^{d}}\times\Pi_{3^{q}}\to\Pi_{3^{d+q}}$ of PI operators accordingly, so that $\mcl{P}[Q]\otimes\mcl{P}[R]=\mcl{P}[B]$.
\end{prop}
\begin{proof}
	Since $Q\in\mcl{N}_{3^{d}}[a,b]$ and $R\in\mcl{N}_{3^{q}}[a,b]$, there exists functions $Q_{\mbs{i}}\in L_{0,2}[[a,b]^{d+1}]$ for $\mbs{i}\in\{0,1,2\}^{d}$ and $R_{\mbs{j}}\in L_{0,2}[[a,b]^{q+1}]$ for $\mbs{j}\in\{0,1,2\}^{q}$ such that
	\begin{align*}
		Q(s,\mbs{\theta}_{(1:d)})=\sum_{\mbs{i}\in\{0,1,2\}^{d}}\mbf{I}_{\mbs{i}}(s,\mbs{\theta}_{(1:d)})Q_{\mbs{i}}(s,\mbs{\theta}_{(1:d)}),	\\
		R(s,\mbs{\theta}_{(1:q)})=\sum_{\mbs{j}\in\{0,1,2\}^{q}}\mbf{I}_{\mbs{j}}(s,\mbs{\theta}_{(1:q)})R_{\mbs{j}}(s,\mbs{\theta}_{(1:q)}),
	\end{align*}
	Define $B_{\mbs{k}}(s,\mbs{\theta}_{(1:d+q)}):=Q_{\mbs{i}}(s,\mbs{\theta}_{(1:d)})R_{\mbs{j}}(s,\mbs{\theta}_{(d+1:d+q)})$ for $\mbs{k}=(\mbs{i},\mbs{j})\in\{0,1,2\}^{d}\times\{0,1,2\}^{q}$. Then, $B_{\mbs{k}}\in L_{0,2}[[a,b]^{d+q}]$ for each $\mbs{k}$, and
	{%\small
	\begin{align*}
		B(s,\mbs{\theta}_{(1:d+q)})&=Q(s,\mbs{\theta}_{(1:d)})R(s,\mbs{\theta}_{(d+1:d+q)})	\\
		&=\!\!\!\!\!\!\sum_{\mbs{i}\in\{0,1,2\}^{d}}\!\!\!\!\mbf{I}_{\mbs{i}}(s,\mbs{\theta}_{(1:d)})\!\!\!\!\sum_{\mbs{j}\in\{0,1,2\}^{q}}\!\!\!\!\mbf{I}_{\mbs{j}}(s,\mbs{\theta}_{(d+1:d+q)})\\
		&\hspace*{2.0cm}[Q_{\mbs{i}}(s,\mbs{\theta}_{(1:d)})R_{\mbs{j}}(s,\mbs{\theta}_{(d+1:d+q)})]	\\
		&=\sum_{\mbs{k}\in\{0,1,2\}^{d+q}}\!\!\mbf{I}_{\mbs{k}}(s,\mbs{\theta}_{(1:d+q)})B_{\mbs{k}}(s,\mbs{\theta}_{(1:d+q)}).
	\end{align*}}
	It follows that $B\in\mcl{N}_{3^{d+q}}[a,b]$.
	Moreover, for any $\mbf{u}\in L_{2}\bl[[a,b]^{d}\br]$ and $\mbf{v}\in L_{2}\bl[[a,b]^{q}\br]$
	\begin{align*}
		&(\mcl{P}[Q]\mbf{u})(s)(\mcl{P}[R]\mbf{v})(s)
		=
		\int_{a}^{b}Q(s,\mbs{\theta}_{(1:d)})\mbf{u}(\mbs{\theta}_{(1:d)}) \thinspace d\mbs{\theta}_{(d:1)} \\
		&\hspace*{1.0cm}\cdot\int_{a}^{b}R(s,\mbs{\theta}_{(d+1:d+q)})\mbf{v}(\mbs{\theta}_{(d+1:d+q)}) \thinspace d\mbs{\theta}_{(d+q:d+1)}	\\
		&=\int_{a}^{b}B(s,\mbs{\theta}_{(1:d+q)})\mbf{u}(\mbs{\theta}_{(1:d)})\mbf{v}(\mbs{\theta}_{(d+1:d+q)})\thinspace d\mbs{\theta}_{(d+q:1)}	\\
		&\hspace*{4.0cm}=\bl(\mcl{P}[B](\mbf{u}\otimes \mbf{v})\br)(s). \\[-3.0em]
	\end{align*}
\end{proof}
Defining a tensor product of PI operators $\mcl{B}_{k},\mcl{C}_{k}\in\Pi_{3^{k}}$ in this manner, we can naturally define a tensor product of aggregate operators $\mcl{B}=\bmat{\mcl{B}_{0}&\hdots&\mcl{B}_{d}}$ and $\mcl{C}=\bmat{\mcl{C}_{0}&\hdots&\mcl{C}_{q}}$ as
\begin{align*}
	&\bmat{\mcl{B}_{0}&\hdots&\mcl{B}_{d}}\otimes\bmat{\mcl{C}_{0}&\hdots&\mcl{C}_{q}}	\\
	&\quad =\bmat{\mcl{B}_{0}\otimes\mcl{C}_{0}&\hdots&\mcl{B}_{0}\otimes\mcl{C}_{q}&\mcl{B}_{1}\otimes\mcl{C}_{0}&\hdots&\mcl{B}_{d}\otimes\mcl{C}_{q}}.
\end{align*}
Similarly, we can define a tensor product of monomial vectors $Z_{\otimes d}(\mbf{u})$ and $Z_{\otimes q}(\mbf{u})$ as \\[-1.9em]
{\small
\begin{align*}
	&Z_{\otimes d}(\mbf{u})\otimes Z_{\otimes q}(\mbf{u})	=\slbmat{\mbf{u}^{\otimes 0}\\\mbf{u}^{\otimes 1}\\~\vdots\\\mbf{u}^{\otimes d}}\otimes
	\slbmat{\mbf{u}^{\otimes 0}\\\mbf{u}^{\otimes 1}\\~\vdots\\\mbf{u}^{\otimes q}}	
	=\slbmat{\mbf{u}^{\otimes 0}\otimes\mbf{u}^{\otimes 0}\\~\vdots\\\mbf{u}^{\otimes0}\otimes\mbf{u}^{\otimes q}\\\mbf{u}^{\otimes 1}\otimes\mbf{u}^{\otimes 0}\\~\vdots\\\mbf{u}^{\otimes d}\otimes\mbf{u}^{\otimes q}}. \\[-1.8em]
	%=\slbmat{\mbf{u}^{\otimes 0}\\~\vdots\\\mbf{u}^{\otimes q}\\\mbf{u}^{\otimes 1}\\~\vdots\\\mbf{u}^{\otimes d+q}}
	%\sim Z_{\otimes (d+q)}(\mbf{u})
\end{align*}}
Finally, we can use these tensor products to define the product of two distributed PI polynomials.
\begin{cor}\label{cor:polynomial_multiplication}
	For given $d,q\in\N$, let $\mcl{B}:=\bmat{\mcl{B}_{0}&\hdots&\mcl{B}_{d}}$ and $\mcl{C}:=\bmat{\mcl{C}_{0}&\hdots&\mcl{C}_{q}}$ for $\mcl{B}_{k},\mcl{C}_{k}\in\Pi_{3^{k}}$ define distributed polynomials $p_1(\mbf{u})=\mcl{B}Z_{\otimes d}(\mbf{u})$ and $p_2(\mbf{u})=\mcl{C}Z_{\otimes q}(\mbf{u})$. Then, for any $\mbf{u}\in L_2$, \\[-1.6em]
	\begin{align*}
		p_1(\mbf{u})p_2(\mbf{u})=[\mcl{B}\otimes\mcl{C}]\bl[Z_{\otimes d}(\mbf{u})\otimes Z_{\otimes q}(\mbf{u})\br].
	\end{align*}
\end{cor}
\medskip
\begin{proof}
	By Proposition~\ref{prop:PI_multiplication}, and by definition of the tensor products $\mcl{B}\otimes\mcl{C}$ and $Z_{\otimes d}(\mbf{u})\otimes Z_{\otimes q}(\mbf{u})$, for any $s\in[a,b]$, \\[-1.6em]
	\begin{align*}
		p_1(\mbf{u})(s)p_2(\mbf{u})(s)%&=\bl(\mcl{B}Z_{\otimes d}(\mbf{u})\br)(s) \thinspace \bl(\mcl{C}Z_{\otimes q}}(\mbf{u})\br)(s)	\\
		&=\sum_{k=0}^{d}(\mcl{B}_{k}\mbf{u}^{\otimes k})(s) \sum_{\ell=0}^{q}(\mcl{C}_{\ell}\mbf{u}^{\otimes \ell})(s)	\\
		&=\sum_{k=0}^{d}\sum_{\ell=0}^{q}\bl([\mcl{B}_{k}\otimes\mcl{C}_{\ell}]\bl[\mbf{u}^{\otimes k}\otimes\mbf{u}^{\otimes \ell}\br]\br)(s)	\\
		&=\bl([\mcl{B}\otimes\mcl{C}]\bl[Z_{\otimes d}(\mbf{u})\otimes Z_{\otimes d}(\mbf{u})\br]\br)(s).%
	\end{align*}
	\	\\[-2.5em]
\end{proof}
Naturally, we can always reduce $Z_{\otimes d}(\mbf{u})\otimes Z_{\otimes q}(\mbf{u})$ to the monomial basis $Z_{\otimes (d+q)}(\mbf{u})$, by merging terms defined by the same monomials, as e.g. \\[-1.7em]
\begin{align*}
	&\bmat{\mcl{B}_{0}\otimes\mcl{C}_{2} &\mcl{B}_{1}\otimes\mcl{C}_{1} & \mcl{B}_{2}\otimes\mcl{C}_{0}}\slbmat{\mbf{u}^{\otimes 0}\otimes\mbf{u}^{\otimes 2}\\ \mbf{u}^{\otimes 1}\otimes\mbf{u}^{\otimes 1}\\ \mbf{u}^{\otimes 2}\otimes\mbf{u}^{\otimes 0}}\\
	&\hspace*{1.5cm} =[\mcl{B}_{0}\otimes\mcl{C}_{2} +\mcl{B}_{1}\otimes\mcl{C}_{1} + \mcl{B}_{2}\otimes\mcl{C}_{0}]\mbf{u}^{\otimes 2}. \\[-2.0em]
\end{align*}
In this manner, the product
of a distributed polynomial of degree $d$ with a distributed polynomial of degree $q$ can always be written as a distributed polynomial of degree $d+q$.

\paragraph*{\tbf{Example}}
For $\mbf{u}\in L_2[0,1]$, let
\begin{align*}
	p_1(\mbf{u})&=\mcl{B}Z_{\otimes 1}(\mbf{u})=B_{0}(s)+(\mcl{B}_{1}\mbf{u})(s),	\\
	p_2(\mbf{u})&=\mcl{C}Z_{\otimes 2}(\mbf{u})=(\mcl{C}_{1}\mbf{u})(s) +(\mcl{C}_{2}[\mbf{u}\otimes\mbf{u}])(s),
\end{align*}
where, for $\mbf{u}\in L_2[0,1]$ and $\mbf{v}\in L_2[[0,1]^2]$, $B_{0}\in L_2[0,1]$, $\mcl{B}_1,\mcl{C}_{1}\in\Pi_{3}$ and $\mcl{C}_{2}\in\Pi_{3^2}$ are defined as
{\small%
\begin{align*}
	&B_{0}(s):=s^2,	&
	(\mcl{B}_{1}\mbf{u})(s)&:=\int_{0}^{s}[s-\theta]\mbf{u}(\theta)d\theta,	\\
	&(\mcl{C}_{1}\mbf{u})(s):=\mbf{u}(s),	&
	(\mcl{C}_{2}\mbf{v})(s)&:=\int_{s}^{1}\int_{0}^{s}\theta\eta\mbf{v}(\theta,\eta) d\eta d\theta.
\end{align*}}
Then	\\[-1.8em]
\begin{align*}
	p(\mbf{u})p_{2}(\mbf{u})%=\mcl{D}Z_{3}^{\otimes}(\mbf{u})
	\!=\!
	\bmat{B_{0}\!\otimes\!\mcl{C}_{1}&(B_{0}\!\otimes\!\mcl{C}_2+\mcl{B}_{1}\!\otimes\!\mcl{C}_1)&\mcl{B}_{1}\!\otimes\!\mcl{C}_{2}}\slbmat{\mbf{u}\\\mbf{u}^{\otimes 2}\\\mbf{u}^{\otimes 3}},
\end{align*}
where, for $\mbf{u}\in L_2[0,1]$, $\mbf{v}\in L_2[[0,1]^2]$, and $\mbf{w}\in L_2[[0,1]^3]$,
{\small
\begin{align*}
	([B_{0}\otimes\mcl{C}_{1}]\mbf{u})(s)&=s^2\mbf{u}(s),	\\
	([B_{0}\otimes\mcl{C}_{2}]\mbf{v})(s)&=\int_{s}^{1}\int_{0}^{s}s^2\theta\eta\mbf{v}(\theta,\eta)d\eta d\theta,	\\
	([B_{1}\otimes\mcl{C}_{1}]\mbf{v})(s)&=\int_{0}^{s}[s-\theta]\mbf{v}(\theta,s)d\theta,	\\
	([B_{1}\otimes\mcl{C}_{2}]\mbf{w})(s)&=\int_{0}^{s}\int_{s}^{1}\int_{0}^{s}[s-\theta]\eta\zeta\mbf{w}(\theta,\eta,\zeta)d\zeta d\eta d\theta.
\end{align*}}
%Letting $\mcl{D}:=\bmat{0&B_{0}\otimes\mcl{C}_{1}&(B_{0}\otimes\mcl{C}_2+\mcl{B}_{1}\otimes\mcl{C}_1)&\mcl{B}_{1}\otimes\mcl{C}_{2}}$, it follows that
%\begin{align*}
%	p(\mbf{u})p_{2}(\mbf{u})=\mcl{D}Z_{3}^{\otimes}(\mbf{u})
%\end{align*}

%Define $\mcl{T},\mcl{R}\in\Pi_{3}$ for $\mbf{v}\in L_2[0,1]$ as
%\begin{align*}
%	\bl(\mcl{T}\mbf{v}\br)(s)&=\int_{0}^{s}[s-1]\theta\mbf{u}(\theta)d\theta + \int_{s}^{1}s[\theta-1]\mbf{u}(\theta)d\theta,	\\
%	\bl(\mcl{R}\mbf{v}\br)(s)&:=\int_{0}^{s}\theta\mbf{u}(\theta)d\theta + \int_{s}^{1}[\theta-1]\mbf{u}(\theta)d\theta.
%\end{align*}
%Then, for any $\mbf{w}\in L_2[[0,1]^2]$,
%{\small%
%\begin{align*}
%	&\bl((\mcl{T}\otimes\mcl{R})\mbf{w}\br)(s)=\int_{0}^{s}\!\int_{0}^{s}\bl([s-1]\theta\br)\eta \mbf{w}(\theta,\eta)d\eta d\theta	\\
%	&+\int_{0}^{s}\!\int_{s}^{1}\bl([s-1]\theta\br)[\eta-1] \mbf{w}(\theta,\eta)d\eta d\theta +\int_{s}^{1}\!\int_{0}^{s}\bl(s[\theta-1]\eta\br) \mbf{w}(\theta,\eta)d\eta d\theta \\
%	&\qquad +\int_{s}^{1}\!\int_{0}^{s}\bl(s[\theta-1]\br)[\eta-1] \mbf{w}(\theta,\eta)d\eta d\theta.
%\end{align*}}


\section{A PIE Representation of Polynomial PDEs}\label{sec:PIE}

In this section, we combine the results from the previous sections to show how a large class of scalar-valued, 1D, polynomial PDEs with linear BCs can be equivalently represented as PIEs. In particular, we restrict our attention to $N$th order PDEs of the form
\begin{align}\label{eq:PDE_nonlinear}
	\tbf{PDE:}\quad &\dot{u}(t,s)=c(s)^T Z_{d}\bl((\mscr{D}^{N}u)(t,s)\br),	\quad s\in[a,b]	\\
	%&=\sum_{k=1}^{d}c_{k}(s)^TZ^{h}_{k}(u(t,s),u_{s}(t,s),u_{ss}(t,s)),	\quad s\in[a,b]	\notag\\
	\tbf{BCs:}\quad &B \bmat{\Delta_{s}^{a}\mscr{D}^{N-1}u\\\Delta_{s}^{b}\mscr{D}^{N-1}u }=0.	\notag
\end{align}
where we defined boundary operators $\Delta_{s}^{a}u = u(a)$ and $\Delta_{s}^{b}u = u(b)$ for $u\in H_{1}[a,b]$, and where for $u\in H_{N}$ and $k\leq N$ we define $\mscr{D}^{k}u$ as the vector of all derivatives of $u$ up to $k$th order as
\begin{align*}
	\mscr{D}^{N}u:=\bmat{u&\partial_{s} u&\hdots &\partial_{s}^{N} u}^T.
\end{align*}
In this system, the state $u(t)$ at each time $t\geq 0$ must be $N$th-order differentiable with respect to the spatial variable $s$, and must satisfy the Boundary Conditions (BCs) defined by $B\in\R^{N\times 2N}$. Accordingly, we define the state space
\begin{align*}
	X_{B}[a,b]:=\bbbl\{\mbf{u}\in H_{N}[a,b]~\bbr\rvert~ B \smallbmat{\Delta_{s}^{a}\mscr{D}^{N-1}u\\\Delta_{s}^{b}\mscr{D}^{N-1}u }=0\bbbr\}.
\end{align*}
We define solutions to the PDE as follows.
\begin{defn}[Classical Solution to the PDE]
	For given initial conditions $\mbf{u}_{0}\in X_{B}$, we say that $\mbf{u}$ is a classical solution to the PDE defined by $\{B,c,d\}$ if $\mbf{u}$ is Frech\'et differentiable, $\mbf{u}(0)=\mbf{u}_{0}$, and for all $t\geq0$, $\mbf{u}(t)\in X_{B}$, and $\mbf{u}(t)$ satisfies~\eqref{eq:PDE_nonlinear}.
\end{defn}
In the remainder of this section, we derive an associated PIE representation of the PDE~\eqref{eq:PDE_nonlinear}, proving equivalence of solutions to the PDE and PIE. %This PIE representation will be free of the BCs and continuity constraints inherent to the PDE representation, and will be expressed in terms of a distributed polynomial.
To derive this representation, in the following subsection, we first prove that we can express the state $u\in X_{B}$ in terms of its derivative $\partial_{s}^{N}u\in L_2$ using a PI operator $\mcl{T}$, allowing us to express the system dynamics in terms of the BCs-free state $\mbf{v}:= \partial_{s}^{N}u$. In the next subsection, we then derive an expression for the monomial vector $Z_{d}(\mscr{D}^{N}u)$ in terms of the distributed monomial vector $Z_{\otimes d}(\mbf{v})$, yielding a representation of the PDE in terms of a distributed polynomial $\mcl{C}Z_{\otimes d}(\mbf{v})$.


\subsection{A Map From Fundamental to PDE State}

In order to derive a BCs-free representation of the PDE~\eqref{eq:PDE_nonlinear}, we first note that, since the state $u\in X_{B}$ is only required to be $N$th-order differentiable with respect to the variable $s$, the $N$th-order derivative $\partial_{s}^{N}u\in L_{2}$ of the state does not have to satisfy any BCs or continuity constraints. Accordingly, we will refer to the state $\mbf{v}:=\partial_{s}^{N}u$ as the \textit{fundamental state} associated to the PDE.
Clearly, then, we can define a map $\partial_{s}^{N}:X_{B}\to L_2$ from the PDE state space to the fundamental state space as a differential operator $\partial_{s}^{N}$.
Moreover, assuming the BCs to be sufficiently well-posed, we can define an inverse map $\mcl{T}:L_2\to X_{B}$ from the fundamental state space to the PDE state space using a PI operator $\mcl{T}\in\Pi_{3}$. In particular, we recall the following result from e.g.~\cite{shivakumar2022GPDE_Arxiv}
\begin{lem}\label{lem:Tmap}
	Let $B\in\R^{N\times 2N}$ define suitably well-posed BCs for the PDE~\eqref{eq:PDE_nonlinear}, as in Defn.~9 in~\cite{shivakumar2022GPDE_Arxiv}. Then, we can define PI operators $\mcl{T}\in\Pi_{3}$ such that, for all $\mbf{u}\in X_{B}[a,b]$ and $\mbf{v}\in L_2[a,b]$,
	\begin{align*}
		\mbf{u}(s)&=\bl(\mcl{T}[\partial_{s}^{N}\mbf{u}]\br)(s),	&	&\text{and}	&
		\mbf{v}(s)&=\bl(\partial_{s}^{N}[\mcl{T}\mbf{v}]\br)(s).
	\end{align*}
	More generally, for any $j\in\{0,\hdots,N-1\}$, we can define an operator $\mcl{R}_{j}\in\Pi_{3}$ such that, for all $\mbf{u}\in X_{B}[a,b]$ and $\mbf{v}\in L_2[a,b]$,
	\begin{align*}
		(\partial_{s}^{j}\mbf{u})(s)&=\bl(\mcl{R}_{j}[\partial_{s}^{N}\mbf{u}]\br)(s),	&	&\text{and}	&
		(\mcl{R}_{j}\mbf{v})(s)&=\bl(\partial_{s}^{j}[\mcl{T}\mbf{v}]\br)(s).
	\end{align*}
\end{lem}
\begin{proof}
	We refer to Thm.~10 and Thm.~12 in\cite{shivakumar2022GPDE_Arxiv} for a proof, as well as for explicit formulae mapping the matrix $B$ to operators $\mcl{T}$ and $\mcl{R}_{j}$.
\end{proof}
Using this lemma, we can express both the PDE state $u$ and any of its derivatives $\partial_{s}^{j}u$ in terms of the fundamental state $\mbf{v}:=\partial_{s}^{N}u$ as $u=\mcl{T}\mbf{v}$ and $\partial_{s}^{j}u=\mcl{R}_{j}\mbf{v}$. Obviously, we can also define an identity operator $\mcl{R}_{N}=I$ so that $\partial_{s}^{N}u=\mcl{R}_{N}\mbf{v}$. Then, defining
\begin{align*}
	\mscr{R}\mbf{v}:=\bmat{\mcl{R}_{0}\mbf{v}&\mcl{R}_{1}\mbf{v}&\hdots&\mcl{R}_{N}\mbf{v}}^T,
\end{align*}
any $u\in X_{B}[a,b]$ will satisfy $\mscr{D}u=\mscr{R}[\partial_{s}^{N}u]=\mscr{R}\mbf{v}$. Substituting this relations into the PDE~\eqref{eq:PDE_nonlinear}, we obtain an equivalent representation of the system as
\begin{align}\label{eq:PIE_intermediate}
	\tbf{PIE:}\enspace &(\mcl{T}\dot{\mbf{v}})(t,s)=c(s)^T Z_{d}\bl((\mscr{R}\mbf{v})(t,s)\br),	\enspace	s\in[a,b],
	%&=\sum_{k=1}^{d}c_{k}(s)^TZ^{h}_{k}(u(t,s),u_{s}(t,s),u_{ss}(t,s)),	\quad s\in[a,b]	\notag
\end{align}
where $\mbf{v}(t)\in L_2$ is free of BCs and continuity constraints. Moreover, the dynamics are now represented as a function of only the fundamental state $\mbf{v}$ and PI operators applied to this state. In the following subsection, we show how this function can be expressed as a distributed polynomial in $\mbf{v}$.


\subsection{A Representation in Terms of Distributed Monomials}

Consider the polynomial function $c(s)^TZ_{d}\bl((\mscr{R}\mbf{v})(s)\br)$ defining the dynamics of the PIE in~\eqref{eq:PIE_intermediate}. In order to express this polynomial in terms of the distributed monomial vector $Z_{\otimes d}$, we first expand it into a sum of homogeneous polynomials $c_{k}(s)^T Z^{h}_{k}$ as
\begin{align*}
	c(s)^T Z_{d}\bl((\mscr{R}\mbf{v})(s)\br)=c_{0}(s) +\sum_{k=1}^{d} c_{k}(s)^T Z^{h}_{k}\bl((\mscr{R}\mbf{v})(s)\br).
\end{align*}
Here, e.g. the vector $Z^{h}_{2}\bl(\mscr{R}\mbf{v}\br)$ consists of all products $(\mcl{R}_{i}\mbf{v})(\mcl{R}_{j}\mbf{v})$ for $i,j\in\{0,\hdots,N\}$. As such, we can define associated coefficients $c_{2,ij}$ for $i,j\in\{0,\hdots,2\}$ such that
\begin{align*}
	&c_{2}(s)^T Z^{h}_{2}\bl((\mscr{R}\mbf{v})(s)\br)	
	=\sum_{i=0}^{N}\sum_{j=0}^{N}c_{2,ij}(s)\thinspace (\mcl{R}_{i}\mbf{v})(s) (\mcl{R}_{j}\mbf{v})(s)	\\
	&\enspace =\bbbl(\sum_{i,j=0}^{N}c_{2,ij}(s)[\mcl{R}_{i}\otimes\mcl{R}_{j}][\mbf{v}\otimes\mbf{v}]\bbbr)(s)
	=\bl(\mcl{C}_{2}Z^{h}_{\otimes 2}(\mbf{v})\br)(s),
\end{align*}
where $\mcl{C}_{2}=\sum_{i,j=0}^{N}c_{2,ij}[\mcl{R}_{i}\otimes\mcl{R}_{j}]$. More generally, for any $k\in\{1,\hdots,d\}$, we can define an operator $\mcl{C}_{k}\in\Pi_{3^{k}}$ as
\begin{align*}%\label{eq:PIE_Cops}
	\mcl{C}_{k}:=\!\!\!\sum_{\mbs{j}\in\{0,\hdots,N\}^{k}}\!\! c_{k,\mbs{j}}(s)\thinspace  [\mcl{R}_{j_1}\otimes\cdots\otimes\mcl{R}_{j_{k}}],
\end{align*}
so that
\begin{align*}
	&c_{k}(s)^T Z^{h}_{k}\bl((\mscr{R}\mbf{v})(s)\br)=\bl(\mcl{C}_{k}Z^{h}_{\otimes k}(\mbf{v})\br)(s).
\end{align*}
More precisely, we have the following result.
\begin{prop}\label{prop:Cmap}
	Let $\mscr{R}:=\bmat{\mcl{R}_{0}&\hdots&\mcl{R}_{N}}^T$, where $\mcl{R}_{j}\in\Pi_{3}$ for $j\in\{0,\hdots,N\}$. For given coefficients $c\in L_{0}^{{N+1+d}\choose{d}}$, define a polynomial on $\mbf{v}\in L_2$ as
	\begin{align*}
		p(\mbf{v})(s)&=c(s)^T Z_{d}\bl((\mscr{R}\mbf{v})(s)\br)	\\
		&\quad=c_{0}(s) +\sum_{k=1}^{d} c_{k}(s)^T Z^{h}_{k}\bl((\mscr{R}\mbf{v})(s)\br),
	\end{align*}
	where, for each $k\in\{1,\hdots, d\}$,
	\begin{align*}
		c_{k}(s)^T Z^{h}_{k}((\mscr{R}\mbf{v})(s)) = \!\!\!\!\!\!\!\sum_{\mbs{j}\in\{0,\hdots,N\}^{k}} \!\!\!\!\!\! c_{k,\mbs{j}}(s)\thinspace (\mcl{R}_{j_1}\mbf{v})(s)\cdots (\mcl{R}_{j_k}\mbf{v})(s).
	\end{align*}
	Finally, define $\mcl{C}:=\bmat{c_{0}&\mcl{C}_{1}&\hdots&\mcl{C}_{d}}$, where for $k\in\{1,\hdots, d\}$
	\begin{align}\label{eq:PIE_Cops}
		\mcl{C}_{k}:=\!\!\!\sum_{\mbs{j}\in\{0,\hdots,N\}^{k}}\!\! c_{k,\mbs{j}}(s)\thinspace  [\mcl{R}_{j_1}\otimes\cdots\otimes\mcl{R}_{j_{k}}]\quad \in\Pi_{3}.
	\end{align}
	Then, for any $\mbf{v}\in L_2$,
	\begin{align*}
		p(\mbf{v})(s)=\bl(\mcl{C} Z_{\otimes d}(\mbf{v})\br)(s),
	\end{align*}
\end{prop}
\begin{proof}	
	Let $\mbf{v}\in L_2$ be arbitrary, and fix $k\in\{1,\hdots,N\}$. Then, by Proposition~\ref{prop:PI_multiplication}, for any $\mbs{j}\in\{0,\hdots,N\}^{d}$,
	\begin{align*}
		(\mcl{R}_{j_1}&\mbf{v})(s)(\mcl{R}_{j_2}\mbf{v})(s)\cdots (\mcl{R}_{j_k}\mbf{v})(s)	\\
		&=([\mcl{R}_{j_1}\otimes\mcl{R}_{j_2}][\mbf{v}^{\otimes 2}])(s)(\mcl{R}_{j_k}\mbf{v})(s)\cdots (\mcl{R}_{j_k}\mbf{v})(s)	\\
		&\enspace \vdots	\\
		&=\bl([\mcl{R}_{j_1}\otimes\mcl{R}_{j_2}\otimes\cdots\otimes\mcl{R}_{j_k}][\mbf{v}^{\otimes k}]\br)(s).
	\end{align*}
	It follows that
	\begin{align*}
		&c_{k}(s)^T Z^{h}_{k}((\mscr{R}\mbf{v})(s)) = \!\!\!\!\!\!\!\sum_{\mbs{j}\in\{0,\hdots,N\}^{k}} \!\!\!\!\!\! c_{k,\mbs{j}}(s)\thinspace (\mcl{R}_{j_1}\mbf{v})(s)\cdots (\mcl{R}_{j_k}\mbf{v})(s) \\
		&\quad =\sum_{\mbs{j}\in\{0,\hdots,N\}^{k}} \!\!\!\!\!\! c_{k,\mbs{j}}(s)\thinspace \bl([\mcl{R}_{j_1}\otimes\cdots\otimes\mcl{R}_{j_k}][\mbf{v}^{\otimes k}]\br)(s)	\\
		&\qquad =\bbbbl(\bbbbl[\sum_{\mbs{j}\in\{0,\hdots,N\}^{k}} \!\!\!\!\!\! c_{k,\mbs{j}}(s)\thinspace [\mcl{R}_{j_1}\otimes\cdots\otimes\mcl{R}_{j_k}]\bbbbr]\mbf{v}^{\otimes k}\bbbbr)(s) \\
		&\qquad\quad =\bl(\mcl{C}_{k}\mbf{v}^{\otimes k}\br)(s).
	\end{align*}
	Since this holds for any $k\in\{1,\hdots,d\}$, we find
	\begin{align*}
		p(\mbf{v})(s)&=c(s)^T Z_{d}\bl((\mscr{R}\mbf{v})(s)\br)	\\
		&\quad=c_{0}(s) +\sum_{k=1}^{d} c_{k}(s)^T Z^{h}_{k}\bl((\mscr{R}\mbf{v})(s)\br)	\\
		&\qquad= c_{0}(s) +\sum_{k=1}^{d}\bl(\mcl{C}_{k}\mbf{v}^{\otimes k}\br)(s)	%\\[-1.6em]
		%&\qquad = \bmat{c_{0}&\mcl{C}_{1}&\hdots&\mcl{C}_{d}}\slbmat{\mbf{v}^{\otimes 0}\\\mbf{v}^{\otimes 1}\\\vdots\\\mbf{v}^{\otimes d}}
		=\bl(\mcl{C}Z_{\otimes d}(\mbf{v})\br)(s).
	\end{align*}
\end{proof}
Defining $\mcl{C}$ as in this proposition, we obtain a representation of the PIE~\eqref{eq:PIE_intermediate} in terms of the distributed monomial vector $Z_{\otimes d}(\mbf{v})$ as
\begin{align}\label{eq:PIE_nonlinear}
	\tbf{PIE:}\quad (\mcl{T}\dot{\mbf{v}})(t,s)&=\bl(\mcl{C}Z_{\otimes d}(\mbf{v})\br)(t,s),	&	s&\in[a,b].
\end{align}
We define solutions to this system as follows.
\begin{defn}[Classical Solution to the PIE]
	For given initial conditions $\mbf{v}_{0}\in L_2[a,b]$, we say that $\mbf{v}$ is a classical solution to the PIE defined by $\{\mcl{T},\mcl{C},d\}$ if $\mbf{v}$ is Frech\'et differentiable, $\mbf{v}(0)=\mbf{v}_{0}$, and for all $t\geq0$, $\mbf{v}(t)$ satisfies~\eqref{eq:PIE_nonlinear}.
\end{defn}
\smallskip
The following lemma proves that there exists an invertible map between classical solutions to the PDE~\eqref{eq:PDE_nonlinear}, and classical solutions to the associated PIE~\eqref{eq:PIE_nonlinear}.
\begin{lem}\label{lem:PDE2PIE}
	Let $\{B,c,d\}$ define a PDE as in~\eqref{eq:PDE_nonlinear}, where $c=\bmat{c_{0}&\hdots&c_{d}}$. Suppose that $B\in\R^{N\times 2N}$ defines suitably well-posed BCs, and let the associated operators $\mcl{T},\mcl{R}_{j}\in\Pi_{3}$ for $j\in\{0,\hdots,N-1\}$ be as defined in Lemma~\ref{lem:Tmap}, further setting $R_{N}=I$. For each $k\in\{1,\hdots,d\}$, define coefficients $c_{k,\mbs{j}}$ for $\mbs{j}\in\{0,\hdots,N\}^k$ such that
	\begin{align*}
		c_{k}(s)^T Z^{h}_{k}(x) = \sum_{\mbs{j}\in\{0,\hdots,N\}^{k}} c_{k,\mbs{j}}(s)\thinspace x_{j_1}\cdots x_{j_k}.
	\end{align*}
	Finally, given these coefficients $c$ and operators $\mcl{R}_{j}$, define an associated operator $\mcl{C}$ as in Proposition~\ref{prop:Cmap}.
	
	Then, $\mbf{v}$ is a classical solution to the PIE defined by $\{\mcl{T},\mcl{C},d\}$ with initial conditions $\mbf{v}_{0}$ if and only if $\mbf{u}$ is a classical solution to the PDE defined by $\{B,c,d\}$ with initial conditions $\mbf{u}_{0}=\mcl{T}\mbf{v}$. Conversely, $\mbf{u}$ is a classical solution to the PDE defined by $\{B,c,d\}$ with initial conditions $\mbf{u}_{0}$ if and only if $\partial_{s}^2\mbf{u}$ is a classical solution to the PIE defined by $\{\mcl{T},\mcl{C},d\}$ with initial conditions $\mbf{v}_{0}=\partial_{s}^2\mbf{u}$.
\end{lem}
\begin{proof}
	Let the operators $\mcl{T},\mcl{R}_{j}\in\Pi_{3}$ for $j\in\{0,\hdots,N\}$ be as defined. Then, by Lemma~\ref{lem:Tmap}, for any $\mbf{v}\in L_2[a,b]$,
	\begin{align*}
		\mbf{v}(s)&=\bl(\partial_{s}^{N}[\mcl{T}\mbf{v}]\br)(s),	&	&\text{and}	&
		(\mcl{R}_{j}\mbf{v})(s)&=\bl(\partial_{s}^{j}[\mcl{T}\mbf{v}]\br)(s).
	\end{align*}
	Letting $\mscr{D}=\bmat{\partial_{s}^{0}&\hdots&\partial_{s}^{N}}^T$ and $\mscr{R}=\bmat{\mcl{R}_{0}&\hdots&\mcl{R}_{N}}^T$, it follows then that
	\begin{align*}
		(\mscr{R}\mbf{v})(s)&=\bl(\mscr{D}[\mcl{T}\mbf{v}]\br)(s).
	\end{align*}
	Then, defining $\mcl{C}$ as in Prop.~\ref{prop:Cmap}, we find
	\begin{align*}
		\bl(\mcl{C}Z_{\otimes d}(\mbf{v})\br)(s)
		&=c(s)^T Z_{d}\bl((\mscr{R}^{N}\mbf{v})(s)\br)	\\
		&\qquad=c(s)^T Z_{d}\bl((\mscr{D}^{N}[\mcl{T}\mbf{v}])(s)\br),
	\end{align*}
	Invoking these relations, it follows that for any $\mbf{v}(t)\in L_2[a,b]$,
	\begin{align*}
		&(\mcl{T}\dot{\mbf{v}})(t,s)=\bl(\mcl{C} Z_{\otimes d}(\mbf{v})\br)(t,s)\hspace*{1.4cm}~ \text{ and }~ \mbf{v}(0)=\mbf{v}_{0}	\\
		&\hspace*{2.0cm} \text{if and only if}	\\
		&(\mcl{T}\dot{\mbf{v}})(t,s)=c(s)^T Z_{d}\bl((\mscr{D}^{N}[\mcl{T}\mbf{v}])(t,s)\br) \text{ and }~\mcl{T}\mbf{v}(0)=\mcl{T}\mbf{v}_{0},
	\end{align*}
	Thus, $\mbf{v}(t)\in L_2[a,b]$ solves the PIE defined by $\{\mcl{T},\mcl{C},d\}$ with initial conditions $\mbf{v}_{0}\in L_2[a,b]$ if and only if $\mcl{T}\mbf{v}(t)\in X_{B}[a,b]$ solves the PDE defined by $\{B,c,d\}$ with initial conditions $\mcl{T}\mbf{v}_{0}\in X_{B}[a,b]$
	
	Conversely, by Lemma~\ref{lem:Tmap}, we also know that for any $u\in X_{B}[a,b]$
	\begin{align*}
		u(s)&=\bl(\mcl{T}[\partial_{s}^{N}u]\br)(s),	&	&\text{and}	&
		\partial_{s}^{j}u(s)&=\bl(\mcl{R}_{j}[\partial_{s}^{N}u]\br)(s),
	\end{align*}
	and therefore
	\begin{align*}
		(\mscr{D}u)(s)&=\bl(\mscr{R}[\partial_{s}^{N}u]\br)(s).
	\end{align*}
	Then, by Prop.~\ref{prop:Cmap},
	\begin{align*}
		c(s)^T Z_{d}\bl((\mscr{D}^{N}u)(s)\br)
		&=c(s)^T Z_{d}\bl((\mscr{R}^{N}[\partial_{s}^{N}u])(s)\br)	\\
		&\qquad=\bl(\mcl{C}Z_{\otimes d}(\partial_{s}^{N}u)\br)(s),
	\end{align*}
	It follows that, for any $u(t)\in X_{B}[a,b]$,
	\begin{align*}
		&\dot{u}(t,s)=c(s)^T Z_{d}\bl((\mscr{D}^{N}u)(t,s)\br)\enspace~ \text{ and }~ u(0)=u_{0}	\\
		&\hspace*{2.0cm} \text{if and only if}	\\
		&(\mcl{T}[\partial_{s}^{N}\dot{u}])(t,s)=\bl(\mcl{C}Z_{\otimes d}(\partial_{s}^{N}u)\br)(t,s)~\text{ and }~\partial_{s}^{N}u(0)=\partial_{s}^{N}u_{0},
	\end{align*}
	Thus, $u(t)\in X_{B}[a,b]$ solves the PDE defined by $\{B,c,d\}$ with initial condition $u_{0}\in X_{B}[a,b]$ if and only if $\partial_{s}^{N}u(t)\in L_2[a,b]$ solves the PIE defined by $\{\mcl{T},\mcl{C},d\}$ with initial conditions $\partial_{s}^{N}u_{0}\in L_2[a,b]$, concluding the proof.
\end{proof}

\paragraph*{\tbf{Example}}
Consider Fisher's equation with mixed Dirichlet-Neumann BCs:
\begin{align*}
	\textbf{PDE:}\quad \dot{u}(s,t)&= \nu u_{ss}(s,t) + ru(s,t)[1-u(s,t)],	&	s&\in(0,1),	\\
	\textbf{BCs:}\quad u(0,t)&=0,\qquad u_{s}(1,t)=0.
\end{align*}
To represent this system as a PIE, we use fundamental state $\mbf{v}:=u_{ss}$. Then, defining $\mcl{T}\in\Pi_{3}$ for $\mbf{v}\in L_2[0,1]$ as
\begin{align*}
	\bl(\mcl{T}\mbf{v}\br)(s):=-\int_{0}^{s}\theta\mbf{v}(\theta)d\theta - \int_{s}^{1}s\mbf{v}(\theta)d\theta,
\end{align*}
the state $u=\mcl{T}\mbf{v}\in H_2[0,1]$ will satisfy the mixed Dirichlet-Neumann BCs. Substituting this relation into the PDE, we find that the system can be equivalently represented as a PIE
\begin{align*}%\label{eq:Burgers_PIE_1}
	\textbf{PIE:}\qquad \mcl{T}\dot{\mbf{v}}(t) &= \nu\mbf{v}(t) + r(\mcl{T}\mbf{v}(t))\bl[1-(\mcl{T}\mbf{v}(t))\br]	\\
	&= \bmat{(\nu+r\mcl{T}) &-r(\mcl{T}\otimes\mcl{T})}\slbmat{\mbf{v}(t)\\\mbf{v}(t)\otimes\mbf{v}(t)},
\end{align*}
where, for any $\mbf{w}\in L_2[[0,1]^2]$,
\begin{align*}
	&((\mcl{T}\otimes\mcl{T})\mbf{w})(s)	
	=-2\int_{0}^{s}\!\!\int_{0}^{\theta}r\theta\nu \mbf{w}(\theta,\nu)d\nu d\theta \\ &\quad -2\int_{s}^{1}\!\!\int_{0}^{s}rs\nu\mbf{w}(\theta,\nu)d\nu d\theta	-2\int_{s}^{1}\!\!\int_{s}^{\theta}rs^2 \mbf{w}(\theta,\nu)d\nu d\theta.
\end{align*}


\section{A Stability Test for Quadratic PDEs}\label{sec:stability}

In the previous section, we showed that a class of common nonlinear PDEs can be represented in a manner that is free of BCs, and with dynamics expressed in terms of a linear map applied to a standardized monomial basis $Z_{\otimes d}$. In order to test stability of such distributed-state systems, we could express polynomial Lyapunov Functions (LFs) in terms of the monomial basis $Z_{\otimes d}$ as well, as e.g.  $V(\mbf{u})=\ip{Z_{\otimes d}(\mbf{u})}{\mcl{P}Z_{\otimes d}(\mbf{u})}_{L_2}$. Then, if there exist operators $\mcl{P}\succ 0$ and $\mcl{Q} \preceq 0$ such that $\dot{V}(\mbf{u})=\ip{Z_{d'}^{\otimes}(\mbf{u})}{\mcl{Q}Z_{d'}^{\otimes}(\mbf{u})}_{L_2}$, this proves stability of the system. However, imposing this stability test would require defining an appropriate class of operators $\mcl{P}$ and $\mcl{Q}$, and deriving the necessary formulae for computing e.g. $\dot{V}(\mbf{u})$ -- topics beyond the scope of this paper. Nevertheless, using only the results from this and earlier works, we can already propose a stability test for quadratic PDEs. In particular, consider a quadratic system in the PIE representation,
\begin{align}\label{eq:PIE_quadratic}
	\mcl{T}\dot{\mbf{v}}(t)&=\mcl{A}\mbf{v}(t) +\mcl{B}[\mbf{v}(t)\otimes\mbf{v}(t)].
\end{align}
Parameterizing a quadratic LF $V(\mbf{v})=\ip{\mcl{T}\mbf{v}}{\mcl{P}\mcl{T}\mbf{v}}_{L_2}$ by a PI operator $\mcl{P}=\mcl{P}^*\in \Pi_{3}$, the derivative of this LF along solutions to the PIE can be represented as
\begin{align*}
	\dot{V}(\mbf{v})&=\ip{\srbmat{\mbf{v}\\\mbf{v}\otimes\mbf{v}}}{\bmat{\mcl{A}^*\mcl{P}\mcl{T}+\mcl{T}^*\mcl{P}\mcl{A}&\mcl{T}^*\mcl{P}\mcl{B}\\\mcl{B}^*\mcl{P}\mcl{T}&0}\srbmat{\mbf{v}\\\mbf{v}\otimes\mbf{v}}},
\end{align*}
where e.g. $\mcl{A}^*\in\Pi_{3}$ denotes the adjoint of $\mcl{A}$. Then, it is not difficult to see that this derivative is negative semidefinite if and only if $\mcl{Q}:=[\mcl{A}^*\mcl{P}\mcl{T}+\mcl{T}^*\mcl{P}\mcl{A}]\preceq 0$ and $\ip{\mcl{P}\mcl{T}\mbf{v}}{\mcl{B}[\mbf{v}\otimes\mbf{v}]}_{L_2}=0$ for all $\mbf{v}\in L_2$, as stated in the following proposition.
\begin{prop}\label{prop:cubic_neg}
	Let $\mcl{Q}\in\Pi_{3}$ and $\mcl{R}\in\Pi_{3^2}$, and define $f:L_2\to\R$ as
	\begin{align*}
		f(\mbf{v}):=\ip{\srbmat{\mbf{v}\\\mbf{v}\otimes\mbf{v}}}{\bmat{\mcl{Q}&\mcl{R}\\\mcl{R}^*&0}\srbmat{\mbf{v}\\\mbf{v}\otimes\mbf{v}}}_{L_2}.
	\end{align*}
	Then, for any $\mcl{E}\in\Pi_{3}$, $f(\mbf{v})\leq -\|\mcl{E}\mbf{v}\|_{L_2}^2\leq 0$ for all $\mbf{v}\in L_2$ if and only if $\mcl{Q}\preceq -\mcl{E}^*\mcl{E}$, and $\ip{\mbf{v}}{\mcl{R}[\mbf{v}\otimes\mbf{v}]}_{L_2}=0$ for all $\mbf{v}\in L_2$.
\end{prop}
\begin{proof}
	To prove this result, we first remark that, by definition, $\ip{\mbf{v}}{\mcl{Q}\mbf{v}}_{L_2}\leq -\|\mcl{E}\mbf{v}\|_{L_2}^2=-\ip{\mbf{v}}{\mcl{E}^*\mcl{E}\mbf{v}}_{L_2}$ for all $\mbf{v}\in L_2$ if and only if $\mcl{Q}\preceq -\mcl{E}^*\mcl{E}$. As such, if $\ip{\mbf{v}}{\mcl{R}[\mbf{v}\otimes\mbf{v}]}_{L_2}=0$, it immediately follows that $f(\mbf{v})=\ip{\mbf{v}}{\mcl{Q}\mbf{v}}_{L_2}\leq -\|\mcl{E}\mbf{v}\|_{L_2}^2$ for all $\mbf{v}\in L_2$ if and only if $\mcl{Q}\preceq -\mcl{E}^*\mcl{E}$.
	
	It remains to prove that if $f(\mbf{v})\leq -\|\mcl{E}\mbf{v}\|_{L_2}^2$ for all $\mbf{v}\in L_2$, then $\ip{\mbf{v}}{\mcl{R}[\mbf{v}\otimes\mbf{v}]}_{L_2}= 0$. To prove this implication, suppose that $f(\mbf{v})\leq -\|\mcl{E}\mbf{v}\|_{L_2}^2\leq 0$ for all $\mbf{v}\in L_2$, but assume for contradiction that there exists a function $\mbf{v}^*\in L_2$ such that $\ip{\mbf{v}}{\mcl{R}[\mbf{v}^*\otimes\mbf{v}^*]}_{L_2}\neq 0$. Without loss of generality, we may assume that $\ip{\mbf{v}^*}{\mcl{R}[\mbf{v}^*\otimes\mbf{v}^*]}_{L_2}> 0$, as otherwise we can simply replace $\mbf{v}^*\leftrightarrow -\mbf{v}^*\in L_2$ to obtain the desired inequality. Since $f(\mbf{v})\leq 0$ for all $\mbf{v}\in L_2$, also $f(\mbf{v}^*)\leq 0$, and thus
	\begin{align*}
		\ip{\mbf{v}^*}{\mcl{Q}\mbf{v}^*}_{L_2}&=f(\mbf{v}^*)-2\ip{\mbf{v}^*}{\mcl{R}[\mbf{v}^*\otimes\mbf{v}^*]}_{L_2} <0%	\\
		%		&\leq -2\ip{\mbf{v}}{\mcl{R}[\mbf{v}\otimes\mbf{v}]}_{L_2}< 0.
	\end{align*}
	Now, define
	\begin{align*}
		\lambda:=-\frac{\ip{\mbf{v}^*}{\mcl{Q}\mbf{v}^*}_{L_2}}{\ip{\mbf{v}^*}{\mcl{R}[\mbf{v}^*\otimes\mbf{v}^*]}_{L_2}}>0,
	\end{align*}
	and let $\hat{\mbf{v}}=\lambda\mbf{v}^*\in L_2$. Then,
	% $[\hat{\mbf{v}}\otimes\hat{\mbf{v}}](s,\theta)=\lambda\mbf{v}(s)\lambda\mbf{v}(\theta)=\lambda^2 \mbf{v}\otimes\mbf{v}$, and it follows that
	\begin{align*}
		f(\hat{\mbf{v}})&=\ip{\hat{\mbf{v}}}{\mcl{Q}\hat{\mbf{v}}}_{L_2} +2\ip{\hat{\mbf{v}}}{\mcl{R}[\hat{\mbf{v}}\otimes\hat{\mbf{v}}]}_{L_2}	\\
		&=\lambda^2 \ip{\mbf{v}^*}{\mcl{Q}\mbf{v}^*}_{L_2} +2\lambda^3 \ip{\mbf{v}^*}{\mcl{R}[\mbf{v}^*\otimes\mbf{v}^*]}_{L_2}	\\	
		&=\lambda^2 \bbl[\ip{\mbf{v}^*}{\mcl{Q}\mbf{v}^*}_{L_2} +2\lambda \ip{\mbf{v}^*}{\mcl{R}[\mbf{v}^*\otimes\mbf{v}^*]}_{L_2}\bbr]	\\
		&\hspace*{3.5cm}=-\lambda^2 \ip{\mbf{v}^*}{\mcl{Q}\mbf{v}^*}_{L_2}>0,
	\end{align*}
	contradicting the fact that $f(\mbf{v})\leq 0$ for all $\mbf{v}\in L_2$. Hence, for any $\mbf{v}\in L_2$ we must have $\ip{\mbf{v}}{\mcl{R}[\mbf{v}\otimes\mbf{v}]}_{L_2}= 0$, concluding the proof.
	%	As such, the function $f$ simplifies to
	%	\begin{align*}
		%		f(\mbf{v})=\ip{\mbf{v}}{\mcl{Q}\mbf{v}}_{L_2}.
		%	\end{align*}
	%	Since $f(\mbf{v})\leq 0$ for all $\mbf{v}\in L_2$, it follows that $\mcl{Q}\preceq 0$ (by definition), concluding the prove.
\end{proof}
By Proposition~\ref{prop:cubic_neg}, the derivative of the LF $V(\mbf{v})=\ip{\mcl{T}\mbf{v}}{\mcl{P}\mcl{T}\mbf{v}}_{L_2}$ along solutions to the quadratic PIE in~\eqref{eq:PIE_quadratic} will be negative semdefinite if and only if $\mcl{Q}:=[\mcl{A}^*\mcl{P}\mcl{T}+\mcl{T}^*\mcl{P}\mcl{A}]\preceq 0$ and $\ip{\mcl{P}\mcl{T}\mbf{v}}{\mcl{B}[\mbf{v}\otimes\mbf{v}]}_{L_2}\equiv 0$. Here, we can explicitly compute the value of $\mcl{Q}\in\Pi_{3}$ using the formulae presented in e.g.~\cite{shivakumar2022GPDE_Arxiv}, or simply using the PIETOOLS software suite~\cite{shivakumar2021PIETOOLS}. Using PIETOOLS, we can also declare and solve an optimization program with constraints $\mcl{P}\succ 0$ and $\mcl{Q}\preceq 0$, leaving only the challenge of enforcing $\ip{\mcl{P}\mcl{T}\mbf{v}}{\mcl{B}[\mbf{v}\otimes\mbf{v}]}_{L_2}\equiv0$. For this, we can use the following result.
\begin{prop}\label{prop:PImap_ip}
	Let PI operators $\mcl{B}=\mcl{P}[B]\in\Pi_{3^{2}}$ and $\mcl{Q}=\mcl{P}[Q]\in\Pi_{3}$ be defined by parameters $B(s,\theta,\eta)=\sum_{i,j=1}^{2}\mbf{I}_{ij}(s,\theta,\eta)B_{ij}(s,\theta,\eta)$ and $Q(s,\theta)=\sum_{i=1}^{2}\mbf{I}_{i}(s,\theta)Q_{i}(s,\theta)$. Define the map $\mcl{K}_{\ip{.}{.}}:\Pi_{3}[a,b]\times\Pi_{3^2}[a,b]\to L_0[[a,b]^3]$ as
	{\small%
		\begin{align*}
			&\mcl{K}_{\ip{.}{.}}(\mcl{Q},\mcl{B})(s,\theta,\eta)	\\
			&:=\int_{s}^{b} \bl[K_{11}(s,\theta,\eta,\zeta) +K_{11}(\theta,s,\eta,\zeta) +K_{11}(\eta,s,\theta,\zeta)\br]d\zeta	\\
			&\enspace +\int_{\theta}^{s} \bl[K_{21}(s,\theta,\eta,\zeta) +K_{12}(\theta,s,\eta,\zeta)+K_{12}(\eta,s,\theta,\zeta)\br]d\zeta \\
			&\quad +\int_{\eta}^{\theta} \bl[K_{22}(s,\theta,\eta,\zeta) +K_{22}(\theta,s,\eta,\zeta) +K_{13}(\eta,s,\theta,\zeta)\br]d\zeta	\\
			&\qquad +\int_{a}^{\eta}\bl[K_{23}(s,\theta,\eta,\zeta) +K_{23}(\theta,s,\eta,\zeta) +K_{23}(\eta,s,\theta,\zeta) \br]d\zeta,
	\end{align*}}
	where $K_{ij}(s,\theta,\eta,\zeta)=Q_{i}(\zeta,s)R_{j}(\zeta,\theta,\eta)$ for $i\in\{1,2\}$ and $j\in\{1,2,3\}$, with
	\begin{align*}
		R_{1}(s,\theta,\eta) &= B_{11}(s,\theta,\eta) + B_{11}(s,\eta,\theta),	\\
		R_{2}(s,\theta,\eta) &= B_{21}(s,\theta,\eta) + B_{12}(s,\eta,\theta),	\\
		R_{3}(s,\theta,\eta) &= B_{22}(s,\theta,\eta) + B_{22}(s,\eta,\theta).
	\end{align*}
	Then, if $K=\mcl{K}_{\ip{.}{.}}(\mcl{Q},\mcl{B})$, for any $\mbf{v}\in L_{2}[a,b]$,
	{\small
	\begin{align*}
		\ip{\mcl{Q}\mbf{v}}{\mcl{B}[\mbf{v}\otimes\mbf{v}]}_{L_2}
		=\int_{a}^{b}\!\!\int_{a}^{s}\!\!\int_{a}^{\theta}K(s,\theta,\eta)\thinspace \mbf{v}(s)\mbf{v}(\theta)\mbf{v}(\eta)\thinspace d\eta d\theta ds.
	\end{align*}}	
\end{prop}
\begin{proof}
	The result follows by substituting the expressions $(\mcl{B}[\mbf{v}\otimes\mbf{v}])(\zeta)=\int_{a}^{b}\!\int_{a}^{b}B(\zeta,\theta,\eta)\mbf{v}(\theta,\eta)d\eta d\theta$ and $(\mcl{Q}\mbf{v})(\zeta)=\int_{a}^{b}Q(\zeta,s)\mbf{v}(s)ds$ into the inner product $\ip{\mcl{Q}\mbf{v}}{\mcl{B}[\mbf{v}\otimes\mbf{v}]}_{L_2}=\int_{a}^{b}\bl[(\mcl{Q}\mbf{v})(\zeta)(\mcl{B}[\mbf{v}\otimes\mbf{v}])(\zeta)\br]d\zeta$, and performing standard algebraic manipulations. A full proof is given in Appendix~\ref{appx:proof_PImap_ip}.
\end{proof}

Defining $\mcl{K}_{\ip{.}{.}}$ as in this proposition, it is clear that $\ip{\mcl{P}\mcl{T}\mbf{v}}{\mcl{B}[\mbf{v}\otimes\mbf{v}]}_{L_2}=0$ for all $\mbf{v}\in L_2$ if and only if $\mcl{K}_{\ip{.}{.}}(\mcl{P}\mcl{T},\mcl{B})=0$. Thus, we can declare an optimization program for testing stability of a quadratic PDE as follows.

\begin{lem}
	Let $\{B,c,2\}$ define a quadratic PDE, with associated PIE representation defined by $\{\mcl{T},[\mcl{A},\mcl{B}],2\}$ as in~\eqref{eq:PIE_quadratic}. Suppose that there exist $\epsilon,\delta>0$ and $\mcl{P}=\mcl{P}^*\in\Pi_{3}$ such that
	\begin{align}\label{eq:stability_LPI}
		&\mcl{P}\succ \epsilon I,	\\
		&\mcl{Q}:=[\mcl{A}^*\mcl{P}\mcl{T}+\mcl{T}^*\mcl{P}\mcl{A}]\preceq -\delta\mcl{T}^*\mcl{T},	\notag\\
		&\mcl{K}_{\ip{.}{.}}(\mcl{P}^*\mcl{T},\mcl{B})=0.	\notag
	\end{align}
	Finally, let $\mu=\|\mcl{P}\|_{\mcl{L}_{L_{2}}}$.
	Then, any solution $\mbf{u}(t)$ to the PDE defined by $\{B,c\}$ satisfies
	\begin{align*}
		\|\mbf{u}(t)\|_{L_2}^2\leq \frac{\mu}{\epsilon}\|\mbf{u}(0)\|_{L_2}^2 e^{-\frac{\delta}{\mu}t}.
	\end{align*}
\end{lem}
\begin{proof}	
	To prove stability of the PDE, consider the candidate LF $V:L_2\rightarrow\R$ defined for arbitrary $\mbf{v}\in L_2$ as
	\begin{align*}
		V(\mbf{v})=\ip{\mcl{T}\mbf{v}}{\mcl{P}\mcl{T}\mbf{v}}_{L_2}\geq \epsilon \|\mcl{T}\mbf{v}\|^2_{L_2}.
	\end{align*}
	Since $\|\mcl{P}\|_{\mcl{L}_{L_2}}=\mu$, this function is bounded from above as
	\begin{align*}
		V(\mbf{v})=
		\ip{\mcl{T}\mbf{v}}{\mcl{P}\mcl{T}\mbf{v}}_{L_2}\leq \mu\|\mcl{T}\mbf{v}\|_{L_2}^2.
	\end{align*}
	Now, let $\mbf{u}$ be an arbitrary solution to the PDE defined by $\{B,c\}$, and fix  $\mbf{v}:=\partial_{s}^{N}\mbf{u}$. Then, by Lemma~\ref{lem:PDE2PIE}, $\mbf{u}=\mcl{T}\mbf{v}$, and $\mbf{v}$ is a solution to the PIE defined by $\{\mcl{T},[\mcl{A},\mcl{B}]\}$. As such the temporal derivative of $V$ along $\mbf{v}$ satisfies
	\begin{align*}
		&\dot{V}(\mbf{v})
		=\ip{\mcl{T}\dot{\mbf{v}}}{\mcl{P}\mcl{T}\mbf{v}}_{L_2}
		+\ip{\mcl{T}\mbf{v}}{\mcl{P}\mcl{T}\dot{\mbf{v}}}_{L_2}    \\
		&=\ip{\bmat{\mcl{A}&\mcl{B}}\srbmat{\mbf{v}\\\mbf{v}\otimes\mbf{v}}}{\mcl{P}\mcl{T}\mbf{v}}_{L_2} \!\!\!\!\!+\ip{\mcl{T}\mbf{v}}{\mcl{P}\bmat{\mcl{A}&\mcl{B}}\srbmat{\mbf{v}\\\mbf{v}\otimes\mbf{v}}}_{L_2} \\
		&=\ip{\srbmat{\mbf{v}\\\mbf{v}\otimes\mbf{v}}}{\bmat{\mcl{A}^*\mcl{P}\mcl{T} \!+\! \mcl{T}^*\mcl{P}\mcl{A}&\mcl{T}^*\mcl{P}\mcl{B}\\\mcl{B}^*\mcl{P}\mcl{T}&0}\srbmat{\mbf{v}\\\mbf{v}\otimes\mbf{v}}}_{L_2}    .
		%		\\
		%		&\qquad +2\int_{a}^{b}\int_{a}^{s}\int_{a}^{\theta}\mcl{K}_{\ip{.}{.}}(\mcl{P}\mcl{T},\mcl{B})[\mbf{v}\otimes\mbf{v}\otimes\mbf{v}](s,\theta,\eta)d\eta d\theta ds\\
		%		&\leq -\delta\|\mcl{T}\mbf{v}\|^2_{L_2}
		%		\leq -\frac{\delta}{\mu}V(\mbf{v}).
	\end{align*}
	Here, since $\mcl{K}_{\ip{.}{.}}(\mcl{P}\mcl{T},\mcl{B})=0$, by Proposition~\ref{prop:PImap_ip} we have $\ip{\mbf{v}}{\mcl{T}^*\mcl{P}\mcl{B}[\mbf{v}\otimes\mbf{v}]}_{L_2}=\ip{\mcl{T}\mbf{v}}{\mcl{P}\mcl{B}[\mbf{v}\otimes\mbf{v}]}_{L_2}=0$. Since also $[\mcl{A}^*\mcl{P}\mcl{T} \!+\! \mcl{T}^*\mcl{P}\mcl{A}]\preceq-\delta\mcl{T}^*\mcl{T}$, by Proposition~\ref{prop:cubic_neg} we find
	\begin{align*}
		\dot{V}(\mbf{v})\leq -\delta\|\mcl{T}\mbf{v}\|^2_{L_2}
		\leq -\frac{\delta}{\mu}V(\mbf{v}).
	\end{align*}
	Applying the Gr\"onwall-Bellman inequality, it immediately follows that
	\begin{align*}
		V(\mbf{v}(t))\leq V(\mbf{v}(0))e^{-\frac{\delta}{\mu} t},
	\end{align*}
	and therefore
	\begin{align*}
		\|\mcl{T}\mbf{v}(t)\|^2_{L_2}\leq \frac{\mu}{\epsilon}\|\mcl{T}\mbf{v}(0)\|_{L_2}^2 e^{-\frac{\delta}{\mu} t}.
	\end{align*}
	Finally, since $\mbf{u}=\mcl{T}\mbf{v}$, we conclude that
	\begin{align*}
		\|\mbf{u}(t)\|^2_{L_2}\leq \frac{\mu}{\epsilon}\|\mbf{u}(0)\|_{L_2}^2 e^{-\frac{\delta}{\mu} t}.
	\end{align*}	
\end{proof}

Using the techniques for parameterizing positive PI operators $\mcl{P}\in\Pi_{3}$ by positive matrices shown in e.g.~\cite{shivakumar2022GPDE_Arxiv}, the optimization program in~\eqref{eq:stability_LPI} can be solved using semidefinite programming. In fact, this optimization program can be declared and solved directly using the PIETOOLS software. In the following subsections, we use PIETOOLS to solve this optimization program and verify stability of several nonlinear PDEs. 

%Although this stability test is limited in its applications, it does illustrate how the polynomial PIE representation opens the door towards developing a more general (SOS-based) framework for stability analysis of infinite-dimensional systems. Moreover, even this rudimentary result can be used to test stability of several PDEs, as we show in the following subsections.



\subsection{Burgers' Equation}

Consider Burgers' equation on $s\in[0,1]$, with an added reaction term and Dirichlet BCs:
\begin{align*}
	\textbf{PDE:}\qquad \dot{u}(s,t)&= u_{ss}(s,t) + ru(s,t) - u(s,t) u_{s}(s,t),	\\
	\textbf{BCs:}\qquad u(0,t)&=0,\qquad u(1,t)=0.
\end{align*}
Define PI operators $\mcl{T},\mcl{R}\in\Pi_{3}$ for $\mbf{v}\in L_2[0,1]$ as
{\small%
\begin{align*}
	\bl(\mcl{T}\mbf{v}\br)(s)&=\int_{0}^{s}[s-1]\theta\mbf{u}(\theta)d\theta + \int_{s}^{1}s[\theta-1]\mbf{u}(\theta)d\theta,	\\
	\bl(\mcl{R}\mbf{v}\br)(s)&:=\int_{0}^{s}\theta\mbf{u}(\theta)d\theta + \int_{s}^{1}[\theta-1]\mbf{u}(\theta)d\theta,
\end{align*}}
Given the fundamental state $\mbf{v}:=u_{ss}$, then, $u=\mcl{T}\mbf{v}$ and $u_{s}=\mcl{R}\mbf{v}$. We obtain an equivalent PIE representation as
\begin{align*}%\label{eq:Burgers_PIE_1}
	\textbf{PIE:}\qquad \mcl{T}\dot{\mbf{v}}(t) %&= \mcl{C}Z_{\otimes 2}(\mbf{v})
	=\bmat{1+r\mcl{T}&-(\mcl{T}\otimes\mcl{R})}\slbmat{\mbf{v}(t)\\\mbf{v}(t)\otimes\mbf{v}(t)}.
\end{align*}
Consider the candidate LF $V(u)=\|u\|_{L_2}^2=\|\mcl{T}\mbf{v}\|_{L_2}^2$, i.e. letting $\mcl{P}=1$ in~\eqref{eq:stability_LPI}. Then, for $r=0$, we find
\begin{align*}
	\dot{V}(\mbf{v})=\ip{\mbf{v}}{[\mcl{T}^*+\mcl{T}]\mbf{v}}_{L_2} - 2\ip{\mcl{T}\mbf{v}}{[\mcl{T}\otimes\mcl{R}][\mbf{v}\otimes\mbf{v}]}_{L_2}.
\end{align*}
Here, defining $\mcl{K}_{\ip{.}{.}}$ as in Prop.~\ref{prop:PImap_ip}, we find that $\mcl{K}_{\ip{.}{.}}(\mcl{T},[\mcl{T}\otimes\mcl{R}])\equiv 0$, and thus $\dot{V}(\mbf{v})=\ip{\mbf{v}}{[\mcl{T}^*+\mcl{T}]\mbf{v}}_{L_2}$. Moreover, we can show that $\mcl{T}^*=\mcl{T}=-\mcl{R}^*\mcl{R}$, and thus $[\mcl{T}^*+\mcl{T}]\preceq 0$. We find that $\dot{V}(\mbf{v})\leq 0$ for all $\mbf{v}\in L_2[0,1]$, proving stability of Burgers' equation with Dirichlet BCs.

More generally, for $r>0$, we can solve the optimization program in~\eqref{eq:stability_LPI} using PIETOOLS. Setting $\epsilon=\delta=10^{-6}$, stability can then be verified for any $r\leq 9.8696\approx \pi^2$, approaching the analytic stability limit $r=\pi^2$~\cite{valmorbida2014semi}.


\subsection{Kortweg-De Vries Equation}

Consider an adapted Korteweg-De Vries (KdV) equation on $s\in[0,1]$, with Dirichlet-Neumann BCs:
\begin{align*}
	\textbf{PDE:}\qquad \dot{u}(s,t)&= -u_{sss}(s,t) + u(s,t)[r u(s,t)+6u_{s}(s,t)],	\\
	\textbf{BCs:}\qquad u(0,t)&=0,\qquad u(1,t)=0,\qquad u_{s}(1,t)=0.
\end{align*}
Define PI operators $\mcl{T},\mcl{R}\in\Pi_{3}$ for $\mbf{v}\in L_2[0,1]$ as
{\small
\begin{align*}
	&\bl(\mcl{T}\mbf{v}\br)(s)
	%&=\int_{0}^{s}T_{1}(s,\theta) \mbf{v}(\theta)d\theta + \int_{s}^{1}T_{2}(s,\theta)\mbf{v}(\theta)d\theta,	\\
	:=\int_{0}^{1}\frac{1}{2}[s-1]^2\theta^2 \mbf{v}(\theta)d\theta - \int_{s}^{1}\frac{1}{2}[s-\theta]^2\mbf{v}(\theta)d\theta,	\\
	&\bl(\mcl{R}\mbf{v}\br)(s)
	%=\int_{0}^{s}R_{1}(s,\theta)\mbf{v}(\theta)d\theta + \int_{s}^{1}R_{2}(s,\theta)\mbf{v}(\theta)d\theta,	\\
	:=\int_{0}^{1}[s-1]\theta^2 \mbf{v}(\theta)d\theta - \int_{s}^{1} [s-\theta]\mbf{v}(\theta)d\theta,	
	%\bl(\mcl{Q}\mbf{v}\br)(s)&:=\int_{0}^{s}\theta^2 \mbf{v}(\theta)d\theta + \int_{s}^{1}[\theta^2-1]\mbf{v}(\theta)d\theta.
\end{align*}}
Then, defining fundamental state $\mbf{v}:=u_{sss}$ we have $u=\mcl{T}\mbf{v}$ and $u_{s}=\mcl{R}\mbf{v}$. Imposing this relation in the PDE, the system can be equivalently represented as a PIE
\begin{align*}%\label{eq:Burgers_PIE_1}
	\textbf{PIE:}\qquad \mcl{T}\dot{\mbf{v}}(t) &%= \mcl{C}Z_{\otimes 2}(\mbf{v}(t))
	=\bmat{-1 &\mcl{T}\otimes(r\mcl{T}+6\mcl{R})}\slbmat{\mbf{v}(t)\\\mbf{v}(t)\otimes\mbf{v}(t)}.
\end{align*}
%where for any $\mbf{w}\in L_2[[0,1]^2]$,
%{\small%
%\begin{align*}
%	&\bl((\mcl{T}\otimes\mcl{R})\mbf{w}\br)(s)=\int_{0}^{s}\!\int_{0}^{s}T_{1}(s,\theta)R_{1}(s,\eta) \mbf{w}(\theta,\eta)d\eta d\theta	\\
%	&+\int_{0}^{s}\!\int_{s}^{1}T_{1}(s,\theta)R_{2}(s,\eta) \mbf{w}(\theta,\eta)d\eta d\theta +\int_{s}^{1}\!\int_{0}^{s}T_{2}(s,\theta)R_{1}(s,\eta) \mbf{w}(\theta,\eta)d\eta d\theta \\
%	&\qquad +\int_{s}^{1}\!\int_{0}^{s}T_{2}(s,\theta)R_{2}(s,\eta) \mbf{w}(\theta,\eta)d\eta d\theta.
%\end{align*}}
%\begin{align*}
%	&(\mcl{C}_{2}\mbf{w})(s)=(6(\mcl{T}\otimes\mcl{R})\mbf{w})(s)	\\
%	&=\int_{0}^{s}\!\int_{0}^{s}3[s-1]^3\theta^2\nu^2\mbf{w}(\theta,\nu)d\nu d\theta \\
%	&\hspace*{0.5cm}+\int_{s}^{1}\!\int_{0}^{s}3\bmat{\bbl([s-1]^2\theta^2 -[s-\theta]^2\bbr)&\bbl([s-1]\theta^2 -[s-\theta]\bbr)}\bmat{[s-1]\nu^2\\ [s-1]^2\nu^2}\mbf{w}(\theta,\nu)d\nu d\theta	\\
%	&\hspace*{1.0cm} +\int_{s}^{1}\!\int_{s}^{1}3\bbl([s-1]^2\theta^2 -[s-\theta]^2\bbr)\bbl([s-1]\nu^2 -[s-\nu]\bbr)\mbf{w}(\theta,\nu)d\nu d\theta
%\end{align*}
In the case that $r=0$, stability can be readily verified with the LF $V(u)=\|u\|_{L_2}^2=\|\mcl{T}\mbf{v}\|_{L_2}^2$. Indeed, we find
\begin{align*}
	\dot{V}(\mbf{v})=\ip{\mbf{v}}{[-\mcl{T}^*-\mcl{T}]\mbf{v}}_{L_2} + 12\ip{\mcl{T}\mbf{v}}{[\mcl{T}\otimes\mcl{R}][\mbf{v}\otimes\mbf{v}]}_{L_2},
\end{align*}
%Here, we find $\mcl{K}_{\ip{.}{.}}(\mcl{T},[\mcl{T}\otimes\mcl{R}])=0$, and thus $\dot{V}(\mbf{v})=\ip{\mbf{v}}{[-\mcl{T}^*-\mcl{T}]\mbf{v}}_{L_2}$. Using PIETOOLS, we can verify $[-\mcl{T}^*-\mcl{T}]\leq 0$, proving that the Korteweg-De Vries equation is stable.
where we can show that $\mcl{K}_{\ip{.}{.}}(\mcl{T},[\mcl{T}\otimes\mcl{R}])\equiv 0$ and $[-\mcl{T}^*-\mcl{T}]\preceq 0$, proving stability of the KdV equation. In the more general case $r\geq 0$, using PIETOOLS, we can verify stability for any $r\leq 1.4693$, letting $\epsilon=\delta=10^{-6}$ in~\eqref{eq:stability_LPI}.


%Here, the adjoint $\mcl{T}^*$ is defined for $\mbf{v}\in L_2[0,1]$ as
%\begin{align*}
%	(\mcl{T}^*\mbf{v})(s)=\int_{0}^{s}\frac{1}{2}\bbl[[\theta-1]^2 s^2 -[s-\theta]^2\bbr]d\theta + \int_{s}^{1}\frac{1}{2}[\theta-1]^2s^2 \mbf{v}(\theta)\mbf{v}(\theta)d\theta,
%\end{align*}
%and thus
%\begin{align*}
%	([\mcl{T}+\mcl{T}^*]\mbf{v})(s)
%	=\int_{0}^{s}s[s-1]\theta[\theta-1]\mbf{v}(\theta)d\theta +\int_{s}^{1}s[s-1]\theta[\theta-1]\mbf{v}(\theta)d\theta.
%\end{align*}
%Using PIETOOLS, we can verify that $-[\mcl{T}+\mcl{T}^*]\prec 0$. In particular, we note that $[\mcl{T}+\mcl{T}^*]=\mcl{P}^*\mcl{P}$, where for $\mbf{v}\in L_2[0,1]$,
%\begin{align*}
%	\bl(\mcl{P}\mbf{v}\br)(s)&:=\int_{0}^{s}[\theta-1]\theta \mbf{v}(\theta)d\theta + \int_{s}^{1}[\theta-1]\theta\mbf{v}(\theta)d\theta.
%\end{align*}
%Moreover, we find that for this system too,
%\begin{align*}
%	\mcl{K}_{\ip{.}{.}}(\mcl{T},[\mcl{T}\otimes\mcl{R}])\equiv 0,
%\end{align*}
%and thus $12\ip{\mcl{T}\mbf{v}}{(\mcl{T}\otimes\mcl{R})(\mbf{v}\otimes\mbf{v})}_{L_2}=0$ for all $\mbf{v}\in L_2[0,1]$. We find that $\dot{V}(\mbf{v})<0$ for all $\mbf{v}\in L_2[0,1]$, proving that the Korteweg-De Vries equation with the proposed BCs is stable.
%More generally, adding a quadratic term $\lambda u^2=\lambda [\mcl{T}\otimes\mcl{T}] [\mbf{v}\otimes\mbf{v}]$ to the dynamics, stability can be verified for any $\lambda\leq 1.4693$, letting $\epsilon=\delta=10^{-6}$ in~\eqref{eq:stability_LPI}.


\subsection{Kuramoto-Sivashinsky Equation}

Consider the Kuramoto-Sivashinsky equation (KSE) with Dirichlet and Neumann BCs
\begin{align*}
	\tbf{PDE:}\quad \dot{u}(t,s)&= -u_{ssss}(t,s) - u_{ss}(t,s) - u(t,s)u_{s}(t,s),\quad		\\
	\tbf{BCs:}\quad u(t,0)&=u(t,1)=u_{s}(t,0)=u_{s}(t,1)=0.
\end{align*}
Define PI operators
{\small%
	\begin{align*}
		\bl(\mcl{T}\mbf{v}\br)(s)&:=-\int_{0}^{s}\frac{1}{6}[s-1]^2 \theta^2 [2s\theta-3s+\theta] \mbf{v}(\theta)d\theta\\
		&\qquad -\int_{s}^{1}\frac{1}{6}[\theta-1]^2 s^2[2s\theta-3\theta+s] \mbf{v}(\theta)d\theta,	\\
		\bl(\mcl{R}_{1}\mbf{v}\br)(s)&:=-\int_{0}^{s}\frac{1}{2}[s-1]\theta^2 [2s\theta-3s+1]\mbf{v}(\theta)d\theta \\
		&\qquad -\int_{s}^{1}\frac{1}{2}s[\theta-1]^2[2s\theta +s-2\theta]\mbf{v}(\theta)d\theta,	\\
		\bl(\mcl{R}_{2}\mbf{v}\br)(s)&:=-\int_{0}^{s}\theta^2 [2s\theta-3s-\theta+2]\mbf{v}(\theta)d\theta \\
		&\qquad -\int_{s}^{1}[\theta-1]^2[2s\theta +s-\theta]\mbf{v}(\theta)d\theta.
\end{align*}}
Then, given fundamental state $\mbf{v}=u_{ssss}$, $u=\mcl{T}\mbf{v}$, $u_{s}=\mcl{R}_{1}\mbf{v}$, and $u_{ss}=\mcl{R}_{2}\mbf{v}$. It follows that we can equivalently represent this system as a PIE
\begin{align*}
	\tbf{PIE:}\quad
	\mcl{T}\dot{\mbf{v}}(t)=\bmat{(-\mcl{T}-\mcl{R}_{2}) &(\mcl{T}\otimes\mcl{R}_{1})}\slbmat{\mbf{v}(t)\\\mbf{v}(t)\otimes\mbf{v}(t)}
\end{align*}
Again, stability can be verified using the quadratic LF $V(u)=\|u\|_{L_2}^2=\|\mcl{T}\mbf{v}\|_{L_2}^2$. Moreover, this same LF can be used to prove stability upon replacing the BC $u_{s}(t,1)=0\mapsto u_{ss}(t,1)=0$.




\section{Conclusion}

In this paper, we proposed a linear representation of polynomials on distributed states $\mbf{u}\in L_2$. We defined a subclass of distributed polynomials parameterized by PI operators, proving that this subclass is closed under addition and multiplication. We showed that a large set of polynomial PDEs with linear BCs can be equivalently represented in terms of such distributed polynomials parameterized by PI operators, proposing an equivalent PIE representation of these PDEs. Finally, using this PIE representation, we formulated a stability test for quadratic PDEs using quadratic LFs, and applied this test to verify stability of several examples. Future work may focus on generalizing this stability test to prove stability of higher degree polynomial PDEs, perhaps also using higher degree polynomial LFs.%, abstracting the sum-of-squares framework for stability analysis of ODEs to distributed-state systems.

\vspace*{-0.1cm}


\bibliographystyle{IEEEtran}
\bibliography{bibfile}










\clearpage

\begin{appendices}




\onecolumn

\section{Proof of Proposition~\ref{prop:PImap_ip}}\label{appx:proof_PImap_ip}

\begin{prop}\label{prop:PImap_ip_appx}
	Let PI operators $\mcl{Q}=\mcl{P}[Q]\in\Pi_{3}[a,b]$ and $\mcl{B}=\mcl{P}[B]\in\Pi_{3^{2}}[a,b]$ be defined by the functions $B(s,\theta,\eta)=\sum_{i,j=1}^{2}\mbf{I}_{ij}(s,\theta,\eta)B_{ij}(s,\theta,\eta)$ and $Q=\sum_{j=1}^{2}\mbf{I}_{j}(s,\theta)Q_{j}(s,\theta)$, so that
	\begin{align*}
		(\mcl{Q}\mbf{v})(s)&=\int_{a}^{s}Q_{1}(s,\theta)\mbf{v}(\theta)d\theta +\int_{a}^{s}Q_{1}(s,\theta)\mbf{v}(\theta)d\theta,	\\
		(\mcl{B}\mbf{w})(s)
		&=\int_{a}^{s}\!\!\int_{a}^{s}B_{11}(s,\theta,\eta)\mbf{w}(\theta,\eta)d\eta d\theta +\int_{s}^{b}\!\!\int_{a}^{s}B_{21}(s,\theta,\eta)\mbf{w}(\theta,\eta)d\eta d\theta	\\	
		&\qquad +\int_{a}^{s}\!\!\int_{s}^{b}B_{12}(s,\theta,\eta)\mbf{w}(\theta,\eta)d\eta d\theta +\int_{s}^{b}\!\!\int_{s}^{b}B_{22}(s,\theta,\eta)\mbf{w}(\theta,\eta)d\eta d\theta,
	\end{align*}
	for $\mbf{v}\in L_2[a,b]$ and $\mbf{w}\in L_2[[a,b]^2]$. Define the map $\mcl{K}_{\ip{.}{.}}:\Pi_{3}[a,b]\times\Pi_{3^2}[a,b]\to L_0[[a,b]^3]$ as
	\begin{align*}
		&\mcl{K}_{\ip{.}{.}}(\mcl{Q},\mcl{B})(s,\theta,\eta)		\\
		&:=\int_{s}^{b} \bl[K_{11}(s,\theta,\eta,\zeta) +K_{11}(\theta,s,\eta,\zeta) +K_{11}(\eta,s,\theta,\zeta)\br]d\zeta	
		+\int_{\theta}^{s} \bl[K_{21}(s,\theta,\eta,\zeta) +K_{12}(\theta,s,\eta,\zeta)+K_{12}(\eta,s,\theta,\zeta)\br]d\zeta 	\\
		&\qquad +\int_{\eta}^{\theta} \bl[K_{22}(s,\theta,\eta,\zeta) +K_{22}(\theta,s,\eta,\zeta) +K_{13}(\eta,s,\theta,\zeta)\br]d\zeta	
		+\int_{a}^{\eta}\bl[K_{23}(s,\theta,\eta,\zeta) +K_{23}(\theta,s,\eta,\zeta) +K_{23}(\eta,s,\theta,\zeta) \br]d\zeta,
	\end{align*}
	where $K_{ij}(s,\theta,\eta,\zeta)=Q_{i}(\zeta,s)R_{j}(\zeta,\theta,\eta)$ for $i\in\{1,2\}$ and $j\in\{1,2,3\}$, where
	\begin{align*}
		R_{1}(s,\theta,\eta) &= B_{11}(s,\theta,\eta) + B_{11}(s,\eta,\theta),	&
		R_{2}(s,\theta,\eta) &= B_{21}(s,\theta,\eta) + B_{12}(s,\eta,\theta),	\\
		R_{3}(s,\theta,\eta) &= B_{22}(s,\theta,\eta) + B_{22}(s,\eta,\theta).
	\end{align*}
	Then, if $K=\mcl{K}_{\ip{.}{.}}(\mcl{Q},\mcl{B})$, for any $\mbf{v}\in L_{2}[a,b]$,
	\begin{align*}
		\ip{\mcl{Q}\mbf{v}}{\mcl{B}[\mbf{v}\otimes\mbf{v}]}_{L_2}
		=\int_{a}^{b}\!\int_{a}^{s}\!\int_{a}^{\theta}K(s,\theta,\eta)\thinspace \mbf{v}(s)\mbf{v}(\theta)\mbf{v}(\eta)\thinspace d\eta d\theta ds.
	\end{align*}
\end{prop}
\bigskip
\begin{proof}
	To prove this result, first note that, for any $\mbf{v}\in L_2[a,b]$,
	\begin{align*}
		(\mcl{B}[\mbf{v}\otimes\mbf{v}])(s)
		&=\int_{a}^{s}\!\!\int_{a}^{s}B_{11}(s,\theta,\eta)\mbf{v}(\theta)\mbf{v}(\eta)d\eta d\theta +\int_{s}^{b}\!\!\int_{a}^{s}B_{21}(s,\theta,\eta)\mbf{v}(\theta)\mbf{v}(\eta)d\eta d\theta	\\	
		&\qquad +\int_{a}^{s}\!\!\int_{s}^{b}B_{12}(s,\theta,\eta)\mbf{v}(\theta)\mbf{v}(\eta)d\eta d\theta +\int_{s}^{b}\!\!\int_{s}^{b}B_{22}(s,\theta,\eta)\mbf{v}(\theta)\mbf{v}(\eta)d\eta d\theta	\\
		&=\int_{a}^{s}\!\!\int_{a}^{\theta}B_{11}(s,\theta,\eta)\mbf{v}(\theta)\mbf{v}(\eta)d\eta d\theta +\int_{a}^{s}\!\!\int_{\theta}^{s}B_{11}(s,\theta,\eta)\mbf{v}(\theta)\mbf{v}(\eta)d\eta d\theta +\int_{s}^{b}\!\!\int_{a}^{s}B_{21}(s,\theta,\eta)\mbf{v}(\theta)\mbf{v}(\eta)d\eta d\theta	\\ &\qquad+\int_{a}^{s}\!\!\int_{s}^{b}B_{12}(s,\eta,\theta)\mbf{v}(\eta)\mbf{v}(\theta)d\theta d\eta +\int_{s}^{b}\!\!\int_{s}^{\theta}B_{22}(s,\theta,\eta)\mbf{v}(\theta)\mbf{v}(\eta)d\eta d\theta +\int_{s}^{b}\!\!\int_{\theta}^{b}B_{22}(s,\theta,\eta)\mbf{v}(\theta)\mbf{v}(\eta)d\eta d\theta	\\
		&=\int_{a}^{s}\!\!\int_{a}^{\theta}B_{11}(s,\theta,\eta)\mbf{v}(\theta)\mbf{v}(\eta)d\eta d\theta +\int_{a}^{s}\!\!\int_{\eta}^{s}B_{11}(s,\eta,\theta)\mbf{v}(\eta)\mbf{v}(\theta)d\theta d\eta +\int_{s}^{b}\!\!\int_{a}^{s}B_{21}(s,\theta,\eta)\mbf{v}(\theta)\mbf{v}(\eta)d\eta d\theta	\\ &\qquad+\int_{s}^{b}\!\!\int_{a}^{s}B_{12}(s,\eta,\theta)\mbf{v}(\theta)\mbf{v}(\eta)d\eta d\theta +\int_{s}^{b}\!\!\int_{s}^{\theta}B_{22}(s,\theta,\eta)\mbf{v}(\theta)\mbf{v}(\eta)d\eta d\theta +\int_{s}^{b}\!\!\int_{\eta}^{b}B_{22}(s,\eta,\theta)\mbf{v}(\eta)\mbf{v}(\theta)d\theta d\eta	\\
		&=\int_{a}^{s}\!\!\int_{a}^{\theta}\bl[B_{11}(s,\theta,\eta)+B_{11}(s,\eta,\theta)\br]\mbf{v}(\theta)\mbf{v}(\eta)d\eta d\theta  +\int_{s}^{b}\!\!\int_{a}^{s}\bl[B_{21}(s,\theta,\eta)+B_{12}(s,\eta,\theta)\br]\mbf{v}(\theta)\mbf{v}(\eta)d\eta d\theta\\
		&\qquad +\int_{s}^{b}\!\!\int_{s}^{\theta}\bl[B_{22}(s,\theta,\eta)+B_{22}(s,\eta,\theta)\br]\mbf{v}(\theta)\mbf{v}(\eta)d\eta d\theta	\\
		&=\int_{a}^{s}\!\!\int_{a}^{\theta}R_{1}(s,\theta,\eta)\mbf{v}(\theta)\mbf{v}(\eta)d\eta d\theta  +\int_{s}^{b}\!\!\int_{a}^{s}R_{2}(s,\theta,\eta)\mbf{v}(\theta)\mbf{v}(\eta)d\eta d\theta +\int_{s}^{b}\!\!\int_{s}^{\theta}R_{3}(s,\theta,\eta)\mbf{v}(\theta)\mbf{v}(\eta)d\eta d\theta.
	\end{align*}
	Substituting this expression into the inner product, it follows that
	{\small
	\begin{align}\label{eq:ip_proof_1}
		&\ip{\mcl{Q}\mbf{v}}{\mcl{B}[\mbf{v}\otimes\mbf{v}]}_{L_{2}}	\notag\\
		&=\int_{a}^{b}\bbbbl[
		\int_{a}^{b}\sum_{k=1}^{2}\mbf{I}_{k}(s,\zeta)Q_{k}(s,\zeta)\mbf{v}(\zeta)d\zeta	\notag\\
		&\hspace*{2.0cm} \bbbbl(\int_{a}^{s}\!\!\int_{a}^{\theta}R_{1}(s,\theta,\eta)\mbf{v}(\theta)\mbf{v}(\eta)d\eta d\theta  +\int_{s}^{b}\!\!\int_{a}^{s}R_{2}(s,\theta,\eta)\mbf{v}(\theta)\mbf{v}(\eta)d\eta d\theta +\int_{s}^{b}\!\!\int_{s}^{\theta}R_{3}(s,\theta,\eta)\mbf{v}(\theta)\mbf{v}(\eta)d\eta d\theta\bbbbr)	
		\bbbbr]ds	\notag\\
		&=\!\int_{a}^{b}\bbbbl[
		\int_{a}^{b}\sum_{k=1}^{2}\mbf{I}_{k}(\zeta,s)Q_{k}(\zeta,s)\mbf{v}(s)ds \notag\\
		&\hspace*{2.0cm} \bbbbl(\int_{a}^{\zeta}\!\!\int_{a}^{\theta}R_{1}(\zeta,\theta,\eta)\mbf{v}(\theta)\mbf{v}(\eta)d\eta d\theta  +\!\int_{\zeta}^{b}\!\!\int_{a}^{\zeta}R_{2}(\zeta,\theta,\eta)\mbf{v}(\theta)\mbf{v}(\eta)d\eta d\theta +\!\int_{\zeta}^{b}\!\!\int_{\zeta}^{\theta}R_{3}(\zeta,\theta,\eta)\mbf{v}(\theta)\mbf{v}(\eta)d\eta d\theta\bbbbr)	
		\bbbbr]d\zeta	\notag\\
		&=\!\int_{a}^{b}\bbbbl[\int_{s}^{b}Q_{1}(\zeta,s)
		\bbbbl(\int_{a}^{\zeta}\!\!\int_{a}^{\theta}R_{1}(\zeta,\theta,\eta)\mbf{v}(\theta)\mbf{v}(\eta)d\eta d\theta  +\!\int_{\zeta}^{b}\!\!\int_{a}^{\zeta}R_{2}(\zeta,\theta,\eta)\mbf{v}(\theta)\mbf{v}(\eta)d\eta d\theta +\!\int_{\zeta}^{b}\!\!\int_{\zeta}^{\theta}R_{3}(\zeta,\theta,\eta)\mbf{v}(\theta)\mbf{v}(\eta)d\eta d\theta\bbbbr)	d\zeta	\notag\\
		&\hspace*{1.5cm} +\!\int_{a}^{s}Q_{2}(\zeta,s)
		\bbbbl(\int_{a}^{\zeta}\!\!\int_{a}^{\theta}R_{1}(\zeta,\theta,\eta)\mbf{v}(\theta)\mbf{v}(\eta)d\eta d\theta  +\!\int_{\zeta}^{b}\!\!\int_{a}^{\zeta}R_{2}(\zeta,\theta,\eta)\mbf{v}(\theta)\mbf{v}(\eta)d\eta d\theta +\!\int_{\zeta}^{b}\!\!\int_{\zeta}^{\theta}R_{3}(\zeta,\theta,\eta)\mbf{v}(\theta)\mbf{v}(\eta)d\eta d\theta\bbbbr)	d\zeta
		\bbbbr] \mbf{v}(s) ds	\notag\\
		&=\!\int_{a}^{b}\bbbbl[
		\int_{s}^{b}\!\!\int_{a}^{\zeta}\!\!\int_{a}^{\theta}K_{11}(s,\theta,\eta,\zeta)\mbf{v}(\theta)\mbf{v}(\eta) d\eta d\theta d\zeta +\!\int_{s}^{b}\!\!\int_{\zeta}^{b}\!\!\int_{a}^{\zeta}K_{12}(s,\theta,\eta,\zeta)\mbf{v}(\theta)\mbf{v}(\eta)d\eta d\theta d\zeta +\!\int_{s}^{b}\!\!\int_{\zeta}^{b}\!\!\int_{\zeta}^{\theta}K_{13}(s,\theta,\eta,\zeta)\mbf{v}(\theta)\mbf{v}(\eta)d\eta d\theta d\zeta		\notag\\
		&\hspace*{0.75cm}
		+\!\int_{a}^{s}\!\!\int_{a}^{\zeta}\!\!\int_{a}^{\theta}K_{21}(s,\theta,\eta,\zeta)\mbf{v}(\theta)\mbf{v}(\eta)d\eta d\theta d\zeta +\!\int_{a}^{s}\!\!\int_{\zeta}^{b}\!\!\int_{a}^{\zeta}K_{22}(s,\theta,\eta,\zeta)\mbf{v}(\theta)\mbf{v}(\eta)d\eta d\theta d\zeta +\!\int_{a}^{s}\!\!\int_{\zeta}^{b}\!\!\int_{\zeta}^{\theta}K_{23}(s,\theta,\eta,\zeta)\mbf{v}(\theta)\mbf{v}(\eta)d\eta d\theta d\zeta
		\bbbbr] \mbf{v}(s) ds
	\end{align}	}
	Here we note that
	{%\small
	\begin{align*}
		\int_{a}^{b}&\!\!\int_{s}^{b}\!\!\int_{a}^{\zeta}\!\!\int_{a}^{\theta}K_{11}(s,\theta,\eta,\zeta)\mbf{v}(s)\mbf{v}(\theta)\mbf{v}(\eta) \; d\eta d\theta d\zeta ds	\\
		&=\int_{a}^{b}\!\!\int_{s}^{b}\!\!\int_{a}^{s}\!\!\int_{a}^{\theta}K_{11}(s,\theta,\eta,\zeta)\mbf{v}(s)\mbf{v}(\theta)\mbf{v}(\eta) \; d\eta d\theta d\zeta ds +\int_{a}^{b}\!\!\int_{s}^{b}\!\!\int_{s}^{\zeta}\!\!\int_{a}^{\theta}K_{11}(s,\theta,\eta,\zeta)\mbf{v}(s)\mbf{v}(\theta)\mbf{v}(\eta) \; d\eta d\theta d\zeta ds	\\
		&=\int_{a}^{b}\!\!\int_{a}^{s}\!\!\int_{s}^{b}\!\!\int_{a}^{\theta}K_{11}(s,\theta,\eta,\zeta)\mbf{v}(s)\mbf{v}(\theta)\mbf{v}(\eta) \; d\eta d\zeta d\theta ds +\int_{a}^{b}\!\!\int_{s}^{b}\!\!\int_{\theta}^{b}\!\!\int_{a}^{\theta}K_{11}(s,\theta,\eta,\zeta)\mbf{v}(s)\mbf{v}(\theta)\mbf{v}(\eta) \; d\eta d\zeta d\theta ds	\\
		&=\int_{a}^{b}\!\!\int_{a}^{s}\!\!\int_{a}^{\theta}\!\!\int_{s}^{b}K_{11}(s,\theta,\eta,\zeta)\mbf{v}(s)\mbf{v}(\theta)\mbf{v}(\eta) \; d\zeta d\eta d\theta ds +\int_{a}^{b}\!\!\int_{s}^{b}\!\!\int_{a}^{\theta}\!\!\int_{\theta}^{b}K_{11}(s,\theta,\eta,\zeta)\mbf{v}(s)\mbf{v}(\theta)\mbf{v}(\eta) \; d\zeta d\eta  d\theta ds	\\
		&=\int_{a}^{b}\!\!\int_{a}^{s}\!\!\int_{a}^{\theta}\!\!\int_{s}^{b}K_{11}(s,\theta,\eta,\zeta)\mbf{v}(s)\mbf{v}(\theta)\mbf{v}(\eta) \; d\zeta d\eta d\theta ds +\int_{a}^{b}\!\!\int_{\theta}^{b}\!\!\int_{a}^{s}\!\!\int_{s}^{b}K_{11}(\theta,s,\eta,\zeta)\mbf{v}(s)\mbf{v}(\theta)\mbf{v}(\eta) d\zeta d\eta ds d\theta	\\
		&=\int_{a}^{b}\!\!\int_{a}^{s}\!\!\int_{a}^{\theta}\!\!\int_{s}^{b}K_{11}(s,\theta,\eta,\zeta)\mbf{v}(s)\mbf{v}(\theta)\mbf{v}(\eta) \; d\zeta d\eta d\theta ds +\int_{a}^{b}\!\!\int_{a}^{s}\!\!\int_{a}^{s}\!\!\int_{s}^{b}K_{11}(\theta,s,\eta,\zeta)\mbf{v}(s)\mbf{v}(\theta)\mbf{v}(\eta)\;  d\zeta d\eta d\theta ds	\\
		&=\int_{a}^{b}\!\!\int_{a}^{s}\!\!\int_{a}^{\theta}\!\!\int_{s}^{b}K_{11}(s,\theta,\eta,\zeta)\mbf{v}(s)\mbf{v}(\theta)\mbf{v}(\eta) \; d\zeta d\eta d\theta ds +\int_{a}^{b}\!\!\int_{a}^{s}\!\!\int_{a}^{\theta}\!\!\int_{s}^{b}K_{11}(\theta,s,\eta,\zeta)\mbf{v}(s)\mbf{v}(\theta)\mbf{v}(\eta) \; d\zeta d\eta d\theta ds	\\ 
		&\qquad +\int_{a}^{b}\!\!\int_{a}^{s}\!\!\int_{\theta}^{s}\!\!\int_{s}^{b}K_{11}(\theta,s,\eta,\zeta)\mbf{v}(s)\mbf{v}(\theta)\mbf{v}(\eta) \; d\zeta d\eta d\theta ds	\\
		&=\int_{a}^{b}\!\!\int_{a}^{s}\!\!\int_{a}^{\theta}\!\!\int_{s}^{b}K_{11}(s,\theta,\eta,\zeta)\mbf{v}(s)\mbf{v}(\theta)\mbf{v}(\eta) \; d\zeta d\eta d\theta ds +\int_{a}^{b}\!\!\int_{a}^{s}\!\!\int_{a}^{\theta}\!\!\int_{s}^{b}K_{11}(\theta,s,\eta,\zeta)\mbf{v}(s)\mbf{v}(\theta)\mbf{v}(\eta) \; d\zeta d\eta d\theta ds	\\ 
		&\qquad +\int_{a}^{b}\!\!\int_{a}^{s}\!\!\int_{\eta}^{s}\!\!\int_{s}^{b}K_{11}(\eta,s,\theta,\zeta)\mbf{v}(s)\mbf{v}(\theta)\mbf{v}(\eta) \; d\zeta d\theta d\eta ds	\\
		&=\int_{a}^{b}\!\!\int_{a}^{s}\!\!\int_{a}^{\theta}\!\!\int_{s}^{b}K_{11}(s,\theta,\eta,\zeta)\mbf{v}(s)\mbf{v}(\theta)\mbf{v}(\eta) \; d\zeta d\eta d\theta ds +\int_{a}^{b}\!\!\int_{a}^{s}\!\!\int_{a}^{\theta}\!\!\int_{s}^{b}K_{11}(\theta,s,\eta,\zeta)\mbf{v}(s)\mbf{v}(\theta)\mbf{v}(\eta) \; d\zeta d\eta d\theta ds	\\ 
		&\qquad +\int_{a}^{b}\!\!\int_{a}^{s}\!\!\int_{a}^{\theta}\!\!\int_{s}^{b}K_{11}(\eta,s,\theta,\zeta)\mbf{v}(s)\mbf{v}(\theta)\mbf{v}(\eta) \; d\zeta d\eta d\theta ds	\\
		&=\int_{a}^{b}\!\!\int_{a}^{s}\!\!\int_{a}^{\theta}\!\!\int_{s}^{b}\bbbl[K_{11}(s,\theta,\eta,\zeta) +K_{11}(\theta,s,\eta,\zeta) +K_{11}(\eta,s,\theta,\zeta) \bbbr]d\zeta\; \mbf{v}(s)\mbf{v}(\theta)\mbf{v}(\eta) d\eta d\theta ds
	\end{align*}}
	Similarly
	{%\small
	\begin{align*}
		\int_{a}^{b}&\!\!\int_{a}^{s}\!\!\int_{a}^{\zeta}\!\!\int_{a}^{\theta}K_{21}(s,\theta,\eta,\zeta) \mbf{v}(s)\mbf{v}(\theta)\mbf{v}(\eta) \; d\eta d\theta d\zeta ds 
		+\int_{a}^{b}\!\!\int_{s}^{b}\!\!\int_{\zeta}^{b}\!\!\int_{a}^{\zeta}K_{12}(s,\theta,\eta,\zeta)\mbf{v}(s)\mbf{v}(\theta)\mbf{v}(\eta) \; d\eta d\theta d\zeta ds 	\\
		&=\int_{a}^{b}\!\!\int_{a}^{s}\!\!\int_{\theta}^{s}\!\!\int_{a}^{\theta}K_{21}(s,\theta,\eta,\zeta) \mbf{v}(s)\mbf{v}(\theta)\mbf{v}(\eta) \; d\eta d\zeta d\theta ds +\int_{a}^{b}\!\!\int_{s}^{b}\!\!\int_{s}^{\theta}\!\!\int_{a}^{\zeta}K_{12}(s,\theta,\eta,\zeta)\mbf{v}(s)\mbf{v}(\theta)\mbf{v}(\eta) \; d\eta d\zeta d\theta ds	\\
		&=\int_{a}^{b}\!\!\int_{a}^{s}\!\!\int_{a}^{\theta}\!\!\int_{\theta}^{s} K_{21}(s,\theta,\eta,\zeta) \mbf{v}(s)\mbf{v}(\theta)\mbf{v}(\eta) \; d\zeta d\eta d\theta ds +\int_{a}^{b}\!\!\int_{s}^{b}\!\!\int_{s}^{\theta}\!\!\int_{a}^{s}K_{12}(s,\theta,\eta,\zeta)\mbf{v}(s)\mbf{v}(\theta)\mbf{v}(\eta) \; d\eta d\zeta d\theta ds	\\ &\qquad +\int_{a}^{b}\!\!\int_{s}^{b}\!\!\int_{s}^{\theta}\!\!\int_{s}^{\zeta}K_{12}(s,\theta,\eta,\zeta)\mbf{v}(s)\mbf{v}(\theta)\mbf{v}(\eta) \; d\eta d\zeta d\theta ds	\\
		&=\int_{a}^{b}\!\!\int_{a}^{s}\!\!\int_{a}^{\theta}\!\!\int_{\theta}^{s} K_{21}(s,\theta,\eta,\zeta) \mbf{v}(s)\mbf{v}(\theta)\mbf{v}(\eta) \; d\zeta d\eta d\theta ds +\int_{a}^{b}\!\!\int_{s}^{b}\!\!\int_{a}^{s}\!\!\int_{s}^{\theta}K_{12}(s,\theta,\eta,\zeta)\mbf{v}(s)\mbf{v}(\theta)\mbf{v}(\eta) \; d\zeta d\eta d\theta ds	\\ &\qquad +\int_{a}^{b}\!\!\int_{s}^{b}\!\!\int_{s}^{\theta}\!\!\int_{\eta}^{\theta}K_{12}(s,\theta,\eta,\zeta)\mbf{v}(s)\mbf{v}(\theta)\mbf{v}(\eta) \; d\zeta d\eta d\theta ds	\\ 
		&=\int_{a}^{b}\!\!\int_{a}^{s}\!\!\int_{a}^{\theta}\!\!\int_{\theta}^{s} K_{21}(s,\theta,\eta,\zeta) \mbf{v}(s)\mbf{v}(\theta)\mbf{v}(\eta) \; d\zeta d\eta d\theta ds +\int_{a}^{b}\!\!\int_{a}^{s}\!\!\int_{a}^{\theta}\!\!\int_{\theta}^{s}K_{12}(\theta,s,\eta,\zeta)\mbf{v}(s)\mbf{v}(\theta)\mbf{v}(\eta) \; d\zeta d\eta d\theta ds	\\ &\qquad +\int_{a}^{b}\!\!\int_{a}^{s}\!\!\int_{\theta}^{s}\!\!\int_{\eta}^{s}K_{12}(\theta,s,\eta,\zeta)\mbf{v}(s)\mbf{v}(\theta)\mbf{v}(\eta) \; d\zeta d\eta d\theta ds	\\
		&=\int_{a}^{b}\!\!\int_{a}^{s}\!\!\int_{a}^{\theta}\!\!\int_{\theta}^{s} K_{21}(s,\theta,\eta,\zeta) \mbf{v}(s)\mbf{v}(\theta)\mbf{v}(\eta) \; d\zeta d\eta d\theta ds +\int_{a}^{b}\!\!\int_{a}^{s}\!\!\int_{a}^{\theta}\!\!\int_{\theta}^{s}K_{12}(\theta,s,\eta,\zeta)\mbf{v}(s)\mbf{v}(\theta)\mbf{v}(\eta) \; d\zeta d\eta d\theta ds	\\ &\qquad +\int_{a}^{b}\!\!\int_{a}^{s}\!\!\int_{a}^{\theta}\!\!\int_{\theta}^{s}K_{12}(\eta,s,\theta,\zeta)\mbf{v}(s)\mbf{v}(\theta)\mbf{v}(\eta) \; d\zeta d\eta d\theta ds	\\
		&=\int_{a}^{b}\!\!\int_{a}^{s}\!\!\int_{a}^{\theta}\!\!\int_{\theta}^{s}\bbbl[K_{21}(s,\theta,\eta,\zeta) +K_{12}(\theta,s,\eta,\zeta) +K_{12}(\eta,s,\theta,\zeta)\bbbr]d\zeta\; \mbf{v}(s)\mbf{v}(\theta)\mbf{v}(\eta) d\eta d\theta ds
	\end{align*}}

	and
	{%\small
	\begin{align*}
		\int_{a}^{b}&\!\!\int_{a}^{s}\!\!\int_{\zeta}^{b}\!\!\int_{a}^{\zeta}K_{22}(s,\theta,\eta,\zeta)\mbf{v}(s)\mbf{v}(\theta)\mbf{v}(\eta) \; d\eta d\theta d\zeta ds
		+\int_{a}^{b}\!\!\int_{s}^{b}\!\!\int_{\zeta}^{b}\!\!\int_{\zeta}^{\theta}K_{13}(s,\theta,\eta,\zeta)\mbf{v}(s)\mbf{v}(\theta)\mbf{v}(\eta) \; d\eta d\theta d\zeta ds	\\
		&=\int_{a}^{b}\!\!\int_{a}^{s}\!\!\int_{\zeta}^{s}\!\!\int_{a}^{\zeta}K_{22}(s,\theta,\eta,\zeta)\mbf{v}(s)\mbf{v}(\theta)\mbf{v}(\eta) \; d\eta d\theta d\zeta ds +\int_{a}^{b}\!\!\int_{a}^{s}\!\!\int_{s}^{b}\!\!\int_{a}^{\zeta}K_{22}(s,\theta,\eta,\zeta)\mbf{v}(s)\mbf{v}(\theta)\mbf{v}(\eta) \; d\eta d\theta d\zeta ds \\
		&\qquad +\int_{a}^{b}\!\!\int_{s}^{b}\!\!\int_{s}^{\theta}\!\!\int_{\zeta}^{\theta}K_{13}(s,\theta,\eta,\zeta)\mbf{v}(s)\mbf{v}(\theta)\mbf{v}(\eta) \; d\eta d\zeta d\theta ds	\\
		&=\int_{a}^{b}\!\!\int_{a}^{s}\!\!\int_{a}^{\theta}\!\!\int_{a}^{\zeta}K_{22}(s,\theta,\eta,\zeta)\mbf{v}(s)\mbf{v}(\theta)\mbf{v}(\eta) \; d\eta d\zeta d\theta ds +\int_{a}^{b}\!\!\int_{s}^{b}\!\!\int_{a}^{s}\!\!\int_{a}^{\zeta}K_{22}(s,\theta,\eta,\zeta)\mbf{v}(s)\mbf{v}(\theta)\mbf{v}(\eta) \; d\eta d\zeta d\theta ds \\
		&\qquad +\int_{a}^{b}\!\!\int_{s}^{b}\!\!\int_{s}^{\theta}\!\!\int_{s}^{\eta}K_{13}(s,\theta,\eta,\zeta)\mbf{v}(s)\mbf{v}(\theta)\mbf{v}(\eta) \; d\zeta d\eta d\theta ds \\
		&=\int_{a}^{b}\!\!\int_{a}^{s}\!\!\int_{a}^{\theta}\!\!\int_{\eta}^{\theta}K_{22}(s,\theta,\eta,\zeta)\mbf{v}(s)\mbf{v}(\theta)\mbf{v}(\eta) \; d\zeta d\eta d\theta ds +\int_{a}^{b}\!\!\int_{a}^{s}\!\!\int_{a}^{\theta}\!\!\int_{a}^{\zeta}K_{22}(\theta,s,\eta,\zeta)\mbf{v}(s)\mbf{v}(\theta)\mbf{v}(\eta) \; d\eta d\zeta d\theta ds \\
		&\qquad +\int_{a}^{b}\!\!\int_{a}^{s}\!\!\int_{a}^{\theta}\!\!\int_{\eta}^{\theta}K_{13}(\eta,s,\theta,\zeta)\mbf{v}(s)\mbf{v}(\theta)\mbf{v}(\eta) \; d\zeta d\eta d\theta ds	\\
		&=\int_{a}^{b}\!\!\int_{a}^{s}\!\!\int_{a}^{\theta}\!\!\int_{\eta}^{\theta}\bbbl[K_{22}(s,\theta,\eta,\zeta) +K_{22}(\theta,s,\eta,\zeta) +K_{13}(\eta,s,\theta,\zeta) \bbbr]d\zeta\; \mbf{v}(s)\mbf{v}(\theta)\mbf{v}(\eta) d\eta d\theta ds
	\end{align*}}

	Finally
	{%\small
	\begin{align*}
		\int_{a}^{b}&\!\!\int_{a}^{s}\!\!\int_{\zeta}^{b}\!\!\int_{\zeta}^{\theta}K_{23}(s,\theta,\eta,\zeta)\mbf{v}(s)\mbf{v}(\theta)\mbf{v}(\eta) \; d\eta d\theta d\zeta ds	\\
		&=\int_{a}^{b}\!\!\int_{a}^{s}\!\!\int_{\zeta}^{s}\!\!\int_{\zeta}^{\theta}K_{23}(s,\theta,\eta,\zeta)\mbf{v}(s)\mbf{v}(\theta)\mbf{v}(\eta) \; d\eta d\theta d\zeta ds +\int_{a}^{b}\!\!\int_{a}^{s}\!\!\int_{s}^{b}\!\!\int_{\zeta}^{\theta}K_{23}(s,\theta,\eta,\zeta)\mbf{v}(s)\mbf{v}(\theta)\mbf{v}(\eta) \; d\eta d\theta d\zeta ds	\\
		&=\int_{a}^{b}\!\!\int_{a}^{s}\!\!\int_{a}^{\theta}\!\!\int_{\zeta}^{\theta}K_{23}(s,\theta,\eta,\zeta)\mbf{v}(s)\mbf{v}(\theta)\mbf{v}(\eta) \; d\eta d\zeta d\theta ds +\int_{a}^{b}\!\!\int_{s}^{b}\!\!\int_{a}^{s}\!\!\int_{\zeta}^{\theta}K_{23}(s,\theta,\eta,\zeta)\mbf{v}(s)\mbf{v}(\theta)\mbf{v}(\eta) \; d\eta d\zeta d\theta ds	\\
		&=\int_{a}^{b}\!\!\int_{a}^{s}\!\!\int_{a}^{\theta}\!\!\int_{a}^{\eta}K_{23}(s,\theta,\eta,\zeta)\mbf{v}(s)\mbf{v}(\theta)\mbf{v}(\eta) \; d\zeta d\eta d\theta ds +\int_{a}^{b}\!\!\int_{s}^{b}\!\!\int_{a}^{s}\!\!\int_{\zeta}^{s}K_{23}(s,\theta,\eta,\zeta)\mbf{v}(s)\mbf{v}(\theta)\mbf{v}(\eta) \; d\eta d\zeta d\theta ds	\\
		&\qquad +\int_{a}^{b}\!\!\int_{s}^{b}\!\!\int_{a}^{s}\!\!\int_{s}^{\theta}K_{23}(s,\theta,\eta,\zeta)\mbf{v}(s)\mbf{v}(\theta)\mbf{v}(\eta) \; d\eta d\zeta d\theta ds	\\
		&=\int_{a}^{b}\!\!\int_{a}^{s}\!\!\int_{a}^{\theta}\!\!\int_{a}^{\eta}K_{23}(s,\theta,\eta,\zeta)\mbf{v}(s)\mbf{v}(\theta)\mbf{v}(\eta) \; d\zeta d\eta d\theta ds +\int_{a}^{b}\!\!\int_{s}^{b}\!\!\int_{a}^{s}\!\!\int_{a}^{\eta}K_{23}(s,\theta,\eta,\zeta)\mbf{v}(s)\mbf{v}(\theta)\mbf{v}(\eta) \; d\zeta d\eta d\theta ds	\\
		&\qquad +\int_{a}^{b}\!\!\int_{s}^{b}\!\!\int_{s}^{\theta}\!\!\int_{a}^{s}K_{23}(s,\theta,\eta,\zeta)\mbf{v}(s)\mbf{v}(\theta)\mbf{v}(\eta) \; d\zeta d\eta d\theta ds	\\
		&=\int_{a}^{b}\!\!\int_{a}^{s}\!\!\int_{a}^{\theta}\!\!\int_{a}^{\eta}K_{23}(s,\theta,\eta,\zeta)\mbf{v}(s)\mbf{v}(\theta)\mbf{v}(\eta) \; d\zeta d\eta d\theta ds +\int_{a}^{b}\!\!\int_{a}^{s}\!\!\int_{a}^{\theta}\!\!\int_{a}^{\eta}K_{23}(\theta,s,\eta,\zeta)\mbf{v}(s)\mbf{v}(\theta)\mbf{v}(\eta) \; d\zeta d\eta d\theta ds	\\
		&\qquad +\int_{a}^{b}\!\!\int_{a}^{s}\!\!\int_{\theta}^{s}\!\!\int_{a}^{\theta}K_{23}(\theta,s,\eta,\zeta)\mbf{v}(s)\mbf{v}(\theta)\mbf{v}(\eta) \; d\zeta d\eta d\theta ds	\\
		&=\int_{a}^{b}\!\!\int_{a}^{s}\!\!\int_{a}^{\theta}\!\!\int_{a}^{\eta}\bbbl[K_{23}(s,\theta,\eta,\zeta) +K_{23}(\theta,s,\eta,\zeta) +K_{23}(\eta,s,\theta,\zeta) \bbbr]d\zeta\; \mbf{v}(s)\mbf{v}(\theta)\mbf{v}(\eta) d\eta d\theta ds
	\end{align*}}

	Substituting these relations into~\eqref{eq:ip_proof_1}, we find
	
	{\small
	\begin{align*}
		&\ip{\mcl{Q}\mbf{v}}{\mcl{B}[\mbf{v}\otimes\mbf{v}]}_{L_{2}}	\\
		&=\!\int_{a}^{b}\bbbbl[
		\int_{s}^{b}\!\!\int_{a}^{\zeta}\!\!\int_{a}^{\theta}K_{11}(s,\theta,\eta,\zeta)\mbf{v}(\theta)\mbf{v}(\eta) d\eta d\theta d\zeta +\!\int_{s}^{b}\!\!\int_{\zeta}^{b}\!\!\int_{a}^{\zeta}K_{12}(s,\theta,\eta,\zeta)\mbf{v}(\theta)\mbf{v}(\eta)d\eta d\theta d\zeta +\!\int_{s}^{b}\!\!\int_{\zeta}^{b}\!\!\int_{\zeta}^{\theta}K_{13}(s,\theta,\eta,\zeta)\mbf{v}(\theta)\mbf{v}(\eta)d\eta d\theta d\zeta		\\
		&\hspace*{0.75cm}
		+\!\int_{a}^{s}\!\!\int_{a}^{\zeta}\!\!\int_{a}^{\theta}K_{21}(s,\theta,\eta,\zeta)\mbf{v}(\theta)\mbf{v}(\eta)d\eta d\theta d\zeta +\!\int_{a}^{s}\!\!\int_{\zeta}^{b}\!\!\int_{a}^{\zeta}K_{22}(s,\theta,\eta,\zeta)\mbf{v}(\theta)\mbf{v}(\eta)d\eta d\theta d\zeta +\!\int_{a}^{s}\!\!\int_{\zeta}^{b}\!\!\int_{\zeta}^{\theta}K_{23}(s,\theta,\eta,\zeta)\mbf{v}(\theta)\mbf{v}(\eta)d\eta d\theta d\zeta
		\bbbbr] \mbf{v}(s) ds	\\
		&=\int_{a}^{b}\!\!\int_{a}^{s}\!\!\int_{a}^{\theta}\bbbbl[
		\int_{s}^{b}K_{11}(s,\theta,\eta,\zeta)d\zeta +\!\int_{s}^{b}K_{11}(\theta,s,\eta,\zeta)d\zeta +\!\int_{s}^{b}K_{11}(\eta,s,\theta,\zeta)d\zeta	\\ 
		&\hspace*{2.0cm}+\!\int_{\theta}^{s}K_{12}(\theta,s,\eta,\zeta)d\zeta +\!\int_{\theta}^{s}K_{12}(\eta,s,\theta,\zeta)d\zeta		+\!\int_{\eta}^{\theta}K_{13}(\eta,s,\theta,\zeta)d\zeta	\\
		&\hspace*{2.5cm}  +\int_{\theta}^{s}K_{21}(s,\theta,\eta,\zeta)d\zeta +\!\int_{\eta}^{\theta}K_{22}(s,\theta,\eta,\zeta)d\zeta  		+\!\int_{\eta}^{\theta}K_{22}(\theta,s,\eta,\zeta)d\zeta	\\
		&\hspace*{3.0cm}+\!\int_{a}^{\eta}K_{23}(s,\theta,\eta,\zeta)d\zeta +\!\int_{a}^{\eta}K_{23}(\theta,s,\eta,\zeta)d\zeta +\!\int_{a}^{\eta}K_{23}(\eta,s,\theta,\zeta)d\zeta
		\bbbbr] \mbf{v}(s)\mbf{v}(\theta)\mbf{v}(\eta) \thinspace d\eta d\theta ds	\\
		&=\int_{a}^{b}\int_{a}^{s}\int_{a}^{\theta}K(s,\theta,\eta)[\mbf{v}\otimes\mbf{v}\otimes\mbf{v}](s,\theta,\eta)d\eta d\theta ds
	\end{align*}}
\end{proof}



\end{appendices}

\end{document} 