\documentclass[letterpaper, 10 pt, journal, twoside]{IEEEtran}
\pdfminorversion=4
\usepackage[noadjust]{cite}
\let\proof\relax
\let\endproof\relax
\usepackage{amsmath,amssymb,amsfonts,amsthm}
\usepackage{algorithm}
\usepackage[algo2e]{algorithm2e}
\usepackage{algorithmic}
\usepackage{wasysym,xcolor}
\usepackage{hyperref}
\usepackage{mathrsfs}
\usepackage{graphicx,float}
\usepackage{capt-of}
\usepackage{textcomp,soul,comment}
\usepackage{pgfplots}
\pgfplotsset{compat=1.10}
\def\BibTeX{{\rm B\kern-.05em{\sc i\kern-.025em b}\kern-.08em
    T\kern-.1667em\lower.7ex\hbox{E}\kern-.125emX}}

\newcommand{\anna}[1]{{\color{teal} #1}}
\newcommand{\rahel}[1]{{\color{blue} #1}}
\newcommand{\JK}[1]{{\color{magenta} #1}}

\newtheorem{remark}{Remark}

%\theoremstyle{definition}
\newtheorem{definition}{Definition}
\newtheorem{theorem}{Theorem}
\newtheorem{proposition}{Proposition}
\newtheorem{assumption}{Assumption}
\newtheorem{lemma}{Lemma}


\allowdisplaybreaks

\begin{document}
\title{Active Learning-based \\
Model Predictive Coverage Control}
\author{Rahel Rickenbach, Johannes K\"ohler, Anna Scampicchio, Melanie N. Zeilinger, Andrea Carron
\thanks{
All authors are members within the Institute for Dynamic Systems and Control (IDSC), ETH Z{\"u}rich.
{\tt\footnotesize [rrahel|jkoehle|ascampicc|\\mzeilinger|carrona]@ethz.ch}}}

\maketitle

\begin{abstract}
The problem of coverage control, i.e., of coordinating multiple agents to optimally cover an area, arises in various applications. 
However, existing coverage algorithms face two major challenges: (1) dealing with nonlinear dynamics while respecting system and safety critical constraints, and (2) performing the task in an initially unknown environment. We solve the coverage problem by using a hierarchical framework, where references are calculated at a central server and passed to the agents' local model predictive control (MPC) tracking schemes.
Furthermore, a probabilistic exploration-exploitation trade-off is deployed to ensure that the environment is actively explored by the agents. In addition, we derive a control framework that avoids the hierarchical structure by integrating the reference optimization in the MPC formulation.  
Active learning is then performed drawing inspiration from Upper Confidence Bound (UCB) approaches. We guarantee recursive feasibility and convergence to an optimal configuration for all developed control architectures. Furthermore, all methods are tested and compared on hardware using a miniature car platform.
\end{abstract}

\begin{IEEEkeywords}
Coverage Control, NL Predictive Control, Cooperative Control, Machine Learning, Agents and Autonomous Systems%, Centroidal Voronoi Partitions.
\end{IEEEkeywords}

\section{Introduction}
\label{sec:introduction}
 
A key problem in multi-agent systems consists in optimally covering a finite area with respect to environmental demands, which  
are reflected by a density function of measurable values of interest. 
This task goes under the name of coverage control, and it can be rephrased as placing the agents at the centroids of their Voronoi partition induced by the density function~\cite{Du1999}. Applications are versatile. One is, e.g.,  
the autonomous re-positioning of self-driving taxis according to the population density, providing a faster and more environmentally friendly service. Another example, illustrated in Figure~\ref{fig:explanatoryfigure}, are firefighting planes that are autonomously and optimally distributing themselves with respect to the heat map of a certain area.
From these examples, the challenges that emerge are twofold and intertwined. The first is designing a coverage control architecture for agents with nonlinear dynamics while ensuring collision avoidance and respecting safety constraints. The second consists in dealing with initially unknown environments, i.e., unknown density functions characterizing the optimal coverage problem. In this case, the considered area needs to be explored during the process relying on the agents' sensing capabilities. 
\begin{figure} [t]
\centering
\includegraphics[width=0.45\textwidth]{figures/coverage_control_explained_9-compressed.pdf}
\caption{Illustration of coverage control problem and partitioning of environment at the example of firefighting planes and environmental demands defined by the resulting heat map.}
\label{fig:explanatoryfigure}
\vspace{-1.2em}
\end{figure} 
The goal of this paper is to design a coverage control framework that addresses both of the aforementioned challenges. 

\subsubsection*{Related Work}
\label{subsubsec:relatedwork}
In classical works such as~\cite{Cortes2004,Bullo2012}, the coverage control problem is rigorously solved under the assumptions that the environment is perfectly known and that dynamics are single integrators. Nonlinear dynamics and state/input constraints can be taken into account by using a  model predictive control (MPC) scheme~\cite{Grune2011,rawlings2017model}, which was pursued in~\cite{Carron2017,Kohler2018,Farina2015}. In all the aforementioned references, the implemented coverage algorithm is hierarchical, i.e., a reference is calculated before being passed to a tracking MPC~\cite{Limon2018}. Moreover, they all require some non-trivial offline design for the terminal ingredients in MPC to prove convergence and recursive feasibility. Furthermore, none of them addressed coverage control in an unknown environment.\\
To cope with this second challenge, ``exploitation" targeted to the coverage control task needs to be combined with ``exploration" given by data collection performed via active learning~\cite{Li2006,Campbell1990} to improve the estimate of the initially unknown density function. The problem of coverage control in an unknown environment is investigated in~\cite{Schwager2009,Todescato2017,McDonald2021,prajapat2022}, where different strategies to balance exploration and exploitation are proposed. 
However, in these works the agents' dynamics are once again assumed to be single integrators. To the best of the authors' knowledge, the general coverage control problem encompassing an unknown environment and nonlinear constrained dynamics has not yet been addressed in the literature. 

\begin{figure} [t]
\centering
\includegraphics[trim={0cm 0cm 0.8cm 0cm},clip,width=0.45\textwidth]{figures/coverage_clarification_diagram_vert-compressed.pdf}
\caption{Simplified illustration of the developed two- and one-layers algorithm. Considering density $\phi$, partitions $\mathbb{O}$, references $r$, state $x$, inputs $u$, positions $p$, as well as setpoint positions $\bar{p}$.} 
\label{fig:coverageclarification}
\vspace{-1.5em}
\end{figure}

\subsubsection*{Contributions}
\label{subsubsec:contributions}
We propose an MPC-based framework to perform coverage control under nonlinear dynamics and constraints, and extend it to consider an unknown environment. In particular, we first consider a coverage control strategy inspired by~\cite{Carron2017}, passing references, calculated by the server, to each agent's individual tracking MPC~\cite{Limon2018}. In this way, the multi-agent architecture is assumed centralized when computing the partitions of the considered area, but decentralized to calculate each agent's control policy. In the remainder of the paper, we will refer to this implementation strategy as ``two-layers approach''. A visual explanation is presented in the scheme on the left of Figure~\ref{fig:coverageclarification}. In order to facilitate application to more complex, nonlinear systems, we use the tools in~\cite{Grune2011,Soloperto2021} to derive a tracking MPC scheme that does not rely on offline designed terminal ingredients. Moreover, thanks to this predictive control set-up we are able to directly include collision avoidance constraints. We extend this approach to an initially unknown environment by leveraging the probabilistic exploration-exploitation decision
proposed in~\cite{Todescato2017} and include it in the hierarchical control set-up. \\
We also demonstrate how to simplify and overcome the hierarchical structure of the latter and directly integrate the calculation of the next optimal configuration into the MPC cost of each agent. To the best of the authors' knowledge, this has not been addressed in the literature. In the following, we refer to this non-hierarchical method, schematized on the right of Figure~\ref{fig:coverageclarification}, as ``one-layer approach". Moreover, to provide a potentially more efficient exploration behaviour, we address unknown environments by using an Upper Confidence Bound (UCB) approach, similar to Bayesian optimization methods~\cite{Auer2002,Srinivas2010}.\\
We prove recursive feasibility, constraint satisfaction and convergence to an optimal configuration for both control architectures (one-layer and two-layers, Figure~\ref{fig:coverageclarification}) and for both known and unknown environments. 
Moreover, all approaches are tested and compared on hardware, using the miniature racing cars Chronos in combination with CRS, an open-source software framework for control and robotics~\cite{carron2022,Froehlich2021,Tearle2021}%\footnote{An accompanying video detailing the contributions of the paper is available at the following link: \url{https://youtu.be/ej0WAizJKVM}}.

\subsubsection*{Outline}
\label{subsubsec:roadmap}
The remainder of the paper is structured as follows. Section~\ref{sec:problemFormulation} states the coverage control problem. This is followed by a presentation of  preliminaries regarding MPC and Bayesian learning in Section~\ref{sec:controlandlearningframework}.  Sections~\ref{sec:twolayers} and~\ref{sec:onelayer} detail the proposed two-layers and one-layer methods, respectively. In addition, we state the main theoretical results on convergence, satisfaction of safety-critical constraints and recursive feasibility for both set-ups with known and unknown environment. The proofs are deferred to the Appendix. Section~\ref{sec:experiments} gathers all the experimental results, and Section~\ref{sec:conclusions} ends the paper by discussing the benefits of the proposed approaches and drawing conclusions.
 
\subsubsection*{Notation}
\label{subsubsec:notation}
Throughout the paper, $\Vert \! \cdot \! \Vert$ indicates the Euclidean norm and $\mathcal{N}$ the normal distribution. The set of all non-negative real numbers is given by $\mathbb{R}_{+}$ and the set of natural numbers by $\mathbb{N}$. We define the ball $\mathbb{B}_{r}^{b} = \{x \in \mathbb{R}^b\vert \,\Vert x \Vert\leq r \}$ and indicate with $\ominus$ the Pontryagin difference.  Further, we consider some arbitrary small but fixed constant $\epsilon > 0$ and for each set $\mathbb{V}\subset\mathbb{R}^b$ we denote $\mathbb{V}^{\mathrm{int}}:= \mathbb{V} \ominus \mathbb{B}_{\epsilon}^{b}$. Given a compact set $\mathbb{D}\subset\mathbb{R}^b$ and a  continuous function $f: \mathbb{D} \rightarrow \mathbb{R}$, we define
\begin{equation}
    f(s)_{\mathbb{D}} := \min_{\tilde{s} \in \mathbb{D}}f(s - \tilde{s}).
    \label{eq:minsetdistance}
\end{equation}

\section{Problem Formulation}
\label{sec:problemFormulation} 
This section is devoted to the statement of the coverage control problem, together with the specification of the nonlinear constrained dynamics that are considered in the paper.

\subsection{Dynamics and Constraints}
We consider $M$ agents on a convex area \mbox{$\mathbb{A}\in \mathbb{R}^{D}$} and the agents are described using time-invariant, discrete-time, nonlinear dynamics. For agent $i$, the dynamics are given by
\begin{equation}
\begin{split}
    &x_{i,k+1} = f_{i}(x_{i,k}, u_{i,k}) \\
    &p_{i,k} = C_{i}x_{i,k},
\end{split}
\label{eq:nonlineardynamics}
\end{equation}
where $x_{i,k} \in \mathbb{R}^{n_{i}}$ and $u_{i,k} \in \mathbb{R}^{m_{i}}$ denote the state and input at time $k$, respectively. The vector $p_{i,k} \in \mathbb{R}^D$ indicates the position of agent $i$ at time $k$, which is usually chosen as Cartesian coordinates in D=2 or D=3. Additionally, all agents are subject to state and input constraints, i.e., $x_{i,k} \in \mathbb{X}_{i}$ and $u_{i,k} \in \mathbb{U}_{i}$ for $k \in \mathbb{N}$. Note that we consider relatively general heterogeneous dynamics for the different agents. Additionally, we make the subsequent assumption.
\begin{assumption} The agent's dynamics $f_{i}$ are assumed to be known and Lipschitz continuous with Lipschitz constant $\mathcal{L}_{i}$. Moreover, each agent's state can be measured perfectly and its state and input constraints are compact.
\label{assumption:dynamics}
\end{assumption}

\subsection{Coverage Control}
\label{subsec:coverage}
We indicate the agents' position with \mbox{$p = [p_{1}, \hdots, p_{M}]^{T} \in \mathbb{R}^{M \times D}$} and denote with $\mathbb{O} = \{\mathbb{O}_{1}, \hdots, \mathbb{O}_{M}\}$ an arbitrary collection of polytopes partitioning $\mathbb{A}$. The problem of coverage control can be mathematically stated as the following optimization problem:
\begin{equation}
\begin{split}
    \min_{p,\,\mathbb{O}} \:\underbrace{\sum_{i=1}^{M} \int_{\mathbb{O}_{i}} g(\Vert q - p _{i} \Vert)\phi(q)dq}_{H(p,\mathbb{O})},\, \ p_{i} \in \mathbb{O}_{i},
\end{split}
\label{eq:cortescost}
\end{equation}
where $\phi: \mathbb{A} \rightarrow \mathbb{R}_+$ is the so-called density function which acts as a measure of information on the environment~$\mathbb{A}$, and \mbox{$g$: $\mathbb{R}_{+} \rightarrow \mathbb{R}_{+}$} is an arbitrary function representing the agents' sensing capability that needs to be continuous, non-decreasing, and is typically chosen as the quadratic function. The function $H(p, \mathbb{O})$ is known as the locational optimization cost. It was shown in~\cite{Du1999} that the optimal set  $\mathbb{O}$ in~\eqref{eq:cortescost} is given by the Voronoi tessellation~\cite{Senechal1995} denoted by \mbox{$\mathbb{W}_{p} = \{\mathbb{W}_{p,1}, \hdots, \mathbb{W}_{p,M}\}$}, with
\begin{equation}
\begin{split}
    \mathbb{W}_{p,i} = \{ q \in \mathbb{A} \; \vert\; \Vert q - p_{i} \Vert \leq \Vert q - p_{j} \Vert, \, \forall j \neq i \},
\end{split}
\label{eq:voronoidef}
\end{equation}
and each agent's optimal position is given by the centroids $c(\mathbb{W}_{p}, \phi) = [c_1(\mathbb{W}_{p,1}, \phi), \hdots, c_M(\mathbb{W}_{p,M}, \phi)]$, with
\begin{equation}
\begin{split}
     c_i(\mathbb{W}_{p,i}, \phi) = \left(\int_{\mathbb{W}_{p,i}}\phi(q)dq\right)^{-1} \left(\int_{\mathbb{W}_{p,i}}q\phi(q)dq\right).
\end{split}
\label{eq:voronoicenterdef}
\end{equation}
The optimal solution in terms of both partition and positions is called centroidal Voronoi configuration. Note that all the sets in $\mathbb{W}_{p}$ are convex provided that $\mathbb{A}$ is convex~\cite{Du1999}. The coverage control problem can hence be reduced to ensuring $p$ converges to $c(\mathbb{W}_{p}, \phi)$ with $\mathbb{W}_{p}$ according to~\eqref{eq:voronoidef}. Classically, for integrator dynamics this is achieved using the Lloyd algorithm~\cite{Lloyd1982}, which consists in an iterative update of the form $p_{i, k+1} = c(\mathbb{W}_{p_{k},i}, \phi)$. We study this problem for more general dynamics, and both for known environments, i.e., a known density~$\phi$, and in case $\phi$ is unknown and needs to be learned online.

\subsection{Collision Avoidance}
\label{subsec:collisionavoidance}
To prevent agents from colliding with each other, it needs to be ensured that, for all time instances $k \in \mathbb{N}$, 
\begin{equation}
\Vert p_{i,k} - p_{j,k}\Vert \geq (r_{i,\max} + r_{j,\max}), \ \forall i\neq j, 
\label{eq:colavoidancerequirement}
\end{equation}
where $r_{i,\max}$ indicates a radius that covers its respective agent. For later use, we define
$r_{\max} = \max\{r_{1,\max},\hdots,r_{M,\max}\}$.
The developed collision avoidance strategy will deploy 
the constructed Voronoi partitions~\eqref{eq:voronoidef}. 

\vspace{-0.7em}
\section{Control and Learning Framework}
\label{sec:controlandlearningframework}
In this section, we introduce the core control and learning tool of the developed methods that allow for a safe coverage movement while respecting the 
given dynamics. Therefore, we recall a nonlinear tracking MPC formulation and complement such a set-up with 
collision avoidance constraints based on the Voronoi partition (Section~\ref{subsec:nonlineartrackingmpc}). Additionally, we define the Bayesian linear regression (BLR) strategy that will be deployed in all the learning-based approaches to deal with an unknown environment (Section~\ref{subsec:measurementcol}).

\subsection{Nonlinear Tracking MPC}
\label{subsec:nonlineartrackingmpc}

To allow for the inclusion of state, input and collision avoidance constraints, each agent $i$ is controlled by a nonlinear tracking MPC~\cite{Limon2018,Soloperto2021}.
Its cost reads as
\begin{equation}
\begin{split}
&J_{i}(x_{i,\cdot \vert k}, u_{i,\cdot \vert k}, s_{i,k}, r_{i,k}) \\ & = V_{N,i}(x_{i,\cdot \vert k},u_{i,\cdot \vert k},s_{i,k}) + \ell_{T,i}(\bar{p}_{i,k} - r_{i,k} )\\
& = \sum_{l=0}^{N-1}\ell_{i}(x_{i,l\vert k}, u_{i,l\vert k}, s_{i,k}) + \ell_{T,i}(\bar{p}_{i,k} - r_{i,k} ),
\end{split}
\label{eq:generalnonlineartrackingmpccost}
\end{equation}
where we use the symbol ``$\cdot$" to denote all values of the index $l=\{0,\dots,N-1\}$. Similar to standard MPC~\cite{Grune2011}, the tracking cost~$V_{N,i}$ sums a continuous stage cost $\ell_i$ over a finite horizon~$N$. Thereby, the stage cost should steer the system to the aritifical setpoint~$s_{i,k} = (\bar{x}_{i,k},\bar{u}_{i,k}) \in \mathbb{S}_{i}$, with
\begin{equation}
\begin{split}
    \mathbb{S}_{i} \! = \! \{(x,u) \! \in \! \mathbb{R}^{n_{i}+m_{i}}\,\vert\, x \! \in \mathbb{X}^{\mathrm{int}}_{i}, u \! \in \mathbb{U}^{\mathrm{int}}_{i}, x \! = \! f_{i}(x,u) \},
\end{split}
\label{eq:steadystateset}
\end{equation} which enables the tracking of piece-wise constant references~$r_{i,k}$. The according steady-state position is obtained as \mbox{$\bar{p}_{i,k} = C_{i}s_{i,k}$} and its difference to the reference is penalized in the continuous target cost $\ell_{T,i}$. To ensure~\eqref{eq:colavoidancerequirement}, we consider the current Voronoi partitions, indicated by $\mathbb{W}$, and define \mbox{$\bar{\mathbb{W}}_{i} := \mathbb{W}_{i} \ominus \mathbb{B}_{r_{\max}}^D$}  for all $i = \{1,\hdots,M\}$. Then, for all $i \neq j$ and all $p_{i} \in \bar{\mathbb{W}}_{i}, p_{j} \in \bar{\mathbb{W}}_{j}$, inequality~\eqref{eq:colavoidancerequirement} holds. \\
Consequently, a set of  position and steady-state constraints is added. The first can be encoded as $p_{i,k} \in \bar{\mathbb{W}}_{i}$, while the latter is given by modifying the set of setpoints as follows:
\begin{equation}
\begin{split}
    \mathbb{S}_{\mathbb{W}_{i}} = & \{(x,u) \in \mathbb{S}_{i}\;\vert \; C_{i}x \in \bar{\mathbb{W}}_{i}^{\mathrm{int}} \},
\end{split}
\label{eq:steadystatesetwithcollavoidance}
\end{equation}
and define its projection on the position space as 
\begin{equation}
\begin{split}
    \mathbb{S}_{\mathbb{W}_{i}}^{\mathrm{p}} = & \{p \in \mathbb{W}_i^{\mathrm{int}} \;\vert \; p = C_{i}x, \forall x : \exists u \text{ s.t. } (x,u) \in \mathbb{S}_{\mathbb{W}_{i}}\} \notag.
\end{split}
\label{eq:steadystatesetwithcollavoidancepositionspace}
\end{equation}
The resulting MPC problem, of agent $i$ at time $k$, given state~$x_i$, reference~$r_i$ and partition $\mathbb{W}_i$, reads as
\begin{subequations}
\begin{align}
\min_{x_{i,\cdot \vert k}, u_{i,\cdot \vert k},s_{i,k}}&J_{i}(x_{i,\cdot \vert k}, u_{i,\cdot \vert k}, s_{i,k}, r_{i,k}) \label{eq:generalcostwithcolavoidance} \\ &\qquad x_{i,0\vert k} = x_{i,k} \label{eq:generalinit}\\
&\qquad x_{i, l+1\vert k} = f_{i}(x_{i,l\vert k}, u_{i,l\vert k}) \label{eq:generaldyn} \\
&\qquad x_{i,\cdot\vert k} \in \mathbb{X}_{i},\; u_{i,\cdot\vert k} \in \mathbb{U}_{i}  \label{eq:generalstateconst}\\
&\qquad V_{N,i}(x_{i,\cdot \vert k},u_{i,\cdot \vert k}, s_{i,k}) \leq V_{\max,i} \label{eq:generalvarbound}\\
&\qquad p_{i,l\vert k} = C_{i}x_{i,l\vert k} \in \bar{\mathbb{W}}_{i} \label{eq:generalvoronoiconst}\\
&\qquad s_{i,k} \in \mathbb{S}_{\mathbb{W}_{i}}. \label{eq:generalsteadystatesetconstwithcolavoidance} \\
& \qquad l = 1, \hdots, N-1 \label{eq:1toN-1withcolavoidance}.
\end{align}
\label{eq:nonlineartrackingmpcwithcolavoidance}
\end{subequations}
Condition~\eqref{eq:generalvarbound} imposes an additional user-chosen bound~$V_{\max,i}$ on the tracking cost $V_{N,i}$, which is a technical condition to ensure closed-loop properties under reasonable assumptions and without terminal ingredients, see~\cite{Soloperto2021,Boccia2014}. For each agent, a solution of~\eqref{eq:nonlineartrackingmpcwithcolavoidance} are the optimal state and input trajectories~$x_{i,\cdot \vert k}^{*}$ and $u_{i,\cdot \vert k}^{*}$, as well as the optimal setpoint  \mbox{$\bar{s}_{i,k}^{*} = (\bar{x}_{i, k}^{*}$, $\bar{u}_{i,k}^{*})$}. It exists, given continuity of $\ell_{i},\ell_{T,i}$, due to the compact constraints. The problem is solved in a receding horizon fashion, i.e., at every time $k$ we solve~\eqref{eq:nonlineartrackingmpcwithcolavoidance} and apply the first input of the obtained input sequence, that is $u_{i,k}=u^*_{i,0|k}$. We denote by $V_{N,i}^{*}(x_{i,k},s_{i,k}^{*},\mathbb{W}_i) = V_{N,i}(x_{i,\cdot \vert k}^{*},u_{i,\cdot \vert k}^{*}, s_{i,k}^{*})$ the optimal tracking cost for the considered partition, by $\ell^*_i(x_k,s_k^*) = \min_{u \in \mathbb{U}_{i}} \ell_{i}(x_{i,k},u,s_{i,k}^*)$ the one step optimal stage cost, and by $u^*_{x_{i,k},s_{i,k}^*}$ its minimizing input, which is assumed to be equal to $\bar{u}_{i,k}^*$.
In the following, we discuss the stability properties of the MPC scheme~\eqref{eq:nonlineartrackingmpcwithcolavoidance} using the theory in~\cite{Boccia2014}. To this end, we require a local stabilizability condition, also known as \textit{exponential cost controllability} (\cite[Assumption 1]{Boccia2014}) in the literature.
\begin{assumption} For every agent $i \in \{1,\hdots,M\}$ there exist constants $c_{i}, \gamma_{i} > 0$ such that for any $\mathbb{W}_{i} \subset \mathbb{A}$, with $\bar{\mathbb{W}}^{\mathrm{int}}_{i} \neq \emptyset$, $s \in \mathbb{S}_{\mathbb{W}_{i}}$, $N \in \mathbb{N}$ and any $x \in \mathbb{R}^{n_{i}}$ satisfying $\ell_{i}^{*}(x,s) \leq c_{i}$, we have
\begin{equation}
    V_{N,i}^{*}(x,s) \leq \gamma_{i} \ell_{i}^{*}(x,s). 
\label{eq:assumption2gammavmax}
\end{equation}
\label{assumption:expocostcontrollability}
\end{assumption}
\vspace{-1.4em}
In addition, we require that the stage cost $\ell_{i}$ is positive definite w.r.t. the setpoint $s_{i}$, which is often achieved using a quadratic stage cost $\ell_{i}$. However, in~\cite{Muller2017}, it was shown that Assumption~\ref{assumption:expocostcontrollability} cannot be satisfied for non-holonomic systems using a quadratic stage cost. To avoid this issue, we use the results in~\cite{Rosenfelder2021, Coron2020, Worthmann2016} to devise a polynomial stage cost for the experiments with non-holonomic robots (cf. Section~\ref{sec:experiments}). To properly characterize such polynomial stage costs we introduce the following distance function
\begin{equation}
d_i(\zeta) = \sqrt{\zeta_1^{a_{i,1}} + \zeta_2^{a_{i,2}} + \cdots + \zeta_{n_i+m_i}^{a_{i,n_i+m_i}}}.\label{eq:distance}
\end{equation} which uses even exponents $a_{i,j} \in \mathbb{N}, j = 1,\hdots,n_i+m_i$.
\begin{assumption} For every agent $i \in \{1,\hdots,M\}$ there are constants $\alpha_{1,i}, \alpha_{2,i} > 0$ such that, for all $x \in \mathbb{X}_{i}$, $s \in \mathbb{S}_{\mathbb{W}_{i}}$, and their respective $u^*_{x,s} \in \mathbb{U}_{i}$
, the following holds: 
\begin{equation}
\begin{split}
    \alpha_{1,i}d_{i}((x,u_{x,s}^{*}) \! - \! s)^{2} \! \leq  \! \ell_{i}^{*}(x,s) \! \leq \! \alpha_{2,i}d_{i}((x,u_{x,s}^{*}) \! - \! s)^{2}.
\end{split}
\label{eq:boundstagecosttwolayer}
\end{equation}
Moreover, there exist constants $\xi_{1,i}, \xi_{2,i} \geq 0$ such that for any two admissible setpoints \mbox{$s_{1} \! = \! (\bar{x}_{1},\bar{u}_{1})$ and $s_{2} \! = \! (\bar{x}_{2},\bar{u}_{2})$ $\in \mathbb{S}_{\mathbb{W}_{i}}$,}
\begin{equation}
\begin{split}
    \ell_{i}(x,u,s_{1}) \leq  \xi_{1,i}\cdot \ell_{i}(x,u,s_{2}) + \xi_{2,i}\cdot d_{i}(s_{1}-s_{2})^{2}.
\end{split}
\label{eq:boundtwosteadystatestwolayer}
\end{equation}
\label{assumption:boundedbyd}
\end{assumption}
\vspace{-1.3em}
\begin{remark}\label{remark:commentAssumption23}
By applying recursively the argument in~\cite[Section A]{Soloperto2021}, it is easily shown that Assumption~\ref{assumption:boundedbyd} holds for polynomial stage costs with only even exponents.\label{remark:remarkgammamax}
\end{remark}
\noindent Given these requirements, it is proven in~\cite{Boccia2014,Soloperto2021} that, for a sufficiently long prediction horizon, the tracking cost is a valid Lyapunov function with respect to a fixed steady state.
\begin{theorem}\cite[Theorem 4]{Boccia2014} Let Assumptions~\ref{assumption:expocostcontrollability} and~\ref{assumption:boundedbyd} hold. Then, for any $\mathbb{W}_{i}\subset \mathbb{A}$, with $\bar{\mathbb{W}}^{\mathrm{int}}_{i} \neq \emptyset$, any $V_{\max,i} > 0$ and any $\alpha_{N,i} \in (0,1)$, there exists a horizon $N^{*} \in \mathbb{N}_{\geq 0}$ such that for all $k \in \mathbb{N}$, $N \geq N^{*}$, $s_k \in \mathbb{S}_{\mathbb{W}_{i}}$ and all $x_k$ with $V_{N,i}^{*}(x_{k},s_k) \leq V_{\max,i}$ 
\begin{equation}
\begin{split}
    &V_{N,i}^{*}(f_i(x_{k},u_{0\vert k}^{*}),s_k^*,\mathbb{W}_i) - V_{N,i}^{*}(x_{k},s_k^*,\mathbb{W}_i) \\ &\leq  - \bar{\alpha}_{N,i} \ell_{i}^{*}(x_{k},s_k^*).
\end{split}
\label{eq:theorem5twolayer}
\end{equation}
\label{theorem:theo4boccia}
\end{theorem}
\vspace{-1.0em}
\noindent Note that the developed approaches, see Figure~\ref{fig:coverageclarification}, will use a Voronoi partion update to ensure a coverage cost decrease. However, feasibility of the nonlinear tracking MPC explicitly depends on the partition $ \bar{\mathbb{W}}_{i}^{\mathrm{int}}$: hence, any update might cause infeasibility of optimization problem~\eqref{eq:nonlineartrackingmpcwithcolavoidance} or invalidate the closed-loop properties in Theorem~\ref{theorem:theo4boccia}. A procedure for preserving feasibility under the time-varying collision avoidance constraints, i.e., taking Voronoi partition updates into account, is introduced in the following. For this purpose, let $\mathbb{W}'$ be the candidate Voronoi update and consider the following input and state sequences
\begin{subequations}
    \begin{align}
    \hat{u}_{i, \cdot \vert k} = &[u^{*}_{i,1\vert k}, \hdots, u^{*}_{i,N-2\vert k}, \bar{u}_{i,k}^{*}, \bar{u}_{i,k}^{*}]^{\top}\label{eq:uhat}, \\
    \hat{x}_{i, \cdot \vert k} = &[x^{*}_{i,1\vert k}, \hdots, x^{*}_{i,N-1\vert k}, f(x^{*}_{i,N-1\vert k},\bar{u}_{i,k}^{*}), \label{eq:xhat} \\ &f(f(x^{*}_{i,N-1\vert k},\bar{u}_{i,k}^{*}),\bar{u}_{i,k}^{*})]^{\top}.\nonumber
    \end{align}
    \label{eq:lemma1proposal}
\end{subequations}
\noindent The idea is to only update the partitions if the candidate sequence remains feasible, i.e., if, for all $l=\{1,...,N+1\}$ and all $i=\{1,...,M\}$, 
\begin{equation}
    \hat{x}_{i,l \vert k}\in\mathbb{X}_i, \ C_{i}\hat{x}_{i,l \vert k} \in \bar{\mathbb{W}}_{i}' \ \text{and} \  C_{i}\bar{x}_{i,k}^{*} \in \bar{\mathbb{W}}'^{\mathrm{int}}_{i},
    \label{eq:feasibilitycond}
\end{equation}
which will be later used to ensure closed-loop properties (cf. proof of Theorem~\ref{theorem:twolayers} and~\ref{theorem:onelayer} in the Appendix).
To ensure that the problem is well-posed, we define a feasible configuration as a configuration $p$ for which the distance between all agents is bigger than or equal to $2(r_{\max} + \epsilon)$.
\begin{lemma}
    Take any feasible configuration $p_w$ and its according Voronoi tessellation $\mathbb{W}_{p_w}$. Then, for any configuration $p$ with $p_{i} \in \bar{\mathbb{W}}^{\mathrm{int}}_{p_w,i}, \forall i \in \{1,\hdots,M\}$, it holds that $p_{i} \in \bar{\mathbb{W}}^{\mathrm{int}}_{p,i}$
\label{lemma:inclusioninnewinterior}
\end{lemma}
\vspace{-0.0em}
\emph{Proof:} %Lemma~\ref{lemma:inclusioninnewinterior} follows, given the fact that 
For all $p_i \in \bar{\mathbb{W}}^{\mathrm{int}}_{p_w,i}$ and $p_j \in \bar{\mathbb{W}}^{\mathrm{int}}_{p_w,j}$, with $i \neq j$, it holds by definition of ${\mathbb{W}}^{\mathrm{int}}_{p_w}$ that $\|p_i-p_j\| \leq 2(r_{\max} + \epsilon)$. Furthermore, following the definition of Voronoi partitions in~\eqref{eq:voronoidef}, the distance of each $p_i$ to its closest Voronoi border is bigger than or equal to half the distance to the surrounding seeds, which indeed is
$(r_{\max} + \epsilon)$.  \\

\noindent 
We also assume that, for an update of its Voronoi partition with respect to a sufficiently close position $p_i$, the steady state stays within the interior of $\bar{\mathbb{W}}_{p_i}$. 
\begin{assumption} 
There exists an $\tilde{\epsilon}>0$ s.t., for any $\bar{p}_i \in \mathbb{S}_{\mathbb{W}_i}^{\mathrm{p}}$ and 
%any
$p_i \in \bar{\mathbb{W}}_i^{int}$ with $\| \bar{p}_i - p_i \|<\tilde{\epsilon}$, it holds: 
$\bar{p}_i \in \bar{\mathbb{W}}_{p,i}^{\mathrm{int}} \text{ and } p_i \in \bar{\mathbb{W}}_{\bar{p},i}$.
\label{assumption:voronoisetneverempty}
\end{assumption}
\vspace{-0.6em}

\subsection{Data Collection and Bayesian Linear Regression}
\label{subsec:measurementcol}
In an unknown environment, the agents are assumed to be equipped with sensors and able to take a noisy measurement of the density $\phi$ at their current position. Note that measurements do not necessarily have to be taken at each time step, but their collection can be e.g. conditioned on reaching a predefined location or triggered after a fixed number of time steps $k$. Indexing by $h$ the number of collected data, and assuming that the noise is independent and identically distributed (i.i.d.) and zero mean Gaussian with variance $\sigma^2$, the measurements model for the $i-$th agent is
\begin{equation}
   m_{i,h} = \phi(p_{i,h}) + \nu_{i,h},\; \text{with } \nu_{i,h} \sim \mathcal{N} (0,\sigma^{2}). 
   \label{eq:measmod}
   \vspace{-0.3em}
\end{equation}
We assume that each agent's data is communicated to a server, so that the estimation of $\phi$ is centralized. Accordingly, we define the data-set collected by all $M$ agents after $t$ measurement steps as
\begin{equation}
\begin{split}
    I_{t} = \{ (p_{i,h},m_{i,h})\: \vert \: i=1,\cdots, M,\; h=1, \cdots, t\}.
\end{split}
\label{eq:setofmeasurements}
\vspace{-0.3em}
\end{equation}
In this set-up, we make use of the following assumption.
\begin{assumption}
The density function $\phi$ is represented by a linear combination of $\upsilon$ known, Lipschitz continuous features collected in a vector $\Phi: \mathbb{A} \rightarrow \mathbb{R}^{1 \times \upsilon}$, i.e., there exists an unknown parameter vector  $\theta \in \mathbb{R}^{\upsilon}$, s.t., $\phi(p) = \Phi(p)\theta$, $\forall p\in\mathbb{A}$. Furthermore, the features are linearly independent on the partition $\mathbb{W}_{p,i}$ in the sense that $\max_{p\in\mathbb{W}_{p,i}}\Phi(p)\Sigma\Phi(p)^\top\geq c_\Phi\|\Sigma\|$ for any positive semi-definite matrix $\Sigma\in\mathbb{R}^{\upsilon\times \upsilon}$ and some constant $c_\Phi>0$.
\label{assumption:bayesianconvergence}
\end{assumption}
Thus, the goal is to estimate the unknown vector $\theta$ from collected data. We solve the problem within the Bayesian linear regression framework~\cite{Sarkka2013}. To this aim, we endow the unknown vector with a Gaussian prior, such that \mbox{$\theta \sim \mathcal{N}(\mu_0,\Sigma_0)$}, and adopt a recursive scheme to update mean and covariance matrix in view of new data. Specifically, denoting by $\bar{\Phi}_{t+1} = [\Phi(p_{1,t+1})^{\top} \: \cdots \: \Phi(p_{M,t+1})^{\top}]^{\top} \in \mathbb{R}^{M \times \upsilon}$ the matrix of features measured at $t+1$, and by $\bar{m}_{t+1} = [m_{1,t+1} \: \cdots \: m_{M,t+1}]^{\top} \in \mathbb{R}^{M}$ the column vector of respective measurements, the updating rule reads as follows:
\begin{subequations}
\begin{align}
    \theta_{t+1} & \sim \mathcal{N}(\mu_{t+1}, \Sigma_{t+1})\\
    \mu_{t+1} & = \Sigma_{t+1}\Big(\Sigma_{t}^{-1}\mu_{t} + \frac{1}{\sigma^{2}}\bar{\Phi}^{\top}_{t+1}\bar{m}_{t+1}\Big)\\
    \Sigma_{t+1} & = \Big(\frac{1}{\sigma^{2}}\bar{\Phi}^{\top}_{t+1}\bar{\Phi}_{t+1} + \Sigma_{t}^{-1}\Big)^{-1}. \label{eq:bayessigmaupdate}
\end{align}
\label{eq:bayesupdate}
\vspace{-0.2em}
\end{subequations}
\noindent Accordingly, at any point $p \in \mathbb{A}$ we can define the estimated density $\hat{\phi}_{t+1}(p)$ and the variance $\text{Var}_{t+1}(p)$ as
\begin{subequations}
\begin{align}
    \hat{\phi}_{t+1}(p) & = \Phi(p)\mu_{t+1} \label{eq:thetaupdate}\\
    \text{Var}_{t+1}(p) & = \Phi(p)\Sigma_{t+1}\Phi(p)^{\top} \label{eq:variancephiblr}.
\end{align}
\label{eq:densityupdateblr}
\end{subequations}
\vspace{-1.5em}
\begin{remark}
Other coverage control approaches in an unknown environment rely on Gaussian Processes (GPs) for estimation of the unknown density function, i.e., in~\cite{Todescato2017} and~\cite{prajapat2022}, but they suffer from increasing complexity and memory requirements. The parametric set-up stated in Assumption~\ref{assumption:bayesianconvergence} thereby allows for a recursive update with fixed computational complexity. This is particularly relevant for the one-layer approach, where the GP would otherwise enter the MPC formulation. 
\end{remark}

\section{Two-layers Coverage MPC Algorithm}
\label{sec:twolayers}
The two-layers approach presented in this section builds upon the MPC scheme proposed in~\cite{Carron2017}. However, the set-up below does not leverage terminal ingredients, allowing for the consideration of more complex dynamical systems and the inclusion of collision avoidance constraints. Section~\ref{subsec:twolayers} studies the solution with known density, while in~\ref{subsec:twolayerslearning} the set-up is adjusted to include learning.

\vspace{-0.3em}
\subsection{Known Environment}
\label{subsec:twolayers}
In this subsection we consider the density $\phi$ as known. The overall here presented strategy encompasses two main tasks: (i) centralized computation of Voronoi partitions according to the agents' current position, and the respective centroids; (ii) solution of a tracking MPC scheme presented in Section~\ref{subsec:nonlineartrackingmpc} having the centroids as references. After solving task (ii), a new Voronoi partition can be computed, and the procedure can be repeated. \\
With $w \leq k$ indicating the time step of the last partition update, we now discuss the conditions that ensure convergence of this iterative set-up to the centroidal Voronoi partition solving problem~\eqref{eq:cortescost}. For the first task, we require a suitable update rule on the partitions. This is built upon~\cite[Proposition 3.3]{Cortes2004}, where sufficient conditions on position updates are given: 
\begin{proposition}\cite[Proposition 3.3]{Cortes2004} Assume that $p_{0} \in \mathbb{A}^{M}$ and the Voronoi partition is updated in finite time.
Further consider a continuous mapping $T: \mathbb{A}^{M} \rightarrow \mathbb{A}^{M}$, with $p_{w_{+}} = T(p_w)$, where $w_{+}$ indicates the next update step. If the mapping $T(p_w)$ fulfills the following properties at each update step $w$:
%time step $k$: 
\begin{itemize}
    \item $ \Vert p_{i,w_{+}} - c_i(\mathbb{W}_{p_{w},i}, \phi) \Vert \leq \Vert p_{i,w} - c_i(\mathbb{W}_{p_{w},i}, \phi) \Vert,$ \newline  $\forall i \in \{1, ..., M\}$;
    \item As long as the positions do not describe a centroidal Voronoi partition,
    $\exists j \in \{1, ..., M\}$ such that$ \newline \Vert p_{j,w_{+}} - c_j(\mathbb{W}_{p_{w},j}, \phi) \Vert < \Vert p_{j,w} - c_j(\mathbb{W}_{p_{w},j}, \phi) \Vert$,
\end{itemize}
then, $p_{w}$ converges to a centroidal Voronoi partition.
\label{proposition:cortesconvergence}
\end{proposition}

Hence, given the ability of integrator dynamics to reduce their Euclidean distance to a given reference at each time-step $k$, their Voronoi partition can be updated accordingly. However, as arbitrary nonlinear dynamics do not generally decrease the distance of agents to their corresponding centroid for all instances in time (see, e.g., the case presented in Figure~\ref{pics:singleintvsarbdyn}),  
\begin{figure}[t]
\begin{minipage}[t]{0.20\textwidth}
\includegraphics[width = \textwidth]{figures/single_int_mov_own_picto.pdf}
\end{minipage}
\hfill
\begin{minipage}[t]{0.20\textwidth}
\includegraphics[width = \textwidth]{figures/complex_dyn_mov_own_picto.pdf}
\end{minipage}
\caption{Left: agent able to decrease Euclidean distance to reference at each time step, indicated by a drone. Right: agent not able to decrease Euclidean distance to reference at each time step, represented by a car.}
\label{pics:singleintvsarbdyn}
\vspace{-0.8em}
\end{figure} a partition update rule has to be applied in order to meet the requirements of Proposition~\ref{proposition:cortesconvergence} and allow for the construction of a suitable mapping $T$. For this purpose, we rely on the conditions proposed in~\cite[Section 4]{Carron2017}: i.e., defining $e_{i,w} \! = \! \Vert p_{i,w} \! - \! c_i(\mathbb{W}_{p_{w},i}, \phi) \Vert$, the partition is updated only if
\begin{subequations}
\begin{align}
    & \bullet \; \Vert p_{i,k} - c_i(\mathbb{W}_{p_{w},i}, \phi) \Vert \leq e_{i,w} \;\: \forall i \in \{1,..., M\}, \label{eq:updatereq1} \\
    & \bullet \; \exists j \! \in \! \{1,..., M\} \text{ s.t. } \Vert p_{j,k} \! - \! c_j(\mathbb{W}_{p_{w},j}, \phi) \Vert \! < \! e_{j,w}. \label{eq:updatereq2} 
\end{align}
\label{eq:updatereq}%
\end{subequations}
Furthermore, the vector combining all errors is indicated with $e_{w}=[e_{1,w}, \hdots, e_{M,w}]$. Before discussing convergence within the second task, i.e., of the tracking MPC having the above computed centroids as reference, we detail its formulation. The optimization problem at time $k$ is defined as in~\eqref{eq:nonlineartrackingmpcwithcolavoidance}, with $r = c(\mathbb{W}_{p_{w}}, \phi)$ and  $p_{w}$ set to the most recent position configuration that fulfilled~\eqref{eq:updatereq} and~\eqref{eq:feasibilitycond}. The cost function $J_{i}(x_{i}, \bar{x}_{i}, u_{i}, \bar{u}_{i}, r_{i})$ follows the definition in~\eqref{eq:generalnonlineartrackingmpccost}. Within this set-up, convergence can be proven leveraging the subsequent assumptions. 
\begin{assumption}  
For any feasible configuration $p \in \mathbb{A}^{M}$ and according reference vector $r \in \mathbb{A}^{M}$ define the set \begin{equation}
\begin{split}
    \mathbb{T}_{\mathbb{W}_{p,i},r} = \{s\in \mathbb{S}_{\mathbb{W}_{p,i}} \: \vert \: \ell_{T,i}(C_{i}\bar{x} - r_{i}) = l_{\mathbb{W}_{p,i},r,\min}\},
\end{split}
\label{eq:Tid2layer}
\end{equation}
which is the union of all setpoints resulting in a target cost value equal to $l_{\mathbb{W}_{p,i},r,\min} = \min_{s\in \mathbb{S}_{\mathbb{W}_{p,i}}} \ell_{T,i}(C_i\bar{x}_i-r_i)$. Then, for every agent $i \in \{1,\hdots,M\}$ there are constants  \mbox{$\beta_{1,i}, \beta_{2,i} > 0$} such that, for any $\bar{s} := (\bar{x},\bar{u}) \in \mathbb{S}_{\mathbb{W}_{p,i}}$ and any $\epsilon \in [0,1]$, there exists a setpoint $\hat{s} := (\hat{x},\hat{u}) \in \mathbb{S}_{\mathbb{W}_{p,i}}$ satisfying 
\begin{subequations}
\begin{align}
    d_{i}(\hat{s}-\bar{s}) &\leq \beta_{1,i}\epsilon d_{i}(\bar{s})_{\mathbb{T}_{\mathbb{W}_{p,i},r}} \label{eq:assumption31twolayer} \\
    \ell_{T,i}(C_{i}\hat{x} \! - \! r_{i}) \! - \! \ell_{T,i}(C_{i}\bar{x}-r_{i}) &\leq \! - \! \beta_{2,i}\epsilon d_{i}(\bar{s})_{\mathbb{T}_{\mathbb{W}_{p,i},r}}^{2}, \label{eq:assumption32twolayer}%
\end{align}
\end{subequations} 
where $d_i(\cdot)$ is the distance function defined in~\eqref{eq:distance} and $d_{i}(\cdot)_{\mathbb{T}_{\mathbb{W}_{p,i},r}}$ follows definition~\eqref{eq:minsetdistance}. 
\label{assumption:steadystatedecreasetwolayers}%
\end{assumption}
\begin{assumption}
Consider any feasible configuration $p \in \mathbb{A}^{M}$ and according reference vector $r \in \mathbb{A}^{M}$. Then, for  every agent $i \in \{1,\hdots,M\}$ there is a constant  \mbox{$\beta_{T,i} > 0$} such that, for any $s := (\bar{x},\bar{u}) \in \mathbb{S}_{\mathbb{W}_{p,i}}$ it holds: 
\begin{align}
    \ell_{T,i}(C\bar{x}-r_i) - l_{\mathbb{W}_{p,i},r,\min} \leq \beta_{T,i}\epsilon d_{i}(s)_{\mathbb{T}_{\mathbb{W}_{p,i},r}}
    \label{eq:upperboundtargetcost}
\end{align}
\label{assumption:upperboundtargetcost}%
\end{assumption}
\vspace{-0.5em}
\noindent Assumption~\ref{assumption:steadystatedecreasetwolayers} states that, for any setpoint $\bar{s}$ not being a minimizer of the target cost, i.e., resulting in a target cost value not equal to its attainable minimal value $l_{\mathbb{W}_{p,i},r,\min}$, there exists another setpoint in its neighborhood such that the target cost can be actually reduced. In Assumption~\ref{assumption:upperboundtargetcost} the target cost of an arbitrary setpoint is upper bounded by a function of its distance to the target cost's minimizing setpoint. Assumption~\ref{assumption:steadystatedecreasetwolayers} and~\ref{assumption:upperboundtargetcost} are generally satisfied by choosing a target cost of the same powers as the stage cost, and by having $\mathbb{S}_{\mathbb{W}_{p,i}}$ and $\mathbb{A}$ as polytopes. The latter condition is not restrictive for most dynamical systems that are of interest in this work. \\
The overall procedure consists in constructing the Voronoi partitions with respect to the agents' position and setting the references equal to their centroids. Then, the MPC defined in~\eqref{eq:nonlineartrackingmpcwithcolavoidance} is applied recursively until conditions~\eqref{eq:updatereq1},~\eqref{eq:updatereq2} and~\eqref{eq:feasibilitycond} are fulfilled and the partitions as well as the centroids are updated. It is summarized in Algorithm~\ref{alg:twolayermpcalg} and we state our main result in Theorem~\ref{theorem:twolayers}, whose proof can be found in Appendix~\ref{subsec:twolayersproof}.

\begin{algorithm}[h!]
\SetAlgoLined
Set $k = 0$, $w = 0$ and construct $\mathbb{W}_{p_{w}}$. \par
Set $r = c(\mathbb{W}_{p_{w}},\phi)$ and calculate $e_{w}$.\par
 \For{k=0,1,\dots}{
 \ForAll{$i \in \{1, \hdots,M\}$}{
 Solve~\eqref{eq:nonlineartrackingmpcwithcolavoidance} and obtain $u^{*}_{i,0\vert k}, \hat{u}_{i,\cdot \vert k}$, $\hat{x}_{i,\cdot \vert k}$. \par
 Apply $u^{*}_{i,0\vert k}$ to obtain  $x_{i,k+1}$. \par}
 Construct $\mathbb{W}^{'}_{p_{k+1}}$. \par
\If{\textup{conditions~\eqref{eq:updatereq1},~\eqref{eq:updatereq2} and~\eqref{eq:feasibilitycond} are fulfilled}}{
   Set $w = k+1$ and update $\mathbb{W}_{p_{w}}$ = $\mathbb{W}^{'}_{p_{k+1}}$.\par
   Update $r = c(\mathbb{W}_{p_{w}},\phi)$ and calculate $e_{w}$.}
   }
 \caption{Two-Layers Coverage MPC}
 \label{alg:twolayermpcalg}
\end{algorithm}

\begin{theorem} 
Let Assumptions~\ref{assumption:dynamics},~\ref{assumption:expocostcontrollability},~\ref{assumption:boundedbyd},~\ref{assumption:voronoisetneverempty},~\ref{assumption:steadystatedecreasetwolayers} and~\ref{assumption:upperboundtargetcost} hold and consider a horizon length $N \geq N^{*}$. Additionally, suppose that at time step $k = 0$ the MPC problem described in~\eqref{eq:nonlineartrackingmpcwithcolavoidance} is feasible for all agents. Then, the overall control problem according to Algorithm~\ref{alg:twolayermpcalg} is recursively feasible, the agents do not collide and satisfy $x_{i,k} \in \mathbb{X}_{i},\; u_{i,k} \in \mathbb{U}_{i}, \forall i \in \{1,\hdots,M\}$ and $\forall k \in \mathbb{N}$. Furthermore, the partition update condition given by~\eqref{eq:updatereq1},~\eqref{eq:updatereq2} and~\eqref{eq:feasibilitycond} is fulfilled after a finite amount of time, and the agents' position configuration converges to a centroidal Voronoi partition.
\label{theorem:twolayers}
\end{theorem}

\vspace{-0.7em}
\subsection{Unknown Environment}
\label{subsec:twolayerslearning}
To deal with an initially unknown $\phi$, the motion planning scheme in the tracking MPC considered above needs to be adjusted to balance density learning and steering towards the current partition's centroids. In particular, the reference of each agent during exploration is chosen as the point of maximal variance within the Voronoi region\footnote{Finding the point of maximal variance over the whole Voronoi region implies evaluating~\eqref{eq:densityupdateblr} on an infinite number of points. In practice, we perform a sufficiently dense gridding of each $\mathbb{W}_{p,i}$ and evaluate variances on those points. For ease of notation, we still refer to such set of points as $\mathbb{W}_{p,i}$.} $\bar{\mathbb{W}}_{p,i}^{\mathrm{int}}$, denoted by $v_i(\mathbb{W}_{p,i}, \text{Var}) = \text{arg}\!\max_{\tilde{p} \in \bar{\mathbb{W}}_{p,i}^{\mathrm{int}}} \text{Var}(\tilde{p})$ and belonging to the vector $v(\mathbb{W}_{p}, \text{Var}) = [v_1(\mathbb{W}_{p,1}, \text{Var}), \hdots, v_M(\mathbb{W}_{p,M}, \text{Var})]$. In case of exploitation, the reference is set to the current partition's centroid with respect to the available density estimate $\hat{\phi}_t$, i.e., $r_{i,k} = c_i(\mathbb{W}_{p_{w},i}, \hat{\phi}_t)$. As proposed in~\cite{Todescato2017}, denoting with $\text{Var}_{\max} = \max_{p \in \mathbb{A}} \text{Var}(p)$ the maximal variance value computed over the set $\mathbb{A}$, and with $F: [0,1] \rightarrow [0,1]$ an arbitrary strictly monotonically increasing function such that \mbox{$F(\epsilon)=0 \Leftrightarrow \epsilon=0$}, the decision between exploration and exploitation is performed according to a Bernoulli random variable $\mathcal{B}(F(\text{Var}_{\max})) \in \{0,1\}$. The rationale is the following: at each time step, a sample from $\mathcal{B}$ is drawn whose probability of success, i.e. of returning a value equal to 1 is $F(\text{Var}_{\max})$. In this case, exploration is selected and both the Voronoi tessellation and the references are kept fixed for all agents. The exploration modality continues regardless of the new samples of $\mathcal{B}(F(\text{Var}_{\max}))$ and the data-set is expanded as $I_{t+1} = I_{t} \cup [(m_{1,k},p_{1,k}), \hdots, (m_{M,k},p_{M,k})]$ at each time-step until the agents' positions are sufficiently close to the point of maximum variance of their Voronoi region, i.e., when $e_{v,i,k} = \Vert v_i(\mathbb{W}_{p,i}, \text{Var}_t) - p_{i,k} \Vert \leq \rho$ for an arbitrarily small $\rho >0$. At this point, exploration modality is left. As for the exploitation phase, Voronoi partitions are updated only if conditions~\eqref{eq:updatereq1},~\eqref{eq:updatereq2} and~\eqref{eq:feasibilitycond} are met. The overall procedure is summarized in Algorithm~\ref{alg:twolayermpcalglearningimp}. The main result is given in the following theorem that is proven in Appendix~\ref{subsec:twolayerslearningproof}.
\begin{algorithm}[h!]
\SetAlgoLined
 Set $k = 0$, $w = 0$ and construct $\mathbb{W}_{p_{w}}$. \par
 Set $t = 1$, $\rho$, $\mu_{0}, \Sigma_{0}$ and $I_{t} = [(m_{1,k},p_{1,k}),..., (m_{M,k},p_{M,k})]$.\par
 Compute $\hat{\phi}_{t}(p), \text{Var}_{t}(p) \ \ \forall p \in \mathbb{A}$. \par
 Compute $c(\mathbb{W}_{p_{w}}, \hat{\phi}_{t}), v(\mathbb{W}_{p_{w}}, \text{Var}_{t}), \text{Var}_{\max,t}$ and $e_{w}$.\par
 Set exploration flag = false. \par
  \For{k=0,1, \dots}{
  \eIf{\textup{exploration flag} $\Vert$ \textup{(}$\mathcal{B}(F(\textup{Var}_{\textup{max},t})) == 1$\textup{)}}{
   Set exploration flag = true. \par
   \ForAll{$i \in \{1, \hdots, M \}$}{
   Set $r_{i,k} = v_i(\mathbb{W}_{p_{w},i}, \text{Var}_{t})$.\par
   Solve~\eqref{eq:nonlineartrackingmpcwithcolavoidance} and obatin $u^{*}_{i,0\vert k}.$ \par
   Apply $u^{*}_{i,0\vert k}$ to obtain $x_{i,k+1}$. \par 
   $e_{v,i,k+1} = \Vert v_i(\mathbb{W}_{p_{w},i}, \text{Var}_{t}) - p_{i,k+1} \Vert$. \par}
   $I_{t+1} \! = \! {\footnotesize I_{t} \! \cup \![(m_{1,k+1},p_{1,k+1}),..., (m_{M,k+1},p_{M,k+1})]}$ \par
   Update~\eqref{eq:thetaupdate}-\eqref{eq:variancephiblr} $\forall p \in \mathbb{A}$. \par
   Update $c(\mathbb{W}_{p_{w}}, \hat{\phi}_{t+1})$,$v(\mathbb{W}_{p_{w}}, \! \text{Var}_{t+1}),\text{Var}_{\max,t+1}$.\par
   $t=t+1$ \par
   \If{$\|e_{v,i,k+1}\|\leq\rho$, 
   $\forall i \in \{1, \hdots, M \}$}
   {
   Set exploration flag = false. \par}
  }
  {\ForAll{$i \in \{1, \hdots, M \}$}{
  Set $r_{i,k} = c(\mathbb{W}_{p_{w},i}, \hat{\phi}_{t})$. \par
  Solve~\eqref{eq:nonlineartrackingmpcwithcolavoidance} and obtain $u^{*}_{i,0\vert k}, \hat{u}_{i,\cdot \vert k}$, $\hat{x}_{i,\cdot \vert k}$. \par
  Apply $u^{*}_{i,0\vert k}$ to obtain $x_{i,k+1}$. \par}
 Construct $\mathbb{W}^{'}_{p_{k+1}}$. \par
\If{\textup{conditions~\eqref{eq:updatereq1},~\eqref{eq:updatereq2} and~\eqref{eq:feasibilitycond} are fulfilled}}{
   $w = k+1$, $\mathbb{W}_{p_{w}}$ = $\mathbb{W}^{'}_{p_{k+1}} \xrightarrow{update}$$c(\mathbb{W}_{p_{w}},\hat{\phi}_t)$, $e_{w}$.}}
   }
  \caption{Two-Layers, Learning-Based Coverage MPC}
 \label{alg:twolayermpcalglearningimp}
\end{algorithm}

\begin{theorem} 
Let Assumptions~\ref{assumption:dynamics},~\ref{assumption:expocostcontrollability},~\ref{assumption:boundedbyd},~\ref{assumption:voronoisetneverempty},~\ref{assumption:bayesianconvergence},~\ref{assumption:steadystatedecreasetwolayers} and~\ref{assumption:upperboundtargetcost} hold and consider a horizon length $N \geq N^{*}$ and $\rho>0$ sufficiently small. Additionally, suppose that at time step $k = 0$ the MPC problem described in~\eqref{eq:nonlineartrackingmpcwithcolavoidance} is feasible for all agents. Then, the overall control problem according to Algorithm~\ref{alg:twolayermpcalglearningimp} is recursively feasible and the agents do not collide and satisfy $x_{i,k} \in \mathbb{X}_{i}, u_{i,k} \in \mathbb{U}_{i}, \forall i \in \{1,\hdots,M\}$ and $\forall k \in \mathbb{N}$. Furthermore, the partition update condition given by~\eqref{eq:updatereq1},~\eqref{eq:updatereq2} and~\eqref{eq:feasibilitycond} is fulfilled after a finite amount of time, and the agents' position configuration converges to a centroidal Voronoi partition.
\label{theorem:twolayerslearning}
\end{theorem}
\section{One-Layer Coverage MPC Algorithm}
\label{sec:onelayer}
In the one-layer approach, the reference is not pre-calculated and passed to the MPC, but the locational optimization function is jointly optimized with the MPC cost, cf. Figure~\ref{fig:coverageclarification}. Furthermore, the Voronoi partitions are constructed with respect to the optimal setpoint positions $\bar{p}_{i,k}^* = C_{i}s_{i,k}^{*}$. The combined optimization is then expected to reduce the time and energy required for exploration.
\vspace{-0.5em}
\subsection{Known Environment}
\label{subsec:onelayer}
For the case of a known density $\phi$, the MPC optimization cost of the one-layer framework will encompass the same tracking cost $V_{N,i}$ considered in the previous sections, but the target cost $\ell_{T,i}$ will be equal to the corresponding summand of the locational optimization cost defined in~\eqref{eq:cortescost}. In particular, the latter is to be optimized with respect to the steady-state position $\bar{p}_{i,k} = C_i\bar{x}_{i,k} \in \mathbb{S}^{\mathrm{p}}_{\mathbb{W}_{\bar{p}^{*}_{w},i}}$ (see~\eqref{eq:steadystatesetwithcollavoidance}), and the integral is evaluated over the current Voronoi partition $\mathbb{W}_{\bar{p}^{*}_{w},i}$.
Thus, the resulting continuous objective reads as follows:
\begin{align}
&J_{i}(x_{i,\cdot \vert k}, u_{i,\cdot \vert k}, s_{i,k},\mathbb{W}_{\bar{p}^{*}_{w},i}) = \label{eq:fullnonlineartrackingmpccostonelayer} \\ & V_{N,i}(x_{i,\cdot \vert k}, u_{i,\cdot \vert k}, s_{i,k}) +   \ell_{T,i}(\bar{p}_{i,k},\mathbb{W}_{\bar{p}^{*}_{w},i}) = \notag\\ 
& \sum_{l=0}^{N-1}\ell_{i}(x_{i,l\vert k}, u_{i,l \vert k}, s_{i,k}) + \lambda \int_{\mathbb{W}_{\bar{p}^{*}_{w},i}} g(\Vert q - \bar{p}_{i,k} \Vert)\phi(q)dq,\notag
\end{align}
with $\lambda \in \mathbb{R}_{+}$ representing a scaling factor. The overall MPC program is then stated as
\begin{equation}
\begin{split}
\min_{x_{i,\cdot \vert k}, u_{i,\cdot \vert k},s_{i,k}}&J_{i}(x_{i,\cdot \vert k}, u_{i,\cdot \vert k}, s_{i,k},\mathbb{W}_{\bar{p}^{*}_{w},i}, \phi) \\
&\eqref{eq:generalinit}-\eqref{eq:1toN-1withcolavoidance} ,
\end{split}
\label{eq:fullnonlineartrackingmpconelayer}
\end{equation}
and is solved within the iterative coverage scheme presented in Algorithm~\ref{alg:onelayeralg}.To show convergence to a centroidal Voronoi configuration while dealing with non-convex target costs, Assumptions~\ref{assumption:ballwithminconvex} and~\ref{assumption:steadystatedecreaseonelayer} are imposed.

\begin{algorithm}[h!]
\caption{One-Layer Coverage MPC}
\SetAlgoLined
Set $k = 0$, $w=0$ and 
construct $\mathbb{W}_{\bar{p}_{w}^{*}}$ with $\bar{p}_0^*=p_0$\par
\For{k=0,1, \dots}{\ForAll{$i \in \{1, \hdots, M \}$}{
 Solve~\eqref{eq:fullnonlineartrackingmpconelayer} and obtain  $\hat{x}_{i,\cdot \vert k}$, $\hat{u}_{i,\cdot \vert k}$,  $\bar{x}^{*}_{i,k}$,  $u^{*}_{i,0\vert k}$. \par
 Apply $u^{*}_{i,0\vert k}$ to obtain $x_{i,k+1}$. \par}
 %$x_{i,k+1}$ = $f_{i}(x_{i,k},u^{*}_{i,0\vert k})$. \par}
 Construct $\mathbb{W}^{'}_{\bar{p}_{k}^{*}}$ according to $\bar{p}_{k}^{*}$. \par
  \If{\textup{condition~\eqref{eq:feasibilitycond} is fulfilled}}{
  $w = k$, $ \mathbb{W}_{\bar{p}_{w}^{*}} = \mathbb{W}_{\bar{p}_{k}^{*}}^{'}$.}
 }
\label{alg:onelayeralg}
\end{algorithm}

\begin{assumption} For every agent $i \in \{1, ..., M\}$, given any $\bar{p} \in \mathbb{S}^{p}_{i}$ and any $\bar{p}' \in \mathbb{S}^{\mathrm{p}}_{\mathbb{W}_{\bar{p},i}}$ there exist an $r > 0$ such that the set $\mathrm{argmin}_{p \in S^{\mathrm{p}}_{\bar{p}',\mathbb{W}_{\bar{p},i}}} \ell_{T,i}(p,\mathbb{W}_{\bar{p},i}), \; \text{with } S^{\mathrm{p}}_{\bar{p}',\mathbb{W}_{\bar{p},i}} := \mathbb{B}_{r}^{D}(\bar{p}') \cap \mathbb{S}^{\mathrm{p}}_{\mathbb{W}_{\bar{p},i}},$ is convex.
\label{assumption:ballwithminconvex}
\end{assumption}
\begin{assumption} 
Let $l_{\bar{p}',\mathbb{W}_{\bar{p},i},\min} \! = \! \min_{p \in S^{\mathrm{p}}_{\bar{p}',\mathbb{W}_{\bar{p},i}}}\ell_{T,i}(p,\mathbb{W}_{\bar{p},i})$, and define the corresponding set of local minimizers as
\begin{equation}
\begin{split}
    \mathbb{T}_{\bar{p}',\mathbb{W}_{\bar{p},i}} \! = \! \{p \in S^{\mathrm{p}}_{\bar{p}',\mathbb{W}_{\bar{p},i}}\,\vert\, \ell_{T,i}(p,\mathbb{W}_{\bar{p},i}) \! = \! l_{\bar{p}',\mathbb{W}_{\bar{p},i},\min}\}.
\end{split}
\label{eq:Tid}
\end{equation}
Then, for every agent $i \in \{1,\hdots,M\}$, any $\bar{p} \in \mathbb{S}^{p}_{i}$ and any $\bar{s}' := (\bar{x}',\bar{u}') \in \mathbb{S}_{\mathbb{W}_{\bar{p},i}}$ there are constants $\beta_{1,i}, \beta_{2,i} > 0$ as well as a $\mathcal{K}$-function $\kappa$ such that, for any $\epsilon \in [0,1]$, there exists a setpoint $\hat{s} := (\hat{x},\hat{u}) \in S_{\bar{p}',\mathbb{W}_{\bar{p},i}}$ satisfying
\begin{subequations}
\begin{align}
    d_{i}(\hat{s}-\bar{s}') &\leq \beta_{1,i}\epsilon \kappa(\Vert \bar{p}' \Vert)_{\mathbb{T}_{\bar{p}',\mathbb{W}_{\bar{p},i}}}
    \label{eq:assumption31},\\
    \ell_{T,i}(\hat{p},\mathbb{W}_{\bar{p},i}) \! - \! \ell_{T,i}(\bar{p}',\mathbb{W}_{\bar{p},i}) &\leq \! - \! \beta_{2,i}\epsilon \kappa(\Vert \bar{p}' \Vert)_{\mathbb{T}_{\bar{p}',\mathbb{W}_{\bar{p},i}}}^{2} \label{eq:assumption32},
\end{align}
\end{subequations}
with $\hat{p} = C_i\hat{x}$ and $\bar{p}' = C_i\bar{x}'$
\label{assumption:steadystatedecreaseonelayer}
\end{assumption}
\begin{remark} 
Assumption~\ref{assumption:ballwithminconvex} builds upon the fact that the constructed Voronoi partitions are never empty (see Section~\ref{subsec:collisionavoidance}) and provides a convenient structure on the set of solutions in view of Assumption~\ref{assumption:steadystatedecreaseonelayer}. Note that, in the considered case of the function $g(\cdot)$ in~\eqref{eq:cortescost} being equal to the square function, $\ell_{T,i}(\bar{p},\mathbb{W}_{\bar{p},i})$ is convex in $p$~\cite[Lemma 6.1]{Bullo2012}, and the set of minimizers is trivially convex: accordingly, Assumption~\ref{assumption:ballwithminconvex} holds with $r = \infty \ \text{such that} \ S^{\mathrm{p}}_{\bar{p}',\mathbb{W}_{\bar{p},i}} =\mathbb{S}^{\mathrm{p}}_{\mathbb{W}_{\bar{p},i}}$.  Finally, Assumption~\ref{assumption:steadystatedecreaseonelayer} is a generalization of Assumption~\ref{assumption:steadystatedecreasetwolayers} to the case of non-convex target costs. In accordance to~\cite[Remark 1]{Soloperto2021}, this condition holds taking functions $d$ and $\kappa$ quadratic if $\ell_{T,i}$ is strongly convex quadratic and sets are polytopic.
\end{remark}
\noindent Next, we state the theorem collecting all the theoretical guarantees of the proposed scheme. The proof is presented in Appendix~\ref{subsubsec:onelayerknownenvironmenttheory}.
\begin{theorem} Let Assumptions~\ref{assumption:dynamics},~\ref{assumption:expocostcontrollability},~\ref{assumption:boundedbyd},~\ref{assumption:voronoisetneverempty},~\ref{assumption:ballwithminconvex} and~\ref{assumption:steadystatedecreaseonelayer} hold, and consider a horizon length $N \geq N^{*}$. Additionally, suppose that at time step $k = 0$ the MPC problem described in~\eqref{eq:fullnonlineartrackingmpconelayer} is feasible for all agents. Then, in accordance to Algorithm~\ref{alg:onelayeralg} it remains recursively feasible, the agents do not collide and satisfy $x_{i,k} \in \mathbb{X}_{i},\; u_{i,k} \in \mathbb{U}_{i}, \forall i \in \{1,\hdots,M\}$ and $\forall k \in \mathbb{N}$. Furthermore, the partition update condition given by~\eqref{eq:feasibilitycond} is fulfilled after a finite amount of time, and the agents' position configuration converges to a centroidal Voronoi partition.
\label{theorem:onelayer}
\end{theorem}

\vspace{-0.8em}
\subsection{Unknown Environment}
\label{subsec:onelayerlearning}
The adaptation of the one-layer approach to an unknown environment involves the following modification of the target cost $\ell_{T,i}$ entering~\eqref{eq:fullnonlineartrackingmpconelayer} inspired by the Upper Confidence Bound method in Bayesian Optimization~\cite{Auer2002,Srinivas2010}. Specifically, the uncertainty on a potential setpoint $\bar{p}_i$, quantified by the variance $\text{Var}(\bar{p}_i)$ and scaled by a parameter $S > 0$, is subtracted from~\eqref{eq:fullnonlineartrackingmpconelayer}. Hence, the target cost now reads as
\begin{equation}
\begin{split}
&\ell_{T,i}(\bar{p}_{i}, \mathbb{W}_{\bar{p}^{*}_{w},i}, \hat{\phi}, \text{Var}) = \\ & \lambda \left(\int_{\mathbb{W}_{\bar{p}^{*}_{w},i}} g(\Vert q - \bar{p}_{i} \Vert)\hat{\phi}(q)dq - S\cdot \text{Var}(\bar{p}_{i}) \right).
\end{split}
\label{eq:fullnonlineartrackingmpccostonelayerlearning}
\end{equation}
Apart from this modification of the cost and the data collection, the overall procedure follows the one proposed in Section~\ref{subsec:onelayer} and is summarized in Algorithm~\ref{alg:onelayerlearningalg}. Note that~\eqref{eq:fullnonlineartrackingmpccostonelayerlearning} is non-convex, so Assumptions~\ref{assumption:ballwithminconvex} and~\ref{assumption:steadystatedecreaseonelayer} become crucial. However, they are considered to hold with respect to the modified target cost $\ell_{T,i}$ defined in~\eqref{eq:fullnonlineartrackingmpccostonelayerlearning}, given its continuity with respect to $\bar{p}_i$, which is proven below. \\

\noindent \textit{Proof of Continuity.} Considering a fixed Voronoi partition $\mathbb{W}_{\bar{p},i}$. By continuity of $g$, it holds that $\lim_{p_{1} \rightarrow p_{2}}g(p_{1}) - g(p_{2}) = 0$ for all $p_{1},\,p_{2} \in \mathbb{A}.$ Accordingly,
\begin{align*}
    &\lim_{p_{1} \rightarrow p_{2}} H_{i}(p_{1}, \mathbb{W}_{\bar{p},i}) - H_{i}(p_{2}, \mathbb{W}_{\bar{p},i}) \\
    &=\lim_{p_{1} \rightarrow p_{2}}\int_{\mathbb{W}_{\bar{p},i}}\Big(g(\Vert q - {p}_{1} \Vert) - g(\Vert q - {p}_{2} \Vert)\Big)\phi(q)dq\\
    &= \int_{\mathbb{W}_{\bar{p},i}} \lim_{p_{1} \rightarrow p_{2}} (g(\Vert q - {p}_{1} \Vert) -  g(\Vert q - {p}_{2} \Vert))\phi(q)dq = 0,
\label{eq:cortescostcontinuity}
\end{align*}
where the last step follows from the fact that the limit and the integration can be interchanged if the limit of the function exists and is integrable. The continuity of the variance follows by its definition.
\hfill{$\blacksquare$} \\
\label{proposition:coveragecostcontinuity}

\noindent Also, Assumption~\ref{assumption:ballwithminconvex} seems natural in such a scenario, ensuring the existence of a direction in which $\ell_{T,i}(\bar{p}',\mathbb{W}_{\bar{p},i})$ is decreasing and fulfilling equation~\eqref{eq:assumption32} as long as $\bar{p}' \notin \mathbb{T}_{\bar{p}',\mathbb{W}_{\bar{p},i}}$.

\begin{algorithm}[h!]
\caption{One-Layer, Learning-Based Coverage MPC}
\SetAlgoLined
Set $k = 0$, $w = 0$ and construct $\mathbb{W}_{\bar{p}^{*}_{w}}$ with $\bar{p}_0^*=p_0$\par
Set $t = 1$, $\mu_{0}$, $\Sigma_{0}$ and $I_{t} = [(m_{1,k},p_{1,k}), \hdots, (m_{M,k},p_{M,k})]$. \par 
Update $\hat{\phi}_{t}(p), \text{Var}_{t}(p) \ \ \forall p \in \mathbb{A}$. \par
 \For{k = 0,1,\dots}{\ForAll{$i \in \{1, \hdots, M \}$}{
  Set $\ell_{T,i}$ = $\ell_{T,i}(\bar{p}, \mathbb{W}_{\bar{p}^{*}_{w},i}, \hat{\phi}_{t}, \text{Var}_{t})$ \par
  Solve \eqref{eq:fullnonlineartrackingmpconelayer} and obtain  $\bar{x}^{*}_{i,k}, u^{*}_{i,0\vert k}, \hat{u}_{i,\cdot \vert k}$, $\hat{x}_{i,\cdot \vert k}$. \par
  Apply $u^{*}_{i,0\vert k}$ to obtain $x_{i,k+1}$. \par}
 $I_{t+1} = I_{t} \cup [(m_{1,k},p_{1,k}), \hdots, (m_{M,k},p_{M,k})]$ \par
 Update~\eqref{eq:thetaupdate}-\eqref{eq:variancephiblr} $\forall p \in \mathbb{A}$. \par
 Construct $\mathbb{W}^{'}_{\bar{p}_{k}^{*}}$ according to $\bar{p}_{k}^{*}$. \par
  \If{\textup{condition~\eqref{eq:feasibilitycond} is fulfilled}}{
  $w = k$, $ \mathbb{W}_{\bar{p}_{w}^{*}} = \mathbb{W}_{\bar{p}_{k}^{*}}^{'}$.} $t=t+1$
 }
\label{alg:onelayerlearningalg}
\end{algorithm}

\noindent Define the set of feasible positions in the MPC problem of agent $i$ at time $k$ by $\mathbb{F}_{i,k}=\{p\in\mathbb{A}|~\exists (x_{\cdot|k},u_{\cdot|k},s_k)$ s.t.~\eqref{eq:fullnonlineartrackingmpconelayer} is feasible with $[C,0]s_k=p\}$ and accordingly $\mathbb{F}_{k}=\bigcup_{i=1}^{M}\mathbb{F}_{i,k}$. 
The theoretical results are summarized in the following theorem, which is proven in Appendix~\ref{subsec:onelayerunknownenvironmenttheory}. 
\begin{theorem} Let  Assumptions~\ref{assumption:dynamics},~\ref{assumption:expocostcontrollability},~\ref{assumption:boundedbyd},~\ref{assumption:voronoisetneverempty},~\ref{assumption:bayesianconvergence},~\ref{assumption:ballwithminconvex} and~\ref{assumption:steadystatedecreaseonelayer} hold regarding the target cost defined in~\eqref{eq:fullnonlineartrackingmpccostonelayerlearning} and consider a horizon length $N \geq N^{*}$. Additionally, suppose that at time step $k = 0$, the MPC problem described in~\eqref{eq:fullnonlineartrackingmpconelayer} is feasible for all agents, Then, the overall control problem according to Algorithm~\ref{alg:onelayerlearningalg} is recursively feasible and the agents do not collide and satisfy $x_{i,k} \in \mathbb{X}_{i},\; u_{i,k} \in \mathbb{U}_{i}, \forall i \in \{1,\hdots,M\}$ and $\forall k \in \mathbb{N}$. The partition update condition given by~\eqref{eq:feasibilitycond} is fulfilled after a finite amount of time, the density estimate converges in probability and the agents' position configuration converges to a centroidal Voronoi partition with respect to the converged density estimate $\hat{\phi}_{\infty}$. Furthermore, there exists a uniform constant $\Delta H\geq 0$, such that
$\sup_{p\in\mathbb{F}_k}\lim_{k\rightarrow\infty} \mathrm{Var}_k(p)\leq \frac{ \Delta H}{S}$, in probability. 
\label{theorem:onelayerlearning}
\end{theorem}
\vspace{-0.1em}
\noindent In accordance to Theorem~\ref{theorem:onelayerlearning}, the remaining uncertainty by the time of convergence can be tuned by altering the variance scaling factor $S$ introduced in equation~\eqref{eq:fullnonlineartrackingmpccostonelayerlearning}, as well as the controller horizon length $N$, and hence the set of feasible positions $\mathbb{F}_{i,k}$.
\vspace{-0.3em}
\section{Experimental Results}
\label{sec:experiments}
In the following section, we introduce the mathematical set-up of our experiments (Section~\ref{subsec:mathematicalsetup} and~\ref{subsec:expermpccost}) and provide details on the used software and hardware framework (Section~\ref{subsec:hardwaredetails}). Furthermore, the obtained experimental hardware results for both control architectures (one-layer and two-layers) and for both known and unknown environments are presented in Sections~\ref{subsec:experimenttwolayer}-~\ref{subsec:experimentonelayerlearning}.
\vspace{-0.6em}
\subsection{Mathematical Set-up}
\label{subsec:mathematicalsetup}
We consider a fleet of 4 cars that covers area of $\mathbb{A}= [-1.55,1.55] \times [-1.85,1.85]$ meters. The nonlinear continuous dynamics for the $i-$th car is modeled using the kinematic bicycle model~\cite{Rajamani2012}, 
\begin{equation}
\begin{bmatrix}
    \dot{x}_{p,i} \\
    \dot{y}_{p,i} \\
    \dot{\psi_{i}}\\
    \dot{\delta_{i}}
    \end{bmatrix} = 
\begin{bmatrix}
    \cos{(\psi_{i})} \\
    \sin{(\psi_{i})} \\
    \frac{1}{L} \cdot \tan{(\delta_{i})}\\
    0
\end{bmatrix} v_{i} + 
\begin{bmatrix}
    0 \\
    0 \\
    0\\
    1
\end{bmatrix} u_{d,i} 
= g_{1,i} v_{i} + g_{2,i} u_{d,i}
\label{eq:dynamics}
\end{equation}
where $p_{i} =[x_{p,i},y_{p,i}]^{\top} \in \mathbb{R}^{2}$ is the position in Cartesian coordinates and~$\psi_{i}$ its orientation with respect to the $x$-axis. The orientation of its wheels with respect to their neutral position is given by~$\delta_{i}$. As an input, steering and velocity are available, indicated as~$v_{i}$ and~$u_{d,i}$. Lastly, we denote with $L$ the wheelbase length. The discrete model is obtained using the forward Euler method, with a sampling time $T_{s}$. It is set to 0.05 seconds for all experiments except for the one-layer, learning-based approach, where due to the increased complexity of its target cost, it has been set to 0.1 seconds. The coverage is performed in accordance to density $\phi_{1}$ in case of a known environment and $\phi_{2}$ for the experiments in which the environmental information is initially unknown, where
\begin{align}
    &\phi_{1}(x_p,y_p) = 5.0e^{-0.5((x_{p}-1.4)^2 + (y_{p}-1.7)^2)} \\
    &\phi_{2}(x_p,y_p) = -0.5x^{2} - 0.5y^{2} - 0.5x - 0.5y + 12. 
\end{align}
The cars are able to collect noisy measurements of the density at their current location according to~\eqref{eq:measmod}, with $\nu_{i,h} \sim \mathcal{N} (0,0.1) \ \forall i \in \{1,...,4\}$. For the application of the Bayesian linear regression, the feature vector is chosen as $\Phi = [x_p^2  \: y_p^2  \: x_p  \: y_p  \: 1]^{\top}$ and the Gaussian prior is set to $\mu_0 = [0  \: 0  \: 0  \: 0  \: 0]^{\top}$ and $\Sigma_0 = \mathbb{I}$.

\subsection{MPC Cost Function}
\label{subsec:expermpccost}
In consideration of the non-holonomic system dynamics presented in Section~\ref{subsec:mathematicalsetup} the stage cost of the MPC is designed according to~\cite{Rosenfelder2021}. Therefore, the Lie brackets of the previously introduced vectors $g_{1,i}$ and $g_{2,i}$, given by
\begin{align}
    & g_{3,i} := [g_{1,i},g_{2,i}] = \begin{bmatrix} 0 & 0 & \frac{1}{L}(\tan^{2}{(\delta_{i})} + 1) & 0 \end{bmatrix}^{\top} \notag \\
    & g_{4,i} := [g_{1,i},g_{3,i}] = \begin{bmatrix} \frac{\sin{(\psi_{i})}}{L(\sin^{2}{(\delta_{i})}-1)} & \frac{\cos{(\psi_{i})}}{L\cos^{2}{(\delta_{i})}} & 0 & 0 \end{bmatrix}^{\top}, \notag
\end{align}
are used to build a stage cost that allows for parallel parking maneuvers. Concerning a full dimensional state and input reference, indicated by $x_{r,i}$ and $u_{r,i}$, it reads as
\begin{align}
    &\ell_{i} = \sum_{j = 1}^{4}Q_{j}(g_{j,i}^{\top}(x_{r})(x_{i}-\bar{x}_{i}))^{\rho_{j}} + \sum_{z = 1}^{2}R_{z}(u_{z,i}-\bar{u}_{z,i})^{\rho_{u}}. \notag 
\end{align}
By setting~$\rho_{1}= \rho_{2} = \rho_{u} = 12$,~$\rho_{3} = 6 $ and~$\rho_{4} = 4$, the stage cost satisfies the cost-controllability condition in Assumption~\ref{assumption:expocostcontrollability}, as shown in~\cite{Rosenfelder2021} based on~\cite{Coron2020}. Hence, the derived stability guarantees are applicable. The reference location is either set to the pre-calculated reference, for the two-layers approaches, or the previously calculated setpoint, for the one-layer approaches. The reference in orientation,~$\psi_{r,i}$, is set to the angle enclosed by the x-axis and the error vector to the reference and~$\delta_{r,i}$ is chosen equal to zero for all agents. While the stage cost is used for all of the presented approaches, the implemented target costs differ. In consideration of Assumption~\ref{assumption:steadystatedecreasetwolayers}, the target cost of the one-layer approaches is formed using the same structure,
\begin{align}
    &\ell_{T,i} \! = \! \sum_{j = 1}^{4}Q_{j}(g_{j,i}^{\top}(x_{r})(\bar{x}_{i}-x_{r,i}))^{\rho_{j}} \! + \! \sum_{z = 1}^{2}R_{n}(\bar{u}_{z,i}-u_{r,z,i})^{\rho_{u}}. \notag
\end{align}
For the one-layer approaches, the target cost follows the structures described in Sections~\ref{subsec:onelayer} and~\ref{subsec:onelayerlearning}. To speed up computation we approximate the included integral with a quadratic function whose coefficients are learned using BLR. The controller's horizon length is set to $N=30$ for all conducted experiments except for the one-layer learning-based approach where, due to the doubled sampling time, it is divided in half and set to $N=15$.

\subsection{Software and Hardware Details}
\label{subsec:hardwaredetails}
For the experiments, the miniature RC cars of Chronos in combination with the CRS software framework are used~\cite{carron2022,Froehlich2021,Tearle2021}. The vehicles' position, velocity and orientation feedback is provided by a VICON motion capture system, and the control inputs are transmitted using WiFi-connected micro-controllers. A Macbook Air with 8 GB RAM and a 1.4GHz Dual-Core Intel Core i5 processor is used as a server. Its operating system is Ubuntu 18.04. The closed-loop system is implemented using ROS (Robotics Operating System) embedded in a C++/ Python framework using ACADOS as a solver~\cite{Verschueren2019,Verschueren2018}. The approximate initial positions and orientations, as well as the car's wheelbase length are given in the following table. The radius covering each agent $i$ is set to its length, i.e., $r_{i,\max} = L_i$.

\begin{table}[h]
\begin{center}
\vspace{1ex}
\begin{tabular}{l|cccccc}
\hline car & $x_{p,i,0}$ & $y_{p,i,0}$ & $\psi_{i,0}$ & $\delta_{i,0}$ & $L_{i}$ \\ \hline  \hline
1  & -1.26m & -1.4m & 0.785 rad & 0 rad & 0.115m \\
2 & -0.86m & -1.1m & 0.785 rad & 0 rad & 0.099m \\
3 & -1.25m & -1.1m & 0.000 rad & 0 rad & 0.115m \\
4  & -0.85m & -1.4m & 1.570 rad & 0 rad & 0.093m  \\
\hline
\end{tabular}
\caption{Values of initial configuration and model parameters of minature RC cars for all conducted experiments.}
\label{tab:model_param}
\end{center}
\end{table}

\vspace{-1.6em}
\subsection{Results Two-Layers Coverage MPC}
\label{subsec:experimenttwolayer}
The application of Algorithm~\ref{alg:twolayermpcalg} in the described set-up using $\phi_{1}$ results in a decrease of the locational optimization cost $H(p,\mathbb{W})$, defined in equation~\eqref{eq:cortescost}, shown in Figure~\ref{fig:twolayercoveragecost}. For a more detailed visualization of the obtained solution, in Figure~\ref{pics:twolayersconfig} the agents' location, their travelled paths, predicted trajectory, as well as the current Voronoi partitions and their centroids, are shown for three instances in time.
\begin{figure} [h]
\centering
% This file was created by matlab2tikz.
%
%The latest updates can be retrieved from
%  http://www.mathworks.com/matlabcentral/fileexchange/22022-matlab2tikz-matlab2tikz
%where you can also make suggestions and rate matlab2tikz.
%
\definecolor{mycolor1}{rgb}{0.47000,0.67000,0.19000}%
%
\begin{tikzpicture}

\begin{axis}[%
height=0.95in,
width=0.40\textwidth,
yshift=0.8cm,
at={(0.0in,0.0in)},
scale only axis,
xmin=0,
xmax=25.2,
xlabel style={font=\color{white!15!black}},
xlabel style={font=\footnotesize},
xlabel style={yshift=0.6ex,},
xlabel={Time [sec]},
ymin=0,
ymax=120,
ylabel style={font=\color{white!15!black}},
ylabel style={font=\footnotesize},
ylabel style={xshift=0.6ex,},
ylabel={H(p,$\mathbb{W}$,$\phi$)},
axis background/.style={fill=white},
tick label style={font=\footnotesize},
xmajorgrids,
ymajorgrids,
legend style={legend cell align=left, align=left, draw=white!15!black},
legend style={font=\footnotesize}
]
\addplot [color=mycolor1, line width=1.4pt]
  table[row sep=crcr]{%
-0.338209390640259	102.647579287503\\
%-0.120396614074707	102.650379790131\\
%-0.0944283008575439	102.649041787027\\
-0.0326046943664551	102.648919358382\\
%0.164487838745117	102.648564205503\\
%0.164519786834717	102.648564205503\\
0.379635572433472	101.650678720284\\
%0.531607627868652	99.9832960661896\\
%0.663259029388428	98.100369679627\\
0.706910610198975	97.3808765327923\\
%0.743441104888916	96.7860726672732\\
%0.837706089019775	95.5592937403551\\
0.938856601715088	94.5803479194664\\
%1.10168814659119	93.1907071946915\\
%1.18570184707642	91.9654483810193\\
1.27984595298767	90.5439519759774\\
%1.42377543449402	89.6583544342939\\
%1.48708319664001	89.5734361120679\\
1.64046311378479	87.2428284696633\\
%1.68371367454529	86.4780777110238\\
%1.75322890281677	86.2340141255883\\
1.93411183357239	84.5748251348349\\
%1.97922921180725	83.6687619904134\\
%2.07647466659546	82.1132428673362\\
2.14527606964111	81.0686089901058\\
%2.31244826316833	80.3882495022819\\
%2.34497833251953	80.385594525887\\
2.48258137702942	78.3813056106173\\
%2.60901021957397	76.8389265473559\\
%2.64507079124451	76.63140487506\\
2.76235008239746	75.2142285441418\\
%2.88889527320862	73.5221709962342\\
%2.9587504863739	72.6765140027681\\
3.0932674407959	71.2825902002498\\
%3.18195343017578	70.5931298408092\\
%3.25392055511475	69.6429952758824\\
3.36086225509644	68.1432177104906\\
%3.54534912109375	66.8982951753832\\
%3.59189772605896	66.0435954305736\\
3.69675660133362	64.8406673330911\\
%3.89260768890381	62.842612135393\\
%3.89263033866882	62.842612135393\\
4.00143790245056	61.7151153289869\\
%4.24608707427979	59.5469180384932\\
%4.24674129486084	59.5469180384932\\
4.29272150993347	59.5368410273629\\
%4.35481643676758	59.5360577899858\\
%4.47510623931885	57.2718949620527\\
4.64856171607971	55.747953944547\\
%4.69350457191467	54.9157191420592\\
%4.75423240661621	54.352073217653\\
%4.88121771812439	52.3926542037478\\
4.99211096763611	51.6617495180296\\
%5.08892917633057	50.7059018886116\\
%5.2321183681488	49.1108412133614\\
5.31547856330872	48.8414744940091\\
%5.37159657478333	48.6816056844934\\
%5.51115584373474	47.3733835335136\\
5.57099938392639	46.5474630841915\\
%5.66567897796631	45.9120434777475\\
%5.74089932441711	45.589434446125\\
5.85310626029968	44.894979550799\\
%5.97864818572998	43.167988559319\\
%6.06009483337402	42.5098431624329\\
6.14809060096741	42.3727758131693\\
%6.33588719367981	41.5770666486947\\
%6.42340469360352	40.593456310887\\
6.4443953037262	40.4232102325236\\
%6.57820129394531	39.2730744627907\\
%6.67769122123718	38.8211054210777\\
6.78823828697205	37.6872490283446\\
%6.86511731147766	36.927867958086\\
%6.99781703948975	35.7591443521735\\
7.08165597915649	35.0885258415117\\
%7.25101828575134	33.9811974919581\\
%7.25152850151062	33.9811974919581\\
7.381103515625	33.0909532862823\\
%7.44877195358276	32.9434803185874\\
%7.54210329055786	32.8291623216915\\
7.64353704452515	32.3261399170909\\
%7.74606490135193	30.7794221711876\\
%7.85150980949402	30.6330383639385\\
7.98551774024963	29.7616368056724\\
%8.05082273483276	29.1703762065677\\
%8.14854264259338	28.2024622910962\\
8.2467200756073	27.5555704733329\\
%8.4806809425354	25.9725538749212\\
%8.48071956634521	25.9725538749212\\
8.55836772918701	25.5478380423889\\
%8.76805377006531	24.7110215864656\\
%8.76827788352966	24.7110215864656\\
8.97729063034058	22.7416741498559\\
%8.97729825973511	22.7416741498559\\
%9.20506644248962	21.6697271905329\\
9.20586514472961	21.6697271905329\\
%9.25269436836243	21.6578597293837\\
%9.42942547798157	20.2956469578275\\
9.48990058898926	20.2956469578275\\
%9.6376473903656	18.7990692647233\\
%9.89465689659119	17.8766149821071\\
9.89466261863708	17.8766149821071\\
%9.89467811584473	17.8766149821071\\
%9.94935369491577	17.5111462161873\\
10.0524241924286	16.9568819517558\\
%10.1427183151245	16.9423351035966\\
%10.2808077335358	16.9249238778767\\
10.3680753707886	15.4988213974109\\
%10.4440491199493	15.0083601987565\\
%10.6120595932007	14.8543471179229\\
10.6800458431244	14.7218314108224\\
%10.7857151031494	14.1918430247996\\
%10.9001326560974	13.6096627991704\\
11.0366690158844	13.088454039026\\
%11.228307723999	12.2179455565088\\
%11.4209961891174	11.5172134253931\\
11.4210088253021	11.5173286269728\\
%11.5448913574219	11.5325327533073\\
%11.7024807929993	11.1991695873527\\
11.9768662452698	9.98089645276595\\
%12.0832796096802	9.6174038449186\\
%12.394145488739	8.70573011443336\\
12.3941540718079	8.70538961653489\\
%12.4929599761963	8.70747932135404\\
%12.5585391521454	8.25759752111568\\
12.6734893321991	8.21666898810312\\
%12.9700100421906	7.04508171925177\\
%13.0806632041931	6.76067759202343\\
13.314270734787	6.4315402211206\\
%13.3243107795715	6.42592000726814\\
%13.5467875003815	5.99258181469493\\
13.6518497467041	5.89952076589098\\
%13.9441342353821	5.47510104742717\\
%13.9441404342651	5.47696449693743\\
14.0644507408142	5.41772180971213\\
%14.0644700527191	5.41772180971213\\
%14.119989156723	5.37110657683742\\
14.3853130340576	4.99618240074548\\
%14.491269826889	4.93901815947285\\
%14.5808637142181	4.9220462165736\\
14.7170324325562	4.78455483894257\\
%14.8021757602692	4.6712354614438\\
%14.949814081192	4.69546979633859\\
15.0867953300476	4.76984227848602\\
%15.1770157814026	4.74039619511538\\
%15.3377788066864	4.50292754953204\\
15.435783624649	4.46366073266869\\
%15.5861015319824	4.47175330200768\\
%15.6999230384827	4.42317167673856\\
15.7909214496613	4.315918881056\\
%15.9280953407288	4.30039354408673\\
%16.1429970264435	4.1909537405945\\
16.2356607913971	4.16836022282263\\
%16.3624284267426	4.18180027912074\\
%16.5106637477875	4.20082468018567\\
16.5498957633972	4.20252487753888\\
%16.6256432533264	4.18929146194679\\
%16.9483022689819	3.9588656861856\\
17.1693933010101	3.95544118351298\\
%17.1693971157074	3.95869832278562\\
%17.2919044494629	3.90368280901756\\
17.3760797977448	3.85324771642473\\
%17.5839233398438	3.80067439299659\\
%17.5839557647705	3.80067439299659\\
17.5840194225311	3.80067439299659\\
%17.8079395294189	3.79316652122807\\
%17.9587700366974	3.78006745841675\\
17.999499797821	3.74737769552958\\
%18.2456641197205	3.75329673928541\\
%18.3619546890259	3.7616428690176\\
19.1916697025299	3.63101849576282\\
%19.1917409896851	3.62508874143829\\
%19.1918005943298	3.62530291240307\\
19.1918063163757	3.62530291240307\\
%19.191930770874	3.62530291240307\\
%19.3209233283997	3.63367444921927\\
19.3210594654083	3.63367444921927\\
%19.3918113708496	3.67092368727737\\
%19.6210758686066	3.61277979219668\\
19.7921669483185	3.55588780549529\\
%19.792172908783	3.55588780549529\\
%19.7921805381775	3.55588780549529\\
20.0480847358704	3.58481868771129\\
%20.0494854450226	3.58481868771129\\
%20.1437504291534	3.58668381640393\\
20.1716260910034	3.58668381640393\\
%20.3706171512604	3.55484827867995\\
%20.4409732818604	3.55557333104345\\
20.5181019306183	3.59219938138603\\
%20.5896918773651	3.64219475174362\\
%20.6878204345703	3.65496290818748\\
20.7951703071594	3.63999618252195\\
%20.936244726181	3.56490340998247\\
%21.0492799282074	3.5644043950615\\
21.342613697052	3.61610290535779\\
%21.3426361083984	3.61610290535779\\
%21.4711365699768	3.60989234712801\\
21.5759742259979	3.6106372548861\\
%21.6383993625641	3.59459692391849\\
%21.7403063774109	3.58744927866076\\
21.9195487499237	3.51613366667814\\
%22.0776562690735	3.48154971242334\\
%22.1792168617249	3.49347423845425\\
22.2911365032196	3.48902603514975\\
%22.3123288154602	3.48309053431046\\
%22.3434128761292	3.47035978619215\\
22.4151690006256	3.4382265377717\\
%22.5262145996094	3.40342663739622\\
%22.6690542697906	3.40804449011352\\
22.7778334617615	3.41473465223552\\
%22.8305082321167	3.41733795529059\\
%22.9587891101837	3.40136534436434\\
23.036957025528	3.37709459915283\\
%23.1890845298767	3.37742842094551\\
%23.3264737129211	3.38863126511409\\
23.3264787197113	3.38863126511409\\
%23.4356791973114	3.38070521625049\\
%23.630350112915	3.385137238515\\
23.6304533481598	3.385137238515\\
%23.7790114879608	3.35368847879553\\
%23.8493964672089	3.33952593740054\\
23.93940782547	3.32267033198935\\
%24.0230920314789	3.31365591276766\\
%24.1369876861572	3.32139813547835\\
24.2461638450623	3.3099582631927\\
%24.3494703769684	3.3071655503769\\
%24.5901727676392	3.32011811253782\\
24.5902140140533	3.32011811253782\\
%24.6975922584534	3.29596610405813\\
%24.7479755878448	3.27565624792515\\
24.8939726352692	3.26971897284158\\
%24.9583642482758	3.27851586426868\\
%25.0035440921783	3.28119474712716\\
25.295355796814	3.346919760632\\
%25.2955074310303	3.346919760632\\
%25.5031731128693	3.30377837913201\\
25.5032165050507	3.30377837913201\\
%25.5576188564301	3.28765293436966\\
%25.6483964920044	3.261116129947\\
25.7305290699005	3.29141016448344\\
%25.8708355426788	3.2996332849801\\
%26.0308413505554	3.34654494235354\\
26.0537099838257	3.34272330405357\\
%26.1888933181763	3.29592935864243\\
%26.6888473033905	3.32093916071875\\
26.689044713974	3.32093916071875\\
%26.6890523433685	3.32093916071875\\
26.6892867088318	3.32093916071875\\
};
\addlegendentry{H(p,$\mathbb{W}$,$\phi$)}

\end{axis}
\end{tikzpicture}%
\vspace{-0.7em}
\caption{Locational optimization cost decrease over time, applying Algorithm~\ref{alg:twolayermpcalg} with respect to a known density $\phi_{1}$.}
\label{fig:twolayercoveragecost}
\vspace{-0.6em}
\end{figure} 
\begin{figure}[h!]
\begin{minipage}[t]{0.15\textwidth}
\centering
\includegraphics[trim={5.2cm 8.1cm 4.6cm 7.7cm},clip, width = 0.8\textwidth]{figures/config_7sec_two_layer_6_220316_1215-compressed.pdf}
\centering
\end{minipage}
\hfill
\begin{minipage}[t]{0.15\textwidth}
\centering
\includegraphics[trim={5.2cm 8.1cm 4.6cm 7.7cm},clip, width = 0.8\textwidth]{figures/config_13sec_two_layer_6_220316_1215-compressed.pdf}
\centering
\end{minipage}
\hfill
\begin{minipage}[t]{0.15\textwidth}
\centering
\includegraphics[trim={5.2cm 8.1cm 4.6cm 7.7cm},clip, width = 0.8\textwidth]{figures/config_31sec_two_layer_6_220316_1215-compressed.pdf}
\centering
\end{minipage}
\caption{Configurations of cars at 2, 8, and 26 seconds for the application of Algorithm~\ref{alg:twolayermpcalg} in the described set-up. The agents' location and their predicted trajectory are given in red, the Voronoi partitions in green, their centroids in blue, and the traveled paths are visualized in light grey.}
\label{pics:twolayersconfig}
\vspace{-0.8em}
\end{figure}While the general trend of the locational optimization cost is pointing downwards, its decrease is not monotonic. Small increases are caused by movements not fulfilling the partition update conditions presented in~\eqref{eq:updatereq} and accordingly not causing a partition and centroid update, i.e. reorientation movements (see Figure~\ref{pics:singleintvsarbdyn}).
\vspace{-0.9em}
\subsection{Results Two-Layers, Learning-Based Coverage MPC}
\label{subsec:experimenttwolayerlearning}
Figure~\ref{fig:twolayerlearningcoveragecost} shows the locational optimization cost and the estimated cost when applying Algorithm~\ref{alg:twolayermpcalglearningimp} with the initially unknown density~$\phi_2$. Further, Figure~\ref{pics:twolayerslearningconfig} shows the according configurations at three instances in time.  
\begin{figure} [h!]
\centering
% This file was created by matlab2tikz.
%
%The latest updates can be retrieved from
%  http://www.mathworks.com/matlabcentral/fileexchange/22022-matlab2tikz-matlab2tikz
%where you can also make suggestions and rate matlab2tikz.
%
\definecolor{mycolor1}{rgb}{0.47000,0.67000,0.19000}%
\definecolor{mycolor2}{rgb}{0.79000,0.49000,0.43000}%
%
\begin{tikzpicture}

\begin{axis}[%
height=0.95in,
width=0.40\textwidth,
yshift=0.8cm,
at={(0.0in,0.0in)},
scale only axis,
xmin=0,
xmax=59,
xlabel style={font=\color{white!15!black}},
xlabel style={font=\footnotesize},
xlabel style={yshift=0.6ex,},
xlabel={Time [sec]},
ymin=0,
ymax=80,
ylabel style={font=\color{white!15!black}},
ylabel style={font=\footnotesize},
ylabel style={xshift=0.6ex,},
ylabel={H(p,$\mathbb{W}$,$\phi$) \& H(p,$\mathbb{W}$,$\hat{\phi}$)},
axis background/.style={fill=white},
tick label style={font=\footnotesize},
xmajorgrids,
ymajorgrids,
legend style={legend cell align=left, align=left, draw=white!15!black},
legend style={font=\footnotesize}
]
\addplot [color=mycolor1, line width=1.4pt]
  table[row sep=crcr]{%
0.420625162124634	73.5375866675462\\
%0.489288282394409	73.5375408803291\\
%0.850826215744019	73.5362744945654\\
%0.897809219360352	73.5048014697291\\
1.00277514457703	73.466953182668\\
%1.12979311943054	73.2981779346007\\
%1.34106035232544	72.4570640735746\\
%1.57837362289429	71.4017589191556\\
1.65951700210571	71.1716612623331\\
%1.81642551422119	71.1597638272605\\
%2.0189115524292	69.9178471609299\\
%2.23138756752014	69.4859842506241\\
2.2315568447113	69.4859842506241\\
%2.33489651679993	69.1036792894039\\
%2.57929892539978	68.1442603482274\\
%2.57936425209045	68.1442603482274\\
2.96555585861206	66.8358267519445\\
%2.96556348800659	66.8358267519445\\
%2.96617336273193	66.8358267519445\\
%3.03980965614319	66.8520834680558\\
3.14297742843628	66.8522551283534\\
%3.32022733688355	65.8372267574719\\
3.46950216293335	65.527680656503\\
%3.46954746246338	65.527680656503\\
%3.60934205055237	65.0736535576782\\
%3.7445876121521	64.3229212628427\\
3.87013382911682	64.197469410914\\
%4.01177234649658	63.3999751770899\\
%4.10301084518433	62.714655722836\\
%4.22800250053406	62.6447232636872\\
4.32021350860596	62.6583039298582\\
%4.49588556289673	62.1427222294875\\
%4.60772318840027	61.2863166841029\\
%4.71755905151367	61.2759177335493\\
4.95907301902771	60.1668470471026\\
%4.95908160209656	60.1668470471026\\
%5.05162043571472	59.9466160759429\\
%5.20451254844666	59.0267676352814\\
5.57901592254639	57.736788072081\\
%5.5790500164032	57.736788072081\\
%5.57970185279846	57.736788072081\\
%5.97623343467712	56.9069622043635\\
5.97626991271973	56.9069622043635\\
%6.00170822143555	56.9069622043635\\
%6.1019136428833	56.907008029128\\
%6.26189942359924	55.5591000850887\\
6.5089394569397	53.9755852682585\\
%6.57637996673584	53.9955850716927\\
%6.69121475219727	53.997409044121\\
%6.94041318893433	53.0832813928958\\
6.99789304733276	52.5969630261488\\
%7.18834848403931	52.3305730235821\\
7.36334104537964	51.3008011886345\\
%7.56809325218201	51.0331426164375\\
%7.56809897422791	51.0331426164375\\
%7.62411541938782	50.8070596020388\\
7.81034440994263	49.8972079780844\\
%7.98921818733215	49.6149413803786\\
%8.05398769378662	49.0792959553913\\
%8.16235013008118	48.6479652079623\\
8.31476492881775	48.5055076053604\\
%8.37018222808838	48.2514073469859\\
%8.72280068397522	47.0504569940987\\
%8.72280616760254	47.0504569940987\\
9.01490659713745	46.3724556804997\\
%9.01496548652649	46.3724556804997\\
%9.01532001495361	46.3724556804997\\
%9.34970684051514	45.2487365612487\\
9.34974498748779	45.2487365612487\\
%9.34978504180908	45.2487365612487\\
%9.5608615398407	44.3143086608143\\
%9.61851806640625	44.3143086608143\\
9.74689025878906	44.1990133621557\\
%9.96127982139587	42.8636564119263\\
%10.0929660320282	42.3934452351759\\
%10.1839732646942	42.5272479141303\\
10.3562631130218	42.0550286055876\\
%10.4828819751739	41.3316572726549\\
%10.5345203399658	40.9415277158797\\
%10.6591269493103	40.7469412165604\\
10.780286026001	40.7444563485648\\
%10.9054092884064	40.3541330197488\\
%11.10487241745	39.6586860962452\\
%11.1388580322266	39.6346375871146\\
11.3006488800049	39.0149047643453\\
%11.4709231376648	38.6421006431391\\
%11.5949499130249	38.5274291137528\\
%11.7224828720093	37.8035733228067\\
11.8191837787628	37.4633633363564\\
%11.9995903491974	37.4541565362086\\
%12.0857843875885	37.1355528800832\\
%12.2231640338898	36.4738953617358\\
12.4825975418091	36.0552588845833\\
%12.6197120666504	35.6627036002165\\
%12.6197244644165	35.6627036002165\\
%12.7353512763977	35.6308769630055\\
12.9461745738983	35.0249768839258\\
%13.029993724823	34.8683290885617\\
%13.1215083122253	34.8417628908651\\
%13.2974931716919	34.2634729099509\\
13.4828106880188	34.0994683297976\\
%13.6017415046692	33.9881874509002\\
%13.7464329719543	33.3249215716421\\
%13.86565990448	32.9759879215199\\
13.9670102119446	32.9799482997859\\
%14.1072391986847	32.7359170324098\\
%14.3973576545715	32.1159042243768\\
%14.3973626613617	32.1159042243768\\
14.4525951862335	32.045044345739\\
%14.6058504104614	31.6504154721102\\
%14.7166051387787	31.6124829641667\\
%14.841578912735	31.1654900783467\\
14.9496008872986	31.0309358261225\\
%15.0831017017364	30.6534998808075\\
%15.2263788700104	30.4441236877527\\
%15.4016568183899	30.2741880817715\\
15.6143710136414	29.773711638385\\
%15.8217999458313	29.5508870781744\\
%15.8224465370178	29.5508870781744\\
%15.9890865802765	29.1389768219882\\
16.1281466007233	28.9295080908731\\
%16.2221219062805	28.8305523720104\\
%16.3112008094788	28.6204266337133\\
%16.5050880432129	28.3071720932582\\
16.6190881252289	28.296291721146\\
%16.7282452106476	28.0017928319732\\
%16.9825152873993	27.6238258261685\\
%17.0168098926544	27.6587037681424\\
17.2105741024017	27.3989993058996\\
%17.472890329361	27.0044018803968\\
%17.5090875148773	26.9980154966202\\
%17.5980572223663	26.997655114299\\
17.7507791042328	26.7800479496383\\
%17.8156904697418	26.4764063413024\\
%17.9230515480041	26.512210460683\\
%18.0529295921326	26.4322941744495\\
18.2837771892548	26.0124599493311\\
%18.2838572978973	26.0124599493311\\
%18.4809062004089	25.9977109772607\\
%18.6011461734772	26.0043022600137\\
18.761940908432	25.7698706639952\\
%18.9734606266022	25.4577711294226\\
%19.1326980113983	25.3472483567037\\
%19.5531069755554	24.9354974673761\\
19.5531358242035	24.9354974673761\\
%20.0045535087585	24.8074904737757\\
%20.0045587539673	24.8074904737757\\
%20.004967880249	24.8074904737757\\
20.0049807548523	24.8074904737757\\
%20.1169054031372	24.6808847358539\\
%20.2384790897369	24.7346631202572\\
%20.3561331748962	24.6329998003482\\
20.5109445571899	24.5918288554217\\
%20.6463200569153	24.5726811180237\\
%20.7514168739319	24.4965342007404\\
%20.9313446998596	24.4174196230076\\
21.095033121109	24.460792474996\\
%21.3913058757782	24.2937296383691\\
%21.6324288368225	24.4003651340537\\
%21.6324719905853	24.4003651340537\\
21.854337644577	24.3649180002364\\
%22.0354251384735	24.4052449035102\\
%22.0354876041412	24.4052449035102\\
%22.2573592185974	24.4727795865282\\
22.2996007919312	24.4727794676533\\
%22.5107860088348	24.5271009709549\\
%22.6832606315613	24.615264823043\\
%23.0745903968811	24.7188612444077\\
23.076425743103	24.7188612444077\\
%23.0765375614166	24.7188612444077\\
%23.2663554668427	24.7861199331851\\
%23.3698591709137	24.9808670451831\\
23.4285540103912	24.9903764072817\\
%23.5829140663147	24.9684456730355\\
%23.7451769828796	25.158861320249\\
%23.8904494762421	25.3023562050003\\
24.0351898193359	25.2809555864351\\
%24.1868371486664	25.4252509854942\\
%24.2919966697693	25.5803009417738\\
%24.4268409729004	25.6322977486586\\
24.5920493125916	25.7601597627214\\
%24.8797428131104	26.1915016068073\\
%24.8797811985016	26.1915016068073\\
%25.0822343349457	26.2470292883343\\
25.1347145557404	26.3475804174992\\
%25.2200073719025	26.4657874374661\\
%25.4428767681122	26.7819756872408\\
%26.3120278835297	27.7427870121645\\
26.3120715141296	27.7427870121645\\
%26.3121389865875	27.7427870121645\\
%26.3122283935547	27.7427870121645\\
%26.3124303340912	27.7427870121645\\
26.3125948429108	27.7427870121645\\
%26.3126947402954	27.7427870121645\\
%26.7981185436249	28.025131861481\\
%26.7995223522186	28.025131861481\\
26.7995843410492	28.025131861481\\
%26.7996272563934	28.025131861481\\
%26.8955528259277	28.5495297909954\\
%27.9453289031982	29.9445182871999\\
27.9453796863556	29.9445182871999\\
%27.9454285621643	29.9445182871999\\
%27.945477437973	29.9445182871999\\
%27.9455241680145	29.9445182871999\\
27.9455747127533	29.9445182871999\\
%29.2150637626648	31.0463797648174\\
%29.2150699615479	31.0463797648174\\
%29.2151073932648	31.0463797648174\\
29.2151443481445	31.0463797648174\\
%29.2151770114899	31.0463797648174\\
%30.8537635326385	34.4601053813249\\
%30.8537714004517	34.4601053813249\\
30.8537902355194	34.4601053813249\\
%30.8538136005402	34.4601053813249\\
%30.8542685031891	34.4601053813249\\
%30.8542816162109	34.4601053813249\\
30.8543009281158	34.4601053813249\\
%30.8543068885803	34.4601053813249\\
%31.2613622665405	36.9326216310509\\
%31.2613920688629	36.9326216310509\\
31.2614261627197	36.9326216310509\\
%31.3491465568542	36.9326216310509\\
%31.6353506565094	37.9753679326956\\
%31.7009007453918	37.9744697365487\\
32.3394157409668	38.8384724198689\\
%32.3394221782684	38.8384724198689\\
%32.7008087158203	40.1030171918157\\
%32.7884866714478	40.2453295986813\\
33.0511624336243	40.3566244660761\\
%33.0511979579926	40.3566244660761\\
%33.2069298744202	41.0789808111384\\
%33.3059620380402	41.4221869574822\\
33.7932095050812	42.5634788236068\\
%33.7932152271271	42.5634788236068\\
%33.7933632850647	42.5634788236068\\
%33.7940322875977	42.5634788236068\\
33.7945327281952	42.5634788236068\\
%34.1493222236633	43.631213597984\\
%34.1493288993835	43.631213597984\\
%34.1493486881256	43.631213597984\\
34.3521542072296	43.6888661559272\\
%34.3521592140198	43.6888661559272\\
%34.407747220993	43.6888663992982\\
%34.5416676521301	44.9407275106139\\
34.7464391708374	44.8383928208856\\
%34.8334414482117	45.2039914061237\\
%34.8939089298248	45.72844604741\\
%34.9977795600891	46.5113375262851\\
35.1801499843597	47.0118112506685\\
%35.1801588058472	47.0118112506685\\
%35.3600363254547	47.0789481202246\\
%35.4869303226471	47.0601361669033\\
35.8174631118774	48.1530362655659\\
%35.8174817085266	48.1530362655659\\
%36.020121049881	48.6366779027739\\
%36.0201854228973	48.6366779027739\\
36.1910464286804	48.8343048837198\\
%36.3291983127594	49.4940169120725\\
%36.386096906662	49.4939806987675\\
%36.4642600536346	50.0900280937696\\
36.7553929805756	50.8367315860877\\
%36.7553977489471	50.8367315860877\\
%37.0482501506805	51.1342018889859\\
%37.0482875823975	51.1342018889859\\
37.2553562641144	51.9869392051622\\
%37.2553903579712	51.9869392051622\\
%37.3868126392365	52.7729987372945\\
%37.5363270759583	53.1262921696331\\
37.708508682251	53.28967470477\\
%37.8455035209656	54.5745427864279\\
%37.9154173851013	54.4772656351833\\
%38.0308119773865	54.5283025210685\\
38.1543263912201	55.3186366839583\\
%38.1543299674988	55.3186366839583\\
%38.3650972366333	55.8838688476104\\
%38.4100870609283	55.9456683162372\\
38.6126198291779	57.0548580838601\\
%38.7188293457031	57.0626160554563\\
%38.8497247219086	57.3420299530953\\
%39.0036797046661	58.1871975510329\\
39.584637594223	61.3519783572304\\
%39.5846428394318	61.3519783572304\\
%39.584646654129	61.3519783572304\\
%39.7133063793182	61.2109262536161\\
39.806764793396	61.2969058850044\\
%39.9291097640991	61.7506435924003\\
%40.0604414463043	63.0570794537602\\
%40.3766531467438	63.8902304461408\\
40.3767680644989	63.8902304461408\\
%40.4598187923431	64.4337424843376\\
%40.5434920310974	65.0121831865502\\
%40.6191706180573	64.906691552179\\
40.853698682785	65.4532986351146\\
%46.0684711456299	21.8434873574566\\
%47.0084621429443	20.6083116830592\\
%48.7002977848053	17.800398726904\\
49.6572093486786	15.9640720286201\\
%51.9544324398041	14.0900402339715\\
%53.0441104888916	13.0878466423575\\
%54.0290650844574	12.5242588001509\\
55.8657269001007	11.6545271716568\\
%55.8678416728973	11.6895764835258\\
%58.9761447429657	11.0770691449773\\
%58.9761514186859	11.0770691449773\\
58.9761569023132	11.0770691449773\\
%58.9761640548706	11.0770691449773\\
%59.3945128440857	11.1363399014387\\
};
\addlegendentry{H(p,$\mathbb{W}$,$\phi$)}

\addplot [color=mycolor2, dotted, line width=1.4pt]
  table[row sep=crcr]{%
0.420625162124634	0\\
%0.489288282394409	0\\
%0.850826215744019	2.55758463000382\\
%0.897809219360352	2.15772537204607\\
1.00277514457703	1.24453286275597\\
%1.12979311943054	1.47183480856073\\
%1.34106035232544	3.29828215010599\\
%1.57837362289429	5.34233359782094\\
1.65951700210571	5.7798936807392\\
%1.81642551422119	6.61835345818524\\
%2.0189115524292	7.251122625088\\
%2.23138756752014	11.3609787142417\\
2.2315568447113	11.3609787142417\\
%2.33489651679993	11.701284993546\\
%2.57929892539978	13.6287190581566\\
%2.57936425209045	13.6287190581566\\
2.96555585861206	16.9244479715568\\
%2.96556348800659	16.9244479715568\\
%2.96617336273193	16.9244479715568\\
%3.03980965614319	18.5606270647292\\
3.14297742843628	18.5606270647292\\
%3.32022733688355	18.1652638337247\\
%3.46950216293335	23.3171214563567\\
%3.46954746246338	23.3171214563567\\
3.60934205055237	23.7101770827624\\
%3.7445876121521	24.5992189137022\\
%3.87013382911682	26.18557185522\\
%4.01177234649658	28.2803097103579\\
4.10301084518433	29.3382214425461\\
%4.22800250053406	30.6743573147169\\
%4.32021350860596	32.1447971801589\\
%4.49588556289673	34.1878351543908\\
4.60772318840027	35.1421697779849\\
%4.71755905151367	38.2476912295269\\
%4.95907301902771	40.5931291689586\\
%4.95908160209656	40.5931291689586\\
5.05162043571472	41.9233088540102\\
%5.20451254844666	42.868962684313\\
%5.57901592254639	47.6609978229145\\
%5.5790500164032	47.6609978229145\\
5.57970185279846	47.6609978229145\\
%5.97623343467712	51.6969037762222\\
%5.97626991271973	51.6969037762222\\
%6.00170822143555	51.6969037762222\\
6.1019136428833	51.6969436217098\\
%6.26189942359924	50.4436131043119\\
%6.5089394569397	55.0606457037389\\
%6.57637996673584	55.8964391925605\\
6.69121475219727	55.9233667160603\\
%6.94041318893433	57.2115488933574\\
%6.99789304733276	56.9687190745104\\
%7.18834848403931	57.8597655466058\\
7.36334104537964	57.8561710337981\\
%7.56809325218201	58.1486929837418\\
%7.56809897422791	58.1486929837418\\
%7.62411541938782	57.8962524800094\\
7.81034440994263	57.4638245506385\\
%7.98921818733215	57.6853920770605\\
%8.05398769378662	57.2743159939745\\
%8.16235013008118	56.9460688532585\\
8.31476492881775	57.0553014728106\\
%8.37018222808838	56.765604075325\\
%8.72280068397522	55.8442025616938\\
%8.72280616760254	55.8442025616938\\
9.01490659713745	55.1764454314031\\
%9.01496548652649	55.1764454314031\\
%9.01532001495361	55.1764454314031\\
%9.34970684051514	53.8458259615487\\
9.34974498748779	53.8458259615487\\
%9.34978504180908	53.8458259615487\\
%9.5608615398407	52.7324158760819\\
%9.61851806640625	52.7320739190264\\
9.74689025878906	52.5282490940332\\
%9.96127982139587	50.8954214183558\\
%10.0929660320282	50.2600085527575\\
%10.1839732646942	50.388922056124\\
10.3562631130218	49.7194412609191\\
%10.4828819751739	48.8042665907686\\
%10.5345203399658	48.343500547155\\
%10.6591269493103	47.9863171731841\\
10.780286026001	47.9053217257699\\
%10.9054092884064	47.2809557462367\\
%11.10487241745	46.2925184703679\\
%11.1388580322266	46.2229514115622\\
11.3006488800049	45.3746839616579\\
%11.4709231376648	44.8139893248047\\
%11.5949499130249	44.5626676858183\\
%11.7224828720093	43.596384772165\\
11.8191837787628	43.1181973686967\\
%11.9995903491974	42.9964801240066\\
%12.0857843875885	42.549078789108\\
%12.2231640338898	41.7021415531438\\
12.4825975418091	41.0357727984244\\
%12.6197120666504	40.4688851579184\\
%12.6197244644165	40.4688851579184\\
%12.7353512763977	40.3635042187878\\
12.9461745738983	39.5995239544965\\
%13.029993724823	39.246461702035\\
%13.1215083122253	39.1527117600987\\
%13.2974931716919	38.4024622151635\\
13.4828106880188	38.1223526194992\\
%13.6017415046692	37.9373870595613\\
%13.7464329719543	37.0772360975861\\
%13.86565990448	36.627847652153\\
13.9670102119446	36.5803877314141\\
%14.1072391986847	36.2239175295587\\
%14.3973576545715	35.3653884550247\\
%14.3973626613617	35.3653884550247\\
14.4525951862335	35.2617022826222\\
%14.6058504104614	34.7442047621457\\
%14.7166051387787	34.7022511101889\\
%14.841578912735	34.10238135489\\
14.9496008872986	33.8822075090928\\
%15.0831017017364	33.41421550328\\
%15.2263788700104	33.0923466713804\\
%15.4016568183899	32.8448597357574\\
15.6143710136414	32.2086300276006\\
%15.8217999458313	31.886225399742\\
%15.8224465370178	31.886225399742\\
%15.9890865802765	31.3804425504608\\
16.1281466007233	31.1124547126107\\
%16.2221219062805	31.0007154271561\\
%16.3112008094788	30.6843747821088\\
%16.5050880432129	30.2703621718314\\
16.6190881252289	30.2236043168669\\
%16.7282452106476	29.8556480335659\\
%16.9825152873993	29.3660727351747\\
%17.0168098926544	29.4038681765912\\
17.2105741024017	29.0598717608283\\
%17.472890329361	28.574583678959\\
%17.5090875148773	28.5391679694731\\
%17.5980572223663	28.5137211849888\\
17.7507791042328	28.2412638121451\\
%17.8156904697418	27.9080807873993\\
%17.9230515480041	27.8993274294601\\
%18.0529295921326	27.786558812474\\
18.2837771892548	27.2810581767505\\
%18.2838572978973	27.2810581767505\\
%18.4809062004089	27.2134405223157\\
%18.6011461734772	27.2186519086694\\
18.761940908432	26.9060278465024\\
%18.9734606266022	26.5209956992494\\
%19.1326980113983	26.3713932795968\\
%19.5531069755554	25.8473295729104\\
19.5531358242035	25.8473295729104\\
%20.0045535087585	25.6451442288206\\
%20.0045587539673	25.6451442288206\\
%20.004967880249	25.6451442288206\\
20.0049807548523	25.6451442288206\\
%20.1169054031372	25.5058160469614\\
%20.2384790897369	25.5543012677634\\
%20.3561331748962	25.4345312642737\\
20.5109445571899	25.3131006560857\\
%20.6463200569153	25.2777510505609\\
%20.7514168739319	25.1781830707798\\
%20.9313446998596	25.0568270038543\\
21.095033121109	25.086368613021\\
%21.3913058757782	24.8482452493295\\
%21.6324288368225	24.9390468097774\\
%21.6324719905853	24.9390468097774\\
21.854337644577	24.8713706152281\\
%22.0354251384735	24.8949179550835\\
%22.0354876041412	24.8949179550835\\
%22.2573592185974	24.9215256123057\\
22.2996007919312	24.9215256123057\\
%22.5107860088348	24.9571377680022\\
%22.6832606315613	25.0207751545251\\
%23.0745903968811	25.0829552157799\\
23.076425743103	25.0829552157799\\
%23.0765375614166	25.0829552157799\\
%23.2663554668427	25.1412452609218\\
%23.3698591709137	25.316559485264\\
23.4285540103912	25.3259991320364\\
%23.5829140663147	25.2788882540625\\
%23.7451769828796	25.4558838604191\\
%23.8904494762421	25.5747364629724\\
24.0351898193359	25.5513222187542\\
%24.1868371486664	25.6780628069906\\
%24.2919966697693	25.8156195700783\\
%24.4268409729004	25.8622582969176\\
24.5920493125916	25.9757592929793\\
%24.8797428131104	26.3812887819492\\
%24.8797811985016	26.3812887819492\\
%25.0822343349457	26.4279707599353\\
25.1347145557404	26.5277332043672\\
%25.2200073719025	26.6394838190937\\
%25.4428767681122	26.9373406253294\\
%26.3120278835297	27.8611236240949\\
26.3120715141296	27.8611236240949\\
%26.3121389865875	27.8611236240949\\
%26.3122283935547	27.8611236240949\\
%26.3124303340912	27.8611236240949\\
26.3125948429108	27.8611236240949\\
%26.3126947402954	27.8611236240949\\
%26.7981185436249	28.1091173798652\\
%26.7995223522186	28.1091173798652\\
26.7995843410492	28.1091173798652\\
%26.7996272563934	28.1091173798652\\
%26.8955528259277	28.6291289268449\\
%27.9453289031982	29.9805378790789\\
27.9453796863556	29.9805378790789\\
%27.9454285621643	29.9805378790789\\
%27.945477437973	29.9805378790789\\
%27.9455241680145	29.9805378790789\\
27.9455747127533	29.9805378790789\\
%29.2150637626648	31.0736369036464\\
%29.2150699615479	31.0736369036464\\
%29.2151073932648	31.0736369036464\\
29.2151443481445	31.0736369036464\\
%29.2151770114899	31.0736369036464\\
%30.8537635326385	34.4154288859593\\
%30.8537714004517	34.4154288859593\\
30.8537902355194	34.4154288859593\\
%30.8538136005402	34.4154288859593\\
%30.8542685031891	34.4154288859593\\
%30.8542816162109	34.4154288859593\\
30.8543009281158	34.4154288859593\\
%30.8543068885803	34.4154288859593\\
%31.2613622665405	36.8683642395026\\
%31.2613920688629	36.8683642395026\\
31.2614261627197	36.8683642395026\\
%31.3491465568542	36.8511280381669\\
%31.6353506565094	37.8858587134803\\
%31.7009007453918	37.8788655202535\\
32.3394157409668	38.7301809730401\\
%32.3394221782684	38.7301809730401\\
%32.7008087158203	39.9834652622286\\
%32.7884866714478	40.125162702238\\
33.0511624336243	40.2323275028906\\
%33.0511979579926	40.2323275028906\\
%33.2069298744202	40.9468282696151\\
%33.3059620380402	41.2876059658287\\
33.7932095050812	42.4179419260436\\
%33.7932152271271	42.4179419260436\\
%33.7933632850647	42.4179419260436\\
%33.7940322875977	42.4179419260436\\
33.7945327281952	42.4179419260436\\
%34.1493222236633	43.4756797007498\\
%34.1493288993835	43.4756797007498\\
%34.1493486881256	43.4756797007498\\
34.3521542072296	43.5325810199121\\
%34.3521592140198	43.5325810199121\\
%34.407747220993	43.5325810199121\\
%34.5416676521301	44.7746089299041\\
34.7464391708374	44.6714122525941\\
%34.8334414482117	45.0338571046725\\
%34.8939089298248	45.5557726329883\\
%34.9977795600891	46.3344080697686\\
35.1801499843597	46.8322477466116\\
%35.1801588058472	46.8322477466116\\
%35.3600363254547	46.8976241085566\\
%35.4869303226471	46.878201273556\\
35.8174631118774	47.9632368166373\\
%35.8174817085266	47.9632368166373\\
%36.020121049881	48.4441558685348\\
%36.0201854228973	48.4441558685348\\
36.1910464286804	48.6401411710322\\
%36.3291983127594	49.2952816046452\\
%36.386096906662	49.2952439062124\\
%36.4642600536346	49.8882323888775\\
36.7553929805756	50.6295928433038\\
%36.7553977489471	50.6295928433038\\
%37.0482501506805	50.9248649604376\\
%37.0482875823975	50.9248649604376\\
37.2553562641144	51.772310436869\\
%37.2553903579712	51.772310436869\\
%37.3868126392365	52.5547044164765\\
%37.5363270759583	52.9065341570193\\
37.708508682251	53.0678330838481\\
%37.8455035209656	54.346392833479\\
%37.9154173851013	54.2495013132324\\
%38.0308119773865	54.298847193798\\
38.1543263912201	55.0862173865411\\
%38.1543299674988	55.0862173865411\\
%38.3650972366333	55.648151418932\\
%38.4100870609283	55.7094374254128\\
38.6126198291779	56.8132196538668\\
%38.7188293457031	56.8204978489272\\
%38.8497247219086	57.0978340898576\\
%39.0036797046661	57.9393326922265\\
39.584637594223	61.0899915748051\\
%39.5846428394318	61.0899915748051\\
%39.584646654129	61.0899915748051\\
%39.7133063793182	60.9497121480064\\
39.806764793396	61.0349812039909\\
%39.9291097640991	61.4856111561478\\
%40.0604414463043	62.7871459082632\\
%40.3766531467438	63.6156078514413\\
40.3767680644989	63.6156078514413\\
%40.4598187923431	64.156948730839\\
%40.5434920310974	64.7332396072264\\
%40.6191706180573	64.628218291899\\
40.853698682785	65.1713574260592\\
%46.0684711456299	21.9036530771897\\
%47.0084621429443	20.4660368352546\\
%48.7002977848053	17.7427754330535\\
49.6572093486786	15.6790764319282\\
%51.9544324398041	14.0478613389792\\
%53.0441104888916	13.0502523004837\\
%54.0290650844574	12.4896510224136\\
55.8657269001007	11.6245911032776\\
%55.8678416728973	11.6597928290454\\
%58.9761447429657	11.051824296352\\
%58.9761514186859	11.051824296352\\
58.9761569023132	11.051824296352\\
%58.9761640548706	11.051824296352\\
%59.3945128440857	11.1103290466325\\
};
\addlegendentry{H(p,$\mathbb{W}$,$\hat{\phi}$)}


\addplot[area legend, draw=none, fill=black, fill opacity=0.09, forget plot]
table[row sep=crcr] {%
x	y\\
0	0\\
40.65	0\\
40.65	80\\
0	80\\
}--cycle;
\end{axis}

\begin{axis}[%
width=0in,
height=0in,
at={(0in,0in)},
scale only axis,
xmin=0,
xmax=1,
ymin=0,
ymax=1,
axis line style={draw=none},
ticks=none,
axis x line*=bottom,
axis y line*=left
]
\end{axis}
\end{tikzpicture}%
\vspace{-0.7em}
\caption{Locational optimization cost (green) as well as estimated locational optimization cost versus time (rose), applying Algorithm~\ref{alg:twolayermpcalglearningimp} in consideration of an initially unknown $\phi_{2}$. The grey background indicates time instances for which the agents are exploring.}
\label{fig:twolayerlearningcoveragecost}
\vspace{-0.8em}
\end{figure}

\begin{figure}[h!]
\begin{minipage}[t]{0.15\textwidth}
\centering
\includegraphics[trim={5.2cm 8.1cm 4.6cm 7.7cm},clip, width = 0.8\textwidth]{figures/config_7sec_two_layer_learning_3_220316_1215-compressed.pdf}
\centering
\end{minipage}
\hfill
\begin{minipage}[t]{0.15\textwidth}
\centering
\includegraphics[trim={5.2cm 8.1cm 4.6cm 7.7cm},clip, width = 0.8\textwidth]{figures/config_42sec_two_layer_learning_3_220316_1215-compressed.pdf}
\centering
\end{minipage}
\hfill
\begin{minipage}[t]{0.15\textwidth}
\centering
\includegraphics[trim={5.2cm 8.1cm 4.6cm 7.7cm},clip, width = 0.8\textwidth]{figures/config_64sec_two_layer_learning_3_220316_1215-compressed.pdf}
\centering
\end{minipage}
\caption{Configurations of cars at 2, 37, and 59 seconds for the application of Algorithm~\ref{alg:twolayermpcalglearningimp} in the described set-up. The agents' location and their predicted trajectory are given in red, the Voronoi partitions in green, the references in blue, and the traveled paths are visualized in light grey. While the first two instances visualize the initial exploration movement, the third shows a covering behavior.}
\label{pics:twolayerslearningconfig}
\vspace{-0.4em}
\end{figure}
Defining~$F(\text{Var}_{\max,t}) = \frac{\text{Var}_{\max,t}}{\text{Var}_{\max,0}}$, the first decision in Algorithm~\ref{alg:twolayermpcalglearningimp} is guaranteed to be exploration and the agents drive towards the point of maximal variance within their initial Voronoi partition. Allowing for a better insight into the learning progress, the decreasing behaviour of the maximal variance over time is presented in Figure~\ref{fig:twolayerslearningmaxvar}. In Figure~\ref{fig:twolayerslearningcoef}, the development of estimated parameter vector $\theta$ is visualized. \\
In Figure~\ref{fig:twolayerlearningcoveragecost}, as well as in Figures~\ref{fig:twolayerslearningmaxvar} and~\ref{fig:twolayerslearningcoef}, the time instances for which the cars are exploring are indicated with a light grey background. Hence, it can be seen that, for the first 40 seconds, the agents keep exploring: correspondingly, any decrease or increase in cost is only accidental. However, by collecting measurements at each time step during exploration, after 25 seconds the mean estimate of $\theta$ is quite precise and further improvements are minor. This is also reflected in the small remaining maximal variance of the estimate. By the time the agents leave exploration mode all mean estimates show a maximal error of less than 0.05 concerning the true coefficient value and a maximal variance of approximately $10^{-3}$. Due to the remaining small uncertainty, the probability of another exploration movement for future instances is strictly positive, but very small, by the time the experiment is interrupted.
\begin{figure} [h!]
\centering
% This file was created by matlab2tikz.
%
%The latest updates can be retrieved from
%  http://www.mathworks.com/matlabcentral/fileexchange/22022-matlab2tikz-matlab2tikz
%where you can also make suggestions and rate matlab2tikz.
%
\definecolor{mycolor1}{rgb}{0.50000,0.75000,0.93000}%
%
\begin{tikzpicture}

\begin{axis}[%
height=0.95in,
width=0.40\textwidth,
yshift=0.8cm,
at={(0.0in,0.0in)},
scale only axis,
xmin=0,
xmax=59,
xlabel style={font=\color{white!15!black}},
xlabel style={font=\footnotesize},
xlabel style={yshift=0.6ex,},
xlabel={Time [sec]},
ymode=log,
ymin=0.001,
ymax=16,
yminorticks=true,
ylabel style={font=\color{white!15!black}},
ylabel style={font=\footnotesize},
ylabel style={xshift = 0.6ex,},
ylabel={$\textup{Var}_{\textup{max}}$},
axis background/.style={fill=white},
tick label style={font=\footnotesize},
xmajorgrids,
ymajorgrids,
legend style={legend cell align=left, align=left, draw=white!15!black},
legend style={font=\footnotesize}
]
\addplot [color=mycolor1, line width=1.4pt]
  table[row sep=crcr]{%
0.420608711242676	15.7830967703483\\
%0.489271116256714	15.7830967703483\\
%0.850548934936524	14.6245570138414\\
%0.897891235351563	14.1722433944758\\
1.00487942695618	14.0529728547349\\
%1.13019318580627	14.0060614962122\\
%1.34105415344238	13.9635522242133\\
%1.58354253768921	13.9393737485683\\
1.65958805084229	13.9209073702252\\
%1.83605880737305	13.8927581338847\\
%2.02010912895203	13.8671135216068\\
%2.23125858306885	13.8484390496997\\
2.23128910064697	13.8276244061326\\
%2.33593697547913	13.801378597783\\
%2.57922835350037	13.7682136749753\\
%2.57926125526428	13.7355905876587\\
2.97208185195923	13.6929408176864\\
%2.97208805084229	13.6559366437151\\
%2.97210378646851	13.6159509347427\\
%3.04132957458496	13.5304053192157\\
3.14291281700134	13.468424680844\\
%3.3281476020813	13.4104069797786\\
%3.46944184303284	13.2966230549345\\
%3.46947021484375	13.197722308189\\
3.60933537483215	13.0581014652781\\
%3.77070636749268	12.9524321702602\\
%3.87096972465515	12.7471656178854\\
%4.01497240066528	12.6779037330863\\
4.10311479568481	12.4595826346881\\
%4.22849364280701	12.2811111161847\\
%4.32065577507019	12.0928123085829\\
%4.49587292671204	11.8278173211615\\
4.61608476638794	11.502309495686\\
%4.75592918395996	11.1817350473227\\
%4.95933599472046	10.967780936598\\
%4.95934052467346	10.5938600603588\\
5.05161447525024	10.4711705999299\\
%5.20468683242798	10.1257033599653\\
%5.58660955429077	9.72549169939791\\
%5.58663554191589	9.25555254922911\\
5.58667726516724	8.86321160841773\\
%5.97616190910339	8.59450745595501\\
%5.9761971950531	8.29634049011872\\
%6.0060314655304	7.87665713160232\\
6.10699505805969	7.49038017350338\\
%6.26185746192932	7.22064244592996\\
%6.50893301963806	6.95298728515185\\
%6.58045120239258	6.46200595131957\\
6.69852275848389	6.10955103005274\\
%6.94101686477661	5.7604182564441\\
%6.99783773422241	5.43084255667882\\
%7.20303030014038	5.11132998638873\\
7.3639084815979	4.87498023872109\\
%7.56926031112671	4.57023369571156\\
%7.56926436424255	4.4182820053396\\
%7.62409467697144	4.27729479189179\\
7.8105945110321	4.13819985352058\\
%7.99015398025513	3.92750240390956\\
%8.05397171974182	3.72257384200093\\
%8.16179533004761	3.59217009248594\\
8.3147575378418	3.45923276494187\\
%8.37038154602051	3.32393659017966\\
%8.72354860305786	3.14192283581168\\
%8.72356266975403	2.97440887188299\\
9.01403756141663	2.793187875584\\
%9.01469678878784	2.67481451854161\\
%9.01470823287964	2.5247064054373\\
%9.35440845489502	2.43254820818452\\
9.35443539619446	2.33851844635211\\
%9.35446949005127	2.24409008152822\\
%9.56097192764282	2.19942568875038\\
%9.61640853881836	2.07729413469248\\
9.74840493202209	1.95559419431344\\
%9.96143360137939	1.83408184212855\\
%10.0929305076599	1.69589843920158\\
%10.1839370250702	1.63167930729271\\
10.3585786342621	1.59799751184939\\
%10.489895772934	1.49616376070298\\
%10.5344924449921	1.40725645944659\\
%10.6591836929321	1.32821700936354\\
10.8146667003632	1.2749669082742\\
%10.9155141830444	1.19941822625751\\
%11.1061386585236	1.13068225140499\\
%11.1409115314484	1.08652167708057\\
11.3015639305115	1.04382070674611\\
%11.4709155082703	0.984675822208764\\
%11.5976314067841	0.929031112888642\\
%11.7236799716949	0.875872423727902\\
11.8284985542297	0.841800829539856\\
%11.999521446228	0.793729501154888\\
%12.1139175415039	0.7495720163883\\
%12.2231506824493	0.720742448491962\\
12.4834331989288	0.692602737538425\\
%12.620437335968	0.65358656467603\\
%12.6204528331757	0.605527388964674\\
%12.7454678535461	0.58265248784589\\
12.9525310516357	0.561240708636517\\
%13.0315119743347	0.530834283385718\\
%13.1214596748352	0.511302435414154\\
%13.3175677776337	0.492565651900369\\
13.4828969955444	0.457965923713734\\
%13.639973115921	0.432876444058652\\
%13.7463969707489	0.408755239535841\\
%13.8656234264374	0.393877378496078\\
13.9674911022186	0.379612643107736\\
%14.1074556827545	0.358840036691641\\
%14.3973459720612	0.339436890570421\\
%14.3973507404327	0.327567646131307\\
14.4662909030914	0.31597551574837\\
%14.6060220718384	0.299087696722978\\
%14.7199427604675	0.288451971275148\\
%14.8448554992676	0.278522264810953\\
14.9525277137756	0.264311114410092\\
%15.0830420970917	0.25058431983507\\
%15.2266444683075	0.237477493685645\\
%15.4015814781189	0.229341566341384\\
15.6156305789948	0.217848900730351\\
%15.8253778934479	0.206737012157931\\
%15.8254236698151	0.196060681133471\\
%16.0789682388306	0.186181201051003\\
16.0789825439453	0.179787960992446\\
%16.2250976085663	0.170559542431173\\
%16.3185827255249	0.162035420768145\\
%16.5040962219238	0.153865908207175\\
16.6190711975098	0.145950795834425\\
%16.7288388729095	0.141085402689773\\
%16.9825038433075	0.134272032764588\\
%17.0167946338654	0.127740763630318\\
17.2107841491699	0.123575935770298\\
%17.4746043205261	0.115723895749298\\
%17.5090643882751	0.108283514171019\\
%17.6079167842865	0.104784516807109\\
17.7507457256317	0.101441355491831\\
%17.8155276298523	0.0981570760757316\\
%17.9232196331024	0.09488481910231\\
%18.0530476093292	0.0917178374548981\\
18.2855603218079	0.0887943294721457\\
%18.2855836868286	0.0832965770587681\\
%18.4808699607849	0.0806758967181523\\
%18.613339138031	0.077006869986217\\
18.7829482078552	0.0724055986402763\\
%18.9732889652252	0.0690337395978646\\
%19.1326312541962	0.0659018393907187\\
%19.5532950878143	0.0620952900767658\\
19.5533368110657	0.0593279495965489\\
%20.0071830272675	0.0567148803688542\\
%20.0074591159821	0.0542965701341282\\
%20.0074688911438	0.0527488715663575\\
20.0078486919403	0.0512287761217785\\
%20.1168634414673	0.0497411148553906\\
%20.2384712219238	0.0476050027757883\\
%20.3564297676086	0.0462406672743978\\
20.510938835144	0.0442662466557554\\
%20.6473092556	0.0423758878786481\\
%20.7513927936554	0.0405571627584486\\
%20.9336759567261	0.0393964101764385\\
21.1055740833282	0.0366022765859343\\
%21.3913001537323	0.0350682612259337\\
%21.6344120025635	0.0327002019431306\\
%21.6344387054443	0.0313434217969938\\
21.8629877090454	0.0304666364794155\\
%22.0352773189545	0.029219044131129\\
%22.0353116512299	0.0280286195942494\\
%22.2573225021362	0.0272490808869758\\
22.3030514240265	0.0264851292451935\\
%22.5352119922638	0.0250565083610626\\
%22.6830250740051	0.0237030616885803\\
%23.0808772563934	0.0230502454020557\\
23.0809125423431	0.0221382535148574\\
%23.0809463977814	0.0212662269881594\\
%23.2663487911224	0.0204184878835058\\
%23.3698398590088	0.0198744756873214\\
23.4285459041595	0.0196146121009167\\
%23.5831117153168	0.018874606181164\\
%23.7452105998993	0.0179215782759634\\
%23.8906676292419	0.0172403166560738\\
24.0453979492187	0.0166113810225361\\
%24.1868304729462	0.0160126045400332\\
%24.2919723510742	0.0154335001146155\\
%24.4269902229309	0.0148802289537308\\
24.6002363681793	0.0143365737872024\\
%24.8852552890778	0.0138087141173824\\
%24.8852624416351	0.0134732807463104\\
%25.0822021484375	0.0128353203403324\\
25.1356396198273	0.0125272964100762\\
%25.2408792495728	0.0122267400019312\\
%25.4529945373535	0.0119297756618887\\
%26.3092822551727	0.011636870219911\\
26.3093142032623	0.0109607919179145\\
%26.3093418598175	0.0107039266578867\\
%26.3093680858612	0.0105779755295412\\
%26.3093945503235	0.0102127907838594\\
26.3113171577454	0.00997692328605271\\
%26.3113731861114	0.0097452881081064\\
%26.7988104343414	0.00940724693107937\\
%26.8010909080505	0.00908214815617266\\
26.8010980606079	0.00886961026068089\\
%26.8011021137238	0.00845737433676775\\
%26.8955358982086	0.00826328604641828\\
%27.9643182277679	0.00798723185450257\\
27.9643666267395	0.0077191922851892\\
%27.9644112110138	0.00745892782911368\\
%27.9644574642181	0.00729689934043308\\
%27.9645018100739	0.00714032707291139\\
27.9645559310913	0.00698584207154221\\
%29.2172905921936	0.00661277278652837\\
%29.2173552036285	0.00619057392786573\\
%29.2237591266632	0.00591468561157236\\
29.2240962505341	0.00572396112983845\\
%29.2241084098816	0.00526675057225291\\
%30.8569306850433	0.00500518058164347\\
%30.8569344997406	0.00480992794798861\\
30.8569392681122	0.00466593414995466\\
%30.8569473743439	0.00452587646196039\\
%30.8569528579712	0.00439075904492508\\
%30.8569590568542	0.00412748726231471\\
30.8569678783417	0.00391824685205869\\
%30.8569728851318	0.00373995240585541\\
%31.2631685256958	0.00364106769987581\\
%31.2631980895996	0.00354275361077865\\
31.2632297992706	0.00344805323284793\\
%31.3501712799072	0.00338800586716429\\
%31.6353339672089	0.00330004887801959\\
%31.7174381732941	0.00321200520678901\\
32.3394011974335	0.00309969512402333\\
%32.3394073963165	0.00294160003574557\\
%32.7060753822327	0.00278868322839854\\
%32.7886142253876	0.00269748000930212\\
33.0536310195923	0.00258311924762098\\
%33.0536603450775	0.00247536619511223\\
%33.2076444149017	0.00245474486173746\\
%33.312033367157	0.0024137338514241\\
33.7903916358948	0.00235324911105958\\
%33.7904042720795	0.0022943615957141\\
%33.7904080867767	0.00227496672940268\\
%33.793186378479	0.00223637654773769\\
33.7932016372681	0.00219873980306393\\
%34.151874256134	0.00216288877125085\\
%34.1518804550171	0.00212851861122582\\
%34.1518837928772	0.00209500323648006\\
34.3524024009705	0.00206195394873343\\
%34.3524059772491	0.00199881118925818\\
%34.407807302475	0.00198385262082955\\
%34.5437044620514	0.00195474856744975\\
34.74682970047	0.00192627386630921\\
%34.8334404945374	0.00188422319335739\\
%34.8965370178223	0.00185711100671631\\
%34.9977204322815	0.00183128378453812\\
35.1817261695862	0.0018064260398685\\
%35.1817316532135	0.00178195604006893\\
%35.3600294113159	0.00175809168936323\\
%35.498969745636	0.00172456292279317\\
35.8174228191376	0.00169235467190599\\
%35.8174547672272	0.00166110617892082\\
%36.0260457515717	0.00163171125997143\\
%36.0260910511017	0.00160324363405115\\
36.1910125732422	0.00157560037505914\\
%36.3341519355774	0.00155820128126245\\
%36.3873586177826	0.0015412773194643\\
%36.4671365737915	0.00152445873840864\\
36.7553820133209	0.00150798597640692\\
%36.7553874969482	0.00148477753564424\\
%37.0521227836609	0.00146995978462246\\
%37.0521311283112	0.00144787587527308\\
37.2682344436646	0.00142621871836717\\
%37.268274974823	0.00141249958185622\\
%37.386811208725	0.00139907940708474\\
%37.5363182544708	0.00138570009254042\\
37.7102128982544	0.00136643087287419\\
%37.8459927558899	0.00135459437648942\\
%37.9178843021393	0.00134298011257275\\
%38.0423507213593	0.00133159483673947\\
38.1556205272675	0.00132425215957541\\
%38.155638885498	0.00132061889638752\\
%38.3648716926575	0.00131339617404279\\
%38.4113175392151	0.00130626601393777\\
38.6137465953827	0.00129588122973122\\
%38.7569715499878	0.00128907241436609\\
%38.8509041786194	0.00127557736760303\\
%39.0035182952881	0.00126578821396484\\
39.5835487365723	0.00125613415337886\\
%39.5835577964783	0.0012434517300384\\
%39.5846256732941	0.00123729507175626\\
%39.713298034668	0.00122524638080086\\
39.8087961196899	0.00121627772356647\\
%39.9287998199463	0.00121043633150012\\
%40.0604319095612	0.00120471802314238\\
%40.3765961647034	0.00119620654716335\\
40.3766471862793	0.00118502130908757\\
%40.4607483863831	0.00118229249966228\\
%40.5434603214264	0.00117422141444668\\
%40.6191648960114	0.00116888812096791\\
40.8539204120636	0.00116623132792332\\
%46.0684799671173	0.00109780374366288\\
%47.0084490299225	0.0011025701024741\\
%48.7003995895386	0.0011233151189909\\
49.6571928977966	0.00112199586348124\\
%51.9605166435242	0.0011125965142259\\
%53.0441836833954	0.00111742731087382\\
%54.0371889591217	0.00111933512473243\\
55.87338347435	0.00112084377070268\\
%55.8733899116516	0.00111732909385705\\
%58.9767014503479	0.00112916238201049\\
%58.9767071723938	0.00112916238201049\\
58.9767109870911	0.00112916238201049\\
%58.9767150402069	0.00112916238201049\\
%59.3944961547852	0.00112916238201049\\
%59.3945006847382	0.00112916238201049\\
59.3945056915283	0.00112916238201049\\
%60.3067969799042	0.00112916238201049\\
%60.3072955131531	0.00112916238201049\\
%60.3073040962219	0.00112916238201049\\
60.9895321846008	0.00112916238201049\\
};
\addlegendentry{Maximal Variance}


\addplot[area legend, draw=none, fill=black, fill opacity=0.09, forget plot]
table[row sep=crcr] {%
x	y\\
0	0.001\\
40.65	0.001\\
40.65	16\\
0	16\\
}--cycle;
\end{axis}

\begin{axis}[%
width=0in,
height=0in,
at={(0in,0in)},
scale only axis,
xmin=0,
xmax=1,
ymin=0,
ymax=1,
axis line style={draw=none},
ticks=none,
axis x line*=bottom,
axis y line*=left
]
\end{axis}
\end{tikzpicture}%
\vspace{-0.7em}
\caption{Maximal variance decrease over time applying Algorithm~\ref{alg:twolayermpcalglearningimp} in consideration of an initially unknown $\phi_{2}$. The grey background indicates time instances for which the agents are exploring.}
\label{fig:twolayerslearningmaxvar}
\vspace{-0.7em}
\end{figure}

\begin{figure} [h!]
\centering
% This file was created by matlab2tikz.
%
%The latest updates can be retrieved from
%  http://www.mathworks.com/matlabcentral/fileexchange/22022-matlab2tikz-matlab2tikz
%where you can also make suggestions and rate matlab2tikz.
%
\definecolor{mycolor1}{rgb}{0.00000,0.44700,0.74100}%
\definecolor{mycolor2}{rgb}{0.85000,0.32500,0.09800}%
\definecolor{mycolor3}{rgb}{0.92900,0.69400,0.12500}%
\definecolor{mycolor4}{rgb}{0.49400,0.18400,0.55600}%
\definecolor{mycolor5}{rgb}{0.46600,0.67400,0.18800}%
%
\begin{tikzpicture}

\begin{axis}[%
height=0.95in,
width=0.40\textwidth,
yshift=0.8cm,
at={(0.0in,0.0in)},
scale only axis,
xmin=0,
xmax=59,
xlabel style={font=\color{white!15!black}},
xlabel style={font=\footnotesize},
xlabel style={yshift=0.6ex,},
xlabel={Time [sec]},
ymin=-4,
ymax=14,
ylabel style={font=\color{white!15!black}},
ylabel style={font=\footnotesize},
ylabel style={xshift = 0.6ex,},
ylabel={Mean value},
axis background/.style={fill=white},
tick label style={font=\footnotesize},
title style={font=\bfseries},
title={},
xmajorgrids,
ymajorgrids,
legend style={legend cell align=left, align=left, draw=white!15!black},
legend style={font=\footnotesize}
]
\addplot [color=mycolor1, line width=1.4pt]
  table[row sep=crcr]{%
-2.3985894203186	0\\
%-2.34845380783081	0\\
%-2.29882984161377	0\\
%-2.24259977340698	0\\
-2.1981945514679	0\\
%-2.14788823127747	0\\
%-2.09815530776978	0\\
%-2.04817252159119	0\\
-1.99881820678711	0\\
%-1.94862990379333	0\\
%-1.87693266868591	0\\
%-1.84271411895752	0\\
-1.79806022644043	0\\
%-1.74720959663391	0\\
%-1.69869332313538	0\\
%-1.6482563495636	0\\
-1.59839563369751	0\\
%-1.548703956604	0\\
%-1.49876074790955	0\\
%-1.44885139465332	0\\
-1.3982711315155	0\\
%-1.34799294471741	0\\
%-1.29795436859131	0\\
%-1.2485387802124	0\\
-1.19843416213989	0\\
%-1.14853506088257	0\\
%-1.09835534095764	0\\
%-1.04829435348511	0\\
-0.998184490203857	0\\
%-0.948882627487182	0\\
%-0.898640441894531	0\\
%-0.847953605651855	0\\
-0.798753547668457	0\\
%-0.748502779006958	0\\
%-0.698466348648071	0\\
%-0.648822116851806	0\\
-0.597503709793091	0\\
%-0.548140096664429	0\\
%-0.498594331741333	0\\
%-0.4485755443573	0\\
-0.398829984664917	0\\
%-0.348826217651367	0\\
%-0.298781204223633	0\\
%-0.248498249053955	0\\
-0.197533178329468	0\\
%-0.147660779953003	0\\
%-0.0983634471893309	0\\
%-0.0481057643890379	0\\
0.0019752502441408	0\\
%0.0514363765716555	0\\
%0.113763284683228	0\\
%0.15691466331482	0\\
0.234845352172852	0\\
%0.308734607696533	0\\
%0.308903646469116	0\\
%0.360757303237915	0\\
0.415858459472656	0\\
%0.497472476959229	-0.559973417767658\\
%0.559248399734497	-0.899734337626924\\
%0.568476867675781	-1.10925055147351\\
0.604575109481812	-1.24944387934536\\
%0.723840665817261	-1.34873634439748\\
%0.723884773254395	-1.41779539380354\\
%0.853632164001465	-1.46391200121724\\
0.853860330581665	-1.49384579024149\\
%0.859521579742432	-1.51278364928373\\
%0.922651720046997	-1.52741710184341\\
%0.971385669708252	-1.53929682449029\\
1.02831001281738	-1.54971825703944\\
%1.07039852142334	-1.55616163072852\\
%1.11899919509888	-1.55727648671302\\
%1.16864533424377	-1.55469541574166\\
1.22722692489624	-1.54899294582253\\
%1.32403750419617	-1.54256641493168\\
%1.3240451335907	-1.53595097919651\\
%1.43628258705139	-1.52859289369781\\
1.4363646030426	-1.52024267556658\\
%1.6345018863678	-1.51058447024684\\
%1.63452954292297	-1.49876239517357\\
%1.63456315994263	-1.48529835768704\\
1.63461179733276	-1.47113700714272\\
%1.67959208488464	-1.45688045659654\\
%1.74344153404236	-1.44048922291154\\
%1.80315823554993	-1.42347798254514\\
1.8214485168457	-1.40725836202978\\
%1.86181755065918	-1.39213314819949\\
%1.95309824943542	-1.37816568328435\\
%1.98877353668213	-1.36294670056736\\
2.10328192710876	-1.34593617381211\\
%2.10334939956665	-1.32642567816492\\
%2.2043318271637	-1.30577334504528\\
%2.20438261032105	-1.2847184665543\\
%2.20442314147949	-1.26585486329441\\
2.31627173423767	-1.24890486367667\\
%2.31644148826599	-1.2325219404479\\
%2.35974855422974	-1.21459074737686\\
%2.40763850212097	-1.19454252665946\\
2.58187789916992	-1.17282459194121\\
%2.58190841674805	-1.15084983513725\\
%2.58211798667908	-1.12985193029817\\
%2.96352286338806	-1.11033909646494\\
2.96353478431702	-1.09161447121596\\
%2.96354002952576	-1.07274543751009\\
%2.96354813575745	-1.05355282901064\\
%2.96359629631042	-1.03262023763727\\
2.96360988616943	-1.01142749943801\\
%2.9636182308197	-0.989950629694249\\
%2.96363134384155	-0.968382030706834\\
%3.06343312263489	-0.947103494935163\\
3.06350989341736	-0.926252939508686\\
%3.14400072097778	-0.905152044353599\\
%3.24244160652161	-0.883589416032919\\
%3.24249835014343	-0.86120358646167\\
3.38695282936096	-0.838195026597077\\
%3.38704962730408	-0.815163810644208\\
%3.38712425231934	-0.792114061841858\\
%3.50973148345947	-0.768394603528122\\
3.50977296829224	-0.744067162938336\\
%3.50982351303101	-0.719182269784142\\
%3.60157866477966	-0.69465065386521\\
%3.61497135162354	-0.670158611636907\\
3.66265697479248	-0.645770915441517\\
%3.77121181488037	-0.622143894413057\\
%3.77127523422241	-0.598392725525173\\
%3.88388056755066	-0.574309109584647\\
3.88406581878662	-0.54977430246754\\
%4.01174230575562	-0.525087157338021\\
%4.01174683570862	-0.500192103786048\\
%4.01175088882446	-0.475722514410791\\
4.10342116355896	-0.452059421023932\\
%4.10781664848328	-0.429036496495428\\
%4.22873516082764	-0.406312036221152\\
%4.22877926826477	-0.383862453565143\\
4.3162703037262	-0.360981536652616\\
%4.31628270149231	-0.337448800913535\\
%4.36952753067017	-0.314229108037125\\
%4.49391932487488	-0.291786537450207\\
4.4939248085022	-0.270326520661456\\
%4.50986928939819	-0.249470168285825\\
%4.60001916885376	-0.229126228075074\\
%4.6112982749939	-0.209131500014337\\
4.713267993927	-0.188690613316339\\
%4.71331377029419	-0.167940108320636\\
%4.84796900749207	-0.147817620037586\\
%4.84803886413574	-0.128720893175341\\
4.9620210647583	-0.110625038106264\\
%4.96202535629272	-0.0931403237786981\\
%4.96383280754089	-0.0761620091698205\\
%5.00998063087463	-0.0588760307009579\\
5.05557627677917	-0.0417917837212372\\
%5.10753936767578	-0.0239077112073574\\
%5.19693036079407	-0.00551631095413541\\
%5.24160213470459	0.0122409928235356\\
5.2747401714325	0.0292288875152735\\
%5.47130937576294	0.0451110921367217\\
%5.47131628990173	0.0595279898848275\\
%5.47132034301758	0.073269811568025\\
5.47196359634399	0.0866038403623861\\
%5.92306728363037	0.100041047549894\\
%5.92307133674622	0.113200935262739\\
%5.92308039665222	0.126013705170408\\
5.92308468818665	0.138163652708499\\
%5.92308826446533	0.14944430638343\\
%5.92309231758118	0.16013362569629\\
%5.92309613227844	0.170366283434305\\
5.92310948371887	0.180365951790463\\
%5.92311329841614	0.189839296399896\\
%5.99617094993591	0.198720188852803\\
%6.01983733177185	0.206759521849563\\
6.06683416366577	0.2140808651925\\
%6.13510556221008	0.220947884536145\\
%6.17488260269165	0.227547998571026\\
%6.34516496658325	0.233959062868962\\
6.34567971229553	0.240189644655402\\
%6.34579391479492	0.245975378350067\\
%6.48088450431824	0.251351098022951\\
%6.48123211860657	0.255882747639589\\
6.48173136711121	0.259741796990056\\
%6.55087299346924	0.262807697839435\\
%6.60183663368225	0.265247754873144\\
%6.61844868659973	0.267637065396684\\
6.6699878692627	0.270131037437295\\
%6.73549313545227	0.272515639494941\\
%6.90011711120605	0.274375889991916\\
%6.90014715194702	0.275523848239573\\
6.90017552375793	0.27589223216728\\
%6.96017570495605	0.275600746891996\\
%6.96023197174072	0.274725053594921\\
%7.02018661499023	0.273706262690212\\
7.08605260848999	0.272937150863072\\
%7.14995141029358	0.272305978316126\\
%7.2400710105896	0.271155677047773\\
%7.24012393951416	0.269472421272212\\
7.35665650367737	0.266949285741475\\
%7.35666365623474	0.264104399438793\\
%7.50300760269165	0.260714482715684\\
%7.50304217338562	0.257657599869617\\
7.50325675010681	0.254830716098695\\
%7.55728693008423	0.252244348255317\\
%7.58989186286926	0.248921087999634\\
%7.60993332862854	0.245472877749989\\
7.71798748970032	0.241678787085675\\
%7.71803326606751	0.237315204887864\\
%7.8084406375885	0.232726741722672\\
%7.8084451675415	0.227709831618085\\
7.86788339614868	0.222919185876435\\
%7.98911018371582	0.218388355758407\\
%7.98919291496277	0.213613449019476\\
%8.03272266387939	0.208850601227368\\
8.08047933578491	0.203738054825408\\
%8.12581033706665	0.198249581723985\\
%8.19591255187988	0.192144412406378\\
%8.21851797103882	0.185667931843909\\
8.32182474136352	0.178557437088102\\
%8.32182974815369	0.171900191201757\\
%8.37021417617798	0.165821013639288\\
%8.40591187477112	0.159848469217422\\
8.59654874801636	0.154083673068271\\
%8.59659929275513	0.148299176849378\\
%8.59668583869934	0.141908961020363\\
%8.75598998069763	0.135137026066104\\
8.75599665641785	0.1279432065935\\
%8.75600190162659	0.120344302716148\\
%8.75602097511291	0.11294289396551\\
%9.02396960258484	0.105143592868217\\
9.02399797439575	0.0980717480999829\\
%9.02402396202087	0.09149910276642\\
%9.02404947280884	0.0850260268239253\\
%9.02408738136291	0.0786315515297247\\
9.34571809768677	0.0721653575163828\\
%9.34574980735779	0.0654071500521241\\
%9.34578461647034	0.0584434284581903\\
%9.34599132537842	0.0515102609449514\\
9.34602160453796	0.0445143288358167\\
%9.34606547355652	0.0378496312108609\\
%9.55178208351135	0.0316379115994323\\
%9.55178709030151	0.0257825491900121\\
9.55179281234741	0.0200467243764706\\
%9.55179829597473	0.0141988140949252\\
%9.61852855682373	0.00825668276070246\\
%9.61856646537781	0.00217741170013142\\
9.6634735584259	-0.0038410094879282\\
%9.70841760635376	-0.00971407401482338\\
%9.81058926582336	-0.015608953905712\\
%9.81059379577637	-0.0210682981663695\\
9.91963858604431	-0.0261775207382016\\
%9.91964693069458	-0.0314276026138032\\
%9.97360939979553	-0.0366813473133334\\
%10.0412783145905	-0.0419071814741301\\
10.1883935451508	-0.0472612147434575\\
%10.1884243011475	-0.053133076330937\\
%10.1884679317474	-0.0591416626344881\\
%10.2828607082367	-0.0649252306658354\\
10.2828685760498	-0.0703041786038057\\
%10.3878936290741	-0.0754842184053359\\
%10.3879322528839	-0.0803223073424988\\
%10.477405500412	-0.0852288245237105\\
10.4774195671082	-0.0901121398579789\\
%10.5185296058655	-0.0950767111512505\\
%10.560958814621	-0.100043860571077\\
%10.6081099033356	-0.104956681653107\\
10.6624338150024	-0.11005356317969\\
%10.7120904445648	-0.114819141703094\\
%10.8078364849091	-0.119581506800955\\
%10.8310131549835	-0.124061819599774\\
10.86188621521	-0.128428665978959\\
%10.9186720371246	-0.132726655341173\\
%10.9605512142181	-0.136996827914402\\
%11.0521792888641	-0.141205568271744\\
11.0872270584106	-0.145473686914045\\
%11.1295003414154	-0.14950554220394\\
%11.1574806690216	-0.153667702261032\\
%11.2049316883087	-0.157614147763638\\
11.264439535141	-0.16138445827994\\
%11.3169879436493	-0.165150292514625\\
%11.3666297912598	-0.16888040719283\\
%11.4129797935486	-0.172625198185813\\
11.487088394165	-0.176277749932297\\
%11.5034069538116	-0.179812138647947\\
%11.5847613334656	-0.183155409429048\\
%11.6668142795563	-0.18674466899347\\
11.6668383598328	-0.190128643130095\\
%11.756715965271	-0.193657400016093\\
%11.7568525791168	-0.197124620617558\\
%11.8283314228058	-0.200465171712949\\
11.861048412323	-0.203653901368739\\
%11.9123050689697	-0.206693464466866\\
%11.9995982170105	-0.209934162777586\\
%12.0273801803589	-0.213005201091903\\
12.0843092918396	-0.215934443401125\\
%12.1415590763092	-0.219170542016911\\
%12.1802153110504	-0.22233697821008\\
%12.2485608577728	-0.225296409800762\\
12.2750324726105	-0.228164077474332\\
%12.3711525917053	-0.230920942856983\\
%12.3711673736572	-0.23378453605477\\
%12.584521484375	-0.236517970376099\\
12.5848187923431	-0.239134875688706\\
%12.5848257064819	-0.242007971600003\\
%12.5848338127136	-0.2447886104585\\
%12.6255592823029	-0.247491059915632\\
12.6870493412018	-0.250103797815683\\
%12.7069229602814	-0.252584422186004\\
%12.7557956695557	-0.255206570592549\\
%12.8823637485504	-0.25773808934045\\
12.882368516922	-0.260144120013138\\
%12.9974901199341	-0.262961875744026\\
%12.9975285053253	-0.265635795439266\\
%13.0484261035919	-0.268300538990729\\
13.0905348777771	-0.270864955766996\\
%13.1565808773041	-0.273340606179829\\
%13.2565376281738	-0.275715210405025\\
%13.2565733909607	-0.277991061650823\\
13.2757715702057	-0.280132398773901\\
%13.3401882171631	-0.282442723020353\\
%13.4026224136353	-0.284647963444172\\
%13.4468314170837	-0.286760778147567\\
13.4851455211639	-0.289196568963035\\
%13.5162672519684	-0.291526889690374\\
%13.5994228839874	-0.293748550336375\\
%13.6809041023254	-0.295718254949463\\
13.6809529781342	-0.297574389485618\\
%13.760929775238	-0.299661620107969\\
%13.7609891414642	-0.301660867681449\\
%13.8658623218536	-0.303578567205193\\
13.8659095287323	-0.305684701532016\\
%13.9651402950287	-0.307694854523817\\
%13.9651853561401	-0.309516796604441\\
%14.013670873642	-0.311247378223072\\
14.0695306777954	-0.312860210818087\\
%14.1660151004791	-0.31448907136749\\
%14.1674413204193	-0.316033766299086\\
%14.3973986625671	-0.317940802345646\\
14.3974039077759	-0.319786955319387\\
%14.397408914566	-0.321678863575471\\
%14.3974151134491	-0.323491485592996\\
%14.4515306472778	-0.325217604286678\\
14.4809290885925	-0.326762134734082\\
%14.507790517807	-0.328206523222093\\
%14.6076409339905	-0.329763413214479\\
%14.6330356121063	-0.331243470539505\\
14.7221893787384	-0.332954008922414\\
%14.7222127437592	-0.334604643507916\\
%14.8223895549774	-0.336194738098584\\
%14.8226484775543	-0.337745045130745\\
14.8592681407928	-0.339213141938515\\
%14.9488486766815	-0.340598795469049\\
%14.9796115875244	-0.341904568732772\\
%15.0424162864685	-0.343126455914032\\
15.0784852027893	-0.344285702671471\\
%15.1091608524323	-0.345707900552142\\
%15.1938590526581	-0.347080457114146\\
%15.2520548820496	-0.348586637030422\\
15.3033129692078	-0.350014225115913\\
%15.3873831748962	-0.351365222628687\\
%15.389208984375	-0.35257500733897\\
%15.5462734222412	-0.353692089308836\\
15.5463118076324	-0.354727733787669\\
%15.5463530540466	-0.355815649743079\\
%15.6087738990784	-0.356856605419665\\
%15.6088122844696	-0.357868015049547\\
15.8284608840942	-0.359233889089097\\
%15.82850689888	-0.360532810360837\\
%15.8285231113434	-0.361756633994247\\
%15.8290633678436	-0.362896518957527\\
15.9087426185608	-0.363929899678183\\
%16.0841259479523	-0.36512075187369\\
%16.0841304779053	-0.366259870975156\\
%16.0841798305511	-0.367363428706952\\
16.0844277858734	-0.368744792095768\\
%16.1285621643066	-0.370059958990694\\
%16.1920978546143	-0.371306908698216\\
%16.2366220474243	-0.372411636949366\\
16.2680780410767	-0.373417264139462\\
%16.3074433326721	-0.374336729178168\\
%16.3852943897247	-0.375268842999404\\
%16.427689743042	-0.37613329446846\\
16.4832610607147	-0.376971639194807\\
%16.5667356967926	-0.378090938731219\\
%16.5667421340942	-0.379174820431729\\
%16.6510304927826	-0.380323693624518\\
16.6845106601715	-0.381405639931643\\
%16.7282096862793	-0.382425558163376\\
%16.7647928714752	-0.383357545967122\\
%16.8415979862213	-0.384212059985764\\
16.9638578414917	-0.384987066657104\\
%16.9638618946075	-0.385710191346199\\
%16.963866186142	-0.386493955405733\\
%17.0969151973724	-0.387571095515106\\
17.0969202041626	-0.388593230039895\\
%17.1762506484985	-0.38954713119864\\
%17.1763615131378	-0.390423939837092\\
%17.2229020118713	-0.391253995943515\\
17.3326086521149	-0.392009401045797\\
%17.3326153278351	-0.392653860855894\\
%17.473627281189	-0.393235491397938\\
%17.4736518383026	-0.393962036712296\\
17.4736754417419	-0.394667442689055\\
%17.5113889694214	-0.395765574483811\\
%17.5604962825775	-0.396813278479939\\
%17.6228019714355	-0.397732442214092\\
17.6881763458252	-0.398584166283683\\
%17.8001989841461	-0.399391115476199\\
%17.8002123355865	-0.400088690858553\\
%17.8233386993408	-0.40084525852599\\
17.9228260040283	-0.401544418285965\\
%17.9228312492371	-0.402515355467443\\
%18.0239114284515	-0.403469739596105\\
%18.023930978775	-0.404478342174166\\
18.2090794563293	-0.405435616584903\\
%18.2091359615326	-0.406321987886926\\
%18.209382724762	-0.4071530550031\\
%18.2096747875214	-0.407858984349417\\
18.3486201286316	-0.408499159262021\\
%18.3489136219025	-0.409268462504137\\
%18.4253313064575	-0.409984628479798\\
%18.4253766059875	-0.410648089044621\\
18.5247117996216	-0.411370603704199\\
%18.5248689174652	-0.412017645238304\\
%18.5989043235779	-0.412612076931111\\
%18.6113500118256	-0.413250386842307\\
18.6828867912292	-0.413795678604515\\
%18.7189335346222	-0.414299255930199\\
%18.7590083599091	-0.414776915344682\\
%18.8376090049744	-0.415528533509477\\
18.8578114032745	-0.416258438943668\\
%18.9627770900726	-0.416930858915197\\
%18.9628245353699	-0.41771938104965\\
%19.0279707431793	-0.418406572968244\\
19.0604125976562	-0.419029080884545\\
%19.1335758686066	-0.419560388882633\\
%19.2351016521454	-0.419991569416261\\
%19.2352606773376	-0.420388476046597\\
19.5428084850311	-0.420773596816794\\
%19.5428378105164	-0.421611628367426\\
%19.5428936004639	-0.422435230482456\\
%19.5429250717163	-0.423214562456927\\
19.5429551124573	-0.423997454545233\\
%19.5429982662201	-0.424705505955757\\
%19.9834553718567	-0.425326564601438\\
%19.9834630012512	-0.425826943018256\\
19.9834668159485	-0.426245211134162\\
%19.9834703922272	-0.426574319957782\\
%19.9834739685059	-0.427140980139326\\
%19.983477306366	-0.427927920771145\\
19.9834856510162	-0.428694521336915\\
%19.9834892272949	-0.429435614896644\\
%19.9834928035736	-0.430114621874228\\
%20.048566532135	-0.430706424076302\\
20.1035956859589	-0.431294878418585\\
%20.1756994247437	-0.431761302106727\\
%20.1757237434387	-0.43218050954124\\
%20.2920464992523	-0.432764631710764\\
20.2920505523682	-0.433338906114939\\
%20.3619036197662	-0.434094100520863\\
%20.4108409404755	-0.43482197334548\\
%20.4923514842987	-0.435473451796256\\
20.4923595905304	-0.436010621031546\\
%20.5578968048096	-0.43647967604002\\
%20.6461452960968	-0.436836277207099\\
%20.6461510181427	-0.43731331304148\\
20.7409729480743	-0.437778603489035\\
%20.775611114502	-0.438503250117757\\
%20.8289811134338	-0.439225567456675\\
%20.855291557312	-0.439928604682141\\
20.8881897449493	-0.440604173117704\\
%20.9274053096771	-0.44122835796756\\
%20.9642960548401	-0.441751215403078\\
%21.0089377880096	-0.442194943005662\\
21.0846263885498	-0.442543033552285\\
%21.1657716751099	-0.442880342427857\\
%21.1659259319305	-0.443164602126431\\
%21.2478870868683	-0.443362867575718\\
21.287251663208	-0.443552211019874\\
%21.3467800140381	-0.443759672265893\\
%21.3878879070282	-0.444204977479764\\
%21.4673904895782	-0.444608509626132\\
21.4674722671509	-0.445085100108382\\
%21.7880739688873	-0.445504522794768\\
%21.7880997180939	-0.445826350876099\\
%21.7881242752075	-0.446210566782096\\
21.788148355484	-0.446572306949649\\
%21.788175535202	-0.446945439586766\\
%21.7882205963135	-0.447464019505873\\
%21.9344484329224	-0.447942242274298\\
21.9345077991486	-0.448369782577867\\
%21.9346115112305	-0.448750029402002\\
%22.1828016757965	-0.449046928056558\\
%22.1828560352325	-0.449343862615316\\
22.1828851222992	-0.449597978017791\\
%22.1829120635986	-0.450010275966354\\
%22.1829561710358	-0.450435466674659\\
%22.2871834754944	-0.450859202851742\\
22.2872302055359	-0.451244600080886\\
%22.3290242671967	-0.451707766745126\\
%22.3717877388	-0.452064975108859\\
%22.4189037799835	-0.452333441364361\\
22.508477640152	-0.452527951025958\\
%22.579109621048	-0.452588686566144\\
%22.5791170120239	-0.452620783026648\\
%22.6909491539001	-0.453036848125222\\
22.6910020828247	-0.453456885266676\\
%23.0378574848175	-0.453897205448976\\
%23.0378658294678	-0.454383706785915\\
%23.0378715515137	-0.45482401179637\\
23.0378772735596	-0.455179408625776\\
%23.0378908634186	-0.455443515085562\\
%23.0379697799683	-0.455599692893526\\
%23.0379788398743	-0.455627654109872\\
23.1805610179901	-0.455800335941731\\
%23.1805998802185	-0.455957651406476\\
%23.2553226470947	-0.456448398393562\\
%23.3742901802063	-0.456936892436044\\
23.3742956638336	-0.457362476503309\\
%23.4069616317749	-0.457731516245889\\
%23.436697435379	-0.457977211084156\\
%23.4717108726501	-0.458166049294002\\
23.4726278305054	-0.458318246874031\\
%23.5186371326447	-0.458343315795162\\
%23.5845655918121	-0.458338678367189\\
%23.6520792961121	-0.458564033536991\\
23.7337271690369	-0.458791454647878\\
%23.7337564945221	-0.459241631625835\\
%23.7942924022675	-0.459673674316743\\
%23.8527013778687	-0.460059470929659\\
23.8903464794159	-0.460294582698307\\
%23.9444241046906	-0.460450388056536\\
%24.0006498813629	-0.460561397862698\\
%24.0766436576843	-0.460742710729317\\
24.0766481876373	-0.460928233248145\\
%24.1547135829926	-0.461127489176469\\
%24.1547355175018	-0.461542117641738\\
%24.2052554607391	-0.461932058489282\\
24.3249959468842	-0.462230359635424\\
%24.3469130516052	-0.462444040260271\\
%24.3948239803314	-0.462575150730496\\
%24.4395904064178	-0.462703856586332\\
24.4598485946655	-0.462752624770188\\
%24.5525354862213	-0.462782035337716\\
%24.5862226009369	-0.462835044282958\\
%24.6285922050476	-0.463194044524668\\
24.8560425758362	-0.463543188615489\\
%24.8560881137848	-0.463873585458396\\
%24.8566789150238	-0.464125007587423\\
%24.8570489406586	-0.464319168807885\\
24.9618987560272	-0.464451642075653\\
%24.9619016170502	-0.464579476464337\\
%24.9667996883392	-0.464700008152159\\
%25.1297199249268	-0.46502697539591\\
25.129741859436	-0.465346922287866\\
%25.1297678470612	-0.465645146707828\\
%25.1699120521545	-0.465894675222092\\
%25.3255557537079	-0.466063957427787\\
25.3258833408356	-0.466172311879216\\
%25.4633335590363	-0.466233700425894\\
%25.4633907794952	-0.466430033060877\\
%25.4634375095367	-0.466633090117744\\
26.2949263572693	-0.466831261543669\\
%26.2949594974518	-0.467010161093334\\
%26.2949900150299	-0.467297165080248\\
%26.2950171947479	-0.467525617869681\\
26.2950567722321	-0.467688482171642\\
%26.2950622558594	-0.467929988918376\\
%26.2950939655304	-0.468132267894077\\
%26.2951228141785	-0.468305494380495\\
26.2951516628265	-0.468533747775567\\
%26.2951850414276	-0.46872929756826\\
%26.2952184200287	-0.468945662639751\\
%26.2952496528625	-0.469116644238982\\
26.2952785015106	-0.469284616042161\\
%26.2953056812286	-0.469395221987821\\
%26.2953350067139	-0.469460036778822\\
%26.2953686237335	-0.469647196488397\\
26.2954256057739	-0.46983349212694\\
%26.5964006900787	-0.470019274949852\\
%26.5964049816132	-0.470292679541837\\
%26.5964097499847	-0.470532438374466\\
26.5964138031006	-0.470736731390391\\
%26.5964183330536	-0.470894788348785\\
%26.5964228630066	-0.471003588200164\\
%26.7749821662903	-0.471052215441905\\
26.774987411499	-0.471103066095637\\
%26.7755240917206	-0.471108678748964\\
%26.8470217704773	-0.471284862610263\\
%26.8470828056335	-0.471465687151914\\
26.89402384758	-0.471654857373098\\
%26.927468252182	-0.471941283833536\\
%27.2203940868378	-0.472195365069885\\
%27.2204248428345	-0.472415457181047\\
27.2204520225525	-0.472559238446287\\
%27.2204794406891	-0.472615132317538\\
%27.2205059051514	-0.472603030010326\\
%27.2205748081207	-0.472592855483516\\
28.8658260822296	-0.472583498483786\\
%28.8658315658569	-0.472863061176602\\
%28.8668162345886	-0.473165975659846\\
%28.866832447052	-0.473436877024458\\
28.8669096946716	-0.473665538633009\\
%28.8669151782989	-0.473857906188432\\
%28.8669273376465	-0.473981340761682\\
%28.8669893264771	-0.474082036898405\\
28.8669952869415	-0.474196588103165\\
%28.8670451164246	-0.474324643731482\\
%28.8670749187469	-0.474471480078124\\
%28.8671030521393	-0.474694924252253\\
28.867130947113	-0.474910853710682\\
%28.8671600341797	-0.475111658806183\\
%28.8671998500824	-0.475297103920221\\
%28.8672048568726	-0.475452694132215\\
28.8672348976135	-0.475566269050928\\
%28.8672766208649	-0.475667634489055\\
%28.8672813892364	-0.475691071980195\\
%28.8674888134003	-0.475686987378687\\
28.8674966812134	-0.475601521435099\\
%28.8675477027893	-0.475596027989383\\
%28.867590379715	-0.475597520213029\\
%28.8676218509674	-0.475625199430294\\
28.8676533222198	-0.47586177457076\\
%28.8676936149597	-0.476124283583145\\
%28.8676976680756	-0.476384277626019\\
%28.8677289009094	-0.476629185603743\\
28.8677708625793	-0.476864362845516\\
%28.8677784919739	-0.477075275826838\\
%28.8678075790405	-0.477251561768298\\
%28.867866230011	-0.477335890248686\\
28.8679084300995	-0.477429468620958\\
%29.2079612731934	-0.477532854730121\\
%29.208006811142	-0.477644463453423\\
%29.2080163478851	-0.477767427968054\\
29.2080223083496	-0.477955635701806\\
%29.2080304145813	-0.478112269935433\\
%29.2080440044403	-0.478235902002547\\
%29.208407831192	-0.478322253987537\\
30.8635506153107	-0.47833232900684\\
%30.8635580062866	-0.478342724597949\\
%30.8635627746582	-0.478374835113833\\
%30.8635682582855	-0.478560296661055\\
30.8635725498199	-0.478762965583725\\
%30.8635766029358	-0.478960207598876\\
%30.8635820865631	-0.47906269297525\\
%30.8635863780975	-0.479165876836571\\
30.8635904312134	-0.479247282273191\\
%30.8635944843292	-0.479294987117733\\
%30.8635987758636	-0.479363033171294\\
%30.8636030673981	-0.479441011742145\\
30.8636075973511	-0.479583390893048\\
%30.8636118888855	-0.479771096707386\\
%30.8636159420013	-0.479996154861421\\
%30.8636209487915	-0.480292651818925\\
30.8636252403259	-0.480668164902987\\
%30.8636300086975	-0.4810286899085\\
%30.8636340618134	-0.48133127591537\\
%30.8636385917664	-0.481608831260306\\
30.8636426448822	-0.481871559551264\\
%30.8636474132538	-0.481719215318954\\
%30.863651227951	-0.481524429465834\\
%30.8636557579041	-0.481191472525689\\
30.8636600494385	-0.480803699550966\\
%30.8636657714844	-0.480571657491886\\
%30.8636707782745	-0.480218467413113\\
%30.863675069809	-0.479893207506231\\
30.8636786460876	-0.479961952955906\\
%30.8636826992035	-0.480091330498819\\
%31.2550806522369	-0.480283393189085\\
%31.2551092624664	-0.480626274022965\\
31.2551359653473	-0.481000057752029\\
%31.2553131103516	-0.481363098627059\\
%31.2553185939789	-0.481481027054118\\
%31.255323600769	-0.481559698890599\\
31.2553271770477	-0.481571516421448\\
%31.2553925037384	-0.481560081444643\\
%31.3403334140778	-0.481548708671831\\
%31.3403391361237	-0.481612436249272\\
31.5086352348328	-0.481702703990154\\
%31.5086521625519	-0.481855087468842\\
%31.5086788654327	-0.482061735221805\\
%31.6173405170441	-0.482279240896627\\
31.6173724651337	-0.482489126126596\\
%31.6975693225861	-0.482691547334213\\
%31.6975774288177	-0.482755732300108\\
%31.7641047954559	-0.482765495329192\\
31.7724740028381	-0.482768053763847\\
%32.082110118866	-0.482757157252799\\
%32.0831522464752	-0.482734657347113\\
%32.0832287788391	-0.482723835590537\\
32.0832550048828	-0.482751746658693\\
%32.0832790851593	-0.482804213469102\\
%32.0845431804657	-0.482880926150139\\
%32.4165415287018	-0.482988857551346\\
32.4165484428406	-0.483112610352379\\
%32.416555595398	-0.48323986267706\\
%32.4167065143585	-0.483370822372644\\
%32.4169623374939	-0.483481104849945\\
32.4169682979584	-0.483574049926037\\
%32.6960920810699	-0.483661402911023\\
%32.6961004257202	-0.483728116458833\\
%32.6961133003235	-0.483710817829643\\
32.6961235523224	-0.483709492327232\\
%32.696128320694	-0.483725183550735\\
%32.6961371421814	-0.483777093685868\\
%32.8045789718628	-0.483900645071492\\
32.8046054363251	-0.484045638191535\\
%32.8438534259796	-0.484184319944838\\
%32.9417709827423	-0.484312157178344\\
%32.9419347763062	-0.484430130135605\\
33.1483494758606	-0.484431675016655\\
%33.1483871459961	-0.484442204009181\\
%33.1485037326813	-0.484469289360943\\
%33.1485960006714	-0.484521358737201\\
33.2275540351868	-0.484601423316123\\
%33.2620851516724	-0.484702003680775\\
%33.4662851810455	-0.48478508845699\\
%33.4662901878357	-0.484878642108663\\
33.4662944793701	-0.484975398964046\\
%33.4663323879242	-0.484997795789448\\
%33.6951984882355	-0.485007576202676\\
%33.6952318668366	-0.485014981961285\\
33.6952564239502	-0.485033873934666\\
%33.6952812194824	-0.48505549566293\\
%33.6953327178955	-0.485079225092719\\
%34.0244714736939	-0.485107654225459\\
34.0244769573212	-0.485133208964898\\
%34.0244836330414	-0.485165193406594\\
%34.0244886398315	-0.485191369474617\\
%34.0244934082031	-0.485220334409641\\
34.0244967460632	-0.485239979736756\\
%34.1406688213348	-0.48525857845509\\
%34.1412770271301	-0.485272508251338\\
%34.1414825439453	-0.48528500926256\\
34.3046380996704	-0.485297251498315\\
%34.3046442985535	-0.485308104091596\\
%34.3046488285065	-0.485322572405609\\
%34.3536202430725	-0.485341956094938\\
34.4075736522675	-0.485360869677663\\
%34.4325148582459	-0.485384243380318\\
%34.5385291099548	-0.485405884654575\\
%34.538582277298	-0.485428886370819\\
34.5713927268982	-0.485449213510318\\
%34.7362963676453	-0.485465201346013\\
%34.7364422798157	-0.485483174709088\\
%34.7365619659424	-0.48549620105433\\
34.8030476093292	-0.485505841393463\\
%34.8534969806671	-0.485521021310296\\
%34.896230173111	-0.485535801866106\\
%34.9908022403717	-0.485563950852891\\
34.9910139560699	-0.485596528043411\\
%35.1785239696503	-0.48562995993788\\
%35.178533744812	-0.485658843502277\\
%35.1785385131836	-0.485683277907535\\
35.1785497188568	-0.485702363956143\\
%35.2687682628632	-0.485717210311568\\
%35.3520075798035	-0.485729512883781\\
%35.3520118713379	-0.485753709157063\\
35.4015700340271	-0.485778889703386\\
%35.4856979370117	-0.485805998075641\\
%35.4857115268707	-0.485841418125368\\
%35.567130279541	-0.485872988770183\\
35.6499711990356	-0.485901752214833\\
%35.6500160217285	-0.485925910876688\\
%35.7923717021942	-0.485948240749384\\
%35.794321012497	-0.485968861938074\\
35.7943298339844	-0.486000622117032\\
%35.9174291610718	-0.486032674721411\\
%35.9174360752106	-0.486065498264843\\
%36.0053495883942	-0.486102740745213\\
36.0053569793701	-0.486134055839983\\
%36.0758685588837	-0.486162147476335\\
%36.0759078979492	-0.486185609741636\\
%36.1870767593384	-0.486206230451073\\
36.1871056079865	-0.486244958031573\\
%36.2769727230072	-0.486285053523197\\
%36.2769793987274	-0.486329671641652\\
%36.3538722515106	-0.48637330847578\\
36.4040152549744	-0.486412986282899\\
%36.4267644405365	-0.486445318862412\\
%36.4882075309753	-0.48647931648447\\
%36.5738670349121	-0.486509992689555\\
36.7500044822693	-0.486539413385927\\
%36.7500144958496	-0.486584546407862\\
%36.7500192642212	-0.486629711908989\\
%36.7500240325928	-0.486677696718736\\
36.9328815460205	-0.486725012143179\\
%36.9328860759735	-0.486767699766581\\
%36.9328906059265	-0.486803868449591\\
%36.9333383560181	-0.486831165052621\\
37.1490315914154	-0.486851512591784\\
%37.1490804672241	-0.486883015318789\\
%37.1497647285461	-0.486932432084121\\
%37.1498274326324	-0.48698273535334\\
37.3611752510071	-0.48703285339835\\
%37.361182641983	-0.48708701557446\\
%37.3623020172119	-0.487138087873898\\
%37.3623060703278	-0.487183854682126\\
37.4374818325043	-0.487220849735113\\
%37.43752617836	-0.487250744964832\\
%37.6287002086639	-0.487285783633197\\
%37.6287106990814	-0.48731941556478\\
37.6287152290344	-0.487352583812918\\
%37.628727388382	-0.487405054580783\\
%37.8067099571228	-0.48745608928682\\
%37.8067221164703	-0.487502577298297\\
37.8067826747894	-0.487548800408177\\
%37.8460576057434	-0.487583120628102\\
%37.9141819000244	-0.48761357442541\\
%37.9315175533295	-0.487654759944018\\
38.0237917423248	-0.487694108575841\\
%38.0862652778626	-0.48774864390904\\
%38.086269569397	-0.487801991427887\\
%38.1961585998535	-0.487854995868865\\
38.196178150177	-0.487905109712582\\
%38.3084940433502	-0.487945754551792\\
%38.3085419654846	-0.487987098753678\\
%38.378288936615	-0.488026600160939\\
38.393293094635	-0.488064714864414\\
%38.4296378612518	-0.488119292514927\\
%38.5430137634277	-0.48817329776973\\
%38.5430688381195	-0.488226554056855\\
38.6240517616272	-0.488272634967146\\
%38.6530174732208	-0.488311146007703\\
%38.7179064273834	-0.488344196495099\\
%38.7788311958313	-0.488382966979934\\
38.7788407325745	-0.488419896321705\\
%38.8368362903595	-0.488457121880749\\
%38.8831307411194	-0.488518637726951\\
%38.919997882843	-0.488577678350826\\
38.9796266078949	-0.488631049773019\\
%39.019500207901	-0.488686574087311\\
%39.0791270256043	-0.488717657563384\\
%39.1526405334473	-0.488731414217103\\
39.4971506118774	-0.488730096523792\\
%39.49718708992	-0.488738084725867\\
%39.4972180843353	-0.488746813990886\\
%39.4972497940064	-0.488825154132343\\
39.497278881073	-0.488903322903422\\
%39.4973082065582	-0.48897944089838\\
%39.4974827289581	-0.489051203116382\\
%39.65032787323	-0.489115087892518\\
39.6503335952759	-0.48916675593523\\
%39.6503650665283	-0.489204460981789\\
%39.7096879005432	-0.489245013946864\\
%39.7588674545288	-0.48928428090975\\
39.8301410198212	-0.489360977504802\\
%39.8644632816315	-0.489436895446769\\
%39.926452589035	-0.489514109131779\\
%39.9270951271057	-0.489584527729381\\
40.0534979820251	-0.489644991389291\\
%40.0535077571869	-0.489693686750048\\
%40.2688049793243	-0.489732653926907\\
%40.2689120292664	-0.489769309211908\\
40.268930387497	-0.489806384834107\\
%40.2689377784729	-0.489875338071446\\
%40.4292718887329	-0.489944553712991\\
%40.4293117046356	-0.490027421682623\\
40.4293787002564	-0.490107245371309\\
%40.4294297218323	-0.490174605405397\\
%40.5301465511322	-0.490232754998086\\
%40.564444732666	-0.490290635513645\\
40.7577771663666	-0.490348630967812\\
%40.7585477352142	-0.490398071027847\\
%40.7585613250732	-0.490474212219063\\
%40.7589544773102	-0.490552867571488\\
41.0156762123108	-0.490632356416743\\
%41.0157110214233	-0.490706654188909\\
%41.0157479763031	-0.490774580902421\\
%41.015776348114	-0.490834996952742\\
41.0157904148102	-0.49088325857689\\
%41.2113515853882	-0.490928037602522\\
%41.2113830566406	-0.490975114456088\\
%41.2114123821259	-0.491025913400428\\
41.2114569664001	-0.491076375149923\\
%41.3561818122864	-0.491124081055631\\
%41.356210899353	-0.491165965862319\\
%41.3562597751617	-0.491198926360184\\
41.4521789073944	-0.491214269827772\\
%41.4522032260895	-0.491219850569806\\
%41.5460445404053	-0.491216362683961\\
%41.6698898792267	-0.491213133248465\\
42.0554856777191	-0.491224104423619\\
%42.0562896251678	-0.491234615785316\\
%42.056302022934	-0.491243687059761\\
%42.0563089370728	-0.491249441123799\\
42.0563282489777	-0.491251076147115\\
%42.0563456535339	-0.491251076147115\\
%42.056353521347	-0.491251076147115\\
%42.0563606739044	-0.491251076147115\\
42.0563671112061	-0.491251076147115\\
%42.0563849925995	-0.491251076147115\\
%42.7761535167694	-0.491251076147115\\
%42.7761735439301	-0.491251076147115\\
43.0857395648956	-0.491251076147115\\
%43.0857460021973	-0.491251076147115\\
%43.0857576847076	-0.491251076147115\\
%43.2397584438324	-0.491251076147115\\
43.2397701263428	-0.491251076147115\\
%43.2397839546204	-0.491251076147115\\
%43.2644345283508	-0.491251076147115\\
%43.6193358421326	-0.491251076147115\\
43.6193403720856	-0.491251076147115\\
%43.6193439483643	-0.491251076147115\\
%43.6193475246429	-0.491251076147115\\
%43.7342502593994	-0.491251076147115\\
43.7342824459076	-0.491251076147115\\
%44.1866940975189	-0.491251076147115\\
%44.1866991043091	-0.491251076147115\\
%44.2737109184265	-0.491251076147115\\
44.2737159252167	-0.491251076147115\\
%44.6853835105896	-0.491251076147115\\
%44.6854300022125	-0.491251076147115\\
%44.8329545974731	-0.491251076147115\\
44.8329832077026	-0.491251076147115\\
%44.8330263614655	-0.491251076147115\\
%45.7841484069824	-0.491251076147115\\
%45.78420753479	-0.491251076147115\\
45.7842416286469	-0.491251076147115\\
%45.7842862129211	-0.491251076147115\\
%45.7843179225922	-0.491251076147115\\
%45.7843596458435	-0.491251076147115\\
45.7843942165375	-0.491251076147115\\
%45.7844330787659	-0.491251076147115\\
%45.7844631195068	-0.491251076147115\\
%45.7844976902008	-0.491251076147115\\
45.784534406662	-0.491251076147115\\
%45.7846402645111	-0.491251076147115\\
%46.0884601593018	-0.491251076147115\\
%46.0884670734406	-0.491251076147115\\
46.1257688522339	-0.491251076147115\\
%46.1662382602692	-0.491251076147115\\
%46.2153467655182	-0.491251076147115\\
%46.6415497779846	-0.491251076147115\\
46.7479912757874	-0.491251076147115\\
%46.7480005741119	-0.491251076147115\\
%46.9038605213165	-0.491251076147115\\
%46.9188446521759	-0.491251076147115\\
47.030423116684	-0.491251076147115\\
%47.8224229335785	-0.491251076147115\\
%47.9271270751953	-0.491251076147115\\
%48.2469987392426	-0.491251076147115\\
48.2470311641693	-0.491251076147115\\
%48.247061920166	-0.491251076147115\\
%48.6642553329468	-0.491251076147115\\
%48.9163457870483	-0.491251076147115\\
48.9163507938385	-0.491251076147115\\
%48.9173902988434	-0.491251076147115\\
%49.5241639137268	-0.491251076147115\\
%49.5241684436798	-0.491251076147115\\
49.525897693634	-0.491251076147115\\
%49.5259534835815	-0.491251076147115\\
%50.0419322967529	-0.491251076147115\\
%51.5745617866516	-0.491251076147115\\
51.5749794960022	-0.491251076147115\\
%51.5749883174896	-0.491251076147115\\
%51.8694738864899	-0.491251076147115\\
%51.869504404068	-0.491251076147115\\
51.8695332527161	-0.491251076147115\\
%51.869561624527	-0.491251076147115\\
%51.8696019172668	-0.491251076147115\\
%51.8696481704712	-0.491251076147115\\
53.5224811553955	-0.491251076147115\\
%53.5224897384644	-0.491251076147115\\
%53.5224947452545	-0.491251076147115\\
%53.5224995136261	-0.491251076147115\\
53.5225045204163	-0.491251076147115\\
%53.5225088119507	-0.491251076147115\\
%53.5225133419037	-0.491251076147115\\
%53.5225185871124	-0.491251076147115\\
53.52253074646	-0.491251076147115\\
%53.5225345611572	-0.491251076147115\\
%53.5225390911102	-0.491251076147115\\
%53.5225431442261	-0.491251076147115\\
53.8357407569885	-0.491251076147115\\
%53.8357474327087	-0.491251076147115\\
%53.8357564926147	-0.491251076147115\\
%53.8360311508179	-0.491251076147115\\
53.8361589431763	-0.491251076147115\\
%53.8363892555237	-0.491251076147115\\
%54.9593457698822	-0.491251076147115\\
%54.9593765258789	-0.491251076147115\\
54.9594101428986	-0.491251076147115\\
%54.9594618797302	-0.491251076147115\\
%54.9596836090088	-0.491251076147115\\
%54.9597208023071	-0.491251076147115\\
54.9597517967224	-0.491251076147115\\
%54.9597842216492	-0.491251076147115\\
%54.9599391937256	-0.491251076147115\\
%54.9599730491638	-0.491251076147115\\
54.9600042819977	-0.491251076147115\\
%54.9601048946381	-0.491251076147115\\
%55.7669550895691	-0.491251076147115\\
%55.766986322403	-0.491251076147115\\
55.767018032074	-0.491251076147115\\
%55.7670630931854	-0.491251076147115\\
%55.7670952796936	-0.491251076147115\\
%55.7671293735504	-0.491251076147115\\
55.767174911499	-0.491251076147115\\
%55.7672061443329	-0.491251076147115\\
%55.76724858284	-0.491251076147115\\
%55.7672833919525	-0.491251076147115\\
55.7673131942749	-0.491251076147115\\
%55.7673434734345	-0.491251076147115\\
%55.7673713684082	-0.491251076147115\\
%55.7674064159393	-0.491251076147115\\
55.7674335956574	-0.491251076147115\\
%55.7674622058868	-0.491251076147115\\
%55.7675101280212	-0.491251076147115\\
%56.4338857650757	-0.491251076147115\\
56.4339029312134	-0.491251076147115\\
%56.4339148521423	-0.491251076147115\\
%56.4339248657227	-0.491251076147115\\
%56.4339437007904	-0.491251076147115\\
56.4340469360352	-0.491251076147115\\
%56.4340550422668	-0.491251076147115\\
%56.4340698242188	-0.491251076147115\\
%56.4340769767761	-0.491251076147115\\
56.4340853214264	-0.491251076147115\\
%56.4340981960297	-0.491251076147115\\
%56.4342569828033	-0.491251076147115\\
%56.4342746257782	-0.491251076147115\\
56.8145236492157	-0.491251076147115\\
%56.9261416912079	-0.491251076147115\\
%57.0169128894806	-0.491251076147115\\
%57.0169181346893	-0.491251076147115\\
58.5536970615387	-0.491251076147115\\
%58.5537037372589	-0.491251076147115\\
%58.553709936142	-0.491251076147115\\
%58.5537163734436	-0.491251076147115\\
58.5537225723267	-0.491251076147115\\
%58.5537290096283	-0.491251076147115\\
%58.5537354469299	-0.491251076147115\\
%58.5537423610687	-0.491251076147115\\
58.5537490367889	-0.491251076147115\\
%58.5537545204163	-0.491251076147115\\
%58.5537600040436	-0.491251076147115\\
%58.5537657260895	-0.491251076147115\\
58.5537721633911	-0.491251076147115\\
%58.5537781238556	-0.491251076147115\\
%58.5537843227387	-0.491251076147115\\
%58.5537902832031	-0.491251076147115\\
58.5551464080811	-0.491251076147115\\
%58.5551602363586	-0.491251076147115\\
%58.5551681041718	-0.491251076147115\\
%58.5551752567291	-0.491251076147115\\
58.5551819324493	-0.491251076147115\\
%58.9172229290009	-0.491251076147115\\
%58.9172555923462	-0.491251076147115\\
%58.9172882556915	-0.491251076147115\\
58.9172951698303	-0.491251076147115\\
%58.9173254489899	-0.491251076147115\\
%58.9173306941986	-0.491251076147115\\
%58.9173583507538	-0.491251076147115\\
58.9174062728882	-0.491251076147115\\
%59.1581761360168	-0.491251076147115\\
%59.1582495689392	-0.491251076147115\\
%59.1583513736725	-0.491251076147115\\
59.1584672451019	-0.491251076147115\\
%59.3863837242126	-0.491251076147115\\
%59.3863975524902	-0.491251076147115\\
%59.3864140033722	-0.491251076147115\\
59.3864302158356	-0.491251076147115\\
%59.3864404678345	-0.491251076147115\\
%59.8614983081818	-0.491251076147115\\
%59.865093421936	-0.491251076147115\\
59.8651005744934	-0.491251076147115\\
%59.8657540798187	-0.491251076147115\\
%59.8657629013062	-0.491251076147115\\
%59.8659624576569	-0.491251076147115\\
59.8659700870514	-0.491251076147115\\
%59.8662604808807	-0.491251076147115\\
%59.8662702560425	-0.491251076147115\\
%60.6861924648285	-0.491251076147115\\
60.686212015152	-0.491251076147115\\
%60.6862289428711	-0.491251076147115\\
%60.6862382411957	-0.491251076147115\\
%60.686255645752	-0.491251076147115\\
60.6862613677979	-0.491251076147115\\
%60.6882712364197	-0.491251076147115\\
%60.6882829189301	-0.491251076147115\\
%60.6882910251617	-0.491251076147115\\
60.6882996082306	-0.491251076147115\\
%60.6883103370667	-0.491251076147115\\
%60.6883182048798	-0.491251076147115\\
%60.6883239269257	-0.491251076147115\\
60.6883284568787	-0.491251076147115\\
%60.6883360862732	-0.491251076147115\\
%60.6883420467377	-0.491251076147115\\
%60.6883496761322	-0.491251076147115\\
60.9728793621063	-0.491251076147115\\
%60.9912290096283	-0.491251076147115\\
%60.9912459373474	-0.491251076147115\\
%60.9912766933441	-0.491251076147115\\
60.9912983894348	-0.491251076147115\\
%60.9992479801178	-0.491251076147115\\
};
\addlegendentry{$\theta_{x_{p}^{2}}$}

\addplot [color=mycolor2, line width=1.4pt]
  table[row sep=crcr]{%
-2.3985894203186	0\\
%-2.34845380783081	0\\
%-2.29882984161377	0\\
%-2.24259977340698	0\\
-2.1981945514679	0\\
%-2.14788823127747	0\\
%-2.09815530776978	0\\
%-2.04817252159119	0\\
-1.99881820678711	0\\
%-1.94862990379333	0\\
%-1.87693266868591	0\\
%-1.84271411895752	0\\
-1.79806022644043	0\\
%-1.74720959663391	0\\
%-1.69869332313538	0\\
%-1.6482563495636	0\\
-1.59839563369751	0\\
%-1.548703956604	0\\
%-1.49876074790955	0\\
%-1.44885139465332	0\\
-1.3982711315155	0\\
%-1.34799294471741	0\\
%-1.29795436859131	0\\
%-1.2485387802124	0\\
-1.19843416213989	0\\
%-1.14853506088257	0\\
%-1.09835534095764	0\\
%-1.04829435348511	0\\
-0.998184490203857	0\\
%-0.948882627487182	0\\
%-0.898640441894531	0\\
%-0.847953605651855	0\\
-0.798753547668457	0\\
%-0.748502779006958	0\\
%-0.698466348648071	0\\
%-0.648822116851806	0\\
-0.597503709793091	0\\
%-0.548140096664429	0\\
%-0.498594331741333	0\\
%-0.4485755443573	0\\
-0.398829984664917	0\\
%-0.348826217651367	0\\
%-0.298781204223633	0\\
%-0.248498249053955	0\\
-0.197533178329468	0\\
%-0.147660779953003	0\\
%-0.0983634471893309	0\\
%-0.0481057643890379	0\\
0.0019752502441408	0\\
%0.0514363765716555	0\\
%0.113763284683228	0\\
%0.15691466331482	0\\
0.234845352172852	0\\
%0.308734607696533	0\\
%0.308903646469116	0\\
%0.360757303237915	0\\
0.415858459472656	0\\
%0.497472476959229	-0.601823436281478\\
%0.559248399734497	-0.944216614712929\\
%0.568476867675781	-1.1506839759997\\
0.604575109481812	-1.28617180254727\\
%0.723840665817261	-1.38076222288419\\
%0.723884773254395	-1.4479623324255\\
%0.853632164001465	-1.49564196386206\\
0.853860330581665	-1.52929431718911\\
%0.859521579742432	-1.55370718698202\\
%0.922651720046997	-1.57502659131239\\
%0.971385669708252	-1.59412649704939\\
1.02831001281738	-1.61156462067913\\
%1.07039852142334	-1.62659288067891\\
%1.11899919509888	-1.63829140169435\\
%1.16864533424377	-1.64816835561032\\
1.22722692489624	-1.65754854071929\\
%1.32403750419617	-1.66654562698159\\
%1.3240451335907	-1.67546286399306\\
%1.43628258705139	-1.68283819174053\\
1.4363646030426	-1.68783123211915\\
%1.6345018863678	-1.68998442064867\\
%1.63452954292297	-1.69091660825143\\
%1.63456315994263	-1.69218552732173\\
1.63461179733276	-1.69421630133343\\
%1.67959208488464	-1.69669166038489\\
%1.74344153404236	-1.69882310366734\\
%1.80315823554993	-1.69989979850629\\
1.8214485168457	-1.70076515927849\\
%1.86181755065918	-1.70170307274589\\
%1.95309824943542	-1.70299704770036\\
%1.98877353668213	-1.70401861908772\\
2.10328192710876	-1.70423630895152\\
%2.10334939956665	-1.7032699650631\\
%2.2043318271637	-1.70145718025242\\
%2.20438261032105	-1.70003433200372\\
2.20442314147949	-1.69952116038303\\
%2.31627173423767	-1.69923612983075\\
%2.31644148826599	-1.69876736444894\\
%2.35974855422974	-1.6973924788058\\
2.40763850212097	-1.69481908960597\\
%2.58187789916992	-1.69159875175728\\
%2.58190841674805	-1.68909023264086\\
%2.58211798667908	-1.68779972657194\\
2.96352286338806	-1.68666898527545\\
%2.96353478431702	-1.68495763665942\\
%2.96354002952576	-1.68239773657206\\
%2.96354813575745	-1.6793269854179\\
2.96359629631042	-1.6756110805818\\
%2.96360988616943	-1.67202189895215\\
%2.9636182308197	-1.66935748410378\\
%2.96363134384155	-1.66737820842445\\
3.06343312263489	-1.6654391541033\\
%3.06350989341736	-1.6631222404676\\
%3.14400072097778	-1.65952947066444\\
%3.24244160652161	-1.65439194432111\\
3.24249835014343	-1.64802714748566\\
%3.38695282936096	-1.64182364339604\\
%3.38704962730408	-1.6365345810209\\
%3.38712425231934	-1.63310074910623\\
3.50973148345947	-1.63032260548789\\
%3.50977296829224	-1.62761754040002\\
%3.50982351303101	-1.6235608010511\\
%3.60157866477966	-1.61739036574772\\
3.61497135162354	-1.60879875579531\\
3.66265697479248	-1.59874204973448\\
%3.77121181488037	-1.58889453369511\\
%3.77127523422241	-1.57972774456312\\
%3.88388056755066	-1.57052630489261\\
3.88406581878662	-1.56103082909203\\
%4.01174230575562	-1.55022737035597\\
%4.01174683570862	-1.53768794036796\\
%4.01175088882446	-1.52400255345765\\
4.10342116355896	-1.51089354644228\\
%4.10781664848328	-1.49839582905133\\
%4.22873516082764	-1.4863754399812\\
%4.22877926826477	-1.47476199861376\\
4.3162703037262	-1.46230505806489\\
%4.31628270149231	-1.44773866996638\\
%4.36952753067017	-1.43105431095319\\
%4.49391932487488	-1.41388331460803\\
4.4939248085022	-1.39772495403213\\
%4.50986928939819	-1.38251860216496\\
%4.60001916885376	-1.36803768145546\\
%4.6112982749939	-1.35385742124436\\
4.713267993927	-1.33844732948728\\
%4.71331377029419	-1.32087891977335\\
%4.84796900749207	-1.30261129247901\\
%4.84803886413574	-1.28481128806197\\
4.9620210647583	-1.26842658569058\\
%4.96202535629272	-1.25282993891597\\
%4.96383280754089	-1.23772614473182\\
%5.00998063087463	-1.22182855525261\\
5.05557627677917	-1.20529708630784\\
%5.10753936767578	-1.18683327532563\\
%5.19693036079407	-1.16535799626945\\
%5.24160213470459	-1.14396526940527\\
5.2747401714325	-1.12309332755376\\
%5.47130937576294	-1.10359917271398\\
%5.47131628990173	-1.08721619254993\\
%5.47132034301758	-1.07214066937195\\
5.47196359634399	-1.05769846726798\\
%5.92306728363037	-1.04218439502893\\
%5.92307133674622	-1.02598247925016\\
%5.92308039665222	-1.00892884052064\\
5.92308468818665	-0.992193548652722\\
%5.92308826446533	-0.976788091305934\\
%5.92309231758118	-0.961700622403441\\
%5.92309613227844	-0.946980625338256\\
5.92310948371887	-0.931830320464542\\
%5.92311329841614	-0.916975575720244\\
%5.99617094993591	-0.90265321864581\\
%6.01983733177185	-0.889229323587188\\
6.06683416366577	-0.876533521917281\\
%6.13510556221008	-0.864083558217771\\
%6.17488260269165	-0.851586037770176\\
%6.34516496658325	-0.838012699159663\\
6.34567971229553	-0.824162698365853\\
%6.34579391479492	-0.809899442794631\\
%6.48088450431824	-0.795644112123682\\
%6.48123211860657	-0.782307186622347\\
6.48173136711121	-0.77008356434726\\
%6.55087299346924	-0.758939646296767\\
%6.60183663368225	-0.748796363539441\\
%6.61844868659973	-0.738470459554264\\
6.6699878692627	-0.72789234703896\\
%6.73549313545227	-0.715289455767106\\
%6.90011711120605	-0.702666836803928\\
%6.90014715194702	-0.690378580700781\\
6.90017552375793	-0.679500779850059\\
%6.96017570495605	-0.6694985991262\\
%6.96023197174072	-0.660246762990937\\
%7.02018661499023	-0.65091182871538\\
7.08605260848999	-0.641621955189748\\
%7.14995141029358	-0.632459371198365\\
%7.2400710105896	-0.622992004129628\\
%7.24012393951416	-0.613875281901528\\
7.35665650367737	-0.60504347406868\\
%7.35666365623474	-0.59651954495348\\
%7.50300760269165	-0.588147875936556\\
%7.50304217338562	-0.580258212540684\\
7.50325675010681	-0.571664731430701\\
%7.55728693008423	-0.563458643058766\\
%7.58989186286926	-0.55573361475308\\
%7.60993332862854	-0.548336739480874\\
7.71798748970032	-0.541125395156541\\
%7.71803326606751	-0.534033044792977\\
%7.8084406375885	-0.527183562527171\\
%7.8084451675415	-0.519595725650106\\
7.86788339614868	-0.512213361993872\\
%7.98911018371582	-0.50510469816254\\
%7.98919291496277	-0.498350784131162\\
%8.03272266387939	-0.491905979564763\\
8.08047933578491	-0.485431917491496\\
%8.12581033706665	-0.479133548536993\\
%8.19591255187988	-0.472370544136538\\
%8.21851797103882	-0.465678649488382\\
8.32182474136352	-0.45898622001323\\
%8.32182974815369	-0.452672969904143\\
%8.37021417617798	-0.446766793590996\\
%8.40591187477112	-0.441288468091443\\
8.59654874801636	-0.436118935775539\\
%8.59659929275513	-0.431231289969674\\
%8.59668583869934	-0.426076942294458\\
%8.75598998069763	-0.420944789906116\\
8.75599665641785	-0.415832616275566\\
%8.75600190162659	-0.410645493981065\\
%8.75602097511291	-0.405306522764931\\
%9.02396960258484	-0.400423147277479\\
9.02399797439575	-0.396020576304181\\
%9.02402396202087	-0.391947652338331\\
%9.02404947280884	-0.387955640902646\\
%9.02408738136291	-0.384143787518724\\
9.34571809768677	-0.380230455930132\\
%9.34574980735779	-0.376305133617507\\
%9.34578461647034	-0.372355515779873\\
%9.34599132537842	-0.368377814546648\\
9.34602160453796	-0.364612103553782\\
%9.34606547355652	-0.36085653223904\\
%9.55178208351135	-0.357397107894258\\
%9.55178709030151	-0.354142781876135\\
9.55179281234741	-0.350959394404526\\
%9.55179829597473	-0.347899505693022\\
%9.61852855682373	-0.344836577523893\\
%9.61856646537781	-0.342145281385683\\
9.6634735584259	-0.339522248869343\\
%9.70841760635376	-0.33692632323914\\
%9.81058926582336	-0.334828149359282\\
%9.81059379577637	-0.33286255272219\\
9.91963858604431	-0.331074120464336\\
%9.91964693069458	-0.329232402065145\\
%9.97360939979553	-0.32753562116477\\
%10.0412783145905	-0.32596100420551\\
10.1883935451508	-0.324006350464288\\
%10.1884243011475	-0.32249867648602\\
%10.1884679317474	-0.320965559267506\\
%10.2828607082367	-0.319420571815044\\
10.2828685760498	-0.31795871637064\\
%10.3878936290741	-0.316938065520617\\
%10.3879322528839	-0.316075602410365\\
%10.477405500412	-0.315363667607016\\
10.4774195671082	-0.314717747333134\\
%10.5185296058655	-0.314107866017366\\
%10.560958814621	-0.313294218447481\\
%10.6081099033356	-0.312487727358757\\
10.6624338150024	-0.31187931780596\\
%10.7120904445648	-0.311262728775546\\
%10.8078364849091	-0.3106646854676\\
%10.8310131549835	-0.31016538971403\\
10.86188621521	-0.309782215092582\\
%10.9186720371246	-0.309467022646459\\
%10.9605512142181	-0.309278919756139\\
%11.0521792888641	-0.309071521220744\\
11.0872270584106	-0.308839660462695\\
%11.1295003414154	-0.308572462571306\\
%11.1574806690216	-0.308270033029203\\
%11.2049316883087	-0.308062975247026\\
11.264439535141	-0.307964914067725\\
%11.3169879436493	-0.308037996476742\\
%11.3666297912598	-0.308291064359537\\
%11.4129797935486	-0.308514234786458\\
11.487088394165	-0.308719909343722\\
%11.5034069538116	-0.308816930196372\\
%11.5847613334656	-0.308880327927795\\
%11.6668142795563	-0.308985191345073\\
11.6668383598328	-0.309187244172222\\
%11.756715965271	-0.309453623059312\\
%11.7568525791168	-0.30982185611083\\
%11.8283314228058	-0.310242193496038\\
11.861048412323	-0.310757678320101\\
%11.9123050689697	-0.311222002504678\\
%11.9995982170105	-0.31156458667806\\
%12.0273801803589	-0.311908435053624\\
12.0843092918396	-0.312306490740525\\
%12.1415590763092	-0.312676691760544\\
%12.1802153110504	-0.313091981020648\\
%12.2485608577728	-0.313686187124176\\
12.2750324726105	-0.31421739594532\\
%12.3711525917053	-0.314694881937385\\
%12.3711673736572	-0.315210841831799\\
%12.584521484375	-0.315751119785887\\
12.5848187923431	-0.316342842324275\\
%12.5848257064819	-0.316873028311647\\
%12.5848338127136	-0.317384125021363\\
%12.6255592823029	-0.318005076587156\\
12.6870493412018	-0.318554559515178\\
%12.7069229602814	-0.319065926763528\\
%12.7557956695557	-0.31953227766428\\
%12.8823637485504	-0.320003113041025\\
12.882368516922	-0.320499876627991\\
%12.9974901199341	-0.320919730336527\\
%12.9975285053253	-0.321345536349327\\
%13.0484261035919	-0.321836170606218\\
13.0905348777771	-0.322256362253654\\
%13.1565808773041	-0.322772170040309\\
%13.2565376281738	-0.323262264386926\\
%13.2565733909607	-0.32376316425831\\
13.2757715702057	-0.324249078320065\\
%13.3401882171631	-0.324738886138789\\
%13.4026224136353	-0.325257210937309\\
%13.4468314170837	-0.325791027861442\\
13.4851455211639	-0.326249480031095\\
%13.5162672519684	-0.326648946603086\\
%13.5994228839874	-0.327002334409599\\
%13.6809041023254	-0.327713608452825\\
13.6809529781342	-0.328420986031119\\
%13.760929775238	-0.329127406963039\\
%13.7609891414642	-0.329839324255261\\
%13.8658623218536	-0.330545381320405\\
13.8659095287323	-0.331251475435636\\
%13.9651402950287	-0.331905796906256\\
%13.9651853561401	-0.332714339357835\\
%14.013670873642	-0.333458993360864\\
14.0695306777954	-0.334191881228492\\
%14.1660151004791	-0.335032224680958\\
%14.1674413204193	-0.33586506600642\\
%14.3973986625671	-0.336583948336113\\
14.3974039077759	-0.337323345161906\\
%14.397408914566	-0.338106449347151\\
%14.3974151134491	-0.338829886288067\\
%14.4515306472778	-0.339470154465062\\
14.4809290885925	-0.340278636945925\\
%14.507790517807	-0.341055712291563\\
%14.6076409339905	-0.341932796721778\\
%14.6330356121063	-0.342809562545597\\
14.7221893787384	-0.343625120248475\\
%14.7222127437592	-0.344421841862811\\
%14.8223895549774	-0.345149507714723\\
%14.8226484775543	-0.345892380335556\\
14.8592681407928	-0.346552981347259\\
%14.9488486766815	-0.34714364374139\\
%14.9796115875244	-0.347884245588631\\
%15.0424162864685	-0.348615466265954\\
15.0784852027893	-0.349339301571234\\
%15.1091608524323	-0.350153423611403\\
%15.1938590526581	-0.350918571502007\\
%15.2520548820496	-0.351693568862231\\
15.3033129692078	-0.352401155086881\\
%15.3873831748962	-0.353051474349769\\
%15.389208984375	-0.353765191318473\\
%15.5462734222412	-0.354433697705076\\
15.5463118076324	-0.355078981619016\\
%15.5463530540466	-0.355787227365795\\
%15.6087738990784	-0.356481171193707\\
%15.6088122844696	-0.357145032331431\\
15.8284608840942	-0.357896244309359\\
%15.82850689888	-0.358577616385134\\
%15.8285231113434	-0.359192694358313\\
%15.8290633678436	-0.359953401891573\\
15.9087426185608	-0.360685168402199\\
%16.0841259479523	-0.361462235423147\\
%16.0841304779053	-0.362198058340795\\
%16.0841798305511	-0.3628956255275\\
16.0844277858734	-0.363631295341833\\
%16.1285621643066	-0.364295190656605\\
%16.1920978546143	-0.364913615903077\\
%16.2366220474243	-0.365622234131195\\
16.2680780410767	-0.366279996820595\\
%16.3074433326721	-0.366897738703273\\
%16.3852943897247	-0.367609696330788\\
%16.427689743042	-0.368288100164648\\
16.4832610607147	-0.368932164790778\\
%16.5667356967926	-0.369662960934818\\
%16.5667421340942	-0.370342448589724\\
%16.6510304927826	-0.371141471041085\\
16.6845106601715	-0.371885244623343\\
%16.7282096862793	-0.372584377246653\\
%16.7647928714752	-0.373280822687008\\
%16.8415979862213	-0.373939763319441\\
16.9638578414917	-0.374565949575242\\
%16.9638618946075	-0.375160526629557\\
%16.963866186142	-0.375782204228514\\
%17.0969151973724	-0.376527419068552\\
17.0969202041626	-0.37721551228401\\
%17.1762506484985	-0.377847275890772\\
%17.1763615131378	-0.378433467127763\\
%17.2229020118713	-0.379067622957649\\
17.3326086521149	-0.379669211424286\\
%17.3326153278351	-0.380240976125677\\
%17.473627281189	-0.380782656160363\\
%17.4736518383026	-0.381377917821737\\
17.4736754417419	-0.381926527750537\\
%17.5113889694214	-0.38257265149754\\
%17.5604962825775	-0.383147423801859\\
%17.6228019714355	-0.383799400395063\\
17.6881763458252	-0.384354052870094\\
%17.8001989841461	-0.384961610423488\\
%17.8002123355865	-0.38550994698754\\
%17.8233386993408	-0.386175983919173\\
17.9228260040283	-0.386812848865119\\
%17.9228312492371	-0.387536548151644\\
%18.0239114284515	-0.388205071667969\\
%18.023930978775	-0.38883679216373\\
18.2090794563293	-0.389401443003763\\
%18.2091359615326	-0.389911009155952\\
%18.209382724762	-0.39037257195443\\
%18.2096747875214	-0.390847452670016\\
18.3486201286316	-0.391274660436537\\
%18.3489136219025	-0.391882753948423\\
%18.4253313064575	-0.392467456238677\\
%18.4253766059875	-0.393014531387578\\
18.5247117996216	-0.393588423858162\\
%18.5248689174652	-0.394115693112212\\
%18.5989043235779	-0.394582984527794\\
%18.6113500118256	-0.395059980482245\\
18.6828867912292	-0.395477182924196\\
%18.7189335346222	-0.395840981796873\\
%18.7590083599091	-0.396157328881827\\
%18.8376090049744	-0.396552667233223\\
18.8578114032745	-0.396922621074609\\
%18.9627770900726	-0.397275768273886\\
%18.9628245353699	-0.397814614632395\\
%19.0279707431793	-0.398336620806717\\
19.0604125976562	-0.398835024476512\\
%19.1335758686066	-0.399361028471922\\
%19.2351016521454	-0.399845023305843\\
%19.2352606773376	-0.400273591046691\\
19.5428084850311	-0.400638461009734\\
%19.5428378105164	-0.401025053227229\\
%19.5428936004639	-0.401361924571113\\
%19.5429250717163	-0.401679742296224\\
19.5429551124573	-0.402130178451884\\
%19.5429982662201	-0.402556870046801\\
%19.9834553718567	-0.402961942007224\\
%19.9834630012512	-0.403468613977843\\
19.9834668159485	-0.403944330684325\\
%19.9834703922272	-0.404366249792815\\
%19.9834739685059	-0.40478646173657\\
%19.983477306366	-0.405208960241218\\
19.9834856510162	-0.405565041367034\\
%19.9834892272949	-0.406009224501325\\
%19.9834928035736	-0.406433125847158\\
%20.048566532135	-0.406834371572351\\
20.1035956859589	-0.407324804771207\\
%20.1756994247437	-0.407786556480218\\
%20.1757237434387	-0.408194735786921\\
%20.2920464992523	-0.408601565204847\\
20.2920505523682	-0.408949576912448\\
%20.3619036197662	-0.409313771994594\\
%20.4108409404755	-0.409613447062313\\
%20.4923514842987	-0.409872863878896\\
20.4923595905304	-0.410237482515175\\
%20.5578968048096	-0.410585606132184\\
%20.6461452960968	-0.41090243496965\\
%20.6461510181427	-0.4113769139299\\
20.7409729480743	-0.411793500372152\\
%20.775611114502	-0.412157738794853\\
%20.8289811134338	-0.412455566617524\\
%20.855291557312	-0.412691543539157\\
20.8881897449493	-0.412904160001204\\
%20.9274053096771	-0.413082797033031\\
%20.9642960548401	-0.413419896382855\\
%21.0089377880096	-0.413748533778346\\
21.0846263885498	-0.414063241118882\\
%21.1657716751099	-0.414607608789311\\
%21.1659259319305	-0.4151155532237\\
%21.2478870868683	-0.415596221500738\\
21.287251663208	-0.416022853672068\\
%21.3467800140381	-0.416367717164917\\
%21.3878879070282	-0.416663608377757\\
%21.4673904895782	-0.416910766649069\\
21.4674722671509	-0.417306044423023\\
%21.7880739688873	-0.417656595803969\\
%21.7880997180939	-0.417994459878738\\
%21.7881242752075	-0.418448855985853\\
21.788148355484	-0.418869320617519\\
%21.788175535202	-0.419233664369008\\
%21.7882205963135	-0.419585727508583\\
%21.9344484329224	-0.419893129539613\\
21.9345077991486	-0.420161533233719\\
%21.9346115112305	-0.420481724331434\\
%22.1828016757965	-0.42076522510739\\
%22.1828560352325	-0.421150212470977\\
22.1828851222992	-0.42148935893016\\
%22.1829120635986	-0.421925315992016\\
%22.1829561710358	-0.422311410686878\\
%22.2871834754944	-0.42267026653922\\
22.2872302055359	-0.422969628837528\\
%22.3290242671967	-0.423285835078631\\
%22.3717877388	-0.423559064998266\\
%22.4189037799835	-0.423793039540734\\
22.508477640152	-0.424004829374487\\
%22.579109621048	-0.424347649263619\\
%22.5791170120239	-0.424669546555698\\
%22.6909491539001	-0.425071601911704\\
22.6910020828247	-0.425448924399332\\
%23.0378574848175	-0.42577735646291\\
%23.0378658294678	-0.426113867788819\\
%23.0378715515137	-0.426399228297551\\
23.0378772735596	-0.426643851653669\\
%23.0378908634186	-0.426970281607159\\
%23.0379697799683	-0.427255767614543\\
%23.0379788398743	-0.427509600125271\\
23.1805610179901	-0.427867249576219\\
%23.1805998802185	-0.428179393563935\\
%23.2553226470947	-0.428544995048097\\
%23.3742901802063	-0.428891823362375\\
23.3742956638336	-0.429271510020515\\
% 23.4069616317749	-0.429617653766522\\
% 23.436697435379	-0.429945519750782\\
% 23.4717108726501	-0.430265403437982\\
23.4726278305054	-0.430535629120131\\
% 23.5186371326447	-0.430788365249644\\
% 23.5845655918121	-0.43100661457318\\
% 23.6520792961121	-0.431265014543193\\
23.7337271690369	-0.431503722609956\\
% 23.7337564945221	-0.431809179961388\\
% 23.7942924022675	-0.432099828056348\\
% 23.8527013778687	-0.432381305184734\\
23.8903464794159	-0.432713494840957\\
% 23.9444241046906	-0.433019749217512\\
% 24.0006498813629	-0.433306131990352\\
% 24.0766436576843	-0.43358618363323\\
24.0766481876373	-0.433817014696914\\
% 24.1547135829926	-0.434001561579019\\
% 24.1547355175018	-0.434221557532187\\
% 24.2052554607391	-0.434407126656486\\
24.3249959468842	-0.434701030766306\\
% 24.3469130516052	-0.434949159813758\\
% 24.3948239803314	-0.435172139446163\\
% 24.4395904064178	-0.435450021605939\\
24.4598485946655	-0.435705904804927\\
% 24.5525354862213	-0.43591596453096\\
% 24.5862226009369	-0.436079072781513\\
% 24.6285922050476	-0.436320304947851\\
24.8560425758362	-0.436541265394808\\
% 24.8560881137848	-0.436865335273513\\
% 24.8566789150238	-0.437158292685788\\
% 24.8570489406586	-0.437426433552776\\
24.9618987560272	-0.437665956577536\\
% 24.9619016170502	-0.43792868753933\\
% 24.9667996883392	-0.438155592620932\\
% 25.1297199249268	-0.438434237999978\\
25.129741859436	-0.438646695182656\\
% 25.1297678470612	-0.43889513952116\\
% 25.1699120521545	-0.439110698732634\\
% 25.3255557537079	-0.439416177242782\\
25.3258833408356	-0.439713595188985\\
% 25.4633335590363	-0.440007323280703\\
% 25.4633907794952	-0.440344465785076\\
% 25.4634375095367	-0.440655556911295\\
26.2949263572693	-0.440952219592628\\
%26.2949594974518	-0.441212591532313\\
%26.2949900150299	-0.441438770070436\\
%26.2950171947479	-0.441633531349762\\
26.2950567722321	-0.44188917781336\\
%26.2950622558594	-0.442207710210504\\
%26.2950939655304	-0.442508326127228\\
%26.2951228141785	-0.442769684573603\\
26.2951516628265	-0.443011296087537\\
%26.2951850414276	-0.443212713051476\\
%26.2952184200287	-0.443432437292845\\
%26.2952496528625	-0.443624297248565\\
26.2952785015106	-0.4438525544442\\
%26.2953056812286	-0.444067212676432\\
%26.2953350067139	-0.444255681890104\\
%26.2953686237335	-0.444497556515145\\
26.2954256057739	-0.44469096900524\\
%26.5964006900787	-0.444839219054673\\
%26.5964049816132	-0.445009772850543\\
%26.5964097499847	-0.445123073914052\\
26.5964138031006	-0.445334156276424\\
%26.5964183330536	-0.445517493881825\\
%26.5964228630066	-0.445672044206972\\
%26.7749821662903	-0.445834183064273\\
26.774987411499	-0.446093775428826\\
%26.7755240917206	-0.446347556357825\\
%26.8470217704773	-0.446641685015843\\
%26.8470828056335	-0.446905864950322\\
26.89402384758	-0.447124688768456\\
%26.927468252182	-0.447293989277288\\
%27.2203940868378	-0.447418370756292\\
%27.2204248428345	-0.447508051165963\\
27.2204520225525	-0.44766254132483\\
%27.2204794406891	-0.447800345408451\\
%27.2205059051514	-0.44795867630987\\
%27.2205748081207	-0.44826962817408\\
28.8658260822296	-0.44857452022821\\
%28.8658315658569	-0.448805991717007\\
%28.8668162345886	-0.448984738193072\\
%28.866832447052	-0.449136220367397\\
28.8669096946716	-0.449244504514128\\
%28.8669151782989	-0.449324836004831\\
%28.8669273376465	-0.449494342505846\\
%28.8669893264771	-0.449633213626498\\
28.8669952869415	-0.449741656715201\\
%28.8670451164246	-0.449822568354623\\
%28.8670749187469	-0.449861700799842\\
%28.8671030521393	-0.449940837216142\\
28.867130947113	-0.4499712140392\\
%28.8671600341797	-0.4499535925329\\
%28.8671998500824	-0.449897814059831\\
%28.8672048568726	-0.449765739039005\\
28.8672348976135	-0.44975682356141\\
%28.8672766208649	-0.449695150123475\\
%28.8672813892364	-0.449623960335632\\
%28.8674888134003	-0.449552667560862\\
28.8674966812134	-0.44952169442591\\
%28.8675477027893	-0.449800680429417\\
%28.867590379715	-0.450083633679672\\
%28.8676218509674	-0.450357081903878\\
28.8676533222198	-0.450632675238481\\
%28.8676936149597	-0.450852885045142\\
%28.8676976680756	-0.451049569150932\\
%28.8677289009094	-0.451204115212434\\
28.8677708625793	-0.451347072853537\\
%28.8677784919739	-0.451440357837029\\
% 28.8678075790405	-0.451507308203869\\
% 28.867866230011	-0.451583533755686\\
 28.8679084300995	-0.45164123512655\\
% 29.2079612731934	-0.451690400669456\\
% 29.208006811142	-0.45173351950608\\
% 29.2080163478851	-0.451752656394728\\
 29.2080223083496	-0.451955166507759\\
% 29.2080304145813	-0.452115004772599\\
% 29.2080440044403	-0.452244205985908\\
% 29.208407831192	-0.452336296214927\\
 30.8635506153107	-0.452463936461786\\
% 30.8635580062866	-0.452543571186396\\
% 30.8635627746582	-0.452585448936384\\
% 30.8635682582855	-0.452725301400607\\
 30.8635725498199	-0.45283433257349\\
% 30.8635766029358	-0.452935343342155\\
% 30.8635820865631	-0.453112430054309\\
% 30.8635863780975	-0.453273309756504\\
 30.8635904312134	-0.453390561385037\\
% 30.8635944843292	-0.453572877420342\\
% 30.8635987758636	-0.453684890654607\\
% 30.8636030673981	-0.453774227844981\\
 30.8636075973511	-0.453765586385369\\
% 30.8636118888855	-0.453694149704891\\
% 30.8636159420013	-0.453575234476622\\
% 30.8636209487915	-0.453369594213147\\
 30.8636252403259	-0.453244086196278\\
% 30.8636300086975	-0.453150559505909\\
% 30.8636340618134	-0.45306555108948\\
% 30.8636385917664	-0.452993704169852\\
 30.8636426448822	-0.4529554003734\\
% 30.8636474132538	-0.453349673161343\\
% 30.863651227951	-0.453757361165927\\
% 30.8636557579041	-0.454223645800573\\
 30.8636600494385	-0.454693406494048\\
% 30.8636657714844	-0.455102104509978\\
% 30.8636707782745	-0.455507121799222\\
% 30.863675069809	-0.455875923693439\\
 30.8636786460876	-0.456036855085866\\
% 30.8636826992035	-0.456134322086786\\
% 31.2550806522369	-0.456161742100896\\
% 31.2551092624664	-0.456209543978877\\
 31.2551359653473	-0.456245625732842\\
% 31.2553131103516	-0.45628534671703\\
% 31.2553185939789	-0.456513412154864\\
% 31.255323600769	-0.456731568496755\\
 31.2553271770477	-0.456966816138516\\
% 31.2553925037384	-0.457187130607334\\
% 31.3403334140778	-0.457356301156249\\
% 31.3403391361237	-0.457473060922792\\
 31.5086352348328	-0.457553657364281\\
% 31.5086521625519	-0.457555945249691\\
% 31.5086788654327	-0.457622336963972\\
% 31.6173405170441	-0.457682645755286\\
 31.6173724651337	-0.45773883459721\\
% 31.6975693225861	-0.457796625257451\\
% 31.6975774288177	-0.458016973068514\\
% 31.7641047954559	-0.458206457814324\\
 31.7724740028381	-0.458387624520913\\
% 32.082110118866	-0.458553079590466\\
% 32.0831522464752	-0.458665873575089\\
% 32.0832287788391	-0.458750764199198\\
 32.0832550048828	-0.458759713724481\\
% 32.0832790851593	-0.458721184389866\\
% 32.0845431804657	-0.458661125739233\\
% 32.4165415287018	-0.458594472434045\\
 32.4165484428406	-0.458520409430525\\
% 32.416555595398	-0.458458057373777\\
% 32.4167065143585	-0.458615773035477\\
% 32.4169623374939	-0.458786117918516\\
 32.4169682979584	-0.458943294581269\\
% 32.6960920810699	-0.459092958684225\\
% 32.6961004257202	-0.459210061801075\\
% 32.6961133003235	-0.459294245467079\\
 32.6961235523224	-0.459345274726772\\
% 32.696128320694	-0.459366964274675\\
% 32.6961371421814	-0.459327486273024\\
% 32.8045789718628	-0.459393820204038\\
 32.8046054363251	-0.459449871415739\\
% 32.8438534259796	-0.459507614756564\\
% 32.9417709827423	-0.459571907861232\\
% 32.9419347763062	-0.459621826718799\\
 33.1483494758606	-0.459832575394005\\
% 33.1483871459961	-0.460019901831798\\
% 33.1485037326813	-0.460169193149824\\
% 33.1485960006714	-0.460334218366318\\
 33.2275540351868	-0.460452102174713\\
% 33.2620851516724	-0.460537597525972\\
% 33.4662851810455	-0.460676560795097\\
% 33.4662901878357	-0.460783246680913\\
 33.4662944793701	-0.46086125089381\\
% 33.4663323879242	-0.461017846016549\\
% 33.6951984882355	-0.461207659319593\\
% 33.6952318668366	-0.461368815694238\\
 33.6952564239502	-0.4615436684033\\
% 33.6952812194824	-0.461712803403344\\
% 33.6953327178955	-0.461938457763836\\
% 34.0244714736939	-0.462165597192104\\
 34.0244769573212	-0.462395511587916\\
% 34.0244836330414	-0.462607708402107\\
% 34.0244886398315	-0.4627949227408\\
% 34.0244934082031	-0.462957617400464\\
 34.0244967460632	-0.463111894151606\\
% 34.1406688213348	-0.463246773486887\\
% 34.1412770271301	-0.463411111199969\\
% 34.1414825439453	-0.463565600269709\\
 34.3046380996704	-0.463718020419964\\
% 34.3046442985535	-0.46387295258393\\
% 34.3046488285065	-0.464063377164821\\
% 34.3536202430725	-0.464261903646689\\
 34.4075736522675	-0.464453023801516\\
% 34.4325148582459	-0.464620222489424\\
% 34.5385291099548	-0.46476058351903\\
% 34.538582277298	-0.46487706719571\\
 34.5713927268982	-0.464972928362471\\
% 34.7362963676453	-0.465042028172961\\
% 34.7364422798157	-0.465157028173506\\
% 34.7365619659424	-0.465274287283263\\
 34.8030476093292	-0.465399291533792\\
% 34.8534969806671	-0.465604562631531\\
% 34.896230173111	-0.465810765891518\\
% 34.9908022403717	-0.46599146554637\\
 34.9910139560699	-0.46613378713473\\
% 35.1785239696503	-0.466253102903336\\
% 35.178533744812	-0.466380387679024\\
% 35.1785385131836	-0.466509695665826\\
 35.1785497188568	-0.466687267745947\\
% 35.2687682628632	-0.466877446713967\\
% 35.3520075798035	-0.467072484218185\\
% 35.3520118713379	-0.467243903942929\\
 35.4015700340271	-0.467398690330604\\
% 35.4856979370117	-0.467528330650458\\
% 35.4857115268707	-0.4676104963748\\
% 35.567130279541	-0.467688047699248\\
 35.6499711990356	-0.467768694249452\\
% 35.6500160217285	-0.467908086836601\\
% 35.7923717021942	-0.46805150559984\\
% 35.794321012497	-0.468194842520637\\
 35.7943298339844	-0.468332482687417\\
% 35.9174291610718	-0.46845562075581\\
% 35.9174360752106	-0.468559476837072\\
% 36.0053495883942	-0.468630702607282\\
 36.0053569793701	-0.468699008598108\\
% 36.0758685588837	-0.468820883325039\\
% 36.0759078979492	-0.468954068299769\\
% 36.1870767593384	-0.469096316553278\\
 36.1871056079865	-0.469241752432143\\
% 36.2769727230072	-0.469370679917974\\
% 36.2769793987274	-0.46945712294348\\
% 36.3538722515106	-0.469520855917267\\
 36.4040152549744	-0.469610534372325\\
% 36.4267644405365	-0.469699723902458\\
% 36.4882075309753	-0.469834451627184\\
% 36.5738670349121	-0.469984768382204\\
 36.7500044822693	-0.47014401363205\\
% 36.7500144958496	-0.470293558268787\\
% 36.7500192642212	-0.470432820370132\\
% 36.7500240325928	-0.470518333001186\\
 36.9328815460205	-0.470580852638646\\
% 36.9328860759735	-0.470617226033118\\
% 36.9328906059265	-0.470652339585158\\
% 36.9333383560181	-0.47068435780135\\
 37.1490315914154	-0.470719633541758\\
% 37.1490804672241	-0.470836658092924\\
% 37.1497647285461	-0.470962676162413\\
% 37.1498274326324	-0.471083188854371\\
 37.3611752510071	-0.471191768528808\\
% 37.361182641983	-0.471266479697645\\
% 37.3623020172119	-0.471324065195567\\
% 37.3623060703278	-0.471393890583364\\
 37.4374818325043	-0.471463465900451\\
% 37.43752617836	-0.471534892916493\\
% 37.6287002086639	-0.471652516826033\\
% 37.6287106990814	-0.471777258159841\\
 37.6287152290344	-0.471898960501843\\
% 37.628727388382	-0.472004979063424\\
% 37.8067099571228	-0.472093019042088\\
% 37.8067221164703	-0.472164222571131\\
 37.8067826747894	-0.472212684497293\\
% 37.8460576057434	-0.472256432546613\\
% 37.9141819000244	-0.472308997336139\\
% 37.9315175533295	-0.472425225998682\\
 38.0237917423248	-0.472546667920371\\
% 38.0862652778626	-0.472655134625738\\
% 38.086269569397	-0.472752722731546\\
% 38.1961585998535	-0.472811263478031\\
 38.196178150177	-0.472888429348851\\
% 38.3084940433502	-0.472964951540433\\
% 38.3085419654846	-0.473066744990174\\
% 38.378288936615	-0.473174208474287\\
 38.393293094635	-0.473284060061127\\
% 38.4296378612518	-0.473383459246913\\
% 38.5430137634277	-0.473468055306206\\
% 38.5430688381195	-0.473524151143472\\
 38.6240517616272	-0.473557502528841\\
% 38.6530174732208	-0.473585581188084\\
% 38.7179064273834	-0.473614904010327\\
% 38.7788311958313	-0.473684135426216\\
 38.7788407325745	-0.473758948235554\\
% 38.8368362903595	-0.473832517418465\\
% 38.8831307411194	-0.4739215087119\\
% 38.919997882843	-0.473993322206013\\
 38.9796266078949	-0.474052366848269\\
% 39.019500207901	-0.474102241730769\\
% 39.0791270256043	-0.474130745252781\\
% 39.1526405334473	-0.474150972725161\\
 39.4971506118774	-0.474170193147783\\
% 39.49718708992	-0.474230350726273\\
% 39.4972180843353	-0.474297930022837\\
% 39.4972497940064	-0.474408875590193\\
 39.497278881073	-0.47451484157223\\
% 39.4973082065582	-0.474600287730091\\
% 39.4974827289581	-0.474674940455262\\
% 39.65032787323	-0.474709444463254\\
 39.6503335952759	-0.47472981570643\\
% 39.6503650665283	-0.474744345684744\\
% 39.7096879005432	-0.474807656242024\\
% 39.7588674545288	-0.474878794259673\\
 39.8301410198212	-0.474985121166503\\
% 39.8644632816315	-0.475086003950311\\
% 39.926452589035	-0.475160765383306\\
% 39.9270951271057	-0.475216799488742\\
 40.0534979820251	-0.475255212405475\\
% 40.0535077571869	-0.475285816670574\\
% 40.2688049793243	-0.475313908814373\\
% 40.2689120292664	-0.475341976341509\\
 40.268930387497	-0.47537531766551\\
% 40.2689377784729	-0.475462792114303\\
% 40.4292718887329	-0.475549972611795\\
% 40.4293117046356	-0.475633156684365\\
 40.4293787002564	-0.475711936638479\\
% 40.4294297218323	-0.475772015822313\\
% 40.5301465511322	-0.475821210907048\\
% 40.564444732666	-0.475874561828862\\
 40.7577771663666	-0.475935441806868\\
% 40.7585477352142	-0.47599221641565\\
% 40.7585613250732	-0.476070212440179\\
% 40.7589544773102	-0.476148898904455\\
 41.0156762123108	-0.476227101238448\\
% 41.0157110214233	-0.476299235705262\\
% 41.0157479763031	-0.476364213959251\\
% 41.015776348114	-0.476423938761242\\
 41.0157904148102	-0.476475736660202\\
% 41.2113515853882	-0.47652573840571\\
% 41.2113830566406	-0.476576996822642\\
% 41.2114123821259	-0.476628549889812\\
 41.2114569664001	-0.476678368510603\\
% 41.3561818122864	-0.476726356320796\\
% 41.356210899353	-0.476771113814937\\
% 41.3562597751617	-0.476810456948099\\
 41.4521789073944	-0.476834923757215\\
% 41.4522032260895	-0.476851670367163\\
% 41.5460445404053	-0.476861714405686\\
% 41.6698898792267	-0.47687181351624\\
 42.0554856777191	-0.476892624013182\\
% 42.0562896251678	-0.476912964571751\\
% 42.056302022934	-0.476931951362362\\
% 42.0563089370728	-0.476948473445688\\
 42.0563282489777	-0.476961571344022\\
% 42.0563456535339	-0.476961571344022\\
% 42.056353521347	-0.476961571344022\\
% 42.0563606739044	-0.476961571344022\\
 42.0563671112061	-0.476961571344022\\
% 42.0563849925995	-0.476961571344022\\
% 42.7761535167694	-0.476961571344022\\
% 42.7761735439301	-0.476961571344022\\
 43.0857395648956	-0.476961571344022\\
% 43.0857460021973	-0.476961571344022\\
% 43.0857576847076	-0.476961571344022\\
% 43.2397584438324	-0.476961571344022\\
 43.2397701263428	-0.476961571344022\\
% 43.2397839546204	-0.476961571344022\\
% 43.2644345283508	-0.476961571344022\\
% 43.6193358421326	-0.476961571344022\\
 43.6193403720856	-0.476961571344022\\
% 43.6193439483643	-0.476961571344022\\
% 43.6193475246429	-0.476961571344022\\
% 43.7342502593994	-0.476961571344022\\
% 43.7342824459076	-0.476961571344022\\
 44.1866940975189	-0.476961571344022\\
% 44.1866991043091	-0.476961571344022\\
% 44.2737109184265	-0.476961571344022\\
% 44.2737159252167	-0.476961571344022\\
 44.6853835105896	-0.476961571344022\\
% 44.6854300022125	-0.476961571344022\\
% 44.8329545974731	-0.476961571344022\\
% 44.8329832077026	-0.476961571344022\\
 44.8330263614655	-0.476961571344022\\
% 45.7841484069824	-0.476961571344022\\
% 45.78420753479	-0.476961571344022\\
% 45.7842416286469	-0.476961571344022\\
 45.7842862129211	-0.476961571344022\\
% 45.7843179225922	-0.476961571344022\\
% 45.7843596458435	-0.476961571344022\\
% 45.7843942165375	-0.476961571344022\\
 45.7844330787659	-0.476961571344022\\
% 45.7844631195068	-0.476961571344022\\
% 45.7844976902008	-0.476961571344022\\
% 45.784534406662	-0.476961571344022\\
 45.7846402645111	-0.476961571344022\\
% 46.0884601593018	-0.476961571344022\\
% 46.0884670734406	-0.476961571344022\\
% 46.1257688522339	-0.476961571344022\\
 46.1662382602692	-0.476961571344022\\
% 46.2153467655182	-0.476961571344022\\
% 46.6415497779846	-0.476961571344022\\
% 46.7479912757874	-0.476961571344022\\
 46.7480005741119	-0.476961571344022\\
% 46.9038605213165	-0.476961571344022\\
% 46.9188446521759	-0.476961571344022\\
% 47.030423116684	-0.476961571344022\\
 47.8224229335785	-0.476961571344022\\
% 47.9271270751953	-0.476961571344022\\
% 48.2469987392426	-0.476961571344022\\
% 48.2470311641693	-0.476961571344022\\
 48.247061920166	-0.476961571344022\\
% 48.6642553329468	-0.476961571344022\\
% 48.9163457870483	-0.476961571344022\\
% 48.9163507938385	-0.476961571344022\\
 48.9173902988434	-0.476961571344022\\
% 49.5241639137268	-0.476961571344022\\
% 49.5241684436798	-0.476961571344022\\
% 49.525897693634	-0.476961571344022\\
 49.5259534835815	-0.476961571344022\\
% 50.0419322967529	-0.476961571344022\\
% 51.5745617866516	-0.476961571344022\\
% 51.5749794960022	-0.476961571344022\\
 51.5749883174896	-0.476961571344022\\
% 51.8694738864899	-0.476961571344022\\
% 51.869504404068	-0.476961571344022\\
% 51.8695332527161	-0.476961571344022\\
 51.869561624527	-0.476961571344022\\
% 51.8696019172668	-0.476961571344022\\
% 51.8696481704712	-0.476961571344022\\
% 53.5224811553955	-0.476961571344022\\
 53.5224897384644	-0.476961571344022\\
% 53.5224947452545	-0.476961571344022\\
% 53.5224995136261	-0.476961571344022\\
% 53.5225045204163	-0.476961571344022\\
 53.5225088119507	-0.476961571344022\\
% 53.5225133419037	-0.476961571344022\\
% 53.5225185871124	-0.476961571344022\\
% 53.52253074646	-0.476961571344022\\
 53.5225345611572	-0.476961571344022\\
% 53.5225390911102	-0.476961571344022\\
% 53.5225431442261	-0.476961571344022\\
% 53.8357407569885	-0.476961571344022\\
 53.8357474327087	-0.476961571344022\\
% 53.8357564926147	-0.476961571344022\\
% 53.8360311508179	-0.476961571344022\\
% 53.8361589431763	-0.476961571344022\\
 53.8363892555237	-0.476961571344022\\
% 54.9593457698822	-0.476961571344022\\
% 54.9593765258789	-0.476961571344022\\
% 54.9594101428986	-0.476961571344022\\
 54.9594618797302	-0.476961571344022\\
% 54.9596836090088	-0.476961571344022\\
% 54.9597208023071	-0.476961571344022\\
% 54.9597517967224	-0.476961571344022\\
 54.9597842216492	-0.476961571344022\\
% 54.9599391937256	-0.476961571344022\\
% 54.9599730491638	-0.476961571344022\\
% 54.9600042819977	-0.476961571344022\\
 54.9601048946381	-0.476961571344022\\
% 55.7669550895691	-0.476961571344022\\
% 55.766986322403	-0.476961571344022\\
% 55.767018032074	-0.476961571344022\\
 55.7670630931854	-0.476961571344022\\
% 55.7670952796936	-0.476961571344022\\
% 55.7671293735504	-0.476961571344022\\
% 55.767174911499	-0.476961571344022\\
 55.7672061443329	-0.476961571344022\\
% 55.76724858284	-0.476961571344022\\
% 55.7672833919525	-0.476961571344022\\
% 55.7673131942749	-0.476961571344022\\
 55.7673434734345	-0.476961571344022\\
% 55.7673713684082	-0.476961571344022\\
% 55.7674064159393	-0.476961571344022\\
% 55.7674335956574	-0.476961571344022\\
 55.7674622058868	-0.476961571344022\\
% 55.7675101280212	-0.476961571344022\\
% 56.4338857650757	-0.476961571344022\\
% 56.4339029312134	-0.476961571344022\\
 56.4339148521423	-0.476961571344022\\
% 56.4339248657227	-0.476961571344022\\
% 56.4339437007904	-0.476961571344022\\
% 56.4340469360352	-0.476961571344022\\
 56.4340550422668	-0.476961571344022\\
% 56.4340698242188	-0.476961571344022\\
% 56.4340769767761	-0.476961571344022\\
% 56.4340853214264	-0.476961571344022\\
 56.4340981960297	-0.476961571344022\\
% 56.4342569828033	-0.476961571344022\\
% 56.4342746257782	-0.476961571344022\\
% 56.8145236492157	-0.476961571344022\\
 56.9261416912079	-0.476961571344022\\
% 57.0169128894806	-0.476961571344022\\
% 57.0169181346893	-0.476961571344022\\
% 58.5536970615387	-0.476961571344022\\
 58.5537037372589	-0.476961571344022\\
% 58.553709936142	-0.476961571344022\\
% 58.5537163734436	-0.476961571344022\\
% 58.5537225723267	-0.476961571344022\\
 58.5537290096283	-0.476961571344022\\
% 58.5537354469299	-0.476961571344022\\
% 58.5537423610687	-0.476961571344022\\
% 58.5537490367889	-0.476961571344022\\
 58.5537545204163	-0.476961571344022\\
% 58.5537600040436	-0.476961571344022\\
% 58.5537657260895	-0.476961571344022\\
% 58.5537721633911	-0.476961571344022\\
 58.5537781238556	-0.476961571344022\\
% 58.5537843227387	-0.476961571344022\\
% 58.5537902832031	-0.476961571344022\\
% 58.5551464080811	-0.476961571344022\\
 58.5551602363586	-0.476961571344022\\
% 58.5551681041718	-0.476961571344022\\
% 58.5551752567291	-0.476961571344022\\
% 58.5551819324493	-0.476961571344022\\
 58.9172229290009	-0.476961571344022\\
% 58.9172555923462	-0.476961571344022\\
% 58.9172882556915	-0.476961571344022\\
% 58.9172951698303	-0.476961571344022\\
 58.9173254489899	-0.476961571344022\\
% 58.9173306941986	-0.476961571344022\\
% 58.9173583507538	-0.476961571344022\\
% 58.9174062728882	-0.476961571344022\\
 59.1581761360168	-0.476961571344022\\
% 59.1582495689392	-0.476961571344022\\
% 59.1583513736725	-0.476961571344022\\
% 59.1584672451019	-0.476961571344022\\
 59.3863837242126	-0.476961571344022\\
% 59.3863975524902	-0.476961571344022\\
% 59.3864140033722	-0.476961571344022\\
% 59.3864302158356	-0.476961571344022\\
 59.3864404678345	-0.476961571344022\\
% 59.8614983081818	-0.476961571344022\\
% 59.865093421936	-0.476961571344022\\
% 59.8651005744934	-0.476961571344022\\
 59.8657540798187	-0.476961571344022\\
% 59.8657629013062	-0.476961571344022\\
% 59.8659624576569	-0.476961571344022\\
% 59.8659700870514	-0.476961571344022\\
 59.8662604808807	-0.476961571344022\\
% 59.8662702560425	-0.476961571344022\\
% 60.6861924648285	-0.476961571344022\\
% 60.686212015152	-0.476961571344022\\
 60.6862289428711	-0.476961571344022\\
% 60.6862382411957	-0.476961571344022\\
% 60.686255645752	-0.476961571344022\\
% 60.6862613677979	-0.476961571344022\\
 60.6882712364197	-0.476961571344022\\
% 60.6882829189301	-0.476961571344022\\
% 60.6882910251617	-0.476961571344022\\
% 60.6882996082306	-0.476961571344022\\
 60.6883103370667	-0.476961571344022\\
% 60.6883182048798	-0.476961571344022\\
% 60.6883239269257	-0.476961571344022\\
% 60.6883284568787	-0.476961571344022\\
 60.6883360862732	-0.476961571344022\\
% 60.6883420467377	-0.476961571344022\\
% 60.6883496761322	-0.476961571344022\\
% 60.9728793621063	-0.476961571344022\\
 60.9912290096283	-0.476961571344022\\
% 60.9912459373474	-0.476961571344022\\
% 60.9912766933441	-0.476961571344022\\
% 60.9912983894348	-0.476961571344022\\
 60.9992479801178	-0.476961571344022\\
};
\addlegendentry{$\theta_{y_{p}^{2}}$}

\addplot [color=mycolor3, line width=1.4pt]
  table[row sep=crcr]{%
 -2.3985894203186	0\\
% -2.34845380783081	0\\
% -2.29882984161377	0\\
% -2.24259977340698	0\\
 -2.1981945514679	0\\
% -2.14788823127747	0\\
% -2.09815530776978	0\\
% -2.04817252159119	0\\
 -1.99881820678711	0\\
% -1.94862990379333	0\\
% -1.87693266868591	0\\
% -1.84271411895752	0\\
 -1.79806022644043	0\\
% -1.74720959663391	0\\
% -1.69869332313538	0\\
% -1.6482563495636	0\\
 -1.59839563369751	0\\
% -1.548703956604	0\\
% -1.49876074790955	0\\
% -1.44885139465332	0\\
 -1.3982711315155	0\\
% -1.34799294471741	0\\
% -1.29795436859131	0\\
% -1.2485387802124	0\\
 -1.19843416213989	0\\
% -1.14853506088257	0\\
% -1.09835534095764	0\\
% -1.04829435348511	0\\
 -0.998184490203857	0\\
% -0.948882627487182	0\\
% -0.898640441894531	0\\
% -0.847953605651855	0\\
 -0.798753547668457	0\\
% -0.748502779006958	0\\
% -0.698466348648071	0\\
% -0.648822116851806	0\\
 -0.597503709793091	0\\
% -0.548140096664429	0\\
% -0.498594331741333	0\\
% -0.4485755443573	0\\
 -0.398829984664917	0\\
% -0.348826217651367	0\\
% -0.298781204223633	0\\
% -0.248498249053955	0\\
 -0.197533178329468	0\\
% -0.147660779953003	0\\
% -0.0983634471893309	0\\
% -0.0481057643890379	0\\
 0.0019752502441408	0\\
% 0.0514363765716555	0\\
% 0.113763284683228	0\\
% 0.15691466331482	0\\
 0.234845352172852	0\\
% 0.308734607696533	0\\
% 0.308903646469116	0\\
% 0.360757303237915	0\\
 0.415858459472656	0\\
% 0.497472476959229	-1.55956442892092\\
% 0.559248399734497	-2.1394979497918\\
% 0.568476867675781	-2.441075008518\\
 0.604575109481812	-2.62966118368422\\
% 0.723840665817261	-2.75154269737516\\
% 0.723884773254395	-2.82454011052459\\
% 0.853632164001465	-2.86225077157383\\
 0.853860330581665	-2.87840081826289\\
% 0.859521579742432	-2.88086408595223\\
% 0.922651720046997	-2.87845475425092\\
% 0.971385669708252	-2.87492314899146\\
 1.02831001281738	-2.86993572459801\\
% 1.07039852142334	-2.85908838024875\\
% 1.11899919509888	-2.83873218547842\\
% 1.16864533424377	-2.81293144771394\\
 1.22722692489624	-2.78125805496666\\
% 1.32403750419617	-2.74957416845473\\
% 1.3240451335907	-2.71872430743133\\
% 1.43628258705139	-2.68930337997153\\
 1.4363646030426	-2.66071849943683\\
% 1.6345018863678	-2.63160478565942\\
% 1.63452954292297	-2.5984292975827\\
% 1.63456315994263	-2.56087638793542\\
 1.63461179733276	-2.52153511375127\\
% 1.67959208488464	-2.48140533206924\\
% 1.74344153404236	-2.43746936422212\\
% 1.80315823554993	-2.39368977447748\\
 1.8214485168457	-2.35281658307758\\
% 1.86181755065918	-2.31412430791761\\
% 1.95309824943542	-2.27811593967272\\
% 1.98877353668213	-2.23938238879828\\
 2.10328192710876	-2.19675649650117\\
% 2.10334939956665	-2.14953745512958\\
% 2.2043318271637	-2.10088933212955\\
% 2.20438261032105	-2.05163005688246\\
 2.20442314147949	-2.00662468598239\\
% 2.31627173423767	-1.96591961333525\\
% 2.31644148826599	-1.92686354697162\\
% 2.35974855422974	-1.88494388634081\\
 2.40763850212097	-1.83903238024959\\
% 2.58187789916992	-1.79002590942673\\
% 2.58190841674805	-1.7405349538094\\
% 2.58211798667908	-1.69247248411557\\
 2.96352286338806	-1.64793807357\\
% 2.96353478431702	-1.6056779714263\\
% 2.96354002952576	-1.56358265236122\\
% 2.96354813575745	-1.52115660529671\\
 2.96359629631042	-1.47505354950772\\
% 2.96360988616943	-1.42836585199984\\
% 2.9636182308197	-1.3806955299433\\
% 2.96363134384155	-1.33299646686828\\
 3.06343312263489	-1.28606437768576\\
% 3.06350989341736	-1.240543265415\\
% 3.14400072097778	-1.19524753236146\\
% 3.24244160652161	-1.14960601572511\\
 3.24249835014343	-1.10265179950147\\
% 3.38695282936096	-1.05395365849745\\
% 3.38704962730408	-1.00510316896361\\
% 3.38712425231934	-0.955483823604482\\
 3.50973148345947	-0.904534654616327\\
% 3.50977296829224	-0.852583643672006\\
% 3.50982351303101	-0.800124834973758\\
% 3.60157866477966	-0.749000405986408\\
 3.61497135162354	-0.698483886135364\\
% 3.66265697479248	-0.648400759401738\\
% 3.77121181488037	-0.599483628330745\\
% 3.77127523422241	-0.550256520489711\\
 3.88388056755066	-0.50033814966082\\
% 3.88406581878662	-0.44975261962918\\
% 4.01174230575562	-0.399272377676425\\
% 4.01174683570862	-0.348966448759711\\
 4.01175088882446	-0.299720463793165\\
% 4.10342116355896	-0.251838606361162\\
% 4.10781664848328	-0.205265127944585\\
% 4.22873516082764	-0.159351065338342\\
 4.22877926826477	-0.113798980487445\\
% 4.3162703037262	-0.0676291229819981\\
% 4.31628270149231	-0.0206733919012549\\
% 4.36952753067017	0.0252989811760926\\
 4.49391932487488	0.0695820987511979\\
% 4.4939248085022	0.11197905982408\\
% 4.50986928939819	0.15317497527667\\
% 4.60001916885376	0.193331441728333\\
 4.6112982749939	0.233017204958742\\
% 4.713267993927	0.273326287583586\\
% 4.71331377029419	0.31380307589825\\
% 4.84796900749207	0.352877337638347\\
 4.84803886413574	0.389949369089663\\
% 4.9620210647583	0.425260817840353\\
% 4.96202535629272	0.45935697827008\\
% 4.96383280754089	0.492511831180309\\
 5.00998063087463	0.52605135826343\\
% 5.05557627677917	0.558974951502023\\
% 5.10753936767578	0.593109552314672\\
% 5.19693036079407	0.627811887253301\\
 5.24160213470459	0.661236755157915\\
% 5.2747401714325	0.693147554646202\\
% 5.47130937576294	0.722978665683513\\
% 5.47131628990173	0.75046569972892\\
 5.47132034301758	0.776659597935577\\
% 5.47196359634399	0.802001829511028\\
% 5.92306728363037	0.827275359653868\\
% 5.92307133674622	0.851790148553846\\
 5.92308039665222	0.875362606056115\\
% 5.92308468818665	0.897602207647651\\
% 5.92308826446533	0.91839981108933\\
% 5.92309231758118	0.938060002138627\\
 5.92309613227844	0.956803249053337\\
% 5.92310948371887	0.974990675202207\\
% 5.92311329841614	0.99205644308131\\
% 5.99617094993591	1.00791544340609\\
 6.01983733177185	1.02208831832831\\
% 6.06683416366577	1.03483648276233\\
% 6.13510556221008	1.04673227227613\\
% 6.17488260269165	1.05811803680899\\
 6.34516496658325	1.06904978299008\\
% 6.34567971229553	1.07959344785741\\
% 6.34579391479492	1.08919956921454\\
% 6.48088450431824	1.09797383190198\\
 6.48123211860657	1.10508296755233\\
% 6.48173136711121	1.11089662099403\\
% 6.55087299346924	1.11516265388946\\
% 6.60183663368225	1.11827582847036\\
 6.61844868659973	1.12124500129448\\
% 6.6699878692627	1.1244145486271\\
% 6.73549313545227	1.12721170751593\\
% 6.90011711120605	1.12899834690506\\
 6.90014715194702	1.12941727891575\\
% 6.90017552375793	1.12848751761157\\
% 6.96017570495605	1.12625659885907\\
% 6.96023197174072	1.12283178274811\\
 7.02018661499023	1.11904989821505\\
% 7.08605260848999	1.11579222837372\\
% 7.14995141029358	1.11287090639689\\
% 7.2400710105896	1.10897433919899\\
 7.24012393951416	1.10407427478003\\
% 7.35665650367737	1.09752294585041\\
% 7.35666365623474	1.09032977419156\\
% 7.50300760269165	1.08202017997883\\
 7.50304217338562	1.07441117558164\\
% 7.50325675010681	1.06721997767727\\
% 7.55728693008423	1.06057012556676\\
% 7.58989186286926	1.05263579208076\\
 7.60993332862854	1.04450411230846\\
% 7.71798748970032	1.03567633504099\\
% 7.71803326606751	1.02570862380389\\
% 7.8084406375885	1.01527834089711\\
 7.8084451675415	1.00390747758547\\
% 7.86788339614868	0.99304207541627\\
% 7.98911018371582	0.982769785627625\\
% 7.98919291496277	0.972212687092906\\
 8.03272266387939	0.961735089195827\\
% 8.08047933578491	0.950526168994656\\
% 8.12581033706665	0.938570887293281\\
% 8.19591255187988	0.925302108455071\\
 8.21851797103882	0.911291224766956\\
% 8.32182474136352	0.896136778841083\\
% 8.32182974815369	0.881935958963822\\
% 8.37021417617798	0.868941527120114\\
 8.40591187477112	0.856290575362891\\
% 8.59654874801636	0.844070795163589\\
% 8.59659929275513	0.831814180362926\\
% 8.59668583869934	0.818266662250608\\
 8.75598998069763	0.803945872173244\\
% 8.75599665641785	0.788793971299128\\
% 8.75600190162659	0.772892525237694\\
% 8.75602097511291	0.757391397135507\\
 9.02396960258484	0.741339669546505\\
% 9.02399797439575	0.726793173391798\\
% 9.02402396202087	0.713259199579852\\
% 9.02404947280884	0.699876367745674\\
 9.02408738136291	0.686653243075625\\
% 9.34571809768677	0.673181472496481\\
% 9.34574980735779	0.65913936403831\\
% 9.34578461647034	0.644723339642724\\
 9.34599132537842	0.630423057316648\\
% 9.34602160453796	0.616153167804441\\
% 9.34606547355652	0.602533832007794\\
% 9.55178208351135	0.589799201156438\\
 9.55178709030151	0.577754381841714\\
% 9.55179281234741	0.565851816415716\\
% 9.55179829597473	0.55368327706492\\
% 9.61852855682373	0.541315063844877\\
 9.61856646537781	0.528729960072951\\
% 9.6634735584259	0.516293633949545\\
% 9.70841760635376	0.504191045336256\\
% 9.81058926582336	0.492242627861742\\
 9.81059379577637	0.48119536734157\\
% 9.91963858604431	0.470845693171213\\
% 9.91964693069458	0.460197668146293\\
% 9.97360939979553	0.449528731769988\\
 10.0412783145905	0.438891106671463\\
% 10.1883935451508	0.427866624249418\\
% 10.1884243011475	0.415930522750841\\
% 10.1884679317474	0.403739271382847\\
 10.2828607082367	0.392011604197933\\
% 10.2828685760498	0.381080876180476\\
% 10.3878936290741	0.370654799766271\\
% 10.3879322528839	0.360900085827552\\
 10.477405500412	0.351020735327438\\
% 10.4774195671082	0.341178642597242\\
% 10.5185296058655	0.331164863939421\\
% 10.560958814621	0.321084499383346\\
 10.6081099033356	0.311128129281656\\
% 10.6624338150024	0.300921561512155\\
% 10.7120904445648	0.291363785065414\\
% 10.8078364849091	0.281829276232997\\
 10.8310131549835	0.272825608333505\\
% 10.86188621521	0.26403358311029\\
% 10.9186720371246	0.255368436128492\\
% 10.9605512142181	0.246767774471436\\
 11.0521792888641	0.238299894170154\\
% 11.0872270584106	0.229757736973852\\
% 11.1295003414154	0.221673235799017\\
% 11.1574806690216	0.213352013931512\\
 11.2049316883087	0.205439028513638\\
% 11.264439535141	0.197858190251054\\
% 11.3169879436493	0.19029318029709\\
% 11.3666297912598	0.182795155464191\\
 11.4129797935486	0.175285830055714\\
% 11.487088394165	0.167962559415741\\
% 11.5034069538116	0.160881228581729\\
% 11.5847613334656	0.154167585099913\\
 11.6668142795563	0.146997753050982\\
% 11.6668383598328	0.140210570633485\\
% 11.756715965271	0.133165425024572\\
% 11.7568525791168	0.126239779483626\\
 11.8283314228058	0.119554866175804\\
% 11.861048412323	0.113164651592797\\
% 11.9123050689697	0.107062398948642\\
% 11.9995982170105	0.100604140655605\\
 12.0273801803589	0.0944667342017738\\
% 12.0843092918396	0.088593441246644\\
% 12.1415590763092	0.0821643707348585\\
% 12.1802153110504	0.0758716673448134\\
 12.2485608577728	0.0699664552805643\\
% 12.2750324726105	0.0642455041015637\\
% 12.3711525917053	0.0587409163326811\\
% 12.3711673736572	0.0530450457234792\\
 12.584521484375	0.0475919203909143\\
% 12.5848187923431	0.0423546599041629\\
% 12.5848257064819	0.0366578608446844\\
% 12.5848338127136	0.0311410996434915\\
 12.6255592823029	0.0257724922621492\\
% 12.6870493412018	0.0205826641463887\\
% 12.7069229602814	0.0156455112612548\\
% 12.7557956695557	0.010466470230142\\
 12.8823637485504	0.00546024127180544\\
% 12.882368516922	0.000684405920367226\\
% 12.9974901199341	-0.00483197804237534\\
% 12.9975285053253	-0.010082998268274\\
 13.0484261035919	-0.0153130694155266\\
% 13.0905348777771	-0.0203467807452853\\
% 13.1565808773041	-0.0252121106958612\\
% 13.2565376281738	-0.0298830900779308\\
 13.2565733909607	-0.0343686413557975\\
% 13.2757715702057	-0.0386070071680251\\
% 13.3401882171631	-0.0431442292152724\\
% 13.4026224136353	-0.0474920330565851\\
 13.4468314170837	-0.0516722354421972\\
% 13.4851455211639	-0.0564253322806962\\
% 13.5162672519684	-0.0609780902194643\\
% 13.5994228839874	-0.0653247153841221\\
 13.6809041023254	-0.0692311709050841\\
% 13.6809529781342	-0.0729275886331493\\
% 13.760929775238	-0.0770254377129618\\
% 13.7609891414642	-0.0809643166148248\\
 13.8658623218536	-0.084755340866252\\
% 13.8659095287323	-0.0888832159669164\\
% 13.9651402950287	-0.092831559690751\\
% 13.9651853561401	-0.0964483811119408\\
 14.013670873642	-0.0998914946131322\\
% 14.0695306777954	-0.103118431266694\\
% 14.1660151004791	-0.106367658776492\\
% 14.1674413204193	-0.109463656703213\\
 14.3973986625671	-0.113191425736375\\
% 14.3974039077759	-0.11681103678319\\
% 14.397408914566	-0.120515995065603\\
% 14.3974151134491	-0.124073233292393\\
 14.4515306472778	-0.127468937922458\\
% 14.4809290885925	-0.130553522549576\\
% 14.507790517807	-0.13345362096004\\
% 14.6076409339905	-0.136536323830171\\
 14.6330356121063	-0.139478875816422\\
% 14.7221893787384	-0.142823762227493\\
% 14.7222127437592	-0.146059703683193\\
% 14.8223895549774	-0.149183820602616\\
 14.8226484775543	-0.15224577581975\\
% 14.8592681407928	-0.15515500315712\\
% 14.9488486766815	-0.157911270095724\\
% 14.9796115875244	-0.160525774819945\\
 15.0424162864685	-0.162987740030417\\
% 15.0784852027893	-0.165335823162678\\
% 15.1091608524323	-0.168126278151107\\
% 15.1938590526581	-0.170826263367417\\
 15.2520548820496	-0.173769933802987\\
% 15.3033129692078	-0.176570329949357\\
% 15.3873831748962	-0.179231213044488\\
% 15.389208984375	-0.181650981028923\\
 15.5462734222412	-0.183903605094628\\
% 15.5463118076324	-0.186009446197914\\
% 15.5463530540466	-0.188192281793306\\
% 15.6087738990784	-0.190288427732071\\
 15.6088122844696	-0.192328570952668\\
% 15.8284608840942	-0.194981916213663\\
% 15.82850689888	-0.197515092004778\\
% 15.8285231113434	-0.199913992925389\\
 15.8290633678436	-0.202184729105966\\
% 15.9087426185608	-0.204266282474521\\
% 16.0841259479523	-0.206598949001233\\
% 16.0841304779053	-0.208838151275671\\
 16.0841798305511	-0.211011525068727\\
% 16.0844277858734	-0.213661969046427\\
% 16.1285621643066	-0.216194956488238\\
% 16.1920978546143	-0.218607023674224\\
 16.2366220474243	-0.220787998234158\\
% 16.2680780410767	-0.222794175933871\\
% 16.3074433326721	-0.224647533374565\\
% 16.3852943897247	-0.226508121981254\\
 16.427689743042	-0.22824750804611\\
% 16.4832610607147	-0.229935247025821\\
% 16.5667356967926	-0.232098365782349\\
% 16.5667421340942	-0.234198873366211\\
 16.6510304927826	-0.236433382398104\\
% 16.6845106601715	-0.238550072138707\\
% 16.7282096862793	-0.240556712834177\\
% 16.7647928714752	-0.242403610680441\\
 16.8415979862213	-0.244112267244617\\
% 16.9638578414917	-0.245678871655315\\
% 16.9638618946075	-0.24715158363712\\
% 16.963866186142	-0.24871804789305\\
 17.0969151973724	-0.250798203013289\\
% 17.0969202041626	-0.252783086506597\\
% 17.1762506484985	-0.25464949963451\\
% 17.1763615131378	-0.256381856592895\\
 17.2229020118713	-0.25803398409036\\
% 17.3326086521149	-0.259555356839954\\
% 17.3326153278351	-0.260883445062817\\
% 17.473627281189	-0.262101940490297\\
 17.4736518383026	-0.263553586100215\\
% 17.4736754417419	-0.264967615458261\\
% 17.5113889694214	-0.267055055202135\\
% 17.5604962825775	-0.269056181583338\\
 17.6228019714355	-0.270835835481449\\
% 17.6881763458252	-0.272495816165005\\
% 17.8001989841461	-0.274072161883723\\
% 17.8002123355865	-0.275459606552516\\
 17.8233386993408	-0.276939653009578\\
% 17.9228260040283	-0.278319755671063\\
% 17.9228312492371	-0.280157995359801\\
% 18.0239114284515	-0.281968875138134\\
 18.023930978775	-0.283870428876384\\
% 18.2090794563293	-0.285681861163795\\
% 18.2091359615326	-0.287371044074945\\
% 18.209382724762	-0.288963503518701\\
 18.2096747875214	-0.290341451685194\\
% 18.3486201286316	-0.291605648246062\\
% 18.3489136219025	-0.293087190378117\\
% 18.4253313064575	-0.294477243914727\\
 18.4253766059875	-0.295776386444913\\
% 18.5247117996216	-0.297171840038118\\
% 18.5248689174652	-0.298436933625453\\
% 18.5989043235779	-0.299612012763291\\
 18.6113500118256	-0.300846092225441\\
% 18.6828867912292	-0.301936224754627\\
% 18.7189335346222	-0.302956312362653\\
% 18.7590083599091	-0.303932815515225\\
 18.8376090049744	-0.305361567249605\\
% 18.8578114032745	-0.306751004378583\\
% 18.9627770900726	-0.308039251339423\\
% 18.9628245353699	-0.309512247376247\\
 19.0279707431793	-0.310809449604136\\
% 19.0604125976562	-0.311996304661946\\
% 19.1335758686066	-0.313037303830384\\
% 19.2351016521454	-0.313917592880649\\
 19.2352606773376	-0.314744246847731\\
% 19.5428084850311	-0.315552217438231\\
% 19.5428378105164	-0.317104839277945\\
% 19.5428936004639	-0.318628853710493\\
 19.5429250717163	-0.320072631474986\\
% 19.5429551124573	-0.321502235561184\\
% 19.5429982662201	-0.322802798838879\\
% 19.9834553718567	-0.323957520016727\\
 19.9834630012512	-0.324939216667168\\
% 19.9834668159485	-0.325793399229738\\
% 19.9834703922272	-0.326511389629026\\
% 19.9834739685059	-0.327624773508944\\
 19.983477306366	-0.329059192234318\\
% 19.9834856510162	-0.330451917278957\\
% 19.9834892272949	-0.331799827642506\\
% 19.9834928035736	-0.333044506073477\\
% 20.048566532135	-0.334147425753947\\
 20.1035956859589	-0.335261940696256\\
% 20.1756994247437	-0.336189640224376\\
% 20.1757237434387	-0.337044233863644\\
% 20.2920464992523	-0.338157322960363\\
 20.2920505523682	-0.339247958356417\\
% 20.3619036197662	-0.340597763343744\\
% 20.4108409404755	-0.341894132062471\\
% 20.4923514842987	-0.34305995346886\\
 20.4923595905304	-0.344051937219028\\
% 20.5578968048096	-0.344937210720559\\
% 20.6461452960968	-0.345653555612273\\
% 20.6461510181427	-0.346596927965159\\
 20.7409729480743	-0.34752231171791\\
% 20.775611114502	-0.348821852458727\\
% 20.8289811134338	-0.350108391312851\\
% 20.855291557312	-0.351351599304031\\
 20.8881897449493	-0.352546203764184\\
% 20.9274053096771	-0.353648119560383\\
% 20.9642960548401	-0.354588476264169\\
% 21.0089377880096	-0.355400646631146\\
 21.0846263885498	-0.356068305011163\\
% 21.1657716751099	-0.356780333163378\\
% 21.1659259319305	-0.357414762434026\\
% 21.2478870868683	-0.357921703655215\\
 21.287251663208	-0.358409556833013\\
% 21.3467800140381	-0.358913775438207\\
% 21.3878879070282	-0.359770953924844\\
% 21.4673904895782	-0.360551737573829\\
 21.4674722671509	-0.361436299886861\\
% 21.7880739688873	-0.362230270692308\\
% 21.7880997180939	-0.362879121375997\\
% 21.7881242752075	-0.363647172955815\\
 21.788148355484	-0.364381447201502\\
% 21.788175535202	-0.365125825910994\\
% 21.7882205963135	-0.366073011359436\\
% 21.9344484329224	-0.366951216515126\\
 21.9345077991486	-0.367743617927449\\
% 21.9346115112305	-0.368462544294932\\
% 22.1828016757965	-0.369055497274452\\
% 22.1828560352325	-0.369676932845685\\
 22.1828851222992	-0.370237305428148\\
% 22.1829120635986	-0.371033752824555\\
% 22.1829561710358	-0.37184179719312\\
% 22.2871834754944	-0.372640462575237\\
 22.2872302055359	-0.373367837730665\\
% 22.3290242671967	-0.374196490984101\\
% 22.3717877388	-0.374850636503592\\
% 22.4189037799835	-0.37536663255597\\
 22.508477640152	-0.375774657927231\\
% 22.579109621048	-0.376030350924356\\
% 22.5791170120239	-0.376247935140185\\
% 22.6909491539001	-0.377046685709583\\
 22.6910020828247	-0.377848418432073\\
% 23.0378574848175	-0.378672257761743\\
% 23.0378658294678	-0.379535412407375\\
% 23.0378715515137	-0.380321158461161\\
 23.0378772735596	-0.380972751747645\\
% 23.0378908634186	-0.381504534947734\\
% 23.0379697799683	-0.381879799109559\\
% 23.0379788398743	-0.382078339264844\\
 23.1805610179901	-0.382504026925712\\
% 23.1805998802185	-0.382907507396194\\
% 23.2553226470947	-0.383761028521594\\
% 23.3742901802063	-0.384604177280757\\
 23.3742956638336	-0.385352816522971\\
% 23.4069616317749	-0.386009605204421\\
% 23.436697435379	-0.386506796830293\\
% 23.4717108726501	-0.386923268443518\\
 23.4726278305054	-0.387288938088441\\
% 23.5186371326447	-0.387484270107883\\
% 23.5845655918121	-0.387643174530947\\
% 23.6520792961121	-0.388147511197239\\
 23.7337271690369	-0.388652173987499\\
% 23.7337564945221	-0.389439539387263\\
% 23.7942924022675	-0.390182487243061\\
% 23.8527013778687	-0.39084608753581\\
 23.8903464794159	-0.391285123774477\\
% 23.9444241046906	-0.391609091011839\\
% 24.0006498813629	-0.39187160443492\\
% 24.0766436576843	-0.392281622607765\\
 24.0766481876373	-0.392695288642843\\
% 24.1547135829926	-0.393121364750602\\
% 24.1547355175018	-0.393845968918102\\
% 24.2052554607391	-0.394524810673422\\
 24.3249959468842	-0.395076436767432\\
% 24.3469130516052	-0.395496986395559\\
% 24.3948239803314	-0.395795295432793\\
% 24.4395904064178	-0.396110414930055\\
 24.4598485946655	-0.396317800708914\\
% 24.5525354862213	-0.396501273566768\\
% 24.5862226009369	-0.396715300796298\\
% 24.6285922050476	-0.39736409477741\\
 24.8560425758362	-0.39799077145331\\
% 24.8560881137848	-0.398579665593692\\
% 24.8566789150238	-0.3990470012684\\
% 24.8570489406586	-0.399428705248924\\
 24.9618987560272	-0.399722882490648\\
% 24.9619016170502	-0.400035353258458\\
% 24.9667996883392	-0.400339241237013\\
% 25.1297199249268	-0.400941764270117\\
 25.129741859436	-0.401525661860507\\
% 25.1297678470612	-0.402063445498573\\
% 25.1699120521545	-0.402525051664028\\
% 25.3255557537079	-0.402882216156694\\
 25.3258833408356	-0.403158125239305\\
% 25.4633335590363	-0.403373984219138\\
% 25.4633907794952	-0.403799602098333\\
% 25.4634375095367	-0.404230835005908\\
 26.2949263572693	-0.404653259396934\\
% 26.2949594974518	-0.405042961254569\\
% 26.2949900150299	-0.405557523121002\\
% 26.2950171947479	-0.405987849716187\\
 26.2950567722321	-0.40634212698094\\
% 26.2950622558594	-0.406815237699853\\
% 26.2950939655304	-0.407236293545386\\
% 26.2951228141785	-0.407612113844031\\
 26.2951516628265	-0.408046247191136\\
% 26.2951850414276	-0.408425717512506\\
% 26.2952184200287	-0.408828496036735\\
% 26.2952496528625	-0.409165783721626\\
 26.2952785015106	-0.409517673970284\\
% 26.2953056812286	-0.409798394286788\\
% 26.2953350067139	-0.410022311499182\\
% 26.2953686237335	-0.410414414750315\\
 26.2954256057739	-0.410797683973152\\
% 26.5964006900787	-0.411167541337925\\
% 26.5964049816132	-0.411632627910667\\
% 26.5964097499847	-0.412037418466931\\
 26.5964138031006	-0.412417839347833\\
% 26.5964183330536	-0.412735937397714\\
% 26.5964228630066	-0.412995376518858\\
% 26.7749821662903	-0.413181815577982\\
 26.774987411499	-0.413399986620606\\
% 26.7755240917206	-0.41355666981584\\
% 26.8470217704773	-0.413913010091547\\
% 26.8470828056335	-0.414264316507533\\
 26.89402384758	-0.414609828395033\\
% 26.927468252182	-0.415067709023914\\
% 27.2203940868378	-0.415473356472503\\
% 27.2204248428345	-0.415827815919634\\
 27.2204520225525	-0.416116882628391\\
% 27.2204794406891	-0.41630482874408\\
% 27.2205059051514	-0.416416703856243\\
% 27.2205748081207	-0.416574975873327\\
 28.8658260822296	-0.416732016114446\\
% 28.8658315658569	-0.417203739726083\\
% 28.8668162345886	-0.417689971204712\\
% 28.866832447052	-0.418129474263329\\
 28.8669096946716	-0.418504886695928\\
% 28.8669151782989	-0.41882986995078\\
% 28.8669273376465	-0.41909810121598\\
% 28.8669893264771	-0.419340331377494\\
 28.8669952869415	-0.419599051898071\\
% 28.8670451164246	-0.419872826861041\\
% 28.8670749187469	-0.420162604669011\\
% 28.8671030521393	-0.420561261793507\\
 28.867130947113	-0.420938356991564\\
% 28.8671600341797	-0.421284509855117\\
% 28.8671998500824	-0.421594657464858\\
% 28.8672048568726	-0.421849127883013\\
 28.8672348976135	-0.422048827849264\\
% 28.8672766208649	-0.422229076548838\\
% 28.8672813892364	-0.422318483398141\\
% 28.8674888134003	-0.422378733816638\\
 28.8674966812134	-0.422360833784335\\
% 28.8675477027893	-0.42251340128088\\
% 28.867590379715	-0.422686770072468\\
% 28.8676218509674	-0.422892020806048\\
 28.8676533222198	-0.423315748001675\\
% 28.8676936149597	-0.423754857736658\\
% 28.8676976680756	-0.424182849506645\\
% 28.8677289009094	-0.42457738064809\\
 28.8677708625793	-0.424956435343177\\
% 28.8677784919739	-0.42529869704996\\
% 28.8678075790405	-0.425604163529995\\
% 28.867866230011	-0.42580626114605\\
 28.8679084300995	-0.42602255588965\\
% 29.2079612731934	-0.426246187774086\\
% 29.208006811142	-0.426472697149436\\
% 29.2080163478851	-0.426700431183955\\
 29.2080223083496	-0.427026517693563\\
% 29.2080304145813	-0.427288272979939\\
% 29.2080440044403	-0.427493114629417\\
% 29.208407831192	-0.427642385551255\\
 30.8635506153107	-0.42772546072008\\
% 30.8635580062866	-0.427816222067756\\
% 30.8635627746582	-0.42793717385303\\
% 30.8635682582855	-0.428296003202478\\
 30.8635725498199	-0.428668449740203\\
% 30.8635766029358	-0.429027135747821\\
% 30.8635820865631	-0.429228524814308\\
% 30.8635863780975	-0.429430477415919\\
 30.8635904312134	-0.429596703106224\\
% 30.8635944843292	-0.429739454624323\\
% 30.8635987758636	-0.429910101059122\\
% 30.8636030673981	-0.430092513998402\\
 30.8636075973511	-0.430351810007109\\
% 30.8636118888855	-0.430663998977046\\
% 30.8636159420013	-0.43101909604462\\
% 30.8636209487915	-0.431450414359591\\
 30.8636252403259	-0.43200573577938\\
% 30.8636300086975	-0.432539640801235\\
% 30.8636340618134	-0.432968535759202\\
% 30.8636385917664	-0.43335857200228\\
 30.8636426448822	-0.433730131997943\\
% 30.8636474132538	-0.433585810413384\\
% 30.863651227951	-0.433380458933169\\
% 30.8636557579041	-0.432990431143546\\
 30.8636600494385	-0.432535273640507\\
% 30.8636657714844	-0.432328445935275\\
% 30.8636707782745	-0.431997339321692\\
% 30.863675069809	-0.431704090391033\\
 30.8636786460876	-0.431928247117083\\
% 30.8636826992035	-0.432224275380568\\
% 31.2550806522369	-0.432590036855013\\
% 31.2551092624664	-0.433098900723309\\
 31.2551359653473	-0.433629821166794\\
% 31.2553131103516	-0.434133957315201\\
% 31.2553185939789	-0.434345159687027\\
% 31.255323600769	-0.434501075773671\\
 31.2553271770477	-0.434627506809259\\
% 31.2553925037384	-0.434729824953055\\
% 31.3403334140778	-0.434839748112026\\
% 31.3403391361237	-0.435047211051364\\
 31.5086352348328	-0.435282408542711\\
% 31.5086521625519	-0.435580436898714\\
% 31.5086788654327	-0.435916253920489\\
% 31.6173405170441	-0.436251487683274\\
 31.6173724651337	-0.43656815415294\\
% 31.6975693225861	-0.436871624077962\\
% 31.6975774288177	-0.43702253842287\\
% 31.7641047954559	-0.437113042870683\\
 31.7724740028381	-0.437196689085967\\
% 32.082110118866	-0.437268855326051\\
% 32.0831522464752	-0.437338063568082\\
% 32.0832287788391	-0.437421517556887\\
 32.0832550048828	-0.43754535637205\\
% 32.0832790851593	-0.437725584863457\\
% 32.0845431804657	-0.437923207455094\\
% 32.4165415287018	-0.438143396060386\\
 32.4165484428406	-0.438360559516255\\
% 32.416555595398	-0.438568301423519\\
% 32.4167065143585	-0.438744184269167\\
% 32.4169623374939	-0.438891943013893\\
 32.4169682979584	-0.439024712195016\\
% 32.6960920810699	-0.439152359530706\\
% 32.6961004257202	-0.439266233277736\\
% 32.6961133003235	-0.439379349651689\\
 32.6961235523224	-0.43950711497362\\
% 32.696128320694	-0.43964837913841\\
% 32.6961371421814	-0.439819984195851\\
% 32.8045789718628	-0.440040291498335\\
 32.8046054363251	-0.440266426301799\\
% 32.8438534259796	-0.440476438140133\\
% 32.9417709827423	-0.440672554274248\\
% 32.9419347763062	-0.440859250481086\\
 33.1483494758606	-0.440949219570201\\
% 33.1483871459961	-0.441062193829195\\
% 33.1485037326813	-0.441184717957433\\
% 33.1485960006714	-0.441327310205415\\
 33.2275540351868	-0.441484354403111\\
% 33.2620851516724	-0.441652838330844\\
% 33.4662851810455	-0.44181800863191\\
% 33.4662901878357	-0.441997247321116\\
 33.4662944793701	-0.442181145159841\\
% 33.4663323879242	-0.442315044326432\\
% 33.6951984882355	-0.442422675977831\\
% 33.6952318668366	-0.44251998547754\\
 33.6952564239502	-0.442623201495139\\
% 33.6952812194824	-0.442717881826749\\
% 33.6953327178955	-0.442823567591987\\
% 34.0244714736939	-0.442932225311192\\
 34.0244769573212	-0.443058353586519\\
% 34.0244836330414	-0.443191003571135\\
% 34.0244886398315	-0.443315521674331\\
% 34.0244934082031	-0.443435190572225\\
 34.0244967460632	-0.443529669491688\\
% 34.1406688213348	-0.443610986453048\\
% 34.1412770271301	-0.443691691559567\\
% 34.1414825439453	-0.443762995059673\\
 34.3046380996704	-0.443833777185065\\
% 34.3046442985535	-0.443903360709526\\
% 34.3046488285065	-0.444001861989558\\
% 34.3536202430725	-0.444118476462539\\
 34.4075736522675	-0.444231214580151\\
% 34.4325148582459	-0.44434501115099\\
% 34.5385291099548	-0.444434624952009\\
% 34.538582277298	-0.444513410099131\\
 34.5713927268982	-0.444573501392092\\
% 34.7362963676453	-0.444607416451752\\
% 34.7364422798157	-0.44466855911269\\
% 34.7365619659424	-0.444717839417659\\
 34.8030476093292	-0.4447632561423\\
% 34.8534969806671	-0.444862392041654\\
% 34.896230173111	-0.444962150957453\\
% 34.9908022403717	-0.445086017602917\\
 34.9910139560699	-0.445203913253521\\
% 35.1785239696503	-0.445311284039544\\
% 35.178533744812	-0.445409440996708\\
% 35.1785385131836	-0.445497509653875\\
 35.1785497188568	-0.445594924844966\\
% 35.2687682628632	-0.445687948025849\\
% 35.3520075798035	-0.445777158168888\\
% 35.3520118713379	-0.445887358089877\\
 35.4015700340271	-0.445991929233788\\
% 35.4856979370117	-0.446088138754716\\
% 35.4857115268707	-0.446182589429513\\
% 35.567130279541	-0.446264412264937\\
 35.6499711990356	-0.446341076850554\\
% 35.6500160217285	-0.446431924003687\\
% 35.7923717021942	-0.446520707765997\\
% 35.794321012497	-0.446605782725528\\
 35.7943298339844	-0.446715346770228\\
% 35.9174291610718	-0.446818212301368\\
% 35.9174360752106	-0.446913394907525\\
% 36.0053495883942	-0.447004106082936\\
 36.0053569793701	-0.447078128125153\\
% 36.0758685588837	-0.447162781999529\\
% 36.0759078979492	-0.447241299972723\\
% 36.1870767593384	-0.447317713248552\\
 36.1871056079865	-0.447437597604286\\
% 36.2769727230072	-0.447553962233152\\
% 36.2769793987274	-0.447664389219805\\
% 36.3538722515106	-0.44776253874372\\
 36.4040152549744	-0.447858134754535\\
% 36.4267644405365	-0.447936034992706\\
% 36.4882075309753	-0.448031421109921\\
% 36.5738670349121	-0.448125894769426\\
 36.7500044822693	-0.448221752565943\\
% 36.7500144958496	-0.448350924831439\\
% 36.7500192642212	-0.44847729717139\\
% 36.7500240325928	-0.448592155259801\\
 36.9328815460205	-0.44869651221534\\
% 36.9328860759735	-0.448780606211294\\
% 36.9328906059265	-0.448849579119152\\
% 36.9333383560181	-0.448898641146113\\
 37.1490315914154	-0.448933928705035\\
% 37.1490804672241	-0.449014138150528\\
% 37.1497647285461	-0.449134601065588\\
% 37.1498274326324	-0.44925584715593\\
 37.3611752510071	-0.449373512848658\\
% 37.361182641983	-0.449489789027107\\
% 37.3623020172119	-0.449594204599911\\
% 37.3623060703278	-0.449689745916729\\
 37.4374818325043	-0.44976599486568\\
% 37.43752617836	-0.449827409915681\\
% 37.6287002086639	-0.449911177522546\\
% 37.6287106990814	-0.449993943652379\\
 37.6287152290344	-0.450075602275622\\
% 37.628727388382	-0.450194312623314\\
% 37.8067099571228	-0.450303830232899\\
% 37.8067221164703	-0.450398480811426\\
 37.8067826747894	-0.450486868568362\\
% 37.8460576057434	-0.450549974305259\\
% 37.9141819000244	-0.45060750937096\\
% 37.9315175533295	-0.450700796154615\\
 38.0237917423248	-0.450792075833825\\
% 38.0862652778626	-0.450910652667476\\
% 38.086269569397	-0.451024160318844\\
% 38.1961585998535	-0.451127354356529\\
 38.196178150177	-0.451228595904655\\
% 38.3084940433502	-0.451310348936726\\
% 38.3085419654846	-0.451398108386802\\
% 38.378288936615	-0.451483747041193\\
 38.393293094635	-0.451567696163049\\
% 38.4296378612518	-0.451681431201302\\
% 38.5430137634277	-0.451790381053458\\
% 38.5430688381195	-0.451890999673484\\
 38.6240517616272	-0.451971744644164\\
% 38.6530174732208	-0.452036902436673\\
% 38.7179064273834	-0.452092137718323\\
% 38.7788311958313	-0.452164847404314\\
 38.7788407325745	-0.452235670746477\\
% 38.8368362903595	-0.45230721074788\\
% 38.8831307411194	-0.452425299232653\\
% 38.919997882843	-0.452534577663228\\
 38.9796266078949	-0.452630256871632\\
% 39.019500207901	-0.452728564284802\\
% 39.0791270256043	-0.452776863661131\\
% 39.1526405334473	-0.452793138353655\\
 39.4971506118774	-0.452783422345605\\
% 39.49718708992	-0.452793523606154\\
% 39.4972180843353	-0.452805624658699\\
% 39.4972497940064	-0.452942205768575\\
 39.497278881073	-0.453078551594847\\
% 39.4973082065582	-0.453207755813843\\
% 39.4974827289581	-0.453327336489377\\
% 39.65032787323	-0.453427731357108\\
 39.6503335952759	-0.453504966457278\\
% 39.6503650665283	-0.453558043224842\\
% 39.7096879005432	-0.453619470439848\\
% 39.7588674545288	-0.453679722788051\\
 39.8301410198212	-0.453804262066172\\
% 39.8644632816315	-0.453927322016095\\
% 39.926452589035	-0.45405008492692\\
% 39.9270951271057	-0.454158742488703\\
 40.0534979820251	-0.454248477144607\\
% 40.0535077571869	-0.454317733526365\\
% 40.2688049793243	-0.454371160726788\\
% 40.2689120292664	-0.454420988024129\\
 40.268930387497	-0.454472205246732\\
% 40.2689377784729	-0.454576775993445\\
% 40.4292718887329	-0.454682060932091\\
% 40.4293117046356	-0.454808440676102\\
 40.4293787002564	-0.45492948486473\\
% 40.4294297218323	-0.455028660475758\\
% 40.5301465511322	-0.455111809803749\\
% 40.564444732666	-0.455192168165284\\
 40.7577771663666	-0.455271442381041\\
% 40.7585477352142	-0.455337246503499\\
% 40.7585613250732	-0.455442762132414\\
% 40.7589544773102	-0.455551968053888\\
 41.0156762123108	-0.45566294467172\\
% 41.0157110214233	-0.455765931331495\\
% 41.0157479763031	-0.455858485520719\\
% 41.015776348114	-0.455938726727056\\
 41.0157904148102	-0.455999553272974\\
% 41.2113515853882	-0.456055250577539\\
% 41.2113830566406	-0.456114846011093\\
% 41.2114123821259	-0.45618010491669\\
 41.2114569664001	-0.456244497273689\\
% 41.3561818122864	-0.45630455434741\\
% 41.356210899353	-0.456355801281033\\
% 41.3562597751617	-0.456393869010903\\
 41.4521789073944	-0.456405464136843\\
% 41.4522032260895	-0.456402930013584\\
% 41.5460445404053	-0.456387474756874\\
% 41.6698898792267	-0.456372392467921\\
 42.0554856777191	-0.456377788433813\\
% 42.0562896251678	-0.456382584453851\\
% 42.056302022934	-0.456385227261578\\
% 42.0563089370728	-0.456383151542129\\
 42.0563282489777	-0.456375076512855\\
% 42.0563456535339	-0.456375076512855\\
% 42.056353521347	-0.456375076512855\\
% 42.0563606739044	-0.456375076512855\\
 42.0563671112061	-0.456375076512855\\
% 42.0563849925995	-0.456375076512855\\
% 42.7761535167694	-0.456375076512855\\
% 42.7761735439301	-0.456375076512855\\
 43.0857395648956	-0.456375076512855\\
% 43.0857460021973	-0.456375076512855\\
% 43.0857576847076	-0.456375076512855\\
% 43.2397584438324	-0.456375076512855\\
 43.2397701263428	-0.456375076512855\\
% 43.2397839546204	-0.456375076512855\\
% 43.2644345283508	-0.456375076512855\\
% 43.6193358421326	-0.456375076512855\\
 43.6193403720856	-0.456375076512855\\
% 43.6193439483643	-0.456375076512855\\
% 43.6193475246429	-0.456375076512855\\
% 43.7342502593994	-0.456375076512855\\
 43.7342824459076	-0.456375076512855\\
% 44.1866940975189	-0.456375076512855\\
% 44.1866991043091	-0.456375076512855\\
% 44.2737109184265	-0.456375076512855\\
 44.2737159252167	-0.456375076512855\\
% 44.6853835105896	-0.456375076512855\\
% 44.6854300022125	-0.456375076512855\\
% 44.8329545974731	-0.456375076512855\\
 44.8329832077026	-0.456375076512855\\
% 44.8330263614655	-0.456375076512855\\
% 45.7841484069824	-0.456375076512855\\
% 45.78420753479	-0.456375076512855\\
 45.7842416286469	-0.456375076512855\\
% 45.7842862129211	-0.456375076512855\\
% 45.7843179225922	-0.456375076512855\\
% 45.7843596458435	-0.456375076512855\\
 45.7843942165375	-0.456375076512855\\
% 45.7844330787659	-0.456375076512855\\
% 45.7844631195068	-0.456375076512855\\
% 45.7844976902008	-0.456375076512855\\
 45.784534406662	-0.456375076512855\\
% 45.7846402645111	-0.456375076512855\\
% 46.0884601593018	-0.456375076512855\\
% 46.0884670734406	-0.456375076512855\\
% 46.1257688522339	-0.456375076512855\\
 46.1662382602692	-0.456375076512855\\
% 46.2153467655182	-0.456375076512855\\
% 46.6415497779846	-0.456375076512855\\
% 46.7479912757874	-0.456375076512855\\
 46.7480005741119	-0.456375076512855\\
% 46.9038605213165	-0.456375076512855\\
% 46.9188446521759	-0.456375076512855\\
% 47.030423116684	-0.456375076512855\\
 47.8224229335785	-0.456375076512855\\
% 47.9271270751953	-0.456375076512855\\
% 48.2469987392426	-0.456375076512855\\
% 48.2470311641693	-0.456375076512855\\
 48.247061920166	-0.456375076512855\\
% 48.6642553329468	-0.456375076512855\\
% 48.9163457870483	-0.456375076512855\\
% 48.9163507938385	-0.456375076512855\\
 48.9173902988434	-0.456375076512855\\
% 49.5241639137268	-0.456375076512855\\
% 49.5241684436798	-0.456375076512855\\
% 49.525897693634	-0.456375076512855\\
 49.5259534835815	-0.456375076512855\\
% 50.0419322967529	-0.456375076512855\\
% 51.5745617866516	-0.456375076512855\\
% 51.5749794960022	-0.456375076512855\\
 51.5749883174896	-0.456375076512855\\
% 51.8694738864899	-0.456375076512855\\
% 51.869504404068	-0.456375076512855\\
% 51.8695332527161	-0.456375076512855\\
 51.869561624527	-0.456375076512855\\
% 51.8696019172668	-0.456375076512855\\
% 51.8696481704712	-0.456375076512855\\
% 53.5224811553955	-0.456375076512855\\
 53.5224897384644	-0.456375076512855\\
% 53.5224947452545	-0.456375076512855\\
% 53.5224995136261	-0.456375076512855\\
% 53.5225045204163	-0.456375076512855\\
 53.5225088119507	-0.456375076512855\\
% 53.5225133419037	-0.456375076512855\\
% 53.5225185871124	-0.456375076512855\\
% 53.52253074646	-0.456375076512855\\
 53.5225345611572	-0.456375076512855\\
% 53.5225390911102	-0.456375076512855\\
% 53.5225431442261	-0.456375076512855\\
% 53.8357407569885	-0.456375076512855\\
 53.8357474327087	-0.456375076512855\\
% 53.8357564926147	-0.456375076512855\\
% 53.8360311508179	-0.456375076512855\\
% 53.8361589431763	-0.456375076512855\\
 53.8363892555237	-0.456375076512855\\
% 54.9593457698822	-0.456375076512855\\
% 54.9593765258789	-0.456375076512855\\
% 54.9594101428986	-0.456375076512855\\
 54.9594618797302	-0.456375076512855\\
% 54.9596836090088	-0.456375076512855\\
% 54.9597208023071	-0.456375076512855\\
% 54.9597517967224	-0.456375076512855\\
 54.9597842216492	-0.456375076512855\\
% 54.9599391937256	-0.456375076512855\\
% 54.9599730491638	-0.456375076512855\\
% 54.9600042819977	-0.456375076512855\\
 54.9601048946381	-0.456375076512855\\
% 55.7669550895691	-0.456375076512855\\
% 55.766986322403	-0.456375076512855\\
% 55.767018032074	-0.456375076512855\\
 55.7670630931854	-0.456375076512855\\
% 55.7670952796936	-0.456375076512855\\
% 55.7671293735504	-0.456375076512855\\
% 55.767174911499	-0.456375076512855\\
 55.7672061443329	-0.456375076512855\\
% 55.76724858284	-0.456375076512855\\
% 55.7672833919525	-0.456375076512855\\
% 55.7673131942749	-0.456375076512855\\
 55.7673434734345	-0.456375076512855\\
% 55.7673713684082	-0.456375076512855\\
% 55.7674064159393	-0.456375076512855\\
% 55.7674335956574	-0.456375076512855\\
 55.7674622058868	-0.456375076512855\\
% 55.7675101280212	-0.456375076512855\\
% 56.4338857650757	-0.456375076512855\\
% 56.4339029312134	-0.456375076512855\\
 56.4339148521423	-0.456375076512855\\
% 56.4339248657227	-0.456375076512855\\
% 56.4339437007904	-0.456375076512855\\
% 56.4340469360352	-0.456375076512855\\
 56.4340550422668	-0.456375076512855\\
% 56.4340698242188	-0.456375076512855\\
% 56.4340769767761	-0.456375076512855\\
% 56.4340853214264	-0.456375076512855\\
 56.4340981960297	-0.456375076512855\\
% 56.4342569828033	-0.456375076512855\\
% 56.4342746257782	-0.456375076512855\\
% 56.8145236492157	-0.456375076512855\\
 56.9261416912079	-0.456375076512855\\
% 57.0169128894806	-0.456375076512855\\
% 57.0169181346893	-0.456375076512855\\
% 58.5536970615387	-0.456375076512855\\
 58.5537037372589	-0.456375076512855\\
% 58.553709936142	-0.456375076512855\\
% 58.5537163734436	-0.456375076512855\\
% 58.5537225723267	-0.456375076512855\\
 58.5537290096283	-0.456375076512855\\
% 58.5537354469299	-0.456375076512855\\
% 58.5537423610687	-0.456375076512855\\
% 58.5537490367889	-0.456375076512855\\
 58.5537545204163	-0.456375076512855\\
% 58.5537600040436	-0.456375076512855\\
% 58.5537657260895	-0.456375076512855\\
% 58.5537721633911	-0.456375076512855\\
 58.5537781238556	-0.456375076512855\\
% 58.5537843227387	-0.456375076512855\\
% 58.5537902832031	-0.456375076512855\\
% 58.5551464080811	-0.456375076512855\\
 58.5551602363586	-0.456375076512855\\
% 58.5551681041718	-0.456375076512855\\
% 58.5551752567291	-0.456375076512855\\
% 58.5551819324493	-0.456375076512855\\
 58.9172229290009	-0.456375076512855\\
% 58.9172555923462	-0.456375076512855\\
% 58.9172882556915	-0.456375076512855\\
% 58.9172951698303	-0.456375076512855\\
 58.9173254489899	-0.456375076512855\\
% 58.9173306941986	-0.456375076512855\\
% 58.9173583507538	-0.456375076512855\\
% 58.9174062728882	-0.456375076512855\\
 59.1581761360168	-0.456375076512855\\
% 59.1582495689392	-0.456375076512855\\
% 59.1583513736725	-0.456375076512855\\
% 59.1584672451019	-0.456375076512855\\
 59.3863837242126	-0.456375076512855\\
% 59.3863975524902	-0.456375076512855\\
% 59.3864140033722	-0.456375076512855\\
% 59.3864302158356	-0.456375076512855\\
 59.3864404678345	-0.456375076512855\\
% 59.8614983081818	-0.456375076512855\\
% 59.865093421936	-0.456375076512855\\
% 59.8651005744934	-0.456375076512855\\
 59.8657540798187	-0.456375076512855\\
% 59.8657629013062	-0.456375076512855\\
% 59.8659624576569	-0.456375076512855\\
% 59.8659700870514	-0.456375076512855\\
 59.8662604808807	-0.456375076512855\\
% 59.8662702560425	-0.456375076512855\\
% 60.6861924648285	-0.456375076512855\\
% 60.686212015152	-0.456375076512855\\
 60.6862289428711	-0.456375076512855\\
% 60.6862382411957	-0.456375076512855\\
% 60.686255645752	-0.456375076512855\\
% 60.6862613677979	-0.456375076512855\\
 60.6882712364197	-0.456375076512855\\
% 60.6882829189301	-0.456375076512855\\
% 60.6882910251617	-0.456375076512855\\
% 60.6882996082306	-0.456375076512855\\
 60.6883103370667	-0.456375076512855\\
% 60.6883182048798	-0.456375076512855\\
% 60.6883239269257	-0.456375076512855\\
% 60.6883284568787	-0.456375076512855\\
 60.6883360862732	-0.456375076512855\\
% 60.6883420467377	-0.456375076512855\\
% 60.6883496761322	-0.456375076512855\\
% 60.9728793621063	-0.456375076512855\\
 60.9912290096283	-0.456375076512855\\
% 60.9912459373474	-0.456375076512855\\
% 60.9912766933441	-0.456375076512855\\
% 60.9912983894348	-0.456375076512855\\
 60.9992479801178	-0.456375076512855\\
};
\addlegendentry{$\theta_{x_{p}}$}

\addplot [color=mycolor4, line width=1.4pt]
  table[row sep=crcr]{%
 -2.3985894203186	0\\
% -2.34845380783081	0\\
% -2.29882984161377	0\\
% -2.24259977340698	0\\
 -2.1981945514679	0\\
% -2.14788823127747	0\\
% -2.09815530776978	0\\
% -2.04817252159119	0\\
% -1.99881820678711	0\\
 -1.94862990379333	0\\
% -1.87693266868591	0\\
% -1.84271411895752	0\\
% -1.79806022644043	0\\
 -1.74720959663391	0\\
% -1.69869332313538	0\\
% -1.6482563495636	0\\
% -1.59839563369751	0\\
 -1.548703956604	0\\
% -1.49876074790955	0\\
% -1.44885139465332	0\\
% -1.3982711315155	0\\
 -1.34799294471741	0\\
% -1.29795436859131	0\\
% -1.2485387802124	0\\
% -1.19843416213989	0\\
 -1.14853506088257	0\\
% -1.09835534095764	0\\
% -1.04829435348511	0\\
% -0.998184490203857	0\\
 -0.948882627487182	0\\
% -0.898640441894531	0\\
% -0.847953605651855	0\\
% -0.798753547668457	0\\
 -0.748502779006958	0\\
% -0.698466348648071	0\\
% -0.648822116851806	0\\
% -0.597503709793091	0\\
 -0.548140096664429	0\\
% -0.498594331741333	0\\
% -0.4485755443573	0\\
% -0.398829984664917	0\\
 -0.348826217651367	0\\
% -0.298781204223633	0\\
% -0.248498249053955	0\\
% -0.197533178329468	0\\
 -0.147660779953003	0\\
% -0.0983634471893309	0\\
% -0.0481057643890379	0\\
% 0.0019752502441408	0\\
 0.0514363765716555	0\\
% 0.113763284683228	0\\
% 0.15691466331482	0\\
% 0.234845352172852	0\\
 0.308734607696533	0\\
% 0.308903646469116	0\\
% 0.360757303237915	0\\
% 0.415858459472656	0\\
 0.497472476959229	-1.81919548833159\\
% 0.559248399734497	-2.50719258745892\\
% 0.568476867675781	-2.87387457444413\\
% 0.604575109481812	-3.09779897230283\\
 0.723840665817261	-3.244560440893\\
% 0.723884773254395	-3.34462599833751\\
% 0.853632164001465	-3.41138424813039\\
% 0.853860330581665	-3.45603656211688\\
 0.859521579742432	-3.48671824457961\\
% 0.922651720046997	-3.51244996395167\\
% 0.971385669708252	-3.53540275980504\\
% 1.02831001281738	-3.55527057630979\\
 1.07039852142334	-3.57181073038555\\
% 1.11899919509888	-3.58267917060266\\
% 1.16864533424377	-3.59104193738358\\
% 1.22722692489624	-3.59891231659867\\
 1.32403750419617	-3.60695256420968\\
% 1.3240451335907	-3.61585722285736\\
% 1.43628258705139	-3.62314745938193\\
% 1.4363646030426	-3.6269104968228\\
 1.6345018863678	-3.62589275529717\\
% 1.63452954292297	-3.62232304481449\\
% 1.63456315994263	-3.61880687639314\\
% 1.63461179733276	-3.61657275324342\\
 1.67959208488464	-3.61487198989198\\
% 1.74344153404236	-3.61299090756438\\
% 1.80315823554993	-3.60992092423567\\
% 1.8214485168457	-3.60696552755599\\
 1.86181755065918	-3.60400882696922\\
% 1.95309824943542	-3.60175368012096\\
% 1.98877353668213	-3.59868241947606\\
% 2.10328192710876	-3.59374310221119\\
 2.10334939956665	-3.58670393687225\\
% 2.2043318271637	-3.57849209073083\\
% 2.20438261032105	-3.57162507767043\\
% 2.20442314147949	-3.56639084543531\\
 2.31627173423767	-3.56153907569114\\
% 2.31644148826599	-3.55644010755532\\
% 2.35974855422974	-3.54949071766532\\
% 2.40763850212097	-3.54010134169675\\
 2.58187789916992	-3.52950852847243\\
% 2.58190841674805	-3.52069920853319\\
% 2.58211798667908	-3.51423722595246\\
% 2.96352286338806	-3.50829392141441\\
 2.96353478431702	-3.50132248673617\\
% 2.96354002952576	-3.49257419101923\\
% 2.96354813575745	-3.48282910107355\\
% 2.96359629631042	-3.47147348192539\\
 2.96360988616943	-3.46032887797628\\
% 2.9636182308197	-3.45106015044621\\
% 2.96363134384155	-3.44360117726683\\
% 3.06343312263489	-3.4362995839947\\
 3.06350989341736	-3.42846356260452\\
% 3.14400072097778	-3.41815832374232\\
% 3.24244160652161	-3.40461070520632\\
% 3.24249835014343	-3.38840059809445\\
 3.38695282936096	-3.37208665934486\\
% 3.38704962730408	-3.35781059898\\
% 3.38712425231934	-3.34717887730403\\
% 3.50973148345947	-3.33814333748342\\
 3.50977296829224	-3.32944476903322\\
% 3.50982351303101	-3.31796097736515\\
% 3.60157866477966	-3.30180467296486\\
% 3.61497135162354	-3.28013615939926\\
 3.66265697479248	-3.2549581229232\\
% 3.77121181488037	-3.22959094345242\\
% 3.77127523422241	-3.20565309971971\\
% 3.88388056755066	-3.18121346362932\\
 3.88406581878662	-3.15624165887857\\
% 4.01174230575562	-3.12848245268833\\
% 4.01174683570862	-3.0969968616464\\
% 4.01175088882446	-3.06285104644849\\
 4.10342116355896	-3.0296582602823\\
% 4.10781664848328	-2.99783009079511\\
% 4.22873516082764	-2.96705603800547\\
% 4.22877926826477	-2.93693087013162\\
 4.3162703037262	-2.90489128065383\\
% 4.31628270149231	-2.86811570950704\\
% 4.36952753067017	-2.82659679660537\\
% 4.49391932487488	-2.78400733233684\\
 4.4939248085022	-2.74357773944575\\
% 4.50986928939819	-2.70530852635056\\
% 4.60001916885376	-2.66870219970042\\
% 4.6112982749939	-2.6323521629638\\
 4.713267993927	-2.59321987798467\\
% 4.71331377029419	-2.54926233241531\\
% 4.84796900749207	-2.50377865424707\\
% 4.84803886413574	-2.45937747054995\\
 4.9620210647583	-2.41804135105122\\
% 4.96202535629272	-2.37853175079181\\
% 4.96383280754089	-2.3400837021909\\
% 5.00998063087463	-2.29984238966972\\
 5.05557627677917	-2.25828469576481\\
% 5.10753936767578	-2.2124069279198\\
% 5.19693036079407	-2.15969506682814\\
% 5.24160213470459	-2.10708382838493\\
 5.2747401714325	-2.05567922659793\\
% 5.47130937576294	-2.0074750644726\\
% 5.47131628990173	-1.9660536196061\\
% 5.47132034301758	-1.92771919243751\\
 5.47196359634399	-1.89099778974378\\
% 5.92306728363037	-1.85194182795931\\
% 5.92307133674622	-1.81143135674392\\
% 5.92308039665222	-1.76909291149423\\
% 5.92308468818665	-1.72752239939473\\
 5.92308826446533	-1.6887992837419\\
% 5.92309231758118	-1.65086102643181\\
% 5.92309613227844	-1.6138926474614\\
% 5.92310948371887	-1.57602733287058\\
 5.92311329841614	-1.53896945248016\\
% 5.99617094993591	-1.50323361618757\\
% 6.01983733177185	-1.46959700568004\\
% 6.06683416366577	-1.43769633852207\\
 6.13510556221008	-1.4063830490777\\
% 6.17488260269165	-1.3750402911428\\
% 6.34516496658325	-1.34130755658407\\
% 6.34567971229553	-1.30707521850854\\
 6.34579391479492	-1.27200350665225\\
% 6.48088450431824	-1.23701062829059\\
% 6.48123211860657	-1.20418659201096\\
% 6.48173136711121	-1.17399284607563\\
 6.55087299346924	-1.14635440805796\\
% 6.60183663368225	-1.12102106513157\\
% 6.61844868659973	-1.09542821992909\\
% 6.6699878692627	-1.06937954761088\\
 6.73549313545227	-1.03883857613437\\
% 6.90011711120605	-1.00829028656881\\
% 6.90014715194702	-0.97853724565357\\
% 6.90017552375793	-0.951906224982849\\
 6.96017570495605	-0.927381678524398\\
% 6.96023197174072	-0.9047268246768\\
% 7.02018661499023	-0.882017183754215\\
% 7.08605260848999	-0.859491321047244\\
 7.14995141029358	-0.837297576178571\\
% 7.2400710105896	-0.81439701331692\\
% 7.24012393951416	-0.792337124635196\\
% 7.35665650367737	-0.771012889591475\\
 7.35666365623474	-0.750478752608615\\
% 7.50300760269165	-0.730406543182653\\
% 7.50304217338562	-0.711486811249415\\
% 7.50325675010681	-0.69103920030102\\
 7.55728693008423	-0.671503372355801\\
% 7.58989186286926	-0.652986534474621\\
% 7.60993332862854	-0.635264919396832\\
% 7.71798748970032	-0.618055839587214\\
 7.71803326606751	-0.601234307519917\\
% 7.8084406375885	-0.58507454624305\\
% 7.8084451675415	-0.567385050448138\\
% 7.86788339614868	-0.550180323758468\\
 7.98911018371582	-0.533587631882256\\
% 7.98919291496277	-0.517656558778071\\
% 8.03272266387939	-0.502472723246683\\
% 8.08047933578491	-0.487342694680592\\
 8.12581033706665	-0.472718940512095\\
% 8.19591255187988	-0.457216040378626\\
% 8.21851797103882	-0.441972818360227\\
% 8.32182474136352	-0.426647053158376\\
 8.32182974815369	-0.41215226877739\\
% 8.37021417617798	-0.398547538579805\\
% 8.40591187477112	-0.385832223305442\\
% 8.59654874801636	-0.373873687344712\\
 8.59659929275513	-0.362623418677686\\
% 8.59668583869934	-0.350992896357639\\
% 8.75598998069763	-0.339538222491683\\
% 8.75599665641785	-0.328232415887214\\
 8.75600190162659	-0.316791956011002\\
% 8.75602097511291	-0.305009334745591\\
% 9.02396960258484	-0.294006981736857\\
% 9.02399797439575	-0.284008472832284\\
 9.02402396202087	-0.274737597534113\\
% 9.02404947280884	-0.265746000372928\\
% 9.02408738136291	-0.257206183092421\\
% 9.34571809768677	-0.248624491912778\\
 9.34574980735779	-0.240102100532567\\
% 9.34578461647034	-0.231557131483896\\
% 9.34599132537842	-0.222934409359368\\
% 9.34602160453796	-0.214647767556016\\
 9.34606547355652	-0.206387485869527\\
% 9.55178208351135	-0.198798370122404\\
% 9.55178709030151	-0.191688564341348\\
% 9.55179281234741	-0.184884745341151\\
 9.55179829597473	-0.178476822209177\\
% 9.61852855682373	-0.17214549392429\\
% 9.61856646537781	-0.166646043947821\\
% 9.6634735584259	-0.161296155706623\\
 9.70841760635376	-0.15597038216265\\
% 9.81058926582336	-0.151541480820924\\
% 9.81059379577637	-0.147342814088461\\
% 9.91963858604431	-0.143517295907714\\
 9.91964693069458	-0.139625704523496\\
% 9.97360939979553	-0.136101774937742\\
% 10.0412783145905	-0.132900928244453\\
% 10.1883935451508	-0.12909207769917\\
 10.1884243011475	-0.126231727193044\\
% 10.1884679317474	-0.123361490891966\\
% 10.2828607082367	-0.120442376466599\\
% 10.2828685760498	-0.117670723461288\\
 10.3878936290741	-0.115718178222778\\
% 10.3879322528839	-0.114084705701316\\
% 10.477405500412	-0.112799243579047\\
% 10.4774195671082	-0.11168923128298\\
 10.5185296058655	-0.110711840877229\\
% 10.560958814621	-0.109401138480735\\
% 10.6081099033356	-0.108105218203491\\
% 10.6624338150024	-0.107154470164005\\
 10.7120904445648	-0.106157683948368\\
% 10.8078364849091	-0.105202051632205\\
% 10.8310131549835	-0.104459614306506\\
% 10.86188621521	-0.103981837277388\\
 10.9186720371246	-0.103683336601989\\
% 10.9605512142181	-0.103666175542457\\
% 11.0521792888641	-0.103606505297648\\
% 11.0872270584106	-0.103477144431054\\
 11.1295003414154	-0.103255798746432\\
% 11.1574806690216	-0.102969467382536\\
% 11.2049316883087	-0.102882666157711\\
% 11.264439535141	-0.103027472066515\\
 11.3169879436493	-0.103553731574493\\
% 11.3666297912598	-0.104479844746265\\
% 11.4129797935486	-0.105344475468883\\
% 11.487088394165	-0.106170343824942\\
 11.5034069538116	-0.106751375318709\\
% 11.5847613334656	-0.107246415556801\\
% 11.6668142795563	-0.107850949966178\\
% 11.6668383598328	-0.108653547458744\\
 11.756715965271	-0.109601864337037\\
% 11.7568525791168	-0.1107713773863\\
% 11.8283314228058	-0.112045495216307\\
% 11.861048412323	-0.113528881462798\\
 11.9123050689697	-0.114888165370758\\
% 11.9995982170105	-0.115998014594425\\
% 12.0273801803589	-0.117099758880158\\
% 12.0843092918396	-0.118311958351796\\
 12.1415590763092	-0.119472030072075\\
% 12.1802153110504	-0.120728159431565\\
% 12.2485608577728	-0.122370227209103\\
% 12.2750324726105	-0.123868120203952\\
 12.3711525917053	-0.125240563722173\\
% 12.3711673736572	-0.126705245642142\\
% 12.584521484375	-0.128213159579673\\
% 12.5848187923431	-0.12982333760533\\
 12.5848257064819	-0.131309518411854\\
% 12.5848338127136	-0.132748385307423\\
% 12.6255592823029	-0.134430116596832\\
% 12.6870493412018	-0.135951332674949\\
 12.7069229602814	-0.137380517421889\\
% 12.7557956695557	-0.138720889036904\\
% 12.8823637485504	-0.140064911762238\\
% 12.882368516922	-0.141453455596547\\
 12.9974901199341	-0.142691355029228\\
% 12.9975285053253	-0.143929552591999\\
% 13.0484261035919	-0.145311835050052\\
% 13.0905348777771	-0.146537217256082\\
 13.1565808773041	-0.147972051606075\\
% 13.2565376281738	-0.149345530379833\\
% 13.2565733909607	-0.150734865145978\\
% 13.2757715702057	-0.152079145011498\\
 13.3401882171631	-0.153433782801244\\
% 13.4026224136353	-0.154837642376833\\
% 13.4468314170837	-0.156263752921291\\
% 13.4851455211639	-0.157545365888154\\
 13.5162672519684	-0.158694725281748\\
% 13.5994228839874	-0.15973923861884\\
% 13.6809041023254	-0.161549997549059\\
% 13.6809529781342	-0.163342540124859\\
 13.760929775238	-0.165140568254174\\
% 13.7609891414642	-0.16693840769949\\
% 13.8658623218536	-0.168712571708056\\
% 13.8659095287323	-0.170495814256924\\
 13.9651402950287	-0.17216176722593\\
% 13.9651853561401	-0.174153270491388\\
% 14.013670873642	-0.176004198497026\\
% 14.0695306777954	-0.177820273361021\\
 14.1660151004791	-0.179863795405936\\
% 14.1674413204193	-0.181880829484385\\
% 14.3973986625671	-0.183671218324662\\
% 14.3974039077759	-0.185491303380331\\
 14.397408914566	-0.187407032658143\\
% 14.3974151134491	-0.189189971859037\\
% 14.4515306472778	-0.190792647321587\\
% 14.4809290885925	-0.192750785078942\\
 14.507790517807	-0.19463524270202\\
% 14.6076409339905	-0.196730443385945\\
% 14.6330356121063	-0.198814156902245\\
% 14.7221893787384	-0.200778189337257\\
 14.7222127437592	-0.202691311515423\\
% 14.8223895549774	-0.204452955069769\\
% 14.8226484775543	-0.206252607061174\\
% 14.8592681407928	-0.207876796063374\\
 14.9488486766815	-0.209350640779348\\
% 14.9796115875244	-0.211134752122348\\
% 15.0424162864685	-0.212889600915446\\
% 15.0784852027893	-0.214618808891373\\
 15.1091608524323	-0.216542643336084\\
% 15.1938590526581	-0.218356530376241\\
% 15.2520548820496	-0.220206277919146\\
% 15.3033129692078	-0.221910743598585\\
 15.3873831748962	-0.223490773718083\\
% 15.389208984375	-0.225197439521537\\
% 15.5462734222412	-0.226802431619092\\
% 15.5463118076324	-0.228349624390603\\
 15.5463530540466	-0.230025397238762\\
% 15.6087738990784	-0.231660139288635\\
% 15.6088122844696	-0.233224046372243\\
% 15.8284608840942	-0.234994911461229\\
 15.82850689888	-0.236617201111414\\
% 15.8285231113434	-0.238098089584525\\
% 15.8290633678436	-0.239884683491653\\
% 15.9087426185608	-0.24160167253541\\
 16.0841259479523	-0.243413459322028\\
% 16.0841304779053	-0.245132395162443\\
% 16.0841798305511	-0.246764706809188\\
% 16.0844277858734	-0.248493881262277\\
 16.1285621643066	-0.250071684377474\\
% 16.1920978546143	-0.251550927991161\\
% 16.2366220474243	-0.253213377640634\\
% 16.2680780410767	-0.254763376649862\\
 16.3074433326721	-0.256222992206972\\
% 16.3852943897247	-0.257866551630343\\
% 16.427689743042	-0.259431296409673\\
% 16.4832610607147	-0.260916673793304\\
 16.5667356967926	-0.262597890573545\\
% 16.5667421340942	-0.264170921538607\\
% 16.6510304927826	-0.266006104626172\\
% 16.6845106601715	-0.267723785083035\\
 16.7282096862793	-0.269344299010584\\
% 16.7647928714752	-0.270946877388326\\
% 16.8415979862213	-0.272463082514776\\
% 16.9638578414917	-0.273901891992978\\
 16.9638618946075	-0.275268150298928\\
% 16.963866186142	-0.276688047041887\\
% 17.0969151973724	-0.278391170876318\\
% 17.0969202041626	-0.279974656928488\\
 17.1762506484985	-0.281439930067023\\
% 17.1763615131378	-0.282806340765319\\
% 17.2229020118713	-0.284263497336042\\
% 17.3326086521149	-0.28564586947499\\
 17.3326153278351	-0.286953687494588\\
% 17.473627281189	-0.28819269879645\\
% 17.4736518383026	-0.289545951146408\\
% 17.4736754417419	-0.29080393540184\\
 17.5113889694214	-0.292299715100285\\
% 17.5604962825775	-0.293650008229577\\
% 17.6228019714355	-0.295137408339457\\
% 17.6881763458252	-0.296427713501686\\
 17.8001989841461	-0.297813442669081\\
% 17.8002123355865	-0.299069271881308\\
% 17.8233386993408	-0.30055809733247\\
% 17.9228260040283	-0.301981530937752\\
 17.9228312492371	-0.303605583452253\\
% 18.0239114284515	-0.305119911038759\\
% 18.023930978775	-0.306567529653478\\
% 18.2090794563293	-0.307876676810537\\
 18.2091359615326	-0.309070280741821\\
% 18.209382724762	-0.310162441312315\\
% 18.2096747875214	-0.311264053536689\\
% 18.3486201286316	-0.312264269603787\\
 18.3489136219025	-0.313630954269314\\
% 18.4253313064575	-0.314944885589512\\
% 18.4253766059875	-0.316178597049316\\
% 18.5247117996216	-0.317470208224108\\
 18.5248689174652	-0.318661033039916\\
% 18.5989043235779	-0.319728930273072\\
% 18.6113500118256	-0.320812666363423\\
% 18.6828867912292	-0.321778840573756\\
 18.7189335346222	-0.322637498166301\\
% 18.7590083599091	-0.323402101858761\\
% 18.8376090049744	-0.324350604333375\\
% 18.8578114032745	-0.325245101425452\\
 18.9627770900726	-0.326097110943522\\
% 18.9628245353699	-0.327313881570191\\
% 19.0279707431793	-0.328482271073085\\
% 19.0604125976562	-0.329595595135501\\
 19.1335758686066	-0.330754312021568\\
% 19.2351016521454	-0.331823173210537\\
% 19.2352606773376	-0.3327843647549\\
% 19.5428084850311	-0.333623659003337\\
 19.5428378105164	-0.334563884647082\\
% 19.5428936004639	-0.335404121367532\\
% 19.5429250717163	-0.336197887107858\\
% 19.5429551124573	-0.337229962520041\\
 19.5429982662201	-0.338203664901627\\
% 19.9834553718567	-0.339123430288822\\
% 19.9834630012512	-0.340235968039181\\
% 19.9834668159485	-0.341283936011521\\
 19.9834703922272	-0.342225487815611\\
% 19.9834739685059	-0.343202622146308\\
% 19.983477306366	-0.344188403212613\\
% 19.9834856510162	-0.345040456826666\\
 19.9834892272949	-0.346051237652205\\
% 19.9834928035736	-0.347014363520827\\
% 20.048566532135	-0.347923845616634\\
% 20.1035956859589	-0.349011472290201\\
 20.1756994247437	-0.350037608701072\\
% 20.1757237434387	-0.350960196833938\\
% 20.2920464992523	-0.351898353711377\\
% 20.2920505523682	-0.352720977835858\\
 20.3619036197662	-0.353575116019968\\
% 20.4108409404755	-0.354299339166047\\
% 20.4923514842987	-0.354931916417826\\
% 20.4923595905304	-0.35574942016035\\
 20.5578968048096	-0.356529704017078\\
% 20.6461452960968	-0.357246088354586\\
% 20.6461510181427	-0.358293242282059\\
% 20.7409729480743	-0.359233387203091\\
 20.775611114502	-0.360084886082333\\
% 20.8289811134338	-0.360807083131661\\
% 20.855291557312	-0.361404335254534\\
% 20.8881897449493	-0.361951596494379\\
 20.9274053096771	-0.362422393376761\\
% 20.9642960548401	-0.363165179082849\\
% 21.0089377880096	-0.363880634594985\\
% 21.0846263885498	-0.364564073131788\\
 21.1657716751099	-0.365705258837657\\
% 21.1659259319305	-0.366778208532764\\
% 21.2478870868683	-0.367795818516903\\
% 21.287251663208	-0.368710177468003\\
 21.3467800140381	-0.369469766973097\\
% 21.3878879070282	-0.370160080216735\\
% 21.4673904895782	-0.370748126029987\\
% 21.4674722671509	-0.371602586495271\\
 21.7880739688873	-0.372370212443457\\
% 21.7880997180939	-0.373107953794111\\
% 21.7881242752075	-0.37407880945382\\
% 21.788148355484	-0.374985086294814\\
% 21.788175535202	-0.37578832193438\\
 21.7882205963135	-0.376572574300212\\
% 21.9344484329224	-0.377266817740612\\
% 21.9345077991486	-0.377878942156187\\
% 21.9346115112305	-0.37857505453249\\
 22.1828016757965	-0.379197149946236\\
% 22.1828560352325	-0.380017322013352\\
% 22.1828851222992	-0.380754728544971\\
% 22.1829120635986	-0.381679854287349\\
 22.1829561710358	-0.382511913592182\\
% 22.2871834754944	-0.383289923228055\\
% 22.2872302055359	-0.383948088898663\\
% 22.3290242671967	-0.384633858984429\\
 22.3717877388	-0.385220744857818\\
% 22.4189037799835	-0.3857249515628\\
% 22.508477640152	-0.386183416393081\\
% 22.579109621048	-0.386886098393383\\
 22.5791170120239	-0.387550766862425\\
% 22.6909491539001	-0.388413195487587\\
% 22.6910020828247	-0.389228960158537\\
% 23.0378574848175	-0.389957335153948\\
 23.0378658294678	-0.390682951684671\\
% 23.0378715515137	-0.391309488244538\\
% 23.0378772735596	-0.391849548395928\\
% 23.0378908634186	-0.392528951555917\\
 23.0379697799683	-0.39312506114479\\
% 23.0379788398743	-0.393658752216641\\
% 23.1805610179901	-0.394399098773469\\
% 23.1805998802185	-0.395059766713722\\
 23.2553226470947	-0.395825745095326\\
% 23.3742901802063	-0.396555289028516\\
% 23.3742956638336	-0.397329723183379\\
% 23.4069616317749	-0.398033651347824\\
 23.436697435379	-0.398704630069421\\
% 23.4717108726501	-0.399356562489391\\
% 23.4726278305054	-0.399921260655933\\
% 23.5186371326447	-0.400449862582171\\
 23.5845655918121	-0.400921144251022\\
% 23.6520792961121	-0.401499422581988\\
% 23.7337271690369	-0.402042691858288\\
% 23.7337564945221	-0.402698509767816\\
 23.7942924022675	-0.403313640067541\\
% 23.8527013778687	-0.40389891545323\\
% 23.8903464794159	-0.404545069991709\\
% 23.9444241046906	-0.405136485127979\\
 24.0006498813629	-0.405690559534609\\
% 24.0766436576843	-0.406275522886506\\
% 24.0766481876373	-0.406776708836851\\
% 24.1547135829926	-0.407197087205395\\
 24.1547355175018	-0.407697045963864\\
% 24.2052554607391	-0.408128585300647\\
% 24.3249959468842	-0.408730490773351\\
% 24.3469130516052	-0.40923791358383\\
 24.3948239803314	-0.409689406425787\\
% 24.4395904064178	-0.410244870227281\\
% 24.4598485946655	-0.410756899795951\\
% 24.5525354862213	-0.4111934186762\\
 24.5862226009369	-0.411554258898523\\
% 24.6285922050476	-0.412085214288137\\
% 24.8560425758362	-0.412575149663269\\
% 24.8560881137848	-0.413223390742459\\
 24.8566789150238	-0.413802764935211\\
% 24.8570489406586	-0.414329788104254\\
% 24.9618987560272	-0.41480062008479\\
% 24.9619016170502	-0.415325537466376\\
 24.9667996883392	-0.415792382293029\\
% 25.1297199249268	-0.416378124143726\\
% 25.129741859436	-0.416848386103345\\
% 25.1297678470612	-0.417362072974234\\
 25.1699120521545	-0.417809919403394\\
% 25.3255557537079	-0.418395196355521\\
% 25.3258833408356	-0.418961345458305\\
% 25.4633335590363	-0.41951863691922\\
 25.4633907794952	-0.420180432129833\\
% 25.4634375095367	-0.420798474549811\\
% 26.2949263572693	-0.42139099676848\\
% 26.2949594974518	-0.421917982436173\\
 26.2949900150299	-0.42238817166521\\
% 26.2950171947479	-0.422797715447937\\
% 26.2950567722321	-0.423304913650288\\
% 26.2950622558594	-0.423931112348312\\
 26.2950939655304	-0.424524533407102\\
% 26.2951228141785	-0.425047121261328\\
% 26.2951516628265	-0.42553427409826\\
% 26.2951850414276	-0.425945792026935\\
 26.2952184200287	-0.426385427175111\\
% 26.2952496528625	-0.426772163845595\\
% 26.2952785015106	-0.427228451424329\\
% 26.2953056812286	-0.427660054873747\\
 26.2953350067139	-0.428047660638585\\
% 26.2953686237335	-0.428538382697546\\
% 26.2954256057739	-0.428947581050341\\
% 26.5964006900787	-0.429276978064737\\
 26.5964049816132	-0.429637420314967\\
% 26.5964097499847	-0.429893913919441\\
% 26.5964138031006	-0.4303125689022\\
% 26.5964183330536	-0.430681849197487\\
 26.5964228630066	-0.431005189643598\\
% 26.7749821662903	-0.431336774935287\\
% 26.774987411499	-0.431838566931467\\
% 26.7755240917206	-0.432323523135607\\
 26.8470217704773	-0.432876726316465\\
% 26.8470828056335	-0.433375922970979\\
% 26.89402384758	-0.433793761681905\\
% 26.927468252182	-0.434137163575704\\
 27.2203940868378	-0.434400592162811\\
% 27.2204248428345	-0.434602040928224\\
% 27.2204520225525	-0.434915738023909\\
% 27.2204794406891	-0.435202280194972\\
 27.2205059051514	-0.435519482896188\\
% 27.2205748081207	-0.436099533153383\\
% 28.8658260822296	-0.436668944176961\\
% 28.8658315658569	-0.437125276460401\\
 28.8668162345886	-0.437492285418803\\
% 28.866832447052	-0.437809426209938\\
% 28.8669096946716	-0.43804879812763\\
% 28.8669151782989	-0.438238815114255\\
 28.8669273376465	-0.438575205066243\\
% 28.8669893264771	-0.438864949249405\\
% 28.8669952869415	-0.439110129640546\\
% 28.8670451164246	-0.439314405307362\\
 28.8670749187469	-0.439453021985456\\
% 28.8671030521393	-0.439668502174762\\
% 28.867130947113	-0.439802006758952\\
% 28.8671600341797	-0.439854618959134\\
 28.8671998500824	-0.439837351766094\\
% 28.8672048568726	-0.439692826359986\\
% 28.8672348976135	-0.43972299086859\\
% 28.8672766208649	-0.439672460981199\\
 28.8672813892364	-0.439604074246484\\
% 28.8674888134003	-0.43953593356488\\
% 28.8674966812134	-0.439532509833488\\
% 28.8675477027893	-0.440039696710462\\
 28.867590379715	-0.44056318035499\\
% 28.8676218509674	-0.441075688341733\\
% 28.8676533222198	-0.441587296396656\\
% 28.8676936149597	-0.442007428133116\\
 28.8676976680756	-0.442386918715839\\
% 28.8677289009094	-0.442692773184461\\
% 28.8677708625793	-0.442978270005483\\
% 28.8677784919739	-0.443184655447407\\
 28.8678075790405	-0.443354073950543\\
% 28.867866230011	-0.443533648203632\\
% 28.8679084300995	-0.443689787593006\\
% 29.2079612731934	-0.4438312527742\\
 29.208006811142	-0.443959564876948\\
% 29.2080163478851	-0.444045497784262\\
% 29.2080223083496	-0.444413523103094\\
% 29.2080304145813	-0.444698972836636\\
 29.2080440044403	-0.444925688032695\\
% 29.208407831192	-0.445088525339532\\
% 30.8635506153107	-0.445311847959784\\
% 30.8635580062866	-0.445472405953073\\
 30.8635627746582	-0.445585118470655\\
% 30.8635682582855	-0.445888093306058\\
% 30.8635725498199	-0.44614391682342\\
% 30.8635766029358	-0.446382679111032\\
 30.8635820865631	-0.446687950453244\\
% 30.8635863780975	-0.446970346750354\\
% 30.8635904312134	-0.447182861079572\\
% 30.8635944843292	-0.447494397501928\\
 30.8635987758636	-0.447712711948776\\
% 30.8636030673981	-0.447900596273394\\
% 30.8636075973511	-0.447960340580158\\
% 30.8636118888855	-0.447938149792438\\
 30.8636159420013	-0.447854101806048\\
% 30.8636209487915	-0.447652709490031\\
% 30.8636252403259	-0.447583996263011\\
% 30.8636300086975	-0.447553293824428\\
 30.8636340618134	-0.447503870218599\\
% 30.8636385917664	-0.447463881194902\\
% 30.8636426448822	-0.447466848205095\\
% 30.8636474132538	-0.448009743859004\\
 30.863651227951	-0.448560667554873\\
% 30.8636557579041	-0.449165666858215\\
% 30.8636600494385	-0.449767510728153\\
% 30.8636657714844	-0.450344239174486\\
 30.8636707782745	-0.450907973499693\\
% 30.863675069809	-0.451427984418785\\
% 30.8636786460876	-0.451746202845638\\
% 30.8636826992035	-0.45198502383543\\
 31.2550806522369	-0.452132426698681\\
% 31.2551092624664	-0.452292743988154\\
% 31.2551359653473	-0.452426146426809\\
% 31.2553131103516	-0.452552966709487\\
 31.2553185939789	-0.452911062602095\\
% 31.255323600769	-0.45324689893838\\
% 31.2553271770477	-0.453633678686482\\
% 31.2553925037384	-0.454000397091564\\
 31.3403334140778	-0.454303898093192\\
% 31.3403391361237	-0.45455123923501\\
% 31.5086352348328	-0.454750634290483\\
% 31.5086521625519	-0.454844605317036\\
 31.5086788654327	-0.45500553094936\\
% 31.6173405170441	-0.455148348084134\\
% 31.6173724651337	-0.455278320590598\\
% 31.6975693225861	-0.455406822175885\\
% 31.6975774288177	-0.455745387149009\\
 31.7641047954559	-0.456042741675181\\
% 31.7724740028381	-0.456329884665843\\
% 32.082110118866	-0.456598580708521\\
% 32.0831522464752	-0.456805766904647\\
 32.0832287788391	-0.456978595965479\\
% 32.0832550048828	-0.457054541489632\\
% 32.0832790851593	-0.457093654930067\\
% 32.0845431804657	-0.457099071595967\\
% 32.4165415287018	-0.457086908052929\\
 32.4165484428406	-0.457049441431819\\
% 32.416555595398	-0.457015905522872\\
% 32.4167065143585	-0.457232852255433\\
% 32.4169623374939	-0.457463016670506\\
 32.4169682979584	-0.457679906284849\\
% 32.6960920810699	-0.457888111424874\\
% 32.6961004257202	-0.458061844864264\\
% 32.6961133003235	-0.458261837141484\\
 32.6961235523224	-0.458419283228544\\
% 32.696128320694	-0.4585373112449\\
% 32.6961371421814	-0.458574582632345\\
% 32.8045789718628	-0.458717501818544\\
 32.8046054363251	-0.458833114434423\\
% 32.8438534259796	-0.458943774644951\\
% 32.9417709827423	-0.459061329875566\\
% 32.9419347763062	-0.459163023159166\\
 33.1483494758606	-0.459485518841232\\
% 33.1483871459961	-0.459789531193292\\
% 33.1485037326813	-0.460040483865086\\
% 33.1485960006714	-0.460304141054217\\
 33.2275540351868	-0.460497380418923\\
% 33.2620851516724	-0.46064216378428\\
% 33.4662851810455	-0.460863522105413\\
% 33.4662901878357	-0.461048266376697\\
 33.4662944793701	-0.461200244464532\\
% 33.4663323879242	-0.461472900198112\\
% 33.6951984882355	-0.461773763459747\\
% 33.6952318668366	-0.462034706566647\\
 33.6952564239502	-0.462305481989009\\
% 33.6952812194824	-0.462559941667917\\
% 33.6953327178955	-0.462887250571581\\
% 34.0244714736939	-0.463214484132217\\
 34.0244769573212	-0.463562782707368\\
% 34.0244836330414	-0.463889470309247\\
% 34.0244886398315	-0.464185907686938\\
% 34.0244934082031	-0.464446892277431\\
 34.0244967460632	-0.464685275437579\\
% 34.1406688213348	-0.464890888868527\\
% 34.1412770271301	-0.465133996700791\\
% 34.1414825439453	-0.46535900606011\\
 34.3046380996704	-0.465581498769035\\
% 34.3046442985535	-0.465807306094149\\
% 34.3046488285065	-0.466095321181129\\
% 34.3536202430725	-0.466403462147747\\
 34.4075736522675	-0.466700155992966\\
% 34.4325148582459	-0.466965770787732\\
% 34.5385291099548	-0.46718124689545\\
% 34.538582277298	-0.467358519346106\\
 34.5713927268982	-0.4674986167353\\
% 34.7362963676453	-0.467589736567534\\
% 34.7364422798157	-0.467754351556206\\
% 34.7365619659424	-0.467916213335277\\
 34.8030476093292	-0.468087043489695\\
% 34.8534969806671	-0.46839026094072\\
% 34.896230173111	-0.468695626649236\\
% 34.9908022403717	-0.468979216811321\\
 34.9910139560699	-0.469209505809292\\
% 35.1785239696503	-0.469403688916035\\
% 35.178533744812	-0.469603293402505\\
% 35.1785385131836	-0.469801009242467\\
 35.1785497188568	-0.470067111337134\\
% 35.2687682628632	-0.470348487086377\\
% 35.3520075798035	-0.470634956682516\\
% 35.3520118713379	-0.470900571820391\\
 35.4015700340271	-0.471141311097913\\
% 35.4856979370117	-0.471344374360085\\
% 35.4857115268707	-0.471484193212251\\
% 35.567130279541	-0.471611889214157\\
 35.6499711990356	-0.471741452534879\\
% 35.6500160217285	-0.471954887870137\\
% 35.7923717021942	-0.472173047605839\\
% 35.794321012497	-0.472389702296869\\
 35.7943298339844	-0.472610188018329\\
% 35.9174291610718	-0.472808230581776\\
% 35.9174360752106	-0.472977145266746\\
% 36.0053495883942	-0.473101206039171\\
 36.0053569793701	-0.473213454636885\\
% 36.0758685588837	-0.473399791750563\\
% 36.0759078979492	-0.473598520025198\\
% 36.1870767593384	-0.473808820319447\\
 36.1871056079865	-0.474043412099143\\
% 36.2769727230072	-0.474254638096472\\
% 36.2769793987274	-0.47440782580455\\
% 36.3538722515106	-0.4745255613762\\
 36.4040152549744	-0.474674837277736\\
% 36.4267644405365	-0.474815235550412\\
% 36.4882075309753	-0.475022442319903\\
% 36.5738670349121	-0.475250440760908\\
 36.7500044822693	-0.47549147319249\\
% 36.7500144958496	-0.475735451941814\\
% 36.7500192642212	-0.47596493234488\\
% 36.7500240325928	-0.476119282836567\\
 36.9328815460205	-0.47623871108048\\
% 36.9328860759735	-0.476314713374222\\
% 36.9328906059265	-0.476382047819101\\
% 36.9333383560181	-0.476436473382643\\
 37.1490315914154	-0.476488940038611\\
% 37.1490804672241	-0.476665255564389\\
% 37.1497647285461	-0.476870835371861\\
% 37.1498274326324	-0.477069828858848\\
 37.3611752510071	-0.477251869476599\\
% 37.361182641983	-0.47738966783694\\
% 37.3623020172119	-0.47750012505941\\
% 37.3623060703278	-0.47762170271305\\
 37.4374818325043	-0.477734087323558\\
% 37.43752617836	-0.477842149294008\\
% 37.6287002086639	-0.478018699120289\\
% 37.6287106990814	-0.478204239741011\\
 37.6287152290344	-0.47838566139871\\
% 37.628727388382	-0.478563129145251\\
% 37.8067099571228	-0.478713065849777\\
% 37.8067221164703	-0.478834677106633\\
 37.8067826747894	-0.478924938942523\\
% 37.8460576057434	-0.478998654771418\\
% 37.9141819000244	-0.479081224516229\\
% 37.9315175533295	-0.479259130765615\\
 38.0237917423248	-0.479443290733063\\
% 38.0862652778626	-0.479621811271491\\
% 38.086269569397	-0.479784382292731\\
% 38.1961585998535	-0.479893171758431\\
 38.196178150177	-0.480024511904247\\
% 38.3084940433502	-0.480147085700648\\
% 38.3085419654846	-0.480303775232282\\
% 38.378288936615	-0.480467114052589\\
 38.393293094635	-0.480633179897978\\
% 38.4296378612518	-0.480797606046286\\
% 38.5430137634277	-0.480941274588674\\
% 38.5430688381195	-0.481045526075059\\
 38.6240517616272	-0.481113197998787\\
% 38.6530174732208	-0.481168180986054\\
% 38.7179064273834	-0.481220907964336\\
% 38.7788311958313	-0.48133019389511\\
 38.7788407325745	-0.481446090007816\\
% 38.8368362903595	-0.481560837416283\\
% 38.8831307411194	-0.481712251050692\\
% 38.919997882843	-0.481838524628609\\
 38.9796266078949	-0.481943554224441\\
% 39.019500207901	-0.482038441506418\\
% 39.0791270256043	-0.482087054865454\\
% 39.1526405334473	-0.482113497966851\\
 39.4971506118774	-0.482129419980105\\
% 39.49718708992	-0.482202990026119\\
% 39.4972180843353	-0.482286571844576\\
% 39.4972497940064	-0.482468135746797\\
 39.497278881073	-0.482643774722735\\
% 39.4973082065582	-0.482790956634883\\
% 39.4974827289581	-0.482921129772344\\
% 39.65032787323	-0.482994647977049\\
 39.6503335952759	-0.483042304561559\\
% 39.6503650665283	-0.483074005262572\\
% 39.7096879005432	-0.483168171848205\\
% 39.7588674545288	-0.483271802686567\\
 39.8301410198212	-0.483440751363343\\
% 39.8644632816315	-0.483602620829917\\
% 39.926452589035	-0.483732080913259\\
% 39.9270951271057	-0.48383332548704\\
 40.0534979820251	-0.483906276655592\\
% 40.0535077571869	-0.483962592061883\\
% 40.2688049793243	-0.484010418600255\\
% 40.2689120292664	-0.484057049895009\\
 40.268930387497	-0.484110735410883\\
% 40.2689377784729	-0.484247936135041\\
% 40.4292718887329	-0.484385166574212\\
% 40.4293117046356	-0.484524483386781\\
 40.4293787002564	-0.484656707426458\\
% 40.4294297218323	-0.484758752319166\\
% 40.5301465511322	-0.484841987644955\\
% 40.564444732666	-0.484928284138495\\
 40.7577771663666	-0.485022859081436\\
% 40.7585477352142	-0.485108209458843\\
% 40.7585613250732	-0.485231075224012\\
% 40.7589544773102	-0.485355666938444\\
 41.0156762123108	-0.485480544869883\\
% 41.0157110214233	-0.485595940166354\\
% 41.0157479763031	-0.485699218718935\\
% 41.015776348114	-0.485791881868438\\
 41.0157904148102	-0.485868413008583\\
% 41.2113515853882	-0.485941322618844\\
% 41.2113830566406	-0.486017276939045\\
% 41.2114123821259	-0.486095337589804\\
 41.2114569664001	-0.486170772134898\\
% 41.3561818122864	-0.48624255052236\\
% 41.356210899353	-0.486307714080969\\
% 41.3562597751617	-0.48636248554092\\
 41.4521789073944	-0.486390852380186\\
% 41.4522032260895	-0.486405766962551\\
% 41.5460445404053	-0.486408788547524\\
% 41.6698898792267	-0.486411989264894\\
 42.0554856777191	-0.486434207244063\\
% 42.0562896251678	-0.486455755761954\\
% 42.056302022934	-0.486474947790987\\
% 42.0563089370728	-0.486489841155743\\
 42.0563282489777	-0.486498732900419\\
% 42.0563456535339	-0.486498732900419\\
% 42.056353521347	-0.486498732900419\\
% 42.0563606739044	-0.486498732900419\\
 42.0563671112061	-0.486498732900419\\
% 42.0563849925995	-0.486498732900419\\
% 42.7761535167694	-0.486498732900419\\
% 42.7761735439301	-0.486498732900419\\
 43.0857395648956	-0.486498732900419\\
% 43.0857460021973	-0.486498732900419\\
% 43.0857576847076	-0.486498732900419\\
% 43.2397584438324	-0.486498732900419\\
 43.2397701263428	-0.486498732900419\\
% 43.2397839546204	-0.486498732900419\\
% 43.2644345283508	-0.486498732900419\\
% 43.6193358421326	-0.486498732900419\\
 43.6193403720856	-0.486498732900419\\
% 43.6193439483643	-0.486498732900419\\
% 43.6193475246429	-0.486498732900419\\
% 43.7342502593994	-0.486498732900419\\
 43.7342824459076	-0.486498732900419\\
% 44.1866940975189	-0.486498732900419\\
% 44.1866991043091	-0.486498732900419\\
% 44.2737109184265	-0.486498732900419\\
 44.2737159252167	-0.486498732900419\\
% 44.6853835105896	-0.486498732900419\\
% 44.6854300022125	-0.486498732900419\\
% 44.8329545974731	-0.486498732900419\\
 44.8329832077026	-0.486498732900419\\
% 44.8330263614655	-0.486498732900419\\
% 45.7841484069824	-0.486498732900419\\
% 45.78420753479	-0.486498732900419\\
 45.7842416286469	-0.486498732900419\\
% 45.7842862129211	-0.486498732900419\\
% 45.7843179225922	-0.486498732900419\\
% 45.7843596458435	-0.486498732900419\\
 45.7843942165375	-0.486498732900419\\
% 45.7844330787659	-0.486498732900419\\
% 45.7844631195068	-0.486498732900419\\
% 45.7844976902008	-0.486498732900419\\
 45.784534406662	-0.486498732900419\\
% 45.7846402645111	-0.486498732900419\\
% 46.0884601593018	-0.486498732900419\\
% 46.0884670734406	-0.486498732900419\\
 46.1257688522339	-0.486498732900419\\
% 46.1662382602692	-0.486498732900419\\
% 46.2153467655182	-0.486498732900419\\
% 46.6415497779846	-0.486498732900419\\
 46.7479912757874	-0.486498732900419\\
% 46.7480005741119	-0.486498732900419\\
% 46.9038605213165	-0.486498732900419\\
% 46.9188446521759	-0.486498732900419\\
 47.030423116684	-0.486498732900419\\
% 47.8224229335785	-0.486498732900419\\
% 47.9271270751953	-0.486498732900419\\
% 48.2469987392426	-0.486498732900419\\
 48.2470311641693	-0.486498732900419\\
% 48.247061920166	-0.486498732900419\\
% 48.6642553329468	-0.486498732900419\\
% 48.9163457870483	-0.486498732900419\\
 48.9163507938385	-0.486498732900419\\
% 48.9173902988434	-0.486498732900419\\
% 49.5241639137268	-0.486498732900419\\
% 49.5241684436798	-0.486498732900419\\
 49.525897693634	-0.486498732900419\\
% 49.5259534835815	-0.486498732900419\\
% 50.0419322967529	-0.486498732900419\\
% 51.5745617866516	-0.486498732900419\\
 51.5749794960022	-0.486498732900419\\
% 51.5749883174896	-0.486498732900419\\
% 51.8694738864899	-0.486498732900419\\
% 51.869504404068	-0.486498732900419\\
 51.8695332527161	-0.486498732900419\\
% 51.869561624527	-0.486498732900419\\
% 51.8696019172668	-0.486498732900419\\
% 51.8696481704712	-0.486498732900419\\
 53.5224811553955	-0.486498732900419\\
% 53.5224897384644	-0.486498732900419\\
% 53.5224947452545	-0.486498732900419\\
% 53.5224995136261	-0.486498732900419\\
 53.5225045204163	-0.486498732900419\\
% 53.5225088119507	-0.486498732900419\\
% 53.5225133419037	-0.486498732900419\\
% 53.5225185871124	-0.486498732900419\\
 53.52253074646	-0.486498732900419\\
% 53.5225345611572	-0.486498732900419\\
% 53.5225390911102	-0.486498732900419\\
% 53.5225431442261	-0.486498732900419\\
 53.8357407569885	-0.486498732900419\\
% 53.8357474327087	-0.486498732900419\\
% 53.8357564926147	-0.486498732900419\\
% 53.8360311508179	-0.486498732900419\\
 53.8361589431763	-0.486498732900419\\
% 53.8363892555237	-0.486498732900419\\
% 54.9593457698822	-0.486498732900419\\
% 54.9593765258789	-0.486498732900419\\
 54.9594101428986	-0.486498732900419\\
% 54.9594618797302	-0.486498732900419\\
% 54.9596836090088	-0.486498732900419\\
% 54.9597208023071	-0.486498732900419\\
 54.9597517967224	-0.486498732900419\\
% 54.9597842216492	-0.486498732900419\\
% 54.9599391937256	-0.486498732900419\\
% 54.9599730491638	-0.486498732900419\\
 54.9600042819977	-0.486498732900419\\
% 54.9601048946381	-0.486498732900419\\
% 55.7669550895691	-0.486498732900419\\
% 55.766986322403	-0.486498732900419\\
 55.767018032074	-0.486498732900419\\
% 55.7670630931854	-0.486498732900419\\
% 55.7670952796936	-0.486498732900419\\
% 55.7671293735504	-0.486498732900419\\
 55.767174911499	-0.486498732900419\\
% 55.7672061443329	-0.486498732900419\\
% 55.76724858284	-0.486498732900419\\
% 55.7672833919525	-0.486498732900419\\
 55.7673131942749	-0.486498732900419\\
% 55.7673434734345	-0.486498732900419\\
% 55.7673713684082	-0.486498732900419\\
% 55.7674064159393	-0.486498732900419\\
 55.7674335956574	-0.486498732900419\\
% 55.7674622058868	-0.486498732900419\\
% 55.7675101280212	-0.486498732900419\\
% 56.4338857650757	-0.486498732900419\\
 56.4339029312134	-0.486498732900419\\
% 56.4339148521423	-0.486498732900419\\
% 56.4339248657227	-0.486498732900419\\
% 56.4339437007904	-0.486498732900419\\
 56.4340469360352	-0.486498732900419\\
% 56.4340550422668	-0.486498732900419\\
% 56.4340698242188	-0.486498732900419\\
% 56.4340769767761	-0.486498732900419\\
 56.4340853214264	-0.486498732900419\\
% 56.4340981960297	-0.486498732900419\\
% 56.4342569828033	-0.486498732900419\\
% 56.4342746257782	-0.486498732900419\\
% 56.8145236492157	-0.486498732900419\\
 56.9261416912079	-0.486498732900419\\
% 57.0169128894806	-0.486498732900419\\
% 57.0169181346893	-0.486498732900419\\
% 58.5536970615387	-0.486498732900419\\
 58.5537037372589	-0.486498732900419\\
% 58.553709936142	-0.486498732900419\\
% 58.5537163734436	-0.486498732900419\\
 58.5537225723267	-0.486498732900419\\
% 58.5537290096283	-0.486498732900419\\
% 58.5537354469299	-0.486498732900419\\
% 58.5537423610687	-0.486498732900419\\
 58.5537490367889	-0.486498732900419\\
% 58.5537545204163	-0.486498732900419\\
% 58.5537600040436	-0.486498732900419\\
% 58.5537657260895	-0.486498732900419\\
 58.5537721633911	-0.486498732900419\\
% 58.5537781238556	-0.486498732900419\\
% 58.5537843227387	-0.486498732900419\\
% 58.5537902832031	-0.486498732900419\\
 58.5551464080811	-0.486498732900419\\
% 58.5551602363586	-0.486498732900419\\
% 58.5551681041718	-0.486498732900419\\
% 58.5551752567291	-0.486498732900419\\
 58.5551819324493	-0.486498732900419\\
% 58.9172229290009	-0.486498732900419\\
% 58.9172555923462	-0.486498732900419\\
% 58.9172882556915	-0.486498732900419\\
 58.9172951698303	-0.486498732900419\\
% 58.9173254489899	-0.486498732900419\\
% 58.9173306941986	-0.486498732900419\\
% 58.9173583507538	-0.486498732900419\\
 58.9174062728882	-0.486498732900419\\
% 59.1581761360168	-0.486498732900419\\
% 59.1582495689392	-0.486498732900419\\
% 59.1583513736725	-0.486498732900419\\
 59.1584672451019	-0.486498732900419\\
% 59.3863837242126	-0.486498732900419\\
% 59.3863975524902	-0.486498732900419\\
% 59.3864140033722	-0.486498732900419\\
 59.3864302158356	-0.486498732900419\\
% 59.3864404678345	-0.486498732900419\\
% 59.8614983081818	-0.486498732900419\\
% 59.865093421936	-0.486498732900419\\
 59.8651005744934	-0.486498732900419\\
% 59.8657540798187	-0.486498732900419\\
% 59.8657629013062	-0.486498732900419\\
% 59.8659624576569	-0.486498732900419\\
 59.8659700870514	-0.486498732900419\\
% 59.8662604808807	-0.486498732900419\\
% 59.8662702560425	-0.486498732900419\\
% 60.6861924648285	-0.486498732900419\\
 60.686212015152	-0.486498732900419\\
% 60.6862289428711	-0.486498732900419\\
% 60.6862382411957	-0.486498732900419\\
% 60.686255645752	-0.486498732900419\\
 60.6862613677979	-0.486498732900419\\
% 60.6882712364197	-0.486498732900419\\
% 60.6882829189301	-0.486498732900419\\
% 60.6882910251617	-0.486498732900419\\
 60.6882996082306	-0.486498732900419\\
% 60.6883103370667	-0.486498732900419\\
% 60.6883182048798	-0.486498732900419\\
% 60.6883239269257	-0.486498732900419\\
 60.6883284568787	-0.486498732900419\\
% 60.6883360862732	-0.486498732900419\\
% 60.6883420467377	-0.486498732900419\\
% 60.6883496761322	-0.486498732900419\\
 60.9728793621063	-0.486498732900419\\
% 60.9912290096283	-0.486498732900419\\
% 60.9912459373474	-0.486498732900419\\
% 60.9912766933441	-0.486498732900419\\
 60.9912983894348	-0.486498732900419\\
% 60.9992479801178	-0.486498732900419\\
};
\addlegendentry{$\theta_{y_{p}}$}

\addplot [color=mycolor5, line width=1.4pt]
  table[row sep=crcr]{%
 -2.3985894203186	0\\
% -2.34845380783081	0\\
% -2.29882984161377	0\\
% -2.24259977340698	0\\
 -2.1981945514679	0\\
% -2.14788823127747	0\\
% -2.09815530776978	0\\
% -2.04817252159119	0\\
 -1.99881820678711	0\\
% -1.94862990379333	0\\
% -1.87693266868591	0\\
% -1.84271411895752	0\\
 -1.79806022644043	0\\
% -1.74720959663391	0\\
% -1.69869332313538	0\\
% -1.6482563495636	0\\
 -1.59839563369751	0\\
% -1.548703956604	0\\
% -1.49876074790955	0\\
% -1.44885139465332	0\\
 -1.3982711315155	0\\
% -1.34799294471741	0\\
% -1.29795436859131	0\\
% -1.2485387802124	0\\
 -1.19843416213989	0\\
% -1.14853506088257	0\\
% -1.09835534095764	0\\
% -1.04829435348511	0\\
 -0.998184490203857	0\\
% -0.948882627487182	0\\
% -0.898640441894531	0\\
% -0.847953605651855	0\\
 -0.798753547668457	0\\
% -0.748502779006958	0\\
% -0.698466348648071	0\\
% -0.648822116851806	0\\
 -0.597503709793091	0\\
% -0.548140096664429	0\\
% -0.498594331741333	0\\
% -0.4485755443573	0\\
 -0.398829984664917	0\\
% -0.348826217651367	0\\
% -0.298781204223633	0\\
% -0.248498249053955	0\\
 -0.197533178329468	0\\
% -0.147660779953003	0\\
% -0.0983634471893309	0\\
% -0.0481057643890379	0\\
 0.0019752502441408	0\\
% 0.0514363765716555	0\\
% 0.113763284683228	0\\
% 0.15691466331482	0\\
 0.234845352172852	0\\
% 0.308734607696533	0\\
% 0.308903646469116	0\\
% 0.360757303237915	0\\
 0.415858459472656	0\\
% 0.497472476959229	3.54432114337104\\
% 0.559248399734497	4.97169233637806\\
% 0.568476867675781	5.74871173361703\\
 0.604575109481812	6.23776798252891\\
% 0.723840665817261	6.58479749668368\\
% 0.723884773254395	6.85111085161066\\
% 0.853632164001465	7.06865595037198\\
 0.853860330581665	7.24890192023531\\
% 0.859521579742432	7.40094127198813\\
% 0.922651720046997	7.5304318069052\\
% 0.971385669708252	7.63976631663081\\
 1.02831001281738	7.73610902232303\\
% 1.07039852142334	7.82350421644969\\
% 1.11899919509888	7.90794324536591\\
% 1.16864533424377	7.9871370140886\\
 1.22722692489624	8.06284643064328\\
% 1.32403750419617	8.13197600264311\\
% 1.3240451335907	8.19479157014439\\
% 1.43628258705139	8.25139555839542\\
 1.4363646030426	8.30372906660023\\
% 1.6345018863678	8.35427859469519\\
% 1.63452954292297	8.40560058521942\\
% 1.63456315994263	8.45800084200118\\
 1.63461179733276	8.50922036938255\\
% 1.67959208488464	8.55943459275795\\
% 1.74344153404236	8.60953159824385\\
% 1.80315823554993	8.65743082570225\\
 1.8214485168457	8.70162534352039\\
% 1.86181755065918	8.74405652737278\\
% 1.95309824943542	8.78370681558499\\
% 1.98877353668213	8.82449714036466\\
 2.10328192710876	8.86760985625824\\
% 2.10334939956665	8.91281113913556\\
% 2.2043318271637	8.95790250611844\\
% 2.20438261032105	9.0015359435447\\
 2.20442314147949	9.0420021568134\\
% 2.31627173423767	9.07948226200915\\
% 2.31644148826599	9.11544429214609\\
% 2.35974855422974	9.15311529791688\\
 2.40763850212097	9.19334942431851\\
% 2.58187789916992	9.23509904395178\\
% 2.58190841674805	9.27560207108309\\
% 2.58211798667908	9.31433882960391\\
 2.96352286338806	9.35038280968183\\
% 2.96353478431702	9.38508548771915\\
% 2.96354002952576	9.42008826992515\\
% 2.96354813575745	9.45528208962605\\
 2.96359629631042	9.49306492848154\\
% 2.96360988616943	9.53084567250244\\
% 2.9636182308197	9.56817975403737\\
% 2.96363134384155	9.60407530842531\\
 3.06343312263489	9.63924729237897\\
% 3.06350989341736	9.67337338334801\\
% 3.14400072097778	9.70795855923825\\
% 3.24244160652161	9.74387310659677\\
 3.24249835014343	9.7814715065042\\
% 3.38695282936096	9.82031103564805\\
% 3.38704962730408	9.8580200367378\\
% 3.38712425231934	9.89463682763017\\
 3.50973148345947	9.9307923649958\\
% 3.50977296829224	9.96689309693465\\
% 3.50982351303101	10.0042128620862\\
% 3.60157866477966	10.0429509236674\\
 3.61497135162354	10.084055024663\\
% 3.66265697479248	10.1267544428065\\
% 3.77121181488037	10.1694577083454\\
% 3.77127523422241	10.211503018736\\
 3.88388056755066	10.2543543164265\\
% 3.88406581878662	10.2975065288083\\
% 4.01174230575562	10.3417203073222\\
% 4.01174683570862	10.3873892327065\\
 4.01175088882446	10.4338436273101\\
% 4.10342116355896	10.4793680014682\\
% 4.10781664848328	10.5234307064275\\
% 4.22873516082764	10.5664962988521\\
 4.22877926826477	10.609213703261\\
% 4.3162703037262	10.6530565751109\\
% 4.31628270149231	10.6995078772634\\
% 4.36952753067017	10.7477815884158\\
 4.49391932487488	10.7956738131397\\
% 4.4939248085022	10.841547530119\\
% 4.50986928939819	10.8856031534233\\
% 4.60001916885376	10.9281698468694\\
 4.6112982749939	10.9706800737295\\
% 4.713267993927	11.0147789120583\\
% 4.71331377029419	11.0612544651685\\
% 4.84796900749207	11.1077216176654\\
 4.84803886413574	11.1526042683163\\
% 4.9620210647583	11.1951421179529\\
% 4.96202535629272	11.2360315323276\\
% 4.96383280754089	11.2759334674174\\
 5.00998063087463	11.3167986435365\\
% 5.05557627677917	11.3578604912327\\
% 5.10753936767578	11.4015218077093\\
% 5.19693036079407	11.4488419292047\\
 5.24160213470459	11.4954836513098\\
% 5.2747401714325	11.5406965877482\\
% 5.47130937576294	11.5831741537895\\
% 5.47131628990173	11.6212941409049\\
 5.47132034301758	11.6570451307252\\
% 5.47196359634399	11.6913570276047\\
% 5.92306728363037	11.7266680764333\\
% 5.92307133674622	11.7621840064458\\
 5.92308039665222	11.7979986711562\\
% 5.92308468818665	11.8326881285484\\
% 5.92308826446533	11.8653699558927\\
% 5.92309231758118	11.8970446481944\\
 5.92309613227844	11.9276508694038\\
% 5.92310948371887	11.9583717768983\\
% 5.92311329841614	11.9879745619655\\
% 5.99617094993591	12.0161775277556\\
 6.01983733177185	12.0423708657745\\
% 6.06683416366577	12.0668731136811\\
% 6.13510556221008	12.0906441402085\\
% 6.17488260269165	12.1141292769125\\
 6.34516496658325	12.1386064121634\\
% 6.34567971229553	12.1630612880554\\
% 6.34579391479492	12.1874190746794\\
% 6.48088450431824	12.2112785402414\\
 6.48123211860657	12.2331194391454\\
% 6.48173136711121	12.2528753184706\\
% 6.55087299346924	12.2704633695121\\
% 6.60183663368225	12.2862726703615\\
 6.61844868659973	12.3020235729873\\
% 6.6699878692627	12.3180271503279\\
% 6.73549313545227	12.3361295448349\\
% 6.90011711120605	12.3537380426678\\
 6.90014715194702	12.3702623936451\\
% 6.90017552375793	12.3846031343101\\
% 6.96017570495605	12.3971297731996\\
% 6.96023197174072	12.4079757840072\\
 7.02018661499023	12.4185670375355\\
% 7.08605260848999	12.4292594011267\\
% 7.14995141029358	12.4399225024217\\
% 7.2400710105896	12.4505397519952\\
 7.24012393951416	12.4602269775587\\
% 7.35665650367737	12.4686984642276\\
% 7.35666365623474	12.4763864198139\\
% 7.50300760269165	12.4832198608474\\
 7.50304217338562	12.4897406159989\\
% 7.50325675010681	12.4972125048062\\
% 7.55728693008423	12.5044510301359\\
% 7.58989186286926	12.5106602786318\\
 7.60993332862854	12.5163403821471\\
% 7.71798748970032	12.5213643635643\\
% 7.71803326606751	12.5255796982165\\
% 7.8084406375885	12.52915646562\\
 7.8084451675415	12.5329936829585\\
% 7.86788339614868	12.5368147639056\\
% 7.98911018371582	12.5406169167663\\
% 7.98919291496277	12.5440981059473\\
 8.03272266387939	12.5472093680019\\
% 8.08047933578491	12.549869382553\\
% 8.12581033706665	12.5518549250155\\
% 8.19591255187988	12.5535719203681\\
 8.21851797103882	12.5547544607562\\
% 8.32182474136352	12.5555578590729\\
% 8.32182974815369	12.5563948696049\\
% 8.37021417617798	12.5573512601809\\
 8.40591187477112	12.558086855287\\
% 8.59654874801636	12.5585897396113\\
% 8.59659929275513	12.558662482314\\
% 8.59668583869934	12.5581862633994\\
 8.75598998069763	12.5571896074021\\
% 8.75599665641785	12.5556835144162\\
% 8.75600190162659	12.5539147022819\\
% 8.75602097511291	12.5525147822141\\
 9.02396960258484	12.5506967502461\\
% 9.02399797439575	12.5491003365851\\
% 9.02402396202087	12.5475966378217\\
% 9.02404947280884	12.5459422197591\\
 9.02408738136291	12.544100870587\\
% 9.34571809768677	12.5420276893196\\
% 9.34574980735779	12.5396248831971\\
% 9.34578461647034	12.5370646734941\\
 9.34599132537842	12.5346284787456\\
% 9.34602160453796	12.5321691921349\\
% 9.34606547355652	12.5299830306431\\
% 9.55178208351135	12.5278366438765\\
 9.55178709030151	12.525734498226\\
% 9.55179281234741	12.5234219677172\\
% 9.55179829597473	12.5207053229906\\
% 9.61852855682373	12.5178110272595\\
 9.61856646537781	12.5144168164752\\
% 9.6634735584259	12.5110286769923\\
% 9.70841760635376	12.5078175634485\\
% 9.81058926582336	12.5044102998572\\
 9.81059379577637	12.5013352078199\\
% 9.91963858604431	12.4983932255545\\
% 9.91964693069458	12.495317924876\\
% 9.97360939979553	12.4920198499644\\
 10.0412783145905	12.488538400095\\
% 10.1883935451508	12.4850183898807\\
% 10.1884243011475	12.480687914056\\
% 10.1884679317474	12.476236058199\\
 10.2828607082367	12.4720349214069\\
% 10.2828685760498	12.4681308329832\\
% 10.3878936290741	12.4641637418008\\
% 10.3879322528839	12.4603480263916\\
 10.477405500412	12.4563016506986\\
% 10.4774195671082	12.4521694636112\\
% 10.5185296058655	12.4478690156234\\
% 10.560958814621	12.4436250158681\\
 10.6081099033356	12.4394391984263\\
% 10.6624338150024	12.4350665359182\\
% 10.7120904445648	12.4310139871336\\
% 10.8078364849091	12.426958136939\\
 10.8310131549835	12.4230212633179\\
% 10.86188621521	12.4190400139919\\
% 10.9186720371246	12.4150094481708\\
% 10.9605512142181	12.410875562231\\
 11.0521792888641	12.4068294112822\\
% 11.0872270584106	12.4028125297052\\
% 11.1295003414154	12.399045821641\\
% 11.1574806690216	12.3952105879273\\
 11.2049316883087	12.3914544865047\\
% 11.264439535141	12.3877291086969\\
% 11.3169879436493	12.3838272489631\\
% 11.3666297912598	12.3797599804473\\
 11.4129797935486	12.3757285990721\\
% 11.487088394165	12.3717997957592\\
% 11.5034069538116	12.3681055830232\\
% 11.5847613334656	12.3646163267285\\
 11.6668142795563	12.360878588117\\
% 11.6668383598328	12.3572100884355\\
% 11.756715965271	12.3533714105174\\
% 11.7568525791168	12.3494825691339\\
 11.8283314228058	12.3456520844798\\
% 11.861048412323	12.3418453320835\\
% 11.9123050689697	12.3382291718603\\
% 11.9995982170105	12.3345911679618\\
 12.0273801803589	12.3310962328318\\
% 12.0843092918396	12.3276609968716\\
% 12.1415590763092	12.3240326550133\\
% 12.1802153110504	12.3204219877239\\
 12.2485608577728	12.316785590467\\
% 12.2750324726105	12.3133062942482\\
% 12.3711525917053	12.3099865098104\\
% 12.3711673736572	12.3065453174611\\
 12.584521484375	12.3031888868352\\
% 12.5848187923431	12.2998780605739\\
% 12.5848257064819	12.2964493201201\\
% 12.5848338127136	12.2931255976242\\
 12.6255592823029	12.2897431554735\\
% 12.6870493412018	12.286519987082\\
% 12.7069229602814	12.2834523797538\\
% 12.7557956695557	12.2803394496871\\
 12.8823637485504	12.2773024443392\\
% 12.882368516922	12.2743434390504\\
% 12.9974901199341	12.27116989922\\
% 12.9975285053253	12.2681094156859\\
 13.0484261035919	12.2649897308933\\
% 13.0905348777771	12.2620311152142\\
% 13.1565808773041	12.2590402213351\\
% 13.2565376281738	12.2561646971176\\
 13.2565733909607	12.2533637478532\\
% 13.2757715702057	12.2506882435428\\
% 13.3401882171631	12.2479020941258\\
% 13.4026224136353	12.2451745676628\\
 13.4468314170837	12.2425098659726\\
% 13.4851455211639	12.2396930043321\\
% 13.5162672519684	12.2370231018533\\
% 13.5994228839874	12.2344898257982\\
 13.6809041023254	12.2317573135135\\
% 13.6809529781342	12.2291237696418\\
% 13.760929775238	12.2263439342379\\
% 13.7609891414642	12.2236342784401\\
 13.8658623218536	12.2210008916549\\
% 13.8659095287323	12.2182350313835\\
% 13.9651402950287	12.2156001408525\\
% 13.9651853561401	12.2129365377106\\
 14.013670873642	12.2104108290126\\
% 14.0695306777954	12.2079918416142\\
% 14.1660151004791	12.205467321515\\
% 14.1674413204193	12.2030221548108\\
 14.3973986625671	12.2004509043336\\
% 14.3974039077759	12.1979189826512\\
% 14.397408914566	12.1953095401086\\
% 14.3974151134491	12.1928228640553\\
 14.4515306472778	12.1904832266677\\
% 14.4809290885925	12.1880887392161\\
% 14.507790517807	12.1858041998299\\
% 14.6076409339905	12.1833721117677\\
 14.6330356121063	12.1810080144372\\
% 14.7221893787384	12.1785538620049\\
% 14.7222127437592	12.1761732426353\\
% 14.8223895549774	12.1739076814172\\
 14.8226484775543	12.1716378562667\\
% 14.8592681407928	12.1695065490071\\
% 14.9488486766815	12.1675041791729\\
% 14.9796115875244	12.1654193343092\\
 15.0424162864685	12.1634132951002\\
% 15.0784852027893	12.1614701724071\\
% 15.1091608524323	12.1592970437615\\
% 15.1938590526581	12.1572117331426\\
 15.2520548820496	12.1550092958614\\
% 15.3033129692078	12.1529270694472\\
% 15.3873831748962	12.1509549821046\\
% 15.389208984375	12.1490127164646\\
 15.5462734222412	12.1471827623128\\
% 15.5463118076324	12.1454402640101\\
% 15.5463530540466	12.1436277811935\\
% 15.6087738990784	12.1418769682357\\
 15.6088122844696	12.1401863645667\\
% 15.8284608840942	12.1381819247752\\
% 15.82850689888	12.1362869872577\\
% 15.8285231113434	12.1345036138399\\
 15.8290633678436	12.1326201900052\\
% 15.9087426185608	12.1308419658387\\
% 16.0841259479523	12.1289551143053\\
% 16.0841304779053	12.1271478910092\\
 16.0841798305511	12.1254088774129\\
% 16.0844277858734	12.1234596306258\\
% 16.1285621643066	12.1216179494676\\
% 16.1920978546143	12.1198635878598\\
 16.2366220474243	12.118102036894\\
% 16.2680780410767	12.1164554256538\\
% 16.3074433326721	12.1149075374177\\
% 16.3852943897247	12.1132987487613\\
 16.427689743042	12.1117749033931\\
% 16.4832610607147	12.1103139209319\\
% 16.5667356967926	12.1086083166777\\
% 16.5667421340942	12.106970878068\\
 16.6510304927826	12.1051540857346\\
% 16.6845106601715	12.1034298656564\\
% 16.7282096862793	12.101789012589\\
% 16.7647928714752	12.1002236257209\\
 16.8415979862213	12.0987514756394\\
% 16.9638578414917	12.0973709000704\\
% 16.9638618946075	12.0960609248524\\
% 16.963866186142	12.0947074694202\\
 17.0969151973724	12.0930440872899\\
% 17.0969202041626	12.0914631895018\\
% 17.1762506484985	12.0899725640161\\
% 17.1763615131378	12.088571940052\\
 17.2229020118713	12.0871672866711\\
% 17.3326086521149	12.0858443119238\\
% 17.3326153278351	12.0846281243229\\
% 17.473627281189	12.0834824752797\\
 17.4736518383026	12.0822222515174\\
% 17.4736754417419	12.0810133347183\\
% 17.5113889694214	12.0794650138849\\
% 17.5604962825775	12.0780031825715\\
 17.6228019714355	12.0765722052331\\
% 17.6881763458252	12.0752624656247\\
% 17.8001989841461	12.073955945426\\
% 17.8002123355865	12.0727717469328\\
 17.8233386993408	12.0714770430832\\
% 17.9228260040283	12.0702462163042\\
% 17.9228312492371	12.0687779557068\\
% 18.0239114284515	12.0673588578399\\
 18.023930978775	12.0659336553498\\
% 18.2090794563293	12.0645964251539\\
% 18.2091359615326	12.0633482031402\\
% 18.209382724762	12.062176071573\\
 18.2096747875214	12.0610809643606\\
% 18.3486201286316	12.0600671364605\\
% 18.3489136219025	12.0588435050178\\
% 18.4253313064575	12.0576741702835\\
 18.4253766059875	12.0565690417105\\
% 18.5247117996216	12.0554128677465\\
% 18.5248689174652	12.0543440075917\\
% 18.5989043235779	12.0533534635401\\
 18.6113500118256	12.0523524302744\\
% 18.6828867912292	12.051431783745\\
% 18.7189335346222	12.0505738296558\\
% 18.7590083599091	12.0497641510645\\
 18.8376090049744	12.0487365292356\\
% 18.8578114032745	12.0477456620863\\
% 18.9627770900726	12.0468118928477\\
% 18.9628245353699	12.0456961123354\\
 19.0279707431793	12.0446677463274\\
% 19.0604125976562	12.0437008036201\\
% 19.1335758686066	12.0427653037242\\
% 19.2351016521454	12.0419110532803\\
 19.2352606773376	12.0411082343013\\
% 19.5428084850311	12.0403520493357\\
% 19.5428378105164	12.0393093358623\\
% 19.5428936004639	12.0383165350822\\
 19.5429250717163	12.0373760213784\\
% 19.5429551124573	12.0363833943676\\
% 19.5429982662201	12.035462149365\\
% 19.9834553718567	12.0346126423248\\
 19.9834630012512	12.0337300436873\\
% 19.9834668159485	12.0329038703567\\
% 19.9834703922272	12.0321472664203\\
% 19.9834739685059	12.0312379969617\\
 19.983477306366	12.0302621188629\\
% 19.9834856510162	12.0293552210943\\
% 19.9834892272949	12.02841742299\\
% 19.9834928035736	12.0275324468303\\
 20.048566532135	12.0267121917317\\
% 20.1035956859589	12.0258120893093\\
% 20.1756994247437	12.0249797446271\\
% 20.1757237434387	12.0242001392022\\
 20.2920464992523	12.0233402293924\\
% 20.2920505523682	12.0225326799288\\
% 20.3619036197662	12.0216697560825\\
% 20.4108409404755	12.020877101171\\
 20.4923514842987	12.0201668337898\\
% 20.4923595905304	12.0194461651936\\
% 20.5578968048096	12.0187677964446\\
% 20.6461452960968	12.0181488057557\\
 20.6461510181427	12.0173173690157\\
% 20.7409729480743	12.0165225627502\\
% 20.775611114502	12.0156834695094\\
% 20.8289811134338	12.0148997011705\\
 20.855291557312	12.0141833143792\\
% 20.8881897449493	12.0135033470057\\
% 20.9274053096771	12.0128890844109\\
% 20.9642960548401	12.0122488166931\\
 21.0089377880096	12.0116597357895\\
% 21.0846263885498	12.0111155619352\\
% 21.1657716751099	12.0103545963735\\
% 21.1659259319305	12.0096289941382\\
 21.2478870868683	12.0089484746183\\
% 21.287251663208	12.0083103991418\\
% 21.3467800140381	12.0077276060882\\
% 21.3878879070282	12.0070815913338\\
 21.4673904895782	12.0065026489383\\
% 21.4674722671509	12.0058340867193\\
% 21.7880739688873	12.0052169708256\\
% 21.7880997180939	12.0046418623939\\
 21.7881242752075	12.0039415786326\\
% 21.788148355484	12.0032687579115\\
% 21.788175535202	12.0026296448952\\
% 21.7882205963135	12.0019652283759\\
 21.9344484329224	12.0013541700153\\
% 21.9345077991486	12.0008004060487\\
% 21.9346115112305	12.0002481645949\\
% 22.1828016757965	11.9997484866828\\
 22.1828560352325	11.9991564569943\\
% 22.1828851222992	11.9985966844154\\
% 22.1829120635986	11.997930077533\\
% 22.1829561710358	11.9972954525242\\
 22.2871834754944	11.9966870794809\\
% 22.2872302055359	11.9961469567749\\
% 22.3290242671967	11.9955931479684\\
% 22.3717877388	11.9951302306222\\
 22.4189037799835	11.9947302469918\\
% 22.508477640152	11.9943658692501\\
% 22.579109621048	11.9939232762085\\
% 22.5791170120239	11.9934946050591\\
 22.6909491539001	11.9928564485691\\
% 22.6910020828247	11.9922341995683\\
% 23.0378574848175	11.9916371053861\\
% 23.0378658294678	11.9910693954537\\
 23.0378715515137	11.9905554017525\\
% 23.0378772735596	11.9901064490492\\
% 23.0378908634186	11.9896372249889\\
% 23.0379697799683	11.9892237591178\\
 23.0379788398743	11.9888524863935\\
% 23.1805610179901	11.9883617986967\\
% 23.1805998802185	11.9878938709957\\
% 23.2553226470947	11.9873450695872\\
 23.3742901802063	11.9868152008934\\
% 23.3742956638336	11.9863034113437\\
% 23.4069616317749	11.9858413551433\\
% 23.436697435379	11.9854007181359\\
 23.4717108726501	11.9849803024716\\
% 23.4726278305054	11.9845873182066\\
% 23.5186371326447	11.9842232569963\\
% 23.5845655918121	11.9838705658095\\
 23.6520792961121	11.9833998416629\\
% 23.7337271690369	11.982939743169\\
% 23.7337564945221	11.9824430096564\\
% 23.7942924022675	11.9819898269099\\
 23.8527013778687	11.9815775233968\\
% 23.8903464794159	11.9812143795712\\
% 23.9444241046906	11.9808900179136\\
% 24.0006498813629	11.9805833867207\\
 24.0766436576843	11.9801797824017\\
% 24.0766481876373	11.9797951700619\\
% 24.1547135829926	11.9794332245646\\
% 24.1547355175018	11.9790039852841\\
 24.2052554607391	11.9786124986532\\
% 24.3249959468842	11.9782133965199\\
% 24.3469130516052	11.977874966722\\
% 24.3948239803314	11.9775798520644\\
 24.4395904064178	11.9772378366554\\
% 24.4598485946655	11.9769195966923\\
% 24.5525354862213	11.9766150114948\\
% 24.5862226009369	11.976320517888\\
 24.6285922050476	11.975904047678\\
% 24.8560425758362	11.9755109229608\\
% 24.8560881137848	11.9751138437141\\
% 24.8566789150238	11.9747682366917\\
 24.8570489406586	11.974456602811\\
% 24.9618987560272	11.9741741162413\\
% 24.9619016170502	11.973847878514\\
% 24.9667996883392	11.9735304796494\\
 25.1297199249268	11.9731219720277\\
% 25.129741859436	11.9727485264318\\
% 25.1297678470612	11.9723995458603\\
% 25.1699120521545	11.9720875250163\\
 25.3255557537079	11.9717691012544\\
% 25.3258833408356	11.9714650955938\\
% 25.4633335590363	11.9711672125666\\
% 25.4633907794952	11.9707840040307\\
 25.4634375095367	11.9704112285844\\
% 26.2949263572693	11.9700467225196\\
% 26.2949594974518	11.9697067268757\\
% 26.2949900150299	11.9693838223225\\
 26.2950171947479	11.9690903058071\\
% 26.2950567722321	11.9687793775952\\
% 26.2950622558594	11.9684161842316\\
% 26.2950939655304	11.9680647268101\\
 26.2951228141785	11.9677392465045\\
% 26.2951516628265	11.9674305647367\\
% 26.2951850414276	11.9671557988206\\
% 26.2952184200287	11.9668810213498\\
 26.2952496528625	11.9666297212611\\
% 26.2952785015106	11.9663435065594\\
% 26.2953056812286	11.9660635462782\\
% 26.2953350067139	11.9657915326562\\
 26.2953686237335	11.9654754743912\\
% 26.2954256057739	11.9651810409447\\
% 26.5964006900787	11.9649156837997\\
% 26.5964049816132	11.9646623206172\\
 26.5964097499847	11.9644506823406\\
% 26.5964138031006	11.9641967023966\\
% 26.5964183330536	11.9639582365153\\
% 26.5964228630066	11.9637233767376\\
 26.7749821662903	11.9634893256207\\
% 26.774987411499	11.9631958491419\\
% 26.7755240917206	11.9629160499712\\
% 26.8470217704773	11.9626352884387\\
 26.8470828056335	11.9623767170431\\
% 26.89402384758	11.9621510329159\\
% 26.927468252182	11.9619386725245\\
% 27.2203940868378	11.9617544445324\\
 27.2204248428345	11.9615916546758\\
% 27.2204520225525	11.9613825609544\\
% 27.2204794406891	11.9611696135501\\
% 27.2205059051514	11.9609467881902\\
 27.2205748081207	11.9606371168354\\
% 28.8658260822296	11.9603314607114\\
% 28.8658315658569	11.960076273698\\
% 28.8668162345886	11.9598497332253\\
 28.866832447052	11.9596412202557\\
% 28.8669096946716	11.9594600205553\\
% 28.8669151782989	11.9592933498191\\
% 28.8669273376465	11.9590813126551\\
 28.8669893264771	11.9588704078418\\
% 28.8669952869415	11.958661684423\\
% 28.8670451164246	11.9584565089897\\
% 28.8670749187469	11.9582645635066\\
 28.8671030521393	11.9580417071495\\
% 28.867130947113	11.9578426619047\\
% 28.8671600341797	11.9576670987388\\
% 28.8671998500824	11.9575219823829\\
 28.8672048568726	11.957411247251\\
% 28.8672348976135	11.9573030164459\\
% 28.8672766208649	11.957203631973\\
% 28.8672813892364	11.9571018991526\\
 28.8674888134003	11.9569947931432\\
% 28.8674966812134	11.9568588438806\\
% 28.8675477027893	11.9565970311737\\
% 28.867590379715	11.9563133005639\\
 28.8676218509674	11.9560276948438\\
% 28.8676533222198	11.955785324028\\
% 28.8676936149597	11.9555711054606\\
% 28.8676976680756	11.9553710679499\\
 28.8677289009094	11.9551975080371\\
% 28.8677708625793	11.9550305201707\\
% 28.8677784919739	11.9548771237849\\
% 28.8678075790405	11.9547168561314\\
 28.867866230011	11.95454813734\\
% 28.8679084300995	11.9543739492342\\
% 29.2079612731934	11.9542066019532\\
% 29.208006811142	11.9540502974633\\
 29.2080163478851	11.9539134326618\\
% 29.2080223083496	11.9537452858422\\
% 29.2080304145813	11.9536199608649\\
% 29.2080440044403	11.9535223553657\\
 29.208407831192	11.9534421919817\\
% 30.8635506153107	11.9533271613735\\
% 30.8635580062866	11.9531994104983\\
% 30.8635627746582	11.9530627783984\\
 30.8635682582855	11.9528432287873\\
% 30.8635725498199	11.9526334393637\\
% 30.8635766029358	11.9524350746077\\
% 30.8635820865631	11.9523113464774\\
 30.8635863780975	11.9521872959818\\
% 30.8635904312134	11.9520767428516\\
% 30.8635944843292	11.9519454450872\\
% 30.8635987758636	11.9518038477122\\
 30.8636030673981	11.9516604742169\\
% 30.8636075973511	11.9515072602327\\
% 30.8636118888855	11.9513485852485\\
% 30.8636159420013	11.951186348499\\
 30.8636209487915	11.9510271192806\\
% 30.8636252403259	11.9508334402994\\
% 30.8636300086975	11.9506503616458\\
% 30.8636340618134	11.9505255243198\\
 30.8636385917664	11.9504162731069\\
% 30.8636426448822	11.9503125447769\\
% 30.8636474132538	11.9502853529345\\
% 30.863651227951	11.9502743073504\\
 30.8636557579041	11.9502971863791\\
% 30.8636600494385	11.9503227493718\\
% 30.8636657714844	11.9502504826207\\
% 30.8636707782745	11.950158209081\\
 30.863675069809	11.9500559864557\\
% 30.8636786460876	11.9498513295149\\
% 30.8636826992035	11.9496385738199\\
% 31.2550806522369	11.9494238447361\\
 31.2551092624664	11.949267387573\\
% 31.2551359653473	11.9491347685489\\
% 31.2553131103516	11.9490245684982\\
% 31.2553185939789	11.9489394190934\\
 31.255323600769	11.9488658146067\\
% 31.2553271770477	11.9487223336913\\
% 31.2553925037384	11.9485720781439\\
% 31.3403334140778	11.948403866293\\
 31.3403391361237	11.9482152167114\\
% 31.5086352348328	11.9480266731838\\
% 31.5086521625519	11.9478425083649\\
% 31.5086788654327	11.9477111733059\\
 31.6173405170441	11.9476006048187\\
% 31.6173724651337	11.9475037630517\\
% 31.6975693225861	11.9474135653096\\
% 31.6975774288177	11.9473318680949\\
 31.7641047954559	11.9472380340763\\
% 31.7724740028381	11.9471400976261\\
% 32.082110118866	11.9470325965915\\
% 32.0831522464752	11.9468990250017\\
 32.0832287788391	11.9467592859781\\
% 32.0832550048828	11.9466121202727\\
% 32.0832790851593	11.9464132503116\\
% 32.0845431804657	11.9462275512289\\
 32.4165415287018	11.9460640576954\\
% 32.4165484428406	11.9459335842628\\
% 32.416555595398	11.9458273765343\\
% 32.4167065143585	11.9458242309917\\
 32.4169623374939	11.9458275323113\\
% 32.4169682979584	11.9458191095084\\
% 32.6960920810699	11.9458066925832\\
% 32.6961004257202	11.9457706601886\\
 32.6961133003235	11.9455775830568\\
% 32.6961235523224	11.9453826226911\\
% 32.696128320694	11.9451903039635\\
% 32.6961371421814	11.9450010579286\\
 32.8045789718628	11.9448980060193\\
% 32.8046054363251	11.9448238589377\\
% 32.8438534259796	11.9447631525615\\
% 32.9417709827423	11.9447051386799\\
 32.9419347763062	11.9446390274844\\
% 33.1483494758606	11.9445504343148\\
% 33.1483871459961	11.9444377293487\\
% 33.1485037326813	11.9443307805332\\
 33.1485960006714	11.9442445695434\\
% 33.2275540351868	11.9441752555957\\
% 33.2620851516724	11.9441174834971\\
% 33.4662851810455	11.9440492005174\\
 33.4662901878357	11.9439690226663\\
% 33.4662944793701	11.9438784023211\\
% 33.4663323879242	11.9437505150835\\
% 33.6951984882355	11.943651174112\\
 33.6952318668366	11.9435543095735\\
% 33.6952564239502	11.943475396349\\
% 33.6952812194824	11.943413187214\\
% 33.6953327178955	11.9433576840822\\
 34.0244714736939	11.9433067497595\\
% 34.0244769573212	11.9432243055512\\
% 34.0244836330414	11.9431383266794\\
% 34.0244886398315	11.9430455264986\\
 34.0244934082031	11.9429577577452\\
% 34.0244967460632	11.942889501579\\
% 34.1406688213348	11.9428333294487\\
% 34.1412770271301	11.9427807229486\\
 34.1414825439453	11.942736661863\\
% 34.3046380996704	11.9426921226609\\
% 34.3046442985535	11.9426475224646\\
% 34.3046488285065	11.9425779695192\\
 34.3536202430725	11.9424929708027\\
% 34.4075736522675	11.9424106127868\\
% 34.4325148582459	11.942326622935\\
% 34.5385291099548	11.942267982228\\
 34.538582277298	11.9422205623676\\
% 34.5713927268982	11.9421894435255\\
% 34.7362963676453	11.9421802267156\\
% 34.7364422798157	11.9421518322602\\
 34.7365619659424	11.9421328852023\\
% 34.8030476093292	11.9421161354978\\
% 34.8534969806671	11.9420596128107\\
% 34.896230173111	11.942001472691\\
 34.9908022403717	11.9419233471234\\
% 34.9910139560699	11.9418485304436\\
% 35.1785239696503	11.9417829492349\\
% 35.178533744812	11.9417262303546\\
 35.1785385131836	11.9416770077885\\
% 35.1785497188568	11.9416220711971\\
% 35.2687682628632	11.9415698502889\\
% 35.3520075798035	11.9415199747666\\
 35.3520118713379	11.9414529623666\\
% 35.4015700340271	11.9413903954882\\
% 35.4856979370117	11.9413348196463\\
% 35.4857115268707	11.9412792086209\\
 35.567130279541	11.9412333850675\\
% 35.6499711990356	11.9411911404788\\
% 35.6500160217285	11.9411418879546\\
% 35.7923717021942	11.9410936442523\\
 35.794321012497	11.9410475793652\\
% 35.7943298339844	11.9409845634762\\
% 35.9174291610718	11.940926660724\\
% 35.9174360752106	11.9408742603871\\
 36.0053495883942	11.9408241455418\\
% 36.0053569793701	11.9407864680282\\
% 36.0758685588837	11.9407481676415\\
% 36.0759078979492	11.9407143048033\\
 36.1870767593384	11.9406813915317\\
% 36.1871056079865	11.9406198148823\\
% 36.2769727230072	11.9405599664922\\
% 36.2769793987274	11.9405010598089\\
 36.3538722515106	11.9404496459985\\
% 36.4040152549744	11.9404052143549\\
% 36.4267644405365	11.9403729809225\\
% 36.4882075309753	11.9403360011463\\
 36.5738670349121	11.9402995813403\\
% 36.7500044822693	11.9402617351734\\
% 36.7500144958496	11.9402008409884\\
% 36.7500192642212	11.9401405053573\\
 36.7500240325928	11.9400819305818\\
% 36.9328815460205	11.9400288230394\\
% 36.9328860759735	11.9399867959497\\
% 36.9328906059265	11.9399546196429\\
 36.9333383560181	11.9399340006295\\
% 37.1490315914154	11.9399219091343\\
% 37.1490804672241	11.9398983363973\\
% 37.1497647285461	11.9398534768214\\
 37.1498274326324	11.9398070319481\\
% 37.3611752510071	11.9397610973707\\
% 37.361182641983	11.9397116382025\\
% 37.3623020172119	11.9396671116449\\
 37.3623060703278	11.9396310028437\\
% 37.4374818325043	11.9396066025223\\
% 37.43752617836	11.9395911949982\\
% 37.6287002086639	11.9395721208394\\
 37.6287106990814	11.9395543392997\\
% 37.6287152290344	11.9395362001875\\
% 37.628727388382	11.9394957013293\\
% 37.8067099571228	11.9394587320981\\
 37.8067221164703	11.9394281007689\\
% 37.8067826747894	11.9393967140793\\
% 37.8460576057434	11.9393775813289\\
% 37.9141819000244	11.9393628469747\\
 37.9315175533295	11.9393438279672\\
% 38.0237917423248	11.9393261177897\\
% 38.0862652778626	11.939293503148\\
% 38.086269569397	11.9392616714367\\
 38.1961585998535	11.9392280007495\\
% 38.196178150177	11.9392000045151\\
% 38.3084940433502	11.9391817116954\\
% 38.3085419654846	11.9391667813475\\
 38.378288936615	11.939153511852\\
% 38.393293094635	11.9391409290312\\
% 38.4296378612518	11.9391136224264\\
% 38.5430137634277	11.9390860424774\\
 38.5430688381195	11.9390570052842\\
% 38.6240517616272	11.939032830113\\
% 38.6530174732208	11.9390142815388\\
% 38.7179064273834	11.9390000893583\\
 38.7788311958313	11.9389891293505\\
% 38.7788407325745	11.9389796975804\\
% 38.8368362903595	11.9389693667257\\
% 38.8831307411194	11.938946219719\\
% 38.919997882843	11.9389236251787\\
 38.9796266078949	11.938904121899\\
% 39.019500207901	11.9388809598927\\
% 39.0791270256043	11.9388731114807\\
% 39.1526405334473	11.9388740245539\\
 39.4971506118774	11.938882447611\\
% 39.49718708992	11.9388979239664\\
% 39.4972180843353	11.9389149018746\\
% 39.4972497940064	11.9389084403994\\
 39.497278881073	11.9389000874607\\
% 39.4973082065582	11.9388894846046\\
% 39.4974827289581	11.9388796217641\\
% 39.65032787323	11.9388644101976\\
 39.6503335952759	11.9388524876448\\
% 39.6503650665283	11.9388457269405\\
% 39.7096879005432	11.9388540352915\\
% 39.7588674545288	11.9388648023029\\
 39.8301410198212	11.9388721922981\\
% 39.8644632816315	11.9388780554687\\
% 39.926452589035	11.9388759026417\\
% 39.9270951271057	11.9388726064995\\
 40.0534979820251	11.938869166709\\
% 40.0535077571869	11.9388686211136\\
% 40.2688049793243	11.9388707587912\\
% 40.2689120292664	11.9388735604558\\
 40.268930387497	11.9388776294878\\
% 40.2689377784729	11.9388916670585\\
% 40.4292718887329	11.9389051256952\\
% 40.4293117046356	11.9389138038099\\
 40.4293787002564	11.9389222657751\\
% 40.4294297218323	11.9389294363535\\
% 40.5301465511322	11.9389366645829\\
% 40.564444732666	11.938949046576\\
 40.7577771663666	11.9389666257077\\
% 40.7585477352142	11.9389848435996\\
% 40.7585613250732	11.9390081056943\\
% 40.7589544773102	11.9390311892911\\
 41.0156762123108	11.9390533112837\\
% 41.0157110214233	11.9390741574035\\
% 41.0157479763031	11.939094247581\\
% 41.015776348114	11.9391146937605\\
 41.0157904148102	11.9391352860054\\
% 41.2113515853882	11.9391556295548\\
% 41.2113830566406	11.9391756168679\\
% 41.2114123821259	11.9391949546471\\
 41.2114569664001	11.9392139841079\\
% 41.3561818122864	11.9392329405196\\
% 41.356210899353	11.9392516504\\
% 41.3562597751617	11.9392693971646\\
 41.4521789073944	11.9392836314565\\
% 41.4522032260895	11.93929538322\\
% 41.5460445404053	11.9393047628917\\
% 41.6698898792267	11.9393141522299\\
 42.0554856777191	11.9393271135925\\
% 42.0562896251678	11.9393398496445\\
% 42.056302022934	11.9393522031653\\
% 42.0563089370728	11.9393636971175\\
 42.0563282489777	11.9393740882781\\
% 42.0563456535339	11.9393740882781\\
% 42.056353521347	11.9393740882781\\
% 42.0563606739044	11.9393740882781\\
 42.0563671112061	11.9393740882781\\
% 42.0563849925995	11.9393740882781\\
% 42.7761535167694	11.9393740882781\\
% 42.7761735439301	11.9393740882781\\
 43.0857395648956	11.9393740882781\\
% 43.0857460021973	11.9393740882781\\
% 43.0857576847076	11.9393740882781\\
% 43.2397584438324	11.9393740882781\\
 43.2397701263428	11.9393740882781\\
% 43.2397839546204	11.9393740882781\\
% 43.2644345283508	11.9393740882781\\
% 43.6193358421326	11.9393740882781\\
 43.6193403720856	11.9393740882781\\
% 43.6193439483643	11.9393740882781\\
% 43.6193475246429	11.9393740882781\\
% 43.7342502593994	11.9393740882781\\
 43.7342824459076	11.9393740882781\\
% 44.1866940975189	11.9393740882781\\
% 44.1866991043091	11.9393740882781\\
% 44.2737109184265	11.9393740882781\\
 44.2737159252167	11.9393740882781\\
% 44.6853835105896	11.9393740882781\\
% 44.6854300022125	11.9393740882781\\
% 44.8329545974731	11.9393740882781\\
 44.8329832077026	11.9393740882781\\
% 44.8330263614655	11.9393740882781\\
% 45.7841484069824	11.9393740882781\\
% 45.78420753479	11.9393740882781\\
 45.7842416286469	11.9393740882781\\
% 45.7842862129211	11.9393740882781\\
% 45.7843179225922	11.9393740882781\\
% 45.7843596458435	11.9393740882781\\
 45.7843942165375	11.9393740882781\\
% 45.7844330787659	11.9393740882781\\
% 45.7844631195068	11.9393740882781\\
% 45.7844976902008	11.9393740882781\\
 45.784534406662	11.9393740882781\\
% 45.7846402645111	11.9393740882781\\
% 46.0884601593018	11.9393740882781\\
% 46.0884670734406	11.9393740882781\\
 46.1257688522339	11.9393740882781\\
% 46.1662382602692	11.9393740882781\\
% 46.2153467655182	11.9393740882781\\
% 46.6415497779846	11.9393740882781\\
 46.7479912757874	11.9393740882781\\
% 46.7480005741119	11.9393740882781\\
% 46.9038605213165	11.9393740882781\\
% 46.9188446521759	11.9393740882781\\
 47.030423116684	11.9393740882781\\
% 47.8224229335785	11.9393740882781\\
% 47.9271270751953	11.9393740882781\\
% 48.2469987392426	11.9393740882781\\
 48.2470311641693	11.9393740882781\\
% 48.247061920166	11.9393740882781\\
% 48.6642553329468	11.9393740882781\\
% 48.9163457870483	11.9393740882781\\
 48.9163507938385	11.9393740882781\\
% 48.9173902988434	11.9393740882781\\
% 49.5241639137268	11.9393740882781\\
% 49.5241684436798	11.9393740882781\\
 49.525897693634	11.9393740882781\\
% 49.5259534835815	11.9393740882781\\
% 50.0419322967529	11.9393740882781\\
% 51.5745617866516	11.9393740882781\\
 51.5749794960022	11.9393740882781\\
% 51.5749883174896	11.9393740882781\\
% 51.8694738864899	11.9393740882781\\
% 51.869504404068	11.9393740882781\\
 51.8695332527161	11.9393740882781\\
% 51.869561624527	11.9393740882781\\
% 51.8696019172668	11.9393740882781\\
% 51.8696481704712	11.9393740882781\\
 53.5224811553955	11.9393740882781\\
% 53.5224897384644	11.9393740882781\\
% 53.5224947452545	11.9393740882781\\
% 53.5224995136261	11.9393740882781\\
 53.5225045204163	11.9393740882781\\
% 53.5225088119507	11.9393740882781\\
% 53.5225133419037	11.9393740882781\\
% 53.5225185871124	11.9393740882781\\
 53.52253074646	11.9393740882781\\
% 53.5225345611572	11.9393740882781\\
% 53.5225390911102	11.9393740882781\\
% 53.5225431442261	11.9393740882781\\
 53.8357407569885	11.9393740882781\\
% 53.8357474327087	11.9393740882781\\
% 53.8357564926147	11.9393740882781\\
% 53.8360311508179	11.9393740882781\\
 53.8361589431763	11.9393740882781\\
% 53.8363892555237	11.9393740882781\\
% 54.9593457698822	11.9393740882781\\
% 54.9593765258789	11.9393740882781\\
 54.9594101428986	11.9393740882781\\
% 54.9594618797302	11.9393740882781\\
% 54.9596836090088	11.9393740882781\\
% 54.9597208023071	11.9393740882781\\
 54.9597517967224	11.9393740882781\\
% 54.9597842216492	11.9393740882781\\
% 54.9599391937256	11.9393740882781\\
% 54.9599730491638	11.9393740882781\\
 54.9600042819977	11.9393740882781\\
% 54.9601048946381	11.9393740882781\\
% 55.7669550895691	11.9393740882781\\
% 55.766986322403	11.9393740882781\\
 55.767018032074	11.9393740882781\\
% 55.7670630931854	11.9393740882781\\
% 55.7670952796936	11.9393740882781\\
% 55.7671293735504	11.9393740882781\\
 55.767174911499	11.9393740882781\\
% 55.7672061443329	11.9393740882781\\
% 55.76724858284	11.9393740882781\\
% 55.7672833919525	11.9393740882781\\
 55.7673131942749	11.9393740882781\\
% 55.7673434734345	11.9393740882781\\
% 55.7673713684082	11.9393740882781\\
% 55.7674064159393	11.9393740882781\\
 55.7674335956574	11.9393740882781\\
% 55.7674622058868	11.9393740882781\\
% 55.7675101280212	11.9393740882781\\
% 56.4338857650757	11.9393740882781\\
 56.4339029312134	11.9393740882781\\
% 56.4339148521423	11.9393740882781\\
% 56.4339248657227	11.9393740882781\\
% 56.4339437007904	11.9393740882781\\
 56.4340469360352	11.9393740882781\\
% 56.4340550422668	11.9393740882781\\
% 56.4340698242188	11.9393740882781\\
% 56.4340769767761	11.9393740882781\\
 56.4340853214264	11.9393740882781\\
% 56.4340981960297	11.9393740882781\\
% 56.4342569828033	11.9393740882781\\
% 56.4342746257782	11.9393740882781\\
 56.8145236492157	11.9393740882781\\
% 56.9261416912079	11.9393740882781\\
% 57.0169128894806	11.9393740882781\\
% 57.0169181346893	11.9393740882781\\
 58.5536970615387	11.9393740882781\\
% 58.5537037372589	11.9393740882781\\
% 58.553709936142	11.9393740882781\\
% 58.5537163734436	11.9393740882781\\
 58.5537225723267	11.9393740882781\\
% 58.5537290096283	11.9393740882781\\
% 58.5537354469299	11.9393740882781\\
% 58.5537423610687	11.9393740882781\\
 58.5537490367889	11.9393740882781\\
% 58.5537545204163	11.9393740882781\\
% 58.5537600040436	11.9393740882781\\
% 58.5537657260895	11.9393740882781\\
 58.5537721633911	11.9393740882781\\
% 58.5537781238556	11.9393740882781\\
% 58.5537843227387	11.9393740882781\\
% 58.5537902832031	11.9393740882781\\
 58.5551464080811	11.9393740882781\\
% 58.5551602363586	11.9393740882781\\
% 58.5551681041718	11.9393740882781\\
% 58.5551752567291	11.9393740882781\\
 58.5551819324493	11.9393740882781\\
% 58.9172229290009	11.9393740882781\\
% 58.9172555923462	11.9393740882781\\
% 58.9172882556915	11.9393740882781\\
 58.9172951698303	11.9393740882781\\
% 58.9173254489899	11.9393740882781\\
% 58.9173306941986	11.9393740882781\\
% 58.9173583507538	11.9393740882781\\
 58.9174062728882	11.9393740882781\\
% 59.1581761360168	11.9393740882781\\
% 59.1582495689392	11.9393740882781\\
% 59.1583513736725	11.9393740882781\\
 59.1584672451019	11.9393740882781\\
% 59.3863837242126	11.9393740882781\\
% 59.3863975524902	11.9393740882781\\
% 59.3864140033722	11.9393740882781\\
 59.3864302158356	11.9393740882781\\
% 59.3864404678345	11.9393740882781\\
% 59.8614983081818	11.9393740882781\\
% 59.865093421936	11.9393740882781\\
 59.8651005744934	11.9393740882781\\
% 59.8657540798187	11.9393740882781\\
% 59.8657629013062	11.9393740882781\\
% 59.8659624576569	11.9393740882781\\
 59.8659700870514	11.9393740882781\\
% 59.8662604808807	11.9393740882781\\
% 59.8662702560425	11.9393740882781\\
% 60.6861924648285	11.9393740882781\\
 60.686212015152	11.9393740882781\\
% 60.6862289428711	11.9393740882781\\
% 60.6862382411957	11.9393740882781\\
% 60.686255645752	11.9393740882781\\
 60.6862613677979	11.9393740882781\\
% 60.6882712364197	11.9393740882781\\
% 60.6882829189301	11.9393740882781\\
% 60.6882910251617	11.9393740882781\\
 60.6882996082306	11.9393740882781\\
% 60.6883103370667	11.9393740882781\\
% 60.6883182048798	11.9393740882781\\
% 60.6883239269257	11.9393740882781\\
 60.6883284568787	11.9393740882781\\
% 60.6883360862732	11.9393740882781\\
% 60.6883420467377	11.9393740882781\\
% 60.6883496761322	11.9393740882781\\
 60.9728793621063	11.9393740882781\\
% 60.9912290096283	11.9393740882781\\
% 60.9912459373474	11.9393740882781\\
% 60.9912766933441	11.9393740882781\\
 60.9912983894348	11.9393740882781\\
% 60.9992479801178	11.9393740882781\\
};
\addlegendentry{$\theta_{1}$}


\addplot[area legend, draw=none, fill=black, fill opacity=0.09, forget plot]
table[row sep=crcr] {%
x	y\\
0	-4\\
40.65	-4\\
40.65	14\\
0	14\\
}--cycle;
\end{axis}

\begin{axis}[%
width=0in,
height=0in,
at={(0in,0in)},
scale only axis,
xmin=0,
xmax=1,
ymin=0,
ymax=1,
axis line style={draw=none},
ticks=none,
axis x line*=bottom,
axis y line*=left
]
\end{axis}
\end{tikzpicture}%
\vspace{-0.7em}
\caption{Coefficient estimation over time applying Algorithm~\ref{alg:twolayermpcalglearningimp} in consideration of an initially unknown $\phi_{2}$. The grey background indicates time instances for which the agents are exploring.}
\label{fig:twolayerslearningcoef}
\vspace{-1.0em}
\end{figure}
\vspace{-0.4em}
\subsection{Results One-Layer Coverage MPC}
\label{subsec:experimentonelayer}
For the application of Algorithm~\ref{alg:onelayeralg} in the described set-up using $\phi_{1}$, we obtain the locational optimization cost as well as the cumulative target cost decrease presented in Figure~\ref{fig:onelayercoveragecost}. 
\begin{figure} [h!]
\centering
% This file was created by matlab2tikz.
%
%The latest updates can be retrieved from
%  http://www.mathworks.com/matlabcentral/fileexchange/22022-matlab2tikz-matlab2tikz
%where you can also make suggestions and rate matlab2tikz.
%
\definecolor{mycolor1}{rgb}{0.47000,0.67000,0.19000}%
\definecolor{mycolor2}{rgb}{0.89000,0.59000,0.53000}%
%
\begin{tikzpicture}

\begin{axis}[%
height=0.95in,
width=0.40\textwidth,
yshift=0.8cm,
at={(0.0in,0.0in)},
scale only axis,
xmin=0,
xmax=25.3307616710663,
xlabel style={font=\color{white!15!black}},
xlabel style={font=\footnotesize},
xlabel style={yshift=0.6ex,},
xlabel={Time [sec]},
ymin=0,
ymax=100,
ylabel style={font=\color{white!15!black}},
ylabel style={font=\footnotesize},
ylabel style={xshift=0.6ex,},
ylabel={H(p,$\mathbb{W}$) \& H($\bar{p}$,$\mathbb{W}$)},
axis background/.style={fill=white},
tick label style={font=\footnotesize},
xmajorgrids,
ymajorgrids,
legend style={legend cell align=left, align=left, draw=white!15!black},
legend style={font=\footnotesize}
]
\addplot [color=mycolor1, line width=1.4pt]
  table[row sep=crcr]{%
 -0.752678871154785	97.9261835074132\\
% -0.61667537689209	97.9235471385173\\
% -0.431777238845825	97.8456419027995\\
% -0.304361581802368	96.9117950255414\\
 -0.170496463775635	95.2026404938776\\
% 0.0511531829833984	93.5435885341303\\
% 0.248857736587524	92.2623745888061\\
% 0.398529291152954	90.026906830959\\
 0.703789710998535	87.6633450806331\\
% 0.818437337875366	86.2824576568983\\
% 1.05893611907959	84.1478843429725\\
% 1.22304081916809	82.8727704346556\\
 1.36801433563232	81.2757739315223\\
% 1.56077289581299	79.7669228490952\\
% 1.75548720359802	77.7285236712399\\
% 1.94184041023254	75.9153063996265\\
 2.09567832946777	74.1793567897177\\
% 2.22142171859741	72.5926707533859\\
% 2.3580162525177	71.3220504958444\\
% 2.54110312461853	69.9788597130346\\
 2.6709840297699	68.95190804127\\
% 2.80001330375671	67.9694032716722\\
% 2.94930195808411	66.685223581493\\
% 3.02313780784607	66.1158218125278\\
 3.21049380302429	64.572825866155\\
% 3.30067825317383	63.4814333340537\\
% 3.60484051704407	61.0058411735866\\
% 3.77515506744385	58.5232118224019\\
 3.84762120246887	58.5206153527019\\
% 4.02376914024353	56.1167730777753\\
% 4.22069692611694	54.575885287451\\
% 4.33111357688904	53.4217035460305\\
 4.43757009506226	52.3762454532487\\
% 4.58952450752258	51.2429490788439\\
% 4.81252813339233	49.656223484726\\
% 4.96443629264832	48.20938275447\\
 5.12998819351196	46.8163853169604\\
% 5.30091428756714	45.4320130155381\\
% 5.35768866539001	44.807745971548\\
% 5.54773354530334	43.1366945479841\\
 5.64689445495605	42.5635902697352\\
% 5.78338193893433	41.6539539356269\\
% 5.91341590881348	40.8248248844356\\
% 6.09682011604309	39.7615217848914\\
 6.27487683296204	38.5734927870055\\
% 6.51981401443481	36.5472926976338\\
% 6.62030243873596	36.0480287676247\\
% 6.77694392204285	34.6582107299174\\
 6.85331034660339	34.2175401174465\\
% 7.00919485092163	32.9338064653659\\
% 7.20971250534058	31.7102028418202\\
% 7.3954074382782	30.9346641163149\\
 7.47528767585754	30.8698548308237\\
% 7.64361238479614	30.6915347847871\\
% 7.78141713142395	30.035178470727\\
% 7.95131397247314	29.1260743660754\\
% 8.1119179725647	28.1307013615035\\
 8.235520362854	27.274923193687\\
% 8.45801091194153	25.7724442636854\\
% 8.69429516792297	24.9753251257048\\
% 8.7844672203064	24.5019533307573\\
 8.93292856216431	23.3761529267306\\
% 9.12631440162659	22.2000789912582\\
% 9.38094067573547	20.9989562372349\\
% 9.55542182922363	19.7432584062385\\
 9.70495367050171	18.8861427396862\\
% 9.84391689300537	18.3771467942708\\
% 10.0529608726501	17.1817419459393\\
% 10.1821112632751	16.3612738007476\\
 10.6786482334137	15.2512509806442\\
% 10.6787111759186	15.2512509806442\\
% 10.8570318222046	13.8968395842906\\
% 10.9726643562317	13.1204965279191\\
 11.1019060611725	12.7397716303064\\
% 11.2957096099854	12.2612365004617\\
% 11.4397356510162	11.8787881542414\\
% 11.5285334587097	11.8234324845714\\
 11.7108900547028	10.9702705398059\\
% 11.9440429210663	11.3767510885031\\
% 12.0432002544403	11.0357824572337\\
% 12.2909953594208	10.1948002403646\\
 12.4994716644287	9.38039045710852\\
% 12.7165651321411	8.69106231472487\\
% 12.9490962028503	8.89631076697001\\
% 13.1039261817932	8.36685850657455\\
 13.3131551742554	7.5287587356547\\
% 13.4769022464752	7.06418406034529\\
% 13.6352708339691	7.43575973792294\\
% 13.7497515678406	6.99168555249827\\
 13.9693222045898	7.1242725255925\\
% 14.0935587882996	6.83600679694388\\
% 14.3345820903778	6.22508578716411\\
% 14.4886283874512	5.95856006704196\\
 14.6366865634918	5.54631747129091\\
% 14.8857729434967	5.10009551460488\\
% 14.9787616729736	5.02617987544193\\
% 15.204327583313	4.79941022034252\\
 15.3200981616974	4.65311734191458\\
% 15.5124068260193	4.26350129878597\\
% 15.6531028747559	4.1410860782638\\
% 15.7691702842712	3.96994171111974\\
 15.8480935096741	3.95065127528456\\
% 16.0110819339752	3.84403754410932\\
% 16.1660482883453	3.62105514418661\\
% 16.2790815830231	3.60814947119849\\
 16.5165019035339	3.4853462316023\\
% 16.5974833965302	3.45849513306002\\
% 16.7026789188385	3.3706868515419\\
% 16.9388484954834	3.22584821654862\\
 16.9673933982849	3.17850059871171\\
% 17.2013874053955	3.13596342712042\\
% 17.3178255558014	3.02973860344121\\
% 17.5166654586792	2.96065357906586\\
 17.8063726425171	2.93003112274219\\
% 17.9396996498108	2.83474529556394\\
% 18.1558725833893	2.90313898728074\\
% 18.3329045772552	2.84608978906323\\
 18.5583703517914	2.95793400067437\\
% 18.8119606971741	2.80672060826011\\
% 18.9484868049622	2.90664396905864\\
% 19.1286723613739	2.83427492169805\\
 19.4161057472229	2.92928277761334\\
% 19.5076582431793	2.85096294155628\\
% 19.6566977500916	2.93962579140266\\
% 19.7684731483459	2.88155024142863\\
 19.9211452007294	2.91914780235644\\
% 20.9479937553406	2.76787368246699\\
% 21.0826144218445	2.75573403761934\\
% 21.1669147014618	2.77035887444555\\
 21.3112151622772	2.79540506016242\\
% 21.4806368350983	2.85128102689137\\
% 21.574159860611	2.86572738152075\\
% 21.6827952861786	2.86692107205378\\
 21.8488836288452	2.86510750561852\\
% 22.004537820816	2.85408554330827\\
% 22.118185043335	2.88363655527461\\
% 22.2612323760986	2.90433145247043\\
 22.4250974655151	2.89663736237433\\
% 22.5900375843048	2.87753588433102\\
% 22.764110326767	2.89351436669891\\
% 22.9147973060608	2.89766880316227\\
 23.0184512138367	2.86506267903808\\
% 23.1471710205078	2.81776466292077\\
% 23.2823204994202	2.81947408227146\\
% 23.5483236312866	2.79762049792858\\
 23.6712720394135	2.75953604498169\\
% 23.806471824646	2.76818393816441\\
% 23.8850417137146	2.78423958568272\\
% 24.0690824985504	2.80342791771918\\
 24.2575709819794	2.78784858251353\\
% 24.3878412246704	2.75094628470568\\
% 24.5714821815491	2.74237073560486\\
% 24.8079340457916	2.77803609365994\\
 25.0052394866943	2.76948265257434\\
% 25.1679096221924	2.73466805363196\\
% 25.3307616710663	2.72010248027264\\
};
\addlegendentry{H(p,W)}

\addplot [color=mycolor2, line width=1.4pt]
  table[row sep=crcr]{%
 -0.752678871154785	7.56285490483829\\
% -0.61667537689209	7.56297384264219\\
% -0.431777238845825	10.0293222275859\\
% -0.304361581802368	18.6688819737576\\
 -0.170496463775635	23.1009022125737\\
% 0.0511531829833984	23.8763961576362\\
% 0.248857736587524	23.616922238403\\
% 0.398529291152954	22.9151019063225\\
 0.703789710998535	21.6316576140315\\
% 0.818437337875366	21.3684391251852\\
% 1.05893611907959	20.5987775403987\\
% 1.22304081916809	19.8419302859445\\
 1.36801433563232	19.4246139278471\\
% 1.56077289581299	18.7912649045521\\
% 1.75548720359802	18.2199743625534\\
% 1.94184041023254	17.4568313967572\\
 2.09567832946777	16.9419081134732\\
% 2.22142171859741	16.623831524297\\
% 2.3580162525177	15.929495459725\\
% 2.54110312461853	15.4859085112028\\
 2.6709840297699	15.1764839576916\\
% 2.80001330375671	14.9598037519708\\
% 2.94930195808411	14.5176920303934\\
% 3.02313780784607	14.4724438678095\\
 3.21049380302429	13.977566018927\\
% 3.30067825317383	13.7946000980355\\
% 3.60484051704407	12.7496669906113\\
% 3.77515506744385	12.4221490359563\\
 3.84762120246887	12.1586459774128\\
% 4.02376914024353	11.8243782178084\\
% 4.22069692611694	11.3050124251582\\
% 4.33111357688904	11.0690007985341\\
 4.43757009506226	10.8587895828632\\
% 4.58952450752258	10.525313540109\\
% 4.81252813339233	10.2301346576364\\
% 4.96443629264832	9.25523354123375\\
 5.12998819351196	8.05194349002117\\
% 5.30091428756714	7.746805915319\\
% 5.35768866539001	7.6019290492317\\
% 5.54773354530334	7.35184486799693\\
 5.64689445495605	7.27543960091199\\
% 5.78338193893433	7.17241861060426\\
% 5.91341590881348	7.06289754913507\\
% 6.09682011604309	6.91935647575397\\
 6.27487683296204	6.74750154916436\\
% 6.51981401443481	6.5234704049213\\
% 6.62030243873596	6.4532297157707\\
% 6.77694392204285	6.39080808318853\\
 6.85331034660339	6.34689457144909\\
% 7.00919485092163	6.2347013746235\\
% 7.20971250534058	6.14473301816492\\
% 7.3954074382782	6.08621391039704\\
 7.47528767585754	6.08606045197086\\
% 7.64361238479614	6.07741342868004\\
% 7.78141713142395	6.05496332780166\\
% 7.95131397247314	6.00969005652814\\
 8.1119179725647	5.97049560618141\\
% 8.235520362854	5.94808109424288\\
% 8.45801091194153	5.90046145511005\\
% 8.69429516792297	5.89397367830342\\
 8.7844672203064	5.88541973178094\\
% 8.93292856216431	5.87060601125861\\
% 9.12631440162659	5.86103360492009\\
% 9.38094067573547	5.85643605559228\\
 9.55542182922363	5.85422132280676\\
% 9.70495367050171	5.85426998596367\\
% 9.84391689300537	4.56963006937618\\
% 10.0529608726501	4.18928403842919\\
 10.1821112632751	3.96524631917966\\
% 10.6786482334137	3.96618124904489\\
% 10.6787111759186	3.96618124904489\\
% 10.8570318222046	3.96696675764339\\
 10.9726643562317	3.96724422073821\\
% 11.1019060611725	3.9673202409883\\
% 11.2957096099854	3.96746849839068\\
% 11.4397356510162	3.96751962161736\\
 11.5285334587097	3.96751983970998\\
% 11.7108900547028	3.96760646612805\\
% 11.9440429210663	3.3702532977907\\
% 12.0432002544403	3.36998831721674\\
 12.2909953594208	3.0904216760129\\
% 12.4994716644287	3.09048072703809\\
% 12.7165651321411	3.09050517532083\\
% 12.9490962028503	2.87345060689621\\
 13.1039261817932	2.84318597702762\\
% 13.3131551742554	2.75031954793988\\
% 13.4769022464752	2.75031275537445\\
% 13.6352708339691	2.65175708041828\\
 13.7497515678406	2.6517154408047\\
% 13.9693222045898	2.58615944244385\\
% 14.0935587882996	2.58614441149627\\
% 14.3345820903778	2.5628522735754\\
 14.4886283874512	2.56172268845075\\
% 14.6366865634918	2.55811791947846\\
% 14.8857729434967	2.56126119584634\\
% 14.9787616729736	2.55169336193798\\
 15.204327583313	2.5543118114905\\
% 15.3200981616974	2.54491911702307\\
% 15.5124068260193	2.5697789018863\\
% 15.6531028747559	2.54392076316301\\
 15.7691702842712	2.54724880670057\\
% 15.8480935096741	2.54451732552229\\
% 16.0110819339752	2.55013060354415\\
% 16.1660482883453	2.5559709838747\\
 16.2790815830231	2.54423835652103\\
% 16.5165019035339	2.55267475224853\\
% 16.5974833965302	2.54936160207208\\
% 16.7026789188385	2.546760993705\\
 16.9388484954834	2.54508498712576\\
% 16.9673933982849	2.54271272815845\\
% 17.2013874053955	2.54654941771184\\
% 17.3178255558014	2.5457527565031\\
 17.5166654586792	2.54624041376523\\
% 17.8063726425171	2.55635364029489\\
% 17.9396996498108	2.54868166404025\\
% 18.1558725833893	2.58378311337337\\
 18.3329045772552	2.58765541802843\\
% 18.5583703517914	2.5824257154562\\
% 18.8119606971741	2.57880744302331\\
% 18.9484868049622	2.60285697673233\\
 19.1286723613739	2.58614494505635\\
% 19.4161057472229	2.57324851703747\\
% 19.5076582431793	2.54727471019\\
% 19.6566977500916	2.5556730232244\\
 19.7684731483459	2.54510481025136\\
% 19.9211452007294	2.55700859289565\\
% 20.9479937553406	2.55044329256388\\
% 21.0826144218445	2.55255797376511\\
 21.1669147014618	2.55008022232164\\
% 21.3112151622772	2.54849370988001\\
% 21.4806368350983	2.54849411628334\\
% 21.574159860611	2.54849731597863\\
 21.6827952861786	2.54848261588244\\
% 21.8488836288452	2.54849384462625\\
% 22.004537820816	2.54849364416863\\
% 22.118185043335	2.54847621355385\\
 22.2612323760986	2.54849977461287\\
% 22.4250974655151	2.54848571366345\\
% 22.5900375843048	2.54849387533637\\
% 22.764110326767	2.54849350007978\\
 22.9147973060608	2.5484813062398\\
% 23.0184512138367	2.54849640058705\\
% 23.1471710205078	2.54849448921071\\
% 23.2823204994202	2.54849380610677\\
 23.5483236312866	2.54849136365606\\
% 23.6712720394135	2.54849393167468\\
% 23.806471824646	2.54849467974347\\
% 23.8850417137146	2.54850193718172\\
 24.0690824985504	2.54849349944944\\
% 24.2575709819794	2.54849171614812\\
% 24.3878412246704	2.54849604642902\\
% 24.5714821815491	2.54849606119896\\
 24.8079340457916	2.54849384161373\\
% 25.0052394866943	2.5484924755392\\
% 25.1679096221924	2.5484939806542\\
 25.3307616710663	2.54849529645036\\
};
\addlegendentry{H($\bar{p}$,$\mathbb{W}$)}

\end{axis}
\end{tikzpicture}%
\vspace{-0.7em}
\caption{Locational optimization cost decrease (green) as well as over all agents summed up target cost decrease over time (rose), applying Algorithm~\ref{alg:onelayeralg} with respect to a known density $\phi_{1}$.}
\label{fig:onelayercoveragecost}
\vspace{-0.5em}
\end{figure}


\begin{figure}[h!]
\begin{minipage}[t]{0.15\textwidth}
\centering
\includegraphics[trim={5.2cm 8.1cm 4.6cm 7.7cm},clip, width = 0.8\textwidth]{figures/config_6sec_one_layer_2_220323_1215-compressed.pdf}
\centering
\end{minipage}
\hfill
\begin{minipage}[t]{0.15\textwidth}
\centering
\includegraphics[trim={5.2cm 8.1cm 4.6cm 7.7cm},clip, width = 0.8\textwidth]{figures/config_13sec_one_layer_2_220323_1215-compressed.pdf}
\centering
\end{minipage}
\hfill
\begin{minipage}[t]{0.15\textwidth}
\centering
\includegraphics[trim={5.2cm 8.1cm 4.6cm 7.7cm},clip, width = 0.8\textwidth]{figures/config_30sec_one_layer_2_220323_1215-compressed.pdf}
\centering
\end{minipage}
\caption{Configurations of cars at 2, 8, and 25 seconds for the application of Algorithm~\ref{alg:onelayeralg} in the described set-up. The agents' location and their predicted trajectory are given in red, the Voronoi partitions in green, the artificial setpoints in blue, and the traveled paths are visualized in light grey.}
\label{pics:onelayerconfig}
\vspace{-0.7em}
\end{figure}

It shows a comparable behavior to the two-layers approach regarding the time required until convergence, as well as the final configuration. While at certain instances in time the locational optimization cost increases slightly, the summed up target cost is strictly decreasing. 
\subsection{Results One-Layer, Learning-Based Coverage MPC}
\label{subsec:experimentonelayerlearning}
In a final experiment, Algorithm~\ref{alg:onelayerlearningalg} is applied, with its corresponding density $\phi_{2}$ initially unknown and $S$=2.5. Allowing for comparability, the same values are investigated as in the previous sections.  
\begin{figure} [h!]
%\centering
% This file was created by matlab2tikz.
%
%The latest updates can be retrieved from
%  http://www.mathworks.com/matlabcentral/fileexchange/22022-matlab2tikz-matlab2tikz
%where you can also make suggestions and rate matlab2tikz.
%
\definecolor{mycolor1}{rgb}{0.47000,0.67000,0.19000}%
\definecolor{mycolor2}{rgb}{0.79000,0.49000,0.43000}%
%
\begin{tikzpicture}

\begin{axis}[%
%width=6.862in,
height=0.95in,
width=0.40\textwidth,
yshift=0.8cm,
at={(0.0in,0.0in)},
scale only axis,
xmin=0,
xmax=24.5,
xlabel style={font=\color{white!15!black}},
xlabel style={font=\footnotesize},
xlabel style={yshift=0.6ex,},
xlabel={Time [sec]},
ymin=0,
ymax=100,
ylabel style={font=\color{white!15!black}},
ylabel style={font=\footnotesize},
ylabel style={xshift = 0.6ex,},
ylabel={H(p,$\mathbb{W}$,$\phi$) \& H(p,$\mathbb{W}$,$\hat{\phi}$)},
axis background/.style={fill=white},
tick label style={font=\footnotesize},
%title style={font=\bfseries},
%title={},
xmajorgrids,
ymajorgrids,
legend style={legend cell align=left, align=left, draw=white!15!black},
legend style={font=\footnotesize}
]
\addplot [color=mycolor1, line width=1.4pt]
  table[row sep=crcr]{%
 0.049284887313843	67.9524050654734\\
% 0.521220636367798	67.955086024249\\
% 0.720788669586182	67.6719910658091\\
% 0.939451169967652	65.9061916721829\\
 1.24817414283752	64.4342323277812\\
% 1.35687227249146	63.7699667521215\\
% 1.50339193344116	62.5301719260544\\
% 1.69203538894653	60.8600801308309\\
 1.92890639305115	58.2299922390615\\
% 2.06619520187378	57.2668032320519\\
% 2.27335591316223	55.9028732246555\\
% 2.48976130485535	54.5760321261274\\
 2.73564667701721	53.1483212962923\\
% 2.89879388809204	52.6110056185696\\
% 3.08453054428101	52.2353256385558\\
% 3.26425213813782	51.553752042749\\
 3.45619125366211	50.9280849292508\\
% 3.65377326011658	50.2792813239695\\
% 3.90644593238831	49.7144826767073\\
% 4.05354161262512	49.3450111422343\\
 4.27878136634827	49.0686523584347\\
% 4.50914735794067	48.6991586849296\\
% 4.65725631713867	48.6123600106592\\
% 4.95371408462524	48.0379359863517\\
 5.10187644958496	47.4874141672045\\
% 5.28157181739807	46.8076169307819\\
% 5.47542161941528	45.9573869002807\\
% 5.76581401824951	44.5300867621344\\
 5.9406313419342	43.5381750373614\\
% 6.15238089561462	43.0164831662849\\
% 6.35639019012451	42.3055493778849\\
% 6.46701831817627	41.7737697538998\\
 6.7211296081543	40.5185260756877\\
% 6.90273065567017	39.0656193544728\\
% 7.05933113098145	38.4943111688127\\
% 7.30348343849182	36.7834866536416\\
 7.47501749992371	35.7849463562373\\
% 7.6738917350769	35.1685099194417\\
% 7.86925811767578	34.0128733001812\\
% 8.17200989723205	32.7632584649244\\
 8.32736535072327	32.1222369666807\\
% 8.46725125312805	27.3067494055079\\
% 8.68271727561951	26.7579496530643\\
% 8.9227189540863	25.4754453075929\\
 9.14097018241882	24.4182296783686\\
% 9.38154573440552	23.9679275744211\\
% 9.46438498497009	23.9431267620717\\
% 9.67332692146301	23.0356233156769\\
 9.86595768928528	22.7481675239553\\
% 10.0558857440948	22.5639723233798\\
% 10.3858830451965	22.5386378711378\\
% 10.4936358451843	22.5484356757803\\
 10.776785326004	22.5774742474748\\
% 10.8545956134796	22.6645200156009\\
% 11.1029264450073	22.7691091086258\\
% 11.2770318508148	22.7481222562891\\
 11.5893284797668	22.6098069752684\\
% 11.6759116172791	22.5658438443616\\
% 11.9269306182861	22.3913469514645\\
% 12.0506445884705	22.3490042924414\\
 12.3151247024536	21.6969076560135\\
% 12.492760848999	21.0560008818383\\
% 12.7263946056366	21.762743001003\\
% 12.8795260906219	20.9923219718744\\
 13.1556639194489	20.2818847012124\\
% 13.3137912273407	19.4749980833408\\
% 13.6068145751953	18.3517481196682\\
% 13.665696811676	18.6354616351512\\
 13.9014417648315	17.6004527639742\\
% 14.1454388618469	16.8136794109449\\
% 14.2738837718964	16.3695585018433\\
% 14.5163492679596	15.6184526247604\\
 14.7443801879883	14.700420685577\\
% 14.9911911010742	13.9778454896733\\
% 15.091393661499	13.698295978214\\
% 15.3195831298828	13.1820765885452\\
 15.4853744029999	12.6669704438\\
% 15.7133118629456	12.1792460731827\\
% 15.9334220409393	11.9082237269886\\
% 16.0791105747223	11.6586300905852\\
 16.266965341568	11.4220834390524\\
% 16.5023197650909	11.1935859798008\\
% 16.6838082790375	11.0971961385648\\
% 16.8699030399323	11.037396092819\\
 17.1469637870789	10.961084953814\\
% 17.2530440807343	10.9270593708809\\
% 17.459480714798	10.8939764977075\\
% 17.7267465114593	10.8680852752909\\
 17.870863866806	10.8622430623763\\
% 18.1831285476685	10.8367666320133\\
% 18.3009466648102	10.8251254705976\\
% 18.4912693023682	10.8144499353545\\
 18.7064876079559	10.8457607285418\\
% 18.9133874893188	10.8746536598965\\
% 19.2820596218109	10.8903853115047\\
% 19.3635892391205	10.8879537980267\\
 19.7252389907837	10.90729773852\\
% 19.7252447128296	10.90729773852\\
% 19.8903939247131	10.9059580790515\\
% 20.168097448349	10.9303853489136\\
 20.3190359592438	10.9263785387802\\
% 20.4911682128906	10.9138176797472\\
% 20.657948923111	10.906470484052\\
% 20.9995102405548	10.8749349794804\\
 21.2234882831573	10.8645804929238\\
% 21.315204334259	10.8553206862775\\
% 21.5460283279419	10.8349860658515\\
% 21.8071853637695	10.834442868285\\
 21.9854676246643	10.830438547175\\
% 22.1622938632965	10.890946269262\\
% 22.2758011341095	10.903769229408\\
% 22.5343856334686	10.8780099249258\\
 22.6996335506439	10.8407676371976\\
% 22.8794180870056	10.8339190258637\\
% 23.0871574401855	10.8185861026012\\
% 23.3353487968445	10.8050065068236\\
 23.537798833847	10.8007718169455\\
% 23.6523868560791	10.794201639302\\
% 23.8663136482239	10.7955686443523\\
% 24.1272313117981	10.7868846929752\\
 24.3637666225433	10.8029684409071\\
% 24.5188874721527	10.8195814970693\\
};
\addlegendentry{H(p,$\mathbb{W}$,$\phi$)}

\addplot [color=mycolor2, dotted, line width=1.4pt]
  table[row sep=crcr]{%
 0.049284887313843	0\\
% 0.521220636367798	4.02641260197319\\
% 0.720788669586182	3.94298141673725\\
% 0.939451169967652	5.17147426994991\\
 1.24817414283752	8.3186572633275\\
% 1.35687227249146	10.1905000151175\\
% 1.50339193344116	9.83124776564314\\
% 1.69203538894653	19.9937706228761\\
 1.92890639305115	27.4812410192702\\
% 2.06619520187378	37.1107078435219\\
% 2.27335591316223	46.4777812168114\\
% 2.48976130485535	56.0607523528025\\
 2.73564667701721	68.7622411253062\\
% 2.89879388809204	72.0906153223525\\
% 3.08453054428101	78.516002001716\\
% 3.26425213813782	83.0586467367706\\
 3.45619125366211	86.0542813178378\\
% 3.65377326011658	87.4381073046365\\
% 3.90644593238831	87.5549080120548\\
% 4.05354161262512	86.8637391411668\\
 4.27878136634827	85.7594267305667\\
% 4.50914735794067	83.9118552547847\\
% 4.65725631713867	82.2934987379881\\
% 4.95371408462524	79.7565726808781\\
 5.10187644958496	76.9483858759916\\
% 5.28157181739807	73.9081903021012\\
% 5.47542161941528	70.6186149339358\\
% 5.76581401824951	65.9388253172175\\
 5.9406313419342	62.9155179898021\\
% 6.15238089561462	61.5232701164209\\
% 6.35639019012451	58.9153444939588\\
% 6.46701831817627	57.632745850861\\
 6.7211296081543	54.829237865703\\
% 6.90273065567017	51.8670968482894\\
% 7.05933113098145	50.6367840494653\\
% 7.30348343849182	47.1373545121347\\
 7.47501749992371	45.1597366895902\\
% 7.6738917350769	43.7605148133252\\
% 7.86925811767578	41.7811121952913\\
% 8.17200989723205	39.4976762818594\\
 8.32736535072327	38.5084418155317\\
% 8.46725125312805	32.827627220541\\
% 8.68271727561951	32.0433480811652\\
% 8.9227189540863	29.8444090638165\\
 9.14097018241882	28.3663402352473\\
% 9.38154573440552	27.6604357884162\\
% 9.46438498497009	27.6480467632753\\
% 9.67332692146301	26.351863014647\\
 9.86595768928528	25.9006495609637\\
% 10.0558857440948	25.5230497019909\\
% 10.3858830451965	25.3298770022809\\
% 10.4936358451843	25.2722186598426\\
 10.776785326004	25.1962161296421\\
% 10.8545956134796	25.2489110046016\\
% 11.1029264450073	25.2612732836228\\
% 11.2770318508148	25.1760243442401\\
 11.5893284797668	24.9436059587239\\
% 11.6759116172791	24.8717405299789\\
% 11.9269306182861	24.6380498881491\\
% 12.0506445884705	24.4619927097546\\
 12.3151247024536	23.6939385918595\\
% 12.492760848999	22.9443271258353\\
% 12.7263946056366	23.7137464984225\\
% 12.8795260906219	22.8289745477458\\
 13.1556639194489	22.0857166727729\\
% 13.3137912273407	20.9751046026078\\
% 13.6068145751953	19.8235190034996\\
% 13.665696811676	20.0451354022005\\
 13.9014417648315	19.0628919299153\\
% 14.1454388618469	17.8654631686208\\
% 14.2738837718964	17.5329579822624\\
% 14.5163492679596	16.5105073657823\\
 14.7443801879883	15.6648381326954\\
% 14.9911911010742	14.6454902042278\\
% 15.091393661499	14.4490736412031\\
% 15.3195831298828	13.7910113592429\\
 15.4853744029999	13.1874435105989\\
% 15.7133118629456	12.6175648257349\\
% 15.9334220409393	12.2551682819766\\
% 16.0791105747223	11.9942845378985\\
 16.266965341568	11.6952085337083\\
% 16.5023197650909	11.3917208175997\\
% 16.6838082790375	11.2604737381342\\
% 16.8699030399323	11.1759076819849\\
 17.1469637870789	11.0768756114862\\
% 17.2530440807343	11.0486317240722\\
% 17.459480714798	11.0006799015649\\
% 17.7267465114593	10.9600670648932\\
 17.870863866806	10.9577057177562\\
% 18.1831285476685	10.9294205901748\\
% 18.3009466648102	10.8742647825635\\
% 18.4912693023682	10.879239266726\\
 18.7064876079559	10.899125139863\\
% 18.9133874893188	10.8916566687274\\
% 19.2820596218109	10.9116088108415\\
% 19.3635892391205	10.9120954598231\\
 19.7252389907837	10.9399505275542\\
% 19.7252447128296	10.9399505275542\\
% 19.8903939247131	10.9280246004344\\
% 20.168097448349	10.9364862919541\\
 20.3190359592438	10.952257239979\\
% 20.4911682128906	10.9365131272145\\
% 20.657948923111	10.9289166637667\\
% 20.9995102405548	10.8659797607977\\
 21.2234882831573	10.8706842970793\\
% 21.315204334259	10.8593684313104\\
% 21.5460283279419	10.8535934303191\\
% 21.8071853637695	10.8201297574367\\
 21.9854676246643	10.8089351219906\\
% 22.1622938632965	10.8380684184019\\
% 22.2758011341095	10.8681290102424\\
% 22.5343856334686	10.8873500776525\\
 22.6996335506439	10.8493210415789\\
% 22.8794180870056	10.825365687572\\
% 23.0871574401855	10.8058294374285\\
% 23.3353487968445	10.7743626341035\\
 23.537798833847	10.7750042030627\\
% 23.6523868560791	10.7661555854868\\
% 23.8663136482239	10.7745067864086\\
% 24.1272313117981	10.755728650736\\
 24.3637666225433	10.757023588112\\
% 24.5188874721527	10.7785316101774\\
};
\addlegendentry{H(p,$\mathbb{W}$,$\hat{\phi}$)}

\end{axis}

\begin{axis}[%
width=0.0in,
height=0.0in,
at={(0in,0.0in)},
scale only axis,
xmin=0,
xmax=1,
ymin=0,
ymax=1,
axis line style={draw=none},
ticks=none,
axis x line*=bottom,
axis y line*=left
]
\end{axis}
\end{tikzpicture}%
%\centering
\vspace{-0.7em}
\caption{Locational optimization cost (green) as well as estimated locational optimization cost over time (rose), applying Algorithm~\ref{alg:onelayerlearningalg} for an initially unknown $\phi_{2}$.}
\label{fig:onelayerlearningcoveragecost}
\vspace{-0.7em}
\end{figure}

\begin{figure}[h!]
\begin{minipage}[t]{0.15\textwidth}
\centering
\includegraphics[trim={5.2cm 8.1cm 4.6cm 7.7cm},clip, width = 0.8\textwidth]{figures/config_7sec_one_layer_learning_2_220322_1215-compressed.pdf}
\centering
\end{minipage}
\hfill
\begin{minipage}[t]{0.15\textwidth}
\centering
\includegraphics[trim={5.2cm 8.1cm 4.6cm 7.7cm},clip, width = 0.8\textwidth]{figures/config_14sec_one_layer_learning_2_220322_1215-compressed.pdf}
\centering
\end{minipage}
\hfill
\begin{minipage}[t]{0.15\textwidth}
\centering
\includegraphics[trim={5.2cm 8.1cm 4.6cm 7.7cm},clip, width = 0.8\textwidth]{figures/config_25sec_one_layer_learning_2_220322_1215-compressed.pdf}
%\input{figures/config_25sec_one_layer_learning_2_220322.tex}
\centering
\end{minipage}
\caption{Configurations of cars at 2, 9, and 20 seconds for the application of Algorithm~\ref{alg:onelayerlearningalg} in the described set-up. The agents' location and their predicted trajectory are given in red, the Voronoi partitions in green, the artificial setpoints in blue, and the traveled paths are visualized in light grey.}
\label{pics:onelayerlearningconfig}
\vspace{-0.7em}
\end{figure}

\begin{figure} [h]
\centering
% This file was created by matlab2tikz.
%
%The latest updates can be retrieved from
%  http://www.mathworks.com/matlabcentral/fileexchange/22022-matlab2tikz-matlab2tikz
%where you can also make suggestions and rate matlab2tikz.
%
\definecolor{mycolor1}{rgb}{0.50000,0.75000,0.93000}%
%
\begin{tikzpicture}

\begin{axis}[%
height=0.95in,
width=0.40\textwidth,
yshift=0.8cm,
at={(0.0in,0.0in)},
scale only axis,
xmin=0,
xmax=24.5,
xlabel style={font=\color{white!15!black}},
xlabel style={font=\footnotesize},
xlabel style={yshift=0.6ex,},
xlabel={Time [sec]},
ymode=log,
ymin=0.001,
ymax=16,
yminorticks=true,
ylabel style={font=\color{white!15!black}},
ylabel style={font=\footnotesize},
ylabel style={xshift=0.6ex,},
ylabel={$\textup{Var}_{\textup{max}}$},
axis background/.style={fill=white},
xmajorgrids,
ymajorgrids,
tick label style={font=\footnotesize},
legend style={legend cell align=left, align=left, draw=white!15!black},
legend style={font=\footnotesize}
]
\addplot [color=mycolor1, line width=1.4pt]
  table[row sep=crcr]{%
 0.0626885414123537	15.3364197320176\\
% 0.523710918426514	14.3830696593064\\
% 0.72285361289978	14.1449330738488\\
% 0.939412784576416	14.0796771279175\\
 1.24876327514648	14.0396275133969\\
% 1.36225457191467	13.9869052450396\\
% 1.50689334869385	13.8956385552936\\
% 1.69302577972412	13.73756318325\\
 1.93132109642029	13.4704330447849\\
% 2.09447355270386	13.0784517469571\\
% 2.29062123298645	12.4879039312465\\
% 2.48974843025208	11.7792878710687\\
 2.74132223129272	10.9325085578703\\
% 2.89878602027893	9.63172380153771\\
% 3.09829182624817	9.19415736184645\\
% 3.29683418273926	8.18427623268525\\
 3.45868558883667	7.20328911437721\\
% 3.66376729011536	6.23326592164169\\
% 3.90756673812866	5.34470942184546\\
% 4.05353422164917	4.5788004285658\\
 4.27873892784119	3.97717370518687\\
% 4.50914044380188	3.43039986537487\\
% 4.65723652839661	2.99252562696411\\
% 4.95369954109192	2.58004508843916\\
 5.10325736999512	2.20957981691045\\
% 5.28414530754089	1.8909236002671\\
% 5.47541422843933	1.63667452651062\\
% 5.77117056846619	1.42160637979225\\
 5.9428297996521	1.23089090844007\\
% 6.15294117927551	1.09677606851152\\
% 6.3600353717804	0.974280580229805\\
% 6.48478074073792	0.859670301988254\\
 6.72261209487915	0.759430244722629\\
% 6.90272254943848	0.668832267140267\\
% 7.06064171791077	0.589156309427439\\
% 7.30467600822449	0.523833865926835\\
 7.47501225471497	0.466541268129628\\
% 7.67383284568787	0.416347919498192\\
% 7.86925287246704	0.374698151706717\\
% 8.19750161170959	0.336544599339761\\
 8.33077235221863	0.30397723756784\\
% 8.47818179130554	0.280931913604564\\
% 8.68704838752747	0.25534012001347\\
% 8.92462844848633	0.232440963393452\\
 9.14202637672424	0.213075693679231\\
% 9.38452429771423	0.195879920411903\\
% 9.48665351867676	0.180894219452174\\
% 9.67648310661316	0.167886286408851\\
 9.87281818389893	0.155768146367349\\
% 10.0724849224091	0.145163689619201\\
% 10.4021653652191	0.135639592996901\\
% 10.5343381881714	0.127157870224558\\
 10.7813014507294	0.119976696443548\\
% 10.8609740257263	0.113698016685015\\
% 11.1029171466827	0.108299047563994\\
% 11.2911762714386	0.103691338257287\\
 11.5894588947296	0.0998696370917303\\
% 11.7196168422699	0.0966313983705688\\
% 11.9270276546478	0.0945372152808792\\
% 12.0594660758972	0.0924330831369313\\
 12.3241943836212	0.0901527002297975\\
% 12.5232164382935	0.0880442695807286\\
% 12.7263862609863	0.0823713859739412\\
% 12.8838002204895	0.0802279218825608\\
 13.1556558132172	0.0814781063375793\\
% 13.3154165267944	0.0786904520074803\\
% 13.611102771759	0.0757376772202163\\
% 13.6771685600281	0.0759121098688563\\
 13.9040469646454	0.0719637018918585\\
% 14.1454307556152	0.06825509348735\\
% 14.2845708847046	0.0641123243844028\\
% 14.5166477680206	0.0606536549040652\\
 14.7482227802277	0.0571946363718322\\
% 14.9929327487946	0.0537858963851067\\
% 15.0913676738739	0.0499800376884479\\
% 15.3198346614838	0.046895519470269\\
 15.5018398284912	0.0410733134071872\\
% 15.7127189159393	0.0384013392700777\\
% 15.9373342514038	0.0359263620394156\\
% 16.0801534175873	0.0336240719090328\\
 16.267839384079	0.0313354498996826\\
% 16.5006865978241	0.0291932316234364\\
% 16.6912109375	0.027222513806584\\
% 16.8856727600098	0.0254106724554785\\
 17.1472770690918	0.0237941505784408\\
% 17.2657367706299	0.0223049140140121\\
% 17.4756175994873	0.0209914628111142\\
% 17.8385262012482	0.0198074356286632\\
 17.8758258342743	0.0182342674861824\\
% 18.1833455085754	0.0177403161292954\\
% 18.3013305187225	0.0167259095603031\\
% 18.4964096069336	0.0158547701773423\\
 18.7318005084991	0.0150539880304\\
% 18.9137486934662	0.014340777598575\\
% 19.2817496776581	0.013722274421712\\
% 19.3635525226593	0.0131831898422715\\
 19.7287744998932	0.0127060095642195\\
% 19.7288724899292	0.0122798967206908\\
% 19.8907618045807	0.0118936500564712\\
% 20.1699037075043	0.0115415390277942\\
 20.3223242282867	0.0112164770629054\\
% 20.5114111423492	0.0109131319308602\\
% 20.6597070217133	0.0106287668844236\\
% 21.0112277984619	0.0103245799127117\\
 21.228457403183	0.0100745079814872\\
% 21.3250350475311	0.00984062439931376\\
% 21.5461937904358	0.00962079029407252\\
% 21.8100671291351	0.00940968116593575\\
 21.9935154438019	0.00920669097824538\\
% 22.1636809825897	0.00901162600313575\\
% 22.2810172557831	0.00882304413363494\\
% 22.5351306915283	0.0086439161882483\\
 22.6996159076691	0.00847396480815522\\
% 22.8962873935699	0.00831397789369778\\
% 23.0881471157074	0.00816239189606354\\
% 23.3356120109558	0.00801977981552799\\
 23.5374648094177	0.00785457621333352\\
% 23.6681346416473	0.00772819657724645\\
% 23.8967995166779	0.00760791392300213\\
% 24.136847448349	0.00749323649311508\\
 24.3644620895386	0.00738775618633427\\
% 24.5188157081604	0.00729265593594685\\
% 24.7046813488007	0.00720314985311679\\
% 24.9304329872131	0.00711019464935537\\
 25.1028103351593	0.00701532098943582\\
% 25.3077227592468	0.00692552773812845\\
 25.6347496032715	0.00684780939987893\\
};
\addlegendentry{Maximal Variance}

\end{axis}

\begin{axis}[%
width=0.854in,
height=0.438in,
at={(0in,0in)},
scale only axis,
xmin=0,
xmax=1,
ymin=0,
ymax=1,
axis line style={draw=none},
ticks=none,
axis x line*=bottom,
axis y line*=left
]
\end{axis}
\end{tikzpicture}%
\vspace{-0.7em}
\caption{Maximal variance over time applying Algorithm~\ref{alg:onelayerlearningalg} for an initially unknown $\phi_{2}$.}
\label{fig:onelayerlearningmaxvar}
\vspace{-0.7em}
\end{figure}

\begin{figure} [h!]
\centering
% This file was created by matlab2tikz.
%
%The latest updates can be retrieved from
%  http://www.mathworks.com/matlabcentral/fileexchange/22022-matlab2tikz-matlab2tikz
%where you can also make suggestions and rate matlab2tikz.
%
\definecolor{mycolor1}{rgb}{0.00000,0.44700,0.74100}%
\definecolor{mycolor2}{rgb}{0.85000,0.32500,0.09800}%
\definecolor{mycolor3}{rgb}{0.92900,0.69400,0.12500}%
\definecolor{mycolor4}{rgb}{0.49400,0.18400,0.55600}%
\definecolor{mycolor5}{rgb}{0.46600,0.67400,0.18800}%
%
\begin{tikzpicture}

\begin{axis}[%
height=0.95in,
width=0.40\textwidth,
yshift=0.8cm,
at={(0.0in,0.0in)},
scale only axis,
xmin=0,
xmax=24.5,
xlabel style={font=\color{white!15!black}},
xlabel style={font=\footnotesize},
xlabel style={yshift=0.6ex,},
xlabel={Time [sec]},
ymin=-4,
ymax=14,
ylabel style={font=\color{white!15!black}},
ylabel={Mean value},
ylabel style={font=\footnotesize},
ylabel style={xshift = 0.6ex,},
axis background/.style={fill=white},
tick label style={font=\footnotesize},
title style={font=\bfseries},
title={},
xmajorgrids,
ymajorgrids,
legend style={legend cell align=left, align=left, draw=white!15!black},
legend style={font=\footnotesize}
]
\addplot [color=mycolor1, line width=1.4pt]
  table[row sep=crcr]{%
 -0.57880334854126	0\\
% -0.48006682395935	0\\
% -0.383595514297485	0\\
% -0.282954502105713	0\\
 -0.182975339889526	0\\
% -0.0815842628479002	0\\
% 0.01931209564209	0\\
% 0.118979644775391	-0.588060988129804\\
 0.21865553855896	-0.937085896245506\\
% 0.321214628219605	-1.15544927422846\\
% 0.428329658508301	-1.3192452898422\\
% 0.53482551574707	-1.4388617250188\\
 0.631938886642456	-1.52391168451607\\
% 0.739639472961426	-1.57976938928732\\
% 0.822497081756592	-1.62700130758913\\
% 0.94228048324585	-1.66172936419105\\
 1.24386186599731	-1.66863333309118\\
% 1.24391431808472	-1.64000055499832\\
% 1.24392242431641	-1.58563284330285\\
% 1.36236472129822	-1.50996833642137\\
 1.46429152488709	-1.43131973698507\\
% 1.53867239952087	-1.34134455550395\\
% 1.62087888717651	-1.2316451541825\\
% 1.72517223358154	-1.11832702071524\\
 1.8207799911499	-1.00275989298848\\
% 1.94416422843933	-0.876496456454788\\
% 2.02091021537781	-0.740074886869934\\
% 2.13180298805237	-0.614468201579029\\
 2.22401900291443	-0.491485107875405\\
% 2.32308192253113	-0.376086856571419\\
% 2.44293112754822	-0.265727482886859\\
% 2.52259345054626	-0.164896258199178\\
 2.62238688468933	-0.0737638840971613\\
% 2.74145216941834	0.00849929488504131\\
% 2.82201690673828	0.0836108632142896\\
% 2.93049092292786	0.148883295959422\\
 3.02306551933289	0.203341881579718\\
% 3.12266702651978	0.247669204468139\\
% 3.22329874038696	0.282996551355609\\
% 3.34136552810669	0.307270263247801\\
 3.44502396583557	0.325510409433718\\
% 3.5300621509552	0.335587901288932\\
% 3.62293739318848	0.338448880607643\\
% 3.72189350128174	0.335801404388008\\
 3.84191269874573	0.332221938604448\\
% 3.93421573638916	0.325022625394922\\
% 4.02698321342468	0.314574164304759\\
% 4.12289471626282	0.298158822502614\\
 4.23660249710083	0.280985419817966\\
% 4.33962697982788	0.263693197615453\\
% 4.44047636985779	0.24645528741064\\
% 4.56040711402893	0.225278296093393\\
 4.630579662323	0.201953165956525\\
% 4.72640986442566	0.177577421942203\\
% 4.87428517341614	0.155394921530444\\
% 5.00027008056641	0.132388293636666\\
 5.03321404457092	0.110411387449744\\
% 5.1221203327179	0.0886821146258967\\
% 5.24035425186157	0.0714428583427775\\
% 5.32195730209351	0.0532894759552391\\
 5.42484803199768	0.0346853859954308\\
% 5.53551692962647	0.0186319060149458\\
% 5.63034815788269	0.00812426518592702\\
% 5.82210655212402	-0.00186352829138059\\
 5.82214756011963	-0.0110925809389641\\
% 5.94401998519897	-0.0199968082164332\\
% 6.02550621032715	-0.0299298084792099\\
% 6.16116161346436	-0.0395873967884484\\
 6.2615348815918	-0.0480551838230099\\
% 6.36849851608276	-0.0574980962511518\\
% 6.4345142364502	-0.0654719166672066\\
% 6.55445404052734	-0.0743841838151695\\
 6.70095343589783	-0.0831459674929818\\
% 6.73470945358276	-0.0920812582687631\\
% 6.82313985824585	-0.098859144330703\\
% 6.92498059272766	-0.108075346746915\\
 7.05544490814209	-0.117230461024803\\
% 7.15404720306397	-0.125972272739631\\
% 7.24332089424133	-0.133386484924614\\
% 7.32086343765259	-0.1420475639502\\
 7.45592446327209	-0.150392622132017\\
% 7.54519577026367	-0.158907374990093\\
% 7.6316234588623	-0.165870250628657\\
% 7.73830981254578	-0.173502143022418\\
 7.85962791442871	-0.180541261411008\\
% 7.94003291130066	-0.186921684714335\\
% 8.05915970802307	-0.192723414809564\\
% 8.19914145469665	-0.198994128247904\\
 8.32294745445251	-0.20472911597308\\
% 8.32404775619507	-0.210246665547345\\
% 8.43282551765442	-0.215100653637279\\
% 8.53713102340698	-0.220336355128495\\
 8.6296338558197	-0.225177148665836\\
% 8.73589963912964	-0.229066810525921\\
% 8.84264702796936	-0.232763958731979\\
% 8.93937177658081	-0.236628465809062\\
 9.09919638633728	-0.240527490975197\\
% 9.15441460609436	-0.244963529152116\\
% 9.23950095176697	-0.24908028611074\\
% 9.38686461448669	-0.254791637097952\\
 9.44052762985229	-0.260379198683658\\
% 9.53487749099731	-0.266770911285285\\
% 9.64274806976318	-0.272907381592745\\
% 9.7318009853363	-0.279925248866249\\
 9.8260941028595	-0.28639470491224\\
% 9.96415657997131	-0.294223248587478\\
% 10.031244468689	-0.301564211358126\\
% 10.1295303821564	-0.309356701447292\\
 10.2355622768402	-0.316382648450144\\
% 10.4032809257507	-0.324289825235525\\
% 10.4906537055969	-0.331454708609023\\
% 10.5653883934021	-0.338660138409605\\
 10.6581940174103	-0.345151956943333\\
% 10.784938287735	-0.351889479219849\\
% 10.841081571579	-0.357846175778445\\
% 10.9676131725311	-0.364454854979275\\
 11.0286945819855	-0.370241138323284\\
% 11.1352338314056	-0.3758299750337\\
% 11.2198607444763	-0.380772517714908\\
% 11.328627538681	-0.38569536398177\\
 11.4415976524353	-0.389955229957744\\
% 11.6073987007141	-0.394291413980047\\
% 11.6468948841095	-0.398079839872485\\
% 11.7516786575317	-0.401764593300546\\
 11.8524360179901	-0.405034746413258\\
% 11.956632566452	-0.407958760747443\\
% 12.0420829772949	-0.410607914132549\\
% 12.1279501438141	-0.41259811540553\\
 12.2416917800903	-0.414422134635745\\
% 12.3331725120544	-0.415913620690806\\
% 12.4218036651611	-0.417325959887279\\
% 12.5414697647095	-0.418304935717744\\
 12.6532685279846	-0.419232843217376\\
% 12.7462152957916	-0.419751951896465\\
% 12.8523222923279	-0.420178974027853\\
% 12.9294535636902	-0.420587422210042\\
 13.0822853565216	-0.420879662151254\\
% 13.1939737319946	-0.421317871227997\\
% 13.2472683906555	-0.421708169151669\\
% 13.378995847702	-0.422346549660128\\
 13.5621871471405	-0.422892371180906\\
% 13.5621916770935	-0.42337098933981\\
% 13.6228856563568	-0.423806498781289\\
% 13.7575747489929	-0.424330863664334\\
 13.8234891414642	-0.424822829502073\\
% 13.9262966632843	-0.425301532714302\\
% 14.0689234256744	-0.425855039779192\\
% 14.1454946517944	-0.426391661927148\\
 14.2291230678558	-0.426976874864934\\
% 14.3729912757874	-0.427489248464429\\
% 14.4506949901581	-0.428066930232761\\
% 14.5584237098694	-0.428565325035812\\
 14.725235414505	-0.429190339169935\\
% 14.7252416133881	-0.429752655725832\\
% 14.8966662406921	-0.430367604604982\\
% 15.0779006004333	-0.430887364832213\\
 15.0799574375153	-0.431527345389426\\
% 15.1401779174805	-0.432242140420939\\
% 15.2216836929321	-0.433132589779841\\
% 15.356421661377	-0.433979606775388\\
 15.4242901325226	-0.434977756750278\\
% 15.5294272422791	-0.435714695457222\\
% 15.6678728580475	-0.436354856007\\
% 15.7614891052246	-0.436736680536264\\
 15.8384966373444	-0.437112597370337\\
% 15.9403559684753	-0.437704417562713\\
% 16.0358340263367	-0.438297617661437\\
% 16.1387001991272	-0.439315133843373\\
 16.2235087871552	-0.440415301447239\\
% 16.4228355407715	-0.441707153828993\\
% 16.4617208957672	-0.443018498140521\\
% 16.536874961853	-0.444151852666003\\
 16.631720495224	-0.44515120605125\\
% 16.7993404388428	-0.446237276910459\\
% 16.8255044937134	-0.447215655046868\\
% 16.9287883758545	-0.448361998803271\\
% 17.0428664207458	-0.449517664824342\\
 17.1474303722382	-0.450962509120671\\
% 17.2291614532471	-0.452440195779744\\
% 17.3465222835541	-0.453823800083381\\
% 17.4238087654114	-0.455075817321777\\
 17.5406684398651	-0.456161715042601\\
% 17.6217422008514	-0.457158591982832\\
% 17.8375637054443	-0.458251501191267\\
% 17.837580871582	-0.459181510557677\\
 17.9298414707184	-0.460304670897845\\
% 18.0336350917816	-0.461279677805067\\
% 18.2846984386444	-0.46255172631054\\
% 18.2847022533417	-0.463796185033225\\
 18.3382050514221	-0.465481739577378\\
% 18.4275156974792	-0.467122783239201\\
% 18.5614318370819	-0.468878288874034\\
% 18.6368317127228	-0.470473416079731\\
 18.7376241207123	-0.471927682077112\\
% 18.8722116470337	-0.473235334293076\\
% 18.9676610946655	-0.474505285875502\\
% 19.2566427707672	-0.47572846804194\\
 19.2566499233246	-0.476967139987224\\
% 19.2566549301147	-0.47820401743946\\
% 19.5335189819336	-0.479365499185949\\
% 19.5335454463959	-0.4804260187592\\
 19.5335821628571	-0.481204062656943\\
% 19.8561558246613	-0.481910242254369\\
% 19.8561932563782	-0.482671378551334\\
% 19.856215429306	-0.483449458495791\\
 19.939593744278	-0.484284964642768\\
% 20.1148948192596	-0.48508871673981\\
% 20.2086979866028	-0.485773406926853\\
% 20.2411860942841	-0.486412975344543\\
 20.3258230209351	-0.486967536546992\\
% 20.4300326824188	-0.487415295655325\\
% 20.5301665782928	-0.487875198014692\\
% 20.6302778244019	-0.488340552663743\\
 20.7319926738739	-0.488944714634144\\
% 20.8867396831512	-0.489568211934845\\
% 21.0112361431122	-0.490207073670049\\
% 21.212642621994	-0.490874908752517\\
 21.212647151947	-0.491429714528884\\
% 21.2460777282715	-0.491946259468797\\
% 21.3257083415985	-0.492337543591034\\
% 21.4222948074341	-0.492619663867075\\
 21.6202604293823	-0.492858497679883\\
% 21.6202783107758	-0.493060088219131\\
% 21.8131494045258	-0.493196517880939\\
% 21.9340204715729	-0.493421003670079\\
 21.9340269088745	-0.493707299811412\\
% 22.0973948955536	-0.494023299816874\\
% 22.1650163650513	-0.494434063860716\\
% 22.2429306030273	-0.494840015525554\\
 22.32282538414	-0.495224826810336\\
% 22.4374601364136	-0.49562429778674\\
% 22.554838848114	-0.495962904783951\\
% 22.6611167907715	-0.496263178055948\\
 22.7834405422211	-0.496512434480444\\
% 22.8527333259583	-0.496744764171943\\
% 22.9370495796204	-0.497051353540428\\
% 23.0243656158447	-0.4973971479426\\
 23.132009935379	-0.497785057701556\\
% 23.2255770683289	-0.498175950543157\\
% 23.3404690742493	-0.498602313181898\\
% 23.4402012348175	-0.498951466852333\\
 23.5462953567505	-0.499205274896298\\
% 23.627540063858	-0.499425035141684\\
% 23.7291373729706	-0.499573359137521\\
% 23.8266167163849	-0.499678542340611\\
 23.9292280197144	-0.499800035878941\\
% 24.0224551677704	-0.499940759357514\\
% 24.1974343776703	-0.500153783326226\\
% 24.3536417007446	-0.50035003933931\\
 24.3537036895752	-0.5004447947138\\
% 24.4515547275543	-0.500509582907384\\
% 24.5563561439514	-0.500486798720374\\
% 24.6695627689362	-0.500453255932726\\
 24.7547094345093	-0.500480839142057\\
% 24.8209278106689	-0.500462514074655\\
% 24.9442512512207	-0.500506023446945\\
% 25.0268671035767	-0.500577496070321\\
 25.1480109214783	-0.500760359179225\\
% 25.2522289276123	-0.500963390272972\\
% 25.3239643096924	-0.50109731518517\\
% 25.6276008605957	-0.501219519130593\\
 25.6276306629181	-0.501226932261886\\
% 25.6276738166809	-0.501199938947195\\
};
\addlegendentry{$\theta_{x_{p}^{2}}$}

\addplot [color=mycolor2, line width=1.4pt]
  table[row sep=crcr]{%
 -0.57880334854126	0\\
% -0.48006682395935	0\\
% -0.383595514297485	0\\
% -0.282954502105713	0\\
 -0.182975339889526	0\\
% -0.0815842628479002	0\\
% 0.01931209564209	0\\
% 0.118979644775391	-0.586831998094254\\
 0.21865553855896	-0.927302042606996\\
% 0.321214628219605	-1.13083693067699\\
% 0.428329658508301	-1.27228337156717\\
% 0.53482551574707	-1.40135021803519\\
 0.631938886642456	-1.5080617085747\\
% 0.739639472961426	-1.5749032695166\\
% 0.822497081756592	-1.61674263183119\\
% 0.94228048324585	-1.63997352201022\\
 1.24386186599731	-1.64168565272769\\
% 1.24391431808472	-1.60073506578522\\
% 1.24392242431641	-1.53825552429578\\
% 1.36236472129822	-1.45715468960964\\
 1.46429152488709	-1.36341336605017\\
% 1.53867239952087	-1.24891165678082\\
% 1.62087888717651	-1.11224853822523\\
% 1.72517223358154	-0.970401007141731\\
 1.8207799911499	-0.827482417587817\\
% 1.94416422843933	-0.671716956139562\\
% 2.02091021537781	-0.503926855525606\\
% 2.13180298805237	-0.343174806181878\\
 2.22401900291443	-0.181578173803018\\
% 2.32308192253113	-0.0266665849674155\\
% 2.44293112754822	0.118485653040807\\
% 2.52259345054626	0.251452160591271\\
 2.62238688468933	0.375419280732558\\
% 2.74145216941834	0.486361365093444\\
% 2.82201690673828	0.586952070246298\\
% 2.93049092292786	0.680329121038994\\
 3.02306551933289	0.763574187236259\\
% 3.12266702651978	0.832236908820164\\
% 3.22329874038696	0.891961994259759\\
% 3.34136552810669	0.936028423412722\\
 3.44502396583557	0.968498184876921\\
% 3.5300621509552	0.987424628889812\\
% 3.62293739318848	0.996844182219206\\
% 3.72189350128174	0.994087310852365\\
 3.84191269874573	0.986867928651236\\
% 3.93421573638916	0.972414081535248\\
% 4.02698321342468	0.953363600135731\\
% 4.12289471626282	0.92842962970542\\
 4.23660249710083	0.900908722952749\\
% 4.33962697982788	0.878091346026224\\
% 4.44047636985779	0.852724191805464\\
% 4.56040711402893	0.82531089160301\\
 4.630579662323	0.795491925363535\\
% 4.72640986442566	0.763436954399253\\
% 4.87428517341614	0.730551318860648\\
% 5.00027008056641	0.693700075199331\\
 5.03321404457092	0.655819142371968\\
% 5.1221203327179	0.618258181315468\\
% 5.24035425186157	0.584565609127253\\
% 5.32195730209351	0.550572749765252\\
 5.42484803199768	0.516477551067567\\
% 5.53551692962647	0.480678229477775\\
% 5.63034815788269	0.445319439516453\\
% 5.82210655212402	0.414678597577591\\
 5.82214756011963	0.388702905284219\\
% 5.94401998519897	0.362855470039904\\
% 6.02550621032715	0.337027860251936\\
% 6.16116161346436	0.309517981606689\\
 6.2615348815918	0.281437600552238\\
% 6.36849851608276	0.252210743761083\\
% 6.4345142364502	0.22363816556097\\
% 6.55445404052734	0.195475061799527\\
 6.70095343589783	0.168927579078954\\
% 6.73470945358276	0.142314688277423\\
% 6.82313985824585	0.115990412786459\\
% 6.92498059272766	0.0924656402333\\
 7.05544490814209	0.0723339657632351\\
% 7.15404720306397	0.0526418516636298\\
% 7.24332089424133	0.0345152502543797\\
% 7.32086343765259	0.0172354418017093\\
 7.45592446327209	0.00231791181819574\\
% 7.54519577026367	-0.0134498630303526\\
% 7.6316234588623	-0.029011913371022\\
% 7.73830981254578	-0.0441185003264941\\
 7.85962791442871	-0.0583247522569259\\
% 7.94003291130066	-0.0718295354546754\\
% 8.05915970802307	-0.0832055484635674\\
% 8.19914145469665	-0.0955324558786099\\
 8.32294745445251	-0.107159857258322\\
% 8.32404775619507	-0.119123485006385\\
% 8.43282551765442	-0.130230651840988\\
% 8.53713102340698	-0.141824213875929\\
 8.6296338558197	-0.152343799516729\\
% 8.73589963912964	-0.16293926914228\\
% 8.84264702796936	-0.171943903198744\\
% 8.93937177658081	-0.181538567056904\\
 9.09919638633728	-0.190024932966933\\
% 9.15441460609436	-0.197781142318234\\
% 9.23950095176697	-0.204731465082643\\
% 9.38686461448669	-0.210312924987631\\
 9.44052762985229	-0.21535760175553\\
% 9.53487749099731	-0.218941407817601\\
% 9.64274806976318	-0.222141161625609\\
% 9.7318009853363	-0.224418070945941\\
 9.8260941028595	-0.226493538156609\\
% 9.96415657997131	-0.227360130581328\\
% 10.031244468689	-0.228071454970051\\
% 10.1295303821564	-0.228074843723858\\
 10.2355622768402	-0.228112908203087\\
% 10.4032809257507	-0.227397181890764\\
% 10.4906537055969	-0.226700954900435\\
% 10.5653883934021	-0.225696039193028\\
 10.6581940174103	-0.224730528399098\\
% 10.784938287735	-0.223326436242449\\
% 10.841081571579	-0.222053681715015\\
% 10.9676131725311	-0.220225911767197\\
 11.0286945819855	-0.218623072193362\\
% 11.1352338314056	-0.216969889552921\\
% 11.2198607444763	-0.215476581743808\\
% 11.328627538681	-0.213896208923253\\
 11.4415976524353	-0.212522407366409\\
% 11.6073987007141	-0.211117514668828\\
% 11.6468948841095	-0.209968613132355\\
% 11.7516786575317	-0.208973012078273\\
 11.8524360179901	-0.208220969434661\\
% 11.956632566452	-0.207777875379456\\
% 12.0420829772949	-0.207481857386949\\
% 12.1279501438141	-0.207643569023919\\
 12.2416917800903	-0.207906230081688\\
% 12.3331725120544	-0.208540311601961\\
% 12.4218036651611	-0.209241085918991\\
% 12.5414697647095	-0.210296501330713\\
 12.6532685279846	-0.21136837741796\\
% 12.7462152957916	-0.213107210443217\\
% 12.8523222923279	-0.214836898462234\\
% 12.9294535636902	-0.217114584709833\\
 13.0822853565216	-0.21935829401481\\
% 13.1939737319946	-0.222211763522525\\
% 13.2472683906555	-0.224983073016658\\
% 13.378995847702	-0.22863434874739\\
 13.5621871471405	-0.232161373822457\\
% 13.5621916770935	-0.236268343093549\\
% 13.6228856563568	-0.24019249544715\\
% 13.7575747489929	-0.244721905145067\\
 13.8234891414642	-0.248985754497051\\
% 13.9262966632843	-0.254035623125509\\
% 14.0689234256744	-0.258774870607327\\
% 14.1454946517944	-0.264298938353308\\
 14.2291230678558	-0.269429524392063\\
% 14.3729912757874	-0.275393582298911\\
% 14.4506949901581	-0.280876513152787\\
% 14.5584237098694	-0.28677539531429\\
 14.725235414505	-0.292098406126531\\
% 14.7252416133881	-0.298077062206008\\
% 14.8966662406921	-0.303471139749931\\
% 15.0779006004333	-0.309624955138162\\
 15.0799574375153	-0.315140250018704\\
% 15.1401779174805	-0.320945641349358\\
% 15.2216836929321	-0.325981650145071\\
% 15.356421661377	-0.331483902937748\\
 15.4242901325226	-0.33628049235902\\
% 15.5294272422791	-0.34193216228341\\
% 15.6678728580475	-0.347000845434785\\
% 15.7614891052246	-0.352514204651499\\
 15.8384966373444	-0.357328469683271\\
% 15.9403559684753	-0.362115374926276\\
% 16.0358340263367	-0.366200403311964\\
% 16.1387001991272	-0.371329667220447\\
 16.2235087871552	-0.375705701884126\\
% 16.4228355407715	-0.380336240857336\\
% 16.4617208957672	-0.384320126609669\\
% 16.536874961853	-0.388664619023896\\
 16.631720495224	-0.3924519887812\\
% 16.7993404388428	-0.396899997507003\\
% 16.8255044937134	-0.400864553065073\\
% 16.9287883758545	-0.404996691954057\\
 17.0428664207458	-0.408620655064205\\
% 17.1474303722382	-0.412226606058443\\
% 17.2291614532471	-0.415432450550622\\
% 17.3465222835541	-0.418842194501028\\
 17.4238087654114	-0.42189140478947\\
% 17.5406684398651	-0.42497103977098\\
% 17.6217422008514	-0.427735200575491\\
% 17.8375637054443	-0.430428017250144\\
 17.837580871582	-0.432782102009806\\
% 17.9298414707184	-0.435258625940257\\
% 18.0336350917816	-0.437386498514016\\
% 18.2846984386444	-0.439452838476459\\
 18.2847022533417	-0.441169213179997\\
% 18.3382050514221	-0.443085866060219\\
% 18.4275156974792	-0.44481807328529\\
% 18.5614318370819	-0.446838896725907\\
 18.6368317127228	-0.448799779242194\\
% 18.7376241207123	-0.450990609305705\\
% 18.8722116470337	-0.453280617606676\\
% 18.9676610946655	-0.455485704077162\\
 19.2566427707672	-0.457590193535406\\
% 19.2566499233246	-0.459241863876173\\
% 19.2566549301147	-0.46087579192081\\
% 19.5335189819336	-0.462408864228109\\
 19.5335454463959	-0.463938910687951\\
% 19.5335821628571	-0.46537332809293\\
% 19.8561558246613	-0.466838952969756\\
% 19.8561932563782	-0.468159304669737\\
 19.856215429306	-0.469395800342205\\
% 19.939593744278	-0.470410598664625\\
% 20.1148948192596	-0.471413078837319\\
% 20.2086979866028	-0.472441189002783\\
 20.2411860942841	-0.473441738212006\\
% 20.3258230209351	-0.474436443272417\\
% 20.4300326824188	-0.475445281750829\\
% 20.5301665782928	-0.476403201387356\\
 20.6302778244019	-0.477317014949271\\
% 20.7319926738739	-0.478123537992875\\
% 20.8867396831512	-0.478890429491763\\
% 21.0112361431122	-0.479629222995801\\
 21.212642621994	-0.480298071137348\\
% 21.212647151947	-0.481018975412034\\
% 21.2460777282715	-0.481734925458049\\
% 21.3257083415985	-0.482494491010666\\
 21.4222948074341	-0.483247276170673\\
% 21.6202604293823	-0.483978161790347\\
% 21.6202783107758	-0.48469194326838\\
% 21.8131494045258	-0.485400589961532\\
 21.9340204715729	-0.486077621995658\\
% 21.9340269088745	-0.486677366138412\\
% 22.0973948955536	-0.487212638806721\\
% 22.1650163650513	-0.487717487045966\\
 22.2429306030273	-0.488189567971482\\
% 22.32282538414	-0.488711778639676\\
% 22.4374601364136	-0.489229492228265\\
% 22.554838848114	-0.489828882995031\\
 22.6611167907715	-0.490435357254071\\
% 22.7834405422211	-0.491032791078553\\
% 22.8527333259583	-0.49165506260972\\
% 22.9370495796204	-0.492219892050571\\
 23.0243656158447	-0.492743928678184\\
% 23.132009935379	-0.493182957080048\\
% 23.2255770683289	-0.493644078102939\\
% 23.3404690742493	-0.494077222654123\\
 23.4402012348175	-0.494526435930499\\
% 23.5462953567505	-0.494993965488721\\
% 23.627540063858	-0.495486987712681\\
% 23.7291373729706	-0.496016041025239\\
 23.8266167163849	-0.496561923605128\\
% 23.9292280197144	-0.497074048880119\\
% 24.0224551677704	-0.497615164380559\\
% 24.1974343776703	-0.498088576460209\\
 24.3536417007446	-0.498570342608765\\
% 24.3537036895752	-0.499008368111003\\
% 24.4515547275543	-0.499435666254566\\
% 24.5563561439514	-0.499768443603504\\
 24.6695627689362	-0.500062931872481\\
% 24.7547094345093	-0.500342134034589\\
% 24.8209278106689	-0.500567145089157\\
% 24.9442512512207	-0.500900627217341\\
 25.0268671035767	-0.501319414194587\\
% 25.1480109214783	-0.501850900045902\\
% 25.2522289276123	-0.502378347243997\\
% 25.3239643096924	-0.502715320645862\\
 25.6276008605957	-0.502963106193121\\
% 25.6276306629181	-0.503105311980523\\
 25.6276738166809	-0.503229896086264\\
};
\addlegendentry{$\theta_{y_{p}^{2}}$}

\addplot [color=mycolor3, line width=1.4pt]
  table[row sep=crcr]{%
 -0.57880334854126	0\\
% -0.48006682395935	0\\
% -0.383595514297485	0\\
% -0.282954502105713	0\\
 -0.182975339889526	0\\
% -0.0815842628479002	0\\
% 0.01931209564209	0\\
% 0.118979644775391	-1.5955978619661\\
 0.21865553855896	-2.17932604788149\\
% 0.321214628219605	-2.49204984116619\\
% 0.428329658508301	-2.65276870636853\\
% 0.53482551574707	-2.70675249640158\\
 0.631938886642456	-2.71442778057133\\
% 0.739639472961426	-2.71253544536376\\
% 0.822497081756592	-2.70018303482607\\
% 0.94228048324585	-2.666187275526\\
 1.24386186599731	-2.5963946958509\\
% 1.24391431808472	-2.48085320159453\\
% 1.24392242431641	-2.33653087582229\\
% 1.36236472129822	-2.16008001348041\\
 1.46429152488709	-1.98664685011136\\
% 1.53867239952087	-1.79984809468328\\
% 1.62087888717651	-1.58600022229325\\
% 1.72517223358154	-1.36590250841982\\
 1.8207799911499	-1.14526063912717\\
% 1.94416422843933	-0.906510797373812\\
% 2.02091021537781	-0.653117132576426\\
% 2.13180298805237	-0.420184521214651\\
 2.22401900291443	-0.194512094567926\\
% 2.32308192253113	0.0185862630100928\\
% 2.44293112754822	0.22286250536672\\
% 2.52259345054626	0.410138948113399\\
 2.62238688468933	0.577415395935986\\
% 2.74145216941834	0.728468230332965\\
% 2.82201690673828	0.866053025055862\\
% 2.93049092292786	0.98599938977577\\
 3.02306551933289	1.08545266458668\\
% 3.12266702651978	1.16611263572668\\
% 3.22329874038696	1.22939610365802\\
% 3.34136552810669	1.27239692017565\\
 3.44502396583557	1.30438739683859\\
% 3.5300621509552	1.3218533824288\\
% 3.62293739318848	1.32571816680138\\
% 3.72189350128174	1.31930289042475\\
 3.84191269874573	1.31109412738397\\
% 3.93421573638916	1.29639806023738\\
% 4.02698321342468	1.27484106291286\\
% 4.12289471626282	1.24099422556083\\
 4.23660249710083	1.20567652852128\\
% 4.33962697982788	1.16943696466137\\
% 4.44047636985779	1.13333852385676\\
% 4.56040711402893	1.08942623924418\\
 4.630579662323	1.04162763117756\\
% 4.72640986442566	0.991757930105109\\
% 4.87428517341614	0.945889319739535\\
% 5.00027008056641	0.898967209023567\\
 5.03321404457092	0.853744677978966\\
% 5.1221203327179	0.808866659053706\\
% 5.24035425186157	0.772396862138748\\
% 5.32195730209351	0.73377387924063\\
 5.42484803199768	0.694411146641983\\
% 5.53551692962647	0.658836361569342\\
% 5.63034815788269	0.633123448706215\\
% 5.82210655212402	0.609071181186778\\
 5.82214756011963	0.58706793796182\\
% 5.94401998519897	0.566014389492921\\
% 6.02550621032715	0.543246920032175\\
% 6.16116161346436	0.521444092503998\\
 6.2615348815918	0.50150024002744\\
% 6.36849851608276	0.479662997320304\\
% 6.4345142364502	0.460314538947102\\
% 6.55445404052734	0.439208785826736\\
 6.70095343589783	0.418643548019986\\
% 6.73470945358276	0.397346581578006\\
% 6.82313985824585	0.379868859157796\\
% 6.92498059272766	0.358605478222614\\
 7.05544490814209	0.338043944468552\\
% 7.15404720306397	0.31764585204121\\
% 7.24332089424133	0.299668669664243\\
% 7.32086343765259	0.279786779224139\\
 7.45592446327209	0.260985180562329\\
% 7.54519577026367	0.242585325878196\\
% 7.6316234588623	0.226956400034112\\
% 7.73830981254578	0.210194489207254\\
 7.85962791442871	0.194574485844782\\
% 7.94003291130066	0.179645375824691\\
% 8.05915970802307	0.165876831711842\\
% 8.19914145469665	0.151491698934507\\
 8.32294745445251	0.138243045847105\\
% 8.32404775619507	0.125859155136197\\
% 8.43282551765442	0.114856275664266\\
% 8.53713102340698	0.103089267933655\\
 8.6296338558197	0.0921432783075318\\
% 8.73589963912964	0.0821740151892527\\
% 8.84264702796936	0.0726987915870438\\
% 8.93937177658081	0.0631346595689308\\
 9.09919638633728	0.0539509635081856\\
% 9.15441460609436	0.0445311029269533\\
% 9.23950095176697	0.0358711182988429\\
% 9.38686461448669	0.0246518818944423\\
 9.44052762985229	0.0140158576214162\\
% 9.53487749099731	0.00215644734798559\\
% 9.64274806976318	-0.00896496242918943\\
% 9.7318009853363	-0.0213709831283921\\
 9.8260941028595	-0.0326925656238188\\
% 9.96415657997131	-0.0459891834866113\\
% 10.031244468689	-0.0582337834348152\\
% 10.1295303821564	-0.0711109404734884\\
 10.2355622768402	-0.0826719347562772\\
% 10.4032809257507	-0.0955430924513223\\
% 10.4906537055969	-0.107211032747117\\
% 10.5653883934021	-0.118890295225256\\
 10.6581940174103	-0.129424406931307\\
% 10.784938287735	-0.14022930426961\\
% 10.841081571579	-0.149842698374783\\
% 10.9676131725311	-0.160356118197804\\
 11.0286945819855	-0.169647559164005\\
% 11.1352338314056	-0.178667413703742\\
% 11.2198607444763	-0.186696867375169\\
% 11.328627538681	-0.194624285389622\\
 11.4415976524353	-0.201597496190999\\
% 11.6073987007141	-0.208661327744728\\
% 11.6468948841095	-0.214967705572285\\
% 11.7516786575317	-0.221192470522205\\
 11.8524360179901	-0.226863129304725\\
% 11.956632566452	-0.232222361188487\\
% 12.0420829772949	-0.237186341299786\\
% 12.1279501438141	-0.241465285400601\\
 12.2416917800903	-0.245497263837414\\
% 12.3331725120544	-0.249203127979136\\
% 12.4218036651611	-0.252759500878046\\
% 12.5414697647095	-0.255808812959202\\
 12.6532685279846	-0.25860309915571\\
% 12.7462152957916	-0.260888197842601\\
% 12.8523222923279	-0.26295035679356\\
% 12.9294535636902	-0.26511188337026\\
 13.0822853565216	-0.26695004039976\\
% 13.1939737319946	-0.269168137719738\\
% 13.2472683906555	-0.271220930673877\\
% 13.378995847702	-0.273762760426994\\
 13.5621871471405	-0.27602401760489\\
% 13.5621916770935	-0.278350647944816\\
% 13.6228856563568	-0.280490993858102\\
% 13.7575747489929	-0.282806085293316\\
 13.8234891414642	-0.284930321740992\\
% 13.9262966632843	-0.287187597369707\\
% 14.0689234256744	-0.289381941186143\\
% 14.1454946517944	-0.291679274089828\\
 14.2291230678558	-0.293865202894146\\
% 14.3729912757874	-0.296133678386289\\
% 14.4506949901581	-0.29830795337779\\
% 14.5584237098694	-0.300438789908078\\
 14.725235414505	-0.302539368026496\\
% 14.7252416133881	-0.304717199733552\\
% 14.8966662406921	-0.306780513630156\\
% 15.0779006004333	-0.308844606578294\\
 15.0799574375153	-0.310900221821237\\
% 15.1401779174805	-0.313119551951204\\
% 15.2216836929321	-0.315350409791348\\
% 15.356421661377	-0.317599678866543\\
 15.4242901325226	-0.319879732833423\\
% 15.5294272422791	-0.321970597273801\\
% 15.6678728580475	-0.323785786749244\\
% 15.7614891052246	-0.325311621550165\\
 15.8384966373444	-0.326653497606359\\
% 15.9403559684753	-0.328219481863386\\
% 16.0358340263367	-0.329620411434299\\
% 16.1387001991272	-0.331688272944292\\
 16.2235087871552	-0.333703464939376\\
% 16.4228355407715	-0.335974607617139\\
% 16.4617208957672	-0.338118411998386\\
% 16.536874961853	-0.34003667493803\\
 16.631720495224	-0.341669023002356\\
% 16.7993404388428	-0.343511078255496\\
% 16.8255044937134	-0.345124767939744\\
% 16.9287883758545	-0.346956682163782\\
 17.0428664207458	-0.348716601492875\\
% 17.1474303722382	-0.35084003181404\\
% 17.2291614532471	-0.352941280367285\\
% 17.3465222835541	-0.3549386917125\\
 17.4238087654114	-0.356714872032358\\
% 17.5406684398651	-0.358285901931223\\
% 17.6217422008514	-0.359681544539265\\
% 17.8375637054443	-0.361143986315821\\
 17.837580871582	-0.362337634312411\\
% 17.9298414707184	-0.36372058286969\\
% 18.0336350917816	-0.364872529298563\\
% 18.2846984386444	-0.366333408685074\\
 18.2847022533417	-0.367717614828595\\
% 18.3382050514221	-0.369568893637125\\
% 18.4275156974792	-0.371359829715522\\
% 18.5614318370819	-0.373275982115075\\
 18.6368317127228	-0.375012053575445\\
% 18.7376241207123	-0.376609221299621\\
% 18.8722116470337	-0.378080011334799\\
% 18.9676610946655	-0.379529024290017\\
 19.2566427707672	-0.380922860360811\\
% 19.2566499233246	-0.382353798879349\\
% 19.2566549301147	-0.38381474633685\\
% 19.5335189819336	-0.385189761695767\\
 19.5335454463959	-0.386447971642716\\
% 19.5335821628571	-0.387384959201507\\
% 19.8561558246613	-0.388253544056036\\
% 19.8561932563782	-0.389192713700898\\
 19.856215429306	-0.390158359986924\\
% 19.939593744278	-0.391186132888609\\
% 20.1148948192596	-0.392176440298987\\
% 20.2086979866028	-0.393023200252599\\
 20.2411860942841	-0.393818436969426\\
% 20.3258230209351	-0.394517099153448\\
% 20.4300326824188	-0.39510148660263\\
% 20.5301665782928	-0.395699960320433\\
 20.6302778244019	-0.396302174249766\\
% 20.7319926738739	-0.39705592400637\\
% 20.8867396831512	-0.397833329552156\\
% 21.0112361431122	-0.398621104119971\\
 21.212642621994	-0.399448938391722\\
% 21.212647151947	-0.40015023907969\\
% 21.2460777282715	-0.400810227225932\\
% 21.3257083415985	-0.401326938024333\\
 21.4222948074341	-0.401745579479707\\
% 21.6202604293823	-0.40212296079391\\
% 21.6202783107758	-0.402459125916018\\
% 21.8131494045258	-0.402731400011364\\
 21.9340204715729	-0.403082076683249\\
% 21.9340269088745	-0.403519480605688\\
% 22.0973948955536	-0.404002084620976\\
% 22.1650163650513	-0.404565272534366\\
 22.2429306030273	-0.40513510412903\\
% 22.32282538414	-0.405650350702263\\
% 22.4374601364136	-0.40616775970937\\
% 22.554838848114	-0.406591594043475\\
 22.6611167907715	-0.406968639212669\\
% 22.7834405422211	-0.407299478286463\\
% 22.8527333259583	-0.407597884413971\\
% 22.9370495796204	-0.407975784899287\\
 23.0243656158447	-0.40839487898792\\
% 23.132009935379	-0.408887037600372\\
% 23.2255770683289	-0.409371412848519\\
% 23.3404690742493	-0.409888404299426\\
 23.4402012348175	-0.410330920192558\\
% 23.5462953567505	-0.410689701553437\\
% 23.627540063858	-0.41101100717101\\
% 23.7291373729706	-0.411267445190365\\
 23.8266167163849	-0.411509214480533\\
% 23.9292280197144	-0.411811238538943\\
% 24.0224551677704	-0.412170430491633\\
% 24.1974343776703	-0.412607991740765\\
 24.3536417007446	-0.413092371909824\\
% 24.3537036895752	-0.413436528087689\\
% 24.4515547275543	-0.41371668184055\\
% 24.5563561439514	-0.41379126047014\\
 24.6695627689362	-0.413817835131859\\
% 24.7547094345093	-0.413905796575439\\
% 24.8209278106689	-0.413915754065157\\
% 24.9442512512207	-0.414029839235131\\
 25.0268671035767	-0.4141975790868\\
% 25.1480109214783	-0.414572472845183\\
% 25.2522289276123	-0.414994725156786\\
% 25.3239643096924	-0.415339452382075\\
 25.6276008605957	-0.415668912471881\\
% 25.6276306629181	-0.415805343977434\\
 25.6276738166809	-0.415890636367823\\
};
\addlegendentry{$\theta_{x_{p}}$}

\addplot [color=mycolor4, line width=1.4pt]
  table[row sep=crcr]{%
 -0.57880334854126	0\\
% -0.48006682395935	0\\
% -0.383595514297485	0\\
% -0.282954502105713	0\\
 -0.182975339889526	0\\
% -0.0815842628479002	0\\
% 0.01931209564209	0\\
% 0.118979644775391	-1.70463620259392\\
 0.21865553855896	-2.3366844387233\\
% 0.321214628219605	-2.64808425534926\\
% 0.428329658508301	-2.80068012938898\\
% 0.53482551574707	-2.94773393214382\\
 0.631938886642456	-3.06043256716714\\
% 0.739639472961426	-3.11333827368094\\
% 0.822497081756592	-3.11107021482917\\
% 0.94228048324585	-3.07250379980587\\
 1.24386186599731	-3.00552371059939\\
% 1.24391431808472	-2.86360435387678\\
% 1.24392242431641	-2.69465183256898\\
% 1.36236472129822	-2.49503269604406\\
 1.46429152488709	-2.27410963750185\\
% 1.53867239952087	-2.01411255034145\\
% 1.62087888717651	-1.71739654943667\\
% 1.72517223358154	-1.41086609273088\\
 1.8207799911499	-1.10656089977647\\
% 1.94416422843933	-0.777200315274513\\
% 2.02091021537781	-0.427658952010006\\
% 2.13180298805237	-0.0929197564314563\\
 2.22401900291443	0.240078044334496\\
% 2.32308192253113	0.559308917312137\\
% 2.44293112754822	0.858107887081587\\
% 2.52259345054626	1.13122723106244\\
 2.62238688468933	1.38259311640422\\
% 2.74145216941834	1.60674007405623\\
% 2.82201690673828	1.80879966697148\\
% 2.93049092292786	1.99516760150618\\
 3.02306551933289	2.15892918584314\\
% 3.12266702651978	2.29257947194856\\
% 3.22329874038696	2.40653303058912\\
% 3.34136552810669	2.48857737490857\\
 3.44502396583557	2.54661514010945\\
% 3.5300621509552	2.57788422161593\\
% 3.62293739318848	2.58935602931774\\
% 3.72189350128174	2.57697461200974\\
 3.84191269874573	2.55561044079627\\
% 3.93421573638916	2.52131788482461\\
% 4.02698321342468	2.47724432434325\\
% 4.12289471626282	2.4204800692637\\
 4.23660249710083	2.35869968405405\\
% 4.33962697982788	2.30566097257724\\
% 4.44047636985779	2.24781432017619\\
% 4.56040711402893	2.18516099610304\\
 4.630579662323	2.11745895747072\\
% 4.72640986442566	2.04536080876642\\
% 4.87428517341614	1.97203598810415\\
% 5.00027008056641	1.89205413453737\\
 5.03321404457092	1.81029441503483\\
% 5.1221203327179	1.7294957619921\\
% 5.24035425186157	1.65724872451597\\
% 5.32195730209351	1.58408559200814\\
 5.42484803199768	1.51082970678374\\
% 5.53551692962647	1.43428989589177\\
% 5.63034815788269	1.35983287384028\\
% 5.82210655212402	1.29516707012209\\
 5.82214756011963	1.23977719775445\\
% 5.94401998519897	1.18469050126572\\
% 6.02550621032715	1.12973098581813\\
% 6.16116161346436	1.07278095631767\\
 6.2615348815918	1.01523326466474\\
% 6.36849851608276	0.955975016686068\\
% 6.4345142364502	0.898554059114019\\
% 6.55445404052734	0.84186643755541\\
 6.70095343589783	0.788557055867017\\
% 6.73470945358276	0.735130219958592\\
% 6.82313985824585	0.683195460474963\\
% 6.92498059272766	0.63619216813828\\
 7.05544490814209	0.595527839001051\\
% 7.15404720306397	0.555194575224391\\
% 7.24332089424133	0.517981115163479\\
% 7.32086343765259	0.482430816376336\\
 7.45592446327209	0.451410988679072\\
% 7.54519577026367	0.419619325403175\\
% 7.6316234588623	0.388908570974152\\
% 7.73830981254578	0.359068230099879\\
 7.85962791442871	0.331014201556215\\
% 7.94003291130066	0.303780769758376\\
% 8.05915970802307	0.280244354182244\\
% 8.19914145469665	0.255234161260802\\
% 8.32294745445251	0.231817485817857\\
 8.32404775619507	0.208551106681043\\
% 8.43282551765442	0.187206592113242\\
% 8.53713102340698	0.164990461699531\\
% 8.6296338558197	0.144788522888419\\
 8.73589963912964	0.124186628716068\\
% 8.84264702796936	0.106228701673253\\
% 8.93937177658081	0.0873478442630997\\
% 9.09919638633728	0.0704973409828398\\
 9.15441460609436	0.0551270380884148\\
% 9.23950095176697	0.0412650715637142\\
% 9.38686461448669	0.0293256897062797\\
% 9.44052762985229	0.0184322909606749\\
 9.53487749099731	0.00966964972469597\\
% 9.64274806976318	0.00166559626437035\\
% 9.7318009853363	-0.00497319046452915\\
% 9.8260941028595	-0.0110927267691885\\
 9.96415657997131	-0.0155564027031225\\
% 10.031244468689	-0.0196122936893097\\
% 10.1295303821564	-0.0226454592171166\\
% 10.2355622768402	-0.0255074852823576\\
 10.4032809257507	-0.0273116537221973\\
% 10.4906537055969	-0.0290143865558328\\
% 10.5653883934021	-0.0302875931175635\\
% 10.6581940174103	-0.0314884388387462\\
 10.784938287735	-0.0320179333818729\\
% 10.841081571579	-0.0325837067585333\\
% 10.9676131725311	-0.032360504485009\\
% 11.0286945819855	-0.0322764771172181\\
 11.1352338314056	-0.03202010113041\\
% 11.2198607444763	-0.0318477281898879\\
% 11.328627538681	-0.0315658985015972\\
% 11.4415976524353	-0.0314407020950576\\
 11.6073987007141	-0.0312985505404235\\
% 11.6468948841095	-0.0313836104886605\\
% 11.7516786575317	-0.0316584342896569\\
% 11.8524360179901	-0.032139511648591\\
 11.956632566452	-0.0330901809666067\\
% 12.0420829772949	-0.0341436169831582\\
% 12.1279501438141	-0.0358129028390977\\
% 12.2416917800903	-0.0375357444160151\\
 12.3331725120544	-0.0398116342095598\\
% 12.4218036651611	-0.0420926276379454\\
% 12.5414697647095	-0.0448283948120078\\
% 12.6532685279846	-0.0474542241692042\\
 12.7462152957916	-0.051082866802501\\
% 12.8523222923279	-0.0546128677944697\\
% 12.9294535636902	-0.0590803638379409\\
% 13.0822853565216	-0.0633269772688383\\
 13.1939737319946	-0.0686242907713748\\
% 13.2472683906555	-0.0736970140808051\\
% 13.378995847702	-0.080285213071762\\
% 13.5621871471405	-0.0865191551725211\\
 13.5621916770935	-0.0936504578142072\\
% 13.6228856563568	-0.100389264642844\\
% 13.7575747489929	-0.108057509262126\\
% 13.8234891414642	-0.115194839871152\\
 13.9262966632843	-0.12353203872957\\
% 14.0689234256744	-0.131311466858325\\
% 14.1454946517944	-0.140252741829511\\
% 14.2291230678558	-0.148499842257519\\
 14.3729912757874	-0.157946165686112\\
% 14.4506949901581	-0.16659344284484\\
% 14.5584237098694	-0.175769221068732\\
% 14.725235414505	-0.184036736354141\\
 14.7252416133881	-0.193225117319369\\
% 14.8966662406921	-0.201502332017154\\
% 15.0779006004333	-0.210787848185428\\
% 15.0799574375153	-0.21913085975383\\
 15.1401779174805	-0.227880825857309\\
% 15.2216836929321	-0.235510878583202\\
% 15.356421661377	-0.243743398937657\\
% 15.4242901325226	-0.250982061130417\\
 15.5294272422791	-0.259286575727229\\
% 15.6678728580475	-0.266673873865908\\
% 15.7614891052246	-0.2745363212856\\
% 15.8384966373444	-0.281355108824284\\
 15.9403559684753	-0.28815017695003\\
% 16.0358340263367	-0.29392578735745\\
% 16.1387001991272	-0.301229041229522\\
% 16.2235087871552	-0.307493697744441\\
 16.4228355407715	-0.314133874677438\\
% 16.4617208957672	-0.319866025407563\\
% 16.536874961853	-0.325976874434462\\
% 16.631720495224	-0.331262789844434\\
 16.7993404388428	-0.337431649956228\\
% 16.8255044937134	-0.342902895181926\\
% 16.9287883758545	-0.348625861612595\\
% 17.0428664207458	-0.353661729696668\\
 17.1474303722382	-0.358748886373762\\
% 17.2291614532471	-0.363317608826417\\
% 17.3465222835541	-0.368103907469649\\
% 17.4238087654114	-0.372360238352188\\
 17.5406684398651	-0.376593197991513\\
% 17.6217422008514	-0.380376650561358\\
% 17.8375637054443	-0.384071530499996\\
% 17.837580871582	-0.387272828223347\\
 17.9298414707184	-0.390653273860119\\
% 18.0336350917816	-0.393544558982988\\
% 18.2846984386444	-0.396410012437407\\
% 18.2847022533417	-0.398825794417398\\
 18.3382050514221	-0.401585828105976\\
% 18.4275156974792	-0.404105216029768\\
% 18.5614318370819	-0.406990648519074\\
% 18.6368317127228	-0.409761208680717\\
 18.7376241207123	-0.412780059017644\\
% 18.8722116470337	-0.415901195268427\\
% 18.9676610946655	-0.418916591995714\\
% 19.2566427707672	-0.421795487037317\\
 19.2566499233246	-0.424098590868402\\
% 19.2566549301147	-0.42639137096354\\
% 19.5335189819336	-0.428525875936273\\
% 19.5335454463959	-0.430630493651995\\
 19.5335821628571	-0.432530006682482\\
% 19.8561558246613	-0.434458118050284\\
% 19.8561932563782	-0.43621573159964\\
% 19.856215429306	-0.437869998268184\\
 19.939593744278	-0.43925581535672\\
% 20.1148948192596	-0.440620779376717\\
% 20.2086979866028	-0.441979969250291\\
% 20.2411860942841	-0.443290261870949\\
 20.3258230209351	-0.444561868245795\\
% 20.4300326824188	-0.445833666279025\\
% 20.5301665782928	-0.447050334865299\\
% 20.6302778244019	-0.44820812008096\\
 20.7319926738739	-0.449256900404853\\
% 20.8867396831512	-0.450262841799958\\
% 21.0112361431122	-0.451238920586789\\
% 21.212642621994	-0.452140908704781\\
 21.212647151947	-0.453072341150431\\
% 21.2460777282715	-0.453984526635727\\
% 21.3257083415985	-0.45490762106526\\
% 21.4222948074341	-0.455807182777106\\
 21.6202604293823	-0.456680416673958\\
% 21.6202783107758	-0.457524918891449\\
% 21.8131494045258	-0.458358040381647\\
% 21.9340204715729	-0.459169636295002\\
 21.9340269088745	-0.459909697681992\\
% 22.0973948955536	-0.460577694944766\\
% 22.1650163650513	-0.461213040117128\\
% 22.2429306030273	-0.461814856488202\\
 22.32282538414	-0.462453397857903\\
% 22.4374601364136	-0.463081582013748\\
% 22.554838848114	-0.463782353792793\\
% 22.6611167907715	-0.464482374744765\\
 22.7834405422211	-0.465169214749359\\
% 22.8527333259583	-0.465883553931386\\
% 22.9370495796204	-0.466544997178302\\
% 23.0243656158447	-0.46715918192718\\
 23.132009935379	-0.467690493103369\\
% 23.2255770683289	-0.468249644559108\\
% 23.3404690742493	-0.468780621960402\\
% 23.4402012348175	-0.469317229693781\\
 23.5462953567505	-0.46985562882773\\
% 23.627540063858	-0.470417532515414\\
% 23.7291373729706	-0.471011424136552\\
% 23.8266167163849	-0.471626060666001\\
 23.9292280197144	-0.47221710584401\\
% 24.0224551677704	-0.472854514350717\\
% 24.1974343776703	-0.473431929589577\\
% 24.3536417007446	-0.47404472747631\\
 24.3537036895752	-0.474569374368169\\
% 24.4515547275543	-0.475062611034652\\
% 24.5563561439514	-0.475381375572391\\
% 24.6695627689362	-0.47564563991007\\
 24.7547094345093	-0.475903463562855\\
% 24.8209278106689	-0.476086380455911\\
% 24.9442512512207	-0.476415011906765\\
% 25.0268671035767	-0.476856306011118\\
 25.1480109214783	-0.477468902347324\\
% 25.2522289276123	-0.478085492601286\\
% 25.3239643096924	-0.478468223039567\\
% 25.6276008605957	-0.478747191502212\\
 25.6276306629181	-0.478857609334598\\
% 25.6276738166809	-0.478937389628491\\
};
\addlegendentry{$\theta_{y_{p}}$}

\addplot [color=mycolor5, line width=1.4pt]
  table[row sep=crcr]{%
 -0.57880334854126	0\\
% -0.48006682395935	0\\
% -0.383595514297485	0\\
% -0.282954502105713	0\\
 -0.182975339889526	0\\
% -0.0815842628479002	0\\
% 0.01931209564209	0\\
% 0.118979644775391	3.7343001305853\\
% 0.21865553855896	5.22872363114783\\
 0.321214628219605	6.05514856710801\\
% 0.428329658508301	6.6798283068922\\
% 0.53482551574707	7.15678515212414\\
% 0.631938886642456	7.53768129582272\\
 0.739639472961426	7.84146798333177\\
% 0.822497081756592	8.12679552114423\\
% 0.94228048324585	8.40361072345536\\
% 1.24386186599731	8.66602122748145\\
 1.24391431808472	8.94890006799301\\
% 1.24392242431641	9.22209956746133\\
% 1.36236472129822	9.50470083858795\\
% 1.46429152488709	9.77910878894068\\
 1.53867239952087	10.0640926207689\\
% 1.62087888717651	10.3618105708488\\
% 1.72517223358154	10.6604223409017\\
% 1.8207799911499	10.9484842440183\\
 1.94416422843933	11.2499909083149\\
% 2.02091021537781	11.5584932789934\\
% 2.13180298805237	11.8466879286184\\
% 2.22401900291443	12.1241718418623\\
% 2.32308192253113	12.3869472731872\\
 2.44293112754822	12.6331506368724\\
% 2.52259345054626	12.8570308632011\\
% 2.62238688468933	13.0570621394495\\
% 2.74145216941834	13.2351231970124\\
 2.82201690673828	13.3945031830867\\
% 2.93049092292786	13.5366254890187\\
% 3.02306551933289	13.6565933531092\\
% 3.12266702651978	13.7529892456782\\
 3.22329874038696	13.8304652760967\\
% 3.34136552810669	13.8833869114096\\
% 3.44502396583557	13.9198648918748\\
% 3.5300621509552	13.9378852901916\\
 3.62293739318848	13.940187831324\\
% 3.72189350128174	13.9273676861608\\
% 3.84191269874573	13.9097162670705\\
% 3.93421573638916	13.884426653887\\
 4.02698321342468	13.851230452246\\
% 4.12289471626282	13.806237921729\\
% 4.23660249710083	13.7586251802447\\
% 4.33962697982788	13.713991399756\\
 4.44047636985779	13.6676336968299\\
% 4.56040711402893	13.6153596478614\\
% 4.630579662323	13.5592548112697\\
% 4.72640986442566	13.5004403326858\\
 4.87428517341614	13.4431056546716\\
% 5.00027008056641	13.3834133639543\\
% 5.03321404457092	13.3239188615665\\
% 5.1221203327179	13.2652256286696\\
 5.24035425186157	13.2142496537672\\
% 5.32195730209351	13.1617639785857\\
% 5.42484803199768	13.109232178473\\
% 5.53551692962647	13.0567281404002\\
 5.63034815788269	13.0096817283816\\
% 5.82210655212402	12.9679581776254\\
% 5.82214756011963	12.9313498089687\\
% 5.94401998519897	12.8956181681137\\
 6.02550621032715	12.8595756162497\\
% 6.16116161346436	12.8241295087172\\
% 6.2615348815918	12.7894777085314\\
% 6.36849851608276	12.7537291442024\\
 6.4345142364502	12.720016101148\\
% 6.55445404052734	12.6860238976336\\
% 6.70095343589783	12.6539772041619\\
% 6.73470945358276	12.6216195573397\\
 6.82313985824585	12.5918765790229\\
% 6.92498059272766	12.5626965789309\\
% 7.05544490814209	12.5364397669344\\
% 7.15404720306397	12.510058000289\\
 7.24332089424133	12.4859282444308\\
% 7.32086343765259	12.4619872099018\\
% 7.45592446327209	12.4404254595074\\
% 7.54519577026367	12.4195142110975\\
 7.6316234588623	12.4004537660054\\
% 7.73830981254578	12.3814545340972\\
% 7.85962791442871	12.3637176177665\\
% 7.94003291130066	12.3461576061767\\
 8.05915970802307	12.3302158706557\\
% 8.19914145469665	12.3138791849662\\
% 8.32294745445251	12.2988226842847\\
% 8.32404775619507	12.2847332462475\\
 8.43282551765442	12.2720887137161\\
% 8.53713102340698	12.2589509320172\\
% 8.6296338558197	12.2469061679412\\
% 8.73589963912964	12.2347206439293\\
 8.84264702796936	12.2235016545074\\
% 8.93937177658081	12.2121639224787\\
% 9.09919638633728	12.201841428605\\
% 9.15441460609436	12.1922401252294\\
 9.23950095176697	12.1834841712596\\
% 9.38686461448669	12.1743071199857\\
% 9.44052762985229	12.1658746972967\\
% 9.53487749099731	12.1575159722743\\
 9.64274806976318	12.1498192775094\\
% 9.7318009853363	12.1421056918019\\
% 9.8260941028595	12.1350508338062\\
% 9.96415657997131	12.1278294226757\\
 10.031244468689	12.1212332885221\\
% 10.1295303821564	12.1147711565893\\
% 10.2355622768402	12.1088940921514\\
% 10.4032809257507	12.1029430028591\\
 10.4906537055969	12.0974558477293\\
% 10.5653883934021	12.0921297356192\\
% 10.6581940174103	12.0872325784771\\
% 10.784938287735	12.0824908360528\\
 10.841081571579	12.0781521989894\\
% 10.9676131725311	12.0738007694022\\
% 11.0286945819855	12.0698374035578\\
% 11.1352338314056	12.0660253306991\\
 11.2198607444763	12.062545190911\\
% 11.328627538681	12.0591428101219\\
% 11.4415976524353	12.0560224806215\\
% 11.6073987007141	12.0529005353727\\
 11.6468948841095	12.0499968038997\\
% 11.7516786575317	12.047065039675\\
% 11.8524360179901	12.0443029611722\\
% 11.956632566452	12.0413898583624\\
 12.0420829772949	12.0386363370239\\
% 12.1279501438141	12.0357866759435\\
% 12.2416917800903	12.0330623029264\\
% 12.3331725120544	12.030205690593\\
 12.4218036651611	12.027485748249\\
% 12.5414697647095	12.0247612323004\\
% 12.6532685279846	12.0222988240569\\
% 12.7462152957916	12.0197033187698\\
 12.8523222923279	12.0173095613451\\
% 12.9294535636902	12.0146597963604\\
% 13.0822853565216	12.0122943420433\\
% 13.1939737319946	12.0095206202491\\
 13.2472683906555	12.0069620612919\\
% 13.378995847702	12.003979947512\\
% 13.5621871471405	12.0012863962735\\
% 13.5621916770935	11.9983846562271\\
 13.6228856563568	11.9957406765129\\
% 13.7575747489929	11.9929695340214\\
% 13.8234891414642	11.9904588399177\\
% 13.9262966632843	11.9877092030414\\
 14.0689234256744	11.9852353777812\\
% 14.1454946517944	11.9826125425629\\
% 14.2291230678558	11.9802646995085\\
% 14.3729912757874	11.9777611593453\\
 14.4506949901581	11.9755322122129\\
% 14.5584237098694	11.9732975059459\\
% 14.725235414505	11.9713179011948\\
% 14.7252416133881	11.9692332966662\\
 14.8966662406921	11.9674008533766\\
% 15.0779006004333	11.965526380085\\
% 15.0799574375153	11.9638492166969\\
% 15.1401779174805	11.9621453247082\\
 15.2216836929321	11.9606531034246\\
% 15.356421661377	11.9591447611964\\
% 15.4242901325226	11.9577982047193\\
% 15.5294272422791	11.9564970211554\\
 15.6678728580475	11.9554235868129\\
% 15.7614891052246	11.9544693261338\\
% 15.8384966373444	11.953745783202\\
% 15.9403559684753	11.9530662587003\\
 16.0358340263367	11.9525827515309\\
% 16.1387001991272	11.9519125081904\\
% 16.2235087871552	11.9513620883085\\
% 16.4228355407715	11.9507799375838\\
 16.4617208957672	11.9503533662007\\
% 16.536874961853	11.9500664540677\\
% 16.631720495224	11.949939430557\\
% 16.7993404388428	11.9497736102143\\
 16.8255044937134	11.9497104847412\\
% 16.9287883758545	11.9496104222905\\
% 17.0428664207458	11.94957481281\\
% 17.1474303722382	11.94946449343\\
 17.2291614532471	11.9493635199011\\
% 17.3465222835541	11.9493076718836\\
% 17.4238087654114	11.9493197723855\\
% 17.5406684398651	11.9494059834398\\
 17.6217422008514	11.9495436778153\\
% 17.8375637054443	11.9497132241564\\
% 17.837580871582	11.9499520222554\\
% 17.9298414707184	11.9502383257259\\
 18.0336350917816	11.9505459529288\\
% 18.2846984386444	11.9508661941403\\
% 18.2847022533417	11.9511725843994\\
% 18.3382050514221	11.9515150134877\\
 18.4275156974792	11.9518438280963\\
% 18.5614318370819	11.9522534066198\\
% 18.6368317127228	11.9526565619128\\
% 18.7376241207123	11.9531079655996\\
 18.8722116470337	11.9535459381044\\
% 18.9676610946655	11.9539312089119\\
% 19.2566427707672	11.9542968430416\\
% 19.2566499233246	11.954593221668\\
 19.2566549301147	11.9548292342153\\
% 19.5335189819336	11.9550910841867\\
% 19.5335454463959	11.9553697620165\\
% 19.5335821628571	11.95569976036\\
 19.8561558246613	11.9560256769642\\
% 19.8561932563782	11.9563098602914\\
% 19.856215429306	11.9565596958173\\
% 19.939593744278	11.9567759373265\\
 20.1148948192596	11.9569906542929\\
% 20.2086979866028	11.957244261306\\
% 20.2411860942841	11.9575028060305\\
% 20.3258230209351	11.9577925670439\\
 20.4300326824188	11.9580691832505\\
% 20.5301665782928	11.9583241764968\\
% 20.6302778244019	11.95858319784\\
% 20.7319926738739	11.958824778939\\
 20.8867396831512	11.9590472437695\\
% 21.0112361431122	11.9592719234351\\
% 21.212642621994	11.959455646341\\
% 21.212647151947	11.9596658053521\\
 21.2460777282715	11.9598827287906\\
% 21.3257083415985	11.9601406123727\\
% 21.4222948074341	11.9603524820272\\
% 21.6202604293823	11.960538230607\\
 21.6202783107758	11.9607281303008\\
% 21.8131494045258	11.9609027503606\\
% 21.9340204715729	11.9610962031477\\
% 21.9340269088745	11.9612223397316\\
 22.0973948955536	11.9613145822199\\
% 22.1650163650513	11.9614454420274\\
% 22.2429306030273	11.9615427391588\\
% 22.32282538414	11.9617165334046\\
 22.4374601364136	11.9619189132794\\
% 22.554838848114	11.9621981806634\\
% 22.6611167907715	11.9624945751143\\
% 22.7834405422211	11.9627707464436\\
 22.8527333259583	11.9630779978738\\
% 22.9370495796204	11.9633727596982\\
% 23.0243656158447	11.9636663732914\\
% 23.132009935379	11.9638787425798\\
 23.2255770683289	11.9641144973047\\
% 23.3404690742493	11.9643589581014\\
% 23.4402012348175	11.9645813190967\\
% 23.5462953567505	11.9647808921259\\
 23.627540063858	11.9649953830774\\
% 23.7291373729706	11.9652093144848\\
% 23.8266167163849	11.9653791619934\\
% 23.9292280197144	11.9654699080942\\
 24.0224551677704	11.9655242592497\\
% 24.1974343776703	11.9655618144129\\
% 24.3536417007446	11.96550221827\\
% 24.3537036895752	11.965482305407\\
 24.4515547275543	11.965500388243\\
% 24.5563561439514	11.9656594866865\\
% 24.6695627689362	11.9658486053778\\
% 24.7547094345093	11.9660406032566\\
 24.8209278106689	11.9662465483424\\
% 24.9442512512207	11.966452085503\\
% 25.0268671035767	11.9666726524448\\
% 25.1480109214783	11.9668323309208\\
 25.2522289276123	11.9669586108247\\
% 25.3239643096924	11.966991144973\\
% 25.6276008605957	11.9669743126725\\
% 25.6276306629181	11.9670070030397\\
 25.6276738166809	11.9670509098307\\
};
\addlegendentry{$\theta_{1}$}

\end{axis}
\end{tikzpicture}%
\vspace{-0.7em}
\caption{Coefficient estimation over time applying Algorithm~\ref{alg:onelayerlearningalg} for an initially unknown $\phi_{2}$.}
\label{fig:onelayerlearningcoef}
\vspace{-0.7em}
\end{figure}
For the first half of the considered time-span the variance dominates the target cost and the setpoints are placed close to the locations of maximal variance within the partitions. The influence of the variance diminishes with its decreasing maximal value and the goal continuously shifts to the coverage problem. It is important to note that the timescale of the graphics highly differs from the two-layers learning approach presented in Section~\ref{subsec:experimenttwolayerlearning}. For the conducted experiments the one-layer learning approach takes approximately one-third of the time to steer its agents in a nearly optimal configuration. However, by the time the experiment has been interrupted, all mean estimates show a maximal error of less than 0.085 concerning the true coefficient value, and maximal variance reads as approximately 0.007. Therefore, both of them are slightly higher than in the two-layers learning-based approach and the obtained estimates of the parameter vector $\theta$ are slightly less certain and precise, due to a reduction in exploration movements. Both values would decrease if a large variance scaling factor $S$ is used.
\vspace{-0.7em}


\section{Discussion and conclusions}
\label{sec:conclusions}
This work presented a framework, consisting of two methods, to optimally cover a predefined convex area  using a non-homogeneous fleet of agents whose movements are determined by their nonlinear dynamics.  To this end, a tracking MPC without terminal constraints was deployed in both approaches to allow collision avoidance, as well as the consideration of additional constraints. While the first method relies on a two-layered structure and passes an online calculated reference to the individual MPC of each agent, the second method overcomes the hierarchical structure and directly integrates the calculation of the succeeding optimal configuration into the cost function of each agent's MPC. The developed methods are proposed for the case with known environment, represented by a known density function $\phi$, and are individually extended to the scenario in which the density function needs to be actively learned by the agents. 
Recursive feasibility and convergence to an optimal configuration were formally proven for both methods in each scenario. Furthermore, hardware results were obtained for all the proposed methods using a miniature racing car platform. With the performance of the two methodologies being rather comparable in a known environment, one of the benefits of the one-layer approach is its simplicity. Furthermore, the one-layer learning approach allows for a more efficient operation in the considered experiments, since it does \textit{not} require an explicit exploration vs. exploitation decision, but automatically adjusts it based on the remaining uncertainty. However, given the integration in its MPC cost function, the one layer approach is computationally more expensive than its two layer opponent. An interesting challenge for further research is a decentralized implementation for the learning and the Voronoi partition calculation~\cite{Bullo2012}, e.g., relying only on gossiping between the agents. 
\vspace{-0.7em}


\bibliographystyle{docstyle/IEEEtran} % official IEEE transaction referencing style
\bibliography{main}


\appendix
\section{Proofs}
\label{sec:proofs}
In this section, the proofs of Theorems~\ref{theorem:twolayers}-\ref{theorem:onelayerlearning} are given. Note that the dependency of the optimal costs, e.g. $J^{*}$ and $V_{N}^{*}$, on their current Voronoi partition will be dropped for notational ease if clear from the context. In all theorems, Theorem~\ref{theorem:theo4boccia} ensures the partitions are updated, while recursive feasibility is ensured by only updating the partitions if the condition presented in~\eqref{eq:feasibilitycond} is fulfilled. Furthermore, recall that $w$ indicates the timestep of the most recent partition update. 

\subsection{Proof of Theorem~\ref{theorem:twolayers}}
\label{subsec:twolayersproof}The proof of Theorem~\ref{theorem:twolayers} is divided into two steps. First, convergence of each agent to its reference $r$ is shown for a fixed Voronoi partition. This is followed by the proof that the fulfillment of the partition update requirement described in equations~\eqref{eq:updatereq1},~\eqref{eq:updatereq2} and~\eqref{eq:feasibilitycond} is satisfied after a finite time. Subsequently, convergence to a centroidal Voronoi configuration follows by  Proposition~\ref{proposition:cortesconvergence}.

\subsubsection{Convergence to Reference} 
The result will be proven by first introducing two intermediate results (Prop.~\ref{proposition:notminsteadynotcosetwolayers} and~\ref{proposition:upperlowerboudJtilde}).
\begin{proposition} Let Assumptions~\ref{assumption:dynamics},~\ref{assumption:expocostcontrollability},~\ref{assumption:boundedbyd} and~\ref{assumption:steadystatedecreasetwolayers} hold. Then there exist a constant $\bar{a}_{i} > 0$ such that, for any $p_w \in \mathbb{A}^M$ and according reference $r$, any feasible state $x_{k} \in \mathbb{X}^{\mathrm{int}}: Cx_k \in \bar{\mathbb{W}}_{p_w,i}^{\mathrm{int}}$, the optimal solution of problem~\eqref{eq:nonlineartrackingmpcwithcolavoidance} satisfies
\begin{equation}
\begin{split}
    \ell_{i}^*(x_{k}, s_{k}^{*}) \geq \bar{a}_{i}d_{i}(s_{k}^{*})_{\mathbb{T}_{\mathbb{W}_{p_w,i},r}}^{2}.
\end{split}
\label{eq:prop12layer}
\end{equation}
\label{proposition:notminsteadynotcosetwolayers}
\end{proposition}
\vspace{-1.9em}
\noindent \emph{Proof:} For time step $k\in \mathbb{N}$, assume for contradiction that $\ell_{i}^*(x_{k}, s_{k}^{*}) < \bar{a}_{i}d_{i}(s_{k}^{*})_{\mathbb{T}_{\mathbb{W}_{p_w,i},r}}^{2}$. Following~\cite[Proposition 1]{Soloperto2021} using compact constraints and $\bar{a}_{i}$ small enough, we can define a setpoint $s'_{k} \in \mathbb{S}_{\mathbb{W}_{p_w,i}}$ according to Assumption~\ref{assumption:steadystatedecreasetwolayers}, such that
\begin{align*}
    &V_{N,i}^{*}(x_{k},s'_{k},\mathbb{W}_{p_w,i}) - V_{N,i}^{*}(x_{k},s_{k}^{*},\mathbb{W}_{p_w,i}) + \ell_{T,i}(\bar{p}'_{k} - r_{i,k})\\
    & -\ell_{T,i}(\bar{p}^*_{k} - r_{i,k}) \overset{\eqref{eq:assumption2gammavmax},~\eqref{eq:assumption32twolayer}}{\leq} \gamma_{i} \ell_{i}^{*}(x_{k},s'_{k}) - \beta_{2,i}\epsilon d_{i}( s_{k}^{*} )_{\mathbb{T}_{\mathbb{W}_{p_w,i},r}}^{2} \\
    &\overset{\eqref{eq:boundtwosteadystatestwolayer}}{\leq} \! \gamma_{i} \xi_{1,i} \ell_{i}^*(x_{k},s_{k}^{*}) \! +  \! \gamma_{i}\xi_{2,i} d_{i}(s'_{k} \! - \! s_{k}^{*})^{2} \! - \! \beta_{2,i}\epsilon d_{i}( s_{k}^{*} )_{\mathbb{T}_{\mathbb{W}_{p_w,i},r}}^{2} \\ 
    &\overset{\eqref{eq:assumption31twolayer}}{<} (\gamma_{i} \xi_{1,i}\bar{a}_i + \gamma_{i} \xi_{2,i}\beta_{1,i}^{2}\epsilon^{2}- \beta_{2,i}\epsilon) d_{i}( s_{k}^{*} )_{\mathbb{T}_{\mathbb{W}_{p_w,i},r}}^{2} \\ &:= - \bar{c}_id_{i}( s_{k}^{*} )_{\mathbb{T}_{\mathbb{W}_{p_w,i},r}}^{2}. 
\label{eq:prop1}
\end{align*}
Given $\bar{a}_i$ small enough, an $\epsilon > 0$ can be found such that $\bar{c}_i > 0$. This implies that the new setpoint $s^{\prime}_k$ yields a smaller cost, which contradicts the optimality of $s_k^*$.\hfill{$\blacksquare$}

\begin{proposition}
Let Assumptions~\ref{assumption:dynamics},~\ref{assumption:expocostcontrollability},~\ref{assumption:boundedbyd},~\ref{assumption:steadystatedecreasetwolayers} and~\ref{assumption:upperboundtargetcost} hold. Defining  $a_{i} = \frac{1}{2} \min{(\bar{a}_{i},\alpha_{1,i})}$ and recalling that $u^*_{x,s}$ describes the minimizer of $\min_{u \in \mathbb{U}_{i}} \ell_{i}(x,u,s)$. Then for for any $p_w \in \mathbb{A}^M$, according reference $r$ and setpoint $s_k^* \in \mathbb{S}_{\mathbb{W}_{p_w,i}}$ and any feasible state $x_{k} \in \mathbb{X}^{\mathrm{int}}: Cx_k \in \bar{\mathbb{W}}_{p_w,i}^{\mathrm{int}}$ it holds
\begin{equation}
    \tilde{J}_{i}^{*}(x_{k}, \mathbb{W}_{p_w,i}) \! \geq \! a_{i}(d_{i}(s_{k}^{*})_{\mathbb{T}_{\mathbb{W}_{p_w,i},r}}^{2} \! + \! d_{i}((x_{k},u_{x_k,s_k^*}^{*}) \! - \! s_{k}^{*})^2).
\label{eq:lowerbound2layer}
\end{equation}
Further, denote by $b_{i} \! = \! \max{(\gamma_{i,V_{\max,i}}\alpha_{2,i}, \beta_{T,i})}$. Then, we have 
\begin{equation}
     \tilde{J}_{i}^{*}(x_{k}, \mathbb{W}_{p_w,i}) \! \leq \! b_{i} (d_{i}(s_{k}^{*})_{\mathbb{T}_{\mathbb{W}_{p_w,i},r}}^{2} \! + \! d_{i}((x_{k},u_{x_k,s_k^*}^{*}) \! - \! s_{k}^{*})^2).
\label{eq:upperbound2layer}
\end{equation}
\label{proposition:upperlowerboudJtilde}
\end{proposition}
\vspace{-1.9em}
\noindent \emph{Proof:} As regards the lower bound, we have 
\begin{align}
     &\tilde{J}_{i}^{*}(x_{k}, \mathbb{W}_{p_w,i}) \geq  V_{N,i}^{*}(x_{k},s_{k}^{*},\mathbb{W}_{p_w,i}) \geq \ell_{i}(x_{k},u_{x_k,s_k^*}^{*},s_{k}^{*}) \notag \\ &\overset{\eqref{eq:boundstagecosttwolayer},\eqref{eq:prop12layer}}{\geq} \frac{\bar{a}_{i}}{2}d_{i}(s_{k}^{*})_{\mathbb{T}_{\mathbb{W}_{p_w,i},r}}^{2} + \frac{\alpha_{1,i}}{2}d_{i}((x_{k},u_{x_k,s_k^*}^{*}) - s_{k}^{*})^2 \label{eq:lowerbound2layertmp} \\ & \geq a_{i}(d_{i}(s_{k}^{*})_{\mathbb{T}_{\mathbb{W}_{p_w,i},r}}^{2} + d_{i}((x_{k},u_{x_k,s_k^*}^{*}) - s_{k}^{*})^2). \notag
\end{align}
To show the existence of an upper bound, we leverage Assumptions~\ref{assumption:expocostcontrollability} and~\ref{assumption:upperboundtargetcost}. In consideration that combining~\cite{Boccia2014} with the constraint $V_{N,i} \leq V_{\max,i}$,  results in $V_{N,i}^{*}(x,s) \leq \gamma_{i,V_{\max,i}} \ell_{i}^{*}(x,s)$, where $\gamma_{i,V_{\max,i}} = \max{(\gamma_{i}, \frac{V_{\max,i}}{c_{i}},2)}$, it follows: 
\begin{align}
     &\tilde{J}_{i}^{*}(x_{k}, \mathbb{W}_{p_w,i}) \notag \\ & = V_{N,i}^{*}(x_{k},s_{k}^{*},\mathbb{W}_{p_w,i}) + \ell_{T,i}(s_{k}^{*}-r) -   l_{\mathbb{W}_{p_w,i},r,\min} \label{eq:upperbound2layer1} \\ & \leq
     %\gamma_{i,V_{\max,i}}\alpha_{2,i} d_{i}((x_{k},u_{x,s}^{*}) - s)^{2} =
     \gamma_{i,V_{\max,i}}\alpha_{2,i} d_{i}((x_{k},u_{x_k,s_k^*}^{*}) - s_{k}^{*}))^2 + \beta_{T,i}d_{i}(s_{k}^{*})_{\mathbb{T}_{\mathbb{W}_{p_w,i},r}}^{2} \notag \\ &  \leq b_{i}(d_{i}(s_{k}^{*})_{\mathbb{T}_{\mathbb{W}_{p_w,i},r}}^{2} + d_{i}((x_{k},u_{x_k,s_k^*}^{*}) - s_{k}^{*})^2). \notag
     \hspace{5.1em}\hfill\ensuremath{\blacksquare}
\end{align}
\vspace{-0.1em}

Prop.~\ref{proposition:notminsteadynotcosetwolayers} follows \cite[Proposition 1]{Soloperto2021} and shows that the optimal setpoint cannot be arbitrarily close to the current state-input pair, unless the latter belongs to the set $\mathbb{T}_{\mathbb{W}_{p_w,i},r}$ defined in~\eqref{eq:Tid2layer}; Prop.~\ref{proposition:upperlowerboudJtilde} later provides a lower and upper bound for $\tilde{J}_{i}^{*}(x_{k}, \mathbb{W}_{p_Wi}) = J_{i}^{*}(x_{k}, \mathbb{W}_{p_w,i}) - l_{\mathbb{W}_{p_w,i},r,\min}$. At this point, considering a fixed partition $\mathbb{W}_{p_w,i}$ and using the candidate $s_{k+1}=s_{k}^{*}$, Theorem~\ref{theorem:theo4boccia} in combination with Proposition~\ref{proposition:notminsteadynotcosetwolayers} and Assumption~\ref{assumption:boundedbyd} yields \begin{align*}
    &\tilde{J}_{i}^{*}(x_{k+1},\mathbb{W}_{p_w,i}) - \tilde{J}_{i}^{*}(x_{k}, \mathbb{W}_{p_w,i})\\ 
    &\leq V^{*}_{N,i}(x_{k+1},s_{k}^{*},\mathbb{W}_{p_w,i}) + \ell_{T,i}(s_{k}^{*} - r_{i}) - l_{\mathbb{W}_{p_w,i},r,\min} \\ 
    &- V^{*}_{N,i}(x_{k},s_{k}^{*},\mathbb{W}_{p_w,i}) - \ell_{T,i}(s_{k}^{*} - r_{i}) + l_{\mathbb{W}_{p_w,i},r,\min} \\
    &\overset{\eqref{eq:theorem5twolayer}}{\leq} - \bar{\alpha}_{N,i}\ell_{i}^{*}(x_{k},s_{k}^{*})\\ &\overset{\eqref{eq:prop12layer}}{\leq} -\frac{\bar{\alpha}_{N,i}}{2}(\ell_{i}^{*}(x_{k},s_{k}^{*}) + \bar{a}_{i}d_{i}(s_{k}^{*})_{\mathbb{T}_{\mathbb{W}_{p_w,i},r}}^{2}) \\ 
    &\overset{\eqref{eq:boundstagecosttwolayer}}{\leq} -\frac{\bar{\alpha}_{N,i}}{2}(\alpha_{1,i}d_{i}((x_{k},u_{x_{k},s_{k}^{*}}^{*})-s_{k}^{*})^{2} + \bar{a}_{i}d_{i}(s_{k}^{*})_{\mathbb{T}_{\mathbb{W}_{p_w,i},r}}^{2})\\  & \leq - \tilde{\alpha}_{N,i}(d_{i}((x_{k},u_{x_{k},s_{k}^{*}}^{*})-s_{k}^{*})^{2} + d_{i}(s_{k}^{*})_{\mathbb{T}_{\mathbb{W}_{p_w,i},r}}^{2}),
    \label{eq:2theorem2twolayer}
\end{align*}
where we defined $\tilde{\alpha}_{N,i}=\frac{\bar{\alpha}_{N,i}}{2}\text{min}\{\alpha_{1,i},\bar{\alpha}_i\}$. Using Proposition~\ref{proposition:upperlowerboudJtilde}, we have 
\begin{equation*}
    \begin{aligned}
    &\tilde{J}_{i}^{*}(x_{k+1},\mathbb{W}_{p_w,i}) - \tilde{J}_{i}^{*}(x_{k}, \mathbb{W}_{p_w,i}) \\
    &\leq - \tilde{\alpha}_{N,i}(d_{i}(s_{k}^{*})_{\mathbb{T}_{\mathbb{W}_{p_w,i},r}}^{2} + d_{i}((x_{k},u_{x_{k},s_{k}^{*}}^{*})-s_{k}^{*})^{2})\\ 
    & \leq  -\frac{\tilde{\alpha}_{N,i}}{b_{i}}\tilde{J}_{i}^{*}(x_{k}, \mathbb{W}_{p_w,i})
    \end{aligned}
\end{equation*}

\noindent and obtain the exponential convergence 
\begin{equation}
\begin{split}
    \tilde{J}_{i}^{*}(x_{k+\tilde{k}},\mathbb{W}_{p_w,i})
    \leq \Big(1 - \frac{\tilde{\alpha}_{N,i}}{b_{i}} \Big)^{\tilde{k}} \tilde{J}_{i}^{*}(x_{k}, \mathbb{W}_{p_w,i})
\end{split}
\label{eq:4fintitetimecovergencetwolayer}
\end{equation}
which is guaranteed to converge, because $b_{i} >\tilde{\alpha}_{N,i}>0$.

\subsubsection{Finite Time Update Condition Fulfillment, Part 1}\label{subsection:finitetimeconditiontwolayersp1}

Given \eqref{eq:4fintitetimecovergencetwolayer}, and the fact that $\tilde{J}_{i}^{*}(x_{k}, \mathbb{W}_{p,i})$ admits a uniform bound due to compact constraints and continuity of the cost, it follows that
for any $\epsilon > 0$ there exists a finite time step $\tilde{k}'>0$ such that $\tilde{J}_{i}^{*}(x_{k+\tilde{k}'},\mathbb{W}_{p,i}) < \epsilon$. 
Hence, by choosing $\epsilon > 0$ small enough and considering the lower bound~\eqref{eq:lowerbound2layer}, the state at time step $\tilde{k}'$ is arbitrary close to the set $\mathbb{T}_{\mathbb{W}_{p_w,i},r}$, and thus the update conditions~\eqref{eq:updatereq1},~\eqref{eq:updatereq2} hold (assuming we are not already in a centroidal Voronoi configuration).

\subsubsection{Finite Time Update Condition Fulfillment, Part 2}\label{subsection:finitetimeconditiontwolayersp2}We now need to ensure that the remaining Voronoi partition update condition~\eqref{eq:feasibilitycond} is fulfilled in finite time with the candidate state and input sequences proposed in~\eqref{eq:lemma1proposal}. For this we leverage the initial feasibility of the problem, the construction of the Voronoi partitions with respect to the agents positions, resulting in a position sequence $C_{i}x_{i,k} \in \bar{\mathbb{W}}_{p_{w},i}^{\mathrm{int}}, \forall k \in \mathbb{N}, \forall i \in \{1, \hdots, M\}$, as well as Assumption~\ref{assumption:voronoisetneverempty}. Specifically, 
 we need to find an upper bound for 
\begin{align}
    & V_{N,i}(x_{k+\tilde{k}'}, \hat{u}, s_{k}^{*}, \mathbb{W}_{p_w}) = \sum_{l=0}^{N-2}\ell_{i}(x_{l\vert k+\tilde{k}'}^{*}, u_{l \vert k+\tilde{k}'}^{*},s_{k}^{*}) + \notag \\ &  \ell_{i}(x_{N-1\vert k+\tilde{k}'}^{*},\bar{u}_{k+\tilde{k}'}^{*},s_{k}^{*}) + \notag \\ & \ell_{i}(f(x_{N-1\vert k+\tilde{k}'}^{*},\bar{u}_{k+\tilde{k}'}^{*}),\bar{u}_{k+\tilde{k}'}^{*},s_{k}^{*}), 
    \label{eq:trackingcostcandidate}
\end{align}
and such a bound is required to be decreasing as $\tilde{k}'$ increases, along the lines of~\eqref{eq:4fintitetimecovergencetwolayer}. To this aim, we will make use of $\epsilon$ as introduced above that can be seen as a decreasing function of $\tilde{k}'$.
The first term of~\eqref{eq:trackingcostcandidate} can be upper bounded by~\eqref{eq:4fintitetimecovergencetwolayer} as 
\begin{align}
    &\epsilon \geq V_{N,i}^{*}(x_{k+\tilde{k}'},s_{k}^{*},\mathbb{W}_{p_w})  \geq \sum_{l=0}^{N-2}\ell_{i}(x_{l\vert k+\tilde{k}'}^{*}, u_{l \vert k+\tilde{k}'}^{*},s_{k}^{*}). \notag %\notag \\ & \geq \sum_{l=0}^{N-2}\alpha_{1}d(({x}^{*}_{l \vert k+\tilde{k}'},u_{x_{l},s_{k}^{*}}^{*}) - s_{k}^{*}).
\label{eq:7finitetimeconvergence}
\end{align}
The upper bound for the last two terms is given by the following proposition.
\begin{proposition} Let Assumption~\ref{assumption:dynamics} and~\ref{assumption:boundedbyd} hold. 
Then, for any $x \in \mathbb{R}^{n_{i}}$ and any $s = (\bar{x},\bar{u}) \in \mathbb{S}_i$, it holds that
\begin{equation}
    \ell_{i}(x,\bar{u},s) + \ell_{i}(f(x,\bar{u}),\bar{u},s)\leq \gamma_{i}' \cdot \ell_{i}^{*}(x,s),
\label{eq:assumption22}
\end{equation}
where $\gamma_{i}'=\frac{2\alpha_{2,i}}{\alpha_{1,i}}\max_j\{1+\mathcal{L}_{i}^{\alpha_j}\}$.
\label{proposition:upperboundsuccedingstagecost}
\end{proposition}

\noindent \emph{Proof:} 
In accordance to equation~\eqref{eq:boundstagecosttwolayer} it holds that
\begin{align}
    &\ell_{i}(x,\bar{u},s) + \ell_{i}(f_{i}(x,\bar{u}),\bar{u},s)\leq \label{eq:fulfillassumption22}\\ 
    &2\alpha_{2,i} {\sum_{j=1}^{n_i}  \Big(\vert x_{j}-\bar{x}_j \vert^{a_j} + \vert f_{i,j}(x,\bar{u})-\bar{x}_j \vert^{a_j}\Big)}.\notag 
\end{align}
Because $\bar{x}$ is a steady state, we have that $f_{i,j}(x,\bar{u}) - \bar{x}_j = f_{i,j}(x,\bar{u}) - f_{i,j}(\bar{x},\bar{u})$. Thus, by Lipschitz continuity and reusing \eqref{eq:boundstagecosttwolayer}, we have that~\eqref{eq:fulfillassumption22} is upper bounded by 
\begin{align*}
    \hspace{3.3em}&2\alpha_{2,i}\sum_{j=1}^{n_i} (1 + \mathcal{L}_{i}^{a_j})\vert x_{j}-\bar{x}_j \vert^{\alpha_j} \leq \gamma_i'\ell_i^*(x,s).
    \hspace{3.3em}\hfill\ensuremath{\blacksquare}
\end{align*}
\vspace{-1.0em}

\noindent Finally, noting also that
\begin{equation}
{\displaystyle \epsilon \geq \ell_{i}(x_{N-1 \vert k+\tilde{k}'}^{*}, u_{N-1 \vert k+\tilde{k}'}^{*}, s_{k}^{*}) \geq \ell^{*}_{i}(x_{N-1 \vert k+\tilde{k}'}^{*},s_{k}^{*})},\notag 
\label{eq:8finitetimeconvergence}
\end{equation}
we obtain the desired bound
\begin{align}
     & V_{N,i}(x_{k+\tilde{k}'}, \hat{u}, s_{k}^{*}, \mathbb{W}_{\bar{p}}) \qquad \text{(see~\eqref{eq:trackingcostcandidate})} \notag \\
     &  \leq \sum_{l=0}^{N-2}\ell_{i}(x_{l\vert k+\tilde{k}'}^{*}, u_{l \vert k+\tilde{k}'}^{*},s_{k}^{*}) + \gamma_{i}' \ell^{*}_{i}(x_{N-1\vert k+\tilde{k}'}^{*},s_{k}^{*}) \notag \\ &
     \leq (1+ \gamma_{i}')\epsilon.
\label{eq:9finitetimeconvergence}
\end{align}
Since it holds for any of the addenda, which in turn are lower-bounded by a function describing the distance between state and setpoint, they will become arbitrarily close. Given Assumption~\ref{assumption:voronoisetneverempty} and that $\bar{\mathbb{W}}_{p_{k+\tilde{k}'}}^{\mathrm{int}}$ (the interior of the candidate partition at $\tilde{k}'$) contains $p_{k+\tilde{k}'}$,~\eqref{eq:feasibilitycond} will hold for a finite~$\tilde{k}'$.

\subsection{Proof of Theorem~\ref{theorem:twolayerslearning}}
\label{subsec:twolayerslearningproof}
The result of Theorem~\ref{theorem:twolayerslearning} builds upon~\cite[Proposition 1]{Todescato2017}. It ensures that the learning-based solution converges to the centroidal Voronoi partition if the estimate $\hat{\phi}_t$ converges in probability to the true density $\phi$. Then, convergence to a centroidal Voronoi configuration follows by Theorem~\ref{theorem:twolayers}. 

\begin{proposition}~\cite[Proposition 1]{Todescato2017} Assume $\hat{\phi}_{t}(p) \xrightarrow{\mathbb{P}} \phi(p)$, where $\mathbb{P}$ denotes convergence in probability in the space of continuous functions. Moreover, assume it is possible to start the Lloyd algorithm from the configuration reached by Algorithm~\ref{alg:twolayermpcalglearningimp} at time $\bar{t}$. Then the centroids $c^{L}_{\bar{t}}$ obtained from the Lloyd algorithm and the estimated ones $\hat{c}_{\bar{t}}$ coincide. Furthermore, for any $0<\delta < 1$, $\epsilon > 0$ an integer $N$ can be picked, such that there exists a $\bar{t}$ sufficiently large such that
\begin{equation}
     \mathbb{P}[ \Vert \hat{c}_{\bar{t} + t} - c^{L}_{\bar{t} + t} \Vert < \epsilon ] > 1 - \delta, \ \ t = 0, \hdots, N,
\label{eq:learningprop1}
\end{equation}
where $\mathbb{P}$ is associated to the Gaussian distribution of the estimated coefficients.
\label{proposition:convergenceofcentroidstwolayerslearning}
\end{proposition}
\noindent As by construction of the algorithm the locations of the agents never coincide, it is possible to start Lloyd's algorithm from an arbitrary instant in time. Then, it remains to show that $\hat{\phi}_{t}(p) \xrightarrow{\mathbb{P}} \phi(p)$ holds by applying Algorithm~\ref{alg:twolayermpcalglearningimp}.

\begin{lemma}
Assumptions~\ref{assumption:dynamics},~\ref{assumption:expocostcontrollability},~\ref{assumption:boundedbyd},~\ref{assumption:voronoisetneverempty},~\ref{assumption:bayesianconvergence},~\ref{assumption:steadystatedecreasetwolayers} and~\ref{assumption:upperboundtargetcost} hold
and consider a horizon length $N \geq N^{*}$ and $\rho>0, r_{i,\max} >0, \epsilon>0$ sufficiently small. Then for any $p \in \mathbb{A}$, $\hat{\phi}_t(p) \xrightarrow{\mathbb{P}} \phi(p)$. 
\end{lemma}
\noindent \emph{Proof:} We prove this by showing that, given any partition configuration $\mathbb{W}_p$, the maximum variance value inside the $i-$th partition, $\text{Var}_{i,\max,t} = \max_{p \in \mathbb{W}_{p,i}} \text{Var}_t(p)$, converges to zero for each $i$. This implies that $\text{Var}_t(p)$ at any point $p$ inside the partition
converges to zero with increasing $t$: this is equivalent to proving $\mathscr{L}^2$-convergence of $\phi_t(p)$, from which convergence in probability follows.\\
For a time instant $t$, define $p^{\text{maxvar}}_{t,i} = \arg\max_{p \in \mathbb{W}_{p,i}} \text{Var}_t(p)$, and consider the position of the $i-$th agent, $p_{t,i}$, such that $\|p^{\text{maxvar}}_{t,i} - p_{t,i}\| \leq (\rho + r_{\max} + \epsilon)$. Next, we show $\text{Var}(p_{t,i}) \geq c\text{Var}(p^{\text{maxvar}}_{t,i})$ for some $c>0$. To this aim, we have
\begin{align*}
    \text{Var}(p^{\text{maxvar}}_{t,i}) &\leq 2\Phi(p_{t,i})\Sigma_{t}\Phi(p_{t,i})^{\top}  \\
    + &2(\Phi(p_{t,i}) - \Phi(p^{\text{maxvar}}_{t,i}))\Sigma_t(\Phi(p_{t,i}) - \Phi(p^{\text{maxvar}}_{t,i}))^{\top}\\
    &\leq 2 \text{Var}(p_{t,i}) + 2\mathcal{L}_{\Phi}^2(\rho + r_{\max} + \epsilon)^2\|\Sigma_t\|,
\end{align*}
with some Lipschitz constant $\mathcal{L}_\Phi>0$ from Assumption~\ref{assumption:bayesianconvergence}. Moreover, by the linear independence of the features in $\Phi$ stated in the same Assumption, we have $\text{Var}(p^{\text{maxvar}}_{t,i}) \geq c_{\Phi}\|\Sigma_t\|$. Applying this inequality yields
\begin{equation*}
   \text{Var}(p_{t,i}) \geq \underbrace{(1 - 2\mathcal{L}_{\Phi}^2(\rho+r_{\max} + \epsilon)^2/c_{\Phi})/2}_{=:c} \text{Var}(p^{\text{maxvar}}_{t,i}), 
\end{equation*}
with $c>0$ for $\gamma>0, r_{i,\max} >0, \epsilon>0$ sufficiently small.\\
The proof is concluded by showing that $\text{Var}(p_{t,i})$ converges to zero. Assume for contradiction that \mbox{$\lim\sup_{t \rightarrow +\infty} \Phi(p_{t,i})\Sigma_t\Phi(p_{t,i})^{\top} \! = \! \tilde{c} > 0$.} We can isolate a subsequence $\{t_{\tau}\}_{\tau}$ such that $\lim_{\tau \rightarrow +\infty} \Phi(p_{t_{\tau},i})\Sigma_{t_{\tau}}\Phi(p_{t_{\tau},i})^{\top} \! = \! \tilde{c}$. \\ However, from recursive application of equation~\eqref{eq:bayessigmaupdate} we have that \mbox{$\lim_{\tau\rightarrow\infty}\Sigma_{t_\tau}^{-1} \! \succeq \! \sum_{\tau=1}^{\infty}\Phi(p_{t_{\tau},i})^{\top}\Phi(p_{t_{\tau},i})$}. 
This implies that $\Phi(p_{t_\tau,i})\Sigma_{t_\tau}\Phi(p_{t_\tau,i})^\top$ tends to 0, contradicting the starting hypothesis. 
\hfill$\blacksquare$

\subsection{Proof of Theorem~\ref{theorem:onelayer}}
\label{subsubsec:onelayerknownenvironmenttheory}
We first present two preliminary results that will be needed in the main proof. In particular, Proposition~\ref{proposition:notminsteadynotcoseonelayer} generalizes Proposition~\ref{proposition:notminsteadynotcosetwolayers} to non-convex target costs and Proposition~\ref{proposition:voronoiupdatecostdecrease} ensures that an update of the partitions in accordance to the current steady state does not increase the cost~\eqref{eq:cortescost}.

\begin{proposition} Let Assumptions~\ref{assumption:dynamics},~\ref{assumption:expocostcontrollability},~\ref{assumption:boundedbyd},~\ref{assumption:ballwithminconvex} and~\ref{assumption:steadystatedecreaseonelayer} hold. Then there exists a constant $\bar{a}_{i} > 0$ such that, for any $\bar{p}_w \in \mathbb{S}_{i}^{\mathrm{p}}$ any feasible state $x_{k} \in \mathbb{X}^{\mathrm{int}}:Cx_k \in \bar{\mathbb{W}}_{i}^{\mathrm{int}}$, the optimal solution of the MPC problem~\eqref{eq:fullnonlineartrackingmpconelayer} satisfies
\begin{equation}
    \ell_{i}^*(x_{k}, s_{k}^{*}) \geq \bar{a}_{i}\kappa(\Vert \bar{p}_{k}^{*} \Vert)_{\mathbb{T}_{\bar{p}_{k}^{*},\mathbb{W}_{\bar{p}_w,i}}}.
\label{eq:prop1}
\end{equation}
\label{proposition:notminsteadynotcoseonelayer}
\end{proposition}
\vspace{-1.2em}
\noindent \emph{Proof:} Follows the same steps as the proof of Proposition 2, with Assumption~\ref{assumption:steadystatedecreasetwolayers} replaced by Assumptions~\ref{assumption:ballwithminconvex} and~\ref{assumption:steadystatedecreaseonelayer}.\hfill{$\blacksquare$} 

\begin{proposition} 
For any $\bar{p},\bar{p}_k\in\mathbb{A}^M$, it holds
\begin{equation}
    H(\bar{p}_{k},\mathbb{W}_{\bar{p}_{k}}) \leq  H(\bar{p}_{k},\mathbb{W}_{\bar{p}}).
\label{eq:condassumption6}
\end{equation}
\label{proposition:voronoiupdatecostdecrease}
\end{proposition}
\vspace{-1.5em}
\noindent \emph{Proof:} 
Given setpoint positions $\bar{p}$, each partition $\mathbb{W}_{\bar{p},i}$ admits itself a partition $\{\mathbb{W}_{\bar{p} \rightarrow \bar{p}_{k},i \rightarrow j}\}_{j=1}^M$, explicitly depending on $\bar{p}_{k}$, where each $\mathbb{W}_{\bar{p} \rightarrow \bar{p}_{k},i \rightarrow j}$ describes the set of points which are currently assigned to the steady-state of agent $i$, but would be assigned to the steady-state of agent $j$ in case of a Voronoi partition update in accordance to $\bar{p}_{k}$. 
Therefore, with $g$ non-decreasing, it holds:
\begin{align*}
    &H(\bar{p}_{k},\mathbb{W}_{\bar{p}})  = \sum_{i=1}^{M}\int_{\mathbb{W}_{\bar{p},i}} g(\Vert q - \bar{p}_{i} \Vert)\phi(q)dq \\ 
    &= \sum_{i=1}^{M}\sum_{j=1}^{M}\int_{\mathbb{W}_{\bar{p}\rightarrow\bar{p}_{k},i\rightarrow j}} g(\Vert q - \bar{p}_{i} \Vert)\phi(q)dq \\ 
    &\overset{\eqref{eq:voronoidef}}{\geq} \sum_{i=1}^{M}\sum_{j=1}^{M}\int_{\mathbb{W}_{\bar{p}\rightarrow\bar{p}_{k},i\rightarrow j}} g(\Vert q - \bar{p}_{j} \Vert)\phi(q)dq \\
    &= \sum_{i=1}^{M}\int_{\mathbb{W}_{\bar{p}_{k},i}} g(\Vert q - \bar{p}_{i} \Vert)\phi(q)dq = H(\bar{p}_{k},\mathbb{W}_{\bar{p}_{k}}).
    \label{eq:assumption6}
    \hspace{3.0em}\hfill\ensuremath{\blacksquare}
\end{align*}

\noindent
The remainder of the proof amounts to showing the following facts: 1) convergence to a fixed setpoint, taking into account a possible update of the Voronoi tessellation; 2) convergence to a centroidal Voronoi partition; and 3) finite-time partition update. The following results thereby rely on a Lyapunov like decrease of the storage function for the MPC in~\eqref{eq:fullnonlineartrackingmpconelayer}.\\
The storage function we are considering is the summation of the optimal MPC cost over all agents, i.e., that is,
\begin{equation}
\begin{aligned}
&J^{*}(x_{k}, \mathbb{W}_{\bar{p}^{*}_{w}}) = V_{N}^{*}(x_{k}, s_{k}^{*},\mathbb{W}_{\bar{p}^{*}_{w}}) + \lambda  H(\bar{p}_{k},\mathbb{W}_{\bar{p}^{*}_{w}}). 
\end{aligned}
\label{eq:lyapunovonelayer}
\end{equation}
When the subscript $i$ is dropped, then we refer to quantities of the overall system. The storage function~\eqref{eq:lyapunovonelayer} has an upper and lower bound, whereas the first exists thanks to the compactness of $\mathbb{A},\mathbb{X},\mathbb{U}$ and the continuity of the cost. Given that the locational optimization cost is always greater or equal to zero the latter follows by
\begin{align}
     &J^{*}(x_{k},\mathbb{W}_{\bar{p}_w}) \geq V_{N}^{*}(x_{k},s_{k}^{*},\mathbb{W}_{\bar{p}_w}) \geq \ell^{*}(x_{k},s_{k}^{*}) \notag \\ &\geq \alpha_{1}d((x_{k},u_{x_{k},s_{k}^{*}}^{*})-s_{k}^{*})^{2} \geq 0.
\label{eq:lowerbound}
\end{align}

\subsubsection{Convergence to Setpoint} We now prove that the optimal cost at time $k+1$ on a candidate update partition $\mathbb{W}_{\bar{p}_{k}^{*}}$ decreases with respect to its value corresponding to the Voronoi partition induced by the setpoint positions $\bar{p}$ at time $w$. In other words, we show that $J^{*}(x_{k+1},\mathbb{W}_{\bar{p}_{k}^{*}}) < J^{*}(x_{k},\mathbb{W}_{\bar{p}_w})$. By definition of the involved storage function, this in turn implies that $x_k$ converges to the current steady-state $s_k^*$.\\
Recalling the definitions of $\gamma_{i}'$ and $\gamma_{i,V_{\max,i}}$ given in Proposition~\ref{proposition:upperboundsuccedingstagecost} and the proof of Proposition~\ref{proposition:upperlowerboudJtilde}, respectively, let $\gamma_{\max} := \max_{i=1,...,M}\max{\{\gamma_{i}',\gamma_{i,V_{\max,i}}\}}$ and $\gamma' = \max_{i=1,...,M}{\{\gamma_{i}'\}}$. Moreover, we will make use of the following result taken from the proof of Theorem~\ref{theorem:theo4boccia}:
\begin{equation}
\begin{split}
    &\left( \frac{\gamma_{i,V_{\max,i}}}{\gamma_{i,V_{\max,i}}-1} \right)^{N-1} \ell_{i}^{*}(\hat{x}_{N-1\vert k},s_{k}^*) \\ & \leq V_{N,i}(x_{k},s_k^*,\mathbb{W}_{\bar{p}_w,i}) \leq \gamma_{i,V_{\max,i}} \ell^{*}_{i}(x_{k},s_k^*).
\end{split}
\label{eq:prooftheom4}
\end{equation}

Next we get the following storage function decrease 
\begin{align}
    &J^{*}(x_{k+1},\mathbb{W}'_{\bar{p}_{k}^{*}}) \leq J^{*}(x_{k+1}, s^{*}_{k},\mathbb{W}'_{\bar{p}_{k}^{*}}) \label{eq:intermediateJ1} \\ & = 
    V_{N}^{*}(x_{k+1},s_{k}^{*},\mathbb{W}'_{\bar{p}_{k}^{*}}) + \lambda H(\bar{p}_{k}^{*},\mathbb{W}'_{\bar{p}_{k}^{*}}) %- \ell_{T,min} 
    \notag \\ &\overset{\eqref{eq:lemma1proposal},\eqref{eq:feasibilitycond}}{\leq} V_{N}(\hat{x}_{\cdot \vert k}, \hat{u}_{\cdot \vert k},s_{k}^{*}) + \lambda H(\bar{p}_{k}^{*},\mathbb{W}'_{\bar{p}_{k}^{*}})  %-\ell_{T,min} 
    \notag \\ &\overset{\eqref{eq:condassumption6},\eqref{eq:feasibilitycond}}{\leq} V_{N}(\hat{x}_{\cdot \vert k}, \hat{u}_{\cdot \vert k},s_{k}^{*}) + \lambda H(\bar{p}_{k}^{*},\mathbb{W}_{\bar{p}_w}) %- \ell_{T,min} 
    \notag \\
    &\overset{\eqref{eq:lemma1proposal}}{=} J^{*}(x_{k},\mathbb{W}_{\bar{p}_w}) - \ell(x_{k},u_{0\vert k}^{*},s_{k}^{*}) - \ell(x_{N-1\vert k}^{*},u_{N-1\vert k}^{*},s_{k}^{*}) \notag \\ & + \ell(\hat{x}_{N-1 \vert k},\bar{u}^{*}_{k},s_{k}^{*}) + \ell(f(\hat{x}_{N-1 \vert k},\bar{u}^{*}_{k}),\bar{u}^{*}_{k},s_{k}^{*}) \notag \\
    & \leq J^{*}(x_{k},\mathbb{W}_{\bar{p}_w}) - \ell^{*}(x_{k},s_{k}^{*}) - \ell^{*}(x_{N-1\vert k}^{*},s_{k}^{*}) \notag \\ & + \ell(\hat{x}_{N-1\vert k},\bar{u}^{*}_{k},s_{k}^{*}) + \ell(f(\hat{x}_{N-1 \vert k},\bar{u}^{*}_{k}),\bar{u}^{*}_{k},s_{k}^{*}) \notag \\
    &\overset{\eqref{eq:assumption22}}{\leq} J^{*}(x_{k},\mathbb{W}_{\bar{p}_w}) - \ell^{*}(x_{k},s_{k}^{*}) + (\gamma' -1)\ell^{*}(x_{N-1\vert k}^{*},s_{k}^{*}) \notag \\
    &\overset{\eqref{eq:prooftheom4}}{\leq} J^{*}(x_{k},\mathbb{W}_{\bar{p}_w}) - \ell^{*}(x_{k},s_{k}^{*})  \notag \\ & +(\gamma_{\max} - 1)\left( \frac{\gamma_{\max}-1}{\gamma_{\max}} \right)^{N-1} \gamma_{\max} \ell^{*}(x_{k},s_{k}^{*}). \notag
\end{align}
Thus, we obtain 
\begin{equation}
 J^{*}(x_{k+1},\mathbb{W}'_{\bar{p}_{k}^{*}}) \leq J^{*}(x_{k},\mathbb{W}_{\bar{p}_w}) - \bar{\alpha}_{N}\ell^{*}(x_{k},s_{k}^{*}),
 \label{eq:1theorem2}
\end{equation}
where $\bar{\alpha}_{N} = 1-\left( \dfrac{(\gamma_{\max}-1)}{\gamma_{\max}} \right)^{N} \gamma_{\max}^{2} > 0$ by choosing $N>N^{*}$ large enough. Thus, inequality~\eqref{eq:1theorem2} in combination with lower and upper bounds on $J^{*}$ and Assumption~\ref{assumption:boundedbyd} ensures that $\lim_{k\rightarrow \infty} \|  x_{k} - s_{k}^{*} \|  = 0$.

\subsubsection{Convergence to Centroidal Voronoi partition} Combining Proposition~\ref{proposition:notminsteadynotcoseonelayer} with the result seen in~\eqref{eq:1theorem2}, it follows that $J^{*}$ decreases as long as $x_{k} \neq s_{k}^{*}$ or $\bar{p}_{k}^{*} \notin \mathbb{T}_{\bar{p}_{k}^{*},\mathbb{W}_{\bar{p}_w,i}}$. In fact,
\begin{align}
    &J^{*}(x_{k+1}, \mathbb{W}_{\bar{p}_w}) - J^{*}(x_{k}, \mathbb{W}_{\bar{p}_w})
    \leq - \bar{\alpha}_{N}\ell^{*}(x_{k},s_{k}^{*}) \notag \\
    &\leq -\frac{\bar{\alpha}_{N}}{2}(\ell^{*}(x_{k},s_{k}^{*}) + \bar{a}\kappa(\Vert \bar{p}_{k}^{*} \Vert)_{\mathbb{T}_{\bar{p}_{k}^{*},\mathbb{W}_{\bar{p},i}}}^{2}).
\label{eq:2theorem2}
\end{align}
Accordingly, it can be concluded that $\lim_{k \to \infty}p_{k} \in \mathbb{T}_{\bar{p}_{k}^{*},\mathbb{W}_{\bar{p},i}}$. 

\subsubsection{Finite Time Partition Update} Finally, it has to be shown that equation~\eqref{eq:feasibilitycond} is fulfilled in finite time and the Voronoi partition can be updated. For this purpose, we leverage the result in~\eqref{eq:1theorem2} with respect to a fixed set of partitions $\mathbb{W}_{\bar{p}_w}$ and candidate $s_{k+1} = s_k^*$ and combine them with Assumption~\ref{assumption:expocostcontrollability}. Then, we obtain 
\begin{equation}
\begin{split}
     V_{N}^{*}(x_{k+\tilde{k}},s_{k}^{*},\mathbb{W}_{\bar{p}_w}) 
     \leq \Big(1 - \frac{\bar{\alpha}_{N}}{\gamma_{\max}} \Big)^{\tilde{k}} V_{N}^{*}(x_{k},s_{k}^{*},\mathbb{W}_{\bar{p}_w}),
\end{split}
\label{eq:4fintitetimecovergence}
\end{equation}
that converges because by definition $\gamma_{\max}> \bar{\alpha}_{N} > 0$. At this point, the same reasoning carried out in Appendix~\ref{subsection:finitetimeconditiontwolayersp2} can be used to show that condition~\eqref{eq:feasibilitycond} is met in finite time. However, we leverage Assumption~\ref{assumption:voronoisetneverempty} in combination with the fact that by construction, starting from an initially feasible configuration, the interior of the candidate partition at $\tilde{k}'$, $\bar{\mathbb{W}}_{\bar{p}_{k+\tilde{k}'}}^{\mathrm{int}}$ contains $\bar{p}_{k+\tilde{k}'}$.

\noindent In conclusion, the algorithm does converge to a configuration in which each agent's position defines a local minimum of $H(p, \mathbb{W})$ and the Voronoi partition does not update anymore: that is, a centroidal Voronoi configuration is obtained. 

\subsection{Proof of Theorem~\ref{theorem:onelayerlearning}}
\label{subsec:onelayerunknownenvironmenttheory}
For this proof we first show convergence to a steady state describing a local minimum of the locational optimization function considering the converged estimate. In a second step, an upper bound for the remaining variance at each reachable position is defined.
\subsubsection{Convergence} Similar to the proof of Theorem~\ref{theorem:onelayer}, we define the storage function as
\begin{align}
&J^{*}(x_{k}, \mathbb{W}_{\bar{p}^{*}_{w}},\hat{\phi}_{k}) \notag \\ & =  V_{N}^{*}(x_{k}, s_{k}^{*},\mathbb{W}_{\bar{p}^{*}_{w}}) + \lambda (H(\bar{p}_{k}^{*},\mathbb{W}_{\bar{p}^{*}_{w}}, \hat{\phi}_{k}) - S\text{Var}_{k}(\bar{p}_{k}^{*}))\notag \\
& = \sum_{i=1}^{M} \bigg( \sum_{l=0}^{N-1}\ell_{i}(x_{i,l\vert k}^{*}, u_{i,l \vert k}^{*}, s_{i,k}^{*}) \label{eq:lyapunovonelayerlearning} \\  &+ \lambda (H_{i}(\bar{p}_{i,k}^{*},\mathbb{W}_{\bar{p}^{*}_{w},i},\hat{\phi}_{k}) - S\text{Var}_{k}(\bar{p}_{i,k}^{*}) \bigg), \notag
\end{align}
Following the reasoning in equations~\eqref{eq:intermediateJ1} and~\eqref{eq:1theorem2} for the adapted storage function, we end up with
\begin{align}
&J^{*}(x_{k+1}, \mathbb{W}'_{\bar{p}^{*}_{k}},\hat{\phi}_{k+1}) \leq J^{*}(x_{k}, \mathbb{W}_{\bar{p}_w^*},\hat{\phi}_{k}) - \underbrace{\bar{\alpha}_{N}\ell^{*}(x_{k},s_{k}^{*})}_{\geq 0} 
\notag \\ &- \lambda\int_{\mathbb{W}_{\bar{p}_w^*}} g(\Vert q - \bar{p}^{*}_{k} \Vert)(\hat{\phi}(q)_{k}-\hat{\phi}(q)_{k+1})dq \notag \\ &- \underbrace{\lambda S (\text{Var}_{k}(\bar{p}^{*}_{k}) - \text{Var}_{k+1}(\bar{p}^{*}_{k}))}_{\geq 0}. \label{eq:lyth5}
\end{align}
Using the Bernstein-von Mises Theorem~\cite[Chapter 10.2]{vanderVaart1998} we have that $\hat{\phi}_{k}(q_k)$ converges in probability to the Gaussian distribution with mean equal to the maximum likelihood estimate, and with variance equal to $1/k$ times the asymptotic variance of the maximum likelihood estimate (i.e., the inverse of the Fisher information matrix). Using~\eqref{eq:bayesupdate}, this convergence implies that $(\hat{\theta}_{k}-\hat{\theta}_{k+1}) \xrightarrow{\mathbb{P}} 0$. Thus, the storage function decrease in~\eqref{eq:lyth5} implies $\lim_{k\rightarrow\infty}\|x_k-s_k^*\|=0$ in probability. Applying Proposition~\ref{proposition:notminsteadynotcoseonelayer} ensures convergence of the steady state to a local minimum of the target cost with respect to the current density~$\phi_k$, similarly to Equation~\eqref{eq:2theorem2}.

\subsubsection{Remaining Variance} 
Given the converged state, the following proposition provides a bound on the variance for all feasible position $p\in\mathbb{F}:=\lim_{k\rightarrow\infty}\mathbb{F}_k$.  

\begin{proposition} 
Suppose the closed loop converges to a setpoint $\overline{s}_i^*$ with state $\bar{x}_{i}^{*}$ and position $\bar{p}_{i}^{*}$. 
Then, each agent $i \in \{1,\hdots,M\}$, there exists a uniform constant $\Delta H_i\geq 0$, such that for any $p\in \mathbb{F}_i$ with corresponding $s_{p}\in\mathbb{S}_i$, the variance is bounded by
\begin{align}
    &\lim_{k \rightarrow \infty}\mathrm{Var}_{k}(p) \leq  \frac{\Delta H_i}{S}. 
\label{eq:upperboundvaronelayerlearningsimplified}
\end{align}
\label{proposition:onelayerlearning}
\end{proposition}
\vspace{-1.5em}
\noindent \emph{Proof:} 
Convergence ensures $\lim_{k\rightarrow\infty} V_N(\overline{x}^*,\overline{s}^*)=0$, $\text{Var}_\infty(\overline{p}^*)=0$ and hence the minimum cost in the MPC problem is given by $\lambda H(\overline{p}^*,\mathbb{W}_{\overline{p}^*,i},\hat{\phi}_\infty)$. 
Note that the cost of any feasible position $p\in\mathbb{F}_i$ with corresponding setpoint $s_p\in\mathbb{S}_i$ is larger than or equal to this minimum, i.e.,
\begin{align}
    &\lambda H(\overline{p}^*,\mathbb{W}_{\overline{p}^*,i},\hat{\phi}_\infty) \label{eq:contraditiononelayerlearning2}\\
    &\leq V_{N}^{*}(\bar{x}_{i}^{*}, s_{p},\mathbb{W}_{\bar{p},i}) +\lambda (H(\overline{p},\mathbb{W}_{\overline{p}^*,i},\hat{\phi}_\infty)-S\text{Var}_\infty({p})).
    \nonumber
\end{align}
Note that $V_N(\overline{x}_i^*,p)\leq V_{\max,i}$; moreover, the definition of the coverage cost $H$ in~\eqref{eq:cortescost}   
with $g$ continuous and $\mathbb{W}_{\overline{p}^*,i}\subseteq\mathbb{A}$ compact ensures that there exists a uniform constant $\Delta H_i\geq 0$ such that
\begin{align*}
    %&S\text{Var}_\infty(p)
    &\dfrac{1}{\lambda}V_{N}^{*}(\bar{x}_{i}^{*}, s_{p},\mathbb{W}_{\bar{p},i})  + H(\overline{p},\mathbb{W}_{\overline{p}^*,i},\hat{\phi}_\infty)-H(\overline{p}^*,\mathbb{W}_{\overline{p}^*,i},\hat{\phi}_\infty)\\
    & \leq \Delta H_i.
    %\label{eq:contraditiononelayerlearning2}
\end{align*}
Inequality~\eqref{eq:upperboundvaronelayerlearningsimplified} follows by applying this bound to inequality~\eqref{eq:contraditiononelayerlearning2}. 
$\blacksquare$\\

\noindent Therefore, for every $\epsilon > 0$, a scaling $S>0$ can be found such that
$\lim_{k\rightarrow\infty}\text{Var}_k(p) \leq \frac{\Delta H}{S} \leq \epsilon$, $\ \forall p \in \mathbb{F}$, whereas $\Delta H = \max_{\Delta H_i}\{\Delta H_1,\hdots,\Delta H_M\}$ 
\vspace{-3.7em}
\begin{IEEEbiography}[{\includegraphics[width=1in,height=1.25in,clip,keepaspectratio]{figures/rahel_rickenbach-min.jpg}}]{Rahel Rickenbach} is a PhD candidate at ETH Zürich. She received her bachelors degree in mechanical engineering in 2019 and her masters degree in robotics, systems and control in 2021, both from ETH Zürich. In 2022 she was awarded the SGA-Förderpreis for her master thesis. Her main research interests lie in the areas of multi agent and coverage control, as well as inverse optimal control. 
\end{IEEEbiography}


\begin{IEEEbiography}[{\includegraphics[width=1in,height=1.25in,clip,keepaspectratio]{figures/koehler-min.jpg}}]{Johannes K\"ohler} received his Master degree in Engineering Cybernetics from the University of Stuttgart, Germany, in 2017. 
In 2021, he obtained a Ph.D. in mechanical engineering, also from the University of Stuttgart,
Germany, for which he received the 2021 European Systems \& Control PhD award.
He is currently a postdoctoral researcher at the Institute for Dynamic Systems and Control (IDSC) at ETH Zürich.
His research interests are in the area of model predictive control and control and estimation for nonlinear uncertain systems. 
\end{IEEEbiography}

\begin{IEEEbiography}[{\includegraphics[width=1in,height=1.25in,clip,keepaspectratio]{figures/scampicchio.jpg}}]{Anna Scampicchio} was born in 1993. She received in 2015 the Bachelor degree in Information Engineering and in 2017 the Masters degree in Automation Engineering, both cum laude, from the University of Padova. In 2017 she was awarded with the Roberto Rocca scholarship for her career during the Masters Degree. She held a visiting position at the Department of Applied Mathematics of University of Washington, Seattle, in 2019. In 2021 she received the Ph.D. in Information Engineering from the University of Padova, and now she is a
postdoctoral researcher at the Institute for Dynamic Systems and Control, ETH Z{\"u}rich. Her research interests lie at the interplay among system identification, machine learning and control design.
\end{IEEEbiography}

\begin{IEEEbiography}[{\includegraphics[width=1in,height=1.25in,clip,keepaspectratio]{figures/melanie_zeilinger.jpg}}]{Melanie N. Zeilinger} 
is an Associate Professor at ETH Zurich, Switzerland. She received the Diploma degree in engineering cybernetics from the University of Stuttgart, Germany, in 2006, and the Ph.D. degree with honors in electrical engineering from ETH Zurich, Switzerland, in 2011. From 2011 to 2012 she was a Postdoctoral Fellow with the Ecole Polytechnique Federale de Lausanne (EPFL), Switzerland. She was a Marie Curie Fellow and Postdoctoral Researcher with the Max Planck Institute for Intelligent Systems, Tübingen, Germany until 2015 and with the Department of Electrical Engineering and Computer Sciences at the University of California at Berkeley, CA, USA, from 2012 to 2014. From 2018 to 2019 she was a professor at the University of Freiburg, Germany. Her current research interests include safe learning-based control, as well as distributed control and optimization, with applications to robotics and human-in-the-loop control.
\end{IEEEbiography}


\begin{IEEEbiography}[{\includegraphics[width=1in,height=1.25in,clip,keepaspectratio]{figures/andrea_carron-min.jpg}}]{Andrea Carron} is a Senior Lecturer at ETH Zürich. He received the bachelor’s, master’s, and Ph.D. degrees in control engineering from the University of Padova, Padova, Italy. During his master and Ph.D. studies, he spent three stays abroad as a Visiting Researcher: the first at the University of California at Riverside, Riverside, CA, USA, the second at the Max Planck Institute, Tubingen, Germany, and the third at the University of California at Santa Barbara, Santa Barbara, CA, USA. From 2016 to 2019, he was a Post-Doctoral Fellow with the Intelligent Control Systems Group, ETH Zürich, Zürich, Switzerland. His is research interests include safe-learning, learning-based control, multiagent systems, and robotics.
\end{IEEEbiography}


\end{document}