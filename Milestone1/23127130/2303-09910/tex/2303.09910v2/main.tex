\documentclass[journal,twoside,web]{ieeecolor}
%\pdfminorversion=4
\pdfoutput=1
\usepackage{generic}
\usepackage[noadjust]{cite}
\let\proof\relax
\let\endproof\relax
\usepackage{amsmath,amssymb,amsfonts,amsthm}
\usepackage{algorithm}
\usepackage[algo2e]{algorithm2e}
\usepackage{algorithmic}
\usepackage{wasysym,xcolor}
\usepackage{hyperref}
\usepackage{mathrsfs}
\usepackage{graphicx,float}
\usepackage{capt-of}
\usepackage{textcomp,soul,comment}
\usepackage{pgfplots}

\usepackage{caption}
\captionsetup{size=footnotesize,
	skip=5pt, position = bottom}

 % Fix for cite + hyperref
\makeatletter
\let\NAT@parse\undefined
\makeatother
\usepackage{hyperref}

\pgfplotsset{compat=1.10}
 \def\BibTeX{{\rm B\kern-.05em{\sc i\kern-.025em b}\kern-.08em
    T\kern-.1667em\lower.7ex\hbox{E}\kern-.125emX}}
% \markboth{\journalname, VOL. XX, NO. XX, XXXX 2017}
% {Author \MakeLowercase{\textit{et al.}}: Preparation of Papers for IEEE TRANSACTIONS and JOURNALS (February 2017)}
% for copyright notice
\usepackage{fancyhdr}
\usepackage{xcolor}

\def\doi{10.1109/TAC.2024.3365569}
\fancyhf{}
\renewcommand{\headrulewidth}{0pt}
\fancyfoot[c]{}
\fancyfootoffset{+0.0em}

\fancypagestyle{copyright}{
    \vspace{-2.0em}
	\cfoot{\vspace{0.0em}\footnotesize\textcopyright{} 2024 IEEE. Personal use of this material is permitted.  Permission from IEEE must be obtained for all other uses, in any current or future media, including reprinting/republishing this material for advertising or promotional purposes, creating new collective works, for resale or redistribution to servers or lists, or reuse of any copyrighted component of this work in other works.}
	%
	\chead{\footnotesize This article has been accepted for publication in IEEE Transactions on Automatic Control. This is the author's version which has not been fully edited and content may change prior to final publication. Citation information: \href{http://dx.doi.org/\doi}{DOI \doi}}
}

\newcommand{\anna}[1]{{\color{black} #1}}
\newcommand{\rahel}[1]{{\color{black} #1}}
\newcommand{\rrahel}[1]{{\color{black} #1}}
\newcommand{\JK}[1]{{\color{black} #1}}

% for look up table
%\usepackage{vmargin}               
%\usepackage{palatino}
\renewcommand{\baselinestretch}{1.05}
\parskip = \medskipamount
                 
\usepackage{amssymb,url,amsmath,mathrsfs}
%\usepackage[square,sort,comma,numbers]{natbib}

%\usepackage{graphicx,multicol,subfigure,enumitem}% for pdf, bitmapped graphics files
\usepackage{epsfig,adjustbox} % for postscript 
\usepackage{times} % assumes new font selection scheme installed
\usepackage{euscript}
\usepackage{xcolor}
\usepackage{xr-hyper}
\usepackage{hhline,multirow,makecell}
\usepackage{color}
\usepackage{comment,soul}
\usepackage{caption}

\captionsetup{tablename=Tab.}

\newcommand{\notn}[2]{& {#1} \quad \quad && \text{#2}\\}

% def
\def\R{{\Bbb R}}
\def\N{{\Bbb N}}
\def\P{{\Bbb P}}
\def\E{{\Bbb E}}

%%%%%%%%%%%%%%%%%%%%%%%%%%%%%%%%%%%%%%%%%%%%%%%%

\newtheorem{remark}{Remark}

\newtheorem{definition}{Definition}
\newtheorem{theorem}{Theorem}
\newtheorem{proposition}{Proposition}
\newtheorem{assumption}{Assumption}
\newtheorem{lemma}{Lemma}


\allowdisplaybreaks

\begin{document}
\title{Active Learning-based \\
Model Predictive Coverage Control}
\author{Rahel Rickenbach, Johannes K\"ohler, Anna Scampicchio, Melanie N. Zeilinger, Andrea Carron
\thanks{
An accompanying video detailing the contributions of the paper is available at the following link: \url{https://youtu.be/3u-JZxx3L3M}. \\
All authors are members within the Institute for Dynamic Systems and Control (IDSC), ETH Z{\"u}rich.
{\tt\footnotesize [rrahel|jkoehle|ascampicc|\\mzeilinger|carrona]@ethz.ch}}}

\maketitle
\thispagestyle{copyright}
\pagestyle{empty}

\begin{abstract}
The problem of coverage control, i.e., of coordinating multiple agents to optimally cover an area, arises in various applications. 
However, coverage applications face two major challenges: (1) dealing with nonlinear dynamics while respecting system and safety critical constraints, and (2) performing the task in an initially unknown environment. We solve the coverage problem by using a hierarchical framework, in which references are calculated at a central server and passed to the agents' local model predictive control (MPC) tracking schemes.
Furthermore, to ensure that the environment is actively explored by the agents a probabilistic exploration-exploitation trade-off is deployed. In addition, we derive a control framework that avoids the hierarchical structure by integrating the reference optimization in the MPC formulation.  
Active learning is then performed drawing inspiration from Upper Confidence Bound (UCB) approaches. For all developed control architectures, we guarantee \rahel{closed-loop constraint satisfaction} and convergence to an optimal configuration. Furthermore, all methods are tested and compared on hardware using a miniature car platform.
\end{abstract}

\begin{IEEEkeywords}
Coverage Control, NL Predictive Control, Cooperative Control, Machine Learning, Agents and Autonomous Systems
\end{IEEEkeywords}

\section{Introduction}
\label{sec:introduction}
 
A key problem in multi-agent systems consists in optimally covering a finite area with respect to environmental demands, which  
are reflected by a density function of measurable values of interest. 
This task goes under the name of \textit{coverage control}, and it can be rephrased as placing the agents at the centroids of their Voronoi partition induced by the density function~\cite{Du1999}. Applications are versatile. One is, e.g.,  
the autonomous re-positioning of self-driving taxis according to the population density, providing a faster and more environmentally friendly service. Another example, illustrated in Figure~\ref{fig:explanatoryfigure}, are firefighting planes that are autonomously and optimally distributing themselves with respect to the heat map of a certain area.
From these examples, the challenges that emerge are twofold and intertwined. The first is designing a coverage control architecture for agents with nonlinear dynamics while ensuring collision avoidance and respecting safety constraints. The second consists in dealing with initially unknown environments, i.e., unknown density functions characterizing the optimal coverage problem. In this case, the considered area needs to be explored during the process, relying on the agents' sensing capabilities. 
\begin{figure} [t]
\centering
\includegraphics[width=0.45\textwidth]{figures/coverage_control_explained_9-compressed.pdf}
\caption{Illustration of coverage control problem and partitioning of environment at the example of firefighting planes and environmental demands defined by the resulting heat map.}
\label{fig:explanatoryfigure}
\vspace{-1.2em}
\end{figure} 
The goal of this paper is to design a coverage control framework that addresses both of the aforementioned challenges. 

\subsubsection*{Related Work}
\label{subsubsec:relatedwork}
In classical works such as~\cite{Cortes2004,Bullo2012}, the coverage control problem is rigorously solved under the assumptions that the environment is perfectly known and that dynamics are single integrators. Nonlinear dynamics and state/input constraints can be taken into account by using a  model predictive control (MPC) scheme~\cite{Grune2011,rawlings2017model}, as pursued in~\cite{Carron2017,Kohler2018,Farina2015}. In all the aforementioned references, the implemented coverage algorithm is hierarchical, i.e., a reference is calculated before being passed to a tracking MPC~\cite{Limon2018}. Moreover, they all require some non-trivial offline design for the terminal ingredients in MPC to prove convergence and recursive feasibility. Furthermore, none of them addressed coverage control in an unknown environment.\\
To cope with this second challenge, ``exploitation" targeted to the coverage control task needs to be combined with ``exploration" given by data collection performed via active learning~\cite{Li2006,Campbell1990} to improve the estimate of the initially unknown density function. The problem of coverage control in an unknown environment is investigated in~\rahel{\cite{Schwager2009,Todescato2017,McDonald2021,prajapat2022,le2012}}, where different strategies to balance exploration and exploitation are proposed. 
However, in these works the agents' dynamics are once again assumed to be single integrators.
The general coverage control problem encompassing an unknown environment and nonlinear constrained dynamics, \rahel{while ensuring persistent collision avoidance,} has not yet been addressed in the literature. 

\begin{figure} [t]
\centering
\includegraphics[trim={0cm 0cm 0.8cm 0cm},clip,width=0.45\textwidth]{figures/coverage_clarification_diagram_vert-compressed.pdf}
\caption{Simplified illustration of the developed two- and one-layers algorithm. Considering density $\phi$, partitions $\mathbb{O}$, references $r$, state $x$, inputs $u$, positions $p$, as well as setpoint positions $\bar{p}$.} 
\label{fig:coverageclarification}
\vspace{-1.5em}
\end{figure}

\subsubsection*{Contributions}
\label{subsubsec:contributions}

In \rahel{presence of complex agents' dynamics and in need of environment exploration, ensuring collision avoidance when agents change their configuration becomes a key challenge in coverage control. To tackle this problem, we propose a tracking MPC-based coverage framework, which (a) includes safety-ensuring constraints, and (b) avoids the difficult design of terminal constraints needed, e.g., in \cite{Carron2017}, by leveraging the tools proposed in ~\cite{Grune2011,Soloperto2021}.
Specifically, we propose and analyze two strategies. The first, visualized on the left of Figure~\ref{fig:coverageclarification}, extends \cite{Carron2017} by including collision avoidance constraints. Its rationale consists in passing references, calculated by the server, to each agent's individual tracking MPC: accordingly, we will refer to this as the ``two-layers approach". The second architecture, summarized on the right of Figure ~\ref{fig:coverageclarification}, overcomes the hierarchical structure by directly integrating the calculation of the next optimal configuration into the MPC cost of each agent; for this reason, it will be referred to as the ``one-layer approach". To the best of the authors' knowledge, this strategy is novel in the literature. Furthermore, we complement both methods with active learning strategies: for the two-layers approach, we leverage the probabilistic exploration-exploitation decision
proposed in~\cite{Todescato2017}, while for the one-layer we draw inspiration from Bayesian optimization methods~\cite{Auer2002,Srinivas2010} and use an Upper Confidence Bound (UCB) approach. \\ Both control architectures are based on an MPC problem, which is challenged by the time-varying nature of its safety-ensuring constraints. In fact, as the latter change with respect to the agents' configuration, recursive feasibility, closed-loop constraint satisfaction and convergence to a locally optimal configuration require special investigation. Rigorous proofs of those two key properties are the main theoretical contributions of this paper, carried out for the two proposed control strategies and in both cases of known and initially unknown environment.
Lastly, we test and compare all the proposed approaches on hardware using the miniature racing cars Chronos in combination with CRS, an open-source software framework for control and robotics~\cite{carron2022}.}


\subsubsection*{Outline}
\label{subsubsec:roadmap}
The remainder of the paper is structured as follows. Section~\ref{sec:problemFormulation} states the coverage control problem, followed by a presentation of preliminaries regarding nonlinear tracking MPC and Bayesian learning in Section~\ref{sec:controlandlearningframework}. \rahel{Section \ref{sec:colavoidanceandrecursivefeasibility} introduces a strategy to ensure safety and recursive feasibility under changing Voronoi partitions.} Sections~\ref{sec:twolayers} and~\ref{sec:onelayer} detail the proposed two-layers and one-layer methods, respectively. In addition, we state the main theoretical results on convergence, satisfaction of safety-critical constraints and recursive feasibility for both set-ups with known and unknown environment. The proofs are deferred to the Appendix. Section~\ref{sec:experiments} gathers all the experimental results, and Section~\ref{sec:conclusions} ends the paper by discussing the benefits of the proposed approaches and drawing conclusions.
 
\subsubsection*{Notation}
\label{subsubsec:notation}
Throughout the paper, $\Vert \! \cdot \! \Vert$ indicates the Euclidean norm and $\mathcal{N}$ the Gaussian distribution. The set of all non-negative real numbers is given by $\mathbb{R}_{+}$ and the set of natural numbers by $\mathbb{N}$. We define the ball $\mathbb{B}_{\rahel{o}}^{b} = \{x \in \mathbb{R}^b\vert \,\Vert x \Vert\leq \rahel{o} \}$ and indicate with $\ominus$ the Pontryagin set difference \cite[Sec.~3.1]{kouvaritakis2016}. Considering some arbitrary small but fixed constant $\epsilon > 0$, for each set $\mathbb{V}\subset\mathbb{R}^b$ we denote $\mathbb{V}^{\mathrm{int}}:= \mathbb{V} \ominus \mathbb{B}_{\epsilon}^{b}$. Given a compact set $\mathbb{D}\subset\mathbb{R}^b$ and a  continuous function $f: \mathbb{D} \rightarrow \mathbb{R}$, we define
\begin{equation}
    f(\rahel{y})_{\mathbb{D}} := \min_{\tilde{\rahel{y}} \in \mathbb{D}}f(\rahel{y} - \tilde{\rahel{y}}).
    \label{eq:minsetdistance}
\end{equation}

\section{Problem Formulation}
\label{sec:problemFormulation} 
This section states the coverage control problem, together with the specification of the nonlinear constrained dynamics that are considered in the paper.

\subsection{Dynamics and Constraints}
We consider a \rrahel{polytopic} area \mbox{$\mathbb{A}\in \mathbb{R}^{D}$}, on which $M$ agents move according to time-invariant, discrete-time, and nonlinear dynamics. For agent $i$, these are
\begin{equation}
\begin{split}
    &x_{i,k+1} = f_{i}(x_{i,k}, u_{i,k}) \\
    &p_{i,k} = C_{i}x_{i,k},
\end{split}
\label{eq:nonlineardynamics}
\end{equation}
where $x_{i,k} \in \mathbb{R}^{n_{i}}$ and $u_{i,k} \in \mathbb{R}^{m_{i}}$ denote the state and input at time $k$, respectively. The vector $p_{i,k} \in \mathbb{R}^D$ indicates the position of agent $i$ at time $k$, which is usually chosen as Cartesian coordinates in $D$=2 or $D$=3. Additionally, all agents are subject to state and input constraints, i.e., $x_{i,k} \in \mathbb{X}_{i}$ and $u_{i,k} \in \mathbb{U}_{i}$ for $k \in \mathbb{N}$. We consider relatively general heterogeneous dynamics for the different agents and make the subsequent assumption.
\begin{assumption} The agent's dynamics $f_{i}$ are assumed to be known and Lipschitz continuous with Lipschitz constant $\mathcal{L}_{i}$. Moreover, each agent's state can be measured perfectly and its state and input constraints are compact.
\label{assumption:dynamics}
\end{assumption}

\subsection{Coverage Control}
\label{subsec:coverage}
We indicate with \mbox{$p = [p_{1}, \hdots, p_{M}]^{T} \in \mathbb{R}^{M \times D}$} the agents' position, and denote with \mbox{$\mathbb{O} = \{\mathbb{O}_{1}, \hdots, \mathbb{O}_{M}\}$} an arbitrary collection of polytopes partitioning $\mathbb{A}$. Accordingly, the task of coverage control can be mathematically stated as the following optimization problem:
\begin{equation}
\begin{split}
    \min_{p,\,\mathbb{O}} \:\underbrace{\sum_{i=1}^{M} \int_{\mathbb{O}_{i}} \Vert q - p _{i} \Vert^{2}\phi(q)dq}_{H(p,\mathbb{O})},\, \ p_{i} \in \mathbb{O}_{i},
\end{split}
\label{eq:cortescost}
\end{equation}
where $\phi: \mathbb{A} \rightarrow \mathbb{R}_+$ is the so-called \textit{density function}, which acts as a measure of information on the environment~$\mathbb{A}$, \rahel{the squared Euclidean norm represents} the agents' \textit{sensing capability}, and $H(p, \mathbb{O})$ is known as the \textit{locational optimization cost}. It was shown in~\cite{Du1999} that the optimal set  $\mathbb{O}$ in~\eqref{eq:cortescost} is given by the \textit{Voronoi tessellation}~\cite{Senechal1995} \rahel{with respect to a generating position configuration $p$, denoted by \mbox{$\mathbb{W}_{p} = \{\mathbb{W}_{p,1}, \hdots, \mathbb{W}_{p,M}\}$}}, with
\begin{equation}
\begin{split}
    \mathbb{W}_{p,i} = \{ q \in \mathbb{A} \; \vert\; \Vert q - p_{i} \Vert \leq \Vert q - p_{j} \Vert, \, \forall j \neq i \}.
\end{split}
\label{eq:voronoidef}
\end{equation}
\rahel{Note that all the sets in $\mathbb{W}_{p}$ are convex, since $\mathbb{A}$ is convex~\cite{Du1999}.}
Each agent's optimal position is given by the \textit{centroids} $c(\mathbb{W}_{p}, \phi) = [c_1(\mathbb{W}_{p,1}, \phi), \hdots, c_M(\mathbb{W}_{p,M}, \phi)]$, with
\begin{equation}
\begin{split}
     c_i(\mathbb{W}_{p,i}, \phi) = \left(\int_{\mathbb{W}_{p,i}}\phi(q)dq\right)^{-1} \left(\int_{\mathbb{W}_{p,i}}q\phi(q)dq\right).
\end{split}
\label{eq:voronoicenterdef}
\end{equation}
\rahel{This locally} optimal solution in terms of both partition and positions is called \rrahel{a} \textit{centroidal Voronoi configuration} and the coverage control problem can hence be reduced to ensuring $p$ converges to $c(\mathbb{W}_{p}, \phi)$ with $\mathbb{W}_{p}$ according to~\eqref{eq:voronoidef}. Classically, for integrator dynamics this is achieved using the Lloyd algorithm~\cite{Lloyd1982}, which consists in an iterative update of the form $p_{i, k+1} = c(\mathbb{W}_{p_{k},i}, \phi)$. We study \rahel{this coverage problem while accounting for} more general nonlinear dynamics, collision avoidance constraints, and both for known environments, i.e., with known density~$\phi$, and in the case in which $\phi$ is unknown and needs to be learned online.

\begin{remark}\label{rmk:g}
    \rahel{The following exposition can be extended to consider a coverage cost \eqref{eq:cortescost} with a more general sensing capability \mbox{$g(\Vert q-p_i \Vert)$: $\mathbb{R}_{+} \rightarrow \mathbb{R}_{+}$} instead of the squared distance~\cite{Cortes2004}. In this case, the optimal positions are minimizers of $H(p, \mathbb{O})$ and have no closed-form solution.}
\end{remark}

\section{Control and Learning Framework}
\label{sec:controlandlearningframework}
In this section, we introduce the core control and learning tools of the developed methods that allow for a safe coverage movement. 
Therefore, we recall a nonlinear tracking MPC formulation and complement such a set-up with \rahel{position constraints }(Section~\ref{subsec:nonlineartrackingmpc}). Further, we define the Bayesian linear regression strategy that will be deployed in all learning-based approaches dealing with an unknown environment (Section~\ref{subsec:measurementcol}).

\subsection{Nonlinear Tracking MPC}
\label{subsec:nonlineartrackingmpc}

\rahel{For each agent, consider the problem of reaching an external reference $r_i \in \mathbb{P}_i$ while satisfying state, input, and position constraints, indicated with $x_i \in \mathbb{X}_i$, $u_i \in \mathbb{U}_i$, and $p_i \in \mathbb{P}_i$ respectively. A way of accomplishing this task consists in resorting to a nonlinear tracking MPC~\cite{Limon2018,Soloperto2021}, which we now review.}\\  
\rrahel{We introduce the concept of agent-specific \textit{artificial setpoints}, $s_i = (\bar{x}_{i},\bar{u}_{i})$, which should lie in the following set that also depends on the position constraint $\mathbb{P}_i$}
\begin{equation}
\begin{split}
    & \mathbb{S}_{\mathbb{P},i} \! = \! \{(x,u) \! \in \! \mathbb{R}^{n_{i}+m_{i}}\,\vert\, \\  & C_{i}x \in \mathbb{P}_{i}^{\mathrm{int}}, x \! \in \mathbb{X}^{\mathrm{int}}_{i}, u \! \in \mathbb{U}^{\mathrm{int}}_{i}, x \! = \! f_{i}(x,u) \}.
\end{split}
\label{eq:steadystatesetwithcollavoidance}
\end{equation}
\rahel{The steady-state position corresponding to $s_i$ is obtained by \mbox{$\bar{p}_{i} = C_{i}\bar{x}_{i}$}, which lies in the projected position space
\begin{equation}
\begin{split}
    \mathbb{S}_{\mathbb{P},i}^{\mathrm{p}} = & \{p \in \mathbb{P}_{i}^{\mathrm{int}} \;\vert \; p = C_{i}x: \exists (x,u) \text{ s.t. } (x,u) \in \mathbb{S}_{\mathbb{P}_{i}}\} \notag.
\end{split}
\label{eq:steadystatesetwithcollavoidancepositionspace}
\end{equation} Following the the MPC notation in~\cite{kouvaritakis2016}, we let $x_{i,l\vert k}$ indicate the predicted state of agent $i$ at prediction step $l$ for a controller applied at time step $k$, and use the symbol ``$\cdot$" to denote all values of the index $l=\{0,\dots,N-1\}$. The nonlinear tracking MPC cost then reads as}
\begin{equation}
\begin{split}
&J_{i}(x_{i,\cdot \vert k}, u_{i,\cdot \vert k}, s_{i,k}, r_{i,k}) \\ & = V_{N,i}(x_{i,\cdot \vert k},u_{i,\cdot \vert k},s_{i,k}) + \ell_{T,i}(\bar{p}_{i,k} - r_{i,k} )\\
& = \sum_{l=0}^{N-1}\ell_{i}(x_{i,l\vert k}, u_{i,l\vert k}, s_{i,k}) + \ell_{T,i}(\bar{p}_{i,k} - r_{i,k} ).
\end{split}
\label{eq:generalnonlineartrackingmpccost}
\end{equation}
Similar to standard MPC~\cite{kouvaritakis2016}, the \textit{tracking cost}~$V_{N,i}$ sums a continuous \textit{stage cost} $\ell_i$ over a finite horizon~$N$, and steers the system to the \rahel{artificial setpoint~$s_{i,k}$. It enables the tracking of piece-wise constant references~$r_{i,k}$, whose difference to their steady-state position $\bar{p}_{i,k}$ is penalized in the continuous \textit{target cost} $\ell_{T,i}$.}
For the $i$-th agent at time $k$, current state~$x_{i,k}$, \rahel{reference~$r_{i,k}$, and position constraint $\mathbb{P}_i$ the resulting MPC problem 
reads as}
\begin{subequations}
\begin{align}
\min_{x_{i,\cdot \vert k}, u_{i,\cdot \vert k},s_{i,k}}&J_{i}(x_{i,\cdot \vert k}, u_{i,\cdot \vert k}, s_{i,k}, r_{i,k}) \label{eq:generalcostwithcolavoidance} \\ &\qquad x_{i,0\vert k} = x_{i,k} \label{eq:generalinit}\\
&\qquad x_{i, l+1\vert k} = f_{i}(x_{i,l\vert k}, u_{i,l\vert k}) \label{eq:generaldyn} \\
&\qquad x_{i,\cdot\vert k} \in \mathbb{X}_{i},\; u_{i,\cdot\vert k} \in \mathbb{U}_{i}  \label{eq:generalstateconst}\\
&\qquad V_{N,i}(x_{i,\cdot \vert k},u_{i,\cdot \vert k}, s_{i,k}) \leq V_{\max,i} \label{eq:generalvarbound}\\
&\qquad p_{i,l\vert k} = C_{i}x_{i,l\vert k} \in \mathbb{P}_{i} \label{eq:generalvoronoiconst}\\
&\qquad s_{i,k} \in \mathbb{S}_{\mathbb{P},i}. \label{eq:generalsteadystatesetconstwithcolavoidance} \\
& \qquad l = 1, \hdots, N-1 \label{eq:1toN-1withcolavoidance}.
\end{align}
\label{eq:nonlineartrackingmpcwithcolavoidance}
\end{subequations}
\noindent \rahel{Its construction follows closely the one of a nonlinear tracking MPC as in~\cite{Limon2018,Soloperto2021}. Therefore, constraint~\eqref{eq:generalvarbound} exploits a user-chosen constant~$V_{\max,i}>0$ to ensure that the agent stays in a region of attraction around the setpoint $s_i$ without the need for utilizing a terminal constraint; see \cite{Soloperto2021,Boccia2014} for further details.} \\
For each agent, a solution of~\eqref{eq:nonlineartrackingmpcwithcolavoidance} exists given continuity of $\ell_{i},\ell_{T,i}$, and compact constraints. It consists of the optimal state and input trajectories, as well as the optimal setpoint, indicated with ~$x_{i,\cdot \vert k}^{*}$, $u_{i,\cdot \vert k}^{*}$, and \mbox{$s_{i,k}^{*} = (\bar{x}_{i, k}^{*}$, $\bar{u}_{i,k}^{*})$}, respectively. We denote by \rahel{$V_{N,i}^{*}(x_{i,k},s_{i,k}, \rahel{\mathbb{P}_{i}}) = \min_{u_{i,\cdot \vert k}, \in \mathbb{U}_{i},x_{i,\cdot \vert k} \in \mathbb{X}_{i}, C_ix_{i,\cdot \vert k} \in \mathbb{P}_{i}}V_{N,i}(x_{i,\cdot \vert k},u_{i,\cdot \vert k}, s_{i,k})$} the optimal tracking cost for the considered \rahel{position constraint and an arbitrary setpoint}. Furthermore, we denote by $\ell^*_i(x_k,\rahel{s_k}) = \min_{u \in \mathbb{U}_{i}} \ell_{i}(x_{i,k},u,s_{i,k})$ the one-step optimal stage cost
%, and by $u^*_{x_{i,k},\rahel{s_{i,k}}}$ its 
whose minimizing input is assumed to be equal to $\bar{u}_{i,k}$. The actual control problem is then solved in a receding horizon fashion, i.e., at every time $k$ we solve~\eqref{eq:nonlineartrackingmpcwithcolavoidance} and apply the first input of the obtained input sequence, that is $u_{i,k}=u^*_{i,0|k}$. \\
In the following, we discuss the stability properties of the MPC scheme~\eqref{eq:nonlineartrackingmpcwithcolavoidance} using the theory in~\cite{Boccia2014}. To this end, we require a local stabilizability condition, also known as \textit{exponential cost controllability} (\cite[Assumption 1]{Boccia2014}) in the literature.
\begin{assumption} For every agent $i \in \{1,\hdots,M\}$ there exist constants $\rahel{\chi_{i}}, \gamma_{i} > 0$ \rahel{such that, for any 
feasible artificial setpoint $s \in \mathbb{S}_{\mathbb{P},i}$, any horizon length $N \in \mathbb{N}$ and any state $x \in \mathbb{R}^{n_{i}}$ with a stage cost $\ell_{i}^{*}(x,s) \leq \rahel{\chi_{i}}$}, we have
\begin{equation}
    V_{N,i}^{*}(x,s) \leq \gamma_{i} \ell_{i}^{*}(x,s). 
\label{eq:assumption2gammavmax}
\end{equation}
\label{assumption:expocostcontrollability}
\end{assumption}
\vspace{-1.4em}
\noindent In addition, we require that the stage cost $\ell_{i}$ is positive definite w.r.t. the setpoint $s_{i}$, which is often achieved using a quadratic stage cost $\ell_{i}$. However, in~\cite{Muller2017}, it was shown that Assumption~\ref{assumption:expocostcontrollability} cannot be satisfied for non-holonomic systems using a quadratic stage cost. To avoid this issue for experiments with non-holonomic robots (cf. Section~\ref{sec:experiments}), we use the results in~\cite{Rosenfelder2021, Coron2020, Worthmann2016} to devise a polynomial stage cost and introduce the following distance function with even exponents $\eta_{i,j} \in \mathbb{N}, j = 1,\hdots,n_i+m_i$:
\begin{equation}
d_i(\zeta) = \sqrt{\zeta_1^{\eta_{i,1}} + \zeta_2^{\eta_{i,2}} + \cdots + \zeta_{n_i+m_i}^{\eta_{i,n_i+m_i}}}.\label{eq:distance}
\end{equation}
\rahel{In accordance to it, we present Assumption~\ref{assumption:boundedbyd}, which is similar to Assumption 1 in~\cite{Soloperto2021}}.
\begin{assumption} For every agent $i \in \{1,\hdots,M\}$ there exist constants $\alpha_{1,i}, \alpha_{2,i} > 0$ such that, for all \rahel{states} $x \in \mathbb{X}_{i}$, \rahel{artifical setpoints} $s \in \mathbb{S}_{\mathbb{P},i}$, and their respective \rahel{stage cost minimizing input} $\bar{u} \in \mathbb{U}_{i}$
, the following holds: 
\begin{equation}
\begin{split}
    \alpha_{1,i}d_{i}((x,\bar{u}) \! - \! s)^{2} \! \leq  \! \ell_{i}^{*}(x,s) \! \leq \! \alpha_{2,i}d_{i}((x,\bar{u}) \! - \! s)^{2}.
\end{split}
\label{eq:boundstagecosttwolayer}
\end{equation}
Moreover, there exist constants $\xi_{1,i}, \xi_{2,i} \geq 0$ such that for any two admissible setpoints \mbox{$s_{1} \! = \! (\bar{x}_{1},\bar{u}_{1})$} and \mbox{$s_{2} \! = \! (\bar{x}_{2},\bar{u}_{2})$ $\in \mathbb{S}_{\mathbb{P},i}$,}
\begin{equation}
\begin{split}
    \ell_{i}(x,u,s_{1}) \leq  \xi_{1,i}\cdot \ell_{i}(x,u,s_{2}) + \xi_{2,i}\cdot d_{i}(s_{1}-s_{2})^{2}.
\end{split}
\label{eq:boundtwosteadystatestwolayer}
\end{equation}
\label{assumption:boundedbyd}
\end{assumption}
\vspace{-1.5em}
\begin{remark}\label{remark:commentAssumption23}
\rahel{Assumptions \ref{assumption:expocostcontrollability} and \ref{assumption:boundedbyd} hold, e.g., with a quadratic stage cost $\ell_i$ and a standard norm $d_{i}(\zeta) = \Vert \zeta \Vert$ if the linearized dynamics are stabilizable~\cite[Section A]{Soloperto2021}.} 
\label{remark:remarkgammamax}
\end{remark}
\vspace{-0.0em}
\noindent Given these requirements, it is proven in~\cite{Boccia2014,Soloperto2021} that, for a sufficiently long prediction horizon, the tracking cost is a valid Lyapunov function with respect to a fixed steady state.
\begin{theorem}\cite[Theorem 4]{Boccia2014} Let Assumptions~\ref{assumption:expocostcontrollability} and~\ref{assumption:boundedbyd} hold. Then, for any \rahel{position constraint set, $\mathbb{P}_{i} \subset \mathbb{A}$, with $\mathbb{P}^{\mathrm{int}}_{i} \neq \emptyset$,} any \rahel{choice of} $V_{\max,i} > 0$ and any $\bar{\alpha}_{N,i} \in (0,1)$, there exists a horizon \rahel{length} $N^{*} \in \mathbb{N}_{\geq 0}$ such that, for all $k \in \mathbb{N}$, $N \geq N^{*}$, $s_k \in \mathbb{S}_{\mathbb{P},i}$ and all $x_k$ with $V_{N,i}^{*}(x_{k},s_k) \leq V_{\max,i}$, 
\begin{equation}
\begin{split}
    &V_{N,i}^{*}(f_i(x_{k},u_{0\vert k}^{*}),s_k^*,\mathbb{P}_i) - V_{N,i}^{*}(x_{k},s_k^*,\mathbb{P}_i) \\ & \leq  - \bar{\alpha}_{N,i} \ell_{i}^{*}(x_{k},s_k^*).
\end{split}
\label{eq:theorem5twolayer}
\end{equation}
\label{theorem:theo4boccia}
\end{theorem}
\vspace{-1.8em}

\subsection{Data Collection and Bayesian Linear Regression}
\label{subsec:measurementcol}
In an unknown environment, the agents are assumed to be equipped with sensors and able to take a noisy measurement of the density $\phi$ at their current position. Note that measurements do not necessarily have to be taken at each time step, but their collection can be, e.g., conditioned on reaching a predefined location, or triggered after a fixed number of time steps. Indexing by $h$ the number of collected data \rahel{per agent}, and assuming that the noise is independent, identically distributed (i.i.d.) and zero mean Gaussian with variance $\sigma^2$, the measurements model for the $i-$th agent is
\begin{equation}
   m_{i,h} = \phi(p_{i,h}) + \nu_{i,h},\; \text{with } \nu_{i,h} \sim \mathcal{N} (0,\sigma^{2}). 
   \label{eq:measmod}
   \vspace{-0.3em}
\end{equation}
\rahel{The data locally collected by the agents are then communicated to a server, which then estimates $\phi$ in a centralized fashion. Further, we define the 
data-set collected by all $M$ agents after $t$ measurement steps as}
\begin{equation}
\begin{split}
    I_{t} = \{ (p_{i,h},m_{i,h})\: \vert \: i=1,\cdots, M,\; h=1, \cdots, t\}.
\end{split}
\label{eq:setofmeasurements}
\vspace{-0.3em}
\end{equation}
In this set-up, we make use of the following assumption\rahel{, also used similar in, e.g.,~\cite{Schwager2009}.}
\begin{assumption}
The density function $\phi$ is represented by a linear combination of $\upsilon$ known, Lipschitz continuous features collected in a vector $\Phi: \mathbb{A} \rightarrow \mathbb{R}^{1 \times \upsilon}$, i.e., there exists an unknown parameter vector  $\theta \in \mathbb{R}^{\upsilon}$, s.t., $\phi(p) = \Phi(p)\theta$, $\forall p\in\mathbb{A}$. Furthermore, the features are linearly independent on the partition $\mathbb{W}_{p,i}$ in the sense that $\max_{p\in\mathbb{W}_{p,i}}\Phi(p)\Sigma\Phi(p)^\top\geq c_\Phi\|\Sigma\|$ for any positive semi-definite matrix $\Sigma\in\mathbb{R}^{\upsilon\times \upsilon}$ and some constant $c_\Phi>0$.
\label{assumption:bayesianconvergence}
\end{assumption}
To estimate the unknown vector $\theta$ from collected data we solve the problem within the Bayesian linear regression framework~\cite{Sarkka2013}. To this aim, we endow the unknown vector with a Gaussian prior, such that \mbox{$\theta \sim \mathcal{N}(\mu_0,\Sigma_0)$}, and adopt a recursive scheme to update mean and covariance matrix in view of new data. Specifically, denoting by $\bar{\Phi}_{t+1} = [\Phi(p_{1,t+1})^{\top} \: \cdots \: \Phi(p_{M,t+1})^{\top}]^{\top} \in \mathbb{R}^{M \times \upsilon}$ the matrix of features measured at $t+1$, and by $\bar{m}_{t+1} = [m_{1,t+1} \: \cdots \: m_{M,t+1}]^{\top} \in \mathbb{R}^{M}$ the column vector of respective measurements, the updating rule reads as follows:
\begin{subequations}
\begin{align}
    \theta_{t+1} & \sim \mathcal{N}(\mu_{t+1}, \Sigma_{t+1})\\
    \mu_{t+1} & = \Sigma_{t+1}\Big(\Sigma_{t}^{-1}\mu_{t} + \frac{1}{\sigma^{2}}\bar{\Phi}^{\top}_{t+1}\bar{m}_{t+1}\Big)\\
    \Sigma_{t+1} & = \Big(\frac{1}{\sigma^{2}}\bar{\Phi}^{\top}_{t+1}\bar{\Phi}_{t+1} + \Sigma_{t}^{-1}\Big)^{-1}. \label{eq:bayessigmaupdate}
\end{align}
\label{eq:bayesupdate}
\vspace{-0.2em}
\end{subequations}
\noindent Accordingly, at any point $p \in \mathbb{A}$ we can define the estimated density $\hat{\phi}_{t+1}(p)$ and the variance $\text{Var}_{t+1}(p)$ as
\begin{subequations}
\begin{align}
    \hat{\phi}_{t+1}(p) & = \Phi(p)\mu_{t+1} \label{eq:thetaupdate}\\
    \text{Var}_{t+1}(p) & = \Phi(p)\Sigma_{t+1}\Phi(p)^{\top} \label{eq:variancephiblr}.
\end{align}
\label{eq:densityupdateblr}
\end{subequations}
\vspace{-1.5em}

\begin{remark}
Other coverage control approaches in an unknown environment rely on Gaussian Processes (GPs) for estimation of the unknown density function, i.e., in~\cite{Todescato2017} and~\cite{prajapat2022}, but they suffer from increasing complexity and memory requirements. The parametric set-up stated in Assumption~\ref{assumption:bayesianconvergence} thereby allows for a recursive update with fixed computational complexity. This is particularly relevant for the one-layer approach, where the GP would otherwise enter the MPC formulation.
\rahel{Note that the adopted parametric representation can be seen as an approximation of a Gaussian process when selecting the rows of $\Phi(\cdot)$ as a Fourier basis: see, e.g., \cite{lazaro-gredilla_sparse_2010}.}
\end{remark}

\vspace{-0.7em}
\section{Partition Update, Collision Avoidance, and Recursive Feasibility}
\label{sec:colavoidanceandrecursivefeasibility}
\rahel{\noindent Like other coverage control approaches, the proposed framework relies on Voronoi partition updates as a key step to further decrease the locational optimization cost.}
Hence, in the following, we introduce our \rahel{partition update strategy, as well as the} safety-guaranteeing position constraint selection. %that builds upon it. 
Since these are intertwined, the update might jeopardize the recursive feasibility of the tracking MPC: therefore, we present a strategy to overcome this issue.\\
\rahel{ \noindent 
Since the nonlinear tracking MPC provides us with optimal setpoints, we update the Voronoi partitions with respect to their positions $\bar{p}_{i,w}^* = [C_{i},0]s_{i,w}^{*}$ for all $i={1,...,M}$, and denoting with $w$ the time index at which the update occurs. 
To prevent agents from colliding with each other, it needs to be ensured that, for all time instances $k \in \mathbb{N}$, 
\begin{equation}
\Vert p_{i,k} - p_{j,k}\Vert \geq (r_{i,\max} + r_{j,\max}), \ \forall i\neq j, 
\label{eq:colavoidancerequirement}
\end{equation}
where $r_{i,\max}$ is the radius of the ball covering the $i-$th agent. %indicates the radius that covers its respective agent. 
Letting $r_{\max} = \max\{r_{1,\max},\hdots,r_{M,\max}\}$, we define \mbox{$\bar{\mathbb{W}}_{\rahel{\bar{p}_w^*},i} := \mathbb{W}_{\rahel{\bar{p}_w^*},i} \ominus \mathbb{B}_{r_{\max}}^D$}  for all $i = \{1,\hdots,M\}$. 
Consequently, $p_{i} \in \bar{\mathbb{W}}_{\rahel{\bar{p}_w^*},i}$ for all $i = \{1,\hdots,M\}$ ensures~\eqref{eq:colavoidancerequirement}. Thus, collision avoidance can be achieved %and hence collision avoidance, which can then be achieved
using the position constraint $\mathbb{P}_i = \bar{\mathbb{W}}_{\rahel{\bar{p}_w^*},i}$ in the tracking MPC \eqref{eq:nonlineartrackingmpcwithcolavoidance}.}
\noindent However, feasibility of the nonlinear tracking MPC \rahel{with collision avoidance now} explicitly depends on the partition $ \bar{\mathbb{W}}_{\bar{p}^*,i}$ and any update might cause infeasibility of optimization problem~\eqref{eq:nonlineartrackingmpcwithcolavoidance} or invalidate the closed-loop properties in Theorem~\ref{theorem:theo4boccia}. \rahel{For this reason, the time instant $w$ will differ from the general time index $k$: i.e., the Voronoi partitions will not be updated at each time step. The proposed decision rule ensuring recursive feasibility given a candidate partition works as follows.}
Let \rahel{$\mathbb{W}_{\bar{p}_{k}^*}$ be the candidate Voronoi update at the current time-step $k>w$,} 
and consider the following input and state sequences
\begin{subequations}
    \begin{align}
    \hat{u}_{i, \cdot \vert k} = &[u^{*}_{i,1\vert k}, \hdots, u^{*}_{i,N-2\vert k}, \bar{u}_{i,k}^{*}, \bar{u}_{i,k}^{*}]^{\top}\label{eq:uhat}, \\
    \hat{x}_{i, \cdot \vert k} = &[x^{*}_{i,1\vert k}, \hdots, x^{*}_{i,N-1\vert k}, f(x^{*}_{i,N-1\vert k},\bar{u}_{i,k}^{*}), \label{eq:xhat} \\ &f(f(x^{*}_{i,N-1\vert k},\bar{u}_{i,k}^{*}),\bar{u}_{i,k}^{*})]^{\top}.\nonumber
    \end{align}
    \label{eq:lemma1proposal}
\end{subequations}
\noindent The idea is to only update the partitions if the candidate sequences remain feasible, i.e., if for all $l=\{1,...,N+1\}$ and all $i=\{1,...,M\}$, 
\begin{equation}
    \hat{x}_{i,l \vert k}\in\mathbb{X}_i, \ C_{i}\hat{x}_{i,l \vert k} \in \bar{\mathbb{W}}_{\bar{p}_{k}^*,i} \ \text{and} \  C_{i}\bar{x}_{i,k}^{*} \in \bar{\mathbb{W}}^{\mathrm{int}}_{\bar{p}_{k}^*,i},
    \label{eq:feasibilitycond}
\end{equation}
which will be later used to ensure closed-loop properties (cf. proof of Theorem~\ref{theorem:twolayers} and~\ref{theorem:onelayer} in the Appendix). \rahel{By construction, a partition updating setpoint $\bar{p}$ remains in $\bar{\mathbb{W}}^{\mathrm{int}}_{\bar{p}_w,i}$, the interior of its respective partition. Thus, if the agents' positions $p$ get close enough to such setpoints $\bar{p}$, then these positions lie in the updated Voronoi partition. This is formalized in the following lemma.  
\begin{lemma}
    For any $\bar{p} \in \mathbb{A}^{M}$ satisfying $\Vert \bar{p}_i - \bar{p}_j \Vert \geq 2(r_{\mathrm{max}} + \epsilon)$ for all $i \neq j$, it holds that $\bar{p} \in \bar{\mathbb{W}}^{\mathrm{int}}_{\bar{p}}$. Furthermore, $\forall p$ with $\Vert \bar{p} - p \Vert \leq \epsilon$, it holds $p \in \bar{\mathbb{W}}_{\bar{p}}$.
\label{lemma:inclusioninnewinterior}
\end{lemma}}
\vspace{-0.0em}

\rahel{\noindent Consequently, starting from a configuration $p$ for which the distance between all agents is bigger than or equal to $2(r_{\max} + \epsilon)$, which we further refer to as an \textit{initially feasible configuration}, the presented update and collision avoidance framework, in combination with Theorem~\ref{theorem:theo4boccia}, provides consistent safety guarantees and keeps the MPC recursively feasible.}

\noindent 

\section{Two-layers Coverage MPC Algorithm}
\label{sec:twolayers}
The two-layers approach presented in this section builds upon the MPC scheme proposed in~\cite{Carron2017}. However, leveraging the results of Sections \ref{subsec:nonlineartrackingmpc} and \ref{sec:colavoidanceandrecursivefeasibility}, the presented set-up does not leverage terminal ingredients, allowing for the consideration of more complex dynamical systems and the inclusion of collision avoidance constraints. In Section~\ref{subsec:twolayers} we study the solution with known density, while in~\ref{subsec:twolayerslearning} the set-up is adjusted to include learning.

\vspace{-0.3em}
\subsection{Known Environment}
\label{subsec:twolayers}
In this subsection, we consider the density $\phi$ as known. The overall presented strategy encompasses two main tasks: (i) \rahel{centralized computation of the Voronoi partitions and their respective centroids, according to the agents' current setpoint position;} (ii) \rahel{solution of the tracking MPC scheme \eqref{eq:nonlineartrackingmpcwithcolavoidance} with collision avoidance strategy from Section~\ref{sec:colavoidanceandrecursivefeasibility} 
%presented in Section~\ref{subsec:nonlineartrackingmpc}
and the centroids as references.} After solving task (ii), a new Voronoi partition can be computed \rahel{as in (i)}, and the procedure can be repeated. \\
We now discuss the conditions that ensure convergence of this iterative set-up to \rrahel{a centroidal} Voronoi partition solving problem~\eqref{eq:cortescost}. For the first task, we require an update rule on the partitions, which is built upon~\cite[Proposition 3.3]{Cortes2004}:  
\begin{proposition}\cite[Proposition 3.3]{Cortes2004} Assume that $p_{0} \in \mathbb{A}^{M}$ and the Voronoi partition is updated in finite time.
Further consider a continuous mapping $T: \mathbb{A}^{M} \rightarrow \mathbb{A}^{M}$, with $\rahel{\bar{p}_{w_{+}}^* = T(\bar{p}_w^*)}$, where $w_{+}$ indicates the next update step. If the mapping \rahel{$T(\bar{p}_w^*)$} fulfills the following properties at each update step $w$:
%time step $k$: 
\begin{itemize}
    \item $ \Vert \rahel{\bar{p}_{i,w_{+}}^*} - c_i(\mathbb{W}_{\rahel{\bar{p}_{w}^*},i}, \phi) \Vert \leq \Vert \rahel{\bar{p}_{i,w}^*} - c_i(\mathbb{W}_{\rahel{\bar{p}_{w}^*},i}, \phi) \Vert,$ \newline  $\forall i \in \{1, ..., M\}$;
    \item As long as the positions do not describe a centroidal Voronoi partition,
    $\exists j \in \{1, ..., M\}$ such that$ \newline \Vert \rahel{\bar{p}_{j,w_{+}}^*} - c_j(\mathbb{W}_{\rahel{\bar{p}_{w}^*},j}, \phi) \Vert < \Vert \rahel{\bar{p}_{j,w}^*} - c_j(\mathbb{W}_{\rahel{\bar{p}_{w}^*},j}, \phi) \Vert$,
\end{itemize}
then, \rahel{$\bar{p}_{w}^*$} converges to a centroidal Voronoi partition.
\label{proposition:cortesconvergence}
\end{proposition}

As integrator dynamics are able to reduce their Euclidean distance to a given reference at each time-step $k$, their partitions can be updated accordingly. However, as arbitrary nonlinear dynamics do not generally decrease the distance to their corresponding centroid for all instances in time, 
a partition update rule has to be applied to meet the requirements of Proposition~\ref{proposition:cortesconvergence} and allow for the construction of a suitable mapping $T$. For this purpose, we rely on the conditions proposed in~\cite[Section~4]{Carron2017}: \rahel{i.e., defining $e_{i,w} \! = \! \Vert \bar{p}_{i,w}^* \! - \! c_i(\mathbb{W}_{\bar{p}_{w}^*,i}, \phi) \Vert$ and for later use also their combining vector $e_{w}=[e_{1,w}, \hdots, e_{M,w}]$}, the partition is updated only if
\begin{subequations}
\begin{align}
    & \bullet \; \Vert \rahel{\bar{p}_{i,k}^*} - c_i(\mathbb{W}_{\rahel{\bar{p}_{w}^*},i}, \phi) \Vert \leq e_{i,w} \;\: \forall i \in \{1,..., M\}, \label{eq:updatereq1} \\
    & \bullet \; \exists j \! \in \! \{1,..., M\} \text{ s.t. } \Vert \rahel{\bar{p}_{j,k}^*} \! - \! c_j(\mathbb{W}_{\rahel{\bar{p}_{w}^*},j}, \phi) \Vert \! < \! e_{j,w}. \label{eq:updatereq2} 
\end{align}
\label{eq:updatereq}%
\end{subequations}
\rahel{Note that, for recursive feasibility, we additionally always require the conditions from Section \ref{sec:colavoidanceandrecursivefeasibility} to hold.}\\
We now study the second task, i.e., the tracking MPC having the centroids as references, and discuss its convergence. The optimization problem at time $k$ is defined as in~\eqref{eq:nonlineartrackingmpcwithcolavoidance}, \rahel{with position constraints \mbox{$\mathbb{P}_i = \mathbb{W}_{\bar{p}_{w}^*}$}, and reference $r = c(\mathbb{W}_{\bar{p}_{w}^*}, \phi)$ with  $\bar{p}_{w}^*$ set to the most recent steady-state position configuration that fulfilled~\eqref{eq:updatereq} and~\eqref{eq:feasibilitycond}. The cost function $J_{i}(x_{i}, \bar{x}_{i}, u_{i}, \bar{u}_{i}, r_{i})$ follows the definition in~\eqref{eq:generalnonlineartrackingmpccost}. To ensure convergence, we leverage the following condition on the offset cost $\ell_T$, adapted from the literature~\cite{Soloperto2021}.}
\begin{assumption}  
For every agent $i \in \{1, ..., M\}$, given any \rahel{$\bar{p}_i \in \mathbb{S}^{\mathrm{P}}_{\mathbb{A},i}$}, corresponding $\mathbb{W}_{\bar{p}}$ and reference vector $r \in \mathbb{A}^{M}$, we define the set \begin{equation}
\begin{split}
    \rahel{\mathbb{T}_{\bar{p},r,i}} = \{s\in \mathbb{S}_{\mathbb{W}_{\bar{p},i}} \: \vert \: \ell_{T,i}(C_{i}\bar{x} - r_{i}) = l_{\mathbb{W}_{\bar{p}},r,i,\min}\},
\end{split}
\label{eq:Tid2layer}
\end{equation}
which is the union of all setpoints resulting in a target cost value equal to $l_{\mathbb{W}_{\bar{p}},r,i,\min} = \min_{s\in \mathbb{S}_{\mathbb{W}_{\bar{p},i}}} \ell_{T,i}(C_i\bar{x}_i-r_i)$. Then, for every agent $i \in \{1,\hdots,M\}$ there are constants  \mbox{$\beta_{1,i}, \beta_{2,i} > 0$} such that, for any $\bar{s} := (\bar{x},\bar{u}) \in \mathbb{S}_{\mathbb{W}_{\bar{p},i}}$ and any $\epsilon^{\prime} \in [0,1]$, there exists a setpoint $\hat{s} := (\hat{x},\hat{u}) \in \mathbb{S}_{\mathbb{W}_{\bar{p},i}}$ satisfying 
\vspace{-0.8em}
\begin{subequations}
\begin{align}
    d_{i}(\hat{s}-\bar{s}) &\leq \beta_{1,i}\epsilon^{\prime} d_{i}(\bar{s})_{\rahel{\mathbb{T}_{\bar{p},r,i}}} \label{eq:assumption31twolayer} \\
    \ell_{T,i}(C_{i}\hat{x} \! - \! r_{i}) \! - \! \ell_{T,i}(C_{i}\bar{x}-r_{i}) &\leq \! - \! \beta_{2,i}\epsilon^{\prime} d_{i}(\bar{s})_{\rahel{\mathbb{T}_{\bar{p},r,i}}}^{2}, \label{eq:assumption32twolayer}%
\end{align}
\label{eq:assumption3twolayer}
\end{subequations} 
where $d_i(\cdot)$ is the distance function defined in~\eqref{eq:distance} and $d_{i}(\cdot)_{\rahel{\mathbb{T}_{\bar{p},r,i}}}$ follows definition~\eqref{eq:minsetdistance}. Furthermore, there is a constant  \mbox{$\beta_{T,i} > 0$} such that, for any $s := (\bar{x},\bar{u}) \in \mathbb{S}_{\mathbb{W}_{\bar{p}_w^*,i}}$,% it holds:
\begin{align}
    \ell_{T,i}(C\bar{x}-r_i) - l_{\mathbb{W}_{\bar{p}},r,i,\min} \leq \beta_{T,i}\epsilon^{\prime} d_{i}(s)_{\rahel{\mathbb{T}_{\bar{p},r,i}}}
    \label{eq:upperboundtargetcost}
\end{align}
\label{assumption:steadystatedecreasetwolayers}%
\end{assumption}
\vspace{-1.0em}
\noindent Assumption~\ref{assumption:steadystatedecreasetwolayers} states that, for any setpoint $\bar{s}$ not being a minimizer of the target cost (i.e., resulting in a target cost value not equal to its attainable minimal value $l_{\mathbb{W}_{\bar{p}_w^*},r,i,\min}$), there exists another setpoint in its neighborhood such that the target cost can be actually reduced. Additionally, it claims that the target cost of an arbitrary setpoint is upper bounded by a function of its distance to the target cost's minimizing setpoint. Assumption~\ref{assumption:steadystatedecreasetwolayers} is generally satisfied by choosing a target cost of the same powers as the stage cost, and by having $\mathbb{S}^{\mathrm{P}}_{\mathbb{W}_{\bar{p}_w^*,i}}$ and $\mathbb{A}$ as polytopes. The latter condition is not restrictive for most dynamical systems of interest in this work. \\
The overall procedure consists in constructing the Voronoi partitions with respect to the agents' steady-state position and setting the references equal to their centroids. Then, the MPC defined with according position constraints is applied recursively until conditions~\eqref{eq:updatereq1},~\eqref{eq:updatereq2} and~\eqref{eq:feasibilitycond} are fulfilled and the partitions, as well as the centroids, are updated. It is summarized in Algorithm~\ref{alg:twolayermpcalg} and we state our main result in Theorem~\ref{theorem:twolayers}, whose proof can be found in Appendix~\ref{subsec:twolayersproof}.

\begin{algorithm}[h!]
\SetAlgoLined
Set $k = 0$, $w=0$ and \rahel{ construct $\mathbb{W}_{\bar{p}_{w}^{*}}$ with $\bar{p}_0^*=p_0$} \par
Set \rahel{$\mathbb{P}_i = \mathbb{W}_{\rahel{\bar{p}_{w}^*}}$,} $r = c(\mathbb{W}_{\rahel{\bar{p}_{w}^*}},\phi)$ and calculate $e_{w}$.\par
 \For{k=0,1,\dots}{
 \ForAll{$i \in \{1, \hdots,M\}$}{
 Solve~\eqref{eq:nonlineartrackingmpcwithcolavoidance} and obtain $\hat{x}_{i,\cdot \vert k}$, $\hat{u}_{i,\cdot \vert k}$, \rahel{ $\bar{x}^{*}_{i,k}$,  $u^{*}_{i,0\vert k}$.} \par
 Apply $u^{*}_{i,0\vert k}$ to obtain  $x_{i,k+1}$. \par}
 Construct $\mathbb{W}_{\rahel{\bar{p}_{k}^{*}}}$ according to \rahel{$\bar{p}_{k}^{*}$}. \par
\If{\textup{conditions~\eqref{eq:updatereq1},~\eqref{eq:updatereq2} and~\eqref{eq:feasibilitycond} are fulfilled}}{
   Set $w = k$ and update $\mathbb{W}_{\rahel{\bar{p}_{w}^*}}$ = $\mathbb{W}_{\rahel{\bar{p}_{k}}}$.\par
   Update \rahel{$\mathbb{P}_i = \mathbb{W}_{\rahel{\bar{p}_{w}^*}}$,} $r = c(\mathbb{W}_{\rahel{\bar{p}_{w}^*}},\phi)$ and calculate $e_{w}$.}
   }
 \caption{Two-Layers Coverage MPC}
 \label{alg:twolayermpcalg}
\end{algorithm}

\begin{theorem} 
Let Assumptions~\ref{assumption:dynamics},~\ref{assumption:expocostcontrollability},~\ref{assumption:boundedbyd}, and~\ref{assumption:steadystatedecreasetwolayers} hold, and consider a horizon length $N \geq N^{*}$. Additionally, suppose that at time step $k = 0$ the MPC problem described in~\eqref{eq:nonlineartrackingmpcwithcolavoidance} is feasible for all agents. Then, the overall control problem according to Algorithm~\ref{alg:twolayermpcalg} is recursively feasible, the agents do not collide and satisfy $x_{i,k} \in \mathbb{X}_{i},\; u_{i,k} \in \mathbb{U}_{i}, \forall i \in \{1,\hdots,M\}$, $\forall k \in \mathbb{N}$. Furthermore, the partition update condition given by~\eqref{eq:updatereq1},~\eqref{eq:updatereq2} and~\eqref{eq:feasibilitycond} is fulfilled after a finite amount of time, and the agents' position configuration converges to a centroidal Voronoi partition.
\label{theorem:twolayers}
\end{theorem}

\vspace{-0.7em}
\subsection{Unknown Environment}
\label{subsec:twolayerslearning}
To deal with an initially unknown $\phi$, the motion planning scheme in the tracking MPC considered above must be adjusted to balance density learning and coverage. In particular, the reference of each agent during exploration is chosen as the point of maximal variance within the Voronoi region\footnote{Finding the point of maximal variance over the whole Voronoi region implies evaluating~\eqref{eq:densityupdateblr} on an infinite number of points. In practice, we perform a sufficiently dense gridding of each $\mathbb{W}_{\rahel{\bar{p}_w^*},i}$ and evaluate variances on those points.} $\bar{\mathbb{W}}_{\rahel{\bar{p}_w^*},i}^{\mathrm{int}}$, denoted by $v_i(\mathbb{W}_{\rahel{\bar{p}_w^*},i}, \text{Var}) = \text{arg}\!\max_{\tilde{p} \in \bar{\mathbb{W}}_{\rahel{\bar{p}_w^*},i}^{\mathrm{int}}} \text{Var}(\tilde{p})$ and belonging to the vector $v(\mathbb{W}_{\rahel{\bar{p}_w^*}}, \text{Var}) = [v_1(\mathbb{W}_{\rahel{\bar{p}_w^*},1}, \text{Var}), \hdots, v_M(\mathbb{W}_{\rahel{\bar{p}_w^*},M}, \text{Var})]$. In case of exploitation, the reference is set to the current partition's centroid with respect to the available density estimate $\hat{\phi}_t$, i.e., $r_{i,k} = c_i(\mathbb{W}_{\rahel{\bar{p}_w^*},i}, \hat{\phi}_t)$. As proposed in~\cite{Todescato2017}, denoting with $\text{Var}_{\max} = \max_{p \in \mathbb{A}} \text{Var}(p)$ the maximal variance value computed over the set $\mathbb{A}$, and with $F: [0,\infty] \rightarrow [0,1]$ an arbitrary strictly monotonically increasing function such that \mbox{$F(\epsilon)=0 \Leftrightarrow \epsilon=0$}, the decision between exploration and exploitation is performed according to a Bernoulli random variable $\mathcal{B}(F(\text{Var}_{\max})) \in \{0,1\}$. The rationale is the following: at each time step, a sample from $\mathcal{B}$ is drawn whose probability of success (i.e., of returning a value equal to 1) is $F(\text{Var}_{\max})$. In this case, exploration is selected, and both the Voronoi tessellation and the references are kept fixed for all agents. The exploration modality continues regardless of the new samples of $\mathcal{B}(F(\text{Var}_{\max}))$ and the data-set is expanded as $I_{t+1} = I_{t} \cup [(m_{1,k},p_{1,k}), \hdots, (m_{M,k},p_{M,k})]$ at each time-step until the agents' positions are sufficiently close to the point of maximum variance of their Voronoi region, i.e., when $e_{v,i,k} = \Vert v_i(\mathbb{W}_{\bar{p}_w^*,i}, \text{Var}_t) - p_{i,k} \Vert \leq \rho$ for an arbitrarily small $\rho >0$. At this point, exploration modality is left and a new sample from $\mathcal{B}$ is drawn. As for the exploitation phase, Voronoi partitions are updated only if conditions~\eqref{eq:updatereq1},~\eqref{eq:updatereq2} and~\eqref{eq:feasibilitycond} are met. The overall procedure is summarized in Algorithm~\ref{alg:twolayermpcalglearningimp}. The main result is given in the following theorem that is proven in Appendix~\ref{subsec:twolayerslearningproof}.
\begin{algorithm}[h!]
\SetAlgoLined
 Set $k = 0$, $w=0$ and \rahel{construct $\mathbb{W}_{\bar{p}_{w}^{*}}$ with $\bar{p}_0^*=p_0$}\par
 Set $t = 1$, $\rho$, $\mu_{0}, \Sigma_{0}$ and $I_{t} = [(m_{1,k},p_{1,k}),..., (m_{M,k},p_{M,k})]$.\par
 Compute $\hat{\phi}_{t}(p), \text{Var}_{t}(p) \ \ \forall p \in \mathbb{A}$. \par
 Compute $c(\mathbb{W}_{\rahel{\bar{p}_{w}^{*}}}, \hat{\phi}_{t}), v(\mathbb{W}_{\rahel{\bar{p}_{w}^{*}}}, \text{Var}_{t}), \text{Var}_{\max,t}$ and $e_{w}$.\par
 Set exploration flag = false. \par
  \For{k=0,1, \dots}{
  \eIf{\textup{exploration flag} $\Vert$ \textup{(}$\mathcal{B}(F(\textup{Var}_{\textup{max},t})) == 1$\textup{)}}{
   Set exploration flag = true. \par
   \ForAll{$i \in \{1, \hdots, M \}$}{
   Set \rahel{$\mathbb{P}_i = \mathbb{W}_{\rahel{\bar{p}_{w}^*}}$,} $r_{i,k} = v_i(\mathbb{W}_{\rahel{\bar{p}_{w}^{*}},i}, \text{Var}_{t})$.\par
   Solve~\eqref{eq:nonlineartrackingmpcwithcolavoidance} and obtain $u^{*}_{i,0\vert k}.$ \par
   Apply $u^{*}_{i,0\vert k}$ to obtain $x_{i,k+1}$ and $e_{v,i,k+1}$. \par} 
   %$e_{v,i,k+1} = \Vert v_i(\mathbb{W}_{\bar{p}_{w}^{*},i}, \text{Var}_{t}) - p_{i,k+1} \Vert$. \par}
   $I_{t+1} \! = \! {\footnotesize I_{t} \! \cup \![(m_{1,k+1},p_{1,k+1}),..., (m_{M,k+1},p_{M,k+1})]}$ \par
   Update~\eqref{eq:thetaupdate}-\eqref{eq:variancephiblr} $\forall p \in \mathbb{A}$. \par
   Update $c(\mathbb{W}_{\rahel{\bar{p}_{w}^{*}}}, \hat{\phi}_{t+1})$,$v(\mathbb{W}_{\bar{p}_{w}^{*}}, \! \text{Var}_{t+1}),\text{Var}_{\max,t+1}$.\par
   $t=t+1$ \par
   \If{$\|e_{v,i,k+1}\|\leq\rho$, 
   $\forall i \in \{1, \hdots, M \}$}
   {
   Set exploration flag = false. \par}
  }
  {\ForAll{$i \in \{1, \hdots, M \}$}{
  Set \rahel{$\mathbb{P}_i = \mathbb{W}_{\rahel{\bar{p}_{w}^*}}$,} $r_{i,k} = c(\mathbb{W}_{\rahel{\bar{p}_{w}^{*}},i}, \hat{\phi}_{t})$. \par
  Solve~\eqref{eq:nonlineartrackingmpcwithcolavoidance} and obtain $\hat{x}_{i,\cdot \vert k}$, $\hat{u}_{i,\cdot \vert k}$, \rahel{  $\bar{x}^{*}_{i,k}$,  $u^{*}_{i,0\vert k}$.} \par
 Apply $u^{*}_{i,0\vert k}$ to obtain  $x_{i,k+1}$. \par}
 Construct $\mathbb{W}_{\rahel{\bar{p}_{k}^{*}}}$ according to $\rahel{\bar{p}_{k}^{*}}$. \par
\If{\textup{conditions~\eqref{eq:updatereq1},~\eqref{eq:updatereq2} and~\eqref{eq:feasibilitycond} are fulfilled}}{
   $w = k$, $\mathbb{W}_{\rahel{\bar{p}_{w}^{*}}}$ = $\mathbb{W}_{\rahel{\bar{p}_{k}}} \xrightarrow{update}$$c(\mathbb{W}_{\rahel{\bar{p}_{w}^{*}}},\hat{\phi}_t)$, $e_{w}$.}}
   }
  \caption{Two-Layers, Learning-Based Coverage MPC}
 \label{alg:twolayermpcalglearningimp}
\end{algorithm}

\begin{theorem} 
Let Assumptions~\ref{assumption:dynamics},~\ref{assumption:expocostcontrollability},~\ref{assumption:boundedbyd},~\ref{assumption:bayesianconvergence}, and~\ref{assumption:steadystatedecreasetwolayers} hold and consider a horizon length $N \geq N^{*}$ and $\rho>0$ sufficiently small. Additionally, suppose that at time step $k = 0$ the MPC problem described in~\eqref{eq:nonlineartrackingmpcwithcolavoidance} is feasible for all agents. Then, the overall control problem according to Algorithm~\ref{alg:twolayermpcalglearningimp} is recursively feasible, the agents do not collide, and satisfy $x_{i,k} \in \mathbb{X}_{i}, u_{i,k} \in \mathbb{U}_{i}, \forall i \in \{1,\hdots,M\}$, $\forall k \in \mathbb{N}$. Furthermore, the partition update condition given by~\eqref{eq:updatereq1},~\eqref{eq:updatereq2} and~\eqref{eq:feasibilitycond} is fulfilled after a finite amount of time, and the agents' position configuration converges to a centroidal Voronoi partition.
\label{theorem:twolayerslearning}
\end{theorem}
\section{One-Layer Coverage MPC Algorithm}
\label{sec:onelayer}
In the one-layer approach, the reference is not pre-calculated and passed to the MPC, but the locational optimization function is jointly optimized with the MPC cost, cf. Figure~\ref{fig:coverageclarification}. The combined optimization is then expected to reduce the time and energy required for exploration.
\vspace{-0.5em}
\subsection{Known Environment}
\label{subsec:onelayer}
For the case of a known density $\phi$, the MPC optimization cost of the one-layer framework will encompass the same tracking cost $V_{N,i}$ considered in the previous sections, but the target cost $\ell_{T,i}$ will be equal to the corresponding summand of the locational optimization cost defined in~\eqref{eq:cortescost}. In particular, the latter is to be optimized with respect to the steady-state position $\bar{p}_{i,k} = C_i\bar{x}_{i,k} \in \mathbb{S}^{\mathrm{p}}_{\mathbb{W}_{\bar{p}^{*}_{w},i}}$ (see~\eqref{eq:steadystatesetwithcollavoidance}), and the integral is evaluated over the current Voronoi partition $\mathbb{W}_{\bar{p}^{*}_{w},i}$.
Thus, the resulting continuous objective reads as follows:
\begin{align}
&J_{i}(x_{i,\cdot \vert k}, u_{i,\cdot \vert k}, s_{i,k},\mathbb{W}_{\bar{p}^{*}_{w},i}, \phi) = \label{eq:fullnonlineartrackingmpccostonelayer} \\ & V_{N,i}(x_{i,\cdot \vert k}, u_{i,\cdot \vert k}, s_{i,k}) +   \ell_{T,i}(\bar{p}_{i,k},\mathbb{W}_{\bar{p}^{*}_{w},i}, \phi) = \notag\\ 
& \sum_{l=0}^{N-1}\ell_{i}(x_{i,l\vert k}, u_{i,l \vert k}, s_{i,k}) + \lambda \int_{\mathbb{W}_{\bar{p}^{*}_{w},i}} \Vert q - \bar{p}_{i,k} \Vert^{2}\phi(q)dq,\notag
\end{align}
with \rahel{$\lambda > 0$} representing a scaling factor. The overall MPC program is then stated as
\begin{equation}
\begin{split}
\min_{x_{i,\cdot \vert k}, u_{i,\cdot \vert k},s_{i,k}}&J_{i}(x_{i,\cdot \vert k}, u_{i,\cdot \vert k}, s_{i,k},\mathbb{W}_{\bar{p}^{*}_{w},i}, \phi) \\
&\eqref{eq:generalinit}-\eqref{eq:1toN-1withcolavoidance} ,
\end{split}
\label{eq:fullnonlineartrackingmpconelayer}
\end{equation}
and is solved within the iterative coverage scheme presented in Algorithm~\ref{alg:onelayeralg}. To show convergence to a centroidal Voronoi configuration while dealing with non-convex target costs, Assumptions~\ref{assumption:ballwithminconvex} and~\ref{assumption:steadystatedecreaseonelayer} are imposed.

\begin{algorithm}[h!]
\caption{One-Layer Coverage MPC}
\SetAlgoLined
Set $k = 0$, $w=0$ and 
construct $\mathbb{W}_{\bar{p}_{w}^{*}}$ with $\bar{p}_0^*=p_0$\par
\For{k=0,1, \dots}{\ForAll{$i \in \{1, \hdots, M \}$}{
 \rahel{Set $\mathbb{P}_i = \mathbb{W}_{\bar{p}_{w}^{*}}$} \par
 Solve~\eqref{eq:fullnonlineartrackingmpconelayer} and obtain  $\hat{x}_{i,\cdot \vert k}$, $\hat{u}_{i,\cdot \vert k}$,  $\bar{x}^{*}_{i,k}$,  $u^{*}_{i,0\vert k}$. \par
 Apply $u^{*}_{i,0\vert k}$ to obtain $x_{i,k+1}$. \par}
 %$x_{i,k+1}$ = $f_{i}(x_{i,k},u^{*}_{i,0\vert k})$. \par}
 Construct $\mathbb{W}_{\bar{p}_{k}^{*}}$ according to $\bar{p}_{k}^{*}$. \par
  \If{\textup{condition~\eqref{eq:feasibilitycond} is fulfilled}}{
  $w = k$, $ \mathbb{W}_{\bar{p}_{w}^{*}} = \mathbb{W}_{\bar{p}_{k}^{*}}^{'}$.}
 }
\label{alg:onelayeralg}
\end{algorithm}

\begin{assumption} For every agent $i \in \{1, ..., M\}$, given any $\bar{p} \in \mathbb{S}^{p}_{\mathbb{A},i}$ and any $\bar{p}' \in \mathbb{S}^{\mathrm{p}}_{\mathbb{W}_{\bar{p},i}}$ there exist an $r > 0$ such that defining $S^{\mathrm{p}}_{\bar{p}',\mathbb{W}_{\bar{p},i}} := \mathbb{B}_{r}^{D}(\bar{p}') \cap \mathbb{S}^{\mathrm{p}}_{\mathbb{W}_{\bar{p},i}},$ the set $\mathrm{argmin}_{p \in S^{\mathrm{p}}_{\bar{p}',\mathbb{W}_{\bar{p},i}}} \ell_{T,i}(p,\mathbb{W}_{\bar{p},i}, \rahel{\phi})$ is convex.
\label{assumption:ballwithminconvex}
\end{assumption}
\begin{assumption} 
Let $l_{\bar{p}',\mathbb{W}_{\bar{p}},i,\min} \! = \! \min_{p \in S^{\mathrm{p}}_{\bar{p}',\mathbb{W}_{\bar{p},i}}}\ell_{T,i}(p,\mathbb{W}_{\bar{p},i},\rahel{\phi})$, and define the corresponding set of local minimizers as
\begin{equation}
\begin{split}
    \rahel{\mathbb{T}_{\bar{p}',\bar{p},i}} \! = \! \{p \in S^{\mathrm{p}}_{\bar{p}',\mathbb{W}_{\bar{p},i}}\,\vert\, \ell_{T,i}(p,\mathbb{W}_{\bar{p},i},\rahel{\phi}) \! = \! l_{\bar{p}',\mathbb{W}_{\bar{p}},i,\min}\}.
\end{split}
\label{eq:Tid}
\end{equation}
Then, for every agent $i \in \{1,\hdots,M\}$, any $\bar{p}^{\prime} \in \mathbb{S}^{p}_{\mathbb{A},i}$ and any $\bar{s}' := (\bar{x}',\bar{u}') \in \mathbb{S}_{\mathbb{W}_{\bar{p},i}}$ there are constants $\beta_{1,i}, \beta_{2,i} > 0$ as well as a $\mathcal{K}$-function $\kappa$ such that, for any $\epsilon^{\prime} \in [0,1]$, there exists a setpoint $\hat{s} := (\hat{x},\hat{u}) \in S_{\bar{p}',\mathbb{W}_{\bar{p},i}}$ satisfying
\begin{subequations}
\begin{align}
    d_{i}(\hat{s}-\bar{s}') &\leq \beta_{1,i}\epsilon^{\prime} \kappa(\Vert \bar{p}' \Vert)_{\rahel{\mathbb{T}_{\bar{p}',\bar{p},i}}}
    \label{eq:assumption31},\\
    \ell_{T,i}(\hat{p},\mathbb{W}_{\bar{p},i},\rahel{\phi}) \! - \! \ell_{T,i}(\bar{p}',\mathbb{W}_{\bar{p},i},\rahel{\phi}) &\leq \! - \! \beta_{2,i}\epsilon^{\prime} \kappa(\Vert \bar{p}' \Vert)_{\rahel{\mathbb{T}_{\bar{p}',\bar{p},i}}}^{2} \label{eq:assumption32},
\end{align}
\label{eq:assumption3}
\end{subequations}
with $\hat{p} = C_i\hat{x}$ and $\bar{p}' = C_i\bar{x}'$. Furthermore, there is a constant  \mbox{$\beta_{T,i} > 0$} such that, for any $s := (\bar{x},\bar{u}) \in \mathbb{S}_{\mathbb{W}_{\bar{p},i}}$ it holds:
\begin{align}
    \ell_{T,i}(C\bar{x}-r_i,\mathbb{W}_{\bar{p},i},\rahel{\phi}) - l_{\bar{p}',\mathbb{W}_{\bar{p}},i,\min} \leq \beta_{T,i}\epsilon^{\prime} d_{i}(s)_{\rahel{\mathbb{T}_{\bar{p}',\bar{p},i}}}
    \label{eq:assumption33}
\end{align}
\label{assumption:steadystatedecreaseonelayer}
\end{assumption}
\vspace{-1.0em}
\noindent Assumption~\ref{assumption:ballwithminconvex} builds upon the fact that the constructed Voronoi partitions are never empty (see Section~\ref{sec:colavoidanceandrecursivefeasibility}) \rahel{and ensures convexity of $\rahel{\mathbb{T}_{\bar{p}',\bar{p},i}}$. Assumption~\ref{assumption:steadystatedecreaseonelayer} is a generalization of Assumption~\ref{assumption:steadystatedecreasetwolayers} to the case of non-convex target costs. This becomes crucial in the next section, when we consider non-convex target costs $\ell_{T,i}$ for the learning. Note that, in the chosen set-up, $\ell_{T,i}(\bar{p},\mathbb{W}_{\bar{p},i})$ is convex in $p$~\cite[Lemma 6.1]{Bullo2012} and the set of minimizers is convex, i.e., Assumption~\ref{assumption:ballwithminconvex} holds with $ S^{\mathrm{p}}_{\bar{p}',\mathbb{W}_{\bar{p},i}} =\mathbb{S}^{\mathrm{p}}_{\mathbb{W}_{\bar{p},i}}$. } In accordance to~\cite[Remark 1]{Soloperto2021}, Assumption~\ref{assumption:steadystatedecreasetwolayers} holds taking functions $d$ and $\kappa$ quadratic if $\ell_{T,i}$ is strongly convex quadratic and sets are polytopic. \\
\noindent Next, we state the theorem collecting all the theoretical guarantees of the proposed scheme. The proof is presented in Appendix~\ref{subsubsec:onelayerknownenvironmenttheory}.
\begin{theorem} Let Assumptions~\ref{assumption:dynamics},~\ref{assumption:expocostcontrollability},~\ref{assumption:boundedbyd},~\ref{assumption:ballwithminconvex} and~\ref{assumption:steadystatedecreaseonelayer} hold, and consider a horizon length $N \geq N^{*}$. Additionally, suppose that at time step $k = 0$ the MPC problem described in~\eqref{eq:fullnonlineartrackingmpconelayer} is feasible for all agents. Then, in accordance to Algorithm~\ref{alg:onelayeralg} it remains recursively feasible, the agents do not collide and satisfy $x_{i,k} \in \mathbb{X}_{i},\; u_{i,k} \in \mathbb{U}_{i}, \forall i \in \{1,\hdots,M\}$, $\forall k \in \mathbb{N}$. Furthermore, the partition update condition given by~\eqref{eq:feasibilitycond} is fulfilled after a finite amount of time, and the agents' position configuration converges to a centroidal Voronoi partition.
\label{theorem:onelayer}
\end{theorem}

\vspace{-0.8em}
\subsection{Unknown Environment}
\label{subsec:onelayerlearning}
The adaptation of the one-layer approach to an unknown environment involves the following modification of the target cost $\ell_{T,i}$ entering~\eqref{eq:fullnonlineartrackingmpconelayer}, inspired by the Upper Confidence Bound method in Bayesian Optimization~\cite{Auer2002,Srinivas2010}. Specifically, the uncertainty on a potential setpoint $\bar{p}_i$, quantified by the variance $\text{Var}(\bar{p}_i)$ and scaled by a parameter $S > 0$, is subtracted from~\eqref{eq:fullnonlineartrackingmpconelayer}. Hence, the target cost now reads as
\begin{equation}
\begin{split}
&\ell_{T,i}(\bar{p}_{i}, \mathbb{W}_{\bar{p}^{*}_{w},i}, \hat{\phi}, \text{Var}) = \\ & \lambda \left(\int_{\mathbb{W}_{\bar{p}^{*}_{w},i}} \Vert q - \bar{p}_{i} \Vert^{2}\hat{\phi}(q)dq - S\cdot \text{Var}(\bar{p}_{i}) \right).
\end{split}
\label{eq:fullnonlineartrackingmpccostonelayerlearning}
\end{equation}
Apart from this modification of the cost and the data collection, the overall procedure follows the one proposed in Section~\ref{subsec:onelayer} and is summarized in Algorithm~\ref{alg:onelayerlearningalg}. Note that~\eqref{eq:fullnonlineartrackingmpccostonelayerlearning} is non-convex, so Assumptions~\ref{assumption:ballwithminconvex} and~\ref{assumption:steadystatedecreaseonelayer} become crucial. However, they hold with respect to the modified target cost $\ell_{T,i}$ defined in~\eqref{eq:fullnonlineartrackingmpccostonelayerlearning}, given its continuity with respect to $\bar{p}_i$\footnote{Given arbitrary positions $p_1$, $p_2$ and $q \in \mathbb{A}$, by continuity of norms we have that $\lim_{p_1 \rightarrow p_2} \|q-p_1\|^2-\|q-p_2\|^2 =0$. By considering the locational cost difference $\lim_{p_1\rightarrow p_2} H_{i}(p_{1}, \mathbb{W}_{\bar{p},i}) - H_{i}(p_{2}, \mathbb{W}_{\bar{p},i})$, and noting that the limit of the integrand exists and is integrable, conclusion follows by swapping limit and integration operations. }%, which is proven below. 
Also, Assumption~\ref{assumption:ballwithminconvex} seems natural in such a scenario, ensuring the existence of a direction in which $\ell_{T,i}(\bar{p}',\mathbb{W}_{\bar{p},i})$ is decreasing and fulfilling equation~\eqref{eq:assumption32} as long as $\bar{p}' \notin \mathbb{T}_{\bar{p}',\bar{p},i}$. 

\begin{algorithm}[h!]
\caption{One-Layer, Learning-Based Coverage MPC}
\SetAlgoLined
Set $k = 0$, $w = 0$ and construct $\mathbb{W}_{\bar{p}^{*}_{w}}$ with $\bar{p}_0^*=p_0$\par
Set $t = 1$, $\mu_{0}$, $\Sigma_{0}$ and $I_{t} = [(m_{1,k},p_{1,k}), \hdots, (m_{M,k},p_{M,k})]$. \par 
Update $\hat{\phi}_{t}(p), \text{Var}_{t}(p) \ \ \forall p \in \mathbb{A}$. \par
 \For{k = 0,1,\dots}{\ForAll{$i \in \{1, \hdots, M \}$}{
  Set \rahel{$\mathbb{P}_i = \mathbb{W}_{\bar{p}_{w}^{*}}$ and}  $\ell_{T,i}$ = $\ell_{T,i}(\bar{p}, \mathbb{W}_{\bar{p}^{*}_{w},i}, \hat{\phi}_{t}, \text{Var}_{t})$ \par
  Solve \eqref{eq:fullnonlineartrackingmpconelayer} and obtain  $\bar{x}^{*}_{i,k}, u^{*}_{i,0\vert k}, \hat{u}_{i,\cdot \vert k}$, $\hat{x}_{i,\cdot \vert k}$. \par
  Apply $u^{*}_{i,0\vert k}$ to obtain $x_{i,k+1}$. \par}
 $I_{t+1} = I_{t} \cup [(m_{1,k},p_{1,k}), \hdots, (m_{M,k},p_{M,k})]$ \par
 Update~\eqref{eq:thetaupdate}-\eqref{eq:variancephiblr} $\forall p \in \mathbb{A}$. \par
 Construct $\mathbb{W}^{'}_{\bar{p}_{k}^{*}}$ according to $\bar{p}_{k}^{*}$. \par
  \If{\textup{condition~\eqref{eq:feasibilitycond} is fulfilled}}{
  $w = k$, $ \mathbb{W}_{\bar{p}_{w}^{*}} = \mathbb{W}_{\bar{p}_{k}^{*}}^{'}$.} $t=t+1$
 }
\label{alg:onelayerlearningalg}
\end{algorithm}

\noindent Define the set of feasible positions in the MPC problem of agent $i$ at time $k$ by $\mathbb{F}_{i,k}=\{p\in\mathbb{A}|~\exists (x_{\cdot|k},u_{\cdot|k},s_k)$ s.t.~\eqref{eq:fullnonlineartrackingmpconelayer} is feasible with $[C_i,0]s_k=p\}$ and accordingly $\mathbb{F}_{k}=\bigcup_{i=1}^{M}\mathbb{F}_{i,k}$. 
The theoretical results are summarized in the following theorem, which is proven in Appendix~\ref{subsec:onelayerunknownenvironmenttheory}. 
\begin{theorem} Let  Assumptions~\ref{assumption:dynamics},~\ref{assumption:expocostcontrollability},~\ref{assumption:boundedbyd},~\ref{assumption:bayesianconvergence},~\ref{assumption:ballwithminconvex} and~\ref{assumption:steadystatedecreaseonelayer} hold regarding the target cost defined in~\eqref{eq:fullnonlineartrackingmpccostonelayerlearning}, and consider a horizon length $N \geq N^{*}$. Additionally, suppose that at time step $k = 0$, the MPC problem described in~\eqref{eq:fullnonlineartrackingmpconelayer} is feasible for all agents. Then, the overall control problem according to Algorithm~\ref{alg:onelayerlearningalg} is recursively feasible, the agents do not collide and satisfy $x_{i,k} \in \mathbb{X}_{i},\; u_{i,k} \in \mathbb{U}_{i}, \forall i \in \{1,\hdots,M\}$, $\forall k \in \mathbb{N}$. The partition update condition given by~\eqref{eq:feasibilitycond} is fulfilled after a finite amount of time, the density estimate converges in probability, and the agents' position configuration converges to a centroidal Voronoi partition with respect to the converged density estimate $\hat{\phi}_{\infty}$. Furthermore, there exists a uniform constant $\Delta H\geq 0$, such that
$\sup_{p\in\mathbb{F}_k}\lim_{k\rightarrow\infty} \mathrm{Var}_k(p)\leq \frac{ \Delta H}{S}$, in probability. 
\label{theorem:onelayerlearning}
\end{theorem}
\vspace{-0.1em}
\noindent In accordance to Theorem~\ref{theorem:onelayerlearning}, the remaining uncertainty by the time of convergence can be tuned by altering the variance scaling factor $S$ introduced in equation~\eqref{eq:fullnonlineartrackingmpccostonelayerlearning}, as well as the controller horizon length $N$, and hence the set of feasible positions $\mathbb{F}_{i,k}$.
\vspace{-0.3em}
\section{Experimental Results}
\label{sec:experiments}
In the following section, we introduce the mathematical set-up of our experiments (Section~\ref{subsec:mathematicalsetup} and~\ref{subsec:expermpccost}) and provide details on the used software and hardware framework (Section~\ref{subsec:hardwaredetails}). Furthermore, the obtained experimental hardware results for both control architectures (one-layer and two-layers) and for both known and unknown environments are presented in Sections~\ref{subsec:experimenttwolayer}-~\ref{subsec:experimentonelayerlearning}.
\vspace{-0.6em}
\subsection{Mathematical Set-up}
\label{subsec:mathematicalsetup}
We consider a fleet of $M$=4 cars that covers an area $\mathbb{A}= [-1.55,1.55] \times [\rahel{-1.55,1.55}]$ meters. The nonlinear continuous dynamics for the $i-$th car are given by the kinematic bicycle model~\cite{Rajamani2012}, 
\begin{equation}
\begin{bmatrix}
    \dot{x}_{p,i} \\
    \dot{y}_{p,i} \\
    \dot{\psi_{i}}\\
    \dot{\delta_{i}}
    \end{bmatrix} = 
\begin{bmatrix}
    \cos{(\psi_{i})} \\
    \sin{(\psi_{i})} \\
    \frac{1}{L_i} \cdot \tan{(\delta_{i})}\\
    0
\end{bmatrix} u_{v,i} + 
\begin{bmatrix}
    0 \\
    0 \\
    0\\
    1
\end{bmatrix} u_{d,i} = g_{1,i} u_{v,i} + g_{2,i} u_{d,i}
\label{eq:dynamics}
\end{equation}
where $p_{i} =[x_{p,i},y_{p,i}]^{\top} \in \mathbb{R}^{2}$ is the position in Cartesian coordinates, and~$\psi_{i}$ its orientation with respect to the $x$-axis. The orientation of the wheels with respect to their neutral position is given by~$\delta_{i}$. Steering~$u_{d,i}$ and velocity~$u_{v,i}$ are available as inputs. Lastly, we denote with $L_i$ the wheelbase length. The discrete model is obtained using the forward Euler method, with a sampling time $T_{s}$. It is set to 0.05 seconds for all experiments, except for the one-layer, learning-based approach, where it has been set to 0.1 seconds due to the increased complexity of its target cost. The coverage is performed in accordance to density $\phi_{1}$ in case of a known environment, and $\phi_{2}$ for the experiments in which the environmental information is initially unknown, where
\begin{align}
    &\phi_{1}(x_p,y_p) = 5.0e^{-0.5((x_{p}-1.4)^2 + (y_{p}-1.7)^2)} \\
    &\phi_{2}(x_p,y_p) = -0.5x^{2} - 0.5y^{2} - 0.5x - 0.5y + 12. 
\end{align}
The cars are able to collect noisy measurements of the density at their current location according to~\eqref{eq:measmod}, with $\nu_{i,h} \sim \mathcal{N} (0,0.1) \ \forall i \in \{1,...,4\}$. For the application of the Bayesian linear regression, the feature vector is chosen as $\Phi = [x_p^2  \: y_p^2  \: x_p  \: y_p  \: 1]^{\top}$ and the Gaussian prior is set to $\mu_0 = [0  \: 0  \: 0  \: 0  \: 0]^{\top}$ and $\Sigma_0 = \mathbb{I}$.

\subsection{MPC Cost Function}
\label{subsec:expermpccost}
In consideration of the non-holonomic system dynamics presented in Section~\ref{subsec:mathematicalsetup}, the stage cost of the MPC is designed according to~\cite{Rosenfelder2021}. Therefore, the Lie brackets of the vectors $g_{1,i}$ and $g_{2,i}$, introduced in \eqref{eq:dynamics}, given by
\begin{align}
    & g_{3,i} := [g_{1,i},g_{2,i}] = \begin{bmatrix} 0 & 0 & \frac{1}{L_i}(\tan^{2}{(\delta_{i})} + 1) & 0 \end{bmatrix}^{\top} \notag \\
    & g_{4,i} := [g_{1,i},g_{3,i}] = \begin{bmatrix} \frac{\sin{(\psi_{i})}}{L_i(\sin^{2}{(\delta_{i})}-1)} & \frac{\cos{(\psi_{i})}}{L_i\cos^{2}{(\delta_{i})}} & 0 & 0 \end{bmatrix}^{\top}, \notag
\end{align}
are used to build a stage cost that allows for parallel parking manoeuvres. Concerning a full dimensional state and input reference, indicated by $x_{r,i}$ and $u_{r,i}$, the stage cost reads as
\begin{align}
    &\ell_{i} = \sum_{j = 1}^{4}Q_{j}(g_{j,i}^{\top}(x_{r})(x_{i}-\bar{x}_{i}))^{\rho_{j}} + \sum_{z = 1}^{2}R_{z}(u_{z,i}-\bar{u}_{z,i})^{\rho_{u}}. \notag 
\end{align}
By setting~$\rho_{1}= \rho_{2} = \rho_{u} = 12$,~$\rho_{3} = 6 $ and~$\rho_{4} = 4$, the stage cost satisfies the cost-controllability condition in Assumption~\ref{assumption:expocostcontrollability}, as shown in~\cite{Rosenfelder2021} based on~\cite{Coron2020}. Hence, the derived stability guarantees are applicable. The reference location is either set to the pre-calculated reference (two-layers approaches), or the previously calculated setpoint, (one-layer approaches). The reference in orientation,~$\psi_{r,i}$, is set to the angle enclosed by the x-axis and the error vector to the reference, and~$\delta_{r,i}$ is chosen equal to zero for all agents. While the stage cost is used for all of the presented approaches, the implemented target costs differ. In consideration of Assumption~\ref{assumption:steadystatedecreasetwolayers}, the target cost of the one-layer approaches is formed using the same structure,
\begin{align}
    &\ell_{T,i} \! = \! \sum_{j = 1}^{4}Q_{j}^{\prime}(g_{j,i}^{\top}(x_{r})(\bar{x}_{i}-x_{r,i}))^{\rho_{j}} \! + \! \sum_{z = 1}^{2}R_{z}^{\prime}(\bar{u}_{z,i}-u_{r,z,i})^{\rho_{u}}. \notag
\end{align}
For the one-layer approaches, the target cost follows the structures described in Sections~\ref{subsec:onelayer} and~\ref{subsec:onelayerlearning}. To speed up computation we approximate the included integral with a quadratic function whose coefficients are learned using Bayesian linear regression. The controller's horizon length is set to $N=30$ for all conducted experiments except for the one-layer learning-based approach where, due to the doubled sampling time, it is divided in half and set to $N=15$.

\subsection{Software and Hardware Details}
\label{subsec:hardwaredetails}
For the experiments, the miniature RC cars of Chronos in combination with the CRS software framework are used~\cite{carron2022}. The vehicles' position, velocity and orientation feedback is provided by a \rahel{Qualisys} motion capture system, and the control inputs are transmitted using WiFi-connected micro-controllers. \rahel{A HP NotebookPro with 32 GB RAM and a Dual-Core Intel Core i7 processor is used as a server. Its operating system is Ubuntu 22.04.} The closed-loop system is implemented using ROS (Robotics Operating System) embedded in a C++/ Python framework using ACADOS as a solver~\cite{Verschueren2019,Verschueren2018}. The approximate initial positions and orientations, as well as the car's wheelbase length, are given in Table \ref{tab:model_param}. The radius covering each agent $i$ is set to its length, i.e., $r_{i,\max} = L_i$.

\begin{table}[h]
\begin{center}
\vspace{1ex}
\begin{tabular}{l|cccccc}
\hline car & $x_{p,i,0}$ & $y_{p,i,0}$ & $\psi_{i,0}$ & $\delta_{i,0}$ & $L_{i}$ \\ \hline  \hline
1  & \rahel{-1.30m} & \rahel{-1.36m} & 0.785 rad & 0 rad & \rahel{0.099m} \\
2 & \rahel{-0.96m} & \rahel{-0.92m} & 0.785 rad & 0 rad & \rahel{0.099m} \\
3 & \rahel{-0.87m} & \rahel{-1.33m} & 0.000 rad & 0 rad & \rahel{0.115m} \\
4  & \rahel{-1.34m} & \rahel{-0.92m} & 1.570 rad & 0 rad & \rahel{0.115m}  \\
\hline
\end{tabular}
\caption{Values of initial configuration and model parameters of miniature RC cars for all conducted experiments.}
\label{tab:model_param}
\end{center}
\end{table}

\vspace{-2.6em}
\subsection{Results Two-Layers Coverage MPC}
\label{subsec:experimenttwolayer}
As shown in Figure~\ref{fig:twolayercoveragecost}, applying Algorithm~\ref{alg:twolayermpcalg} results in a decrease of the locational optimization cost $H(p,\mathbb{W})$ defined in~\eqref{eq:cortescost}. We further display in Figure~\ref{pics:twolayersconfig} the agents' location, traveled paths, and predicted trajectories, as well as the current Voronoi partitions and their centroids, for three instances of time during operation. \rahel{It can thus be seen that the locational optimization cost is decreasing}.    
\begin{figure} [h]
\centering
% This file was created by matlab2tikz.
%
%The latest updates can be retrieved from
%  http://www.mathworks.com/matlabcentral/fileexchange/22022-matlab2tikz-matlab2tikz
%where you can also make suggestions and rate matlab2tikz.
%
\definecolor{mycolor1}{rgb}{0.47000,0.67000,0.19000}%
%
\begin{tikzpicture}

\begin{axis}[%
height=0.85in,
width=0.40\textwidth,
yshift=0.8cm,
at={(0.0in,0.0in)},
scale only axis,
xmin=0,
xmax=25.2,
xlabel style={font=\color{white!15!black}},
xlabel style={font=\footnotesize},
xlabel style={yshift=0.6ex,},
xlabel={Time [sec]},
ymin=0,
ymax=80,
ylabel style={font=\color{white!15!black}},
ylabel style={font=\footnotesize},
ylabel style={xshift=0.6ex,},
ylabel={H(p,$\mathbb{W}$,$\phi$)},
axis background/.style={fill=white},
tick label style={font=\footnotesize},
xmajorgrids,
ymajorgrids,
legend style={legend cell align=left, align=left, draw=white!15!black},
legend style={font=\footnotesize}
]
\addplot [color=mycolor1, line width=1.4pt]
  table[row sep=crcr]{%
-0.136847496032715	65.7147318811555\\
%-0.0371048450469971	65.7141025791914\\
%0.0620768070220947	65.7176653673454\\
%0.161469697952271	65.714992569575\\
0.261643886566162	65.713941761781\\
%0.36226224899292	65.6652244674498\\
%0.461460828781128	65.4882893140563\\
%0.56434154510498	65.2521625540132\\
0.662614583969116	64.9424268331387\\
%0.762358427047729	64.5976259912785\\
%0.8625168800354	64.2088010780966\\
%0.962496280670166	63.8198908966269\\
1.06378626823425	63.4503842347759\\
%1.16183519363403	63.1284460205569\\
%1.26234865188599	62.7294133634805\\
%1.36193776130676	62.2576065619985\\
1.46243357658386	61.7165175247938\\
%1.56223154067993	61.1674400837633\\
%1.66147470474243	60.636496122732\\
%1.76231551170349	60.1319121713593\\
1.86265349388123	59.6351750745504\\
%1.96229076385498	59.1149756027786\\
%2.06291961669922	58.529791540016\\
%2.16290998458862	57.8991339544115\\
2.26398348808289	57.245873350798\\
%2.36443758010864	56.6084131849239\\
%2.46303391456604	56.0438375820128\\
%2.56176090240479	55.3781242516295\\
2.66133499145508	54.6572166217308\\
%2.76162910461426	53.9185408265481\\
%2.86209678649902	53.2031548060683\\
%2.96219873428345	52.5045456809734\\
3.06272625923157	51.7766843330984\\
%3.16386413574219	51.0032157193064\\
%3.26152968406677	50.2331871758604\\
%3.36196136474609	49.5596130508612\\
3.46389150619507	48.7597699321254\\
%3.56191039085388	48.0863606288969\\
%3.66256523132324	47.3643591647446\\
%3.76379084587097	46.6502217792357\\
3.86300754547119	45.9028228719395\\
%3.96175622940063	45.1227380663116\\
%4.06194043159485	44.3508637339247\\
%4.16191387176514	43.6050208193601\\
4.26197743415833	42.8673546027746\\
%4.36165142059326	42.0899316216143\\
%4.46216106414795	41.2829559652887\\
%4.56300568580627	40.539011556116\\
4.66434669494629	39.7940568288115\\
%4.76181197166443	39.0983196749146\\
%4.86113023757935	38.3345616700073\\
%4.96095561981201	37.4096805196616\\
5.06158518791199	36.6602959328186\\
%5.16249418258667	35.8861295739377\\
%5.26262211799622	35.1484551872026\\
%5.36246418952942	34.426668584906\\
5.46387887001038	33.7006766761758\\
%5.56208968162537	32.9716223334315\\
%5.66188097000122	32.3400212783282\\
%5.76174116134644	31.6349346723542\\
5.86149024963379	30.9012184512809\\
%5.96248912811279	30.1900024437948\\
%6.06258392333984	29.5142011147849\\
%6.16169762611389	28.838723857839\\
6.26238393783569	28.1569581424115\\
%6.36211895942688	27.5040193982834\\
%6.46333408355713	26.8212412745664\\
%6.56176900863647	26.1885895220867\\
6.66133618354797	25.61971445358\\
%6.76202297210693	24.9727735543982\\
%6.86214709281921	24.30491924498\\
%6.96356296539307	23.7319887004845\\
7.06277966499329	23.1667972435316\\
%7.16180324554443	22.5612716307008\\
%7.26324582099915	21.9647734648916\\
%7.36150026321411	21.3644754026979\\
7.46163249015808	20.7926761609671\\
%7.56277251243591	20.2261970317116\\
%7.66220688819885	19.7037261546474\\
%7.76242470741272	19.2174816294838\\
7.86262941360474	18.6596736239189\\
%7.96322584152222	18.1665298698935\\
%8.062007188797	17.7029894420565\\
%8.16241884231567	17.264271721128\\
8.2621328830719	16.7402689824588\\
%8.3619282245636	16.2962172863856\\
%8.46186399459839	15.8194234056849\\
%8.56250953674316	15.3751376556221\\
8.6623363494873	14.8630088707818\\
%8.76244473457336	14.3546447071225\\
%8.86219096183777	13.8812574057446\\
%8.96091961860657	13.4166692136251\\
9.06167483329773	12.9200037707895\\
%9.16328835487366	12.3992720354634\\
%9.26235842704773	11.994014173149\\
%9.36294913291931	11.5715527650693\\
9.46277642250061	11.147992468496\\
%9.56219148635864	10.7389957570298\\
%9.66214561462402	10.3262533971241\\
%9.76171827316284	9.94120674718422\\
9.86226558685303	9.52656332560766\\
%9.96249175071716	9.16619668388716\\
%10.0628128051758	8.78337678310621\\
%10.1626462936401	8.44873822492959\\
10.262323141098	8.11607069832383\\
%10.3620719909668	7.79208753627541\\
%10.4613265991211	7.47751042963883\\
%10.5636489391327	7.16069796147218\\
10.661961555481	6.89549508810877\\
%10.762024641037	6.58218501431248\\
%10.8623445034027	6.29934793931739\\
%10.9647569656372	6.01378123974181\\
11.0619549751282	5.78970202737477\\
%11.161342382431	5.57234014007227\\
%11.2640092372894	5.35796188948003\\
%11.3620743751526	5.15227489857847\\
11.4620730876923	4.90880380392596\\
%11.5623366832733	4.72267154373488\\
%11.6637694835663	4.55998399091304\\
%11.7639300823212	4.34149585481214\\
11.863787651062	4.14346999719322\\
%11.9626297950745	3.98427028495415\\
%12.0626294612885	3.84170425759642\\
%12.1619465351105	3.68831681043112\\
12.263040304184	3.5420274360348\\
%12.3631663322449	3.41029206627869\\
%12.4637322425842	3.31389160827643\\
%12.5636265277863	3.2372010530717\\
12.6638104915619	3.1220137787308\\
%12.7621366977692	3.02932255833074\\
%12.8632388114929	2.97316530324849\\
%12.9620780944824	2.92116855319453\\
13.0626950263977	2.90044294724279\\
%13.1630253791809	2.87071800257281\\
%13.2646701335907	2.84723777919927\\
%13.364804983139	2.82723338637856\\
13.4636492729187	2.81366682728727\\
%13.5642302036285	2.81612127118996\\
%13.663932800293	2.81964128837875\\
%13.7629618644714	2.8273952925911\\
13.8622810840607	2.83849221471343\\
%13.9622378349304	2.82096295888399\\
%14.0628867149353	2.81104663839387\\
%14.1643960475922	2.79646414953051\\
14.2636156082153	2.77711064007257\\
%14.3626174926758	2.76308570375007\\
%14.4630918502808	2.75299838592583\\
%14.5635931491852	2.76259967418067\\
14.6629374027252	2.75746648531866\\
%14.7625286579132	2.7442219503865\\
%14.8634414672852	2.7214507786205\\
%14.9644274711609	2.71415361681687\\
15.0634999275208	2.72049891209565\\
%15.1640720367432	2.72566010588011\\
%15.2647752761841	2.70037150087345\\
%15.3630213737488	2.66842346762632\\
15.4634771347046	2.6278762493031\\
%15.5639541149139	2.62083262807003\\
%15.6644651889801	2.61033608851626\\
%15.7641949653625	2.61791616341151\\
15.8636064529419	2.60432503603001\\
%15.9632527828217	2.5881850124139\\
%16.0621263980865	2.58652627962028\\
%16.1643273830414	2.61006969790038\\
16.2650966644287	2.60488605102246\\
%16.3625254631042	2.5893136121987\\
%16.4633474349976	2.56757449236355\\
%16.5644986629486	2.55718418250969\\
16.6632504463196	2.56857143037381\\
%16.7634224891663	2.58868140818764\\
%16.8623278141022	2.59204949386819\\
%16.9642863273621	2.58193999461847\\
17.0650916099548	2.56251422347987\\
%17.1624326705933	2.54988377924405\\
%17.2625813484192	2.55729591713552\\
%17.3641028404236	2.57599618744149\\
17.4646759033203	2.58587245850263\\
%17.5623912811279	2.58142922998299\\
%17.6642105579376	2.56634292611161\\
%17.7630805969238	2.54649731128394\\
17.8621718883514	2.54776205886609\\
%17.9642775058746	2.56613795535787\\
%18.0626151561737	2.58593375350397\\
%18.163590669632	2.58445220784649\\
18.2646324634552	2.58331441396454\\
%18.3645942211151	2.59412779035123\\
%18.4627721309662	2.60627852774233\\
%18.5626373291016	2.59244107287249\\
18.6641886234283	2.56079005396539\\
%18.7632658481598	2.52896174848355\\
%18.8632364273071	2.51690753341883\\
%18.9634046554565	2.51458541593067\\
19.0637781620026	2.51438374877378\\
%19.1628761291504	2.49478550620986\\
%19.2616717815399	2.48114561232696\\
%19.3626668453217	2.44740539097551\\
19.4627630710602	2.435021320006\\
%19.5639314651489	2.42438853633271\\
%19.6643121242523	2.42467011648091\\
%19.7649931907654	2.40872958334131\\
19.8631455898285	2.38581667166785\\
%19.9641597270966	2.35141537636668\\
%20.0627551078796	2.32360954175013\\
%20.1643521785736	2.31014430906271\\
20.2628207206726	2.30238492994856\\
%20.3643596172333	2.30105616579606\\
%20.4624674320221	2.30564522956168\\
%20.5642185211182	2.28550979352876\\
20.6629610061646	2.2650666734644\\
%20.7624547481537	2.22614651591165\\
%20.8629891872406	2.20945883944709\\
%20.9629504680634	2.22515994690902\\
21.0651729106903	2.20579302941369\\
%21.1630001068115	2.17778417403491\\
%21.2640700340271	2.15174184919369\\
%21.3629946708679	2.14624239651952\\
21.4653496742249	2.13897968806679\\
%21.5631816387177	2.11192245007102\\
%21.6641440391541	2.08570974120647\\
%21.7636394500732	2.07397927855536\\
21.8637981414795	2.07950273578768\\
%21.9664578437805	2.05928867237268\\
%22.0653104782104	2.02511564814404\\
%22.1647293567657	2.0016273409874\\
22.2668449878693	1.98745614641427\\
%22.3628029823303	1.97425669457265\\
%22.4630630016327	1.95632998106732\\
%22.5624330043793	1.9516411195511\\
22.6629841327667	1.95870424911883\\
%22.7641706466675	1.96436500010176\\
%22.864452123642	1.95987263222597\\
%22.9645540714264	1.95032461113674\\
23.0635523796082	1.9415329117847\\
%23.1631093025208	1.93705280433436\\
%23.2643055915833	1.93844141681833\\
%23.3625559806824	1.94655513139405\\
23.4628214836121	1.94698298183309\\
%23.5630297660828	1.93808588852657\\
%23.6638770103455	1.92368155026377\\
%23.7644026279449	1.91115163224994\\
23.8648905754089	1.91502075048318\\
%23.9630286693573	1.91368182581174\\
%24.0629198551178	1.91135769990123\\
%24.1633512973785	1.91765597555124\\
24.264963388443	1.92656114692747\\
%24.3635349273682	1.92604928293205\\
%24.4638168811798	1.92763295517115\\
%24.5639774799347	1.93015998619557\\
24.6638417243958	1.92517452167286\\
%24.7640142440796	1.91869268268013\\
%24.8629016876221	1.92332013066937\\
%24.9625117778778	1.9293942652426\\
25.0634026527405	1.92990211541636\\
%25.1637403964996	1.92068073597409\\
%25.264505147934	1.92620936136289\\
%25.3662180900574	1.93361186419332\\
25.4639046192169	1.92712631971054\\
%25.5626134872437	1.92348743631465\\
%25.6630389690399	1.92890187838976\\
%25.763174533844	1.93425375577921\\
25.8641333580017	1.92875745826823\\
};
\addlegendentry{H(p,$\mathbb{W}$,$\phi$)}

\end{axis}
\end{tikzpicture}%
\vspace{-0.7em}
\caption{\rahel{Locational optimization cost decrease over time, applying Algorithm~\ref{alg:twolayermpcalg} with respect to a known density $\phi_{1}$.}}
\label{fig:twolayercoveragecost}
\vspace{-0.6em}
\end{figure} 
\begin{figure}[h!]
\begin{minipage}[t]{0.15\textwidth}
\centering
\includegraphics[trim={5.2cm 8.1cm 4.6cm 7.7cm},clip, width = 0.8\textwidth]{figures/config_7sec_two_layer_1_0_take_one_compressed.pdf}
\centering
\end{minipage}
\hfill
\begin{minipage}[t]{0.15\textwidth}
\centering
\includegraphics[trim={5.2cm 8.1cm 4.6cm 7.7cm},clip, width = 0.8\textwidth]{figures/config_13sec_two_layer_1_0_take_one_compressed.pdf}
\centering
\end{minipage}
\hfill
\begin{minipage}[t]{0.15\textwidth}
\centering
\includegraphics[trim={5.2cm 8.1cm 4.6cm 7.7cm},clip, width = 0.8\textwidth]{figures/config_29sec_two_layer_1_0_take_one_compressed.pdf}
\centering
\end{minipage}
\caption{\rahel{Configurations of cars at 2, 8, and 24 seconds applying Algorithm~\ref{alg:twolayermpcalg} in the described set-up. The agents' location and their predicted trajectory are given in red, the Voronoi partitions in green, their centroids in blue, and the traveled paths are visualized in light grey.}}
\label{pics:twolayersconfig}
\vspace{-0.8em}
\end{figure}
\vspace{-0.9em}
\subsection{Results Two-Layers, Learning-Based Coverage MPC}
\label{subsec:experimenttwolayerlearning}
Figure~\ref{fig:twolayerlearningcoveragecost} shows the locational optimization cost and the estimated cost when applying Algorithm~\ref{alg:twolayermpcalglearningimp} with the initially unknown density~$\phi_2$. Further, Figure~\ref{pics:twolayerslearningconfig} shows the obtained configurations at three instances in time.  
\begin{figure} [h!]
\centering
% This file was created by matlab2tikz.
%
%The latest updates can be retrieved from
%  http://www.mathworks.com/matlabcentral/fileexchange/22022-matlab2tikz-matlab2tikz
%where you can also make suggestions and rate matlab2tikz.
%
\definecolor{mycolor1}{rgb}{0.47000,0.67000,0.19000}%
\definecolor{mycolor2}{rgb}{0.79000,0.49000,0.43000}%
%
\begin{tikzpicture}

\begin{axis}[%
height=0.95in,
width=0.40\textwidth,
yshift=0.8cm,
at={(0.0in,0.0in)},
scale only axis,
xmin=0,
xmax=54,
xlabel style={font=\color{white!15!black}},
xlabel style={font=\footnotesize},
xlabel style={yshift=0.6ex,},
xlabel={Time [sec]},
ymin=0,
ymax=70,
ylabel style={font=\color{white!15!black}},
ylabel style={font=\footnotesize},
ylabel style={xshift=0.6ex,},
ylabel={H(p,$\mathbb{W}$,$\phi$) \& H(p,$\mathbb{W}$,$\hat{\phi}$)},
axis background/.style={fill=white},
tick label style={font=\footnotesize},
xmajorgrids,
ymajorgrids,
legend style={legend cell align=left, align=left, draw=white!15!black},
legend style={font=\footnotesize}
]
\addplot [color=mycolor1, line width=1.4pt]
  table[row sep=crcr]{%
-0.3027024269104	55.6448905300431\\
%-0.204407453536987	55.6445605335339\\
%-0.00332498550415039	55.6444435667724\\
0.0976643562316895	55.6437842953774\\
%0.195189237594604	55.5928035929491\\
%0.295344114303589	55.3454732213708\\
0.393577814102173	55.1000471036426\\
%0.495812654495239	54.9592693546433\\
%0.595274448394775	54.7794032647213\\
0.697245836257935	54.4941562576551\\
%0.795575380325317	54.2765281989494\\
%0.896119117736816	54.1115032379665\\
0.996686458587646	53.9228724496997\\
%1.09528946876526	53.5991235923043\\
%1.19543051719666	53.3401822659869\\
1.29686903953552	53.1961710347564\\
%1.39723205566406	52.8611862970196\\
%1.49804735183716	52.5261503821488\\
1.59539771080017	52.4395732350599\\
%1.69622015953064	52.2254547327256\\
%1.79628205299377	51.7592166073933\\
1.89486718177795	51.5351045732547\\
%1.99460434913635	51.484468928725\\
%2.09470582008362	51.1522036615691\\
2.19493699073792	50.6709604988086\\
%2.29509472846985	50.506167134881\\
%2.39675807952881	50.4013856461966\\
2.49606013298035	49.9471221751262\\
%2.59892535209656	49.6830063637171\\
%2.69621014595032	49.5967288706158\\
2.79557943344116	49.2173056344831\\
%2.89679265022278	48.7679788168519\\
%2.99537801742554	48.6658861485765\\
3.09799122810364	48.4848243708458\\
%3.19628190994263	48.0326603973333\\
%3.29523468017578	47.6444569122725\\
3.39598107337952	47.6102336622782\\
%3.49721097946167	47.2822554058838\\
%3.59590768814087	46.730262687347\\
3.69483256340027	46.5593052629565\\
%3.794677734375	46.4601423558629\\
%3.89759874343872	45.9247166841674\\
3.99593710899353	45.4607610471854\\
%4.09496021270752	45.449389333006\\
%4.19439911842346	45.1961050639755\\
4.29492950439453	44.6200298416493\\
%4.39417624473572	44.4417889779748\\
%4.49639844894409	44.3942589553393\\
4.59689235687256	43.8955333958328\\
%4.69826722145081	43.3438263577577\\
%4.79972457885742	43.3141160077291\\
4.89654588699341	43.1288018307258\\
%4.99674820899963	42.5190943238922\\
%5.09615421295166	42.1507865042412\\
5.19613456726074	42.095182956867\\
%5.29454684257507	41.8442171345676\\
%5.39432430267334	41.5038320238755\\
5.50427985191345	41.2636341957749\\
%5.59798741340637	41.0859903495264\\
%5.6969621181488	40.7670877716329\\
5.79630017280579	40.554633488274\\
%5.8956778049469	40.2219904562953\\
%5.99362421035767	39.8565513969533\\
6.09515595436096	39.5915796644303\\
%6.19423890113831	39.403195706652\\
%6.29831290245056	39.1523212609064\\
6.39703941345215	38.7542019276908\\
%6.49727249145508	38.4460688756863\\
%6.59456992149353	38.3268572824327\\
6.69639205932617	38.0705991085796\\
%6.79680919647217	37.5989672592609\\
%6.89485692977905	37.3282774268723\\
6.99404382705688	37.1526960652489\\
%7.09676909446716	36.7176999464666\\
%7.19661045074463	36.2791330503225\\
7.29652333259583	36.2079436668063\\
%7.3973970413208	35.9899710321138\\
%7.49491548538208	35.4355925254983\\
7.59483218193054	35.0816851850785\\
%7.69463491439819	35.124159794816\\
%7.79643225669861	34.806821723418\\
7.89787697792053	34.1778769545131\\
%7.99566459655762	33.9885249374232\\
%8.09709501266479	33.9681748432515\\
8.1970043182373	33.4808887004695\\
%8.29438591003418	32.9359061670461\\
%8.39681267738342	32.8719570228112\\
8.49688982963562	32.7618056770042\\
%8.5955216884613	32.2521246514505\\
%8.69606304168701	31.895829641309\\
8.7971670627594	31.8695675785133\\
%8.89605140686035	31.5864529596444\\
%8.99571561813354	31.0641169092132\\
9.09747910499573	30.8761242783908\\
%9.19631099700928	30.5440618278512\\
%9.29610395431519	30.0746289273981\\
9.39601469039917	29.9224601343534\\
%9.49678158760071	29.7926011493481\\
%9.59709072113037	29.380975877275\\
9.69534707069397	29.0543933890062\\
%9.79486393928528	28.9581891941082\\
%9.8989634513855	28.6613634887856\\
9.99624752998352	28.2326692390484\\
%10.0953075885773	28.0853666202694\\
%10.1943938732147	27.9993703197415\\
10.2988548278809	27.573225022295\\
%10.3973481655121	27.2178426862035\\
%10.4963929653168	27.2285581263737\\
10.5952639579773	27.0837718313324\\
%10.695898771286	26.6026635446797\\
%10.7955617904663	26.3218186701166\\
10.8951668739319	26.361103813473\\
%10.9942538738251	26.1308623622926\\
%11.0986571311951	25.6372116215053\\
11.1985907554626	25.4942418257901\\
%11.2959842681885	25.4310598322967\\
%11.396155834198	25.1203939987004\\
11.4962804317474	24.7570022935693\\
%11.5944316387177	24.7346092270717\\
%11.6955852508545	24.6274422213127\\
11.7964098453522	24.2161010305582\\
%11.897780418396	23.9117882696265\\
%11.9974286556244	23.8950782468176\\
12.0974180698395	23.749219097194\\
%12.1979539394379	23.3887266112793\\
%12.2949485778809	23.2527766231323\\
12.3952643871307	23.2006390274161\\
%12.4969043731689	22.9881552586352\\
%12.5967862606049	22.6651154446435\\
12.6941664218903	22.5896511201456\\
%12.7970929145813	22.4961345149358\\
%12.8979876041412	22.2050631393991\\
12.9957091808319	21.9782570989638\\
%13.097021818161	21.9399678503182\\
%13.1961116790771	21.7855150497665\\
13.2949576377869	21.5146547018269\\
%13.3957409858704	21.4002833264466\\
%13.4961285591125	21.3529342416283\\
13.5980474948883	21.1332241175198\\
%13.696790933609	20.897822130659\\
%13.7954096794128	20.8313766240195\\
13.8983070850372	20.7784771839211\\
%13.996826171875	20.5365005063936\\
%14.094925403595	20.3075359565997\\
14.1947116851807	20.2875992691348\\
%14.2948770523071	20.1996192660744\\
%14.3961243629456	19.9507127779328\\
14.4968063831329	19.7819338613906\\
%14.5953252315521	19.7834656381039\\
%14.6954717636108	19.6845564094897\\
14.7958970069885	19.4557169694763\\
%14.896258354187	19.3717259228931\\
%14.9960539340973	19.3622553032533\\
15.0969412326813	19.2135513830366\\
%15.1948711872101	19.0098399521646\\
%15.2987542152405	18.9756432208449\\
15.3961992263794	18.9492904995139\\
%15.4965529441833	18.7889789489075\\
%15.5953741073608	18.6200455573016\\
15.6950442790985	18.6328935337581\\
%15.7977945804596	18.612722751482\\
%15.8981750011444	18.4834942672792\\
15.996687412262	18.3655074631898\\
%16.0970618724823	18.3425748638263\\
%16.1949391365051	18.284904729428\\
16.2953879833221	18.1473077250693\\
%16.3949494361877	18.0685003658919\\
%16.4954373836517	18.0661033399415\\
16.5964007377625	17.9345260145572\\
%16.6946594715118	17.8289760564213\\
%16.796350479126	17.8639786670093\\
16.8955898284912	17.8490576628005\\
%16.9947648048401	17.7503466778932\\
%17.100373506546	17.6795435510801\\
17.19553399086	17.6762758101225\\
%17.2948462963104	17.6712174869333\\
%17.3937425613403	17.6231634882735\\
17.4970827102661	17.5671388817921\\
%17.5959758758545	17.5551370875548\\
%17.6955008506775	17.5474042395112\\
17.7950415611267	17.4937708149664\\
%17.8959901332855	17.4505794195529\\
%17.9963026046753	17.4681950153301\\
18.0950508117676	17.4755228702488\\
%18.1964402198792	17.3898572890724\\
%18.2948694229126	17.3289815251424\\
18.3968381881714	17.3788017287155\\
%18.4950091838837	17.3789851948895\\
%18.5958671569824	17.2960415922313\\
18.6950407028198	17.2825296234331\\
%18.7952744960785	17.347757985054\\
%18.8955545425415	17.3370174525574\\
18.9941639900208	17.271671228115\\
%19.0975036621094	17.2908970371561\\
%19.1973111629486	17.3565584094633\\
19.2952461242676	17.3578547590392\\
%19.3976235389709	17.3095195014094\\
%19.4953064918518	17.3379129376196\\
19.5960757732391	17.4083711915999\\
%19.6974213123322	17.4368600707278\\
%19.7959299087524	17.4063103036257\\
19.8959774971008	17.4331347654418\\
%19.9968333244324	17.5150397748626\\
%20.0981433391571	17.5666586166457\\
20.1953341960907	17.5527309814203\\
%20.2971794605255	17.5865068807457\\
%20.3956689834595	17.6830607310154\\
20.497052192688	17.7609806974695\\
%20.595240354538	17.7559638917269\\
%20.6994514465332	17.78639029246\\
20.795618057251	17.9135400753972\\
%20.8962006568909	18.0307546553221\\
%20.9973442554474	18.0116785189629\\
21.0969412326813	18.0602372685287\\
%21.1951310634613	18.2074210638318\\
%21.2955241203308	18.3079184349188\\
21.3965122699738	18.2876240689908\\
%21.4959852695465	18.3715341374918\\
%21.5969209671021	18.5252381416092\\
21.6978921890259	18.6039742680351\\
%21.7961347103119	18.6371614727788\\
%21.8951568603516	18.7412132194337\\
21.995100736618	18.880962317914\\
%22.0968828201294	18.9786655478155\\
%22.1957552433014	19.043324850093\\
22.2985866069794	19.1701990614172\\
%22.3976767063141	19.3039822531437\\
%22.497936964035	19.4298057055756\\
22.5984561443329	19.532245319084\\
%22.6955976486206	19.5971089410902\\
%22.795291185379	19.7535792022972\\
22.8961069583893	19.9298450035625\\
%22.9963226318359	19.9578658892241\\
%23.0974667072296	20.02350240621\\
23.1954870223999	20.2141354405193\\
%23.2972660064697	20.4422449838679\\
%23.3962876796722	20.4956840556837\\
23.4945378303528	20.548665865367\\
%23.5949668884277	20.8387881946963\\
%23.6956467628479	21.1032766659025\\
23.7949657440186	21.0836803780203\\
%23.8968379497528	21.1831917237771\\
%23.9955539703369	21.502217506946\\
24.095534324646	21.7365318895982\\
%24.195175409317	21.6976207120639\\
%24.2950949668884	21.8027590924678\\
24.397723197937	22.1607080040078\\
%24.4962141513824	22.392713338985\\
%24.5954253673553	22.4529394619299\\
24.6950843334198	22.5980328694538\\
%24.796327829361	22.8019938521711\\
%24.8997488021851	22.9319479902249\\
24.995391368866	23.0847925049269\\
%25.0983858108521	23.3163428653441\\
%25.1943018436432	23.4084759374247\\
25.2951481342316	23.4917806353248\\
%25.3948314189911	23.7298734345565\\
%25.4972231388092	23.9597739402717\\
25.5975432395935	24.0031841391873\\
%25.6968541145325	24.111367203753\\
%25.7961668968201	24.3877497827233\\
25.8950951099396	24.63912161174\\
%25.9958007335663	24.6642686518395\\
%26.0963604450226	24.7922575875582\\
26.1970686912537	25.1580919499532\\
%26.2952060699463	25.4070173067177\\
%26.3958468437195	25.3915230771492\\
26.4963295459747	25.5562526075687\\
%26.5961065292358	25.9566293013279\\
%26.6963200569153	26.1496623036272\\
26.7971513271332	26.1504284804323\\
%26.8980207443237	26.4516213191749\\
%26.9955034255981	26.8431182510907\\
27.0978033542633	26.9428176180782\\
%27.1978311538696	27.0435238953561\\
%27.2983016967773	27.4594765085467\\
27.3952560424805	27.7527403622506\\
%27.4951612949371	27.9039670868909\\
%27.5959136486053	28.219559320064\\
27.697301864624	28.5434008224991\\
%27.7966742515564	28.6621928349859\\
%27.8960103988647	28.994819399259\\
27.9982404708862	29.4039534854098\\
%28.0984890460968	29.4900844628064\\
%28.1960735321045	29.6250926093525\\
28.2973930835724	30.0553630946016\\
%28.3956468105316	30.4319092479141\\
%28.4950547218323	30.4558999834994\\
28.5952968597412	30.6326138855797\\
%28.6980776786804	31.1707111445093\\
%28.7960538864136	31.3970328885956\\
28.8973968029022	31.3892755741239\\
%28.9953141212463	31.6064449604207\\
%29.0977954864502	32.0624435888716\\
29.1961948871613	32.1288823431368\\
%29.2956788539886	32.2030523671112\\
%29.3945882320404	32.6169103111231\\
29.4956109523773	32.672753494854\\
%29.5962133407593	32.7321979822599\\
%29.6969833374023	33.1075018500233\\
29.7962291240692	33.2342450281331\\
%29.8951134681702	33.2780230246289\\
%29.9963798522949	33.6437791955147\\
30.0954389572144	33.9277226792639\\
%30.1946218013763	33.9168989481429\\
%30.2983458042145	34.1742054341637\\
30.397833108902	34.6429456885098\\
%30.496143579483	34.7256949558631\\
%30.5962183475494	34.726425539417\\
30.6954398155212	35.0642479832341\\
%30.796008348465	35.4153889102323\\
%30.8968033790588	35.5806085999589\\
30.9957027435303	35.7671852425061\\
%31.0974082946777	36.1717436258936\\
%31.1986017227173	36.4677214515119\\
31.2970654964447	36.5004903584966\\
%31.3960077762604	36.9803466546089\\
%31.4961841106415	37.542298685413\\
31.5973048210144	37.6083837107639\\
%31.6965961456299	37.9473651080987\\
%31.7958209514618	38.4262105052504\\
31.8982129096985	38.622800152485\\
%31.9969327449799	38.7347143743279\\
%32.0949926376343	39.3361713580005\\
32.1962532997131	39.8258517897591\\
%32.2969284057617	39.8247635814065\\
%32.3958101272583	40.0573609811841\\
32.4963457584381	40.6141116326489\\
%32.5960640907288	40.9860244539333\\
%32.6969578266144	40.9795911951437\\
32.7965381145477	41.3383254810733\\
%32.8974368572235	42.0254427027297\\
%32.9969608783722	42.2487570831257\\
33.098441362381	42.1799964367878\\
%33.1956195831299	42.5350738386219\\
%33.2960820198059	43.1708051971815\\
33.3965842723846	43.2217165676186\\
%33.4956648349762	43.2317845975817\\
%33.5960750579834	43.5916854204635\\
33.6956932544708	43.9494001431193\\
%33.7968392372131	43.9281760965428\\
%33.8974401950836	16.0190941272265\\
33.9979965686798	16.1246272211347\\
%34.0977871417999	16.1315690609387\\
%34.1991693973541	16.140726870486\\
34.2988502979279	16.2429527059587\\
%34.3993678092957	16.3200851604808\\
%34.4985780715942	16.3523483660124\\
34.5994029045105	16.3790833519397\\
%34.6982367038727	16.3453708652284\\
%34.7970509529114	16.3638833424942\\
34.8972249031067	16.4863502144478\\
%35.0004985332489	16.5326647907524\\
%35.0996613502502	16.4763984030944\\
35.1983637809753	16.5058858301774\\
%35.2983539104462	16.6326860455021\\
%35.3992595672607	16.6786359036432\\
35.4964368343353	16.5990567193105\\
%35.5987293720245	16.514429385273\\
%35.6982982158661	16.4674428104906\\
35.7991690635681	16.5192454306228\\
%35.8992547988892	16.6120128848222\\
%35.9990017414093	16.5741460225683\\
36.0990765094757	16.36787959314\\
%36.1988270282745	16.2655548062029\\
%36.2991209030151	16.302521932675\\
36.3967835903168	16.2317012579996\\
%36.4997007846832	16.0023543203535\\
%36.5963776111603	15.9257551015501\\
36.7007505893707	15.9605588719857\\
%36.79927110672	15.8590275544143\\
%36.8988215923309	15.655162781935\\
36.99835896492	15.5812157503144\\
%37.0977628231049	15.599078903617\\
%37.1978387832642	15.4306440509927\\
37.2986805438995	15.2220231681066\\
%37.3977646827698	15.1656509544441\\
%37.4994828701019	15.1276164086888\\
37.5977094173431	14.9688564213548\\
%37.7001557350159	14.796677094295\\
%37.7987766265869	14.7256305203417\\
37.8987610340118	14.6724508238382\\
%37.9977929592133	14.5247029065001\\
%38.0994288921356	14.3435585865855\\
38.1991164684296	14.2831708333369\\
%38.2976996898651	14.2266132561967\\
%38.3978955745697	14.0638967011326\\
38.4974372386932	13.9152338950045\\
%38.5984590053558	13.8463645291066\\
%38.6983511447906	13.7738220111082\\
38.7975261211395	13.6037293294205\\
%38.8969886302948	13.4879050647834\\
%38.9986202716827	13.4363914452042\\
39.0978474617004	13.3378028836567\\
%39.1986021995544	13.1712001918453\\
%39.2987122535706	13.0417929212527\\
39.3982963562012	12.9398400332076\\
%39.4985549449921	12.8449985541884\\
%39.5998070240021	12.7194171944511\\
39.6968386173248	12.5674306514131\\
%39.7995557785034	12.4519683905429\\
%39.8975603580475	12.3686050865702\\
39.9990828037262	12.2496980447172\\
%40.0985217094421	12.1071025575814\\
%40.1979424953461	11.9872525462325\\
40.2979881763458	11.8826043450243\\
%40.3978669643402	11.7797593963084\\
%40.4979383945465	11.6516247812249\\
40.5992522239685	11.5185676771682\\
%40.6984276771545	11.4198139986561\\
%40.7988903522491	11.3366346695251\\
40.8982491493225	11.2075299354916\\
%40.9970371723175	11.0609615740811\\
%41.097158908844	10.9799147837584\\
41.1970839500427	10.9038382120088\\
%41.3006472587585	10.7721256543357\\
%41.3970010280609	10.6388694112962\\
41.4979462623596	10.5644802802808\\
%41.5988345146179	10.5003617264669\\
%41.6986339092255	10.3636528545828\\
41.7987580299377	10.2105014549673\\
%41.898280620575	10.1512274361449\\
%41.9970042705536	10.0885675478318\\
42.0969915390015	9.96149616974038\\
%42.1989109516144	9.81717194857299\\
%42.2988197803497	9.77281360894598\\
42.3990840911865	9.72639808124173\\
%42.4996585845947	9.60156119015831\\
%42.5977330207825	9.45692128997169\\
42.6975464820862	9.40577250052448\\
%42.7976016998291	9.37231873583881\\
%42.8980731964111	9.23936240780032\\
42.9992232322693	9.0745409867368\\
%43.099645614624	9.0492196123343\\
%43.1977679729462	9.03795783700015\\
43.3005847930908	8.91427329662141\\
%43.3998730182648	8.76660372085676\\
%43.497976064682	8.75110237690993\\
43.5978007316589	8.75738754667072\\
%43.6980624198914	8.67329051689503\\
%43.7975356578827	8.55271898454498\\
43.9009740352631	8.53260742941788\\
%43.9991044998169	8.53503046102915\\
%44.0993976593018	8.45988359966371\\
44.1966049671173	8.37742389845512\\
%44.2981135845184	8.37256163545783\\
%44.3979256153107	8.38787590503425\\
44.4983563423157	8.34160827537734\\
%44.5976676940918	8.2638156969769\\
%44.6995811462402	8.2724339269572\\
44.7986752986908	8.29609790858319\\
%44.8985450267792	8.22069744547098\\
%44.9989039897919	8.11077018775572\\
45.0958788394928	8.09849977556838\\
%45.1988072395325	8.09791829853451\\
%45.2986025810242	8.02820773202829\\
45.4006898403168	7.98472718718479\\
%45.4995272159576	7.99293544199053\\
%45.5981070995331	7.95911789811679\\
45.6987905502319	7.92533489044842\\
%45.7980630397797	7.93487851478197\\
%45.8978490829468	7.93711958505304\\
45.9975397586823	7.91075698886097\\
%46.0997786521912	7.89027642506014\\
%46.1985127925873	7.89360412221531\\
46.2988638877869	7.89840643406655\\
%46.3977031707764	7.87732769285748\\
%46.4992260932922	7.839284071987\\
46.598242521286	7.83492383591496\\
%46.6982197761536	7.85037986389635\\
%46.7964286804199	7.82889017871074\\
46.8979287147522	7.79072043171923\\
%47.0007612705231	7.77244692476397\\
%47.099461555481	7.77315174688821\\
47.1973867416382	7.7525696478608\\
%47.2976024150848	7.73500684194044\\
%47.3962533473969	7.74458524605134\\
47.4976968765259	7.74959548473272\\
%47.5979177951813	7.72244461001897\\
%47.6981735229492	7.69899584976267\\
47.7990231513977	7.70834736975279\\
%47.8997867107391	7.71443014367402\\
%47.999951839447	7.69725464126241\\
48.0984971523285	7.69164002621035\\
%48.1977996826172	7.70434910779465\\
%48.2982051372528	7.70780851880899\\
48.397274017334	7.68943743488405\\
%48.4983186721802	7.68440391411816\\
%48.5980792045593	7.69635301991242\\
48.6991057395935	7.69135267058732\\
%48.7981367111206	7.67357633810924\\
%48.8980414867401	7.66946897225097\\
48.9978730678558	7.67591371551318\\
%49.0970058441162	7.67403782771344\\
%49.1963555812836	7.66538671740914\\
49.2969927787781	7.66131212066346\\
%49.3988502025604	7.66193391606008\\
%49.498318195343	7.66245207723119\\
49.5989005565643	7.6607420246152\\
%49.6973714828491	7.66128645236969\\
%49.797901391983	7.66116855593455\\
49.8976213932037	7.66029078330162\\
%49.9978492259979	7.65931228265287\\
%50.0978472232819	7.6590652944368\\
50.2000968456268	7.6591347735566\\
%50.3002262115479	7.65939956514082\\
%50.3988947868347	7.6588722729644\\
50.4979214668274	7.65881179889964\\
%50.5981481075287	7.65916233873136\\
%50.6984331607819	7.65876941952508\\
50.7982943058014	7.6559762958769\\
%50.9011902809143	7.64732435664369\\
%50.9977343082428	7.64630040288322\\
51.0983502864838	7.64656726874181\\
%51.1994557380676	7.64601782375085\\
%51.2987756729126	7.64602572190649\\
51.3972861766815	7.64293050369674\\
%51.4997346401215	7.63862734880052\\
%51.5964462757111	7.64037734761764\\
51.698677778244	7.64066069371714\\
%51.7978341579437	7.64029277403493\\
%51.9003474712372	7.64036005345733\\
51.9989249706268	7.64055981732716\\
%52.0975363254547	7.64036140634576\\
%52.1990420818329	7.64019848604191\\
52.2984504699707	7.640321750418\\
%52.3979661464691	7.64067503383119\\
%52.4984498023987	7.64040582886553\\
52.5998876094818	7.64032990185417\\
%52.6995544433594	7.64061629994296\\
%52.7968590259552	7.64047679999293\\
52.8977110385895	7.64023984401505\\
%52.9981262683868	7.64072051272619\\
%53.0989315509796	7.64058724114244\\
53.1972990036011	7.64061129562711\\
%53.2967021465302	7.64032117437473\\
%53.3984217643738	7.6405612311073\\
53.4997849464417	7.64070008111707\\
%53.5988943576813	7.64040224662139\\
%53.6983377933502	7.64058334114078\\
53.7982711791992	7.64052516268386\\
%53.8991680145264	7.6402561293538\\
%53.9992122650146	7.64037455706845\\
54.0977897644043	7.6406256836301\\
%54.1989953517914	7.64068252659098\\
%54.2989671230316	7.63925751176575\\
54.4001793861389	7.63870168639603\\
%54.4991703033447	7.63897823831836\\
%54.5983467102051	7.63871020920959\\
54.6985468864441	7.63877455528261\\
%54.7985153198242	7.63890613378715\\
%54.8970367908478	7.63900171878129\\
54.9963908195496	7.63877858953665\\
};
\addlegendentry{H(p,$\mathbb{W}$,$\phi$)}

\addplot [color=mycolor2, dotted, line width=1.4pt]
  table[row sep=crcr]{%
-0.3027024269104	0\\
%-0.204407453536987	0\\
%-0.00332498550415039	0\\
0.0976643562316895	0\\
%0.195189237594604	4.96115496599351\\
%0.295344114303589	6.72276650306742\\
0.393577814102173	7.73851141534208\\
%0.495812654495239	8.23510834174255\\
%0.595274448394775	8.35599520993234\\
0.697245836257935	8.4934583789001\\
%0.795575380325317	8.78674323669304\\
%0.896119117736816	9.0625464218325\\
0.996686458587646	9.28954939793286\\
%1.09528946876526	9.46060934819767\\
%1.19543051719666	9.68975721405352\\
1.29686903953552	10.0070511842665\\
%1.39723205566406	10.2957409631536\\
%1.49804735183716	10.5307687027232\\
1.59539771080017	10.8185572171799\\
%1.69622015953064	11.1448214572038\\
%1.79628205299377	11.4715954373854\\
1.89486718177795	11.8267378125789\\
%1.99460434913635	12.1989674137448\\
%2.09470582008362	12.5679065263658\\
2.19493699073792	12.9625045682983\\
%2.29509472846985	13.4272063742444\\
%2.39675807952881	13.8848478462349\\
2.49606013298035	14.3036010915053\\
%2.59892535209656	14.8865983687501\\
%2.69621014595032	15.4820220010379\\
2.79557943344116	15.9047887542883\\
%2.89679265022278	16.4010247507315\\
%2.99537801742554	17.1401423000125\\
3.09799122810364	17.7988980660689\\
%3.19628190994263	18.2680874777487\\
%3.29523468017578	18.8436839002498\\
3.39598107337952	19.6697525500844\\
%3.49721097946167	20.2901179215027\\
%3.59590768814087	20.7466619155785\\
3.69483256340027	21.4613323298946\\
%3.794677734375	22.2499662884326\\
%3.89759874343872	22.7453289204697\\
3.99593710899353	23.2105935677705\\
%4.09496021270752	23.9578015817637\\
%4.19439911842346	24.5770832509378\\
4.29492950439453	24.9664241063872\\
%4.39417624473572	25.5439085543736\\
%4.49639844894409	26.2006963679512\\
4.59689235687256	26.5735225364863\\
%4.69826722145081	26.885884932887\\
%4.79972457885742	27.4943571283801\\
4.89654588699341	27.9818140884456\\
%4.99674820899963	28.1763945718638\\
%5.09615421295166	28.5304591233387\\
5.19613456726074	29.0556476555614\\
%5.29454684257507	29.4180851159404\\
%5.39432430267334	29.7108661660236\\
5.50427985191345	30.0580882690539\\
%5.59798741340637	30.3973746682697\\
%5.6969621181488	30.6050842739941\\
5.79630017280579	30.8945985998994\\
%5.8956778049469	30.8560617784899\\
%5.99362421035767	31.187416408057\\
6.09515595436096	31.3506638147803\\
%6.19423890113831	31.5782498520671\\
%6.29831290245056	31.7457201238333\\
6.39703941345215	31.7402924491471\\
%6.49727249145508	31.7847390003458\\
%6.59456992149353	31.993937441361\\
6.69639205932617	32.0799838568742\\
%6.79680919647217	31.9381230195179\\
%6.89485692977905	31.9477595300537\\
6.99404382705688	32.0432467475086\\
%7.09676909446716	31.9087301093574\\
%7.19661045074463	31.7362089821008\\
7.29652333259583	31.8685766527528\\
%7.3973970413208	31.8738913512882\\
%7.49491548538208	31.5748981079217\\
7.59483218193054	31.4269055847694\\
%7.69463491439819	31.6172930605021\\
%7.79643225669861	31.4833139274339\\
7.89787697792053	31.0671069219518\\
%7.99566459655762	31.0283232392952\\
%8.09709501266479	31.1257634858581\\
8.1970043182373	30.7969529069337\\
%8.29438591003418	30.4188201198089\\
%8.39681267738342	30.4642970355414\\
8.49688982963562	30.4507023364314\\
%8.5955216884613	30.0701714409769\\
%8.69606304168701	29.8361871150931\\
8.7971670627594	29.8933016627378\\
%8.89605140686035	29.6961384783418\\
%8.99571561813354	29.2786907189825\\
9.09747910499573	29.1822058456369\\
%9.19631099700928	28.9310721348762\\
%9.29610395431519	28.5407995624595\\
9.39601469039917	28.4562414128904\\
%9.49678158760071	28.3955992143865\\
%9.59709072113037	28.0531196137507\\
9.69534707069397	27.7840842338102\\
%9.79486393928528	27.7389128106921\\
%9.8989634513855	27.5035069210036\\
9.99624752998352	27.1320955255859\\
%10.0953075885773	27.0250367079775\\
%10.1943938732147	26.979574335248\\
10.2988548278809	26.6092910448854\\
%10.3973481655121	26.2988279990919\\
%10.4963929653168	26.3359929029212\\
10.5952639579773	26.2252525493371\\
%10.695898771286	25.7920682471734\\
%10.7955617904663	25.5448729743454\\
10.8951668739319	25.6040785729448\\
%10.9942538738251	25.4036335010276\\
%11.0986571311951	24.950262378618\\
11.1985907554626	24.8298031148737\\
%11.2959842681885	24.7833838085437\\
%11.396155834198	24.4979208347667\\
11.4962804317474	24.1641034680546\\
%11.5944316387177	24.1567497423454\\
%11.6955852508545	24.0634851220294\\
11.7964098453522	23.676629688942\\
%11.897780418396	23.397010263287\\
%11.9974286556244	23.3938111131953\\
12.0974180698395	23.2614769767169\\
%12.1979539394379	22.9216744139025\\
%12.2949485778809	22.8033524671136\\
12.3952643871307	22.7635002094582\\
%12.4969043731689	22.5641727027049\\
%12.5967862606049	22.2589541090917\\
12.6941664218903	22.1978686620985\\
%12.7970929145813	22.1151754998746\\
%12.8979876041412	21.8368519292923\\
12.9957091808319	21.6231557475284\\
%13.097021818161	21.595425259488\\
%13.1961116790771	21.4506258643235\\
13.2949576377869	21.1906753821847\\
%13.3957409858704	21.08619202363\\
%13.4961285591125	21.0480530175875\\
13.5980474948883	20.8374923275355\\
%13.696790933609	20.610561515605\\
%13.7954096794128	20.5510670471248\\
13.8983070850372	20.5054429505054\\
%13.996826171875	20.2719719413365\\
%14.094925403595	20.050061579369\\
14.1947116851807	20.0348733950412\\
%14.2948770523071	19.9534283068576\\
%14.3961243629456	19.7121371966041\\
14.4968063831329	19.5489653363965\\
%14.5953252315521	19.5539290643327\\
%14.6954717636108	19.4604722412024\\
14.7958970069885	19.2386825486961\\
%14.896258354187	19.1587449970157\\
%14.9960539340973	19.1522202767006\\
15.0969412326813	19.0085797614333\\
%15.1948711872101	18.8112089040259\\
%15.2987542152405	18.7803231483568\\
15.3961992263794	18.7563036970664\\
%15.4965529441833	18.6003135654237\\
%15.5953741073608	18.436803031535\\
15.6950442790985	18.4515498942173\\
%15.7977945804596	18.4329614458122\\
%15.8981750011444	18.3074281945123\\
15.996687412262	18.1938833461759\\
%16.0970618724823	18.1732339118627\\
%16.1949391365051	18.1171980403924\\
16.2953879833221	17.9830018959349\\
%16.3949494361877	17.9080763178548\\
%16.4954373836517	17.9073612686181\\
16.5964007377625	17.778418835844\\
%16.6946594715118	17.6760726745834\\
%16.796350479126	17.7129814168682\\
16.8955898284912	17.6987629599674\\
%16.9947648048401	17.6015190991165\\
%17.100373506546	17.5330654620584\\
17.19553399086	17.5318657148653\\
%17.2948462963104	17.5277009582865\\
%17.3937425613403	17.4806140130188\\
17.4970827102661	17.4261956993587\\
%17.5959758758545	17.4160089932488\\
%17.6955008506775	17.4093146959278\\
17.7950415611267	17.357045980259\\
%17.8959901332855	17.3156807721789\\
%17.9963026046753	17.3346647206793\\
18.0950508117676	17.3425996427856\\
%18.1964402198792	17.258671508874\\
%18.2948694229126	17.199336125723\\
18.3968381881714	17.249913886415\\
%18.4950091838837	17.2510178271105\\
%18.5958671569824	17.1695581695858\\
18.6950407028198	17.1569300167434\\
%18.7952744960785	17.2227502769633\\
%18.8955545425415	17.2131459877813\\
18.9941639900208	17.1492945644126\\
%19.0975036621094	17.1690759860254\\
%19.1973111629486	17.2352458924515\\
19.2952461242676	17.2375761847553\\
%19.3976235389709	17.1905595761837\\
%19.4953064918518	17.2189353619779\\
19.5960757732391	17.2896549788893\\
%19.6974213123322	17.3191174045855\\
%19.7959299087524	17.2893459103113\\
19.8959774971008	17.3159650030746\\
%19.9968333244324	17.3980752126517\\
%20.0981433391571	17.4506177629771\\
20.1953341960907	17.4375083264802\\
%20.2971794605255	17.4712106833518\\
%20.3956689834595	17.5678076578443\\
20.497052192688	17.6464277558269\\
%20.595240354538	17.6423554301805\\
%20.6994514465332	17.6726630731743\\
20.795618057251	17.7998186993656\\
%20.8962006568909	17.9178785519461\\
%20.9973442554474	17.8997861147332\\
21.0969412326813	17.9481776480812\\
%21.1951310634613	18.095182616408\\
%21.2955241203308	18.1962326015314\\
21.3965122699738	18.1766243883332\\
%21.4959852695465	18.2601424229381\\
%21.5969209671021	18.4134825959472\\
21.6978921890259	18.4927799413798\\
%21.7961347103119	18.526348030004\\
%21.8951568603516	18.6298803761462\\
21.995100736618	18.7694510506435\\
%22.0968828201294	18.8677267265706\\
%22.1957552433014	18.9326243457631\\
22.2985866069794	19.0589162484097\\
%22.3976767063141	19.1924528115109\\
%22.497936964035	19.3187248048628\\
22.5984561443329	19.4211731582115\\
%22.6955976486206	19.485583176573\\
%22.795291185379	19.6416466855533\\
22.8961069583893	19.8180771693361\\
%22.9963226318359	19.8463457115353\\
%23.0974667072296	19.9113473952512\\
23.1954870223999	20.1012206698539\\
%23.2972660064697	20.3291764232019\\
%23.3962876796722	20.3828846865818\\
23.4945378303528	20.4354370465276\\
%23.5949668884277	20.7244024104695\\
%23.6956467628479	20.9885486093634\\
23.7949657440186	20.9696098846912\\
%23.8968379497528	21.068466278909\\
%23.9955539703369	21.3861445095768\\
24.095534324646	21.62026282102\\
%24.195175409317	21.5820742204687\\
%24.2950949668884	21.68645770241\\
24.397723197937	22.0427188353573\\
%24.4962141513824	22.2742630379099\\
%24.5954253673553	22.3348624182829\\
24.6950843334198	22.4793833543905\\
%24.796327829361	22.6824651084209\\
%24.8997488021851	22.8121184811293\\
24.995391368866	22.9643767498642\\
%25.0983858108521	23.1949850631457\\
%25.1943018436432	23.2869823762911\\
25.2951481342316	23.3702077440732\\
%25.3948314189911	23.6073257358399\\
%25.4972231388092	23.8362078649828\\
25.5975432395935	23.8796563543636\\
%25.6968541145325	23.9877034177335\\
%25.7961668968201	24.2629565315454\\
25.8950951099396	24.5132665394162\\
%25.9958007335663	24.5385765860521\\
%26.0963604450226	24.6663241766767\\
26.1970686912537	25.0306471213667\\
%26.2952060699463	25.2785471061533\\
%26.3958468437195	25.2634300271119\\
26.4963295459747	25.4276633155902\\
%26.5961065292358	25.8263421360842\\
%26.6963200569153	26.0187287526214\\
26.7971513271332	26.0197670012022\\
%26.8980207443237	26.3197856224826\\
%26.9955034255981	26.7097829571393\\
27.0978033542633	26.8093437672891\\
%27.1978311538696	26.9097278963179\\
%27.2983016967773	27.3240213396591\\
27.3952560424805	27.6163955370417\\
%27.4951612949371	27.7672383362594\\
%27.5959136486053	28.0815458964361\\
27.697301864624	28.4042782691416\\
%27.7966742515564	28.5229672038853\\
%27.8960103988647	28.8543019133919\\
27.9982404708862	29.2618155409241\\
%28.0984890460968	29.347994732836\\
%28.1960735321045	29.4827345276881\\
28.2973930835724	29.9112192398351\\
%28.3956468105316	30.2862844504951\\
%28.4950547218323	30.3105255114627\\
28.5952968597412	30.4866936494572\\
%28.6980776786804	31.0225214785464\\
%28.7960538864136	31.2481120839176\\
28.8973968029022	31.2406919900947\\
%28.9953141212463	31.4570333669992\\
%29.0977954864502	31.9111214840976\\
29.1961948871613	31.9775787584645\\
%29.2956788539886	32.05161554531\\
%29.3945882320404	32.4637113945536\\
29.4956109523773	32.5195838819179\\
%29.5962133407593	32.5789725301418\\
%29.6969833374023	32.9526905304623\\
29.7962291240692	33.0791119560776\\
%29.8951134681702	33.1229327667151\\
%29.9963798522949	33.4871684330157\\
30.0954389572144	33.7700549581206\\
%30.1946218013763	33.7595566046913\\
%30.2983458042145	34.015858966578\\
30.397833108902	34.4826694481014\\
%30.496143579483	34.5653379955108\\
%30.5962183475494	34.5662881254293\\
30.6954398155212	34.9027346242111\\
%30.796008348465	35.2524854579039\\
%30.8968033790588	35.4172101054724\\
30.9957027435303	35.6031595850583\\
%31.0974082946777	36.0060807687198\\
%31.1986017227173	36.3009635255781\\
31.2970654964447	36.3338375968513\\
%31.3960077762604	36.8116768135838\\
%31.4961841106415	37.3712948345259\\
31.5973048210144	37.4373887322063\\
%31.6965961456299	37.7750243856638\\
%31.7958209514618	38.2518633824985\\
31.8982129096985	38.4478086535613\\
%31.9969327449799	38.5594229765378\\
%32.0949926376343	39.1582654432094\\
32.1962532997131	39.6459256681162\\
%32.2969284057617	39.6451024387881\\
%32.3958101272583	39.8767814784516\\
32.4963457584381	40.4311079378754\\
%32.5960640907288	40.8015353130781\\
%32.6969578266144	40.7953424435819\\
32.7965381145477	41.1525476999052\\
%32.8974368572235	41.8366487839153\\
%32.9969608783722	42.0591622125977\\
33.098441362381	41.99088590334\\
%33.1956195831299	42.3444154262443\\
%33.2960820198059	42.9773379211427\\
33.3965842723846	43.028198289001\\
%33.4956648349762	43.0383381680471\\
%33.5960750579834	43.3966544619268\\
33.6956932544708	43.7528348331261\\
%33.7968392372131	43.7318280031126\\
%33.8974401950836	15.9436829012325\\
33.9979965686798	16.0486930109178\\
%34.0977871417999	16.0556723143576\\
%34.1991693973541	16.0648092367385\\
34.2988502979279	16.1664982430084\\
%34.3993678092957	16.2432545654198\\
%34.4985780715942	16.2754879291678\\
34.5994029045105	16.302173487643\\
%34.6982367038727	16.2686013308792\\
%34.7970509529114	16.2870406513309\\
34.8972249031067	16.4090320759602\\
%35.0004985332489	16.4551747812532\\
%35.0996613502502	16.3991365136086\\
35.1983637809753	16.4285081994117\\
%35.2983539104462	16.5548161191064\\
%35.3992595672607	16.6005967271033\\
35.4964368343353	16.5213391827368\\
%35.5987293720245	16.4370530165711\\
%35.6982982158661	16.3902771405291\\
35.7991690635681	16.4418918491075\\
%35.8992547988892	16.5342905026045\\
%35.9990017414093	16.4965796554706\\
36.0990765094757	16.2911712739116\\
%36.1988270282745	16.1892810859819\\
%36.2991209030151	16.2261094131704\\
36.3967835903168	16.1555814809488\\
%36.4997007846832	15.9271966642454\\
%36.5963776111603	15.8509366886479\\
36.7007505893707	15.8856121929559\\
%36.79927110672	15.7844968829899\\
%36.8988215923309	15.5814836349269\\
36.99835896492	15.5078676078706\\
%37.0977628231049	15.5256824955594\\
%37.1978387832642	15.357996420832\\
37.2986805438995	15.1502746918084\\
%37.3977646827698	15.0941683815252\\
%37.4994828701019	15.0563578785206\\
37.5977094173431	14.8982890141439\\
%37.7001557350159	14.7268300743842\\
%37.7987766265869	14.6561223156873\\
37.8987610340118	14.6032177081635\\
%37.9977929592133	14.4561092377277\\
%38.0994288921356	14.2757408499112\\
38.1991164684296	14.2156514975226\\
38.2976996898651	14.1593715089189\\
%38.3978955745697	13.9973546566231\\
%38.4974372386932	13.849340079139\\
38.5984590053558	13.7808075019489\\
%38.6983511447906	13.7086056444845\\
%38.7975261211395	13.539249818324\\
38.8969886302948	13.4239258509268\\
%38.9986202716827	13.3726632284682\\
%39.0978474617004	13.2745253280445\\
39.1986021995544	13.1086428373067\\
%39.2987122535706	12.9798129392392\\
%39.3982963562012	12.8783382852955\\
39.4985549449921	12.7839228052151\\
%39.5998070240021	12.6588853610608\\
%39.6968386173248	12.507590912543\\
39.7995557785034	12.3926713755401\\
%9.8975603580475	12.3096745863683\\
%39.9990828037262	12.1912848658449\\
40.0985217094421	12.049351858834\\
%40.1979424953461	11.9300620707118\\
%40.2979881763458	11.8258662882378\\
40.3978669643402	11.7234802472691\\
%40.4979383945465	11.5959488715877\\
%40.5992522239685	11.4634996444046\\
40.6984276771545	11.3651627845737\\
%40.7988903522491	11.2823570721191\\
%40.8982491493225	11.1538669783434\\
40.9970371723175	11.0079651200191\\
%41.097158908844	10.9272560670087\\
%41.1970839500427	10.8515336174188\\
41.3006472587585	10.7204522367095\\
%41.3970010280609	10.5878010877923\\
%41.4979462623596	10.5137263812965\\
41.5988345146179	10.4499148383606\\
%41.6986339092255	10.3138626033183\\
%41.7987580299377	10.1614019642839\\
41.898280620575	10.1023730102327\\
%41.9970042705536	10.0400126401755\\
%42.0969915390015	9.91355342411357\\
42.1989109516144	9.76987276522074\\
%42.2988197803497	9.72569848594112\\
%42.3990840911865	9.67951554553401\\
42.4996585845947	9.55526968369815\\
%42.5977330207825	9.41128133167297\\
%42.6975464820862	9.36035103209345\\
42.7976016998291	9.32706353153899\\
%42.8980731964111	9.19473490592726\\
%42.9992232322693	9.03065543048921\\
43.099645614624	9.00543892711164\\
%43.1977679729462	8.99424172781802\\
%43.3005847930908	8.87112900822277\\
43.3998730182648	8.72411596300301\\
%43.497976064682	8.70867796131803\\
%43.5978007316589	8.71494542371628\\
43.6980624198914	8.63123284791156\\
%43.7975356578827	8.51119019876741\\
%43.9009740352631	8.49116007123352\\
43.9991044998169	8.49357901011576\\
%44.0993976593018	8.41876464223778\\
%44.1966049671173	8.3366586111003\\
44.2981135845184	8.33182152080188\\
%44.3979256153107	8.34707454238278\\
%44.4983563423157	8.30100113064165\\
44.5976676940918	8.22353809995218\\
%44.6995811462402	8.23212455673355\\
%44.7986752986908	8.25569109186716\\
44.8985450267792	8.18061164711559\\
%44.9989039897919	8.07116483383077\\
%45.0958788394928	8.05896548546056\\
45.1988072395325	8.05838962667141\\
%45.2986025810242	7.98896414003348\\
%45.4006898403168	7.94566551246516\\
45.4995272159576	7.95385193836406\\
%45.5981070995331	7.920178315373\\
%45.6987905502319	7.88652625585478\\
45.7980630397797	7.89602978961424\\
%45.8978490829468	7.89826921294095\\
%45.9975397586823	7.87201669153311\\
46.0997786521912	7.8516174204625\\
%46.1985127925873	7.85492898500484\\
%46.2988638877869	7.85971347466012\\
46.3977031707764	7.83871967822303\\
%46.4992260932922	7.80082866616132\\
%46.598242521286	7.79648632136381\\
46.6982197761536	7.81187963636086\\
%46.7964286804199	7.79047525047833\\
%46.8979287147522	7.75245776083849\\
47.0007612705231	7.73425706558692\\
%47.099461555481	7.7349585194702\\
%47.1973867416382	7.71445723331785\\
47.2976024150848	7.69696271827801\\
%47.3962533473969	7.70650369477193\\
%47.4976968765259	7.71149394850536\\
47.5979177951813	7.68444825743228\\
%47.6981735229492	7.66108952030101\\
%47.7990231513977	7.67040346757448\\
47.8997867107391	7.67646324258757\\
%47.999951839447	7.65935471821028\\
%48.0984971523285	7.65376240088038\\
48.1977996826172	7.66642178064507\\
%48.2982051372528	7.66986668028813\\
%48.397274017334	7.65156851388986\\
48.4983186721802	7.64655521008297\\
%48.5980792045593	7.65845657002347\\
%48.6991057395935	7.65347533346745\\
48.7981367111206	7.6357681164933\\
%48.8980414867401	7.63167540432084\\
%48.9978730678558	7.63809426523989\\
49.0970058441162	7.63622544631797\\
%49.1963555812836	7.62760775245212\\
%49.2969927787781	7.62354881419668\\
49.3988502025604	7.62416750643234\\
%49.498318195343	7.62468420205536\\
%49.5989005565643	7.62298101990451\\
49.6973714828491	7.62352236878343\\
%49.797901391983	7.62340526444167\\
%49.8976213932037	7.62253175176939\\
49.9978492259979	7.62155663027932\\
%50.0978472232819	7.62131042471654\\
%50.2000968456268	7.62138006318507\\
50.3002262115479	7.62164400292132\\
%50.3988947868347	7.62111824367638\\
%50.4979214668274	7.62105843153128\\
50.5981481075287	7.62140773002663\\
%50.6984331607819	7.62101597112662\\
%50.7982943058014	7.61823406054113\\
50.9011902809143	7.60961818039272\\
%50.9977343082428	7.60859866361877\\
%51.0983502864838	7.60886374679633\\
51.1994557380676	7.60831690921982\\
%51.2987756729126	7.60832526679255\\
%51.3972861766815	7.60524207036761\\
51.4997346401215	7.60095641784262\\
%51.5964462757111	7.60269991417995\\
%51.698677778244	7.60298165570827\\
51.7978341579437	7.60261477563652\\
%51.9003474712372	7.6026822088433\\
%51.9989249706268	7.60288187915039\\
52.0975363254547	7.60268361075552\\
%52.1990420818329	7.60252110117285\\
%52.2984504699707	7.60264440469967\\
52.3979661464691	7.60299662126431\\
%52.4984498023987	7.60272731993891\\
%52.5998876094818	7.6026521361001\\
52.6995544433594	7.60293760639476\\
%52.7968590259552	7.60279846244591\\
%52.8977110385895	7.60256218815895\\
52.9981262683868	7.60304132547281\\
%53.0989315509796	7.60290908376836\\
%53.1972990036011	7.6029327923497\\
53.2967021465302	7.60264337355673\\
%53.3984217643738	7.6028827965836\\
%53.4997849464417	7.60302101476021\\
53.5988943576813	7.60272394696622\\
%53.6983377933502	7.60290461095835\\
%53.7982711791992	7.60284735403454\\
53.8991680145264	7.60257846406496\\
%53.9992122650146	7.60269655619888\\
%54.0977897644043	7.60294739716441\\
54.1989953517914	7.60300342798453\\
%54.2989671230316	7.60158352717816\\
%54.4001793861389	7.60102996802456\\
54.4991703033447	7.60130566571751\\
%54.5983467102051	7.60103798555883\\
%54.6985468864441	7.60110214328459\\
54.7985153198242	7.60123431008813\\
%54.8970367908478	7.60132859621398\\
%54.9963908195496	7.60110605298061\\
};
\addlegendentry{H(p,$\mathbb{W}$,$\hat{\phi}$)}


\addplot[area legend, draw=none, fill=black, fill opacity=0.09, forget plot]
table[row sep=crcr] {%
x	y\\
0	0\\
33.4	0\\
33.4	70\\
0	70\\
}--cycle;
\end{axis}

\begin{axis}[%
width=0in,
height=0in,
at={(0in,0in)},
scale only axis,
xmin=0,
xmax=1,
ymin=0,
ymax=1,
axis line style={draw=none},
ticks=none,
axis x line*=bottom,
axis y line*=left
]
\end{axis}
\end{tikzpicture}%
\vspace{-0.7em}
\caption{\rahel{Locational optimization cost (green) as well as estimated locational optimization cost versus time (rose), applying Algorithm~\ref{alg:twolayermpcalglearningimp} in consideration of an initially unknown $\phi_{2}$. The grey background indicates time instances for which the agents are exploring.}}
\label{fig:twolayerlearningcoveragecost}
\vspace{-0.8em}
\end{figure}

\begin{figure}[h!]
\begin{minipage}[t]{0.15\textwidth}
\centering
\includegraphics[trim={5.2cm 8.1cm 4.6cm 7.7cm},clip, width = 0.8\textwidth]{figures/config_7sec_two_layer_learning_take_one_0_1_compressed.pdf}
\centering
\end{minipage}
\hfill
\begin{minipage}[t]{0.15\textwidth}
\centering
\includegraphics[trim={5.2cm 8.1cm 4.6cm 7.7cm},clip, width = 0.8\textwidth]{figures/config_38sec_two_layer_learning_take_one_0_1_compressed.pdf}
\centering
\end{minipage}
\hfill
\begin{minipage}[t]{0.15\textwidth}
\centering
\includegraphics[trim={5.2cm 8.1cm 4.6cm 7.7cm},clip, width = 0.8\textwidth]{figures/config_58sec_two_layer_learning_take_one_0_1_compressed.pdf}
\centering
\end{minipage}
\caption{\rahel{Configurations of cars at 2, 33, and 53 seconds applying  Algorithm~\ref{alg:twolayermpcalglearningimp} in the described set-up. The agents' location and their predicted trajectory are given in red, the Voronoi partitions in green, the references in blue, and the traveled paths are visualized in light grey. While the first two instances visualize the initial exploration movement, the third shows a covering behavior.}}
\label{pics:twolayerslearningconfig}
\vspace{-0.4em}
\end{figure}
\noindent Defining~$F(\text{Var}_{\max,t}) = \frac{\text{Var}_{\max,t}}{\text{Var}_{\max,0}}$, the first decision in Algorithm~\ref{alg:twolayermpcalglearningimp} is guaranteed to be exploration, and the agents drive towards the point of maximal variance within their initial Voronoi partition. Allowing for a better insight into the learning progress, the decreasing behaviour of the maximal variance over time is presented in Figure~\ref{fig:twolayerslearningmaxvar}. In Figure~\ref{fig:twolayerslearningcoef}, the development of estimated parameter vector $\theta$ is visualized.
In Figure~\ref{fig:twolayerlearningcoveragecost}, as well as in Figures~\ref{fig:twolayerslearningmaxvar} and~\ref{fig:twolayerslearningcoef}, the time instances for which the cars are exploring are indicated with a light grey background. It can be seen that, for the first \rahel{33} seconds, the agents keep exploring: correspondingly, any decrease or increase in cost is only accidental. However, by collecting measurements at each time step during exploration, after \rahel{15} seconds the mean estimate of $\theta$ is quite precise and further improvements are minor. This is also reflected in the small remaining maximal variance of the estimate. By the time the agents leave exploration mode, all mean estimates of the true coefficient $\theta$ show a maximal error of less than \rahel{0.06} and a maximal variance of approximately $10^{-3}$. Due to the remaining small uncertainty, by the time the experiment is interrupted the probability of another exploration movement for future instances is strictly positive, but very small.
\begin{figure} [h!]
\centering
% This file was created by matlab2tikz.
%
%The latest updates can be retrieved from
%  http://www.mathworks.com/matlabcentral/fileexchange/22022-matlab2tikz-matlab2tikz
%where you can also make suggestions and rate matlab2tikz.
%
\definecolor{mycolor1}{rgb}{0.50000,0.75000,0.93000}%
%
\begin{tikzpicture}

\begin{axis}[%
height=0.85in,
width=0.40\textwidth,
yshift=0.8cm,
at={(0.0in,0.0in)},
scale only axis,
xmin=0,
xmax=54,
xlabel style={font=\color{white!15!black}},
xlabel style={font=\footnotesize},
xlabel style={yshift=0.6ex,},
xlabel={Time [sec]},
ymode=log,
ymin=0.001,
ymax=13,
yminorticks=true,
ylabel style={font=\color{white!15!black}},
ylabel style={font=\footnotesize},
ylabel style={xshift = 0.6ex,},
ylabel={$\textup{Var}_{\textup{max}}$},
axis background/.style={fill=white},
tick label style={font=\footnotesize},
xmajorgrids,
ymajorgrids,
legend style={legend cell align=left, align=left, draw=white!15!black},
legend style={font=\footnotesize}
]
\addplot [color=mycolor1, line width=1.4pt]
  table[row sep=crcr]{%
-0.102641630172729	11.7509203731866\\
-0.00399475097656232	11.7509203731866\\
0.197481822967529	11.2573278430165\\
0.297660303115845	10.8505898973323\\
0.395315361022949	10.7644744091478\\
0.496318769454956	10.7181912858578\\
0.594118309020996	10.6857820293292\\
0.696222257614136	10.6606622739937\\
0.796583127975464	10.6373765051286\\
0.897525024414063	10.6090385061855\\
0.995674800872803	10.5810121391838\\
1.09674854278564	10.5561656117164\\
1.19668192863464	10.5235577948856\\
1.29528160095215	10.4768771867359\\
1.39562602043152	10.4316857950609\\
1.49685688018799	10.3902547747594\\
1.59779186248779	10.3355168720236\\
1.69930715560913	10.2639309788096\\
1.79555983543396	10.1936675172094\\
1.8978611946106	10.1260055653088\\
1.99712080955505	10.0441680565736\\
2.09485359191895	9.94376778051811\\
2.19515223503113	9.84635725289906\\
2.29523344039917	9.75747448793587\\
2.39508051872253	9.66186762723755\\
2.49601120948792	9.54299739650042\\
2.59701557159424	9.41568655431464\\
2.69605584144592	9.29520764264453\\
2.7990264415741	9.17206056414415\\
2.89691681861877	9.02624122314248\\
2.99577827453613	8.87221542348229\\
3.0970148563385	8.72615445350894\\
3.19535584449768	8.5754427957272\\
3.29989237785339	8.4002985665436\\
3.39770526885986	8.22359089176197\\
3.49521751403809	8.05886227385404\\
3.59622402191162	7.8853798260223\\
3.69739217758179	7.69012738501594\\
3.79589910507202	7.49891847749245\\
3.89509339332581	7.32215449809617\\
3.99589152336121	7.14069091541574\\
4.09866471290588	6.94420173487156\\
4.19591970443726	6.7545876595114\\
4.29492802619934	6.57476243726578\\
4.39473552703857	6.39173940753303\\
4.49513740539551	6.20189405753166\\
4.59427614212036	6.01967782479169\\
4.69665236473084	5.84040708477064\\
4.79721279144287	5.65780971942414\\
4.89971823692322	5.47740771731946\\
4.99971623420715	5.30272426289607\\
5.09674663543701	5.12636893609592\\
5.19711680412292	4.94828171034644\\
5.29613680839539	4.77370030818991\\
5.39612264633179	4.61098715855097\\
5.49452586174011	4.45116228702712\\
5.59431190490723	4.28975922296248\\
5.70431060791016	4.13109619886798\\
5.79796380996704	3.98260698109043\\
5.89721817970276	3.83633867760645\\
5.99629421234131	3.68806598921937\\
6.09697599411011	3.54346222274896\\
6.1944531917572	3.40864985599635\\
6.29653282165527	3.27685757676491\\
6.39521975517273	3.14491258517996\\
6.49830098152161	3.01708720825438\\
6.5970103263855	2.89805199433001\\
6.69804544448853	2.78176059824196\\
6.79470391273499	2.66468345902076\\
6.89596815109253	2.55224301931898\\
6.99677081108093	2.44930379174377\\
7.09484143257141	2.3492539667924\\
7.19564433097839	2.24958144162671\\
7.29785127639771	2.15303445636818\\
7.3964430809021	2.06326924707875\\
7.49649543762207	1.97510263850776\\
7.59733147621155	1.88712675752293\\
7.69563360214233	1.80445109438426\\
7.79482383728027	1.72877829592683\\
7.89493722915649	1.65431760097826\\
7.99771327972412	1.58173834708925\\
8.09831328392029	1.51363674057802\\
8.19566221237183	1.4505243804539\\
8.29708476066589	1.38832773832785\\
8.3971161365509	1.32763706200675\\
8.49455518722534	1.27132388464454\\
8.596808385849	1.21917428387761\\
8.69723243713379	1.16817097208008\\
8.79802293777466	1.11941115045008\\
8.89604587554932	1.07389895602251\\
8.99716157913208	1.0299876920294\\
9.09651894569397	0.98660497127301\\
9.19569869041443	0.944576509633502\\
9.29771966934204	0.90525618769386\\
9.39675970077515	0.868456109766679\\
9.49634737968445	0.83243060919062\\
9.59599800109863	0.797249013646737\\
9.69701762199402	0.764668601704531\\
9.79708094596863	0.734320984870786\\
9.89533562660217	0.704548240157783\\
9.99642462730408	0.675370959900486\\
10.0989460468292	0.648096717335077\\
10.1970824718475	0.622650426034246\\
10.2955255031586	0.597810061589321\\
10.3951494216919	0.573425176816013\\
10.4988402843475	0.550546689238708\\
10.5975560665131	0.529210608103583\\
10.6963927268982	0.508294504259403\\
10.7953311920166	0.487752439351762\\
10.8961343288422	0.468565593246573\\
10.9970607280731	0.450543009761434\\
11.0951551914215	0.43284139290088\\
11.1950246810913	0.41557852892216\\
11.2986406803131	0.399600681697165\\
11.3985943317413	0.38462288985733\\
11.4964956760406	0.369978479753251\\
11.5962562084198	0.355877483282452\\
11.6961719512939	0.34262671047326\\
11.7945816040039	0.330077455999779\\
11.8962785720825	0.317766858124408\\
11.9970404624939	0.305874602025842\\
12.0981022834778	0.294581189449658\\
12.1976756572723	0.283866704012376\\
12.297647190094	0.273393847853638\\
12.3979465484619	0.263291511528522\\
12.4956640720367	0.253573144807234\\
12.5957223892212	0.244380634532613\\
12.6969048500061	0.23546323771661\\
12.7963842868805	0.226851986388769\\
12.8941533088684	0.21861381647458\\
12.9970893383026	0.210831595626277\\
13.0981628417969	0.203358088879928\\
13.1958713054657	0.196126348996357\\
13.2970160961151	0.189124847655879\\
13.3970913410187	0.182473214122332\\
13.4954921722412	0.176079817690392\\
13.5960585594177	0.169932545144856\\
13.6961168766022	0.163958493062449\\
13.7979001522064	0.158300599396171\\
13.8967771053314	0.152861541247127\\
13.9955553531647	0.147574124605765\\
14.0983836174011	0.142447330835847\\
14.1969835281372	0.137599615732212\\
14.2950086116791	0.132917817068089\\
14.3953346729279	0.128352921922508\\
14.4976608276367	0.123937792525953\\
14.5960981369019	0.1197844253569\\
14.6967906475067	0.115787586515799\\
14.7962595939636	0.11189378505444\\
14.895455789566	0.108113352213044\\
14.9958805561066	0.10455032506092\\
15.0966526985168	0.101106677703453\\
15.1965777397156	0.0977151706909612\\
15.2969314575195	0.0944362548173842\\
15.3959002017975	0.0913697227260864\\
15.4992904186249	0.0883983459939962\\
15.5964524269104	0.0854999472095525\\
15.6968938827515	0.0827105673898284\\
15.7952544212341	0.0800761774520717\\
15.8950328350067	0.0775209175022645\\
15.9980420589447	0.0750150304923154\\
16.0985359668732	0.0725976346142464\\
16.1967639446259	0.0703230229572608\\
16.2970499515533	0.0681066474272125\\
16.3958527565002	0.0659244494980544\\
16.4959725856781	0.0638252948695584\\
16.5949379920959	0.0618352738689621\\
16.6982783794403	0.0598868741650092\\
16.796982717514	0.0580020628732429\\
16.8951115131378	0.0562010998969423\\
16.9973370552063	0.0544829308363552\\
17.0955852985382	0.0528103944341852\\
17.1948601722717	0.051177212636883\\
17.3003346443176	0.0496016474488353\\
17.3955173015594	0.0481022064456822\\
17.4954607009888	0.0466473000814996\\
17.5943259716034	0.0452219702267513\\
17.6972994327545	0.0438382434593588\\
17.7960125923157	0.0425183884372971\\
17.8956210136414	0.0412337412348947\\
17.9953317165375	0.0399805868889353\\
18.0961772918701	0.0387634822678264\\
18.1964308738708	0.0375974849683019\\
18.2951261520386	0.0364682628572815\\
18.3980986595154	0.0353611506733615\\
18.4948031425476	0.0342879991849828\\
18.5971197605133	0.0332648084877982\\
18.6954788684845	0.0322716239628609\\
18.7960478782654	0.0312980163656593\\
18.8955831050873	0.0303524903910901\\
18.9958488464355	0.0294460491159485\\
19.0958041667938	0.0285692497154672\\
19.1941401481628	0.0277132747475731\\
19.2980067253113	0.026880645183012\\
19.397473526001	0.0260842045139516\\
19.4952361106873	0.0253213524196592\\
19.5980996608734	0.0245712947422579\\
19.6952943325043	0.0238415815165219\\
19.7962779521942	0.023152216028381\\
19.8974170207977	0.0224914967337737\\
19.9963166236877	0.0218390319014746\\
20.095966053009	0.0212014425539117\\
20.1977030754089	0.0205947699729977\\
20.2981388092041	0.0200113770594404\\
20.3957421302795	0.0194368345648399\\
20.4971506118774	0.01887426037629\\
20.5960132598877	0.0183354843681672\\
20.6970345497131	0.0178206392983738\\
20.795324754715	0.0173106894454483\\
20.8998586654663	0.0168066644066699\\
20.9955911159515	0.0163272129756598\\
21.0970391750336	0.015868665375892\\
21.1975664615631	0.0154134297618945\\
21.2972383022308	0.0149664729393212\\
21.3958482265472	0.0145448070089352\\
21.4955183982849	0.014140741968618\\
21.5967411518097	0.0137370018304369\\
21.6974296092987	0.0133435678169968\\
21.797936630249	0.0129741304698876\\
21.897910785675	0.0126162595605064\\
21.9961254119873	0.0122579219915416\\
22.0951427936554	0.0119099795466644\\
22.1953200817108	0.0115831676766515\\
22.2968997478485	0.0112657604418373\\
22.3957356929779	0.0109490264227191\\
22.4985556125641	0.0106425708221365\\
22.5987534046173	0.0103532318662335\\
22.6979319572449	0.0100726596985576\\
22.7984454154968	0.00979279985802551\\
22.8956765651703	0.00952021352369118\\
22.9952852249146	0.00926530721657878\\
23.0960955142975	0.00902048921183612\\
23.1963202476501	0.00877689681031076\\
23.2982692241669	0.00853700071599082\\
23.395704460144	0.00831080839877355\\
23.4972602844238	0.0080942512727936\\
23.5962728977203	0.0078776198619385\\
23.6951977729797	0.00766339732003276\\
23.7953948497772	0.00746166107926908\\
23.8959335803986	0.00726884161790321\\
23.9949538230896	0.00707532930361321\\
24.0971569538116	0.00688331629978366\\
24.1964091777802	0.00670375641974355\\
24.2955185890198	0.00653430209009964\\
24.3954247951508	0.00636607405432334\\
24.4950892448425	0.00619887499835346\\
24.5977506160736	0.00603991217847521\\
24.6964554309845	0.00588899595832358\\
24.7965447425842	0.00574120403480175\\
24.8952652931213	0.00559884946109637\\
24.9963130474091	0.00546370658033992\\
25.0997437953949	0.00533288571306614\\
25.19569439888	0.00520510632304203\\
25.2984697341919	0.00508241219033061\\
25.3945843696594	0.00496596745837952\\
25.4953708171844	0.0048533281705229\\
25.5948240280151	0.00474297195893774\\
25.697218132019	0.00463643721328153\\
25.7978610515594	0.00453451989605821\\
25.8968491077423	0.00443543719138322\\
25.9962591648102	0.00433864786316784\\
26.0950316905975	0.00424500172691527\\
26.1957902431488	0.00415462553060766\\
26.2969285964966	0.00406647176393032\\
26.397035074234	0.0039808487746683\\
26.4961301803589	0.00389835162722799\\
26.5958349227905	0.00381812569765149\\
26.696764421463	0.00373976048693204\\
26.7960946083069	0.00366432617295952\\
26.8963512897491	0.00359111103845091\\
26.9971394062042	0.00351919893647658\\
27.0980145454407	0.00345005342495553\\
27.1954867362976	0.00338362092057585\\
27.2977942943573	0.00331803771827633\\
27.3979982852936	0.00325349843964391\\
27.498285484314	0.00319176403675254\\
27.5956844806671	0.00313157299932467\\
27.6955816268921	0.00307169916592267\\
27.7959031581879	0.00301321184139541\\
27.8978302001953	0.00295649690574323\\
27.9968165874481	0.00290051707680569\\
28.0959853649139	0.00284528224688203\\
28.1990952014923	0.0027916267839095\\
28.2990755558014	0.00273958603690777\\
28.3968019008636	0.00268840001305485\\
28.4977595329285	0.00263835739487919\\
28.5956415653229	0.00258978684819408\\
28.6952630996704	0.00254227675806001\\
28.796107006073	0.00249579759600966\\
28.8989147663116	0.00245089272494752\\
28.9964420318604	0.00240733357641936\\
29.0973846435547	0.00236458476986312\\
29.1952988624573	0.00232295113410592\\
29.297781419754	0.00228281952098237\\
29.3961817741394	0.00224356311511033\\
29.495747756958	0.00220527593553532\\
29.5961651325226	0.00216841210664653\\
29.6956064224243	0.00213242003912669\\
29.7963809490204	0.00209720102555867\\
29.8977269649506	0.0020632186684925\\
29.9961711883545	0.00203010244114927\\
30.0960766792297	0.00199753400669186\\
30.1963569641113	0.00196599535555785\\
30.2955946445465	0.00193537726025139\\
30.3946065425873	0.00190522338277723\\
30.4984733581543	0.00187570323213339\\
30.5978180885315	0.00184714577816629\\
30.6964194297791	0.00181928774233519\\
30.7962066650391	0.00179192512304708\\
30.8956923007965	0.00176513105019232\\
30.9959980964661	0.00173906241927248\\
31.0967654705048	0.00171344522332578\\
31.1960083961487	0.00168820996983177\\
31.2985543727875	0.00166366046523921\\
31.3986517906189	0.00163960912915114\\
31.4971270084381	0.00161568742711678\\
31.5964040279388	0.00159221182274048\\
31.6963140487671	0.00156931149027833\\
31.7980613231659	0.00154669483665757\\
31.8965267658234	0.00152437564011195\\
31.9970397472382	0.00150269484233836\\
32.0989522457123	0.00148130603311189\\
32.1969158172607	0.00146771286794942\\
32.2958316326141	0.00145458448627479\\
32.3970858573914	0.00144177303636024\\
32.496911239624	0.00142918649083976\\
32.5960337638855	0.00141681151576914\\
32.6966475963593	0.00140473440865564\\
32.7968043804169	0.0013929253161122\\
32.8977918148041	0.00138130379138036\\
32.9977538108826	0.00136989174118776\\
33.0974013328552	0.00135876036610494\\
33.196910572052	0.00134787497983752\\
33.2986182689667	0.00133716298782231\\
33.3956074237824	0.00132665839567381\\
33.4963221073151	0.00131640636744158\\
33.5969666957855	0.00130636399287633\\
33.6959208965302	0.00129649602978389\\
33.7972809791565	0.00128684196996334\\
33.8963171958923	0.00127739086368829\\
33.9972002029419	0.00126808705696875\\
34.0974368572235	0.00126311265335843\\
34.1982259273529	0.00125403006060988\\
34.2978076457977	0.00122676087964584\\
34.3997306346893	0.00122594807957366\\
34.4999250888824	0.00125356862658326\\
34.5993508815765	0.00124568667565476\\
34.698733997345	0.00124419644563143\\
34.7993893146515	0.00124398317307299\\
34.8988301277161	0.00123244195194868\\
34.9975063323975	0.00123244195194868\\
35.0976547718048	0.00123485272028592\\
35.2006651878357	0.00125223986138496\\
35.2998816490173	0.00123353684958708\\
35.3983435153961	0.00123353684958708\\
35.4988638877869	0.00122607018841824\\
35.5992395401001	0.00122525004431067\\
35.6974098205566	0.00122095912114928\\
35.7990953445435	0.00124541587270605\\
35.8984739303589	0.0012251527881473\\
35.9991373538971	0.00121987041348809\\
36.0992421627045	0.00123147901841601\\
36.1995939731598	0.00124575058649676\\
36.2995116233826	0.00124362825073473\\
36.3988010406494	0.00124384185077587\\
36.4991075515747	0.00124066885541785\\
36.5975064754486	0.00124207746950343\\
36.6998371601105	0.0012328288610679\\
36.7991470813751	0.00123662225967291\\
36.9011051177979	0.00124925868751883\\
36.9994184494019	0.0012255968866883\\
37.0990733623505	0.00124900460695837\\
37.1988157749176	0.00124905611356369\\
37.2981652736664	0.00122311224211267\\
37.3983177661896	0.00122381474314633\\
37.4988364696503	0.0012491261524081\\
37.5981459140778	0.00125417145779982\\
37.6999842643738	0.00123023007686638\\
37.7979850292206	0.0012276397046446\\
37.9001514434814	0.00122276062971303\\
37.9992446422577	0.00122936800136015\\
38.09906001091	0.00124089967857013\\
38.1977645874023	0.00123447283697119\\
38.2998115539551	0.0012288151738128\\
38.3999709606171	0.00124125095671919\\
38.4986514568329	0.00124559114024958\\
38.5986451625824	0.00124247640765437\\
38.6977481365204	0.00124022670691073\\
38.7988166332245	0.00124981094831515\\
38.8983344554901	0.00125231397698857\\
38.9985215187073	0.00125215048954695\\
39.0979029655457	0.0012548149920002\\
39.1986093044281	0.00122729405585539\\
39.2984861850739	0.00122848795235286\\
39.3988687515259	0.00122668661614561\\
39.4992932796478	0.00122727874795802\\
39.5986649513245	0.00123023762231126\\
39.6984800815582	0.0012312274998237\\
39.8000394821167	0.00123057450616801\\
39.8976871490479	0.00123351562027939\\
39.9995448112488	0.0012340688429556\\
40.098398399353	0.00123587271791105\\
40.1997789859772	0.00123465972854594\\
40.2985596179962	0.00123994293384627\\
40.3981627941132	0.00123853866280863\\
40.4982318401337	0.00124093981437822\\
40.5980152606964	0.00123925492787516\\
40.6980852603912	0.00124635717671336\\
40.7991847515106	0.00124517927656415\\
40.8989045143127	0.00124732940860616\\
40.9993845939636	0.00124873198425198\\
41.0983955383301	0.0012523802820549\\
41.1980945587158	0.00124669798781676\\
41.2979053974152	0.00124415847167531\\
41.3978716850281	0.00125257744307335\\
41.5006420135498	0.00121598290906369\\
41.5981866836548	0.00124939628862585\\
41.698610496521	0.00124906722234805\\
41.7993831157684	0.00122031137941692\\
41.8989188194275	0.00122231250932501\\
41.9987523078918	0.0012238891951432\\
42.0987452983856	0.00122479509260518\\
42.1969921112061	0.00123143530991246\\
42.2976343154907	0.00123001781407103\\
42.3991317272186	0.00122949561311427\\
42.4998847961426	0.00123004941271986\\
42.5993759155273	0.00123962100445539\\
42.6998688697815	0.00123512149813326\\
42.7977230072022	0.00123558779926818\\
42.8983890533447	0.00123543089374851\\
42.998481464386	0.00124262466439407\\
43.0982572555542	0.0012358778602894\\
43.200212430954	0.00123798214741641\\
43.2996308326721	0.00123888369365786\\
43.3981952190399	0.00124578931721613\\
43.5008167743683	0.00124163255084981\\
43.6000844955444	0.00123638914503846\\
43.6979612827301	0.00123671867717928\\
43.798176240921	0.00124777309478566\\
43.8982347965241	0.00124447620564471\\
43.9983796596527	0.00123361908274486\\
44.1013307094574	0.00123727025427045\\
44.1997320175171	0.00125208759439369\\
44.2993864536285	0.00124137184785563\\
44.3969835758209	0.00123218164352999\\
44.4983205318451	0.00124278524922125\\
44.5983383178711	0.00122561585558545\\
44.6986577033997	0.00124579906397556\\
44.7979838371277	0.00123937293312144\\
44.8995577812195	0.00124763222201725\\
44.9992038726807	0.00125034901528625\\
45.0985345363617	0.0012380172166442\\
45.1998519420624	0.00123707785690681\\
45.296613407135	0.0012415250002008\\
45.3990528106689	0.00123731544442308\\
45.498597574234	0.00124160655166392\\
45.6006803035736	0.00123929928058519\\
45.6997265338898	0.00123616427030831\\
45.7984370708466	0.0012394265841299\\
45.8987800598145	0.00124175001268021\\
45.9983410358429	0.00123949688453456\\
46.0978343009949	0.00123560354797122\\
46.1979047775269	0.00123543321425362\\
46.3004221439362	0.00123625005217762\\
46.3988713741303	0.00123339619010164\\
46.4992043495178	0.00122865113847691\\
46.5981831073761	0.00122878423664686\\
46.6992108345032	0.00123219738631113\\
46.7986704826355	0.00123176036980306\\
46.8982145309448	0.00125225538313976\\
46.9968122959137	0.00125150942662049\\
47.0982305526733	0.00125059285941719\\
47.2007486343384	0.00124868777685873\\
47.2994477272034	0.00124573966554869\\
47.3982145309448	0.00124471680817601\\
47.4983245849609	0.00124429397669799\\
47.5979132175446	0.00124185404289442\\
47.6981746673584	0.00123547770943767\\
47.7979065895081	0.00123386316164202\\
47.8985058784485	0.00123359557731767\\
47.9990026473999	0.00123423651482632\\
48.0997802734375	0.00123473678796891\\
48.1998104572296	0.00123391374201559\\
48.2986614227295	0.00123374407252074\\
48.3977844238281	0.00123446740497182\\
48.4982063293457	0.00123415123296208\\
48.5981866836548	0.00123319701298903\\
48.6983060359955	0.00123277557188319\\
48.7980682373047	0.00122968030981742\\
48.8994664669037	0.00122806960274069\\
48.9986116409302	0.00122808628080302\\
49.0981363773346	0.00122832261557019\\
49.1985058307648	0.00122779149419266\\
49.2973339080811	0.00122727521946292\\
49.3965417861939	0.00122727023237428\\
49.4969806194305	0.00122741967832079\\
49.5988392353058	0.00122785378464586\\
49.6983053207397	0.00122783245530632\\
49.7995328426361	0.00122774973842558\\
49.8977550983429	0.00122802410752651\\
49.9981278896332	0.00122789326611012\\
50.0985233306885	0.00122783584992273\\
50.1980664253235	0.00122787986094145\\
50.2980305671692	0.00122793617073447\\
50.4000823020935	0.00122782035543801\\
50.50042719841	0.00122773376375649\\
50.5993928432465	0.00122789036662694\\
50.698119354248	0.00122794870172845\\
50.7988230705261	0.00122778990736767\\
50.8988437175751	0.00122807740267905\\
50.998971414566	0.0012285560031907\\
51.1012002944946	0.00122849460994253\\
51.1983432292938	0.00122879993770462\\
51.3002111434937	0.00122871946211157\\
51.3994435787201	0.00123000564024844\\
51.4987613677979	0.00123221462237627\\
51.5982357978821	0.00123143428791059\\
51.6998898506165	0.00123118280896381\\
51.7980644226074	0.00123127838135594\\
51.8986653804779	0.00123141283138108\\
51.9986044883728	0.00123124107439053\\
52.1003365039825	0.00123117363720209\\
52.1989116191864	0.0012313515897784\\
52.2984992980957	0.00123131143470864\\
52.3991014480591	0.00123113763007425\\
52.4986261844635	0.0012312748610773\\
52.5979811668396	0.00123135755576328\\
52.6984385967255	0.00123125241578513\\
52.8002340316772	0.00123120468895769\\
52.8995310783386	0.00123132733248384\\
52.9980551719666	0.00123125697361719\\
53.0982808589935	0.00123114363477755\\
53.1981179237366	0.00123124823556097\\
53.2991795063019	0.0012313540381295\\
53.3977806091309	0.00123122309396641\\
53.4977111339569	0.00123116958414548\\
53.6009675979614	0.00123129344195525\\
53.699773979187	0.00123128201425986\\
53.7995035171509	0.00123114821541315\\
53.8985678672791	0.00123119296032299\\
53.9982547283173	0.00123133675636707\\
54.0993465900421	0.00123116648834233\\
54.1993054866791	0.00123114590256365\\
54.2980393886566	0.00123066906538842\\
54.3992440223694	0.00123026635969834\\
54.4989540100098	0.00123009209589323\\
54.6004807472229	0.00123018013100602\\
54.6991581439972	0.00123030362157297\\
54.7983424186707	0.0012301711702234\\
54.8992936134338	0.00123007768865152\\
54.9986073493958	0.00123027753784811\\
55.0982775211334	0.00123023820840747\\
55.1963874816895	0.00123000012403496\\
55.3000180244446	0.00123014315006932\\
55.3994614601135	0.00123033014956347\\
55.4992069721222	0.00123014969507156\\
55.5982458114624	0.0012300416274634\\
55.6981310367584	0.00123031012025672\\
};
\addlegendentry{Maximal Variance}


\addplot[area legend, draw=none, fill=black, fill opacity=0.09, forget plot]
table[row sep=crcr] {%
x	y\\
0	0.001\\
33.4	0.001\\
33.4	16\\
0	16\\
}--cycle;
\end{axis}

\begin{axis}[%
width=0in,
height=0in,
at={(0in,0in)},
scale only axis,
xmin=0,
xmax=1,
ymin=0,
ymax=1,
axis line style={draw=none},
ticks=none,
axis x line*=bottom,
axis y line*=left
]
\end{axis}
\end{tikzpicture}%
\vspace{-0.7em}
\caption{\rahel{Maximal variance decrease over time applying Algorithm~\ref{alg:twolayermpcalglearningimp} in consideration of an initially unknown $\phi_{2}$. The grey background indicates time instances for which the agents are exploring.}}
\label{fig:twolayerslearningmaxvar}
\vspace{-0.7em}
\end{figure}

\begin{figure} [h!]
\centering
% This file was created by matlab2tikz.
%
%The latest updates can be retrieved from
%  http://www.mathworks.com/matlabcentral/fileexchange/22022-matlab2tikz-matlab2tikz
%where you can also make suggestions and rate matlab2tikz.
%
\definecolor{mycolor1}{rgb}{0.00000,0.44700,0.74100}%
\definecolor{mycolor2}{rgb}{0.85000,0.32500,0.09800}%
\definecolor{mycolor3}{rgb}{0.92900,0.69400,0.12500}%
\definecolor{mycolor4}{rgb}{0.49400,0.18400,0.55600}%
\definecolor{mycolor5}{rgb}{0.46600,0.67400,0.18800}%
%
\begin{tikzpicture}

\begin{axis}[%
height=0.9in,
width=0.40\textwidth,
yshift=0.8cm,
at={(0.0in,0.0in)},
scale only axis,
xmin=0,
xmax=54,
xlabel style={font=\color{white!15!black}},
xlabel style={font=\footnotesize},
xlabel style={yshift=0.6ex,},
xlabel={Time [sec]},
ymin=-4,
ymax=14,
ylabel style={font=\color{white!15!black}},
ylabel style={font=\footnotesize},
ylabel style={xshift = 0.6ex,},
ylabel={Mean value},
axis background/.style={fill=white},
tick label style={font=\footnotesize},
title style={font=\bfseries},
title={},
xmajorgrids,
ymajorgrids,
legend style={legend cell align=left, align=left, draw=white!15!black},
legend style={font=\footnotesize}
]
\addplot [color=mycolor1, line width=1.4pt]
  table[row sep=crcr]{%
-3.41395025253296	0\\
-3.36380581855774	0\\
-3.31373219490051	0\\
-3.26373772621155	0\\
-3.21386485099792	0\\
-3.16371612548828	0\\
-3.11396126747131	0\\
-3.0639102935791	0\\
-3.01393465995789	0\\
-2.96376042366028	0\\
-2.91368894577026	0\\
-2.86386590003967	0\\
-2.81392412185669	0\\
-2.76383023262024	0\\
-2.71365766525269	0\\
-2.66386275291443	0\\
-2.61371665000916	0\\
-2.56373028755188	0\\
-2.51375107765198	0\\
-2.46365218162537	0\\
-2.41376595497131	0\\
-2.36374454498291	0\\
-2.31372838020325	0\\
-2.26391820907593	0\\
-2.21392397880554	0\\
-2.16394262313843	0\\
-2.11378173828125	0\\
-2.06323270797729	0\\
-2.01386647224426	0\\
-1.96378617286682	0\\
-1.91375093460083	0\\
-1.86376647949219	0\\
-1.81371264457703	0\\
-1.76388936042786	0\\
-1.71374897956848	0\\
-1.66370325088501	0\\
-1.61389498710632	0\\
-1.56357769966125	0\\
-1.51373534202576	0\\
-1.46395378112793	0\\
-1.41391592025757	0\\
-1.36373047828674	0\\
-1.31355028152466	0\\
-1.26392798423767	0\\
-1.21389083862305	0\\
-1.1639223575592	0\\
-1.11384873390198	0\\
-1.06395726203918	0\\
-1.01393632888794	0\\
-0.96373085975647	0\\
-0.913796234130859	0\\
-0.863754081726074	0\\
-0.81366925239563	0\\
-0.763712215423584	0\\
-0.71373085975647	0\\
-0.663900899887085	0\\
-0.61391978263855	0\\
-0.563617038726806	0\\
-0.51369457244873	0\\
-0.46377067565918	0\\
-0.413990783691406	0\\
-0.363846826553345	0\\
-0.313869285583496	0\\
-0.263829755783081	0\\
-0.213733005523681	0\\
-0.163687038421631	0\\
-0.113701391220093	0\\
-0.0636923789978026	0\\
-0.0136535644531248	0\\
0.0374466896057131	-0.527580896901753\\
0.0868460655212404	-0.861548999430013\\
0.137080144882202	-1.06904640492971\\
0.187013101577759	-1.20501484555072\\
0.237029504776001	-1.30639080350272\\
0.288115215301514	-1.38787176339338\\
0.336934280395508	-1.457058641451\\
0.387251806259155	-1.51445511240144\\
0.436960172653198	-1.56087376242522\\
0.487322521209717	-1.59706566213652\\
0.537163209915161	-1.62470813123991\\
0.587262582778931	-1.64714148554549\\
0.638643217086792	-1.66738683429969\\
0.687006664276123	-1.68706474399119\\
0.737519216537476	-1.70622674199785\\
0.787285280227661	-1.72389125508607\\
0.838375759124756	-1.73937740637609\\
0.887129497528076	-1.75209842684626\\
0.937354993820191	-1.76267169846915\\
0.98739595413208	-1.77159830900177\\
1.03824419975281	-1.78037069260881\\
1.08832807540894	-1.7893643270927\\
1.13733215332031	-1.7988797531475\\
1.18734998703003	-1.80761536368163\\
1.23741598129272	-1.8147303232272\\
1.28700251579285	-1.82044899135462\\
1.33715624809265	-1.82533864718357\\
1.3871440410614	-1.83084934076442\\
1.43721837997437	-1.83712216459094\\
1.48703689575195	-1.8438225840157\\
1.53733153343201	-1.85054426852162\\
1.58698172569275	-1.85574883415597\\
1.6369044303894	-1.85898885501751\\
1.68689270019531	-1.86085862158802\\
1.73750514984131	-1.86249457353938\\
1.78736562728882	-1.86498927517778\\
1.83805890083313	-1.86871212046344\\
1.8878821849823	-1.8726201725533\\
1.93722839355469	-1.8763317108469\\
1.9872805595398	-1.8783360318721\\
2.03788895606995	-1.87878100759463\\
2.08703417778015	-1.87731874469455\\
2.13730902671814	-1.87528801960025\\
2.18721742630005	-1.87393258078282\\
2.2393238067627	-1.87358577445275\\
2.28719348907471	-1.87357327768314\\
2.33826942443848	-1.87277436710474\\
2.38713521957397	-1.87044323565942\\
2.43748707771301	-1.86643020891916\\
2.487189245224	-1.86108840612042\\
2.53701467514038	-1.85669421446028\\
2.58730955123901	-1.85331847975567\\
2.6372718334198	-1.85094081471243\\
2.68767471313477	-1.84865683987618\\
2.73776693344116	-1.84519554773169\\
2.78724522590637	-1.8400663261541\\
2.83687014579773	-1.83258709276515\\
2.88723630905151	-1.82365848866311\\
2.93726320266724	-1.81548227176654\\
2.98756213188171	-1.8086422992742\\
3.03776831626892	-1.80314643380007\\
3.08719940185547	-1.79778321233425\\
3.13819379806519	-1.79165046589742\\
3.1871039390564	-1.7837637719349\\
3.2383677482605	-1.77390402599667\\
3.28719182014465	-1.76307323969468\\
3.33709425926209	-1.75272322605861\\
3.3875627040863	-1.74373537135853\\
3.43703336715698	-1.73585218311564\\
3.48778195381165	-1.72856727545786\\
3.53756542205811	-1.72075195815569\\
3.58713026046753	-1.71139358002711\\
3.63820476531982	-1.70012226968902\\
3.68715305328369	-1.68751663738612\\
3.7370192527771	-1.67560748087499\\
3.78696150779724	-1.66554550642604\\
3.83697123527527	-1.65703799218727\\
3.88720889091492	-1.64909147692197\\
3.93718166351318	-1.64114314691005\\
3.98680086135864	-1.63218791827921\\
4.0373019695282	-1.62197145823393\\
4.087202501297	-1.61056234064836\\
4.13750429153442	-1.5991603508819\\
4.18730540275574	-1.58898915701457\\
4.23692674636841	-1.57985780850004\\
4.28764815330505	-1.57174637076014\\
4.33715195655823	-1.56354304230445\\
4.38732810020447	-1.55425083368027\\
4.43775720596313	-1.54363549854043\\
4.48715395927429	-1.53175343965313\\
4.53803582191467	-1.52004697980647\\
4.58724708557129	-1.50957318477685\\
4.63719172477722	-1.50034383213415\\
4.68718476295471	-1.4922671158788\\
4.73849172592163	-1.48405666569397\\
4.78692073822022	-1.47471704348709\\
4.83752150535584	-1.46396215244931\\
4.88743896484375	-1.45190818953097\\
4.93707461357117	-1.43992140207547\\
4.98722429275513	-1.42926512775557\\
5.03730530738831	-1.41987004567181\\
5.08796877861023	-1.41161847850117\\
5.13730926513672	-1.40321136676857\\
5.1875111579895	-1.39382189629532\\
5.2383083820343	-1.38327969105876\\
5.28709311485291	-1.37172626645815\\
5.33733768463135	-1.36013557033084\\
5.38735957145691	-1.3497371389476\\
5.43840856552124	-1.3405044682263\\
5.4869279384613	-1.33219797581842\\
5.53821368217468	-1.32369300552705\\
5.58721609115601	-1.31461768550662\\
5.63834400177002	-1.30455647065401\\
5.70256943702698	-1.2935675332908\\
5.73696870803833	-1.2825061363078\\
5.78678030967712	-1.27240681227522\\
5.83695526123047	-1.26342616169882\\
5.88792462348938	-1.25530302863103\\
5.93726081848145	-1.24722756941787\\
5.98705382347107	-1.23852049051379\\
6.037406873703	-1.22893983374752\\
6.08693976402283	-1.21856132109315\\
6.13725109100342	-1.20820667172529\\
6.1868757724762	-1.1988096525431\\
6.23866934776306	-1.1904632510948\\
6.28684301376343	-1.18292885813776\\
6.3376051902771	-1.17537464244879\\
6.38692278862	-1.16726932009169\\
6.43746538162231	-1.15834000459927\\
6.48725814819336	-1.14873027485692\\
6.53829855918884	-1.13925029424877\\
6.58715434074402	-1.13051151531204\\
6.63686246871948	-1.12272946178518\\
6.68743295669556	-1.11561911120816\\
6.73686356544495	-1.10860740030444\\
6.78764243125916	-1.10121659524179\\
6.83729786872864	-1.09318418812131\\
6.8867949962616	-1.0845643441005\\
6.93705792427063	-1.07574392984634\\
6.98714203834534	-1.06756460004704\\
7.03722019195557	-1.06024863356561\\
7.0867495059967	-1.05353226450029\\
7.13819832801819	-1.04696825837391\\
7.18691987991333	-1.0401315996121\\
7.23969264030457	-1.0327373788989\\
7.28743596076965	-1.02486509737059\\
7.33690996170044	-1.01690966385786\\
7.38727087974548	-1.0094297740477\\
7.43710012435913	-1.00275197710448\\
7.48817820549011	-0.996629296282663\\
7.5378050327301	-0.990740982258558\\
7.58700842857361	-0.984643375008091\\
7.63773460388184	-0.97804275223848\\
7.68709034919739	-0.970971387680947\\
7.7376603603363	-0.963721636679111\\
7.78713698387146	-0.956898807745802\\
7.83752889633179	-0.950857510805008\\
7.88738055229187	-0.945315624859518\\
7.93729419708252	-0.940070215587753\\
7.98730368614197	-0.934659114249826\\
8.03704233169556	-0.92874864990506\\
8.08665652275085	-0.922380754395704\\
8.13837356567383	-0.915827037971326\\
8.18696184158325	-0.909644148990651\\
8.23794646263122	-0.904217251864338\\
8.28800339698791	-0.899277153620801\\
8.33725209236145	-0.894530261185196\\
8.38739914894104	-0.889618913743334\\
8.43756384849548	-0.88426286475385\\
8.48700923919678	-0.878541327915059\\
8.53716988563538	-0.872790014885481\\
8.5874659538269	-0.86737108727209\\
8.63724226951599	-0.862667552802577\\
8.68716926574707	-0.858340166156154\\
8.73704977035522	-0.854162996889841\\
8.78744978904724	-0.849890730576021\\
8.83704800605774	-0.845039118562909\\
8.88674945831299	-0.839787436103506\\
8.93740029335022	-0.834551165540347\\
8.98759908676147	-0.829659926144359\\
9.03756065368652	-0.825456473546524\\
9.08785433769226	-0.821576909385726\\
9.13731617927551	-0.817829641061735\\
9.18759603500366	-0.813905646876265\\
9.23747487068176	-0.809563509202007\\
9.28715224266052	-0.804917740314522\\
9.33708448410034	-0.800302607570757\\
9.38704843521118	-0.795985548601834\\
9.43731732368469	-0.792345921931656\\
9.48708744049072	-0.789005276206808\\
9.5378701210022	-0.785724740189949\\
9.58680291175842	-0.782282793222748\\
9.63790078163147	-0.778462251071574\\
9.68698759078979	-0.774383517391811\\
9.73723216056824	-0.770395329941664\\
9.78691930770874	-0.766640780219973\\
9.83674449920654	-0.763472678666176\\
9.8874725818634	-0.76053182022423\\
9.93801159858704	-0.757631329596677\\
9.98698897361755	-0.754582920226653\\
10.0372972011566	-0.751186835754879\\
10.0883295059204	-0.747498253630908\\
10.1372539520264	-0.743883886305582\\
10.1879517555237	-0.740515581326918\\
10.2371825695038	-0.737774587192973\\
10.287105512619	-0.735265768144188\\
10.3395609378815	-0.732843462139613\\
10.387139749527	-0.730273896385199\\
10.4386827468872	-0.727283124193661\\
10.4871279716492	-0.723988730683857\\
10.537340593338	-0.720813675242141\\
10.5872203826904	-0.717841990680768\\
10.6368417263031	-0.71550508763022\\
10.6872877597809	-0.713368899795569\\
10.7376646518707	-0.711213158570729\\
10.7870134830475	-0.708911361798243\\
10.8377515792847	-0.70621770276216\\
10.8868004798889	-0.703244876130327\\
10.9370796203613	-0.700464211448264\\
10.9869966030121	-0.697882610932311\\
11.0377678394318	-0.695869823392172\\
11.0870306015015	-0.694033342747957\\
11.1382483959198	-0.692159040992578\\
11.1873485565186	-0.690207024848291\\
11.2382921695709	-0.687925365246087\\
11.2869455337524	-0.685419965224639\\
11.337321472168	-0.682996981368433\\
11.3868953704834	-0.680723821760466\\
11.4372324466705	-0.678940559585499\\
11.4881226539612	-0.677317360087102\\
11.5377847671509	-0.675686089270329\\
11.5873648643494	-0.673919238112887\\
11.6380569458008	-0.671819333780547\\
11.6869468212128	-0.669503142094641\\
11.7386147499084	-0.667235704177813\\
11.7868301391602	-0.665081327900992\\
11.8376504898071	-0.663411987280995\\
11.8868886947632	-0.661870348755826\\
11.9389214038849	-0.660355422350989\\
11.9870438098907	-0.658712151052327\\
12.0372139930725	-0.656746031821655\\
12.0867568969727	-0.654551422683028\\
12.1371871948242	-0.652404252554959\\
12.1868307113647	-0.650364222791524\\
12.2372700691223	-0.648851488728809\\
12.2880680084229	-0.647465471830998\\
12.3380994319916	-0.646136959461643\\
12.3877601146698	-0.644611252498521\\
12.4371673583984	-0.642780544451682\\
12.4874753475189	-0.640712260114668\\
12.5367342948914	-0.638700849349618\\
12.5872320652008	-0.636801254057019\\
12.6379358291626	-0.635487710991214\\
12.6869699478149	-0.634321272223616\\
12.737691116333	-0.633199948297374\\
12.7872032642364	-0.631945598579193\\
12.837261390686	-0.630325556655691\\
12.8871109008789	-0.628509451880603\\
12.9367858886719	-0.626786420052582\\
12.9869508266449	-0.6251545735241\\
13.03677983284	-0.623995797604323\\
13.0889579772949	-0.62293613588497\\
13.1373173713684	-0.621913188273879\\
13.1868133068085	-0.620769726599889\\
13.2383656024933	-0.619327972512565\\
13.2870273113251	-0.617693507550911\\
13.3382465362549	-0.616106336807192\\
13.3877489089966	-0.614591715098229\\
13.4376754283905	-0.613514021361226\\
13.4869431972504	-0.612542721482484\\
13.5382995128632	-0.611704355994405\\
13.5876717090607	-0.610779453449368\\
13.6373288154602	-0.60956503083969\\
13.6871709346771	-0.608167122534141\\
13.7372221469879	-0.606731681039903\\
13.7870430469513	-0.605297235447395\\
13.8368041038513	-0.604310556696419\\
13.8874387264252	-0.603411725069535\\
13.9370603084564	-0.602686647022821\\
13.9871797084808	-0.601865675629\\
14.0383731842041	-0.600782239755972\\
14.0870515823364	-0.599508141377484\\
14.1367220401764	-0.598179295675052\\
14.1868392944336	-0.596852927410964\\
14.2372829437256	-0.595968543242151\\
14.2870769023895	-0.595173880470398\\
14.3382911205292	-0.594550507434064\\
14.3871123313904	-0.593836072661531\\
14.4397968769073	-0.59287534977679\\
14.4871720790863	-0.591732605553986\\
14.5370795249939	-0.590502636410818\\
14.587115240097	-0.589246344668211\\
14.6373021125793	-0.588379393293494\\
14.6878222942352	-0.587599727138631\\
14.7388114452362	-0.587075059128416\\
14.7875110626221	-0.586496619543041\\
14.8372318267822	-0.585713108996991\\
14.8873505115509	-0.584760770694025\\
14.9386431694031	-0.583682232664671\\
14.9871870994568	-0.58251538909802\\
15.0373312950134	-0.581703485574998\\
15.0867146968842	-0.58097434561666\\
15.1386305809021	-0.58052078054633\\
15.1868702888489	-0.580018654108187\\
15.2371518135071	-0.579326613246614\\
15.2869948863983	-0.578473425754652\\
15.3384689807892	-0.57749448159926\\
15.3868526935577	-0.576412911707777\\
15.4374238967896	-0.575668866580266\\
15.4874579429626	-0.575011605157655\\
15.5373572826385	-0.57464761132443\\
15.5874838352203	-0.57425455176795\\
15.6373798370361	-0.573662243327526\\
15.6868905544281	-0.572896557826759\\
15.737205696106	-0.57197895124164\\
15.7869731903076	-0.570972251813721\\
15.8384644508362	-0.570252726758142\\
15.887641620636	-0.569609988816705\\
15.938517999649	-0.569283021326644\\
15.9870197296143	-0.568943867385084\\
16.0382165431976	-0.568449650521078\\
16.087006521225	-0.567792577694203\\
16.1369261264801	-0.566965367006933\\
16.187291097641	-0.565997141647138\\
16.2368158817291	-0.565303247645723\\
16.2877594947815	-0.564683598974533\\
16.3371722221375	-0.564394059208755\\
16.3871049404144	-0.564097779231691\\
16.4373132705688	-0.563634796630083\\
16.4878031730652	-0.562999814101968\\
16.5373649120331	-0.562176806340123\\
16.5867781162262	-0.561207077684456\\
16.6383754730225	-0.560569616606363\\
16.6873669147491	-0.560031267576841\\
16.7393970012665	-0.559839761990416\\
16.7873897075653	-0.559636161954721\\
16.8394999027252	-0.559282503878519\\
16.8871914863586	-0.558779676404132\\
16.9372188568115	-0.558069508324156\\
16.9868101596832	-0.557213145392621\\
17.0378970623016	-0.556608601606259\\
17.0877410888672	-0.556071640935081\\
17.1387340545654	-0.555872733361192\\
17.1871625900269	-0.555684010362417\\
17.2373549461365	-0.555354920973104\\
17.2877740383148	-0.554883240803835\\
17.3370270252228	-0.554225685071569\\
17.3883034706116	-0.553435930784724\\
17.4389266490936	-0.552894905116869\\
17.4871525287628	-0.552420551695501\\
17.5380241394043	-0.552258776337305\\
17.5868267536163	-0.552097336154825\\
17.6374902248383	-0.551787553657117\\
17.6872567653656	-0.551322666554098\\
17.7372834205627	-0.550665047406204\\
17.7898430347443	-0.549866600298101\\
17.8382009983063	-0.549352788507855\\
17.8908442974091	-0.548930112913272\\
17.9381300926208	-0.548859642492287\\
17.9873158454895	-0.54880771054032\\
18.038357925415	-0.548611544068917\\
18.0870062828064	-0.548288841463322\\
18.1371778964996	-0.547751400704518\\
18.1874336719513	-0.547086845260684\\
18.2375645160675	-0.54658754795036\\
18.2877232551575	-0.546143606480918\\
18.3384027004242	-0.546018101028139\\
18.3875796318054	-0.545926169636402\\
18.4372620105743	-0.545751517325341\\
18.4870087623596	-0.545456447650498\\
18.5369572162628	-0.544948342399241\\
18.5870844841003	-0.544303575566801\\
18.6373524188995	-0.543811334221088\\
18.6875247478485	-0.543367844626143\\
18.7381965637207	-0.543263189707174\\
18.787005853653	-0.543192537438753\\
18.8369290351868	-0.543050711872835\\
18.8874282360077	-0.542791465773224\\
18.9384850978851	-0.542314581787888\\
18.9872798442841	-0.541690645840925\\
19.0383550643921	-0.541190489493967\\
19.0876242637634	-0.540733885532713\\
19.13878865242	-0.540634642074378\\
19.1871997833252	-0.54058369122213\\
19.238090467453	-0.540521807900745\\
19.2875425338745	-0.540372588607534\\
19.3374142169952	-0.540007230753902\\
19.3872191429138	-0.539518897145609\\
19.4372310161591	-0.539065547476262\\
19.4873599529266	-0.538613149979076\\
19.5374044895172	-0.538452563871813\\
19.5872218132019	-0.538332095854079\\
19.636870098114	-0.538274689173747\\
19.6870867729187	-0.538142815488818\\
19.7377478599548	-0.537825871130885\\
19.7868735313416	-0.53738131982325\\
19.8382687091827	-0.53693684472859\\
19.8874859333038	-0.536485700000892\\
19.9384080886841	-0.536311710472717\\
19.9873585224152	-0.536172088194348\\
20.0380374908447	-0.536118673170346\\
20.0872175216675	-0.535982555481279\\
20.1384467601776	-0.535667962800726\\
20.1871039390564	-0.535216292384479\\
20.2371427536011	-0.534757529038522\\
20.2878586769104	-0.534284493141637\\
20.3390125751495	-0.534114383014714\\
20.3873836517334	-0.533984526349617\\
20.4368681430817	-0.533959687242305\\
20.4869358062744	-0.533870519619175\\
20.5373003005981	-0.533612118258965\\
20.5874337673187	-0.533227842777208\\
20.6376835823059	-0.532800720909719\\
20.6868483543396	-0.532320045801718\\
20.7382573604584	-0.532114095309133\\
20.7872662067413	-0.53195136736748\\
20.8385295391083	-0.531953450740453\\
20.8871445178986	-0.531898551605412\\
20.9377784252167	-0.531674424941677\\
20.9873167991638	-0.531321146921883\\
21.0377876281738	-0.530908337409358\\
21.0882877826691	-0.530444541705421\\
21.1374673366547	-0.530253943232701\\
21.1871928691864	-0.530114229115991\\
21.2374176502228	-0.530143534195627\\
21.2877287387848	-0.530120370488039\\
21.3368806362152	-0.529933411644205\\
21.3871657371521	-0.529619768640753\\
21.4381284236908	-0.52921975248789\\
21.4872109413147	-0.528752000251146\\
21.5376049995422	-0.528552843457934\\
21.5874728679657	-0.5284088745065\\
21.6380285739899	-0.528447648248331\\
21.6870588779449	-0.528448645680008\\
21.739088010788	-0.528300922913321\\
21.7871963500977	-0.528033554970897\\
21.8373071670532	-0.527676418261695\\
21.8888287067413	-0.527234088743025\\
21.9381830215454	-0.5270200738074\\
21.9872478961945	-0.526851215123344\\
22.0378989696503	-0.526878743671421\\
22.0872938156128	-0.526861923406553\\
22.138090801239	-0.526712067067425\\
22.1868211746216	-0.526451160366474\\
22.2377860069275	-0.526120836203901\\
22.2874586105347	-0.525715718202754\\
22.3384885311127	-0.525522047902044\\
22.3870710849762	-0.525359803347378\\
22.4387132644653	-0.525364156241206\\
22.4868266105652	-0.525321673338684\\
22.5373594284058	-0.525167827296073\\
22.5873472213745	-0.524914028503904\\
22.6370512962341	-0.524607179947683\\
22.6884295463562	-0.524215234907622\\
22.7378239154816	-0.524007163254929\\
22.787751865387	-0.523827954570098\\
22.8374096870422	-0.523829100409188\\
22.8873886585236	-0.523819610605582\\
22.9379829883575	-0.523732415831073\\
22.9869961261749	-0.523563200450202\\
23.037150812149	-0.52329824786559\\
23.0874580860138	-0.52293937910807\\
23.1373772144318	-0.522693177377491\\
23.1871411323547	-0.522462107716688\\
23.237290096283	-0.522427943369703\\
23.2868723392487	-0.522397343852724\\
23.3371350288391	-0.522328892707623\\
23.3872100830078	-0.52218581740669\\
23.4371966838837	-0.521941684778213\\
23.4883241176605	-0.521597679412022\\
23.5368735313416	-0.521340164315401\\
23.5883466720581	-0.521102100760778\\
23.6373395442963	-0.521077508290574\\
23.6870235919952	-0.521068007042015\\
23.7381245613098	-0.521025965667221\\
23.7868911743164	-0.52090811114353\\
23.8377303600311	-0.520663191493927\\
23.8871185302734	-0.520316319792201\\
23.9379884719849	-0.520060400210038\\
23.9873661518097	-0.519830411113439\\
24.0375952243805	-0.519828626804753\\
24.0870673179626	-0.519854658220336\\
24.1371151924133	-0.519863332021853\\
24.1869523048401	-0.51980893513004\\
24.2370559692383	-0.519607326817628\\
24.2891807079315	-0.519313472560174\\
24.3380689144135	-0.519097494745164\\
24.3871647834778	-0.518922434593033\\
24.4377190589905	-0.518965737401395\\
24.4868864536285	-0.519029095886065\\
24.5379740715027	-0.519040813682441\\
24.5869614601135	-0.518983687922759\\
24.6376108646393	-0.518789832098996\\
24.6870998859406	-0.518520794746451\\
24.7377688407898	-0.51834189255176\\
24.7873253345489	-0.518193860406388\\
24.8372883319855	-0.518233530126111\\
24.8869270801544	-0.518299981442436\\
24.9384078502655	-0.518362055878782\\
24.9870955467224	-0.518391405389217\\
25.0383190631866	-0.518320784192852\\
25.0877341747284	-0.518219499678278\\
25.1370951652527	-0.518120058437816\\
25.1875030517578	-0.518029354759967\\
25.2380036830902	-0.518037260133386\\
25.2871031284332	-0.518063244456179\\
25.338099193573	-0.51812452665974\\
25.386954498291	-0.518192953448199\\
25.4379999160767	-0.518193638745558\\
25.4874267101288	-0.518167764097534\\
25.5373079299927	-0.518088588598854\\
25.5872232437134	-0.517981381183247\\
25.6382319450378	-0.517926789987502\\
25.6868111610413	-0.517884332873827\\
25.7375020503998	-0.517902794708633\\
25.7869216918945	-0.517937709411605\\
25.8372537612915	-0.517939768531125\\
25.8889283657074	-0.51790918872473\\
25.9383606433868	-0.517798459507976\\
25.9875232696533	-0.517637850261945\\
26.0369369506836	-0.517488874050834\\
26.0872263431549	-0.517330979818414\\
26.1380128383636	-0.517283438474937\\
26.1871096611023	-0.517250607185137\\
26.2377979278564	-0.517280366601877\\
26.2872809886932	-0.517279168884787\\
26.3384959220886	-0.51720520213772\\
26.387292098999	-0.517073726307924\\
26.4373895645142	-0.516922633842082\\
26.4868809700012	-0.51674883945001\\
26.53678150177	-0.516681970057256\\
26.5870432376862	-0.51663123521169\\
26.6371018409729	-0.516662210291013\\
26.6869751930237	-0.516685351103323\\
26.7378837585449	-0.516648457060889\\
26.7877766609192	-0.516555856700354\\
26.8378419399261	-0.516408519156101\\
26.8871218681335	-0.516230322690177\\
26.9376534938812	-0.516173255735932\\
26.987083864212	-0.516147455235242\\
27.0375470638275	-0.516219482786717\\
27.0869986534119	-0.516279641987733\\
27.137539100647	-0.516244156353341\\
27.1868445396423	-0.516148621598433\\
27.2373196601868	-0.515996611711639\\
27.286718082428	-0.515837790720013\\
27.3369435787201	-0.515809514905264\\
27.3872541904449	-0.515813094734508\\
27.4369372844696	-0.515891195263421\\
27.4883272171021	-0.515943202286017\\
27.5371679782867	-0.51588721202582\\
27.5876988887787	-0.515782274178534\\
27.6373111724854	-0.515665718026732\\
27.6871180057526	-0.515548201900607\\
27.7367758274078	-0.515528272205607\\
27.7870089530945	-0.515523094469128\\
27.8375215053558	-0.515550110177919\\
27.8871633529663	-0.515555324318534\\
27.940003824234	-0.515507125293841\\
27.9871308326721	-0.515435564264045\\
28.0378810882568	-0.515360488090682\\
28.0873820304871	-0.515278494973192\\
28.1369366168976	-0.515252398665051\\
28.1869079589844	-0.515227244411339\\
28.2368185043335	-0.515222511057562\\
28.2877447128296	-0.515214358590288\\
28.3373779773712	-0.51519385482862\\
28.3870558261871	-0.515166860149812\\
28.4374107837677	-0.515136519983407\\
28.4871124744415	-0.515095720903741\\
28.5371062278748	-0.515055660941087\\
28.5868112564087	-0.515003748743236\\
28.6371838569641	-0.514960693193363\\
28.6869341850281	-0.514911157680874\\
28.7380611419678	-0.514885505060494\\
28.7870323181152	-0.514860397591697\\
28.8377327442169	-0.514850724129147\\
28.8874563694	-0.514837248497063\\
28.9368562221527	-0.514813598024794\\
28.9870872020721	-0.514774836798225\\
29.0385212421417	-0.514736129865959\\
29.0873689174652	-0.514697712745814\\
29.1368615150452	-0.514682779834079\\
29.187535238266	-0.514668248898008\\
29.2372958183289	-0.514652273407261\\
29.2871133804321	-0.514623532931488\\
29.3380169391632	-0.514588837368798\\
29.3868567466736	-0.514553797889665\\
29.4373459339142	-0.514539430856658\\
29.4873344421387	-0.514523661134294\\
29.5384788036346	-0.514505983886114\\
29.5869259357452	-0.514476426373413\\
29.6376266002655	-0.514440683377853\\
29.6871110916138	-0.514404493114075\\
29.7390245914459	-0.514387204955298\\
29.7868015289307	-0.514368967607798\\
29.8379787921906	-0.514354253016467\\
29.8871769428253	-0.514329349217103\\
29.9371747493744	-0.514291935368277\\
29.987563085556	-0.514246639563737\\
30.0383028507233	-0.514220558268011\\
30.0876338005066	-0.51419384728516\\
30.1368801116943	-0.514183512155363\\
30.1873883724213	-0.514166384453741\\
30.2370848178864	-0.514135293650646\\
30.2871500968933	-0.514090992085933\\
30.3385204792023	-0.514050526522272\\
30.386811208725	-0.514009942331452\\
30.4384216785431	-0.513993114795809\\
30.4868461608887	-0.513974421224816\\
30.5371896743774	-0.513956010776774\\
30.5867259025574	-0.513931816561655\\
30.6380993843079	-0.513898974380034\\
30.6876494407654	-0.513856083517748\\
30.7367903709412	-0.513821158044699\\
30.7870404243469	-0.51378179785539\\
30.8369943618774	-0.513758780000018\\
30.8871693134308	-0.513729841393097\\
30.9381038665771	-0.513701651007029\\
30.986993265152	-0.513661744893408\\
31.0379008769989	-0.513617992866831\\
31.0870007991791	-0.513565582766699\\
31.1389050006866	-0.513533319828188\\
31.187396478653	-0.513501501935572\\
31.2372066497803	-0.513480293844541\\
31.2869581699371	-0.513445267028334\\
31.3370429992676	-0.513392624338195\\
31.3868891716003	-0.513323712235668\\
31.4371098995209	-0.513269372037776\\
31.488369178772	-0.513216614410901\\
31.5368852138519	-0.5131922061314\\
31.5877389431	-0.513157450458024\\
31.6370863437653	-0.513117079278697\\
31.6871823787689	-0.513061411981887\\
31.738449048996	-0.513005767779656\\
31.7871157646179	-0.512946925066394\\
31.8383152008057	-0.5129175728907\\
31.8873555183411	-0.512885293118483\\
31.9394099235535	-0.512855017520359\\
31.9871487140656	-0.51280675533004\\
32.0381509780884	-0.512742868660371\\
32.088035774231	-0.512665744410782\\
32.1374802112579	-0.512616773139015\\
32.187110376358	-0.512569513767115\\
32.2387628078461	-0.512550910042094\\
32.2874614715576	-0.512524893376536\\
32.3373488903046	-0.512486090065007\\
32.3874296665192	-0.512428721111419\\
32.4374224662781	-0.512364506306955\\
32.4878580093384	-0.512293510172622\\
32.5376955986023	-0.512253411073349\\
32.5874928951263	-0.51221387516139\\
32.6386348724365	-0.512194335471978\\
32.6873406887054	-0.512163346771509\\
32.7388579368591	-0.512114262225332\\
32.7873272418976	-0.512045402818451\\
32.8374335289001	-0.511976373677417\\
32.8873004436493	-0.511903511224219\\
32.937156867981	-0.511871257637829\\
32.9873573303223	-0.511842059161705\\
33.0380663394928	-0.511824924043726\\
33.0887851238251	-0.511797687760208\\
33.1397532939911	-0.511748378065185\\
33.1896943569183	-0.51168037219087\\
33.2378904342651	-0.511617776900827\\
33.2873873233795	-0.511556763753902\\
33.337309551239	-0.511534532585806\\
33.3870064735413	-0.511511631791709\\
33.4376117706299	-0.511489269371705\\
33.4877395153046	-0.511454833024363\\
33.5375718593597	-0.511403516239895\\
33.5879704475403	-0.51134521044083\\
33.6378068447113	-0.511304403817231\\
33.6874708652496	-0.511264368129119\\
33.7375998020172	-0.511243439339623\\
33.7882034301758	-0.511213217531161\\
33.8377136707306	-0.511166359542006\\
33.8892471313477	-0.511103327701839\\
33.9370285987854	-0.511045995092896\\
33.9873494625092	-0.510990424055322\\
34.0370041847229	-0.510968686057669\\
34.0873434066772	-0.510946001310788\\
34.1362752437592	-0.510946001310788\\
34.1862122535706	-0.510946001310788\\
34.2362687110901	-0.510946001310788\\
34.2865228176117	-0.510946001310788\\
34.3363995075226	-0.510946001310788\\
34.3862075328827	-0.510946001310788\\
34.4361023426056	-0.510946001310788\\
34.4861790657043	-0.510946001310788\\
34.5361194133759	-0.510946001310788\\
34.586292219162	-0.510946001310788\\
34.6364039897919	-0.510946001310788\\
34.6863061904907	-0.510946001310788\\
34.7361094474793	-0.510946001310788\\
34.7864546298981	-0.510946001310788\\
34.83685297966	-0.510946001310788\\
34.8863391399384	-0.510946001310788\\
34.9360925674439	-0.510946001310788\\
34.986105632782	-0.510946001310788\\
35.0454177379608	-0.510946001310788\\
35.0912179470062	-0.510946001310788\\
35.136381816864	-0.510946001310788\\
35.186456155777	-0.510946001310788\\
35.2363311767578	-0.510946001310788\\
35.2861754417419	-0.510946001310788\\
35.3360485553741	-0.510946001310788\\
35.3861937046051	-0.510946001310788\\
35.4363774776459	-0.510946001310788\\
35.486319732666	-0.510946001310788\\
35.536076259613	-0.510946001310788\\
35.5862481117249	-0.510946001310788\\
35.6363558292389	-0.510946001310788\\
35.6861931800842	-0.510946001310788\\
35.7363302230835	-0.510946001310788\\
35.786288690567	-0.510946001310788\\
35.8361827850342	-0.510946001310788\\
35.8864717006683	-0.510946001310788\\
35.9363789081574	-0.510946001310788\\
35.9867715358734	-0.510946001310788\\
36.0360667228699	-0.510946001310788\\
36.0863480091095	-0.510946001310788\\
36.1363653659821	-0.510946001310788\\
36.1864337444305	-0.510946001310788\\
36.2360584259033	-0.510946001310788\\
36.2864779949188	-0.510946001310788\\
36.336106967926	-0.510946001310788\\
36.3864702701569	-0.510946001310788\\
36.4362844944	-0.510946001310788\\
36.4863364219666	-0.510946001310788\\
36.5363025188446	-0.510946001310788\\
36.5861410617828	-0.510946001310788\\
36.6363481998444	-0.510946001310788\\
36.6863061904907	-0.510946001310788\\
36.7361874103546	-0.510946001310788\\
36.7863311290741	-0.510946001310788\\
36.8363565921783	-0.510946001310788\\
36.8863608360291	-0.510946001310788\\
36.9363264560699	-0.510946001310788\\
36.9860896587372	-0.510946001310788\\
37.0362286090851	-0.510946001310788\\
37.0864164352417	-0.510946001310788\\
37.1363679885864	-0.510946001310788\\
37.1863879680634	-0.510946001310788\\
37.2362648963928	-0.510946001310788\\
37.2863585472107	-0.510946001310788\\
37.3360833644867	-0.510946001310788\\
37.3862823963165	-0.510946001310788\\
37.4368352413178	-0.510946001310788\\
37.4861456871033	-0.510946001310788\\
37.5361048698425	-0.510946001310788\\
37.5862543106079	-0.510946001310788\\
37.6362061023712	-0.510946001310788\\
37.6863986968994	-0.510946001310788\\
37.736749124527	-0.510946001310788\\
37.7863394737244	-0.510946001310788\\
37.8361865997314	-0.510946001310788\\
37.888374042511	-0.510946001310788\\
37.9363502979279	-0.510946001310788\\
37.9863593101501	-0.510946001310788\\
38.0362567424774	-0.510946001310788\\
38.0862390518189	-0.510946001310788\\
38.1364056587219	-0.510946001310788\\
38.1864752292633	-0.510946001310788\\
38.2363790988922	-0.510946001310788\\
38.2863652229309	-0.510946001310788\\
38.3364450454712	-0.510946001310788\\
38.3863930225372	-0.510946001310788\\
38.4377541065216	-0.510946001310788\\
38.4863030433655	-0.510946001310788\\
38.5364169597626	-0.510946001310788\\
38.5863399028778	-0.510946001310788\\
38.6360737800598	-0.510946001310788\\
38.686456155777	-0.510946001310788\\
38.7360977649689	-0.510946001310788\\
38.7863506793976	-0.510946001310788\\
38.8363575458527	-0.510946001310788\\
38.8868772506714	-0.510946001310788\\
38.9363157272339	-0.510946001310788\\
38.9863695621491	-0.510946001310788\\
39.036302280426	-0.510946001310788\\
39.0863775730133	-0.510946001310788\\
39.1364001750946	-0.510946001310788\\
39.1863421916962	-0.510946001310788\\
39.2363414287567	-0.510946001310788\\
39.2865790843964	-0.510946001310788\\
39.3364316940308	-0.510946001310788\\
39.3863877773285	-0.510946001310788\\
39.4363846302032	-0.510946001310788\\
39.4864034175873	-0.510946001310788\\
39.5362460136414	-0.510946001310788\\
39.5864672183991	-0.510946001310788\\
39.6376175403595	-0.510946001310788\\
39.6865121841431	-0.510946001310788\\
39.7360577106476	-0.510946001310788\\
39.7860879421234	-0.510946001310788\\
39.8361377239227	-0.510946001310788\\
39.8862397193909	-0.510946001310788\\
39.9363758087158	-0.510946001310788\\
39.9866580486298	-0.510946001310788\\
40.0361229896545	-0.510946001310788\\
40.0863337039948	-0.510946001310788\\
40.1362952709198	-0.510946001310788\\
40.1863722324371	-0.510946001310788\\
40.2360808372498	-0.510946001310788\\
40.2867063999176	-0.510946001310788\\
40.3362371444702	-0.510946001310788\\
40.3864602565765	-0.510946001310788\\
40.4362022399902	-0.510946001310788\\
40.4862391471863	-0.510946001310788\\
40.536110830307	-0.510946001310788\\
40.586754989624	-0.510946001310788\\
40.6363067150116	-0.510946001310788\\
40.6862320423126	-0.510946001310788\\
40.7363345146179	-0.510946001310788\\
40.7863447189331	-0.510946001310788\\
40.8362771987915	-0.510946001310788\\
40.886261177063	-0.510946001310788\\
40.936424446106	-0.510946001310788\\
40.9864334583282	-0.510946001310788\\
41.0363933563232	-0.510946001310788\\
41.0863849639893	-0.510946001310788\\
41.1362373352051	-0.510946001310788\\
41.1860804080963	-0.510946001310788\\
41.2361869335175	-0.510946001310788\\
41.2864078998566	-0.510946001310788\\
41.3360969543457	-0.510946001310788\\
41.3861855983734	-0.510946001310788\\
41.4376759052277	-0.510946001310788\\
41.486256313324	-0.510946001310788\\
41.536371421814	-0.510946001310788\\
41.5862829208374	-0.510946001310788\\
41.636062335968	-0.510946001310788\\
41.686300945282	-0.510946001310788\\
41.7360789299011	-0.510946001310788\\
41.7862710475922	-0.510946001310788\\
41.8360845565796	-0.510946001310788\\
41.8863353252411	-0.510946001310788\\
41.9362205982208	-0.510946001310788\\
41.986283493042	-0.510946001310788\\
42.0364117145538	-0.510946001310788\\
42.0863775730133	-0.510946001310788\\
42.1363970756531	-0.510946001310788\\
42.186447095871	-0.510946001310788\\
42.2362341403961	-0.510946001310788\\
42.2862476825714	-0.510946001310788\\
42.3368155479431	-0.510946001310788\\
42.3862549781799	-0.510946001310788\\
42.4363796234131	-0.510946001310788\\
42.4863807678223	-0.510946001310788\\
42.5361065387726	-0.510946001310788\\
42.586087179184	-0.510946001310788\\
42.6363203048706	-0.510946001310788\\
42.6863209724426	-0.510946001310788\\
42.7362560749054	-0.510946001310788\\
42.7863132476807	-0.510946001310788\\
42.8362674236298	-0.510946001310788\\
42.8862435340881	-0.510946001310788\\
42.9362756729126	-0.510946001310788\\
42.9860562801361	-0.510946001310788\\
43.0366630077362	-0.510946001310788\\
43.0862891197205	-0.510946001310788\\
43.1369411468506	-0.510946001310788\\
43.1862763881683	-0.510946001310788\\
43.2363652706146	-0.510946001310788\\
43.2866267681122	-0.510946001310788\\
43.3361598968506	-0.510946001310788\\
43.3862552165985	-0.510946001310788\\
43.4363893985748	-0.510946001310788\\
43.4867214679718	-0.510946001310788\\
43.5363444805145	-0.510946001310788\\
43.5866703510284	-0.510946001310788\\
43.6362726211548	-0.510946001310788\\
43.6863271713257	-0.510946001310788\\
43.7361363887787	-0.510946001310788\\
43.7873098373413	-0.510946001310788\\
43.8363158226013	-0.510946001310788\\
43.8864092350006	-0.510946001310788\\
43.9364137172699	-0.510946001310788\\
43.9862827777863	-0.510946001310788\\
44.0369226455689	-0.510946001310788\\
44.0863043785095	-0.510946001310788\\
44.1360961914063	-0.510946001310788\\
44.1862127304077	-0.510946001310788\\
44.236283493042	-0.510946001310788\\
44.2865638256073	-0.510946001310788\\
44.3360812187195	-0.510946001310788\\
44.386487197876	-0.510946001310788\\
44.4361028194428	-0.510946001310788\\
44.4864933013916	-0.510946001310788\\
44.536284160614	-0.510946001310788\\
44.5863046169281	-0.510946001310788\\
44.6362940788269	-0.510946001310788\\
44.6862978458405	-0.510946001310788\\
44.7364215373993	-0.510946001310788\\
44.7863480567932	-0.510946001310788\\
44.8362671852112	-0.510946001310788\\
44.8862499713898	-0.510946001310788\\
44.9360589504242	-0.510946001310788\\
44.9889430522919	-0.510946001310788\\
45.0363490104675	-0.510946001310788\\
45.0864402770996	-0.510946001310788\\
45.1361271858215	-0.510946001310788\\
45.1861283302307	-0.510946001310788\\
45.2363178253174	-0.510946001310788\\
45.2860579013824	-0.510946001310788\\
45.3360833644867	-0.510946001310788\\
45.3861248016357	-0.510946001310788\\
45.4362778186798	-0.510946001310788\\
45.4862968444824	-0.510946001310788\\
45.5364520072937	-0.510946001310788\\
45.5863546848297	-0.510946001310788\\
45.6363763332367	-0.510946001310788\\
45.6862480163574	-0.510946001310788\\
45.7361382961273	-0.510946001310788\\
45.7863018035889	-0.510946001310788\\
45.8363010406494	-0.510946001310788\\
45.8863670349121	-0.510946001310788\\
45.9360751628876	-0.510946001310788\\
45.9862565517426	-0.510946001310788\\
46.036564540863	-0.510946001310788\\
46.086417388916	-0.510946001310788\\
46.136324596405	-0.510946001310788\\
46.1863011837006	-0.510946001310788\\
46.236340713501	-0.510946001310788\\
46.286172580719	-0.510946001310788\\
46.3363940238953	-0.510946001310788\\
46.3862518787384	-0.510946001310788\\
46.4363107204437	-0.510946001310788\\
46.4862446308136	-0.510946001310788\\
46.5362638950348	-0.510946001310788\\
46.5863682746887	-0.510946001310788\\
46.6360892772675	-0.510946001310788\\
46.6863905906677	-0.510946001310788\\
46.7362088680267	-0.510946001310788\\
46.7864021778107	-0.510946001310788\\
46.8362369060516	-0.510946001310788\\
46.8863255500793	-0.510946001310788\\
46.9363812923431	-0.510946001310788\\
46.9860479354858	-0.510946001310788\\
47.0364589214325	-0.510946001310788\\
47.086301279068	-0.510946001310788\\
47.1364011287689	-0.510946001310788\\
47.1863252639771	-0.510946001310788\\
47.2360765457153	-0.510946001310788\\
47.2863361358643	-0.510946001310788\\
47.336061668396	-0.510946001310788\\
47.3863124370575	-0.510946001310788\\
47.436296415329	-0.510946001310788\\
47.4863233089447	-0.510946001310788\\
47.5360545635223	-0.510946001310788\\
47.5861608505249	-0.510946001310788\\
47.6362637996674	-0.510946001310788\\
47.6860692024231	-0.510946001310788\\
47.7363540649414	-0.510946001310788\\
47.7863378047943	-0.510946001310788\\
47.8363644599915	-0.510946001310788\\
47.8862695217133	-0.510946001310788\\
47.9362542152405	-0.510946001310788\\
47.9863764762878	-0.510946001310788\\
48.0360953330994	-0.510946001310788\\
48.0863441944122	-0.510946001310788\\
48.1360756874085	-0.510946001310788\\
48.186307144165	-0.510946001310788\\
48.2360848903656	-0.510946001310788\\
48.2862736701965	-0.510946001310788\\
48.3363034248352	-0.510946001310788\\
48.3863589286804	-0.510946001310788\\
48.4373104095459	-0.510946001310788\\
48.4862935066223	-0.510946001310788\\
48.5363015651703	-0.510946001310788\\
48.5862574100494	-0.510946001310788\\
48.6363148212433	-0.510946001310788\\
48.6862716197968	-0.510946001310788\\
48.7360643863678	-0.510946001310788\\
48.7863087177277	-0.510946001310788\\
48.8364052295685	-0.510946001310788\\
48.8860489845276	-0.510946001310788\\
48.9366774082184	-0.510946001310788\\
48.9863929271698	-0.510946001310788\\
49.0368101119995	-0.510946001310788\\
49.0863396644592	-0.510946001310788\\
49.1362618923187	-0.510946001310788\\
49.1863467216492	-0.510946001310788\\
49.2363132953644	-0.510946001310788\\
49.2862998962402	-0.510946001310788\\
49.3360840797424	-0.510946001310788\\
49.3862306594849	-0.510946001310788\\
49.4362706661224	-0.510946001310788\\
49.486224603653	-0.510946001310788\\
49.5363234996796	-0.510946001310788\\
49.5862879276276	-0.510946001310788\\
49.6361152648926	-0.510946001310788\\
49.6863047599793	-0.510946001310788\\
49.7360612869263	-0.510946001310788\\
49.7862214565277	-0.510946001310788\\
49.8363184452057	-0.510946001310788\\
49.8879334449768	-0.510946001310788\\
49.9363319396973	-0.510946001310788\\
49.9862370014191	-0.510946001310788\\
50.0362956047058	-0.510946001310788\\
50.0868522644043	-0.510946001310788\\
50.1361271858215	-0.510946001310788\\
50.1862706661224	-0.510946001310788\\
50.2362811088562	-0.510946001310788\\
50.2863530635834	-0.510946001310788\\
50.3364486217499	-0.510946001310788\\
50.3862692832947	-0.510946001310788\\
50.4361993789673	-0.510946001310788\\
50.4862770557404	-0.510946001310788\\
50.5360641002655	-0.510946001310788\\
50.5860704898834	-0.510946001310788\\
50.636267375946	-0.510946001310788\\
50.6864208698273	-0.510946001310788\\
50.7364587306976	-0.510946001310788\\
50.786306810379	-0.510946001310788\\
50.8363444328308	-0.510946001310788\\
50.8863732337952	-0.510946001310788\\
50.9360637187958	-0.510946001310788\\
50.9860643863678	-0.510946001310788\\
51.0364918231964	-0.510946001310788\\
51.086265039444	-0.510946001310788\\
51.1363150596619	-0.510946001310788\\
51.1863760471344	-0.510946001310788\\
51.2363709926605	-0.510946001310788\\
51.2863130092621	-0.510946001310788\\
51.3363859176636	-0.510946001310788\\
51.3862800121307	-0.510946001310788\\
51.4363278865814	-0.510946001310788\\
51.4863607406616	-0.510946001310788\\
51.5360641002655	-0.510946001310788\\
51.5886544704437	-0.510946001310788\\
51.6360978603363	-0.510946001310788\\
51.688673210144	-0.510946001310788\\
51.7367107391357	-0.510946001310788\\
51.7860705375671	-0.510946001310788\\
51.8363744735718	-0.510946001310788\\
51.8862721443176	-0.510946001310788\\
51.936353635788	-0.510946001310788\\
51.9860617637634	-0.510946001310788\\
52.0368484973907	-0.510946001310788\\
52.0862779140472	-0.510946001310788\\
52.1360904693604	-0.510946001310788\\
52.1863877296448	-0.510946001310788\\
52.2361249446869	-0.510946001310788\\
52.2863645076752	-0.510946001310788\\
52.3361343860626	-0.510946001310788\\
52.3863236427307	-0.510946001310788\\
52.4363097667694	-0.510946001310788\\
52.4860250473022	-0.510946001310788\\
52.5360769748688	-0.510946001310788\\
52.5860502243042	-0.510946001310788\\
52.6363713264465	-0.510946001310788\\
52.690006685257	-0.510946001310788\\
52.7364186763763	-0.510946001310788\\
52.7863256454468	-0.510946001310788\\
52.8360683441162	-0.510946001310788\\
52.8861855983734	-0.510946001310788\\
52.9360858917236	-0.510946001310788\\
52.9860407829285	-0.510946001310788\\
53.0362543582916	-0.510946001310788\\
53.0863141536713	-0.510946001310788\\
53.1362888336182	-0.510946001310788\\
53.1872043132782	-0.510946001310788\\
53.2363035202026	-0.510946001310788\\
53.2860457420349	-0.510946001310788\\
53.3363325119019	-0.510946001310788\\
53.3861095428467	-0.510946001310788\\
53.436364364624	-0.510946001310788\\
53.4863154411316	-0.510946001310788\\
53.5363397121429	-0.510946001310788\\
53.5863167762756	-0.510946001310788\\
53.6361658096314	-0.510946001310788\\
53.6863488674164	-0.510946001310788\\
53.7363729000092	-0.510946001310788\\
53.7860550403595	-0.510946001310788\\
53.836292219162	-0.510946001310788\\
53.8862983703613	-0.510946001310788\\
53.9362475395203	-0.510946001310788\\
53.9861213684082	-0.510946001310788\\
54.0362629413605	-0.510946001310788\\
54.0862345218658	-0.510946001310788\\
54.1360654354095	-0.510946001310788\\
54.1862487316132	-0.510946001310788\\
54.2360968112946	-0.510946001310788\\
54.2862512588501	-0.510946001310788\\
54.3363804340363	-0.510946001310788\\
54.3863088607788	-0.510946001310788\\
54.4362613677979	-0.510946001310788\\
54.4864789962769	-0.510946001310788\\
54.5363761901855	-0.510946001310788\\
54.5860692977905	-0.510946001310788\\
54.6361002445221	-0.510946001310788\\
54.6862933158875	-0.510946001310788\\
54.7361409187317	-0.510946001310788\\
54.7861282348633	-0.510946001310788\\
54.8360788345337	-0.510946001310788\\
54.8863553524017	-0.510946001310788\\
54.9360541820526	-0.510946001310788\\
54.9860584259033	-0.510946001310788\\
55.0363721370697	-0.510946001310788\\
55.0868153095245	-0.510946001310788\\
55.136411857605	-0.510946001310788\\
55.1862184524536	-0.510946001310788\\
55.2393674373627	-0.510946001310788\\
55.2863261222839	-0.510946001310788\\
55.33689661026	-0.510946001310788\\
55.3863090991974	-0.510946001310788\\
55.4367873191834	-0.510946001310788\\
55.4863738536835	-0.510946001310788\\
55.5360693454742	-0.510946001310788\\
55.586337518692	-0.510946001310788\\
55.6370651245117	-0.510946001310788\\
55.6863228797913	-0.510946001310788\\
55.7360960960388	-0.510946001310788\\
55.7863347053528	-0.510946001310788\\
};
\addlegendentry{$\theta_{x_{p}^{2}}$}

\addplot [color=mycolor2, line width=1.4pt]
  table[row sep=crcr]{%
-3.41395025253296	0\\
-3.36380581855774	0\\
-3.31373219490051	0\\
-3.26373772621155	0\\
-3.21386485099792	0\\
-3.16371612548828	0\\
-3.11396126747131	0\\
-3.0639102935791	0\\
-3.01393465995789	0\\
-2.96376042366028	0\\
-2.91368894577026	0\\
-2.86386590003967	0\\
-2.81392412185669	0\\
-2.76383023262024	0\\
-2.71365766525269	0\\
-2.66386275291443	0\\
-2.61371665000916	0\\
-2.56373028755188	0\\
-2.51375107765198	0\\
-2.46365218162537	0\\
-2.41376595497131	0\\
-2.36374454498291	0\\
-2.31372838020325	0\\
-2.26391820907593	0\\
-2.21392397880554	0\\
-2.16394262313843	0\\
-2.11378173828125	0\\
-2.06323270797729	0\\
-2.01386647224426	0\\
-1.96378617286682	0\\
-1.91375093460083	0\\
-1.86376647949219	0\\
-1.81371264457703	0\\
-1.76388936042786	0\\
-1.71374897956848	0\\
-1.66370325088501	0\\
-1.61389498710632	0\\
-1.56357769966125	0\\
-1.51373534202576	0\\
-1.46395378112793	0\\
-1.41391592025757	0\\
-1.36373047828674	0\\
-1.31355028152466	0\\
-1.26392798423767	0\\
-1.21389083862305	0\\
-1.1639223575592	0\\
-1.11384873390198	0\\
-1.06395726203918	0\\
-1.01393632888794	0\\
-0.96373085975647	0\\
-0.913796234130859	0\\
-0.863754081726074	0\\
-0.81366925239563	0\\
-0.763712215423584	0\\
-0.71373085975647	0\\
-0.663900899887085	0\\
-0.61391978263855	0\\
-0.563617038726806	0\\
-0.51369457244873	0\\
-0.46377067565918	0\\
-0.413990783691406	0\\
-0.363846826553345	0\\
-0.313869285583496	0\\
-0.263829755783081	0\\
-0.213733005523681	0\\
-0.163687038421631	0\\
-0.113701391220093	0\\
-0.0636923789978026	0\\
-0.0136535644531248	0\\
0.0374466896057131	-0.550162313670157\\
0.0868460655212404	-0.88179676681618\\
0.137080144882202	-1.08092420905598\\
0.187013101577759	-1.20511896133124\\
0.237029504776001	-1.29457579938844\\
0.288115215301514	-1.36628006451247\\
0.336934280395508	-1.42818629275439\\
0.387251806259155	-1.47982073052242\\
0.436960172653198	-1.52093883363716\\
0.487322521209717	-1.55153599319158\\
0.537163209915161	-1.57320040263903\\
0.587262582778931	-1.58894685683629\\
0.638643217086792	-1.60276180658002\\
0.687006664276123	-1.61639744208014\\
0.737519216537476	-1.6299040358756\\
0.787285280227661	-1.64160805454458\\
0.838375759124756	-1.65075701202863\\
0.887129497528076	-1.65751652611107\\
0.937354993820191	-1.6619350931901\\
0.98739595413208	-1.66490232887554\\
1.03824419975281	-1.66688103487877\\
1.08832807540894	-1.66805550386403\\
1.13733215332031	-1.66840627749889\\
1.18734998703003	-1.66786405710286\\
1.23741598129272	-1.66738977537693\\
1.28700251579285	-1.66667162301064\\
1.33715624809265	-1.6655896577156\\
1.3871440410614	-1.66327994186349\\
1.43721837997437	-1.6599440358782\\
1.48703689575195	-1.65522924980519\\
1.53733153343201	-1.64887312267388\\
1.58698172569275	-1.64214118257564\\
1.6369044303894	-1.63609500164443\\
1.68689270019531	-1.63063667467713\\
1.73750514984131	-1.62545219896492\\
1.78736562728882	-1.61952955954894\\
1.83805890083313	-1.61261612207727\\
1.8878821849823	-1.60412490897897\\
1.93722839355469	-1.59396935936775\\
1.9872805595398	-1.58359852206377\\
2.03788895606995	-1.57394108942026\\
2.08703417778015	-1.5651657823305\\
2.13730902671814	-1.55655421772326\\
2.18721742630005	-1.54699278544331\\
2.2393238067627	-1.53635560570001\\
2.28719348907471	-1.52437911010566\\
2.33826942443848	-1.51117140936367\\
2.38713521957397	-1.49798188431896\\
2.43748707771301	-1.48558247175424\\
2.487189245224	-1.47443027438294\\
2.53701467514038	-1.46416705258935\\
2.58730955123901	-1.45300093818696\\
2.6372718334198	-1.44008090478701\\
2.68767471313477	-1.42500960604866\\
2.73776693344116	-1.40805136836525\\
2.78724522590637	-1.39124559880156\\
2.83687014579773	-1.37592147432088\\
2.88723630905151	-1.36268318447355\\
2.93726320266724	-1.35047797340803\\
2.98756213188171	-1.33771701633304\\
3.03776831626892	-1.32349759657745\\
3.08719940185547	-1.30707768639331\\
3.13819379806519	-1.28873113358304\\
3.1871039390564	-1.27030525444297\\
3.2383677482605	-1.25356800533973\\
3.28719182014465	-1.23895479523708\\
3.33709425926209	-1.22535274178654\\
3.3875627040863	-1.21100996504902\\
3.43703336715698	-1.19507837834453\\
3.48778195381165	-1.17704032210759\\
3.53756542205811	-1.15743368579024\\
3.58713026046753	-1.13899905857988\\
3.63820476531982	-1.1223687891279\\
3.68715305328369	-1.10783275597078\\
3.7370192527771	-1.09381441849473\\
3.78696150779724	-1.07887864801887\\
3.83697123527527	-1.0625145313229\\
3.88720889091492	-1.04466591494611\\
3.93718166351318	-1.02592515807737\\
3.98680086135864	-1.00829695387938\\
4.0373019695282	-0.992024881189081\\
4.087202501297	-0.97707079921156\\
4.13750429153442	-0.962751760072024\\
4.18730540275574	-0.947541263538369\\
4.23692674636841	-0.931680134086491\\
4.28764815330505	-0.915077850060243\\
4.33715195655823	-0.898222250127219\\
4.38732810020447	-0.882684824119224\\
4.43775720596313	-0.868177608724181\\
4.48715395927429	-0.854763508383257\\
4.53803582191467	-0.841278533705463\\
4.58724708557129	-0.827227024437434\\
4.63719172477722	-0.812547917832489\\
4.68718476295471	-0.797495437325324\\
4.73849172592163	-0.782653077676514\\
4.78692073822022	-0.768998894104698\\
4.83752150535584	-0.756038526595603\\
4.88743896484375	-0.74392329968623\\
4.93707461357117	-0.731706472985934\\
4.98722429275513	-0.719018323824002\\
5.03730530738831	-0.706072098554614\\
5.08796877861023	-0.69289563240045\\
5.13730926513672	-0.679881075639059\\
5.1875111579895	-0.667951546710356\\
5.2383083820343	-0.656769880727097\\
5.28709311485291	-0.646556866881838\\
5.33733768463135	-0.636655553843411\\
5.38735957145691	-0.626663636353442\\
5.43840856552124	-0.616522213661483\\
5.4869279384613	-0.606237457477164\\
5.53821368217468	-0.59608195063538\\
5.58721609115601	-0.586684018259348\\
5.63834400177002	-0.57780489828383\\
5.70256943702698	-0.56968557383243\\
5.73696870803833	-0.561767984511789\\
5.78678030967712	-0.553821564983082\\
5.83695526123047	-0.545725872657385\\
5.88792462348938	-0.537535526534157\\
5.93726081848145	-0.529440153332871\\
5.98705382347107	-0.52201358938828\\
6.037406873703	-0.515080394225151\\
6.08693976402283	-0.508821583054669\\
6.13725109100342	-0.502845957155387\\
6.1868757724762	-0.496879266062024\\
6.23866934776306	-0.490814118775688\\
6.28684301376343	-0.484691600071528\\
6.3376051902771	-0.478664248727\\
6.38692278862	-0.473137905802105\\
6.43746538162231	-0.468068421389944\\
6.48725814819336	-0.463577289116529\\
6.53829855918884	-0.459279888222227\\
6.58715434074402	-0.454998653444363\\
6.63686246871948	-0.450607711769408\\
6.68743295669556	-0.446164141956615\\
6.73686356544495	-0.441845014740011\\
6.78764243125916	-0.437894899678213\\
6.83729786872864	-0.434366123028667\\
6.8867949962616	-0.431261742143306\\
6.93705792427063	-0.428336730658884\\
6.98714203834534	-0.425393367416291\\
7.03722019195557	-0.422318364930106\\
7.0867495059967	-0.419189745860422\\
7.13819832801819	-0.416170062692345\\
7.18691987991333	-0.413471680696148\\
7.23969264030457	-0.411182825108881\\
7.28743596076965	-0.409186440261465\\
7.33690996170044	-0.407335951428422\\
7.38727087974548	-0.405437174801818\\
7.43710012435913	-0.403418766397863\\
7.48817820549011	-0.40142405902543\\
7.5378050327301	-0.399654454103938\\
7.58700842857361	-0.398099115183868\\
7.63773460388184	-0.396880868473318\\
7.68709034919739	-0.395839458753244\\
7.7376603603363	-0.39488764955081\\
7.78713698387146	-0.393905352183538\\
7.83752889633179	-0.392816039723812\\
7.88738055229187	-0.391785954542911\\
7.93729419708252	-0.390907229867707\\
7.98730368614197	-0.390148270501186\\
8.03704233169556	-0.389633801394439\\
8.08665652275085	-0.389214518701465\\
8.13837356567383	-0.388876694031524\\
8.18696184158325	-0.388529636806879\\
8.23794646263122	-0.388153696540712\\
8.28800339698791	-0.38784374214174\\
8.33725209236145	-0.387657622926099\\
8.38739914894104	-0.387510250878677\\
8.43756384849548	-0.387518732029775\\
8.48700923919678	-0.387573068290067\\
8.53716988563538	-0.387628529480025\\
8.5874659538269	-0.387691476423015\\
8.63724226951599	-0.387678327421099\\
8.68716926574707	-0.387674897986244\\
8.73704977035522	-0.387763849429575\\
8.78744978904724	-0.387857392679507\\
8.83704800605774	-0.388157611346969\\
8.88674945831299	-0.388507876286894\\
8.93740029335022	-0.388887769158146\\
8.98759908676147	-0.389267404584416\\
9.03756065368652	-0.389598261586116\\
9.08785433769226	-0.389862175744611\\
9.13731617927551	-0.390204817465928\\
9.18759603500366	-0.390511166572651\\
9.23747487068176	-0.391029089165158\\
9.28715224266052	-0.391639201665271\\
9.33708448410034	-0.392228361296702\\
9.38704843521118	-0.392785620955692\\
9.43731732368469	-0.39322038715602\\
9.48708744049072	-0.393556228632974\\
9.5378701210022	-0.394045143200628\\
9.58680291175842	-0.394494284621828\\
9.63790078163147	-0.395196482505639\\
9.68698759078979	-0.395963375094425\\
9.73723216056824	-0.396640064381359\\
9.78691930770874	-0.397246189352245\\
9.83674449920654	-0.397695521169908\\
9.8874725818634	-0.398020540478683\\
9.93801159858704	-0.398519034567016\\
9.98698897361755	-0.399036362763923\\
10.0372972011566	-0.399848589201838\\
10.0883295059204	-0.40070088144671\\
10.1372539520264	-0.401458287470973\\
10.1879517555237	-0.402085646261185\\
10.2371825695038	-0.402487426318999\\
10.287105512619	-0.402753024895787\\
10.3395609378815	-0.403221833325347\\
10.387139749527	-0.403731356422668\\
10.4386827468872	-0.404623417179778\\
10.4871279716492	-0.405524810389466\\
10.537340593338	-0.406298561496783\\
10.5872203826904	-0.406932966115505\\
10.6368417263031	-0.407325672399793\\
10.6872877597809	-0.407593929757013\\
10.7376646518707	-0.408126933847711\\
10.7870134830475	-0.408696740153005\\
10.8377515792847	-0.409635539904798\\
10.8868004798889	-0.410569780947299\\
10.9370796203613	-0.411335857349854\\
10.9869966030121	-0.411960751725815\\
11.0377678394318	-0.412411091508261\\
11.0870306015015	-0.412786292031683\\
11.1382483959198	-0.413424799633958\\
11.1873485565186	-0.414046721572561\\
11.2382921695709	-0.414913715244666\\
11.2869455337524	-0.415754643242451\\
11.337321472168	-0.416467066525342\\
11.3868953704834	-0.417093194448114\\
11.4372324466705	-0.417595017619931\\
11.4881226539612	-0.418020231303785\\
11.5377847671509	-0.418608296725495\\
11.5873648643494	-0.419172731779838\\
11.6380569458008	-0.419973438693432\\
11.6869468212128	-0.42073012978733\\
11.7386147499084	-0.421401797281703\\
11.7868301391602	-0.421975814714472\\
11.8376504898071	-0.422423089300537\\
11.8868886947632	-0.422773717279028\\
11.9389214038849	-0.423277268789946\\
11.9870438098907	-0.423767656580083\\
12.0372139930725	-0.424533440579353\\
12.0867568969727	-0.425298580104425\\
12.1371871948242	-0.426011469337539\\
12.1868307113647	-0.426590692461218\\
12.2372700691223	-0.4269952749558\\
12.2880680084229	-0.427259196768148\\
12.3380994319916	-0.42766667308905\\
12.3877601146698	-0.428069629052118\\
12.4371673583984	-0.428813469928514\\
12.4874753475189	-0.429566860052773\\
12.5367342948914	-0.430274029467796\\
12.5872320652008	-0.43087970954366\\
12.6379358291626	-0.431205566558219\\
12.6869699478149	-0.431368392340232\\
12.737691116333	-0.431705410539792\\
12.7872032642364	-0.432032260425501\\
12.837261390686	-0.432766950083192\\
12.8871109008789	-0.433502453729886\\
12.9367858886719	-0.434173881017358\\
12.9869508266449	-0.434753538708748\\
13.03677983284	-0.435082881914951\\
13.0889579772949	-0.435271693840178\\
13.1373173713684	-0.435560212098494\\
13.1868133068085	-0.435833236006147\\
13.2383656024933	-0.436465561560908\\
13.2870273113251	-0.43711652040659\\
13.3382465362549	-0.437792779389213\\
13.3877489089966	-0.438376940002541\\
13.4376754283905	-0.438743338668473\\
13.4869431972504	-0.43896672596594\\
13.5382995128632	-0.439222857635627\\
13.5876717090607	-0.439448970848531\\
13.6373288154602	-0.440017257135139\\
13.6871709346771	-0.440608268021549\\
13.7372221469879	-0.441283459542376\\
13.7870430469513	-0.441890321065046\\
13.8368041038513	-0.442272443971405\\
13.8874387264252	-0.442520141651841\\
13.9370603084564	-0.442784950800728\\
13.9871797084808	-0.443026710026516\\
14.0383731842041	-0.443597353023961\\
14.0870515823364	-0.444198097658358\\
14.1367220401764	-0.444863471188086\\
14.1868392944336	-0.445442914505804\\
14.2372829437256	-0.445784053159315\\
14.2870769023895	-0.445969843762668\\
14.3382911205292	-0.446185480535291\\
14.3871123313904	-0.446394405504634\\
14.4397968769073	-0.446943270372486\\
14.4871720790863	-0.447526384431768\\
14.5370795249939	-0.448199981492735\\
14.587115240097	-0.448797312425143\\
14.6373021125793	-0.449132604405946\\
14.6878222942352	-0.449314984948373\\
14.7388114452362	-0.449459303435532\\
14.7875110626221	-0.449562627087573\\
14.8372318267822	-0.450029200366401\\
14.8873505115509	-0.450528576000522\\
14.9386431694031	-0.451182710219967\\
14.9871870994568	-0.451744473465368\\
15.0373312950134	-0.452046471345668\\
15.0867146968842	-0.452171523414776\\
15.1386305809021	-0.452268791140156\\
15.1868702888489	-0.452359789651879\\
15.2371518135071	-0.452853189171407\\
15.2869948863983	-0.45338901923536\\
15.3384689807892	-0.454019730683711\\
15.3868526935577	-0.454571707009762\\
15.4374238967896	-0.454860307938603\\
15.4874579429626	-0.454990125089509\\
15.5373572826385	-0.455085595575369\\
15.5874838352203	-0.455152910985277\\
15.6373798370361	-0.455593673195494\\
15.6868905544281	-0.456063493804066\\
15.737205696106	-0.456640460917143\\
15.7869731903076	-0.457145624201118\\
15.8384644508362	-0.457431841824516\\
15.887641620636	-0.45757000714508\\
15.938517999649	-0.457671180357821\\
15.9870197296143	-0.457747220887512\\
16.0382165431976	-0.458158314520119\\
16.087006521225	-0.458596169180495\\
16.1369261264801	-0.459152177930207\\
16.187291097641	-0.459643191567693\\
16.2368158817291	-0.459912357661679\\
16.2877594947815	-0.460046084597735\\
16.3371722221375	-0.460158844523413\\
16.3871049404144	-0.460236571570725\\
16.4373132705688	-0.460624840484044\\
16.4878031730652	-0.461032489441912\\
16.5373649120331	-0.461551253486256\\
16.5867781162262	-0.462015314339169\\
16.6383754730225	-0.462299130795337\\
16.6873669147491	-0.462492931953285\\
16.7393970012665	-0.462636356683092\\
16.7873897075653	-0.462717919155047\\
16.8394999027252	-0.463030973079597\\
16.8871914863586	-0.463357184329006\\
16.9372188568115	-0.46384745838408\\
16.9868101596832	-0.464290700791423\\
17.0378970623016	-0.464589419748286\\
17.0877410888672	-0.464769111445165\\
17.1387340545654	-0.464879433571774\\
17.1871625900269	-0.464917679348737\\
17.2373549461365	-0.465215851656687\\
17.2877740383148	-0.465540835839995\\
17.3370270252228	-0.466048240750347\\
17.3883034706116	-0.466499933935623\\
17.4389266490936	-0.466777881451002\\
17.4871525287628	-0.46692059500873\\
17.5380241394043	-0.466977183582241\\
17.5868267536163	-0.466957157150823\\
17.6374902248383	-0.467200405110088\\
17.6872567653656	-0.467477559699802\\
17.7372834205627	-0.46798348495712\\
17.7898430347443	-0.468442157847974\\
17.8382009983063	-0.468714515674492\\
17.8908442974091	-0.468844190629888\\
17.9381300926208	-0.468850503745259\\
17.9873158454895	-0.468730370063746\\
18.038357925415	-0.468891248030793\\
18.0870062828064	-0.469086204691219\\
18.1371778964996	-0.469565758150438\\
18.1874336719513	-0.469995459028837\\
18.2375645160675	-0.470268911284109\\
18.2877232551575	-0.470406824330574\\
18.3384027004242	-0.47042999337258\\
18.3875796318054	-0.470356719706686\\
18.4372620105743	-0.470558089719052\\
18.4870087623596	-0.4707797068437\\
18.5369572162628	-0.471234184377471\\
18.5870844841003	-0.471640589792498\\
18.6373524188995	-0.471899059656989\\
18.6875247478485	-0.472026317762412\\
18.7381965637207	-0.472040189115836\\
18.787005853653	-0.47195203527575\\
18.8369290351868	-0.472112026283069\\
18.8874282360077	-0.472288944139333\\
18.9384850978851	-0.472709439703795\\
18.9872798442841	-0.47311320993483\\
19.0383550643921	-0.473411722738566\\
19.0876242637634	-0.473594971277425\\
19.13878865242	-0.473631033434363\\
19.1871997833252	-0.473533974554552\\
19.238090467453	-0.473634261127446\\
19.2875425338745	-0.473760757587488\\
19.3374142169952	-0.474171004878502\\
19.3872191429138	-0.474554488981507\\
19.4372310161591	-0.474832159735519\\
19.4873599529266	-0.474997119901551\\
19.5374044895172	-0.475042733307486\\
19.5872218132019	-0.475003511694562\\
19.636870098114	-0.475176282318921\\
19.6870867729187	-0.475378801827807\\
19.7377478599548	-0.475795509887916\\
19.7868735313416	-0.476183310676546\\
19.8382687091827	-0.476440619028083\\
19.8874859333038	-0.476592960449292\\
19.9384080886841	-0.476625479442962\\
19.9873585224152	-0.476557928366817\\
20.0380374908447	-0.476681444473108\\
20.0872175216675	-0.476828490920754\\
20.1384467601776	-0.477195710547794\\
20.1871039390564	-0.477557828424209\\
20.2371427536011	-0.477840437039577\\
20.2878586769104	-0.478022147263765\\
20.3390125751495	-0.478062602092541\\
20.3873836517334	-0.477981039180099\\
20.4368681430817	-0.478036640388949\\
20.4869358062744	-0.478117822010711\\
20.5373003005981	-0.478473294884999\\
20.5874337673187	-0.47882438603582\\
20.6376835823059	-0.479092816975992\\
20.6868483543396	-0.479248392353385\\
20.7382573604584	-0.479234111705125\\
20.7872662067413	-0.479092249434466\\
20.8385295391083	-0.479125006278296\\
20.8871445178986	-0.479195530571833\\
20.9377784252167	-0.479547630641891\\
20.9873167991638	-0.479879911717438\\
21.0377876281738	-0.480111900140839\\
21.0882877826691	-0.480231617758152\\
21.1374673366547	-0.480193164185629\\
21.1871928691864	-0.480045066494505\\
21.2374176502228	-0.480100174089271\\
21.2877287387848	-0.480189411739724\\
21.3368806362152	-0.480527492876524\\
21.3871657371521	-0.480846261763009\\
21.4381284236908	-0.481045536501908\\
21.4872109413147	-0.481135572758797\\
21.5376049995422	-0.481108184680346\\
21.5874728679657	-0.481010325383536\\
21.6380285739899	-0.481122889136728\\
21.6870588779449	-0.481249679529764\\
21.739088010788	-0.481545098459003\\
21.7871963500977	-0.481782617554918\\
21.8373071670532	-0.481882033950646\\
21.8888287067413	-0.481900375057689\\
21.9381830215454	-0.481865474972199\\
21.9872478961945	-0.481791859402293\\
22.0378989696503	-0.48193942866968\\
22.0872938156128	-0.482097202703471\\
22.138090801239	-0.482365638001335\\
22.1868211746216	-0.482572838328812\\
22.2377860069275	-0.482656957404119\\
22.2874586105347	-0.482640805098676\\
22.3384885311127	-0.482614094139031\\
22.3870710849762	-0.48256667355319\\
22.4387132644653	-0.482746001654927\\
22.4868266105652	-0.482925772729265\\
22.5373594284058	-0.483158242987454\\
22.5873472213745	-0.483344283939109\\
22.6370512962341	-0.483403925296104\\
22.6884295463562	-0.483390217992987\\
22.7378239154816	-0.483361498603973\\
22.787751865387	-0.48331347492778\\
22.8374096870422	-0.483481264305441\\
22.8873886585236	-0.483668001286365\\
22.9379829883575	-0.483929442494528\\
22.9869961261749	-0.484148315026662\\
23.037150812149	-0.484241122791005\\
23.0874580860138	-0.48424613942132\\
23.1373772144318	-0.48422477065137\\
23.1871411323547	-0.484170061578805\\
23.237290096283	-0.484298057226187\\
23.2868723392487	-0.484439190173976\\
23.3371350288391	-0.484683020863809\\
23.3872100830078	-0.484889228593843\\
23.4371966838837	-0.484963986886392\\
23.4883241176605	-0.484964286509236\\
23.5368735313416	-0.484918885314926\\
23.5883466720581	-0.484826354290561\\
23.6373395442963	-0.484903516680692\\
23.6870235919952	-0.484995349359369\\
23.7381245613098	-0.485236447833358\\
23.7868911743164	-0.4854271807848\\
23.8377303600311	-0.485507744661586\\
23.8871185302734	-0.485485032305913\\
23.9379884719849	-0.485404580723003\\
23.9873661518097	-0.485274733213863\\
24.0375952243805	-0.485367359617374\\
24.0870673179626	-0.485492622433263\\
24.1371151924133	-0.485780593533313\\
24.1869523048401	-0.486058436273343\\
24.2370559692383	-0.48621897671708\\
24.2891807079315	-0.486296032539737\\
24.3380689144135	-0.486245033083641\\
24.3871647834778	-0.486093508477225\\
24.4377190589905	-0.486056844770527\\
24.4868864536285	-0.48604264764316\\
24.5379740715027	-0.486255673706903\\
24.5869614601135	-0.486475960057003\\
24.6376108646393	-0.486666059953333\\
24.6870998859406	-0.486827431272118\\
24.7377688407898	-0.486921188841212\\
24.7873253345489	-0.486997131447808\\
24.8372883319855	-0.487104719890105\\
24.8869270801544	-0.487176738075252\\
24.9384078502655	-0.487247328297659\\
24.9870955467224	-0.487273864134284\\
25.0383190631866	-0.48731812808667\\
25.0877341747284	-0.487338412930598\\
25.1370951652527	-0.487423848552684\\
25.1875030517578	-0.487505242011636\\
25.2380036830902	-0.487644881020952\\
25.2871031284332	-0.487774516353518\\
25.338099193573	-0.487870122175302\\
25.386954498291	-0.487907165267636\\
25.4379999160767	-0.487898083812379\\
25.4874267101288	-0.487844145281716\\
25.5373079299927	-0.487891367119243\\
25.5872232437134	-0.487952273182405\\
25.6382319450378	-0.488112381074529\\
25.6868111610413	-0.488261619384712\\
25.7375020503998	-0.488346387471541\\
25.7869216918945	-0.488375750068357\\
25.8372537612915	-0.488362230164913\\
25.8889283657074	-0.488324109863647\\
25.9383606433868	-0.488404259564\\
25.9875232696533	-0.488509801891176\\
26.0369369506836	-0.488698187358803\\
26.0872263431549	-0.488882584932806\\
26.1380128383636	-0.488976291536105\\
26.1871096611023	-0.489012385518372\\
26.2377979278564	-0.488991136396134\\
26.2872809886932	-0.488967924347936\\
26.3384959220886	-0.489075891120056\\
26.387292098999	-0.489207890794592\\
26.4373895645142	-0.489386941010594\\
26.4868809700012	-0.489541373989862\\
26.53678150177	-0.489597946258353\\
26.5870432376862	-0.489621106750739\\
26.6371018409729	-0.489633318455997\\
26.6869751930237	-0.489640262665723\\
26.7378837585449	-0.489727236124207\\
26.7877766609192	-0.489818078730558\\
26.8378419399261	-0.489925721205182\\
26.8871218681335	-0.490024059243861\\
26.9376534938812	-0.490105447583623\\
26.987083864212	-0.490173975343399\\
27.0375470638275	-0.490227385103214\\
27.0869986534119	-0.490259612494057\\
27.137539100647	-0.49029090802205\\
27.1868445396423	-0.490308706486218\\
27.2373196601868	-0.490357708811121\\
27.286718082428	-0.49040364494557\\
27.3369435787201	-0.490475720537141\\
27.3872541904449	-0.490524547829114\\
27.4369372844696	-0.490543333959842\\
27.4883272171021	-0.490526884933008\\
27.5371679782867	-0.490509509463802\\
27.5876988887787	-0.490469146150829\\
27.6373111724854	-0.490489445154456\\
27.6871180057526	-0.490500745654753\\
27.7367758274078	-0.490538122201272\\
27.7870089530945	-0.490541770033769\\
27.8375215053558	-0.490515118960779\\
27.8871633529663	-0.490445196245098\\
27.940003824234	-0.490403428966339\\
27.9871308326721	-0.490343656036849\\
28.0378810882568	-0.490374172315999\\
28.0873820304871	-0.49039868206798\\
28.1369366168976	-0.49046019475405\\
28.1869079589844	-0.490498065697345\\
28.2368185043335	-0.490514672724057\\
28.2877447128296	-0.490494734965811\\
28.3373779773712	-0.490487716404473\\
28.3870558261871	-0.490464057927774\\
28.4374107837677	-0.490492998691821\\
28.4871124744415	-0.49050954179603\\
28.5371062278748	-0.490554076480787\\
28.5868112564087	-0.4905832669204\\
28.6371838569641	-0.49062128631801\\
28.6869341850281	-0.490650422807548\\
28.7380611419678	-0.490699805769538\\
28.7870323181152	-0.490737559775162\\
28.8377327442169	-0.490782613418041\\
28.8874563694	-0.490810204060082\\
28.9368562221527	-0.490835398336838\\
28.9870872020721	-0.490841636321715\\
29.0385212421417	-0.49087208419072\\
29.0873689174652	-0.490903902652608\\
29.1368615150452	-0.490966886807033\\
29.187535238266	-0.491020823003133\\
29.2372958183289	-0.491059839549088\\
29.2871133804321	-0.49107780373657\\
29.3380169391632	-0.491096591785205\\
29.3868567466736	-0.491117261292583\\
29.4373459339142	-0.491177629318091\\
29.4873344421387	-0.491237420758349\\
29.5384788036346	-0.491290489891181\\
29.5869259357452	-0.491326204832635\\
29.6376266002655	-0.491344485523879\\
29.6871110916138	-0.491360207723812\\
29.7390245914459	-0.491401589636045\\
29.7868015289307	-0.49144340622221\\
29.8379787921906	-0.491493436757459\\
29.8871769428253	-0.491530007906388\\
29.9371747493744	-0.491550365076062\\
29.987563085556	-0.491562583088321\\
30.0383028507233	-0.491595323341116\\
30.0876338005066	-0.491627819502536\\
30.1368801116943	-0.491674991409397\\
30.1873883724213	-0.491713809903096\\
30.2370848178864	-0.491738632530538\\
30.2871500968933	-0.491749127592346\\
30.3385204792023	-0.491764404298703\\
30.386811208725	-0.49177932059558\\
30.4384216785431	-0.491816898026528\\
30.4868461608887	-0.491851930515473\\
30.5371896743774	-0.491887113997244\\
30.5867259025574	-0.491915162827397\\
30.6380993843079	-0.491935692205839\\
30.6876494407654	-0.491945563978632\\
30.7367903709412	-0.491962661468815\\
30.7870404243469	-0.491975470129766\\
30.8369943618774	-0.492003191130745\\
30.8871693134308	-0.492024738881323\\
30.9381038665771	-0.492046479002372\\
30.986993265152	-0.492056246297629\\
31.0379008769989	-0.492063079208222\\
31.0870007991791	-0.492061060477521\\
31.1389050006866	-0.492078022253068\\
31.187396478653	-0.492095103803329\\
31.2372066497803	-0.492121378668511\\
31.2869581699371	-0.492133769658857\\
31.3370429992676	-0.492129539166839\\
31.3868891716003	-0.492108919407848\\
31.4371098995209	-0.492102140427381\\
31.488369178772	-0.492096405878325\\
31.5368852138519	-0.492116684570441\\
31.5877389431	-0.492126938494728\\
31.6370863437653	-0.492130824709491\\
31.6871823787689	-0.492119629873965\\
31.738449048996	-0.492107655670754\\
31.7871157646179	-0.492092167023422\\
31.8383152008057	-0.492104444982537\\
31.8873555183411	-0.492113758041448\\
31.9394099235535	-0.492124047948835\\
31.9871487140656	-0.492116281986334\\
32.0381509780884	-0.492093047851096\\
32.088035774231	-0.492055018910069\\
32.1374802112579	-0.492044423439218\\
32.187110376358	-0.492034922616384\\
32.2387628078461	-0.492053392299642\\
32.2874614715576	-0.492064420346068\\
32.3373488903046	-0.492061832253242\\
32.3874296665192	-0.492040390734317\\
32.4374224662781	-0.492011010504815\\
32.4878580093384	-0.491974354807368\\
32.5376955986023	-0.491967928989588\\
32.5874928951263	-0.491961902212665\\
32.6386348724365	-0.491975147658223\\
32.6873406887054	-0.491977035977197\\
32.7388579368591	-0.491960248354573\\
32.7873272418976	-0.491922992060946\\
32.8374335289001	-0.491885619603142\\
32.8873004436493	-0.49184438287228\\
32.937156867981	-0.491842735752236\\
32.9873573303223	-0.491844142328691\\
33.0380663394928	-0.491857478989637\\
33.0887851238251	-0.491860177229576\\
33.1397532939911	-0.491840886488784\\
33.1896943569183	-0.491802113291079\\
33.2378904342651	-0.491769113073346\\
33.2873873233795	-0.491737744973065\\
33.337309551239	-0.491744406002812\\
33.3870064735413	-0.491750140127078\\
33.4376117706299	-0.491755466717812\\
33.4877395153046	-0.491748276527535\\
33.5375718593597	-0.49172486050891\\
33.5879704475403	-0.491693891827278\\
33.6378068447113	-0.491680643512368\\
33.6874708652496	-0.491667929894888\\
33.7375998020172	-0.4916740942401\\
33.7882034301758	-0.491670567264897\\
33.8377136707306	-0.491650123107238\\
33.8892471313477	-0.491613085740671\\
33.9370285987854	-0.491581975498192\\
33.9873494625092	-0.491552519276222\\
34.0370041847229	-0.491556162947479\\
34.0873434066772	-0.491558381578656\\
34.1362752437592	-0.491558381578656\\
34.1862122535706	-0.491558381578656\\
34.2362687110901	-0.491558381578656\\
34.2865228176117	-0.491558381578656\\
34.3363995075226	-0.491558381578656\\
34.3862075328827	-0.491558381578656\\
34.4361023426056	-0.491558381578656\\
34.4861790657043	-0.491558381578656\\
34.5361194133759	-0.491558381578656\\
34.586292219162	-0.491558381578656\\
34.6364039897919	-0.491558381578656\\
34.6863061904907	-0.491558381578656\\
34.7361094474793	-0.491558381578656\\
34.7864546298981	-0.491558381578656\\
34.83685297966	-0.491558381578656\\
34.8863391399384	-0.491558381578656\\
34.9360925674439	-0.491558381578656\\
34.986105632782	-0.491558381578656\\
35.0454177379608	-0.491558381578656\\
35.0912179470062	-0.491558381578656\\
35.136381816864	-0.491558381578656\\
35.186456155777	-0.491558381578656\\
35.2363311767578	-0.491558381578656\\
35.2861754417419	-0.491558381578656\\
35.3360485553741	-0.491558381578656\\
35.3861937046051	-0.491558381578656\\
35.4363774776459	-0.491558381578656\\
35.486319732666	-0.491558381578656\\
35.536076259613	-0.491558381578656\\
35.5862481117249	-0.491558381578656\\
35.6363558292389	-0.491558381578656\\
35.6861931800842	-0.491558381578656\\
35.7363302230835	-0.491558381578656\\
35.786288690567	-0.491558381578656\\
35.8361827850342	-0.491558381578656\\
35.8864717006683	-0.491558381578656\\
35.9363789081574	-0.491558381578656\\
35.9867715358734	-0.491558381578656\\
36.0360667228699	-0.491558381578656\\
36.0863480091095	-0.491558381578656\\
36.1363653659821	-0.491558381578656\\
36.1864337444305	-0.491558381578656\\
36.2360584259033	-0.491558381578656\\
36.2864779949188	-0.491558381578656\\
36.336106967926	-0.491558381578656\\
36.3864702701569	-0.491558381578656\\
36.4362844944	-0.491558381578656\\
36.4863364219666	-0.491558381578656\\
36.5363025188446	-0.491558381578656\\
36.5861410617828	-0.491558381578656\\
36.6363481998444	-0.491558381578656\\
36.6863061904907	-0.491558381578656\\
36.7361874103546	-0.491558381578656\\
36.7863311290741	-0.491558381578656\\
36.8363565921783	-0.491558381578656\\
36.8863608360291	-0.491558381578656\\
36.9363264560699	-0.491558381578656\\
36.9860896587372	-0.491558381578656\\
37.0362286090851	-0.491558381578656\\
37.0864164352417	-0.491558381578656\\
37.1363679885864	-0.491558381578656\\
37.1863879680634	-0.491558381578656\\
37.2362648963928	-0.491558381578656\\
37.2863585472107	-0.491558381578656\\
37.3360833644867	-0.491558381578656\\
37.3862823963165	-0.491558381578656\\
37.4368352413178	-0.491558381578656\\
37.4861456871033	-0.491558381578656\\
37.5361048698425	-0.491558381578656\\
37.5862543106079	-0.491558381578656\\
37.6362061023712	-0.491558381578656\\
37.6863986968994	-0.491558381578656\\
37.736749124527	-0.491558381578656\\
37.7863394737244	-0.491558381578656\\
37.8361865997314	-0.491558381578656\\
37.888374042511	-0.491558381578656\\
37.9363502979279	-0.491558381578656\\
37.9863593101501	-0.491558381578656\\
38.0362567424774	-0.491558381578656\\
38.0862390518189	-0.491558381578656\\
38.1364056587219	-0.491558381578656\\
38.1864752292633	-0.491558381578656\\
38.2363790988922	-0.491558381578656\\
38.2863652229309	-0.491558381578656\\
38.3364450454712	-0.491558381578656\\
38.3863930225372	-0.491558381578656\\
38.4377541065216	-0.491558381578656\\
38.4863030433655	-0.491558381578656\\
38.5364169597626	-0.491558381578656\\
38.5863399028778	-0.491558381578656\\
38.6360737800598	-0.491558381578656\\
38.686456155777	-0.491558381578656\\
38.7360977649689	-0.491558381578656\\
38.7863506793976	-0.491558381578656\\
38.8363575458527	-0.491558381578656\\
38.8868772506714	-0.491558381578656\\
38.9363157272339	-0.491558381578656\\
38.9863695621491	-0.491558381578656\\
39.036302280426	-0.491558381578656\\
39.0863775730133	-0.491558381578656\\
39.1364001750946	-0.491558381578656\\
39.1863421916962	-0.491558381578656\\
39.2363414287567	-0.491558381578656\\
39.2865790843964	-0.491558381578656\\
39.3364316940308	-0.491558381578656\\
39.3863877773285	-0.491558381578656\\
39.4363846302032	-0.491558381578656\\
39.4864034175873	-0.491558381578656\\
39.5362460136414	-0.491558381578656\\
39.5864672183991	-0.491558381578656\\
39.6376175403595	-0.491558381578656\\
39.6865121841431	-0.491558381578656\\
39.7360577106476	-0.491558381578656\\
39.7860879421234	-0.491558381578656\\
39.8361377239227	-0.491558381578656\\
39.8862397193909	-0.491558381578656\\
39.9363758087158	-0.491558381578656\\
39.9866580486298	-0.491558381578656\\
40.0361229896545	-0.491558381578656\\
40.0863337039948	-0.491558381578656\\
40.1362952709198	-0.491558381578656\\
40.1863722324371	-0.491558381578656\\
40.2360808372498	-0.491558381578656\\
40.2867063999176	-0.491558381578656\\
40.3362371444702	-0.491558381578656\\
40.3864602565765	-0.491558381578656\\
40.4362022399902	-0.491558381578656\\
40.4862391471863	-0.491558381578656\\
40.536110830307	-0.491558381578656\\
40.586754989624	-0.491558381578656\\
40.6363067150116	-0.491558381578656\\
40.6862320423126	-0.491558381578656\\
40.7363345146179	-0.491558381578656\\
40.7863447189331	-0.491558381578656\\
40.8362771987915	-0.491558381578656\\
40.886261177063	-0.491558381578656\\
40.936424446106	-0.491558381578656\\
40.9864334583282	-0.491558381578656\\
41.0363933563232	-0.491558381578656\\
41.0863849639893	-0.491558381578656\\
41.1362373352051	-0.491558381578656\\
41.1860804080963	-0.491558381578656\\
41.2361869335175	-0.491558381578656\\
41.2864078998566	-0.491558381578656\\
41.3360969543457	-0.491558381578656\\
41.3861855983734	-0.491558381578656\\
41.4376759052277	-0.491558381578656\\
41.486256313324	-0.491558381578656\\
41.536371421814	-0.491558381578656\\
41.5862829208374	-0.491558381578656\\
41.636062335968	-0.491558381578656\\
41.686300945282	-0.491558381578656\\
41.7360789299011	-0.491558381578656\\
41.7862710475922	-0.491558381578656\\
41.8360845565796	-0.491558381578656\\
41.8863353252411	-0.491558381578656\\
41.9362205982208	-0.491558381578656\\
41.986283493042	-0.491558381578656\\
42.0364117145538	-0.491558381578656\\
42.0863775730133	-0.491558381578656\\
42.1363970756531	-0.491558381578656\\
42.186447095871	-0.491558381578656\\
42.2362341403961	-0.491558381578656\\
42.2862476825714	-0.491558381578656\\
42.3368155479431	-0.491558381578656\\
42.3862549781799	-0.491558381578656\\
42.4363796234131	-0.491558381578656\\
42.4863807678223	-0.491558381578656\\
42.5361065387726	-0.491558381578656\\
42.586087179184	-0.491558381578656\\
42.6363203048706	-0.491558381578656\\
42.6863209724426	-0.491558381578656\\
42.7362560749054	-0.491558381578656\\
42.7863132476807	-0.491558381578656\\
42.8362674236298	-0.491558381578656\\
42.8862435340881	-0.491558381578656\\
42.9362756729126	-0.491558381578656\\
42.9860562801361	-0.491558381578656\\
43.0366630077362	-0.491558381578656\\
43.0862891197205	-0.491558381578656\\
43.1369411468506	-0.491558381578656\\
43.1862763881683	-0.491558381578656\\
43.2363652706146	-0.491558381578656\\
43.2866267681122	-0.491558381578656\\
43.3361598968506	-0.491558381578656\\
43.3862552165985	-0.491558381578656\\
43.4363893985748	-0.491558381578656\\
43.4867214679718	-0.491558381578656\\
43.5363444805145	-0.491558381578656\\
43.5866703510284	-0.491558381578656\\
43.6362726211548	-0.491558381578656\\
43.6863271713257	-0.491558381578656\\
43.7361363887787	-0.491558381578656\\
43.7873098373413	-0.491558381578656\\
43.8363158226013	-0.491558381578656\\
43.8864092350006	-0.491558381578656\\
43.9364137172699	-0.491558381578656\\
43.9862827777863	-0.491558381578656\\
44.0369226455689	-0.491558381578656\\
44.0863043785095	-0.491558381578656\\
44.1360961914063	-0.491558381578656\\
44.1862127304077	-0.491558381578656\\
44.236283493042	-0.491558381578656\\
44.2865638256073	-0.491558381578656\\
44.3360812187195	-0.491558381578656\\
44.386487197876	-0.491558381578656\\
44.4361028194428	-0.491558381578656\\
44.4864933013916	-0.491558381578656\\
44.536284160614	-0.491558381578656\\
44.5863046169281	-0.491558381578656\\
44.6362940788269	-0.491558381578656\\
44.6862978458405	-0.491558381578656\\
44.7364215373993	-0.491558381578656\\
44.7863480567932	-0.491558381578656\\
44.8362671852112	-0.491558381578656\\
44.8862499713898	-0.491558381578656\\
44.9360589504242	-0.491558381578656\\
44.9889430522919	-0.491558381578656\\
45.0363490104675	-0.491558381578656\\
45.0864402770996	-0.491558381578656\\
45.1361271858215	-0.491558381578656\\
45.1861283302307	-0.491558381578656\\
45.2363178253174	-0.491558381578656\\
45.2860579013824	-0.491558381578656\\
45.3360833644867	-0.491558381578656\\
45.3861248016357	-0.491558381578656\\
45.4362778186798	-0.491558381578656\\
45.4862968444824	-0.491558381578656\\
45.5364520072937	-0.491558381578656\\
45.5863546848297	-0.491558381578656\\
45.6363763332367	-0.491558381578656\\
45.6862480163574	-0.491558381578656\\
45.7361382961273	-0.491558381578656\\
45.7863018035889	-0.491558381578656\\
45.8363010406494	-0.491558381578656\\
45.8863670349121	-0.491558381578656\\
45.9360751628876	-0.491558381578656\\
45.9862565517426	-0.491558381578656\\
46.036564540863	-0.491558381578656\\
46.086417388916	-0.491558381578656\\
46.136324596405	-0.491558381578656\\
46.1863011837006	-0.491558381578656\\
46.236340713501	-0.491558381578656\\
46.286172580719	-0.491558381578656\\
46.3363940238953	-0.491558381578656\\
46.3862518787384	-0.491558381578656\\
46.4363107204437	-0.491558381578656\\
46.4862446308136	-0.491558381578656\\
46.5362638950348	-0.491558381578656\\
46.5863682746887	-0.491558381578656\\
46.6360892772675	-0.491558381578656\\
46.6863905906677	-0.491558381578656\\
46.7362088680267	-0.491558381578656\\
46.7864021778107	-0.491558381578656\\
46.8362369060516	-0.491558381578656\\
46.8863255500793	-0.491558381578656\\
46.9363812923431	-0.491558381578656\\
46.9860479354858	-0.491558381578656\\
47.0364589214325	-0.491558381578656\\
47.086301279068	-0.491558381578656\\
47.1364011287689	-0.491558381578656\\
47.1863252639771	-0.491558381578656\\
47.2360765457153	-0.491558381578656\\
47.2863361358643	-0.491558381578656\\
47.336061668396	-0.491558381578656\\
47.3863124370575	-0.491558381578656\\
47.436296415329	-0.491558381578656\\
47.4863233089447	-0.491558381578656\\
47.5360545635223	-0.491558381578656\\
47.5861608505249	-0.491558381578656\\
47.6362637996674	-0.491558381578656\\
47.6860692024231	-0.491558381578656\\
47.7363540649414	-0.491558381578656\\
47.7863378047943	-0.491558381578656\\
47.8363644599915	-0.491558381578656\\
47.8862695217133	-0.491558381578656\\
47.9362542152405	-0.491558381578656\\
47.9863764762878	-0.491558381578656\\
48.0360953330994	-0.491558381578656\\
48.0863441944122	-0.491558381578656\\
48.1360756874085	-0.491558381578656\\
48.186307144165	-0.491558381578656\\
48.2360848903656	-0.491558381578656\\
48.2862736701965	-0.491558381578656\\
48.3363034248352	-0.491558381578656\\
48.3863589286804	-0.491558381578656\\
48.4373104095459	-0.491558381578656\\
48.4862935066223	-0.491558381578656\\
48.5363015651703	-0.491558381578656\\
48.5862574100494	-0.491558381578656\\
48.6363148212433	-0.491558381578656\\
48.6862716197968	-0.491558381578656\\
48.7360643863678	-0.491558381578656\\
48.7863087177277	-0.491558381578656\\
48.8364052295685	-0.491558381578656\\
48.8860489845276	-0.491558381578656\\
48.9366774082184	-0.491558381578656\\
48.9863929271698	-0.491558381578656\\
49.0368101119995	-0.491558381578656\\
49.0863396644592	-0.491558381578656\\
49.1362618923187	-0.491558381578656\\
49.1863467216492	-0.491558381578656\\
49.2363132953644	-0.491558381578656\\
49.2862998962402	-0.491558381578656\\
49.3360840797424	-0.491558381578656\\
49.3862306594849	-0.491558381578656\\
49.4362706661224	-0.491558381578656\\
49.486224603653	-0.491558381578656\\
49.5363234996796	-0.491558381578656\\
49.5862879276276	-0.491558381578656\\
49.6361152648926	-0.491558381578656\\
49.6863047599793	-0.491558381578656\\
49.7360612869263	-0.491558381578656\\
49.7862214565277	-0.491558381578656\\
49.8363184452057	-0.491558381578656\\
49.8879334449768	-0.491558381578656\\
49.9363319396973	-0.491558381578656\\
49.9862370014191	-0.491558381578656\\
50.0362956047058	-0.491558381578656\\
50.0868522644043	-0.491558381578656\\
50.1361271858215	-0.491558381578656\\
50.1862706661224	-0.491558381578656\\
50.2362811088562	-0.491558381578656\\
50.2863530635834	-0.491558381578656\\
50.3364486217499	-0.491558381578656\\
50.3862692832947	-0.491558381578656\\
50.4361993789673	-0.491558381578656\\
50.4862770557404	-0.491558381578656\\
50.5360641002655	-0.491558381578656\\
50.5860704898834	-0.491558381578656\\
50.636267375946	-0.491558381578656\\
50.6864208698273	-0.491558381578656\\
50.7364587306976	-0.491558381578656\\
50.786306810379	-0.491558381578656\\
50.8363444328308	-0.491558381578656\\
50.8863732337952	-0.491558381578656\\
50.9360637187958	-0.491558381578656\\
50.9860643863678	-0.491558381578656\\
51.0364918231964	-0.491558381578656\\
51.086265039444	-0.491558381578656\\
51.1363150596619	-0.491558381578656\\
51.1863760471344	-0.491558381578656\\
51.2363709926605	-0.491558381578656\\
51.2863130092621	-0.491558381578656\\
51.3363859176636	-0.491558381578656\\
51.3862800121307	-0.491558381578656\\
51.4363278865814	-0.491558381578656\\
51.4863607406616	-0.491558381578656\\
51.5360641002655	-0.491558381578656\\
51.5886544704437	-0.491558381578656\\
51.6360978603363	-0.491558381578656\\
51.688673210144	-0.491558381578656\\
51.7367107391357	-0.491558381578656\\
51.7860705375671	-0.491558381578656\\
51.8363744735718	-0.491558381578656\\
51.8862721443176	-0.491558381578656\\
51.936353635788	-0.491558381578656\\
51.9860617637634	-0.491558381578656\\
52.0368484973907	-0.491558381578656\\
52.0862779140472	-0.491558381578656\\
52.1360904693604	-0.491558381578656\\
52.1863877296448	-0.491558381578656\\
52.2361249446869	-0.491558381578656\\
52.2863645076752	-0.491558381578656\\
52.3361343860626	-0.491558381578656\\
52.3863236427307	-0.491558381578656\\
52.4363097667694	-0.491558381578656\\
52.4860250473022	-0.491558381578656\\
52.5360769748688	-0.491558381578656\\
52.5860502243042	-0.491558381578656\\
52.6363713264465	-0.491558381578656\\
52.690006685257	-0.491558381578656\\
52.7364186763763	-0.491558381578656\\
52.7863256454468	-0.491558381578656\\
52.8360683441162	-0.491558381578656\\
52.8861855983734	-0.491558381578656\\
52.9360858917236	-0.491558381578656\\
52.9860407829285	-0.491558381578656\\
53.0362543582916	-0.491558381578656\\
53.0863141536713	-0.491558381578656\\
53.1362888336182	-0.491558381578656\\
53.1872043132782	-0.491558381578656\\
53.2363035202026	-0.491558381578656\\
53.2860457420349	-0.491558381578656\\
53.3363325119019	-0.491558381578656\\
53.3861095428467	-0.491558381578656\\
53.436364364624	-0.491558381578656\\
53.4863154411316	-0.491558381578656\\
53.5363397121429	-0.491558381578656\\
53.5863167762756	-0.491558381578656\\
53.6361658096314	-0.491558381578656\\
53.6863488674164	-0.491558381578656\\
53.7363729000092	-0.491558381578656\\
53.7860550403595	-0.491558381578656\\
53.836292219162	-0.491558381578656\\
53.8862983703613	-0.491558381578656\\
53.9362475395203	-0.491558381578656\\
53.9861213684082	-0.491558381578656\\
54.0362629413605	-0.491558381578656\\
54.0862345218658	-0.491558381578656\\
54.1360654354095	-0.491558381578656\\
54.1862487316132	-0.491558381578656\\
54.2360968112946	-0.491558381578656\\
54.2862512588501	-0.491558381578656\\
54.3363804340363	-0.491558381578656\\
54.3863088607788	-0.491558381578656\\
54.4362613677979	-0.491558381578656\\
54.4864789962769	-0.491558381578656\\
54.5363761901855	-0.491558381578656\\
54.5860692977905	-0.491558381578656\\
54.6361002445221	-0.491558381578656\\
54.6862933158875	-0.491558381578656\\
54.7361409187317	-0.491558381578656\\
54.7861282348633	-0.491558381578656\\
54.8360788345337	-0.491558381578656\\
54.8863553524017	-0.491558381578656\\
54.9360541820526	-0.491558381578656\\
54.9860584259033	-0.491558381578656\\
55.0363721370697	-0.491558381578656\\
55.0868153095245	-0.491558381578656\\
55.136411857605	-0.491558381578656\\
55.1862184524536	-0.491558381578656\\
55.2393674373627	-0.491558381578656\\
55.2863261222839	-0.491558381578656\\
55.33689661026	-0.491558381578656\\
55.3863090991974	-0.491558381578656\\
55.4367873191834	-0.491558381578656\\
55.4863738536835	-0.491558381578656\\
55.5360693454742	-0.491558381578656\\
55.586337518692	-0.491558381578656\\
55.6370651245117	-0.491558381578656\\
55.6863228797913	-0.491558381578656\\
55.7360960960388	-0.491558381578656\\
55.7863347053528	-0.491558381578656\\
};
\addlegendentry{$\theta_{y_{p}^{2}}$}

\addplot [color=mycolor3, line width=1.4pt]
  table[row sep=crcr]{%
-3.41395025253296	0\\
-3.36380581855774	0\\
-3.31373219490051	0\\
-3.26373772621155	0\\
-3.21386485099792	0\\
-3.16371612548828	0\\
-3.11396126747131	0\\
-3.0639102935791	0\\
-3.01393465995789	0\\
-2.96376042366028	0\\
-2.91368894577026	0\\
-2.86386590003967	0\\
-2.81392412185669	0\\
-2.76383023262024	0\\
-2.71365766525269	0\\
-2.66386275291443	0\\
-2.61371665000916	0\\
-2.56373028755188	0\\
-2.51375107765198	0\\
-2.46365218162537	0\\
-2.41376595497131	0\\
-2.36374454498291	0\\
-2.31372838020325	0\\
-2.26391820907593	0\\
-2.21392397880554	0\\
-2.16394262313843	0\\
-2.11378173828125	0\\
-2.06323270797729	0\\
-2.01386647224426	0\\
-1.96378617286682	0\\
-1.91375093460083	0\\
-1.86376647949219	0\\
-1.81371264457703	0\\
-1.76388936042786	0\\
-1.71374897956848	0\\
-1.66370325088501	0\\
-1.61389498710632	0\\
-1.56357769966125	0\\
-1.51373534202576	0\\
-1.46395378112793	0\\
-1.41391592025757	0\\
-1.36373047828674	0\\
-1.31355028152466	0\\
-1.26392798423767	0\\
-1.21389083862305	0\\
-1.1639223575592	0\\
-1.11384873390198	0\\
-1.06395726203918	0\\
-1.01393632888794	0\\
-0.96373085975647	0\\
-0.913796234130859	0\\
-0.863754081726074	0\\
-0.81366925239563	0\\
-0.763712215423584	0\\
-0.71373085975647	0\\
-0.663900899887085	0\\
-0.61391978263855	0\\
-0.563617038726806	0\\
-0.51369457244873	0\\
-0.46377067565918	0\\
-0.413990783691406	0\\
-0.363846826553345	0\\
-0.313869285583496	0\\
-0.263829755783081	0\\
-0.213733005523681	0\\
-0.163687038421631	0\\
-0.113701391220093	0\\
-0.0636923789978026	0\\
-0.0136535644531248	0\\
0.0374466896057131	-1.66449057945178\\
0.0868460655212404	-2.29529939530613\\
0.137080144882202	-2.62939126479137\\
0.187013101577759	-2.8255618381861\\
0.237029504776001	-2.95709865994183\\
0.288115215301514	-3.05415776542225\\
0.336934280395508	-3.13275484889334\\
0.387251806259155	-3.19737686664143\\
0.436960172653198	-3.25072360957324\\
0.487322521209717	-3.29296792621244\\
0.537163209915161	-3.32580386561835\\
0.587262582778931	-3.35044015698259\\
0.638643217086792	-3.36966691934663\\
0.687006664276123	-3.38636716448605\\
0.737519216537476	-3.40131081330151\\
0.787285280227661	-3.41507502005834\\
0.838375759124756	-3.4291701282109\\
0.887129497528076	-3.44101683673125\\
0.937354993820191	-3.45156165384014\\
0.98739595413208	-3.45902763773802\\
1.03824419975281	-3.46498159054136\\
1.08832807540894	-3.47009607853033\\
1.13733215332031	-3.47560543227883\\
1.18734998703003	-3.48011240880669\\
1.23741598129272	-3.48343186679085\\
1.28700251579285	-3.48542812018968\\
1.33715624809265	-3.48720912407589\\
1.3871440410614	-3.49044789730488\\
1.43721837997437	-3.49424480824291\\
1.48703689575195	-3.49833563410311\\
1.53733153343201	-3.50225873443424\\
1.58698172569275	-3.50321906834233\\
1.6369044303894	-3.50124083636365\\
1.68689270019531	-3.49699261217756\\
1.73750514984131	-3.49301803871458\\
1.78736562728882	-3.49100384437406\\
1.83805890083313	-3.4910597316366\\
1.8878821849823	-3.49112811709529\\
1.93722839355469	-3.49068564853133\\
1.9872805595398	-3.4867339850407\\
2.03788895606995	-3.48042249331047\\
2.08703417778015	-3.4707600552083\\
2.13730902671814	-3.46071211028811\\
2.18721742630005	-3.45300909349953\\
2.2393238067627	-3.44737170805092\\
2.28719348907471	-3.44230819266068\\
2.33826942443848	-3.43473490898123\\
2.38713521957397	-3.42322600085481\\
2.43748707771301	-3.40806198875907\\
2.487189245224	-3.39023356922644\\
2.53701467514038	-3.37520886145921\\
2.58730955123901	-3.36320282479028\\
2.6372718334198	-3.35401199777789\\
2.68767471313477	-3.34530093300873\\
2.73776693344116	-3.33375116404477\\
2.78724522590637	-3.31820951399914\\
2.83687014579773	-3.29723855598877\\
2.88723630905151	-3.27310985135318\\
2.93726320266724	-3.25086945886505\\
2.98756213188171	-3.2319428447845\\
3.03776831626892	-3.21632544277236\\
3.08719940185547	-3.20124019737978\\
3.13819379806519	-3.18454224819925\\
3.1871039390564	-3.16396870646167\\
3.2383677482605	-3.13901514873578\\
3.28719182014465	-3.11207875358105\\
3.33709425926209	-3.08626793215444\\
3.3875627040863	-3.06362961455125\\
3.43703336715698	-3.04351409051378\\
3.48778195381165	-3.02475871293427\\
3.53756542205811	-3.00485112350634\\
3.58713026046753	-2.98171308921883\\
3.63820476531982	-2.95445830082917\\
3.68715305328369	-2.92447719567463\\
3.7370192527771	-2.8961320455719\\
3.78696150779724	-2.87192501817481\\
3.83697123527527	-2.85113559660294\\
3.88720889091492	-2.83159897445603\\
3.93718166351318	-2.81202845117605\\
3.98680086135864	-2.79040324343896\\
4.0373019695282	-2.76611922293432\\
4.087202501297	-2.73939960945791\\
4.13750429153442	-2.71285149503819\\
4.18730540275574	-2.68906141402204\\
4.23692674636841	-2.66750893253356\\
4.28764815330505	-2.64821591692271\\
4.33715195655823	-2.62867333750364\\
4.38732810020447	-2.60691537982802\\
4.43775720596313	-2.58244568763439\\
4.48715395927429	-2.55536627515903\\
4.53803582191467	-2.52871847501137\\
4.58724708557129	-2.50477053693066\\
4.63719172477722	-2.48342253949249\\
4.68718476295471	-2.46462226452422\\
4.73849172592163	-2.44556771391763\\
4.78692073822022	-2.42423898498782\\
4.83752150535584	-2.40000835814953\\
4.88743896484375	-2.37309428425306\\
4.93707461357117	-2.34630570578065\\
4.98722429275513	-2.32240372203705\\
5.03730530738831	-2.30112938856928\\
5.08796877861023	-2.28234508827609\\
5.13730926513672	-2.26328191394532\\
5.1875111579895	-2.24225656704584\\
5.2383083820343	-2.21890730915311\\
5.28709311485291	-2.19351964848283\\
5.33733768463135	-2.16805347435547\\
5.38735957145691	-2.14516413410001\\
5.43840856552124	-2.124733290113\\
5.4869279384613	-2.10628950656883\\
5.53821368217468	-2.08746281079675\\
5.58721609115601	-2.06751986840936\\
5.63834400177002	-2.04557908936658\\
5.70256943702698	-2.02180776168007\\
5.73696870803833	-1.9979226373207\\
5.78678030967712	-1.97608689354774\\
5.83695526123047	-1.95657166870751\\
5.88792462348938	-1.93887232926409\\
5.93726081848145	-1.92129219288745\\
5.98705382347107	-1.90247350304162\\
6.037406873703	-1.88188486256877\\
6.08693976402283	-1.85971597967136\\
6.13725109100342	-1.83761405760924\\
6.1868757724762	-1.81753355247702\\
6.23866934776306	-1.79964145788063\\
6.28684301376343	-1.78346094205699\\
6.3376051902771	-1.76727357961227\\
6.38692278862	-1.75000877968978\\
6.43746538162231	-1.73107764714132\\
6.48725814819336	-1.71081244069637\\
6.53829855918884	-1.69082376866891\\
6.58715434074402	-1.67239145168924\\
6.63686246871948	-1.65594043511919\\
6.68743295669556	-1.64089729030457\\
6.73686356544495	-1.62608670851569\\
6.78764243125916	-1.61055063427284\\
6.83729786872864	-1.59372067580989\\
6.8867949962616	-1.57574662059278\\
6.93705792427063	-1.55738018483771\\
6.98714203834534	-1.54035032097772\\
7.03722019195557	-1.52509918370151\\
7.0867495059967	-1.5110937265631\\
7.13819832801819	-1.49740691126044\\
7.18691987991333	-1.48320978214997\\
7.23969264030457	-1.46790190864795\\
7.28743596076965	-1.45167346924018\\
7.33690996170044	-1.43530821317472\\
7.38727087974548	-1.41992848187601\\
7.43710012435913	-1.4061840036411\\
7.48817820549011	-1.39357838186106\\
7.5378050327301	-1.38144996324581\\
7.58700842857361	-1.36893654711321\\
7.63773460388184	-1.35543023630817\\
7.68709034919739	-1.34101958450947\\
7.7376603603363	-1.32628901100361\\
7.78713698387146	-1.31243393153727\\
7.83752889633179	-1.30015247014853\\
7.88738055229187	-1.28888798660228\\
7.93729419708252	-1.27821354347657\\
7.98730368614197	-1.2672389060665\\
8.03704233169556	-1.2552915652268\\
8.08665652275085	-1.24246744863194\\
8.13837356567383	-1.22930383069547\\
8.18696184158325	-1.21689270151114\\
8.23794646263122	-1.20598357760969\\
8.28800339698791	-1.19605611946099\\
8.33725209236145	-1.18651594725679\\
8.38739914894104	-1.17667511433638\\
8.43756384849548	-1.16597818256992\\
8.48700923919678	-1.15458701044531\\
8.53716988563538	-1.14315964985713\\
8.5874659538269	-1.13240055754613\\
8.63724226951599	-1.12305519031474\\
8.68716926574707	-1.11446179457016\\
8.73704977035522	-1.10617304761354\\
8.78744978904724	-1.0977144447404\\
8.83704800605774	-1.08814350302578\\
8.88674945831299	-1.07781430604382\\
8.93740029335022	-1.06753438039823\\
8.98759908676147	-1.05794169143951\\
9.03756065368652	-1.04969640356467\\
9.08785433769226	-1.04208699739979\\
9.13731617927551	-1.03473955322829\\
9.18759603500366	-1.0270575012496\\
9.23747487068176	-1.01857833271265\\
9.28715224266052	-1.00953706390328\\
9.33708448410034	-1.00058098108298\\
9.38704843521118	-0.992213129217816\\
9.43731732368469	-0.985161005999998\\
9.48708744049072	-0.978688119791059\\
9.5378701210022	-0.972332800500681\\
9.58680291175842	-0.965672513798836\\
9.63790078163147	-0.958298573156981\\
9.68698759078979	-0.950453038675732\\
9.73723216056824	-0.942806166478363\\
9.78691930770874	-0.935612408378006\\
9.83674449920654	-0.929541562869986\\
9.8874725818634	-0.923903766483818\\
9.93801159858704	-0.918346094914455\\
9.98698897361755	-0.91251647018521\\
10.0372972011566	-0.906048123753749\\
10.0883295059204	-0.899043752865964\\
10.1372539520264	-0.892198620045747\\
10.1879517555237	-0.885819489734558\\
10.2371825695038	-0.880625833428439\\
10.287105512619	-0.875868606391009\\
10.3395609378815	-0.871285656856458\\
10.387139749527	-0.866436794265155\\
10.4386827468872	-0.860817742103109\\
10.4871279716492	-0.854648496246796\\
10.537340593338	-0.848717896446828\\
10.5872203826904	-0.843164462107893\\
10.6368417263031	-0.8387929105578\\
10.6872877597809	-0.834794839346273\\
10.7376646518707	-0.830775992229519\\
10.7870134830475	-0.826495252721656\\
10.8377515792847	-0.821504894328655\\
10.8868004798889	-0.816006262863311\\
10.9370796203613	-0.810866385729469\\
10.9869966030121	-0.806091787092669\\
11.0377678394318	-0.802375125857566\\
11.0870306015015	-0.798990432849337\\
11.1382483959198	-0.795561641864424\\
11.1873485565186	-0.791997589759035\\
11.2382921695709	-0.787834985754898\\
11.2869455337524	-0.783263726328641\\
11.337321472168	-0.778836429522471\\
11.3868953704834	-0.774682786706762\\
11.4372324466705	-0.771441954400188\\
11.4881226539612	-0.768506729936007\\
11.5377847671509	-0.765591495539582\\
11.5873648643494	-0.762444697784247\\
11.6380569458008	-0.758691333236186\\
11.6869468212128	-0.754543238824567\\
11.7386147499084	-0.750470471987398\\
11.7868301391602	-0.746599829263459\\
11.8376504898071	-0.74363012785733\\
11.8868886947632	-0.740906694570072\\
11.9389214038849	-0.738264298728438\\
11.9870438098907	-0.735397650618552\\
12.0372139930725	-0.731937276606857\\
12.0867568969727	-0.7280672753825\\
12.1371871948242	-0.724282211340778\\
12.1868307113647	-0.720685159129289\\
12.2372700691223	-0.718064864435405\\
12.2880680084229	-0.715686842095636\\
12.3380994319916	-0.71342362150159\\
12.3877601146698	-0.710806651891289\\
12.4371673583984	-0.70762104759234\\
12.4874753475189	-0.704010052540525\\
12.5367342948914	-0.700511869591935\\
12.5872320652008	-0.697218109192761\\
12.6379358291626	-0.694994969955587\\
12.6869699478149	-0.693032345483005\\
12.737691116333	-0.691139735804029\\
12.7872032642364	-0.688999806808965\\
12.837261390686	-0.686202569113959\\
12.8871109008789	-0.683059704529569\\
12.9367858886719	-0.680111880268726\\
12.9869508266449	-0.677328315147463\\
13.03677983284	-0.675392519867614\\
13.0889579772949	-0.673626010101771\\
13.1373173713684	-0.671910472828756\\
13.1868133068085	-0.669972919698637\\
13.2383656024933	-0.667513505037391\\
13.2870273113251	-0.664723284992419\\
13.3382465362549	-0.662048786677758\\
13.3877489089966	-0.659503838626165\\
13.4376754283905	-0.657722234463307\\
13.4869431972504	-0.656114940179862\\
13.5382995128632	-0.654721064096236\\
13.5876717090607	-0.653168469765575\\
13.6373288154602	-0.651121084537635\\
13.6871709346771	-0.648763142326231\\
13.7372221469879	-0.646376478938294\\
13.7870430469513	-0.643996273489051\\
13.8368041038513	-0.642383601079644\\
13.8874387264252	-0.640909539153114\\
13.9370603084564	-0.639714565571296\\
13.9871797084808	-0.638343353550908\\
14.0383731842041	-0.636541694622764\\
14.0870515823364	-0.63442348871456\\
14.1367220401764	-0.632248648583982\\
14.1868392944336	-0.630078938293735\\
14.2372829437256	-0.62864744895225\\
14.2870769023895	-0.62735366575194\\
14.3382911205292	-0.626334843832979\\
14.3871123313904	-0.625152751938174\\
14.4397968769073	-0.623581641672132\\
14.4871720790863	-0.621711875861649\\
14.5370795249939	-0.619725707609092\\
14.587115240097	-0.61769236430402\\
14.6373021125793	-0.616293914162213\\
14.6878222942352	-0.615030702219201\\
14.7388114452362	-0.614184663457792\\
14.7875110626221	-0.613246571238705\\
14.8372318267822	-0.611995877056614\\
14.8873505115509	-0.610469626558881\\
14.9386431694031	-0.608747570160915\\
14.9871870994568	-0.606875976529068\\
15.0373312950134	-0.605574246233346\\
15.0867146968842	-0.604395728177735\\
15.1386305809021	-0.603675079386932\\
15.1868702888489	-0.60287497953172\\
15.2371518135071	-0.601790353553909\\
15.2869948863983	-0.600443946193593\\
15.3384689807892	-0.598895069912189\\
15.3868526935577	-0.597172791182786\\
15.4374238967896	-0.595985969190821\\
15.4874579429626	-0.594928001972278\\
15.5373572826385	-0.594360185188421\\
15.5874838352203	-0.593749435388986\\
15.6373798370361	-0.592845384071296\\
15.6868905544281	-0.591662409692958\\
15.737205696106	-0.590229365070158\\
15.7869731903076	-0.588650649117611\\
15.8384644508362	-0.587520254312835\\
15.887641620636	-0.586502398515051\\
15.938517999649	-0.585997710412414\\
15.9870197296143	-0.585473959885817\\
16.0382165431976	-0.584723109976423\\
16.087006521225	-0.583708855234264\\
16.1369261264801	-0.5824218310741\\
16.187291097641	-0.580903783921016\\
16.2368158817291	-0.579816069508212\\
16.2877594947815	-0.578839765296848\\
16.3371722221375	-0.578408697910632\\
16.3871049404144	-0.577969014106642\\
16.4373132705688	-0.577276165225797\\
16.4878031730652	-0.576298819891761\\
16.5373649120331	-0.575009863065901\\
16.5867781162262	-0.573484758946478\\
16.6383754730225	-0.57250464374529\\
16.6873669147491	-0.571684266598396\\
16.7393970012665	-0.571435038900972\\
16.7873897075653	-0.571165442791219\\
16.8394999027252	-0.570647564342515\\
16.8871914863586	-0.569887859289707\\
16.9372188568115	-0.568795826711749\\
16.9868101596832	-0.567474827256369\\
17.0378970623016	-0.566559528217603\\
17.0877410888672	-0.565746712661053\\
17.1387340545654	-0.565472485205945\\
17.1871625900269	-0.565214385248076\\
17.2373549461365	-0.564742953881641\\
17.2877740383148	-0.564048934588534\\
17.3370270252228	-0.563063910945104\\
17.3883034706116	-0.561873730432389\\
17.4389266490936	-0.561064078331206\\
17.4871525287628	-0.560352326719268\\
17.5380241394043	-0.560129945682256\\
17.5868267536163	-0.55990800959848\\
17.6374902248383	-0.559468165899467\\
17.6872567653656	-0.558788401671514\\
17.7372834205627	-0.557805554363604\\
17.7898430347443	-0.556607959793027\\
17.8382009983063	-0.555842487580371\\
17.8908442974091	-0.555215241892306\\
17.9381300926208	-0.555137654293901\\
17.9873158454895	-0.55509040503339\\
18.038357925415	-0.554825123067978\\
18.0870062828064	-0.554356027881688\\
18.1371778964996	-0.553544052829601\\
18.1874336719513	-0.552532379117537\\
18.2375645160675	-0.55178936594919\\
18.2877232551575	-0.551134646571171\\
18.3384027004242	-0.550984488918871\\
18.3875796318054	-0.550881955854258\\
18.4372620105743	-0.550639690636505\\
18.4870087623596	-0.550202996367307\\
18.5369572162628	-0.549437714955083\\
18.5870844841003	-0.548469201920071\\
18.6373524188995	-0.547760465564131\\
18.6875247478485	-0.547128386821409\\
18.7381965637207	-0.547009669999415\\
18.787005853653	-0.54693202676032\\
18.8369290351868	-0.546723311396448\\
18.8874282360077	-0.546324193239933\\
18.9384850978851	-0.545611254394739\\
18.9872798442841	-0.544691652365447\\
19.0383550643921	-0.543995524374314\\
19.0876242637634	-0.543366242111524\\
19.13878865242	-0.543247964561516\\
19.1871997833252	-0.543181049504518\\
19.238090467453	-0.543074892358558\\
19.2875425338745	-0.542826992814685\\
19.3374142169952	-0.542289429068617\\
19.3872191429138	-0.541584553172335\\
19.4372310161591	-0.540970687405284\\
19.4873599529266	-0.540356447328804\\
19.5374044895172	-0.540142637678631\\
19.5872218132019	-0.539975839190532\\
19.636870098114	-0.539888210871808\\
19.6870867729187	-0.53968551479354\\
19.7377478599548	-0.539234293056174\\
19.7868735313416	-0.538605188288088\\
19.8382687091827	-0.537994577584376\\
19.8874859333038	-0.537375357831603\\
19.9384080886841	-0.537140407067916\\
19.9873585224152	-0.536943875344976\\
20.0380374908447	-0.536865029916743\\
20.0872175216675	-0.536661758272498\\
20.1384467601776	-0.536222191010971\\
20.1871039390564	-0.535595353341163\\
20.2371427536011	-0.534982485540477\\
20.2878586769104	-0.534349229577774\\
20.3390125751495	-0.534125361885819\\
20.3873836517334	-0.533944173787754\\
20.4368681430817	-0.533895516383625\\
20.4869358062744	-0.533747083503915\\
20.5373003005981	-0.533389527809215\\
20.5874337673187	-0.532865951608784\\
20.6376835823059	-0.532306930955752\\
20.6868483543396	-0.531663075390227\\
20.7382573604584	-0.53136515089524\\
20.7872662067413	-0.531108367792374\\
20.8385295391083	-0.531086283699126\\
20.8871445178986	-0.530987065540216\\
20.9377784252167	-0.530694502905457\\
20.9873167991638	-0.53022195295656\\
21.0377876281738	-0.529672838341594\\
21.0882877826691	-0.529046933200806\\
21.1374673366547	-0.528783436238909\\
21.1871928691864	-0.528586383074515\\
21.2374176502228	-0.528634271733325\\
21.2877287387848	-0.528611860735883\\
21.3368806362152	-0.528373412431942\\
21.3871657371521	-0.527963368577185\\
21.4381284236908	-0.52742330355732\\
21.4872109413147	-0.526796282750352\\
21.5376049995422	-0.526547687834711\\
21.5874728679657	-0.526373914048375\\
21.6380285739899	-0.526444093455042\\
21.6870588779449	-0.526453145402005\\
21.739088010788	-0.52624429877222\\
21.7871963500977	-0.525869097178366\\
21.8373071670532	-0.525400966760885\\
21.8888287067413	-0.52483205923413\\
21.9381830215454	-0.524586443787077\\
21.9872478961945	-0.52439680259657\\
22.0378989696503	-0.524440818416315\\
22.0872938156128	-0.524412363579138\\
22.138090801239	-0.52419920850529\\
22.1868211746216	-0.523841989045041\\
22.2377860069275	-0.523434177764292\\
22.2874586105347	-0.522941182420632\\
22.3384885311127	-0.522733992532206\\
22.3870710849762	-0.522556707052925\\
22.4387132644653	-0.522558478928234\\
22.4868266105652	-0.522486551141643\\
22.5373594284058	-0.522282782971314\\
22.5873472213745	-0.521961522192678\\
22.6370512962341	-0.521615620249512\\
22.6884295463562	-0.52116716027438\\
22.7378239154816	-0.520934117019654\\
22.787751865387	-0.520727310110167\\
22.8374096870422	-0.520728563029972\\
22.8873886585236	-0.520710159349449\\
22.9379829883575	-0.520613303097602\\
22.9869961261749	-0.520418034503816\\
23.037150812149	-0.520123848192028\\
23.0874580860138	-0.519715243770827\\
23.1373772144318	-0.519435268069851\\
23.1871411323547	-0.519161215028997\\
23.237290096283	-0.519114034888895\\
23.2868723392487	-0.519063692704262\\
23.3371350288391	-0.518990770588079\\
23.3872100830078	-0.518828762590177\\
23.4371966838837	-0.518559596473652\\
23.4883241176605	-0.518159425398538\\
23.5368735313416	-0.517848684002352\\
23.5883466720581	-0.517549523462765\\
23.6373395442963	-0.517522890370452\\
23.6870235919952	-0.517515751548915\\
23.7381245613098	-0.517492760220227\\
23.7868911743164	-0.517373759924172\\
23.8377303600311	-0.517096948558481\\
23.8871185302734	-0.516693905227477\\
23.9379884719849	-0.516391898442232\\
23.9873661518097	-0.516114784214874\\
24.0375952243805	-0.516124958082555\\
24.0870673179626	-0.516165880925868\\
24.1371151924133	-0.516195789418849\\
24.1869523048401	-0.516139968278102\\
24.2370559692383	-0.515901529930905\\
24.2891807079315	-0.515543463859683\\
24.3380689144135	-0.515285915505223\\
24.3871647834778	-0.515076012975394\\
24.4377190589905	-0.515145660273786\\
24.4868864536285	-0.515228817806958\\
24.5379740715027	-0.515244397029214\\
24.5869614601135	-0.515165587998208\\
24.6376108646393	-0.514932198413711\\
24.6870998859406	-0.514608705588038\\
24.7377688407898	-0.514417770208135\\
24.7873253345489	-0.514260366242494\\
24.8372883319855	-0.514324831852064\\
24.8869270801544	-0.514407750981753\\
24.9384078502655	-0.51447122754202\\
24.9870955467224	-0.514482703549259\\
25.0383190631866	-0.514391593953953\\
25.0877341747284	-0.514266147833368\\
25.1370951652527	-0.514175513248768\\
25.1875030517578	-0.514092194978668\\
25.2380036830902	-0.514114237671565\\
25.2871031284332	-0.514139143948188\\
25.338099193573	-0.514188745595973\\
25.386954498291	-0.514235536439205\\
25.4379999160767	-0.514233524547587\\
25.4874267101288	-0.514205614080858\\
25.5373079299927	-0.514145568035527\\
25.5872232437134	-0.514047872087913\\
25.6382319450378	-0.513991070130867\\
25.6868111610413	-0.513928824808917\\
25.7375020503998	-0.513923859158972\\
25.7869216918945	-0.513929984302103\\
25.8372537612915	-0.513930503467042\\
25.8889283657074	-0.513897637580491\\
25.9383606433868	-0.513802139246702\\
25.9875232696533	-0.513644125317146\\
26.0369369506836	-0.513479499140823\\
26.0872263431549	-0.513286149780041\\
26.1380128383636	-0.513209369573496\\
26.1871096611023	-0.513142596995982\\
26.2377979278564	-0.513181680873435\\
26.2872809886932	-0.513187889701546\\
26.3384959220886	-0.513129321842649\\
26.387292098999	-0.512996652507569\\
26.4373895645142	-0.512822243953453\\
26.4868809700012	-0.512607675912679\\
26.53678150177	-0.512514214520865\\
26.5870432376862	-0.512435258324931\\
26.6371018409729	-0.512476401323404\\
26.6869751930237	-0.512509069733158\\
26.7378837585449	-0.512484364710285\\
26.7877766609192	-0.512387162163326\\
26.8378419399261	-0.51221036406543\\
26.8871218681335	-0.511982759494334\\
26.9376534938812	-0.511896966331515\\
26.987083864212	-0.511843975363359\\
27.0375470638275	-0.511933670616222\\
27.0869986534119	-0.512005411939305\\
27.137539100647	-0.511969429917137\\
27.1868445396423	-0.511852345794472\\
27.2373196601868	-0.511665789638759\\
27.286718082428	-0.511463425531407\\
27.3369435787201	-0.51142939795729\\
27.3872541904449	-0.511424551287746\\
27.4369372844696	-0.511511828169161\\
27.4883272171021	-0.511559014620331\\
27.5371679782867	-0.51148697752471\\
27.5876988887787	-0.511354252803659\\
27.6373111724854	-0.511231978351541\\
27.6871180057526	-0.511110041037445\\
27.7367758274078	-0.511104188414492\\
27.7870089530945	-0.511104708805726\\
27.8375215053558	-0.5111237837767\\
27.8871633529663	-0.511107460108389\\
27.940003824234	-0.511048737259534\\
27.9871308326721	-0.510968324877781\\
28.0378810882568	-0.510911867453027\\
28.0873820304871	-0.510846019428701\\
28.1369366168976	-0.510828080281863\\
28.1869079589844	-0.51079614584594\\
28.2368185043335	-0.510771904617991\\
28.2877447128296	-0.510733462257877\\
28.3373779773712	-0.51070648849477\\
28.3870558261871	-0.510675522647383\\
28.4374107837677	-0.510667566244911\\
28.4871124744415	-0.510644093681196\\
28.5371062278748	-0.510607189174447\\
28.5868112564087	-0.510541277134292\\
28.6371838569641	-0.510473752486298\\
28.6869341850281	-0.510391507056081\\
28.7380611419678	-0.510360926087674\\
28.7870323181152	-0.510332604890626\\
28.8377327442169	-0.510344550085069\\
28.8874563694	-0.510346165490276\\
28.9368562221527	-0.510322933247534\\
28.9870872020721	-0.510269490836225\\
29.0385212421417	-0.510218172739946\\
29.0873689174652	-0.510166472053227\\
29.1368615150452	-0.510164660098738\\
29.187535238266	-0.510162350371383\\
29.2372958183289	-0.510157333372607\\
29.2871133804321	-0.510126464764994\\
29.3380169391632	-0.510083305019436\\
29.3868567466736	-0.510039427894826\\
29.4373459339142	-0.510039115472868\\
29.4873344421387	-0.510035418867643\\
29.5384788036346	-0.510027828921378\\
29.5869259357452	-0.509995458198835\\
29.6376266002655	-0.509950937948265\\
29.6871110916138	-0.509905335135798\\
29.7390245914459	-0.509898527919491\\
29.7868015289307	-0.509889826298853\\
29.8379787921906	-0.509888504222076\\
29.8871769428253	-0.509866102515204\\
29.9371747493744	-0.50981980355217\\
29.987563085556	-0.509758194592743\\
30.0383028507233	-0.509734555704956\\
30.0876338005066	-0.509710085093789\\
30.1368801116943	-0.509718374639401\\
30.1873883724213	-0.509712693338564\\
30.2370848178864	-0.509680396496239\\
30.2871500968933	-0.509622908152235\\
30.3385204792023	-0.509574668265282\\
30.386811208725	-0.509525982798523\\
30.4384216785431	-0.509523531613049\\
30.4868461608887	-0.509517243801767\\
30.5371896743774	-0.509512038791554\\
30.5867259025574	-0.509495044907685\\
30.6380993843079	-0.509462992312756\\
30.6876494407654	-0.509412205593238\\
30.7367903709412	-0.509377284272827\\
30.7870404243469	-0.509334430546554\\
30.8369943618774	-0.509321955268947\\
30.8871693134308	-0.509298685132091\\
30.9381038665771	-0.509277341949669\\
30.986993265152	-0.509235069859397\\
31.0379008769989	-0.509187542224161\\
31.0870007991791	-0.50912494308317\\
31.1389050006866	-0.50909869106003\\
31.187396478653	-0.509072984595566\\
31.2372066497803	-0.509066674024808\\
31.2869581699371	-0.509036228714018\\
31.3370429992676	-0.508976404024882\\
31.3868891716003	-0.508889329513611\\
31.4371098995209	-0.508828659640254\\
31.488369178772	-0.508770111570168\\
31.5368852138519	-0.50876104084353\\
31.5877389431	-0.508734581368767\\
31.6370863437653	-0.508698834121412\\
31.6871823787689	-0.508638038945179\\
31.738449048996	-0.508578160080685\\
31.7871157646179	-0.50851259785956\\
31.8383152008057	-0.508496659650731\\
31.8873555183411	-0.508475772035689\\
31.9394099235535	-0.508458735898652\\
31.9871487140656	-0.508412690158714\\
32.0381509780884	-0.5083424753044\\
32.088035774231	-0.508251135830547\\
32.1374802112579	-0.508205732001801\\
32.187110376358	-0.508162441872102\\
32.2387628078461	-0.508165803621346\\
32.2874614715576	-0.508157355569266\\
32.3373488903046	-0.508128693000245\\
32.3874296665192	-0.508071342150816\\
32.4374224662781	-0.508004069797773\\
32.4878580093384	-0.507926342888162\\
32.5376955986023	-0.507896958656123\\
32.5874928951263	-0.507868099006703\\
32.6386348724365	-0.507870857550298\\
32.6873406887054	-0.507856119118383\\
32.7388579368591	-0.507813934720586\\
32.7873272418976	-0.507742377753597\\
32.8374335289001	-0.50767214942528\\
32.8873004436493	-0.507596234420422\\
32.937156867981	-0.507581333000706\\
32.9873573303223	-0.507570813057594\\
33.0380663394928	-0.507578751869537\\
33.0887851238251	-0.507571505698546\\
33.1397532939911	-0.507531969100438\\
33.1896943569183	-0.507465457904212\\
33.2378904342651	-0.507407964680365\\
33.2873873233795	-0.507352419679838\\
33.337309551239	-0.50735395401676\\
33.3870064735413	-0.507354292137413\\
33.4376117706299	-0.507355260605334\\
33.4877395153046	-0.507338675166295\\
33.5375718593597	-0.507298454088126\\
33.5879704475403	-0.507248242071598\\
33.6378068447113	-0.507223389756568\\
33.6874708652496	-0.507199383013773\\
33.7375998020172	-0.50720326829201\\
33.7882034301758	-0.507193960714161\\
33.8377136707306	-0.507161221110085\\
33.8892471313477	-0.507106192020041\\
33.9370285987854	-0.507059773109977\\
33.9873494625092	-0.507015522321891\\
34.0370041847229	-0.507019084931557\\
34.0873434066772	-0.50702103377076\\
34.1362752437592	-0.50702103377076\\
34.1862122535706	-0.50702103377076\\
34.2362687110901	-0.50702103377076\\
34.2865228176117	-0.50702103377076\\
34.3363995075226	-0.50702103377076\\
34.3862075328827	-0.50702103377076\\
34.4361023426056	-0.50702103377076\\
34.4861790657043	-0.50702103377076\\
34.5361194133759	-0.50702103377076\\
34.586292219162	-0.50702103377076\\
34.6364039897919	-0.50702103377076\\
34.6863061904907	-0.50702103377076\\
34.7361094474793	-0.50702103377076\\
34.7864546298981	-0.50702103377076\\
34.83685297966	-0.50702103377076\\
34.8863391399384	-0.50702103377076\\
34.9360925674439	-0.50702103377076\\
34.986105632782	-0.50702103377076\\
35.0454177379608	-0.50702103377076\\
35.0912179470062	-0.50702103377076\\
35.136381816864	-0.50702103377076\\
35.186456155777	-0.50702103377076\\
35.2363311767578	-0.50702103377076\\
35.2861754417419	-0.50702103377076\\
35.3360485553741	-0.50702103377076\\
35.3861937046051	-0.50702103377076\\
35.4363774776459	-0.50702103377076\\
35.486319732666	-0.50702103377076\\
35.536076259613	-0.50702103377076\\
35.5862481117249	-0.50702103377076\\
35.6363558292389	-0.50702103377076\\
35.6861931800842	-0.50702103377076\\
35.7363302230835	-0.50702103377076\\
35.786288690567	-0.50702103377076\\
35.8361827850342	-0.50702103377076\\
35.8864717006683	-0.50702103377076\\
35.9363789081574	-0.50702103377076\\
35.9867715358734	-0.50702103377076\\
36.0360667228699	-0.50702103377076\\
36.0863480091095	-0.50702103377076\\
36.1363653659821	-0.50702103377076\\
36.1864337444305	-0.50702103377076\\
36.2360584259033	-0.50702103377076\\
36.2864779949188	-0.50702103377076\\
36.336106967926	-0.50702103377076\\
36.3864702701569	-0.50702103377076\\
36.4362844944	-0.50702103377076\\
36.4863364219666	-0.50702103377076\\
36.5363025188446	-0.50702103377076\\
36.5861410617828	-0.50702103377076\\
36.6363481998444	-0.50702103377076\\
36.6863061904907	-0.50702103377076\\
36.7361874103546	-0.50702103377076\\
36.7863311290741	-0.50702103377076\\
36.8363565921783	-0.50702103377076\\
36.8863608360291	-0.50702103377076\\
36.9363264560699	-0.50702103377076\\
36.9860896587372	-0.50702103377076\\
37.0362286090851	-0.50702103377076\\
37.0864164352417	-0.50702103377076\\
37.1363679885864	-0.50702103377076\\
37.1863879680634	-0.50702103377076\\
37.2362648963928	-0.50702103377076\\
37.2863585472107	-0.50702103377076\\
37.3360833644867	-0.50702103377076\\
37.3862823963165	-0.50702103377076\\
37.4368352413178	-0.50702103377076\\
37.4861456871033	-0.50702103377076\\
37.5361048698425	-0.50702103377076\\
37.5862543106079	-0.50702103377076\\
37.6362061023712	-0.50702103377076\\
37.6863986968994	-0.50702103377076\\
37.736749124527	-0.50702103377076\\
37.7863394737244	-0.50702103377076\\
37.8361865997314	-0.50702103377076\\
37.888374042511	-0.50702103377076\\
37.9363502979279	-0.50702103377076\\
37.9863593101501	-0.50702103377076\\
38.0362567424774	-0.50702103377076\\
38.0862390518189	-0.50702103377076\\
38.1364056587219	-0.50702103377076\\
38.1864752292633	-0.50702103377076\\
38.2363790988922	-0.50702103377076\\
38.2863652229309	-0.50702103377076\\
38.3364450454712	-0.50702103377076\\
38.3863930225372	-0.50702103377076\\
38.4377541065216	-0.50702103377076\\
38.4863030433655	-0.50702103377076\\
38.5364169597626	-0.50702103377076\\
38.5863399028778	-0.50702103377076\\
38.6360737800598	-0.50702103377076\\
38.686456155777	-0.50702103377076\\
38.7360977649689	-0.50702103377076\\
38.7863506793976	-0.50702103377076\\
38.8363575458527	-0.50702103377076\\
38.8868772506714	-0.50702103377076\\
38.9363157272339	-0.50702103377076\\
38.9863695621491	-0.50702103377076\\
39.036302280426	-0.50702103377076\\
39.0863775730133	-0.50702103377076\\
39.1364001750946	-0.50702103377076\\
39.1863421916962	-0.50702103377076\\
39.2363414287567	-0.50702103377076\\
39.2865790843964	-0.50702103377076\\
39.3364316940308	-0.50702103377076\\
39.3863877773285	-0.50702103377076\\
39.4363846302032	-0.50702103377076\\
39.4864034175873	-0.50702103377076\\
39.5362460136414	-0.50702103377076\\
39.5864672183991	-0.50702103377076\\
39.6376175403595	-0.50702103377076\\
39.6865121841431	-0.50702103377076\\
39.7360577106476	-0.50702103377076\\
39.7860879421234	-0.50702103377076\\
39.8361377239227	-0.50702103377076\\
39.8862397193909	-0.50702103377076\\
39.9363758087158	-0.50702103377076\\
39.9866580486298	-0.50702103377076\\
40.0361229896545	-0.50702103377076\\
40.0863337039948	-0.50702103377076\\
40.1362952709198	-0.50702103377076\\
40.1863722324371	-0.50702103377076\\
40.2360808372498	-0.50702103377076\\
40.2867063999176	-0.50702103377076\\
40.3362371444702	-0.50702103377076\\
40.3864602565765	-0.50702103377076\\
40.4362022399902	-0.50702103377076\\
40.4862391471863	-0.50702103377076\\
40.536110830307	-0.50702103377076\\
40.586754989624	-0.50702103377076\\
40.6363067150116	-0.50702103377076\\
40.6862320423126	-0.50702103377076\\
40.7363345146179	-0.50702103377076\\
40.7863447189331	-0.50702103377076\\
40.8362771987915	-0.50702103377076\\
40.886261177063	-0.50702103377076\\
40.936424446106	-0.50702103377076\\
40.9864334583282	-0.50702103377076\\
41.0363933563232	-0.50702103377076\\
41.0863849639893	-0.50702103377076\\
41.1362373352051	-0.50702103377076\\
41.1860804080963	-0.50702103377076\\
41.2361869335175	-0.50702103377076\\
41.2864078998566	-0.50702103377076\\
41.3360969543457	-0.50702103377076\\
41.3861855983734	-0.50702103377076\\
41.4376759052277	-0.50702103377076\\
41.486256313324	-0.50702103377076\\
41.536371421814	-0.50702103377076\\
41.5862829208374	-0.50702103377076\\
41.636062335968	-0.50702103377076\\
41.686300945282	-0.50702103377076\\
41.7360789299011	-0.50702103377076\\
41.7862710475922	-0.50702103377076\\
41.8360845565796	-0.50702103377076\\
41.8863353252411	-0.50702103377076\\
41.9362205982208	-0.50702103377076\\
41.986283493042	-0.50702103377076\\
42.0364117145538	-0.50702103377076\\
42.0863775730133	-0.50702103377076\\
42.1363970756531	-0.50702103377076\\
42.186447095871	-0.50702103377076\\
42.2362341403961	-0.50702103377076\\
42.2862476825714	-0.50702103377076\\
42.3368155479431	-0.50702103377076\\
42.3862549781799	-0.50702103377076\\
42.4363796234131	-0.50702103377076\\
42.4863807678223	-0.50702103377076\\
42.5361065387726	-0.50702103377076\\
42.586087179184	-0.50702103377076\\
42.6363203048706	-0.50702103377076\\
42.6863209724426	-0.50702103377076\\
42.7362560749054	-0.50702103377076\\
42.7863132476807	-0.50702103377076\\
42.8362674236298	-0.50702103377076\\
42.8862435340881	-0.50702103377076\\
42.9362756729126	-0.50702103377076\\
42.9860562801361	-0.50702103377076\\
43.0366630077362	-0.50702103377076\\
43.0862891197205	-0.50702103377076\\
43.1369411468506	-0.50702103377076\\
43.1862763881683	-0.50702103377076\\
43.2363652706146	-0.50702103377076\\
43.2866267681122	-0.50702103377076\\
43.3361598968506	-0.50702103377076\\
43.3862552165985	-0.50702103377076\\
43.4363893985748	-0.50702103377076\\
43.4867214679718	-0.50702103377076\\
43.5363444805145	-0.50702103377076\\
43.5866703510284	-0.50702103377076\\
43.6362726211548	-0.50702103377076\\
43.6863271713257	-0.50702103377076\\
43.7361363887787	-0.50702103377076\\
43.7873098373413	-0.50702103377076\\
43.8363158226013	-0.50702103377076\\
43.8864092350006	-0.50702103377076\\
43.9364137172699	-0.50702103377076\\
43.9862827777863	-0.50702103377076\\
44.0369226455689	-0.50702103377076\\
44.0863043785095	-0.50702103377076\\
44.1360961914063	-0.50702103377076\\
44.1862127304077	-0.50702103377076\\
44.236283493042	-0.50702103377076\\
44.2865638256073	-0.50702103377076\\
44.3360812187195	-0.50702103377076\\
44.386487197876	-0.50702103377076\\
44.4361028194428	-0.50702103377076\\
44.4864933013916	-0.50702103377076\\
44.536284160614	-0.50702103377076\\
44.5863046169281	-0.50702103377076\\
44.6362940788269	-0.50702103377076\\
44.6862978458405	-0.50702103377076\\
44.7364215373993	-0.50702103377076\\
44.7863480567932	-0.50702103377076\\
44.8362671852112	-0.50702103377076\\
44.8862499713898	-0.50702103377076\\
44.9360589504242	-0.50702103377076\\
44.9889430522919	-0.50702103377076\\
45.0363490104675	-0.50702103377076\\
45.0864402770996	-0.50702103377076\\
45.1361271858215	-0.50702103377076\\
45.1861283302307	-0.50702103377076\\
45.2363178253174	-0.50702103377076\\
45.2860579013824	-0.50702103377076\\
45.3360833644867	-0.50702103377076\\
45.3861248016357	-0.50702103377076\\
45.4362778186798	-0.50702103377076\\
45.4862968444824	-0.50702103377076\\
45.5364520072937	-0.50702103377076\\
45.5863546848297	-0.50702103377076\\
45.6363763332367	-0.50702103377076\\
45.6862480163574	-0.50702103377076\\
45.7361382961273	-0.50702103377076\\
45.7863018035889	-0.50702103377076\\
45.8363010406494	-0.50702103377076\\
45.8863670349121	-0.50702103377076\\
45.9360751628876	-0.50702103377076\\
45.9862565517426	-0.50702103377076\\
46.036564540863	-0.50702103377076\\
46.086417388916	-0.50702103377076\\
46.136324596405	-0.50702103377076\\
46.1863011837006	-0.50702103377076\\
46.236340713501	-0.50702103377076\\
46.286172580719	-0.50702103377076\\
46.3363940238953	-0.50702103377076\\
46.3862518787384	-0.50702103377076\\
46.4363107204437	-0.50702103377076\\
46.4862446308136	-0.50702103377076\\
46.5362638950348	-0.50702103377076\\
46.5863682746887	-0.50702103377076\\
46.6360892772675	-0.50702103377076\\
46.6863905906677	-0.50702103377076\\
46.7362088680267	-0.50702103377076\\
46.7864021778107	-0.50702103377076\\
46.8362369060516	-0.50702103377076\\
46.8863255500793	-0.50702103377076\\
46.9363812923431	-0.50702103377076\\
46.9860479354858	-0.50702103377076\\
47.0364589214325	-0.50702103377076\\
47.086301279068	-0.50702103377076\\
47.1364011287689	-0.50702103377076\\
47.1863252639771	-0.50702103377076\\
47.2360765457153	-0.50702103377076\\
47.2863361358643	-0.50702103377076\\
47.336061668396	-0.50702103377076\\
47.3863124370575	-0.50702103377076\\
47.436296415329	-0.50702103377076\\
47.4863233089447	-0.50702103377076\\
47.5360545635223	-0.50702103377076\\
47.5861608505249	-0.50702103377076\\
47.6362637996674	-0.50702103377076\\
47.6860692024231	-0.50702103377076\\
47.7363540649414	-0.50702103377076\\
47.7863378047943	-0.50702103377076\\
47.8363644599915	-0.50702103377076\\
47.8862695217133	-0.50702103377076\\
47.9362542152405	-0.50702103377076\\
47.9863764762878	-0.50702103377076\\
48.0360953330994	-0.50702103377076\\
48.0863441944122	-0.50702103377076\\
48.1360756874085	-0.50702103377076\\
48.186307144165	-0.50702103377076\\
48.2360848903656	-0.50702103377076\\
48.2862736701965	-0.50702103377076\\
48.3363034248352	-0.50702103377076\\
48.3863589286804	-0.50702103377076\\
48.4373104095459	-0.50702103377076\\
48.4862935066223	-0.50702103377076\\
48.5363015651703	-0.50702103377076\\
48.5862574100494	-0.50702103377076\\
48.6363148212433	-0.50702103377076\\
48.6862716197968	-0.50702103377076\\
48.7360643863678	-0.50702103377076\\
48.7863087177277	-0.50702103377076\\
48.8364052295685	-0.50702103377076\\
48.8860489845276	-0.50702103377076\\
48.9366774082184	-0.50702103377076\\
48.9863929271698	-0.50702103377076\\
49.0368101119995	-0.50702103377076\\
49.0863396644592	-0.50702103377076\\
49.1362618923187	-0.50702103377076\\
49.1863467216492	-0.50702103377076\\
49.2363132953644	-0.50702103377076\\
49.2862998962402	-0.50702103377076\\
49.3360840797424	-0.50702103377076\\
49.3862306594849	-0.50702103377076\\
49.4362706661224	-0.50702103377076\\
49.486224603653	-0.50702103377076\\
49.5363234996796	-0.50702103377076\\
49.5862879276276	-0.50702103377076\\
49.6361152648926	-0.50702103377076\\
49.6863047599793	-0.50702103377076\\
49.7360612869263	-0.50702103377076\\
49.7862214565277	-0.50702103377076\\
49.8363184452057	-0.50702103377076\\
49.8879334449768	-0.50702103377076\\
49.9363319396973	-0.50702103377076\\
49.9862370014191	-0.50702103377076\\
50.0362956047058	-0.50702103377076\\
50.0868522644043	-0.50702103377076\\
50.1361271858215	-0.50702103377076\\
50.1862706661224	-0.50702103377076\\
50.2362811088562	-0.50702103377076\\
50.2863530635834	-0.50702103377076\\
50.3364486217499	-0.50702103377076\\
50.3862692832947	-0.50702103377076\\
50.4361993789673	-0.50702103377076\\
50.4862770557404	-0.50702103377076\\
50.5360641002655	-0.50702103377076\\
50.5860704898834	-0.50702103377076\\
50.636267375946	-0.50702103377076\\
50.6864208698273	-0.50702103377076\\
50.7364587306976	-0.50702103377076\\
50.786306810379	-0.50702103377076\\
50.8363444328308	-0.50702103377076\\
50.8863732337952	-0.50702103377076\\
50.9360637187958	-0.50702103377076\\
50.9860643863678	-0.50702103377076\\
51.0364918231964	-0.50702103377076\\
51.086265039444	-0.50702103377076\\
51.1363150596619	-0.50702103377076\\
51.1863760471344	-0.50702103377076\\
51.2363709926605	-0.50702103377076\\
51.2863130092621	-0.50702103377076\\
51.3363859176636	-0.50702103377076\\
51.3862800121307	-0.50702103377076\\
51.4363278865814	-0.50702103377076\\
51.4863607406616	-0.50702103377076\\
51.5360641002655	-0.50702103377076\\
51.5886544704437	-0.50702103377076\\
51.6360978603363	-0.50702103377076\\
51.688673210144	-0.50702103377076\\
51.7367107391357	-0.50702103377076\\
51.7860705375671	-0.50702103377076\\
51.8363744735718	-0.50702103377076\\
51.8862721443176	-0.50702103377076\\
51.936353635788	-0.50702103377076\\
51.9860617637634	-0.50702103377076\\
52.0368484973907	-0.50702103377076\\
52.0862779140472	-0.50702103377076\\
52.1360904693604	-0.50702103377076\\
52.1863877296448	-0.50702103377076\\
52.2361249446869	-0.50702103377076\\
52.2863645076752	-0.50702103377076\\
52.3361343860626	-0.50702103377076\\
52.3863236427307	-0.50702103377076\\
52.4363097667694	-0.50702103377076\\
52.4860250473022	-0.50702103377076\\
52.5360769748688	-0.50702103377076\\
52.5860502243042	-0.50702103377076\\
52.6363713264465	-0.50702103377076\\
52.690006685257	-0.50702103377076\\
52.7364186763763	-0.50702103377076\\
52.7863256454468	-0.50702103377076\\
52.8360683441162	-0.50702103377076\\
52.8861855983734	-0.50702103377076\\
52.9360858917236	-0.50702103377076\\
52.9860407829285	-0.50702103377076\\
53.0362543582916	-0.50702103377076\\
53.0863141536713	-0.50702103377076\\
53.1362888336182	-0.50702103377076\\
53.1872043132782	-0.50702103377076\\
53.2363035202026	-0.50702103377076\\
53.2860457420349	-0.50702103377076\\
53.3363325119019	-0.50702103377076\\
53.3861095428467	-0.50702103377076\\
53.436364364624	-0.50702103377076\\
53.4863154411316	-0.50702103377076\\
53.5363397121429	-0.50702103377076\\
53.5863167762756	-0.50702103377076\\
53.6361658096314	-0.50702103377076\\
53.6863488674164	-0.50702103377076\\
53.7363729000092	-0.50702103377076\\
53.7860550403595	-0.50702103377076\\
53.836292219162	-0.50702103377076\\
53.8862983703613	-0.50702103377076\\
53.9362475395203	-0.50702103377076\\
53.9861213684082	-0.50702103377076\\
54.0362629413605	-0.50702103377076\\
54.0862345218658	-0.50702103377076\\
54.1360654354095	-0.50702103377076\\
54.1862487316132	-0.50702103377076\\
54.2360968112946	-0.50702103377076\\
54.2862512588501	-0.50702103377076\\
54.3363804340363	-0.50702103377076\\
54.3863088607788	-0.50702103377076\\
54.4362613677979	-0.50702103377076\\
54.4864789962769	-0.50702103377076\\
54.5363761901855	-0.50702103377076\\
54.5860692977905	-0.50702103377076\\
54.6361002445221	-0.50702103377076\\
54.6862933158875	-0.50702103377076\\
54.7361409187317	-0.50702103377076\\
54.7861282348633	-0.50702103377076\\
54.8360788345337	-0.50702103377076\\
54.8863553524017	-0.50702103377076\\
54.9360541820526	-0.50702103377076\\
54.9860584259033	-0.50702103377076\\
55.0363721370697	-0.50702103377076\\
55.0868153095245	-0.50702103377076\\
55.136411857605	-0.50702103377076\\
55.1862184524536	-0.50702103377076\\
55.2393674373627	-0.50702103377076\\
55.2863261222839	-0.50702103377076\\
55.33689661026	-0.50702103377076\\
55.3863090991974	-0.50702103377076\\
55.4367873191834	-0.50702103377076\\
55.4863738536835	-0.50702103377076\\
55.5360693454742	-0.50702103377076\\
55.586337518692	-0.50702103377076\\
55.6370651245117	-0.50702103377076\\
55.6863228797913	-0.50702103377076\\
55.7360960960388	-0.50702103377076\\
55.7863347053528	-0.50702103377076\\
};
\addlegendentry{$\theta_{x_{p}}$}

\addplot [color=mycolor4, line width=1.4pt]
  table[row sep=crcr]{%
-3.41395025253296	0\\
-3.36380581855774	0\\
-3.31373219490051	0\\
-3.26373772621155	0\\
-3.21386485099792	0\\
-3.16371612548828	0\\
-3.11396126747131	0\\
-3.0639102935791	0\\
-3.01393465995789	0\\
-2.96376042366028	0\\
-2.91368894577026	0\\
-2.86386590003967	0\\
-2.81392412185669	0\\
-2.76383023262024	0\\
-2.71365766525269	0\\
-2.66386275291443	0\\
-2.61371665000916	0\\
-2.56373028755188	0\\
-2.51375107765198	0\\
-2.46365218162537	0\\
-2.41376595497131	0\\
-2.36374454498291	0\\
-2.31372838020325	0\\
-2.26391820907593	0\\
-2.21392397880554	0\\
-2.16394262313843	0\\
-2.11378173828125	0\\
-2.06323270797729	0\\
-2.01386647224426	0\\
-1.96378617286682	0\\
-1.91375093460083	0\\
-1.86376647949219	0\\
-1.81371264457703	0\\
-1.76388936042786	0\\
-1.71374897956848	0\\
-1.66370325088501	0\\
-1.61389498710632	0\\
-1.56357769966125	0\\
-1.51373534202576	0\\
-1.46395378112793	0\\
-1.41391592025757	0\\
-1.36373047828674	0\\
-1.31355028152466	0\\
-1.26392798423767	0\\
-1.21389083862305	0\\
-1.1639223575592	0\\
-1.11384873390198	0\\
-1.06395726203918	0\\
-1.01393632888794	0\\
-0.96373085975647	0\\
-0.913796234130859	0\\
-0.863754081726074	0\\
-0.81366925239563	0\\
-0.763712215423584	0\\
-0.71373085975647	0\\
-0.663900899887085	0\\
-0.61391978263855	0\\
-0.563617038726806	0\\
-0.51369457244873	0\\
-0.46377067565918	0\\
-0.413990783691406	0\\
-0.363846826553345	0\\
-0.313869285583496	0\\
-0.263829755783081	0\\
-0.213733005523681	0\\
-0.163687038421631	0\\
-0.113701391220093	0\\
-0.0636923789978026	0\\
-0.0136535644531248	0\\
0.0374466896057131	-1.69170757309664\\
0.0868460655212404	-2.3250264160551\\
0.137080144882202	-2.64211996880562\\
0.187013101577759	-2.81527456617505\\
0.237029504776001	-2.92302093710896\\
0.288115215301514	-3.00122080095548\\
0.336934280395508	-3.06646023719475\\
0.387251806259155	-3.12142419070051\\
0.436960172653198	-3.16624017320373\\
0.487322521209717	-3.1988263276487\\
0.537163209915161	-3.22145926690223\\
0.587262582778931	-3.23376032434726\\
0.638643217086792	-3.24116649265579\\
0.687006664276123	-3.24674516507162\\
0.737519216537476	-3.25113905499393\\
0.787285280227661	-3.25338103398599\\
0.838375759124756	-3.25488521565103\\
0.887129497528076	-3.25480946593871\\
0.937354993820191	-3.25280463552554\\
0.98739595413208	-3.24816395265543\\
1.03824419975281	-3.24010520244883\\
1.08832807540894	-3.22889729136705\\
1.13733215332031	-3.21511352938251\\
1.18734998703003	-3.20020191830008\\
1.23741598129272	-3.18792253311494\\
1.28700251579285	-3.17680419903445\\
1.33715624809265	-3.16634284790916\\
1.3871440410614	-3.15298237960315\\
1.43721837997437	-3.13645830606674\\
1.48703689575195	-3.11638828195191\\
1.53733153343201	-3.092589730295\\
1.58698172569275	-3.06861130118978\\
1.6369044303894	-3.0476473273975\\
1.68689270019531	-3.028825531027\\
1.73750514984131	-3.01118810599928\\
1.78736562728882	-2.99180700387751\\
1.83805890083313	-2.96975328654207\\
1.8878821849823	-2.94391604158136\\
1.93722839355469	-2.91445417344403\\
1.9872805595398	-2.8849720264634\\
2.03788895606995	-2.85808234254637\\
2.08703417778015	-2.83404999966024\\
2.13730902671814	-2.8110865236722\\
2.18721742630005	-2.78662356865061\\
2.2393238067627	-2.75968610060227\\
2.28719348907471	-2.72960234670973\\
2.33826942443848	-2.69600339030603\\
2.38713521957397	-2.6621649286426\\
2.43748707771301	-2.63017551362827\\
2.487189245224	-2.60133732806753\\
2.53701467514038	-2.57521410477966\\
2.58730955123901	-2.54767980554243\\
2.6372718334198	-2.51687813128046\\
2.68767471313477	-2.48154994234937\\
2.73776693344116	-2.44174202960676\\
2.78724522590637	-2.40216362023784\\
2.83687014579773	-2.36554056062505\\
2.88723630905151	-2.33365381537988\\
2.93726320266724	-2.30427292448167\\
2.98756213188171	-2.27390778271274\\
3.03776831626892	-2.24081424010092\\
3.08719940185547	-2.20309579998957\\
3.13819379806519	-2.16121673025282\\
3.1871039390564	-2.11927052488409\\
3.2383677482605	-2.08093944711891\\
3.28719182014465	-2.04747108531956\\
3.33709425926209	-2.01623283207664\\
3.3875627040863	-1.98347141954764\\
3.43703336715698	-1.94740330091645\\
3.48778195381165	-1.90678239525369\\
3.53756542205811	-1.86279546357855\\
3.58713026046753	-1.82158811130921\\
3.63820476531982	-1.78435589404125\\
3.68715305328369	-1.75189985943689\\
3.7370192527771	-1.72063636583698\\
3.78696150779724	-1.68743258393897\\
3.83697123527527	-1.65121126835675\\
3.88720889091492	-1.61183695673753\\
3.93718166351318	-1.57054410856108\\
3.98680086135864	-1.5318419790292\\
4.0373019695282	-1.49611035063481\\
4.087202501297	-1.46339624874327\\
4.13750429153442	-1.43217093504882\\
4.18730540275574	-1.39910860289547\\
4.23692674636841	-1.36469452608708\\
4.28764815330505	-1.32876992555157\\
4.33715195655823	-1.29229389768898\\
4.38732810020447	-1.25880012980087\\
4.43775720596313	-1.2275846537932\\
4.48715395927429	-1.19882092832995\\
4.53803582191467	-1.16995883909021\\
4.58724708557129	-1.13994422912037\\
4.63719172477722	-1.10860187355865\\
4.68718476295471	-1.07657743938671\\
4.73849172592163	-1.04504204275963\\
4.78692073822022	-1.01616858904436\\
4.83752150535584	-0.98882500448417\\
4.88743896484375	-0.963354624267595\\
4.93707461357117	-0.937673158440703\\
4.98722429275513	-0.911049923116479\\
5.03730530738831	-0.88388612108065\\
5.08796877861023	-0.856345471583154\\
5.13730926513672	-0.829194534497219\\
5.1875111579895	-0.804426330024398\\
5.2383083820343	-0.781242910817127\\
5.28709311485291	-0.760165874115955\\
5.33733768463135	-0.739764935607127\\
5.38735957145691	-0.719242162874252\\
5.43840856552124	-0.698459265318888\\
5.4869279384613	-0.677435215460719\\
5.53821368217468	-0.65670152182156\\
5.58721609115601	-0.637607010144166\\
5.63834400177002	-0.619601529975625\\
5.70256943702698	-0.603263629193862\\
5.73696870803833	-0.587387593377571\\
5.78678030967712	-0.57148504318593\\
5.83695526123047	-0.555269011766541\\
5.88792462348938	-0.538891791688911\\
5.93726081848145	-0.522712007919836\\
5.98705382347107	-0.507963694118189\\
6.037406873703	-0.494231451692031\\
6.08693976402283	-0.481937560639381\\
6.13725109100342	-0.47023457503019\\
6.1868757724762	-0.458575858361655\\
6.23866934776306	-0.446724911678757\\
6.28684301376343	-0.43477867420188\\
6.3376051902771	-0.423043022609818\\
6.38692278862	-0.41236799545095\\
6.43746538162231	-0.402631826352263\\
6.48725814819336	-0.394115702711133\\
6.53829855918884	-0.385985905399139\\
6.58715434074402	-0.377905472025304\\
6.63686246871948	-0.369604390168206\\
6.68743295669556	-0.361209997455262\\
6.73686356544495	-0.353082822209217\\
6.78764243125916	-0.345728838625291\\
6.83729786872864	-0.339225949862339\\
6.8867949962616	-0.333610920906722\\
6.93705792427063	-0.328352333815019\\
6.98714203834534	-0.323064804927526\\
7.03722019195557	-0.317504370939787\\
7.0867495059967	-0.311842629523198\\
7.13819832801819	-0.306394378349978\\
7.18691987991333	-0.301615514252262\\
7.23969264030457	-0.297668284713836\\
7.28743596076965	-0.294327444234568\\
7.33690996170044	-0.2912819013236\\
7.38727087974548	-0.288141912678384\\
7.43710012435913	-0.284747103751215\\
7.48817820549011	-0.281404224893322\\
7.5378050327301	-0.278503363982509\\
7.58700842857361	-0.276042228728329\\
7.63773460388184	-0.274258578328045\\
7.68709034919739	-0.272839631898478\\
7.7376603603363	-0.271605771785289\\
7.78713698387146	-0.270309431163696\\
7.83752889633179	-0.268775288802772\\
7.88738055229187	-0.267360503643431\\
7.93729419708252	-0.266230928948517\\
7.98730368614197	-0.265344670885952\\
8.03704233169556	-0.264951449695673\\
8.08665652275085	-0.26475129226219\\
8.13837356567383	-0.264711637314122\\
8.18696184158325	-0.264650155364961\\
8.23794646263122	-0.26450372895124\\
8.28800339698791	-0.264487528807877\\
8.33725209236145	-0.264703195137827\\
8.38739914894104	-0.264997583754848\\
8.43756384849548	-0.265609745851634\\
8.48700923919678	-0.266313426303896\\
8.53716988563538	-0.267008322714446\\
8.5874659538269	-0.267712325110551\\
8.63724226951599	-0.268240714758122\\
8.68716926574707	-0.268782992643992\\
8.73704977035522	-0.269501969951307\\
8.78744978904724	-0.270227383942711\\
8.83704800605774	-0.271369804244387\\
8.88674945831299	-0.272609749919411\\
8.93740029335022	-0.27389663194117\\
8.98759908676147	-0.275177477934449\\
9.03756065368652	-0.276340669349793\\
9.08785433769226	-0.277360810766822\\
9.13731617927551	-0.278528197473861\\
9.18759603500366	-0.279619751872588\\
9.23747487068176	-0.281134843809696\\
9.28715224266052	-0.282832874826909\\
9.33708448410034	-0.284481740826777\\
9.38704843521118	-0.286060544654219\\
9.43731732368469	-0.287376583350579\\
9.48708744049072	-0.288485939578777\\
9.5378701210022	-0.28988896023543\\
9.58680291175842	-0.291210457354964\\
9.63790078163147	-0.293039673176963\\
9.68698759078979	-0.294997934125831\\
9.73723216056824	-0.296773458497682\\
9.78691930770874	-0.298399023444972\\
9.83674449920654	-0.29969235591534\\
9.8874725818634	-0.300728626156797\\
9.93801159858704	-0.302098487832609\\
9.98698897361755	-0.303507214371848\\
10.0372972011566	-0.305507305002038\\
10.0883295059204	-0.307589193935428\\
10.1372539520264	-0.309478695320195\\
10.1879517555237	-0.311100445969259\\
10.2371825695038	-0.312254597458605\\
10.287105512619	-0.313128647341728\\
10.3395609378815	-0.314397187043824\\
10.387139749527	-0.315750711389001\\
10.4386827468872	-0.317868756706275\\
10.4871279716492	-0.320012736757008\\
10.537340593338	-0.321899873665643\\
10.5872203826904	-0.323500141750628\\
10.6368417263031	-0.324597732868142\\
10.6872877597809	-0.325441822621997\\
10.7376646518707	-0.326802760680721\\
10.7870134830475	-0.32824052779597\\
10.8377515792847	-0.330407605910978\\
10.8868004798889	-0.332565631737907\\
10.9370796203613	-0.334379463748064\\
10.9869966030121	-0.335907271356291\\
11.0377678394318	-0.337079311666798\\
11.0870306015015	-0.338101233451198\\
11.1382483959198	-0.339643779338417\\
11.1873485565186	-0.341160603801654\\
11.2382921695709	-0.343155275441177\\
11.2869455337524	-0.345096172245235\\
11.337321472168	-0.346765674999574\\
11.3868953704834	-0.348257523573466\\
11.4372324466705	-0.349500395947075\\
11.4881226539612	-0.350598308717167\\
11.5377847671509	-0.35202965323856\\
11.5873648643494	-0.353430748045525\\
11.6380569458008	-0.35528149149718\\
11.6869468212128	-0.357044243379619\\
11.7386147499084	-0.358613125553916\\
11.7868301391602	-0.359983964234289\\
11.8376504898071	-0.361109378338597\\
11.8868886947632	-0.362056005962813\\
11.9389214038849	-0.363310539522445\\
11.9870438098907	-0.364547684505965\\
12.0372139930725	-0.366292918904946\\
12.0867568969727	-0.368035301714372\\
12.1371871948242	-0.369660942617429\\
12.1868307113647	-0.371022705384576\\
12.2372700691223	-0.372062588214021\\
12.2880680084229	-0.372843622984419\\
12.3380994319916	-0.373893122067216\\
12.3877601146698	-0.374935814752689\\
12.4371673583984	-0.376592847379698\\
12.4874753475189	-0.378264599853537\\
12.5367342948914	-0.379845557657973\\
12.5872320652008	-0.381233258082602\\
12.6379358291626	-0.382106710152812\\
12.6869699478149	-0.382670097174241\\
12.737691116333	-0.383538586015788\\
12.7872032642364	-0.384380244362774\\
12.837261390686	-0.385970678832336\\
12.8871109008789	-0.387562354088232\\
12.9367858886719	-0.389051257723452\\
12.9869508266449	-0.390366221124566\\
13.03677983284	-0.391216162515434\\
13.0889579772949	-0.391797143631379\\
13.1373173713684	-0.392541787758148\\
13.1868133068085	-0.393249820814674\\
13.2383656024933	-0.394618490433572\\
13.2870273113251	-0.396025017631416\\
13.3382465362549	-0.397495972747379\\
13.3877489089966	-0.398794471777819\\
13.4376754283905	-0.399684676770221\\
13.4869431972504	-0.400298537321191\\
13.5382995128632	-0.40094705311094\\
13.5876717090607	-0.40153417015614\\
13.6373288154602	-0.402751540715428\\
13.6871709346771	-0.404016293768109\\
13.7372221469879	-0.4054578879975\\
13.7870430469513	-0.406772948091032\\
13.8368041038513	-0.407663666439873\\
13.8874387264252	-0.408294529857983\\
13.9370603084564	-0.408928820672202\\
13.9871797084808	-0.409513027867774\\
14.0383731842041	-0.41070791861523\\
14.0870515823364	-0.411965782915843\\
14.1367220401764	-0.413364429741989\\
14.1868392944336	-0.414603875425598\\
14.2372829437256	-0.41539084460959\\
14.2870769023895	-0.415882256952898\\
14.3382911205292	-0.416403464046937\\
14.3871123313904	-0.416906868134902\\
14.4397968769073	-0.418042063253154\\
14.4871720790863	-0.419246989880151\\
14.5370795249939	-0.420634789770986\\
14.587115240097	-0.421880061985803\\
14.6373021125793	-0.422631987027557\\
14.6878222942352	-0.423098793811647\\
14.7388114452362	-0.423478551536192\\
14.7875110626221	-0.423784527506513\\
14.8372318267822	-0.424758266243145\\
14.8873505115509	-0.425796797355758\\
14.9386431694031	-0.427117258799456\\
14.9871870994568	-0.428268108707357\\
15.0373312950134	-0.428935494253295\\
15.0867146968842	-0.429276894679163\\
15.1386305809021	-0.429560784334228\\
15.1868702888489	-0.42983516126634\\
15.2371518135071	-0.430837854240082\\
15.2869948863983	-0.43191947542411\\
15.3384689807892	-0.433167264459087\\
15.3868526935577	-0.434269311212688\\
15.4374238967896	-0.434890273822361\\
15.4874579429626	-0.435220491723982\\
15.5373572826385	-0.435491435087215\\
15.5874838352203	-0.435717476702052\\
15.6373798370361	-0.43661790118378\\
15.6868905544281	-0.437572873691408\\
15.737205696106	-0.438707946202726\\
15.7869731903076	-0.439713849788035\\
15.8384644508362	-0.440320267998914\\
15.887641620636	-0.440658149704802\\
15.938517999649	-0.440925817109076\\
15.9870197296143	-0.441150442852461\\
16.0382165431976	-0.441968619472249\\
16.087006521225	-0.442833738570016\\
16.1369261264801	-0.443900145571405\\
16.187291097641	-0.444846730278499\\
16.2368158817291	-0.44540053940591\\
16.2877594947815	-0.445713787880692\\
16.3371722221375	-0.445998513015951\\
16.3871049404144	-0.446225514041714\\
16.4373132705688	-0.446988648485757\\
16.4878031730652	-0.447777028830387\\
16.5373649120331	-0.448739694702923\\
16.5867781162262	-0.449601984855718\\
16.6383754730225	-0.450172334199422\\
16.6873669147491	-0.450590917310151\\
16.7393970012665	-0.450938225358911\\
16.7873897075653	-0.451180001164744\\
16.8394999027252	-0.451797876346006\\
16.8871914863586	-0.452433121624608\\
16.9372188568115	-0.453335518077893\\
16.9868101596832	-0.454155224082896\\
17.0378970623016	-0.454739585669131\\
17.0877410888672	-0.455120480413015\\
17.1387340545654	-0.455389922470033\\
17.1871625900269	-0.455539541723866\\
17.2373549461365	-0.456121918158232\\
17.2877740383148	-0.456748922392421\\
17.3370270252228	-0.457673039970175\\
17.3883034706116	-0.458500704944257\\
17.4389266490936	-0.459035255005261\\
17.4871525287628	-0.459341092399001\\
17.5380241394043	-0.459507925142105\\
17.5868267536163	-0.45954947708077\\
17.6374902248383	-0.460028411326633\\
17.6872567653656	-0.460561634298472\\
17.7372834205627	-0.461457641658711\\
17.7898430347443	-0.46227246827813\\
17.8382009983063	-0.462780333207419\\
17.8908442974091	-0.463054489914256\\
17.9381300926208	-0.463138038986941\\
17.9873158454895	-0.463019928850002\\
18.038357925415	-0.463353447888409\\
18.0870062828064	-0.463731936482574\\
18.1371778964996	-0.464550679135087\\
18.1874336719513	-0.465279619125738\\
18.2375645160675	-0.465772767498962\\
18.2877232551575	-0.46605137632686\\
18.3384027004242	-0.466167576156622\\
18.3875796318054	-0.466127159580914\\
18.4372620105743	-0.466507475965656\\
18.4870087623596	-0.466908385680348\\
18.5369572162628	-0.467669826319295\\
18.5870844841003	-0.468352013205255\\
18.6373524188995	-0.46882641157363\\
18.6875247478485	-0.469095126708041\\
18.7381965637207	-0.46919192302709\\
18.787005853653	-0.46911919419373\\
18.8369290351868	-0.469410559269114\\
18.8874282360077	-0.469714335403495\\
18.9384850978851	-0.470406263614692\\
18.9872798442841	-0.471077374520426\\
19.0383550643921	-0.471618358986326\\
19.0876242637634	-0.471979839059962\\
19.13878865242	-0.472099273729079\\
19.1871997833252	-0.471993426934141\\
19.238090467453	-0.472169515069531\\
19.2875425338745	-0.472374361346347\\
19.3374142169952	-0.473041925470078\\
19.3872191429138	-0.473675591480792\\
19.4372310161591	-0.4741791166716\\
19.4873599529266	-0.474504324583556\\
19.5374044895172	-0.474625428529521\\
19.5872218132019	-0.474605822335718\\
19.636870098114	-0.474896707503099\\
19.6870867729187	-0.475227073547487\\
19.7377478599548	-0.475901344905473\\
19.7868735313416	-0.476532960918426\\
19.8382687091827	-0.476978843580611\\
19.8874859333038	-0.477263164308084\\
19.9384080886841	-0.477355631020302\\
19.9873585224152	-0.477286692443066\\
20.0380374908447	-0.477498846444679\\
20.0872175216675	-0.477740614857698\\
20.1384467601776	-0.478328641942927\\
20.1871039390564	-0.478912298507144\\
20.2371427536011	-0.479394801272765\\
20.2878586769104	-0.479722711534066\\
20.3390125751495	-0.479825567464282\\
20.3873836517334	-0.479733498175619\\
20.4368681430817	-0.479829696450206\\
20.4869358062744	-0.479954991541781\\
20.5373003005981	-0.480512280980527\\
20.5874337673187	-0.481069676512952\\
20.6376835823059	-0.481523387474361\\
20.6868483543396	-0.481796695294104\\
20.7382573604584	-0.481789175186293\\
20.7872662067413	-0.481576957594626\\
20.8385295391083	-0.481626498166587\\
20.8871445178986	-0.481733248914736\\
20.9377784252167	-0.482289134172984\\
20.9873167991638	-0.482811216024967\\
21.0377876281738	-0.483187871252873\\
21.0882877826691	-0.48339099592096\\
21.1374673366547	-0.483356586086916\\
21.1871928691864	-0.483161173774217\\
21.2374176502228	-0.483273195594048\\
21.2877287387848	-0.483435965590878\\
21.3368806362152	-0.483963559552201\\
21.3871657371521	-0.484461016973196\\
21.4381284236908	-0.484769348042153\\
21.4872109413147	-0.484920619369511\\
21.5376049995422	-0.484922447083266\\
21.5874728679657	-0.484824661893684\\
21.6380285739899	-0.48502632877284\\
21.6870588779449	-0.485238508575229\\
21.739088010788	-0.485672661681008\\
21.7871963500977	-0.486014862828341\\
21.8373071670532	-0.486180799228777\\
21.8888287067413	-0.486239549134558\\
21.9381830215454	-0.486244944247726\\
21.9872478961945	-0.486192740316403\\
22.0378989696503	-0.486428326718343\\
22.0872938156128	-0.486665013252949\\
22.138090801239	-0.487047308367258\\
22.1868211746216	-0.487341395556912\\
22.2377860069275	-0.48749662371341\\
22.2874586105347	-0.487519973818358\\
22.3384885311127	-0.487540182475188\\
22.3870710849762	-0.487521377087148\\
22.4387132644653	-0.487782367687732\\
22.4868266105652	-0.488031042495678\\
22.5373594284058	-0.488363245674364\\
22.5873472213745	-0.488638981828759\\
22.6370512962341	-0.48877726907706\\
22.6884295463562	-0.48881748821799\\
22.7378239154816	-0.488819578992702\\
22.787751865387	-0.488785128865469\\
22.8374096870422	-0.48902575721389\\
22.8873886585236	-0.489286854722025\\
22.9379829883575	-0.489669118555295\\
22.9869961261749	-0.489996474491922\\
23.037150812149	-0.490169324749313\\
23.0874580860138	-0.490223048642422\\
23.1373772144318	-0.490228677014506\\
23.1871411323547	-0.490176664240335\\
23.237290096283	-0.490353356153383\\
23.2868723392487	-0.490540541857388\\
23.3371350288391	-0.490890052115671\\
23.3872100830078	-0.49119095711292\\
23.4371966838837	-0.491329411168318\\
23.4883241176605	-0.491358232559435\\
23.5368735313416	-0.491309443306324\\
23.5883466720581	-0.491186276240312\\
23.6373395442963	-0.491299347865229\\
23.6870235919952	-0.491432875159141\\
23.7381245613098	-0.49178798486577\\
23.7868911743164	-0.492071916513844\\
23.8377303600311	-0.492204389447298\\
23.8871185302734	-0.492195376328438\\
23.9379884719849	-0.492102097719735\\
23.9873661518097	-0.49193846419409\\
24.0375952243805	-0.492077967162359\\
24.0870673179626	-0.492257818303115\\
24.1371151924133	-0.492661079659156\\
24.1869523048401	-0.493042030660662\\
24.2370559692383	-0.493261653421562\\
24.2891807079315	-0.493360669143289\\
24.3380689144135	-0.493299863352929\\
24.3871647834778	-0.493108257188339\\
24.4377190589905	-0.493084007380427\\
24.4868864536285	-0.49307859233986\\
24.5379740715027	-0.493361065361001\\
24.5869614601135	-0.493641569040961\\
24.6376108646393	-0.49388895549188\\
24.6870998859406	-0.494096262411208\\
24.7377688407898	-0.494241490975773\\
24.7873253345489	-0.494361078295982\\
24.8372883319855	-0.494522636808654\\
24.8869270801544	-0.494625004475884\\
24.9384078502655	-0.494710789086011\\
24.9870955467224	-0.494726662227346\\
25.0383190631866	-0.494779361680863\\
25.0877341747284	-0.494802802143389\\
25.1370951652527	-0.494940797699621\\
25.1875030517578	-0.495070797040546\\
25.2380036830902	-0.495265118255956\\
25.2871031284332	-0.49542852783941\\
25.338099193573	-0.495530798422295\\
25.386954498291	-0.49554617314326\\
25.4379999160767	-0.495533138227633\\
25.4874267101288	-0.495468724958868\\
25.5373079299927	-0.495561664974162\\
25.5872232437134	-0.495667328024515\\
25.6382319450378	-0.495878033554451\\
25.6868111610413	-0.49605575090131\\
25.7375020503998	-0.496137140438585\\
25.7869216918945	-0.496139462754172\\
25.8372537612915	-0.496120508804615\\
25.8889283657074	-0.496075960616613\\
25.9383606433868	-0.496209641598889\\
25.9875232696533	-0.496370980554287\\
26.0369369506836	-0.496617000516769\\
26.0872263431549	-0.496840648165135\\
26.1380128383636	-0.496939132753896\\
26.1871096611023	-0.496957644831454\\
26.2377979278564	-0.496936716789409\\
26.2872809886932	-0.496916770971236\\
26.3384959220886	-0.497079135762099\\
26.387292098999	-0.497264411132111\\
26.4373895645142	-0.497490681932106\\
26.4868809700012	-0.497672178267376\\
26.53678150177	-0.497730339983604\\
26.5870432376862	-0.497742457372209\\
26.6371018409729	-0.49776634813527\\
26.6869751930237	-0.497784273449213\\
26.7378837585449	-0.497912498643234\\
26.7877766609192	-0.498037132486879\\
26.8378419399261	-0.498165793956921\\
26.8871218681335	-0.498267557864883\\
26.9376534938812	-0.49835429606942\\
26.987083864212	-0.498422080009706\\
27.0375470638275	-0.498500845659693\\
27.0869986534119	-0.498548756644621\\
27.137539100647	-0.498596031472214\\
27.1868445396423	-0.49861344270072\\
27.2373196601868	-0.498662919341361\\
27.286718082428	-0.498700695325799\\
27.3369435787201	-0.498793010208959\\
27.3872541904449	-0.498849230009139\\
27.4369372844696	-0.498875047424253\\
27.4883272171021	-0.498846806886877\\
27.5371679782867	-0.498818963285077\\
27.5876988887787	-0.498756624020716\\
27.6373111724854	-0.49879097556518\\
27.6871180057526	-0.498815478617701\\
27.7367758274078	-0.498879357847037\\
27.7870089530945	-0.498891454548516\\
27.8375215053558	-0.49884849086667\\
27.8871633529663	-0.498741768154879\\
27.940003824234	-0.498685760747647\\
27.9871308326721	-0.498612440320951\\
28.0378810882568	-0.498674915164145\\
28.0873820304871	-0.49872836086001\\
28.1369366168976	-0.498812373558936\\
28.1869079589844	-0.498852478363664\\
28.2368185043335	-0.498851624590337\\
28.2877447128296	-0.498795988183972\\
28.3373779773712	-0.498781027025557\\
28.3870558261871	-0.498749581117871\\
28.4374107837677	-0.498807347624113\\
28.4871124744415	-0.498846403832005\\
28.5371062278748	-0.498903811439231\\
28.5868112564087	-0.498926868186661\\
28.6371838569641	-0.498948878918412\\
28.6869341850281	-0.498952826352189\\
28.7380611419678	-0.499006440447575\\
28.7870323181152	-0.499047835762443\\
28.8377327442169	-0.499121381035917\\
28.8874563694	-0.499167987977725\\
28.9368562221527	-0.499197770739337\\
28.9870872020721	-0.499191633262535\\
29.0385212421417	-0.499216073804198\\
29.0873689174652	-0.499241457277833\\
29.1368615150452	-0.499327774147361\\
29.187535238266	-0.499402349386604\\
29.2372958183289	-0.499458275123899\\
29.2871133804321	-0.499477539805459\\
29.3380169391632	-0.499491953850458\\
29.3868567466736	-0.499508294827027\\
29.4373459339142	-0.499592159616931\\
29.4873344421387	-0.499673517969036\\
29.5384788036346	-0.499745233562842\\
29.5869259357452	-0.499784733149944\\
29.6376266002655	-0.499798393698614\\
29.6871110916138	-0.499808510694326\\
29.7390245914459	-0.499866768613662\\
29.7868015289307	-0.499924685179379\\
29.8379787921906	-0.499995751514704\\
29.8871769428253	-0.500041060246605\\
29.9371747493744	-0.500057080541828\\
29.987563085556	-0.500056887902329\\
30.0383028507233	-0.500097714834717\\
30.0876338005066	-0.500138190182736\\
30.1368801116943	-0.500210671004544\\
30.1873883724213	-0.500266744301591\\
30.2370848178864	-0.500294807691762\\
30.2871500968933	-0.500294998019356\\
30.3385204792023	-0.500305673092974\\
30.386811208725	-0.500315565526886\\
30.4384216785431	-0.500372668506539\\
30.4868461608887	-0.500424934252458\\
30.5371896743774	-0.500478069839368\\
30.5867259025574	-0.500517191317411\\
30.6380993843079	-0.500541513583937\\
30.6876494407654	-0.500545170884513\\
30.7367903709412	-0.500564784274413\\
30.7870404243469	-0.50057613475137\\
30.8369943618774	-0.50061779146622\\
30.8871693134308	-0.500647624238253\\
30.9381038665771	-0.500678787135907\\
30.986993265152	-0.500687143032461\\
31.0379008769989	-0.500690632120347\\
31.0870007991791	-0.500677764764876\\
31.1389050006866	-0.500702157479178\\
31.187396478653	-0.500726790963348\\
31.2372066497803	-0.500770314038248\\
31.2869581699371	-0.500787884247257\\
31.3370429992676	-0.5007751024884\\
31.3868891716003	-0.50073290490026\\
31.4371098995209	-0.500718090546115\\
31.488369178772	-0.500704948210281\\
31.5368852138519	-0.50074192056332\\
31.5877389431	-0.500760747952582\\
31.6370863437653	-0.500768738822767\\
31.6871823787689	-0.500750254739327\\
31.738449048996	-0.500732061469706\\
31.7871157646179	-0.500707433791729\\
31.8383152008057	-0.500733729002572\\
31.8873555183411	-0.50075473801991\\
31.9394099235535	-0.500778725710945\\
31.9871487140656	-0.500771873852364\\
32.0381509780884	-0.500739640836398\\
32.088035774231	-0.500683385605091\\
32.1374802112579	-0.500675213938454\\
32.187110376358	-0.500668583727146\\
32.2387628078461	-0.500710653375999\\
32.2874614715576	-0.500740232400411\\
32.3373488903046	-0.500747461505731\\
32.3874296665192	-0.500724190491122\\
32.4374224662781	-0.500689349301917\\
32.4878580093384	-0.500642880094494\\
32.5376955986023	-0.500646703770389\\
32.5874928951263	-0.500650962733877\\
32.6386348724365	-0.500687769895437\\
32.6873406887054	-0.50070623769129\\
32.7388579368591	-0.500695179762706\\
32.7873272418976	-0.500652361099567\\
32.8374335289001	-0.500610882754088\\
32.8873004436493	-0.500563188386037\\
32.937156867981	-0.50057884410507\\
32.9873573303223	-0.500599133984962\\
33.0380663394928	-0.500638731938649\\
33.0887851238251	-0.500661714493921\\
33.1397532939911	-0.50065073529028\\
33.1896943569183	-0.500610399538459\\
33.2378904342651	-0.500579862148701\\
33.2873873233795	-0.500551481219228\\
33.337309551239	-0.500582421705266\\
33.3870064735413	-0.500611915661932\\
33.4376117706299	-0.500640942997497\\
33.4877395153046	-0.500650845375385\\
33.5375718593597	-0.50063654897944\\
33.5879704475403	-0.500610971008847\\
33.6378068447113	-0.500612539775088\\
33.6874708652496	-0.500614815401552\\
33.7375998020172	-0.500646260977676\\
33.7882034301758	-0.500663201874414\\
33.8377136707306	-0.500655190980199\\
33.8892471313477	-0.500623162594798\\
33.9370285987854	-0.500600400964218\\
33.9873494625092	-0.500579833400609\\
34.0370041847229	-0.500608950915256\\
34.0873434066772	-0.500635807755218\\
34.1362752437592	-0.500635807755218\\
34.1862122535706	-0.500635807755218\\
34.2362687110901	-0.500635807755218\\
34.2865228176117	-0.500635807755218\\
34.3363995075226	-0.500635807755218\\
34.3862075328827	-0.500635807755218\\
34.4361023426056	-0.500635807755218\\
34.4861790657043	-0.500635807755218\\
34.5361194133759	-0.500635807755218\\
34.586292219162	-0.500635807755218\\
34.6364039897919	-0.500635807755218\\
34.6863061904907	-0.500635807755218\\
34.7361094474793	-0.500635807755218\\
34.7864546298981	-0.500635807755218\\
34.83685297966	-0.500635807755218\\
34.8863391399384	-0.500635807755218\\
34.9360925674439	-0.500635807755218\\
34.986105632782	-0.500635807755218\\
35.0454177379608	-0.500635807755218\\
35.0912179470062	-0.500635807755218\\
35.136381816864	-0.500635807755218\\
35.186456155777	-0.500635807755218\\
35.2363311767578	-0.500635807755218\\
35.2861754417419	-0.500635807755218\\
35.3360485553741	-0.500635807755218\\
35.3861937046051	-0.500635807755218\\
35.4363774776459	-0.500635807755218\\
35.486319732666	-0.500635807755218\\
35.536076259613	-0.500635807755218\\
35.5862481117249	-0.500635807755218\\
35.6363558292389	-0.500635807755218\\
35.6861931800842	-0.500635807755218\\
35.7363302230835	-0.500635807755218\\
35.786288690567	-0.500635807755218\\
35.8361827850342	-0.500635807755218\\
35.8864717006683	-0.500635807755218\\
35.9363789081574	-0.500635807755218\\
35.9867715358734	-0.500635807755218\\
36.0360667228699	-0.500635807755218\\
36.0863480091095	-0.500635807755218\\
36.1363653659821	-0.500635807755218\\
36.1864337444305	-0.500635807755218\\
36.2360584259033	-0.500635807755218\\
36.2864779949188	-0.500635807755218\\
36.336106967926	-0.500635807755218\\
36.3864702701569	-0.500635807755218\\
36.4362844944	-0.500635807755218\\
36.4863364219666	-0.500635807755218\\
36.5363025188446	-0.500635807755218\\
36.5861410617828	-0.500635807755218\\
36.6363481998444	-0.500635807755218\\
36.6863061904907	-0.500635807755218\\
36.7361874103546	-0.500635807755218\\
36.7863311290741	-0.500635807755218\\
36.8363565921783	-0.500635807755218\\
36.8863608360291	-0.500635807755218\\
36.9363264560699	-0.500635807755218\\
36.9860896587372	-0.500635807755218\\
37.0362286090851	-0.500635807755218\\
37.0864164352417	-0.500635807755218\\
37.1363679885864	-0.500635807755218\\
37.1863879680634	-0.500635807755218\\
37.2362648963928	-0.500635807755218\\
37.2863585472107	-0.500635807755218\\
37.3360833644867	-0.500635807755218\\
37.3862823963165	-0.500635807755218\\
37.4368352413178	-0.500635807755218\\
37.4861456871033	-0.500635807755218\\
37.5361048698425	-0.500635807755218\\
37.5862543106079	-0.500635807755218\\
37.6362061023712	-0.500635807755218\\
37.6863986968994	-0.500635807755218\\
37.736749124527	-0.500635807755218\\
37.7863394737244	-0.500635807755218\\
37.8361865997314	-0.500635807755218\\
37.888374042511	-0.500635807755218\\
37.9363502979279	-0.500635807755218\\
37.9863593101501	-0.500635807755218\\
38.0362567424774	-0.500635807755218\\
38.0862390518189	-0.500635807755218\\
38.1364056587219	-0.500635807755218\\
38.1864752292633	-0.500635807755218\\
38.2363790988922	-0.500635807755218\\
38.2863652229309	-0.500635807755218\\
38.3364450454712	-0.500635807755218\\
38.3863930225372	-0.500635807755218\\
38.4377541065216	-0.500635807755218\\
38.4863030433655	-0.500635807755218\\
38.5364169597626	-0.500635807755218\\
38.5863399028778	-0.500635807755218\\
38.6360737800598	-0.500635807755218\\
38.686456155777	-0.500635807755218\\
38.7360977649689	-0.500635807755218\\
38.7863506793976	-0.500635807755218\\
38.8363575458527	-0.500635807755218\\
38.8868772506714	-0.500635807755218\\
38.9363157272339	-0.500635807755218\\
38.9863695621491	-0.500635807755218\\
39.036302280426	-0.500635807755218\\
39.0863775730133	-0.500635807755218\\
39.1364001750946	-0.500635807755218\\
39.1863421916962	-0.500635807755218\\
39.2363414287567	-0.500635807755218\\
39.2865790843964	-0.500635807755218\\
39.3364316940308	-0.500635807755218\\
39.3863877773285	-0.500635807755218\\
39.4363846302032	-0.500635807755218\\
39.4864034175873	-0.500635807755218\\
39.5362460136414	-0.500635807755218\\
39.5864672183991	-0.500635807755218\\
39.6376175403595	-0.500635807755218\\
39.6865121841431	-0.500635807755218\\
39.7360577106476	-0.500635807755218\\
39.7860879421234	-0.500635807755218\\
39.8361377239227	-0.500635807755218\\
39.8862397193909	-0.500635807755218\\
39.9363758087158	-0.500635807755218\\
39.9866580486298	-0.500635807755218\\
40.0361229896545	-0.500635807755218\\
40.0863337039948	-0.500635807755218\\
40.1362952709198	-0.500635807755218\\
40.1863722324371	-0.500635807755218\\
40.2360808372498	-0.500635807755218\\
40.2867063999176	-0.500635807755218\\
40.3362371444702	-0.500635807755218\\
40.3864602565765	-0.500635807755218\\
40.4362022399902	-0.500635807755218\\
40.4862391471863	-0.500635807755218\\
40.536110830307	-0.500635807755218\\
40.586754989624	-0.500635807755218\\
40.6363067150116	-0.500635807755218\\
40.6862320423126	-0.500635807755218\\
40.7363345146179	-0.500635807755218\\
40.7863447189331	-0.500635807755218\\
40.8362771987915	-0.500635807755218\\
40.886261177063	-0.500635807755218\\
40.936424446106	-0.500635807755218\\
40.9864334583282	-0.500635807755218\\
41.0363933563232	-0.500635807755218\\
41.0863849639893	-0.500635807755218\\
41.1362373352051	-0.500635807755218\\
41.1860804080963	-0.500635807755218\\
41.2361869335175	-0.500635807755218\\
41.2864078998566	-0.500635807755218\\
41.3360969543457	-0.500635807755218\\
41.3861855983734	-0.500635807755218\\
41.4376759052277	-0.500635807755218\\
41.486256313324	-0.500635807755218\\
41.536371421814	-0.500635807755218\\
41.5862829208374	-0.500635807755218\\
41.636062335968	-0.500635807755218\\
41.686300945282	-0.500635807755218\\
41.7360789299011	-0.500635807755218\\
41.7862710475922	-0.500635807755218\\
41.8360845565796	-0.500635807755218\\
41.8863353252411	-0.500635807755218\\
41.9362205982208	-0.500635807755218\\
41.986283493042	-0.500635807755218\\
42.0364117145538	-0.500635807755218\\
42.0863775730133	-0.500635807755218\\
42.1363970756531	-0.500635807755218\\
42.186447095871	-0.500635807755218\\
42.2362341403961	-0.500635807755218\\
42.2862476825714	-0.500635807755218\\
42.3368155479431	-0.500635807755218\\
42.3862549781799	-0.500635807755218\\
42.4363796234131	-0.500635807755218\\
42.4863807678223	-0.500635807755218\\
42.5361065387726	-0.500635807755218\\
42.586087179184	-0.500635807755218\\
42.6363203048706	-0.500635807755218\\
42.6863209724426	-0.500635807755218\\
42.7362560749054	-0.500635807755218\\
42.7863132476807	-0.500635807755218\\
42.8362674236298	-0.500635807755218\\
42.8862435340881	-0.500635807755218\\
42.9362756729126	-0.500635807755218\\
42.9860562801361	-0.500635807755218\\
43.0366630077362	-0.500635807755218\\
43.0862891197205	-0.500635807755218\\
43.1369411468506	-0.500635807755218\\
43.1862763881683	-0.500635807755218\\
43.2363652706146	-0.500635807755218\\
43.2866267681122	-0.500635807755218\\
43.3361598968506	-0.500635807755218\\
43.3862552165985	-0.500635807755218\\
43.4363893985748	-0.500635807755218\\
43.4867214679718	-0.500635807755218\\
43.5363444805145	-0.500635807755218\\
43.5866703510284	-0.500635807755218\\
43.6362726211548	-0.500635807755218\\
43.6863271713257	-0.500635807755218\\
43.7361363887787	-0.500635807755218\\
43.7873098373413	-0.500635807755218\\
43.8363158226013	-0.500635807755218\\
43.8864092350006	-0.500635807755218\\
43.9364137172699	-0.500635807755218\\
43.9862827777863	-0.500635807755218\\
44.0369226455689	-0.500635807755218\\
44.0863043785095	-0.500635807755218\\
44.1360961914063	-0.500635807755218\\
44.1862127304077	-0.500635807755218\\
44.236283493042	-0.500635807755218\\
44.2865638256073	-0.500635807755218\\
44.3360812187195	-0.500635807755218\\
44.386487197876	-0.500635807755218\\
44.4361028194428	-0.500635807755218\\
44.4864933013916	-0.500635807755218\\
44.536284160614	-0.500635807755218\\
44.5863046169281	-0.500635807755218\\
44.6362940788269	-0.500635807755218\\
44.6862978458405	-0.500635807755218\\
44.7364215373993	-0.500635807755218\\
44.7863480567932	-0.500635807755218\\
44.8362671852112	-0.500635807755218\\
44.8862499713898	-0.500635807755218\\
44.9360589504242	-0.500635807755218\\
44.9889430522919	-0.500635807755218\\
45.0363490104675	-0.500635807755218\\
45.0864402770996	-0.500635807755218\\
45.1361271858215	-0.500635807755218\\
45.1861283302307	-0.500635807755218\\
45.2363178253174	-0.500635807755218\\
45.2860579013824	-0.500635807755218\\
45.3360833644867	-0.500635807755218\\
45.3861248016357	-0.500635807755218\\
45.4362778186798	-0.500635807755218\\
45.4862968444824	-0.500635807755218\\
45.5364520072937	-0.500635807755218\\
45.5863546848297	-0.500635807755218\\
45.6363763332367	-0.500635807755218\\
45.6862480163574	-0.500635807755218\\
45.7361382961273	-0.500635807755218\\
45.7863018035889	-0.500635807755218\\
45.8363010406494	-0.500635807755218\\
45.8863670349121	-0.500635807755218\\
45.9360751628876	-0.500635807755218\\
45.9862565517426	-0.500635807755218\\
46.036564540863	-0.500635807755218\\
46.086417388916	-0.500635807755218\\
46.136324596405	-0.500635807755218\\
46.1863011837006	-0.500635807755218\\
46.236340713501	-0.500635807755218\\
46.286172580719	-0.500635807755218\\
46.3363940238953	-0.500635807755218\\
46.3862518787384	-0.500635807755218\\
46.4363107204437	-0.500635807755218\\
46.4862446308136	-0.500635807755218\\
46.5362638950348	-0.500635807755218\\
46.5863682746887	-0.500635807755218\\
46.6360892772675	-0.500635807755218\\
46.6863905906677	-0.500635807755218\\
46.7362088680267	-0.500635807755218\\
46.7864021778107	-0.500635807755218\\
46.8362369060516	-0.500635807755218\\
46.8863255500793	-0.500635807755218\\
46.9363812923431	-0.500635807755218\\
46.9860479354858	-0.500635807755218\\
47.0364589214325	-0.500635807755218\\
47.086301279068	-0.500635807755218\\
47.1364011287689	-0.500635807755218\\
47.1863252639771	-0.500635807755218\\
47.2360765457153	-0.500635807755218\\
47.2863361358643	-0.500635807755218\\
47.336061668396	-0.500635807755218\\
47.3863124370575	-0.500635807755218\\
47.436296415329	-0.500635807755218\\
47.4863233089447	-0.500635807755218\\
47.5360545635223	-0.500635807755218\\
47.5861608505249	-0.500635807755218\\
47.6362637996674	-0.500635807755218\\
47.6860692024231	-0.500635807755218\\
47.7363540649414	-0.500635807755218\\
47.7863378047943	-0.500635807755218\\
47.8363644599915	-0.500635807755218\\
47.8862695217133	-0.500635807755218\\
47.9362542152405	-0.500635807755218\\
47.9863764762878	-0.500635807755218\\
48.0360953330994	-0.500635807755218\\
48.0863441944122	-0.500635807755218\\
48.1360756874085	-0.500635807755218\\
48.186307144165	-0.500635807755218\\
48.2360848903656	-0.500635807755218\\
48.2862736701965	-0.500635807755218\\
48.3363034248352	-0.500635807755218\\
48.3863589286804	-0.500635807755218\\
48.4373104095459	-0.500635807755218\\
48.4862935066223	-0.500635807755218\\
48.5363015651703	-0.500635807755218\\
48.5862574100494	-0.500635807755218\\
48.6363148212433	-0.500635807755218\\
48.6862716197968	-0.500635807755218\\
48.7360643863678	-0.500635807755218\\
48.7863087177277	-0.500635807755218\\
48.8364052295685	-0.500635807755218\\
48.8860489845276	-0.500635807755218\\
48.9366774082184	-0.500635807755218\\
48.9863929271698	-0.500635807755218\\
49.0368101119995	-0.500635807755218\\
49.0863396644592	-0.500635807755218\\
49.1362618923187	-0.500635807755218\\
49.1863467216492	-0.500635807755218\\
49.2363132953644	-0.500635807755218\\
49.2862998962402	-0.500635807755218\\
49.3360840797424	-0.500635807755218\\
49.3862306594849	-0.500635807755218\\
49.4362706661224	-0.500635807755218\\
49.486224603653	-0.500635807755218\\
49.5363234996796	-0.500635807755218\\
49.5862879276276	-0.500635807755218\\
49.6361152648926	-0.500635807755218\\
49.6863047599793	-0.500635807755218\\
49.7360612869263	-0.500635807755218\\
49.7862214565277	-0.500635807755218\\
49.8363184452057	-0.500635807755218\\
49.8879334449768	-0.500635807755218\\
49.9363319396973	-0.500635807755218\\
49.9862370014191	-0.500635807755218\\
50.0362956047058	-0.500635807755218\\
50.0868522644043	-0.500635807755218\\
50.1361271858215	-0.500635807755218\\
50.1862706661224	-0.500635807755218\\
50.2362811088562	-0.500635807755218\\
50.2863530635834	-0.500635807755218\\
50.3364486217499	-0.500635807755218\\
50.3862692832947	-0.500635807755218\\
50.4361993789673	-0.500635807755218\\
50.4862770557404	-0.500635807755218\\
50.5360641002655	-0.500635807755218\\
50.5860704898834	-0.500635807755218\\
50.636267375946	-0.500635807755218\\
50.6864208698273	-0.500635807755218\\
50.7364587306976	-0.500635807755218\\
50.786306810379	-0.500635807755218\\
50.8363444328308	-0.500635807755218\\
50.8863732337952	-0.500635807755218\\
50.9360637187958	-0.500635807755218\\
50.9860643863678	-0.500635807755218\\
51.0364918231964	-0.500635807755218\\
51.086265039444	-0.500635807755218\\
51.1363150596619	-0.500635807755218\\
51.1863760471344	-0.500635807755218\\
51.2363709926605	-0.500635807755218\\
51.2863130092621	-0.500635807755218\\
51.3363859176636	-0.500635807755218\\
51.3862800121307	-0.500635807755218\\
51.4363278865814	-0.500635807755218\\
51.4863607406616	-0.500635807755218\\
51.5360641002655	-0.500635807755218\\
51.5886544704437	-0.500635807755218\\
51.6360978603363	-0.500635807755218\\
51.688673210144	-0.500635807755218\\
51.7367107391357	-0.500635807755218\\
51.7860705375671	-0.500635807755218\\
51.8363744735718	-0.500635807755218\\
51.8862721443176	-0.500635807755218\\
51.936353635788	-0.500635807755218\\
51.9860617637634	-0.500635807755218\\
52.0368484973907	-0.500635807755218\\
52.0862779140472	-0.500635807755218\\
52.1360904693604	-0.500635807755218\\
52.1863877296448	-0.500635807755218\\
52.2361249446869	-0.500635807755218\\
52.2863645076752	-0.500635807755218\\
52.3361343860626	-0.500635807755218\\
52.3863236427307	-0.500635807755218\\
52.4363097667694	-0.500635807755218\\
52.4860250473022	-0.500635807755218\\
52.5360769748688	-0.500635807755218\\
52.5860502243042	-0.500635807755218\\
52.6363713264465	-0.500635807755218\\
52.690006685257	-0.500635807755218\\
52.7364186763763	-0.500635807755218\\
52.7863256454468	-0.500635807755218\\
52.8360683441162	-0.500635807755218\\
52.8861855983734	-0.500635807755218\\
52.9360858917236	-0.500635807755218\\
52.9860407829285	-0.500635807755218\\
53.0362543582916	-0.500635807755218\\
53.0863141536713	-0.500635807755218\\
53.1362888336182	-0.500635807755218\\
53.1872043132782	-0.500635807755218\\
53.2363035202026	-0.500635807755218\\
53.2860457420349	-0.500635807755218\\
53.3363325119019	-0.500635807755218\\
53.3861095428467	-0.500635807755218\\
53.436364364624	-0.500635807755218\\
53.4863154411316	-0.500635807755218\\
53.5363397121429	-0.500635807755218\\
53.5863167762756	-0.500635807755218\\
53.6361658096314	-0.500635807755218\\
53.6863488674164	-0.500635807755218\\
53.7363729000092	-0.500635807755218\\
53.7860550403595	-0.500635807755218\\
53.836292219162	-0.500635807755218\\
53.8862983703613	-0.500635807755218\\
53.9362475395203	-0.500635807755218\\
53.9861213684082	-0.500635807755218\\
54.0362629413605	-0.500635807755218\\
54.0862345218658	-0.500635807755218\\
54.1360654354095	-0.500635807755218\\
54.1862487316132	-0.500635807755218\\
54.2360968112946	-0.500635807755218\\
54.2862512588501	-0.500635807755218\\
54.3363804340363	-0.500635807755218\\
54.3863088607788	-0.500635807755218\\
54.4362613677979	-0.500635807755218\\
54.4864789962769	-0.500635807755218\\
54.5363761901855	-0.500635807755218\\
54.5860692977905	-0.500635807755218\\
54.6361002445221	-0.500635807755218\\
54.6862933158875	-0.500635807755218\\
54.7361409187317	-0.500635807755218\\
54.7861282348633	-0.500635807755218\\
54.8360788345337	-0.500635807755218\\
54.8863553524017	-0.500635807755218\\
54.9360541820526	-0.500635807755218\\
54.9860584259033	-0.500635807755218\\
55.0363721370697	-0.500635807755218\\
55.0868153095245	-0.500635807755218\\
55.136411857605	-0.500635807755218\\
55.1862184524536	-0.500635807755218\\
55.2393674373627	-0.500635807755218\\
55.2863261222839	-0.500635807755218\\
55.33689661026	-0.500635807755218\\
55.3863090991974	-0.500635807755218\\
55.4367873191834	-0.500635807755218\\
55.4863738536835	-0.500635807755218\\
55.5360693454742	-0.500635807755218\\
55.586337518692	-0.500635807755218\\
55.6370651245117	-0.500635807755218\\
55.6863228797913	-0.500635807755218\\
55.7360960960388	-0.500635807755218\\
55.7863347053528	-0.500635807755218\\
};
\addlegendentry{$\theta_{y_{p}}$}

\addplot [color=mycolor5, line width=1.4pt]
  table[row sep=crcr]{%
-3.41395025253296	0\\
-3.36380581855774	0\\
-3.31373219490051	0\\
-3.26373772621155	0\\
-3.21386485099792	0\\
-3.16371612548828	0\\
-3.11396126747131	0\\
-3.0639102935791	0\\
-3.01393465995789	0\\
-2.96376042366028	0\\
-2.91368894577026	0\\
-2.86386590003967	0\\
-2.81392412185669	0\\
-2.76383023262024	0\\
-2.71365766525269	0\\
-2.66386275291443	0\\
-2.61371665000916	0\\
-2.56373028755188	0\\
-2.51375107765198	0\\
-2.46365218162537	0\\
-2.41376595497131	0\\
-2.36374454498291	0\\
-2.31372838020325	0\\
-2.26391820907593	0\\
-2.21392397880554	0\\
-2.16394262313843	0\\
-2.11378173828125	0\\
-2.06323270797729	0\\
-2.01386647224426	0\\
-1.96378617286682	0\\
-1.91375093460083	0\\
-1.86376647949219	0\\
-1.81371264457703	0\\
-1.76388936042786	0\\
-1.71374897956848	0\\
-1.66370325088501	0\\
-1.61389498710632	0\\
-1.56357769966125	0\\
-1.51373534202576	0\\
-1.46395378112793	0\\
-1.41391592025757	0\\
-1.36373047828674	0\\
-1.31355028152466	0\\
-1.26392798423767	0\\
-1.21389083862305	0\\
-1.1639223575592	0\\
-1.11384873390198	0\\
-1.06395726203918	0\\
-1.01393632888794	0\\
-0.96373085975647	0\\
-0.913796234130859	0\\
-0.863754081726074	0\\
-0.81366925239563	0\\
-0.763712215423584	0\\
-0.71373085975647	0\\
-0.663900899887085	0\\
-0.61391978263855	0\\
-0.563617038726806	0\\
-0.51369457244873	0\\
-0.46377067565918	0\\
-0.413990783691406	0\\
-0.363846826553345	0\\
-0.313869285583496	0\\
-0.263829755783081	0\\
-0.213733005523681	0\\
-0.163687038421631	0\\
-0.113701391220093	0\\
-0.0636923789978026	0\\
-0.0136535644531248	0\\
0.0374466896057131	3.47260465528922\\
0.0868460655212404	4.8880385866841\\
0.137080144882202	5.67595078875331\\
0.187013101577759	6.19599635019279\\
0.237029504776001	6.57428952691873\\
0.288115215301514	6.86206232111817\\
0.336934280395508	7.08526311727701\\
0.387251806259155	7.25925725260663\\
0.436960172653198	7.39652163971914\\
0.487322521209717	7.50867034281055\\
0.537163209915161	7.60211021445548\\
0.587262582778931	7.68781187546779\\
0.638643217086792	7.76857779466036\\
0.687006664276123	7.84421877500927\\
0.737519216537476	7.91502163651694\\
0.787285280227661	7.97922531274253\\
0.838375759124756	8.03326164868076\\
0.887129497528076	8.08074684961002\\
0.937354993820191	8.12279408220638\\
0.98739595413208	8.16411936605618\\
1.03824419975281	8.20668200928458\\
1.08832807540894	8.25048383440708\\
1.13733215332031	8.2942580777235\\
1.18734998703003	8.33641190427034\\
1.23741598129272	8.37374009437781\\
1.28700251579285	8.40775193324362\\
1.33715624809265	8.43867238141411\\
1.3871440410614	8.46920537890855\\
1.43721837997437	8.50112690477454\\
1.48703689575195	8.53440281111398\\
1.53733153343201	8.56913337872538\\
1.58698172569275	8.6041552821689\\
1.6369044303894	8.63703034050013\\
1.68689270019531	8.6684150571009\\
1.73750514984131	8.69774302926089\\
1.78736562728882	8.72636343487284\\
1.83805890083313	8.7553553421285\\
1.8878821849823	8.78619517645438\\
1.93722839355469	8.81885577645789\\
1.9872805595398	8.85258374435307\\
2.03788895606995	8.88482536130141\\
2.08703417778015	8.91602891642651\\
2.13730902671814	8.94576749143198\\
2.18721742630005	8.97394681620517\\
2.2393238067627	9.00215114027742\\
2.28719348907471	9.03160080787802\\
2.33826942443848	9.06476032951468\\
2.38713521957397	9.10029514945563\\
2.43748707771301	9.13651918874166\\
2.487189245224	9.17189687445898\\
2.53701467514038	9.20350081827564\\
2.58730955123901	9.23333332883794\\
2.6372718334198	9.26262560280793\\
2.68767471313477	9.29362405412303\\
2.73776693344116	9.3285792958477\\
2.78724522590637	9.36556923771695\\
2.83687014579773	9.40400345706621\\
2.88723630905151	9.44133932399563\\
2.93726320266724	9.47603569689545\\
2.98756213188171	9.50902488415431\\
3.03776831626892	9.54117874989424\\
3.08719940185547	9.57513086832205\\
3.13819379806519	9.61192769852005\\
3.1871039390564	9.65064201071345\\
3.2383677482605	9.6897366198391\\
3.28719182014465	9.72701735586452\\
3.33709425926209	9.7624107813881\\
3.3875627040863	9.79666741960864\\
3.43703336715698	9.83109037608574\\
3.48778195381165	9.86702137482189\\
3.53756542205811	9.90520025399292\\
3.58713026046753	9.94334476452514\\
3.63820476531982	9.98147947838061\\
3.68715305328369	10.0182638874312\\
3.7370192527771	10.053407553939\\
3.78696150779724	10.0872153437722\\
3.83697123527527	10.1206527284439\\
3.88720889091492	10.1549324749822\\
3.93718166351318	10.1901709903941\\
3.98680086135864	10.2249076523685\\
4.0373019695282	10.2594017027932\\
4.087202501297	10.2933772447623\\
4.13750429153442	10.3263004745941\\
4.18730540275574	10.3585736336199\\
4.23692674636841	10.3902940649868\\
4.28764815330505	10.4214774527218\\
4.33715195655823	10.4530571013065\\
4.38732810020447	10.4840562775653\\
4.43775720596313	10.515149079999\\
4.48715395927429	10.5461675283204\\
4.53803582191467	10.5769027369197\\
4.58724708557129	10.6067197802358\\
4.63719172477722	10.6358614839201\\
4.68718476295471	10.663890519987\\
4.73849172592163	10.6917402176969\\
4.78692073822022	10.7192252087298\\
4.83752150535584	10.7472877750479\\
4.88743896484375	10.7756425800426\\
4.93707461357117	10.8039967644281\\
4.98722429275513	10.8312511214508\\
5.03730530738831	10.8574403202892\\
5.08796877861023	10.8824236869432\\
5.13730926513672	10.9072768161268\\
5.1875111579895	10.9317878094325\\
5.2383083820343	10.9565830664151\\
5.28709311485291	10.9812358753034\\
5.33733768463135	11.0055488527978\\
5.38735957145691	11.0285246949315\\
5.43840856552124	11.050338843309\\
5.4869279384613	11.0711905738308\\
5.53821368217468	11.0920317704031\\
5.58721609115601	11.112522899125\\
5.63834400177002	11.1334124891091\\
5.70256943702698	11.1542662258453\\
5.73696870803833	11.1748881203885\\
5.78678030967712	11.1944386180858\\
5.83695526123047	11.2129726901112\\
5.88792462348938	11.2306242662617\\
5.93726081848145	11.2480861969507\\
5.98705382347107	11.2653762516966\\
6.037406873703	11.2830089080198\\
6.08693976402283	11.3006353311303\\
6.13725109100342	11.3179131206489\\
6.1868757724762	11.3341229637062\\
6.23866934776306	11.3493208719237\\
6.28684301376343	11.3636764156236\\
6.3376051902771	11.377888072313\\
6.38692278862	11.3920477976681\\
6.43746538162231	11.4065379862423\\
6.48725814819336	11.4210210382826\\
6.53829855918884	11.4351747563373\\
6.58715434074402	11.4485070108876\\
6.63686246871948	11.4609568390911\\
6.68743295669556	11.4727290804231\\
6.73686356544495	11.4842220338855\\
6.78764243125916	11.4956441750423\\
6.83729786872864	11.5072716893001\\
6.8867949962616	11.5189711623266\\
6.93705792427063	11.5306747264463\\
6.98714203834534	11.5417167823202\\
7.03722019195557	11.552014344923\\
7.0867495059967	11.5617328675335\\
7.13819832801819	11.5711879644964\\
7.18691987991333	11.5805274251602\\
7.23969264030457	11.5899850987553\\
7.28743596076965	11.5995488947547\\
7.33690996170044	11.6090098326345\\
7.38727087974548	11.6180212763279\\
7.43710012435913	11.6263667459039\\
7.48817820549011	11.6341280658489\\
7.5378050327301	11.6414582226303\\
7.58700842857361	11.6487325637409\\
7.63773460388184	11.6561422315435\\
7.68709034919739	11.6637675249713\\
7.7376603603363	11.6714199780426\\
7.78713698387146	11.6786724761496\\
7.83752889633179	11.6853016732448\\
7.88738055229187	11.6913858363714\\
7.93729419708252	11.6970727334469\\
7.98730368614197	11.702761641327\\
8.03704233169556	11.7086515115807\\
8.08665652275085	11.7148270925509\\
8.13837356567383	11.7210593517229\\
8.18696184158325	11.7269406443193\\
8.23794646263122	11.7321807612639\\
8.28800339698791	11.7368985581384\\
8.33725209236145	11.7413355153249\\
8.38739914894104	11.7458533277302\\
8.43756384849548	11.7505923353401\\
8.48700923919678	11.7555818577166\\
8.53716988563538	11.7605772081504\\
8.5874659538269	11.7652554618028\\
8.63724226951599	11.7693802956553\\
8.68716926574707	11.7731511037302\\
8.73704977035522	11.7766987462358\\
8.78744978904724	11.7803061834188\\
8.83704800605774	11.7842011566831\\
8.88674945831299	11.7883720248556\\
8.93740029335022	11.7924895121027\\
8.98759908676147	11.7962946987125\\
9.03756065368652	11.7995572503742\\
9.08785433769226	11.8025994275888\\
9.13731617927551	11.8054550329236\\
9.18759603500366	11.8084838674758\\
9.23747487068176	11.8116591645055\\
9.28715224266052	11.8149808844996\\
9.33708448410034	11.8182733301023\\
9.38704843521118	11.8213333644464\\
9.43731732368469	11.8239368295144\\
9.48708744049072	11.826380195954\\
9.5378701210022	11.8286395777382\\
9.58680291175842	11.8310614841448\\
9.63790078163147	11.8335508032719\\
9.68698759078979	11.8361699134534\\
9.73723216056824	11.8387690569102\\
9.78691930770874	11.841239049862\\
9.83674449920654	11.8433794584523\\
9.8874725818634	11.8454513737165\\
9.93801159858704	11.8473395236758\\
9.98698897361755	11.8493169397645\\
10.0372972011566	11.8512838124352\\
10.0883295059204	11.8534270376001\\
10.1372539520264	11.8555762020469\\
10.1879517555237	11.8576498986484\\
10.2371825695038	11.8594374927909\\
10.287105512619	11.8611694812756\\
10.3395609378815	11.8626478091352\\
10.387139749527	11.8641889101991\\
10.4386827468872	11.8656949558639\\
10.4871279716492	11.867401823437\\
10.537340593338	11.8691137605803\\
10.5872203826904	11.8708017593443\\
10.6368417263031	11.8722321080577\\
10.6872877597809	11.873622542262\\
10.7376646518707	11.8747851455678\\
10.7870134830475	11.8760117257996\\
10.8377515792847	11.8771948573393\\
10.8868004798889	11.8785815356141\\
10.9370796203613	11.8799750949405\\
10.9869966030121	11.88135015655\\
11.0377678394318	11.8824511523178\\
11.0870306015015	11.883480182088\\
11.1382483959198	11.8842814513606\\
11.1873485565186	11.8851391406367\\
11.2382921695709	11.8860242976866\\
11.2869455337524	11.8870945724983\\
11.337321472168	11.8882336132302\\
11.3868953704834	11.8893443255364\\
11.4372324466705	11.8901884494447\\
11.4881226539612	11.8909641716024\\
11.5377847671509	11.8915606543691\\
11.5873648643494	11.8922450720623\\
11.6380569458008	11.8929801473087\\
11.6869468212128	11.8939082584984\\
11.7386147499084	11.8949034957319\\
11.7868301391602	11.8959091648335\\
11.8376504898071	11.8966487096195\\
11.8868886947632	11.8973559405823\\
11.9389214038849	11.8978763259912\\
11.9870438098907	11.8984824721034\\
12.0372139930725	11.8991205299931\\
12.0867568969727	11.8999154226836\\
12.1371871948242	11.9007285803219\\
12.1868307113647	11.9015854738552\\
12.2372700691223	11.9021746480012\\
12.2880680084229	11.9027657257939\\
12.3380994319916	11.903198615403\\
12.3877601146698	11.9037678999826\\
12.4371673583984	11.9043281792216\\
12.4874753475189	11.9050436249988\\
12.5367342948914	11.9057430379424\\
12.5872320652008	11.9064414409096\\
12.6379358291626	11.906919274193\\
12.6869699478149	11.9074261502916\\
12.737691116333	11.9078015699172\\
12.7872032642364	11.9082896509879\\
12.837261390686	11.908735895449\\
12.8871109008789	11.9093076844752\\
12.9367858886719	11.9098172846372\\
12.9869508266449	11.9103343870519\\
13.03677983284	11.9107105217266\\
13.0889579772949	11.9111370474226\\
13.1373173713684	11.9115006860515\\
13.1868133068085	11.9119691021702\\
13.2383656024933	11.9123705462096\\
13.2870273113251	11.9128698830772\\
13.3382465362549	11.9132694484543\\
13.3877489089966	11.9136885255282\\
13.4376754283905	11.9139918181951\\
13.4869431972504	11.914352581479\\
13.5382995128632	11.9146486576203\\
13.5876717090607	11.9150324989061\\
13.6373288154602	11.9153441542619\\
13.6871709346771	11.9157387725408\\
13.7372221469879	11.9160383514154\\
13.7870430469513	11.916381364178\\
13.8368041038513	11.916625263144\\
13.8874387264252	11.9169344825866\\
13.9370603084564	11.9171725596828\\
13.9871797084808	11.9175013321675\\
14.0383731842041	11.9177297489947\\
14.0870515823364	11.9180318351277\\
14.1367220401764	11.9182619238359\\
14.1868392944336	11.9185506217275\\
14.2372829437256	11.9187757055992\\
14.2870769023895	11.9190804007355\\
14.3382911205292	11.9193039444145\\
14.3871123313904	11.9195969905344\\
14.4397968769073	11.9197593275061\\
14.4871720790863	11.9199848221166\\
14.5370795249939	11.9201492181013\\
14.587115240097	11.9203856575215\\
14.6373021125793	11.9206162966902\\
14.6878222942352	11.9209185830057\\
14.7388114452362	11.9211319284661\\
14.7875110626221	11.9214028668044\\
14.8372318267822	11.9215050608074\\
14.8873505115509	11.9216650535111\\
14.9386431694031	11.9217735035598\\
14.9871870994568	11.9219946273034\\
15.0373312950134	11.9222290543823\\
15.0867146968842	11.9225545220603\\
15.1386305809021	11.9227571372294\\
15.1868702888489	11.922986301157\\
15.2371518135071	11.9230212608419\\
15.2869948863983	11.9231052298283\\
15.3384689807892	11.9231914987325\\
15.3868526935577	11.9233889983717\\
15.4374238967896	11.9236079287804\\
15.4874579429626	11.9239021299985\\
15.5373572826385	11.9240569231229\\
15.5874838352203	11.9242343197847\\
15.6373798370361	11.9242379963705\\
15.6868905544281	11.9243027469783\\
15.737205696106	11.924387653538\\
15.7869731903076	11.9245627336053\\
15.8384644508362	11.9247593155923\\
15.887641620636	11.9250261603388\\
15.938517999649	11.9251642019936\\
15.9870197296143	11.9253211030967\\
16.0382165431976	11.9253247272721\\
16.087006521225	11.9253900488068\\
16.1369261264801	11.9254692544815\\
16.187291097641	11.9256622228884\\
16.2368158817291	11.9258703668431\\
16.2877594947815	11.9261337968021\\
16.3371722221375	11.9262321556338\\
16.3871049404144	11.9263484063656\\
16.4373132705688	11.9263532182095\\
16.4878031730652	11.9264417611526\\
16.5373649120331	11.9265798198151\\
16.5867781162262	11.9268219929338\\
16.6383754730225	11.926981344832\\
16.6873669147491	11.9271413368301\\
16.7393970012665	11.9271396621868\\
16.7873897075653	11.9271775491091\\
16.8394999027252	11.927184797821\\
16.8871914863586	11.927260093746\\
16.9372188568115	11.9273533381581\\
16.9868101596832	11.9275352795563\\
17.0378970623016	11.9276663947444\\
17.0877410888672	11.9278362508405\\
17.1387340545654	11.9278847003562\\
17.1871625900269	11.9279639594629\\
17.2373549461365	11.9279675217104\\
17.2877740383148	11.9280164597684\\
17.3370270252228	11.928058693977\\
17.3883034706116	11.9281848937301\\
17.4389266490936	11.9283040878796\\
17.4871525287628	11.9284712420678\\
17.5380241394043	11.9285453754813\\
17.5868267536163	11.9286544310495\\
17.6374902248383	11.9286837183691\\
17.6872567653656	11.928759541427\\
17.7372834205627	11.9288208326971\\
17.7898430347443	11.928959765862\\
17.8382009983063	11.9290806999469\\
17.8908442974091	11.9292340467444\\
17.9381300926208	11.929288960435\\
17.9873158454895	11.9293874972809\\
18.038357925415	11.9294175987162\\
18.0870062828064	11.9294988800771\\
18.1371778964996	11.9295678996177\\
18.1874336719513	11.9297190537922\\
18.2375645160675	11.9298450374131\\
18.2877232551575	11.9300017972534\\
18.3384027004242	11.9300502427319\\
18.3875796318054	11.9301316794683\\
18.4372620105743	11.9301592267459\\
18.4870087623596	11.9302429850671\\
18.5369572162628	11.9303219289774\\
18.5870844841003	11.9304677498597\\
18.6373524188995	11.9305663735558\\
18.6875247478485	11.9306955387183\\
18.7381965637207	11.93074116947\\
18.787005853653	11.9308316919133\\
18.8369290351868	11.9308974239272\\
18.8874282360077	11.9310220222525\\
18.9384850978851	11.9311097315664\\
18.9872798442841	11.9312354096317\\
19.0383550643921	11.9312912147819\\
19.0876242637634	11.9313735955073\\
19.13878865242	11.9314238520686\\
19.1871997833252	11.9315419070784\\
19.238090467453	11.9316356598333\\
19.2875425338745	11.9317693006797\\
19.3374142169952	11.931820443196\\
19.3872191429138	11.9318994841372\\
19.4372310161591	11.931937425175\\
19.4873599529266	11.9320202037051\\
19.5374044895172	11.9320996911031\\
19.5872218132019	11.9322159572761\\
19.636870098114	11.9322666329615\\
19.6870867729187	11.9323401524114\\
19.7377478599548	11.9323611675032\\
19.7868735313416	11.9324199872361\\
19.8382687091827	11.9324921233138\\
19.8874859333038	11.9326009260096\\
19.9384080886841	11.9326944905511\\
19.9873585224152	11.9328288798067\\
20.0380374908447	11.9328926109254\\
20.0872175216675	11.9329813345547\\
20.1384467601776	11.933018153807\\
20.1871039390564	11.9330810048834\\
20.2371427536011	11.9331305891253\\
20.2878586769104	11.9332170061001\\
20.3390125751495	11.9333003650487\\
20.3873836517334	11.9334332946318\\
20.4368681430817	11.9335298345174\\
20.4869358062744	11.9336503364588\\
20.5373003005981	11.9336845148341\\
20.5874337673187	11.9337333357622\\
20.6376835823059	11.9337750369785\\
20.6868483543396	11.9338805929508\\
20.7382573604584	11.9340312576476\\
20.7872662067413	11.934237826357\\
20.8385295391083	11.934352525441\\
20.8871445178986	11.93446618925\\
20.9377784252167	11.9344726688575\\
20.9873167991638	11.9345162750518\\
21.0377876281738	11.9345909872402\\
21.0882877826691	11.9347182160189\\
21.1374673366547	11.9348477920544\\
21.1871928691864	11.9349981291156\\
21.2374176502228	11.9350434783489\\
21.2877287387848	11.935088171546\\
21.3368806362152	11.9350938688297\\
21.3871657371521	11.9351283856692\\
21.4381284236908	11.9352317976726\\
21.4872109413147	11.9353676820921\\
21.5376049995422	11.9354523671949\\
21.5874728679657	11.9355417809711\\
21.6380285739899	11.9355579970288\\
21.6870588779449	11.9355959517052\\
21.739088010788	11.9356548515237\\
21.7871963500977	11.9357592427316\\
21.8373071670532	11.9358629614526\\
21.8888287067413	11.9359780339592\\
21.9381830215454	11.9360306118869\\
21.9872478961945	11.9360899604091\\
22.0378989696503	11.9361224197871\\
22.0872938156128	11.9361839600023\\
22.138090801239	11.9362614785216\\
22.1868211746216	11.9363672645392\\
22.2377860069275	11.9364379770922\\
22.2874586105347	11.9365170162673\\
22.3384885311127	11.9365486954453\\
22.3870710849762	11.9365986671433\\
22.4387132644653	11.9366554609285\\
22.4868266105652	11.9367422409713\\
22.5373594284058	11.9368139581253\\
22.5873472213745	11.9368878387305\\
22.6370512962341	11.9369109161463\\
22.6884295463562	11.9369441885429\\
22.7378239154816	11.9369925315519\\
22.787751865387	11.9370597640877\\
22.8374096870422	11.9371194916859\\
22.8873886585236	11.9371887757977\\
22.9379829883575	11.9372201778622\\
22.9869961261749	11.9372570382985\\
23.037150812149	11.9372797313037\\
23.0874580860138	11.9373167810828\\
23.1373772144318	11.9373695510337\\
23.1871411323547	11.9374482563234\\
23.237290096283	11.9375261605311\\
23.2868723392487	11.9376160424157\\
23.3371350288391	11.9376529309202\\
23.3872100830078	11.9376935474976\\
23.4371966838837	11.937723068798\\
23.4883241176605	11.9377797344322\\
23.5368735313416	11.9378662380954\\
23.5883466720581	11.9379778534376\\
23.6373395442963	11.9380477073421\\
23.6870235919952	11.9381142493959\\
23.7381245613098	11.938124530207\\
23.7868911743164	11.9381483976882\\
23.8377303600311	11.9381916893933\\
23.8871185302734	11.9382521651281\\
23.9379884719849	11.9383305586116\\
23.9873661518097	11.9384226740124\\
24.0375952243805	11.9384745174579\\
24.0870673179626	11.9385277981365\\
24.1371151924133	11.9385507910867\\
24.1869523048401	11.9385913643176\\
24.2370559692383	11.938653405367\\
24.2891807079315	11.9387389379875\\
24.3380689144135	11.938819645191\\
24.3871647834778	11.938907350143\\
24.4377190589905	11.9389559962282\\
24.4868864536285	11.9390208031686\\
24.5379740715027	11.9390831008097\\
24.5869614601135	11.9391645463917\\
24.6376108646393	11.9392341503775\\
24.6870998859406	11.9393074366604\\
24.7377688407898	11.9393417977046\\
24.7873253345489	11.939382294684\\
24.8372883319855	11.9394230614786\\
24.8869270801544	11.939487950613\\
24.9384078502655	11.9395765266825\\
24.9870955467224	11.9396884068412\\
25.0383190631866	11.9397725389371\\
25.0877341747284	11.9398540050326\\
25.1370951652527	11.9398793904315\\
25.1875030517578	11.9399097685685\\
25.2380036830902	11.9399578386454\\
25.2871031284332	11.9400367397384\\
25.338099193573	11.9401458273652\\
25.386954498291	11.9402759638089\\
25.4379999160767	11.9403542975316\\
25.4874267101288	11.9404238424002\\
25.5373079299927	11.940440132353\\
25.5872232437134	11.9404629265327\\
25.6382319450378	11.9405197451287\\
25.6868111610413	11.9406087881011\\
25.7375020503998	11.9407240062208\\
25.7869216918945	11.9408543907301\\
25.8372537612915	11.9409297486623\\
25.8889283657074	11.9409958132944\\
25.9383606433868	11.9410082565832\\
25.9875232696533	11.941025396581\\
26.0369369506836	11.9410747327197\\
26.0872263431549	11.9411523932426\\
26.1380128383636	11.9412535276476\\
26.1871096611023	11.9413672272819\\
26.2377979278564	11.9414299175762\\
26.2872809886932	11.9414858742784\\
26.3384959220886	11.9415077335206\\
26.387292098999	11.9415382735157\\
26.4373895645142	11.9415973652608\\
26.4868809700012	11.941677247955\\
26.53678150177	11.941764887514\\
26.5870432376862	11.9418598822156\\
26.6371018409729	11.9419178414125\\
26.6869751930237	11.9419742676792\\
26.7378837585449	11.9420097198825\\
26.7877766609192	11.9420549105517\\
26.8378419399261	11.9421231651191\\
26.8871218681335	11.9422130097207\\
26.9376534938812	11.9423031545945\\
26.987083864212	11.9423999614527\\
27.0375470638275	11.942454795935\\
27.0869986534119	11.9425155046861\\
27.137539100647	11.9425672610967\\
27.1868445396423	11.9426340381697\\
27.2373196601868	11.9427056127306\\
27.286718082428	11.9427886577389\\
27.3369435787201	11.942849544611\\
27.3872541904449	11.942923495793\\
27.4369372844696	11.9429911104665\\
27.4883272171021	11.9430717555406\\
27.5371679782867	11.9431389673454\\
27.5876988887787	11.9432095710955\\
27.6373111724854	11.9432456445223\\
27.6871180057526	11.9432792153254\\
27.7367758274078	11.9433122589797\\
27.7870089530945	11.9433610027144\\
27.8375215053558	11.9434385162728\\
27.8871633529663	11.9435280271755\\
27.940003824234	11.9435872731719\\
27.9871308326721	11.9436361657151\\
28.0378810882568	11.9436482531948\\
28.0873820304871	11.9436617372842\\
28.1369366168976	11.9437053380163\\
28.1869079589844	11.9437702752182\\
28.2368185043335	11.943858430608\\
28.2877447128296	11.9439591544807\\
28.3373779773712	11.9440210780638\\
28.3870558261871	11.9440760191235\\
28.4374107837677	11.9440944474785\\
28.4871124744415	11.9441163899804\\
28.5371062278748	11.9441625800979\\
28.5868112564087	11.9442294309877\\
28.6371838569641	11.9443145833958\\
28.6869341850281	11.9444089700123\\
28.7380611419678	11.9444699779104\\
28.7870323181152	11.9445276334849\\
28.8377327442169	11.9445528157446\\
28.8874563694	11.9445849332071\\
28.9368562221527	11.944635685831\\
28.9870872020721	11.9447020532161\\
29.0385212421417	11.9447673496896\\
29.0873689174652	11.9448337622431\\
29.1368615150452	11.9448707021748\\
29.187535238266	11.944908223508\\
29.2372958183289	11.9449453164383\\
29.2871133804321	11.9449954456475\\
29.3380169391632	11.9450529355575\\
29.3868567466736	11.9451109420754\\
29.4373459339142	11.9451453945931\\
29.4873344421387	11.9451821709221\\
29.5384788036346	11.9452202714743\\
29.5869259357452	11.9452714550797\\
29.6376266002655	11.9453271943386\\
29.6871110916138	11.945383320938\\
29.7390245914459	11.9454188060216\\
29.7868015289307	11.9454553795255\\
29.8379787921906	11.9454880201144\\
29.8871769428253	11.9455315835002\\
29.9371747493744	11.9455855522628\\
29.987563085556	11.9456465024226\\
30.0383028507233	11.9456884175972\\
30.0876338005066	11.9457303175905\\
30.1368801116943	11.9457550336427\\
30.1873883724213	11.9457868422166\\
30.2370848178864	11.9458307242874\\
30.2871500968933	11.9458856306061\\
30.3385204792023	11.945934168295\\
30.386811208725	11.9459830635218\\
30.4384216785431	11.9460095845804\\
30.4868461608887	11.9460379285245\\
30.5371896743774	11.9460650868041\\
30.5867259025574	11.9460979488309\\
30.6380993843079	11.9461361912476\\
30.6876494407654	11.9461820475251\\
30.7367903709412	11.9462195469848\\
30.7870404243469	11.9462599117763\\
30.8369943618774	11.9462871522421\\
30.8871693134308	11.9463185475094\\
30.9381038665771	11.9463481762609\\
30.986993265152	11.946385359131\\
31.0379008769989	11.9464226652889\\
31.0870007991791	11.946464761103\\
31.1389050006866	11.9464927292351\\
31.187396478653	11.9465206369641\\
31.2372066497803	11.946540469972\\
31.2869581699371	11.9465681265209\\
31.3370429992676	11.9466036594051\\
31.3868891716003	11.9466461615225\\
31.4371098995209	11.9466783715382\\
31.488369178772	11.9467104518381\\
31.5368852138519	11.9467259257429\\
31.5877389431	11.946745895081\\
31.6370863437653	11.9467677362382\\
31.6871823787689	11.9467949770092\\
31.738449048996	11.9468199961794\\
31.7871157646179	11.946846531038\\
31.8383152008057	11.9468596969833\\
31.8873555183411	11.946874005689\\
31.9394099235535	11.9468861332457\\
31.9871487140656	11.9469034438771\\
32.0381509780884	11.9469234121809\\
32.088035774231	11.9469459525377\\
32.1374802112579	11.9469587693882\\
32.187110376358	11.94697175976\\
32.2387628078461	11.9469752587689\\
32.2874614715576	11.9469805008027\\
32.3373488903046	11.9469883139626\\
32.3874296665192	11.9469987630863\\
32.4374224662781	11.9470084317734\\
32.4878580093384	11.9470187539319\\
32.5376955986023	11.9470227569853\\
32.5874928951263	11.9470269680928\\
32.6386348724365	11.9470265250366\\
32.6873406887054	11.9470274183138\\
32.7388579368591	11.9470296946911\\
32.7873272418976	11.9470323073716\\
32.8374335289001	11.9470321628349\\
32.8873004436493	11.9470320585185\\
32.937156867981	11.9470284788218\\
32.9873573303223	11.947024959246\\
33.0380663394928	11.947019866829\\
33.0887851238251	11.9470151158508\\
33.1397532939911	11.9470102265508\\
33.1896943569183	11.947003979119\\
33.2378904342651	11.9469960083311\\
33.2873873233795	11.9469884482243\\
33.337309551239	11.9469803946072\\
33.3870064735413	11.9469724196651\\
33.4376117706299	11.9469642447862\\
33.4877395153046	11.9469553927818\\
33.5375718593597	11.9469446445478\\
33.5879704475403	11.9469330184101\\
33.6378068447113	11.946922032608\\
33.6874708652496	11.9469112671543\\
33.7375998020172	11.9469008599267\\
33.7882034301758	11.9468894259297\\
33.8377136707306	11.9468756529644\\
33.8892471313477	11.9468587404523\\
33.9370285987854	11.9468414518981\\
33.9873494625092	11.9468246888973\\
34.0370041847229	11.9468115361114\\
34.0873434066772	11.9467983945378\\
34.1362752437592	11.9467983945378\\
34.1862122535706	11.9467983945378\\
34.2362687110901	11.9467983945378\\
34.2865228176117	11.9467983945378\\
34.3363995075226	11.9467983945378\\
34.3862075328827	11.9467983945378\\
34.4361023426056	11.9467983945378\\
34.4861790657043	11.9467983945378\\
34.5361194133759	11.9467983945378\\
34.586292219162	11.9467983945378\\
34.6364039897919	11.9467983945378\\
34.6863061904907	11.9467983945378\\
34.7361094474793	11.9467983945378\\
34.7864546298981	11.9467983945378\\
34.83685297966	11.9467983945378\\
34.8863391399384	11.9467983945378\\
34.9360925674439	11.9467983945378\\
34.986105632782	11.9467983945378\\
35.0454177379608	11.9467983945378\\
35.0912179470062	11.9467983945378\\
35.136381816864	11.9467983945378\\
35.186456155777	11.9467983945378\\
35.2363311767578	11.9467983945378\\
35.2861754417419	11.9467983945378\\
35.3360485553741	11.9467983945378\\
35.3861937046051	11.9467983945378\\
35.4363774776459	11.9467983945378\\
35.486319732666	11.9467983945378\\
35.536076259613	11.9467983945378\\
35.5862481117249	11.9467983945378\\
35.6363558292389	11.9467983945378\\
35.6861931800842	11.9467983945378\\
35.7363302230835	11.9467983945378\\
35.786288690567	11.9467983945378\\
35.8361827850342	11.9467983945378\\
35.8864717006683	11.9467983945378\\
35.9363789081574	11.9467983945378\\
35.9867715358734	11.9467983945378\\
36.0360667228699	11.9467983945378\\
36.0863480091095	11.9467983945378\\
36.1363653659821	11.9467983945378\\
36.1864337444305	11.9467983945378\\
36.2360584259033	11.9467983945378\\
36.2864779949188	11.9467983945378\\
36.336106967926	11.9467983945378\\
36.3864702701569	11.9467983945378\\
36.4362844944	11.9467983945378\\
36.4863364219666	11.9467983945378\\
36.5363025188446	11.9467983945378\\
36.5861410617828	11.9467983945378\\
36.6363481998444	11.9467983945378\\
36.6863061904907	11.9467983945378\\
36.7361874103546	11.9467983945378\\
36.7863311290741	11.9467983945378\\
36.8363565921783	11.9467983945378\\
36.8863608360291	11.9467983945378\\
36.9363264560699	11.9467983945378\\
36.9860896587372	11.9467983945378\\
37.0362286090851	11.9467983945378\\
37.0864164352417	11.9467983945378\\
37.1363679885864	11.9467983945378\\
37.1863879680634	11.9467983945378\\
37.2362648963928	11.9467983945378\\
37.2863585472107	11.9467983945378\\
37.3360833644867	11.9467983945378\\
37.3862823963165	11.9467983945378\\
37.4368352413178	11.9467983945378\\
37.4861456871033	11.9467983945378\\
37.5361048698425	11.9467983945378\\
37.5862543106079	11.9467983945378\\
37.6362061023712	11.9467983945378\\
37.6863986968994	11.9467983945378\\
37.736749124527	11.9467983945378\\
37.7863394737244	11.9467983945378\\
37.8361865997314	11.9467983945378\\
37.888374042511	11.9467983945378\\
37.9363502979279	11.9467983945378\\
37.9863593101501	11.9467983945378\\
38.0362567424774	11.9467983945378\\
38.0862390518189	11.9467983945378\\
38.1364056587219	11.9467983945378\\
38.1864752292633	11.9467983945378\\
38.2363790988922	11.9467983945378\\
38.2863652229309	11.9467983945378\\
38.3364450454712	11.9467983945378\\
38.3863930225372	11.9467983945378\\
38.4377541065216	11.9467983945378\\
38.4863030433655	11.9467983945378\\
38.5364169597626	11.9467983945378\\
38.5863399028778	11.9467983945378\\
38.6360737800598	11.9467983945378\\
38.686456155777	11.9467983945378\\
38.7360977649689	11.9467983945378\\
38.7863506793976	11.9467983945378\\
38.8363575458527	11.9467983945378\\
38.8868772506714	11.9467983945378\\
38.9363157272339	11.9467983945378\\
38.9863695621491	11.9467983945378\\
39.036302280426	11.9467983945378\\
39.0863775730133	11.9467983945378\\
39.1364001750946	11.9467983945378\\
39.1863421916962	11.9467983945378\\
39.2363414287567	11.9467983945378\\
39.2865790843964	11.9467983945378\\
39.3364316940308	11.9467983945378\\
39.3863877773285	11.9467983945378\\
39.4363846302032	11.9467983945378\\
39.4864034175873	11.9467983945378\\
39.5362460136414	11.9467983945378\\
39.5864672183991	11.9467983945378\\
39.6376175403595	11.9467983945378\\
39.6865121841431	11.9467983945378\\
39.7360577106476	11.9467983945378\\
39.7860879421234	11.9467983945378\\
39.8361377239227	11.9467983945378\\
39.8862397193909	11.9467983945378\\
39.9363758087158	11.9467983945378\\
39.9866580486298	11.9467983945378\\
40.0361229896545	11.9467983945378\\
40.0863337039948	11.9467983945378\\
40.1362952709198	11.9467983945378\\
40.1863722324371	11.9467983945378\\
40.2360808372498	11.9467983945378\\
40.2867063999176	11.9467983945378\\
40.3362371444702	11.9467983945378\\
40.3864602565765	11.9467983945378\\
40.4362022399902	11.9467983945378\\
40.4862391471863	11.9467983945378\\
40.536110830307	11.9467983945378\\
40.586754989624	11.9467983945378\\
40.6363067150116	11.9467983945378\\
40.6862320423126	11.9467983945378\\
40.7363345146179	11.9467983945378\\
40.7863447189331	11.9467983945378\\
40.8362771987915	11.9467983945378\\
40.886261177063	11.9467983945378\\
40.936424446106	11.9467983945378\\
40.9864334583282	11.9467983945378\\
41.0363933563232	11.9467983945378\\
41.0863849639893	11.9467983945378\\
41.1362373352051	11.9467983945378\\
41.1860804080963	11.9467983945378\\
41.2361869335175	11.9467983945378\\
41.2864078998566	11.9467983945378\\
41.3360969543457	11.9467983945378\\
41.3861855983734	11.9467983945378\\
41.4376759052277	11.9467983945378\\
41.486256313324	11.9467983945378\\
41.536371421814	11.9467983945378\\
41.5862829208374	11.9467983945378\\
41.636062335968	11.9467983945378\\
41.686300945282	11.9467983945378\\
41.7360789299011	11.9467983945378\\
41.7862710475922	11.9467983945378\\
41.8360845565796	11.9467983945378\\
41.8863353252411	11.9467983945378\\
41.9362205982208	11.9467983945378\\
41.986283493042	11.9467983945378\\
42.0364117145538	11.9467983945378\\
42.0863775730133	11.9467983945378\\
42.1363970756531	11.9467983945378\\
42.186447095871	11.9467983945378\\
42.2362341403961	11.9467983945378\\
42.2862476825714	11.9467983945378\\
42.3368155479431	11.9467983945378\\
42.3862549781799	11.9467983945378\\
42.4363796234131	11.9467983945378\\
42.4863807678223	11.9467983945378\\
42.5361065387726	11.9467983945378\\
42.586087179184	11.9467983945378\\
42.6363203048706	11.9467983945378\\
42.6863209724426	11.9467983945378\\
42.7362560749054	11.9467983945378\\
42.7863132476807	11.9467983945378\\
42.8362674236298	11.9467983945378\\
42.8862435340881	11.9467983945378\\
42.9362756729126	11.9467983945378\\
42.9860562801361	11.9467983945378\\
43.0366630077362	11.9467983945378\\
43.0862891197205	11.9467983945378\\
43.1369411468506	11.9467983945378\\
43.1862763881683	11.9467983945378\\
43.2363652706146	11.9467983945378\\
43.2866267681122	11.9467983945378\\
43.3361598968506	11.9467983945378\\
43.3862552165985	11.9467983945378\\
43.4363893985748	11.9467983945378\\
43.4867214679718	11.9467983945378\\
43.5363444805145	11.9467983945378\\
43.5866703510284	11.9467983945378\\
43.6362726211548	11.9467983945378\\
43.6863271713257	11.9467983945378\\
43.7361363887787	11.9467983945378\\
43.7873098373413	11.9467983945378\\
43.8363158226013	11.9467983945378\\
43.8864092350006	11.9467983945378\\
43.9364137172699	11.9467983945378\\
43.9862827777863	11.9467983945378\\
44.0369226455689	11.9467983945378\\
44.0863043785095	11.9467983945378\\
44.1360961914063	11.9467983945378\\
44.1862127304077	11.9467983945378\\
44.236283493042	11.9467983945378\\
44.2865638256073	11.9467983945378\\
44.3360812187195	11.9467983945378\\
44.386487197876	11.9467983945378\\
44.4361028194428	11.9467983945378\\
44.4864933013916	11.9467983945378\\
44.536284160614	11.9467983945378\\
44.5863046169281	11.9467983945378\\
44.6362940788269	11.9467983945378\\
44.6862978458405	11.9467983945378\\
44.7364215373993	11.9467983945378\\
44.7863480567932	11.9467983945378\\
44.8362671852112	11.9467983945378\\
44.8862499713898	11.9467983945378\\
44.9360589504242	11.9467983945378\\
44.9889430522919	11.9467983945378\\
45.0363490104675	11.9467983945378\\
45.0864402770996	11.9467983945378\\
45.1361271858215	11.9467983945378\\
45.1861283302307	11.9467983945378\\
45.2363178253174	11.9467983945378\\
45.2860579013824	11.9467983945378\\
45.3360833644867	11.9467983945378\\
45.3861248016357	11.9467983945378\\
45.4362778186798	11.9467983945378\\
45.4862968444824	11.9467983945378\\
45.5364520072937	11.9467983945378\\
45.5863546848297	11.9467983945378\\
45.6363763332367	11.9467983945378\\
45.6862480163574	11.9467983945378\\
45.7361382961273	11.9467983945378\\
45.7863018035889	11.9467983945378\\
45.8363010406494	11.9467983945378\\
45.8863670349121	11.9467983945378\\
45.9360751628876	11.9467983945378\\
45.9862565517426	11.9467983945378\\
46.036564540863	11.9467983945378\\
46.086417388916	11.9467983945378\\
46.136324596405	11.9467983945378\\
46.1863011837006	11.9467983945378\\
46.236340713501	11.9467983945378\\
46.286172580719	11.9467983945378\\
46.3363940238953	11.9467983945378\\
46.3862518787384	11.9467983945378\\
46.4363107204437	11.9467983945378\\
46.4862446308136	11.9467983945378\\
46.5362638950348	11.9467983945378\\
46.5863682746887	11.9467983945378\\
46.6360892772675	11.9467983945378\\
46.6863905906677	11.9467983945378\\
46.7362088680267	11.9467983945378\\
46.7864021778107	11.9467983945378\\
46.8362369060516	11.9467983945378\\
46.8863255500793	11.9467983945378\\
46.9363812923431	11.9467983945378\\
46.9860479354858	11.9467983945378\\
47.0364589214325	11.9467983945378\\
47.086301279068	11.9467983945378\\
47.1364011287689	11.9467983945378\\
47.1863252639771	11.9467983945378\\
47.2360765457153	11.9467983945378\\
47.2863361358643	11.9467983945378\\
47.336061668396	11.9467983945378\\
47.3863124370575	11.9467983945378\\
47.436296415329	11.9467983945378\\
47.4863233089447	11.9467983945378\\
47.5360545635223	11.9467983945378\\
47.5861608505249	11.9467983945378\\
47.6362637996674	11.9467983945378\\
47.6860692024231	11.9467983945378\\
47.7363540649414	11.9467983945378\\
47.7863378047943	11.9467983945378\\
47.8363644599915	11.9467983945378\\
47.8862695217133	11.9467983945378\\
47.9362542152405	11.9467983945378\\
47.9863764762878	11.9467983945378\\
48.0360953330994	11.9467983945378\\
48.0863441944122	11.9467983945378\\
48.1360756874085	11.9467983945378\\
48.186307144165	11.9467983945378\\
48.2360848903656	11.9467983945378\\
48.2862736701965	11.9467983945378\\
48.3363034248352	11.9467983945378\\
48.3863589286804	11.9467983945378\\
48.4373104095459	11.9467983945378\\
48.4862935066223	11.9467983945378\\
48.5363015651703	11.9467983945378\\
48.5862574100494	11.9467983945378\\
48.6363148212433	11.9467983945378\\
48.6862716197968	11.9467983945378\\
48.7360643863678	11.9467983945378\\
48.7863087177277	11.9467983945378\\
48.8364052295685	11.9467983945378\\
48.8860489845276	11.9467983945378\\
48.9366774082184	11.9467983945378\\
48.9863929271698	11.9467983945378\\
49.0368101119995	11.9467983945378\\
49.0863396644592	11.9467983945378\\
49.1362618923187	11.9467983945378\\
49.1863467216492	11.9467983945378\\
49.2363132953644	11.9467983945378\\
49.2862998962402	11.9467983945378\\
49.3360840797424	11.9467983945378\\
49.3862306594849	11.9467983945378\\
49.4362706661224	11.9467983945378\\
49.486224603653	11.9467983945378\\
49.5363234996796	11.9467983945378\\
49.5862879276276	11.9467983945378\\
49.6361152648926	11.9467983945378\\
49.6863047599793	11.9467983945378\\
49.7360612869263	11.9467983945378\\
49.7862214565277	11.9467983945378\\
49.8363184452057	11.9467983945378\\
49.8879334449768	11.9467983945378\\
49.9363319396973	11.9467983945378\\
49.9862370014191	11.9467983945378\\
50.0362956047058	11.9467983945378\\
50.0868522644043	11.9467983945378\\
50.1361271858215	11.9467983945378\\
50.1862706661224	11.9467983945378\\
50.2362811088562	11.9467983945378\\
50.2863530635834	11.9467983945378\\
50.3364486217499	11.9467983945378\\
50.3862692832947	11.9467983945378\\
50.4361993789673	11.9467983945378\\
50.4862770557404	11.9467983945378\\
50.5360641002655	11.9467983945378\\
50.5860704898834	11.9467983945378\\
50.636267375946	11.9467983945378\\
50.6864208698273	11.9467983945378\\
50.7364587306976	11.9467983945378\\
50.786306810379	11.9467983945378\\
50.8363444328308	11.9467983945378\\
50.8863732337952	11.9467983945378\\
50.9360637187958	11.9467983945378\\
50.9860643863678	11.9467983945378\\
51.0364918231964	11.9467983945378\\
51.086265039444	11.9467983945378\\
51.1363150596619	11.9467983945378\\
51.1863760471344	11.9467983945378\\
51.2363709926605	11.9467983945378\\
51.2863130092621	11.9467983945378\\
51.3363859176636	11.9467983945378\\
51.3862800121307	11.9467983945378\\
51.4363278865814	11.9467983945378\\
51.4863607406616	11.9467983945378\\
51.5360641002655	11.9467983945378\\
51.5886544704437	11.9467983945378\\
51.6360978603363	11.9467983945378\\
51.688673210144	11.9467983945378\\
51.7367107391357	11.9467983945378\\
51.7860705375671	11.9467983945378\\
51.8363744735718	11.9467983945378\\
51.8862721443176	11.9467983945378\\
51.936353635788	11.9467983945378\\
51.9860617637634	11.9467983945378\\
52.0368484973907	11.9467983945378\\
52.0862779140472	11.9467983945378\\
52.1360904693604	11.9467983945378\\
52.1863877296448	11.9467983945378\\
52.2361249446869	11.9467983945378\\
52.2863645076752	11.9467983945378\\
52.3361343860626	11.9467983945378\\
52.3863236427307	11.9467983945378\\
52.4363097667694	11.9467983945378\\
52.4860250473022	11.9467983945378\\
52.5360769748688	11.9467983945378\\
52.5860502243042	11.9467983945378\\
52.6363713264465	11.9467983945378\\
52.690006685257	11.9467983945378\\
52.7364186763763	11.9467983945378\\
52.7863256454468	11.9467983945378\\
52.8360683441162	11.9467983945378\\
52.8861855983734	11.9467983945378\\
52.9360858917236	11.9467983945378\\
52.9860407829285	11.9467983945378\\
53.0362543582916	11.9467983945378\\
53.0863141536713	11.9467983945378\\
53.1362888336182	11.9467983945378\\
53.1872043132782	11.9467983945378\\
53.2363035202026	11.9467983945378\\
53.2860457420349	11.9467983945378\\
53.3363325119019	11.9467983945378\\
53.3861095428467	11.9467983945378\\
53.436364364624	11.9467983945378\\
53.4863154411316	11.9467983945378\\
53.5363397121429	11.9467983945378\\
53.5863167762756	11.9467983945378\\
53.6361658096314	11.9467983945378\\
53.6863488674164	11.9467983945378\\
53.7363729000092	11.9467983945378\\
53.7860550403595	11.9467983945378\\
53.836292219162	11.9467983945378\\
53.8862983703613	11.9467983945378\\
53.9362475395203	11.9467983945378\\
53.9861213684082	11.9467983945378\\
54.0362629413605	11.9467983945378\\
54.0862345218658	11.9467983945378\\
54.1360654354095	11.9467983945378\\
54.1862487316132	11.9467983945378\\
54.2360968112946	11.9467983945378\\
54.2862512588501	11.9467983945378\\
54.3363804340363	11.9467983945378\\
54.3863088607788	11.9467983945378\\
54.4362613677979	11.9467983945378\\
54.4864789962769	11.9467983945378\\
54.5363761901855	11.9467983945378\\
54.5860692977905	11.9467983945378\\
54.6361002445221	11.9467983945378\\
54.6862933158875	11.9467983945378\\
54.7361409187317	11.9467983945378\\
54.7861282348633	11.9467983945378\\
54.8360788345337	11.9467983945378\\
54.8863553524017	11.9467983945378\\
54.9360541820526	11.9467983945378\\
54.9860584259033	11.9467983945378\\
55.0363721370697	11.9467983945378\\
55.0868153095245	11.9467983945378\\
55.136411857605	11.9467983945378\\
55.1862184524536	11.9467983945378\\
55.2393674373627	11.9467983945378\\
55.2863261222839	11.9467983945378\\
55.33689661026	11.9467983945378\\
55.3863090991974	11.9467983945378\\
55.4367873191834	11.9467983945378\\
55.4863738536835	11.9467983945378\\
55.5360693454742	11.9467983945378\\
55.586337518692	11.9467983945378\\
55.6370651245117	11.9467983945378\\
55.6863228797913	11.9467983945378\\
55.7360960960388	11.9467983945378\\
55.7863347053528	11.9467983945378\\
};
\addlegendentry{$\theta_{1}$}


\addplot[area legend, draw=none, fill=black, fill opacity=0.09, forget plot]
table[row sep=crcr] {%
x	y\\
0	-4\\
33.4	-4\\
33.4	14\\
0	14\\
}--cycle;
\end{axis}

\begin{axis}[%
width=0in,
height=0in,
at={(0in,0in)},
scale only axis,
xmin=0,
xmax=1,
ymin=0,
ymax=1,
axis line style={draw=none},
ticks=none,
axis x line*=bottom,
axis y line*=left
]
\end{axis}
\end{tikzpicture}%
\vspace{-0.7em}
\caption{\rahel{Coefficient estimation over time applying Algorithm~\ref{alg:twolayermpcalglearningimp} in consideration of an initially unknown $\phi_{2}$. The grey background indicates time instances for which the agents are exploring.}}
\label{fig:twolayerslearningcoef}
\vspace{-1.0em}
\end{figure}
\vspace{-0.4em}
\subsection{Results One-Layer Coverage MPC}
\label{subsec:experimentonelayer}
Applying Algorithm~\ref{alg:onelayeralg} in the described set-up using $\phi_{1}$, we obtain the locational optimization cost as well as the cumulative target cost decrease presented in Figure~\ref{fig:onelayercoveragecost}. 
\begin{figure} [h!]
\centering
% This file was created by matlab2tikz.
%
%The latest updates can be retrieved from
%  http://www.mathworks.com/matlabcentral/fileexchange/22022-matlab2tikz-matlab2tikz
%where you can also make suggestions and rate matlab2tikz.
%
\definecolor{mycolor1}{rgb}{0.47000,0.67000,0.19000}%
\definecolor{mycolor2}{rgb}{0.89000,0.59000,0.53000}%
%
\begin{tikzpicture}

\begin{axis}[%
height=0.85in,
width=0.40\textwidth,
yshift=0.8cm,
at={(0.0in,0.0in)},
scale only axis,
xmin=0,
xmax=22,
xlabel style={font=\color{white!15!black}},
xlabel style={font=\footnotesize},
xlabel style={yshift=0.6ex,},
xlabel={Time [sec]},
ymin=0,
ymax=70,
ylabel style={font=\color{white!15!black}},
ylabel style={font=\footnotesize},
ylabel style={xshift=0.6ex,},
ylabel={H(p,$\mathbb{W}$) \& H($\bar{p}$,$\mathbb{W}$)},
axis background/.style={fill=white},
tick label style={font=\footnotesize},
xmajorgrids,
ymajorgrids,
legend style={legend cell align=left, align=left, draw=white!15!black},
legend style={font=\footnotesize}
]
\addplot [color=mycolor1, line width=1.4pt]
  table[row sep=crcr]{%
-1.14863276481628	65.562193201352\\
-0.9499192237854	65.5624347419556\\
-0.74800181388855	65.5609329674382\\
-0.548374652862549	65.2822277839491\\
-0.348526477813721	65.0303386888569\\
-0.146867752075195	64.3525586599515\\
0.0536508560180664	63.3171338541951\\
0.254810094833374	62.328330128019\\
0.452388525009155	61.3898731351064\\
0.653592109680176	60.3493058384483\\
0.855054140090942	59.2457307249114\\
1.05077505111694	58.3207244305405\\
1.25211906433105	57.2236254489755\\
1.45325183868408	56.0760044833987\\
1.65678715705872	54.8605719037616\\
1.85086393356323	53.773023367262\\
2.05186653137207	52.5128513380267\\
2.2526798248291	51.2926420275172\\
2.45258402824402	50.0099449309508\\
2.65606117248535	48.7309763530255\\
2.86030268669128	47.6195702193418\\
3.05085372924805	46.4754120151667\\
3.25225734710693	45.1484445480075\\
3.45727562904358	43.8475214178303\\
3.65331101417542	42.6404587216085\\
3.85134172439575	41.509864736026\\
4.05105924606323	40.1871730257659\\
4.25282216072083	38.9286886695794\\
4.45276713371277	37.6999430989049\\
4.65254688262939	36.4447367606383\\
4.85205054283142	35.2889340970934\\
5.05226540565491	34.1067977397443\\
5.25319838523865	32.8289335533659\\
5.45644354820251	31.9367821664029\\
5.65510845184326	30.6456142239336\\
5.85332989692688	29.4693239590976\\
6.05252838134766	28.3861047297461\\
6.25445294380188	27.0183760487706\\
6.46015214920044	25.905552687018\\
6.65498781204224	24.8764802416462\\
6.85395312309265	23.8645700332329\\
7.05144262313843	22.8743926276081\\
7.25302982330322	21.8226515302632\\
7.45259237289429	20.8498981183392\\
7.65178751945496	19.957950482174\\
7.85272026062012	19.0657912901285\\
8.05256104469299	18.1877506863703\\
8.25225877761841	17.4217944352581\\
8.45404386520386	16.5954410798695\\
8.65307760238647	15.7682165240718\\
8.85222029685974	14.9648181657544\\
9.05230665206909	14.1323156000588\\
9.25369954109192	13.3189219443762\\
9.45354962348938	12.5514388092422\\
9.6605167388916	11.8010218445597\\
9.85368680953979	11.1050340674596\\
10.0528147220612	10.8020312150005\\
10.2548942565918	10.1171839341643\\
10.4560463428497	9.33008651742342\\
10.6537096500397	8.70989280118209\\
10.852068901062	8.03524231914427\\
11.060142993927	7.44238125393548\\
11.2601866722107	6.94278358139857\\
11.453777551651	7.32208323535132\\
11.6524991989136	6.83365635289609\\
11.8529818058014	6.34175185751041\\
12.056131362915	5.77047468900845\\
12.2538182735443	5.28006582155066\\
12.4540915489197	5.51241340222508\\
12.6534292697906	5.01978566371832\\
12.852790594101	4.81567696009788\\
13.0552260875702	4.75377938461791\\
13.2522768974304	4.58126525518347\\
13.4556972980499	4.18152189290704\\
13.6580104827881	4.13796743892915\\
13.8601644039154	3.71228822124642\\
14.053829908371	3.49217254587481\\
14.2517249584198	3.23820424471994\\
14.4562628269196	3.04667752550112\\
14.6524398326874	2.84352885815419\\
14.8541240692139	2.69258297693489\\
15.0627791881561	2.61331699740651\\
15.2535579204559	2.48846082211692\\
15.4558203220367	2.34031304423724\\
15.6550364494324	2.23859256397869\\
15.853856086731	2.16405872275812\\
16.0536758899689	2.10244873744507\\
16.2539458274841	2.07627310813357\\
16.4536452293396	2.05658222181167\\
16.6541950702667	2.0303097331946\\
16.8526608943939	1.99458243081649\\
17.0542204380035	1.9588635763843\\
17.25421833992	1.92722415584726\\
17.4558839797974	1.89774068567154\\
17.6550118923187	1.87518103937933\\
17.8538489341736	1.8590554486696\\
18.0541191101074	1.84503837558478\\
18.25346326828	1.82541058036337\\
18.4529554843903	1.80984249510558\\
18.6553075313568	1.79847834759857\\
18.8579256534576	1.79420429428453\\
19.0531616210938	1.79390771014297\\
19.2545323371887	1.79735861391717\\
19.4557473659515	1.79764517956737\\
19.6673305034637	1.79333963301262\\
19.8524878025055	1.79010651031494\\
20.0534772872925	1.78926342507603\\
20.2548887729645	1.78924748815141\\
20.4541759490967	1.79134845389358\\
20.654465675354	1.79320624091296\\
20.8551850318909	1.79078057287302\\
21.0532202720642	1.78970154931529\\
21.2552809715271	1.78938267537597\\
21.4548418521881	1.78990505638254\\
21.6552832126617	1.79337986339714\\
21.8562107086182	1.79157638908723\\
22.0539293289185	1.78996910279734\\
22.2615072727203	1.78973453507758\\
};
\addlegendentry{H(p,W)}

\addplot [color=mycolor2, line width=1.4pt]
  table[row sep=crcr]{%
-1.14863276481628	4.82682191063969\\
-0.9499192237854	4.82682191063969\\
-0.74800181388855	5.18783651934293\\
-0.548374652862549	8.09393827843221\\
-0.348526477813721	12.2152705177284\\
-0.146867752075195	15.2022015626517\\
0.0536508560180664	15.6784610496897\\
0.254810094833374	15.3241592751222\\
0.452388525009155	14.9420886289247\\
0.653592109680176	14.5634875431525\\
0.855054140090942	14.1706006139336\\
1.05077505111694	13.7963053359219\\
1.25211906433105	13.3839974045554\\
1.45325183868408	12.9950163049752\\
1.65678715705872	12.5863879105637\\
1.85086393356323	12.1880321235988\\
2.05186653137207	11.7371531518192\\
2.2526798248291	11.3556276691377\\
2.45258402824402	10.9076012184155\\
2.65606117248535	10.5417216366012\\
2.86030268669128	10.1556514170574\\
3.05085372924805	9.79698410083848\\
3.25225734710693	9.42663582088015\\
3.45727562904358	9.05110002582203\\
3.65331101417542	8.71821829867112\\
3.85134172439575	8.40069574321629\\
4.05105924606323	8.07257895962971\\
4.25282216072083	7.73527326196105\\
4.45276713371277	7.44132066342783\\
4.65254688262939	7.1800521708258\\
4.85205054283142	6.91575574543485\\
5.05226540565491	6.65305281167643\\
5.25319838523865	6.42329877143479\\
5.45644354820251	6.20350286498015\\
5.65510845184326	6.01263056378562\\
5.85332989692688	5.83727697105999\\
6.05252838134766	5.67405977630017\\
6.25445294380188	4.86353653658054\\
6.46015214920044	4.41266911639833\\
6.65498781204224	4.09275741262297\\
6.85395312309265	3.98603371078364\\
7.05144262313843	3.89288785440247\\
7.25302982330322	3.8171757480794\\
7.45259237289429	3.74568689519841\\
7.65178751945496	3.69986506029771\\
7.85272026062012	3.65626231730288\\
8.05256104469299	3.62504639736417\\
8.25225877761841	3.60009623187219\\
8.45404386520386	3.58112633144317\\
8.65307760238647	3.56674747810836\\
8.85222029685974	3.55671910399462\\
9.05230665206909	3.54994951876785\\
9.25369954109192	3.54530090533223\\
9.45354962348938	3.54239101330009\\
9.6605167388916	3.54075027399415\\
9.85368680953979	3.53968124478077\\
10.0528147220612	2.82378958486142\\
10.2548942565918	2.57383343166985\\
10.4560463428497	2.45444175849939\\
10.6537096500397	2.45419794996822\\
10.852068901062	2.45386431791954\\
11.060142993927	2.45370043420302\\
11.2601866722107	2.45362748011544\\
11.453777551651	2.12377546297446\\
11.6524991989136	2.05272612448421\\
11.8529818058014	1.93783525919921\\
12.056131362915	1.93718634967001\\
12.2538182735443	1.93680043874723\\
12.4540915489197	1.82512050980072\\
12.6534292697906	1.79855932352391\\
12.852790594101	1.72494126595411\\
13.0552260875702	1.69429234498362\\
13.2522768974304	1.68688711185345\\
13.4556972980499	1.65832145309634\\
13.6580104827881	1.64788755926484\\
13.8601644039154	1.63932854401094\\
14.053829908371	1.64445108836856\\
14.2517249584198	1.6392472517355\\
14.4562628269196	1.63161252224022\\
14.6524398326874	1.63092709560399\\
14.8541240692139	1.63286171531763\\
15.0627791881561	1.62805126480581\\
15.2535579204559	1.63235923479764\\
15.4558203220367	1.63335088684578\\
15.6550364494324	1.63402580788887\\
15.853856086731	1.63365569063574\\
16.0536758899689	1.63365286341622\\
16.2539458274841	1.63365557838884\\
16.4536452293396	1.63365561466941\\
16.6541950702667	1.63365561825339\\
16.8526608943939	1.63365562787564\\
17.0542204380035	1.63365563689708\\
17.25421833992	1.63365563589644\\
17.4558839797974	1.63365563400017\\
17.6550118923187	1.63365560346543\\
17.8538489341736	1.63365553786641\\
18.0541191101074	1.63365540998605\\
18.25346326828	1.63365506864634\\
18.4529554843903	1.63365420995785\\
18.6553075313568	1.63365084811231\\
18.8579256534576	1.63365409048682\\
19.0531616210938	1.63365587691128\\
19.2545323371887	1.63365493218485\\
19.4557473659515	1.63365528769902\\
19.6673305034637	1.63365417037062\\
19.8524878025055	1.63364928183907\\
20.0534772872925	1.63365716967745\\
20.2548887729645	1.63365570900968\\
20.4541759490967	1.63365552768294\\
20.654465675354	1.63365537775702\\
20.8551850318909	1.63365384007193\\
21.0532202720642	1.63365645297589\\
21.2552809715271	1.63365565433818\\
21.4548418521881	1.63365478745126\\
21.6552832126617	1.63365544804109\\
21.8562107086182	1.63365444889696\\
22.0539293289185	1.63365674356721\\
22.2615072727203	1.63365565456281\\
};
\addlegendentry{H($\bar{p}$,$\mathbb{W}$)}

\end{axis}
\end{tikzpicture}%
\vspace{-0.7em}
\caption{\rahel{Locational optimization cost decrease (green) as well as the over all agents summed up target cost decrease (rose) over time, applying Algorithm~\ref{alg:onelayeralg} with respect to a known density $\phi_{1}$.}}
\label{fig:onelayercoveragecost}
\vspace{-0.5em}
\end{figure}


\begin{figure}[h!]
\begin{minipage}[t]{0.15\textwidth}
\centering
\includegraphics[trim={5.2cm 8.1cm 4.6cm 7.7cm},clip, width = 0.8\textwidth]{figures/config_7sec_one_layer_1_0_take_two_compressed.pdf}
\centering
\end{minipage}
\hfill
\begin{minipage}[t]{0.15\textwidth}
\centering
\includegraphics[trim={5.2cm 8.1cm 4.6cm 7.7cm},clip, width = 0.8\textwidth]{figures/config_13sec_one_layer_1_0_take_two_compressed.pdf}
\centering
\end{minipage}
\hfill
\begin{minipage}[t]{0.15\textwidth}
\centering
\includegraphics[trim={5.2cm 8.1cm 4.6cm 7.7cm},clip, width = 0.8\textwidth]{figures/config_26sec_one_layer_1_0_take_two_compressed.pdf}
\centering
\end{minipage}
\caption{\rahel{Configurations of cars at 2, 8, and 21 seconds applying Algorithm~\ref{alg:onelayeralg} in the described set-up. The agents' location and their predicted trajectory are given in red, the Voronoi partitions in green, the artificial setpoints in blue, and the traveled paths are visualized in light grey.}}
\label{pics:onelayerconfig}
\vspace{-0.7em}
\end{figure}

\noindent It shows a comparable behavior to the two-layers approach regarding the time required until convergence, as well as the final configuration. While at certain instances in time the locational optimization cost increases slightly, the summed up target cost is strictly decreasing. 
\subsection{Results One-Layer, Learning-Based Coverage MPC}
\label{subsec:experimentonelayerlearning}
In a final experiment, Algorithm~\ref{alg:onelayerlearningalg} is applied, with density $\phi_{2}$ initially unknown and $S$=2.5. 
\begin{figure} [h!]
%\centering
% This file was created by matlab2tikz.
%
%The latest updates can be retrieved from
%  http://www.mathworks.com/matlabcentral/fileexchange/22022-matlab2tikz-matlab2tikz
%where you can also make suggestions and rate matlab2tikz.
%
\definecolor{mycolor1}{rgb}{0.47000,0.67000,0.19000}%
\definecolor{mycolor2}{rgb}{0.79000,0.49000,0.43000}%
%
\begin{tikzpicture}

\begin{axis}[%
%width=6.862in,
height=0.85in,
width=0.40\textwidth,
yshift=0.8cm,
at={(0.0in,0.0in)},
scale only axis,
xmin=0,
xmax=21,
xlabel style={font=\color{white!15!black}},
xlabel style={font=\footnotesize},
xlabel style={yshift=0.6ex,},
xlabel={Time [sec]},
ymin=0,
ymax=70,
ylabel style={font=\color{white!15!black}},
ylabel style={font=\footnotesize},
ylabel style={xshift = 0.6ex,},
ylabel={H(p,$\mathbb{W}$,$\phi$) \& H(p,$\mathbb{W}$,$\hat{\phi}$)},
axis background/.style={fill=white},
tick label style={font=\footnotesize},
%title style={font=\bfseries},
%title={},
xmajorgrids,
ymajorgrids,
legend style={legend cell align=left, align=left, draw=white!15!black},
legend style={font=\footnotesize}
]
\addplot [color=mycolor1, line width=1.4pt]
  table[row sep=crcr]{%
-0.117774486541748	55.6416455837685\\
0.280490159988403	55.6453627549429\\
0.483649969100952	55.361083891311\\
0.682579755783081	54.8773618427026\\
0.883516311645508	54.4365677614694\\
1.08183526992798	53.8771787478607\\
1.28105568885803	52.7542535672557\\
1.4802074432373	51.6136157983385\\
1.6821653842926	50.4360011825191\\
1.8819944858551	49.4240281906488\\
2.08089518547058	48.4058542677896\\
2.28198099136353	47.3090191264453\\
2.48251509666443	46.3847905869479\\
2.68301582336426	45.6359750824814\\
2.8805935382843	45.029363172859\\
3.08038401603699	44.4180656231274\\
3.2818455696106	43.5742824500728\\
3.48093891143799	42.8493418005347\\
3.68154144287109	42.2362567113429\\
3.88178730010986	41.5072348794846\\
4.08241820335388	40.9321376153649\\
4.28357720375061	40.3427885174429\\
4.48154473304749	39.6726479886624\\
4.68079805374146	39.0314837426816\\
4.88374209403992	38.2690356066683\\
5.08170390129089	37.4957600645149\\
5.28213214874268	36.6608746608028\\
5.48064565658569	35.798442589528\\
5.6825487613678	34.693506745643\\
5.88136625289917	33.7996035091865\\
6.08316135406494	32.7448490956232\\
6.28040170669556	31.7546724163282\\
6.48473501205444	30.7255765475058\\
6.68359088897705	29.7565531932265\\
6.8818998336792	28.844052477089\\
7.08079028129578	28.0449804973609\\
7.28340697288513	27.2355326202412\\
7.4817681312561	26.4590145323513\\
7.68242835998535	25.7297394324212\\
7.88226366043091	25.0451935509965\\
8.08400750160217	21.3410933718077\\
8.28272938728333	20.7647311329084\\
8.48226618766785	20.1426341001193\\
8.68258595466614	19.52805074703\\
8.88428592681885	18.9668370496759\\
9.08221316337585	18.4405659707563\\
9.28181314468384	17.9469920160315\\
9.48220610618591	17.4299038113296\\
9.6818208694458	18.3242499781337\\
9.88172578811646	17.9735928518606\\
10.0821895599365	17.669638109797\\
10.2835392951965	17.3846642457329\\
10.4819307327271	17.1264844208831\\
10.6823532581329	16.6955435589931\\
10.8809492588043	16.324843202953\\
11.0806744098663	15.9516226898534\\
11.2827413082123	15.5441400341789\\
11.4807531833649	15.1649371911249\\
11.6827719211578	14.7359950204209\\
11.8808121681213	14.204652680973\\
12.0822365283966	13.7627185072749\\
12.2819607257843	13.2446165806699\\
12.482825756073	12.7426418872089\\
12.6820688247681	13.6182524074829\\
12.8831100463867	12.9932256099771\\
13.0828530788422	12.434824934557\\
13.2831931114197	12.848650264405\\
13.4814794063568	12.3879141763603\\
13.6822912693024	12.2708504510673\\
13.8833720684052	11.9899648522295\\
14.0813863277435	11.7371021230744\\
14.2816686630249	11.5166552083777\\
14.4826619625092	11.2839397286643\\
14.6806054115295	10.9476933473091\\
14.8825922012329	10.5617598837446\\
15.0800611972809	10.1971533405991\\
15.2834858894348	9.86447391244495\\
15.4836881160736	9.55526590479222\\
15.6811118125916	9.31695098847601\\
15.8810572624207	9.11008817662245\\
16.0815989971161	8.92641957598823\\
16.2841033935547	8.70922488775289\\
16.4819452762604	8.49097062595203\\
16.6800713539124	8.29315991785297\\
16.881413936615	8.13959434449154\\
17.0820443630219	8.03574777971558\\
17.2811107635498	8.03160866844738\\
17.4821639060974	8.02398643818547\\
17.6829349994659	8.01370806790568\\
17.8820822238922	7.96308694941008\\
18.082211971283	7.86668473000067\\
18.2817482948303	7.81194675754939\\
18.4820413589478	7.800862629438\\
18.6805729866028	7.80084693423395\\
18.8808255195618	7.80114947930034\\
19.0800879001617	7.80099185310746\\
19.2812404632568	7.8011984181981\\
19.4819047451019	7.8037018629369\\
19.6810040473938	7.80737793468351\\
19.8802058696747	7.80778929055742\\
20.081912279129	7.80842422173864\\
20.2811563014984	7.81327702564884\\
20.4812014102936	7.81614655889011\\
20.6810367107391	7.81611511608263\\
20.8817536830902	7.8164073447897\\
21.0817904472351	7.81649368399212\\
21.2805547714233	7.81879393713733\\
21.4817001819611	7.82025101883706\\
21.6815104484558	7.82105345996906\\
};
\addlegendentry{H(p,$\mathbb{W}$,$\phi$)}

\addplot [color=mycolor2, dotted, line width=1.4pt]
  table[row sep=crcr]{%
-0.117774486541748	0\\
0.280490159988403	5.01004814758321\\
0.483649969100952	6.58125594200134\\
0.682579755783081	7.43546088675415\\
0.883516311645508	8.30547631840922\\
1.08183526992798	9.79947514525108\\
1.28105568885803	11.4852891880853\\
1.4802074432373	13.5225527201427\\
1.6821653842926	15.779279662806\\
1.8819944858551	18.3177005873401\\
2.08089518547058	20.924079349164\\
2.28198099136353	23.5584393611047\\
2.48251509666443	26.2968107621151\\
2.68301582336426	29.0085142052109\\
2.8805935382843	31.6513171451911\\
3.08038401603699	34.0424879842239\\
3.2818455696106	35.9442474621927\\
3.48093891143799	37.620181898234\\
3.68154144287109	39.0402009464921\\
3.88178730010986	40.002763878236\\
4.08241820335388	40.7911141981991\\
4.28357720375061	41.2518007428867\\
4.48154473304749	41.3452060256601\\
4.68079805374146	41.2273358379846\\
4.88374209403992	40.7775567672026\\
5.08170390129089	40.1603621529442\\
5.28213214874268	39.3584878016996\\
5.48064565658569	38.4414985511979\\
5.6825487613678	37.2035576918124\\
5.88136625289917	36.1621549470681\\
6.08316135406494	34.9335389076995\\
6.28040170669556	33.7649651855327\\
6.48473501205444	32.5556951984043\\
6.68359088897705	31.4215310540775\\
6.8818998336792	30.3590771239661\\
7.08079028129578	29.4288444559117\\
7.28340697288513	28.5102477334512\\
7.4817681312561	27.6455868598057\\
7.68242835998535	26.8416545433883\\
7.88226366043091	26.092894716062\\
8.08400750160217	22.3217015314451\\
8.28272938728333	21.6891398417603\\
8.48226618766785	20.9980455679924\\
8.68258595466614	20.32173289701\\
8.88428592681885	19.7067657204088\\
9.08221316337585	19.1281227279209\\
9.28181314468384	18.5910780690894\\
9.48220610618591	18.036346768048\\
9.6818208694458	18.9019513344367\\
9.88172578811646	18.5243069614519\\
10.0821895599365	18.1952165330867\\
10.2835392951965	17.8873222638923\\
10.4819307327271	17.6090498681704\\
10.6823532581329	17.1570800762272\\
10.8809492588043	16.7628644918813\\
11.0806744098663	16.3650182976867\\
11.2827413082123	15.9323844290872\\
11.4807531833649	15.5324814002502\\
11.6827719211578	15.0835282181021\\
11.8808121681213	14.5330918118539\\
12.0822365283966	14.0700612358026\\
12.2819607257843	13.5309162371155\\
12.482825756073	13.0094650196548\\
12.6820688247681	13.8768285908668\\
12.8831100463867	13.2260533591586\\
13.0828530788422	12.6417166628174\\
13.2831931114197	13.0396586141099\\
13.4814794063568	12.5567486544518\\
13.6822912693024	12.4247238171294\\
13.8833720684052	12.1240601836556\\
14.0813863277435	11.8571121692981\\
14.2816686630249	11.6152620018859\\
14.4826619625092	11.3650247944024\\
14.6806054115295	11.0116585668762\\
14.8825922012329	10.6102443171718\\
15.0800611972809	10.2354556739546\\
15.2834858894348	9.89300860784614\\
15.4836881160736	9.57548757764806\\
15.6811118125916	9.3293640822962\\
15.8810572624207	9.11356538304075\\
16.0815989971161	8.92130998333853\\
16.2841033935547	8.69747925405955\\
16.4819452762604	8.47408157789424\\
16.6800713539124	8.2715294333268\\
16.881413936615	8.1134766655674\\
17.0820443630219	8.00581457808346\\
17.2811107635498	7.99852300447833\\
17.4821639060974	7.98763074788972\\
17.6829349994659	7.97435657218688\\
17.8820822238922	7.9205794945088\\
18.082211971283	7.82112121148361\\
18.2817482948303	7.76385912447984\\
18.4820413589478	7.7506163607135\\
18.6805729866028	7.74855502627953\\
18.8808255195618	7.74691551663029\\
19.0800879001617	7.74496149681459\\
19.2812404632568	7.74349603381707\\
19.4819047451019	7.74442190074292\\
19.6810040473938	7.74657260146394\\
19.8802058696747	7.74560350285448\\
20.081912279129	7.74494909081099\\
20.2811563014984	7.74848730863917\\
20.4812014102936	7.75021705136891\\
20.6810367107391	7.74909563439767\\
20.8817536830902	7.74835421852609\\
21.0817904472351	7.74746135002338\\
21.2805547714233	7.74876982322129\\
21.4817001819611	7.74929804564092\\
21.6815104484558	7.7492348773708\\
};
\addlegendentry{H(p,$\mathbb{W}$,$\hat{\phi}$)}

\end{axis}

\begin{axis}[%
width=0.0in,
height=0.0in,
at={(0in,0.0in)},
scale only axis,
xmin=0,
xmax=1,
ymin=0,
ymax=1,
axis line style={draw=none},
ticks=none,
axis x line*=bottom,
axis y line*=left
]
\end{axis}
\end{tikzpicture}%
%\centering
\vspace{-0.7em}
\caption{\rahel{Locational optimization cost (green) as well as estimated locational optimization cost over time (rose), applying Algorithm~\ref{alg:onelayerlearningalg} for an initially unknown $\phi_{2}$.}}
\label{fig:onelayerlearningcoveragecost}
\vspace{-0.7em}
\end{figure}

\begin{figure}[h!]
\begin{minipage}[t]{0.15\textwidth}
\centering
\includegraphics[trim={5.2cm 8.1cm 4.6cm 7.7cm},clip, width = 0.8\textwidth]{figures/config_7sec_one_layer_learning_take_one_0_1_compressed.pdf}
\centering
\end{minipage}
\hfill
\begin{minipage}[t]{0.15\textwidth}
\centering
\includegraphics[trim={5.2cm 8.1cm 4.6cm 7.7cm},clip, width = 0.8\textwidth]{figures/config_13sec_one_layer_learning_take_one_0_1_compressed.pdf}
\centering
\end{minipage}
\hfill
\begin{minipage}[t]{0.15\textwidth}
\centering
\includegraphics[trim={5.2cm 8.1cm 4.6cm 7.7cm},clip, width = 0.8\textwidth]{figures/config_24sec_one_layer_learning_take_one_0_1_compressed.pdf}
%\input{figures/config_25sec_one_layer_learning_2_220322.tex}
\centering
\end{minipage}
\caption{\rahel{Configurations of cars at 2, 8, and 19 seconds applying Algorithm~\ref{alg:onelayerlearningalg} in the described set-up. The agents' location and their predicted trajectory are given in red, the Voronoi partitions in green, the artificial setpoints in blue, and the traveled paths are visualized in light grey.}}
\label{pics:onelayerlearningconfig}
\vspace{-0.7em}
\end{figure}

\begin{figure} [h]
\centering
% This file was created by matlab2tikz.
%
%The latest updates can be retrieved from
%  http://www.mathworks.com/matlabcentral/fileexchange/22022-matlab2tikz-matlab2tikz
%where you can also make suggestions and rate matlab2tikz.
%
\definecolor{mycolor1}{rgb}{0.50000,0.75000,0.93000}%
%
\begin{tikzpicture}

\begin{axis}[%
height=0.85in,
width=0.40\textwidth,
yshift=0.8cm,
at={(0.0in,0.0in)},
scale only axis,
xmin=0,
xmax=21,
xlabel style={font=\color{white!15!black}},
xlabel style={font=\footnotesize},
xlabel style={yshift=0.6ex,},
xlabel={Time [sec]},
ymode=log,
ymin=0.001,
ymax=16,
yminorticks=true,
ylabel style={font=\color{white!15!black}},
ylabel style={font=\footnotesize},
ylabel style={xshift=0.6ex,},
ylabel={$\textup{Var}_{\textup{max}}$},
axis background/.style={fill=white},
xmajorgrids,
ymajorgrids,
tick label style={font=\footnotesize},
legend style={legend cell align=left, align=left, draw=white!15!black},
legend style={font=\footnotesize}
]
\addplot [color=mycolor1, line width=1.4pt]
  table[row sep=crcr]{%
0.0824691772460939	11.2337052528901\\
0.4823326587677	10.5834347445309\\
0.683907461166382	10.4061974846819\\
0.883398008346558	10.3354384794997\\
1.08480448722839	10.2818153068785\\
1.28220982551575	10.2204790628016\\
1.48219556808472	10.1323660477821\\
1.68094701766968	10.0053900948755\\
1.88225836753845	9.82408095091294\\
2.08258218765259	9.59112377292213\\
2.2808596611023	9.29913952176296\\
2.48196144104004	8.94703675651548\\
2.68293137550354	8.53658175788563\\
2.88405508995056	8.07128169054354\\
3.08245701789856	7.55296232245986\\
3.28137726783752	6.99860414966163\\
3.48191614151001	6.42310359934903\\
3.68113107681274	5.84317107685419\\
3.88187260627747	5.2664518989589\\
4.08333582878113	4.70616917185519\\
4.28269715309143	4.17263326435623\\
4.48381371498108	3.6762895750628\\
4.68152732849121	3.22386750685917\\
4.88279004096985	2.81580338147141\\
5.08373804092407	2.45072128311792\\
5.28267259597778	2.12887830728859\\
5.48323984146118	1.84837001349638\\
5.68104691505432	1.60601493124275\\
5.88389511108398	1.39661087706699\\
6.08287544250488	1.21484168417437\\
6.28336806297302	1.0590220222861\\
6.48188943862915	0.923991761114672\\
6.68472023010254	0.806200499104494\\
6.8842405796051	0.705939462445048\\
7.08238549232483	0.625696316323029\\
7.28184170722961	0.562319007430952\\
7.48439021110535	0.509984810043683\\
7.68310136795044	0.466189474829822\\
7.88349456787109	0.429028948957453\\
8.08247299194336	0.396833193076347\\
8.2843870639801	0.333261330811213\\
8.4837438583374	0.312696687996926\\
8.68244643211365	0.295508510609875\\
8.8829309463501	0.280165109836974\\
9.08465547561645	0.267516057492097\\
9.2827887058258	0.257512727187911\\
9.48206562995911	0.248932539309797\\
9.68278117179871	0.241664533717542\\
9.88401408195496	0.241339579086565\\
10.0824365139008	0.23538743273719\\
10.2821797847748	0.229637126899569\\
10.4837989330292	0.223927427930646\\
10.6825706481934	0.218196215831969\\
10.8826281547546	0.212176326133362\\
11.0825271129608	0.205808639400006\\
11.2814244747162	0.199032069362536\\
11.4827241420746	0.191898015255592\\
11.6821047782898	0.184710950527391\\
11.8838531494141	0.177591485211253\\
12.0817265033722	0.170498813820227\\
12.2838096141815	0.163439087492615\\
12.484863948822	0.156294093750672\\
12.6830825328827	0.148975366014786\\
12.8830816268921	0.161841139402378\\
13.0848724365234	0.153071112254505\\
13.2828008651733	0.144064740665686\\
13.4831900119781	0.1341764995115\\
13.6818553924561	0.124956744614517\\
13.8834168434143	0.115050977117623\\
14.0844420909882	0.106360875883648\\
14.2820920467377	0.0991818053534859\\
14.4824532985687	0.0930265314485906\\
14.6831182956696	0.0872917197029991\\
14.881488275528	0.0825366207049508\\
15.0828029632568	0.0777433250492063\\
15.2814306735992	0.0730081374593279\\
15.4833886146545	0.0684125273087926\\
15.6849286079407	0.0640936098605778\\
15.8820456981659	0.0602642504062827\\
16.0815531730652	0.0569464397761256\\
16.2836560726166	0.0540106877732077\\
16.48409075737	0.0513814085461716\\
16.6819352626801	0.0490157823376638\\
16.8810507774353	0.0468855763544784\\
17.0838777542114	0.0449503714112135\\
17.2820245742798	0.0431881698980211\\
17.481378030777	0.0416058370373098\\
17.6838738441467	0.0401777079008201\\
17.8839389801025	0.0388819537210294\\
18.0822820186615	0.0376947549997477\\
18.2837690830231	0.0365897220303644\\
18.4817387580872	0.0355625031166764\\
18.6826481342316	0.0346153559752507\\
18.8818845272064	0.0337424943581691\\
19.0817939758301	0.032935628248512\\
19.2805633068085	0.0321868891538603\\
19.4819049358368	0.0314901884846144\\
19.6831392765045	0.0308406835927521\\
19.8814782619476	0.0302331541395085\\
20.0811514377594	0.0296633590094357\\
20.2840866565704	0.029127819526675\\
20.4836422920227	0.0286236977570073\\
20.6818365573883	0.0281480096076934\\
20.8817939281464	0.0276980069061547\\
21.0834691047668	0.0272715168931879\\
21.2821485519409	0.0268666820895049\\
21.4827281951904	0.0264821955919711\\
21.6823594093323	0.0261161636836615\\
21.8826069355011	0.0257671785670892\\
22.0820765018463	0.0254341736544502\\
22.2821170806885	0.0251160933380441\\
22.4816207408905	0.0248118713004267\\
22.6831359386444	0.0245203431247609\\
22.8840643882751	0.0242406837178302\\
};
\addlegendentry{Maximal Variance}

\end{axis}

\begin{axis}[%
width=0.854in,
height=0.438in,
at={(0in,0in)},
scale only axis,
xmin=0,
xmax=1,
ymin=0,
ymax=1,
axis line style={draw=none},
ticks=none,
axis x line*=bottom,
axis y line*=left
]
\end{axis}
\end{tikzpicture}%
\vspace{-0.7em}
\caption{\rahel{Maximal variance over time applying Algorithm~\ref{alg:onelayerlearningalg} for an initially unknown $\phi_{2}$.}}
\label{fig:onelayerlearningmaxvar}
\vspace{-0.7em}
\end{figure}

\begin{figure} [h!]
\centering
% This file was created by matlab2tikz.
%
%The latest updates can be retrieved from
%  http://www.mathworks.com/matlabcentral/fileexchange/22022-matlab2tikz-matlab2tikz
%where you can also make suggestions and rate matlab2tikz.
%
\definecolor{mycolor1}{rgb}{0.00000,0.44700,0.74100}%
\definecolor{mycolor2}{rgb}{0.85000,0.32500,0.09800}%
\definecolor{mycolor3}{rgb}{0.92900,0.69400,0.12500}%
\definecolor{mycolor4}{rgb}{0.49400,0.18400,0.55600}%
\definecolor{mycolor5}{rgb}{0.46600,0.67400,0.18800}%
%
\begin{tikzpicture}

\begin{axis}[%
height=0.9in,
width=0.40\textwidth,
yshift=0.8cm,
at={(0.0in,0.0in)},
scale only axis,
xmin=0,
xmax=21,
xlabel style={font=\color{white!15!black}},
xlabel style={font=\footnotesize},
xlabel style={yshift=0.6ex,},
xlabel={Time [sec]},
ymin=-4,
ymax=15,
ylabel style={font=\color{white!15!black}},
ylabel={Mean value},
ylabel style={font=\footnotesize},
ylabel style={xshift = 0.6ex,},
axis background/.style={fill=white},
tick label style={font=\footnotesize},
title style={font=\bfseries},
title={},
xmajorgrids,
ymajorgrids,
legend style={legend cell align=left, align=left, draw=white!15!black},
legend style={font=\footnotesize}
]
\addplot [color=mycolor1, line width=1.4pt]
  table[row sep=crcr]{%
-3.30612330436707	0\\
-3.20646886825562	0\\
-3.10659484863281	0\\
-3.00668745040894	0\\
-2.9067156791687	0\\
-2.80673222541809	0\\
-2.70655851364136	0\\
-2.60655527114868	0\\
-2.50651769638062	0\\
-2.4063539981842	0\\
-2.30651407241821	0\\
-2.20650224685669	0\\
-2.10655479431152	0\\
-2.0065110206604	0\\
-1.90649728775024	0\\
-1.80657606124878	0\\
-1.70637445449829	0\\
-1.60666470527649	0\\
-1.506618309021	0\\
-1.40652709007263	0\\
-1.30670337677002	0\\
-1.20660214424133	0\\
-1.10664420127869	0\\
-1.00659852027893	0\\
-0.906519937515259	0\\
-0.806342649459839	0\\
-0.706490087509155	0\\
-0.6064706325531	0\\
-0.50660662651062	0\\
-0.4065833568573	0\\
-0.306743431091308	0\\
-0.206654357910156	0\\
-0.106565284729004	0\\
-0.00667457580566388	0\\
0.09395809173584	-0.523592972851006\\
0.193818521499634	-0.855111684769\\
0.293800067901612	-1.06259916437074\\
0.394488763809204	-1.21223519520478\\
0.49393744468689	-1.32548425885992\\
0.594617795944214	-1.40657175352908\\
0.694251489639282	-1.46157928690837\\
0.79453010559082	-1.49365272895767\\
0.894382190704346	-1.51100884102226\\
0.993970346450806	-1.51916166657395\\
1.09413213729858	-1.51958476809307\\
1.19380946159363	-1.51434617276357\\
1.29390091896057	-1.50513443766488\\
1.39381856918335	-1.49172777701108\\
1.49456901550293	-1.47318504172449\\
1.59414882659912	-1.44942927870045\\
1.69416828155518	-1.42300841490055\\
1.79447693824768	-1.39205969277873\\
1.89404244422913	-1.35788475730192\\
1.99382014274597	-1.32189493703538\\
2.09440393447876	-1.28427707926119\\
2.19404430389404	-1.24561435257124\\
2.29452223777771	-1.20570119056322\\
2.39429755210876	-1.16568352466697\\
2.49435157775879	-1.12502944665448\\
2.59410924911499	-1.08440241800633\\
2.69441957473755	-1.04370351034652\\
2.79399509429932	-1.00431841472903\\
2.89404768943787	-0.966123682102079\\
2.99460310935974	-0.930447311654802\\
3.09801001548767	-0.896908959893494\\
3.19434351921082	-0.866198190203249\\
3.29437227249146	-0.837661575383663\\
3.39399857521057	-0.811267117880448\\
3.49391670227051	-0.787054385614283\\
3.59435768127441	-0.764710867414124\\
3.69422359466553	-0.744717974424816\\
3.79430932998657	-0.726202330970636\\
3.89424223899841	-0.709368408374075\\
3.9946937084198	-0.69366281177156\\
4.09427065849304	-0.679542127718264\\
4.19411582946777	-0.66622819329757\\
4.29436988830566	-0.654802382131152\\
4.3939537525177	-0.644038089567118\\
4.49408383369446	-0.634674030911583\\
4.59400362968445	-0.625636254497863\\
4.69436283111572	-0.617691773105946\\
4.79416580200195	-0.60986499451576\\
4.89397211074829	-0.603187688623734\\
4.99416513442993	-0.59646573752741\\
5.09401054382324	-0.590673307151064\\
5.19386262893677	-0.584619026230087\\
5.29427857398987	-0.579611936491801\\
5.39438290596008	-0.574521513704269\\
5.49426670074463	-0.57033942760151\\
5.59454793930054	-0.565653142774948\\
5.69384164810181	-0.56183507315157\\
5.79383273124695	-0.557486154505028\\
5.89450521469116	-0.554229324307983\\
5.99575634002686	-0.550500691528015\\
6.09436769485474	-0.548090728855556\\
6.19436354637146	-0.545349779437913\\
6.29518957138062	-0.543646105845966\\
6.39615125656128	-0.541170931739288\\
6.49447984695435	-0.539677350812568\\
6.59447736740112	-0.53757392200383\\
6.69415373802185	-0.536655038728796\\
6.79385895729065	-0.535104051907638\\
6.89376873970032	-0.534499110143642\\
6.99494786262512	-0.533097405138463\\
7.0942928314209	-0.532257847942915\\
7.19487829208374	-0.530598898208716\\
7.29400796890259	-0.529869772789148\\
7.39391846656799	-0.528287567626357\\
7.4946560382843	-0.527691531095115\\
7.5947274684906	-0.526175594128823\\
7.69392747879028	-0.52556677902038\\
7.79379243850708	-0.523982595204593\\
7.89449687004089	-0.52332144482099\\
7.99416227340698	-0.521983018028314\\
8.09386534690857	-0.5229829498223\\
8.19469327926636	-0.524374015256674\\
8.29435892105102	-0.526336757277668\\
8.39448161125183	-0.528239966464629\\
8.49439902305603	-0.52912963111265\\
8.59412689208984	-0.530189742257867\\
8.69434328079224	-0.531115671625995\\
8.79458684921265	-0.532262832643147\\
8.89440197944641	-0.532918043466611\\
8.99430246353149	-0.533643561340376\\
9.09426374435425	-0.533768172123885\\
9.19427580833435	-0.534009804736568\\
9.29399795532226	-0.533879956368204\\
9.39400095939636	-0.533922686863701\\
9.49448127746582	-0.533659014499634\\
9.59459300041199	-0.533581227553558\\
9.69432420730591	-0.533227337426687\\
9.79443378448486	-0.533057171955648\\
9.89469833374023	-0.532606054009771\\
9.99415941238403	-0.532311509839693\\
10.0939476013184	-0.531803108261515\\
10.1943997859955	-0.531447214359509\\
10.2945494174957	-0.530886345311862\\
10.396132183075	-0.530542789303102\\
10.4946388721466	-0.530110048317179\\
10.5941822052002	-0.529866638256123\\
10.6943089485168	-0.529540863086481\\
10.7942625999451	-0.52933679520125\\
10.8941337585449	-0.529012700838484\\
10.9948989868164	-0.528782313128499\\
11.0947746753693	-0.528472833099694\\
11.1950318336487	-0.528226442340825\\
11.2942704677582	-0.5279350158782\\
11.3940650939941	-0.527650198320837\\
11.4944948673248	-0.527358628936144\\
11.5940901756287	-0.527064503734936\\
11.6940326213837	-0.526972688049623\\
11.7943291187286	-0.527038482934155\\
11.8944835186005	-0.527371219860634\\
11.9943615913391	-0.527671394864967\\
12.0940055370331	-0.528147720291\\
12.1948394298553	-0.528499515529909\\
12.2944278240204	-0.529087128904614\\
12.3944353580475	-0.529513957574569\\
12.4942023277283	-0.530370362367419\\
12.5944509029388	-0.530945196775683\\
12.6944336414337	-0.531890968031234\\
12.7945048332214	-0.532537398168794\\
12.8942729949951	-0.533620724134305\\
12.9943153381348	-0.534358376333589\\
13.0942151069641	-0.535474234973492\\
13.1943997859955	-0.536225176902672\\
13.2945842266083	-0.537358102837459\\
13.3941556930542	-0.538140930104237\\
13.4959408760071	-0.539274338775794\\
13.5952534198761	-0.540059000293667\\
13.6944038391113	-0.541196341743248\\
13.7940704345703	-0.542113063988051\\
13.8939878463745	-0.543606013980671\\
13.994055223465	-0.545007538133575\\
14.0940544128418	-0.546499259698024\\
14.1938771724701	-0.54787354883517\\
14.2945451259613	-0.549243075501415\\
14.3944327354431	-0.550509048545472\\
14.4946231365204	-0.55175403907446\\
14.5942589759827	-0.552857921335026\\
14.6945983886719	-0.553931743903426\\
14.7947313308716	-0.554733258010501\\
14.894251537323	-0.555691797416536\\
14.9940084934235	-0.556405140011385\\
15.0941972255707	-0.557449545906898\\
15.1944472312927	-0.558183365323217\\
15.2941483974457	-0.559167896629205\\
15.3943430900574	-0.559900373856337\\
15.4941553592682	-0.560990608220365\\
15.5943800926208	-0.561955732488208\\
15.6945120811462	-0.563114591643263\\
15.7940501689911	-0.564164096595036\\
15.8942977905273	-0.565148606631304\\
15.9943353652954	-0.566027469652454\\
16.0946883678436	-0.566820547996024\\
16.1947559833527	-0.567471379052634\\
16.2944976806641	-0.568037016406478\\
16.393835735321	-0.568459762649056\\
16.4941844463348	-0.568919720554717\\
16.5940634727478	-0.569287530033264\\
16.6945108890533	-0.569718525688387\\
16.7945806503296	-0.570091328469289\\
16.8942512989044	-0.570488928502421\\
16.9942724227905	-0.570847847097312\\
17.0940279483795	-0.571224734689558\\
17.1941775798798	-0.571596267474916\\
17.2939688682556	-0.571992975536219\\
17.3942429542541	-0.572362656938459\\
17.4944741249084	-0.572764800119963\\
17.594543170929	-0.573184237447855\\
17.6945220947266	-0.573668559830999\\
17.7943229198456	-0.574181097262837\\
17.8947572231293	-0.574691718752184\\
17.993910741806	-0.575238523367311\\
18.0943593502045	-0.575707011462313\\
18.1944233894348	-0.576191611134924\\
18.2944530963898	-0.576570686005176\\
18.394628238678	-0.576955361508389\\
18.4941641807556	-0.57729530328627\\
18.5942115306854	-0.577640559760695\\
18.6940993785858	-0.577988553016823\\
18.7940663814545	-0.578338864128099\\
18.894183588028	-0.578689468884008\\
18.9942590713501	-0.579043203877511\\
19.0946690559387	-0.579400801532216\\
19.1968643188477	-0.579758866763342\\
19.29445977211	-0.580122543887761\\
19.3940042972565	-0.580501479224807\\
19.4939903736114	-0.580874282698765\\
19.5939936161041	-0.581247420367077\\
19.6944453239441	-0.581610875203012\\
19.7941777229309	-0.581973792141561\\
19.8944935321808	-0.582335608069194\\
19.9946469783783	-0.582705841292679\\
20.0940992355347	-0.58308384564971\\
20.1944553375244	-0.583456392404088\\
20.2938937664032	-0.583811107441406\\
20.3942555904388	-0.584165833746106\\
20.4941558361053	-0.58452529628223\\
20.5939862251282	-0.584883614501649\\
20.6948296546936	-0.585240436958294\\
20.7943338871002	-0.585595322757072\\
20.8942765712738	-0.585948958901191\\
20.9943639755249	-0.586302104879415\\
21.0939878940582	-0.586661920850055\\
21.1938986301422	-0.587026321972019\\
21.2944957733154	-0.587379289575566\\
21.3939191818237	-0.587731915584648\\
21.4943480014801	-0.588077436405783\\
21.5946044445038	-0.588421171706429\\
21.6944186210632	-0.588761843539672\\
21.7945346355438	-0.58909888193503\\
21.8940104961395	-0.589434487137652\\
21.9942833900452	-0.589768676905777\\
22.0937165737152	-0.59010319269359\\
22.1942743778229	-0.590436106687569\\
22.2950155258179	-0.590771751372259\\
22.3950175762177	-0.591106130602185\\
22.4941741943359	-0.591437066029115\\
22.5940436840057	-0.591766500430097\\
22.6943318367004	-0.592095767175607\\
22.7942213535309	-0.592427228362695\\
22.8943209171295	-0.592766329475225\\
22.9939038276672	-0.593112433166596\\
};
\addlegendentry{$\theta_{x_{p}^{2}}$}

\addplot [color=mycolor2, line width=1.4pt]
  table[row sep=crcr]{%
-3.30612330436707	0\\
-3.20646886825562	0\\
-3.10659484863281	0\\
-3.00668745040894	0\\
-2.9067156791687	0\\
-2.80673222541809	0\\
-2.70655851364136	0\\
-2.60655527114868	0\\
-2.50651769638062	0\\
-2.4063539981842	0\\
-2.30651407241821	0\\
-2.20650224685669	0\\
-2.10655479431152	0\\
-2.0065110206604	0\\
-1.90649728775024	0\\
-1.80657606124878	0\\
-1.70637445449829	0\\
-1.60666470527649	0\\
-1.506618309021	0\\
-1.40652709007263	0\\
-1.30670337677002	0\\
-1.20660214424133	0\\
-1.10664420127869	0\\
-1.00659852027893	0\\
-0.906519937515259	0\\
-0.806342649459839	0\\
-0.706490087509155	0\\
-0.6064706325531	0\\
-0.50660662651062	0\\
-0.4065833568573	0\\
-0.306743431091308	0\\
-0.206654357910156	0\\
-0.106565284729004	0\\
-0.00667457580566388	0\\
0.09395809173584	-0.544923557234917\\
0.193818521499634	-0.875058488754888\\
0.293800067901612	-1.07688036028716\\
0.394488763809204	-1.21580744969782\\
0.49393744468689	-1.33187536508723\\
0.594617795944214	-1.43262687346186\\
0.694251489639282	-1.51450284580443\\
0.79453010559082	-1.56899080638607\\
0.894382190704346	-1.60244386334313\\
0.993970346450806	-1.62817923530099\\
1.09413213729858	-1.64973050887863\\
1.19380946159363	-1.66390996125313\\
1.29390091896057	-1.66919639852551\\
1.39381856918335	-1.66771989852657\\
1.49456901550293	-1.65975043501578\\
1.59414882659912	-1.64431867439419\\
1.69416828155518	-1.62102096162528\\
1.79447693824768	-1.59328207290673\\
1.89404244422913	-1.56002168363011\\
1.99382014274597	-1.5211106267061\\
2.09440393447876	-1.47554075588323\\
2.19404430389404	-1.4250763322068\\
2.29452223777771	-1.3703543420927\\
2.39429755210876	-1.31386072459372\\
2.49435157775879	-1.25534903139578\\
2.59410924911499	-1.19602222024287\\
2.69441957473755	-1.13518384916324\\
2.79399509429932	-1.07470185677971\\
2.89404768943787	-1.0146373136447\\
2.99460310935974	-0.955061440322424\\
3.09801001548767	-0.895157204881343\\
3.19434351921082	-0.83612592763393\\
3.29437227249146	-0.778035692374942\\
3.39399857521057	-0.72278837448971\\
3.49391670227051	-0.67011212019338\\
3.59435768127441	-0.620548134578257\\
3.69422359466553	-0.573143284541175\\
3.79430932998657	-0.529153746073234\\
3.89424223899841	-0.488564508261561\\
3.9946937084198	-0.4521274347087\\
4.09427065849304	-0.419473699042555\\
4.19411582946777	-0.390530754822521\\
4.29436988830566	-0.365068486970358\\
4.3939537525177	-0.342843502467986\\
4.49408383369446	-0.324442845478302\\
4.59400362968445	-0.309009947080654\\
4.69436283111572	-0.296799891047613\\
4.79416580200195	-0.286944196591094\\
4.89397211074829	-0.27991741811536\\
4.99416513442993	-0.274757234094182\\
5.09401054382324	-0.27199301215569\\
5.19386262893677	-0.270677685141322\\
5.29427857398987	-0.271220980721864\\
5.39438290596008	-0.272662845349714\\
5.49426670074463	-0.275393170371899\\
5.59454793930054	-0.278799115796062\\
5.69384164810181	-0.283406610860766\\
5.79383273124695	-0.288113504501666\\
5.89450521469116	-0.293497016291553\\
5.99575634002686	-0.29884895256555\\
6.09436769485474	-0.304879739694968\\
6.19436354637146	-0.310475868017591\\
6.29518957138062	-0.316639285286556\\
6.39615125656128	-0.322060048868479\\
6.49447984695435	-0.328164755066553\\
6.59447736740112	-0.333234066525819\\
6.69415373802185	-0.338764351492301\\
6.79385895729065	-0.343413965853586\\
6.89376873970032	-0.347883328249139\\
6.99494786262512	-0.351763184945952\\
7.0942928314209	-0.354798384439562\\
7.19487829208374	-0.357514022203134\\
7.29400796890259	-0.359606403875119\\
7.39391846656799	-0.361526872888447\\
7.4946560382843	-0.362956712279583\\
7.5947274684906	-0.364314636809013\\
7.69392747879028	-0.365296858689419\\
7.79379243850708	-0.366251301230051\\
7.89449687004089	-0.36689949465466\\
7.99416227340698	-0.367522682978432\\
8.09386534690857	-0.367735034779969\\
8.19469327926636	-0.367755299330668\\
8.29435892105102	-0.367553198656395\\
8.39448161125183	-0.367227707512598\\
8.49439902305603	-0.36712253105263\\
8.59412689208984	-0.366852353034176\\
8.69434328079224	-0.366896755487019\\
8.79458684921265	-0.366863711057562\\
8.89440197944641	-0.366872547751157\\
8.99430246353149	-0.36658500636814\\
9.09426374435425	-0.365796339362333\\
9.19427580833435	-0.364776436498914\\
9.29399795532226	-0.363681977893606\\
9.39400095939636	-0.36242699887832\\
9.49448127746582	-0.361217956589471\\
9.59459300041199	-0.359901277008561\\
9.69432420730591	-0.358617421351781\\
9.79443378448486	-0.357210030868202\\
9.89469833374023	-0.355956014781\\
9.99415941238403	-0.354547231019581\\
10.0939476013184	-0.353360917527183\\
10.1943997859955	-0.351989322492784\\
10.2945494174957	-0.35100703856736\\
10.396132183075	-0.349989824175765\\
10.4946388721466	-0.349545359373394\\
10.5941822052002	-0.348971589963995\\
10.6943089485168	-0.348917482088765\\
10.7942625999451	-0.348548337797055\\
10.8941337585449	-0.34868038181871\\
10.9948989868164	-0.348433716621997\\
11.0947746753693	-0.348815205992409\\
11.1950318336487	-0.34877122625403\\
11.2942704677582	-0.349345937647108\\
11.3940650939941	-0.349536563204445\\
11.4944948673248	-0.350294539753634\\
11.5940901756287	-0.350667959324625\\
11.6940326213837	-0.351753585302021\\
11.7943291187286	-0.352777752120769\\
11.8944835186005	-0.35441539958223\\
11.9943615913391	-0.35570798205945\\
12.0940055370331	-0.357534517097381\\
12.1948394298553	-0.359077805569882\\
12.2944278240204	-0.361206963709527\\
12.3944353580475	-0.362993039734826\\
12.4942023277283	-0.365489417537201\\
12.5944509029388	-0.367463613173481\\
12.6944336414337	-0.370084697377123\\
12.7945048332214	-0.372098839691579\\
12.8942729949951	-0.374809060231809\\
12.9943153381348	-0.376789042140695\\
13.0942151069641	-0.379535644050374\\
13.1943997859955	-0.381512069784875\\
13.2945842266083	-0.384232767665771\\
13.3941556930542	-0.38612487210419\\
13.4959408760071	-0.388911165480593\\
13.5952534198761	-0.390851696542208\\
13.6944038391113	-0.393756425309457\\
13.7940704345703	-0.396022814389255\\
13.8939878463745	-0.400102754863255\\
13.994055223465	-0.404142873230338\\
14.0940544128418	-0.408505949330497\\
14.1938771724701	-0.412591075030051\\
14.2945451259613	-0.416359739164641\\
14.3944327354431	-0.419950117212817\\
14.4946231365204	-0.423401675363166\\
14.5942589759827	-0.426353402065754\\
14.6945983886719	-0.428795436228842\\
14.7947313308716	-0.430430126564925\\
14.894251537323	-0.432647379896181\\
14.9940084934235	-0.434263742166596\\
15.0941972255707	-0.436692589772946\\
15.1944472312927	-0.438427385381353\\
15.2941483974457	-0.440834431438446\\
15.3943430900574	-0.442810921910265\\
15.4941553592682	-0.445649279800513\\
15.5943800926208	-0.448345201485317\\
15.6945120811462	-0.451413851259773\\
15.7940501689911	-0.454270020933684\\
15.8942977905273	-0.456935386617396\\
15.9943353652954	-0.459428791049724\\
16.0946883678436	-0.461925242553114\\
16.1947559833527	-0.464273848028157\\
16.2944976806641	-0.466692371497395\\
16.393835735321	-0.469062592954309\\
16.4941844463348	-0.471470580139432\\
16.5940634727478	-0.473793370352297\\
16.6945108890533	-0.476026644700866\\
16.7945806503296	-0.478147451282098\\
16.8942512989044	-0.480251220916074\\
16.9942724227905	-0.482255575766182\\
17.0940279483795	-0.484295963715724\\
17.1941775798798	-0.486237709282221\\
17.2939688682556	-0.488102402648181\\
17.3942429542541	-0.489897051298072\\
17.4944741249084	-0.491573047520468\\
17.594543170929	-0.493155234717683\\
17.6945220947266	-0.494608973085327\\
17.7943229198456	-0.495954866613395\\
17.8947572231293	-0.497233203340315\\
17.993910741806	-0.498396227005905\\
18.0943593502045	-0.499650785484882\\
18.1944233894348	-0.500824182354727\\
18.2944530963898	-0.502147500006473\\
18.394628238678	-0.503409810945826\\
18.4941641807556	-0.50470309388664\\
18.5942115306854	-0.505940339444841\\
18.6940993785858	-0.507133333499809\\
18.7940663814545	-0.508278668289087\\
18.894183588028	-0.50938010675327\\
18.9942590713501	-0.510438974442145\\
19.0946690559387	-0.511458772532414\\
19.1968643188477	-0.512441569888779\\
19.29445977211	-0.51338649994041\\
19.3940042972565	-0.514287392008939\\
19.4939903736114	-0.515162090466916\\
19.5939936161041	-0.516007253171395\\
19.6944453239441	-0.516832268846798\\
19.7941777229309	-0.517630697968286\\
19.8944935321808	-0.518404496571918\\
19.9946469783783	-0.519147499706293\\
20.0940992355347	-0.519861272006146\\
20.1944553375244	-0.520557780636139\\
20.2938937664032	-0.521248636072407\\
20.3942555904388	-0.521919441183584\\
20.4941558361053	-0.522564426190009\\
20.5939862251282	-0.523191391883586\\
20.6948296546936	-0.523802076281193\\
20.7943338871002	-0.524397565308675\\
20.8942765712738	-0.524977084074983\\
20.9943639755249	-0.525541727735572\\
21.0939878940582	-0.52608564505946\\
21.1938986301422	-0.526611089321257\\
21.2944957733154	-0.527132436625486\\
21.3939191818237	-0.527640348384324\\
21.4943480014801	-0.528139926788484\\
21.5946044445038	-0.52862908248138\\
21.6944186210632	-0.529108540234603\\
21.7945346355438	-0.529580041945465\\
21.8940104961395	-0.530041000017896\\
21.9942833900452	-0.53049187500342\\
22.0937165737152	-0.53092832233115\\
22.1942743778229	-0.531355701268858\\
22.2950155258179	-0.531768343560884\\
22.3950175762177	-0.53217274390726\\
22.4941741943359	-0.532570070166464\\
22.5940436840057	-0.532959433603864\\
22.6943318367004	-0.533340069919291\\
22.7942213535309	-0.533719583276714\\
22.8943209171295	-0.534093859776057\\
22.9939038276672	-0.534455444937784\\
};
\addlegendentry{$\theta_{y_{p}^{2}}$}


\addplot [color=mycolor3, line width=1.4pt]
  table[row sep=crcr]{%
-3.30612330436707	0\\
-3.20646886825562	0\\
-3.10659484863281	0\\
-3.00668745040894	0\\
-2.9067156791687	0\\
-2.80673222541809	0\\
-2.70655851364136	0\\
-2.60655527114868	0\\
-2.50651769638062	0\\
-2.4063539981842	0\\
-2.30651407241821	0\\
-2.20650224685669	0\\
-2.10655479431152	0\\
-2.0065110206604	0\\
-1.90649728775024	0\\
-1.80657606124878	0\\
-1.70637445449829	0\\
-1.60666470527649	0\\
-1.506618309021	0\\
-1.40652709007263	0\\
-1.30670337677002	0\\
-1.20660214424133	0\\
-1.10664420127869	0\\
-1.00659852027893	0\\
-0.906519937515259	0\\
-0.806342649459839	0\\
-0.706490087509155	0\\
-0.6064706325531	0\\
-0.50660662651062	0\\
-0.4065833568573	0\\
-0.306743431091308	0\\
-0.206654357910156	0\\
-0.106565284729004	0\\
-0.00667457580566388	0\\
0.09395809173584	-1.66533245531147\\
0.193818521499634	-2.29479174781443\\
0.293800067901612	-2.63191783511758\\
0.394488763809204	-2.83503547020189\\
0.49393744468689	-2.94242040705535\\
0.594617795944214	-2.98921424113996\\
0.694251489639282	-2.99700262769613\\
0.79453010559082	-2.98068741830184\\
0.894382190704346	-2.95230794502231\\
0.993970346450806	-2.90846752284028\\
1.09413213729858	-2.84638420522924\\
1.19380946159363	-2.77336805705954\\
1.29390091896057	-2.69655884045255\\
1.39381856918335	-2.61342509774931\\
1.49456901550293	-2.52245402144626\\
1.59414882659912	-2.42482583583046\\
1.69416828155518	-2.32742791998726\\
1.79447693824768	-2.2253910619479\\
1.89404244422913	-2.12037752626338\\
1.99382014274597	-2.0159664326552\\
2.09440393447876	-1.91168771942193\\
2.19404430389404	-1.80890323799031\\
2.29452223777771	-1.70645499219563\\
2.39429755210876	-1.60702326073988\\
2.49435157775879	-1.50857394679269\\
2.59410924911499	-1.41281657224454\\
2.69441957473755	-1.31899174193904\\
2.79399509429932	-1.23003133484963\\
2.89404768943787	-1.14521053884755\\
2.99460310935974	-1.06721864813835\\
3.09801001548767	-0.994917227740643\\
3.19434351921082	-0.929611001705553\\
3.29437227249146	-0.869766403520998\\
3.39399857521057	-0.815247549033757\\
3.49391670227051	-0.765977714850578\\
3.59435768127441	-0.721249835552953\\
3.69422359466553	-0.682030289398426\\
3.79430932998657	-0.646417113943926\\
3.89424223899841	-0.614779237663015\\
3.9946937084198	-0.585835580367302\\
4.09427065849304	-0.560548475929636\\
4.19411582946777	-0.537224821620612\\
4.29436988830566	-0.518039565553863\\
4.3939537525177	-0.500371283506865\\
4.49408383369446	-0.485678152762489\\
4.59400362968445	-0.471773003475391\\
4.69436283111572	-0.460281157939676\\
4.79416580200195	-0.44914439537888\\
4.89397211074829	-0.440426984815645\\
4.99416513442993	-0.431683381717676\\
5.09401054382324	-0.424876745685651\\
5.19386262893677	-0.417626044754002\\
5.29427857398987	-0.412469980678793\\
5.39438290596008	-0.407137359467924\\
5.49426670074463	-0.403604433178089\\
5.59454793930054	-0.399172648904198\\
5.69384164810181	-0.396348801379474\\
5.79383273124695	-0.392527540372384\\
5.89450521469116	-0.390663253087496\\
5.99575634002686	-0.387870748116484\\
6.09436769485474	-0.387289874417434\\
6.19436354637146	-0.385959381351313\\
6.29518957138062	-0.386335181574054\\
6.39615125656128	-0.385330991329738\\
6.49447984695435	-0.38599995655909\\
6.59447736740112	-0.385581257026814\\
6.69415373802185	-0.387054375712324\\
6.79385895729065	-0.387401523345233\\
6.89376873970032	-0.389246003104063\\
6.99494786262512	-0.389660915459388\\
7.0942928314209	-0.390916076419472\\
7.19487829208374	-0.390755575259249\\
7.29400796890259	-0.392099839478277\\
7.39391846656799	-0.391968225959687\\
7.4946560382843	-0.393382332558019\\
7.5947274684906	-0.393228435479188\\
7.69392747879028	-0.39449275597994\\
7.79379243850708	-0.394110060777138\\
7.89449687004089	-0.395164279351761\\
7.99416227340698	-0.395072237777484\\
8.09386534690857	-0.398724529238905\\
8.19469327926636	-0.402976247779904\\
8.29435892105102	-0.408142507521902\\
8.39448161125183	-0.41316119064777\\
8.49439902305603	-0.416444686647836\\
8.59412689208984	-0.419953337519622\\
8.69434328079224	-0.423199518584653\\
8.79458684921265	-0.426733715193812\\
8.89440197944641	-0.429526212363839\\
8.99430246353149	-0.432467964044172\\
9.09426374435425	-0.434584264039131\\
9.19427580833435	-0.436878278697243\\
9.29399795532226	-0.438618579032585\\
9.39400095939636	-0.440552232776788\\
9.49448127746582	-0.441997441534113\\
9.59459300041199	-0.443612732733442\\
9.69432420730591	-0.444884982154235\\
9.79443378448486	-0.446339918537497\\
9.89469833374023	-0.447470657409283\\
9.99415941238403	-0.448713795841726\\
10.0939476013184	-0.449704790566955\\
10.1943997859955	-0.450808190214907\\
10.2945494174957	-0.451711011258212\\
10.396132183075	-0.452681677513709\\
10.4946388721466	-0.453506786553845\\
10.5941822052002	-0.454396152567014\\
10.6943089485168	-0.455377981553696\\
10.7942625999451	-0.456381092773448\\
10.8941337585449	-0.457404982525738\\
10.9948989868164	-0.458429980345159\\
11.0947746753693	-0.459466306612505\\
11.1950318336487	-0.460450403356113\\
11.2942704677582	-0.461447167910698\\
11.3940650939941	-0.462322420184492\\
11.4944948673248	-0.463262873259055\\
11.5940901756287	-0.464048066566228\\
11.6940326213837	-0.465060145882703\\
11.7943291187286	-0.465981394517989\\
11.8944835186005	-0.467248172835831\\
11.9943615913391	-0.468277219127629\\
12.0940055370331	-0.469609440801317\\
12.1948394298553	-0.470525217742541\\
12.2944278240204	-0.471744680484569\\
12.3944353580475	-0.472576378331706\\
12.4942023277283	-0.474060410856271\\
12.5944509029388	-0.475002458749476\\
12.6944336414337	-0.476489461942876\\
12.7945048332214	-0.477472480839101\\
12.8942729949951	-0.479104405150538\\
12.9943153381348	-0.480201779720593\\
13.0942151069641	-0.481816913926739\\
13.1943997859955	-0.482887174247763\\
13.2945842266083	-0.484464310655265\\
13.3941556930542	-0.485539806692981\\
13.4959408760071	-0.487067375563473\\
13.5952534198761	-0.488111133250587\\
13.6944038391113	-0.489596895635889\\
13.7940704345703	-0.490779697747803\\
13.8939878463745	-0.492647273642468\\
13.994055223465	-0.494339642335103\\
14.0940544128418	-0.496099092466156\\
14.1938771724701	-0.497690033521768\\
14.2945451259613	-0.499223563555187\\
14.3944327354431	-0.500588022386063\\
14.4946231365204	-0.501873642568183\\
14.5942589759827	-0.502949985295963\\
14.6945983886719	-0.503927049187597\\
14.7947313308716	-0.504544139963294\\
14.894251537323	-0.50530470525673\\
14.9940084934235	-0.505760337985564\\
15.0941972255707	-0.506579049158585\\
15.1944472312927	-0.507027700680185\\
15.2941483974457	-0.507743316137779\\
15.3943430900574	-0.508166093184192\\
15.4941553592682	-0.508988233039442\\
15.5943800926208	-0.509672790260588\\
15.6945120811462	-0.510590509757154\\
15.7940501689911	-0.511393408073467\\
15.8942977905273	-0.512137497537395\\
15.9943353652954	-0.512766311128413\\
16.0946883678436	-0.513294867367918\\
16.1947559833527	-0.513657459981452\\
16.2944976806641	-0.513918816770751\\
16.393835735321	-0.514008477132361\\
16.4941844463348	-0.514156900890004\\
16.5940634727478	-0.514199189895841\\
16.6945108890533	-0.514338030008197\\
16.7945806503296	-0.5144129079696\\
16.8942512989044	-0.514529646266919\\
16.9942724227905	-0.514605648037733\\
17.0940279483795	-0.514710504847962\\
17.1941775798798	-0.514815409561734\\
17.2939688682556	-0.514959206937551\\
17.3942429542541	-0.515074576822716\\
17.4944741249084	-0.515237845006151\\
17.594543170929	-0.515429052584548\\
17.6945220947266	-0.515709814564449\\
17.7943229198456	-0.516033585086422\\
17.8947572231293	-0.516362188002034\\
17.993910741806	-0.516744132624893\\
18.0943593502045	-0.517033055590016\\
18.1944233894348	-0.517347648152906\\
18.2944530963898	-0.517533880128813\\
18.394628238678	-0.517731067066702\\
18.4941641807556	-0.517875543662232\\
18.5942115306854	-0.518030157675259\\
18.6940993785858	-0.518191462840257\\
18.7940663814545	-0.518358749744614\\
18.894183588028	-0.518529375035364\\
18.9942590713501	-0.518706820623319\\
19.0946690559387	-0.518891843614163\\
19.1968643188477	-0.519080092942271\\
19.29445977211	-0.519278170845944\\
19.3940042972565	-0.519498617077161\\
19.4939903736114	-0.519713266402203\\
19.5939936161041	-0.519930726802905\\
19.6944453239441	-0.520137529400774\\
19.7941777229309	-0.52034588844284\\
19.8944935321808	-0.520554997864945\\
19.9946469783783	-0.520777332822806\\
20.0940992355347	-0.52101213654818\\
20.1944553375244	-0.52124167740326\\
20.2938937664032	-0.521449585516031\\
20.3942555904388	-0.521659524345147\\
20.4941558361053	-0.521877123081886\\
20.5939862251282	-0.522095087109392\\
20.6948296546936	-0.522312939272471\\
20.7943338871002	-0.522530036518077\\
20.8942765712738	-0.522747221590574\\
20.9943639755249	-0.52296550091749\\
21.0939878940582	-0.52319455411493\\
21.1938986301422	-0.523431476100203\\
21.2944957733154	-0.523654785846812\\
21.3939191818237	-0.523879314253453\\
21.4943480014801	-0.524096107897488\\
21.5946044445038	-0.524312055255997\\
21.6944186210632	-0.524525431444488\\
21.7945346355438	-0.52473523189202\\
21.8940104961395	-0.524944431503847\\
21.9942833900452	-0.52515322529705\\
22.0937165737152	-0.525363192771472\\
22.1942743778229	-0.525572417272993\\
22.2950155258179	-0.525786401549794\\
22.3950175762177	-0.526000049975345\\
22.4941741943359	-0.526210500558552\\
22.5940436840057	-0.52642029855404\\
22.6943318367004	-0.526631163943211\\
22.7942213535309	-0.526845451871427\\
22.8943209171295	-0.527070594220625\\
22.9939038276672	-0.527306652504922\\
};
\addlegendentry{$\theta_{x_{p}}$}

\addplot [color=mycolor4, line width=1.4pt]
  table[row sep=crcr]{%
-3.30612330436707	0\\
-3.20646886825562	0\\
-3.10659484863281	0\\
-3.00668745040894	0\\
-2.9067156791687	0\\
-2.80673222541809	0\\
-2.70655851364136	0\\
-2.60655527114868	0\\
-2.50651769638062	0\\
-2.4063539981842	0\\
-2.30651407241821	0\\
-2.20650224685669	0\\
-2.10655479431152	0\\
-2.0065110206604	0\\
-1.90649728775024	0\\
-1.80657606124878	0\\
-1.70637445449829	0\\
-1.60666470527649	0\\
-1.506618309021	0\\
-1.40652709007263	0\\
-1.30670337677002	0\\
-1.20660214424133	0\\
-1.10664420127869	0\\
-1.00659852027893	0\\
-0.906519937515259	0\\
-0.806342649459839	0\\
-0.706490087509155	0\\
-0.6064706325531	0\\
-0.50660662651062	0\\
-0.4065833568573	0\\
-0.306743431091308	0\\
-0.206654357910156	0\\
-0.106565284729004	0\\
-0.00667457580566388	0\\
0.09395809173584	-1.68436139787107\\
0.193818521499634	-2.32020176092163\\
0.293800067901612	-2.6552703415573\\
0.394488763809204	-2.8428723849463\\
0.49393744468689	-2.9814260746075\\
0.594617795944214	-3.10094986814795\\
0.694251489639282	-3.19130370252958\\
0.79453010559082	-3.23710807430894\\
0.894382190704346	-3.2519097099098\\
0.993970346450806	-3.25283140572992\\
1.09413213729858	-3.24235849108436\\
1.19380946159363	-3.21572574051447\\
1.29390091896057	-3.17281129615367\\
1.39381856918335	-3.11726968897256\\
1.49456901550293	-3.0502463385169\\
1.59414882659912	-2.97057103486395\\
1.69416828155518	-2.87846173263233\\
1.79447693824768	-2.78163569805156\\
1.89404244422913	-2.67571042076497\\
1.99382014274597	-2.56135676353415\\
2.09440393447876	-2.43524438587292\\
2.19404430389404	-2.30173529285184\\
2.29452223777771	-2.16134436514176\\
2.39429755210876	-2.02049130535852\\
2.49435157775879	-1.87752317990271\\
2.59410924911499	-1.73574285339328\\
2.69441957473755	-1.59295359317275\\
2.79399509429932	-1.45356383141598\\
2.89404768943787	-1.31708353425529\\
2.99460310935974	-1.18391288689963\\
3.09801001548767	-1.05186051789724\\
3.19434351921082	-0.923657936666586\\
3.29437227249146	-0.799072167644681\\
3.39399857521057	-0.682256055030848\\
3.49391670227051	-0.572199113926672\\
3.59435768127441	-0.470115858651297\\
3.69422359466553	-0.373777164555577\\
3.79430932998657	-0.285739393969834\\
3.89424223899841	-0.205775932539837\\
3.9946937084198	-0.13526227172224\\
4.09427065849304	-0.0733155089515094\\
4.19411582946777	-0.0196004697454555\\
4.29436988830566	0.0263234219964943\\
4.3939537525177	0.0652524241929768\\
4.49408383369446	0.0960563135513439\\
4.59400362968445	0.120623128319039\\
4.69436283111572	0.138378892837864\\
4.79416580200195	0.151314137358895\\
4.89397211074829	0.158438543679949\\
4.99416513442993	0.161889757694553\\
5.09401054382324	0.160551470591372\\
5.19386262893677	0.156463040853851\\
5.29427857398987	0.148753265889354\\
5.39438290596008	0.139489186123683\\
5.49426670074463	0.127790584889226\\
5.59454793930054	0.11496954376355\\
5.69384164810181	0.100010004081469\\
5.79383273124695	0.0851167045848342\\
5.89450521469116	0.0691119126896638\\
5.99575634002686	0.0534607208082889\\
6.09436769485474	0.0368203798938112\\
6.19436354637146	0.0213326688092366\\
6.29518957138062	0.00501661413727561\\
6.39615125656128	-0.0096990567670332\\
6.49447984695435	-0.025579357284812\\
6.59447736740112	-0.0393811928832974\\
6.69415373802185	-0.0538898672264168\\
6.79385895729065	-0.0666643320769822\\
6.89376873970032	-0.0790155932409675\\
6.99494786262512	-0.090132505758163\\
7.0942928314209	-0.0995852254214356\\
7.19487829208374	-0.108322620377578\\
7.29400796890259	-0.115883794958961\\
7.39391846656799	-0.12298487561651\\
7.4946560382843	-0.12914307659014\\
7.5947274684906	-0.135023317665286\\
7.69392747879028	-0.140188916451223\\
7.79379243850708	-0.145146450296579\\
7.89449687004089	-0.149529258612176\\
7.99416227340698	-0.153745209989779\\
8.09386534690857	-0.157343986781569\\
8.19469327926636	-0.160556080124991\\
8.29435892105102	-0.163352775775081\\
8.39448161125183	-0.165841918654763\\
8.49439902305603	-0.168562000492528\\
8.59412689208984	-0.170918423435012\\
8.69434328079224	-0.173782115429304\\
8.79458684921265	-0.176475493100355\\
8.89440197944641	-0.179313141530088\\
8.99430246353149	-0.18169197010269\\
9.09426374435425	-0.18322295879517\\
9.19427580833435	-0.184375929827183\\
9.29399795532226	-0.185377727001452\\
9.39400095939636	-0.186061028050659\\
9.49448127746582	-0.186732484895373\\
9.59459300041199	-0.187142913143987\\
9.69432420730591	-0.18758988405645\\
9.79443378448486	-0.187747464733576\\
9.89469833374023	-0.188169106059604\\
9.99415941238403	-0.188250590158447\\
10.0939476013184	-0.188704230393057\\
10.1943997859955	-0.188775663980692\\
10.2945494174957	-0.189471040530719\\
10.396132183075	-0.189930775294869\\
10.4946388721466	-0.191200706565382\\
10.5941822052002	-0.192115945787293\\
10.6943089485168	-0.19391646606536\\
10.7942625999451	-0.1951312245628\\
10.8941337585449	-0.197190245481025\\
10.9948989868164	-0.198584184177321\\
11.0947746753693	-0.200964445182649\\
11.1950318336487	-0.20261599855246\\
11.2942704677582	-0.205235986125558\\
11.3940650939941	-0.207204175821374\\
11.4944948673248	-0.210050063792821\\
11.5940901756287	-0.212225178422784\\
11.6940326213837	-0.215350471744713\\
11.7943291187286	-0.218112607289427\\
11.8944835186005	-0.221720241897529\\
11.9943615913391	-0.224659413548522\\
12.0940055370331	-0.228492854597789\\
12.1948394298553	-0.231681497606047\\
12.2944278240204	-0.235684579958622\\
12.3944353580475	-0.239009267902112\\
12.4942023277283	-0.243442235370168\\
12.5944509029388	-0.246940938669326\\
12.6944336414337	-0.251426667589698\\
12.7945048332214	-0.254904404438193\\
12.8942729949951	-0.259494091550884\\
12.9943153381348	-0.262929618324822\\
13.0942151069641	-0.267552621536908\\
13.1943997859955	-0.27097416675813\\
13.2945842266083	-0.275519977556105\\
13.3941556930542	-0.278800814629292\\
13.4959408760071	-0.283400627004831\\
13.5952534198761	-0.2867298369118\\
13.6944038391113	-0.291458136205119\\
13.7940704345703	-0.295231056349692\\
13.8939878463745	-0.301542554642968\\
13.994055223465	-0.307687846227935\\
14.0940544128418	-0.314254772238272\\
14.1938771724701	-0.320376905179098\\
14.2945451259613	-0.326028415685477\\
14.3944327354431	-0.3313564965293\\
14.4946231365204	-0.336444928127123\\
14.5942589759827	-0.340771867906412\\
14.6945983886719	-0.34434039752515\\
14.7947313308716	-0.346738793627066\\
14.894251537323	-0.349920965542951\\
14.9940084934235	-0.352245724580509\\
15.0941972255707	-0.355675157333982\\
15.1944472312927	-0.35813031354108\\
15.2941483974457	-0.361490534310002\\
15.3943430900574	-0.364247318347939\\
15.4941553592682	-0.368168773028161\\
15.5943800926208	-0.371885534607291\\
15.6945120811462	-0.376107850976718\\
15.7940501689911	-0.380028427964863\\
15.8942977905273	-0.383688293792073\\
15.9943353652954	-0.387096503777808\\
16.0946883678436	-0.3905009529193\\
16.1947559833527	-0.393676940350375\\
16.2944976806641	-0.396940230047441\\
16.393835735321	-0.400101507366578\\
16.4941844463348	-0.403320179982298\\
16.5940634727478	-0.406389374823956\\
16.6945108890533	-0.409356132749313\\
16.7945806503296	-0.412140276532696\\
16.8942512989044	-0.41491929452398\\
16.9942724227905	-0.417550025092236\\
17.0940279483795	-0.420268424850491\\
17.1941775798798	-0.422856239109596\\
17.2939688682556	-0.425352949976695\\
17.3942429542541	-0.427752492750557\\
17.4944741249084	-0.429998329696559\\
17.594543170929	-0.432119243522436\\
17.6945220947266	-0.434072063619212\\
17.7943229198456	-0.435870315131844\\
17.8947572231293	-0.437565021320793\\
17.993910741806	-0.439096042893599\\
18.0943593502045	-0.440745946883634\\
18.1944233894348	-0.442284453802696\\
18.2944530963898	-0.44402496872382\\
18.394628238678	-0.445684942063497\\
18.4941641807556	-0.447388930212519\\
18.5942115306854	-0.449019784007962\\
18.6940993785858	-0.450593564936188\\
18.7940663814545	-0.452105175869445\\
18.894183588028	-0.453559433756269\\
18.9942590713501	-0.454958290219292\\
19.0946690559387	-0.456306618732945\\
19.1968643188477	-0.457606586515062\\
19.29445977211	-0.458858148496248\\
19.3940042972565	-0.460055110150019\\
19.4939903736114	-0.461217447090092\\
19.5939936161041	-0.462341297949589\\
19.6944453239441	-0.463436786897589\\
19.7941777229309	-0.464497671775508\\
19.8944935321808	-0.465526689829201\\
19.9946469783783	-0.466517096806296\\
20.0940992355347	-0.467470907624239\\
20.1944553375244	-0.468401897026379\\
20.2938937664032	-0.469324601102059\\
20.3942555904388	-0.470221163339126\\
20.4941558361053	-0.47108397513043\\
20.5939862251282	-0.471923285826556\\
20.6948296546936	-0.472741709036629\\
20.7943338871002	-0.473540396603468\\
20.8942765712738	-0.474318411025362\\
20.9943639755249	-0.475077463915607\\
21.0939878940582	-0.4758109871788\\
21.1938986301422	-0.476521700867309\\
21.2944957733154	-0.477225521102568\\
21.3939191818237	-0.477912291007438\\
21.4943480014801	-0.478587465170225\\
21.5946044445038	-0.479249365191244\\
21.6944186210632	-0.479899114792879\\
21.7945346355438	-0.480539832078286\\
21.8940104961395	-0.481167396611305\\
21.9942833900452	-0.481781902581858\\
22.0937165737152	-0.48237838296307\\
22.1942743778229	-0.482963312151384\\
22.2950155258179	-0.483528524291877\\
22.3950175762177	-0.484083352761544\\
22.4941741943359	-0.484629039660021\\
22.5940436840057	-0.485164438056756\\
22.6943318367004	-0.485688839054873\\
22.7942213535309	-0.486211872341507\\
22.8943209171295	-0.486729124199748\\
22.9939038276672	-0.48723155670767\\
};
\addlegendentry{$\theta_{y_{p}}$}

\addplot [color=mycolor5, line width=1.4pt]
  table[row sep=crcr]{%
-3.30612330436707	0\\
-3.20646886825562	0\\
-3.10659484863281	0\\
-3.00668745040894	0\\
-2.9067156791687	0\\
-2.80673222541809	0\\
-2.70655851364136	0\\
-2.60655527114868	0\\
-2.50651769638062	0\\
-2.4063539981842	0\\
-2.30651407241821	0\\
-2.20650224685669	0\\
-2.10655479431152	0\\
-2.0065110206604	0\\
-1.90649728775024	0\\
-1.80657606124878	0\\
-1.70637445449829	0\\
-1.60666470527649	0\\
-1.506618309021	0\\
-1.40652709007263	0\\
-1.30670337677002	0\\
-1.20660214424133	0\\
-1.10664420127869	0\\
-1.00659852027893	0\\
-0.906519937515259	0\\
-0.806342649459839	0\\
-0.706490087509155	0\\
-0.6064706325531	0\\
-0.50660662651062	0\\
-0.4065833568573	0\\
-0.306743431091308	0\\
-0.206654357910156	0\\
-0.106565284729004	0\\
-0.00667457580566388	0\\
0.09395809173584	3.46900406965293\\
0.193818521499634	4.87751157109278\\
0.293800067901612	5.64463574510739\\
0.394488763809204	6.17750483979262\\
0.49393744468689	6.59835888616442\\
0.594617795944214	6.93303768242367\\
0.694251489639282	7.21532782087024\\
0.79453010559082	7.46145673390288\\
0.894382190704346	7.67723753891369\\
0.993970346450806	7.87909226793181\\
1.09413213729858	8.0802127135637\\
1.19380946159363	8.27997563449912\\
1.29390091896057	8.47477890731989\\
1.39381856918335	8.66696712157636\\
1.49456901550293	8.85873032232575\\
1.59414882659912	9.04913647905278\\
1.69416828155518	9.23493327927054\\
1.79447693824768	9.41501172530707\\
1.89404244422913	9.59382335676764\\
1.99382014274597	9.76835938567956\\
2.09440393447876	9.9425711292688\\
2.19404430389404	10.1131205136235\\
2.29452223777771	10.2818350143598\\
2.39429755210876	10.4433083450895\\
2.49435157775879	10.6010277291211\\
2.59410924911499	10.7518557869039\\
2.69441957473755	10.8983386753066\\
2.79399509429932	11.0365463299408\\
2.89404768943787	11.1679404985725\\
2.99460310935974	11.2909331722267\\
3.09801001548767	11.4078015716868\\
3.19434351921082	11.5165429456233\\
3.29437227249146	11.6184923112119\\
3.39399857521057	11.7116597232789\\
3.49391670227051	11.7970436328874\\
3.59435768127441	11.8744164497443\\
3.69422359466553	11.9446338812963\\
3.79430932998657	12.0075176456673\\
3.89424223899841	12.0632066404683\\
3.9946937084198	12.1119121280519\\
4.09427065849304	12.1536929894173\\
4.19411582946777	12.1897382731238\\
4.29436988830566	12.2192416164903\\
4.3939537525177	12.2441273490276\\
4.49408383369446	12.2632530298722\\
4.59400362968445	12.2786863425808\\
4.69436283111572	12.2893352461861\\
4.79416580200195	12.2973096577662\\
4.89397211074829	12.3011858601762\\
4.99416513442993	12.3032704444936\\
5.09401054382324	12.3021080830666\\
5.19386262893677	12.2998355594015\\
5.29427857398987	12.2949125303957\\
5.39438290596008	12.2894397342657\\
5.49426670074463	12.2820526815549\\
5.59454793930054	12.2745691812189\\
5.69384164810181	12.2655573657471\\
5.79383273124695	12.2570955666831\\
5.89450521469116	12.2474690217386\\
5.99575634002686	12.2385192557117\\
6.09436769485474	12.2285295664695\\
6.19436354637146	12.2195543595714\\
6.29518957138062	12.2098112390386\\
6.39615125656128	12.2013348632665\\
6.49447984695435	12.1918762464298\\
6.59447736740112	12.1837174624631\\
6.69415373802185	12.1747700200254\\
6.79385895729065	12.166956501741\\
6.89376873970032	12.1588893119534\\
6.99494786262512	12.1519073883209\\
7.0942928314209	12.145401349975\\
7.19487829208374	12.1397220817579\\
7.29400796890259	12.1340227886191\\
7.39391846656799	12.1290844988491\\
7.4946560382843	12.1240700347446\\
7.5947274684906	12.1197638921084\\
7.69392747879028	12.1153237134164\\
7.79379243850708	12.1115887909867\\
7.89449687004089	12.1076599206298\\
7.99416227340698	12.1042297577704\\
8.09386534690857	12.0998165179206\\
8.19469327926636	12.0953898159325\\
8.29435892105102	12.090829898184\\
8.39448161125183	12.0864982959419\\
8.49439902305603	12.0827667764996\\
8.59412689208984	12.0791580292631\\
8.69434328079224	12.0754880860277\\
8.79458684921265	12.0718375654641\\
8.89440197944641	12.0682804835677\\
8.99430246353149	12.0647797646102\\
9.09426374435425	12.0616901363726\\
9.19427580833435	12.0586656167649\\
9.29399795532226	12.0557954412378\\
9.39400095939636	12.0530420687333\\
9.49448127746582	12.050441955296\\
9.59459300041199	12.0479917460303\\
9.69432420730591	12.0455583352156\\
9.79443378448486	12.0432693849599\\
9.89469833374023	12.0408997522431\\
9.99415941238403	12.0387298086795\\
10.0939476013184	12.0364562040647\\
10.1943997859955	12.0343822286238\\
10.2945494174957	12.0321063375369\\
10.396132183075	12.0301487150438\\
10.4946388721466	12.0280718016799\\
10.5941822052002	12.0262994445611\\
10.6943089485168	12.0241228774361\\
10.7942625999451	12.0222534532193\\
10.8941337585449	12.0200268828372\\
10.9948989868164	12.0181046965449\\
11.0947746753693	12.0158505734462\\
11.1950318336487	12.0139279976399\\
11.2942704677582	12.0117088445007\\
11.3940650939941	12.0097992606533\\
11.4944948673248	12.0076136891218\\
11.5940901756287	12.0057704544594\\
11.6940326213837	12.0037679504539\\
11.7943291187286	12.0022159987123\\
11.8944835186005	12.0004823759468\\
11.9943615913391	11.9991491444205\\
12.0940055370331	11.99756456157\\
12.1948394298553	11.9964687040413\\
12.2944278240204	11.9952231170107\\
12.3944353580475	11.9943741844497\\
12.4942023277283	11.9931866893985\\
12.5944509029388	11.9924594529597\\
12.6944336414337	11.9914657660275\\
12.7945048332214	11.9908695918182\\
12.8942729949951	11.9899196199136\\
12.9943153381348	11.9893559937998\\
13.0942151069641	11.9884594309279\\
13.1943997859955	11.9879331704408\\
13.2945842266083	11.9871110603165\\
13.3941556930542	11.9866280482349\\
13.4959408760071	11.985856023994\\
13.5952534198761	11.985391888669\\
13.6944038391113	11.9846561902205\\
13.7940704345703	11.9841297961334\\
13.8939878463745	11.9831872928969\\
13.994055223465	11.9823523020659\\
14.0940544128418	11.9815035061481\\
14.1938771724701	11.9807594916628\\
14.2945451259613	11.980186423465\\
14.3944327354431	11.9797089447013\\
14.4946231365204	11.9793463833589\\
14.5942589759827	11.9791393958346\\
14.6945983886719	11.9791536374024\\
14.7947313308716	11.9793588028664\\
14.894251537323	11.979531466424\\
14.9940084934235	11.9798187440649\\
15.0941972255707	11.9800277794389\\
15.1944472312927	11.9803501856335\\
15.2941483974457	11.9806172442362\\
15.3943430900574	11.9809467632518\\
15.4941553592682	11.9811868711504\\
15.5943800926208	11.9814463976953\\
15.6945120811462	11.9816532969561\\
15.7940501689911	11.9818898399175\\
15.8942977905273	11.9821332463126\\
15.9943353652954	11.9824120231158\\
16.0946883678436	11.9826894635727\\
16.1947559833527	11.9830167704629\\
16.2944976806641	11.9833308863475\\
16.393835735321	11.9836946010687\\
16.4941844463348	11.9840222305963\\
16.5940634727478	11.9844002824364\\
16.6945108890533	11.984740879974\\
16.7945806503296	11.9851327565173\\
16.8942512989044	11.9854817625314\\
16.9942724227905	11.9858585784222\\
17.0940279483795	11.9861583763603\\
17.1941775798798	11.9864634762778\\
17.2939688682556	11.9867523543567\\
17.3942429542541	11.9870505147221\\
17.4944741249084	11.9873511193347\\
17.594543170929	11.9876576608506\\
17.6945220947266	11.9879683730905\\
17.7943229198456	11.9883007913788\\
17.8947572231293	11.9886526818021\\
17.993910741806	11.9890273791054\\
18.0943593502045	11.9893802301676\\
18.1944233894348	11.9897436680334\\
18.2944530963898	11.9900682779754\\
18.394628238678	11.990396080495\\
18.4941641807556	11.9907077637005\\
18.5942115306854	11.9910213181898\\
18.6940993785858	11.9913343979306\\
18.7940663814545	11.9916487577368\\
18.894183588028	11.991964213018\\
18.9942590713501	11.992280338738\\
19.0946690559387	11.9925966732547\\
19.1968643188477	11.9929135118649\\
19.29445977211	11.9932293404827\\
19.3940042972565	11.9935414574133\\
19.4939903736114	11.9938545204588\\
19.5939936161041	11.994167355438\\
19.6944453239441	11.9944836578957\\
19.7941777229309	11.9947995700868\\
19.8944935321808	11.9951148570411\\
19.9946469783783	11.9954274276308\\
20.0940992355347	11.9957372279529\\
20.1944553375244	11.9960470472637\\
20.2938937664032	11.9963574507647\\
20.3942555904388	11.9966672310362\\
20.4941558361053	11.9969777205283\\
20.5939862251282	11.9972874484903\\
20.6948296546936	11.9975958034346\\
20.7943338871002	11.997903242757\\
20.8942765712738	11.9982096993394\\
20.9943639755249	11.9985146221972\\
21.0939878940582	11.9988159337352\\
21.1938986301422	11.9991143097399\\
21.2944957733154	11.9994141604409\\
21.3939191818237	11.9997123355035\\
21.4943480014801	12.0000105971404\\
21.5946044445038	12.0003074204106\\
21.6944186210632	12.0006024628195\\
21.7945346355438	12.000894659066\\
21.8940104961395	12.0011853182045\\
21.9942833900452	12.0014747233517\\
22.0937165737152	12.0017635178342\\
22.1942743778229	12.0020508584357\\
22.2950155258179	12.0023387351036\\
22.3950175762177	12.0026249868874\\
22.4941741943359	12.0029100173117\\
22.5940436840057	12.0031937509676\\
22.6943318367004	12.0034758619986\\
22.7942213535309	12.0037563955506\\
22.8943209171295	12.0040335757281\\
22.9939038276672	12.0043064910791\\
};
\addlegendentry{$\theta_{1}$}

\end{axis}
\end{tikzpicture}%
\vspace{-0.7em}
\caption{\rahel{Coefficient estimation over time applying Algorithm~\ref{alg:onelayerlearningalg} for an initially unknown $\phi_{2}$.}}
\label{fig:onelayerlearningcoef}
\vspace{-0.7em}
\end{figure}
\noindent For the first half of the considered time-span, the variance dominates the target cost, and the setpoints are placed close to the locations of maximal variance within the partitions. The influence of the variance diminishes with its decreasing maximal value, and the goal continuously shifts to the coverage problem. It is important to note that the timescale of the graphics highly differs from the two-layers learning approach presented in Section~\ref{subsec:experimenttwolayerlearning}: for the conducted experiments, the one-layer learning approach takes \rahel{approximately half} of the time to steer the agents in a nearly optimal configuration. However, by the time the experiment was interrupted, all mean estimates showed a maximal error of less than \rahel{0.093} concerning the true coefficient value, and the maximal variance reads as approximately \rahel{0.02}. Both are slightly higher than in the two-layers learning-based approach, and the obtained estimates of the parameter vector $\theta$ are slightly less certain and precise due to a reduction in exploration movements. Both values would decrease if a large variance scaling factor $S$ was used.
\vspace{-0.7em}


\section{Discussion and conclusions}
\label{sec:conclusions}
This work presents a framework to optimally cover a predefined convex area using a non-homogeneous fleet of agents, whose movements are determined by their known nonlinear dynamics. Specifically, two methods based on tracking MPC without terminal constraints and accounting for collision avoidance constraints are introduced. While the first method relies on a two-layered structure and passes an online calculated reference to the individual MPC of each agent, the second method overcomes the hierarchical structure and directly integrates the calculation of the succeeding optimal configuration into the cost function of each agent's MPC. The developed methods are proposed for the case with a known environment, represented by a known density function $\phi$, and are individually extended to the scenario in which the density function needs to be actively learned by the agents. 
Recursive feasibility and convergence to an optimal configuration are formally proven for both methods in each scenario. Furthermore, hardware results are obtained for all the proposed methods using a miniature racing car platform. With the performance of the two methodologies being rather comparable in a known environment, one of the benefits of the one-layer approach is its simplicity. Furthermore, the one-layer learning approach allows for a more efficient operation in the considered experiments, since it does \textit{not} require an explicit exploration vs.~exploitation decision, but automatically adjusts it based on the remaining uncertainty. However, given the integration in its MPC cost function, the one-layer approach is computationally more expensive than its two-layers opponent. An interesting challenge for further research is a decentralized implementation for the learning and the Voronoi partition calculation~\cite{Bullo2012}, e.g., relying only on gossiping between the agents. 
\vspace{-0.7em}


\bibliographystyle{docstyle/IEEEtran} % official IEEE transaction referencing style
\bibliography{bibliography}


\appendix
\section{Proofs}
\label{sec:proofs}
In this section, the proofs of Theorems~\ref{theorem:twolayers}-\ref{theorem:onelayerlearning} are given. In all theorems, 
recursive feasibility is ensured by only updating the partitions if the condition presented in~\eqref{eq:feasibilitycond} is fulfilled. Furthermore, recall that $w$ indicates the timestep of the most recent partition update. 

\subsection{Proof of Theorem~\ref{theorem:twolayers}}
\label{subsec:twolayersproof}The proof of Theorem~\ref{theorem:twolayers} is divided into two steps. First, convergence of each agent to its reference $r$ is shown for a fixed Voronoi partition. This is followed by the proof that the fulfillment of the partition update requirement described in equations~\eqref{eq:updatereq1},~\eqref{eq:updatereq2} and~\eqref{eq:feasibilitycond} is satisfied after a finite time. Subsequently, convergence to a centroidal Voronoi configuration follows by  Proposition~\ref{proposition:cortesconvergence}.

\subsubsection{Convergence to Reference} 
The result will be proven by first introducing two intermediate results (Prop.~\ref{proposition:notminsteadynotcosetwolayers} and~\ref{proposition:upperlowerboudJtilde}). Prop.~\ref{proposition:notminsteadynotcosetwolayers} follows \cite[Proposition 1]{Soloperto2021} and shows that the optimal setpoint cannot be arbitrarily close to the current state-input pair, unless the latter belongs to the set $\rahel{\mathbb{T}_{\bar{p}_w^*,r,i}}$ defined in~\eqref{eq:Tid2layer}; Prop.~\ref{proposition:upperlowerboudJtilde} later provides a lower and upper bound for $\tilde{J}_{i}^{*}(x_{k}, \mathbb{W}_{\rahel{\bar{p}_w^*},i}) = J_{i}^{*}(x_{k}, \mathbb{W}_{\rahel{\bar{p}_w^*},i}) - l_{\mathbb{W}_{\rahel{\bar{p}_w^*}},r,i,\min}$.
\begin{proposition} Let Assumptions~\ref{assumption:dynamics},~\ref{assumption:expocostcontrollability},~\ref{assumption:boundedbyd} and~\ref{assumption:steadystatedecreasetwolayers} hold. Then there exist a constant $\bar{a}_{i} > 0$ such that, for any \rahel{$\bar{p}_{w,i}^*$} and corresponding reference $r$, and for any feasible state $x_{k} \in \mathbb{X}^{\mathrm{int}}: Cx_k \in \bar{\mathbb{W}}_{\rahel{\bar{p}_w^*},i}^{\mathrm{int}}$, the optimal one-step stage cost of problem~\eqref{eq:nonlineartrackingmpcwithcolavoidance} satisfies,
\begin{equation}
\begin{split}
    \ell_{i}^*(x_{k}, s_{k}^{*}) \geq \bar{a}_{i}d_{i}(s_{k}^{*})_{\rahel{\mathbb{T}_{\bar{p}_w^*,r,i}}}^{2}.
\end{split}
\label{eq:prop12layer}
\end{equation}
\label{proposition:notminsteadynotcosetwolayers}
\end{proposition}
\vspace{-1.9em}
\noindent \emph{Proof:} For time step $k\in \mathbb{N}$, assume for contradiction that $\ell_{i}^*(x_{k}, s_{k}^{*}) < \bar{a}_{i}d_{i}(s_{k}^{*})_{\rahel{\mathbb{T}_{\bar{p}_w^*,r,i}}}^{2}$. Following~\cite[Proposition 1]{Soloperto2021}, using compact constraints and $\bar{a}_{i}$ small enough, we can define a setpoint $s'_{k} \in \mathbb{S}_{\mathbb{W}_{\rahel{\bar{p}_w^*},i}}$ according to Assumption~\ref{assumption:steadystatedecreasetwolayers}, such that
\begin{align*}
    &V_{N,i}^{*}(x_{k},s'_{k},\mathbb{W}_{\rahel{\bar{p}_w^*},i}) - V_{N,i}^{*}(x_{k},s_{k}^{*},\mathbb{W}_{\rahel{\bar{p}_w^*},i}) + \ell_{T,i}(\bar{p}'_{k} - r_{i,k})\\
    & -\ell_{T,i}(\bar{p}^*_{k} - r_{i,k}) \overset{\eqref{eq:assumption2gammavmax},~\eqref{eq:assumption32twolayer}}{\leq} \gamma_{i} \ell_{i}^{*}(x_{k},s'_{k}) - \beta_{2,i}\epsilon^{\prime} d_{i}( s_{k}^{*} )_{\rahel{\mathbb{T}_{\bar{p}_w^*,r,i}}}^{2} \\
    &\overset{\eqref{eq:boundtwosteadystatestwolayer}}{\leq} \! \gamma_{i} \xi_{1,i} \ell_{i}^*(x_{k},s_{k}^{*}) \! +  \! \gamma_{i}\xi_{2,i} d_{i}(s'_{k} \! - \! s_{k}^{*})^{2} \! - \! \beta_{2,i}\epsilon^{\prime} d_{i}( s_{k}^{*} )_{\rahel{\mathbb{T}_{\bar{p}_w^*,r,i}}}^{2} \\ 
    &\overset{\eqref{eq:assumption31twolayer}}{<} (\gamma_{i} \xi_{1,i}\bar{a}_i + \gamma_{i} \xi_{2,i}\beta_{1,i}^{2}\epsilon^{\prime,2}- \beta_{2,i}\epsilon^{\prime}) d_{i}( s_{k}^{*} )_{\rahel{\mathbb{T}_{\bar{p}_w^*,r,i}}}^{2} \\ &:= - \bar{c}_id_{i}( s_{k}^{*} )_{\rahel{\mathbb{T}_{\bar{p}_w^*,r,i}}}^{2}. 
\label{eq:prop1}
\end{align*}
Given $\bar{a}_i$ small enough, an $\epsilon^{\prime} > 0$ can be found such that $\bar{c}_i > 0$. This implies that the new setpoint $s^{\prime}_k$ yields a smaller cost, which contradicts the optimality of $s_k^*$.\hfill{$\blacksquare$}

\begin{proposition}
Let Assumptions~\ref{assumption:dynamics},~\ref{assumption:expocostcontrollability},~\ref{assumption:boundedbyd},and~\ref{assumption:steadystatedecreasetwolayers} hold. Define  $a_{i} = \frac{1}{2} \min{(\bar{a}_{i},\alpha_{1,i})}$ and recall that $\bar{u}$ describes the minimizer of $\min_{u \in \mathbb{U}_{i}} \ell_{i}(x,u,s)$. Then, for any $\rahel{\bar{p}_w^*} \in \mathbb{A}^M$, the corresponding reference $r$ and setpoint $s_k^* \in \mathbb{S}_{\mathbb{W}_{\rahel{\bar{p}_w^*},i}}$, and for any feasible state $x_{k} \in \mathbb{X}^{\mathrm{int}}: Cx_k \in \bar{\mathbb{W}}_{\rahel{\bar{p}_w^*},i}^{\mathrm{int}}$, it holds
\begin{equation}
    \tilde{J}_{i}^{*}(x_{k}, \mathbb{W}_{\rahel{\bar{p}_w^*},i}) \! \geq \! a_{i}(d_{i}(s_{k}^{*})_{\rahel{\mathbb{T}_{\bar{p}_w^*,r,i}}}^{2} \! + \! d_{i}((x_{k},u_{x_k,s_k^*}^{*}) \! - \! s_{k}^{*})^2).
\label{eq:lowerbound2layer}
\end{equation}
Further, defining $\gamma_{i,V_{\max,i}} = \max{(\gamma_{i},\frac{V_{\max,i}}{\chi_{i}},2)}$ and denoting by $b_{i} \! = \! \max{(\gamma_{i,V_{\max,i}}\alpha_{2,i}, \beta_{T,i})}$, we have 
\begin{equation}
     \tilde{J}_{i}^{*}(x_{k}, \mathbb{W}_{\rahel{\bar{p}_w^*},i}) \! \leq \! b_{i} (d_{i}(s_{k}^{*})_{\rahel{\mathbb{T}_{\bar{p}_w^*,r,i}}}^{2} \! + \! d_{i}((x_{k},u_{x_k,s_k^*}^{*}) \! - \! s_{k}^{*})^2).
\label{eq:upperbound2layer}
\end{equation}
\label{proposition:upperlowerboudJtilde}
\end{proposition}
\vspace{-1.9em}
\noindent \emph{Proof:} As regards the lower bound, we have 
\begin{align}
     &\tilde{J}_{i}^{*}(x_{k}, \mathbb{W}_{\rahel{\bar{p}_w^*},i}) \geq  V_{N,i}^{*}(x_{k},s_{k}^{*},\mathbb{W}_{\rahel{\bar{p}_w^*},i}) \geq \ell_{i}(x_{k},u_{x_k,s_k^*}^{*},s_{k}^{*}) \notag \\ &\overset{\eqref{eq:boundstagecosttwolayer},\eqref{eq:prop12layer}}{\geq} \frac{\bar{a}_{i}}{2}d_{i}(s_{k}^{*})_{\rahel{\mathbb{T}_{\bar{p}_w^*,r,i}}}^{2} + \frac{\alpha_{1,i}}{2}d_{i}((x_{k},u_{x_k,s_k^*}^{*}) - s_{k}^{*})^2 \label{eq:lowerbound2layertmp} 
\end{align}
To show the existence of an upper bound, we leverage Assumptions~\ref{assumption:expocostcontrollability} and~\ref{assumption:steadystatedecreasetwolayers}. Therefore, we combine~\cite{Boccia2014} with the constraints $V_{N,i} \leq V_{\max,i}$,  to obtain $V_{N,i}^{*}(x,s) \leq \gamma_{i,V_{\max,i}} \ell_{i}^{*}(x,s)$, which yields: 
\begin{align}
     &\tilde{J}_{i}^{*}(x_{k}, \mathbb{W}_{\rahel{\bar{p}_w^*},i}) \notag \\ & = V_{N,i}^{*}(x_{k},s_{k}^{*},\mathbb{W}_{\rahel{\bar{p}_w^*},i}) + \ell_{T,i}(s_{k}^{*}-r) -   l_{\mathbb{W}_{\rahel{\bar{p}_w^*},i},r,\min} \label{eq:upperbound2layer1} \\ & \leq
     \gamma_{i,V_{\max,i}}\alpha_{2,i} d_{i}((x_{k},u_{x_k,s_k^*}^{*}) - s_{k}^{*}))^2 + \beta_{T,i}d_{i}(s_{k}^{*})_{\rahel{\mathbb{T}_{\bar{p}_w^*,r,i}}}^{2} \notag 
     \hspace{0.1em}\hfill\ensuremath{\blacksquare}
\end{align}
\vspace{-0.1em}

At this point, considering a fixed partition $\mathbb{W}_{\rahel{\bar{p}_w^*},i}$ and using the candidate $s_{k+1}=s_{k}^{*}$, combining Theorem~\ref{theorem:theo4boccia} with Proposition~\ref{proposition:notminsteadynotcosetwolayers} and Assumption~\ref{assumption:boundedbyd} yields \begin{align*}
    &\tilde{J}_{i}^{*}(x_{k+1},\mathbb{W}_{\rahel{\bar{p}_w^*},i}) - \tilde{J}_{i}^{*}(x_{k}, \mathbb{W}_{\rahel{\bar{p}_w^*},i})\\ 
    &\leq V^{*}_{N,i}(x_{k+1},s_{k}^{*},\mathbb{W}_{\rahel{\bar{p}_w^*},i}) + \ell_{T,i}(s_{k}^{*} - r_{i}) - l_{\mathbb{W}_{\rahel{\bar{p}_w^*}},r,i,\min} \\ 
    &- V^{*}_{N,i}(x_{k},s_{k}^{*},\mathbb{W}_{\rahel{\bar{p}_w^*},i}) - \ell_{T,i}(s_{k}^{*} - r_{i}) + l_{\mathbb{W}_{\rahel{\bar{p}_w^*}},r,i,\min} \\
    &\overset{\eqref{eq:theorem5twolayer}}{\leq} - \bar{\alpha}_{N,i}\ell_{i}^{*}(x_{k},s_{k}^{*})\\ &\overset{\eqref{eq:prop12layer}}{\leq} -\frac{\bar{\alpha}_{N,i}}{2}(\ell_{i}^{*}(x_{k},s_{k}^{*}) + \bar{a}_{i}d_{i}(s_{k}^{*})_{\rahel{\mathbb{T}_{\bar{p}_w^*,r,i}}}^{2}) \\ 
    &\overset{\eqref{eq:boundstagecosttwolayer}}{\leq} -\frac{\bar{\alpha}_{N,i}}{2}(\alpha_{1,i}d_{i}((x_{k},u_{x_{k},s_{k}^{*}}^{*})-s_{k}^{*})^{2} + \bar{a}_{i}d_{i}(s_{k}^{*})_{\rahel{\mathbb{T}_{\bar{p}_w^*,r,i}}}^{2})\\  & \leq - \tilde{\alpha}_{N,i}(d_{i}((x_{k},u_{x_{k},s_{k}^{*}}^{*})-s_{k}^{*})^{2} + d_{i}(s_{k}^{*})_{\rahel{\mathbb{T}_{\bar{p}_w^*,r,i}}}^{2}),
    \label{eq:2theorem2twolayer}
\end{align*}
where we defined $\tilde{\alpha}_{N,i}=\frac{\bar{\alpha}_{N,i}}{2}\text{min}\{\alpha_{1,i},\bar{\alpha}_i\}$. Using Proposition~\ref{proposition:upperlowerboudJtilde}, we have 
\begin{equation*}
    \begin{aligned}
    &\tilde{J}_{i}^{*}(x_{k+1},\mathbb{W}_{\rahel{\bar{p}_w^*},i}) - \tilde{J}_{i}^{*}(x_{k}, \mathbb{W}_{\rahel{\bar{p}_w^*},i}) \\
    &\leq - \tilde{\alpha}_{N,i}(d_{i}(s_{k}^{*})_{\rahel{\mathbb{T}_{\bar{p}_w^*,r,i}}}^{2} + d_{i}((x_{k},u_{x_{k},s_{k}^{*}}^{*})-s_{k}^{*})^{2})\\ 
    & \leq  -\frac{\tilde{\alpha}_{N,i}}{b_{i}}\tilde{J}_{i}^{*}(x_{k}, \mathbb{W}_{\rahel{\bar{p}_w^*},i})
    \end{aligned}
\end{equation*}

\noindent and obtain the exponential convergence relation for $\tilde{J}_{i}^{*}(x_{k+\tilde{k}},\mathbb{W}_{\rahel{\bar{p}_w^*},i})$
\begin{equation}
\begin{split}
    \tilde{J}_{i}^{*}(x_{k+\tilde{k}},\mathbb{W}_{\rahel{\bar{p}_w^*},i})
    \leq \Big(1 - \frac{\tilde{\alpha}_{N,i}}{b_{i}} \Big)^{\tilde{k}} \tilde{J}_{i}^{*}(x_{k}, \mathbb{W}_{\rahel{\bar{p}_w^*},i}),
\end{split}
\label{eq:4fintitetimecovergencetwolayer}
\end{equation}
which is guaranteed to converge, because $b_{i} >\tilde{\alpha}_{N,i}>0$.

\subsubsection{Finite Time Update Condition Fulfillment, Part 1}\label{subsection:finitetimeconditiontwolayersp1}

Given \eqref{eq:4fintitetimecovergencetwolayer}, and the fact that $\tilde{J}_{i}^{*}(x_{k}, \mathbb{W}_{\rahel{\bar{p}_w^*},i})$ admits a uniform bound due to compact constraints and continuity of the cost, it follows that,
for any $\bar{\epsilon} > 0$, there exists a finite time step $\tilde{k}'>0$ such that $\tilde{J}_{i}^{*}(x_{k+\tilde{k}'},\mathbb{W}_{\rahel{\bar{p}_w^*},i}) < \bar{\epsilon}$. 
Hence, by choosing $\bar{\epsilon} > 0$ small enough and considering the lower bound~\eqref{eq:lowerbound2layer}, the state at time step $\tilde{k}'$ is arbitrary close to the set $\rahel{\mathbb{T}_{\bar{p}_w^*,r,i}}$, and thus the update conditions~\eqref{eq:updatereq1},~\eqref{eq:updatereq2} hold (assuming we are not already in a centroidal Voronoi configuration).

\subsubsection{Finite Time Update Condition Fulfillment, Part 2}\label{subsection:finitetimeconditiontwolayersp2}We now need to ensure that the remaining Voronoi partition update condition~\eqref{eq:feasibilitycond} is fulfilled in finite time with the candidate state and input sequences proposed in~\eqref{eq:lemma1proposal}.  
Specifically, we need to find an upper bound for 
\begin{align}
    & V_{N,i}(x_{k+\tilde{k}'}, \hat{u}, s_{k}^{*}, \mathbb{W}_{\rahel{\bar{p}_w^*}}) = \sum_{l=0}^{N-2}\ell_{i}(x_{l\vert k+\tilde{k}'}^{*}, u_{l \vert k+\tilde{k}'}^{*},s_{k}^{*}) + \notag \\ &  \ell_{i}(x_{N-1\vert k+\tilde{k}'}^{*},\bar{u}_{k+\tilde{k}'}^{*},s_{k}^{*}) + \notag \\ & \ell_{i}(f(x_{N-1\vert k+\tilde{k}'}^{*},\bar{u}_{k+\tilde{k}'}^{*}),\bar{u}_{k+\tilde{k}'}^{*},s_{k}^{*}), 
    \label{eq:trackingcostcandidate}
\end{align}
and such a bound is required to be decreasing as $\tilde{k}'$ increases, along the lines of~\eqref{eq:4fintitetimecovergencetwolayer}. To this aim, we will make use of $\bar{\epsilon}$ as introduced above that can be seen as a decreasing function of $\tilde{k}'$.
The first term of~\eqref{eq:trackingcostcandidate} can be upper bounded by~\eqref{eq:4fintitetimecovergencetwolayer} as 
\begin{align}
    &\epsilon \geq V_{N,i}^{*}(x_{k+\tilde{k}'},s_{k}^{*},\mathbb{W}_{\rahel{\bar{p}_w^*}})  \geq \sum_{l=0}^{N-2}\ell_{i}(x_{l\vert k+\tilde{k}'}^{*}, u_{l \vert k+\tilde{k}'}^{*},s_{k}^{*}). \notag %\notag \\ & \geq \sum_{l=0}^{N-2}\alpha_{1}d(({x}^{*}_{l \vert k+\tilde{k}'},u_{x_{l},s_{k}^{*}}^{*}) - s_{k}^{*}).
\label{eq:7finitetimeconvergence}
\end{align}
The upper bound for the last two terms is given by the following proposition.
\begin{proposition} Let Assumption~\ref{assumption:dynamics} and~\ref{assumption:boundedbyd} hold. 
Then, for any $x \in \mathbb{R}^{n_{i}}$ and any $s = (\bar{x},\bar{u}) \in \mathbb{S}_i$, it holds that
\begin{equation}
    \ell_{i}(x,\bar{u},s) + \ell_{i}(f(x,\bar{u}),\bar{u},s)\leq \gamma_{i}' \cdot \ell_{i}^{*}(x,s),
\label{eq:assumption22}
\end{equation}
where $\gamma_{i}'=\frac{2\alpha_{2,i}}{\alpha_{1,i}}\max_j\{1+\mathcal{L}_{i}^{\eta_j}\}$.
\label{proposition:upperboundsuccedingstagecost}
\end{proposition}

\noindent \emph{Proof:} 
In accordance to equation~\eqref{eq:boundstagecosttwolayer}, it holds that
\begin{align}
    &\ell_{i}(x,\bar{u},s) + \ell_{i}(f_{i}(x,\bar{u}),\bar{u},s)\leq \label{eq:fulfillassumption22}\\ 
    &2\alpha_{2,i} {\sum_{j=1}^{n_i}  \Big(\vert x_{j}-\bar{x}_j \vert^{\eta_j} + \vert f_{i,j}(x,\bar{u})-\bar{x}_j \vert^{\eta_j}\Big)}.\notag 
\end{align}
Because $\bar{x}$ is a steady state, we have that $f_{i,j}(x,\bar{u}) - \bar{x}_j = f_{i,j}(x,\bar{u}) - f_{i,j}(\bar{x},\bar{u})$. Thus, by Lipschitz continuity and reusing \eqref{eq:boundstagecosttwolayer}, we have that~\eqref{eq:fulfillassumption22} is upper bounded by 
\begin{align*}
    \hspace{3.3em}&2\alpha_{2,i}\sum_{j=1}^{n_i} (1 + \mathcal{L}_{i}^{\eta_j})\vert x_{j}-\bar{x}_j \vert^{\eta_j} \leq \gamma_i'\ell_i^*(x,s).
    \hspace{3.3em}\hfill\ensuremath{\blacksquare}
\end{align*}
\vspace{-1.0em}

\noindent Finally, noting also that
\begin{equation}
{\displaystyle \bar{\epsilon} \geq \ell_{i}(x_{N-1 \vert k+\tilde{k}'}^{*}, u_{N-1 \vert k+\tilde{k}'}^{*}, s_{k}^{*}) \geq \ell^{*}_{i}(x_{N-1 \vert k+\tilde{k}'}^{*},s_{k}^{*})},\notag 
\label{eq:8finitetimeconvergence}
\end{equation}
we obtain the desired bound
\begin{align}
     & V_{N,i}(x_{k+\tilde{k}'}, \hat{u}, s_{k}^{*}, \mathbb{W}_{\rahel{\bar{p}_w^*}}) \qquad \text{(see~\eqref{eq:trackingcostcandidate})} \notag \\
     &  \leq \sum_{l=0}^{N-2}\ell_{i}(x_{l\vert k+\tilde{k}'}^{*}, u_{l \vert k+\tilde{k}'}^{*},s_{k}^{*}) + \gamma_{i}' \ell^{*}_{i}(x_{N-1\vert k+\tilde{k}'}^{*},s_{k}^{*}) \notag \\ &
     \leq (1+ \gamma_{i}')\bar{\epsilon}.
\label{eq:9finitetimeconvergence}
\end{align}
Since it holds for any of the addenda, which in turn are lower-bounded by a function describing the distance between state and setpoint, they will become arbitrarily close. \rahel{In combination with Lemma~\ref{lemma:inclusioninnewinterior}, the update condition in~\eqref{eq:feasibilitycond} will then hold for a finite~$\tilde{k}'$.}

\subsection{Proof of Theorem~\ref{theorem:twolayerslearning}}
\label{subsec:twolayerslearningproof}
The result of Theorem~\ref{theorem:twolayerslearning} builds upon~\cite[Proposition 1]{Todescato2017}. It ensures that the learning-based solution converges to \rrahel{a centroidal} Voronoi partition if the estimate $\hat{\phi}_t$ converges in probability to the true density $\phi$. Then, convergence to a centroidal Voronoi configuration follows by Theorem~\ref{theorem:twolayers}. 

\begin{proposition}~\cite[Proposition 1]{Todescato2017} Assume $\hat{\phi}_{t}(p) \xrightarrow{\mathcal{P}} \phi(p)$, where $\mathcal{P}$ denotes convergence in probability in the space of continuous functions. Moreover, assume it is possible to start the Lloyd algorithm from the configuration reached by Algorithm~\ref{alg:twolayermpcalglearningimp} at time $\bar{t}$. Then the centroids $c^{L}_{\bar{t}}$ obtained from the Lloyd algorithm and the estimated ones $\hat{c}_{\bar{t}}$ coincide. Furthermore, for any $0<\bar{\delta} < 1$, $\varepsilon > 0$ an integer $N$ can be picked, such that there exists a $\bar{t}$ sufficiently large such that
\begin{equation}
     \mathcal{P}[ \Vert \hat{c}_{\bar{t} + t} - c^{L}_{\bar{t} + t} \Vert < \varepsilon ] > 1 - \bar{\delta}, \ \ t = 0, \hdots, N,
\label{eq:learningprop1}
\end{equation}
where $\mathcal{P}$ is associated to the Gaussian distribution of the estimated coefficients.
\label{proposition:convergenceofcentroidstwolayerslearning}
\end{proposition}
\noindent By construction, the locations of the agents never coincide, so it is possible to start Lloyd's algorithm from an arbitrary instant in time. Then, it remains to show that $\hat{\phi}_{t}(p) \xrightarrow{\mathcal{P}} \phi(p)$ holds by applying Algorithm~\ref{alg:twolayermpcalglearningimp}.

\begin{lemma}
\rahel{Let Assumptions~\ref{assumption:dynamics},~\ref{assumption:expocostcontrollability},~\ref{assumption:boundedbyd},~\ref{assumption:bayesianconvergence},and~\ref{assumption:steadystatedecreasetwolayers} hold.}
Consider a horizon length $N \geq N^{*}$, constants $\rho>0, r_{i,\max} >0$, and $\varepsilon>0$ sufficiently small. Then, for any $p \in \mathbb{A}$, $\hat{\phi}_t(p) \xrightarrow{\mathcal{P}} \phi(p)$.
\label{lemma:learning}
\end{lemma}
\noindent \emph{Proof:} We prove this by showing that, given any partition configuration $\mathbb{W}_p$, the maximum variance value inside the $i-$th partition, $\text{Var}_{i,\max,t} = \max_{p \in \mathbb{W}_{p,i}} \text{Var}_t(p)$, converges to zero for each $i$. This implies that $\text{Var}_t(p)$ at any point $p$ inside the partition
converges to zero with increasing $t$: this is equivalent to proving $\mathscr{L}^2$-convergence of $\phi_t(p)$, from which convergence in probability follows.\\
For a time instant $t$, define $p^{\text{maxvar}}_{t,i} = \arg\max_{p \in \mathbb{W}_{p,i}} \text{Var}_t(p)$, and consider the position of the $i-$th agent, $p_{t,i}$, such that $\|p^{\text{maxvar}}_{t,i} - p_{t,i}\| \leq (\rho + r_{\max} + \varepsilon)$. Next, we show $\text{Var}(p_{t,i}) \geq \tilde{c}\text{Var}(p^{\text{maxvar}}_{t,i})$ for some $\tilde{c}>0$. To this aim, we have
\begin{align*}
    \text{Var}(p^{\text{maxvar}}_{t,i}) &\leq 2\Phi(p_{t,i})\Sigma_{t}\Phi(p_{t,i})^{\top}  \\
    + &2(\Phi(p_{t,i}) - \Phi(p^{\text{maxvar}}_{t,i}))\Sigma_t(\Phi(p_{t,i}) - \Phi(p^{\text{maxvar}}_{t,i}))^{\top}\\
    &\leq 2 \text{Var}(p_{t,i}) + 2\mathcal{L}_{\Phi}^2(\rho + r_{\max} + \varepsilon)^2\|\Sigma_t\|,
\end{align*}
with some Lipschitz constant $\mathcal{L}_\Phi>0$ from Assumption~\ref{assumption:bayesianconvergence}. Moreover, by the linear independence of the features in $\Phi$ stated in the same Assumption, we have $\text{Var}(p^{\text{maxvar}}_{t,i}) \geq c_{\Phi}\|\Sigma_t\|$. Applying this inequality yields
\begin{equation*}
   \text{Var}(p_{t,i}) \geq \underbrace{(1 - 2\mathcal{L}_{\Phi}^2(\rho+r_{\max} + \varepsilon)^2/c_{\Phi})/2}_{=:\tilde{c}} \text{Var}(p^{\text{maxvar}}_{t,i}), 
\end{equation*}
with $\tilde{c}>0$ for $\gamma>0, r_{i,\max} >0, \varepsilon>0$ sufficiently small.\\
The proof is concluded by showing that $\text{Var}(p_{t,i})$ converges to zero. Assume for contradiction that \mbox{$\lim\sup_{t \rightarrow +\infty} \Phi(p_{t,i})\Sigma_t\Phi(p_{t,i})^{\top} \! = \! c^{\prime} > 0$.} We can isolate a subsequence $\{t_{\tau}\}_{\tau}$ such that $\lim_{\tau \rightarrow +\infty} \Phi(p_{t_{\tau},i})\Sigma_{t_{\tau}}\Phi(p_{t_{\tau},i})^{\top} \! = \! c^{\prime}$. However, from recursive application of equation~\eqref{eq:bayessigmaupdate} we have that \mbox{$\lim_{\tau\rightarrow\infty}\Sigma_{t_\tau}^{-1} \! \succeq \! \sum_{\tau=1}^{\infty}\Phi(p_{t_{\tau},i})^{\top}\Phi(p_{t_{\tau},i})$}. 
This implies that $\Phi(p_{t_\tau,i})\Sigma_{t_\tau}\Phi(p_{t_\tau,i})^\top$ tends to 0, contradicting the starting hypothesis. 
\hfill$\blacksquare$

\subsection{Proof of Theorem~\ref{theorem:onelayer}}
\label{subsubsec:onelayerknownenvironmenttheory}
We first present two preliminary results that will be needed in the main proof. In particular, Proposition~\ref{proposition:notminsteadynotcoseonelayer} generalizes Proposition~\ref{proposition:notminsteadynotcosetwolayers} to non-convex target costs, and Proposition~\ref{proposition:voronoiupdatecostdecrease} ensures that an update of the partitions in accordance to the current steady state does not increase the cost~\eqref{eq:cortescost}.

\begin{proposition} Let Assumptions~\ref{assumption:dynamics},~\ref{assumption:expocostcontrollability},~\ref{assumption:boundedbyd},~\ref{assumption:ballwithminconvex} and~\ref{assumption:steadystatedecreaseonelayer} hold. Then there exists a constant $\bar{a}_{i} > 0$ such that, for any $\bar{p}_{w,i}^* \in \mathbb{S}_{\mathbb{A},i}^{\mathrm{p}}$ any feasible state $x_{k} \in \mathbb{X}^{\mathrm{int}}:Cx_k \in \bar{\mathbb{W}}_{\bar{p}_{w}^*,i}^{\mathrm{int}}$, the optimal solution of the MPC problem~\eqref{eq:fullnonlineartrackingmpconelayer} satisfies
\begin{equation}
    \ell_{i}^*(x_{k}, s_{k}^{*}) \geq \bar{a}_{i}\kappa(\Vert \bar{p}_{k}^{*} \Vert)_{\rahel{\mathbb{T}_{\bar{p}_{k}^{*},\bar{p}_w^*,i}}}.
\label{eq:prop1}
\end{equation}
\label{proposition:notminsteadynotcoseonelayer}
\end{proposition}
\vspace{-1.2em}
\noindent \emph{Proof:} Follows the same steps as the proof of Proposition 2, with Assumption~\ref{assumption:steadystatedecreasetwolayers} replaced by Assumptions~\ref{assumption:ballwithminconvex} and~\ref{assumption:steadystatedecreaseonelayer}.\hfill{$\blacksquare$} 

\begin{proposition} 
For any $\bar{p},\bar{p}_k\in\mathbb{A}^M$, it holds
\begin{equation}
    H(\bar{p}_{k},\mathbb{W}_{\bar{p}_{k}}) \leq  H(\bar{p}_{k},\mathbb{W}_{\bar{p}}).
\label{eq:condassumption6}
\end{equation}
\label{proposition:voronoiupdatecostdecrease}
\end{proposition}
\vspace{-1.5em}
\noindent \emph{Proof:} 
Given setpoint positions $\bar{p}$, each partition $\mathbb{W}_{\bar{p},i}$ admits itself a partition $\{\mathbb{W}_{\bar{p} \rightarrow \bar{p}_{k},i \rightarrow j}\}_{j=1}^M$, explicitly depending on $\bar{p}_{k}$, where each $\mathbb{W}_{\bar{p} \rightarrow \bar{p}_{k},i \rightarrow j}$ describes the set of points which are currently assigned to the steady-state of agent $i$, but would be assigned to the steady-state of agent $j$ in case of a Voronoi partition update in accordance to $\bar{p}_{k}$. 
Therefore, with $g$ non-decreasing, it holds:
\begin{align*}
    &H(\bar{p}_{k},\mathbb{W}_{\bar{p}})  = \sum_{i=1}^{M}\int_{\mathbb{W}_{\bar{p},i}} g(\Vert q - \bar{p}_{i} \Vert)\phi(q)dq \\ 
    &= \sum_{i=1}^{M}\sum_{j=1}^{M}\int_{\mathbb{W}_{\bar{p}\rightarrow\bar{p}_{k},i\rightarrow j}} g(\Vert q - \bar{p}_{i} \Vert)\phi(q)dq \\ 
    &\overset{\eqref{eq:voronoidef}}{\geq} \sum_{i=1}^{M}\sum_{j=1}^{M}\int_{\mathbb{W}_{\bar{p}\rightarrow\bar{p}_{k},i\rightarrow j}} g(\Vert q - \bar{p}_{j} \Vert)\phi(q)dq \\
    &= \sum_{i=1}^{M}\int_{\mathbb{W}_{\bar{p}_{k},i}} g(\Vert q - \bar{p}_{i} \Vert)\phi(q)dq = H(\bar{p}_{k},\mathbb{W}_{\bar{p}_{k}}).
    \label{eq:assumption6}
    \hspace{3.0em}\hfill\ensuremath{\blacksquare}
\end{align*}

\noindent
The remainder of the proof amounts to showing the following facts: 1) convergence to a fixed setpoint, taking into account a possible update of the Voronoi tessellation; 2) convergence to a centroidal Voronoi partition; and 3) finite-time partition update. The following results thereby rely on a Lyapunov like decrease of the storage function for the MPC in~\eqref{eq:fullnonlineartrackingmpconelayer}.\\
The storage function we are considering is the summation of the optimal MPC cost over all agents, that is
\begin{equation}
\begin{aligned}
&J^{*}(x_{k}, \mathbb{W}_{\rahel{\bar{p}^{*}_{w}}}) = V_{N}^{*}(x_{k}, s_{k}^{*},\mathbb{W}_{\bar{p}^{*}_{w}}) + \lambda  H(\bar{p}_{k},\mathbb{W}_{\bar{p}^{*}_{w}}). 
\end{aligned}
\label{eq:lyapunovonelayer}
\end{equation}
When the subscript $i$ is dropped, then we refer to quantities of the overall system. The storage function~\eqref{eq:lyapunovonelayer} has an upper and lower bound. The first exists thanks to the compactness of $\mathbb{A},\mathbb{X},\mathbb{U}$ and the continuity of the cost. Given that the locational optimization cost is always greater or equal to zero, the latter follows by
\begin{align}
     &J^{*}(x_{k},\mathbb{W}_{\rahel{\bar{p}_w^*}}) \geq V_{N}^{*}(x_{k},s_{k}^{*},\mathbb{W}_{\rahel{\bar{p}_w^*}}) \geq \ell^{*}(x_{k},s_{k}^{*}) \notag \\ &\overset{\eqref{eq:boundstagecosttwolayer}}{\geq} \alpha_{1}d((x_{k},u_{x_{k},s_{k}^{*}}^{*})-s_{k}^{*})^{2} \geq 0.
\label{eq:lowerbound}
\end{align}

\subsubsection{Convergence to Setpoint} We now prove that the optimal cost at time $k+1$ on a candidate update partition $\mathbb{W}_{\bar{p}_{k}^{*}}$ decreases with respect to its value corresponding to the Voronoi partition induced by the setpoint positions $\bar{p}$ at time $w$. In other words, we show that $J^{*}(x_{k+1},\mathbb{W}_{\bar{p}_{k}^{*}}) < J^{*}(x_{k},\mathbb{W}_{\rahel{\bar{p}_w^*}})$. By definition of the involved storage function, this in turn implies that $x_k$ converges to the current steady-state $s_k^*$.\\
Recalling the definitions of $\gamma_{i}'$ and $\gamma_{i,V_{\max,i}}$ given in Proposition~\ref{proposition:upperboundsuccedingstagecost} and the proof of Proposition~\ref{proposition:upperlowerboudJtilde}, respectively, let $\gamma_{\max} := \max_{i=1,...,M}\max{\{\gamma_{i}',\gamma_{i,V_{\max,i}}\}}$ and $\gamma' = \max_{i=1,...,M}{\{\gamma_{i}'\}}$. Moreover, we will make use of the following result taken from the proof of Theorem~\ref{theorem:theo4boccia}:
\begin{equation}
\begin{split}
    &\left( \frac{\gamma_{i,V_{\max,i}}}{\gamma_{i,V_{\max,i}}-1} \right)^{N-1} \ell_{i}^{*}(\hat{x}_{N-1\vert k},s_{k}^*) \\ & \leq V_{N,i}(x_{k},s_k^*,\mathbb{W}_{\rahel{\bar{p}_w^*},i}) \leq \gamma_{i,V_{\max,i}} \ell^{*}_{i}(x_{k},s_k^*).
\end{split}
\label{eq:prooftheom4}
\end{equation}

Next, we get the following storage function decrease: 
\begin{align}
    &J^{*}(x_{k+1},\mathbb{W}'_{\bar{p}_{k}^{*}}) \leq J^{*}(x_{k+1}, s^{*}_{k},\mathbb{W}_{\bar{p}_{k}^{*}}) \label{eq:intermediateJ1} \\ & = 
    V_{N}^{*}(x_{k+1},s_{k}^{*},\mathbb{W}_{\bar{p}_{k}^{*}}) + \lambda H(\bar{p}_{k}^{*},\mathbb{W}'_{\bar{p}_{k}^{*}}) %- \ell_{T,min} 
    \notag \\ &\overset{\eqref{eq:lemma1proposal},\eqref{eq:feasibilitycond}}{\leq} V_{N}(\hat{x}_{\cdot \vert k}, \hat{u}_{\cdot \vert k},s_{k}^{*}) + \lambda H(\bar{p}_{k}^{*},\mathbb{W}_{\bar{p}_{k}^{*}})  %-\ell_{T,min} 
    \notag \\ &\overset{\eqref{eq:condassumption6},\eqref{eq:feasibilitycond}}{\leq} V_{N}(\hat{x}_{\cdot \vert k}, \hat{u}_{\cdot \vert k},s_{k}^{*}) + \lambda H(\bar{p}_{k}^{*},\mathbb{W}_{\rahel{\bar{p}_w^*}}) %- \ell_{T,min} 
    \notag \\
    &\overset{\eqref{eq:lemma1proposal}}{=} J^{*}(x_{k},\mathbb{W}_{\rahel{\bar{p}_w^*}}) - \ell(x_{k},u_{0\vert k}^{*},s_{k}^{*}) - \ell(x_{N-1\vert k}^{*},u_{N-1\vert k}^{*},s_{k}^{*}) \notag \\ & + \ell(\hat{x}_{N-1 \vert k},\bar{u}^{*}_{k},s_{k}^{*}) + \ell(f(\hat{x}_{N-1 \vert k},\bar{u}^{*}_{k}),\bar{u}^{*}_{k},s_{k}^{*}) \notag \\
    & \leq J^{*}(x_{k},\mathbb{W}_{\rahel{\bar{p}_w^*}}) - \ell^{*}(x_{k},s_{k}^{*}) - \ell^{*}(x_{N-1\vert k}^{*},s_{k}^{*}) \notag \\ & + \ell(\hat{x}_{N-1\vert k},\bar{u}^{*}_{k},s_{k}^{*}) + \ell(f(\hat{x}_{N-1 \vert k},\bar{u}^{*}_{k}),\bar{u}^{*}_{k},s_{k}^{*}) \notag \\
    &\overset{\eqref{eq:assumption22}}{\leq} J^{*}(x_{k},\mathbb{W}_{\rahel{\bar{p}_w^*}}) - \ell^{*}(x_{k},s_{k}^{*}) + (\gamma' -1)\ell^{*}(x_{N-1\vert k}^{*},s_{k}^{*}) \notag \\
    &\overset{\eqref{eq:prooftheom4}}{\leq} J^{*}(x_{k},\mathbb{W}_{\rahel{\bar{p}_w^*}}) - \ell^{*}(x_{k},s_{k}^{*})  \notag \\ & +(\gamma_{\max} - 1)\left( \frac{\gamma_{\max}-1}{\gamma_{\max}} \right)^{N-1} \gamma_{\max} \ell^{*}(x_{k},s_{k}^{*}). \notag
\end{align}
Thus, we obtain 
\begin{equation}
 J^{*}(x_{k+1},\mathbb{W}'_{\bar{p}_{k}^{*}}) \leq J^{*}(x_{k},\mathbb{W}_{\rahel{\bar{p}_w^*}}) - \bar{\alpha}_{N}\ell^{*}(x_{k},s_{k}^{*}),
 \label{eq:1theorem2}
\end{equation}
where $\bar{\alpha}_{N} = 1-\left( \dfrac{(\gamma_{\max}-1)}{\gamma_{\max}} \right)^{N} \gamma_{\max}^{2} > 0$ by choosing $N>N^{*}$ large enough. Thus, inequality~\eqref{eq:1theorem2}, in combination with lower and upper bounds on $J^{*}$ and Assumption~\ref{assumption:boundedbyd}, ensures that $\lim_{k\rightarrow \infty} \|  x_{k} - s_{k}^{*} \|  = 0$.

\subsubsection{Convergence to Centroidal Voronoi partition} Combining Proposition~\ref{proposition:notminsteadynotcoseonelayer} with the result seen in~\eqref{eq:1theorem2}, it follows that $J^{*}$ decreases as long as $x_{k} \neq s_{k}^{*}$ or $\bar{p}_{i,k}^{*} \notin \rahel{\mathbb{T}_{\bar{p}_{i,k}^{*},\bar{p}_w^*,i}}$ for all $i \in \{1,\hdots,M \}$. In fact,
\begin{align}
    &J^{*}(x_{k+1}, \mathbb{W}_{\rahel{\bar{p}_w^*}}) - J^{*}(x_{k}, \mathbb{W}_{\rahel{\bar{p}_w^*}})
    \leq - \bar{\alpha}_{N}\ell^{*}(x_{k},s_{k}^{*}) \notag \\
    &\leq -\frac{\bar{\alpha}_{N}}{2}(\ell^{*}(x_{k},s_{k}^{*}) + \bar{a}\kappa(\Vert \bar{p}_{k}^{*} \Vert)_{\rahel{\mathbb{T}_{\bar{p}_{k}^{*},\bar{p}_w^*}}}^{2}),
\label{eq:2theorem2}
\end{align}
with $\bar{a}\kappa(\Vert \bar{p}_{k}^{*} \Vert)_{\rahel{\mathbb{T}_{\bar{p}_{k}^{*},\bar{p}_w^*}}}^{2} = \max_{i} \bar{a}_i \sum_{i=1}^{M}\kappa(\Vert \bar{p}_{i,k}^{*} \Vert)_{\rahel{\mathbb{T}_{\bar{p}_{k}^{*},\bar{p}_w^*,i}}}^{2}$.
Accordingly, it can be concluded that $\lim_{k \to \infty}p_{i,k} \in \rahel{\mathbb{T}_{\bar{p}_{k}^{*},\bar{p}_w^*,i}}$. 

\subsubsection{Finite Time Partition Update} Finally, it has to be shown that equation~\eqref{eq:feasibilitycond} is fulfilled in finite time and the Voronoi partition can be updated. For this purpose, we leverage the result in~\eqref{eq:1theorem2} with respect to a fixed set of partitions $\mathbb{W}_{\rahel{\bar{p}_w^*}}$ and candidate $s_{k+1} = s_k^*$, and combine them with Assumption~\ref{assumption:expocostcontrollability}. Then, we obtain 
\begin{equation}
\begin{split}
     V_{N}^{*}(x_{k+k^{\prime}},s_{k}^{*},\mathbb{W}_{\rahel{\bar{p}_w^*}}) 
     \leq \Big(1 - \frac{\bar{\alpha}_{N}}{\gamma_{\max}} \Big)^{k^{\prime}} V_{N}^{*}(x_{k},s_{k}^{*},\mathbb{W}_{\rahel{\bar{p}_w^*}}),
\end{split}
\label{eq:4fintitetimecovergence}
\end{equation}
that converges because by definition $\gamma_{\max}> \bar{\alpha}_{N} > 0$. At this point, the same reasoning carried out in Appendix~\ref{subsection:finitetimeconditiontwolayersp2} can be used to show that condition~\eqref{eq:feasibilitycond} is met in finite time.

\noindent In conclusion, the algorithm does converge to a configuration in which each agent's position defines a local minimum of $H(p, \mathbb{W})$ and the Voronoi partition does not update anymore: that is, a centroidal Voronoi configuration is obtained. 

\subsection{Proof of Theorem~\ref{theorem:onelayerlearning}}
\label{subsec:onelayerunknownenvironmenttheory}
For this proof we first show convergence to a steady state describing a local minimum of the locational optimization function considering the converged estimate. In a second step, an upper bound for the remaining variance at each reachable position is defined.
\subsubsection{Convergence} Similar to the proof of Theorem~\ref{theorem:onelayer}, we define the storage function as
\begin{align}
&J^{*}(x_{k}, \mathbb{W}_{\bar{p}^{*}_{w}},\hat{\phi}_{k}) \notag \\ & =  V_{N}^{*}(x_{k}, s_{k}^{*},\mathbb{W}_{\bar{p}^{*}_{w}}) + \lambda (H(\bar{p}_{k}^{*},\mathbb{W}_{\bar{p}^{*}_{w}}, \hat{\phi}_{k}) - S\text{Var}_{k}(\bar{p}_{k}^{*}))\notag \\
& = \sum_{i=1}^{M} \bigg( \sum_{l=0}^{N-1}\ell_{i}(x_{i,l\vert k}^{*}, u_{i,l \vert k}^{*}, s_{i,k}^{*}) \label{eq:lyapunovonelayerlearning} \\  &+ \lambda (H_{i}(\bar{p}_{i,k}^{*},\mathbb{W}_{\bar{p}^{*}_{w},i},\hat{\phi}_{k}) - S\text{Var}_{M,k}(\bar{p}_{i,k}^{*}) \bigg), \notag
\end{align}
Following the reasoning in equations~\eqref{eq:intermediateJ1} and~\eqref{eq:1theorem2} for the adapted storage function, we end up with
\begin{align}
&J^{*}(x_{k+1}, \mathbb{W}'_{\bar{p}^{*}_{k}},\hat{\phi}_{k+1}) \leq J^{*}(x_{k}, \mathbb{W}_{\bar{p}_w^*},\hat{\phi}_{k}) - \underbrace{\bar{\alpha}_{N}\ell^{*}(x_{k},s_{k}^{*})}_{\geq 0} 
\notag \\ &- \lambda \sum_{i=1}^{M} \int_{\mathbb{W}_{\bar{p}_{w}^*},i} g(\Vert q - \bar{p}^{*}_{i,k} \Vert)(\hat{\phi}(q)_{k}-\hat{\phi}(q)_{k+1})dq \notag \\ &- \underbrace{\lambda S (\text{Var}_{M,k}(\bar{p}^{*}_{k}) - \text{Var}_{M,k+1}(\bar{p}^{*}_{k}))}_{\geq 0}. \label{eq:lyth5}
\end{align}
Using the Bernstein-von Mises Theorem~\cite[Chapter 10.2]{vanderVaart1998} we have that $\hat{\phi}_{k}(q_k)$ converges in probability to the Gaussian distribution with mean equal to the maximum likelihood estimate, and with variance equal to $1/k$ times the asymptotic variance of the maximum likelihood estimate (i.e., the inverse of the Fisher information matrix). Using~\eqref{eq:bayesupdate}, this convergence implies that $(\hat{\theta}_{k}-\hat{\theta}_{k+1}) \xrightarrow{\mathcal{P}} 0$. Thus, the storage function decrease in~\eqref{eq:lyth5} implies $\lim_{k\rightarrow\infty}\|x_k-s_k^*\|=0$ in probability. Applying Proposition~\ref{proposition:notminsteadynotcoseonelayer} ensures convergence of the steady state to a local minimum of the target cost with respect to the current density~$\phi_k$, similarly to Equation~\eqref{eq:2theorem2}.

\subsubsection{Remaining Variance} 
Given the converged state, the following proposition provides a bound on the variance for all feasible position $p\in\mathbb{F}:=\lim_{k\rightarrow\infty}\mathbb{F}_k$.  

\begin{proposition} 
Suppose the closed loop converges to a setpoint $\overline{s}_i^*$ with state $\bar{x}_{i}^{*}$ and position $\bar{p}_{i}^{*}$. 
Then, for each agent $i \in \{1,\hdots,M\}$, there exists a uniform constant $\Delta H_i\geq 0$ such that, for any $p\in \mathbb{F}_i$ with corresponding $s_{p}\in\mathbb{S}_{\rahel{\bar{p}_w^*},i}$, the variance is bounded by
\begin{align}
    &\lim_{k \rightarrow \infty}\mathrm{Var}_{k}(p) \leq  \frac{\Delta H_i}{S}. 
\label{eq:upperboundvaronelayerlearningsimplified}
\end{align}
\label{proposition:onelayerlearning}
\end{proposition}
\vspace{-1.5em}
\noindent \emph{Proof:} 
Convergence ensures $\lim_{k\rightarrow\infty} V_N(\overline{x}^*,\overline{s}^*)=0$, $\text{Var}_\infty(\overline{p}^*)=0$ and hence the minimum cost in the MPC problem is given by $\lambda H(\overline{p}^*,\mathbb{W}_{\overline{p}^*,i},\hat{\phi}_\infty)$. 
The cost of any feasible position $p\in\mathbb{F}_i$ with corresponding setpoint $s_p\in\mathbb{S}_i$ is larger than or equal to this minimum, i.e.,
\begin{align}
    &\lambda H(\overline{p}^*,\mathbb{W}_{\overline{p}^*,i},\hat{\phi}_\infty) \label{eq:contraditiononelayerlearning2}\\
    &\leq V_{N}^{*}(\bar{x}_{i}^{*}, s_{p},\mathbb{W}_{\bar{p},i}) +\lambda (H(\overline{p},\mathbb{W}_{\overline{p}^*,i},\hat{\phi}_\infty)-S\text{Var}_\infty({p})).
    \nonumber
\end{align}
Note that $V_N(\overline{x}_i^*,p)\leq V_{\max,i}$; moreover, the definition of the coverage cost $H$ in~\eqref{eq:cortescost}   
with $g$ continuous and $\mathbb{W}_{\overline{p}^*,i}\subseteq\mathbb{A}$ compact ensures that there exists a uniform constant $\Delta H_i\geq 0$ such that
\begin{align*}
    %&S\text{Var}_\infty(p)
    &\dfrac{1}{\lambda}V_{N}^{*}(\bar{x}_{i}^{*}, s_{p},\mathbb{W}_{\bar{p},i})  + H(\overline{p},\mathbb{W}_{\overline{p}^*,i},\hat{\phi}_\infty)-H(\overline{p}^*,\mathbb{W}_{\overline{p}^*,i},\hat{\phi}_\infty)\\
    & \leq \Delta H_i.
    %\label{eq:contraditiononelayerlearning2}
\end{align*}
Inequality~\eqref{eq:upperboundvaronelayerlearningsimplified} follows by applying this bound to inequality~\eqref{eq:contraditiononelayerlearning2}. 
$\blacksquare$\\

\noindent Therefore, for every $\tilde{\epsilon} > 0$, a scaling $S>0$ can be found such that
$\lim_{k\rightarrow\infty}\text{Var}_{M,k}(p) \leq \frac{\Delta H}{S} \leq \tilde{\epsilon}$, $\ \forall p \in \mathbb{F}$, where $\Delta H = \max_{\Delta H_i}\{\Delta H_1,\hdots,\Delta H_M\}$. 
\vspace{-1.7em}
\begin{IEEEbiography}[{\includegraphics[width=1in,height=1.25in,clip,keepaspectratio]{figures/rahel_rickenbach-min.jpg}}]{Rahel Rickenbach} is a PhD candidate at ETH Zürich. She received her bachelors degree in mechanical engineering in 2019 and her masters degree in robotics, systems and control in 2021, both from ETH Zürich. In 2022 she was awarded the SGA-Förderpreis for her master thesis. Her main research interests lie in the areas of multi agent and coverage control, as well as inverse optimal control. 
\end{IEEEbiography}

\begin{IEEEbiography}[{\includegraphics[width=1in,height=1.25in,clip,keepaspectratio]{figures/koehler-min.jpg}}]{Johannes K\"ohler} received his Master degree in Engineering Cybernetics from the University of Stuttgart, Germany, in 2017. 
In 2021, he obtained a Ph.D. in mechanical engineering, also from the University of Stuttgart,
Germany, for which he received the 2021 European Systems \& Control PhD award.
He is currently a postdoctoral researcher at the Institute for Dynamic Systems and Control (IDSC) at ETH Zürich.
His research interests are in the area of model predictive control and control and estimation for nonlinear uncertain systems. 
\end{IEEEbiography}

\begin{IEEEbiography}[{\includegraphics[width=1in,height=1.25in,clip,keepaspectratio]{figures/scampicchio.jpg}}]{Anna Scampicchio} was born in 1993. She received in 2015 the Bachelor degree in Information Engineering and in 2017 the Masters degree in Automation Engineering, both cum laude, from the University of Padova. In 2017 she was awarded with the Roberto Rocca scholarship for her career during the Masters Degree. She held a visiting position at the Department of Applied Mathematics of University of Washington, Seattle, in 2019. In 2021 she received the Ph.D. in Information Engineering from the University of Padova, and now she is a
postdoctoral researcher at the Institute for Dynamic Systems and Control, ETH Z{\"u}rich. Her research interests lie at the interplay among system identification, machine learning and control design.
\end{IEEEbiography}

\begin{IEEEbiography}[{\includegraphics[width=1in,height=1.25in,clip,keepaspectratio]{figures/melanie_zeilinger.jpg}}]{Melanie N. Zeilinger} 
is an Associate Professor at ETH Zurich, Switzerland. She received the Diploma degree in engineering cybernetics from the University of Stuttgart, Germany, in 2006, and the Ph.D. degree with honors in electrical engineering from ETH Zurich, Switzerland, in 2011. From 2011 to 2012 she was a Postdoctoral Fellow with the Ecole Polytechnique Federale de Lausanne (EPFL), Switzerland. She was a Marie Curie Fellow and Postdoctoral Researcher with the Max Planck Institute for Intelligent Systems, Tübingen, Germany until 2015 and with the Department of Electrical Engineering and Computer Sciences at the University of California at Berkeley, CA, USA, from 2012 to 2014. From 2018 to 2019 she was a professor at the University of Freiburg, Germany. Her current research interests include safe learning-based control, as well as distributed control and optimization, with applications to robotics and human-in-the-loop control.
\end{IEEEbiography}
 \vspace{-30em}

\begin{IEEEbiography}[{\includegraphics[width=1in,height=1.25in,clip,keepaspectratio]{figures/andrea_carron-min.jpg}}]{Andrea Carron} is a Senior Lecturer at ETH Zürich. He received the bachelor’s, master’s, and Ph.D. degrees in control engineering from the University of Padova, Padova, Italy. During his master and Ph.D. studies, he spent three stays abroad as a Visiting Researcher: the first at the University of California at Riverside, Riverside, CA, USA, the second at the Max Planck Institute, Tubingen, Germany, and the third at the University of California at Santa Barbara, Santa Barbara, CA, USA. From 2016 to 2019, he was a Post-Doctoral Fellow with the Intelligent Control Systems Group, ETH Zürich, Zürich, Switzerland. His is research interests include safe-learning, learning-based control, multiagent systems, and robotics.
\end{IEEEbiography}.
\newpage


%%%%%%%%%%%%%%%%%%%%%%%%%%%%%%%%%%%%%%%%%%%%%%%%%%%%%%%%%%%%%%%%%%%%%%%%%%%%%%%%%%%%%
%\input{Addendum_ExplanatoryTables}
%\input{TableNotation}
\onecolumn 

\section*{Look-up notation table in alphabetic order}
%The provided tables should simplify an initial read of the paper, making it possible to conveniently look up variables meaning, whenever it reappears after its introduction.

In the following, we present and contextualize all the symbols used throughout the paper. The list is divided in four subsections: general sets, operators and functions; indices; set cardinalities and vector dimensions; and variables and sets. Within each subsection, symbols are ordered in a dictionary-like lexicographic order, where capital letters are presented separately from lower-case ones, and grouped by font type.

\subsection*{General sets, operators and functions}
\begin{flalign*}
\notn{\|\cdot \|}{Euclidean norm}
\notn{\ominus}{Pontryagin set difference}
\notn{f(y)_{\mathbb{D}}}{$\min_{\tilde{{y}} \in \mathbb{D}}f({y} - \tilde{{y}})$, for continuous $f(\cdot)$ and compact $\mathbb{D}$}
\notn{\text{Var}_{t}(p_i)}{variance at arbitrary position $p_i$ at time $t$, \eqref{eq:variancephiblr}}
\notn{\text{Var}_{M,t}(p)}{summed up $\text{Var}_{t}(p_i)$ over all agents}
\notn{\mathcal{B}(\cdot)}{Bernoulli random variable with success probability $\cdot$}
\notn{\mathcal{N}(\mu_0,\Sigma_0)}{Gaussian distribution with mean $\mu_0$ and covariance $\Sigma_0$}
\notn{\mathcal{P}[\cdot]}{Probability operator}
\notn{\mathbb{B}_{o}^{b}}{Ball with radius $o$: $\{x \in \mathbb{R}^b\vert \,\Vert x \Vert\leq o \}$}
\notn{\mathbb{N}}{Set of natural numbers}
\notn{\mathbb{R}_+}{Set of non-negative real numbers}
\notn{\mathbb{V}^{\mathrm{int}},\; \mathbb{V}\subset\mathbb{R}^b}{$\mathbb{V} \ominus \mathbb{B}_{\epsilon}^{b}$ for problem-dependent, fixed $\epsilon >0$ \hspace{14.7em}}
\end{flalign*}

\subsection*{Indices}
\begin{flalign*}
\notn{h}{time index at which a density measurement is collected}
    \notn{i}{agent index}
    \notn{j}{dummy index}
    \notn{k}{time step}
    \notn{l}{prediction step}
    \notn{t}{time index for data collection \eqref{eq:setofmeasurements}}
    \notn{w}{time index for partition update}
    \notn{w_+}{next partition update time index \hspace{19.5em}}
\end{flalign*}

\subsection*{Set cardinalities and vector dimensions}
\begin{flalign*}
\notn{D}{coverage domain dimension (usually $D=2$ or $D=3$)}
\notn{M}{number of agents}
\notn{N}{horizon length of MPC problem}
\notn{N^*}{minimum horizon length as in Theorem \ref{theorem:theo4boccia}}
\notn{}{}
\notn{m_i}{dimension of input vector $u_{i,k}$}
\notn{n_i}{dimension of state vector $x_{i,k}$} 
\notn{}{}
\notn{\upsilon}{number of features describing density function $\phi$ \hspace{12.5em}}
\end{flalign*}

\subsection*{Variables and sets}
\begin{flalign*}
\notn{C_i}{$D\times n_i$ matrix for $i-$th agent state-position map \eqref{eq:nonlineardynamics}}
\notn{F}{arbitrary, strictly monotonically increasing function such that \mbox{$F(\epsilon)=0 \Leftrightarrow \epsilon=0$}}
\notn{H(p,\mathbb{O})}{locational optimization cost in \eqref{eq:cortescost}}
\notn{H(p,\mathbb{O},\hat{\phi})}{locational optimization cost in \eqref{eq:cortescost} with respect to estimated density $\hat{\phi}$}
\notn{I_t}{set of measurements \eqref{eq:setofmeasurements}}
\notn{J_i}{MPC cost for the $i-$th agent \eqref{eq:generalnonlineartrackingmpccost}}
\notn{J^{*}(x_{k}, \mathbb{W}_{\bar{p}_w^*})}{summation of $J_{i}^{*}(x_{k}, \mathbb{W}_{\bar{p}_w^*,i})$ over all agents}
\notn{J^{*}(x_{k}, \mathbb{W}_{\bar{p}_w^*},\hat{\phi_k})}{$J^{*}(x_{k}, \mathbb{W}_{\bar{p}_w^*,i})$ with respect to estimated density at time step $k$}
\notn{J_{i}^{*}(x_{k}, \mathbb{W}_{\bar{p}_w^*,i})}{optimal $J_i$ with respect to initial state $x_{k}$ and partition $\mathbb{W}_{\bar{p}_w^*,i}$}
\notn{\tilde{J}_{i}^{*}(x_{k}, \mathbb{W}_{\bar{p}_w^*,i})}{Two-layers approach Lyapunov function, equal to $J_{i}^{*}(x_{k}, \mathbb{W}_{\bar{p}_w^*,i})- l_{\mathbb{W}_{\bar{p}_w^*},r,i,\min}$}
\notn{\mathcal{L}_i}{Lipshitz constant of the $i-$th agent dynamics}
\notn{\mathcal{L}_{\Phi}}{Lipshitz constant from Assumption \ref{assumption:bayesianconvergence}}
\notn{Q_j}{scaling matrix for state vector in stage cost, Section \ref{subsec:expermpccost}}
\notn{Q_j^{\prime}}{scaling matrix for state vector in target cost, Section \ref{subsec:expermpccost}}
\notn{R_z}{scaling matrix for input vector in stage cost, Section \ref{subsec:expermpccost}}
\notn{R_z^{\prime}}{scaling matrix for input vector in target cost, Section \ref{subsec:expermpccost}}
\notn{S}{scaling factor for variance term in one-layer MPC cost, \eqref{eq:fullnonlineartrackingmpccostonelayerlearning}}
\notn{T}{continuous mapping entering Proposition \ref{proposition:cortesconvergence}}
\notn{T_s}{sampling time}
\notn{\text{Var}_{\max}}{maximal variance value computed over the set $\mathbb{A}$}
\notn{\text{Var}_{i,\max,t}}{maximal variance value computed over the $i$-th partition at time step $t$}
\notn{V_{N,i}}{tracking cost for the $i-$th agent}
\notn{V_{N,i}^{*}(x_{i,k},s_{i,k}, \mathbb{P}_{i})}{optimal $V_{N,i}$ with arbitrary initial condition, setpoint and position constraint}
\notn{V_{N}^{*}(x_{k}, s_{k}^{*},\mathbb{W}_{\bar{p}^{*}_{w}})}{summation of $V_{N,i}^{*}(x_{i,k},s_{i,k}, \mathbb{W}_{\bar{p}^{*}_{w,i}})$ over all agents}
\notn{V_{\max,i}}{user chosen bound on tracking cost \eqref{eq:generalvarbound}}
\notn{}{}
\notn{a_{i}}{positive constant entering \eqref{eq:lowerbound2layer}, Proposition \ref{proposition:upperlowerboudJtilde} and \eqref{eq:prop1}, Proposition \ref{proposition:notminsteadynotcoseonelayer}}
\notn{\bar{a}}{maximal value of $\bar{a}_{i}$ over all agents}
\notn{\bar{a}_{i}}{positive constant in lower bound of optimal stage cost \eqref{eq:prop12layer}, Proposition \ref{proposition:notminsteadynotcosetwolayers}}
\notn{b_{i}}{positive constant entering \eqref{eq:upperbound2layer}, Proposition \ref{proposition:upperlowerboudJtilde}}
\notn{\tilde{c}}{constant in proof of Lemma \ref{lemma:learning}}
\notn{c^{\prime}}{constant in proof of Lemma \ref{lemma:learning}}
\notn{\bar{c}_{i}}{positive constant in proof of Proposition \ref{proposition:notminsteadynotcosetwolayers}}
\notn{c(\mathbb{W}_{p}, \phi)}{centroids of the Voronoi tessellation $\mathbb{W}_{p}$}
\notn{c_i(\mathbb{W}_{p,i}, \phi)}{$i-$th centroid \eqref{eq:voronoicenterdef}}
\notn{\hat{c}_{\bar{t}}}{estimated centroids at time step $\bar{t}$}
\notn{c^{L}_{\bar{t}}}{centroids at time step $\bar{t}$ obtained with Lloyd algorithm}
\notn{c_{\Phi}}{constant entering bound in Assumption \ref{assumption:bayesianconvergence}}
\notn{d(\cdot)}{summed up distance function over all agents}
\notn{d_i(\cdot)}{distance function defining the adopted stage cost, \eqref{eq:distance}}
\notn{e_{w}}{vector of errors $e_{i,w}$}
\notn{e_{i,w}}{$i-$th agent position error $\Vert \bar{p}_{i,w}^* \! - \! c_i(\mathbb{W}_{\bar{p}_{w}^*,i}, \phi) \Vert$}
\notn{e_{v,i,k}}{distance from point of maximum variance, $\Vert v_i(\mathbb{W}_{\bar{p}_w^*,i}, \text{Var}_t) - p_{i,k} \Vert$}
\notn{f_i(\cdot,\cdot)}{state transition map for $i-$th agent \eqref{eq:nonlineardynamics}}
\notn{g(\cdot)}{general sensing capability (Remark \ref{rmk:g})}
\notn{g_{1,i},g_{2,i}}{position and state-dependent dynamics vector of the $i$-th car}
\notn{g_{3,i},g_{4,i}}{Lie brackets of $g_{1,i}$ and $g_{2,i}$}
\notn{k'}{time steps into future in \eqref{eq:4fintitetimecovergence}}
\notn{\tilde{k}'}{Finite time step at which $\tilde{J}_{i}^{*}(x_{k+\tilde{k}'},\mathbb{W}_{\bar{p}_w^*,i}) < \bar{\epsilon}$}
\notn{l_{\mathbb{W}_{\bar{p}},r,i,\min}}{minimal target cost entering \eqref{eq:Tid2layer}, Assumption \ref{assumption:steadystatedecreasetwolayers}}
\notn{l_{\bar{p}',\mathbb{W}_{\bar{p}},i,\min}}{minimal target cost entering \eqref{eq:Tid}, Assumption \ref{assumption:steadystatedecreaseonelayer}}
\notn{\ell_i}{stage cost for the $i-$th agent \eqref{eq:generalnonlineartrackingmpccost}}
\notn{\ell^*_i(x_k,s_k)}{one-step optimal stage cost for $i-$th agent}
\notn{\ell^*(x_k,s_k)}{summed up $\ell^*_i(x_k,s_k)$ over all agents}
\notn{\ell_{T,i}}{target cost for the $i-$th agent \eqref{eq:generalnonlineartrackingmpccost}}
\notn{m_{i,h}}{density measure of $i-th$ agent at time $h$ \eqref{eq:measmod}}
\notn{\bar{m}_{t+1}}{vector of measurements $[m_{1,t+1} \: \cdots \: m_{M,t+1}]^{\top}$}
\notn{p}{Collection of positions $[p_1,\cdots,p_M]^{\top}$}
\notn{p_i}{position of the $i-$th agent}
    \notn{p_{i,k}}{position of the $i-$th agent at time $k$}
    \notn{\bar{p}_i}{position setpoint for the $i-$th agent}
    \notn{\bar{p}'}{position setpoint in Assumptions \ref{assumption:steadystatedecreasetwolayers} and \ref{assumption:steadystatedecreaseonelayer}}
    \notn{\bar{p}_{i,k}}{position setpoint for the $i-$th agent at time $k$}
    \notn{\bar{p}_{i,w}^*}{position setpoint on which the Voronoi partition is built}
    \notn{p^{\text{maxvar}}_{t,i}}{position with maximal variance in the $i$-th partition at time step $t$}
    \notn{r_i}{general external reference for the $i-$th agent}
     \notn{r_{i,k}}{external reference for the $i-$th agent at time $k$}
     \notn{r_{\max}}{maximum of all $r_{i,\max}$}
     \notn{r_{i,\max}}{radius of the ball covering the $i-$th agent, \eqref{eq:colavoidancerequirement}}
    \notn{s_i}{artificial setpoint for $i-$th agent, $s_i = (\bar{x}_{i},\bar{u}_{i})$}
    \notn{s_{i,k}}{artificial setpoint for the $i-$th agent at time $k$}
    \notn{s_{i,k}^*}{optimal artificial setpoint for the $i-$th agent at time $k$}
    \notn{\hat{s}}{candidate setpoint, Assumption \ref{assumption:steadystatedecreasetwolayers}}
    \notn{\bar{t}}{investigated time step in Proposition \ref{proposition:convergenceofcentroidstwolayerslearning}}
    \notn{u_{i,k}}{input of the $i-th$ agent at time instant $k$}
    \notn{u_{i,l\vert k}}{$l-$steps-ahead predicted input of agent $i$ at time $k$}
    \notn{u_{i,\cdot\vert k}}{predicted inputs $x_{i,l\vert k}$ with $l=0,...,N-1$}
    \notn{u_{i,\cdot\vert k}^*}{optimal input sequence for $i-$th agent}
    \notn{\bar{u}_i}{input setpoint for $i-$th agent}
    \notn{u_{d,i}}{steering input of the $i$-th car}
    \notn{u_{r,i}}{input reference of the $i$-th car}
    \notn{u_{v,i}}{velocity input of the $i$-th car}
    \notn{v(\mathbb{W}_{\bar{p}_w^*}, \text{Var})}{vector collecting all $v_i(\mathbb{W}_{\bar{p}_w^*,i}, \text{Var})$}
    \notn{v_i(\mathbb{W}_{\bar{p}_w^*,i}, \text{Var})}{point of maximum variance within the $i-$th partition}
    \notn{x_{i,k}}{state of the $i-th$ agent at time instant $k$}
    \notn{x_{i,l\vert k}}{$l-$steps-ahead predicted state of agent $i$ at time $k$}
    \notn{x_{i,\cdot\vert k}}{predicted states $x_{i,l\vert k}$ with $l=0,...,N-1$}
    \notn{x_{i,\cdot\vert k}^*}{optimal state sequence for $i-$th agent}
    \notn{\bar{x}_i}{state setpoint for $i-$th agent}
    \notn{x_{p,i}}{$x$-coordinate position of the $i$-th car}
    \notn{x_{r,i}}{state reference of the $i$-th car}
    \notn{y_{p,i}}{$y$-coordinate position of the $i$-th car}
    \notn{}{}
    \notn{\mathbb{A}}{convex area in $\mathbb{R}^D$}
    \notn{\mathbb{F}_{k}}{union of $\mathbb{F}_{i,k}$}
    \notn{\mathbb{F}_{i,k}}{set of positions that are feasible w.r.t.~\eqref{eq:fullnonlineartrackingmpconelayer}}
    \notn{\mathbb{O}}{arbitrary partition of $\mathbb{A}$, $\mathbb{O} = \{\mathbb{O}_{1}, \hdots, \mathbb{O}_{M}\}$}
    \notn{\mathbb{O}_i}{polytope associated to the $i-$th agent}
    \notn{\mathbb{P}_i}{general position constraint for the $i-$th agent}
    \notn{\mathbb{S}_{\mathbb{P},i}}{set of artificial setpoints \eqref{eq:steadystatesetwithcollavoidance}}
\notn{\mathbb{S}_{\mathbb{P},i}^{\mathrm{p}}}{set of positions satisfying the artificial setpoints set conditions}
\notn{\mathbb{S}_{\bar{p}^{\prime},\mathbb{P},i}^{\mathrm{p}}}{$\mathbb{B}_{r}^{D}(\bar{p}') \cap \mathbb{S}_{\mathbb{P},i}^{\mathrm{p}}$}
\notn{\mathbb{T}_{\bar{p},r,i}}{union of all $\bar{p} \in \mathbb{S}_{\mathbb{P},i}^{\mathrm{p}}$ resulting in a minimal target cost value, \eqref{eq:Tid2layer}}
\notn{\mathbb{T}_{\bar{p}^{\prime},\bar{p},r,i}}{union of all $\bar{p} \in \mathbb{S}_{\bar{p^{\prime}}\mathbb{P},i}^{\mathrm{p}}$ resulting in a minimal target cost value, \eqref{eq:Tid}}
    \notn{\mathbb{X}_i}{state constraints for $i-$th agent}
    \notn{\mathbb{U}_i}{input constraints for $i-$th agent}
    \notn{\mathbb{W}_{p}}{Voronoi tessellation built according to $p$}
    \notn{\mathbb{W}_{p,i}}{set of the Voronoi tessellation associated to the $i-$th agent \eqref{eq:voronoidef}}
    \notn{\bar{\mathbb{W}}_{\bar{p}_w^*},i}{$\mathbb{W}_{\bar{p}_w^*,i} \ominus \mathbb{B}_{r_{\max}}^D$}
    \notn{\mathbb{W}_{\bar{p} \rightarrow \bar{p}_{k},i \rightarrow j}}{set of points changing partition ($i \rightarrow j$) for a Voronoi update considering $\bar{p}_{k}$}
    \notn{}{}
    \notn{\Delta H}{uniform constant for asymptotic variance bound in Theorem \ref{theorem:onelayerlearning}}
    \notn{\Delta H_i}{individual $\Delta H$ for each agent with $\Delta H = \textstyle{\max_{\Delta H_i}}\{\Delta H_1,\hdots,\Delta H_M\}$}
    \notn{\Phi}{vector collecting density features, Assumption \ref{assumption:bayesianconvergence}}
    \notn{\bar{\Phi}_{t+1}}{matrix of features $[\Phi(p_{1,t+1})^{\top} \: \cdots \: \Phi(p_{M,t+1})^{\top}]^{\top}$}
    \notn{\Sigma}{positive semi-definite matrix, possibly associated to a covariance}
    \notn{}{}
    \notn{\alpha_{1}}{maximal value of $\alpha_{1,i}$ over all agents}
    \notn{\alpha_{1,i},\alpha_{2,i}}{constants entering bounds for stage cost \eqref{eq:boundstagecosttwolayer}, Assumption \ref{assumption:boundedbyd}}
    \notn{\bar{\alpha}_{N}}{maximal value of $\bar{\alpha}_{N,i}$ over all agents}
    \notn{\bar{\alpha}_{N,i}}{constant entering bound \eqref{eq:theorem5twolayer}, Theorem \ref{theorem:theo4boccia}}
    \notn{\tilde{\alpha}_{N,i}}{positive constant used to show exponential convergence in \eqref{eq:4fintitetimecovergencetwolayer}}
    \notn{\beta_{1,i},\beta_{2,i}}{constants entering bounds \eqref{eq:assumption3twolayer}, Assumption \ref{assumption:steadystatedecreasetwolayers} and \eqref{eq:assumption3}, Assumption \ref{assumption:steadystatedecreaseonelayer}}
    \notn{\beta_{T,i}}{constant entering bound \eqref{eq:upperboundtargetcost}, Assumption \ref{assumption:steadystatedecreasetwolayers} and \eqref{eq:assumption33}, Assumption \ref{assumption:steadystatedecreaseonelayer}}
    \notn{\gamma_i}{constant for tracking cost bound \eqref{eq:assumption2gammavmax}, Assumption \ref{assumption:expocostcontrollability}}
    \notn{\gamma'}{maximal value of $\gamma_{i}'$ over all agents}
    \notn{\gamma_{i}'}{constant entering \eqref{eq:assumption22}, Proposition \ref{proposition:upperboundsuccedingstagecost}}
    \notn{\gamma_{V_{\max}}}{maximal value of $\gamma_{i}'$ and $\gamma_{i,V_{\max,i}}$ over all agents}
    \notn{\gamma_{i,V_{\max,i}}}{constant entering \eqref{eq:upperbound2layer1}, in proof of Proposition \ref{proposition:upperlowerboudJtilde}}
    \notn{\delta_i}{wheel orientation with respect to neutral position of the $i$-th car}
    \notn{\delta_{r,i}}{reference wheel orientation with respect to neutral position of the $i$-th car}
    \notn{\bar{\delta}}{constant between 0 and 1 entering \eqref{eq:learningprop1} in Proposition \ref{proposition:convergenceofcentroidstwolayerslearning}}
    \notn{\epsilon}{small but fixed positive constant to define the interior of a set}
    \notn{\epsilon^{\prime}}{constant between 0 and 1 entering bounds in Assumptions \ref{assumption:steadystatedecreasetwolayers} and \ref{assumption:steadystatedecreaseonelayer}}
    \notn{\varepsilon}{positive constant in \eqref{eq:learningprop1}, Proposition \ref{proposition:convergenceofcentroidstwolayerslearning}}
    \notn{\bar{\epsilon}}{positive constant to show fulfillment of \eqref{eq:updatereq1},~\eqref{eq:updatereq2} in Section \ref{subsection:finitetimeconditiontwolayersp1}}
    \notn{\tilde{\epsilon}}{positive constant in proof of Proposition \ref{proposition:onelayerlearning}}
    \notn{\zeta}{dummy variable entering the definition of $d_i$}
    \notn{\eta_{i,j}}{dummy exponent entering the definition of $d_i$}
    \notn{\theta}{vector of linear combinators describing the density, Assumption \ref{assumption:bayesianconvergence}}
    \notn{\kappa}{$\mathcal{K}$-function}
    \notn{\lambda}{scaling factor in one-layer MPC cost, \eqref{eq:fullnonlineartrackingmpccostonelayer}}
    \notn{\mu}{mean vector}
    \notn{\nu_{i,h}}{additive noise entering density measurement model \eqref{eq:measmod}}
    \notn{\xi_{1,i},\xi_{2,i}}{constants entering bound \eqref{eq:boundtwosteadystatestwolayer}, Assumption \ref{assumption:expocostcontrollability}}
    \notn{\rho}{bound for position error $e_{v,i,k}$}
    \notn{\rho_{1,...,4}, \rho_{u}}{stage cost exponents used in the experiments}
    \notn{\sigma^2}{noise variance entering density measurements \eqref{eq:measmod}}
    \notn{\phi}{density function, $\phi: \mathbb{A} \rightarrow \mathbb{R}_+$}
    \notn{\phi_1}{density function used in experiments for known environments}
    \notn{\phi_2}{density function used in experiments for unknown environments}
    \notn{\hat{\phi}_{t}}{estimated density at time index $t$}
    \notn{\hat{\phi}_{t}(p)}{estimated density at position $p$ and time index $t$ \eqref{eq:thetaupdate}}
    \notn{\hat{\phi}_{\infty}}{converged estimate for $\phi$}
    \notn{\chi_i}{bound on optimal stage cost, Assumption \ref{assumption:expocostcontrollability}}
    \notn{\psi_i}{orientation with respect to x-axis of the $i$-th car}
    \notn{\psi_{r,i}}{reference orientation with respect to x-axis of the $i$-th car}
\end{flalign*}

\end{document}