\section{A test-space only discretization}

As indicated in the previous section, we propose to use the normal equation~\eqref{eq:transport:continuousNormalEq} to define an optimally stable approximation scheme.
It is thus obvious, that a discretization can be fully based on a discrete approximate test space $\ycal^\delta$.

\subsection{Discrete normal equation and functional reconstruction}

Let $\ycal^\delta \subseteq \ycal$ be a conforming discretization of the optimal test space (\ie using a standard Lagrange finite element space). Based on~\eqref{eq:transport:continuousNormalEq} we then define the \textit{discrete} normal equation using Galerkin-projection.
\begin{equation}\label{eq:discreteNormalEq}
\text{Find}\; w^\delta \in \ycal^\delta: \quad {(A^*[w^\delta], A^*[v^\delta])}_{L^2(\Omega)} = f(v^\delta) \qquad \forall v^\delta \in \ycal^\delta.
\end{equation}
Note that this is still an optimally conditioned problem. Given the discrete solution $w^\delta$ we may reconstruct the discrete solution $u^\delta = A^*[w^\delta]$. Technically, this solution lies in the finite-dimensional subspace $\xcal^\delta := A^*[\ycal^\delta] \subseteq \xcal$, however, due to its non-accessible structure this space is of no practical use.

Previous work often used knowledge of the structure of $A^*$ to determine a larger, more traditional (DG-)space $\zcal^\delta \supsetneq \xcal^\delta$ and then assembled the matrix $\underline{A}$ representing the operator $A^*: \xcal^\delta \rightarrow \zcal^\delta$ in the respective standard FE-bases. In this case, one can determine the system matrix $\underline{A}^{NE}$ of the normal equation as $\underline{A}^{NE} = \underline{A}^T \underline{M}_\zcal \underline{A}$ (where $\underline{M}_{\zcal}$ denotes the inner-product matrix in $\zcal^\delta$), solve the linear system\vspace*{-0.5em}
\begin{equation}\label{eq:linearEquationSystem}
  \underline{A}^{NE}\underline{w} = \underline{f}\vspace*{-0.5em}
\end{equation}
and compute the coefficients $\underline{u}$ of $u^\delta \in\xcal^\delta \subset \zcal^\delta$ in the basis of $\zcal^\delta$ by simply computing $\underline{u} = \underline{A}\,\underline{w}$.

However, this is suboptimal as the construction of a discrete larger space $\zcal^\delta$ is only feasible or even possible with suitable additional assumptions on the data, e.g. (elementwise) constant data functions. For non-constant reaction or velocities one has to resort to a nonconforming choice $\zcal^\delta \not\supset \xcal^\delta$ introducing an additional projection error which might be difficult to estimate or control.

Here, we propose an approach that avoids ever computing a matrix $\underline{A}$ representing the operator $A^*$. The system matrix of the normal equation $\underline{A}^{NE}$ can also be directly assembled in a basis of $\ycal^\delta$ which means basically assembling a normal equation using the full infinite dimensional trial space $\xcal$. The reconstruction $u^\delta := A^*[w^\delta]$ is now seen as an element of $\xcal$ (we technically know that it lies in the finite dimensional subspace $\xcal^\delta \subset \xcal$ but this does not give us any useful information). The crucial insight is that in almost all applications only functional evaluations of $u^\delta$ are needed. Examples include point-evaluations for the visualization of $u^\delta$ or the computation of quantities of interest via numerical quadrature (\ie $\norm{u^\delta}$). Therefore, we replace the reconstruction by functional evaluations and e.g. do a pointwise reconstruction. Note that in this way we do not introduce any additional projection error.

\subsection{Conditioning of the system matrix and solving the linear system}
Solving the linear equation system~\eqref{eq:linearEquationSystem} is actually quite a challenging task - a problem that has to our knowledge not been discussed so far. Although Problem~\eqref{eq:discreteNormalEq} is optimally stable in theory, the condition of the system matrix $\underline{A}^{NE}$ still scales quadratically in the inverse grid width $h^{-1}$ and is thus a significant challenge even for moderately large problems. To better understand these seemingly conflicting statements consider the non-symmetric formulation of~\eqref{eq:discreteNormalEq}:
\begin{equation}\label{eq:discreteNonsymmetricEq}
\text{Find}\; u^\delta \in \xcal^\delta: \quad (u^\delta, A^*[v^\delta]) = f(v^\delta) \qquad \forall v^\delta \in \ycal^\delta.
\end{equation}
Let $\{\psi_i\}_{i=1}^{N}$ be a basis of $\ycal^\delta$ (\eg a finite element basis). Then, the set $\{ \varphi_i \}_{i=1}^N$, $\varphi_i := A^*[\psi_i]$ forms a basis of $\xcal^\delta$ and the matrix $\underline{A}$ representing $A^*$ in these bases is the identity matrix. The condition of the system matrix $\underline{A}^{NE}$ is still of order $\mathcal{O}(h^{-2})$ since the trial functions $\varphi_i$ have, contrary to classic finite elements, in this case a magnitude of $\mathcal{O}(h^{-1})$.

As mentioned in Remark~\ref{rmk:strongTestProblem}, the normal equation can also be seen as the weak form of a specific Poisson-problem with rank-deficient diffusion tensor $D$. In the following numerical experiments we thus employed an algebraic multigrid for preconditioning and a conjugate gradient (CG) solver - methods that are known to perform well for this type of problems. For more complex problems (\eg for velocity fields with (locally) small magnitude) the efficient preconditioning and solving of the linear equation system~\eqref{eq:linearEquationSystem} still needs further investigation.
