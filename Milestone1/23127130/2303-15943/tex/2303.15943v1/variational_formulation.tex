\section{Ultraweak Petrov-Galerkin variational formulation and related normal equation}
\newcommand{\Ltwoout}{L^2(\Gamma_+, |\vec{b}\vec{\nu}|)}

%Let $\Omega\subset\R^n$ be an open, bounded polyhedral domain with Lipschitz-boundary $\Gamma := \partial\Omega$. Given a divergence-free\footnote{One can also consider non-divergence free fields.} transport field $\vec{b}\in H^1(\mathop{div}, \Omega)$ and reaction $c\in L^\infty(\Omega)$ we consider the linear transport equation
%\begin{equation}\label{eq:transport:strongProblem}
%\begin{cases}
%A_\circ u := \nabla\cdot(\vec{b} u) + cu &= f_\circ \qquad  \text{in }\;\Omega,\\
%\hfill u &= g_D \hfill \text{on }\Gamma_- \\
%\end{cases}
%\end{equation}
%with source term $f_\circ\in L^2(\Omega)$ and boundary values $g_D\in L^2(\Gamma_-)$. Here, the boundary parts $\Gamma_\pm$ are defined as the in- and outflow boundary, \ie
%\[
%\Gamma_\pm := \{ z\in\Gamma \;|\; \vec{b}(z)\cdot \nu(z) \gtrless 0\}
%\]
%where $\nu(z)$ denotes the outer unit normal at $z$.
%For continuous spaces $U=C^\infty_{\Gamma_-}(\Omega)$\footnote{This notation shall denote the $C^\infty$-functions vanishing on $\Gamma_-$.} and $V=C^\infty(\Omega)$ we obtain the variational formulation
%\begin{equation}
%\text{Find}\; u\in U: \qquad a(u,v) \;=\; f(v) \qquad \forall v\in V
%\end{equation}
%where
%\begin{equation}
%a(u,v) := {(A_\circ u, v)}_{L^2(\Omega)}, \qquad f(v) := {(f_\circ, v)}_{L^2(\Omega)}
%\end{equation}

Following the ideas developed in \cite{BrunkenSmetanaUrban}, we derive an ultraweak variational Petrov-Galerkin formulation for the reactive transport problem~\eqref{eq:transport:strongProblem}
as well as a corresponding normal equation that only depends on the
test space.
% In order to do so, let us
We
start with the definition of the formal adjoint operator $A_\circ^*:V \to U'$ given through
\begin{equation}
  v\mapsto A_\circ^*[v], \qquad (A_\circ^*[v])(u) := {(u,-b\nabla v + cv)}_{L^2(\Omega)} + {(u\restr{\Gamma_+}, v\restr{\Gamma_+})}_{\Ltwoout}
\end{equation}
using the weighted inner product
$
{(u,v)}_{L^2(\Gamma_{\pm}, |\vec{b}\vec{\nu}|)} := \int_{\Gamma_\pm} u\,v\,|\vec{b}\vec{\nu}| \diff s .
$
%of a weighted $L^2$-product\footnote{The corresponding term on the inflow boundary will later appear in the righthand side.}.

\subsection{Choice of appropriate function spaces}

Let us start with regular spaces $U=C^\infty_{\Gamma_-}(\Omega)$, the space of $C^\infty$-functions vanishing on $\Gamma_-$ and $V=C^\infty(\Omega)$.
For given $v\in V$ the operator $A_\circ^*[v]$ is continuous, and hence in $U'$, if we choose
\begin{equation}
  \norm{u}_U := \sqrt{\norm{u}_{L^2(\Omega)}^2 + \norm{u\restr{\Gamma_+}}_{\Ltwoout}^2}
\end{equation}
as the norm on $U$. Via closure we obtain the ultraweak trial space $\xcal := \clos_{\norm{\cdot}_U}(U)$ equipped with the continuous extension of the norm $\norm{\cdot}_U$.
For this trial space we have the following characterization (without proof).

\begin{proposition}
There exists a linear and continuous trace operator
\begin{equation*}
\gamma: \xcal \rightarrow \Ltwoout
\end{equation*}
as well as a linear and continuous projection operator
\begin{equation}
  \mathrm{pr}_{L^2}: \xcal \rightarrow L^2(\Omega)
\end{equation}
fulfilling $\gamma(u) = u\restr{\Gamma_+}$ and $\mathrm{pr}_{L^2}(u) = u$ for all $u\in U \subset \xcal$.
\end{proposition}
%\begin{proof}
%  Define the linear and bounded operators $\gamma(u) := u\restr{\Gamma_+}$ and $\mathrm{pr}_{L^2}(u) := u$ for $u\in U$ and continuously extend to $\xcal$.
%\end{proof}

\begin{proposition}\label{thm:trialIsometry}
  The trial space $\xcal$ is isometrically isomorphic to the Sobolev-space $\xcal_{L^2}$ defined as
  \[
  \xcal_{L^2} := L^2(\Omega) \times \Ltwoout
  \]
  equipped with the canonical norm $\norm{(u, \hat{u})}_{\xcal_{L^2}}^2 := \norm{u}_{L^2(\Omega)}^2 + \norm{\hat{u}}_{\Ltwoout}^2$.
  The isometry $\Phi: \xcal \rightarrow \xcal_{L^2}$ is given as
  \begin{equation*}
    \Phi(u) := (pr_{L^2}(u), \gamma(u)).
  \end{equation*} 

\end{proposition}
%\begin{proof}
%  Define the linear map
%  \begin{equation*}
%    \Phi: U \rightarrow \xcal_{L^2}, \qquad \Phi(u) := (pr_{L^2}(u), \gamma(u)).
%  \end{equation*}
%  Then, we have by definition $\norm{u}_U = \norm{\Phi(u)}_{\xcal_{L^2}}$. Since $\xcal = \clos_{\norm{\cdot}_U}(U)$, it remains to show that also
%  \begin{equation}
%    \clos_{\norm{\Phi(\cdot)}_{\xcal_{L^2}}}(U) = L^2(\Omega) \times \Ltwoout
%  \end{equation}
%
%  To that end, let $(u,\hat{u})\in L^2(\Omega)\times \Ltwoout$. Then there exist sequences
%  \begin{align*}
%    &(\varphi_k) \subset U, \qquad &pr_{L^2}(\varphi_k) \xrightarrow{L^2(\Omega)} u, \quad &\gamma(\varphi_k) = 0 \; &\text{($C_0^\infty(\Omega)$ is dense in $L^2(\Omega)$)}\\
%    &(\hat{\varphi}_l) \subset U, \qquad &pr_{L^2}(\hat{\varphi}_l) \xrightarrow{L^2(\Omega)} 0, \quad &\gamma(\hat{\varphi}_l) \xrightarrow{\Ltwoout} \hat{u} \; &\text{(Dirac-sequence)}
%  \end{align*}
%  it follows that
%  \begin{equation*}
%    \Phi(\varphi_k + \hat{\varphi}_k) \;\xrightarrow{\xcal_{L^2}}\; (u,\hat{u})
%  \end{equation*}
%  and thus the claim.
%\end{proof}

\begin{corollary}
  It holds that
\begin{align*}
  \norm{x}_\xcal^2 &=\quad \norm{\mathrm{pr}_{L^2}(x)}_{L^2(\Omega)}^2 + \norm{\gamma(x)}_{\Ltwoout}^2 = {(x,x)}_\xcal \\
\text{with } \  (x,x')_\xcal &:=\quad {(\mathrm{pr}_{L^2}(x), \mathrm{pr}_{L^2}(x'))}_{L^2(\Omega)} + {(\gamma(x), \gamma(x'))}_{\Ltwoout}.
\end{align*}
In particular, $\xcal$ is a Hilbert space and there exists the Riesz-map $R_\xcal: \xcal' \rightarrow \xcal$.
\end{corollary}
\par
Hence, by continuous extension we can interpret $A_\circ^*[v]$ as an operator acting on $\xcal$, where in the following we will write $(u, \hat{u})\in\xcal$ in the sense of  Prop.~\ref{thm:trialIsometry}, \ie
\begin{equation}
A_\circ^*[v](u,\hat{u}) := {(u, -b\nabla v +cv)}_{L^2(\Omega)} + {(\hat{u}, v\restr{\Gamma_+})}_{\Ltwoout}.
\end{equation}

Provided that $A_\circ^*$ is injective on $V$ (which holds under mild assumptions on the data functions, see~\cite[Prop.~2.2]{BrunkenSmetanaUrban}) we are now in the position to define a norm on $V$ as follows
\begin{equation}
  \norm{v}_V := \norm{A_\circ^*[v]}_{\xcal'} = \norm{R_\xcal(A_\circ^*[v])}_\xcal.
\end{equation}

By a simple variational argument one sees that for a given $v\in V$ the Riesz-representative $r_v := R_\xcal(A_\circ^*[v])\in\xcal$ fulfills $\Phi(r_v) = (-b\nabla v + cv, v\restr{\Gamma_+})$. We thus obtain
\begin{equation*}
\norm{v}_V^2 = \norm{r_v}_\xcal^2 = \norm{\Phi(r_v)}_{\xcal_{L^2}}^2 = \norm{-b\nabla v + cv}_{L^2(\Omega)}^2 + \norm{v\restr{\Gamma_+}}_{\Ltwoout}^2.
\end{equation*}
By closure we finally obtain the test space $\ycal := \clos_{\norm{\cdot}_V}(V)$ of our ultraweak formulation equipped with the norm
\begin{equation}
\norm{y}_\ycal := \norm{A^*[y]}_{\xcal'}
\end{equation}
where $A^*:\ycal \rightarrow \xcal'$ is the continuous extension of $A_\circ^*$ to $\ycal$.

The following conjecture states our current intuition regarding the structure of the test space $\ycal$:
\begin{conjecture}
  The test space $\ycal$ is isometrically isomorphic to the Sobolev-space
  \begin{equation}
H^1(\vec{b}, \Omega) := \{v\in L^2(\Omega) \;|\; \vec{b}\nabla v \in L^2(\Omega)\}.
  \end{equation}
\end{conjecture}
%\info{The conjecture requires a proof that boundedness of $\norm{\vec{b}\nabla v}_{L^2}$ implies boundedness of $\norm{\gamma(v)}_{\Ltwoout}$}

\subsection{Optimally stable ultraweak formulation and normal equation}
With the definitions of the trial space $\xcal$, the test space $\ycal$ and the adjoint operator $A^*$ we are now prepared to give an ultraweak 
variational formulation for reactive transport as follows.
\begin{definition}[Ultraweak variational formulation of reactive transport]
$u\in\xcal$ is called a solution of the ultraweak variational formulation, if it satisfies
\begin{equation} \label{eq:continuousProblem}
 \quad {(u,A^*[v])}_{\xcal \times \xcal'} = f(v) \qquad \forall v\in\ycal.
\end{equation}
with righthand side
$
f(v) := {(f_\circ, v)}_{L^2(\Omega)} - {(g_D, v)}_{\Ltwoout}.
$
\end{definition}
\begin{proposition}\label{prop:genericIsometry}
The mappings $A: \xcal\rightarrow\ycal'$ and $A^*:\ycal\rightarrow \xcal'$ are isometries, \ie
\[
\ycal = A^{-*}\xcal', \quad \xcal = A^{-1}\ycal'
\]
and
\[
\norm{A}_{\mathcal{L}(\xcal,\ycal')} = \norm{A^*}_{\mathcal{L}(\ycal,\xcal')}
= \norm{A^{-1}}_{\mathcal{L}(\ycal',\xcal)} = \norm{A^{-*}}_{\mathcal{L}(\xcal',\ycal)}
= 1.
\]
\end{proposition}
\begin{proof}
%  \todo{Show the assumptions on $A^*$}
  The proof follows the argumentation in~\cite[Prop.~2.1]{DahmenHuangSchwab}.
\end{proof}

As a consequence we obtain the following corollary, which shows optimal stability of the ultraweak variational formulation.
\begin{corollary}[Optimal stability]\label{cor:generalProblem}
The variational formulation~\eqref{eq:continuousProblem}
is well-posed and has optimal condition number $\kappa_{\xcal,\ycal}(A) := \norm{A}\,\norm{A^{-1}} = 1$.
\end{corollary}
%\begin{proof}
%It only remains to show that the chosen $f$ is a linear and continuous functional on $\ycal$.
%\end{proof}

Since by Proposition\ \ref{prop:genericIsometry} every $u\in\xcal$ has a representation $u = R_\xcal A^*[w]$ for some $w\in\ycal$ we can substitute $u$
in the ultraweak formulation~\eqref{eq:continuousProblem} to obtain the equivalent (continuous) normal equation.
\begin{definition}[Normal equation of the ultraweak formulation]
$w\in\ycal$ is called a solution of the normal equation of the ultraweak formulation, if it satisfies
\begin{equation}\label{eq:transport:continuousNormalEq}
	 {(A^*[w], A^*[v])}_{\xcal'} \;=\;  f(v) \qquad \forall v\in\ycal
\end{equation}
or equivalently
\begin{equation}
 {(-b\nabla w + cw, -b\nabla v + cv)}_{L^2(\Omega)} + {(\gamma(w),\gamma(v))}_{\Ltwoout} \quad =\quad f(v).
\end{equation}
\end{definition}
Note, that this is not a `classic` normal equation since we only substituted the solution variable but did not modify the righthand side $f$.
\begin{remark}\label{rmk:strongTestProblem}
Let $w$ denote a solution of the normal equation~\eqref{eq:transport:continuousNormalEq}. If $w$ is regular enough (\eg $w\in C^2(\Omega)$), then $w$ solves the degenerated Poisson problem
\begin{equation}
\begin{cases}
  -\nabla\cdot(D\nabla w) &= f_\circ \qquad \text{in}\; \Omega \\
  -b\nabla w &= g_D \qquad \text{on}\; \Gamma_- \\
  (D\nabla w)\vec{\nu} + w &= 0 \qquad \text{on}\; \Gamma_+ \\
\end{cases}
\end{equation}
with a rank-$1$ diffusion tensor $D := \vec{b}\otimes\vec{b}$.
\end{remark}
The equivalent formulation of the normal equation will serve as the starting point for the definition of an optimal stable approximation method in the following section.
