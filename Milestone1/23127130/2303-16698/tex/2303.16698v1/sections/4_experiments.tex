\section{Experiments}
\label{sec:experiments}

We evaluated our proposed method on simulated data of 
two classic control tasks (Pendulum and CartPole) and
two behavioral human tasks (reaching and navigation).
To evaluate the accuracy of the parameter estimates obtained by our method and to compare it against a baseline, we computed absolute relative errors per parameter $\lvert (\theta - \hat{\theta}) / \theta \rvert$. This makes averages across parameters on different scales more interpretable compared to other metrics like root mean squared errors.
For each task, we simulated 100 sets of parameters from a uniform distribution in logarithmic space. For each set of parameters, we simulated 50 trajectories. We then maximized the log likelihood using gradient-based optimization with automatic differentiation \citep[L-BFGS algorithm;][]{zhu1997algorithm}. See \cref{app:hyperparams} for a summary of the hyperparameters of our experiments.

\begin{figure}[H]
    \centering
    \includegraphics[width=\linewidth]{figures/reaching-loglike.pdf}
    \vspace{-0.5cm}
    \caption{\textbf{IOC likelihood for the non-linear reaching task.} \textbf{(a)} Simulated reaching trajectories for eight targets. Increasing the cost of actions and the motor noise has an effect on the trajectories, since perfectly reaching the target becomes less important to the agent and variability increases. \textbf{(b)} IOC log likelihood for two of the model parameters, action costs $c_a$ and motor noise $\sigma_m$. The likelihood has its maximum (pink cross) close to the ground truth parameter values (black dot). \textbf{(c)} Simulated trajectories using the MLEs from (b). The simulations are visually indistinguishable from the ground truth data.}
    \label{fig:reaching-likelihood}
\end{figure}

All tasks we consider have four free parameters: cost of actions $c_a$, cost of velocity at the final time step $c_v$, motor noise $\sigma_m$, and observation noise $\sigma_o$. In the fully observable case, we leave out the observation noise parameter and only infer the three remaining parameters. For concrete definitions of the parameters in each specific tasks, see \cref{app:tasks}.

\subsection{Baseline method}
For a comparison to previously proposed methods, we applied a baseline method based on the maximum causal entropy (MCE) approach \citep{ziebart2010modeling}, as these formulations have been successfully used for IOC in non-linear stochastic systems. As to the best of our knowledge, no past method can be directly applied in the setting we consider (non-linear and stochastic dynamics, unknown controls, partial observability, finite horizon), we choose a straight-forward implementation of the MCE formulation for this case. The tasks we consider can be well solved by linearizing the dynamics locally, so an accurate approximation of the optimal MCE controller is given by the optimal MCE controller for the linearized dynamics (see \cref{sec:mcerl}). An estimate of the parameters is then obtained by maximizing the likelihood of the approximate maximum entropy policy for the given set of states and controls.
To apply this baseline to the setting where control signals are missing, we use the estimates of the controls as we determine in our proposed method for the data-based linearization (\cref{sec:fixed_linearization}). As past IOC methods do not have an explicit model of partial observability, except for a few exceptions which are limited to specific tasks, 
we follow the usual formulation of the policy acting directly on the states. 
To show that this approach constitutes a suitable baseline, in \cref{app:results_controls}, we provide results for the case where the true control signals are known and there is no partial observability.

\begin{figure*}[h]
    \centering
    \includegraphics[width=.9\linewidth]{figures/estimates_po_reaching.pdf}
    \vspace{-0.6 cm}
    \caption{\textbf{Maximum likelihood estimates for reaching task} True parameter values plotted against the maximum likelihood parameter estimates for the partially observable reaching task. Top row: our method, bottom row: MCE baseline. The columns contain the four different model parameters (action cost $c_a$, velocity cost $c_v$, motor noise $\sigma_m$, observation noise $\sigma_o$).}
    \label{fig:reaching-mle}
\end{figure*}



\subsection{Reaching task}
We evaluate the method on a manual reaching task with a non-linear biomechanical model of a two-link arm. The agent's goal is to move its arm towards a target in the horizontal plane by controlling its two joints. For a more detailed description, see \cref{app:reaching}. Note that the cost function is non-linear in states because the positions are a non-linear function of the joint angles that comprise the state of the system.
We use a fully observable version of the task \citep{todorov2005generalized} and a version in which the agent receives noisy observations \citep{li2007iterative}. This model has been applied to reaching movements in the sensorimotor neuroscience literature \citep[e.g.,][]{nagengast2009optimal, knill2011flexible}.
\cref{fig:reaching-likelihood}a shows simulations from the model using iLQG with two different parameter settings. We evaluated the likelihood function for a grid of two of the model parameters (\cref{fig:reaching-likelihood}b) to illustrate that it has well-defined maxima close to the true parameter values. In this example, simulated data using the maximum likelihood estimates look indistinguishable from the ground truth data (\cref{fig:reaching-likelihood}c).

In \cref{fig:reaching-mle} we present maximum likelihood parameter estimates and true values for repeated runs with different random parameter settings. One can observe that the parameter estimates of our method closely align with the true parameter values, showing that our method can successfully recover the parameters from data. The baseline method, in contrast, shows considerably worse performance, in particular for estimating noises. Estimates for the fully observable case are provided in \cref{app:results_fullobs}.
To quantify the accuracy of the maximum likelihood estimates, we computed the absolute relative errors. The results are shown separately for the fully observable and partially observable cases in \cref{fig:results-aggregate}.
The median absolute relative errors of our method were 0.11, while they were 0.93 for the baseline.
The influence of missing control signals and of the lacking explicit observation model in the baseline can be observed by comparing the results to the fully-observable case and the case of given control signals in \cref{app:results_fullobs} and \cref{app:results_controls}.


\subsection{Navigation task}
In the navigation task, we consider an agent navigating to a target under non-linear dynamics while receiving noisy observations from a non-linear observation model. To reach the target, the agent can control the angular velocity of their heading direction and the acceleration with which they move forward. The agent observes noisy versions of the distance to the target and the target's bearing angle. We provide more details about the experiment in \cref{app:navigation}.

Maximum likelihood parameter estimates for the navigation task are shown for the partially observable case in \cref{fig:estimates_po_navigation} and for the fully observable case in \cref{fig:estimates_navigation}. As for the reaching task, our method provides parameter estimates close to the true ones, while the estimates of the baseline deviate for a large number of trials. Median absolute relative errors of our method were 0.31, while they were 1.99 for the baseline (\cref{fig:results-aggregate}).


\subsection{Classic control tasks}
Lastly, we evaluate our method on two classic control tasks (Pendulum and Cart Pole) based on the implementations in the \texttt{gym} library \citep{brockman2016openai}. Because these tasks are neither stochastic nor partially observable in their standard formulations, we introduce noise on the dynamics and turn them into partially-observed problems by defining a stochastic observation function (see \cref{app:classic-control}).
In \cref{app:add_results} we show the parameter estimates for the Pendulum (\cref{fig:estimates_po_pendulum}) and for the Cart Pole (\cref{fig:estimates_po_cartpole}) for the partially observable case, while \cref{fig:estimates_pendulum} and \cref{fig:estimates_cartpole} show the fully observable case, respectively. One can observe that the results match the ones of the reaching and navigation task, showing that our method provides accurate estimates of the parameters. Median absolute relative errors of our method were 0.12 and 0.41, while for the baseline they were 2.21 and 3.82 (\cref{fig:results-aggregate}).





\begin{figure*}
    \centering
    \includegraphics[width=\linewidth]{figures/errors.pdf}
    \vspace{-1cm}
    \caption{\textbf{Evaluation across tasks. } Absolute relative errors (log scale) for different tasks. Our method consistently outperforms the MCE baseline.}
    \label{fig:results-aggregate}
\end{figure*}

