% Recommended, but optional, packages for figures and better typesetting:
\usepackage{microtype}
\usepackage{graphicx}
\usepackage{subfigure}
\usepackage{booktabs} % for professional tables
\usepackage{wrapfig}
\usepackage[affil-it]{authblk}


% hyperref makes hyperlinks in the resulting PDF.
% If your build breaks (sometimes temporarily if a hyperlink spans a page)
% please comment out the following usepackage line and replace
\usepackage{hyperref}




% Attempt to make hyperref and algorithmic work together better:
\newcommand{\theHalgorithm}{\arabic{algorithm}}

% Use the following line for the initial blind version submitted for review:
\usepackage{preprint}


% For theorems and such
\usepackage{amsmath}
\usepackage{amssymb}
\usepackage{mathtools}
\usepackage{amsthm}
\usepackage{amsfonts}
\usepackage{bm}             % bold math symbols

\usepackage{layouts}

% if you use cleveref..
\usepackage[capitalize,noabbrev]{cleveref}


%%%%%%%%%%%%%%%%%%%%%%%%%%%%%%%%
% THEOREMS
%%%%%%%%%%%%%%%%%%%%%%%%%%%%%%%%
\theoremstyle{plain}
\newtheorem{theorem}{Theorem}[section]
\newtheorem{proposition}[theorem]{Proposition}
\newtheorem{lemma}[theorem]{Lemma}
\newtheorem{corollary}[theorem]{Corollary}
\theoremstyle{definition}
\newtheorem{definition}[theorem]{Definition}
\newtheorem{assumption}[theorem]{Assumption}
\theoremstyle{remark}
\newtheorem{remark}[theorem]{Remark}

% Todonotes is useful during development; simply uncomment the next line
%    and comment out the line below the next line to turn off comments
%\usepackage[disable,textsize=tiny]{todonotes}
\usepackage[textsize=tiny]{todonotes}

       % blackboard math symbols
% \usepackage{nicefrac}       % compact symbols for 1/2, etc.
% \usepackage{microtype}      % microtypography
\usepackage{xcolor}         % colors

% \usepackage[ruled,vlined]{algorithm2e}
% \usepackage{amssymb}
% \usepackage{mathtools}
% hyperref makes hyperlinks in the resulting PDF.
% If your build breaks (sometimes temporarily if a hyperlink spans a page)
% please comment out the following usepackage line and replace
% \usepackage{hyperref}
% \usepackage[capitalize,noabbrev]{cleveref}

\graphicspath{{figures/}{fig/}}

\newcommand{\eg}{e.g.,\xspace}
\newcommand{\ie}{i.e.,\xspace}
\newcommand{\wrt}{w.r.t.\@\xspace}

%%%% Math Operators %%%%
%Linear Algebra
\DeclareMathOperator{\diag}{\mathrm{diag}\,}
\DeclareMathOperator{\tr}{\mathrm{tr}\,}

%Optimization
\DeclareMathOperator*{\argmin}{\arg\min}
\DeclareMathOperator*{\argmax}{\arg\max}

%Sets
\DeclareMathOperator{\R}{\mathbb{R}}
\DeclareMathOperator{\N}{\mathbb{N}}
\DeclareMathOperator{\Z}{\mathbb{Z}}

%Indicator
\DeclareMathOperator{\1}{\mathbbm{1} }

%Probability
\DeclareMathOperator{\Cov}{Cov\,}

\newcommand{\innermid}{\nonscript\;\delimsize\vert\nonscript\;}
\newcommand{\activatebar}{%
  \begingroup\lccode`\~=`\|
  \lowercase{\endgroup\let~}\innermid 
  \mathcode`|=\string"8000
}

\DeclareMathOperator{\Var}{\mathrm{Var}\,}
\DeclareMathOperator{\KL}{\mathrm{KL}\,}
\newcommand{\Div}{\;\Vert\;} %Divergence operartor


%Misc
\newcommand\littleO{o}
\DeclareMathOperator{\BetaFun}{\mathrm{B}}

%Distribution
\DeclareMathOperator{\GamDis}{\mathrm{Gamma}}
\DeclareMathOperator{\BetaDis}{\mathrm{Beta}}
\DeclareMathOperator{\CatDis}{\mathrm{Cat}}
\DeclareMathOperator{\ExpDis}{\mathrm{Exp}}
\DeclareMathOperator{\CRPDis}{\mathrm{CRP}}
\DeclareMathOperator{\LogNDis}{\mathrm{Lognorm}}
\DeclareMathOperator{\UniformDis}{\mathrm{Uniform}}
\DeclareMathOperator{\NDis}{\mathcal{N}}
\DeclareMathOperator{\NormalDis}{\mathcal{N}}
\DeclareMathOperator{\GIGDis}{\mathcal{G}\mathcal{I}\mathcal{G}}
\DeclareMathOperator{\MultDis}{\mathrm{Mult}}
\DeclareMathOperator{\BinDis}{\mathrm{Bin}}
\DeclareMathOperator{\DirDis}{\mathrm{Dir}}
\DeclareMathOperator{\PGDis}{\mathrm{PG}}
\DeclareMathOperator{\PoisDis}{\mathrm{Pois}}
\DeclareMathOperator{\onehot}{\mathrm{OneHot}}
\DeclareMathOperator{\PP}{\mathcal{PP}}

\newcommand{\td}{\mathrm{d}}

\usepackage{acronym}
\acrodef{RL}{reinforcement learning}
\acrodef{MDP}{Markov Decision Process}
\acrodefplural{MDP}{Markov Decision Processes}

% Bayes-nets
\usepackage{tikz}
\usetikzlibrary{bayesnet}

% rescaling
\usepackage{adjustbox}

% state action spaces
\newcommand{\stateset}{\mathcal{X}}
\newcommand{\actionset}{\mathcal{U}}
% \newcommand{\beliefset}{\mathcal{B}}
\newcommand{\beliefset}{\R^b}
\newcommand{\params}{{\bm \theta}}

\newcommand{\Rplus}{\R^+_0}

% probabilities and stuff
\newcommand{\Normal}{\mathcal{N}}
\DeclareMathOperator{\E}{\mathbb{E}}
\DeclareMathOperator{\Identity}{\mathit{I}}
\newcommand{\norm}[1]{\left\lVert#1\right\rVert}

% dynamical system
\newcommand{\state}{\bm x}
\newcommand{\statenoise}{{\bm v}}
\newcommand{\action}{\bm u}
\newcommand{\obs}{{\bm y}}
\newcommand{\obsnoise}{{\bm w}}
\newcommand{\eobs}{{\bm o}}
\newcommand{\eobsnoise}{{\bm \epsilon}}
\newcommand{\eobsdynamics}{j}
\newcommand{\dynamics}{f}

% belief
\newcommand{\belief}{\mathbf b}
\newcommand{\beliefdynamics}{\beta}

% policy
\newcommand{\policy}{\pi}
\newcommand{\policynoise}{{\bm \xi}}
\newcommand{\controlmatrix}{L}
\newcommand{\controloffset}{{\mathbf m}}

% joint system
\newcommand{\jointstate}{\mathbf z}

% control
% linearization
\newcommand{\nomstate}{\bar{\state}}
\newcommand{\nomaction}{\bar{\action}}

\newcommand{\lstate}{\mathbf x}
\newcommand{\laction}{\mathbf u}
\newcommand{\lobs}{\mathbf y}

% value function
\newcommand{\Sxx}{S}
\newcommand{\sx}{\mathbf s}
\newcommand{\s}{s}

% nicer equations
\newcommand\given[1][]{\:#1\vert\:}
\newcommand{\eqdef}{=\mathrel{\mathop:}}
\newcommand{\defeq}{\mathrel{\mathop:}=}

\renewcommand{\algorithmicrequire}{\textbf{Input:}}
\renewcommand{\algorithmicensure}{\textbf{Output:}}
