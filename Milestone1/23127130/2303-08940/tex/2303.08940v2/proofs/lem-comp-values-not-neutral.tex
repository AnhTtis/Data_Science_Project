\begin{proof}
    By case analysis on the form of $t \in \val$:
    \begin{itemize}
        \item Case $t = x$. Then we have to consider three cases according to the last rule used in $\Phi$:
        \begin{itemize}
            \item Case $\Phi$ ends with rule (\ruleAx), then $t$ can only be assigned $\sig$. Therefore, this case does not apply.
            \item Case $\Phi$ ends with rule (\ruleMany), then $\tau = \M \neq \tneutral$.
            \item Case $\Phi$ ends with rule (\ruleAxP), then $\tau = \nott{\tneutral} \neq \tneutral$.
        \end{itemize}
        \item Case $t = \lam x.t$. Then we have to consider three cases according to the last rule used in $\Phi$:
        \begin{itemize}
            \item Case $\Phi$ ends with rule (\ruleLam), then $t$ can only be assigned $\sig$. Therefore, this case does not apply.
            \item Case $\Phi$ ends with rule (\ruleMany), then $\tau = \M  \neq \tneutral$.
            \item Case $\Phi$ ends with rule (\ruleLamP), then $\tau = \vl \neq \tneutral$.
        \end{itemize}
    \end{itemize}
\end{proof}