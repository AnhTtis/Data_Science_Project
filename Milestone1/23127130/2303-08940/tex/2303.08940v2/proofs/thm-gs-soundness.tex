%\begin{proof}
    The proof follows by induction over $b+m$:
    \begin{itemize}
        \item Case $b+m = 0$. Then $b=m=0$, therefore $t \in \normal$, by point (1) of~\cref{lem:zero-counters}, and  $d = \size{t}$,  by point (2) of~\cref{lem:zero-counters}. Let $u = t$ and $q=s$, then  we can conclude since $\size{(u,q)} = \size{u} =\size{t} = d$.
        \item Case $b+m > 0$. Then $b>0$ or $m>0$, and in either case $t \not\in \normal$,  by~\cref{lem:zero-nfs}. Note that $(t,s)$ is not final because $t$ is unblocked by~\cref{prop:typed-unblock}. Therefore, by~\cref{prop:normal-iff-final} there exists $(t',s')$ such that $(t,s) \gsred (t',s')$. By~\cref{lem-exact-red-exp}.\ref{lem:subj-comp-red}, there exists $\Phi' \tr \seqi{\Gam}{(t',s')}{\ctype}{(b',m',d)}$, such that $b'+m'=b+m-1$. By the \ih, there exists $(u,q)$, such that $u\in \normal$, $(t',s') \gsrred^{(b',m')} (u,q)$ and $d = \size{(u,q)}$. So we can conclude with $(t,s) \gsred (t',s') \gsrred^{(b',m')} (u,q)$, which means that $(t,s) \drred^{(b,m)} (u,q)$, as expected.
    \end{itemize}
%\end{proof}
