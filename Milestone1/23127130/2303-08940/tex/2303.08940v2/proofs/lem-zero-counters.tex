%\begin{proof} 
  We replace the statement by the following three ones.
  \begin{enumerate}
    \item[(1.1)] If $\del = \tcomptype{\stype}{\tneutral}{\stype'}$, then $t \in \neutral$.
    \item[(1.2)] If $\del = \tcomptype{\stype}{\tightt}{\stype'}$, then $t \in \normal$.
    \item[(2)] $d = \size{t}$.
  \end{enumerate}
  We reason by simultaneous induction on tight derivations.
  \begin{itemize}
    \item Case $\Phi$ ends with (\ruleAx). This case does not apply since the resulting type is not a monadic type.
    \item Case $\Phi$ ends with (\ruleLam). This case does not apply since the resulting type is not a monadic type.
    \item Case $\Phi$ ends with (\ruleApp). Then the first counter in the conclusion of the derivation is necessarily greater  than $0$ and thus this case does not apply. 
    \item Case $\Phi$ ends with (\ruleMany). This case does not apply since the resulting type is not a monadic type.
    \item Case $\Phi$ ends with (\ruleLift). Then $\Phi$ cannot be tight. 
    \item Case $\Phi$ ends with (\ruleGet). Then the second counter in the conclusion of the derivation is necessarily greater than $0$ and thus this case does not apply. 
    \item Case $\Phi$ ends with (\ruleSet). Then the second counter in the conclusion of the derivation is necessarily greater than $0$ and thus this case does not apply.
    \item Case $\Phi$ ends with (\ruleAxP), so that $t=x$ and $s=0$. The condition of case (1.1) is not possible by construction. In case (1.2) we can conclude $x \in \val \subseteq \normal$. The statement (2) $d=0 = \size{x}$ is straightforward.  
    \item Case $\Phi$ ends with (\ruleLamP) so that $t=\lam x.u$ and $d=0$. The condition of case (1.1) is not possible by construction. In case (1.2) we can conclude $\lam x.u \in \val \subseteq \normal$. The statement (2) $d=0 = \size{\lam x.u}$ is straightforward. \item Case $\Phi$ ends with (\ruleAppPOne), so that $t = xu$ and $d = 1+d'$. If the condition of case (1.1) holds for $t$, that means that the condition of case (1.2) holds for $u$. By the \ih (1.2) $u \in \normal$ so that $xu \in \neutral$. The condition of case (1.2) holds for $t$ only for  $\tightt=\tneutral$, and then $xu \in \neutral$ holds by case (1.1), which implies $xu \in \normal$ by definition. To show statement (2), we apply the \ih (2) to $u$ and obtain $d'=\size{u}$, then $d=1+d' = 1 + \size{u} = \size{t}$.
    \item Case $\Phi$ ends with (\ruleAppPTwo), so that $t = (\lam x. p)u$ and $d = 1+d'$. If the condition of case (1.1) holds for $t$, that means that
    the condition of case (1.1) holds for $u$. By the \ih (1.1) $u \in \neutral$ so that $t \in \neutral$. The condition of case (1.2) holds for $t$ only for  $\tightt=\tneutral$, and then $t \in \neutral$ holds by case (1.1), which implies $t  \in \normal$ by definition. To show statement (3), we apply the \ih (3) to $u$ and obtain $d'=\size{u}$, then $d=1+d' = 1 + \size{u} = \size{t}$.
  \end{itemize}
%\end{proof}