\subsection{Soundness and Completeness}
\label{s:sound-complete-lambda-gs}

In this section we show the main properties of the type system $\sysgs$ with respect to the operational semantics of the $\lam$-calculus with global state. The properties of type system $\sysgs$ are similar to the ones for $\syscbv$, but now with respect to configurations instead of terms. \emph{Soundness} does not only state that a (tightly) typable configuration $(t,s)$ is terminating, but also gives exact (and split) measures concerning the reduction sequence from $(t,s)$ to a final form. \emph{Completeness} guarantees that a terminating configuration $(t,s)$ is tightly typable, where the measures of the associated reduction sequence of $(t,s)$ to final form are reflected in the counters of the resulting type derivation of $(t,s)$. This is the first work providing a model for a language with global memory being able to count the number of memory accesses.

We start by noting that type system $\sysgs$ does not type blocked configurations, which is exactly the notion that we want to capture.

\begin{restatable}[]{proposition}{proptypedunblock}
    \label{prop:typed-unblock}
    If $\Phi \tr \seqi{\Gam}{(t,s)}{\ctype}{(b,m,d)}$, then $(t,s)$ is unblocked.
\end{restatable}

We also show that counters capture the notion of normal form correctly, both for terms and states.

\begin{restatable}[]{lemma}{lemzerocounters} \mbox{}
  \label{lem:zero-counters-size-store}
  \begin{enumerate}
    \item \label{lem:zero-counters} Let $\Phi \tr \seqi{\Gam}{t}{\del}{(0,0,d)}$ be tight. Then, (1) $t \in \normal$ and (2) $d = \size{t}$.
    \item \label{lem:zero-size-store} Let $\Phi \tr \seqi{\Del}{s}{\stype}{(0,0,d)}$ be tight. Then $d = 0$.
  \end{enumerate}
\end{restatable}

In fact, we can show the following stronger property with respect to the counters for the number of $\betav$- and $\getname/\setname$-steps.

\begin{restatable}[]{lemma}{lemzeronfs}
    \label{lem:zero-nfs}
    Let $\Phi \tr \seqi{\Gam}{t}{\del}{(b,m,d)}$ be tight. Then, $b = m = 0$ iff $t \in \normal$.
\end{restatable}

The following property is essential for tight type systems~\cite{Accattoli2020}, and it shows that tightness of types spreads throughout type derivations of neutral terms, just as long as the environments are tight. 

\begin{restatable}[{\bf Tight Spreading}]{lemma}{lemcomtightspreading}
    \label{lem:comp-tight-spreading}
    Let $\Phi \tr \seqi{\Gam}{t}{\tcomptype{\stype}{\tau}{\stype'}}{(b,m,d)}$, such that $\Gam$ is tight. If  $t \in \neutral$, then $\tau \in \tightt$.
\end{restatable}

The two following properties ensure tight typability of final configurations. For that we need to be able to \emph{tightly} type any state, as well as any normal form. In fact, normal forms do not depend on a particular state since they are irreducible, so we can type them with any state type.

\begin{restatable}[{\bf Typability of States and Normal Forms}]{lemma}{typstates} \mbox{}
  \label{lem:typestatesnfs}
  \begin{enumerate}
      \item \label{lem:typ-states} Let $s$ be a state. Then, there exists $\Phi\ \tr \seqi{}{s}{\stype}{(0,0,0)}$ tight.
      \item \label{lem:comp-typ-nfs} Let  $t \in \normal$. Then for any tight $\stype$ there exists $\Phi \tr \seqi{\Gam}{t}{\tcomptype{\stype}{\tightt}{\stype}}{(\cmiguel{b,m}{0,0},d)}$ tight s.t. $d = \size{t}$.
  \end{enumerate}
\end{restatable}

Finally, we state the usual basic properties.

\begin{restatable}[{\bf Substitution} and {\bf Anti-Substitution}]{lemma}{lemcompsubsantisubs} \mbox{}
    \label{lem:comp-subs-antisubs}    
    \begin{enumerate}
        \item {\bf (Substitution)} \label{lem:comp-subs} If $\Phi_t \tr \seqi{\Gam_t; x : \M}{t}{\del}{(b_t,m_t,d_t)}$ and $\Phi_v \tr \seqi{\Gam_v}{v}{\M}{(b_v,m_v,d_v)}$, then $\Phi_{t \subs{x}{v}} \tr \seqi{\Gam_t + \Gam_v}{t \subs{x}{v}}{\del}{(b_t+b_v,m_t+m_v,d_t+d_v)}$.
        \item {\bf (Anti-Substitution)} \label{lem:comp-antisubs} If $\Phi_{t \subs{x}{v}} \tr \seqi{\Gam_{t \subs{x}{v}}}{t \subs{x}{v}}{\del}{(b,m,d)}$, then $\Phi_t \tr \seqi{\Gam_t; x : \M}{t}{\del}{(b_t,m_t,d_t)}$ and $\Phi_v \tr \seqi{\Gam_v}{v}{\M}{(b_v,m_v,d_v)}$, such that $\Gam_{t \subs{x}{v}} = \Gam_t + \Gam_v$, $b = b_t+b_v$, $m = m_t+m_v$, and $d = d_t + d_v$.
    \end{enumerate}
\end{restatable}

\begin{restatable}[{\bf Split Exact Subject Reduction} and {\bf Expansion}]{lemma}{lemexactredexp}
  \label{lem-exact-red-exp} \mbox{}
  \begin{enumerate}
    \item {\bf (Subject Reduction)} \label{lem:subj-comp-red} Let $(t,s) \red[\gname] (u,q)$. If $\Phi \tr \seqi{\Gam}{(t,s)}{\ctype}{(b,m,d)}$ is tight, then $\Phi' \tr \seqi{\Gam}{(u,q)}{\ctype}{(b',m',d)}$, where $\gname =\beta$ implies $b' = b - 1$ and $m' = m$, while $\gname \in \{\getname, \setname\}$ implies $b'=b$ and  $m' = m - 1$.
    \item {\bf (Subject Expansion)} \label{lem:comp-subj-exp} Let $(t,s) \red[\gname] (u,q)$. If $\Phi' \tr \seqi{\Gam}{(u,q)}{\ctype}{(b',m',d)}$ is tight, then $\Phi \tr \seqi{\Gam}{(t,s)}{\ctype}{(b,m,d)}$, where $\gname =\beta$ implies $b' = b - 1$ and $m' = m$, while $\gname \in \{\getname, \setname\}$ implies $b'=b$ and  $m' = m - 1$.
  \end{enumerate}
\end{restatable}

Soundness (resp. completeness) is based on exact subject reduction (resp. expansion) respectively, in turn based on the previous substitution (resp. anti-substitution) lemma.

\begin{restatable}[{\bf Quantitative Soundness} and {\bf Completeness}]{theorem}{compsoundness} \mbox{}
    \begin{enumerate}
        \item {\bf (Soundness)} If $\Phi \tr \seqi{\Gam}{(t,s)}{\kap}{(b,m,d)}$ tight, then there exists $(u,q)$ such that $u \in \normal$ and $(t,s) \gsrred^{(b,m)} (u,q)$ with $b$ $\beta$-steps, $m$ $\getname/\setname$-steps, and $\size{(u,q)} = d$.
        \item {\bf (Completeness)} If $(t,s) \rra^{(b,m,d)} (u,q)$ and $u \in \normal$, then there exists $\Phi \tr \seqi{\Gam}{(t,s)}{\ctype}{(b,m,\size{(u,q)})}$ tight.
    \end{enumerate}
\end{restatable}

\begin{example}
  Consider again configuration $c_0$ from~\cref{ex:globalstate} and its associated tight derivation $\Phi_{c_0}$. The first two counters of $\Phi_c$ are different from $0$: this means that $c$ is not a final configuration, but normalizes in one $\betav$-step ($b = 1$) and two $\getname/\setname$-steps ($m = 2$), to a final configuration having size $d = 0 = \size{z} = \size{(z, \upd{l}{I}{\estate})}$.
\end{example}
