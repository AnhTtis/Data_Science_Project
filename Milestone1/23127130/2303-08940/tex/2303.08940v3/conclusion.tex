\section{Conclusion and Related Work}
\label{s:conclusion}

This paper provides a foundational step into the development of quantitative models for programming languages with effects. We focus on a simple language with global memory access capabilities. Due to the inherent lack of confluence in such framework we fix a particular evaluation strategy following a (weak) CBV approach. We provide a type system for our language that is able to (both) extract and discriminate between (exact) measures for the length of evaluation, number of memory accesses and size of normal forms. This study provides a valuable insight into time and space analysis of languages with global memory.

In future work we would like to explore effectful computations involving global memory in a more general framework being able to capture different models of computation, such as the CBPV~\cite{Levy99} or the bang calculus~\cite{BucciarelliKRV20}. Furthermore, we would like to apply our  quantitative techniques to other  effects that can be found in programming languages, such as non-termination, exceptions, non-determinism, I/O.

{\bf Related Work.}  Several papers proposed quantitative approaches for different notions of CBV (without effects). But none of them exploits the idea of exact \emph{and} split tight typing. Indeed, the first non-idempotent intersection type system for Plotkin's CBV is~\cite{Ehrhard12}, where reduction is allowed under abstractions, and terms are considered to be closed.  This work was further extended to~\cite{CarraroG14}, where commutation rules are added to the calculus. None of these contributions extracts quantitative bounds from the type derivations. A calculus for open CBV is proposed in~\cite{AccattoliG18}, where \emph{fireball} --normal forms-- can be either erased or duplicated. Quantitative results are obtained, but no split measures. Other similar approaches appear in~\cite{Guerrieri19}. A logical characterization of CBV solvability is given in~\cite{AccattoliG22}, the resulting non-impotent system gives quantitative information of the \emph{solvable} associated reduction relation. A similar notion of solvability for CBV for generalized applications was studied in~\cite{KesnerP22}, together with a logical characterization provided by a quantitative system.
Other non-idempotent systems for CBV were proposed~\cite{Manzonetto2019,Kerinecetal21}, but they are defective in the sense that they do not enjoy subject reduction and expansion. Split measures for (strong) open CBV are developed in~\cite{KesnerV22}.

In~\cite{Dezani-CiancagliniGR09}, a system with universally quantified intersection and reference types is introduced for a language belonging to the ML-family. However, intersections are idempotent and only (qualitative) soundness is proved.

Concerning  (exact) quantitative models for programming languages with global state the state of the art is still underexplored. Some sound but not complete approaches are given in~\cite{Benton2009,Davies2000}, and quantitative results are not provided. Our work is inspired by a recent idempotent (thus only qualitative and not quantitative) model for CBV with global memory proposed by~\cite{deLiguoroT21}. This work was further extended in~\cite{GTV23} to a more generic framework of algebraic effectful computation, still the model does not provide any quantitative information about the evaluation of programs and the size of their results.

