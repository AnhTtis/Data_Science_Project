\begin{proof}\
  \begin{itemize}
    \item[$\Ra$)] By point (1) of~\cref{lem:zero-counters}.
    \item[$\La$)] By induction on $t$: \begin{itemize}
    \item Case $t \in \val$. There are six cases to consider for $\Phi$:
    \begin{itemize}    
      \item $\Phi$ ends with (\ruleAx). This case does not apply since the resulting type is not a monadic type. %Then $\Phi \tr \seqi{x:\mul{\rdel}}{x}{\rdel}{(0,0,0)}$ and the conclusion holds trivially.
      \item $\Phi$ ends with (\ruleLam). This case does not apply since the resulting type is not a monadic type.
      \item $\Phi$ ends with (\ruleMany). This case does not apply since the resulting type is not a monadic type.
      \item $\Phi$ ends with (\ruleLift). This case does not apply, since $\del = \tcomptype{\stype}{\M}{\stype'}$, but $\M \not\in \tightt$.
      \item $\Phi$ ends with (\ruleAxP). Then $\Phi \tr \seqi{x:\mul{\nott{\tneutral}}}{x}{\tcomptype{\stype}{\nott{\tneutral}}{\stype}}{(0,0,0)}$, with $\stype$ tight, and the conclusion holds trivially.
      \item $\Phi$ ends with (\ruleLamP). Then $\Phi \tr \seqi{}{\lambda x.t}{\tcomptype{\stype}{\vl}{\stype}}{(0,0,0)}$, with $\stype$ tight, and the conclusion holds trivially. 
    \end{itemize}
    \item Case $t = xu$. Then $u \in \normal$, by definition and there are two cases to consider for $\Phi$:
    \begin{itemize}
      \item If $\Phi$ ends with (\ruleApp). Then $\Phi_u \tr \seqi{\Gam_u}{u}{\tcomptype{\stype}{\M}{\stype'}}{(b_u,m_u,d_u)}$, $\Phi_x \tr \seqi{x : \M \ta (\comptype{\stype'}{\ctype})}{x}{\M \ta (\comptype{\stype'}{\ctype})}{(b_x,m_x,d_x)}$, such that $\Gam =  (x:\mul{\M \ta (\comptype{\stype'}{\ctype})}) + \Gam_u$ is tight. Absurd, since $\M \ta (\comptype{\stype'}{\ctype})$ is not tight, therefore this case does not apply.
      \item If $\Phi$ ends with (\ruleAppPOne). Then $\Phi_u \tr \seqi{\Gam_u}{u}{\tcomptype{\stype}{\tightt}{\stype}}{(b_u,m_u,d_u)}$, such that $\Gam = (x: \mul{\tvar})+\Gam_u$ is tight, $b = b_u$, $m =m_u$, $d = d_u+ 1$, and $\stype$ is tight. By the \ih\ on $u$, we have $b_u=m_u=0$, therefore $b = m = 0$.
      \end{itemize}
      \item Case $t = (\lam x.p) u$. Then $u \in \neutral$, by definition and there are two cases to consider for $\Phi$:
      \begin{itemize}
        \item If $\Phi$ ends with (\ruleApp). Then $\Phi_u \tr \seqi{\Gam_u}{u}{\tcomptype{\stype}{\M}{\stype'}}{(b_u,m_u,d_u)}$, $\Phi_{\lam x.p} \tr \seqi{\Gam_{\lam x.p}}{\lam x.p}{\M \ta (\comptype{\stype'}{\ctype})}{(b_p,m_p,d_p)}$, such that $\Gam = \Gam_u + \Gam_{\lambda x.p}$ is tight, $b = 1+b_l+b_u$, $m = m_l+m_u$, $d = d_l+ d_m$. Since $\Gam_u$ is tight and $u\in\neutral$, by~\cref{lem:comp-tight-spreading}, $\M \in \tightt$, which is absurd. Therefore, this case does not apply.
        \item If $\Phi$ ends with (\ruleAppPTwo). Then $\Phi_u \tr \seqi{\Gam_u}{u}{\tcomptype{\stype}{\tneutral}{\stype}}{(b_u,m_u,d_u)}$, such that $\Gam = \Gam_u$ is tight, $b = b_u$, $m=m_u$, $d = d_u+ 1$ and $\stype_f$ is tight. By the \ih\ on $u$, we have $b_u=m_u=0$. Therefore $b = m = 0$.
      \end{itemize}
    \end{itemize}
  \end{itemize}
\end{proof}
