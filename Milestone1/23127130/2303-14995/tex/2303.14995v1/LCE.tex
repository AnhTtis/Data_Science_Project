\documentclass[10pt,english]{article}
\usepackage[letterpaper,bindingoffset=0.2in,% 
            left=0.75in,right=0.75in,top=1in,bottom=1in,%
            footskip=.25in]{geometry}
%\userpackage{natbib}
%{\parfillskip=0pt\par} 
%\userpackage{url}
\usepackage{cite}%sort&compress citations
\usepackage{notoccite}%forces citation sequential order, didn't work.
\makeatletter%removes brackets from around ref list
\renewcommand\@biblabel[1]{#1.}
\makeatother
%\newcommand\myeq{\mathrel{\overset{\makebox[0pt]{\mbox{\normalfont\tiny\sffamily def}}}{=}}}%amsmath def= command
%\usepackage{orcidlink}
\usepackage{amsthm}
\newlength{\bibitemsep}\setlength{\bibitemsep}{0\baselineskip plus .0\baselineskip minus 0\baselineskip}%changes spacing between references
%\newlength{\bibitemsep}\setlength{\bibitemsep}{.2\baselineskip plus .05\baselineskip minus .05\baselineskip}%changes spacing between references
\newlength{\bibparskip}\setlength{\bibparskip}{0pt}
\let\oldthebibliography\thebibliography
\renewcommand\thebibliography[1]{%
  \oldthebibliography{#1}%
  \setlength{\parskip}{\bibitemsep}%
  \setlength{\itemsep}{\bibparskip}%
}
%\userpackage[]{apacite}
%\userpackage{spbasic}
%\documentclass[smallextended]{svjour3} 
%\usepackage[natbibapa]{apacite}
%\userpackage{vancouver}
%\setcitestyle{numbers,comma,square} 
%\citeindextrue
\usepackage{epsfig, xcolor, setspace, graphicx,pict2e}
%\usepackage[outdir=./]{epstopdf}
%\usepackage{xfrac}

%orcid icon
\newsavebox{\ORCIDlogo}
\savebox{\ORCIDlogo}{%
\setlength{\unitlength}{\dimexpr 1em/256\relax}%
\begin{picture}(256,256)%
  \color[HTML]{A6CE39}\put(128,128){\circle*{256}}%
  \color{white}%
  \put(78.6,199.2){\circle*{20}}%
  \moveto(70.9,176,9)\lineto(86.3,176,9)\lineto(86.3,69.8)\lineto(70.9,69.8)%
  \closepath\fillpath%
  \moveto(108.9,176.9)\lineto(150.5,176.9)%
  \curveto(190.1,176.9)(207.5,148.6)(207.5 ,123.3)%
  \curveto(207.5,95,8)(186,69.7)(150.7,69.7)%
  \lineto(108.9,69.7)%
  \closepath\fillpath%
  \color[HTML]{A6CE39}%
  \moveto(124.3,83.6)\lineto(148.8,83.6)%
  \curveto(183.7,83.6)(191.7,110.1)(191.7,123.3)%
  \curveto(191.7,144.8)(178,163)(148,163)%
  \lineto(124.3,163)%
  \closepath\fillpath%
\end{picture}%
}

\newcommand\orcidicon[1]{\href{https://orcid.org/#1}{\usebox{\ORCIDlogo}}}
%orcid icon

\newenvironment{myfont}{\fontfamily{cm}\selectfont}{\par}%define font for mathematica code !!!

\setcounter{tocdepth}{6}
\setcounter{secnumdepth}{6}
\usepackage[leftcaption]{sidecap}
\usepackage{lscape,amsmath,nccmath,nicefrac}
\newcommand\myeq{\mathrel{\overset{\makebox[0pt]{\mbox{\normalfont\tiny\sffamily def}}}{=}}}%amsmath def= command
\usepackage{relsize}
\usepackage{setspace}%line spacing 
\renewcommand{\baselinestretch}{1.5} 
\usepackage[T1]{fontenc}
\usepackage{footmisc}%reference footnotes must appear before hyperref package
\usepackage{hyperref} %includes \nameref{} command
%\usepackage[hyperfootnotes=false,...]{hyperref}
\hypersetup{colorlinks,breaklinks,
	unicode = true,
            urlcolor=[rgb]{0,0,.5},
            linkcolor=[rgb]{1,0,0},
            citecolor=[rgb]{0,0,1}}%\hypersetup{
%    colorlinks = true,
   % linkbordercolor = {white}
    %<your other options...>,
%}
\usepackage[all]{hypcap} %for going to the top of an image
\usepackage{caption}
\usepackage{makecell,multirow}
\usepackage{hhline}% http://ctan.org/pkg/hhline
\usepackage[normalem]{ulem} % simple strickout package
\usepackage{float}% figure float position
%\usepackage{floatrow}
\usepackage{centernot}% negation of other symbols like \implies
\usepackage{etoolbox}
\usepackage{amssymb}%therefore symbol
\usepackage{tabularx}%table width
\usepackage{wrapfig,booktabs}%text wrapping

%\usepackage{lineno}%line numbers
%\linenumbers%!!!

\sidecaptionvpos{figure}{t}%for sidecap


%lineno_general

\apptocmd{\lim}{\limits}{}{}%Always put \to under \lim
\newcolumntype{C}[1]{>{\centering\arraybackslash}p{#1}}
\renewcommand{\vec}[1]{\mathbf{#1}}
\newcommand{\cmmnt}[1]{\ignorespaces}
\newcolumntype{Y}{>{\centering\arraybackslash}X} %converts tabularx X to centred Y
\usepackage[euler]{textgreek}
\usepackage{cleveref}
\renewcommand\bfdefault{b}% rather than bx

%\usepackage{filecontents} obsolete command

%\usepackage{tikz}
%\usetikzlibrary{shapes.geometric, arrows, shapes, shapes.multipart}

\title{On renal insufficiency measurement and reference standards using the logarithm of a cumulative exponential and multiple other plasma and renal clearance models\vspace{1.5cm}}

\author{{\bf Carl A. Wesolowski\,\orcidicon{0000-0003-0134-9346}$^{\dag\ddag\footnote{Corresponding Author: telephone :(306) 665 1515, e-mail: carl.wesolowski@gmail.com}}$} \vspace{0.5cm}\\%, Jane Alcorn$^{\dag}$, and Geoffrey T. Tucker$^{\S}$ 
       $^{\dag}$  {\normalsize College of Pharmacy and Nutrition}\\
       {\normalsize University of Saskatchewan, 104 Clinic Place, Saskatoon, SK, S7N 2Z4, Canada}\\
       $^{\ddag}$ {\normalsize Department of Medical Imaging, Royal University Hospital, College of Medicine}\\
       {\normalsize University of Saskatchewan, 103 Hospital Drive, Saskatoon, SK, S7N 0W8, Canada}}%\\
       %$^{\S}$ {\normalsize Department of Human Metabolism, Medical and Biological Sciences, University of Sheffield, Sheffield, UK}}

\date{}
\begin{document}  

%\tikzstyle{beginend} = [rounded rectangle, minimum width=2.0cm, minimum height=1cm,text centered, draw=black ]%, fill=red!30
%\tikzstyle{decision} = [diamond, aspect=2, minimum width=2.5cm, minimum height=.8cm, text centered, text height=0.25cm, draw=black ]%, fill=green!30
%\tikzstyle{io} = [trapezium, trapezium left angle=70, trapezium right angle=110, minimum width=2.5cm, minimum height=0.8cm, text centered, text width=1.5cm, draw=black ]%, fill=blue!30
%\tikzstyle{process} = [rectangle, minimum width=2.cm, minimum height=1cm, text centered, text width=2.0cm, text height=0.25cm, draw=black ]%, fill=orange!30
%\tikzstyle{arrow} = [thick,->,>=stealth ]
%\tikzstyle{pp} = [rectangle split, rectangle split horizontal, rectangle split parts=3, minimum height=1cm, text centered, text width=2cm, text height=0.4cm ]%fill=orange!30, 


\begin{onehalfspacing}
\maketitle \vspace{0in} \noindent

\begin{abstract}
\noindent For current models and methods, glomerular filtration rates below 20 ml/min in adults resulted in modelling concentration tails that were frequently unseen on linear-log plotting. The resulting sometimes unobservable tail was predicted using the negative logarithm of a cumulative exponential (LCE), from the latter of its two asymptotes; a logarithm for decreasing time and an exponential tail as time increases. Lambert's Omega is the scaled time at which the two asymptotes are equal. The LCE formula uses two plasma samples, minimum, and fit 13 24 h $^{51}$Cr-EDTA studies with an 8\% standard deviation of residuals compared to 20\% error for monoexponentials. The LCE model was unbiased for prediction of 43 5 h urinary $^{51}$Cr-EDTA activity cases whereas the mono- and bi-exponential, as well as, adaptively regularised gamma variate models were relatively overestimating. Reference standard corrections were explored. The LCE model detected two otherwise unidentified absent renal function cases (GFR < 0.01 ml/min) in a 41 case $^{169}$Yb-DTPA dataset suggesting its use for detecting anephric conditions. Prospective clinical testing, and metabolic scaling of renal insufficiency is advised for potential changes to patient triage, e.g., for conservative management, dialysis, and kidney or liver transplantation. 

\vspace{2em}

\noindent $\mathbf{Keywords}$: Measured glomerular filtration rate, Logarithm of cumulative exponential, Severe renal insufficiency, Transplantation, Statistical distributions
\end{abstract}

\section*{Introduction}
Glomerular filtration rate, GFR, can be defined as the effective volume of arterial blood plasma per unit time totally cleared of nonindigenous, entirely-solvated, low-enough molecular-weight inert markers to be freely eliminated by renal filtration alone. GFR is widely considered to be the most useful measure of renal function \cite{Stevens2006}. This usefulness is likely due to a homeostatic balance between normal glomerular elimination of the products of metabolism and metabolic rate itself, such that reduced GFR signifies increased plasma concentration of a host of metabolites \cite{wesolowski2006improved}. This work presents and tests a new bolus intravenous GFR plasma model for use with venous sampling of radiochelates and other nonindigenous GFR markers. Most bolus intravenous injection pharmacokinetic models are venous plasma concentration sampling models of two principle types. The simplest type is the washout model; monotonically decreasing functions of time that have maximum concentration initially, at $t=0$. Models of the second type allow for the increasing concentration from an initial zero concentration in a peripheral sampling site, i.e., $C(0)=0$, and have more parameters than washout models. This work reports a potentially more useful type of two-parameter washout model than the exponential type of model in current use, and a comparison of the results of multiple model types from three different series and two different radiopharmaceuticals.

In 1960, Schloerb \cite{schloerb1960total} 
published the results of intravenous infusion of tritiated water, urea, and creatinine in nephrectomised dogs. Schloerb noted that plasma concentration of creatinine decreased with elapsing time and appeared to come to equilibrium after 4 hours, but then noted that this was only an apparent equilibrium as the expected complete equilibrium with total body water had not been achieved even at 24 h. He concluded that a near infinite number of compartments would need to be invoked to explain his results. That is, if we were to fit a monoexponential (E1) to Schloerb's disappearance curves, we would obtain a finite AUC, where AUC would have to be infinite to be consistent with the actual renal clearance of zero in a nephrectomised animal. Thus, monoexponentials and their sums fit to concentration curves from an infusion with data acquired for a short time exaggerate clearance. Moreover, most current models of plasma and renal clearance, be they from bolus intravenous injections, constant infusion, or subcutaneous injections do not reliably quantify severe renal insufficiency defined here as less than or equal to 20 ml/min for an adult. We refer to this problem as the Schloerb challenge, that is, to find a plasma disappearance curve model having a limiting infinite AUC with zero plasma clearance as renal clearance goes to zero.

Typical clinical measurements using monoexponential (E1) models collect two or more time-samples between 2 and 4 hours.  However, in severe renal insufficiency and/or fluid overload (ascites, tumour) the first time-sample should be collected at two or five h and the last at 24 h \cite{BroechnerMortensen1981,BroechnerMortensen1985,wickham2013development}, and even then the E1 results from 2 h to 24 h sample-times required correction for AUC underestimation \cite{wickham2013development}. One way to address the Schloerb challenge is to ignore plasma concentration models and instead measure GFR markers in urine. As Schloerb predicted, comparative measurements of E1 models of plasma clearance with renal (urine) clearance have shown that exponential plasma models predict substantial clearance values, when renal clearance was zero, i.e., causing an irreducible intercept error, e.g., 11.3 ml$\cdot$min$^{-1}$ \cite{LaFrance1988}. Current correction methods do not address the overestimation of zero renal clearance by plasma E1 models. That is, both the Chantler-Barratt and  Br{\o}chner-Mortensen corrections of E1 clearance ($\text{CL}_{\text{E1}}$) lack the appropriate nonlinearity at zero renal clearance to correct for linear constant bias (respectively, $\text{CL}\approx 0.87\, \text{CL}_{\text{E1}}$ and $\text{CL}\approx 0.990778\, \text{CL}_{\text{E1}}-0.001218\,\text{CL}_{\text{E1}}^2$)\cite{Chantler1972,BroechnerMortensen1972,Murray2013}.  A secondary adjustment to an estimated body surface area (BSA) of 1.73 m$^2$ is often preformed. Although one can argue that creatinine plasma level scales approximately as BSA (circa weight to the 2/3 power), GFR certainly does not (circa weight to the 3/4 power)  \cite{adolph1949,wesolowski2006improved}. This, and other factors, have led to a divergence between pharmacokinetics and nephrology with nephrology current guidelines suggesting multiple timed voluntary urine collections for a noisy \textit{underestimating} approximate body surface area normalised renal clearance reference standard from  subcutaneous injections of $^{125}$I-iothalimate, a marker with circa 18\% non-renal clearance \cite{Prueksaritanont1986}, see \nameref{Uprob} in the Methods section. That standard is currently recommended for calibrating a heuristic endogenous plasma creatinine GFR index \cite{delgado2022unifying}. Creatinine, in turn, is a mixed GFR and tubular extraction marker, and overestimates renal filtration in a variety of clinical conditions most notoriously in liver failure and renal insufficiency \cite{Wesolowski1992}. Another difficulty occurs in acute renal failure, which can be defined clinically by: creatinine levels (however, creatinine levels take days to build up); by loss of GFR, (presumably as GFR-indices from creatinine levels); or by 12 h of anuria or 24 h of severe oliguria of < 0.3 ml$\cdot$h$^{-1}$ per kg body weight \cite{Hilton2011}. In anuria, or severe oliguria, urine collection volumes are nonexistent or inaccurate.   On the other hand, pharmacokinetics is concerned with drug effects most often correlated to venous plasma drug concentrations (GFR is arterial), utilise plasma (not renal) models that are tailored for route of administration, and might body scale per kilogram body mass for veterinary work, or occasionally BSA body scale for dose calculations, and would not likely claim that an 18\% non-renal cleared marker is a GFR marker. Thus, it is important to answer the Schloerb challenge as neither nephrologist nor pharmacokineticist has accurate methodology to offer the renal insufficient patient.

    Our first attempt to answer the Schloerb challenge produced the more accurate measurement of GFR obtained using the Tikhonov adaptively regularised gamma variate fitting (Tk-GV) method, which smooths the data to obtain that flattened curve that best reduces the relative error of propagation of the rate parameter of a gamma variate \cite{wesolowski2011validation,wickham2013development,wanasundara2016}. Because of this curve flattening, which becomes severe for renal failure, the Tk-GV algorithm is not a curve fit method in the ordinary sense. Compared to Tk-GV GFR-values, E1 and biexponential (E2) GFR values are larger, especially in severe renal insufficiency, because exponential methods overall underestimate both early and late concentrations \cite{wesolowski2010tikhonov,wesolowski2011validation,wickham2013development}. The use of the \text{Tk-GV} algorithm for measuring GFR was unique enough that patents were granted in the USA and multiple other jurisdictions \cite{wesolowski2014method}. 

For bolus intravenous injections, mixing takes a long time, thus concentration does not decrease in proportion to the logarithm of concentration. Indeed, in a prior publication, concentration before 2 to 4 h following a bolus injection of a GFR marker more accurately back-extrapolated as the logarithm of time, than as an underestimating exponential, or an overestimating power function \cite{wanasundara2015early}. The intent here was to characterise, test and present a modified logarithm of time function as a better two parameter paradigm than that offered by E1 or other published models with special attention to Schloerb's implied challenge of identifying the renal failure patient.

\section*{The linear-logarithm hypothesis}

For a very long time it has been supposed that as a first approximation, the concentration of an intravenously injected GFR marker is proportional to the logarithm of concentration. That supposition implies instant mixing, and is incorrect. In 2015, it was noted that during the first few hours following intravenous injections of a GFR marker, concentration decreased \textit{less} exponentially, i.e., \textit{less} linearly with the logarithm of \textit{concentration}, and decreased \textit{more} linearly with the logarithm of \textit{time} \cite{wanasundara2015early}. It would be less onerous to assume that early concentration is logarithmic and becomes exponential when a terminal volume of drug distribution is approached. Of the many such formulas a potentially useful one assigns concentration as proportional to $-\ln(1-e^{-\beta\,t})$, called the logarithm of cumulative exponential function (LCE, as below), which was our working hypothesis for the severe renal insufficiency measurement problem. The LCE \textit{density} functions are developed by solving for the constant of proportionality, with their properties compared to exponential densities in Table \ref{dists}. 

Concentration models having a finite area under the curve (AUC) from $t=0$ to $t=\infty$ and can be written as 


\begin{equation}
\label{eq1}C(t)=\text{AUC }f(t)\;\;,
\end{equation}

\noindent where $f(t)$ is a density function with a total area of 1; $\int_0^{\infty } f(t)\,dt \myeq1$. To be clear, AUC is from curve fitting but is the area under the entire curve, not just the data from the first to last time-samples. For example, for E1, let,

$$f(t)=\lambda\, e^{-\lambda\,t},\;\;\;C(t)=\text{AUC}\,\lambda \,e^{-\lambda\,t},$$



\begin{table}[H]
\centering
%\small
\captionsetup{justification=justified,margin=0cm}
 \caption {Comparison of the monoexponential (E1) and Logarithm of Cumulative Exponential (LCE) distributions.}
 \vspace*{-.5em}
\label{dists} 
\begin{tabularx}{\textwidth}{@{\hspace{0cm}}l@{\hspace{.4cm}}l@{\hspace{.5cm}}ll}
%\hline
 \Xhline{2\arrayrulewidth} 

\vspace{0em}

&E1 Distribution&LCE Distribution&Notes\\\midrule
 \vspace{0em}
 
Type&Washout&Washout&Monotonic decreasing\\
\vspace{0em}

Parameters&$ \lambda>0$, rate &$\beta>0$, rate & Rate is 1/scale\\
 \vspace{.4em}

Support&$t\in[0,\infty)$&$t\in[0,\infty)$&Semi-infinite support\\
 \vspace{0em}
 
Density function, $f(t)$&$\lambda\,e^{-\lambda\,t}$&$-\mfrac{6\, \beta }{\pi ^2}\ln \left(1-e^{-\beta\, t}\right)$&Probability $f(t)$ only: PDF\\
 \vspace{0em}

CDF, $F(t)$&$1-e^{-\lambda\,t}$&$1-\mfrac{6 }{\pi ^2}\text{Li}_2\left(e^{-\beta\, t}\right)$&Li$_n(z);$  polylogarithm function $^\text{a}$\\
 \vspace{0em}

CCDF$=1-$CDF, $S(t)$&$e^{-\lambda\,t}$&$\mfrac{6 }{\pi ^2}\text{Li}_2\left(e^{-\beta\, t}\right)$&Complementary CDF $^\text{b}$\\
 \vspace{.2em}

$t_{m}:F(t_m)=\mfrac{1}{2}$&$\mfrac{\ln(2)}{\lambda}\approx \mfrac{0.693147}{\lambda}$&$t_{m}\approx \mfrac{0.415389}{\beta}$&Median residence time\\
 \vspace{.2em}

$t_{x}:\,e^{-\beta \,t}=-\ln (\beta \,t)$&$\cdots$&$\mfrac{\Omega}{\beta}\approx\mfrac{0.567143}{\beta}$&$\Omega$ is Lambert's $W(1)$\\
\vspace{.2em}

MRT $=\int_0^\infty t\,f(t)\,dt$&$\mfrac{1}{\lambda}$&$\mfrac{6\, \zeta (3)}{\pi ^2\, \beta}\approx\mfrac{0.730763}{\beta }$&$\zeta (n)$ is the zeta function\\
 \vspace{.1em}

 V$_\text{MRT}=\text{CL MRT}$&$\mfrac{\text{CL}}{\lambda}=\mfrac{\text{Dose}}{\text{AUC}}\mfrac{1}{\lambda}$&$\mfrac{\text{CL}}{\beta}\mfrac{6\, \zeta (3)}{\pi ^2}$&\textit{Pharm.}: V$_{\text{SS}}$; Vol. steady state\\
 \vspace{.2em}
 
 $V_\text{d}(t)$&$\mfrac{\text{CL}}{\lambda}$&$0\leq-\mfrac{\text{CL}}{\beta}\mfrac{\text{Li}_2\left(e^{-\beta \,t}\right)}{\ln \left(1-e^{-\beta \,t}\right)}\leq\mfrac{\text{CL}}{\beta}$&LCE: $V_\text{d}(0)\leq V_\text{d}(t)\leq V_\text{d}(\infty)$\\
  \vspace{.1em}

 $M_{\text{urine}}(t)$&$M_0\left(1-e^{-\lambda \,t}\right)$&\makecell{$M_0\mfrac{3}{\pi ^2} \Big[\ln ^2(e^{\beta\, t})-2\, \text{Li}_2(1-e^{\beta\, t})$\\$\hspace*{1.5cm}-\ln (e^{2 \beta\, t}) \ln (e^{\beta\, t}-1)\Big]$}&Dose ($M_0$) in urine at time $t$\\

 \hline
\end{tabularx}
 \begin{tabularx}{1\textwidth}{X} 
 
 $^\text{a }$CDF is the cumulative density function, i.e., the integral from 0 to $t$ of the density function.\\
 $^\text{b }$CCDF, the complementary cumulative density function, is sometimes called a survival function, $S(t)$.
\end{tabularx}
\end{table}

\noindent where setting $c=\text{AUC}\,\lambda$ yields the more common notation $C(t)=c \,e^{-\lambda\,t}$. In addition to extracting AUC-values immediately from data fitting, identifying the density function makes the rules for its manipulation immediately available. One such rule is the cumulative density function, CDF, also written as $F(t)$, where $F(t) \myeq \int_0^t  f(\tau)\,d\tau$, i.e., the 0 to $t$ integral of $f(t)$, such that $\lim_{t\to\infty}F(t)=1$. The CDF of an exponential density, $\lambda\,e^{-\lambda\,t}$, is thus 

$$F(t)= \int_0^t  \lambda\,e^{-\lambda\,t}\,dx=1-e^{-\lambda\,t}\;\;.$$ 

\noindent This equation was used to name the new model named as follows. As the inside of $-\ln(1-e^{-\beta\,t})$, i.e., $1-e^{-\beta\,t}$, is a cumulative exponential, $-\ln(1-e^{-\beta\,t})$ is a negative Logarithm of a Cumulative Exponential, or LCE as an acronym. To make the LCE into a density function, $-\ln(1-e^{-\beta\,t})$ is multiplied by a constant that makes its total area equal to one, $\lim_{t\to\infty}F(t)=1$. That is,

\begin{equation}\label{eq2}
f(t)=-\frac{6\, \beta }{\pi ^2}\ln \left(1-e^{-\beta\, t}\right)\;\;,
\end{equation}

\noindent where that constant is $\mfrac{6\, \beta }{\pi ^2}$. The proof of this and exposition of the other formulas of Table \ref{dists} are listed in the \nameref{sec:appendix} section. Combining Eqs.~\eqref{eq1} and \eqref{eq2} yields the fit equation for the LCE model used in this manuscript,

\begin{equation}\label{eq3}
C(t)=\text{AUC}\cdot f(t)=-\text{AUC}\,\frac{6\, \beta }{\pi ^2}\ln \left(1-e^{-\beta\, t}\right)\;\;.
\end{equation}

\noindent One can rewrite this model in pharmacokinetic form using a constant of proportionality, $c_p=\text{AUC}\frac{6\, \beta }{\pi ^2}$, 

\begin{equation}\label{eq4}
C(t)=-c_p\ln \left(1-e^{-\beta\, t}\right);\;\;\;
\text{AUC}=c_p\frac{\pi ^2}{6\, \beta }\;\;\;.
\end{equation}

\noindent As shown in Figure \ref{fig_1}, Eq.~\eqref{eq4} has two asymptotes; the first a logarithm as $t\to0$ and the second an exponential as $t\to\infty$. Then, the time when these are equal is obtained by solving for when their magnitude is the same, in other words we seek $\beta\, t$ such that,

$$-c_p\ln (\beta \,t)\equiv c_p\,e^{-\beta \,t}\;\;.$$  
Let $u=\beta\, t$, then as $c_p$ cancels, this equation becomes $-\ln (u)=e^{-u}$, whose solution is $u=\Omega$, where $\Omega$, is Lambert's Omega or $W(1)\approx0.567143$. Also called the product logarithm function, Lambert's $W(z)$, satisfies $w\, e^w=z$. In our case, $\Omega e^{\Omega}=1$, and we can write the intersection time for the asymptotes, $t_x$, as, 

$$t_{x}=\Omega\,\beta^{-1}\approx0.567143\,\beta^{-1}\;\;,$$


\noindent where $t_{x}$ is a time before which the LCE is predominantly a logarithmic function, and after which the LCE is relatively more exponential. Note (Table \ref{dists} and the \nameref{sec:appendix} section) that for the LCE model $t_m<t_{x}<\text{MRT}$. That is, the \textit{median} residence time ($t_{m}\approx 0.415389\,\beta^{-1}$) occurs when the LCE density is predominantly a logarithmic function of time, whereas its mean residence time (MRT $\approx 0.730763\,\beta^{-1}$), occurs when the LCE is more exponential. 



\begin{SCfigure}[][ht]\centering\includegraphics[scale=0.7]{fig_1.png}
\caption {Shown is an example LCE model, $C(t)=-c_p\ln(1-e^{-\beta \,t})$ (black), with asymptotes plotted as linear concentration versus time scaled logarithmically. The LCE asymptotes are $-c_p\ln(\beta\,t)$ (dot-dashed grey) and $c_p\,e^{-\beta\,t}$ (dashed grey), which intersect at $t_{x}=\frac{\Omega}{\beta}$. Displayed are example curves using the parameters AUC $=27.34$ (unit dose min per litre), and $\beta=0.0005027$ (per min). From $\beta$ then $t_x=1128.2$ min.}
\vspace{-1em}
\label{fig_1}
\end{SCfigure}


For LCE models and sums of exponential term (SET) models the constants of proportionality are equal to the models' concentrations at different times. For SETs the total concentration $C(0)=c_1+c_2+c_3+\cdots+c_n$ at $t=0$. For LCE the time when its model concentration equals $c_p$ occurs at $t:\ln(e-1)\,\beta^{-1}\approx 0.541325\,\beta^{-1}$. As per Table \ref{dists} and Figure \ref{fig_2}, the LCE initial volume of distribution is zero; $V_d(0)=0$, which is unlike the SET value, $V_c>0$, that is, the central (i.e., initial, non-zero) volume of distribution. The hypothetical volume of drug distribution at which concentration curve shape becomes more exponential is 81\% $V_z$ occurring at time $t_x = \Omega\,\beta^{-1}$ and is a substantial portion of $V_z$, the terminal volume. This is from the LCE volume equation, $V_d(t)$, as follows,

\begin{equation}
V_\Omega=-\mfrac{\text{Li}_2\left(e^{-\Omega}\right)}{\ln \left(1-e^{-\Omega}\right)}\;V_z\approx 0.81000437\;V_z\;\;,
\end{equation}
where $V_\Omega$ is $V_d(t_x)$ and almost exactly 81\% of LCEs $V_z$.  For SETs, $V_c>0$, and $V_c$ is an initially (instantly) mixed volume that is unphysical as instant mixing of incompressible fluids is physically impossible. This does not occur for the LCE model as the initial volume of distribution is zero as shown by $V_d(t)$ in Table \ref{dists} and as plotted in Figure \ref{fig_2}. Both the LCE and Tk-GV models have zero initial volume of distribution, which requires an infinite  concentration at $t=0$. For the Tk-GV\footnote{where GV is a gamma variate; $C(t)=c\,t^{\alpha-1}e^{-\beta\,t}$} model, this is accomplished by adaptive fitting yielding $\alpha<1$ and is not


\begin{SCfigure}[][ht]\centering\includegraphics[scale=0.6]{fig_2.png}
\caption {Shown is a plot of LCE volume of distribution as a function of time, $V_d(t)$ (Table \ref{dists}), with reuse of the same parameters used to create Figure \ref{fig_1}. 
   Note that $V_\Omega \approx 0.81V_z $, where $V_\Omega $ occurs at $t_{x}=\Omega\,\beta^{-1}$.}
   \vspace*{-2em}
   \label{fig_2}
\end{SCfigure}

\noindent a constrained solution. For both models the infinity is \textit{integrable} and better mimics arterial concentration before the first sample times for small molecules like EDTA and DTPA chelates, and less so for inulin \cite{Cousins1997} and is our preferred method of adjusting venous sampling to arterial GFR conditions.

\section*{Methods}\label{Methods}

\subsection*{Datasets 1-3}\label{S1-3}

\textbf{Dataset 1} was a group of 13 adult liver transplant candidates most having ascites who underwent bolus $^{51}$Cr-EDTA intravenous injections followed by plasma collection of a total of 162 time-samples drawn at 5 min to 24 h for routine assessment of renal function. Approval was obtained from the Royal Free Hospital Research Ethics Committee for the required extra blood sampling (REC reference number 07/H07211/70). The time-samples were obtained at circa 5, 10, 15, 20, 30, 40, 50, 60, 90, 120, 180, 240, 360, 480, 720, and 1440 min. The results of E1 and Tk-GV renal modelling appeared elsewhere \cite{wesolowski2011validation,wickham2013development}. \textbf{Dataset 2} was from  44 adults with cirrhosis and moderate to tense ascites from a project approved by the Ethics Committee for Medical Research in Copenhagen (J. nr. (KF) 11-110/02), i.e., group I of reference \cite{Henriksen2015}. These subjects underwent bolus $^{51}$Cr-EDTA intravenous injection followed by plasma collection of 555 time-samples drawn at 5 min to 5 h, as well as circa 5 h of voluntary urine collection with assay of accumulated urinary drug activity. Time-samples were acquired at 0, 5, 10, 15, 30, 60, 90, 120, 150, 180, 240, and 300 min.
\textbf{Dataset 3} contains data from  41 $^{169}$Yb-DTPA adults in whom 328 plasma samples were processed for activity at 10 min to 4 h following bolus intravenous injection. The eight time-samples in each study were collected at circa 10, 20, 30, 45, 60, 120, 180, and 240 min. These data are from an older study prior to routine publication of ethics committee identification numbers, but were nevertheless ethically obtained \cite{Russell1985}. At that time, there were problems with DTPA-chelate plasma binding \cite{Carlsen1980}, likely due to improper pH buffering in certain commercial DTPA chelation kits, and the  $^{169}$Yb-DTPA time-samples were plasma protein binding corrected using ultrafiltration. This group had subjects whose renal function varied from renal failure to normal renal function \textit{without} evidence of fluid disturbance.

\subsection*{Urinary measurement considerations}\label{Uprob} 

In their classical work, Walser and Bodenlos, using E1 models, noted an unexpected 30 to 90 min delay between disappearance of radiolabeled urea from plasma and its appearance in urine \cite{walser1959urea}. This should serve as a remainder that $\text{CL}=\frac{\text{U V}}{\text{P}}$ is only defined for P (plasma concentration) under steady-state conditions. The P-value correction to account for the delay between zero time and marker first appearance in urine during a bolus experiment has been estimated as circa four min average, where literature estimates of average times were 2.5-8 min \cite{Ekins1966}. However, this time is longer in dilated urine collecting structures, e.g., renal pelvises and ureters, and for other reasons, e.g., renal insufficiency or intermittent obstructive disease. This time delay includes circulatory mixing time. That is, renal glomeruli filter arterial, not venous, blood. All of the plasma samples in this report are venous. Cousins \textit{et al.} showed negative arteriovenous differences for individual inulin and $^{99m}$Tc-DTPA time-samples at 30 min and beyond\cite{Cousins1997}. Thus, the concentration appropriate as a divisor for the U V mass product, i.e., Urine drug concentration times Volume of urine, is a later, smaller, venous plasma concentration than the venous plasma concentration occurring at the time of urine collection with the effect that renal clearance will be otherwise underestimated. 

For the classical renal clearance formula, CL~=~U~V~/~P, as P is not used to calculate U~V drug mass, the renal clearance variables can be separated into mass predicted to be excreted from a plasma concentration bolus model (Table \ref{dists}) and U~V, which then allowed for a mass balance comparison between plasma modelling and urine. Moreover, voluntary urine collection suffers from marked inaccuracy from multiple causes: neglecting to save a voided volume \cite{John2016}; post void residual urine in the adult bladder \cite{Uensal2004}; worse and more variable residuals in the elderly from genitourinary pathology (including uterine prolapse and prostatic hypertrophy) \cite{Griffiths1996}; bladder resorption of x-ray contrast \cite{Currarino1977} and other drugs with resorption made worse with long elapsed time between voids \cite{Dalton1994,Wood1983}. Review of 24 h urine collections suggested that catheterisation avoids the problem of incomplete urine collection in the one sense of neglecting to save a voided volume. Moreover, bladder catheterisation may correct some of the problems of residual urine in the bladder post void. However, even with catheterisation improper catheter placement itself led to residual bladder urine 26\% of the time \cite{Stoller1989}. Another problem is that there can be so little urine output in severe renal insufficiency that a small amount of bladder residual can render renal clearance based upon urine collection problematic.

In Dataset 2, case 6 of 44 had 6.5\% more urine mass collected (4.21 cpm) than administered (3.9533 cpm), which is unphysical. That case was excluded from mass balance comparisons. The other 43 cases were processed in two stages, initial screening, which showed an acceptable confidence interval agreement of mass balance between urine drug mass collected and the LCE method of predicting urine drug mass. Subsequently, to test whether the agreement was only a statistical aberration, the LCE prediction was adjusted to occur four minutes earlier as per \cite{Ekins1966}, the voided volume was augmented by a positional average post void bladder residual of 13.014 ml as per \cite{Uensal2004} followed by discard of those voided volumes that were less than 70\% of predicted as recommended \cite{John2016}, wherein the frequency of incomplete urine collections was noted as 6\% to 47\%.  This procedure was repeated after dropping the initial time-samples to discover that LCE urine mass predictions from models whose first time-sample started at 15 min agreed slightly better with the urinary mass calculations.   

\subsection*{Statistical methods}


 
\subsubsection*{Regression analysis}

For each data set several regression targets were tested for accuracy including: ordinary least squares (OLS), $\frac{1}{C_{obs}}$ weighted OLS, $\frac{1}{C_{obs}^2}$ weighted OLS, and OLS regression of log-log transformed $C_{obs}$ and sample times, where $C_{obs}$ are the observed concentrations. Of the regression targets tested, the $\frac{1}{C_{obs}^2}$ weighted OLS, also called proportional error modelling, proved the most accurate with the exception that log-log transformed regression is native to the Tk-GV clearance method, and not very different from proportional error modelling, see Eq.~(39) and surrounding text in reference \cite{wesolowski2020comparison}. For the Tk-GV method, the regression target is not curve fitting, but minimisation of the propagated proportional error of either clearance (CL) or of the exponential rate parameter ($\beta$) of a gamma distribution. Apart from the Tk-GV results, only the proportional minimum norm results are presented here. The regression method used for all targets was Nelder-Mead, which is more robust for absolute minimisation than gradient descent and most other methods, and is the most popular informed choice for regression analysis. Some pharmacokineticists prefer an adaptation of the maximum likelihood regression method from random variate minimisation, however, that was not tested here. The implementation was performed using the Mathematica 13.2.1.0 language on an Apple M1 iMac. All LCE model regressions converged rapidly, e.g., for Dataset 1 in 156.2 iterations at 52 milliseconds per case (mean values). For biexponentials, in one case of 57, the convergence was to a degenerate model of the $\lambda_2=\infty$ type; Dataset 2, case 19, 1470 iterations, 725 milliseconds, $C(t)= 0.100126 e^{-0.00755689 \,t}+0.0148127$, which 1.75\% failure rate is consistent with the circa 2\% failure rate reported elsewhere \cite{russell2002bayesian,wanasundara2016}.

 Widely used for clinical laboratory assay calibration, Passing-Bablok type I linear regression was applied to the results including comparison of predicted and observed urine mass \cite{Bablok1983}. Passing-Bablok 
 type I regressions are used to evaluate replacement same-scale methods and are bivariate nonparametric regressions. In specific, these regressions find least squares in $x$ and $y$ where the regression target is replacement, that is, a best linear functional relationship, whereas ordinary (OLS) regression yields a minimum error line for predicting $y$-values. %This is done to mitigate omitted variable bias appearing in the univariate case %\cite(Clarke2005}.
 \subsection*{Moving average and extrapolation testing}\label{average}
For residual analysis, i.e., of the difference between the concentrations of model values and time-samples, there is a need to examine how the models perform on average. As there are multiple plasma samples drawn at the same time following injection, one can take the number of earliest time-samples and average them to create a mean prediction for all the same model types. Next, one can drop an averaged time-sample from that group and bring in another averaged value from the next later group of time-samples, and assign that new group to have occurred at a new averaged time. This is performed until all the time samples have been average-averaged. This may seem contrived. However, if one were to drop and include unaveraged concentration values in each sample-time group, one would create a curve whose shape is dependant upon an arbitrary selection order of time-sample concentrations dropped or included. Finally, as each averaged, average-value is from the same number of averaged time-samples, it is equal-value weighted over the whole curve, which makes it possible to do statistical analysis, such as finding a reliable standard deviation that shows how well model curve shapes match those of noise reduced data, which is asymptotically correct  as the number of samples increases. 

Extrapolation testing is done without withholding data by testing with Wilcoxon signed-rank sum one-sample differences from zero of the first and also the last time-samples of each curve in a dataset. Small probabilities indicate that it is unlikely that the model extrapolates properly.   


\section*{Results}%

\subsection*{Dataset 1 results}

The parameters of the LCE model are AUC and $\beta$, and are obtained directly from the data curve fitting of that model as stated in the general case, i.e., Eq.~\eqref{eq1}. Figure \ref{fig_3} shows two competing plot types for viewing Dataset~1. The overall linear grouping of Figure \ref{fig_3}\textbf{a} can be interpreted as concentration propagating in time as a negative logarithm. However, negative logarithms would eventually yield negative concentrations. Thus, at some point in time, the logarithm should convert to an $x$-axis asymptote. Panel \textbf{b} shows relatively smooth but pronounced{\parfillskip=0pt\par} 

\begin{figure}[H]
\centering
\includegraphics[scale=.44]{fig_3.png}\captionsetup{belowskip=0pt}
\caption {Dataset 1 had 13 data series collected between 5 min and 24 h. These are shown as connected line segments, and plotted in two different ways. Panel \textbf{a} shows the cases plotted as linear concentration versus time on a logarithmic scale. Note the near linearity until late time of the line segments. Panel \textbf{b} shows semilog plots of the same data. Note the early time curvilinearity of the connected line segments.}
   \vspace{-1em}
   \label{fig_3}
  %\end{center}
\end{figure}  

\noindent  early-time log convexity, which are not linear and therefore not exponential for early-time on semi-log plotting. The curve fitting errors for those methods using proportional error modelling are displayed as residual plots in Figure \ref{fig_4}. Even though Dataset 1 has 13 cases, only 12 cases have 5 min time-samples and another 12 have 24 h 

\begin{figure}[H]
\centering
\includegraphics[scale=.3105]{fig_4.png}\captionsetup{belowskip=0pt}
\caption {Dataset 1 residuals for panel \textbf{a} the LCE models, panel \textbf{b} E1 models, and panel \textbf{c} biexponential models. The circles are proportional modelling errors. The heavy black curves are 12 sample moving averages. The probabilities are the likelihood of the earliest and latest 12 samples containing zero error of fitting.}
\label{fig_4}\vspace{-1em}
\end{figure}  

\noindent time-samples. A \textit{stationary} adaptation of a so-called \textit{moving} average of same sample-time averages was used as per the \nameref{average} Methods subsection. The standard deviation of those averages is a 2.38\% error of mean error of fitting for the LCE models, 2.87\% error for the E2 models and 14.17\% for the E1 models. For the LCE models, the 12 earliest and 12 latest time sample errors were insignificantly different from zero, and very significantly different for the E1 and E2 models. This suggests that on average for accuracy of curve fitting, LCE models with only two parameters outperformed E1 and E2, despite the latter having an extra two fit parameters. The standard deviations of the residuals themselves worsen in a different order, 5.69\% for E2, 8.42\% for LCE, and 20.07\% for E1.  Thus, the E2 fits, compared to the LCE fits are overfit (see the probabilities in Figure \ref{fig_4}c), and overfitting can cause a spurious reduction of error under the curve, and does cause erroneous extrapolation \cite{Hawkins2004}, which given the significant earliest and latest time-sample underestimation causes underestimation of AUC and overestimation of CL. The results appear in Table \ref{S1}, which shows that ability to capture MRT values longer than the 24 h (>1440 min, bold) data acquisition decreased in the following order LCE, Tk-GV, E2, E1 having respectively 7, 4, 2, 1 of 13 total MRT-values longer than 24 h.  The longer MRT-values led to  more inclusive AUC-values, and smaller clearances. The number of CL-values in the severe renal insufficiency range $(<20\text{ ml}\cdot\text{min}^{-1},$ bold type) decreased as LCE, Tk-GV, E2, E1 having respectively 5, 3, 3, 1 of those CL-{\parfillskip=0pt\par}

\begin{table}[H]
\centering
\captionsetup{justification=raggedright,margin=0cm}
 \caption {Dataset 1, some LCE, Tk-GV, biexponential (E2) and monoexponential (E1) model results.$^{\text{ a}}$}
 \vspace*{-.5em}
\label{S1} 
\small
\begin{tabularx}{\linewidth}{@{\hspace{.05cm}}l@{\hspace{.1cm}}c@{\hspace{.2cm}}c@{\hspace{.25cm}}c@{\hspace{.3cm}}c@{\hspace{.39cm}}c@{\hspace{.2cm}}c@{\hspace{.2cm}}c@{\hspace{.3cm}}c@{\hspace{.44cm}}c@{\hspace{.2cm}}c@{\hspace{.2cm}}c@{\hspace{.3cm}}c@{\hspace{.45cm}}c@{\hspace{.2cm}}c@{\hspace{.2cm}}c@{\hspace{.3cm}}c}
 %\hline
 \Xhline{2\arrayrulewidth} 
&\multicolumn{4}{c}{MRT (min)}&\multicolumn{4}{c}{AUC $(\text{min}\cdot\text{L}^{-1}$)}&\multicolumn{4}{c}{CL (ml$\cdot$min$^{-1}$)}&\multicolumn{4}{c}{$V_\text{MRT}$ (L)}\\
\hhline{~~~~~----~~~~----}
&LCE&\scalebox{.8}[1.0]{Tk-GV}&\,E2&E1&LCE&\scalebox{.8}[1.0]{Tk-GV}&\,E2&E1&LCE&\scalebox{.8}[1.0]{Tk-GV}&\,E2&E1&LCE&\scalebox{.8}[1.0]{Tk-GV}&\,E2&E1\\
\hhline{~----~~~~----~~~~}

Min&389&373&373&349&11.9&12.5&12.5&11.5&\textbf{2.4}&\textbf{4.0}&\textbf{7.6}&\textbf{12.0}&20.3&17.6&16.5&13.2\\
1st Quartile&451&453&437&365&19.6&20.2&19.8&19.1&\textbf{14.4}&\textbf{18.7}&\textbf{18.8}&20.8&24.8&20.7&20.3&17.1\\
Median&\textbf{1454}&830&812&598&27.3&26.7&24.9&21.8&36.6&37.5&40.2&45.9&33.1&26.8&27.9&20.3\\
        3rd Quartile&\textbf{2873}&\textbf{1995}&1189&863&75.5&55.8&53.9&48.0&51.5&49.8&51.0&52.6&51.6&39.9&35.3&30.5\\
    Max&\textbf{32735}&\textbf{8395}&\textbf{4251}&\textbf{2285}&414.7&252.0&131.6&83.3&84.4&79.7&80.3&86.6&78.9&59.7&44.2&35.4\\
\hhline{~----~~~~----~~~~}
Mean&\textbf{4096}&\textbf{1595}&1115&731&74.4&51.7&40.6&32.6&37.3&37.6&39.7&42.7&38.4&31.0&28.4&23.6\\
 \Xhline{2\arrayrulewidth} 
 \end{tabularx}
 \begin{tabularx}{1\textwidth}{X} 
 
 $^{\text{a }}$AUC is unit dose scaled. Results corresponding to MRT > 24 h and CL < 20 $\text{ml}\cdot\text{min}^{-1}$ are in \textbf{bold} font type.\\

\end{tabularx}
\vspace{-2em}
\end{table}



\noindent values. The best correlation between CL-values is from LCE and E2 (0.99765), and the worse is from LCE and E1 (0.97903). The smallest CL-value (LCE: 2.4 ml per min) had the longest MRT: 32735 min. The volumes of distribution (as $V_\text{MRT}$) decreased overall in the sequence LCE, Tk-GV, E2, E1. As mentioned in the Introduction, in severe renal insufficiency and/or fluid overload, there are two published suggestions for not using early time-samples to form better E1 model CL-prediction using 24 hours of data. The Wickham \textit{et al.} E1 $\geq$ 2 h method \cite{wickham2013development} would have us discard data before 2 h to improve CL-values overall, and the Br{\o}chner-Mortensen and Freund E1 $\geq$ 5 h method would have us discard data before 5 h to better predict severe renal insufficiency CL-values \cite{BroechnerMortensen1981}.  We compared proportional error regression for all of the applicable time-samples > 2 or 5 h to compare those procedures with the LCE results. Table \ref{S1b} shows Passing-Bablok regression line comparison CL-values normalised and CL $=1000/$AUC led to smaller CL-values.  Perhaps because of the use of proportional modelling, the slope

\begin{wraptable}{r}{10.3cm}\small
\vspace{-1em}
\renewcommand{\arraystretch}{1} % this reduces the vertical spacing between rows
\linespread{1}\selectfont\centering
\caption{Dataset 1, Passing-Bablok regression line, $y=m\,x+b$, and confidence intervals (CI) of CL-values of LCE ($x,\, \text{ml}\cdot\text{min}^{-1}$) versus E1 ($y$) models with various first time-samples, and correlations ($r$).}\label{S1b}
\vspace*{-2em}
\begin{tabular}{@{\hspace{0cm}}l@{\hspace{.4cm}}c@{\hspace{.2cm}}lc@{\hspace{.2cm}}lc}\\\toprule  
LCE ($x$),&\hspace{0cm}$b$,&\hspace{.1em}   95\% CI $\left(\frac{\text{ml}}{\text{min}}\right)$&$m$,&\hspace{.1em}95\% CI&$r$\\\midrule
E1 $>0$ h&11.63,&\;\;8.24 \;to 14.0&0.823,&0.709 to 0.992&0.97903\\ 
E1 $\geq$ 2 h&6.886,&\;\;2.38 \;to 9.11&0.988,&0.903 to 1.103&0.99091\\
E1 $\geq$ 5 h&4.362,&-0.327 to 6.65&1.144,&1.011 to 1.269&0.98635\\  \bottomrule
\vspace{-2em}
\end{tabular}
\end{wraptable}



\noindent  for E1 $\geq$ 2 h, i.e., 0.988, does not need to be further decremented using body surface area correction, but the intercept of 6.886 ml$\cdot$min$^{-1}$ is significantly greater than zero. Although still substantial at 4.362 ml$\cdot$min$^{-1}$, the E1 $\geq$ 5 h CL intercept with LCE CLs are now within the 95\% confidence interval (CI), such that the claim to better measure severe renal insufficiency seems plausible, but at the cost of significant exaggeration of CL values in the normal range as the slope is > 1. That is, if the LCE method results can be relied upon as an accurate basis for comparison.

These results suggest better stratification and detection of severe renal insufficiency for the LCE model than for the other methods examined. However, there is a need to validate the low CL-values. One quick way to check LCE accuracy is to take the LCE mean CL and divide that by the mean E1 CL, which yields a ratio of 0.873. That ratio agrees with the Chantler-Barratt \cite{Chantler1972} E1 correction factor of 0.87, so the LCE mean CL-value, at least, appears to be plausible. However, as our objective was explore the entire range of CL-values with special attention to the severe renal insufficiency range of values, it behoved us to do the same thing that Chantler and Barratt did, compare with urinary drug mass excreted. Thus, we next analysed Dataset 2, which has that information.


\subsection*{Dataset 2 results}

Figure \ref{fig_5} shows Passing-Bablok regression lines for Dataset 2's 43 subjects having physical urinary mass for $^{51}$Cr-EDTA radioactivity in urine at circa 300 min compared to urinary radioactivity predicted using the E1, E2, Tk-GV and LCE models, the latter with and without start time adjustment, adjusted urinary mass excretion, and urinary transit time adjustment. The urinary mass excretion corrections are presented in the \nameref{Uprob} Methods subsection. The confidence intervals for these regression formulas as well as E1 $\geq 2$ h fit results appear in Table \ref{S2}. All of the methods had 95\% confidence intervals for slope that included a slope of 1. However, only the LCE regressions had intercepts of zero within their 95\% confidence intervals. As a further demonstration that the error between the LCE model and urine mass collected is negligible, that difference was reduced to 0.4\% by correction of urine count rate for 13.014 ml expected bladder residual, for a urine transit time of four min, by a slight improvement in the LCE fits by dropping early time-samples leaving a median collection

\begin{figure}[H]
\centering\includegraphics[scale=.575]{fig_5.png}
\caption {As used for clinical laboratory assay calibration, Passing-Bablok type I regressions were used to evaluate equivalent or replacement same-scale methods for Dataset 2's 44 cases of voluntary urine mass as 10$^6$ counts per min (cpm) of $^{51}$Cr-EDTA activity. Panel \textbf{a} shows mono- and bi-exponential (E1 \& E2), Tk-GV and LCE urinary mass predictions. Panel \textbf{b} shows bladder residual adjusted urine mass versus and LCE urinary mass predicted 4 min earlier from fits starting with $\geq 14$ min plasma data and then discard of 7 cases with less than 70\% of the LCE predicted urine drug mass. Only the LCE models had no significant difference (95\% CI's) between slopes of 1 \textit{and} intercepts of 0, and was the only method that could be adjusted to approximate modified urinary mass conditions.}
\label{fig_5}
\end{figure}



\begin{wraptable}{r}{12.3cm}
\small
\vspace{-1.5em}
\caption{Dataset 2, Passing-Bablok regression line, $y=m\,x+b$, and confidence intervals (CI) for 6 models versus urine $^{51}$Cr-EDTA $10^6\cdot$cpm, number of cases (\textit{n}) and correlations ($r$).}\label{S2}
\vspace*{-2em}
%renewcommand{\arraystretch}{1} % this reduces the vertical spacing between rows
\linespread{1}\selectfont\centering
\begin{tabular}{@{\hspace{0cm}}l@{\hspace{.4cm}}c@{\hspace{.2cm}}lc@{\hspace{.2cm}}lcc}\\\toprule  
Urine $10^6\cdot$cpm,&$b$,&\;\;\;95\% CI&\hspace*{.3cm}$m$,

\;&95\% CI&\textit{n}&$r$\\\midrule
Adj. LCE \& urine&0.129,&$-$0.188 to 0.527&1.002,&0.893 to 1.119&36&0.95429\\
LCE&0.367,&$-$0.191 to 0.835&1.070,&0.903 to 1.232&43&0.90124\\
Tk-GV&0.910,&\;\;\;0.402 to 1.337&1.018,&0.877 to 1.168&43&0.89411\\
E1 $\geq 2$ h&0.989&\;\;\;0.537 to 1.490&0.991&0.826 to 1.149&43&0.88570\\
E2&1.098,&\;\;\;0.624 to 1.649&1.027,&0.849 to 1.179&42&0.88507\\
E1&1.676,&\;\;\;0.820 to 2.065&1.046,&0.846 to 1.248&43&0.87096\\
 \bottomrule
 \vspace{-2em}
\end{tabular}
\end{wraptable}


\noindent start time of 15 min (LCE-15), and finally by discard of the 7 recalculated urine samples with less than 70\% of the then adjusted LCE predicted activity to adjust for missing urine collections. This yielded tighter confidence intervals and better correlation. The 15 min rather than 5 min start time for LCE fitting also significantly improved its correlation with Tk-GV CL-values from 0.98035 to 0.98653 ($p<0.001,$ dependent $r$-value testing where $r=0.99858$ between LCE and LCE-15). The fit error improved from 4.4\% to 3.2\% using LCE-15. However, the early sample probability of bias decreased from 0.094 to 0.055, and the late sample bias probability decreased from 0.025 to 0.0002 with late sample bias being in the direction of overestimation of concentration. As these are highly correlated probability calculations, the differences are likely significant.  
It is not known in absolute terms that the voluntary urine collections used here were incomplete \cite{Henriksen2015} and the literature is quite clear that a 70\% cutoff is heuristic \cite{John2016}. However, the ratios of urine mass to LCE mass suggested this was the case, e.g., see Figure \ref{fig_6}. Using Cauchy distributions as the approximate ratio distributions\footnote{The ratio of two $\mu=0$ normal distributions, $\frac{\mathcal{N}(0,\sigma_1)}{\mathcal{N}(0,\sigma_2)}$, is  Cauchy distributed with median zero.} there may have been two peaks, very approximately at ratios of 0.78 and 0.96 with the volume deficient ratio peaked at 0.78 constituting 27\% of the cases. Moreover, using other distributions and other models, the area of the smaller ratio model could easily have been 0.5 of the total. Thus, one cannot rule out{\parfillskip=0pt\par} 

%MixtureDistribution[{0.273711,0.726289},{CauchyDistribution[0.777374,0.0564116],CauchyDistribution[0.964429,0.0990638]}]

\begin{SCfigure}[][ht]
\centering\includegraphics[scale=0.7]{fig_6.png}
\caption {Ratios of adjusted LCE and adjusted urinary drug masses are plotted as a histogram with a superimposed mixture model consisting of two summed Cauchy distributions (solid curve), the smaller being 27\% of the total density at a peak ratio of 0.78, and the larger 73\% having a peak ratio of 0.96 (both dashed). Although one cannot discard the fit quality with $n=43$, we know \textit{a priori} that the ratios are not exactly Cauchy distributed such that the results are only qualitative and vary depending on which density function(s) is (are) used for fitting.}
\vspace{-1em}
\label{fig_6}
\end{SCfigure}

\noindent missed urine collections, and on the other hand deletion of the 7 cases (16.3\%) with ratios < 0.7 may be too conservative. For example, applying the same procedure for the second closest procedure to urine mass, Tk-GV mass excretion, required discard of 17 urine collection cases or 39.5\%. E1 and E2 were more completely out of scope for such a procedure.  

There were few results in the very severe renal insufficiency range in Dataset 2, with the least plasma CL values for LCE, Tk-GV, E2 and E1 being 13.0, 24.3, 25.0 and 26.4 ml$\cdot$min$^{-1}$ respectively, with only (uncorrected) LCE having results ($\times 3$) less than 20.0 ml$\cdot$min$^{-1}$. The Chantler-Barratt style correction factor (mean LCE CL/mean E1 CL) for Dataset 2 was 0.783.

\subsection*{Dataset 3 results}\label{russel}

Dataset 3 consists of 41 $^{169}$Yb-DTPA studies conducted in adults with eight time-samples collected from 10 min to 4 h.  The issues of interest for this dataset were how the LCE formula behaved 1) for other GFR markers, 2) for subjects who did not have evidence of fluid disturbance and 3) for severe renal insufficiency in those conditions, especially with respect a lack of function in sense of the Schloerb challenge. Upon LCE identification of nominal CL-values $<20\text{ ml}\cdot$min, the dataset was sorted into cases with and without evidence of severe renal insufficiency. This is shown in Figure \ref{fig_7} as a clear difference between the behaviour of those two groups of studies.  That is, the suspected severely renal insufficient cases changed only slightly in concentration over 4 h, (Figure \ref{fig_7}c) as linearly decreased  concentration with elapsed time on a logarithmic scale. As seen for Dataset 2, 15 min may be a better LCE model first-sample time, and pruning the data would decrease the standard deviation of error especially due to the nonlinearity of the first time-sample of study number 31 of Figure \ref{fig_7}c. The more normal renal cases, Figure \ref{fig_7}b, approached the $t$-axis in late time as a  group,{\parfillskip=0pt\par}

\begin{figure}[H]
\centering
\includegraphics[scale=.46]{fig_7.png}
\caption {Dataset 3 linear-log plots. Panel \textbf{a} all cases, panel \textbf{b} without severe renal insufficiency, and panel \textbf{c} with severe renal insufficiency.}
\label{fig_7}
\vspace{-1em}
\end{figure}  

\noindent   with sometimes slight asymptotic flattening in late time. The LCE model fit error for all 41 cases (Figure \ref{fig_7}a) was significantly greater than the fit error of the renal insufficient cases (4.88\%), or only 2.14\% without case 31. By comparison, the E1 models' error of fitting to these eight cases was significantly more variable than LCE fit error (Conover $p=0.003$) at 6.87\%, or 4.57\% without case 31. 


Figure \ref{fig_8} shows plots of the minimum and maximum plasma clearances cases for LCE and E1, where the LCE CL-values ranged from $9.27\times10^{-10}$ to 163.7 ml$\cdot$min$^{-1}$, and for E1 from 4.30 to 176.1 ml$\cdot$min$^{-1}$, respectively for cases 19 and 15. Overall, the fits for the LCE models have a 4.85\% standard deviation of proportional error, compared to 10.10\%  for E1. Note that these errors are approximately 1/2 of the values seen for Dataset 1, where Dataset 1 data was acquired for six times as long, i.e., 24 h. Figure \ref{fig_8}a shows an asymptotic approach to the time-{\parfillskip=0pt\par}

\begin{figure}[H]
\vspace{-1em}
\centering
\includegraphics[scale=.55]{fig_8.png} 
\vspace{-1em}
\caption {Dataset 3 linear-log plots of the greatest (case 15) and least (case 19) plasma CL-values from LCE and E1 models. Panel \textbf{a}, greatest CL LCE model. Panel \textbf{b}, greatest CL E1 model. Panel \textbf{c}, renal failure LCE model, and Panel \textbf{d}, renal failure E1 model. In panels \textbf{a} \& \textbf{c} the solid lines are the LCE models, the straight dot-dashed lines are the logarithmic early asymptotes and the dashed lines are the terminal exponentials. In Panel \textbf{c}, the early asymptote and model curve are superimposed.}
\label{fig_8}
\vspace{-1em}
\end{figure}  

\noindent  axis after $t_x$, the intersection of the exponential curve and the early time asymptote; the  straight-line rending of a negative sloped logarithm on linear-log plotting. However, in Figure \ref{fig_8}c, the LCE model and its early asymptotic logarithm are both linear and superimposed and the exponential (dashed) has been flattened so much as to appear to be a constant. That is because the predicted elapsed time for $t_x$, the intersection of the logarithmic and exponential asymptotes, is geologically long; 4979 millennia in this worst case scenario. In Table \ref{renfail}  the largest LCE CL-value of 3.17 ml$\cdot\text{min}^{-1}$ for these suspected renal failure cases had the shortest $t_x$ at 7.27 days; still largely beyond the capacity for validation for most experiments.  The E1 model only identified half of the eight{\parfillskip=0pt\par} 

\begin{wraptable}{r}{7.3cm}
\small
\vspace{-0.2em}
\caption{Dataset 3 renal failure candidates'  LCE, Tk-GV, E2 \& E1 model CLs (ml$\cdot\text{min}^{-1}$).}
\label{renfail}
\vspace*{-2em}
\linespread{1}\selectfont\centering
\begin{tabular}{@{\hspace{.1cm}}c@{\hspace{.2cm}}c@{\hspace{.4cm}}c@{\hspace{.5cm}}c@{\hspace{.7cm}}c}\\\toprule
%\begin{tabular}{@{\hspace{0cm}}l@{\hspace{.4cm}}r@{\hspace{.2cm}}lc@{\hspace{.2cm}}lc}\\\toprule  
%LCE vs.,&\multicolumn{2}{l}{Intercept, 95\% CI}&Slope,&95\% CI&r\\
Study N\textsuperscript{\underline{\scriptsize o}}&LCE&Tk-GV&E2&E1\\\midrule

19&9.27$\cdot10^{-10}$&1.24&2.60&4.30\\
6&1.19$\cdot10^{-6}$&2.85&5.56&7.05\\
36&0.0312&6.29&5.63&18.2\\
41&0.406&10.0&11.7&22.3\\
3&1.06&9.49&13.9&20.3\\
31&1.13&5.72&8.30&17.3\\
18&2.89&27.2&43.5&48.7\\
40&3.17&17.0&20.7&30.0\\
  \bottomrule
  \vspace*{-2em}
\end{tabular}
\end{wraptable}


\noindent  severe renal insufficiency candidates of the LCE models. Proper identification of renal failure from E1 model usage is implausible as all 41 E1 models of Dataset 3 underestimated the concentrations of the first sample-times and 39 of 41 underestimated their last sample-time concentrations (Wilcoxon one-sample two-tailed $\textit{p}\ll0.0001$), and %{\parfillskip=0pt\par}  
which correspond to systematic overestimation of plasma CL, just as Schloerb observed. Similarly, the E2 first and last time-samples were significantly underestimating. The Tk-GV model identified seven of the eight cases having LCE CL $<20$, but at multiples of the LCE predicted plasma clearance values. The Chantler-Barratt style correction factor for Dataset 3 using LCE as the reference standard was 0.810.

\subsection*{Results, all datasets}
For the total of 98 subjects analysed, there were 16, 10, 9, and 6 having GFR-values < 20 ml$\cdot$min$^{-1}$ respectively for the LCE, Tk-GV, E2 and E1 models. The 95\% reference intervals for GFR were for: the LCE model from 0.015 to 167.9 ml$\cdot$min$^{-1}$; the Tk-GV model 3.38 to 163.9; the E2 model 5.59 to 174.0, and for E1 9.40 to 182.2 ml$\cdot$min$^{-1}$, which explains the frequency of detection of the methods for GFR-values < 20 ml$\cdot$min$^{-1}$, e.g., E1 was unlikely to return a GFR value lower than 9.40 ml$\cdot$min$^{-1}$. Figure \ref{fig_9} shows how this occurred by quantile-quantile (Q-Q) plotting of all 98 GFR measurements for the LCE and the E1 models. This type of plot shows how{\parfillskip=0pt\par}
 
\begin{SCfigure}[][ht]
\centering\includegraphics[scale=0.45]{fig_9.png}
\caption {Superimposed are the Q-Q plots for LEC (open circles) and E1 (open triangles) GFR-values. The solid grey lines give the locations of theoretically normally distributed values. The LCE values become abnormal very close to zero clearance and transition to the normal distribution line proximity at only a few ml$\cdot$min$^{-1}$. However, the E1 GFR-values are approximately at 20 ml$\cdot$min$^{-1}$ before they 'normalise,' which suggests why it is difficult to measure GFR-values < 20 ml$\cdot$min$^{-1}$ using current methods.}
\vspace{-1em}
\label{fig_9}
\end{SCfigure}

\noindent  measured values depart from the theoretical distribution used; in this case, the normal distribution. If one supposes that GFR-values are normally distributed a problem occurs because normal distributions extend from negative infinity to positive infinity, but GFR values cannot be less than zero. In practice that means that there should be a departure from normally distributed GFR-values in the region near zero GFR. Indeed, that is the case for the LCE model values, but not as abruptly so for the E1 values. To investigate how abrupt this change should be the correlations between CL and fluid volume divided by weight, $\mfrac{V_\text{MRT}}{W}$, were examined, which although variable between datasets, could be ranked from least correlated to most correlated for all 98 patients sequentially as Tk-GV, E2, E1, and LCE. For the three datasets, Tk-GV had zero significant correlations, two negative and one positive correlation, the latter with Dataset 2. E2 and E1 had three positive correlations each, two not significant and one significant with Dataset 2. LCE had three negative correlations, two significant and one not significant with Dataset 2, which dataset had few low GFR-values and where the correlation was associated with large volumes for renal insufficient GFR values that ceased to be significant above 25 ml$\cdot$min$^{-1}$. Taking at face value, it would seem that Tk-GV CL-values are the more reliable in the severe renal insufficiency range. 

Both Kruskal-Wallis rank testing and 1-way ANOVA showed significant differences between the hepatorenal compromised subjects in Dataset~1 (mean 37.3 ml$\cdot$min$^{-1}$)
and the other datasets. However, there was no significant difference between Datasets 2 and 3 (means 73.6 and 77.0 ml$\cdot$min$^{-1}$, respectively), for LCE CL-values from $^{51}$Cr-EDTA and $^{169}$Yb-DTPA despite moderate to tense ascites in the former and the lack of fluid disturbance in the latter. Although a close agreement is necessary to insure that the same thing is being measured for each pharmaceutical, the lack of a significant difference between pharmaceutical CL-values could also be coincidental. Finally, the Chantler-Barratt style E1 correction factor using the default LCE model as the standard for all 98 cases was 0.801.


\section*{Discussion}

This paper is a first presentation of plasma clearance for GFR markers using the logarithm of cumulative exponential concentration (LCE) model; the product of the LCE density function Eq.~\eqref{eq2} and a constant; AUC, the area under the curve being modelled, which can either be obtained during curve fitting, Eq~\eqref{eq3}, or solved for later, Eq.~\eqref{eq4}. The LCE model was not very sensitive to changes in the last time of plasma sampling. This is because the model has a terminal exponential that in the absence of terminal concentration behaviour is assigned by the early-time logarithmic asymptote; a fixed feature of the model itself. That is, the terminal exponential first becomes more predominant at a time $\left(t_x = \frac{\Omega}{\beta}\right)$ long enough that it was not observed in the presence of severe renal insufficiency, although it was seen for high normal renal function, e.g., see Figure \ref{fig_8}a. The time at which the early and late asymptotes are equal from $\beta\,t$ equals Lambert's $\Omega\approx 0.567143$, and could be varied by varying $\alpha$ for models proportional to{\parfillskip=0pt\par}  
$$\ln \left(\frac{\alpha}{e^{\beta \,t}-1}+1\right)\;\;,$$
where physiologically $\alpha$ is likely to be between 0 and 2. The LCE model results from setting $\alpha=1$. For $\alpha>1$, $t_x$ occurs sooner (CL $\uparrow$), and for $\alpha<1$, $t_x$ occurs later (CL $\downarrow$). Even 24 h data often did not have enough tail information to allow $\alpha$ to vary consistently, and it is possible that an $\alpha>1$ should be used if it can be shown that the model sufficiently underestimates urinary drug mass excreted. The Tk-GV CL-values were least correlated to relative fluid volume. One can make a physiologic argument for a negative correlation of fluid excess in liver failure with decreasing CL, but it is also possible that LCE underestimates low CL-values, and if so, the urine drug mass corrected for missing urine samples may be underestimating as well. There is no apparent physiologic argument for positive correlation of fluid volume and increasing CL, which casts doubt on the reliability of E1 and E2 CL-values as both had positive correlations. Given all the problems known for bolus intravenous voluntary urine collection, it may be that only constant infusion calculations with a host of precautions has a chance of being used as a reference standard without fear of contradiction. However, even with precautions, urinary measurement of GFR in the renal failure range is still problematic such that plasma clearance using methods like Tk-GV or LCE as is or as some suitable variant may be the only practical measurement systems for those patients.

%; using drug administration rate as the standard, bladder catheterisation, ultrasound control of bladder residual and urinary tract pathology with injection time to bladder delay correction of plasma sampling, dose loading and very long infusion times 

The initial concentration in a peripheral veinous sampling site is zero at the time of a bolus intravenous injection in a different vein.  To model the entire concentration curve including the zero initial concentration requires more parameters, data, processing and theory than are typically used for routine drug assessment \cite{Wesolowski2016GDC,wesolowski2020comparison}. The alternative is to use incomplete models that do not model the very rapidly changing early vascular effects with the caveat that first time-sampling be drawn some minutes or hours following the time of peak venous concentration.  How many minutes or hours following injection one should wait to take samples depends on the model. For the Tk-GV model, 5 min is enough. For bolus injections of inulin and $^{99m}$Tc-DTPA, 25 minutes in adult humans was the time at which arteriovenous differences of concentration concentrations equalised \cite{Cousins1997}. The LCE model produced possibly slightly better results with sampling times starting at 15 min rather than 5 min for Dataset 2, compared to start times beginning at 2 or 5 hours and ongoing for 24 h for E1 as suggested by Wickham \textit{et al.} \cite{wickham2013development} and Br{\o}chner-Mortensen \cite{BroechnerMortensen1981}, respectively. Table \ref{S1b} shows this effect for Dataset 1, the only dataset with 24 h data. Compared to the LCE model, using an E1 model with 24 h data beginning at 2 h proved more accurate than fitting E1 to the complete data beginning at 5 min, but E1 > 2 h still significantly exaggerated small CL-values, which is not correctable using current methods (Chantler-Barratt and  Br{\o}chner-Mortensen corrections). Fitting data starting at 5 h no longer significantly exaggerated smaller CL-values but exaggerated larger CL values compared to 24-h LCE CL-values (Dataset 2 was unusable at a 5 h start time with only 5 h total urinary drug mass data). 

Not unexpectedly,  the results showed that the attempts to fit E1 or E2 to time-limited data resulted in poor quality fits of the AUC underestimating type because the curve shape of the data was more linear-logarithmic than exponential. This was the same problem for all three datasets, and is shown for Dataset 1 in Figure \ref{fig_4}. The change in concentration as apportioned in time logarithmically is not unknown. For instance, in Datasets 1 and 3 above, the time-samples were independently selected to be drawn at times that form a nearly geometric progression, where for example, a perfectly geometric progression would be a doubling time: 15, 30, 60, 120, 240, 480,$\dots$ min. Such a scale is equidistant when its logarithm is taken, where the motive for doing so is to acquire data such that the change in concentration is more or less linear and detectable between time-samples. So clearly equal log-time, time-sample spacing is appreciated by some experimentalists. The search for incorporating that observation into a plausible model that forms a better basis for quantifying concentration curves than exponentials yielded a simple result, the LCE model.

An important consideration is robustness, which is the ability of an algorithm to produce a physical answer. Without an explicit $\lambda>0$ constraint, a E1 $=c\,e^{-\lambda\,t}$ two time-sample exact solution algorithm will echo negative values, \{CL, $\lambda\} < 0$, when the formula has a positive slope and an undefined ($+\infty$) E1 survival function. The LCE CL is never less than zero. Hint, $\ln\left(\mathbb{R}_{<0}\right)\in\mathbb{C}$. That is, the LCE model is more robust than even a E1, never mind E2, which all too frequently produces answers that are unphysical. 

The accuracy of the LCE measurements was not definitively established. This was examined compared to urine activity for Dataset 2, for which no other plasma CL method examined yielded a result below 20 ml$\cdot$min$^{-1}$ clearance. The only uncorrected plasma model drug elimination method that did not have a significantly positive intercept with urine mass collected corresponding to presumed overestimation of low CL-values was the LCE method. Without adequate controls, only \textit{post hoc} corrections, as outlined in the literature, were available for investigating the conditions that may allow for still better agreement between the modified reference standard candidates, and did not allow for preferring the renal model or the plasma model as a stand-alone reference standard. The reconciliation scheme used to find good agreement between corrected LCE mass eliminated and corrected urine drug mass only made sense for plasma and urine models that were already quite similar. To be clear, there was no gold standard assumption made and the candidate reference standard were Tk-GV, corrected LCE or corrected urine collection. Dataset 3, Figure \ref{fig_7}c illustrates that in renal failure the attempt to fit gently down-sloping, almost perfectly straight linear-log plot lines with E1 functions did a poor job of fitting, i.e., the model curve shapes were not appropriate. Figure \ref{fig_7}c is supplemented by Table \ref{renfail}, which shows that the obvious slope differences for the suspected renal failure group had very low GFR-values for the LCE method alone. A competing method for CL-measurements in the renal failure range is Tk-GV. Tk-GV CL-values were unrelated to its predicted relative fluid content (V/W). However, LCE CL and relative fluid content were correlated, especially below 25 ml$\cdot$min$^{-1}$. 

Using current methods, few radiometric plasma GFR clinical studies are performed for patients having less than 20 ml$\cdot$min$^{-1}$,  e.g., there were none in Dataset 2. However, a patient having 10 ml$\cdot$min$^{-1}$ clearance may merit different management than one with 0 or 20 ml$\cdot$min$^{-1}$.  In lieu of a direct measurement of GFR, a current practice is, for example, to use the average of creatinine and urea renal CL-values or 24 h creatinine renal CL-values as well as urinary albumin levels as rough indicators of what the appropriate clinical management may be \cite{BroechnerMortensen1981}. Even using exogenous radiotracers, bolus injection urinary collection measurements are, however, problematic, see the \nameref{Uprob} Methods subsection for details. For example, a severely oliguric patient does not produce enough urine to yield accurate renal clearance values. In prior work, Tk-GV clearances were more accurate and precise than E2 clearances \cite{wanasundara2016}. In this work, Figure \ref{fig_9} suggests more of a sharp division between normally distributed CL values and renal failure CL for the LCE method. That is, it appears that LCE method satisfies the Schloerb challenge to quantify a lack of renal function, where current plasma clearance methods do not do so. However, only prospective studies using more sensitive methods like Tk-GV and LCE can determine how those methods agree with other patient  management indicators, e.g., for selecting patients for dialysis.  



%This is much more compact than the current practice of doing the same thing by transforming each different $C(t)$ equation into Laplace space, convolving in that space by a Laplace space transformed unit box infusion function of duration $\tau$ and then inverse Laplace space transforming the convolution to obtain a result, as was, for example performed at greater length and complexity for a biexponential and only a biexponential by Hermann  \cite{hermann2017application}.


\subsection*{Limitations}

Much of the mathematical exploration and statistical testing performed to generate this report have been omitted in order to present the most important observations without undue burden placed upon the reader. For example, the LCE density function was identified from simplification of a four parameter model and proportional error modelling was selected as best of class from four methods. Many formulas, e.g., for constant infusion, half-life of volume and concentration as functions of time were similarly omitted. The Appendix section outlines those derivations specific to the logarithm of cumulative exponential (LCE) density function needed in this report, where a more complete set of equations is merely a routine application of the calculus to a more complete set of general equations as previously presented and applied respectively for the gamma and gamma-Pareto distributions in \cite{Wesolowski2016PLoS,wesolowski2020comparison}. 

Scaling of measured GFR is needed to classify sufficiency of renal function versus metabolic demand and should be done with respect to normal measured GFR by 1) normalising powers of variables like volume of distribution and body weight over 2) at least an 8-fold weight range, as well as over 3) a range of abnormal fluid balance, e.g., see \cite{wesolowski2006improved,wesolowski2011validation}. This then would provide for a reference standard for calculating estimating formulas for creatinine, cystatin-C and any other endogenous metabolite. This has not been done in this introductory paper. 
Unfortunately, much of the clinical history for Dataset 3 has been lost in time, and clinical correlation, as well as body scaling are needed for final interpretation of the value of the LCE method.

\section*{Conclusions}\label{sec:conclusions}

The two methods that appear to have potential applicability to the severe renal insufficiency measurement problem are the Tk-GV method and the LCE method, the latter introduced in this work. Of these, the LCE method more clearly appears to meet the Schloerb challenge of quantifying anephric conditions, as well as being the only plasma clearance method tested to agree with renal clearance within its 95\% confidence intervals for both slope and intercept from bivariate nonparametric linear regression.

\section*{Acknowledgements}\label{sec:acknowledgements}

Prof. Geoffrey T. Tucker of the University of Sheffield, Sheffield, UK is thanked for his suggestions concerning this manuscript. Maria T. Burniston and coauthors in the UK \cite{wesolowski2011validation} are thanked for graciously providing Dataset 1. Prof. Jens H. Henriksen of the University of Copenhagen, Denmark is thanked for providing Dataset 2. Prof. Charles D. Russell of the University of Alabama at Birmingham is thanked for providing Dataset 3. Surajith N. Wanasundara is thanked for his help with computer implementation of an earlier version of the Tk-GV processing program.


\section*{Appendix}\label{sec:appendix}


\noindent \textit{Theorem}. The density function for $-\ln \left(1-e^{-\beta\, t}\right)$ is $f(t)=-\dfrac{6\, \beta }{\pi ^2}\ln \left(1-e^{-\beta\, t}\right)$, i.e., the log cumulative exponential (LCE) distribution. \\\textit{Proof.} We first note the derivative that yields $-\ln \left(1-e^{-\beta\, t}\right)$,


\begin{equation}\label{eqA1}
\frac{d}{d\,t}\left[-\frac{1}{\beta}\text{Li}_2\left(e^{-\beta\, t}\right)\right]=-\ln \left(1-e^{-\beta\, t}\right)\;\;,
\end{equation}
where $\text{Li}_n(z)=\sum _{k=1}^{\infty } \frac{z^k}{k^n}$ is the polylogarithm function of order $n=2$. Next, we scale $-\ln \left(1-e^{-\beta\, t}\right)$ to be a density function by dividing by the total area from 0 to $t\to \infty$ of its antiderivative. That is since, 
 
\begin{equation}\int_0^{\infty}\frac{d}{d\,t}\left[-\frac{1}{\beta}\text{Li}_2\left(e^{-\beta\, t}\right)\right]=\lim_{t\to \infty}\left[-\frac{1}{\beta}\text{Li}_2\left(e^{-\beta\, t}\right)\right]+\frac{1}{\beta}\text{Li}_2\left(e^{-\beta\cdot 0}\right)=0+\frac{\pi ^2}{6\, \beta}\;\;,\end{equation}

\noindent then,
$$f(t)=-\ln \left(1-e^{-\beta\, t}\right)\bigg/ \frac{\pi ^2}{6\, \beta}=-\frac{6\, \beta }{\pi ^2}\ln \left(1-e^{-\beta\, t}\right)\;\;.\qed$$
\textit{Corollary.} Similarly, the CDF and CCDF = CDF $-$ 1, are from the antiderivative evaluated between 0 and $t$,

\begin{equation}\label{Ft}
F(t)=1-\frac{6 }{\pi ^2}\text{Li}_2\left(e^{-\beta\, t}\right),\;\;\;S(t)=\frac{6 }{\pi ^2}\text{Li}_2\left(e^{-\beta\, t}\right)\;\;,
\end{equation}
where CCDF is the complementary CDF.  The CCDF is symbolised $S(t)$ here, even though  $S(t)$, survival functions, are technically mass functions and not density functions. Note that how long it takes for $F(t)$ to converge to 1 is dependent on a single parameter, $\beta$; the smaller $\beta$ is, the longer it takes. 
The mean residence time, where MRT $=\int_0^\infty t\,f(t)\,dt$ for the LCE density function, was found from evaluating its antiderivative from $t$ equals 0 to $\infty$,

$$\text{MRT}=\frac{6\, \zeta (3)}{\pi ^2\, \beta}\approx\frac{0.730763}{\beta }\;\;,$$
where the zeta ($\zeta$) function of 3 is approximately 1.20206. Note that the ratio of MRT$_\text{LCE}$ and $t_x$ is a constant equal to $\mfrac{6\, \zeta (3)}{\pi ^2\, \Omega}$. That is,  MRT$_\text{LCE}$ occurs at a time approximately 1.2885 times longer than $t_x$. Note that the MRT occurs when the tail is already predominantly exponential. The \textit{median} residence time (LCE half-survival) was calculated by Newton-Raphson's method for $u$ such that the $S(u)=\frac{6 }{\pi ^2}\text{Li}_2\left(e^{-u}\right)=\frac{1}{2}$. Then, let $u=\beta\,t_{m}$, and solve for $t_{m}$, which yields,

$$t_{m}\approx \frac{0.415389}{\beta}\;\;.$$

%With the exception of the instantly mixed monoexponential model, half-life varies with time elapsed since dose administration. 

%As a further example of the utility of density notation, let us show the general formula for concentration following a constant infusion of infinite duration, 


%\begin{equation}\label{Cin}
%\begin{aligned}
%{{C_{in}(t)}}&=\int_0^t C( x )\,dx=\int_0^t \text{AUC}_{\mathrm{bolus}}\;f(x) =\text{AUC}_{\mathrm{bolus}}\int_0^t f(x) \,dx \\
%&=\text{AUC}_{\mathrm{bolus}}\,F(t)\\
%&\equiv { C_{\textit{SS}} }\,F(t) =\dfrac{D_R}{ \text{CL} }F(t), 
%\end{aligned}
%\end{equation}

%\noindent where the 'in' of $C_{in}$ is for 'infusion', and where $C_{SS}=D_R/$CL is the steady state concentration. $D_R$, is an acronym for dosing rate and elsewhere R$_0$ or $k_0$ are often used. This is general for any $F(t)$ and merely requires substitution of whichever $\text{CDF}=F(t)$ one is using, which for the LCE density that would be left Eq~\eqref{Ft}, thus, $C_{in}=C_{\textit{SS}}\left[1-\frac{6 }{\pi ^2}\text{Li}_2\left(e^{-\beta\, t}\right)\right]$, and note that the limit as $t\to\infty$ of this is indeed $C_{\textit{SS}}$. If instead we wish to infuse starting at exactly $t=0\text{ and ending at }\tau$, i.e., a $C_{\tau}(t)$, we would convolve an $f(t)$ model with a unit box function to obtain,
%https://www.certara.com/knowledge-base/understanding-steady-state-pharmacokinetics/
%\begin{equation}\label{Ctau}
 %C_{\tau}(t) =\dfrac {\text{AUC}}{\tau}
 %\begin{cases}
%F(t), & 0 < t \leq \tau \\[2\jot]
 % F(t)-F(t-\tau), & t>\tau
 % \end{cases}\;\;\;,
%\end{equation}

%\noindent where again AUC$_\text{bolus}\equiv C_{\textit{SS}}$, such that $C_{\textit{SS}}$ can be predicted from a bolus curve or obtained from curve fitting of infusion data. Substituting $F(t)$ from the LCE CDF, left Eq.~\eqref{Ft}, into this yields, 

%\begin{equation}
 %C_{\tau}(t) = \dfrac{6\,\text{AUC}}{\pi ^2\,\tau}
% \begin{cases}
 % \mfrac{\pi ^2 }{6}-\text{Li}_2\left(e^{-\beta\, t}\right), & 0 < t \leq \tau \\[2\jot]
% \text{Li}_2\left(e^{-\beta\, (t-\tau)}\right) -\text{Li}_2\left(e^{-\beta\, t}\right), & t>\tau
%  \end{cases}\;\;\;,
%\end{equation}


% allows the constant infusion, $C_{in}(t)$, equation to be written

%$$C_{in}(t)=\text{AUC}_{bolus}\,F(t)=C_{SS}F(t);\;\;C_{SS}=\frac{D_R}{\textit{CL}}$$

%Seeking a shape coefficient generalisation of the (apparently) new ln-coth density function. Obviously, the ln-coth function itself needs to be presented first. However, the following is limited in scope to the bare bones of the ln-coth density function's properties, as follows. For a rate parameter $\beta>0,$

%$$\int_0^{\infty } \ln \left[\coth \left(\frac{\beta\,  x}{2}\right)\right] \, dx=\frac{\pi ^2}{4 \,\beta }\;\;.$$
%Thus, 
%$$f(x)=\frac{4\, \beta }{\pi ^2}\,\ln \left[\coth \left(\frac{\beta\,  x}{2}\right)\right]$$
%is a density function with support on $x=(0,\infty).$ Inquiring minds may want to know why $\mfrac{\beta\,  x}{2}$ is used rather than just $\beta\,  x.$ For one thing $\coth \left(\mfrac{w}{2}\right)=\mfrac{2}{e^w-1}+1$, which allows for nice properties, e.g., asymptotic equivalence of the right tail of the ln-coth distribution to a scaled exponential density function, i.e., 
%$$\underset{x\to \infty }{\text{lim}}\,\frac{4\, \beta  }{\pi ^2}\,\ln\left[\coth \left(\frac{\beta \, x}{2}\right)\right]\sim \frac{8}{\pi ^2}\beta \,e^{-\beta \, x}\;\;,$$
%where $f(x)=\beta \,e^{-\beta \, x}$ is the exponential distribution. It is possible that the scale $\mfrac{8}{\pi ^2}\approx 0.810569$ is a limiting correction factor for $\beta$ large for total area under the curve (AUC) calculations for scaled exponential distribution fits (AUC$\,\times\,\beta \,e^{-\beta\,  x}$) to certain data types, i.e., those for which the ln-coth function is hypothetically much more accurate.

%The left hand asymptotic equivalence is   $$\underset{x\to 0 }{\text{lim}}\,\dfrac{4 \,\beta  }{\pi ^2}\,\ln\left[\coth \left(\dfrac{\beta \, x}{2}\right)\right]\sim\dfrac{4\, \beta  }{\pi ^2}\,\ln \left(\dfrac{2}{\beta\,  x}\right)\;,$$ 

%\noindent which shows that the ln-coth distribution is a straight line with descending slope on a linear-$y$, log-$x$ plot, where the discontinuity at the origin has zero asymptotic area. As such a line would  eventually lead to prohibited negative $y$-values, the right hand asymptote was chosen as above to be exponentially convergent to zero for $x$ increasing.

%To create a generalized ln-coth density function, I want to include a shape parameter $\alpha>0.$ Currently, I am using numerical methods (double exponential) to integrate $$ \int_0^{\infty } \ln^\alpha \left[\coth \left(\frac{\beta\,  x}{2}\right)\right] \, dx,$$ because I cannot solve that definite integral, at least for now. Can anyone express that latter definite integral as a simple function of $\alpha$ and $\beta\,$?

%The motive for doing such a thing is to allow for closed form expressions of the generalized distribution that allows the otherwise straight line on a linear-log plot to bend to better accommodate the data. For example, the figure below shows a $\ln^\alpha\!\text{-}\coth$ fit to $^{51}$Cr-EDTA bolus injection plasma clearance concentration data from a prior study.

\end{onehalfspacing}

\begin{small}
%\bibliographystyle{apacite}
%\bibliographystyle{natbib}
%\bibliographystyle{vancouver}
\bibliographystyle{myvancouver2}
%\bibliographystyle[square]{apacite}
%\bibliographystyle{spmpsci}
%\bibliographystyle{spbasic}
\bibliography{Reff3}
\end{small}

\end{document}