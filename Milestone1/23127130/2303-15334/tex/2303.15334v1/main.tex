

\documentclass[10pt,journal,compsoc]{IEEEtran}

\ifCLASSOPTIONcompsoc
  \usepackage[nocompress]{cite}
\else
  \usepackage{cite}
\fi

\usepackage{times}
\usepackage{epsfig}
\usepackage{graphicx}
\usepackage{amsmath}
\usepackage{amssymb}
\usepackage{subfigure}
\usepackage{hyperref}
\usepackage{seqsplit}
\usepackage{url}
% \usepackage{algorithmic}
% \usepackage{algorithm}
\usepackage{multirow}
\usepackage{threeparttable}
\usepackage{tablefootnote}
\usepackage{booktabs}
\usepackage{arydshln}
\usepackage{color}
\usepackage{xcolor}
\usepackage{tikz}
\usepackage{bm}
\usepackage{subfiles}
\usepackage{colortbl}
\newcommand{\PreserveBackslash}[1]{\let\temp=\\#1\let\\=\temp}
\newcolumntype{C}[1]{>{\PreserveBackslash\centering}p{#1}}

\ifCLASSOPTIONcompsoc
  \usepackage[nocompress]{cite}
\else
  \usepackage{cite}
\fi

\ifCLASSINFOpdf
\else
\fi

\hypersetup{
    colorlinks=true,
    linkcolor=blue,
    filecolor=magenta,      
    urlcolor=magenta,
}
\def\etal{\emph{et al}. }
\def\ie{\emph{i.e.} }
\def\eg{\emph{e.g.} }

\definecolor{Gray}{gray}{0.92}

\definecolor{codegreen}{rgb}{0.0,0.6,0.0}

\def\httilde{\mbox{\tt\raisebox{-.5ex}{\symbol{126}}}}

\usepackage[ruled,vlined,linesnumbered]{algorithm2e}
\makeatletter
\newcommand{\algorithmfootnote}[2][\footnotesize]{%
  \let\old@algocf@finish\@algocf@finish% Store algorithm finish macro
  \def\@algocf@finish{\old@algocf@finish% Update finish macro to insert "footnote"
    \leavevmode\rlap{\begin{minipage}{\linewidth}
    #1#2
    \end{minipage}}%
  }%
}
\makeatother

% \usepackage[pagebackref,breaklinks,colorlinks]{hyperref}

\usepackage[accsupp]{axessibility}

% Support for easy cross-referencing
\usepackage[capitalize]{cleveref}
\crefname{section}{Sec.}{Secs.}
\Crefname{section}{Section}{Sections}
\Crefname{table}{Table}{Tables}
\crefname{table}{Tab.}{Tabs.}

\newcommand{\xwnote}[1]{\footnote{\textcolor{red}{\small #1}}}
\newcommand{\myparagraph}[1]{{\vspace{.5em} \noindent \bf #1}}
\newcommand{\ye}[1]{{\color{blue} #1}}
\newcommand{\yifu}[1]{{\color{red} #1}}
\begin{document}

\newcommand{\modelname}{ByteTrackV2\xspace}


\title{\modelname: 2D and 3D Multi-Object Tracking by Associating Every Detection Box}

\author{Yifu Zhang,
        Xinggang Wang,
        Xiaoqing Ye,
        Wei Zhang,
        Jincheng Lu,
        Xiao Tan,
        Errui Ding,
        Peize Sun,
        Jingdong Wang
        
\IEEEcompsocitemizethanks{\IEEEcompsocthanksitem 
Corresponding author: Xinggang Wang. Email: xgwang@hust.edu.cn\\
\IEEEcompsocthanksitem Y. Zhang, X. Ye, W. Zhang, J. Lu, X. Tan, E. Ding, J. Wang are with Baidu Inc., China.\protect\\
\IEEEcompsocthanksitem X. Wang is with Huazhong University of Science and Technology, China.\protect\\
\IEEEcompsocthanksitem P. Sun is with The University of Hong Kong, China.\protect\\}
% E-mail: \{chnuwa, wezeng\}@microsoft.com
% \IEEEcompsocthanksitem Yifu Zhang, Xinggang Wang and Wenyu Liu are with Huazhong University of Science and Technology.\protect\\
% Email: \{yifuzhang, xgwang, liuwy\}@hust.edu.cn}% <-this % stops an unwanted space
}

% \markboth{Journal of \LaTeX\ Class Files,~Vol.~14, No.~8, August~2015}%
% {Shell \MakeLowercase{\textit{et al.}}: Bare Demo of IEEEtran.cls for Computer Society Journals}

\IEEEcompsoctitleabstractindextext{
\begin{abstract}
% Multi-object tracking (MOT) aims at estimating bounding boxes and identities of objects across video frames. 2D and 3D MOT tasks have been independently addressed as separate challenges by different research communities, which fails to fully leverage the commonalities of the two tasks. We propose \modelname to tackle the two tasks in a unified framework, including object detection, motion prediction, and detection-driven hierarchical data association. Detection boxes serve as the basis of the entire framework. We present a hierarchical data association strategy based on detection scores to mine the true objects in low-score detection boxes, which alleviates the problems of true object missing and fragmented trajectories. The generic data association strategy shows effectiveness under both 2D and 3D settings. In 3D scenarios, it is much easier for the tracker to predict object velocities in the world coordinate. We propose a complementary motion prediction strategy that incorporates the detected velocities with a Kalman filter to address the problem of abrupt motion and short-term disappearing.  \modelname leads the nuScenes 3D MOT leaderboard, ranking first in both camera (56.4\% AMOTA) and LiDAR (70.1\% AMOTA) modalities.  Furthermore, it is nonparametric and can be integrated with various detectors, facilitating the unification of 2D and 3D MOT tasks.

Multi-object tracking (MOT) aims at estimating bounding boxes and identities of objects across video frames. Detection boxes serve as the basis of both 2D and 3D MOT. The inevitable changing of detection scores leads to object missing after tracking. We propose a hierarchical data association strategy to mine the true objects in low-score detection boxes, which alleviates the problems of object missing and fragmented trajectories. The simple and generic data association strategy shows effectiveness under both 2D and 3D settings. In 3D scenarios, it is much easier for the tracker to predict object velocities in the world coordinate. We propose a complementary motion prediction strategy that incorporates the detected velocities with a Kalman filter to address the problem of abrupt motion and short-term disappearing. \modelname leads the nuScenes 3D MOT leaderboard in both camera (56.4\% AMOTA) and LiDAR (70.1\% AMOTA) modalities. Furthermore, it is nonparametric and can be integrated with various detectors, making it appealing in real applications. The source code is released at \url{https://github.com/ifzhang/ByteTrack-V2}.
 
\end{abstract}

\begin{IEEEkeywords}
2D\&3D Multi-Object Tracking, Motion Prediction, Data Association
\end{IEEEkeywords}}

\maketitle

\IEEEdisplaynotcompsoctitleabstractindextext

\IEEEpeerreviewmaketitle



%\IEEEraisesectionheading{\section{Introduction}}
\section{Introduction}
\subfile{1introduction}


\section{Related Work}
\subfile{2relatedwork}


\section{ByteTrackV2}
\subfile{3ByteTrack}


\section{Datasets and Metrics}
\subfile{5datasets}


\section{Experiments}
\subfile{6experiments}

\section{Conclusion}
\subfile{7conclusion}

% \ifCLASSOPTIONcompsoc
%   \section*{Acknowledgments}
% \else
%   \section*{Acknowledgment}
% \fi




\ifCLASSOPTIONcaptionsoff
  \newpage
\fi


\bibliographystyle{IEEEtran}      % basic style, author-year
\bibliography{egbib}   % name your BibTeX data base

% \begin{IEEEbiography}[{\includegraphics[width=1in,height=1.25in,clip,keepaspectratio]{authors/yifu.jpg}}]{Yifu Zhang} received the M.S. degree from the School of Electronic Information and Communications, Huazhong University of Science and Technology, Wuhan, China, in 2022. He is currently a senior engineer at the Department of Computer Vision Technology (VIS), Baidu Inc., China. His research mainly focuses on computer vision, with specific interests in multi-object tracking. He has contributed several papers in top conferences and journals, such as ECCV, TPAMI, and IJCV. 
% \end{IEEEbiography}

% \vspace{11pt}

% \begin{IEEEbiography}[{\includegraphics[width=1in,height=1.25in,clip,keepaspectratio]{authors/xinggang_wang.jpeg}}] {Xinggang Wang (M’17)} received the B.S. and Ph.D. degrees in Electronics and Information Engineering from Huazhong University of Science and Technology (HUST), Wuhan, China, in 2009 and 2014, respectively. He is currently a Professor at the School of Electronic Information and Communications, HUST. His research interests include computer vision and machine learning. He serves as the Co-Editor-in-Chief of the Image and Vision Computing journal and an associate editor of the Pattern Recognition  journal.
% \end{IEEEbiography}

% \vspace{11pt}

% \begin{IEEEbiography}[{\includegraphics[width=1in,height=1.25in,clip]{authors/xiaoqing.jpg}}]{Xiaoqing Ye}
%  received the B.S. degree from Wuhan University, China, in 2014, and the Ph.D. degree in information and communication engineering from the University of Chinese Academy of Sciences, China, in 2019. She is currently with the department of computer vision technology, Baidu, as a Senior Engineer. Her research interests include 3D Vision and autonomous driving. She has contributed several papers in top conferences and journals, such as CVPR, ICCV, ECCV, NeurIPS, and IEEE TPAMI.
% \end{IEEEbiography}

% \vspace{11pt}

% \begin{IEEEbiography}[{\includegraphics[width=1in,height=1.25in,clip]{authors/wzhang.jpg}}]{Wei Zhang} received a Ph.D. degree from the University of Hong Kong in 2017. He is currently a senior R\&D Engineer in the Department of Computer Vision Technology at Baidu Inc. His research interests include 2D/3D object detection, segmentation, and tracking. He is dedicated to developing computer vision models for autonomous driving and intelligent transportation systems. He has contributed several papers to top conferences, such as CVPR, ECCV, AAAI, and TPAMI.
% \end{IEEEbiography}

% \vspace{11pt}

% \begin{IEEEbiography}[{\includegraphics[width=1in,height=1.25in,clip,keepaspectratio]{authors/ljc.jpg}}]{Jincheng Lu}
% received the B.E. degree from Shanghai Jiao Tong University, Shanghai, China, and the M.S. degree from The Chinese University of Hong Kong, Hong Kong. He is currently a senior engineer at the Department of Computer Vision Technology (VIS), Baidu Inc., China. His current research interests include computer vision, image processing, and deep learning.
% \end{IEEEbiography}

% \vspace{11pt}

% \begin{IEEEbiography}[{\includegraphics[width=1in,height=1.25in,clip]{authors/tan.png}}]{Xiao Tan}
%  received the Ph.D. degree in computer vision from the University of New South Wales, Sydney, in 2014. His research interests include computer vision, pattern recognition, and image processing. He is currently with the department of computer vision technology, Baidu as a Senior Engineer, leading the tech-team to develop visual systems for AI-city and autonomous driving. He has contributed 10+ papers in top conferences and journals, such as CVPR, ECCV, ICCV, and TIP.
% \end{IEEEbiography}

% \vspace{11pt}

% \begin{IEEEbiography}[{\includegraphics[width=1in,height=1.25in,clip,keepaspectratio]{authors/errui.JPG}}]{Errui Ding} received Ph.D degree from Xidian University in 2008 and currently is the director of Computer Vision Technology Department (VIS) of Baidu Inc. In recent years, he has published tens of papers on top-tier conferences and was awarded Best Paper Runner-up at ICDAR 2019. He co-organized several competitions and workshops at recent ICDAR and CVPR. He is also a member of CSIG, CCF and CAAI.
% \end{IEEEbiography}

% \vspace{11pt}

% \begin{IEEEbiography}[{\includegraphics[width=1in,height=1.25in,clip,keepaspectratio]{authors/peize.jpg}}]{Peize Sun} is currently a Ph.D. student with the Department of Computer Science, The University of Hong Kong(HKU). His research interests include deep learning, computer vision, especially in object detection, segmentation and tracking. He received his B.E. and M.E. degree from Xi'an Jiaotong University, China, in 2017 and 2020. 
% \end{IEEEbiography}

% \vspace{11pt}

% \begin{IEEEbiography}[{\includegraphics[width=1in,height=1.25in,clip,keepaspectratio]{authors/jingdong.jpg}}]{Jingdong Wang} is a Chief Scientist for Computer Vision with Baidu. His team is focusing on conducting product-driven and cutting-edge computer vision/deep learning/AI research and developing practical computer vision applications. Before joining Baidu, he was a Senior Principal Researcher at Microsoft Research Asia. His areas of interest are computer vision, deep learning, and multimedia search. His representative works include deep high-resolution network (HRNet), discriminative regional feature integration (DRFI) for supervised saliency detection, neighborhood graph search (NGS, SPTAG) for large scale similarity search. He has been serving/served as an Associate Editor of IEEE TPAMI, IJCV, IEEE TMM, and IEEE TCSVT, and an area chair of leading conferences in vision, multimedia, and AI, such as CVPR, ICCV, ECCV, ACM MM, IJCAI, and AAAI. He was elected as an ACM Distinguished Member, a Fellow of IAPR, and a Fellow of IEEE, for his contributions to visual content understanding and retrieval.
% \end{IEEEbiography}

\vfill

\end{document}



