\documentclass[english,aps,prb,superscriptaddress,reprint,longbibliograhy]{revtex4-2}
%\documentclass[english,aps,prb,showpacs,10pt]{revtex4-2}
\usepackage[T1]{fontenc}
\usepackage[latin9]{inputenc}
\usepackage{babel}
\usepackage{graphicx}
\usepackage{bm}
\usepackage{amsmath}
\usepackage{amssymb}
\usepackage{amsfonts}
\usepackage{dcolumn}
\usepackage{bbold}
\usepackage{tikz}
\usepackage{xfp}
%\usepackage{subfig}
%\usepackage{multicol, lipsum}
\usepackage{verbatim}

\usepackage[margin=15mm]{geometry}
\colorlet{linkequation}{blue}
\usepackage[colorlinks=true,linkcolor=cyan,citecolor=red,urlcolor=cyan]{hyperref}
\usetikzlibrary{shadings}

\newcommand{\ak}[1]{\textcolor{red}{#1}}

\usepackage{color}
%\usepackage{booktabs}
%\usepackage[x11names, svgnames, dvipsnames]{xcolor}
\newcommand{\blue}{\color{blue}}
\newcommand{\red}{\color{red}}
\newcommand{\green}{\color{green}}
\newcommand{\cyan}{\color{cyan}}
%\definecolor{Burgundy}{RGB}{144,0,32}

% new good colors
\definecolor{goodred}{RGB}{183,15,58}
\definecolor{goodblue}{RGB}{93,128,180}
\definecolor{goodgreen}{RGB}{95,118,33}

% color text
\usepackage{color}
\newcommand{\gblue}{\color{goodblue}}
\newcommand{\gred}{\color{goodred}}


%\newcommand{\minline}[2]{$\mbox{\fontsize{#1}{10}\selectfont #2}$}
%\newcommand{\nsum}[1][1.4]{% only for \displaystyle
 %   \mathop{%
   %     \raisebox
    %        {-#1\depthofsumsign+1\depthofsumsign}
   %         {\scalebox
   %             {#1}
     %           {$\displaystyle\sum$}%
        %    }
   % }
%}

\newcommand\SupplementaryMaterials{%
  \xdef\presupfigures{\arabic{figure}}% save the current figure number
  \xdef\presupsections{\arabic{section}}% save the current section number
  \renewcommand\thefigure{SM\fpeval{\arabic{figure}-\presupfigures}}
  \renewcommand\thesection{SM\fpeval{\arabic{section}-\presupsections}}
}

\begin{document}

%----------------------------------------------------------------------
\begin{abstract}
The dynamics of magnetic moments consists of a precession around the magnetic field direction and a relaxation towards the field to minimize the energy. While the magnetic moment and the angular momentum are conventionally assumed to be parallel to each other, at ultrafast time scales their directions become separated due to inertial effects. The inertial dynamics gives rise to additional high-frequency modes in the excitation spectrum of magnetic materials. 
%Classical spin dynamics has so far been described by the traditional Landau-Lifshitz-Gilbert (LLG) equation of motion, consisting of spin precession and energy dissipation in terms of the Gilbert damping. However, at ultrafast timescales additional magnetic inertial effects appear which cannot be covered by the LLG equation. One requires to go beyond LLG spin dynamics and add an extra spin torque term that gives rise to spin {inertia}. Due to such a term an additional resonance at higher frequency has been predicted theoretically in the frequency spectrum of ferromagnetic resonance (FMR). Very recently, experimental indications for such inertial dynamics have been observed in CoFeB and NiFe films at about 0.5 THz. Owing to this fundamental THz resonance property it is now possible to manipulate spins at the THz regime in ferromagnets. Such high-frequency manipulation of ferromagnetic spins might lead to new mechanisms of future technological interest.  
{ Here, we}  
review the recent theoretical and %groundbreaking
experimental advances in this emerging topic and discuss the open challenges and opportunities %that underpin 
in the detection and the potential applications of inertial spin dynamics.     
%The spin dynamics have so far been described by the traditional Landau-Lifshitz-Gilbert equation of motion, consisting of spin precession and an energy dissipation in terms of the Gilbert damping. However, an additional inertial magnetic dynamics appears at ultrafast timescales, which cannot be described by the LLG equation. Recently, such inertial dynamics have been observed in the experiments. Here, we briefly review the theoretical and groundbreaking experimental advances of such emerging topic and foresee the  open challenges and opportunities that underpin the potential of the inertial spin dynamics. 
\\

%{\gred Cherkasskii suggests deleting the text highlighted in red}

%{\gblue Cherkasskii suggests inserting blue text}

\end{abstract}
%----------------------------------------------------------------------

%\title{A perspective on the inertial spin dynamics}
\title{Inertial effects in ultrafast spin dynamics}

\author{Ritwik Mondal}
\email[]{ritwik@iitism.ac.in}


\affiliation{Department of Physics, Indian Institute of Technology (ISM) Dhanbad, IN-826004, Dhanbad, India}

%\affiliation{Department of Physics and Astronomy, Uppsala University, Box 516, Uppsala, SE-75120, Sweden}

\author{Levente R\'ozsa}
%\email[]{levente.rozsa@uni-konstanz.de}
 
\altaffiliation{Present address: Department of Theoretical Solid State Physics, Institute of Solid State Physics and Optics, Wigner Research Centre for Physics, H-1525 Budapest, Hungary}
\altaffiliation{Department of Theoretical Physics, Budapest University of Technology and Economics, H-1111 Budapest, Hungary}
\affiliation{Department of Physics, University of Konstanz, DE-78457 Konstanz, Germany}

\author{Michael Farle}
\affiliation{Faculty of Physics, University of Duisburg-Essen, DE-47048 Duisburg, Germany}


% \author{Igor Barsukov}
% \affiliation{Physics and Astronomy, University of California, Riverside, California 92521, United States}

\author{Peter M. Oppeneer}
\affiliation{Department of Physics and Astronomy, Uppsala University, Box 516, SE-75120 Uppsala, Sweden }

\author{Ulrich Nowak}
\affiliation{Department of Physics, University of Konstanz, DE-78457 Konstanz, Germany} 

\author{Mikhail Cherkasskii}
\email[]{macherkasskii@hotmail.com}
\altaffiliation{Present address: Institute for Theoretical Solid State Physics, RWTH Aachen University, DE-52074 Aachen, Germany}
\affiliation{Faculty of Physics, University of Duisburg-Essen, DE-47048 Duisburg, Germany}


\maketitle
\date{\today}

\section{Introduction}
The increasing challenge of processing and storing a %of the 
rapidly growing amount of digital information requires novel technological solutions operating at smaller length scales and at increased speed, yet in a more energy-efficient manner. %From this perspective, spin-based information handling is a possible route with potential for a major breakthrough. In the current magnetic data storage devices the magnetic bits i.e., spin up and down are used to encode binary logic ``0'' and ``1'' \cite{IBM1981}.    
While current magnetic devices enable data storage on short length scales with a low energy consumption, reading and rewriting the bits using magnetic field pulses~\cite{SIEGMANN1995L8} is not possible below the nanosecond time scale. %Such limitation in timescales is restricted by the magnetization precession frequency in ferromagnets i.e., gigahertz. 

%To alter the magnetic bits, the conventional way is to apply a magnetic field in the opposite direction of the spin arrangement \cite{SIEGMANN1995L8}. The fastest magnetic reversal can be obtained by the precession of spin angular momentum with the frequency proportional to the magnitude of the applied magnetic field.    

%The crucial question is: how can the spins be manipulated efficiently on time scales shorter than a nanosecond? 
To manipulate the spins on shorter time scales, electrical currents and ultrafast optical laser pulses have been employed. These methods enable ultrafast demagnetization within femtoseconds~\cite{Bigot1996} and magnetization switching within picoseconds in a broad variety of magnetic materials~\cite{Stanciu2007,Radu2011,ostler12,LeGuyader2015,Mangin2014,KOPLAK2017}. Many aspects of ultrafast demagnetization and switching can be successfully described either phenomenologically~\cite{Vahaplar2009,Mentink2012}, or microscopically based on the Landau--Lifshitz--Gilbert (LLG) equation~\cite{landau35,Gilbert2004} in its stochastic form~\cite{brown1963micromagnetics,Barker2013,Wienholdt2013}. While the latter approach is widely applied to modelling magnetization dynamics in the presence of thermal fluctuations, it relies on the crucial assumption that the spin degrees of freedom are coupled to a heat bath responsible for the dissipation as well as the thermal noise, while details of the considerably faster electronic and lattice degrees of freedom constituting the heat bath are neglected~\cite{brown1963micromagnetics,Antropov1996}. Recent derivations of the LLG equation based on a relativistic theory~\cite{Mondal2017Nutation,Mondal2018JPCM} have proven that this approximation is no longer justified if the spin directions significantly vary over the course of femtoseconds. %While ultrafast demagnetization and switching can be successfully described within a phenomenological framework~\cite{Vahaplar2009,Mentink2012,Barker2013}, a {\gred complete} microscopic theoretical understanding of these processes is lacking. One of the reasons for this is that the Landau--Lifshitz--Gilbert (LLG) equation~\cite{landau35,Gilbert2004} is widely {\gred applied} to modelling magnetization dynamics, but it relies on the crucial assumption that the spin degrees of freedom are adiabatically decoupled from the considerably faster electronic and lattice motion~\cite{brown1963micromagnetics,Antropov1996}. Such an approximation is no longer justified if the spin directions considerably vary over the course of femtoseconds.
%A pioneering work of laser-induced ultrafast demagnetization in ferromagnetic Ni showed that the spins can even be controlled at femtosecond (fs) timescales \cite{Bigot1996}. This unexpected but remarkable experimental achievement was  followed by several other theoretical and experimental investigations of femtosecond manipulation of magnetism. Later, it was realised that the ultrafast magnetic %bits
%switching can be achieved in ferrimagnetic GdFeCo at picoseconds timescale \cite{Stanciu2007,Radu2011,ostler12,LeGuyader2015}, {and also in} a broad variety of magnetic materials including multi-layers and heterostructures \cite{Mangin2014,KOPLAK2017}. While the ultrafast ferrimagnetic switching mechanism is mediated via the transfer of angular momentum from one sublattice to the other, the ferromagnetic switching is governed by the spin precession at the resonance frequency i.e., GHz. Such precessional switching sets a speed limit in ferromagnets.  

At ultrashort time scales, the LLG equation has to be corrected by accounting for the fact that the magnetization direction can no longer instantaneously follow the angular momentum. This delay can be described by appending an inertial term including the second time derivative of the magnetization to the LLG equation~\cite{Suhl1998,wegrowe2000thermokinetic,Ciornei2011,Wegrowe2012,giordano2020derivation,Jingwen2022}. This phenomenological consideration is supported by various derivations of the inertial term based on microscopic relativistic quantum theories~\cite{Fahnle2011,Bhattacharjee2012,Mondal2017Nutation,Mondal2018JPCM}. There are numerous theoretical predictions on how the signatures of inertial dynamics can be detected, but experimental observations are limited so far. Most likely this can be attributed to the fact that conventional magnetic measurements focus on the low-frequency regime, typically on the GHz range in ferromagnets, where the inertia plays little role and its effects may alternatively be explained based on the conventional LLG equation. However, the magnetic moments not only experience precession around the effective field in the presence of the inertial term, but they also perform a high-frequency nutation around the angular momentum, see Fig.\,\ref{fig:Fig1}. Hence, the nutation gives rise to an additional peak in the ferromagnetic resonance spectrum in the high-frequency regime~\cite{Olive2012}. %, {\gred and nutational spin waves~\cite{Kikuchi}.}
This resonance is typically found in the THz range in contrast to the conventional precession resonance at GHz frequencies. The most convincing experimental signatures of inertial dynamics to date are based on the observation of this high-frequency response in NiFe, CoFeB~\cite{neeraj2019experimental} and Co~\cite{unikandanunni2021inertial} films. {{Propagating nutational spin waves have also been predicted to possess frequencies in the THz regime~\cite{Kikuchi}, but have not been observed experimentally so far.}} %However, a recent experimental discovery of magnetic inertial dynamics suggests that the ferromagnetic spins can even be manipulated at THz resonance frequencies, meaning picosecond timescales \cite{neeraj2019experimental}. In fact one finds that the ferromagnetic resonance spectrum not only consists of the precession resonance peak at GHz frequency, but also an additional nutational resonance peak at THz frequency. Although the amplitude of the additional resonance is smaller than the GHz precession resonance, it cannot be neglected in the evaluation of high-frequency spin dynamics. Direct evidence of spin inertia was found with the pump-probe technique in NiFe, CoFeB and Co films \cite{neeraj2019experimental, unikandanunni2021inertial}. In antiferromagnetic HoFeO, inertial spin behaviour is discovered at switching processes and considered as a phenomenon caused by the exchange interaction \cite{kimel2009inertia}.
%{\gred Such a faster manipulation of spins questions our current understanding of spin dynamics. It has been suggested that the spin dynamics of this kind cannot be explained by the traditional Landau-Lifshitz-Gilbert (LLG) equation of motion anymore. An additional spin torque term involving the second-order time-derivative of magnetization is necessary to explain such high frequency spin dynamics \cite{Suhl1998}}
%{Such a fast spin dynamics questions our understanding relying on the traditional Landau-Lifshitz-Gilbert (LLG) equation. It was suggested that an inertial term including the second-order time-derivative of magnetization has to be appended to this equation \cite{Suhl1998}}. Nevertheless, such picosecond control of spins might offer considerably faster future technological solutions.

%Theoretical and experimental investigations  have been put forward to understand and predict the spin dynamics in the high-frequency regime. 
%{\gblue XXX Add papers here XXX}. 
%The consensus among all of them is that the magnetic inertial dynamics leads to the high-frequency nutation of the spins. When the spins are in equilibrium with the effective field, they align parallel to the field. An external drive creates a nonequilibrium situation where the spins not only experience the precession around the field, but also they nutate at the ultrafast timescales (Fig.\,\ref{fig:Fig1}). 
%The spin dynamics is equivalent to the classical rigid-body dynamics where the precession, damping and nutation dynamics are also realized.      

In this review, we first describe the inertial LLG equation by motivating the precession, damping and inertial terms. We discuss the consequences of inertial dynamics on resonance spectra, on the spin-wave dispersion and on switching processes not only in ferromagnets, but also in antiferromagnets and ferrimagnets. We also outline the challenges and opportunities concerning the experimental observation of inertial spin dynamics, paving the way towards a microscopic understanding and possible technological applications of the evolution of magnetic moments on ultrafast time scales.

%In this perspective, we first describe traditional LLG spin dynamics involving the spin precession and energy dissipation in terms of damping. Further we discuss the theoretical and experimental advances in high frequency spin dynamics. We also outline the unexplored territories of inertial dynamics not only in ferromagnets, but also in multi-sublattice magnets such as antiferromagnets and ferrimagnets. Even though there are several publications already reinforcing the inertial nature of the dynamics, a complete theoretical and experimental understanding is far from being achieved in the field.

\section{Magnetization dynamics}

Here, we summarize the main {{points}} of LLG dynamics, and point out in which aspects it has to be modified at ultrashort time scales, culminating in the formulation of the inertial LLG equation.

\subsection{Precession and damping dynamics}
{When a %classical magnet with 
magnetic moment {$\boldsymbol{ M}_{0}$} is placed in an external magnetic field $\boldsymbol{ B}$, the corresponding energy is {$\mathcal{H}=-\boldsymbol{ M}_{0}\cdot \boldsymbol{ B}$}}.
%{\gblue When a classical magnetic moment having magnetization $\boldsymbol{ M}$ is placed in an external magnetic field $\boldsymbol{ B}$, it gains Zeeman energy equal to $-\boldsymbol{ M}\cdot \boldsymbol{ B}$.}.
The energy is minimized when the direction of the magnetic moment is parallel to the direction of the magnetic field. %On the other hand, the magnetization is proportional to the angular momentum $\boldsymbol{L}$ via the relation $\boldsymbol{M} = \gamma \boldsymbol{L}$, where $\gamma$ is the gyromagnetic ratio.
In classical electrodynamics, the magnetic moment is represented by a charged particle moving along a closed curve, establishing a relation between its angular momentum {$\boldsymbol{L}_{0}$} and magnetic moment {$\boldsymbol{M}_{0}$} via the relation {$\boldsymbol{M}_{0} = \gamma \boldsymbol{L}_{0}$}, where $\gamma$ is the gyromagnetic ratio. %The torque acting on the magnetization is equal to the rate of change of angular momentum, obtaining the precessional motion of magnetization \cite{blundell01}
The rate of change of angular momentum is equal to the torque, leading to the precessional motion of the magnetic moment~\cite{blundell01}
\begin{align}
    \dot{\boldsymbol{M}}_{0} = -\gamma \boldsymbol{M}_{0}\times \boldsymbol{B}\,,
    \label{Eq1}
\end{align}
for electrons with a negative charge $-e$ and mass $m$. 
% where the time-derivative of the magnetization is defined as $\partial \bf{M}/\partial t = \dot\boldsymbol{{M}}$. 
Note that the magnitude of the magnetic moment $M_{0}$ %\vert \boldsymbol{ M}_{0}\vert$}} 
remains constant. %independent of motion. 
An identical equation of motion may be derived by treating the moment quantum-mechanically. The only difference is in the value of the gyromagnetic ratio $\gamma=ge/\left(2m\right)$, where the gyromagnetic factor is $g=1$ for classical particles and is close to $g\approx 2$ for electrons in a solid where the quantum-mechanical spin angular momentum is the dominant contribution to the magnetic moment. The value $\gamma=1.76\cdot 10^{11}$\,T$^{-1}$s$^{-1}$ for $g=2$ sets the %typical 
characteristic frequencies of magnetic moment dynamics in the gigahertz range for typically achievable magnetic field values of a few Tesla.
%From experimental observation of hysteresis curve, it is well known that the magnetization saturates beyond a critical value of applied magnetic field. The latter suggests that although the magnetization precesses around the field, it should become parallel to the field at saturation \cite{lakshmanan11}.

It is known that the magnetic moment of ferromagnets does not only precess around the field, but also minimizes its energy by becoming parallel to it within microscopic time scales. To incorporate this experimental fact, adding a phenomenological damping term to the equation of motion was suggested by Landau and Lifshitz~\cite{landau35}. An alternative formulation of the damped equation of motion was proposed by Gilbert~\cite{Gilbert2004}%\cite{gilbert56}
, setting an upper bound on the damping coefficient in the Landau--Lifshitz formalism to better accommodate experimental observations. 
For an ensemble of interacting magnetic moments, the magnetization dynamics can be described by the Landau--Lifshitz--Gilbert (LLG) equation of motion %of following kind 
\cite{landau35,Gilbert2004},
\begin{align}
    \dot{\boldsymbol{M}}_{ i}(t) = - \gamma_{ i} \boldsymbol{ M}_{ i} \times \boldsymbol{ B}^{\rm eff}_{ i} + \frac{\alpha_{ i}}{M_{ \textrm{0},i}}  \boldsymbol{ M}_{ i} \times \dot{\boldsymbol{ M}}_{ i} \,,  \label{Eq2}
\end{align}
where ${ i}$ stands for the indices of the magnetic moments and $M_{\textrm{0},i}$ are the magnitudes of the moments, which are still conserved during the time evolution. % of the equilibrium magnetization. 
$\alpha_{ i}$ are the Gilbert damping parameters that phenomenologically describe the energy dissipation in terms of the coupling of the magnetic moments to the considerably faster degrees of freedom. % of electrons and phonons. 
The typical frequency scale of the dissipation is given by $\alpha_{i}\gamma_{i}/\left(1+\alpha_{i}^{2}\right)B$, which is usually much slower than the precession dynamics for common values of $\alpha_{i}\sim 10^{-5}-10^{-2}$. Equation~\eqref{Eq2} can be readily rewritten for a continuous magnetization field $\boldsymbol{M}(\boldsymbol{r})$ with saturation value $M_{ \textrm{S}}$, as originally proposed by Landau and Lifshitz~\cite{landau35}. The effective field $\boldsymbol{ B}^{\rm eff}_{ i}$ can be calculated from the Hamiltonian $\mathcal{H}$ of a magnetic system following the definition $\boldsymbol{ B}^{\rm eff}_{ i} = -\partial\mathcal{H}/\partial \boldsymbol{ M}_{ i}$ in the discrete case, replaced by the free energy $\mathcal{F}$ and $\boldsymbol{ B}^{\rm eff}_{ i} = -\delta\mathcal{F}/\delta \boldsymbol{ M}$ in the continuum limit. The Hamiltonian contains interactions of the magnetic moments with the external field through the Zeeman term, with the atomic lattice through magnetocrystalline anisotropy terms, and with each other in the form of dipolar and exchange interactions.


%{\gblue The LLG equation \ (\ref{Eq2}) can be reshaped as
%\begin{align}
%   (1+\alpha_i^2) \dot\boldsymbol{ M}_i(t) & = -\gamma_i \boldsymbol{ M}_{ i} \times \boldsymbol{ B}^{\rm eff}_{ i}\nonumber\\
%   &- \frac{\gamma_i \alpha_i}{M_{S,i}} \boldsymbol{ M}_{ i} \times \left(\boldsymbol{ M}_{ i} \times \boldsymbol{ B}^{\rm eff}_{ i}\right),
 %  \label{Eq3_LL}
%\end{align}
%This equation is known as the Landau-Lifshitz (LL) equation. Both eqns. (\ref{Eq2}) and (\ref{Eq3_LL}) are preserve the length of magnetization \cite{lakshmanan11}.}

%Although the Gilbert damping parameter was introduced as a scalar quantity, it was later found to be a tensor \cite{Brataas2008,Thonig_2014,Mondal2016,Thonig2018,Mondal2018PRB}. The effective field $\boldsymbol{ B}^{\rm eff}_{ i}$ can be calculated from the free energy $\mathcal{F}$ of a magnetic system following the definition $\boldsymbol{ B}^{\rm eff}_{ i} = -\delta\mathcal{F}/\delta \boldsymbol{ M}_{ i}$. The two terms in the LLG spin dynamics are referred to as the field torque and the damping torque, respectively. In equilibrium, the magnetization remains parallel to $\boldsymbol{ B}^{\rm eff}$. Out of equilibrium, a precessional dynamics is induced, followed by a transverse damping. The typical timescale of the precessional motion is given by $\vert \gamma \boldsymbol{ B}^{\rm eff} \vert^{-1}$ which belongs to nanoseconds in ferromagnets. The free energy $\mathcal{F}$ is rather specific to the magnetic system  which can have many contributions e.g., Zeeman energy due to an external magnetic field, magnetic anisotropy energy, magnetic exchange interactions, etc.

Though the original LLG equation is based on a phenomenological description~\cite{landau35,Gilbert2004}, several theories on the microscopic origins of the Gilbert damping have been put forward. In particular, the Gilbert damping has been proposed to originate from the breathing Fermi surface model \cite{kambersky70}, the torque-torque correlation model \cite{kambersky76,kambersky07,Thonig2018}, scattering theory formalism \cite{Brataas2008}, linear-response theory \cite{EbertPRL2011}, and relativistic Dirac theory \cite{Mondal2016,Mondal2018PRB}. The damping coefficient has also been generalized to a tensor~\cite{Brataas2008,Thonig_2014,Mondal2016,Thonig2018,Mondal2018PRB}, which is responsible for anisotropic damping observed in experiments~\cite{platow1998correlations, Farle2013}. Since the magnetic moment primarily stems from the spin angular momentum while the damping describes coupling to the lattice degrees of freedom, a common point of these microscopic theories is that the damping originates from the spin--orbit coupling. %We mention that a different form of spin relaxation has been predicted that accounts for the spatial dispersion of the magnetic exchange interaction \cite{baryakhtar84,baryakhtar13r}.     

% {The LLG equation of motion in Eq.~\eqref{Eq2} can be explicitly rewritten by taking a vector %cross 
% product of $\boldsymbol{ M}_i$ on both sides. The LLG equation thus takes
% the following form:
% \begin{align}
% %    \boldsymbol{ M}_i \times \dot\boldsymbol{ M}_i & = -\gamma_i \boldsymbol{ M}_{ i} \times \left(\boldsymbol{ M}_{ i} \times \boldsymbol{ B}^{\rm eff}_{ i}\right) + \frac{\alpha_{ i}}{M_{ S,i}}  \boldsymbol{ M}_{ i} \times \left(\boldsymbol{ M}_{ i} \times \dot\boldsymbol{ M}_{ i}\right)\nonumber\\
% %    & = -\gamma_i \boldsymbol{ M}_{ i} \times \left(\boldsymbol{ M}_{ i} \times \boldsymbol{ B}^{\rm eff}_{ i}\right) - \alpha_i M_{ S,i}   \dot\boldsymbol{ M}_{ i}\nonumber\\
% %  \frac{\alpha_i}{M_{S,i}}\boldsymbol{ M}_i \times \dot\boldsymbol{ M}_i  & = -\frac{\gamma_i \alpha_i}{M_{S,i}} \boldsymbol{ M}_{ i} \times \left(\boldsymbol{ M}_{ i} \times \boldsymbol{ B}^{\rm eff}_{ i}\right) - \alpha_i^2   \dot\boldsymbol{ M}_{ i} \\
%    (1+\alpha_i^2) \dot{\boldsymbol{ M}}_i(t) & = -\gamma_i \boldsymbol{ M}_{ i} \times \boldsymbol{ B}^{\rm eff}_{ i}\nonumber\\
%    &- \frac{\gamma_i \alpha_i}{M_{S,i}} \boldsymbol{ M}_{ i} \times \left(\boldsymbol{ M}_{ i} \times \boldsymbol{ B}^{\rm eff}_{ i}\right)
%    \label{Eq3}
% \end{align}
% This equation is known as the Landau-Lifshitz (LL) equation of motion. Equations (\ref{Eq2}) and (\ref{Eq3}) are mathematically equivalent, however, only the LL form i.e., Eq.\ (\ref{Eq3}) is explicit. An important property is that both of these equations preserve the length of the magnetization \cite{lakshmanan11}.}

% The LLG equation~\eqref{Eq2} can be employed to investigate spin dynamics at zero temperature %~\cite{nowak2007handbook}. To include finite-temperature effects, Brown proposed~\cite{Brown_1963} that a thermal noise term should be added to the effective field such that $\boldsymbol{ B}^{\rm eff}_{ i} = -\frac{\partial \mathcal{H}}{\partial{ M}_{ i}} + \boldsymbol{\zeta}_{ i}(t)$, turning it into a stochastic differential equation. This approach was generalized to interacting spin systems later~\cite{Lyberatos1993,CHUBYKALO2003}. Assuming that the system follows the Boltzmann distribution in thermal equilibrium, this noise term has the following properties:

{The LLG equation~\eqref{Eq2}, which describes the dynamics of the mean value of the magnetization, can be augmented to incorporate the effects of thermal fluctuations. Brown proposed~\cite{Brown_1963,coffey2004langevin} %(see Ref.~\cite[p.~128]{coffey2004langevin} and~\cite{Brown_1963}) 
that a thermal noise term should be added to the effective field such that $\boldsymbol{ B}^{\rm eff}_{ i} = -\frac{\partial \mathcal{H}}{\partial{ M}_{ i}} + \boldsymbol{\zeta}_{ i}(t)$, turning it into a stochastic differential equation.} This approach was generalized to interacting spin systems later~\cite{Lyberatos1993,CHUBYKALO2003}. Assuming that the system follows the Boltzmann distribution in thermal equilibrium, this noise term has the following properties: 
\begin{align}
    \langle \zeta_{i\eta}(t)\rangle & = 0\label{Eq3}\\
    \langle \zeta_{i\eta}(t) \zeta_{j\theta}(t^\prime) \rangle & = \delta_{ij}\delta_{\eta\theta} \delta(t-t^\prime) \frac{2\alpha_{i} k_{\textrm{B}} T }{\gamma_{i} M_{0,i}}
    \label{Eq4}
\end{align}
where $\eta$ and $\theta$ denote Cartesian components, %the indices $i$ and $j$ denote lattice sites. 
$k_{\textrm{B}}$ is the Boltzmann constant and $T$ denotes the temperature of the system. This corresponds to white noise with zero expectation value, which is uncorrelated in space, time and Cartesian components. Similarly to the damping term, the noise describes coupling to the faster %lattice 
electronic degrees of freedom, which can be considered to be uncorrelated at the time scale of the spin dynamics. Regarding the phononic degrees of freedom, the separation of time scales is less straightforward, and the microscopic description of the coupling between spins and phonons is a subject of current research~\cite{Tauchert2022}. The connection between dissipation and fluctuations is also expressed by the Einstein relation~\eqref{Eq4}. An alternative form of the stochastic LLG equation was proposed by Kubo and Hashitsume~\cite{Kubo1970}, primarily differing in the scaling of the parameters from Brown's formulation, similarly to the Landau--Lifshitz and Gilbert forms of the LLG equation. If it is assumed that the heat bath consisting of phonons and electrons evolves at faster time scales than the spin system, including a white noise in the equation is justified. %Inclusion of the white noise is based on the assumption that the heat bath consisting of phonons and electrons evolves at faster timescales than the spin system. %In this regard, the electronic and phononic degrees of freedom can be averaged out and a stochastic field with white noise correlation function is used instead. 
However, such a separation and averaging out becomes invalid for femtosecond magnetization dynamics because the electron relaxation time in metals is on the order of 10 fs~\cite{Stohr2006}. Using a stochastic field with a coloured noise may be more accurate in such cases~\cite{Atxitia2009}.        

%The LLG equation has been very useful in understanding the spin dynamics at {shorter}  timescales. 
%The magnetic response (susceptibility) via the simulation of LLG equation provides a peak -- known as FMR resonance frequency which has been observed in the experiments. Moreover, the  line-width of the FMR resonance peak is related to the Gilbert damping $\alpha$ \cite{Usadel2005,Yokoyama2010,Dreher2012,Zhao2016,Celinski1991,LAYADI1990,LAYADI1997}. The LLG equation has also been employed to model the magnetization dynamics at ultrafast timescales in ferromagnets \cite{Bigot1996}, ferrimagnets \cite{ostler12,Radu2011} and antiferromagnets \cite{Kampfrath2011,Baierl2016,Sonke2012}.   

%Apart from the LLG spin dynamics, several other spin torques can be dominant in magnetization dynamics. These spin torques include the current-induced spin transfer torque \cite{slonczewski96,Berger1996,Zhang2002PRL,Li2003STT}, the spin-orbit torque \cite{Zhang2002,Brataas2014,Demidov2020}, the field-derivative torque \cite{Mondal2019PRB,Blank2021THz}, and the optical spin-orbit torque \cite{tesarova13,mondal2021terahertz,Choi2020}. %{\gred Some sentences about LLB spin dynamics as well?}

%{\gred \bf RM + LR + MC}

\subsection{Inertial dynamics}

As emphasized above, both the dissipation and the thermal noise term in the stochastic LLG equation were introduced under the assumption that the relatively slow motion of the magnetic moments
%may be adiabatically decoupled from the dynamics of electrons and phonons, and the magnetic subsystem
is only influenced by an average of the other degrees of freedom. At shorter time scales, additional effects have to be included in the equation of motion. First, describing the evolution of the magnetic moments on time intervals comparable to the time between electron and phonon scattering events requires going beyond the instantaneous values of the magnetic moments in the LLG equations by including memory effects~\cite{Suhl1998,Fahnle2011}. %in classical electrodynamics retardation effects become prominent as the velocity of the charge carriers is increased, i.e., the interaction between the particles is mediated by the electromagnetic field which requires time for propagation. Similar memory effects are also expected for the magnetic moment dynamics~\cite{Suhl1998,Fahnle2011}. 
Second, it has been already pointed out by Gilbert~\cite{Gilbert2004} that although precession also exists in classical mechanics, the correspondence between the dynamics of a magnetic moment and a spinning top is incomplete since the former does not possess a physical inertial tensor when described by the LLG equation. Third, at these time scales the excitation energies of the magnetic moments become comparable to those of electronic excitations, requiring a common quantum treatment of the degrees of freedom.

\subsubsection{Classical theory}

\begin{figure}[tbh!]
    \centering
    \includegraphics[scale = 0.9]{NutSchematic.png}
    \caption{Schematic diagram of ILLG spin dynamics displaying the torques responsible for spin precession (purple), damping (green) and nutation (blue)%including spin precession, damping and nutation
    ; from Ref.~\cite{Cherkasskii2022Anisotropy}.  %have been shown.   %{\gred We must mention this figure in text} 
    }
    \label{fig:Fig1}
\end{figure}

While quantum electrodynamics provides an accurate description of the motion at high energy scales, here we discuss time scales ranging from 1~fs to 1~ps where a quasiclassical description remains valid. Both memory effects and the problem of the inertial tensor of magnetic moments may be treated by adding a second time derivative to Eq.~\eqref{Eq2}, resulting in the inertial LLG (ILLG) equation
\begin{equation} \label{Eq5}
\begin{split}
    \dot{\boldsymbol{ M}}_{ i}(t) &= - \gamma_{ i} \boldsymbol{ M}_{ i} \times \boldsymbol{ B}^{\rm eff}_{ i} + \frac{\alpha_{ i}}{M_{ 0,i}}  \boldsymbol{ M}_{ i} \times \dot{\boldsymbol{ M}}_{ i} \\
    &+ \frac{\eta_{ i}}{M_{ 0,i}}  \boldsymbol{ M}_{ i} \times \ddot{\boldsymbol{ M}}_{ i} \,,  
\end{split}
\end{equation}
Here, $\eta_{i}$ is the inertial relaxation time~\cite{neeraj2019experimental}, and the $\boldsymbol{ M}_{ i} \times \ddot{\boldsymbol{ M}}_{ i}$ form of the last term ensures the conservation of the length of the magnetic moments.

Memory effects can be fully treated by transforming the LLG equation into an integro-differential equation, as was derived in Ref.~\cite{Suhl1998} for spin-lattice and in Refs.~\cite{Fahnle2011,fahnle2013erratum} for spin-electron coupling. These types of equations are difficult to treat even numerically and require an expansion of the time integral, which leads to the damping and inertial terms containing first and second time derivatives, respectively. Third-order time derivatives were also included in Ref.~\cite{Suhl1998}, although it was emphasized that this form of the equation is not applicable at high frequencies. Indeed, higher-order derivatives are expected to lead to causality breaking, as is also known from the example of the Abraham--Lorentz force in electrodynamics. %The appearance of the second time derivative of the magnetic moment is analogous to the presence of the acceleration of charged particles in the electromagnetic field derived from Li\'{e}nard-Wiechert potentials. Truncating the expansion at the second derivative is recommended, since higher-order derivatives in the equation of motion are expected to lead to causality breaking, as is also known from the electrodynamics in the example of the Abraham--Lorentz force.

{{An alternative approach to derive Eq.~\eqref{Eq5} is based on the mechanical analogy with rigid-body motion, as is explained in detail in Refs.~\cite{Ciornei2011,Wegrowe2012}. The magnetic moment is pictured as a symmetric top, where $\boldsymbol{M}_{i}/M_{0,i}$ describes the direction of the axis of the top, which is now allowed to deviate from the direction of the angular momentum $\boldsymbol{ L}_{i}$. In the rotating frame where the axis of the top is fixed along the $\hat{\boldsymbol{z}}$ direction, the inertial tensor reads }}
\begin{align}
    \overline{\overline{I_{i}}} & = \begin{pmatrix}
    I_{i,1} & 0 & 0\\
    0 & I_{i,1} & 0\\
    0 & 0 & I_{i,3}
    \end{pmatrix}\, ,
\end{align}
and the connection between the angular momentum $\boldsymbol{ L}_{i} = (L_{i,1},L_{i,2},L_{i,3})$ and the angular velocity $\boldsymbol{ \Omega}_{i} = (\Omega_{i,1},\Omega_{i,2},\Omega_{i,3})$ is given by
$\boldsymbol{ L}_{i} = - \overline{\overline{I_{i}}} \boldsymbol{ \Omega_{i}}$, where the negative sign is introduced to follow the sign convention for $\gamma_{i}$ used here. The direction of $\boldsymbol{M}_{i}$ follows the time evolution
\begin{align}
    \dot{\boldsymbol{ M}_{i}} & = \boldsymbol{ \Omega}_{i} \times \boldsymbol{ M}_{i} \,,
\end{align}
by rotating with angular velocity $\boldsymbol{\Omega}_{i}$.
Taking a cross product on both sides with $\boldsymbol{ M}_{i} = M_{0,i}\hat{\boldsymbol{ z}}$, utilizing the double cross product $\boldsymbol{ M}_{i}\times \left(\boldsymbol{ \Omega}_{i}\times \boldsymbol{ M}_{i}\right) = \boldsymbol{ \Omega}_{i} M_{0,i}^2 - \boldsymbol{ M}_{i} \left(\boldsymbol{ \Omega}_{i} \cdot \boldsymbol{ M}_{i}\right) = \boldsymbol{ \Omega}_{i} M_{0,i}^2 - M_{0,i}^2\Omega_{i,3} \hat{\boldsymbol{ z}} $ and multiplying by $-\overline{\overline{I_{i}}}$, one obtains
\begin{align}
    \boldsymbol{ L}_{i} & = \frac{1}{\gamma_{i}}\boldsymbol{ M}_{i}-\frac{\eta_{i}}{\gamma_{i}M_{0,i}}\boldsymbol{ M}_{i}\times \dot{\boldsymbol{ M}}_{i}\, , \label{Angular_momentum}
\end{align}
where the notations $M_{0,i}/\gamma_{i}=-I_{i,3}\Omega_{i,3}$ and $\eta_{i}/\gamma_{i}=I_{i,1}/M_{0,i}$ were introduced. The time evolution of the angular momentum is governed by the precession and damping torques known from the LLG equation,
\begin{align}
    \dot{\boldsymbol{L}}_{ i}(t) = - \boldsymbol{ M}_{ i} \times \boldsymbol{ B}^{\rm eff}_{ i} + \frac{\alpha_{ i}}{\gamma_{i}M_{ 0,i}}  \boldsymbol{ M}_{ i} \times \dot{\boldsymbol{ M}}_{ i} \,.  \label{EqL}
\end{align}
Substituting Eq.~\eqref{Angular_momentum} into Eq.~\eqref{EqL} yields the ILLG equation \eqref{Eq5}.

In Gilbert's derivation of the LLG equation, $I_{i,1}$ was set to zero while $I_{i,3}$ was finite, which cannot occur for any mechanical rigid body~\cite{Gilbert2004}. For a finite $I_{i,1}$ or $\eta_{i}$, the angular momentum and the axis of the spinning top identified with the magnetic moment direction are no longer parallel to each other, and $\boldsymbol{M}_{i}$ performs a fast nutation around $\boldsymbol{L}_{i}$. {It is interesting to note that these fast and slow degrees of freedom are well separated \cite{Wegrowe2016JPCM}.} A schematic diagram of the ILLG equation is shown in Fig.~\ref{fig:Fig1}, displaying spin precession, relaxation and nutation. %{\gred The peculiarity of the ILLG equation is that it conserves both angular and magnetic momentum, which may not be parallel to each other. The damping term also preserves the angular momentum while pushing it towards its equilibrium orientation. Therefore, this model can be called mechanical.}

The ratio $\eta_{i}/\gamma_{i}$ stemming from the moment of inertia $I_{i,1}$ must necessarily be positive, which supports the interpretation of $\eta_{i}$ as an inertial relaxation time. %The inertial relaxation time 
This coefficient enables the introduction of the kinetic energy term
\begin{align}
\mathcal{T}=\sum_{i}\frac{\eta_{i}}{2\gamma_{i}M_{0,i}}\dot{\boldsymbol{M}}_{i}^{2}\, ,\label{eqT}
\end{align}
the lack of which was also pointed out by Gilbert for $I_{i,1}=0$. Taking the cross product of Eq.~\eqref{Eq5} with $\boldsymbol{M}_{i}$, then the scalar product with $\dot{\boldsymbol{M}}_{i}$ results in 
\begin{align}
\dot{\mathcal{T}}+\dot{\mathcal{H}}+\sum_{i}\frac{\alpha_{i}}{\gamma_{i}M_{0,i}}\dot{\boldsymbol{M}}_{i}^{2}=0\, ,\label{eqE}
\end{align}
describing the conservation of the total energy $\mathcal{T}+\mathcal{H}$ in the absence of damping~\cite{Mondal2020nutation}. The difference $\mathcal{T}-\mathcal{H}$ corresponds to the Lagrangian~\cite{Wegrowe2012}. 
%The sum of the kinetic $\mathcal{T}$ and potential $\mathcal{H}$ energies is conserved by the ILLG equation~\cite{Mondal2020nutation}, while $\mathcal{T}-\mathcal{H}$ corresponds to the Lagrangian~\cite{Wegrowe2012}.

Inertial magnetization dynamics of ferromagnetic nanoparticles including thermal excitations was investigated in Ref.~\cite{Titov2021Inertial}. It was found that adding the thermal noise term $\zeta_{i}$ with the moments given by Eqs.~\eqref{Eq3} and \eqref{Eq4} to the effective field can correctly account for the thermal fluctuations within the ILLG equation as well. The equilibrium Boltzmann distribution is defined by the sum of the kinetic and potential energies $\mathcal{T}+\mathcal{H}$ in this case, instead of only the potential energy for the stochastic LLG equation. The shorter time scales of inertial dynamics support the arguments in favour of replacing the white noise with a coloured noise~\cite{Atxitia2009}, which has not been considered in the ILLG formalism so far.

\subsubsection{Microscopic theory}

On a microscopic level, the ILLG equation has been derived from an extension of the breathing Fermi surface model~\cite{Fahnle2011,fahnle2013erratum}, from the torque-torque correlation model~\cite{Thonig2017}, as well as in atomistic~\cite{Bhattacharjee2012} and in Dirac relativistic quantum~\cite{Mondal2017Nutation,Mondal2018JPCM} frameworks. The latter approach is based on the derivation of a Pauli--Schr\"{o}dinger Hamilton operator from the Dirac equation,
\begin{align}
    \mathcal{H}_{\rm FW} & = \frac{\left(\boldsymbol{ p}-e\boldsymbol{ A}\right)^2}{2m} + V  - \frac{e\hbar}{2m}\, \boldsymbol{ \sigma}\cdot \boldsymbol{ B}\nonumber\\
    & + \mathcal{O}\left(\frac{1}{m^2c^2}\right) + \mathcal{O}\left(\frac{1}{m^3c^4}\right) + \dots 
    \label{FW_Hamiltonian}
\end{align}
by applying the Foldy-Wouthuysen transformation~\cite{foldy50}. The Zeeman term $e\hbar/\left(2m\right) \boldsymbol{ \sigma}\cdot \boldsymbol{ B}$ is responsible for the precession, where $\boldsymbol{ \sigma}$ denotes the vector of Pauli matrices. The spin-dependent part of the first-order relativistic correction term $\mathcal{O}\left(\frac{1}{m^2c^2}\right)$ results in the Gilbert damping, which contributes to the imaginary part of the magnetic susceptibility or the finite lifetime of excitations. The spin-dependent part of the second-order relativistic correction $\mathcal{O}\left(\frac{1}{m^3c^4}\right)$ includes higher-order spin--orbit coupling terms, and leads to intrinsic inertial dynamics~\cite{Mondal2017Nutation,Mondal2018JPCM} modifying the real part of the susceptibility. {%The relativistic theory of Landau-Lifshitz-Gilbert equation predicts that 
The intrinsic Gilbert damping parameter $\alpha_i$ and inertial relaxation time $\eta_i$ %, both have relativistic origin. These parameters 
are generally considered to be constant in Eq.~\eqref{Eq5}. %for theoretical modeling. 
However, we emphasize that $\alpha_i$ and $\eta_i$ have to be time-dependent for pulsed, non-harmonic applied fields~\cite{Mondal2016,Mondal2017Nutation}, since the ILLG equation with the constant parameters may not capture the expected dynamics in the ultrafast regime~\cite{Hammar2017,Bajpai2019}. }

One of the pivotal questions of inertial spin dynamics is the time scales on which it is applicable, defined by the inertial relaxation time $\eta_{i}$. {The experimentally determined and theoretically predicted values of $\eta_{i}$ are summarized in Table~\ref{tab:eta}.} Although the phenomenological theory is not capable of calculating $\eta_{i}$, values of around 1-100~fs were proposed in Refs.~\cite{Ciornei2011,Wegrowe2012}. A value close to a single femtosecond was proposed in Ref. \cite{Bhattacharjee2012}, %Similar values were obtained based on first-principles calculations in Refs.~\cite{Bhattacharjee2012,Thonig2017} 
and deduced from ferromagnetic resonance measurements of the precession frequency in Ref.~\cite{Li2015}. First-principles calculations in Ref.\ \cite{Thonig2017} obtained smaller {absolute} values of $\eta_i \approx 10^{-3}$ fs{{, while in Ref.~\cite{Bouaziz2019} inertial relaxation times typically in the range of $10-100$~fs have been determined from ab initio simulations of the dynamical magnetic susceptibility}}. The detection of the resonant excitation of nutation by time-resolved magneto-optical measurements reported in Ref.~\cite{neeraj2019experimental} arrived at a value for $\eta_i$ on the order of $100$~fs in our convention. A recent measurement on Co films gave a value $\eta_i/\alpha_i \approx 750$~fs \cite{unikandanunni2021inertial} which also suggests that $\eta_i\approx 100$~fs. The large deviation between the values is remarkable since although the methods were different, almost all of the ab initio calculations and the experiments were performed for the $3d$ transition metal ferromagnets Fe, Co and Ni or their alloys. Even less reassuring is the fact that the $\eta_{i}/\gamma_{i}$ values were found to be negative in certain cases~\cite{Suhl1998,fahnle2013erratum,Li2015,Thonig2017,Bouaziz2019}. While a negative $\eta_{i}/\gamma_{i}$ may be substituted in the linear-response regime as was done in Ref.~\cite{Li2015}, the complete non-linear ILLG equation~\eqref{Eq5} is not meaningful for $\eta_{i}/\gamma_{i}<0$, since the magnetic moments could accelerate infinitely to decrease their kinetic energy in Eq.~\eqref{eqT}. A similar restriction is obtained for the damping $\alpha_{i}/\gamma_{i}>0$, otherwise the dissipation term would increase the energy over time in Eq.~\eqref{eqE}. Note that although Refs.~\cite{Ciornei2011,Kikuchi} use an opposite sign convention for the precessional term, the $\alpha_{i}/\gamma_{i}$ and $\eta_{i}/\gamma_{i}$ ratios are positive as required by energetic considerations. The estimated values of $\eta_{i}\approx 1-100$~fs are comparable to electron relaxation times in metals. Since the quasiclassical Boltzmann equation has proven successful in describing the non-equilibrium distribution of electrons and phonons on similar time scales, the ILLG equation can similarly be expected to correctly account for magnetic moment dynamics in this regime.

\begin{table}[tbh!]
\caption{\label{tab:eta}{{Comparison between the values of the inertial relaxation time $\eta_{i}$ obtained from various experimental and theoretical methods.}}}
\begin{ruledtabular}
\begin{tabular}{|>{\centering}p{3cm}|>{\centering}p{2cm}|>{\centering}p{2cm}|}
\hline 
Sample & $\eta_{i}\:\left(\text{fs}\right)$ & Ref.\tabularnewline
\hline 
\hline 
\multicolumn{3}{|c|}{Experiment}\tabularnewline
\hline 
CoFeB, NiFe & 284--318 & \cite{neeraj2019experimental}\tabularnewline
% \hline 
% CoFeB & 317 & \cite{neeraj2019experimental}\tabularnewline
% \hline 
% Epitaxial NiFe & 284 & \cite{neeraj2019experimental}\tabularnewline
% \hline 
% Polycrystalline NiFe & 276 & \cite{neeraj2019experimental}\tabularnewline
\hline 
Py, Co & 0.83--3.1 & \cite{Li2015}\tabularnewline
% \hline 
% Py 6 nm & 2.1 & \cite{Li2015}\tabularnewline
% \hline 
% Py 10 nm & 0.94 & \cite{Li2015}\tabularnewline
% \hline 
% Py 30 nm & 0.83 & \cite{Li2015}\tabularnewline
% \hline 
% Co 6 nm & 3.1 & \cite{Li2015}\tabularnewline
% \hline 
% Co 10 nm & 1.4 & \cite{Li2015}\tabularnewline
% \hline 
% Co 15 nm & 1.6 & \cite{Li2015}\tabularnewline
% \hline 
% Co 30 nm & 1.5 & \cite{Li2015}\tabularnewline
\hline 
Co & 75--120 & \cite{unikandanunni2021inertial}\tabularnewline
% \hline 
% Co fcc & 120 & \cite{unikandanunni2021inertial}\tabularnewline
% \hline 
% Co bcc & 110 & \cite{unikandanunni2021inertial}\tabularnewline
% \hline 
% Co hcp & 75 & \cite{unikandanunni2021inertial}\tabularnewline
\hline 
\multicolumn{3}{|c|}{Theory}\tabularnewline
\hline 
estimates & $\approx$1-100 & \cite{Ciornei2011, Bhattacharjee2012, Wegrowe2012}\tabularnewline
% Generic Ferromagnet & 1-100 & \cite{Ciornei2011, Bhattacharjee2012, Wegrowe2012}\tabularnewline
\hline 
bulk Fe, Co, Ni & $5.9-6.5\times10^{-3}$ & \cite{Thonig2017}\tabularnewline
\hline 
$3d$ and $4d$ impurities & $\approx$10--100 & \cite{Bouaziz2019}\tabularnewline
% \hline 
% Ni fcc & $\pm 6.1\times10^{-3}$ & \cite{Thonig2017}\tabularnewline
% \hline 
% Fe bcc & $\pm 5.9\times10^{-3}$ & \cite{Thonig2017}\tabularnewline
% \hline 
% Co fcc & $\pm 6.5\times10^{-3}$ & \cite{Thonig2017}\tabularnewline
\hline 
\end{tabular}
\end{ruledtabular}


\end{table}




%{\blue \bf A paragraph on time-dependent effects?? FDT and higher FDT}

We note that non-harmonic time-dependent fields can also produce field-derivative torques, along with the ILLG spin dynamics. These spin torques are relativistic in nature and have been derived from Dirac theory \cite{Mondal2019PRB,Mondal2017Nutation}. 
\\
\\
% While deriving the equation, Gilbert already suggested that the LLG equation does not contain any inertial terms and therefore, the angular momentum of the spin is proportional to the magnetic moment \cite{gilbert04}. Later, it has been shown that in the presence of magnetic inertia this proportionality breaks down \cite{Wegrowe2012}. Thus the magnetic inertia of spin directly provides an intrinsic deviation of the parallel alignment of the angular momentum from the magnetic moment. This nonalignment brings a rich variety of physics at high frequency and ultrafast spin dynamics. In fact, the magnetic moment starts to rotate around the angular momentum that gives rise to the inertial dynamics.

% In order to theoretically include inertial spin dynamics, an additional spin torque term of the form $\boldsymbol{ M}\times \ddot{\boldsymbol{ M}}$ has to be included in Eq.\ (\ref{Eq2}). To incorporate such a torque term, the LLG spin dynamics
% was generalized with the extension of the breathing Fermi surface model \cite{Fahnle2011}, in the framework of mesoscopic nonequilibrium thermodynamics theory  \cite{Ciornei2011}, with a Lagrangian approach \cite{Wegrowe2012} , in an atomistic \cite{Bhattacharjee2012} and in a Dirac relativistic quantum \cite{Mondal2017Nutation,Mondal2018JPCM} frameworks. Below two of these methods are discussed. However, the origin of inertial dynamics is still an open question. 

% To derive the ILLG equation of motion, a theoretical approach was adapted In Ref.~\cite{Ciornei2011,Wegrowe2012} where a mechanical analogy of a rigid-body motion was employed.  
% %\begin{figure}[tbh!]
% %    \centering
% %    \includegraphics[scale = 0.6]{Figure/Coordinates.png}
% %    \caption{The coordinate system representing a rigid body motion. The body-fixed frame is $\{\Vec{e_1},\Vec{e_2},\Vec{e_3}\}$ with the radius $M_s$. The figure is taken from Ref. \cite{Wegrowe2012}. {\gred We may try to sketch this.}} {\gred I think we must describe $\phi, \psi$, or draw another figure}
% %    \label{fig:my_label}
% %\end{figure}
% In such a case, the relation between the angular momentum vector $\boldsymbol{ L} = (L_1,L_2,L_3)$ and the angular velocity $\boldsymbol{ \Omega} = (\Omega_1,\Omega_2,\Omega_3)$ is
% $\boldsymbol{ L} = I \boldsymbol{ \Omega}$, where $I$ is the inertial tensor of the rigid body. In the fixed-body frame, the inertial tensor reduces to the moments of inertia along the three principal axes of rotation as $\{I_1,I_2,I_3\}$. Furthermore, the symmetry of the rotation suggests that $I_1 = I_2$ such that  
% \begin{align}
%     I & = \begin{pmatrix}
%     I_1 & 0 & 0\\
%     0 & I_1 & 0\\
%     0 & 0 & I_3
%     \end{pmatrix}
% \end{align}
% In the rotating frame, it is known that any vector $\bf{M}$ should follow the evolution as \begin{align}
%     \dot{\boldsymbol{ M}} & = \boldsymbol{ \Omega} \times \boldsymbol{ M} 
% \end{align}
% Taking a cross product on both sides by $\boldsymbol{ M} = M_s\hat{\boldsymbol{ z}}$ and utilizing the double cross product $\boldsymbol{ M}\times \left(\boldsymbol{ \Omega}\times \boldsymbol{ M}\right) = \boldsymbol{ \Omega} M_S^2 - \boldsymbol{ M} \left(\boldsymbol{ \Omega} \cdot \boldsymbol{ M}\right) = \boldsymbol{ \Omega} M_S^2 - M_S^2\Omega_3 \hat{\boldsymbol{ z}} $, one obtains
% \begin{align}
%     \boldsymbol{\Omega} & = \frac{\boldsymbol{M}\times \dot{\boldsymbol{M}}}{M_S^2}  + \Omega_3 \hat{\boldsymbol{ z}} 
% \end{align}
% The relation between angular momentum and angular velocity then suggests that 
% \begin{align}
%     \boldsymbol{ L} & = \frac{I_1}{M_S^2}\boldsymbol{ M}\times \dot{\boldsymbol{ M}} + L_3  \hat{\boldsymbol{ z}}\nonumber\\
%     & =  \frac{I_1}{M_S^2}\boldsymbol{ M}\times \dot{\boldsymbol{ M}} + \frac{M_S}{\gamma}\hat{\boldsymbol{ z}}
%     \label{Angular_momentum}
% \end{align}
% As pointed out earlier, Gilbert's original derivation suggested $I_1 = 0$. By doing so, one obtains the LLG spin dynamics. However, it is necessary to make the inertial terms nonzero meaning $I_1 \neq 0$ because there exists hardly any rigid body for which $I_1 = 0$. Therefore, if $I_1\neq 0$, the angular momentum does not point parallel to the direction of the magnetic moment. Such an approach derives the inertial dynamics from the first term of Eq.\ (\ref{Angular_momentum}) \cite{Wegrowe2012}.

%{\gred Write more!}
 
% On the other hand, the ILLG spin dynamics has also been derived within the relativistic Dirac formalism \cite{Mondal2015a,Mondal2017Nutation,Mondal2018JPCM}. To this end, the Dirac Hamiltonian {describing the particles and antiparticles} has been considered in the presence of an external electromagnetic field as
% \begin{align}
%     \mathcal{H}_{\rm Dirac} & = c\boldsymbol{ \alpha} \cdot \left(\boldsymbol{ p}-e\boldsymbol{ A}\right) + \beta mc^2 + V
% \end{align}
% where $\bf{\alpha}$ and $\beta$ are the Dirac matrices, $\bf{A}$ is the external electromagnetic vector potential, and $V$ is the scalar potential. The Dirac matrices %$\bf{\sigma}$
% contain the Pauli spin matrices $\sigma$, representing the spin. The magnetic field associated with the external electromagnetic field is $\bf{B} = \bf{\nabla} \times \bf{A}$. %{\gred The Dirac Hamiltonian describes the particles and antiparticles.}
% To separate the particles from antiparticles, the Foldy-Wouthuysen transformation \cite{foldy50} has been applied. The electrons can be described by the following Hamiltonian
% \begin{align}
%     \mathcal{H}_{\rm FW} & = \frac{\left(\boldsymbol{ p}-e\boldsymbol{ A}\right)^2}{2m} + V  - \frac{e\hbar}{2m}\, \boldsymbol{ \sigma}\cdot \boldsymbol{ B}\nonumber\\
%     & + \mathcal{O}\left(\frac{1}{m^2c^2}\right) + \mathcal{O}\left(\frac{1}{m^3c^4}\right) + \dots 
%     \label{FW_Hamiltonian}
% \end{align}
% Here, the first three terms are nonrelativistic contributions while the rest are considered to be relativistic corrections. The first-order relativistic correction term  $\mathcal{O}\left(\frac{1}{m^2c^2}\right)$ contains many %relativistic contributions 
% spin-dependent and spin-independent terms, the most significant one being the spin-orbit coupling (SOC). Similarly, the second-order relativistic correction term $\mathcal{O}\left(\frac{1}{m^3c^4}\right)$ includes the higher-order SOC terms. The spin dynamical equation has been constructed by deriving the commutator of spin angular momentum with the above Hamiltonian in Eq.\ (\ref{FW_Hamiltonian}) \cite{white07,Mondal2016}. Apparently, all the spin-dependent terms in the Hamiltonian can contribute to the spin dynamics. As a result, it was shown that the Zeeman-like field-spin coupling terms cause the precessional motion of spins. While the relativistic SOC leads to the dissipative Gilbert damping, the  higher-order SOC is responsible for the magnetic inertial dynamics \cite{Mondal2017Nutation, Mondal2018JPCM}. 
% Furthermore, the relativistic theory also explains that while the Gilbert damping dynamics is related to the imaginary part of the magnetic susceptibility, the inertial dynamics is related to the real part of the susceptibility \cite{Mondal2018JPCM}. 

% The inertial LLG (ILLG) spin dynamics including the inertial term takes the form
% \begin{align}
%     \dot{\boldsymbol{ M}}_{ i}(t) = - \gamma_{ i} \boldsymbol{ M}_{ i} \times \boldsymbol{ B}^{\rm eff}_{ i} + \frac{\alpha_{ i}}{M_{ S,i}}  \boldsymbol{ M}_{ i} \times \dot{\boldsymbol{ M}}_{ i} + \frac{\eta_{ i}}{M_{ S,i}}  \boldsymbol{ M}_{ i} \times \ddot{\boldsymbol{ M}}_{ i} \,,  \label{Eq5}
% \end{align}
% The last term essentially describes the magnetic inertial dynamics where the double time-derivative of magnetization is defined as $\partial^2\boldsymbol{ M}/\partial t^2 = \ddot{\boldsymbol{ M}}$. This dynamical equation is an extension of LLG dynamics in Eq.\ (\ref{Eq2}). In fact at the limit $\eta \rightarrow 0$, the ILLG equation completely transforms into the LLG equation. Note that while the Gilbert damping parameter $\alpha$ is dimensionless, the parameter describing the strength of inertial dynamics $\eta$ has the dimension of time. Thus, the parameter $\eta$ is called relaxation time \cite{neeraj2019experimental}. Similar to the Gilbert damping, the relaxation time $\eta$ is found to be a tensor \cite{Mondal2017Nutation}. Moreover, it is investigated that the imaginary (dissipative) and real (reactive) parts of the magnetic  susceptibility are related to the Gilbert damping parameter and relaxation time, respectively \cite{Mondal2018JPCM}. Several theoretical and experimental methods enable the calculation of the Gilbert damping parameter $\alpha$ in magnetic systems \cite{Schoen2016}. But, there are not enough investigations of the calculation of relaxation time $\eta$ \cite{Thonig2017}. Nevertheless, the parameter $\eta$ has been predicted to belong in the sub-picosecond timescales e.g., 1 fs - 1 ps \cite{Bhattacharjee2012,Ciornei2011}. A schematic diagram of the ILLG equation is shown in Fig.\ \ref{fig:Fig1}, that is accompanied by spin precession, relaxation and nutation.

% We mention that earlier investigations used a different notation to introduce the inertial dynamics \cite{Ciornei2011,Olive2012,Wegrowe2015JAP,Wegrowe2016JPCM,cherkasskii2020nutation,neeraj2019experimental}. In these works, the inertial relaxation time was denoted by $\alpha \tau$ %has been considered 
% instead of by $\eta$.% for inertial dynamics, where $\tau$ is the inertial relaxation time. 
% This formalism directly links the Gilbert damping dynamics to the inertial dynamics. However, it has been realized that inertial dynamics is independent of the Gilbert damping dynamics, thus the introduction of a separate parameter $\eta$ is more profound for inertial dynamics \cite{Mondal2017Nutation,Mondal2020nutation}.    
% %{\gred Write something on $\eta = \alpha \tau$ and sign of $\eta$}

%\begin{figure}[tbh!]
%    \centering
%    \includegraphics[scale = 0.35]{Figure/Nutation_Peak.png}
%    \caption{The magnetic response as a function of frequency. The FMR resonance peak can already be obtained from LLG spin dynamics. Both the nutation and the FMR peaks can be observed from the simulation of ILLG spin dynamics. The figure is taken from Ref.\ \cite{neeraj2019experimental}.}
%    \label{Nutation_peak}
%\end{figure}

% The theoretical calculation of the magnetic response via the ILLG spin dynamics suggests that two resonance frequencies should be observed \cite{Olive2012}.  Along with the FMR resonance peak, the magnetic inertial dynamics introduces an additional resonance peak at the THz frequencies called nutation resonance. Such a THz resonance has recently been observed experimentally in ferromagnets \cite{neeraj2019experimental}. While the FMR resonance occurs as the effect of the spin precession around the effective field, the nutation resonance arises from the fact that the magnetic moment rotates around the angular momentum. Moreover, the nutation resonance shifts the precession resonance frequency towards a lower value \cite{cherkasskii2020nutation,Mondal2020nutation}. We shall discuss the effect of inertial dynamics applied to specific magnetic systems in the next sections.      

% In addition to nutation caused by spin inertia, other types of nutation can also be noted. Transient nutation (Rabi oscillations) have been widely investigated in nuclear magnetic resonance \cite{Torrey1949} and electron spin resonance \cite{Verma1973time,Fedoruk2002}, and they were recently addressed in ferromagnets \cite{Capua2017}. A complex dynamics and Josephson nutation of a local spin $S=1/2$ as well as of a large spin cluster embedded in the tunnel junction between ferromagnetic leads was shown to occur due to a coupling to the Josephson current \cite{Fransson2008,Nussinov2005,Zhu2004}. Low-frequency nutation was observed in nanomagnets exhibiting a non-linear FMR with the large-angle precession of magnetization where the onset of spin wave instabilities can be delayed due to geometric confinement \cite{Li2019}. 

\section{Inertial effects in ferromagnetic resonance}
\subsection{Ferromagnets}
%{\gred \bf MC }

For testing the accuracy of the model, it is necessary to connect the theoretical predictions based on the ILLG equation~\eqref{Eq5} to experimentally observable quantities. %Since nutation has a relatively low amplitude and is expected to occur at very short time scales where real-time measurements are unfeasible~\cite{Bottcher2012}, experiments in the frequency regime appear to be most suitable for this purpose. 
One of the possible methods is ferromagnetic resonance (FMR) where the linear response to a spatially homogeneous time-dependent external field is measured~\cite{Kittel1948}. A ferromagnet placed in a static external field $B_{\textrm{ext}}$ may be treated as a macrospin in FMR, which was investigated using numerical simulations of the ILLG equation in Ref.~\cite{Olive2012}. The magnetic susceptibility of the macrospin to a circularly polarized excitation %field 
of frequency $\omega$ is given by~\cite{cherkasskii2020nutation,Mondal2020nutation} 

\begin{equation} 
%	{\chi _ \pm (\omega)} = \dfrac{{\gamma {\mu _0}M_s}}{{\gamma {\mu _0}{H_{\rm ext}} - \omega  - \eta {\omega ^2} \pm i\alpha \omega }}.
%	{\chi _ \pm (\omega)} = \dfrac{{\gamma {\mu _0}M_s}}{{\gamma B_{\textrm{ext}} - \omega  - \eta {\omega ^2} \pm i\alpha \omega }}.
	{\chi (\omega)} = \dfrac{{\gamma M_{0}}}{{\gamma B_{\textrm{ext}} - \omega  - \eta {\omega ^2} + i\alpha \omega }}.
\end{equation}

With the help of this susceptibility, one can calculate the dissipated power $P = \omega {\tt Im}[\chi(\omega)]$, shown in Fig.\,\ref{fig:suscep_FMR_nut}. The dissipated power shows peaks in the vicinity of the poles of the susceptibility,
\begin{equation} %\label{eq:w_nut_weak}
\begin{split}
%{\omega _n} = \frac{{1 + \sqrt {1 + 4\eta \gamma {\mu _0}{H_{\rm ext}}} }}{{2\eta }} \approx \frac{1}{\eta }.
{\omega _{\textrm{p}}} &= \frac{{-1 + \sqrt {1 + 4\eta \gamma B_{\textrm{ext}}} }}{{2\eta }} \\
&\approx \gamma B_{\textrm{ext}}\left(1-\eta\gamma B_{\textrm{ext}}\right)
\end{split}\label{eqprecFM}
\end{equation} 
and
\begin{equation}
\begin{split}
%{\omega _n} = \frac{{1 + \sqrt {1 + 4\eta \gamma {\mu _0}{H_{\rm ext}}} }}{{2\eta }} \approx \frac{1}{\eta }.
{\omega _{\textrm{n}}} &= \frac{{-1 - \sqrt {1 + 4\eta \gamma B_{\textrm{ext}}} }}{{2\eta }} \\
&\approx -\frac{1}{\eta }-\gamma B_{\textrm{ext}}\left(1-\eta\gamma B_{\textrm{ext}}\right)\,,
\end{split} \label{eqnutFM}
\end{equation}
where $\omega_{\rm p}$ and $\omega_{\rm n}$ denote the precession and nutation frequencies, respectively.
Approximate expressions for the frequencies were already derived in Ref.~\cite{Olive2015}, which reproduce the first term in the expansion.

{{The inertia causes a redshift of the precession frequency $\omega_{\textrm{p}}$ in Eq.~\eqref{eqprecFM}, as is visible in Fig.~\ref{fig:suscep_FMR_nut}. Unfortunately, this effect is not directly observable experimentally, since the inertia cannot be turned off in magnetic materials. Moreover, the resonance frequency may also be shifted by anisotropy effects discussed below, and the strength of the anisotropy terms would have to be also determined from the position of the FMR peak. As shown in Fig.~\ref{fig:FreqH}, the inertia also influences the dependence of the precession frequency on the external field. The effective gyromagnetic ratio $\gamma_{\textrm{eff}}=\partial\omega_{\textrm{p}}/\partial B_{\textrm{ext}}$ is decreased, and the frequency is no longer linear in the external field but also contains a term quadratic in $B_{\textrm{ext}}$ with a negative sign, which could represent an experimentally detectable signature.}} %Somewhat more promising is that the inertia also renormalizes the gyromagnetic ratio, leading to a term quadratic in $B_{\textrm{ext}}$ with a negative sign which could be possible to identify in field-dependent measurements; cf. Fig.~\ref{fig:FreqH}. 
This direction was pursued in Ref.~\cite{Li2015}, where it was observed that the frequency is actually blueshifted at high fields, i.e., the coefficient of the $B_{\textrm{ext}}^{2}$ term is positive. Based on the energetical considerations fixing the sign of $\eta/\gamma>0$ discussed above, this seems to indicate that the redshift caused by the inertia is obscured by further effects not taken into account in Eq.~\eqref{eqprecFM}.

More promising for the observation is the emergence of a second nutational resonance peak $\omega_{\textrm{n}}$ in Fig.~\ref{fig:suscep_FMR_nut}. The negative frequency denotes the opposite handedness of this excitation compared to the counterclockwise precession~\cite{Kikuchi,Mondal2020nutation}. Since the LLG equation is not capable of explaining the emergence of magnetic excitations in ferromagnets at such high frequencies, a resonant excitation at the nutation frequency provides a distinct signature of inertial dynamics. Such high-frequency resonances were  recently detected via time-resolved magneto-optical pump-probe techniques in Refs.~\cite{neeraj2019experimental,unikandanunni2021inertial}. { {Typical experimental spectra from Ref.~\cite{unikandanunni2021inertial} are shown in Fig.\,~\ref{fig:nut_exper}. The presence of higher harmonics in the experimental signal not predicted by linear-response theory might represent an indication of non-linear processes. %These spectra are the Fourier transform of the temporal traces of magnetization at time delays $t > 1.7 \; ps$, when the THz pump field has left the cobalt thin films sample. The simulation performed in Ref.~\cite{unikandanunni2021inertial} using the ILLG equation and linear-response theory predicts one nutation resonance peak, but the higher order harmonics were detected. 
}}

{{
A further possibility for the experimental investigation of the nutation resonances is based on spin pumping. In this case, a spin current is injected from an externally excited ferromagnet into an adjacent non-magnetic metal. This spin current is predicted to change sign as the frequency is changed from the precession to the nutation resonance~\cite{Mondal2021PRBSpinCurrent}.}}

% Various theoretical approaches were employed to investigate the inertial dynamics in ferromagnets. Kikuchi \textit{et al.} show that the nonadiabatic contribution of environmental degrees of freedom yields spin inertia in metallic ferromagnets using the expansion of the spin effective action \cite{Kikuchi}.  The inertia has two effects: the superimposition of the nutation on the conventional precession and the redshift of the precession frequency. These effects originate due to the nonadiabatic component in the relation between the spin and the angular momentum, and consequently, the spin acquires a separate degree of freedom from the angular momentum. As a result, the spin undergoes nutation.%, and a shift along the direction of the time derivative of applied magnetic fields. 

% The comprehensive numerical analysis of precession and nutation resonances using the ILLG equation is performed by E. Olive \textit{et al.} \cite{Olive2012, Olive2015}. The precession and nutation resonance frequencies are found to depend on the %inertial dynamics characteristic 
% relaxation time $\tau$, the dimensionless damping $\alpha$, and the static magnetic field $B_{\textrm{ext}}$. As discussed earlier, $\eta = \alpha \tau$ is independent of the damping within the classical Lagrangian approach \cite{Wegrowe2012}, whereas the breathing Fermi surface model suggests the relation between damping and inertial %characteristic 
% relaxation time \cite{Fahnle2011, fahnle2013erratum}. In Refs. \cite{Olive2012, Olive2015}, it was found that the frequency of nutation resonance can be estimated as
% \begin{equation} \label{eq:w_nut_Olive}
% %{\omega' _n} = \dfrac{{\sqrt {1 + \eta \gamma {\mu _0}{H_{\rm ext}}} }}{{\eta }} \approx \dfrac{1}{\eta }
% {\omega' _{\textrm{n}}} = \dfrac{{\sqrt {1 + \eta \gamma B_{\textrm{ext}}} }}{{\eta }} \approx \dfrac{1}{\eta }.\end{equation} 
% %{where  $\mu_0$ is the free space magnetic permeability.} 
% This estimation was found based on the damped oscillator model.

% The frequency-dependent susceptibility $\chi_\pm(\omega)$ exhibiting ferromagnetic and nutation resonances was derived in Ref. \cite{cherkasskii2020nutation} taking into account only the Zeeman energy. One can consider this susceptibility as a response of the ferromagnet to the external excitation. For circularly polarized fields, the following expression is found:
% \begin{equation} 
% %	{\chi _ \pm (\omega)} = \dfrac{{\gamma {\mu _0}M_s}}{{\gamma {\mu _0}{H_{\rm ext}} - \omega  - \eta {\omega ^2} \pm i\alpha \omega }}.
% 	{\chi _ \pm (\omega)} = \dfrac{{\gamma {\mu _0}M_s}}{{\gamma B_{\textrm{ext}} - \omega  - \eta {\omega ^2} \pm i\alpha \omega }}.
% \end{equation}
% The frequency dependence of the reactive and dissipative parts is ${\tt Re}[\chi_ \pm(\omega)]$ and ${\tt Im}[\chi_ \pm(\omega)]$, respectively.
% %of susceptibilities given by ${\chi _ \pm } = {\chi '_ \pm } - i{\chi ''_ \pm },$. 
% With the help of this susceptibility, one can calculate the dissipated power that is: $P = \omega {\tt Im}[\chi_+(\omega)]$, which is shown in Fig.\,\ref{fig:suscep_FMR_nut}.  
% %The plus and minus subscripts correspond to right-hand and left-hand direction of circular polarization. 
% The FMR occurs for right-handed precession, i.e. positive polarization of the microwave excitation, while nutation resonance can be observed in the opposite case. Based on the susceptibility, another estimation of the nutation resonance frequency was derived:
% \begin{equation} \label{eq:w_nut_weak}
% %{\omega _n} = \frac{{1 + \sqrt {1 + 4\eta \gamma {\mu _0}{H_{\rm ext}}} }}{{2\eta }} \approx \frac{1}{\eta }.
% {\omega _{\textrm{n}}} = \frac{{1 + \sqrt {1 + 4\eta \gamma B_{\textrm{ext}}} }}{{2\eta }} \approx \frac{1}{\eta }.
% \end{equation} 
% Note that both estimation (\ref{eq:w_nut_Olive}) and (\ref{eq:w_nut_weak}) were found for negligibly small boundary and magnetic anisotropy effects.

\begin{figure}[tbh!]
    \centering
    \includegraphics[scale = 0.45]{dissipation_ferro_1012new.pdf}
    \caption{Ferromagnetic resonance for a single macrospin. The dissipated power compared between the inertia-free ($\eta=0$~ps) and inertial ($\eta=1$~ps) cases. The negative frequency of nutation indicates the opposite handedness of this motion compared to precession. The Hamiltonian is given by $\mathcal{H}=-M_{z}B_{\rm ext}-KM_{z}^{2}/M_{0}^{2}$. The calculation parameters are $ \gamma = 1.76\times 10^{11}$ T$^{-1}$s$^{-1}$, on-site anisotropy energy
$K = 10^{-23}$ {J},
$M_{ 0}  = 2 \mu_{\rm B}$, $ \alpha = 0.05$, and $ B_{\rm ext} = 1$ T. The data are taken from Ref. \cite{Mondal2020nutation}. %(a) The FMR peak in non-inertial ferromagnets $\left(\eta = 0 \right)$ (dashed blue) and the FMR peak redshifted by inertia  $\left(\eta = 284 \:\rm{fs} \cdot \rm{rad}^{-1} \right)$ (orange). (b) The nutation resonance. The calculation was performed for $\gamma/\left(2 \pi \right) = 28 \:\rm{GHz} \cdot \rm{T}^{-1}$, ${\mu _0}{M_s} =0.9 \:\rm{T}$, ${\mu _0}{H_{\rm{ext}}} =0.1 \:\rm{T}$, $\alpha=0.0058$.
    }
    \label{fig:suscep_FMR_nut}
\end{figure}

\begin{figure}
    \centering
    %\includegraphics[scale = 0.3]{Figure/exper.pdf}
    \includegraphics[width=\columnwidth]{exper1.pdf}
    \caption{{{Experimental signatures of the inertial dynamics. Comparison between Fourier transforms of the probe signal in pump-probe measurements (solid line, open circles) and numerical simulations based on Eq.~\eqref{Eq5} for a single macrospin for (a) face-centered cubic (blue), (b) body-centered cubic (orange), and (c) hexagonal close-packed (green) cobalt thin films. The nutational resonance and its higher harmonics are visible in the experimental spectra. (d) Simulated response of the magnetization in the time domain. The semitransparent line shows the time integral of the pump pulse. Figure from Ref.~\cite{unikandanunni2021inertial}.}}
    %Experimental (solid line, open circles) and simulated (semitransparent line) Fourier transform of the magnetization dynamics in (a) face-centered cubic (fcc, blue), (b) body-centered cubic (bcc, orange), and (c) hexagonal close-packed (hcp, green) cobalt thin films. (d) Solid lines, simulated response of magnetization to the terahertz excitation $H_{\rm{THz}}$ with Eq.\;~\ref{Eq5}. Dashed line, integral over time of $H_{\rm{THz}}$, from Ref.~\cite{unikandanunni2021inertial}
    }
    \label{fig:nut_exper}
\end{figure}

%The interplay between magnetic field, magnetic anisotropy and inertial effects 
The inertial dynamics of a general anisotropic macrospin in the linear-response formalism was investigated in Refs.~\cite{Cherkasskii2022Anisotropy, titov2022ferromagnetic}. 
%The secular equation of an inertial spin system was derived in analogy to the ubiquitous Smit-Beljers formalism, which allows one to study the influence of free-energy landscapes on eigenfrequencies. Since this formalism requires the investigation in spherical coordinates, the ILLG equation is written as
%     \begin{equation} \label{eq:ILLGsph_approx} 
%     \left\{ \begin{array}{l}
%     {\partial _{tt}}\theta  = \dfrac{{\gamma {B_\theta }}}{\eta } - \dfrac{{\alpha \gamma {B_\varphi }}}{\eta } 
%      + \dfrac{{{\partial _t}\varphi \sin \theta }}{\eta } \\ 
%     + {\left( {{\partial _t}\varphi } \right)^2}\sin \theta \cos \theta ,
%     \\  
%     {\partial _{tt}}\varphi \sin \theta  = \dfrac{{\gamma {B_\varphi }}}{\eta } + \dfrac{{\alpha \gamma {B_\theta }}}{\eta } 
%     - \dfrac{{{\partial _t}\theta }}{\eta } \\ 
%     - 2{\partial _t}\varphi {\partial _t}\theta \cos \theta,
%     \end{array}
%     \right.
%     \end{equation}
% where $\theta$ and $\varphi$ are the angles of the magnetization. Using the small-angle approximation, the system of equations can be linearized. Employing the periodic solution ansatz, a fourth-order characteristic polynomial is obtained, which constitutes the secular equation of the inertial spin system:
After expressing the free-energy density $F$ in polar coordinates $\vartheta$ and $\varphi$, the excitation frequencies are found from the solution of the fourth-order secular equation
\begin{figure}[t]
    \centering
    \includegraphics[]{w_H_nut.png}
    \caption{Frequency-field relation of precessional and nutational resonances. The explicit solution of Eq.\,\eqref{eq:SBC} for the precessional resonance (orange line) shows a redshift compared to the non-inertial Smit--Beljers case (dashed blue line), and the nutational resonance (red) demonstrates a blueshift compared to the zeroth-order approximation $1/\eta$. The calculation parameters for a thin film with cubic magnetocrystalline anisotropy are %$\gamma/\left(2 \pi \right) = 28 \:\rm{GHz} \cdot \rm{T}^{-1}$, 
    ${\mu _0}{M_{\textrm{S}}} =0.9 \:\rm{T}$, $\alpha=0.0058$, $\eta = 284 \:\rm{fs}$, $K_{\rm{cub1}} =  4.9 \times 10^4 \: \rm{ J} \cdot \rm{m}^{-3}$. The data are taken from\,Ref.\,\cite{Cherkasskii2022Anisotropy}}
    \label{fig:FreqH}
\end{figure}

    \begin{equation} \label{eq:SBC}
    \begin{split}
		 & \left[ \frac{\omega ^2}{\gamma^2} - \frac{{\left( {1 + {\alpha ^2}} \right)}}{{M_{\textrm{S}}^2{{\sin }^2}{\vartheta}}}\left( {{\partial _{\vartheta \vartheta }}F{\partial _{\phi \phi }}F - {{\left( {{\partial _{\vartheta \phi }}F} \right)}^2}} \right) \right] \\
		 & - {\eta ^2 \omega ^2} \left[ \frac{{\omega ^2}}{ \gamma ^2} - \frac{1}{\eta \gamma M_{\textrm{S}}}\left( {{\partial _{\vartheta \vartheta }}F + \frac{{{\partial _{\varphi \varphi }}F}}{{{{\sin }^2}{\vartheta}}}} \right) \right]\\
		 & - i\omega \frac{{\alpha}}{\gamma M_{\textrm{S}}}\left( {{\partial _{\vartheta \vartheta }}F + \frac{{{\partial _{\varphi \varphi }}F}}{{{{\sin }^2}{\vartheta}}}} \right) = 0,
%		 & \left[ \frac{\omega ^2}{\gamma^2} - \frac{{\left( {1 + {\alpha ^2}} \right)}}{{M_S^2{{\sin }^2}{\vartheta}}}\left( {{\partial _{\vartheta \vartheta }}F{\partial _{\phi \phi }}F - {{\left( {{\partial _{\vartheta \phi }}F} \right)}^2}} \right) \right] \\
%		 & - {\eta ^2 \omega ^2} \left[ \frac{{\omega ^2}}{ \gamma ^2} - \frac{1}{\eta \gamma M_S}\left( {{\partial _{\vartheta \vartheta }}F + \frac{{{\partial _{\varphi \varphi }}F}}{{{{\sin }^2}{\vartheta}}}} \right) \right]\\
%		 & - i\omega \frac{{\alpha}}{\gamma M_S}\left( {{\partial _{\vartheta \vartheta }}F + \frac{{{\partial _{\varphi \varphi }}F}}{{{{\sin }^2}{\vartheta}}}} \right) = 0,
  		% & \left[ \frac{\omega ^2}{\gamma^2} - \frac{{\left( {1 + {\alpha ^2}} \right)}}{{M_{\textrm{S}}^2{{\sin }^2}{\vartheta}}}\left( {{\partial _{\vartheta \vartheta }}\mathcal{H}{\partial _{\phi \phi }}\mathcal{H} - {{\left( {{\partial _{\vartheta \phi }}\mathcal{H}} \right)}^2}} \right) \right] \\
		 % & - {\eta ^2 \omega ^2} \left[ \frac{{\omega ^2}}{ \gamma ^2} - \frac{1}{\eta \gamma M_{\textrm{S}}}\left( {{\partial _{\vartheta \vartheta }}\mathcal{H} + \frac{{{\partial _{\varphi \varphi }}\mathcal{H}}}{{{{\sin }^2}{\vartheta}}}} \right) \right]\\
		 % & - i\omega \frac{{\alpha}}{\gamma M_{\textrm{S}}}\left( {{\partial _{\vartheta \vartheta }}\mathcal{H} + \frac{{{\partial _{\varphi \varphi }}\mathcal{H}}}{{{{\sin }^2}{\vartheta}}}} \right) = 0.
	\end{split}
    \end{equation}
%where $F$ is the free-energy density.
The first %group of terms 
line corresponds to the Smit--Beljers equation. The second %group of terms 
line includes the inertia. The third %group of terms 
line induces the frequency-domain linewidth of the FMR. Note that the solutions of Eq.~\eqref{eq:SBC} can be grouped into pairs of $\omega$ and $-\omega^{*}$ due to a particle-hole constraint, and the two frequency pairs describe precessional and nutational excitations. %A free-energy density including magnetocrystalline, shape and other types of anisotropy can be substituted into this equation to find the eigenfrequencies.

For instance, if the magnetic field $B_{\textrm{ext}}$ is applied out-of-plane with respect to the surface of a film demonstrating cubic magnetocrystalline anisotropy $K_{\textrm{cub1}}$, %and $\eta < 100 \, \rm{fs} \cdot \rm{rad}^{ - 1}$, 
the free-energy density is given by

\begin{equation}
\begin{split}
F= & -B_{{\rm ext}}M_{z}+\dfrac{1}{2}\mu_{0}M_{z}^{2}\\
 & +\frac{K_{{\rm cub1}}}{M_{\textrm{S}}^{4}} \left(M_{x}^{2}M_{y}^{2}+M_{y}^{2}M_{z}^{2}+M_{x}^{2}M_{z}^{2}\right),
%F= & -{\bf B}_{{\rm ext}}{\bf M}+\dfrac{1}{2}\mu_{0}N_{\rm{zz}}{\bf M}^{2}\\
% & +K_{{\rm cub1}} \left(\alpha_{1}^{2}\alpha_{2}^{2}+\alpha_{2}^{2}\alpha_{3}^{2}+\alpha_{1}^{2}\alpha_{3}^{2}\right)
\end{split}
\end{equation}

allowing to find the approximate solution of Eq.~\eqref{eq:SBC}:
\begin{equation} \label{eq:w_approx_prec_b_cub_H_OOP}
    \begin{split}
%    &{\omega_{\rm{FMR}}}^2 = 
%    \left| \gamma  \right|^2\left(1+ \alpha ^2 \right) \left(-\mu_0 M_0 + \mu_0 H_{\rm{ext}}  + \dfrac{2 K_{\rm{cub1}}}{M} \right)^2 \\
%    & \times \left[1-\eta\left| \gamma  \right|\left(-2 \mu_0 M_0 +2 \mu_0 H_{\rm{ext}}  + \dfrac{4 K_{\rm{cub1}}}{M}\right)\right],
    & {\omega_{\rm{p}}}^2 \approx  
    \gamma^2\left(1+ \alpha ^2 \right) \left(-\mu_0 M_{\textrm{S}} + B_{\textrm{ext}}  + \dfrac{2 K_{\rm{cub1}}}{M_{\textrm{S}}} \right)^2 \\
    & \times \left[1-\eta\gamma\left(-2 \mu_0 M_{\textrm{S}} +2 B_{\textrm{ext}}  + \dfrac{4 K_{\rm{cub1}}}{M_{\textrm{S}}}\right)\right],
    \end{split}
    \end{equation}

\begin{equation} \label{eq:w_approx_b_nut_cub_H_OOP}
%	{\omega}_{\rm{n}} = \frac{1}{\eta } + \left| \gamma  \right|\left( { - {\mu _0}{M_0} + {\mu _0}{H_{\rm{ext}}} + \frac{{2{K_{{\rm{cub1}}}}}}{{{M_0}}}} \right)
	{\omega}_{\rm{n}} \approx \frac{1}{\eta } + \gamma\left( { - {\mu _0}{M_{\textrm{S}}} + B_{\textrm{ext}} + \frac{{2{K_{{\rm{cub1}}}}}}{{{M_{\textrm{S}}}}}} \right).
\end{equation}
%where $N_{\rm{zz}}$ is the demagnetization factor, $\alpha_{i}$ are directional cosines and $K_{{\rm cub1}}$ is the constants of cubic anisotropy. 

The numerical solutions of Eq.\,(\ref{eq:SBC}) are plotted in Fig.\,\ref{fig:FreqH}. In agreement with Eqs.~\eqref{eqprecFM} and \eqref{eqnutFM}, this approximation shows a redshift for the precessional resonance compared to the non-inertial Smit--Beljers case, and a blueshift of the nutational resonance compared to the zeroth-order approximation $1/\eta$. 

In the undamped limit of the non-linear ILLG equation, analytic solutions for the magnetization of a macrospin with uniaxial magnetocrystalline anisotropy parallel to the external field direction were obtained in terms of the Jacobi elliptic functions and elliptic integrals in Ref.~\cite{Titov2021Deterministic}. In this work, the nutation frequency was determined in terms of the inverse period of the Jacobi elliptic function. In addition, the equilibrium correlation functions of the magnetization at short times were investigated in Ref.~\cite{TitovCorrelation2023}.

For the sake of completeness, it should be mentioned that high resonance frequencies have also been predicted based on the conventional LLG equation due to surface anisotropy effects in ferromagnetic nanoparticles~\cite{Bastardis2018}. These must be distinguished from the resonances caused by the inertial motion of a homogeneous magnetization discussed here.
%that the inertial distortions in spin dynamics discussed here for homogeneous magnetization must be distinguished from similar predictions of higher frequency resonances due to non-collinear spin structures induced by surface anisotropy at nanoparticle surfaces~\cite{Bastardis2018}.

%From this approximation one can clearly see that the frequency of the nutational resonance is generally increased due to the presence of magnetic anisotropy and applied magnetic field, and the FMR frequency is decreased by inertia. 

%In the undamped limit of the ILLG equation, analytic solutions for the magnetization of a single-domain ferromagnetic nanoparticle with the uniaxial magnetocrystalline anisotropy were obtained in terms of the Jacobi elliptic functions and elliptic integrals \cite{Titov2021Deterministic}. In this paper, the nutation frequency was determined in terms of the inverse period of the Jacobi elliptic function. Furthermore, nutation is theoretically analyzed as a part of magnetization dynamics for a single magnetic moment, a chain of Fe atoms, and Co islands on Cu(111) \cite{Bottcher2012}. It is demonstrated that the nutation is important at low-coordination sites and on the timescale of about 100 fs.

%Inertial magnetization dynamics of ferromagnetic nanoparticles including thermal agitation was investigated with the ILLG equation augmented by thermal noise \cite{Titov2021Inertial}. It is shown that the inertial stochastic magnetization dynamics of ferromagnetic nanoparticles has an analogy with the stochastic dynamics of the electric dipole moment of a polar molecule. Such a polar molecule can be visualized as a dipole lying along the axis of symmetry of a symmetric top ignoring friction %about 
%around that axis.

%Besides that, nutation induced by surface effects in nanomagnets is found using the atomistic approach within the framework of the LLG equation, i.e., without taking inertia into account \cite{Bastardis2018}. It is demonstrated that surface anisotropy causes the spin misalignment, which in turn exhibits nutation.

\subsection{Antiferromagnets and ferrimagnets}
%{\gred \bf RM + LR }

Unlike ferromagnets, antiferromagnets and ferrimagnets consist of %two 
multiple %different 
magnetic sublattices %equal in magnetic moments, but 
pointing along different directions. %opposite to each other. Because of the 
In two-sublattice systems, the magnetic susceptibility shows two resonances with opposite handedness, similarly to the precessional and nutational resonances in ferromagnets. Furthermore, the antiferromagnetic exchange coupling typically shifts both resonances in antiferromagnets %~\cite{Kittel1951,Takeo1951,Keffer1952} 
and one of the resonances in ferrimagnets to the THz regime~\cite{Kittel1951,Takeo1951,Keffer1952}. The other ferrimagnetic resonance usually resides in the GHz range, because ferrimagnets have a net magnetic moment similarly to ferromagnets. %These two AFMR frequencies remain exactly the same but with an opposite sign meaning that they have opposite rotational senses.

It was investigated in Ref.~\cite{Mondal2020nutation} how these resonances are influenced by the inertia. Two-sublattice systems may be treated as two interacting macrospins $\boldsymbol{M}_{A}$ and $\boldsymbol{M}_{B}$ described by the Hamiltonian
\begin{align}
\mathcal{H}=&-\left(M_{A,z}+M_{B,z}\right)B_{\textrm{ext}}-\frac{K_{A}}{M_{\textrm{0},A}^{2}}M_{A,z}^{2}-\frac{K_{B}}{M_{\textrm{0},B}^{2}}M_{B,z}^{2} \nonumber
\\
&+\frac{J}{M_{\textrm{0},A}M_{\textrm{0},B}}\boldsymbol{M}_{A}\cdot\boldsymbol{M}_{B}\, ,\label{eqAFMHam}
\end{align}
where $K_{A/B}$ are the uniaxial anisotropy coefficients, $M_{0,A}$ and $M_{0,B}$ are the sizes of the magnetic moments and $J$ is the exchange interaction. The linear response is calculated around the state where $\boldsymbol{M}_{A}$ and $\boldsymbol{M}_{B}$ are parallel and antiparallel to the external field, respectively. The resonance frequencies can be identified as the poles of the susceptibility tensor. Although Eq.~\eqref{eqAFMHam} possesses a cylindrical symmetry simplifying the calculations, this requires solving a fourth-order algebraic equation due to the two sublattices. %{\gred having equilibrium magnetic moments $M_{0,A}$ and $M_{0,B}$}.

\begin{figure}[tbh!]
    \centering
    \includegraphics[scale = 0.5]{dissipation_AFM.pdf}
    \caption{The precessional and nutational resonance frequencies for antiferromagnets. The dissipated power is compared between the inertia-free ($\eta=0$~ps) and inertial ($\eta=1$~ps) cases. The calculation parameters are $\gamma_A = \gamma_B   = \gamma = 1.76\times 10^{11}$ T$^{-1}$s$^{-1}$, $J  = 10^{-21}$ {J},
$K_A = K_B = K = 10^{-23}$ {J},
$M_{0,A}  = M_{0,B} = 2 \mu_{\rm B}$, $\alpha_{A}  = \alpha_{B} = \alpha = 0.05$, and $ B_{\rm ext} = 1$ T. The data are taken from Ref. \cite{Mondal2020nutation}.}
    \label{fig:fig4}
\end{figure}

In the antiferromagnetic limit with $\eta_{A}=\eta_{B}=\eta$, %$\alpha_{A}=\alpha_{B}=\alpha$
$\gamma_{A}=\gamma_{B}=\gamma$, $M_{\textrm{0},A}=M_{\textrm{0},B}=M_{\textrm{0}}$ and $K_{A}=K_{B}=K$, the undamped excitation frequencies may be approximated as~\cite{Mondal2020nutation}
\begin{align}
\omega_{\textrm{p}\pm}=&\pm\frac{\gamma}{M_{\textrm{0}}}\frac{\sqrt{4KJ}}{\sqrt{1+2\eta\gamma J/M_{\textrm{0}}}}+\frac{\gamma B_{\textrm{ext}}}{1+2\eta\gamma J/M_{\textrm{0}}}\, ,\label{eqprecAFM}
\\
\omega_{\textrm{n}\pm}=&\pm\frac{1}{\eta}\sqrt{1+2\eta\gamma J/M_{\textrm{0}}}-\frac{\gamma B_{\textrm{ext}}}{1+2\eta\gamma J/M_{\textrm{0}}}\, ,\label{eqnutAFM}
\end{align}
for $K,M_{\textrm{0}}B_{\textrm{ext}}\ll J$. These resonances are observable as peaks in the dissipated power in Fig.~\ref{fig:fig4}. Similarly to ferromagnets, the number of peaks is doubled and the precessional resonance frequency is reduced with the introduction of magnetic inertia. Note that the redshift of the precessional resonance frequency is determined by the dimensionless parameter $\gamma\eta J/M_{0}$ in antiferromagnets and $\gamma\eta B_{\textrm{ext}}$ in ferromagnets, therefore it is exchange enhanced in the former. As mentioned in the ferromagnetic case, this shift itself is not detectable experimentally since it is not possible to distinguish the influence of inertia from, e.g., a different value of the anisotropy. Furthermore, antiferromagnetic precessional resonance peaks have a much lower intensity as can be seen from the comparison between Fig.~\ref{fig:suscep_FMR_nut} and Fig.~\ref{fig:fig4}. However, the nutational peaks have a much higher intensity and a sharp lineshape, since the exchange enhancement of the effective damping parameter only affects the precessional peaks~\cite{Mondal2020nutation}. Although so far there are no experimental investigations of the magnetic inertia in antiferromagnets reported in the literature, these properties indicate that observing the nutational resonances would be possible in them just as %well {\gred as} 
in ferromagnets. Furthermore, since the precessional resonances also have higher frequencies, the same THz methods could be used for the detection of precessional and nutational resonances, while the GHz precessional frequencies in ferromagnets are typically measured using a different approach. %typically require a further instrument.  
\begin{figure}[tbh!]
    \centering
    \includegraphics[scale = 0.25]{Ferri_omegar_eta.pdf}
    \caption{The precessional and nutational resonance frequencies for ferrimagnets as a function of the inertial parameter $\eta$. The calculation parameters are $M_{0,A} = 2\mu_B, M_{0,B} = 5M_{0,A} = 10\mu_B, \gamma_A = \gamma_B  = 1.76\times 10^{11}$ T$^{-1}$s$^{-1}$, $J  = 10^{-21}$ {J},
$K_A = K_B  = 10^{-23}$ {J},
%$M_{B0}  = 2 \mu_{\rm B}$, 
$\alpha_{A}  = \alpha_{B}  = 0.05$, and $ B_{\rm ext} = 1$ T. The figure is taken from Ref. \cite{Mondal2020nutation}.}
    \label{fig:fig5}
\end{figure}

The numerically calculated resonance frequencies in a ferrimagnet are displayed in Fig.~\ref{fig:fig5}. The precessional frequency $\omega_{\textrm{p}+}$ only starts to be influenced by the nutational frequency $\omega_{\textrm{n}-}$ for large values of $\eta$, similarly to the ferromagnetic case. In contrast, the strong interaction between the $\omega_{\textrm{p}-}$ and $\omega_{\textrm{n}+}$ frequencies resembles the antiferromagnetic case. Therefore, the same considerations as above concerning the possible experimental detection of the nutational resonance apply here.

%{\gred Inertial spin dynamics was also studied in the 3d and 4d impurities (Fe, Co, etc.) embedded in prototypical topological insulators $\text{Bi}_{2}\text{Te}_{3}$ and $\text{Bi}_{2}\text{Se}_{3}$~\cite{Bouaziz2019}. It was shown that the magnetic susceptibility can be calculated with time-dependent density functional theory and that this susceptibility is related to the magnetic anisotropy energy as well as to the Gilbert damping and inertia tensors. The impurity induced states are found to have a drastic effect on the spin excitation spectra, resulting in very long lifetimes, up to microseconds. In particular, this leads to a non-negligible shift of the spin excitation resonance in the high-frequency regime.}
%Similarly,  the effective damping reduces in the presence of spin nutation. This reduction in the effective damping is stronger in antiferromagnets than in ferromagnets.

% Unlike ferromagnets, antiferromagnets have two different magnetic sublattices equal in magnetic moments, but pointing opposite to each other. In these materials, the antiferromagnetic resonance (AFMR) frequencies already occur in the THz regime. Therefore, it is possible to manipulate the antiferromagnetic spins at ultrafast  (e.g., picosecond) timescales. Nevertheless, due to the presence of magnetic inertia, nutation resonances have been predicted to occur at even higher THz frequencies. A linear-response theory applied to the ILLG equation shows that the nutation resonance might be more reliably detected in antiferromagnets \cite{Mondal2020nutation}.    

% The spin dynamics in antiferromagnets have successfully been described by the LLG equation with two sublattices coupled by the magnetic exchange interaction \cite{BaierlPRL2016,baierl2016nonlinear,Kampfrath2011,Gomonai,Sonke2012}. As antiferromagnets have two sublattices, the magnetic response (susceptibility) shows two AFMR frequencies typically at THz regime \cite{Kittel1951,Takeo1951,Keffer1952}. These two AFMR frequencies remain exactly the same but with an opposite sign meaning that they have opposite rotational senses. At AFMR frequencies the eigenmode aligns such that the two sublattices stay almost anti-parallel to each other. The two degenerate AFMR frequencies can be separated with the application of an external magnetic field, as the field will shift one frequency to a higher value and the other one to a lower value  \cite{Mondal2020nutation}. Denoting the two sub-lattices as $A$ and $B$, the system of ILLG equations for an antiferromagnet becomes    
% \begin{align}
% \label{Antiferro_Eq1}
%     \dot{\boldsymbol{ M}}_{ A}(t) & = - \gamma_{ A} \boldsymbol{ M}_{ A} \times \boldsymbol{ B}^{\rm eff}_{ A} + \frac{\alpha_{ A}}{M_{ S,A}}  \boldsymbol{ M}_{ A} \times \dot{\boldsymbol{ M}}_{ A}\nonumber\\
%     & + \frac{\eta_{ A}}{M_{ s,A}}  \boldsymbol{ M}_{ A} \times \ddot{\boldsymbol{ M}}_{ A}\,, \\
%     \dot{\boldsymbol{ M}}_{ B}(t) & = - \gamma_{ B} \boldsymbol{ M}_{ B} \times \boldsymbol{ B}^{\rm eff}_{ B} + \frac{\alpha_{ A}}{M_{ S,B}}  \boldsymbol{ M}_{ B} \times \dot{\boldsymbol{ M}}_{ B}\nonumber\\
%     & + \frac{\eta_{ B}}{M_{ s,B}}  \boldsymbol{ M}_{ B} \times \ddot{\boldsymbol{ M}}_{ B}\,,
%     \label{Antiferro_Eq2}
% \end{align}
% where the two sublattices are coupled in the effective field $\boldsymbol{ B}_{A,B}^{\rm eff}$ through the exchange interaction. Note that the effect of inter-sublattice Gilbert and inertial dynamics can also be included within the above description \cite{Kamra2018,Mondal2021PRB}. For antiferromagnets, one considers %the ground state is defined by 
% $\vert \bm{M}_A\vert  = \vert \bm{M}_B\vert  = M_S$. Along the same line, we can also set the other parameters as $\gamma_A = \gamma_B = \gamma$, $\alpha_A = \alpha_B = \alpha$, $\eta_A = \eta_B = \eta$. Within the approach of linear-response theory, the magnetic susceptibility becomes a tensor that depends on the inertial relaxation time $\eta$ and the Gilbert damping $\alpha$. The frequency-dependent susceptibility $\chi_{\pm}^{AB}$ can be written in the following form \cite{Mondal2020nutation}
% \begin{align}
%     \left(\chi_{\pm}%^{AB}
%     (\omega)\right)^{-1} & = \begin{pmatrix}
% \chi_{\pm}^{-1,AA}(\omega) & \chi_{\pm}^{-1,AB}  \\
% \chi_{\pm}^{-1,BA} & \chi_{\pm}^{-1,BB}(\omega) 
% \end{pmatrix}\,,
% \end{align}
% with the following definitions
% \begin{align}
%     \chi_{\pm}^{-1,AA} & = \frac{1}{\gamma M_S}\left[ \Omega_A \pm i\alpha\omega  -\eta \omega^2 + \omega\right]\,, \\
%     \chi_{\pm}^{-1,BB} & = \frac{1}{\gamma M_S}\left[ \Omega_B \pm i\alpha\omega  -\eta \omega^2 - \omega\right]\,, \\
%     \chi_{\pm}^{-1,AB} & = - \frac{J}{M_S^2}\,,\\
%     \chi_{\pm}^{-1,BA} & = - \frac{J}{M_S^2}\,, 
% \end{align}
% The expressions for $\Omega_A$ and $\Omega_B$ can be obtained from the free energy of the antiferromagnet \cite{Mondal2021PRB}. Typically if the two sublattcies have uniaxial anisotropies in terms of $K_A$ and $K_B$  with $K_A = K_B = K$ and they are coupled via exchange energy $J$, the expressions become $\Omega_A = \frac{\gamma}{M_S}(J + 2K + B_{\rm ext} M_S) $ and $\Omega_B = \frac{\gamma}{M_S}(J + 2K - B_{\rm ext} M_S) $.
% %{\gred Can you say something about the susceptibility tensor? Because it seems that the text was cut off.}


% We point out that so far there is no experimental investigations reported on magnetic inertia in antiferromagnets. However, the analytical calculations and simulations of the ILLG equations suggest that the additional two antiferromagnetic nutation resonances (AFMNR) are expected at even higher THz frequencies. Due to the same magnetic moment in the two sublattices, the AFMNR frequency values are exactly the same with an  approximate value $\sim 1/\eta$, considering $\eta_A = \eta_B = \eta$. Moreover, the AFMNR peaks are found to be exchange-enhanced by a factor of $\sqrt{J/K}$, where $J$ is inter-sublattice exchange energy and $K$ is the uniaxial anisotropy energy of the antiferromagnet \cite{Mondal2020nutation}. However, the effect of other types of anisotropy (e.g., bi-axial,  cubic etc.)  on the spin nutation has not yet been investigated in antiferromagnets. 



% Similar to ferromagnets, the precession resonance frequency reduces with the introduction of magnetic inertia in antiferromagnets. Note that the redshift of the precession resonance frequency is much stronger in antiferromagnets than in ferromagnets. Similarly,  the effective damping reduces in the presence of spin nutation. This reduction in the effective damping is stronger in antiferromagnets than in ferromagnets.  These results are analytically calculated using linear-response theory in Ref.\ \cite{Mondal2020nutation}.

% All of the above inertial effects have also been compared with the numerical simulations by solving the ILLG Eqs. (\ref{Antiferro_Eq1}) and (\ref{Antiferro_Eq2}). The comparison of analytical and numerical results show strong agreement \cite{Mondal2020nutation}. Following this analysis, the detection of spin nutation should be %better 
% easier to achieve in antiferromagnets.
% %{However, magnetic inertial dynamics is yet to be detected in antiferromagnets.} 
% %With the model parameters  (resembles to antiferromagnetic NiO or CoO) used in calculations,          




% \subsection{ Ferrimagnets}
% %{\gred \bf RM + LR }
% %{\gred In a ferrimagnet, there are two magnetic sublattices as similar to an antiferromagnet. But, unlike antiferromagnets, the magnetic moments of the two sublattices are different in ferrimagnets.} 
% {In a ferrimagnet, there are two magnetic sublattices possessing different magnetic moments.} The two precession resonance frequencies are therefore distinct in ferrimagnets. Compared to antiferromagnets, these two precession resonance frequencies could be observed  in the GHz and the THz regimes, respectively \cite{Schlickeiser2012}. The resonance in the GHz frequency is called ferromagnetic-like mode, while the other at the THz regime is termed the antiferromagnetic-like mode\cite{Mondal2020nutation}.



% The magnetic inertia additionally introduces two other nutation resonance frequencies in the THz regime \cite{Mondal2020nutation}. These frequencies have been calculated in Ref. \cite{Mondal2020nutation} and they are distinct even if one considers $\eta_A = \eta_B$. On the other hand, due to magnetic inertia, the redshift of the precession frequency is more significant in the antiferromagnetic-like mode. A similar observation can also be noted for the calculation of effective damping in ferrimagnets, i.e., that the antiferromagnetic-mode shows strong reduction of effective damping. So far, there is only one investigation of magnetic inertia in ferrimagnets focusing on resonance frequencies and effective damping \cite{Mondal2020nutation}. No experiments have been performed in ferrimagnetic system to detect the magnetic inertia as well. Therefore, there are many challenges and opportunities of investigating inertia in ferrimagnets.

\section{Nutational spin waves}
\begin{figure*}[tbh!]
    \centering
\includegraphics[width=1\textwidth]{Nutation_wave_magnetost.png}
    \caption{(a) Precessional spin wave without inertia. %(red curve). 
    The blue arrows indicate the motion of the magnetic moments $\boldsymbol{M}_{i}$  in a ferromagnet. (b) Nutational %surface 
    spin wave %(purple curve) 
    with a frequency considerably higher than in (a) plotted with small blue circles on top of the ``frozen'' precessional motion. Panels (a) and (b) from Ref. \cite{Cherkasskii2021}. (c) The dispersion branches of nutational surface spin waves in the THz %terahertz 
    range %(purple curve)
    $\omega_{\textrm{n}}$, the precessional Damon-Eshbach mode without inertia %(red curve)
    $\omega_{\textrm{p}}|_{\eta=0}$, and the precessional Damon-Eshbach mode shifted by inertia %(green curve)
    $\omega_{\textrm{p}}$. 
    The calculation parameters for the thin film are: %$\gamma/\left(2 \pi \right) = 28 \:\rm{GHz} \cdot \rm{T}^{-1}$, 
    ${\mu _0}{M_{\textrm{S}}} =0.9 \:\rm{T}$,  $B_{\textrm{ext}} =0.1 \:\rm{T}$, $\alpha=0.0058$, $\eta = 284 \:\rm{fs}$, film thickness $L =  40\: \rm{nm}$.}
    \label{fig:NutWaves}
\end{figure*}
%In general, the following interactions must be taken into account to describe the dynamics of spin waves: Zeeman, spin-orbit, exchange, and dipole-dipole interactions. 
Due to the interaction between the magnetic moments, the linearized ILLG equation also possesses propagating solutions known as spin waves, illustrated in Fig.\,\ref{fig:NutWaves}(a). %dipole-dipole and exchange coupling, the phase shifts between precessing magnetic moments propagate as a spin wave through ferromagnets (Fig.\,\ref{fig:NutWaves}(a)). 
Taking inertia into account, nutational spin waves also appear alongside the conventional  precessional spin waves. %one finds that the deviation of localized moments will propagate through the spin system in the form of both precessional and nutational motion, i.e., in ferromagnetic materials one needs to add to all ``conventional'' spin wave modes a high frequency wave-like motion with small amplitude caused by the inertia. The latter were called nutational spin waves.
Since nutational spin waves have THz frequencies compared to the typically GHz frequencies of the precessional spin-wave modes in ferromagnets, they can be imagined as a small deviation on top of a ``frozen'' precessional motion, shown in Fig.\,\ref{fig:NutWaves}(b). Conventionally the spin-wave dispersion relation is separated into two regimes: at long wave vectors comparable to the sample sizes magnetostatic effects dominate, while at shorter wavelengths the short-ranged exchange interactions play the most important role.

%{\gred Nutational waves introduced by Kikuchi \textit{et al}.~\cite{Kikuchi} were investigated in the above mentioned regimes.} 
The magnetostatic nutational waves were studied in in-plane magnetized ferromagnetic thin films in Ref.~\cite{Cherkasskii2021}. It was found that for the spin waves propagating perpendicular to the applied magnetic field (Damon--Eshbach configuration), inertial effects on magnons are twofold: the frequency of precessional waves is reduced and %the other, i.e.
nutational surface spin waves emerge, as shown in Fig.~\ref{fig:NutWaves}(c). Notably, nutational spin waves propagate with a group velocity opposite to their wave vector, which is only observed for precessional spin waves with wave vectors parallel to the magnetic field (backward volume modes). %These waves 
%The dispersion of nutation waves start at frequency $\eta^{-1}$, and they propagate with a group velocity significantly lower than the velocity of conventional spin waves.
The interaction of spin waves described by the ILLG equation with electromagnetic waves in ferromagnets was investigated %by S. Titov \textit{et al} 
in Ref.~\cite{Titov2022NutationWaves}. The interaction leads to the hybridization between magnons and photons and the opening of avoided crossings in the spectrum. Since the typical nutational spin-wave frequencies are in the range of $\eta^{-1}\sim 10^{13}-10^{15}$~s$^{-1}$, the wave vectors of electromagnetic waves hybridizing with these modes are around $k\sim 10^{5}-10^{7}$~m$^{-1}$, falling into the magnetostatic regime.



Nutational exchange spin waves were discussed in Refs.~\cite{Kikuchi,Makhfudz2020,Titov2022NutationWaves,Lomonosov2021,Mondal2022}. For a nearest-neighbour ferromagnetic exchange interaction $J$, the dispersion relation may be approximated in the long-wavelength regime %{\gred{($ka\ll 1$, where $k$ is the wave vector and $a$ is the lattice constant)}} 
as
\begin{align}
\omega_{\textrm{p},\boldsymbol{k}}\approx &\frac{zJa^{2}}{2}\boldsymbol{k}^{2}\left(1-\eta \frac{zJa^{2}}{2}\boldsymbol{k}^{2}\right)\, ,\label{eq:ferroprecwave}
\\
\omega_{\textrm{n},\boldsymbol{k}}\approx &-\frac{1}{\eta}-\frac{zJa^{2}}{2}\boldsymbol{k}^{2}\left(1-\eta \frac{zJa^{2}}{2}\boldsymbol{k}^{2}\right)\, \label{eq:ferronutwave}
\end{align}
for precessional and nutational spin waves, respectively. Here, $z$ is the number of nearest neighbours and $a$ is the distance between the corresponding sites. The negative sign of the nutational frequency indicates an opposite handedness compared to the precessional waves~\cite{Kikuchi}, as already mentioned for the $\boldsymbol{k}=\boldsymbol{0}$ FMR mode in Eqs.~\eqref{eqprecFM} and \eqref{eqnutFM}. If the spin-wave dispersion becomes non-reciprocal, for example due to the presence of the Dzyaloshinskii--Moriya interaction, the opposite handedness gives rise to a minimum in the dispersion relation for opposite wave vectors in the two branches~\cite{Mondal2022}. The precessional and nutational branches differ by a constant shift $\eta^{-1}$, and the frequencies of the precessional modes are decreased due to the inertia, as illustrated in Fig.~\ref{fig:NutWaves_Titov}. 
%and their interaction with electromagnetic waves in ferromagnets were investigated by S. Titov \textit{et al} \cite{Titov2022NutationWaves}. Dispersion relations describing the propagation of the hybrid nutation spin waves in an arbitrary direction relative to the applied magnetic field were derived via Maxwell's equations. As in the magnetostatic case, inertia yields the additional nutation dispersion branch. Furthermore, it was found that the dispersion slope of conventional exchange spin waves deviates from the non-inertial parabolic dependence $\omega_{\rm{FMR}}+A k^2$ (compare red and green curves in Fig.\,\ref{fig:NutWaves_Titov}). Similar effects were found in ferromagnetic nanostructures with uniaxial anisotropy taking inertia into account \cite{Lomonosov2021}.

\begin{figure}[t]
    \centering
    \includegraphics[]{Nut_Exchange_El_SW.png}
    \caption{Dispersion relation of exchange spin waves without and with inertia in ferromagnets: nutational spin waves %(purple)
    $\omega_{\textrm{n}}$, inertial precessional spin waves %(green)
    $\omega_{\textrm{p}}$, and non-inertial precessional spin waves %(red)
    $\omega_{\textrm{p}}|_{\eta=0}$. Here $\omega_M = \gamma  \mu_0 M_{\textrm{S}}$, $k_M = \sqrt{\varepsilon_r} \omega_M / c$, where $c$ is the speed of light and $\varepsilon_r$ is the relative permittivity of ferromagnets. The calculation parameters are the following: %$\gamma/\left(2 \pi \right) = 28 \:\rm{GHz} \cdot \rm{T}^{-1}$, 
    ${\mu _0}{M_{\textrm{S}}} =0.2 \:\rm{T}$,  $B_{\textrm{ext}} =0.1 \:\rm{T}$, $\eta = 1 \:\rm{ps}$, and $\varepsilon_r =  15.5$.}
    \label{fig:NutWaves_Titov}
\end{figure}

%Nutation waves were discovered in Heisenberg spin chains with the isotropic exchange interaction, where the single-particle excitations were identified as relativistic massive particles \cite{Makhfudz2020}. These particles, which were called ``nutatons'', acquire their mass via the Brout--Englert--Higgs mechanism, through the interaction of the wave with an emergent topological gauge field. 

%In the last decades, a variety of linear \cite{stancil2009spin, Rozsa2017Temperature, Rozsa2018Effective} and nonlinear spin-wave effects \cite{lvov2012wave, ordonez2013direct, wang2014formation, wang2015observation} were investigated in thin and ultrathin magnetic films \cite{kalinikos1986theory, Rozsa2016Domain}, layered magnetic structures \cite{hillebrands1990spin, Rozsa2018Localized, cherkasskii2013envelope}, and periodic magnetic crystals \cite{chumak2009spin, drozdovskii2010formation}, neglecting inertial spin dynamics. Since the inclusion of spin inertia into the consideration dramatically changes the viewpoint, future experimental and theoretical studies are required.

%\section{Nonlinear inertial dynamics}
%{\gred \bf RM + MC + LR}

Although the dispersion relation of the precessional spin waves is modified by inertia, similar frequency shifts may also be explained within the LLG equation by choosing a different saturation magnetization, magnetocrystalline anisotropy term or taking exchange interactions with further neighbours into account. Unless these parameters are known from independent measurements of static properties which are not expected to be affected by the inertia, such as the temperature dependence of the magnetization or the critical temperature, a measurement of the precessional branch only is unlikely to result in a convincing indication for the prevalence of inertial phenomena. The same argument holds for the group velocity, the gyromagnetic ratio or the effective damping parameter which are also influenced by the inertia~\cite{Mondal2022,Lomonosov2021,Makhfudz2020}, as mentioned above in the case of the ferromagnetic resonance. Experimental results on nutational spin waves are not available at the moment, but they could provide sufficient evidence for the theoretical predictions based on the ILLG equation. Of particular interest would be the investigation of nutational spin-wave modes with a group velocity opposite to their wave vector, as discussed in the Damon--Eshbach configuration for ferromagnets above and for exchange spin waves in antiferromagnets in Ref.~\cite{Mondal2022}.

\section{Magnetization switching in the inertial regime}

While linear-response and linear spin-wave theories describe the time evolution close to the equilibrium state, the switching between different equilibrium states is a non-linear effect that is also influenced by the inertial dynamics. The ILLG equation was solved numerically for a single uniaxial macrospin under the influence of a magnetic field pulse with zero frequency perpendicular to the easy-axis direction in Ref.~\cite{neeraj2021magnetization}. The switching time was found to be lower in a wide range of pulse durations in the inertial case than for the non-inertial LLG equation. However, as emphasized before such a quantitative effect may be difficult to probe experimentally where inertial and non-inertial dynamics cannot be compared directly. 

In Ref.~\cite{Winter2022}, it was found by combining analytical calculations and numerical simulations that a resonant excitation of the nutation amplitude gives rise to a torque that is capable of switching the magnetic moment, which effect is unparalleled in the LLG equation. This phenomenon is illustrated in Fig.~\ref{fig:fig8}. Since the switching velocity was found to be proportional to the square of the nutation amplitude which scales with the amplitude of the ac excitation field itself, the velocity increases quadratically with the field amplitude instead of linearly in the case of switching based on the Larmor precession. This enables lower switching times compared to precessional switching for intermediate field strengths. Furthermore, it was demonstrated that both $90^{\circ}$ and $180^{\circ}$ switching may be achieved for a single macrospin depending on the linear or circular polarization of the excitation field, while in antiferromagnets switching to states with the magnetic moments either perpendicular to or in the plane of the excitation field may be realized depending on the excitation frequency. 

\begin{figure}[tbh!]
    \centering
    \includegraphics[width=\columnwidth]{nutationalswitching.png}
    \caption{Illustration of magnetization switching in the inertial regime. (a) The angular momentum $\boldsymbol{L}_{i}$ and the magnetic moment $\gamma^{-1}\boldsymbol{M}_{i}$ differ by the nutation vector $\Delta\boldsymbol{L}_{i}$ due to the inertial dynamics. The nutation is excited by the oscillating external magnetic field $\mu_{0}\boldsymbol{H}_{\textrm{osc}}$ resonantly at the nutation frequency $\omega_{\textrm{n}}$. (b)-(d) The oscillating field exerts a torque $-\gamma\Delta\boldsymbol{L}_{i}\times\mu_{0}\boldsymbol{H}_{\textrm{osc}}$ on the angular momentum. Since the nutation vector follows the excitation field with a finite phase shift, the average of the torque over a period of the excitation is finite. (e) Over several nutation periods, the torque causes a switching of the angular momentum and the magnetic moment. Figure from Ref.~\cite{Winter2022}.}
    \label{fig:fig8}
\end{figure}

Based on the solution of the Fokker--Planck equation for the inertial Landau--Lifshitz--Gilbert--Bloch equation, it was argued in Ref.~\cite{Makhfudz2022} that inertial effects together with thermal excitations could explain puzzling effects observed in all-optical magnetization switching, including its dependence on the polarization of the laser pulse.

%\section{Future of inertial dynamics}
%{\gred \bf RM + MC + LR}

\section{Conclusion}
% %{\gred \bf RM + MC + LR}

We reviewed how the inclusion of an inertial term in the quasiclassical Landau--Lifshitz--Gilbert equation influences the resonance frequencies and the spin-wave modes in the linear-response regime, as well as the switching times in magnetic nanoparticles at ultrafast time scales. Apart from quantitative changes compared to the Landau--Lifshitz--Gilbert dynamics, the inertia gives rise to qualitatively new phenomena such as nutational spin waves and resonances, the excitation of which opens up faster paths for the reversal of the magnetic moments. Precisely these new phenomena provide the most promising way to detect signatures of the inertia, since the quantitative changes may also be interpreted using a different choice of parameters within the LLG dynamics, and the values of these parameters are not known a priori.

Experimental observations of these effects have been restricted to nutational resonances in ferromagnets so far~\cite{neeraj2019experimental,unikandanunni2021inertial}, which have considerably higher frequencies compared to the typical precessional modes. Optical methods appear to be particularly suitable for obtaining further experimental evidence on inertial effects, because the frequency of the electromagnetic waves used in these methods falls into the range where nutational spin waves are expected to emerge, and the possible control over their polarization may enable different switching paths. Antiferromagnets may be particularly appealing for this purpose, since precessional and nutational spin waves in them are located in the same frequency range, possibly enabling their simultaneous observation using the same technique. The requirement for adjusting the frequency represents a challenge, in particular because estimates for the nutational resonance frequency in the literature differ by two orders of magnitude for similar materials. Further first-principles calculations of the inertial relaxation time may provide guidance on the choice of materials for the experiments. Theoretical calculations based on microscopic models of the interactions of the magnetic degrees of freedom with electrons, phonons or photons could provide important comparisons with the quasiclassical description discussed here. A joint effort from the experimental and theoretical sides is expected to provide valuable insight into the limits of applicability of inertial spin dynamics at femtosecond time scales and beyond.

%The emergence of nutation resonance through magnetic inertial dynamics has entered the spotlight in ultrafast manipulation of spins.  Several theoretical and experimental works have already laid down the foundations of inertial dynamics. However, this research area is still %belongs to 
%in its initial phase. %days.
%There is a need for further theoretical and experimental investigations to have a fundamental understanding and possible applications.  

%One of the main challenges is to understand the role of inertia in resonance experiments and in resonance and wave phenomena in several classes of magnetic materials. The connection between the Gilbert damping parameter and the inertial relaxation time is not revealed at the microscopic level yet. The experimental detection of nutation waves in ferromagnets and antiferromagnets can open new paths for realizing ultrafast spin dynamics. In addition, nutation waves have not been investigated in multilayered magnetic structures.
%The effects of inertia on various non-linear phenomena, namely three- and four-magnon interactions, chaotic dynamics and soliton formation, also hold promising prospects.


\section*{Acknowledgments}
We are grateful to Anna Semisalova, Igor Barsukov, Jean-Eric Wegrowe and Sebastian T. B. Goennenwein for fruitful discussions. Financial support by the faculty research scheme at IIT (ISM) Dhanbad, India under Project No. FRS(196)/2023-2024/PHYSICS, by the National Research, Development, and Innovation Office (NRDI) of Hungary under Project Nos. K131938 and FK142601, by the Young Scholar Fund at the University of Konstanz, by the Ministry of Culture and Innovation; National Research, Development and Innovation Office within the Quantum Information National Laboratory of Hungary (Grant No. 2022-2.1.1-NL-2022-00004), by the Swedish Research Council (VR), and by the K. and A. Wallenberg Foundation (Grant No. 2022.0079) is gratefully acknowledged.

 %Financial support by the National Research, Development, and Innovation Office (NRDI) of Hungary under Project No. K131938 and by the Young Scholar Fund at the University of Konstanz is gratefully acknowledged.

%\bibliography{Ref}
%\bibliographystyle{apsrev4-2}

%\end{document}%apsrev4-2.bst 2019-01-14 (MD) hand-edited version of apsrev4-1.bst
%Control: key (0)
%Control: author (8) initials jnrlst
%Control: editor formatted (1) identically to author
%Control: production of article title (0) allowed
%Control: page (0) single
%Control: year (1) truncated
%Control: production of eprint (0) enabled
\providecommand{\noopsort}[1]{}\providecommand{\singleletter}[1]{#1}%
\begin{thebibliography}{81}%
\makeatletter
\providecommand \@ifxundefined [1]{%
 \@ifx{#1\undefined}
}%
\providecommand \@ifnum [1]{%
 \ifnum #1\expandafter \@firstoftwo
 \else \expandafter \@secondoftwo
 \fi
}%
\providecommand \@ifx [1]{%
 \ifx #1\expandafter \@firstoftwo
 \else \expandafter \@secondoftwo
 \fi
}%
\providecommand \natexlab [1]{#1}%
\providecommand \enquote  [1]{``#1''}%
\providecommand \bibnamefont  [1]{#1}%
\providecommand \bibfnamefont [1]{#1}%
\providecommand \citenamefont [1]{#1}%
\providecommand \href@noop [0]{\@secondoftwo}%
\providecommand \href [0]{\begingroup \@sanitize@url \@href}%
\providecommand \@href[1]{\@@startlink{#1}\@@href}%
\providecommand \@@href[1]{\endgroup#1\@@endlink}%
\providecommand \@sanitize@url [0]{\catcode `\\12\catcode `\$12\catcode
  `\&12\catcode `\#12\catcode `\^12\catcode `\_12\catcode `\%12\relax}%
\providecommand \@@startlink[1]{}%
\providecommand \@@endlink[0]{}%
\providecommand \url  [0]{\begingroup\@sanitize@url \@url }%
\providecommand \@url [1]{\endgroup\@href {#1}{\urlprefix }}%
\providecommand \urlprefix  [0]{URL }%
\providecommand \Eprint [0]{\href }%
\providecommand \doibase [0]{https://doi.org/}%
\providecommand \selectlanguage [0]{\@gobble}%
\providecommand \bibinfo  [0]{\@secondoftwo}%
\providecommand \bibfield  [0]{\@secondoftwo}%
\providecommand \translation [1]{[#1]}%
\providecommand \BibitemOpen [0]{}%
\providecommand \bibitemStop [0]{}%
\providecommand \bibitemNoStop [0]{.\EOS\space}%
\providecommand \EOS [0]{\spacefactor3000\relax}%
\providecommand \BibitemShut  [1]{\csname bibitem#1\endcsname}%
\let\auto@bib@innerbib\@empty
%</preamble>
\bibitem [{\citenamefont {Siegmann}\ \emph {et~al.}(1995)\citenamefont
  {Siegmann}, \citenamefont {Garwin}, \citenamefont {Prescott}, \citenamefont
  {Heidmann}, \citenamefont {Mauri}, \citenamefont {Weller}, \citenamefont
  {Allenspach},\ and\ \citenamefont {Weber}}]{SIEGMANN1995L8}%
  \BibitemOpen
  \bibfield  {author} {\bibinfo {author} {\bibfnamefont {H.~C.}\ \bibnamefont
  {Siegmann}}, \bibinfo {author} {\bibfnamefont {E.~L.}\ \bibnamefont
  {Garwin}}, \bibinfo {author} {\bibfnamefont {C.~Y.}\ \bibnamefont
  {Prescott}}, \bibinfo {author} {\bibfnamefont {J.}~\bibnamefont {Heidmann}},
  \bibinfo {author} {\bibfnamefont {D.}~\bibnamefont {Mauri}}, \bibinfo
  {author} {\bibfnamefont {D.}~\bibnamefont {Weller}}, \bibinfo {author}
  {\bibfnamefont {R.}~\bibnamefont {Allenspach}},\ and\ \bibinfo {author}
  {\bibfnamefont {W.}~\bibnamefont {Weber}},\ }\bibfield  {title} {\bibinfo
  {title} {Magnetism with picosecond field pulses},\ }\href
  {https://doi.org/http://dx.doi.org/10.1016/0304-8853(95)00602-8} {\bibfield
  {journal} {\bibinfo  {journal} {J. Magn. Magn. Mater.}\ }\textbf {\bibinfo
  {volume} {151}},\ \bibinfo {pages} {L8 } (\bibinfo {year}
  {1995})}\BibitemShut {NoStop}%
\bibitem [{\citenamefont {Beaurepaire}\ \emph {et~al.}(1996)\citenamefont
  {Beaurepaire}, \citenamefont {Merle}, \citenamefont {Daunois},\ and\
  \citenamefont {Bigot}}]{Bigot1996}%
  \BibitemOpen
  \bibfield  {author} {\bibinfo {author} {\bibfnamefont {E.}~\bibnamefont
  {Beaurepaire}}, \bibinfo {author} {\bibfnamefont {J.-C.}\ \bibnamefont
  {Merle}}, \bibinfo {author} {\bibfnamefont {A.}~\bibnamefont {Daunois}},\
  and\ \bibinfo {author} {\bibfnamefont {J.-Y.}\ \bibnamefont {Bigot}},\
  }\bibfield  {title} {\bibinfo {title} {Ultrafast spin dynamics in
  ferromagnetic nickel},\ }\href {https://doi.org/10.1103/PhysRevLett.76.4250}
  {\bibfield  {journal} {\bibinfo  {journal} {Phys. Rev. Lett.}\ }\textbf
  {\bibinfo {volume} {76}},\ \bibinfo {pages} {4250} (\bibinfo {year}
  {1996})}\BibitemShut {NoStop}%
\bibitem [{\citenamefont {Stanciu}\ \emph {et~al.}(2007)\citenamefont
  {Stanciu}, \citenamefont {Hansteen}, \citenamefont {Kimel}, \citenamefont
  {Kirilyuk}, \citenamefont {Tsukamoto}, \citenamefont {Itoh},\ and\
  \citenamefont {Rasing}}]{Stanciu2007}%
  \BibitemOpen
  \bibfield  {author} {\bibinfo {author} {\bibfnamefont {C.~D.}\ \bibnamefont
  {Stanciu}}, \bibinfo {author} {\bibfnamefont {F.}~\bibnamefont {Hansteen}},
  \bibinfo {author} {\bibfnamefont {A.~V.}\ \bibnamefont {Kimel}}, \bibinfo
  {author} {\bibfnamefont {A.}~\bibnamefont {Kirilyuk}}, \bibinfo {author}
  {\bibfnamefont {A.}~\bibnamefont {Tsukamoto}}, \bibinfo {author}
  {\bibfnamefont {A.}~\bibnamefont {Itoh}},\ and\ \bibinfo {author}
  {\bibfnamefont {{\rm Th}.}~\bibnamefont {Rasing}},\ }\bibfield  {title}
  {\bibinfo {title} {All-optical magnetic recording with circularly polarized
  light},\ }\href {https://doi.org/10.1103/PhysRevLett.99.047601} {\bibfield
  {journal} {\bibinfo  {journal} {Phys. Rev. Lett.}\ }\textbf {\bibinfo
  {volume} {99}},\ \bibinfo {pages} {047601} (\bibinfo {year}
  {2007})}\BibitemShut {NoStop}%
\bibitem [{\citenamefont {Radu}\ \emph {et~al.}(2011)\citenamefont {Radu},
  \citenamefont {Vahaplar}, \citenamefont {Stamm}, \citenamefont {Kachel},
  \citenamefont {Pontius}, \citenamefont {D{\"u}rr}, \citenamefont {Ostler},
  \citenamefont {Barker}, \citenamefont {Evans}, \citenamefont {Chantrell},
  \citenamefont {Tsukamoto}, \citenamefont {Itoh}, \citenamefont {Kirilyuk},
  \citenamefont {Rasing},\ and\ \citenamefont {Kimel}}]{Radu2011}%
  \BibitemOpen
  \bibfield  {author} {\bibinfo {author} {\bibfnamefont {I.}~\bibnamefont
  {Radu}}, \bibinfo {author} {\bibfnamefont {K.}~\bibnamefont {Vahaplar}},
  \bibinfo {author} {\bibfnamefont {C.}~\bibnamefont {Stamm}}, \bibinfo
  {author} {\bibfnamefont {T.}~\bibnamefont {Kachel}}, \bibinfo {author}
  {\bibfnamefont {N.}~\bibnamefont {Pontius}}, \bibinfo {author} {\bibfnamefont
  {H.~A.}\ \bibnamefont {D{\"u}rr}}, \bibinfo {author} {\bibfnamefont {T.~A.}\
  \bibnamefont {Ostler}}, \bibinfo {author} {\bibfnamefont {J.}~\bibnamefont
  {Barker}}, \bibinfo {author} {\bibfnamefont {R.~F.~L.}\ \bibnamefont
  {Evans}}, \bibinfo {author} {\bibfnamefont {R.~W.}\ \bibnamefont
  {Chantrell}}, \bibinfo {author} {\bibfnamefont {A.}~\bibnamefont
  {Tsukamoto}}, \bibinfo {author} {\bibfnamefont {A.}~\bibnamefont {Itoh}},
  \bibinfo {author} {\bibfnamefont {A.}~\bibnamefont {Kirilyuk}}, \bibinfo
  {author} {\bibfnamefont {T.}~\bibnamefont {Rasing}},\ and\ \bibinfo {author}
  {\bibfnamefont {A.~V.}\ \bibnamefont {Kimel}},\ }\bibfield  {title} {\bibinfo
  {title} {Transient ferromagnetic-like state mediating ultrafast reversal of
  antiferromagnetically coupled spins},\ }\href
  {https://doi.org/10.1038/nature09901} {\bibfield  {journal} {\bibinfo
  {journal} {Nature}\ }\textbf {\bibinfo {volume} {472}},\ \bibinfo {pages}
  {205} (\bibinfo {year} {2011})}\BibitemShut {NoStop}%
\bibitem [{\citenamefont {Ostler}\ \emph {et~al.}(2012)\citenamefont {Ostler},
  \citenamefont {Barker}, \citenamefont {Evans}, \citenamefont {Chantrell},
  \citenamefont {Atxitia}, \citenamefont {Chubykalo-Fesenko}, \citenamefont
  {El~Moussaoui}, \citenamefont {Le~Guyader}, \citenamefont {Mengotti},
  \citenamefont {Heyderman}, \citenamefont {Nolting}, \citenamefont
  {Tsukamoto}, \citenamefont {Itoh}, \citenamefont {Afanasiev}, \citenamefont
  {Ivanov}, \citenamefont {Kalashnikova}, \citenamefont {Vahaplar},
  \citenamefont {Mentink}, \citenamefont {Kirilyuk}, \citenamefont {Rasing},\
  and\ \citenamefont {Kimel}}]{ostler12}%
  \BibitemOpen
  \bibfield  {author} {\bibinfo {author} {\bibfnamefont {T.~A.}\ \bibnamefont
  {Ostler}}, \bibinfo {author} {\bibfnamefont {J.}~\bibnamefont {Barker}},
  \bibinfo {author} {\bibfnamefont {R.~F.~L.}\ \bibnamefont {Evans}}, \bibinfo
  {author} {\bibfnamefont {R.~W.}\ \bibnamefont {Chantrell}}, \bibinfo {author}
  {\bibfnamefont {U.}~\bibnamefont {Atxitia}}, \bibinfo {author} {\bibfnamefont
  {O.}~\bibnamefont {Chubykalo-Fesenko}}, \bibinfo {author} {\bibfnamefont
  {S.}~\bibnamefont {El~Moussaoui}}, \bibinfo {author} {\bibfnamefont
  {L.}~\bibnamefont {Le~Guyader}}, \bibinfo {author} {\bibfnamefont
  {E.}~\bibnamefont {Mengotti}}, \bibinfo {author} {\bibfnamefont {L.~J.}\
  \bibnamefont {Heyderman}}, \bibinfo {author} {\bibfnamefont {F.}~\bibnamefont
  {Nolting}}, \bibinfo {author} {\bibfnamefont {A.}~\bibnamefont {Tsukamoto}},
  \bibinfo {author} {\bibfnamefont {A.}~\bibnamefont {Itoh}}, \bibinfo {author}
  {\bibfnamefont {D.}~\bibnamefont {Afanasiev}}, \bibinfo {author}
  {\bibfnamefont {B.~A.}\ \bibnamefont {Ivanov}}, \bibinfo {author}
  {\bibfnamefont {A.~M.}\ \bibnamefont {Kalashnikova}}, \bibinfo {author}
  {\bibfnamefont {K.}~\bibnamefont {Vahaplar}}, \bibinfo {author}
  {\bibfnamefont {J.}~\bibnamefont {Mentink}}, \bibinfo {author} {\bibfnamefont
  {A.}~\bibnamefont {Kirilyuk}}, \bibinfo {author} {\bibfnamefont {{\rm
  Th}.}~\bibnamefont {Rasing}},\ and\ \bibinfo {author} {\bibfnamefont {A.~V.}\
  \bibnamefont {Kimel}},\ }\bibfield  {title} {\bibinfo {title} {Ultrafast
  heating as a sufficient stimulus for magnetization reversal in a
  ferrimagnet},\ }\href
  {https://www.nature.com/articles/ncomms1666#:~:text=Here%20we%20show%20numerically%20and,presence%20of%20a%20magnetic%20field.}
  {\bibfield  {journal} {\bibinfo  {journal} {Nat. Commun.}\ }\textbf {\bibinfo
  {volume} {3}},\ \bibinfo {pages} {666} (\bibinfo {year} {2012})}\BibitemShut
  {NoStop}%
\bibitem [{\citenamefont {Le~Guyader}\ \emph {et~al.}(2015)\citenamefont
  {Le~Guyader}, \citenamefont {Savoini}, \citenamefont {El~Moussaoui},
  \citenamefont {Buzzi}, \citenamefont {Tsukamoto}, \citenamefont {Itoh},
  \citenamefont {Kirilyuk}, \citenamefont {Rasing}, \citenamefont {Kimel},\
  and\ \citenamefont {Nolting}}]{LeGuyader2015}%
  \BibitemOpen
  \bibfield  {author} {\bibinfo {author} {\bibfnamefont {L.}~\bibnamefont
  {Le~Guyader}}, \bibinfo {author} {\bibfnamefont {M.}~\bibnamefont {Savoini}},
  \bibinfo {author} {\bibfnamefont {S.}~\bibnamefont {El~Moussaoui}}, \bibinfo
  {author} {\bibfnamefont {M.}~\bibnamefont {Buzzi}}, \bibinfo {author}
  {\bibfnamefont {A.}~\bibnamefont {Tsukamoto}}, \bibinfo {author}
  {\bibfnamefont {A.}~\bibnamefont {Itoh}}, \bibinfo {author} {\bibfnamefont
  {A.}~\bibnamefont {Kirilyuk}}, \bibinfo {author} {\bibfnamefont
  {T.}~\bibnamefont {Rasing}}, \bibinfo {author} {\bibfnamefont {A.~V.}\
  \bibnamefont {Kimel}},\ and\ \bibinfo {author} {\bibfnamefont
  {F.}~\bibnamefont {Nolting}},\ }\bibfield  {title} {\bibinfo {title}
  {{Nanoscale sub-100 picosecond all-optical magnetization switching in GdFeCo
  microstructures}},\ }\href {https://doi.org/10.1038/ncomms6839} {\bibfield
  {journal} {\bibinfo  {journal} {Nat. Commun.}\ }\textbf {\bibinfo {volume}
  {6}},\ \bibinfo {pages} {5839} (\bibinfo {year} {2015})}\BibitemShut
  {NoStop}%
\bibitem [{\citenamefont {Mangin}\ \emph {et~al.}(2014)\citenamefont {Mangin},
  \citenamefont {Gottwald}, \citenamefont {Lambert}, \citenamefont {Steil},
  \citenamefont {Uhl{\'i}{\v r}}, \citenamefont {Pang}, \citenamefont {Hehn},
  \citenamefont {Alebrand}, \citenamefont {Cinchetti}, \citenamefont
  {Malinowski}, \citenamefont {Fainman}, \citenamefont {Aeschlimann},\ and\
  \citenamefont {Fullerton}}]{Mangin2014}%
  \BibitemOpen
  \bibfield  {author} {\bibinfo {author} {\bibfnamefont {S.}~\bibnamefont
  {Mangin}}, \bibinfo {author} {\bibfnamefont {M.}~\bibnamefont {Gottwald}},
  \bibinfo {author} {\bibfnamefont {C.-H.}\ \bibnamefont {Lambert}}, \bibinfo
  {author} {\bibfnamefont {D.}~\bibnamefont {Steil}}, \bibinfo {author}
  {\bibfnamefont {V.}~\bibnamefont {Uhl{\'i}{\v r}}}, \bibinfo {author}
  {\bibfnamefont {L.}~\bibnamefont {Pang}}, \bibinfo {author} {\bibfnamefont
  {M.}~\bibnamefont {Hehn}}, \bibinfo {author} {\bibfnamefont {S.}~\bibnamefont
  {Alebrand}}, \bibinfo {author} {\bibfnamefont {M.}~\bibnamefont {Cinchetti}},
  \bibinfo {author} {\bibfnamefont {G.}~\bibnamefont {Malinowski}}, \bibinfo
  {author} {\bibfnamefont {Y.}~\bibnamefont {Fainman}}, \bibinfo {author}
  {\bibfnamefont {M.}~\bibnamefont {Aeschlimann}},\ and\ \bibinfo {author}
  {\bibfnamefont {E.~E.}\ \bibnamefont {Fullerton}},\ }\bibfield  {title}
  {\bibinfo {title} {Engineered materials for all-optical helicity-dependent
  magnetic switching},\ }\href {http://dx.doi.org/10.1038/nmat3864} {\bibfield
  {journal} {\bibinfo  {journal} {Nat. Mater.}\ }\textbf {\bibinfo {volume}
  {13}},\ \bibinfo {pages} {286} (\bibinfo {year} {2014})}\BibitemShut
  {NoStop}%
\bibitem [{\citenamefont {Koplak}\ \emph {et~al.}(2017)\citenamefont {Koplak},
  \citenamefont {Talantsev}, \citenamefont {Lu}, \citenamefont {Hamadeh},
  \citenamefont {Pirro}, \citenamefont {Hauet}, \citenamefont {Morgunov},\ and\
  \citenamefont {Mangin}}]{KOPLAK2017}%
  \BibitemOpen
  \bibfield  {author} {\bibinfo {author} {\bibfnamefont {O.}~\bibnamefont
  {Koplak}}, \bibinfo {author} {\bibfnamefont {A.}~\bibnamefont {Talantsev}},
  \bibinfo {author} {\bibfnamefont {Y.}~\bibnamefont {Lu}}, \bibinfo {author}
  {\bibfnamefont {A.}~\bibnamefont {Hamadeh}}, \bibinfo {author} {\bibfnamefont
  {P.}~\bibnamefont {Pirro}}, \bibinfo {author} {\bibfnamefont
  {T.}~\bibnamefont {Hauet}}, \bibinfo {author} {\bibfnamefont
  {R.}~\bibnamefont {Morgunov}},\ and\ \bibinfo {author} {\bibfnamefont
  {S.}~\bibnamefont {Mangin}},\ }\bibfield  {title} {\bibinfo {title}
  {{Magnetization switching diagram of a perpendicular synthetic ferrimagnet
  CoFeB/Ta/CoFeB bilayer}},\ }\href
  {https://doi.org/https://doi.org/10.1016/j.jmmm.2017.02.047} {\bibfield
  {journal} {\bibinfo  {journal} {J. Magn. Magn. Mater.}\ }\textbf {\bibinfo
  {volume} {433}},\ \bibinfo {pages} {91} (\bibinfo {year} {2017})}\BibitemShut
  {NoStop}%
\bibitem [{\citenamefont {Vahaplar}\ \emph {et~al.}(2009)\citenamefont
  {Vahaplar}, \citenamefont {Kalashnikova}, \citenamefont {Kimel},
  \citenamefont {Hinzke}, \citenamefont {Nowak}, \citenamefont {Chantrell},
  \citenamefont {Tsukamoto}, \citenamefont {Itoh}, \citenamefont {Kirilyuk},\
  and\ \citenamefont {Rasing}}]{Vahaplar2009}%
  \BibitemOpen
  \bibfield  {author} {\bibinfo {author} {\bibfnamefont {K.}~\bibnamefont
  {Vahaplar}}, \bibinfo {author} {\bibfnamefont {A.~M.}\ \bibnamefont
  {Kalashnikova}}, \bibinfo {author} {\bibfnamefont {A.~V.}\ \bibnamefont
  {Kimel}}, \bibinfo {author} {\bibfnamefont {D.}~\bibnamefont {Hinzke}},
  \bibinfo {author} {\bibfnamefont {U.}~\bibnamefont {Nowak}}, \bibinfo
  {author} {\bibfnamefont {R.}~\bibnamefont {Chantrell}}, \bibinfo {author}
  {\bibfnamefont {A.}~\bibnamefont {Tsukamoto}}, \bibinfo {author}
  {\bibfnamefont {A.}~\bibnamefont {Itoh}}, \bibinfo {author} {\bibfnamefont
  {A.}~\bibnamefont {Kirilyuk}},\ and\ \bibinfo {author} {\bibfnamefont
  {T.}~\bibnamefont {Rasing}},\ }\bibfield  {title} {\bibinfo {title}
  {Ultrafast path for optical magnetization reversal via a strongly
  nonequilibrium state},\ }\href
  {https://doi.org/10.1103/PhysRevLett.103.117201} {\bibfield  {journal}
  {\bibinfo  {journal} {Phys. Rev. Lett.}\ }\textbf {\bibinfo {volume} {103}},\
  \bibinfo {pages} {117201} (\bibinfo {year} {2009})}\BibitemShut {NoStop}%
\bibitem [{\citenamefont {Mentink}\ \emph {et~al.}(2012)\citenamefont
  {Mentink}, \citenamefont {Hellsvik}, \citenamefont {Afanasiev}, \citenamefont
  {Ivanov}, \citenamefont {Kirilyuk}, \citenamefont {Kimel}, \citenamefont
  {Eriksson}, \citenamefont {Katsnelson},\ and\ \citenamefont
  {Rasing}}]{Mentink2012}%
  \BibitemOpen
  \bibfield  {author} {\bibinfo {author} {\bibfnamefont {J.~H.}\ \bibnamefont
  {Mentink}}, \bibinfo {author} {\bibfnamefont {J.}~\bibnamefont {Hellsvik}},
  \bibinfo {author} {\bibfnamefont {D.~V.}\ \bibnamefont {Afanasiev}}, \bibinfo
  {author} {\bibfnamefont {B.~A.}\ \bibnamefont {Ivanov}}, \bibinfo {author}
  {\bibfnamefont {A.}~\bibnamefont {Kirilyuk}}, \bibinfo {author}
  {\bibfnamefont {A.~V.}\ \bibnamefont {Kimel}}, \bibinfo {author}
  {\bibfnamefont {O.}~\bibnamefont {Eriksson}}, \bibinfo {author}
  {\bibfnamefont {M.~I.}\ \bibnamefont {Katsnelson}},\ and\ \bibinfo {author}
  {\bibfnamefont {T.}~\bibnamefont {Rasing}},\ }\bibfield  {title} {\bibinfo
  {title} {Ultrafast spin dynamics in multisublattice magnets},\ }\href
  {https://doi.org/10.1103/PhysRevLett.108.057202} {\bibfield  {journal}
  {\bibinfo  {journal} {Phys. Rev. Lett.}\ }\textbf {\bibinfo {volume} {108}},\
  \bibinfo {pages} {057202} (\bibinfo {year} {2012})}\BibitemShut {NoStop}%
\bibitem [{\citenamefont {Landau}\ and\ \citenamefont
  {Lifshitz}(1935)}]{landau35}%
  \BibitemOpen
  \bibfield  {author} {\bibinfo {author} {\bibfnamefont {L.~D.}\ \bibnamefont
  {Landau}}\ and\ \bibinfo {author} {\bibfnamefont {E.~M.}\ \bibnamefont
  {Lifshitz}},\ }\bibfield  {title} {\bibinfo {title} {On the theory of the
  dispersion of magnetic permeability in ferromagnetic bodies},\ }\href@noop {}
  {\bibfield  {journal} {\bibinfo  {journal} {Phys. Z. Sowjetunion}\ }\textbf
  {\bibinfo {volume} {8}},\ \bibinfo {pages} {153} (\bibinfo {year}
  {1935})}\BibitemShut {NoStop}%
\bibitem [{\citenamefont {{Gilbert}}(2004)}]{Gilbert2004}%
  \BibitemOpen
  \bibfield  {author} {\bibinfo {author} {\bibfnamefont {T.~L.}\ \bibnamefont
  {{Gilbert}}},\ }\bibfield  {title} {\bibinfo {title} {A phenomenological
  theory of damping in ferromagnetic materials},\ }\href
  {https://doi.org/10.1109/TMAG.2004.836740} {\bibfield  {journal} {\bibinfo
  {journal} {IEEE Trans. Magn.}\ }\textbf {\bibinfo {volume} {40}},\ \bibinfo
  {pages} {3443} (\bibinfo {year} {2004})}\BibitemShut {NoStop}%
\bibitem [{\citenamefont {Brown}(1963)}]{brown1963micromagnetics}%
  \BibitemOpen
  \bibfield  {author} {\bibinfo {author} {\bibfnamefont {W.}~\bibnamefont
  {Brown}},\ }\href {https://books.google.se/books?id=KvuXnAEACAAJ} {\emph
  {\bibinfo {title} {Micromagnetics}}},\ Interscience tracts on physics and
  astronomy\ (\bibinfo  {publisher} {Interscience Publishers},\ \bibinfo
  {address} {New York},\ \bibinfo {year} {1963})\BibitemShut {NoStop}%
\bibitem [{\citenamefont {Barker}\ \emph {et~al.}(2013)\citenamefont {Barker},
  \citenamefont {Atxitia}, \citenamefont {Ostler}, \citenamefont {Hovorka},
  \citenamefont {Chubykalo-Fesenko},\ and\ \citenamefont
  {Chantrell}}]{Barker2013}%
  \BibitemOpen
  \bibfield  {author} {\bibinfo {author} {\bibfnamefont {J.}~\bibnamefont
  {Barker}}, \bibinfo {author} {\bibfnamefont {U.}~\bibnamefont {Atxitia}},
  \bibinfo {author} {\bibfnamefont {T.~A.}\ \bibnamefont {Ostler}}, \bibinfo
  {author} {\bibfnamefont {O.}~\bibnamefont {Hovorka}}, \bibinfo {author}
  {\bibfnamefont {O.}~\bibnamefont {Chubykalo-Fesenko}},\ and\ \bibinfo
  {author} {\bibfnamefont {R.~W.}\ \bibnamefont {Chantrell}},\ }\bibfield
  {title} {\bibinfo {title} {{Two-magnon bound state causes ultrafast thermally
  induced magnetisation switching}},\ }\href
  {https://doi.org/10.1038/srep03262} {\bibfield  {journal} {\bibinfo
  {journal} {Sci. Rep.}\ }\textbf {\bibinfo {volume} {3}},\ \bibinfo {pages}
  {3262} (\bibinfo {year} {2013})}\BibitemShut {NoStop}%
\bibitem [{\citenamefont {Wienholdt}\ \emph {et~al.}(2013)\citenamefont
  {Wienholdt}, \citenamefont {Hinzke}, \citenamefont {Carva}, \citenamefont
  {Oppeneer},\ and\ \citenamefont {Nowak}}]{Wienholdt2013}%
  \BibitemOpen
  \bibfield  {author} {\bibinfo {author} {\bibfnamefont {S.}~\bibnamefont
  {Wienholdt}}, \bibinfo {author} {\bibfnamefont {D.}~\bibnamefont {Hinzke}},
  \bibinfo {author} {\bibfnamefont {K.}~\bibnamefont {Carva}}, \bibinfo
  {author} {\bibfnamefont {P.~M.}\ \bibnamefont {Oppeneer}},\ and\ \bibinfo
  {author} {\bibfnamefont {U.}~\bibnamefont {Nowak}},\ }\bibfield  {title}
  {\bibinfo {title} {Orbital-resolved spin model for thermal magnetization
  switching in rare-earth-based ferrimagnets},\ }\href
  {https://doi.org/10.1103/PhysRevB.88.020406} {\bibfield  {journal} {\bibinfo
  {journal} {Phys. Rev. B}\ }\textbf {\bibinfo {volume} {88}},\ \bibinfo
  {pages} {020406} (\bibinfo {year} {2013})}\BibitemShut {NoStop}%
\bibitem [{\citenamefont {Antropov}\ \emph {et~al.}(1996)\citenamefont
  {Antropov}, \citenamefont {Katsnelson}, \citenamefont {Harmon}, \citenamefont
  {van Schilfgaarde},\ and\ \citenamefont {Kusnezov}}]{Antropov1996}%
  \BibitemOpen
  \bibfield  {author} {\bibinfo {author} {\bibfnamefont {V.~P.}\ \bibnamefont
  {Antropov}}, \bibinfo {author} {\bibfnamefont {M.~I.}\ \bibnamefont
  {Katsnelson}}, \bibinfo {author} {\bibfnamefont {B.~N.}\ \bibnamefont
  {Harmon}}, \bibinfo {author} {\bibfnamefont {M.}~\bibnamefont {van
  Schilfgaarde}},\ and\ \bibinfo {author} {\bibfnamefont {D.}~\bibnamefont
  {Kusnezov}},\ }\bibfield  {title} {\bibinfo {title} {Spin dynamics in
  magnets: Equation of motion and finite temperature effects},\ }\href
  {https://doi.org/10.1103/PhysRevB.54.1019} {\bibfield  {journal} {\bibinfo
  {journal} {Phys. Rev. B}\ }\textbf {\bibinfo {volume} {54}},\ \bibinfo
  {pages} {1019} (\bibinfo {year} {1996})}\BibitemShut {NoStop}%
\bibitem [{\citenamefont {Mondal}\ \emph {et~al.}(2017)\citenamefont {Mondal},
  \citenamefont {Berritta}, \citenamefont {Nandy},\ and\ \citenamefont
  {Oppeneer}}]{Mondal2017Nutation}%
  \BibitemOpen
  \bibfield  {author} {\bibinfo {author} {\bibfnamefont {R.}~\bibnamefont
  {Mondal}}, \bibinfo {author} {\bibfnamefont {M.}~\bibnamefont {Berritta}},
  \bibinfo {author} {\bibfnamefont {A.~K.}\ \bibnamefont {Nandy}},\ and\
  \bibinfo {author} {\bibfnamefont {P.~M.}\ \bibnamefont {Oppeneer}},\
  }\bibfield  {title} {\bibinfo {title} {Relativistic theory of magnetic
  inertia in ultrafast spin dynamics},\ }\href
  {https://doi.org/10.1103/PhysRevB.96.024425} {\bibfield  {journal} {\bibinfo
  {journal} {Phys. Rev. B}\ }\textbf {\bibinfo {volume} {96}},\ \bibinfo
  {pages} {024425} (\bibinfo {year} {2017})}\BibitemShut {NoStop}%
\bibitem [{\citenamefont {Mondal}\ \emph
  {et~al.}(2018{\natexlab{a}})\citenamefont {Mondal}, \citenamefont
  {Berritta},\ and\ \citenamefont {Oppeneer}}]{Mondal2018JPCM}%
  \BibitemOpen
  \bibfield  {author} {\bibinfo {author} {\bibfnamefont {R.}~\bibnamefont
  {Mondal}}, \bibinfo {author} {\bibfnamefont {M.}~\bibnamefont {Berritta}},\
  and\ \bibinfo {author} {\bibfnamefont {P.~M.}\ \bibnamefont {Oppeneer}},\
  }\bibfield  {title} {\bibinfo {title} {Generalisation of {G}ilbert damping
  and magnetic inertia parameter as a series of higher-order relativistic
  terms},\ }\href {http://stacks.iop.org/0953-8984/30/i=26/a=265801} {\bibfield
   {journal} {\bibinfo  {journal} {J. Phys.: Condens. Matter}\ }\textbf
  {\bibinfo {volume} {30}},\ \bibinfo {pages} {265801} (\bibinfo {year}
  {2018}{\natexlab{a}})}\BibitemShut {NoStop}%
\bibitem [{\citenamefont {Suhl}(1998)}]{Suhl1998}%
  \BibitemOpen
  \bibfield  {author} {\bibinfo {author} {\bibfnamefont {H.}~\bibnamefont
  {Suhl}},\ }\bibfield  {title} {\bibinfo {title} {Theory of the magnetic
  damping constant},\ }\href {https://doi.org/10.1109/20.706720} {\bibfield
  {journal} {\bibinfo  {journal} {IEEE Trans. Magn.}\ }\textbf {\bibinfo
  {volume} {34}},\ \bibinfo {pages} {1834} (\bibinfo {year}
  {1998})}\BibitemShut {NoStop}%
\bibitem [{\citenamefont {Wegrowe}(2000)}]{wegrowe2000thermokinetic}%
  \BibitemOpen
  \bibfield  {author} {\bibinfo {author} {\bibfnamefont {J.-E.}\ \bibnamefont
  {Wegrowe}},\ }\bibfield  {title} {\bibinfo {title} {{Thermokinetic approach
  of the generalized Landau-Lifshitz-Gilbert equation with spin-polarized
  current}},\ }\href {https://doi.org/10.1103/PhysRevB.62.1067} {\bibfield
  {journal} {\bibinfo  {journal} {Phys. Rev. B}\ }\textbf {\bibinfo {volume}
  {62}},\ \bibinfo {pages} {1067} (\bibinfo {year} {2000})}\BibitemShut
  {NoStop}%
\bibitem [{\citenamefont {Ciornei}\ \emph {et~al.}(2011)\citenamefont
  {Ciornei}, \citenamefont {Rub\'{\i}},\ and\ \citenamefont
  {Wegrowe}}]{Ciornei2011}%
  \BibitemOpen
  \bibfield  {author} {\bibinfo {author} {\bibfnamefont {M.-C.}\ \bibnamefont
  {Ciornei}}, \bibinfo {author} {\bibfnamefont {J.~M.}\ \bibnamefont
  {Rub\'{\i}}},\ and\ \bibinfo {author} {\bibfnamefont {J.-E.}\ \bibnamefont
  {Wegrowe}},\ }\bibfield  {title} {\bibinfo {title} {Magnetization dynamics in
  the inertial regime: Nutation predicted at short time scales},\ }\href
  {https://doi.org/10.1103/PhysRevB.83.020410} {\bibfield  {journal} {\bibinfo
  {journal} {Phys. Rev. B}\ }\textbf {\bibinfo {volume} {83}},\ \bibinfo
  {pages} {020410} (\bibinfo {year} {2011})}\BibitemShut {NoStop}%
\bibitem [{\citenamefont {Wegrowe}\ and\ \citenamefont
  {Ciornei}(2012)}]{Wegrowe2012}%
  \BibitemOpen
  \bibfield  {author} {\bibinfo {author} {\bibfnamefont {J.-E.}\ \bibnamefont
  {Wegrowe}}\ and\ \bibinfo {author} {\bibfnamefont {M.-C.}\ \bibnamefont
  {Ciornei}},\ }\bibfield  {title} {\bibinfo {title} {Magnetization dynamics,
  gyromagnetic relation, and inertial effects},\ }\href
  {https://doi.org/10.1119/1.4709188} {\bibfield  {journal} {\bibinfo
  {journal} {Am. J. Phys.}\ }\textbf {\bibinfo {volume} {80}},\ \bibinfo
  {pages} {607} (\bibinfo {year} {2012})}\BibitemShut {NoStop}%
\bibitem [{\citenamefont {Giordano}\ and\ \citenamefont
  {D\'ejardin}(2020)}]{giordano2020derivation}%
  \BibitemOpen
  \bibfield  {author} {\bibinfo {author} {\bibfnamefont {S.}~\bibnamefont
  {Giordano}}\ and\ \bibinfo {author} {\bibfnamefont {P.-M.}\ \bibnamefont
  {D\'ejardin}},\ }\bibfield  {title} {\bibinfo {title} {Derivation of magnetic
  inertial effects from the classical mechanics of a circular current loop},\
  }\href {https://doi.org/10.1103/PhysRevB.102.214406} {\bibfield  {journal}
  {\bibinfo  {journal} {Phys. Rev. B}\ }\textbf {\bibinfo {volume} {102}},\
  \bibinfo {pages} {214406} (\bibinfo {year} {2020})}\BibitemShut {NoStop}%
\bibitem [{\citenamefont {Li}\ \emph {et~al.}(2022)\citenamefont {Li},
  \citenamefont {Yang}, \citenamefont {Mondal}, \citenamefont {Tzschaschel},\
  and\ \citenamefont {Pal}}]{Jingwen2022}%
  \BibitemOpen
  \bibfield  {author} {\bibinfo {author} {\bibfnamefont {J.}~\bibnamefont
  {Li}}, \bibinfo {author} {\bibfnamefont {C.-J.}\ \bibnamefont {Yang}},
  \bibinfo {author} {\bibfnamefont {R.}~\bibnamefont {Mondal}}, \bibinfo
  {author} {\bibfnamefont {C.}~\bibnamefont {Tzschaschel}},\ and\ \bibinfo
  {author} {\bibfnamefont {S.}~\bibnamefont {Pal}},\ }\bibfield  {title}
  {\bibinfo {title} {{A perspective on nonlinearities in coherent magnetization
  dynamics}},\ }\href {https://doi.org/10.1063/5.0075999} {\bibfield  {journal}
  {\bibinfo  {journal} {Appl. Phys. Lett.}\ }\textbf {\bibinfo {volume}
  {120}},\ \bibinfo {pages} {050501} (\bibinfo {year} {2022})}\BibitemShut
  {NoStop}%
\bibitem [{\citenamefont {F\"ahnle}\ \emph {et~al.}(2011)\citenamefont
  {F\"ahnle}, \citenamefont {Steiauf},\ and\ \citenamefont
  {Illg}}]{Fahnle2011}%
  \BibitemOpen
  \bibfield  {author} {\bibinfo {author} {\bibfnamefont {M.}~\bibnamefont
  {F\"ahnle}}, \bibinfo {author} {\bibfnamefont {D.}~\bibnamefont {Steiauf}},\
  and\ \bibinfo {author} {\bibfnamefont {C.}~\bibnamefont {Illg}},\ }\bibfield
  {title} {\bibinfo {title} {{Generalized Gilbert equation including inertial
  damping: Derivation from an extended breathing Fermi surface model}},\ }\href
  {https://doi.org/10.1103/PhysRevB.84.172403} {\bibfield  {journal} {\bibinfo
  {journal} {Phys. Rev. B}\ }\textbf {\bibinfo {volume} {84}},\ \bibinfo
  {pages} {172403} (\bibinfo {year} {2011})}\BibitemShut {NoStop}%
\bibitem [{\citenamefont {Bhattacharjee}\ \emph {et~al.}(2012)\citenamefont
  {Bhattacharjee}, \citenamefont {Nordstr\"om},\ and\ \citenamefont
  {Fransson}}]{Bhattacharjee2012}%
  \BibitemOpen
  \bibfield  {author} {\bibinfo {author} {\bibfnamefont {S.}~\bibnamefont
  {Bhattacharjee}}, \bibinfo {author} {\bibfnamefont {L.}~\bibnamefont
  {Nordstr\"om}},\ and\ \bibinfo {author} {\bibfnamefont {J.}~\bibnamefont
  {Fransson}},\ }\bibfield  {title} {\bibinfo {title} {Atomistic spin dynamic
  method with both damping and moment of inertia effects included from first
  principles},\ }\href {https://doi.org/10.1103/PhysRevLett.108.057204}
  {\bibfield  {journal} {\bibinfo  {journal} {Phys. Rev. Lett.}\ }\textbf
  {\bibinfo {volume} {108}},\ \bibinfo {pages} {057204} (\bibinfo {year}
  {2012})}\BibitemShut {NoStop}%
\bibitem [{\citenamefont {Olive}\ \emph {et~al.}(2012)\citenamefont {Olive},
  \citenamefont {Lansac},\ and\ \citenamefont {Wegrowe}}]{Olive2012}%
  \BibitemOpen
  \bibfield  {author} {\bibinfo {author} {\bibfnamefont {E.}~\bibnamefont
  {Olive}}, \bibinfo {author} {\bibfnamefont {Y.}~\bibnamefont {Lansac}},\ and\
  \bibinfo {author} {\bibfnamefont {J.-E.}\ \bibnamefont {Wegrowe}},\
  }\bibfield  {title} {\bibinfo {title} {Beyond ferromagnetic resonance: The
  inertial regime of the magnetization},\ }\href
  {https://doi.org/10.1063/1.4712056} {\bibfield  {journal} {\bibinfo
  {journal} {Appl. Phys. Lett.}\ }\textbf {\bibinfo {volume} {100}},\ \bibinfo
  {pages} {192407} (\bibinfo {year} {2012})}\BibitemShut {NoStop}%
\bibitem [{\citenamefont {Neeraj}\ \emph {et~al.}(2021)\citenamefont {Neeraj},
  \citenamefont {Awari}, \citenamefont {Kovalev}, \citenamefont {Polley},
  \citenamefont {Zhou~Hagstr{\"o}m}, \citenamefont {Arekapudi}, \citenamefont
  {Semisalova}, \citenamefont {Lenz}, \citenamefont {Green}, \citenamefont
  {Deinert}, \citenamefont {Ilyakov}, \citenamefont {Chen}, \citenamefont
  {Bawatna}, \citenamefont {Scalera}, \citenamefont {d'Aquino}, \citenamefont
  {Serpico}, \citenamefont {Hellwig}, \citenamefont {Wegrowe}, \citenamefont
  {Gensch},\ and\ \citenamefont {Bonetti}}]{neeraj2019experimental}%
  \BibitemOpen
  \bibfield  {author} {\bibinfo {author} {\bibfnamefont {K.}~\bibnamefont
  {Neeraj}}, \bibinfo {author} {\bibfnamefont {N.}~\bibnamefont {Awari}},
  \bibinfo {author} {\bibfnamefont {S.}~\bibnamefont {Kovalev}}, \bibinfo
  {author} {\bibfnamefont {D.}~\bibnamefont {Polley}}, \bibinfo {author}
  {\bibfnamefont {N.}~\bibnamefont {Zhou~Hagstr{\"o}m}}, \bibinfo {author}
  {\bibfnamefont {S.~S. P.~K.}\ \bibnamefont {Arekapudi}}, \bibinfo {author}
  {\bibfnamefont {A.}~\bibnamefont {Semisalova}}, \bibinfo {author}
  {\bibfnamefont {K.}~\bibnamefont {Lenz}}, \bibinfo {author} {\bibfnamefont
  {B.}~\bibnamefont {Green}}, \bibinfo {author} {\bibfnamefont {J.-C.}\
  \bibnamefont {Deinert}}, \bibinfo {author} {\bibfnamefont {I.}~\bibnamefont
  {Ilyakov}}, \bibinfo {author} {\bibfnamefont {M.}~\bibnamefont {Chen}},
  \bibinfo {author} {\bibfnamefont {M.}~\bibnamefont {Bawatna}}, \bibinfo
  {author} {\bibfnamefont {V.}~\bibnamefont {Scalera}}, \bibinfo {author}
  {\bibfnamefont {M.}~\bibnamefont {d'Aquino}}, \bibinfo {author}
  {\bibfnamefont {C.}~\bibnamefont {Serpico}}, \bibinfo {author} {\bibfnamefont
  {O.}~\bibnamefont {Hellwig}}, \bibinfo {author} {\bibfnamefont {J.-E.}\
  \bibnamefont {Wegrowe}}, \bibinfo {author} {\bibfnamefont {M.}~\bibnamefont
  {Gensch}},\ and\ \bibinfo {author} {\bibfnamefont {S.}~\bibnamefont
  {Bonetti}},\ }\bibfield  {title} {\bibinfo {title} {Inertial spin dynamics in
  ferromagnets},\ }\href {https://doi.org/10.1038/s41567-020-01040-y}
  {\bibfield  {journal} {\bibinfo  {journal} {Nat. Phys.}\ }\textbf {\bibinfo
  {volume} {17}},\ \bibinfo {pages} {245} (\bibinfo {year} {2021})}\BibitemShut
  {NoStop}%
\bibitem [{\citenamefont {Unikandanunni}\ \emph {et~al.}(2022)\citenamefont
  {Unikandanunni}, \citenamefont {Medapalli}, \citenamefont {Asa},
  \citenamefont {Albisetti}, \citenamefont {Petti}, \citenamefont {Bertacco},
  \citenamefont {Fullerton},\ and\ \citenamefont
  {Bonetti}}]{unikandanunni2021inertial}%
  \BibitemOpen
  \bibfield  {author} {\bibinfo {author} {\bibfnamefont {V.}~\bibnamefont
  {Unikandanunni}}, \bibinfo {author} {\bibfnamefont {R.}~\bibnamefont
  {Medapalli}}, \bibinfo {author} {\bibfnamefont {M.}~\bibnamefont {Asa}},
  \bibinfo {author} {\bibfnamefont {E.}~\bibnamefont {Albisetti}}, \bibinfo
  {author} {\bibfnamefont {D.}~\bibnamefont {Petti}}, \bibinfo {author}
  {\bibfnamefont {R.}~\bibnamefont {Bertacco}}, \bibinfo {author}
  {\bibfnamefont {E.~E.}\ \bibnamefont {Fullerton}},\ and\ \bibinfo {author}
  {\bibfnamefont {S.}~\bibnamefont {Bonetti}},\ }\bibfield  {title} {\bibinfo
  {title} {Inertial spin dynamics in epitaxial cobalt films},\ }\href
  {https://doi.org/10.1103/PhysRevLett.129.237201} {\bibfield  {journal}
  {\bibinfo  {journal} {Phys. Rev. Lett.}\ }\textbf {\bibinfo {volume} {129}},\
  \bibinfo {pages} {237201} (\bibinfo {year} {2022})}\BibitemShut {NoStop}%
\bibitem [{\citenamefont {Kikuchi}\ and\ \citenamefont
  {Tatara}(2015)}]{Kikuchi}%
  \BibitemOpen
  \bibfield  {author} {\bibinfo {author} {\bibfnamefont {T.}~\bibnamefont
  {Kikuchi}}\ and\ \bibinfo {author} {\bibfnamefont {G.}~\bibnamefont
  {Tatara}},\ }\bibfield  {title} {\bibinfo {title} {Spin dynamics with inertia
  in metallic ferromagnets},\ }\href
  {https://doi.org/10.1103/PhysRevB.92.184410} {\bibfield  {journal} {\bibinfo
  {journal} {Phys. Rev. B}\ }\textbf {\bibinfo {volume} {92}},\ \bibinfo
  {pages} {184410} (\bibinfo {year} {2015})}\BibitemShut {NoStop}%
\bibitem [{\citenamefont {Blundell}(2001)}]{blundell01}%
  \BibitemOpen
  \bibfield  {author} {\bibinfo {author} {\bibfnamefont {S.}~\bibnamefont
  {Blundell}},\ }\href
  {http://www.amazon.com/exec/obidos/redirect?tag=citeulike07-
  20\&path=ASIN/0198505914} {\emph {\bibinfo {title} {{Magnetism in Condensed
  Matter}}}}\ (\bibinfo  {publisher} {Oxford University Press, Oxford},\
  \bibinfo {year} {2001})\BibitemShut {NoStop}%
\bibitem [{\citenamefont {Kambersk{\'y}}(1970)}]{kambersky70}%
  \BibitemOpen
  \bibfield  {author} {\bibinfo {author} {\bibfnamefont {V.}~\bibnamefont
  {Kambersk{\'y}}},\ }\bibfield  {title} {\bibinfo {title} {{On the
  Landau-Lifshitz relaxation in ferromagnetic metals}},\ }\href
  {https://doi.org/10.1139/p70-361} {\bibfield  {journal} {\bibinfo  {journal}
  {Can. J. Phys.}\ }\textbf {\bibinfo {volume} {48}},\ \bibinfo {pages} {2906}
  (\bibinfo {year} {1970})}\BibitemShut {NoStop}%
\bibitem [{\citenamefont {Kambersk{\'y}}(1976)}]{kambersky76}%
  \BibitemOpen
  \bibfield  {author} {\bibinfo {author} {\bibfnamefont {V.}~\bibnamefont
  {Kambersk{\'y}}},\ }\bibfield  {title} {{\selectlanguage {English}\bibinfo
  {title} {On ferromagnetic resonance damping in metals}},\ }\href
  {https://doi.org/10.1007/BF01587621} {\bibfield  {journal} {\bibinfo
  {journal} {Czech. J. Phys. B}\ }\textbf {\bibinfo {volume} {26}},\ \bibinfo
  {pages} {1366} (\bibinfo {year} {1976})}\BibitemShut {NoStop}%
\bibitem [{\citenamefont {Kambersk{\'y}}(2007)}]{kambersky07}%
  \BibitemOpen
  \bibfield  {author} {\bibinfo {author} {\bibfnamefont {V.}~\bibnamefont
  {Kambersk{\'y}}},\ }\bibfield  {title} {\bibinfo {title} {{Spin-orbital
  Gilbert damping in common magnetic metals}},\ }\href
  {https://doi.org/10.1103/PhysRevB.76.134416} {\bibfield  {journal} {\bibinfo
  {journal} {Phys. Rev. B}\ }\textbf {\bibinfo {volume} {76}},\ \bibinfo
  {pages} {134416} (\bibinfo {year} {2007})}\BibitemShut {NoStop}%
\bibitem [{\citenamefont {Thonig}\ \emph {et~al.}(2018)\citenamefont {Thonig},
  \citenamefont {Kvashnin}, \citenamefont {Eriksson},\ and\ \citenamefont
  {Pereiro}}]{Thonig2018}%
  \BibitemOpen
  \bibfield  {author} {\bibinfo {author} {\bibfnamefont {D.}~\bibnamefont
  {Thonig}}, \bibinfo {author} {\bibfnamefont {Y.}~\bibnamefont {Kvashnin}},
  \bibinfo {author} {\bibfnamefont {O.}~\bibnamefont {Eriksson}},\ and\
  \bibinfo {author} {\bibfnamefont {M.}~\bibnamefont {Pereiro}},\ }\bibfield
  {title} {\bibinfo {title} {{Nonlocal Gilbert damping tensor within the
  torque-torque correlation model}},\ }\href
  {https://doi.org/10.1103/PhysRevMaterials.2.013801} {\bibfield  {journal}
  {\bibinfo  {journal} {Phys. Rev. Materials}\ }\textbf {\bibinfo {volume}
  {2}},\ \bibinfo {pages} {013801} (\bibinfo {year} {2018})}\BibitemShut
  {NoStop}%
\bibitem [{\citenamefont {Brataas}\ \emph {et~al.}(2008)\citenamefont
  {Brataas}, \citenamefont {Tserkovnyak},\ and\ \citenamefont
  {Bauer}}]{Brataas2008}%
  \BibitemOpen
  \bibfield  {author} {\bibinfo {author} {\bibfnamefont {A.}~\bibnamefont
  {Brataas}}, \bibinfo {author} {\bibfnamefont {Y.}~\bibnamefont
  {Tserkovnyak}},\ and\ \bibinfo {author} {\bibfnamefont {G.~E.~W.}\
  \bibnamefont {Bauer}},\ }\bibfield  {title} {\bibinfo {title} {{Scattering
  Theory of Gilbert Damping}},\ }\href
  {https://doi.org/10.1103/PhysRevLett.101.037207} {\bibfield  {journal}
  {\bibinfo  {journal} {Phys. Rev. Lett.}\ }\textbf {\bibinfo {volume} {101}},\
  \bibinfo {pages} {037207} (\bibinfo {year} {2008})}\BibitemShut {NoStop}%
\bibitem [{\citenamefont {Ebert}\ \emph {et~al.}(2011)\citenamefont {Ebert},
  \citenamefont {Mankovsky}, \citenamefont {K\"odderitzsch},\ and\
  \citenamefont {Kelly}}]{EbertPRL2011}%
  \BibitemOpen
  \bibfield  {author} {\bibinfo {author} {\bibfnamefont {H.}~\bibnamefont
  {Ebert}}, \bibinfo {author} {\bibfnamefont {S.}~\bibnamefont {Mankovsky}},
  \bibinfo {author} {\bibfnamefont {D.}~\bibnamefont {K\"odderitzsch}},\ and\
  \bibinfo {author} {\bibfnamefont {P.~J.}\ \bibnamefont {Kelly}},\ }\bibfield
  {title} {\bibinfo {title} {{\textit{Ab Initio} Calculation of the Gilbert
  Damping Parameter via the Linear Response Formalism}},\ }\href
  {https://doi.org/10.1103/PhysRevLett.107.066603} {\bibfield  {journal}
  {\bibinfo  {journal} {Phys. Rev. Lett.}\ }\textbf {\bibinfo {volume} {107}},\
  \bibinfo {pages} {066603} (\bibinfo {year} {2011})}\BibitemShut {NoStop}%
\bibitem [{\citenamefont {Mondal}\ \emph {et~al.}(2016)\citenamefont {Mondal},
  \citenamefont {Berritta},\ and\ \citenamefont {Oppeneer}}]{Mondal2016}%
  \BibitemOpen
  \bibfield  {author} {\bibinfo {author} {\bibfnamefont {R.}~\bibnamefont
  {Mondal}}, \bibinfo {author} {\bibfnamefont {M.}~\bibnamefont {Berritta}},\
  and\ \bibinfo {author} {\bibfnamefont {P.~M.}\ \bibnamefont {Oppeneer}},\
  }\bibfield  {title} {\bibinfo {title} {Relativistic theory of spin relaxation
  mechanisms in the {L}andau-{L}ifshitz-{G}ilbert equation of spin dynamics},\
  }\href {https://doi.org/10.1103/PhysRevB.94.144419} {\bibfield  {journal}
  {\bibinfo  {journal} {Phys. Rev. B}\ }\textbf {\bibinfo {volume} {94}},\
  \bibinfo {pages} {144419} (\bibinfo {year} {2016})}\BibitemShut {NoStop}%
\bibitem [{\citenamefont {Mondal}\ \emph
  {et~al.}(2018{\natexlab{b}})\citenamefont {Mondal}, \citenamefont
  {Berritta},\ and\ \citenamefont {Oppeneer}}]{Mondal2018PRB}%
  \BibitemOpen
  \bibfield  {author} {\bibinfo {author} {\bibfnamefont {R.}~\bibnamefont
  {Mondal}}, \bibinfo {author} {\bibfnamefont {M.}~\bibnamefont {Berritta}},\
  and\ \bibinfo {author} {\bibfnamefont {P.~M.}\ \bibnamefont {Oppeneer}},\
  }\bibfield  {title} {\bibinfo {title} {Unified theory of magnetization
  dynamics with relativistic and nonrelativistic spin torques},\ }\href
  {https://doi.org/10.1103/PhysRevB.98.214429} {\bibfield  {journal} {\bibinfo
  {journal} {Phys. Rev. B}\ }\textbf {\bibinfo {volume} {98}},\ \bibinfo
  {pages} {214429} (\bibinfo {year} {2018}{\natexlab{b}})}\BibitemShut
  {NoStop}%
\bibitem [{\citenamefont {Thonig}\ and\ \citenamefont
  {Henk}(2014)}]{Thonig_2014}%
  \BibitemOpen
  \bibfield  {author} {\bibinfo {author} {\bibfnamefont {D.}~\bibnamefont
  {Thonig}}\ and\ \bibinfo {author} {\bibfnamefont {J.}~\bibnamefont {Henk}},\
  }\bibfield  {title} {\bibinfo {title} {{Gilbert damping tensor within the
  breathing Fermi surface model: anisotropy and non-locality}},\ }\href
  {https://doi.org/10.1088/1367-2630/16/1/013032} {\bibfield  {journal}
  {\bibinfo  {journal} {New J. Phys.}\ }\textbf {\bibinfo {volume} {16}},\
  \bibinfo {pages} {013032} (\bibinfo {year} {2014})}\BibitemShut {NoStop}%
\bibitem [{\citenamefont {Platow}\ \emph {et~al.}(1998)\citenamefont {Platow},
  \citenamefont {Anisimov}, \citenamefont {Dunifer}, \citenamefont {Farle},\
  and\ \citenamefont {Baberschke}}]{platow1998correlations}%
  \BibitemOpen
  \bibfield  {author} {\bibinfo {author} {\bibfnamefont {W.}~\bibnamefont
  {Platow}}, \bibinfo {author} {\bibfnamefont {A.~N.}\ \bibnamefont
  {Anisimov}}, \bibinfo {author} {\bibfnamefont {G.~L.}\ \bibnamefont
  {Dunifer}}, \bibinfo {author} {\bibfnamefont {M.}~\bibnamefont {Farle}},\
  and\ \bibinfo {author} {\bibfnamefont {K.}~\bibnamefont {Baberschke}},\
  }\bibfield  {title} {\bibinfo {title} {Correlations between
  ferromagnetic-resonance linewidths and sample quality in the study of
  metallic ultrathin films},\ }\href {https://doi.org/10.1103/PhysRevB.58.5611}
  {\bibfield  {journal} {\bibinfo  {journal} {Phys. Rev. B}\ }\textbf {\bibinfo
  {volume} {58}},\ \bibinfo {pages} {5611} (\bibinfo {year}
  {1998})}\BibitemShut {NoStop}%
\bibitem [{\citenamefont {Farle}\ \emph {et~al.}(2013)\citenamefont {Farle},
  \citenamefont {Silva},\ and\ \citenamefont {Woltersdorf}}]{Farle2013}%
  \BibitemOpen
  \bibfield  {author} {\bibinfo {author} {\bibfnamefont {M.}~\bibnamefont
  {Farle}}, \bibinfo {author} {\bibfnamefont {T.}~\bibnamefont {Silva}},\ and\
  \bibinfo {author} {\bibfnamefont {G.}~\bibnamefont {Woltersdorf}},\ }\bibinfo
  {title} {Spin dynamics in the time and frequency domain},\ in\ \href
  {https://doi.org/10.1007/978-3-642-32042-2_2} {\emph {\bibinfo {booktitle}
  {Magnetic Nanostructures: Spin Dynamics and Spin Transport}}},\ \bibinfo
  {editor} {edited by\ \bibinfo {editor} {\bibfnamefont {H.}~\bibnamefont
  {Zabel}}\ and\ \bibinfo {editor} {\bibfnamefont {M.}~\bibnamefont {Farle}}}\
  (\bibinfo  {publisher} {Springer Berlin Heidelberg},\ \bibinfo {address}
  {Berlin, Heidelberg},\ \bibinfo {year} {2013})\ pp.\ \bibinfo {pages}
  {37--83}\BibitemShut {NoStop}%
\bibitem [{\citenamefont {{Brown Jr.}}(1963)}]{Brown_1963}%
  \BibitemOpen
  \bibfield  {author} {\bibinfo {author} {\bibfnamefont {W.~F.}\ \bibnamefont
  {{Brown Jr.}}},\ }\bibfield  {title} {\bibinfo {title} {Thermal fluctuations
  of a single-domain particle},\ }\href
  {https://doi.org/10.1103/PhysRev.130.1677} {\bibfield  {journal} {\bibinfo
  {journal} {Phys. Rev.}\ }\textbf {\bibinfo {volume} {130}},\ \bibinfo {pages}
  {1677} (\bibinfo {year} {1963})}\BibitemShut {NoStop}%
\bibitem [{\citenamefont {Coffey}\ \emph {et~al.}(2004)\citenamefont {Coffey},
  \citenamefont {Kalmykov},\ and\ \citenamefont
  {Waldron}}]{coffey2004langevin}%
  \BibitemOpen
  \bibfield  {author} {\bibinfo {author} {\bibfnamefont {W.}~\bibnamefont
  {Coffey}}, \bibinfo {author} {\bibfnamefont {Y.~P.}\ \bibnamefont
  {Kalmykov}},\ and\ \bibinfo {author} {\bibfnamefont {J.~T.}\ \bibnamefont
  {Waldron}},\ }\href {https://www.worldscientific.com/doi/abs/10.1142/5343}
  {\emph {\bibinfo {title} {The Langevin equation: with applications to
  stochastic problems in physics, chemistry and electrical engineering}}},\
  \bibinfo {series} {World Scientific Series in Contemporary Chemical Physics},
  Vol.~\bibinfo {volume} {14}\ (\bibinfo  {publisher} {World Scientific},\
  \bibinfo {address} {Singapore},\ \bibinfo {year} {2004})\BibitemShut
  {NoStop}%
\bibitem [{\citenamefont {Lyberatos}\ \emph {et~al.}(1993)\citenamefont
  {Lyberatos}, \citenamefont {Berkov},\ and\ \citenamefont
  {Chantrell}}]{Lyberatos1993}%
  \BibitemOpen
  \bibfield  {author} {\bibinfo {author} {\bibfnamefont {A.}~\bibnamefont
  {Lyberatos}}, \bibinfo {author} {\bibfnamefont {D.~V.}\ \bibnamefont
  {Berkov}},\ and\ \bibinfo {author} {\bibfnamefont {R.~W.}\ \bibnamefont
  {Chantrell}},\ }\bibfield  {title} {\bibinfo {title} {A method for the
  numerical simulation of the thermal magnetization fluctuations in
  micromagnetics},\ }\href {https://doi.org/10.1088/0953-8984/5/47/016}
  {\bibfield  {journal} {\bibinfo  {journal} {J. Phys.: Condens. Matter}\
  }\textbf {\bibinfo {volume} {5}},\ \bibinfo {pages} {8911} (\bibinfo {year}
  {1993})}\BibitemShut {NoStop}%
\bibitem [{\citenamefont {Chubykalo}\ \emph {et~al.}(2003)\citenamefont
  {Chubykalo}, \citenamefont {Smirnov-Rueda}, \citenamefont {Gonzalez},
  \citenamefont {Wongsam}, \citenamefont {Chantrell},\ and\ \citenamefont
  {Nowak}}]{CHUBYKALO2003}%
  \BibitemOpen
  \bibfield  {author} {\bibinfo {author} {\bibfnamefont {O.}~\bibnamefont
  {Chubykalo}}, \bibinfo {author} {\bibfnamefont {R.}~\bibnamefont
  {Smirnov-Rueda}}, \bibinfo {author} {\bibfnamefont {J.}~\bibnamefont
  {Gonzalez}}, \bibinfo {author} {\bibfnamefont {M.}~\bibnamefont {Wongsam}},
  \bibinfo {author} {\bibfnamefont {R.}~\bibnamefont {Chantrell}},\ and\
  \bibinfo {author} {\bibfnamefont {U.}~\bibnamefont {Nowak}},\ }\bibfield
  {title} {\bibinfo {title} {Brownian dynamics approach to interacting magnetic
  moments},\ }\href
  {https://doi.org/https://doi.org/10.1016/S0304-8853(03)00452-9} {\bibfield
  {journal} {\bibinfo  {journal} {J. Magn. Magn. Mater.}\ }\textbf {\bibinfo
  {volume} {266}},\ \bibinfo {pages} {28} (\bibinfo {year} {2003})}\BibitemShut
  {NoStop}%
\bibitem [{\citenamefont {Tauchert}\ \emph {et~al.}(2022)\citenamefont
  {Tauchert}, \citenamefont {Volkov}, \citenamefont {Ehberger}, \citenamefont
  {Kazenwadel}, \citenamefont {Evers}, \citenamefont {Lange}, \citenamefont
  {Donges}, \citenamefont {Book}, \citenamefont {Kreuzpaintner}, \citenamefont
  {Nowak},\ and\ \citenamefont {Baum}}]{Tauchert2022}%
  \BibitemOpen
  \bibfield  {author} {\bibinfo {author} {\bibfnamefont {S.~R.}\ \bibnamefont
  {Tauchert}}, \bibinfo {author} {\bibfnamefont {M.}~\bibnamefont {Volkov}},
  \bibinfo {author} {\bibfnamefont {D.}~\bibnamefont {Ehberger}}, \bibinfo
  {author} {\bibfnamefont {D.}~\bibnamefont {Kazenwadel}}, \bibinfo {author}
  {\bibfnamefont {M.}~\bibnamefont {Evers}}, \bibinfo {author} {\bibfnamefont
  {H.}~\bibnamefont {Lange}}, \bibinfo {author} {\bibfnamefont
  {A.}~\bibnamefont {Donges}}, \bibinfo {author} {\bibfnamefont
  {A.}~\bibnamefont {Book}}, \bibinfo {author} {\bibfnamefont {W.}~\bibnamefont
  {Kreuzpaintner}}, \bibinfo {author} {\bibfnamefont {U.}~\bibnamefont
  {Nowak}},\ and\ \bibinfo {author} {\bibfnamefont {P.}~\bibnamefont {Baum}},\
  }\bibfield  {title} {\bibinfo {title} {Polarized phonons carry angular
  momentum in ultrafast demagnetization},\ }\href
  {https://doi.org/10.1038/s41586-021-04306-4} {\bibfield  {journal} {\bibinfo
  {journal} {Nature}\ }\textbf {\bibinfo {volume} {602}},\ \bibinfo {pages}
  {73} (\bibinfo {year} {2022})}\BibitemShut {NoStop}%
\bibitem [{\citenamefont {Kubo}\ and\ \citenamefont
  {Hashitsume}(1970)}]{Kubo1970}%
  \BibitemOpen
  \bibfield  {author} {\bibinfo {author} {\bibfnamefont {R.}~\bibnamefont
  {Kubo}}\ and\ \bibinfo {author} {\bibfnamefont {N.}~\bibnamefont
  {Hashitsume}},\ }\bibfield  {title} {\bibinfo {title} {{Brownian Motion of
  Spins}},\ }\href {https://doi.org/10.1143/PTPS.46.210} {\bibfield  {journal}
  {\bibinfo  {journal} {Prog. Theor. Phys. Suppl.}\ }\textbf {\bibinfo {volume}
  {46}},\ \bibinfo {pages} {210} (\bibinfo {year} {1970})}\BibitemShut
  {NoStop}%
\bibitem [{\citenamefont {St\"{o}hr}\ and\ \citenamefont
  {Siegmann}(2006)}]{Stohr2006}%
  \BibitemOpen
  \bibfield  {author} {\bibinfo {author} {\bibfnamefont {J.}~\bibnamefont
  {St\"{o}hr}}\ and\ \bibinfo {author} {\bibfnamefont {H.~C.}\ \bibnamefont
  {Siegmann}},\ }\bibinfo {title} {Magnetism: From fundamentals to nanoscale
  dynamics}\ (\bibinfo  {publisher} {Springer, Berlin Heidelberg},\ \bibinfo
  {year} {2006})\ Chap.\ \bibinfo {chapter} {Ultrafast Magnetization Dynamics},
  pp.\ \bibinfo {pages} {679--759}\BibitemShut {NoStop}%
\bibitem [{\citenamefont {Atxitia}\ \emph {et~al.}(2009)\citenamefont
  {Atxitia}, \citenamefont {Chubykalo-Fesenko}, \citenamefont {Chantrell},
  \citenamefont {Nowak},\ and\ \citenamefont {Rebei}}]{Atxitia2009}%
  \BibitemOpen
  \bibfield  {author} {\bibinfo {author} {\bibfnamefont {U.}~\bibnamefont
  {Atxitia}}, \bibinfo {author} {\bibfnamefont {O.}~\bibnamefont
  {Chubykalo-Fesenko}}, \bibinfo {author} {\bibfnamefont {R.~W.}\ \bibnamefont
  {Chantrell}}, \bibinfo {author} {\bibfnamefont {U.}~\bibnamefont {Nowak}},\
  and\ \bibinfo {author} {\bibfnamefont {A.}~\bibnamefont {Rebei}},\ }\bibfield
   {title} {\bibinfo {title} {Ultrafast spin dynamics: The effect of colored
  noise},\ }\href {https://doi.org/10.1103/PhysRevLett.102.057203} {\bibfield
  {journal} {\bibinfo  {journal} {Phys. Rev. Lett.}\ }\textbf {\bibinfo
  {volume} {102}},\ \bibinfo {pages} {057203} (\bibinfo {year}
  {2009})}\BibitemShut {NoStop}%
\bibitem [{\citenamefont {Cherkasskii}\ \emph {et~al.}(2022)\citenamefont
  {Cherkasskii}, \citenamefont {Barsukov}, \citenamefont {Mondal},
  \citenamefont {Farle},\ and\ \citenamefont
  {Semisalova}}]{Cherkasskii2022Anisotropy}%
  \BibitemOpen
  \bibfield  {author} {\bibinfo {author} {\bibfnamefont {M.}~\bibnamefont
  {Cherkasskii}}, \bibinfo {author} {\bibfnamefont {I.}~\bibnamefont
  {Barsukov}}, \bibinfo {author} {\bibfnamefont {R.}~\bibnamefont {Mondal}},
  \bibinfo {author} {\bibfnamefont {M.}~\bibnamefont {Farle}},\ and\ \bibinfo
  {author} {\bibfnamefont {A.}~\bibnamefont {Semisalova}},\ }\bibfield  {title}
  {\bibinfo {title} {Theory of inertial spin dynamics in anisotropic
  ferromagnets},\ }\href {https://doi.org/10.1103/PhysRevB.106.054428}
  {\bibfield  {journal} {\bibinfo  {journal} {Phys. Rev. B}\ }\textbf {\bibinfo
  {volume} {106}},\ \bibinfo {pages} {054428} (\bibinfo {year}
  {2022})}\BibitemShut {NoStop}%
\bibitem [{\citenamefont {F\"ahnle}\ \emph {et~al.}(2013)\citenamefont
  {F\"ahnle}, \citenamefont {Steiauf},\ and\ \citenamefont
  {Illg}}]{fahnle2013erratum}%
  \BibitemOpen
  \bibfield  {author} {\bibinfo {author} {\bibfnamefont {M.}~\bibnamefont
  {F\"ahnle}}, \bibinfo {author} {\bibfnamefont {D.}~\bibnamefont {Steiauf}},\
  and\ \bibinfo {author} {\bibfnamefont {C.}~\bibnamefont {Illg}},\ }\bibfield
  {title} {\bibinfo {title} {{Erratum: Generalized Gilbert equation including
  inertial damping: Derivation from an extended breathing Fermi surface model
  [Phys. Rev. B 84, 172403 (2011)]}},\ }\href
  {https://doi.org/10.1103/PhysRevB.88.219905} {\bibfield  {journal} {\bibinfo
  {journal} {Phys. Rev. B}\ }\textbf {\bibinfo {volume} {88}},\ \bibinfo
  {pages} {219905(E)} (\bibinfo {year} {2013})}\BibitemShut {NoStop}%
\bibitem [{\citenamefont {Wegrowe}\ and\ \citenamefont
  {Olive}(2016)}]{Wegrowe2016JPCM}%
  \BibitemOpen
  \bibfield  {author} {\bibinfo {author} {\bibfnamefont {J.-E.}\ \bibnamefont
  {Wegrowe}}\ and\ \bibinfo {author} {\bibfnamefont {E.}~\bibnamefont
  {Olive}},\ }\bibfield  {title} {\bibinfo {title} {The magnetic monopole and
  the separation between fast and slow magnetic degrees of freedom},\ }\href
  {http://stacks.iop.org/0953-8984/28/i=10/a=106001} {\bibfield  {journal}
  {\bibinfo  {journal} {J. Phys.: Condens. Matter}\ }\textbf {\bibinfo {volume}
  {28}},\ \bibinfo {pages} {106001} (\bibinfo {year} {2016})}\BibitemShut
  {NoStop}%
\bibitem [{\citenamefont {Mondal}\ \emph {et~al.}(2021)\citenamefont {Mondal},
  \citenamefont {Gro\ss{}enbach}, \citenamefont {R\'ozsa},\ and\ \citenamefont
  {Nowak}}]{Mondal2020nutation}%
  \BibitemOpen
  \bibfield  {author} {\bibinfo {author} {\bibfnamefont {R.}~\bibnamefont
  {Mondal}}, \bibinfo {author} {\bibfnamefont {S.}~\bibnamefont
  {Gro\ss{}enbach}}, \bibinfo {author} {\bibfnamefont {L.}~\bibnamefont
  {R\'ozsa}},\ and\ \bibinfo {author} {\bibfnamefont {U.}~\bibnamefont
  {Nowak}},\ }\bibfield  {title} {\bibinfo {title} {Nutation in
  antiferromagnetic resonance},\ }\href
  {https://doi.org/10.1103/PhysRevB.103.104404} {\bibfield  {journal} {\bibinfo
   {journal} {Phys. Rev. B}\ }\textbf {\bibinfo {volume} {103}},\ \bibinfo
  {pages} {104404} (\bibinfo {year} {2021})}\BibitemShut {NoStop}%
\bibitem [{\citenamefont {Titov}\ \emph
  {et~al.}(2021{\natexlab{a}})\citenamefont {Titov}, \citenamefont {Coffey},
  \citenamefont {Kalmykov}, \citenamefont {Zarifakis},\ and\ \citenamefont
  {Titov}}]{Titov2021Inertial}%
  \BibitemOpen
  \bibfield  {author} {\bibinfo {author} {\bibfnamefont {S.~V.}\ \bibnamefont
  {Titov}}, \bibinfo {author} {\bibfnamefont {W.~T.}\ \bibnamefont {Coffey}},
  \bibinfo {author} {\bibfnamefont {Y.~P.}\ \bibnamefont {Kalmykov}}, \bibinfo
  {author} {\bibfnamefont {M.}~\bibnamefont {Zarifakis}},\ and\ \bibinfo
  {author} {\bibfnamefont {A.~S.}\ \bibnamefont {Titov}},\ }\bibfield  {title}
  {\bibinfo {title} {Inertial magnetization dynamics of ferromagnetic
  nanoparticles including thermal agitation},\ }\href
  {https://doi.org/10.1103/PhysRevB.103.144433} {\bibfield  {journal} {\bibinfo
   {journal} {Phys. Rev. B}\ }\textbf {\bibinfo {volume} {103}},\ \bibinfo
  {pages} {144433} (\bibinfo {year} {2021}{\natexlab{a}})}\BibitemShut
  {NoStop}%
\bibitem [{\citenamefont {Thonig}\ \emph {et~al.}(2017)\citenamefont {Thonig},
  \citenamefont {Eriksson},\ and\ \citenamefont {Pereiro}}]{Thonig2017}%
  \BibitemOpen
  \bibfield  {author} {\bibinfo {author} {\bibfnamefont {D.}~\bibnamefont
  {Thonig}}, \bibinfo {author} {\bibfnamefont {O.}~\bibnamefont {Eriksson}},\
  and\ \bibinfo {author} {\bibfnamefont {M.}~\bibnamefont {Pereiro}},\
  }\bibfield  {title} {\bibinfo {title} {Magnetic moment of inertia within the
  torque-torque correlation model},\ }\href
  {https://doi.org/10.1038/s41598-017-01081-z} {\bibfield  {journal} {\bibinfo
  {journal} {Sci. Rep.}\ }\textbf {\bibinfo {volume} {7}},\ \bibinfo {pages}
  {931} (\bibinfo {year} {2017})}\BibitemShut {NoStop}%
\bibitem [{\citenamefont {Foldy}\ and\ \citenamefont
  {Wouthuysen}(1950)}]{foldy50}%
  \BibitemOpen
  \bibfield  {author} {\bibinfo {author} {\bibfnamefont {L.~L.}\ \bibnamefont
  {Foldy}}\ and\ \bibinfo {author} {\bibfnamefont {S.~A.}\ \bibnamefont
  {Wouthuysen}},\ }\bibfield  {title} {\bibinfo {title} {{On the Dirac Theory
  of Spin 1/2 Particles and Its Non-Relativistic Limit}},\ }\href
  {https://doi.org/10.1103/PhysRev.78.29} {\bibfield  {journal} {\bibinfo
  {journal} {Phys. Rev.}\ }\textbf {\bibinfo {volume} {78}},\ \bibinfo {pages}
  {29} (\bibinfo {year} {1950})}\BibitemShut {NoStop}%
\bibitem [{\citenamefont {Hammar}\ and\ \citenamefont
  {Fransson}(2017)}]{Hammar2017}%
  \BibitemOpen
  \bibfield  {author} {\bibinfo {author} {\bibfnamefont {H.}~\bibnamefont
  {Hammar}}\ and\ \bibinfo {author} {\bibfnamefont {J.}~\bibnamefont
  {Fransson}},\ }\bibfield  {title} {\bibinfo {title} {Transient spin dynamics
  in a single-molecule magnet},\ }\href
  {https://doi.org/10.1103/PhysRevB.96.214401} {\bibfield  {journal} {\bibinfo
  {journal} {Phys. Rev. B}\ }\textbf {\bibinfo {volume} {96}},\ \bibinfo
  {pages} {214401} (\bibinfo {year} {2017})}\BibitemShut {NoStop}%
\bibitem [{\citenamefont {Bajpai}\ and\ \citenamefont
  {Nikoli\ifmmode~\acute{c}\else \'{c}\fi{}}(2019)}]{Bajpai2019}%
  \BibitemOpen
  \bibfield  {author} {\bibinfo {author} {\bibfnamefont {U.}~\bibnamefont
  {Bajpai}}\ and\ \bibinfo {author} {\bibfnamefont {B.~K.}\ \bibnamefont
  {Nikoli\ifmmode~\acute{c}\else \'{c}\fi{}}},\ }\bibfield  {title} {\bibinfo
  {title} {{Time-retarded damping and magnetic inertia in the
  {Landau-Lifshitz-Gilbert} equation self-consistently coupled to electronic
  time-dependent nonequilibrium Green functions}},\ }\href
  {https://doi.org/10.1103/PhysRevB.99.134409} {\bibfield  {journal} {\bibinfo
  {journal} {Phys. Rev. B}\ }\textbf {\bibinfo {volume} {99}},\ \bibinfo
  {pages} {134409} (\bibinfo {year} {2019})}\BibitemShut {NoStop}%
\bibitem [{\citenamefont {Li}\ \emph {et~al.}(2015)\citenamefont {Li},
  \citenamefont {Barra}, \citenamefont {Auffret}, \citenamefont {Ebels},\ and\
  \citenamefont {Bailey}}]{Li2015}%
  \BibitemOpen
  \bibfield  {author} {\bibinfo {author} {\bibfnamefont {Y.}~\bibnamefont
  {Li}}, \bibinfo {author} {\bibfnamefont {A.-L.}\ \bibnamefont {Barra}},
  \bibinfo {author} {\bibfnamefont {S.}~\bibnamefont {Auffret}}, \bibinfo
  {author} {\bibfnamefont {U.}~\bibnamefont {Ebels}},\ and\ \bibinfo {author}
  {\bibfnamefont {W.~E.}\ \bibnamefont {Bailey}},\ }\bibfield  {title}
  {\bibinfo {title} {Inertial terms to magnetization dynamics in ferromagnetic
  thin films},\ }\href {https://doi.org/10.1103/PhysRevB.92.140413} {\bibfield
  {journal} {\bibinfo  {journal} {Phys. Rev. B}\ }\textbf {\bibinfo {volume}
  {92}},\ \bibinfo {pages} {140413} (\bibinfo {year} {2015})}\BibitemShut
  {NoStop}%
\bibitem [{\citenamefont {Bouaziz}\ \emph {et~al.}(2019)\citenamefont
  {Bouaziz}, \citenamefont {Dias}, \citenamefont {Guimar\~aes},\ and\
  \citenamefont {Lounis}}]{Bouaziz2019}%
  \BibitemOpen
  \bibfield  {author} {\bibinfo {author} {\bibfnamefont {J.}~\bibnamefont
  {Bouaziz}}, \bibinfo {author} {\bibfnamefont {M.~d.~S.}\ \bibnamefont
  {Dias}}, \bibinfo {author} {\bibfnamefont {F.~S.~M.}\ \bibnamefont
  {Guimar\~aes}},\ and\ \bibinfo {author} {\bibfnamefont {S.}~\bibnamefont
  {Lounis}},\ }\bibfield  {title} {\bibinfo {title} {Spin dynamics of $3d$ and
  $4d$ impurities embedded in prototypical topological insulators},\ }\href
  {https://doi.org/10.1103/PhysRevMaterials.3.054201} {\bibfield  {journal}
  {\bibinfo  {journal} {Phys. Rev. Mater.}\ }\textbf {\bibinfo {volume} {3}},\
  \bibinfo {pages} {054201} (\bibinfo {year} {2019})}\BibitemShut {NoStop}%
\bibitem [{\citenamefont {Mondal}\ \emph {et~al.}(2019)\citenamefont {Mondal},
  \citenamefont {Donges}, \citenamefont {Ritzmann}, \citenamefont {Oppeneer},\
  and\ \citenamefont {Nowak}}]{Mondal2019PRB}%
  \BibitemOpen
  \bibfield  {author} {\bibinfo {author} {\bibfnamefont {R.}~\bibnamefont
  {Mondal}}, \bibinfo {author} {\bibfnamefont {A.}~\bibnamefont {Donges}},
  \bibinfo {author} {\bibfnamefont {U.}~\bibnamefont {Ritzmann}}, \bibinfo
  {author} {\bibfnamefont {P.~M.}\ \bibnamefont {Oppeneer}},\ and\ \bibinfo
  {author} {\bibfnamefont {U.}~\bibnamefont {Nowak}},\ }\bibfield  {title}
  {\bibinfo {title} {Terahertz spin dynamics driven by a field-derivative
  torque},\ }\href {https://doi.org/10.1103/PhysRevB.100.060409} {\bibfield
  {journal} {\bibinfo  {journal} {Phys. Rev. B}\ }\textbf {\bibinfo {volume}
  {100}},\ \bibinfo {pages} {060409(R)} (\bibinfo {year} {2019})}\BibitemShut
  {NoStop}%
\bibitem [{\citenamefont {Kittel}(1948)}]{Kittel1948}%
  \BibitemOpen
  \bibfield  {author} {\bibinfo {author} {\bibfnamefont {C.}~\bibnamefont
  {Kittel}},\ }\bibfield  {title} {\bibinfo {title} {On the theory of
  ferromagnetic resonance absorption},\ }\href
  {https://doi.org/10.1103/PhysRev.73.155} {\bibfield  {journal} {\bibinfo
  {journal} {Phys. Rev.}\ }\textbf {\bibinfo {volume} {73}},\ \bibinfo {pages}
  {155} (\bibinfo {year} {1948})}\BibitemShut {NoStop}%
\bibitem [{\citenamefont {Cherkasskii}\ \emph {et~al.}(2020)\citenamefont
  {Cherkasskii}, \citenamefont {Farle},\ and\ \citenamefont
  {Semisalova}}]{cherkasskii2020nutation}%
  \BibitemOpen
  \bibfield  {author} {\bibinfo {author} {\bibfnamefont {M.}~\bibnamefont
  {Cherkasskii}}, \bibinfo {author} {\bibfnamefont {M.}~\bibnamefont {Farle}},\
  and\ \bibinfo {author} {\bibfnamefont {A.}~\bibnamefont {Semisalova}},\
  }\bibfield  {title} {\bibinfo {title} {Nutation resonance in ferromagnets},\
  }\href {https://doi.org/10.1103/PhysRevB.102.184432} {\bibfield  {journal}
  {\bibinfo  {journal} {Phys. Rev. B}\ }\textbf {\bibinfo {volume} {102}},\
  \bibinfo {pages} {184432} (\bibinfo {year} {2020})}\BibitemShut {NoStop}%
\bibitem [{\citenamefont {Olive}\ \emph {et~al.}(2015)\citenamefont {Olive},
  \citenamefont {Lansac}, \citenamefont {Meyer}, \citenamefont {Hayoun},\ and\
  \citenamefont {Wegrowe}}]{Olive2015}%
  \BibitemOpen
  \bibfield  {author} {\bibinfo {author} {\bibfnamefont {E.}~\bibnamefont
  {Olive}}, \bibinfo {author} {\bibfnamefont {Y.}~\bibnamefont {Lansac}},
  \bibinfo {author} {\bibfnamefont {M.}~\bibnamefont {Meyer}}, \bibinfo
  {author} {\bibfnamefont {M.}~\bibnamefont {Hayoun}},\ and\ \bibinfo {author}
  {\bibfnamefont {J.-E.}\ \bibnamefont {Wegrowe}},\ }\bibfield  {title}
  {\bibinfo {title} {{Deviation from the Landau-Lifshitz-Gilbert equation in
  the inertial regime of the magnetization}},\ }\href
  {https://doi.org/10.1063/1.4921908} {\bibfield  {journal} {\bibinfo
  {journal} {J. Appl. Phys.}\ }\textbf {\bibinfo {volume} {117}},\ \bibinfo
  {pages} {213904} (\bibinfo {year} {2015})}\BibitemShut {NoStop}%
\bibitem [{\citenamefont {Mondal}\ and\ \citenamefont
  {Kamra}(2021)}]{Mondal2021PRBSpinCurrent}%
  \BibitemOpen
  \bibfield  {author} {\bibinfo {author} {\bibfnamefont {R.}~\bibnamefont
  {Mondal}}\ and\ \bibinfo {author} {\bibfnamefont {A.}~\bibnamefont {Kamra}},\
  }\bibfield  {title} {\bibinfo {title} {Spin pumping at terahertz nutation
  resonances},\ }\href {https://doi.org/10.1103/PhysRevB.104.214426} {\bibfield
   {journal} {\bibinfo  {journal} {Phys. Rev. B}\ }\textbf {\bibinfo {volume}
  {104}},\ \bibinfo {pages} {214426} (\bibinfo {year} {2021})}\BibitemShut
  {NoStop}%
\bibitem [{\citenamefont {Titov}\ \emph
  {et~al.}(2022{\natexlab{a}})\citenamefont {Titov}, \citenamefont {Dowling},\
  and\ \citenamefont {Kalmykov}}]{titov2022ferromagnetic}%
  \BibitemOpen
  \bibfield  {author} {\bibinfo {author} {\bibfnamefont {S.~V.}\ \bibnamefont
  {Titov}}, \bibinfo {author} {\bibfnamefont {W.~J.}\ \bibnamefont {Dowling}},\
  and\ \bibinfo {author} {\bibfnamefont {Y.~P.}\ \bibnamefont {Kalmykov}},\
  }\bibfield  {title} {\bibinfo {title} {Ferromagnetic and nutation resonance
  frequencies of nanomagnets with various magnetocrystalline anisotropies},\
  }\href {https://doi.org/10.1063/5.0093226} {\bibfield  {journal} {\bibinfo
  {journal} {J. Appl. Phys.}\ }\textbf {\bibinfo {volume} {131}},\ \bibinfo
  {pages} {193901} (\bibinfo {year} {2022}{\natexlab{a}})}\BibitemShut
  {NoStop}%
\bibitem [{\citenamefont {Titov}\ \emph
  {et~al.}(2021{\natexlab{b}})\citenamefont {Titov}, \citenamefont {Coffey},
  \citenamefont {Kalmykov},\ and\ \citenamefont
  {Zarifakis}}]{Titov2021Deterministic}%
  \BibitemOpen
  \bibfield  {author} {\bibinfo {author} {\bibfnamefont {S.~V.}\ \bibnamefont
  {Titov}}, \bibinfo {author} {\bibfnamefont {W.~T.}\ \bibnamefont {Coffey}},
  \bibinfo {author} {\bibfnamefont {Y.~P.}\ \bibnamefont {Kalmykov}},\ and\
  \bibinfo {author} {\bibfnamefont {M.}~\bibnamefont {Zarifakis}},\ }\bibfield
  {title} {\bibinfo {title} {Deterministic inertial dynamics of the
  magnetization of nanoscale ferromagnets},\ }\href
  {https://doi.org/10.1103/PhysRevB.103.214444} {\bibfield  {journal} {\bibinfo
   {journal} {Phys. Rev. B}\ }\textbf {\bibinfo {volume} {103}},\ \bibinfo
  {pages} {214444} (\bibinfo {year} {2021}{\natexlab{b}})}\BibitemShut
  {NoStop}%
\bibitem [{\citenamefont {Titov}\ \emph {et~al.}(2023)\citenamefont {Titov},
  \citenamefont {Dowling}, \citenamefont {Titov}, \citenamefont {Nikitov},\
  and\ \citenamefont {Cherkasskii}}]{TitovCorrelation2023}%
  \BibitemOpen
  \bibfield  {author} {\bibinfo {author} {\bibfnamefont {S.~V.}\ \bibnamefont
  {Titov}}, \bibinfo {author} {\bibfnamefont {W.~J.}\ \bibnamefont {Dowling}},
  \bibinfo {author} {\bibfnamefont {A.~S.}\ \bibnamefont {Titov}}, \bibinfo
  {author} {\bibfnamefont {S.~A.}\ \bibnamefont {Nikitov}},\ and\ \bibinfo
  {author} {\bibfnamefont {M.}~\bibnamefont {Cherkasskii}},\ }\bibfield
  {title} {\bibinfo {title} {Inertial dynamics and equilibrium correlation
  functions of magnetization at short times},\ }\href
  {https://doi.org/10.1103/PhysRevB.107.104416} {\bibfield  {journal} {\bibinfo
   {journal} {Phys. Rev. B}\ }\textbf {\bibinfo {volume} {107}},\ \bibinfo
  {pages} {104416} (\bibinfo {year} {2023})}\BibitemShut {NoStop}%
\bibitem [{\citenamefont {Bastardis}\ \emph {et~al.}(2018)\citenamefont
  {Bastardis}, \citenamefont {Vernay},\ and\ \citenamefont
  {Kachkachi}}]{Bastardis2018}%
  \BibitemOpen
  \bibfield  {author} {\bibinfo {author} {\bibfnamefont {R.}~\bibnamefont
  {Bastardis}}, \bibinfo {author} {\bibfnamefont {F.}~\bibnamefont {Vernay}},\
  and\ \bibinfo {author} {\bibfnamefont {H.}~\bibnamefont {Kachkachi}},\
  }\bibfield  {title} {\bibinfo {title} {Magnetization nutation induced by
  surface effects in nanomagnets},\ }\href
  {https://doi.org/10.1103/PhysRevB.98.165444} {\bibfield  {journal} {\bibinfo
  {journal} {Phys. Rev. B}\ }\textbf {\bibinfo {volume} {98}},\ \bibinfo
  {pages} {165444} (\bibinfo {year} {2018})}\BibitemShut {NoStop}%
\bibitem [{\citenamefont {Kittel}(1951)}]{Kittel1951}%
  \BibitemOpen
  \bibfield  {author} {\bibinfo {author} {\bibfnamefont {C.}~\bibnamefont
  {Kittel}},\ }\bibfield  {title} {\bibinfo {title} {Theory of
  antiferromagnetic resonance},\ }\href
  {https://doi.org/10.1103/PhysRev.82.565} {\bibfield  {journal} {\bibinfo
  {journal} {Phys. Rev.}\ }\textbf {\bibinfo {volume} {82}},\ \bibinfo {pages}
  {565} (\bibinfo {year} {1951})}\BibitemShut {NoStop}%
\bibitem [{\citenamefont {Nagamiya}(1951)}]{Takeo1951}%
  \BibitemOpen
  \bibfield  {author} {\bibinfo {author} {\bibfnamefont {T.}~\bibnamefont
  {Nagamiya}},\ }\bibfield  {title} {\bibinfo {title} {{Theory of
  Antiferromagnetism and Antiferromagnetic Resonance Absorption, II}},\ }\href
  {https://doi.org/10.1143/ptp/6.3.350} {\bibfield  {journal} {\bibinfo
  {journal} {Prog. Theor. Phys.}\ }\textbf {\bibinfo {volume} {6}},\ \bibinfo
  {pages} {350} (\bibinfo {year} {1951})}\BibitemShut {NoStop}%
\bibitem [{\citenamefont {Keffer}\ and\ \citenamefont
  {Kittel}(1952)}]{Keffer1952}%
  \BibitemOpen
  \bibfield  {author} {\bibinfo {author} {\bibfnamefont {F.}~\bibnamefont
  {Keffer}}\ and\ \bibinfo {author} {\bibfnamefont {C.}~\bibnamefont
  {Kittel}},\ }\bibfield  {title} {\bibinfo {title} {Theory of
  antiferromagnetic resonance},\ }\href
  {https://doi.org/10.1103/PhysRev.85.329} {\bibfield  {journal} {\bibinfo
  {journal} {Phys. Rev.}\ }\textbf {\bibinfo {volume} {85}},\ \bibinfo {pages}
  {329} (\bibinfo {year} {1952})}\BibitemShut {NoStop}%
\bibitem [{\citenamefont {Cherkasskii}\ \emph {et~al.}(2021)\citenamefont
  {Cherkasskii}, \citenamefont {Farle},\ and\ \citenamefont
  {Semisalova}}]{Cherkasskii2021}%
  \BibitemOpen
  \bibfield  {author} {\bibinfo {author} {\bibfnamefont {M.}~\bibnamefont
  {Cherkasskii}}, \bibinfo {author} {\bibfnamefont {M.}~\bibnamefont {Farle}},\
  and\ \bibinfo {author} {\bibfnamefont {A.}~\bibnamefont {Semisalova}},\
  }\bibfield  {title} {\bibinfo {title} {Dispersion relation of nutation
  surface spin waves in ferromagnets},\ }\href
  {https://doi.org/10.1103/PhysRevB.103.174435} {\bibfield  {journal} {\bibinfo
   {journal} {Phys. Rev. B}\ }\textbf {\bibinfo {volume} {103}},\ \bibinfo
  {pages} {174435} (\bibinfo {year} {2021})}\BibitemShut {NoStop}%
\bibitem [{\citenamefont {Titov}\ \emph
  {et~al.}(2022{\natexlab{b}})\citenamefont {Titov}, \citenamefont {Dowling},
  \citenamefont {Kalmykov},\ and\ \citenamefont
  {Cherkasskii}}]{Titov2022NutationWaves}%
  \BibitemOpen
  \bibfield  {author} {\bibinfo {author} {\bibfnamefont {S.~V.}\ \bibnamefont
  {Titov}}, \bibinfo {author} {\bibfnamefont {W.~J.}\ \bibnamefont {Dowling}},
  \bibinfo {author} {\bibfnamefont {Y.~P.}\ \bibnamefont {Kalmykov}},\ and\
  \bibinfo {author} {\bibfnamefont {M.}~\bibnamefont {Cherkasskii}},\
  }\bibfield  {title} {\bibinfo {title} {Nutation spin waves in ferromagnets},\
  }\href {https://doi.org/10.1103/PhysRevB.105.214414} {\bibfield  {journal}
  {\bibinfo  {journal} {Phys. Rev. B}\ }\textbf {\bibinfo {volume} {105}},\
  \bibinfo {pages} {214414} (\bibinfo {year} {2022}{\natexlab{b}})}\BibitemShut
  {NoStop}%
\bibitem [{\citenamefont {Makhfudz}\ \emph {et~al.}(2020)\citenamefont
  {Makhfudz}, \citenamefont {Olive},\ and\ \citenamefont
  {Nicolis}}]{Makhfudz2020}%
  \BibitemOpen
  \bibfield  {author} {\bibinfo {author} {\bibfnamefont {I.}~\bibnamefont
  {Makhfudz}}, \bibinfo {author} {\bibfnamefont {E.}~\bibnamefont {Olive}},\
  and\ \bibinfo {author} {\bibfnamefont {S.}~\bibnamefont {Nicolis}},\
  }\bibfield  {title} {\bibinfo {title} {Nutation wave as a platform for
  ultrafast spin dynamics in ferromagnets},\ }\href
  {https://doi.org/10.1063/5.0013062} {\bibfield  {journal} {\bibinfo
  {journal} {Appl. Phys. Lett.}\ }\textbf {\bibinfo {volume} {117}},\ \bibinfo
  {pages} {132403} (\bibinfo {year} {2020})}\BibitemShut {NoStop}%
\bibitem [{\citenamefont {Lomonosov}\ \emph {et~al.}(2021)\citenamefont
  {Lomonosov}, \citenamefont {Temnov},\ and\ \citenamefont
  {Wegrowe}}]{Lomonosov2021}%
  \BibitemOpen
  \bibfield  {author} {\bibinfo {author} {\bibfnamefont {A.~M.}\ \bibnamefont
  {Lomonosov}}, \bibinfo {author} {\bibfnamefont {V.~V.}\ \bibnamefont
  {Temnov}},\ and\ \bibinfo {author} {\bibfnamefont {J.-E.}\ \bibnamefont
  {Wegrowe}},\ }\bibfield  {title} {\bibinfo {title} {{Anatomy of inertial
  magnons in ferromagnetic nanostructures}},\ }\href
  {https://doi.org/10.1103/PhysRevB.104.054425} {\bibfield  {journal} {\bibinfo
   {journal} {Phys. Rev. B}\ }\textbf {\bibinfo {volume} {104}},\ \bibinfo
  {pages} {054425} (\bibinfo {year} {2021})}\BibitemShut {NoStop}%
\bibitem [{\citenamefont {Mondal}\ and\ \citenamefont
  {R\'ozsa}(2022)}]{Mondal2022}%
  \BibitemOpen
  \bibfield  {author} {\bibinfo {author} {\bibfnamefont {R.}~\bibnamefont
  {Mondal}}\ and\ \bibinfo {author} {\bibfnamefont {L.}~\bibnamefont
  {R\'ozsa}},\ }\bibfield  {title} {\bibinfo {title} {Inertial spin waves in
  ferromagnets and antiferromagnets},\ }\href
  {https://doi.org/10.1103/PhysRevB.106.134422} {\bibfield  {journal} {\bibinfo
   {journal} {Phys. Rev. B}\ }\textbf {\bibinfo {volume} {106}},\ \bibinfo
  {pages} {134422} (\bibinfo {year} {2022})}\BibitemShut {NoStop}%
\bibitem [{\citenamefont {Neeraj}\ \emph {et~al.}(2022)\citenamefont {Neeraj},
  \citenamefont {Pancaldi}, \citenamefont {Scalera}, \citenamefont {Perna},
  \citenamefont {d'Aquino}, \citenamefont {Serpico},\ and\ \citenamefont
  {Bonetti}}]{neeraj2021magnetization}%
  \BibitemOpen
  \bibfield  {author} {\bibinfo {author} {\bibfnamefont {K.}~\bibnamefont
  {Neeraj}}, \bibinfo {author} {\bibfnamefont {M.}~\bibnamefont {Pancaldi}},
  \bibinfo {author} {\bibfnamefont {V.}~\bibnamefont {Scalera}}, \bibinfo
  {author} {\bibfnamefont {S.}~\bibnamefont {Perna}}, \bibinfo {author}
  {\bibfnamefont {M.}~\bibnamefont {d'Aquino}}, \bibinfo {author}
  {\bibfnamefont {C.}~\bibnamefont {Serpico}},\ and\ \bibinfo {author}
  {\bibfnamefont {S.}~\bibnamefont {Bonetti}},\ }\bibfield  {title} {\bibinfo
  {title} {Magnetization switching in the inertial regime},\ }\href
  {https://doi.org/10.1103/PhysRevB.105.054415} {\bibfield  {journal} {\bibinfo
   {journal} {Phys. Rev. B}\ }\textbf {\bibinfo {volume} {105}},\ \bibinfo
  {pages} {054415} (\bibinfo {year} {2022})}\BibitemShut {NoStop}%
\bibitem [{\citenamefont {Winter}\ \emph {et~al.}(2022)\citenamefont {Winter},
  \citenamefont {Gro\ss{}enbach}, \citenamefont {Nowak},\ and\ \citenamefont
  {R\'ozsa}}]{Winter2022}%
  \BibitemOpen
  \bibfield  {author} {\bibinfo {author} {\bibfnamefont {L.}~\bibnamefont
  {Winter}}, \bibinfo {author} {\bibfnamefont {S.}~\bibnamefont
  {Gro\ss{}enbach}}, \bibinfo {author} {\bibfnamefont {U.}~\bibnamefont
  {Nowak}},\ and\ \bibinfo {author} {\bibfnamefont {L.}~\bibnamefont
  {R\'ozsa}},\ }\bibfield  {title} {\bibinfo {title} {Nutational switching in
  ferromagnets and antiferromagnets},\ }\href
  {https://doi.org/10.1103/PhysRevB.106.214403} {\bibfield  {journal} {\bibinfo
   {journal} {Phys. Rev. B}\ }\textbf {\bibinfo {volume} {106}},\ \bibinfo
  {pages} {214403} (\bibinfo {year} {2022})}\BibitemShut {NoStop}%
\bibitem [{\citenamefont {Makhfudz}\ \emph {et~al.}(2022)\citenamefont
  {Makhfudz}, \citenamefont {Hajati},\ and\ \citenamefont
  {Olive}}]{Makhfudz2022}%
  \BibitemOpen
  \bibfield  {author} {\bibinfo {author} {\bibfnamefont {I.}~\bibnamefont
  {Makhfudz}}, \bibinfo {author} {\bibfnamefont {Y.}~\bibnamefont {Hajati}},\
  and\ \bibinfo {author} {\bibfnamefont {E.}~\bibnamefont {Olive}},\ }\bibfield
   {title} {\bibinfo {title} {High-temperature magnetization reversal in the
  inertial regime},\ }\href {https://doi.org/10.1103/PhysRevB.106.134415}
  {\bibfield  {journal} {\bibinfo  {journal} {Phys. Rev. B}\ }\textbf {\bibinfo
  {volume} {106}},\ \bibinfo {pages} {134415} (\bibinfo {year}
  {2022})}\BibitemShut {NoStop}%
\end{thebibliography}%

\end{document}
