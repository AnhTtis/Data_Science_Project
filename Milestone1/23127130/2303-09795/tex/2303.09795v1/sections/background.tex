\section{Background}
\label{sec:background}

% We present open-source AI repositories and the issue trackers in GitHub.

\begin{figure*}[!t]
	\centering
	\includegraphics[width=0.9\linewidth]{figures/issue-example.pdf}
	\caption{An example of an issue in an AI repository hosted on GitHub.
    }
    \label{fig:issue-example}
\end{figure*}

\subsection{Open-Source AI Repositories}


The continued success of open-source software (OSS) in the field of AI, as exemplified by projects such as Tensorflow, has been widely acknowledged by both practitioners and researchers. The critical role that OSS plays in the advancement and development of AI techniques has been emphasized in various studies~\cite{sonnenburg2007need,Spohrer_2021}. In particular, the following five AI-related components should be made open source: design, implementation, frameworks, benchmarks, and models~\cite{Spohrer_2021}.


Artificial intelligence (AI) papers and technical reports that describe the design of algorithms and models are often made available to the public in the form of open-source materials. For example, authors are encouraged to publish their papers as open-access or release preprint versions on platforms such as \texttt{arXiv}. Additionally, the review process for many AI conferences is conducted publicly on platforms like \texttt{OpenReview}. Furthermore, the implementation of AI systems is commonly made open source as well. Developers release their code on platforms such as GitHub and BitBucket, enabling researchers and practitioners to replicate results and conduct further extensions. The key frameworks that support the implementation of AI software, such as \texttt{Tensorflow} and \texttt{PyTorch}, are also made open source to facilitate the development of various AI models. There are emerging open-source frameworks for specific tasks, such as \texttt{Carla} for autonomous driving~\cite{dosovitskiy2017carla} and \texttt{OpenAI Gym} for reinforcement learning~\cite{brockman2016openai}.

% and \texttt{ChatGPT} for chatbot~\cite{tutekchatgpt}.  Zhou: ChatGPT is not open-source I think?


Open-source public benchmark platforms, e.g., \texttt{CodaLab}\footnote{\url{https://codalab.org/}} and \texttt{Kaggle},\footnote{\url{https://www.kaggle.com/}} provide developers with opportunities to participate in competitions and share their AI solutions. Platforms also offer large pre-trained AI models that can be fine-tuned for downstream tasks. For instance, HugginingFace\footnote{\url{https://huggingface.co/}} is a platform that shares a variety of Transformer-based models and collections of datasets for various tasks, including natural language processing, audio processing, and speech recognition. Similarly, ModelZoo\footnote{\url{https://modelzoo.co/}} is another platform that shares pre-trained AI models. Giant tech companies like IBM and Google also actively participate in the open-source community by releasing their AI-related projects~\cite{strickland2019ibm, carvalho2019off, goralski2020artificial, tomavsev2020ai}.
Among these open-source platforms, GitHub is one of the most popular and hosts numerous AI repositories~\cite{zhang2019explorative, gonzalez2020state}. Therefore, in this study, we conduct an analysis of open-source AI repositories on GitHub to gain insights into their usage and popularity.

\subsection{Tracking Issues in GitHub}


The software system is not always reliable~\cite{heaven_2022, gundersen2018reproducible}. Developers of open-source software (OSS) repositories may encounter difficulties when employing these repositories. Introduced in 2009 for the first time, GitHub Issues~\cite{preston-werner_2009} is a built-in issue-tracking system integrated into GitHub that allows developers to raise questions, request features, etc.


Figure~\ref{fig:issue-example} illustrates an example of an issue in an open-source AI repository hosted on GitHub.\footnote{\url{https://github.com/deepinsight/insightface/issues/1281}} The person raising the issue, referred to as the \textit{issue raiser}, provides a title and a description of the issue. The issue can be in either an \textit{open} or \textit{closed} status, indicating whether it has been resolved or not. Any GitHub users who can access the repository can provide \textit{comments} to discuss the issue, such as adding extra details or suggesting solutions. The issue tracker has two basic functions for issue management: a \textit{label(s)} and assign issues to an \textit{assignee(s)}. Anyone with write access to the repository can create a label, and anyone with triage access to the repository can apply and dismiss labels. Only those with write access to the repository can specify the assignee(s). All the users involved in an issue, including the issue raiser, the commenter(s), and the assignee(s) are referred to as \textit{participants}. 

%\jh{Maybe we should indicate the issue raiser, the commenter, and the assignee in the figure 1. It is not clear now.}


There are various challenges when employing open-source AI repositories. For example, missing datasets and descriptions can make it difficult for users to evaluate AI models and replicate their results. Additionally, pre-trained AI models should be provided so that users can fine-tune them for other downstream tasks. Furthermore, frameworks used to develop AI software applications are constantly evolving; therefore, installing compatible development environments can be challenging for new users. Moreover, the documentation of open-source AI repositories may only mention the names of installed packages without specifying their versions, which can lead to runtime errors. Lastly, AI models are sensitive to hyperparameters~\cite{yang2020estimating, zhuang2022randomness}, and attempts to replicate the results may fail when using random hyperparameters. These issues can impede the usability of AI repositories. In this paper, we aim to understand these challenges to enable users to make better use of open-source AI repositories when developing their software applications.