\begin{thebibliography}{9}

\bibitem{nielsen} M. A. Nielsen and I. L. Chuang,
{\it Quantum Computation and Quantum Information},
Cambridge University Press, Cambridge, UK, 2000.

\bibitem{preskill} J. Preskill, Lecture Notes for the Course on Quantum
Computation, {\tt http://www.theory.caltech.edu/people/preskill/ph219/}

\bibitem{mukunda} N. Mukunda and R. Simon,
{\it Quantum kinematic approach to the geometric phase:
I. General formalism}, Ann. Phys. 228 (1993) 205-268.

\bibitem{wigner} E. P. Wigner,
{\it On the quantum correction to thermodynamic equilibrium},
Phys. Rev. 40 (1932) 749-759.

\bibitem{dimodd} See for example, F. A. Buot,
{\it Method for calculating Tr ${\cal H}^n$ in solid-state theory},
Phys. Rev. B 10 (1974) 3700-3705.

\bibitem{primefac} W.K. Wootters,
{\it A Wigner-function formulation of finite-state quantum mechanics},
Ann. Phys. 176 (1987) 1-21.

\bibitem{feyn_neg} R. Feynman, {\it Negative probability},
in {\it Quantum Implications: Essays in honour of David Bohm},
B.J. Hiley and F. David Peat (Eds.),
Routledge \& Kegan Paul, London (1987), 235-248.

\bibitem{bell} J. S. Bell,
{\it Speakable and Unspeakable in Quantum Mechanics},
Cambridge University Press, Cambridge, UK (1987).

\bibitem{eisert} A. Mari and J. Eisert,
{\it Positive Wigner functions render classical simulation of quantum
computation efficient}, Phys. Rev. Lett. 109 (2012) 230503.

\bibitem{gottesman} D. Gottesman,
{\it The Heisenberg represenation of quantum computers},
Proc. XXII International Collquium on Group Theoretical Methods in Physics,
S.P. Corney, R. Delbourgo and P.D. Jarvis (Eds.),
International Press, Cambridge, UK (1999), 32-43.

\bibitem{qchaos} S. Chaudhury, A. Smith, B. E. Anderson, S. Ghose and
P. S. Jessen, {\it Quantum signatures of chaos in a kicked top},
Nature 461 (2009) 768-771.

\bibitem{BinYan} B. Yan and W. Chemissany,
{\it Quantum chaos on complexity geometry}, {\tt arXiv:2004.03501} (2020).

\bibitem{kickedtop} See for example, V. Madhok, S. Dogra and A. Lakshminarayan,
{\it Quantum correlations as probes of chaos and ergodicity},
Optics Communications 420 (2018) 189-193.

\bibitem{discretelog} Y. Liu, S. Arunachalam and K. Temme,
{\it A rigorous and robust quantum speed-up in supervised machine learning},
Nature Physics 17 (2021) 1013-1017.

\bibitem{optbasis} H.-Y. Huang, R. Kueng and J. Preskill,
{\it Information-theoretic bounds on quantum advantage in machine learning},
Phys. Rev. Lett. 126 (2021) 190505.

\bibitem{hubregsten} T. Hubregsten, D. Wierichs, E. Gil-Fuster, P.-J. H.S.Derks,
P.K. Faehrmann and J.J. Meyer,
{\it Training quantum embedding kernels on near-term quantum computers},
Phys. Rev. A 106 (2022) 042431.

\bibitem{ankit} A. Khandelwal,
{\it Explorations in quantum machine learning},
M.S. Thesis, IISc (2022).

\bibitem{NISQ} J. Preskill,
{\it Quantum computing in the NISQ era and beyond},
Quantum 2 (2018) 79.

\bibitem{qcsimulators} See for instance,
{\tt http://quantiki.org/wiki/list-qc-simulators}

\bibitem{Qsim} H. Chaudhary, B. Mahato, L. Priyadarshi, N. Roshan,
Utkarsh and A. D. Patel,
{\it A software simulator for noisy quantum circuits},
Int. J. Mod. Phys. C 33 (2022) 2250103.

\bibitem{aakash} See
{\tt https://github.com/indian-institute-of-science-qc/qiskit-aakash}

\bibitem{qiskit} See
{\tt https://qiskit.org/} and {\tt https://github.com/Qiskit/qiskit-terra}\\
Qiskit has copyright under Apache License 2.0.

\bibitem{qctoolkit} See
{\tt https://qctoolkit.in}

\bibitem{shadows} H.-Y. Huang, R. Kueng and J. Preskill,
{\it Predicting many properties of a quantum system from very few measurements},
Nature Physics 16 (2022) 1050-1057.

\bibitem{qml} M. Schuld and F. Petruccione,
{\it Machine Learning with Quantum Computers},
Second edition, Springer Nature, Cham, Switzerland, 2021.

\end{thebibliography}
