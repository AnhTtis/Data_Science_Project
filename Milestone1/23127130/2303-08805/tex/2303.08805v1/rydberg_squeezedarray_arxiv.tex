\documentclass[aps, prl, reprint, superscriptaddress, twocolumn]{revtex4-1}
%\bibliographystyle{apsrev4-1}

% Imports
\usepackage{graphicx, subfigure}
\usepackage{amsmath, amssymb}
\usepackage{mathtools}
\usepackage{multirow}
\usepackage{textcomp}
\usepackage{float}
\usepackage{color}
\usepackage{xcolor}
\usepackage{ulem}
\usepackage{dsfont}
\usepackage{upgreek}
\usepackage{comment}
\usepackage{enumitem}
\usepackage{bm}
\usepackage{pdfpages}

\makeatletter
\AtBeginDocument{\let\LS@rot\@undefined}
\makeatother

% user comments
% \newcommand{\mss}[1]{\textcolor{blue}{\it#1}}
% \newcommand{\jah}[1]{\textcolor{red}{\it#1}}
% \newcommand{\glm}[1]{\textcolor{orange}{\it#1}}
% \definecolor{bluegreen}{HTML}{16b5b2}
% \newcommand{\svr}[1]{\textcolor{bluegreen}{\it#1}}
% \newcommand{\mdw}[1]{\textcolor{violet}{\it#1}}
% \newcommand{\nal}[1]{\textcolor{green}{\it#1}}

% units
\newcommand{\micro}[1]{\ensuremath{\upmu\mathrm{#1}}}
\newcommand{\nano}[1]{\ensuremath{\mathrm{n#1}}}
\newcommand{\milli}[1]{\ensuremath{\mathrm{m#1}}}
\newcommand{\kilo}[1]{\ensuremath{\mathrm{k#1}}}
\newcommand{\mega}[1]{\ensuremath{{\mathrm{M#1}}}}

% cesium
\newcommand{\Cs}{\ensuremath{^{133}\textrm{Cs}}}
\newcommand{\mCs}{\ensuremath{m_\mathrm{Cs}}}

% math
\newcommand{\abs}[1]{\left\lvert#1\right\rvert}
\renewcommand{\vec}[1]{\ensuremath{\bm{#1}}}
\newcommand{\uvec}[1]{\ensuremath{\mathbf{\hat{#1}}}}

% general quantum
\newcommand{\ket}[1]{\ensuremath{\left\vert{#1}\right\rangle}}
\newcommand{\expval}[1]{\ensuremath{\left\langle{#1}\right\rangle}}

% states
\newcommand{\state}[3]{\ensuremath{{#1}\textrm{#2}_{#3}}}
\newcommand{\pairstate}[6]{\ensuremath{ \ket{\state{#1}{#2}{#3}, \state{#4}{#5}{#6}} }}
\newcommand{\up}{\ensuremath{\ket{\uparrow}}}
\newcommand{\down}{\ensuremath{\ket{\downarrow}}}
\newcommand{\updressed}{\ensuremath{\tilde{\ket{\uparrow}}}}
\newcommand{\ryd}{\ensuremath{\ket{r}}}
\newcommand{\rydc}{\ensuremath{\ket{c}}}

% spins
\newcommand{\Sx}{\ensuremath{S_x}}
\newcommand{\Sy}{\ensuremath{S_y}}
\newcommand{\Sz}{\ensuremath{S_z}}
\newcommand{\contrast}{\ensuremath{\mathcal{C}}}

% populations
\newcommand{\nup}{\ensuremath{N_{c}^\uparrow}}
\newcommand{\nc}{\ensuremath{N_{c}}}
\newcommand{\fup}{\ensuremath{f_\uparrow}}
\newcommand{\loss}{\ensuremath{\ell}}

% stroboscopic pulse sequence
\newcommand{\tint}{\ensuremath{\tau_\mathrm{int}}}
\newcommand{\tp}{\ensuremath{\tau_p}}
\newcommand{\td}{\ensuremath{\tau_d}}
\newcommand{\npulses}{\ensuremath{M}}

% frequencies
\newcommand{\tDelta}{\ensuremath{\tilde{\Delta}}}
\newcommand{\Umax}{\ensuremath{U_\mathrm{max}}}
\newcommand{\Omegamax}{\ensuremath{\Omega_\mathrm{max}}}
\newcommand{\DeltaSq}{\ensuremath{\Delta_*}}

% rydberg dressing
\newcommand{\rc}{\ensuremath{r_c}}
\newcommand{\dressingfraction}{\ensuremath{\epsilon^2}}
\newcommand{\phiryd}{\ensuremath{\phi}}

% ramsey spectroscopy
\newcommand{\phiramsey}{\ensuremath{\varphi}}

% squeezing
\newcommand{\aopt}{\ensuremath{\alpha_\mathrm{opt}}}
\newcommand{\xs}{\ensuremath{\xi^2}}
\newcommand{\xsmin}{\ensuremath{\xi^2_\mathrm{min}}}
\newcommand{\xsmax}{\ensuremath{\xi^2_\mathrm{max}}}

% numbers
\newcommand{\averagenc}{\ensuremath{13}}
\newcommand{\minimumsqueezing}{\ensuremath{0.77(9)}}

\begin{document}
\preprint{APS/123-QED}
\title{Spin Squeezing by Rydberg Dressing in an Array of Atomic Ensembles}

% arxiv uses full first names
\author{Jacob~A.~Hines}
\affiliation{Department of Physics, Stanford University, Stanford, California 94305, USA}
\author{Shankari~V.~Rajagopal}
\affiliation{Department of Physics, Stanford University, Stanford, California 94305, USA}
\author{Gabriel~L.~Moreau}
\affiliation{Department of Physics, Stanford University, Stanford, California 94305, USA}
\author{Michael~D.~Wahrman}
\affiliation{Department of Applied Physics, Stanford University, Stanford, California 94305, USA}
\author{Neomi~A.~Lewis}
\affiliation{Department of Applied Physics, Stanford University, Stanford, California 94305, USA}
\author{Ognjen~Markovi\'{c}}
\affiliation{Department of Physics, Stanford University, Stanford, California 94305, USA}
\affiliation{Department of Physics, Harvard University, Cambridge, MA 02138, USA}
\author{Monika~Schleier-Smith}
\affiliation{Department of Physics, Stanford University, Stanford, California 94305, USA}

% PRL uses initials
% \author{J.~A.~Hines}
% \affiliation{Department of Physics, Stanford University, Stanford, California 94305, USA}
% \author{S.~V.~Rajagopal}
% \affiliation{Department of Physics, Stanford University, Stanford, California 94305, USA}
% \author{G.~L.~Moreau}
% \affiliation{Department of Physics, Stanford University, Stanford, California 94305, USA}
% \author{M.~D.~Wahrman}
% \affiliation{Department of Applied Physics, Stanford University, Stanford, California 94305, USA}
% \author{N.~A.~Lewis}
% \affiliation{Department of Applied Physics, Stanford University, Stanford, California 94305, USA}
% \author{O.~Markovi\'{c}}
% \affiliation{Department of Physics, Harvard University, Cambridge, MA 02138, USA}
% \author{M.~Schleier-Smith}
% \affiliation{Department of Physics, Stanford University, Stanford, California 94305, USA}

\date{\today}

\begin{abstract}
We report on the creation of an array of spin-squeezed ensembles of cesium atoms via Rydberg dressing, a technique that offers optical control over local interactions between neutral atoms. We optimize the coherence of the interactions by a stroboscopic dressing sequence that suppresses super-Poissonian loss.  We thereby prepare squeezed states of $N=200$ atoms with a metrological squeezing parameter $\xi^2 = \minimumsqueezing$ quantifying the reduction in phase variance below the standard quantum limit.  We realize metrological gain across three spatially separated ensembles in parallel, with the strength of squeezing controlled by the local intensity of the dressing light. Our method can be applied to enhance the precision of tests of fundamental physics based on arrays of atomic clocks and to enable quantum-enhanced imaging of electromagnetic fields.
\end{abstract}

\maketitle

%%% INTRODUCTION %%%
Quantum projection noise limits the precision of state-of-the-art measurements of time, acceleration, and electromagnetic fields based on spectroscopy of ensembles of atoms. Entanglement among the constituent two-state atoms, or equivalently spins, can enable enhanced precision by squeezing the quantum noise~\cite{kitagawa1993squeezed, wineland1994squeezed, pezze2018quantum}. Spin squeezing has been demonstrated in several experimental platforms featuring all-to-all interactions, including atoms in optical cavities~\cite{leroux2010implementation, hosten2016quantum, pedrozo2020entanglement, greve2022entanglement}, Bose-Einstein condensates~\cite{esteve2008squeezing, gross2010nonlinear, riedel2010atom, lucke2011twin, hamley2012spin, berrada2013integrated, ockeloen2013quantum, muessel2014scalable}, and ions coupled by collective motion~\cite{meyer2001experimental, bohnet2016quantum}. However, a wide range of metrological tasks stand to benefit from instead generating spin squeezing with local interactions. Notably, entangling Rydberg atoms~\cite{gil2014spin, bouchoule2002spin, opatrny2012spin, van2021impacts, kaubruegger2019variational, young2022enhancing}, cold molecules~\cite{bilitewski2021dynamical}, or solid-state spins~\cite{bennett2013phonon, xia2016generating} via their native interactions offers prospects for applying squeezing in optical tweezer clocks~\cite{norcia2019seconds, madjarov2019atomic}, electrometers~\cite{arias2019realization}, molecular spectroscopy~\cite{bilitewski2021dynamical}, and compact magnetometers~\cite{barry2020sensitivity}.

For local control of spin squeezing in systems of neutral atoms, several proposals have envisioned applying the method of Rydberg dressing~\cite{gil2014spin, kaubruegger2019variational, van2021impacts, young2022enhancing, mitra2022practical}. Here, an off-resonant laser field hybridizes one of two ground spin states with a Rydberg state to induce interactions with a characteristic range on the few-micron scale~\cite{pupillo2010strongly, johnson2010interactions, henkel2010three}. Such short-range interactions are ideally suited to generating arrays of independent squeezed states for spatially resolved sensing~\cite{muessel2014scalable}. By offering local and dynamical optical control~\cite{borish2020transverse, hollerith2022realizing}, Rydberg dressing further promises to enable metrological protocols employing multiple internally entangled ensembles to maximize the dynamic range of a sensor or the stability of a clock~\cite{borregaard2013efficient,rosenband2013exponential, kessler2014heisenberg}.

Several experiments have demonstrated coherent Rydberg-dressed interactions in small systems~\cite{jau2016entangling, zeiher2016many, zeiher2017coherent}. However, maintaining sufficient coherence to scalably engineer many-body entanglement has so far proven challenging~\cite{goldschmidt2016anomalous, aman2016trap, boulier2017spontaneous, hollerith2022realizing}. Decoherence is dominated by facilitated excitation, wherein a single atom that decays into a Rydberg state shifts the atomic transition for surrounding atoms into resonance with the dressing light, triggering an avalanche of subsequent excitations~\cite{desalvo2016rydberg, goldschmidt2016anomalous, aman2016trap, young2018dissipation, festa2022blackbody}. This accelerated multi-body excitation cycle is of greatest issue in systems with high dimensionality, such as 3D optical lattices~\cite{goldschmidt2016anomalous, boulier2017spontaneous} and bulk gases~\cite{aman2016trap}. These systems are particularly relevant in metrological applications, where increased particle number enables increased measurement precision.

\begin{figure}[t]
    \includegraphics[width=\columnwidth]{figures/1/figure1.pdf}
    \caption{\textbf{Experimental setup and Rydberg-dressed interactions}. (a)~Ensembles of cesium atoms are held in a one-dimensional array of microtraps and locally illuminated with $319~\nano{m}$ Rydberg dressing light to induce interactions of characteristic range $r_c$. (b)~Level diagrams for a single atom (left) and a pair of atoms (right), where $\ket{+} = (\ket{r\uparrow}+\ket{\uparrow r})/\sqrt{2}$ and $\updressed$ denotes the Rydberg-dressed state. (c)~The interactions generate an $S_z$-dependent precession (twisting) of the collective spin $\vec{S}$ that shears the uncertainty distribution of a coherent state (left) to create a squeezed spin state (right).}
    \label{fig:overview}
\end{figure}

In this Letter, we report on the generation of an array of spin-squeezed atomic ensembles by Rydberg dressing. For pseudospins encoded in the hyperfine clock states of cesium, we generate Ising interactions by off-resonantly coupling one clock state to a Rydberg state. Whereas applying the dressing light continuously induces super-Poissonian loss, a stroboscopic pulse sequence suppresses this loss to enable coherent interactions. We observe the dependence of the resulting spin squeezing on the local intensity of the dressing light across an array of atomic ensembles. We detect squeezing in three adjacent ensembles, with a minimum metrological squeezing parameter $\xs = \minimumsqueezing$.

%%% EXPERIMENTAL SETUP %%%
Our experiments are conducted in a one-dimensional array of optical microtraps [Fig.~\ref{fig:overview}(a)], consisting of nine sites with $25$~\micro{m} spacing. Each array site contains a cloud of typically $N=200$ cesium atoms with rms dimensions $[1.7(2),\, 1.7(2),\, 19(2)]~\micro{m}$, corresponding to a peak density $\rho_0=2.3(3)\times 10^{11}~\mathrm{cm}^{-3}$. We prepare the atoms in a superposition of the hyperfine clock states $\down=\ket{6S_{1/2}, F=3, m_F=0}$ and $\up=\ket{6S_{1/2}, F=4, m_F=0}$. To introduce Ising interactions within each array site, we dress the state $\up$ using a $319~\nano{m}$ laser field detuned by an amount $\Delta$ from the $60P_{3/2}$ Rydberg state $\ryd$, as illustrated in Fig.~\ref{fig:overview}(b). A nonuniform intensity of the dressing light across the array allows us to perform experiments at multiple Rabi frequencies $\Omega$ in parallel.

The interactions induced by the dressing light can be understood as a suppression of the ac Stark shift on each atom due to the influence of nearby atoms.  This effect is most pronounced for an ensemble of $N$ atoms localized within a critical length scale $\rc \approx \abs{C_6/2\Delta}^{1/6}$, below which the van der Waals interaction $V_\mathrm{R} = C_6/r^6$ between two Rydberg atoms exceeds the pair-state detuning $2\Delta$.  In this idealized limit, the Hamiltonian takes the form
\begin{equation}\label{eq:twisting_hamiltonian}
    H \approx U_0 \Sz - \frac{\chi}{N} \Sz^2,
\end{equation}
where $\Sz = (N_\uparrow - N_\downarrow)/2$ denotes the population difference between the clock states.  Here, $U_0$ denotes an overall ac Stark shift that can readily be removed by spin echo, while $\chi$ parameterizes the mean-field interaction, which manifests in an $\Sz$-dependence of the ac Stark shift.  The resulting $\Sz$-dependent spin precession, termed one-axis twisting~\cite{kitagawa1993squeezed}, provides a means of squeezing quantum fluctuations, as shown in Fig.~\ref{fig:overview}(c).

Our experiment is guided by this idealized model of spin squeezing by one-axis twisting but must contend with two key factors beyond it.  Firstly, we operate with atomic clouds larger than the interaction ellipsoid with radii $\rc^{\mathrm{x,y,z}} \approx (3,5,5)~\micro{m}$, so the collective spin model in Eq.~\ref{eq:twisting_hamiltonian} only approximately describes the dynamics~\cite{SM}.  Secondly, the squeezing must compete with decay of the Rydberg-dressed state, which in practice is often exacerbated by multi-body loss processes that induce super-Poissonian noise~\cite{goldschmidt2016anomalous, aman2016trap, boulier2017spontaneous}.  We focus first on minimizing such loss to optimize the coherence of the dressing, before examining the role of the finite interaction range and observing the resulting squeezing.

%%% COHERENCE %%%
\begin{figure}[tb]
    \includegraphics[width=\columnwidth]{figures/2/figure2.pdf}
    \caption{\textbf{Maximizing coherence by stroboscopic Rydberg dressing}. (a)~An equal superposition of states $\up$ and $\down$ is prepared by a $\pi/2$ microwave rotation (purple) and subjected to Rydberg dressing pulses of length $\tau_p$ (blue) at intervals $\tau_d$.  (b)~Schematic showing the creation of atoms in contaminant states $\rydc$ (orange) and their influence on dressed atoms $\updressed$ (purple-blue). (c)~Histograms of $S_z$ for a single microtrap when the dressing light is (blue) or is not (gray) applied. The broadening observed for a continuous pulse (top, $\tau_d=0$) is suppressed by dressing with short pulses separated by a delay time $\tau_d = 250~\micro{s}$ (bottom). (d)~Normalized variance versus loss, plotted across all microtraps for $\td=0~\micro{s}$. A linear fit (red line) reports atoms being lost in groups of size $g=17(1)$, while Poissonian loss ($g=1$) is shown in green. (e)~Normalized variance, averaged across three central microtraps, versus pulse delay for $\tp=628~\nano{s}$ and $\npulses=48$. A fit of the form $\sigma^2 = A\exp(-\gamma\td)+1$ yields a characteristic timescale $\gamma^{-1}=29(9)~\micro{s}$. The ``$\boldsymbol{\times}$'' markers in (d) and (e) denote the microtrap and pulse delays for the data shown in~(c).}
    \label{fig:coherence}
\end{figure}

We maximize coherence in our system by implementing a stroboscopic dressing sequence designed to suppress facilitated excitation to the Rydberg state~[Fig.~\ref{fig:coherence}(a-b)].  We apply the dressing light in a sequence of pulses, each smoothly shaped to ensure that the dressing is adiabatic and ideally leaves all atoms in the ground state $\up$ at the end of the pulse.  Non-idealities, including incoherent excitation due to laser phase noise and blackbody decay to nearby Rydberg $S$ and $D$ states that are dipole-coupled to the dressing state $\ryd$, can nevertheless lead to atoms populating the Rydberg manifold.  A separation $\td$ between the pulses provides time for any contaminant atoms to decay or be expelled by antitrapping, thereby averting effects where interactions with a contaminant atom shift the surrounding atoms into resonance with the dressing light.  Such stroboscopic dressing was proposed in Refs.~\cite{zeiher2016many,boulier2017spontaneous} and implemented in Ref.~\cite{borish2020transverse} for measurements of the mean-field dynamics induced by Rydberg dressing.

For spin squeezing, optimization of the dressing pulse sequence is essential to avoiding even subtle loss processes that add percent-level noise to the quantum state.  To probe such loss, we first prepare each atom in an equal superposition state $\ket{\pi/2} = \left(\up+\down\right)/\sqrt{2}$, obtained by a $\pi/2$ microwave rotation of the initial state $\down$ [Fig.~\ref{fig:coherence}(a)].  We then apply a sequence of dressing pulses separated by a variable time $\td$.  Since the dressing light affects only state $\up$, any light-induced loss manifests in a population difference between the two spin states, which we read out by state-sensitive fluorescence imaging.

Figure~\ref{fig:coherence}(c) shows representative histograms of the normalized population imbalance $\Sz/N$ between the two clock states in a single microtrap after a total dressing time $\tint = 30$~\micro{s}.  For dressing light applied in a single long pulse ($\td=0$), we observe loss from state $\up$ and accompanying noise in the atomic state populations.  Performing the same analysis for all microtraps, which experience different levels of loss due to the spatially varying intensity of the dressing light, we plot the variance of the population imbalance versus loss in Fig.~\ref{fig:coherence}(d).  Specifically, we define $\sigma^2 = 4(\Delta\Sz)^2/N$ as the variance normalized to that of a coherent spin state and plot $\sigma^2$ as a function of the fractional loss $\loss = (\expval{\Sz}_0 - \expval{\Sz})/N$, where $\expval{\Sz}_0\approx 0$ is the population imbalance measured in the absence of dressing light.  For small loss $\loss$, we observe a growth $\sigma^2 \approx 1 + g\loss$, where the slope $g = 17(1)$ exceeding unity evidences the super-Poissonian statistics of the loss.

The loss induced by the dressing light is suppressed by introducing a delay between the pulses.  In particular, we divide the total dressing time $\tint$ into $\npulses = 48$ pulses spaced by a variable delay time.  The histogram of the state populations after this dressing sequence with delay $\td = 250$~\micro{s} [Fig.~\ref{fig:coherence}(c)] exhibits substantially reduced loss and negligible broadening.  We plot the dependence of the spin noise $\sigma^2$ on the delay $\td$ in Fig.~\ref{fig:coherence}(e).  Fitting these data reveals that the noise decays to the quantum projection noise level $\sigma^2=1$ on a characteristic timescale $\gamma^{-1} = 29(9)~\micro{s}$. We attribute this timescale to a combination of radiative decay from the Rydberg manifold and ejection of Rydberg atoms out of the microtraps due to the repulsive ponderomotive force, both of which occur on times of order $100~\micro{s}$ ~\cite{dutta2000ponderomotive,SM}. For the remainder of this work, we set $\td=100~\micro{s}$ to ensure negligible broadening of the $S_z$ distribution.

%%% LIGHT SHIFT %%%
Observing the interactions induced by Rydberg dressing requires measuring the $\Sz$-dependent phase accrual due to the dressing light.  Specifically, the dressing light shifts the clock transition of each atom by an amount
\begin{equation}\label{eq:light_shift}
    U = \frac{\Omega^2}{4\Delta}\cdot\frac{1}{\sqrt{1 + \nup\left(\Omega/\Delta\right)^2}},
\end{equation}
in units where $\hbar=1$.  Here $N_{c}^{(\uparrow)}$ denotes the number of surrounding atoms (in state $\up$) within the interaction ellipsoid of radii $\vec{r}_c(\Delta)$.  In the ideal case where all atoms are confined within the interaction range ($\nc = N$), expanding Eq.~\ref{eq:light_shift} in powers of $\Sz = \nup - N/2$ yields the one-axis twisting Hamiltonian in Eq.~\ref{eq:twisting_hamiltonian}.  More generally, we expect similar twisting dynamics~\cite{gil2014spin,borish2020transverse,SM} with a collective interaction strength $\chi$ set by the number of neighbors $\nc$ as
\begin{equation}
\chi = -\left(\frac{N}{2}\frac{dU}{d\Sz}\right)_{\Sz=0} = \frac{\nc}{16}\frac{\Omega^4}{\tDelta^3},
\end{equation}
where $\tDelta = \sqrt{\Delta^2 + \nc\Omega^2/2}$.

\begin{figure}[tb]
    \includegraphics[width=\columnwidth]{figures/3/figure3.pdf}
    \caption{\textbf{Quantifying interactions}. (a)~Sequence of microwave (purple) and Rydberg dressing (blue) pulses for measuring the ac Stark shift $U$ by Ramsey spectroscopy. (b)~Measured light shift $U$ versus detuning $\Delta$ for $\theta =$ ($\pi$/4,\, $\pi$/2,\,3$\pi$/4) (green squares, orange triangles, blue circles). Solid lines show fits to Eq.~\ref{eq:light_shift} used to determine $\Omega$ and $\nup$. Inset: visualization of the accumulated phase $\phiryd$ as a function of initial tilt $\theta$. (c)~Fitted $\nup(\DeltaSq)$ versus $\cos^2\left(\theta/2\right)$. Red line denotes a linear fit with slope $\nc$. (d)~Fitted number of neighbors $\nc$ (bottom) and Rabi frequency $\Omegamax$ (top) by microtrap index.}
    \label{fig:lightshift}
\end{figure}

To experimentally determine the number of interacting neighbors and the collective interaction strength, we measure the ac Stark shift for atoms prepared in different initial states $\ket{\theta} = \cos{\left(\theta/2\right)}\up + \sin{\left(\theta/2\right)}\down$.  We perform each measurement via the Ramsey sequence shown in Fig.~\ref{fig:lightshift}(a), where stroboscopic dressing is followed by a $\pi/2$ microwave rotation that converts the acquired phase into a measurable population difference.  Based on the total phase shift $\phiryd = \int U(t)\, dt$ measured in the Ramsey sequence and the known shape of the dressing pulse, we determine the peak ac Stark shift $\Umax$~\cite{SM}.  The dependence of $\Umax$ on detuning $\Delta$ is shown in Fig.~\ref{fig:lightshift}(b) for three different polar angles $\theta = (\pi/4,\, \pi/2,\, 3\pi/4)$ of the collective Bloch vector.  The suppression of the ac Stark shift with decreasing polar angle evidences interactions among the Rydberg-dressed atoms in state $\up$.

We quantify the interactions by fitting the dependence of the ac Stark shift $\Umax$ on detuning.  These fits reveal both the peak Rabi frequency $\Omegamax$ and the number of interacting neighbors $\nup \propto \rho \cos^2(\theta/2)\prod_\alpha \rc^\alpha(\Delta)$ for each microtrap.  The data corroborate the expected dependence $\nup \propto \cos^2(\theta/2)$ of the number of interacting neighbors on the tilt of the Bloch vector, shown in Fig.~\ref{fig:lightshift}(c) for a representative detuning $\DeltaSq = 2\pi\times 8~\mathrm{MHz}$.  Linear fits of the form $\nup= \nc\cos^2(\theta/2)$ reveal the total number of neighbors $\nc$ within the interaction ellipsoid in each microtrap, which we summarize in Fig.~\ref{fig:lightshift}(d).  The result of $\nc \sim \averagenc$ neighbors, approximately consistent with the atomic density and the calculated Rydberg-dressed potential~\cite{SM}, confirms that the total system size is approximately $N/\nc = 15$ times larger than the interaction ellipsoid, as illustrated in Fig.~\ref{fig:overview}(a).

%%% SQUEEZING %%%
\begin{figure*}[tb]
    \includegraphics[width=\textwidth]{figures/4/figure4.pdf}
    \caption{\textbf{Spin squeezing}. (a)~Groups of four dressing pulses and one spin echo pulse are applied to implement one-axis twisting on an initial state $\ket{\pi/2}$. The resulting squeezed state is rotated by an angle $\alpha$ about the mean spin vector prior to readout of $\Sz$. (b)~Squeezing parameter $\xs$ for one microtrap as a function of final rotation angle $\alpha$, measured with (blue circles) and without (gray squares) dressing pulses applied. (c)~Minimum squeezing parameter $\xsmin$ (green circles) and maximum squeezing parameter $\xsmax$ (orange diamonds) in the presence of dressing light, as well as twisting strength $Q$ (blue circles), are plotted versus microtrap index. Gray squares represent $\xs$ evaluated at $\aopt$ with no dressing light, while the gray dotted line denotes $\contrast_0^{-2}$. Blue shaded region serves as a guide to the eye for the intensity of the dressing light, which is used to predict squeezing (orange shaded) and antisqueezing (green shaded). Yellow shading indicates the microtrap shown in (b). A fluorescence image of the microtrap array is shown at top. (d)~Squeezing (green circles) and antisqueezing (orange diamonds) versus twisting strength, plotted across microtraps. Solid lines denote parameter-free model of one-axis twisting for $N_c$ atoms with initial contrast $\contrast_0$. Dotted lines represent the same model with 7\% additional technical noise, in accordance with excess measured noise on the data without dressing light in (c).}
    \label{fig:squeezing}
    
\end{figure*}

To examine the influence of these local interactions on quantum fluctuations, we initialize all atoms in a spin-polarized state $\ket{\theta}=\ket{\pi/2}$ along the $x$-axis and apply stroboscopic Rydberg dressing.  We isolate the twisting effect of the Ising term in Eq.~\ref{eq:twisting_hamiltonian} by a spin echo sequence that removes the average ac Stark shift $U_0$.  The sequence consists of microwave $\pi$ pulses applied between groups of four Rydberg dressing pulses [Fig.~\ref{fig:squeezing}(a)], frequently enough to combat motional dephasing.  After applying $\npulses = 48$ dressing pulses at a detuning $\DeltaSq=2\pi\times 8~\mega{Hz}$ and a central Rabi frequency $\Omega \approx 2\pi\times 1.2~\mega{Hz}$, we measure the spin projection in a given quadrature $S_\alpha = \Sz\cos(\alpha) + \Sy\sin(\alpha)$ by performing a final microwave rotation by an angle $\alpha$ about the mean spin vector $\expval{\vec{S}}\propto\uvec{x}$ and reading out $\Sz$ via state-sensitive fluorescence imaging.

To quantify spin squeezing [Fig.~\ref{fig:squeezing}(b)], we plot the Wineland parameter~\cite{wineland1994squeezed}
\begin{equation}\label{eq:squeezing_parameter}
    \xs = \frac{\left(\Delta\phi\right)^2}{\left(\Delta\phi_\mathrm{SQL}\right)^2} = \frac{\left(\Delta S_\alpha\right)^2}{\abs{\left\langle \mathbf{S} \right\rangle}^2/N},
\end{equation}
defined such that $\xs < 1$ enables an improvement in phase sensitivity due to entanglement.  Here $\Delta\phi_\mathrm{SQL} = 1/\sqrt{N}$ is the angular uncertainty of a coherent spin state of $N$ atoms and $\abs{\left\langle \mathbf{S} \right\rangle}$ is the length of the total spin vector as determined from the contrast $\contrast = 2\abs{\left\langle \mathbf{S} \right\rangle}/N$ of a Ramsey fringe. We calibrate the atom number $N$ by measurements of the quantum projection noise of coherent spin states~\cite{SM}, where we remove a small amount of common-mode technical noise by a linear regression across microtraps. This same regression is applied to measurements of $\Delta S_\alpha$. The contrast is limited to $\contrast_0 = 0.95(1)$ by inhomogeneous trap light shifts that are imperfectly canceled by spin echo due to atomic motion.  We choose a sufficiently short interaction time that the additional contrast loss due to the dressing light, including atom loss, is at most 1\%.

To investigate the dependence of the squeezing on the interaction strength $\chi$, we leverage the variation in intensity of the dressing light across the array.  Figure~\ref{fig:squeezing}(c) shows the minimum squeezing parameter $\xsmin\equiv\xs(\aopt)$ (green circles) for each of the nine array sites, compared with the value for the same quadrature in the absence of dressing light (gray squares).  In addition, we plot an independent calibration of the twisting strength $Q\equiv \int\chi(t)\,dt$, based on the phase accumulated by coherent states $\ket{\theta}$ with different initial tilts in the full dressing sequence with spin echo~\cite{borish2020transverse,SM}.  The twisting strength varies as a function of position in the array due to the intensity profile of the dressing light~\cite{SM}.  We observe the strongest squeezing in the array sites with the largest twisting strength, and correspondingly strong antisqueezing $\xsmax\equiv\xs(\aopt-\pi/2)$ in the orthogonal quadrature (orange diamonds).

The dependence of squeezing and antisqueezing on twisting strength is summarized in Fig.~\ref{fig:squeezing}(d).  For comparison, the solid curves show a model of one-axis twisting with $N_c = \averagenc$ neighbors, accounting for the finite initial contrast $\contrast_0$. The squeezing and antisqueezing are consistent with the model predictions augmented by a small amount of technical noise. Our measurement includes all detection noise, which is on the scale of 3\% of the quantum projection noise. Other factors contributing excess noise may include laser intensity fluctuations and residual effects of rare contaminant atoms.

The observed improvement in squeezing with increasing twisting strength suggests that stronger squeezing is attainable at higher laser intensity or longer interaction time.  We limit the duration of each dressing pulse to $\tau_p \approx600~\mathrm{ns}$ to avoid excess contrast loss attributable to contaminant atoms~\cite{SM}.  We also limit the total duration of the stroboscopic dressing sequence to minimize trap-induced dephasing. This effect could be mitigated by improved cooling or state-insensitive trapping to access longer interaction times in future experiments.  In addition, a significantly higher intensity of the dressing light could be achieved by addressing the ensembles sequentially with a focused beam.

Accessing stronger twisting will allow for observing limits to squeezing due to the finite interaction range.   The inhomogeneous density of the atomic cloud is predicted to limit the squeezing to approximately $\xi^2_\mathrm{inh}=0.3$~\cite{SM, van2021impacts} for our parameters.  Overcoming this limit, either in an ordered array or in a trapping potential that is shaped to provide uniform density, would enable squeezing by an amount  $\xi^2_c \propto N_c^{-2/3}$ set by the number of interacting neighbors, which might further be improved by addition of a transverse field~\cite{young2022enhancing,borish2020transverse} or by leveraging atomic motion to spread correlations beyond the interaction range.

%Ultimately, the inhomogeneous density of the atomic cloud is predicted to limit the squeezing to approximately $\xi^2_\mathrm{inh}=0.3$~\cite{SM, van2021impacts}.  This limit can be overcome either in an ordered array or in a trapping potential that is shaped to provide uniform density.

Our demonstration of optically controlled spin squeezing in an array of atomic ensembles opens the door to enhancing multiplexed atomic clocks and sensors.  Prospective applications include clock comparisons for tests of fundamental physics~\cite{zheng2022differential}, cascaded interrogation schemes for clocks limited by local oscillator noise~\cite{borregaard2013efficient,kessler2014heisenberg}, and quantum-enhanced imaging of magnetic~\cite{muessel2014scalable, yang2020nematic} or electric~\cite{arias2019realization} fields.  The ability to access many-body entanglement by stroboscopic Rydberg dressing further promises to advance quantum simulations of lattice spin models that benefit from optical control of long-range interactions~\cite{glaetzle2014quantum,glaetzle2015designing,potirniche2017floquet,zeiher2017coherent,van2021impacts,steinert2022spatially}.

\textit{Note}: During completion of this manuscript, we became aware of related work on spin squeezing by Rydberg dressing in an optical tweezer clock~\cite{kaufman2023}.

%%% END %%%
\begin{acknowledgments}
This work was supported by the ARO under grant numbers W911NF-20-1-0136 and W911NF-16-1-0490.  We additionally acknowledge support from the AFOSR under grant No. FA9550-20-1-0059 (J.~A.~H. and N.~L.), the Stanford Science Fellowship (S.~V.~R.), the National Science Foundation Graduate Research Fellowship (G.~M.),  the ONR under grant No. N00014-17-1-2279 (O.~M.), and the DOE Q-NEXT Quantum Center (M.~S.-S.).  We thank Adam Kaufman, Immanuel Bloch, Dan Stamper-Kurn, and Manuel Endres for stimulating discussions.
\end{acknowledgments}

\bibliography{references/references, references/zotero}

\clearpage
\includepdf[pages={{},1,{},2,{},3,{},4,{},5,{},6,{},7,{},8,{},9,{},10}]{rydberg_squeezedarray_supplement.pdf}

\end{document}