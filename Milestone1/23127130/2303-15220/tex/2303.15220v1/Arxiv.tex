\documentclass{article}


\usepackage{PRIMEarxiv}
\usepackage[utf8]{inputenc} % allow utf-8 input
\usepackage[T1]{fontenc}    % use 8-bit T1 fonts
\usepackage{multirow}
\usepackage{hyperref}       % hyperlinks
\usepackage{url}            % simple URL typesetting
\usepackage{booktabs}       % professional-quality tables
\usepackage{amsfonts}       % blackboard math symbols
\usepackage{nicefrac}       % compact symbols for 1/2, etc.
\usepackage{microtype}      % microtypography
\usepackage{lipsum}
\usepackage{fancyhdr}       % header
\usepackage{graphicx}       % graphics
\graphicspath{{media/}}     % organize your images and other figures under media/ folder

%Header
\pagestyle{fancy}
\thispagestyle{empty}
\rhead{ \textit{ }} 

% Update your Headers here
\fancyhead[LO]{Soft Skills Centrality in Graduate Studies Offerings}
% \fancyhead[RE]{Firstauthor and Secondauthor} % Firstauthor et al. if more than 2 - must use \documentclass[twoside]{article}



  
%% Title
\title{Soft Skills Centrality in Graduate Studies Offerings
%%%% Cite as
%%%% Update your official citation here when published 
% \thanks{\textit{\underline{Citation}}: 
% \textbf{Authors. Title. Pages.... DOI:000000/11111.}} 
}

\author{
  María del Pilar García-Chitiva\href{https://orcid.org/0000-0001-6776-3422}{{\includegraphics[width=0.3cm]{orcid.jpg}}} \& Juan C. Correa\href{https://orcid.org/0000-0002-0301-5641}{{\includegraphics[width=0.3cm]{orcid.jpg}}} \\
  Instituto Tecnológico y de Estudios Superiores de Monterrey \\
  Monterrey, México\\
  \texttt{\{pilargarciach, jcorrean\}@tec.mx} \\
  %% examples of more authors
  %  \And
  % Author3 \\
  % Affiliation \\
  % Univ \\
  % City\\
  % \texttt{email@email} \\
  % %% \AND
  % %% Coauthor \\
  % %% Affiliation \\
  % %% Address \\
  % %% \texttt{email} \\
  % %% \And
  % %% Coauthor \\
  % %% Affiliation \\
  % %% Address \\
  % %% \texttt{email} \\
  % %% \And
  % %% Coauthor \\
  % %% Affiliation \\
  % %% Address \\
  %% \texttt{email} \\
}


\begin{document}
\maketitle


\begin{abstract}
Is it possible to measure how important soft skills like leadership or teamwork are from the point of view of graduate studies offerings? This paper provides a framework that introduces the concept of network centrality as a practical way to measure the individual importance of soft skills in graduate studies. We examine 230 graduate programs offered by 49 higher education institutions in Colombia to estimate the empirical importance of soft skills in the context of graduate studies offerings. The results show that: a) graduate programs in Colombia tend to share a common set of 31 soft skills in their intended learning outcomes; b) the centrality of these skills varies as a function of the graduate pro- gram, although this variation was not statistically significant; and  c)  while  most central soft skills tend to be those related to creativity (i.e., creation or generation of ideas or projects), leadership (to lead or teamwork),  and  analytical  orientation (e.g., evaluating situations and solving problems), less central were those related to empathy (i.e., understanding others and acknowledgment of others), ethical thinking, and critical thinking, posing the question if too much emphasis on most visible skills might imply an unbalance in the opportunities to enhancing other soft skills such as ethical thinking.
\end{abstract}


% keywords can be removed
\keywords{Higher Education \and Sustainable Development Goals \and Soft-Skills Training \and Bipartite Network \and Natural Language Processing}


\section{Introduction}
How well-prepared are university graduates to cope with the challenges of their work environment? This question captures the attention of professors, educational policy-makers, and university rectors alike. In the past, answering attempts included but were not limited to follow-up interviews with graduates, philosophical debates, professional workshops, and stakeholders' perceptions and opinions \cite{Crossman2010599}. More recently, however, the sustainable development goals posed by United Nations are gaining momentum with knowledge and skills that graduates should demonstrate when looking for jobs \cite{Chaka2022,Succi20201834,Jamison2021145}. Higher education institutions, in this sense, are summoned to deal with global challenges, such as sustainability, roughly understood as the societal goal of achieving safe and prosperous co-existence for people on Earth over time \cite{Diez2020509,Cottafava2019521,Ilham2020121}. From this viewpoint, it makes sense for members of the higher education system to ask how professionals should be educated and trained to work on projects with deadlines and non-trivial goals, such as reversing global warming. 

When the success of a project depends on individual goals achievement with pre-established deadlines, procrastination is a risk that demands self-control \cite{Ariely2002}, but this is a psychological feature fundamentally disconnected from the mainstream research focusing on soft skills in higher education (see for example,\cite{Andrews2008411,Ritter201880,Botke2018130}. According to \cite{Scheerens2020}, soft skills are also \textit{social and emotional}. As a movement, the rise of soft skills has to do with keeping education responsive to societies' needs (e.g., changing demands of the labor market) and cultural trends (e.g., emotional intelligence to respect others' opinions and behavior). These needs and trends go hand-in-hand with developing social-emotional learning (SELs) programs for student-faculty interaction \cite{Awang-Hashim2022635}. This complex set of needs, demands, and trends co-exist in a labor market where graduates compete regarding employability, knowledge, skills, and experience.  

The relevance of soft skills (e.g., leadership, critical thinking, persuasion) for the labor market is evident in several professional disciplines \cite{Coelho202278}, including but not limited to: data scientists \cite{Borner201812630}, mechanical engineers \cite{Rovida20231541}, physicians \cite{Riskiyana20222174}, librarians \cite{Hamid2022263}, or foreign languages \cite{Medvedeva2022}. Nonetheless, while these soft skills are paramount, recent evidence shows that students are not well-trained in how to talk about or show their soft skills. In a recent study, for example, it was reported that students tended to leave blank professional key sections in their LinkedIn profiles, could not communicate their unique value propositions, and described their experience section with poor messages, revealing that the profiles of already-employed individuals tended to be better than those unemployed \cite{Daniels202390}. Based on these facts, some scholars have claimed the use of LinkedIn as a pedagogical tool for careers and employability learning in higher education \cite{Healy2023106}.

When one studies the literature on soft skills, it is easy to find these social and emotional abilities in existent taxonomic attempts at a more ample range of skills like Burning Glass Technologies, O*NET, or similar skills classifications \cite{Borner201812630}. Despite these attempts, in a recent systematic review, it was showed that soft skills are ill-defined and suffer from theoretical dispersion \cite{Marin2022}. This work aims to extend the knowledge of soft skills in at least three significant orientations. First, we argue the value of  examining the explicit presence of soft skills in the academic offerings available in higher education institutions. To our knowledge, such analysis does not exist in the literature on higher education. Above and beyond this gap, however, we are keenly aware of the strategic nature of the relationship between soft skills and graduate studies offerings aiming at increasing the number of enrolled students in formal courses. Finally, the scrutiny of these sources of information becomes relevant for researchers interested in describing the complexity surrounding soft skills training in the context of graduate studies. Using the concept of \textit{centrality} from the perspective of complex network structures \cite{Estrada2011}, we show how non-trivial insights might be evident for university rectors and educational policy-makers alike. An added value is present in this work. Interested researchers might learn from our supplemental materials that reveal how to proceed with the analysis of the data with sufficient computational detail. As compared with other studies that rely on structured data collected through surveys or questionnaires, in this work the data analysis requires the combination of natural language processing with bipartite network analysis. Although this combination of techniques is well-known in social sciences (see, for example,  \cite{Bail2016}), its application in higher education research is unknown.

The early stage of this particular research orientation leads us to define a sample choice for analytical purposes. We take the case of higher education institutions in Colombia for two main reasons. First, Colombia has been a systematic object of study in higher education research during the last decade \cite{Alvarez2022,Duque2021669,Bradford2018909,Berry2014,Melguizo2011}. Despite these investigations, the literature on soft skills training remains scarce in Colombia \cite{Jaimes2022, Renteria2022} and other Latin American countries. The second reason to sample higher education institutions from Colombia is that this country became member of the \textit{Organization for Economic Co-operation and Development} (OECD) on April 28, 2020, which implies that some soft skills like leadership, teamwork, or creativity are paramount, as suggested by previous OECD reports \cite{Ocde2016-nq}. Developing these skills allows future professionals to work in a more globalized scenario where Colombia is expected to boost its current capacities \cite{Zarate2023}. In this context, for example, competent graduates might work on attracting foreign investors, but they require to master their persuasion skills should they want to be successful \cite{Dellavigna2010}. As graduates need to be trained to compete under the rules of new economic agents, it is widely known that executive business programs, for example, are the standard mechanism that facilitates the next generation of managers to succeed \cite{Lorange2019}. In this work, we also extend this view by sampling various graduate studies with and without a business-oriented purpose. 

The organization for the remainder of this paper is as follows. In the next section, we describe Colombia and its higher education offering for graduate studies. Then, we elaborate on the concept of centrality from the perspective of complex networks \cite{Estrada2011}. Although the concept of centrality is not novel in higher education research \cite{Glass2023}, our approach introduces the concept of \textit{bipartite networks} as a framework that provides a helpful way to quantify the relationship between graduate studies offerings and soft skills. In the section on materials and methods, we describe the data collection and analysis procedure and provide their details following the standards of reproducible research \cite{Gandrud2018}, so all interested readers can reproduce the results reported here. In the end, the paper offers a discussion section that pinpoints relevant topics for further research. 

\subsection{Graduate Programs Offering in Colombia}

In Colombia, public and private institutions with or without profit-oriented purposes can freely propose graduate programs. The Minister of Education evaluates these proposals and authorizes their commercial launch, provided the institutional proponent complies with all requirements. The legal considerations for launching new graduate programs in this country are hierarchically organized as follows. The General Education Law (Law 115 from 1994) provides the broader reference, complemented by the public service of higher education (Law 30 from 1992), resources and competencies to organize educational services (Legislative act 1 from 2001), and a technical compendium (Decree 2566 from 2003, Regulatory Single Decree 1075 from 2015) with all educational norms that apply in this nation. Apart from these normative prescriptions, each institution is free to organize a technical committee in charge of documenting the specifics of a new graduate program proposal, along with the general description, the duration, the name of the courses, the list of professors, as well as the intended learning outcomes and the graduates' profile. 

In Colombia, higher education institutions offer three types of graduate programs (i.e., specializations, masters, and doctorates) with two types of accreditation that distinguish them. The first qualification is mandatory according to local regulations and is known as ``qualified accreditation'' (``\textit{registro calificado}''). The second is optional and is known as ``high-quality accreditation'' (``\textit{acreditado de alta calidad}''). While the first qualification applies to all three types of graduate programs, high-quality accreditations apply for masters and doctorates exclusively, except for specializations in health-related disciplines such as medicine, dentistry, epidemiology, or neurology. A practical way to differentiate specializations from masters and doctorates is the time required to attend classes to get a diploma. While the standard duration for specializations is one year or two semesters, masters require two years or four semesters. Doctorates, in contrast, can last at least three years or six semesters. In addition, developing research skills is essential for master's and doctorate but not for specializations. Thus, students enrolled in specializations do not have to prepare a thesis to get a diploma, but students enrolled in a master's or a doctorate need to work on a research thesis that shows how the student applied the scientific knowledge and the scientific method to understand any phenomenon with a critical and data-driven perspective. 


The high-quality certification for graduate programs goes back to September 2008, with the Guidelines for High-quality accreditation for Master and Doctorate programs, issued by the National Accreditation Council (CNA) \cite{CNA2008}. The guidelines followed the standards of the National Council of Higher Education (CESU), an independent, related unit to the Minister of Education that is in charge of planning, assessing, coordinating, and recommending best practices for increasing the quality of higher education in Colombia. A critical reader might argue that, at this point, graduate studies offered in Colombia should show a significant change in how soft skills are visible for all professionals looking at continuing their studies. 

Although we are inclined to say that this visibility has undoubtedly increased since the Guidelines for High-quality accreditation publication, we are unaware of previous efforts supporting this idea. Nevertheless, the contribution of the current work in this regard should be evident, as it is the first empirical attempt to uncover the explicit appearance of soft skills in the official description of graduate studies. The quantitative analysis of this information is possible through the concepts provided in the next section. 

\subsection{Soft Skills Centrality in Graduate Studies: A Bipartite Network Approach}

As mentioned earlier, social-emotional abilities mingle with other job requirements in non-trivial ways. For example, data scientists working with large volumes of customer reviews from an e-commerce platform can affect teamwork if big data technologies are not tested for automatically detecting patterns in natural language data through information-entropy-related metrics \cite{Correa2020}. In this case, the untested application of existing technologies introduces uncertainty or risk in achieving a project's goal, which might be intimidating for several professionals. These factors, however, should be solved in one way or another by professionals in charge of developing solutions. Moreover, soft skills, in this case, become a competitive advantage for the employees who work to satisfy customers' needs. Nonetheless, finding solutions might not be easy for graduates without field experience. From this viewpoint, the following questions frequently pop up among young graduates and employers: Are graduate studies appropriate for training professionals' soft skills? If so, how can we quantify the link between graduate studies offering and soft skills?

The concept of a bipartite network, as a particular case of complex network structures \cite{Estrada2011}, provides a natural and powerful framework to quantify the relationship between two disjoint sets of entities. In our case, one set is defined by the list of soft skills, and the list of graduate studies represents the other set. The connection between these two sets relies on each link depicted as straight red lines, as shown in Figure \ref{F1}.  In each set, each entity is formally known as a node and the concept of \textit{node adjacency} refers to the existence of an edge (i.e., straight red lines) that connects them. By looking at Figure \ref{F1}, it should be evident that the number of connections varies for each node in both sets. For example, while negotiation connects with only one program (i.e., Master in business administration),  curiosity connects with two (i.e., Doctorate in High-Energy Physics and Master in Data Science). It is also evident that critical thinking and empathy are even in terms of being the soft skills with the highest number of connections with graduate studies (each one with six edges). If we count the number of red lines that connects every program with each skill, we should be able to see that in this case only two programs have the highest connectivity with soft skills (i.e., Master in Business Administration and Specialization in Human Resources) and one program is the least connected with soft skills (i.e., Specialization in Maxillofacial Dentistry). 

\begin{figure}[h!]
\begin{center}
\includegraphics[width=16cm]{F1.jpg}% This is a *.eps file
\end{center}
\caption{A bipartite network visualization for soft skills centrality in graduate studies}
\label{F1}
\end{figure}

Until this point, it should be clear that the concept of a bipartite network proves to be natural when it comes to understanding the relationship between two independent sets. With the bipartite network, one can examine the direct connections among the nodes of one set at a time (i.e., soft skills and graduate studies). To do this one can extract and visualize the unipartite projections of the original bipartite network. It is beyond the scope of this work to elaborate upon the mathematical and computational details related to these visualizations and projections. Still, interested readers are encouraged to revise the textbooks of \cite{Estrada2011} and \cite{Luke2015} to dive deep into these concepts. Figure \ref{F2} shows the one-mode or unipartite network projections for soft skills (with green nodes on the left) and graduate studies (with pink nodes on the right). 

\begin{figure}[h!]
\begin{center}
\includegraphics[width=16cm]{F2.jpg}% This is a *.eps file
\end{center}
\caption{Two resulting unipartite networks from the bipartite network visualization for soft skills and graduate studies}
\label{F2}
\end{figure}

With these unipartite networks at hand, the calculus of \textit{node centrality} is fairly straightforward. However, a note of warning is worthy of mention here. Intuitively, node centrality refers to the network's ``most important'' node. One can think that counting the number of links or ties for each network's node and then ranking them from the most connected to the least connected might suffice to express this idea. \textit{Degree centrality} does just that. Nonetheless, in network analysis, several proxies exist for a node's importance. \cite{Oldham2019}, for example, summarized and analyzed 17 different metrics of node centrality. Despite this variety, it is important to understand the concept of centrality beyond mere intuition. According to \cite{Estrada2011}, ``a node is more central or more influential than another in a network if the degree of the first is larger than that of the second'' (p. 122). Degree centrality, in other words, relates to a direct connection between any pair of nodes. Apart from this idea, one can also estimate how close each node is to all other nodes, which leads us to an alternative called \textit{closeness centrality}. \cite{Luke2015} defines closeness centrality as ``the inverse of the sum of all the distances between one node and all the other nodes in the network'' (p. 94). Another alternative is known as \textit{betweenness centrality}, which captures the extent that a node exists ``in-between'' pair of other nodes; that is, the edge between two nodes has to go through that node. A fourth alternative is known as \textit{Eigenvector centrality}, which measures nodes' transitive influence. A high eigenvector score means that a node is connected to many nodes with high scores. Although it is known that these centrality metrics share strong correlations in theoretical network models, their correlations for real-world networks tend to be lower \cite{ronqui2015,Oldham2019}. Given the statistical behavior of these metrics, it becomes relevant to analyze them in an unknown scenario like the one typically studied in higher education, in this case, the relevance of soft skills for graduate studies. 

To complete our bipartite network approach to the problem of quantifying soft skills centrality in graduate studies, it is mandatory to illustrate the principles that allow us to establish an objective connection or tie between any graduate program sampled from, say, the official website of a higher education institution, and any given soft skill. In essence, linking soft skills with graduate studies might be understood as one of the tasks of a behavioral data scientist. \cite{Saura2022} define behavioral data science as ``a new and emerging interdisciplinary field that
combines techniques from behavioral sciences, psychology, sociology, economics, and business, and uses the processes from computer science, data-centric engineering, statistical models, information science, or mathematics, in order to understand and predict human behavior using artificial intelligence.'' (p. 2). One concrete way to define an objective link between soft skills and graduate programs is through the use of information retrieval techniques based on  \textit{natural language processing}. As per \cite{Manning2008}, information retrieval refers to ``finding material (usually documents) of an unstructured nature (usually text) that satisfies an information need from within large collections (usually stored on computers)'' (p. 1).  

An objective link between any program of graduate studies and any given soft skill can be defined through a term-document matrix, where each column represents a concrete graduate program, and each row represents a specific word. In this matrix, cell entries can assume two possible values only. If the word is present in the document, the value of the cell will be one and zero otherwise. This matrix is equivalent to an adjacency matrix, which is the standard computational input for bipartite network analyses.  Figure \ref{F3} shows an example of a term-document matrix with three graduate studies (i.e., Program 1, Program 2, Program 3) and a set of words (e.g., innovators, leaders). In this matrix, the complete message from Programs 1, 2, and 3 can be read as ``\textit{our graduates are innovators},'' ``\textit{after completing our program you will be business leaders},'' and ``\textit{our graduates will be program innovators},'' respectively. The example here shows that while Program 2 is designed for training business leaders, programs 1 and 3 are intended to train innovators. Given that mathematical and computational details for treating text as data are already available in other sources, we encourage interested readers to revise the textbook of \cite{Manning2008}. At this point, we wish to highlight another note of warning. 

\begin{figure}[h!]
\begin{center}
\includegraphics[width=16cm]{F3.jpg}% This is a *.eps file
\end{center}
\caption{A term-document matrix with cell entries that define objective links or ties between soft skills and graduate studies.}
\label{F3}
\end{figure}


Even though our approach's mathematical and computational methods are reasonably proven in a disparate set of disciplines, their application in higher education research should be considered in a nascent stage that prevents us from posing empirical hypotheses regarding soft skills centrality for graduate studies. In other words, the motivation in this work is to provide descriptive evidence rather than inferential relationships. Nevertheless, we regard this circumstance as an opportunity to illustrate a promising methodological procedure that others can use to extend and contrast the results that follow after the next section.

\section{Materials and Method}

Our methodological approach relies on information publicly available from the Colombian Ministry of Education website and its National Information System for Higher Education (i.e., ``Sistema Nacional de Información de la Educación Superior'' or SNIES) as our first set of data sources.  

\subsection{Data sources}

By the end of February 2023, the National Information System for Higher Education reported 7,268 active graduate programs and only 424 complied with conditions for high-quality accreditation. We used this information to conduct a query-based search within SNIES' official website with the option of ``public consultation by programs'' (i.e., ``consultas públicas por programas'') including the following additional filters: \texttt{Institution State}: ``Active''; \texttt{Branch type}: ``Principal''; \texttt{Program State}: ``Active''; \texttt{Academic Sector}: ``All'', and \texttt{Academic level}: ``Graduate studies''. This query-based search allowed us to retrieve a list with the names of all graduate studies programs during the first semester of 2022. 

The resulting list was downloaded as a Microsoft Excel file for pre-processing data purposes, including a stratified sampling procedure and its resulting sample of randomly-selected programs. We used the names of the institutions and their programs to conduct a website-oriented search to identify and store the textual description of each selected program as captured by their online commercial offerings. After carefully exploring the official websites of sampled institutions, we identified and downloaded the official description of 282 graduate programs. Preliminary analyses of the information revealed that Colombian institutions do not follow a standard template or electronic format (e.g., docx, txt, html, or pdf) when documenting their programs and making them visible through their official websites. Given this variability, we also relied on historical reports from SNIES to ensure that our sampled programs had at least one promotion of graduates between 2018 and 2021, and we found that during this period, the total of graduates from private institutions (314,606) outnumbered that from public institutions (123,984). For all practical purposes, however, this work focuses on 230 valid programs, as summarized in Table \ref{T2}. 

\begin{table}[h!]
\centering
% \scriptsize
\caption{Random Stratified Sample of Graduate Studies during the first semester of 2022 in Colombia by Institution, Program, and Accreditation type}
\label{T2}
% Please add the following required packages to your document preamble:
% \usepackage{multirow}
\begin{tabular}{|ll|c|c|c|}
\hline
\multicolumn{1}{|c|}{\begin{tabular}[c]{@{}c@{}}Accreditation \\ Type\end{tabular}}                         & Program Type   & \multicolumn{1}{l|}{Public} & \multicolumn{1}{l|}{Private} & \multicolumn{1}{l|}{Total} \\ \hline
\multicolumn{1}{|l|}{\multirow{3}{*}{\begin{tabular}[c]{@{}l@{}}High-quality\\ Accreditation\end{tabular}}} & Specialization & 3                           & 3                            & 6                          \\ \cline{2-5} 
\multicolumn{1}{|l|}{}                                                                                      & Masters        & 3                           & 8                            & 11                         \\ \cline{2-5} 
\multicolumn{1}{|l|}{}                                                                                      & Doctorate      & 5                           & 2                            & 7                          \\ \hline
\multicolumn{1}{|l|}{\multirow{3}{*}{\begin{tabular}[c]{@{}l@{}}Qualified\\ Accreditation\end{tabular}}}    & Specialization & 26                          & 83                           & 109                        \\ \cline{2-5} 
\multicolumn{1}{|l|}{}                                                                                      & Masters        & 32                          & 45                           & 77                         \\ \cline{2-5} 
\multicolumn{1}{|l|}{}                                                                                      & Doctorate      & 8                           & 12                           & 20                         \\ \hline
\multicolumn{2}{|c|}{Total}                                                                                                  & 77                          & 153                          & 230                        \\ \hline
\end{tabular}
\end{table}
% \newpage



\subsection{Data Analysis}
We conducted our analyses in the R system \cite{RCore2022}. The official records gathered from the Colombian Ministry of Education and its National Information System for Higher Education were processed with standard data science techniques \cite{Wickham2019}. Text analyses were treated with quantitative text-mining procedures \cite{Feinerer2008,Benoit2018}. 

Text analyses proceeded as follows. We created a local working folder directory that stores 230 files with the textual information from each graduate program. During the process of information gathering, we noticed that all programs were available as online websites (.html), and some of them did not have a downloadable version or had as portable document formats (.pdf) or Microsoft Word documents (.docx). We stored these documents in our local folder and used them to create a text corpus in Spanish. This corpus served as the input for a standard document-term matrix where documents were arranged as rows, words were arranged as columns, and the presence of a particular word in a specific document was automatically registered as one (1) and with zero (0) otherwise. 

Preliminary analyses on this document-term matrix allowed us to develop a double-step procedure to quantify the relationship between soft skills defined by \cite{Scheerens2020} and \cite{Zins2004} and the information described in the academic program where these skills were explicitly mentioned. As a first step, we applied a \textit{keyword-in-context} search guided by a list of 50 terms and bigrams that matched the conceptual classification of socio-emotional skills proposed by \cite{Scheerens2020} and \cite{Zins2004}. The \textit{keyword-in-context} search led us to build one detailed data frame for each keyword, where each keyword was stored along with the set of terms that explicitly appeared before and after the keyword in the corpus. We iterated this procedure and identified 43 keywords explicitly mentioned in at least one sampled program. We merged all resulting data frames into one single data set used as the input for our second step. We removed the words before and after the explicit keyword in our corpus and reduced the data frame to a filtered edge list. With the help of the \texttt{quanteda} R package \cite{Benoit2018}, we developed an ultimate term-document matrix with Spanish as the default input language. In this matrix, all numeric characters, punctuation characters, Spanish stopwords (i.e., most frequent words, including pronouns, prepositions, articles, and other part-of-the-speech tokens with nil semantic information), and common words that appeared in the educational jargon were removed to decrease matrix sparsity. This matrix was then coerced or transformed as a filtered edge list. According to \cite{Luke2015}, an edge list is a data format that depicts network information by simply listing every tie in the network. In our case, a tie is just the connection between the keyword and a program. For example, if in program ``\textit{A}'' the keyword ``\textit{lead}'' explicitly occurs as ``\textit{The graduate will be able to lead teams to achieve common institutional goals},'' here we have a tie between program \textit{A} and keyword \textit{lead}. This procedure allowed us to describe and quantify the explicit presence of these skills in our sampled documents by adopting a \textit{bipartite network analysis} \cite{Estrada2011}. In a bipartite network, also known as an affiliation or two-mode network, two types of nodes (i.e., soft skills and graduate programs) co-exist, and its co-existence is quantified through the number of times each node of one type is connected to nodes of the other types. Even though such analysis has been applied in social sciences \cite{Bail2016}, its application in higher education research remains ignored. In this regard, the contribution of the present article is evident as it provides a helpful methodological approach for higher education researchers. This approach is developed following the standards of reproducible research with data and code available for those interested in knowing the computational details of data analysis \cite{Gandrud2018}.  The data, codes, and analyses are available in a public Github repository (\url{https://github.com/jcorrean/SoftSkillsUniversityPrograms}). 

\section{Results}

We begin our analyses by describing the statistical behavior of the resulting 43 terms or n-grams that emerged as soft skills from our bipartite network approach (see Panel A of Figure \ref{F4}). 

\begin{figure}[h!]
\begin{center}
\includegraphics[width=13.5cm]{F4.jpg}% This is a *.eps file
\end{center}
\caption{\textbf{(A)} Unipartite Projection Network of Emergent Soft Skills from Sampled Graduate Studies.  \textbf{(B)} Spearman correlation matrix plot for the five items of leadership and four items of personality. \textbf{(C)} Bipartite Network of Top-10 soft skills (in green rectangles on the top) emerging from sampled programs (in pink rectangles on the bottom). The symbol *** indicates the correlation is significant at or below 0.001.}
\label{F4}
\end{figure}

As expected from the statistical behavior of different centrality measures, panel B of Figure \ref{F4} depicts that the centrality of these skills (i.e., each one showed as a black point in the scatterplot included in the lower triangular matrix) tends to be fairly similar regardless of the mathematical procedure used to estimate soft skills' importance in the bipartite network (i.e., Pearson correlation is greater than or equal to 0.90). In contrast, when the projection of the bipartite network is taken as an input for describing the statistical behavior of soft skills' centrality, we found two highly-correlated centrality metrics (i.e., the correlation between degree and Eigenvector on the one hand, and the correlation between Closeness centrality and Betweenness centrality). This finding implies that all centrality measures seem to be good proxies for calculating the importance of soft skills as nodes of a bipartite network but not as nodes of a projected unipartite network. We relied, however, on Eigenvector centrality as our preferred proxy for identifying soft skills' centrality. 

% \newpage



To identify a subset of soft skills with the highest importance for all graduate studies, we ranked the emerging 43 terms, or n-grams, using the four centrality metrics for the bipartite network we built from the term-document matrix.  In panel C of Figure \ref{F4}, the bipartite network shows the top 10 soft skills that emerged from ranking their Eigenvector centrality. The results above relate to all sampled programs. To dive deep into these results, we generated five specific term-document matrices (one for each program level plus two for each accreditation type). This computational treatment allowed us to estimate soft skills' centralities better under the assumption that each program responds to the educational needs matching their target students; that is, a specialization program is qualitatively different than a master's or a doctorate. We found that although soft skills' centrality varies depending on the type of accreditation (F = 1.49; df = 1; p = 0.226) and the academic program level (F = 2.558; df = 2; p = 0.0823), their variation differences proved to be statistically non-significant. Panels (A) and (B) of Figure \ref{F5} reveal the statistical distribution of soft skills centrality. This result contrasts when it comes to spotting the centrality of each soft skill. Panel (C) of Figure \ref{F5} reveals 31 transversal soft skills explicitly present in specializations, master's, and doctorates. It is evident that the value of a soft skill centrality is context-dependent in the sense of the sampled documents used as inputs for creating the matrix for the bipartite network representation subjected to analysis.  

\begin{figure}[h!]
\begin{center}
\includegraphics[width=15cm]{F5.jpg}% This is a *.eps file
\end{center}
\caption{\textbf{(A)} Distribution of Soft Skills Centrality by Academic Program.  \textbf{(B)} Distribution of Soft Skills Centrality by Type of Accreditation. \textbf{(C)} List of Transversal Soft Skills' centrality.}
\label{F5}
\end{figure}

A final interesting result relates to soft skills' centrality ranking and how some particular soft skills change their position in this ranking depending on the centrality index used. Figure \ref{F6} shows 31 soft skills shared by all academic programs sampled. Soft skills with the highest centrality are those related to creativity (for example, in the creation or generation of ideas or projects), leadership (in the sense of leading others and teamwork), and analytical orientation (for example, evaluating situations and solving problems). In contrast, soft skills with the lowest centrality relate to understanding others, acknowledgment of others, ethical thinking, critical thinking, innovation, accountability, guidance, sharing information, or social interaction.

\begin{figure}[h!]
\begin{center}
\includegraphics[width=16cm]{F6.jpg}% This is a *.eps file
\end{center}
\caption{Centrality of Soft Skills estimated by \textbf{(A)} degree centrality, \textbf{(B)} Closeness centrality, \textbf{(C)} Betweenness centrality, and \textbf{(D)} Eigenvector centrality. Centrality metrics for all nodes were rescaled to facilitate interpretations.}
\label{F6}
\end{figure}

\section{Discussion}

During the last decade, soft skills have caught the attention of different members of the higher education system \cite{Fletcher2023}. Only recently, however, scholars have attempted to advance our knowledge of soft skills and differentiate them from individual competencies, which in turn, relate to employees' personality traits \cite{Marin2022}. As these soft skills are parts of individual competencies, there seems to be an implicit consensus that ``the interaction between soft skills and training programs appears to be vital to enhancing employee job performance.'' \cite[p. 991]{Marin2022}. In this work, we have extended these ideas in several ways. 

The first contribution of this work is the concept of bipartite networks and how it offers a promising framework that tackles the problem of quantifying the importance of soft skills in the context of the intended learning outcomes written in the commercial offering of graduate studies. Despite the large body of studies on soft skills, to our knowledge, we are unaware of previous attempts to examine soft skills from this well-known perspective in the emerging field of complex systems \cite{Estrada2011}. This statement, however, should be pondered in the field of higher education research. It is important to recall that bipartite networks are a special case of social networks \cite{Luke2015}, and quite recently, some scholars have leveraged the concept of social network analysis to understand shifts in international student mobility and world university rankings in a time window of twenty years \cite{Glass2023}. 

A second contribution of this work relates to using information retrieval and natural language processing techniques as methodological procedures that pave the way for higher education researchers to use written materials as unstructured data that can be scrutinized. This contribution, again, demands ponderation. In higher education research, natural language processing techniques were certainly unknown until recently. For example, \cite{LiCausi2022} used these techniques to understand disciplinary identity in the field of linguistics, and \cite{Mantai2022} used text-mining techniques to identify the expected skills, qualifications, and attributes to do a doctorate. Although the combination of bipartite network analysis with natural language processing techniques is not novel for social and political scientists (see, for example, \cite{Bail2016}), their combined application to higher education research remained unnoticed.

A third contribution of this work should be regarded as a corollary of the contributions mentioned above. The content-oriented search and document sampling shown in Figure \ref{F3} open new horizons regarding how to use available data easily accessible through the official websites of higher education institutions. Higher education researchers, in this sense, are encouraged to explore possible ways to escalate the procedure developed in this work to reach more documents at a continental scale, which might lead to analyzing soft skills centrality from the viewpoint of a cross-national comparison. In this panorama, this work should be regarded as an initial endeavor with a clear research agenda. We took the case of higher education institutions in Colombia, given the academic trajectory of the authors in this country. As one of the most recent Latin-American members of the OECD, Colombia and its higher education system faces the challenge of increasing their productivity and economic capacities to remain competitive in a globalized market \cite{Zarate2023}. We argue that competitiveness at a national scale relies, among many other factors, on competent individuals in different areas or disciplines for which graduate programs are expected to fill professionals' needs. While the main motivation for professionals who search for a graduate program is to learn new knowledge in a concrete discipline, it is often assumed that this learning might demand some challenges in soft skills and other personal attributes \cite{Suyansah2023,Feraco2023}. Our experience as professors with teaching and research trajectories in Colombia leads us to think that higher education institutions in this country are particularly demanding for all professionals. Most people enrolled in graduate programs are in the dual role of employees and students, which demands discipline in terms of time management and schedules in a context with few options for sponsoring tuition fees through academic scholarships. If pursuing a master's or a doctorate demands personal accountability to read and assimilate new technical material, but these cognitive tasks are compromised by poor critical thinking, for example, students' attainment and satisfaction might suffer. Thus, we concur with \cite{Mantai2022} that graduate programs ``would do well in communicating how the attributes required for PhD admission will be applied and further developed during the PhD.'' (p. 2283), as empirical evidence shows that, among several factors, delays in PhD candidates relate to some skills such as communication to prevent supervisors-related and personal-related issues \cite{van2013}. 

Our results showed that graduate programs in Colombia tend to emphasize soft skills like creativity (in the sense of creating or generating ideas or projects), leadership (e.g., leading and teamwork), and analytical orientation (e.g., evaluating situations and solving problems). Although these soft skills seem reasonable regarding what a well-trained professional requires in a concrete technical discipline, we noticed that other skills, such as empathy (i.e., understanding others and acknowledgment of others), ethical thinking, and critical thinking, prove to be less central. Even though we regard this result in the context of our exploratory research, it leads us to question if too much emphasis on the most visible soft skills might jeopardize society's long-term wellness. For example, recent evidence shows a dark side of leadership \cite{Benlahcene2022} comprising unethical behaviors that result from ineffective training programs and poor ethics education. As per \cite{Hassan2023511}, unethical leadership relates to behaviors and decisions that are anti-moral, most often illegal, with an outrageous intent to instigate
unethical behaviors among employees, producing deleterious impacts on organizations. In the context of higher education, \cite{Crawford20231} poses a timely reflection: ``With the rise of software to make cheating easier, opportunities to outsource dissertations and a more turbulent sector, it will be the leaders who sustain teams, and build good educational outcomes.'' (p. 1).

Apart from previous considerations, with this work, we aimed to justify a novel yet promising data analysis procedure that poses some implications for more traditional techniques, such as surveys or questionnaires. Although these data collection techniques will continue to be preferred by several researchers, we argue that they should be regarded as complements or optional means to collect relevant data on soft skills. Thus, we foresee a rise in natural language processing techniques as data collection procedures for soft skills research. Although data collection was certainly mentioned as a related factor to soft skills research in recent reviews like the one provided by recent reviews \cite{Marin2022}, the absence of bipartite networks and natural language processing in such endeavors is remarkable. 



%Bibliography
\begin{thebibliography}{10}

\bibitem{Crossman2010599}
J.E. Crossman and M.~Clarke.
\newblock International experience and graduate employability: Stakeholder
  perceptions on the connection.
\newblock {\em Higher Education}, 59(5):599--613, 2010.

\bibitem{Chaka2022}
C.~Chaka.
\newblock {Is Education 4.0 a Sufficient Innovative, and Disruptive Educational
  Trend to Promote Sustainable Open Education for Higher Education
  Institutions? A Review of Literature Trends}.
\newblock {\em Frontiers in Education}, 7, 2022.

\bibitem{Succi20201834}
C.~Succi and M.~Canovi.
\newblock Soft skills to enhance graduate employability: comparing students and
  employers’ perceptions.
\newblock {\em Studies in Higher Education}, 45(9):1834--1847, 2020.

\bibitem{Jamison2021145}
A.~Jamison and M.~Madden.
\newblock Developing capacities for meeting the sdgs: exploring the role of a
  public land-grant institution in the civic engagement of its african alumni.
\newblock {\em Higher Education}, 81(1):145--162, 2021.

\bibitem{Diez2020509}
M.~Diez, J.~Corral, A.~Zubizarreta, and C.~Pinto.
\newblock Including the united nations sustainable development goals in
  teaching in engineering: A practical approach.
\newblock {\em Mechanisms and Machine Science}, 89:509--518, 2020.

\bibitem{Cottafava2019521}
D.~Cottafava, G.~Cavaglià, and L.~Corazza.
\newblock Education of sustainable development goals through students’ active
  engagement: A transformative learning experience.
\newblock {\em Sustainability Accounting, Management and Policy Journal},
  10(3):521--544, 2019.

\bibitem{Ilham2020121}
J.I.J. Ilham, M.H. Zaihan, S.M. Hakimi, M.H. Ibrahim, and S.~Shahrul.
\newblock Mobilising the sustainable development goals through universities:
  Case studies of sustainable campuses in malaysia.
\newblock {\em World Sustainability Series}, pages 121--133, 2020.

\bibitem{Ariely2002}
D.~Ariely and K.~Wertenbroch.
\newblock Procrastination, deadlines, and performance: Self-control by
  precommitment.
\newblock {\em Psychological Science}, 13(3):219--224, 2002.

\bibitem{Andrews2008411}
J.~Andrews and H.~Higson.
\newblock {Graduate employability, `soft skills' versus `hard' business
  knowledge: A European study}.
\newblock {\em Higher Education in Europe}, 33(4):411--422, 2008.

\bibitem{Ritter201880}
B.A. Ritter, E.E. Small, J.W. Mortimer, and J.L. Doll.
\newblock Designing management curriculum for workplace readiness: Developing
  students’ soft skills.
\newblock {\em Journal of Management Education}, 42(1):80--103, 2018.

\bibitem{Botke2018130}
J.A. Botke, P.G.W. Jansen, S.N. Khapova, and M.~Tims.
\newblock Work factors influencing the transfer stages of soft skills training:
  A literature review.
\newblock {\em Educational Research Review}, 24:130--147, 2018.

\bibitem{Scheerens2020}
J.~Scheerens, Greetje van~der Werf, and Hester de~Boer.
\newblock {\em Soft Skills in Education: Putting the evidence in perspective}.
\newblock Springer, Cham, 2020.

\bibitem{Awang-Hashim2022635}
R.~Awang-Hashim, A.~Kaur, N.~Yusof, S.K./S. Shanmugam, N.A.A. Manaf, A.M.
  Zubairi, A.Y.S. Voon, and M.A. Malek.
\newblock Reflective and integrative learning and the role of instructors and
  institutions—evidence from malaysia.
\newblock {\em Higher Education}, 83(3):635--654, 2022.

\bibitem{Coelho202278}
M.J. Coelho and H.~Martins.
\newblock The future of soft skills development: a systematic review of the
  literature of the digital training practices for soft skills.
\newblock {\em Journal of E-Learning and Knowledge Society}, 18(2):78--85,
  2022.

\bibitem{Borner201812630}
K.~Börner, O.~Scrivner, M.~Gallant, S.~Ma, X.~Liu, K.~Chewning, L.~Wu, and
  J.A. Evans.
\newblock Skill discrepancies between research, education, and jobs reveal the
  critical need to supply soft skills for the data economy.
\newblock {\em Proceedings of the National Academy of Sciences of the United
  States of America}, 115(50):12630--12637, 2018.

\bibitem{Rovida20231541}
E.G.M. Rovida, A.~Gianotti, and G.~Zafferri.
\newblock Soft skills teaching proposal for “designers”.
\newblock {\em Lecture Notes in Mechanical Engineering}, pages 1541--1551,
  2023.

\bibitem{Riskiyana20222174}
R.~Riskiyana, N.~Qomariyah, R.N. Hidayah, and M.~Claramita.
\newblock Towards improving soft skills of medical education in the 21st
  century: A literature review.
\newblock {\em International Journal of Evaluation and Research in Education},
  11(4):2174--2181, 2022.

\bibitem{Hamid2022263}
A.~Hamid and M.~Younus.
\newblock Why soft skills matter: Analyzing the relationship between soft
  skills and productivity in workplace of academic library professionals.
\newblock {\em Libri}, 72(3):263--277, 2022.

\bibitem{Medvedeva2022}
O.D. Medvedeva, A.V. Rubtsova, A.V. Vilkova, and V.V. Ischenko.
\newblock Digital monitoring of students’ soft skills development as an
  interactive method of foreign language learning.
\newblock {\em Education Sciences}, 12(8), 2022.

\bibitem{Daniels202390}
R.A. Daniels, S.D. Pemble, D.~Allen, G.~Lain, and L.A. Miller.
\newblock Linkedin blunders: A mixed method study of college students’
  profiles.
\newblock {\em Community College Journal of Research and Practice},
  47(2):90--105, 2023.

\bibitem{Healy2023106}
M.~Healy, S.~Cochrane, P.~Grant, and M.~Basson.
\newblock Linkedin as a pedagogical tool for careers and employability
  learning: a scoping review of the literature.
\newblock {\em Education and Training}, 65(1):106--125, 2023.

\bibitem{Marin2022}
S.I. Marin-Zapata, J.P. Román-Calderón, C.~Robledo-Ardila, and M.A.
  Jaramillo-Serna.
\newblock Soft skills, do we know what we are talking about?
\newblock {\em Review of Managerial Science}, 16(4):969--1000, 2022.

\bibitem{Estrada2011}
E.~Estrada.
\newblock {\em {The Structure of Complex Networks: Theory and Applications}}.
\newblock Oxford University Press, 2011.

\bibitem{Bail2016}
C.A. Bail.
\newblock Combining natural language processing and network analysis to examine
  how advocacy organizations stimulate conversation on social media.
\newblock {\em Proceedings of the National Academy of Sciences of the United
  States of America}, 113(42):11823--11828, 2016.

\bibitem{Alvarez2022}
M.J. Alvarez-Rivadulla, A.M. Jaramillo, F.~Fajardo, L.~Cely, A.~Molano, and
  F.~Montes.
\newblock College integration and social class.
\newblock {\em Higher Education}, 84(3):647--669, 2022.

\bibitem{Duque2021669}
J.F. Duque.
\newblock A comparative analysis of the chilean and colombian systems of
  quality assurance in higher education.
\newblock {\em Higher Education}, 82(3):669--683, 2021.

\bibitem{Bradford2018909}
H.~Bradford, A.~Guzmán, J.M. Restrepo, and M.-A. Trujillo.
\newblock Who controls the board in non-profit organizations? the case of
  private higher education institutions in colombia.
\newblock {\em Higher Education}, 75(5):909--924, 2018.

\bibitem{Berry2014}
C.~Berry and J.~Taylor.
\newblock Internationalisation in higher education in latin america: Policies
  and practice in colombia and mexico.
\newblock {\em Higher Education}, 67(5):585--601, 2014.

\bibitem{Melguizo2011}
T.~Melguizo, F.S. Torres, and H.~Jaime.
\newblock The association between financial aid availability and the college
  dropout rates in colombia.
\newblock {\em Higher Education}, 62(2):231--247, 2011.

\bibitem{Jaimes2022}
Y.-C. Jaimes-Acero, A.~Granados-Comba, and R.~Bolivar-Leon.
\newblock {Soft Skills Requirements for Engineering Entrepreneurship}.
\newblock {\em {Revista Facultad de Ingeniería}}, 31, 03 2022.

\bibitem{Renteria2022}
J.~A. Rentería-Vera, E.~M. Hincapi-Montoya, Y.~J. Rodríguez-Caro, C.~K.
  Vélez-Castaneda, B.~E. Osorio-Vélez, and J.~A. Durango-Marín.
\newblock {Competencia global para el desarrollo sostenible: una oportunidad
  para la educación superior}.
\newblock {\em {Entramado}}, 18:e208, 06 2022.

\bibitem{Ocde2016-nq}
OECD.
\newblock {\em Perspectivas económicas de América Latina 2017}.
\newblock OECD Publishing, 2016.

\bibitem{Zarate2023}
R.~Zarate-Torres and J.C. Correa.
\newblock How good is the myers-briggs type indicator for predicting
  leadership-related behaviors?
\newblock {\em Frontiers in Psychology}, 14:14:940961, 2023.

\bibitem{Dellavigna2010}
Stefano DellaVigna and Matthew Gentzkow.
\newblock Persuasion: empirical evidence.
\newblock {\em Annual Review of Economics}, 2(1):643--669, 2010.

\bibitem{Lorange2019}
P.~Lorange.
\newblock {\em The Business School of the Future}.
\newblock Cambridge University Press, 2019.

\bibitem{Glass2023}
C.~R. Glass and N.~I. Cruz.
\newblock Moving towards multipolarity: shifts in the core‑periphery
  structure of international student mobility and world rankings (2000–2019).
\newblock {\em Higher Education}, 85:415--435, 2023.

\bibitem{Gandrud2018}
Christopher Gandrud.
\newblock {\em Reproducible research with R and RStudio}.
\newblock Chapman and Hall/CRC, 2018.

\bibitem{CNA2008}
{Consejo Nacional de Acreditación}.
\newblock {\em {Lineamientos para la Acreditación de Alta Calidad de Programas
  de Maestría y Doctorado}}.
\newblock Ministerio de Educación, Bogotá, Colombia, 2008.

\bibitem{Correa2020}
J.~C. Correa.
\newblock {Metrics of Emergence, Self-Organization, and Complexity for EWOM
  Research}.
\newblock {\em Frontiers in Physics}, 8:35, 2020.

\bibitem{Luke2015}
D.~Luke.
\newblock {\em {A User’s Guide to Network Analysis in R}}.
\newblock Springer, Cham, 2015.

\bibitem{Oldham2019}
Stuart Oldham, Ben Fulcher, Linden Parkes, Aurina Arnatkeviciüte,
  Chao Suo, and Alex Fornito.
\newblock Consistency and differences between centrality measures across
  distinct classes of networks.
\newblock {\em PloS one}, 14(7):e0220061, 2019.

\bibitem{ronqui2015}
Jos{\'e} Ricardo~Furlan Ronqui and Gonzalo Travieso.
\newblock Analyzing complex networks through correlations in centrality
  measurements.
\newblock {\em Journal of Statistical Mechanics: Theory and Experiment},
  2015(5):P05030, 2015.

\bibitem{Saura2022}
J.R. Saura, D.~Ribeiro-Soriano, and D.~Palacios-Marqués.
\newblock Assessing behavioral data science privacy issues in government
  artificial intelligence deployment.
\newblock {\em Government Information Quarterly}, 39(4), 2022.

\bibitem{Manning2008}
C.~D. Manning, P.~Raghavan, and H.~Schütze.
\newblock {\em Introduction to Information Retrieval}.
\newblock Cambridge Univesity Press, 2008.

\bibitem{RCore2022}
{R Core Team}.
\newblock {\em R: A Language and Environment for Statistical Computing}.
\newblock R Foundation for Statistical Computing, Vienna, Austria, 2023.

\bibitem{Wickham2019}
H.~Wickham and G.~Grolemund.
\newblock {\em {R for Data Science}}.
\newblock O'Reilly, California, USA, 2017.

\bibitem{Feinerer2008}
Ingo Feinerer, Kurt Hornik, and David Meyer.
\newblock Text mining infrastructure in r.
\newblock {\em Journal of Statistical Software}, 25(5):1--54, 2008.

\bibitem{Benoit2018}
Kenneth Benoit, Kohei Watanabe, Haiyan Wang, Paul Nulty, Adam Obeng, Stefan
  Müller, and Akitaka Matsuo.
\newblock quanteda: An r package for the quantitative analysis of textual data.
\newblock {\em Journal of Open Source Software}, 3(30):774, 2018.

\bibitem{Zins2004}
J.~E. Zins, R.~P. Weissberg, M.~C. Wang, and H.~J. Walberg.
\newblock {\em Building academic success on social and emotional learning: What
  does the research say?}
\newblock Teachers College Press, New York, 2004.

\bibitem{Fletcher2023}
S.~Fletcher and K.R.V. Thornton.
\newblock The top 10 soft skills in business today compared to 2012.
\newblock {\em Business and Professional Communication Quarterly}, 2023.

\bibitem{LiCausi2022}
T.J. LiCausi and D.A. McFarland.
\newblock Abstract(s) at the core: a case study of disciplinary identity in the
  field of linguistics.
\newblock {\em Higher Education}, 84(5):955--978, 2022.

\bibitem{Mantai2022}
L.~Mantai and M.~Marrone.
\newblock Identifying skills, qualifications, and attributes expected to do a
  phd.
\newblock {\em Studies in Higher Education}, 47(11):2273--2286, 2022.

\bibitem{Suyansah2023}
Q.~Suyansah, D.~Gabda, E.~Jawing, K.~Kamlun, R.P. Tibok, H.~Wendy, and W.N.Y.
  Xe.
\newblock Students' academic performance and soft skills on graduate
  employability among students in universiti malaysia sabah.
\newblock volume 2500, 2023.

\bibitem{Feraco2023}
T.~Feraco, D.~Resnati, D.~Fregonese, A.~Spoto, and C.~Meneghetti.
\newblock An integrated model of school students’ academic achievement and
  life satisfaction. linking soft skills, extracurricular activities,
  self-regulated learning, motivation, and emotions.
\newblock {\em European Journal of Psychology of Education}, 38(1):109--130,
  2023.

\bibitem{van2013}
Rens Van~de Schoot, Mara~A Yerkes, Jolien~M Mouw, and Hans Sonneveld.
\newblock What took them so long? explaining phd delays among doctoral
  candidates.
\newblock {\em PloS one}, 8(7):e68839, 2013.

\bibitem{Benlahcene2022}
A.~Benlahcene, O.~Saoula, M.~Jaganathan, A.~Ramdani, and N.A. AlQershi.
\newblock The dark side of leadership: How ineffective training and poor ethics
  education trigger unethical behavior?
\newblock {\em Frontiers in Psychology}, 13, 2022.

\bibitem{Hassan2023511}
S.~Hassan, P.~Kaur, M.~Muchiri, C.~Ogbonnaya, and A.~Dhir.
\newblock Unethical leadership: Review, synthesis and directions for future
  research.
\newblock {\em Journal of Business Ethics}, 183(2):511--550, 2023.

\bibitem{Crawford20231}
J.~Crawford.
\newblock Editorial: The need for good leaders in higher education.
\newblock {\em Journal of University Teaching and Learning Practice},
  20(1):1--7, 2023.

\end{thebibliography}

\end{document}
