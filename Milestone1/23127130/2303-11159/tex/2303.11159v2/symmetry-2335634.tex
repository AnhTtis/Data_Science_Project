\pdfoutput=1
%  LaTeX support: latex@mdpi.com
%  For support, please attach all files needed for compiling as well as the log file, and specify your operating system, LaTeX version, and LaTeX editor.

%=================================================================
\documentclass[symmetry,article,accept,oneauthor]{Definitions/mdpi}
%\usepackage{showframe}
%--------------------
% Class Options:
%--------------------
%----------
% journal
%----------
% Choose between the following MDPI journals:
% symmetry
% For posting an early version of this manuscript as a preprint, you may use "preprints" as the journal. Changing "submit" to "accept" before posting will remove line numbers.


%----------
% submit
%----------
% The class option "submit" will be changed to "accept" by the Editorial Office when the paper is accepted. This will only make changes to the frontpage (e.g., the logo of the journal will get visible), the headings, and the copyright information. Also, line numbering will be removed. Journal info and pagination for accepted papers will also be assigned by the Editorial Office.

%------------------
% moreauthors
%------------------
% If there is only one author the class option oneauthor should be used. Otherwise use the class option moreauthors.

%---------
% pdftex
%---------
% The option pdftex is for use with pdfLaTeX. Remove "pdftex" for (1) compiling with LaTeX & dvi2pdf (if eps figures are used) or for (2) compiling with XeLaTeX.

%=================================================================
% MDPI internal commands - do not modify
\firstpage{1}
\makeatletter
\setcounter{page}{\@firstpage}
\makeatother
\pubvolume{15}
\issuenum{4}
\articlenumber{950}
\pubyear{2023}
\copyrightyear{2023}
\externaleditor{{Academic Editor: Alexey V. Lukoyanov} %MDPI: please add academic editor
}
\datereceived{25 March 2023}
\daterevised{13 April 2023} % Comment out if no revised date
\dateaccepted{17 April 2023}
\datepublished{21 April 2023 }
%\datecorrected{} % For corrected papers: "Corrected: XXX" date in the original paper.
%\dateretracted{} % For corrected papers: "Retracted: XXX" date in the original paper.
\hreflink{https:// doi.org/10.3390/sym15040950} % If needed use \linebreak
%\doinum{}
%\pdfoutput=1 % Uncommented for upload to arXiv.org
%\usepackage[none]{hyphenat}
%=================================================================
% Add packages and commands here. The following packages are loaded in our class file: fontenc, inputenc, calc, indentfirst, fancyhdr, graphicx, epstopdf, lastpage, ifthen, float, amsmath, amssymb, lineno, setspace, enumitem, mathpazo, booktabs, titlesec, etoolbox, tabto, xcolor, colortbl, soul, multirow, microtype, tikz, totcount, changepage, attrib, upgreek, array, tabularx, pbox, ragged2e, tocloft, marginnote, marginfix, enotez, amsthm, natbib, hyperref, cleveref, scrextend, url, geometry, newfloat, caption, draftwatermark, seqsplit
% cleveref: load \crefname definitions after \begin{document}


\newcommand{\pl}{\partial}
\newcommand{\plb}{\bar{\partial}}

\newcommand{\g}{\gamma}
\newcommand{\D}{\Delta}
\newcommand{\e}{\epsilon}
\newcommand\ddfrac[2]{\frac{\displaystyle #1}{\displaystyle #2}}
\DeclareMathOperator{\Tr}{\mathrm{tr}}


%=================================================================
% Please use the following mathematics environments: Theorem, Lemma, Corollary, Proposition, Characterization, Property, Problem, Example, ExamplesandDefinitions, Hypothesis, Remark, Definition, Notation, Assumption
%% For proofs, please use the proof environment (the amsthm package is loaded by the MDPI class).

%=================================================================
% Full title of the paper (Capitalized)
\Title{On Correlation Functions as Higher-Spin Invariants}

% MDPI internal command: Title for citation in the left column
\TitleCitation{\textls[+15]{On Correlation Functions as Higher-Spin Invariants}}

% Author Orchid ID: enter ID or remove command
%\newcommand{\orcidauthorA}{0000-0000-0000-000X} % Add \orcidA{} behind the author's name
%\newcommand{\orcidauthorB}{0000-0000-0000-000X} % Add \orcidB{} behind the author's name

% Authors, for the paper (add full first names)
\Author{{Adrien Scalea} 
}

%\longauthorlist{yes}

% MDPI internal command: Authors, for metadata in PDF
\AuthorNames{Adrien Scalea}

% MDPI internal command: Authors, for citation in the left column
\AuthorCitation{Scalea, A.}
% If this is a Chicago style journal: Lastname, Firstname, Firstname Lastname, and Firstname Lastname.

% Affiliations / Addresses (Add [1] after \address if there is only one affiliation.)
\address[1]{%
Physique de l’Univers, Champs et Gravitation, Université de Mons—UMONS, 20 Place du Parc,
\linebreak B-7000 Mons, Belgium; adrien.scalea@student.umons.ac.be}

% Contact information of the corresponding author
%\corres{Correspondence: Adrien.Scalea@student.umons.ac.be}

% Current address and/or shared authorship
%\firstnote{Current address: Affiliation 3.}
%\secondnote{These authors contributed equally to this work.}
% The commands \thirdnote{} till \eighthnote{} are available for further notes

%\simplesumm{} % Simple summary

%\conference{} % An extended version of a conference paper

% Abstract (Do not insert blank lines, i.e., \\)
\abstract{(Chern--Simons) vector models exhibit an infinite-dimensional symmetry, the slightly-broken higher-spin symmetry with the unbroken higher-spin symmetry being the first approximation. \textls[-15]{In this note, we compute the $n$-point correlation functions of the higher-spin currents as higher-spin invariants directly on the CFT side, which complements earlier results that have a holographic~perspective.}}

% Keywords
\keyword{higher spin; correlation function; vector model}

% The fields PACS, MSC, and JEL may be left empty or commented out if not applicable
%\PACS{J0101}
%\MSC{}
%\JEL{}


%%%%%%%%%%%%%%%%%%%%%%%%%%%%%%%%%%%%%%%%%%
\begin{document}

%%%%%%%%%%%%%%%%%%%%%%%%%%%%%%%%%%%%%%%%%%

% The order of the section titles is different for some journals. Please refer to the "Instructions for Authors” on the journal homepage.

\section{Introduction}

(Chern--Simons) vector models are a rich class of three-dimensional conformal field theories that can be of interest for a number of reasons. Firstly, these CFTs describe critical behaviour of many physical systems, e.g., the famous Ising model, which can be realised as the $O(N)$-vector model, describes the critical behaviour of the $O(N)$-magnetic at the Curie point. Secondly, a hypothetical bulk AdS/CFT dual of these CFTs \cite{Sezgin:2002rt,Klebanov:2002ja,Sezgin:2003pt,Leigh:2003gk} gives a class of higher-spin gravities, the latter can, at least formally, be defined by inverting the correlation functions \cite{Das:2003vw,Bekaert:2015tva,deMelloKoch:2018ivk,Aharony:2020omh}.~Thirdly, Chern--Simons vector models were conjectured to exhibit a number of remarkable dualities \cite{Giombi:2011kc, Maldacena:2012sf, Aharony:2012nh,Aharony:2015mjs,Karch:2016sxi,Seiberg:2016gmd}, of which, in this work, we concentrate on the three-dimensional bosonisation duality.

The simplest gauge-invariant operators in Chern--Simons vector models are higher-spin currents, that are operators of type $J_s=\bar \phi D\dots D \phi+\dots $ or $J_s=\bar \psi\gamma D\dots D \psi+\dots $, depending on whether the model's matter is bosons $\phi$ or fermions $\psi$. {The term 'higher-spin current' is jargon.} %MDPI: footnote is not allowed in this journal, we move it into maintext, please check if it is OK. The same as following highlights.
The higher-spin currents are not conserved unless we deal with a free or very large-$N$ vector model and even in this case they are conserved tensors for $s>1$ rather than currents (to obtain a current, one needs to contract it with a conformal Killing tensor). In addition, $s=0$ and $s=1$, whenever present, are included into the multiplet of higher-spin currents. We will, however, stick with this unfortunate terminology. Higher-spin currents also turn out to be single-trace operators from the holographic perspective, and are dual to massless higher-spin fields in $AdS_4$. To prove the $3d$ bosonisation duality, it is sufficient to show that all $n$-point correlation functions of the dual theories are the same, provided we relate the free parameters appropriately. Therefore, we can concentrate on the higher-spin currents and ignore all other local gauge-invariant operators.

In the very large-$N$ limit, the higher-spin currents are conserved. Via the Noether theorem, they lead to an infinite-dimensional extension of the conformal symmetry $so(3,2)$, with the spin-two current, the stress-tensor, manifesting $so(3,2)$. The resulting algebra is usually called the higher-spin algebra and it is also the symmetry algebra of the free boson's and free fermion's equations of motion, e.g., refs. \cite{Dirac:1963ta, Gunaydin:1981yq,Gunaydin:1983yj,Vasiliev:1986qx, Eastwood:2002su,Joung:2014qya}. Historically, it was identified~\cite{Dirac:1963ta} as the even subalgebra of the Weyl algebra $A_2$, which is the algebra of observables of the $2d$ harmonic oscillator.

The (unbroken) higher-spin symmetry is the usual symmetry, i.e., there is a Lie algebra acting on the physical spectrum of operators. Interestingly, the free matter fields and the multiplet of the higher-spin currents are the simplest representation of the higher-spin algebra \cite{Dirac:1963ta,Flato:1978qz,Craigie:1983fb}. The algebra admits an invariant trace $\Tr[a\star b-b\star a]=0$, and the series of invariants

\begin{align*}
\langle J\dots J\rangle = \Tr[\Psi \star \ldots  \star \Psi]
\end{align*}
computes the correlation functions, provided the wave-functions $\Psi$ are chosen wisely. Calculations of this kind were performed in \cite{Colombo:2012jx,Didenko:2012tv,Didenko:2013bj, Bonezzi:2017vha}, where $\Psi$ was taken to represent a multiplet of massless higher-spin fields in $AdS_4$. It is important that the unbroken higher-spin symmetry is a signature of the free CFT's behaviour in $d\geq3$ \cite{Maldacena:2011jn,Boulanger:2013zza,Alba:2013yda,Alba:2015upa}. {The results of} \cite{Maldacena:2011jn} require finite $N$, but the higher-spin currents are conserved at $N=\infty$ as well. Importantly, they are also conserved at $N=\infty$ {in the interacting vector models.} Uniqueness of this type of invariants can also be shown \cite{Sharapov:2020quq}.

When interactions are turned on, either directly or by departing from the very large-$N$ limit, the higher-spin currents cease to be conserved, except for the stress-tensor, $s=2$, and for the global symmetry current, $s=1$. As a simple consequence of the smallness of operators' spectrum of vector models, the non-conservation operator has a very restricted form of a composite operator, built of $J$s themselves \cite{Giombi:2011kc, Maldacena:2012sf}. As a result, the non-conservation is still very useful to impose on correlation functions \cite{Maldacena:2012sf} and this type of symmetry breaking was dubbed slightly-broken higher-spin symmetry in \cite{Maldacena:2012sf}. Mathematically, the slightly-broken higher-spin symmetry is not a symmetry. It is not realised as an action of some Lie algebra on a multiplet of operators. However, it can be understood as a strong homotopy algebra, that deforms the action of the higher-spin algebra \cite{Sharapov:2018kjz,Gerasimenko:2021sxj,Skvortsov:2022abz}.

While Chern--Simons vector models have simple actions, the (slightly-broken) higher-spin symmetry is by far the most efficient way to compute correlation functions of higher-spin currents in vector models, e.g., refs. \cite{Li:2019twz,Kalloor:2019xjb,Turiaci:2018nua,Jain:2020puw,Jain:2021vrv,Jain:2021gwa,Jain:2021whr,Silva:2021ece}. For free or very large-$N$ vector models, this calculation was  performed in \cite{Giombi:2010vg,Colombo:2012jx,Didenko:2012tv,Didenko:2013bj, Bonezzi:2017vha} {(note that the calculation of} \cite{Giombi:2010vg} applied a {regularisation that effectively replaced the ill-defined vertices} \cite{Boulanger:2015ova,Skvortsov:2015lja} with the higher-spin invariant) with an additional proviso of identifying the correlation functions with the invariants of the higher-spin algebra. The correlators can also be computed via the textbook Wick contractions \cite{Gelfond:2013xt} of free fields.

The main point of the present note is to exclude the holographic aspect present in~\cite{Giombi:2010vg,Colombo:2012jx,Didenko:2012tv,Didenko:2013bj, Bonezzi:2017vha}. In other words, in this note we adopt a purely CFT view on the higher-spin symmetry. The wave-functions of \cite{Giombi:2010vg,Colombo:2012jx,Didenko:2012tv,Didenko:2013bj, Bonezzi:2017vha} represent a multiplet of higher-spin fields that are duals of higher-spin currents. Due to the fact that the bulk dual of Chern--Simons vector models is not known, and is unlikely to exist as a reasonably local field theory, one cannot just extract the Chern--Simons vector models' correlators from holography. Fortunately, the key features of the higher-spin symmetry (such as mixing spins and derivatives) that generically invalidate the field theory approach are harmless on the CFT side. {It should be mentioned that the dual theory has a closed local subsector} \cite{Metsaev:2018xip,Skvortsov:2018uru,Sharapov:2022awp,Sharapov:2022wpz,Sharapov:2022nps}, which is an $AdS_4$-deformation of the chiral higher-spin gravity in flat space \cite{Metsaev:1991mt,Metsaev:1991nb,Ponomarev:2016lrm,Ponomarev:2017nrr,Skvortsov:2018jea,Skvortsov:2020wtf,Skvortsov:2022syz,Sharapov:2022faa}, see, e.g., ref. \cite{Bekaert:2022poo} for more on higher-spin gravities. It would be interesting to compute holographic correlation functions in this model, {but we believe it can be achieved more efficiently on the CFT side.} We hope that this is a useful first step in the programme of computing correlation functions of higher-spin currents in Chern--Simons vector models, as invariants of the slightly-broken higher-spin symmetry.

The note is organised as follows. In Section \ref{sec:correlators in 3d} we recall the results of \cite{Giombi:2011rz} on the general structure of $3d$ conformal correlators. In Section \ref{sec:HS invariants} we define the wave-functions that represent a generating function of higher-spin currents and compute the higher-spin invariants. Two appendices collect some useful identities and definitions.


%%%%%%%%%%%%%%%%%%%%%%%%%%%%%%%%%%%%%%%%%%
\section{Structure of Correlation Functions in Three Dimensions}
\label{sec:correlators in 3d}
%%%%%%%%%%%%%%%%%%%%%%%%%%%%%%%%%%%%%%%%%%%%%%%%%%%%%%%%%%%%%
In three dimensions, one has the isomorphism $so(2,1)\simeq sl(2,\mathbb R)$, implying that a traceless rank-$s$ Lorentz tensor can be represented by a rank-$2s$ spin-tensor. In Appendix \ref{app:vec-to-spin dictionary}, we detail notations and conventions, but in brief, we note that a 3-vector $\mathbf{x}^m$ ($m=0,1,2$) can be mapped into a symmetric bi-spinor $X^{\alpha\beta}$ ($\alpha,\beta=1,2$).

Higher-spin currents are symmetric and traceless tensors $J_{a_1 \dots a_s}(\mathbf{x})$.~In addition, they are conserved $\partial^b J_{b a_2\dots a_s}= 0$, so that they are primary fields. Thanks to the isomorphism, they are mapped to $J^{\alpha_1\dots \alpha_{2s}}(X)$ and one can pack them into a generating function \linebreak $j(X,\eta) = J^{\alpha_1\dots \alpha_{2s}}(X)\,\eta_{\alpha_1}\dots \eta_{\alpha_{2s}}$, where $\eta_{\alpha}$ is an auxiliary polarisation spinor. The conservation implies
\begin{equation}
\frac{\partial}{\partial X^{\alpha\beta}} \dfrac{\partial^{2}}{\partial\eta_{\alpha}\partial\eta_{\beta}}\, j(X,\eta) = 0\,. \label{eq:conservation eq}
\end{equation}
{It turns out, that} conformally invariant correlation functions of tensor operators can depend on very few atomic conformally invariant structures \cite{Giombi:2011rz}. There are two parity-even atomic~structures
\begin{equation}
\begin{aligned}
P_{ij} &= \eta^\alpha_i \eta^\beta_j  \big(X^{-1}_{ij} \big)_{\alpha\beta}\,, &\qquad
Q^i_{j,k} &= \eta^\alpha_i \eta^\beta_i \big(X^{-1}_{ji} - X^{-1}_{ki} \big)_{\alpha\beta}\,, \label{eq:conformal structures P,Q}
\end{aligned}
\end{equation}
where $X^{-1}_{ij} \equiv \big(X_{ij}\big)^{-1} \equiv \big(X_{i} - X_{j}\big)^{-1}$. We have $P_{ij} = -P_{ji}$ and $Q^i_{j,k} = -Q^i_{k,j}$. Defining the inversion map as
\begin{equation}
R\eta^i_{\alpha} = -\dfrac{X_{i\,\alpha}{}^\beta \eta^i_\beta}{|X_i|}= (X_i^{-1})_\alpha{}^\beta \eta^i_\beta\,,
\end{equation}
we observe $RP_{ij} = P_{ij}$ and $RQ^i_{j,k} = Q^i_{j,k}$, which proves the structures to be parity-even. There is also one parity-odd invariant structure
\begin{equation}
S^i_{jk} = \dfrac{\eta^\alpha_k \, (X_{ik} X_{ij})_{\alpha\beta}\, \eta^\beta_j }{x_{ij} x_{ik} x_{jk}}\,, \label{eq:conformal structure S}
\end{equation}
where $x_{ij} \equiv |\mathbf{x}_{ij}| = \sqrt{-|X_{ij}|}$. Conformally invariant correlation functions depend on the cross-ratios, and are polynomials in $P$s, $Q$s, and $S$s. The exponents of $P$s, $Q$s, and $S$s are constrained by the spin of the operators in an obvious way. For example, the simplest two- and three-point correlators
\begin{align}
\langle j_{s}(X_1,\eta_1) j_{s}(X_2,\eta_2) \rangle &
\sim\frac{1}{x_{12}^2} (P_{12})^{2s}\ ,
\\
\langle j_{s_1}(X_1,\eta_1) j_0(X_2) j_0(X_3)\rangle &
\sim\frac{1}{{x_{12}x_{23}x_{31}}} (Q_1)^{s_1}\ ,
\end{align}
where $j_s$ is a conserved higher-spin current and $j_0$ is a scalar operator of dimension $1$.




%%%%%%%%%%%%%%%%%%%%%%%%%%%%%%%%%%%%%%%%%%%%%%%%%%%%%%%%%%%%%
\section{Correlation Functions as Higher-Spin Invariants}
\label{sec:HS invariants}
%%%%%%%%%%%%%%%%%%%%%%%%%%%%%%%%%%%%%%%%%%%%%%%%%%%%%%%%%%%%%
In this section, we first recall the definition of the relevant higher-spin algebra, together with the star-product, as a convenient tool to work with it. We also introduce a conformally friendly basis for the generators. Next, we fix the form of the wave-functions $\Psi$ and compute the correlation functions.

%%%%%%%%%%%%%%%%%%%%%%%%%%%%%%%%%%%%%%%%%%%%%%%%%%%%%%%%%%%%%
\subsection{Higher-Spin Algebra}
%%%%%%%%%%%%%%%%%%%%%%%%%%%%%%%%%%%%%%%%%%%%%%%%%%%%%%%%%%%%%
The isomorphism $so(3,2)\simeq sp(4,\mathbb{R})$, allows us to use the $sp(4)$ generators $T_{AB}= T_{BA}$, $A,B=1,\dots,4$ , such that
\begin{equation}
[T_{AB}, T_{CD}] = T_{AC} \,\epsilon_{BD} + \text{3 terms}\,,
\end{equation}
where $\epsilon_{AB} = -\epsilon_{BA}$ and $\epsilon_{AB} \epsilon^{AC} = \delta_B{}^C$. With the four operators $Y_A$ satisfying the canonical commutation relations $[Y_A,Y_B] = 2i\epsilon_{AB}$, one can realise the above commutation relations as $T_{AB} = \tfrac{-i}{4} \{Y_A, Y_B\}$, which is the standard oscillator realisation. The associative algebra of functions $f(Y)$ in $Y^A$ is the Weyl algebra $A_2$ (the subscript $2$ is the number of canonical pairs). Its even subalgebra $A_2^e$ of functions $f(Y)=f(-Y)$ is the higher-spin algebra we need. We can also split $Y_A = \big(y_\alpha,\, \bar y_{\alpha}\big)$ and $\epsilon_{AB} = \mathrm{diag}(\epsilon_{\alpha\beta},\epsilon_{\alpha \beta})$, with $\epsilon_{12}=1$.

More abstractly, a higher-spin algebra can be defined, for any irreducible representation of the conformal algebra, as the quotient of the universal enveloping algebra by a two-sided ideal that is the annihilator of this representation or, in the field theory language, as the symmetry algebra of the corresponding conformally invariant field equation \cite{Eastwood:2002su}. An important fact, is that for the free fermion and free boson representations, this ideal gets resolved by the oscillator realisation. Another important fact for the $3d$ bosonisation duality to take place, is that the higher-spin algebras of the free fermion and the free boson are isomorphic to the same $A_2^e$, which is explicit already in \cite{Dirac:1963ta}.

Higher-spin algebras turn out to be infinite-dimensional associative algebras that contain the conformal algebra as a Lie subalgebra (any associative algebra leads to a Lie algebra where the Lie bracket is defined as the commutator). Therefore, any higher-spin algebra can be viewed as an infinite-dimensional extension of the conformal symmetry.

\subsubsection{Star-Product}
Instead of working with an algebra of operators, it is convenient to use the algebra of functions in commuting variables $Y_A$ (symbols) with the (associative) Moyal--Weyl star-product. It admits an integral form and a more standard differential form (simple $(2\pi)^{-4}$ prefactor is omitted or included into the definition of $\int$ below)
\begin{equation}
f(Y)\star g(Y) = \int \mathrm{d}^4 U\, \mathrm{d}^4 V\ f(Y+U) g(Y+V)\, e^{iV^A U_A} = f(Y)\exp \left[i \overset{\leftarrow}{\partial}_A \epsilon^{AB} \overset{\rightarrow}{\partial}_B \right] g(Y)\,. \label{eq:star-product}
\end{equation}
{We will also} have to go outside of the space of polynomial functions, e.g., admitting delta function $\delta^2(y) = \int \mathrm{d}^2 s\, e^{is^\alpha y_\alpha}$. With the star-product, we have $[Y_A,Y_B]_\star :=
Y_A \star Y_B - Y_B \star Y_A = 2i\epsilon_{AB}$ and the unit element is $1$, i.e., $f\star 1=1\star f = f$. We also find useful relations
\begin{equation}
Y_A \star f(Y) = Y_A f + i \dfrac{\partial f}{\partial Y^A}\,, \qquad f(Y) \star Y_A = Y_A f - i \dfrac{\partial f}{\partial Y^A}\,. \label{eq: Y star f}
\end{equation}
{The even} subalgebra of the Weyl algebra admits an invariant trace operation, which in terms of symbols $f(Y)$ reads
\begin{equation}
\mathrm{tr}\big(f(Y) \big) = f(0)  \label{eq:trace}
\end{equation}
{such that }$\mathrm{tr}(f\star g) = \mathrm{tr}(g\star f)$.

\subsubsection{Conformally Adapted Basis}
In view of the CFT nature of the problem, it is convenient to split $T_{AB}$ in such a way as to make the standard basis of conformal generators explicit, e.g., refs. \cite{Gunaydin:1981yq,Vasiliev:2012vf}. We define $y^-_\alpha = \frac{1}{2}(\bar y_\alpha - iy_\alpha)$ and $y^{+\alpha} = \frac{1}{2}(y^\alpha-i\bar y^\alpha)$ that obey $[y^-_{\alpha}, y^{+\beta}]_\star = \delta_\alpha{}^\beta$, which implies that $y^\pm_\alpha$ are the standard creation/annihilation operators. Indeed, the conformal generators read~\cite{Vasiliev:2012vf}
\begin{equation}
\begin{aligned}
P_{\alpha\beta} &= iy^-_\alpha y^-_\beta\,, \qquad\ \  K^{\alpha\beta} = -iy^{+\alpha} y^{+\beta}\,,\\
D &= \tfrac{1}{2} y^{+\alpha} y^-_\alpha\,, \qquad L^{\alpha}{}_{\beta} = y^{+\alpha} y^-_\beta - \tfrac{1}{2} \delta^\alpha_\beta \, y^{+\gamma} y^-_\gamma\,. \label{eq:conformal alg with oscillators}
\end{aligned}
\end{equation}
{With the reality} %MDPI: please check if the noindent format is correct? the same as following highlights
conditions $(y^-_\alpha)^\dagger = y^{+\alpha}$, one has $D^\dagger = D$, $(L^\alpha{}_\beta)^\dagger = L_\beta{}^\alpha$ and $P^\dagger_{\alpha\beta} = K^{\alpha\beta}$. The mass-shell condition is manifest since $P^2 = 0$. The basic star-product relations \eqref{eq: Y star f} in terms of $y^\pm$ read
\begin{equation}
\begin{aligned}
y^\pm_\alpha \star f(y^+, y^-) &= y^\pm_\alpha + \tfrac{1}{2} \partial^\mp_\alpha f\,, &\qquad
f(y^+, y^-) \star y^\pm_\alpha  &= y^\pm_\alpha - \tfrac{1}{2} \partial^\mp_\alpha f\,.
\end{aligned}
\end{equation}
{As a result,} $[y^\pm_\alpha, f]_\star = \partial^\mp_\alpha f$ and
\begin{equation}
[y^a_\alpha y^b_\beta, f]_\star = y^a_\alpha \partial^{\bar b}_\beta f + y^b_\beta \partial^{\bar a}_\alpha f\,, \qquad a,b \in \{+,-\}\,,\quad  \bar a \equiv -a\,.
\end{equation}
{The action} of the conformal generators \eqref{eq:conformal alg with oscillators} reads
\begin{equation}
\begin{aligned}\relax
[P_{\alpha\beta}, f]_\star &= +i \big( y^-_\alpha \partial^+_\beta + y^-_\beta \partial^+_\alpha \big) f\,, \\
[K_{\alpha\beta}, f]_\star &= -i \big( y^+_\alpha \partial^-_\beta + y^+_\beta \partial^-_\alpha \big) f\,, \\
[D,f]_\star &= \tfrac{1}{2} \big(y^{+\alpha} \partial^+_\alpha - y^{-\alpha} \partial^-_\alpha \big)f\,, \\
[L_{\alpha\beta}, f]_\star &= \Big( y^+_\alpha \partial^+_\beta + y^-_\beta \partial^-_\alpha +  y^+_\beta \partial^+_\alpha + y^-_\alpha \partial^-_\beta \Big)f\,.
\end{aligned} \label{eq:action of the conformal alg}
\end{equation}
{We observe} that $D$ counts the number of $y^+$ minus the number of $y^-$.


%%%%%%%%%%%%%%%%%%%%%%%%%%%%%%%%%%%%%%%%%%%%%%%%%%%%%%%%%%%%%
\subsection{Wave-Functions}
%%%%%%%%%%%%%%%%%%%%%%%%%%%%%%%%%%%%%%%%%%%%%%%%%%%%%%%%%%%%%
Given that the higher-spin algebra is an infinite-dimensional extension of the conformal algebra, it should not be surprising that correlation functions of the higher-spin currents can be computed as simple invariants of this algebra \cite{Colombo:2012jx,Didenko:2012tv,Didenko:2013bj, Bonezzi:2017vha}. The invariants know nothing about correlation functions per se and must be fed with appropriate wave-functions $\Psi$, that contain the information about the operators' positions and spins. In \cite{Colombo:2012jx,Didenko:2012tv,Didenko:2013bj, Bonezzi:2017vha}, $\Psi$ was defined to reside in $AdS_4$ and it represents a collection of massless fields in $AdS_4$. In addition, $\Psi$ of \cite{Colombo:2012jx,Didenko:2012tv,Didenko:2013bj, Bonezzi:2017vha} does not transform in the adjoint representation. Below, we fix the form of $\Psi$ on the CFT side, which is the main difference compared to \cite{Colombo:2012jx,Didenko:2012tv,Didenko:2013bj, Bonezzi:2017vha}. We will find that the wave-function $\Psi$ is simpler than its $AdS_4$ cousin.

\subsubsection*{Wave-Functions' Properties}
The main building block of correlation functions of higher-spin currents is
\begin{equation}\label{invcor}
O_n = \mathrm{tr}\big(\Psi_1 \star \dots \star \Psi_n \big)\,.
\end{equation}
{It is invariant} under higher-spin transformations (hence, conformally invariant as well) provided that $\Psi_i \equiv \Psi(X_i,\eta_i|Y)$ transforms in the adjoint representation of the higher-spin algebra, i.e., $\delta_\xi \Psi = [\Psi,\xi]_\star$. To relate this observable to higher-spin currents, we also need to make sure that $\Psi$ obeys the conservation condition \eqref{eq:conservation eq}.

Concerning \eqref{invcor}, it is worth noting that it has only cyclic symmetry, which is exactly the symmetry of the correlation functions in vector models with (leftover) global symmetries that are not gauged via the Chern--Simons term. Indeed, if there is a global symmetry, say $U(M)$, the higher-spin currents have a pair of indices $J_i{}^k$. Correlation functions of such currents have only cyclic symmetry. In some sense, \eqref{invcor} is the master higher-spin invariant and all the others can be obtained by projecting it (say on the bosonic currents, since by default it contains the super-currents as well) and symmetrising over the external legs.


In \cite{Colombo:2012jx,Didenko:2012tv,Didenko:2013bj, Bonezzi:2017vha}, the authors used the AdS/CFT formalism, where all physical information is encoded in the master field $B(X,z,\eta|Y)$ ($z$ is the radial coordinate on $AdS_4$). This field transforms in a twisted-adjoint representation. However, one can build $\Psi = B\star \delta^2(y)$ that transforms in the adjoint one, which still resides in the bulk. To give an idea of the functional class used for the holographic calculations in \cite{Colombo:2012jx,Didenko:2012tv,Didenko:2013bj, Bonezzi:2017vha}, the main building block of $B$ was found to be
\begin{equation}
\Phi(F,\xi,\theta) = \mathbold{K}\,\exp i (-yF\bar y + \xi y)\,,
\end{equation}
where $\mathbold{K}$ is the scalar boundary-to-bulk propagator. Matrix $F$, spinor $\xi$, and $\mathbold{K}$ depend on the bulk and boundary coordinates. Explicit formulas can be found in \cite{Colombo:2012jx,Didenko:2012tv,Didenko:2013bj, Bonezzi:2017vha}. Let us note that $B$ does not have any obvious boundary limit.

In order to find the wave-function $\Psi(X,\eta| y^\pm)$ directly on the CFT side, we will use the following physical conditions. (1) $\Psi$ must be a generating function of quasi-primary operators at $X=0$; (2) $\Psi$ must satisfy the equation of motion---covariant constancy condition, which reconstructs $X$-dependence; (3) it has to be a generating function of conserved higher-spin currents. We should also take into account that there is no unique solution $\Psi$ that satisfies (1--3), since we can always rescale any spin-$s$ component by some numerical factor.

The first condition can be translated into a simple equation
\begin{equation}
[K_{\alpha\beta}, \Psi(0,\eta|y^\pm)]_\star =  (y^+_{\alpha} \pl^-_{\beta}+y^+_{\beta} \pl^-_{\alpha})\Psi=0\,. \label{eq:constraint primary}
\end{equation}
{Its obvious solution} is $\Psi(0,\eta|y^\pm) = \Psi(0,\eta|y^+)$. There is a less obvious solution
\begin{align}
\widetilde\Psi(0|y^\pm) = \epsilon^{\alpha\beta}y^-_\alpha \pl^+_\beta \delta (y^+)\label{foot:second sol}\,.\end{align}
{Note that in the case} of one variable $x\delta'(x)=-\delta(x)$, but $(y^+_\alpha \pl^+_\beta+y^+_\beta \pl^+_\alpha )\delta (y^+)=0$ in the symplectic case. The latter identity needs to be used to check that $\widetilde\Psi$ is a solution. This solution corresponds to the $\Delta=2$ operator $\bar\psi\psi$ that is present in the free fermion theory. It stands out of the main higher-spin current multiplet and we will not discuss it further, except for the two-point function.

Next, we need to recover the $X$-dependence in such a way that $\Psi$ obeys
\begin{align}
\partial^x_{\alpha\beta} \Psi + \tfrac{i}{2}\, [P_{\alpha\beta}, \Psi]_\star &= 0\,. \label{eq: d + P =0}
\end{align}
{Indeed, the flat }space is realised with the help of the gauge function $g= \exp \frac{i}{2} X^{\alpha\beta} P_{\alpha\beta}$. The corresponding connection $g^{-1} \star \mathrm{d} g=\frac{i}{2} \mathrm{d} X^{\alpha\beta} P_{\alpha\beta}$, is a flat connection of the conformal algebra.
Since $\Psi$ must be in the adjoint representation, the $X$-dependence is determined by
\begin{equation*}
\Psi(X,\eta|y^\pm) = g^{-1}(X)\star \Psi(0,\eta|y^+) \star g(X)\,.
\end{equation*}
{This is a direct a}nalog of $O(X)=\exp{[X\cdot P]} O(0) \exp{[-X\cdot P]}$ in the standard CFT language, the only difference being that $\Psi$ is a generating function of infinitely many quasi-primary operators. With the help of the oscillator realisation \eqref{eq:conformal alg with oscillators}, we find
\begin{align}
g^{-1} \star y^+_\gamma \star g &= y^+_\gamma + X_{\gamma}{}^{\alpha} y^-_\alpha \,, &
g^{-1} \star y^-_\gamma \star g &= y^-_\gamma\,.
\end{align}
{Therefore,} the wave-function is constrained now to be
\begin{align}
\Psi(X,\eta|y^\pm)=\Psi(\eta|y^+_\gamma + X_{\gamma}{}^{\alpha} y^-_\alpha)\,. \label{eq:constraint evolution}
\end{align}

{Lastly,} we need to impose the conservation condition, which determines the $\eta$-dependence
\begin{align}
\frac{\partial}{\partial X^{\alpha\beta}} \dfrac{\partial^{2}}{\partial\eta_{\alpha}\partial\eta_{\beta}}\Psi(X,\eta|y^\pm)=0\,. \label{eq:constraint spin}
\end{align}
{It is helpful }to represent the wave-function as a Fourier integral
\begin{align}
\Psi(X,\eta|y^\pm)=\int \mathrm{d}^2s\ f(s,\eta) \exp{is^\gamma[y^+_\gamma + X_{\gamma}{}^{\alpha} y^-_\alpha]}\,.
\end{align}
{Imposing} the conservation condition we find
\begin{align}
\int \mathrm{d}^2s\  (s_\alpha y^-_\beta)\, \pl_\eta^\alpha \pl_\eta^\beta f(s,\eta)\, \exp{is^\gamma[y^+_\gamma + X_{\gamma}{}^{\alpha} y^-_\alpha]}=0\,.
\end{align}
{Since} $f$ cannot depend on $y^-_\alpha$, otherwise it spoils the solution of the other two conditions, we have to take $f(s,\eta)=f(s^\gamma\eta_\gamma)$.
%EE: Please check meaning retained
This function of one variable is the expected ambiguity in normalisation of the higher-spin currents. We, of course, fix it to be $f=\exp{ i s^\gamma \eta_\gamma}$. Finally, the wave-function is found to be
{\small\begin{align}
\Psi(X,\eta|y^\pm) &= \int \mathrm{d}^2s\, \exp{is^\gamma[y^+_\gamma + X_{\gamma}{}^{\alpha} y^-_\alpha+\eta_\gamma]}= \delta^2(y^+_\gamma + X_{\gamma}{}^{\alpha} y^-_\alpha+\eta_\gamma)\equiv \delta^2\big(\Gamma(X) \big)\,. \label{eq:Psi solution}
\end{align}}

{For completeness}, let us note that the Lorentz generators act canonically
\begin{align}
[L_{\alpha\beta}, \Psi]_\star \big|_{X=0} &= \Big( y^+_\alpha \partial^+_\beta +  y^+_\beta \partial^+_\alpha \Big)\delta^2(y^++\eta)\,.
\end{align}
{Further, }the dilation operator relates the conformal weight to the spin
\begin{align}
[D,\Psi]_\star \big|_{X=0} &=\tfrac12 y^{+\alpha} \partial^+_\alpha \delta^2(y^++\eta)\,.
\end{align}
{The full} higher-spin symmetry can also be seen to act. Its parameters are contained in a covariantly constant generating function of Killing tensors $\xi$, that obeys
\begin{align}
\partial^x_{\alpha\beta} \xi + \tfrac{i}{2}\, [P_{\alpha\beta}, \xi]_\star &= 0\,.
\end{align}
{The conformal} and genuine higher-spin symmetries act as $\delta_\xi \Psi=[\Psi,\xi]_\star$.


%%%%%%%%%%%%%%%%%%%%%%%%%%%%%%%%%%%%%%%%%%%%%%%%%%%%%%%%%%%%%
\subsection{Correlation Functions}
%%%%%%%%%%%%%%%%%%%%%%%%%%%%%%%%%%%%%%%%%%%%%%%%%%%%%%%%%%%%%
As stated before, the main building block of correlation functions of higher-spin currents is
\begin{equation}
O_n = \mathrm{tr}\big(\Psi_1 \star \dots \star \Psi_n \big)\,.
\end{equation}
{In order to explicitly} compute it, we begin with $O_2$.

\subsubsection{Two-Point Correlators}
It is useful to first check the two-point functions. Here, we return to the variables $Y^A=(y^\alpha, \bar{y}^\alpha)$ to compute the star-product in what follows. The solution \eqref{eq:Psi solution} is rewritten~as
\begin{equation}
\Psi(X,\eta|Y) = \kappa\, \delta^2\big(\Gamma(X) \big)\,,\qquad\text{with }\Gamma^{\alpha} = A^{\alpha\beta}(X) \bar y_\beta + B^{\alpha\beta}(X) y_\beta + C^{\alpha}\,,
\end{equation}
with $C^\alpha = c\,\eta^\alpha$,
\begin{equation}
\begin{aligned}
A^{\alpha\beta}(X) &= a X^{\alpha\beta} - i a\, \e^{\alpha\beta}\,, \\
B^{\alpha\beta}(X) &= -ia X^{\alpha\beta} + a\, \e^{\alpha\beta} \equiv -iA^\mathrm{T}(X) \equiv iA(-X) \,.
\end{aligned} \label{eq:A, B final form}
\end{equation}
\textls[+20]{{Indeed the most }general solution is defined up to a multiplicative constant, $\kappa$, and we can always rescale $y^\pm$ and $\eta$ without affecting Equations \eqref{eq:constraint primary}, \eqref{eq:constraint evolution}, and \eqref{eq:constraint spin}. We keep $\kappa$, $a$, and $c$ arbitrary for the time being, in order to derive more general formulas for the star-products and have an additional control over the calculations. Denoting $\Psi_i \equiv \Psi(X_i,\eta_i|Y)$, one can show that}
\begin{equation}
\Psi_1 \star \Psi_2 = \dfrac{-\kappa^2}{|M_{12} |} \exp i \left[(\bar y A^\mathrm{T}_2  + y B^\mathrm{T}_2 - C_2)\, \big(M_{12}\big)^{-1}\, (A_1 \bar y + B_1 y + C_1) \right], \label{eq:psi_i psi_j}
\end{equation}
with $M_{12} := A_1 A^\mathrm{T}_2 + B_1 B^\mathrm{T}_2$.
With \eqref{eq:A, B final form}, we have $M_{12} = 2ia^2\, X_{12}$, so that tracing gives
\begin{equation}
\mathrm{tr}\, \Psi_1 \star \Psi_2 = \dfrac{-\kappa^2}{4a^4\, \mathbf{x}^2_{12}} \exp i \left[\frac{-i}{2a^2} C_{1\alpha}\, (X^{-1}_{12} )^{\alpha\beta}\,C_{2\beta} \right].
\end{equation}
{In order to} match with the two-point function $O_2 = \frac{1}{\mathbf{x}^2_{12}} \exp iP_{12}$ \cite{Didenko:2012tv}, we need
\begin{align}
\dfrac{-\kappa^2}{4a^4} &= 1, \label{eq:cond on M,kappa}\\
\frac{-i}{2a^2} C_{1\alpha}\, (X^{-1}_{12} )^{\alpha\beta}\,C_{2\beta} &= \eta_{1\alpha} (X^{-1}_{12})^{\alpha\beta} \eta_{2\beta} \,. \label{eq:condition on C,M}
\end{align}
{With the choice} $C_i = c\,\eta_i$, the second equation leads to $\dfrac{i\,c^2}{2a^2}= -1$. In particular, our wave-function $\Psi$ leads to the correct two-point function. Let us note that
\begin{equation}
\Psi_{12} := \Psi_1 \star \Psi_2 = \dfrac{1}{\mathbf{x}^2_{12}} \,\exp i \left[ \frac{1}{2} \Sigma^{AB}_{12}\, Y_A Y_B + \xi^A_{12}\, Y_A + P_{12} \right]\, \label{eq:2pt gaussian}
\end{equation}
with
\begin{align}
\Sigma^{AB}_{12} &= \frac{i}{a^2}
\begin{pmatrix}
B^{\mathrm{T}}_2 X^{-1}_{12} B_1 & B^{\mathrm{T}}_2 X^{-1}_{12} A_1 \\
A^{\mathrm{T}}_2 X^{-1}_{12} B_1 & B^{\mathrm{T}}_2 X^{-1}_{12} A_1
\end{pmatrix}^{(AB)}, \label{eq:Sigma12} \\
\xi^A_{12} &=  \frac{i}{2a^2} \begin{pmatrix} c\,\eta_1 & c\,\eta_2        \end{pmatrix}^B  \begin{pmatrix}
X^{-1}_{12} B_2 & X^{-1}_{12} A_2 \\
X^{-1}_{12} B_1 & X^{-1}_{12} A_1
\end{pmatrix}_B^{\phantom{Bi}A} \equiv \eta^B_{12}\, \rho_{12}{}_B{}^A\,. \label{eq:xi12}
\end{align}
{We will denote the Gaussian \eqref{eq:2pt gaussian} as} $\Phi(\Sigma_{12},\xi_{12},\theta_{12})$. Crucial properties to compute higher-order correlators are $(\Sigma_{12}^2)^{AB}\equiv (\Sigma_{12})^{AC} (\Sigma_{12})_C{}^B =\e^{AB}$ and $\Sigma_{12}= -\Sigma_{21}$.
\vspace{3mm}

Let us also check that the second solution, \eqref{foot:second sol}, leads to the correct two-point function. The solution can be rewritten as
\begin{equation}
\widetilde\Psi(X|y^\pm) = \dfrac{\partial}{\partial\chi} \delta^2\big(y^+_\alpha + X_\alpha{}^\beta y^-_\beta + \chi y^-_\alpha \big) \Big|_{\chi=0}\,.
\end{equation}
{In terms of} $(y_\alpha, \bar y_\alpha)$, we write the argument of the delta function as $\widetilde\Gamma(X) = \widetilde A(X)\bar y + \widetilde B(X) y$, with
\begin{equation}
\begin{aligned}
\widetilde A^{\alpha\beta}(X) &= bX^{\alpha\beta} + b(\chi - i)\, \epsilon^{\alpha\beta}\,, & \qquad
\widetilde B^{\alpha\beta}(X) &= -ib X^{\alpha\beta} + b(1 - i\chi)\, \epsilon^{\alpha\beta}\,,
\end{aligned}
\end{equation}
where we introduced an arbitrary constant $b$. With those notations, we can use the previous result \eqref{eq:psi_i psi_j} with $C_i \to 0$, $\kappa\to \widetilde \kappa$ and $A,B\to \widetilde A, \widetilde B$ to get
\begin{equation}
\widetilde\Psi(X_1|Y) \star \widetilde \Psi(X_2|Y) = \widetilde \kappa^2\, \dfrac{\partial^2}{\partial\chi_1 \partial\chi_2} \left[ \dfrac{1}{|\widetilde M_{12}|}\, \exp i \left[\tfrac{1}{2}\, \widetilde\Sigma^{AB}_{12}\, Y_A Y_B \right] \right]\Big|_{\chi_1=\chi_2=0}\,,
\end{equation}
with $\widetilde M_{12} = 2ib^2\, X_{12} + (\chi_1 + \chi_2)\epsilon$ and
\begin{equation}
\widetilde\Sigma^{AB}_{12} = 2
\begin{pmatrix}
\widetilde B^{\mathrm{T}}_2 \widetilde M^{-1}_{12} \widetilde B_1 & \widetilde B^{\mathrm{T}}_2 \widetilde M^{-1}_{12} \widetilde A_1 \\
\widetilde A^{\mathrm{T}}_2 \widetilde M^{-1}_{12} \widetilde B_1 & \widetilde B^{\mathrm{T}}_2 \widetilde M^{-1}_{12} \widetilde A_1
\end{pmatrix}^{(AB)}\,.
\end{equation}
{Tracing gives}
\begin{equation}
\Tr \widetilde\Psi_1 \star \widetilde\Psi_2 = \dfrac{\widetilde \kappa^2}{2\,b^4} \dfrac{1}{\mathbf{x}^4_{12}}\,,
\end{equation}
which is the two-point function of the $\Delta=2$ operator, which in our case is $\bar \psi \psi$.

\subsubsection{Higher-Point Procedure}
Having the building block, we can now describe the procedure to obtain $O_n$. We begin with $O_{2n}$ ($n\in\mathbb{N}_0$). Since the star-product is associative, we can compute it recursively as
\begin{equation}
O_{2n} = \Tr\Big( (\Psi_{1,2} \star \Psi_{3,4}\star \dots\star \Psi_{2n-3,2n-2})  \star \Psi_{2n-1,2n}\Big)\,. \label{eq:2n-pt}
\end{equation}
{Since we know} that $\Psi_{i,j}$ is a Gaussian in $Y^A$, we would have to compute the star-product of Gaussians. This will be performed below, but we already note that the star-product of Gaussians is a Gaussian. For the $(2n+1)$-point correlator, we have
\begin{equation}
O_{2n+1} = \Tr\Big( (\Psi_{1,2} \star \Psi_{3,4}\star \dots\star \Psi_{2n-1,2n}) \star \Psi_{2n+1} \Big)\,. \label{eq:2n+1-pt procedure}
\end{equation}
{Since Gaussians }form a closed subalgebra under the star-product, the term in the small parentheses is a Gaussian of the form $r(\mathbf{x}) \, \Phi(\Sigma,\xi,\theta)$, and one can show that
\begin{equation}
\Phi(\Sigma,\xi,\theta) \star \Psi_{k} = \dfrac{i\kappa\, \Phi(\Sigma,\xi,\theta)}{\sqrt{|-\gamma_k \Sigma\gamma^{\mathrm{T}}_k|}} \exp \frac{i}{2} b\, \big(\gamma_k \Sigma\gamma^{\mathrm{T}}_k\big)^{-1} b\, \label{eq:2n+1 pts}
\end{equation}
with
\begin{equation}
\begin{aligned}
\gamma^{\alpha B}_k &= \begin{pmatrix}
B^{\alpha\beta}_k & A^{\alpha\dot\beta}_k
\end{pmatrix}, &\qquad
b^A &= \big(\gamma_k Y - \gamma_k \Sigma Y - \gamma_k \xi + C_k \big)^A\,.
\end{aligned}
\end{equation}
{We also note }that $\big(\gamma_k \Sigma\gamma^{\mathrm{T}}_k \big)^{-1} = -\big(\gamma_k \Sigma\gamma_k^{\mathrm{T}}\big) \big/ |\gamma_k \Sigma\gamma_k^{\mathrm{T}}|$. In the following, we denote
\begin{equation}
\Psi_{1,2}\star\dots\star \Psi_{n-1,n} \sim \exp  i\left[\frac{1}{2} \big(\Sigma_{[n]}\big)_{AB}\, Y^A Y^B + \xi_{[n]}^A Y_A + \theta_{[n]}\right].
\end{equation}
Therefore, taking $Y=0$, the argument of the exponential for \eqref{eq:2n+1-pt procedure} reads
\begin{equation}
\theta_{[2n+1]} = \theta_{[2n]} - \frac{1}{2} \big(\xi_{[2n]}\gamma^\mathrm{T}_{2n+1} + \eta_{2n+1} \big)  \dfrac{\gamma_{2n+1}\,\Sigma_{1,2n}\, \gamma^\mathrm{T}_{2n+1}}{|\gamma_{2n+1}\,\Sigma_{1,2n}\, \gamma^\mathrm{T}_{2n+1}|} \big(-\gamma_{2n+1}\xi_{[2n]} +\eta_{2n+1} \big)\,. \label{eq:theta for 2n+1 pts}
\end{equation}
{This suggests} that we first need to compute the $2n$-point correlators. Let us see how the star-product of Gaussians works.



\subsubsection{Star-Product of Gaussians}
Let us be more general and consider a Gaussian
\begin{equation}
\Phi(f,\xi, \theta) = \exp i\left[\frac{1}{2} f_{AB}\, Y^A Y^B + \xi^A Y_A + \theta\right], \quad A,B=1,\dots,2N, \label{eq:general gaussian}
\end{equation}
with $f_{AB}$ a symmetric matrix, $\xi^A$ a commuting spinor, and $\theta$ a constant. One can show that~\cite{Didenko:2003aa,Didenko:2012tv}
\begin{equation}
\Phi(f,\xi, 0) \star \Phi(g,\eta, 0) = \dfrac{(-1)^N}{\sqrt{|f|\,|g+f^{-1}|}}\, \Phi(f\circ g, \xi\circ \eta, q) \label{eq:star product of gaussians}
\end{equation}
where (matrix $1$ in \eqref{eq:star product of gaussians features} should be understood as $\epsilon_{AB}$)
\begin{equation}
\begin{aligned}
(f\circ g)_{AB} &= \dfrac{1}{1+gf} (1+g) - \dfrac{1}{1+fg} (1-f),\\
(\xi\circ\eta)^A &= \xi^B\, \left[\dfrac{1}{1+gf}(1+g)\right]_B{}^A + \eta^B\, \left[\dfrac{1}{1+fg}(1-f)\right]_B{}^A,\\
q &= \frac{1}{2} \left( \dfrac{1}{1+gf}\,g\right)_{AB} \xi^A \xi^B + \frac{1}{2} \left( \dfrac{1}{1+fg}\, f\right)_{AB} \eta^A \eta^B \\
&\qquad - \left( \dfrac{1}{1+gf} \right)_{AB} \xi^A \eta^B.
\end{aligned}\label{eq:star product of gaussians features}
\end{equation}
{The proof follows }from Gaussian integration. In our $\Psi_{i,j}$, \eqref{eq:2pt gaussian}, we have $f^2=\epsilon=g^2$. In that case, one has
\begin{equation}
\begin{aligned}
(f\circ g)_{AB} &= \dfrac{1}{f+g} (2+g-f),\\
(\xi\circ\eta)^A &= \frac{1}{2}\xi^B (1 +f\circ g)_B{}^A + \frac{1}{2}\eta^B (1 - f\circ g)_B{}^A,\\
q &= \frac{1}{8} \{f,g\}_{\circ AB} \left(\xi^A \xi^B +  \eta^A \eta^B\right) - \frac{1}{2} \left(1+\frac{1}{2} [f,g]_{\circ}\right)_{AB} \xi^A \eta^B,\\
\end{aligned}\label{eq:star product of gaussians features f^2=1}
\end{equation}
and $|f||g+f^{-1}| = |f+g|$. We note that $(f\circ g)_{AB}$ is symmetric.


\subsubsection{Higher-Point Correlators}
Now that we have simple formulas, we can proceed to compute \eqref{eq:2n-pt}, whose building block is $\Psi_{i,j}\star \Psi_{k,l}$. A crucial property of our $\Sigma$ matrices is that
\begin{equation}
\Sigma_{ij}\circ \Sigma_{kl} = \Sigma_{il}.
\end{equation}
{This tells us that the} only matrices $f$, $g$ and $f\circ g$ that appear in \eqref{eq:star product of gaussians features f^2=1} are our $\Sigma$. Therefore, it is useful to use the projectors $\pi^\pm_{ij} := \frac{1}{2}(\e \pm \Sigma_{ij})$ that satisfy the following properties:
\begin{equation}
\begin{aligned}
\big(\pi^\pm_{ij}\big)^\mathrm{T} &= -\pi^\mp_{ij}\,,\qquad &\phantom{\pi^\pm_{ij}}\pi^\pm_{ji} = \pi^\mp_{ij}\,,\\
\pi^\pm_{ij} \pi^\mp_{ij} &= 0,\qquad &\pi^\pm_{ij} \pi^\pm_{ij} = \pi^\pm_{ij}\,,\\
\pi^-_{ij} \pi^+_{ik} &= 0,\qquad &\pi^+_{ij} \pi^+_{ik} = \pi^+_{ik}\,.
\end{aligned} \label{eq:projectors properties}
\end{equation}
%We also note that $\{\Sigma_{ij},\Sigma_{kl}\}_{\circ AB} = \Sigma_{il} + \Sigma_{kj} \equiv \pm 2 \big( \pi^\pm_{il} - \pi^\mp_{kj} \big)$ and $\left(1+\frac{1}{2} [\Sigma_{ij},\Sigma_{kl}]_{\circ}\right) \equiv \pi^+_{il} + \pi^+_{jk}$.

Now we proceed recursively. For the two-point correlator, we had \eqref{eq:2pt gaussian}. For the four-point one, writing \eqref{eq:star product of gaussians features f^2=1} in terms of projectors, we have
\begin{equation}
\begin{aligned}
\Sigma_{[4]} &:= \Sigma_{12}\circ \Sigma_{34} \equiv \Sigma_{14}\,,\\
\xi_{[4]} &:= \xi_{12} \circ \xi_{34} = \xi^B_{12} \big(\pi^+_{14}\big)_B{}^A + \xi^B_{34} \big(\pi^-_{14}\big)_B{}^A\,,\\
\theta_{[4]} &= P_{12} + P_{34} + \frac{1}{4} \big(\pi^+_{14} - \pi^+_{23}\big)_{AB} \big(\xi^A_{12}\xi^B_{12} + \xi^A_{34}\xi^B_{34} \big) - \frac{1}{2} \big(\pi^+_{14} + \pi^+_{23}\big)_{AB}\, \xi^A_{12}\xi^B_{34}\,. \label{eq:theta 4pts first form}
\end{aligned}
\end{equation}
{Thanks to the structure}s listed in Appendix \ref{app:conformal structures}, one obtains
\begin{equation}
\theta_{[4]} = \frac{1}{2} \big[ P_{12} + P_{23} + P_{34} - P_{41} \big] + \frac{1}{4} \big[ Q^1_{24} + Q^2_{31} + Q^3_{42} + Q^4_{13} \big]\,. \label{eq:theta 4pts}
\end{equation}
{The properties of} the projectors \eqref{eq:projectors properties} help us to easily generalise
\begin{equation}
\begin{aligned}
\Sigma_{[2n]} &= \Sigma_{1,2n}\,,\\
\hspace{1cm}\xi^A_{[2n]} &= \xi^B_{12} \big(\pi^+_{1,2n}\big)_B{}^A + \xi^B_{2n-1,2n} \big(\pi^-_{1,2n}\big)_B{}^A\,, \\
\theta_{[2n]} &= \theta_{[2n-2]} + \frac{1}{4} \big(\pi^+_{1,2n} - \pi^+_{2n-2,2n-1}\big)_{AB} \big(\xi^A_{[2n-2]}\,\xi^B_{[2n-2]} + \xi^A_{2n-1,2n}\,\xi^B_{2n-1,2n} \big) \\
&\phantom{= \theta_{[2n-1]}}\ - \frac{1}{2} \big(\pi^+_{1,2n} + \pi^+_{2n-2,2n-1}\big)_{AB}\, \xi^A_{[2n-2]}\xi^B_{2n-1,2n} + P_{2n-1,2n}\,.
\end{aligned}
\end{equation}
{One finds}
\begin{equation}
\begin{aligned}
\theta_{[2n]} &= \theta_{[2n-2]} + \frac{1}{4} \big[Q^1_{2n-2,2n} + Q^{2n-2}_{2n-1,1} + Q^{2n-1}_{2n,2n-2} + Q^{2n}_{1,2n-1} \big] \\
&\phantom{= \theta_{[2n-1]}}\ + \frac{1}{2} \left[P_{1,2n} - P_{1,2n-2} + P_{2n-2,2n-1} + P_{2n-1,2n} \right],
\end{aligned}
\end{equation}
which reduces to
\begin{equation}
\theta_{[2n]} = \frac{1}{4} \sum_{i=1}^{2n} Q^i_{i+1,i-1} + \frac{1}{2} \sum_{i=1}^{2n} (-)^{\delta_{i,2n}} P_{i,i+1} \, ,
\end{equation}
where the sum is understood to be mod $2n$.
% EE: Please check meaning retained
For prefactors, we have $|f+g|^{-1/2}$ from \eqref{eq:star product of gaussians}, but one should also take into account the prefactor of every $\Psi_{i,j}$, e.g., $x_{12}^{-2}$. Due to
\begin{equation}
|\Sigma_{ij} + \Sigma_{kl}| = 16\, \dfrac{|X_{jk}||X_{il}|}{|X_{ij}||X_{kl}|}\,,
\end{equation}
and requiring \eqref{eq:cond on M,kappa}, one can find the prefactor for $O_{2n}$
\begin{equation}
\left[2^{2n-2} \prod_{i=1}^{2n} x_{i,i+1} \right]^{-1}.
\end{equation}


\subsubsection{$\boldsymbol{(2n+1)}$-pt Functions}
We can now compute \eqref{eq:2n+1-pt procedure}. With the help of \eqref{eq:theta for 2n+1 pts} and thanks to the structures in Appendix \ref{app:conformal structures}, one has
\begin{equation}
\begin{aligned}
\theta_{[2n+1]} &= \theta_{[2n]} + \frac{1}{4} \big( Q^1_{2n,2n+1} + Q^{2n}_{2n+1,1} + Q^{2n+1}_{1,2n} \big)+ \frac{1}{2} \big(P_{1,2n+1} + P_{2n,2n+1} - P_{1,2n} \big )\\
&= \frac{1}{4} \sum_{i=1}^{2n+1} Q^i_{i+1,i-1} + \frac{1}{2} \sum_{i=1}^{2n+1} (-)^{\delta_{i,2n+1}} P_{i,i+1}\,.
\end{aligned}
\end{equation}
{From} \eqref{eq:2n+1 pts} and with $|\gamma_k \Sigma_{ij} \gamma^{\mathrm{T}}_k| = -16 a^4\, |X_{ik}| |X_{jk}| / |X_{ij}|$, the prefactor is found to be ({the sign of} $\kappa$ can be taken as $-1$ to obtain a {positive function)} %MDPI: we remove the words ``, see also footnote.'' since we have remove the footnote
\begin{equation}
\left[2^{2n-1} \prod_{i=1}^{2n+1} x_{i,i+1} \right]^{-1}\,.
\end{equation}
\vspace{1mm}
{Therefore,} we can summarise the results for any $n\in \mathbb{N}_0$ by
\begin{equation}
O_{n} = \dfrac{1}{2^{n-2} \prod_{i=1}^{n} x_{i,i+1}} \, \exp i\left[ \frac{1}{4} \sum_{i=1}^{n} Q^i_{i+1,i-1} + \frac{1}{2} \sum_{i=1}^{n} (-)^{\delta_{i,n}} P_{i,i+1} \right]\,, \label{eq: O_n result}
\end{equation}
where the sum and the product are understood to be mod $n$.
% EE: Please check meaning retained


%%%%%%%%%%%%%%%%%%%%%%%%%%%%%%%%%%%%%%%%%%
\section{Conclusions and Discussion}
The main results of the paper are the wave-functions that represent higher-spin currents multiplets in the higher-spin algebra, and the calculation of correlators \eqref{eq: O_n result}. The result \eqref{eq: O_n result} is exactly the conformally invariant generating function of correlators found in~\cite{Colombo:2012jx,Didenko:2012tv, Bonezzi:2017vha}. Insertions of the $\Delta=2$ wave-function $\widetilde\Psi$, \eqref{foot:second sol}, will have to lead to the results of \cite{Didenko:2013bj}. It should be noted that the wave-functions on the CFT side found in this paper are much simpler than those on $AdS_4$ of \cite{Colombo:2012jx,Didenko:2012tv, Didenko:2013bj, Bonezzi:2017vha}. We hope that the CFT wave-functions provide a useful first step to compute the deformed invariants \cite{Gerasimenko:2021sxj} of the slightly-broken higher-spin symmetry.

Lastly, it is also worth mentioning recent works \cite{Ponomarev:2022qkx,Ponomarev:2022ryp,Ponomarev:2022atv} that apply similar ideas to directly looking for higher-spin invariant observables. The higher-spin algebra of these papers is a commutative limit of the higher-spin algebra we use in this paper along two (out of four) oscillators. In particular, the amplitudes (and wave-functions) of \cite{Ponomarev:2022atv} should be some 'flat limits' of the correlation functions of the present paper. It would also be interesting to extend the results of this paper to other dimensions and to find the CFT counterpart of the rather complicated bulk-to-boundary propagator found in \cite{Didenko:2012vh}.



%%%%%%%%%%%%%%%%%%%%%%%%%%%%%%%%%%%%%%%%%%
\vspace{6pt}

%%%%%%%%%%%%%%%%%%%%%%%%%%%%%%%%%%%%%%%%%%
%% optional
%\supplementary{The following supporting information can be downloaded at:  \linksupplementary{s1}, Figure S1: title; Table S1: title; Video S1: title.}

% Only for the journal Methods and Protocols:
% If you wish to submit a video article, please do so with any other supplementary material.
% \supplementary{The following supporting information can be downloaded at: \linksupplementary{s1}, Figure S1: title; Table S1: title; Video S1: title. A supporting video article is available at doi: link.}

%%%%%%%%%%%%%%%%%%%%%%%%%%%%%%%%%%%%%%%%%%

\funding{This research was supported by the European Research Council (ERC) under the European Union’s Horizon 2020 research and innovation programme (grant agreement no. 101002551).}



\institutionalreview{Not applicable.}

\informedconsent{Not applicable.}



\acknowledgments{This work is a part of a Master's thesis, defended at University of Mons (Belgium) in June 2022 under the supervision of E. Skvortsov. The author is grateful to E. Skvortsov for many stimulating discussions.}

\conflictsofinterest{The author declares no conflict of interest.}

%%%%%%%%%%%%%%%%%%%%%%%%%%%%%%%%%%%%%%%%%%
%% Optional
%\sampleavailability{Samples of the compounds ... are available from the authors.}




%%%%%%%%%%%%%%%%%%%%%%%%%%%%%%%%%%%%%%%%%%
%% Optional
\appendixtitles{yes} % Leave argument "no" if all appendix headings stay EMPTY (then no dot is printed after "Appendix A"). If the appendix sections contain a heading then change the argument to "yes".
\appendixstart
\appendix
\section[\appendixname~\thesection]{Vector-Spinor Dictionary}
%\subsection[\appendixname~\thesubsection]{}

\label{app:vec-to-spin dictionary}

In $3d$, the Lorentz algebra is $so(2,1)$, isomorphic to $sl(2,\mathbb{R})$. From this fact, any Lorentz vector can be represented as a $2\times 2$ symmetric matrix. Indeed, if $\{\mathbf{x}^m\}_{m=0,1,2}$ are the components of a 3-vector $\mathbf{x}$, with the following Pauli matrices
\begin{equation}
(\sigma_0)^{\alpha\beta} =
\begin{pmatrix}
1	&	0\\
0	& 	1
\end{pmatrix}\,,%^{\alpha\beta}
\qquad
(\sigma_1)^{\alpha\beta} =
\begin{pmatrix}
1	&	0\\
0	& 	-1
\end{pmatrix}\,,%^{\alpha\beta}
\qquad
(\sigma_2)^{\alpha\beta} =
\begin{pmatrix}
0	&	1\\
1	& 	0
\end{pmatrix},  %^{\alpha\beta}
\end{equation}
we can form the matrix $X$ of components $X^{\alpha\beta} = \mathbf{x}^m\, (\sigma_m)^{\alpha\beta}$. A Lorentz transformation corresponds to an $\mathrm{SL}(2,\mathbb{R})$ matrix $A_{\alpha}{}^\beta$ acting as $X^{\alpha\beta} \to X^{\gamma\delta} A_{\gamma}{}^\alpha A_{\delta}{}^\beta$. For $\mathbf{x}^m = (t,x,y)$, we~have
\begin{equation}
X^{\alpha\beta} = \begin{pmatrix}
t+x	&	y\\
y	& 	t-x
\end{pmatrix}%^{\alpha\beta}
\qquad \text{and}\qquad
(X^{-1})^{\alpha\beta} = \frac{-1}{|X|} X^{\alpha\beta}\,,
\end{equation}
where the determinant $|X|$ is $-|\mathbf{x}|^2 = -\eta_{mn} \mathbf{x}^m \mathbf{x}^{n}$, since we take $\eta_{mn}$ as the Minkowski metric of signature $(-++)$. %EE: Please check meaning retained
We observe that $X^{-1}$ is obtained from an inversion \linebreak $R\mathbf{x}^m = \mathbf{x}^m/|\mathbf{x}|^2$, i.e., $RX^{\alpha\beta} = (X^{-1})^{\alpha\beta}$. We also note that
\begin{equation}
\dfrac{\partial}{\partial X^{\alpha\beta}} X^{\gamma\delta} := \partial_{\alpha\beta} X^{\gamma\delta} = \frac{1}{2} \left(\delta_\alpha{}^\gamma \delta_\beta{}^\delta + \delta_\alpha{}^\delta \delta_\beta{}^\gamma  \right)\,.
\end{equation}

We introduce the $\mathrm{SL}(2,\mathbb{R})$-invariant tensor $\e_{\alpha\beta} = -\e_{\beta\alpha}$, with $\e_{12} = +1$ and its inverse, such that $\e_{\alpha\beta}\e^{\gamma\beta} = \delta_\alpha{}^\gamma$, i.e., $\e^{12}=+1$. With them, one can raise and lower spinorial indices. For a spinor $\xi^\alpha$, we use Penrose's conventions:
\begin{equation}
\xi_\alpha = \xi^\beta\, \e_{\beta\alpha}\,, \qquad \xi^\alpha = \e^{\alpha\beta} \, \xi_\beta\,. \label{eq:eps raise lower conv}
\end{equation}
{We also define} $\partial_\alpha\equiv \frac{\partial}{\partial\xi^\alpha}$, such that $\partial_\alpha \xi_\beta = \e_{\alpha\beta}$ and $\partial^{\alpha} = \e^{\alpha\beta} \partial_\beta$. The contraction between two spinors $\chi,\xi$ is defined as $\chi\xi \equiv \chi^\alpha\xi_\alpha = -\xi\chi$. Finally, we note that any bi-spinor $A_{\alpha\beta}$ can be written as
\begin{equation}
A_{\alpha\beta} = S_{\alpha\beta} + \tfrac{1}{2} A_\lambda{}^\lambda \e_{\alpha\beta} \label{eq:arbitrary spinor matrix}
\end{equation}
with $S_{\alpha\beta}=S_{\beta\alpha}$ and $A_\lambda{}^\lambda = \e^{\lambda\gamma} A_{\lambda \gamma}$ the symplectic trace of $A_{\alpha\beta}$. In addition, we write the matrix multiplication between two matrices $A$ and $B$ as $(AB)_{\alpha}{}^\gamma \equiv A_{\alpha}{}^\beta B_\beta{}^{\gamma}$ and  \linebreak$(A\xi)^{\alpha} \equiv A^{\alpha\beta}\xi_{\beta}$. Then, $\mathbf{x}^m = \frac{1}{2} \,(X\sigma^m)_{\alpha}{}^{\alpha}$.



\section[\appendixname~\thesection]{Conformal Structures}
\label{app:conformal structures}

We first recall the notation we already introduced, \eqref{eq:xi12},
\begin{align}
\eta_{ij} &= c
\begin{pmatrix}
\eta^A_i & \eta_j
\end{pmatrix}^A\,, &
\rho^{AB}_{ij} &= \frac{i}{2a^2}
\begin{pmatrix}
X^{-1}_{12} B_2 & X^{-1}_{12} A_2 \\
X^{-1}_{12} B_1 & X^{-1}_{12} A_1
\end{pmatrix}^{AB}\,.
\end{align}
{For} $2n$-pt functions, we needed the following structures ({we always have a factor} $\frac{-ic^2}{2a^2}$ but, due to \eqref{eq:condition on C,M}, {this factor is 1})
\begin{eqnarray}
\eta^A_{ij}\, \eta^B_{ij} \ \big[\rho_{ij}\, \pi^+_{ik}\, \rho^\mathrm{T}_{ij} \big]_{AB} = Q^i_{kj} + P_{ij}\,, \quad &\eta^A_{ij}\, \eta^B_{ij}\  \big[\rho_{ij}\, \pi^+_{jk}\, \rho^\mathrm{T}_{ij} \big]_{AB} = Q^j_{ki} - P_{ij}\,, \\
\eta^A_{ij}\, \eta^B_{ij} \ \big[\rho_{ij}\, \pi^+_{ik} \pi^-_{il}\, \rho^\mathrm{T}_{ij} \big]_{AB} = Q^i_{kl}\,, \quad &\eta^A_{ij}\, \eta^B_{ij} \ \big[\rho_{ij}\, \pi^+_{jk} \pi^-_{jl}\, \rho^\mathrm{T}_{ij} \big]_{AB} = Q^j_{kl}\,,\\
\eta^A_{ij}\, \eta^B_{kl} \ \big[\rho_{ij}\, \pi^+_{il}\, \rho^\mathrm{T}_{kl} \big]_{AB} = P_{il}\,, \quad &\eta^A_{ij}\, \eta^B_{kl} \ \big[\rho_{ij}\, \pi^+_{jk}\, \rho^\mathrm{T}_{kl} \big]_{AB} =  P_{jk}\,.
\end{eqnarray}
{For} $(2n+1)$-pt functions, with $f := (\gamma^\mathrm{T}\gamma)_{2n+1} \Sigma_{1,2n} (\gamma^\mathrm{T}\gamma)_{2n+1} \big/ |\gamma\Sigma_{1,2n}\gamma|$,
\begin{gather}
\eta^A_{12}\, \eta^B_{12} \ \big[\rho_{12}\, \pi^+_{1,2n}\, f\, \pi^-_{1,2n}\, \rho^\mathrm{T}_{12} \big]_{AB} = \tfrac{1}{2} Q^1_{2n,2n+1}\,,\\
\eta^A_{2n-1,2n}\, \eta^B_{2n-1,2n} \ \big[\rho_{2n-1,2n}\, \pi^-_{1,2n}\, f\, \pi^+_{1,2n}\, \rho^\mathrm{T}_{2n-1,2n} \big]_{AB} = -\tfrac{1}{2} Q^{2n}_{1,2n+1}\,,\\
\eta^A_{12}\, \eta^B_{2n-1,2n} \ \big[\rho_{12}\, \pi^+_{1,2n}\, f\, \pi^+_{1,2n}\, \rho^\mathrm{T}_{2n-1,2n} \big]_{AB} = -\tfrac{1}{2} P_{1,2n} \,,
\end{gather}
and, with $g := (\gamma^\mathrm{T}\gamma)_{2n+1} \Sigma_{1,2n} \gamma^\mathrm{T}_{2n+1}$,
\begin{gather}
\eta^A_{12}\, \eta^\beta_{2n+1} \ \big[\rho_{12}\, \pi^+_{1,2n}\, g\big]_{A\beta} = -\tfrac{1}{2} P_{1,2n+1}\,,\\
\eta^A_{2n-1,2n}\, \eta^\beta_{2n+1} \ \big[\rho_{2n-1,2n}\, \pi^-_{1,2n}\, g\big]_{A\beta} = -\tfrac{1}{2} P_{2n,2n+1}\,.
\end{gather}
{Lastly,}
\begin{equation}
\eta^\alpha_{2n+1}\, \eta^\beta_{2n+1}\ \left[ \dfrac{\gamma_{2n+1}\Sigma_{1,2n}\gamma^\mathrm{T}_{2n+1}}{|\gamma\Sigma\gamma|} \right]_{\alpha\beta} = \tfrac{1}{2} Q^{2n+1}_{1,2n}\,.
\end{equation}






%%%%%%%%%%%%%%%%%%%%%%%%%%%%%%%%%%%%%%%%%%
\begin{adjustwidth}{-\extralength}{0cm}
\printendnotes[custom] % Un-comment to print a list of endnotes

\reftitle{References}

% Please provide either the correct journal abbreviation (e.g., according to the “List of Title Word Abbreviations” http://www.issn.org/services/online-services/access-to-the-ltwa/) or the full name of the journal.
% Citations and References in Supplementary files are permitted provided that they also appear in the reference list here.

%=====================================
% References, variant A: external bibliography
%=====================================
%\bibliography{megabib}
\begin{thebibliography}{999}

\bibitem[Sezgin and Sundell(2002)]{Sezgin:2002rt}
Sezgin, E.; Sundell, P.
\newblock {Massless higher spins and holography}.
\newblock {\em Nucl. Phys.} {\bf 2002}, {\em B644},~{303--370}. [\href{http://doi.org/10.1016/S0550-3213(02)00739-3}{CrossRef}]

%  \href{http://xxx.lanl.gov/abs/hep-th/0205131}{{\normalfont
%  [arXiv:hep-th/hep-th/0205131]}}.

\bibitem[Klebanov and Polyakov(2002)]{Klebanov:2002ja}
Klebanov, I.R.; Polyakov, A.M.
\newblock {AdS dual of the critical $O(N)$ vector model}.
\newblock {\em Phys. Lett.} {\bf 2002}, {\em B550},~213--219. [\href{http://dx.doi.org/10.1016/S0370-2693(02)02980-5}{CrossRef}]
%  \href{http://xxx.lanl.gov/abs/hep-th/0210114}{{\normalfont
%  [hep-th/0210114]}}.

\bibitem[Sezgin and Sundell(2005)]{Sezgin:2003pt}
Sezgin, E.; Sundell, P.
\newblock {Holography in 4D (super) higher spin theories and a test via cubic
scalar couplings}.
\newblock {\em J. High Energy Phys.} {\bf 2005}, {\em 0507},~044. [\href{http://dx.doi.org/10.1088/1126-6708/2005/07/044}{CrossRef}]
%  \href{http://xxx.lanl.gov/abs/hep-th/0305040}{{\normalfont
%  [arXiv:hep-th/hep-th/0305040]}}.

\bibitem[Leigh and Petkou(2003)]{Leigh:2003gk}
Leigh, R.G.; Petkou, A.C.
\newblock {Holography of the N=1 higher spin theory on AdS(4)}.
\newblock {\em J. High Energy Phys.} {\bf 2003}, {\em 0306},~011. [\href{http://dx.doi.org/10.1088/1126-6708/2003/06/011}{CrossRef}]
%  \href{http://xxx.lanl.gov/abs/hep-th/0304217}{{\normalfont
%  [arXiv:hep-th/hep-th/0304217]}}.

\bibitem[Das and Jevicki(2003)]{Das:2003vw}
Das, S.R.; Jevicki, A.
\newblock {Large N collective fields and holography}.
\newblock {\em Phys. Rev. D} {\bf 2003}, {\em 68},~044011. [\href{http://dx.doi.org/10.1103/PhysRevD.68.044011}{CrossRef}]
%  \href{http://xxx.lanl.gov/abs/hep-th/0304093}{{\normalfont
%  [hep-th/0304093]}}.
%\newblock.

\bibitem[Bekaert \em{et~al.}(2015)Bekaert, Erdmenger, Ponomarev, and
Sleight]{Bekaert:2015tva}
Bekaert, X.; Erdmenger, J.; Ponomarev, D.; Sleight, C.
\newblock {Quartic AdS Interactions in Higher-Spin Gravity from Conformal Field
Theory}.
\newblock {\em J. High Energy Phys.} {\bf 2015}, {\em 11},~149. [\href{http://dx.doi.org/10.1007/JHEP11(2015)149}{CrossRef}]
%  \href{http://xxx.lanl.gov/abs/1508.04292}{{\normalfont
%  [arXiv:hep-th/1508.04292]}}.

\bibitem[de~Mello~Koch \em{et~al.}(2019)de~Mello~Koch, Jevicki, Suzuki, and
Yoon]{deMelloKoch:2018ivk}
de~Mello~Koch, R.; Jevicki, A.; Suzuki, K.; Yoon, J.
\newblock {AdS Maps and Diagrams of Bi-local Holography}.
\newblock {\em J. High Energy Phys.} {\bf 2019}, {\em 03},~133. [\href{http://dx.doi.org/10.1007/JHEP03(2019)133}{CrossRef}]
%  \href{http://xxx.lanl.gov/abs/1810.02332}{{\normalfont
%  [arXiv:hep-th/1810.02332]}}.

\bibitem[Aharony \em{et~al.}(2020)Aharony, Chester, and
Urbach]{Aharony:2020omh}
Aharony, O.; Chester, S.M.; Urbach, E.Y.
\newblock {A Derivation of AdS/CFT for Vector Models}. \emph{J. High Energy Phys.} \textbf{2020}, {1--72} 
. [\href{http://dx.doi.org/10.1007/JHEP03(2021)208}{CrossRef}]

%{\bf 2020}.
%\newblock  \href{http://xxx.lanl.gov/abs/2011.06328}{{\normalfont
%  [arXiv:hep-th/2011.06328]}}.

\bibitem[Giombi \em{et~al.}(2012)Giombi, Minwalla, Prakash, Trivedi, Wadia, and
Yin]{Giombi:2011kc}
Giombi, S.; Minwalla, S.; Prakash, S.; Trivedi, S.P.; Wadia, S.R.; Yin, X.
\newblock {Chern-Simons Theory with Vector Fermion Matter}.
\newblock {\em Eur. Phys. J.} {\bf 2012}, {\em C72},~2112. [\href{http://dx.doi.org/10.1140/epjc/s10052-012-2112-0}{CrossRef}]
%  \href{http://xxx.lanl.gov/abs/1110.4386}{{\normalfont
%  [arXiv:hep-th/1110.4386]}}.

\bibitem[Maldacena and Zhiboedov(2012)]{Maldacena:2012sf}
Maldacena, J.; Zhiboedov, A.
\newblock {Constraining conformal field theories with a slightly broken higher
spin symmetry}. \emph{Class. Quantum Gravity} {\bf 2012}, \emph{30}, 104003. [\href{http://dx.doi.org/10.1088/0264-9381/30/10/104003}{CrossRef}]
%\newblock  \href{http://xxx.lanl.gov/abs/1204.3882}{{\normalfont
%  [arXiv:hep-th/1204.3882]}}.

\bibitem[Aharony \em{et~al.}(2012)Aharony, Gur-Ari, and Yacoby]{Aharony:2012nh}
Aharony, O.; Gur-Ari, G.; Yacoby, R.
\newblock {Correlation Functions of Large N Chern-Simons-Matter Theories and
Bosonization in Three Dimensions}.
\newblock {\em J. High Energy Phys.} {\bf 2012}, {\em 12},~028. [\href{http://dx.doi.org/10.1007/JHEP12(2012)028}{CrossRef}]
%  \href{http://xxx.lanl.gov/abs/1207.4593}{{\normalfont
%  [arXiv:hep-th/1207.4593]}}.

\bibitem[Aharony(2016)]{Aharony:2015mjs}
Aharony, O.
\newblock {Baryons, monopoles and dualities in Chern-Simons-matter theories}.
\newblock {\em J. High Energy Phys.} {\bf 2016}, {\em 02},~093. [\href{http://dx.doi.org/10.1007/JHEP02(2016)093}{CrossRef}]
%  \href{http://xxx.lanl.gov/abs/1512.00161}{{\normalfont
%  [arXiv:hep-th/1512.00161]}}.

\bibitem[Karch and Tong(2016)]{Karch:2016sxi}
Karch, A.; Tong, D.
\newblock {Particle-Vortex Duality from 3d Bosonization}.
\newblock {\em Phys. Rev.} {\bf 2016}, {\em X6},~031043. [\href{http://dx.doi.org/10.1103/PhysRevX.6.031043}{CrossRef}]
%  \href{http://xxx.lanl.gov/abs/1606.01893}{{\normalfont
%  [arXiv:hep-th/1606.01893]}}.

\bibitem[Seiberg \em{et~al.}(2016)Seiberg, Senthil, Wang, and
Witten]{Seiberg:2016gmd}
Seiberg, N.; Senthil, T.; Wang, C.; Witten, E.
\newblock {A Duality Web in 2+1 Dimensions and Condensed Matter Physics}.
\newblock {\em Annals Phys.} {\bf 2016}, {\em 374},~395--433. [\href{http://dx.doi.org/10.1016/j.aop.2016.08.007}{CrossRef}]
%  \href{http://xxx.lanl.gov/abs/1606.01989}{{\normalfont
%  [arXiv:hep-th/1606.01989]}}.

\bibitem[Dirac(1963)]{Dirac:1963ta}
Dirac, P.A.M.
\newblock {A Remarkable representation of the 3 + 2 de Sitter group}.
\newblock {\em J. Math. Phys.} {\bf 1963}, {\em 4},~901--909. [\href{http://dx.doi.org/10.1063/1.1704016}{CrossRef}]

\bibitem[G{\"u}naydin and Saclioglu(1982)]{Gunaydin:1981yq}
G{\"u}naydin, M.; Saclioglu, C.
\newblock {Oscillator Like Unitary Representations of Noncompact Groups With a
Jordan Structure and the Noncompact Groups of Supergravity}.
\newblock {\em Commun. Math. Phys.} {\bf 1982}, {\em 87},~159. [\href{http://dx.doi.org/10.1007/BF01218560}{CrossRef}]

\bibitem[G{\"u}naydin(1983)]{Gunaydin:1983yj}
G{\"u}naydin, M.
\newblock {Oscillator like unitary representations of noncompact groups and
supergroups and extended supergravity theories}.
\newblock In Proceedings of the {Group Theoretical Methods in Physics, 11th International Colloquium, Istanbul, Turkey, 23--28 August 1982}; pp. 192--213.

\bibitem[Vasiliev(1988)]{Vasiliev:1986qx}
Vasiliev, M.A.
\newblock Extended higher spin superalgebras and their realizations in terms of
quantum operators.
\newblock {\em Fortsch. Phys.} {\bf 1988}, {\em 36},~33--62. [\href{http://dx.doi.org/10.1002/prop.2190360104}{CrossRef}]

\bibitem[Eastwood(2005)]{Eastwood:2002su}
Eastwood, M.G.
\newblock {Higher symmetries of the Laplacian}.
\newblock {\em Ann. Math.} {\bf 2005}, {\em 161},~1645--1665. [\href{http://dx.doi.org/10.4007/annals.2005.161.1645}{CrossRef}]
%  \href{http://xxx.lanl.gov/abs/hep-th/0206233}{{\normalfont
%  [arXiv:hep-th/hep-th/0206233]}}.

\bibitem[Joung and Mkrtchyan(2014)]{Joung:2014qya}
Joung, E.; Mkrtchyan, K.
\newblock {Notes on higher-spin algebras: Minimal representations and structure
constants}.
\newblock {\em J. High Energy Phys.} {\bf 2014}, {\em 05},~103. [\href{http://dx.doi.org/10.1007/JHEP05(2014)103}{CrossRef}]
%  \href{http://xxx.lanl.gov/abs/1401.7977}{{\normalfont
%  [arXiv:hep-th/1401.7977]}}.

\bibitem[Flato and Fronsdal(1978)]{Flato:1978qz}
Flato, M.; Fronsdal, C.
\newblock {One Massless Particle Equals Two Dirac Singletons: Elementary
Particles in a Curved Space.}
\newblock {\em Lett. Math. Phys.} {\bf 1978}, {\em 2},~421--426. [\href{http://dx.doi.org/10.1007/BF00400170}{CrossRef}]

\bibitem[Craigie \em{et~al.}(1985)Craigie, Dobrev, and Todorov]{Craigie:1983fb}
Craigie, N.S.; Dobrev, V.K.; Todorov, I.T.
\newblock {Conformally Covariant Composite Operators in Quantum
Chromodynamics}.
\newblock {\em Annals Phys.} {\bf 1985}, {\em 159},~411--444. [\href{http://dx.doi.org/10.1016/0003-4916(85)90118-6}{CrossRef}]

\bibitem[Colombo and Sundell(2012)]{Colombo:2012jx}
Colombo, N.; Sundell, P.
\newblock {Higher Spin Gravity Amplitudes From Zero-form Charges}.
\emph{arXiv } {\bf 2012}, arXiv:1208.3880.

%{\bf 2012}.
%\newblock  \href{http://xxx.lanl.gov/abs/1208.3880}{{\normalfont
%  [arXiv:hep-th/1208.3880]}}.

\bibitem[Didenko and Skvortsov(2013)]{Didenko:2012tv}
Didenko, V.; Skvortsov, E.
\newblock {Exact higher-spin symmetry in CFT: All correlators in unbroken
Vasiliev theory}.
\newblock {\em J. High Energy Phys.} {\bf 2013}, {\em 1304},~158. [\href{http://dx.doi.org/10.1007/JHEP04(2013)158}{CrossRef}]
%  \href{http://xxx.lanl.gov/abs/1210.7963}{{\normalfont
%  [arXiv:hep-th/1210.7963]}}.

\bibitem[Didenko \em{et~al.}(2013)Didenko, Mei, and Skvortsov]{Didenko:2013bj}
Didenko, V.E.; Mei, J.; Skvortsov, E.D.
\newblock {Exact higher-spin symmetry in CFT: Free fermion correlators from
Vasiliev Theory}.
\newblock {\em Phys. Rev.} {\bf 2013}, {\em D88},~046011. [\href{http://dx.doi.org/10.1103/PhysRevD.88.046011}{CrossRef}]
%  \href{http://xxx.lanl.gov/abs/1301.4166}{{\normalfont
%  [arXiv:hep-th/1301.4166]}}.

\bibitem[Bonezzi \em{et~al.}(2017)Bonezzi, Boulanger, De~Filippi, and
Sundell]{Bonezzi:2017vha}
Bonezzi, R.; Boulanger, N.; De~Filippi, D.; Sundell, P.
\newblock {Noncommutative Wilson lines in higher-spin theory and correlation
functions of conserved currents for free conformal fields}.
\newblock {\em J. Phys.} {\bf 2017}, {\em A50},~475401. [\href{http://dx.doi.org/10.1088/1751-8121/aa8efa}{CrossRef}]
%  \href{http://xxx.lanl.gov/abs/1705.03928}{{\normalfont
%  [arXiv:hep-th/1705.03928]}}.

\bibitem[Maldacena and Zhiboedov(2011)]{Maldacena:2011jn}
Maldacena, J.; Zhiboedov, A.
\newblock {Constraining Conformal Field Theories with A Higher Spin Symmetry}.
\emph{ J. Phys. Math. Theor.} {\bf 2011}, \emph{46}, 214011. [\href{http://dx.doi.org/10.1088/1751-8113/46/21/214011}{CrossRef}]
%  .
%\newblock  \href{http://xxx.lanl.gov/abs/1112.1016}{{\normalfont
%  [arXiv:hep-th/1112.1016]}}.

\bibitem[Boulanger \em{et~al.}(2013)Boulanger, Ponomarev, Skvortsov, and
Taronna]{Boulanger:2013zza}
Boulanger, N.; Ponomarev, D.; Skvortsov, E.D.; Taronna, M.
\newblock {On the uniqueness of higher-spin symmetries in AdS and CFT}.
\newblock {\em Int. J. Mod. Phys.} {\bf 2013}, {\em A28},~1350162. [\href{http://dx.doi.org/10.1142/S0217751X13501625}{CrossRef}]
%  \href{http://xxx.lanl.gov/abs/1305.5180}{{\normalfont
%  [arXiv:hep-th/1305.5180]}}.
%\newblock.

\bibitem[Alba and Diab(2013)]{Alba:2013yda}
Alba, V.; Diab, K.
\newblock {Constraining conformal field theories with a higher spin symmetry in
d = 4.} \emph{arXiv } {\bf 2013}, arXiv:1307.8092.
%\newblock  \href{http://xxx.lanl.gov/abs/1307.8092}{{\normalfont
%  [arXiv:hep-th/1307.8092]}}.

\bibitem[Alba and Diab(2015)]{Alba:2015upa}
Alba, V.; Diab, K.
\newblock {Constraining conformal field theories with a higher spin symmetry in
$d> 3$ dimensions.} \emph{J. High Energy Phys.} {\bf 2015}, {44.}  [\href{http://dx.doi.org/10.1007/JHEP03(2016)044}{CrossRef}]

%\newblock  \href{http://xxx.lanl.gov/abs/1510.02535}{{\normalfont
%  [arXiv:hep-th/1510.02535]}}.

\bibitem[Sharapov and Skvortsov(2020)]{Sharapov:2020quq}
Sharapov, A.; Skvortsov, E.
\newblock {Characteristic Cohomology and Observables in Higher Spin Gravity}.
\newblock {\em J. High Energy Phys.} {\bf 2020}, {\em 12},~190,
%  \href{http://xxx.lanl.gov/abs/2006.13986}{{\normalfont
%  [arXiv:hep-th/2006.13986]}}.
\newblock  [\href{http://dx.doi.org/10.1007/JHEP12(2020)190}{CrossRef}]

\bibitem[Sharapov and Skvortsov(2019)]{Sharapov:2018kjz}
Sharapov, A.; Skvortsov, E.
\newblock {$A_\infty$ algebras from slightly broken higher spin symmetries}.
\newblock {\em J. High Energy Phys.} {\bf 2019}, {\em 09},~024.
%  \href{http://xxx.lanl.gov/abs/1809.10027}{{\normalfont
%  [arXiv:hep-th/1809.10027]}}.
\newblock  [\href{http://dx.doi.org/10.1007/JHEP09(2019)024}{CrossRef}]

\bibitem[Gerasimenko \em{et~al.}(2022)Gerasimenko, Sharapov, and
Skvortsov]{Gerasimenko:2021sxj}
Gerasimenko, P.; Sharapov, A.; Skvortsov, E.
\newblock {Slightly broken higher spin symmetry: General structure of
correlators}.
\newblock {\em J. High Energy Phys.} {\bf 2022}, {\em 01},~097.
%  \href{http://xxx.lanl.gov/abs/2108.05441}{{\normalfont
%  [arXiv:hep-th/2108.05441]}}. 
[\href{http://dx.doi.org/10.1007/JHEP01(2022)097}{CrossRef}]

\bibitem[Skvortsov and Sharapov(2022)]{Skvortsov:2022abz}
Skvortsov, E.; Sharapov, A.
\newblock {Integrable models from non-commutative geometry, with applications
to 3d dualities}.
\newblock {\em PoS} {\bf 2022}, {\em CORFU2021},~253.
%  \href{http://xxx.lanl.gov/abs/2204.08903}{{\normalfont
%  [arXiv:hep-th/2204.08903]}}. 
[\href{http://dx.doi.org/10.22323/1.406.0253}{CrossRef}]

\bibitem[Li(2020)]{Li:2019twz}
Li, Z.
\newblock {Bootstrapping conformal four-point correlators with slightly broken
higher spin symmetry and $3D$ bosonization}.
\newblock {\em J. High Energy Phys.} {\bf 2020}, {\em 10},~007.
%  \href{http://xxx.lanl.gov/abs/1906.05834}{{\normalfont
%  [arXiv:hep-th/1906.05834]}}. 
[\href{http://dx.doi.org/10.1007/JHEP10(2020)007}{CrossRef}]

\bibitem[Kalloor(2020)]{Kalloor:2019xjb}
Kalloor, R.R.
\newblock {Four-point functions in large $N$ Chern-Simons fermionic theories}.
\newblock {\em J. High Energy Phys.} {\bf 2020}, {\em 10},~028.
%  \href{http://xxx.lanl.gov/abs/1910.14617}{{\normalfont
%  [arXiv:hep-th/1910.14617]}}.
\newblock  
[\href{http://dx.doi.org/10.1007/JHEP10(2020)028}{CrossRef}]

\bibitem[Turiaci and Zhiboedov(2018)]{Turiaci:2018nua}
Turiaci, G.J.; Zhiboedov, A.
\newblock {Veneziano Amplitude of Vasiliev Theory}.
\newblock {\em J. High Energy Phys.} {\bf 2018}, {\em 10},~034.
%  \href{http://xxx.lanl.gov/abs/1802.04390}{{\normalfont
%  [arXiv:hep-th/1802.04390]}}.
\newblock  [\href{http://dx.doi.org/10.1007/JHEP10(2018)034}{CrossRef}]

\bibitem[Jain \em{et~al.}(2021{\natexlab{a}})Jain, John, and
Malvimat]{Jain:2020puw}
Jain, S.; John, R.R.; Malvimat, V.
\newblock {Constraining momentum space correlators using slightly broken higher
spin symmetry}.
\newblock {\em J. High Energy Phys.} {\bf 2021}, {\em 04},~231.
%  \href{http://xxx.lanl.gov/abs/2008.08610}{{\normalfont
%  [arXiv:hep-th/2008.08610]}}. 
[\href{http://dx.doi.org/10.1007/JHEP04(2021)231}{CrossRef}]

\bibitem[Jain \em{et~al.}(2021{\natexlab{b}})Jain, John, Mehta, Nizami, and
Suresh]{Jain:2021vrv}
Jain, S.; John, R.R.; Mehta, A.; Nizami, A.A.; Suresh, A.
\newblock {Higher spin 3-point functions in 3d CFT using spinor-helicity
variables}.
\newblock {\em J. High Energy Phys.} {\bf 2021}, {\em 09},~041.
%  \href{http://xxx.lanl.gov/abs/2106.00016}{{\normalfont
%  [arXiv:hep-th/2106.00016]}}. 
[\href{http://dx.doi.org/10.1007/JHEP09(2021)041}{CrossRef}]

\bibitem[Jain and John(2021)]{Jain:2021gwa}
Jain, S.; John, R.R.
\newblock {Relation between parity-even and parity-odd CFT correlation
functions in three dimensions.} \emph{J. High Energy Phys.} {\bf 2021}, {67.} [\href{http://dx.doi.org/10.1007/JHEP12(2021)067}{CrossRef}]
%\newblock  \href{http://xxx.lanl.gov/abs/2107.00695}{{\normalfont
%  [arXiv:hep-th/2107.00695]}}.

\bibitem[Jain \em{et~al.}(2022)Jain, John, Mehta, and S]{Jain:2021whr}
Jain, S.; John, R.R.; Mehta, A.; S, D.K.
\newblock {Constraining momentum space CFT correlators with consistent position
space OPE limit and the collider bound}.
\newblock {\em J. High Energy Phys.} {\bf 2022}, {\em 02},~084.
%  \href{http://xxx.lanl.gov/abs/2111.08024}{{\normalfont
%  [arXiv:hep-th/2111.08024]}}. 
[\href{http://dx.doi.org/10.1007/JHEP02(2022)084}{CrossRef}]

\bibitem[Silva(2021)]{Silva:2021ece}
Silva, J.A.
\newblock {Four point functions in CFT\textquoteright{}s with slightly broken
higher spin symmetry}.
\newblock {\em J. High Energy Phys.} {\bf 2021}, {\em 05},~097.
%  \href{http://xxx.lanl.gov/abs/2103.00275}{{\normalfont
%  [arXiv:hep-th/2103.00275]}}.
\newblock  [\href{http://dx.doi.org/10.1007/JHEP05(2021)097}{CrossRef}]

\bibitem[Giombi and Yin(2011)]{Giombi:2010vg}
Giombi, S.; Yin, X.
\newblock {Higher Spins in AdS and Twistorial Holography}.
\newblock {\em J. High Energy Phys.} {\bf 2011}, {\em 1104},~086. [\href{http://dx.doi.org/10.1007/JHEP04(2011)086}{CrossRef}]
%  \href{http://xxx.lanl.gov/abs/1004.3736}{{\normalfont
%  [arXiv:hep-th/1004.3736]}}.


\bibitem[Boulanger \em{et~al.}(2016)Boulanger, Kessel, Skvortsov, and
Taronna]{Boulanger:2015ova}
Boulanger, N.; Kessel, P.; Skvortsov, E.D.; Taronna, M.
\newblock {Higher spin interactions in four-dimensions: Vasiliev versus
Fronsdal}.
\newblock {\em J. Phys.} {\bf 2016}, {\em A49},~095402. [\href{http://dx.doi.org/10.1088/1751-8113/49/9/095402}{CrossRef}]
%  \href{http://xxx.lanl.gov/abs/1508.04139}{{\normalfont
%  [arXiv:hep-th/1508.04139]}}.

\bibitem[Skvortsov and Taronna(2015)]{Skvortsov:2015lja}
Skvortsov, E.D.; Taronna, M.
\newblock {On Locality, Holography and Unfolding}.
\newblock {\em J. High Energy Phys.} {\bf 2015}, {\em 11},~044. [\href{http://dx.doi.org/10.1007/JHEP11(2015)044}{CrossRef}]
%  \href{http://xxx.lanl.gov/abs/1508.04764}{{\normalfont
%  [arXiv:hep-th/1508.04764]}}.


\bibitem[Gelfond and Vasiliev(2013)]{Gelfond:2013xt}
Gelfond, O.A.; Vasiliev, M.A.
\newblock {Operator algebra of free conformal currents via twistors}.
\newblock {\em Nucl. Phys.} {\bf 2013}, {\em B876},~871--917. [\href{http://dx.doi.org/10.1016/j.nuclphysb.2013.09.001}{CrossRef}]
%  \href{http://xxx.lanl.gov/abs/1301.3123}{{\normalfont
%  [arXiv:hep-th/1301.3123]}}.

\bibitem[Metsaev(2018)]{Metsaev:2018xip}
Metsaev, R.R.
\newblock {Light-cone gauge cubic interaction vertices for massless fields in
AdS(4)}.
\newblock {\em Nucl. Phys.} {\bf 2018}, {\em B936},~320--351.
%  \href{http://xxx.lanl.gov/abs/1807.07542}{{\normalfont
%  [arXiv:hep-th/1807.07542]}}. 
[\href{http://dx.doi.org/10.1016/j.nuclphysb.2018.09.021}{CrossRef}]

\bibitem[Skvortsov(2019)]{Skvortsov:2018uru}
Skvortsov, E.
\newblock {Light-Front Bootstrap for Chern-Simons Matter Theories}.
\newblock {\em J. High Energy Phys.} {\bf 2019}, {\em 06},~058. [\href{http://dx.doi.org/10.1007/JHEP06(2019)058}{CrossRef}]
%  \href{http://xxx.lanl.gov/abs/1811.12333}{{\normalfont
%  [arXiv:hep-th/1811.12333]}}.

\bibitem[Sharapov and Skvortsov(2022)]{Sharapov:2022awp}
Sharapov, A.; Skvortsov, E.
\newblock {Chiral Higher Spin Gravity in (A)dS${}_4$ and secrets of
Chern--Simons Matter Theories.} \emph{Nucl. Phys. B} {\bf 2022}, \emph{985}, 115982. [\href{http://dx.doi.org/10.1016/j.nuclphysb.2022.115982}{CrossRef}]
%\newblock  \href{http://xxx.lanl.gov/abs/2205.15293}{{\normalfont
%  [arXiv:hep-th/2205.15293]}}.

\bibitem[Sharapov \em{et~al.}(2022{\natexlab{a}})Sharapov, Skvortsov, and
Van~Dongen]{Sharapov:2022wpz}
Sharapov, A.; Skvortsov, E.; Van~Dongen, R.
\newblock {Chiral Higher Spin Gravity and Convex Geometry}. 
\emph{arXiv} \textbf{2022}, 	arXiv:2209.01796.
[\href{https://doi.org/10.48550/arXiv.2209.01796}{CrossRef}]

\bibitem[Sharapov \em{et~al.}(2022{\natexlab{b}})Sharapov, Skvortsov, Sukhanov,
and Van~Dongen]{Sharapov:2022nps}
Sharapov, A.; Skvortsov, E.; Sukhanov, A.; Van~Dongen, R.
\newblock {More on Chiral Higher Spin Gravity and Convex Geometry.}  \emph{Nucl. Phys. B} {\bf 2022}, \emph{990}, 116152. [\href{http://dx.doi.org/10.1016/j.nuclphysb.2023.116152}{CrossRef}]
%\newblock  \href{http://xxx.lanl.gov/abs/2209.15441}{{\normalfont
%  [arXiv:hep-th/2209.15441]}}.


\bibitem[Metsaev(1991{\natexlab{a}})]{Metsaev:1991mt}
Metsaev, R.R.
\newblock {Poincare invariant dynamics of massless higher spins: Fourth order
analysis on mass shell}.
\newblock {\em Mod. Phys. Lett.} {\bf 1991}, {\em A6},~359--367. [\href{http://dx.doi.org/10.1142/S0217732391000348}{CrossRef}]

\bibitem[Metsaev(1991{\natexlab{b}})]{Metsaev:1991nb}
Metsaev, R.R.
\newblock {$S$ matrix approach to massless higher spins theory. 2: The Case of
internal symmetry}.
\newblock {\em Mod. Phys. Lett.} {\bf 1991}, {\em A6},~2411--2421. [\href{http://dx.doi.org/10.1142/S0217732391002839}{CrossRef}]

\bibitem[Ponomarev and Skvortsov(2017)]{Ponomarev:2016lrm}
Ponomarev, D.; Skvortsov, E.D.
\newblock {Light-Front Higher-Spin Theories in Flat Space}.
\newblock {\em J. Phys.} {\bf 2017}, {\em A50},~095401. [\href{http://dx.doi.org/10.1088/1751-8121/aa56e7}{CrossRef}]
%  \href{http://xxx.lanl.gov/abs/1609.04655}{{\normalfont
%  [arXiv:hep-th/1609.04655]}}.

\bibitem[Ponomarev(2017)]{Ponomarev:2017nrr}
Ponomarev, D.
\newblock {Chiral Higher Spin Theories and Self-Duality}.
\newblock {\em  J. High Energy Phys.} {\bf 2017}, {\em 12},~141. [\href{http://dx.doi.org/10.1007/JHEP12(2017)141}{CrossRef}]
%  \href{http://xxx.lanl.gov/abs/1710.00270}{{\normalfont
%  [arXiv:hep-th/1710.00270]}}.

\bibitem[Skvortsov \em{et~al.}(2018)Skvortsov, Tran, and
Tsulaia]{Skvortsov:2018jea}
Skvortsov, E.D.; Tran, T.; Tsulaia, M.
\newblock {Quantum Chiral Higher Spin Gravity}.
\newblock {\em Phys. Rev. Lett.} {\bf 2018}, {\em 121},~031601. [\href{http://dx.doi.org/10.1103/PhysRevLett.121.031601}{CrossRef}]
%  \href{http://xxx.lanl.gov/abs/1805.00048}{{\normalfont
%  [arXiv:hep-th/1805.00048]}}.

\bibitem[Skvortsov \em{et~al.}(2020)Skvortsov, Tran, and
Tsulaia]{Skvortsov:2020wtf}
Skvortsov, E.; Tran, T.; Tsulaia, M.
\newblock {More on Quantum Chiral Higher Spin Gravity}.
\newblock {\em Phys. Rev.} {\bf 2020}, {\em D101},~106001. [\href{http://dx.doi.org/10.1103/PhysRevD.101.106001}{CrossRef}]
%  \href{http://xxx.lanl.gov/abs/2002.08487}{{\normalfont
%  [arXiv:hep-th/2002.08487]}}.

\bibitem[Skvortsov and Van~Dongen(2022)]{Skvortsov:2022syz}
Skvortsov, E.; Van~Dongen, R.
\newblock {Minimal models of field theories: Chiral Higher Spin Gravity.} \emph{Phys. Rev. D} {\bf
2022}, \emph{106}, 045006. [\href{http://dx.doi.org/10.1103/PhysRevD.106.045006}{CrossRef}]
%\newblock  \href{http://xxx.lanl.gov/abs/2204.10285}{{\normalfont
%  [arXiv:hep-th/2204.10285]}}.

\bibitem[Sharapov \em{et~al.}(2022)Sharapov, Skvortsov, Sukhanov, and
Van~Dongen]{Sharapov:2022faa}
Sharapov, A.; Skvortsov, E.; Sukhanov, A.; Van~Dongen, R.
\newblock {Minimal model of Chiral Higher Spin Gravity.} \emph{J. High Energy Phys.} {\bf 2022},  {134}. [\href{http://dx.doi.org/10.1007/JHEP09(2022)134}{CrossRef}]
%\newblock  \href{http://xxx.lanl.gov/abs/2205.07794}{{\normalfont
%  [arXiv:hep-th/2205.07794]}}.

\bibitem[Bekaert \em{et~al.}(2022)Bekaert, Boulanger, Campoleoni, Chiodaroli,
Francia, Grigoriev, Sezgin, and Skvortsov]{Bekaert:2022poo}
Bekaert, X.; Boulanger, N.; Campoleoni, A.; Chiodaroli, M.; Francia, D.;
Grigoriev, M.; Sezgin, E.; Skvortsov, E.
\newblock {Snowmass White Paper: Higher Spin Gravity and Higher Spin Symmetry.} \emph{arXiv } {\bf 2022}, arXiv:2205.01567.

\bibitem[Giombi \em{et~al.}(2011)Giombi, Prakash, and Yin]{Giombi:2011rz}
Giombi, S.; Prakash, S.; Yin, X.
\newblock {A Note on CFT Correlators in Three Dimensions.} \emph{J. High Energy Phys.} {\bf 2011}, {105}. [\href{http://dx.doi.org/10.1007/JHEP07(2013)105}{CrossRef}]
%\newblock  \href{http://xxx.lanl.gov/abs/1104.4317}{{\normalfont
%  [arXiv:hep-th/1104.4317]}}.

\bibitem[Vasiliev(2013)]{Vasiliev:2012vf}
Vasiliev, M.A.
\newblock {Holography, Unfolding and Higher-Spin Theory}.
\newblock {\em J. Phys.} {\bf 2013}, {\em A46},~214013. [\href{http://dx.doi.org/10.1088/1751-8113/46/21/214013}{CrossRef}]
%  \href{http://xxx.lanl.gov/abs/1203.5554}{{\normalfont
%  [arXiv:hep-th/1203.5554]}}.

\bibitem[Didenko and Vasiliev(2004)]{Didenko:2003aa}
Didenko, V.E.; Vasiliev, M.A.
\newblock {Free field dynamics in the generalized AdS (super)space}.
\newblock {\em J. Math. Phys.} {\bf 2004}, {\em 45},~197--215.
%  \href{http://xxx.lanl.gov/abs/hep-th/0301054}{{\normalfont
%  [hep-th/0301054]}}.
\newblock  [\href{http://dx.doi.org/10.1063/1.1633022}{CrossRef}]

\bibitem[Ponomarev(2023{\natexlab{a}})]{Ponomarev:2022qkx}
Ponomarev, D.
\newblock {Chiral higher-spin holography in flat space: The Flato-Fronsdal
theorem and lower-point functions}.
\newblock {\em J. High Energy Phys.} {\bf 2023}, {\em 01},~048.
%  \href{http://xxx.lanl.gov/abs/2210.04036}{{\normalfont
%  [arXiv:hep-th/2210.04036]}}.
\newblock [\href{http://dx.doi.org/10.1007/JHEP01(2023)048}{CrossRef}]

\bibitem[Ponomarev(2023{\natexlab{b}})]{Ponomarev:2022ryp}
Ponomarev, D.
\newblock {Towards higher-spin holography in flat space}.
\newblock {\em J. High Energy Phys.} {\bf 2023}, {\em 01},~084.
%  \href{http://xxx.lanl.gov/abs/2210.04035}{{\normalfont
%  [arXiv:hep-th/2210.04035]}}.
\newblock  [\href{http://dx.doi.org/10.1007/JHEP01(2023)084}{CrossRef}]

\bibitem[Ponomarev(2022)]{Ponomarev:2022atv}
Ponomarev, D.
\newblock {Invariant traces of the flat space chiral higher-spin algebra as
scattering amplitudes}.
\newblock {\em J. High Energy Phys.} {\bf 2022}, {\em 09},~086.
%  \href{http://xxx.lanl.gov/abs/2205.09654}{{\normalfont
%  [arXiv:hep-th/2205.09654]}}.
\newblock  [\href{http://dx.doi.org/10.1007/JHEP09(2022)086}{CrossRef}]

\bibitem[Didenko and Skvortsov(2013)]{Didenko:2012vh}
Didenko, V.E.; Skvortsov, E.D.
\newblock {Towards higher-spin holography in ambient space of any dimension}.
\newblock {\em J. Phys.} {\bf 2013}, {\em A46},~214010. [\href{http://dx.doi.org/10.1088/1751-8113/46/21/214010}{CrossRef}]
%  \href{http://xxx.lanl.gov/abs/1207.6786}{{\normalfont
%  [arXiv:hep-th/1207.6786]}}.






%\newblock  \href{http://xxx.lanl.gov/abs/2205.01567}{{\normalfont
%  [arXiv:hep-th/2205.01567]}}.

\end{thebibliography}


% If authors have biography, please use the format below
%\section*{Short Biography of Authors}
%\bio
%{\raisebox{-0.35cm}{\includegraphics[width=3.5cm,height=5.3cm,clip,keepaspectratio]{Definitions/author1.pdf}}}
%{\textbf{Firstname Lastname} Biography of first author}
%
%\bio
%{\raisebox{-0.35cm}{\includegraphics[width=3.5cm,height=5.3cm,clip,keepaspectratio]{Definitions/author2.jpg}}}
%{\textbf{Firstname Lastname} Biography of second author}



\PublishersNote{}
\end{adjustwidth}
\end{document}

