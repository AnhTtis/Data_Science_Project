\section{Additional Experimental Results}
\subsection{Training for longer steps}
We provide quantitative results on training \ourmethod and instant-NGP for a longer time on the Free dataset (Table~\ref{tab:supp_longer}) and NeRF-360-V2 dataset (Table~\ref{tab:supp_longer_360}). When trained for a longer time, \ourmethod and Instant-NGP can obtain better rendering quality. As shown in Table~\ref{tab:supp_longer}, on the Free dataset, training Instant-NGP for a longer time (15m, 50k steps) does not achieve better rendering quality than \ourmethod (12m, 20k steps). Moreover, increasing the hash table size from $2^{19}$ to $2^{20}$ helps improve the performance of \ourmethod on the Free dataset (\ourmethod$_{\rm 50k-large}$).

\begin{table}[t!]
    \centering
    \begin{tabular}{@{}l@{\hskip 3pt}|@{\hskip 3pt}c@{\hskip 8pt}c@{\hskip 8pt}c@{\hskip 8pt}c@{}}
    \hline
    Method & Tr. time & PSNR{\scriptsize$\uparrow$} & SSIM{\scriptsize$\uparrow$} & LPIPS{\scriptsize(VGG)$\downarrow$} \\
    \hline\hline
    {\small Instant-NGP$_{\rm 20k}$} & 6m & 24.43 & 0.677 & 0.413 \\
    {\small Instant-NGP$_{\rm 50k}$} & 15m & 25.07 & 0.703 & 0.376 \\
    \hline
    {\small \ourmethod$_{\rm 20k}$} & 12m & 26.32 & 0.779 & 0.276 \\
    {\small \ourmethod$_{\rm 50k}$} & 30m & 26.85 & 0.811 & 0.235 \\
    {\small \ourmethod$_{\rm 50k-large}$} & 36m & 27.19 & 0.833 & 0.204 \\
    \hline
    \end{tabular}
    \vspace{-1em}
    \caption[]{{\bf Training Instant-NGP and \ourmethod for longer time on the Free dataset.}}
    \label{tab:supp_longer}
\end{table}

\begin{table}[t!]
    \centering
    \begin{tabular}{@{}l@{\hskip 3pt}|@{\hskip 3pt}c@{\hskip 8pt}c@{\hskip 8pt}c@{\hskip 8pt}c@{}}
    \hline
    Method & Tr. time & PSNR{\scriptsize$\uparrow$} & SSIM{\scriptsize$\uparrow$} & LPIPS{\scriptsize(VGG)$\downarrow$} \\
    \hline\hline
    {\small Instant-NGP$_{\rm 20k}$} & 6m & 26.24 & 0.716 & 0.404 \\
    {\small Instant-NGP$_{\rm 50k}$} & 17m & 26.55 & 0.733 & 0.382 \\
    \hline
    {\small \ourmethod$_{\rm 20k}$} & 14m & 26.39 & 0.746 & 0.361 \\
    {\small \ourmethod$_{\rm 50k}$} & 33m & 26.92 & 0.771 & 0.333 \\
    \hline
    \end{tabular}
    \vspace{-1em}
    \caption[]{{\bf Training Instant-NGP and \ourmethod for longer time on the NeRF-360-V2 dataset.}}
    \label{tab:supp_longer_360}
\end{table}


\begin{table*}[htb]
    \centering
    \begin{tabular}{l|cccccccc|c}
    \hline
    Warping method & Fern & Flower & Fortress & Horns & Leaves & Orchids & Room & Trex & Mean \\
    \hline\hline
    {NDC Warp} & {\bf 24.82} & 27.87 & {\bf 31.22} & {\bf 27.37} & 20.74 & {\bf 19.91} & {\bf 31.75} & 26.77 & {\bf 26.31} \\
    {Inv. Warp} & 24.40 & 27.40 & 30.96 & 27.19 & 20.59 & 19.72 & 31.23 & 26.66 & 26.02 \\
    {Pers. Warp} & 24.71 & {\bf 27.88} & {\bf 31.22} & 27.22 & {\bf 20.84} & 19.82 & 31.64 & {\bf 27.03} & 26.29 \\
    \hline
    \end{tabular}
    \vspace{-1em}
    \caption[]{{\bf Different warping functions on MLP-based NeRFs on the LLFF dataset.}}
    \label{tab:supp_diff_warp_llff}
    \vspace{-0em}
\end{table*}

\begin{figure}[!b]
  \includegraphics[width=\linewidth]{figure_resources/supp_exp_mlp_nerf.pdf} \caption{{\bf Visual comparions among different warping methods on the ``Room'' case of LLFF dataset. }}
  \label{fig:ablation_mlp_nerf}
\end{figure}

\subsection{Compatibility with MLP-based NeRF}
In this section, we provide results of applying perspective warping on MLP-based NeRF.
In this experiment, for each setting, we train a neural radiance field represented by an 8-layer fully-connected MLP for 250K steps, and use different warping functions before feeding the positional encoding to the MLP. For the perspective warping, we do not subdivide the spaces and also only use one single MLP. We also provide results of other warping functions using the same MLP-based NeRF, as shown in Table~\ref{tab:supp_diff_warp_llff}. In the forward-facing setting, our perspective warping (mean PSNR: 26.29) and NDC warping (26.31) perform better than the inverse sphere warping (26.02).
The PSNR of NDC warping is slightly worse than the reported PSNR by the original paper~\cite{MildenhallSTBRN20} due to fewer training steps (ours: 250K steps, official: 1M steps). Fig.~\ref{fig:ablation_mlp_nerf} provides qualitative results on the ``Room'' case of LLFF dataset, and our perspective warping method presents more visual details in the synthesized image, which demonstrates that the proposed perspective warping is compatible with MLP-based NeRF.


\subsection{View extrapolation}
Here we additionally conduct an experiment for view extrapolation on the ``Lego'' case from the NeRF synthetic dataset. In this experiment, we choose images with elevation angles less than 30$^{\circ}$ for training and the others for testing. As shown in Table~\ref{tab:rebuttal_extrapolation} and Fig.~\ref{fig:rebuttal_extrapolation}, the result of our perspective warping method is similar to that using original Euclidean space.

\begin{table}[htb]
    \centering
    \begin{tabular}{@{}l@{\hskip 3pt}|@{\hskip 3pt}c@{\hskip 8pt}c@{\hskip 8pt}c@{\hskip 8pt}c@{}}
    \hline
    Elevation & $[0^\circ, 30^\circ)$ & $[30^\circ, 60^\circ)$ & $[60^\circ, 90^\circ)$ \\
    \hline\hline
    {\small Pers. warp} & 37.63 & 30.19 & 27.35 \\
    {\small w/o warp} & 37.15 & 30.35 & 27.03 \\
    \hline
    \end{tabular}
    \vspace{-1em}
    \caption[]{\bf Results on view extrapolation in the metric of PSNR.}
    \label{tab:rebuttal_extrapolation}
\end{table}

\begin{table}[htb]
    \centering
    \begin{tabular}{@{}l@{\hskip 3pt}|@{\hskip 3pt}c@{\hskip 8pt}c@{\hskip 8pt}c@{\hskip 8pt}c@{}}
    \hline
    Elevation & $[0^\circ, 30^\circ)$ & $[30^\circ, 60^\circ)$ & $[60^\circ, 90^\circ)$ \\
    \hline\hline
    {\small Pers. warp} & 37.63 & 30.19 & 27.35 \\
    {\small w/o warp} & 37.15 & 30.35 & 27.03 \\
    \hline
    \end{tabular}
    \vspace{-1em}
    \caption[]{\bf Results on view extrapolation in the metric of PSNR.}
    \label{tab:rebuttal_extrapolation}
\end{table}


\subsection{Additional ablations}
\begin{table*}[t!]
    \centering
    \begin{tabular}{c|cccccccc}
    \hline
    Setting & Hydrant & Lab & Pillar & Road & Sky & Stair & Grass \\
    \hline\hline
    {w/o warp + Disp. sample} & 23.23 & 25.06 & 26.81 & 25.46 & 25.44 & 27.64 & 21.33 \\
    {Inv. warp + Disp. sample} & 24.05 & 25.43 & 27.09 & 26.24 & 26.27 & 27.66 & 22.34 \\
    {Inv. warp + Exp. sample} & 24.15 & 25.58 & 27.86 & 26.21 & 26.27 & 28.41 & 21.94 \\
    {Pers. warp + Exp. sample} & 24.31 & 25.79 & 28.65 & 26.60 & 26.15 & 29.08 & {\bf 22.89} \\
    {Pers. warp + Pers. sample} & {\bf 24.34} & {\bf 25.92} & {\bf 28.76} & {\bf 26.76} & {\bf 26.41} & {\bf 29.19} & 22.87 \\
    \hline
    \end{tabular}
    \vspace{-1em}
    \caption[]{{\bf Scene breakdown of our ablation studies on the warping and sampling methods.}}
    \label{tab:ablation_break_down}
    \vspace{-0em}
\end{table*}
 

\textbf{Single v.s. multiple hash tables.} We test \ourmethod on the setting with multiple hash tables, i.e., one hash table for each octree node,  with the same budget of parameters as the setting of a single hash table used in the paper. In this setting, the size of each hash table is $L/n_l$, where $L=2^{19}$ is the overall table size and $n_l$ is the number of leaf octree nodes. Fig.~\ref{fig:ablation_hash_table} shows that when using multiple hash tables, the quality degrades clearly compared to the setting of using a single hash table with multiple hash functions. The reason is that using a global hash table has more flexibility in allocating the representation capacity to different regions.

\begin{figure}[!t]
  \includegraphics[width=\linewidth]{figure_resources/supp_exp_hash_table.pdf} \caption{{\bf Visual comparison between using multiple hash tables (a) and single hash table (b) on the ``Sky'' case of Free dataset.}}
  \label{fig:ablation_hash_table}
\end{figure}

\begin{figure}[!b]
  \includegraphics[width=\linewidth]{figure_resources/supp_exp_reg.pdf} \caption{{\bf Visual comparison between w/o regularization losses (a) and w/ regularization losses (b) on the ``Bonsai'' case in NeRF-360-V2 dataset.}}
  \label{fig:ablation_reg}
\end{figure}


\textbf{Effect of regularization losses.} As shown in Fig.~\ref{fig:ablation_reg}, when regularization losses are not used, the foggy artifacts appear and the rendered result is not clear, especially on the regions with pure colors.

\begin{table*}[htb]
    \centering
    \begin{tabular}{l|ccccccc}
    \hline
    Method & Hydrant & Lab & Pillar & Road & Sky & Stair & Grass \\
    \hline\hline
    NeRF++~\cite{ZhangRSK20} & 22.21 & 21.82 & 25.73 & 23.29 & 23.91 & 26.08 & 21.26 \\
    mip-NeRF-360 & {\bf 25.03} & {\bf 26.57} & {\bf 29.22} & {\bf 27.07} & {\bf 26.99} & {\bf 29.79} & {\bf 24.39} \\
    \hline
    mip-NeRF-360 (short)~\cite{BarronMVSH22} & 21.01 & 21.17 & 24.12 & 21.49 & 22.29 & 24.27 & 19.87 \\
    Plenoxels~\cite{YuFTCR22} & 19.82 & 18.12 & 18.74 & 21.31 & 18.22 & 21.41 & 16.28 \\
    DVGO~\cite{SunSC22} & 22.10 & 23.78 & 26.22 & 23.53 & 24.26 & 26.65 & 20.75 \\
    Instant-NGP~\cite{mueller2022instant} & 22.30 & 23.21 & 25.88 & 24.24 & 25.80 & 27.79 & 21.82 \\
    \ourmethod & {\bf 24.34} & {\bf 25.92} & {\bf 28.76} & {\bf 26.76} & {\bf 26.41} & {\bf 29.19} & {\bf 22.87} \\
    \hline
    \end{tabular}
    \vspace{-1em}
    \caption[]{{\bf Scene breakdown on the Free dataset.}}
    \label{tab:compare_free_break_down}
    \vspace{-0em}
\end{table*}

\begin{table*}[htb]
    \centering
    \begin{tabular}{l|cccccccc}
    \hline
    Method & Fern & Flower & Fortress & Horns & Leaves & Orchids & Room & Trex \\
    \hline\hline
    NeRF~\cite{MildenhallSTBRN20} & {\bf 25.17} & 27.40 & 31.16 & 27.45 & 20.92 & {\bf 20.36} & {\bf 32.70} & 26.80 \\
    mip-NeRF~\cite{BarronMTHMS21} & 25.12 & {\bf 27.79} & {\bf 31.42} & {\bf 27.55} & {\bf 20.97} & 20.28 & 32.52 & {\bf 27.16} \\
    \hline
    Plenoxels~\cite{YuFTCR22} & {\bf 25.46} & 27.83 & 31.09 & 27.58 & {\bf 21.41} & 20.24 & 30.22 & 26.48 \\
    TensoRF~\cite{ChenXGYS22} & 25.27 & {\bf 28.60} & 31.36 & {\bf 28.14} & 21.30 & 19.87 & {\bf 32.35} & 26.97 \\
    DVGO~\cite{SunSC22} & 25.08 & 27.62 & 30.44 & 27.59 & 21.00 & {\bf 20.33} & 31.53 & 27.17 \\
    Instant-NGP~\cite{mueller2022instant} & 25.13 & 27.07 & 30.96 & 27.32 & 12.08 & 19.80 & 31.56 & 26.82 \\
    \ourmethod & 25.26 & 27.48 & {\bf 31.49} & 27.84 & 20.68 & 20.10 & 32.23 & {\bf 27.26} \\
    \hline
    \end{tabular}
    \caption[]{{\bf Scene breakdown on the LLFF dataset.}}
    \label{tab:compare_llff_break_down}
\end{table*}

\begin{table*}[htb]
    \centering
    \begin{tabular}{l|ccccccc}
    \hline
    Method & Bicycle & Bonsai & Counter & Garden & Kitchen & Room & Stump \\
    \hline\hline
    NeRF++~\cite{ZhangRSK20} & 22.64 & 29.15 & 26.38 & 24.32 & 27.80 & 28.87 & 24.34 \\
    mip-NeRF-360~\cite{BarronMVSH22} & {\bf 23.99} & {\bf 33.06} & {\bf 29.51} & {\bf 26.10} & {\bf 32.13} & {\bf 31.53} & {\bf 26.27} \\
    \hline
    Plenoxels~\cite{YuFTCR22} & 21.39 & 23.65 & 25.23 & 22.71 & 24.00 &	26.38 &	20.08 \\
    DVGO~\cite{SunSC22} & {\bf 22.12} & 27.80 & 25.76 & 24.34 & 26.00 & 28.33 & 23.59 \\
    Instant-NGP~\cite{mueller2022instant} & 22.08 & {\bf 29.86} & {\bf 26.37} & 24.26 & 28.27 & 28.90 & 23.93 \\
    \ourmethod & 22.11 & 29.65 & 25.36 & {\bf 24.76} & {\bf 28.97} & {\bf 29.30} & {\bf 24.60} \\
    \hline
    \end{tabular}
    \vspace{-1em}
    \caption[]{{\bf Scene breakdown on the NeRF-360-V2 dataset.}}
    \label{tab:compare_360_break_down}
    \vspace{-0em}
\end{table*}


\subsection{Per-scene results}
We provide the per-scene results on the Free dataset, NeRF-360-V2 dataset, and LLFF dataset in Table~\ref{tab:compare_free_break_down}, Table~\ref{tab:compare_360_break_down}, and Table~\ref{tab:compare_llff_break_down} respectively. The results are reported in the metric of PSNR. Table~\ref{tab:ablation_break_down} provides per-scene results on different warping and sampling methods on the Free dataset. We note that the longer the trajectory is (e.g., the ``stair'' and ``grass''), the relatively better performance our perspective warping method with the perspective sampling achieves than the inverse sphere warping method.

\begin{figure*}[htb]
  \includegraphics[width=\linewidth]{figure_resources/supp_exp_compare_free.pdf} \caption{{\bf Additional visual comparions on the Free dataset.}}
  \label{fig:addtional_compare_free}
\end{figure*}



\subsection{More visual results}
We provide more visual comparisons on the Free dataset in 
Fig.~\ref{fig:addtional_compare_free}.
