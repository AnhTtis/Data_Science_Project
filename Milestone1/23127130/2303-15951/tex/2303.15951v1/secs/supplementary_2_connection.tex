\begin{figure}[htb]
  \includegraphics[width=\linewidth]{figure_resources/supp_connection.pdf}
  \caption{Different warp coordinates.}
  \label{fig:rebuttal_connection}
\end{figure}


\section{Connection to NDC warping and Inv. sphere warping} 
In the main paper, we intuitively show the connection of our proposed perspective warping to NDC warping and inverse sphere warping. Here we mathematically 
 analyze the connections using two forward-facing 1D cameras that project 2D points onto their 1D camera plane as shown in Fig.~\ref{fig:rebuttal_connection}. 
The proper perspective warping utilizes the image coordinates of two cameras as the warping coordinates. Thus, the point with coordinate $(0,y)$ in the original Euclidean space will be mapped to $(0, -\frac{\Delta_x}{y})$ in the warping space.
Meanwhile, the coordinates of this point in the NDC space and inverse sphere space are $(0, \frac{f+n}{f-n} - \frac{2fn}{f-n}\cdot\frac{1}{y})$ and $(0, 2 - \frac{r}{y})$ respectively, where $n,f$ are preset near-far depth and $r$ is preset sphere radius. When $\Delta_x=\frac{2fn}{f-n}$ or $\Delta_x=r$, the perspective warping is equivalent to NDC warping or inverse sphere warping with a constant offset. However, theoretically proving such connections in general cases of the 3D space is very complex since it involves sampling points and PCA analysis on sample points.
