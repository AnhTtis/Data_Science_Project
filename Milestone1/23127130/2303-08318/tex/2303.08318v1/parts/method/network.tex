\subsection{Video-tag Network} \label{sec:video_tag_network}
We build a directed heterogeneous video-tag network to incorporate social network and tag ontology, as shown in \autoref{fig:overview}(a).

Our raw data is composed of users, micro-videos, and tags along with two relations: \textit{has\_tag} and \textit{is\_followed\_by}. The latter provides a clue about Behavior Spread. 

We first simplify the raw data by deleting user nodes, and letting their created videos inherit \textit{is\_followed\_by} relations. Concretely, if $\textmd{user}_A$ is followed by $\textmd{user}_B$, then all $\textmd{user}_A$'s videos are followed by $\textmd{user}_B$'s videos. Note that we only preserve the \textit{is\_followed\_by} relations pointing from previously-uploaded videos to newly-uploaded ones, because only old videos can influence new ones. After simplification, we incorporate the \textit{is\_subtopic\_of} relations from the constructed tag ontology to get the video-tag network.

Formally, we denote the micro-video set as $\mathcal{V}_{\textmd{video}}$ and tag set as $\mathcal{V}_{\textmd{tag}}$. They form the node set $\mathcal{V} = \mathcal{V}_{\textmd{video}} \cup \mathcal{V}_{\textmd{tag}}$. For convenience, we denote the three relations \textit{is\_subtopic\_of}, \textit{has\_tag}, and \textit{is\_followed\_by} as $r_1,r_2$, and $r_3$, respectively. Their corresponding relation sets are $\mathcal{E}_{r_1}$, $\mathcal{E}_{r_2}$, $\mathcal{E}_{r_3}$, respectively. They form the edge set $\mathcal{E} = \mathcal{E}_{r_1} \cup \mathcal{E}_{r_2} \cup \mathcal{E}_{r_3}$.