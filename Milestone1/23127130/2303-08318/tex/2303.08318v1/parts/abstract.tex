\begin{abstract}
The last decade has witnessed the proliferation of micro-videos on various user-generated content platforms. 
According to our statistics, around 85.7\% of micro-videos lack annotation. 
In this paper, we focus on annotating micro-videos with tags.
%Existing methods mostly focus on analyzing video content, neglecting users' social influence and tag relation \ADD{construction}.
Existing methods mostly focus on analyzing video content, neglecting users' social influence and tag relation.
%%gantian
%Users' social influence is valuable since they may imitate their influential social network neighbors to create videos with similar tags. 
%Such phenomenon, namely Behavior Spread by sociologists, has not been well-explored for video tagging before.
% Tag relation can be represented by tag ontology. 
Meanwhile, existing tag relation construction methods
%\DEL{can enhance tagging by providing external knowledge} 
suffer from either deficient performance or low tag coverage.
% Existing tag ontology construction methods suffer from either deficient performance or low tag coverage.
To jointly model social influence and tag relation, 
we formulate micro-video tagging as a link prediction problem in a constructed heterogeneous network.
%%
%For ontology construction, we build a tag ontology in a semi-supervised manner.
%Afterwards, we derive a better video and tag representation through Behavior Spread modeling and visual and linguistic knowledge aggregation. 
%Finally, for each micro-video, we estimate its semantic similarity among all candidate tags.
%%
%\TODO{how to construct video-tag network?}
Specifically, the tag relation (represented by tag ontology) is constructed in a semi-supervised manner.
Then, we combine tag relation, video-tag annotation, and user follow relation to build the network.
Afterward, a better video and tag representation are derived through Behavior Spread modeling and visual and linguistic knowledge aggregation. 
Finally, the semantic similarity between each micro-video and all candidate tags is calculated in this video-tag network.
%%
Extensive experiments on industrial datasets of three verticals verify the superiority of our model compared with several state-of-the-art baselines. 
\end{abstract}

% The last decade has witnessed the proliferation of micro-videos in various user-generated content platforms. According to our statistics, around 85.7\% of micro-videos lack annotation. In this paper, we focus on annotating micro-videos with tags. Existing methods mostly focus on analyzing video content, neglecting users' social influence and tag relation. Users' social influence is valuable since they may imitate their influential social network neighbors to create videos with similar tags. Meanwhile, existing tag relation construction methods suffer from either deficient performance or low tag coverage. To jointly model social influence and tag relation, we formulate micro-video tagging as a link prediction problem in a constructed heterogeneous network. Specifically, the tag relation (represented by tag ontology) is constructed in a semi-supervised manner. Then, we combine tag relation, video-tag annotation, and user follow relation to build the network. Afterwards, a better video and tag representation are derived through Behavior Spread modeling and visual and linguistic knowledge aggregation. Finally, the semantic similarity between each micro-video and all candidate tags is calculated in this video-tag network. Extensive experiments on industrial datasets of three verticals verify the superiority of our model compared with several state-of-the-art baselines.