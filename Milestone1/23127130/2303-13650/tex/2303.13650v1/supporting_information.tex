\documentclass[aps,
               prb,
               preprint,
               unsortedaddress,
               superscriptaddress,
               showkeys,
               notitlepage]{revtex4-1}

\usepackage[utf8]{inputenc}
\usepackage[T1]{fontenc}
\usepackage{graphicx}
\usepackage{amsmath, amsfonts, amssymb, mathtools, xcolor}	
\usepackage[breaklinks, linkcolor = black]{hyperref}
\usepackage{bm}					% Bold Greek letters in math mode.
\usepackage{latexsym} 				% Additional symbols. 
\usepackage{tikz}
\usepackage{tabularx}
\usepackage{braket}
\usepackage{diagbox}

%\usepackage[labelfont=bf, format=plain, justification=justified]{caption}
\usepackage{setspace}

\graphicspath{{figures/}}
\newcommand{\red}[1]{\textcolor{red}{#1}}

%\bibliographystyle{}

   \makeatletter
     \renewcommand\@make@capt@title[2]{%
      \@ifx@empty\float@link{\@firstofone}{\expandafter\href\expandafter{\float@link}}%
       {\textbf{#1}}\@caption@fignum@sep#2\quad}%
     \makeatother
 
\makeatletter 
\renewcommand{\fnum@figure}{\textbf{Fig.~\thefigure}}
\makeatother
\renewcommand{\thefigure}{S\arabic{figure}}

\def\theequation{S\arabic{equation}}

\begin{document}

\title{{Supporting Information \\ Reduced absorption due to defect-localized interlayer excitons in \\ transition metal dichalcogenide--graphene heterostructures}}


\author{Daniel \surname{Hernang\'{o}mez-P\'{e}rez}}

%\email[]{daniel.hernangomez@weizmann.ac.il}

\affiliation{Department of Molecular Chemistry and Materials Science, Weizmann Institute of Science, Rehovot 7610001, Israel}


\author{Amir \surname{Kleiner}}

\affiliation{Department of Molecular Chemistry and Materials Science, Weizmann Institute of Science, Rehovot 7610001, Israel}

\author{Sivan Refaely-Abramson}

%\email[]{sivan.refaely-abramson@weizmann.ac.il}

\affiliation{Department of Molecular Chemistry and Materials Science, Weizmann Institute of Science, Rehovot 7610001, Israel}

\keywords{2D materials, transition-metal dichalcogenides, heterostructures, defects, graphene, excitons}

\maketitle

\begin{center}
 \vspace{-.6cm} 
 Email: daniel.hernangomez@weizmann.ac.il \\ \hspace{1.8cm} sivan.refaely-abramson@weizmann.ac.il
 \end{center}

\tableofcontents
\clearpage


\section{Computational details and methods}\label{app:computational}

%
\subparagraph{Geometry.} As a starting point for the supercell optimization in the presence of the vacancy (see details below), we consider the geometry of a pristine heterobilayer  with a lattice parameter whose length corresponds to {an average} nearest-neighbor metal-metal distance (equal to the in-plane monolayer TMDC lattice constant) of $\bar{d}_{\textnormal{X--X}} = 3.15 $\,\AA \, both for MoS\textsubscript{2} and WS\textsubscript{2}. 
%
This value is almost equal to the experimental monolayer lattice parameter of both TMDC (MoS\textsubscript{2}; 3.15 \AA\, \onlinecite{Nicklow1975} and WS\textsubscript{2}; 3.153 \AA\, \onlinecite{Schutte1987}).
%
As stated in the main text, the supercell is composed of $4\times4$ TMDC unit cells and $5\times 5$ graphene unit cells. Therefore, it is made of $97$ atoms and possesses a rhomboedral shape with in-plane lattice vectors of length $|\mathbf{R}_{1,2}| \simeq 12.6$ \AA\, (see Fig. \ref{f5}).
%
The vacuum distance between the periodic replicas in the out-of-plane direction was taken to be $\sim 10$ \AA\, and the average interlayer distance within the supercell was $\bar{d}_\textnormal{inter} = 3.43$ \AA\, {both for Mo and W}. 


\begin{figure}[h]
   \includegraphics[width=0.6\linewidth]{f4.pdf}

%
    \caption{Top view of the WS\textsubscript{2}--Gr supercell. Four supercells are shown, each supercell forms a parallelepiped whose lateral boundaries are given by the straight red lines (the in-plane supercell lattice vectors are labeled as $\mathbf{R}_1$, $\mathbf{R}_2$). The chalcogen vacancy, located in the TMDC layer on the opposite side of the graphene layer, is indicated by a red triangle.
    }\label{f5} 
\end{figure}
% DFT
\subparagraph{Density functional theory.} The DFT calculations were performed employing the implementation of DFT of \textsc{Quantum Espresso}. \cite{Giannozzi2009, Giannozzi2017} %
We used the non-empirical PBE generalized gradient approximation for the exchange--correlation functional \cite{Perdew1996}.
%
We employed a plane-wave basis set and included spin-orbit interaction by means of full relativistic norm-conserving pseudopotentials \cite{Dojo2019}. We considered a basis cut-off of $50$ Ry for both TMDC--Gr interfaces.
%
The self-consistent charge density was converged within a $6 \times 6 \times 1$ \textbf{k}-grid, with Fermi-Dirac smearing of $10^{-5}$ Ry for fractional occupations. The calculation was considered to be converged only if the total energy difference between consecutive iterations within the self-consistent field cycle was smaller than the threshold value of $10^{-9}$ Ry.

The supercells were initially preoptimized with VASP \cite{Kresse1996} in the absence of chalcogen vacancies. For the exchange-correlation functional a local density approximation (LDA) was employed, with a basis set energy cut-off of $600$ eV. The self-consistent charge density for the geometry relaxations was converged in a $6\times 6 \times 1$ \textbf{k}-grid as well. 
%
The supercells were subsequently relaxed, fixing only the position of the supercell lattice vectors and optimizing the position of the TMDC atoms within the supercell after the chalcogen atom was removed. 
%
The position of the atoms was relaxed until all components of the forces were smaller than a threshold value of $10^{-3}$ Ry/$a$\textsubscript{0}.
%
This second optimization was done with \textsc{Quantum Espresso}, using PBE and the van-der Waals corrected functional \texttt{vdw-df-09} \cite{Thonhauser2007, Troy2009, Berland2015} to properly account for changes in the interlayer separation.

% GW-BSE
\subparagraph{GW.} Using the DFT wavefunctions and energies as a starting point, we computed the corrected quasi-particle energy spectrum by performing a one-shot non-self-consistent GW calculation (G\textsubscript{0}W\textsubscript{0}). Our GW calculations were performed with the package Berkeley{GW}, including spin-orbit interaction \cite{Deslippe2012, Rohlfing2000, Wu2020, Louie2022}.
%
The dielectric function was obtained with the generalized plasmon-pole model of Hybertsen-Louie \cite{Hybertsen1986}.
%
We employed a cut-off of $5$ Ry in the dielectric screening and a total of \textcolor{black}{$2499$} states for the summation over the occupied and unoccupied states.
%
We used a non-uniform neck subsampling scheme to sample the Brillouin zone close to $|\mathbf{q}| \rightarrow 0$ and speed up  the convergence with respect to the 
\textbf{k}-grid sampling \cite{Qiu2017}. Within this scheme, we considered a $6\times 6 \times 1$ uniform \textbf{q}-grid and $10$ additional \textbf{q}-points around $\mathbf{q}= \mathbf{0}$.
%
A truncated Coulomb interaction was considered in the perpendicular direction to the heterostructure to prevent spurious interactions between the periodic replicas in this direction \cite{Ismail2006}. \textcolor{black}{This set of parameters yields converged quasiparticle gaps within 100 meV.}


\subparagraph{BSE.}
%
%
To study the excitonic features, we solved the Bethe-Salpether equation (BSE) \cite{Rohlfing1998, Rohlfing2000} 
\begin{equation}
    (E_{c\mathbf{k}} - E_{v\mathbf{k}}) A_{vc\mathbf{k}}^S + \sum_{v'c' \mathbf{k}'} K^{\textnormal{eh}}_{vc\mathbf{k}; v'c'\mathbf{k}'} A_{v'c'\mathbf{k}'}^S  = \Omega_S A_{vc\mathbf{k}}^S,\label{e3}
\end{equation}
where $E_{c\mathbf{k}}$ (resp. $E_{v\mathbf{k}}$) are the quasiparticle energies of the conduction (resp. valence) bands, $K^{\textnormal{eh}}_{vc\mathbf{k}; v'c'\mathbf{k}'}  = \langle vc \mathbf{k}|\hat{K}^{\textnormal{eh}}|v'c' \mathbf{k}' \rangle$ are the matrix elements of the electron-hole interaction kernel, defined from the addition of an attractive screened direct and a repulsive bare exchange Coulomb contributions, $\Omega_S$ is the exciton energy and $A^S_{vc\mathbf{k}}$ is the amplitude of the exciton state $|\Psi^S \rangle$.
%
This equation sets an eigenvalue problem, $\hat{H}^\textnormal{BSE}|\Psi^S \rangle = \Omega_S |\Psi^S \rangle$, where the matrix elements of the BSE Hamiltonian in the electron-hole basis are given by
\begin{equation}
    {H}^\textnormal{BSE}_{cv\mathbf{k};c'v'\mathbf{k}'} = (E_{c\mathbf{k}} - E_{v\mathbf{k}}) \delta_{c,c'}\delta_{v,v'}\delta_{\mathbf{k},\mathbf{k}'} + K^{\textnormal{eh}}_{vc\mathbf{k}; v'c'\mathbf{k}'}.
\end{equation}
This representation of the BSE assumes that the (real-space) direct exciton wavefunction is described as the coherent superposition of electrons and holes at each $\mathbf{k}$-point,
\begin{equation}
   \langle \mathbf{r}_e, \mathbf{r}_h | \Psi^S \rangle = \sum_{vc\mathbf{k}} A^S_{vc\mathbf{k}}  \psi^\ast_{v\mathbf{k}}(\mathbf{r}_h) \psi_{c\mathbf{k}}(\mathbf{r}_e), 
\end{equation}
with $\psi_{c\mathbf{k}}(\mathbf{r}_e)$ being the spinor wavefunction describing the electron at position $\mathbf{r}_e$ with conduction band quantum number $c$ and crystal momentum $\mathbf{k}$ (correspondingly, $\psi_{v\mathbf{k}}(\mathbf{r}_h)$ is the spinor wavefunction describing a hole at position $\mathbf{r}_h$ and characterized by the valence band quantum number $v$ and same crystal momentum $\mathbf{k}$).

%%%%%%%%%%%%%%%%%%%%%%%%%%%%%%%%%%%%%%%%%%%%%%%%%%%%%%%%%%%
Eq. \eqref{e3} was solved using the Berkeley{GW} package \cite{Deslippe2012, Rohlfing2000, Wu2020}. 
%
The matrix elements were computed on a Monkhorst-Pack  $9 \times 9 \times 1$ \textbf{k}-grid and the result interpolated to a uniform \textcolor{black}{$27 \times 27 \times 1$ \textbf{k}-grid} that we employ in any subsequent analysis and calculation.
%
We employed the Tamm-Dancoff approximation and evaluated the Coulomb interaction kernel for all possible transitions between pairs of bands $(n,m) \rightarrow (n', m')$, with an energy cut-off of $5$ Ry for the dielectric matrix within the electron-hole kernel matrix elements.
%
We considered for the main paper a total of \textcolor{black}{$28$ bands ($14$ valence and $14$ conduction bands)} in the absorption calculations, which include both the defect bands as well as all the relevant low energy pristine valence and conduction bands of the heterobilayer.
%
%
%
\textcolor{black}{These parameters converged the calculated excitonic spectra within 100 meV for the defect-dominated subgap features and below 10 meV for the most prominent absorption resonances in the region where intralayer non-defect TMDC excitons become more relevant.}
%
In the absorption calculation, we also avoid the heavy calculation of the $\mathbf{q}$-shifted wavefunctions on the interpolation grid by evaluation of the matrix elements of the velocity operator, $\hat{\mathbf{v}}$, instead of the momentum, $\mathbf{p} = \mathfrak{i} \nabla$. This involves neglecting terms in the sums proportional to $|\langle 0 |[\hat{V}_{\textnormal{ps}}, \hat{\mathbf{r}}]| S \rangle|^2$, where $\hat{V}_{\textnormal{ps}}$ is the non-local part of the pseudopotential \cite{Rohlfing2000, Deslippe2012}. Including the non-local terms has been shown to not qualitatively change the shape of water X-ray absorption spectrum \cite{Qiu2022}.


\subparagraph{Projected density of states.} We compute the layer contribution of each band to a given exciton state, first by obtaining the \textbf{k}-projected density of states (DoS) of each layer, $l = \{\textnormal{WS}_2, \textnormal{Gr}\}$ from the \textbf{k}-resolved projected DoS
\begin{equation}\label{e1}
    g^{l}_{n\mathbf{k}}(E) = \sum_{\{i_A,A\} \in l} |\langle \phi_{i_A}^A| \psi_{n\mathbf{k}} \rangle|^2 \delta(E - E_{n\mathbf{k}}),
\end{equation}
where $|\psi_{n\mathbf{k}} \rangle $ and $E_{n\mathbf{k}}$ are the Kohn-Sham states and energies and the sum runs over atoms $A$ and orbitals $i_A$ of the corresponding layer $l$.
%
We further normalize this quantity for each layer as $\tilde{g}(E) = g(E) / \textnormal{max}[g(E)]$ so that 
\begin{equation}\label{e2}
\tilde{g}_{n\mathbf{k}}(E) = \tilde{g}^{\textnormal{WS}_2}_{n\mathbf{k}}(E) - \tilde{g}^{\textnormal{Gr}}_{n\mathbf{k}}(E),
\end{equation}
 is defined in the range $[-1,1]$. This way, $\tilde{g}_{n\mathbf{k}}(E) = -1$ corresponds exclusively to graphene contribution and $\tilde{g}_{n\mathbf{k}}(E) = 1$  exclusively to TMDC contribution.

 \subparagraph{Heterostructure decomposition.} 
 We employ Eq. \eqref{e2} to display the color of the band contributions to the exciton decomposition  as well as the absorbance decomposition into intralayer and interlayer parts in the main paper. In particular, for Fig. 2 and Fig. \textbf{S5}, the contributions to a given conduction band are obtained as $\sum_{vc\mathbf{k}} \tilde{g}_{v\mathbf{k}}|A^S_{vc\mathbf{k}}|^2$ while the contributions to a given valence band result from $\sum_{vc\mathbf{k}} \tilde{g}_{c\mathbf{k}}|A^S_{vc\mathbf{k}}|^2$. 
 %
 The \textbf{k}-resolved decomposition in Fig. 3(b) of the main paper (see also Figs. \ref{f18}, \ref{f7}, \ref{f8}) are obtained using the same expressions but without the summation over the crystal momentum.


\subparagraph{Absorbance.} 

From the absorption, we compute the associated absorbance using\cite{Li2009}
\begin{equation}
A(\omega) = \dfrac{\omega L_z}{c}  \epsilon_2(\omega),
\end{equation}
where $\omega$ is the photon frequency, $c$  the speed of light, $L_z$  the distance between the heterostructure and its periodic replicas and $\epsilon_2(\omega)$  the imaginary part of the dielectric function.

\subparagraph{Exciton binding energy.} 
% 
The excitonic binding energies are calculated from the difference between the expectation values of the diagonal and the full BSE Hamiltonian,
\begin{align}
    E^S_\textnormal{bind} &=
    \langle \Psi^S | \hat{H}^{\textnormal{BSE}} - \hat{K}^{\textnormal{eh}} | \Psi^S \rangle - \langle \Psi^S | \hat{H}^{\textnormal{BSE}} | \Psi^S \rangle, \notag \\
    &=\sum_{vc\mathbf{k}}|A^S_{vc\mathbf{k}}|^2 (E_{c\mathbf{k}} - E_{v\mathbf{k}}) - \Omega_S,\label{eq:binding}
\end{align}
where $\hat{H}^{\textnormal{BSE}}$ is the BSE Hamiltonian and  $\hat{K}^{\textnormal{eh}}$ the electron-hole interaction kernel; $E_{\alpha\mathbf{k}}$, the quasi-particle bands; and $\Omega_S$, the exciton energy.

\subparagraph{Intrinsic decay rate.}
To compute the intrinsic decay rate of the zero-momentum excitons we follow Refs. \onlinecite{Spataru2005, Palummo2015, Bernardi2018}
 \begin{equation}
      \gamma_S = \dfrac{4 \pi e^2}{m^2 c} \dfrac{\mu_S}{A_{c} \Omega_S},
 \end{equation}
 where $m$ is the electron mass, $\mu_S$ is the BSE oscillator strength of the exciton with energy $\Omega_S$ and $A_c$ the area of the supercell.

\subparagraph{Non-defected heterostructure.} The absorbance data for the non-defected heterobilayer was taken from the forthcoming publication \onlinecite{Kleiner2023}, where a GW-BSE calculation in a relaxed supercell of the same size but without vacancy was performed.

\section{Additional results}

\subparagraph{Geometry optimization.}
%
After relaxation in the presence of the vacancy, we find that close to the defect position, the nearest-neighbor metal-metal distance decreases to $d_{\textnormal{W--W}} = 3.02 $\,\AA\, and $d_{\textnormal{Mo--Mo}} = 3.05 $\,\AA. This corresponds to a reduction of about $3-4$\% in units of the TMDC monolayer lattice constant, $\bar{d}_{\textnormal{X--X}}$. Because the supercell volume is constant during the relaxation, the TMDC layer is strained in the vicinity of the defect where the distance to the next-nearest-neighbor metallic atoms increases up to $\simeq 3.2$ \AA\, for the functional employed here. Similar behavior is also observed for the sulfur atoms surrounding the vacant site, which rigidly follow the motion of the metallic atoms.
%
Geometry optimization also reduces the average interlayer distance in the heterostructure by around $3-4\%$. 
% 
In particular, we find a reduction of the interlayer distance to $\bar{d}_\textnormal{inter} = 3.29$ \AA\, in the case of MoS\textsubscript{2}--Gr and $\bar{d}_\textnormal{inter} = 3.31$ \AA\, for WS\textsubscript{2}--Gr for our relaxation criteria and our employed van der Waals scheme.

\subparagraph{DFT bandstructures.}
For completeness, we show the bandstructures obtained using the DFT relaxed structures in the presence of spin-orbit interaction in Fig. \ref{f17}.

\begin{figure}
   \includegraphics[width=1.0\linewidth]{figures/f17.pdf}

     %
    \caption{Band structures of the TMDC--Gr heterostructures in the presence of spin-orbit interaction computed along the $\bar{\textnormal{M}}-\bar{\textnormal{K}}-\bar{\Gamma}$ path in the supercell (see inset) (a) WS\textsubscript{2}--Gr heterobilayer (b) MoS\textsubscript{2}--Gr heterobilayer.}\label{f17} 
\end{figure}

\subparagraph{GW.}
%
As stated in the main text, our GW results qualitatively follow the DFT band structures discussed in Ref. \onlinecite{Hernangomez2023}. We summarize here for completeness the main features. In essence, the band structures of graphene and the TMDC appear mostly superimposed, with the graphene Dirac cone centered at the $\bar{\textnormal{K}}$ (and $\bar{\textnormal{K}}$') points. The Dirac point sets the charge neutrality point %
within the pristine TMDC band gap. 
%
The combination of the missing chalcogen atom and the symmetry of the host lattice results in four empty in-gap and two occupied spin-orbit split bands. %
In addition, the reduction of the original TMDC symmetry due to graphene adsorption, and the residual defect-defect interaction that results from the lattice mismatch between layers within the supercell as well as the supercell size (\textit{i.e.} the defect density) make these bands non-degenerate and weakly dispersive. The breaking of the degeneracy is stronger in the vicinity of the $\bar{\textnormal{K}}$ and $\bar{\textnormal{K}}'$ points.
%
Defect states also hybridize with the graphene states in certain regions of the supercell Brillouin zone. In particular, the interlayer hybridization is relatively important between the defect and the graphene bands in a ring around the $\bar{\textnormal{K}}$ and $\bar{\textnormal{K}}'$ points. This interlayer hybridization was already  measured \cite{Batzill2015, Pierucci2016} and predicted \cite{Jain2018} for occupied bands in MoS\textsubscript{2}--Gr far from the charge neutrality point.

%%%%%%%%%%%%%%%%%%%%%%%%%%%%%%%%%%%%%%%%%%%%%%%%%
Graphene adsorption and geometry relaxation within the supercell breaks the original lattice symmetry \cite{Hernangomez2023}, lifting the degeneracy of the TMDC conduction states at the $\Lambda$ and K \textbf{k}-points, both folded into the point $\bar{\textnormal{K}}$ of the supercell Brillouin zone due to lattice commensuration \cite{Kleiner2023}. For WS\textsubscript{2}--Gr, this results in the lowest conduction band having $\Lambda$ nature with K states being at higher energy 
%
already at the DFT level. 
%
The identification is performed from the pseudo-charge density, $|\Psi|^2$ at $\mathbf{k} = \bar{\textnormal{K}}$, which provides the orbital contribution, comparing  %
to the pristine TMDC states (see also Ref. \onlinecite{Kleiner2023}). 
%
Screening affects more strongly the more delocalized K states (essentially due to the defect effect), shifting them higher in energy compared to the folded $\Lambda$ states (by $\sim 0.25$ eV), the latter determining thus the pristine band gap. %
For MoS\textsubscript{2}--Gr, %
already at the DFT level the K bands are below the $\Lambda$ bands by $\sim 0.1$ eV and this situation reverses at the GW level, where the K bands appear $\sim 0.15$ eV above. %
This again gives a relative shift of $\sim 0.25$ eV, suggesting that in both heterostructures the levels shifting results from the graphene dielectric screening.

% 
The screening and exchange renormalize also the graphene Fermi velocity associated to the slope of the Dirac cone.
%
In particular, we obtain an increase of the Fermi velocity of $\sim 34$\% for WS\textsubscript{2}--Gr and $\sim 41$\% for  MoS\textsubscript{2}--Gr, consistent with a reported increase of the Fermi velocity of $\sim 34$\% in Ref. \onlinecite{Li2009} and slightly larger than the $\sim 18\%$ reported in Ref. \onlinecite{Olevano2008} (both calculations for the isolated graphene monolayer).
% 
In addition, we find that for  WS\textsubscript{2}--Gr (resp. MoS\textsubscript{2}--Gr) the Fermi energy shifts from $3.80$ eV in DFT to $4.68$ eV in GW with respect to the vacuum level (resp. $3.60$ eV to $4.50$ eV).
%

%%%%%%%%%%%%%%%%%%%%%%%%%%%%%%%%%%%%%%%%%%%%%%%%%%%%%%%%%%%%%%%%%%%%%%%%%
\section{Absorption convergence checks}

\begin{figure}
   \includegraphics[width=1.0\linewidth]{figures/f9.pdf}

     %
    \caption{Brillouin zone \textbf{k}-grid convergence of the absorption spectrum for different supercell sampling. (a) WS\textsubscript{2}--Gr heterobilayer (b) MoS\textsubscript{2}--Gr heterobilayer.}\label{f1} 
\end{figure}

\begin{figure}
   \includegraphics[width=0.6\linewidth]{figures/f16.pdf}  

     %
    \caption{Convergence of the absorption spectrum of WS\textsubscript{2}--Gr heterobilayer with the number of bands inclued in the BSE calculation.}\label{f16} 
\end{figure}

Ensuring a qualitatively $\mathbf{k}$-grid converged absorption is crucial for understanding the defect-induced phenomena presented in this paper.
%
In Fig. \ref{f1}, we show the absorption spectrum for different supercell Brillouin zone uniform \textbf{k}-grid samplings of the interpolated absorption grid for (a) WS\textsubscript{2}--Gr (b) MoS\textsubscript{2}--Gr .
%
We find that the absorption spectra are quantitatively well converged %for the computationally available grids 
in the visible range, especially above optical energies $\gtrsim 2 $ eV for WS\textsubscript{2}--Gr and $\gtrsim 1.75 $ eV for MoS\textsubscript{2}--Gr. In this energy range, convergence is achieved already with a  $24 \times 24 \times 1$ \textbf{k}-grid up to $\sim 10$ meV. However, the position and height of the absorption resonances in the infrared range, especially for energies roughly below $0.3-0.5$ eV, are not yet fully converged. This region, which  is also known to be strongly dominated by intraband graphene resonances \cite{Li2009},  requires very dense \textbf{k}-grids for smoothness and quantitative convergence. 
%
In the intermediate optical range, up to $\sim 1.6-1.7$ eV, the results are qualitatively converged, but the main absorption features are still quantitatively dependent on the employed \textbf{k}-grid. At   energies between $\sim 0.5-1.6$ eV, the results qualitatively agree between different \textbf{k}-grids, and, importantly, the height of the absorption resonances is stable. The importance of this observation for defect-based sub-gap features and their expected survival in the limit of dense \textbf{k}-sampling is discussed the main text.
% 
Finally, we also ensured that the excitonic features discussed in the main paper are converged in the number of bands included in the BSE calculation, see \textit{e.g.} Fig. \ref{f16} 


\section{Absorbance spectra and defect resonances in MoS\textsubscript{2}-graphene}\label{sec:MoS2}

 \begin{figure}
    \includegraphics[width=0.65\linewidth]{f12.pdf} 
%
     \caption{Absorbance and exciton contributions for the defected MoS\textsubscript{2}--Gr heterobilayer. (Top) Absorbance calculated along one of the main in-plane polarization directions, as well as its decomposition into graphene and MoS\textsubscript{2} contributions (interlayer contributions are read from the difference of the three traces). The dashed horizontal black line marks the $2.4\%$ universal limit of graphene absorbance at infrared energies.
    The shaded box represents the estimated range for which we expect a smooth and monotonic spectrum dominated by graphene (instead of resonances that result from finite \textbf{k}-grid sampling). The vertical dotted line denotes the optical ranges below which excitons transition from being dominated by defect-graphene sub-gap transitions to have transitions which involve pristine TMDC bands.
    %
    (Bottom) For each exciton composing the absorbance resonances, we represent the contribution of each electron and hole bands. Each dot corresponds to the band contribution to a given exciton summed over all $\mathbf{k}$ points (only bright contributions whose oscillator strength are $> 5$ a.u. are shown). For reasons of clarity, all dots with value $\geq 10^3$ a.u. have the same area. 
    %
    The color code corresponds to the layer composition  of each contribution and the dotted box marks the position of the transitions towards graphene-defect empty bands.  
     }\label{f3} 
 \end{figure}
As mentioned in the main text, the strong mixing that occurs in  WS\textsubscript{2}--Gr is also qualitatively similar for the case of the MoS\textsubscript{2}--Gr interface.  We show in Fig. \ref{f3}, top panel, the absorbance spectrum, its decomposition into intralayer graphene, intralayer TMDC and interlayer contributions as well as the graphene universal absorbance limit at infrared energies (dashed horizontal line). Here, however, the impact of the defect bands in the absorbance is more dramatic since the defect bands are closer to the Dirac point and thus interlayer contributions affect the absorbance even at lower energies. Consequently, for MoS\textsubscript{2}--Gr  the absorbance spectrum is very different to that obtained for this TMDC  with chalcogen vacancies in the absence of  graphene layer \cite{Mitterreiter2021, Amit2022}. 
%
In Fig. \ref{f3}, bottom panel, we show the band contribution of each  exciton. Note that, as compared to WS\textsubscript{2}--Gr, here the four spin-orbit split defect bands can be identified clearly as being below the graphene Dirac cone already at optical energies $\sim 0.5$ eV. 



%%%%%%%%%%%%%%%%%%%%%%%%%%%%%%%%%%%%%%%%%%%%%%%%%%%%%%%%%%%%%%%
\section{Wavefunction densities of relevant conduction bands}\label{app:wfn}

\begin{figure*}
   \includegraphics[width=1.0\linewidth]{f7.pdf}
   %
    \caption{(a) Top view of the Kohn-Sham pseudo-charge density, $|\Psi_{n,\mathbf{k}}(\mathbf{r})|^2$, for the conduction band states, derived from the monolayer K point. These densities are evaluated at the point $\mathbf{k} = \bar{\textnormal{K}}$ of the supercell Brillouin zone of the defected WS\textsubscript{2}--Gr heterobilayer.
    (b) Same as in (a) but for the states derived from the folded $\Lambda$ point, at $\mathbf{k} = \bar{\textnormal{K}}$. (c) Same as in (a) but for the two defect states with different ``orbital'' quantum number. \cite{Hernangomez2023}
    }\label{f4} 
\end{figure*}

In order to understand how the defect has such a strong impact on the absorption features, it is instructive to look at the wavefunction densities of relevant states that participate in the transitions. In Fig. \ref{f4}, we show the pseudo-charge density for three sets of bands at the relevant $\bar{\textnormal{K}}$ point of the supercell Brillouin zone: (a) states with K nature, \textit{i.e.} which have orbital contribution resembling the orbitals found in the monolayer at the K point; (b) states originally coming from the monolayer $\Lambda$ point, folded into $\bar{\textnormal{K}}$, and (c) defect states. 
%
While the states with K nature are weakly affected by the missing atom forming the vacancy, the perturbation due to the missing atom  affects very strongly the $\Lambda$-like states, which have a new orbital pattern that varies at longer length scales (nanometer scale). As such, we expect these states to react differently to the dielectric screening once this is taken into account. This difference can explain therefore, as stated in the main text, why the pristine TMDC band-gap at the GW level is determined by the states shown in Fig. \ref{f4} (b) and not in (a). For comparison, we display in (c) the pseudo-densities of two defect states. These states present an orbital signature consistent with that seen in panel (b), therefore, we conclude that the behavior and shape of the $\Lambda$-derived states is exclusively determined by the vacancy.

\clearpage

\section{Defect-induced exciton hybridization}



\begin{figure}[h]
   \includegraphics[width=0.575\linewidth]{f18.pdf} 

%
    \caption{Brillouin zone exciton distribution plotted for all the WS\textsubscript{2}--Gr excitons within an energy window of $\pm 5 $ meV marked in the absorption spectrum by \textcircled{1} (centered at 2.19 eV), \textcircled{2} (centered at 2.4 eV) and \textcircled{3} (centered at 2.7 eV). Even a small energy window close to the center of the excitation peak shows larger hybridization between the WS\textsubscript{2} and the graphene layers.
    }\label{f18} 
\end{figure}

\begin{figure}[h]
   \includegraphics[width=0.575\linewidth]{f19.pdf} 

%
    \caption{Sketch of most prominent transitions and transition band diagram for the absorption peaks in Fig. \ref{f18}. The excitons have all been added up within an energy window of $\pm 5 $ meV.
    }\label{f19} 
\end{figure}



\clearpage 

\section{Additional figures}\label{app:add_figs}
\vspace{-0.2cm}
\begin{figure}[h]
   \includegraphics[width=0.5\linewidth]{f6.pdf}

%
    \caption{Exciton energies, $\Omega_S$, represented as a function of the binding energy, $E_\textnormal{bind}$, for the excitons in the MoS\textsubscript{2}--Gr heterobilayer.
    %
    The binding energy is computed using Eq. \eqref{eq:binding}. We only show the excitons with $E_\textnormal{bind} > 2.5$ meV (around $\sim 14000$ out of $142 884$ excitons for this \textbf{k}-grid sampling and number of bands). The size of each dot is proportional to the oscillator strength (rescaled by a factor of two for visibility).
    }\label{f6} 
\end{figure}

\begin{figure}[h]
   \includegraphics[width=0.575\linewidth]{f14.pdf} 
    \caption{Brillouin zone exciton distribution for the two most bound excitons  in Fig. \ref{f6}. The Brillouin zone in the top row displays $\sum_{v}|A^S_{cv\mathbf{k}}|^2$, while the lower row shows $\sum_{c}|A^S_{cv\mathbf{k}}|^2$. These two excitons present  graphene-defect transitions and are largely delocalized in \textbf{k}-space.
    }\label{f7} 
\end{figure}

\begin{figure}[h]
   \includegraphics[width=0.6\linewidth]{f13.pdf} 
    \caption{Same as in Fig. \ref{f7} but for two graphene dominated excitons. Both excitons are mostly localized in the vicinity of the $\bar{\textnormal{K}}$ valleys. These excitons have a large transition dipole, with the value of $\mu_S$ being $1.5 \cdot 10^2$ a.u. and $1.2 \cdot 10^2$ a.u. respectively, but small binding energy.
    }\label{f8} 
\end{figure}

\begin{figure}
   \includegraphics[width=0.6\linewidth]{figures/f11.pdf}

   %
     %
    \caption{Intrinsic radiative lifetimes at low temperatures for the grey excitons with binding energy larger than $25$ meV, both for WS\textsubscript{2}--Gr (blue circles) and MoS\textsubscript{2}--Gr (red squares). The size of the points is proportional to the oscillator strength, rescaled for clarity by a factor of 20. 
    %
    }\label{f11} 
\end{figure}

\begin{figure}[h]
   \includegraphics[width=0.6\linewidth]{f15.pdf} 
    \caption{Intrinsic radiative lifetimes at low temperatures for the bright excitons in the WS\textsubscript{2}--Gr heterobilayer with and without defects (data for the pristine heterobilayer taken from Ref. \onlinecite{Kleiner2023}). The size of the symbols is proportional to the oscillator strength $\mu_S$, which is chosen to be larger than the threshold value of $10^1$ a.u.. For clarity,  we fix the value of $\mu_S$ to be $10^{3}$ a.u if larger or equal. The grey shaded area corresponds to the region of the spectrum strongly dominated by graphene.
    %
    We observe that the heterostructure without the vacancy has brighter excitons with systematically shorter lifetimes and larger oscillator strengths in the visible region, where the defected structure shows exciton quenching. 
    %
    }\label{f9} 
\end{figure}

\clearpage
\bibliography{biblio}

\end{document}
