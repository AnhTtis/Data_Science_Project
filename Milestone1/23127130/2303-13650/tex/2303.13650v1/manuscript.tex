\documentclass[%
      aps,
      prb, %jcp,
      % nofootinbib,
      % bmf,
      % sd,
      % rsi,
      % reprint,
      twocolumn,
      % preprint,
      superscriptaddress,
      showkeys]{revtex4-1}
      
\usepackage[utf8]{inputenc}
\usepackage[T1]{fontenc}
\usepackage{dcolumn}
\usepackage{bm}					% Bold Greek letters in math mode.
\usepackage{graphicx}
\usepackage{amsmath, amsfonts, amssymb, mathtools}	
\usepackage{latexsym} 				% Additional symbols. 
\usepackage{xcolor}
\usepackage[caption = false]{subfig}
\usepackage[breaklinks, linkcolor = black]{hyperref}
\usepackage{tikz}
\usepackage{tabularx}
\usepackage{braket}
\usepackage{diagbox}
%\usepackage{ulem}
\usepackage[normalem]{ulem}
\usepackage{textcomp}
\usepackage{soul}
%\usepackage[mathlines]{lineno} % Enable numbering of text and display math
%\usepackage{mathptmx}
%\usepackage{blindtext}
\usepackage{booktabs,eqparbox}
% \usepackage{duckuments}
% \usepackage{coffee4}
\graphicspath{{figures/}}

%\graphicspath{{figures/}}
\newcommand{\red}[1]{\textcolor{red}{#1}}

%\makeatletter
%\let\@fnsymbol\@fnsymbol@latex
%\@booleanfalse\altaffilletter@sw
%\makeatother

\begin{document}
\title{Reduced absorption due to defect-localized interlayer excitons in \\ transition metal dichalcogenide--graphene heterostructures}


%%%%%%%%%%%%%%%%%%%%%%%%%%%%%%%%%%%%%%%%%%%%%%%%%%%
% 
 \author{Daniel \surname{Hernang\'{o}mez-P\'{e}rez}}

\email[]{daniel.hernangomez@weizmann.ac.il}

\affiliation{Department of Molecular Chemistry and Materials Science, Weizmann Institute of Science, Rehovot 7610001, Israel}

\author{Amir \surname{Kleiner}}

\affiliation{Department of Molecular Chemistry and Materials Science, Weizmann Institute of Science, Rehovot 7610001, Israel}

\author{Sivan \surname{Refaely-Abramson}}

\email[]{sivan.refaely-abramson@weizmann.ac.il}

\affiliation{Department of Molecular Chemistry and Materials Science, Weizmann Institute of Science, Rehovot 7610001, Israel}


\date{\today}

\begin{abstract}
Associating the presence of atomic vacancies to excited-state transport phenomena in two dimensional semiconductors is of emerging interest, and demands detailed understanding of the involved exciton transitions. Here we study the effect of such defects on the electronic and optical properties of WS\textsubscript{2}--graphene and MoS\textsubscript{2}--graphene van der Waals heterobilayers by employing many-body perturbation theory. We find that the combination of chalcogen defects and graphene adsorption onto the transition metal dichalcogenide layer can radically alter the optical properties of the heterobilayer, due to a combination of dielectric screening, the impact of the missing chalcogen atoms in the intralayer and interlayer optical transitions, and the different nature of each layer. By analyzing the intrinsic radiative rates of the most stable subgap excitonic features, we find that while the presence of defects introduces low-lying optical transitions, resulting in excitons with larger oscillator strength, it also decreases the optical response associated to the pristine-like transition-metal dichalcogenide intralayer excitons. Our findings relate excitonic features with interface design for defect engineering in photovoltaic and transport applications.
% 
\end{abstract} 
 
\keywords{2D materials, transition-metal dichalcogenides, heterostructures, defects, graphene, excitons}
 
\maketitle


%
Van der Waals heterostructures \cite{Wang2012, Geim2013, CastroNeto2016, Liu2016}, formed by vertically stacking  atomically-thin two-dimensional layers  through weak interlayer interaction, are considered one of the most promising systems for the next-generation of ultrathin optoelectronic and photovoltaic high-performance components with tunable properties and tailored functionalities modifiable at the atomic scale \cite{Radisavljevic2011, Andras2013, Ross2014, Hersam2014, Pospischil2014, Ferrari2016}.  % add more papers from Lindenberg2021, Koppens2015
%
An important example of such heterostructures is the heterobilayer formed by stacking graphene \cite{Novoselov2005, Geim2009} with a monolayer transition metal dichalcogenide (TMDC) of the type XS\textsubscript{2}, where X is W, Mo \cite{Gierz2020, Gierz2021, Wang2021, Steinberg2021, Hernangomez2023, FariaJunior2023}. 
%
These are type I heterostructures which combine the high carrier mobility \cite{Mayorov2011}, high thermal conductivity \cite{Balandin2008} and semi-metallic character of graphene with the pseudospin circular dichroism \cite{Niu2008, Cao2012, Ye2017}, large quantum confinement, strong light absorption properties \cite{Bernardi2013} and sizeable spin-orbit interaction of a direct band gap TMDC \cite{Zhu2011, Lanzara2015}. 
% 
%

The electronic and optical properties of layered TMDCs and their heterostructures are sensitive to the potential created by defects \cite{Lischner2018, RefaelyAbramson2018, RefaelyAbramson2019, barja2019identifying, Lischner2020}.
%
In particular, the most abundant and stable point defects in these systems are monoatomic chalcogen vacancies \cite{Idrobo2013, Robertson2013, Noh2014}. Electron-hole optical transitions between the defect and pristine states are known to produce novel sub-gap excitonic features \cite{Attaccalite2011, RefaelyAbramson2018, Mitterreiter2021, Hotger2021, Sigger2022, Micevic2022} and form localized excitons that were shown to intrinsically reduce the degree of valley polarization even without additional scattering mechanisms \cite{Koratkar2015, RefaelyAbramson2018, Mitterreiter2021, Amit2022, Hotger2023}.
%
%
Changes in the dielectric environment can impact the TMDC intrinsic light emission properties \cite{Steinhoff2015, Rana2016, Marie2016}. For instance, interlayer coupling between graphene and TMDC results in a notable quenching of excitonic photoluminescence \cite{Berciaud2020} or a widening of the exciton linewidth \cite{Heinz2017}. Engineering the exciton spontaneous decay time is also possible by microcavity formation by additional adsorbed layers and consequent Purcell effect \cite{Massicotte2016, Marie2019}. This effect has been shown to give low-temperature picosecond exciton photoresponses.
%
%
Therefore, and due to the defects ubiquitous nature, a  microscopic understanding of the emergent electronic and excitonic properties of TMDC--graphene (Gr) heterobilayers in the presence of vacancies is interesting for dynamical modelling of optoelectronic devices and applications.

%%%%%%%%%%%%%%%%%%%%%%%%%%%%%%%%%%%%%%%%%%%%%%%%%%%%%%%%%%%%

In this work, we investigate the electronic and optical properties of WS\textsubscript{2}--Gr and MoS\textsubscript{2}--Gr heterobilayers with monoatomic chalcogen vacancies. We employ a GW-BSE approach \cite{Hedin1965, Hybertsen1985, Hybertsen1986, Albrecht1998, Rohlfing1998, Rohlfing2000, Deslippe2012} to compute the many-body electronic properties and optical characteristics and 
%
%
 find that due to the combination of screening and strong optical hybridization, absorption resonances of well-known pristine TMDC excitons are strongly quenched in the heterostructure, resulting in substantially altered absorbance properties compared to the pristine TMDC--Gr heterobilayer or defected TMDC without graphene. These pristine-like TMDC ``A'' and ``B'' peaks are largely reduced due to the mixing of the optical transitions with both graphene and defect electronic states and the electron-hole transitions determining their excitonic composition is fundamentally altered. This manifests also in a strong reduction of the excitonic binding energy for those absorption peaks. In addition, Fermi-level alignment of the defect transition levels and the magnitude of the spin-orbit interaction, determined by the choice of the TMDC, can result in additional substantial changes in the heterostructure optical properties. 
 %
 We obtain the intrinsic radiative rates for excitons with the largest binding energy, which create strongly mixed subgap resonances, and show that the associated inverse rates are comparable to those calculated for pristine TMDC monolayers.  
 %


%%%%%%%%%%%%%%%%%%%%%%%%%%%%%%%%%%%%%%%%%%%%%%%%%%%%%%%%%%%%%%%%%%%%%%%
%
%
%
\begin{figure}
        \includegraphics[width=0.95\linewidth]{f1.pdf}
   %
    \caption{%
    (a) Schematics of a WS\textsubscript{2}--Gr heterobilayer. 
    %
    %
    Each supercell (two are shown here) forms a parallelepiped with lateral boundaries marked by the straight red lines (in-plane supercell lattice vectors are denoted by $\mathbf{R}_1$, $\mathbf{R}_2$). The monoatomic chalcogen vacancy position, located opposite to the graphene layer, is indicated by a red triangle.
    (b) DFT and GW calculated valence and conduction band energies at the $\bar{\textnormal{K}}$ point. The blue bars on the left-hand side correspond to WS\textsubscript{2}--Gr, the red bars on the right-hand side, to  MoS\textsubscript{2}--Gr. 
%     %
%     %%%%%
     The dashed lines denote the defect energy levels and the solid lines mark the top of the pristine TMDC valence and conduction bands. 
     Energies are related to the Fermi energy of the system defined as $E_\textnormal{F} = (E_{\textnormal{val}} + E_{\textnormal{cond}})/2$.
     %
     The grey sketch of the Dirac cone represents the region of the TMDC pristine band gap occupied by graphene.
%     %
     Conduction levels with predominantly $\Lambda$ nature are represented by the dashed dotted lines, with those with predominantly $\textnormal{K}$ nature represented by dotted lines.
     %
    }\label{f1} 
\end{figure}
%
We employ a commensurate supercell composed of $4 \times 4$ WS\textsubscript{2} (resp. MoS\textsubscript{2})  and $5 \times 5$ graphene elementary cells, see Fig. \ref{f1} (a) and Supporting Information (SI).  
% 
We consider a vacancy concentration of $\sim 3$\% (at least one order of magnitude larger than the typical intrinsic vacancy concentration \cite{Abhay2019}) corresponding to a single monoatomic sulphur vacancy per supercell, located at the opposite side of the graphene layer. 
%
We first perform a geometry optimization of the supercell atomic positions, keeping the supercell volume constant (see \onlinecite{Hernangomez2023} and SI). This optimization reduces the nearest-neighbor bonds close to the vacancy (which shrinks and strains the TMDC lattice) as well as the interlayer distance between the TMDC and the graphene layer.
%
%
%%%%%%%%%%%%%%%%%%%%%%%%%%%%%%%%%%%%%%%%%%%%%%%%%%%%%%%%%%%%%%%%%
%
%%%%%%%%%%%%%%%%%%%%%%%%%%%%%%%%%%%%%%%%%%%%%%%%%%%%%%%%%%%%%%%%%
Using DFT as a starting point (with the PBE functional \cite{Perdew1996}), we perform a one-shot GW (G\textsubscript{0}W\textsubscript{0}) calculation for each TMDC--Gr heterobilayer (see computational details in the SI). 
%
%
%
% 
Fig. \ref{f1} (b) shows the DFT and GW energies at the point $\bar{\textnormal{K}}$ of the supercell Brillouin zone in an energy level diagram. As expected based on previous studies \cite{Qiu2013, RefaelyAbramson2018, Jain2018}, GW increases the gap in both systems with the quasiparticle corrections, qualitatively conserving the DFT picture for these heterobilayers \cite{Hernangomez2023}.
%
%
There are four spin-orbit split defect states, shifted upon the GW calculation to higher energy with respect to the Fermi level. %
%
%
Far from the Fermi level, we find the valence band splitting due to spin-orbit interaction to be $\sim 460$ meV for WS\textsubscript{2}--Gr and $150$ meV for MoS\textsubscript{2}--Gr, which is consistent with $425 \pm 18$ meV and $170 \pm 2$ meV obtained from high-resolution ARPES measurements \cite{Lanzara2015}. 
%
The dielectric screening also shifts the pair of occupied defect states to lower energies (by $\sim 400-450$ meV for WS\textsubscript{2}--Gr and $\sim 400$ meV for MoS\textsubscript{2}--Gr).
%
Simultaneously, the pristine-like band gap of WS\textsubscript{2} increases from $1.69$ eV at the DFT level to $2.20$ eV at the GW level.
Similarly, the pristine-like band gap of MoS\textsubscript{2}--Gr changes from $1.73$ eV to $2.19$ eV.
%
These values reflect a significant renormalization of the TMDC band gaps compared to the monolayer case, as expected due to the quasi-metallic  character of graphene with GW corrections including image charge effects \cite{Jain2018}. 
Our GW results are in agreement with previous calculations with reported band gap reduction of $\sim 300-350$ meV \cite{Jain2018, Thygesen2015} compared to the pristine counterpart \cite{Qiu2013}. %
They are also well consistent with the experimental values of the quasi-particle bandgap found in MoS\textsubscript{2}--Gr heterostructures, reported to be $\sim 2.0-2.2$ eV \cite{Shi2015, Hersam2016}, and in WS\textsubscript{2}--Gr, which ranges $\sim 2.0-2.3$ eV \cite{Giusca2019, Gierz2021, Raja2017}. 
%
%


\begin{figure}
   \includegraphics[width=1.0\linewidth]{figures/f2.pdf}
%
    \caption{Absorbance and exciton contributions for the defected WS\textsubscript{2}--Gr heterobilayer. (Top) Absorbance calculated along one of the main in-plane polarization directions, as well as its decomposition into graphene and WS\textsubscript{2} contributions (interlayer contributions are read from the difference of the three traces).
    For comparison, we also show the absorbance of the pristine WS\textsubscript{2}-Gr heterobilayer (from \onlinecite{Kleiner2023}).
    %
    The dashed horizontal black line marks the $2.4\%$ universal limit of graphene absorbance at infrared energies.
    The shaded box represents the estimated range for which we expect a smooth and monotonic spectrum dominated by graphene (instead of resonances resulting from finite \textbf{k}-grid sampling). The vertical dotted line marks the optical ranges below which excitons are dominated by defect-graphene sub-gap transitions to a range where excitons present larger intralayer TMDC composition.
    %
    (Bottom) For each exciton composing the absorbance resonances, we represent the contribution of each electron and hole bands. Each dot corresponds to the band contribution to a given exciton summed over all $\mathbf{k}$ points %
     %
    (only bright contributions whose 
    %
    oscillator strength are $> 5$ a.u. are shown). For clarity, all %
    dots with value $\geq 10^3$ a.u. have the same area. 
    %
    The color code corresponds to the layer composition  of each contribution and the dotted box marks the graphene-defect empty bands. 
    %
    }\label{f5} 
\end{figure}

%
% %%%%%%%%%%%%%%%%%%%%%%%%%%%%%%%%%%%%%%%%%%%%%%%%%
%
 
%%%%%%%%%%%%%%%%%%%%%%%%%%%%%%%%%%%%%%%%%%%%%%%%%%%%%%%%%%%%%%%
%
%%%%%%%%%%%%%%%%%%%%%%%%%%%%%%%%%%%%%%%%%%%%%%%%%%%%%%%%%%%%%%%%
%
Next, we examine the excitonic properties of defected  WS\textsubscript{2}--Gr and MoS\textsubscript{2}--Gr heterobilayers using the Bethe-Salpether equation \cite{Rohlfing1998, Rohlfing2000} within the Tamm-Dancoff approximation (see SI).
%
%
We show in Fig. \ref{f5}, top panel, the absorbance of WS\textsubscript{2}--Gr as well as its decomposition into intralayer graphene, intralayer TMDC and interlayer contributions (see SI for the case of MoS\textsubscript{2}--Gr). 
%
%
%
% 
At low optical energies, intralayer graphene electron-hole excitonic features are found to be most prominent. Graphene intraband transitions (not considered here, as well as temperature effects) are known to dominate this regime \cite{Li2009}, marked by a dashed grey rectangle. The resonant peaks are a consequence of the finite \textbf{k}-grid sampling of the graphene Dirac cone and therefore, the absorption in this region is actually continuous in the dense grid limit.
%
In the high infrared spectral range, excitonic peaks corresponding to interlayer graphene--defect and graphene--pristine optical transitions appear  at higher energies while graphene intralayer contributions become less relevant. At optical energies $\gtrsim 2.0$ eV, intralayer TMDC contributions (in the form of defect-defect, defect-pristine and pristine-pristine band transitions) become the dominant features of the spectrum over the interlayer contributions.
%
%%%%%%%%%%%
Out of all optical transitions sampled for our \textbf{k}-grid and energy range, 0.4\% belong to intralayer graphene while 62\% correspond to intralayer TMDC transitions, with the remaining 37\% representing a large degree of interlayer mixing. 
%
% 

The spectral absorbance of graphene for infrared light is almost constant at $2.4\%$ \cite{Li2009, Nair2008, Heinz2008, Gusynin2006, Peres2006, Carbotte2008, Stauber2008}. This limit is represented by a dashed horizontal line in Fig. \ref{f5}. We find that absorbance resonances in the optical range between $\sim 1.0-1.6$ eV oscillate around values larger than the graphene infrared constant limit. These features are likely due to excitons that involve defects and should persist in the dense grid limit.  
The computed absorbance values in this range are also consistent with those calculated for defected MoSe\textsubscript{2}  in the absence of graphene \cite{RefaelyAbramson2018}.
% %
To further validate our findings, we compare the absorbance for WS\textsubscript{2}--Gr with and without vacancies. We observe that, unlike the defected heterostructure, the absorbance of the pristine heterobilayer oscillates around the graphene  limit in the same energy range , supporting our previous conclusion.
%
In the visible range (\textit{i.e.} $\sim 1.6-3.2$ eV), experiments on WS\textsubscript{2} have reported a red shift and a significant quenching of the excitonic resonances in photoluminescence upon graphene adsorption \cite{Heinz2017, Magnozzi2020}. These effects were attributed to changes in the dielectric environment  of the TMDC due to the adsorbed quasi-metallic layer.
In defected WS\textsubscript{2}--Gr heterobilayers, we observe 
%
a reduction of the strength of the pristine-like TMDC resonances in addition to the strong interlayer mixing in the sub-gap optical region associated with the defect states. This effect is attributed to the strong optical mixing of TMDC and graphene, which results in additional interlayer optical transitions and redistribution of the oscillator strength due to the defects.
These combined effects of the graphene and vacancies quench the pristine-like ``A'' and ``B'' resonances \cite{Ramasubramaniam2012, Druppel2018} and also broaden them \cite{Amit2022}. As a consequence, they are no longer as dominant in the spectrum. Moreover, the composition of the absorption peaks, clearly defined at $\sim 2.2$, $\sim 2.4$ and $\sim 2.7$ eV, also changes drastically compared to the expectation for the pristine or defected TMDC monolayers (see SI).
%
%
%
%


 \begin{figure*}
   \includegraphics[width=1.0\linewidth]{figures/f8.pdf}
    \caption{(a) Exciton energies for WS\textsubscript{2}--Gr, $\Omega_S$, as a function of their binding energy, $E_\textnormal{bind}$.
    %
    %
    Only excitons for which $E_\textnormal{bind} > 2.5$ meV are shown ($\sim 8000$ out of $142 884$ excitons for our \textbf{k}-grid sampling and bands). The size of each dot is proportional  to the oscillator strength, $\mu_S$. %
    %
    %
    %
    Bright excitons, in particular those dominated by intralayer graphene transitions have very small binding energies (smaller than the thermal energy at room temperature, marked by a dashed black vertical line).
    Excitons within the energy range where the pristine ``A'' and ``B'' features would be expected are dark and also have a very small binding energy.    
    %
    The grey rectangle corresponds to  excitons with  binding energies  larger than $25$ meV. 
    %    
    %
    (b) Brillouin zone exciton distribution and transition band diagram for the exciton marked with a purple circle in  (a). 
    The top Brillouin zone represents transitions to the conduction bands; the bottom one, transitions from the valence bands. 
    %
    %
    (c) Transition band diagram and sketch of the optical transitions at selected \textbf{k}-points  marked with purple arrows in (b). 
    }\label{f8} 
\end{figure*}
%




%%%%%%%%%%%%%
To further understand this effect, we show in Fig. \ref{f5}, bottom panel, the contribution of each band to the exciton (similarly to our previous analysis of exciton state mixing in defected systems \cite{RefaelyAbramson2018, Mitterreiter2021, Steinitz2022}).
%
Each excitonic state, $|\Psi^S \rangle$, defined by its wavefunction amplitude $A^S_{vc\mathbf{k}}$ and energy $\Omega_S$, is represented by a column of dots whose area is proportional to $\sum_{v\mathbf{k}}|A^S_{vc\mathbf{k}}|^2$, for each electron ($c$), and $\sum_{c\mathbf{k}}|A^S_{vc\mathbf{k}}|^2$, for each hole ($v$).%  %
The color of the dot represents the target layer from which ($c$) or to which ($v$) the transitions occur. 
%
We observe intralayer graphene optical transitions in the low energy region ($\lesssim 0.5$ eV), while excitonic resonances with interlayer character appear only above $0.5$ eV.
%
%
The dispersive nature of graphene can be seen from the increase of the conduction band number for graphene with the energy.
%
As expected, the quenched high-energy resonances %
%
%
%
show significant contribution of mixed TMDC and graphene optical transitions, thus, although they appear in similar positions they no longer possess true ``A'' and ``B'' characters (see SI). We note that while graphene reduces absorption properties without the defects as well \cite{Kleiner2023}, here the effect is further pronounced due the electron-hole defect and non-defect mixing in the sub-gap region.  
%
%
%
%%%%%%%%%%%%%%%%%%%%%%%%%%%%%%%%%%%%%%%%%%%%%%%%%%%%%%%%%%%%%%%%%%%%%%%%%%%
%%%
%
% %
%
%
%

%%%%%%%%%%%%%%%%%%%%%%%%%%%%%%%%%%%%%%%%%%%%%%
%
% %%%%%%%%%%%%%%%%%%%%%%%%%%%%%%%%%%%%%%%%%%%%%%%%%%%%%%%%
The binding energy of the exciton quantifies how strongly bound are the electrons and holes participating in the excitation. %  forming the exciton.
%
Intralayer graphene excitonic features have vanishing small binding energies ($\sim 0-1$ meV, see also \onlinecite{Li2009, Li2011}) despite their strong oscillator strength. This differs significantly from pristine or encapsulated TMDC excitons, which have large oscillator strength and binding energies.
%
In particular, experimental estimations of the binding energies are $0.3-0.4$ eV for MoS\textsubscript{2}, and $0.3-0.7$ eV for WS\textsubscript{2} pristine ``A'' excitons \cite{Urbaszek2018}. %
Theoretical predictions give $0.6$ eV for MoS\textsubscript{2} and $0.2$ eV for WS\textsubscript{2} pristine ``B'' excitons \cite{Druppel2018}. %
%
%
%
%
%
%
%
In Fig. \ref{f8} (a), we present the exciton energy as a function of binding energy for WS\textsubscript{2}--Gr (see SI for MoS\textsubscript{2}--Gr). For excitons in the optical region where the pristine ``A'' and ``B''  resonances  would be expected, we observe that excitons have substantially lower binding energies compared to the pristine or encapsulated counterparts, as well as substantially smaller oscillator strength. We attribute this to a redistribution of the oscillator strength due to the combined effect of substantial hybridization of the defect and non-defect electron-hole transitions with graphene, as well as the small binding properties of excitons in graphene due to its quasi-metallic nature at low energies.
%
Importantly, we also find excitons (with oscillator strength in the range $10^{-2}-1.0$ a.u.) in the optical region $1.5-2.0$ with a binding energy comparable to that of pristine excitons in the absence of graphene layer. These excitons result from intralayer optical transitions to defect states and interlayer graphene-defect transitions.
%
%
To gain insight into the nature of these excitons, we show in Fig. \ref{f8} (b) the \textbf{k}-space distribution for a representative  case  marked by a circle in panel (a). 
%
It is worth noting that their degree of localization cannot be used to infer  the strength of the binding, as excitons with similar
binding  may exhibit optical transitions in very different regions of the Brillouin zone due to the dispersive nature of the graphene bands and the delocalization of defect states in \textbf{k}-space.
%
Fig. \ref{f8} (c) displays the optical transitions at selected points in the Brillouin zone, supporting our analysis that this exciton is formed by a combination of defect-defect (notably at $\bar{\textnormal{K}}$), graphene-valence and graphene-defect transitions (for the \textbf{k}-point noted as $\mathbf{k}^\ast$).

%


% 
Finally, we relate our findings to the intrinsic radiative decay rates of zero-momentum excitons, which can be computed from the GW-BSE oscillator strength and excitation energy \cite{Spataru2005, Palummo2015, Bernardi2018}. We consider the inverse rate, which scales as $\gamma_S^{-1}:=\tau_S  \sim \Omega_S/\mu_S$.
%
This rate accounts only for part of the radiative linewidth, as other contributions, \textit{e.g.} electron-phonon and exciton-phonon terms are not included, and is useful to evaluate the significance of the oscillator strength. 
%
%
Our analysis reveals that the inverse rates for the excitons with binding $>50$ meV can be as large as $\tau_S \sim 0.1$ ps for both heterobilayers. However, depending on the oscillator strength, they can be shorter, even as small as $\tau_S \sim 0.1$ fs for MoS\textsubscript{2}--Gr (see SI). Intralayer graphene excitonic features, which have significantly small binding, have substantially larger intrinsic rates due to their large optical transition dipole. % maybe compute one to give a number
%
%
Dark interlayer excitons with large binding have larger inverse rates, as large as $\sim 100$ ps for WS\textsubscript{2}--Gr, due to the smaller oscillator strength.
%
For WS\textsubscript{2}--Gr without defects bright ``A'' and ``B'' excitons  in the TMDC layer, $\tau_S$ can be even shorter, essentially due to the increased oscillator strength (between two and three orders of magnitude compared to the defect-related excitons, see SI) which yields $\tau_S \sim 10^{-4}-10^{-5}$ fs. We note that compared to pristine TMDCs \cite{Palummo2015}, graphene adsorption has a strong impact on $\tau_S$, which only become comparable again to the pristine ones in the presence of defects due to the strong exciton hybridization of the graphene and the subgap vacancy-related features.
%
% 
%
%
%
%
%
Furthermore, charge transfer times of photocarriers at TMDC--graphene interfaces \cite{Jin2018, Yuan2018, Gierz2020, Gierz2021, Wang2021}, where single-particle defect tunneling is understood to be the dominating coherent transport channel \cite{Gierz2021, Hernangomez2023} can be of similar order of magnitude. In this scenario, defects slow down coherent charge transfer due to relatively small interlayer tunneling. Interestingly, in the presence of graphene, defects optically enhance transitions associated to them, resulting in  excitons with significantly higher oscillator strength, compared to the reduced oscillator strength of the original pristine-like ``A'' and ``B'' TMDC excitons.

%
%%%%%%%
%


%%%%%%%%%%%%%%%%%%%%%%%%%%%%%%%%%%%%%%%%%%%%%%%%%%%%%%%%%%%%
%
In conclusion, we have studied the electronic and optical properties of WS\textsubscript{2}--Gr and MoS\textsubscript{2}--Gr heterobilayers with chalcogen vacancies employing first-principles many-body perturbation theory. 
%
We find that strong hybridization of the defect states with graphene gives rise to subgap features, which manifest as strong resonances in the optical absorbance spectrum, while quenching the  ``A'' and ``B'' pristine exciton peaks originally coming from intralayer TMDC transitions. These altered absorption features may be used to extend the functionality in the infrared of solar cells. We have analyzed the stability of the excitons and found a strong reduction of the binding energy for those TMDC excitons, while strongly hybridized interlayer and defect-dominated excitons have binding energies up to $\sim 250$ meV.
%
We computed the intrinsic radiative decay rate of these excitons and found inverse rates of up to $0.1$ ps. Overall, our results demonstrate how point-like defects can be used to design optical features in graphene-based van der Waals heterostructures, where excitons inherit properties from two well-distinct layers in a non-trivial way, pointing to the relevance of a first-principles understanding of many-body effects in the description of these systems for transport and potential optoelectronic applications.


%%%%%%%%%%%%%%%%%%%%%%%%%%%%%%%%%%%%%%%%%%%%%%
%
\subparagraph{Acknowledgments.} 
%
The authors acknowledge Tomer Amit, Mar\'ia Camarasa--G\'omez, Alexey Chernikov, Florian Dirnberger, Paulo E. Faria Junior and Alexander Holleitner for insightful discussions.
%
The authors are thankful to Simone Latini, Lede Xian and \'Angel Rubio for the initial geometry employed as a starting point of the calculations performed in this manuscript.
%
The computations were carried out in the Chemfarm local cluster at the Weizmann Institute of Science and the Max Planck Computing and Data Facility cluster. 
%
D. H.-P. and A. K. acknowledge a Minerva Foundation grant 7135421. This research is supported by the German Research Foundation (DFG) through the Collaborative Research Center SFB 1277 (Project-ID 314695032, project B10). S. R. A. is an incumbent of the Leah Omenn Career Development Chair.
This project has received funding from the European Research Council (ERC), Grant agreement No. 101041159.


%%%%%%%%%%%%%%%%%%%%%%%%%%%%%%%%%%%%%%%%%%%%%%%%%%%%%%%%%%%%
%
\subparagraph{Supporting Information.}
%
The Supporting Information is available free of charge on the ACS Publications website at \texttt{http://pubs.acs.org}.\\
%
Computational details and methods, additional results, convergence checks, absorbance and additional results for MoS\textsubscript{2}--Gr, wavefunction density of relevant conduction states, exciton reciprocal space composition for relevant energies in the WS\textsubscript{2}--Gr spectra, additional figures.

%%%%%%%%%%%%%%%%%%%%%%%%%%%%%%%%%%%%%%%%%%%%%%%%%%%%%%



\bibliography{biblio}

\end{document}



