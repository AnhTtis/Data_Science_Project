
%% bare_jrnl_compsoc.tex
%% V1.4b
%% 2015/08/26
%% by Michael Shell
%% See:
%% http://www.michaelshell.org/
%% for current contact information.
%%
%% This is a skeleton file demonstrating the use of IEEEtran.cls
%% (requires IEEEtran.cls version 1.8b or later) with an IEEE
%% Computer Society journal paper.
%%
%% Support sites:
%% http://www.michaelshell.org/tex/ieeetran/
%% http://www.ctan.org/pkg/ieeetran
%% and
%% http://www.ieee.org/

%%*************************************************************************
%% Legal Notice:
%% This code is offered as-is without any warranty either expressed or
%% implied; without even the implied warranty of MERCHANTABILITY or
%% FITNESS FOR A PARTICULAR PURPOSE! 
%% User assumes all risk.
%% In no event shall the IEEE or any contributor to this code be liable for
%% any damages or losses, including, but not limited to, incidental,
%% consequential, or any other damages, resulting from the use or misuse
%% of any information contained here.
%%
%% All comments are the opinions of their respective authors and are not
%% necessarily endorsed by the IEEE.
%%
%% This work is distributed under the LaTeX Project Public License (LPPL)
%% ( http://www.latex-project.org/ ) version 1.3, and may be freely used,
%% distributed and modified. A copy of the LPPL, version 1.3, is included
%% in the base LaTeX documentation of all distributions of LaTeX released
%% 2003/12/01 or later.
%% Retain all contribution notices and credits.
%% ** Modified files should be clearly indicated as such, including  **
%% ** renaming them and changing author support contact information. **
%%*************************************************************************


% *** Authors should verify (and, if needed, correct) their LaTeX system  ***
% *** with the testflow diagnostic prior to trusting their LaTeX platform ***
% *** with production work. The IEEE's font choices and paper sizes can   ***
% *** trigger bugs that do not appear when using other class files.       ***                          ***
% The testflow support page is at:
% http://www.michaelshell.org/tex/testflow/


\documentclass[10pt,journal,compsoc]{IEEEtran}
%
% If IEEEtran.cls has not been installed into the LaTeX system files,
% manually specify the path to it like:
% \documentclass[10pt,journal,compsoc]{../sty/IEEEtran}





% Some very useful LaTeX packages include:
% (uncomment the ones you want to load)


% *** MISC UTILITY PACKAGES ***
%
%\usepackage{ifpdf}
% Heiko Oberdiek's ifpdf.sty is very useful if you need conditional
% compilation based on whether the output is pdf or dvi.
% usage:
% \ifpdf
%   % pdf code
% \else
%   % dvi code
% \fi
% The latest version of ifpdf.sty can be obtained from:
% http://www.ctan.org/pkg/ifpdf
% Also, note that IEEEtran.cls V1.7 and later provides a builtin
% \ifCLASSINFOpdf conditional that works the same way.
% When switching from latex to pdflatex and vice-versa, the compiler may
% have to be run twice to clear warning/error messages.






% *** CITATION PACKAGES ***
%
\ifCLASSOPTIONcompsoc
  % IEEE Computer Society needs nocompress option
  % requires cite.sty v4.0 or later (November 2003)
  \usepackage[nocompress]{cite}
\else
  % normal IEEE
  \usepackage{cite}
\fi
% cite.sty was written by Donald Arseneau
% V1.6 and later of IEEEtran pre-defines the format of the cite.sty package
% \cite{} output to follow that of the IEEE. Loading the cite package will
% result in citation numbers being automatically sorted and properly
% "compressed/ranged". e.g., [1], [9], [2], [7], [5], [6] without using
% cite.sty will become [1], [2], [5]--[7], [9] using cite.sty. cite.sty's
% \cite will automatically add leading space, if needed. Use cite.sty's
% noadjust option (cite.sty V3.8 and later) if you want to turn this off
% such as if a citation ever needs to be enclosed in parenthesis.
% cite.sty is already installed on most LaTeX systems. Be sure and use
% version 5.0 (2009-03-20) and later if using hyperref.sty.
% The latest version can be obtained at:
% http://www.ctan.org/pkg/cite
% The documentation is contained in the cite.sty file itself.
%
% Note that some packages require special options to format as the Computer
% Society requires. In particular, Computer Society  papers do not use
% compressed citation ranges as is done in typical IEEE papers
% (e.g., [1]-[4]). Instead, they list every citation separately in order
% (e.g., [1], [2], [3], [4]). To get the latter we need to load the cite
% package with the nocompress option which is supported by cite.sty v4.0
% and later. Note also the use of a CLASSOPTION conditional provided by
% IEEEtran.cls V1.7 and later.





% *** GRAPHICS RELATED PACKAGES ***
%
\ifCLASSINFOpdf
  % \usepackage[pdftex]{graphicx}
  % declare the path(s) where your graphic files are
  % \graphicspath{{../pdf/}{../jpeg/}}
  % and their extensions so you won't have to specify these with
  % every instance of \includegraphics
  % \DeclareGraphicsExtensions{.pdf,.jpeg,.png}
\else
  % or other class option (dvipsone, dvipdf, if not using dvips). graphicx
  % will default to the driver specified in the system graphics.cfg if no
  % driver is specified.
  % \usepackage[dvips]{graphicx}
  % declare the path(s) where your graphic files are
  % \graphicspath{{../eps/}}
  % and their extensions so you won't have to specify these with
  % every instance of \includegraphics
  % \DeclareGraphicsExtensions{.eps}
\fi
% graphicx was written by David Carlisle and Sebastian Rahtz. It is
% required if you want graphics, photos, etc. graphicx.sty is already
% installed on most LaTeX systems. The latest version and documentation
% can be obtained at: 
% http://www.ctan.org/pkg/graphicx
% Another good source of documentation is "Using Imported Graphics in
% LaTeX2e" by Keith Reckdahl which can be found at:
% http://www.ctan.org/pkg/epslatex
%
% latex, and pdflatex in dvi mode, support graphics in encapsulated
% postscript (.eps) format. pdflatex in pdf mode supports graphics
% in .pdf, .jpeg, .png and .mps (metapost) formats. Users should ensure
% that all non-photo figures use a vector format (.eps, .pdf, .mps) and
% not a bitmapped formats (.jpeg, .png). The IEEE frowns on bitmapped formats
% which can result in "jaggedy"/blurry rendering of lines and letters as
% well as large increases in file sizes.
%
% You can find documentation about the pdfTeX application at:
% http://www.tug.org/applications/pdftex






% *** MATH PACKAGES ***
%
%\usepackage{amsmath}
% A popular package from the American Mathematical Society that provides
% many useful and powerful commands for dealing with mathematics.
%
% Note that the amsmath package sets \interdisplaylinepenalty to 10000
% thus preventing page breaks from occurring within multiline equations. Use:
%\interdisplaylinepenalty=2500
% after loading amsmath to restore such page breaks as IEEEtran.cls normally
% does. amsmath.sty is already installed on most LaTeX systems. The latest
% version and documentation can be obtained at:
% http://www.ctan.org/pkg/amsmath





% *** SPECIALIZED LIST PACKAGES ***
%
%\usepackage{algorithmic}
% algorithmic.sty was written by Peter Williams and Rogerio Brito.
% This package provides an algorithmic environment fo describing algorithms.
% You can use the algorithmic environment in-text or within a figure
% environment to provide for a floating algorithm. Do NOT use the algorithm
% floating environment provided by algorithm.sty (by the same authors) or
% algorithm2e.sty (by Christophe Fiorio) as the IEEE does not use dedicated
% algorithm float types and packages that provide these will not provide
% correct IEEE style captions. The latest version and documentation of
% algorithmic.sty can be obtained at:
% http://www.ctan.org/pkg/algorithms
% Also of interest may be the (relatively newer and more customizable)
% algorithmicx.sty package by Szasz Janos:
% http://www.ctan.org/pkg/algorithmicx




% *** ALIGNMENT PACKAGES ***
%
%\usepackage{array}
% Frank Mittelbach's and David Carlisle's array.sty patches and improves
% the standard LaTeX2e array and tabular environments to provide better
% appearance and additional user controls. As the default LaTeX2e table
% generation code is lacking to the point of almost being broken with
% respect to the quality of the end results, all users are strongly
% advised to use an enhanced (at the very least that provided by array.sty)
% set of table tools. array.sty is already installed on most systems. The
% latest version and documentation can be obtained at:
% http://www.ctan.org/pkg/array


% IEEEtran contains the IEEEeqnarray family of commands that can be used to
% generate multiline equations as well as matrices, tables, etc., of high
% quality.




% *** SUBFIGURE PACKAGES ***
%\ifCLASSOPTIONcompsoc
%  \usepackage[caption=false,font=footnotesize,labelfont=sf,textfont=sf]{subfig}
%\else
%  \usepackage[caption=false,font=footnotesize]{subfig}
%\fi
% subfig.sty, written by Steven Douglas Cochran, is the modern replacement
% for subfigure.sty, the latter of which is no longer maintained and is
% incompatible with some LaTeX packages including fixltx2e. However,
% subfig.sty requires and automatically loads Axel Sommerfeldt's caption.sty
% which will override IEEEtran.cls' handling of captions and this will result
% in non-IEEE style figure/table captions. To prevent this problem, be sure
% and invoke subfig.sty's "caption=false" package option (available since
% subfig.sty version 1.3, 2005/06/28) as this is will preserve IEEEtran.cls
% handling of captions.
% Note that the Computer Society format requires a sans serif font rather
% than the serif font used in traditional IEEE formatting and thus the need
% to invoke different subfig.sty package options depending on whether
% compsoc mode has been enabled.
%
% The latest version and documentation of subfig.sty can be obtained at:
% http://www.ctan.org/pkg/subfig




% *** FLOAT PACKAGES ***
%
%\usepackage{fixltx2e}
% fixltx2e, the successor to the earlier fix2col.sty, was written by
% Frank Mittelbach and David Carlisle. This package corrects a few problems
% in the LaTeX2e kernel, the most notable of which is that in current
% LaTeX2e releases, the ordering of single and double column floats is not
% guaranteed to be preserved. Thus, an unpatched LaTeX2e can allow a
% single column figure to be placed prior to an earlier double column
% figure.
% Be aware that LaTeX2e kernels dated 2015 and later have fixltx2e.sty's
% corrections already built into the system in which case a warning will
% be issued if an attempt is made to load fixltx2e.sty as it is no longer
% needed.
% The latest version and documentation can be found at:
% http://www.ctan.org/pkg/fixltx2e


%\usepackage{stfloats}
% stfloats.sty was written by Sigitas Tolusis. This package gives LaTeX2e
% the ability to do double column floats at the bottom of the page as well
% as the top. (e.g., "\begin{figure*}[!b]" is not normally possible in
% LaTeX2e). It also provides a command:
%\fnbelowfloat
% to enable the placement of footnotes below bottom floats (the standard
% LaTeX2e kernel puts them above bottom floats). This is an invasive package
% which rewrites many portions of the LaTeX2e float routines. It may not work
% with other packages that modify the LaTeX2e float routines. The latest
% version and documentation can be obtained at:
% http://www.ctan.org/pkg/stfloats
% Do not use the stfloats baselinefloat ability as the IEEE does not allow
% \baselineskip to stretch. Authors submitting work to the IEEE should note
% that the IEEE rarely uses double column equations and that authors should try
% to avoid such use. Do not be tempted to use the cuted.sty or midfloat.sty
% packages (also by Sigitas Tolusis) as the IEEE does not format its papers in
% such ways.
% Do not attempt to use stfloats with fixltx2e as they are incompatible.
% Instead, use Morten Hogholm'a dblfloatfix which combines the features
% of both fixltx2e and stfloats:
%
% \usepackage{dblfloatfix}
% The latest version can be found at:
% http://www.ctan.org/pkg/dblfloatfix




%\ifCLASSOPTIONcaptionsoff
%  \usepackage[nomarkers]{endfloat}
% \let\MYoriglatexcaption\caption
% \renewcommand{\caption}[2][\relax]{\MYoriglatexcaption[#2]{#2}}
%\fi
% endfloat.sty was written by James Darrell McCauley, Jeff Goldberg and 
% Axel Sommerfeldt. This package may be useful when used in conjunction with 
% IEEEtran.cls'  captionsoff option. Some IEEE journals/societies require that
% submissions have lists of figures/tables at the end of the paper and that
% figures/tables without any captions are placed on a page by themselves at
% the end of the document. If needed, the draftcls IEEEtran class option or
% \CLASSINPUTbaselinestretch interface can be used to increase the line
% spacing as well. Be sure and use the nomarkers option of endfloat to
% prevent endfloat from "marking" where the figures would have been placed
% in the text. The two hack lines of code above are a slight modification of
% that suggested by in the endfloat docs (section 8.4.1) to ensure that
% the full captions always appear in the list of figures/tables - even if
% the user used the short optional argument of \caption[]{}.
% IEEE papers do not typically make use of \caption[]'s optional argument,
% so this should not be an issue. A similar trick can be used to disable
% captions of packages such as subfig.sty that lack options to turn off
% the subcaptions:
% For subfig.sty:
% \let\MYorigsubfloat\subfloat
% \renewcommand{\subfloat}[2][\relax]{\MYorigsubfloat[]{#2}}
% However, the above trick will not work if both optional arguments of
% the \subfloat command are used. Furthermore, there needs to be a
% description of each subfigure *somewhere* and endfloat does not add
% subfigure captions to its list of figures. Thus, the best approach is to
% avoid the use of subfigure captions (many IEEE journals avoid them anyway)
% and instead reference/explain all the subfigures within the main caption.
% The latest version of endfloat.sty and its documentation can obtained at:
% http://www.ctan.org/pkg/endfloat
%
% The IEEEtran \ifCLASSOPTIONcaptionsoff conditional can also be used
% later in the document, say, to conditionally put the References on a 
% page by themselves.




% *** PDF, URL AND HYPERLINK PACKAGES ***
%
%\usepackage{url}
% url.sty was written by Donald Arseneau. It provides better support for
% handling and breaking URLs. url.sty is already installed on most LaTeX
% systems. The latest version and documentation can be obtained at:
% http://www.ctan.org/pkg/url
% Basically, \url{my_url_here}.





% *** Do not adjust lengths that control margins, column widths, etc. ***
% *** Do not use packages that alter fonts (such as pslatex).         ***
% There should be no need to do such things with IEEEtran.cls V1.6 and later.
% (Unless specifically asked to do so by the journal or conference you plan
% to submit to, of course. )


\IEEEoverridecommandlockouts
\hyphenation{op-tical net-works semi-conduc-tor}
\usepackage{graphicx}
\usepackage{color}
\usepackage{placeins}
\usepackage{float}
\usepackage{tabularx,colortbl}
\usepackage{subcaption}
\usepackage{nccmath}

\usepackage[latin1]{inputenc}
%\usepackage[cmex10]{amsmath}
\usepackage{amsfonts}
\usepackage{amssymb}
\usepackage{epstopdf}
\usepackage[ruled,vlined]{algorithm2e}
\usepackage{dsfont}



\bibliographystyle{IEEEbib}

\usepackage{amsthm}
%\usepackage{savetrees}

\newtheorem*{remark}{Remark}

\theoremstyle{definition}
\newtheorem{definition}{Definition}[section]
\newtheorem{theorem}{Theorem}[section]
\newtheorem{corollary}{Corollary}[section]
\newtheorem{lemma}{Lemma}[section]



\begin{document}
%
% paper title
% Titles are generally capitalized except for words such as a, an, and, as,
% at, but, by, for, in, nor, of, on, or, the, to and up, which are usually
% not capitalized unless they are the first or last word of the title.
% Linebreaks \\ can be used within to get better formatting as desired.
% Do not put math or special symbols in the title.
\title{Estimating Parameters of Large CTMP from Single Trajectory with Application to Stochastic Network Epidemics Models}
%
%
% author names and IEEE memberships
% note positions of commas and nonbreaking spaces ( ~ ) LaTeX will not break
% a structure at a ~ so this keeps an author's name from being broken across
% two lines.
% use \thanks{} to gain access to the first footnote area
% a separate \thanks must be used for each paragraph as LaTeX2e's \thanks
% was not built to handle multiple paragraphs
%
%
%\IEEEcompsocitemizethanks is a special \thanks that produces the bulleted
% lists the Computer Society journals use for "first footnote" author
% affiliations. Use \IEEEcompsocthanksitem which works much like \item
% for each affiliation group. When not in compsoc mode,
% \IEEEcompsocitemizethanks becomes like \thanks and
% \IEEEcompsocthanksitem becomes a line break with idention. This
% facilitates dual compilation, although admittedly the differences in the
% desired content of \author between the different types of papers makes a
% one-size-fits-all approach a daunting prospect. For instance, compsoc 
% journal papers have the author affiliations above the "Manuscript
% received ..."  text while in non-compsoc journals this is reversed. Sigh.

\author{Seyyed A. Fatemi,~\IEEEmembership{Member,~IEEE,}
        June Zhang,~\IEEEmembership{Member,~IEEE}% <-this % stops a space
\IEEEcompsocitemizethanks{\IEEEcompsocthanksitem J. Zhang is with the Department
of Electrical and Computer Engineering, University of Hawai'i at M\={a}noa, Honolulu, HI, 96822.\protect\\
% note need leading \protect in front of \\ to get a newline within \thanks as
% \\ is fragile and will error, could use \hfil\break instead.
E-mail: zjz@hawaii.edu
%\IEEEcompsocthanksitem J. Doe and J. Doe are with Anonymous University. 
}% <-this % stops an unwanted space
\thanks{Manuscript received April 19, 2005; revised August 26, 2015.}}

% note the % following the last \IEEEmembership and also \thanks - 
% these prevent an unwanted space from occurring between the last author name
% and the end of the author line. i.e., if you had this:
% 
% \author{....lastname \thanks{...} \thanks{...} }
%                     ^------------^------------^----Do not want these spaces!
%
% a space would be appended to the last name and could cause every name on that
% line to be shifted left slightly. This is one of those "LaTeX things". For
% instance, "\textbf{A} \textbf{B}" will typeset as "A B" not "AB". To get
% "AB" then you have to do: "\textbf{A}\textbf{B}"
% \thanks is no different in this regard, so shield the last } of each \thanks
% that ends a line with a % and do not let a space in before the next \thanks.
% Spaces after \IEEEmembership other than the last one are OK (and needed) as
% you are supposed to have spaces between the names. For what it is worth,
% this is a minor point as most people would not even notice if the said evil
% space somehow managed to creep in.



% The paper headers
\markboth{Journal of \LaTeX\ Class Files,~Vol.~14, No.~8, August~2015}%
{Shell \MakeLowercase{\textit{et al.}}: Bare Demo of IEEEtran.cls for Computer Society Journals}
% The only time the second header will appear is for the odd numbered pages
% after the title page when using the twoside option.
% 
% *** Note that you probably will NOT want to include the author's ***
% *** name in the headers of peer review papers.                   ***
% You can use \ifCLASSOPTIONpeerreview for conditional compilation here if
% you desire.



% The publisher's ID mark at the bottom of the page is less important with
% Computer Society journal papers as those publications place the marks
% outside of the main text columns and, therefore, unlike regular IEEE
% journals, the available text space is not reduced by their presence.
% If you want to put a publisher's ID mark on the page you can do it like
% this:
%\IEEEpubid{0000--0000/00\$00.00~\copyright~2015 IEEE}
% or like this to get the Computer Society new two part style.
%\IEEEpubid{\makebox[\columnwidth]{\hfill 0000--0000/00/\$00.00~\copyright~2015 IEEE}%
%\hspace{\columnsep}\makebox[\columnwidth]{Published by the IEEE Computer Society\hfill}}
% Remember, if you use this you must call \IEEEpubidadjcol in the second
% column for its text to clear the IEEEpubid mark (Computer Society jorunal
% papers don't need this extra clearance.)



% use for special paper notices
%\IEEEspecialpapernotice{(Invited Paper)}



% for Computer Society papers, we must declare the abstract and index terms
% PRIOR to the title within the \IEEEtitleabstractindextext IEEEtran
% command as these need to go into the title area created by \maketitle.
% As a general rule, do not put math, special symbols or citations
% in the abstract or keywords.
\IEEEtitleabstractindextext{%
\begin{abstract}
Graph dynamical systems (GDS) model dynamic processes on a (static) graph. Stochastic GDS has been used for network-based epidemics models such as the contact process and the reversible contact process. In this paper, we consider stochastic GDS that are also continuous-time Markov processes (CTMP), whose transition rates are linear functions of some dynamics parameters $\theta$ of interest (i.e., healing, exogeneous, and endogeneous infection rates). Our goal is to estimate $\theta$ from a single, finite-time, continuously observed trajectory of the CTMP. Parameter estimation of CTMP is challenging when the state space is large; for GDS, the number of Markov states are \emph{exponential} in the number of nodes of the graph. We showed that holding classes (i.e., Markov states with the same holding time distribution) give efficient partitions of the state space of GDS. We derived an upperbound on the number of holding classes for the contact process, which is polynomial in the number of nodes. We utilized holding classes to solve a smaller system of linear equations to find $\theta$. Experimental results show that finding reasonable results can be achieved even for short trajectories, particularly for the contact process. In fact, trajectory length does not significantly affect estimation error.

%In this paper, we considered a subclass of continuous-time Markov processes where the transition rates are linear functions of a set of parameters of much smaller dimensionality than the size of the state space of the Markov process. We showed that interaction particle systems (IPS) are one such class of CTMP.  However, size of the state of IPS models grow exponentially with the number of nodes in the contact network so that even a moderately system has a prohibitively large state space.




\end{abstract}

% Note that keywords are not normally used for peerreview papers.
\begin{IEEEkeywords}
graph dynamical system, interacting particle system, CTMP, Markov process, network-based epidemics
\end{IEEEkeywords}}


% make the title area
\maketitle


% To allow for easy dual compilation without having to reenter the
% abstract/keywords data, the \IEEEtitleabstractindextext text will
% not be used in maketitle, but will appear (i.e., to be "transported")
% here as \IEEEdisplaynontitleabstractindextext when the compsoc 
% or transmag modes are not selected <OR> if conference mode is selected 
% - because all conference papers position the abstract like regular
% papers do.
\IEEEdisplaynontitleabstractindextext
% \IEEEdisplaynontitleabstractindextext has no effect when using
% compsoc or transmag under a non-conference mode.



% For peer review papers, you can put extra information on the cover
% page as needed:
% \ifCLASSOPTIONpeerreview
% \begin{center} \bfseries EDICS Category: 3-BBND \end{center}
% \fi
%
% For peerreview papers, this IEEEtran command inserts a page break and
% creates the second title. It will be ignored for other modes.
\IEEEpeerreviewmaketitle



\IEEEraisesectionheading{\section{Introduction}\label{sec:introduction}}
% Computer Society journal (but not conference!) papers do something unusual
% with the very first section heading (almost always called "Introduction").
% They place it ABOVE the main text! IEEEtran.cls does not automatically do
% this for you, but you can achieve this effect with the provided
% \IEEEraisesectionheading{} command. Note the need to keep any \label that
% is to refer to the section immediately after \section in the above as
% \IEEEraisesectionheading puts \section within a raised box.




% The very first letter is a 2 line initial drop letter followed
% by the rest of the first word in caps (small caps for compsoc).
% 
% form to use if the first word consists of a single letter:
% \IEEEPARstart{A}{demo} file is ....
% 
% form to use if you need the single drop letter followed by
% normal text (unknown if ever used by the IEEE):
% \IEEEPARstart{A}{}demo file is ....
% 
% Some journals put the first two words in caps:
% \IEEEPARstart{T}{his demo} file is ....
% 
% Here we have the typical use of a "T" for an initial drop letter
% and "HIS" in caps to complete the first word.
Graphs are used to represent dependencies between variables. There has been a great deal of interest in graph-based data analysis. Graph/network-based dynamical systems are less well-studied and used in application. Graph dynamical systems (GDS) model processes on graphs. The behavior of these systems are determined by: 1) a graph, $G$, which characterizes the dependencies amongst variables (i.e., nodes), and 2) dynamics rules that describe how the variable values change over time. Often, the state of a variable is assumed to be explicitly dependent on the states of its neighboring nodes in $G$. In this paper, we will focus on systems where $G$ has a \emph{finite} number of nodes and is \emph{static} over time. Previously, graph dynamical systems over infinitely-size graph has been studied as either cellular automata or interacting particle systems, depending on if the system has discrete-time or continuous-time dynamics.

In stochastic GDS, the dynamic rules have a random component. This means that the future state of the system is characterized by a probability distribution. Stochastic GDS that are used to model the spread of epidemics over contact networks are known as \emph{stochastic network epidemics models}.

The Markov process is a natural mathematical tool for studying such dynamics. However, they are considered to be impractical for modeling GDS because the size of the state space, $D$, is exponential in the number of variables. Additionally, $D(D-1)$ parameters are needed to completely specify the Markov process. When the state space is very large, obtaining a sufficient number of observations to compute the likelihood for $D(D-1)$ parameters would require an infeasibly long observation window. 


However, GDS-Markov processes generally have more structure than an arbitrary Markov process.  In many models, the pairwise transition probabilities/rates are functions of only a few variables. For example, the contact process is one of the most well-known GDS-continuous-time Markov process model \cite{harris1974contact}. It models infections between infected and healthy agents (i.e., nodes) in a known contact network. Only three parameters characterize the continuous-time dynamics of the model: 1) healing rate $\mu$, 2) exogeneous infection rate $\beta$ (i.e., infection from outside the contact network), 3) endogeneous infection rate $\delta$ (i.e., infection from contact with infected neighbor). Furthermore, the pairwise transition rates of the contact process are linearly dependent on $\mu, \beta, \delta$. We show that this induces equivalence classes in the diagonal entries of the transition rate matrix. These equivalence classes, which we call \emph{holding classes}, effectively reduces the dimensionality of the Markov process. This makes parameter estimation from a moderate-length trajectory possible. 

Leveraging the existence of holding classes and efficient least squares computation, we present an algorithm for learning the dynamics parameter, $\theta$, of GDS-continuous-time Markov process (CTMP) models whose transition rates are \emph{linearly} dependent on $\theta$. In this paper, we assume that the system is continuously observable (i.e., no missing observation) and that the graph $G$ is known. We applied our algorithm to the contact process and the reversible contact process. 

In Section~\ref{sec:contact} and~\ref{sec:rev}, we derived upperbounds on the number of holding classes for the contact process and the reversible contact process on arbitrary graph $G$. For the contact process, can be seen that this upperbound is polynomial, instead of exponential, in the number of variables. Therefore, the number of holding classes relative to the dimension of the state space \emph{decreases} with increasing $G$; we confirm this experimentally by exhaustively counting the number number of holding classes for small Erdos-Renyi and Watts-Strongatz graphs in Section~\ref{sec:numholdingexp}. Section~\ref{sec:exp} uses our algorithm in numerical experiments to estimate $\mu, \beta, \delta$ from synthetically generated trajectories on both 100-node random graphs and a 100-bus test power grid.


%estimating the dynamics parameters from a given finite-duration trajectory for GDS-continuous-time Markov process (CTMP) models whose transition rates are \emph{linearly} dependent on the parameters of 
%
%
%
%In this paper, we assume that states of all the variables in the system are directly observable. We present an algorithm for estimating the dynamics parameters from a given finite-duration trajectory for GDS-continuous-time Markov process (CTMP) models whose transition rates are \emph{linearly} dependent on a few parameters.  First, we show that linear dependency induces equivalence classes in the diagonal entries of the transition rate matrix, which we call holding classes. Equivalences effectively reduces the dimensionality of the state space of the Markov process. 








\section{Related Works}
We will review some of the current works on graph dynamical systems for the case where the underlaying network is \emph{static}. Related works can be found in various areas. Like the contact process, many such graph dynamical systems models are meant to model contagion-like behavior. As a result, there are a lot of overlap between works in epidemiology and general work on graph dynamical systems. We also review existing literatures on parameter estimation problems of continuous-time Markov processes; these approaches are intended for arbitrary Markov process and have not been applied to GDS due to the prohibitively large state space.



\subsection{Network-based Epidemics models and Social Contagion Models}
Efforts for mathematical modeling of contagion goes back to 17th century and Daniel Bernoulli's seminal work on smallpox disease spread. Typically, compartmental models are used, which segment individuals in a population by status such as susceptible ($S$, healthy), exposed ($E$, not infectious), infected ($I$, exposed and infectious), removed, etc. Epidemics models can be deterministic or stochastic \cite{kermack1927contribution,anderson1992infectious, dietz2002daniel}. 

Deterministic epidemics models predict the number/fraction of infected individuals in a population at some future time $t$ whereas stochastic models infer the probability distribution of the quantity of interest. Deterministic models are derived using mean-field approximation, which assume full-mixing (i.e every individual interacts with all other individuals with the same probability) and analyze the asymptotic (assuming an infinitely large population) behavior stochastic models. Classic epidemics models are time-series based; there are many approaches to learning the dynamics parameters of classic epidemics models \cite{Pan2014,Zheng2017, o2002tutorial, kypraios2017tutorial, Dutta2018}. 

One challenge is that the infection process are often not directly observable, and the likelihood is difficult to compute. Bayesian methods, which treats the unknown dynamics parameters and any unobserved quantities as random variables, are often favored. These approaches can be combined with with simple simulation models to infer the parameters that best fit the observed data. Reference~\cite{Dutta2018} uses approximate Bayesian computing (ABC) to estimate the dynamics parameter, $\theta$, and the single infection source of for a discrete-time Susceptible-Infected (SI) epidemics process.


Recently, greater attention has been paid to the inclusion of a heterogeneous contact networks within the epidemics models. These models are known as network epidemics models \cite{Pastor-Satorras2015a, Pellis2015}. Deterministic network-epidemics models extend mean-field results by incorporating statistics of the underlying network into the dynamics equations \cite{gleeson2013binary}. For example, the pairwise approximation model characterizes the dynamic of the number of contacts between an infected and susceptible individual,
\[
\frac{d [SI]}{dt} = -\tau[SI] + \tau[SSI] - \tau[ISI] + g[II] -g[SI],
\]
where $\tau$ and $g$ are the infection rate and healing rate respectively \cite{doi:10.1098/rsif.2005.0051}. 


Stochastic network epidemics models are usually discrete-time (DTMP) or continuous-time (CTMP) Markov processes. A state in the Markov process is a possible configuration of the population. In discrete-time Markov model, the probability that a healthy node will be infected in the next time step is assumed to depend on how many infected neighbors it has in $G$ \cite{gomez2010discrete,Wang2017}. In continuous-time Markov model, the probability that a healthy node will be infected in a time interval $\tau$ is dependent on how many infected neighbors it has in $G$ \cite{Zhang2014,Zhang2015,Zhang2017,VanMieghem2012}. Stochastic network epidemics models have been studied in probability as interacting particle systems (IPS). The most well-known IPS model is the contact process \cite{harris1974contact, liggett1985interacting, griffeath1983binary}. These models also share many similarities with spin system models in physics \cite{glauber1963time}. 


Research has mostly focused on the analysis of network epidemics models such as the impact of network topology on the epidemic threshold \cite{Wang2017, nowzari2016analysis}. Typically, these analytical results are derived for $t \to \infty$ and/or for an infinitely large network. Fewer work have focused on learning network epidemics models from data. The inclusion of $G$ induces a very large (discrete) state space, which makes parameter estimation difficult even if the infection process can be directly observed. 


\subsection{Other Graph Dynamical System Models}

In social science, networked-based models of social contagion have been studied \cite{Centola2007}. These work differ from epidemics-based models in the contagion mechanism. For example, social contagion models often assume a directed contact network $G$. The edge direction implies that one person has more impact to affect the behavior/state of the other individual. Social contagion model also differentiate between simple and complex contagion. In simple contagion, a susceptible agent becomes infected with a fixed probability as a result of a one-time exposure to an infected neighbor. In complex contagion, a susceptible agent becomes infected with a probability that scales with the number of exposure to infected neighbors. 

In graph signal processing, the interest is in analyzing and filtering graph signal (real or complex variables associated with nodes in a static graph). Graph dynamical systems can be studied from the perspective of time-vertex graph signals \cite{loukas2019stationary, grassi2017time}. Various stationary properties are used to characterize the covariance between the nodal values in the graph and over time.

Dynamic graph models have also been studied as (time-varying) probabilistic graphical models. Reference~\cite{nodelman2012continuous} and~\cite{el2011continuous} introduced the continuous-time Bayesian network and continuous-time Markov network for representing dynamic directed (Bayesian) or undirected (Markov) probabilistic graphical model. Similar to works in network epidemics/graph dynamical systems, the dynamic of a variable (node in a network) is only dependent on the state of neighboring nodes. The CTMP model proposed by reference~\cite{el2011continuous}, called continuous-time Markov network (CTMN) studied, considered \emph{reversible} CTMPs only. A maximum likelihood estimator for learning the model parameters was presented for a binary-state CTMN. However, this was studied on a 4-node network, and the state space is only $D = 2^4$.


\subsection{Learning Continuous-time Markov Process}

Many work in parameter estimation of large-scale continuous-time Markov processes come from chemistry, where Markov State Models (MSM) are used to model molecular dynamics \cite{mcgibbon2015efficient, MSMpaper}. In these problems, only discrete-time observations are available. Therefore, a transition probability matrix, $P$, is learned from data and the used to estimate the transition rate matrix, $Q$. Large state-spaces are difficult to handle. If the methods do not assume nor exploit any additional structure in the Markov process, the runtime is on the order of $O(D^3)$ to $O(D^5)$, where $D$ is the number of Markov states. Reference~\cite{el2011continuous} proposed a more efficient maximum likelihood estimator of the transition rate matrix by exploiting the additional structure of \emph{reversible} continuous-time Markov processes.

%%%%%%%%%%%%%%%%%%%%%%%%%%
%%%%%%%%%%%%%%%%%%%%%%%%%%
%%%%%%%%%%%%%%%%%%%%%%%%%%


\section{Continuous-time Markov Process}\label{sec:contmarkov}

The dynamic of a continuous-time Markov process, ${X(t), t\ge 0}$, taking values in a countable state space $\mathcal{D} = \{1,2, \ldots, D\}$, is completely described by the transition rate matrix, $Q =[q(i,j)], i,j \in \mathcal{D}$ (also known as the infinitesimal generator) \cite{norris1998markov, Kelly}. The transition rate from Markov state $i$ to state $j$ is
\[
q(i,j) = \lim_{\tau \to 0} \frac{P(X(t+\tau) = j | X(t) = i)}{\tau}, i \neq j.
\]
The diagonal entries of $Q$, $q(i,i), i \in \mathcal{D}$, are such that 
\begin{equation}\label{eq:diagonal}
\sum_{j \in \mathcal{D}} q(i,j) = 0.
\end{equation}
Therefore, 
\begin{equation}\label{eq:diagonal2}
q(i,i) = -\sum_{j \in \mathcal{D}} q(i,j).
\end{equation}
The continuous-time Markov process, $X(t)$, remains in state $i$ for a length of time that is exponentially distributed with rate $|q(i,i)|$; this is known as the \emph{holding time}. To reduce notation, whenever we refer to a diagonal entry $q(i,i)$, we mean the absolute value. 

\begin{definition}\label{def:holdingclass}
Two Markov states, $i, j \in \mathcal{D}, i \neq j$ belongs to the same \emph{holding class}, $\mathcal{H}$, if $q(i,i) = q(j,j)$. Then $q(\mathcal{H}, \mathcal{H}) = q(i,i) = q(j,j)$.
\end{definition}

Equivalently, two Markov states belong to the same holding class if their holding time has the same distribution. The state space of $X(t)$ can be partitioned into different holding classes: $\mathcal{D} = \{\mathcal{H}_1 \cup \mathcal{H}_2 \cup \ldots  \cup\mathcal{H}_K| \mathcal{H}_i \cap \mathcal{H}_j = \emptyset \}$.

\subsection{Continuous-time Markov Process with Linearly Dependent Transition Rates}

In this paper, we consider a sub-class of continuous-time Markov processes whose transition rates are linear functions depending on some underlying dynamic parameter,  $\theta \in \mathds{R}^b$, where $b << D$. This means that~\eqref{eq:diagonal2} can expressed in vector form as $F\theta$, where $F$ is some $D \times b$ matrix determined by how the transitions depend on $\theta$. However, we only need to be concerned with the unique diagonal values of $Q$ (i.e., holding classes), then
\begin{equation}\label{eq:linearsystem}
F\theta = \begin{bmatrix} 
q(\mathcal{H}_1,\mathcal{H}_1)\\ 
\vdots\\ 
q(\mathcal{H}_K,\mathcal{H}_K) 
\end{bmatrix}.
\end{equation}
As a result, the matrix $F$ is $K \times b$. Solving this system of equation is much more efficient when the number of holding classes is much smaller than the number of Markov states (i.e,. $K << D$). We will show that this is the case for network epidemics models such as the contact process. However, in order to solve the system of linear equations to find $\theta$, we need to be able to estimate the holding class rates, $q(\mathcal{H}_i, \mathcal{H}_i)$, from observation.

\subsection{Estimating Transitions Rates from Trajectory}

%Therefore, in general, learning a transition rate matrix from observation, $Q$, means estimating $d(d-1)$ parameters. Depending on the application, there may be two types of observations 1) continuous-time observations, 2) discrete-time observations.

We assume that the states are continuously observed without noise. This means that 1) we do not miss any observations, 2) we know both the sampling time $t_i$ and the state of the system at $t_i$. Given a finite duration ($T= t_{M-1} - t_0$) trajectory
\[
\Sigma = \{x(t_0), x(t_1), x(t_2), \ldots, x(t_{M-1})\},
\]
%, \quad X(t_i) \neq X(t_j),
it is known that the maximum likelihood estimator (MLE) of transition rate $q(i,j)$ is 
\begin{equation}\label{eq:MLEtransitionrate}
\widehat{q}(i,j)_{\text{MLE}}= \frac{N_{ij}(T)}{R_i(T)},
\end{equation}
where $N_{ij}(T)$ is the number of transitions from state $i \in \mathcal{D}$ to state $j \in \mathcal{D}$ in interval $T$ and
\[
R_i(T) = \int_{t_0}^{t_{M-1}} \mathds{1}(X(t) = i) dt, \quad i \in \mathcal{D}
\]
is the total amount of time $X(t)$ was in state $i$. The function $\mathds{1}(\cdot)$ is the indicator function. The MLE for the holding time rate is
\begin{equation}\label{eq:MLEdiagonaltransitionrate}
|\widehat{q}(i,i)_{\text{MLE}}| = \frac{\sum_{i\neq j }N_{ij}(T)}{R_i(T)}.
\end{equation}

Instead of considering individual Markov states, we can consider the holding classes. The MLE estimate of $q(\mathcal{H}_i,\mathcal{H}_i)$ is
\begin{equation}\label{eq:MLEdiagonaltransitionrate2}
\widehat{q}(\mathcal{H}_i, \mathcal{H}_i)_{\text{MLE}} =  \frac{\sum_{\mathcal{H}_i \neq \mathcal{H}_j} N_{\mathcal{H}_i, \mathcal{H}_j}(T)}{R_{\mathcal{H}_i}(T)},
\end{equation}
where $N_{\mathcal{H}_i, \mathcal{H}_j}(T)$ is the number of transitions from Markov states in holding class $\mathcal{H}_i$ to states in holding class $\mathcal{H}_j$ in $T$ intervals, and $R_{\mathcal{H}_i}(T)$ is the total amount of time the process remained in Markov states belonging to holding class $\mathcal{H}_i$.

It is known however, that the MLE of the exponential distribution is a biased estimate. An alternative to the MLE estimate is the uniformly minimum variance unbiased estimator (UMVUE):
\begin{equation}\label{eq:configdiag}
\widehat{q}(\mathcal{H}_i, \mathcal{H}_i)_{\text{UMVUE}} =  \frac{\sum_{\mathcal{H}_i \neq \mathcal{H}_j} N_{\mathcal{H}_i, \mathcal{H}_j}(T) - 1}{R_{\mathcal{H}_i}(T)}.
\end{equation}
Other estimators include the minimum mean squared error estimator (MMSE). Reference~\ref{cohen1973estimation} has more reviews of the different estimators of the exponential distribution.

\subsection{Finding $\widehat{\theta}$}

Depending on the duration of the observed trajectory, not all transitions between pairs of holding classes will be observed. Therefore, it may be that not $\widehat{q}(\mathcal{H}_1, \mathcal{H}_1), \ldots \widehat{q}(\mathcal{H}_K, \mathcal{H}_K)$ can be estimated. Therefore~\eqref{eq:linearsystem} may be overdetermined or underdetermined. We can account for the differences in estimation quality of each term by using the inverse sample variance. We can estimate $\theta$ by solving either the weighted least squares problem 
\begin{equation}\label{eq:wls}
\widehat{\theta} = \arg \min_{\theta} ||W^{\frac{1}{2}}(F\theta - [\widehat{q}(\mathcal{H}_1, \mathcal{H}_1), \ldots, \widehat{q}(\mathcal{H}_K, \mathcal{H}_K)]^T)||_2,
\end{equation}
or the weighted least deviation problem.
\begin{equation}\label{eq:lad}
\widehat{\theta} = \arg \min_{\theta} ||W^{\frac{1}{2}}(F\theta - [\widehat{q}(\mathcal{H}_1, \mathcal{H}_1), \ldots, \widehat{q}(\mathcal{H}_K, \mathcal{H}_K)]^T)||_1.
\end{equation}
The matrix $W$ is a diagonal weight matrix. The weighted least squares problem has a closed-form solution. The weighted least deviation problem can be solved numerically. Algorithm~\ref{alg} summarizes the approach to estimate $\theta$ from a finite-duration trajectory $\Sigma$. 

\begin{algorithm}[ht]
\SetAlgoLined
\KwResult{$\widehat{\theta}$}
Given continuous-time trajectory $\Sigma = \{x(t_0), x(t_1), x(t_2), \ldots, x(t_{M-1})$, find the holding class for each $X(t_i)$: $\mathcal{H}= \{ \mathcal{H}_1, \mathcal{H}_2, \ldots \mathcal{H}_m\}, m \le K$ to form the matrix $F$, which would be $m \times b$.

\For{$\mathcal{H}_i \in \mathcal{H}$}{
Estimate $\widehat{q}(\mathcal{H}_i, \mathcal{H}_i)$ from the given observations using equation~\eqref{eq:MLEdiagonaltransitionrate2}. Estimate the corresponding sample variance $var(\widehat{q}(\mathcal{H}_i, \mathcal{H}_i))$ to determine the diagonal entries of $W$. 

%From the given observations, find $N_{\mathcal{H}_i}-1$, the total number of transitions in $\Sigma$ from any configuration in holding class $\mathcal{H}_i$. $R_{\mathcal{H}_i}$, the total amount of time the system was in holding class $\mathcal{H}_i$. Estimate $\widehat{q}(\mathcal{H}_i, \mathcal{H}_i)$ from these values.


%	\begin{equation}\label{eq:diagestimate}
%	\widehat{q}(\mathcal{H}_i, \mathcal{H}_i) = \frac{N_{\mathcal{H}_i}-1}{R_{\mathcal{H}_i}}
%	\end{equation}
%	
%	\begin{equation}\label{eq:weight}
%	W(\mathcal{H}_i, \mathcal{H}_i) = \frac{N_{\mathcal{H}_i}-2}{\widehat{q}(\mathcal{H}_i, \mathcal{H}_i)^2}
%	\end{equation}
}
Estimate $\widehat{\theta}$ by solving the weighted least squares~\eqref{eq:wls} or weighted least deviation~\eqref{eq:lad}.  

\caption{Estimate $\theta$ from continuous-time trajectory $\Sigma$}\label{alg}
\end{algorithm}

%%%%%%%%%%%%%%%%%%%%%%%%%%%
%%%%%%%%%%%%%%%%%%%%%%%%%%%


\section{Contact Process}\label{sec:contact}

Interacting particle systems (IPS) models random interactions amongst $N$ particles (this paper assumes that $N$ is finite). Without interactions, the system would consist of $N$ independent continuous-time Markov processes. With interactions, the dynamic of 
dynamics of the particles becomes coupled and the evolution of the individual particles loses their Markovian property. 

Typically, the structure of interaction is characterized by an unweighted, undirected graph $G(V,E)$, where $|V| = N$. We will call this the \emph{contact network}. The most well-known IPS model is the contact process, which models infection and healing of the particles (i.e., nodes) according to the network-based susceptible-infected-susceptible (SIS) epidemics framework. IPS models such as the reversible contact process and the dynamic bond percolation processes have also been proposed and studied \cite{PhysRevE.86.016116, JZhang}. 

A Markov state $i \in \mathcal{D} = \{1, 2, \ldots D\}$ is a vector whose components describes the state of each node; we will refer to the state as a \emph{configuration} when we wish to emphasize the graphical nature of the system. The contact process assumes that each node can be in one of two states, $\{0,1\}$, representing healthy or infected state respectively. At time $t$, the network configuration is
\[
\mathbf{x}(t) = [x_1(t), x_2(t), \ldots x_N(t)]^T, \text{ where } x_i(t) = \{0,1\}.
\]
We see then that $D = 2^N$. The size of the state space is exponential in the number of nodes in the contact network. 

The contact process assumes that multiple nodes can not change states simultaneously. There are two types of transitions: 1) healing of infected agents and 2) infection of susceptible agents.

\begin{enumerate}
\item
Consider a configuration
\[ 
\mathbf{x} = [x_1,x_2, \ldots, x_j = 1, x_{j+1}, \ldots x_N]^T.
\] 
Let $T^-_j\mathbf{x}$ be the configuration where the $j$th node heals: 
\[
T^-_j\mathbf{x} =  [x_1,x_2, \ldots, x_j = 0, x_{j+1}, \ldots x_N]^T.
\] 
The transition rate from state $\mathbf{x}$ to $T^-_j\mathbf{x}$ is
\begin{equation}\label{eq:contactheal}
q(\mathbf{x}, T^-_j\mathbf{x}) = \mu,
\end{equation}
where $\mu \ge 0$ is the \emph{healing rate} (num. of healing events/unit time). If infected nodes can not heal, then $\mu = 0$.

\item Consider a configuration 
\[
\mathbf{x} = [x_1,x_2, \ldots, x_{k-1}, x_k = 0, \ldots x_N]^T.
\]
Let $T^+_k\mathbf{x}$ be the configuration where the $k$th node becomes infected:
\[
T^+_k\mathbf{x} = [x_1,x_2, \ldots, x_{k-1}, x_k = 1, \ldots x_N]^T.
\]
The transition rate from state $\mathbf{x}$ to $T^+_k\mathbf{x}$ is
\begin{equation}\label{eq:contactinfect}
q(\mathbf{x}, T^+_k\mathbf{x}) = \beta + \delta m_k, 
\end{equation}
where $m_k$ is the number of infected neighbors of node $k$ in configuration $\mathbf{x}$. Let $A = [A_{ik}]$ be the adjacency matrix of the contact network $G(V,E)$, then
\begin{equation}\label{eq:numinfneighbors}
m_k = \sum_{i =1}^N x_iA_{ik}.
\end{equation}

The parameter $\beta \ge 0$ is the \emph{exogeneous infection rate} (num. of infection from outside the network/unit time) and $\delta \ge 0$ is the \emph{endogeneous infection rate} (num. of contagion from infected neighbors/unit time). When $\delta> 0$, the infection rate of the contact process is linearly dependent on the number of infected neighbors. If susceptible nodes can only be infected by neighboring infected nodes (i.e., there can be no infection from outside the network), then $\beta= 0$.
\end{enumerate}


While the transition rate matrix, $Q$, of the contact process is $2^N \times 2^N$, we see can see from~\eqref{eq:contactheal} and~\eqref{eq:contactinfect} that the transition rates are linear functions of only three parameters: healing rate, $\mu$, exogeneous infection rate, $\beta$, and endogeneous infection rate, $\delta$.


%%%%%%%%%%%%%%%%%%%%%%%%%%%%%%%%%%
%%%%%%%%%%%%%%%%%%%%%%%%%%%%%%%%%%
%%%%%%%%%%%%%%%%%%%%%%%%%%%%%%%%%%
\subsection{Upperbound on the Number of Holding Classes}



\begin{lemma}\label{prop:1}
Let $\mathcal{S}(\mathbf{x}) \subset V$ denote the set of susceptible nodes in a configuration $\mathbf{x}$. For the contact process, two different configurations $\mathbf{x}$ and $\mathbf{x}'$ belongs to the same holding class, $\mathcal{H}$, if and only if
\begin{equation}\label{eq:contactcond1}
| \mathcal{S}(\mathbf{x}) | = |\mathcal{S}(\mathbf{x}') |, \text{ and}
\end{equation}
\begin{equation}\label{eq:contactcond2}
\sum_{k \in \mathcal{S}(\mathbf{x})} m_k = \sum_{k \in \mathcal{S}(\mathbf{x}')} m_k,
\end{equation}
where $m_k$ is the total number of infected neighbors of node $k$. For the contact process, the holding classes are determined by the total number of susceptible nodes and the sum of infected neighbors of all the susceptible nodes.  
\end{lemma}

\begin{proof}
For two Markov states, corresponding to configurations $\mathbf{x}$ and $\mathbf{x}'$, the diagonal entries of the transition rate matrix $Q$ are 
\begin{align*}
&q(\mathbf{x}, \mathbf{x}) = -\sum_{\tilde{\mathbf{x}} \in \mathcal{D}} q(\mathbf{x}, \tilde{\mathbf{x}})\\
&= -\left((N- | \mathcal{S}(\mathbf{x}) |)\mu + (| \mathcal{S}(\mathbf{x}) |) \beta +   \left(\sum_{k \in \mathcal{S}(\mathbf{x})} m_k\right)\delta \right)
\end{align*}
and
\begin{align*}
&q(\mathbf{x}', \mathbf{x}') = -\sum_{\tilde{\mathbf{x}} \in \mathcal{D}} q(\mathbf{x}', \tilde{\mathbf{x}})\\
&= -\left((N- | \mathcal{S}(\mathbf{x}') |)\mu + (| \mathcal{S}(\mathbf{x}') |) \beta +   \left(\sum_{k \in \mathcal{S}(\mathbf{x}')} m_k\right)\delta \right).
\end{align*}

When equations~\eqref{eq:contactcond1} and~\eqref{eq:contactcond2} are true, then $q(\mathbf{x}, \mathbf{x}) = q(\mathbf{x}', \mathbf{x}')$. By definition~\ref{def:holdingclass}, Markov states $\mathbf{x}
$ and $\mathbf{x}'$ belongs to the same holding class, $\mathcal{H}$.
\end{proof}

Using Preposition~\ref{prop:1}, we can partition the $2^N$-sized state space of the contact process into holding classes $\{\mathcal{H}_1 \cup \mathcal{H}_2 \cup \ldots  \cup\mathcal{H}_K\}$. The total number of holding classes, $K$, depends on \emph{both} the infection/healing rates and the structure of the contact network $G(V,E)$. It is not easy to find $K$ without enumerating over all $2^N$ possible configurations. However, we can derive an upperbound on $K$ to see that it is much smaller than $2^N$.


%%%%%%%%%%%%%%%

\begin{theorem}\label{thm:1}
For a contact process with contact network $G(V,E), |V| = N$ and dynamics parameter $\theta$, the number of holding class 
\[
K \le \frac{(N+1)(N^2 - N +6)}{6}.
\]
\end{theorem}

\begin{proof}
Consider Lemma~\ref{prop:1}. If the holding class is only determined by~\eqref{eq:contactcond1}, then there would be $N+1$ different holding classes corresponding to $|\mathcal{S}(\mathbf{x})| = 0, 1, \ldots N$. It is intuitive that the number of holding classes should be much larger than $O(N)$. Therefore, the number of holding classes, $K$, is determined by how many possible unique values $\sum_{k \in \mathcal{S}(\mathbf{x})} m_k$ may take for all $\mathbf{x} \in \mathcal{D}$. 

Assuming that there are $|\mathcal{S}(\mathbf{x})|$ susceptible nodes in a configuration $\mathbf{x}$, this means that there are $N- |\mathcal{S}(\mathbf{x})|$ total number of infected nodes. Without any additional knowledge of the structure of $G(V,E)$, we can conclude that that for any node $k \in \mathcal{S}(\mathbf{x})$, $m_k$ can be take on \emph{at most} $N- |\mathcal{S}(\mathbf{x})| +1 $ different values (i.e.,$\{0, 1, 2, \ldots N- |\mathcal{S}(\mathbf{x})|\}$). Then,

% $\sum_{k \in \mathcal{S}(\mathbf{x})} m_k$ can take on \emph{at most} $|\mathcal{S}(\mathbf{x})|(N- |\mathcal{S}(\mathbf{x})|) +1$ different values. To consider the entire state space $\mathcal{D}$, we need to consider all possible sized set of susceptible nodes from $0$ to $N$.


%Assuming that there are $|\mathcal{S}(\mathbf{x})|$ susceptible nodes in a configuration $\mathbf{x}$, this means that there are $N- |\mathcal{S}(\mathbf{x})|$ total number of infected nodes. Without any additional knowledge of the structure of $G(V,E)$, we can conclude that that $m_k \le N-s$ for any node $k \in \mathcal{S}(\mathbf{x})$. 

%\[
%\sum_{k \in \mathcal{S}(\mathbf{x})} m_k \le |\mathcal{S}(\mathbf{x})|(N-|\mathcal{S}(\mathbf{x})|) +1.
%\]
%Since $\sum_{k \in \mathcal{S}(\mathbf{x})} m_k$ can only take on nonnegative integer values, this means that the summation can take on  $|\mathcal{S}(\mathbf{x})|(N-|\mathcal{S}(\mathbf{x})|)+1$ possible values for a fixed $|\mathcal{S}(\mathbf{x})|$. Sum over all possible values of $|\mathcal{S}(\mathbf{x})|$ from $0$ to $N$, then
%\[
%K \le  \sum_{|\mathcal{S}(\mathbf{x})|=0}^N |\mathcal{S}(\mathbf{x})|(N-|\mathcal{S}(\mathbf{x})|).
%\]


Let $|\mathcal{S}(\mathbf{x})| = s$, summing over all possible values of $s$ from $0$ to $N$,
\begin{align}\label{eq:contactsum}
\sum_{s=0}^N &s(N-s) +1 = N\sum_{s=0}^N s -  \sum_{s=0}^N s^2 +  \sum_{s=0}^N 1\\
&\medmath{= N\left(\frac{N(N+1)}{2}\right) - \frac{N(N+1)(2N+1)}{6} +(N+1)}\\
&\medmath{=\frac{(N+1)(N^2 - N +6)}{6}}.
\end{align}
Therefore
\[
 K \le \frac{(N+1)(N^2 - N +6)}{6}.
\]


\end{proof}


Theorem~\ref{thm:1} shows that the number of holding classes can not be more than cubic in the number of nodes, $N$. This makes $K$ much smaller than $D = 2^N$. 

%%%%%%%%%%%%%%%%%%%%%%%%%%
 \begin{corollary}
For a contact process with contact network $G(V,E), |V| = N$ and dynamics parameter $\theta$, if we know that $G(V,E)$ is a degree-bounded graph such that all the nodal degrees are less than or equal to $d_{\max}$, then the number of holding class 
\begin{equation}\label{eq:upperbound}
K \le \sum_{s = 0}^N s(\min(N-s, d_{\max}))+1 \le \frac{(N+1)(N^2 - N +6)}{6}.
\end{equation}

 \end{corollary}


\begin{proof}
From the proof of Theorem~\ref{thm:1}, for a given number of susceptible nodes, $s$, we assumed that the number of infected neighbors of a susceptible node can range from $0, 1, \ldots, N-s$. When the contact network has bounded degree, we know that the number of infected neighbors of a susceptible node can range from $0,1, \ldots, \min(N-s, d_{\max})$. Therefore, \eqref{eq:contactsum} becomes
\begin{align*}
\sum_{s=0}^N s(\min(N-s, d_{\max})) + 1.
\end{align*}
\end{proof}

%%%%%%%%%%%%%%%%%%%%%%%%%%
 \begin{lemma}
For a contact process with contact network $G(V,E), |V| = N$ and dynamics parameter $\theta$, if we know that $G(V,E)$ is a complete graph, then the number of holding class 
\[
K = N+1.
\]
 \end{lemma}
\begin{proof}
Let $\mathbf{x}$ and $\mathbf{x}'$ denote any two configurations such that $|\mathcal{S}(\mathbf{x})| = |\mathcal{S}(\mathbf{x}')|$. Then we know that 
\[
\sum_{s \in \mathcal{S}(\mathbf{x})} m_s = \sum_{s \in \mathcal{S}(\mathbf{x'})} m_s = s(N-s), \forall \mathbf{x}, \mathbf{x}'.
\]
Since $|\mathcal{S}(\mathbf{x})|$ can be $0,1,\dots N$, the number of holding classes is $N+1$.

\end{proof}

%%%%%%%%%%%%%%%%%%%%%%%%%%%
%%%%%%%%%%%%%%%%%%%%%%%%%%%
\section{Reversible Contact Process}\label{sec:rev}

In the contact process, we see that the infection rate of a susceptible node is linearly dependent on the number infected neighbors. In~\cite{Zhang2014}, a process similar to the contact process (called scaled SIS process) was analyzed. In this model, the infection rate is exponentially dependent on the number of infected neighbors. This modification made the underlying continuous-time Markov process a \emph{reversible} process: a stochastic process that is statistically the same forward and backward in time. Additionally, the equilibrium distribution of the reversible contact process can be derived in closed-form. 

The transition rates of the reversible contact process are
\begin{enumerate}
\item 
\begin{equation}\label{eq:scaledheal}
q(\mathbf{x}, T^-_j\mathbf{x}) = \mu,
\end{equation}
where $\mu > 0$ is the \emph{healing rate} (\# of healing events/unit time). The reversible contact process can not have $\mu = 0$, or the reversibility property would be lost. 

\item 
\begin{equation}\label{eq:scaledinfect}
q(\mathbf{x}, T^+_k\mathbf{x}) = \beta(\delta)^{m_k}, 
\end{equation}
where $m_k$ is the number of infected neighbors of node $k$ in configuration $\mathbf{x}$. Like in the contact process, we can think of $\beta >0$ as the exogeneous infection rate (\# of infected events/unit). Unlike the contact process, $\delta$ is not a rate but a unitless factor. We see that the number of infected neighbors, $m_k$, induces a scaling of the infection rate from $\beta$. This scaling has increases the infection rate with increasing $m_k$ if $\delta > 1$ and decreases infection rate if $\delta < 1$. The reversible contact process can not have $\beta =0$ or the reversibility property would be lost. 

\end{enumerate}

The transition rate matrix, $Q$, of the reversible contact process is $2^N \times 2^N$. We can see from~\eqref{eq:scaledheal} and~\eqref{eq:scaledinfect} that unlike the contact process, the transition rates are not linear functions of $\mu, \beta, \delta$. Instead, the transition rates can be written as functions of $\theta = [\mu, \beta, \beta\delta, \beta\delta^2, \ldots \beta\delta^{\text{dmax}}]^T$, where $dmax$ is the maximum degree of $G(V,E)$. A simple additional step is needed after Algorithm~\ref{alg} to compute $\delta$ from $\beta\delta, \ldots \beta\delta^{\text{dmax}}$, but this may induce additional errors.


%%%%%%%%%%%%%%%%%%%%%%%%%
%%%%%%%%%%%%%%%%%%%%%%%%%
%%%%%%%%%%%%%%%%%%%%%%%%%
\subsection{Upperbound on the Number of Holding Classes}

%We assume that the reversible contact process $X(t)$ has been continuously observed for some time interval $[t_0, t_{M-1}]$:   
%\[
%\Sigma = \{ \mathbf{x}(t_0),  \mathbf{x}(t_1), \ldots, \mathbf{x}(t_{M-1}) \}
%\]
%To use Algorithm~\ref{alg}, we need to `lift' the to estimate $\theta = [\mu, \beta, \beta\delta, \beta\delta^2, \ldots \beta\delta^{\text{dmax}}]^T$, we need to determine the holding classes of the reversible contact process.

%Proofs for the section can be found in the Appendix as they follow the same logic as those in Section~\ref{sec:contact}.

\begin{lemma}\label{lemmascaled}
Let $\mathcal{S}(\mathbf{x}) \subset V$ denote the set of susceptible nodes in a configuration $\mathbf{x}$. For the reversible contact process, two different configurations $\mathbf{x}$ and $\mathbf{x}'$ belongs to the same holding class if and only if
\begin{equation}\label{eq:contactcond3}
| \mathcal{S}(\mathbf{x}) | = |\mathcal{S}(\mathbf{x}') |, \text{ and}
\end{equation}
\begin{equation}\label{eq:contactcond4}
\{m_k: k \in S(\mathbf{x})\} = \{m_k: k \in S(\mathbf{x}')\},
\end{equation}
where $m_k$ is the number of infected neighbors of node $k$. The set $\{m_k: k \in S(\mathbf{x})\}$ is the set of the number of infected neighbors of each susceptible node in configuration $\mathbf{x}$. Equality in~\eqref{eq:contactcond4} is set equality; the ordering of the values do not matter.
\end{lemma}
%%%%%%%%%%%%%%%%%%
\begin{proof}
For two Markov states, corresponding to configurations $\mathbf{x}$ and $\mathbf{x}'$, the diagonal entries of the transition rate matrix $Q$ are 
\begin{align*}
&q(\mathbf{x}, \mathbf{x}) = -\sum_{\tilde{\mathbf{x}} \in \mathcal{D}} q(\mathbf{x}, \tilde{\mathbf{x}})\\
&\medmath{= -((N- | \mathcal{S}(\mathbf{x}) |)\mu +  |{ s \in S(\mathbf{x}): m_s = 0 }|(\beta)}+\\
&\medmath{ |\{s \in S(\mathbf{x}): m_s = 1\}|(\beta\delta) \ldots + |\{s \in S(\mathbf{x}): m_s = d_{\max}\}|(\beta\delta^{d_{\max}}))}, 
\end{align*}
and
\begin{align*}
&q(\mathbf{x}', \mathbf{x}') = -\sum_{\tilde{\mathbf{x}} \in \mathcal{D}} q(\mathbf{x}', \tilde{\mathbf{x}})\\
&= \medmath{-((N- | \mathcal{S}(\mathbf{x}') |)\mu +  (|{ s \in S(\mathbf{x}'): m_s = 0 }|)\beta}+\\
&\medmath{(|\{s \in S(\mathbf{x}'): m_s = 1\}|)\beta\delta \ldots + (|\{s \in S(\mathbf{x}'): m_s = d_{\max}\}|)\beta\delta^{d_{\max}}}.
\end{align*}

When equations~\eqref{eq:contactcond3} and \eqref{eq:contactcond4} are true, then $q(\mathbf{x}, \mathbf{x}) = q(\mathbf{x}', \mathbf{x}')$. By Definition~\ref{def:holdingclass}, Markov state $\mathbf{x}
$ and $\mathbf{x}'$ belongs to the same holding class, $\mathcal{H}$.
\end{proof}
%%%%%%%%%%%%%%%%%%

In contrast the contact process, the holding classes of the reversible process is determined by the number of infected neighbors of each susceptible node instead of simply by the total number of infected neighbors. Consequently, the number of holding classes in the reversible process is larger than the number of holding classes in the contact process. Reversible contact processes also have a larger number of values to estimate $\theta = [\mu, \beta, \beta\delta, \beta\delta^2, \ldots \beta\delta^{\text{dmax}}]^T$. It may be useful then, to approximate the reversible process by the contact process to efficiently estimate the healing and infection rates. Reference~\ref{Zhang2017} gives some insight as to when the two processes are equivalent.


\begin{theorem}\label{thm:revtran}
For a reversible contact process with interaction network $G(V,E), |V| = N$ and dynamics parameter $\theta$, the number of holding class 
\begin{equation}\label{eq:revbound}
K \le 2^N.
\end{equation}
\end{theorem}

%%%%%%%%%%%%%%
\begin{proof}
If the holding class is only determined by~\eqref{eq:contactcond3}, then there would be $N+1$ different holding classes corresponding to $|\mathcal{S}(\mathbf{x})| = 0, 1, \ldots N$. It is intuitive that the number of holding classes should be much larger than $O(N)$. Therefore, the number of holding classes, $K$, is determined by the number of unique sets $\{m_k: k \in S(\mathbf{x})\}$.

Assuming that there are $|\mathcal{S}(\mathbf{x})|$ susceptible nodes in a configuration $\mathbf{x}$, this means that there are $N- |\mathcal{S}(\mathbf{x})|$ total number of infected nodes. Without any additional knowledge of the structure of $G(V,E)$, we can conclude that that $m_k$ can range in value from $\{0,1, \ldots N- |\mathcal{S}(\mathbf{x})|\}$ for any node $k \in \mathcal{S}(\mathbf{x})$. The number of possible unique set $\{m_k: k \in S(\mathbf{x})\}$ is a combination with replacement problem where we want to choose $N-|\mathcal{S}(\mathbf{x})|+1$ values for $|\mathcal{S}(\mathbf{x})|$ entires. 

Let $|\mathcal{S}(\mathbf{x})| =s$. The number of possibilities is
\[
\frac{(N-s+1+s - 1)!}{s!(N-s+1-1)!} = \frac{N!}{s!(N-s)!}.
\]
Summing over all possible number of susceptible nodes results in
\begin{equation}\label{eq:revsum}
\sum_{s=0}^N \frac{N!}{s!(N-s)!}.
\end{equation}
By the binomial theorem, we know that~\eqref{eq:revsum} is equal to $2^N$.
\end{proof}
%%%%%%%%%%%%%%




 \begin{corollary}\label{coro:revtran}
For a reversible contact process with interaction network $G(V,E), |V| = N$ and dynamics parameter $\theta$, if we know that $G(V,E)$ is a degree-bounded graph such that all the nodal degrees are less than or equal to $d_{\max}$, then the number of holding class 
\[
K \le \sum_{s=0}^{N - d_{\max}}  \frac{(d_{\max} +s)!}{s!(d_{\max})!}  + \sum_{s = N -d_{\max}+1}^{N} \frac{N!}{s!(N-s)!}.
\]
 \end{corollary}

%%%%%%%%%%%%%%%%%%
\begin{proof}
From the proof of Theorem~\ref{thm:revtran}, for a given number of susceptible nodes, $s$, we assumed that the number of infected neighbors of a susceptible node can range from $0, 1, \ldots, N-s$. When the interaction network has bounded degree, we know that the number of infected neighbors of a susceptible node can range from $0,1, \ldots, \min(N-s, d_{\max})$. Therefore, the sum~\eqref{eq:revsum} becomes
\begin{align*}
&\sum_{s=0}^N \frac{N!}{s!(N-s)!}\\ 
&= \medmath{\sum_{s=0}^{N - d_{\max}}  \frac{(d_{\max} +s)!}{s!(d_{\max})!}      + \sum_{N -d_{\max}+1}^{N} \frac{(N-s+1+s - 1)!}{s!(N-s+1-1)!}}\\
& = \sum_{s=0}^{N - d_{\max}}  \frac{(d_{\max} +s)!}{s!(d_{\max})!}      + \sum_{s = N -d_{\max}+1}^{N} \frac{N!}{s!(N-s)!}
\end{align*}

\end{proof}
%%%%%%%%%%%%%%%%%%

%%%%%%%%%%%%%%%%%%%%%%%%%
%%%%%%%%%%%%%%%%%%%%%%%%%
%%%%%%%%%%%%%%%%%%%%%%%%%
\section{Number of Holding Classes and Contact Network Structure}\label{sec:numholdingexp}
 \begin{figure*}[ht]
\begin{subfigure}{.5\textwidth}
  \centering
  % include first image
  \includegraphics[width=0.7\linewidth]{./images/contact_numholding-eps-converted-to}  
  \caption{Line indicates the theoretical upperbound~\eqref{eq:upperbound}}
  \label{fig:contactclass}
\end{subfigure}
\begin{subfigure}{.5\textwidth}
  \centering
  % include first image
  \includegraphics[width=0.7\linewidth]{./images/contact_numholdingratio-eps-converted-to}  
  \caption{Line indicates the theoretical upperbound~\eqref{eq:upperbound} divided by $D=2^N$}
  \label{fig:contactclassratio}
\end{subfigure}
\caption{Contact Process Num. of Holding Classes $(K)$ and Ratio to State Space Size $(K/D)$}
%\label{fig:contactclass}
\end{figure*} 
  
 
 \begin{figure*}[ht]
\begin{subfigure}{.5\textwidth}
  \centering
  % include first image
  \includegraphics[width=0.7\linewidth]{./images/rev_numholding-eps-converted-to}  
  \caption{Line indicates the theoretical upperbound~\eqref{eq:revbound}}
  \label{fig:revclass}
\end{subfigure}
\begin{subfigure}{.5\textwidth}
  \centering
  % include first image
  \includegraphics[width=0.7\linewidth]{./images/rev_numholdingratio-eps-converted-to}  
  \caption{Line indicates the theoretical upperbound~\eqref{eq:revbound} divided by $D= 2^N$}
  \label{fig:revclassratio}
\end{subfigure}
\caption{Reversible Contact Process Num. of Holding Classes $(K)$ and Ratio to State Space Size $(K/D)$}
%\label{fig:contactER100result}
\end{figure*}
 
The total number of holding classes, $K$, depends on the topology of the contact network, $G(V,E)$ and the formulation of the transition rates. For small sized networks, we can numerically compute the number of holding classes. First, we generated 50 different Erd\H{o}s-R\'{e}nyi (ER) graphs with the same number of nodes ($N$), but randomly chosen edge probability $p \sim U\left(\frac{\log N}{N}, 0.2\frac{\log N}{N}\right)$ assuming the contact process dynamics (\ref{eq:contactheal},~\ref{eq:contactinfect}) and the reversible contact process dynamics (\ref{eq:scaledheal},~\ref{eq:scaledinfect}). Then we generated 50 different Watts-Strogatz (WS) graphs of the same size $N$, but randomly chosen neighbor size, $nei \sim U\left(3, \frac{N}{2}\right)$ and rewiring probability $p \sim U(0.2, 0.8)$.

Figures~\ref{fig:contactclass} and~\ref{fig:revclass} show the total number of holding classes $K$ for different network size $N$ assuming the contact process and reversible contact process dynamics, respectively. The line shows the theoretical upperbound in~\eqref{eq:upperbound} and~\eqref{eq:revbound}. The actual number of holding classes is much smaller than the theoretical upperbound for both ER and WS graphs. Figures~\ref{fig:contactclassratio} and~\ref{fig:revclassratio} show the ratio of the number of holding class to the state space dimensionality. For the contact process, the ratio $\frac{K}{D}$ provably \emph{decreases} with increasing $N$ since the upperbound is polynomial instead of exponential in $N$. We see that this remains true for the reversible contact process experimentally. 




\section{Parameter Estimation Experiments}\label{sec:exp}

%We assume that the contact process $X(t)$ has been continuously observed for some time interval $[t_0, t_{M-1}]$:   
%\[
%\Sigma = \{ \mathbf{x}(t_0),  \mathbf{x}(t_1), \ldots, \mathbf{x}(t_{M-1}) \}
%\]
%We use $\mathbf{x}(t)$ to emphasize that each an observation of the contact process at time $t$ is vector consisting of the states of all the particles in the system. To use Algorithm~\ref{alg} to estimate $\theta = [\mu, \beta, \delta]^T$, we need to determine the holding classes of the contact process.


We will study the performance of Algorithm~\ref{alg} for estimating the dynamics parameters for both the contact process and the reversible contact process using generated trajectories on synthetic and real-world graphs. We will look at two types of error: absolute error and relative error. The mean absolute error (MAE) is the absolute difference between the true healing rate, endeogenous infection rate, exogeneous infection rate ($\mu, \beta, \delta$) and the estimates ($\widehat{\mu}, \widehat{\beta}, \widehat{\delta}$): 
\[
\mu_\text{MAE} = \frac{1}{L} \sum_{i=1}^L |\widehat{\mu}_i- \mu_i |
\]

%\[
%\beta_\text{MAE} = \frac{1}{L} \sum_{i=1}^L |\widehat{\beta}_i- \beta_i|    
%\]
%
%\[
%\delta_\text{MAE} = \frac{1}{L} \sum_{i=1}^L |\widehat{\delta}_i- \delta_i |    
%\]

However, absolute error can be misleading when the true value is close to zero. Relative error compares the absolute error to the true and estimated values. We use symmetric mean absolute percentage error (SMAPE) \cite{flores1986pragmatic}:

\[
\mu_\text{SMAPE} = \frac{100\%}{L}\sum_{i=1}^{L} \frac{|\widehat{\mu}_i- \mu_i    |}{|\widehat{\mu}_i| + |\mu_i|},
\]
%
%\[
%\beta_\text{SMAPE} = \frac{100\%}{L}\sum_{t=1}^{L} \frac{|\widehat{\beta}_i- \beta_i   |}{|\widehat{\beta}_i| + |\beta_i|},
%\]
%
%\[
%\delta_\text{SMAPE} = \frac{100\%}{L}\sum_{t=1}^{L} \frac{|\widehat{\delta}_i- \delta_i    |}{|\widehat{\delta}_i| + |\delta_i|},
%\]
which is bounded between $0\%$ and $100\%$. A second reason that we used SMAPE is because an estimated value of $0$ when the true value is nonzero will have $100\%$ error. This is particularly important for learning parameters of graph dynamical systems because the parameters of interests are rates, which can be small but is generally nonzero. 


%We will consider the error between the true parameter $\theta = [\mu, \beta, \delta]$ and the estimate $\widehat{\theta}= [\widehat{\mu}, \widehat{\beta}, \widehat{\delta}]$ as:
%\[
%\theta_\text{error} = ||\theta - \widehat{\theta}||_2.
%\]





\subsection{Parameter Estimation Performance for Contact Processes}

\subsubsection{\textbf{Synthetic Erd\H{o}s-R\'{e}nyi (ER) Random Graph}}

We generated fifty 100-node unweighted, undirected, connected Erd\H{o}s-R\'{e}nyi (ER) graphs. The state space of the contact process is $D = 2^{100}\approx 1.27e^{30}$. Per Theorem~\ref{thm:1}, we know that the maximum number of holding classes can be is 166,751. For each graph, a \emph{single} contact process trajectory of length $M$ was simulated, $\Sigma = \{\mathbf{x}(t_0), \ldots \mathbf{x}(t_{M-1})\}$. 

%In addition of the structure of the contact network, the total number of observed holding classes in a finite duration also depends on infection/healing rates since some holding classes may not show up due to small likelihood governed by the contagion dynamics

The initial configuration, $\mathbf{x}(t_0)$, is randomly generated; a probability $p_0$ is chosen from Uniform(0,1). The state of each node is independently assigned as a Bernoulli random variable with probability $p_0$. The infection and healing rates, $\mu, \beta, \delta$ are chosen independently from Uniform(0,3).

Given $\Sigma = \{\mathbf{x}(t_0), \ldots \mathbf{x}(t_{M-1})\}$, we used Algorithm~\ref{alg}. We used three different methods to find $\widehat{\theta}$: 1) weighted least squares (WLS), which solves~\eqref{eq:wls}; 2) non-negative (constrained) weighted least squares (NNLS), which solves~\eqref{eq:wls} but with the additional constraint that the estimated values need to be non-negative; 3) weighted least absolute deviation (LAD), which solves the weighed least absolute deviation problem~\eqref{eq:lad}.

\begin{figure*}[ht]
\begin{subfigure}{.5\textwidth}
  \centering
  % include first image
  \includegraphics[width=0.95\linewidth]{./images/contact_MAE_mu-eps-converted-to}  
  \caption{MAE of Healing Rate $\mu$}
  \label{fig:contact_MAE_mu}
\end{subfigure}
\begin{subfigure}{.5\textwidth}
  \centering
  % include first image
  \includegraphics[width=0.95\linewidth]{./images/contact_SMAPE_mu-eps-converted-to}  
  \caption{SMAPE of Healing Rate $\mu$}
  \label{fig:contact_SMAPE_mu}
\end{subfigure}
\hfill

\begin{subfigure}{.5\textwidth}
  \centering
  % include second image
  \includegraphics[width=0.95\linewidth]{./images/contact_MAE_lambda-eps-converted-to}  
  \caption{MAE of Exogeneous Infection Rate $\beta$}
  \label{fig:contact_MAE_lambda}
\end{subfigure}
\begin{subfigure}{.5\textwidth}
  \centering
  % include second image
  \includegraphics[width=0.95\linewidth]{./images/contact_SMAPE_lambda-eps-converted-to}  
  \caption{SMAPE of Exogeneous Infection Rate $\beta$}
  \label{fig:contact_SMAPE_lambda}
\end{subfigure}
\hfill

\begin{subfigure}{.5\textwidth}
  \centering
  % include second image
  \includegraphics[width=0.95\linewidth]{./images/contact_MAE_gamma-eps-converted-to}  
  \caption{MAE of Endogeneous Infection Rate $\delta$}
  \label{fig:contact_MAE_gamma}
\end{subfigure}
\begin{subfigure}{.5\textwidth}
  \centering
  % include second image
  \includegraphics[width=0.95\linewidth]{./images/contact_SMAPE_gamma-eps-converted-to}  
  \caption{SMAPE of Endogeneous Infection Rate $\delta$}
  \label{fig:contact_SMAPE_gamma}
\end{subfigure}
\caption{Contact Process: Error vs. Trajectory Length ($M$) for Erd\H{o}s-R\'{e}nyi Graphs}
\label{fig:contactER100result}
\end{figure*}


Figure~\ref{fig:contactER100result} shows the MAE and SMAPE error for the healing rate, endogeneous infection rate, and exogeneous infection rate. The black line indicates the standard deviation. The error does not decrease with increasing number of observations. Partly, this is because successive observations in the trajectory are \emph{not independent}. As a result the error is more influenced by the choice of the initial state and the topology of the $G(V,E)$. 

Looking at the absolute error, we can see that the standard deviation can sometimes be large be larger than the mean value. This means that the estimate is overdispersed. The weighted least squares (WLS) estimator has the largest error, but generally has the smallest standard deviation. If we consider relative error, all three estimator perform similarly. The reason that SMAPE of non-negative least squares (NNLS) and least absolute deviation (LAD) estimators increased compare to WLS is because both estimators may give results that are exactly zero. Therefore, they are penalized by SMAPE, which assigns a 100\% error to an estimate that is exactly zero. From this perspective, the performance of the WLS estimator is comparable to both NNLS and LAD.  


The estimate of the healing rate always give the lowest error compared to estimates of infection rates. This is because there is only one type of healing events so there are comparatively more observations for estimating the healing rate. For the contact process, the error for the endogeneous infection rate, $\delta$ is smaller than the error for the exogeneous infection rate, $\beta$. One reason may be that direct observation of a node becoming infected via exogeneous infection (i.e., infection from outside the network) is rarely observed. Therefore, it can be difficult to disentangle $\beta$ from $\delta$. If we consider SMAPE, then NNLS estimator gives the lowest error for the endogeneous infection rate, while the LAD estimator gives the lowest error for the exogeneous infection rate.



\subsubsection{\textbf{118-Bus Power Grid Graph}}


\begin{figure}[h] 
\includegraphics[width=0.3\textwidth]{./images/118bus-eps-converted-to}
\centering
 \caption{Network Visualization of 118-bus Power Flow Test System}
 \label{fig:118bus}
\end{figure}

In this section, we consider a real-world network, the IEEE 118-bus Power Flow Test System from \cite{christie1993power}. The nodes (N = 118) of the network correspond to the bus in the power grid, which is the location where a line or several lines connects at. The number of edges is 179. The maximum degree of the network is 9.  A node can be in the failed ($x_i =1$) or working ($x_i = 0)$ state. The graph representation of the system is shown in Figure~\ref{fig:118bus}.

For this graph, \emph{fifty} independent trajectories of length $M$, $\Sigma = \{\mathbf{x}(t_0), \ldots \mathbf{x}(t_{M-1})\}$ from the contact process was simulated. The initial configuration $\mathbf{x}(t_0)$ and dynamics parameters, $\mu, \beta, \delta$, are chosen randomly using the same procedure described previously. Figure~\ref{fig:contactpowerresult} shows the MAE and SMAPE error for $\mu, \beta, \delta$. Again we see that the length of the observed trajectory do not have much effect on the error. The error is smaller compared to experiments involving random graphs. This suggest that the topology of the network has a large impact on how well parameters can be learned from data.


%Similarly, the NNLS estimator gives the lowest error for the endogeneous infection rate while the LAD estimator gives the lowest error for the exogeneous infection rate.



\begin{figure*}[ht]
\begin{subfigure}{.5\textwidth}
  \centering
  % include first image
  \includegraphics[width=0.95\linewidth]{./images/real_contact_MAE_mu-eps-converted-to}  
  \caption{MAE of Healing Rate $\mu$}
  \label{fig:118buscontact_MAE_mu}
\end{subfigure}
\begin{subfigure}{.5\textwidth}
  \centering
  % include first image
  \includegraphics[width=0.95\linewidth]{./images/real_contact_SMAPE_mu-eps-converted-to}  
  \caption{SMAPE of Healing Rate $\mu$}
  \label{fig:118buscontact_SMAPE_mu}
\end{subfigure}
\hfill

\begin{subfigure}{.5\textwidth}
  \centering
  % include second image
  \includegraphics[width=0.95\linewidth]{./images/real_contact_MAE_lambda-eps-converted-to}  
  \caption{MAE of Exogeneous Infection Rate $\beta$}
  \label{fig:ER100SMAPE}
\end{subfigure}
\begin{subfigure}{.5\textwidth}
  \centering
  % include second image
  \includegraphics[width=0.95\linewidth]{./images/real_contact_SMAPE_lambda-eps-converted-to}  
  \caption{SMAPE of Exogeneous Infection Rate $\beta$}
  \label{fig:ER100SMAPE}
\end{subfigure}
\hfill

\begin{subfigure}{.5\textwidth}
  \centering
  % include second image
  \includegraphics[width=0.95\linewidth]{./images/real_contact_MAE_gamma-eps-converted-to}  
  \caption{MAE of Endogeneous Infection Rate $\delta$}
  \label{fig:ER100SMAPE}
\end{subfigure}
\begin{subfigure}{.5\textwidth}
  \centering
  % include second image
  \includegraphics[width=0.95\linewidth]{./images/real_contact_SMAPE_gamma-eps-converted-to}  
  \caption{SMAPE of Endogeneous Infection Rate $\delta$}
  \label{fig:ER100SMAPE}
\end{subfigure}
\caption{Contact Process: Error vs. Trajectory Length ($M$) for 100-Bus Power Grid}
\label{fig:contactpowerresult}
\end{figure*}

%Note that from Section~\ref{}, the transition rates of the reversible contact process are not directly a linear function of the healing and infection rates. Instead,  it is linear in higher dimensions (i.e., the lifting `trick'). As a results, there are more parameters, $\theta = [\mu, \beta, \beta\delta, \beta\delta^2, \ldots \beta\delta^{d_{\max}}]$, that need to be estimated via Algorithm~\ref{alg} than compared to the contact process. While we can directly obtain $\mu$ and $\beta$ from the estimates, we need to obtain $\delta$ from $ \beta\delta, \beta\delta^2, \ldots \beta\delta^{d_{\max}}$. As a result, if the estimates of these quantities are inconsistent, we the estimates of $\delta$ will be poor.
%
%
%Figure~\ref{fig:ER100MAE2} shows the plot of mean absolute error vs. trajectory length $M$ for the healing rate, endogeneous infection rate, and exogeneous infection rate. Note that the absolute error is shown in \emph{log} scale. The estimate and the true dynamics parameters can be off by several magnitudes. Figure~\ref{fig:ER100SMAPE2} shows the plot of the mean relative error (SMAPE) vs. number of observations $M$. The relative error of $\widehat{\delta}$ is nearly 100\% for all three estimators.


%%%%%%%%%%%%%%%%%%%%%%%%%%%%%%%
%%%%%%%%%%%%%%%%%%%%%%%%%%%%%%%
%%%%%%%%%%%%%%%%%%%%%%%%%%%%%%%
%%%%%%%%%%%%%%%%%%%%%%%%%%%%%%%

\subsection{Parameter Estimation Performance for Reversible Contact Processes}

\subsubsection{Reversible Contact Processes: Synthetic Erd\H{o}s-R\'{e}nyi (ER) Graphs}
We used the same ER graphs and generation process as in the previous section. The contact process dynamic is replaced by the reversible contact process. Note that from Section~\ref{sec:rev}, the transition rates of the reversible contact process are not directly a linear function of the healing and infection rates. Instead,  it is linear in higher dimensions. As a results, there are more parameters, $\theta = [\mu, \beta, \beta\delta, \beta\delta^2, \ldots \beta\delta^{d_{\max}}]^T$ need to be estimated via Algorithm~\ref{alg} than compared to the contact process. While we can directly obtain estimates of $\mu$ and $\beta$, an additional step is need to find $\delta$ from $ \beta\delta, \beta\delta^2, \ldots \beta\delta^{d_{\max}}$. In contrast to the contact process, the estimate of the endogenous infection rate $\delta$ is poorer than the exogeneous infection rate $\beta$.

Figure~\ref{fig:revER100result} shows the the MAE and SMAPE error for the healing rate, endogeneous infection rate, and exogeneous infection rate. The estimates for the reversible contact process have larger error compared to the contact process. Note that the absolute error is shown in \emph{log} scale. The standard deviation of the estimates can also be very large. 




\begin{figure*}[ht]
\begin{subfigure}{0.5\textwidth}
  \centering
  % include first image
  \includegraphics[width=0.95\linewidth]{./images/rev_MAE_mu-eps-converted-to}  
  \caption{Mean Absolute Error of Healing Rate $\mu$}
  \label{fig:ER100MAE}
\end{subfigure}
\begin{subfigure}{0.5\textwidth}
  \centering
  % include first image
  \includegraphics[width=0.95\linewidth]{./images/rev_SMAPE_mu-eps-converted-to}  
  \caption{SMAPE of Healing Rate $\mu$}
  \label{fig:ER100MAE}
\end{subfigure}
\hfill

\begin{subfigure}{0.5\textwidth}
  \centering
  % include second image
  \includegraphics[width=0.95\linewidth]{./images/rev_MAE_lambda-eps-converted-to}  
  \caption{Mean Absolute Error of Exogeneous Infection Rate $\beta$}
  \label{fig:ER100SMAPE}
\end{subfigure}
\begin{subfigure}{.5\textwidth}
  \centering
  % include second image
  \includegraphics[width=0.95\linewidth]{./images/rev_SMAPE_lambda-eps-converted-to}  
  \caption{SMAPE of Exogeneous Infection Rate $\beta$}
  \label{fig:ER100SMAPE}
\end{subfigure}
\hfill

\begin{subfigure}{.5\textwidth}
  \centering
  % include second image
  \includegraphics[width=0.95\linewidth]{./images/rev_MAE_gamma-eps-converted-to}  
  \caption{Mean Absolute Error of Endogeneous Infection Rate $\delta$}
  \label{fig:ER100SMAPE}
\end{subfigure}
\begin{subfigure}{.5\textwidth}
  \centering
  % include second image
  \includegraphics[width=0.95\linewidth]{./images/rev_SMAPE_gamma-eps-converted-to}  
  \caption{SMAPE of Endogeneous Infection Rate $\delta$}
  \label{fig:ER100SMAPE}
\end{subfigure}
\caption{Reversible Contact Process: Error vs. Trajectory Length ($M$) for Erd\H{o}s-R\'{e}nyi Graphs}
\label{fig:revER100result}
\end{figure*}





%%%%%%%%%%%%%%%%%%%%%%%%%%%%%%%%
\subsubsection{\textbf{Reversible Contact Processes: 118-bus Power Grid Graph}}

Figure~\ref{fig:revbus118result} shows the mean absolute error and SMAPE from 50 independent trajectories $\Sigma$ from the reversible contact process with random initialization and dynamics parameters. The errors are much lower for $\mu, \beta, \delta$ compared to experiments using randomly generated ER graphs. This may be because the 110-bus network has very low average degree, and the maximum degree is limited to $9$. The NNLS estimator gives slightly lower error than the other estimators. We can also see that the error of the reversible contact process is in general larger than the contact process. 




%Figure~\ref{fig:118scaledspread_MAE} and~\ref{fig:118scaledspread_SMAPE} show the distribution of the absolute and relative errors. The spread of absolute error can be several magnitudes for the reversible contact process; this is a much larger spread that compared to the contact process. The NNLS estimator give the best estimates for the exogeneous infection rate, $\delta$, compared to WLS and LAD estimators
\begin{figure*}[ht]


\begin{subfigure}{.5\textwidth}
  \centering
  % include first image
  \includegraphics[width=0.95\linewidth]{./images/real_rev_MAE_mu-eps-converted-to}  
  \caption{Mean Absolute Error of Healing Rate $\mu$}
  \label{fig:118BusMAE}
\end{subfigure}
\begin{subfigure}{.5\textwidth}
  \centering
  % include first image
  \includegraphics[width=0.95\linewidth]{./images/real_rev_SMAPE_mu-eps-converted-to}  
  \caption{SMAPE of Healing Rate $\mu$}
  \label{fig:ER100MAE}
\end{subfigure}
\hfill

\begin{subfigure}{.5\textwidth}
  \centering
  % include second image
  \includegraphics[width=0.95\linewidth]{./images/real_rev_MAE_lambda-eps-converted-to}  
  \caption{Mean Absolute Error of Exogeneous Infection Rate $\beta$}
  \label{fig:ER100SMAPE}
\end{subfigure}
\begin{subfigure}{.5\textwidth}
  \centering
  % include second image
  \includegraphics[width=0.95\linewidth]{./images/real_rev_SMAPE_lambda-eps-converted-to}  
  \caption{SMAPE of Exogeneous Infection Rate $\beta$}
  \label{fig:ER100SMAPE}
\end{subfigure}
\hfill

\begin{subfigure}{.5\textwidth}
  \centering
  % include second image
  \includegraphics[width=0.95\linewidth]{./images/real_rev_MAE_gamma-eps-converted-to}  
  \caption{Mean Absolute Error of Endogeneous Infection Rate $\delta$}
  \label{fig:ER100SMAPE}
\end{subfigure}
\begin{subfigure}{.5\textwidth}
  \centering
  % include second image
  \includegraphics[width=0.95\linewidth]{./images/real_rev_SMAPE_gamma-eps-converted-to}  
  \caption{SMAPE of Endogeneous Infection Rate $\delta$}
  \label{fig:ER100SMAPE}
\end{subfigure}
\caption{Reversible Contact Process: Error vs. Trajectory Length ($M$) for 100-Bus Power Grid}
\label{fig:revbus118result}
\end{figure*}

%%%%%%%%%%%%%%%%%%%%%%%%%%%%%%%%%%%%%%%%%%%%%
%%%%%%%%%%%%%%%%%%%%%%%%%%%%%%%%%%%%%%%%%%%%%
%%%%%%%%%%%%%%%%%%%%%%%%%%%%%%%%%%%%%%%%%%%%%
%%%%%%%%%%%%%%%%%%%%%%%%%%%%%%%%%%%%%%%%%%%%%
\section{Conclusion}\label{Sec:Con}

We studied continuous-time Markov processes whose transition rates are linear functions of a smaller set of dynamics parameters; our goal is to develop a method to estimate these parameters from a single continuously-observed trajectory (i.e., no missing observations). We defined the concept of holding classes, which are Markov states in a continuous-time Markov process that have the same expected holding time. It can be shown that for network-based epidemics models, it is more efficient to consider holding classes than individual Markov states. A much smaller system of equation can be formulated to estimate the dynamics parameters via weighted least squares methods. We showed with numerical experiments on synthetic and real-world graphs that reasonable estimates of dynamics parameters can be obtained even the length of the trajectory is many magnitudes smaller than the dimensionality of the Markov process. In all criteria, the estimation method works better for the contact process than the reversible contact process. 


%The number of holding classes, $K$, maybe be much smaller than the number of Markov states. We derived the upperbounds on $K$ for the contact process and the reversible contact process. For the contact process, the upperbound on $K$ scales polynomially with the size of the graph instead of exponentially. With numerical experiments, we showed that $K$ is also much smaller than the number of Markov states for the reversible contact process.

There are several directions for future work. It is not clear the dependency between network structure and the number of holding classes. Better upperbound may be derived, especially the study of asymptotic behavior of $K$ as $N \to \infty$. There are many more issues to address with estimation.  Realistic observations would have noise and/or missing values (i.e., discrete-time observations). We believe that estimation results can be improved using generalized least squares as the errors for different holding classes are not (realistically) independent. The challenge is on how to estimate the (non-diagonal) covariance matrix from a single trajectory. 


%Second, especially in the application to epidemics modeling, it is more likely that system will not be
%
%the trajectory can only be observed at discrete time periods. 
%
%Thirdly, the performance of the estimator is not very good for infection rates. One modification is to consider a Bayesian perspective to parameter estimation or by using regularization to improve the estimates.




%
%For interacting particle system models involving small graphs ($< 16$ nodes), we can enumerate over all possible holding classes. We can show that for both contact and reversible contact processes, the ratio of the number of holding classes to the number of Markov states is 1) much smaller than $1$, 2) decreases with the size of the graph.
%
%Therefore for very large interacting particle systems, it is more efficient to work with holding classes than with Markov states. We can formulate a much 
%smaller system of equation to estimate the dynamics parameters via weighted least squares methods. We showed with numerical experiments on synthetic and real-world graphs that reasonable estimates of dynamics parameters can be obtained even the length of the trajectory is many magnitudes smaller than the dimensionality of the Markov process. In all criteria, the estimation method works better for the contact process than the reversible contact process. 
%
%There are several direction for future work. First, it is not clear the dependency between network structure and the number of holding classes. Second, especially in the application to epidemics modeling, it is more likely that the trajectory can only be observed at discrete time periods. This means that state transitions may have been missed, which will affect the 
%
%Thirdly, the performance of the estimator is not very good for infection rates. One modification is to consider a Bayesian perspective to parameter estimation or by using regularization to improve the estimates.
%%
%
%
%
%
% For interacting particle systems, which are a special type of continuous-time Markov processes, the number of holding classes can be much smaller than the number of Markov states. 
%
%
%
%We found an upperbound on the number of holding classes f
%
%
%
%We showed that the number of holding classes in IPS models can be much smaller than the dimensionality of the state space, which is exponential in the number of interacting particles. As a result, given a continuously observed trajectory of the system, $\Sigma$, whose length can be many orders smaller than the dimensionality of the state space, the dynamics parameter underlying the IPS model can be computed efficiently using weighted least squares method. Because we assumed that the system is continuously observed, the state of the system is completely known. We will consider in future work the more realistic scenario of missing observations. This may occur in `space' when not all the nodes in the contact network are directly observable. Alternatively, missing observations may be in time in that the system can not be continuously observed but rather can only be observed at discrete-time intervals. Estimating transition rates of CTMP from discrete-time observations is known as the \emph{embedding problem}, and there is a large body of research work dedicated to the problem \cite{}. The primary issue is that computational complexity precludes the study of CTMP with large state space.


%%%%%%%%%



% An example of a floating figure using the graphicx package.
% Note that \label must occur AFTER (or within) \caption.
% For figures, \caption should occur after the \includegraphics.
% Note that IEEEtran v1.7 and later has special internal code that
% is designed to preserve the operation of \label within \caption
% even when the captionsoff option is in effect. However, because
% of issues like this, it may be the safest practice to put all your
% \label just after \caption rather than within \caption{}.
%
% Reminder: the "draftcls" or "draftclsnofoot", not "draft", class
% option should be used if it is desired that the figures are to be
% displayed while in draft mode.
%
%\begin{figure}[!t]
%\centering
%\includegraphics[width=2.5in]{myfigure}
% where an .eps filename suffix will be assumed under latex, 
% and a .pdf suffix will be assumed for pdflatex; or what has been declared
% via \DeclareGraphicsExtensions.
%\caption{Simulation results for the network.}
%\label{fig_sim}
%\end{figure}

% Note that the IEEE typically puts floats only at the top, even when this
% results in a large percentage of a column being occupied by floats.
% However, the Computer Society has been known to put floats at the bottom.


% An example of a double column floating figure using two subfigures.
% (The subfig.sty package must be loaded for this to work.)
% The subfigure \label commands are set within each subfloat command,
% and the \label for the overall figure must come after \caption.
% \hfil is used as a separator to get equal spacing.
% Watch out that the combined width of all the subfigures on a 
% line do not exceed the text width or a line break will occur.
%
%\begin{figure*}[!t]
%\centering
%\subfloat[Case I]{\includegraphics[width=2.5in]{box}%
%\label{fig_first_case}}
%\hfil
%\subfloat[Case II]{\includegraphics[width=2.5in]{box}%
%\label{fig_second_case}}
%\caption{Simulation results for the network.}
%\label{fig_sim}
%\end{figure*}
%
% Note that often IEEE papers with subfigures do not employ subfigure
% captions (using the optional argument to \subfloat[]), but instead will
% reference/describe all of them (a), (b), etc., within the main caption.
% Be aware that for subfig.sty to generate the (a), (b), etc., subfigure
% labels, the optional argument to \subfloat must be present. If a
% subcaption is not desired, just leave its contents blank,
% e.g., \subfloat[].


% An example of a floating table. Note that, for IEEE style tables, the
% \caption command should come BEFORE the table and, given that table
% captions serve much like titles, are usually capitalized except for words
% such as a, an, and, as, at, but, by, for, in, nor, of, on, or, the, to
% and up, which are usually not capitalized unless they are the first or
% last word of the caption. Table text will default to \footnotesize as
% the IEEE normally uses this smaller font for tables.
% The \label must come after \caption as always.
%
%\begin{table}[!t]
%% increase table row spacing, adjust to taste
%\renewcommand{\arraystretch}{1.3}
% if using array.sty, it might be a good idea to tweak the value of
% \extrarowheight as needed to properly center the text within the cells
%\caption{An Example of a Table}
%\label{table_example}
%\centering
%% Some packages, such as MDW tools, offer better commands for making tables
%% than the plain LaTeX2e tabular which is used here.
%\begin{tabular}{|c||c|}
%\hline
%One & Two\\
%\hline
%Three & Four\\
%\hline
%\end{tabular}
%\end{table}


% Note that the IEEE does not put floats in the very first column
% - or typically anywhere on the first page for that matter. Also,
% in-text middle ("here") positioning is typically not used, but it
% is allowed and encouraged for Computer Society conferences (but
% not Computer Society journals). Most IEEE journals/conferences use
% top floats exclusively. 
% Note that, LaTeX2e, unlike IEEE journals/conferences, places
% footnotes above bottom floats. This can be corrected via the
% \fnbelowfloat command of the stfloats package.








% if have a single appendix:
%\appendix[Proof of the Zonklar Equations]
% or
%\appendix  % for no appendix heading
% do not use \section anymore after \appendix, only \section*
% is possibly needed

% use appendices with more than one appendix
% then use \section to start each appendix
% you must declare a \section before using any
% \subsection or using \label (\appendices by itself
% starts a section numbered zero.)
%




% use section* for acknowledgment
\ifCLASSOPTIONcompsoc
  % The Computer Society usually uses the plural form
  \section*{Acknowledgments}
\else
  % regular IEEE prefers the singular form
  \section*{Acknowledgment}
\fi
This work was supported in part by the National Science Foundation AI Institute in Dynamic Systems (Grant No. 2112085).


\bibliography{library}

% biography section
% 
% If you have an EPS/PDF photo (graphicx package needed) extra braces are
% needed around the contents of the optional argument to biography to prevent
% the LaTeX parser from getting confused when it sees the complicated
% \includegraphics command within an optional argument. (You could create
% your own custom macro containing the \includegraphics command to make things
% simpler here.)
%\begin{IEEEbiography}[{\includegraphics[width=1in,height=1.25in,clip,keepaspectratio]{mshell}}]{Michael Shell}
% or if you just want to reserve a space for a photo:

\begin{IEEEbiography}[{\includegraphics[width=1in,height=1.25in,clip,keepaspectratio]{./images/Seyyed.jpg}}]{Seyyed A. Fatemi} (Ph.D. U. of Hawaii Electrical Engineering) is a former post-doctoral researcher at the University of Hawai'i at M\={a}noa specializing in artificial intelligence, stochastic optimization, reinforcement learning, probabilistic and  machine learning modeling, statistical learning, estimation, prediction, time series analysis. His research interests include areas such as statistical signal processing, network science, big data and artificial intelligence, renewable energy and smart grid.
\end{IEEEbiography}

% if you will not have a photo at all:
\begin{IEEEbiography}[{\includegraphics[width=1in,height=1.25in,clip,keepaspectratio]{./images/portrait.jpg}}]{June Zhang} is an assistant professor at the University of Hawai'i at M\={a}noa in the Department of Electrical and Computer Engineering. She received her B.S. with Highest Honor in Electrical and Computer Engineering from the Georgia Institute of Technology and M.S. in Electrical and Computer Engineering from Stanford University. She received a Ph.D. in Electrical and Computer Engineering from Carnegie Mellon University in December 2015. She is a co-PI of the NSF AI Institute in Dynamic Systems. Her research interests are graph dynamical systems, graph coding, representation learning, anomaly detection, graph signal processing, and natural language processing. 
\end{IEEEbiography}

% insert where needed to balance the two columns on the last page with
% biographies
%\newpage


% You can push biographies down or up by placing
% a \vfill before or after them. The appropriate
% use of \vfill depends on what kind of text is
% on the last page and whether or not the columns
% are being equalized.

%\vfill

% Can be used to pull up biographies so that the bottom of the last one
% is flush with the other column.
%\enlargethispage{-5in}


 
% \newpage
% 
%\appendices
%\section{Proof Lemma~\ref{lemmascaled}}
%\begin{proof}
%For two Markov states, corresponding to configurations $\mathbf{x}$ and $\mathbf{x}'$, the diagonal entries of the transition rate matrix $Q$ are 
%\begin{align*}
%&q(\mathbf{x}, \mathbf{x}) = -\sum_{\tilde{\mathbf{x}} \in \mathcal{D}} q(\mathbf{x}, \tilde{\mathbf{x}})\\
%&\medmath{= -((N- | \mathcal{S}(\mathbf{x}) |)\mu +  |{ s \in S(\mathbf{x}): m_s = 0 }|(\beta)}+\\
%&\medmath{ |\{s \in S(\mathbf{x}): m_s = 1\}|(\beta\delta) \ldots + |\{s \in S(\mathbf{x}): m_s = d_{\max}\}|(\beta\delta^{d_{\max}}))}, 
%\end{align*}
%and
%\begin{align*}
%&q(\mathbf{x}', \mathbf{x}') = -\sum_{\tilde{\mathbf{x}} \in \mathcal{D}} q(\mathbf{x}', \tilde{\mathbf{x}})\\
%&= \medmath{-((N- | \mathcal{S}(\mathbf{x}') |)\mu +  (|{ s \in S(\mathbf{x}'): m_s = 0 }|)\beta}+\\
%&\medmath{(|\{s \in S(\mathbf{x}'): m_s = 1\}|)\beta\delta \ldots + (|\{s \in S(\mathbf{x}'): m_s = d_{\max}\}|)\beta\delta^{d_{\max}}}.
%\end{align*}
%
%When equations~\eqref{eq:contactcond3} and \eqref{eq:contactcond4} are true, then $q(\mathbf{x}, \mathbf{x}) = q(\mathbf{x}', \mathbf{x}')$. By Definition~\ref{def:holdingclass}, Markov state $\mathbf{x}
%$ and $\mathbf{x}'$ belongs to the same holding class, $\mathcal{H}$.
%\end{proof}
%
%\section{Proof of Theorem~\ref{thm:revtran}}
%\begin{proof}
%Consider Lemma~\ref{lemmascaled}. If the holding class is only determined by~\eqref{eq:contactcond3}, then there would be $N+1$ different holding classes corresponding to $|\mathcal{S}(\mathbf{x})| = 0, 1, \ldots N$. It is intuitive that the number of holding classes should be much larger than $O(N)$. Therefore, the number of holding classes, $K$, is determined by how many possible unique values $\{m_k: k \in S(\mathbf{x})\}$ may take for all $\mathbf{x} \in \mathcal{D}$,
%
%Assuming that there are $|\mathcal{S}(\mathbf{x})|$ susceptible nodes in a configuration $\mathbf{x}$, this means that there are $N- |\mathcal{S}(\mathbf{x})|$ total number of infected nodes. Without any additional knowledge of the structure of $G(V,E)$, we can conclude that that $m_k$ can range in value from $0,1, \ldots N- |\mathcal{S}(\mathbf{x})|$ for any node $k \in \mathcal{S}(\mathbf{x})$. The number of possible unique set $\{m_k: k \in S(\mathbf{x})\}$ is a combination with replacement problem where we want to choose $N-|\mathcal{S}(\mathbf{x})|+1$ values for $|\mathcal{S}(\mathbf{x})|$ entires. 
%
%Let $s = |\mathcal{S}(\mathbf{x})|$. The number of possibilities is
%\[
%\frac{(N-s+1+s - 1)!}{s!(N-s+1-1)!} = \frac{N!}{s!(N-s)!}.
%\]
%Summing over all possible number of susceptible nodes in $\mathcal{D}$ results in
%\begin{equation}\label{eq:revsum}
%\sum_{s=0}^N \frac{N!}{s!(N-s)!}.
%\end{equation}
%By the binomial theorem, we know that~\eqref{eq:revsum} is equal to $2^N$.
%\end{proof}
%
%
%\section{Proof of Corollary~\ref{coro:revtran}}
%\begin{proof}
%From the proof of Theorem~\ref{thm:revtran}, for a given number of susceptible nodes, $s$, we assumed that the number of infected neighbors of a susceptible node can range from $0, 1, \ldots, N-s$. When the interaction network has bounded degree, we know that the number of infected neighbors of a susceptible node can range from $0,1, \ldots, \min(N-s, d_{\max})$. Therefore, the sum~\eqref{eq:revsum} becomes
%\begin{align*}
%&\sum_{s=0}^N \frac{N!}{s!(N-s)!}\\ 
%&= \medmath{\sum_{s=0}^{N - d_{\max}}  \frac{(d_{\max} +s)!}{s!(d_{\max})!}      + \sum_{N -d_{\max}+1}^{N} \frac{(N-s+1+s - 1)!}{s!(N-s+1-1)!}}\\
%& = \sum_{s=0}^{N - d_{\max}}  \frac{(d_{\max} +s)!}{s!(d_{\max})!}      + \sum_{s = N -d_{\max}+1}^{N} \frac{N!}{s!(N-s)!}
%\end{align*}
%
%\end{proof}



% that's all folks
\end{document}


