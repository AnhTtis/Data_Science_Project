%%%%%%%%%%%%%%%%%%%%%%%%%%%%%%%%%%%%%%%%%%%%%%%%%%%%%%%%%%%%%%%%%%%%%%%%%%%%%%
%%                                                                          %%
%% Tame homeomorphisms of surfaces of infinite type                         %%
%%                                                                          %%
%%%%%%%%%%%%%%%%%%%%%%%%%%%%%%%%%%%%%%%%%%%%%%%%%%%%%%%%%%%%%%%%%%%%%%%%%%%%%%

%section{Preamble}

  \documentclass[11pt]{amsart}

  \usepackage[margin=3cm]{geometry}
  \usepackage{latexsym}
  \usepackage{comment}
  \usepackage{amsmath,amsthm,amsfonts,amssymb,graphicx,epsfig,latexsym,float}
  \usepackage[dvipsnames]{xcolor}
  \definecolor{PrettyGreen}{RGB}{0, 200, 0}
  \definecolor{FigGreen}{RGB}{0, 180, 0}
  \definecolor{FigLightBlue}{RGB}{0, 170, 255}
  
  \usepackage{epsf,enumerate}
  \usepackage{overpic}  
  \usepackage[normalem]{ulem}

  \usepackage[breaklinks,colorlinks=true,allcolors=BurntOrange]{hyperref}

  %\usepackage[notcite,notref]{showkeys}
  
  \makeatletter
  \def\paragraph{\@startsection{paragraph}{4}%
    \z@\z@{-\fontdimen2\font}%
    {\normalfont\bfseries}}
  \makeatother

  \def\rank{\operatorname{rank}}
  \def\int{\operatorname{int}}
  \newcommand{\diagclosure}[1]{\langle\!\langle #1\rangle\!\rangle}

  \newtheorem{thm}{Theorem}[section]
  \newtheorem{lemma}[thm]{Lemma}
  \newtheorem{claim}[thm]{Claim}
  \newtheorem{cor}[thm]{Corollary}
  \newtheorem{prop}[thm]{Proposition}
  \newtheorem*{main}{Main Theorem}{}
  \newtheorem*{rrt}{Rank Rigidity Theorem}{}

  \newtheorem{thmintro}{Theorem}
  \renewcommand*{\thethmintro}{\Alph{thmintro}}
  \newtheorem{conj}[thmintro]{Conjecture}
  \newtheorem{propintro}[thmintro]{Proposition}
  
  \theoremstyle{remark}
  \newtheorem{example}[thm]{Example}
  \newtheorem{exercise}[thm]{Exercise}
  \newtheorem{definition}[thm]{Definition}
  \newtheorem{remark}[thm]{Remark}
  \newtheorem{examples}[thm]{Examples}
  \newtheorem*{definition*}{Definition}
  \newtheorem*{remark*}{Remark}
  \newtheorem{question}[thm]{Question}
  
  \newcommand{\mb}[1]{\marginpar{\color{blue}\tiny #1 --mb}}

  \newcommand{\ff}[1]{\marginpar{\color{olive}\tiny #1 --ff}}

  \newcommand{\jt}[1]{\marginpar{\color{violet}\tiny #1 --jt}}
  
  \newcommand{\note}[1]{{\color{orange} #1}}
  \newcommand\red[1]{{\color{red}#1}}
  \newcommand{\gray}[1]{{\color{gray} #1}}
  \newcommand\blue[1]{{\color{blue}#1}}

  \bibliographystyle{plain}   %plain

  \def\R{{\mathbb R}}
  \def\Q{{\mathbb Q}}
  \def\Z{{\mathbb Z}}
  \def\N{{\mathbb N}}
  \def\H{{\mathbb H}}  
  \def\G{{\Gamma}}
  \def\S{{\mathcal S}}
  \def\D{{\mathcal D}}
  \def\s{{\sigma}}
  \def\t{{\tau}}
  \def\e{{\varepsilon}}
  \def\F{{\mathcal F}}
  \def\X{{\mathcal X}}
  \def\P{{\mathcal P}}
  \def\pa{pseudo-Anosov\ }
  \def\diam{\operatorname{diam}}
  \def\spec{\operatorname{spec}}
  \def\stab{\operatorname{Stab}}
  \def\diag{\operatorname{diag}}
  \def\SL{\operatorname{SL}}
  \def\GL{\operatorname{GL}}
  \def\Sym{\operatorname{Sym}}
  \def\FF{{\mathbb F}}
  \def\A{{\mathcal A}}
  \def\l{{\lambda}}
  \def\C{{\mathcal C}}
  \def\Cper{{\mathcal C}_{\text{per}}}
  \def\Cinf{{\mathcal C}_\infty}
  \def\PSL{{PS}}
  \def\PSper{\tilde{S}_{\text{per}}}
  \def\Sper{{S}_{\text{per}}}
  \def\PS0{\tilde{S}_0}
  \def\Sinf{{S}_\infty}
  \DeclareMathOperator{\Ends}{Ends}
  \DeclareMathOperator{\interior}{int}
  \DeclareMathOperator{\mcg}{Map}
  \newcommand{\Cis}{\C_{\text{is}}}
  \newcommand{\Cint}{\C_{\text{int}}}

  \newcommand{\<}{\langle}
  \renewcommand{\>}{\rangle}
  \renewcommand{\L}{\mathcal{L}}
  \newcommand{\st}{\;|\;}
  \newcommand{\ssm}{\smallsetminus}
  \newcommand{\Teich}{Teichm\"uller }
  
  \begin{document}

%section{Title and abstract}

  \title[Towards Nielsen--Thurston classification]{Towards
  Nielsen--Thurston classification\\ for surfaces of infinite type}
  
  \author{Mladen Bestvina}
  \address{Dept of Math, Univ of Utah, Salt Lake City, UT 84112, USA}
  \email{bestvina@math.utah.edu}
  \author{Federica Fanoni}
  \address{CNRS, Univ Paris Est Creteil, Univ Gustave Eiffel, LAMA UMR8050,
  F-94010 Creteil, France}
  \email{federica.fanoni@u-pec.fr} 
  \author{Jing Tao}
  \address{Dept of Math, Univ of Oklahoma, Norman, OK 73019, USA}
\email{jing@ou.edu}
  
  \date{}

  
  \begin{abstract}
    We introduce and study tame homeomorphisms of surfaces of infinite
    type. These are maps for which curves under iterations do not
    accumulate onto geodesic laminations with non-proper leaves, but rather
    just a union of possibly intersecting curves or proper lines. Assuming
    an additional finiteness condition on the accumulation set, we prove a
    Nielsen--Thurston type classification theorem. We prove that for such
    maps there is a canonical decomposition of the surface into invariant
    subsurfaces on which the first return is either periodic or a
    translation.
  \end{abstract}
   
  \thispagestyle{empty}

  \maketitle

\section{Introduction}

  In this paper, we study the isotopy classes of self-homeomorphisms, or
  mapping classes, of a connected and oriented surface $S$ of
  \emph{infinite} type. In the classical setting of surfaces of finite
  type, the Nielsen--Thurston classification theorem states that a mapping
  class is either periodic, reducible, or pseudo-Anosov. By Nielsen realization, a periodic class is represented by an
  isometry of some hyperbolic metric on $S$. Instead a pseudo-Anosov map
  preserves a pair of transverse geodesic laminations which are minimal and
  filling. In the reducible case, one can decompose $S$ along an invariant
  multicurve such that the first return map to a complementary subsurface
  is either periodic or pseudo-Anosov. The long term goal of this project
  is to extend this understanding to surfaces of infinite type. Here, we
  introduce the notion of \emph{tame} maps and prove a structure theorem
  for the subclass of \emph{extra tame} maps.
	
  To motivate our definitions, let's first consider the various approaches
  to the Nielsen--Thurston classification and how they may generalize to
  surfaces of infinite type. Thurston's original proof
  \cite{thurston_geometry,flp_travaux} finds a fixed point in the
  compactified \Teich space. Bers' proof \cite{bers_extremal} also uses
  \Teich space, but from the point of view of extremal quasiconformal maps.
  Casson's proof \cite{cb_automorphisms} finds an invariant geodesic
  lamination by iterating a curve, and Nielsen's original approach
  \cite{Nielsen_UntersuchungenI, Nielsen_UntersuchungenII,
  Nielsen_UntersuchungenIII}, completed by Miller \cite{miller} and
  Handel--Thurston \cite{ht_new}, analyzes the dynamics of the action on
  the circle at infinity of the lifts to the universal cover. The
  Bestvina--Handel proof \cite{BH} looks for efficient spines of the
  surface leading to invariant train tracks, and relies on the
  Perron--Frobenius theorem. 

  In the infinite-type setting, \Teich space is very complicated: in particular, it is infinite-dimensional, it has uncountably many connected components and there are  maps that do not preserve any component. So it is hard to imagine
  how one could adapt Thurston's or Bers' approach to the infinite-type
  setup. The train track approach would require considering infinite matrices
  without a good analog of the Perron--Frobenius theory. Instead, the
  viewpoints of Casson and Nielsen--Handel--Thurston seem more amenable, and in this
  paper we take Casson's approach. 

  Given a map $f$, the first step in Casson's program is to construct a
  geodesic lamination $\lambda$ as the limit of a subsequence
  $f^{n_i}(\alpha)$ of iterates of a curve $\alpha$, pulled tight. In order
  to ensure that $f^n(\lambda)$ has no transverse intersections with
  $\lambda$ (in which case the closure of $\bigcup_n f^n(\lambda)$ is an
  $f$-invariant geodesic lamination), this limiting process should be
  ``robust'', in the sense that more and more strands of $f^{n_i}(\alpha)$
  converge to the leaves of $\lambda$. This robustness is essentially
  automatic in the finite type case and fails only when $\alpha$ intersects
  the subsurface on which $f$ is periodic.

  For infinite-type surfaces, there are several new phenomena, and in
  particular the robustness mentioned above is far from being automatic.
  First, there are infinite-order isometries, such as the translation
  $(x,y) \mapsto (x+1,y)$ on the surface $\mathbb{R}^2 \smallsetminus \Z^2$
  which sends every curve to infinity. This is an incidence of a general
  \emph{translation} --- a map that generates an
  infinite cyclic group acting properly on the surface. For a more complicated example, consider the same surface
  but now restrict the translation on $y \le 1/3$, act with its inverse on $y\geq
  2/3$, and interpolate between the two maps by horizontal translations in
  the strip $1/3< y< 2/3$. In this case, a curve that intersects the
  strip essentially will converge to a line homotopic into the strip, but
  the convergence will not be robust. Another example comes from taking a
  homeomorphism of the equator in $S^2$ with minimal invariant subset a
  Cantor set $C$, and extending it to a homeomorphism $f$ of $S^2
  \smallsetminus C$, which we call an \emph{irrational rotation}. In this
  example, a convergent subsequence $f^{n_i}(\alpha)$ limits onto a line in
  a non-robust way, and the union of all such limits is a non
  locally-finite collection of lines, some of which intersect --- in
  particular, it's not a geodesic lamination.

  This brings us to the notion of \emph{tame} maps, which are exactly those
  maps where the convergence is never robust for any curve; in other words,
  a tame map does not exhibit pseudo-Anosov behavior anywhere on the
  surface. A quick definition is that $f$ is tame if for any two curves
  $\alpha,\beta$ there is a uniform bound on the intersection numbers
  $i(f^n(\alpha),\beta)$, for $n\in\Z$ (see Section \ref{sec:tame}).  As shown in Section \ref{sec:tame},  under a tame map $f$, a
  curve $\alpha$ either leaves every compact set, or has limit set
  $\L(\alpha)$ a collection of curves or properly embedded lines (rather
  than more complicated laminations). A tame map is \emph{extra tame} if in
  addition the limit set of every curve is a {\it finite} collection of
  lines and curves.

  On a finite-type surface, being tame is equivalent to being periodic. For
  infinite-type surfaces, prototypes of tame maps are periodic maps,
  translations, and irrational rotations, with the first two being extra
  tame. More complicated examples can be constructed by gluing multiple
  maps together in a controlled way, such as the gluing of two translations
  along a fixed line like our second example above, and one can easily
  adapt this procedure to infinitely many maps.

  Despite their name, the behavior of tame maps can often be quite
  complicated. Nevertheless, we believe that there is a Nielsen--Thurston
  type theory for tame maps and conjecture that our fundamental examples
  are the only building blocks of such maps. Our main result is a proof of
  this conjectural picture for the case of extra tame maps. As we shall
  explain below, the full strength of our theorem cannot be generalized to
  all tame maps.

  \begin{thmintro} \label{thm:mainintro} 
    
    Let $f$ be an extra tame map of a surface $S$ of infinite type. There
    is a canonical decomposition of $S$ into three $f$--invariant
    subsurfaces $\Sper$, $\Sinf$ and $S_0$ and a hyperbolic metric on
    $S$, such that for every component $X$ of $\Sper$, $\Sinf$ and $S_0$, $f$ returns to $X$ and is isotopic to: 
    \begin{itemize}
      \item a periodic isometry, if $X\subset \Sper\cup S_0$, 
      \item an isometric translation, if $X\subset \Sinf$.
    \end{itemize} 
Furthermore, components of $S_0$ contain no essential non-peripheral curves and at most one essential (peripheral) curve.
  \end{thmintro}
  
  A component of $S_0$ should be regarded as the transition piece
  interpolating the action of $g$ on its neighboring pieces --- for instance, if we have a map given by three translations with disjoint supports (instead of two as in the example described above), we could have a component of $S_0$ homeomorphic to an ideal triangle (see for instance the first map in Figure \ref{fig:componentsofS0}).

 For general tame maps, the statement of Theorem \ref{thm:mainintro}
  first needs to be adjusted to include maps with irrational rotation behavior. However, it
  turns out the full analogue of the statement cannot hold, as there is not
  always a nice decomposition for general tame maps. More precisely, there
  are tame maps for which the natural invariant pieces are no longer
  subsurfaces, meaning that although they have non-empty interior, their
  boundaries 
 accumulate in a complicated way (see the \emph{Strategy of
  proof} section for some more details). Even replacing finiteness of limit sets by local finiteness is not enough to address this issue (see Section \ref{sec:examples}).
  Therefore, we propose the following conjecture for general tame maps. 
		
  \begin{conj}
          
    Let $f$ be a tame map of a surface $S$ of infinite type. Then there
    exists an $f$--invariant collection $\Gamma$ of pairwise disjoint lines
    and curves on $S$, such that for each component $X$ of $S
    \smallsetminus \Gamma$, the first return of $f$ to $X$ is either
    periodic, a translation, or has irrational rotation behavior.
          
  \end{conj}

	
  If $f$ is not tame, then one could try to construct an
  $f$--invariant geodesic lamination using the Casson technique. This
  approach has been successfully carried out for the class of
  \emph{irreducible end periodic} maps by Handel and Miller
  \cite{ccf_endperiodic}. In our vision, a comprehensive structure theorem
  for an arbitrary map $f$ should begin by decomposing the surface $S$ into
  $f$--invariant pieces on which the first return is either tame or
  preserves two transverse geodesic laminations. Therefore, a structure
  theorem for tame maps is an essential part of the theory, and Theorem
  \ref{thm:mainintro} is the first step in this program. 
	
  \subsection*{Strategy of proof}

  For any map $f$ on $S$, the general strategy toward decomposing $S$ into
  $f$--invariant pieces is to organize curves according to their behavior
  under $f$, and look for the smallest subsurface \emph{spanned} by curves
  of the same type. For instance, consider the collection $\Cper$ of
  $f$--periodic curves. If the smallest subsurface $\Sper$ spanned by
  $\Cper$ exists, then $\Sper$ is $f$--invariant and it is not hard to see
  that the action of $f$ on each component of $\Sper$ is periodic
  (see Section \ref{sec:structure}). We can then repeat this process and look for the
  subsurface $S_\infty$ spanned by the collection $\C_\infty$ of
  \emph{wandering} curves (curves that leave every compact set under
  iteration). Here, with some work, one can show that if $S_\infty$ exists,
  then the action of $f$ on each component of $S_\infty$ is by a
  translation. This characterization can be thought of as the analogue of
  Brouwer’s plane translation theorem (\cite{brouwer_beweis}).
 

  \begin{thmintro}

    Suppose $f$ is a homeomorphism of a surface $S$ such that every curve
    of $S$ is wandering. Then there is a hyperbolic metric on $S$ on which
    $f$ is isotopic to an isometric translation.

  \end{thmintro}

  If $f$ is tame, to prove the conjecture it remains to show that
  $S \smallsetminus \Sper \cup S_\infty$ can be decomposed into pieces by cutting along lines and curves, and that the map has irrational rotation behavior on all components with nontrivial topology. When $f$ is extra tame,
  there is no irrational rotation behavior, so we just need to show that the
  components of $S \smallsetminus \Sper \cup S_\infty$ have trivial
  topology.

  As the above procedure suggests, one of the technical steps is to
  establish the existence of the subsurface spanned by a collection of
  curves. With a finite set of curves, this subsurface is always
  well-defined. To deal with an infinite collection, we can take the span
  $S_n$ of the first $n$ curves and consider the limiting object. When the
  underlying surface $S$ has finite-type, then the topology of $S_n$'s
  stabilizes by Euler characteristic considerations, and the procedure
  described above leads to the Nielsen--Thurston decomposition of $f$. On
  the other hand, if $S$ has infinite type, then the limiting object may
  not  be a subsurface of $S$. Indeed, there are examples of tame maps
  where the collection of periodic or wandering curves displays this
  behavior (see Section \ref{sec:examples}). This issue turns out to be a
  significant obstacle in proving the conjecture in full generality.
  However, under the extra tameness assumption, the bad behavior disappears,
  allowing for this fundamental step to be completed successfully. 

  \begin{propintro}
          
    If $f$ is extra tame, then the collection of $f$--periodic curves
    spans a subsurface, and the collection of $f$--wandering curves
    spans a subsurface. Moreover, the complement of their union
    in $S$ is a subsurface with no essential non-peripheral curves.

  \end{propintro}
	
  \subsection*{Connection to actions on graphs}
	  
  A variation of Nielsen-Thurston classification, perhaps a more tractable
  problem, is to study the action of mapping classes on certain
  Gromov-hyperbolic graphs associated to the surface and classify them according to their type as
  isometries of the graphs. For instance, in the finite-type setting, pseudo-Anosov maps
  act hyperbolically on the curve graph of $S$ and all other maps act
  elliptically.  
		
  In the infinite-type setting, the curve graph always has bounded diameter,
  but for many surfaces there are other interesting infinite-diameter, Gromov-hyperbolic
  graphs. One example is the ray graph defined by Bavard
  \cite{bavard_hyperbolicite} for the plane minus the Cantor set
  $\mathbb{R}^2 \smallsetminus C$, which has been generalized to surfaces
  with one isolated end by Aramayona--Fossas--Parlier in \cite{afp_arc} and even further by Bar-Natan--Verberne in \cite{bv_grand}, where they define the \emph{grand-arc graph}. In
  \cite{bavard_hyperbolicite}, Bavard produced a map of $\mathbb{R}^2
  \smallsetminus C$ that acts hyperbolically on the ray graph, and her
  construction has been generalized to a large class of surfaces by Abbott--Miller--Patel in
  \cite{amp_infinite}. However, the classification of isometries remains
  open at large, and even the following question is unknown.
		
  \begin{question}
          
  Are there maps of $\mathbb{R}^2 \smallsetminus C$ that act as parabolic
  isometries of the ray graph? More generally, are there maps that act as
  parabolic isometries of the grand arc graphs?
          
  \end{question}
  
  By our work above, extra tame map are elliptic isometries on the
  ray/grand arc graph, and our conjectural picture of tame maps implies
  that they are also elliptic isometries.

  \subsection*{Acknowledgements}
  
  The authors would like to thank the American Institute of Mathematics for
  the hospitality during the workshop ``Surfaces of infinite type'', during
  which this work started. M.B.\ and J.T.\ gratefully acknowledge the
  support by the National Science Foundation under grant numbers
  DMS-1905720 and DMS-1651963 respectively. F.F.\ thanks Peter Feller for useful conversations.

\section{Background}
  
  By a surface $S$ we will mean a connected orientable 2-manifold, without
  boundary unless otherwise stated.  When we need to consider surfaces with boundary, we will usually say
  \emph{bordered} surface for emphasis.
  
   A surface $S$ has \emph{finite type} if
  $\pi_1(S)$ is finitely generated and \emph{infinite type} otherwise. By
  the classification of surfaces, $S$ is of finite type if and only if $S$ is
  homeomorphic to a closed genus $g$ surface minus finitely many points.   A \emph{pair of pants} is a (possibly bordered) surface homeomorphic to either a three-holed sphere, or a two-holed, once-punctured sphere, or a twice-punctured disk.
  
  To a surface we can associate its
  \emph{ends}, which can be \emph{planar} and \emph{nonplanar}. A
  \emph{puncture} is a planar isolated end. We refer to \cite{av_big} for
  definitions and properties of ends of a surface.
    
  The \emph{mapping class group} of $S$ is the group $\mcg(S)$ of isotopy
  classes of orientation-preserving homeomorphisms of $S$.

  By a \emph{hyperbolic metric} on $S$ we always mean a complete hyperbolic metric
  without funnels or half-planes (i.e.\ the hyperbolic surface coincides
  with its convex core, which is also called a metric of the \emph{first
  kind} -- see \cite{ar_structure}). Hyperbolic metrics on surfaces with
  boundary will be assumed to contain no funnels or half planes and to have
  totally geodesic boundary.
  
  \emph{Curves} on a surface $S$ are assumed to be simple, closed and
  homotopically nontrivial. A curve is \emph{essential} if it is not
  homotopic to a puncture. If $S$ has boundary, then a curve is called
  \emph{peripheral} if it is homotopic to a boundary component. When $S$ is
  endowed with a hyperbolic metric with geodesic boundary, then every
  essential curve is realized by a unique geodesic $\alpha^*$ representing
  its homotopy class.
    
  A \emph{line} on $S$ is the image of a proper embedding of $\R \to S$
  which admits a geodesic representative with respect to some (and hence
  any) hyperbolic metric on $S$. Note that such lines are essential, in the
  sense that they are not properly homotopic to an end of $S$. If the image
  is a geodesic, then we will sometimes call it a \emph{geodesic line} for
  emphasis. We will usually consider lines up to proper isotopy relative to its ends.
  
   A \emph{ray} is the image of $[0,\infty)$ under a proper embedding and an \emph{arc} is the image of an embedding of $[0,1]$. Note that under our definition arcs are compact, which is not always the case in the literature. 
  If $S$ has boundary, a line (or ray or arc) is \emph{peripheral} if it is properly isotopic, relative to its ends, into the boundary of $S$. The \email{geodesic representative} of a non-peripheral ray (respectively, arc) with endpoint(s) on $\partial S$ is the unique geodesic ray (respectively, arc) orthogonal to $\partial S$ and properly isotopic to the ray (respectively, arc) relative to $\partial S$.
  
  A \emph{geodesic lamination} is a closed subset of $S$ which is the union
  of complete simple and pairwise disjoint geodesics (with
  respect to some hyperbolic metric on $S$), called the \emph{leaves} of the
  lamination. An essential curve or a line is an example of a geodesic lamination with a single leaf. 
    
  The \emph{geometric intersection number} $i(\alpha,\beta)$ is the minimum number of intersections between representatives of $\alpha$ and $\beta$, where $\alpha$ and
  $\beta$ could each be either a line, curve, or a leaf of a lamination.
  Note that $i(\alpha,\beta)$ could be infinite, but we have the following
  characterization of lines.

  \begin{lemma} \label{lem:proper}

    Let $S$ be a surface possibly with boundary and equipped with a
    hyperbolic metric. Then a complete simple geodesic $\ell \subset S$ is
    a line or a curve if and only if $i(\ell,\alpha) < \infty$ for all curves
    $\alpha$.

  \end{lemma}

  \begin{proof}

    Clearly, if $i(\ell,\alpha) = \infty$ for some curve $\alpha$, then
    $\ell$ cannot be proper. Now suppose $i(\ell,\alpha) < \infty$ for all
    curves $\alpha$. Let $K \subset S$ be a compact subsurface. Since
    $\sum_{\alpha \subset \partial K} i(\ell, \alpha) < \infty$, $\ell \cap
    K$ has a finite number of arcs. If a component $\tau \subset \ell \cap
    K$ has infinite length, then $\tau$ must accumulate on some geodesic
    lamination $\alpha$, or on a boundary component $\alpha$ of $K$. But
    then any curve $\beta$ intersecting $\alpha$ will have $i(\ell,\beta)
    \ge i(\tau,\beta) = \infty$. This shows $\ell \cap K$ is always a
    finite number of arcs, hence $\ell$ is proper. \qedhere 

  \end{proof}
  
  \subsection{Subsurfaces}

  A closed subset $X \subset S$ is called a \emph{subsurface} if it is the
  image of a proper embedding of a bordered, possibly disconnected,
  surface. We further require each compact boundary component of $X$ to be an
  essential curve. In particular, no component of $X$ can be a closed disk with at most one puncture. A connected subsurface $X$ is called
  \emph{essential} if its double has either negative Euler characteristic
  or is of infinite type. Topologically, this rules out annuli and closed disks with at most two points removed from the boundary. Note that $X$ is a subsurface if and only if $Y = \overline{S
  \setminus X}$ is also a subsurface. We will usually consider a subsurface
  up to proper isotopy.
  
  Two subsurfaces in $S$ are \emph{disjoint} if they have disjoint
  representatives. Similarly, a curve or line or geodesic lamination is
  disjoint from a subsurface if they have disjoint representatives. Note
  that in this definition, the boundary of a subsurface is consider
  disjoint from the subsurface. 
  
  If we fix a hyperbolic metric on $S$, then for a non-annular subsurface
  $X$ of $S$, the interior of $X$ admits a canonical representative
  $X^\circ$ whose  metric completion $X^*$ is a hyperbolic surface
  with totally geodesic boundary homeomorphic to $X$. We will call $X^\circ$ the \emph{geodesic representative of the
  interior} of $X$. 
  
   Let $\overline{X^\circ}$ be the closure of $X^\circ$, and $\partial X^\circ =
  \overline{X^\circ} \smallsetminus X^\circ$, which is a disjoint union of
  simple closed geodesics or geodesic lines. Let $\partial_{sa} X^\circ$ be the the collection of components of $\partial X^\circ$ such that both boundary components of a regular neighborhood are contained in $X^\circ$. We define the \emph{almost geodesic representative} of $X$ as
  $$X^*=\overline{X^\circ}\ssm \bigcup_{\alpha\in \partial_{sa} X^\circ} N(\alpha)$$
  where the $N(\alpha)$ are pairwise disjoint open regular neighborhoods of the components. Note that $\overline{X^\circ}$ is homeomorphic to $X^*$ if and only if $\partial_{sa}X^\circ$ is empty.

\begin{figure}[h]
	\begin{center}
		\includegraphics{almostgeodrep}
		\caption{An example of a subsurface $X$ so that $\overline{X^\circ}$ is not homeomorphic to $X^*$}
	\end{center}

\end{figure}

We say that a collection $\C$ of curves and lines \emph{fills} a (sub)surface if every non-peripheral curve has positive intersection number with some curve or line in $\C$.
  \subsection{Some results on isometries of surfaces}

  The following well known statement follows from the proof of
  \cite[Theorem 2.7]{cb_automorphisms}.

  \begin{lemma}\label{lem:periodic-finite-type}
    Let $F$ be a finite-type surface of negative Euler characteristic and
    $f$ a mapping class of $F$. If $F$ is filled by a finite collection of
    $f$--periodic curves, then $f$ has finite order.
  \end{lemma}

  \begin{remark}
    Note that the order of $f$ in the previous lemma can be arbitrarily
    higher than the periods of the curves. For instance, look at the torus
    $\R^2 /\Z^2$ and at the curves $\alpha$ and $\beta$, obtained as the
    quotients of the lines $y=nx$ and $y=-nx$. Puncture the torus at points
    that are the projections of $\left(\frac{1+2k}{2n},0\right)$, for
    $k=0,\dots, n-1$, and $\left(\frac{k}{n},\frac{1}{2}\right)$, for
    $k=0,\dots, n-1$. Then the map induced by $(x,y)\mapsto
    \left(x+\frac{1}{2n},y+\frac{1}{2}\right)$ has order $2n$, but it fixes
    both $\alpha$ and $\beta$.
  \end{remark}

  We will also need a result about isometries of infinite-type
  surfaces, whose proof is essentially borrowed from
  Afton--Calegari--Chen--Lyman's work (\cite{accl_Nielsen}).

  \begin{prop}\label{prop:Nielsenrealization}
    Let $S$ be an infinite-type surface with boundary and $f$ a
    homeomorphism of $S$ with $f^n$ homotopic to the identity for some $n >
    0$. For each compact boundary $\alpha$ of $S$, choose an $f$-invariant
    positive number $\ell(\alpha)$. Then there exists a hyperbolic metric
    on $S$ with totally geodesic boundary such that each compact boundary
    $\alpha$ has length $\ell(\alpha)$ and a periodic isometry of $S$
    isotopic to $f$.
  \end{prop}

  \begin{proof}
    Suppose first that $S$ has only compact boundary components. By using
    Nielsen's work \cite{nielsen_abbildungsklassen} (which shows that the
    proposition holds if $S$ is of finite type and has compact boundary)
    and Afton--Calegari--Chen--Lyman's argument (see the proof of
    \cite[Theorem 2]{accl_Nielsen}), we can prove the proposition in this
    case.

    If there are noncompact boundary components, then let $D(S)$ be the
    double of $S$ along its noncompact boundary components and extend the
    map $f$ to a map $\hat{f}$ of $D(S)$. Let $i$ be the involution on
    $D(S)$ with quotient $S$, which commutes with $\hat{f}$ by construction. By
    \cite{accl_Nielsen}, there is an $i$--invariant metric on $S$ and a
    periodic isometry $g$ isotopic to $\hat{f}$. Since $i$ is an isometry,
    the boundaries of $S$ are realized as geodesics in $D(S)$. Therefore,
    $g$ also commutes with $i$, and we get a periodic isometry of $S$
    isotopic to $f$. \qedhere
  \end{proof}

\section{Limit sets}

  Throughout this section, $S$ will be a surface without boundary and
  equipped with a hyperbolic metric, though everything we say will be
  independent of the choice of such metric. Indeed, the following fact
  holds (see \cite[Theorem 3.6]{saric_train} for the result in the
  infinite-type setting).
  
  \begin{thm}\label{thm:geods_correspondence} 
    Let $m$ and $m'$ be two hyperbolic metrics on $S$. Then the identity map on $S$ yields a
    natural identification of the boundary at infinity of the universal
    cover of $(S,m)$ and the boundary at infinity of the universal cover of
    $(S,m')$. This induces a homeomorphism between complete geodesics in
    the universal cover of $(S,m)$ and complete geodesics in the universal
    cover of $(S,m')$, which descends to a homeomorphism between complete
    $m$-geodesics and complete $m'$-geodesics on $S$.
  \end{thm}

  Given a complete $m$-geodesic $\ell$, we call the corresponding
  $m'$-geodesic the \emph{$m'$--straightening} of $\ell$. For a collection
  of $m$--geodesics, its \emph{$m'$--straightening} is the union of the
  $m'$--straightenings of its geodesics. A consequence of Theorem
  \ref{thm:geods_correspondence} is the following: 
    
  \begin{lemma}
    Let $m$ and $m'$ be two hyperbolic metrics on $S$. If two complete
    $m$--geodesics $\ell_1$ and $\ell_2$ are simple and disjoint, then the
    $m'$--straightenings $\ell_1$ and $\ell_2$ are also simple and
    disjoint. If $\lambda$ is an $m$--geodesic lamination, its
    $m'$--straightening is an $m'$--geodesic lamination.
  \end{lemma}
  
  The reason why the lemma holds is that (self-)intersections correspond to
  linked pairs of endpoints at infinity of lifts and convergence of
  geodesics corresponds to convergence of pairs of endpoints at infinity of
  lifts. Moreover, we have:
  
  \begin{prop}
    Let $m$ and $m'$ be hyperbolic metrics on $S$ and $\lambda$ an
    $m$--geodesic lamination. Then there is an ambient isotopy which maps
    $\lambda$, leaf by leaf, to its $m'$--straightening $\lambda'$.
  \end{prop}

  \begin{proof}
    Pick an $m$--geodesic pants decomposition $\mathcal{P}$ which doesn't
    contain any leaf of $\lambda$ (this can be done by choosing a pants
    decomposition and doing enough elementary moves until none of its
    curves are  in $\lambda$). In particular $\mathcal{P}$ is in minimal
    position with respect to $\lambda$. We can then find an ambient isotopy
    $H_1$ of $S$ sending $\mathcal{P}$ to $\mathcal{P}'$, the
    $m'$--straightening of $\mathcal{P}$. Let $\lambda_1$ be the image
    under the isotopy of $\lambda$. Note that $\mathcal{P}'$ is in minimal
    position with respect to $\lambda_1$ and doesn't contain any leaf of
    $\lambda_1$.

    Next, given a curve $\alpha$ in the pants decomposition $\mathcal{P}'$,
    lift $\alpha$, $\lambda_1$ and $\lambda'$ to the universal cover of
    $(S,m')$. Look at a lift $\beta$ of $\alpha$ and pick an orientation on
    it; every lift $\ell_1$ of a leaf in $\lambda_1$ intersecting $\beta$
    corresponds to a lift $\ell_1'$ of a leaf of $\lambda'$ with the same
    endpoints, thus $\ell_1'$ intersects $\beta$ as well. Moreover, if
    $\ell_2$ is another lift of a leaf in $\lambda_1$ intersecting $\beta$
    and $\ell_2'$ the corresponding $m'$--geodesic, the intersections of
    $\ell_1$ and $\ell_2$ with $\beta$ come in the same order as the
    intersections of $\ell_1'$ and $\ell_2'$ with $\beta$. So we can find
    an order-preserving homeomorphism from
    $\tilde{\lambda_1}\cap\beta$ to $\tilde{\lambda'}\cap\beta$ sending
    $\ell\cap\beta$ to $\ell'\cap \beta$, for every $\ell\subset
    \tilde{\lambda_1}$, where $\ell'$ is the $m'$-- straightening of
    $\ell$. Extend this to an equivariant isotopy of a small neighborhood
    of $\beta$. By doing this equivariantly at each lift and on disjoint
    neighborhoods, we get an isotopy of the universal cover which descends
    to an ambient isotopy $H_2$ of $S$, sending $\lambda_1$ to $\lambda_2$
    such that for every $\alpha$ in $\mathcal{P}'$,
    $\lambda_2\cap\alpha=\lambda'\cap \alpha$.

    \begin{figure}[ht]
    \begin{center}
    \begin{overpic}{lifts}
    \put(-1,18){$\beta$}
    \put(68,-6){$\ell_1$}
    \put(100,30){$\ell_2$}
    \put(22,61){\color{PrettyGreen}$\ell_1'$}
    \put(70,55){\color{PrettyGreen}$\ell_2'$}
    \end{overpic}
    \caption{Lifting a curve in $\mathcal{P}'$}
    \end{center}
    \end{figure}

    Look at a pair of pants $P$ of $\mathcal{P}'$. By the transversality of
    $\lambda_2$ and of $\lambda'$, $\lambda_2\cap P$ (respectively,
    $\lambda'\cap P$) is a union of arcs from $\partial P$ to $\partial P$.
    We claim that we can find an isotopy of $P$, fixing the boundary
    pointwise, sending $\lambda_2\cap P$ to $\lambda'\cap P$. Indeed, we
    can divide the arcs according to their homotopy class relative to the
    boundary. There are at most three classes, which are the same for the
    arcs in $\lambda_2$ and in $\lambda$, by how $\lambda_2$ was
    constructed. For each class we can find a rectangle $R$ with two sides
    on $\partial P$ and two arcs from $\lambda_2\cap P$ in the given
    homotopy class containing all arcs of $\lambda_2\cap P$ in the homotopy
    class. We can do the same for $\lambda'$ and get a rectangle $R'$, with
    the extra assumption that the four corners of $R$ are the same as the
    four corners of $R$. We can then choose an ambient isotopy, fixing
    $\partial P$ pointwise, sending $R$ to $R'$ and all arcs of
    $\lambda_2\cap R$ to the corresponding arcs of $\lambda'\cap R'$.

    \begin{figure}[ht]
    \begin{center}
    \begin{overpic}{rectanglesinpants}
    \put(20,35){$R$}
    \put(72,35){$R'$}
    \end{overpic}
    \caption{Isotopying rectangles}
    \end{center}
    \end{figure}

    By doing this for all homotopy classes we get the required isotopy of
    $P$. Repeating the procedure in each pair of pants, we get an ambient
    isotopy $H_3$ of $S$, fixing all curves in $P$ pointwise, and sending
    $\lambda_2$ to $\lambda'$.

    The composition of the three ambient isotopies is the required
    isotopy.\qedhere

  \end{proof}

  Let $\lambda \subset S$ be a closed subset. We say that a sequence of
  curves $\{\alpha_n\st n\in \N\}$ \emph{converges} to $\lambda$, and that
  $\lambda$ is the \emph{limit} of $\{\alpha_n\st n\in \N\}$, if for every
  finite-type subsurface $K \subset S$ with compact boundary, $\alpha_n^* \cap K$ converges to
  $\lambda \cap K$ with respect to the Hausdorff distance. We will also
  denote this by $\alpha_n \to \lambda$.

  Since we restrict to finite-type surfaces to define convergence, the standard proof (see
  e.g.\ \cite{cb_automorphisms}) applies to show:

  \begin{lemma}\label{lem:limit}
    If a closed subset $\lambda\subset S$ is the limit of a sequence of
    curves  $\{\alpha_n\st n\in \N\}$, then $\lambda$ is a geodesic
    lamination.
  \end{lemma}

  The \emph{limit set} of a sequence of curves $\{\alpha_n\st n\in \N\}$ is
  the set $\L(\{\alpha_n\st n\in \N\})$ given by all complete geodesics
  contained in the limit of some subsequence of $\{\alpha_n\st n\in \N\}$.

  Changing the metric on $S$ does not change the limit set in the following
  sense. 

  \begin{lemma}
    Let $m$ and $m'$ be two hyperbolic metrics on $S$. For a sequence
    $\{\alpha_n\}$ of curves, let $\L(\{\alpha_n\st n\in \N\},m)$ and
    $\L(\{\alpha_n\st n\in \N\},m')$ be their respective limit sets in the
    two metrics. Then a subsequence $\{\alpha_{n_j}\}_{j\in\N}$ converges
    to $\lambda \subset \L(\{\alpha_n\st n\in\N\},m)$ if and only if
    $\{\alpha_{n_j}\}_{j\in\N}$ converges to $\lambda' \subset
    \L(\{\alpha_n\st n\in \N\},m')$, where $\lambda'$ is the
    $m'$--straightening of $\lambda$.
  \end{lemma}

  \begin{proof}
    As convergence of geodesics corresponds to convergence of pairs of
    endpoints at the boundary at infinity, this follows from Theorem
    \ref{thm:geods_correspondence}.\qedhere
  \end{proof}

  This justifies our notation $\L(\{\alpha_n\st n\in\N\})$, which makes no
  reference to the metric on $S$. 
    
  \subsection{Limit sets under a homeomorphism}

  For a homeomorphism $f$ of $S$ and a curve $\alpha$, we say $\alpha$ is
  $f$--\emph{periodic} if there exists $n \ge 1$ such that $f^n(\alpha)$ is
  isotopic to $\alpha$. The smallest such $n$ is called the
  $f$--\emph{period} of $\alpha$. We say $\alpha$ is $f$--\emph{forward
  wandering} if $\{f^n(\alpha)^*\}_{n \ge 0}$ leaves every compact set of
  $S$, and $f$--\emph{backward wandering} if it is $f^{-1}$--forward
  wandering. The curve $\alpha$ is $f$--\emph{wandering} if it is both
  $f$--forward and $f$--backward wandering. These properties are
  independent of the hyperbolic metric on $S$ as well as perturbing $f$ by
  an isotopy. Thus, we will also adopt the same definition for a mapping
  class of $S$.  We will usually suppress the reference to $f$ when the
  context is clear. 

  Note that we can extend the definition of $f$--wandering to 
 lines, non-peripheral arcs or rays with endpoints on the boundary of $S$, 
  as well as to subsurfaces of $S$, by which we mean the geodesic representatives (of the interior, in case of subsurfaces) of their images under iterations by $f$ leave every compact set in
  $S$. 
     
  For $\alpha$ a curve, line, or non-peripheral arc or ray with endpoints on the boundary of $S$ , let $\L^+(\alpha) := \L(\{f^n(\alpha)\st n \ge
  0\})$ be the \emph{forward limit} of $\alpha$ under $f$, and
  $\L^-(\alpha) := \L(\{f^n(\alpha)\st n \le 0\})$ the \emph{backward
  limit}. Set $\L(\alpha):=\L^+(\alpha)\cup\L^-(\alpha)$. When $\alpha$ is
  $f$--periodic, then \[ \L^{\pm}(\alpha) =  \{\alpha^*, f(\alpha)^*, \dots
  , f^{n-1}(\alpha)^*\}, \] where $n \ge 1$ is the $f$--period of $\alpha$.
  Moreover, $\alpha$ is forward wandering if and only if $\L^+(\alpha) =
  \emptyset$, and it is backward wandering if and only if $\L^-(\alpha) =
  \emptyset$.

  We establish some basic facts.

  \begin{lemma}\label{lem:tame-intersection}
    
    Let $f$ be a homeomorphism and $\alpha$ a curve. Suppose $\lambda$ is
    the limit of a sequence of iterates $\{f^{n_j}(\alpha)\st j\in\N\}$.
    Then for any curve $\beta$, $i(\beta,\lambda) \ne 0$ if and only if
    $i(\beta,f^{n_j}(\alpha)) \ne 0$ for all sufficiently large $j$.
    Moreover, if $i(\beta,\lambda) \ge k$, then $i(\beta,f^{n_j}(\alpha))
    \ge k$ for all sufficiently large $j$. 

  \end{lemma}

  \begin{proof}

    If we look at an annular neighborhood $K$ of $\beta$,
    $f^{n_j}(\alpha)^*\cap K$ converges to $\lambda \cap K$ in the
    Hausdorff metric, so the first statement follows. If $\beta$ intersects
    $\lambda$ at least $k$ times, then we can find $k$ arcs in $K \cap
    \lambda$. Choose sufficiently small neighborhood about each arc so that
    they are pairwise disjoint. Then for all sufficiently large $j$,
    $f^{n_j}(\alpha) \cap K$ will pass through each of these neighborhoods,
    so $i(\beta,f^{n_j}(\alpha)) \ge k$. \qedhere 

  \end{proof}

  \begin{lemma}\label{lem:duality}
    Let \(f\) be a homeomorphism. Then for any two (not necessarily
    distinct) curves \(\alpha\) and \(\beta\) we have:
    \[i(\alpha,\L^+(\beta))\neq 0 \; \Leftrightarrow \;
    i(\L^-(\alpha),\beta)\neq 0.\]
  \end{lemma}

  \begin{proof}
    Suppose \(i \left( \alpha,\L^+(\beta) \right)\neq 0\). Let
    \(n_j\to\infty\) be a sequence so that $f^{n_j}(\beta) \to \lambda
    \subset \L^+(\alpha)$ with \(i(\alpha,\lambda)\neq 0\). By Lemma
    \ref{lem:tame-intersection}, for every \(j\) large enough,
    \[i(\alpha,f^{n_j}(\beta))\neq 0\]
    and thus
    \[i(f^{-n_j}(\alpha),\beta)\neq 0.\] In particular, $\alpha$ is not
    backward wandering. By taking a further subsequence so that
    $f^{-n_{j_k}}(\alpha) \to \lambda' \subset \L^-(\alpha)$, we have,
    again by Lemma \ref{lem:tame-intersection}, \(i(\lambda',\beta)\neq
    0\), i.e.\ \(i(\L^-(\alpha),\beta)\neq 0\). By replacing $f$ by
    $f^{-1}$ we get the other implication. \qedhere 

  \end{proof}

  \begin{lemma}\label{lem:not-wandering} 

    Let $f$ be a homeomorphism. Then a curve $\alpha$ is forward wandering
    if and only if \(i(\alpha,\L^-(\beta)) = 0\) for all curves $\beta$.
    Similarly, $\alpha$ is backward wandering if and only if 
    \(i(\alpha,\L^+(\beta)) = 0\) for all $\beta$.

  \end{lemma}

  \begin{proof}
    
    If $\alpha$ is not forward wandering, then we can choose a curve
    $\beta$ with $i(\L^+(\alpha),\beta) \ne 0$. By Lemma
    \ref{lem:duality}, this implies $i(\alpha,\L^-(\beta)) \ne 0$. Conversely, if
    \(i(\alpha,\L^-(\beta))\neq 0\), then again by Lemma \ref{lem:duality},
    \(i(\L^+(\alpha),\beta)\neq 0\). In particular
    \(\L^+(\alpha)\neq\emptyset\), so \(\alpha\) is not forward wandering.
    \qedhere

  \end{proof}
  
  \begin{cor}\label{cor:not-wandering}
    Let $f$ be a homeomorphism. If $\alpha$ is a periodic curve and $\beta$
    is wandering, $i(\alpha,\beta)=0$.
  \end{cor}
  \begin{proof}
    If $i(\alpha,\beta)\neq 0$, $i(\L^+(\alpha),\beta)\neq 0$ (as
    $\alpha\in \L^+(\alpha)$), so $\beta$ is not wandering by Lemma
    \ref{lem:not-wandering}.\qedhere
  \end{proof}

\section{Tame maps}

  \label{sec:tame}

  As usual, assume $S$ is a surface without boundary endowed with a
  hyperbolic metric. We say a homeomorphism $f$ of $S$ is \emph{tame} if
  for every finite-type subsurface $K \subset S$ with compact boundary and every curve $\alpha$ there
  are $N=N(K,\alpha)\in \N$ and finitely many isotopy (relative to
  $\partial K$) classes of arcs from $\partial K$ to $\partial K$ and curves in $K$ such that, for every
  $n\in\N$, the intersection $f^n(\alpha)^*\cap K$ has at most $N$
  components and each is in one of the given isotopy classes. Tameness can
  be characterized by looking at intersections of curves, as the
  following lemma shows. 


  \begin{lemma} \label{lem:tame-properties}
   
    Let $f$ be a homeomorphism of $S$. The following statements are
    equivalent. 
    
    \begin{enumerate}
      \item $f$ is tame.
      \item $f^{-1}$ is tame.
      \item For curves $\alpha$ and $\beta$, $i(f^n(\alpha),\beta)$ is
        uniformly bounded for all $n \ge 0$.
      \item For curves $\alpha$ and $\beta$, $i(f^n(\alpha),\beta)$ is
        uniformly bounded for all $n \le 0$.
      \item For curves $\alpha$ and $\beta$, $i(f^n(\alpha),\beta)$ is
        uniformly bounded for all $n$.
    \end{enumerate}
    In particular, the notion of tame is independent of the hyperbolic
    metric on $S$.
  \end{lemma}

  \begin{proof}
    The equivalence of (3), (4), and (5) are immediate as the action of $f$
    preserves geometric intersection number. That (1) implies (3) follows
    from taking the compact surface to be an annular neighborhood of the
    curve $\beta$. On the other hand, if $f$ is not tame, then there exists
    a finite-type subsurface $K$ with compact boundary, a curve $\alpha$, and a subsequence $\{n_j\}
    \subset \N$ such that $f^{n_j}(\alpha)^* \cap K$ has $N_j$ components
    with $N_j \to \infty$ and $j \to \infty$, or includes components with
    increasing multiplicity. Then there exists a curve $\beta\subset K$
    such that $i(f^{n_j}(\alpha),\beta) \to \infty$. This shows the
    equivalence of (1) and (3), as well as (2) and (4) by symmetry, whence
    the equivalence of all statements.  Property (5) is independent of the
    hyperbolic metric on $S$, so the same is true of being tame. \qedhere 
  \end{proof}

  We say a mapping class $f \in \mcg(S)$ is \emph{tame} if it has a tame
  representative. From the characterization of Lemma
  \ref{lem:tame-properties}, if $f$ has a tame representative, then all of
  its representatives are tame. We call a
  tame homeomorphism $f$ \emph{extra tame} if the limit set \(\L(\alpha)\)
  is finite for every curve \(\alpha\). This property is also inherited by
  the mapping class of $f$. 
  
  As an example, for a map of a finite-type surface it is equivalent to be tame, extra tame, periodic and isotopic to an isometry for some hyperbolic structure. In the infinite-type case, the situation is more complicated. Being periodic implies being isotopic to an isometry (with respect to some hyperbolic structure), which implies being extra tame, and extra tame maps are tame by definition. No implication is an equivalence, though. Consider for instance a translation of a surface $S$, which --- as mentioned in the introduction --- is a map $f$ that generates an
  infinite cyclic group acting properly on $S$. If $S$ has infinite type,
  then the quotient surface $X=S/\langle f \rangle$ has negative Euler
  characteristic or has infinite type, so we can lift a hyperbolic metric from $X$ to $S$ so
  that $f$ acts by an (infinite-order) isometry. In this case, every curve on $S$ is
  wandering, so $f$ is (extra) tame. 
  
 Another example is what we call an \emph{irrational rotation}. Here, we
  start with a homeomorphism of the circle with minimal invariant subset a
  Cantor set $C$ (see the Denjoy construction \cite[Section
  12.2]{kh_introduction}). Think of the circle as the equator of the
  two-sphere $S^2$, and extend the map to a homeomorphism of $S=S^2
  \smallsetminus C$.
  This map is tame but \emph{not} extra tame: every curve has an infinite limit set, given by lines which can intersect. More complicated examples of (extra) tame maps can be found in Sections \ref{sec:examples} and \ref{sec:structure}.
  
  On the other hand, a Dehn twist is not tame: if $\tau$ is the twist about a curve $\alpha$, let $\beta$ be a curve intersecting $\alpha$ essentially. If $K$ is a finite-type subsurface with compact boundary containing $\alpha$ and $\beta$, the curves $\tau^n(\beta)\cap K=\tau^n(\beta)$ are all distinct. Another example of a map which is not tame is a homeomorphism which restricts to a pseudo-Anosov on some finite-type subsurface (with compact boundary).


  The goal of the remainder of this section is to describe properties of limit sets of
  curves under tame homeomorphisms. The first result is that if a curve is
  not periodic, its limit set can only contain lines.

  \begin{prop} \label{prop:tame-limits}
    
    Let $f$ be a tame homeomorphism and $\alpha$ a non-periodic curve.
    If $\alpha$ is not forward wandering, then $\L^+(\alpha)$ is a
    non-empty collection of lines. Similarly, if $\alpha$ is not
    backward wandering, then $\L^-(\alpha)$ is a non-empty collection of lines.

  \end{prop}

  \begin{proof}
    
    Assume that $\alpha$ is not forward wandering. For simplicity, let
    $\alpha_n = f^n(\alpha)^*$. If $\L^+(\alpha)$ is empty, then for every
    compact subsurface $K \subset S$, $\bigcup_{n \ge 0} \alpha_n \cap K$
    has no accumulation points. That is, $\bigcup_{n \ge 0} \alpha_n \cap
    K$ is given by finitely many arcs or curves. If two geodesics coincide
    along a non-trivial arc, then they are equal. Since $\alpha$ is not
    $f$--periodic, we must have $\alpha_n \cap K = \emptyset$ for all
    sufficiently large $n$. But this contradicts the assumption that
    $\alpha$ is not forward wandering. So $\L^+(\alpha)$ is a non-empty
    union of complete simple geodesics by Lemma \ref{lem:limit}. If
    $\L^+(\alpha)$ has a non-proper leaf $\ell$, then there exists a curve
    $\beta$ such that $i(\beta,\ell) = \infty$, by Lemma \ref{lem:proper}.
    Let $\lambda \subset \L^+(\alpha)$ be the limit of some subsequence
    $\{\alpha_{n_j}\}$ containing $\ell$. By Lemma
    \ref{lem:tame-intersection}
    $i(f^{n_j}(\alpha),\beta)=i(\alpha_{n_j},\beta) \to \infty$,
    contradicting the characterization of tameness given by Lemma
    \ref{lem:tame-properties}. If $\L^+(\alpha)$ has a compact component
    $\beta$, then since $\L^+(\alpha)$ has no non-proper leaves, $\beta$ is
    the limit of some subsequence $\{\alpha_{n_i}\}$, which has to be
    eventually constant (again by tameness). But this contradicts the
    non-periodicity of $\alpha$. Thus $\L^+(\alpha)$ is a non-empty union
    of lines. By replacing $f$ by $f^{-1}$ we get the second statement.
    \qedhere 

  \end{proof}
  
  As $\L^{\pm}(\alpha)$ is well defined for mapping classes, we have
  the following immediate corollary.
  
  \begin{cor} \label{cor:tame-limits}
    For a tame mapping class $f$ and a curve $\alpha$, the following
    statements hold. 
    \begin{enumerate}
      \item $\L^+(\alpha)=\emptyset$ if and only if $\alpha$ is
        forward wandering. 
      \item $\L^-(\alpha)=\emptyset$ if and only if $\alpha$ is
        backward wandering. 
      \item $\L(\alpha)=\emptyset$ if and only if $\alpha$
        is wandering. 
      \item $\L(\alpha)$ contains a curve if and only if
        $\alpha$ is periodic, in which case
        $\L(\alpha)=\L^+(\alpha)=\L^-(\alpha)$ is the orbit of $\alpha$
        under $f$.
    \end{enumerate}
  \end{cor}

  Another consequence of tameness is that limits of iterates of curves can
  only intersect finitely many times.

  \begin{lemma}
  
    Let $f$ be a tame homeomorphism. Then for any two (not necessarily
    distinct) curves $\alpha$ and $\beta$, if $f^{n_j}(\alpha) \to
    \lambda \subset \L(\alpha)$ and $f^{m_k}(\beta) \to \lambda' \subset
    \L(\beta)$, for some sequences $n_j$ and $m_k$, then
    $i(\lambda,\lambda') < \infty$.

  \end{lemma}

  \begin{proof}

    If $i(\lambda,\lambda') = \infty$, then
    for every $N$ there are $j,k$ such that
    $i(f^{n_j}(\alpha),f^{m_k}(\beta))\geq N$. Thus, $i(\alpha,
    f^{m_k-n_j}(\beta))\geq N$, so $f$ is not tame by Lemma
    \ref{lem:tame-properties}.\qedhere
    
  \end{proof}
  
\section{Subsurfaces and spanning sets of curves and lines}

  \label{sec:subsurfaces}

  Let $\C$ be a collection of curves or lines on $S$. Consider the collection of
  subsurfaces $\Sigma \subset S$, defined up to proper isotopy, such that $\Sigma$
  contains the proper isotopy class of every element of $\C$, possibly as a boundary
  component. We allow $\Sigma$ to coincide with $S$. The collection of
  all such subsurfaces is nonempty and partially ordered by inclusion. A
  minimal subsurface within this collection, if it exists, is said to be
  \emph{spanned} by $\C$.
  
  When $\C$ is a finite collection, then there is always a subsurface $F$
  (of finite type) spanned by $\C$. Namely, put the elements in $\C$ in
  general position. Then $F$ is obtained by taking a regular neighborhood
  of the union of the curves and lines and filling in all disks with at most one
  puncture. More generally, a subsurface spanned by $\C$ exists if $\C$ is
  \emph{locally finite}, by which we mean every compact set of $S$
  essentially intersects only finitely many elements in $\C$. In general,
  however, there are collections of curves and lines which do not span a subsurface
  -- see for instance Figure \ref{fig:notspanning}. The main goal of this
  section is to give a condition on $\C$ which ensures that it spans a
  subsurface.

  \begin{figure}[ht]
  \begin{center}
  \includegraphics{notspanning}
  \caption{The beginning of a collection of curves not spanning a
    subsurface}\label{fig:notspanning}
  \end{center}
  \end{figure}
\subsection{Concatenations of curves}
  Fix a hyperbolic metric on $S$. We say a curve $\alpha$ is a
  \emph{concatenation} of curves in $\C$ if it is homotopic to a finite
  concatenation of piecewise geodesic arcs coming from the geodesic
  representatives of curves in $\S$. We say $\C$ is \emph{closed under
  concatenation} if every curve obtained as a concatenation of curves in
  $\C$ belongs to $\C$. 
  
  Note that if $\C$ spans a subsurface $\Sigma$, then necessarily $\C$
  \emph{fills} $\Sigma$, i.e.\ every essential and non-peripheral curve or
  line in $\Sigma$ intersects some element of $\C$. We further have:

  \begin{lemma}\label{lem:concatenation} Let $\C$ be a collection of curves
  spanning a subsurface $\Sigma$. Then any essential curve in $\Sigma$ is a
  concatenation of curves in $\C$. \end{lemma}

\begin{proof}
Fix a hyperbolic structure on $S$. Let $\gamma$ be a curve in $\Sigma$ and suppose first it is non-peripheral. Enumerate the curves in $\C$ as $\C=\{\gamma_1,\gamma_2,\dots\}$ and assume that they are geodesic. As $\gamma$ is compact, we can find a finite-type subsurface $K$ of $\Sigma$ with compact totally geodesic boundary such that $\gamma\subset K$ and is non-peripheral in $K$. Inductively define the following curves:
\begin{itemize}
	\item $\beta_1=\gamma_i$, where $i$ is the smallest index so that $\gamma_i\cap K\neq \emptyset$;
	\item given $\beta_1,\dots,\beta_j$, let $\beta_{j+1}$ be $\gamma_i$, where $i$ is the smallest index so that $\gamma_i\cap K\neq \emptyset$ and the components of $\gamma_i\cap K$ are not all among the homotopy classes of the components of $\bigcup_{k\leq j}\beta_k\cap K$ --- unless such a $\gamma_i$ doesn't exist, in which case we stop the process.
\end{itemize}
Since $K$ is of finite type, the process stops and we end up with a finite collection $\C_K=\{\beta_1,\dots\beta_m\}$. Let $F$ be the subsurface spanned by $\C_K$. Note that $K\subset F$ (up to homotopy): otherwise there is some curve $\alpha\subset K\ssm F$, and since $\C$ fills $\Sigma$, there is some $\gamma_i\in \C$ intersecting $\alpha$. Therefore the components of $\gamma_i\cap K$ are not all among the homotopy classes of the components of $\bigcup_{k\leq m}\beta_k\cap K$, a contradiction.

Since $F$ can be constructed by taking a regular neighborhood of $\bigcup_{k\leq m}\beta_k$ and filling in disks with at most one puncture, and $\gamma\subset F$, it is easy to show that $\gamma$ is concatenation of curves in $\C_K$, and thus in $\C$.

If now $\gamma$ is peripheral, one can show that it is the concatenation of non-peripheral curves in $\Sigma$. By using the result for these curves, we deduce that also $\gamma$ is a concatenation of curves in $\C$.
\end{proof}

  \begin{lemma}\label{lem:periodic&wandering}
    Let $f$ be a mapping class on $S$. Every closed curve obtained as a
    finite concatenation of $f$--periodic curves is periodic, and every
    closed curve obtained as a finite concatenation of wandering curves is
    wandering. 
  \end{lemma}

  \begin{proof}
    We first consider periodic curves. A curve is a finite concatenation of
    curves $\gamma_1,\dots, \gamma_n$ if and only if it is a curve in the
    subsurface $F$ spanned by $\{\gamma_1,\ldots,\gamma_n\}$ (by Lemma
    \ref{lem:concatenation}). Note that $F$ is connected and has negative
    Euler characteristic. Since each $\gamma_i$ is $f$--periodic, we can
    take a suitable power so that $f^k$ fixes $\gamma_i$ for all $i$. Since
    $F$ is spanned by $\{\gamma_i\}$, $f^k(F)$ is spanned by
    $\{f^k(\gamma_i)\}=\{\gamma_i\}$, so $f^k(F)$ is isotopic to $F$. Since
    $F$ has negative Euler characteristic, $f^k$ restricted to $F$ is
    isotopic to a periodic map of $F$, so every curve in $F$ is
    $f$--periodic. 
    
    The proof for wandering curves is similar. Let $F$ be the subsurface
    spanned by a finite collection $\{\gamma_1,\ldots,\gamma_n\}$ of
    wandering curves. It is enough to prove that for all compact subsurface
    $K \subset S$, there exists $k_0 \in \N$ such that $f^{\pm k}(F)$ can
    be homotoped to be disjoint from $K$ for all $k \ge k_0$. Since
    $\{\gamma_1,\ldots,\gamma_n\}$ are wandering, we can find such $k_0$ so
    that $f^{\pm k}(\gamma_i)^* \cap K = \emptyset$ for all $k \ge k_0$ and
    all $i$. Then $f^{\pm k}(F)$ is spanned by $\{f^{\pm k}(\gamma_i)\}$,
    so $f^{\pm k}(F)$ can be homotoped away from $K$. \qedhere
  \end{proof}

  \subsection{Good collection of curves}

  In the following, fix a hyperbolic metric on $S$ and let $\C$ be a
  collection of curves. We
  would like to construct a candidate for the subsurface spanned by $\C$,
  by enumerating the curves in $\C$ and taking the geodesic subsurface
  spanned by the first $n$ curves and then taking their union. However, this union might not be the right candidate, if there are annular components or
  boundaries that are homotopic --- think for instance of the subsurface spanned by all curves disjoint from a given one. To deal with these issues, we collect
  those curves in $\C$ that contribute to the annular components, namely
  let 
  \[
    \Cis:=\{\gamma\in\C\st \forall\gamma'\in\C,\; i(\gamma,\gamma')=0\}.
  \] 
  Enumerate the remaining curves: \(\Cint:=\C\ssm\Cis=\{\gamma_0,\gamma_1,\dots\}\). Up to reordering, we can assume that
  $i(\gamma_0,\gamma_1) \ne 0$. Then for every \(i\geq 1\), let
  \[\Cint^{i}:=\{\gamma_j\st ´j\leq i \;\mbox{and}\; \exists\;j'\leq
  i \text{ such that } i(\gamma_j,\gamma_{j'})\neq 0\}.\] Note that
  $\Cint^{1} = \{\gamma_0,\gamma_1\}$, $\Cint^{i} \subset \Cint^{i+1}$, and
  \(\Cint=\bigcup_{i\geq 1}\Cint^{i}\). Moreover, to go from $i$ to $i+1$, either we add
  a curve that intersects some curve in $\Cint^i$, or we add two curves
  that intersect each other.  In particular, each component of the subsurface spanned by $\Cint^{i}$ has negative Euler characteristic. 
  
  We now let \(F^{i}=F^{i}(\Cint)\) be the
  geodesic representative of the \emph{interior} of the subsurface spanned
  by \(\Cint^{i}\) and define
  \[F=F(\Cint):=\bigcup_{i\geq 1}F^{i}.\]
  By construction every connected component of \(F^{i}\) is an open
  subset (homotopic to a subsurface), and thus $F$ is an open subset of $S$. We call $F$ the
  \emph{set spanned} by $\Cint$.

  \begin{definition}
  We will a collection $\C$ of curves is \emph{good} if:
  \begin{enumerate}
    \item[(a)] the curves in \(\Cis\) do not accumulate anywhere, and
    \item[(b)] for every \(p\in\partial F\) there is a closed disk \(B\)
      centered at \(p\) such that \(B\cap F\) is either \(B\) minus a
      diameter or a connected component of \(B\) minus a diameter. 
  \end{enumerate}
  \end{definition}

  \begin{figure}[htp]
  \begin{center}
  \begin{overpic}{notspanning2}
  \put(49,-4){$p$}
  \end{overpic}
  \vspace{.2cm}
  \caption{A collection of curves which is not good}\label{fig:notspanning2}
  \end{center}
  \end{figure}
  
  For instance, the collection of curves depicted in Figure
  \ref{fig:notspanning} is not good, because isolated curves accumulate
  somewhere. Similarly, the collection in Figure \ref{fig:notspanning2} is
  not good either, because condition (b) is not satisfied at the point
  \(p\) in the drawing (onto which the curves accumulate). In both
  examples, the collection of curves do not span a subsurface. Indeed, we
  will show that the condition of being good is equivalent to $\C$
  admitting a spanning subsurface.
  
  Let $K \subset S$ be a finite-type subsurface with compact totally geodesic boundary.
  Each $F^{i}$ intersects $K$ in a disjoint union of open subsurfaces and
  disks with at most one puncture. By a \emph{rectangle} we mean a
  connected component $R$, homeomorphic to a closed disk, of $K \cap F^{i}$, with boundary split into \emph{horizontal sides}
  lying in $\partial F^{i}$ and \emph{vertical sides} lying in $\partial
  K$. All components of $K \cap F^{i}$ which are not rectangles are called
  \emph{essential}. Let $K^{i}_e$ be the union of all the essential
  components of $K \cap F^{i}$.  
  
  \begin{figure}[h]
  	\begin{center}
  		\includegraphics[scale=.8]{Ke+rectangles}
  	\caption{A compact subsurface $K$, whose boundary is in green, and in orange the boundary of a subsurface $F^i$. The shaded part is the intersection, formed by an essential component and a rectangle}
  	\end{center}
  \end{figure}

  \begin{lemma}\label{lem:stable-essential}
    For all $i \ge 2$, $K^{i}_e \subset K^{i+1}_e$, and for
    large $i$ it stabilizes. More precisely, there exists an open
    subsurface $K_e\subset K$ and such that for all sufficiently large $i$,
    $K^{i}_e$ is isotopic to $K_e$.
  \end{lemma}

  \begin{proof}
    This follows from Euler characteristic considerations. \qedhere
  \end{proof}

  So for any finite-type subsurface $K$ with compact totally geodesic boundary, there
  exists $I$ such that for all $i \ge I$, the topology of the essential
  components $K^{i}_e$ stabilizes, and further increasing $i$ only adds
  new rectangles or adjoins previous ones.  We define:
    \[ \mathcal{R}(K) =  \{ R: R \subset K \cap F^{i} \text{
    is a rectangle for some $i \ge I$} \}.\]
  Two parallel rectangles are \emph{equivalent} if they are contained in
  the same connected component of $F^{(i)}\cap K$ for some $i$.
  
  The following result is not hard to prove, and we leave it as an exercise.

  \begin{lemma}
    The relation define above is an equivalence relation on
    $\mathcal{R}(K)$. Moreover, suppose $R_1$, $R_2$, $R_3$ are three
    parallel rectangles with $R_2$ contained in the rectangle between $R_1$
    and $R_3$. If $R_1$ and $R_3$ are equivalent, then all three are
    equivalent.
  \end{lemma}

  So we can naturally organize parallel rectangles into equivalence classes
  of rectangles. As $i$ further increases the number of equivalence classes may
  increase. We have the following proposition.
  
  \begin{prop}\label{prop:span}
    Suppose \(\C\) is a collection of curves closed under concatenation.
    The following statements are equivalent:
    \begin{enumerate}
    \item there is a subsurface spanned by \(\C\); 
    \item the curves in \(\Cis\) do not accumulate anywhere and for every
      compact subsurface \(K\) with totally geodesic boundary, the number
      of equivalence classes in \(\mathcal{R}(K)\) is finite;
    \item \(\C\) is good.
    \end{enumerate}

Moreover, if there is a subsurface spanned by $\C$, it is properly homotopic to $F$.
  \end{prop}

  \begin{proof}

    [(1) \(\Rightarrow\) (2)] Let $\Sigma$ be a surface spanned by $\C$. By
    contradiction, suppose the curves in $\Cis$ accumulate somewhere. Then
    we can find a sequence of curves $\gamma_j\in\Cis$ and a curve $\alpha$
    such that $$i(\gamma_j,\alpha)\neq 0.$$
    $\Sigma$ is a subsurface, so $\Sigma\cap \alpha$ consists of finitely
    many arcs, with at least one, denoted $a$, intersecting at least
    three curves, say $\gamma_{j_1},\gamma_{j_2}$ and $\gamma_{j_3}$.
    Assume that $\gamma_{j_2}\cap a$ is between $\gamma_{j_1}\cap a$ and
    $\gamma_{j_3}\cap a$. Then the pair of pants containing
    $\gamma_{j_1}\cup a\cup \gamma_{j_3}$  is in $\Sigma$ and contains a
    curve $\beta$ intersecting $\gamma_{j_2}$ essentially. As $\beta$ is in
    $\Sigma$, by Lemma \ref{lem:concatenation} $\beta$ is a concatenation of curves in $\C$, and since $\C$ is closed under concatenation, $\beta\in\C$, which contradicts the definition of $\Cis$. So
    the curves in $\Cis$ cannot accumulate anywhere.

    If instead there is a finite-type subsurface $K$ with compact totally geodesic
    boundary such that $\mathcal{R}(K)$ has infinitely many equivalence
    classes, we find infinitely many parallel
    rectangles $R_i$ which are not equivalent. Let $\gamma_i$ be a curve in
    $F$ passing through $R_i$. As the $\gamma_i$ are, by construction,
    concatenations of pieces of curves in $\C$, they all are in $\Sigma$.
    As $\Sigma$ is a subsurface, $\Sigma\cap K$ is a finite union of
    connected components. In particular there is some component $X$ of
    $\Sigma\cap K$ containing $\gamma_i\cap K$ for infinitely many $i$.
    This implies that we can find distinct curves $\gamma_{i_1}$ and
    $\gamma_{i_2}$ such that the rectangle $R$ between $\gamma_{i_1}\cap K$
    and $\gamma_{i_2}\cap K$ is included in $\Sigma$. Thus the boundary of
    this rectangle is a concatenation of curves in $\C$, which implies that
    the rectangle between them is contained in $F^{(l)}\cap K$ for some
    $l$, so $R_{i_1}$ and $R_{i_2}$ are equivalent, a contradiction.

    [(2) \(\Rightarrow\) (3)] We just need to show property (b). By construction, $\partial F$ is a union of limits of simple closed
    geodesics. We claim that for every finite-type subsurface $K$ with compact totally geodesic boundary, $\partial F\cap K$ has finitely many connected components. If not, as $K$ is of finite type, there are infinitely
    many components $l_j$, $j\in \N$, of $\partial F \cap K$ which are parallel
    to a given arc from $\partial K$ to $\partial K$. Thus there are infinitely many rectangles between
    the $l_j$ which are not in $F$ and therefore infinitely many rectangles
    that are not equivalent, a contradiction.

    So for every $p\in\ell\subset F$ we can find a sufficiently small ball $B$ such that $\ell\cap B$ coincided with $\partial F\cap B$ and it is a single arc from $\partial B$ to $\partial B$. In particular, $B\ssm \partial F=B\ssm \ell$ has two components, each of which is either entirely contained in $F$ or in $S\ssm F$. Moreover, since $\ell\subset \partial F$, at least one of the components intersects, and hence is contained in, $F$. So $B$ is the required disk.
    
  [(3) \(\Rightarrow\) (1)] The condition on the boundary points shows that
    $\partial F$ is a union of lines and curves (i.e.\ it contains no
    non-proper component). Moreover, since each component is the limit of
    geodesics, for any $\ell\subset \partial F$, either
  \begin{itemize}
    \item for every $p\in\ell$ there is a closed disk $B$ centered at $p$
      such that $B\cap F$ is a connected component of a disk minus a
      diameter, or
    \item for every $p\in\ell$ there is a closed disk $B$ centered at $p$
      such that $B\cap F$ is a a disk minus a diameter.
  \end{itemize}
  Let $T_1$ be the collection of lines and curves in $\partial F$ for which
  the first condition holds and $T_2$ the other lines and curves in
  $\partial F$.

  For every $\ell\in T_2 \cup \Cis$ we can find an open regular
  neighborhood $N(\ell)$ such that:
  \begin{itemize}
  \item $N(\ell)$ is disjoint from $\partial F\cup \Cis\ssm\{\ell\}$, and
  \item if $\ell_1\neq \ell_2$, then $N(\ell_1)\cap N(\ell_2)=\emptyset$, and
  \item the neighborhoods don't accumulate anywhere. \end{itemize}
  
  We can find such neighborhoods by properness of components of $\partial
  F$ and by the non-accumulation assumption for $\Cis$ (we can just fix a
  compact exhaustion of the surface and construct the neighborhoods piece
  by piece).

  Define
  $$\Sigma:=\left(F\cup\bigcup_{\ell\in T_1}\ell\cup
  \bigcup_{\gamma\in\Cis}N(\gamma) \right)\ssm \bigcup_{\ell\in
  T_2}N(\ell).$$

  Let $X$ be the metric completion of
  \[F\cup\bigcup_{\gamma\in \Cis}N(\gamma).\]
  Then $X$ is a surface with boundary and we can construct a  proper
  embedding of $X$ into $S$ so that its image is $\Sigma$ (as before, we
  can look at a compact exhaustion of $S$ and construct proper embeddings
  of subsurfaces of $X$ to subsurfaces of $\Sigma$ so that their limit is
  the required embedding). So $\Sigma$ is a subsurface. Moreover, since
  $\Sigma$ is properly homotopic to $F$, it is spanned by $\C$.\qedhere
  \end{proof}
 
  We record here a consequence of having a non-good collection of curves.
  We will use it to prove that certain collections of curves span a
  subsurface.
  \begin{lemma}\label{lem:not-good}
 Suppose $\C$ is not a good collection of curves but closed under 
finite concatenations. Then there are simple closed geodesics $\alpha,\beta_i,\gamma_j$, for $i,j\in\N$, an arc $a=[p,q]\subset\alpha$ and a sequence $\{i_j\}\subset \N$ going to infinity so that:
\begin{enumerate}
	\item $\alpha,\gamma_j\notin \C$, for every $j$ and $\beta_i\in\C$ for every $i$,
	\item for every $i$, $a\cap \beta_i\neq \emptyset$ and $q$ is a limit of a sequence of intersections of $a$ with the $\beta_i$,
	\item for every $j$, there is a subarc $\tau_j$ of $a$ from $\beta_{i_j}$ to $\beta_{i_{j+1}}$ so that the union of the three spans a pair of pants whose third boundary component is $\gamma_j$ and $\tau_j$ is disjoint from $\beta_{i_k}$ for $k\neq j,j+1$.
\end{enumerate}
\end{lemma}

\begin{remark}\label{rmk:tau^n}
In the situation of the previous lemma, and given a homeomorphism $f$, fix an index \(j\). Look at \(f^n(\alpha), f^n(\beta_{i_j})\) and
\(f^n(\beta_{i_{j+1}})\) and consider an ambient isotopy which sends
them to their geodesic representatives. The
image through this isotopy of \(f^n(\tau_j)\) is an arc $\widetilde{f^n(\tau_j)}$. Then we denote by \(\tau_j^n\) the
geodesic arc in the homotopy class, relative endpoints, of $\widetilde{f^n(\tau_j)}$.
\end{remark}
 
  \begin{proof}
    By Proposition \ref{prop:span}, either there is a compact subsurface $K$
    with totally geodesic boundary such that $\mathcal{R}(K)$ contains
    infinitely many equivalence classes, or the curves in \(\Cis\)
    accumulate somewhere. 
    
    In the first case, we can find a sequence $R_i\in \mathcal{R}(K)$ which
    are all parallel and pairwise not equivalent. Let $\alpha$ be a
    boundary curve $K$ containing vertical sides of all $R_i$. Note that
    since $\alpha$ is compact, the vertical sides of the rectangles have
    accumulation points on $\alpha$. Up to passing to a subsequence, we can
    assume that there is a compact subarc $a\subset \alpha$ with endpoints
    $p$ and $q$ such that, if we orient $a$ from $p$ to $q$, we can find points $v_i\in R_i\cap a$ which converge monotonically to $q$.
  
    For every $i$ we can moreover find a curve $\beta_i\in \C$ passing
    through $R_i$; let $b_i$ be an arc of $\beta_i\cap K$ homotopic to
    $R_i$. 

    Let \(\beta_{i_j}\) be a subsequence of pairwise distinct curves so that:
    \begin{itemize}
      \item all strands of \(\beta_{i_j}\) passing though the rectangless
        intersect \(a\) before all the strands of \(\beta_{i_{j+1}}\), and
      \item between the last strand of \(\beta_{i_j}\) and the first strand
        of \(\beta_{i_{j+1}}\), \(a\) intersects at least two other
        \(\beta_k\).
    \end{itemize}
  
    Let \(\tau_j\) be the subarc of \(a\) between the last intersection of
    \(a\) and \(\beta_{i_j}\) and the first intersection of \(a\) and
    \(\beta_{i_{j+1}}\). Note that $\beta_{i_j}\cup\tau_j\cup
    \beta_{i_{j+1}}$ spans a pair of pants with $\beta_{i_j}$ and
    $\beta_{i_{j+1}}$ as two boundary components; let $\gamma_j$ be the
    third boundary component. Since the $R_i$ are not equivalent,
    $\gamma_j\notin \C$.

    Fix an index \(j\). Look at \(f^n(\alpha), f^n(\beta_{i_j})\) and
    \(f^n(\beta_{i_{j+1}})\) and consider an ambient isotopy which sends
    them to their geodesic representatives. We denote by \(\tau_j^n\) the
    image through this isotopy of \(f^n(\tau_j)\). Then \(\tau_j^n\) is a
    geodesic arc between the geodesic representative of
    \(f^n(\beta_{i_j})\) and the geodesic representative of
    \(f^n(\beta_{i_{j+1}})\).

    If the curves in \(\Cis\) accumulate somewhere, we can find a curve
    $\alpha$ intersecting infinitely many $\beta_i\subset \Cis$. As before,
    up to passing to a subsequence, we can find a compact subarc $a$ of
    $\alpha$ with endpoints $p$ and $q$ such that, if we orient $a$ from
    $p$ to $q$, $v_i:=\beta_i\cap a$ converges to $q$ monotonically. We
    then define $\beta_{i_j}$, $\gamma_j$ and $\tau_j^n$ as before. Since
    each $\gamma_j$ intersects a curve in $\Cis$, $\gamma_j\notin\C$.
    \qedhere

  \end{proof}
  
  \begin{figure}[H]
  \begin{center}
  \begin{overpic}{not-subsurface}
  \put(29.5,5){$a$}
  \put(84,31){$p$}
  \put(84,-2){$q$}
  \put(101,28){\color{FigGreen}$\beta_{i_1}$}
  \put(101,14){\color{FigGreen}$\beta_{i_2}$}
  \put(95,26){\color{FigLightBlue}$\gamma_1$}
  \put(86,24){\color{blue}$\tau_1$}
  \end{overpic}
  \vspace{.2cm}
  \caption{A choice for $a$ for the curves in Figure \ref{fig:notspanning},
    on the left-hand side; on the right-hand side, the general situation of
    Lemma \ref{lem:not-good}}
  \end{center}
  \end{figure}

  \subsection{Diagonal closure of lines}

  
  Given two rays $r_1$ and $r_2$, we say that they \emph{cobound a half-strip}
  if there is a proper embedding $\varphi:\R_{\geq 0}\times [0,1]\to S$
  such that $\varphi(\R_{\geq 0}\times \{0\})=r_1$ and $\varphi(\R_{\geq
  0}\times \{1\})=r_2$. We say that two lines $\ell_1$ and $\ell_2$ are
  \emph{asymptotic} in some direction if they contain rays $r_1\subset
  \ell_1$ and $r_2\subset \ell_2$ which cobound a half-strip.
  
  \begin{figure}[ht]
  \begin{center}
  \begin{overpic}{asymptotic}
  \put(35,50){$\ell_1$}
  \put(15,10){$\ell_2$}
  \end{overpic}
  \caption{Two asymptotic lines, with the half-strip cobounded by two
    rays}\label{fig:asymptotic}
  \end{center}
  \end{figure}
  
  If $L$ is a finite collection of proper lines, we denote by
  $\diagclosure{L}$ the subsurface of $S$ obtained as follows. First
  realize the lines in $L$ as geodesics. Then take the subsurface $\langle
  L \rangle$ spanned by $L$,  which is obtained by taking the union of the
  regular neighborhoods of the lines in $L$ and fill in every disk with at
  most one puncture. Then for every two (not necessarily distinct) lines
  and a direction in which they are asymptotic, we add to $\langle L
  \rangle$ the half-strip between two rays. Then fill in any additional disks
  with at most one puncture. We call $\diagclosure{L}$ the \emph{diagonal
  closure} of $L$, and the following lemma gives a characterization of
  $\diagclosure{L}$.  

   \begin{lemma}\label{lem:lines-outside-surfaces}
     Let $L=\{\ell_1,\dots,\ell_N\}$ be a finite collection of lines that
     do not end in an isolated planar end of $S$ and any two lines
     pairwise intersect finitely many times. Then for any line $\ell$ that
     does not end in an isolated planar end, the following
     statements are equivalent.
    \begin{enumerate}
      \item $\ell$ can be homotoped into $\diagclosure{L}$;
      \item for any curve $\alpha$, if $i(\alpha,\ell)\neq 0$, then
        $i(\alpha,\ell_j)\neq 0$ for some $j$;
      \item for every compact subsurface $K$, $\ell$ can be homotoped to
        $\ell'$ such that $\ell'\cap K\subset \langle L\rangle$.
    \end{enumerate}
    Moreover there is a finite union of curves $\mu$ such that if $\ell$ is
     a line in $\diagclosure{L}$, then $i(\mu,\ell)\neq 0$.
  \end{lemma}
  
  \begin{proof} We first prove the equivalence of the three conditions.
  
    [(1) $\Rightarrow$ (3)] For any compact subsurface $K$, there is a
    homotopy that homotopes all half-strips of $\diagclosure{L}$ away from $K$.
    We can then set $\ell'$ to be the image under this homotopy of $\ell$.

    [(3)$\Rightarrow$(2)] Let $\alpha$ be a curve with $i(\alpha,\ell)\neq
    0$. Let $K$ be an annulus around $\alpha$ and $\ell'$ a line so that
    $\ell'\cap K\subset \langle L\rangle$. Since $\ell'$ intersects
    $\alpha$, there is an arc of $\langle L\rangle$ crossing $\alpha$, so
    there is some $j$ such that $i(\alpha,\ell_j)\neq 0$.

    [(2)$\Rightarrow$ (1)] We prove that if $\ell$ cannot be homotoped into
    $\diagclosure{L}$, there is some curve $\alpha$ intersecting $\ell$ and
    disjoint from all $\ell_j$. Assume that all lines are geodesic and
    $\ell$ is in minimal position with respect to $\diagclosure{L}$.

    Note that $\diagclosure{L}$ is the union of a compact surface with
    finitely many half-strips. In particular the boundary of $\diagclosure{L}$
    is a finite union of circles and lines. So the same holds for
    $\Sigma:=\overline{S\ssm \diagclosure{L}}$. Note moreover that $\Sigma$
    doesn't contain rays in boundary cobounding a half-strip in \(\Sigma\),
    because such a half-strip would be contained in $\diagclosure{L}$ by
    construction.
    
    Let $a$ be a component of $\ell\cap\Sigma$ and $F$ the component of
    $\Sigma$ containing $a$. If $a$ is nonseparating in $F$, we can find a
    curve contained in $F$ intersecting $a$ once and we are done. If $a$ is
    separating and both components of $F\ssm a$ are not contractible, we
    can find two non-nullhomotopic curves $\alpha_1$ and $\alpha_2$, one in
    each component of $F\ssm a$, and a simple arc $b$ connecting them. The
    regular neighborhood of $\alpha_1\cup b\cup\alpha_2$ is a pair of pants
    whose third boundary component is the required curve. So we can assume
    that $a$ is separating and at least one component $C$ of $F\ssm a$ is
    contractible. But then, by the classification of contractible surfaces
    with boundary, $C\cup a$ is a closed disk with points removed from the
    boundary. Since the boundary of $\Sigma$ contains only finitely many
    components, $C\cup a$ is a closed disk with finitely many points
    removed from its boundary. As $a$ cannot be homotoped into
    $\diagclosure{L}$, one such point is not an end of $a$. But then
    $\Sigma$ contains a half-strip between two rays in its boundary, a
    contradiction.

    For the last statement, write $\diagclosure{L}$ as 
    \[K\sqcup S_1\sqcup\dots\sqcup S_k,\]
    where \(K\) is a compact surface and the $S_i$ are pairwise
    disjoint half-strips. For every $j=1,\dots,k$ there is a line $\ell_{i_j}$
    going through $S_j$. As no line ends in a puncture, there is a sequence
    of essential separating curves, pairwise not homotopic, converging to
    the end of $\ell_{i_j}$ contained in $S_j$; we can choose one such
    curve $\alpha_j$ sufficiently far out so that $\alpha_j$ intersects
    $\ell_{i_j}\cap S_j$ and is disjoint from $K$. Let
    $$\mu=\bigcup_{j=1}^k\alpha_j.$$

    If $\ell$ is a line in $\diagclosure{L}$, by properness $\ell$ contains
    the ray $\ell_{i_j}\cap S_j$ for some $j$. So it intersects the
    corresponding $\alpha_j$ and thus $\mu$.\qedhere

  \end{proof}
 
  \begin{figure}[ht]

  \begin{center}
  \begin{overpic}{contractible}
  \put(70,56){$F$}
  \put(35,31){$a$}
  \put(50,22){$C$}
  \end{overpic}
  \caption{The situation in the proof of Lemma
    \ref{lem:lines-outside-surfaces}}
  \end{center}
  \end{figure}

  \begin{remark}
    At first, one might think that condition (2) in the previous lemma is
    equivalent to $\ell$ being in $\langle L\rangle$ (up to proper
    homotopy). The example depicted in Figure \ref{fig:diagclosure} shows
    that this is not the case: condition (2) holds, but $\ell$ cannot be
    properly homotoped into $\langle L\rangle$.
  \end{remark}
  
  \begin{figure}[hb]
  \begin{center}
  \begin{overpic}[width=.5\textwidth]{diagclosure}
  \put(20,30){\color{orange} $\ell$}
  \put(75,30){\color{orange} $\ell$}
  \put(18,-5){\color{PrettyGreen} $\langle L\rangle$}
  \put(72,-5){\color{PrettyGreen} $\diagclosure{L}$}
  \end{overpic}
  \vspace{.5cm}
  \caption{$L$ is the union of the two green lines and $\ell$ is the orange
    one. Then $\ell$ is not properly homotopic into $\langle L\rangle$, but
    it is in $\diagclosure{L}$.}\label{fig:diagclosure}
  \end{center}
  \end{figure}

\section{Extra tame maps: canonical decomposition}

  \label{sec:decomposition}

  By a \emph{decomposition} of $S$ we mean a collection $\{X_i\}$ of pairwise
  disjoint subsurfaces of $S$ (each of which $X_i$ may be disconnected), such that  we can find representatives $\tilde{X_i}$ so that $S = \bigcup \tilde{X_i^*}$ and the
  only overlap between $\tilde{X_i}$ and $\tilde{X_j}$, $i \ne j$, are common boundary
  components. When $f$ is a homeomorphism of $S$, a decomposition $\{X_i\}$
  of $S$ is called $f$--invariant if $f(X_i)$ is isotopic to $X_i$ for all
  $i$. For a subsurface $X \subset S$, we say that $f$ \emph{returns to}
  $X$ if there is a power $k>0$ such that $f^k(X)$ is isotopic to $X$. We
  call any such $k>0$ a \emph{returning time} of $f$ to $X$. Note that all
  returning times of $f$ to $X$ are multiples of the smallest returning
  time.   
  
  Our main proposition in this section is the following. 
  
  \begin{prop} \label{prop:decomposition}

    Suppose $f$ is extra tame. Then there is an $f$--invariant
    decomposition of $S$ into three subsurfaces, $\Sper$, $\Sinf$, and
    $S_0$, with the following properties.
    \begin{itemize}
      \item A curve in $S$ is periodic if and only if it can be
        homotoped into $\Sper$. 
      \item A curve in $S$ is wandering if and only if it can be
        homotoped into $\Sinf$ 
      \item $S_0$ contains no essential, non-peripheral curves.  
    \end{itemize}
    We will call $\Sper$, $\Sinf$, and $S_0$ the \emph{canonical
    decomposition} of $f$ on $S$.
  \end{prop}

The decomposition is defined as follows. Denote by $\Cper$ the collection of $f$--periodic curves, and $\Cinf$ the
  collection of $f$--wandering curves.
 Let $\PSper$ be the surface spanned by $\Cper$ (whose existence is proven in Lemma \ref{lem:SperSubs}), $\Sinf$ the subsurface spanned by $\Cinf$ (whose existence is proven in Lemma \ref{lem:SinftySubs})
 
  Relative to a hyperbolic metric on $S$, we will pick an almost geodesic
 representative for the components of $\PSper$ and $\Sinf$ as follows. For
 each non-annular component $X$, let $X^*$ be the almost geodesic
 representative of $X$. Recall this means that we first take the geodesic
 representative $X^\circ$ of the interior of $X$ and then remove a small
 regular neighborhood of a boundary component of $X^\circ$ which is homotopic to another boundary component of $X^\circ$. By
 an abuse of notation, we will continue to use $\PSper$ and $\Sinf$ as the
 union of the representatives of their components. We define  $$\PS0:=\overline{S\ssm(\PSper\cup\Sinf)},$$
 whose isotopy class is independent of the choice of the hyperbolic metric on
 $S$.
 
 Then:
  \begin{itemize}
	\item $\Sper$ is the components of the union of $\PSper$ (the subsurface spanned by periodic curves) and the negative Euler characteristic components of $\PS0$,
	\item $\Sinf$ is --- as already said --- the subsurface spanned by wandering curves,
	\item $S_0$ is the union of all components in $\PS0$ which have non-negative Euler characteristic.
\end{itemize}
  
  As just seen, a fundamental step in the proof of the proposition is to show that, for an extra tame map, $\Cper$ and $\Cinf$
  span subsurfaces $\PSper$ and $\Sinf$. We then need to show that $\PS0$ has no interesting topology. All three proofs are done via
  contradiction. Namely, we will show that if the required statement s
  false, then we can use Lemma \ref{lem:not-good} to find an essential curve $\alpha$
  with infinite limit set $\L(\alpha)$. The proofs in the cases of the $\PSper$ and $\Sinf$ are independent of each other. The proof for $\PS0$ is more
  technical, and it relies on the fact that we already know it is a
  subsurface, being the complement of two subsurfaces, and further no curve
  in $\PS0$ can be periodic or wandering, and thus must limit onto a
  non-empty collection of lines.  


Let us then prove that $\Sinf$ and $\PSper$ are subsurfaces.

  \begin{lemma}\label{lem:SinftySubs}
    If \(f\) is extra tame, then $\Cinf$ spans a subsurface $\Sinf$. Every
    curve in $\Sinf$ is wandering, and if $\alpha$ is any curve with
    $\L(\alpha) \ne \emptyset$, then $\L(\alpha)$ does not intersect
    $S_\infty$ essentially.
  \end{lemma}

  \begin{proof}

    Suppose there is no subsurface spanned by $\Sinf$. Then we can find
    curves $\alpha$, $\beta_i$ and $\gamma_j$ and arcs $\tau_j$ as in
    Lemma \ref{lem:not-good} and Remark \ref{rmk:tau^n}. Since \(\gamma_j\) is not
    wandering, \(\L(\gamma_j)\) is nonempty and by construction it is
    contained in \(\L(\alpha)\), which is finite. So there is some
    \(\ell\in\L(\alpha)\) which is contained in infinitely many
    \(\L(\gamma_j)\).

    Let \(\delta\) be a curve intersecting \(\ell\); then \(\L(\delta)\)
    intersects infinitely many \(\gamma_j\). By finiteness of $\L(\delta)$,
    there is \(\ell'\in\L(\delta)\) which intersects infinitely many
    \(\gamma_j\). If \(i(\ell',\beta_{i_k})\neq 0\), then
    $i(\delta,\L(\beta_{i_k}))\neq 0$, which is a contradiction (since the
    $\beta_{i_k}$ are wandering). So \(\ell'\) needs to intersect
    infinitely many \(\tau_j\), and thus it intersects \(\alpha\)
    infinitely many times, a contradiction.

    Every curve $\beta$ in $\Sinf$ is a concatenation of wandering curves
    and hence is wandering by Lemma \ref{lem:periodic&wandering}. If
    $\L(\alpha)$ intersects the interior of $\Sinf$, since $\Sinf$ is
    spanned by wandering curves, there would be some $\beta\subset\Sinf$
    intersecting $\L(\alpha)$, but this contradicts Lemma
    \ref{lem:not-wandering}.\qedhere

  \end{proof}

  \begin{lemma}\label{lem:SperSubs}
    If \(f\) is extra tame, then $\Cper$ spans a subsurface $\PSper$, and every curve in $\PSper$ is periodic.
  \end{lemma}

  \begin{proof}

    Suppose there is no subsurface spanned by periodic curves. Then we can
    find curves $\alpha$, $\beta_i$ and $\gamma_j$ and arcs $\tau_j$ as in
    Lemma \ref{lem:not-good} and Remark \ref{rmk:tau^n}. We know from the lemma that
    \(\alpha\) is not periodic; moreover, since it intersects periodic
    curves, it is not wandering either. Thus \(\L(\alpha)\) is non-empty
    and contains only lines.

    As the pair of pants spanned by \(\beta_{i_j}\cup
    \tau_j\cup\beta_{i_{j+1}}\) is not \(f\)--periodic, the length of
    \(\tau_j^n\) is not bounded (because there are finitely many isotopy
    classes of arcs of bounded length between the geodesic representative
    of \(f^n(\beta_{i_j})\) and the geodesic representative of
    \(f^n(\beta_{i_{j+1}})\), for \(n\geq 0\)). Moreover, the smallest
    angle of intersection of \(\tau_j^n\) with \(f^n(\beta_{i_j})\) and
    \(f^n(\beta_{i_{j+1}})\) is bounded away from zero, otherwise there is
    a sequence of iterates of \(\alpha\) wrapping around some iterate of
    \(\beta_{i_j}\) or \(\beta_{i_{j+1}}\) more and more times,
    contradicting tameness. So the \(\tau_j^n\) leave every compact, but
    also intersect periodic curves. Hence the sequence \(\{\tau_j^n\st
    n\geq 0\}\) accumulates somewhere, and the accumulation set contains
    rays  starting at iterates of \(\beta_{i_j}\) or \(\beta_{i_{j+1}}\).
    For every \(j\), fix one such ray \(r_j\). By construction, \(\tau_j\)
    intersects only \(\beta_{i_k}\) for \(k=j,j+1\), so the \(r_j\) doesn't
    intersect any iterate of any \(\beta_{i_k}\) for \(k\neq j,j+1\). In
    particular the rays \(r_{2j}\) are pairwise distinct and not nested and
    each of them is contained in some line of \(\L(\alpha)\). As
    \(\L(\alpha)\) is finite, this gives us a contradiction. 
    
    Every curve in $\Sper$ is a concatenation of periodic curves and hence
    periodic  by Lemma \ref{lem:periodic&wandering}. \qedhere

  \end{proof}

  \begin{lemma} \label{lem:S0}
  
    If $f$ is extra tame, then $\PSper$ and $\Sinf$ are disjoint
    subsurfaces. 

  \end{lemma}

  \begin{proof}

    Since they are subsurfaces spanned by curves, if they do not have
    disjoint representatives, then there is curve $\alpha$ in $\PSper$ and a
    curve $\beta$ in \(S_\infty\) such that \[i(\alpha,\beta) > 0,\] which
    is a contradiction.\qedhere

  \end{proof}

 
  \begin{lemma}\label{prop:no-ess-nonper}

    If \(f\) is extra tame, $\PS0$ has no essential,
    non-peripheral curves. 
   
 
  \end{lemma}

  \begin{proof}
  
    We prove this by contradiction. Suppose there is a curve $\delta$ in
    $\PS0$ which is essential and non-peripheral. Then $\delta$ cannot be
    wandering or periodic, so $L:=\L(\delta)$ is a nonempty collection of
    lines. Let $L^\pm:=\L^\pm(\delta).$ We define $$\C=\{\beta\subset
    S_0\st \L^\pm(\beta)\subset \diagclosure{L^\pm}\}.$$ Note that $\C$ is
    closed under taking finite concatenation of curves in $\C$: indeed, if
    a curve $\beta$ is a concatenation $b_1\ast\dots\ast b_k$ of pieces
    $b_i\subset\beta_j\in\C$, as the $\beta_j$ are in $S_0$, so is $\beta$.
    Moreover, suppose a line $\ell\in\L^\pm(\beta)$ were not in
    $\diagclosure{L^\pm}$. Then by Lemma \ref{lem:lines-outside-surfaces}
    there is a curve $\eta$ intersecting $\ell$ and disjoint from all lines
    in $L^\pm$. Thus there are infinitely many iterates of $\beta$
    intersecting $\eta$ and hence some $j$ so that infinitely many iterates
    of $\beta_j$ intersect $\eta$. Thus $\L^\pm(\beta_j)$ contains a line
    intersecting $\eta$, but this contradicts Lemma
    \ref{lem:lines-outside-surfaces} as $\L^\pm(\beta_j)\subset
    \diagclosure{L^\pm}$.
   
    \begin{claim}
    There is a subsurface $S_{L}$ spanned by $\C$.
    \end{claim}
   
    \begin{proof}

      By contradiction, suppose there is no subsurface spanned by $S_L$. Then
      we can find curves $\alpha$, $\beta_i$ and $\gamma_j$ and arcs $\tau_j$
      as in Lemma \ref{lem:not-good} and Remark \ref{rmk:tau^n}. Note that we can choose
      $\alpha\subset \PS0$, because if $S_L$ were not tame, it wouldn't be
      homotopic to a subsurface of $\PS0$ either, since $\PS0$ is a subsurface.

      Let $\L^\pm(\tau_j)$ be $\L^\pm(\{\tau^n_j\st n\in\Z\})$. Note that for
      every $j$ at least one between $\L^+(\tau_j)$ and $\L^-(\tau_j)$ is
      nonempty, otherwise
      $\L^\pm(\gamma_j)\subset\diagclosure{\L^\pm(\beta_j)\cup\L^\pm(\beta_{j+1})}\subset
      \diagclosure{L^\pm}$, so $\gamma_j\subset S_{L}$, which is
      impossible.
      So $\L^\pm(\tau_j)$ is a nonempty collection of arcs, rays and lines
      contained in $\L^\pm(\alpha)\ssm\diagclosure{L^\pm}$, where the
      segments go from $\partial\diagclosure{L^\pm}$ to
      $\partial\diagclosure{L^\pm}$ and rays start from
      $\partial\diagclosure{L^\pm}$. Note moreover that $\ell\ssm
      \diagclosure{L^\pm}$ has finitely many components for every
      $\ell\in\L^\pm(\alpha)$. As $\L^\pm(\alpha)$ and $\diagclosure{L^\pm}$
      are $f$--invariant, up to passing to a power we can assume that each
      component of $\ell\ssm \diagclosure{L^\pm}$ is $f$--invariant, for
      every $\ell\in \L^\pm(\alpha)$. Let $\{A_1,\dots, A_k\}$ be the
      collection of all these components.

      Without loss of generality, assume that $L^+\neq \emptyset$; up to
      passing to a subsequence, we can assume that $\L^+(\tau_j)\neq
      \emptyset$ for every $j$.

      Suppose that an arc, ray or line $A$ belongs to $\L^+(\tau_j)$. Then
      there is some sequence $l_i\to\infty$ so that $\tau_j^{l_i}$ converges
      and the limit contains $A$. But as all components of
      $\ell\ssm\diagclosure{L^+}$, for all $\ell$, are $f$--invariant,
      $\tau_j^n$ has the same limit containing $A$ as $n\to\infty$.

      Fix a compact subsurface intersecting all components of $\ell\ssm
      \diagclosure{L^+}$, for every $\ell$. By tameness, for every component
      $A_i$ there is a number $N(A_i)$ so that for every $n$,
      $f^n(\alpha)\cap K$ has at most $n(A_i)$ strands parallel to $A_i\cap
      K$.

      Let $J=\sum_{i=1}^k N(A_i)+1$ and consider $\tau_1,\dots,\tau_J$. By
      our assumptions, for every $i$ there are at most $N(A_i)$ among the
      $\tau_j$ with $A_i\subset\L^+(\tau_j)$. But then there is at least one
      $\tau_j$ which cannot have any $A_i$ in the limit, a
      contradiction.\qedhere

    \end{proof}

    Up to replacing $f$ by its inverse, we can assume that \(L^+\) is
    nonempty. By Lemma \ref{lem:lines-outside-surfaces}, we can find a
    finite collection of curves $\{\delta_1^+,\ldots,\delta_k^+\}$
    intersecting all lines in \(\diagclosure{L^+}\). Then by Lemma
    \ref{lem:duality}, for every curve \(\gamma\) in \(S_L\), either
    \begin{enumerate}[(a)]
    \item \(\L^+(\gamma)=\emptyset\), or
    \item \(i(\gamma, \L^-(\delta_i))\neq 0\) for some $i$.
    \end{enumerate}

    If all curves in \(S_L\) satisfy (b), then every connected component of
    \(S_L\) is filled by a finite collection of lines, so it is of finite
    type. If one connected component is \(f\)--periodic, it contains a
    periodic curve; otherwise all connected components go to infinity (they
    cannot accumulate, because \(S_{L}\) is a subsurface), so \(S_{L}\)
    contains a wandering curve. In either situation, we have a
    contradiction.

    So we can assume that there is a curve \(\gamma\) in \(S_{L}\) satisfying
    (a), which is then contained in \(S_{L}\ssm \L^-(\delta_i)\). Note that
    \(S_{L}\ssm \L^-(\delta_i)\) is an \(f\)--invariant subsurface.

    Let $\{ \delta_j^-\}$ a finite collection of curves intersecting all
    lines in \(\diagclosure{L^-}\). If \(\gamma \subset S_{L}\ssm
    \L^-(\delta_2)\), \(\L^+(\gamma)=\emptyset\). As \(\gamma\) isn't
    wandering, \(\L^-(\gamma)\neq \emptyset\). Since \(\L^-(\gamma)\subset
    \diagclosure{L^-}\), by Lemma \ref{lem:duality}
    \(i(\gamma,\L^+(\delta_3))\neq 0\). Hence every connected component of
    \(S_{L}\ssm \L^-(\delta_2)\) is filled by finitely many lines that
    intersect finitely many times and hence is of finite type. As before we
    deduce that there is either a periodic curve or a curve going to infinity
    in \(S_{L}\ssm \L^-(\delta_2)\), a contradiction. \qedhere

  \end{proof}


A consequence of the lemma is the following.
\begin{cor}
	Any negative Euler characteristic component of $\PS0$ is a pair of pants.
\end{cor}

\begin{proof}
If the Euler characteristic were smaller than $-1$, there would be essential non-peripheral curves. So the Euler characteristic is $-1$ . If the component is not a pair of pants, it is a pair of pants with some points removed from its boundary, and therefore contains an essential non-peripheral curve, a contradiction.
\end{proof}


  \subsection{Examples} 

  \label{sec:examples}
 
  We end this section with some examples of tame but not extra tame maps
  for which the conclusions of Proposition \ref{prop:decomposition} are
  false.

  \begin{lemma} \label{ex:Spernotsurf}
  
    There exists a tame map $f$ such that $\L(\alpha)$ is locally finite
    for every curve $\alpha$, but the collection of periodic curves does
    not span a subsurface.

  \end{lemma}

  \begin{proof}

    Consider the cylinder $\Sigma:=S^1\times \R$ and fix a Cantor set
    $K\subset S^1$. Let $S$ be the surface obtained as
    \[S:=\Sigma\ssm\bigcup_{n=-\infty}^\infty K\times\{t_n\},\]
    where $t_{\pm\infty}=\pm 1$, $t_0=0$, $t_n=-1+\frac{1}{2^{-n}}$ for
    $n<0$ and $t_n=1-\frac{1}{2^{n}}$ for $n>0$ (see Figure
    \ref{fig:cylinder}).

    Fix a monotone non-increasing homeomorphism $s:\R\to\R$ such that
    \begin{align*}
    s(t_{\pm\infty})&=t_{\pm\infty}\\
    s(t_k)&=t_{k+1} \hspace{.5cm} \forall k\in\Z.
    \end{align*}
    Then the map $\text{id}\times s:\Sigma\to\Sigma$ induces a tame
    homeomorphism $f$ of $S$, which has no wandering curves. The periodic
    curves are also fixed curves, which are those that can be homotoped to
    be vertical between height $-1$ and height $1$. In particular $\Cper$
    does not span a subsurface. \qedhere

  \end{proof}

  \begin{figure}[ht]
  \begin{center}
  \begin{overpic}[scale=1.2]{cylinder}
  \put(46,88){$1$}
  \put(46,8){$-1$}
  \end{overpic}
  \caption{A sketch of the surface in Lemma \ref{ex:Spernotsurf}, with
    a periodic curve (in green) and the Cantor sets in light
    blue}\label{fig:cylinder}
  \end{center}
  \end{figure}


  \begin{lemma} \label{ex:Sinfnotsurf}

    There exists a tame map $f$ such that $\L(\alpha)$ is locally finite
    for every curve $\alpha$, but $\Cinf$ does not span a subsurface.
    Moreover, we can construct such an example so that a single component of $\Sinf$ is not a subsurface.
   
  \end{lemma}
 
  \begin{proof}

    In the plane, for any $k\in\N$, let $\ell_k$ be the boundary of
    \[ [k+1,\infty) \times
    \left[\frac{1}{k+1}+\varepsilon_k,\frac{1}{k}-\varepsilon_k\right], \]
    where $\varepsilon_k=\frac{1}{10k^2}$ (see Figure \ref{fig:lightblue}
    for a schematic picture). Let $S$ be the plane punctured at the points
    $\left(\frac{1}{n},m\right)$, for $n\in\N$ and $m\in\Z$, at the points
    $(0,m)$, for $m\in\Z$, and for every line $\ell_k$, $k\in\N$, at a
    sequence of points on the line not accumulating anywhere.

   \begin{figure}[htp]
   \begin{center}
   \begin{overpic}[width=.6\textwidth]{lightblue}
   \put(0,5){$(0,0)$}
   \put(24,5){$(1,0)$}
   \put(43,5){$(2,0)$}
   \put(61,5){$(3,0)$}
   \put(80,5){$(4,0)$}
   \put(0,27){$(0,1)$}
   \put(-4,18){$(0,1/2)$}
   \put(-4,14){$(0,1/3)$}
   \put(100,26){$\ell_1$}
   \put(100,16){$\ell_2$}
   \put(100,13){$\ell_3$}
   \end{overpic}
   \caption{The lines $\ell_k$ in the construction of Lemma
     \ref{ex:Sinfnotsurf}}\label{fig:lightblue}
   \end{center}
   \end{figure}
   The homeomorphism $f$ that we consider is given by:
   \begin{itemize}
   \item the map induced by $(x,y)\mapsto (x+1,y)$ on a small regular
     neighborhood of each horizontal line $y=\frac{1}{n}$ ($n\in \N$) and
       $y=0$,
   \item a puncture shift supported on a small regular neighborhood of
     $\ell_k$, for every $k$ tapered to the identity on the rest of the
       surface.
   \end{itemize}

    One can show that $f$ is tame. Moreover, $F(\Cinf)$ is the disjoint
    union of one strip per line $y=\frac{1}{n}$, $n\in\N$, and of the
    strips $(k,\infty) \times \left(
    \frac{1}{k+1}+\frac{\varepsilon_k}{2},\frac{1}{k}-\frac{\varepsilon_k}{2}\right)$.
    The closure of each component is a subsurface, but the horizontal
    strips accumulate onto $y=0$, so $\Cinf$ doesn't span a subsurface.
    Note moreover that there are no periodic curves.

    We can also modify the surface by adding handles connecting consecutive
    horizontal strips in a translation invariant way (see Figure
    \ref{fig:connectedSinf} for a schematic picture). The map induced by
    $f$ on this new surface is also tame and has no periodic curves. Now
    though $F(\Cinf)$ is given by one component per line $\ell_k$ and a
    single nonplanar component. The nonplanar component is not homotopic to
    a subsurface. \qedhere

  \end{proof}

  \begin{figure}[ht]
  \begin{center}
  \includegraphics[width=.5\textwidth]{connectedSinf}
  \caption{The modified surface in Lemma
    \ref{ex:Sinfnotsurf}}\label{fig:connectedSinf}
  \end{center}
  \end{figure}

\section{Characterization of translations}

  \label{sec:translation}

  In this section, we give a characterization of a translation in terms of
  the dynamics of its action on curves. The main application will be to an
  extra tame map $f$ and its action on the invariant subsurface $\Sinf$ in
  its canonical decomposition. Therefore, in this section, we need to work
  with surfaces with boundary. Everything in this section is independent of
  the previous sections.


  \begin{lemma}\label{lem:e_pm}

    Let $X$ be a surface possibly with boundary and $f$ a homeomorphisms of
    $X$ such that every curve in $X$ is $f$--wandering. Then there are ends
    $e_\pm$ of $X$, possibly $e_+ = e_-$,  such that for every curve
    $\alpha$ in $X$ $f^{\pm n}(\alpha)^*$ converges to $e_\pm$ as $n\to
    \pm\infty$.
  \end{lemma}

  We will call $e_+$ and $e_-$ respectively the $f$--\emph{attracting} and
  $f$--\emph{repelling} ends of $X$.

  \begin{proof}
    Let $\alpha$ be a curve in $X$. We first show that $f^n(\alpha)^*$
    converges to an end. Let
    \[\alpha^*=\alpha_0,\alpha_1,\dots\alpha_k=f(\alpha)^*\] be a chain of
    geodesic curves in \(X\) connecting \(\alpha^*\) and \(f(\alpha)^*\).
    Given any compact set $K$ there is $n_0$ so that if $n >n_0$, the
    $f^n(\alpha_i)^*$ are disjoint from $K$. In particular, $f^n(\alpha)^*$
    and $f^{n+1}(\alpha)^*$ are in the same complementary component of $K$,
    for $n>n_0$, so $f^n(\alpha)^*$ goes to a single end $e_+$ as
    $n\to\infty$. The same argument shows that $f^n(\alpha)^*$ converges to
    an end $e_-$ as $n\to -\infty$.

    If $\beta$ is another curve in $X$, by looking at a chain of curves in
    $X$ connecting $\alpha^*$ to $\beta^*$ and repeating the same argument as
    above, we deduce that $f^n(\beta)^\ast\to e_\pm$ as
    $n\to\pm\infty$.\qedhere

  \end{proof}
 
  Note that a surface $X$ does not necessarily coincide with the subsurface spanned by its curves. This is for instance not the case for a surface which doesn't contain any curve (such as a closed disk with points removed from the boundary) or for surfaces with rays in the boundary cobounding a half-strip (for instance, a one-holed torus with points removed from the boundary). We say that $X$ is \emph{filled by its curves} if it does coincide with the subsurface spanned by its curves. 
  \begin{lemma}\label{lem:delta}

    Let $X$ be a surface possibly with boundary and $f$ a homeomorphism on
    $X$ such that every curve in $X$ is $f$--wandering. Suppose $X$ is
    filled by its curves and the $f$--attracting and repelling ends $e_+$
    and $e_-$ are distinct. Then for any a hyperbolic metric on $X$, there
    is a curve or a finite collection of arcs $\delta$ separating $e_+$ and
    $e_-$, and $f^n(\delta)^*$ is disjoint from $\delta^*$ for all
    sufficiently large $n$. 

  \end{lemma}

  \begin{proof}
   
    Let $D(X)$ be the double of $X$ along its boundary (which coincides
    with $X$ if $X$ has no boundary components), with the double hyperbolic
    structure. The map $f$ induces a map $\hat{f}$ on the double. Let $i :
    X \hookrightarrow D(X)$ be the inclusion map and $i_*$ the induced map
    from $\Ends(X)$ to $\Ends(D(X))$. Note that $i_*$ is injective, because
    if $e_1,e_2\in\Ends(X)$ are distinct ends, there is a compact subset
    $K$ separating them, and therefore $i_*(e_1)$ and $i_*(e_2)$ are
    separated by the double of $K$ in $D(X)$.

    Since $D(X)$ has no boundary, there is a curve $\hat{\delta}\subset
    D(X)$ separating $i_*(e_+)$ and $i_*(e_-)$. Assume $\hat{\delta}$ is in
    minimal position with respect to $i(\partial X)$ and let
    $\delta:=i^{-1}(\hat{\delta})$. Since the boundary of $X$ are realized
    as geodesic lines in $D(X)$, $\hat{\delta}$ intersects each boundary
    component of $i(X)$ at most finitely many times, and the boundaries of
    $i(X)$ in $D(X)$ cannot accumulate, so $\delta$ has only finitely many
    components. 


    \begin{claim}
    $\delta$ separates $e_+$ and $e_-$.
    \end{claim}
    \begin{proof}
      If not, we can find a line $\ell$ in $X$ from $e_+$ to $e_-$ disjoint
      from $\delta$, and hence a line $i(\ell)$ from $i_*(e_+)$ to
      $i_*(e_-)$ disjoint from $\hat{\delta}$.\qedhere
    \end{proof}

    Let $H^+$ be the component of $X \ssm \delta^*$ containing $e_+$. We
    claim there exists $n \ge 1$ such that $f^n(\delta)^* \subset H^+$, and
    hence in particular that $\delta^*$ and $f^n(\delta)^*$ are disjoint.
    This is immediate if $\delta$ is a curve since it is wandering and
    converging to $e_+$. If $\delta$ is an arc, then let $L_1, L_2$ be the
    boundary components of $X$ containing the endpoints of $\delta$. Since
    $X$ is filled by its curves, we can find a finite-type subsurface $F
    \subset X$ such that $\delta^* \ssm F^*$ is contained in $N_1 \cup
    N_2$, where $N_i$ is a regular neighborhood about $L_i$ and $F^*$ is
    the geodesic representative of $F$. Since $F$ is filled by curves in
    $X$, for all sufficiently large $n$, $f^n(F)^* \subset H^+$. If
    $f^n(\delta)^*$ meets $\delta$, then $f^n(\delta)^*$ would have to
    intersect $\delta$ essentially in $N_1 \cup N_2$, but this is not
    possible. Thus the ends of $f^n(\delta)^*$ also have to lie in $H^+$.
    This shows $f^n(\delta)^* \subset H^+$ for all sufficiently large $n$.
    If $\delta$ is a finite union of arcs, we can repeat the same argument
    for every component.\qedhere

  \end{proof}

  \begin{prop}\label{prop:wandering-arcs}
    Let $X$ be a surface possibly with boundary and $f$ a homeomorphism on
    $X$ such that every curve in $X$ is $f$--wandering and $X$ filled by
    its curves. Then every arc in $X$ with endpoints on $\partial X$ is $f$--wandering.
 
  \end{prop}

  \begin{proof}
   
    Fix a complete hyperbolic structure on $X$ with totally geodesic
    boundary, so that $X$ has injectivity radius greater than some positive
    constant $\epsilon$. Let $\delta = \delta(\epsilon)$ be a constant such
    that if $\beta$ is a piecewise geodesic loop with each segment at least
    $\epsilon$ long and angles between consecutive segments at least
    $\pi/2$, then $\beta$ is $\delta$ close to its geodesic representative
    $\beta^*$. 

    Note that there is nothing to show if $\ell$ is $f$--wandering. Thus we
    can assume that $f^k(\ell) = \ell$ for some $k \ge 1$.
  
    For any arc $\gamma$, let $\gamma^*$ be its geodesic representative
    which is orthogonal to the boundary of $X$. Our goal is to show
    $f^n(\gamma)^*$ leaves every compact set of $X$. It is enough to show
    $f^{kn}(\gamma)^* \to e_+$ for some $k \ge 1$, as this implies that
    $f^{kn+m}(\gamma)^* \to e_+$ for every \(m\). Thus up to replacing
    $f$ by a power, we may assume $f(\ell) = \ell$ and $\gamma$ is an arc
    with both endpoints on $\ell$. 

    Set $\gamma_n = f^n(\gamma)$. As for curves, we denote by
    $\L^\pm(\gamma)$ the accumulation set of $\gamma_n^*$ as $n \to \pm
    \infty$. Applying the same argument as in Lemma \ref{lem:duality} to
    arcs, we have that \[ \L(\gamma) \cap \beta \ne \emptyset
    \Longleftrightarrow \L(\beta) \cap \gamma \ne \emptyset,\] where
    $\beta$ is any curve or arc in $S$. Since curves in $X$ are wandering,
    $\L(\gamma)$ cannot intersect the interior of $X$ essentially.
    Parametrize $\ell$ by $\R$ isometrically, and let $a_n, b_n \in \R$ be
    the endpoints of $\gamma_n^*$ with $a_n < b_n$. 

    \begin{claim}
       
      Both sequences $a_n$ and $b_n$ converge simultaneously to $\infty$ or
      to $-\infty$. 

    \end{claim}

    To see the claim, let $\alpha_n$ be the curve in $X$ obtained by
    concatenating $\gamma_n^*$ with the segment $[a_n,b_n] \subset \ell$.
    Since $f^n(\gamma)$ and $f^n(\gamma^*)$ have the same geodesic
    representative $\gamma_n^*$, $f^n(\alpha_0)^* = \alpha_n^*$, so
    $\alpha_n^*$ converges to an end of $X$. If there is a subsequence
    $a_{n_i}$ that converges to $a \in \ell$, then $\gamma_{n_i}^*$ would
    converge to a ray in $X$ emanating orthogonally from $a$, but this
    impossible since such ray meets the interior of $X$ essentially, and thus
    a curve in \(X\). Similarly for $b_n$. On the other hand, if some
    subsequence $a_{n_i} \to \pm\infty$ but $b_{n_i} \to \mp \infty$, then
    for sufficiently large $i$, $\alpha_{n_i}$ is a quasi-geodesic with
    uniform constants, so it stays close to its geodesic representative
    $\alpha_{n_i}^*$. But in this case, $\alpha_{n_i}^*$ cannot converge to
    an end of $X$. This finishes the proof of the claim.
    
    Henceforth, without loss of generality assume that $a_n, b_n \to \infty$.
    Choose $i \ge 1$ so that $a_0 < b_0 < a_i < b_i$, and note that then $a_n
    < b_n < a_{n+i} < b_{n+i}$ for all $n \ge 0$. If $\gamma$ is not
    wandering, then there is a compact convex subsurface $K$ that intersects
    infinitely many $\gamma_n^*$, for which we can assume that
    \(a_n,b_n\notin K\). 

    \begin{claim}\label{claim:quadrilateral}

      There exists a boundary component $\tau$ of $K$ and $n$ such that for
      all $j \ge 0$, there exists a quadrilateral $Q_j$ bounded above by an
      arc of $\tau$, laterally by an arc of $\gamma_n^*$ and an arc of
      $\gamma_{n+ij}^*$, and below by $[a_n,b_{n+ij}]$. 

    \end{claim}
    
    \begin{figure}[ht]
    \begin{center}
    \begin{overpic}{quadrilateral}
    \put(0,-7){$\ell$}
    \put(12,2){$K$}
    \put(30,36){$\tau$}
    \put(75,-7){$a_n$}
    \put(95,-7){$b_{n+ij}$}
    \put(78,8){$Q_j$}
    \end{overpic}
    \caption{The situation of Claim \ref{claim:quadrilateral}}
    \end{center}
    \end{figure}

    We form another family of curves by surgery as follows. Let $\beta_n$ be
    the closed (not necessarily simple) curve obtained by joining
    $\gamma_n^* \cup \gamma_{n+i}^* \cup [b_n,a_{n+i}] \cup [a_n,b_{n+i}]$.
    Again, because $f^n(\gamma)$ and $f^n(\gamma^*)$ have the same geodesic
    representative, $f^n(\beta_0)^* = \beta_n^*$.

    \begin{figure}[hb]
    \begin{center}
    \begin{overpic}{beta_n}
    \put(3,2){$\ell$}
    \put(20,37){$\gamma_n^*$}
    \put(88,28){$\gamma_{n+ij}^*$}
    \end{overpic}
    \caption{The curve \(\beta_n\), in green}
    \end{center}
    \end{figure}  
    
    Let $N_{2\delta}(K)$ be the $2\delta$ neighborhood of $K$. Since
    $\alpha_0$ and $\beta_0$ are wandering, there exists $N$ such that
    $\alpha_n^*$ and $\beta_n^*$ are disjoint from $N_{2\delta}(K)$ for all
    $n \ge N$. Fix some $n \ge N$ so that $\gamma_n^*$ intersects $K$ and
    $a_n,b_n\notin K$. Let $\tau \subset \partial K$ be the first boundary
    component that intersects $\gamma_n^*$ as we start from $a_n$. We will
    prove the claim by induction on $j$. 
    
    First note that since $n+ij \ge N$, $\alpha_{n+ij}^*$ is disjoint from
    $K$, so all intersections of $\alpha_{n+ij}$ with $K$, if they exist, are
    inessential. For $j \ge 0$, if $\gamma_{n+ij}^*$ meets $\tau$, let $A_j$
    be the subarc of $\gamma_{n+ij}$ that starts from $a_{n+ij}$ until its
    first meeting point $p_j \in \tau$, and $B_j$ the subarc starting from
    $b_{n+ij}$ until $q_j \in \tau$. Since $p_j$ and $q_j$ are inessential
    intersections, there is a subarc $[p_j,q_j] \subset \tau$ such that the
    geodesic loop $A_j \cup [p_j,q_j] \cup B_j \cup [a_{n+ij},b_{n+ij}]$
    bounds a quadrilateral $R_j$. 
    
    When $j=0$, $\tau$ was chosen so it intersects $\gamma_n^*$, so the
    existence of $Q_0=R_0$ is the base case. Note that $q_0$ is the first
    place that $\gamma_n^*$ intersects $K$ starting from $b_n$. For $j \ge
    0$, we will assume that $Q_j$ has top $[p_0,q_j] \subset \tau$ and sides
    $A_0$ and $B_j$. Moreover $q_j$ is the first place that $\gamma_{n+ij}^*$
    intersects $K$ starting from $b_{n+ij}$. 
    
    Consider the pieceswise geodesic curve $\beta_{n+ij}$. If $a_{n+i(j+1)} -
    b_{n+ij} \ge \epsilon$, then $\beta_{n+ij}$ is a quasi-geodesic that
    stays $\delta$ close to $\beta_{n+ij}^*$. Since $n+ij \ge N$,
    $\beta_{n+ij}^*$ misses $N_{2\delta}(K)$, so the $\delta$--neighborhood
    of $\beta_{n+ij}^*$ misses $K$. But this is a contraction since
    $\beta_{n+ij}$ intersects $\tau$ (as $\gamma_{n+ij}^*$ does) and hence
    $K$. So $a_{n+i(j+1)} - b_{n+ij} < \epsilon$.
    
    Note that if $\gamma_n^*$ and $\gamma_m^*$ are at most
    \(\varepsilon\)--apart, they are homotopic, which is impossible since the
    $a_n$ go to infinity. So we can start from $b_{n+ij}$ and $a_{n+i(j+1)}$
    and travel along $\gamma_{n+ij}^*$ and $\gamma_{n+i(j+1)}^*$ respectively
    until the first moment they become $\epsilon$--apart, say at $x \in
    \gamma_{n+ij}^*$ and $y \in \gamma_{n+i(j+1)}^*$. Let $B_j'$ and
    $A_{j+1}'$ be the corresponding subarcs, and $[x,y]$ the segment of
    length $\epsilon$. Since the injectivity radius of $X$ is greater than
    $\epsilon$, the piecewise geodesic loop $B_j' \cup [x,y] \cup A_{j+1}'
    \cup [b_{n+ij},a_{n+i(j+1)}]$ bounds a quadrilateral $Q'$. Homotope the
    inner segment $[b_{n+ij},a_{n+i(j+1)}]$ of $\beta_{n+ij}$ along $Q'$
    until $[x,y]$. If $b_{n+i(j+1)} - a_{n+ij} < \epsilon$, we can also
    homotope the outer segment  of $\beta_{n+ij}$ along a (shorter)
    quadrilateral until they are $\epsilon$-apart. The result is a piecewise
    geodesic curve $\beta_{n+ij}'$ homotopic to $\beta_{n+ij}$, but now
    $\beta_{n+ij}'$ is a quasi-geodesic contained in the
    $\delta$--neighborhood of $\beta_{n+ij}^*$, so it cannot intersect $K$.
    In particular, the homotopy through $Q'$ passes through $K$. Since $\tau$
    is the first boundary of $K$ that meets $\gamma_{n+ij}^*$ starting from
    $b_{n+ij}$, it must intersect \(B_j'\) and hence \(A_{j+1}'\).  In other
    words, $Q'$ contains a subquadrilateral $Q$ bounded above by
    $[q_j,p_{j+1}] \subset \tau$ and laterally by $B_j \subset B_j'$ and
    $A_{j+1} \subset A_{j+1}'$. In particular, $\gamma_{n+i(j+1)}$ intersects
    $\tau$, so $R_{j+1}$ exists. The desired $Q_{j+1}$ is the extension of
    $Q_j$ by $Q$ and $R_{j+1}$. This finishes the proof of the claim.

    To finish the proof of the proposition, note that $Q_j$ has right angles
    at $a_n$ and $b_{n+ij}$. By basic hyperbolic geometry, we have \[
      \sinh(d_j/2) \le \frac{\sinh(s_j/2)}{\cosh(h_j)}, \] where
    $d_j=b_{n+ij} - a_n$, $s_j$ is the length of the top of $Q_j$, and $h_j$
    is the minimal length of the two sides of $Q_j$. Since $s_j$ is bounded
    by the length of $\tau$ and $h_j \ge 0$, $d_j$ is bounded but this
    contradicts the fact that $b_k \to \infty$. \qedhere 

  \end{proof}
  
  \begin{cor}\label{cor:double-is-wandering}
    Let $X$ be a surface possibly with boundary and $f$ a homeomorphism on
    $X$ such that every curve in $X$ is $f$--wandering and $X$ filled by
    its curves. Let \(D(X)\) be its double and  the map induced on
    it. Then every curve in \(D(X)\) is $\hat{f}$-wandering.
  \end{cor}

  \begin{proof}
    Let $\mathcal{A}$ be the collection of all (essential, i.e.\ not
    homotopic into the boundary) arcs with both endpoints on a boundary
    component of $X$. Then the double $D(X)$ is filled by the curves in
    the two copies of $X$ and by the curves obtained as doubles of arcs
    in $\mathcal{A}$. By Proposition \ref{prop:wandering-arcs}, all of
    these curves are wandering, so by Lemma \ref{lem:periodic&wandering}
    all curves in $D(X)$ are wandering.
  \end{proof}

  \begin{thm}\label{thm:translation}
    Let $X$ be a (possibly bordered) surface which is filled by its curves. Let $f$ be a homeomorphism of
    $X$ such that every curve in $X$ is $f$--wandering. Then there is a hyperbolic metric on $X$ and an isometric
    translation on $X$ isotopic to $f$.
  \end{thm}

  \begin{proof}

    By Lemma \ref{lem:e_pm} we know that there are ends $e_+$ and $e_-$ of
    $X$ to which all curves converge under forward/backward iteration. In
    particular, $f(e_\pm)=e_\pm$. There are now two cases.

    \uline{Case 1: $e_+\neq e_-$.} Let $\delta$ and $n$ be as in Lemma
    \ref{lem:delta}.  Set $\delta_i = f^{n_i}(\delta)^*$, each of which is
    an arc separating $e_+$ and $e_-$, and $\delta_{i+1} \subset H_i^+$,
    where $H^+_i$ is the component of $X \ssm \delta_i$ containing $e_+$.
    This shows that the dual graph is a line on which $f^n$ acts as a
    translation. We can homotope $f^n$ so it is a translation on $X$, with
    the region between $\delta_0$ and $\delta_1$ a fundamental domain for
    the action of $f^n$.

    We now argue that $f$ acts as an isometric translation of $X$. Let
    $Q=X/<f^n>$ be the quotient hyperbolic surface for the action of $f^n$.
    Since $f$ commutes with $f^n$, $f$ descends to a map on $Q$, whose
    $n$-th power is the identity map on $Q$, so $f$ is a periodic map of
    $Q$. By Proposition \ref{prop:Nielsenrealization}, there is an
    hyperbolic metric on $Q$ for which $f$ acts as an isometry of $Q$.
    Lifting this metric to $X$ also makes $f$ an isometry of $X$. Since
    $f^n$ is a translation of $X$, $f$ acts on $X$ as a covering
    transformation, so it is also a translation of $X$. 

    \uline{Case 2: $e_+=e_-$.} Let $D(X)$ be the double of $X$ along its
    boundary, if $\partial X\neq \emptyset$, otherwise $D(X)=X$. By
    Corollary \ref{cor:double-is-wandering}, every curve in $D(X)$ is
    wandering with respect to the map induced by $f$ (which we will also
    denote by $f$, abusing notation). Let $e$ the end of $D(X)$ which is
    the image of $e^\pm$ under the natural map $\Ends(X)\to\Ends(D(X))$.
    Note that all curves of $D(X)$ converge to $e$ under forward and
    backward iteration of $f$. If $D(X)\neq X$, we will do every
    operation and make every choice below respecting the symmetry given
    by the doubling.

    The goal is to construct an exhaustion of $D(X)$ by $f$-invariant
    subsurfaces, each of which falls in the previous case. To construct
    this exhaustion, fix a curve $\alpha_1$ and choose a sequence of curves
    \[\alpha_1,\alpha_2,\ldots, \alpha_k = f(\alpha_1)\] in $D(X)$ so
    that $i(\alpha_i,\alpha_{i+1}) > 0$. Let $F$ be the subsurfaces spanned
    by $ \{\alpha_i\}$, which by construction is connected, and $F \cap
    f(F)$ is also non-empty. But since the $\alpha_i$'s converge to $e$, $F
    \cap f^n(F) = \emptyset$ for all sufficiently large $n$. Set $T_1 =
    \bigcup f^i(F)$, which is $f$--invariant up to isotopy. Note that $T_1$
    is a subsurface, as it is spanned by the locally finite collection of
    iterates of $\alpha_i$'s.
    
    Consider the nerve $N$ of the cover of $T_1$ by the translates of $F$.
    We claim $N$ is $2$-ended. Indeed, $N$ is a locally-finite simplicial
    complex, since $f^n(F)\cap F=\emptyset$ for $|n|$ large. The map $f$
    acts on $N$ by simplicial automorphism. Moreover, $\<f\>\cong\Z$ acts
    properly discontinuously and cocompactly. It follows that $N$ has two
    ends --- in fact it is quasi-isometric to $\Z$. Since the iterates of
    $F$ converge to $e$ in both directions, the two ends of $N$ are two
    copies of $e$, say $t_+$ and $t_-$, with $f^i(F) \to t_\pm$ for $i\to
    \pm \infty$. We claim these two ends $t_+$ and $t_-$ are distinct in
    $T_1$.
    
    Choose a finite subcomplex of $N$ that separates the ends --- say the
    subcomplex spanned by the vertices $\{f^i(F)\mid i\in [-m,m]\}$. If
    $t_+$ and $t_-$ are the same in $T_1$, then there is a path that
    joins $f^n(F)$ with $f^{-n}(F)$ for a large $n \gg m$, and misses
    $\bigcup_{i\in [-m,m]}f^i(F)$. But then tracing the translates of $F$
    that the path intersects we get a path in $N$ that joins points close
    to the two ends and misses the finite subcomplex that separates them, a
    contradiction. It follows that $T_1$ has two ends $t_\pm$ such that
    all curves in $T_1$ go to these ends under forward/backward
    iteration. Modify $f$ by an isotopy so that $f(T_1) = T_1$.

    We now do the same for a larger collection of curves and we get a
    sequence of subsurfaces $T_1 \subset T_2 \subset \cdots$ exhausting
    $D(X)$, each of which is $2$-ended and $f$--invariant.
    
    \begin{figure}[ht]
    \begin{center}
    \begin{overpic}[width=.5\textwidth]{Tis}
    \put(102,33){\color{PrettyGreen}$T_1$}
    \put(-7,29){\color{orange}$T_2$}
    \end{overpic}
    \caption{An example of the exhaustion $T_1\subset T_2\subset\dots$,
      where the map is the horizontal shift to the right, $T_1$ is the
      surface spanned by the orbit of the green curve and $T_2$ the surface
      spanned by the orbit of the green and the orange
      curve}\label{fig:Tis}
    \end{center}
    \end{figure}
    
    As in case 1, we can find exponents $n_i$ so that $Q_i:=T_i/<f^{n_i}>$
    is a surface and $f$ induces a finite-order map on each $Q_i$; we can
    assume that for every $i$, $n_i$ divides $n_{i+1}$. Note moreover that the $Q_i$ are of finite-type, since the $T_i$ are spanned by finitely many orbits of curves. By Proposition
    \ref{prop:Nielsenrealization}, we can find a hyperbolic structure $Y_1$
    on $Q_1$ which is $f$--invariant and thus can be lifted to a
    $f$--invariant hyperbolic structure $X_1$ on $T_1$. Lift $Y_1$ to a
    hyperbolic structure $Y_1'$ on $T_1/<f^{n_2}>\subset
    T_2/<f^{n_2}>=Q_2$. By \cite{nielsen_abbildungsklassen}, we can find a
    hyperbolic structure $Y_2$ on $Q_2$ which is $f$--invariant and extends
    $Y_1'$. Lift $Y_2$ to a hyperbolic structure $X_2$ on $T_2$ which is
    $f$--invariant and note that $X_2$ extends $X_1$. Repeating this
    procedure we obtain a sequence of hyperbolic structures $X_i$ on $T_i$
    such that $X_i$ restricts to $X_{i-1}$ on $T_{i-1}$. So we get a
    hyperbolic structure on $D(X)$ with respect to which $f$ is a
    translation. By restricting the hyperbolic metric to $X$, we get the
    required structure.
    \qedhere
    
    \end{proof}
    
    \begin{remark}
      In the previous proof, in the case $e_+=e_-$, if we don't look at
      the double and we follow the same procedure, we are not sure that
      the surfaces $T_i$ exhaust $X$: their union will contain the
      interior of $X$, but not necessarily all of the boundary
      components. If this were the case, we would not get a hyperbolic
      structure on the whole of $X$ (some of its boundary components
      might be at infinity with respect to the metric on the interior of
      $X$).
    \end{remark}

\section{Structure theorem of extra tame maps}

  \label{sec:structure}
  
  Our goal in this section is to prove the main theorem of the
  introduction. We first develop some properties of the canonical
  decomposition for an extra tame map $f$. Namely, we will establish how
  the components of $\Sper$, $\Sinf$, and $S_0$ can neighbor each other. We
  will also show that, for each component $X$ of the decomposition, $f$
  returns to $X$. We then combine the results of Section
  \ref{sec:translation} and the work of Afton--Calegari--Chen-Lyman \cite{accl_Nielsen} to prove the main theorem.

  \subsection{Properties of the canonical decomposition}

  For an extra tame map, recall the almost geodesic representatives of
  $\Sper$, $\Sinf$, and $S_0$ defined in Section \ref{sec:decomposition}. An essential component of $\Sper$ and $\Sinf$
  either has infinite type or has negative Euler characteristic; in
  particular, such a component always has an essential curve. 
  
 Two components of the decomposition are \emph{adjacent} if they have boundary components which are properly isotopic (or equivalently, they have representatives sharing at least one boundary component).
  A component $X$ is \emph{self-adjacent} along $\alpha$ if $\alpha$ is properly homotopic to two distinct boundary components of $X$. Equivalently, $X$ is self-adjacent if there is a non-essential component $Y$ which shares two boundary components with $X$.
  
  One of the goals in this section is to show there
  are in fact no self-adjacent components (Lemma \ref{lem:no-self-adjacent}) and no strips (Corollary \ref{cor:no-strips}). We will also show that there is no wandering component of the decomposition and prove that the first return map to each component is either periodic or a translation, and can be realized as an isometry for some hyperbolic structure with totally geodesic boundary (Lemmas \ref{lem:per-on-S0}, \ref{lem:per-on-per} and \ref{lem:Sinf-not-wandering}). Finally we determine the topological types of the components of $S_0$ (Lemma \ref{lem:top-S0}).

We start by showing that certain boundary components cannot be wandering.

  \begin{lemma} \label{lem:jointly-wandering1}
    
    If two components of $S_\infty$ share a common boundary $\alpha$, then
    $\alpha$ is not wandering. Similarly, if a component of $\Sinf$ is
    self-adjacent along $\alpha$, then $\alpha$ is not wandering.
  
  \end{lemma}

  \begin{proof}

    First let $X$ and $Y$ be components of $S_\infty$ with a common
    boundary $\alpha$. By contradiction, assume that $\alpha$ is wandering.
    Neither $X$ nor $Y$ is inessential, so we can find a curve
    $\beta\subset \int(X) \cup \int(Y) \cup \alpha$ that intersects
    $\alpha$ essentially. As $\beta$ is not contained in $\Sinf$, it is not
    wandering, so $\L(\beta)\neq \emptyset$. Let $\gamma$ be a curve
    intersecting $\L(\beta)$. By Lemma \ref{lem:duality}, $\L(\gamma)$
    intersects $\beta$; in particular, $\L(\gamma)$ is nonempty. Let $L \in
    \L(\gamma)$ be such that $L$ intersect $\beta$. By
    \ref{lem:SinftySubs}, $L$ cannot intersect $X$ and $Y$ essentially, but
    it intersects $\beta$, so $L = \alpha$. By $f$-invariance of limits,
    the orbits of $\alpha$ is contained in $\L(\gamma)$, which contradicts
    the finiteness of $\L(\gamma)$. The proof that $X$ is a component of
    $\Sinf$ with self-adjacency along $\alpha$ is similar. \qedhere
    
  \end{proof}

  \begin{lemma} \label{lem:jointly-wandering2}
     
    Let $X$ be a component of $\PS0$. If $X$ either
    \begin{itemize}
      \item shares at least one boundary component with $S_\infty$ but
        contains a non-contractible curve not homotopic to
        that boundary component;
        or \item shares at least two boundary components with $\Sinf$,
    \end{itemize}
    then $X$ is not wandering.

  \end{lemma}

  \begin{proof}
     
    In the first case, let $\alpha$ be a non-contractible curve in $X$ and
    let $Y$ be a component of $\Sinf$ sharing another boundary component
    $\alpha'$ with $X$, such that $\alpha$ is not homotopic to $\alpha'$.
    $Y$ is essential, so we can find an essential curve $\beta$ in $X \cup
    Y$ crossing $\alpha'$, by joining a curve in $Y$ along an arc to
    $\alpha$ and taking the boundary of their regular neighborhood. In the
    second case, let $Y$ and $Z$ be components of $\Sinf$ (possibly $Y=Z$)
    sharing boundary components (called $\alpha'$ and $\alpha''$)
    respectively with $X$. We now find a curve $\beta$ in $Y \cup X \cup Z$
    intersecting $\alpha$ and $\alpha''$, by joining a curve in $Y$ and a
    curve in $Z$ by an arc that crosses $\alpha'$ and $\alpha''$. We will
    show that if $X$ is wandering, then $\beta$ is wandering, which is a
    contradiction.
 
    Indeed, suppose $\beta$ is not wandering. Then $\L(\beta)\neq
    \emptyset$ but $\L(\beta)$ cannot intersect $f^n(Y)$ or $f^n(Z)$
    essentially. On the other hand, $f^n(\beta)$ is contained in $f^n(X)
    \cup f^n(X) \cup f^n(Z)$, so some component $L$ of $\L(\beta)$ must be
    homotopic into $f^n(X)$ for some $n$. Since $\L(\beta)$ is
    $f$--invariant and $X$ is wandering, there must be infinitely many
    iterates of $L$ in $\L(\beta)$, but this contradicts the finiteness of
    $\L(\beta)$. \qedhere

  \end{proof}

We then show that all boundary components of $\Sinf$ are lines.

  \begin{lemma} \label{lem:no-compact}
    
    No component of $\Sinf$ has a compact boundary component. 
    
  \end{lemma}

  \begin{proof}

    We argue by contradiction. First suppose $X$ is non-annular and
    $\alpha$ is a compact boundary of $X$, since $X$ is spanned by curves,
    $\alpha$ is a wandering curve. So the neighboring component $Y$ of $X$
    joined along $\alpha$ cannot belong to $\Sper$ (since boundary curves
    of $\Sper$ are periodic) or $S_\infty$ (by Lemma
    \ref{lem:jointly-wandering1}). 
    
    So suppose $Y$ is in $S_0$. If there is no $n>0$ such that $f^n(Y)$ is
    isotopic to $Y$, then $Y$ is wandering (as $S_0$ is a subsurface). By
    Lemma \ref{lem:jointly-wandering2}, $Y$ cannot have a non-contractible
    curve different from $\alpha$. But $Y$ cannot be a disk or a
    punctured-disk, so it must have at least two boundary components, both
    of which are wandering. Hence the two boundary components belong to
    $\Sinf$, but this contradicts Lemma \ref{lem:jointly-wandering1}.
    
    So for some $n>0$, $f^n(Y)$ is isotopic to $Y$. As $\alpha$ is
    wandering, $Y$ contains infinitely many iterates of $\alpha$, and
    therefore an essential non-peripheral curve, contradicting Proposition
    \ref{prop:no-ess-nonper}. 
    
    If $X$ is annular, then it has a neighbor $Y$ which cannot belong in
    $\Sper$ since $X$ is wandering, and $Y$ cannot belong in $\S_0$ since
    it would have to have essential non-peripheral curves. So $Y$ belongs
    to $\Sinf$, but this contradicts the previous case. \qedhere
    
  \end{proof}


Using the previous results, we can deduce that no component of $\PS0$ is wandering, and even more, that for each of them the first return map has finite order.

  \begin{lemma}\label{lem:per-on-S0} 
    
    For every component $X$ of $\PS0$, $f$ returns to $X$ and there is a
    hyperbolic structure, with compact boundary components of length one,
    such that the first return $f^k$ is isotopic to a periodic isometry of
    $X$. 
  
  \end{lemma}

  \begin{proof}
  
    Recall that $X$ has no essential, non-peripheral curves, and by our
    construction of $\PSper$ and $\Sinf$, $X$ cannot be a closed disk or a
    closed disk with one puncture. If $f$ doesn't return to $X$, then since
    $\PS0$ is a subsurface and $f$--invariant, $X$ is wandering. In
    particular, $X$ cannot border $\PSper$, so it cannot have a compact
    boundary by Lemma \ref{lem:no-compact}. In this case, $X$ is a disk
    with at most one puncture and points removed from the boundary, from
    which we can get a contradiction by Lemma \ref{lem:jointly-wandering2}. 

    So $f$ returns to $X$. Up to replacing $f$ with a power, we can assume
    $f(X)=X$. If $X$ has finitely many boundary components, then $f$ is
    finite order. Otherwise, $X$ is a disk with infinitely many points
    removed from the boundary and possibly a puncture or a disk removed
    from its interior. Look at the action of $f$ on the non-compact
    boundary components. If the $f$--orbit of every component is infinite,
    then $f$ is semi-conjugate to an irrational rotation of the disk, which
    violates tameness. If there exists a finite orbit, then all periodic
    components of $\partial X$ have the same period. Let $Y \subset X$ be
    the smallest convex subsurface containing the puncture or the compact
    boundary component of $X$, if it exists, all the periodic boundary
    components of $X$ and all the periodic ends of $X$. Up to replacing $f$
    by a further power if necessary, we can assume $f$ fixes $Y$. We claim
    $X \ssm Y$ belongs to $S_\infty$. Let $Z$ be a component of $X \ssm Y$.
    Topologically, this is disk with points removed from the boundary which
    shares a unique common boundary $L$ with $Y$, which is fixed by $f$,
    and all other boundary components are wandering. Further, there exist
    ends $e_\pm$ (possibly $e_+=e_-$) of $L$, such that for any other $L'
    \subset \partial Z$, $f^i(L')$ converges to $e)\pm$ as $i \to
    \pm\infty$. Fix a wandering boundary component $L'$, and let $L_0$ be
    the line in $Z$ spanned by $e$ and the end of $L'$ such that $f^i(L')$,
    $i \ge 0$, lies on one side of $Z \ssm L_0$. Let $Z_0$ be the
    subsurface corresponding to that side, and let $Z_i = f^i(Z_0)$. Then
    $Z_i$ converges to $e_\pm$ as $i\to\pm\infty$. Each wandering component
    of $\partial Z$ must be a boundary component of $S_\infty$. Thus we can
    find some curve $\alpha$ that intersects $L'$ and $X$ essentially, and
    $\alpha \subset Z_0 \cup S_\infty$. Such curve must have $\L(\alpha)
    \ne \emptyset$, since $\alpha$ does not belong to $S_\infty$, but this
    is impossible since $f^i(\alpha) \subset Z_i$. This shows $X\ssm Y$
    cannot have any wandering boundary components, so $X=Y$.
    
    Let $k$ be the first returning time to $X$. Then we can get a
    hyperbolic metric on $X$ with respect to which $f^k$ is isotopic to a
    finite order isometry by looking at the quotient $X/\langle f^{k}
    \rangle$, choosing a hyperbolic structure with corners on the quotient
    and lifting it to $X$.  \qedhere

  \end{proof}

We can now establish the properties of the first return map for components of $\Sper$.

\begin{lemma}\label{lem:per-on-per}
	
	Let $X$ be a component of $\Sper$. Then $f$ returns to $X$
	and there is a hyperbolic metric on $X$ with compact boundary
	components of length one such that the first return $f^k$ is isotopic
	to a periodic isometry of $X$. 
	
\end{lemma}

\begin{proof}
	
	Since $X$ is connected and every curve in $X$ is $f$--periodic, $f$
	must return to $X$. Up to replacing $f$ by a power we may assume $f(X)$
	is isotopic to $X$. If $X$ has bounded topology, i.e.\ $X$ is
	finite-type or the double of $X$ has finite type, then $f$ is isotopic
	to a periodic map of $X$. Otherwise, choose a finite collection $\C$ of
	orbits of curves in $X$ and let $F=F(\C)$ be the subsurface spanned by
	$\C$. After possibly enlarging $\C$, we can assume $F$ is connected and
	has negative Euler characteristic. Since $\C$ is $f$--invariant so is
	$F$, and since all the curves in $\C$ are $f$--periodic, $f|_F$ is
	isotopic to a periodic map of $F$ by Lemma
	\ref{lem:periodic-finite-type}. Let $n > 0$ be the period of $f|_F$.
	For any curve $\alpha$ not in $F$, we can find a larger connected
	$f$--invariant finite-type subsurface $F'$ containing both $F$ and
	$\alpha$. By the same reasoning, $f$ is also isotopic to a periodic map
	on $F'$, but since $F' \supset F$, $f|_{F'}$ also has period $n$ (by
	the classical Nielsen--Thurston theory). In particular, $f^n(\alpha)$
	is isotopic to $\alpha$ for all $\alpha$.  By the Alexander method
	(\cite{hmv_Alexander}), this implies that $f|_X$ is isotopic to a
	periodic map of $X$. 
	
	The fact that we can realize the first return $f^k$ to $X$ by a
	periodic isometry is a direct consequence of Proposition
	\ref{prop:Nielsenrealization} (see \cite{accl_Nielsen}). \qedhere
	
\end{proof}


As a consequence, we can describe more precisely what changes when modifying the decomposition from $\Sinf$, $\PSper$ and $\PS0$ to $\Sinf$, $\Sper$ and $S_0$.
\begin{cor}\label{cor:changing-the-decomposition}
	Each pair of pants component of $\PS0$ yields a pair of pants component of $\Sper$, which contains at least one annular component of $\PSper$. Furthermore, no component of $\PS0$ is self-adjacent.
\end{cor}
\begin{proof}
	Since a pair of pants component $X$ of $\PS0$ has only compact boundary components, it must border at least one component  $Y$ of $\PSper$. If $Y$ is not an annulus, we can construct a curve $\alpha\subset X\cup Y$ intersecting the boundary of $X$ essentially. As some power of $f$ is periodic on both $X$ and $Y$, and by tameness there is no twisting allowed along the boundary, $\alpha$ is periodic, a contradiction.
	
	If a component $X$ of $\PS0$ were self-adjacent along a boundary component $\alpha$, then $\alpha$ must be a curve --- otherwise there is a component $Y$ which is a strip with both boundary components in $X$. As $Y$ contains no curves, it is contained in $\PS0$, but then by construction $X$ and $Y$ would be part of the same component of $\PS0$. But we know by Lemma \ref{lem:per-on-S0} that $\alpha$ is periodic, so the annulus $Z$ about $\alpha$ is a component of $\PSper$. Since $X$ has at least two compact boundary components, it is a pair of pants, but then as before we could construct a $\beta$ contained in the genus-one surface $X\cup Z$ which intersects $\alpha$ essentially and is periodic, a contradiction. \qedhere
\end{proof}

Our next step is to exclude (self-)adjacency of components of $\Sper$.


\begin{lemma} \label{lem:Sper-neighbor}
	
	No two components of $\Sper$ can be adjacent, neither can a component of $\Sper$ be self-adjacent. 
	
\end{lemma}

\begin{proof}
	
	First suppose $X$ and $Y$ are essential components of $\Sper$ and let
	$\alpha$ be a common boundary component. Choose a regular neighborhood
	$N(\alpha)$ about $\alpha$ contained in $X \cup Y$. There exists $n$
	such that $f^n$ is isotopic to the identity on $X$ and $Y$. Isotope
	$f^n$ so that it takes $N(\alpha)$ to $N(\alpha)$ fixing $\partial
	N(\alpha)$ pointwise, and further isotope $f^n$ so that it's the
	identity on truncated subsurfaces $X \setminus N(\alpha)$ and $Y
	\setminus N(\alpha)$. If $N(\alpha)$ is a strip, then we can further
	isotope $f^n$ (rel boundary) inside of $N(\alpha)$ to the identity map.
	If $\alpha$ is a curve, then since $f$ is (extra) tame, it cannot twist
	about $\alpha$, so $f^n$ is also isotopic (rel boundary) to the
	identity map on $N(\alpha)$. But this implies $f^n$ is isotopic to the
	identity on $X \cup Y$, and in particular, we can find an essential
	curve $\beta$ in $X \cup Y$ crossing $\alpha$ which is $f$--periodic. 
	
	The proof that a component $X$ has self-adjacency along a periodic
	annular component $N(\alpha)$ is similar. In this case, we can isotope
	$f^n$ to be identity on $X \cup N(\alpha)$, which contradicts that $X$
	and $N(\alpha)$ are components of $\Sper$. \qedhere 
	
\end{proof}

With all these results at hand, we can prove the wanted properties of the first return map of components of $\Sinf$.
  
  \begin{lemma}\label{lem:Sinf-not-wandering}

    No component $X$ of $\Sinf$ has a wandering boundary component. In
    particular, $f$ returns to $X$, and there is a hyperbolic metric on
    $X$, such that, the first return $f^n$ is isotopic to an isometric
    translation on $X$.

  \end{lemma}
  
  \begin{proof} 
    
    Let $\alpha$ be a wandering boundary component of $X$. Since $f$
    returns to each component of $\Sper$ and $S_0$ and the first return
    maps are periodic, the neighbor $Y$ of $X$ along $\alpha$ must belong
    to $\Sinf$, but this contradicts Lemma \ref{lem:jointly-wandering1}.
    Therefore $X$ has no wandering boundary component, so $f$ must return
    to $X$. By Theorem \ref{thm:translation}, the first return $f^n$
    up to isotopy preserves a hyperbolic metric on $X$ on which it acts as
    an isometric translation. \qedhere

  \end{proof}

Our next goal is to show that there is no self-adjacent component of the decomposition. We need a preliminary result.
  
  \begin{lemma} \label{lem:S0-neighbor}
    
    A component $X$ of $S_0$ can share at most one boundary with $\Sper$,
    and if $X$ shares a boundary $\alpha$ with $\Sper$, then every non-contractible curve in $X$ is homotopic to $\alpha$. 

  \end{lemma}

  \begin{proof}

    We now know $X$ is $f$--periodic, so the proof is similar to
    \ref{lem:Sper-neighbor}. The assumptions on $X$ are to rule out the
    existence of a periodic curve $\beta$ crossing the common boundary of
    $X$ with $\Sper$ essentially. \qedhere  

  \end{proof}

We can now show the no self-adjacency result.

  \begin{lemma}\label{lem:no-self-adjacent}

    No component of the canonical decomposition is self-adjacent.

  \end{lemma}

\begin{proof}
Suppose $X$ is a self-adjacent component. By Corollary \ref{cor:changing-the-decomposition} and Lemma \ref{lem:Sper-neighbor}, $X$ is a component of $\Sinf$. Since $\Sinf$ has no compact boundary components by Lemma \ref{lem:no-compact}, there is a strip component $Y$ of $S_0$, with core line $\ell$, so that $X$ is the only neighbor of $Y$. By Lemma \ref{lem:Sinf-not-wandering}, there is some $n\geq 1$ so that $f^n$ fixes $\ell_1$ (and therefore $\ell_2$, since they are homotopic). Since $Y$ is a strip, an orientation on $\ell$ will induce orientations on $\ell_1$ and $\ell_2$. Now pick a curve $\alpha$ contained in $X\cup Y$ and intersecting both $\ell_1$ and $\ell_2$ once. Then $\alpha\cap X$ is an arc from $\partial X$ to $\partial X$, and by Proposition \ref{prop:wandering-arcs} it is wandering. This implies that $f^n$ acts as a positive translation on $\ell_1$ (with respect to the chosen orientation) if and only if it acts as a positive translation on $\ell_2$. But then we can homotope $f^n$ on the strip, leaving it invariant on its boundary, so that it's a translation, and therefore $\alpha$ is wandering, a contradiction.
\end{proof}

An easy consequence of the previous lemma is the fact that there are no components of the decomposition which are strips:

\begin{cor}\label{cor:no-strips}
No component of $S_0$, $\Sper$ or $\Sinf$ is a strip.
\end{cor}
\begin{proof}
Since by construction every component of $\Sper$ and $\Sinf$ contains an essential (possibly peripheral) curve, if there is a strip, it must be a component of $S_0$. But by construction, a strip can arise only if a component of $\Sper$ or $\Sinf$ is self-adjacent, which is impossible by Lemma \ref{lem:no-self-adjacent}.\qedhere
\end{proof}

The next result is a description of the possible components of $S_0$.

  \begin{lemma}\label{lem:top-S0}

    The following are the topological possibilities for
    a component $X$ of $S_0$:

    \begin{enumerate}
        \item A closed disk with at least three points removed from the
          boundary. In this case, at most one
          neighbor of $X$ is in $\Sper$.
        \item A once-punctured closed disk with at least one point removed
          from the boundary. In this case, all neighbors of $X$ are in $\Sinf$.
        \item A cylinder with one compact boundary and at least one point
          removed from the other boundary (a crown) In this case, $X$ has exactly one neighbor in $\Sper$ along its compact boundary. 
    \end{enumerate}

  \end{lemma}

  \begin{figure}[ht]
  \begin{center}
  \includegraphics{S0}
  \caption{The topological possibilities for the components of $S_0$}
  \end{center}
  \end{figure}  

  
  \begin{proof}
  
    Note first that $X$ is not a sphere, being a subsurface of $S$. It also cannot be a disk or a once-punctured disk, by the construction of $\Sinf$ and $\Sper$. Moreover by construction $X$ has non-negative Euler characteristic, thus $\chi(X)$ is either $1$ or $0$.
    
    If $\chi(X)=1$, then $X$ is a disk with $n$ points removed from its
    boundary. If $n$ were one, the boundary of $X$ would be homotopically trivial, which is impossible. Moreover, $n\neq 2$ by Corollary \ref{cor:no-strips}, so $n\geq 3$. Since the boundary
    components of $X$ are non-compact, $X$ cannot border annular
    components of $\Sper$, so it has at most one neighbor in $\Sper$.

    If $\chi(X)=0$, then the interior of $X$ is an open annulus. We already
    know that $X$ cannot be the once-punctured disk, and it cannot be a
    closed annulus. Thus, $X$ is either a once-punctured disk or a closed
    annulus with at least one point removed from its boundary. If it is an
    annulus with points removed from both boundaries, then it would contain
    an essential non-peripheral curve, contradicting Proposition
    \ref{prop:no-ess-nonper}. If $X$ is a once-punctured disk with points
    removed from its boundary, then $X$ has a non-contractible curve which
    is non-peripheral, so it cannot border $\Sper$. If $X$ is an annulus
    with points removed from one of its boundary, then the core curve of
    $X$ is not homotopic to the non-compact boundaries of $X$, so $X$ has
    at most one neighbor in $\Sper$ joined along its only compact boundary.
    \qedhere 

  \end{proof}

  
  We end this section with some examples showing the optimality of some of our results. First, we note that all topological possibilities from Lemma \ref{lem:top-S0} occur for components of
  $S_0$, and that there can be pairs of pants components of $\PS0$. Indeed, Figure \ref{fig:componentsofS0} exhibits examples of tame
  mapping classes with different topological types for the components of $\PS0$ and $S_0$. In each case, the map is a puncture shift in each shaded strip,
  in the direction indicated by the arrow, and it's the identity outside
  the strips.
  
  \begin{figure}[ht]
  	\begin{center}
  		\begin{overpic}{componentsofS0}
  			\put(-2,67){$f_1:$}
  			\put(5,49){$\PS0\simeq$}
  			\put(35,67){$f_2:$}
  			\put(40,49){$\PS0\simeq$}
  			\put(69,67){$f_3:$}
  			\put(74,49){$\PS0\simeq$}
  			\put(-7,28){$f_4:$} 
  			\put(-6,5){$\PS0\simeq$}
  			\put(37,28){$f_5:$}
  			\put(38,5){$\PS0\simeq$}
  			\put(67,28){$f_6:$}
  			\put(70,5){$\PS0\simeq$}
  		\end{overpic}
  		\caption{Examples of extra tame mapping classes}\label{fig:componentsofS0}
  	\end{center}
  \end{figure}
  
  Since an extra tame map returns to each component of $S_0$, $\Sper$ and
  $\Sinf$, one might wonder if there is a uniform returning time for all
  components. The answer is negative, as shown by the following lemma.
  Moreover, given a component of $\Sinf$, there is not always a uniform
  power of $f$ fixing all of its boundary components at once.
  
  \begin{lemma}\label{lem:no-uniform-return}
  	There is an extra tame map so that there is no uniform returning time for all components of the canonical decomposition. Furthermore, the map can be chosen so that there is a component of $\Sinf$ so that there isn't a uniform returning time for its boundary components.
  \end{lemma}
  
  \begin{proof}
  	Consider the bordered surface $$P=\left(\R_{\geq 0}\times
  	\R\right)\ssm\bigcup_{n\geq 1}B_n,$$ where $B_n$ is the open disk of
  	center $(n,0)$ and radius $\frac{1}{4}$. Let $a_n$ be the oriented
  	segment of the $x$-axis from $B_n$ to $B_{n+1}$ and $a_0$ the oriented
  	segment of the $x$-axis from the $y$-axis to $B_1$.  Define the map
  	$\psi:\pi_1(P)\to \Z$ by $$\psi(\alpha)=\sum_{n\geq
  		0}2^n\hat{\iota}(\alpha,a_n),$$ where $\hat{\iota}(\cdot,\cdot)$ is the
  	algebraic intersection form.
  	
  	\begin{figure}[hb]
  		\begin{center}
  			\begin{overpic}[scale=1.4]{cover}
  				\put(2,32){$a_0$}
  				\put(21,32){$a_1$}
  				\put(40,32){$a_2$}
  				\put(59,32){$a_3$}
  				\put(78,32){$a_4$}
  				\put(11,26){$B_1$}
  				\put(30,26){$B_2$}
  				\put(49,26){$B_3$}
  				\put(68,26){$B_4$}
  				\put(87,26){$B_1$}
  			\end{overpic}
  			\caption{The surface $P$ with the arcs $a_n$ in the proof of Lemma \ref{lem:no-uniform-return}}
  		\end{center}
  	\end{figure}
  	
  	Let $p:\Sigma\to P$ be the associated cover and $f$ a generator of the
  	deck group. Then $\Sigma$ is a bordered subsurface, whose boundary is a
  	union of lines, given by lifts of the $y$-axis and of the curves
  	$\partial B_n$. Note that, for any $n$, the number of lifts of
  	$\partial B_n$ is $2^n-2^{n-1}$ and they are cyclically permuted by
  	$f$. So there is no uniform power of $f$ fixing all boundary components
  	of $\Sigma$.
  	
  	To turn this into an example of an extra tame map on a (borderless)
  	surface, glue to each boundary component of $\Sigma$ a copy of the
  	thrice-punctured half-plane $H$. We get a surface $S$ and a
  	homeomorphism $\bar{f}$ of $S$ so that $\bar{f}|_\Sigma$ is $f$ and for
  	every copy of $H$, the first return map is the identity.
  	
  	The map $\bar{f}$ is extra tame and:
  	\begin{itemize}
  		\item $\Sinf=\Sigma$ and $\bar{f}|_{\Sinf}=f$, so there is no bound on
  		the first returning time of the boundary components of $\Sinf$;
  		\item $\Sper$ is the union, for every copy of $H$, of a disk with three
  		punctures. The first returning time of a component of $X$ is the same
  		as the first returning time of the boundary component of the copy of
  		$H$ in which $X$ is contained, so also these are unbounded;
  		\item $S_0$ is the union, for every copy of $H$, of an annulus with a
  		point removed from the boundary, and again there is no uniform bound on
  		the first returning times of components of $S_0$.
  	\end{itemize} 
  	
  \end{proof}
  

  \subsection{Proof of main theorem}
  
Our main theorem is now an easy consequence of the results proved so far.

  \begin{thm}\label{thm:mainthm} 
 Let $f$ be an extra tame map of a surface $S$ of infinite type. There
    is a canonical decomposition of $S$ into three $f$--invariant
    subsurfaces $\Sper$, $\Sinf$ and $S_0$ and a hyperbolic metric on
    $S$ such that for every component $X$ of $\Sper$, $\Sinf$ and $S_0$, $f$ returns to $X$ and is isotopic to:
    \begin{itemize}
      \item a periodic isometry, if $X\subset \Sper\cup S_0$, 
      \item an isometric translation, if $X\subset \Sinf$.
    \end{itemize} 
Furthermore, components of $S_0$ contain no essential non-peripheral curves and at most one essential (peripheral) curve.
  \end{thm}

  \begin{proof}
  
  Let $S_0$, $\Sper$ and $\Sinf$ be given by Proposition \ref{prop:decomposition}. HThe hyperbolic structure on $S$ is given by gluing the hyperbolic structures provided by Lemmas \ref{lem:per-on-per} and \ref{lem:per-on-S0} and Theorem \ref{thm:translation}: since there are no strips (Corollary \ref{cor:no-strips}), the hyperbolic structure contains no funnels mor half-planes. Moreover the same results guarantee that for every component of the decomposition the first return map of $f$ exists and is isotopic to a periodic isometry or an isometric translation as required.
  \end{proof}



  \subsection{Constructing extra tame maps}

Theorem \ref{thm:mainthm} shows that every extra tame map is obtained by gluing together translations and periodic maps supported on disjoint subsurfaces, not accumulating anywhere. If we want to construct extra tame maps by following this procedure, we need to be careful when two supporting subsurfaces share a compact boundary component: we have to make sure that the gluing doesn't create a Dehn twist.

On the other hand, there is no need for special care when gluing along lines. More precisely: 

\begin{lemma}
	Let $\{\Sigma_i\}_{i\in I}$ be a decomposition of a surface $S$ into essential subsurfaces without self-adjacency, not accumulating anywhere and without compact boundary components. For every $i$, let $f_i$ be either a periodic map or a translation of $\Sigma_i$, and assume that for every line $\ell\subset\partial\Sigma_i\cap\partial\Sigma_j$, $f_i(\ell)=f_j(\ell)$.
	Then there is an extra tame mapping class $f$ on $S$ such that the restriction of $f$ on each $\Sigma_i$ is properly homotopic to $f_i$.
\end{lemma}

\begin{proof}
Define $S'$ to be the surface, homeomorphic to $S$, given by inserting strips at each line in the boundary of the decomposition. More precisely, we take the disjoint union of the $\Sigma_i$ and of a strip $S_\ell=\ell\times[0,1]$ for every $\ell$ in the boundary, and if $\ell\subset\partial\Sigma_i\cap\partial\Sigma_j$, we glue $\ell\times\{0\}$ with the copy of $\ell$ in $\Sigma_i$ and  
 $\ell\times\{1\}$ with the copy of $\ell$ in $\Sigma_j$. Define then $f_\ell$ on $S_\ell$ by linearly interpolating $f_i$ and $f_j$:
 $$f_\ell(p,t)=(1-t)f_i(p)+t f_j(p)$$ 
 where we have fixed an identification of $\ell$ with $\R$.
Then we obtain a homeomorphism $f$ of $S'$ by gluing the $f_i$ and the $f_\ell$. We claim that $f$ is extra tame.

Indeed, let $\alpha$ and $\beta$ be two curves in $S'$. By the properties of the decomposition, and up to modifying the curves by a homotopy, we know that there are finitely many indices $j\in I$ and lines $\ell$ in the boundary of the decomposition so that $\alpha\cap\Sigma_j$, $\beta\cap\Sigma_j$, $\alpha\cap S_\ell$ or $\beta\cap S_\ell$ are not empty. By tameness of the $f_j$,
$$i(f_j^n(\alpha\cap\Sigma_j),\beta\cap\Sigma_j)$$
is uniformly bounded and
$$i(f_\ell(\alpha\cap S_\ell), \beta\cap S_\ell)$$
is bounded by the number of intersection of $\alpha$ and $\beta$ with the boundary of the decomposition, so it is also uniformly bounded. Therefore $i(f^n(\alpha),\beta)$ is uniformly bounded and hence $f$ is tame. A similar argument shows the finiteness of the limit set of $\alpha$.
\end{proof}

Finally, note that if we glue periodic maps and translation and we obtain an extra tame map, the canonical decomposition might not coincide with the collection of supporting subsurfaces of the maps we are gluing. For instance, let $S=\R^2\ssm \Z^2$,
$\Sigma_1=S\cap \{y\geq\frac{1}{2}\}$ and $\Sigma_1=S\cap \{y\leq\frac{1}{2}\}$. Let $f_1$ be the map $(x,y)\mapsto
(x+1,y)$ and $f_2$ a map homotopic
to $(x,y)\to (x+2,y)$ and agreeing with $f_1$ on $\partial\Sigma_1=\partial\Sigma_2$. Then the map obtained by gluing $f_1$ and $f_2$ is a translation on $S$. In particular, $S=\Sinf$.
\bibliographystyle{plain}
\bibliography{bibliography}

\end{document} 
