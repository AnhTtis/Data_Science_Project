\documentclass{FeasibleOLG_main.tex}{subfiles}

\begin{document}





In overlapping generation (OLG) repeated games, players play for finite periods and are replaced by their next generation. This class of games has been used to study cooperation among finitely-lived players in long-run organizations (e.g., \cite{Hammond_1975} and \cite{Cremer_1986_QJE}).


In this paper, we study the feasible payoff set of OLG repeated games. In the literature on OLG repeated games, including studies on folk theorems, the convex hull of the stage game payoffs is typically considered the feasible payoff set of interest. However, the overlapping structure of these games allows players to achieve average discounted payoffs beyond the convex hull of the static payoffs: Although players share the same discount factor, they discount payoffs differently depending on their position in the lifecycle. Thus, it is not obvious which payoffs are feasible. Our purpose in this study is to understand how the OLG structure affects what players can obtain. 

 We study the feasible payoff set when players' discount factor and the period of overlap are fixed, which departs from most studies in the literature, which usually focus on the asymptotic case. Specifically, we focus on ``stationary'' feasible payoffs in which each generation of the same player plays the same sequence of actions during their lifetime.\footnote{In general, each generation of the same player plays different sequences of action profiles. In this case, each player has an infinite sequence of feasible payoffs. We discuss this further in \Cref{subsec:discuss}.} 

		
		
First, we completely characterize the feasible payoff set of OLG repeated games. We find that the feasible payoff set can be characterized by the convex hull of the set of average discounted payoffs that can be achieved by playing length-$n$ sequences of action profiles, where $n$ is the number of players and each of the action profiles is supposed to be played for $T$ (the interaction length) times consecutively. For such sequences of action profiles, we can calculate the average discounted payoff \emph{as if} players play length-$n$ sequences (rather than $nT$), while effectively discounting by $\delta^T$ (rather than by $\delta$). Thus, this characterization substantially simplifies the set of action profiles we need to consider to obtain the feasible payoff set. In fact, our characterization allows a closed-form expression of the feasible payoff set of any stage game.

Second, we analyze the comparative statics of the feasible payoff set with respect to $\delta$ and $T$. We find that the feasible payoff set is decreasing  (in the set-inclusion sense) in the effective discount factor $\delta^T$. Interestingly, this implies that the set is \emph{decreasing} in $\delta$. When the effective discount factor is $1$, the feasible payoff set coincides with the convex hull of the static payoffs. When it is close to $0$, it is an $n$-dimensional cube, where for each player, the maximum (resp. minimum) feasible payoff coincides with the maximum (resp. minimum) stage game payoff. For intermediate effective discount factors, it is an $n$-dimensional polytope. Furthermore, we examine the condition under which the above monotonicity holds in a strict sense. We find that unless the one-shot feasible set is already a (multidimensional) cube, the OLG feasible set satisfies the strict monotonicity with respect to the parameters.





	
\textbf{Related Literature} 
	
	Previous research on OLG repeated games has mainly focused on folk-theorem-like approaches such as those in \cite{Kandori_1992_RES} and \cite{Smith_1992_GEB}. Both studies regard the convex hull of the stage game payoffs as the feasible payoff set, which is targeted by their folk theorems. Since their folk theorems first choose sufficiently large $T$ and then $\delta$ close to 1, the feasible set for such a case (as shown in the current study) is the convex hull of the stage game payoffs.\footnote{More precisely, \cite{Kandori_1992_RES} assumes $\delta =1$ throughout the paper, but mentions that the analogous result holds with sufficiently large $\delta<1$ because of the strictness of the punishments. \cite{Smith_1992_GEB} considers two types of folk theorems: non-uniform and uniform. In the non-uniform folk theorem, he first chooses sufficiently large $T$ and then chooses $\delta$. In the uniform folk theorem, he chooses $T$ and $\delta$ simultaneously.} More recently, \cite{Morooka_2021_IJGT} provides an alternative OLG folk theorem with the opposite order of choosing parameters: It shows that if $\delta$ is chosen first and then $T$ is chosen, any feasible and strictly individually rational payoffs can be achieved by subgame perfect equilibrium payoffs, where the feasible payoff set considered is larger than the convex hull of the stage game payoffs. Alternatively, in this study, we analyze OLG repeated games with fixed $\delta$ and $T$. As mentioned, we provide a complete characterization of the feasible payoff set. By studying the feasible payoff set, we provide a natural benchmark for the equilibrium payoff set, which may be of greater interest. 


The present study is also related to the literature on repeated games with differential discounting of players, which has been studied since \cite{LP_1999_ECMA}. They study infinitely repeated games between two players with different discount factors and show that some payoffs outside the convex hull of the stage game payoffs can be obtained by intertemporal trading of payoffs, i.e., the more patient player concedes payoffs in early periods to obtain higher payoffs in later periods. \cite{CT_2012_GEB} generalize this analysis to the case of more than two players and provide a folk theorem. \cite{Sugaya_2015_TE} proves a folk theorem for $n$-player infinitely repeated games with imperfect public monitoring. \cite{DG_2022_JET} provide a more constructive approach to studying feasible and equilibrium payoffs of repeated games with perfect monitoring. On the other hand, \cite{Chen_2007_EL} and \cite{CF_2013_IJGT} study finitely repeated games between two players. These papers examine whether the feasible payoff set expands as the length of the game increases. The latter paper, built on the result of the former, shows that this is indeed the case for any two-player stage games.\footnote{They leave the question open for a more general case involving an arbitrary number of players.} 



Compared to the literature on repeated games with differential discounting, in our model of OLG repeated games, players share the \emph{same} discount factor. Nevertheless, players can trade payoffs across periods due to the overlapping generation structure: In a given period, players are located in different positions in their lifecycle (``age''), resulting in different discounting of some future payoffs. Notice that players discount their payoffs in the same way when they are the same age. In this sense, their discounting is ``symmetric,'' which makes our analysis relatively tractable. This results in our characterization of the feasible payoff set, allowing a closed-form expression of the feasible payoff set given any stage game for each discount factor and the length of each generation's lifespan.\footnote{In the literature on repeated games with differential discounting, \cite{Chen_2007_EL} provides an explicit characterization of the feasible payoffs of finitely repeated games for a specific two-player stage game. \cite{Sugaya_2015_TE} provides a recursive characterization of the feasible payoff set of infinitely repeated games for general stage games. \cite{DG_2022_JET} provide several characterizations of the feasible payoff set; in particular, they characterize it when players can have large discount factors.} 


The remainder of this paper is organized as follows. In \Cref{sec:2}, we introduce the model of OLG repeated games. In \Cref{sec:3}, we present our first main result, which is a complete characterization of the feasible payoff set of OLG repeated games. In \Cref{sec:4}, we provide the comparative statics results of the feasible payoff set with respect to $\delta$ and $T$. In \Cref{sec:5}, we provide two examples to illustrate our main results. In \Cref{sec:6}, we generalize the model and extend our main results. In \Cref{sec:7}, we have some discussions. \Cref{sec:8} concludes the paper. Omitted proofs can be found in the appendix.


	

	



	
\end{document}
