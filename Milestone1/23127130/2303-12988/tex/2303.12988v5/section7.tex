\documentclass{FeasibleOLG_main.tex}{subfiles}
\begin{document}



\subsection{Non-stationary Play}
\label{subsec:discuss}



Thus far, we have restricted our attention to stationary feasible payoffs. In this subsection, we discuss how the relaxation of this restriction would affect the monotonicity result in terms of $\delta$. We consider non-stationary sequences of players' action profiles, in which different generations of the same player may play different sequences of action profiles during their lifetime. One way to extend the concept of monotonicity in this case would be as follows: Given a sequence $(\bar{u}^y(\delta))_{y \in \mathbb{N}}$ of players' average discounted payoffs for each generation $y=1,2,\dots$, which is feasible with discount factor $\delta$, we ask whether the same sequence is feasible with $\delta' < \delta$. 

The following example shows that this is not the case: Consider the OLG repeated game with two players and two possible stage game payoffs of $(1,0)$ and $(0,1)$. Suppose $T=1$. Consider the following sequence of payoff vectors for each $t = 1,2, \dots$:
$(1,0), (1,0), (0,1), (0,1), \dots.$
 That is, the first two periods give $(1,0)$, followed by $(0,1)$ forever. The corresponding average payoff for each generation of player 1 is $\bar{u}_1^1 = 1$ for the first generation, and $\bar{u}_1^y= 0$ for any $y \geq 2$. Observe that player 2 in the first generation, who is born at $t=2$, obtains the average payoff $\bar{u}_2^1 = \frac{ \delta}{ 1+ \delta}$ and $\bar{u}_2^y = 1$ for any $y \geq 2$.

Now consider $\delta' < \delta$. Since $\bar{u}_1^1= 1$, the first two periods must give players $(1,0), (1,0)$. Note that the maximum payoff of player 2 in the first generation is obtained when $(0,1)$ is given at $t=3$, yielding the average payoff $\frac{\delta'}{1+\delta'}$, which is strictly smaller than $\frac{\delta}{ 1+ \delta}$. 


 \subsection{Implications for Equilibrium Payoffs}
 \label{sec:6.2}


The monotonicity result of the feasible payoff set can also shed light on the equilibrium (Nash equilibrium or subgame perfect equilibrium) payoff set. It follows from our findings that the set of ``stationary'' equilibrium (i.e., each generation of a player employs the same strategy) payoffs is, in general, not increasing in $\delta$. Although it is obvious, we state this as a proposition below.\footnote{This contrasts with the case of repeated games with infinitely-lived players with a PRD \cite[]{APS_1990_ECMA}, where the subgame perfect equilibrium payoff set is monotonically increasing in players' discount factor. When there is no PRD, this monotonicity might not hold (see \cite{MOS_2002_GEB,Yamamoto_2010_IJGT}).} 



\begin{prop}
Consider an OLG repeated game with the stage game $G$. Suppose that any extreme point of $V$ is a payoff vector of a static Nash equilibrium of $G$. Then the stationary subgame perfect equilibrium payoff set is non-increasing in $\delta$ and non-decreasing in $T$. 
\end{prop}

\begin{proof}
Under the hypothesis, the OLG feasible payoff set coincides with the stationary subgame perfect equilibrium payoff set for any given $\delta$ and $T$. Thus, \autoref{them:2} implies the result.
\end{proof}





Intuitively, if any extreme points of $V$ can be achieved by some Nash equilibrium, there is no issue of providing incentives to players to play a certain action. In general, to implement a certain action, appropriate incentives should be given by varying continuation payoffs. This suggests that there would be a trade-off of players being more patient in OLG repeated games: Being more patient, players are more easily disciplined as they care more about future payoffs. On the other hand, it is also costly, as it makes players more difficult to make intertemporal trades of their payoffs. This suggests that the ``optimal'' discount factor of players may be some intermediate value for some games.

Let us consider a simple example to illustrate the point made in the previous paragraph. Consider the OLG repeated games with $T=1$, where the stage game is described in \autoref{fig:6} and let $\lambda = (1, 1)$. We claim that the optimal $\delta$ that maximizes the $\lambda$-weighted sum of players' average discounted payoffs in (``stationary'' subgame perfect) equilibrium is $\delta = \frac{1}{2}$. To see this, we first observe that if $\delta \geq 1/2$, playing $(L, L),(R,R)$ (they are the action profiles that give the maximum sum of players' payoffs) for odd and even periods, respectively, can be achieved by a stationary subgame perfect equilibrium. This is because when players are old, they have no incentive to deviate; in addition, each player's action when they are young can be supported by punishing any deviation from it with $(M,M)$ (this is a one-shot Nash profile and gives the minmax payoff to each player): $2 + \delta  \geq 3 - \delta$ or $\delta \geq \frac{1}{2}$. As a result, the monotonicity result implies that the smallest $\delta = \frac{1}{2}$ yields the largest weighted sum of payoffs. 
\begin{figure}[]
\centering
\begin{tabular}{|c|c|c|c|}
\hline
    & $L$     & $R$     & $M$     \\ \hline
$L$ & $2,1$   & $-3,-3$ & $-3,-1$ \\ \hline
$R$ & $-3,-3$ & $1,2$   & $-5,3$  \\ \hline
$M$ & $3,-5$  & $-1,-3$ & $-1,-1$ \\ \hline
\end{tabular}
\caption{The stage game used in the example in \Cref{sec:6.2} }
\label{fig:6}
\end{figure}










\section{Conclusion} 
\label{sec:8}

In the present study, we analyze the feasible payoff set of OLG repeated games. First, we show that the set can be characterized by a convex combination of the average discounted payoffs of length-$n$ sequences of action profiles, where each of the action profiles is played for $T$ periods consecutively. Second, we find that the feasible payoff set is monotonically decreasing in $\delta$ and increasing in $T$. 


A natural and important direction for future research is to study the set of equilibrium payoffs. In particular, it would be meaningful to extend our understanding of the trade-off of having more patient players, which is briefly described in \Cref{sec:6.2}. Relatedly, one might also think of a recursive characterization of the subgame perfect equilibrium payoff set, as in \cite{APS_1990_ECMA}. Making such a recursive characterization would be relatively straightforward. However, unlike the standard repeated games with infinitely-lived players, the object of the recursive characterization would be the set of the equilibrium payoff \emph{sequences} (of the length of $nT$) even for stationary equilibria, where each generation plays the same action, because players' ages keep changing. One might also consider the class of strongly symmetric equilibria \cite[]{APS_1986_JET} for symmetric stage games, as it allows a much simpler recursive characterization for standard repeated games. However, note that, in OLG repeated games, players are not symmetric, as their ages are different, even when the underlying stage game is symmetric. 



\end{document}
