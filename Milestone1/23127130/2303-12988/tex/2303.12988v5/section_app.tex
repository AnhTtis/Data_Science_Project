\documentclass{FeasibleOLG_main.tex}{subfiles}
\begin{document}

\subsection{Proof of \autoref{lem:1}}
\label{sec:proof_lem1}

\begin{proof}
 

Denote the ``age'' of player $i$ at overlap $k$ by $y_k (i) \in \{ 1,\dots, n\}$. 


Consider overlap $k  \in \{ 1, \dots, n\}$ and $m \in \{ 1,\dots, n-1\}$. Let $k' \equiv k +m \text{ (mod $n$)}$. Then, from the optimality of $u^k$ at overlap $k$, 
		$$\sum_{ i =1}^n   \Delta^{ y_k (i)-1 } \lambda_i  u_i^k   \geq \sum_{i=1}^n \Delta^{y_k(i)-1} \lambda_i u_i^{k'} .$$
		By multiplying both sides by $\Delta^m$,
$$\sum_{ i =1}^n   \Delta^{ y_k (i)+ m-1 } \lambda_i  u_i^k  \geq \sum_{i=1}^n \Delta^{y_k(i) + m-1} \lambda_i u_i^{k'} .$$
Note that $y_{k'} (i) = y_k (i) + m$ if $y_k (i) + m \leq n$; otherwise $y_{k'} (i)= y_k (i) + m-n$.
\begin{multline*}
\sum_{ i : y_k(i) + m \leq n}   \Delta^{ y_{k'} (i)-1 } \lambda_i  u_i^k + \Delta^n \sum_{ i : y_k(i) + m >n}   \Delta^{ y_{k'} (i)-1 } \lambda_i  u_i^k \\
\geq \sum_{i : y_k(i) + m \leq n} \Delta^{y_{k'}(i)-1} \lambda_i u_i^{k'}  +\Delta^n\sum_{i : y_k(i) + m >n} \Delta^{y_{k'}(i)-1} \lambda_i u_i^{k'},	
\end{multline*}
or, equivalently, 
\begin{multline}
\label{eq:nnn6}
\sum_{ i : y_k(i) + m \leq n}   \Delta^{ y_{k'} (i)-1 } \lambda_i  u_i^k - \sum_{i : y_k(i) + m \leq n} \Delta^{y_{k'}(i)-1} \lambda_i u_i^{k'}  \\
\geq   \Delta^n \left ( \sum_{i : y_k(i) + m >n} \Delta^{y_{k'}(i)-1} \lambda_i u_i^{k'} -  \sum_{ i : y_k(i) + m >n}   \Delta^{ y_{k'} (i)-1 } \lambda_i  u_i^k \right).	
\end{multline}
On the other hand, from optimality at overlap $k'$, we have 
\begin{multline*}
\sum_{ i : y_k(i) + m \leq n}   \Delta^{ y_{k'} (i)-1 } \lambda_i  u_i^{k'}  + \sum_{ i : y_k(i) + m >n}   \Delta^{ y_{k'} (i)-1 } \lambda_i  u_i^{k'}  \\
\geq \sum_{i : y_k(i) + m \leq n} \Delta^{y_{k'}(i)-1} \lambda_i u_i^k  +\sum_{i : y_k(i) + m >n} \Delta^{y_{k'}(i)-1} \lambda_i u_i^k, 
\end{multline*}
or
\begin{multline}
\label{eq:nnn7}
\sum_{i : y_k(i) + m \leq n} \Delta^{y_{k'}(i)-1} \lambda_i u_i^k- \sum_{ i : y_k(i) + m \leq n}   \Delta^{ y_{k'} (i)-1 } \lambda_i  u_i^{k'}    \\
\leq     \sum_{ i : y_k(i) + m >n}   \Delta^{ y_{k'} (i)-1 } \lambda_i  u_i^{k'} -\sum_{i : y_k(i) + m >n} \Delta^{y_{k'}(i)-1} \lambda_i u_i^k.
\end{multline}
In order to satisfy both \eqref{eq:nnn6} and \eqref{eq:nnn7}, it must be 
$$\sum_{ i : y_k(i) + m >n}   \Delta^{ y_{k'} (i)-1 } \lambda_i  u_i^{k'}  \geq \sum_{i : y_k(i) + m >n} \Delta^{y_{k'}(i)-1} \lambda_i u_i^k,$$
or, equivalently,
$$\sum_{ i : y_{k'}(i) \in \{ 1, \dots, m\} }   \Delta^{ y_{k'} (i)-1 } \lambda_i  u_i^{k'}  \geq \sum_{i : y_{k'}(i) \in \{ 1, \dots, m\}} \Delta^{y_{k'}(i)-1} \lambda_i u_i^k .$$
Summing up both sides over $k \in \{1, \dots, n\}$, we have the inequality in the statement.
	\end{proof}


\subsection{Proof of \autoref{lem:2}}
\label{proof:lem2}
\begin{proof}








For $a^{[n]} \in \mathcal{A}^{*}_{\lambda}(\Delta)$, denote the numerator of $\frac{\partial W_{ \lambda}}{\partial \Delta } (a^{[n]},\Delta )$ by $\eta(\Delta)$. We will show that $\eta (\Delta) \leq 0$, thereby $\frac{\partial W_{ \lambda}}{\partial \Delta } (a^{[n]},\Delta ) \leq 0$. From \eqref{eq:nn3}, observe that for $u^{[n]} \in \mathcal{U}^{*}_{\lambda}(\Delta)$,
\begin{align*}
\eta (\Delta) &= \left (\sum_{k=1}^n \frac{d\Delta^{k-1} }{d\Delta }w_k (u^{[n]}) \right) \left(\sum_{m=1}^n \Delta^{m- 1} \right) - \left( \frac{d \sum_{m=1}^{n} \Delta^{m-1}} { d\Delta}\right)  \sum_{k=1}^n \Delta^{k-1} w_k(u^{[n]})\\
&=\left (\sum_{k=1}^n (k-1)\Delta^{k-2} w_k(u^{[n]}) \right) \left(\sum_{m=1}^n \Delta^{m-1} \right) - \left(\sum_{k=m}^{n} (m-1) \Delta^{m-2}\right)  \sum_{k=1}^n \Delta^{k-1} w_k(u^{[n]})\\
&=\sum_{k=1}^n \left ((k-1) \Delta^{k-2} \sum_{m=1}^{n} \Delta^{m-1} - \left( \sum_{m=1}^n (m-1) \Delta^{m-1}  \right) \Delta^{k-2}  \right) w_k(u^{[n]})\\
&=\sum_{k=1}^n  \left( \sum_{m=1}^n (k-m) \Delta^{k+m-3} \right)w_k(u^{[n]}).
\end{align*}
Recall the $m$-th constraint, $m \in \{ 1, \dots, n-1\}$, is
$$-\sum_{k=1}^m \Delta^{k-1} w_k (u^{[n]})  + \sum_{k=1}^m \Delta^{k-1} w_{n-m+k} (u^{[n]})\leq 0.   \quad (\# m)$$
We will show that 
$$\eta (\Delta) = \sum _{m=1}^{n-1}p_m\times (LHS\ of\ \# m),$$
where for $1\leq m\leq n-1$
$$p_m \equiv \Delta ^{n-2}+\Delta ^{n-3}+\cdots +\Delta ^{n-m-1}.$$
Note that as $p_m \geq 0$, this implies $\eta (\Delta) \leq 0$. To see the equality, fix $j \in \{ 1,\dots, n\}$. Observe that for any constraint $m\geq j$, there exists a non-positive term $- \Delta^{j-1}w_j (u^{[n]})$. In addition, for any constraint $m$ with $j \geq n-m+1$, there exists a non-negative term $\Delta^{j-n+m-1} w_j (u^{[n]})$. Therefore, the coefficient of $w_j (u^{[n]})$ is:
\begin{align*}
&\sum_{m = j}^{n-1} p_m (-\Delta ^{j-1} ) + \sum_{m =n+1-j}^{n-1} p_m \Delta^{j-n+m-1}\\
&=\sum_{m = j}^{n-1} (\Delta^{n-2} + \cdots + \Delta^{n-m-1}) (-\Delta^{j-1}) + \sum_{ m=n+1-j}^{n-1} (\Delta^{n-2} + \cdots + \Delta^{n-m-1}) \Delta^{j-n+m-1} \\
&=\sum_{ m=j}^{n-1} (-\Delta^{n+j-3} - \cdots - \Delta^{n-m+j-2})  + \sum_{m=n+1-j}^{n-1} (\Delta^{j+m-3} +\cdots + \Delta^{j-2})\\
&=(- \Delta^{n+j-3}- \cdots - \Delta^{n-2}) + (- \Delta^{n+j-3} - \cdots - \Delta^{n-1}) + \cdots + (-\Delta^{n+j-3} - \cdots - \Delta^{j-1})  \\
&+ (\Delta^{n-2} + \cdots + \Delta^{j-2}) + (\Delta^{n-1} + \cdots + \Delta^{j-2})  + \cdots + (\Delta^{j+n-4} + \cdots + \Delta^{j-2})\\
&=- \Delta^{n+j-3} ((n-j)-0) - \Delta^{n+j-4} ((n-j)-1)- \cdots - \Delta^{n-2} ((n-j) -(j-1))\\
&- \Delta^{n-3} ((n-j-1)- (j-1))- \cdots - \Delta^{j-1} (1- (j-1))- \Delta^{j-2} (0-(j-1))\\
&=\sum_{m=1}^n (j-m) \Delta^{j+m-3}.
\end{align*}
\end{proof}

\subsection{Proof of \autoref{prop:nnn1}}
\label{proof:prop1}



Let $S^0 \equiv \{a^{[n]}\in A^n: u(a^1)=u(a^2)=\dots =u(a^n)\}\subset A^n$ and $S^C \equiv A^n\setminus S^0$. First, we provide a few lemmas. 


\begin{lem}
\label{lem:nn3}
$V \neq  F^*$ if and only if there exist two points $u$, $\tilde{u} \in V$ with $u \neq \tilde{u}$ that satisfy the following two conditions:
\begin{enumerate}
	\item $u$ and $\tilde{u}$ are vertices of an edge of $V$, i.e., there exists $\lambda \in \mathbb{R}^n \setminus \{ \mathbf{0} \}$ such that $\bar{u} \in \argmax_{ u' \in V} \lambda  \cdot u'$ holds if and only if
$\bar{u}  \in \{  t u +(1- t)\tilde{u} : t \in [0,1]\}.$
\item There exist at least two players $i$ such that $u_i \neq \tilde{u}_i$.
\end{enumerate}
	
\end{lem}

\begin{proof}
	$(\Leftarrow)$ Suppose not, i.e., $V = F^*$. Observe that any $u$ and $\tilde{u}$ on an edge of $F^*$ can have only one player whose payoff differs.
	
	$(\Rightarrow)$ Suppose not. Then, either $V$ does not have an edge or any edge of $V$ does not satisfy Condition 2. In the first case, trivially $V = F^*$. In the second case, there is only one player $i$ such that $u_i \neq \tilde{u}_i$. That is, any edge is parallel to some axis. Since $V$ is convex, this implies that $V$ is an $n$-dimensional cube. In turn, this implies that $V = F^*$, as $V$ contains the best and worst stage payoff profile for each player $i$ (i.e., $u(a)$ for some $a \in \argmax_{a' \in A} u_i (a')$ and $u(a)$ for some $a \in \argmin_{a' \in A} u_i (a')$). A contradiction. 
\end{proof}











\begin{lem}
\label{lem:nn4}
Suppose that there exist $\lambda \in\mathbb{R}^n \setminus \{\mathbf{0}\}$ and two points $u$, $\tilde{u} \in V$ that satisfy Conditions 1 and 2 in \autoref{lem:nn3}. Then, for any $\Delta \in (0,1)$, $W^*_\lambda (\Delta )>\lambda \cdot u$.
\end{lem}
\begin{proof}
Without loss of generality, let $u=(0,\dots,0)$. Note that $\lambda \cdot u = \lambda \cdot \tilde{u} =0$. Let $a, \tilde{a} \in A$ be such that $u(a) = u$ and $u(\tilde{a} ) = \tilde{u}$. If $\tilde{a}$ (resp. $a$) is played when player $n$ is the youngest (resp. not the youngest), players' average payoff vector is $u' \equiv c(\Delta ^{n-1}\tilde{u}_1,\Delta ^{n-2} \tilde{u}_2,\dots,\Delta \tilde{u}_{n-1},\tilde{u}_n)$, where $c \equiv \frac{1}{1+\Delta+\dots +\Delta ^{n-1}}$. Instead, if $a$ (resp. $\tilde{a}$) is played when player $n$ is the youngest (resp. not the youngest), the average payoff vector is $u'' \equiv \tilde{u}-c(\Delta ^{n-1}\tilde{u}_1,\Delta ^{n-2}\tilde{u}_2,\dots, \Delta \tilde{u}_{n-1}, \tilde{u}_n)= \tilde{u}-u'$.


%We claim that either $u'$ \textcolor{red}{or} $u''$ \textcolor{red}{is} not in $V$. \textcolor{red}{Suppose} $u' \in V$. 



Observe that for any $\Delta \in (0,1)$, $u'$ is not on the line segment between $u$ and $\tilde{u}$. This is because $\tilde{u}$ has at least two non-zero components due to Condition 2, and these components are multiplied by $\Delta$ different numbers of times. Thus, $\lambda \cdot u' \neq 0$. Since $\lambda \cdot u'' = \lambda \cdot \tilde{u} - \lambda \cdot u'=- \lambda \cdot u'$, either $\lambda \cdot u' >0$ or $\lambda \cdot u'' >0$. Therefore, $W^*_\lambda (\Delta) \geq \max\{ \lambda \cdot u', \lambda \cdot u'' \}>0.$
%\textcolor{red}{Since $u$ and $\tilde{u}$ are on an edge,} this implies that $\lambda \cdot u' <0$. 
%\textcolor{red}{In turn, this implies that} $\lambda \cdot u'' = \lambda \cdot \tilde{u} - \lambda \cdot u'=- \lambda \cdot u' >0$; hence, $u'' \notin V$. 
\end{proof}








\begin{lem}
\label{lem:nnn5}
When $\mathcal{A}^*_\lambda (\Delta )\subset S^C$ for some $\lambda \in \mathbb{R}^n \setminus \{ \mathbf{0} \}$, the following strict inequality holds for some $m\in \{1,\dots,n-1\}$:\\
$$-\sum_{k=1}^m \Delta^{k-1} w_k (u^{[n]})  + \sum_{k=1}^m \Delta^{k-1} w_{n-m+k} (u^{[n]})<0.$$
\end{lem}
\begin{proof}
We use proof by contradiction. Suppose that the above inequality does not hold for any $m$, i.e., for any $m\in \{1,\dots,n-1\}$,
\begin{equation}
\label{eq:nn11}
\sum_{k=1}^m \Delta^{k-1} w_k (u^{[n]})  - \sum_{k=1}^m \Delta^{k-1} w_{n-m+k} (u^{[n]})=0.
\end{equation}
We claim that the above equalities imply $w_1(u^{[n]})=w_2(u^{[n]})=\cdots =w_n(u^{[n]})$. To show this, let us regard $w_n (u^{[n]})$ as a parameter and the rest $w_1( u^{[n]}), \dots, w_{n-1} (u^{[n]} )$ as unknowns. Clearly, $w_1 (u^{[n]}) = w_n (u^{[n]}), \cdots, w_{n-1} (u^{[n]}) = w_n (u^{[n]})$ is a solution of the system. Next, we show that it is a unique solution. Consider the submatrix consisting of columns 1 to $n-1$ of the coefficient matrix (i.e., the $(n-1)\times n$ matrix in which the $m$-th row corresponds to the coefficients in equation \eqref{eq:nn11} for $m$), which is an $(n-1) \times (n-1)$ matrix. Our purpose is to show that this has the full rank. Let $r_k$ be the $k$-th row of the submatrix, $k=1, \dots, n-1$. To show that this matrix is invertible, we show that $r_1, \dots, r_{n-1}$ are linearly independent or, equivalently, $(\alpha_k)_k \in \mathbb{R}^{n-1}$ satisfies $\sum_{k=1}^{n-1} \alpha_k r_k = \mathbf{0}$ if and only if $\alpha_k = 0$ for all $k$. 
By rearranging each column $l = 1, \dots, n-1$ of $\sum_{k=1}^{n-1} \alpha_k r_k =\mathbf{0}$, we have:
\begin{align*}
(\alpha_1 + \cdots + \alpha_{n-1})  &=0,\\
\Delta (\alpha_2 + \cdots + \alpha_{n-1}) &=\alpha_{n-1},\\
\Delta^2 (\alpha_3 + \cdots + \alpha_{n-1}) &= \alpha_{n-2} + \Delta \alpha_{n-1},\\
\vdots\\
\Delta^{n-2} \alpha_{n-1} &= \alpha_2 + \cdots + \Delta^{n-3}  \alpha_{n-1}.
\end{align*}
Using $\alpha_1 + \alpha_2 + \cdots + \alpha_{n-1} =0$,
\begin{align*}
- \alpha_1 \Delta &= \alpha_{n-1},\\
- (\alpha_1 + \alpha_2) \Delta^2 &= \alpha_{n-2} + \alpha_{n-1} \Delta,\\
\vdots\\
- (\alpha_1 +  \cdots + \alpha_{n-2})\Delta^{n-2} &= \alpha_2 + \alpha_3 \Delta   + \cdots +  \alpha_{n-1}\Delta^{n-3}.
\end{align*}
By substituting each equation into the next, we have:
\begin{align*}
- \alpha_1 \Delta &= \alpha_{n-1},\\
- \alpha_2 \Delta^2 &= \alpha_{n-2},\\
\vdots\\
- \alpha_{n-2} \Delta^{n-2} &= \alpha_2.	
\end{align*}
Then, by comparing the equations with the same variables (for instance, the second with the last), we obtain $\alpha_2 =\alpha_3 = \cdots = \alpha_{n-2}=0$, and using the first equation and $\alpha_1 + \cdots + \alpha_{n-1}=0$, we have $\alpha_1 = \alpha_{n-1}=0$. This concludes the proof of the claim.





By the definition of each $w_k(u^{[n]})$ and the above claim, playing $(u^1,\cdots, u^1)$ yields the same $\lambda$-weighted score as playing $u^{[n]}$. Therefore, $(u^1,\cdots, u^1)\in\mathcal{U}^* _{\lambda}(\Delta )$ must hold, and there exists an action profile $a\in A$ that satisfies $u^1=u(a)$ and $(a,\cdots ,a)\in\mathcal{A}^*_\lambda (\Delta )$. This contradicts $\mathcal{A}^*_\lambda (\Delta )\subset S^C$, completing the proof.
\end{proof}


\begin{proof}[Proof of \autoref{prop:nnn1}] 
Observe that \autoref{lem:nn3} and \autoref{lem:nn4} imply that $V\neq F^*$ if and only if for any $\delta  \in (0,1)$ and $T\in \mathbb{N}$, $F(\delta,T )\neq V$.




Suppose $\delta  \in (0, 1)$, $T\in \mathbb{N}$, and $F^* \neq V$ so that $F(\delta,T )\neq V$. Then, clearly for some $\lambda \neq 0$, $\mathcal{A}^*_\lambda (\Delta )\subset S^C$, where $\Delta=\delta^T$.
Then, by \autoref{lem:nnn5}, the inequality \eqref{eq:2} strictly holds for at least one particular direction $\lambda \in \mathbb{R}^n \setminus \{ \mathbf{0} \}$, and this concludes the proof.
\end{proof}

\subsection{Proof of \autoref{them:4}}

\begin{proof}
Consider an OLG repeated game $OLG (G, \delta, T, \mathbf{M})$. Consider an auxiliary OLG repeated game defined as follows. For each $i \in N$, we divide $i$'th overlap (whose length is $M_i T$) into smaller ``dummy overlaps'' $(i,l_i)$, $l_i = 1, \dots, M_i$, with the same length of $T$. Then, there are $\sum_{i=1}^n M_i$ dummy overlaps. In addition, for some dummy overlap $(i,l_i)$, if no player is born, then we introduce a ``dummy player'' $(i,l_i)$ who is born at the beginning of the dummy overlap $(i,l_i)$. Assume that this dummy player has a single available action, and the payoff function assigns $0$ to any action profile. Let $\tilde{G}$ be the stage game with such dummy players, where the original players' available actions and payoffs are the same as those of $G$. Since in every dummy overlap, either an original or a dummy player is born, this OLG repeated game with the dummy overlaps and players can be regarded as the OLG repeated game $OLG (\tilde{G}, \delta, T)$ in \Cref{sec:2}. Therefore, we can apply our previous results to this auxiliary game. Thus, for each direction $\lambda$, the $\lambda$-weighted welfare is monotonically increasing as $\delta$ decreases and $T$ increases by \autoref{them:2}. Lastly, we observe that the welfare is the same as the one without the dummy players, as their payoffs are assumed to be 0. 
\end{proof}


\end{document}
