\documentclass{FeasibleOLG_main.tex}{subfiles}
\begin{document}

In this section, we illustrate our main results in the previous section using well-known stage games in the repeated game literature.  

\subsection{OLG Prisoners' Dilemma}
\label{subsec:pd}

\begin{figure}[t]
\centering
\begin{tabular}{|c|c|c|}
\hline
    & $C$      & $D$      \\ \hline
$C$ & $1,1$    & $-1,2$ \\ \hline
$D$ & $2,-1$ & $0,0$    \\ \hline
\end{tabular}
\caption{The Prisoners' Dilemma}
\label{fig:1}
\end{figure}
\begin{figure*}
\centering
\includegraphics[width=0.45\textwidth]{fig_pd1.pdf}
\includegraphics[width=0.45\textwidth]{fig_pd2.pdf}
\includegraphics[width=0.45\textwidth]{fig_pd3.pdf}
	\caption{In each figure, the Grey region represents the feasible payoff set (for $\Delta =\frac{2}{3}$, $\Delta = \frac{1}{3}$ and $\Delta \to 0$ clockwise) of the OLG PD game. In each figure, the region surrounded by the dotted lines is the convex hull of the stage game payoffs; the Red region represents the convex hull of the four payoffs, $v(CC, a^2)$, $a^2 \in \{ C,D\}^2$ in \eqref{eq:n2}. Similarly, the Green, Orange, and Blue represent the counterparts for $DC, DD$, and $CD$, respectively.}
	\label{fig:n1}
\end{figure*}

Consider the OLG repeated game $OLG(G,\delta ,T)$, where the stage game $G$ is Prisoners' Dilemma (see \autoref{fig:1}). By \autoref{them:1}, the OLG feasible payoff set is
\begin{equation}
\label{eq:n2}
	F(\delta, T) = co  \left( \bigcup_{ a \in \{CC, DC, DD, CD\}}  \{ v( a, CC), v( a, DC), v(a, DD), v(a, CD)  \}  \right ), 
\end{equation}
where $v (a^1, a^2) = \left (\frac{u_1 (a^1)  + \Delta u_1 (a^2) }{1+ \Delta}, \frac{\Delta u_2 
(a^1)  + u_2 (a^2)}{1 + \Delta } \right)$ for any $(a^1, a^2) \in A^2$ as previous and $\Delta = \delta^T$ (see \autoref{fig:n1}). 




As $\Delta$ becomes smaller (either because players interact for a longer period or they discount more), players put more weight on the payoffs obtained when they are young. Each colored region in the figure corresponding to a certain $a^1 \in A$ represents the payoff vectors that can be achieved from playing $a^1$ over the first overlap and various $a^2 \in \{CC, DC, DD, CD \}$ over the second overlap. The effect of smaller $\Delta$ is represented by each region becoming horizontally narrower as player 1's payoff of the four payoff vectors becomes close to $u_1 (a^1)$, and vertically more spread out as player 2's payoff becomes close to $u_2(a^2)$. Thus in the limit of $\Delta$ converging to $0$, each region becomes a vertical line, and so their convex combination becomes the rectangle. 

It is notable that as $\Delta$ changes the sequences of action profiles that achieve the extreme points of the feasible payoff set may change. For instance, when $\Delta$ is large enough (e.g., $\Delta = 2/3$), $(CC, CC)$ yields an extreme point. Thus no intertemporal trading is needed to achieve the payoff. As $\Delta$ becomes smaller (e.g., $\Delta = 1/3$), $(CC, CC)$ does not anymore generate an extreme point, while $(DC, CD)$ becomes a new sequence corresponding to one of the extreme points. Intuitively as $\Delta$ becomes smaller, the benefit of intertemporal trading such as $(DC,CD)$ becomes larger. Note that for sufficiently small $\Delta$, each player should play the action profile that maximizes her/his payoff in order to be on the efficient frontier. 




\subsection{A 3-player Example from \cite{FM_1986_ECMA}}
\label{subsec:fm}
Consider another example involving three players. The stage game is described in \autoref{fig:2}. The feasible payoff set of the standard infinitely repeated game with this stage game is equal to $V$, which is the line segment between $(0,0,0)$ and $(1,1,1)$. Note that the dimension of $V$ (i.e., 1) is less than the number of players (i.e., 3).\footnote{\cite{FM_1986_ECMA} use this stage game to show that a folk theorem may fail for standard repeated games with infinitely-lived players if the stage game does not satisfy the full dimensionality, which is a sufficient condition of their folk theorem. \cite{Smith_1992_GEB} shows that the full-dimensionality is not necessary for his folk theorem for OLG repeated games. Since his folk theorem first chooses $T$ and then chooses sufficiently large $\delta$, it concerns the case when $\delta^T$ is close to 1. The feasible payoff set in this case is the ``smallest'' according to our characterization, which is the line segment between $(0,0,0)$ and $(1,1,1)$.}
\begin{figure}
\centering
\begin{tabular}{|c|c|c|}
\hline
    & $A$     & $B$     \\ \hline
$A$ & $1,1,1$ & $0,0,0$ \\ \hline
$B$ & $0,0,0$ & $0,0,0$ \\ \hline
\end{tabular}
\quad 
\begin{tabular}{|c|c|c|}
\hline
    & $A$     & $B$     \\ \hline
$A$ & $0,0,0$ & $0,0,0$ \\ \hline
$B$ & $0,0,0$ & $1,1,1$ \\ \hline
\end{tabular}
\caption{The stage game of a 3-player pure coordination game}	
\label{fig:2}
\end{figure}

On the other hand, the feasible payoff set of the OLG repeated game features the full dimensionality (i.e., it is equal to the number of players) for any $\Delta \in (0,1)$. When $\Delta = 1$, it is the line segment between $(0,0,0)$ and $(1,1,1)$, which coincides with the convex hull of the stage game payoffs. On the other hand, when $\Delta \in (0,1)$, it is a polytope with a nonempty interior. 
\begin{figure}[t]
\centering
\includegraphics[width=0.5\textwidth]{fig_fm.pdf}
\caption{The OLG feasible payoff set of the pure coordination game when $\Delta = 1$ (Black), $\Delta = 2/3$ (Dark Orange), $\Delta = 1/2$ (Light Orange), $\Delta = 1/3$ (Yellow)} and $\Delta= 0$ (Gray).
\label{}
\end{figure}


\end{document}
