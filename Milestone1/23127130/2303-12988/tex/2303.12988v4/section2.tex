\documentclass{FeasibleOLG_main.tex}{subfiles}

\begin{document}


\subsection{Stage Game}
A stage game is defined as a triple $G=(N,(A_i)_i,(u_i)_i)$, where $N=\{ 1,2,\cdots ,n\}$, for some $n\geq 2$, is the set of players, $A_i$ is a finite set of pure actions available to player $i$, and  $u_i: \prod_i A_i \to \mathbb{R}$ is player $i$'s one-shot payoff function. Let $A \equiv \prod_{i \in N} A_i$ be the set of action profiles. 

Given a set $B \subseteq \mathbb{R}^N$, let $co (B)$ be the convex hull of $B$. Let $V \subseteq \mathbb{R}^n$ be the feasible set of one-shot payoffs, defined as 
$$V \equiv co \left(  \{ u(a) : a \in A\} \right).$$
Let $F^*\equiv \prod _{i\in N} \left [\min _{a\in A}u_i(a),\ \max _{a\in A}u_i(a) \right].$
That is, $F^*$ is the smallest (multidimensional) cube which contains the one-shot feasible set $V$. 


\subsection{OLG Repeated Game}
Given a stage game $G$ defined above, $\delta \in (0,1]$ and $T \in \mathbb{N}$, we define the \emph{OLG repeated game} $OLG(G,\delta ,T)$ as follows:\footnote{
We generalize the model and extend the main results to some extent in \Cref{sec:6}.}
\begin{itemize}
	\item In every period $t \in \mathbb{N}$, $G$ is played by $n$ finitely-lived players.
	\item For $i\in N$ and $d \in \mathbb{N}$, the player with $A_i$ in generation $d$ joins in the game at the beginning of period $(d-1)nT+(i-1)T+1$, and lives for the following $nT$ periods until he retires at the end of period $dnT+(i-1)T$. The only exceptions are the players with $A_i$ for $i\in N\setminus\{ 1\}$ in generation 0, who participate in the game between periods 1 and $(i-1)T$ (see \autoref{o1}).\footnote{Such an overlapping structure can be found, for instance, in an organization with a fixed retirement age. Employees of the organization work with other employees of different ages. Employees join the organization when they are young. However, at the onset of the organization, some employees are old (generation 0).}
	\item Each player's per-period payoffs are discounted at a common discount factor $\delta$. 
\end{itemize}
\begin{figure}[t]
\centering
\scalebox{0.7}{
\begin{tabular}{|c|c|c|c|c|c|c|c|c|}\hline
Period&$1\sim T$&$T+1\sim 2T$&$2T+1\sim 3T$&$3T+1\sim 4T$&$4T+1\sim 5T$&$5T+1\sim 6T$&$6T+1\sim 7T$&$\cdots$\\\hline
$A_1$&\multicolumn{3}{c|}{Generation 1}&\multicolumn{3}{c|}{Generation 2}&\multicolumn{2}{c|}{Generation 3 $\cdots$}\\\hline
$A_2$&Generation 0&\multicolumn{3}{c|}{Generation 1}&\multicolumn{3}{c|}{Generation 2}&$\cdots$\\\hline
$A_3$&\multicolumn{2}{c|}{Generation 0}&\multicolumn{3}{c|}{Generation 1}&\multicolumn{3}{c|}{Generation 2 $\cdots$}\\\hline
\end{tabular}
}
\caption{Structure of OLG repeated game with $n=3$}
\label{o1}
\end{figure}

Note that a player interacts with the same opponents (with unchanged generation) for $T$ periods. We refer to such $T$ periods as an \emph{overlap}. For each $i \in N$, we refer to any player whose action set is $A_i$ as ``player $i$." Whenever necessary, we explicitly mention players' generation.

 



When a sequence of actions $(a (t)) _{t=1}^{nT}\in A^{nT}$ is played throughout a player's life with $A_i$, her/his average payoff is as follows:$\footnote{For the player with $A_i$ for $i\in N\setminus\{ 1\}$ in generation 0, replace $nT$ with $(i-1)T$.}$
\begin{eqnarray}
\frac{1}{\sum _{t=1}^{nT}\delta^{t-1}}\sum _{t=1}^{nT}\delta ^{t-1}u_i(a (t)).
\nonumber\end{eqnarray}



We maintain the following assumption throughout this paper.\footnote{This assumption is employed also by \cite{Chen_2007_EL} and \cite{CF_2013_IJGT}.} 
\begin{ass}
\label{ass}
Players can access a Public Randomization Device (henceforth PRD) at the beginning of every period. Players observe the realizations of PRDs after their birth. 
\end{ass}























\end{document}
