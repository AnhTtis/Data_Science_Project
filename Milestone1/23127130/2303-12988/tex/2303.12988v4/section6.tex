\documentclass{FeasibleOLG_main.tex}{subfiles}
\begin{document}


In this section we generalize the model in \Cref{sec:2} and extend our main results in \Cref{sec:4} to some extent.


Let $G = (N,(A_i)_i,(u_i)_i)$ be a stage game, $\delta \in [0,1]$ and $T \in \mathbb{N}$. Also let $\mathbf{M}=(M_1,M_2,\cdots ,M_n)\in\mathbb{N}^n$ be an \emph{overlapping structure}. An \emph{OLG repeated game with overlapping structure $\mathbf{M}$}, $OLG(G,\delta,T,\mathbf{M})$, is defined as follows:\footnote{\cite{Kandori_1992_RES} studies this model and provides a folk theorem.} 
\begin{itemize}
	\item 
For $i\in N$ and $d\in\mathbb{N}$, the player with $A_i$ in generation $d$ joins in the game at the beginning of period $(d-1)\overline{M}_nT+\overline{M}_{i-1}T+1$, and lives for the following $\overline{M}_n T$ periods, until he retires at the end of period $d\overline{M}_nT+\overline{M}_{i-1}T$, where $\overline{M}_0=0$ and $\overline{M}_i=\sum_{j=1}^iM_j$ for $i\in N$. 
\item The only exceptions are the players with $A_i$ for $i\in N\setminus\{ 1\}$ in generation 0, who participate in the game between periods 1 and $\overline{M}_{i-1}T$.
\end{itemize}

\begin{figure}[t]
\centering
\scalebox{0.5}{
\begin{tabular}{|c|c|c|c|c|c|c|c|c|}\hline
Period&$1\sim \overline{M}_1T$&$\overline{M}_1T+1\sim \overline{M}_2T$&$\overline{M}_2T+1\sim \overline{M}_3T$&$\overline{M}_3T+1\sim \overline{M}_3T+\overline{M}_1T$&$\overline{M}_3T+\overline{M}_1T+1\sim \overline{M}_3T+\overline{M}_2T$&$\overline{M}_3T+\overline{M}_2T+1\sim 2\overline{M}_3T$&$2\overline{M}_3T+1\sim 2\overline{M}_3T+\overline{M}_1T$&$\cdots$\\\hline
$A_1$&\multicolumn{3}{c|}{Generation 1}&\multicolumn{3}{c|}{Generation 2}&\multicolumn{2}{c|}{Generation 3 $\cdots$}\\\hline
$A_2$&Generation 0&\multicolumn{3}{c|}{Generation 1}&\multicolumn{3}{c|}{Generation 2}&$\cdots$\\\hline
$A_3$&\multicolumn{2}{c|}{Generation 0}&\multicolumn{3}{c|}{Generation 1}&\multicolumn{3}{c|}{Generation 2 $\cdots$}\\\hline
\end{tabular}
}
\caption{Structure of generalized OLG repeated game with $n=3$}
\label{aaaa}
\end{figure}




Note that for each $i \in \{ 1, \dots, n-1\}$, player $(i +1) $ is born later than player $i$ by $M_i \times T$ periods. Let us call such $M_i \times T$ periods an ``overlap.'' Thus in this extension, overlaps may have different lengths. Note that the model in \Cref{sec:2} is the special case of this generalized model when $\mathbf{M}= (1, \dots, 1) \in \mathbb{R}^n$.\footnote{The generalized model is, however, not fully general. For example, suppose that the stage game has three players. We can consider an OLG repeated game where player 1 in each generation is born in an odd period and dies in the next period, whereas player 2 and 3 in each generation together are born in an even period and die in the next period. This model belongs to one structure of OLG repeated games. However, the OLG model in the present paper does not include such a model.}





We define the feasible payoff set of $OLG (G, \delta, T,\mathbf{M})$ by 
$$
F(\delta,T, \mathbf{M}):=co \left(\{ U^{\mathbf{M}}(a^{[\overline{M}_nT]}):a^{[\overline{M}_nT]}\in A^{\overline{M}_nT}\} \right),
$$
where $U^{\mathbf{M}}: A^{\overline{M}_nT} \to \mathbb{R}^n$ is defined similarly as in \eqref{eq:nnnn1}. 

Recall that a sequence of action profiles is stable if players play the same action profile during each overlap $k=1, \dots, n$. So such a sequence can be identified with some $a^{[n]} \in A^n$. We define player $i$'s average discounted payoff from a stable sequence of action profiles by
$$
v_i^{\mathbf{M}}(a^{[n]}):=\frac{1}{\sum _{k=1}^{\overline{M}_n}\Delta ^{k-1}}
\left(\sum _{k=1}^n\Delta ^{\overline{M}_{k-1}} \left( \sum_{s=1}^{M_k}\Delta^{s-1}u_i(a^{i+k-1}) \right ) \right), \quad \forall a^{[n]} \in A^n. \label{eq:nnn2}
$$
Note that when $\mathbf{M} = (1, \dots, 1)$, $F(\delta, T, \mathbf{M})$ and $v_i^{M}$ reduce to $F(\delta, T)$ and $v_i$ in \Cref{sec:3}, respectively. We obtain the following theorems extending the previous one.




\begin{them}
	 For any $\delta \in (0,1]$, $T \in \mathbb{N}$ and $\mathbf{M}\in\mathbb{N}^n$, 
$$
F(\delta,T,\mathbf{M}) =co\left( \left \{v^{\mathbf{M}}(a^{[n]}): {a^{[n]}\in A^n} \right \} \right).
$$
\end{them}


That is, the feasible payoff set for the extended model is characterized similarly to the previous one, i.e., it is sufficient to consider legnth-$n$ sequence of action profiles regardless of $\delta$ and $T$, and now also regardless of overlapping structure $\mathbf{M}$. The idea behind the result is the same as \autoref{them:1}, so we omit the proof. 



\begin{them}
\label{them:4}
 The following hold for any fixed $\mathbf{M} \in \mathbb{N}^n$:
\begin{enumerate}
	\item Given any $T \in \mathbb{N}$, $F(\delta ',T, \mathbf{M})\subseteq F(\delta ,T,\mathbf{M})$ for any $\delta, \delta'$ with $0< \delta <\delta ' \leq 1$.
\item Given any $\delta \in (0,1]$, $F(\delta ,T,\mathbf{M})\subseteq F(\delta ,T+1,\mathbf{M})$ for any $T \in \mathbb{N}$.	
\end{enumerate}
	
\end{them}
That is, the monotonicity of the feasible payoff set holds for general overlapping structures. 



The basic idea of the proof is to regard the extended model as the previous model in \Cref{sec:2} with some ``dummies.'' For instance, consider an OLG repeated game $OLG (G, \delta, T=1, \mathbf{M}=(1,3))$, where $G$ is a stage game with two player $i=1,2$ for some payoff function $u_1$ and $u_2$ (see \autoref{fig:aaaa1}). 
\begin{figure}[t]
\center
\scalebox{1.0}{
\begin{tabular}{|c|c|c|c|c|c|c|c|c|c|c|c|c|c|}\hline
Period&1&2&3&4&5&6&7&8&9&10&11&12&$\cdots$\\\hline
$A_1$&\multicolumn{4}{c|}{Generation 1}&\multicolumn{4}{c|}{Generation 2}&\multicolumn{4}{c|}{Generation 3}&$\cdots$\\\hline
$A_2$&0&\multicolumn{4}{c|}{Generation 1}&\multicolumn{4}{c|}{Generation 2}&\multicolumn{4}{c|}{Generation 3 $\ \cdots$}\\\hline
\end{tabular}
}
\caption{$OLG (G, \delta, T=1, \mathbf{M}=(1,3))$}
\label{fig:aaaa1}
\end{figure}
Now let $\tilde{G}$ be a stage game in which there are two dummy players, player 3 and 4, who have a singleton action set and whose payoffs are assumed to be constantly 0, in addition to player 1 and 2 (player 1 and 2's payoffs in $\tilde{G}$ are the same as in $G$). Then consider an auxiliary OLG repeated game $OLG (\tilde{G}, \delta, T=1, \tilde{\mathbf{M}} = (1,1,1,1))$, where player $i \in \{1,2,3,4\}$ in generation 1 is born at $t=i$ (see \autoref{fig:aaaa2}). Note that this game has the OLG structure we studied in the previous sections. Since player 3 and 4 have the payoffs 0 always, their payoffs do not contribute to the calculation of the $\lambda$-weighted welfare in \eqref{eq:nn2}.

\begin{figure}[t]
\center
\scalebox{1.0}{
\begin{tabular}{|c|c|c|c|c|c|c|c|c|c|c|c|c|c|}\hline
Period&1&2&3&4&5&6&7&8&9&10&11&12&$\cdots$\\\hline
$A_1$&\multicolumn{4}{c|}{Generation 1}&\multicolumn{4}{c|}{Generation 2}&\multicolumn{4}{c|}{Generation 3}&$\cdots$\\\hline
$A_2$&0&\multicolumn{4}{c|}{Generation 1}&\multicolumn{4}{c|}{Generation 2}&\multicolumn{4}{c|}{Generation 3 $\ \cdots$}\\\hline
$\tilde{A}_3$&\multicolumn{2}{c|}{0}&\multicolumn{4}{c|}{Generation 1}&\multicolumn{4}{c|}{Generation 2}&\multicolumn{3}{c|}{Generation 3 $\ \cdots$}\\\hline
$\tilde{A}_4$&\multicolumn{3}{c|}{0}&\multicolumn{4}{c|}{Generation 1}&\multicolumn{4}{c|}{Generation 2}&\multicolumn{2}{c|}{Generation 3 $\ \cdots$}\\\hline
\end{tabular}
}
\caption{$OLG (\tilde{G}, \delta, T=1, \tilde{\mathbf{M}} = (1,1,1,1))$}
\label{fig:aaaa2}
\end{figure}













\end{document}
