\documentclass[a4paper,12pt]{article}
\usepackage[top=1in,bottom=1in,left=1in,right=1in]{geometry}


\usepackage{amssymb}
\usepackage{amsthm}
\usepackage{amsmath}
\usepackage{color}
\usepackage{graphicx}
\usepackage{hyperref}
\usepackage{cleveref}
\usepackage{natbib}
\usepackage{tikz}
\usepackage{setspace}
\usepackage{subfiles}
\usepackage{comment}

				
\DeclareMathOperator*{\argmax}{arg\,max}
\DeclareMathOperator*{\argmin}{arg\,min}


\theoremstyle{definition}
\newtheorem{defi}{Definition}
\providecommand*{\defiautorefname}{Definition}
\newtheorem{exmp}{Example}
\providecommand*{\exmpautorefname}{Example}

\newtheorem{conj}{Conjecture}
\newtheorem{lem}{Lemma}
\providecommand*{\lemautorefname}{Lemma}
\newtheorem{claim}{Claim}
\providecommand*{\claimautorefname}{Claim}
\newtheorem{them}{Theorem}
\providecommand*{\themautorefname}{Theorem}
\providecommand*{\figureautorefname}{Figure}

%\theoremstyle{plain}
\newtheorem{coro}{Corollary}
\providecommand*{\coroautorefname}{Corollary}
\newtheorem{prop}{Proposition}
\providecommand*{\propautorefname}{Proposition}
\newtheorem{ass}{Assumption}
\providecommand*{\assautorefname}{Assumption}
\newtheorem{step}{Step}

\newtheorem{lemadd}{LemAdd}
\providecommand*{\lemaddautorefname}{LemAdd}

\title{Characterizing the Feasible Payoff Set of OLG Repeated Games\thanks{We thank two anonymous referees and the coeditor for helpful comments and discussions. We are grateful to seminar and conference participants at 2023 Asian Meeting of the Econometric Society (Beijing), 2023 Africa Meeting of the Econometric Society (Nairobi) and CIRJE Microeconomics Workshop at Tokyo University. We also thank Michihiro Kandori and Akihiko Matsui for their helpful suggestions. Chihiro Morooka was supported by the Japan Society for the Promotion of Science (JSPS) KAKENHI Grant Number JP22K13360. All remaining errors are ours.}}
\author{Daehyun Kim\thanks{Division of Humanities and Social Sciences, POSTECH. Email: \href{mailto:dkim85@outlook.com}{dkim85@outlook.com}}\and Chihiro Morooka\thanks{School of Science and Engineering, Tokyo Denki University. Email: \href{mailto:c-morooka@mail.dendai.ac.jp}{c-morooka@mail.dendai.ac.jp}}}
\date{July 14, 2024}

\begin{document}
\maketitle
\begin{abstract}
We study the set of (stationary) feasible payoffs of OLG repeated games that can be achieved by action sequences in which every generation of players plays the same sequence of action profiles. Our first main result completely characterizes the set of feasible payoffs given any fixed discount factor of players and the length of interaction. We can use this result to obtain the feasible payoff set in closed form. Second, we provide novel comparative statics of the feasible payoff set with respect to the discount factor and the length of interaction. Perhaps interestingly, the feasible payoff set becomes \emph{smaller} as players' discount factor becomes larger. In addition, we identify a necessary and sufficient condition for this monotonicity to be strict. 
\end{abstract}


\textbf{Keywords:} Overlapping generation, repeated games

\strut

\textbf{JEL Classification Numbers:} C72, C73 


\onehalfspacing

\newpage

\renewcommand{\arraystretch}{1.3}

\section{Introduction}
\subfile{section1}
\label{sec:1}

\section{Model}
\subfile{section2}
\label{sec:2}

\section{A Complete Characterization of the Feasible Payoff Set}
\label{sec:3}
\subfile{section3.tex}


\section{Comparative Statics of the Feasible Payoff Set}
\label{sec:4}
\subfile{section4.tex}


\section{Examples}
\label{sec:5}
\subfile{section5}


\section{Extension}
\label{sec:6}
\subfile{section6}


\section{Discussions}
\label{sec:7}
\subfile{section7}



\appendix
\section{Omitted Proofs}
\label{sec:app}
\subfile{section_app}



\bibliography{FeasibleOLG_bib}
\bibliographystyle{chicago}


\end{document}
