\documentclass{FeasibleOLG_main.tex}{subfiles}

\begin{document}

In this section, we study comparative statics of the feasible set w.r.t. both $T$ and $\delta$. From \autoref{them:1} in the previous section, we know that the feasible payoff set depends on $\delta$ and $T$ only through $\delta^T$. 



\subsection{Monotonicity}
Our main result in this section asserts that the feasible payoff set is non-increasing in $\delta^T$, implying, perhaps surprisingly, it is \emph{non-increasing} in $\delta$.





\begin{them}
\label{them:2}
The following hold:
\begin{enumerate}
	\item Given any $T \in \mathbb{N}$, $F(\delta ',T)\subseteq F(\delta ,T)$ For any $\delta, \delta'$ with $0< \delta <\delta ' \leq 1$. 
\item Given any $\delta \in (0,1]$, $F(\delta , T)\subseteq F(\delta ,T+1)$ for any $T \in \mathbb{N}$.	
\end{enumerate} \end{them}






Payoffs outside the convex hull of the stage game (i.e., $V$) are generated by trading payoffs across different generations. Such intertemporal trading can be beneficial, because players have different ages, so they weigh the payoffs differently even if their $\delta$ is the same. This means that one player's loss can differ from another's gain. A larger $T$ allows more interaction and more trading opportunities among the players. On the other hand, when $\delta$ increases with fixed $T$, all players put similar weights on all periods, so a player's gain is approximately another player's loss. Hence the feasible set shrinks as $\delta$ increases to 1 with $T$ fixed. 



 




The rest of this section is devoted to proving \autoref{them:2}.\footnote{The weak set-inclusions in the result cannot be strengthened to be strict. For instance, if the stage game payoff set is an $n$-dimensional rectangle, the feasible payoff set of the corresponding OLG repeated game would be unchanged with respect to the parameters.} In doing so, we shall also provide a characterization of players' optimal play to maximize the welfare given some weights for players, which might have some independent interest. 

\autoref{them:1} implies that it is without loss of generality to consider a sequence of $n$ pure action profiles for studying the monotonicity. That is, players play the same action profile during an overlap which consists of $T$ periods. In addition, since the feasible payoff set is convex, it is enough to show the maximum ``score'' increases as $\delta$ (resp. $T$) becomes smaller (resp. larger) for each non-zero direction. 

Fix $\lambda \in \mathbb{R}^n \setminus \{ \mathbf{0}\}$. For a given $\delta \in (0,1]$ and $T \in \mathbb{N}$, let $\Delta \equiv  \Delta (\delta, T) =  \delta^T$.  Define $\lambda$-weighted welfare as
\begin{equation}
\label{eq:nn2}
W_{ \lambda}^*(\Delta) :=\max_{ a^{[n]} =(a^1, \dots, a^n) \in A^n} W_\lambda (a^{[n]},\Delta),
\end{equation}
where
$$W_\lambda (a^{[n]}, \Delta ) := \sum_{i=1}^n \lambda_i v_i (a^1, \dots, a^n), \quad \forall a^{[n]} \in A^n$$
and $v_i$ is defined in \eqref{eq:nnn1}.\footnote{$W_\lambda$ becomes a function of $\Delta$ because $v_i$ depends on $\delta$ and $T$ only through $\Delta$.}

We want to show that $W_{ \lambda}^* (\Delta)$ is non-increasing in $\Delta$ (as a result, non-decreasing in $T$ and non-increasing in $\delta$). 







We introduce a few more notations. Denote the set of optimal solutions of \eqref{eq:nn2} by $\mathcal{A}^*_\lambda (\Delta) \subseteq A^n$. Let $\mathcal{U}^*_\lambda (\Delta) \equiv \left \{ (u^k)_k : u^k = u (a^k), a^{[n]} \in \mathcal{A}_\lambda^* (\Delta) \right \} \subseteq \mathbb{R}^{n\times n}$. That is, the set of sequences of the optimal payoff vectors. Notice that $\mathcal{U}^*_\lambda (\Delta)$ may be a singleton even when $\mathcal{A}^*_\lambda (\Delta)$ is not.  Lastly, given $u^{[n]} = (u^1, \dots, u^n) \in \mathbb{R}^{n \times n}$, for each $k \in \{ 1, \dots, n\}$, let
$$w_k (u^{[n]}) := \sum_{ i =1}^n \lambda_i u_i^{i+k-1},$$
where $u_i^m=u_i^{m-n}$ if $m>n$. That is, $w_k (u^{[n]})$ is the weighted sum of players' payoffs when their ``age'' is $k$ (i.e., they are in the $k$-th overlap in their lifetime). For instance, given $u^{[3]} = (u^1, u^2, u^3) \in \mathbb{R}^{3 \times 3}$ and $\lambda = (1,1,1)$, $w_1 (u^{[3]}) = u^1_1  + u^2_2   + u^3_3 $, $w_2 (u^{[3]}) = u^2_1  + u^3_2  + u^1_3  $, $w_3 (u^{[3]}) =u^3_1   + u^1_2  + u^2_3 $. 







Observe that for each $u^{ [n]}=(u^1,\dots, u^n) \in \mathcal{U}^*_\lambda (\Delta)$, 
\begin{equation}
\label{eq:nn3}
W_{ \lambda}^* (\Delta) = \frac{\sum_{k=1}^n \Delta^{k-1} w_k (u^{[n]}) }{\sum_{k=1}^n \Delta^{k-1}}.
\end{equation}

The following lemma is the key to proving the monotonicity. 
\begin{lem}
\label{lem:1}
Let $\Delta \in (0,1]$ and $u^{[n]}\in \mathcal{U}^*_\lambda (\Delta)$. For each $m =1, \dots, n-1$, 
\begin{equation}
\label{eq:2}
\sum_{k=1}^m \Delta^{k-1} w_k (u^{[n]}) \geq  \sum_{k=1}^m \Delta^{k-1} w_{n-m+k} (u^{[n]}).
\end{equation}
\end{lem}





In words, the lemma means, at optimum, the payoffs in the earlier stages of a player are higher than those of later stages (in the sense that for any $m \leq n-1$, the first $m$ payoffs should be larger than the last $m$ payoffs). When there are only two players, this reduces to the condition that the payoffs when they are ``young'' must be larger than those when they are ``old'' for optimality. When there are more than two players, it is \emph{a priori} not clear what could be the corresponding expression. According to the lemma, for the case of three players, it is $w_1 (u^{[3]}) \geq w_3 (u^{[3]}) $ and $w_1 (u^{[3]})   + \Delta w_2 (u^{[3]})  \geq w_2 (u^{[3]})  + \Delta w_3 (u^{[3]}) $. 


Let us explain the crux of the idea of the proof with three players and $\lambda = (1,1,1)$. Assume $u^{[3]} = (u^1, u^2, u^3)$ be an optimal sequence of payoff vectors. From optimality of $u^1$ at $k=1$, it should be $u^1_1 + \Delta^2 u^1_2 + \Delta u^1_3 \geq u^2_1 + \Delta^2 u^2_2+ \Delta u^2_3$. By multiplying both sides by $\Delta$, 
		$$\Delta u^1_1 + \Delta^3 u^1_2 + \Delta^2 u^1_3 \geq \Delta u^2_1 + \Delta^3 u^2_2 + \Delta^2 u^2_3.$$
On the other hand, by optimality of $u^2$ at overlap $k=2$,
		$$\Delta u^2_1 + u^2_2 + \Delta^2 u^2_3\geq \Delta u^1_1 + u^1_2 + \Delta^2 u^1_3.$$
		From the two inequalities, we can conclude $u^2_2 \geq u^1_2$. Intuitively, since the same generation of player 1 and 3 is active both at $k=1$ and $k=2$, while player 2 is replaced by the next generation, in order for $u^2$ to give a larger aggregate payoff at $k=2$, player 2 should have a larger payoff at $u^2$ than at $u^1$. A symmetric argument results in $u^3_3 \geq u^2_3$ and $u^1_1 \geq u^3_1$, and summing them up results in $w_1 (u^{[3]}) \geq w_3 (u^{[3]}) $. Applying a similar argument to ``two-overlap apart,'' we have $\Delta u_2^1 + u_3^1 \leq \Delta u_2^3 + u_3^3$ (as a result, $w_1 (u^{[3]})   + \Delta w_2 (u^{[3]})  \geq w_2 (u^{[3]})  + \Delta w_3 (u^{[3]})$).














Perhaps surprisingly, the following lemma says that the inequality \eqref{eq:2} in \autoref{lem:1} is sufficient to prove that the derivative of the aggregate payoff is non-positive. 
\begin{lem}
\label{lem:2}
For $\Delta \in (0,1)$,
\begin{equation}
\label{eq:nnnn7}
\frac{\partial W_{ \lambda} }{\partial \Delta }(a^{[n]},\Delta)  \leq 0
\end{equation}
for any $a^{[n]}  \in \mathcal{A}_\lambda^* (\Delta)$.
\end{lem}

In the proof of this lemma, we show that the numerator of the derivative in \eqref{eq:nnnn7} is a weighted sum of $m-1$ terms each of which is $(RHS-LHS)$ for $m=1, \dots, n-1$ in \autoref{lem:1}. This immediately implies the inequality in \eqref{eq:nnnn7}.








\begin{proof}[Proof of \autoref{them:2}]




We first observe that $W_\lambda^*(\Delta)$ is continuous in $\Delta$ everywhere because it is the maximum of continuous functions. 

In addition, we argue that $W_\lambda^*(\Delta)$ is differentiable at all but finite $\Delta$. To see this, observe that for any $a^{[n]}, \tilde{a}^{[n]} \in A^n$ with $\lambda \cdot u(a^k) \neq \lambda \cdot u(\tilde{a}^k)$ for some $k \in \{1,\dots, n\}$, the number of solutions of the equation $W_\lambda (a^{[n]}, \Delta) - W_\lambda (\tilde{a}^{[n]}, \Delta)=0$ is at most $n-1$, because it is a polynomial equation of degree $(n-1)$. Thus, the total number of intersections of $W_\lambda ( \cdot, \Delta)$ which any pair of action sequences can make is at most $\frac{|A^n||A^n -1|}{2} (n-1)$. This implies that we can find a partition $\{ (\underline{b}_l,\overline{b}_l]: l = 1,\dots, L \}$ of $(0,1]$, where $L \in \mathbb{N}$, such that for each $l = 1, \dots, L$, there is a unique payoff sequence $u_l^{[n]} = (u(a_l^k))_{k=1}^n$ corresponding to a solution $a_l^{[n]} =(a_l^k)_{k=1}^n$ which is optimal for any $\Delta \in (\underline{b}_l,\overline{b}_l]$. Hence, $W_\lambda^* (\Delta)$ is differentiable in the interior of $(\underline{b}_l,\overline{b}_l]$.\footnote{For $\Delta$ on the boundary, the derivative may not exist. For instance, in the OLG repeated game with the Prisoners' Dilemma depicted in \autoref{fig:1}, for large $\Delta$ and $T=1$, when the direction is $\lambda = (1,1)$, $(CC, CC)$ is optimal, while for some small $\Delta$, the optimal sequence is $(DC,CD)$. There is a cutoff $\Delta$ for this change, and neither $(CC,CC)$ or $(DC,CD)$ does not satisfy the description above: $(CC, CC)$ (or the corresponding payoff sequence) remains optimal for $(\Delta, \Delta + \epsilon)$, while $(DC, CD)$ remains optimal for $(\Delta - \epsilon, \Delta)$ for some small $\epsilon>0$.
 } And the derivative is non-positive by \autoref{lem:2}, i.e., 
\begin{equation*}
\label{eq:nn6}
\frac{d W^*_\lambda (\Delta) }{d\Delta } = \frac{\partial W_{ \lambda} }{\partial \Delta }(a_l^{[n]},\Delta) \leq 0.
\end{equation*}
Together with the continuity of $W^*_\lambda (\Delta)$, this implies the monotonicity. 
\end{proof}





\subsection{A Limit Characterization of the Feasible Payoff Set}

For the asymptotic cases (i.e., $\delta^T \nearrow 1$ or $\delta^T \searrow 0$), we can be more explicit about the shape of the feasible payoff set. 

When the players' effective discount factor $\delta^T$ is close to 1, there is not much difference in discounting of young and old players. So, the scope of intertemporal trade of payoffs is little and what is the best is to maximize the stage game payoffs for a given welfare weight. On the other hand, when the effective discount factor is close to 0, the difference in discounting of young and old players is large. Hence intertemporal trading can be very helpful. The extreme form of such trading is to maximize the youngest player's (weighted) payoff. We summarize this discussion as a corollary:

\begin{coro}
The following hold:
\begin{enumerate}
\item For $\delta^T$ sufficiently close to 1, the solution of the optimization problem \eqref{eq:nn2} is to play some $a^k \in \argmax_{a' \in A} \lambda \cdot u(a')$ for each overlap $k$. As a result, $\lim_{ \delta^T \nearrow 1} F(\delta, T) = V.$
	\item For $\delta^T$ sufficiently close to 0, the solution of the optimization problem \eqref{eq:nn2} is to play $a^k \in \argmax_{a' \in A} \lambda_{i_k} u_{i_k} (a')$, where $i_k \in N$ is the youngest player in overlap $k$. As a result, $\lim _{\delta^T\searrow 0}F(\delta, T)=  \prod _{i\in N} \left [\min _{a\in A}u_i(a),\ \max _{a\in A}u_i(a) \right]$.	 	
\end{enumerate}

\end{coro}

Thus, for any $\delta \in (0,1)$, as $T \to \infty$, the feasible set converges to the $n$-dimensional cube. Similarly, for any $T$, as $\delta \to 0$, the feasible set converges to the $n$-dimensional cube. 


   




\subsection{Strict Monotonicity}


In this subsection, we identify a necessary and sufficient condition for the strict expansion of the feasible set as $\Delta \equiv  \Delta (\delta, T) = \delta^T$ becomes smaller. 


Denote the OLG feasible set by $F(\Delta)  \equiv  F(\delta ,T)$. We show that, when $V\neq F^*$, the OLG feasible set satisfies the following strict monotonicity:
\begin{prop}
\label{prop:nnn1}
Suppose that $V\neq F^*$ holds. Then $F(\Delta ')\subsetneq F(\Delta)$ holds for $\Delta <\Delta '$. Conversely, if $V = F^*$, then $F(\Delta)=V$ for any $\Delta$.
\end{prop}

That is, unless the one-shot feasible set is already a (multidimensional) cube, the OLG feasible set satisfies the strict monotonicity with respect to $\Delta$.


Intuitively, if $\Delta <1$ and $V \neq F^*$, players can find some opportunity of intertemporally trading payoffs (i.e., playing a non-constant action profile sequence) so that they can achieve payoffs beyond $V$ toward some direction $\lambda$. It turns out that the benefit from the most efficient trading (i.e., that achieves $W^*_\lambda (\Delta)$) is strictly larger with smaller $\Delta$.








\end{document}
