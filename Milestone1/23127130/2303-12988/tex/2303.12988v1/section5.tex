\documentclass{FeasibleOLG_main.tex}{subfiles}
\begin{document}

\subsection{OLG Prisoners' Dilemma}
\label{subsec:pd}

\begin{figure}[t]
\centering
\begin{tabular}{|c|c|c|}
\hline
    & $C$      & $D$      \\ \hline
$C$ & $1,1$    & $-1,2$ \\ \hline
$D$ & $2,-1$ & $0,0$    \\ \hline
\end{tabular}
\caption{The Prisoners' Dilemma}
\label{fig:1}
\end{figure}
\begin{figure*}
\centering
\includegraphics[width=0.45\textwidth]{fig_pd1.pdf}
\includegraphics[width=0.45\textwidth]{fig_pd2.pdf}
\includegraphics[width=0.45\textwidth]{fig_pd3.pdf}
	\caption{In each figure, the Grey region represents the feasible payoff set (for $\Delta =\frac{2}{3}$, $\Delta = \frac{1}{3}$ and $\Delta \to 0$ clockwise) of the OLG PD game. In each figure, the region surrounded by the dotted lines is the convex hull of the stage game payoffs; the Red region represents the convex hull of the four payoffs, $v(CC, a^2)$, $a^2 \in \{ C,D\}^2$ in \eqref{eq:n2}. Similarly, the Green, Orange and Blue represent the counterparts for $DC, DD$ and $CD$, respectively.    }
	\label{fig:n1}
\end{figure*}

Consider the OLG repeated games with the stage game of Prisoners' Dilemma (see \autoref{fig:1}). By \autoref{them:1} 
\begin{equation}
\label{eq:n2}
	F(\delta, T) = co  \left( \bigcup_{ a \in \{CC, DC, DD, CD\}}  \{ v( a, CC), v( a, DC), v(a, DD), v(a, CD) ) \}  \right ), 
\end{equation}
where $v (a^1, a^2) = \left (\frac{u_1 (a^1)  + \Delta u_1 (a^2) }{1+ \Delta}, \frac{\Delta u_2 
(a^1)  + u_2 (a^2)}{1 + \Delta } \right)$ for any $(a^1, a^2) \in A^2$ as previous (see \autoref{fig:n1}). 


\begin{comment}
In \autoref{fig:n1} we draw $F(\delta, T)$ for different $\Delta$s, $\Delta = 2/3$, $\Delta= 1/3$ and $\Delta = 0$. In the figure, we classify $v(a^1, a^2)$ based on $a^1$ with different colors. 	
\end{comment}

It is notable that as $\Delta$ changes the sequences of action profiles that generate the extreme points of the feasible payoff set may change. For instance, when $\Delta$ is large enough (e.g., $\Delta = 2/3$), $(CC, CC)$ yields an extreme point. As $\Delta$ becomes smaller (e.g., $\Delta = 1/3$), it is not anymore an extreme point while $(DC, CD)$ becomes a new sequence corresponding one of the extreme points. As $\Delta$ becomes even smaller, $(CC, CD)$ is ``dominated'' by $(DC, CD)$. Intuitively, when $\Delta$ is sufficiently small each player should play the action profile that maximizes her/his payoff in order to be on the efficient frontier. 




\subsection{A 3-player Example from Fudenberg and Maskin (1986)}
\label{subsec:fm}
The second example (see \autoref{fig:2}) involves three players. \cite{FM_1986_ECMA} used this stage game to show that the folk theorem fails for the standard repeated games with infinitely-lived players. In particular, this stage game does not satisfy the full dimensionality, a sufficient condition of their folk theorem.\footnote{\cite{Smith_1992_GEB} shows that the full-dimensionality is not necessary for his folk theorem for OLG repeated games. Since the folk theorem first chooses $T$ then chooses sufficiently large $\delta$, it concerns the case when $\delta^T$ is close to 1. The feasible payoff set in this case is the ``smallest'' according to our characterization, which is the line segment between $(0,0,0)$ and $(1,1,1)$.} 
\begin{figure}
\centering
\begin{tabular}{|c|c|c|}
\hline
    & $A$     & $B$     \\ \hline
$A$ & $1,1,1$ & $0,0,0$ \\ \hline
$B$ & $0,0,0$ & $0,0,0$ \\ \hline
\end{tabular}
\quad 
\begin{tabular}{|c|c|c|}
\hline
    & $A$     & $B$     \\ \hline
$A$ & $0,0,0$ & $0,0,0$ \\ \hline
$B$ & $0,0,0$ & $1,1,1$ \\ \hline
\end{tabular}
\caption{The stage game of a 3-player pure coordination game}	
\label{fig:2}
\end{figure}

Nevertheless we observe that the feasible payoff set of the OLG game exhibits the full dimension for any $\Delta \in (0,1)$. When $\Delta = 1$, it is the line segment between $(0,0,0)$ and $(1,1,1)$, which coincides with the convex hull of the stage game payoffs. On the other hand, when $\Delta \in (0,1)$, it is a polytope with nonempty interior. 
\begin{figure}[t]
\centering
\includegraphics[width=0.5\textwidth]{fig_fm.pdf}
\caption{The OLG feasible payoff set of the pure coordination game when $\Delta = 1$ (Black) $\Delta = 2/3$ (Yellow), $\Delta = 1/2$ (Orange), $\Delta = 1/3$ (Red) and $\Delta= 0$ (Gray).}
\label{}
\end{figure}


\end{document}
