\documentclass[a4paper,12pt]{article}
\usepackage[top=1in,bottom=1in,left=1in,right=1in]{geometry}


\usepackage{amssymb}
\usepackage{amsthm}
\usepackage{amsmath}
\usepackage{color}
\usepackage[]{graphicx}
\usepackage[]{hyperref}
\usepackage{cleveref}
\usepackage{natbib}
\usepackage{tikz}
\usepackage{setspace}
\usepackage{subfiles}
\usepackage{comment}
\usepackage{bookmark}
				
\DeclareMathOperator*{\argmax}{arg\,max}
\DeclareMathOperator*{\argmin}{arg\,min}


\theoremstyle{definition}
\newtheorem{defi}{Definition}
\providecommand*{\defiautorefname}{Definition}
\newtheorem{exmp}{Example}
\providecommand*{\exmpautorefname}{Example}

\newtheorem{conj}{Conjecture}
\newtheorem{lem}{Lemma}
\providecommand*{\lemautorefname}{Lemma}
\newtheorem{claim}{Claim}
\providecommand*{\claimautorefname}{Claim}
\newtheorem{them}{Theorem}
\providecommand*{\themautorefname}{Theorem}
\providecommand*{\figureautorefname}{Figure}

%\theoremstyle{plain}
\newtheorem{coro}{Corollary}
\providecommand*{\coroautorefname}{Corollary}
\newtheorem{prop}{Proposition}
\providecommand*{\propautorefname}{Proposition}
\newtheorem{ass}{Assumption}
\providecommand*{\assautorefname}{Assumption}
\newtheorem{step}{Step}


\title{Characterizing the Feasible Payoff Set of OLG Repeated Games\thanks{Chihiro Morooka was supported by the Japan Society for the Promotion of Science (JSPS) KAKENHI Grant Number JP22K13360. All errors are ours.}}
\author{Daehyun Kim\thanks{Division of Humanities and Social Sciences, POSTECH. Email: \href{mailto:dkim85@outlook.com}{dkim85@outlook.com}}\and Chihiro Morooka\thanks{School of Science and Engineering, Tokyo Denki University. Email: \href{mailto:c-morooka@mail.dendai.ac.jp}{c-morooka@mail.dendai.ac.jp}}}
\date{January 4, 2023}

\begin{document}
\maketitle
\begin{abstract}
We study the set of feasible payoffs of OLG repeated games. We first provide a complete characterization of the feasible payoffs. Second, we provide a novel comparative statics of the feasible payoff set with respect to players' discount factor and the length of interaction. Perhaps surprisingly, the feasible payoff set becomes \emph{smaller} as the players' discount factor approaches to one. 
\end{abstract}


\onehalfspacing

\newpage


\section{Introduction}
\subfile{section1}
\label{sec:1}

\section{Model}
\subfile{section2}
\label{sec:2}

\section{A Complete Characterization of the Feasible Payoff Set}
\label{sec:3}
\subfile{section3.tex}


\section{Monotonicity}
\label{sec:4}
\subfile{section4.tex}


\section{Examples}
\label{sec:5}
\subfile{section5}





\section{Discussion and Conclusion}
\label{sec:6}
\subfile{section6}



\appendix
\section{Omitted Proofs}
\label{sec:app}
\subfile{section_app}



\bibliography{FeasibleOLG_bib}
\bibliographystyle{chicago}


\end{document}
