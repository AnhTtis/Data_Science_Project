\documentclass{FeasibleOLG_main.tex}{subfiles}

\begin{document}





In overlapping generation (OLG) repeated games, players play for finite periods and are replaced by their next generation. This class of games has been used to study cooperation among finitely-lived players in long-run organizations (e.g., \cite{Hammond_1975} and \cite{Cremer_1986_QJE}).

In this paper, we study the feasible payoff set of OLG repeated games. In the literature of OLG repeated games, including studies of the folk theorems, the convex hull of the stage game payoffs is mostly used as the feasible payoff set of interest. However, the overlapping structure allows players to achieve average discounted payoffs beyond the convex hull of the static payoffs: although players share the same discount factor, depending on where they are located in their lifecycle, players discount payoffs differently. Thus, it is not obvious which payoffs are feasible. Our purpose in this paper is to understand how the OLG structure affects what players could obtain. 

 We study the feasible payoff set when players' discount factor and the period of overlap are fixed, departing from the most studies in the literature, which usually focus on the asymptotic case. On the other hand, as typical in the literature, we focus on ``periodic'' feasible payoffs in which each generation of the same player plays in the same sequence of actions during their lifetime.\footnote{More generally, each generation of the same player plays different sequences of action profiles. In this case, each player has an infinite sequence of feasible payoffs. We discuss more about it in \Cref{subsec:discuss}.} 

		
		
Our first main result concerns a complete characterization of the feasible payoff set. We find that it can be characterized by the convex hull of the set of the average discount payoffs that can be achieved by playing $n$-length sequence of action profiles, where $n$ is the number of players and each of the action profiles is to be played for $T$ (the interaction length) times consecutively. For such sequences of action profiles, we could calculate the average discounted payoff \emph{as if} players play $n$-length sequence (rather than $nT$), while effectively discounting $\delta^T$ (rather than $\delta$). Thus, this characterization substantially simplifies the set of action profiles we should consider. For example, for the Prisoners' Dilemma game, where each player has two actions, the result implies, it is sufficient to consider $4 \times 4 = 16$ sequences of action profiles to obtain the feasible payoff set, regardless of $\delta$ and $T$. In particular, this is true even when $T$ is a large number. In fact, our characterization allows us to calculate the feasible payoff set in closed forms.


Our second main result is about a comparative statics of the feasible payoff set with respect to $\delta$ and $T$. We find that the feasible payoff set is decreasing  (in the set-inclusion sense) in the effective discount factor $\delta^T$. Perhaps surprisingly, this implies that the set is \emph{decreasing} in $\delta$. When the effective discount factor is $1$, the feasible payoff set coincides with the convex hull of the static payoffs. When it is close to $0$, it is a $n$-dimensional cube, where for each player the maximum (resp. minimum) feasible payoff coincides with the maximum (resp. minimum) stage payoff. For intermediate effective discount factors, it is a $n$-dimensional polytope. 





	
\textbf{Literature review} 
	
	Previous researches on OLG games mainly focused on folk-like approaches as in \cite{Kandori_1992_RES} and \cite{Smith_1992_GEB}.\footnote{Recently,  \cite{Morooka_2021_IJGT} provides an alternative folk theorem with an opposite order of choosing parameters: it shows that if $\delta$ is chosen first then $T$ is chosen, any feasible and strictly individually rational payoffs can be achieved by subgame perfect equilibrium payoffs. The feasible payoff set considered is larger than the convex hull of the stage game payoffs.} Instead, we study the OLG repeated games with fixed $\delta$ and $T$. One would be more interested in the equilibrium payoff set in this case. By studying the feasible payoff set, we provide a natural benchmark for studying equilibrium payoffs. 
	


For finitely repeated games when players discount factors are different, \cite{Chen_2007_EL} and \cite{CF_2013_IJGT} study a related question to ours: in finitely repeated games, they ask whether the feasible payoff set becomes larger as the length of the game becomes longer. They show that for any two-player stage game, this is indeed the case, leaving the question open for the case with more than two players. On the other hand, we assume players share a common discount factor, while the OLG structure makes players located in a different position in their lifecycle, resulting in different discounting of payoffs. Note that players discount in the same manner if their ``age'' is the same. The symmetric structure of discount factor makes the analysis more tractable than when players have unequal discount factors, which allows us to characterize the feasible payoff set for any stage games with any number of players.
	

	


The remainder of the paper is organized as follows. In \Cref{sec:2}, we introduce the model of OLG repeated games. In \Cref{sec:3}, we present our first main result which is a complete characterization of the feasible payoff set of OLG games. In \Cref{sec:4}, we provide a comparative statics of the feasible payoff set with respect to $\delta$ and $T$. We provide two examples in \Cref{sec:5} to illustrate our main results. \Cref{sec:6} concludes after discussions.

	
\end{document}
