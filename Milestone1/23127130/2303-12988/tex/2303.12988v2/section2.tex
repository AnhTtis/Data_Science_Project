\documentclass{FeasibleOLG_main.tex}{subfiles}

\begin{document}


\subsection{Stage Game}
A stage game is defined as a triple $G=(N,(A_i)_i,(u_i)_i)$, where $N=\{ 1,2,\cdots ,n\}$, for some $n\geq 2$, is the set of one-shot players, $A_i$ is a finite set of pure actions available to player $i$,\footnote{As long as a Public Randomizing Device is available, our result by mixed actions is the same with the one by pure actions. Therefore, we only consider pure actions throughout this paper.} and  $u_i: \prod_i A_i \to \mathbb{R}$ is player $i$'s one-shot payoff function. Let $A \equiv \prod_{i \in N} A_i$ be the set of action profiles. 

Given a set $B \subseteq \mathbb{R}^N$, let $co (B)$ be the convex hull of $B$. Let $V$ be the feasible set of one-shot payoffs, defined as follows:
$$V \equiv co \{ u(a) : a \in A\}.$$


\subsection{OLG Repeated Game}
Given a stage game $G$ defined above, $\delta \in (0,1]$ and $T \in \mathbb{N}$, we define the OLG repeated game $OLG(G,\delta ,T)$ as follows (also see \autoref{o1}):
\begin{itemize}
	\item In every period $t \in \mathbb{N}$, $G$ is played by $n$ finitely-lived players.
	\item For $i\in N$ and $d \in \mathbb{N}$, the player with $A_i$ in generation $d$ joins in the game at the beginning of period $(d-1)nT+(i-1)T+1$, and lives for the following $n$ \emph{overlaps} each of which consists of $T$ periods, until he retires at the end of period $dnT+(i-1)T$. The only exceptions are the players with $A_i$ for $i\in N\setminus\{ 1\}$ in generation 0, who participates in the game between periods 1 and $(i-1)T$.
	\item Each player's per-period payoffs are discounted at a common discount factor $\delta$.
\end{itemize}
\begin{table}[t]
\centering
\scalebox{0.7}{
\begin{tabular}{|c|c|c|c|c|c|c|c|c|}\hline
Period&$1\sim T$&$T+1\sim 2T$&$2T+1\sim 3T$&$3T+1\sim 4T$&$4T+1\sim 5T$&$5T+1\sim 6T$&$6T+1\sim 7T$&$\cdots$\\\hline
$A_1$&\multicolumn{3}{c|}{Generation 1}&\multicolumn{3}{c|}{Generation 2}&\multicolumn{2}{c|}{Generation 3 $\cdots$}\\\hline
$A_2$&Generation 0&\multicolumn{3}{c|}{Generation 1}&\multicolumn{3}{c|}{Generation 2}&$\cdots$\\\hline
$A_3$&\multicolumn{2}{c|}{Generation 0}&\multicolumn{3}{c|}{Generation 1}&\multicolumn{3}{c|}{Generation 2 $\cdots$}\\\hline
\end{tabular}
}
\caption{Structure of OLG repeated game with $n=3$}
\label{o1}
\end{table}
When a sequence of actions $(a (t)) _{t=1}^{nT}\in A^{nT}$ is played throughout a player's life with $A_i$, her/his average payoff is as follows:$\footnote{For the player with $A_i$ for $i\in N\setminus\{ 1\}$ in generation 0, replace $nT$ with $(i-1)T$.}$
\begin{eqnarray}
\frac{1}{\sum _{t=1}^{nT}\delta^{t-1}}\sum _{t=1}^{nT}\delta ^{t-1}u_i(a (t)).
\nonumber\end{eqnarray}

We assume that players can access a Public Randomizing Device (henceforth PRD) uniformly distributed over the unit interval at the beginning of every period.\footnote{This assumption is employed also by \cite{Chen_2007_EL} and \cite{CF_2013_IJGT}.} We assume each player can observe the realization of the PRD at each period after her/his birth.


















\begin{comment}

\subsection{Stage Games}
A stage game $G=(N, A,u)$ is defined as follows. $N=\{ 1,2,\cdots ,n\}$ for $n\geq 2$ is a set of players. Each player $i$ is endowed with a finite set of available actions $A_i$. Let $A = \prod_{i \in N} A_i$. Player $i$'s payoff function is given by $u_i: A \to \mathbb{R}$.  


Let 
$$V \equiv co \{ g(a) : a \in A\}.$$


\subsection{OLG Games}
















=====


$OLG(G, \delta ,T)$ is defined as follows.


The life span is $nT$ and the overlapping periods is $T$. 

For player $i$ and $k \geq 1$, the player in generation $k$ joins... 







Importantly, players can access a public randomization device (henceforth PRD).\footnote{This assumption is employed by \cite{CF_2013_IJGT}.} \textcolor{red}{We assume that the result of a public randomization at some period can also be observed by the next generation. I think we need this for the first result.}



 \textcolor{blue}{(Chihiro 9/22) It's still unclear for me that how the ``degree of the accessibility to past PRD'' affects the set of OLG feasible payoffs. Also, how to define the exact ``stable OLG feasible set of payoffs'' is still unclear. Maybe it would be better for us to provide another subsection to discuss this issue separately. Or we would like to see an example of 2-player OLG game where the difference of the assumption affects its feasible set.}

\color{red}
Daehyun (09/23): Here is what I thought (but now I think the two definitions of feasible payoff lead the same set; I left a proof for this as a part of Theorem 1. The current version of the proof seems unnecessarily complicated though): for example, consider $(1,0)$ and $(0,1)$. With ``stronger'' PRD usage, first there will be 4 payoff points. Then, we convexfy them. If we use weaker one, then we fist convexfy $(1,0)$ and $(0,1)$ (here we get a line segment), then we get 4 points for each of two points on this line segment. The two procedures lead the same set.

\color{black}

\end{comment}





\end{document}
