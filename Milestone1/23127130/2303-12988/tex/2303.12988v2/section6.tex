\documentclass{FeasibleOLG_main.tex}{subfiles}
\begin{document}


\subsection{Discussion}
\label{subsec:discuss}
\begin{comment}
	\textcolor{blue}{(Chihiro 9/22) Do you know how to draw some graph which describes non-stationally feasible set in 2-D plane? It would be helpful if we got some graphical image, but because each generation gets different payoff, the visual expression will be uneasy.}
\end{comment}


Thus far we have restricted our attention to periodic feasible payoffs. In this subsection, we discuss how the relaxation of this restriction would affect the monotonicity result in terms of $\delta$. We consider non-stationary sequences of action profiles of players, in which each generation of the same player may play different sequences of action profiles during their lifetime. One way to extend the concept of the monotonicity in this case would be as follows: Given a sequence $(\bar{u}^y(\delta))_{y \in \mathbb{N}}$ of players' average discounted payoffs for each generation $y=1,2,\dots$, which is feasible with discount factor $\delta$, we ask whether the same sequence is feasible with $\delta' < \delta$. 

The following example shows that this is not the case: Consider the OLG repeated game with two players with two possible stage game payoffs of $(1,0)$ and $(0,1)$. Suppose $T=1$. Consider the following sequence of payoff vectors for each $t = 1,2, \dots$:
$(1,0), (1,0), (0,1), (0,1), \dots.$
 That is, the first two periods gives $(1,0)$, followed by $(0,1)$ forever. The corresponding average payoff for each generation of player 1 is $\bar{u}_1^1 = 1$ for the first generation, and $\bar{u}_1^y= 0$ for any $y \geq 2$. Observe that player 2 in the first generation, who is born at $t=2$, obtains the average payoff $\bar{u}_2^1 = \frac{ \delta}{ 1+ \delta}$ and $\bar{u}_2^y = 1$ for any $y \geq 2$.

Now consider $\delta' < \delta$. Since $\bar{u}_1^1= 1$. the first two period must give players $(1,0), (1,0)$. Note that the maximum payoff of player 2 in the first generation is obtained when $(0,1)$ is given at $t=3$, yielding the average payoff $\frac{\delta'}{1+\delta'}$, which is strictly smaller than $\frac{\delta}{ 1+ \delta}$. 


 
\subsection{Concluding Remarks} 
In the present paper we study the feasible payoff set of OLG repeated games. In our first result, we show that the set can be characterized by a convex combination of the average discounted payoffs of $n$-period sequences of action profiles, where each of the action profiles is played for $T$ periods consecutively. Our second main result shows that the feasible payoff set is monotonely decreasing in $\delta$ and increasing in $T$. 

We note that our monotonicity result of the feasible payoff set can shed also light on the monotonicity of the equilibrium (Nash equilibrium or subgame perfect equilibrium) payoff set: The set of stationary equilibrium (i.e., each generation of a player employs the same strategy) payoffs is, in general, not increasing in $\delta$. This contrasts from the case of repeated games with infinitely-lived players with a PRD \cite[]{APS_1990_ECMA}.\footnote{When there is no PRD, the monotonicity might not hold (see \cite{MOS_2002_GEB,Yamamoto_2010_IJGT}).} As an example, consider the OLG game with the stage game of the Battle of the Sexes, where the coordination gives $(2,1)$ or $(2,1)$ and mis-coordination results in $(0,0)$ (see \autoref{fig:5}). 
\begin{figure}[t]
\centering
\begin{tabular}{|c|c|c|}
\hline
    & $A$   & $B$   \\ \hline
$A$ & $2,1$ & $0,0$ \\ \hline
$B$ & $0,0$ & $1,2$ \\ \hline
\end{tabular}
\caption{The Battle of the Sexes}
\label{fig:5}
\end{figure}
In this case, the Pareto efficient payoffs, $(2,1)$ and $(1,2)$ can be achieved static Nash equilibria and so any convex combination of playing the two equilibria is also a subgame perfect equilibrium. Our result of the decreasing feasible payoff set then translates into the decreasing efficient equilibrium payoffs as $\delta$ increases.  





\end{document}
