\documentclass{FeasibleOLG_main.tex}{subfiles}

\begin{document}





In overlapping generation (OLG) repeated games, players play for finite periods and are replaced by their next generation. This class of games has been used to study cooperation among finitely-lived players in long-run organizations (e.g., \cite{Hammond_1975} and \cite{Cremer_1986_QJE}).

In this paper, we study the feasible payoff set of OLG repeated games. In the literature of OLG repeated games, including studies of the folk theorems, the convex hull of the stage game payoffs is mostly used as the feasible payoff set of interest. However, the overlapping structure allows players to achieve average discounted payoffs beyond the convex hull of the static payoffs: Although players share the same discount factor, depending on where they are located in their lifecycle, players discount payoffs differently. Thus, it is not obvious which payoffs are feasible. Our purpose in this paper is to understand how the OLG structure affects what players could obtain. 

 We study the feasible payoff set when players' discount factor and the period of overlap are fixed, departing from the most studies in the literature, which usually focus on the asymptotic case. On the other hand, as typical in the literature, we focus on ``periodic'' feasible payoffs in which each generation of the same player plays in the same sequence of actions during their lifetime.\footnote{More generally, each generation of the same player plays different sequences of action profiles. In this case, each player has an infinite sequence of feasible payoffs. We discuss more about it in \Cref{subsec:discuss}.} 

		
		
Our first main result concerns a complete characterization of the feasible payoff set of OLG repeated games. We find that it can be characterized by the convex hull of the set of the average discounted payoffs that can be achieved by playing $n$-length sequences of action profiles, where $n$ is the number of players and each of the action profiles is supposed to be played for $T$ (the interaction length) times consecutively. For such sequences of action profiles, we could calculate the average discounted payoff \emph{as if} players play $n$-length sequence (rather than $nT$), while effectively discounting $\delta^T$ (rather than $\delta$). Thus, this characterization substantially simplifies the set of action profiles we should consider in obtaining the feasible payoff set. In fact, our characterization allows a closed-form expression of the feasible payoff set given any stage game.

Our second main result is about a comparative statics of the feasible payoff set with respect to $\delta$ and $T$. We find that the feasible payoff set is decreasing  (in the set-inclusion sense) in the effective discount factor $\delta^T$. Perhaps surprisingly, this implies that the set is \emph{decreasing} in $\delta$. When the effective discount factor is $1$, the feasible payoff set coincides with the convex hull of the static payoffs. When it is close to $0$, it is a $n$-dimensional cube, where for each player the maximum (resp. minimum) feasible payoff coincides with the maximum (resp. minimum) stage payoff. For intermediate effective discount factors, it is a $n$-dimensional polytope. 





	
\textbf{Related Literature} 
	
	Previous researches on OLG repeated games have mainly focused on folk-theorem-like approaches as in \cite{Kandori_1992_RES} and \cite{Smith_1992_GEB}.\footnote{Recently,  \cite{Morooka_2021_IJGT} provides an alternative folk theorem with an opposite order of choosing parameters: It shows that if $\delta$ is chosen first then $T$ is chosen, any feasible and strictly individually rational payoffs can be achieved by subgame perfect equilibrium payoffs. The feasible payoff set considered is larger than the convex hull of the stage game payoffs.} Alternatively, in this paper, we study the OLG repeated games with fixed $\delta$ and $T$. By studying the feasible payoff set, we provide a natural benchmark for the equilibrium payoff set which one might be more interested in.
	


The present paper is also related to the literature of repeated games with differential discounting of players, which has been studied since \cite{LP_1999_ECMA}. They study infinitely repeated games between two players who have different discount factors, and show that some payoffs outside the convex hull of the stage game payoffs can be obtained by intertemporal trading of payoffs: The more patient player gives payoffs in early periods to have more in later periods. Allowing differential discounting of players, \cite{Sugaya_2015_TE} proves a folk theorem for $n$-player infinitely repeated games with imperfect public monitoring. \cite{DG_2022_JET} provides a more constructive approach to study feasible and equilibrium payoffs of repeated games with perfect monitoring. On the other hand, \cite{Chen_2007_EL} and \cite{CF_2013_IJGT} study finitely repeated games between two players. These papers examine whether the feasible payoff set becomes larger as the length of the game becomes longer. The latter paper, based on the result of the former, shows that for any two-player stage games, this is indeed the case.\footnote{They leave the question for more general case of arbitrary number of players as open.} 



Comparing to the literature of repeated games with differential discounting, in our model of OLG repeated games, players share the \emph{same} discount factor. Nevertheless, players can trade payoffs across periods due to the overlapping generation structure: In a given period, players are located in a different position in their lifecycle (``age''), resulting in different discounting of some future payoffs. Notice that players discount in the same way when they have the same age. In this sense, there is a ``symmetricity'' in their discounting, which makes our analysis relatively tractable. It results in our characterization of the feasible payoff set, allowing a closed-form expression of the feasible payoff set given any stage game for each discount factor and the length of each generation's lifespan.\footnote{In the literature of repeated games with differential discounting, \cite{Chen_2007_EL} provides an explicit characterization of the feasible payoffs of finitely repeated games for a specific two-player stage game. \cite{Sugaya_2015_TE} provides a recursive characterization of the feasible payoff set of infinitely repeated games for general stage games. \cite{DG_2022_JET} provides several characterizations of the feasible payoff set; in particular, they characterize it when players can have some large discount factors.} 



	

	


The remainder of the paper is organized as follows. In \Cref{sec:2}, we introduce the model of OLG repeated games. In \Cref{sec:3}, we present our first main result which is a complete characterization of the feasible payoff set of OLG games. In \Cref{sec:4}, we provide comparative statics results of the feasible payoff set with respect to $\delta$ and $T$. We provide two examples in \Cref{sec:5} to illustrate our main results. \Cref{sec:6} concludes after discussions.

	
\end{document}
