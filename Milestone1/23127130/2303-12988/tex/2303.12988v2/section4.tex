\documentclass{FeasibleOLG_main.tex}{subfiles}

\begin{document}

In this section, we study comparative statics of the feasible set w.r.t. both $T$ and $\delta$. From \autoref{them:1} in the previous section, we know that the feasible payoff set depends on $\delta$ and $T$ only through $\delta^T$. 



\subsection{Monotonicity}
Our main result in this section asserts that the feasible payoff set is non-increasing in $\delta^T$, implying, perhaps surprisingly, it is \emph{non-increasing} in $\delta$.





\begin{them}
\label{them:2}
The following hold:
\begin{enumerate}
	\item For any $\delta, \delta'$ with $0< \delta <\delta ' \leq 1$, $F(\delta ',\ T)\subseteq F(\delta ,\ T)$.
\item Given any $\delta \in (0,1]$, $F(\delta ,\ T)\subseteq F(\delta ,\ T+1)$.	
\end{enumerate} \end{them}

The rest of this section devotes to proving \autoref{them:2}.\footnote{The weak set-inclusions in the result cannot be strengthened to be strict. For instance, if the stage game payoff set is an $n$-dimensional rectangle, the feasible payoff set of the corresponding OLG game would be unchanged with respect to the parameters.} In doing so, we shall also provide a characterization of players' optimal play to maximize the welfare given some weights for players, which might have some independent interest. 

\autoref{them:1} implies that it is without loss of generality to consider a sequence of $n$ pure action profiles for studying the monotonicity. That is, players play the same action profile during an overlap which consists of $T$ periods. In addition, since the feasible payoff set is convex, it is enough to show the maximum ``score'' increases as $\delta$ (resp. $T$) becomes smaller (resp. larger) for each non-zero direction. 

Fix $\lambda \in \mathbb{R}^n \setminus \{ \mathbf{0}\}$. For a given $\delta \in (0,1]$ and $T \in \mathbb{N}$, let $\Delta \equiv  \Delta (\delta, T) =  \delta^T$.  Define $\lambda$-weighted welfare as
\begin{equation}
\label{eq:nn2}
W_{ \lambda}^*(\Delta) :=\max_{ a^{[n]} =(a^1, \dots, a^n) \in A^n} W_\lambda (a^{[n]},\Delta),
\end{equation}
where
$$W_\lambda (a^{[n]}, \Delta ) := \sum_{i=1}^n \lambda_i v_i (a^1, \dots, a^n), \quad \forall a^{[n]} \in A^n.$$

We want to show that $W_{ \lambda}^* (\Delta)$ is non-increasing in $\Delta$ (as a result, non-decreasing in $T$ and non-increasing in $\delta$). 







\begin{comment}
	\begin{lem}
When $\Delta$ is sufficiently small, the following hold: 
\begin{enumerate}
	\item For each phase $k  \in \{ 1, \dots, n\}$, the optimal action profile  is an action profile that maximizes $\lambda_k u_k (a)$. Thus, if $\lambda_t >0$, it is the action profile that maximizes player $t$'s payoff, while $\lambda_t <0$, then it is the one that minimize player $t$'s payoff.
\item The derivative of the aggregate sum is negative. 
\end{enumerate}
\end{lem}

\begin{proof}
It is almost obvious that when $\Delta$ is sufficiently small, for each phase $k \in \{1, \dots, n\}$
$$ \argmax_{ a \in A}\lambda_k u_k (a) = \sum_{i \in N} \lambda_i u_i^k (a^k) $$
because players other than $k$ are not important as $\Delta$ is small. 


Next we show that this observation implies that the derivative of the aggregate sum is negative. To see this first note the identity 
$$\sum_{k=1}^n \frac{\Delta^{k-1}}{1+ \Delta +\cdots + \Delta^{n-1}} = 1.$$
Hence, the derivative of the LHS w.r.t. $\Delta$ is 0. Thus,
$$\underbrace{\frac{d}{d\Delta}\left( \frac{1}{ 1+ \Delta + \cdots + \Delta^{n-1}} \right)}_{<0} =- \frac{d}{d\Delta}\left( \sum_{k=2}^{n} \frac{\Delta^{k-1}}{1+ \Delta+ \cdots + \Delta^{T-1}}\right).$$
Since, by the first item, each player $i$ has the highest $\lambda_i u_i (a^k)$ at phase $k=i$, the derivative of the discounted payoff of player $i$ is negative. 


\end{proof}





\begin{lem}
When $\Delta$ is sufficiently large, then the following hold:
\begin{enumerate}
	\item The action profile for each period is the same across periods and it is the action profile that maximizes the $\lambda$-weight sum of players' payoffs, i.e., 
$$a \in \argmax \sum_{i=1}^n \lambda_i u_i (a)$$
	\item The derivative of the aggregate sum is 0.
\end{enumerate} 	
\end{lem}

\begin{proof}
Observe when $\Delta = 1$, the optimal action profile is clearly the one that maximizes the $\lambda$-weight sum of players' payoff. Then, by continuity, there exists some range of $\Delta$ for which the action profile is still optimal.


\end{proof}


New argument ...

The aggregate average payoff can be written as 
$$\frac{u_1 (a^1) + u_2 (a^2) +u_3 (a^3) + \delta (u_1 (a^2) + u_2 (a^3) + u_3 (a^1) ) + \delta^2 (u_1 (a^3) + u_2 (a^1) + u_3 (a^2))  }{1+ \delta +\delta^2 }$$
Note that there is a ``symmetry'' in the numerator. Since for any $\delta$, the first term has the highest weight, 
$$u_1 (a^1) + u_2 (a^2) + u_3 (a^3) \geq \max \{ u_1 (a^2) + u_2 (a^3) + u_3 (a^1) , u_1 (a^3) + u_2 (a^1) + u_3 (a^2)) \}$$
* This assertion is wrong: $u^1=(1,0,100), u^2 = (0,140,0), u^3 = (0,140,0)$ (which violates the inequality) can be a solution for some range of $\delta$ (for this particular example, this range is very small $\delta \in (10 \left(7-4 \sqrt{3}\right), \frac{1}{70} \left(2 \sqrt{165}+25\right))$. If $\delta$ is larger than $\frac{1}{70} \left(2 \sqrt{165}+25\right)$, then $u^1$ is also $(0,140,0)$).

In this example, the aggregate payoff changes as follows:
\begin{itemize}
	\item $\frac{241+100\delta + \delta^2}{1+\delta+\delta^2}$ for $\delta \in (0, (10 \left(7-4 \sqrt{3}\right))$
	\item $\frac{141+140\delta}{1+\delta + \delta^2}$ for $\delta \in ((10 \left(7-4 \sqrt{3}\right), \frac{1}{70} \left(2 \sqrt{165}+25\right)))$
	\item $140$ for $\delta \in (\frac{1}{70} \left(2 \sqrt{165}+25\right)),1)$
\end{itemize}

Some ideas: 
\begin{itemize}
\item At the point of $\delta$ where the optimal action changes for some period $t$, the derivative w.r.t. $\delta$ associated with the new solution is larger that that of the previous solution. That is, we know \underline{at least} for ``around'' those $\delta$, the derivative w.r.t. $\delta$ for period $t$ is indeed increasing. 
\end{itemize}

\end{comment}
Let us introduce a few more notations. Denote the set of optimal solution of \eqref{eq:nn2} by $\mathcal{A}^*_\lambda (\Delta) \subseteq A^n$. Let $\mathcal{U}^*_\lambda (\Delta) = \{ (u^k)_k : u^k = u (a^k), a^{[n]} \in \mathcal{A}_\lambda^* (\Delta) \} \subseteq \mathbb{R}^{n\times n}$. That is, the set of sequences of the optimal payoff vectors. Notice that $\mathcal{U}^*_\lambda (\Delta)$ may be a singleton even when $\mathcal{A}^*_\lambda (\Delta)$ is not.  Lastly, given $u^{[n]} = (u^1, \dots, u^n) \in \mathbb{R}^{n \times n}$, for each $k \in \{ 1, \dots, n\}$, let
%\begin{equation}
%\label{eq:nnn3}
$$w_k (u^{[n]}) := \sum_{ i =1}^n \lambda_i u_i^{i+k-1},$$
%\end{equation}
where $u_i^m=u_i^{m-n}$ if $m>n$. That is, $w_k (u^{[n]})$ is the weighted sum of players' payoffs when their ``age'' is $k$ (i.e., they are in the $k$-th overlap in their lifetime). For instance, given $u^{[3]} = (u^1, u^2, u^3) \in \mathbb{R}^{3 \times 3}$ and $\lambda = (1,1,1)$, $w_1 (u^{[3]}) = u^1_1  + u^2_2   + u^3_3 $, $w_2 (u^{[3]}) = u^2_1  + u^3_2  + u^1_3  $, $w_3 (u^{[3]}) =u^3_1   + u^1_2  + u^2_3 $. 







Observe that for each $u^{ [n]}=(u^1,\dots, u^n) \in \mathcal{U}^*_\lambda (\Delta)$. 
\begin{equation}
\label{eq:nn3}
W_{ \lambda}^* (\Delta) = \frac{\sum_{k=1}^n \Delta^{k-1} w_k (u^{[n]}) }{\sum_{k=1}^n \Delta^{k-1}}.
\end{equation}

The following lemma is the key to prove the monotonicity. 
\begin{lem}
\label{lem:1}
Let $\Delta \in (0,1]$ and $u^{[n]}\in \mathcal{U}^*_\lambda (\Delta)$. For each $m =1, \dots, n-1$, 
\begin{equation}
\label{eq:2}
\sum_{k=1}^m \Delta^{k-1} w_k (u^{[n]}) \geq  \sum_{k=1}^m \Delta^{k-1} w_{n-m+k} (u^{[n]}).
\end{equation}
\end{lem}
\begin{comment}
w_1 (\Delta) + \Delta w_2 (\Delta)  + \Delta^2 w_3 (\Delta) + \cdots + \Delta^{m-2} w_{m-1} (\Delta) + \Delta^{m-1} w_m (\Delta)	\\
\geq w_{n-m+1} (\Delta) + \Delta w_{n-m+2} (\Delta)  + \cdots + \Delta^{m-2} w_{n-1} (\Delta)  + \Delta^{m-1} w_n (\Delta).	
\end{comment}


In words, the lemma means, at optimum, the payoffs in the earlier stages of a player are higher than those of later stages (in the sense that for any $m \leq n-1$, the first $m$ payoffs should be larger than the last $m$ payoffs). When there are only two players, this reduces to the condition that the payoffs when they are ``young'' must be larger than those when they are ``old'' for optimality. When there are more than two players, it is \emph{a priori} not clear what could be the corresponding expression. According to the lemma, for the case of three players, it is $w_1 (u^{[3]}) \geq w_3 (u^{[3]}) $ and $w_1 (u^{[3]})   + \Delta w_2 (u^{[3]})  \geq w_2 (u^{[3]})  + \Delta w_3 (u^{[3]}) $. 


Let us explain the crux of the idea of the proof with 3 players and $\lambda = (1,1,1)$. Assume $u^{[3]} = (u^1, u^2, u^3)$ be an optimal sequence of payoff vectors. From optimality of $u^1$ at $k=1$, it should be $u^1_1 + \Delta^2 u^1_2 + \Delta u^1_3 \geq u^2_1 + \Delta^2 u^2_2+ \Delta u^2_3$. By multiplying both sides by $\Delta$, 
		$$\Delta u^1_1 + \Delta^3 u^1_2 + \Delta^2 u^1_3 \geq \Delta u^2_1 + \Delta^3 u^2_2 + \Delta^2 u^2_3.$$
On the other hand, by optimality of $u^2$ at overlap $k=2$,
		$$\Delta u^2_1 + u^2_2 + \Delta^2 u^2_3\geq \Delta u^1_1 + u^1_2 + \Delta^2 u^1_3.$$
		From the two inequalities, we can conclude $u^2_2 \geq u^1_2$. Intuitively, since the same generation of player 1 and 3 is active both at $k=1$ and $k=2$, while player 2 is replaced by the next generation, in order for $u^2$ to give a larger aggregate payoff at $k=2$, player 2 should have a larger payoff at $u^2$ than at $u^1$. A symmetric argument results in $u^3_3 \geq u^2_3$ and $u^1_1 \geq u^3_1$, and summing them up results in $w_1 (u^{[3]}) \geq w_3 (u^{[3]}) $. Applying a similar argument to ``two-overlap apart,'' we have $\Delta u_2^1 + u_3^1 \leq \Delta u_2^3 + u_3^3$ (as a result, $w_1 (u^{[3]})   + \Delta w_2 (u^{[3]})  \geq w_2 (u^{[3]})  + \Delta w_3 (u^{[3]})$).






\begin{proof}
 

Denote the ``age'' of player $i$ at overlap $k$ by $y_k (i) \in \{ 1,\dots, n\}$. 


Consider overlap $k  \in \{ 1, \dots, n\}$ and $m \in \{ 1,\dots, n-1\}$. Let $k' = k +m \text{ (mod $n$)}$. Then, from optimality of $u^k$ at overlap $k$, 
		$$\sum_{ i =1}^n   \Delta^{ y_k (i)-1 } \lambda_i  u_i^k   \geq \sum_{i=1}^n \Delta^{y_k(i)-1} \lambda_i u_i^{k'} .$$
		By multiplying both sides by $\Delta^m$,
$$\sum_{ i =1}^n   \Delta^{ y_k (i)+ m-1 } \lambda_i  u_i^k  \geq \sum_{i=1}^n \Delta^{y_k(i) + m-1} \lambda_i u_i^{k'} .$$
Note that $y_{k'} (i) = y_k (i) + m$ if $y_k (i) + m \leq n$; otherwise $y_{k'} (i)= y_k (i) + m-n$.
\begin{multline*}
\sum_{ i : y_k(i) + m \leq n}   \Delta^{ y_{k'} (i)-1 } \lambda_i  u_i^k + \Delta^n \sum_{ i : y_k(i) + m >n}   \Delta^{ y_{k'} (i)-1 } \lambda_i  u_i^k \\
\geq \sum_{i : y_k(i) + m \leq n} \Delta^{y_{k'}(i)-1} \lambda_i u_i^{k'}  +\Delta^n\sum_{i : y_k(i) + m >n} \Delta^{y_{k'}(i)-1} \lambda_i u_i^{k'}	
\end{multline*}
or equivalently, 
\begin{multline}
\label{eq:nnn6}
\sum_{ i : y_k(i) + m \leq n}   \Delta^{ y_{k'} (i)-1 } \lambda_i  u_i^k - \sum_{i : y_k(i) + m \leq n} \Delta^{y_{k'}(i)-1} \lambda_i u_i^{k'}  \\
\geq   \Delta^n \left ( \sum_{i : y_k(i) + m >n} \Delta^{y_{k'}(i)-1} \lambda_i u_i^{k'} -  \sum_{ i : y_k(i) + m >n}   \Delta^{ y_{k'} (i)-1 } \lambda_i  u_i^k \right).	
\end{multline}
On the other hand, from optimality at overlap $k'$, we have 
\begin{multline*}
\sum_{ i : y_k(i) + m \leq n}   \Delta^{ y_{k'} (i)-1 } \lambda_i  u_i^{k'}  + \sum_{ i : y_k(i) + m >n}   \Delta^{ y_{k'} (i)-1 } \lambda_i  u_i^{k'}  \\
\geq \sum_{i : y_k(i) + m \leq n} \Delta^{y_{k'}(i)-1} \lambda_i u_i^k  +\sum_{i : y_k(i) + m >n} \Delta^{y_{k'}(i)-1} \lambda_i u_i^k 
\end{multline*}
or
\begin{multline}
\label{eq:nnn7}
\sum_{i : y_k(i) + m \leq n} \Delta^{y_{k'}(i)-1} \lambda_i u_i^k- \sum_{ i : y_k(i) + m \leq n}   \Delta^{ y_{k'} (i)-1 } \lambda_i  u_i^{k'}    \\
\leq     \sum_{ i : y_k(i) + m >n}   \Delta^{ y_{k'} (i)-1 } \lambda_i  u_i^{k'} -\sum_{i : y_k(i) + m >n} \Delta^{y_{k'}(i)-1} \lambda_i u_i^k.
\end{multline}
In order to satisfy both \eqref{eq:nnn6} and \eqref{eq:nnn7}, it must be 
$$\sum_{ i : y_k(i) + m >n}   \Delta^{ y_{k'} (i)-1 } \lambda_i  u_i^{k'}  \geq \sum_{i : y_k(i) + m >n} \Delta^{y_{k'}(i)-1} \lambda_i u_i^k$$
or equivalently,
$$\sum_{ i : y_{k'}(i) \in \{ 1, \dots, m\} }   \Delta^{ y_{k'} (i)-1 } \lambda_i  u_i^{k'}  \geq \sum_{i : y_{k'}(i) \in \{ 1, \dots, m\}} \Delta^{y_{k'}(i)-1} \lambda_i u_i^k .$$
Summing up over $k \in \{1, \dots, n\}$ both sides, we have the inequality in the statement.

\begin{comment} % commented out n=3 casea
	\begin{enumerate}
		\item Daehyun (08/05) Let me add some details:
		Consider $t=1$, then by optimality, we have
		$$u^1_1 + \delta^2 u^1_2 + \delta u^3_1 \geq u^2_1 + \delta^2 u^2_2+ \delta u^2_3$$
		

		
		
		(i.e., here $u^1 \equiv u(a^1), u^2 \equiv u(a^2)$ where $a^1$ and $a^2$ are the optimal actions for $t=1$ and $t=2$, respectively; so this inequality means that playing $a^1$ is weakly better than playing $a^2$ for $t=1$) 
		By multiplying $\delta$ to both sides,
		$$\delta u^1_1 + \delta^3 u^1_2 + \delta^2 u^3_2 \geq \delta u^2_1 + \delta^3 u^2_2 + \delta^2 u^2_3 $$
$$\iff \delta u_1^1 + \delta^2 u_2^3 - (\delta u_1^2 + \delta^2 u^2_3) \geq \delta^3 (u^2_2 - u^1_2)$$
		Now consider $t=2$. Then similarly by optimality (now for $t=2$),
		$$\delta u^2_1 + u^2_2 + \delta^2 u^2_3\geq \delta u^1_1 + u^1_2 + \delta^2 u^1_3$$
		$$\iff \delta u_1^1 + \delta^2 u^1_3 -( \delta u^2_1 + \delta^2 u^2_3) \leq u^2_2 - u^1_2$$
		Combining the two,
		$$ \delta^3 (u^2_2 - u^1_2)\leq \delta u_1^1 + \delta^2 u^1_3 -( \delta u^2_1 + \delta^2 u^2_3) \leq u^2_2 - u^1_2.$$
		From this we can conclude it should be $u^2_2 \geq u^1_2$. Similarly, $u^3_3 \geq u^2_3$ and $u^1_1 \geq u^3_1$. Summing up both sides, then we obtain $a \geq c$. 
		\item Similar to the first item, although now we use the inequalities with ``two-period'' ahead:
$$u^1_1 + \delta^2 u^1_2 + \delta u^1_3 \geq u^3_1 + \delta^2 u^3_2 + \delta u^3_3$$
Multiplying $\delta^2$, 
$$\delta^2 u^1_1 + \delta^3 (\delta u^1_2 + \delta^1_3) \geq \delta^2 u_1^3 + \delta^3 ( \delta u_2^3 + u_3^3)$$
On the other hand, at $t=3$, 
$$\delta^2 u^3_1 + \delta u^3_2 + u^3_3 \geq \delta^2 u^1_1 + \delta u^1_2 + u^1_3$$
Comparing it with above, we know it should be 
$$\delta u_2^1 + u_3^1 \leq \delta u_2^3 + u_3^3.$$
Similarly, 
$$\delta u_3^2 + u_1^2 \leq \delta u_3^1 + u_1^1$$
$$\delta u_1^3 + u_2^3 \leq \delta u_1^2 + u_2^2.$$		
Adding up the inequalities, we obtain
$$\delta c + b \leq \delta b + a.$$
	\end{enumerate}

\end{comment}
	\end{proof}








Perhaps surprisingly, the following lemma says that the inequality \eqref{eq:2} in \autoref{lem:1} is sufficient to prove that the derivative of the aggregate payoff is non-positive. 
\begin{lem}
\label{lem:2}
For $\Delta \in (0,1)$,
$$\frac{\partial W_{ \lambda} }{\partial \Delta }(a^{[n]},\Delta)  \leq 0$$
for any $a^{[n]}  \in \mathcal{A}_\lambda^* (\Delta)$.
\end{lem}

\begin{proof}
See \Cref{proof:lem2}.
\end{proof}


\begin{comment}
\begin{lem}
Given $\lambda$, as $\Delta$ changes from $0$ to $1$ the optimal solution $u^k (\Delta)$ changes finitely many times. Thus, for ``almost any'' $\Delta \in (0,1)$ (except set of $\Delta$ with Lebesgue measure 0), there exists $\epsilon >0$ s.t. for any $\Delta' \in (\Delta-\epsilon, \Delta+ \epsilon)$ has the same optimal payoff. This also means that $W(\Delta)$ is differentiable almost everywhere.  
\end{lem}
	
\end{comment}



\begin{comment}
The optimization problem can be divided into $K$ different problems (each for each ``block''). For each $k$, there are finite extreme points of $V$; as $\Delta$ changes, the optimal solution changes.
	
\end{comment}



\begin{comment}
	Note that 
$$\frac{d}{d \delta }\left( \frac{a + \delta b + \delta^2 c }{1+ \delta + \delta^2} \right)$$
has the numerator 
$$\delta^2 (c-b) + 2\delta (c-a) + b-a.$$
The question is whether we can show that this expression has a negative value, from the above two inequalities. 
\end{comment}




\begin{proof}[Proof of \autoref{them:2} ]
We shall show that for each $\Delta \in (0,1)$, there exists $a^{[n]} \in \mathcal{A}_\lambda^* (\Delta)$ such that
\begin{equation}
\label{eq:nn6}
\frac{d W^*_\lambda (\Delta) }{d\Delta } = \frac{\partial W_{ \lambda} }{\partial \Delta }(a^{[n]},\Delta).\end{equation}
Once we prove it, \autoref{lem:2} immediately implies the conclusion. 



For a generic $\Delta \in (0,1)$, $\mathcal{U}^*_\lambda (\Delta)$ is a singleton. In this case, the $\lambda$-weighted discounted sum of $u^{[n]} \in \mathcal{U}^*_\lambda (\Delta)$ forms a ``vertex'' of the OLG feasible payoff set. Clearly, this means that there exists $\epsilon >0$ such that for all $\Delta '\in (\Delta-\epsilon,\Delta +\epsilon)$, $u^{[n]} \in \mathcal{U}^*_\lambda (\Delta')$; that is, $W_\lambda^* (\Delta ')$ is obtained still from $u^{[n]}$. This implies \eqref{eq:nn6}. 


Next, consider $\Delta \in (0,1)$ for which $\mathcal{U}^*_\lambda (\Delta)$ contains uncountable elements. In this case, the collection of the $\lambda$-weighted discounted sums of the elements of $\mathcal{U}^*_\lambda (\Delta)$ forms an ``edge'' or ``face'' of the OLG feasible set. Note that there exists at least one element $u^{[n]} \in \mathcal{U}^*_\lambda (\Delta)$ and $\epsilon >0$ such that, for all $\Delta '\in (\Delta-\epsilon,\Delta +\epsilon)$ such that  $u^{[n]} \in \mathcal{U}^*_\lambda (\Delta')$. Again, this implies \eqref{eq:nn6}.  

\begin{comment}
We observe $W_\lambda^* (\Delta)$ is an upper envelope of the family: For each $\tilde{\Delta}$,
$$W_\lambda^* (\tilde{\Delta}) = W_\lambda (a^{*[n]}(\tilde{\Delta}),\tilde{\Delta} )$$
and 
$$W_\lambda^* (\Delta ) \geq  W_\lambda (a^{*[n]}(\tilde{\Delta}),\Delta ),\quad \forall \Delta $$

The remaining is to show that $W^*_\lambda (\Delta)$ is differentiable. 
	
\end{comment}







\end{proof}

\subsection{A Limit Characterization of the Feasible Payoff Set}

For the asymptotic cases (i.e., $\delta^T \nearrow 1$ or $\delta^T \searrow 0$), we can be more explicit about the shape of the feasible payoff set. 

When the players' effective discount factor $\delta^T$ is close to 1, there is not much difference in discounting of young and old players. So, the scope of intertemporal trade of payoffs is little and what is the best is to maximize the stage game payoffs for a given welfare weight. On the other hand, when the effective discount factor is close to 0, the difference in discounting of young and old players is large. Hence intertemporal trading can be very helpful. The extreme form of such trading is to maximize the youngest player's (weighted) payoff. We summarize this discussion as a corollary:

\begin{coro}
The following hold:
\begin{enumerate}
\item For $\delta^T$ sufficiently close to 1, the solution of the optimization problem \eqref{eq:nn2} is to play some $a^k \in \argmax_{a' \in A} \lambda \cdot u(a')$ for each overlap $k$. As a result, $\lim_{ \delta^T \nearrow 1} F(\delta, T) = V.$
	\item For $\delta^T$ sufficiently close to 0, the solution of the optimization problem \eqref{eq:nn2} is to play $a^k \in \argmax_{a' \in A} \lambda_{i_k} u_{i_k} (a')$, where $i_k \in N$ is the youngest player in overlap $k$. As a result, $\lim _{\delta^T\searrow 0}F(\delta, T)=  \prod _{i\in N} \left [\min _{a\in A}u_i(a),\ \max _{a\in A}u_i(a) \right]$.	 	
\end{enumerate}

\end{coro}

Thus, for any $\delta \in (0,1)$, as $T \to \infty$, the feasible set converges to the $n$-dimensional cube. Similarly, for any $T$, as $\delta \to 0$, the feasible set converges to the $n$-dimensional cube. 


   

\begin{comment}
\begin{proof}
Choose any $v\in \prod _{i\in N}[\min _{a\in A}u_i(a),\ \max _{a\in A}u_i(a)]$. For each $i$, there exists a vector $w^i\in V$ which satisfies $w^i_i=v_i$. When $i$ is the youngest, players play correlated actions which generate $w^i_i$. Such sequence converges to $v$ as $T\rightarrow\infty$.	
\end{proof}
	
\end{comment}





\end{document}
