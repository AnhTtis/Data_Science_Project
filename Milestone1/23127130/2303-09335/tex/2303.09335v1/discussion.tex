\section{Discussion}
%Pensar que puede estar limitando la red

\begin{figure*}
	\centering
	\begin{subfigure}{.5\textwidth}
		\centering
		\includegraphics[width=1\linewidth]{precision_x_periodo}
%		\captionsetup{labelformat=empty}
		\caption{Precision depending on the period.}
	\end{subfigure}%
	\begin{subfigure}{.5\textwidth}
		\centering
		\includegraphics[width=1\linewidth]{recall_x_periodo}
%		\captionsetup{labelformat=empty}
		\caption{Recall depending on the period.}
		
	\end{subfigure}
	\caption{Precision and recall as a function of the period of the detected planets, using the threshold (thr) 0.77 for the network and 0.95 for the FAP (see Sect.~\ref{lbl:thrsearch}.}
	\label{fig:pres_recall_x_periodo}
\end{figure*}

\begin{figure*}
	\centering
	\begin{subfigure}{.5\textwidth}
		\centering
		\includegraphics[width=1\linewidth]{precision_x_potencia}
%		\captionsetup{labelformat=empty}
		\caption{Precision depending on the power.}
		
	\end{subfigure}%
	\begin{subfigure}{.5\textwidth}
		\centering
		\includegraphics[width=1\linewidth]{exhaustividad_x_potencia}
%		\captionsetup{labelformat=empty}
		\caption{Recall depending on the power}
	\end{subfigure}
	\caption{Precision and recall as a function of the power in the periodogram of the detected planets, using the threshold (thr) 0.77 for the network and 0.95 for the FAP.}
	\label{fig:pres_recall_x_potencia}
\end{figure*}

%\subsection{Characteristics of the predictions}

Having analyzed the general behaviour of the FAP and \texttt{ExoplANNET} as methods for identifying \emph{bona fide} planets in the presence of correlated noise, we now turn to explore the dependence of the performance of the methods with planetary parameters.

In Figure \ref{fig:pres_recall_x_periodo} we plot the precision and recall metrics as a function of the periods of the signals identified as planets, binned every five days. Both methods lose precision in the period range between 30 and 60 days. But while the precision of the FAP method drops below 0.75, the precision of ExoplANNET predictions stays above 0.85. On the other hand, the FAP shows a relatively constant recall across orbital periods, while the network exhibits a clear decrease at around the same period range.

This period range is associated with stellar rotation, and therefore many false positive signals are expected to appear here. Our interpretation is that ExoplANNET learns to be more ``cautious'' when labelling peaks appearing in this range, which incidentally is the same way a professional astronomer usually behaves. As a consequence, the recall is degraded but the precision is kept relatively high. The FAP, on the other hand, does not exhibit any dependence in recall with orbital period and is therefore labelling more noise peaks as planets. As a consequence, the precision is degraded even more than for the network. 

In Figure \ref{fig:pres_recall_x_potencia} precision and recall are plotted as a function of the periodogram peak powers, binned every 10 units of relative power. Although the curves for both methods are  similar, for low-power peaks ExoplANNET is more precise. This leads us to think that the network must be considering characteristics of the periodogram other than the maximum peak power, which allows it to better identify the relatively faint planetary peaks. It is in this range where the network produces less false detection than the FAP method, which explains the results of the previous sections.

