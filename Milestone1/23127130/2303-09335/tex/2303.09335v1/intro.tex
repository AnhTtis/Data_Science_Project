\section{Introduction}
The study of extrasolar planets is a relatively new field of research. Although the first evidence of the existence of this type of bodies dates from 1917 \citep{Landau:2017}, it was not until the 1990s that the first confirmed detections took place. In 1992, by analysing the variations in the period of the pulses received from the radio millisecond pulsar PSR1257+12  \citet{Wolszczan1992} concluded that at least two Earth-mass planets are in orbit around the pulsar. Three years later, \citet{1995Natur.378..355M} discovered the first exoplanet orbiting a solar-type star, 51 Peg b, by measuring the variations in the line-of-sight (radial) velocity of the host star induced by the unseen companion. In the years that followed this first detection, the radial velocity technique allowed unveiling  a large number of planet candidates and information-rich systems, some with masses as small as a few times the mass of the Earth \citep[e.g.][]{lovis2006, mayor2009, wright2016, angladaescude2016a, astudillodefru2017c, astudillodefru2017b, feng2017,delisle2018, bonfils2018, diaz2019, zechmeister2019, dreizler2020} and/or that are promising candidates for future atmospheric characterisation \citep[e.g.][]{bonfils2018, diaz2019}.

These discoveries were in large fraction possible thanks to the continuous improvement in instrumentation that provides an ever increasing precision in the measurement of the radial velocity of the stars. Going from the pioneer ELODIE spectrograph \citep{baranne1996}, to HARPS \citep{mayor2003}, SOPHIE \citep{perruchot2008, bouchy2013}, and CARMENES \citep{quirrenbach2014}, and finally, the ultra-high precision spectrographs such as ESPRESSO \citep{pepe2021} and EXPRES \citep{jurgenson2016, blackman2020}, the precision was improved by over two orders of magnitude, attaining now the level of 10 cm s$^{-1}$.

% What about EXPRESS \citep{mahadevan2014}

However, planet detection with these observations are not limited only by instrumental precision, but by intrinsic variability in the star. Even for the least active, slowly-rotating stars, the phenomena collectively called \emph{stellar activity} can produce spurious radial velocity variations with amplitudes of up to a few meters per second, and timescales ranging from a few minutes \citep[pulsations and granulation; e.g.][]{dumusque2011a} to decades \citep[activity cycles; e.g.][]{lovis2011b, diaz2016a}. Particularly worrying are the variations produced by the rotational modulation of the star, as they tend to exhibit power in the same frequency range as some of the most interesting planetary candidates \citep[e.g.][]{Saar_1997, boisse2009, dumusque2011b, nielsen2013}. The main consequence of the influence of stellar activity is the difficulty to detect exoplanets producing RV variations smaller than 1-ms$^{-1}$.

Some of the most relevant stellar phenomena that can generate this type of noise are:
	\begin{enumerate}
		\item{\textit{Stellar oscillation}:} Pressure waves (p-modes) propagate at the surface of solar type stars causing the contraction and expansion of the outer layers over timescales of a few minutes (5-15 min for the Sun, \cite{Broomhall2009,schrijver_zwaan_2000}). The radial-velocity signature of these modes typically varies between 10 and 400 cm s$^{-1}$, depending on the star type and evolutionary stage \citep{schrijver_zwaan_2000}.

%Granulation corresponds to a small convective pattern
%with a lifetime shorter than 25 min, and a diameter smaller than
%2 Mm (Title et al. 1989; Del Moro 2004). On much larger scales,
%we can find supergranulation. This phenomenon linked to very
%large convective patterns from 15 to 40 Mm, can have a lifetime
%up to 33 h in the Sun (Del Moro et al. 2004). Mesogranulation is
%a convective phenomenon that can be located between the granulation and supergranulation, in terms of size and lifetime (Harvey
%1984; Palle et al. 1995; Schrijver & Zwann 2000).

		\item{\textit{Granulation}:} Various convective motions in the photosphere cause this phenomenon. These \textit{granules}, that emerge from the interior of the star into the photosphere, are hotter than those that cool down and descend. This produces a spurious redshift \citep{refId0, dumusque2011a} that changes depending on the convection pattern and can range from a few minutes to about 48 hours.
		
		\item{\textit{Rotational modulation}:} The rotation of the star can transport various structures on the surface, causing them to appear and disappear at regular intervals breaking the flux balance between the red-shifted and the blue-shifted halves of the star and creating the illusion of a stellar wobble \citep{Saar_1997, Lagrange2010, dumusque2011b}.
		
		\end{enumerate}
%	\end{itemize}

% A summary of these advances can be seen in Figures \ref{fig:plyearmethod} and \ref{fig:masadiscyear}.

From a statistical standpoint, one of the main issues is that these phenomena produce correlated error terms, that invalidate some of the most commonly used techniques for planet detection that rely on ordinary least squares, such as the standard periodogram analyses \citep{baluev-fap, zechmeisterkurster2009}. This problem has often been approached by including correlated noise errors in the modelling of the data. In particular, the use of Gaussian process regression \citep{rasmussenwilliams2005} has been widely adopted by the community \citep[e.g.][just to cite a few among a vast body of literature using this technique]{haywood2014, rajpaul2015, yu2017, cloutier2017, persson2018, bonfils2018, diaz2019, luque2019, suarezmascareno2020}, fuelled by the availability of specific computer code to perform the necessary calculations effectively \citep{foreman-mackey2017, espinoza2019, delisle2022}. However, methods to compare models including this kind of error terms can be time-consuming or unreliable \citep{nelson2020}.

% ONE PARAGRAPH WITH ML SOLUTIONS
A different approach is to bypass the problem of explicitly modelling the effects of activity all together, and rely instead on machine learning models to perform the detection and classification tasks. This has been used mostly for the detection and veto of transiting planet candidates with photometric timeseries \citep[e.g.][]{zucker2018}, mainly from the space missions \emph{Kepler} \citep[e.g.][]{armstrong2017, Shallue_2018, pearson2018, ansdell2018}, \emph{K2} \citep[e.g.][]{dattilo2019}, or \emph{TESS} \citep[e.g.][]{yu2019, osborn2020, rao2021}, but also from ground-based transit surveys such as the Next Generation Transit Survey \citep[e.g.][]{McCauliff_2015, armstrong2018}. These models mostly rely on deep convolutional networks \citep[e.g.][]{lecun1998}, that have proven extremely proficient in many fields, mainly computer vision \citep[e.g.][]{krizhevsky2012, he2016} and natural language processing \citep[e.g.][]{peters2018, devlin2019}. It is not a surprise that these methods have been advanced mostly for spaced-based photometric surveys, as these missions provide rich datasets that are needed to train this kind of machine learning models. But machine learning has also been employed in the context of exoplanets to study their atmospheres \citep[e.g.][]{Marquez-Neila, 2016ApJ...820..107W}, or to classify planets according to their potential habitability \citep[e.g.][]{Basak}. Recently, \citet{debeurs2020identifying} used neural networks to remove the stellar activity signal from simulated and real RV observations and showed how this can help in exoplanet detection.

% OUTLINE
In this article, we train a convolutional neural network (CNN) with simulated data to perform the detection of extrasolar planets in radial velocity time series. We frame the question as a classification problem and train the model to distinguish periodograms with and without planetary signals. We show that this method produces better results than the traditional periodogram analyses and correctly identifies more low-mass planets, being also much faster.


