\section{Conclusions}

%Through this work an automated mechanism to search for planets (dubbed \textit{virtual astronomer}) was presented.
We implemented a Convolutional Neural Network model (named \texttt{ExoplANNET}) to evaluate the presence of planetary signals in radial velocity periodograms. The algorithms was trained on simulated data including red noise and various planetary companions. Its performance was evaluated using previously unseen data and compared with the traditional method based on null  hypothesis significance testing, a.k.a. false alarm probability (FAP). 

Our method outperforms the FAP method on single periodograms, as evaluated by the precision-recall area-under-curve metrics. In other words, a threshold can be chosen, so that it produces a similar level of recall with about 30\% less false positives. In addition, the network is at least five orders of magnitude faster than the traditional FAP method. However, remember that in this work, the noise realizations used to compute the significance of observed peaks are performed using correlated noise, which is certainly slower than methods using only white noise, and could in addition be further optimised by using dedicated methods \citep[e.g.][]{CELERITE, GEORGE}. Finally, \texttt{ExoplANNET} employs barely 1.7kB of computer memory. The trained network is provided in h5 format in the repository \texttt{https://github.com/nicklessagus/ExoplANNET}, together with a sample of labelled periodograms for testing.

% Astronomo virtual
The detection method was implemented within an iterative procedure to sequentially explore the presence of planetary signals in a radial velocity data set. This ``virtual astronomer'' evaluates the maximum peak present in a given periodogram and if it decides it belongs to a planet removes a sinusoidal signal at the detected period and proceeds using the model residuals. If a peak is classified as not belonging to a planet, the procedure stops. The FAP method was again used as comparison for the performance of \texttt{ExoplANNET} used as part of the virtual astronomer. Overall, our new method exhibits a better precision metric than the FAP (0.93 vs. 0.91), while keeping a similar recall metric. 

However, using the virtual astronomer with the FAP method more planets are detected, at the expense of a larger number of false positives. It is probably the higher false positive rate that allows the virtual astronomer with the FAP method to evaluate periodograms which follow from a false detection. The same periodogram might be ignored by the virtual astronomer using the \texttt{ExoplANNET} model if the first peak was correctly identified as noise. Indeed, we see that the number of missed planets (i.e. planets that are not even evaluated by the algorithm) is larger under \texttt{ExoplANNET}. 

% It's really learning!
The behaviour of the detection metrics as a function of  the period of the signal being detected shows a qualitative difference between \texttt{ExoplANNET} and the traditional FAP method. While the FAP retains a relatively constant recall across the entire period range, the neural network seems to have learnt to be more cautious in the period range associated with stellar rotation --and hence activity. As a consequence, while both methods lose precision at this period range, \texttt{ExoplANNET}'s precision remains larger than the one computed with the FAP. This is precisely the reason implementations using machine learning algorithms are useful. As mentioned in the Introduction, the objective is to bypass the modelling of the stellar activity and being able to distinguish between bona fide planets and spurious RV signals nevertheless. In this sense, the comparison with the FAP method is unfair: the traditional method is not even intended to distinguish activity from planets, but merely to claim statistical significance of a signal. Our machine learning method is therefore producing two results at the same time: assessing the significance of a signal and identifying its nature.

% Future work
Future work shall focus on evaluating the performance on real time series, both by performing simulations with realistic time sampling, and on actual observed data. In addition, neural networks can be provided with rich context to the classification task in the form of additional data. In our case, \texttt{ExoplANNET} takes the position and power of the largest peak in the periodogram. Additional information can come in the form of the stellar rotational period and spectral type, for example. Implementing and exploring these possibilities will constitute avenues for future advancement.

%Signals with planetary disturbances and various components of intrinsic noise, typical of solar-type stars, were simulated and some 11,000 time series were generated. With these series, approximately 44,000 periodograms were constructed. These periodograms were used to evaluate the quality of detections using the FAP method and to train and evaluate a convolutional neural network designed to replace it.

%The area under the curve in the precision-completeness graphs categorically showed that the network is a better overall solution than the FAP. By searching for a suitable threshold, improvements in the confidence of the predictions were obtained. The number of false positives is thus decreased by 28 \%, increasing the \textit{virtual astronomer} precision.



