%%%%%%%% ICML 2023 EXAMPLE LATEX SUBMISSION FILE %%%%%%%%%%%%%%%%%

\documentclass{article}

% Recommended, but optional, packages for figures and better typesetting:
\usepackage{microtype}
\usepackage{graphicx}
%\usepackage{subfigure}
\usepackage{booktabs} % for professional tables
%\usepackage{caption}
\usepackage{subcaption}

% hyperref makes hyperlinks in the resulting PDF.
% If your build breaks (sometimes temporarily if a hyperlink spans a page)
% please comment out the following usepackage line and replace
% \usepackage{icml2023} with \usepackage[nohyperref]{icml2023} above.
\usepackage{hyperref}


% Attempt to make hyperref and algorithmic work together better:
\newcommand{\theHalgorithm}{\arabic{algorithm}}

% Use the following line for the initial blind version submitted for review:
\usepackage[accepted]{icml2023}

% If accepted, instead use the following line for the camera-ready submission:
% \usepackage[accepted]{icml2023}

% For theorems and such
\usepackage{amsmath}
\usepackage{amssymb}
\usepackage{mathtools}
\usepackage{amsthm}
\usepackage{amsfonts}
\usepackage{comment}
%\usepackage{hyperref}

% if you use cleveref..
\usepackage[capitalize,noabbrev,nameinlink]{cleveref}

%%%%%%%%%%%%%%%%%%%%%%%%%%%%%%%%
% THEOREMS
%%%%%%%%%%%%%%%%%%%%%%%%%%%%%%%%
\usepackage{thmtools}
\usepackage{thm-restate}

% Todonotes is useful during development; simply uncomment the next line
%    and comment out the line below the next line to turn off comments
%\usepackage[disable,textsize=tiny]{todonotes}
\usepackage[textsize=tiny]{todonotes}

\usepackage[utf8]{inputenc}
\usepackage{aliases}
\usepackage{url}
\usepackage{comment}
\usepackage{pifont}
\usepackage{bbold}
\usepackage{dsfont}
\usepackage{thm-restate}
%\newcommand{\theHalgorithm}{\arabic{algorithm}}
\usepackage[ruled,vlined]{algorithm2e}
\crefname{algocf}{Algorithm}{Algorithms}
\Crefname{algocf}{Algorithm}{Algorithms}
\usepackage[normalem]{ulem}


% For Tikz
\usepackage{physics}
\usepackage{amsmath}
\usepackage{tikz}
\usepackage{mathdots}
\usepackage{yhmath}
\usepackage{cancel}
\usepackage{color}
\usepackage{siunitx}
\usepackage{array}
\usepackage{multirow}
\usepackage{amssymb}
\usepackage{gensymb}
\usepackage{tabularx}
\usepackage{extarrows}
\usepackage{booktabs}
\usetikzlibrary{fadings}
\usetikzlibrary{patterns}
\usetikzlibrary{shadows.blur}
\usetikzlibrary{shapes}




\newcommand\intodo[1]{\textcolor{red}{#1}}
\newcommand\skeleton{\textcolor{gray}{\textit{This is a skeleton for this part. Redaction is on its way.\\}}}

% The \icmltitle you define below is probably too long as a header.
% Therefore, a short form for the running title is supplied here:
\icmltitlerunning{Robust Consensus Ranking}

\begin{document}

\twocolumn[
\icmltitle{Robust Consensus in Ranking Data Analysis:\\ Definitions, Properties and Computational Issues}

% It is OKAY to include author information, even for blind
% submissions: the style file will automatically remove it for you
% unless you've provided the [accepted] option to the icml2023
% package.

% List of affiliations: The first argument should be a (short)
% identifier you will use later to specify author affiliations
% Academic affiliations should list Department, University, City, Region, Country
% Industry affiliations should list Company, City, Region, Country

% You can specify symbols, otherwise they are numbered in order.
% Ideally, you should not use this facility. Affiliations will be numbered
% in order of appearance and this is the preferred way.
\icmlsetsymbol{equal}{*}

\begin{icmlauthorlist}
\icmlauthor{Morgane Goibert}{yyy,comp}
\icmlauthor{Clément Calauzènes}{yyy}
\icmlauthor{Ekhine Irurozki}{comp}
\icmlauthor{Stéphan Clémençon}{comp}

%\icmlauthor{}{sch}
%\icmlauthor{}{sch}
%\icmlauthor{}{sch}
\end{icmlauthorlist}

\icmlaffiliation{yyy}{Criteo AI Lab, France}
\icmlaffiliation{comp}{Télécom Paris, France}

\icmlcorrespondingauthor{Morgane Goibert}{morgane.goibert@gmail.com}

% You may provide any keywords that you
% find helpful for describing your paper; these are used to populate
% the "keywords" metadata in the PDF but will not be shown in the document
\icmlkeywords{Machine Learning, ICML}

\vskip 0.3in
]

% this must go after the closing bracket ] following \twocolumn[ ...

% This command actually creates the footnote in the first column
% listing the affiliations and the copyright notice.
% The command takes one argument, which is text to display at the start of the footnote.
% The \icmlEqualContribution command is standard text for equal contribution.
% Remove it (just {}) if you do not need this facility.

%\printAffiliationsAndNotice{}  % leave blank if no need to mention equal contribution
\printAffiliationsAndNotice{} % otherwise use the standard text.


\begin{abstract}
%The level of delegation to be granted to AI systems will in particular heavily depend on how methodological research replies to questions of robustness. This naturally brings us back to the development of statistical learning techniques that are reliable even in presence of partly contaminated data, due to biases in measurements or the deliberate intention to impair the operation of the automated system. Preference data, observed in the form of (complete) rankings in the simplest situations, are no exception of course and the demand for appropriate concepts and tools is all the more pressing given that technologies fed by or producing this type of data (\textit{e.g.} search engines, recommending systems) are now massively deployed. The lack of vector space structure for the set of rankings (\textit{i.e.} the symmetric group $\mathfrak{S}_n$) and the very complex nature of statistics usually considered in ranking data analysis make the formulation of robustness objectives in this domain extremely challenging. In this paper, we introduce notions of robustness, together with dedicated statistical methods, for \textit{Consensus Ranking}, the flagship problem in ranking data analysis, aiming at summarizing a probability distribution on $\mathfrak{S}_n$ by a \textit{median} ranking. Precisely, we propose specific extensions of the popular concept of \textit{breakdown point}, tailored to consensus ranking, and address the related computational issues. Beyond the theoretical contributions, the relevance of the approach proposed is supported by a detailed experimental study.

As the issue of robustness in AI systems becomes vital, statistical learning techniques that are reliable even in presence of partly contaminated data have to be developed. Preference data, in the form of (complete) rankings in the simplest situations, are no exception and the demand for appropriate concepts and tools is all the more pressing given that technologies fed by or producing this type of data (\textit{e.g.} search engines, recommending systems) are now massively deployed. However, the lack of vector space structure for the set of rankings (\textit{i.e.} the symmetric group $\mathfrak{S}_n$) and the complex nature of statistics considered in ranking data analysis make the formulation of robustness objectives in this domain challenging. In this paper, we introduce notions of robustness, together with dedicated statistical methods, for \textit{Consensus Ranking} the flagship problem in ranking data analysis, aiming at summarizing a probability distribution on $\mathfrak{S}_n$ by a \textit{median} ranking. Precisely, we propose specific extensions of the popular concept of \textit{breakdown point}, tailored to consensus ranking, and address the related computational issues. Beyond the theoretical contributions, the relevance of the approach proposed is supported by an experimental study.

\end{abstract}

\section{Introduction}
Object detection~\cite{fasterrcnn, ssd, yolo, fcos} is a fundamental computer vision task, aiming to localize and recognize the objects of predefined categories in an image. Owing to the rapid development of deep neural networks (DNN)~\cite{vgg,resnet,densenet,mobilenets,googlenet1,googlenet2,googlenet3}, the detection performance has been significantly improved in the past decade. During the evolution of object detectors, one important trend is to remove the hand-crafted components to achieve end-to-end detection.

One hand-crafted component in object detection is the design of training samples. For decades, anchor boxes have been dominantly used in modern object detectors such as Faster RCNN~\cite{fasterrcnn}, SSD~\cite{ssd} and RetinaNet~\cite{focalloss}. However, the performance of anchor-based detectors is sensitive to the shape and size of anchor boxes. To mitigate this issue, anchor-free~\cite{fcos,foveabox} and query-based~\cite{detr,deformdetr,dynamicdetr,conditionaldetr} detectors have been proposed to replace anchor boxes by anchor points and learnable positional queries, respectively.

Another hand-crafted component is non-maximum suppression (NMS) to remove duplicated predictions. The necessity of NMS comes from the one-to-many (o2m) label assignment~\cite{primesample,ota,atss,paa,gfocal,gfocalv2}, which assigns multiple positive samples to each GT object during the training process. This can result in duplicated predictions in inference and impede the detection performance. Since NMS has hyper-parameters to tune and introduces additional cost, NMS-free end-to-end object detection is highly desired.

\begin{figure}[tbp]
    \centering
    \includegraphics[width=0.4\textwidth]{fig_in_intro.pdf}
    \caption{The positive and negative weights of different anchors (A, B, C and D) in the classification loss during early and later training stages. Each anchor has a positive loss weight $t$ (in orange color) and a negative loss weight $1-t$ (in blue color). In our method,  A is a fully positive anchor, D is a fully negative anchor, and B and C are ambiguous anchors. One can see that for o2o and o2m label assignment schemes, the weights for all anchors are fixed during the training process, while for our o2f scheme, the weights for ambiguous anchors are dynamically adjusted.}
    \label{fig_in_intro}
    \vspace{-3mm}
\end{figure}

With a transformer architecture, DETR~\cite{detr} achieves competitive end-to-end detection performance. Subsequent studies~\cite{poto,onenet} find that the one-to-one (o2o) label assignment in DETR plays a key role for its success. Consequently, the o2o strategy has been introduced in fully convolutional network (FCN) based dense detectors for lightweight end-to-end detection. However, o2o can impede the training efficiency due to the limited number of positive samples. This issue becomes severe in dense detectors, which usually have more than 10k anchors in an image. What’s more, two semantically similar anchors can be adversely defined as positive and negative anchors, respectively. Such a ‘label conflicts’ problem further decreases the discrimination of feature representation. As a result, the performance of end-to-end dense detectors still lags behind the ones with NMS. Recent studies~\cite{dndetr,groupdetr,hybriddetr} on DETR try to overcome this shortcoming of o2o scheme by introducing independent query groups to increase the number of positive samples. The independency between different query groups is ensured by the self-attention computed in the decoder, which is however infeasible for FCN-based detectors.

In this paper, we aim to develop an efficient FCN-based dense detector, which is NMS-free yet end-to-end trainable. We observe that it is inappropriate to set the ambiguous anchors that are semantically similar to the positive sample as fully negative ones in o2o. Instead, they can be used to compute both positive and negative losses during training, without influencing the end-to-end capacity if the loss weights are carefully designed. Based on the above observation, we propose to assign dynamic soft classification labels for those ambiguous anchors. As shown in Fig.~\ref{fig_in_intro}, unlike o2o which sets an ambiguous anchor (anchor B or C) as a fully negative sample, we label each ambiguous anchor as partially positive and partially negative. The degrees of positive and negative labels are adaptively adjusted during training to keep a good balance between ‘representation learning’ and ‘duplicated prediction removal’. In particular, we begin with a large positive degree and a small negative degree in the early training stage so that the network can learn  the feature representation ability more efficiently, while in the later training stage, we gradually increase the negative degrees of ambiguous anchors to supervise the network learning to remove duplicated predictions. 
We name our method as a one-to-few (o2f) label assignment since one object can have a few soft anchors. We instantiate the o2f LA into dense detector FCOS, and our experiments on COCO~\cite{coco} and CrowHuman~\cite{crowdhuman} demonstrate that it achieves on-par or even better performance than the detectors with NMS.

\section{
%Setting 
Framework and Problem Statement}
\label{sec:setting}

We start with a reminder of key concepts in ranking data analysis and \textit{Robust Statistics}. The interested reader can refer to \citet{AY14,Huber} for more details. Here and throughout,
%\paragraph{Notation.} A 
a ranking over a set of $n\geq 1$ items is represented as a permutation $\sigma \in \pS$ where $\pS$ is the symmetric group. By convention, the rank $r$ of an item $i\in[n]$ is $r=\sigma(i)$. For any measurable space $\cX$, $\cM^1_+(\cX)$ is the set of probability measures on $\cX$, ${\rm TV}(p,q)$ the total variation distance between $p$ and $q$ in $\cM^1_+(\cX)$.

\subsection{Ranking Data and Summary Statistics}

The descriptive analysis of probability distributions, or datasets for their empirical counterparts, is a fundamental problem in statistics. For distributions on Euclidean spaces such as $\lR^d$, this problem has been widely studied and covered by the literature, with the study of statistics ranging from the simplistic sample mean to more sophisticated data functionals, such as $U/L/R/M$-statistics or depth functions for instance \cite{vdV98}. 

Defining similar notions for probability distributions on $\pS$, the space of rankings, is challenging due to the absence of vector space structure. However, fueled by the recent surge of applications using preference data, such as \textit{e.g.} recommender systems, the statistical analysis of ranking data has recently regained attention and certain classic problems have been revisited, as for instance those related to consensus rankings and their generalization ability (see \textit{e.g.}~\citet{Korba2017} and the references therein) or to the extension of depth functions to ranking data \cite{goibert2022depthranking}.

\paragraph{Central tendency or location.} Statistics measuring centrality, such as the mean (or the median for univariate distribution), can be seen as barycenters of the sampling observations w.r.t a certain distance. Consensus Ranking / Ranking Aggregation extends this idea to probability distributions on $\pS$ \cite{Deza}. Given a (pseudo-)metric $d$ defined on $\pS$ and a distribution $p\in\distribs$, a \emph{ranking median} $\sigma^{\rm med}_{p,d}\in\pS$ can be defined as
\begin{align}
    \label{eq:ranking_median}
    \sigma^{\rm med}_{d}(p) := \argmin_{\sigma \in \pS} \lE_{\Sigma \sim p}(d(\sigma, \Sigma)).
\end{align}
A well-studied instance of ranking median is the \emph{Kemeny consensus}, which corresponds to the situation where $d$ is the \emph{Kendall Tau} distance: for all $\sigma,\;  \nu$ in $\pS$,
\begin{align}
    d_{\tau}(\sigma, \nu) = \frac{2}{n(n-1)}\sum_{i<j}\indicator{\sigma(i)<\sigma(j)}\indicator{\nu(i) > \nu(j)}
    \label{eq:kendall_tau}
\end{align}
Another common choice is the \emph{Borda count} when $d$ is the \emph{Spearman Rho}, see \cref{app:additional_metrics} for more details. %\clem{Add something about median scalability.}
Moreover, when $d$ is the Kendall tau, Borda is a $O(n \log n)$, 5-approximation of the Kemeny ranking~\cite{Caragiannis2013,JKS16,Coppersmith:2010}, which is a NP-hard to compute~\cite{Dwork}. 
%Such median is a simple, yet effective, method to define \emph{location} for the ranking distribution w.r.t. the metric $d$. 



\paragraph{More complex statistics based on ranking data.} Often, the information carried by a location statistic must be complemented. For instance, a notion of \emph{dispersion} or \emph{shape} is generally key to assessing convergence results or building confidence regions. To this end, the notion of \emph{statistical depth function} has been developed for multivariate data (in Euclidean spaces) (see \cite{ZuoSerfling00} and the references therein) and recently adapted to ranking, refer to \cite{goibert2022depthranking}.
However, as more complex statistics are more likely to exhibit robustness issues, we focus on simple statistics estimating location for ranking distribution.

\subsection{Robust Statistics}

To evaluate the robustness of a statistic, the notion of \textit{breakdown function} has been introduced in the seminal work of \cite{huber1964robust}. Informally, the breakdown function for a statistic $T$ on a distribution $p$ measures the minimal attack budget required for an adversarial distribution to change the outcome of the statistic $T$ by an amount at least $\delta > 0$. 

% \begin{definition}
%     {\sc{Breakdown Point}.} The breakdown point of level $\delta$, for distance on rankings $d$ is denoted by $\varepsilon^\star_{\delta, d}: \Delta^{\frak{S}_n} \times \mathcal{T} \to \mathbb{R}_{+}$ and defined by:

%     $$ \varepsilon^\star_{\delta, d}(p,T) = \inf \left\{ \varepsilon > 0 \, | \, \sup_{q | TV(p,q) \leq \varepsilon} d(T(p), T(q)) \geq \delta \right\} $$
% \end{definition}

\begin{restatable}{definition}{defbreakdownfunction}\label{def:breakdown_function}
    {\sc{(Breakdown Function)}} Let $\cX$ and $\cY$ be measurable spaces, $p\in\cM^1_+(\cX)$, $T: \cM^1_+(\cX) \to \cY$ a measurable function and $d$ a metric on $\cY$.
    For any level $\delta \geq 0$, the breakdown function of the functional $T$ at $p$ is 
    \begin{align*}
        \varepsilon^\star_{d, p,T}(\delta) = \inf \left\{ \varepsilon > 0 \, \middle| \, \sup_{q : {\rm TV}(p,q) \leq \varepsilon} d(T(p), T(q)) \geq \delta \right\}.
    \end{align*}
\end{restatable}
In the traditional case $\cX=\cY=\mathbb{R}$, the level $\delta$ is generally set to $+\infty$ and the budget required is referred to as \emph{breakdown point}. 
In the extreme case, when $T$ is the identity and $\delta = 0^+$, $\varepsilon^\star$ quantifies the budget of attack under which \emph{identifiability} of the distribution is possible (which requires the additional knowledge that $p$ belongs to some family). 


\paragraph{Application to Ranking Data.} In \citet{agarwal2020rank} such a study on identifiability is provided for the Bradley-Terry-Luce \cite{BT1952, Luce59} model under a budget constraint on pairwise marginals rather than the Total Variation, and \citet{jin2018} on the Heterogeneous Thurstone Models \cite{thurstone1927}.
However, summary statistics, such as a central tendency, are generally harder to break than the full distribution itself, so the breakdown function provides a finer quantification of robustness than the identifiability of the distribution.
Since the distances on $\pS$ are bounded, in general, the full breakdown function needs to be considered and one cannot focus only on a particular level such as $\delta = 0^+$ or $\delta = +\infty$. From here and throughout, the distance $d$ and the attack amplitude $\delta$ are normalized to lie between $0$ and $1$. 

The robustness of the median statistic when an adversary is allowed to attack with any strategy a pairwise model has also been studied ~\cite{datar2022byzantine}. They characterize the robustness of two statistics in terms of the L2 distance on distributions. We propose in Definition~\ref{def:breakdown_function} a more general and natural measure for robustness as a function of  the distance between the true and a corrupted statistic. 

\paragraph{Bucket Rankings as a robustness candidate.} In rankings, adversarial attacks often target pairs of items that are ``close" in some sense \cite{agarwal2020rank}: consecutive ranks, a pairwise marginal probability close to $\frac{1}{2}$, \ldots Thus, a simple and efficient way to robustify a ranking median is to accept \emph{ties}, rather than being restricted to a strict order.



\subsection{Challenges and Contributions}



% Robustness of ranking medians -- breakdown function (CONTEXT)
% Often no closed-form for the median (CHALLENGE)
% General lower bound + upper bound for Kemeny (CONTRIBUTION)
There is a wide number of median statistic studies motivated by the lack of analytical expression and the computational and statistical challenges that arise in the estimation process.
However, robustness results for ranking statistics are rare and not rigorous enough for comparing different estimators. % or even studying the most prominent rule, Kemeny. 


\begin{contribution}
Using \cref{def:breakdown_function} with the Kendall tau distance provides a straightforward measure of robustness for ranking medians.
In \cref{sec:bd_kem} we provide a lower-bound on the breakdown function for a ranking median (\cref{thm:ubbreakdownfunctionmedian}) and a tight upper-bound for the Kemeny consensus (\cref{thm:ubbreakdownfunctionmedian}). %We conclude that there is not one median that is universally more robust than another.
\end{contribution}


% Extension of breakdown functions to bucket rankings (CONTEXT)
% Need to extend metrics and to handle the non-continuity of the objective (CHALLENGE)
% An empirical estimator for the breakdown function (CONTRIBUTION)
Moreover, slight perturbations in the pairwise relations of items that are similar to each other can imply breaking a median estimator, showing a lack of robustness. It is natural to propose more robust estimators by allowing pairs of items to be ``equally ranked", i.e., by considering bucket ranking statistics. However, generalizations of the breakdown function for bucket rankings require the use of Kendall tau for buckets, which is  computationally impractical. 

\begin{contribution}
In \cref{sec:buckets} we propose an extension of the breakdown function for bucket rankings which is built upon a Hausdorff generalization of the Kendall tau distance. We also develop an optimization algorithm to approximate this breakdown function that overcomes the computational issue of having a piece-wise constant objective function.
%for which we propose a computationally efficient algorithm. The close form expression is difficult to obtain in general since it requires optimizing a piece-wise constant function. However, we propose an empirical estimator building on Lagrangian relaxation to overcome the computational issues. 
\end{contribution}


% How to robustify classical ranking medians with bucket rankings (CONTEXT)
% Direct extension of notion of median to \wO is intractable (CHALLENGE)
% A plug-in method to robustify ranking medians backed with XP (CONTRIBUTION)

We illustrate and show empirically that bucket rankings are more robust median estimators than rankings. However, finding the optimal bucket order statistic requires exhaustively searching the space of bucket rankings $\wO$, which is even larger than the space of permutations, of factorial cardinality, and therefore, it is totally infeasible. 

\begin{contribution}
In \cref{sec:our_stats} we propose a general method for robustifying medians: given a ranking median, our algorithm successively merges ``similar'' items together into the same bucket. 
We evaluate this statistic in \cref{sec:exps}, showing an improvement of robustness w.r.t. Kemeny's median without sacrificing its precision. 
\end{contribution}




% \subsection{Main Contributions}

%  We develop our framework both theoretically and experimentally, and provide a method to evaluate said robustness in practice.

% Our main contributions are thus the following:
% \begin{itemize}
%     \item We extend the notions of distances and statistics to bucket rankings.
%     \item We provide a definition and a practical algorithm to estimate robustness for summary statistics that output a bucket ranking.
%     \item We provide an in-depth comparison of different relevant statistics, both in terms of accuracy and robustness.
% \end{itemize}


% \subsection{Accuracy and Robustness of Statistics}
% Several criteria are usually considered to compare summary statistics. 

% \begin{definition}
%     {\sc{Accuracy}.} Let $d: \frak{S}_n \to \mathbb{R}$ be a (pseudo-) distance. The accuracy of a statistic $T$ is defined by:

%     $$ \text{Acc}_{p,d}(T) := - \mathbb{E}_{\Sigma \sim p}(d(T(p), \Sigma)) $$
% \end{definition}

% By definition, the ERM statistic is optimal with respect to the accuracy criteria.

% However, the evolution of the applications using preference data have shed lights on the problems raised by malevolent manipulation of data (e.g. fake comments), which justify the need for \textit{robust} statistics.

% In $\mathbb{R}$, robust statistics have been studied since the seminal works of Huber, in which the classical attacks consists of \textit{adversarial attacks}, and the classical way to evaluate the robustness of a statistic is to compute its \textit{breakdown point.}

% In the field of rankings, very few works have started to analyze statistics through the scope of robustness. Agarwal \intodo{Add limitations}. Depth functions paper \intodo{Add limitations}.



%From the representation compactness point of view, bucket rankings are similar to rankings (strict orders), as they can be encoded by a vector of $n$ scores that generates the order by sorting them, one per item, where ties are represented by equal scores. 

%In terms of expressiveness to summarize distributions over permutations, bucket orders have a strong advantage over strict orders that goes further than the fact that strict orders are a particular case of bucket orders. A bucket order $\pi \in \wO$ can be seen as a uniform distribution over a subset of permutations (those generated by breaking ties of $\pi$ uniformly at random). Because $|\wO| \gg |\pS|$, bucket orders are providing a much finer way to grid the simplex $\Delta^\pS$ than strict orders which only lies at the corners. Hence, as a summary statistic, bucket orders can be more precise (closer to the summarized distribution). \clem{I'm not happy with this paragraph. The idea is here, but badly explained.}

\section{Robustness - Breakdown Function for Ranking and Bucket Rankings}
\label{sec:robustness}

This section first details how to apply the notion of \textit{breakdown function} $\varepsilon^\star_{d, p, T}$. This allows providing insights into the robustness of classical location statistics such as the Kemeny consensus. These results advocate for the introduction of a more robust type of statistics based on bucket orders that are also developed in this section.

\subsection{Breakdown Function for the Kemeny Consensus}\label{sec:bd_kem}

We explore the robustness of ranking medians $\sigma^{\rm med}_{d}(p)$ as defined in \cref{eq:ranking_median} for different metrics $d$ over $\pS$ as defined by the breakdown function $\varepsilon^\star_{d_\tau, p, T}$. In particular, it is possible to tightly sandwich the breakdown function for the Kemeny median.

\begin{restatable}{theorem}{thmbreakdownfunctionkemeny}\label{thm:breakdownfunctionkemeny}
For $p\in\cM_+^1(\pS)$, ~ $\sigma^\star_p = \sigma^{\rm med}_{d_\tau}(p)$ (Kemeny median) and $\delta\geq0$, if $\varepsilon^+(\delta) \leq 2 p(\sigma_{p}^{*})$ then $\varepsilon^\star_{d_\tau, p,\sigma^\star_p}(\delta) \leq \varepsilon^+(\delta)$ with
    \begin{align*}
        \varepsilon^+(\delta) =
    \min_{\substack{\sigma \in\pS \\ d_\tau(\sigma, \sigma^\star_p) \geq \delta}} 
    \max_{\substack{\nu\in\pS \\ d_\tau(\nu, \sigma^\star_{p}) < \delta}} 
    \frac{\lE_{\Sigma\sim p}\left[d_\tau(\Sigma, \sigma) - d_\tau(\Sigma, \nu)\right]}
    {d_\tau(\sigma^\star_p, \sigma) - d_\tau(\sigma^\star_p, \nu) }\,.
    \end{align*}
\end{restatable}
\begin{proofsketch} Detailed Proof can be found in \cref{app:breakdown_function_kemeny_ub}
The proof relies on showing that, for $\varepsilon>0$, the \emph{attack} distribution $\bar{q}_\varepsilon = p - \frac{\varepsilon}{2}\indicator{\cdot = \sigma^*_p} + \frac{\varepsilon}{2}\indicator{\cdot = \sigma^{\star, {\rm rev}}_p}$, where $\sigma^{\star, {\rm rev}}_p$ is the reverse of $\sigma^\star_p$, is in the feasible set of the optimization problem $\sup_{q: \textsc{tv}(p,q)\leq \varepsilon}d_\tau(\sigma^*_p, \sigma^*_{q})$ (see \cref{def:breakdown_function}). 

Using $\bar{q}_\varepsilon$ provides a way to link $\varepsilon$ and $\delta$.
The condition $\varepsilon^+(\delta) \leq 2 p(\sigma^\star_p)$ ensures $\bar{q}_\varepsilon$ is well-defined.
\end{proofsketch}

It is also possible to provide a lower bound on the breakdown function for any generic ranking median.

\begin{restatable}{theorem}{thmubbreakdownfunctionmedian}\label{thm:ubbreakdownfunctionmedian}
For $p\in\cM_+^1(\pS)$, $m$ and $d$ being two metrics on $\pS$, ~ $\sigma^\star_p = \sigma^{\rm med}_{d}(p)$ and $\delta\geq0$, we have $\varepsilon^\star_{m, p,\sigma^\star_p}(\delta) \geq \varepsilon^-(\delta)$ with
    \begin{align*}
        \varepsilon^-(\delta) =
    \min_{\substack{\sigma \in\pS \\ m(\sigma, \sigma^\star_p) \geq \delta}} 
    \max_{\substack{\nu\in\pS \\ \nu \neq \sigma}} 
    \frac{\lE_{\Sigma\sim p}\left[d(\Sigma, \sigma) - d(\Sigma, \nu)\right]}
    {\max_{\sigma'\in\pS} d(\sigma', \sigma) - d(\sigma', \nu)}
    \end{align*}
\end{restatable}
\begin{proof}
Detailed proof can be found in \cref{app:breakdown_function_median_lb}.
\end{proof}


\begin{figure}[htbp]
    \centering
    \includegraphics[width=0.45\textwidth]{img/bounds_breakdown_borda_footrule_kemeny.pdf}
    \caption{An illustration of $\varepsilon^+(\delta)$ and $\varepsilon^-(\delta)$ (from \cref{thm:breakdownfunctionkemeny} and \cref{thm:ubbreakdownfunctionmedian}) for a distribution on permutations of 4 items. For Borda and the median associated with Spearman footrule, only the lower bound is displayed.}
    \label{fig:breakdown_of_medians}
\end{figure}
\cref{fig:breakdown_of_medians} shows that no choice of $d$ makes the median uniformly more robust than another. Then, unfortunately, it also illustrates the fragility of median statistics against corruption of the distribution. In this example, impacting the distribution $p$ by less than $5\%$ allows changing the Kemeny median by flipping more than half item pairs ($\delta \geq 0.5$).

\paragraph{Sensitivity to similar items.} 
To further illustrate the fragility of Kemeny's median, \cref{fig:kemeny_on_indifference} shows its breakdown function on specific distributions. As could be expected, if all items are almost indifferent (uniform distribution - purple curve), then a ranking median is very fragile: a small nudge on $p$ is enough to change the Kemeny median from one ranking to its reverse. On the contrary, when $p$ is a point mass at a given ranking (blue curve), it requires a large attack on $p$ to impact the median. 

The green curve shows a weakness in the median: despite $p$ being concentrated on two neighbouring rankings (identical up to a pair of adjacent items), the robustness is very low for $\delta \leq 0.2$. This highlights a mechanism underlying adversarial attacks in real-world recommender systems (ex: fake reviews...): at a small cost, it is possible to be systematically ranked on top of close alternatives. This calls for using the natural alternative to (strict) rankings, which incorporates indifference between items: \emph{bucket rankings}.

\begin{figure}[htbp]
    \centering
    \includegraphics[width=0.45\textwidth]{img/theoretical_kemeny.pdf}
    \caption{Breakdown function for Kemeny's median for different distributions $p$. "Uniform" denotes an almost uniform distribution; "Point mass" an almost point mass distribution, and "Bucket" an almost point mass distribution on two neighboring rankings.}
    \label{fig:kemeny_on_indifference}
\end{figure}


%\section{Robustness with Bucket Rankings}
\subsection{Bucket Ranking - Extended Ranking Consensus}\label{sec:buckets}

Intuitively, bucket rankings are rankings with ties allowed. Formally, they can equivalently be defined as a total preorder -- \emph{i.e.} a homogeneous binary relation that satisfies transitivity and reflexivity (preorder) in which any two elements are comparable (total) -- or as a strict weak ordering -- \emph{i.e.} a strict total order over equivalence classes of items (buckets).

\begin{restatable}{definition}{defbucketrankings}\label{def:bucket_rankings}{\sc (Bucket ranking)}
A bucket order $\pi$ is a strict weak order defined by an ordered partition of $[n]$, \emph{i.e.} a sequence $(\pi^{(1)}, \dots , \pi^{(k)})$ of $k \geq 1$ pairwise disjoint non empty subsets (buckets) of $[n]$ such that: 
\begin{enumerate}
    \item[(i)] $i \prec_\pi j ~~\Leftrightarrow~~ \exists l<l' \in [k], (i,j) \in \pi^{(l)} \times \pi^{(l')}$,  
    \item[(ii)] $i \sim_\pi j ~~\Leftrightarrow~~ \exists l \in [k], (i,j) \in \pi^{(l)}\times\pi^{(l)}$,  
\end{enumerate}
We denote $\wO$ the set of bucket rankings, which is of size $\sum_{k=1}^n k! S(n,k)$\footnote{$S(n,k)$ are Stirling numbers of the second kind.} (vs $n!$ for $\pS$).
\end{restatable}
The indifference between items that bucket rankings can incorporate is an interesting feature to gain robustness, because the statistic can output alternatives between several strict orders, making it harder to attack.


\paragraph{As sets of permutations.} A bucket ranking $\pi\in\wO$ can be equivalently mapped to a subset of permutations, generated through the different ways to break ties. We say that a permutation $\sigma\in\pS$ is \emph{compatible} with a bucket ranking $\pi\in\wO$ -- denoted $\sigma\in\pi$ -- if for any $i,j\in[n]$, $\sigma(i)<\sigma(j) ~~\Leftrightarrow~~ i\prec_\pi j$ or $i\sim_\pi j$. For two bucket orders $\pi_1, \pi_2$, we say that $\pi_1$ is \emph{stricter} that $\pi_2$, denoted $\pi_1 \subseteq \pi_2$, iff for any $\sigma\in\pS, ~~\sigma\in\pi_1 \Rightarrow \sigma\in\pi_2$. 

\paragraph{As a distribution.} Being a set of permutations, a bucket order $\pi\in\wO$ can also be seen as a uniform distribution with restricted support. This point of view is particularly intuitive from a robustness perspective: a randomized output is generally harder to attack for an adversary.


\paragraph{Distances between bucket rankings.}
A key to applying the breakdown function from \cref{def:breakdown_function} to bucket orders statistics is to have a metric on $\wO$ that extends those defined on $\pS$.
To this end, we use the previous remark that weak orders are sets of rankings as well as a classical Hausdorff extension of metrics to sets. More precisely, we define:

\begin{restatable}{definition}{defnonsymmetrichausdorff}\label{def:non_symmetric_hausdorff}
    {\sc{(Non-symmetric Hausdorff)}} Let $d$ be a metric on $\pS$. The non-symmetric Hausdorff pseudoquasi-metric between two bucket rankings $\pi_1, \pi_2\in \wO$ is 
    \begin{align*}
        H^\text{\sc ns}_d(\pi_1, \pi_2) = \max_{\sigma_2 \in \pi_2} \min_{\sigma_1 \in \pi_1} d(\sigma_1, \sigma_2)  \,.      
    \end{align*}
\end{restatable}

Even though it is not a metric, $H^\text{\sc ns}_d$ is well-suited to ranking with ties. Intuitively, its lack of symmetry allows differentiating adversarial attacks whose effect is on the strict part of the bucket order (e.g. swapping two items that are strictly ordered) from those whose effect is "only" to disambiguate a tie. More precisely, if $\pi_2 \subseteq \pi_1$, then $H^\text{\sc ns}_d(\pi_1, \pi_2) = 0$. Depending on the application, one may want to focus on the first type of attacks, in which case $H^\text{\sc ns}_d$ is a suitable choice to define the breakdown function as $\varepsilon^\star_{H^\text{\sc ns}_d, p, T}$.
Otherwise, it is possible (and usual) to symmetrize the Hausdorff metric.
\begin{restatable}{definition}{defsymmetrichausdorff}\label{def:symmetric_hausdorff}
    {\sc{($1/2$-symmetric Hausdorff)}} Let $d$ be a metric on $\pS$. The $1/2$-symmetric Hausdorff metric between two bucket rankings $\pi_1, \pi_2\in \wO$ is defined by
    \begin{align*}
        H^{(1/2)}_d(\pi_1, \pi_2) = \frac{1}{2} \Big(H^\text{\sc ns}_d(\pi_1, \pi_2) + H^\text{\sc ns}_d(\pi_2, \pi_1)\Big)\,.
    \end{align*}
\end{restatable}
Usual symmetrization of the Hausdorff metric uses a maximum rather than an average \cite{fagin2006comparing}. However, under the Kendall-tau distance, the average version is computationally simpler (see \cref{app:hausdorff_kendall} for more details).

\subsection{The Breakdown Function in Ranking Data Analysis - Definition and Estimation}

\paragraph{Definition.}
Putting all the pieces together, from now on, the statistic $T : \cM_+^1(\pS) \to \wO$ 
 summarizes a distribution over $\pS$ by a bucket ranking in $\wO$. Then, we use either $H^{(NS)}_{d_\tau}(\pi_1, \pi_2)$ (see \cref{def:non_symmetric_hausdorff}) or $H^{(1/2)}_{d_\tau}(\pi_1, \pi_2)$ on $\wO$ where $d_\tau$ is the Kendall tau (see \cref{eq:kendall_tau}). Finally, the breakdown function $\varepsilon^\star_{H^{(NS)}_{d_\tau}, p, T}$ is the result of the following optimization problem
 \begin{align}
     \inf \left\{ \varepsilon > 0 \, \middle| \, \sup_{q : {\rm TV}(p,q) \leq \varepsilon} H^{(NS)}_{d_\tau}(T(p), T(q)) \geq \delta \right\}
     \label{eq:breakdown_fct_bucket}
 \end{align}

\paragraph{The Empirical Breakdown Function.}
Computing a closed-form expression for the breakdown point for any statistic $T$ and distribution $p$ is challenging in general. However, it can be estimated empirically: the extended expression of the breakdown function in \cref{eq:breakdown_fct_bucket} can be simplified so that it is the solution to the following Lagrangian-relaxed optimization problem.
\begin{equation}
    \inf_{q \in \Delta^{\frak{S}_n}} \sup_{\lambda \geq 0} 1/2 \| p-q \|_1 + \lambda(\delta - H^{(NS)}_{d_\tau}(T(p), T(q)) )
    \label{eq:lagrangian_relax_bkdwn}
\end{equation}
\paragraph{Smoothing.} As $H^{(NS)}_{d_\tau}(T(p), T(q)))$ is piece-wise constant as a function of $q$ (with a combinatorial number of pieces), Problem \eqref{eq:lagrangian_relax_bkdwn} cannot directly be solve using standard optimization techniques.
To solve this issue, we used a smoothing procedure by convolving this function with a smoothing kernel $k_\gamma$ with scale $\gamma$. Thus, after the relaxation, the optimization problem \eqref{eq:lagrangian_relax_bkdwn} becomes:
\begin{equation}
    \inf_{q \in \Delta^{\frak{S}_n}} \sup_{\lambda \geq 0} 1/2 \| p-q \|_1 + \lambda(\delta - \rho_T(p, q) ),
    \label{eq:smoothing_version_bkdwn}
\end{equation}
with 
\begin{equation}
    \begin{split}
        \rho_T(p,q) &= H^{(NS)}_{d_\tau}(T(p), T(q)) \star k_{\gamma}(q) \\
        &= \int_{u} H^{(NS)}_{d_\tau}(T(p), T(u)) \times k_{\gamma}(q-u) \text{d}u,
    \end{split}
\end{equation}
On a practical note, a simple way to build a convolution kernel $k_\gamma$ on a simplex like $\distribs$, is to use a convolution kernel $\kappa_\gamma$ on the whole euclidean space -- for instance an independent Gaussian density $\kappa_{\gamma}(x) = \frac{1}{\sqrt{ (2 \pi \gamma)^{n!} }} \exp{( -\frac{x^{\text{T}} x }{2 \gamma^2} )}$ 
-- and set $k_\gamma$ to be the density of the push-forward through a \emph{softmax} function. We denote $\varepsilon^\gamma_{p,T}(\delta)$ the limiting value of $\|p-q\|_1/2$ at the solution of \eqref{eq:smoothing_version_bkdwn}. Note the bias induced by such definition of $k_\gamma$ fades away when $\gamma$ goes to $0$ in the same way as the bias induced by the convolution.
This smoothing ensures $\rho_T$ is a continuous, differentiable function with respect to $q$. Moreover, it can easily be estimated using a Monte-Carlo sampling, using the following remark: $\rho_T(p,q) = \mathbb{E}_{u \sim k_{(p, \gamma)}} (H^{(NS)}_{d_\tau}(T(u), T(q))$.

% \paragraph{Complexity.} Solving \eqref{eq:smoothing_version_bkdwn} empirically is possible, but very costly, as the objective requires $n!$ computation just to be evaluated. However, this cost comes from computing $\|p-q\|_1$ as it is possible to reformulate $H_{d_\tau}^{(1/2)}$ so it has a much lower computational cost as states the following:
% \begin{restatable}{proposition}{propcomplexityhausdorffkendall}\label{prop:complexity_hausdorff_kendall}
% For any $\pi_1, \pi_2\in\wO$, the computation cost of $H_{d_\tau}^{\textsc{ns}}(\pi_1, \pi_2)$ and $H_{d_\tau}^{(1/2)}(\pi_1, \pi_2)$ is $\cO(n^2)$.
% \end{restatable}
% \begin{proof}
% Proof can be found in \cref{app:hausdorff_kendall}. It relies of reformulations inspired from the work of \citet{fagin2006comparing}.
% \end{proof}

\begin{comment}
As the second term in \eqref{eq:smoothing_version_bkdwn} can be computed efficiently, replacing the $\|\cdot\|_1$ budget constraint by one on pairwise marginal probabilities provides the following objective:
\begin{align}
    \label{eq:smoothing_pairwise_version_bkdwn}
    \inf_{q \in \Delta^{\frak{S}_n}} \sup_{\lambda > 0} \frac{1}{2} \| P\,\text{\scalebox{0.5}[1.0]{$-$}}\,Q \|_1 + \lambda(\delta \,\text{\scalebox{0.5}[1.0]{$-$}}\,\rho_T(p,q) ),
\end{align}
where $P\in[0,1]^{n^2}$ (resp $Q$) is the matrix of pairwise marginal probabilities of $p$ (resp $q$), defined as follows.


In the end, each Monte-Carlo sample to estimate of the objective requires $\cO(n^2 + C(T))$ computations, where $C(T)$ stands for the computational cost of the statistic $T$ and \cref{fig:proba_vs_pairwise} shows both ways to constrain the attack budget are well-correlated.


\begin{figure}[h]
\centering
\label{fig:proba_vs_pairwise}
\includegraphics[width=0.8\linewidth]{icml_submission/img/proba_vs_pairwise.png}
\caption{Relation between $TV(p_1,p_2)$ and $\|P_1-Q1\|_1$ (normalized), where distributions were generated using Plackett-Lice with random logits.}
\end{figure}
\end{comment}

\paragraph{Optimization.} When using Monte-Carlo estimation for $\rho_T$, \cref{eq:smoothing_version_bkdwn} is a stochastic saddle-point problem. To solve such problems, gradient/ascent has a rate of convergence of $\cO(t^{1/2})$ for its ergodic average ($t$ being the number of steps) \cite{Nemirovski2002}. %Recently, better-performing algorithms have also been proposed, such as the \emph{extra-gradient method} (e.g. \citet{hsieh2019}).
Our empirical optimization algorithm for computing the breakdown functions relies on stochastic gradient descent and is able to provide good approximations, as illustrated in \cref{fig:theory_exp_kemeny_maxpair}. We denote $\hat{\varepsilon}^\gamma_{p,T}(\delta) = \|p - \bar{q}_t\|_1$, where $\bar{q}_t$ is the ergodic average of the iterates $(q_s)_{s\leq t}$ obtained during the optimization.

Let's make a couple of remarks on the empirical breakdown function $\hat{\varepsilon}^\gamma_{p,T}$. First, it is a noisy estimate of $\varepsilon^\gamma_{p, T}$ as $\rho_T$ and its gradients are estimated via Monte-Carlo. Thus, the choice of $\gamma$ and $t$ should trade-off the variance of $\hat{\varepsilon}^\gamma_{p,T}$ and the bias $|\varepsilon^\gamma_{p, T}-\varepsilon^\star_{d_\tau, p, T}|$. Second, as the term $\|p - q\|_1$ is minimized in \eqref{eq:smoothing_version_bkdwn}, it is expected $\hat{\varepsilon}^\gamma_{p,T}$ over-estimates $\varepsilon^\gamma_{p, T}$.

% \begin{figure}[h]
% \centering
% \includegraphics[width=0.95\linewidth]{icml_submission/img/theory_exp_Kemeny.pdf}
% \caption{Placeholder -- Theorical bounds and empirical values for the breakdown function of Kemeny's median.}
% \label{fig:theory_exp_kemeny}
% \end{figure}



% \subsection{Definition of Breakdown Points}

% \intodo{all this sections needs to be placed somewhere else / removed}

% In the field of ranking data, very few works have studied adversarial attacks~\cite{datar2022byzantine, agarwal2020rank,jin2018, lu2012bayesian},  (\intodo{Add refs}). In this paper, we will focus on poisoning attacks, where the attacker is allowed to modify the original distribution $p$ by a budget $\varepsilon$, in order to fool the Ranking Aggregation statistic $T$.

% \begin{definition}
%     {\sc{Adversarial Distribution}.} An adversarial distribution against the probability distribution $p \in \Delta^{\frak{S}_n}$ for statistic $T$ with budget $\varepsilon$, shortened $(p, T, \varepsilon)$-adversarial distribution, is a probability distribution $q_{P, T, \varepsilon}$ such that:
    
%     1) the budget constraint is satisfied: $\text{TV}(p, q_{P, \varepsilon}) = \frac{1}{2} \sum_{\sigma \in \frak{S}_n} |p(\sigma) - q_{P, T, \varepsilon}(\sigma)| \leq \varepsilon$, where $\text{TV}$ is the total-variation distance, and $\varepsilon \geq 0$ is the attack budget.

%     2) If possible, $q_{p, T, \varepsilon}$ fools the statistics $T$: $T(p) \neq T(q_{p, T, \varepsilon})$.
% \end{definition}

% To improve readability, whenever the context is clear, the adversarial distribution $q_{p, T, \varepsilon}$ bill be simply denoted by $q$. Note also that since the ranking probability distribution space $\Delta^{\frak{S}_n}$ is discrete and finite, we have $TV(p,q) = 1/2 \sum_{\sigma \in \frak{S}_n} | p(\sigma) - q(\sigma) | = 1/2 \| p-q \|_1  $. \\


\section{Robust Consensus Ranking Statistics}
\label{sec:our_stats}

As proved by \cref{thm:breakdownfunctionkemeny}, the classical median statistics as defined by \eqref{eq:ranking_median} can be easily broken, which motivates defining more robust statistics, based on bucket rankings. As illustrated by \cref{fig:kemeny_on_indifference}, the weakness of median statistics comes from being ``forced" to rank all items, even those which are (almost) indistinguishable. Bucket rankings seem to be a natural solution to this problem, but \emph{what is a good way to build a bucket order statistic?}

As $H^{(NS)}_{d_\tau}$ defines a (pseudoquasi-) distance on $\wO$, we could adapt the idea of a median as in \eqref{eq:ranking_median} for bucket rankings. However, contrarily Borda medians which can be computed in a scalable way \cite{Caragiannis2013}, Hausdorff-based medians would require to optimize over $\wO$. As its cardinality is larger than $\frak{S}_n$ this problem can be more computationally challenging than Kemeny's median.

A more scalable approach is to start from a ranking median such as the Kemeny or Borda consensus and to robustify it using a plug-in method based on merging items that are close into buckets. \cref{fig:merge_limit} illustrates this idea. The left graph describes pairwise marginal probabilities for which the Kemeny consensus is $A\prec B\prec C\prec D$. Intuitively, merging either $C$ and $D$ (as $\lP(C \prec D) = 0.51)$ or $B$ and $C$ (as $\lP(B \prec C) = 0.52)$ leads to bucket rankings (i) and (ii), which will be harder to attack. However, this example also highlights that there is no unique way of merging items. For instance, if the constraint is to only merge items whose pairwise preference probability is in $[0.4, 0.6]$, it is possible to merge $B,C$ or $C,D$, but not $B,C,D$ as $\lP(B\prec D) = 0.7$: \emph{pairwise indistinguishability is not transitive}.


\begin{figure*}
\centering
\tikzset{every picture/.style={line width=0.75pt}} %set default line width to 0.75pt        

\tikzset{
    fullgraph/.pic = {
        % Text Node
        \node[draw,circle] (A) at (-50, 50) {A};
        \node[draw,circle] (B) at (50, 50) {B};
        \node[draw,circle] (C) at (-50, -50) {C};
        \node[draw,circle] (D) at (50, -50) {D};
        \draw [-stealth, color=black ,draw opacity=1 ][line width=1.5]    (A.east) -- node [above,midway] {$0.69$} (B.west) ;
        \draw [-stealth, color=black ,draw opacity=1 ][line width=3]    (A.south) -- node [left,midway] {$0.9$} (C.north) ;
        \draw [-stealth, color=black ,draw opacity=1 ][line width=3]    (A.south east) -- node [below left,near start,xshift=2] {$0.9$}(D.north west) ;
        \draw [-stealth, color=black ,draw opacity=1 ][line width=1.5]    (B.south) -- node [right, midway] {$0.7$}(D.north) ;
        \draw [-stealth, color=black ,draw opacity=1 ][dash pattern={on 4.5pt off 4.5pt}]    (B.south west) -- node [below right,near start,xshift=-5] {$0.52$}(C.north east) ;
        \draw [-stealth, color=black ,draw opacity=1 ][dash pattern={on 4.5pt off 4.5pt}]    (C.east) -- node [below,midway] {$0.51$}(D.west) ;
    },
    boi/.pic = {
        % Text Node
        \node[draw,circle] (A) at (-40, 50) {A};
        \node[draw,circle] (B) at (40, 50) {B};
        \node[draw={rgb, 255:red, 208; green, 2; blue, 27 },circle,text={rgb, 255:red, 208; green, 2; blue, 27 }] (CD) at (0, -20) {C,D};
        \node (i) at (0,-55) {(i)};
        \draw [-stealth, color=black ,draw opacity=1 ][line width=1.5]    (A.east) -- (B.west) ;
        \draw [-stealth, color=black ,draw opacity=1 ][line width=3]    (A.300) -- (CD.north west) ;
        \draw [-stealth, color=black ,draw opacity=1 ][line width=1.5]    (B.240) -- (CD.north east) ;
    },
    boii/.pic = {
        % Text Node
        \node[draw,circle] (A) at (-30, 50) {A};
        \node[draw,circle] (D) at (-30, -20) {D};
        \node[draw={rgb, 255:red, 208; green, 2; blue, 27 },circle,text={rgb, 255:red, 208; green, 2; blue, 27 }] (BC) at (30, 15) {B,C};
        \node (ii) at (0,-55) {(ii)};
        \draw [-stealth, color=black ,draw opacity=1 ][line width=3]    (A.330) -- (BC.150) ;
        \draw [-stealth, color=black ,draw opacity=1 ][line width=3]    (A.south) -- (D.north) ;
        \draw [-stealth, color=black ,draw opacity=1 ][line width=1.5]    (BC.210) -- (D.30) ;
    },
    boiii/.pic = {
        % Text Node
        \node[draw={rgb, 255:red, 208; green, 2; blue, 27 },circle,text={rgb, 255:red, 208; green, 2; blue, 27 }] (AB) at (0, 50) {A,B};
        \node[draw={rgb, 255:red, 208; green, 2; blue, 27 },circle,text={rgb, 255:red, 208; green, 2; blue, 27 }] (CD) at (0, -20) {C,D};
        \node (iii) at (0,-55) {(iii)};
        \draw [-stealth, color=black ,draw opacity=1 ][line width=3]    (AB.south) -- (CD.north) ;
    },
    boiv/.pic = {
        % Text Node
        \node[draw,circle] (A) at (0, 50) {A};
        \node[draw={rgb, 255:red, 208; green, 2; blue, 27 },circle,text={rgb, 255:red, 208; green, 2; blue, 27 }] (BCD) at (0, -20) {B,C,D};
        \node (iii) at (0,-55) {(iii)};
        \draw [-stealth, color=black ,draw opacity=1 ][line width=3]    (A.south) -- (BCD.north) ;
    }
}

\begin{tikzpicture}[x=0.75pt,y=0.75pt,yscale=1,xscale=1]
    \node (top) at (0,40){};
    \node (bottom) at (0,-40){};
    \begin{pgflowlevelscope}{\pgftransformscale{0.8}}
     \draw pic at (-320,0) {fullgraph};
     \draw pic at (-120,0) {boi};
     \draw pic at (60,0) {boii};
     \draw pic at (220,0) {boiii};
     \draw pic at (350,0) {boiv};
\end{pgflowlevelscope}
\end{tikzpicture}

% \begin{tikzpicture}[x=0.75pt,y=0.75pt,yscale=-1,xscale=1]
% %uncomment if require: \path (0,476); %set diagram left start at 0, and has height of 476
% %\begin{pgflowlevelscope}{\pgftransformscale{0.8}}
% %Straight Lines [id:da39407235377586813] 
% \draw [color={rgb, 255:red, 0; green, 0; blue, 0 }  ,draw opacity=1 ][line width=1.5]    (43.33,243.33) -- (111,243.65) ;
% \draw [shift={(115,243.67)}, rotate = 180.27] [fill={rgb, 255:red, 0; green, 0; blue, 0 }  ,fill opacity=1 ][line width=0.08]  [draw opacity=0] (11.61,-5.58) -- (0,0) -- (11.61,5.58) -- cycle    ;
% %Straight Lines [id:da1587085343720771] 
% \draw [color={rgb, 255:red, 0; green, 0; blue, 0 }  ,draw opacity=1 ][line width=3]    (29.33,256.33) -- (29.33,323.67) ;
% \draw [shift={(29.33,329.67)}, rotate = 270] [fill={rgb, 255:red, 0; green, 0; blue, 0 }  ,fill opacity=1 ][line width=0.08]  [draw opacity=0] (16.97,-8.15) -- (0,0) -- (16.97,8.15) -- cycle    ;
% %Straight Lines [id:da11989847707717982] 
% \draw [color={rgb, 255:red, 0; green, 0; blue, 0 }  ,draw opacity=1 ][line width=3]    (39.33,252.83) -- (115.42,328.92) ;
% \draw [shift={(119.67,333.17)}, rotate = 225] [fill={rgb, 255:red, 0; green, 0; blue, 0 }  ,fill opacity=1 ][line width=0.08]  [draw opacity=0] (16.97,-8.15) -- (0,0) -- (16.97,8.15) -- cycle    ;
% %Straight Lines [id:da08794643011610992] 
% \draw [color={rgb, 255:red, 0; green, 0; blue, 0 }  ,draw opacity=1 ][line width=1.5]    (129.33,257.33) -- (129.33,324.33) ;
% \draw [shift={(129.33,328.33)}, rotate = 270] [fill={rgb, 255:red, 0; green, 0; blue, 0 }  ,fill opacity=1 ][line width=0.08]  [draw opacity=0] (11.61,-5.58) -- (0,0) -- (11.61,5.58) -- cycle    ;
% %Straight Lines [id:da19306114007203612] 
% \draw [color={rgb, 255:red, 0; green, 0; blue, 0 }  ,draw opacity=1 ] [dash pattern={on 4.5pt off 4.5pt}]  (118.33,253) -- (41.66,333.44) ;
% \draw [shift={(39.33,334.33)}, rotate = 315.84] [fill={rgb, 255:red, 0; green, 0; blue, 0 }  ,fill opacity=1 ][line width=0.08]  [draw opacity=0] (8.93,-4.29) -- (0,0) -- (8.93,4.29) -- cycle    ;
% %Straight Lines [id:da6200489668777374] 
% \draw [color={rgb, 255:red, 0; green, 0; blue, 0 }  ,draw opacity=1 ] [dash pattern={on 4.5pt off 4.5pt}]  (43.33,343.33) -- (112,343.33) ;
% \draw [shift={(115,343.33)}, rotate = 180] [fill={rgb, 255:red, 0; green, 0; blue, 0 }  ,fill opacity=1 ][line width=0.08]  [draw opacity=0] (8.93,-4.29) -- (0,0) -- (8.93,4.29) -- cycle    ;
% %Straight Lines [id:da018013407856699137] 
% \draw [color={rgb, 255:red, 0; green, 0; blue, 0 }  ,draw opacity=1 ][line width=1.5]    (194.33,244) -- (239.67,244) ;
% \draw [shift={(243.67,244)}, rotate = 180] [fill={rgb, 255:red, 0; green, 0; blue, 0 }  ,fill opacity=1 ][line width=0.08]  [draw opacity=0] (11.61,-5.58) -- (0,0) -- (11.61,5.58) -- cycle    ;
% %Straight Lines [id:da46173605077443924] 
% \draw [color={rgb, 255:red, 0; green, 0; blue, 0 }  ,draw opacity=1 ][line width=3]    (185,257) -- (206.51,304.21) ;
% \draw [shift={(209,309.67)}, rotate = 245.5] [fill={rgb, 255:red, 0; green, 0; blue, 0 }  ,fill opacity=1 ][line width=0.08]  [draw opacity=0] (16.97,-8.15) -- (0,0) -- (16.97,8.15) -- cycle    ;
% %Straight Lines [id:da9365818041464551] 
% \draw [color={rgb, 255:red, 0; green, 0; blue, 0 }  ,draw opacity=1 ][line width=1.5]    (252.33,257) -- (231.89,305.32) ;
% \draw [shift={(230.33,309)}, rotate = 292.93] [fill={rgb, 255:red, 0; green, 0; blue, 0 }  ,fill opacity=1 ][line width=0.08]  [draw opacity=0] (11.61,-5.58) -- (0,0) -- (11.61,5.58) -- cycle    ;
% %Straight Lines [id:da6583680218203549] 
% \draw [color={rgb, 255:red, 0; green, 0; blue, 0 }  ,draw opacity=1 ][line width=3]    (317.33,253) -- (349.58,273.15) ;
% \draw [shift={(354.67,276.33)}, rotate = 212.01] [fill={rgb, 255:red, 0; green, 0; blue, 0 }  ,fill opacity=1 ][line width=0.08]  [draw opacity=0] (16.97,-8.15) -- (0,0) -- (16.97,8.15) -- cycle    ;
% %Straight Lines [id:da953569147846199] 
% \draw [color={rgb, 255:red, 0; green, 0; blue, 0 }  ,draw opacity=1 ][line width=3]    (306,258.67) -- (306,310.67) ;
% \draw [shift={(306,316.67)}, rotate = 270] [fill={rgb, 255:red, 0; green, 0; blue, 0 }  ,fill opacity=1 ][line width=0.08]  [draw opacity=0] (16.97,-8.15) -- (0,0) -- (16.97,8.15) -- cycle    ;
% %Straight Lines [id:da6888773351924493] 
% \draw [color={rgb, 255:red, 0; green, 0; blue, 0 }  ,draw opacity=1 ][line width=1.5]    (352,299) -- (321.88,321.29) ;
% \draw [shift={(318.67,323.67)}, rotate = 323.5] [fill={rgb, 255:red, 0; green, 0; blue, 0 }  ,fill opacity=1 ][line width=0.08]  [draw opacity=0] (11.61,-5.58) -- (0,0) -- (11.61,5.58) -- cycle    ;
% %Straight Lines [id:da8755116874223301] 
% \draw [color={rgb, 255:red, 0; green, 0; blue, 0 }  ,draw opacity=1 ][line width=3]    (419.67,261.67) -- (419.67,305.67) ;
% \draw [shift={(419.67,311.67)}, rotate = 270] [fill={rgb, 255:red, 0; green, 0; blue, 0 }  ,fill opacity=1 ][line width=0.08]  [draw opacity=0] (16.97,-8.15) -- (0,0) -- (16.97,8.15) -- cycle    ;
% %Straight Lines [id:da25136871570485164] 
% \draw [color={rgb, 255:red, 0; green, 0; blue, 0 }  ,draw opacity=1 ][line width=3]    (488.33,250.67) -- (488.33,294.67) ;
% \draw [shift={(488.33,300.67)}, rotate = 270] [fill={rgb, 255:red, 0; green, 0; blue, 0 }  ,fill opacity=1 ][line width=0.08]  [draw opacity=0] (16.97,-8.15) -- (0,0) -- (16.97,8.15) -- cycle    ;

% % Text Node
% \draw    (29.33, 242.83) circle [radius=14.01]  node {A} ;
% %\draw (21.33,231.33) node [anchor=north west][inner sep=0.75pt]   [align=left] {A};
% % Text Node
% \draw    (129.33, 242.83) circle [radius=14.01] node {B}   ;
% %\draw (121.33,231.33) node [anchor=north west][inner sep=0.75pt]   [align=left] {B};
% % Text Node
% \draw    (28.83, 343.83) circle [radius=14.01] node {C}   ;
% %\draw (20.33,332.33) node [anchor=north west][inner sep=0.75pt]   [align=left] {C};
% % Text Node
% \draw    (129.83, 343.83) circle [radius=14.01] node {D}   ;
% %\draw (121.33,332.33) node [anchor=north west][inner sep=0.75pt]   [align=left] {D};
% % Text Node
% \draw (62.33,228.33) node [anchor=north west][inner sep=0.75pt]   [align=left] {{\small \textit{0.69}}};
% % Text Node
% \draw (45.67,280.67) node [anchor=north west][inner sep=0.75pt]   [align=left] {{\small \textit{0.9}}};
% % Text Node
% \draw (7.33,280.33) node [anchor=north west][inner sep=0.75pt]   [align=left] {{\small \textit{0.9}}};
% % Text Node
% \draw (90.33,280.67) node [anchor=north west][inner sep=0.75pt]   [align=left] {{\small \textit{0.52}}};
% % Text Node
% \draw (132.33,280.33) node [anchor=north west][inner sep=0.75pt]   [align=left] {{\small \textit{0.7}}};
% % Text Node
% \draw (62.33,347.33) node [anchor=north west][inner sep=0.75pt]   [align=left] {{\small \textit{0.51}}};
% % Text Node
% \draw    (180.33, 243.5) circle [radius=14.01] node {A}   ;
% %\draw (172.33,232) node [anchor=north west][inner sep=0.75pt]   [align=left] {A};
% % Text Node
% \draw    (257.33, 243.5) circle [radius=14.01] node {B}  ;
% %\draw (249.33,232) node [anchor=north west][inner sep=0.75pt]   [align=left] {B};
% % Text Node
% \draw  [color={rgb, 255:red, 208; green, 2; blue, 27 }  ,draw opacity=1 ]  (219.67, 326.5) circle [x radius= 19.7, y radius= 19.7] node  {\textcolor[rgb]{0.82,0.01,0.11}{C,D}} ;
% %\draw (203.67,315) node [anchor=north west][inner sep=0.75pt]  [color={rgb, 255:red, 208; green, 2; blue, 27 }  ,opacity=1 ] [align=left] {\textcolor[rgb]{0.82,0.01,0.11}{C,D}};
% % Text Node
% \draw    (306, 245.17) circle [radius=14.01] node {A} ;
% %\draw (298,233.67) node [anchor=north west][inner sep=0.75pt]   [align=left] {A};
% % Text Node
% \draw  [color={rgb, 255:red, 208; green, 2; blue, 27 }  ,draw opacity=1 ]  (369.5, 288.17) circle [x radius= 19.3, y radius= 19.3] node {\textcolor[rgb]{0.82,0.01,0.11}{B,C}}  ;
% %\draw (354,276.67) node [anchor=north west][inner sep=0.75pt]   [align=left] {\textcolor[rgb]{0.82,0.01,0.11}{B,C}};
% % Text Node
% \draw    (305.5, 330.17) circle [x radius= 14.3, y radius= 14.3] node {D}  ;
% %\draw (297,318.67) node [anchor=north west][inner sep=0.75pt]   [align=left] {D};
% % Text Node
% \draw  [color={rgb, 255:red, 208; green, 2; blue, 27 }  ,draw opacity=1 ]  (419.67, 242.17) circle [x radius= 18.9, y radius= 18.9]  node {\textcolor[rgb]{0.82,0.01,0.11}{A,B}}  ;
% % Text Node
% \draw  [color={rgb, 255:red, 208; green, 2; blue, 27 }  ,draw opacity=1 ]  (419.67, 332.17) circle [x radius= 19.7, y radius= 19.7] node {\textcolor[rgb]{0.82,0.01,0.11}{C,D}}  ;
% % Text Node
% \draw  [color={rgb, 255:red, 0; green, 0; blue, 0 }  ,draw opacity=1 ]  (488.67, 236.17) circle [radius=14.01] node {A}  ;
% % Text Node
% \draw  [color={rgb, 255:red, 208; green, 2; blue, 27 }  ,draw opacity=1 ]  (488.33, 325.17) circle [x radius= 25.71, y radius= 25.71] node {\textcolor[rgb]{0.82,0.01,0.11}{B,C,D}}  ;
% % Text Node
% \draw (204.67,355) node [anchor=north west][inner sep=0.75pt]   [align=left] {(i)};
% % Text Node
% \draw (314,355) node [anchor=north west][inner sep=0.75pt]   [align=left] {(ii)};
% % Text Node
% \draw (407.33,355) node [anchor=north west][inner sep=0.75pt]   [align=left] {(iii)};
% % Text Node
% \draw (475.33,355) node [anchor=north west][inner sep=0.75pt]   [align=left] {(iv)};

% %\end{pgflowlevelscope}
% \end{tikzpicture}
        
\caption{Left: Directed Graph that summarizes a pairwise marginal probability matrix. (i-iv) Graph representations of bucket orders that are compatible with merging items which pairwise preference probability is below 0.52 (i, ii) and below 0.7 (iii,iv).}
\label{fig:merge_limit}
\end{figure*}

\subsection{Na\"ive Merge Statistic}

% Introduce Merge algorithm
In order to formalize the latter intuition and to derive a first (na\"ive) plug-in rule, we define the pairwise preference probability between two items, which provides a relevant notion of closeness between items.
\begin{definition}
    {\sc (Pairwise probabilities)}. For $p\in\cM_+^1(\pS)$, the pairwise preference probability between items $i \text{ and } j$, denoted $P_{i,j}$, is defined for $i\neq j$ by: $P_{i,j} = \mathbb{P}_{\Sigma \sim p}(\Sigma(i) < \Sigma(j))$. By convention, $P_{ii} = 0.5$. We define the pairwise matrix of $p$ as $P:= [P_{i,j}]_{1 \leq i,j \leq n}$.
    \label{def:pairwise_matrix}
\end{definition}
Then, given a bucket ranking $\pi\in\wO$, we  formalize the notion that two buckets can be merged, with the constraint of not changing the strict order between buckets. To this end, we define $\bar{P}_{i}(\pi)$, the \emph{strongest deviation from indifference} between any two items within the $i^{\rm th}$ bucket $\pi^{(i)}$.
\begin{align}
\bar{P}_{i}(\pi) = \max \left\{\left|P_{l,l'} - 0.5\right| : (l,l')\in \pi^{(i)}\right\}
\end{align}
Then, one needs to quantify the value of $\bar{P}_{i}(\pi)$ that would result from merging bucket $i$ to bucket $j$,
\begin{align}
\bar{P}_{ij}(\pi) = \max \left\{\left|P_{l,l'} - \frac{1}{2}\right| : (l,l')\in \bigcup_{\substack{l\in[n]\\i\leq l \leq j}}\pi^{(l)}\right\}
\end{align}
Finally, given a threshold $\theta\in[0,0.5]$ on the acceptable deviation from indifference, we define the set of pairs of buckets that can be merged while keeping $\bar{P}$ below $\theta$,
\begin{align}
    \cG(\pi,\theta) = \left\{(i,j)\in [n]^2: \bar{P}_{ij}(\pi) \leq \theta\right\}
\end{align}
%A straightforward way to create ties is thus to put items in the same bucket if their pairwise probability is close to $1/2$. However, this first step needs to account for \textit{transitivity}. Intuitively, in cases where pairs of items (A,B) and (B,C) are close to each other, but not the pair (A,C), one must decide to either create a large bucket with the three items (and thus allow for transitivity) or not. We made the choice not to allow transitivity to prevent from having items too far from each other in the same bucket, and to ensure that all items in the same buckets have at most a fixed closeness.
% Moreover, inter-buckets ranking remains challenging in a general case, which can be addressed when restricting the analysis to \textit{(Strictly) Stochastically Transitive} distributions.
% \begin{definition}
%     {\sc (Stochastic transitivity)} A distribution $p\in\cM_+^1(\pS)$ is said to be stochastically transitive (ST) iif $\forall \, (i,j,k) \in [\![1,n]\!], P_{i,j} \geq 1/2$ and $P{j,k} \geq 1/2 \Rightarrow P_{i,k} \geq 1/2$. It is said to be strictly stochastically transitive if the inequalities are strict.
% \end{definition}
% When distribution $p\in\cM_+^1(\pS)$ is SST, the pairwise matrices provide a simpler way to characterize Kemeny's median:
% \begin{remark} {\sc (Uniqueness of the ranking median)} Let $p \in \Delta^{\frak{S}_n}$ be a SST distribution, and let $d_{\tau}$ be Kendall-tau distance. Then, the median $\sigma^{med}_{d_{\tau}}(p)$ is unique and can be defined as $\forall i \in [\![1, n ]\!],  \sigma^{med}_{d_{\tau}}(p)(i) = 1 + \sum_{k \neq i} \mathbb{1}(P_{i,k} < 1/2)$~ \cite{Korba2017}.
% \label{rk:sst_unique}
% \end{remark}
% This result derives from the definition of SST distributions and the following simplification remark.
% \begin{remark} {\sc Pairwise matrix simplification.} Let $d_{\tau}$ be the Kendall-tau distance, $p\in\cM_+^1(\pS)$ a distribution and $\sigma \in \frak{S}_n$ a ranking. Then $\mathbb{E}_{\Sigma \sim p}(d_{\tau}(\Sigma, \sigma)) = \sum_{i<j} P_{i,j} \mathbb{1}(\sigma(i) > \sigma(j)) + (1-P_{i,j})\mathbb{1}(\sigma(i) < \sigma(j))$~ \cite{Korba2017}.
% \label{rk:pairwise_simpl}
% \end{remark}
The first intuition is to merge buckets iteratively, starting with the most indifferent ones, as described in \cref{algo_naive_merge}.

\begin{algorithm}
\DontPrintSemicolon
\SetKwInOut{Input}{Input}
\SetKwInOut{Output}{Output}
\Input{Pairwise matrix $P$, Ranking median $\sigma$, threshold $\theta \in [0, 0.5]$.}
$\pi \gets \sigma$ \tcp*{$\sigma$ as a bucket ranking}
\While{$\cG(\pi, \theta) \neq \emptyset$}{
    $(i^*, j^*) = \argmin_{(i,j)\in\cG(\pi,\theta)} \bar{P}_{ij}(\pi)$ \;
    update $\pi$ by merging all buckets between $i^*$ and $j^*$
    \vskip -2em
    \begin{flushleft}
        \begin{flalign*} 
            \begin{cases}
                \pi^{(i)} &\gets \pi^{(i)} ~~~\text{for}~ i < i^*\\
                \pi^{(i^*)} & \gets \bigcup_{l\in[n], i^*\leq l\leq j^*}\pi^{(l)}\\
                \pi^{(i - j^* + i^*)} & \gets \pi^{(i)} ~~~\text{for}~ i > j^*
            \end{cases}&&
        \end{flalign*}
    \end{flushleft}
    }
    \Output{$\pi$}
\caption{Na\"ive Merge}
\label{algo_naive_merge}
\end{algorithm}

Termination of \cref{algo_naive_merge} is guaranteed by the fact that the number of buckets in $\pi$ strictly decreases at each iteration. Then, by definition of $\cG(\pi, \theta)$, the resulting bucket ranking $\pi$ is such that any of its bucket $i$ satisfies $\bar{P}_i(\pi) \leq \theta$ -- \emph{i.e.} no two items with higher deviation than $\theta$ have been merged.
%Thus, when $p\in\cM_+^1(\pS)$ is SST, we derived the Naïve Merge algorithm to create a statistic allowing for bucket orders. Depending on a threshold $t$, a pair of items are put in the same bucket if their pairwise probability is the closest to $1/2$ and less than $1/2+t$. Then, the probability matrix is updated, and the algorithm continues recursively. Formally, the computation of the statistics is done using \cref{algo_merge}.
%A first remark on the construction of the Naïve Merge algorithm is the use of the max operator, which, as desired, prevents transitivity. 

Despite being very natural, this algorithm suffers from an important limitation: when changing the threshold $\theta$, its output only spans a limited subset of valid bucket rankings. In the example provided by \cref{fig:merge_limit}, the na\"ive merge method plugged-in on the Kemeny consensus can only output (i) and (iii). Whatever the value of $\theta$, it can never output (ii) or (iv). This limitation is induced by its outputs being a monotonic (w.r.t. to inclusion) function of $\theta$ -- \emph{i.e.} for $\theta_1 \leq \theta_2$, the resulting bucket rankings satisfy $\pi_{\theta_1} \subseteq \pi_{\theta_2}$. 


\subsection{Downward Merge Statistic}
% Introduce MaxPair algorithm

Overcoming this limitation only requires a small change in the algorithm which results in our main plug-in method named \emph{Downward Merge}, shown in \cref{algo_downward_merge}. 
Downward Merge algorithm selects the two buckets $(i^*, j^*)$ whose deviation from indifference $\bar{P}_{ij}(\pi)$ is maximal among those $\bar{P}_{ij}(\pi)\leq \theta$. \footnote{Instead of taking the most similar buckets, as in the previous statistic, we take the most different pair among those that are ``similar enough".} Then, all the buckets $l$ such that $i^*\leq l\leq j^*$ are merged. This process is repeated while there exist pairs of buckets whose deviation from indifference $\bar{P}_{ij}(\pi)\leq \theta$ and thus termination is guaranteed.

%Instead of starting by merging buckets that are the most indifferent, it starts by merging buckets which deviation from indifference is the highest (withing $\cG(\pi,\theta)$) and carries on going down. Formally, the Downward Merge algorithm is defined in \cref{algo_downward_merge}.
\begin{algorithm}
\DontPrintSemicolon
\SetKwInOut{Input}{Input}
\SetKwInOut{Output}{Output}
\Input{Pairwise matrix $P$, Ranking median $\sigma$, threshold $t \in [0, 0.5]$.}
$\pi \gets \sigma$ \tcp*{$\sigma$ as a bucket ranking}
\While{$\cG(\pi, t) \neq \emptyset$}{
    $(i^*, j^*) = \argmax_{(i,j)\in\cG(\pi,t)} \bar{P}_{ij}(\pi)$ \;
    update $\pi$ by merging all buckets between $i^*$ and $j^*$
    \vskip -2em
    \begin{flushleft}
        \begin{flalign*} 
            \begin{cases}
                \pi^{(i)} &\gets \pi^{(i)} ~~~\text{for}~ i < i^*\\
                \pi^{(i^*)} & \gets \bigcup_{l\in[n], i^*\leq l\leq j^*}\pi^{(l)}\\
                \pi^{(i - j^* + i^*)} & \gets \pi^{(i)} ~~~\text{for}~ i > j^*
            \end{cases}&&
        \end{flalign*}
    \end{flushleft}
    }
    \Output{$\pi$}
\caption{Downward Merge}
\label{algo_downward_merge}
\end{algorithm}

The Downward Merge method is thus able to span a larger set of bucket orders when varying $\theta$. In the example from \cref{fig:merge_limit}, the Downward Merge method plugged-in on the Kemeny consensus can generate all four bucket rankings (i-iv) for $\theta\in \{0.51, 0.52, 0.69, 0.7)\}$.


The next experimental section illustrates the robustness improvement brought by this plug-in method over a ranking median.









\section{Numerical Experiments}
\label{sec:exps}

In this section, we illustrate the relevance of the statistic outputted by our Downward Merge plug-in on Kemeny's median (called our \textit{Downward Merge statistic} for short) by running several illustrative experiments for various settings and comparing with the baseline provided by the usual Kemeny's median. The code is available \href{https://github.com/RobustConsensusRanking/RobustConsensusRanking}{here}. 

\subsection{Empirical Robustness}
\label{subsec:emp_rob}

\begin{figure}
\centering
\includegraphics[width=\linewidth]{img/theory_exp_kemeny_maxpair_v2.png}
    \caption{Breakdown function $\hat{\varepsilon}^{\gamma}_{p,T}(\delta)$ as a function of attack amplitude $\delta$ for a bucket distribution $p$ (almost a point mass on two neighboring rankings) with $n=4$. The plain blue line denotes the theoretical value for Kemeny's median $\varepsilon^*_{p}(\delta)$, blue crosses (resp. red dots) the empirical approximation $\hat{\varepsilon}^{\gamma}_{p,T}$ for Kemeny's median (resp. Down. Merge statistic for different thresholds $\theta$).}
    \label{fig:theory_exp_kemeny_maxpair}
\end{figure}

Our Downward Merge plug-in aims at providing a robustified statistic. To illustrate its usefulness, we ran experiments computing the approximate breakdown functions $\hat{\varepsilon}^{\gamma}_{p,T}(\delta)$ for the Kemeny's median as a baseline and our statistic when varying $\delta$. \cref{fig:theory_exp_kemeny_maxpair} shows the robustness as a function of attack amplitude $\delta$ and for a hand-picked distribution $p$ that is almost a point mass on a bucket ranking.

%When the threshold is much too small, with $t=0.0001$, \textcolor{red}{0.0001 has been removed from the plot?} one can see that the behavior of the Downward Merge statistic is similar to Kemeny's median, since the threshold does not allow the creation of any bucket order. Indeed, for the smallest values of $\delta$, the approximate robustness of the Downward Merge statistic is close to $0$, as for the Downward Merge statistic. \textcolor{red}{What was the bottom line here?}

When the threshold is set to a sensible value (here $\theta = 0.05$), the Downward Merge algorithm outputs a bucket order as a statistic: thus, the robustness increases very strongly to reach nearly optimal values even for very small values of $\delta$, which illustrates its efficiency. When $\theta=0.5$, the statistic is the bucket order regrouping all items. In this case, the statistic cannot be broken, and provide optimal values for the breakdown function. However, such a statistic does not provide any information about the distribution under analysis: its accuracy of location is very poor. Formally, the accuracy of location of a statistic $T$ is defined by its closeness (under the same metric $d$ used in its definition) to the whole ranking distribution: $AL_{d, p}(T) := \| d \|_{\infty} -\mathbb{E}_{p}(d(T(p), \Sigma))$, which is the opposite of the \textit{loss}, as simply defined by $Loss_{d,p}(T) = \mathbb{E}_{p}(d(T(p), \Sigma))$. By definition, under metric $d = d_{\tau}$, Kemeny's median has the highest accuracy of location, \textit{i.e.} the smallest loss. On the other hand, the Downward Merge statistic when $\theta=0.5$ has a very high loss, which makes it irrelevant in most cases. These observations justify the analysis of the loss/robustness tradeoff of our Downward Merge statistic compared to Kemeny's median.


\subsection{Tradeoffs between Loss and Robustness}
\label{subsec:tradeoffs_classic}

\begin{figure}
\centering
    \includegraphics[width=\linewidth]{img/tradeoffs_acc_rob_v2.png}
\caption{Loss/Robustness tradeoffs for different $p$ with $\delta=1$. Pairs of points linked by a black line denote results for Kemeny's median and Down. Merge statistics on the same distribution $p$ with $n=4$. "Buckets" are hand-picked distributions generated to be almost a point mass on a bucket order, "Uniform" (resp. "Point mass") "is an almost uniform (resp. point mass) hand-picked distribution, and "PL distribs." are random Plackett-Luce distributions.}
\label{fig:res_classic}
\end{figure}

We ran experiments for various distributions $p$ and computed the loss and the breakdown function of Kemeny's median and our Downward Merge algorithm to show the loss/robustness tradeoff for each statistic. \cref{fig:res_classic} shows the results for different choices of distribution $p$ when the number of items $n=4$, and for $\delta=1/6$ (normalized value of $\delta$ that requires at least a switch between two items to break the statistic).

The point mass (resp. the uniform) distribution represents the extreme case for which Kemeny's median is very robust (resp. not robust at all) and for which we expect no improvement from using the Downward Merge statistic. This intuition is verified in both cases, and we can see that the Downward Merge statistic yields the same results (in loss and in robustness) as Kemeny's median.

The bucket distributions (for which the gap between the probabilities for two rankings in the bucket order is respectively $0.1$ and $0.01$) represent the settings to which our Downward Merge is best suited. As expected, the improvement in robustness when using our Downward Merge statistic is high, and the increase in loss is negligible.

Finally, the Plackett Luce distributions (for which the parameters were generated randomly) represent a random setting. The results are interestingly very similar to those for the bucket distributions: the gain in robustness is high and the increase in loss is negligible. This random setting illustrates the usefulness of our Downward Merge statistic in general cases and shows that, overall, it yields a much better compromise than Kemeny's median.

\begin{comment}
\subsection{Scalable approximations}
\label{subsec:scalable_exp}

The breakdown function definition provided in \cref{eq:breakdown_fct_bucket} depends on TV distance between probability distributions $p$ and $q$ (the adversarial distribution for $p$), which needs to be computed at each step of our SGD-based optimization algorithm. These probability distributions are vectors of size $n!$, meaning that the aforementioned algorithm is overly computationally complex. A simple workaround to simplify the computation is to use remarks \cref{rk:sst_unique, rk:pairwise_simpl} and to relax the constraint from the breakdown function definition that states that the adversarial distribution must differ from $p$ by a budget measured using the TV distance. By replacing this constraint and measuring the difference between $p$ and $q$ using a norm on their respective probability matrix, for example, using $\| P-Q\|_1$, the computation is then able to run in $\text{poly}(n)$, which is much faster and allows for scalable experiments with a higher number of items $n$.

Such a relaxation induces modifying the computation of the breakdown function and thus may result in an imprecise approximation. However, the relation between the TV difference of probability distribution and the $L_1$ norm difference of the probability matrices is almost linear as illustrated by \cref{fig:proba_vs_pairwise}, meaning that we conserve a good approximation for the breakdown function.

\begin{figure}[h]
\centering
\label{fig:proba_vs_pairwise}
\includegraphics[width=0.8\linewidth]{icml_submission/img/proba_vs_pairwise.png}
\caption{Relation between $TV(p,q)$ and $\|P-Q\|_1$ (normalized).}
\end{figure}

On the same settings as in \cref{subsec:tradeoffs_classic} for a relevant distribution $p$, the tradeoffs between loss and robustness remain similar with this new pairwise relaxation, as illustrated by \cref{subfig:res_pairwise_n4}. Furthermore, \cref{subfig:res_pairwise_n8,subfig:res_pairwise_n8_delta3} also show that Maxpair statistics is also a better compromise in a more complex setting where the number of items $n = 8$ and where $\delta = 1$ or $3$. Note that in all the settings of \cref{fig:res_pairwise}, not only the robustness provided by Maxpair statistic is much higher, but also the loss is very close to that of Kemeny's median.

\begin{figure}
\centering
\begin{subfigure}{0.95\linewidth}
    \includegraphics[width=\linewidth]{icml_submission/img/pairwise_two_untied_n=4_seed=279.png}
    \caption{$n=4$}
    \label{subfig:res_pairwise_n4}
\end{subfigure}
\hfill
\begin{subfigure}{0.95\linewidth}
    \includegraphics[width=\linewidth]{icml_submission/img/pairwise_two_untied_n=8_seed=279.png}
    \caption{$n=8$}
    \label{subfig:res_pairwise_n8}
\end{subfigure}
\hfill
\begin{subfigure}{0.95\linewidth}
    \includegraphics[width=\linewidth]{icml_submission/img/pairwise_two_untied_n=8_delta=3_seed=279.png}
    \caption{$n=8, \delta=3$}
    \label{subfig:res_pairwise_n8_delta3}
\end{subfigure}
        
\caption{Loss/robustness tradeoffs for different values of $n$, with $\delta=1$ (\cref{subfig:res_pairwise_n4,subfig:res_pairwise_n8}) or $\delta=3$ (\cref{subfig:res_pairwise_n8_delta3}).}
\label{fig:res_pairwise}
\end{figure}

Overall, the experiments show that our Maxpair statistic provides a way to solve Consensus Aggregation in rankings in a scalable way.
\end{comment}




\section{Conclusion}
\label{sec:conclusion}

We consider top-down attention by explaining from an Analysis-by-Synthesis (AbS) view of vision. Starting from previous work on the functional equivalence between visual attention and sparse reconstruction, we show that AbS optimizes a similar sparse reconstruction objective but modulates it with a goal-directed top-down modulation, thus simulating top-down attention. We propose \model, a top-down modulated ViT model that variationally approximates AbS. We show that \model achieves controllable top-down attention and improves over baselines on V\&L tasks as well as image classification and robustness.

\bibliography{icml_main}
\bibliographystyle{icml2023}


%%%%%%%%%%%%%%%%%%%%%%%%%%%%%%%%%%%%%%%%%%%%%%%%%%%%%%%%%%%%%%%%%%%%%%%%%%%%%%%
%%%%%%%%%%%%%%%%%%%%%%%%%%%%%%%%%%%%%%%%%%%%%%%%%%%%%%%%%%%%%%%%%%%%%%%%%%%%%%%
% APPENDIX
%%%%%%%%%%%%%%%%%%%%%%%%%%%%%%%%%%%%%%%%%%%%%%%%%%%%%%%%%%%%%%%%%%%%%%%%%%%%%%%
%%%%%%%%%%%%%%%%%%%%%%%%%%%%%%%%%%%%%%%%%%%%%%%%%%%%%%%%%%%%%%%%%%%%%%%%%%%%%%%
\newpage
\appendix
\onecolumn

\section{Additional Metrics on $\pS$}
\label{app:additional_metrics}

\paragraph{The Kendall Tau} is the metric used all along the main part of the paper, the proportion of misordered pairs,
\begin{align*}
    d_{\tau}(\sigma, \nu) = \frac{2}{n(n-1)}\sum_{i<j}\indicator{\sigma(i)<\sigma(j)}\indicator{\nu(i) > \nu(j)}\,.
\end{align*}
The Kemeny consensus is the median associated with the Kendall Tau metric.

\paragraph{The Spearman Rho} is a normalized quadratic distance between the rank vectors,
\begin{align}
    d_{\tau}(\sigma, \nu) = \frac{6}{n(n^2-1)}\sum_{i}\left(\nu(i) - \sigma(i)\right)^2\,.
\end{align}
The Borda count is the median associated with the Spearman Rho (\emph{e.g.} see \citet{calauzenes2013}).

\paragraph{The Spearman footrule} is a absolute value distance between the rank vectors,
\begin{align}
    d_{\tau}(\sigma, \nu) = \sum_{i}\left|\nu(i) - \sigma(i)\right|\,.
\end{align}


\section{Notation for Appendix}
\label{app:notation}
For the sake of clarity of the proofs, we switch to matrix notation in the appendix.
%Its worth noting the direct implementation of the formulations below should be avoided since the factorial cardinality of the permutation space makes it impratical for medioum size problems. Instead more efficient one versions are proposed in the main paper, which should be used in practical applications. 
We fix an arbitrary indexation $\{\sigma^{(1)}, \dots, \sigma^{(n!)}\}$ of $\pS$. Using this indexation, given a metric $d$ on $\pS$, we can defined the (symmetric) metric matrix $D = (d(\sigma^{(i)},\sigma^{(j)}))_{i,j\in[n!]}$. Identifying a ranking $\sigma$ with its corresponding basis vector $\be_i$ s.t. $\sigma = \sigma^{(i)}$, we write for two rankings $\sigma, \sigma', \nu\in\pS$,
\begin{align}
    \nu^\top D \sigma := d(\nu, \sigma) ~~~~~~~\text{or}~~~~~~~ \nu^\top D (\sigma - \sigma') := d(\nu, \sigma) - d(\nu, \sigma')
\end{align}
Further, a distribution $p\in\cM_+^1(\pS)$ on permutation can now be seen as a $n!$-dimensional vector in $\mathbb{R}^{n!}$. This allows to write, for $p\in\distribs$, $\sigma\in\pS$,
\begin{align}
    p^\top D\sigma := \lE_{\Sigma\sim p}[d(\Sigma,\sigma)]
\end{align}







\section{Proof: Bound on Breakdown Function for Ranking Medians}
\label{app:breakdown_function_medians}

\subsection{Upper-bound}
\label{app:breakdown_function_kemeny_ub}

We first remind \cref{thm:breakdownfunctionkemeny}.
\thmbreakdownfunctionkemeny*

We re-state the theorem with the matrix notation defined in \cref{app:notation} and used all along the appendix.

\begin{theorem}
For $p\in\cM_+^1(\pS)$, $\sigma^\star_p = \sigma^{\rm med}_{d_\tau}(p)$ and $S_\delta = \{\sigma\in\pS | d_\tau(\sigma, \sigma^\star_p) \geq \delta\}$, if $\varepsilon^+(\delta) \leq 2 p(\sigma^\star_p)$, then $\varepsilon^\star_{d_\tau, p, \sigma^\star_p} \leq \varepsilon^+(\delta)$.
\begin{align}
    \varepsilon^+(\delta)
    =
    \min_{\sigma\in S_\delta}\max_{\nu\in N_\delta}
    \frac{\mkendall{(\sigma-\nu)}{p}}
    {\mkendall{(\sigma - \nu)}{\sigma^\star_p}}\,,
\end{align}

\end{theorem}
\begin{proof}
    \begin{align}
        \varepsilon^\star_{d_\tau, p, \sigma^\star_p} & = \inf \left\{\varepsilon>0\middle|\sup_{q: \textsc{tv}(p, q)\leq \varepsilon}d_\tau(\sigma^\star_p, \sigma^\star_q) \geq \delta\right\} \\
        & = \inf \left\{\varepsilon>0\middle|\exists q, s.t. \textsc{tv}(p, q)\leq \varepsilon ~\text{and}~ d_\tau(\sigma^\star_p, \sigma^\star_q) \geq \delta\right\}\\
        & = \inf \underbrace{\left\{\varepsilon>0\middle|\exists q, s.t. \textsc{tv}(p, q)\leq \varepsilon ~\text{and}~ \argmin_{\sigma\in\pS}\mkendall{\sigma}{q} \subseteq S_\delta\right\}}_{=: E} ~~~\text{with}~S_\delta = \{\sigma\in\pS | d_\tau(\sigma, \sigma^\star_p) \geq \delta\}
    \end{align}
    Further, we define 
    $N_\delta = \pS \setminus S_\delta$, 
    $\sigma^{\star, {\rm rev}}_p$ the reverse of $\sigma^{\star}_p$, i.e., $\sigma^{\star, {\rm rev}}_p (i)=\sigma^{\star}_p(n-i-1)$ and the \emph{attack} distribution 
    ${\bar{q}_\varepsilon = p - \frac{\varepsilon}{2}\indicator{\cdot = \sigma^\star_p} + \frac{\varepsilon}{2}\indicator{\cdot = \sigma^{\star, {\rm rev}}_p}}$ that removes the probability mass from the median to put it on the farthest point. 
    We also define 
    ${{E} = \left\{\varepsilon | \argmin_{\sigma\in\pS}\mkendall{\sigma}{\bar{q}_\varepsilon} \subseteq S_\delta\right\}}$  and 
    ${\tilde{E} = \left\{0<\varepsilon \leq 2 p(\sigma^\star_p) \middle | \argmin_{\sigma\in\pS}\mkendall{\sigma}{\bar{q}_\varepsilon} \subseteq S_\delta\right\} \subseteq E \cap (0, 2 p(\sigma^\star_p)]}$.


    Let $\varepsilon>0$ be such that $\varepsilon \leq 2 p(\sigma^\star_p)$. Then
    \begin{align}
        \varepsilon\in\tilde{E} 
        & \Leftrightarrow \exists \sigma \in S_\delta, \forall \nu \in N_\delta, \mkendall{\sigma}{\bar{q}_\varepsilon} \leq \mkendall{\nu}{\bar{q}_\varepsilon}\\
        & \Leftrightarrow \exists \sigma \in S_\delta, \forall \nu \in N_\delta, \mkendall{(\sigma - \nu)}{p} + \frac{\varepsilon}{2}\left(\mkendall{\sigma^{\star, {\rm rev}}_p}{\sigma} - \mkendall{\sigma^{\star}_p}{\sigma} + \mkendall{\sigma^{\star}_p}{\nu} - \mkendall{\sigma^{\star, {\rm rev}}_p}{\nu}\right)\leq 0\\
        & \Leftrightarrow \exists \sigma \in S_\delta, \forall \nu \in N_\delta, \mkendall{(\sigma - \nu)}{p} 
        \leq 
        \frac{\varepsilon}{2}\left(
             \mkendall{(\sigma - \nu)}{(\sigma^\star_p - \sigma^{\star, {\rm rev}}_p)}
        \right)\\
        & \Leftrightarrow \exists \sigma \in S_\delta, \forall \nu \in N_\delta, \mkendall{(\sigma - \nu)}{p} 
        \leq 
        \varepsilon\left(
             \mkendall{(\sigma - \nu)}{\sigma^\star_p}
        \right)&\!\!\!\!\!\!\!\!\!\!\!\!\!\!\!\!\!\!\!\!\!\!\!\!\!\!\!\!\!\!\!\!\!\!\!\!\!\!\!\!\!\!\!\!\!\!\!\!\!\!\!\!\!\!\!\!\!\!\!\!\!\!\!\!\!\!\!\!\!\!\!\!\!\!\!\!\!\!\!\!\text{as }\mkendall{\cdot}{\sigma^{\star, {\rm rev}}_p} = \|D_\tau\|_\infty - \mkendall{\cdot}{\sigma^{\star}_p}\\
        & \Leftrightarrow \exists \sigma \in S_\delta, \forall \nu \in N_\delta, \frac{\mkendall{(\sigma - \nu)}{p}}{\mkendall{(\sigma - \nu)}{\sigma^\star_p}}
        \leq 
        \varepsilon\\
        & \Leftrightarrow \min_{\sigma \in S_\delta}\max_{\nu \in N_\delta} \frac{\mkendall{(\sigma - \nu)}{p}}{\mkendall{(\sigma - \nu)}{\sigma^\star_p}}
        \leq 
        \varepsilon \label{eq:in_tilde_E}
    \end{align}
    Now, denoting $\varepsilon^+(\delta) = \min_{\sigma \in S_\delta}\max_{\nu \in N_\delta} \frac{\mkendall{(\sigma - \nu)}{p}}{\mkendall{(\sigma - \nu)}{\sigma^\star_p}}$, by definition $\varepsilon^+(\delta)$ satisfies \cref{eq:in_tilde_E}, which means $\varepsilon^+(\delta) \in \tilde{E}$ iff $\varepsilon^+(\delta) \leq 2 p(\sigma^\star_p)$.
    Thus, if $\varepsilon^+(\delta) \leq 2 p(\sigma^\star_p)$, then 
    \begin{align}
         \varepsilon^+(\delta) = \inf \tilde{E} \geq \inf E = \varepsilon^\star_{d_\tau, p, \sigma^\star_p}.
    \end{align}
\end{proof}

\subsection{Lower-bound}
\label{app:breakdown_function_median_lb}
We first remind \cref{thm:ubbreakdownfunctionmedian}.
\thmubbreakdownfunctionmedian*

We re-state the theorem with the matrix notation defined in \cref{app:notation}.

\begin{theorem}
For $p\in\cM_+^1(\pS)$, $d$ and $m$ two metrics on $\pS$ and $\sigma^\star_p = \sigma^{\rm med}_{d}(p)$, we have
\begin{align}
    \varepsilon^\star_{m, p, \sigma^\star_p} \geq \min_{\sigma\in S_\delta}\max_{\nu\in\pS: \nu\neq\sigma}\frac{\mdist{(\sigma-\nu)}{p}}{\|D(\sigma-\nu)\|_\infty}\,,
\end{align}
where $S_\delta = \{\sigma\in\pS | d_\tau(\sigma, \sigma^\star_p) \geq \delta\}$.

\end{theorem}
\begin{proof}
Let $S_\delta, N_\delta, E, \tilde{E}$ are defined as above.
    \begin{align}
        \varepsilon^\star_{m, p, \sigma^\star_p} & = \inf \left\{\varepsilon>0\middle|\sup_{q: \textsc{tv}(p, q)\leq \varepsilon}m(\sigma^\star_p, \sigma^\star_q) \geq \delta\right\} \\
        & = \inf \left\{\varepsilon>0\middle|\exists q, s.t. \textsc{tv}(p, q)\leq \varepsilon ~\text{and}~ m(\sigma^\star_p, \sigma^\star_q) \geq \delta\right\}\\
        & = \inf \underbrace{\left\{\varepsilon>0\middle|\exists q, s.t. \textsc{tv}(p, q)\leq \varepsilon ~\text{and}~ \argmin_{\sigma\in\pS}\mdist{\sigma}{q} \subseteq S_\delta\right\}}_{=: E} ~~~\text{with}~S_\delta = \{\sigma\in\pS | m(\sigma, \sigma^\star_p) \geq \delta\}.
    \end{align}
    Now,
    \begin{align}
        \varepsilon \in E & \Leftrightarrow \exists q, s.t. \textsc{tv}(p, q)\leq \varepsilon ~\text{and}~ \argmin_{\sigma\in\pS}\mdist{\sigma}{q} \subseteq S_\delta\\
        & \Leftrightarrow \exists q \in \Delta^\pS, \textsc{tv}(p, q)\leq \varepsilon ~\text{and}~ \exists \sigma\in S_\delta, \forall \nu\in\pS, \mdist{\sigma}{q} \leq \mdist{\nu}{q}\\
        & \Leftrightarrow \exists q \in \Delta^\pS, \textsc{tv}(p, q)\leq \varepsilon ~\text{and}~ \exists \sigma\in S_\delta, \forall \nu\in\pS, \mdist{(\sigma-\nu)}{p} \leq \mdist{(\sigma-\nu)}{(q_- - q_+)}\\
        & ~~~~~~~\text{where}~ q_+ = (q-p)_+ ~~\text{and}~~ q_- = (p-q)_+\nonumber\\
        & \Rightarrow \exists q \in \Delta^\pS, \textsc{tv}(p, q)\leq \varepsilon ~\text{and}~ \exists \sigma\in S_\delta, \forall \nu\in\pS, \mdist{(\sigma-\nu)}{p} \leq \|q_+-q_-\|_1 \|D(\sigma-\nu)\|_\infty\\
        & \Rightarrow \exists \sigma\in S_\delta, \forall \nu\in\pS, \mdist{(\sigma-\nu)}{p} \leq \varepsilon\|D(\sigma-\nu)\|_\infty & \!\!\!\!\!\!\!\!\!\!\!\!\!\!\!\!\!\!\!\!\!\!\!\!\!\!\!\!\!\!\!\!\!\text{as }\|q_+-q_-\|_1\leq \varepsilon\\
        & \Rightarrow \exists \sigma\in S_\delta, \forall \nu\in\pS, s.t. \sigma \neq \nu, \frac{\mdist{(\sigma-\nu)}{p}}{\|D(\sigma-\nu)\|_\infty} \leq \varepsilon\\
        & \Rightarrow \min_{\sigma\in S_\delta}\max_{\nu\in\pS: \nu\neq\sigma}\frac{\mdist{(\sigma-\nu)}{p}}{\|D(\sigma-\nu)\|_\infty} \leq \varepsilon.
    \end{align}
    Finally,
    \begin{align}
        \varepsilon^\star_{m, p, \sigma^\star_p} & = \inf E \geq \min_{\sigma\in S_\delta}\max_{\nu\in\pS: \nu\neq\sigma}\frac{\mdist{(\sigma-\nu)}{p}}{\|D(\sigma-\nu)\|_\infty}\,.
    \end{align}
\end{proof}



% \begin{proof}
% As a reminder, for the Kemeny rule, the breakdown function is
% \begin{align}
%     \varepsilon^\star_{p,\sigma^\star_p}(\delta) 
%     & = \inf \left\{\varepsilon>0\middle|\sup_{q: \textsc{tv}(p, q)\leq \varepsilon}d_\tau(\sigma^\star(p), \sigma^\star(q)) \geq \delta\right\}\,.
% \end{align}


% The inner part of the breakdown function problem of level $\delta$ consists in finding the adversarial distribution $q^*_{\sigma^*}$ on which Kemeny's rule outputs a median at distance at least $\delta$ from $\sigma^*$. Let us order the losses for each $\sigma$, writing $[p^T D_{\tau}]_{(1)} \geq ... \geq [p^T D_{\tau}]_{(n!)}$ the ordered losses for $\sigma_{(1)}, ..., \sigma_{(n!)}$ respectively. Note that with this notation, $\sigma^{med}_{p, d_{\tau}} := \sigma^* = \sigma_{(n!)}$. Then, the goal of $q^*_{\sigma^*}$ is to modify optimally $p$ (under probability and budget constraints) so that there exists a $\sigma_{(l)}$ such that $d_{\tau}(\sigma_{(l)}, \sigma_{(n!)}) \geq \delta$ for which $(q^*_{\sigma^*})^T D_{\tau} \sigma_{(l)}$ is minimal. To achieve this, an important remark stemming from the nature of the Kendall tau distance is the following: when modifying $p$, the relative increase of loss for any $\nu \in \frak{S}_n$ with respect to $\sigma_{(n!)}$ does not depend on the modification of mass granted to $\nu$, but only to that of $\sigma_{(n!)}$.

% More specifically, let us consider the simple case where $\forall \sigma, \varepsilon \leq 2 \min(1-p(\sigma), p(\sigma_{opp}))$. Let's define two adversarial distributions:

% \begin{equation}
% \begin{split}
%     & q_{1}(\sigma_{(n!)}) = p(\sigma_{(n!)}) - \varepsilon/2 \\
%     & q_{1}(\sigma_{(l)}) = p(\sigma_{(l)}) + \varepsilon/2 \\
%     & q_{1}(\sigma) = p(\sigma) \text{ } \forall \sigma \neq (\sigma_{(n!)}, \sigma_{(l)})
% \end{split}
% \end{equation}

% \begin{equation}
% \begin{split}
%     & q_{2}(\sigma_{(n!)}) = p(\sigma_{(n!)}) - \varepsilon/2 \\
%     & q_{2}((\sigma_{(n!)})_{opp}) = p((\sigma_{(n!)})_{opp}) + \varepsilon/2 \\
%     & q_{2}(\sigma) = p(\sigma) \text{ } \forall \sigma \neq (\sigma_{(n!)}, (\sigma_{(n!)})_{opp})
% \end{split}
% \end{equation}

% Then we have that $q_1^T D_{\tau} \sigma_{(l)} - q_1^T D_{\tau} \sigma_{(n!)} = q_2^T D_{\tau} \sigma_{(l)} - q_2^T D_{\tau} \sigma_{(n!)} = p^T D_{\tau} \sigma_{(l)} - p^T D_{\tau} \sigma_{(n!)} - \varepsilon \delta$. Thus, it is not needed to decide/find which $\sigma_{(l)}$ should be the output of Kemeny's rule for $q^*_{\sigma^*}$ since an optimal strategy is simply to remove as much probability mass as possible from $\sigma^*$ and to allocate it to $\sigma^*_{opp}$.

% %The objective of $q^*_{\sigma^*}$ is thus to increase sufficiently the loss of $\sigma^*$, so to solve:

% %\begin{equation}
% %\begin{split}
% %    q^*_{\sigma^*} & = \max_{q \in \Delta^{n!}} \mathbb{E}_{\Sigma \sim q}(d_{\tau}(\sigma^*, \Sigma)) \text{ s.t. } TV(p,q) \leq \varepsilon \\
% %    & = \max_{q \in \Delta^{n!}} q^{T} D_{\tau}(\sigma^*) \text{ s.t. } TV(p,q) \leq \varepsilon
% %\end{split}
% %\label{eq:median}
% %\end{equation}

% %where $D_{\tau}(\sigma^*) := \left[ d_{\tau}(\sigma^*, \sigma_i) \right]_{1 \leq i \leq n!}$ is the vector of Kendall-tau distances with respect to $\sigma^*$. \cref{eq:median} is thus a linear problem under budget and probability constraint. Let us write $(\nu_i)_{1 \leq i \leq n!}$ and arbitrary sequence ordered as follows: $d_{\tau}(\sigma^*, \nu_1) \geq ... \geq d_{\tau}(\sigma^*, \nu_{n!})$, so that $\nu_1 = (\sigma^*)^{R}$ and $\nu_{n!} = \sigma^*$, where $(\sigma^*)^{R}$ denotes the opposite ranking to $\sigma^*$. The intuition is that $q^*_{\sigma^*}$ will take as much budget as possible (under the aforementioned constraints) from $\nu_1 = \sigma^*$ to give it to $\nu_{n!} = (\sigma^*)^{R}$.\\

% %In the simple case where $\varepsilon \leq 2 \min(1-p((\sigma^*)^{R}), p(\sigma^*))$, we have:

% %\begin{equation}
% %\begin{split}
% %    & q^*_{\sigma^*}(\sigma^*) = p(\sigma^*) - \varepsilon/2 \\
% %    & q^*_{\sigma^*}((\sigma^*)^{R}) = p((\sigma^*)^{R}) + \varepsilon/2 \\
% %    & q^*_{\sigma^*}(\sigma) = p(\sigma) \text{ } \forall \sigma \neq (\sigma^*, (\sigma^*)^{R})
% %\end{split}
% %\label{eq:adv_distrib_median_v1}
% %\end{equation}

% In all generality, $\exists k \text{ s.t. } 2 \sum_{l=1}^{k-1} p(\nu_l) < \varepsilon \leq 2 \sum_{l=1}^k p(\nu_l)$ and we withdraw budget sequentially from the $\nu$ closest to $\sigma^*$ until reaching $\varepsilon/2$ budget (which happens for $\nu_k$) to add it sequentially to $\nu$ closest to $(\sigma^*)^R$.
% \begin{equation}
% \begin{split}
%     & q^*_{\sigma^*}(\nu_l) = 0 \text{ } \forall l > k \\
%     & q^*_{\sigma^*}(\nu_k) = p(\nu_k) - \left( \frac{\varepsilon}{2} - \sum_{l=1}^{k-1} p(\nu_l) \right) \\
%     & q^*_{\sigma^*}(\nu_l) = \max \left( p(\nu_l), \min\left[ 1, p(\nu_l) + \frac{\varepsilon}{2} - \sum_{m=l}^{n!} (1-p(\nu_m)) \right] \right) \text{ } \forall l \leq k-1
% \end{split}
% \label{eq:adv_distrib_median_v2}
% \end{equation}

% In the simple setting from \cref{eq:adv_distrib_median_v1}, we have $\forall \nu, \; (q^*_{\sigma^*})^T D_{\tau} \nu = p^T D_{\tau} \nu - \varepsilon/2 ( ((\sigma^*)^{R})^T D_{\tau} \nu - (\sigma^*)^T D_{\tau} \nu ) )$ and thus in particular $(q^*_{\sigma^*})^T D_{\tau} \sigma^* = p^T D_{\tau} \sigma^* - \varepsilon/2 \| d_{\tau} \| $, where $D_{\tau}$ is Kendall-tau distance matrix. Thus,

% \begin{equation}
% \begin{split}
%     (q^*_{\sigma^*})^T D_{\tau} \sigma \leq (q^*_{\sigma^*})^T D_{\tau} \nu & \Leftrightarrow p^T D_{\tau} \sigma - p^T D_{\tau} \nu - \frac{\varepsilon}{2}(\| d_{\tau} \| - (\sigma^{R})^T D_{\tau} \nu + \sigma^T D_{\tau} \nu) \leq 0 \\
%     & \Leftrightarrow \varepsilon \geq 2 \frac{p^T D_{\tau} \sigma - p^T D_{\tau} \nu }{\|d_{\tau}\| - (\sigma^{R})^T D_{\tau} \nu + \sigma^T D_{\tau} \nu)}
% \end{split}
% \label{eq:bkdwn_median_compute}
% \end{equation}

% If one wants a distance $\delta$ at least between $\sigma^*$ and $\nu$, we need $(q^*_{\sigma^*})^T D_{\tau} \sigma \leq (q^*_{\sigma^*})^T D_{\tau} \nu$ for any $\nu$ at distance $\delta$ or more to $\sigma^*$ and for all $\nu$ at distance strictly less than $\delta$ tp $\sigma^*$. Thus:

% \begin{equation}
% \begin{split}
%     \varepsilon^\star_{\delta, d_{\tau}}(p, \text{Kemeny rule}) = \min_{\sigma | d_{\tau}(\sigma, \sigma^*_p) \geq \delta} \max_{\nu | d_{\tau}(\nu, \sigma^*_{p}) < \delta} 2 \frac{p^T D_{\tau} \sigma - p^T D_{\tau} \nu}{\|d_{\tau}\| - ((\sigma^*)^{R})^T D_{\tau} \nu + (\sigma^*)^T D_{\tau} \nu) }
% \end{split}
% \label{eq:bkdwn_median}
% \end{equation}



% \end{proof}



\section{Hausdorff Extensions of Kendall Tau}
\label{app:hausdorff_kendall}

We remind first the Kendall-tau distance, defined by: $$d_{\tau}: (\sigma_1, \sigma_2) \in \frak{S}_n \times \frak{S}_n \to \sum_{i<j} \mathbb{1}( (\sigma_1(i)-\sigma_1(j))(\sigma_2(i)-\sigma_2(j)) < 0 )$$  and the  \cref{def:non_symmetric_hausdorff,def:symmetric_hausdorff} of the Hausdorff extensions of the Kendall tau metric.

\defnonsymmetrichausdorff*

\defsymmetrichausdorff*

\begin{restatable}{proposition}{propcomplexityhausdorffkendall}\label{prop:complexity_hausdorff_kendall}
For any $\pi_1, \pi_2\in\wO$, the computation cost of $H_{d_\tau}^{\textsc{ns}}(\pi_1, \pi_2)$ and $H_{d_\tau}^{(1/2)}(\pi_1, \pi_2)$ is $\cO(n^2)$.
\end{restatable}


The average Hausdorff distance can be expressed with various expressions, necessitating the following notations (see \cite{fagin2006comparing}): 
\begin{enumerate}
    \item $\forall \, i \in [\![1,n]\!] \quad \bar{\pi}(i) = \sum_{\sigma \in \pi} \sigma(i)$ is the rank of item $i$ according to weak order $\pi$.
    \item $S(\pi_1, \pi_2) = \{ (i <j ) \; | \; \bar{\pi}_1(i)\neq\bar{\pi}_1(j), [\bar{\pi}_1(i)-\bar{\pi}_1(j)][\bar{\pi}_2(i)-\bar{\pi}_2(j)] < 0 \}$ is the set of item pairs $(i<j)$ that are in different buckets in both $\pi_1$ and $\pi_2$, and that are in different orders in $\pi_1$ and $\pi_2$.
    \item $S(\pi_1 \setminus \pi_2) = \{(i<j) \; | \; \bar{\pi}_1(i) = \bar{\pi}_1(j) \text{ and } \bar{\pi}_2(i) \neq \bar{\pi}_2(j) \}$ is the set of item pairs $(i<j)$ such that both items are in the same bucket in $\pi_1$ but in different ones in $\pi_2$.
    \item $\prof(\pi) = (\prof(\pi)_{i,j})_{i<j}$, where $\forall \; i<j, \prof(\pi)_{i,j} = 1/2$ if $\bar{\pi}(i) < \bar{\pi}(j)$, $= 0$ if $\bar{\pi}(i) = \bar{\pi}(j)$ and $= -1/2$ if $\bar{\pi}(i) > \bar{\pi}(j)$. $\prof(\pi)$ is called the profile vector of $\pi$.
\end{enumerate}

We have the following equivalent expressions for the average Hausdorff distance:

\begin{restatable}[Average Hausdorff distance]{proposition}{propavghausdorff}\label{prop:avg_hausdorff_expressions}
\begin{align}
    H_K^{(1/2)}(\pi_1, \pi_2) &:= \# S(\pi_1, \pi_2) + \frac{1}{2} \left( \#S(\pi_1 \setminus \pi_2) + \#S(\pi_2 \setminus \pi_1) \right) \\
    &= \sum_{i<j} \mathbb{1}\left( [\bar{\pi}_1(i)-\bar{\pi}_1(j)][\bar{\pi}_2(i)-\bar{\pi}_2(j)] < 0 \right) + \nonumber \\
    & \quad \quad \quad \frac{1}{2} \mathbb{1}\left( [\bar{\pi}_1(i)=\bar{\pi}_1(j)] \right)\mathbb{1}\left( [\bar{\pi}_2(i)\neq\bar{\pi}_2(j)] \right) + \nonumber \\
    & \quad \quad \quad\frac{1}{2} \mathbb{1}\left( [\bar{\pi}_2(i)=\bar{\pi}_2(j)] \right)\mathbb{1}\left( [\bar{\pi}_1(i)\neq\bar{\pi}_1(j)] \right) \\
    &= \| \prof(\pi_1) - \prof(\pi_2) \|_1
    \label{eq:avg_hausdorff_expressions}
\end{align}
\end{restatable}

\begin{restatable}[Avergage Hausdorff distance - Proof]{proof}{proofavghausdorff}\label{proof:avg_hausdorff_expressions}
Let $\pi_1$, $\pi_2$ be two weak orders associated with buckets $(B^1_1,...B^1_{t_1})$ and $(B^2_1,...B^2_{t_2})$ respectively. Such buckets are sets of items $i$ forming a partition of $[\!1,n]\!]$ such that $i \in B^1_k$ iif $\bar{\pi}_1(i) = \sum_{k'< k} \#B^1_{k'} + \frac{\#B^1_k + 1}{2}$  (see \cite{fagin2006comparing} for a more formal definition). Let us define, as in \cite{critchlow2012metric, fagin2006comparing}, ${\forall \; i \leq t_1, \forall \; j \leq t_2, \quad n_{i,j}= \#(B_i \cap B_j)}$.

Then we have \cite{critchlow2012metric}[Chapter IV]: $H_K^{(1/2)} = \frac{1}{2} \left( \sum_{i<i', j \geq j'}n_{i,j}n_{i',j'} + \sum_{i \leq i', j > j'}n_{i,j}n_{i',j'} \right)$.

By noting that $2 \# S(\pi_1, \pi_2) = \sum_{i<i', j > j'}n_{i,j}n_{i',j'}$ and $2 \#S(\pi_1 \setminus \pi_2) = \sum_{i=i', j > j'}n_{i,j}n_{i',j'}$, we derive our first equality. The second equality directly comes from re-expressing the first one. The third equality comes from \cite{fagin2006comparing}.


\end{restatable}



%\section{Bucket Orders are Optimal for the DRO Statistics}
\label{app:dro}

\begin{restatable}{proposition}{propbucketorderoptimalfordro}\label{prop:bucket_order_optimal_for_dro}
$\exists p \in \cM_+^1(\pS), \sigma\in\pS, r\in[n-1]$ s.t. 
\begin{align*}
    \{\sigma, \tau_{r,r+1}\sigma\} = \argmin_{S \subseteq \pS} \max_{q: \textsc{tv}(p,q)\leq \varepsilon} \lE_{\Sigma\sim q}\left[\max_{\sigma\in S}d(\Sigma, \sigma)\right]
\end{align*}
\end{restatable}

\begin{proof}
    For $\eta, \lambda\in[0,1]$, let $\sigma_0\in\pS$, $r\in[n-1]$ and $p\in\distribs$ be such that for any $\sigma\in\pS$,
    \begin{align}
    p(\sigma) =
        \begin{cases}
        \frac{1-\eta}{n!} + \lambda & \text{if }\nu =\tau_{r,r+1} \sigma_0\\
        \frac{1-\eta}{n!} + \eta - \lambda & \text{if }\nu = \sigma_0\\
        \frac{1-\eta}{n!} & \text{otherwise.}
        \end{cases}
    \end{align}

    Given $S\subseteq\pS$, we denote $q_S = \argmax_{q: \textsc{tv}(p,q)\leq \varepsilon}\lE_{\Sigma\sim q}\left[\max_{\sigma\in S}d(\Sigma, \sigma)\right]$. 
\end{proof}
%
\section{Distributionally Robust Optimization (DRO)}

Distributionally Robust Optimization (DRO) is a classical setting for incorporating risk-aversion, used in various statistical learning communities. In the context of ranking, this notion has not \intodo{Sure?} been defined and used yet. We propose to extend the DRO statistic to ranking data.

\begin{definition}\label{def:dro}
    {\sc{DRO Statistic}} Let $p\in\cM_+^1(\pS)$ and $d$ be a metric on $\pS$. The DRO statistic is defined by:
    \begin{align*}
        \sigma^{DRO}_{d, p} \in \argmin_{\sigma \in \pS} \max_{q | TV(p,q) \leq \varepsilon} \mathbb{E}_{\Sigma \sim q}(d(\sigma, \Sigma))\,.
    \end{align*}
\end{definition}

As the DRO statistic inherently optimize on the worst-case adversarial distribution for $p$ (which is reminiscent, but not necessarilly equivalent, of the inner term in the breakdown function definition), it makes a good robust statistic candidate. 

\paragraph{Bucket orders are solution for the DRO.} \cref{sec:setting} briefly mentioned the intuition that letting bucket orders be the output of statistics is a good approach to increase robustness. The DRO statistic motivates this remark because it can be shown that for some distributions $p\in\cM_+^1(\pS)$, whe have $\pi^{DRO}_{p,H} \in \Pi_n \setminus \frak{S}_n$, meaning that bucket orders are indeed solutions to the robust DRO statistic. 

As an example, let us fix $\varepsilon < 2 \min_{\sigma \in \frak{S}_n} p(\sigma)$, let us write $\sigma^* = \sigma^{med}_{p,d_{\tau}}$ the Kemeny median for Kendall-tau distance and let us fix $\sigma \in \frak{S}_n$ and $\tau$ a transposition. Then, let's define the following distribution set

\begin{equation*}
    \begin{split}
    \mathcal{U}_{\varepsilon, \sigma, \tau} = \left\{ p \in \Delta^{n!} \; | \; : \forall \nu \neq \sigma, \sigma\tau, \, p(\nu)=\frac{1-\eta}{n!} \right. \\
     p(\sigma\tau) \in [\max(\frac{1-\eta}{n!}, p(\sigma) - \varepsilon)], \\
     \left. p(\sigma) \in [\frac{1-\eta}{n!}, 1], \, \forall \eta \in [0, 1-n! \varepsilon] \right\}
    \end{split}
\end{equation*}

On the set $\mathcal{U}_{\varepsilon, \sigma, \tau}$, which is of positive mass, the DRO statistic is a non-strict bucket order. 

Here is a sketch of proof. Define $p \in \mathcal{U}_{\varepsilon, \sigma, \tau}$ the following distribution: $\forall \nu \neq \sigma, \sigma \tau, \,  p(\nu) = \frac{1-\eta}{n!}; p(\sigma) = \frac{1-\eta}{n!} + \eta - \lambda$ and $p(\sigma \tau) = \frac{1-\eta}{n!} + \lambda$ for a $\eta \in [0, 1-n! \varepsilon], \; \lambda \in [\max(0, \frac{\eta - \varepsilon}{2}), \frac{\eta}{2}]$.

It can be easily checked that $\sigma = \sigma^{med}_{p, d_{\tau}}$. Moreover, the adversarial distribution for the DRO (i.e. the inner max part) is the following: $\forall \mu \neq \nu, \nu_{opp} \; q_{\varepsilon, \nu}(\mu)= p(\mu), \, q_{\nu}(\nu) = p(\nu) - \varepsilon/2$ and $q_{\nu}(\nu_{opp}) = p(\nu) + \varepsilon/2$. Then, let's write $\pi = \{\sigma, \sigma \tau\}$ we have the following result: $q_{\sigma}^T D_{\tau} \sigma - q_{\pi}^T D_{\tau} \pi = q_{\sigma}^T D_{\tau} \sigma - q_{\sigma}^T D_{\tau} \pi = \frac{1}{2}(p(\sigma \tau) - p(\sigma) + \varepsilon) \geq 0$.


\newpage
\section{Alternative Formulation for Algo}

Let $\pi\in\wO$ be  a bucket ranking and remember that $\pi^{(i)}$ denotes the $i^{\rm th}$ bucket of $\pi$. Let $P$ be a pairwise marginal matrix $P$, we define

$\bar{P}_{ij}(\pi) = \max \left\{\left|P_{l,l'} - \frac{1}{2}\right| : (l,l')\in\pi^{(m)}\times\pi^{(m')}, (m,m')\in[n]^2 ~s.t.~ i\leq m \leq m'\leq j\right\}$

$\cG(\pi,t) = \left\{(i,j)\in [n]^2: \bar{P}_{ij}(\pi) \leq t\right\}$

\begin{algorithm}
\DontPrintSemicolon
\SetKwInOut{Input}{Input}
\SetKwInOut{Output}{Output}
\Input{Pairwise matrix $P$, Ranking median $\sigma$, threshold $t \in [0, 0.5]$.}
$\pi \gets \sigma$ \tcp*{$\sigma$ being a specific bucket order}
\While{$\cG(\pi, t) \neq \emptyset$}{
    $(i^*, j^*) = \argmin_{(i,j)\in\cG(\pi,t)} \bar{P}_{ij}(\pi)$ \;
    update $\pi$ by merging all buckets between $i^*$ and $j^*$
    \vskip -2em
    \begin{flushleft}
        \begin{flalign*} 
            \begin{cases}
                \pi^{(i)} &\gets \pi^{(i)} ~~~\text{for}~ i < i^*\\
                \pi^{(i^*)} & \gets \bigcup_{l\in[n], i^*\leq l\leq j^*}\pi^{(l)}\\
                \pi^{(i - j^* + i^*)} & \gets \pi^{(i)} ~~~\text{for}~ i > j^*
            \end{cases}&&
        \end{flalign*}
    \end{flushleft}
    }
    \Output{$\pi$}
\caption{Na\"ive Merge}
\label{algo_maxpair}
\end{algorithm}

\begin{algorithm}
\DontPrintSemicolon
\SetKwInOut{Input}{Input}
\SetKwInOut{Output}{Output}
\Input{Pairwise matrix $P$, Ranking median $\sigma$, threshold $t \in [0, 0.5]$.}
$\pi \gets \sigma$ \tcp*{$\sigma$ being a specific bucket order}
\While{$\cG(\pi, t) \neq \emptyset$}{
    $(i^*, j^*) = \argmax_{(i,j)\in\cG(\pi,t)} \bar{P}_{ij}(\pi)$ \;
    update $\pi$ by merging all buckets between $i^*$ and $j^*$
    \vskip -2em
    \begin{flushleft}
        \begin{flalign*} 
            \begin{cases}
                \pi^{(i)} &\gets \pi^{(i)} ~~~\text{for}~ i < i^*\\
                \pi^{(i^*)} & \gets \bigcup_{l\in[n], i^*\leq l\leq j^*}\pi^{(l)}\\
                \pi^{(i - j^* + i^*)} & \gets \pi^{(i)} ~~~\text{for}~ i > j^*
            \end{cases}&&
        \end{flalign*}
    \end{flushleft}
    }
    \Output{$\pi$}
\caption{Downward Merge}
\label{algo_maxpair}
\end{algorithm}


\section{Material removed}

Moreover, inter-buckets ranking remains challenging in a general case, which can be addressed when restricting the analysis to \textit{(Strictly) Stochastically Transitive} distributions.

\begin{definition}
    {\sc (Stochastic transitivity)} A distribution $p\in\cM_+^1(\pS)$ is said to be stochastically transitive (ST) iif $\forall \, (i,j,k) \in [\![1,n]\!], P_{i,j} \geq 1/2$ and $P{j,k} \geq 1/2 \Rightarrow P_{i,k} \geq 1/2$. It is said to be strictly stochastically transitive if the inequalities are strict.
\end{definition}

When distribution $p\in\cM_+^1(\pS)$ is SST, the pairwise matrices provide a simpler way to characterize Kemeny's median:

\begin{remark} {\sc (Uniqueness of the ranking median)} Let $p \in \Delta^{\frak{S}_n}$ be a SST distribution, and let $d_{\tau}$ be Kendall-tau distance. Then, the median $\sigma^{med}_{d_{\tau}}(p)$ is unique and can be defined as $\forall i \in [\![1, n ]\!],  \sigma^{med}_{d_{\tau}}(p)(i) = 1 + \sum_{k \neq i} \mathbb{1}(P_{i,k} < 1/2)$~ \cite{Korba2017}.
\label{rk:sst_unique}
\end{remark}

This result derives from the definition of SST distributions and the following simplification remark.

\begin{remark} {\sc Pairwise matrix simplification.} Let $d_{\tau}$ be the Kendall-tau distance, $p\in\cM_+^1(\pS)$ a distribution and $\sigma \in \frak{S}_n$ a ranking. Then $\mathbb{E}_{\Sigma \sim p}(d_{\tau}(\Sigma, \sigma)) = \sum_{i<j} P_{i,j} \mathbb{1}(\sigma(i) > \sigma(j)) + (1-P_{i,j})\mathbb{1}(\sigma(i) < \sigma(j))$~ \cite{Korba2017}.
\label{rk:pairwise_simpl}
\end{remark}


\begin{algorithm}
\SetKwInOut{Input}{Input}
\SetKwInOut{Output}{Output}
\Input{Pairwise matrix $P$ of a SST distribution, threshold $t \in [0, 0.5]$.}
\Output{Naïve Merge statistic $\pi^{merge}(p)$}
\While{$\exists (i,j)$ s.t. $i<j, |P_{i,j} - 1/2| \leq t$}{
    $(i^*, j^*) = \argmin_{i<j} |P_{i,j} - 1/2|$ \;
    Put $(i^*, j^*)$ in the same bucket \;
    Update pairwise matrix $P$: $\forall i,j, \; P_{i^*, j} = P_{j^*,j} = max(P_{i^*,j} , P_{j^*,j}); \; P_{i,i^*} = P_{i, j^*} = max(P_{i,i^*}, P_{i,j^*})$ and $P_{i^*,j^*} = 1/2$
    }
    Return $\pi^{NM}(p)$ s.t. $\forall i \in [\![1,n]\!]\, \pi^{NM}(p)(i) = 1 + \sum_{k \neq i} \mathbb{1}(P_{i,k} < 1/2)$
\caption{Naïve Merge Algorithm}
\label{algo_merge}
\end{algorithm}


\begin{algorithm}
\SetKwInOut{Input}{Input}
\SetKwInOut{Output}{Output}
\Input{Pairwise matrix $P$ of a SST distribution, threshold $t \in [0, 0.5]$.}
\Output{Downward Merge statistic $\pi^{maxpair}(p)$}
\While{$\exists (i,j)$ s.t. $|P_{i,j} - 1/2| \leq t$}{
    $(i^*, j^*) = \argmin_{i,j : i \succ j} \left| |P_{i,j} - 1/2|- t \right|$ \;
    $\forall \, l \text{ s.t. } i^* \succ l \succ j^*$ and s.t. $|P_{i^*,l} - 1/2| \leq t$ and $|P_{l,j^*}-1/2| \leq t$ put $i^*, j^*, l$ in the same bucket \;
    Update pairwise matrix $P$: $\forall i,j, \; P_{i^*, j} = P_{j^*,j} = P_{l,j} = \max(P_{i^*, j}, P_{j^*,j}, \{P_{l,j}\}); \; P_{i,i^*} = P_{i, j^*} = P_{i,l} = \max(P_{i,i^*}, P_{i, j^*}, \{P_{i,l}\})$ and $P_{i^*,j^*} = P_{i^*,l} = P_{l,j^*} = 1/2$
    }
    Return $\pi^{DM}(p)$ s.t. $\forall i \in [\![1,n]\!]\, \pi^{DM}(p)(i) = 1 + \sum_{k \neq i} \mathbb{1}(P_{i,k} < 1/2)$
\caption{Downward Merge Algorithm}
\label{algo_maxpair}
\end{algorithm}


%%%%%%%%%%%%%%%%%%%%%%%%%%%%%%%%%%%%%%%%%%%%%%%%%%%%%%%%%%%%%%%%%%%%%%%%%%%%%%%
%%%%%%%%%%%%%%%%%%%%%%%%%%%%%%%%%%%%%%%%%%%%%%%%%%%%%%%%%%%%%%%%%%%%%%%%%%%%%%%


\end{document}


% This document was modified from the file originally made available by
% Pat Langley and Andrea Danyluk for ICML-2K. This version was created
% by Iain Murray in 2018, and modified by Alexandre Bouchard in
% 2019 and 2021 and by Csaba Szepesvari, Gang Niu and Sivan Sabato in 2022.
% Modified again in 2023 by Sivan Sabato and Jonathan Scarlett.
% Previous contributors include Dan Roy, Lise Getoor and Tobias
% Scheffer, which was slightly modified from the 2010 version by
% Thorsten Joachims & Johannes Fuernkranz, slightly modified from the
% 2009 version by Kiri Wagstaff and Sam Roweis's 2008 version, which is
% slightly modified from Prasad Tadepalli's 2007 version which is a
% lightly changed version of the previous year's version by Andrew
% Moore, which was in turn edited from those of Kristian Kersting and
% Codrina Lauth. Alex Smola contributed to the algorithmic style files.
