%The level of delegation to be granted to AI systems will in particular heavily depend on how methodological research replies to questions of robustness. This naturally brings us back to the development of statistical learning techniques that are reliable even in presence of partly contaminated data, due to biases in measurements or the deliberate intention to impair the operation of the automated system. Preference data, observed in the form of (complete) rankings in the simplest situations, are no exception of course and the demand for appropriate concepts and tools is all the more pressing given that technologies fed by or producing this type of data (\textit{e.g.} search engines, recommending systems) are now massively deployed. The lack of vector space structure for the set of rankings (\textit{i.e.} the symmetric group $\mathfrak{S}_n$) and the very complex nature of statistics usually considered in ranking data analysis make the formulation of robustness objectives in this domain extremely challenging. In this paper, we introduce notions of robustness, together with dedicated statistical methods, for \textit{Consensus Ranking}, the flagship problem in ranking data analysis, aiming at summarizing a probability distribution on $\mathfrak{S}_n$ by a \textit{median} ranking. Precisely, we propose specific extensions of the popular concept of \textit{breakdown point}, tailored to consensus ranking, and address the related computational issues. Beyond the theoretical contributions, the relevance of the approach proposed is supported by a detailed experimental study.

As the issue of robustness in AI systems becomes vital, statistical learning techniques that are reliable even in presence of partly contaminated data have to be developed. Preference data, in the form of (complete) rankings in the simplest situations, are no exception and the demand for appropriate concepts and tools is all the more pressing given that technologies fed by or producing this type of data (\textit{e.g.} search engines, recommending systems) are now massively deployed. However, the lack of vector space structure for the set of rankings (\textit{i.e.} the symmetric group $\mathfrak{S}_n$) and the complex nature of statistics considered in ranking data analysis make the formulation of robustness objectives in this domain challenging. In this paper, we introduce notions of robustness, together with dedicated statistical methods, for \textit{Consensus Ranking} the flagship problem in ranking data analysis, aiming at summarizing a probability distribution on $\mathfrak{S}_n$ by a \textit{median} ranking. Precisely, we propose specific extensions of the popular concept of \textit{breakdown point}, tailored to consensus ranking, and address the related computational issues. Beyond the theoretical contributions, the relevance of the approach proposed is supported by an experimental study.
