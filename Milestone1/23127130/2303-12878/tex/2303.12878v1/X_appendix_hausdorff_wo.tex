\section{Hausdorff Extensions of Kendall Tau}
\label{app:hausdorff_kendall}

We remind first the Kendall-tau distance, defined by: $$d_{\tau}: (\sigma_1, \sigma_2) \in \frak{S}_n \times \frak{S}_n \to \sum_{i<j} \mathbb{1}( (\sigma_1(i)-\sigma_1(j))(\sigma_2(i)-\sigma_2(j)) < 0 )$$  and the  \cref{def:non_symmetric_hausdorff,def:symmetric_hausdorff} of the Hausdorff extensions of the Kendall tau metric.

\defnonsymmetrichausdorff*

\defsymmetrichausdorff*

\begin{restatable}{proposition}{propcomplexityhausdorffkendall}\label{prop:complexity_hausdorff_kendall}
For any $\pi_1, \pi_2\in\wO$, the computation cost of $H_{d_\tau}^{\textsc{ns}}(\pi_1, \pi_2)$ and $H_{d_\tau}^{(1/2)}(\pi_1, \pi_2)$ is $\cO(n^2)$.
\end{restatable}


The average Hausdorff distance can be expressed with various expressions, necessitating the following notations (see \cite{fagin2006comparing}): 
\begin{enumerate}
    \item $\forall \, i \in [\![1,n]\!] \quad \bar{\pi}(i) = \sum_{\sigma \in \pi} \sigma(i)$ is the rank of item $i$ according to weak order $\pi$.
    \item $S(\pi_1, \pi_2) = \{ (i <j ) \; | \; \bar{\pi}_1(i)\neq\bar{\pi}_1(j), [\bar{\pi}_1(i)-\bar{\pi}_1(j)][\bar{\pi}_2(i)-\bar{\pi}_2(j)] < 0 \}$ is the set of item pairs $(i<j)$ that are in different buckets in both $\pi_1$ and $\pi_2$, and that are in different orders in $\pi_1$ and $\pi_2$.
    \item $S(\pi_1 \setminus \pi_2) = \{(i<j) \; | \; \bar{\pi}_1(i) = \bar{\pi}_1(j) \text{ and } \bar{\pi}_2(i) \neq \bar{\pi}_2(j) \}$ is the set of item pairs $(i<j)$ such that both items are in the same bucket in $\pi_1$ but in different ones in $\pi_2$.
    \item $\prof(\pi) = (\prof(\pi)_{i,j})_{i<j}$, where $\forall \; i<j, \prof(\pi)_{i,j} = 1/2$ if $\bar{\pi}(i) < \bar{\pi}(j)$, $= 0$ if $\bar{\pi}(i) = \bar{\pi}(j)$ and $= -1/2$ if $\bar{\pi}(i) > \bar{\pi}(j)$. $\prof(\pi)$ is called the profile vector of $\pi$.
\end{enumerate}

We have the following equivalent expressions for the average Hausdorff distance:

\begin{restatable}[Average Hausdorff distance]{proposition}{propavghausdorff}\label{prop:avg_hausdorff_expressions}
\begin{align}
    H_K^{(1/2)}(\pi_1, \pi_2) &:= \# S(\pi_1, \pi_2) + \frac{1}{2} \left( \#S(\pi_1 \setminus \pi_2) + \#S(\pi_2 \setminus \pi_1) \right) \\
    &= \sum_{i<j} \mathbb{1}\left( [\bar{\pi}_1(i)-\bar{\pi}_1(j)][\bar{\pi}_2(i)-\bar{\pi}_2(j)] < 0 \right) + \nonumber \\
    & \quad \quad \quad \frac{1}{2} \mathbb{1}\left( [\bar{\pi}_1(i)=\bar{\pi}_1(j)] \right)\mathbb{1}\left( [\bar{\pi}_2(i)\neq\bar{\pi}_2(j)] \right) + \nonumber \\
    & \quad \quad \quad\frac{1}{2} \mathbb{1}\left( [\bar{\pi}_2(i)=\bar{\pi}_2(j)] \right)\mathbb{1}\left( [\bar{\pi}_1(i)\neq\bar{\pi}_1(j)] \right) \\
    &= \| \prof(\pi_1) - \prof(\pi_2) \|_1
    \label{eq:avg_hausdorff_expressions}
\end{align}
\end{restatable}

\begin{restatable}[Avergage Hausdorff distance - Proof]{proof}{proofavghausdorff}\label{proof:avg_hausdorff_expressions}
Let $\pi_1$, $\pi_2$ be two weak orders associated with buckets $(B^1_1,...B^1_{t_1})$ and $(B^2_1,...B^2_{t_2})$ respectively. Such buckets are sets of items $i$ forming a partition of $[\!1,n]\!]$ such that $i \in B^1_k$ iif $\bar{\pi}_1(i) = \sum_{k'< k} \#B^1_{k'} + \frac{\#B^1_k + 1}{2}$  (see \cite{fagin2006comparing} for a more formal definition). Let us define, as in \cite{critchlow2012metric, fagin2006comparing}, ${\forall \; i \leq t_1, \forall \; j \leq t_2, \quad n_{i,j}= \#(B_i \cap B_j)}$.

Then we have \cite{critchlow2012metric}[Chapter IV]: $H_K^{(1/2)} = \frac{1}{2} \left( \sum_{i<i', j \geq j'}n_{i,j}n_{i',j'} + \sum_{i \leq i', j > j'}n_{i,j}n_{i',j'} \right)$.

By noting that $2 \# S(\pi_1, \pi_2) = \sum_{i<i', j > j'}n_{i,j}n_{i',j'}$ and $2 \#S(\pi_1 \setminus \pi_2) = \sum_{i=i', j > j'}n_{i,j}n_{i',j'}$, we derive our first equality. The second equality directly comes from re-expressing the first one. The third equality comes from \cite{fagin2006comparing}.


\end{restatable}


