\section{Robustness - Breakdown Function for Ranking and Bucket Rankings}
\label{sec:robustness}

This section first details how to apply the notion of \textit{breakdown function} $\varepsilon^\star_{d, p, T}$. This allows providing insights into the robustness of classical location statistics such as the Kemeny consensus. These results advocate for the introduction of a more robust type of statistics based on bucket orders that are also developed in this section.

\subsection{Breakdown Function for the Kemeny Consensus}\label{sec:bd_kem}

We explore the robustness of ranking medians $\sigma^{\rm med}_{d}(p)$ as defined in \cref{eq:ranking_median} for different metrics $d$ over $\pS$ as defined by the breakdown function $\varepsilon^\star_{d_\tau, p, T}$. In particular, it is possible to tightly sandwich the breakdown function for the Kemeny median.

\begin{restatable}{theorem}{thmbreakdownfunctionkemeny}\label{thm:breakdownfunctionkemeny}
For $p\in\cM_+^1(\pS)$, ~ $\sigma^\star_p = \sigma^{\rm med}_{d_\tau}(p)$ (Kemeny median) and $\delta\geq0$, if $\varepsilon^+(\delta) \leq 2 p(\sigma_{p}^{*})$ then $\varepsilon^\star_{d_\tau, p,\sigma^\star_p}(\delta) \leq \varepsilon^+(\delta)$ with
    \begin{align*}
        \varepsilon^+(\delta) =
    \min_{\substack{\sigma \in\pS \\ d_\tau(\sigma, \sigma^\star_p) \geq \delta}} 
    \max_{\substack{\nu\in\pS \\ d_\tau(\nu, \sigma^\star_{p}) < \delta}} 
    \frac{\lE_{\Sigma\sim p}\left[d_\tau(\Sigma, \sigma) - d_\tau(\Sigma, \nu)\right]}
    {d_\tau(\sigma^\star_p, \sigma) - d_\tau(\sigma^\star_p, \nu) }\,.
    \end{align*}
\end{restatable}
\begin{proofsketch} Detailed Proof can be found in \cref{app:breakdown_function_kemeny_ub}
The proof relies on showing that, for $\varepsilon>0$, the \emph{attack} distribution $\bar{q}_\varepsilon = p - \frac{\varepsilon}{2}\indicator{\cdot = \sigma^*_p} + \frac{\varepsilon}{2}\indicator{\cdot = \sigma^{\star, {\rm rev}}_p}$, where $\sigma^{\star, {\rm rev}}_p$ is the reverse of $\sigma^\star_p$, is in the feasible set of the optimization problem $\sup_{q: \textsc{tv}(p,q)\leq \varepsilon}d_\tau(\sigma^*_p, \sigma^*_{q})$ (see \cref{def:breakdown_function}). 

Using $\bar{q}_\varepsilon$ provides a way to link $\varepsilon$ and $\delta$.
The condition $\varepsilon^+(\delta) \leq 2 p(\sigma^\star_p)$ ensures $\bar{q}_\varepsilon$ is well-defined.
\end{proofsketch}

It is also possible to provide a lower bound on the breakdown function for any generic ranking median.

\begin{restatable}{theorem}{thmubbreakdownfunctionmedian}\label{thm:ubbreakdownfunctionmedian}
For $p\in\cM_+^1(\pS)$, $m$ and $d$ being two metrics on $\pS$, ~ $\sigma^\star_p = \sigma^{\rm med}_{d}(p)$ and $\delta\geq0$, we have $\varepsilon^\star_{m, p,\sigma^\star_p}(\delta) \geq \varepsilon^-(\delta)$ with
    \begin{align*}
        \varepsilon^-(\delta) =
    \min_{\substack{\sigma \in\pS \\ m(\sigma, \sigma^\star_p) \geq \delta}} 
    \max_{\substack{\nu\in\pS \\ \nu \neq \sigma}} 
    \frac{\lE_{\Sigma\sim p}\left[d(\Sigma, \sigma) - d(\Sigma, \nu)\right]}
    {\max_{\sigma'\in\pS} d(\sigma', \sigma) - d(\sigma', \nu)}
    \end{align*}
\end{restatable}
\begin{proof}
Detailed proof can be found in \cref{app:breakdown_function_median_lb}.
\end{proof}


\begin{figure}[htbp]
    \centering
    \includegraphics[width=0.45\textwidth]{img/bounds_breakdown_borda_footrule_kemeny.pdf}
    \caption{An illustration of $\varepsilon^+(\delta)$ and $\varepsilon^-(\delta)$ (from \cref{thm:breakdownfunctionkemeny} and \cref{thm:ubbreakdownfunctionmedian}) for a distribution on permutations of 4 items. For Borda and the median associated with Spearman footrule, only the lower bound is displayed.}
    \label{fig:breakdown_of_medians}
\end{figure}
\cref{fig:breakdown_of_medians} shows that no choice of $d$ makes the median uniformly more robust than another. Then, unfortunately, it also illustrates the fragility of median statistics against corruption of the distribution. In this example, impacting the distribution $p$ by less than $5\%$ allows changing the Kemeny median by flipping more than half item pairs ($\delta \geq 0.5$).

\paragraph{Sensitivity to similar items.} 
To further illustrate the fragility of Kemeny's median, \cref{fig:kemeny_on_indifference} shows its breakdown function on specific distributions. As could be expected, if all items are almost indifferent (uniform distribution - purple curve), then a ranking median is very fragile: a small nudge on $p$ is enough to change the Kemeny median from one ranking to its reverse. On the contrary, when $p$ is a point mass at a given ranking (blue curve), it requires a large attack on $p$ to impact the median. 

The green curve shows a weakness in the median: despite $p$ being concentrated on two neighbouring rankings (identical up to a pair of adjacent items), the robustness is very low for $\delta \leq 0.2$. This highlights a mechanism underlying adversarial attacks in real-world recommender systems (ex: fake reviews...): at a small cost, it is possible to be systematically ranked on top of close alternatives. This calls for using the natural alternative to (strict) rankings, which incorporates indifference between items: \emph{bucket rankings}.

\begin{figure}[htbp]
    \centering
    \includegraphics[width=0.45\textwidth]{img/theoretical_kemeny.pdf}
    \caption{Breakdown function for Kemeny's median for different distributions $p$. "Uniform" denotes an almost uniform distribution; "Point mass" an almost point mass distribution, and "Bucket" an almost point mass distribution on two neighboring rankings.}
    \label{fig:kemeny_on_indifference}
\end{figure}

