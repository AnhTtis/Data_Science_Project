\section{Proof: Bound on Breakdown Function for Ranking Medians}
\label{app:breakdown_function_medians}

\subsection{Upper-bound}
\label{app:breakdown_function_kemeny_ub}

We first remind \cref{thm:breakdownfunctionkemeny}.
\thmbreakdownfunctionkemeny*

We re-state the theorem with the matrix notation defined in \cref{app:notation} and used all along the appendix.

\begin{theorem}
For $p\in\cM_+^1(\pS)$, $\sigma^\star_p = \sigma^{\rm med}_{d_\tau}(p)$ and $S_\delta = \{\sigma\in\pS | d_\tau(\sigma, \sigma^\star_p) \geq \delta\}$, if $\varepsilon^+(\delta) \leq 2 p(\sigma^\star_p)$, then $\varepsilon^\star_{d_\tau, p, \sigma^\star_p} \leq \varepsilon^+(\delta)$.
\begin{align}
    \varepsilon^+(\delta)
    =
    \min_{\sigma\in S_\delta}\max_{\nu\in N_\delta}
    \frac{\mkendall{(\sigma-\nu)}{p}}
    {\mkendall{(\sigma - \nu)}{\sigma^\star_p}}\,,
\end{align}

\end{theorem}
\begin{proof}
    \begin{align}
        \varepsilon^\star_{d_\tau, p, \sigma^\star_p} & = \inf \left\{\varepsilon>0\middle|\sup_{q: \textsc{tv}(p, q)\leq \varepsilon}d_\tau(\sigma^\star_p, \sigma^\star_q) \geq \delta\right\} \\
        & = \inf \left\{\varepsilon>0\middle|\exists q, s.t. \textsc{tv}(p, q)\leq \varepsilon ~\text{and}~ d_\tau(\sigma^\star_p, \sigma^\star_q) \geq \delta\right\}\\
        & = \inf \underbrace{\left\{\varepsilon>0\middle|\exists q, s.t. \textsc{tv}(p, q)\leq \varepsilon ~\text{and}~ \argmin_{\sigma\in\pS}\mkendall{\sigma}{q} \subseteq S_\delta\right\}}_{=: E} ~~~\text{with}~S_\delta = \{\sigma\in\pS | d_\tau(\sigma, \sigma^\star_p) \geq \delta\}
    \end{align}
    Further, we define 
    $N_\delta = \pS \setminus S_\delta$, 
    $\sigma^{\star, {\rm rev}}_p$ the reverse of $\sigma^{\star}_p$, i.e., $\sigma^{\star, {\rm rev}}_p (i)=\sigma^{\star}_p(n-i-1)$ and the \emph{attack} distribution 
    ${\bar{q}_\varepsilon = p - \frac{\varepsilon}{2}\indicator{\cdot = \sigma^\star_p} + \frac{\varepsilon}{2}\indicator{\cdot = \sigma^{\star, {\rm rev}}_p}}$ that removes the probability mass from the median to put it on the farthest point. 
    We also define 
    ${{E} = \left\{\varepsilon | \argmin_{\sigma\in\pS}\mkendall{\sigma}{\bar{q}_\varepsilon} \subseteq S_\delta\right\}}$  and 
    ${\tilde{E} = \left\{0<\varepsilon \leq 2 p(\sigma^\star_p) \middle | \argmin_{\sigma\in\pS}\mkendall{\sigma}{\bar{q}_\varepsilon} \subseteq S_\delta\right\} \subseteq E \cap (0, 2 p(\sigma^\star_p)]}$.


    Let $\varepsilon>0$ be such that $\varepsilon \leq 2 p(\sigma^\star_p)$. Then
    \begin{align}
        \varepsilon\in\tilde{E} 
        & \Leftrightarrow \exists \sigma \in S_\delta, \forall \nu \in N_\delta, \mkendall{\sigma}{\bar{q}_\varepsilon} \leq \mkendall{\nu}{\bar{q}_\varepsilon}\\
        & \Leftrightarrow \exists \sigma \in S_\delta, \forall \nu \in N_\delta, \mkendall{(\sigma - \nu)}{p} + \frac{\varepsilon}{2}\left(\mkendall{\sigma^{\star, {\rm rev}}_p}{\sigma} - \mkendall{\sigma^{\star}_p}{\sigma} + \mkendall{\sigma^{\star}_p}{\nu} - \mkendall{\sigma^{\star, {\rm rev}}_p}{\nu}\right)\leq 0\\
        & \Leftrightarrow \exists \sigma \in S_\delta, \forall \nu \in N_\delta, \mkendall{(\sigma - \nu)}{p} 
        \leq 
        \frac{\varepsilon}{2}\left(
             \mkendall{(\sigma - \nu)}{(\sigma^\star_p - \sigma^{\star, {\rm rev}}_p)}
        \right)\\
        & \Leftrightarrow \exists \sigma \in S_\delta, \forall \nu \in N_\delta, \mkendall{(\sigma - \nu)}{p} 
        \leq 
        \varepsilon\left(
             \mkendall{(\sigma - \nu)}{\sigma^\star_p}
        \right)&\!\!\!\!\!\!\!\!\!\!\!\!\!\!\!\!\!\!\!\!\!\!\!\!\!\!\!\!\!\!\!\!\!\!\!\!\!\!\!\!\!\!\!\!\!\!\!\!\!\!\!\!\!\!\!\!\!\!\!\!\!\!\!\!\!\!\!\!\!\!\!\!\!\!\!\!\!\!\!\!\text{as }\mkendall{\cdot}{\sigma^{\star, {\rm rev}}_p} = \|D_\tau\|_\infty - \mkendall{\cdot}{\sigma^{\star}_p}\\
        & \Leftrightarrow \exists \sigma \in S_\delta, \forall \nu \in N_\delta, \frac{\mkendall{(\sigma - \nu)}{p}}{\mkendall{(\sigma - \nu)}{\sigma^\star_p}}
        \leq 
        \varepsilon\\
        & \Leftrightarrow \min_{\sigma \in S_\delta}\max_{\nu \in N_\delta} \frac{\mkendall{(\sigma - \nu)}{p}}{\mkendall{(\sigma - \nu)}{\sigma^\star_p}}
        \leq 
        \varepsilon \label{eq:in_tilde_E}
    \end{align}
    Now, denoting $\varepsilon^+(\delta) = \min_{\sigma \in S_\delta}\max_{\nu \in N_\delta} \frac{\mkendall{(\sigma - \nu)}{p}}{\mkendall{(\sigma - \nu)}{\sigma^\star_p}}$, by definition $\varepsilon^+(\delta)$ satisfies \cref{eq:in_tilde_E}, which means $\varepsilon^+(\delta) \in \tilde{E}$ iff $\varepsilon^+(\delta) \leq 2 p(\sigma^\star_p)$.
    Thus, if $\varepsilon^+(\delta) \leq 2 p(\sigma^\star_p)$, then 
    \begin{align}
         \varepsilon^+(\delta) = \inf \tilde{E} \geq \inf E = \varepsilon^\star_{d_\tau, p, \sigma^\star_p}.
    \end{align}
\end{proof}

\subsection{Lower-bound}
\label{app:breakdown_function_median_lb}
We first remind \cref{thm:ubbreakdownfunctionmedian}.
\thmubbreakdownfunctionmedian*

We re-state the theorem with the matrix notation defined in \cref{app:notation}.

\begin{theorem}
For $p\in\cM_+^1(\pS)$, $d$ and $m$ two metrics on $\pS$ and $\sigma^\star_p = \sigma^{\rm med}_{d}(p)$, we have
\begin{align}
    \varepsilon^\star_{m, p, \sigma^\star_p} \geq \min_{\sigma\in S_\delta}\max_{\nu\in\pS: \nu\neq\sigma}\frac{\mdist{(\sigma-\nu)}{p}}{\|D(\sigma-\nu)\|_\infty}\,,
\end{align}
where $S_\delta = \{\sigma\in\pS | d_\tau(\sigma, \sigma^\star_p) \geq \delta\}$.

\end{theorem}
\begin{proof}
Let $S_\delta, N_\delta, E, \tilde{E}$ are defined as above.
    \begin{align}
        \varepsilon^\star_{m, p, \sigma^\star_p} & = \inf \left\{\varepsilon>0\middle|\sup_{q: \textsc{tv}(p, q)\leq \varepsilon}m(\sigma^\star_p, \sigma^\star_q) \geq \delta\right\} \\
        & = \inf \left\{\varepsilon>0\middle|\exists q, s.t. \textsc{tv}(p, q)\leq \varepsilon ~\text{and}~ m(\sigma^\star_p, \sigma^\star_q) \geq \delta\right\}\\
        & = \inf \underbrace{\left\{\varepsilon>0\middle|\exists q, s.t. \textsc{tv}(p, q)\leq \varepsilon ~\text{and}~ \argmin_{\sigma\in\pS}\mdist{\sigma}{q} \subseteq S_\delta\right\}}_{=: E} ~~~\text{with}~S_\delta = \{\sigma\in\pS | m(\sigma, \sigma^\star_p) \geq \delta\}.
    \end{align}
    Now,
    \begin{align}
        \varepsilon \in E & \Leftrightarrow \exists q, s.t. \textsc{tv}(p, q)\leq \varepsilon ~\text{and}~ \argmin_{\sigma\in\pS}\mdist{\sigma}{q} \subseteq S_\delta\\
        & \Leftrightarrow \exists q \in \Delta^\pS, \textsc{tv}(p, q)\leq \varepsilon ~\text{and}~ \exists \sigma\in S_\delta, \forall \nu\in\pS, \mdist{\sigma}{q} \leq \mdist{\nu}{q}\\
        & \Leftrightarrow \exists q \in \Delta^\pS, \textsc{tv}(p, q)\leq \varepsilon ~\text{and}~ \exists \sigma\in S_\delta, \forall \nu\in\pS, \mdist{(\sigma-\nu)}{p} \leq \mdist{(\sigma-\nu)}{(q_- - q_+)}\\
        & ~~~~~~~\text{where}~ q_+ = (q-p)_+ ~~\text{and}~~ q_- = (p-q)_+\nonumber\\
        & \Rightarrow \exists q \in \Delta^\pS, \textsc{tv}(p, q)\leq \varepsilon ~\text{and}~ \exists \sigma\in S_\delta, \forall \nu\in\pS, \mdist{(\sigma-\nu)}{p} \leq \|q_+-q_-\|_1 \|D(\sigma-\nu)\|_\infty\\
        & \Rightarrow \exists \sigma\in S_\delta, \forall \nu\in\pS, \mdist{(\sigma-\nu)}{p} \leq \varepsilon\|D(\sigma-\nu)\|_\infty & \!\!\!\!\!\!\!\!\!\!\!\!\!\!\!\!\!\!\!\!\!\!\!\!\!\!\!\!\!\!\!\!\!\text{as }\|q_+-q_-\|_1\leq \varepsilon\\
        & \Rightarrow \exists \sigma\in S_\delta, \forall \nu\in\pS, s.t. \sigma \neq \nu, \frac{\mdist{(\sigma-\nu)}{p}}{\|D(\sigma-\nu)\|_\infty} \leq \varepsilon\\
        & \Rightarrow \min_{\sigma\in S_\delta}\max_{\nu\in\pS: \nu\neq\sigma}\frac{\mdist{(\sigma-\nu)}{p}}{\|D(\sigma-\nu)\|_\infty} \leq \varepsilon.
    \end{align}
    Finally,
    \begin{align}
        \varepsilon^\star_{m, p, \sigma^\star_p} & = \inf E \geq \min_{\sigma\in S_\delta}\max_{\nu\in\pS: \nu\neq\sigma}\frac{\mdist{(\sigma-\nu)}{p}}{\|D(\sigma-\nu)\|_\infty}\,.
    \end{align}
\end{proof}



% \begin{proof}
% As a reminder, for the Kemeny rule, the breakdown function is
% \begin{align}
%     \varepsilon^\star_{p,\sigma^\star_p}(\delta) 
%     & = \inf \left\{\varepsilon>0\middle|\sup_{q: \textsc{tv}(p, q)\leq \varepsilon}d_\tau(\sigma^\star(p), \sigma^\star(q)) \geq \delta\right\}\,.
% \end{align}


% The inner part of the breakdown function problem of level $\delta$ consists in finding the adversarial distribution $q^*_{\sigma^*}$ on which Kemeny's rule outputs a median at distance at least $\delta$ from $\sigma^*$. Let us order the losses for each $\sigma$, writing $[p^T D_{\tau}]_{(1)} \geq ... \geq [p^T D_{\tau}]_{(n!)}$ the ordered losses for $\sigma_{(1)}, ..., \sigma_{(n!)}$ respectively. Note that with this notation, $\sigma^{med}_{p, d_{\tau}} := \sigma^* = \sigma_{(n!)}$. Then, the goal of $q^*_{\sigma^*}$ is to modify optimally $p$ (under probability and budget constraints) so that there exists a $\sigma_{(l)}$ such that $d_{\tau}(\sigma_{(l)}, \sigma_{(n!)}) \geq \delta$ for which $(q^*_{\sigma^*})^T D_{\tau} \sigma_{(l)}$ is minimal. To achieve this, an important remark stemming from the nature of the Kendall tau distance is the following: when modifying $p$, the relative increase of loss for any $\nu \in \frak{S}_n$ with respect to $\sigma_{(n!)}$ does not depend on the modification of mass granted to $\nu$, but only to that of $\sigma_{(n!)}$.

% More specifically, let us consider the simple case where $\forall \sigma, \varepsilon \leq 2 \min(1-p(\sigma), p(\sigma_{opp}))$. Let's define two adversarial distributions:

% \begin{equation}
% \begin{split}
%     & q_{1}(\sigma_{(n!)}) = p(\sigma_{(n!)}) - \varepsilon/2 \\
%     & q_{1}(\sigma_{(l)}) = p(\sigma_{(l)}) + \varepsilon/2 \\
%     & q_{1}(\sigma) = p(\sigma) \text{ } \forall \sigma \neq (\sigma_{(n!)}, \sigma_{(l)})
% \end{split}
% \end{equation}

% \begin{equation}
% \begin{split}
%     & q_{2}(\sigma_{(n!)}) = p(\sigma_{(n!)}) - \varepsilon/2 \\
%     & q_{2}((\sigma_{(n!)})_{opp}) = p((\sigma_{(n!)})_{opp}) + \varepsilon/2 \\
%     & q_{2}(\sigma) = p(\sigma) \text{ } \forall \sigma \neq (\sigma_{(n!)}, (\sigma_{(n!)})_{opp})
% \end{split}
% \end{equation}

% Then we have that $q_1^T D_{\tau} \sigma_{(l)} - q_1^T D_{\tau} \sigma_{(n!)} = q_2^T D_{\tau} \sigma_{(l)} - q_2^T D_{\tau} \sigma_{(n!)} = p^T D_{\tau} \sigma_{(l)} - p^T D_{\tau} \sigma_{(n!)} - \varepsilon \delta$. Thus, it is not needed to decide/find which $\sigma_{(l)}$ should be the output of Kemeny's rule for $q^*_{\sigma^*}$ since an optimal strategy is simply to remove as much probability mass as possible from $\sigma^*$ and to allocate it to $\sigma^*_{opp}$.

% %The objective of $q^*_{\sigma^*}$ is thus to increase sufficiently the loss of $\sigma^*$, so to solve:

% %\begin{equation}
% %\begin{split}
% %    q^*_{\sigma^*} & = \max_{q \in \Delta^{n!}} \mathbb{E}_{\Sigma \sim q}(d_{\tau}(\sigma^*, \Sigma)) \text{ s.t. } TV(p,q) \leq \varepsilon \\
% %    & = \max_{q \in \Delta^{n!}} q^{T} D_{\tau}(\sigma^*) \text{ s.t. } TV(p,q) \leq \varepsilon
% %\end{split}
% %\label{eq:median}
% %\end{equation}

% %where $D_{\tau}(\sigma^*) := \left[ d_{\tau}(\sigma^*, \sigma_i) \right]_{1 \leq i \leq n!}$ is the vector of Kendall-tau distances with respect to $\sigma^*$. \cref{eq:median} is thus a linear problem under budget and probability constraint. Let us write $(\nu_i)_{1 \leq i \leq n!}$ and arbitrary sequence ordered as follows: $d_{\tau}(\sigma^*, \nu_1) \geq ... \geq d_{\tau}(\sigma^*, \nu_{n!})$, so that $\nu_1 = (\sigma^*)^{R}$ and $\nu_{n!} = \sigma^*$, where $(\sigma^*)^{R}$ denotes the opposite ranking to $\sigma^*$. The intuition is that $q^*_{\sigma^*}$ will take as much budget as possible (under the aforementioned constraints) from $\nu_1 = \sigma^*$ to give it to $\nu_{n!} = (\sigma^*)^{R}$.\\

% %In the simple case where $\varepsilon \leq 2 \min(1-p((\sigma^*)^{R}), p(\sigma^*))$, we have:

% %\begin{equation}
% %\begin{split}
% %    & q^*_{\sigma^*}(\sigma^*) = p(\sigma^*) - \varepsilon/2 \\
% %    & q^*_{\sigma^*}((\sigma^*)^{R}) = p((\sigma^*)^{R}) + \varepsilon/2 \\
% %    & q^*_{\sigma^*}(\sigma) = p(\sigma) \text{ } \forall \sigma \neq (\sigma^*, (\sigma^*)^{R})
% %\end{split}
% %\label{eq:adv_distrib_median_v1}
% %\end{equation}

% In all generality, $\exists k \text{ s.t. } 2 \sum_{l=1}^{k-1} p(\nu_l) < \varepsilon \leq 2 \sum_{l=1}^k p(\nu_l)$ and we withdraw budget sequentially from the $\nu$ closest to $\sigma^*$ until reaching $\varepsilon/2$ budget (which happens for $\nu_k$) to add it sequentially to $\nu$ closest to $(\sigma^*)^R$.
% \begin{equation}
% \begin{split}
%     & q^*_{\sigma^*}(\nu_l) = 0 \text{ } \forall l > k \\
%     & q^*_{\sigma^*}(\nu_k) = p(\nu_k) - \left( \frac{\varepsilon}{2} - \sum_{l=1}^{k-1} p(\nu_l) \right) \\
%     & q^*_{\sigma^*}(\nu_l) = \max \left( p(\nu_l), \min\left[ 1, p(\nu_l) + \frac{\varepsilon}{2} - \sum_{m=l}^{n!} (1-p(\nu_m)) \right] \right) \text{ } \forall l \leq k-1
% \end{split}
% \label{eq:adv_distrib_median_v2}
% \end{equation}

% In the simple setting from \cref{eq:adv_distrib_median_v1}, we have $\forall \nu, \; (q^*_{\sigma^*})^T D_{\tau} \nu = p^T D_{\tau} \nu - \varepsilon/2 ( ((\sigma^*)^{R})^T D_{\tau} \nu - (\sigma^*)^T D_{\tau} \nu ) )$ and thus in particular $(q^*_{\sigma^*})^T D_{\tau} \sigma^* = p^T D_{\tau} \sigma^* - \varepsilon/2 \| d_{\tau} \| $, where $D_{\tau}$ is Kendall-tau distance matrix. Thus,

% \begin{equation}
% \begin{split}
%     (q^*_{\sigma^*})^T D_{\tau} \sigma \leq (q^*_{\sigma^*})^T D_{\tau} \nu & \Leftrightarrow p^T D_{\tau} \sigma - p^T D_{\tau} \nu - \frac{\varepsilon}{2}(\| d_{\tau} \| - (\sigma^{R})^T D_{\tau} \nu + \sigma^T D_{\tau} \nu) \leq 0 \\
%     & \Leftrightarrow \varepsilon \geq 2 \frac{p^T D_{\tau} \sigma - p^T D_{\tau} \nu }{\|d_{\tau}\| - (\sigma^{R})^T D_{\tau} \nu + \sigma^T D_{\tau} \nu)}
% \end{split}
% \label{eq:bkdwn_median_compute}
% \end{equation}

% If one wants a distance $\delta$ at least between $\sigma^*$ and $\nu$, we need $(q^*_{\sigma^*})^T D_{\tau} \sigma \leq (q^*_{\sigma^*})^T D_{\tau} \nu$ for any $\nu$ at distance $\delta$ or more to $\sigma^*$ and for all $\nu$ at distance strictly less than $\delta$ tp $\sigma^*$. Thus:

% \begin{equation}
% \begin{split}
%     \varepsilon^\star_{\delta, d_{\tau}}(p, \text{Kemeny rule}) = \min_{\sigma | d_{\tau}(\sigma, \sigma^*_p) \geq \delta} \max_{\nu | d_{\tau}(\nu, \sigma^*_{p}) < \delta} 2 \frac{p^T D_{\tau} \sigma - p^T D_{\tau} \nu}{\|d_{\tau}\| - ((\sigma^*)^{R})^T D_{\tau} \nu + (\sigma^*)^T D_{\tau} \nu) }
% \end{split}
% \label{eq:bkdwn_median}
% \end{equation}



% \end{proof}


