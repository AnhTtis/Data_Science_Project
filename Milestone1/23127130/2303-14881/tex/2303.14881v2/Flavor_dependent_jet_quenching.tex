
\documentclass[aps,twocolumn,superscriptaddress,notitlepage,nofootinbib]{revtex4-1}

\usepackage{amsmath,bm}
\usepackage{graphicx}
\usepackage{epstopdf}
\usepackage{subfigure}
\usepackage{epsfig}
\usepackage{amsmath,amssymb,amsfonts}
\usepackage{color}
\usepackage[utf8]{inputenc}
\usepackage{verbatim}
\usepackage{array}
\setlength{\textheight}{24.0cm}
\usepackage{hyperref}
\usepackage{ulem}
\usepackage{multirow}

\newcommand{\is}[1]{\textcolor{red}{#1}}
\newcommand{\iss}[2]{\textcolor{blue}{\sout{#1}#2}}

\begin{document}


\renewcommand{\topfraction}{0.85}
\renewcommand{\textfraction}{0.1}
\renewcommand{\floatpagefraction}{0.75}


\title{Flavor dependence of jet quenching in heavy-ion collisions from a Bayesian analysis}

\date{\today  \hspace{1ex}}


\author{Shan-Liang Zhang}
\email{Corresponding author.\\
zhangshanl@m.scnu.edu.cn}
\affiliation{Key Laboratory of Atomic and Subatomic Structure and Quantum Control (MOE), Guangdong Basic Research Center of Excellence for Structure and Fundamental Interactions of Matter, Institute of Quantum Matter, South China Normal University, Guangzhou 510006, China}
\affiliation{Guangdong-Hong Kong Joint Laboratory of Quantum Matter, Guangdong Provincial Key Laboratory of Nuclear Science, Southern Nuclear Science Computing Center, South China Normal University, Guangzhou 510006, China}
\affiliation{
Department of Physics, Hubei University, Wuhan 430062, China
}

\author{Enke Wang}
\email{wangek@scnu.edu.cn}
\affiliation{Key Laboratory of Atomic and Subatomic Structure and Quantum Control (MOE), Guangdong Basic Research Center of Excellence for Structure and Fundamental Interactions of Matter, Institute of Quantum Matter, South China Normal University, Guangzhou 510006, China}
\affiliation{Guangdong-Hong Kong Joint Laboratory of Quantum Matter, Guangdong Provincial Key Laboratory of Nuclear Science, Southern Nuclear Science Computing Center, South China Normal University, Guangzhou 510006, China}

\author{Hongxi Xing}
\email{Corresponding author.\\
hxing@m.scnu.edu.cn}
\affiliation{Key Laboratory of Atomic and Subatomic Structure and Quantum Control (MOE), Guangdong Basic Research Center of Excellence for Structure and Fundamental Interactions of Matter, Institute of Quantum Matter, South China Normal University, Guangzhou 510006, China}
\affiliation{Guangdong-Hong Kong Joint Laboratory of Quantum Matter, Guangdong Provincial Key Laboratory of Nuclear Science, Southern Nuclear Science Computing Center, South China Normal University, Guangzhou 510006, China}
\affiliation{Southern Center for Nuclear-Science Theory (SCNT), Institute of Modern Physics, Chinese Academy of Sciences, Huizhou 516000, China}

\author{Ben-Wei Zhang}\email[]{bwzhang@mail.ccnu.edu.cn}
\affiliation{Institute of Particle Physics and Key Laboratory of Quarks and Lepton Physics (MOE), Central China Normal University, Wuhan 430079, China}



\begin{abstract}
We investigate the flavor dependence of jet quenching, by performing a systematic analysis of medium modifications on the inclusive jet, $\gamma$+jet, and $b$-jet in Pb+Pb collisions at the LHC. Our results from MadGraph+PYTHIA exhibit excellent agreement with experimental measurements of the inclusive jet, $\gamma$+jet and $b$-jet simultaneously  in p+p collisions. We then utilize a Bayesian data-driven method to extract systematically the flavor-dependent jet energy loss distributions from experimental data, where the gluon, light quark and $b$-quark initiated energy loss distributions are well constrained and satisfy the predicted flavor hierarchy of jet quenching, i.e. $\langle \Delta E_g \rangle > \langle\Delta E_q\rangle > \langle\Delta E_b\rangle$. It is shown that the 
quark-initiated jet energy loss distribution shows weaker centrality and $p_\text{T}$ dependence than the gluon-initiated one. We demonstrate the impacts of the slope of initial spectra,
color-charge as well as parton mass dependent jet energy attenuation on the $\gamma/b$-jet suppression observed in heavy-ion collisions.
  %We find that large quark-initiated jet fraction contribute $80\%$ to large $p_\text{T}$ $\gamma$+jet suppression, while the flat spectra have the dominate contributions to low $p_\text{T}$ $\gamma$+jet suppression.
  %Furthermore, $b$-initiated jets is less suppressed compared to light-quark initiated jets,  but the mass effects  decrease as $p_\text{T}$.
   %Except for mass effect, smaller fraction of gluon contributions to $b$-jet also lead to less suppression of $b$-jet compared to inclusive jet in low $p_\text{T}$ jet region.


\end{abstract}

\pacs{13.87.-a; 12.38.Mh; 25.75.-q}

\maketitle

%\section{introduction}
%\label{sec:Intro}

\section{Introduction}
%Jet quenching has long been proposed as an excellent probe of the properties of the Quark-Gluon Plasma (QGP) \cite{Wang:1991xy,Wang:1998ww}, formed in high-energy heavy-ion collisions at Relativistic Heavy Ion Collider (RHIC)\cite{Adcox:2001jp,Adler:2002xw, Adler:2002tq} and the Large Hadron Collider (LHC) \cite{Aad:2010bu,Chatrchyan:2012gt,Chatrchyan:2013kwa,Aad:2014bxa}.
The understanding of strongly interacting nuclear matter at extremely high temperature and energy density is one of the primary subjects in the study of high-energy nuclear collisions at Relativistic Heavy Ion Collider (RHIC)\cite{Adler:2002xw, Adler:2002tq} and the Large Hadron Collider (LHC) \cite{Aad:2010bu,Chatrchyan:2012gt,Chatrchyan:2013kwa,Aad:2014bxa}. Jet quenching has long been identified as a very powerful tool to investigate the phase transition from hadron gas to  the quark-gluon plasma (QGP) with deconfined quarks and gluons~\cite{ Wang:1991xy, Wang:1998ww}, and numerous studies have shown that parton energy loss in the QGP  may lead to the  suppression of the single inclusive hadron/jet spectra~\cite{Qin:2007rn,Chen:2011vt,Buzzatti:2011vt,Majumder:2011uk,Aamodt:2010jd,CMS:2012aa,Qiu:2019sfj},
 the shift of $\gamma$/Z+hadron/jet correlations \cite{Wang:1996yh,Renk:2006qg,Zhang:2009rn,Qin:2009bk,Adare:2009vd,Abelev:2009gu,Chen:2017zte,Dai:2012am,Neufeld:2010fj,Kang:2017xnc,Luo:2018pto,Zhang:2018urd,Sirunyan:2017jic,Yang:2021qtl,Zhang:2021oki,Zhang:2022bhq} and dihadron transverse momentum asymmetry~\cite{Zhang:2007ja,stardihadron,Ayala:2009fe,Ayala:2011ii}, the modification of jet internal structures~\cite{ATLAS:2019dsv,Zhang:2021sua,Chang:2019sae,Chang:2019sae,KunnawalkamElayavalli:2017hxo,Kang:2017mda,Ma:2013uqa,Chatrchyan:2014ava,Aaboud:2017bzv,Vitev:2008rz}, as well as the azimuthal anisotropy ($v_2$) of hadrons and jets~\cite{STAR:2003wqp,CMS:2017xgk,ATLAS:2018ezv,He:2022evt} with the large transverse momentum ($p_\text{T}$) in nucleus-nucleus (A+A) collisions, by comparison with those in proton-proton (p+p) collisions~\cite{Qin:2015srf,CMS:2021vui,Apolinario:2022vzg}.

% However, the details of the
%parton’s interactions with the medium, as well as the relative importance of each interaction mechanism, are not yet fully understood~\cite{Qin:2015srf,CMS:2021vui,Apolinario:2022vzg}.


 %Phenomenological studies of experimental data on these observables at RHIC and the LHC have provided important constraints on the properties of QGP, such as the extraction of jet transport coefficient $\hat{q}$, which indicates the transverse momentum broadening squared per unit distance caused by the jet-medium interaction, from a set of single inclusive hadron spectra by JET Collaboration \cite{Burke:2013yra} and from other observables \cite{Dai:2017piq,Dai:2017tuy,Ma:2018swx, Xie:2019oxg,Andres:2016iys,JETSCAPE:2021ehl,JETSCAPE:2020shq}.

%Recently, jet cone dependence of jet modification factor have attracted  many attentions. Jet spectra measurements are extended to large area jets~\cite{CMS:2021vui}, with an anti-$k_\text{T}$ distance parameter R up to 1.0. The new data  shows little $R$ dependence of $R_\text{AA}$  and  place further constraints on the underlying jet quenching mechanisms, as well as a challenge to theoretical models.Besides, STAR  have measured $\gamma/\pi^0$ tagged charged jet,  it  show that trigger tagged charged jet are less  suppressed with larger jet cone $R$. While ALICE find that  charged jet are  more suppressed compared  with  large jet cone $R$. Those experimental measurements show different  modification pattern on jet cone $R$ dependence  $R_\text{AA}$. Which also challenge our understanding that jet with larger $R$ will recover more radiated energy,  leading to less suppression.

%The fragmentations of large transverse momentum($p_\text{T}$) quarks or gluons result in groups of angularly-correlated particles, referred to as jets, which pave a new way to study the interactions between jet partons and the medium constituents. Production and suppression of fully reconstructed single jets~\cite{Aad:2014bxa,Adam:2015ewa,Khachatryan:2016jfl,Wang:2016fds,Chien:2015hda,Casalderrey-Solana:2014bpa,Tachibana:2017syd,Chen:2019gqo,Yan:2020zrz,He:2020iow,He:2018gks,CMS:2012ulu,ALICE:2013dpt}, di-jets~\cite{Aad:2010bu,Chatrchyan:2011sx,Vitev:2009rd,Qin:2010mn,Young:2011qx,He:2011pd,Renk:2012cx} and $\gamma/Z^0$-jets~\cite{Dai:2012am,Neufeld:2010fj,Kang:2017xnc,Chatrchyan:2012gt,Luo:2018pto,Zhang:2018urd,Sirunyan:2017jic,Yang:2021qtl} have also been studied in heavy-ioncollisions and they can provide additional constraints on the jet-medium interaction and jet transport coefficient.Investigations on the jet substructures: jet shape~\cite{Vitev:2008rz,Chatrchyan:2013kwa,Sirunyan:2018ncy,Chang:2019sae,KunnawalkamElayavalli:2017hxo,Kang:2017mda,Ma:2013uqa}, jet fragmentation functions~\cite{Chatrchyan:2014ava,Aaboud:2017bzv,Sirunyan:2018qec,Aaboud:2019oac,Chen:2020tbl}, groomed subjets~\cite{Sirunyan:2017bsd,Sirunyan:2018gct,Acharya:2019djg,Andrews:2018jcm,Casalderrey-Solana:2019ubu}, and jet-novel subjet~\cite{Apolinario:2017qay} put new insight into the energy-loss process and, ultimately, the properties of the QGP itself.

The interaction between an energetic parton and the QGP is sensitive to the colour charge and the mass of the parton, while medium-induced gluon
radiation is expected to be enhanced for gluon due to its larger color factor, and  to be suppressed for heavy quarks by the dead-cone effect relative to that for 
light quarks~\cite{Dokshitzer:2001zm,Zhang:2003wk,Djordjevic:2003qk,Armesto:2003jh}.
%While comprehensive efforts have been devoted to extract key transport properties of QGP based on jet quenching frameworks (see e.g. by JET and JETSCAPE collaborations \cite{JET:2013cls,JETSCAPE:2021ehl,JETSCAPE:2020shq}), the information on the jet quenching of specific parton type is still limited.
Such a predicted flavor hierachy of jet quenching can be identified by a separate determination of quark and gluon jet energy loss, which could play a significant role in revealing the fundamental color structures of the QGP and testing the color representation dependence of the jet-medium interaction~\cite{Frye:2017yrw,Gras:2017jty}. This however proves difficult, as the final state hadronic observables are a mixture of quark and gluon contributions. A clean method for identifying quark or gluon energy loss remains a challenge, despite many past attempts such as the multivariate analysis of jet substructure observables~\cite{Chien:2018dfn}, the proposal of using the averaged jet charge~\cite{Chen:2019gqo,CMS:2020plq,Li:2019dre}  and electroweak gauge boson tagged jet~\cite{Dai:2012am,Neufeld:2010fj,Kang:2017xnc,Luo:2018pto,Zhang:2018urd,CMS:2017eqd,Yang:2021qtl, Yan:2020zrz,Zhang:2021oki,Aaboud:2019oac}.


One recent important measurement by the ATLAS Collaboration, i.e. the  nuclear modification factor for  $\gamma$-tagged and $b$-tagged jets~\cite{ATLAS:2022cim,ATLAS:2022fgb}, shows quite different modification pattern from that of single inclusive jets~\cite{ATLAS:2018gwx}. It is reported that the $\gamma$-tagged jets $R_\text{AA}$~\cite{ATLAS:2022cim} are much higher and show a weaker centrality dependence than inclusive jet $R_\text{AA}$~\cite{ATLAS:2018gwx} , indicating a sensitive observation of color factor dependence of jet-medium interaction. In addition, the ratio of $R_\text{AA}$ between $\gamma$-tagged jet and inclusive jets are above most of the theoretical model calculations~\cite{ATLAS:2022cim}, which challenges the implemented color charge dependence of energy loss in these models. Likewise, systematic difference between $b$-jets and inclusive jets $R_\text{AA}$ are also observed~\cite{ATLAS:2022fgb} and , suggesting a role for mass and colour charge effects in partonic energy loss in heavy-ion collisions. Those differences may arise from not only the inclusive jet mixture of quarks and gluons, where gluon lose more energy, but also the slope of their initial spectrum~\cite{He:2018xjv}. 
 Meanwhile, most theoretical models can capture the inclusive jet $R_\text{AA}$~\cite{ATLAS:2018gwx}. However,  discrepancies arise when examining the latest photon/b-tagged jet $R_\text{AA}$ data points and the double ratio $R^{\gamma/b\text{+jet}}_\text{AA}/R^\text{inclusive jet}_\text{AA}$~\cite{ATLAS:2022cim,ATLAS:2022fgb}. In general, these quantities tend to surpass the predictions of many jet quenching models grounded in pQCD calculations and kinetic theory. Such contradictions strongly motivate a data-driven Bayesian analysis to extract the model-independent yet flavor-dependent jet energy loss distributions, which can not only identify the transport properties of QGP~\cite{Zhang:2022rby}, but also help to pin down the uncertainties of jet quenching models.
%\sout{Therefore, it is necessary to have explicit knowledge of model-independent but flavor-dependent jet energy loss distributions, which can help to constrain jet quenching model uncertainties and to identify the transport properties of QGP~\cite{Zhang:2022rby}.}

The purpose of this work is to extract the flavor-dependent and  model-independent jet energy loss distributions by performing a systematic study of the medium suppression of  the inclusive jet, $\gamma$+jet, and $b$-jet in Pb+Pb collisions relative to that in p+p in a unified framework simultaneously. In the numerical calculation of the p+p baseline, a Monte Carlo event generator MadGraph5+PYTHIA8~\cite{Alwall:2014hca}, which can perform the next-to-leading order (NLO) matrix element (ME) matched to the resummation of parton shower (PS) calculations, is employed to simulate initial hard partons with shower partons and jet cross sections in p+p collisions.
Specifically, a Bayesian data-driven analysis~\cite{He:2018gks} of the nuclear modification factors of inclusive jet~\cite{ATLAS:2018gwx}, $\gamma$+jet~\cite{ATLAS:2022cim}, and $b$-jet~\cite{ATLAS:2022fgb} is performed to quantitatively extract the flavor dependent jet energy loss distributions, which satisfies the predicted flavor hierarchy of jet quenching. We study the relative contributions from the slope of initial spectra, color-charge as well as parton mass dependent jet energy attenuation to the $\gamma/b$-jet suppression in heavy-ion collisions at the same time.

The remainder of the paper is organized as follows.  In Sec.~\ref{sec:Fra} we first introduce the framework.
%And then we present the evaluation of the inclusive jet, $\gamma$+jet, and $b$-jet cross section in p+p collisions using MadGarph+Pythia~\cite{Alwall:2014hca} simulations
With a systematic study of the inclusive jet,  $\gamma$+jet, and  $b$-jet productions in p+p collisions using MadGarph+Pythia, a Bayesian data-driven analysis of nuclear modification factors of these processes is performed to quantitatively extract flavor dependent jet energy loss distributions in Sec.~\ref{sec:nume}.  Finally, a summary is presented in Sec.\ref{summary}.


%\label{numerical}
%\section{Results from LBT simulation}
\section{Framework}
\label{sec:Fra}
In order to study the flavor dependence of jet energy loss, we express the final observable of the nuclear modification factor $R_\text{AA}$ in a given centrality in terms of the flavor dependent $R_\text{AA}^{i,C}$,
\begin{equation}
\begin{split}\label{EQ_raa}
  R_\text{AA}^{C}&= \frac{\sum_i R_\text{AA}^{i,C}d\sigma^{i}_\text{pp} }{\sum_id\sigma^{i}_\text{pp}}=R_\text{AA}^{g,C}+ \sum_{i\neq g}(R_\text{AA}^{i,C}-R_\text{AA}^{g,C} )f_{i},
 % &= \frac{\sigma_{c\rightarrow J/\psi}R_\text{AA}^c+\sigma_{g\rightarrow J/\psi}R_\text{AA}^g}{\sigma_{c\rightarrow J/\psi}+\sigma_{g\rightarrow J/\psi} }\\
 % &=R_\text{AA}^g+\frac{\sigma_{c\rightarrow J/\psi}(R_\text{AA}^c-R_\text{AA}^g )}{\sigma_{c\rightarrow J/\psi}+\sigma_{g\rightarrow J/\psi}}\\
 % &=R_\text{AA}^g+ (R_\text{AA}^q-R_\text{AA}^g )f_{q}
\end{split}
\end{equation}
%where $\sigma^{i}_\text{pp}$ is the cross section for parton $i$ in $p+p$ collisions, and $\sum_i \sigma^{i}_\text{pp}$ corresponds to the final observed cross section in $p+p$ collisions.
where the superscripts $i$ and $C$ stand for the parton flavor and centrality, respectively. $d\sigma^{i}_\text{pp}$ is the differential cross section for parton $i$ initiated jet in p+p collisions, $f_{i}=d\sigma^{i}_\text{pp}/\sum_id\sigma_\text{pp}^i$ is the fraction of the total jet cross section from the parton $i$ initiated one.

In our analysis, the flavor and centrality dependent nuclear modification factor $R_\text{AA}^{i,C}$ is assumed to be factorized as the convolution of its cross section in p+p collisions and the corresponding parton energy loss distribution~\cite{He:2018xjv,He:2018gks}
%\begin{equation}\label{straight}
%\frac{d \sigma^{g^\prime}}{ dp^{g^\prime}_\text{T}}=   \int \frac{dp_\text{T}}{\langle \Delta p_\text{T}\rangle} \frac{d \sigma^{pp}}{dp_\text{T}}(p_\text{T} )  \times W_\text{AA}(x)
%\end{equation}
\begin{equation}
R_\text{AA}^{i,C}(p_\text{T})= \frac{\int d\Delta p_\text{T} d\sigma_\text{pp}^i(p_\text{T}+\Delta p_\text{T})\otimes W_\text{AA}^{i,C}(x)}{d\sigma_\text{pp}^i(p_\text{T})},
\label{sigma_AQ}
\end{equation}
where $x=\Delta p_\text{T}/\langle \Delta p_\text{T}\rangle$ is the scaled variable with $\Delta p_\text{T}$ the amount of energy loss and $\langle \Delta p_\text{T}\rangle$ the averaged jet energy loss, which can be parametrized as $\langle \Delta p_\text{T}\rangle=\beta_i (p_\text{T})^{\gamma_i} \log(p_\text{T})$ following Refs.~\cite{He:2018xjv,Zhang:2021sua}.
 %We assume that the centrality and $x$ dependence of the energy loss distribution can be factorized as $W_\text{AA}^{i,C}(x)=\omega^{i,C}W^i_\text{AA}(x)$, where $f^{i,C}$ is the centrality dependent part, while $W^i_\text{AA}(x)=\frac{\alpha_i^{\alpha_i} x^{\alpha_i-1}e^{-\alpha_i x} }{\Gamma(\alpha_i)}$ is the scaled centrality-independent energy loss distributions of parton $i$.
In Eq. (\ref{sigma_AQ}), $W_\text{AA}^{i,C}$  is the scaled energy loss distribution of parton $i$ in a given centrality class $C$ of A+A collisions and can be assumed as:
\begin{equation}
W^{i}_\text{AA}(x)=\frac{\alpha_{i}^{\alpha_{i}} x^{\alpha_{i}-1}e^{-\alpha_{i} x} }{\Gamma(\alpha_{i})}
\label{Waa}
\end{equation}
where $\Gamma$ is the standard Gamma-function, and the above functional form can be empirically interpreted as the energy loss distribution
resulting from $\alpha_i$ number of jet-medium scattering in the medium.


In this setup, for each parton flavor $i$, the scaled jet energy loss distributions  $W_\text{AA}^{i}(x)$  can be determined by three parameters,  $\alpha_i, \beta_i, \gamma_i$. According to this flavor decomposition, one can extract $\alpha_i, \beta_i, \gamma_i,$ for each parton flavor $i$ to determine the flavor and  centrality dependent jet energy loss distributions  $W_\text{AA}^{i}(x)$ through a global analysis by combining the simulations of p+p cross section and the measurements of nuclear modification factor $R_\text{AA}$ for jet related observables.

We apply an advanced statistical tool, i.e. Bayesian analysis, for this purpose. Such a method has been successfully employed to extract the bulk and heavy quark~\cite{Xu:2017obm},  jet~\cite{He:2018gks} and gluon~\cite{Zhang:2022rby} energy loss distributions as well as  jet transport coefficients~\cite{Xie:2022ght,JETSCAPE:2021ehl} in heavy-ion collisions. The process can be summarized as
\begin{equation}
P(\theta|data) =\frac{P(\theta)P(data|\theta) }{ P(data)},
\label{Bayesian}
\end{equation}
where $P(\theta|data)$ is the posterior distribution of parameters $\theta$ given the experimental data, $P(\theta)$
is the prior distribution of $\theta$, $P(data|\theta)$ is the Gaussian likelihood between experimental data and the output for any given set of parameters and $P(data)$ is the evidence.  Uncorrelated uncertainties in experimental data are used in the evaluation
of the Gaussian likelihood.  To estimate the posterior distribution given by Eq.~\ref{Bayesian},
the Markov chain Monte Carlo (MCMC) process is carried out using the Methropolis-Hasting algorithm~\cite{Andrieu}. A uniform prior distribution $P(\theta)$ in the region $[\alpha_i, \beta_i, \gamma_i] \in [(0,10),(0,8),(0,0.8)]$ is used for the  Bayesian analysis.  We
first run $2 \times 10^6$ burn-in MCMC steps to allow the chain
to reach equilibrium, and then generate $2 \times 10^6$ MCMC steps
in parameter space. 
%{\color{red}It may be noted that the Bayesian analysis here uses specific functional form for the parameterzation, thus introducing long-range correlations in the parameter space which may potentially bias the extracted parameters. A possible solution to tackle such issue is to use information field based approach as presented in Ref.~\cite{Xie:2022ght}.}
%\textcolor{red}{what is $P(data)$? what's the prior distribution of $\omega_i^C$?}

\section{Results and Discussions}
\label{sec:nume}

\subsection{Cross sections in p+p }
In our analysis, we consider three different observables, i.e. the inclusive jet, $\gamma$+jet and $b$-jet, to study the flavor dependence of jet energy loss distribution. Considering the facts NLO matching have considerable contributions to  b-jet cross section~\cite{Banfi:2007gu}  and $\gamma$+jet~\cite{Zhang:2018urd},  we simulate $d\sigma^{i}_\text{pp}$ using a Monte Carlo event generator MadGraph5+PYTHIA8~\cite{Alwall:2014hca}, which combines the NLO matrix element (ME) with the matched parton shower (PS). Furthermore, those shower partons are reconstructed using the anti-$k_\text{T}$  algorithm~\cite{Cacciari:2008gp} implemented in the FastJet~\cite{Cacciari:2011ma}. In order to compare with the $b$-jet measurements, we define $b$-jet to be the one that contains at least one $b$-quark (or $\bar{b}$-quark) with momentum $p_\text{T}>5$ GeV/c and a radial separation from the reconstructed jet axis $\Delta R<0.3$. 
%\sout{In ATLAS measurements~[65-67], the jets are accepted in the rapidity range of $|y|<2.8$ for inclusive jet and $\gamma$+jet, $|y|<2.1$ for $b$-jet.}
In ATLAS measurements [65–67], inclusive jet and $\gamma$+jet
are reconstructed with $R=0.4$ and  accepted in the rapidity range of $|y|<2.8$,  while b-jet are reconstructed with $R=0.2$ and  accepted within $|y|<2.1$. Besides, for $\gamma$+jet event, $\gamma$ is required to have $p_\text{T}^\gamma>$ 50 GeV/$c$, and a cut $\Delta \phi_{\text{j}\gamma }>\pi/2$ is imposed to select the back-to-back $\gamma$+jet pairs. In our simulations, we implement correspondingly the same kinematic cuts adopted by experiments.

\begin{figure}[!t]
  \centering
  \vspace{15pt}
  \includegraphics[width=0.95\linewidth]{./pp_ref.pdf}
%\includegraphics[scale=0.34]{figures/delta_phi_0.3.png}
  \caption{(Color online)  Up: Transverse momentum distributions of: (a) inclusive jet, (b) $\gamma$-tagged jet, and (c) $b$-jet simulated by MadGraph+Pythia8 (lines) and the comparison with experimental data (samples)~\cite{ATLAS:2018gwx,ATLAS:2022cim,ATLAS:2022fgb} in p+p collisions.  The
inset in (a) is the  ratio of $\gamma$-tagged jet (blue solid) and $b$-jet (red dashed) to inclusive jet cross section.
   Bottom: fraction of quark (Dashed blue line) and gluon (Solid red line) initiated jet of : (d) inclusive jet, (e) $\gamma$-tagged jet, and (f) $b$-jet  as a function of jet $p_\text{T}$ in p+p collisions.
   }\label{ref_pp_lbt}
\end{figure}

In the top panel of Fig.~\ref{ref_pp_lbt}, we plot the differential cross section of: (a) inclusive jet (denoted as
“Incl. jet” in the figure in the following), (b) $\gamma$+jet, and (c) $b$-jet  as a function of jet transverse momentum $p_\text{T}$ obtained from MadGraph+Pythia8 simulation at 5.02 TeV in p+p collisions. Through the comparison with experimental data~\cite{ATLAS:2018gwx,ATLAS:2022cim,ATLAS:2022fgb}, one can see clearly that the simulations give very well descriptions of all experimental data. Notice that the inset of Fig.~\ref{ref_pp_lbt}(a) is the scaled ratio of $\gamma$+jet (blue solid) and  $b$-jet (red dashed) cross section to that of inclusive jet. In Fig. \ref{ref_pp_lbt}(a-c), one can see that the inclusive jet spectrum is much steeper than $\gamma$+jet, while $b$-jet have similar slope as the inclusive jet, consistent with the results of Refs.~\cite{ATLAS:2022cim,ATLAS:2022fgb}.

In order to study the flavor dependence of jet energy attenuation in heavy ion collisions, we present the relevant contributions in terms of jet flavor, which is defined as the flavor of the hard parton that fragments into the final observed jet\footnote{If $\geq 2$ hard partons locate in the final observed jet, the flavor of a jet is defined as that of the hardest parton.}.
In the  bottom panel of Fig.~\ref{ref_pp_lbt}, we show the fraction from quark- and gluon- initiated jet in: (d) inclusive jet, (e) $\gamma$+jet, and (f) $b$-jet as a function of jet $p_\text{T}$. One can see that for inclusive jet, the contribution from gluon (quark) initiated jet dominates in low (large) $p_\text{T}$ region, and gradually decreases (increases) with increasing $p_\text{T}$. While for $\gamma$+jet, the quark initiated jet dominates ($\sim 80\%$) in the whole $p_\text{T}$ region. For $b$-jet, it can be generated either from the initial hard scattering or from the parton showers via gluon and quark splitting. In the first case, it is the $b$-quark that initiates the $b$-jet, the relevant contribution is shown by $b$-quark in Fig.~\ref{ref_pp_lbt}(f). In heavy-ion collisions, the medium modification to such $b$-jet has direct connection to the heavy quark energy loss~\cite{Zhang:2003wk,Djordjevic:2003qk,Armesto:2003jh,Dai:2022sjk}. On the other hand, the medium modification on the latter two cases (with gluon and quark splitting) would resemble that of a massive quark or gluon jets. As can be seen, $b$-jet from gluon initiated contributes about $40\%$ to the cross section in the whole $p_\text{T}$ region, while the light quark initiated contribution goes up with increasing $p_\text{T}$.

\begin{figure}
  \centering
  \includegraphics[width=0.5\textwidth]{./correlations.pdf}
%\includegraphics[scale=0.34]{figures/delta_phi_0.3.png}
  \caption{(Color online) Distributions of and the correlations between the Bayesian-extracted parameters for gluon (left) and quark (right) initiated jet energy loss via fitting to $R_\text{AA}$ of inclusive jet and $\gamma$-tagged jet  in central 0-10$\%$ Pb+Pb collisions at 5.02 TeV~\cite{ATLAS:2018gwx,ATLAS:2022cim}.  }\label{parameter}
\end{figure}

%{\color{red}\sout{To benchmark the medium effect which bridges the initial flavor origin and the final observed jet attenuation in A+A collisions, \textcolor{blue}{and compare with the later model independent Bayesian analysis,} we present in the bottom panel of Fig.~\ref{ref_pp_lbt} the nuclear modification factor $R_\text{AA}$ evaluated as a function of jet $p_\text{T}$ for: (g) inclusive jet, (h) $\gamma$+jet, and (i) $b$-jet, and compared with ATLAS data~\cite{ATLAS:2022fgb,ATLAS:2018gwx,ATLAS:2022cim}. The theory curves are obtained from LBT model~\cite{He:2015pra,Cao:2016gvr}, which is based on the Boltzmann equation for both jet shower and recoil partons with perturbative QCD (pQCD) leading-order elastic scattering and induced gluon radiation according to the high-twist approach~\cite{Guo:2000nz,Zhang:2003yn,Zhang:2003wk,Zhang:2004qm}. The dynamic evolution of bulk medium is given by 3+1D CLVisc hydrodynamical model~\cite{Pang:2012he} with parameters fixed by reproducing hadron spectra from experimental measurement. More details on the implementation of the LBT model with the higher-twist formalism~\cite{Guo:2000nz,Zhang:2003yn,Zhang:2003wk,Zhang:2004qm} can be found in Refs.~\cite{He:2015pra,Cao:2016gvr,Luo:2023nsi}. Our results from LBT with $\alpha_s=0.18$, which is the only parameter in LBT that controls the strength of parton interaction, show excellent agreements with the experimental data for inclusive jet~\cite{ATLAS:2018gwx}, $\gamma$+jet~\cite{ATLAS:2022cim} and $b$-jet~\cite{ATLAS:2022fgb}. As can be seen from the figure, the nuclear modification factor $R_\text{AA}$ for $\gamma$+jet is larger than that of inclusive jet. This is attributed to their different quark and gluon origins and the slope of the reference spectra in p+p collisions. The nuclear modification factor $R_\text{AA}$ for $b$-jet is also larger than that of inclusive jet in low $p_\text{T}$ region, while the difference disappear at large $p_\text{T}$ region, which should be a mixed effect of color charge and parton mass dependence of jet quenching in medium.}}


\subsection{Colour-charge dependence of $R_\text{AA}$}



In Fig.~\ref{parameter}, we present the distributions of the final-extracted parameters for gluon (left) and quark (right) initiated jet energy loss as well as their correlations, via Bayesian-fitting  to the ATLAS data~\cite{ATLAS:2018gwx,ATLAS:2022cim} on $R_\text{AA}$ of inclusive jets and $\gamma$-tagged jets in 0-10$\% $ Pb+Pb collisions at 5.02 TeV simultaneously.
%We do not take the effect of different original position of $\gamma$-tagged jet and inclusive jet  in the medium on the $R_\text{AA}$ into consideration at present.
As can be seen, $\beta_i$ and $\gamma_i$, which  reflect the average energy loss,  are strongly correlated and  well constrained for quark and gluon initiated jet.   The  mean value
as well as its standard deviation of those final extracted parameters for gluon and charm quark energy loss distribution are summarized in Table~\ref{table:nrqcd}.

\begin{figure}
  \centering
 \includegraphics[width=0.45\textwidth]{./raa_ref_qg.pdf}
%\includegraphics[scale=0.34]{figures/delta_phi_0.3.png}
  \caption{(Color online) (a) Data-driven Bayesian fitted  nuclear modification factor $R_\text{AA}$ of inclusive jet (orange) and $\gamma$-tagged jet (gray) and the comparison to experimental data at~\cite{ATLAS:2018gwx,ATLAS:2022cim}.  (b) Data-driven extracted  nuclear modification factor of quark (blue) and gluon (red) initiated $\gamma$+jet. (c) Fraction of jet average energy loss of light quark (blue) and gluon (red) initiated jet, (d) scaled energy loss distributions $W_\text{AA}^i(x)$ of  quark (blue) and gluon (red) initiated jet.  }\label{raa_ref}
\end{figure}

The final fitted nuclear modification factor $R_\text{AA}$ of inclusive jet  and $\gamma$-tagged jet as well as the comparison to experimental data~\cite{ATLAS:2018gwx,ATLAS:2022cim} in 0-10$\%$ centrality at 5.02 TeV are shown in Fig.~\ref{raa_ref}.(a), and data-driven extracted  nuclear modification factor of quark- and gluon- initiated $\gamma$+jet are shown in Fig.~\ref{raa_ref}.(b). The corresponding bands are results with one sigma deviation from the average fits of $R_\text{AA}$.
Considering the fact that the training process will minimize the Gaussian likelihood function between experimental data and the output for any given set of parameters, the final fitted results are almost close to the central value of data points. Moreover, considering the limited experimental data sets, our parametrization for the energy loss distribution shown in Eq. (\ref{Waa}) is limited to three parameters for each flavor, which could introduce correlations between different data bins. Therefore, the MCMC bands is more restricted than the uncertainty in the experimental data. We leave a more detailed analysis of the uncertainties for a future publication.
%The quark initiated jet  $R_\text{AA}$ and energy lose can be well constrained when only $\gamma$-tagged jet data  are taken into consideration, due to large fraction of quark jet, while gluon initiated jet $R_\text{AA}$ and energy lose are weakly constrained.  When experimental data of inclusive jet are included, the gluon jet  $R_\text{AA}$ and energy lose  can be further constrained.
Data-driven extracted average energy loss fraction $\langle \Delta p_\text{T}\rangle/ p_\text{T}$ and scaled energy loss distributions $W_\text{AA}(x)$ of quark and gluon initiated jet are also presented in Fig.~\ref{raa_ref}.(c) and Fig.~\ref{raa_ref}.(d).  %{\color{red}\sout{ The results from LBT model simulations are also presented in Fig.~\ref{raa_ref} for comparison. The agreement with LBT simulation validates the Bayesian analysis in extracting flavor dependent jet energy loss.}  }
%\sout{\color{red}Fig.~\ref{raa_ref} also show the LBT results which are in agreement with the Bayesian results. }
As can be seen, average energy loss of gluon  and quark jet is well constrained in $p_\text{T}<200$ GeV/$c$, but is weaker constrained at high $p_\text{T}$ due to large experimental data errors and the scarcity of  $\gamma$-tagged jet experimental data at such high $p_\text{T}$. The quark-initiated jets lose less fraction of its energy and shows a weaker dependence on the jet $p_\text{T}$ compared to gluon-initiated jets due to its color factor as expected.
Since jet showers also contain gluons even if they are initiated by a hard quark, the net energy loss of a gluon-tagged jet is always larger than that of a quark-tagged jet but the ratio is smaller than 9/4 from the naive leading order estimation~\cite{Wang:1998bha,Liu:2006sf,Chen:2008vha}.






%\section{Bayesian analysis}

\begin{figure}
  \centering
  %\includegraphics[width=0.5\textwidth]{./figures/RAA_c_\text{T}est.pdf}
  \includegraphics[width=0.45\textwidth]{./RAA_gamma_ratio0.pdf}%RAA_c_\text{T}est   RAA_bjet_\text{T}est
%\includegraphics[scale=0.34]{figures/delta_phi_0.3.png}
  \caption{(Color online) (a) The reference $R_\text{AA}^{\text{ref}}$ (magenta) and the comparison with $R_\text{AA}$ of $\gamma$+jet (grey) and  inclusive jet (orange) jet in 0-10$\%$ centrality at 5.02 TeV, and the comparison with experimental data~\cite{ATLAS:2022cim}.  (b) The relative contribution fraction  from large quark fraction to the less suppression of $\gamma$+jet $R_\text{AA}$ compared to inclusive jet $R_\text{AA}$. }\label{Raa_gamma}
\end{figure}

Fig.~\ref{raa_ref}.(a) shows that 
 $\gamma$-tagged jet $R_\text{AA}$ is less suppressed compared to that for inclusive jet, which is a mix effect of the slope of initial spectra  and parton color-charge in p+p collisions.
To clarify the relative contributions from the color-charge effect and the initial parton spectra between $\gamma$-tagged jet and inclusive jet, we calculate an artificial reference $R_\text{AA}^{\text{ref}}$ following Eq.(\ref{EQ_raa}), by assuming the inclusive jet production has the same fraction of quark jet as $\gamma$+jet. This reference $R_\text{AA}^{\text{ref}}$ is shown by magenta lines in Fig.~\ref{Raa_gamma}.(a). 
The difference between $R_\text{AA}^{\text{ref}}$ and inclusive jet $R_\text{AA}$  (denoted as ``$R_\text{AA}^{\text{jet}}$")  should be attributed largely to the different color-charge effect between quark-medium and gluon-medium interactions, %  larger fraction of quark jet will lead to larger $R_\text{AA}$.
while the distinction between $R_\text{AA}^{\text{ref}}$ and $\gamma$+jet $R_\text{AA}$  (denoted as ``$R_\text{AA}^{\gamma\text{+jet}}$")  should be attributed mostly to the slope of reference spectra in p+p.

Fig.~\ref{Raa_gamma}.(b) shows the relative contribution fraction  from large quark fraction, evaluated as  $f^{\text{flavor}}=(R_\text{AA}^{\text{ref}}-R_\text{AA}^{\text{jet}})/(R_\text{AA}^{\gamma\text{+jet}}-R_\text{AA}^{\text{jet}})$, to the less suppression of $\gamma$+jet $R_\text{AA}$ compared to inclusive jet $R_\text{AA}$.  The increased quark jet fraction in inclusive jet production give the dominant contributions to the difference of  $R_\text{AA}$  between $\gamma$+jet and inclusive jet %$R_\text{AA}^{\gamma\text{+jet}}/ R_\text{AA}^{\text{jet}}$
at $p_\text{T}>60$ GeV/$c$. Then $1-f^{\text{flavor}}$ characterized approximately  the relative contribution from the slope of reference spectra, which plays a dominated role in the suppression at low $p_\text{T}$.
Besides, the distinction between $\gamma$+jet $R_\text{AA}$ and inclusive jet $R_\text{AA}$ will diminish with increasing $p_\text{T}$, because quark-initiated jets contribute a lion's share to the yields of both $\gamma$+jet and the inclusive jet at very large $p_\text{T}$, which can be verified with the upcoming high precision measurements at the LHC.
%The ratio of extracted nuclear modification factor $R_\text{AA}$ of $\gamma$+jet to inclusive jet as well as $R_\text{AA}^r/R_\text{AA}^{inclusive \ jet}$ in 0-10$\%$ centrality are also shown in Fig.~\ref{Raa_gamma}.(b). The artificial increased quark jet fraction in inclusive jet production leads to more than $80\%$ contributions to the ratio of $R_\text{AA}^{\gamma\text{+jet}}/ R_\text{AA}^{inclusive\  jet}$ in $80<p_\text{T}^{jet}<200$ GeV region, while the slope of reference spectra dominate in low $p_\text{T}$ region.\is{(don't quite understand this statement)} Besides, the difference between $\gamma$+jet $R_\text{AA}$ and inclusive jet $R_\text{AA}$ will decrease with increasing $p_\text{T}$, because the quark/ gluon fraction tend to be similar in large $p_\text{T}$ region, which can be verified with the upcoming high precision measurements at LHC.% \is{(how about the slop in large pt?)}.



\subsection{Centrality dependence of $R_\text{AA}$}
\begin{figure}
  \centering
  \includegraphics[width=0.5\textwidth]{./Raa_c_ref.pdf}
  %\includegraphics[width=0.5\textwidth]{./RAA_c_qg_global.pdf} %\includegraphics[width=0.5\textwidth]{./RAA_c_pre.pdf}
%\includegraphics[scale=0.34]{figures/delta_phi_0.3.png}
  \caption{(Color online) Data-driven fitted  nuclear modification factor $R_\text{AA}$ of the inclusive jet 
  $R_\text{AA}$~\cite{ATLAS:2018gwx} and  $\gamma$+jet $R_\text{AA}$~\cite{ATLAS:2022cim} in 10-30$\%$, 30-80$\%$ centrality bins  and predictions of inclusive jet $R_\text{AA}$ in 10-20$\%$, 20-30$\%$, 30-40$\%$ and 0-20$\%$   centrality bins as well as  the comparison with experimental data~\cite{ATLAS:2018gwx}.
 }\label{Raa_c_fit}
\end{figure}

\begin{footnotesize}
\begin{table}[!t]
\caption{Parameters [$\alpha_i$, $\gamma_i$, $\beta_i$] of quark and gluon jet energy loss distribution from Bayesian fits to experimental data~\cite{ATLAS:2018gwx,ATLAS:2022cim} on inclusive jet and $\gamma$+jet suppressions at 5.02 TeV. }
\label{table:nrqcd}
\begin{center}
  \begin{tabular}{l|c c c c}
  \hline
    &  &  {$\alpha_i$ } &   {$\beta_i$} & {$\gamma_i$}  \\
   \hline
   \multirow{2}*{0-10$\%$} & gluon & 4.36$\pm$2.07    & 1.78$\pm$0.38   &    0.25$\pm$0.03 \\
                           ~&quark & 0.5$\pm$0.07    & 0.39$\pm$0.17   &    0.32$\pm$0.13 \\ \hline

    \multirow{2}*{10-30$\%$} & gluon & 2.17$\pm$0.94    & 1.47$\pm$0.44   &    0.25$\pm$0.04 \\
                             ~&quark & 5.81$\pm$1.8    & 1.27$\pm$0.12   &    0.09$\pm$0.02 \\ \hline

    \multirow{2}*{30-80$\%$} & gluon & 4.78$\pm$1.87    & 1.16$\pm$0.17   &    0.11$\pm$0.03 \\
                             ~&quark & 6.4$\pm$2.63    & 0.7$\pm$0.05   &    0.09$\pm$0.01 \\ \hline
  % Chao 1 & 1.16$\pm$0.2 & 8.9$\pm$0.98 & 0.30$\pm$0.12  & 1.26$\pm$0.47 \\ \hline
  % Chao 2 & 1.16$\pm$0.2 & 11 &  0  & 0 \\ \hline
  % Gong & 1.16$\pm$0.2 & 9.7$\pm$0.9 &  -0.46$\pm$0.13 & -2.14$\pm$0.56 \\ \hline
  \end{tabular}
\end{center}
\end{table}
  \end{footnotesize}

Moreover, we extract the centrality dependent  quark and gluon jet energy loss distributions before exploring parton-mass effect on jet quenching  motivated by two reasons.
First, $\gamma$-tagged jet $R_\text{AA}$~\cite{ATLAS:2022cim} shows a weaker dependence on centrality compared to inclusive jet~\cite{ATLAS:2018gwx}, indicating that gluon-initiated jets may show a distinct centrality dependence with quark-initiated jets.
%Besides, we can not well constrain $b$-quark jet energy loss, light-quark and gluon initiated jet energy loss, when  only b\textrm{-}jet $R_\text{AA}$ and inclusive jet $R_\text{AA}$  are  included in the fitting due to their subdominant contributions  and the limited data points with large uncertainties.
Second, the experimental data of $\gamma$+jet $R_\text{AA}$~\cite{ATLAS:2022cim}, inclusive jet $R_\text{AA}$~\cite{ATLAS:2018gwx}  and $b$-jet  $R_\text{AA}$~\cite{ATLAS:2022fgb} are in different centrality bins.
We need centrality dependent quark and gluon jet energy loss distributions to fit $\gamma$+jet $R_\text{AA}$, inclusive jet $R_\text{AA}$  and $b$-jet  $R_\text{AA}$ simultaneously.



As a matter of fact, there are no experimental data of inclusive jet $R_\text{AA}$ and $\gamma$+jet $R_\text{AA}$ in the same centrality class except in central 0-10$\%$ centrality. For inclusive jet measurements, the existing measurements are provided for centrality bins 0-10$\%$, 10-20$\%$, 20-30$\%$,
30-40$\%$, 40-50$\%$,50-60$\%$, 60-70$\%$, 70-80$\%$~\cite{ATLAS:2018gwx}, while for $\gamma$+jet $R_\text{AA}$, it is  limited to 0-10$\%$, 10-30$\%$, 30-80$\%$~\cite{ATLAS:2022cim}. In order to take full advantage of the existing measurements for inclusive jet $R_\text{AA}$ in different centrality bins, we generate the inclusive jet $R_\text{AA}$ as well as the corresponding errors in 10-30$\%$ and 30-80$\%$ centrality bins according to $R_\text{AA}^C=\sum_{c\in C} P^{c} R_\text{AA}^c$, where $P^c =N^c_\text{bin}/\sum_c N^c_\text{bin}$ is the probability of finding jet events in a given centrality bin following Ref.~\cite{Xing:2019xae}. With such an extension, we can perform a simultaneous fit for both inclusive jet $R_\text{AA}$ and $\gamma$+jet $R_\text{AA}$ in 10-30$\%$ and 30-80$\%$ centralities. In Fig.~\ref{Raa_c_fit}, we present the data-driven fitted nuclear modification factor $R_\text{AA}$ of inclusive jet~\cite{ATLAS:2018gwx} and $\gamma$+jet ~\cite{ATLAS:2022cim} in 10-30$\%$ and 30-80$\%$ centralities and the comparison with experimental data at 5.02 TeV. %\iss{Similarly, those procedures are also applied  to inclusive jet $R_\text{AA}$ and $\gamma$+jet $R_\text{AA}$ in 30-80$\%$ centrality and the final fitted results are also shown in  Fig.~\ref{Raa_c_fit}.}
All final spectra based on Eq.~(\ref{EQ_raa}) and Eq.~(\ref{sigma_AQ}) are in nice agreement with the experimental data. The corresponding mean value
as well as its standard deviation of those final extracted parameters for gluon and light quark energy loss distribution are summarized in Table~\ref{table:nrqcd}.

Meanwhile, we obtain $R_\text{AA}$ for quark-initiated jets and gluon-initiated jets  in 10-30$\%$, 30-80$\%$ centrality. Combined with the flavor dependent $R_\text{AA}$ in 0-10$\%$ as extracted in the previous section (Fig.~\ref{raa_ref}.(b)), we obtain the centrality dependence of final
fitted gluon-initiated jet, quark-initiated jet and inclusive jet $R_\text{AA}$.  In Fig.~\ref{Raa_c}, we show the centrality dependence of final
fitted gluon jet (red), quark jet (blue) and inclusive jet (green)
$R_\text{AA}$ in Pb+Pb collisions  in the region $100<p_\text{T}<112 $ GeV/$c$ by step lines. One finds that the quark-initiated jet has weaker dependence on the centrality than that for gluon-initiated jet. %The open symbols are the centrality corresponding to the impact parameters in the three different centrality bins respectively. \is{(The $R_\text{AA}^{\text{jet}}$ in 30-80$\%$ is similar equal to that for 40-50$\%$.   )}.


\begin{figure}[t]
  \centering
  %\includegraphics[width=0.5\textwidth]{./Raa_c_ref.pdf}
  \includegraphics[width=0.5\textwidth]{./RAA_c_qg_global.pdf} %\includegraphics[width=0.5\textwidth]{./RAA_c_pre.pdf}
%\includegraphics[scale=0.34]{figures/delta_phi_0.3.png}
  \caption{(Color online) The centrality dependence of final fitted  gluon jet (red), quark jet (blue) and inclusive jet (green) $R_\text{AA}$ in Pb+Pb collisions at 5.02 TeV.  }\label{Raa_c}
\end{figure}

Next, we can fit the centrality dependent $R_\text{AA}$ of quark- and gluon- initiated jet via a simple parametrization $h^i(C)=a_i C^2+b_i C+c_i$, with $C$ stands for the centrality. The best fit curves of $h^i(C)$ are shown in Fig.~\ref{Raa_c} by blue and red band, and the corresponding best-fit parameter values are presented in Table.~\ref{table:parameters}.
%\iss{The fitted $f^i(c) $ is greater than one and unreasonable in peripheral collisions, which need to be further constrained  by  more centrality dependent $\gamma$-tagged jet  experimental data.}
Notice that the extrapolation to peripheral collisions ($>80\%$) is greater than one and can not be trusted, a reasonable identification of jet energy loss distribution for peripheral collisions would require a corresponding extension of experimental measurements.
If we ignore the $p_\text{T}$ dependence of $h^i(C)$, $R_\text{AA}^{i,C}$ for any centrality $C$ can be simply obtained by $R_\text{AA}^{i,C}=h^i(C)*R_\text{AA}^{i,rc}/h^i(rc)$, where $rc$ stands for reference centrality. Based on Eq.~(\ref{EQ_raa}) and the above extracted centrality dependent quark and gluon jet $h^i(C)$, the predication of inclusive jet $R_\text{AA}$ in 0-20$\%$, 10-20$\%$, 20-30$\%$, 30-40$\%$ are presented in Fig.~\ref{Raa_c_fit}. One can see that our extracted centrality dependence of quark and gluon jet energy loss distributions can describe the experimental data  $R_\text{AA}$~\cite{ATLAS:2018gwx} very well.
%\is{there are pt-dependence in those curves, need to explain the pt-dependence of $f^i(C)$.}

\begin{table}[!t]
  \caption{The best-fitted Parameters [$a_i$, $b_i$, $c_i$] for centrality dependent quark and gluon jet energy loss distributions. }
\begin{center}
 % \begin{tabular}{l|c|c|c|c}
  \begin{tabular}{p{1.0cm}<{\centering}|p{2.5cm}<{\centering}|p{2.2cm}<{\centering}|p{2.cm}<{\centering}}

   \hline
            & $a_i \  (\times 10^{-5})$          &    $b_i \ (\times 10^{-3}) $        &  $c_i$    \\ \hline
    Quark & 12.39$\pm$2.83    &    -2.95$\pm$1.74 &   0.7$\pm$0.021   \\ \hline
   Gluon &3.36$\pm$2.45   &  6.65$\pm$1.20   & 0.309$\pm$0.00879 \\  \hline
  \end{tabular}

  \label{table:parameters}
\end{center}
\vspace{-10pt}
\end{table}

%The jet energy loss distribution $W_\text{AA}^i(x)$  has an scaling behavior in the scaled variable $ x=\Delta p_\text{T}/\langle \Delta p_\text{T}\rangle$ and  approximately independent of the  centrality and the colliding energy for a given of heavy-ion collisions~\cite{He:2018xjv}. Therefore, the energy loss distribution for a given centrality class of A+A collisions can be assumed as: $W_\text{AA}^{i,c}(x)=  f^{i,c} W_\text{AA}^i(x)$,

%And then, the  centrality dependence distributions  of gluon jet and quark jet energy lose $f^{g,c}$  and $f^{q,c}$, can be extracted from a global  Bayesian  analysis with experimental data of inclusive jet $R_\text{AA}$ ~\cite{ATLAS:2018gwx} in  0-10$\%$,10-20$\%$,20-30$\%$,30-40$\%$ and $\gamma$+jet $R_\text{AA}$~\cite{ATLAS:2022cim} in $10-30\%$ centrality bins. The final fitted results are shown in Fig.~\ref{Raa_c}.
%Then the  centrality dependence  of  inclusive jet, gluon initiated jet and quark  initiated jet $R_\text{AA}$ are shown in the bottom panel  of  Fig.~\ref{Raa_c}. Specially, we need to point out that, we can not constrain the centrality dependence  of gluon and quark jet $R_\text{AA}$ with only one process.
%We can see that quark jet show weaker dependence on the centrality, while gluon jet  show a strong dependence on the centrality in central Pb+Pb collisions. And they tend to be similar for  peripheral collisions, which need to be further verified by  comparing with more experimental data  of centrality dependence of $\gamma$+jet $R_\text{AA}$ in the future.
%With those extracted centrality dependence quark and gluon jet energy lose distributions, we can make predictions of inclusive jet $R_\text{AA}$  in  any given  centrality  $C$  according: $R_\text{AA}^C=\sum_{c\in C} p^{c} R_\text{AA}^c$, where $p^c =N^c_\text{bin}/\sum_c N^c_\text{bin}$ is the probability of finding jet events in a given centrality bin. The predication of inclusive jet $R_\text{AA}$ in 0-20$\%$ and $\gamma$-tagged jet $R_\text{AA}$  in  30-80$\%$  as well as comparison with experimental data~\cite{ATLAS:2022cim} are also presented in Fig.~\ref{Raa_c}.(a). Our extracted centrality dependence jet energy lose distributions can well describe   $R_\text{AA}$  in any  centralities in heavy-ion collisions.
%$f^{g,c}=0.897+0.021*c$  and $f^{q,c}=0.99+0.0023*c$

\begin{table}[!t]
  \caption{Parameters [$\alpha_i$, $\gamma_i$, $\beta_i$] for gluon, quark and $b$-quark  energy loss in 0-20$\%$  centrality from  fitting to  $\gamma$+jet $R_\text{AA}$~\cite{ATLAS:2022cim}, inclusive jet $R_\text{AA}$~\cite{ATLAS:2018gwx}  and $b$-jet  $R_\text{AA}$~\cite{ATLAS:2022fgb} at $\sqrt{s}=5.02$ TeV simultaneously. }
\begin{center}
 % \begin{tabular}{l|c|c|c|c}
  \begin{tabular}{p{1.5cm}<{\centering}|p{2.cm}<{\centering}|p{2.cm}<{\centering}|p{2.cm}<{\centering}}
  \hline
  \multicolumn{4}{c} {(0-20$\%$) 5.02 TeV }   \\
   \hline
            & $\alpha_i$          &    $\beta_i $        &  $\gamma_i$    \\ \hline
    Gluon & 4.60$\pm$2.96   &    2.18$\pm$1.12 & 0.21$\pm$0.12  \\ \hline
    Quark & 4.12$\pm$2.71   &  0.86$\pm$0.38   & 0.24$\pm$0.11  \\ \hline
    $b$-quark &5.32$\pm$2.84   &  0.80$\pm$0.54   & 0.2$\pm$0.17  \\ \hline
  \end{tabular}

  \label{table:parameters_b}
\end{center}
\vspace{-10pt}
\end{table}

\subsection{Parton-mass  dependence of $R_\text{AA}$}
Finally, with the extracted centrality dependent quark and gluon energy loss distributions, we also extract $b$-jet energy loss in the same framework based on  Eqs.~(\ref{EQ_raa}) and (\ref{sigma_AQ}) through fitting to the experimental data of $b$-jet $R_\text{AA}$~\cite{ATLAS:2022fgb}, inclusive jet $R_\text{AA} $ in  0-20$\%$ centrality~\cite{ATLAS:2018gwx} and $\gamma$-tagged jet $R_\text{AA}$ in  0-10$\%$~\cite{ATLAS:2022cim} simultaneously.
 %{\color{blue} $b$-jets $R_\text{AA}$, inclusive jet $R_\text{AA} $ and $\gamma$-tagged jet $R_\text{AA}$ are in different centrality, so we need  centrality dependence of quark and gluon energy lose distributions to fit those $R_\text{AA}$ simultaneously. For example, $\gamma$-tagged jet $R_\text{AA}$ in 0-20$\%$ centrality, which is generated during the fitting,  is changed to $R_\text{AA}$ in 10-30$\%$ to compare with experimental data directly.}.
Considering the recent CMS measurements~\cite{CMS:2016uxf,CMS:2021vui}, as well as an earlier ATLAS measurement~\cite{ATLAS:2012tjt}, where the ratio of inclusive jet $R_\text{AA}$ with jet cone R=0.4 to R=0.2 show no deviation from one at large $p_\text{T}$, and  the limited b-jet data points,  we ignore the jet cone dependence at present and  mainly focus on a qualitative investigation on the parton mass/flavor effects on the b-jet $R_\text{AA}$.


 
The final fitted  nuclear modification factor $R_\text{AA}$ of $b$-jet (lime green band), inclusive jet (magenta band)  and $\gamma$-tagged jet (gray band) as well as the comparison with experimental data~\cite{ATLAS:2018gwx,ATLAS:2022cim,ATLAS:2022fgb} are shown in Fig.~\ref{Raa_bjet}(a). The corresponding bands are results with one sigma deviation from the average fits of $R_\text{AA}$.
Meanwhile, Fig.~\ref{Raa_bjet}(b) shows the extracted nuclear modification factor $R_\text{AA}$ for $b$-quark initiated (green, denoted as ``$R_\text{AA}^{b}$"), light-quark (denoted as ``$R_\text{AA}^{\text{quark}}$") and gluon (denoted as ``$R_\text{AA}^{\text{gluon}}$") initiated $b$-jet in 0-20$\%$ centrality, with the corresponding  parameters for gluon, quark and $b$-quark energy loss distribution summarized in Table.\ref{table:parameters_b}.  
The final extracted light-quark and gluon initiated jet energy loss distributions are consistent with our previous results in the same centrality, while $b$-quark initiated jets is less suppressed compared to light-quark initiated jets due to its large mass in low $p_\text{T}$ region.
 Our result shows
a clear flavor hierarchy of jet energy loss at high energies, $\langle \Delta E_g \rangle > \langle\Delta E_q\rangle > \langle\Delta E_b\rangle$ inside a hot
nuclear matter, consistent with perturbative QCD expectation. 
%\sout{To verify the rationality of the extracted flavor dependence of jet quenching, we compare the results to LBT simulations shown in lines in Fig.~\ref{Raa_bjet}. The agreement between Bayesian analysis and LBT validates the expected flavor hierarchy of jet quenching, i.e. $\langle \Delta E_g \rangle > \langle\Delta E_q\rangle > \langle\Delta E_b\rangle$.}



To explore the underlying $b$-jet suppression mechanism in heavy-ion collisions, we also present in Fig.~\ref{Raa_bjet}(c) the ratio of  $b$-quark initiated jets $R_\text{AA}$ to light quark initialed 
jet $R_\text{AA}$ as $R_\text{AA}^{b}/R_\text{AA}^{\text{quark}}$, and also in Fig.~\ref{Raa_bjet}(d) the ratio of $b$-jet $R_\text{AA}$ (denoted as ``$R_\text{AA}^{b\text{-jet}}$") to inclusive jet $R_\text{AA}$ $R_\text{AA}^{b\text{-jet}}/R_\text{AA}^{\text{jet}}$  extracted  from  global analysis  and the comparison  to the experimental measurements~\cite{ATLAS:2022fgb}. Our numerical results can describe the experimental data within large uncertainties~\cite{ATLAS:2022fgb}. Those ratios are greater than unity and  go down with increasing $p_\text{T}$, indicating that the parton mass effect is reduced with increasing  $p_\text{T}$~\cite{Huang:2013vaa}.
However, the mass effect for $b$-jet could persist to large $p_\text{T}$, even at $p_\text{T}\sim300$ GeV/$c$, and is consistent with the current data and a model based on strong coupling (via the
AdS/CFT correspondence)~\cite{Horowitz:2007su}, in contrast to Ref.~\cite{Huang:2013vaa,Xing:2019xae}  in which mass effects are
expected to be small at  $p_\text{T}>70 $ GeV/$c$. Those disagreements  may be explained  by two reasons. For one thing, due to the subdominant contributions and the limited $b$-jet $R_\text{AA}$ data points with large uncertainties, especially at large $p_\text{T}$, which have weak constraints on $b$-quark initialed jet, $b$-quark initiated jet energy loss distributions is weakly constrained at present.  For another, this may be attributed to the mixture of mass effect and color effect, as we may show below.

%In fact,  \is{(why we don't see the consistency between light quark and b-quark in the large pt region? one should expect the vanish of mass effect in large pt region such as 300 GeV)}
\begin{figure}
  \centering
  \includegraphics[width=0.5\textwidth]{./RAA_bjet_cpmparison1.pdf}   % RAA_bjet_\text{T}est   RAA_bjet_cpmparison
  %\includegraphics[width=0.5\textwidth]{./RAA_bjet_ratio.pdf}
%\includegraphics[scale=0.34]{figures/delta_phi_0.3.png}
  \caption{(Color online) (a): final fitted nuclear modification
factor $R_\text{AA}$ of  $b$-jets (lime green), inclusive jet (magenta)  and $\gamma$-tagged jet (gray) and the comparison with experimental data~\cite{ATLAS:2018gwx,ATLAS:2022cim,ATLAS:2022fgb}. (b): the data-driven extracted $R_\text{AA}$ of gluon (red), light quark (blue), and $b$-quark (green) initiated jets.
  (c): The ratio of $R_\text{AA}^{b}/R_\text{AA}^\text{quark}$ from  global analysis.
  (d):  the ratio of $R_\text{AA}^{b\text{-jet}}/R_\text{AA}^{\text{jet}}$  extracted  from  global analysis and the comparison to the experimental measurements.  The quark mass effect (yellow) and less gluon fraction effect (green)  to the ratio  of $R_\text{AA}$ are also presented.  %{\color{red}\sout{The solid, dotted and dashed lines are from LBT simulations.}} 
  }\label{Raa_bjet}
\end{figure}





%This may be attributed to the jet substructures.  The splitting function of $b$-quark peaks at $z\sim1$ in PYTHIA. Therefore, b-quark initiated jets are harder and lazier than light-quark/gluon initiated jets. Jets with more complex constituents tend to lose more energy compared to those jets with simpler substructures. Thus  $b$-initiated jets will lose less energy compared to quark/gluon jets due to the substructures, which can be verified though the comparison of jet fragmentation function $z=p_\text{T}^{h}/p_\text{T}^j$ of light and $b/c$-jets. \is{this argument is not clear enough.}
Notice that b-jet spectra~\cite{ATLAS:2022fgb} have similar slope as the inclusive jet as shown in the inset of Fig.  1(a),  so the 
 parton mass and  color effects may give the dominated contributions  to the  difference between inclusive jet $R_\text{AA}$ and $b$-jet $R_\text{AA}$.
 For further demonstration of $b$-quark mass effect on the suppression of $b$-jet, we show in Fig.~\ref{Raa_bjet}(d) (yellow band) the ratio of $b$-jet $R_\text{AA}$ to inclusive jet $R_\text{AA}$, assuming $b$-jet has the same fraction of gluon initiated jet as inclusive jet (denoted as ``$f^{b\text{-jet}}=f^{\text{jet}}$").  The difference between this ratio and  $R_\text{AA}^{b\text{-jet}}/R_\text{AA}^{\text{jet}}$
should be attributed to the $b$-quark mass effect.
 One can see that the deviation between $b$-jet and the inclusive jet is moderately reduced with increasing $p_\text{T}$. %Despite $b$-initiated jets is less suppressed compared to light-quark initiated jets,
 The mass effect roughly give considerable contributions to the ratio of $R_\text{AA}^{b\text{-jet}}/R_\text{AA}^{\text{jet}}$ and are expected to be small at $p_\text{T} \sim 300 $ GeV/$c$.




To further illustrate the color-charge effect on the suppression of $b$-jet,  we also calculated the  ratio of $b$-jet $R_\text{AA}$ to inclusive jet $R_\text{AA}$, assuming $b$-quark jet lose the same fraction of energy as light-quark  initiated jet (denoted as ``$R_\text{AA}^{b}=R_\text{AA}^{\text{quark}}$"), as shown by green band in Fig.~\ref{Raa_bjet}(d). The difference between this ratio and  $R_\text{AA}^{b\text{-jet}}/R_\text{AA}^{\text{jet}}$
should be attributed to the different gluon and quark fraction.
 As can be seen, those ratio is significantly enhanced and also show a downward tendency with  increasing $p_\text{T}$, indicating that, less gluon initiated jet contribution also lead to the less suppression of $b$-jet compared to inclusive jet in heavy-ion collisions, especially in low $p_\text{T}$ region.  Therefore, we can see that the color charge effect have greater impacts to the ratio $R_\text{AA}^{b\text{-jet}}/R_\text{AA}^{\text{jet}}$ than parton mass effect in heavy-ion collisions.
 Furthermore, the contribution from gluon initialed jet to inclusive jet production is greater than that to $b$-jet in the $p_\text{T}<300$ GeV/$c$ region  as shown in Fig.~\ref{ref_pp_lbt}. Thus $b$-jet $R_\text{AA}$ is moderately larger than inclusive jet $R_\text{AA}$.





\section{Summary}
\label{summary}
 We have carried out a systematic  investigation of parton color-charge  and parton mass dependence of nuclear modification factor by a systematic study  of  the medium modifications on  three full jet observables: the inclusive  jet, $\gamma$+jet, and $b$-jet, in Pb+Pb collisions relative to that in p+p at the LHC.  Our  results  from MadGraph+PYTHIA  give very nice descriptions of the experimental data for these three jet observables  in p+p.  Then a Bayesian data-driven method is applied to extract the model-independent but
flavor-dependent jet energy loss distributions. Fitting to those experimental data simultaneously, we present the first quantitative extraction of  gluon, light quark and $b$-quark initiated jet energy loss distributions in  heavy-ion collisions.  It is seen that the energy loss of quark-initiated jets shows a weaker centrality dependence and weaker $p_\text{T}$ dependence compared to that of gluon-initiated jets. Our result shows
a clear flavor hierarchy of jet energy loss at high energies, $\langle \Delta E_g \rangle > \langle\Delta E_q\rangle > \langle\Delta E_b\rangle$ inside a hot
nuclear matter, consistent with perturbative QCD expectation. % {\color{red}\sout{and the LBT simulations}}.

Furthermore, we analysed the relative contributions from the
slope of initial spectra, parton color-charge as well as parton
mass dependent jet energy attenuation to the $\gamma$/b-jet suppression in heavy-ion collisions.
 We find that large quark-initiated jet fraction  underlies $\gamma$+jet suppression at large $p_\text{T}$, while the flat spectra give the dominate contribution to $\gamma$+jet suppression at low $p_\text{T}$. We demonstrate that the color charge effect have greater impacts to the ratio $R_\text{AA}^{b\text{-jet}}/R_\text{AA}^{\text{jet}}$ than parton mass effect,  which decrease moderately  at $p_\text{T} \sim 300 $ GeV/$c$.
 %Except for mass effect, smaller fraction of gluon contributions lead to smaller suppression of $b$-jet compared to inclusive jet in low $p_\text{T}$ jet region.
Such a systematic extraction of
jet energy loss distributions can help constrain model
uncertainties  and pave the way to precise predictions of
the properties of the hot QCD medium created in relativistic heavy-ion collisions\footnote{When finalizing this paper, the authors notice a very recent parallel study of extracting the flavor dependence of parton energy loss~\cite{Xing:2023ciw}, but from the nuclear modifications of various hadron species instead of jet observables presented in our work.}.
%%%%%%%%%%%%%???%%%%%%%%%%%%%%%%

Several caveats should be mentioned. First, due to the limited data, we ignore the jet cone dependence~\cite{CMS:2021vui,ATLAS:2012tjt} at present and mainly focus on a qualitative investigation on the effects from the initial spectrum and parton mass/flavor on the jet $R_\text{AA}$. For a more strict study, we should use measurements with the same R in a global analysis. With
the upcoming high precision measurements of b-jet $R_\text{AA}$ at the LHC, one can quantitatively  analyse those mass/ flavor dependent jet quenching. 
%Second, the impact of the isospin and nPDF effects for the three processes are ignored in our analysis.  Even though the isospin effect for inclusive jets is negligible, we should take the impact of the isospin and nPDF effects for the other two processes into consideration. 
Second, the MCMC bands is more restricted than the uncertainty in the experimental data. A more detailed treatment in the Bayesian analysis of the uncertainties is needed in the future. Finally,  the Bayesian analysis here uses specific functional form for the parameterzation, thus introducing long-range correlations in the parameter space which may potentially bias the extracted parameters. A possible solution to tackle such issue is to use information field based approach as presented in Ref.~\cite{Xie:2022ght}.


{\bf Acknowledgments:} This research is supported by National Natural Science Foundation of China with Project Nos. 12035007, 12022512, 12147131, Guangdong Major Project of Basic and Applied Basic Research No. 2020B0301030008. S.Z. is further supported by the MOE Key Laboratory of Quark and Lepton Physics (CCNU) under Project
No. QLPL2021P01.

\vspace*{-.6cm}

\begin{thebibliography}{99}


%%%%%%%%%%%%%%%%   begin jet quenching measurements %%%%%%%%%%




%\cite{Adler:2002xw}
\bibitem{Adler:2002xw}
  C.~Adler {\it et al.},
  %[STAR Collaboration],
  %``Centrality dependence of high p(T) hadron suppression in Au + Au collisions
  %at s(NN)**(1/2) = 130-GeV,''
  Phys.\ Rev.\ Lett.\  {\bf 89}, 202301 (2002).
  %%CITATION = PRLTA,89,202301;%%


%\cite{Adler:2002tq}
\bibitem{Adler:2002tq}
C.~Adler \textit{et al.} [STAR],
%``Disappearance of back-to-back high $p_{T}$ hadron correlations in central Au+Au collisions at $\sqrt{s_{NN}}$ = 200-GeV,''
Phys. Rev. Lett. \textbf{90} (2003), 082302
%doi:10.1103/PhysRevLett.90.082302
%[arXiv:nucl-ex/0210033 [nucl-ex]].
%834 citations counted in INSPIRE as of 01 Mar 2021


%%%%%LHC
  %\cite{Aad:2010bu}
\bibitem{Aad:2010bu}
  G.~Aad {\it et al.}  [ATLAS Collaboration],
  %``Observation of a Centrality-Dependent Dijet Asymmetry in Lead-Lead Collisions at $\sqrt{s_{NN}}=2.77$ TeV with the ATLAS Detector at the LHC,''
  Phys.\ Rev.\ Lett.\  {\bf 105}, 252303 (2010).
%  [arXiv:1011.6182 [hep-ex]].
  %%CITATION = ARXIV:1011.6182;%%
  %311 citations counted in INSPIRE as of 11 Apr 201


%\cite{Chatrchyan:2012gt}
\bibitem{Chatrchyan:2012gt}
  S.~Chatrchyan {\it et al.}  [CMS Collaboration],
  %``Studies of jet quenching using isolated-photon+jet correlations in PbPb and $pp$ collisions at $\sqrt{s_{NN}}=2.76$ TeV,''
  Phys.\ Lett.\ B {\bf 718}, 773 (2013);
%  [arXiv:1205.0206 [nucl-ex]].
  %%CITATION = ARXIV:1205.0206;%%
  %67 citations counted in INSPIRE as of 11 Apr 2014
%  \cite{CMS:2013oua}
%\bibitem{CMS:2013oua}
%  CMS Collaboration [CMS Collaboration],
  %``Study of Isolated photon jet correlation in PbPb and p+p collisions at 2.76TeV  and pPb collisions at 5.02TeV,''
 % CMS-PAS-HIN-13-006.
  %%CITATION = CMS-PAS-HIN-13-006;%%
  %15 citations counted in INSPIRE as of 08 Mar 2018


    %\cite{Aad:2014bxa}
\bibitem{Aad:2014bxa}
  G.~Aad {\it et al.} [ATLAS Collaboration],
  %``Measurements of the Nuclear Modification Factor for Jets in Pb+Pb Collisions at $\sqrt{s_{\mathrm{NN}}}=2.76$ TeV with the ATLAS Detector,''
  Phys.\ Rev.\ Lett.\  {\bf 114}, no. 7, 072302 (2015)
 % doi:10.1103/PhysRevLett.114.072302
 % [arXiv:1411.2357 [hep-ex]].
  %%CITATION = doi:10.1103/PhysRevLett.114.072302;%%
  %118 citations counted in INSPIRE as of 08 Mar 2018

    %\cite{Chatrchyan:2013kwa}
\bibitem{Chatrchyan:2013kwa}
  S.~Chatrchyan {\it et al.} [CMS Collaboration],
  %``Modification of jet shapes in PbPb collisions at $\sqrt {s_{NN}} = 2.76$ TeV,''
  Phys.\ Lett.\ B {\bf 730}, 243 (2014)
 % doi:10.1016/j.physletb.2014.01.042
 % [arXiv:1310.0878 [nucl-ex]].
  %%CITATION = doi:10.1016/j.physletb.2014.01.042;%%
  %131 citations counted in INSPIRE as of 08 Mar 2018

% Wang:1991xy, Wang:1998ww

\bibitem{Wang:1991xy}
  X.~N.~Wang and M.~Gyulassy,
  %``Gluon shadowing and jet quenching in A + A collisions at s**(1/2) =
  %200-GeV,''
  Phys.\ Rev.\ Lett.\  {\bf 68}, 1480 (1992).
  %%CITATION = PRLTA,68,1480;%%

  %\cite{Wang:1998ww}
\bibitem{Wang:1998ww}
  X.~N.~Wang,
  %``Systematic study of high $p_{T}$ hadron spectra in $p p$, $p$ A and A A collisions from SPS to RHIC energies,''
  Phys.\ Rev.\ C {\bf 61}, 064910 (2000)
  %doi:10.1103/PhysRevC.61.064910
  %[nucl-th/9812021].
  %%CITATION = doi:10.1103/PhysRevC.61.064910;%%
  %270 citations counted in INSPIRE as of 08 Mar 2018




  %%%RHIC   Adcox:2001jp,Adler:2002xw, Adler:2002tq



  %\cite{Qin:2007rn}
\bibitem{Qin:2007rn}
  G.~Y.~Qin, J.~Ruppert, C.~Gale, S.~Jeon, G.~D.~Moore and M.~G.~Mustafa,
  %``Radiative and Collisional Jet Energy Loss in the Quark-Gluon Plasma at
  %RHIC,''
  Phys.\ Rev.\ Lett.\  {\bf 100}, 072301 (2008).
%  [arXiv:0710.0605 [hep-ph]].
  %%CITATION = PRLTA,100,072301;%%

  %\cite{Chen:2011vt}
\bibitem{Chen:2011vt}
  X.~-F.~Chen, T.~Hirano, E.~Wang, X.~-N.~Wang and H.~Zhang,
  %``Suppression of high $p_{T}$ hadrons in $Pb+Pb$ Collisions at LHC,''
  Phys.\ Rev.\ C {\bf 84}, 034902 (2011).
%  [arXiv:1102.5614 [nucl-th]].
  %%CITATION = ARXIV:1102.5614;%%
  %33 citations counted in INSPIRE as of 18 Sep 2013

  %\cite{Buzzatti:2011vt}
\bibitem{Buzzatti:2011vt}
  A.~Buzzatti and M.~Gyulassy,
  %CUJET1.0,
  %``Jet Flavor Tomography of Quark Gluon Plasmas at RHIC and LHC,''
  Phys. Rev. Lett. {\bf 108}, 022301 (2012).
%  [arXiv:1106.3061 [hep-ph]].


  %\cite{Majumder:2011uk}
\bibitem{Majumder:2011uk}
  A.~Majumder and C.~Shen,
  %``Suppression of the High $p_\text{T}$ Charged Hadron $R_\text{AA}$ at the LHC,''
  Phys.\ Rev.\ Lett.\  {\bf 109}, 202301 (2012).
%  [arXiv:1103.0809 [hep-ph]].
  %%CITATION = ARXIV:1103.0809;%%
  %20 citations counted in INSPIRE as of 18 Sep 201





%\cite{Aamodt:2010jd}
\bibitem{Aamodt:2010jd}
  K.~Aamodt {\it et al.}  [ALICE Collaboration],
  %``Suppression of Charged Particle Production at Large Transverse Momentum in Central Pb--Pb Collisions at $\sqrt{s_{NN}} = 2.76$ TeV,''
  Phys.\ Lett.\ B {\bf 696}, 30 (2011).
%  [arXiv:1012.1004 [nucl-ex]].
  %%CITATION = ARXIV:1012.1004;%%
  %307 citations counted in INSPIRE as of 11 Apr 2014

  %\cite{CMS:2012aa}
\bibitem{CMS:2012aa}
  S.~Chatrchyan {\it et al.}  [CMS Collaboration],
  %``Study of high-pT charged particle suppression in PbPb compared to $pp$ collisions at $\sqrt{s_{NN}}=2.76$ TeV,''
  Eur.\ Phys.\ J.\ C {\bf 72}, 1945 (2012).
%  [arXiv:1202.2554 [nucl-ex]].
  %%CITATION = ARXIV:1202.2554;%%
  %118 citations counted in INSPIRE as of 11 Apr 2014

%\cite{Qiu:2019sfj}
\bibitem{Qiu:2019sfj}
J.~W.~Qiu, F.~Ringer, N.~Sato and P.~Zurita,
%``Factorization of jet cross sections in heavy-ion collisions,''
Phys. Rev. Lett. \textbf{122}, no.25, 252301 (2019)
%doi:10.1103/PhysRevLett.122.252301
%[arXiv:1903.01993 [hep-ph]].
%32 citations counted in INSPIRE as of 17 Oct 2022

%\cite{Wang:1996yh}
\bibitem{Wang:1996yh}
  X.~N.~Wang, Z.~Huang and I.~Sarcevic,
  %``Jet quenching in the opposite direction of a tagged photon in high-energy heavy ion collisions,''
  Phys.\ Rev.\ Lett.\  {\bf 77}, 231 (1996)
  %doi:10.1103/PhysRevLett.77.231
  %[hep-ph/9605213].
  %%CITATION = doi:10.1103/PhysRevLett.77.231;%%
  %270 citations counted in INSPIRE as of 08 Mar 2018

%\cite{Renk:2006qg}
\bibitem{Renk:2006qg}
  T.~Renk,
  %``Towards jet tomography: gamma-hadron correlations,''
  Phys.\ Rev.\ C {\bf 74}, 034906 (2006).
%  [hep-ph/0607166].
  %%CITATION = HEP-PH/0607166;%%
  %61 citations counted in INSPIRE as of 09 Dec 2013

  %\cite{Zhang:2009rn}
\bibitem{Zhang:2009rn}
  H.~Zhang, J.~F.~Owens, E.~Wang and X.-N. Wang,
  %``Tomography of high-energy nuclear collisions with photon-hadron correlations,''
  Phys.\ Rev.\ Lett.\  {\bf 103}, 032302 (2009).
%  [arXiv:0902.4000 [nucl-th]].

%\cite{Qin:2009bk}
\bibitem{Qin:2009bk}
  G.~Y.~Qin, J.~Ruppert, C.~Gale, S.~Jeon and G.~D.~Moore,
  %``Jet energy loss, photon production, and photon-hadron correlations at
  %RHIC,''
  Phys.\ Rev.\  C {\bf 80}, 054909 (2009).
%  [arXiv:0906.3280 [hep-ph]].
  %%CITATION = PHRVA,C80,054909;%%



%\cite{Adare:2009vd}
\bibitem{Adare:2009vd}
  A.~Adare {\it et al.}  [PHENIX Collaboration],
  %``Photon-Hadron Jet Correlations in p+p and Au+Au Collisions at s**(1/2) = 200-GeV,''
  Phys.\ Rev.\ C {\bf 80}, 024908 (2009).
%  [arXiv:0903.3399 [nucl-ex]].
  %%CITATION = ARXIV:0903.3399;%%
  %55 citations counted in INSPIRE as of 11 Apr 2014

  %\cite{Abelev:2009gu}
\bibitem{Abelev:2009gu}
  B.~I.~Abelev {\it et al.}  [STAR Collaboration],
  %``Studying Parton Energy Loss in Heavy-Ion Collisions via Direct-Photon and Charged-Particle Azimuthal Correlations,''
  Phys.\ Rev.\ C {\bf 82}, 034909 (2010).
%  [arXiv:0912.1871 [nucl-ex]].
  %%CITATION = ARXIV:0912.1871;%%
  %30 citations counted in INSPIRE as of 11 Apr 2014

  %\cite{Chen:2017zte}
\bibitem{Chen:2017zte}
  W.~Chen, S.~Cao, T.~Luo, L.~G.~Pang and X.~N.~Wang,
  %``Effects of jet-induced medium excitation in $\gamma$-hadron correlation in A+A collisions,''
  Phys.\ Lett.\ B {\bf 777}, 86 (2018)
 % doi:10.1016/j.physletb.2017.12.015
  %[arXiv:1704.03648 [nucl-th]].
  %%CITATION = doi:10.1016/j.physletb.2017.12.015;%%
  %9 citations counted in INSPIRE as of 08 Mar 2018


  %\cite{Dai:2012am}
\bibitem{Dai:2012am}
  W.~Dai, I.~Vitev and B.~-W.~Zhang,
  %``Momentum imbalance of isolated photon-tagged jet production at RHIC and LHC,''
  Phys.\ Rev.\ Lett.\  {\bf 110}, 142001 (2013).
%  [arXiv:1207.5177 [hep-ph]].
  %%CITATION = ARXIV:1207.5177;%%
  %18 citations counted in INSPIRE as of 11 Apr 2014

%\cite{Neufeld:2010fj}
\bibitem{Neufeld:2010fj}
R.~B.~Neufeld, I.~Vitev and B.~W.~Zhang,
%``The Physics of $Z^0/\gamma^*$-tagged jets at the LHC,''
Phys. Rev. C \textbf{83}, 034902 (2011)
%doi:10.1103/PhysRevC.83.034902
%[arXiv:1006.2389 [hep-ph]].
%74 citations counted in INSPIRE as of 11 May 2021



  %\cite{Kang:2017xnc}
\bibitem{Kang:2017xnc}
  Z.~B.~Kang, I.~Vitev and H.~Xing,
  %``Vector-boson-tagged jet production in heavy ion collisions at energies available at the CERN Large Hadron Collider,''
  Phys.\ Rev.\ C {\bf 96}, no. 1, 014912 (2017)
  %doi:10.1103/PhysRevC.96.014912
  %[arXiv:1702.07276 [hep-ph]].
  %%CITATION = doi:10.1103/PhysRevC.96.014912;%%
  %6 citations counted in INSPIRE as of 21 Mar 2018



%\cite{Luo:2018pto}
\bibitem{Luo:2018pto}
T.~Luo, S.~Cao, Y.~He and X.~N.~Wang,
%``Multiple jets and $\gamma$+jet correlation in high-energy heavy-ion collisions,''
Phys. Lett. B \textbf{782} (2018), 707-716
%doi:10.1016/j.physletb.2018.06.025
%[arXiv:1803.06785 [hep-ph]].
%40 citations counted in INSPIRE as of 02 Mar 2021


%\cite{Zhang:2018urd}
\bibitem{Zhang:2018urd}
S.~L.~Zhang, T.~Luo, X.~N.~Wang and B.~W.~Zhang,
%``Z+jet correlation with NLO-matched parton-shower and jet-medium interaction in high-energy nuclear collisions,''
Phys. Rev. C \textbf{98} (2018), 021901
%doi:10.1103/PhysRevC.98.021901
%[arXiv:1804.11041 [nucl-th]].
%25 citations counted in INSPIRE as of 02 Mar 2021

%\cite{Zhang:2018kjl,Zhang:2021oki}
\bibitem{Zhang:2021oki}
S.~L.~Zhang, X.~N.~Wang and B.~W.~Zhang,
%``Quenching of jets tagged with W bosons in high-energy nuclear collisions,''
Phys. Rev. C \textbf{105}, no.5, 5 (2022)
%doi:10.1103/PhysRevC.105.054902
%[arXiv:2103.07836 [hep-ph]].
%5 citations counted in INSPIRE as of 26 Mar 2023




%\cite{Sirunyan:2017jic}
\bibitem{Sirunyan:2017jic}
A.~M.~Sirunyan \textit{et al.} [CMS],
%``Study of Jet Quenching with $Z+\text{jet}$ Correlations in Pb-Pb and $pp$ Collisions at ${\sqrt{s}}_{NN}=5.02\text{ }\text{ }\mathrm{TeV}$,''
Phys. Rev. Lett. \textbf{119}, no.8, 082301 (2017)
%doi:10.1103/PhysRevLett.119.082301
%[arXiv:1702.01060 [nucl-ex]].
%65 citations counted in INSPIRE as of 11 May 2021

%\cite{Yang:2021qtl}
\bibitem{Yang:2021qtl}
Z.~Yang, W.~Chen, Y.~He, W.~Ke, L.~Pang and X.~N.~Wang,
%``Search for the Elusive Jet-Induced Diffusion Wake in $Z/\gamma$-Jets with 2D Jet Tomography in High-Energy Heavy-Ion Collisions,''
Phys. Rev. Lett. \textbf{127} (2021) no.8, 082301
%doi:10.1103/PhysRevLett.127.082301
%[arXiv:2101.05422 [hep-ph]].
%6 citations counted in INSPIRE as of 20 Nov 2021

%\cite{Zhang:2022bhq}
\bibitem{Zhang:2022bhq}
S.~L.~Zhang, H.~Xing and B.~W.~Zhang,
%``Hadron productions and jet substructures associated with Z$^{0}$/\ensuremath{\gamma} in Pb+Pb collisions at the LHC,''
Sci. China Phys. Mech. Astron. \textbf{66}, no.12, 121012 (2023)
%doi:10.1007/s11433-023-2251-4
%[arXiv:2209.15336 [hep-ph]].
%3 citations counted in INSPIRE as of 23 Feb 2024

  %\cite{Zhang:2007ja}
\bibitem{Zhang:2007ja}
  H.~Zhang, J.~F.~Owens, E.~Wang and X.~N.~Wang,
  %``Dihadron Tomography of High-Energy Nuclear Collisions in NLO pQCD,''
  Phys.\ Rev.\ Lett.\  {\bf 98}, 212301 (2007).
%  [arXiv:nucl-th/0701045];
  %%CITATION = PRLTA,98,212301;%%




\bibitem{stardihadron} C.~Adler {\it et al.},
  %``Disappearance of back-to-back high p(T) hadron correlations in central Au +
  %Au collisions at s(NN)**(1/2) = 200-GeV,''
  Phys.\ Rev.\ Lett.\  {\bf 90}, 082302 (2003).
  %%CITATION = NUCL-EX 0210033;%%


  %\cite{Ayala:2009fe,Ayala:2011ii}
\bibitem{Ayala:2009fe}
  A.~Ayala, J.~Jalilian-Marian, J.~Magnin, A.~Ortiz, G.~Paic and M.~E.~Tejeda-Yeomans,
  %``Three and two-hadron correlations in s(NN)**(1/2) = 200-GeV proton-proton and nucleus-nucleus collisions,''
  Phys.\ Rev.\ Lett.\  {\bf 104}, 042301 (2010)
  %doi:10.1103/PhysRevLett.104.042301
  %[arXiv:0911.4738 [nucl-th]].
  %%CITATION = doi:10.1103/PhysRevLett.104.042301;%%
  %16 citations counted in INSPIRE as of 12 Mar 2018

  %\cite{Ayala:2011ii}
\bibitem{Ayala:2011ii}
  A.~Ayala, J.~Jalilian-Marian, A.~Ortiz, G.~Paic, J.~Magnin and M.~E.~Tejeda-Yeomans,
  %``Three-hadron angular correlations in high-energy proton-proton and nucleus-nucleus collisions from perturbative QCD,''
  Phys.\ Rev.\ C {\bf 84}, 024915 (2011)
  %doi:10.1103/PhysRevC.84.024915
  %[arXiv:1107.0340 [hep-ph]].
  %%CITATION = doi:10.1103/PhysRevC.84.024915;%%
  %8 citations counted in INSPIRE as of 12 Mar 2018
  %\cite{Ayala:2015jaa}




  %\cite{Burke:2013yra}
%\bibitem{Burke:2013yra}
%  K.~M.~Burke {\it et al.} [JET Collaboration],
  %``Extracting the jet transport coefficient from jet quenching in high-energy heavy-ion collisions,''
 % Phys.\ Rev.\ C {\bf 90}, no. 1, 014909 (2014)
  %doi:10.1103/PhysRevC.90.014909
  %[arXiv:1312.5003 [nucl-th]].
  %%CITATION = doi:10.1103/PhysRevC.90.014909;%%
  %167 citations counted in INSPIRE as of 08 Mar 2018

%\cite{Zhang:2021sua}
\bibitem{Zhang:2021sua}
S.~L.~Zhang, M.~Q.~Yang and B.~W.~Zhang,
%``Parton splitting scales of reclustered large-radius jets in high-energy nuclear collisions,''
Eur. Phys. J. C \textbf{82}, no.5, 414 (2022)
%doi:10.1140/epjc/s10052-022-10340-x
%[arXiv:2105.04955 [hep-ph]].
%3 citations counted in INSPIRE as of 04 Oct 2022

%\cite{Chang:2019sae}
\bibitem{Chang:2019sae}
N.~B.~Chang, Y.~Tachibana and G.~Y.~Qin,
%``Nuclear modification of jet shape for inclusive jets and $\gamma$+jets at the LHC energies,''
Phys. Lett. B \textbf{801} (2020), 135181
%doi:10.1016/j.physletb.2019.135181
%[arXiv:1906.09562 [nucl-th]].
%21 citations counted in INSPIRE as of 20 Nov 2021
%Chatrchyan:2013kwa,Chatrchyan:2014ava,Aad:2014wha,Chatrchyan:2012gw

%\cite{KunnawalkamElayavalli:2017hxo}
\bibitem{KunnawalkamElayavalli:2017hxo}
R.~Kunnawalkam Elayavalli and K.~C.~Zapp,
%``Medium response in JEWEL and its impact on jet shape observables in heavy ion collisions,''
JHEP \textbf{07} (2017), 141
%doi:10.1007/JHEP07(2017)141
%[arXiv:1707.01539 [hep-ph]].
%90 citations counted in INSPIRE as of 20 Nov 2021

%
%\cite{Kang:2017mda}
\bibitem{Kang:2017mda}
Z.~B.~Kang, F.~Ringer and W.~J.~Waalewijn,
%``The Energy Distribution of Subjets and the Jet Shape,''
JHEP \textbf{07} (2017), 064
%doi:10.1007/JHEP07(2017)064
%[arXiv:1705.05375 [hep-ph]].
%43 citations counted in INSPIRE as of 20 Nov 2021

%\cite{Ma:2013uqa}
\bibitem{Ma:2013uqa}
G.~L.~Ma,
%``Medium modifications of jet shapes in Pb+Pb collisions at $\sqrt{s_{_{\rm NN}}}$ = 2.76 TeV within a multiphase transport model,''
Phys. Rev. C \textbf{89} (2014) no.2, 024902
%doi:10.1103/PhysRevC.89.024902
%[arXiv:1309.5555 [nucl-th]].
%21 citations counted in INSPIRE as of 20 Nov 2021


%\cite{Chatrchyan:2014ava,Aaboud:2017bzv}
\bibitem{Chatrchyan:2014ava}
  S.~Chatrchyan {\it et al.} [CMS Collaboration],
  %``Measurement of jet fragmentation in PbPb and p+p collisions at $\sqrt{s_{NN}}=2.76$ TeV,''
  Phys.\ Rev.\ C {\bf 90}, no. 2, 024908 (2014)
 % doi:10.1103/PhysRevC.90.024908
  %[arXiv:1406.0932 [nucl-ex]].
  %%CITATION = doi:10.1103/PhysRevC.90.024908;%%
  %123 citations counted in INSPIRE as of 08 Mar 2018


%\cite{ATLAS:2019dsv}
\bibitem{ATLAS:2019dsv}
M.~Aaboud \textit{et al.} [ATLAS],
%``Comparison of Fragmentation Functions for Jets Dominated by Light Quarks and Gluons from $pp$ and Pb+Pb Collisions in ATLAS,''
Phys. Rev. Lett. \textbf{123}, no.4, 042001 (2019)
%doi:10.1103/PhysRevLett.123.042001
%[arXiv:1902.10007 [nucl-ex]].
%45 citations counted in INSPIRE as of 04 Oct 2022

  %\cite{Aaboud:2017bzv}
\bibitem{Aaboud:2017bzv}
  M.~Aaboud {\it et al.} [ATLAS Collaboration],
  %``Measurement of jet fragmentation in Pb+Pb and $pp$ collisions at $\sqrt{{s_\mathrm{NN}}} = 2.76$ TeV with the ATLAS detector at the LHC,''
  Eur.\ Phys.\ J.\ C {\bf 77} (2017) no.6,  379
 % doi:10.1140/epjc/s10052-017-4915-5
 % [arXiv:1702.00674 [hep-ex]].
  %%CITATION = doi:10.1140/epjc/s10052-017-4915-5;%%
  %15 citations counted in INSPIRE as of 08 Mar 2018
%%%  Dai:2017piq,Dai:2017tuy,Ma:2018swx, Xie:2019oxg


%\cite{Vitev:2008rz}
\bibitem{Vitev:2008rz}
I.~Vitev, S.~Wicks and B.~W.~Zhang,
%``A Theory of jet shapes and cross sections: From hadrons to nuclei,''
JHEP \textbf{11} (2008), 093
%doi:10.1088/1126-6708/2008/11/093
%[arXiv:0810.2807 [hep-ph]].
%150 citations counted in INSPIRE as of 20 Nov 2021




%\cite{STAR:2003wqp}
\bibitem{STAR:2003wqp}
J.~Adams \textit{et al.} [STAR],
%``Particle type dependence of azimuthal anisotropy and nuclear modification of particle production in Au + Au collisions at s(NN)**(1/2) = 200-GeV,''
Phys. Rev. Lett. \textbf{92}, 052302 (2004)
%doi:10.1103/PhysRevLett.92.052302
%[arXiv:nucl-ex/0306007 [nucl-ex]].
%694 citations counted in INSPIRE as of 04 Oct 2022

%\cite{He:2022evt}
\bibitem{He:2022evt}
Y.~He, W.~Chen, T.~Luo, S.~Cao, L.~G.~Pang and X.~N.~Wang,
%``Event-by-event jet anisotropy and hard-soft tomography of the quark-gluon plasma,''
Phys. Rev. C \textbf{106}, no.4, 044904 (2022)
%doi:10.1103/PhysRevC.106.044904
%[arXiv:2201.08408 [hep-ph]].
%18 citations counted in INSPIRE as of 23 Feb 2024

%\cite{CMS:2017xgk}
\bibitem{CMS:2017xgk}
A.~M.~Sirunyan \textit{et al.} [CMS],
%``Azimuthal anisotropy of charged particles with transverse momentum up to 100 GeV/ c in PbPb collisions at $\sqrt {s}_{{NN}}$=5.02 TeV,''
Phys. Lett. B \textbf{776}, 195-216 (2018)
%doi:10.1016/j.physletb.2017.11.041
%[arXiv:1702.00630 [hep-ex]].
%85 citations counted in INSPIRE as of 04 Oct 2022

%\cite{ATLAS:2018ezv}
\bibitem{ATLAS:2018ezv}
M.~Aaboud \textit{et al.} [ATLAS],
%``Measurement of the azimuthal anisotropy of charged particles produced in $\sqrt{s_{_\text {NN}}}$ = 5.02 TeV Pb+Pb collisions with the ATLAS detector,''
Eur. Phys. J. C \textbf{78}, no.12, 997 (2018)
%doi:10.1140/epjc/s10052-018-6468-7
%[arXiv:1808.03951 [nucl-ex]].
%72 citations counted in INSPIRE as of 04 Oct 2022

%\cite{Qin:2015srf}
\bibitem{Qin:2015srf}
G.~Y.~Qin and X.~N.~Wang,
%``Jet quenching in high-energy heavy-ion collisions,''
Int. J. Mod. Phys. E \textbf{24}, no.11, 1530014 (2015)
%doi:10.1142/S0218301315300143
%[arXiv:1511.00790 [hep-ph]].
%253 citations counted in INSPIRE as of 04 Oct 2022

%\cite{CMS:2021vui}
\bibitem{CMS:2021vui}
A.~M.~Sirunyan \textit{et al.} [CMS],
%``First measurement of large area jet transverse momentum spectra in heavy-ion collisions,''
JHEP \textbf{05}, 284 (2021)
%doi:10.1007/JHEP05(2021)284
%[arXiv:2102.13080 [hep-ex]].
%30 citations counted in INSPIRE as of 11 Oct 2022

%\cite{Apolinario:2022vzg}
\bibitem{Apolinario:2022vzg}
L.~Apolin\'ario, Y.~J.~Lee and M.~Winn,
%``Heavy quarks and jets as probes of the QGP,''
Prog. Part. Nucl. Phys. \textbf{127}, 103990 (2022)
%doi:10.1016/j.ppnp.2022.103990
%[arXiv:2203.16352 [hep-ph]].
%66 citations counted in INSPIRE as of 23 Feb 2024

\bibitem{Dokshitzer:2001zm}
Yuri~L. Dokshitzer and D.~E. Kharzeev.
%\newblock {Heavy quark colorimetry of QCD matter}.
Phys. Lett. B, 519:199-206, 2001.

%\cite{Zhang:2003wk}
\bibitem{Zhang:2003wk}
B.~W.~Zhang, E.~Wang and X.~N.~Wang,
%``Heavy quark energy loss in nuclear medium,''
Phys. Rev. Lett. \textbf{93}, 072301 (2004)
%doi:10.1103/PhysRevLett.93.072301
%[arXiv:nucl-th/0309040 [nucl-th]].

%\cite{Djordjevic:2003qk}
\bibitem{Djordjevic:2003qk}
M.~Djordjevic and M.~Gyulassy,
%``Where is the charm quark energy loss at RHIC?,''
Phys. Lett. B \textbf{560}, 37-43 (2003)
%doi:10.1016/S0370-2693(03)00327-7
%[arXiv:nucl-th/0302069 [nucl-th]].

%\cite{Armesto:2003jh}
\bibitem{Armesto:2003jh}
N.~Armesto, C.~A.~Salgado and U.~A.~Wiedemann,
%``Medium induced gluon radiation off massive quarks fills the dead cone,''
Phys. Rev. D \textbf{69}, 114003 (2004).
%doi:10.1103/PhysRevD.69.114003
%[arXiv:hep-ph/0312106 [hep-ph]].

%\cite{JET:2013cls}
%\bibitem{JET:2013cls}
%K.~M.~Burke \textit{et al.} [JET],
%``Extracting the jet transport coefficient from jet quenching in high-energy heavy-ion collisions,''
%Phys. Rev. C \textbf{90}, no.1, 014909 (2014)
%doi:10.1103/PhysRevC.90.014909
%[arXiv:1312.5003 [nucl-th]].
%404 citations counted in INSPIRE as of 04 Oct 2022























%\cite{JETSCAPE:2021ehl}
%\bibitem{JETSCAPE:2021ehl}
%S.~Cao \textit{et al.} [JETSCAPE],
%``Determining the jet transport coefficient q̂ from inclusive hadron suppression measurements using Bayesian parameter estimation,''
%Phys. Rev. C \textbf{104} (2021) no.2, 024905
%doi:10.1103/PhysRevC.104.024905
%[arXiv:2102.11337 [nucl-th]].
%16 citations counted in INSPIRE as of 20 Nov 2021


%%\cite{JETSCAPE:2020shq}
%\bibitem{JETSCAPE:2020shq}
%D.~Everett \textit{et al.} [JETSCAPE],
%``Phenomenological constraints on the transport properties of QCD matter with data-driven model averaging,''
%Phys. Rev. Lett. \textbf{126} (2021) no.24, 242301
%doi:10.1103/PhysRevLett.126.242301
%[arXiv:2010.03928 [hep-ph]].
%36 citations counted in INSPIRE as of 20 Nov 2021
%%%%%%%%jet energy loss


%\cite{Gras:2017jty}
\bibitem{Gras:2017jty}
P.~Gras, S.~H\"oche, D.~Kar, A.~Larkoski, L.~L\"onnblad, S.~Pl\"atzer, A.~Si\'odmok, P.~Skands, G.~Soyez and J.~Thaler,
%``Systematics of quark/gluon tagging,''
JHEP \textbf{07}, 091 (2017)
%doi:10.1007/JHEP07(2017)091
%[arXiv:1704.03878 [hep-ph]].
%112 citations counted in INSPIRE as of 04 Oct 2022

%\cite{Frye:2017yrw}
\bibitem{Frye:2017yrw}
C.~Frye, A.~J.~Larkoski, J.~Thaler and K.~Zhou,
%``Casimir Meets Poisson: Improved Quark/Gluon Discrimination with Counting Observables,''
JHEP \textbf{09}, 083 (2017)
%doi:10.1007/JHEP09(2017)083
%[arXiv:1704.06266 [hep-ph]].
%74 citations counted in INSPIRE as of 04 Oct 2022

%\cite{Chien:2018dfn}
\bibitem{Chien:2018dfn}
Y.~T.~Chien and R.~Kunnawalkam Elayavalli,
%``Probing heavy ion collisions using quark and gluon jet substructure,''
[arXiv:1803.03589 [hep-ph]].
%43 citations counted in INSPIRE as of 04 Oct 2022
































  %\cite{Wang:2016fds}
%\bibitem{Wang:2016fds}
%  X.~N.~Wang, S.~Y.~Wei and H.~Z.~Zhang,
  %``Effect of medium recoil and $p_\text{T}$ broadening on single inclusive jet suppression in high-energy heavy-ion collisions in the high-twist approach,''
 % Phys.\ Rev.\ C {\bf 96}, no. 3, 034903 (2017)
 % doi:10.1103/PhysRevC.96.034903
 % [arXiv:1611.07211 [hep-ph]].
  %%CITATION = doi:10.1103/PhysRevC.96.034903;%%
  %8 citations counted in INSPIRE as of 08 Mar 2018

%\cite{Chien:2015hda}
%\bibitem{Chien:2015hda}
%  Y.~T.~Chien and I.~Vitev,
  %``Towards the understanding of jet shapes and cross sections in heavy ion collisions using soft-collinear effective theory,''
%  JHEP {\bf 1605}, 023 (2016)
 % doi:10.1007/JHEP05(2016)023
  %[arXiv:1509.07257 [hep-ph]].
  %%CITATION = doi:10.1007/JHEP05(2016)023;%%
  %44 citations counted in INSPIRE as of 21 Mar 2018

  %\cite{Casalderrey-Solana:2014bpa}
%\bibitem{Casalderrey-Solana:2014bpa}
%  J.~Casalderrey-Solana, D.~C.~Gulhan, J.~G.~Milhano, D.~Pablos and K.~Rajagopal,
  %``A Hybrid Strong/Weak Coupling Approach to Jet Quenching,''
%  JHEP {\bf 1410}, 019 (2014)
%  Erratum: [JHEP {\bf 1509}, 175 (2015)]
 % doi:10.1007/JHEP09(2015)175, 10.1007/JHEP10(2014)019
  %[arXiv:1405.3864 [hep-ph]].
  %%CITATION = doi:10.1007/JHEP09(2015)175, 10.1007/JHEP10(2014)019;%%


  %\cite{Tachibana:2017syd}
%\bibitem{Tachibana:2017syd}
%  Y.~Tachibana, N.~B.~Chang and G.~Y.~Qin,
  %``Full jet in quark-gluon plasma with hydrodynamic medium response,''
%  Phys.\ Rev.\ C {\bf 95}, no. 4, 044909 (2017)
 % doi:10.1103/PhysRevC.95.044909
  %[arXiv:1701.07951 [nucl-th]].
  %%CITATION = doi:10.1103/PhysRevC.95.044909;%%
  %19 citations counted in INSPIRE as of 08 Mar 2018


%\cite{Chen:2019gqo}
\bibitem{Chen:2019gqo}
S.~Y.~Chen, B.~W.~Zhang and E.~K.~Wang,
%``Jet charge in high energy nuclear collisions,''
Chin. Phys. C \textbf{44}, no.2, 024103 (2020)
%doi:10.1088/1674-1137/44/2/024103
%[arXiv:1908.01518 [nucl-th]].
%12 citations counted in INSPIRE as of 11 May 2021

%\cite{CMS:2020plq}
\bibitem{CMS:2020plq}
A.~M.~Sirunyan \textit{et al.} [CMS],
%``Measurement of quark- and gluon-like jet fractions using jet charge in PbPb and pp collisions at 5.02 TeV,''
JHEP \textbf{07}, 115 (2020)
%doi:10.1007/JHEP07(2020)115
%[arXiv:2004.00602 [hep-ex]].
%25 citations counted in INSPIRE as of 15 Feb 2023

%\cite{Li:2019dre}
\bibitem{Li:2019dre}
H.~T.~Li and I.~Vitev,
%``Jet charge modification in dense QCD matter,''
Phys. Rev. D \textbf{101}, 076020 (2020)
%doi:10.1103/PhysRevD.101.076020
%[arXiv:1908.06979 [hep-ph]].
%30 citations counted in INSPIRE as of 15 Feb 2023

%\cite{Yan:2020zrz}
\bibitem{Yan:2020zrz}
J.~Yan, S.~Y.~Chen, W.~Dai, B.~W.~Zhangy and E.~Wang,
%``Medium modifications of girth distributions for inclusive jets and $Z^0+{\rm jet}$ in relativistic heavy-ion collisions at the LHC,''
Chin. Phys. C \textbf{45}, no.2, 024102 (2021)
%doi:10.1088/1674-1137/abca2b
%[arXiv:2005.01093 [hep-ph]].
%4 citations counted in INSPIRE as of 11 May 2021

%\cite{He:2020iow}
%\bibitem{He:2020iow}
%Y.~He, L.~G.~Pang and X.~N.~Wang,
%``Gradient Tomography of Jet Quenching in Heavy-Ion Collisions,''
%Phys. Rev. Lett. \textbf{125}, no.12, 122301 (2020)
%doi:10.1103/PhysRevLett.125.122301
%[arXiv:2001.08273 [hep-ph]].
%4 citations counted in INSPIRE as of 11 May 2021

%\cite{CMS:2017eqd}
\bibitem{CMS:2017eqd}
A.~M.~Sirunyan \textit{et al.} [CMS],
%``Study of Jet Quenching with $Z+\text{jet}$ Correlations in Pb-Pb and $pp$ Collisions at ${\sqrt{s}}_{NN}=5.02\text{ }\text{ }\mathrm{TeV}$,''
Phys. Rev. Lett. \textbf{119}, no.8, 082301 (2017)
%doi:10.1103/PhysRevLett.119.082301
%[arXiv:1702.01060 [nucl-ex]].
%90 citations counted in INSPIRE as of 15 Feb 2023








%\cite{Aaboud:2019oac}
\bibitem{Aaboud:2019oac}
M.~Aaboud \textit{et al.} [ATLAS],
%``Comparison of Fragmentation Functions for Jets Dominated by Light Quarks and Gluons from $pp$ and Pb+Pb Collisions in ATLAS,''
Phys. Rev. Lett. \textbf{123} (2019) no.4, 042001
%doi:10.1103/PhysRevLett.123.042001
%[arXiv:1902.10007 [nucl-ex]].
%24 citations counted in INSPIRE as of 01 Mar 2021



%\cite{ATLAS:2022cim}
\bibitem{ATLAS:2022cim}
G.~Aad \textit{et al.} [ATLAS],
%``Comparison of inclusive and photon-tagged jet suppression in 5.02 TeV Pb+Pb collisions with ATLAS,''
Phys. Lett. B \textbf{846}, 138154 (2023)
%doi:10.1016/j.physletb.2023.138154
%[arXiv:2303.10090 [nucl-ex]].
%10 citations counted in INSPIRE as of 23 Feb 2024



%\cite{ATLAS:2022fgb}
\bibitem{ATLAS:2022fgb}
%\cite{ATLAS:2022agz}
G.~Aad \textit{et al.} [ATLAS],
%``Measurement of the nuclear modification factor of $b$-jets in 5.02~TeV Pb+Pb collisions with the ATLAS detector,''
Eur. Phys. J. C \textbf{83} (2023) no.5, 438
%doi:10.1140/epjc/s10052-023-11427-9
%[arXiv:2204.13530 [nucl-ex]].
%16 citations counted in INSPIRE as of 20 Sep 2023


%\cite{ATLAS:2018gwx}
\bibitem{ATLAS:2018gwx}
M.~Aaboud \textit{et al.} [ATLAS],
%``Measurement of the nuclear modification factor for inclusive jets in Pb+Pb collisions at $\sqrt{s_\mathrm{NN}}=5.02$ TeV with the ATLAS detector,''
Phys. Lett. B \textbf{790}, 108-128 (2019)
%doi:10.1016/j.physletb.2018.10.076
%[arXiv:1805.05635 [nucl-ex]].
%139 citations counted in INSPIRE as of 07 Jul 2022








%\cite{He:2018xjv}
\bibitem{He:2018xjv}
Y.~He, S.~Cao, W.~Chen, T.~Luo, L.~G.~Pang and X.~N.~Wang,
%``Interplaying mechanisms behind single inclusive jet suppression in heavy-ion collisions,''
Phys. Rev. C \textbf{99}, no.5, 054911 (2019)
%doi:10.1103/PhysRevC.99.054911
%[arXiv:1809.02525 [nucl-th]].
%54 citations counted in INSPIRE as of 07 Jul 2022

%\cite{Zhang:2022rby}
\bibitem{Zhang:2022rby}
S.~L.~Zhang, J.~Liao, G.~Y.~Qin, E.~Wang and H.~Xing,
%``Unraveling gluon jet quenching through J/\ensuremath{\psi} production in heavy-ion collisions,''
Sci. Bull. \textbf{68}, 2003-2009 (2023)
%doi:10.1016/j.scib.2023.07.029
%[arXiv:2208.08323 [hep-ph]].
%8 citations counted in INSPIRE as of 23 Feb 2024


%\cite{Alwall:2014hca}
\bibitem{Alwall:2014hca}
J.~Alwall, R.~Frederix, S.~Frixione, V.~Hirschi, F.~Maltoni, O.~Mattelaer, H.~S.~Shao, T.~Stelzer, P.~Torrielli and M.~Zaro,
%``The automated computation of tree-level and next-to-leading order differential cross sections, and their matching to parton shower simulations,''
JHEP \textbf{07}, 079 (2014)
%doi:10.1007/JHEP07(2014)079
%[arXiv:1405.0301 [hep-ph]].
%6778 citations counted in INSPIRE as of 07 Jul 2022


%\cite{He:2018gks}
\bibitem{He:2018gks}
Y.~He, L.~G.~Pang and X.~N.~Wang,
%``Bayesian extraction of jet energy loss distributions in heavy-ion collisions,''
Phys. Rev. Lett. \textbf{122}, no.25, 252302 (2019)
%doi:10.1103/PhysRevLett.122.252302
%[arXiv:1808.05310 [hep-ph]].
%18 citations counted in INSPIRE as of 11 May 2021


%\cite{He:2015pra}
%\bibitem{He:2015pra}
%  Y.~He, T.~Luo, X.~N.~Wang and Y.~Zhu,
  %``Linear Boltzmann Transport for Jet Propagation in the Quark-Gluon Plasma: Elastic Processes and Medium Recoil,''
%  Phys.\ Rev.\ C {\bf 91}, 054908 (2015)
%  Erratum: [Phys.\ Rev.\ C {\bf 97}, no. 1, 019902 (2018)]
 % doi:10.1103/PhysRevC.97.019902, 10.1103/PhysRevC.91.054908
  %[arXiv:1503.03313 [nucl-th]].
  %%CITATION = doi:10.1103/PhysRevC.97.019902, 10.1103/PhysRevC.91.054908;%%
  %41 citations counted in INSPIRE as of 08 Mar 2018

  %\cite{Cao:2016gvr}
%\bibitem{Cao:2016gvr}
%  S.~Cao, T.~Luo, G.~Y.~Qin and X.~N.~Wang,
  %``Linearized Boltzmann transport model for jet propagation in the quark-gluon plasma: Heavy quark evolution,''
 % Phys.\ Rev.\ C {\bf 94}, no. 1, 014909 (2016)
 % doi:10.1103/PhysRevC.94.014909
  %[arXiv:1605.06447 [nucl-th]].
  %%CITATION = doi:10.1103/PhysRevC.94.014909;%%
  %37 citations counted in INSPIRE as of 08 Mar 2018

  %\cite{Cao:2017hhk}
%\bibitem{Cao:2017hhk}
%  S.~Cao, T.~Luo, G.~Y.~Qin and X.~N.~Wang,
  %``Heavy and light flavor jet quenching at RHIC and LHC energies,''
%  Phys.\ Lett.\ B {\bf 777}, 255 (2018)
 % doi:10.1016/j.physletb.2017.12.023
 % [arXiv:1703.00822 [nucl-th]].
  %%CITATION = doi:10.1016/j.physletb.2017.12.023;%%
  %10 citations counted in INSPIRE as of 08 Mar 2018






%\cite{Xu:2017obm}
\bibitem{Xu:2017obm}
Y.~Xu, J.~E.~Bernhard, S.~A.~Bass, M.~Nahrgang and S.~Cao,
%``Data-driven analysis for the temperature and momentum dependence of the heavy-quark diffusion coefficient in relativistic heavy-ion collisions,''
Phys. Rev. C \textbf{97}, no.1, 014907 (2018)
%doi:10.1103/PhysRevC.97.014907
%[arXiv:1710.00807 [nucl-th]].
%93 citations counted in INSPIRE as of 15 Mar 2023

%\cite{JETSCAPE:2021ehl}
\bibitem{JETSCAPE:2021ehl}
S.~Cao \textit{et al.} [JETSCAPE],
%``Determining the jet transport coefficient q̂ from inclusive hadron suppression measurements using Bayesian parameter estimation,''
Phys. Rev. C \textbf{104}, no.2, 024905 (2021)
%doi:10.1103/PhysRevC.104.024905
%[arXiv:2102.11337 [nucl-th]].
%118 citations counted in INSPIRE as of 18 Feb 2024

%\cite{Xie:2022ght}%\cite{C. Andrieu}
%\cite{Xie:2022ght}
\bibitem{Xie:2022ght}
M.~Xie, W.~Ke, H.~Zhang and X.~N.~Wang,
%``Information-field-based global Bayesian inference of the jet transport coefficient,''
Phys. Rev. C \textbf{108} (2023) no.1, L011901
%doi:10.1103/PhysRevC.108.L011901
%[arXiv:2206.01340 [hep-ph]].
%20 citations counted in INSPIRE as of 20 Sep 2023


%\cite{C. Andrieu}
\bibitem{Andrieu}
C. Andrieu, N. de Freitas, A. Doucet, M.I. Jordan
Mach.Learn. 50 (2003) 5. 
%DOI: 10.1023/A:1020281327116




%\cite{Banfi:2007gu}
\bibitem{Banfi:2007gu}
A.~Banfi, G.~P.~Salam and G.~Zanderighi,
%``Accurate QCD predictions for heavy-quark jets at the Tevatron and LHC,''
JHEP \textbf{07}, 026 (2007)
%doi:10.1088/1126-6708/2007/07/026
%[arXiv:0704.2999 [hep-ph]].
%79 citations counted in INSPIRE as of 06 Sep 2023


  %\cite{Cacciari:2008gp}
\bibitem{Cacciari:2008gp}
M.~Cacciari, G.~P.~Salam and G.~Soyez,
%``The anti-$k_\text{T}$ jet clustering algorithm,''
JHEP \textbf{04} (2008), 063
%doi:10.1088/1126-6708/2008/04/063
%[arXiv:0802.1189 [hep-ph]].
%7690 citations counted in INSPIRE as of 02 Mar 2021



  %\cite{Cacciari:2011ma}
\bibitem{Cacciari:2011ma}
  M.~Cacciari, G.~P.~Salam and G.~Soyez,
  %``FastJet User Manual,''
  Eur.\ Phys.\ J.\ C {\bf 72}, 1896 (2012).
%  [arXiv:1111.6097 [hep-ph]].
  %%CITATION = ARXIV:1111.6097;%%


%\cite{Dai:2022sjk}
\bibitem{Dai:2022sjk}
W.~Dai, M.~Z.~Li, B.~W.~Zhang and E.~Wang,
%``Exposing the dead-cone effect of jet quenching in QCD medium,''
[arXiv:2205.14668 [hep-ph]].





%\bibitem{Guo:2000nz}
%  X.~F.~Guo and X.~N.~Wang,
  %``Multiple Scattering, Parton Energy Loss and Modified Fragmentation
  %Functions in Deeply Inelastic eA Scattering,''
%  Phys.\ Rev.\ Lett.\  {\bf 85}, 3591 (2000).
  %[arXiv:hep-ph/0005044].
  %%CITATION = PRLTA,85,3591;%%

%\bibitem{Zhang:2003yn}
%B.~W.~Zhang and X.~N.~Wang,
%``Multiple parton scattering in nuclei: Beyond helicity amplitude approximation,''
%Nucl. Phys. A \textbf{720}, 429-451 (2003)
%doi:10.1016/S0375-9474(03)01003-0
%[arXiv:hep-ph/0301195 [hep-ph]].


  %\cite{Zhang:2004qm}
%\bibitem{Zhang:2004qm}
%  B.~W.~Zhang, E.~Wang and X.~N.~Wang,
  %``Multiple parton scattering in nuclei: Heavy quark energy loss and modified fragmentation functions,''
%  Nucl.\ Phys.\ A {\bf 757}, 493 (2005)
  %doi:10.1016/j.nuclphysa.2005.04.022
 % [hep-ph/0412060].
  %%CITATION = doi:10.1016/j.nuclphysa.2005.04.022;%%
  %27 citations counted in INSPIRE as of 21 Mar 2018


%\cite{Pang:2012he}
%\bibitem{Pang:2012he}
%L.~Pang, Q.~Wang and X.~N.~Wang,
%``Effects of initial flow velocity fluctuation in event-by-event (3+1)D hydrodynamics,''
%Phys. Rev. C \textbf{86}, 024911 (2012)
%doi:10.1103/PhysRevC.86.024911
%[arXiv:1205.5019 [nucl-th]].
%200 citations counted in INSPIRE as of 06 Sep 2023

%\cite{Luo:2023nsi}
%\bibitem{Luo:2023nsi}
%T.~Luo, Y.~He, S.~Cao and X.~N.~Wang,
%``Linear Boltzmann transport for jet propagation in the quark-gluon plasma: Inelastic processes and jet modification,''
%[arXiv:2306.13742 [nucl-th]].
%1 citations counted in INSPIRE as of 14 Sep 2023

%\cite{Wang:1998bha}
\bibitem{Wang:1998bha}
X.~N.~Wang,
%``Effect of jet quenching on high $p_{T}$ hadron spectra in high-energy nuclear collisions,''
Phys. Rev. C \textbf{58}, 2321 (1998)
%doi:10.1103/PhysRevC.58.2321
%[arXiv:hep-ph/9804357 [hep-ph]].

%\cite{Liu:2006sf}
\bibitem{Liu:2006sf}
W.~Liu, C.~M.~Ko and B.~W.~Zhang,
%``Jet conversions in a quark-gluon plasma,''
Phys. Rev. C \textbf{75}, 051901 (2007)
%doi:10.1103/PhysRevC.75.051901
%[arXiv:nucl-th/0607047 [nucl-th]].

%\cite{Chen:2008vha}
\bibitem{Chen:2008vha}
X.~Chen, H.~Zhang, B.~W.~Zhang and E.~Wang,
%``A Study on the anomaly of p over pi ratios in Au+Au collisions with jet quenching,''
J. Phys. \textbf{37}, 015004 (2010)
%doi:10.1088/0954-3899/37/1/015004
%[arXiv:0806.0556 [hep-ph]].





%%%   Guo:2000nz,Zhang:2003yn,Zhang:2003wk




%\cite{Pang:2012he}
%\bibitem{Pang:2012he}
%L.~Pang, Q.~Wang and X.~N.~Wang,
%``Effects of initial flow velocity fluctuation in event-by-event (3+1)D hydrodynamics,''
%Phys. Rev. C \textbf{86} (2012), 024911
%doi:10.1103/PhysRevC.86.024911
%[arXiv:1205.5019 [nucl-th]].
%145 citations counted in INSPIRE as of 02 Mar 2021

















%\cite{Xing:2019xae}
\bibitem{Xing:2019xae}
W.~J.~Xing, S.~Cao, G.~Y.~Qin and H.~Xing,
%``Flavor hierarchy of jet quenching in relativistic heavy-ion collisions,''
Phys. Lett. B \textbf{805}, 135424 (2020)
%doi:10.1016/j.physletb.2020.135424
%[arXiv:1906.00413 [hep-ph]].
%25 citations counted in INSPIRE as of 28 Nov 2022


%\cite{CMS:2016uxf}
\bibitem{CMS:2016uxf}
V.~Khachatryan \textit{et al.} [CMS],
%``Measurement of inclusive jet cross sections in $pp$ and PbPb collisions at $\sqrt{s_{NN}}=$ 2.76 TeV,''
Phys. Rev. C \textbf{96}, no.1, 015202 (2017)
%doi:10.1103/PhysRevC.96.015202
%[arXiv:1609.05383 [nucl-ex]].
%178 citations counted in INSPIRE as of 17 Feb 2024


%\cite{ATLAS:2012tjt}
\bibitem{ATLAS:2012tjt}
G.~Aad \textit{et al.} [ATLAS],
%``Measurement of the jet radius and transverse momentum dependence of inclusive jet suppression in lead-lead collisions at $\sqrt{s_{NN}}$= 2.76 TeV with the ATLAS detector,''
Phys. Lett. B \textbf{719}, 220-241 (2013)
%doi:10.1016/j.physletb.2013.01.024
%[arXiv:1208.1967 [hep-ex]].
%358 citations counted in INSPIRE as of 06 Jul 2023





%\cite{Huang:2013vaa}
\bibitem{Huang:2013vaa}
J.~Huang, Z.~B.~Kang and I.~Vitev,
%``Inclusive b\textrm{-}jet production in heavy ion collisions at the LHC,''
Phys. Lett. B \textbf{726}, 251-256 (2013)
%doi:10.1016/j.physletb.2013.08.009
%[arXiv:1306.0909 [hep-ph]].
%63 citations counted in INSPIRE as of 07 Oct 2022



%\cite{Horowitz:2007su}
\bibitem{Horowitz:2007su}
W.~A.~Horowitz and M.~Gyulassy,
%``Heavy quark jet tomography of Pb + Pb at LHC: AdS/CFT drag or pQCD energy loss?,''
Phys. Lett. B \textbf{666}, 320-323 (2008)
%doi:10.1016/j.physletb.2008.04.065
%[arXiv:0706.2336 [nucl-th]].
%114 citations counted in INSPIRE as of 11 Mar 2023







%\cite{Xing:2023ciw}
\bibitem{Xing:2023ciw}
W.~J.~Xing, S.~Cao and G.~Y.~Qin,
%``Flavor hierarchy of parton energy loss in quark-gluon plasma from a Bayesian analysis,''
Phys. Lett. B \textbf{850}, 138523 (2024)
%doi:10.1016/j.physletb.2024.138523
%[arXiv:2303.12485 [hep-ph]].
%5 citations counted in INSPIRE as of 17 Feb 2024
\end{thebibliography}


\end{document}


