% ****** Start of file apssamp.tex ******
%
%   This file is part of the APS files in the REVTeX 4.2 distribution.
%   Version 4.2a of REVTeX, December 2014
%
%   Copyright (c) 2014 The American Physical Society.
%
%   See the REVTeX 4 README file for restrictions and more information.
%
% TeX'ing this file requires that you have AMS-LaTeX 2.0 installed
% as well as the rest of the prerequisites for REVTeX 4.2
%
% See the REVTeX 4 README file
% It also requires running BibTeX. The commands are as follows:
%
%  1)  latex apssamp.tex
%  2)  bibtex apssamp
%  3)  latex apssamp.tex
%  4)  latex apssamp.tex
%
\documentclass[%
reprint,
superscriptaddress,
%groupedaddress,
%unsortedaddress,
%runinaddress,
%frontmatterverbose, 
floatfix,
%preprint,
%preprintnumbers,
%nofootinbib,
%nobibnotes,
%bibnotes,
amsmath,amssymb,
aps,
prl,
%pra,
%prb,
%rmp,
%prstab,
%prstper,
%floatfix,
]{revtex4-2}

\usepackage{graphicx}% Include figure files
\usepackage{dcolumn}% Align table columns on decimal point
\usepackage{bm}% bold math
%\usepackage{hyperref}% add hypertext capabilities
%\usepackage[mathlines]{lineno}% Enable numbering of text and display math
%\linenumbers\relax % Commence numbering lines

%\usepackage[showframe,%Uncomment any one of the following lines to test 
%%scale=0.7, marginratio={1:1, 2:3}, ignoreall,% default settings
%%text={7in,10in},centering,
%%margin=1.5in,
%%total={6.5in,8.75in}, top=1.2in, left=0.9in, includefoot,
%%height=10in,a5paper,hmargin={3cm,0.8in},
%]{geometry}

%\usepackage{placeins}
\usepackage{xcolor}
%\renewcommand{\textcolor{red}}{\textcolor{???}}
%\renewcommand{T_\mathrm{c0}}{T_\mathrm{c0}}
%\renewcommand{k_\mathrm{B}}{k_\mathrm{B}}

\newcommand{\beginsupplement}{%
	\setcounter{table}{0}
	\renewcommand{\thetable}{S\arabic{table}}%
	\setcounter{figure}{0}
	\renewcommand{\thefigure}{S\arabic{figure}}%
	\renewcommand{\theequation}{S.\arabic{equation}}
	%\renewcommand{\thesection}{S.\arabic{section}}
}


\begin{document}
	
	\preprint{APS/123-QED}
	
	%	\title{ }
	\title{Sharpness of the Berezinskii-Kosterlitz-Thouless transition in ultrathin NbN films}% Force line breaks with \\
	%\thanks{A footnote to the article title}%
	
	\author{Alexander Weitzel}
	\altaffiliation[]{These authors contributed equally.}%Lines break automatically or can be forced with \\
	\affiliation{
		Institute for Experimental and Applied Physics, University of Regensburg, D-93040 Regensburg, Germany}
	
	\author{Lea Pfaffinger}%
	%\\
	% \email{Second.Author@institution.edu}
	\altaffiliation[]{These authors contributed equally.}
	\affiliation{
		Institute for Experimental and Applied Physics, University of Regensburg, D-93040 Regensburg, Germany}
	
	
	
	\author{Ilaria Maccari}%
	%\\
	% \email{Second.Author@institution.edu}
	\affiliation{
		Department of Physics, Stockholm University, Stockholm SE-10691, Sweden}
	
	
	
	
	
	\author{Klaus Kronfeldner}
	\affiliation{
		Institute for Experimental and Applied Physics, University of Regensburg, D-93040 Regensburg, Germany}
	%\altaffiliation[]{These authors contributed equally.}%Lines break automatically or can be forced 
	
	\author{Thomas Huber}
	\affiliation{
		Institute for Experimental and Applied Physics, University of Regensburg, D-93040 Regensburg, Germany}
	%\altaffiliation[]{These authors contributed equally.}%Lines break automatically or can be forced 
	
	\author{Lorenz Fuchs}
	\affiliation{
		Institute for Experimental and Applied Physics, University of Regensburg, D-93040 Regensburg, Germany}
	%\altaffiliation[]{These authors contributed equally.}%Lines break automatically or can be forced 
	
	\author{Sven Linzen}
	\affiliation{
		Leibniz Institute of Photonic Technology, D-07745 Jena, Germany}
	%\altaffiliation[]{These authors contributed equally.}%Lines break automatically or can be forced 
	
	\author{Evgeni Il'ichev}
	\affiliation{
		Leibniz Institute of Photonic Technology, D-07745 Jena, Germany}
	%\altaffiliation[]{These authors contributed equally.}%Lines break automatically or can be forced 
	
	\author{Nicola Paradiso}
	\affiliation{
		Institute for Experimental and Applied Physics, University of Regensburg, D-93040 Regensburg, Germany}
	%\altaffiliation[]{These authors contributed equally.}%Lines break automatically or can be forced 
	
	\author{Christoph Strunk}
	\affiliation{
		Institute for Experimental and Applied Physics, University of Regensburg, D-93040 Regensburg, Germany}
	%\altaffiliation[]{These authors contributed equally.}%Lines break automatically or can be forced 
	
	
	%\collaboration{MUSO Collaboration}%\noaffiliation
	
	%\author{Charlie Author}
	%\homepage{http://www.Second.institution.edu/~Charlie.Author}
	%\affiliation{
		% Second institution and/or address\\
		% This line break forced% with \\
		%}%
	%\affiliation{
		% Third institution, the second for Charlie Author
		%}%
	%\author{Delta Author}
	%\affiliation{%
		% Authors' institution and/or address\\
		% This line break forced with \textbackslash\textbackslash
		%}%
	%
	%\collaboration{CLEO Collaboration}%\noaffiliation
	
	\date{\today}% It is always \today, today,
	%  but any date may be explicitly specified
	
	\begin{abstract}
		%\textcolor{gray}{
			We present a comprehensive investigation of the Berezinskii-Kosterlitz-Thouless (BKT) transition in ultrathin strongly disordered NbN films. Measurements of resistance, current-voltage characteristics and kinetic inductance on the very same device reveal a consistent picture of a sharp unbinding transition of vortex-antivortex pairs that fit standard renormalization group theory without extra assumptions in terms of inhomogeneity. Our experiments demonstrate that the previously observed broadening of the transition is not an intrinsic feature of strongly disordered superconductors and provide a clean starting point for the study of dynamical effects at the BKT transition.  
			%} 
		
		
		
	\end{abstract}
	
	
	\maketitle
	
	
	In two dimensions, the superfluid transition is governed by the presence of thermally excited vortex-antivortex pairs \cite{KT_1973, Kosterlitz_1974}.  For superfluid $^4$He films, the defining features of the Berezinskii-Kosterlitz-Thouless (BKT) transition are well understood \cite{NelsonKosterlitz1977,McQueeney1984}. In thin-film superconductors, where an analogous behavior is expected, the situation is much less clear due to the extremely high degree of disorder required to access the BKT-regime. Also in this case, the transition is caused by dissociation of vortex-antivortex pairs. The transition is manifested as a discontinuous jump in the superfluid phase stiffness $J_s$ at a temperature $T_\mathrm{BKT}$ below the mean-field transition temperature $T_\mathrm{c0}$. Moreover, below $T_\mathrm{BKT}$ the voltage-current characteristics are nonlinear, $V\propto I^{\alpha(T)}$, with a temperature dependent exponent $\alpha$. In the thermodynamic limit, a linear voltage response regime exists above $T_\mathrm{BKT}$ only. 
	
	Physics of the BKT-transition is controlled by two energy scales \cite{Benfatto2009}. In order to thermally excite a vortex-antivortex pair in a film, the energy cost for the generation of vortex cores (also called vortex fugacity) $\mu \propto \xi^2$ as well as the energy scale for the pair dissociation $J_s \propto 1/\lambda^2$ must be sufficiently small. Here $\xi$ and $\lambda$ are the coherence length and magnetic penetration depth, respectively.  In the dirty limit, both $\xi^2$ and $1/\lambda^2$ are proportional to the elastic mean free path. Owing to their small $\mu$ and $J_s$, ultrathin films of strongly disordered superconductors are the preferred choice for materials that feature a large separation between $T_\mathrm{BKT}$ and the mean field critical temperature $T_{c0}$.  
	
	In the past $J_s(T)$ and $V(I)$ were studied for InO and NbN thin films \cite{FioryHebardGlaberson_1983, Yong2013, Venditti2019} using the two-coil method \cite{Turneaure2000} and standard transport measurements. The two observables, however, require strongly different sample geometries. Thus they could not be studied in the same device before. This limits the validity of consistency checks. While a qualitative agreement with original theory was observed, measurements of strongly disorderd NbN-films always displayed a strong broadening of the BKT-transition, far stronger than expected for, e.g., finite size effects alone \cite{Benfatto2009, Mondal_2011b}. At present, such broadening is believed to be typical for highly disordered superconducting films that are known to feature \textit{emergent granularity} \cite{Ghosal1998,Ghosal2001b, Sacepe2008, Carbillet2016, Carbillet2020, Stosiek2020}. Local variations of the modulus of the order parameter and superfluid stiffness could, in principle, explain the observed smearing of the expected discontinuous jump in $J_s$. On the other hand, the smearing introduces an additional free parameter that inevitably obscures the quantitative analysis.
	
	Within the generally accepted picture, individual signatures of the BKT transition have been observed \cite{Mondal_2011b, Yong2013, FioryHebardGlaberson_1983, Turneaure2000, Crane2007, Mandal2020, Broun2007, Kamal1994, Yong2012, Zuev2005, Maccari2017, Venditti2019, Medveyeva2000,Ganguly2015, Mallik2022}.  In recent years, however, it turned out that each of these signatures is affected by experimental subtleties that need to be controlled in order to reliably test the level of  consistency \cite{Tamir2019, Benyamini2020}. The most popular signature, the non-linearity of $V(I)$, is also the most difficult to interpret, as many other effects affect it. For example, any fluctuation broadening of the resistive transition leads to non-linear $V(I)$ via heating. This can mimic power-law behavior, in particular close to the normal state resistance and for materials with $T_\mathrm{c0}\lesssim1\,$K \cite{Levinson2019}. To address this issue, a set of techniques is desirable that do not extrinsically broaden the transition and allows for all types of measurements to be performed on the very same device.
	
	%new attempt Oct, 30th:
	In this Letter, we observe a sharp BKT-transition in ultra-thin NbN films. We find an excellent  agreement between the measured superfluid stiffness, current voltage characteristics,  temperature- and magnetic field-dependent resistance.  This demonstrates that the previously observed broadening of the transition is \textit{not} a generic  feature of strongly disordered superconductors. Using a low-frequency resonator technique compatible with four terminal dc-measurements, we unambiguously identify the BKT- and mean field transition temperatures. %
	Our results provide a solid basis for the study of more complex non-equilibrium properties of ultra-thin and strongly disordered superconductors.
	
	
	Our NbN films are grown by atomic layer deposition (ALD) \cite{Linzen_2017} with a thickness $d = 3$~nm on top of a thermally oxidized silicon wafer. Using standard electron beam lithography and selective etching techniques we prepared long ($\sim$ 100-200 squares) meander structures of widths 10 - 200~\textmu m with kinetic inductance on the order 100 nH. The samples are mounted into a cold RLC circuit, whose resonance frequency provides access to the sheet kinetic inductance $L_\square$ of the sample \cite{Baumgartner_2020,Supplement}. The resonance frequency of the circuit varies between 0.5\,-\,3\,MHz, depending on $L_\square$.
	From the kinetic inductance, the superfluid stiffness is inferred as 
	\begin{align}\label{eq:J_S}
		J_s\ =\ \frac{\hbar^2d}{4e^2k_\mathrm{B}\mu_0\lambda^2}\ =\ \frac{\hbar^2}{4e^2k_\mathrm{B} L_{\square}}\;, 
	\end{align}
	where  $h$ is Planck's constant, $e$ the electron charge and $k_\mathrm{B}$ being Boltzmann's constant. 
	%
	The parameters of our films are well in line with those of \cite{Mondal_2011a,Mondal_2011b}, albeit with lower thickness for the same values of  $k_F\ell$ and $T_\mathrm{c0}$. Additional voltage probes allow for measurement of DC $V(I)$ characteristics on the same device. Resistance values were always extracted from the linear regime of $V(I)$. %Nevertheless, electron heating remains as an extrinsic source of non-linearity, in particular above and close to $T_\mathrm{BKT}$. 
	
	\begin{figure}[t]
		\includegraphics[width=.45\textwidth]{R_T_final_final_final}% Here is how to import EPS art
		\caption[R(T)]{Sheet resistance, $R$(T), as function of temperature of a 10~\textmu m wide and 2~mm long NbN meander on a logarithmic scale. In the normal state, we measure $R_N=R(15\,\text{K})=4.1\,$k$\Omega$ an electron density $n \simeq 2\cdot 10^{22}\mathrm{cm}^{-3}$ and $k_F\ell\simeq 2$.  Black: data, red: fit including amplitude fluctuations above mean field critical temperature $T_{c0}=5.203\,$,K and dephasing parameter $\delta=0.424$ \cite{Supplement}, blue: fit to square-root-cusp expression corresponding to $T_\mathrm{BKT}=4.471\,$K and  $b=0.76$  (see text). Inset: Linear scale of $R$-axis emphasizes amplitude fluctuations.}
		\label{fig:R(T)}
	\end{figure}
	%
	
	We start by establishing DC transport properties.  Figure \ref{fig:R(T)} shows resistance as function of temperature for a typical meander (width 10~\textmu m, length 2~mm). The transition is strongly broadened by fluctuations of both amplitude and phase fluctuations of the order parameter \cite{Baturina_2012, Postolova2015, LarkinVarlamov2005, Supplement}. 
	
	
	% a 10\,\textmu m-wide meander with 200 squares in series. The transition from normal to superconducting state is broadened over several Kelvin, as shown by the inset in Fig.~\ref{fig:R(T)}. Two types of superconducting fluctuations describe the transition quantitatively: above the mean field transition temperature $T_{c0}$, the temperature-dependent conductivity is affected by paraconductivity due to thermal fluctuations of the amplitude of the superconducting order parameter, weak localization and electron-electron interactions \cite{Baturina_2012, Supplement}. 
	
	
	Fitting $R(T)$ in Fig.~\ref{fig:R(T)}  for $R>0.6\,R_N$ (red line) reveals a mean-field transition temperature $T_{c0}=5.203$\,K. Below $T_{c0}$, phase fluctuations of the order parameter generate resistance, where $R(T)$ is described by the 'square root cusp' expression $R(T)\propto \exp(b/\sqrt{T/T_\mathrm{BKT}-1})$ \cite{HalperinNelson_1979, Baturina_2012, Postolova2015, Supplement}. Good agreement is found between theory (blue line) and experiment.  %Values of $T_{c0}$ and $T_{\mathrm{BKT}}$  can be directly compared to those extracted from $V(I)$ and $L(T)$ data of the same device discussed below. 
	Very similar results are found also for other devices with different width and length.% \cite{Supplement}.
	
	
	
	\begin{figure}[t]
		\includegraphics[width=.48\textwidth]{IV_final.pdf}% Here is how to import EPS art
		\caption[IV]{Voltage-current ($V(I)$) characteristics at different temperatures.  When collecting $V(I)$ over a wide range of current, we used fast sweeps (few sec), in order to avoid heating of the chips.  Straight lines in log-log display indicate power law behavior $V\propto I^{\alpha(T)}$, where $\alpha$ is related to $J_s(T)$ \cite{HalperinNelson_1979}. Dotted and dashed black lines corresponds to $\alpha=1$ and $3$, respectively. Red solid line in the upper left corner corresponds to $V=R_NI$ in the normal state. Slope $\alpha=3$ corresponds to $T=4.475$ (yellow), very close to the value $T_\mathrm{BKT}=4.471$ obtained in Fig.~\ref{fig:R(T)}. $J_s(T)$ extracted from the power law exponents $\alpha(T)$ is displayed in Fig.~\ref{fig:L(T)}. }
		\label{fig:IV}
	\end{figure}
	
	According to Halperin-Nelson (HN) theory, $V(I,T)$ takes the form \cite{HalperinNelson_1979} 
	\begin{equation}\label{eq:HN-IV}
		V(I,T)=A(T)\cdot I^{\alpha(T)}
	\end{equation}
	%where $I_0(T)$ is a characteristic current in the problem sometimes dubbed scaling current. %Within HN-theory the expression $I^\mathrm{HN}_0(T) = wek_\mathrm{B} T_c/[\hbar \xi(T)]$ is found.
	%In short, $V(I,t)$ takes the form  $V(I,T)=A(T)\cdot I^{\alpha(T)}$ 
	with exponent $\alpha(T)={\pi J_s(T)}/{T}+1$ and prefactor $A(T)$. Hence, power-law behavior of $V(I)$-characteristics below $T_{\mathrm{BKT}}$, is another hallmark of the BKT-transition. Increasing temperature decreases $J_s$ and thus $\alpha$. At the universal transition line $\pi J_S(T_\mathrm{BKT})=2T_\mathrm{BKT}$ (dashed in Fig.~\ref{fig:IV}) a characteristic jump of $\alpha(T_\mathrm{BKT})=3$ to $\alpha=1$ is predicted. 
	
	In Figure~\ref{fig:IV} we present the evolution of $V(I)$ with temperature. At low temperatures and voltages the double-logarithmic plot reveals the expected power-law dependence. At higher temperatures and in a wider voltage range $V(I)$ turns out to be much more complex  \cite{Baturina_2012,Postolova2015, Supplement}. 
	Above $T_\mathrm{BKT}$ and low current $V(I)$ is expected to be linear. The linear regime is limited first by current-induced dissociation of vortex-antivortex pairs, leading again to power-law behavior of $V(I)$, but now with values of $\alpha$ smaller than 3. At very high currents also heating effects play a role.  Note that the voltage level is orders of magnitude below $R_NI$.
	
	%At the highest temperatures $T\gtrsim T_\mathrm{c0}$, the linear part of the $V(I)$-characteristics becomes more and more extended until they merge into the normal state resistance at $T\simeq 8-10\,$K (not shown). 
	
	%To minimize heating effects  \cite{Levinson2019}, we performed fast current sweeps, with an approximate duration of a few seconds. A more conventional point-by-point measurement takes about 10~min leads to stronger overheating of the NbN-film and significantly higher voltages for a given current with respect to results of the fast measurement (Fig.~\ref{fig:sup:IV_fast_v_slow} in \cite{Supplement}).
	
	
	\begin{figure}[t]
		\includegraphics[width=.48\textwidth]{L_T_final_final.pdf}% Here is how to import EPS art
		\caption[L(T)]{Superfluid stiffness, $J_s$, vs. temperature, $T$ for the same device as in Figs.~\ref{fig:R(T)} and \ref{fig:IV}. Black dots: $J_s$ extracted from kinetic inductance, red: $J_s$ extracted from nonlinear $IV$-characteristics (Fig.~\ref{fig:IV}), black: BCS-fit of low temperature part, light blue: Renormalization group calculation, dashed black: Nelson-Kosterlitz universal line. Inset: Zoom to the critical region near the jump. The fit parameters for the BCS fit are: $T_{c0}=5.188$\,K, $\Delta(0)/k_\mathrm{B} = 13.2$\,K and $J_s(0)=7.216\,$K. The intersection of the data points with the universal line occurs at $T_{\mathrm{BKT}}= 4.488\,$K. The value of the vortex core energy extracted from the RG-fit is $\mu=18.0$\,K.}
		\label{fig:L(T)}
	\end{figure}
	
	The superfluid stiffness $J_s(T)=T[\alpha(T)-1]/\pi$ is shown as red dots in Fig.~\ref{fig:L(T)} together with $J_s(T)$ (black dots) measured directly via the kinetic inductance $L_\square(T)$. The excellent agreement between the two independent data sets nicely substantiates our analysis of the DC measurements.
	The gradual decrease of $J_s$ towards higher $T$ can be described by the expression \cite{Yong2013,Mondal_2011b}
	\begin{align}\label{eq:J_BCS}
		J_s(T) &= 
		%\pi\hbar\Delta(T)/(4e^2k_\mathrm{B} R_N)\cdot
		J_s(0)\cdot\Delta(T)/\Delta(0)\cdot\tanh[\Delta(T)/(2k_\mathrm{B} T)]
	\end{align}
	(grey line), which accounts for the depletion of $J_s$ by quasiparticle excitations. In order to obtain a good match, it is established practice \cite{Yong2013,Mondal_2011b} to use $J_s(0)$, $\Delta(0)$ and $T_{c0}$ as independent fitting parameters \cite{Supplement}. 	The best fit is obtained for $T_{c0}=5.188$\,K, $J_s(0)=\beta J_\mathrm{BCS}(0)$ and $\Delta(0)=\gamma 1.764k_\mathrm{B}T_\mathrm{c0}$ with $\beta=0.657$, $\gamma=1.522$. While $T_{c0}$ is in excellent agreement with the value obtained from the amplitude fluctuations of the order parameter (Fig.~\ref{fig:R(T)}), the ratio $\Delta(0)/k_\mathrm{B} T_\mathrm{c0}=2.685$ exceeds the BCS-value of 1.764 as observed earlier \cite{Yong2013,Semenov2009,Mondal_2011a,Mandal2020}. Moreover, $J_s(0)$ is smaller than the dirty limit BCS-prediction $J_\mathrm{BCS}(0)=\pi\hbar\Delta(0)/(4e^2k_\mathrm{B} R_N)$, consistent with the conjectured suppression of $J_s(0)$ by phase fluctuations \cite{Mondal_2011a}. 
	The ratio $R_NJ_s(0)/T_\mathrm{c0}$ for several of our films with $R_N\simeq4$~k$\Omega$ agrees within a few percent with the BCS-value of $1.764\pi\hbar/(4e^2)=5.692$~k$\Omega$. This indicates that disorder effects in $J_s(0)/T_\mathrm{c0}$ are accounted for by $R_N$ alone, while both $J_s(0)$ and $T_\mathrm{c0}$ substantially differ from their dirty-limit BCS-expressions. An independent confirmation of the value of $\Delta(0)$  is highly desirable, e.g.~via tunneling spectroscopy on the very same material.
	
	Based on direct measurements of $\Delta(0)$ via tunneling spectroscopy, Carbillet et al.~proposed an interpretation of the large  $\Delta(0)/k_\mathrm{B} T_{c0}$ in terms of an underestimation of $T_\mathrm{c0}$ \cite{Carbillet2020}. In the latter work, $T_\mathrm{c0}$ was associated with the onset of the resistance, rather than $T_\mathrm{BKT}$. Here we can exclude this possibility, as our analysis allows for an unambiguous determination of $T_\mathrm{BKT}$ and $T_\mathrm{c0}$. 
	
	
	%The fit describes our data nicely, albeit only when leaving the ratio $\Delta(0)/k_\mathrm{B} T_{c0}$ as a free parameter. The BCS-fit in Fig.~\ref{fig:L(T)} requires (\textcolor{red}{ to fit the curvature in Fig.~3}) a value $\Delta(0)/k_\mathrm{B} T_{c0}=2.5$ is obtained, similar to values reported in Ref.~\cite{Yong2013}, but larger than the BCS weak coupling prediction of $\Delta(0)/k_\mathrm{B} T_{c0}=1.76$. The prefactor $J_s(0)$ is fixed by the experimental data at 
	%Alternatively, using $\Delta(0)=2.5k_\mathrm{B} T_{c0}$ gives $J_s(0)=10.65$ K. These values are reasonably close to the experimental value, given the ambiguity of determining $R_N$ in a material with strong quantum fluctuations of the conductivity and uncertainty in $\Delta(0)$. 
	
	
	
	At higher temperatures, very close to the universal transition point at $\pi J_s(T_{\mathrm{BKT}})=2T_{\mathrm{BKT}}$, $J_s(T)$ sharply drops to zero. The inset in Fig.~\ref{fig:L(T)} shows a zoom into the transition region. 
	Excellent agreement between the inductance data and the entirely independent nonlinear $IV$-characteristics is found: $T_{c0}$ and $T_{\mathrm{BKT}}$ obtained from $R(T)$ and $J_s(T)$, respectively, match within 1\%.  The observed broadening amounts to $\simeq 50\,$mK or $0.07\cdot(T_\mathrm{c0}-T_\mathrm{BKT})$, much smaller than in previous experiments on ultrathin NbN films \cite{Yong2013, Mondal_2011a, Mandal2020}. 
	
	
	We can theoretically describe the  drop of $J_s(T)$, taking the the BCS-fit to $J_s(T)$ as input for the BKT renormalization group (RG) equations \cite{Benfatto2009,Maccari2017}. In this way,  data is closely reproduced (grey dashed line in Fig.~\ref{fig:L(T)}), assuming a vortex fugacity $\mu=18.0\,$K, or $\mu/J_s(0)\approx 2.5$, similar to values reported, e.g., in Ref.~\cite{Yong2013}.  It is instructive to compare $\mu$ with the loss of condensation energy $u_\mathrm{cond}$ in the vortex cores with effective radius $r_v$. We write $u_\mathrm{cond}=\mu/(\pi r_v^2d)\equiv B_c^2/2\mu_0=1/(2\mu_0)\cdot[\hbar/(2\sqrt{2}e\xi\lambda)]^2$, where $B_c$ is the thermodynamic critical field, $\mu_0$ being the vacuum permeability. %and $\lambda(0)= 1.5$~\textmu m the penetration depth obtained from the kinetic inductance for $T\rightarrow 0$. while $d=3$~nm is the thickness of the film.  
	%From $R(B)$ (see below) we estimate the upper critical field as function of temperature \cite{Supplement}, and find $\xi(T_\mathrm{BKT})\simeq12.6$~nm. 
	From the equation for $\mu$, we find $r_v/\xi(T_\mathrm{BKT})\simeq 2.2$.
	Using the expression $T_\mathrm{BKT}=T_\mathrm{c0}(1-4\,Gi)$  with $Gi=4\,e^27\zeta(3)R_N/(\pi^3h)=0.168$ being the Ginzburg-Levanyuk number \cite{Koenig2015}, we expect $T_\mathrm{BKT}^\mathrm{theo}=4.315$\,K, which is only 4\% smaller than $T_\mathrm{BKT}=4.488$\,K extracted from Fig.~\ref{fig:L(T)}.
	
	
	
	
	Finally, we investigate  signatures of the BKT-transition in magnetic field perpendicular to the film. In the high-field regime and near $T_\mathrm{BKT}$, $R(B)$ is expected to cross over from sublinear to superlinear behavior \cite{Garland1987}, signaling a transition from amplitude fluctuations of the order parameter to vortex pinning. In Fig.~\ref{fig:R(B)}a we  observe such this cross-over at $T=4.3\,$K (blue line) slightly below the range  extracted from the other observables. This discrepancy is probably caused by lack of thermal cycling between curves. For the low-field regime, Minnhagen has derived the scaling law  \cite{Minnhagen1984}
	%
	\begin{equation}
		\label{eq:RB_scaling}
		\frac{B}{B_{c2}} =  \frac{R(B)}{R_N}\left[ 1-\left(\frac{1}{\nu}\ln\frac{R(0)}{R_N}\right)^2
		\left(\frac{R(B)}{R_N}\right)^{-2/\nu}
		\right]^{1/2}
	\end{equation}
	%
	where $B_{c2}(T)=\Phi_0/2\pi\xi^2(T)$ is the upper critical field and the scaling parameter $\nu$ is a universal function of $T$ and $B$. In the Ginzburg-Landau (GL) limit, the coherence length can be written in the form $\xi(T)=\xi(T_\mathrm{BKT})[(T_\mathrm{c0}-T_\mathrm{BKT})/(T_\mathrm{c0}-T)]^{1/2}$. In Fig.~\ref{fig:R(B)}b we show separately measured low field data with field cooling procedure. From the measured $R(T,B)$ at fixed value of $B$, the function $\nu(T)$ can be determined from Eq.~\ref{eq:RB_scaling}, if $B_{c2}$ is given. Adjusting $B_{c2}(T_\mathrm{BKT})=6.1\pm1.7\,$\,T leads to a collapse of the set of $\nu(T)$-curves at low field (Fig.~\ref{fig:R(B)}b) that corresponds to a coherence length of $\xi(T_\mathrm{BKT})=6.7\pm0.6$\,nm. The error margins mark a deviation from optimal scaling by one dot size. This implies that $\nu(T,B)$ only weakly depends of $B$. The BKT-transition temperature is reflected as a cusp in  $\nu(T,B)$ which is located within 50~mK of $T_\mathrm{BKT}$ from the $R(T)$-curve (arrows in  Fig.~\ref{fig:R(B)}b).  
	
	We use the scaling function $\nu(T)$ in order to predict $R(T,B)$ vs.~$T$ at low $B$. Figure~\ref{fig:R(B)}c demonstrates that the result agrees rather nicely with our directly measured $R(T)$ data and that the scaling expression of Eq.~\ref{eq:RB_scaling} works reasonably well.
	The value $B_{c2}(T_\mathrm{BKT})= 6.1$~T significantly deviates from the 2.45~T estimated from the linear high-field magnetoresistance \cite{Supplement}. As the $R(B)$ curves are extremely broad, we suspect that a theory that combines vortex pinning with the more established modeling of superconducting fluctuations \cite{Tikhonov2012} is required to reliably extract $B_{c2}(T)$ from $R(B)$.
	
	At magnetic field below 5-10~\textmu T the resistance becomes independent of magnetic field. This observation, which is not reflected by the scaling law, is in line with the fact, that one expects one vortex per $(10~$\textmu m$)^2$ area at $21~$\textmu T. For lower fields, nucleation of vortices becomes unfavorable, because magnetic screening is negligible then. 
	
	\begin{figure}[t]
		\includegraphics[width=.45\textwidth]{R_B_final_final.pdf}% Here is how to import EPS art
		\caption[]{a) Magnetoresistance in the vicinity of $T_\mathrm{BKT}$ at different $T$. Curves correspond to temperatures (in K from top to bottom): 4.8, 4.6, 4.55, 4.4, 4.3, 4.0, 3.5, 2.5. Blue curve ($T=4.3$ K) shows linear slope, red curve ($T=4.45$ K) is closest to $T_\mathrm{BKT}$ from $J_s$. b) Low-field magnetoresistance expressed in term of the scaling parameter $\nu(T)$ (see text).  Arrows correspond to $T_\mathrm{BKT}$ from $R(T)$, $V(I)$ and $J_s(T)$, respectively. c) $R(T)$ at very low fields together with the scaling function (Eq.~\ref{eq:RB_scaling}). }
		\label{fig:R(B)}
	\end{figure}
	
	
	%Fitting $R(B)$ using Eq.~\ref{eq:RB_scaling} results in the curves $\nu(T,B)$ depicted for weak magnetic field in Fig.~\ref{fig:R(B)}b. Near $T_\mathrm{BKT}$ a relatively weak $B$-dependence of $\nu$ is observed. Most curves display a divergence close to $T_\mathrm{BKT}$ within 50~mK of $T_\mathrm{BKT}$ from $J_s(T)$ \textcolor{red}{???}. Finally, we use the resulting $\nu(T,B)$ in order to compare $R(T)$ at low $B$ with our experimental data, demonstrating that Eq.~\ref{eq:RB_scaling} describes the data reasonably well \textcolor{red}{???}.
	
	
	\vspace{1mm}
	\emph{Discussion:} 
	The central result of our work is the observation of a sharp, textbook-like BKT-transition in strongly disordered ultrathin NbN-films. Hence, the previously observed broadening \cite{Yong2013,Mondal_2011b} is no genuine consequence of strong (but homogeneous) disorder in superconducting thin films. The data can be consistently described in terms of the numerical renormalization group, without introducing inhomogeneity or finite size effects.  
	%
	A possible explanation for the discrepancy is a higher degree of homogeneity in our ALD-deposited films, as opposed to the sputter deposited films  in earlier work  \cite{Semenov2009,Carbillet2016,Carbillet2020}. Long-range correlated disorder can explain the observed broadening of the transition  in terms of a spatial variation of $J_s$ \cite{Benfatto2009, Benfatto_2013}. 
	%
	%At present, we can only conclude that our ALD-grown films appear not to display observable variations of $J_s$ that would manifest themselves as an extrinsic broadening of the BKT-transition. 
	On the other hand, short-range emergent granularity has been observed in STS for both sputter \cite{Carbillet2016, Kamlapure2013} and ALD \cite{Sacepe2008} deposited films alike. At least at the level of disorder in our present films, intrinsic inhomogeneity in the gap distribution appears to be irrelevant at the large length scales that determine the BKT-transition. 
	
	Most often, the presence of a BKT-transition is deduced from the non-linearity of $V(I)$-characteristics. However, this is not straightforward, as $V(I)$ often is strongly affected by heating phenomena. First, power-law behavior can occur also slightly above $T_\mathrm{BKT}$, where only relatively few vortex-anti-vortex pairs are dissociated. A linear regime exist at the lowest currents only, while already at current densities $\lesssim100\,$A/m$^2$ current-induced dissociation dominates over thermal dissociation, leading to power law behavior with $\alpha < 3$ that are not considered by standard theory.
	
	These observations are important, because a substantial fraction of the recent literature on ultra-thin materials analyzes $IV$-characteristics in the high-power regime above $T_{\mathrm{BKT}}$ in terms of BKT-behavior (see e.g.~\cite{Reyren2009,Zhang_2014,Lu_2015,cao_super_2018,Park_2021,Zhou_Young_2021}). Our work shows that power-law exponents obtained in this regime are unrelated to BKT-physics. %They lead to an overestimation of both $T_\mathrm{BKT}$ and the power law exponents $\alpha$ in the whole range between $T_\mathrm{BKT}$ and $T_\mathrm{c0}$. In high resistivity films, heating always dominates close to $R_N$. Although a vortex-unbinding transition probably exists at lower temperatures, such data do not provide unambiguous evidence for it.
	
	
	
	\emph{Conclusions:} We have shown that ultra-thin superconducting films with strong, but homogeneous, disorder feature a sharp BKT-transition without significant broadening. All relevant observables display quantitatively consistent results, allowing for a precise determination of the BKT- and mean-field transition temperatures as well as other parameters governing the films. Our study lays the ground for future controlled studies of the statics and dynamics of the BKT transition in ultra-thin superconductors when approaching to the superconductor-insulator transition.    
	
	%\vspace{2mm}
	\begin{acknowledgments}
		%{\it Acknowledgements:} 
		We would like to thank M.~Ziegler and V.~Ripka for NbN film deposition by ALD in the Leibniz IPHT clean room. E.~König, I.~Gornyi, A.~Mirlin,  P.~Raychaudhuri, A.~Ghosal and F.~Evers are acknowledged for helpful comments. The work was financially supported by the European Union’s Horizon 2020 Research
		and Innovation Program under grant agreements No 862660 QUANTUM E-LEAPS.
	\end{acknowledgments}
	%\clearpage

	
	%\nocite{*}
	
%	\bibliography{bibliography} % Entries are in the refs.bib file
	

%apsrev4-2.bst 2019-01-14 (MD) hand-edited version of apsrev4-1.bst
%Control: key (0)
%Control: author (8) initials jnrlst
%Control: editor formatted (1) identically to author
%Control: production of article title (0) allowed
%Control: page (0) single
%Control: year (1) truncated
%Control: production of eprint (0) enabled
\begin{thebibliography}{50}%
	\makeatletter
	\providecommand \@ifxundefined [1]{%
		\@ifx{#1\undefined}
	}%
	\providecommand \@ifnum [1]{%
		\ifnum #1\expandafter \@firstoftwo
		\else \expandafter \@secondoftwo
		\fi
	}%
	\providecommand \@ifx [1]{%
		\ifx #1\expandafter \@firstoftwo
		\else \expandafter \@secondoftwo
		\fi
	}%
	\providecommand \natexlab [1]{#1}%
	\providecommand \enquote  [1]{``#1''}%
	\providecommand \bibnamefont  [1]{#1}%
	\providecommand \bibfnamefont [1]{#1}%
	\providecommand \citenamefont [1]{#1}%
	\providecommand \href@noop [0]{\@secondoftwo}%
	\providecommand \href [0]{\begingroup \@sanitize@url \@href}%
	\providecommand \@href[1]{\@@startlink{#1}\@@href}%
	\providecommand \@@href[1]{\endgroup#1\@@endlink}%
	\providecommand \@sanitize@url [0]{\catcode `\\12\catcode `\$12\catcode
		`\&12\catcode `\#12\catcode `\^12\catcode `\_12\catcode `\%12\relax}%
	\providecommand \@@startlink[1]{}%
	\providecommand \@@endlink[0]{}%
	\providecommand \url  [0]{\begingroup\@sanitize@url \@url }%
	\providecommand \@url [1]{\endgroup\@href {#1}{\urlprefix }}%
	\providecommand \urlprefix  [0]{URL }%
	\providecommand \Eprint [0]{\href }%
	\providecommand \doibase [0]{https://doi.org/}%
	\providecommand \selectlanguage [0]{\@gobble}%
	\providecommand \bibinfo  [0]{\@secondoftwo}%
	\providecommand \bibfield  [0]{\@secondoftwo}%
	\providecommand \translation [1]{[#1]}%
	\providecommand \BibitemOpen [0]{}%
	\providecommand \bibitemStop [0]{}%
	\providecommand \bibitemNoStop [0]{.\EOS\space}%
	\providecommand \EOS [0]{\spacefactor3000\relax}%
	\providecommand \BibitemShut  [1]{\csname bibitem#1\endcsname}%
	\let\auto@bib@innerbib\@empty
	%</preamble>
	\bibitem [{\citenamefont {Kosterlitz}\ and\ \citenamefont
		{Thouless}(1973)}]{KT_1973}%
	\BibitemOpen
	\bibfield  {author} {\bibinfo {author} {\bibfnamefont {J.~M.}\ \bibnamefont
			{Kosterlitz}}\ and\ \bibinfo {author} {\bibfnamefont {D.~J.}\ \bibnamefont
			{Thouless}},\ }\bibfield  {title} {\bibinfo {title} {Ordering, metastability
			and phase transitions in two-dimensional systems},\ }\href@noop {} {\bibfield
		{journal} {\bibinfo  {journal} {J. Phys. C: Solid State Phys.}\ }\textbf
		{\bibinfo {volume} {6}},\ \bibinfo {pages} {1181} (\bibinfo {year}
		{1973})}\BibitemShut {NoStop}%
	\bibitem [{\citenamefont {Kosterlitz}(1974)}]{Kosterlitz_1974}%
	\BibitemOpen
	\bibfield  {author} {\bibinfo {author} {\bibfnamefont {J.~M.}\ \bibnamefont
			{Kosterlitz}},\ }\bibfield  {title} {\bibinfo {title} {The critical
			properties of the two-dimensional xy-model},\ }\href@noop {} {\bibfield
		{journal} {\bibinfo  {journal} {J. Phys. C: Solid State Phys.}\ }\textbf
		{\bibinfo {volume} {7}},\ \bibinfo {pages} {1046} (\bibinfo {year}
		{1974})}\BibitemShut {NoStop}%
	\bibitem [{\citenamefont {Nelson}\ and\ \citenamefont
		{Kosterlitz}(1977)}]{NelsonKosterlitz1977}%
	\BibitemOpen
	\bibfield  {author} {\bibinfo {author} {\bibfnamefont {D.~R.}\ \bibnamefont
			{Nelson}}\ and\ \bibinfo {author} {\bibfnamefont {J.~M.}\ \bibnamefont
			{Kosterlitz}},\ }\bibfield  {title} {\bibinfo {title} {{Universal Jump in the
				Superfluid Density of Two-Dimensional Superfluids}},\ }\href
	{https://doi.org/10.1103/PhysRevLett.39.1201} {\bibfield  {journal} {\bibinfo
			{journal} {Phys. Rev. Lett.}\ }\textbf {\bibinfo {volume} {39}},\ \bibinfo
		{pages} {1201} (\bibinfo {year} {1977})}\BibitemShut {NoStop}%
	\bibitem [{\citenamefont {McQueeney}\ \emph {et~al.}(1984)\citenamefont
		{McQueeney}, \citenamefont {Agnolet},\ and\ \citenamefont
		{Reppy}}]{McQueeney1984}%
	\BibitemOpen
	\bibfield  {author} {\bibinfo {author} {\bibfnamefont {D.}~\bibnamefont
			{McQueeney}}, \bibinfo {author} {\bibfnamefont {G.}~\bibnamefont {Agnolet}},\
		and\ \bibinfo {author} {\bibfnamefont {J.~D.}\ \bibnamefont {Reppy}},\
	}\bibfield  {title} {\bibinfo {title} {{Surface Superfluidity in Dilute
				$^4$He - $^3$He Mixtures}},\ }\href@noop {} {\bibfield  {journal} {\bibinfo
			{journal} {Phys. Rev. Lett.}\ }\textbf {\bibinfo {volume} {52}},\ \bibinfo
		{pages} {1325} (\bibinfo {year} {1984})}\BibitemShut {NoStop}%
	\bibitem [{\citenamefont {Benfatto}\ \emph {et~al.}(2009)\citenamefont
		{Benfatto}, \citenamefont {Castellani},\ and\ \citenamefont
		{Giamarchi}}]{Benfatto2009}%
	\BibitemOpen
	\bibfield  {author} {\bibinfo {author} {\bibfnamefont {L.}~\bibnamefont
			{Benfatto}}, \bibinfo {author} {\bibfnamefont {C.}~\bibnamefont
			{Castellani}},\ and\ \bibinfo {author} {\bibfnamefont {T.}~\bibnamefont
			{Giamarchi}},\ }\bibfield  {title} {\bibinfo {title} {{Broadening of the
				Berezinskii-Kosterlitz-Thouless superconducting transition by inhomogeneity
				and finite-size effects}},\ }\href
	{https://doi.org/10.1103/PhysRevB.80.214506} {\bibfield  {journal} {\bibinfo
			{journal} {Phys. Rev. B}\ }\textbf {\bibinfo {volume} {80}},\ \bibinfo
		{pages} {214506} (\bibinfo {year} {2009})}\BibitemShut {NoStop}%
	\bibitem [{\citenamefont {A.~T.~Fiory}\ and\ \citenamefont
		{Glaberson}(1983)}]{FioryHebardGlaberson_1983}%
	\BibitemOpen
	\bibfield  {author} {\bibinfo {author}
			\bibnamefont {A.~T.~Fiory}},\  \bibinfo {author} \ {\bibfnamefont {A.~F.~Hebard}\ and\ \bibinfo {author} {\bibfnamefont {W.~I.}\
			\bibnamefont {Glaberson}},\ }\bibfield  {title} {\bibinfo {title}
		{Superconducting phase transitions in indium/indium-oxide thin-film
			composites},\ }\href@noop {} {\bibfield  {journal} {\bibinfo  {journal}
			{Phys. Rev. B}\ }\textbf {\bibinfo {volume} {28}},\ \bibinfo {pages} {5075}
		(\bibinfo {year} {1983})}\BibitemShut {NoStop}%
	\bibitem [{\citenamefont {Yong}\ \emph {et~al.}(2013)\citenamefont {Yong},
		\citenamefont {Lemberger}, \citenamefont {Benfatto}, \citenamefont {Ilin},\
		and\ \citenamefont {Siegel}}]{Yong2013}%
	\BibitemOpen
	\bibfield  {author} {\bibinfo {author} {\bibfnamefont {J.}~\bibnamefont
			{Yong}}, \bibinfo {author} {\bibfnamefont {T.~R.}\ \bibnamefont {Lemberger}},
		\bibinfo {author} {\bibfnamefont {L.}~\bibnamefont {Benfatto}}, \bibinfo
		{author} {\bibfnamefont {K.}~\bibnamefont {Ilin}},\ and\ \bibinfo {author}
		{\bibfnamefont {M.}~\bibnamefont {Siegel}},\ }\bibfield  {title} {\bibinfo
		{title} {{Robustness of the Berezinskii-Kosterlitz-Thouless transition in
				ultrathin NbN films near the superconductor-insulator transition}},\ }\href
	{https://doi.org/10.1103/PhysRevB.87.184505} {\bibfield  {journal} {\bibinfo
			{journal} {Phys. Rev. B}\ }\textbf {\bibinfo {volume} {87}},\ \bibinfo
		{pages} {184505} (\bibinfo {year} {2013})}\BibitemShut {NoStop}%
	\bibitem [{\citenamefont {Venditti}\ \emph {et~al.}(2019)\citenamefont
		{Venditti}, \citenamefont {Biscaras}, \citenamefont {Hurand}, \citenamefont
		{Bergeal}, \citenamefont {Lesueur}, \citenamefont {Dogra}, \citenamefont
		{Budhani}, \citenamefont {Mondal}, \citenamefont {Jesudasan}, \citenamefont
		{Raychaudhuri}, \citenamefont {Caprara},\ and\ \citenamefont
		{Benfatto}}]{Venditti2019}%
	\BibitemOpen
	\bibfield  {author} {\bibinfo {author} {\bibfnamefont {G.}~\bibnamefont
			{Venditti}}, \bibinfo {author} {\bibfnamefont {J.}~\bibnamefont {Biscaras}},
		\bibinfo {author} {\bibfnamefont {S.}~\bibnamefont {Hurand}}, \bibinfo
		{author} {\bibfnamefont {N.}~\bibnamefont {Bergeal}}, \bibinfo {author}
		{\bibfnamefont {J.}~\bibnamefont {Lesueur}}, \bibinfo {author} {\bibfnamefont
			{A.}~\bibnamefont {Dogra}}, \bibinfo {author} {\bibfnamefont {R.~C.}\
			\bibnamefont {Budhani}}, \bibinfo {author} {\bibfnamefont {M.}~\bibnamefont
			{Mondal}}, \bibinfo {author} {\bibfnamefont {J.}~\bibnamefont {Jesudasan}},
		\bibinfo {author} {\bibfnamefont {P.}~\bibnamefont {Raychaudhuri}}, \bibinfo
		{author} {\bibfnamefont {S.}~\bibnamefont {Caprara}},\ and\ \bibinfo {author}
		{\bibfnamefont {L.}~\bibnamefont {Benfatto}},\ }\bibfield  {title} {\bibinfo
		{title} {{Nonlinear $I\text{\ensuremath{-}}V$ characteristics of
				two-dimensional superconductors: Berezinskii-Kosterlitz-Thouless physics
				versus inhomogeneity}},\ }\href {https://doi.org/10.1103/PhysRevB.100.064506}
	{\bibfield  {journal} {\bibinfo  {journal} {Phys. Rev. B}\ }\textbf {\bibinfo
			{volume} {100}},\ \bibinfo {pages} {064506} (\bibinfo {year}
		{2019})}\BibitemShut {NoStop}%
	\bibitem [{\citenamefont {Turneaure}\ \emph {et~al.}(2000)\citenamefont
		{Turneaure}, \citenamefont {Lemberger},\ and\ \citenamefont
		{Graybeal}}]{Turneaure2000}%
	\BibitemOpen
	\bibfield  {author} {\bibinfo {author} {\bibfnamefont {S.~J.}\ \bibnamefont
			{Turneaure}}, \bibinfo {author} {\bibfnamefont {T.~R.}\ \bibnamefont
			{Lemberger}},\ and\ \bibinfo {author} {\bibfnamefont {J.~M.}\ \bibnamefont
			{Graybeal}},\ }\bibfield  {title} {\bibinfo {title} {{Effect of Thermal Phase
				Fluctuations on the Superfluid Density of Two-Dimensional Superconducting
				Films}},\ }\href {https://doi.org/10.1103/PhysRevLett.84.987} {\bibfield
		{journal} {\bibinfo  {journal} {Phys. Rev. Lett.}\ }\textbf {\bibinfo
			{volume} {84}},\ \bibinfo {pages} {987} (\bibinfo {year} {2000})}\BibitemShut
	{NoStop}%
	\bibitem [{\citenamefont {Mondal}\ \emph
		{et~al.}(2011{\natexlab{a}})\citenamefont {Mondal}, \citenamefont {Kumar},
		\citenamefont {Chand}, \citenamefont {Kamlapure}, \citenamefont {Saraswat},
		\citenamefont {Seibold}, \citenamefont {Benfatto},\ and\ \citenamefont
		{Raychaudhuri}}]{Mondal_2011b}%
	\BibitemOpen
	\bibfield  {author} {\bibinfo {author} {\bibfnamefont {M.}~\bibnamefont
			{Mondal}}, \bibinfo {author} {\bibfnamefont {S.}~\bibnamefont {Kumar}},
		\bibinfo {author} {\bibfnamefont {M.}~\bibnamefont {Chand}}, \bibinfo
		{author} {\bibfnamefont {A.}~\bibnamefont {Kamlapure}}, \bibinfo {author}
		{\bibfnamefont {G.}~\bibnamefont {Saraswat}}, \bibinfo {author}
		{\bibfnamefont {G.}~\bibnamefont {Seibold}}, \bibinfo {author} {\bibfnamefont
			{L.}~\bibnamefont {Benfatto}},\ and\ \bibinfo {author} {\bibfnamefont
			{P.}~\bibnamefont {Raychaudhuri}},\ }\bibfield  {title} {\bibinfo {title}
		{{Role of the Vortex-Core Energy on the Berezinskii-Kosterlitz-Thouless
				Transition in Thin Films of NbN}},\ }\href
	{https://doi.org/10.1103/PhysRevLett.107.217003} {\bibfield  {journal}
		{\bibinfo  {journal} {Phys. Rev. Lett.}\ }\textbf {\bibinfo {volume} {107}},\
		\bibinfo {pages} {217003} (\bibinfo {year} {2011}{\natexlab{a}})}\BibitemShut
	{NoStop}%
	\bibitem [{\citenamefont {Ghosal}\ \emph
		{et~al.}(1998{\natexlab{a}})\citenamefont {Ghosal}, \citenamefont
		{Randeria},\ and\ \citenamefont {Trivedi}}]{Ghosal1998}%
	\BibitemOpen
	\bibfield  {author} {\bibinfo {author} {\bibfnamefont {A.}~\bibnamefont
			{Ghosal}}, \bibinfo {author} {\bibfnamefont {M.}~\bibnamefont {Randeria}},\
		and\ \bibinfo {author} {\bibfnamefont {N.}~\bibnamefont {Trivedi}},\
	}\bibfield  {title} {\bibinfo {title} {Role of spatial amplitude fluctuations
			in highly disordered s-wave superconductors},\ }\href@noop {} {\bibfield
		{journal} {\bibinfo  {journal} {Phys. Rev. Lett.}\ }\textbf {\bibinfo
			{volume} {81}},\ \bibinfo {pages} {3940} (\bibinfo {year}
		{1998}{\natexlab{a}})}\BibitemShut {NoStop}%
	\bibitem [{\citenamefont {Ghosal}\ \emph
		{et~al.}(2001{\natexlab{b}})\citenamefont {Ghosal}, \citenamefont
		{Randeria},\ and\ \citenamefont {Trivedi}}]{Ghosal2001b}%
	\BibitemOpen
	\bibfield  {author} {\bibinfo {author} {\bibfnamefont {A.}~\bibnamefont
			{Ghosal}}, \bibinfo {author} {\bibfnamefont {M.}~\bibnamefont {Randeria}},\
		and\ \bibinfo {author} {\bibfnamefont {N.}~\bibnamefont {Trivedi}},\
	}\bibfield  {title} {\bibinfo {title} {Inhomogeneous pairing in highly
			disordered s-wave superconductors},\ }\href@noop {} {\bibfield  {journal}
		{\bibinfo  {journal} {Phys. Rev. B}\ }\textbf {\bibinfo {volume} {65}},\
		\bibinfo {pages} {014501} (\bibinfo {year} {2001}{\natexlab{b}})}\BibitemShut
	{NoStop}%
	\bibitem [{\citenamefont {Sac\'ep\'e}\ \emph {et~al.}(2008)\citenamefont
		{Sac\'ep\'e}, \citenamefont {Chapelier}, \citenamefont {Baturina},
		\citenamefont {Vinokur}, \citenamefont {Baklanov},\ and\ \citenamefont
		{Sanquer}}]{Sacepe2008}%
	\BibitemOpen
	\bibfield  {author} {\bibinfo {author} {\bibfnamefont {B.}~\bibnamefont
			{Sac\'ep\'e}}, \bibinfo {author} {\bibfnamefont {C.}~\bibnamefont
			{Chapelier}}, \bibinfo {author} {\bibfnamefont {T.~I.}\ \bibnamefont
			{Baturina}}, \bibinfo {author} {\bibfnamefont {V.~M.}\ \bibnamefont
			{Vinokur}}, \bibinfo {author} {\bibfnamefont {M.~R.}\ \bibnamefont
			{Baklanov}},\ and\ \bibinfo {author} {\bibfnamefont {M.}~\bibnamefont
			{Sanquer}},\ }\bibfield  {title} {\bibinfo {title} {Disorder-induced
			inhomogeneities of the superconducting state close to the
			superconductor-insulator transition},\ }\href
	{https://doi.org/10.1103/PhysRevLett.101.157006} {\bibfield  {journal}
		{\bibinfo  {journal} {Phys. Rev. Lett.}\ }\textbf {\bibinfo {volume} {101}},\
		\bibinfo {pages} {157006} (\bibinfo {year} {2008})}\BibitemShut {NoStop}%
	\bibitem [{\citenamefont {Carbillet}\ \emph {et~al.}(2016)\citenamefont
		{Carbillet}, \citenamefont {Caprara}, \citenamefont {Grilli}, \citenamefont
		{Brun}, \citenamefont {Cren}, \citenamefont {Debontridder}, \citenamefont
		{Vignolle}, \citenamefont {Tabis}, \citenamefont {Demaille}, \citenamefont
		{Largeau}, \citenamefont {Ilin}, \citenamefont {Siegel}, \citenamefont
		{Roditchev},\ and\ \citenamefont {Leridon}}]{Carbillet2016}%
	\BibitemOpen
	\bibfield  {author} {\bibinfo {author} {\bibfnamefont {C.}~\bibnamefont
			{Carbillet}}, \bibinfo {author} {\bibfnamefont {S.}~\bibnamefont {Caprara}},
		\bibinfo {author} {\bibfnamefont {M.}~\bibnamefont {Grilli}}, \bibinfo
		{author} {\bibfnamefont {C.}~\bibnamefont {Brun}}, \bibinfo {author}
		{\bibfnamefont {T.}~\bibnamefont {Cren}}, \bibinfo {author} {\bibfnamefont
			{F.}~\bibnamefont {Debontridder}}, \bibinfo {author} {\bibfnamefont
			{B.}~\bibnamefont {Vignolle}}, \bibinfo {author} {\bibfnamefont
			{W.}~\bibnamefont {Tabis}}, \bibinfo {author} {\bibfnamefont
			{D.}~\bibnamefont {Demaille}}, \bibinfo {author} {\bibfnamefont
			{L.}~\bibnamefont {Largeau}}, \bibinfo {author} {\bibfnamefont
			{K.}~\bibnamefont {Ilin}}, \bibinfo {author} {\bibfnamefont {M.}~\bibnamefont
			{Siegel}}, \bibinfo {author} {\bibfnamefont {D.}~\bibnamefont {Roditchev}},\
		and\ \bibinfo {author} {\bibfnamefont {B.}~\bibnamefont {Leridon}},\
	}\bibfield  {title} {\bibinfo {title} {Confinement of superconducting
			fluctuations due to emergent electronic inhomogeneities},\ }\href
	{https://doi.org/10.1103/PhysRevB.93.144509} {\bibfield  {journal} {\bibinfo
			{journal} {Phys. Rev. B}\ }\textbf {\bibinfo {volume} {93}},\ \bibinfo
		{pages} {144509} (\bibinfo {year} {2016})}\BibitemShut {NoStop}%
	\bibitem [{\citenamefont {Carbillet}\ \emph {et~al.}(2020)\citenamefont
		{Carbillet}, \citenamefont {Cherkez}, \citenamefont {Skvortsov},
		\citenamefont {Feigel'man}, \citenamefont {Debontridder}, \citenamefont
		{Ioffe}, \citenamefont {Stolyarov}, \citenamefont {Ilin}, \citenamefont
		{Siegel}, \citenamefont {Roditchev}, \citenamefont {Cren},\ and\
		\citenamefont {Brun}}]{Carbillet2020}%
	\BibitemOpen
	\bibfield  {author} {\bibinfo {author} {\bibfnamefont {C.}~\bibnamefont
			{Carbillet}}, \bibinfo {author} {\bibfnamefont {V.}~\bibnamefont {Cherkez}},
		\bibinfo {author} {\bibfnamefont {M.~A.}\ \bibnamefont {Skvortsov}}, \bibinfo
		{author} {\bibfnamefont {M.~V.}\ \bibnamefont {Feigel'man}}, \bibinfo
		{author} {\bibfnamefont {F.}~\bibnamefont {Debontridder}}, \bibinfo {author}
		{\bibfnamefont {L.~B.}\ \bibnamefont {Ioffe}}, \bibinfo {author}
		{\bibfnamefont {V.~S.}\ \bibnamefont {Stolyarov}}, \bibinfo {author}
		{\bibfnamefont {K.}~\bibnamefont {Ilin}}, \bibinfo {author} {\bibfnamefont
			{M.}~\bibnamefont {Siegel}}, \bibinfo {author} {\bibfnamefont
			{D.}~\bibnamefont {Roditchev}}, \bibinfo {author} {\bibfnamefont
			{T.}~\bibnamefont {Cren}},\ and\ \bibinfo {author} {\bibfnamefont
			{C.}~\bibnamefont {Brun}},\ }\bibfield  {title} {\bibinfo {title}
		{Spectroscopic evidence for strong correlations between local superconducting
			gap and local altshuler-aronov density of states suppression in ultrathin nbn
			films},\ }\href {https://doi.org/10.1103/PhysRevB.102.024504} {\bibfield
		{journal} {\bibinfo  {journal} {Phys. Rev. B}\ }\textbf {\bibinfo {volume}
			{102}},\ \bibinfo {pages} {024504} (\bibinfo {year} {2020})}\BibitemShut
	{NoStop}%
	\bibitem [{\citenamefont {Stosiek}\ \emph {et~al.}(2020)\citenamefont
		{Stosiek}, \citenamefont {Lang},\ and\ \citenamefont {Evers}}]{Stosiek2020}%
	\BibitemOpen
	\bibfield  {author} {\bibinfo {author} {\bibfnamefont {M.}~\bibnamefont
			{Stosiek}}, \bibinfo {author} {\bibfnamefont {B.}~\bibnamefont {Lang}},\ and\
		\bibinfo {author} {\bibfnamefont {F.}~\bibnamefont {Evers}},\ }\bibfield
	{title} {\bibinfo {title} {Self-consistent-field ensembles of disordered
			hamiltonians: Efficient solver and application to superconducting films},\
	}\href {https://doi.org/10.1103/PhysRevB.101.144503} {\bibfield  {journal}
		{\bibinfo  {journal} {Phys. Rev. B}\ }\textbf {\bibinfo {volume} {101}},\
		\bibinfo {pages} {144503} (\bibinfo {year} {2020})}\BibitemShut {NoStop}%
	\bibitem [{\citenamefont {Crane}\ \emph {et~al.}(2007)\citenamefont {Crane},
		\citenamefont {Armitage}, \citenamefont {Johansson}, \citenamefont
		{Sambandamurthy}, \citenamefont {Shahar},\ and\ \citenamefont
		{Gr\"uner}}]{Crane2007}%
	\BibitemOpen
	\bibfield  {author} {\bibinfo {author} {\bibfnamefont {R.~W.}\ \bibnamefont
			{Crane}}, \bibinfo {author} {\bibfnamefont {N.~P.}\ \bibnamefont {Armitage}},
		\bibinfo {author} {\bibfnamefont {A.}~\bibnamefont {Johansson}}, \bibinfo
		{author} {\bibfnamefont {G.}~\bibnamefont {Sambandamurthy}}, \bibinfo
		{author} {\bibfnamefont {D.}~\bibnamefont {Shahar}},\ and\ \bibinfo {author}
		{\bibfnamefont {G.}~\bibnamefont {Gr\"uner}},\ }\bibfield  {title} {\bibinfo
		{title} {Fluctuations, dissipation, and nonuniversal superfluid jumps in
			two-dimensional superconductors},\ }\href
	{https://doi.org/10.1103/PhysRevB.75.094506} {\bibfield  {journal} {\bibinfo
			{journal} {Phys. Rev. B}\ }\textbf {\bibinfo {volume} {75}},\ \bibinfo
		{pages} {094506(R)} (\bibinfo {year} {2007})}\BibitemShut {NoStop}%
	\bibitem [{\citenamefont {Mandal}\ \emph {et~al.}(2020)\citenamefont {Mandal},
		\citenamefont {Dutta}, \citenamefont {Basistha}, \citenamefont {Roy},
		\citenamefont {Jesudasan}, \citenamefont {Bagwe}, \citenamefont {Benfatto},
		\citenamefont {Thamizhavel},\ and\ \citenamefont
		{Raychaudhuri}}]{Mandal2020}%
	\BibitemOpen
	\bibfield  {author} {\bibinfo {author} {\bibfnamefont {S.}~\bibnamefont
			{Mandal}}, \bibinfo {author} {\bibfnamefont {S.}~\bibnamefont {Dutta}},
		\bibinfo {author} {\bibfnamefont {S.}~\bibnamefont {Basistha}}, \bibinfo
		{author} {\bibfnamefont {I.}~\bibnamefont {Roy}}, \bibinfo {author}
		{\bibfnamefont {J.}~\bibnamefont {Jesudasan}}, \bibinfo {author}
		{\bibfnamefont {V.}~\bibnamefont {Bagwe}}, \bibinfo {author} {\bibfnamefont
			{L.}~\bibnamefont {Benfatto}}, \bibinfo {author} {\bibfnamefont
			{A.}~\bibnamefont {Thamizhavel}},\ and\ \bibinfo {author} {\bibfnamefont
			{P.}~\bibnamefont {Raychaudhuri}},\ }\bibfield  {title} {\bibinfo {title}
		{{Destruction of superconductivity through phase fluctuations in ultrathin
				$a$-MoGe films}},\ }\href {https://doi.org/10.1103/PhysRevB.102.060501}
	{\bibfield  {journal} {\bibinfo  {journal} {Phys. Rev. B}\ }\textbf {\bibinfo
			{volume} {102}},\ \bibinfo {pages} {060501(R)} (\bibinfo {year}
		{2020})}\BibitemShut {NoStop}%
	\bibitem [{\citenamefont {Broun}\ \emph {et~al.}(2007)\citenamefont {Broun},
		\citenamefont {Huttema}, \citenamefont {Turner}, \citenamefont {\"Ozcan},
		\citenamefont {Morgan}, \citenamefont {Liang}, \citenamefont {Hardy},\ and\
		\citenamefont {Bonn}}]{Broun2007}%
	\BibitemOpen
	\bibfield  {author} {\bibinfo {author} {\bibfnamefont {D.~M.}\ \bibnamefont
			{Broun}}, \bibinfo {author} {\bibfnamefont {W.~A.}\ \bibnamefont {Huttema}},
		\bibinfo {author} {\bibfnamefont {P.~J.}\ \bibnamefont {Turner}}, \bibinfo
		{author} {\bibfnamefont {S.}~\bibnamefont {\"Ozcan}}, \bibinfo {author}
		{\bibfnamefont {B.}~\bibnamefont {Morgan}}, \bibinfo {author} {\bibfnamefont
			{R.}~\bibnamefont {Liang}}, \bibinfo {author} {\bibfnamefont {W.~N.}\
			\bibnamefont {Hardy}},\ and\ \bibinfo {author} {\bibfnamefont {D.~A.}\
			\bibnamefont {Bonn}},\ }\bibfield  {title} {\bibinfo {title} {{Superfluid
				Density in a Highly Underdoped
				${\mathrm{YBa}}_{2}{\mathrm{Cu}}_{3}{\mathrm{O}}_{6+y}$ Superconductor}},\
	}\href {https://doi.org/10.1103/PhysRevLett.99.237003} {\bibfield  {journal}
		{\bibinfo  {journal} {Phys. Rev. Lett.}\ }\textbf {\bibinfo {volume} {99}},\
		\bibinfo {pages} {237003} (\bibinfo {year} {2007})}\BibitemShut {NoStop}%
	\bibitem [{\citenamefont {Kamal}\ \emph {et~al.}(1994)\citenamefont {Kamal},
		\citenamefont {Bonn}, \citenamefont {Goldenfeld}, \citenamefont {Hirschfeld},
		\citenamefont {Liang},\ and\ \citenamefont {Hardy}}]{Kamal1994}%
	\BibitemOpen
	\bibfield  {author} {\bibinfo {author} {\bibfnamefont {S.}~\bibnamefont
			{Kamal}}, \bibinfo {author} {\bibfnamefont {D.~A.}\ \bibnamefont {Bonn}},
		\bibinfo {author} {\bibfnamefont {N.}~\bibnamefont {Goldenfeld}}, \bibinfo
		{author} {\bibfnamefont {P.~J.}\ \bibnamefont {Hirschfeld}}, \bibinfo
		{author} {\bibfnamefont {R.}~\bibnamefont {Liang}},\ and\ \bibinfo {author}
		{\bibfnamefont {W.~N.}\ \bibnamefont {Hardy}},\ }\bibfield  {title} {\bibinfo
		{title} {{Penetration Depth Measurements of 3D $\mathrm{XY}$ Critical
				Behavior in ${\mathrm{YBa}}_{2}$${\mathrm{Cu}}_{3}$${\mathrm{O}}_{6.95}$
				Crystals}},\ }\href {https://doi.org/10.1103/PhysRevLett.73.1845} {\bibfield
		{journal} {\bibinfo  {journal} {Phys. Rev. Lett.}\ }\textbf {\bibinfo
			{volume} {73}},\ \bibinfo {pages} {1845} (\bibinfo {year}
		{1994})}\BibitemShut {NoStop}%
	\bibitem [{\citenamefont {Yong}\ \emph {et~al.}(2012)\citenamefont {Yong},
		\citenamefont {Hinton}, \citenamefont {McCray}, \citenamefont {Randeria},
		\citenamefont {Naamneh}, \citenamefont {Kanigel},\ and\ \citenamefont
		{Lemberger}}]{Yong2012}%
	\BibitemOpen
	\bibfield  {author} {\bibinfo {author} {\bibfnamefont {J.}~\bibnamefont
			{Yong}}, \bibinfo {author} {\bibfnamefont {M.~J.}\ \bibnamefont {Hinton}},
		\bibinfo {author} {\bibfnamefont {A.}~\bibnamefont {McCray}}, \bibinfo
		{author} {\bibfnamefont {M.}~\bibnamefont {Randeria}}, \bibinfo {author}
		{\bibfnamefont {M.}~\bibnamefont {Naamneh}}, \bibinfo {author} {\bibfnamefont
			{A.}~\bibnamefont {Kanigel}},\ and\ \bibinfo {author} {\bibfnamefont {T.~R.}\
			\bibnamefont {Lemberger}},\ }\bibfield  {title} {\bibinfo {title} {{Evidence
				of two-dimensional quantum critical behavior in the superfluid density of
				extremely underdoped Bi${}_{2}$Sr${}_{2}$CaCu${}_{2}$O${}_{8+x}$}},\ }\href
	{https://doi.org/10.1103/PhysRevB.85.180507} {\bibfield  {journal} {\bibinfo
			{journal} {Phys. Rev. B}\ }\textbf {\bibinfo {volume} {85}},\ \bibinfo
		{pages} {180507(R)} (\bibinfo {year} {2012})}\BibitemShut {NoStop}%
	\bibitem [{\citenamefont {Zuev}\ \emph {et~al.}(2005)\citenamefont {Zuev},
		\citenamefont {Seog~Kim},\ and\ \citenamefont {Lemberger}}]{Zuev2005}%
	\BibitemOpen
	\bibfield  {author} {\bibinfo {author} {\bibfnamefont {Y.}~\bibnamefont
			{Zuev}}, \bibinfo {author} {\bibfnamefont {M.~S.}~\bibnamefont {Kim}},\
		and\ \bibinfo {author} {\bibfnamefont {T.~R.}\ \bibnamefont {Lemberger}},\
	}\bibfield  {title} {\bibinfo {title} {{Correlation between Superfluid
				Density and ${T}_{C}$ of Underdoped
				${\mathrm{YBa}}_{2}{\mathrm{Cu}}_{3}{\mathrm{O}}_{6+x}$ Near the
				Superconductor-Insulator Transition}},\ }\href
	{https://doi.org/10.1103/PhysRevLett.95.137002} {\bibfield  {journal}
		{\bibinfo  {journal} {Phys. Rev. Lett.}\ }\textbf {\bibinfo {volume} {95}},\
		\bibinfo {pages} {137002} (\bibinfo {year} {2005})}\BibitemShut {NoStop}%
	\bibitem [{\citenamefont {Maccari}\ \emph {et~al.}(2017)\citenamefont
		{Maccari}, \citenamefont {Benfatto},\ and\ \citenamefont
		{Castellani}}]{Maccari2017}%
	\BibitemOpen
	\bibfield  {author} {\bibinfo {author} {\bibfnamefont {I.}~\bibnamefont
			{Maccari}}, \bibinfo {author} {\bibfnamefont {L.}~\bibnamefont {Benfatto}},\
		and\ \bibinfo {author} {\bibfnamefont {C.}~\bibnamefont {Castellani}},\
	}\bibfield  {title} {\bibinfo {title} {{Broadening of the
				Berezinskii-Kosterlitz-Thouless transition by correlated disorder}},\ }\href
	{https://doi.org/10.1103/PhysRevB.96.060508} {\bibfield  {journal} {\bibinfo
			{journal} {Phys. Rev. B}\ }\textbf {\bibinfo {volume} {96}},\ \bibinfo
		{pages} {060508(R)} (\bibinfo {year} {2017})}\BibitemShut {NoStop}%
	\bibitem [{\citenamefont {Medvedyeva}\ \emph {et~al.}(2000)\citenamefont
		{Medvedyeva}, \citenamefont {Kim},\ and\ \citenamefont
		{Minnhagen}}]{Medveyeva2000}%
	\BibitemOpen
	\bibfield  {author} {\bibinfo {author} {\bibfnamefont {K.}~\bibnamefont
			{Medvedyeva}}, \bibinfo {author} {\bibfnamefont {B.~J.}\ \bibnamefont
			{Kim}},\ and\ \bibinfo {author} {\bibfnamefont {P.}~\bibnamefont
			{Minnhagen}},\ }\bibfield  {title} {\bibinfo {title} {{Analysis of
				current-voltage characteristics of two-dimensional superconductors:
				Finite-size scaling behavior in the vicinity of the Kosterlitz-Thouless
				transition}},\ }\href {https://doi.org/10.1103/PhysRevB.62.14531} {\bibfield
		{journal} {\bibinfo  {journal} {Phys. Rev. B}\ }\textbf {\bibinfo {volume}
			{62}},\ \bibinfo {pages} {14531} (\bibinfo {year} {2000})}\BibitemShut
	{NoStop}%
	\bibitem [{\citenamefont {Ganguly}\ \emph {et~al.}(2015)\citenamefont
		{Ganguly}, \citenamefont {Chaudhuri}, \citenamefont {Raychaudhuri},\ and\
		\citenamefont {Benfatto}}]{Ganguly2015}%
	\BibitemOpen
	\bibfield  {author} {\bibinfo {author} {\bibfnamefont {R.}~\bibnamefont
			{Ganguly}}, \bibinfo {author} {\bibfnamefont {D.}~\bibnamefont {Chaudhuri}},
		\bibinfo {author} {\bibfnamefont {P.}~\bibnamefont {Raychaudhuri}},\ and\
		\bibinfo {author} {\bibfnamefont {L.}~\bibnamefont {Benfatto}},\ }\bibfield
	{title} {\bibinfo {title} {{Slowing down of vortex motion at the
				Berezinskii-Kosterlitz-Thouless transition in ultrathin NbN films}},\ }\href
	{https://doi.org/10.1103/PhysRevB.91.054514} {\bibfield  {journal} {\bibinfo
			{journal} {Phys. Rev. B}\ }\textbf {\bibinfo {volume} {91}},\ \bibinfo
		{pages} {054514} (\bibinfo {year} {2015})}\BibitemShut {NoStop}%
	\bibitem [{\citenamefont {Mallik}\ \emph {et~al.}(2022)\citenamefont {Mallik},
		\citenamefont {Ménard}, \citenamefont {Saïz}, \citenamefont {Witt},
		\citenamefont {Lesueur}, \citenamefont {Gloter}, \citenamefont {Benfatto},
		\citenamefont {Bibes},\ and\ \citenamefont {Bergeal}}]{Mallik2022}%
	\BibitemOpen
	\bibfield  {author} {\bibinfo {author} {\bibfnamefont {S.}~\bibnamefont
			{Mallik}}, \bibinfo {author} {\bibfnamefont {G.}~\bibnamefont {Ménard}},
		\bibinfo {author} {\bibfnamefont {G.}~\bibnamefont {Saïz}}, \bibinfo
		{author} {\bibfnamefont {H.}~\bibnamefont {Witt}}, \bibinfo {author}
		{\bibfnamefont {J.}~\bibnamefont {Lesueur}}, \bibinfo {author} {\bibfnamefont
			{A.}~\bibnamefont {Gloter}}, \bibinfo {author} {\bibfnamefont
			{L.}~\bibnamefont {Benfatto}}, \bibinfo {author} {\bibfnamefont
			{M.}~\bibnamefont {Bibes}},\ and\ \bibinfo {author} {\bibfnamefont
			{N.}~\bibnamefont {Bergeal}},\ }\bibfield  {title} {\bibinfo {title}
		{{Superfluid stiffness of a KTaO$_3$-based two-dimensional electron gas}},\
	}\bibfield  {journal} {\bibinfo  {journal} {arXiv:2204.09094}}
	(\bibinfo {year} {2022})\BibitemShut {NoStop}%
	\bibitem [{\citenamefont {Tamir}\ \emph {et~al.}(2019)\citenamefont {Tamir},
		\citenamefont {Benyamini}, \citenamefont {Telford}, \citenamefont
		{Gorniaczyk}, \citenamefont {Doron}, \citenamefont {Levinson}, \citenamefont
		{Wang}, \citenamefont {Gay}, \citenamefont {Sacepe}, \citenamefont {Hone},
		\citenamefont {Watanabe}, \citenamefont {Taniguchi}, \citenamefont {Dean},
		\citenamefont {Pasupathy},\ and\ \citenamefont {Shahar}}]{Tamir2019}%
	\BibitemOpen
	\bibfield  {author} {\bibinfo {author} {\bibfnamefont {I.}~\bibnamefont
			{Tamir}}, \bibinfo {author} {\bibfnamefont {A.}~\bibnamefont {Benyamini}},
		\bibinfo {author} {\bibfnamefont {E.~J.}\ \bibnamefont {Telford}}, \bibinfo
		{author} {\bibfnamefont {F.}~\bibnamefont {Gorniaczyk}}, \bibinfo {author}
		{\bibfnamefont {A.}~\bibnamefont {Doron}}, \bibinfo {author} {\bibfnamefont
			{T.}~\bibnamefont {Levinson}}, \bibinfo {author} {\bibfnamefont
			{D.}~\bibnamefont {Wang}}, \bibinfo {author} {\bibfnamefont {F.}~\bibnamefont
			{Gay}}, \bibinfo {author} {\bibfnamefont {B.}~\bibnamefont {Sacepe}},
		\bibinfo {author} {\bibfnamefont {J.}~\bibnamefont {Hone}}, \bibinfo {author}
		{\bibfnamefont {K.}~\bibnamefont {Watanabe}}, \bibinfo {author}
		{\bibfnamefont {T.}~\bibnamefont {Taniguchi}}, \bibinfo {author}
		{\bibfnamefont {C.~R.}\ \bibnamefont {Dean}}, \bibinfo {author}
		{\bibfnamefont {A.~N.}\ \bibnamefont {Pasupathy}},\ and\ \bibinfo {author}
		{\bibfnamefont {D.}~\bibnamefont {Shahar}},\ }\bibfield  {title} {\bibinfo
		{title} {Sensitivity of the superconducting state in thin films},\ }\bibfield
	{journal} {\bibinfo  {journal} {{Sci.~Advances}}\ }\textbf {\bibinfo
		{volume} {5}},\ (\bibinfo {year} {2019})\BibitemShut {NoStop}%
	\bibitem [{\citenamefont {Benyamini}\ \emph {et~al.}(2019)\citenamefont
		{Benyamini}, \citenamefont {Telford}, \citenamefont {Kennes}, \citenamefont
		{Wang}, \citenamefont {Williams}, \citenamefont {Watanabe}, \citenamefont
		{Taniguchi}, \citenamefont {Shahar}, \citenamefont {Hone}, \citenamefont
		{Dean}, \citenamefont {Millis},\ and\ \citenamefont
		{Pasupathy}}]{Benyamini2020}%
	\BibitemOpen
	\bibfield  {author} {\bibinfo {author} {\bibfnamefont {A.}~\bibnamefont
			{Benyamini}}, \bibinfo {author} {\bibfnamefont {E.~J.}\ \bibnamefont
			{Telford}}, \bibinfo {author} {\bibfnamefont {D.~M.}\ \bibnamefont {Kennes}},
		\bibinfo {author} {\bibfnamefont {D.}~\bibnamefont {Wang}}, \bibinfo {author}
		{\bibfnamefont {A.}~\bibnamefont {Williams}}, \bibinfo {author}
		{\bibfnamefont {K.}~\bibnamefont {Watanabe}}, \bibinfo {author}
		{\bibfnamefont {T.}~\bibnamefont {Taniguchi}}, \bibinfo {author}
		{\bibfnamefont {D.}~\bibnamefont {Shahar}}, \bibinfo {author} {\bibfnamefont
			{J.}~\bibnamefont {Hone}}, \bibinfo {author} {\bibfnamefont {C.~R.}\
			\bibnamefont {Dean}}, \bibinfo {author} {\bibfnamefont {A.~J.}\ \bibnamefont
			{Millis}},\ and\ \bibinfo {author} {\bibfnamefont {A.~N.}\ \bibnamefont
			{Pasupathy}},\ }\bibfield  {title} {\bibinfo {title} {Fragility of the
			dissipationless state in clean two-dimensional superconductors},\ }\href
	{https://doi.org/10.1038/s41567-019-0571-z} {\bibfield  {journal} {\bibinfo
			{journal} {{Nat.~Physics}}\ }\textbf {\bibinfo {volume} {15}},\ \bibinfo
		{pages} {947} (\bibinfo {year} {2019})}\BibitemShut {NoStop}%
	\bibitem [{\citenamefont {Levinson}\ \emph {et~al.}(2019)\citenamefont
		{Levinson}, \citenamefont {Doron}, \citenamefont {Gorniaczyk},\ and\
		\citenamefont {Shahar}}]{Levinson2019}%
	\BibitemOpen
	\bibfield  {author} {\bibinfo {author} {\bibfnamefont {T.}~\bibnamefont
			{Levinson}}, \bibinfo {author} {\bibfnamefont {A.}~\bibnamefont {Doron}},
		\bibinfo {author} {\bibfnamefont {F.}~\bibnamefont {Gorniaczyk}},\ and\
		\bibinfo {author} {\bibfnamefont {D.}~\bibnamefont {Shahar}},\ }\bibfield
	{title} {\bibinfo {title} {Electron-phonon coupling across the
			superconductor-insulator transition},\ }\href
	{https://doi.org/10.1103/PhysRevB.100.184508} {\bibfield  {journal} {\bibinfo
			{journal} {{Phys.~Rev.~B}}\ }\textbf {\bibinfo {volume} {100}},\ \bibinfo
		{pages} {184508} (\bibinfo {year} {2019})}\BibitemShut {NoStop}%
	\bibitem [{\citenamefont {Linzen}\ \emph {et~al.}(2017)\citenamefont {Linzen},
		\citenamefont {Ziegler}, \citenamefont {Astafiev}, \citenamefont {Schmelz},
		\citenamefont {Hübner}, \citenamefont {Diegel}, \citenamefont {Il’ichev},\
		and\ \citenamefont {Meyer}}]{Linzen_2017}%
	\BibitemOpen
	\bibfield  {author} {\bibinfo {author} {\bibfnamefont {S.}~\bibnamefont
			{Linzen}}, \bibinfo {author} {\bibfnamefont {M.}~\bibnamefont {Ziegler}},
		\bibinfo {author} {\bibfnamefont {O.~V.}\ \bibnamefont {Astafiev}}, \bibinfo
		{author} {\bibfnamefont {M.}~\bibnamefont {Schmelz}}, \bibinfo {author}
		{\bibfnamefont {U.}~\bibnamefont {Hübner}}, \bibinfo {author} {\bibfnamefont
			{M.}~\bibnamefont {Diegel}}, \bibinfo {author} {\bibfnamefont
			{E.}~\bibnamefont {Il’ichev}},\ and\ \bibinfo {author} {\bibfnamefont
			{H.-G.}\ \bibnamefont {Meyer}},\ }\bibfield  {title} {\bibinfo {title}
		{Structural and electrical properties of ultrathin niobium nitride films
			grown by atomic layer deposition},\ }\href
	{https://doi.org/10.1088/1361-6668/aa572a} {\bibfield  {journal} {\bibinfo
			{journal} {Superconductor Science and Technology}\ }\textbf {\bibinfo
			{volume} {30}},\ \bibinfo {pages} {035010} (\bibinfo {year}
		{2017})}\BibitemShut {NoStop}%
	\bibitem [{\citenamefont {Baumgartner}\ \emph {et~al.}(2021)\citenamefont
		{Baumgartner}, \citenamefont {Fuchs}, \citenamefont {Fr\'esz}, \citenamefont
		{Reinhardt}, \citenamefont {Gronin}, \citenamefont {Gardner}, \citenamefont
		{Manfra}, \citenamefont {Paradiso},\ and\ \citenamefont
		{Strunk}}]{Baumgartner_2020}%
	\BibitemOpen
	\bibfield  {author} {\bibinfo {author} {\bibfnamefont {C.}~\bibnamefont
			{Baumgartner}}, \bibinfo {author} {\bibfnamefont {L.}~\bibnamefont {Fuchs}},
		\bibinfo {author} {\bibfnamefont {L.}~\bibnamefont {Fr\'esz}}, \bibinfo
		{author} {\bibfnamefont {S.}~\bibnamefont {Reinhardt}}, \bibinfo {author}
		{\bibfnamefont {S.}~\bibnamefont {Gronin}}, \bibinfo {author} {\bibfnamefont
			{G.~C.}\ \bibnamefont {Gardner}}, \bibinfo {author} {\bibfnamefont {M.~J.}\
			\bibnamefont {Manfra}}, \bibinfo {author} {\bibfnamefont {N.}~\bibnamefont
			{Paradiso}},\ and\ \bibinfo {author} {\bibfnamefont {C.}~\bibnamefont
			{Strunk}},\ }\bibfield  {title} {\bibinfo {title} {{Josephson inductance as a
				probe for highly ballistic semiconductor-superconductor weak links}},\ }\href
	{https://doi.org/10.1103/PhysRevLett.126.037001} {\bibfield  {journal}
		{\bibinfo  {journal} {Phys. Rev. Lett.}\ }\textbf {\bibinfo {volume} {126}},\
		\bibinfo {pages} {037001} (\bibinfo {year} {2021})}\BibitemShut {NoStop}%
	%
		\bibitem [{Sup()}]{Supplement}%
	\BibitemOpen
	\href@noop {} {\bibinfo {title} {{See Supplementary Information for further
				details.}}}\BibitemShut {Stop}%
	%
	\bibitem [{\citenamefont {Mondal}\ \emph
		{et~al.}(2011{\natexlab{b}})\citenamefont {Mondal}, \citenamefont
		{Kamlapure}, \citenamefont {Chand}, \citenamefont {Saraswat}, \citenamefont
		{Kumar}, \citenamefont {Jesudasan}, \citenamefont {Benfatto}, \citenamefont
		{Tripathi},\ and\ \citenamefont {Raychaudhuri}}]{Mondal_2011a}%
	\BibitemOpen
	\bibfield  {author} {\bibinfo {author} {\bibfnamefont {M.}~\bibnamefont
			{Mondal}}, \bibinfo {author} {\bibfnamefont {A.}~\bibnamefont {Kamlapure}},
		\bibinfo {author} {\bibfnamefont {M.}~\bibnamefont {Chand}}, \bibinfo
		{author} {\bibfnamefont {G.}~\bibnamefont {Saraswat}}, \bibinfo {author}
		{\bibfnamefont {S.}~\bibnamefont {Kumar}}, \bibinfo {author} {\bibfnamefont
			{J.}~\bibnamefont {Jesudasan}}, \bibinfo {author} {\bibfnamefont
			{L.}~\bibnamefont {Benfatto}}, \bibinfo {author} {\bibfnamefont
			{V.}~\bibnamefont {Tripathi}},\ and\ \bibinfo {author} {\bibfnamefont
			{P.}~\bibnamefont {Raychaudhuri}},\ }\bibfield  {title} {\bibinfo {title}
		{{Phase Fluctuations in a Strongly Disordered $s$-Wave {NbN} Superconductor
				Close to the Metal-Insulator Transition}},\ }\href
	{https://doi.org/10.1103/PhysRevLett.106.047001} {\bibfield  {journal}
		{\bibinfo  {journal} {Phys. Rev. Lett.}\ }\textbf {\bibinfo {volume} {106}},\
		\bibinfo {pages} {047001} (\bibinfo {year} {2011}{\natexlab{b}})}\BibitemShut
	{NoStop}%
	%
	\bibitem [{\citenamefont {Baturina}\ \emph {et~al.}(2012)\citenamefont
		{Baturina}, \citenamefont {Postolova}, \citenamefont {Mironov}, \citenamefont
		{Glatz}, \citenamefont {Baklanov},\ and\ \citenamefont
		{Vinokur}}]{Baturina_2012}%
	\BibitemOpen
	\bibfield  {author} {\bibinfo {author} {\bibfnamefont {T.~I.}\ \bibnamefont
			{Baturina}}, \bibinfo {author} {\bibfnamefont {S.~V.}\ \bibnamefont
			{Postolova}}, \bibinfo {author} {\bibfnamefont {A.~Y.}\ \bibnamefont
			{Mironov}}, \bibinfo {author} {\bibfnamefont {A.}~\bibnamefont {Glatz}},
		\bibinfo {author} {\bibfnamefont {M.~R.}\ \bibnamefont {Baklanov}},\ and\
		\bibinfo {author} {\bibfnamefont {V.~M.}\ \bibnamefont {Vinokur}},\
	}\bibfield  {title} {\bibinfo {title} {{Superconducting phase transitions in
			ultrathin {TiN} films}},\ }\href {https://doi.org/10.1209/0295-5075/97/17012}
	{\bibfield  {journal} {\bibinfo  {journal} {Europhys.~Lett.}\ }\textbf
		{\bibinfo {volume} {97}},\ \bibinfo {pages} {17012} (\bibinfo {year}
		{2012})}\BibitemShut {NoStop}%
	%
	%
	\bibitem [{\citenamefont {Postolova}\ \emph {et~al.}(2015)\citenamefont
		{Postolova}, \citenamefont {Mironov},\ and\ \citenamefont
		{Baturina}}]{Postolova2015}%
	\BibitemOpen
	\bibfield  {author} {\bibinfo {author} {\bibfnamefont {S.~V.}\ \bibnamefont
			{Postolova}}, \bibinfo {author} {\bibfnamefont {A.~Y.}\ \bibnamefont
			{Mironov}},\ and\ \bibinfo {author} {\bibfnamefont {T.~I.}\ \bibnamefont
			{Baturina}},\ }\bibfield  {title} {\bibinfo {title} {{Nonequilibrium transport
			near the superconducting transition in tin films}},\ }\href
	{https://doi.org/10.1134/S0021364014220135} {\bibfield  {journal} {\bibinfo
			{journal} {JETP Letters}\ }\textbf {\bibinfo {volume} {100}},\ \bibinfo
		{pages} {635} (\bibinfo {year} {2015})}\BibitemShut {NoStop}%
	%
	\bibitem [{\citenamefont {Larkin}\ and\ \citenamefont
		{Varlamov}(2005)}]{LarkinVarlamov2005}%
	\BibitemOpen
	\bibfield  {author} {\bibinfo {author} {\bibfnamefont {A.~I.}\ \bibnamefont
			{Larkin}}\ and\ \bibinfo {author} {\bibfnamefont {A.}~\bibnamefont
			{Varlamov}},\ }\href@noop {} {\emph {\bibinfo {title} {{Theory of
					Fluctuations in Superconductors}}}}\ (\bibinfo  {publisher} {Clarendon Press,
		Oxford},\ \bibinfo {year} {2005})\BibitemShut {NoStop}%
	%	\bibitem{LarkinVarlamov2005} 
	%	A.~Larkin and A.~Varlamov,  Theory of Fluctuations in Superconductors (Oxford: Clarendon Press) (2005)
	%
		\bibitem [{\citenamefont {Halperin}\ and\ \citenamefont
		{Nelson}(1979)}]{HalperinNelson_1979}%
	\BibitemOpen
	\bibfield  {author} {\bibinfo {author} {\bibfnamefont {B.~I.}\ \bibnamefont
			{Halperin}}\ and\ \bibinfo {author} {\bibfnamefont {D.~R.}\ \bibnamefont
			{Nelson}},\ }\bibfield  {title} {\bibinfo {title} {{Resistive Transition in
				Superconducting Films}},\ }\href@noop {} {\bibfield  {journal} {\bibinfo
			{journal} {{J.~Low Temp.~Phys.}}\ }\textbf {\bibinfo {volume} {36}},\
		\bibinfo {pages} {599} (\bibinfo {year} {1979})}\BibitemShut {NoStop}%
	%
	\bibitem [{\citenamefont {Semenov}\ \emph {et~al.}(2009)\citenamefont
		{Semenov}, \citenamefont {G\"unther}, \citenamefont {B\"ottger},
		\citenamefont {H\"ubers}, \citenamefont {Bartolf}, \citenamefont {Engel},
		\citenamefont {Schilling}, \citenamefont {Ilin}, \citenamefont {Siegel},
		\citenamefont {Schneider}, \citenamefont {Gerthsen},\ and\ \citenamefont
		{Gippius}}]{Semenov2009}%
	\BibitemOpen
	\bibfield  {author} {\bibinfo {author} {\bibfnamefont {A.}~\bibnamefont
			{Semenov}}, \bibinfo {author} {\bibfnamefont {B.}~\bibnamefont {G\"unther}},
		\bibinfo {author} {\bibfnamefont {U.}~\bibnamefont {B\"ottger}}, \bibinfo
		{author} {\bibfnamefont {H.-W.}\ \bibnamefont {H\"ubers}}, \bibinfo {author}
		{\bibfnamefont {H.}~\bibnamefont {Bartolf}}, \bibinfo {author} {\bibfnamefont
			{A.}~\bibnamefont {Engel}}, \bibinfo {author} {\bibfnamefont
			{A.}~\bibnamefont {Schilling}}, \bibinfo {author} {\bibfnamefont
			{K.}~\bibnamefont {Ilin}}, \bibinfo {author} {\bibfnamefont {M.}~\bibnamefont
			{Siegel}}, \bibinfo {author} {\bibfnamefont {R.}~\bibnamefont {Schneider}},
		\bibinfo {author} {\bibfnamefont {D.}~\bibnamefont {Gerthsen}},\ and\
		\bibinfo {author} {\bibfnamefont {N.~A.}\ \bibnamefont {Gippius}},\
	}\bibfield  {title} {\bibinfo {title} {{Optical and transport properties of
			ultrathin NbN films and nanostructures}},\ }\href
	{https://doi.org/10.1103/PhysRevB.80.054510} {\bibfield  {journal} {\bibinfo
			{journal} {Phys. Rev. B}\ }\textbf {\bibinfo {volume} {80}},\ \bibinfo
		{pages} {054510} (\bibinfo {year} {2009})}\BibitemShut {NoStop}%
	\bibitem [{\citenamefont {K\"onig}\ \emph {et~al.}(2015)\citenamefont
		{K\"onig}, \citenamefont {Levchenko}, \citenamefont {Protopopov},
		\citenamefont {Gornyi}, \citenamefont {Burmistrov},\ and\ \citenamefont
		{Mirlin}}]{Koenig2015}%
	\BibitemOpen
	\bibfield  {author} {\bibinfo {author} {\bibfnamefont {E.~J.}\ \bibnamefont
			{K\"onig}}, \bibinfo {author} {\bibfnamefont {A.}~\bibnamefont {Levchenko}},
		\bibinfo {author} {\bibfnamefont {I.~V.}\ \bibnamefont {Protopopov}},
		\bibinfo {author} {\bibfnamefont {I.~V.}\ \bibnamefont {Gornyi}}, \bibinfo
		{author} {\bibfnamefont {I.~S.}\ \bibnamefont {Burmistrov}},\ and\ \bibinfo
		{author} {\bibfnamefont {A.~D.}\ \bibnamefont {Mirlin}},\ }\bibfield  {title}
	{\bibinfo {title} {Berezinskii-kosterlitz-thouless transition in
			homogeneously disordered superconducting films},\ }\href
	{https://doi.org/10.1103/PhysRevB.92.214503} {\bibfield  {journal} {\bibinfo
			{journal} {Phys. Rev. B}\ }\textbf {\bibinfo {volume} {92}},\ \bibinfo
		{pages} {214503} (\bibinfo {year} {2015})}\BibitemShut {NoStop}%
	\bibitem [{\citenamefont {Garland}\ and\ \citenamefont
		{Lee}(1987)}]{Garland1987}%
	\BibitemOpen
	\bibfield  {author} {\bibinfo {author} {\bibfnamefont {J.~C.}\ \bibnamefont
			{Garland}}\ and\ \bibinfo {author} {\bibfnamefont {H.~J.}\ \bibnamefont
			{Lee}},\ }\bibfield  {title} {\bibinfo {title} {Influence of a magnetic field
			on the two-dimensional phase transition in thin-film superconductors},\
	}\href {https://doi.org/10.1103/PhysRevB.36.3638} {\bibfield  {journal}
		{\bibinfo  {journal} {Phys. Rev. B}\ }\textbf {\bibinfo {volume} {36}},\
		\bibinfo {pages} {3638} (\bibinfo {year} {1987})}\BibitemShut {NoStop}%
	\bibitem [{\citenamefont {Minnhagen}(1984)}]{Minnhagen1984}%
	\BibitemOpen
	\bibfield  {author} {\bibinfo {author} {\bibfnamefont {P.}~\bibnamefont
			{Minnhagen}},\ }\bibfield  {title} {\bibinfo {title} {Evidence of magnetic
			field scaling for two-dimensional superconductors},\ }\href
	{https://doi.org/10.1103/PhysRevB.29.1440} {\bibfield  {journal} {\bibinfo
			{journal} {Phys. Rev. B}\ }\textbf {\bibinfo {volume} {29}},\ \bibinfo
		{pages} {1440} (\bibinfo {year} {1984})}\BibitemShut {NoStop}%
	\bibitem [{\citenamefont {Tikhonov}\ \emph {et~al.}(2012)\citenamefont
		{Tikhonov}, \citenamefont {Schwiete},\ and\ \citenamefont
		{Finkel'stein}}]{Tikhonov2012}%
	\BibitemOpen
	\bibfield  {author} {\bibinfo {author} {\bibfnamefont {K.~S.}\ \bibnamefont
			{Tikhonov}}, \bibinfo {author} {\bibfnamefont {G.}~\bibnamefont {Schwiete}},\
		and\ \bibinfo {author} {\bibfnamefont {A.~M.}\ \bibnamefont {Finkel'stein}},\
	}\bibfield  {title} {\bibinfo {title} {Fluctuation conductivity in disordered
			superconducting films},\ }\href {https://doi.org/10.1103/PhysRevB.85.174527}
	{\bibfield  {journal} {\bibinfo  {journal} {Phys. Rev. B}\ }\textbf {\bibinfo
			{volume} {85}},\ \bibinfo {pages} {174527} (\bibinfo {year}
		{2012})}\BibitemShut {NoStop}%
	\bibitem [{\citenamefont {Benfatto}\ \emph {et~al.}(2013)\citenamefont
		{Benfatto}, \citenamefont {Castellani},\ and\ \citenamefont
		{Giamarchi}}]{Benfatto_2013}%
	\BibitemOpen
	\bibfield  {author} {\bibinfo {author} {\bibfnamefont {L.}~\bibnamefont
			{Benfatto}}, \bibinfo {author} {\bibfnamefont {C.}~\bibnamefont
			{Castellani}},\ and\ \bibinfo {author} {\bibfnamefont {T.}~\bibnamefont
			{Giamarchi}},\ }\bibfield  {title} {\bibinfo {title}
		{{B}erezinskii–{K}osterlitz–{T}houless transition within the
			sine-{G}ordon approach: The role of the vortex-core energy},\ }\href
	{https://doi.org/10.1142/9789814417648_0005} {\bibfield  {journal} {\bibinfo
			{journal} {40 Years of Berezinskii–Kosterlitz–Thouless Theory}\ ,\
			\bibinfo {pages} {161–199}} (\bibinfo {year} {2013})}\BibitemShut {NoStop}%
	\bibitem [{\citenamefont {Kamlapure}\ \emph {et~al.}(2013)\citenamefont
		{Kamlapure}, \citenamefont {Das}, \citenamefont {Ganguli}, \citenamefont
		{Parmar}, \citenamefont {Bhattacharyya},\ and\ \citenamefont
		{Raychaudhuri}}]{Kamlapure2013}%
	\BibitemOpen
	\bibfield  {author} {\bibinfo {author} {\bibfnamefont {A.}~\bibnamefont
			{Kamlapure}}, \bibinfo {author} {\bibfnamefont {T.}~\bibnamefont {Das}},
		\bibinfo {author} {\bibfnamefont {S.~C.}\ \bibnamefont {Ganguli}}, \bibinfo
		{author} {\bibfnamefont {J.~B.}\ \bibnamefont {Parmar}}, \bibinfo {author}
		{\bibfnamefont {S.}~\bibnamefont {Bhattacharyya}},\ and\ \bibinfo {author}
		{\bibfnamefont {P.}~\bibnamefont {Raychaudhuri}},\ }\bibfield  {title}
	{\bibinfo {title} {Emergence of nanoscale inhomogeneity in the
			superconducting state of a homogeneously disordered conventional
			superconductor},\ }\href {https://doi.org/10.1038/srep02979} {\bibfield
		{journal} {\bibinfo  {journal} {Scientific Reports}\ }\textbf {\bibinfo
			{volume} {3}},\ \bibinfo {pages} {2979} (\bibinfo {year} {2013})}\BibitemShut
	{NoStop}%
	\bibitem [{\citenamefont {Reyren}\ \emph {et~al.}(2007)\citenamefont {Reyren},
		\citenamefont {Thiel}, \citenamefont {Caviglia}, \citenamefont {Kourkoutis},
		\citenamefont {Hammerl}, \citenamefont {Richter}, \citenamefont {Schneider},
		\citenamefont {Kopp}, \citenamefont {Rüetschi}, \citenamefont {Jaccard},
		\citenamefont {Gabay}, \citenamefont {Muller}, \citenamefont {Triscone},\
		and\ \citenamefont {Mannhart}}]{Reyren2009}%
	\BibitemOpen
	\bibfield  {author} {\bibinfo {author} {\bibfnamefont {N.}~\bibnamefont
			{Reyren}}, \bibinfo {author} {\bibfnamefont {S.}~\bibnamefont {Thiel}},
		\bibinfo {author} {\bibfnamefont {A.~D.}\ \bibnamefont {Caviglia}}, \bibinfo
		{author} {\bibfnamefont {L.~F.}\ \bibnamefont {Kourkoutis}}, \bibinfo
		{author} {\bibfnamefont {G.}~\bibnamefont {Hammerl}}, \bibinfo {author}
		{\bibfnamefont {C.}~\bibnamefont {Richter}}, \bibinfo {author} {\bibfnamefont
			{C.~W.}\ \bibnamefont {Schneider}}, \bibinfo {author} {\bibfnamefont
			{T.}~\bibnamefont {Kopp}}, \bibinfo {author} {\bibfnamefont {A.-S.}\
			\bibnamefont {Rüetschi}}, \bibinfo {author} {\bibfnamefont {D.}~\bibnamefont
			{Jaccard}}, \bibinfo {author} {\bibfnamefont {M.}~\bibnamefont {Gabay}},
		\bibinfo {author} {\bibfnamefont {D.~A.}\ \bibnamefont {Muller}}, \bibinfo
		{author} {\bibfnamefont {J.-M.}\ \bibnamefont {Triscone}},\ and\ \bibinfo
		{author} {\bibfnamefont {J.}~\bibnamefont {Mannhart}},\ }\bibfield  {title}
	{\bibinfo {title} {Superconducting interfaces between insulating oxides},\
	} {\bibfield  {journal}
		{\bibinfo  {journal} {Science}\ }\textbf {\bibinfo {volume} {317}},\ \bibinfo
		{pages} {1196} (\bibinfo {year} {2007})}\  \BibitemShut
	{NoStop}%
	\bibitem [{\citenamefont {{\it et al.}}(2014)}]{Zhang_2014}%
	\BibitemOpen
	\bibfield  {author} {\bibinfo {author} {\bibfnamefont {Z.~W.-H.}\
			\bibnamefont {{\it et al.}}},\ }\bibfield  {title} {\bibinfo {title} {Direct
			observation of high-temperature superconductivity in one-unit-cell fese
			films},\ }\href {https://doi.org/10.1088/0256-307X/31/1/017401} {\bibfield
		{journal} {\bibinfo  {journal} {Chinese Physics Letters}\ }\textbf {\bibinfo
			{volume} {31}},\ \bibinfo {pages} {017401} (\bibinfo {year}
		{2014})}\BibitemShut {NoStop}%
	\bibitem [{\citenamefont {Lu}\ \emph {et~al.}(2015)\citenamefont {Lu},
		\citenamefont {Zheliuk}, \citenamefont {Leermakers}, \citenamefont {Yuan},
		\citenamefont {Zeitler}, \citenamefont {Law},\ and\ \citenamefont
		{Ye}}]{Lu_2015}%
	\BibitemOpen
	\bibfield  {author} {\bibinfo {author} {\bibfnamefont {J.~M.}\ \bibnamefont
			{Lu}}, \bibinfo {author} {\bibfnamefont {O.}~\bibnamefont {Zheliuk}},
		\bibinfo {author} {\bibfnamefont {I.}~\bibnamefont {Leermakers}}, \bibinfo
		{author} {\bibfnamefont {N.~F.~Q.}\ \bibnamefont {Yuan}}, \bibinfo {author}
		{\bibfnamefont {U.}~\bibnamefont {Zeitler}}, \bibinfo {author} {\bibfnamefont
			{K.~T.}\ \bibnamefont {Law}},\ and\ \bibinfo {author} {\bibfnamefont {J.~T.}\
			\bibnamefont {Ye}},\ }\bibfield  {title} {\bibinfo {title} {{Evidence for
			two-dimensional Ising superconductivity in gated MoS$_2$}},\ }
	{\bibfield  {journal} {\bibinfo
			{journal} {Science}\ }\textbf {\bibinfo {volume} {350}},\ \bibinfo {pages}
		{1353} (\bibinfo {year} {2015})},\  \BibitemShut
	{NoStop}%
	\bibitem [{\citenamefont {Cao}\ \emph {et~al.}(2018)\citenamefont {Cao},
		\citenamefont {Fatemi}, \citenamefont {Fang}, \citenamefont {Watanabe},
		\citenamefont {Taniguchi}, \citenamefont {Kaxiras},\ and\ \citenamefont
		{Jarillo-Herrero}}]{cao_super_2018}%
	\BibitemOpen
	\bibfield  {author} {\bibinfo {author} {\bibfnamefont {Y.}~\bibnamefont
			{Cao}}, \bibinfo {author} {\bibfnamefont {V.}~\bibnamefont {Fatemi}},
		\bibinfo {author} {\bibfnamefont {S.}~\bibnamefont {Fang}}, \bibinfo {author}
		{\bibfnamefont {K.}~\bibnamefont {Watanabe}}, \bibinfo {author}
		{\bibfnamefont {T.}~\bibnamefont {Taniguchi}}, \bibinfo {author}
		{\bibfnamefont {E.}~\bibnamefont {Kaxiras}},\ and\ \bibinfo {author}
		{\bibfnamefont {P.}~\bibnamefont {Jarillo-Herrero}},\ }\bibfield  {title}
	{\bibinfo {title} {Unconventional superconductivity in magic-angle graphene
			superlattices},\ }\href {https://doi.org/10.1038/nature26160} {\bibfield
		{journal} {\bibinfo  {journal} {Nature}\ }\textbf {\bibinfo {volume} {556}},\
		\bibinfo {pages} {43} (\bibinfo {year} {2018})}\BibitemShut {NoStop}%
	\bibitem [{\citenamefont {Park}\ \emph {et~al.}(2021)\citenamefont {Park},
		\citenamefont {Cao}, \citenamefont {Watanabe}, \citenamefont {Taniguchi},\
		and\ \citenamefont {Jarillo-Herrero}}]{Park_2021}%
	\BibitemOpen
	\bibfield  {author} {\bibinfo {author} {\bibfnamefont {J.~M.}\ \bibnamefont
			{Park}}, \bibinfo {author} {\bibfnamefont {Y.}~\bibnamefont {Cao}}, \bibinfo
		{author} {\bibfnamefont {K.}~\bibnamefont {Watanabe}}, \bibinfo {author}
		{\bibfnamefont {T.}~\bibnamefont {Taniguchi}},\ and\ \bibinfo {author}
		{\bibfnamefont {P.}~\bibnamefont {Jarillo-Herrero}},\ }\bibfield  {title}
	{\bibinfo {title} {Tunable strongly coupled superconductivity in magic-angle
			twisted trilayer graphene},\ }\href
	{https://doi.org/10.1038/s41586-021-03192-0} {\bibfield  {journal} {\bibinfo
			{journal} {Nature}\ }\textbf {\bibinfo {volume} {590}},\ \bibinfo {pages}
		{249} (\bibinfo {year} {2021})}\BibitemShut {NoStop}%
	\bibitem [{\citenamefont {Zhou}\ \emph {et~al.}(2021)\citenamefont {Zhou},
		\citenamefont {Xie}, \citenamefont {Taniguchi}, \citenamefont {Watanabe},\
		and\ \citenamefont {Young}}]{Zhou_Young_2021}%
	\BibitemOpen
	\bibfield  {author} {\bibinfo {author} {\bibfnamefont {H.}~\bibnamefont
			{Zhou}}, \bibinfo {author} {\bibfnamefont {T.}~\bibnamefont {Xie}}, \bibinfo
		{author} {\bibfnamefont {T.}~\bibnamefont {Taniguchi}}, \bibinfo {author}
		{\bibfnamefont {K.}~\bibnamefont {Watanabe}},\ and\ \bibinfo {author}
		{\bibfnamefont {A.~F.}\ \bibnamefont {Young}},\ }\bibfield  {title} {\bibinfo
		{title} {Superconductivity in rhombohedral trilayer graphene},\ }\href
	{https://doi.org/10.1038/s41586-021-03926-0} {\bibfield  {journal} {\bibinfo
			{journal} {Nature}\ }\textbf {\bibinfo {volume} {598}},\ \bibinfo {pages}
		{434} (\bibinfo {year} {2021})}\BibitemShut {NoStop}%
%
	\bibitem [{\citenamefont {Lopes~dos Santos}\ and\ \citenamefont
		{Abrahams}(1985)}]{SantosAbrahams1985}%
	\BibitemOpen
	\bibfield  {author} {\bibinfo {author} {\bibfnamefont {J.~M.~B.}\
			\bibnamefont {Lopes~dos Santos}}\ and\ \bibinfo {author} {\bibfnamefont
			{E.}~\bibnamefont {Abrahams}},\ }\bibfield  {title} {\bibinfo {title}
		{Superconducting fluctuation conductivity in a magnetic field in two
			dimensions},\ }\href {https://doi.org/10.1103/PhysRevB.31.172} {\bibfield
		{journal} {\bibinfo  {journal} {Phys. Rev. B}\ }\textbf {\bibinfo {volume}
			{31}},\ \bibinfo {pages} {172} (\bibinfo {year} {1985})}\BibitemShut
	{NoStop}%
	\bibitem [{\citenamefont {Tinkham}(1996)}]{Tinkham1996}%
	\BibitemOpen
	\bibfield  {author} {\bibinfo {author} {\bibfnamefont {M.}~\bibnamefont
			{Tinkham}},\ }\href@noop {} {\emph {\bibinfo {title} {{Introduction to
					Superconductivity}}}}\ (\bibinfo  {publisher} {McGraw-Hill},\ \bibinfo {year}
	{1996})\BibitemShut {NoStop}%
\end{thebibliography}%


	
	
	\let\oldaddcontentsline\addcontentsline% Store \addcontentsline
	\renewcommand{\addcontentsline}[5]{}
	%% Make \addcontentsline a no-op
	%%\begin{thebibliography}{}%
	%%\bibliographystyle{revtex} % We choose the "plain" reference style
	%
	%%\end{thebibliography}%
	\let\addcontentsline\oldaddcontentsline% Restore \addcontentsline
	
	%----------------------------------------------------------------------------------------

\clearpage
\newpage
\beginsupplement

\begin{center}
	{{\bf\Large SUPPLEMENTARY MATERIAL}\\
		Sharpness of the Berezinskii-Kosterlitz-Thouless transition in ultrathin NbN films\\
		A. Weitzel {\it et al.}\\
		Institut für Experimentelle und Angewandte Physik, University of Regensburg, Regensburg, Germany
	}
\end{center}

\tableofcontents


\begin{figure}[t]
	\includegraphics[width=.35\textwidth]{setup_final.pdf}
	%\includegraphics[width=.3\textwidth]{Grafiken Supplement/C0-Fit_Feb2022.PNG}
	\caption[]{(a) Photograph of the sample discussed in the main text. Green is NbN, blue SiO$_2$/Si. (b)  Photograph of a typical RLC circuit. The brown block is the Tecasinth sample block for loading LCC20 chip carriers. (c) Circuit diagram used for measurements presented in the main text. Input and output of the VNA are connect to ports 1 and 2, the DC current source to port 1.}
	\label{fig:sup:RLC}
\end{figure}


\section{Materials and Methods}

We fabricate our samples from 3 nm thin  NbN films grown by atomic layer deposition (ALD) on top of an amorphous, about 550 nm thick SiO$_2$ layer onto silicon substrate (thermally oxidized silicon wafer with [100] orientation). 
The growth conditions are completely different for ALD and sputtering. The average growth rate is more than one order of magnitude higher in sputtering. Furthermore, the substrate temperatures are about 300 K higher as well as the kinetic energy of the impinging particles (atoms, ions, electrons). The combination leads to a pronounced growth of NbN crystallites with different orientations linked to each other via grain boundaries \cite{Semenov2009,Carbillet2016,Carbillet2020} In contrast, the  layer-by-layer growth of ALD in combination with a low atom mobility on the substrate surface leads to high structural (lateral) homogeneity. Thus, a high density of point defects is accompanied by less pronounced grain boundary formation. %Thus, the thin ALD films do not show a Josephson junction network behavior like in parts observed for sputtered films \cite{Sharma2022}. 
A detailed comparison of the structure for ALD-grown and sputtered ultrathin ($\lesssim$5~nm)  NbN films has not been reported yet and may warrant further study. We believe that such large-scale extrinsic inhomogeneity is much weaker in our ALD-grown films, as compared to epitaxial films. 

A typical sample chip with a 10~\textmu m wide meander and bond wires is shown in Fig.~\ref{fig:sup:RLC}a.
%If sputter deposited films are not homogeneously disordered, but display long-range correlated disorder, the observed broadening of the transition can be explained in terms of a spatial variation of $J_s$ \cite{Benfatto2009, Benfatto_2013}. We believe that such large-scale extrinsic inhomogeneity is much weaker in our ALD-grown films, as compared to epitaxial films. 





$R(T)$ presented in the main text is measured in 4-contact geometry with a standard voltage source (Yokogawa GS 200) and Nanovoltmeter (Agilent 34420A). Each data point corresponds to an $V(I)$ characteristic from which the zero bias resistance is inferred by linear fit. The current is swept in a small range from -10 to +10 nA to minimize heating effects. 
%
$L(T)$ is measured with a vector network analyzer (VNA, Rohde und Schwarz ZNL3) coupled capacitively to high frequency lines in the cryostat. The power $P=-80$ dBm is chosen such that a reduction in power does not affect the quality factor of the circuit. We amplify the signal at room temperature by 56 dB with a Miteq AU 1447 amplifier.
%
Fast $V(I)$ characteristics are measured in 4-contact with a FPGA-base source-measure unit (Nanonis Tramea) with sweep duration of 1-7 s. We measure low voltage regime by using a Femto DLPVA with amplification 80 dB and bandwidth 100 kHz. In parallel, we measure $V(I)$ in the high voltage regime without amplification. Low $V$ and high $V$ regimes overlap over roughly an order of magnitude. All lines are filtered at room temperature by $\pi$ filters with cutoff frequency 100 MHz.  


\begin{figure}[t]
	\includegraphics[width=.5\textwidth]{C0}
	%\includegraphics[width=.3\textwidth]{Grafiken Supplement/C0-Fit_Feb2022.PNG}
	\caption[]{$V^2(f)$ measured above $T_c$ (black dots), together with best fit according to Eqn. \ref{eq:C0} (red line) to determine the capacitance $C_0$ of the LC-circuit. Grey shaded area indicates region used to perform the fit. The downturn at frequencies $\gtrsim 10\,$MHz results from stray capacitance between the pogo pins.}
	\label{fig:sup:C0}
\end{figure}



\section{Circuit Design}
The sample chip is placed in a LCC20 chip carrier and is then mounted into a cold RLC circuit. Figure~\ref{fig:sup:RLC}b shows a photo of a typical set-up, with green PCB-board as well as brown sample holder beneath. The sample holder is made from Tecasinth (polyimide) and holds an array of pogo pins that establish electric and thermal contact to standard LCC20 chip carriers, and via Al bond wires also to the sample on a Si-chip.  To drive and read-out the circuit coaxial cables are needed, which are connected via the SMA-ports on top of the PCB. The other two ports are used to measure the voltage drop over the sample.  
Fig.~\ref{fig:sup:RLC}c shows the corresponding circuit diagram. Four resistors $R_D$ are added to decouple the resonator from the environment, i.e. the lead impedances. For the sample discussed in the main text, a SMD capacitor with capacitance $C_0$ is placed in parallel to the sample with kinetic inductance $L_s$ and resistance $R_s$, comprising the resonator. $R_0$ denotes residual resistance due to contact resistance, bond wires, etc. In early measurements (Fig.~\ref{fig:J_s_200um}) we had 
placed an additional copper coil with resistance $R_c$ and inductance $L_0$  in series with the sample and $R_0$, resulting in a total inductance $L_\mathrm{tot}=L_s+L_0$. 






%
Using the transmission matrix formalism, the element $S_{21}$ of the transmission matrix between terminal 1 and 2 (Fig.~\ref{fig:sup:RLC}) can be calculated:
\begin{equation}
	S_{21}(\omega)= \frac{2Z_lQ}{C_0R_x^2}\frac{1}{\omega_0+2iQ{(\omega-\omega_0)}}\;.
\end{equation}
Therefore, the measured signal can be described by
\begin{equation}
	V^2(f)=A\cdot\bigg|\frac{Z_lQ}{\pi C_0R_x^2}\frac{1}{f_0+2iQ({f-f_0})}\bigg|^2\;,
	\label{eq:sup:V^2(f)}
\end{equation}
where $V$ is the voltage at the input of the network analyzer, $f$ is the frequency, $A$ is a scaling parameter containing the drive level, $Q$ is the quality factor, $R_x=R_D+Z_l$ with $R_D=992\,\Omega$ being decoupling resistors that separate the circuit from lead impedances and $Z_l=50\Omega$ is the impedance of the cables.
Changes of resonance frequency can be directly connected to changes of the sample's kinetic inductance.

%\begin{figure}[h]
%	\includegraphics[width=.3\textwidth]{Grafiken Supplement/paper_setup.png}
%	\caption[]{Schematic setup of resonance measurements. The cold circuit with sample marked in blue is connected to VNA by coaxial cables, while other two DC ports are left floating}
%	\label{fig:sup:setup}
%\end{figure}

%\FloatBarrier

\section{Circuit Calibration}
We determine $C_0$, the capacitance, by heating the circuit with sample to a temperature above $T_c$, such that the resistance of the sample is on the order of hundreds of k$\Omega$. In this regime, the circuit can be effectively modeled as a low-pass filter, with transfer matrix element:
%
\begin{equation}
	%V^2(f)=A\cdot\bigg(\frac{Z_l}{R_x+iR_x^2(\pi C_0 f)}\bigg)^2
	V^2(f)=A\cdot\bigg|\frac{Z_l}{R_x+i\pi C_0 fR_x^2}\bigg|^2~\;.
	\label{eq:C0}
\end{equation}
%
Figure~\ref{fig:sup:C0} shows measured $V^2(f)$ together with the best fit according to Eq.~\ref{eq:C0}, giving $C_0=10.34$ nF, close to the nominal value of the capacitors ($10$~nF). Grey shaded area in Fig.~\ref{fig:sup:C0} indicates the region used for fitting the data to Eq.~\ref{eq:C0}. It is chosen as the largest region where low-pass behavior is clearly discernible. Variation of the fitting region to higher frequency leads to a change in $C_0$ of $\sim2$ \%, which determines the uncertainty in $J_s$ from $C_0$ calibration to the same value of $\sim2$ \%. %On the other hand, including the first point at low frequency in the fit leads to $C_0=8.15$ nF, smaller than the nominal capacitance of the capacitor and thus likely spurious.



\begin{figure}[t]
	\includegraphics[width=.5\textwidth]{Example_spectra.pdf}
	%\includegraphics[width=.3\textwidth]{Grafiken Supplement/C0-Fit_Feb2022.PNG}
	\caption[]{Three exemplary spectra, far away from $T_\mathrm{BKT}$ (blue), close to (orange) and at $T_\mathrm{BKT}$ (red). Solid lines are fits to Eq.~\ref{eq:sup:V^2(f)}. Corresponding $Q$-factors are 35 (blue), 3 (orange), 1.5 (red).}
	\label{fig:sup:spectra}
\end{figure}

For resonators with an additional inductor in series with the sample, we determine its inductance $L_0$ by replacing the sample with a straight bond wire of negligible resistance (50~m$\Omega$) and inductance ($\simeq1$~nH). From the resulting resonance frequency $f_{00}$, we  determine 
\begin{equation}
	L_0=\frac{1}{(2\pi f_{00})^2C_0}
	%L_0=\frac{1}{4\pi^2}\frac{1}{C_0f_{00}^2}
	\label{eq:L0}
\end{equation}
and find typical values around 220~nH.

\section{Exemplary Spectra}

We present some exemplary spectra in Fig.~\ref{fig:sup:spectra}. At $T_\mathrm{BKT}=4.488$ K a resonance is still clearly resolvable, albeit with a strongly suppressed $Q$ factor compared to low temperatures.  $Q(T)$ close to $T_\mathrm{BKT}$ is shown Fig.~\ref{fig:sup:Q(T)}. About 0.5~K below $T_\mathrm{BKT}$, $Q(T)$ decreases rather linearly towards $T_\mathrm{BKT}$. A possible explanation for the decrease is that the oscillatory motion of the increasing number of thermally excited, but still bound, vortex-antivortex pairs around their equilibrium distance gives rise to an additional dissipation channel.

\section{Quality Factor}
%When analyzing the circuit of Fig.~\ref{fig:sup:RLC}, an inner and an outer circuit can be distinguished. The quality factor Q of the resonance curve can thus be defined as
The quality factor of a resonance curve is defined as the ratio of resonance frequency over the full width at half maximum (FWHM). For the circuit displayed in Fig.~\ref{fig:sup:RLC}, $Q$ can be written as the combination of the circuit's internal and external quality factors $Q_e$ and $Q_i$:
\begin{equation}
	Q=\frac{f_0}{\Delta f_\mathrm{FWHM}}
	%=\bigg[ \frac{1}{Q_i}+\frac{2}{C_0\omega_0(R_D+Z_l)} \bigg]^{-1}
	=\bigg[ \frac{1}{Q_i}+\frac{1}{Q_e} \bigg]^{-1}\;.
\end{equation}
where 
\begin{equation}
	Q_i = \frac{1}{2\pi f_0C_0 R_s}\quad \text{and}\quad Q_e=\frac{\omega_0C_0(R_D+Z_l)}{2}\;.
\end{equation}
%
%the latter is defined by the built-in capacitance, decoupling resistance and the resonance frequency.
%
%The internal quality factor $Q_i$ is proportional to the circuit resistance $R=R_s+R_0$ via
%\begin{equation}
%   Q_i
%=\omega_0 C_0 R_{eq}
%=2\pi f_0C_0 R_{eq}
%=2\pi f_0 \frac{L}{R_s}
%=\frac{1}{\omega_0 L} \frac{L}{R_sC_0}
%=\frac{1}{\omega_0 R_sC_0}
%   =\frac{1}{2\pi f_0C_0 R}
%\end{equation}
If $C_0$ as well as $Z_l$ and  $R_0$ are known, small sample resistances $R_s$ can be determined from the internal quality factor. 


\begin{figure}[t]
	\includegraphics[width=.5\textwidth]{QT.pdf}
	%\includegraphics[width=.3\textwidth]{Grafiken Supplement/C0-Fit_Feb2022.PNG}
	\caption[]{Quality factor  as function of temperature close to $T_\mathrm{BKT}$.}
	\label{fig:sup:Q(T)}
\end{figure}



\begin{figure}[b]
	\includegraphics[width=.5\textwidth]{QB.pdf}
	\caption[]{Quality factor  as function of magnetic field for the device discussed in the main text. Solid lines are guides to the eye. Arrow indicates the optimal compensation field as determined from the data.}
	\label{fig:sup:Q(B)}
\end{figure}

\section{Magnetic Field Compensation}
By optimization of the quality factor $Q(B)$, the perpendicular magnetic field can be compensated down to a few \textmu T. In Fig.~\ref{fig:sup:Q(B)} $Q(B)$ shows a sharp maximum with a width of $\simeq~10\,$\textmu T. The precise location of the maximum is obtained from the linear extrapolation of $Q(B)$ from both sides of the maximum. Negligible hysteresis was observed for  up- and down-sweeps. For the experiments presented in the main text, we chose $B_{\mathrm{comp}}=238$~\textmu T to compensate 
residual fields, indicated by an arrow in Fig.~\ref{fig:sup:Q(B)}. Experience shows that after heating the solenoid above $T_c$  strongly improves the stability of the residual field such that compensation is reliable over several days. All experiments in zero field presented in the main text were performed within six days after field compensation.



\section{Fluctuation Contributions to $R(T)$ }
\label{sec:qcc}
Our fit of the intrinsically broadened $R(T)$-curve consists of several contributions (see \cite{LarkinVarlamov2005,Baturina_2012,Postolova2015} and the references therein):
\begin{equation}
	R(T)=\bigg[ \frac{1}{R_N} +\Delta G_{AL} +\Delta G_{MT} +\Delta G_{WL} +\Delta G_{ID} \bigg]^{-1}
\end{equation}
%
Besides the normal-state resistance  $R_N$, the Aslamazov-Larkin (AL) contribution $\Delta G_{AL}$ describes the fluctuation of Cooper pairs above $T_\mathrm{c0}$
\begin{align}
	\Delta G_{AL} &= G_{00}\cdot 
	\frac{\pi^2}{8\,\ln(T/T_\mathrm{c0})}\;.
\end{align}
Fluctuating Cooper pairs also affect the diffusion coefficient of unpaired electrons due to the interaction with the Cooper pairs. This included as $\Delta G_{MT}$, named after Maki and Thompson
\begin{align}
	\Delta G_{MT} &= G_{00}\cdot
	\beta\big( T/T_\mathrm{c0}, \delta\big)\cdot
	\ln{ \left( \frac{\ln(T/T_\mathrm{c0})}{\delta} \right) }\;,
\end{align}
where $G_{00}$ is the conductance quantum, while the Larkin function $\beta$ and the Maki-Thompson pair breaking parameter $\delta$ are given by
\begin{align}
	\delta &= \frac{\pi \hbar}{8k_\mathrm{B} T \tau_\phi}\qquad\mathrm{and,}\\
	\beta\big( T/T_\mathrm{c0}, \delta\big) &\approx \frac{\pi^2}{4}\ \frac{1}{\ln(T/T_\mathrm{c0})-\delta}
	\label{eq:supp:beta}
\end{align}
%
where the approximate form of $\beta(T)$ in Eqn. \ref{eq:supp:beta} is valid in the limit $\ln(T/T_\mathrm{c0})\ll 1$ \cite{SantosAbrahams1985}.
Finally, the normal state weak localization and interaction corrections are responsible for the resistance maximum above $T_\mathrm{c0}$ and read:
\begin{align}
	\Delta G_{WL}+\Delta G_{ID} &= G_{00}\cdot
	A \cdot \ln \left( \frac{k_\mathrm{B} T\tau}{\hbar} \right) \;.
\end{align}
As some of the five parameters $T_\mathrm{c0}$, A, $\delta$, $\tau_\phi$ and $R_N$, are determined independently, essentially only three of them ($T_\mathrm{c0}, A, \delta$), are adjusted freely. Due to the divergence of the fluctuation corrections, the fit vanishes at  $T_\mathrm{c0}$.

\begin{figure}[t]
	\includegraphics[width=.5\textwidth]{Param_var.pdf}
	\caption[]{Comparison of one, two and three parameter fits of Eq.~\ref{eq:supp:J_BCS} to $J_s(T)$ data presented in the main text. Three free parameters $T_\mathrm{c0}$, $\beta$, $\gamma$ are required to obtain agreement between data and fit, see text.}
	\label{fig:sup:J_s_param_var}
\end{figure}

\section{Dirty limit BCS-fit to $J_s(T)$}
%
We fit $J_s(T)$ data using
%\begin{subequations}
\begin{align}
	J_s(T)\ &=\  
	%\pi\hbar\Delta(T)/(4e^2k_\mathrm{B} R_N)\cdot
	J_s(0)\cdot\frac{\Delta(T)}{\Delta(0)}\cdot\tanh\bigg(\frac{\Delta(T)}{2k_\mathrm{B} T}\bigg)\;,%\\
	%J_s^\mathrm{BCS}(0)\&=\ %\beta\cdot\\\
	%\Delta(0)\ &=\ \gamma\cdot1.76k_\mathrm{B} T_\mathrm{c0}
	\label{eq:supp:J_BCS}
\end{align}
%\end{subequations}
%
where $J_s(0) =\beta\cdot J^\mathrm{BCS}_s(0)$. Here, $\beta$ and $\gamma$ are freely adjustable parameters, $J^\mathrm{BCS}_s(0)=\pi\hbar\Delta(T)/(4e^2k_\mathrm{B} R_N)$ is the dirty-limit BCS-expression for $J_S(0)$ \cite{Tinkham1996}, 
$\Delta(0) = \gamma\cdot 1.764\,k_\mathrm{B} T_\mathrm{c0}$,  and $R_N$ being the normal state resistance determined near the maximum in $R(T)$ near 12\,K. As in Refs.~\cite{Yong2013,Mondal_2011b}, we need three free parameters $T_\mathrm{c0}$, $\beta$, and $\gamma$ to achieve a good fit.  Since $\Delta(0)$ also affects the shape of the curve via argument of $\tanh[\Delta(T)/2k_\mathrm{B} T]$ in Eq.~\ref{eq:supp:J_BCS}, independent variation $\beta$ and $\gamma$ is needed to reproduce the data (purple curve). This becomes evident from fits with one and two free parameters in Fig.~\ref{fig:sup:J_s_param_var}. The one-parameter (red) and two-parameter (blue and green) fits fail to reproduce the curvature of $J_s(T)$ as well as the absolute values of $J_s(0)$ and the independently measured value of $T_\mathrm{c0}$.




\section{Magnetoresistance}

\begin{figure}[t]
	\includegraphics[width=.5\textwidth]{RB.pdf}
	\caption[]{Magnetoresistance isotherms. Solid line corresponds to $R(B)=R_t$, see text. Dashed line corresponds to $R=R_N$. Black star denotes $R(B)=0.46R_N$ corresponding to the magnetic field at which scaling is obtained of the  low-field resistance data (Fig.~\ref{fig:R(B)}) in the main text.}
	\label{fig:sup:R(B)}
\end{figure}


Figure~\ref{fig:sup:R(B)} shows magnetoresistance isotherms measured in a  Nb$_3$Sn solenoid  up to 9\,T.  Determination of $B_{c2}(T)$ is difficult because the jump in $R(B)$ is smeared out over a field range of more than 10~T. As criterion for $B_{c2}(T)$, we defined a threshold resistance $R_t=R(B=B_{c2})=0.22R_N$ (solid horizontal line in \ref{fig:sup:R(B)}) such that $B_\mathrm{c2}(T)$ extrapolates to $T_\mathrm{c0}$ (red dots). Note that the value of $B_{c2}(T_\mathrm{BKT})$ extracted from the scaling of the low-field magnetoresistance in Fig.~\ref{fig:R(B)} corresponds to a resistance of $0.46R_N$ and is thus much closer to the standard criterion $0.5R_N$. The latter criterion, however, fails close to $T_\mathrm{c0}$ because the corresponding curve (blue dots) extratolates to much to high temperatures.  From $B_{c2}(T)$ we estimate $\xi_\mathrm{GL}(T)$ via
%
%\begin{align}
%B_{c2}(T)=\frac{\Phi_0}{2\pi\xi_\mathrm{GL}^2(T)}
%\label{eq:sup:Bc(T)}
%\end{align}
%
%
\begin{align}
	\xi_\mathrm{GL}(T)=\frac{\xi_\mathrm{GL}(0)}{(1-T/T_\mathrm{c0})^{1/2}}=\sqrt{\frac{\Phi_0}{2\pi B_{c2}(T)}}\;.
	\label{eq:sup:xi(T)}
\end{align}
with $\xi_\mathrm{GL}(0)=4.26$ nm. 

%\textcolor{red}{xi, lambda, Bc, etc.}


%\begin{figure}[h!]
%	\includegraphics[width=.4\textwidth]{Grafiken Supplement/R_B_paper_22-9-14.png}
%	\caption[]{Magnetoresistance curves of a 40µm wide Hallbar.}
%	\label{fig:sup:R_highB}
%\end{figure}




\begin{figure}[h!]
	\includegraphics[width=.45\textwidth]{B_cT.pdf}
	\caption[]{Upper critical field $B_{c2}(T)$ according to the criteria  $R_t=0.22R_N$ (red) and $R_t=0.5R_N$ (blue), determined from magnetoresistance isotherms (see text).  The red dashed line is linear fit to $B_{c2}(T)$ close to $T_\mathrm{c0}$. The vertical dashed line indicates $T_\mathrm{c0}=5.188$ K, determined via BCS fit to $J_s(T)$.}
	\label{fig:sup:bc}
\end{figure}

%The critical magnetic field is taken at half of the room temperature resistance and its temperature dependence is shown in Fig \ref{fig:sup:bc}.




\begin{figure}[t]
	\includegraphics[width=.48\textwidth]{IV_supp_var.pdf}
	\caption[]{$V(I)$ characteristics in the immediate vicinity of $T_\mathrm{BKT}$ measured on the device discussed in the main text. The dot colors indicate both the temperature and the measurement scheme [fast (7 seconds) vs.~slow (10 minutes per sweep)]. Lines correspond to fits using Eq.~\ref{eq:HN-IV} with $\alpha=1$ (dotted) and $\alpha=2.3, 3.2,  3.5, 3.9, 4.3$ (dashed, from  top to bottom). Solid line indicates $V=R_N\cdot I$, where $R_N$ is the normal state resistance. Black arrow points out beginning deviations from HN-behavior due to heating. Grey shaded area displays the fitting range for the extraction of the power law exponent $\alpha(T)$. Grey arrow points at beginning heating of the sample chip in slow sweeps.}
	\label{fig:sup:IV_fast_v_slow}
\end{figure}

\section{$V(I)$-Characteristics}


As discussed in the main text, the superfluid stiffness  $J_s(T)$ can also be extracted from the power-law exponent $\alpha(T)$ of the $V(I)$ characteristics using Halperin-Nelson theory \cite{HalperinNelson_1979} (Figs. \ref{fig:IV} and \ref{fig:L(T)}). Fig.~\ref{fig:sup:IV_fast_v_slow} shows typical $V(I)$ characteristics in a very narrow temperature regime $0.92~T_\mathrm{BKT}<T<1.01~T_\mathrm{BKT}$. In this regime, we employed two different measurement schemes to evaluate the importance of heating effects: Blue curves are fast sweeps of duration 1-7~s, to minimize heating and measurement time. Measurements shown in Fig.~\ref{fig:L(T)} in the main text were performed in the fast scheme. We determine power-law exponents from fast sweeps (red dots in Fig. ~\ref{fig:L(T)}) in the main text (Fig.~\ref{fig:L(T)}) by fitting the data in a region indicated by the shaded grey area to a power law  with exponent~$\alpha(T)$.   Red curves are slow sweeps (duration: ~10 minutes) using a nanovoltmeter. 

We emphasize the importance of extracting power-law exponents at the lowest possible power regime ($I\rightarrow0$, $V\rightarrow0$), to minimize effects of electron heating, which can alter the shape of the $V(I)$-curve. For both the fast and slow measurement scheme, such effects appear  in Fig.~\ref{fig:sup:IV_fast_v_slow} alredy at power levels of a few picowatts, where data start to deviate from power-law behavior (black arrow), which we attribute to electron overheating \cite{Levinson2019}. At much higher  power $P\sim2~\mu$W (grey arrow) we observe a clear divergence of the slowly measured curves (red) from the fast measured curves (blue). Analysis of data in both electron- and chip-heating regimes will likely result in erroneous values of $\alpha(T)$. Heating effects that push the film towards the normal state can be approximated by power law $V(I)$-characteristics in limited $T$-intervals.  A typical signature of such analysis is a wide temperature range in which $1\leq\alpha\leq3$ corresponding to the total width of the fluctuation regime between $T_\mathrm{BKT}$ and the temperatures where $R(T)$ approaches $R_N$. As indicated by the grey shaded area in Fig.~\ref{fig:sup:IV_fast_v_slow}), the  range of $V(I)$ that is governed by the current-induced vortex-anti-vortex depairing is rather small

\begin{figure}[t]
	\includegraphics[width=.48\textwidth]{I_c.pdf}% Here is how to import EPS art
	\caption[critcur]{Characteristic currents of our films: HN-scaling current $	I_{c}^\mathrm{GL}$ from Eq.~\ref{eq:I_0_IV} (red dots), plotted together with the critical current $I_c^\mathrm{GL}(T)$ according to Eq.~\ref{eq:sup:GL Ic} (black dots), and the standard Ginzburg-Landau critical current $I_c^\mathrm{BCS}$ according to the BCS-extrapolation of $J_s(T)$ towards $T_\mathrm{c0}$ (black line).}
	\label{fig:currents}
\end{figure}


% At voltage levels between  0.01 and 1\,mV, power and $T<T_\mathrm{BKT}$ power law exponents are observed in good agreement with the direct measurement of $J_S(T)$. 
Above 1\,mV, corresponding to power levels of 10\,pW/square in our devices, $V(I)$-gradually starts to bend upward because of electron heating. In a limited voltage range also these $V(I)$-characteristics can mimick power-law behavior, leading to an apparently broadened transition. At even higher voltage levels $\gtrsim 100\,$mV, or power levels $\gtrsim 2$~\textmu W in the whole meander, heating of the sample stage becomes noticable. This leads to heating instabilities and back-bending of the $V(I)$-characteristics. 


Slightly below $T_\mathrm{BKT}$ we observe ohmic tails in the slow measurement at low voltage $V<10^{-6}$ V (red and dark red curve in \ref{fig:sup:IV_fast_v_slow}), that are not discernible in the fast sweeps. As pointed out in \cite{Tamir2019, Benyamini2020}, these tails can result from current noise due to insufficient filtering even at $^4$He-temperatures. In our set-up, we use $\pi$-filters with cutoff frequency 100~MHz for all measurement leads. For the slow measurement rounding towards ohmic behavior occurs at slightly higher voltages when compared to the fast measurement. A clarification of this effect requires further study.


\section{Characteristic Currents}
Another interesting comparison can be performed between the Ginzburg-Landau (GL) critical current $I_{c}^\mathrm{GL}$  and the Halperin-Nelson (HN) scaling current $I_0$ entering the prefactor $A(T)$ in Eq.~\ref{eq:HN-IV}. Withing the error margins of $\xi_\mathrm{GL}$ and $J_s(T)\propto 1/\lambda^2$ we can estimate $I_{c}^\mathrm{GL}$ via
%
\begin{equation}
	I_{c}^\mathrm{GL}(T)\ =\ \frac{\hbar}{3\sqrt{3}e\mu_0}\frac{wd}{\lambda^2(T)\xi(T)}=\frac{\Phi_0}{3\sqrt{3}\pi}\frac{w}{L_\mathrm{kin}(T)\xi(T)}\;.
	\label{eq:sup:GL Ic}
\end{equation}
%
The error margin is mainly set by the uncertainty of $\xi_\mathrm{GL}$, which is hard to determine reliably from the broad magnetoresistance curves.
The scaling currrent $I_0^\mathrm{exp}(T)$ is extracted  from the independent measurement of $V(I)$. On the other hand, the full expression for the $V(I)$-characteristics within 
%
HN theory reads	 \cite{HalperinNelson_1979}:
\begin{align}
	V(I,T)\ =\ I\cdot R_N\cdot[2 \pi J_s(T)/T-4]\cdot[{I}/{I_0(T)}]^{\pi J_s(T)/T}\,.
\end{align}
%
From this equation we infer the HN scaling current as:
%
\begin{equation}
	I_0^\mathrm{exp}(T)\ =\ [R_N({2 \alpha(T)-6})/{A(T)}]^{1/[\alpha(T)-1]}
	\label{eq:I_0_IV}
\end{equation}
%
where $A(T)$ and $\alpha(T)$ are fit parameters in the fit of double-logarithmic $V(I)$, see also Eq.~\ref{eq:HN-IV} in the main text. 




Note  that the definition of $I_0$, given in Ref. \cite{HalperinNelson_1979}
%
\begin{align}
	I_0^{\mathrm{theo}}(T)\ =\ wek_\mathrm{B} T_\mathrm{BKT}/(\hbar\xi(T))
	\label{eq:I_0_theo}
\end{align}
%
is valid only very near $T_\mathrm{BKT}$.  %an approximation of $I_{c}^\mathrm{GL}(T)$ (Eq.~\ref{eq:sup:GL Ic}) close to $T_\mathrm{BKT}$, 
%
In order to slightly generalize Eq.~\ref{eq:I_0_theo} we replace $T_\mathrm{BKT}$ by   $\pi J_s(T\lesssim T_\mathrm{BKT})/2$ and use the relation $J_s(T)\propto1/\lambda^2(T)$ (Eq.~\ref{eq:J_S}), we find that an expression for $I_0(T)$  that reads identical to the GL-critical current (Eq.~\ref{eq:sup:GL Ic}), the only difference being that $1/\lambda^2(T)$ can now be taken form the measurement of $J_s(T)$ rather than assuming the standard GL-form $\lambda(T)=\lambda(0)/\sqrt{1-T/T_\mathrm{c0}}$ in Eq.~\ref{eq:sup:GL Ic}.

When plotting the so-obtained data for $I_c^\mathrm{GL}(T)$ in Fig.~\ref{fig:currents} together with the experimentally determined HN-scaling current $I_0^\mathrm{exp}(T)$ we find a fair agreement. A more accurate derivation of Eq.~\ref{eq:I_0_theo} may provide quantitative understanding also of the prefactor $A(T)$ in Eq.~\ref{eq:HN-IV} in the main text.



%
%
%\section{Other Samples}
%\label{sec:othersamples}
%
%
%We have performed measurements of $J_s(T)$ on several similar devices made from the same wafer over a period of one year. All devices show a sharp jump of $J_s(T)$ near their intersection with the BKT universal line. They differ in width, while the number of squares was kept around 100, except for the 10~\textmu m wide device, where if was 200. There is no systematic dependence on the width; the variations result from different oxidation states of the film. Over the course of several months the films gradually increase in normal state resistance. The measurements were performed in two different resonators with resonance frequencies between 4 and 50~MHz.
%
%\begin{figure}[t]
%	\includegraphics[width=.5\textwidth]{J_supp}% Here is how to import EPS art
%	\caption[200um]{Superfluid stiffness of four NbN meanders with similar number of squares and different width. The black dashed line is the universal transition line. The small differences in $T_\mathrm{c0}$ and $J_s(0)$ are caused by different levels of oxidation of the films in air that occurs on the scale of several months. One of the devices has been measured in a resonator with higher frequency  (blue dots, 30 MHz).}
%	\label{fig:J_s_200um}
%\end{figure}



%\fi


%---------------------------




\end{document}
%
% ****** End of file apssamp.tex ******






