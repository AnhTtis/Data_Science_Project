\section{Preliminaries}\label{secPrelim}

\subsection{Orders of growth and generalized entropy}\label{subOGGE}


Let us briefly recall how the complete set of the orders of growth and the generalized entropy of a map are defined in \cite{CoPu21}. First, we consider the space of non-decreasing sequences in $[0,\infty)$: $$\mathcal{O}=\{a:\mathbb{N}\rightarrow [0,\infty):a(n)\leq a(n+1),\, \forall n\in \mathbb{N}\}.$$
Next, we define the equivalence relationship $\approx$ in $\OR$ by $a(n)\approx b(n)$ if and only if there exist $c_1,c_2\in (0,\infty)$ such that $c_1 a(n)\leq b(n)\leq c_2 a(n)$ for all $n\in \mathbb{N}$. Since two sequences are related, if both have the same order of growth, we call the quotient space $\displaystyle \mathbb{O}=\mathcal{O}/_{\approx}$ the space of the orders of growth. If $a(n)$ belongs to $\mathcal{O}$, we are going to denote $[a(n)]$ as the associated class in $\mathbb{O}$. If a sequence is defined by a formula (for example, $n^2$), then the order of growth associated will be represented by the formula between the brackets ($[n^2]\in \mathbb{O}$).

We define in $\mathbb{O}$ a natural partial order. We say that $[a(n)] \leq [b(n)]$ if there exists $C>0$ such that $a(n) \leq Cb(n)$, for all $n\in\mathbb{N}$. Observe that $\leq$ is well defined. We consider $\overline{\mathbb{O}}$ the Dedekind-MacNeille completion. This is the smallest complete lattice that contains $\mathbb{O}$. In particular, it is uniquely defined and we will consider $\mathbb{O}\subset \overline{\mathbb{O}}$. We will also call $\overline{\mathbb{O}}$ the complete set of the orders of growth. 

Now, we proceed to define the generalized entropy of a dynamical system in the complete space of the orders of growth. Given $(M,d)$ is a compact metric space and $f:M\rightarrow M$ a continuous map, we define in $M$ the distance 
\[d^{f}_{n}(x,y)=\max \{d(f^k(x),f^k(y)); 0\leq k \leq n-1\},\]
 and we denote the dynamical ball as $B(x,n,\e)=\{y\in M; d^{f}_{n}(x,y)<\e\}$. A set $G\subset M$ is a $(n,\e)$-generator if $\displaystyle M=\cup_{x\in G} B(x,n,\e)$. We define $g_{f,\e}(n)$ as the smallest possible cardinality of a finite $(n,\e)$-generator. If we fix $\e>0$, then $g_{f,\e}(n)$ is a non-decreasing sequence of natural numbers, and thus, $g_{f,\e}(n) \in \mathcal{O}$. For a fixed $n$, if $\e_1<\e_2$, then $g_{f,\e_1} (n) \geq g_{f, \e_2}(n)$, and therefore, $[g_{f,\e_1}(n)]\geq [g_{f,\e_2}(n)]$ in $\mathbb{O}$. We define the generalized entropy of $f$ as 
\[o(f)=\text{\textquotedblleft}\lim_{\e\rightarrow 0}"[g_{f,\e}(n)] =\sup \{[g_{f,\e}(n)]\in \mathbb{O}:\e>0\}\in \overline{\mathbb{O}}. \]

This object is a dynamical invariant.
\begin{theorem}[Correa-Pujals \cite{CoPu21}]\label{TeoCoPu01}
	Let $M$ and $N$ be two compact metric spaces and $f:M\to M$ and $g:N\to N$ be two continuous maps. Suppose there exists $h:M\to N$, a homeomorphism, such that $h\circ f = g \circ h$. Then, $o(f)=o(g)$.
\end{theorem}

We also define the generalized entropy through the point of view of $(n,\e)$-separated. We say that $E\subset M$ is $(n,\e)$-separated if $B(x,n,\e)\cap E = \{x\}$, for all $x\in E$. We define $s_{f,\e}(n)$ as the maximal cardinality of a $(n,\e)$-separated set. Analogously, if we fix $\e>0$, then $s_{f,\e}(n)$ is a non-decreasing sequence of natural numbers. Again, for a fixed $n$, if $\e_1<\e_2$, then $s_{f,\e_1} (n) \geq s_{f, \e_2}(n)$, and therefore, $[s_{f,\e_1}(n)]\geq [s_{f,\e_2}(n)]$. By standard arguments, it is verified that
\[o(f)=\sup \{[s_{f,\e}(n)]\in \mathbb{O}:\e>0\}.\]

The order of growth defined through open coverings is also equivalent. Consider $\U=\{U_1,\cdots, U_k\}$ an open covering of $M$ and define 
\[\U^n=\{U_{i_0}\cap f^{-1}(U_{i_1})\cap \cdots \cap f^{-n}(U_{i_n})\neq \emptyset: i_0,\cdots, i_n\in \{1,\cdots, k\}\},\]
which is also an open covering of $M$. Next, we define the sequence $a_{f,\U}(n)$ as the cardinal of the smallest sub-covering of $\U^n$. Then, it holds
\[o(f)=\sup\{[a_{f,\U}(n)]: \U\text{ is a finite open covering of }M\}.\]
 

Now, we shall explain how the generalized topological entropy is related to the classical notion of topological entropy and polynomial entropy. Given a dynamical system $f$, we recall that the topological entropy of $f$ is 
\[h(f) = \lim_{\e\to 0}\limsup_{n\rightarrow \infty} \frac{log(g_{f,\e}(n))}{n},\]
and the polynomial entropy of $f$ is
\[h_{pol}(f) = \lim_{\e\to 0}\limsup_{n\rightarrow \infty} \frac{log(g_{f,\e}(n))}{log(n)}.\]

We define the family of exponential orders of growth as the set $\mathbb{E}=\{[\exp(tn)]; t \in(0,\infty) \} \subset \mathbb{O}$ and the family of polynomial orders of growth as the set $\mathbb{P}=\{[n^t];t\in(0,\infty)\}$. Given $o\in \overline{\mathbb{O}}$ we define the intervals $I(o,\mathbb{E})=\{t\in (0,\infty):o\leq [\exp(tn)]\}$ and $I(o,\mathbb{P})=\{t\in (0,\infty):o\leq [n^t]\}$. With these intervals, we define the projections  $\pi_{\mathbb{E}}:\overline{\mathbb{O}}\rightarrow [0,\infty]$ and $\pi_{\mathbb{P}}:\overline{\mathbb{O}}\rightarrow [0,\infty]$ by
\[
\pi_{\mathbb{E}}(o)=
\begin{cases} \inf(I(o,\mathbb{E}))&\text{ if }I(o,\mathbb{E})\neq \emptyset \\
\infty &\text{ if } I(o,\mathbb{E})= \emptyset
\end{cases}
\] 
and 
\[
\pi_{\mathbb{P}}(o)=
\begin{cases} \inf(I(o,\mathbb{P}))&\text{ if }I(o,\mathbb{P})\neq \emptyset \\
\infty &\text{ if } I(o,\mathbb{P})= \emptyset
\end{cases}
\] 

The projection $\pi_{\Exp}$ verifies for elements in $\OG$, $\pi_{\Exp}([a(n)])=\limsup_{n\rightarrow \infty} \frac{log(a(n))}{n}$. Generalized entropy, polynomial entropy and classical entropy are related by the following result. 

\begin{theorem}[Correa-Pujals \cite{CoPu21}]\label{teo122}
	Let $M$ be a compact metric space and $f : M \rightarrow M$, a continuous map. Then, $\pi_\Exp(o(f))=h(f)$ and $\pi_\Pol(o(f))=h_{pol}(f)$.
\end{theorem}



\subsection{Coding of orbits in wandering dynamics}

 Let us consider $M$ a compact metric space and $f:M\to M$ a homeomorphism such that $\Omega(f)=\{p_1,\cdots, p_k\}$. Let $\F$ be a finite family of non-empty subsets of $M\setminus \Omega(f)$. We denote by $\cup \F$ the union of all the elements of $\F$ and by $\infty_{\F}$ the complement of $\cup \F$. Let us fix a positive integer $n$ and consider $\underline{x}=(x_0,\cdots,x_{n-1})$ a finite sequence of points in $M$ and $\underline{w}=(w_0,\cdots,w_{n-1})$ a finite sequence of elements of $\F\cup \{\infty_{\F}\}$. We say that $\underline{w}$ is a coding of $\underline{x}$, relative to $\F$, if for every $i=0,\cdots, n-1$, $x_i\in w_i$. Whenever the family $\F$ is fixed, we simplify the notation by using $\infty$ instead of $\infty_{\F}$. Note that if the sets of $\F$ are not disjoint, we can have more than one coding for a given sequence. We denote the set of all the codings of all orbits $(x,f(x),\cdots, f^{n-1}(x))$ of length $n$ by $\A_{n}(f,\F)$. We define the sequence $c_{f,\F}(n)=\# \A_{n}(f,\F)$, and it is easy to see that $c_{f,\F}(n)\in \mathcal{O}$.


We say that:
\begin{itemize}
\item a set $Y$ is wandering if $f^n(Y)\cap Y =\emptyset$ for every $n\geq 1$. 
\item  $Y$ is a compact neighborhood if it is compact and is the closure of an open set. 
\item the subsets $Y_1,\cdots, Y_L$ of $M\setminus \{\Omega(f)\}$ are mutually singular if, for every $N>0$, there exists a point $x$ and times $n_1,\cdots, n_L$ such that $f^{n_i}(x)\in Y_i$ for every $i=1,\cdots, L$, and $|n_i -n_j|>N$ for every $i\neq j$. 
\end{itemize}


Let us call $\Sigma$ the family of finite families of wandering compact neighborhoods that are mutually singular. Given $\delta>0$, we define $\Sigma_\delta$ as the subset of $\Sigma$ formed by every family whose every element has a diameter smaller than $\delta$. 

\begin{theorem}[Correa - de Paula \cite{CoPa23}]\label{TeoCod}
Let $M$ be a compact metric space and $f:M\rightarrow M$ a homeomorphism such that $\Omega(f)$ is finite. Then, 
\[o(f)=\sup\{[c_{f,\F}(n)]\in \OG: \F\in \Sigma\}.\]
In addition, the equation also holds if we switch $\Sigma$ by $\Sigma_\delta$. 
\end{theorem}



\subsection{Bounded Jump property}\label{subBJP}

We say that a class of orders of growth $[a(n)]$ verifies the bounded jump property (BJP) if there exists a constant $C>0$ such that $a(n+1)\leq Ca(n)$. Note that this definition does not depend on the choice of the class representative. We call $\B\subset \OG$ the set of orders of growth that verify the BJP and $\CB$ the subset in $\COG$ defined by:
\[\CB=\{sup(\Gamma)\in \COG:\Gamma\subset\B\text{ and }\Gamma\text{ is countable}\}.\]
The set $\CB$ is the countable completion of $\B$. 


\begin{proposition}\label{propB}
Let us consider $M$ a compact metric space and $f:M\to M$ a homeomorphism. Then, it is verified
\begin{enumerate}
\item For $\U$ a finite covering of $M$,  $[a_{f,\U}(n)]\in \B$.  
\item  $o(f)\in \CB$.
\end{enumerate}  
\end{proposition}

\begin{proof}
We shall begin with proving the result for open coverings. Consider $\U=\{U_1,\cdots, U_k\}$ a finite open covering of $M$ and fix $n\in \N$. Suppose $\V\subset \U^n$ is a sub-covering such that $a_{f,\U}(n)=\# \V$. Then, 
$\V'=\{V\cap f^{-(n+1)}(U_i):V\in \V, U_i\in \U\}\subset \U^{n+1}$ is also a sub-covering with at most $\#\U\#\V$ elements. Therefore, $ a_{f,\U}(n+1)\leq \#\U a_{f,\U}(n)$. 

Now that we know $[a_{f,\U}(n)]\in \B$, we only need to show that $o(f)$ is obtained as the supremum of a countable family of open coverings. By lemma A.1 in \cite{CoPu21}, given $\e>0$ if $diam(\U)\leq \e$, then $[a_{f,\U}(n)]\geq [s_{f,\e}(n)]$. Therefore, if $\{\U_k\}_{k\in\N}$ is a family of open coverings such that $diam(\U_k)\leq \frac{1}{k}$, then $o(f)=\sup\{[a_{f,\U_k}(n)]:k\in\N\}$. 
\end{proof}

In \cite{CoPa23}, we proved an analogous result for $[c_{f,\F}(n)]$.

\begin{lemma}[Correa - de Paula \cite{CoPa23}]\label{LemBouJum}
Let us consider $M$ a compact metric space and $f:M\to M$ a homeomorphism whose non-wandering set is finite. Given $\F\in \Sigma$, the class $[c_{f,\F}(n)]\in \B$.
\end{lemma}

The next result show how we use the BJP. We recall that a set $\S\subset \N$ is syndetic if there exists $N\in \N$ such that for all $n$, the interval $[n,n+N]$ contains at least one point of $\S$. 

\begin{lemma}[Correa - de Paula \cite{CoPa23}]\label{LemSyn}
Let us consider $[a(n)]\in \B$, a syndetic set $\S$ and a sequence $b(n)\in \OR$. If there exist two constants $c_1$ and $c_2$ such that $c_1 b(n)\leq a(n)\leq c_2 b(n)$ for all $n\in \S$, then $[a(n)] =[b(n)]$. 
\end{lemma} 

This lemma tell us that if we know an order of growth in a syndetic set and said order of growth verifies the BJP, then we understand the order of growth in $\N$. For example, if we prove that $c_1 n^k\leq c_{f,\F}(n)\leq c_2 n^k$ for every  $n\in \S$, then $[c_{f,\F}(n)]=[n^k]$. 



\subsection{Linearly invariant property} \label{subLIP}

We say that a class of orders of growth $[a(n)]$ verifies the linearly invariant property if there exists an integer $m\geq 2$ such that $[a(n)]= [a(mn)]$. We call $\L\subset \OG$ as the set of orders of growth that verify the LIP and $\CL$ the subset in $\COG$ defined by:
\[\CL=\{sup(\Gamma)\in \COG:\Gamma\subset\L\text{ and }\Gamma\text{ is countable}\}.\]
The set $\CL$ is the countable completion of $\L$. 

Orders of growth in $\L$ verify the following properties

\begin{proposition}\label{propLIPProp}
Let $[a(n)]\in \L$. Then, it is verified:
\begin{enumerate}
\item $[a(n)]\in \B$.
\item For all integer $m\geq 2$, it is verified $[a(mn)]=[a(n)]$.
\item $\pi_{\Exp}([a(n)])=0$.
\end{enumerate} 
\end{proposition}

\begin{proof}
$(1)$. The BJP is equivalent to $[a(n)]=[a(n+1)]$. Suppose $[a(mn)]=[a(n)]$ for some $m\geq 2$. Since $a(n)\leq a(n+1) \leq a(mn)$ for $n\geq 1$, we infer  $[a(n)]=[a(n+1)]$.

$(2)$. Let us suppose $[a(2n)]=[a(n)]$ and consider $c_2>c_1>0$ such that $c_1 \leq \frac{a(2n)}{a(n)} \leq  c_2$. Observe that
\[c_1^2 \leq \frac{a(4n)}{a(2n)} \frac{a(2n)}{a(n)}\leq c_2^2\ \forall n\in \N.\]
From this, we deduce that $[a(4n)]=[a(n)]$ and by induction, we conclude $[a(2^k n)]=[a(n)]$. Now, we fix $m\in \N$ and choose $2^k> m$. Since $a(n)\leq a(mn) \leq a(2^kn)$ and $[a(2^k n)]=[a(n)]$ we see
$[a(mn)]=[a(n)]$.

$(3)$. It is simple to observe that
\begin{align*}
 \pi_{\Exp}([a(n)])  & = \pi_{\Exp}([a(mn)])\\
 & =  \limsup_n \frac{log(a(mn))}{n} \\
  &= m \limsup_n \frac{log(a(mn))}{mn} \\
  &\leq m  \limsup_n \frac{log(a(n))}{n} \\
  & =  m\pi_{\Exp}([a(n)]).
\end{align*}
However, $m\geq 2$ and therefore, $\pi_{\Exp}([a(n)])=0$. 
\end{proof}

A similar reasoning as the one done in proposition \ref{propB} and the third property of the previous proposition implies the following result. 

\begin{proposition}
Consider $f:M\to M$ a dynamical system such that $[a_{f,\U}(n)]\in \L$ for every finite open covering  $\U$. Then,
\begin{enumerate}
\item $o(f)\in \CL$, 
\item and $h(f)=0$. 
\end{enumerate}
\end{proposition}