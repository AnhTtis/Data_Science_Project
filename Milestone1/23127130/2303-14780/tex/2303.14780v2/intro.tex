
\section{Introduction}

An important problem in dynamical systems is to measure the complexity of a map in terms of its orbits. The topological entropy of a system studies the exponential growth rate at which orbits are separated and has become a crucial tool for classifying highly chaotic dynamical systems. However, there are many interesting families of dynamical systems, where every system has vanishing entropy, and therefore, another object is needed. The object that studies the polynomial growth rate is called polynomial entropy and was introduced by J. P. Marco in  \cite{Ma13}. S. Galatolo introduced in \cite{Ga03} the concept of generalized topological entropy that extends classical and polynomial entropy for one-parameter families of orders of growth. This concept is a translation of A. Katok and J. P. Thouvenot notion of slow entropy defined in \cite{KaTh97}.   For a more detailed exposition on the works related to the previous quantities check \cite{CoPu21}, \cite{CoPa23} and \cite{KaKaWe20}. In this article, we are going to work with the notion of generalized entropy introduced by the first author and E. Pujals in \cite{CoPu21}. This object, instead of quantifying the complexity of a system with a single number, works directly in the space of the orders of growth $\mathbb{O}$. In subsection \ref{subOGGE} we show how is constructed $o(f)$, the generalized entropy of a map $f$. 

In this text, we are interested in the families of dynamical systems on the sphere whose non-wandering set consist in only one fixed point. We shall denote by $S^2$ the sphere and $\Omega(f)$ the non-wandering set of a map $f:S^2\to S^2$. We define $\H$ as the family of homeomorphisms $f:S^2\to S^2$ such that $\#\Omega(f)=1$.  A simple consequence of the variational principle is that any map in $\H$ has vanishing classical entropy. The family  $\H$ is naturally interesting because the orientation preserving ones are in a direct bijection with Brouwer homeomorphisms of the plane (through compactifiyng the plane with one point).

For the family $\H$, in the context of polynomial entropy we have the result of  L. Hauseux and F. Le Roux in \cite{HaRo19}. We shall call the polynomial entropy of a map by $h_{pol}(f)$. 

\begin{theorem}[L. Haseux and F. Le Roux]\label{theoHLR}
The following two happens:
\begin{enumerate}
\item For any $f\in \H$, $h_{pol}(f)\in \{1\}\cup [2,\infty]$. 
\item For any $t\in \{1\}\cup [2,\infty]$, there exists $f\in \H$ such that $h_{pol}(f)=t$. Moreover, $h_{pol}(f)=1$ if and only if $f$ is the compactification of a map conjugate to a translation in $\R^2$. 
\end{enumerate}
\end{theorem}

The first part of the theorem gives natural boundaries and the second part is a flexibility result in the sense introduced by J. Bochi, A. Katok and F. Rodriguez-Hertz in \cite{BoKaRH19}.  The article \cite{HaRo19} has two parts. First, a way to code the orbits of a dynamical system with  $\#\Omega(f)=1$, that allows the computation of $h_{pol}(f)$ and second, the construction of maps in the point $(2)$ of theorem \ref{theoHLR}. We have extended the first part in \cite{CoPa23} to generalized entropy and also to maps with finite non-wandering set. With this tool, we computed the generalized entropy of Morse-Smale diffeomorphisms on surfaces and some maps in the boundary of chaos. Our main result in this article is an extension of the second part of the work by L. Hauseux and F. Le Roux.

In order to enunciate our theorem, we need to fix some terminology associated to $\OG$ the space of orders of growth. To us, an order of growth is the class of non-decreasing sequences in $(0,\infty)$ that have the same asymptotic speed  at $\infty$. We represent this classes by $[a(n)]$ for the general case and the formula between brackets if it is defined by one, for instance $[n^2]$. The family of polynomial orders of growth is $\Pol=\{[n^t]:0<t<\infty\}$. We say that an order of growth $[a(n)]$ verifies the linearly invariant property (LIP) if for some $m\geq 2$, $[a(mn)]=[a(n)]$. The set of orders of growth that verifies LIP is represented by $\L$ (observe that $\Pol \subset \L$).  The abstract completion of $\OG$ by its partial order is the complete set of orders of growth $\COG$ and the generalized entropy of a map $o(f)$ is defined in $\COG$. General elements of $\COG$ are represented by $o$. We define $\CL$ as the countable completion of $\L$ in $\COG$ (Check subsections \ref{subOGGE}, \ref{subBJP} and \ref{subLIP} for more details). 


 By our results in \cite{CoPa23}, is immediate that for any $f\in \H$, either $o(f)=[n]$ or $[n^2]\leq o(f) \leq \sup(\Pol)$. The main result in this article is the following. 

\begin{theorem}\label{teoFlex}
For any $o\in \CL$ verifying $o=[n]$ or $[n^2]\leq o \leq \sup(\Pol)$, there exists $f\in \H$ such that $o(f)=o$. 
\end{theorem}

We observe that the order of growth $[n^2 log(n)]$ belongs to $\CL$ and therefore the following corollary holds.

\begin{corollary}\label{coroPolEqu}
There exists $f_1,f_2\in \H$ such that $o(f_1)=[n^2]$ and $o(f_2)=[n^2 log(n)]$. In particular, $h_{pol}(f_1)=h_{pol}(f_2)$ yet $o(f_1)\neq o(f_2)$. 
\end{corollary}

The first part of theorem \ref{theoHLR} implies that polynomial orders of growth is the right scope to analyze the dispersion of orbits for maps in $\H$. However, we argue, in the light corollary \ref{coroPolEqu}, that generalized entropy allows to appreciate on a deeper level how rich the universe of possible complexities actually is.

Let us comment on the hypothesis $\CL$. There is a weaker property an order of growth may verify that we call the bounded jump property (BJP) and was introduced in \cite{CoPa23}. We shall use $\B$ to represent the elements in $\OG$ that verifies BJP and $\CB$ to represent the countable completion of $\B$.  The BJP is in fact necessary: for any $f:M\to M$ with $M$ a compact metric space, $o(f)\in \CB$ (see proposition \ref{propB}). Therefore, the best version of theorem \ref{teoFlex} that may be true is for any $o\in \CB$. The linearly invariant property implies the bounded jump property  ($\L\subset \B$) but the converse is false. For example any exponential order of growth $[e^{tn}]$ fails to verify LIP. The LIP is in fact associated to sub-exponential orders of growth (see proposition \ref{propLIPProp}), but even in this context we believe to be possible the construction of elements in $[a(n)]\in \B\setminus \L$ such that $ [a(n)]< \sup(\Pol)$. On the other hand, any simple example of $[a(n)]< \sup(\Pol)$ constructed by formulas verify LIP. For instance $[n^t]$, $[log(n)]$, $[log(log(n))]$ belong to $\L$. Moreover, we consider it an nice hypothesis to work with. Since it is naturally associated to sub-exponential orders of growth, any proof involving LIP, will use arguments intrinsic to the world of vanishing entropy that will not be an adaptation from the positive entropy setting. 

We would like to finish this introduction with some comments and open questions. Consider $M$ a compact metric space and $f:M\to M$ a homeomorphism. A simple consequence of the variational principle is that $h(f_{|\Omega(f)})=h(f)$. However, for generalized entropy it is only true $o(f_{|\Omega(f)})\leq o(f)$. In order to have a clean notation we call $o(f,\Omega(f))= o(f_{|\Omega(f)})$. As it was observed in \cite{CoPu21} the generalized entropy may ``spike" from $o(f,\Omega(f))$ to $o(f)$. Moreover, for the class of homeomorphisms worked in this article $\H$, every map verifies $o(f,\Omega(f))$ is the class of the constant sequence $[c]$ and $[n]\leq o(f) \leq \sup(\Pol)$. Therefore, $o(f,\Omega(f))< o(f)$. We consider that a possible path to understand this jumps in the general case is the following. Define $\hat M = M/\Omega(f)$, this is, collapse the non-wandering set of $f$ to a single point. The map $f$, induces a homeomorphism $\hat f: \hat M \to \hat M$ whose non-wandering set contains only one point. Now,  with the examples we have worked so far, the inequalities 
\[\max\{o(f,\Omega(f)), o(\hat f) \}\leq o(f) \leq o(f,\Omega(f))o(\hat f), \]
always seems to hold and we ask if this is true in general. It is simple to construct examples where one inequality is strict. If the answer is positive and $o(f,\Omega(f))=[c]$, then $o(f)= o(\hat f)$ because $[c]$ is the identity for the product in $\COG$. Therefore, a simpler version of our previous question is the following: If $o(f,\Omega(f))=[c]$, is $o(f)= o(\hat f)$?


This work is structured as follows:
\begin{itemize}
\item In section \ref{secPrelim}, we show the construction of the set of orders growth and generalized entropy. We also explain the coding of orbits for maps in $\H$, and prove some simple results about the bounded jump and linearly invariant properties. 
\item In section \ref{secTeoFlex}, we prove theorem \ref{teoFlex}. We split the proof in two cases. We address first when $o\in \B$ and then when $o\in \CB$.   
\end{itemize}