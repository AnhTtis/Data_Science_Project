%%%%%%%% HK 16.3.2023 %%%%%%%%%%%%%%%%%
%%%%%%%% MINIMAL VERSION %%%%%%%%%%%%%%%%
\documentclass[twocolumn]{revtex4}
\usepackage{graphicx}
\usepackage{amssymb}


\def\al{\alpha}
\def\be{\beta}
\def\ga{\gamma}
\def\Ga{\Gamma}
\def\de{\delta}
\def\De{\Delta}
\def\eps{\varepsilon}
\def\th{\theta}
\def\la{\lambda}
\def\si{\sigma}
\def\bsi{\bar\sigma}
\def\Si{\Sigma}
\def\om{\omega}
\def\Om{\Omega}
\def\La{\Lambda}
\def\la{\lambda}
\def\vp{\varphi}

\def\E{{\rm E}_{10}}
\def\KE{{{\rm K}(\E})}


\def\rS{{\rm S}}
\def\rG{{\rm G}}

\def\bbi{{\bar{\bf a}}}
\def\bbj{{\bar{\bf b}}}
\def\bbk{{\bar{\bf c}}}
\def\bbl{{\bar{\bf d}}}
\def\bbm{\bar{\bf m}}

\def\mJ{\mathfrak{J}}
\def\mG{\mathfrak{G}}
\def\mN{\mathfrak{N}}

\def\eq{\!=\!}

\def\cA{{\mathcal A}}
\def\cC{{\mathcal C}}
\def\cD{{\mathcal D}}
\def\cL{{\mathcal L}}
\def\cJ{{\mathcal J}}
\def\cM{{\mathcal M}}
\def\cN{{\mathcal N}}
\def\cO{{\mathcal O}}
\def\cS{{\mathcal S}}
\def\cR{{\mathcal R}}
\def\cU{{\mathcal U}}
\def\11{{\mathbb 1}}
\def\II{{\mathbb I}}
\def\JJ{{\mathbb J}}
\def\MM{{\mathbb M}}
\def\NN{{\mathbb N}}
\def\RR{{\mathbb R}}
\def\ZZ{{\mathbb Z}}
\def\re{{\rm e}}
\def\ri{{\rm i}}
\def\rd{{\rm d}}

\def\tM{\tilde{M}}
\def\JJBL{\JJ_{B - L}}
\def\cJBL{\cJ_{B - L}}

\def\rX{{\rm X}}
\def\hrX{\widehat{\rm X}}
\def\hcJ{\widehat{\mathcal{J}}}
\def\hJJ{\widehat{\mathbb{J}}}
\def\tF{\widetilde{F}}
\def\tG{\widetilde{G}}
\def\tW{\widetilde{W}}


\def\eV{\,\rm eV}
\def\keV{\,\rm keV}
\def\MeV{\,\rm MeV}
\def\GeV{\,\rm GeV}
\def\TeV{\,\rm TeV}

\def\MPL{M_{\rm Pl}}
\def\cd{\!\cdot\!}


\def\vev{\langle\varphi\rangle}

\def\beq{\begin{equation}}
\def\eeq{\end{equation}}
\def\bea{\begin{eqnarray}}
\def\eea{\end{eqnarray}}
\def\nn{\nonumber}


\def\ra{\rightarrow}
\def\Ra{\Rightarrow}
\def\d{\partial}
\def\ri{\text{i}}
\def\Tr{\text{Tr}}



\begin{document}
\title{Evidence for a stable supermassive gravitino with charge\, 2/3?}
\author{Krzysztof A. Meissner$^1$ and Hermann Nicolai$^2$\\}
\affiliation{\\
$^1$Faculty of Physics,
University of Warsaw\\
Pasteura 5, 02-093 Warsaw, Poland\\
$^2$Max-Planck-Institut f\"ur Gravitationsphysik
(Albert-Einstein-Institut)\\
M\"uhlenberg 1, D-14476 Potsdam, Germany
}

\vspace{10mm}

\begin{abstract} Some time ago it was suggested that dark matter may consist 
in part of an extremely dilute gas of supermassive fractionally charged gravitinos
\cite{MeissnerNicolai2019}. This scheme makes the definite (and falsifiable) prediction that massive 
gravitinos are the {\em only} new fermionic degrees of freedom  beyond the known 
three generations of quarks and leptons of the Standard Model of Particle Physics. 
In this note we re-examine one special event reported and subsequently 
discarded by the MACRO collaboration \cite{MACRO,MACRO1} in the light of 
this proposal. 
\end{abstract}
%\pacs{95.35.+d,04.65.+e}
\maketitle

\vspace{5mm}

%%%%%%%%%%%%%%%%%%%%%%%%%%%%%%%
\section{Introduction.} 
The nature of dark matter (DM) continues to be one of the most vexing
questions of modern physics. While current DM scenarios are usually based 
on the assumption of ultralight constituents (such as axions) or 
TeV scale WIMPs, the possibility has been raised in recent work \cite{MeissnerNicolai2019}
that DM could consist at least in part of an extremely dilute gas  of 
supermassive stable gravitinos with charge $q=\pm\frac23$ in units of the
elementary charge $e$. This proposal has its roots in Gell-Mann's old observation 
that the fermion content of the Standard Model of Particle Physics (SM)
with three, and only three, generations of quarks and leptons (including
right-chiral neutrinos) can be matched with the spin-$\frac12$ content of the
maximal $N\!=\!8$ supermultiplet after removal of eight Goldstinos \cite{GM,NW}.
The only extra fermions beyond the known three generations of SM fermions would 
thus be eight massive gravitinos, but nothing else.
Crucially, the matching of U(1)$_{em}$ charges requires 
a `spurion shift' of $\delta q = \pm \frac16$ \cite{GM} that is not part of $N\!=\!8$ 
supergravity and that, in terms of the original spin-$\frac12$
fermions of $N\!=\!8$ supergravity, takes the very special form given in \cite{MN2,KM0}.
The main new step taken in \cite{KM0,MeissnerNicolai2019} consisted in extending these
considerations to the eight massive gravitinos which split as
\beq\label{GravCharges}
         \left({\bf 3}\,,\,\frac13\right) \oplus \left(\bar{\bf 3}\,,\,-\frac13\right)
         \oplus \left({\bf 1}\,,\,\frac23\right) \oplus \left({\bf 1}\,,\, -\frac23\right)
\eeq
under SU(3)$\,\times\,$U(1)$_{em}$. All gravitinos would thus carry 
{\em fractional electric charges}. If one identifies the SU(3) in (\ref{GravCharges}) 
with SU(3)$_c$ as in \cite{MeissnerNicolai2019}, a complex triplet of gravitinos would 
be subject to strong interactions. Importantly, the U(1)$_{em}$ charge 
assignments for the gravitinos include the spurion shift needed for matching 
the spin-$\frac12$ sectors~\footnote{The explanation of how to extend the spurion shift 
to the gravitinos (whose U(1) charges would otherwise be $\pm \frac16$ 
and $\pm \frac12$) requires a detour via $D=11$ supergravity \cite{KN}. 
The incorporation of this shift requires enlarging the SU(8) R-symmetry of
$N\!=\!8$ supergravity to $K({\rm E}_{10})$.}.
Although the combined spin-$\frac12$ and 
spin-$\frac32$ content would thus coincide with the fermionic part of 
the $N\!=\!8$ supergravity multiplet, the underlying theory would need to be 
a very specific, but as yet unknown, extension
of (gauged) $N\!=\!8$ supergravity, which in its original form \cite{CJ,dWN} 
cannot be correct for reasons that have been known for more than 40 years.

An important consequence of (\ref{GravCharges}) is that, due to their 
fractional charges the gravitinos cannot decay into SM fermions, and 
are therefore stable independently 
of their mass. Their stability against decays makes them natural candidates for 
DM \cite{MeissnerNicolai2019}. While a very large mass is strongly suggested by the absence
of low energy supersymmetry at LHC, an even more compelling argument for large
mass is the bound on the charge $q$ of any putative DM particle of mass $m$
derived in \cite{milli1,milli2,NNP}
\beq\label{qbound}
|q|\;\lesssim \; 7.6\cdot 10^{-10}\left(\frac{m}{1\TeV}\right)^\frac12
\eeq
which immediately shows that gravitinos with charges (\ref{GravCharges})
can only be viable DM candidates
if their masses are close to the Planck scale. As we argued in \cite{MeissnerNicolai2019} the
strongly interacting gravitinos would have mostly disappeared during the cosmic
evolution (but could play a role in explaining UHECRs and the predominance 
of heavy ions in such events \cite{MN3}). By contrast, the abundance of the 
color singlet gravitinos cannot be estimated since they were never in thermal 
equilibrium, but one can plausibly assume their abundance in first approximation
to be given by the average DM density inside galaxies \cite{WdB}, {\it viz.}
\beq\label{DM}
\rho_{DM} \,\sim \,0.3\cdot 10^6 \GeV\cd m^{-3}  \,.
\eeq 
If DM were entirely made out of nearly Planck mass particles, this would
amount to $\sim 3\cdot 10^{-13}$ particles per cubic meter 
within galaxies (the average in the Universe is a million times smaller). Nevertheless, for the
estimation of flux rates there remain important uncertainties, for instance
concerning possible inhomogeneities in the DM distribution within
galaxies or even stellar systems, as well as the average velocity of 
superheavy DM particles w.r.t. the earth. 

A distinctive feature of the present proposal is that the DM gravitinos {\em do} 
participate in SM interactions. This is in contrast to other scenarios 
involving supermassive DM particles which are assumed to have only 
(super-)weak and gravitational interactions with SM matter
\cite{Markov,ACN,Sriv,SHDM1,SHDM2,PIDM,PIDM1,SHDM6}, and which
are mainly motivated by inflationary cosmology, whence
the mass of those DM constituents would still be well below the Planck 
scale, on the order of the scale of inflation $\lesssim 10^{16}\,$ GeV
(and thus, by (\ref{qbound}), compatible only with milli-charged particles).
By contrast the gravitinos in (\ref{GravCharges})  could in principle be detected 
if a way could be found to overcome their low abundance. 
A superheavy electrically charged particle could easily pass 
through the earth without deflection, leaving  
a very straight but tiny ionized track in the earth's crust.

This leaves us basically with 
two options for discovery. Either one searches for such tracks in old and very
 stable rock with a paleodetector, or otherwise one sets up an underground 
detector with sufficiently large fiducial area/volume and waits for the candidate  
particle to come by. The paleodetector option has been tried in the past with 
MICA samples \cite{MICA0,MICA}, again to search for magnetic
monopoles;  a general difficulty here is that the 
tracks would have to remain unaffected by geological processes 
over very long times, and the detection technique must be such as not 
to destroy the tracks (this favors MICA which comes with a naturally
layered structure). The other and perhaps more promising option is to look for ionized
tracks with suitable underground detectors and time of flight measurements, 
focusing on {\em slow} ionizing particles.
The main background would come from cosmic ray muons, but a possible
way to rule those out would be to look for slow particles {\em moving bottom up}
which must have traversed a substantial part of the earth before being
registered by the detector. In this context, the possible relevance of the MACRO 
experiment \cite{MACRO} was already pointed out in \cite{MeissnerNicolai2019} where it was 
suggested to have a second look at the data collected over many years.
This is what we will now do, focusing on one special event.


\section{The MACRO experiment and a special event}

The MACRO experiment \cite{MACRO} was originally designed to search for magnetic
monopoles, finishing with a null result after several years of taking data. 
We refer to the summary paper \cite{MACRO} for a detailed description of the 
experiment and of what the detector was capable of doing, as well as a summary
of the collected results. The search covered a large part of parameter space, including 
the full range of velocities from relativistic particles down to `slow' particles 
with $\beta \sim 4 \times 10^{-5}$, coming in from all directions. In this way the 
detector was able to search not only for magnetic monopoles, but also for other, 
and unknown kinds of ionizing particles, including fractionally charged particles. 
Results of the latter search which concentrated specifically on {\em lightly
ionizing particles} (LIPs) appeared in a separate publication \cite{MACRO1}.

While \cite{MACRO} mentions 40 events (out of a total of about 35\,000) that were 
subsequently discarded as spurious and not further discussed,
Ref. \cite{MACRO1} reports one special event of a type different from an 
expected monopole signal. The relevant information about  this event is contained 
in figure 3 of \cite{MACRO1} which we here reproduce for the reader's 
convenience, together with its figure caption. 
\begin{figure}
\centering
\null\hfill\includegraphics[width=.6\textwidth, clip=true, trim = 115mm 180mm 0mm 20mm]{MACRO.eps}
\caption{Energy loss as measured by PHRASE for the LIP events
that passed the track quality and geometry cuts and satisfied the requirement
of a maximum energy loss rate (measured by ERP) less than 1.1 MeV/cm. 
[...]
The signal region is in the [0, 1.35] MeV/cm interval. Figure copied from \cite{MACRO1}.}
\end{figure}
Let us also quote the accompanying 
part of the text in \cite{MACRO1} which says:
``As one can see [...] there is one event (run 15871,
event 5649) that appears in the signal region. It corresponds
to a maximum energy loss of 0.66 MeV/cm, i.e., about 20\% lower 
than what expected for a particle of charge $2e/3$ and about
a factor of 3 higher than what expected for a particle of charge
$e/3$ . Three scintillator counters were involved in this trigger; the
first in one of the upper vertical layers, the second in the central
horizontal layer and the third in the lower horizontal layer.
[...] The position along the counter for this particular
box measured by the PHRASE and by the streamer
tube track geometry were in agreement (within 15 cm). We
have examined this event by hand relying primarily on the
wave forms as recorded for all the counters involved in the
trigger. The apparent amplitude of the recorded wave forms
was consistent with the energy thresholds for the ERP and the
PHRASE. Having three scintillator counters involved in the
trigger we have checked for a consistency in the relative timing
of them with the crossing of a single particle of constant
velocity. The relative timing between the counter in the upper
part of the detector and that in the central part was consistent
with the passage of a relativistic particle coming from above
while the relative timing between the box in the lower part of
the detector and any of the other two hits was consistent with a
slowly moving upward-going particle. We thus discarded this
event from the signal region.'' 

The main reason for discarding this event was therefore not some obvious
instrumental glitch or any indication of a fake signal, but rather the fact that
there appears to be no way to reconcile all three scintillator signals
(as well as the fact that the signal does not conform with
expectations for a magnetic monopole), whence one
concludes that one of the three signals must be ascribed to a different origin.
A crucial additional fact is that there is unmistakable and independent evidence 
from the streamer tubes for a single particle track running through the whole detector.
The `obvious' interpretation of this event would seem to be in terms of
a relativistic particle corresponding to a  cosmic ray muon 
coming from above, and this remains perhaps the most plausible explanation.
However, this interpretation would not only require disregarding the 
earlier scintillator signal in the lower part of the detector, but also 
explaining why a third signal in the same scintillator, coincident with the central and upper 
scintillators, is missing. It would also require an explanation
why the energy loss rate is well below the allowed minimum in Fig.~1
since the measured charge of $|q|=\frac23$ is simply not compatible with a muon.
By contrast, the alternative second interpretation with a slow particle moving upward
requires only the signal in the upper scintillator to be due to some other cause. 
While it appears that the question cannot be finally resolved on the basis of 
the existing MACRO data, 
we here wish to raise attention to the hypothesis 
that  it is the signal in the upper layer that was caused by a different cause,
which could have masked the presumed 
later arrival of the slow particle in the upper detector. The event could 
then correspond to the passage of a `slow' gravitino through the 
MACRO detector; this hypothesis is also supported by the 
measured low ionization pointing to charge $\pm \frac23$. If this 
assumption is made we have two additional indications for the correctness
of our hypothesis, as emphasized in the above quote from \cite{MACRO1}, namely
\begin{itemize}
\item the track as seen by the streamer tubes was 
         consistent with the tracks in the lower and central scintillators; and
\item the time of flight was consistent with the same slow particle moving 
         bottom up in the lower and central scintillators 
\end{itemize}
With these additional consistency checks let us re-iterate that a `slow' particle 
moving bottom up would be difficult to explain in terms of known physics. First
of all, going up, it obviously cannot be a muon. Second, it cannot be a magnetic 
monopole, nor a dyon, because monopoles generally are not expected to be lightly 
ionizing  \footnote{In principle an electrically neutral monopole can acquire a very 
small electric charge proportional to the CP violating $\theta$-angle by means 
of the Witten effect \cite{Witten}. However, given the known upper limits on the
value of $\theta$ the ionization would be dominated by magnetic interactions.}; 
although monopoles may in principle be able to traverse the  earth \cite{MM1}, 
according to \cite{MM2}, the energy loss in the scintillator would be 
$\sim 1$ GeV/cm for the exemplary value $\beta\sim 0.01$, and the light yield
would be saturated and bigger than the observed $0.66$ MeV/cm; 
for a dyon the energy loss would be even bigger.
Third, the full track as reconstructed from the streamer tubes
cannot be the result of a radioactive decay in the surrounding rock, 
since such products have energies of at most several MeV, so they could not 
penetrate the scintillator more deeply than a few centimeters. Therefore a
superheavy fractionally charged particle seems to be the most plausible 
explanation if one adopts our hypothesis. 

While \cite{MACRO1} does not explicitly quantify what `slow' means 
a more precise knowledge of the velocity would be useful as it would enable 
us to make a first estimate of the expected gravitino flux. A potential difficulty 
here is that the trigger used in \cite{MACRO1} was sensitive only to fast 
lightly ionizing particles:  as stated in the abstract of \cite{MACRO1}, the trigger 
was set for ionizing particles with velocities $\beta > 0.25$, hence any slower 
particle could have escaped detection without such an accompanying 
triggering signal,  and thus possible events involving only a slow particle 
could have been missed. Therefore the rate of one event for the five-year 
cycle covered by \cite{MACRO1} could be an underestimate of the 
actual abundance and flux rate for supermassive gravitinos, thus
explaining why no further events of this type were observed (even 
taking into account the expected rarity of superheavy gravitinos). For this reason
we cannot at this point reliably deduce the actual gravitino density in the vicinity 
of the Earth from the given data. This, as well as the confirmation (or refutation)
of our hypothesis, would require a dedicated new experiment, concentrating on
slow lightly ionizing particles.


\section{Conclusions}
Re-inspection of the special MACRO event reported in \cite{MACRO1} 
has revealed the possibility of an explanation different from the `obvious' one 
in terms of a cosmic ray muon, namely

\begin{itemize}
\item a slow particle {\em moving bottom up}, which thus must have
         traversed a substantial part of the earth; and
\item the fact that this particle carries fractional charge $q=\frac23$ 
         within a $20\%$ error margin.
\end{itemize}

Obviously, further and independent confirmation
is needed to find out whether this is real physics or
just a fluke and we hope that in the not-so-distant future 
some dedicated experiments will decide. 
We stress that the spin of  the putative DM particle remains 
unknown, so even if the event is due to a superheavy particle 
it remains to determine its spin to confirm (or not) 
spin-$\frac32$, as advocated in \cite{MeissnerNicolai2019}.

On the other hand, corroboration of our new interpretation 
of this event would have several implications. In particular, it would 
bring $N\!=\!8$ supergravity back into focus
for unification, although in an unexpected way. We emphasize that the 
considerations leading to (\ref{GravCharges}) are so far purely kinematical,
and that the dynamics underlying the present scheme remains unknown,
possibly requiring a framework beyond space-time based quantum field theory.
Nevertheless our findings may indicate that $N\!=\!8$ supergravity could be closer 
to the truth than is widely thought (as is also suggested by the finiteness properties
of the theory \cite{Bern} and various anomaly cancellations
\cite{Marcus,Kallosh,Zvi,MN4}). We also note that the spurion shift required 
to match the spin-$\frac12$ sectors of the theory with three generations
of quarks and leptons \cite{GM},  and here extended to the gravitinos \cite{KM0},
appears to be  incompatible with space-time supersymmetry. This could mean 
that, contrary to many expectations, (maximal) space-time supersymmetry might 
not be a relevant concept for unification after all, but, through its fermionic 
(spin-$\frac12$ and spin-$\frac32$) content, merely a
theoretical crutch to guide us to the right answer.

\vspace{0.5cm}
\noindent
 {\bf Acknowledgments:} 
 We would like to thank Barry Barish for helpful correspondence. Furthermore, 
 we are greatly indebted to Erik Katsavounidis for alerting us to \cite{MACRO1}
 and the strange event, and for many explanations and clarifications 
 concerning the MACRO experiment without which this article would not have 
 seen the light of the day. K.A.~Meissner was partially 
 supported by the Polish National Science Center grant UMO-2020/39/B/ST2/01279.
 The work of  H.~Nicolai has received funding from the European Research 
 Council (ERC) under the  European Union's Horizon 2020 research and 
 innovation programme (grant agreement No 740209). 
 
  
\vspace{0.8cm}


\begin{thebibliography}{99}

\bibitem{MeissnerNicolai2019}  K.A.~Meissner and H.~Nicolai, Phys. Rev. {\bf D100} (2019) 035001

\bibitem{MACRO} M. Ambrosio et al. (MACRO coll.), Eur.Phys.J. {\bf C25} (2002) 511 

\bibitem{MACRO1} M.~Ambrosio {\em et al}, {\it Final search for lightly ionizing particles with
    the MACRO detector}, {\tt arXiv:hep-ex/0402006}

\bibitem{GM} M.~Gell-Mann, in Proceedings of the 1983 Shelter Island Conference on
    Quantum Field Theory and the Fundamental Problems of Physics, eds. R.~Jackiw, N.N.~Khuri,
    S.~Weinberg and E.~Witten, Dover Publications, Mineola, New York (1985)
       
\bibitem{NW} H.~Nicolai and N.P.~Warner, Nucl. Phys. {\bf B259} (1985) 412

\bibitem{MN2} K.A.~Meissner and H.~Nicolai, Phys. Rev. {\bf D91} (2015) 065029

\bibitem{KM0}  K.A.~Meissner and H.~Nicolai, Phys. Rev. Lett. {\bf 121} (2018)  091601

\bibitem{KN} A.~Kleinschmidt and H.~Nicolai, Phys. Lett. {\bf B747} (2015) 251

\bibitem{CJ} E. Cremmer and B. Julia, Nucl. Phys. {\bf B159} (1979) 141

\bibitem{dWN} B. de Wit and H. Nicolai, Nucl. Phys. {\bf B208} (1982) 323

\bibitem{milli1} S.  D.  McDermott,  H.  B.  Yu  and  K.  M.  Zurek,  
        Phys.  Rev.  {\bf D83}(2011) 063509
%doi:10.1103/PhysRevD.83.063509 [arXiv:1011.2907 [hep-ph]].

\bibitem{milli2}  A. D. Dolgov, S. L. Dubovsky, G. I. Rubtsov and I. I. Tkachev, 
     Phys. Rev. {\bf D88}(2013)117701 
% [arXiv:1310.2376 [hep-ph]].

\bibitem{NNP} E. Del Nobile, M. Nardecchia and P. Panci, JCAP {\bf 1604} (2016) no.04, 048 


\bibitem{WdB} M. Weber and W. de Boer, Astron.Astrophys. {\bf 509} (2010) A25 

\bibitem{MN3}  K.A.~Meissner and H.~Nicolai,
%"Superheavy gravitinos and ultra-high energy cosmic rays",
JCAP {\bf 09} (2019) 041
%{\tt arXiv:1906.07262[astro-ph.HE]}

\bibitem{RGS} A. De R{\'u}jula, S.L. Glashow and U. Sarid, Nucl.Phys.  {\bf B333} (1990) 173

\bibitem{BJRR} C. Burrage, J. Jaeckel, J. Redondo and A. Ringwald,  JCAP {\bf 0911} (2009) 002 

\bibitem{CK} L. Chuzhoy and E.W. Kolb, JCAP {\bf 0907} (2009) 014

\bibitem{Markov} M.A. Markov, JETP {\bf 24} (1967) 584

\bibitem{ACN} Y. Aharonov, A. Casher and S. Nussinov, Phys.Lett {\bf B191} (1987) 51

\bibitem{Sriv} A.M. Srivastava,Phys.Rev. {\bf D36} (1987) 2368

\bibitem{SHDM1} D.J.H.~Chung, E.W.~Kolb and A.~Riotto,
%"Superheavy Dark Matter"
Phys. Rev. {\bf D59} (1998) 023501;
%"On the gravitational production of superheavy dark matter"
Phys. Rev. {\bf D64}(2001)043503

\bibitem{SHDM2} E.W.~Kolb, A.A. Starobinsky and I.I. Tkachev,
%"Trans-Planckian Wimpzillas"
JCAP {\bf 0707} (2007) 005

\bibitem{PIDM} M. Garny, McCullen Sandora and M.S. Sloth,
%"Planckian Interacting Massibve Particles as Dark Matter"
Phys. Rev. Lett. {\bf 116} (2016) 101302

\bibitem{PIDM1}  M. Garny, McCullen Sandora and M.S. Sloth,
%"Phenomenology of Planckain interacting massive particles as dark matter
{\tt arXiv:1709.09688}

\bibitem{SHDM6} K. Kannike, A. Racioppi and M. Raidal,
%"Superheavy dark matter -towards predictive scenarios from inflation"
{\tt arXiv:1605.09378[hep-ph]}

\bibitem{MICA0} P. Price and M. Salamon,
%" Search for supermassive magnetic monopoles using Mica crystals"
Phys. Rev. Lett. {\bf 56} (1986) 1226

\bibitem{MICA} J.F. Acevedo, J. Bramante and A. Goodman,
%"Old rocks, new limits: excavated ancient Mica searches for dark matter
{\tt arXiv:2205.06473[hep-ph]}

\bibitem{MM1} J.~Derkaoui et al.,
Astroparticle Physics {\bf 9} (1998) 173

\bibitem{MM2} J.~Derkaoui et al.,
Astroparticle Physics {\bf 10} (1999) 339

\bibitem{Witten} E. Witten, Phys. Lett. {\bf B86} (1979) 283


\bibitem{Bern} Z. Bern, J.J.~Carrasco, W.M.~Chen, A.~Edison, 
H.~Johansson, J.~Parra-Martinez, R.~Roiban and M.~Zeng,
Phys. Rev. {\bf D98} (2018) 086021

\bibitem{Marcus} N.~Marcus, Phys.Lett. {\bf B157} (1985) 383                                    

\bibitem{Kallosh} J.J.M. Carrasco, R. Kallosh, R. Roiban and A.A. Tseytlin, 
%On the U(1) duality anomaly and the S-matrix of N=4 supergravity 
          JHEP {\bf 1307} (2013) 029 

\bibitem{Zvi} Z. Bern, S. Davies and T. Dennen,
%Enhanced ultraviolet cancellations in N=5 supergravity at four loops 
                  Phys.Rev. {\bf D90} (2014) 105011          

\bibitem{MN4} K.A.~Meissner, and H.~Nicolai, Phys. Lett. {\bf B772} (2017) 169


\end{thebibliography}





\end{document}



