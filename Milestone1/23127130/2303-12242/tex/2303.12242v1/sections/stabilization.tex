\section{Data-Driven Stabilization}
\label{sec:stab}

This section will detail the data-driven positive-stabilization problem and its solution using robust linear programming.

\subsection{Problem Setting}


A set of $T$ observations are recorded of system \eqref{eq:sys} as corrupted by a noise process $w \in \R^n$,
\begin{equation}
    \delta x(t) = A x(t) + B u(t) + w(t).\label{eq:sys_corrupt}
\end{equation}

These observations are collected into the data $\dc = (\bx, \bu, \xd)$ with the expressions,
\begin{align}
\label{eq:data}
    \begin{array}{ccllll}
        \bx & := & [x(0) & x(1) &  \ldots & x(T-1) ]   \\
        \bu & := & [u(0) & u(1) & \ldots & u(T-1)] \\ 
        \xd & := & [\delta x(0) & \delta x(1) &  \ldots & \delta x(T-1) ].
    \end{array}
\end{align}

The discrepancy matrix $\bw$ satisfies the relation,
\begin{equation}
\label{eq:discrepancy}
    \bw = \xd - (A \bx + B \bu).
\end{equation}


The noise model that we will use is that each $w(t)$ (column of $\bw$) is $L_\infty$-norm-bounded by some given $\epsilon \geq 0$ ($\norm{w(t)}_\infty \leq \epsilon)$. 


The set of all system matrices $(A, B)$ that are compatible with the $L_\infty$-corrupted data in $\dc$ forms a polytopic consistency set $\Sigma_{\dc}$. 
% Defining $a = \textrm{vec}(A)$ and $b = \textrm{vec}(B)$, this consistency set may be represented as,
% \begin{equation}
%     \Sigma_\dc = \left\{ (a, b) \mid \begin{bmatrix}
%     \urg{\text{$\otimes$ terms}}
%     \end{bmatrix}\begin{bmatrix} a \\ b\end{bmatrix} = \begin{bmatrix} \urg{\text{$\otimes$ terms}}\end{bmatrix} \right\}
% \end{equation}
If it is known \textit{a priori} that $A$ is Metzler/Nonnegative and/or $B$ is nonnegative, then these constraints in $(A, B)$ may be adjoined to $\Sigma_\dc$.


The data-driven positive-stabilization problem is:
\begin{prob}
\label{prob:stab}
Find a vector $v \in \R^n_{>0}$ and a controller $K \in \scr$ such that $\max(x./v)$ is a common \ac{DLCLF} ensuring positive-stability of 
$A+BK$ for all $(A, B) \in \Sigma_\dc$.
\end{prob}


% \begin{assum}
% Sufficient data $\dc$ is collected such that $\Sigma_\dc$ is a compact set.
% \label{assum:compact}
% \end{assum}

\subsection{Polytope Description}
We will describe $K$-stabilized and $\dc$-consistent polytopes that will be used in solving Problem \ref{prob:stab}
Throughout this section, the column-vectorization of the plant matrices will be defined as $a = \vvec{A}, b = \vvec{B}$. The identity $\vvec{U V W} = (W^T \otimes U) \vvec{V}$ for matrices $(U, V, W)$ of compatible dimensions will be judiciously used in derivations.

\subsubsection{Data-Consistent Polytopes}
% The following matrices may be formed from the data in $\dc$ in order to pose the polytopic form of $\Sigma_\dc$:
% \begin{align}
% \\
%     -\bx^T \otimes I_n & -\bu^T \otimes I_n \\
%     \end{bmatrix} & \beta &= \epsilon + \begin{bmatrix}\vvec{\xd} \\ -\vvec{\xd}\end{bmatrix}. \label{eq:sigma_data}
% \end{align}

The polytopic set $\Sigma_\dc^{\text{data}}$ of plants consistent with the data in $\dc$ may be represented as
% Defining , this consistency set may be represented as,
% \begin{subequations}
% \begin{align}
%     \Sigma_\dc^{\text{data}} = \left\{ (A, B) \mid G^{\textrm{data}}_1 \begin{bmatrix} a^T & b^T\end{bmatrix}^T = \beta \right\}.
% \end{align}
% \end{subequations}
\begin{subequations}
\label{eq:data_single}
\begin{align}
    G^{\textrm{data}}_1 &= \begin{bmatrix} \bx^T \otimes I_n & \bu^T \otimes I_n \end{bmatrix}\\
    \Sigma_\dc^{\text{data}} &= \left\{ (A, B) \mid G^{\textrm{data}}_1 \begin{bmatrix} a \\ b\end{bmatrix} \leq   \begin{bmatrix}\epsilon\1_{nT} +\vvec{\xd} \\ \epsilon\1_{nT} -\vvec{\xd}\end{bmatrix} \right\}.
\end{align}
\end{subequations}

The consistency set of plants $\Sigma_\dc$ is the intersection of $\Sigma_\dc^{\textrm{data}}$ and the prior knowledge on system-positivity of $(A, B)$ (linear constraints) described in $\Sigma^{\text{prior}}$. As an example, where $A$ is a positive system in discrete-time, then $\Sigma^{\text{prior}} = \{A \mid A \in \R^{n \times n}_+\}$, $G_1^{\textrm{prior}} = -I_{n^2}$, and $h_1^{\textrm{prior}} = \0_{n^2}$.
Let $(G_1, h_1)$ be matrices such that the polytopic data-consistency set $\Sigma_\dc = \Sigma^{\textrm{data}}_\dc \cap \Sigma^{\textrm{prior}}$ can be expressed as
\begin{subequations}
\begin{align}
    \Sigma_\dc = P_1 =  \left\{ (A, B) \mid G_1 \begin{bmatrix} a \\ b\end{bmatrix} \leq h_1 \right\}. \label{eq:sigma_dc}
\end{align}
\end{subequations}



\subsubsection{Controller-Stabilizing Polytopes}
\label{sec:stab_poly}
In order to apply the Extended Farkas Lemma \ref{lem:ext_farkas}, we will convert the strict inequalities in \eqref{eq:stab_clean} and in $v \in \R^n_{>0}$ to non-strict inequalities by utilizing a sufficiently small $\eta > 0$.
\begin{subequations}
\label{eq:stab_delta}
\begin{align}
    -(A X + B Y) \1_n -\eta \1_n& \in \R_{\geq 0}  & & \text{(\ac{CTS})} \label{eq:stab_delta_c} \\
    v-(A X + B Y) \1_n -\eta \1_n& \in \R_{\geq 0} & & \text{(\ac{DTS})} \label{eq:stab_delta_d}.
\end{align}
\end{subequations}

Define the canonical Metzler-indexing matrix $M_n \in \R^{n(n-1) \times n^2}$ as a 0/1-valued matrix that extracts off-diagonal elements, such as
\begin{equation}
    M_2 \text{vec} \left(\begin{bmatrix} 1 & 3 \\ 2 & 4\end{bmatrix}\right)= \begin{bmatrix} 2 \\ 3\end{bmatrix}.
\end{equation}

The polytope $P^C_2$ of continuous-time plants $(A, B)$ that can be positive-stabilized via \eqref{eq:stab_delta_c} under a state-feedback controller $K \in \scr$ with \iac{DLCLF} $\max(x./v)$ such that $Y = K X$ can be described by
\begin{subequations}
\label{eq:poly_stab_c}
\begin{align}
    G^C_2 &= \begin{bmatrix} v^T \otimes I_n & (Y\1_n)^T \otimes I_n \\
    -M_n( X \otimes I_n) & -M_n(Y^T \otimes I_n)\end{bmatrix} \\
    P^C_2 &= \left\{(A, B) \mid G^C_2 \begin{bmatrix} a \\ b \end{bmatrix} \leq \begin{bmatrix} -\eta \1_n \\ \0_{n(n-1)} \end{bmatrix}\right\}.
\end{align}
\end{subequations}

The top row of $G^C_2$ is the \ac{DLCLF} stabilization criterion, and the bottom row enforces that $AX+BY$ is Metzler.

The polytope $P^D_2$ of discrete-time plants $(A, B)$ positive-stabilized by $(K, Y)$ under the same conditions is
\begin{subequations}
\label{eq:poly_stab_d}
\begin{align}
    G^D_2 &= \begin{bmatrix} v^T \otimes I_n & (Y\1_n)^T \otimes I_n \\
     -X \otimes I_n & -Y^T \otimes I_n\end{bmatrix} \\
    P^D_2 &= \left\{(A, B) \mid G^D_2 \begin{bmatrix} a \\ b \end{bmatrix} \leq \begin{bmatrix} v -\eta \1_n \\ \0_{n^2} \end{bmatrix}\right\}.
\end{align}
\end{subequations}




\subsection{Stabilizing Programs using the Extended Farkas Lemma}
\label{sec:stab_program}

To unite notation, let $P_2$ be the appropriate stabilizing polytope for continuous-time $(P_2^C)$ or discrete-time $(P_2^D)$ from Section \ref{sec:stab_poly}. The number of constraints in the stabilizing polytope $P_2$ (length of $h_2$) is $q = n + n(n-1)$ for continuous-time and $q = n + n^2$ for discrete-time.
The polytope $P_2$ has a constraint matrix $G_2\in \R^{q \times n(n+m)}$ and vector $h_2 \in \R^{q}$ such that $P_2 = \{(A, B) \mid G_2 [a^T \ b^T]^T \leq h_2\}$. The entries in $G_2$ and $h_2$ are affinely-dependent on $(v, Y)$.

Problem \ref{prob:stab} may be expressed in the language of polytope-containment as,
\begin{prob}
\label{prob:poly_contain}
Find a vector $v \in \R^n_{> 0}$ and a matrix $Y \in \scr$ such that $P_1 \subseteq P_2$.
\end{prob}

\begin{thm}
\label{thm:poly_contain_lp}
Problem \ref{prob:poly_contain} (equivalent to \eqref{prob:stab}) has a solution iff the following \ac{LP} involving variables $(v, Y, Z)$ is feasible:
\begin{subequations}
\label{eq:lp_stab}
\begin{align}
    \find_{v, Y, Z} \qquad & Z G_1 = G_2(v, Y), \qquad Z h_1 \leq  h_2(v, Y) \label{eq:lp_stab_ext}\\
    & v-\eta \1_n \in \R_{\geq 0}^n, \ Y \in \scr, \ Z \in \R_{\geq 0} ^{q \times 2 n T}, \label{eq:lp_stab_var}
\end{align}
\end{subequations}
whereby the state-feedback gain $K \in \scr$ can be recovered by calculating $K = Y X^{-1}$.
\end{thm}
\begin{proof}
The \ac{LP} in \eqref{eq:lp_stab} is a direct application of the Extended Farkas Lemma \ref{lem:ext_farkas} to prove polytope containment $P_1 \subseteq P_2$.
\end{proof}

\subsection{Computational Complexity}

% The \ac{LP} in \eqref{eq:lp_stab} has $n + mn + (2 n T) q$ scalar variables distributed into $(v, Y, Z)$. 
Table \ref{tab:num_cons} computes the number of inequality and equality constraints required to represent Program \eqref{eq:lp_stab_ext}. The number of equality constraints associated with $Y$ is set to 0 because zero-valued entries of $Y$ will be removed and will not be treated as scalar variables.
The \ac{LP} in \eqref{eq:lp_stab} has up to $n + mn + (2 n T) q$ scalar variables distributed into $(v, Y, Z)$, plus $q$ additional nonnegative slack variables required to represent the inequalities in constraint \eqref{eq:lp_stab_ext}.

\begin{table}[h]
    \centering
        \caption{Number of Inequality and Equality constraints in Program \eqref{eq:lp_stab}}
    \label{tab:num_cons}
    \begin{tabular}{c c c}
        & \# Ineq. & \# Eq.\\ \hline
        $v$ & $n$ & 0 \\
        $Y$ & $\leq mn$ & 0\\
        $Z$ & $(2 n T) q$ & 0\\
        \eqref{eq:lp_stab_ext} & $q$ & $q n(n+m)$
    \end{tabular}
\end{table}

In discrete-time with $q = n^2+n$ and no value-restrictions on $K$ ($Y \in \R^{m \times n}$), Program \eqref{eq:lp_stab_ext} will have $N = (2nT+1)(n^2 + n) + (2m + 1)n$ nonnegative scalar variables (representing $Y = Y^+ - Y^-$ where both $Y^+$ and $Y^-$ are nonnegative) and $(n^2+1)n(n+m)$ equality constraints.

% In order to satisfy Assumption \ref{assum:compact}, the number of constraints must satisfy $2 n T > n(n+m)$. As a result, both the number of variables and the number of inequality constraints scales on the order of $\Omega(n^4)$ (lower bound on complexity). 

The running-time of an Interior Point Method solver for \acp{LP} up to $\gamma$-optimality is approximately $O(N^{\omega+0.5} \abs{\log(1/\gamma)})$ \cite{wright1997primal}, where  $\omega$ is the matrix-multiplication constant. Our \ac{DDC} algorithm therefore has performance on the order of $(Tn^3)^{\omega+0.5} \sim n^{12.5}$. Significant gains in performance may be realized by noting that the matrices $(G_1, G_2)$ are sparse and are highly structured.
% $N \sqrt{NM + M}$

\begin{rmk}
\label{rmk:nonredundant_face}
The polytope $\Sigma_\dc$ may possess a large number of redundant faces. These half-space constraints may be removed to improve computational performance without affecting the description of $\Sigma_\dc$. Nonredundant faces may be discovered by linear programming over the polytope \cite{caron1989degenerate}. 
% The number of preserved faces must still remain larger than $n(n+m)$ to ensure Assumption \ref{assum:compact} holds.
\end{rmk}

\begin{rmk}
An alternative approach is to perform vertex enumeration, in which relations \eqref{eq:stab_clean} hold at every vertex of $\Sigma_\dc$. The polytopes $\Sigma_\dc$ that are gathered as part of the data-acquisition process empirically have a number of vertices that scales exponentially with dimension, for which the face-based approach of the Extended Farkas Lemma is more favorable.
\end{rmk}

\begin{rmk}
This paper focused on the case of $L_\infty$-bounded noise. This set-containment framework will also be nonconservative when applied to other with other semidefinite-representable noise processes, such as when each column of the discrepency matrix $\mathbf{W}$ in \eqref{eq:discrepancy} has bounded $L_2$ norm. The Extended Farkas Lemma \ref{lem:ext_farkas} is a specific instance of a more general Robust Counterpart posed over a system of linear inequalities \cite[Theorem 1.3.14]{ben2009robust}.  In the $L_2$ case, each inequality constraint in the polytope in $P_2$ over the uncertain $(a, b)$ is replaced via a robust counterpart by $n(n+m)$ second-order-cone variables, $n(n+m)$ linear equality constraints, and one linear inequality constraint. This procedure is performed programmatically in \cite{lofberg2012robust} under the `duality' option.
% We note that other semidefinite-representable noise processes (e.g. $L_2$) may be more appro
\end{rmk}

% Given that the running-time of an Interior Point Method to solve an \ac{LP} in standard form is 



% The variable $Z$ obeys $q \times 2 nT$ inequality constraints, and $Y$ possesses  up to $m n$ inequality constraints depending on if the pattern $\scr$ contains $\oplus$ or $\ominus$ elements.

% \urg{Briefly note about the complexity of solving the LPs. Number of variables, constraints. Discuss that the number of nonredundant faces is typically far fewer than the number of vertices. }






% The \ac{LP} that solves Problem \ref{prob:stab} 

% \begin{rmk}
% Problems \ref{prob:stab_c} and \ref{prob:stab_d} can be optionally normalized by setting $\1_n^T v= 1$. This convention will be used in all experiments. 
% \end{rmk}

% \subsection{Linear Program using the Extended Farkas Lemma}

% \urg{I think that Tianyu's Extended Farkas Lemma might be cleaner here.}

% Problems \ref{prob:stab_c} and \ref{prob:stab_d} require the satisfaction of a set of linear inequality constraints (in $(v, Y)$) as perturbed by polytope-bounded uncertain parameters $(A, B)$.
% We briefly review how to eliminate the $(A, B)$ parameters using a robust counterpart,

% \begin{lem}
% Let $\zeta \in \R^L$ be an uncertain parameter restricted to the non-empty compact polytope $\Sigma_\zeta = \{\zeta \mid C \zeta \leq d\}$ with $C \in \R^{q \times L}, \ d \in \R^q$. 
% Define constraint  matrices $A_j \in \R^{m \times n}, \ b_j \in \R^n$ for $j=0..L$ to form the robust system of inequalities in $\gamma \in \R^n$
% \begin{align}
%     \textstyle (A_0 + \sum_{j=1}^L \zeta_j A_j)\gamma \leq (b_0 + \sum_{j=1}^L \zeta_j b_j) & & \forall \zeta \in \Sigma_\zeta. \label{eq:robust_with_uncertainty}
% \end{align}
% Defining $ C^{\text{row}}_j$ as the $j$-th row of $C$, a feasible $\gamma$ exists for \eqref{eq:robust_with_uncertainty} if and only if a feasible $\gamma$ exists for the following robust counterpart \cite{ben2009robust},
% \begin{align}
%     & \exists \Lambda \in \R^{q \times L} \\
%     & b_0 + \Lambda^T d - A_0 \gamma \in \R^{m}_{\geq 0} \\
%     & b_j - A_j \gamma = \Lambda^T C^{\text{row}}_j & \forall j=1..m.
% \end{align}

% \end{lem}

% \urg{Form the robust linear program, based on the problem setting. Describe the robust counterparts and algorithms in continuous-time and discrete-time.}
