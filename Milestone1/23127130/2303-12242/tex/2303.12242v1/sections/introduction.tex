\section{Introduction}
\label{sec:introduction}


This paper performs \ac{DDC} of Positive \ac{LTI} \acp{CTS} and \acp{DTS} by finding full-state-feedback stabilizing controllers. These controllers, which stabilize all possible plants that are consistent with observed data, are formulated as the solution to \iac{LP}. 

%Positive Systems
Positive systems are a class of dynamical systems whose state and output responses to positive (nonnegative) initial conditions and inputs remain positive (nonnegative) 
 for all time \cite{luenberger1979dynamic, berman1989nonnegative, farina2000positive, kaczorek2012positive}. Instances of positive systems include population models \cite{hirsch1988systems}, chemical networks \cite{blanchini2014piecewise}, radio communications \cite{ZAPPAVIGNA2012219}, queuing \cite{shorten2006positive}, 
% image processing, 
and Markov chains \cite{seneta2006non}. 
% Positivity properties may also be defined for input-output relations: an externally positive system has a positive output value when the input remains positive with no sign-requirement on the state, while an internally positive system additionally requires that the state remains positive \cite{farina2000positive}.
Full-state-feedback stabilization of known \ac{LTI} positive systems can be accomplished by solving \iac{LP} to find control  (dual) linear copositive Lyapunov functions \cite{rami2007controller}.
Alternatively, one can perform stabilization by formulating \iac{SDP} to find a quadratic Lyapunov function \cite{gao2005control, shafai2013stabilityradius}.

% to formulate a control \acp{DLCLF}, as simplified from a previously developed \ac{SDP} \cite{gao2005control}.
The peak-to-peak ($L_\infty \rightarrow L_\infty$ for \iac{CTS} or $\ell_\infty \rightarrow \ell_\infty$ for \iac{DTS}) gain of an extended positive plant can be calculated and regulated using \iac{LP} \cite{ebihara2011l1, briat2013positive, ebihara2013stability, naghnaeian2014linfinity}, which has also been derived using stability radius formulas \cite{shafai2019stability}.
Analysis and stabilization results can be extended to uncertain and switched positive systems \cite{blanchini2015switched}, as well as time-delay positive systems \cite{shafai2014positive}. The tutorial in \cite{rantzer2018tutorial} is a  survey of topics about stabilization and performance regulation for positive linear systems.
% Monotone dynamical systems are closely tied to positive systems, be


% Data Driven Control
\ac{DDC} is a method that synthesizes controllers for a class of data-consistent plants without first performing a possibly expensive and inaccurate system identification step \cite{HOU20133, hou2017datasurvey}. 
Methods that require a reference signal include iterative feedback tuning \cite{hjalmarsson1998iterative}, virtual reference feedback tuning \cite{campi2002virtual}, \cite{bazanella2011data}, and correlation-based tuning \cite{karimi2004iterative}, but these algorithms lack stability guarantees for all consistent systems.
Data-driven predictive control through input-output data can be accomplished through Willem's Fundamental Lemma, assuming that a rank condition of the Hankel matrices is satisfied (persistency of excitation) \cite{willems2005note}. Stabilization,  worst-case-optimal control, and Model Predictive Control problems can be solved through the use of this Lemma \cite{waarde2020informativity, depersis2020formulas, coulson2019data, berberich2020mpc}, but the Lemma is vulnerable to noise sensitivity (even with regularization).

Prior knowledge of noise characteristics can be employed to synthesize controllers that will stabilize all plants that are consistent with data. $L_\infty$-bounded noise arises from  bounds on the time-derivative of the state (\ac{CTS}) or discretization of continuous-time finite-difference approximations (\ac{DTS}). Work addressing \ac{DDC} of $L_\infty$-bounded noise by solving \acp{LP} includes \cite{cheng2015} using an Extended Farkas Lemma \cite{hennet1989farkas}. Tools from polynomial optimization may be applied to the $L_\infty$ setting, such as for quadratic stabilization \cite{dai2020data}, switched systems \cite{dai2018moments, dai2018data}, and error-in-variables control \cite{miller2022eiv_short, miller2022eivarx}.
Quadratic Matrix Inequalities may be used to represent consistency sets (including energy-based or $L_2$-bounded noise) \cite{waarde2020noisy, berberich2020combining}, and stabilizing controllers may be synthesized by solving \acp{SDP} using a Matrix S-Lemma \cite{yakubovich1997s}. The work in \cite{martin2021data} employs polynomial optimization for \ac{DDC} under the assumption that magnitude bounds on Taylor polynomial coefficients and residual terms are known.


The work in \cite{shafai2022data} utilizes the Fundamental Lemma \cite{willems2005note} to perform \ac{DDC} of positive systems by solving \iac{SDP}. System identification of positive systems is performed in \cite{grussler2017identification}. The method in \cite{bianchi2022data} uses data-driven Lyapunov-Metzler inequalities to perform switched positive-systems control at the expense of solving Bilinear Matrix Inequalities.

% be employed to perform control over all systems consistent noise des


% , dai2018data,DAI20203965}.



%Data Driven Control and Positive Systems
% Another noise model for \ac{DDC} includes the  $L_\infty$-bounded noise (polytopic) case,  arising from finite-difference approximations in the conversion of continuous-time signals to discrete-time. Work addressing \ac{DDC} of $L_\infty$-bounded noise includes \cite{cheng2015, dai2018data,DAI20203965}.



% \urg{Introduction goes here. Examples of positive systems.}

% \urg{Literature review here. Discuss positive systems and \ac{DDC} methods.}


