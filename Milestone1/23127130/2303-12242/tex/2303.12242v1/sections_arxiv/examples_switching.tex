\section{Numerical Examples}

\label{sec:examples}

MATLAB 2021a Code to reproduce the experiments is available at \url{https://github.com/jarmill/data_driven_pos}, and includes Mosek \cite{mosek92} and YALMIP \cite{lofberg2004yalmip} dependencies. All provided experiments have parameters of $\eta = 10^{-3}$ and $\epsilon = 0.1$.
 
\subsection{Continuous-Time Stabilization}

The ground-truth continuous-time system in this example has $n=3$ inputs and $m=2$ outputs
\begin{align}
\label{eq:cont_stab_sys_true}
    A &= \begin{bmatrix} -0.55 & 0.3 & 0.65\\ 0.06 & -1.35 & 0.25 \\ 0.1 & 0.15 & 0.4   \end{bmatrix}& B = \begin{bmatrix} 0.18 & 0.08 \\ 0.47 & 0.25 \\0.07 & 0.95 \end{bmatrix}.
\end{align}

System \eqref{eq:cont_stab_sys_true} is internally positive but is open-loop unstable (poles of $0.4907, -0.6055, -1.3851$). The stabilization task in \eqref{eq:lp_stab} with $T=5$ and an additional normalization constraint that $\1_n^T v = 1$ results in
\begin{subequations}
 \begin{align}
     v &= \begin{bmatrix}0.5570 & 0.1401& 0.3029\end{bmatrix}^T \label{eq:example_cont_stab_v}\\
    K &= \begin{bmatrix} 0.0279 &    -0.2660 &  0.5041 \\
    0.0107  &  -0.0222 & -0.8650\end{bmatrix}. \label{eq:example_cont_stab_K}
 \end{align}
\end{subequations}
%      Y &=  \begin{bmatrix} 0.0155  &  -0.0373  &  0.1527 \\
    %  0.0060  & -0.0031 & -0.2620 \end{bmatrix}\\

Figure \ref{fig:cont_stab} visualizes 100 controlled trajectories (red curves) starting from $x(0)=[1;1;1]$ (black circle). Each trajectory follows $\dot{x}(t) = (A+ BK)x(t)$ in the times $t \in [0, 20],$ where the plants $(A, B)$ are randomly sampled from $\Sigma_D$ and $K$ is the controller in \eqref{eq:example_cont_stab_K}.

\begin{figure}[h]
    \centering
    \includegraphics[width=0.5\linewidth]{img/sys_3_2_cont_T5.png}
    \caption{Application of the controller $u=Kx$ from \eqref{eq:example_cont_stab_K} to positively-stabilize $100$ consistent systems in $\Sigma_D$.}
    \label{fig:cont_stab}
\end{figure}

Figure \ref{fig:cont_max_lyap} plots values of the Lyapunov function $\max(x./v)$ (for the $v$ in \eqref{eq:example_cont_stab_v}) along the 100 systems in \ref{fig:cont_stab}.
\begin{figure}[h]
    \centering
    \includegraphics[width=0.75\linewidth]{img/Lyap_max.pdf}
    \caption{\ac{DLCLF} along the 100 trajectories.}
    \label{fig:cont_max_lyap}
\end{figure}

\subsection{Discrete-Time Stabilization}

This example involves a discrete-time system with $n=5$ states and $m=3$ inputs. The ground-truth system is internally positive, and is unstable with poles of $1.3094, \ -0.1218 \pm 0.0992 \mathbf{j}, \-0.1201 \pm 0.1108 \mathbf{j}$.  With $T=60$ observations the following \ac{DLCLF} and stabilizing controller is recovered
\begin{align}
v &= \begin{bmatrix}0.2076& 0.1212&   0.2651&   0.2516 & 0\end{bmatrix}^T     \\
K &= \begin{bmatrix}0.0483 &   0.0088  & -0.1326  & -0.0188  & -0.4273\\
   -0.3243  &  0.0115  &  0.0299 &  -0.2980 &   0.0337\\
    0.1601 &   0.0749  & -0.5962 &  -0.3537 &  -0.2194 \end{bmatrix}. \nonumber
\end{align}
 
 
%  \urg{Add an experiment with sign patterns}
 
% A stabilizing controller for all consistent plants that obeys the sign pattern
It is now desired to obtain a stabilizing controller for all consistent plants that obeys the sign pattern
\begin{subequations}
\begin{align}
    \scr &=  \begin{bmatrix}\odot &   \odot  & \odot  & \odot  & \ominus\\
   \odot  &  \odot  &  \circledast & \odot &   \oplus\\
    \odot &   \odot  & \odot &  \circledast &  \circledast \end{bmatrix}
\intertext{Such  \iac{DLCLF} certificate  and controller is}
v &= \begin{bmatrix} 0.2147&   0.1259&    0.2448&     0.2516&     0.1630\end{bmatrix}^T\\
K &= \begin{bmatrix}0&   0  & 0  & 0  & -0.6853\\
   0  &  0  &  -0.3206 & 0 &   0.1206\\
    0 &   0  & 0 &  -0.5604 &  -0.3317 \end{bmatrix}.
\end{align}

\end{subequations}
\subsection{Continuous-Time Peak-to-Peak}

The following ground-truth positive-stable continuous-time system has $n=3$ inputs and $m=2$ outputs
\begin{align}
\label{eq:p2p_sys_true}
    A &= \begin{bmatrix} -0.2 & 0.2 & 0.2\\ 0.4 & -0.7 & 0.2 \\ 0 & 0.8 & -3   \end{bmatrix}& B = \begin{bmatrix} -0.4 & 0.5 \\ 0.2 & -0.8 \\ -1 & 2 \end{bmatrix}.
\end{align}

This system has  $e=2$ external input channels and $p=5$ controlled outputs with
\begin{align}
\label{eq:p2p_param}
    C &= \begin{bmatrix} I_3 \\ \0_{2 \times 3}
    \end{bmatrix}, & D &= \begin{bmatrix} \0_{3 \times 2} \\ I_2
    \end{bmatrix}, & E  &= \begin{bmatrix} I_2 \\ \0_{1 \times 2}
    \end{bmatrix}, & F  &= \0_{5 \times 2}.
\end{align}
 
The peak-to-peak gain of the ground-truth \eqref{eq:p2p_sys_true} under the parameters in \eqref{eq:p2p_param} when uncontrolled $(K=\0_{2 \times 3})$ is $\gamma^* = 32.178$. Lemma \ref{lem:p2p_clean} synthesizes a controller for the ground-truth system resulting in a gain of $\gamma^* = 3.742$. The constraint $CX + DY \in \R^{q \times n}_{\geq0}$ with the values in \eqref{eq:p2p_param} imposes that all elements of $Y$ and $K$ are nonnegative $(\oplus)$.

Table \ref{tab:p2p_result} collects the worst-case peak-to-peak gains obtained by \eqref{eq:p2p} as a function of the number of samples $T$. These gains decrease as $T$ increases and the consistency set $\Sigma_\dc$ shrinks. The top row of \eqref{tab:p2p_result} incorporates the prior knowledge that the ground-truth $A$ from \eqref{eq:p2p_sys_true} is Metzler when constructing the polytope $\Sigma_D$. The bottom row does not impose this positivity (Metzler) prior on $A$, and therefore yields peak to peak bounds that are always greater than or equal to the Metzler-imposed bounds.
 
 
 \begin{table}[h]
     \centering 
     \caption{Worst-case peak-to-peak gain $\gamma^*$ computed by \eqref{eq:p2p} decreases as the number of samples $T$ increases}
     \begin{tabular}{rccccc}
     $T$& 20 & 30 & 50 & 80 & 120 \\ \hline
          $A$ Metzler & 6.4539 & 5.0182 & 4.4967 & 4.0619 & 4.0028 \\
          No Prior& 6.4823 & 5.0719 & 4.5292 & 4.0659 & 4.0029
     \end{tabular}
     \label{tab:p2p_result}
 \end{table}
 
The system with $T=50$ and a Metzler-prior on $A$ has a worst-case peak-to-peak gain of $\gamma^*=4.4967$ and solution outputs of
\begin{subequations}
 \begin{align}
     v &= \begin{bmatrix}4.4967 &4.2021 & 0.4303\end{bmatrix}^T \\
    K &= \begin{bmatrix} 0.5095 &    0.4765  &  0.4727 \\
    0.2587  &  0 & 0\end{bmatrix}.
 \end{align}
\end{subequations}
    %  Y &=  \begin{bmatrix} 2.2910  &  2.0023  &  0.2034 \\
    % 1.1634  & 0 & 0 \end{bmatrix}\\
 
The polytope $\Sigma_D$ under the Metzler-prior has $2nT + (n^2-n) = 300 + 6 = 306$ faces and 308,672 vertices, of which 62 faces are nonredundant (see Remark \ref{rmk:nonredundant_face}). The nonnegative Farkas matrix is $Z \in \R^{9 \times 62}_{\geq 0}$.
% \subsection{Continuous-Time}


% \subsection{Discrete-Time}


\subsection{Switched System Control}

This example will involve a continuous-time system with $n=3$ inputs, $m=2$ outputs, and $N = 2$ subsystems
\begin{align}
    A_1^{\textrm{true}} &= \begin{bmatrix} -0.55 & 0.3 & 0.65\\ 0.06 & -1.35 & 0.25 \\ 0.1 & 0.15 & 0.4   \end{bmatrix}& B_1^{\textrm{true}} &= \begin{bmatrix} 0.18 & 0.08 \\ 0.47 & 0.25 \\0.07 & 0.95 \end{bmatrix} \nonumber\\
    A_2^{\textrm{true}} &= \begin{bmatrix} 0.1& 0.1 & 0.1\\ 0.1 & -1.9 & 0.15 \\ 0.1 & 0.1 & 0.6   \end{bmatrix}& B_2^{\textrm{true}} &= \begin{bmatrix} 1 & 0\\ 0 & 0 \\0 & 1 \end{bmatrix} \label{eq:three_state_common}. 
\end{align}


A set of $T=55$ observations of system \eqref{eq:three_state_common}, 28 of which in 
are $s=1$ and the remaining 27 in $s=2$. The polytope $P_1^{c}$ has $N n(n+m) = 30$ dimensions, 108 nonredundant faces, and 246 redundant faces. 

The recovered controller (and \ac{DLCLF} vector) that simultaneously stabilizes both systems in \eqref{eq:three_state_common} (Section \ref{sec:common_controller}) are
\begin{subequations}
\begin{align}
    v &= \begin{bmatrix}
        0.4989 & 0.0572 & 0.4439
    \end{bmatrix} \\
    K &= \begin{bmatrix}        
   -0.1390 &  -0.0860  & -0.0663\\
    0.0362 &  -0.0810  & -0.8146
    \end{bmatrix} \label{eq:three_state_common_K}.
\end{align}
\end{subequations}
Figure \ref{fig:cont_common} plots controlled trajectories of the system in \eqref{eq:three_state_common} with the gain in \eqref{eq:three_state_common_K} starting from $x(0) = [0.5, 1.5, 1]$. The switching time of each trajectory to a new subsystem is exponentially distributed with a mean of 0.3 time units. The red trajectory on the left subplot highlights the ground truth system in \eqref{eq:three_state_common}, and the other blue curves are trajectories of 15 subsystems inside $P_1^\textrm{c}$ when the identical switching sequence is applied. The right subplot overlays trajectories of 30  switching sequences. 

\begin{figure}[ht]
    \centering
    % \includegraphics[width=0.63\linewidth]{img/param_seq_box2_cont_2_trace1.png}
      \includegraphics[width=0.85\linewidth]{img/sys_switch_cont_3.png}
    \caption{Controlled switched trajectories using the gain in \eqref{eq:three_state_common_K} }
    \label{fig:cont_common}
    % \vspace{-1cm}
\end{figure}

Figure \ref{fig:cont_common} is generated with $T=55$ datapoints. When only $T=20$ observations are collected, it is infeasible to find a common \ac{DLCLF} and controller. However, a common \ac{DLCLF} and a pair of subsystem controllers that can stabilize both systems in \eqref{eq:three_state_common} (Section \ref{sec:different_controller}) are
 
\begin{subequations}
\begin{align}
    v &= \begin{bmatrix}
        0.5423 & 0.1327 & 0.3250
    \end{bmatrix} \\
    K_1 &= \begin{bmatrix}        
    0.0444 &  -0.3097  &  0.4207 \\
   -0.0010 &   0.2910  & -1.0869
    \end{bmatrix} \\
    K_2 &= \begin{bmatrix}        
   -0.4223  &  0.1510  &  0.1520 \\
    0.0059  & -0.0171  & -0.9607
    \end{bmatrix}\label{eq:three_state_diff_K}.
\end{align}
\end{subequations}

Figure \ref{fig:cont_diff} plots switching-aware controlled trajectories of \eqref{eq:three_state_common}   based on \eqref{eq:three_state_diff_K}.
\begin{figure}[ht]
    \centering
    % \includegraphics[width=0.63\linewidth]{img/param_seq_box2_cont_2_trace1.png}
      \includegraphics[width=0.85\linewidth]{img/sys_switch_cont_diffK.png}
    \caption{Controlled switched trajectories using the subsystem-dependent gains in \eqref{eq:three_state_diff_K} }
    \label{fig:cont_diff}
    % \vspace{-1cm}
\end{figure}

% \subsection{Switching-Dependent Control}

\subsection{LPV System}

The considered ground truth continuous-time system with a parameter set
% $n=m=L=2$ with 
$\Theta = \{1\} \times  [-1, 1] \times [-0.5,0.9]$ is
\begin{align}
    A_1^{\textrm{true}} &= \begin{bmatrix} -0.9190  & 0.5555 \\ 0.4936 & -0.5761 \end{bmatrix}, \nonumber &     A_2^{\textrm{true}} &= \begin{bmatrix} -1.2653  & 0.0574 \\ 0.2981 & 0.2455 \end{bmatrix} \nonumber\\
    A_3^{\textrm{true}} &= \begin{bmatrix} 0.9328  & 0.5702 \\ 0.0636 & -1.0487 \end{bmatrix}, & B^{\textrm{true}} &= \begin{bmatrix} 0.4570 & 0.2828 \\ 0.2115 & 0.8863 \end{bmatrix}.\label{eq:two_state_lpv}
\end{align}

This internally positive system has parameters $n = 2, m = 2$ and $L = 3$.
% All $A$ matrices in \eqref{eq:two_state_lpv} are Metzler and the $B$ matrix is nonnegative. 
The plant matrix corresponding to the vertex $\omega = [1, -1, 0.9] \in \Theta$ is unstable, because $A_1^{\textrm{true}} - A_2^{\textrm{true}} + A_3^{\textrm{true}}$ has a positive eigenvalue of $1.2700$.

Data with a horizon of $T=10$ (10 state-input-transition tuples) was collected with $\epsilon =0.1$. The polytope $\Sigma_{LPV}^{\textrm{data}}$ from \eqref{eq:lpv_consistent} has $n(Ln + m) = 16$ dimensions,  42 nonredundant faces, 8 redundant faces, and 607590 vertices.


The following continuous-time vertex controllers positive-stabilize all consistent $\dc$-plants by \eqref{eq:stab_delta_lpva_c} using the stabilizing polytope \eqref{eq:poly_stab_lpva_c}.

\begin{align}
    K_{(1, -1, -0.5)} &= \begin{bmatrix}
  -14.2950  &  9.9057 \\
    6.2326  & -5.8745
    \end{bmatrix}   \nonumber\\  
    K_{(1, -1, 0.9)} &= \begin{bmatrix}
     -20.8043  &  9.7975 \\
    8.5165  & -6.9282
    \end{bmatrix} \nonumber \\
        K_{(1, 1, -0.5)} &= \begin{bmatrix}
   -6.9969 &   6.5542 \\
    2.5340 &  -4.3078
    \end{bmatrix} \label{eq:two_state_k} \\      K_{(1, 1, 0.9)} &= \begin{bmatrix}
   -5.2847  &  2.3813 \\
    2.0480  & -2.3017
    \end{bmatrix}. \nonumber
\end{align}

These controllers were synthesized under the prior knowledge $(\Sigma_{\textrm{prior}})$ that each ground-truth $\{A_\ell\}$ is Metzler and $B$ is nonnegative.

The controllers in \eqref{eq:two_state_k} have an associated common \ac{DLCLF}
\begin{align}
   V(x) &= \max(x_1/0.4482, x_2/0.5518). 
\end{align}

Figure \ref{fig:param_seq} plots system trajectories starting from the black-circle point $x(0) = [0.5, 1.5]$. Parameter values in $\theta$ are drawn uniformly from the box $\Theta$, and change values at mean-$0.05$ exponentially distributed switching times. The top plot of Figure \ref{fig:param_seq} plots controlled trajectory execution for the ground truth in \eqref{eq:two_state_lpv} (red) as well as 15 other systems randomly drawn from $P_1^{\textrm{c,LPV}}$. The bottom plot displays controlled trajectories arising from 30 parameter-switching sequences for each of the 16 sampled systems.



\begin{figure}[ht]
    \centering
    % \includegraphics[width=0.63\linewidth]{img/param_seq_box2_cont_2_trace1.png}
      \includegraphics[width=0.6\linewidth]{img/sys_lpv_cont.png}
    \caption{Controlled \ac{LPV} trajectories using the gains in \eqref{eq:two_state_k} }
    \label{fig:param_seq}
    % \vspace{-1cm}
\end{figure}