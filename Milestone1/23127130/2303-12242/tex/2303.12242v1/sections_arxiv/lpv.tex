\section{Positive Linear Parameter-Varying Systems}
\label{sec:lpv}

% \urg{To be done finished.}

This section will perform data-driven positive-stabilzation for a class of \ac{LPV} systems. The presented approach is similar to subsystem-aware switched systems stabilization from Section \ref{sec:different_controller}.

\subsection{Problem Setting}

The \ac{LPV} framework involves parameters $\theta$ restricted to a known set $\Theta \subset \R^L$ that are measured on-line during operation. The general \ac{LPV} structure is
\begin{align}    \delta x &= A(\theta) x + B(\theta) u. \label{eq:lpv_general}
\intertext{We will focus on the \acf{LPVA} structure \cite{besselmannlofberg2012}, involving a set of matrices $\forall \ell: A_\ell \in \R^{n \times n}$ and a constant $B \in \R^{n \times m}$}
    \delta x &= \textstyle \left( \sum_{\ell=1}^L A_\ell \theta_\ell\right )x + B u. \label{eq:LPVA}
\end{align}
% A trajectory of \eqref{eq:lpv_general} will be expressed as 
The open-loop system $\delta x = A(\theta) x$ is asymptotically stable if $\lim_{t\rightarrow \infty} x(t)=0$ for all possible parameter sequences $\theta(\cdot)$ taking values inside $\Theta$.

Data will be collected from \eqref{eq:lpv_general} with an $L_\infty$ noise bound of $\epsilon$. The observed data $\dc$ with $T$ records is 
\begin{align}
\label{eq:data_lpv}
    \begin{array}{ccllll}
        \bth & := & [\theta(0) & \theta(1) &  \ldots & \theta(T-1) ]\\
        \bx & := & [x(0) & x(1) &  \ldots & x(T-1) ]   \\
        \bu & := & [u(0) & u(1) & \ldots & u(T-1)] \\ 
        \xd & := & [\delta x(0) & \delta x(1) &  \ldots & \delta x(T-1) ]. \\
    \end{array}
\end{align}
The data in \eqref{eq:data_lpv} is collected into $\dc = (\bth, \bx, \bu, \xd)$.

Our goal is to find \iac{DLCLF} with parameter $v>0$ and a control policy $u(t) = K(\theta(t)) x(t)$ such that the closed-loop \ac{LPVA} system $\delta{x} = A(\theta) + BK(\theta)$ is stable and positive (Metzler for \ac{CTS} or Nonnegative for \ac{DTS}) for all $\theta \in \Theta$.

We will assume that there exists a finite $N_c \in \N$ and a bounded discrete set of points $\Omega = \{\omega_c\}_{c=1}^{N_c} \in \Theta$ such that $\Theta$ equals the convex hull of $\Omega$. We will refer to $\Omega$ as the set of `vertices' of $\Theta$. The controller will have knowledge of $\Omega$ and $\theta$ during operation.

\subsection{Data-Consistency Polytope}

Expression of the \ac{LPVA} data-consistency polytope will use the column-wise Khatri-Rao product for matrices $M_1 \in \R^{m \times n}, \ M_2 \in \R^{p \times n}$ (notated as $\kcol$)
\cite{khatri1968solutions}
\begin{equation}
    \label{eq:kron_col}
    M_1 \kcol M_2 = (\1_{p \times 1} \otimes M_1) \odot (M_2 \otimes \1_{m \times 1}).
\end{equation}

Data consistency of the \ac{LPVA} plant $\{A_\ell, B\}$ from \eqref{eq:LPVA} with $L_\infty$-norm error $\epsilon$ w.r.t. $\dc$ requires that
\begin{align}
    \forall t&\in0..T-1: & \norm{\dx(t) - \textstyle (\sum_{\ell=1}^L A_\ell \theta(t)_\ell) - B u(t)}_\infty \leq \epsilon. \label{eq:lpva_consistent_linf}
\end{align}

Define $a_\ell$ as $\vvec{A_\ell}$ for all $\ell=1..L$. Constraint \eqref{eq:lpva_consistent_linf} is equivalent to requiring that all $T$ columns of the following matrix $\bw$ have $L_\infty$-norm $\leq \epsilon$:
\begin{align}
\label{eq:lpv_noise}
    \bw &= \xd - \textstyle \left(\sum_{\ell=1}^L \bth_\ell \kcol A_\ell\right) \xn- B \bu.
\end{align} 

The data-consistent \ac{LPVA} polytope arising from \eqref{eq:lpv_noise} is
\begin{subequations}
\label{eq:lpv_consistent}
\begin{align}
    G^{\textrm{data}}_{LPV} &= \begin{bmatrix} (\bx \kcol \bth)^T \kron I_n & \bu^T \kron I_n\end{bmatrix} \\
    \Sigma^{\textrm{data}}_{LPV} &= \left\{ \{A_\ell, B\} \mid G^{\textrm{data}}_{LPV}  \begin{bmatrix} a_1^T, \ a_2^T, \ \ldots, \  a_L^T, b_T\end{bmatrix}^T \leq   \begin{bmatrix}\epsilon\1_{nT} +\vvec{\xd^s} \\ \epsilon\1_{nT} -\vvec{\xd^s}\end{bmatrix} \right\}
\end{align}
\end{subequations}

\subsection{Stabilizing Polytope}

Positive-Stabilization of consistent \ac{LPVA} will occur using a gain-scheduled controller \cite{rugh2000research} based on vertex-interpolation \cite{apkarian1995self}. 

The $\theta$-dependent control gain $K(\theta)$ will be constructed from a linear combination of controllers $\{K_c\}_{c=1}^{N_c}$. Each vertex $\omega_c \in \Omega$ has a corresponding vertex-controller $K_c \in \R^{m \times n}$ for every $c=1..N_c$. The control policy $u = K(\theta) x$ at a given $\theta$ will be found by first finding a feasible solution $\beta$ to the following \ac{LP}
\begin{subequations}
\label{eq:gain_sched}
\begin{align}
    \textrm{find} \ \beta & \in \R^{N_c}_+ &  \textstyle \sum_{c=1}^{N_c} \beta_c & = 1 &\textstyle   \sum_{c=1}^{N_c} \beta_c \omega_c & = \theta,
    \label{eq:interp_cert}
\end{align}
and subsequently returning the linear combination
\begin{align}
     K(\theta) &= \textstyle \sum_{c=1}^{N_c} \beta_c K_c & u &= K(\theta) x. \label{eq:gain_sched_out}
\end{align}
\end{subequations}

We define the vertex-plant $A_c$ corresponding to $\omega_c \in \Omega$ as
\begin{align}
    A_c &= \textstyle \sum_{\ell=1}^L \omega_{v \ell} A_\ell & & \forall v\in1..N_c
\end{align}



\begin{thm}
\label{thm:positive_stable}
The \ac{LPVA} system \eqref{eq:LPVA} is positive-stabilized by the gain-scheduled feedback controller $u=K(\theta) x$ from \eqref{eq:gain_sched_out} if there exists \iac{DLCLF} with $v > 0$, a matrix $X = \diag{v}$, a tolerance $\eta>0$, and matrices $Y_c \in \R^{m \times n}$ such that $\forall c=1..N_c:$
\begin{subequations}
\begin{align}
    -(A_c X + B Y_c) \1_n -\eta \1_n& \in \R_{\geq 0}^n  & & \text{(\ac{CTS})} \label{eq:stab_delta_lpva_c} \\
    v-(A_c X + B Y_c) \1_n -\eta \1_n& \in \R_{\geq 0}^n & & \text{(\ac{DTS})} \label{eq:stab_delta_lpva_d}.
\end{align}
The subsystem controllers $\{K_c\}_{c=1}^{N_c}$ may be recovered by $K_c = Y_c X^{-1}$.
\end{subequations}
\end{thm}
\begin{proof}
The \ac{LPVA} framework may be interpreted as switching in the sense of Theorem \ref{thm:diff_sys} between systems $(A_c, B)$ for $c\in1..N_c$. Compatibility between imposing positive-stabilization conditions on the vertices $\Omega$ and requiring the property to hold $\forall \theta \in \Theta$ is assured by Lemma 2.1 of \cite{miller2022lpvqmi}.
\end{proof}


The continuous-time \ac{LPVA} stabilization polytope from \eqref{eq:stab_delta_lpva_c} is
\begin{subequations}
\label{eq:poly_stab_lpva_c}
\begin{align}
    G^C_{2c}  &= \begin{bmatrix} \omega_c^T \kron (v^T \otimes I_n) & (Y\1_n)^T \otimes I_n \\
    -M_n( \omega_c^T \kron (X \otimes I_n) )& -M_n(Y^T \otimes I_n)\end{bmatrix} \\
    P^C_{2 LPV} &= \left\{(\{A_\ell\}, B) \mid \ \forall c=1..N_c: \ G^C_{2c} \begin{bmatrix} a_1^T, a_2^T, \ldots, a_L^T,  b^T \end{bmatrix}^T \leq \begin{bmatrix} -\eta \1_n \\ \0_{n(n-1)} \end{bmatrix} \right\}.
\end{align}
\end{subequations}

The discrete-time \ac{LPVA} stabilization polytope from \eqref{eq:stab_delta_lpva_d} is
\begin{subequations}
\label{eq:poly_stab_lpva_d}
\begin{align}
    G^D_{2c}  &= \begin{bmatrix} \omega_c^T \kron (v^T \otimes I_n) & (Y\1_n)^T \otimes I_n \\
    -( \omega_c^T \kron (X \otimes I_n) )& -(Y^T \otimes I_n)\end{bmatrix} \\
    P^D_{2 LPV} &= \left\{(\{A_\ell\}, B) \mid \forall c=1..N_c: \ G^D_{2c} \begin{bmatrix} a_1^T, a_2^T, \ldots, a_L^T,  b^T \end{bmatrix}^T \leq \begin{bmatrix} v-\eta \1_n \\ \0_{n^2} \end{bmatrix} \ \right\}.
\end{align}
\end{subequations}

The stabilizing polytopes $P^C_{2 LPV}$ and $P^D_{2 LPV}$ can be used in conjunction with the $\dc$-consistency polytope $\Sigma^{\textrm{data}}_{LPV}$ to form \ac{LPVA} \ac{DDC} \acp{LP} by the Extended Farkas Lemma.