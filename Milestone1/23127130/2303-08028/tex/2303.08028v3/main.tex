% VLDB template version of 2020-08-03 enhances the ACM template, version 1.7.0:
% https://www.acm.org/publications/proceedings-template
% The ACM Latex guide provides further information about the ACM template

% \documentclass[sigconf, nonacm]{acmart} % VLDB format
% \documentclass[sigconf, nonacm]{acmart}
\documentclass[letterpaper,twocolumn,10pt]{article}
\usepackage{usenix-2020-09}
\usepackage{tikz}

% inlined bib file
\usepackage{filecontents}


% %% The following content must be adapted for the final version
% % paper-specific
% \newcommand\vldbdoi{XX.XX/XXX.XX}
% \newcommand\vldbpages{XXX-XXX}
% % issue-specific
% \newcommand\vldbvolume{14}
% \newcommand\vldbissue{1}
% \newcommand\vldbyear{2020}
% % should be fine as it is
% \newcommand\vldbauthors{\authors}
% \newcommand\vldbtitle{\shorttitle}
% % leave empty if no availability url should be set
% \newcommand\vldbavailabilityurl{https://github.com/swjz/EdgeServe/}
% % whether page numbers should be shown or not, use 'plain' for review versions, 'empty' for camera ready
% \newcommand\vldbpagestyle{plain} 

% Recommended, but optional, packages for figures and better typesetting:
\usepackage{microtype}
\usepackage{graphicx}
% \usepackage{subfigure}
\usepackage{caption}
\usepackage{subcaption}
\usepackage{booktabs} % for professional tables
\let\Bbbk\relax
\usepackage{amssymb}
\usepackage{listings}
\usepackage{array}
\usepackage{xspace} 
\usepackage[pdf]{graphviz}
\usepackage[normalem]{ulem} 
\usepackage{comment}
\graphicspath{ {./figures/} }
\usepackage{varwidth}
\usepackage[ruled, linesnumbered]{algorithm2e}
\usepackage{enumitem}
\usepackage{footnote}
\usepackage{multirow}
\usepackage{tablefootnote}
\usepackage{xpatch}
\usepackage{algorithm2e}
\usepackage{balance}
\usepackage{makecell}
\usepackage{amsmath}
\usepackage{amsthm}
\theoremstyle{definition}  % This defines the style of the environment. 'definition' is one option.
\newtheorem{example}{Example}  % This defines the 'example' environment.
\newtheorem{definition}{Definition}
\usepackage[normalem]{ulem}
%\usepackage{setspace}
%\setstretch{0.5}

\usepackage{listings, xcolor}
\lstset{
tabsize = 4, %% set tab space width
language = Python,
showstringspaces = false, %% prevent space marking in strings, string is defined as the text that is generally printed directly to the console
commentstyle = \color{green}, %% set comment color
keywordstyle = \color{blue}, %% set keyword color
stringstyle = \color{red}, %% set string color
rulecolor = \color{black}, %% set frame color to avoid being affected by text color
basicstyle = \fontsize{8}{8}\selectfont\ttfamily , %% set listing font and size
breaklines = true, %% enable line breaking
numberstyle = \tiny,
}

\usepackage{hyperref}

% Attempt to make hyperref and algorithmic work together better:
\newcommand{\theHalgorithm}{\arabic{algorithm}}

%don't want date printed, part of usenix template
\date{}

\begin{document}
% \title{StreamServe: Low-Latency Decision Making over Distributed Data Streams}
\title{\Large \bf EdgeServe: A Streaming System for Decentralized Model Serving}
\newcommand{\sys}{EdgeServe\xspace}

%% ACM format
%% The "author" command and its associated commands are used to define the authors and their affiliations.
% \author{Ted Shaowang}
% \affiliation{%
%   \institution{The University of Chicago}
% }
% \email{swjz@uchicago.edu}

% \author{Sanjay Krishnan}
% \affiliation{%
%   \institution{The University of Chicago}
% }
% \email{skr@uchicago.edu}

% USENIX format
 \author{
 {\rm Ted Shaowang}\\
 The University of Chicago
 \and
 {\rm Sanjay Krishnan}\\
 The University of Chicago
 }
%\author{Anonymous}
\maketitle

%\setlength{\baselineskip}{11.5pt}

\begin{abstract}
The relevant features for a machine learning task may arrive as one or more continuous streams of data. Serving machine learning models over streams of data creates a number of interesting systems challenges in managing data routing, time-synchronization, and rate control. This paper presents \sys, a distributed streaming system that can serve predictions from machine learning models in real time.
% \sys relies on a low-latency message broker to route data through a network to nodes that can serve predictions.
% \sys relies on a series of novel optimizations that can trade-off computation, communication, and accuracy.
We evaluate \sys on three streaming prediction tasks: (1) human activity recognition, (2) autonomous driving, and (3) network intrusion detection.
\end{abstract}



\newcommand{\todo}[1]{\textcolor{red}{\bf [TODO: #1]}}

%%% do not modify the following VLDB block %%
%%% VLDB block start %%%
% \pagestyle{\vldbpagestyle}
% \begingroup\small\noindent\raggedright\textbf{PVLDB Reference Format:}\\
% \vldbauthors. \vldbtitle. PVLDB, \vldbvolume(\vldbissue): \vldbpages, \vldbyear.\\
% \href{https://doi.org/\vldbdoi}{doi:\vldbdoi}
% \endgroup
% \begingroup
% \renewcommand\thefootnote{}\footnote{\noindent
% This work is licensed under the Creative Commons BY-NC-ND 4.0 International License. Visit \url{https://creativecommons.org/licenses/by-nc-nd/4.0/} to view a copy of this license. For any use beyond those covered by this license, obtain permission by emailing \href{mailto:info@vldb.org}{info@vldb.org}. Copyright is held by the owner/author(s). Publication rights licensed to the VLDB Endowment. \\
% \raggedright Proceedings of the VLDB Endowment, Vol. \vldbvolume, No. \vldbissue\ %
% ISSN 2150-8097. \\
% \href{https://doi.org/\vldbdoi}{doi:\vldbdoi} \\
% }\addtocounter{footnote}{-1}\endgroup
%%% VLDB block end %%%

%%% do not modify the following VLDB block %%
%%% VLDB block start %%%
% \ifdefempty{\vldbavailabilityurl}{}{
% \vspace{.3cm}
% \begingroup\small\noindent\raggedright\textbf{PVLDB Artifact Availability:}\\
% The source code, data, and/or other artifacts have been made available at \url{\vldbavailabilityurl}.
% \endgroup
% }
%%% VLDB block end %%%

\section{Introduction}
\label{sec:introduction}
% \begin{itemize}
%     % Diffusion of FL
%     \item {\st{Diffusion of FL}}
%     % Security threats to FL
%     \item {\st{Security threats to FL with particular focus on model poisoning}}
%     % Limitations of existing countermeasures
%     \item {\st{Current countermeasures (e.g., KRUM) and their limitations}}
%     % Proposed method and its advantages
%     \item {\st{Intuitive description of the proposed method and its difference (i.e., advantages) w.r.t. state of the art}}
%     % Main contributions
%     \item {\st{Summary of the main contributions of this work}}
%     % Paper's structure and organization
%     \item {\st{Paper's structure and organization}}
% \end{itemize}

% Diffusion of FL
Recently, {\em federated learning} (FL) has emerged as the leading paradigm for training distributed, large-scale, and privacy-preserving machine learning (ML) systems~\cite{mcmahan2017googleai,mcmahan2017aistats}. 
The core idea of FL is to allow multiple edge clients to collaboratively train a shared, global model without disclosing their local private training data.
%Specifically, an FL system consists of a central server and many edge clients; 
A typical FL round involves the following steps: {\em(i)} the server randomly picks some clients and sends them the current, global model; {\em(ii)} each selected client locally trains its model with its own private data; then, it sends the resulting local model to the server;\footnote{Whenever we refer to global/local model, we mean global/local model {\em parameters}.} {\em(iii)} the server updates the global model by computing an \emph{aggregation function}, usually the average (FedAvg), on the local models received from clients.
% \begin{enumerate}
%     \item[{\em(i)}] the server sends the current, global model to the clients and appoints some of them for training;
%     \item[{\em(ii)}] each selected client locally trains its copy of the global model with its own private data; then, it sends the resulting local model back to the server;\footnote{Whenever we refer to global/local model, we mean global/local model {\em parameters}.}
%     \item[{\em(iii)}] the server updates the global model by computing an \emph{aggregation function} on the local models received from clients (by default, the average, also referred to as FedAvg~\cite{mcmahan2017aistats}).
% \end{enumerate}
This process goes on until the global model converges. %(e.g., after a certain number of rounds or other similar stopping criteria).
%\\
% The advantages of FL over the traditional, centralized learning paradigm are undoubtedly clear in terms of flexibility/scalability (clients can join/disconnect from the FL network dynamically), network communications (only model weights\footnote{We will use \textit{parameters} and \textit{weights} interchangeably.} are exchanged between clients and server), and privacy (each client's private training data is kept local at the client's end and not uploaded to the server).
\\
% Security threats to FL
%However, the growing adoption of FL also raises security concerns~\cite{costa2022covert}, particularly about its confidentiality, integrity, and availability.
Although its advantages over standard ML, FL also raises security concerns~\cite{costa2022covert}. %, particularly about its confidentiality, integrity, and availability~\cite{costa2022covert}.
% OLD, LONG VERSION
% Indeed, some work deals with privacy leakage that may expose the local data of some clients~\cite{melis2019sp}. 
% A large body of work, instead, investigates attacks that usually aim to detriment the predictive accuracy of the learned global model. For instance, \emph{data poisoning} attacks achieve this goal by letting an adversary pollute the training set of some corrupt FL clients with maliciously crafted examples~\cite{jagielski2018sp}.
% Similarly, in \emph{model poisoning} the attacker attempts to tweak the global model weights~\cite{bhagoji2019pmlr} by directly perturbing the local model's weights of some infected FL clients before these are sent to the central server for aggregation, usually via so-called Byzantine attacks. 
% It turns out that Byzantine model poisoning attacks severely impact standard FedAvg; therefore, more robust aggregation functions must be designed to make FL systems secure.
Here, we focus on \emph{untargeted model poisoning} attacks~\cite{bhagoji2019pmlr}, where an adversary attempts to tweak the global model weights %\footnote{We will use the terms \textit{parameters} and \textit{weights} interchangeably.} 
by directly perturbing the local model's parameters of some infected clients before these are sent to the central server for aggregation.
In doing so, the adversary aims to jeopardize the global model \textit{indiscriminately} at inference time.
Such model poisoning attacks severely impact standard FedAvg; therefore, more robust aggregation functions must be designed to secure FL systems.
\\
% In this paper, we focus on designing a novel robust aggregation scheme at the server's end to contrast the effect of Byzantine model poisoning attacks.
%
% Current countermeasures and their limitations
%Several countermeasures have been proposed in the literature to combat model poisoning attacks on FL systems.
% Some methods use simple statistics more robust than plain average to smooth the impact of malicious updates (e.g., Trimmed Mean and FedMedian~\cite{yin2018icml}). 
% Other defenses implement outlier detection techniques to discard malicious updates from the aggregation performed at the server's end. Those are either based on heuristics (e.g., Krum/Multi-Krum~\cite{blanchard2017nips} and Bulyan~\cite{mhamdi2018pmlr}) or data-driven approaches (e.g., K-means clustering~\cite{shen2016acm} or DnC via spectral analysis~\cite{shejwalkar2021ndss}). 
% Finally, some strategies rely on a centralized ``source of trust'' to spot potential malicious updates (e.g., FLTrust~\cite{cao2020fltrust}).
% Several countermeasures have been proposed in the literature to combat model poisoning attacks on FL systems, i.e., to discard possible malicious local updates from the aggregation performed at the server's end. 
% These techniques range from simple statistics more robust than plain average (e.g., Trimmed Mean and FedMedian~\cite{yin2018icml}) to outlier detection heuristics (e.g., Krum/Multi-Krum~\cite{blanchard2017nips} and Bulyan~\cite{mhamdi2018pmlr}) or data-driven approaches (e.g., spectral analysis via K-means clustering~\cite{shen2016acm} or spectral analysis), or methods based on ``source of trust'' (e.g., FLTrust~\cite{cao2020fltrust}).
% OLD, LONG VERSION
%Several countermeasures have been proposed in the literature to combat Byzantine model poisoning attacks on FL systems.
% Descriptive statistics
% For example, Trimmed Mean and FedMedian aggregate local model updates using more robust statistics than standard average~\cite{yin2018icml}.
%
% % Heuristics for outlier detection
% Many existing Byzantine-resilient strategies implement some outlier detection heuristics to discard the model updates sent by potentially malicious clients from the input of the aggregation function.
% One of the most popular heuristics is Krum~\cite{blanchard2017nips}.
% This strategy tries to mitigate the impact of Byzantine attacks by selecting as a global model the local model with the smallest sum of Euclidean distances to {\em all} the other local models.
% Although powerful, Krum requires the server to know (or, at least, estimate) the number of malicious FL clients upfront, which is generally impossible in a realistic attack scenario. %
% Moreover, Krum may become ineffective for complex, high-dimensional model parameter spaces due to the curse of dimensionality.
% Bulyan~\cite{mhamdi2018pmlr} tries to overcome this issue by combining Krum with a variant of Trimmed Mean.
% % Data-driven outlier detection
% Other strategies use data-driven outlier detection techniques -- e.g., via K-means clustering~\cite{shen2016acm} -- to spot potential malicious local model updates. 
% %For instance, Shen et al. propose to cluster local model updates with K-means and thus identify outliers.
%
% % Other techniques
% As far as the server is concerned, any local model received can be from a potential malicious client. 
% FLTrust~\cite{cao2020fltrust} assumes the server acts as a client, i.e., trains a local model on an additional {\em trustworthy} dataset at the server's end and compares it against all the local models from other clients. 
% This way, the server can rely on some ``source of trust'' when discarding potentially malicious clients.
%\\
% Limitations of existing Byzantine-resilient strategies
Unfortunately, existing defense mechanisms either rely on simple heuristics (e.g., Trimmed Mean and FedMedian by~\cite{yin2018icml}) or need strong and unrealistic assumptions to work effectively (e.g., foreknowledge or estimation of the number of malicious clients in the FL system, as for Krum/Multi-Krum~\cite{blanchard2017nips} and Bulyan~\cite{mhamdi2018pmlr}, which, however, cannot exceed a fixed threshold).
Furthermore, outlier detection methods using K-means clustering~\cite{shen2016acm} or spectral analysis like DnC~\cite{shejwalkar2021ndss} do not directly consider the temporal evolution of local model updates received.
Finally, strategies like FLTrust~\cite{cao2020fltrust} require the server to collect its own dataset and act as a proper client, thereby altering the standard FL protocol.
\\
% OLD, LONG VERSION
% Overall, existing Byzantine-resilient strategies are either simple heuristics (e.g., FedMedian) or, if they are more complex, they rely on strong and unrealistic assumptions to work effectively (e.g., knowing the number of malicious clients in the FL system in advance, as for Krum and alike).
% Furthermore, data-driven outlier detection methods do not consider the temporary evolution of local model updates received (e.g., K-means clustering). 
% Finally, strategies like FLTrust requires the server to collect its own dataset and act as a proper client, thereby altering the standard FL protocol.
%
% Description of the proposed method
This work introduces a novel pre-aggregation \textit{filter} robust to untargeted model poisoning attacks. Notably, this filter $(i)$ operates without requiring prior knowledge or constraints on the number of malicious clients and $(ii)$ inherently integrates temporal dependencies. 
The FL server can employ this filter as a preprocessing step before applying \textit{any} aggregation function, be it standard like FedAvg or robust like Krum or Bulyan.
Specifically, we formulate the problem of identifying corrupted updates as a multidimensional (i.e., matrix-valued) time series anomaly detection task. 
The key idea is that legitimate local updates, resulting from well-calibrated iterative procedures like stochastic gradient descent (SGD) with an appropriate learning rate, show \textit{higher predictability} compared to malicious updates. This hypothesis stems from the fact that the sequence of gradients (thus, model parameters) observed during legitimate training exhibit regular patterns, as validated in Section~\ref{subsec:intuition}. %until convergence. 
%This regularity may be more pronounced for smooth convex loss functions, but it can still be captured within an appropriate time window, even for more complex and convoluted loss surfaces. 
%We provide evidence of this claim in Appendix~B, where we show that the average mutual information (i.e., ``predictability''), calculated over pairs of legitimate model updates sent at different FL rounds, is significantly higher than the corresponding computation for a malicious client.
\\
Inspired by the matrix autoregressive (MAR) framework for multidimensional time series forecasting~\cite{chen2021je}, we propose the FLANDERS ({\em \textbf{F}ederated \textbf{L}earning meets \textbf{AN}omaly \textbf{DE}tection for a \textbf{R}obust and \textbf{S}ecure}) filter.
The main advantages of FLANDERS over existing strategies like FLDetector~\cite{zhao2020multivariate} are its resilience to large-scale attacks, where $50\%$ or more FL participants are hostile, and the capability of working under realistic non-iid scenarios.
We attribute such a capability to two key factors: $(i)$ FLANDERS works without knowing a priori the ratio of corrupted clients, and $(ii)$ it embodies temporal dependencies between intra- and inter-client updates, quickly recognizing local model drifts caused by evil players. Below, we summarize our main contributions:

\begin{itemize}
\item[{\em(i)}]
We provide empirical evidence that the sequence of models sent by legitimate clients is more predictable than those of malicious participants performing untargeted model poisoning attacks.
\\
\item[{\em(ii)}] 
We introduce FLANDERS, the first pre-aggregation filter for FL robust to untargeted model poisoning based on multidimensional time series anomaly detection.
\\
\item[{\em(iii)}] 
We integrate FLANDERS into Flower,\footnote{\scriptsize{\url{https://flower.dev/}}} a popular FL simulation framework for reproducibility.
\\
\item[{\em(iv)}] 
We show that FLANDERS improves the robustness of the existing aggregation methods under multiple settings: different datasets, client's data distribution (non-iid), models, and attack scenarios.
\\
\item[{\em(v)}] 
We publicly release all the implementation code of FLANDERS along with our experiments.\footnote{\scriptsize{\url{https://anonymous.4open.science/r/flanders_exp-7EEB}}}
\end{itemize}

% Paper's structure and organization
The remainder of the paper is structured as follows. %some related work and the current state-of-the-art solutions to security issues that FL entails. 
Section~\ref{sec:background} covers background and preliminaries. 
In Section~\ref{sec:related}, we discuss related work.
Section~\ref{sec:problem} and Section~\ref{sec:method} describe the problem formulation and the method proposed. % to tackle it. 
Section~\ref{sec:experiments} gathers experimental results. %, and Section~\ref{sec:limitations} discusses some limitations of this work.
Finally, we conclude in Section~\ref{sec:conclusion}.
 %discusses the limitations of this work and draws future research directions.
%reports conclusions and draws perspectives for future research directions.

%%%%%%% OLD %%%%%%%
%to overcome the resilience of Byzantine failures in distributed Stochastic Gradient Descent computations. 
% The strength of Krum is its time complexity, which is linear in the gradient dimension. 
% However, the robustness of the approach is guaranteed for gradient-based learning applications only when the majority of the clients are not compromised. 
% Besides, the aggregation mechanism of Krum, as well as that of similar methods, is robust from a coarse-grained perspective and does not provide solutions to errors and perturbations that may occur at inference time.
%A related approach to~\cite{blanchard2017nips} is the work of Su et al.~\cite{su2016dc}. Here, the authors propose an iterated approximate agreement to tackle a multi-layer scenario attacked by Byzantine agents. 
%However, the method works efficiently on the sole discrete context and it is inapplicable to continuous state environments.
%\gabri{Maybe, we should just talk about the main limitations of existing countermeasures without digging into their details (or, we can just mention Krum as this is the most popular one). I will move the description of all these methods to the Related Work section.}
\section{Background on Network Calculus}
\label{sec: background}


\begin{figure*}[tbh]
\centering
\begin{subfigure}[b]{0.3\textwidth}
    \centering
    \includegraphics[width=\linewidth]{images/in-out.png}
    \caption{Arrival and departure data and their relation with delay $d(t)$ and backlog $b(t)$. For a FIFO system, the delay is the horizontal distance between $R(t)$ and $R^*(t)$ but some other multiplexing techniques may shift the data to a later priority, causing a longer delay.}
    \label{fig: data in-out}
\end{subfigure}
\hfill
\begin{subfigure}[b]{0.35\textwidth}
    \centering
    \includegraphics[width=\linewidth]{images/arrival-service.png}
    \caption{Characteristics of an arrival curve and a service curve. From any point of observation, the arriving data never exceeds its arrival curve; the departure data is also never less than the service curve with respect to the data arrival.}
    \label{fig: arrival-service curves}
\end{subfigure}
\hfill
\begin{subfigure}[b]{0.33\textwidth}
    \centering
    \includegraphics[width=\linewidth]{images/bound.png}
    \caption{Delay and backlog bounds of a system. Backlog is the maximum vertical distance between $\alpha(t)$ and $\beta(t)$; FIFO delay is their maximum horizontal distance; but for arbitrary multiplexing, the delay guarantee is when the system clears its buffer, thus it's the intersection of $\alpha(t)$ and $\beta(t)$.}
    \label{fig: system bounds}
\end{subfigure}
\caption{Network calculus framework. We let $R(t)$ and $R^*(t)$ be the arrival and departure data flow of a system; $\alpha(t)$ be the piecewise linear concave arrival curve and $\beta(t)$ be the piecewise linear convex service curve of a system.}
% \hossein{Better to show piece-wise linear concave arrival curve and piece-wise linear convex service curve instead of token-bucket and rate-latency.}}
\end{figure*}

We recall some of the network calculus essentials for a better understanding of the framework used in Saihu. In the following context, we use the following notation: $\mbb{R}^+$ is the set of non-negative real numbers; $[x]_+$ denotes $\max(0, x)$

The data flow is by convention modeled as a left-continuous wide-sense increasing function $R(t): \mbb{R}^+ \mapsto \mbb{R}^+$ with respect to time $t$~\cite{ncbook2001leboudec}. 

A system $\mcal{S}$ receives arrival data described as a cumulative function $R(t)$ and delivers departure data as another cumulative function $R^*(t)$. Figure~\ref{fig: data in-out} illustrates such a system $\mcal{S}$. The benefit of representing a system like this is that we can observe system backlog and delay with such a model. 

\begin{definition}[Backlog and Delay~\cite{ncbook2001leboudec}]
    The backlog of a system at time~$t$ is
    \begin{equation}
        b(t) = R(t) - R^*(t)
    \end{equation}
    
    The virtual delay of a FIFO system at time $t$ is
    \begin{equation}
        d_{FIFO}(t) = \inf \lbp \tau \geq 0 : R(t) \leq R^*(t+\tau) \rbp
    \end{equation}
\end{definition}



The backlog of a system can be viewed as the vertical distance between $R$ and $R^*$. The FIFO (\textit{First-in First-out}) delay is the horizontal distance between $R$ and $R^*$. One may obtain other delay values if the multiplexing technique is not FIFO.

% \begin{figure}
%     \centering
%     \includegraphics[width=0.9\linewidth]{images/in-out.png}
%     \caption{In/out data flow; delay and backlog}
%     \label{fig: data in-out}
% \end{figure}

Since we are interested in the system guarantee instead of a single instance of data flow, we would like to have general bounds to the arrival and departure data flows. Therefore, we define \textit{arrival curve} and \textit{service curve} as the bounds of arrival and departure data flows.

\begin{definition}[Arrival Curve~\cite{ncbook2001leboudec}]
    Given a wide-sense increasing function $\alpha: \mbb{R}^+ \mapsto \mbb{R}^+$, we say that a flow $R(t)$ is $\alpha$-constrained if and only if for all $s \leq t$:
    \begin{equation}
        R(t) - R(s) \leq \alpha(t-s)
    \end{equation}
    We say $R(t)$ has $\alpha$ as an arrival curve.
\end{definition}

\begin{definition}[Service Curve~\cite{ncbook2001leboudec}]
    Given a wide-sense increasing function $\beta: \mbb{R}^+ \mapsto \mbb{R}^+$ and $\beta(0) = 0$. A system $\mcal{S}$ having $R(t)$ and $R^*(t)$ as its arrival and departure flows. We say $\mcal{S}$ offers a service curve $\beta$ if and only if
    \begin{equation}
        R^*(t) \geq (R \otimes \beta)(t) =: \inf_{s \leq t} \lbp R(s) + \beta(t-s) \rbp
    \end{equation}
    where $\otimes$ denotes the min-plus convolution
\end{definition}

Figure~\ref{fig: arrival-service curves} illustrates the arrival and service curves. Any segment of arrival flow $R(t)$ is constrained by arrival curve $\alpha$ and the output curve $R^*(t)$ is always no less than the curve $R\otimes\beta$. As a result, an arrival curve upper bounds the incoming traffic, and a service curve lower bounds the outgoing traffic.

% \begin{figure}
%     \centering
%     \includegraphics[width=\linewidth]{images/arrival-service.png}
%     \caption{Arrival/Service curve}
%     \label{fig: arrival-service curves}
% \end{figure}

We consider 2 special types of curves throughout this paper, \textit{token-bucket} (or sometimes called \textit{leaky-bucket}) curve and \textit{rate-Latency} curve.

\begin{definition}[Token-bucket and Rate-latency~\cite{ncbook2001leboudec}]
    A token-bucket curve $\gamma_{r,b}$ with arrival rate $r$ and burst $b$ is defined as
    \begin{equation}
        \gamma_{r,b}(t) = b + rt
    \end{equation}

    A rate-latency curve $\beta_{R,T}$ with service rate $R$ and latency $T$ is defined as
    \begin{equation}
        \beta_{R,T}(t) = R \lb t - T \rb_+
    \end{equation}
\end{definition}

A token-bucket curve is determined by a burst $b$ and an arrival rate~$r$. Burst represents the maximum possible data volume that can arrive simultaneously, and arrival rate represents the maximum long-term data rate~\cite{bouillard2022tradeoff}.
A rate-latency curve is determined by a latency~$T$ and a service rate~$R$. Latency represents the time a server needs before starting to process the incoming data, and service rate represents the minimum rate to process data after the initial latency.

With the help of arrival and service curves, we can derive delay and backlog bounds for a system $\mcal{S}$ illustrated in Figure~\ref{fig: system bounds}. Suppose a system $\mcal{S}$ has arrival curve $\alpha$ and service curve~$\beta$, its worst-case backlog $b^*$ is the maximum vertical distance between~$\alpha$ and~$\beta$. Similarly, depending on the multiplexing technique applied to the system, its worst-case delay bound $d^*$ is the maximum horizontal distance between $\alpha$ and $\beta$ if $\mcal{S}$ is a FIFO system. If we don't have any information about its multiplexing technique, referred to as arbitrary multiplexing, the best we can say is that when $\alpha$ and $\beta$ intersect each other, where all data has been delivered out of the system. Consequently, the worst-case delay bound for arbitrary multiplexing is the time required for $\mcal{S}$ to clear its buffer.

% \begin{figure}
%     \centering
%     \includegraphics[width=\linewidth]{images/bound.png}
%     \caption{System delay/backlog bounds}
%     \label{fig: system bounds}
% \end{figure}

While a service curve captures the slowest possible output speed of a system, a link's transmission capacity limits the speed as well. Hence, we model this phenomenon using a \textit{greedy shaper} with a sub-additive function $\sigma: \mbb{R}^+ \mapsto \mbb{R}^+$ concatenated with a server. We consider a concatenation as shown in Figure \ref{fig: system}. By convention we assume $\sigma(0) = 0$ and $\beta(t) \leq \sigma(t), \forall t \in \mbb{R}^+$, meaning that the buffer is cleared at the beginning and the service never exceed its physical limitation. With the above definition, such greedy shaper conserves the service provided by the system due to theorem \ref{thm: shaping}.

\begin{figure}[thb]
    \centering
    \includegraphics[width=0.7\linewidth]{images/system.png}
    \caption{Shaping of departure data. A flow that has an arrival curve $\alpha$ feeds into a server with an arrival data flow $R(t)$. The server having service curve $\beta$ takes $R(t)$ and gives a departure data flow $R^*(t)$ to a shaper with shaping function $\sigma$. The shaper takes $R^*(t)$ and shape the data flow as another departure $D(t)$.}
    \label{fig: system}
\end{figure}


\begin{theorem}[Shaping conserves service \cite{ncbook2001leboudec}]
\label{thm: shaping}
Following the system shown in Figure \ref{fig: system}, we have
\begin{equation}
     D = R^* \otimes \sigma \geq \lp R \otimes \beta \rp \otimes \sigma = R \otimes \lp \beta \otimes \sigma \rp = R \otimes \beta
\end{equation}
\end{theorem}

In the following context, we model the shaping function $\sigma$ as a token-bucket curve $\gamma_{C,L}$ with transmission capacity $C$ and the packet size $L$ to capture the link capacity and packetization~\cite{bouillard2022tradeoff}.

\section{\system Framework Design}
\label{sec:system}
We now explain how \system helps author widgets that support transparent, reusable, and customizable user actions. 

% \danc{here do we want to emphasize a) we revise conventional widget design to a statefule design or b) megneton can convert your existing widgets to improve it with our proposed characteristics? It sounds more like b) to me but not sure if that's the intention. } \saj{it's actually a}
%In this section, we discuss the key components that provide the foundation for \system widgets, which instruments the design goals outlined in Section~\ref{sec:design_goal}. 
%We then explain the system architecture of \system.

\subsection{Widget Frameworks: Design and Limitations}
Widgets are interactive elements, \eg sliders, text boxes, buttons, that have representations both in the kernel, \ie where code is executed, and the front-end, \ie the notebook web interface. However, recent frameworks for authoring widgets~\cite{idomjp} also enable integration of interactive dashboards in the front-end~\cite{wu2020b2,bauerle2022symphony, zhang2023meganno}.  

 \begin{figure}[!htb] 
 \centering
  \includegraphics[width=0.8\linewidth]{figures/stateful-widget-redesign-basic.png}
  \caption{Design of basic, \ie traditional widgets.}
  \label{fig:base_widget} 
  \Description{The basic widget design.}
\end{figure}

 
 As shown in Figure~\ref{fig:base_widget}, Widgets (\eg \emph{ipywidgets}~\cite{IPyWidgets}) maintain their state both at the back-end kernel (called \emph{Widget Base}) and the front-end (called \emph{Widget Model}.) The Widget Base and Widget Model remain in-sync via the communication API called \emph{Comm}. 
 %\danc{The previous statement is bit hard to follow. Break down into defining what are widget base and model, and then explain how they work together?} 
However, only the most recent state is maintained, making the widgets essentially \emph{memoryless}. The \emph{Widget Manager} coordinates the display of the widget in the front-end \emph{Widget View}. The Widget View is a container for rendering interactive components using front-end libraries and web frameworks. The Widget View only registers low-level event listeners corresponding to user interactions on the components. %For example, a \emph{drag} interaction that updates position of slider is registered as an \emph{onChange} listener. 
 For example, a \emph{selection} interaction on the graph node in Figure~\ref{fig:teaser}B that updates the bar charts is registered as an \emph{onClick} listener.
 Therefore, these widgets are \emph{agnostic} of the user's high-level interaction types and additional context, such as where the interaction happened and which components were updated. The \emph{memoryless} and \emph{interaction agnostic} nature of widgets prevent tracking of the user's interaction history and the corresponding widget states.
Moreover, such a design primarily serves to parameterize data operations in the kernel using front-end events --- a widget state variable (\eg current node identifier) impacted by a low-level event (\eg \emph{onClick}) serves as an input parameter to a data operation (\eg distribution computation). Any change in the widget variable triggers a recomputation of the data operation. In the notebook, the users can programmatically access and update the parameters of the data operations. However, the data operations in the kernel, designed by widget developers, are neither accessible nor customizable from the front-end. The lack of affordances to override data operations limit 
the end-user's capability to customize the widgets designed by the developers. We describe enhancement of existing widgets with such features next.

\subsection{Towards Persistent, Interaction-Aware, and Customizable Widgets}

%We augment existing widgets to introduce new features such as interaction history, reusable sates, and on-demand customization of data operations. 
We create a persistent and interaction-aware widget called \emph{stateful widget} by extending the Widget Base with state and interaction history management capabilities (see Figure~\ref{fig:stateful_widget}.) Within a stateful widget, the state manager maintains each state updates corresponding to user interactions within a list called \emph{Data States}. The state manager registers the following in the \emph{action history}: (a) context of each event (\eg the front-end interaction type and the component and element where interaction occurred) and (b) the corresponding state identifier in Data States. Since the default Widget View only registers low-level events, we create a Widget View Wrapper that records each event's context as an action via an Action Wrapper. The action wrapper dispatches an action consisting of the event context mentioned earlier via the Comm API. Users can view the interaction history in a separate notebook cell which shows the details of an interaction and the corresponding data state as shown in Figure~\ref{fig:history}, thereby ensuring transparency. The history view is synchronized with the corresponding widget. Therefore, users can leverage the history to load previous states in the Widget View using the \emph{Restore} button. Moreover, users can also access the widget state as a \code{JSON} object using a declarative command as shown in Figure~\ref{fig:teaser}E, thereby ensuring reusability. Such a design also enables users to employ visualizations as a medium for capturing 
and exporting ``actions interactively
performed in the component''~\cite{batch2017interactive} --- 
the outcomes of these interactions are often utilized in 
subsequent steps of a data science workflow~\cite{rahman2022ie}.

 \begin{figure}[!htb] 
  \centering
  \includegraphics[width=\linewidth]{figures/stateful-widget-redesign-mag.png}
  \caption{Design of \system widgets. The dashed (``- -'') elements, \ie the stateful widget and widget view wrappers, are introduced by \system.}
  \label{fig:stateful_widget} 
  \Description{The stateful widget design.}
\end{figure}

% Therefore, interaction-aware state management \todo{via stateful widget} ensures transparency and reusability of user actions. \todo{elaborate}

\begin{figure*}[!htb] 
  \centering
  \includegraphics[width=\linewidth]{figures/history.png}
  \caption{The history view of a widget (\code{widget.history.show()}). Clicking the \emph{Restore} button loads previous state visualizations. }
  \label{fig:history} 
  \Description{The history view of a widget. Clicking the Restore button loads the previous states and their visualizations.}
\end{figure*}

%\hkc{why is it called 'shared'? also which opreations are customizable and which are not?} all data operations are customizable if they are defined as shared
Since data operations in the kernel correspond to user interactions in the front-end component, we introduce the concept of \emph{shared actions}.
Shared actions are data operations that end-users can override from the notebook. The operation definitions are essentially shared between the kernel and front-end. In the \system framework, developers can define a data operation to be shared. 
 For example, a shared data operation may return a distribution sorted by descending order of frequency. However, the user may prefer viewing the distribution in the alphabetic order of labels. As shown in Figure~\ref{fig:teaser}C and~\ref{fig:teaser}D, a user-defined function (UDF) written in the notebook --- which reflects the updated sort order --- is mapped to these the actions during widget instantiation time. 
In the kernel, the state manager parses the UDFs using custom serializers and overrides the data operation corresponding to the shared action. 
Such a design expands the ``events parameterizing code'' paradigm of widgets to ``operations parameterizing code'' and offers more flexible customization capabilities --- users can keep updating the shared actions to explore different objectives by modifying the function defined in the notebook.
Note that developers may implement data operations such as schema generation and distributions computation using standalone libraries or from scratch. In the case studies described in Section~\ref{sec:study}, we used an in-house graph query library, which was published as a Python package. 
% , among others. The operations are part of an in-house graph query library, which is published as a Python package for internal usage.
%We provide examples of these features in the supplementary material. 


%\todo{ADD CODE BLOCK}
\stitle{Components in \system Widget View.} 
%\danc{this paragraph renamed to technical/implementation details? Or is component a special term in megneton? } 
We use the React web framework~\cite{react} to develop the front-end components and the IDOM-Jupyter package~\cite{idomjp} for component rendering in the Widget View. 
The components are TypeScript~\cite{typescript} modules that enable the rendering of a wide range web-based visualization libraries. For example, we used a custom graph visualization library to render the schema graph~\cite{franz2016cytoscape}, Vega-lite~\cite{satyanarayan2016vega} to render the bar charts, and a JavaScript library to render tables. 
%We provide a detailed list of all the components in the supplementary material.
As TypeScript supports static typing, developers can define application-specific data types and use those across the modules. 
Therefore, using TypeScript ensures a tighter integration between the Widget Base in the kernel and the Widget Model in the front-end. 
Moreover, when customizing data operations defined as shared actions, the pre-defined types provide hints to the user about the expected return type of the customized function. Each of the components rendered in the Widget View is fully interactive. These interactions, derived from existing visualization research~\cite{yi2007toward, amar2005low} are reactively synchronized across \system components, enabling multiple-coordinated visualizations (\eg Figure~\ref{fig:teaser}B.) 
\subsection{Model Decomposition}\label{sec:model-decomposition}
While this paper focuses on system optimizations, it is worth devoting some time to understanding what types of machine learning models can effectively use a decentralized serving system. We will show it is relatively straightforward to do these decompositions in a generic way, and thus, is not a primary contribution of this paper.

Since models are the unit of placement and computation in \sys, the goal of model decomposition is to increase opportunities for optimizing placement. The idea is to approximate a single model with an ensemble or mixture of smaller local models.

\subsubsection{Strategy 1. Ensemble Models}
Ensemble machine learning models are techniques that combine multiple models to improve the accuracy and robustness of predictions. The idea behind ensemble modeling is that by combining the predictions of multiple models, we can reduce the risk of individual models making incorrect predictions and improve the overall performance of the model.

\emph{We can use an ensemble of models over different feature partitions to create decomposed models that can effectively take advantage of \sys.} Let's imagine that we have $p$ features and $n$ examples with an example matrix $X$ and a label vector $Y$.
Different subsets of these features are constructed on $m$ data sources on the network. Each source generates a partition of features $f_i$, i.e., $X[:, f_i]$ is the source-specific projection of training data. 

Stacking is an ensembling technique where multiple models are trained, and their predictions are used as inputs to a final model. The final model learns to weigh the predictions of each model and make a final prediction based on the weighted inputs. This helps capture the strengths of each individual model and produce a more accurate prediction.

We can train the following models. For each feature subset $f_i$, we train a model (from any model family) that uses only the subset of features to predict the label.
\[
g_i \leftarrow \textsf{train}(X[:, f_i], Y)
\]
After training each of these models over the $m$ subset, we train a stacking model that combines the prediction. This is a learned function of the outputs of each $g_i$ that predicts a single final label:
\[
h \leftarrow \textsf{train}([g_1,...,g_m], Y)
\]

Stacking models are well-studied in literature and are not new~\cite{sagi2018ensemble}. For multi-modal prediction tasks, our prior work has found that such models do not sacrifice accuracy and sometimes actually improve accuracy~\cite{shaowang2023amir}. A stacked model is a form of regularization that does not learn complex interactions across sources. In cases where some individual sources are unreliable or noisy, a stacked model is often more robust than a monolithic one. Regardless, stacking gives a generic training tool to decompose a single model into smaller source-specific models.  


\subsection{Strategy 2. Mixture of Experts Models}
Similarly, there are neural network architectures that can be trained end-to-end to take advantage of \sys. Mixture of Experts (MoE) is a deep learning architecture that combines multiple models or ``experts' to make predictions on a given task. Each expert is a specialized neural network that excels at solving a particular aspect of the problem. The MoE architecture uses a gating network to determine which expert should be used for a particular input.

The basic idea of the MoE architecture is to divide the input space into regions and assign an expert to each region. The gating network takes the input, decides which region it belongs to, and then selects the corresponding expert to make the prediction. The gating network then weights the output of each expert, and the final prediction is the weighted sum of the expert predictions. MoE architectures have been applied to a wide range of tasks, including language modeling, image classification, and speech recognition~\cite{eigen2013learning}.

In the context of \sys, each expert can be placed independently once trained. Like the stacked model above, we can align the experts to feature subsets generated on different source nodes. However, unlike the stacked model, there is no restriction that the local models output interpretable predictions. These local models might simply output a vector to be combined on a different node.





\section{Highly-Responsive Data Stream Joins}\label{sec:join}
In typical time-series databases, band-joins are used to integrate such series~\cite{dewitt1991evaluation, khayyat2015lightning}, where all items within a certain time-delta are grouped together.
True band-joins are challenging in streaming systems where data may arrive out-of-order or in a bursty way leading to potentially unbounded buffering, so existing streaming systems offer an approximation using tumbling time windows~\cite{flink, apachestorm, apache-samza, spark-structured-streaming,photon,facebook-streaming-join}.
All items that fall within the same tumbling time window are grouped together.
We refer to this type of streaming join as a \textit{time-triggered join}, i.e, the join condition is triggered by a clock tick.

This type of join can cause delays that affect the timeliness of predictions.
A multimodal model's response time to new data is limited by the width of the tumbling window.
An alternative to time-triggered joins is to join eagerly as soon as a new item is published to the stream, which we call a \textit{data-triggered join}.
The novelty of this approach is that it is responsive to new data, while having a bounded buffer size.
We will discuss both join methods in more detail below.

\subsection{Time-triggered Join}\label{sec:time-join}
As the name suggests, a time-triggered join buffers incoming messages from all streams over a time window, and triggers a join result at the end of the time window.
Within each time window, only the latest message from each stream is kept as that stream's input. The definition of `latest' here can be either event time or processing time. These latest messages are combined as a tuple before sending downstream.

\begin{figure}[t]
    \centering
    \begin{minipage}{.5\columnwidth}
      \centering
      \includegraphics[width=\linewidth]{figures/time-triggered-join.pdf}
      \caption{Time-triggered join.}
      \label{fig:time-triggered-join}
    \end{minipage}%
    \begin{minipage}{.5\columnwidth}
      \centering
      \includegraphics[width=\linewidth]{figures/data-triggered-join.pdf}
      \caption{Data-triggered join.}
      \label{fig:data-triggered-join}
    \end{minipage}
\end{figure}
Figure~\ref{fig:time-triggered-join} is an example of a time-triggered join.
% A join is issued whenever a fixed time window is passed.
In this example, we assume event time and processing time are the same for simplicity. The join results are as follows:
join 1 (A1, B1, C1, D1);
join 2 (A1, B2, C1, D2);
join 3 (A1, B3, C2, D2);
join 4 (A2, B3, C2, D2).
% They can be performed based on either event-time or processing-time.
Intuitively, waiting for a time-triggered join resembles waiting for a bus.
Since B2 arrives immediately after the join at $t=T$ was issued, it will have to wait until $t=2T$ to get processed, resulting in a longer waiting time.
On the other hand, time-triggered joins are beneficial when joins are desired at fixed frequencies, as they smooth out the burstiness of incoming data.

The state management for a time-triggered join is rather straightforward.
If the join is based on processing time, only the latest messages from each stream need to be buffered.
If the join is based on event time, all messages within a fixed time window need to be additionally buffered, in case they arrive out of order in terms of event time.

\subsection{Data-triggered Join}\label{sec:data-join}
An alternative to a time-triggered join is to perform a join whenever a new piece of data from any stream arrives.
Intuitively, the latest known data from all streams are buffered in order to join with the \underline{new data item} (underlined in the following example). We will show how this works precisely in the two-way case, and it should be clear how to extend this to a multi-way join.

Given two streams StreamA and StreamB, the algorithm tracks the latest known item from each stream and its timestamp.
Each time the stream publishes a new data item, it is joined with the latest known item from the other stream.
As before, the timestamp can refer to either event time or processing time.
However, the order of joined tuples is not guaranteed in terms of event time, since the joining process depends on when the data is actually received by the joiner.

\vspace{0.25em}
\noindent\fbox{%
\footnotesize
    \parbox{0.8\columnwidth}{%
        \noindent \textbf{Data-Triggered Join Algorithm} \\

\textbf{Given: } StreamA, StreamB 

\textbf{Set: $(a, a_t) \leftarrow (\emptyset, -\infty)$, $(b, b_t) \leftarrow (\emptyset, -\infty)$} \\

\textsf{onStreamA(x: data, t: timestamp): }
\begin{enumerate}
    \item If b is not $\emptyset$, yield $(x, b)$
    \item If $t > a_t$, $(a, a_t) \leftarrow (x, t)$
\end{enumerate} 

\textsf{onStreamB(x: data, t: timestamp): }
\begin{enumerate}
    \item If a is not $\emptyset$, yield $(a, x)$
    \item If $t > b_t$, $(b, b_t) \leftarrow (x, t)$
\end{enumerate}
    }
    %
}

Figure~\ref{fig:data-triggered-join} is an example of a data-triggered join.
Again for simplicity, we assume event time and processing time are the same. The join results are as follows:
join 1 (A1, B1, C1, \underline{D1});
join 2 (A1, \underline{B2}, C1, D1);
join 3 (A1, B2, C1, \underline{D2});
join 4 (A1, \underline{B3}, C1, D2);
join 5 (A1, B3, \underline{C2}, D2);
join 6 (\underline{A2}, B3, C2, D2).
In this way, we ensure that the system reacts immediately to new data at the expense of more frequent joins.
Data-triggered join is preferred when at least one data stream is bursty, as it is difficult to set a good time window with bursty data involved.

\subsection{When Is Data-triggered Join Better}
Data-triggered joins are more suitable for event-based streams whereas time windows are good aggregators for continuous data streams (e.g. sensor data).
In certain cases such as activity recognition, there is no activity of interest for most of the time.
For example, a Nest Cam only emits data when it finds people, vehicles, or animals in sight.
Data-triggered joins can capture these events as soon as they happen.
It is possible to combine time-triggered join and data-triggered join to get the best of two worlds.
\S\ref{sec:target-pred-freq} describes a hybrid method that primarily operates on a data-triggered basis, while strategically integrating time-triggered elements as a rate limiter.
Data-triggered joins further provide a completeness guarantee to the downstream data consumer. 
\emph{Every message is guaranteed to be present in at least one join tuple.}

% Relevant video clips can be combined with network traffic data collected from IoT devices to infer human activity~\cite{shaowang2023amir}.
% We can treat the event streams (e.g. occasional Nest Cam clips) as ``primary keys'' that trigger data-triggered joins, and set a fixed time window for continuous data streams (e.g. network traffic data).

In both time-triggered and data-triggered joins, we see repeated data appearing in multiple join results due to the lower frequency of some streams. Ideally, we want to send at most one copy of the same data over the network to avoid unnecessary bandwidth usage, especially if the data payload is large. Section~\ref{sec:lazy} proposes a novel technique called lazy data routing to address this problem.

% This would result in repeated occurrences of old data from other streams, but it ensures the timeliness of joins.
% This ensures the timeliness of data for latency-sensitive applications.

\section{Communication Primitives}
\label{sec:lazy}
Next, we describe one of the core optimizations in \sys that allow for efficient and low-latency data movement. 

\subsection{Overview}
The machine learning setting has a number of key traits that differ from typical stream processing deployments:
\begin{itemize}
    \item \textbf{Large Message Payloads.} In a number of sensing and imaging applications, each message processed by the message broker can be quite large, e.g., a high-resolution image. 
    
    \item \textbf{Operator Revision.} As models are redesigned and retrained, the core operators in the streaming pipeline change.   When the computational characteristics of the model change, the entire pipeline might have to change. For example, if a smaller model is replaced by a larger model, the stream might have to be down-sampled to avoid backlog.

\end{itemize}

\subsection{Distributed Task Model}
\begin{figure}[t]
    \centering
    \includegraphics[width=0.6\columnwidth]{figures/aggregate-distribute.pdf}
    \caption{Aggregate and Shared Queue Operators}
    \label{fig:aggregate-distribute}
\end{figure}
Before describing how data are communicated, it is worth clarifying the semantics of the task model in \sys
\sys has two key distributed operators illustrated in Figure~\ref{fig:aggregate-distribute}:
\texttt{aggregate(delay)} and \texttt{shared\_queue()}. These are analogous to their non-streaming counterparts (e.g., reduce). These two operators sit between data consumers and producers to ensure that computation is appropriately placed in the network.
They can actually be thought of as special ``models'' in our system whose sole purpose is multiplexing and demultiplexing data streams.

\noindent \texttt{aggregate(delay).} Whenever streaming data from multiple nodes need to be combined in order to make a prediction, the \textit{aggregate} operator is applied. An aggregate operator consumes data from multiple streams and produces a single iterator interface for a data consumer.
The operator takes a user-specified delay as a parameter --- the longest tolerable time skew between data sources.
The operator waits until the skew timeout is met for data from all senders to arrive and yields a combined tuple.
This tuple might contain missing values for streams that did not produce data within the time window.

\noindent \texttt{shared\_queue().}
The \texttt{shared\_queue()} operator multiplexes a stream into multiple streams of data, or demultiplexes multiple streams into one stream of data.
Nodes can consume data from each individual stream and perform model inferences.
After inference, we can also use an \texttt{aggregate} operator to merge the predictions back into a single time-synchronized stream.

\subsection{Lazy Data Routing}
A message broker system consists of a leader that orchestrates the entire message flow and multiple producers/consumers as message endpoints.
Data streams as producers publish data to the leader, and models as consumers consume data from the leader.
With this architecture, the leader can quickly become a point of contention since it has to process all the messages from/to all the different nodes.
Furthermore, large message payloads (e.g., images) can lead to a crucial networking bottleneck at the leader, as message broker systems are not designed to handle large messages.

\sys uses a novel messaging protocol to efficiently transfer data between nodes without placing an undue burden on the leader.
A message sent to the leader only contains message headers: a timestamp and a global source path.
The actual message payloads are not transferred; instead, they are kept and indexed on the node that collected the data.
A model subscribes to the topic and reads the headers as they come in.
If it wants a particular data payload, it retrieves that data lazily from the source node.

Figure~\ref{fig:lazy} illustrates this protocol. When collecting data, every data item added to a \texttt{DataStream} is annotated with a header (Figure \ref{fig:lazy}-1). We can think of this as a stream of $(h,d)$ tuples (header and data, respectively). After the tuple is created, the node locally writes the data to a time-indexed log (Figure \ref{fig:lazy}-2).
After this data is durably written, the header is published to the message broker on the leader (Figure \ref{fig:lazy}-3).
Nodes on the network can subscribe to streams of these headers.
Model inference requires the data payload, and that can be requested from the headers (Figure \ref{fig:lazy}-4). This data is transferred in a peer-to-peer fashion, and inferences happen over these streams (Figure \ref{fig:lazy}-5).

\begin{figure}[t]
    \centering
    \includegraphics[width=0.8\columnwidth]{figures/lazy.png}
    \caption{A figure illustrating the order of operations in the lazy data routing system used by \sys.}
    \label{fig:lazy}
\end{figure}

Lazy retrieval has a number of essential benefits for typical model-serving tasks. In general, these benefits are analogous to that of lazy computation. First, many models predict at rates much slower than the rate of data collection. For example, a model that takes 30ms to evaluate can only process one example every 30ms. If the data collection rate is significantly faster than that, the model often has to downsample the input data. Lazy data routing allows us to avoid transferring the data payload to the leader in these cases. 
Next, this strategy reduces the size of the messages processed by the message broker reducing overheads in checkpointing and serialization/de-serialization.
As a result, we also find that it can enable increased parallelism as well.
Both of these benefits can be tied back to the traits of the machine learning setting mentioned above.

\subsubsection{Other Considerations}
Practically, every edge node has a limited amount of local storage. \sys handles this by having a timeout for the data payloads, where a node on the network can only send a retrieval request within that timeout.
This allows the edge node to overwrite/free up that space periodically.

In certain cases, we allow users to force \sys to have eager message passing. Small messages, such as 1D arrays, can be transferred from data source nodes to worker nodes via the leader node. Essentially, this embeds the payload in the message headers.
In some networks, peer-to-peer communication is not available or not efficient.
We can default to eager message passing when needed to support these cases.
\section{Modeling of Tumor Growth}\label{Sec:Modeling}



\noindent We propose mathematical oncology models that abstract a number of the known significant mechanisms involved in tumor growth, decline, and therapeutic therapy in real tissue.
The systems are designed to reflect mesoscale and macroscale dynamics, with fields representing volume fractions of mass concentrations of diverse species that determine tumor composition.
Several authors, including \cite{araujo2004history},  \cite{fritz2022well}, \cite{garcke2016cahn,garcke2018multiphase}, \cite{lima2014hybrid} and \cite{wise2008three}, have produced localized versions of multiphase models over the past decade. Balance laws of continuum mixture theory are used to derive the model equation, see also \cite{byrne2003modelling}, \cite{cristini2009nonlinear}, and \cite{oden2016toward, oden2010general}.

In \cref{Sec:Math:Proto}, we present the prototype system for modeling tumor growth -- the Cahn--Hilliard equation with concentration-dependent mobility. In a generic framework, we provide in \cref{Sec:Mod:Mult} a multiple constituent model derived from the mass balance law and a Ginzburg--Landau type energy.
As an illustration, we provide the four-species model developed by \cite{hawkins2012numerical}.
In \cref{Sec:Mod:ECM}, we incorporate stratification and invasion due to ECM deterioration into the model. In the following subsections, additional biological phenomena will be added to the stratified tumor model.  We incorporate spatial and temporal nonlocalities in \cref{Sec:Mod:Nonlocal}, stochasticity by a cylindrical Wiener process in \cref{Sec:Mod:Stochastic}, mechanical deformation in \cref{Sec:Mod:Mechanical}, chemotherapeutic influence in \cref{Sec:Mod:Chemo}, and lastly, angiogenesis in mixed-dimensional couplings in \cref{Mod:Mixed}.


\subsection{Prototype model: The Cahn--Hilliard equation} \label{Sec:Math:Proto}
The Cahn--Hilliard equation is the prototypical model for tumor growth.
It is a phase-field equation of the diffuse-interface type, and it possesses the essential attribute of having a solution that is either 0 or 1, or a smooth transition phase in between.
We define the 1-phase as the manifestation of tumor cells, whereas the 0-phase represents the absence of malignant cells. 

Let $\phi_1$ and $\phi_2$ represent the concentrations of two components, and it holds $\phi_1+\phi_2=1$. This indicates that the concentrations describe local portions, such as those found in binary alloys.
They comply with the mass conservation law 
	\begin{equation*}
	\p_t \phi_i = - \div J_i, \quad i \in \{1,2\},
	\end{equation*}
	where the mass flux of the $i$-th component is denotes by $J_i$. We assume that the fluxes fulfill the condition $J_1+J_2=0$ and we reduce the equations by defining the quantities $\phi=\phi_1-\phi_2$ and $J=J_1-J_2$, which yields
	$$  \p_t \phi = - \div J.$$
	Here, the flux $J$ is given by the negative of the gradient of the chemical potential $\mu$, i.e., $J=-\nabla \mu$. In \cite{gurtin1996generalized}, a mechanical version of the second law of thermodynamics was introduced
by providing an augmented mass flux $J=-m(\phi)\nabla\mu$ with some mobility function $m$ for describing microscopic interactions.
	Following \cite{cahn1958free}, the chemical potential $\mu$ is given by the G\^ateaux derivative of the Ginzburg--Landau energy functional
	\begin{equation} \label{Eq:Ginzburg}
	\calE(\phi) = \int_\Omega \Big\{ \Psi(\phi) + \frac{\eps^2}{2} \vert \nabla \phi\vert ^2 \Big\} \, \dd x.\end{equation}
	Here, the parameter $\eps$ expresses the interfacial width and $\Psi$ describes a double-well potential with zeros at $0$ and $1$, e.g., the Landau potential
	$
	\Psi(\phi)=\frac14 \phi^2 (1-\phi)^2. 
	$
 Hence, the Cahn--Hilliard equation with concentration-dependent mobility reads
\begin{equation} \label{Eq:Cahn} \boxed{
		\begin{gathered}
		\textbf{Cahn--Hilliard equation}  \\
		\begin{aligned}
	\p_t \phi ={}& \div(m(\phi) \nabla \mu) \\  \mu ={}& \Psi'(\phi) - \eps^2 \Delta \phi \end{aligned}\end{gathered}}
\end{equation}
	  Usually, the mobility function takes the form $m(\phi)=M\phi^2(1-\phi)^2$ for some $M>0$. The scenario of constant mobility has been exhaustively examined, and well-posedness can be demonstrated through the use of sufficient assumptions, as done in \cite{miranville2019cahn}. A proof or counterexample of uniqueness in the case of degenerate mobility remains unsolved for the class of degenerate fourth-order parabolic equations. 



\subsection{Base system: Multiple constituent model} \label{Sec:Mod:Mult}
Multiple mechanical and chemical species can coexist at a given place $x$ in a given domain $\Omega \subset \R^d$, $d \in \N$, within the continuum mixture theory paradigm.
For a medium with $N$ interacting constituents, the volume fraction of each species is therefore represented by a field $\phi_\alpha$, $1\leq \alpha \leq N$, with value $\phi_\alpha(t,x)$ at $x\in \Omega$, and time $t\geq 0$.
For convenience, we compile the model's components in the following $N$-tuple
$$
 \phi_\A =  (\phi_\alpha)_{\alpha \in \A},
$$
where $\A$ is an index set that is further disjointly separated between the phase-field index set $\CH$, the reaction-diffusion indices $\RD$, and the evolution indices $\OD$ that correspond to abstract ordinary differential equations (ODEs).

Following \cite{lima2014hybrid,lima2015analysis}, the constituents $\phi_\alpha$, $\alpha \in \A$, are governed by the  extended mass balance law
\begin{equation} \label{Eq:MassBalance}
\p_t \phi_\alpha+\text{div}(\phi_\alpha v_\alpha)=- \text{div} J_\alpha(\phi_\A) +S_\alpha(\phi_\A).
\end{equation}
Here, $v_\alpha$ is the cell velocity of the $\alpha$-th  constituent, and $S_\alpha$ is a species-dependent mass source term.
We refer to the system as closed if it holds $\sum_{\alpha \in \A} S_\alpha(\phi_\A)=0$.
In addition, $J_\alpha$ represents the flux of the $\alpha$-th constituent, which is proportional to the negative gradient of the chemical potential multiplied by a mobility function
\begin{equation} \label{Eq:Flux}
J_\alpha(\phi_\A) = - m_\alpha(\phi_\A) \nabla \mu_\alpha.
\end{equation}
Here, $\mu_\alpha$ represents the chemical potential of the $\alpha$-th species, and $m_\alpha$ represents the mobility function, which may depend on all constituents.
In our applications, we typically take the mobilities
\begin{equation} \label{Eq:Mobility} \begin{aligned} m_\alpha(\phi_\A) ={}&M_\alpha \phi_\alpha^2 (1-\phi_\alpha)^2, && \alpha \in \CH, \\ m_\beta(\phi_\A) ={}& M_\beta, && \beta \in \RD, \\
m_{\gamma}(\phi_\A) ={}&0, && \gamma \in \OD,&&
\end{aligned}\end{equation} where $M_\alpha>0$ are constants.  Similarly to the prototype model, see \cref{Sec:Math:Proto}, we define the chemical potential $\mu_\alpha$ as the G\^ateaux derivative of the Ginzburg--Landau energy with respect to $\phi_\alpha$. We propose the system's energy
\begin{equation}\mathcal{E}(\phi_\A)= \int_\Omega \Big\{ \Psi(\phi_{\CH}) + \Phi(\phi_\A)+ \sum_{\alpha \in \CH}  \frac{\eps_\alpha^2}{2} \lvert\nabla \phi_\alpha\rvert^2  + \sum_{\beta \in \RD} \frac{D_\beta}{2} \phi_\beta^2 \Big\}  \text{ d}x,
\label{Eq:GinzburgLandau}
\end{equation}
where $\varepsilon_\alpha$, $\alpha \in \CH$, is a parameter related to the thickness of the contact separating the various cell kinds. As we will see later, the function $\Phi$ explains adhesion mechanisms such as chemotaxis and haptotaxis.
Finally, $\Psi$ represents a double-well potential as in the generic Cahn--Hilliard equation \cref{Eq:Cahn}, e.g., it may be of Landau type, for which we list two alternatives
$$\begin{aligned} \Psi(\phi_\CH)&={}C_{\Psi} \bigg(\sum_{\alpha \in \CH} \phi_\alpha\bigg)^2 \bigg(1-\sum_{\alpha \in \CH} \phi_\alpha \bigg)^2, \\
\Psi(\phi_\CH) &={} \sum_{\alpha \in \CH} C_{\Psi_\alpha} \phi_\alpha^2(1-\phi_\alpha)^2, \end{aligned}
$$
where $C_\Psi$ and  $C_{\Psi_\alpha}$ are given prefactors. As another possibility, we could select a logarithmic potential of Flory--Huggins type, see \cite{cherfils2011cahn} and \cite{frigeri2018on}.

We calculate the G\^{a}teaux derivatives of the Ginzburg--Landau energy \cref{Eq:GinzburgLandau} with respect to the stated constituents and therefore, the corresponding chemical potentials read
\begin{equation*} %
\begin{aligned}
\mu_\alpha ={}& \p_{\phi_\alpha} \Psi(\phi_\CH) + \p_{\phi_\alpha} \Phi(\phi_\A) - \eps^2_\alpha \Delta \phi_\alpha, && \alpha \in \CH , \\
\mu_\beta ={}& D_\beta \phi_\beta  + \p_{\phi_\beta} \Phi(\phi_\A), && \beta \in \RD , \\
\mu_\gamma ={}& \p_{\phi_\gamma} \Phi(\phi_\A), &&\gamma \in \OD,
\end{aligned}
\end{equation*}
and combining the chemical potentials with the mass balance laws \cref{Eq:MassBalance}--\cref{Eq:Mobility}, it yields the multispecies model:
\begin{equation} \label{Eq:MultipleGeneral} \boxed{ 
	\begin{gathered} 
	\textbf{Multiple constituent model} \\
	\begin{aligned}
\pt \phi_\alpha\!+\! \div(\phi_\alpha v_\alpha) ={}& \div \big(M_\alpha\phi_\alpha^2(1-\phi_\alpha)^2 \nabla \mu_\alpha\big) + S_\alpha( \phi_\A) &&\alpha \in \CH   \\
\mu_\alpha ={}& \p_{\phi_\alpha} \Psi(\phi_\CH)+ \p_{\phi_\alpha} \Phi(\phi_\A) - \eps^2_\alpha \Delta \phi_\alpha  &&\alpha \in \CH \\
\pt \phi_\beta \!+\!    \div(\phi_\beta v_\beta)                     ={}& \div\big(M_\beta \nabla \big( D_{\beta} \phi_\beta \!+\! \p_{\phi_\beta} \Phi(\phi_\A) \big) \big)\! +\!S_{\beta}( \phi_\A) &&\beta \in \RD \\
\pt \phi_\gamma                        ={}& S_{\gamma}( \phi_\A) &&\gamma \in \OD                     
\end{aligned} \end{gathered}} \end{equation}


\subsubsection{Four-species tumor growth model}
We begin with a straightforward illustration of a tumor growth model based on the suggested multiple constituent system \cref{Eq:MultipleGeneral}.

The article by \cite{hawkins2012numerical} presents the most fundamental model of tumor growth, which forms the basis of this theory.
The volume fractions of cancer cells, healthy cells, nutrient-rich extracellular water, and nutrient-poor extracellular water were taken into account.
Such a system is referred to as the “four-species model,” and \cite{garcke2016global,garcke2017analysis,garcke2017well} investigated the model's mathematical well-posedness.
In addition, we cite \cite{frigeri2015diffuse,frigeri2017diffuse} for an examination of degenerating mobility functions.
Due to the fact that the model is based on a fourth-order PDE with concentration-dependent mobilities, even for the prototype model \cref{Eq:Cahn}, the uniqueness of weak solutions is unresolved; see \cite{elliott1996cahn} for more information.
\cite{colli2017optimal} investigated the four-species model in relation to an optimal control problem, whereas \cite{miranville2019long} and \cite{cavaterra2011cahn} investigated the long-term behavior of the solution.
Various velocity models have been introduced to the four-species model to account for fluid movement in the progression of cancer.
The cells are represented as viscous, inertia-free fluids, and the fluid mixture's velocity is modeled in a volume-averaged sense.
Such an assumption is reasonable, given that the cells are densely packed.
\cite{garcke2016cahn} modeled the velocity by the Darcy law in the four-species model, and \cite{garcke2018cahn} examined this model analytically.
In \cite{ebenbeck2019analysis,ebenbeck2019cahn} and in \cite{fritz2019unsteady}, this law was extended to the Darcy--Brinkman equation and the time-dependent Darcy--Forchheimer--Brinkman equation, respectively.
Authors have also approximated the velocity as a Stokes flow (see \cite{franks2003interactions} and \cite{friedman2006free,friedman2016free}), and the Darcy--Brinkman equation can be viewed as an interpolation between Darcy and Stokes flow.
The inclusion of a velocity equation in a Cahn--Hilliard system is not innovative in and of itself, as it has been done by \cite{lee2002modeling} without the application to tumor growth.
These strategies have been modified to accommodate the new system, which incorporates nontrivial effects such as chemotaxis, proliferation, and nonlinear source functions. 

We choose $\vert\A\vert=2$ constituents and set $\A=\{T,\sigma\}$, $\CH=\{T\}$, $\RD=\{\sigma\}$, and $\OD=\emptyset$. It is understood that the volume fraction of tumor cells $\phi_T$ represents an averaged cell concentration, a homogenized representation of several thousands of cells.
Field $\phi_\sigma$ is representative of the local nutrient content.
In addition, we present the adhesion function $\Phi(\phi_T,\phi_\sigma)=-\chi_c\phi_T \phi_\sigma$ in energy \cref{Eq:GinzburgLandau} for a particular chemotaxis parameter $\chi_c>0$.
For the tumor cells and the nutrients, we assume a volume-averaged velocity.
This assumption of a volume-averaged velocity is fair given the dense packing of the cells.
When all the assumptions are inserted into the multispecies model, the result is the so-called four-species model. 
\begin{equation} \label{Eq:FourSpecies} \boxed{ 
	\begin{gathered} 
	\textbf{Four-species model} \\
	\begin{aligned}
		\pt \phi_T+ \div(\phi_T v) ={}& \div \big(M_T\phi_T^2(1-\phi_T)^2  \nabla \mu_T\big) + S_T( \phi_T,\phi_\sigma)    \\
		\mu_T ={}& \Psi'(\phi_T) - \chi_c \phi_\sigma- \eps_T^2 \Delta \phi_T \\
		\pt \phi_\sigma +  \div(\phi_\sigma v)                     ={}& \div\big(M_\sigma \nabla( D_{\sigma} \phi_\sigma -\chi_c\phi_T  ) \big) +S_{\sigma}( \phi_T,\phi_\sigma) 
\end{aligned} \end{gathered} } \end{equation}
In the case of an absent velocity $v=0$, this model is studied in \cite{garcke2017analysis,garcke2017well} with respect to the existence of weak solutions. If the flow is governed by Darcy's law $$\begin{aligned} v&=-K\nabla p + S_v(\phi_T,\phi_\sigma), \\ \div \, v &= 0,\end{aligned}$$ then we refer to \cite{garcke2016cahn} and \cite{garcke2016global}. The pressure is denoted by $p$, the permeability factor by $K>0$, and $S_v$ is called the Korteweg force \cite{frigeri2018on}. Alternatively, the flow has been governed by the Brinkman law \citep{ebenbeck2019analysis,ebenbeck2019cahn}, the unsteady Darcy--Forchheimer--Brinkman law \citep{fritz2019unsteady}, and the Navier--Stokes equations \citep{lam2017thermodynamically,he2021global} in literature. Numerically, we present a comparison of different flow models and their influence in the four-species model, see \cref{Fig:Image_Flow}. We refer to \cref{Sec:Numerics} below for further details on the techniques for discretizing the PDEs in time and space. We notice that the flow is highly influential on the evolution of the tumor by drastically changing the growth directions of the tumor mass.
\begin{figure}[htb!]
    \centering
    \includegraphics[width=.8\textwidth]{Images/Image_Flow.pdf}
    \caption{Evolution of tumor mass $\phi_T$ with a slightly elliptic initial condition on the 9th, 15th, 21st and 27th day; we present three different variations of the model: I. without velocity, II. unsteady Darcy--Brinkman law, III. unsteady Darcy--Forchheimer--Brinkman law; figure taken with permission from Figure 7 in \cite{fritz2019unsteady}.}
    \label{Fig:Image_Flow}
\end{figure}

Source functions that are expressed as sink and source terms are of particular importance.
Tumors absorb the nutrients; hence, tumor growth is proportional to nutrient depletion.
In addition, programmed cell death (also known as apoptosis) occurs, and these dead cells become nutrients.
Consequently, we consider the source function
$$S_T(\phi_T,\phi_\sigma)=-S_\sigma(\phi_T,\phi_\sigma)=\lambda^\pro_T \phi_\sigma\phi_T(1-\phi_T) - \lambda^\apo_T \phi_T,$$
where $\lambda_T^\pro$ is called proliferation rate and $\lambda_T^\apo$ apoptosis rate.

The system \cref{Eq:FourSpecies} is also referred to as the “four-species model,” \citep{hawkins2012numerical,oden2010general,lima2014hybrid} because it can be derived from four constituents: the volume fraction of tumor cells $\phi_T$, healthy cells $\phi_C$, nutrient-rich extracellular water $\phi_\sigma$, and its nutrient-poor counterpart $\phi_{\sigma_0}$.
Consequently, the four variables are governed by the law of mass balance, see \cref{Eq:MassBalance}, for $\A=\{T,C,\sigma,\sigma_0\}$. One sets $\phi_T=1-\phi_C$ and $\phi_\sigma=1-\phi_{\sigma_0}$. Thus, one can eliminate the superfluous constituents $\phi_C$ and $\phi_{\sigma_0}$ from the system and obtains the four-species system \cref{Eq:FourSpecies}.




\subsection{Phase separation in an ECM} \label{Sec:Mod:ECM}
The “microenvironment” of a solid tumor is a patch of vascularized tissue in a living subject, such as within an organ, that contains a colony of tumor cells and other components.
The tumor is contained within an open-bounded region $\Omega \subset \R^3$ and is supported by a network of collagen, enzymes, and other proteins that comprise the extracellular matrix (ECM).
We are focusing on developing phenomenological descriptions of tumor cell colony growth that capture both mesoscale and macroscale phenomena.


When tumor cells endure hypoxia or necrosis, these four-species models are inadequate for representing the formation of an early tumor whose evolution is primarily determined by proliferation.
Indeed, a larger and more advanced tumor tends to become stratified \citep{roose2007mathematical}, meaning that the tumor tissue is subdivided into numerous layers, each with its own properties.
Typically, tumors are separated into three phases: \medskip
\begin{itemize} \itemsep0.5em
	\item Rapidly proliferating outer rim.
	\item Intermediate quiescent layer with cells suffering from hypoxia.
	\item Necrotic core with perished cells. \medskip
\end{itemize}
	Multiphase models with multiple cell species and nutrients have been studied in the works \cite{wise2008three}, \cite{escher2011analysis}, \cite{sciume2014three}, \cite{garcke2018multiphase}, \cite{araujo2004history}, \cite{astanin2008multiphase}, \cite{frieboes2010three}, \cite{frigeri2018on}, \cite{dai2017analysis}, and \cite{fritz2019local,fritz2021analysis,fritz2021modeling}.
In the hypoxic phase, tumor cells are quiescent and release matrix-degrading enzymes (MDEs), which degrade the ECM and allow nutrients to flow.
This procedure allows tumor cells to move into the tissue and is the initial stage in simulating metastasis.
Simply put, the ECM works as a wall that regulates the flow of nutrients around the tumor.
Several authors \citep{chaplain2011mathematical,engwer2017structured,stinner2014global,sfakianakis2020hybrid,shuttleworth2020cell,sciume2014tumor} have examined the ECM in reaction-diffusion type tumor models.
We investigated the ECM in a Cahn--Hilliard type model \citep{fritz2019local}, and it was also included in our successive research \citep{fritz2021analysis,fritz2021modeling}.
 

The field of the tumor cells $\phi_T$ can be represented by the sum $$\phi_T=\phi_P+\phi_H+\phi_N,$$ of the three components $\phi_P$, $\phi_H$, $\phi_N$ that describe the volume fractions of the proliferative, hypoxic, and necrotic cells, respectively. They are characterized by: \medskip

 \begin{itemize} \itemsep.5em
 	\item Proliferative cells $\phi_P$ are those with a high probability of undergoing mitosis, dividing into twin cells, and fostering tumor growth. 
 	\item Hypoxic cells $\phi_H$ are tumor cells that lack sufficient resources, such as oxygen, to proliferate or continue to proliferate. 
 	\item Necrotic cells $\phi_N$ have died owing to nutrient deficiency. \medskip
 	\end{itemize} 
  
\noindent In response to hypoxia, tumor cells produce an enzyme that promotes cell motility and stimulates the secretion of angiogenesis-stimulating substances $\phi_\TAF$.
The most frequently mentioned of these substances is vascular endothelial growth factor (VEGF), which induces endothelial cells to proliferate and form the tubular shape of blood vessels, which then extend to form new arteries that supply nutrition to hypoxic cells.

In addition, hypoxic cells release MDEs such as urokinase-plasminogen and matrix metalloproteinases, as indicated by the volume fraction $\phi_\MDE$, which erode the ECM, whose density is represented by $\phi_\ECM$.
This procedure permits tumor cells $\phi_T$ to infiltrate, hence increasing the number of tumor cells in the ECM domain and the probability of metastasis.
The following is a simplified explanation of the impacts of the tumor's evolution and it is also depicted in \cref{Fig:Separation}. \medskip

\begin{figure}[htb!]
    \centering
    \includegraphics[width=.245\textwidth]{Images/steps1.pdf}\!
\includegraphics[width=.245\textwidth]{Images/steps2.pdf}\!
\includegraphics[width=.245\textwidth]{Images/steps3.pdf}\!
\includegraphics[width=.245\textwidth]{Images/steps4.pdf}
    \caption{Depiction of angiogenesis and growth of capillaries after the proliferative tumor phase becomes hypoxic due to nutrient shortage.}
    \label{Fig:Separation}
\end{figure}

\begin{enumerate} \itemsep.5em
	\item[(1)] Outer proliferative layer absorbs nutrients and expands ($\phi_P\!\!\uparrow$,  $\phi_{\sigma}\!\downarrow$).
	\item[(2)] Inner tumor layer changes to hypoxic ($\phi_H\!\uparrow$).
	\item[(3)] Tumor core changes to necrotic ($\phi_N \uparrow$).
	\item[(4)] Hypoxic cells send out MDEs and TAFs ($\taf\!\uparrow$, $\mde\!\uparrow$).
	\item[(5)] TAFs trigger angiogenesis and initiate the sprouting of vessels  ($\phi_H\!\!\downarrow, \phi_P\!\uparrow$), \\[0.1cm] and MDEs erode the ECM, i.e., tumor cells migrate  ($\ecm\!\downarrow$, $\phi_H\!\downarrow, \phi_P\!\uparrow$). \medskip
\end{enumerate}




\noindent We collect the constituents within the following tuple:
$$
\phi_\A = (\phi_\alpha)_{\alpha \in \A} = ( \phi_P, \phi_H, \phi_N, \phi_\sigma, \ecm, \mde, \taf  ),
$$
with $\A=\{P,H,N,\sigma,\ECM,\MDE,\TAF\}$. We differentiate between the tumor phase-field indices $\CH=\{P,H,N\}$, the reaction-diffusion indices $\RD=\{\sigma,\MDE,\TAF\}$, and the evolution index set $\OD=\{\ECM\}$ using the setup of the multiple constituent model \cref{Eq:MultipleGeneral} in \cref{Sec:Mod:Mult}.
The necrotic cells are immobile and only gain mass from the hypoxic cells, which lack nutrients.
Therefore, the necrotic cells' mobility is set to zero, i.e., it holds $m_N=v_N=0$.
Still, necrotic cells are counted as a phase-field variable and constitute a component of $\CH$ rather than the ODEs because they influence the double-well potential and inherit their phase-field structure from the hypoxic phase-field variable.
Assuming that haptotaxis and chemotaxis are part of the system, we calculate the adhesion force $$\Phi(\phi_\A)=-(\phi_P+\phi_H)(\chi_c \phi_\sigma + \chi_h \phi_\ECM),$$ 
where $\chi_c$ and $\chi_h$ are the chemotaxis and haptotaxis components, respectively.
The adhesion force only operates on live (proliferative and hypoxic) cells, while necrotic cells are excluded from this process.
Consequently, the equations for the phase-field variables $(\phi_\alpha)_{\alpha \in \CH}$ are derived from the multiple constituent model \cref{Eq:MultipleGeneral} and read as follows:
\begin{equation} \label{Eq:Stratified} \boxed{
	\begin{gathered} 
	\textbf{Stratified tumor growth model with ECM: $\CH$} \\
	\begin{aligned}
		\pt \phi_P+ \div(\phi_P v) ={}& \div \big(M_P \phi_P^2(1-\phi_P)^2 \nabla \mu_P\big) + S_P( \phi_\A)    \\
		\mu_P                      ={}&   \p_{\phi_P} \Psi(\phi_\CH) - \eps^2_P \Delta \phi_P - \chi_c \phi_\sigma-\chi_h \ecm \\
		\pt \phi_H+ \div(\phi_H v) ={}& \div \big(M_H \phi_H^2(1-\phi_H)^2 \nabla \mu_H\big)+S_H( \phi_\A)      \\
		\mu_H                      ={}&   \p_{\phi_H} \Psi(\phi_\CH) - \eps^2_H \Delta \phi_H - \chi_c \phi_\sigma-\chi_h \ecm \\
		\pt \phi_N                 ={}& S_N( \phi_\A)
	\end{aligned} \end{gathered} }\end{equation}
We assume a volume-averaged velocity $v=v_\alpha$ for the fields $\phi_P$, $\phi_H$ and $\phi_\sigma$, which shall be governed by Darcy law for the sake of simplicity. Moreover, we consider the following source functions
$$
	\begin{aligned}
S_P(\phi_\A) ={}& \lambda_P^\pro \phi_\sigma \phi_P(1- \phi_T) - \lambda_P^\apo \phi_P - \lambda_{P\!H}  \calH(\sigma_{P\!H} - \phi_\sigma)\phi_P \\& + \lambda_{H\!P} \calH(\phi_\sigma - \sigma_{H\!P}) \phi_H,  \\
S_H(\phi_\A) ={}&  \lambda_{H}^\pro \phi_\sigma \phi_H (1-\phi_T)-\lambda_H^\apo \phi_H + \lambda_{P\!H}  \calH(\sigma_{P\!H} - \phi_\sigma) \phi_P  \\& - \lambda_{H\!P} \calH(\phi_\sigma - \sigma_{H\!P}) \phi_H - \lambda_{H\!N} \calH(\sigma_{H\!N} - \phi_\sigma)\phi_H,  \\
S_N(\phi_\A) ={}& \lambda_{H\!N} \calH(\sigma_{H\!N} - \phi_\sigma) \phi_H.
	\end{aligned}
$$
The parameters $\lambda_\alpha^\pro$ and $\lambda_\alpha^\apo$ are the proliferation and apoptosis rates corresponding to the $\alpha$-th species. Furthermore, $\lambda_{P\!H}$ denotes the transition rate from the proliferative to the hypoxic phase below the nutrient level $\sigma_{P\!H}$, $\lambda_{H\!P}$ the transition rate from the hypoxic to the proliferative phase above the nutrient level $\sigma_{H\!P}$, and $\lambda_{H\!N}$ the transition rate from the hypoxic to the necrotic phase below the nutrient level $\sigma_{H\!N}$. Lastly, $\calH$ represents the Heaviside step function that can be replaced by the Sigmoid function if a sufficiently smooth right-hand side is required.

In the instance of diffusion-type models, \cite{tao2011chemotaxis, tao2014energy}, \cite{engwer2017structured} and \cite{sfakianakis2020hybrid} discuss related theories of ECM degradation due to MDEs generated by hypoxic cell concentrations and subsequent tumor invasion and metastasis.  Following these references, we present an ECM evolution equation in the form of:
$$
	\boxed{\begin{gathered} 
			\textbf{Stratified tumor growth model with ECM: $\OD$} \\\begin{aligned}
	\pt \ecm ={}& S_{\ECM}( \phi_\A) \\
	={}&-\lambda_{\ECM}^\deg \ecm\mde + \lambda_{\ECM}^\pro \phi_\sigma (1- \ecm) \calH(\ecm - \phi_{\ECM}^\pro)
	\end{aligned}  \end{gathered} }
$$
Here, $\lambda_{\ECM}^\deg$ denotes the degradation rate of ECM fibers due to the matrix degrading enzymes, and $\lambda_{\ECM}^\pro$ is the production rate of ECM fibers above the threshold level $\phi_{\ECM}^\pro$. Further, for $(\phi_\beta)_{\beta \in \RD}$ we arrive at the following set of equations:
$$\boxed{\begin{gathered} 
	\textbf{Stratified tumor growth model with ECM: $\RD$} \\
	\begin{aligned}
	\pt \phi_\sigma + \div(\phi_\sigma v) ={}& \div \big(M_\sigma \nabla (D_\sigma \phi_\sigma - \chi_c  (\phi_P +\phi_H )\big)+S_\sigma( \phi_\A) \\
	\pt \mde                        ={}& M_{\MDE} D_{\MDE} \Delta \mde+S_{\MDE}( \phi_\A)
	\\
	\pt \taf                        ={}& M_{\TAF} D_{\TAF} \Delta \taf+S_{\TAF}( \phi_\A)                      
\end{aligned}\end{gathered}} $$
where the source functions are given by
$$
	\begin{aligned}
S_{\sigma}(\phi_\A) ={}&  \lambda_P^\apo \phi_P + \lambda_H^\apo \phi_H-\lambda_P^\pro \phi_\sigma\phi_P(1-\phi_T) -\lambda_{H}^\pro \phi_\sigma\phi_H(1-\phi_T)   \\
& + \lambda_{\ECM}^\deg \ecm\mde- \lambda_{\ECM}^\pro \phi_\sigma (1- \ecm) \calH(\ecm - \phi_{\ECM}^\pro) ,  \\
S_{\MDE}(\phi_\A) ={}&  \lambda_{\MDE}^\pro (\phi_P + \phi_H) \ecm\frac{\sigma_{H\!P}}{\sigma_{H\!P} + \phi_\sigma} (1-\mde)-\lambda_{\MDE}^\deg \mde  \\[-.25cm] &- \lambda_{\ECM}^\deg \ecm \mde,  \\
S_{\TAF}(\phi_\A) ={}& \lambda_{\TAF}^\pro (1- \taf) \phi_H \calH(\phi_H-\phi_{H}^\pro) - \lambda_{\TAF}^\deg \taf.  
\end{aligned}
$$
The parameters $\lambda_{\MDE}^\deg$ and $\lambda_{\TAF}^\deg$ denote the decay rates of the MDEs and TAFs, respectively. Moreover, $\lambda_{\MDE}^\pro$  represents the production rate of MDEs, and $\lambda_{\TAF}^\pro$ is the production rate of the $\taf$ due to the release by hypoxic cells above a threshold value of $\phi_{H}^\pro$. 

We notice that the cell species $\phi_\alpha$, $\alpha \in \{P,H,N,\sigma,\ECM\}$, form a mass conserving subsystem in the sense that their source terms add to zero. The constituents  $\phi_\MDE$ and $\phi_\TAF$ do not belong to a mass exchanging closed subsystem since they are signals and show natural degradation factors that are not absorbed by the other constituents.

Numerically, we depict a simulation of a tumor with the degradation of the ECM in \cref{Fig:Image_ECM}. The viable part of the tumor consists of the proliferative and hypoxic phase. It absorbs the nutrients and starts to grow until $t=5$. Then the nutrients are sufficiently deprived in the sense that a necrotic core forms. The tumor moves towards the right and cell-to-cell and cell-to-matrix adhesion effects can be observed, i.e., tumor cells move towards nutrients due to chemotaxis and tumor cells move towards the ECM due to haptotaxis.

\begin{figure}[htb!]
    \centering
    \includegraphics[width=.95\textwidth]{Images/NumericsECM2D.pdf}
    \caption{Evolution of the tumor mass under the influence of the extracellular matrix, taken with permission from Figure 1 and 2 in \cite{fritz2019unsteady}.}
    \label{Fig:Image_ECM}
\end{figure}

\subsection{Nonlocal phenomena} \label{Sec:Mod:Nonlocal}
In this section, the nonlocal impacts of tumor evolution models are discussed.
There are two types of nonlocality: spatial and temporal.
The first phenomenon relates to a time-fractional derivative in the PDE and is known as the memory effect.
In the second scenario, one must deal with a space integral, which reflects long-range interactions.

In addition, nonlocal events are incorporated into mathematical models of cancer cells.
These effects demonstrate long-distance interactions and may be geographical or temporal in character.
In the situation of spatial nonlocality, cell-to-matrix and cell-to-cell adhesion qualities are crucial to tumor growth modeling and encourage the proliferation of tumor cells.
Due to the structure of integro-differential systems, these events are nonlocal in space and require a special mathematical treatment.
\cite{fritz2019unsteady} explored cell-to-cell adhesion, whereas \cite{fritz2019local} investigated cell-to-matrix adhesion.
Further, we mention the articles by \cite{scarpa2021class} and \cite{frigeri2017diffuse} that studied nonlocal cell-to-cell adhesion properties  in phase-field models with applications to tumor growth.

In the case of temporal nonlocality, not only does the outcome of the previous step affect the current evolution, but it is also taken into account that cells have innate memories \citep{meir2020single}.
The past consequently effects the present.
In contrast to the normal Fickian diffusion process, memory effects are handled using a time-fractional derivative and fractional heat equations reflect the process of subdiffusivity.
As evidenced by  the in vitro and in vivo experimental findings of \cite{jiang2014anomalous}, tumors migrate via both traditional Fickian diffusion and subdiffusion.
\cite{fritz2021timefractional,fritz2021equivalence} investigated the memory effect in connection to the time-fractional Cahn--Hilliard equation with degenerating mobility.
Additionally, \cite{fritz2021subdiffusive} examined a fractional tumor model including subdiffusion, nutritional couplings, and mechanical deformation. 

\subsubsection{Nonlocal-in-space: cell-to-cell and cell-to-matrix adhesion}
If events or cell concentrations at one site in the tumor domain are dependent on events at other points within a defined neighborhood, it is said that the model is spatially nonlocal.
Long-distance interactions, such as cell-to-cell adhesion, are among the several processes that affect the mobility and migration of tumor cells.
cell-to-cell adhesion is a crucial aspect of tissue formation, stability, and degeneration, as well as a major contributor to cancer cell invasion and metastasis. 

Following \cite{chaplain2011mathematical} and \cite{frigeri2017diffuse}, we address cell-to-cell adhesion effects, which are responsible for the binding of two or more cells via protein processes on their respective cell surfaces.
The Ginzburg--Landau free energy functional generates separation and surface tension effects \citep{frigeri2017diffuse}, hence it is reasonable to incorporate cell-to-cell adhesion.
Therefore, tumor cells prefer to adhere to each other rather than healthy cells.
The physicists \cite{giacomin1996exact,giacomin1997phase} studied the problem of phase separation from a microscopic background using statistical mechanics and obtained the Helmholtz energy functional
$$\begin{aligned}
\mathcal{E}(\phi_T)={}&\int_\Omega \Psi(\phi_T)\, \text{d}x  + \frac14 \int_\Omega \int_\Omega J(x-y) \big(\phi_T(x)-\phi_T(y)\big)^2 \, \text{d}y \,\text{d}x.\end{aligned}$$
In this equation, we assume that $J:\mathbb{R}^d \to \mathbb{R}$ is a convolution kernel with the essential symmetry property $J(-x)=J(x)$. One obtains the Ginzburg--Landau energy by choosing a particular kernel sequence and passing the limit \citep{frigeri2015a}.
We modify the energy to account for chemotaxis and consider
\begin{equation*} \begin{aligned}
\mathcal{E}(\phi_T,\phi_\sigma) =&\int_\Omega \Psi(\phi_T) + \frac{D_\sigma}{2} \phi_\sigma^2 - \chi_c \phi_T \phi_\sigma \text{d}x \\ &+ \frac14 \int_\Omega \int_\Omega J(x-y) \big(\phi_T(x)-\phi_T(y)\big)^2 \, \text{d}y \, \text{d}x. \end{aligned}
\end{equation*}
Hence, we propose a class of long-range interactions, which are represented by chemical potentials of the form
\begin{equation*} \label{mu}
\mu_T = \frac{\delta \mathcal{E}}{\delta \phi_T} = \Psi'(\phi_T) - \chi_c \phi_\sigma + \int_\Omega J(x-y) \big(\phi_T(x)-\phi_T(y)\big) \, \textup{d} y.
\end{equation*}
This immediately results in the nonlocal system:
\begin{equation} \boxed{
	\begin{gathered} 
	\textbf{Four-species model with cell-to-cell adhesion} \\
	\begin{aligned}
	\partial_t \phi_T + \div(\phi_T v) ={}&  \div \big(M_T \phi_T^2(1-\phi_T)^2 \nabla \mu_T\big) + S_T(\phi_T,\phi_\sigma) \\
	\mu_T ={}& \Psi'(\phi_T)- \chi_c \phi_\sigma + \phi_T \cdot J*1 - J*\phi_T  \\
	\partial_t \phi_\sigma + \div(\phi_\sigma v) ={}&  \div\big(M_\sigma \nabla (D_\sigma  \phi_\sigma- \chi_c  \phi_T) \big) + S_\sigma(\phi_T,\phi_\sigma)
	\end{aligned}\end{gathered} }
\label{Eq:Nonlocal} \end{equation}

Included in models that account for cell-to-matrix adhesion effects are MDEs that erode the ECM; hence, this mechanism permits cell migration into tissue.
Such systems have been extensively investigated in \cite{engwer2017structured} and \cite{chaplain2011mathematical}.
In contrast to the fourth-order Cahn--Hilliard phase-field equation in our case, a reaction-diffusion equation is used to describe the tumor volume fraction in these publications.
The cell-to-matrix adhesion flux can be categorized as a local gradient-based haptotaxis effect \citep{stinner2014global,tao2011chemotaxis,walker2007global} or a nonlocal adhesion-based haptotaxis effect \citep{armstrong2006continuum,chaplain2011mathematical,gerisch2008mathematical}.
We consider the respective fluxes of the form
$$
	J_\alpha(\phi_\A) = \chi_h \phi_V \cdot \begin{cases} \nabla \ecm, &\alpha = \text{local}, \\ k*\ecm, &\alpha=\text{nonlocal},\end{cases}
$$
where $k$ is a vector-valued kernel function. This adhesion flux is factored into the equation of the extended mass balance law for the volume fraction of viable cells as follows: 
$$\boxed{\pt \phi_V + \div(\phi_V v) = \div(m_V(\phi_\A) \nabla \mu_V) + \div J_\alpha + S_V(\phi_\A)}$$

In the following, we numerically investigate the effects of the different haptotaxis parameters on the growth of the tumor mass in \cref{Fig:Image_Nonlocal}. We distinguish between three different values, and we compare the local gradient-based adhesion flux to the nonlocal one with two different values of $\eps$. We notice that different pairing of $\eps$ and $\chi_h$ result in similar simulations. A larger value of $\chi_h$ results in a larger difference between the local and nonlocal model. 
Comparing the local and nonlocal models, we find a greater adhesion impact in the local model, as the tumor mass moves further to the right of the boundary where nourishment is put in the local model. By selecting a smaller haptotaxis parameter, the local model can resemble the nonlocal model for a fixed value of $\eps$. The subtleties of the discretization of the nonlocal model is further discussed in \cref{Sec:Numerics}. 
\begin{figure}[htb!]
    \centering
    \includegraphics[width=.9\textwidth,page=4]{Images/Image_ECM.pdf}
    \caption{Simulation of the tumor volume fraction $\phi_T$ for different haptotaxis parameters $\chi_h \in \{0.0005, 0.001, 0.002\}$ and different values of $\eps \in \{\eps_1,\eps_2,\eps_3\}=\{0,0.00275,0.00525\}$ for a fixed time; taken with permission from Figure 5 in \cite{fritz2019local}.}
    \label{Fig:Image_Nonlocal}
\end{figure}


\subsubsection{Nonlocal-in-time: The memory effect}
According to \cite{Balkwill12}, \cite{wang17} and \cite{yuan16}, the tumor microenvironment significantly influences the proliferation and migration of tumor cells.
In addition to Fickian diffusion and subdiffusion, tumor cells travel through a variety of methods.
The results of the experiments of \cite{jiang2014anomalous} show anomalous diffusion in the progression of cancer.
In addition to clinical data from patients with adrenal and liver tumors, they discovered subdiffusion during in vitro tests of generating cultured cells from the breast line and during in vitro trials of developing cultured cells from the liver line.

In earlier sections, the phenomenological law $J_T=-m_T(\phi_\A)\nabla\mu_T$ was used to depict the typical relationship between flow and the gradient of the chemical potential.
A more complicated phenomenological link that could account for hypothesized nonlocal, nonlinear, and memory effects \citep{gorenflo2002time, povstenko2017two}, can be substituted for this law without contradicting the conservation law suggested by the continuity equation.
\cite{seki2003recombination} and \cite{yuste2004reaction} simulate subdiffusion-limited reactions on a tiny scale by employing fractional derivatives in flux and reaction terms.
Consequently, we propose
$$
\begin{aligned}
J^\rel_T(\phi_\A)&=-\p_t \big (g_\alpha \circledast (m_T(\phi_\A)\nabla\mu_T)\big),\\
S^\rel_T(\phi_\A)&=\p_t \big(g_\alpha \circledast S_T(\phi_\A) \big),
\end{aligned}
$$
for $\alpha\in(0,1)$. Inserting the relaxed flux and source  into the law of conservation of mass \cref{Eq:MassBalance}, it yields for the tumor species $\phi_T$ 
\begin{equation*}
\begin{aligned}
\p_t \phi_T ={}& -\div J_T^\rel(\phi_\A) + S_T^\rel(\phi_\A) = \p_t \big( g_\alpha \circledast \big( \div(m_T(\phi_\A)\nabla\mu_T) +  S_T(\phi_\A) \big) \big).
\end{aligned}
\end{equation*}
We can equivalently rewrite this equation by taking the convolution with $g_{1-\alpha}$ on both sides of the equation and using the inverse convolution property \cref{Eq:InverseConvolution}. Therefore, we obtain
\begin{equation*}  \label{Eq:CHCaputo}
\boxed{\begin{aligned}
\p_t^\alpha \phi_T ={}& \div(m_T(\phi_\A)\nabla\mu_T) + S_T(\phi_\A).
\end{aligned}}
\end{equation*}
The chemical potential reads $\mu_T=\Psi'(\phi_T)-\eps_T^2\Delta \phi_T$ in the case of the Ginzburg--Landau energy \cref{Eq:Ginzburg}. This model is called the time-fractional Cahn--Hilliard equation  \citep{fritz2021timefractional,fritz2021equivalence}. If one selects the Dirichlet energy $\calE(\phi_T)=\int_\Omega \phi_T^2 \dx$, then one obtains the time-fractional reaction-diffusion equation as studied in a tumor growth setting in \cite{fritz2021subdiffusive}.

Typically, subdiffusive models do not exhibit linear growth if enough nutrients are in the tumor environment, as it can be typically observed for integer-order tumor growth models. Indeed, subdiffusive models have a larger growth in the beginning and the growth becomes damped afterwards. This can be explained by the memory effect of cells that first try to absorb everything that they can get and, afterwards, become more lenient with the available nutrients. The details for the implementation of the time-fractional derivative are considered in \cref{Sec:Numerics}.
\begin{figure}[htb!]
    \centering
    \includegraphics[width=.6\textwidth]{Images/Image_Frac.pdf}
    \caption{Evolution of tumor mass $\phi_T$ with different parameters of the fractional order $\alpha$; taken with permission from Figure 5 in \cite{fritz2021timefractional}.}
    \label{Fig:Image_Frac}
\end{figure}

\subsection{Uncertainty in tumor modeling} \label{Sec:Mod:Stochastic}
In thin subdomains at the interfaces of the phase fields, stochastic variations of the phase concentrations are possible, see the works by \cite{orrieri2020optimal} and \cite{fritz2021modeling,fritz2023stochastic}. The variances of these regions of random behavior are constrained by noise parameters $\phi_\alpha^\omega$ with noise intensity $\omega_\alpha$. In fact, the variations in $\phi_\alpha$ are restricted to interface regions by making use of the operator
$$G_\alpha(\phi_\alpha)=\omega_\alpha \mathcal{H}((\phi_\alpha-\phi_\alpha^\omega)(1-\phi_\alpha-\phi_\alpha^\omega)).$$ %
Generally, the model incorporates the randomness in the evolution of species along the interface as a cylindrical Wiener process on $L^2(\Omega)$. We refer to \cite{da1996stochastic}, \cite{cardon2001cahn}, and \cite{elezovic1991stochastic} on the stochastic Cahn--Hilliard equation. Further details on stochastic PDEs can be found in the textbooks \cite{lord2014introduction}, \cite{prevot2007concise}, and \cite{liu2015stochastic}. 

Modeling-wise, we add $G_\alpha \dot W_\alpha$  to the mass balance equation for $\phi_\alpha$ and to keep the mass balance equations in standard form, we slightly abuse the standard notation by writing $\dot W_\alpha$ in the sense $\dot W_\alpha \dt = \dd W_\alpha$. In the case of the simplified four-species model, we obtain the following stochastic version of it.

$$\label{Eq:Wiener} \boxed{
		\begin{gathered}
	\textbf{Stochastic four-species model} \\
	\begin{aligned} 
	\pt \phi_T \!+\! \div(\phi_T v) ={}& \div (M_T \phi_T^2(1-\phi_T)^2 \nabla \mu_T) + S_T( \phi_T,\phi_\sigma) + G_\alpha(\phi_T) \omega_T \text{d}W_T    \\
	\mu_T ={}& \Psi'(\phi_T) - \chi_c \phi_\sigma- \eps_T^2 \Delta \phi_T \\
	\pt \phi_\sigma \!+\!  \div(\phi_\sigma v)                     ={}& \div\Big(M_\sigma \nabla\big( D_{\sigma} \phi_\sigma -\chi_c\phi_T  \big) \Big) +S_{\sigma}( \phi_T,\phi_\sigma) \\
	\end{aligned}\end{gathered} } $$

 Numerically, we investigate the influence of the stochasticity in the tumor growth model in \cref{Fig:stochastic}. We first consider the deterministic model that corresponds to $\omega_T=0$ and, afterwards, compare it to the cases with two different values for $\omega_T$. We notice that a larger value of $\omega_T$ results in a non-regular shape of the tumor's interface. The implementation of the Wiener process is discussed in \cref{Sec:Numerics}.

\begin{figure}[htb!]
    \centering
    \includegraphics[width=.9\textwidth]{Images/wiener.pdf}
    \caption{Evolution of the tumor volume fraction for different values of the noise intensity $\omega_T$; we consider the cases of $\omega_T\in\{\omega_1,\omega_2,\omega_3\}=\{0,0.001,0.1\}$.}
    \label{Fig:stochastic}
\end{figure}


\subsection{Mechanical deformation} \label{Sec:Mod:Mechanical}
As the tumor grows, the surrounding host tissues generate mechanical stress, restricting the tumor's growth.
In the papers \citep{faghihi2020coupled,lima2016selection,lima2017selection}, mechanical deformation in a tumor development model was first mentioned, and in terms of analysis, it was first examined in \cite{fritz2021subdiffusive} in a diffusion-type tumor model and subsequently \cite{garcke2021phase} in a Cahn--Hilliard type system.
Such models with elasticity are referred to as Cahn--Larch\'e equations. It had previously been incorporated into the Cahn--Hilliard equation without being applied to tumor growth or traditional source variables in \cite{garcke2003cahn,garcke2005cahn}.

As the tumor grows, the surrounding host tissues generate mechanical stress, restricting the tumor's growth.
Regarding mathematical modeling and sensitivity studies, several works \citep{lima2016selection,lima2017selection,hormuth2018mechanically,faghihi2020coupled} have employed reaction-diffusion equations with mechanical coupling to predict tumor progression.
\cite{fritz2021subdiffusive} examined the well-posedness of a model in which similar mechanical factors were incorporated.
The underlying energy functional now contains the stored energy potential $W(\phi_T,\eps(u))$, which is dependent on the tumor volume fraction $\phi_T$ and the symmetric strain measure $\eps(u)=\frac12(\nabla u+\nabla u^\top)$ of the displacement field $u$.
Assuming minor deformations, we consider the specific stored energy potential 
\begin{equation} \label{Eq:Stored}
W(\phi_T,\eps(u)) = \frac{1}{2} \eps(u):T_M(\phi_T)\eps(u) + \eps(u):T_S(\phi_T),
\end{equation}
where $T_S(\phi_T)=\lambda \phi_T \bbI$ is the symmetric compositional stress tensor with $\lambda>0$ and $T_M$ is the linear elastic inhomogeneous material tensor.
The symbol $\bbI$ signifies the $(d\times d)$-dimensional identity matrix in this instance.
The displacement field $u$ is governed by the conservation equation of linear and angular momentum
$$
\begin{aligned}
\p_t (\phi_T v) + \div(\phi_T v\otimes v) &= \div\, T_C + \phi_T b + p, \\
T_C- T_C^\top &= \text{m},
\end{aligned}
$$
where $v$ is the volume-averaged velocity, $b$ is the body force, $p$ is the momentum contributed by other components, and $\text{m}$ is the intrinsic moment of momentum.
First variations of the energy functional $\mathcal{E}$ with respect to $\phi_T$ and $\eps(u)$, respectively, determine the chemical potential $\mu_T$ and Cauchy stress tensor $T_C$.
We minimize the system's complexity by using the typical simplification assumptions of \cite{lima2016selection}.
Particularly, we assume constant mass density $\text{m}=0$ and a monopolar material $b=0$.
In addition, we disregard inertial forces and set $\div(\phi_T v\otimes v)=p=0$.
We assume that the mechanical equilibrium is reached quicker than diffusion, i.e., that the time derivative on the left-hand side disappears.
After the simplifications, the mechanical deformation equation \cref{Eq:Stored} becomes 
$$0 = \div\, T_C = \div\, \frac{\delta \calE(\phi_T,\phi_\sigma,\eps(u))}{\delta \eps(u)} = \div\, \frac{\partial W(\phi_T,\eps(u))}{\partial \eps(u)}.$$
We assume that the tumor is an isotropic and homogeneous material, i.e., that its material tensor $C_M(\phi)=C_M$ has the form $$C_M\eps(u)=2G\eps(u)+\frac{2G\nu}{1-2\nu} \tr\,\eps(u)\bbI,$$ where $G>0$ and $\nu<\frac{1}{2}$ represent the shear modulus and Poisson ratio, respectively.
For the stored energy potential $$
W(\phi_T,\eps(u)) = \frac{1}{2} \eps(u):(2G\eps(u)+\frac{2G\nu}{1-2\nu} \tr\,\eps(u)\bbI)\eps(u) + \eps(u):(\lambda \phi_T \bbI).
$$
and its partial derivatives with respect to $\phi_T$ and $\eps(u)$, we may therefore write 
$$\begin{aligned}\frac{\partial W(\phi_T,\eps(u))}{\partial \phi_T} &= \lambda \div u, \\
\frac{\p W(\phi_T,\eps(u))}{\p \eps(u)} &= 2 G \eps(u) + \frac{2 G \nu}{1-2\nu} \tr(\eps(u)) \bbI + \lambda \phi_T \bbI.
\end{aligned}$$

With mechanical deformation, it provides the model
$$  \boxed{
	\begin{gathered}
	\textbf{Four-species model with mechanical deformation} \\
	 \begin{aligned}
	\pt \phi_T+ \div(\phi_T v) ={}& \div \big(M_T \phi_T^2(1-\phi_T)^2  \nabla \mu_T\big) + S_T( \phi_T,\phi_\sigma)    \\
	\pt \phi_\sigma +  \div(\phi_\sigma v)                     ={}& \div\big(M_\sigma \nabla( D_{\sigma} \phi_\sigma -\chi_c\phi_T  ) \big) +S_{\sigma}( \phi_T,\phi_\sigma) \\
	0={}& \div \Big(2 G \eps(u) + \frac{2 G \nu}{1-2\nu} \tr(\eps(u)) \bbI + \lambda \phi_T \bbI\Big)  
	\end{aligned} \end{gathered}} $$
As \cite{fritz2021subdiffusive} demonstrated in their study, the Ginzburg--Landau energy yields $$\mu_T = \Psi'(\phi_T) - \chi_c \phi_\sigma- \eps_T^2 \Delta \phi_T + \lambda \div u,$$ whereas the Dirichlet energy yields $\mu_T =D_T \phi_T - \chi_c \phi_\sigma + \lambda \div u$, i.e.,
$$\calE(\phi_T,\phi_\sigma)= \int_\Omega \Big\{ \frac{D_T}{2} \phi_T^2 + \frac{D_\sigma}{2} \phi_\sigma^2 - \chi_c \phi_T \phi_\sigma + W(\phi_T,\eps(u))\Big\} \dx.$$

\subsection{Chemotherapeutic influence} \label{Sec:Mod:Chemo}
In addition to precisely simulating the tumor's growth, mathematicians are interested in treating the tumor and stopping its growth.
Currently, chemotherapy, surgery, immunotherapy, and radiotherapy are used to treat malignancies.
Angiogenesis is one of the primary mechanisms by which tumors grow, hence anti-angiogenic drugs that inhibit the production of new vascular structures are commonly identified as one of the methods to delay or stop cancer growth.
Consequently, a realistic model of angiogenesis is essential for evaluating the efficiency of anti-angiogenic drugs; for the ideal dosage of medication, see the optimal control problems discussed in \cite{colli2020mathematical,colli2021optimal}.
Chemotherapy was incorporated into our research \citep{fritz2021subdiffusive} with a reaction-diffusion equation and subdiffusive tumor growth, as well as in the articles \citep{ebenbeck2019optimal,signori2019penalisation,garcke2018optimal,colli2020mathematical,colli2021optimal} on optimum control issues for the optimal drug dosage. Moreover, in the work \cite{wagner2023phase} it was assumed that the immunotherapeutic concentration follows the Hill–Langmuir equation. 

In addition to studying the growth of tumors, we also incorporate a substance that inhibits their spread.
Current cancer treatments include: \medskip 
\begin{itemize}  \itemsep0.3em
	\item Surgery: Removing the tumor by an operation. 
	\item Immunotherapy: Strengthening the immune system.
	\item Radiotherapy: Employing radiation to eradicate cancerous cells. 
	\item Chemotherapy: Utilizing medications to destroy the tumor. \medskip
\end{itemize} 
These therapies, excepting surgery, are administered in cycles, with each cycle consisting of a period of therapy followed by a period of rest to allow the patient's body to repair and regenerate new, healthy cells.
These therapeutic procedures should diminish the tumor to a degree where surgical removal is feasible. 

The mass density of chemotherapy $\cmt$ is considered to be driven by a reaction-diffusion equation that links to the tumor equation and, if chemotherapy is present, degrades the tumor.
Therefore, we recommend adding the index $\CMT$ to the index set $\RD$ and propose the model:
$$\label{Eq:Chemo} \boxed{
		\begin{gathered}
	\textbf{Four-species model with chemotherapy} \\
	\begin{aligned} 
	\pt \phi_T+ \div(\phi_T v) ={}& \div (M_T \phi_T^2(1-\phi_T)^2 \nabla \mu_T) + S_T( \phi_T,\phi_\sigma,\cmt)    \\
	\mu_T ={}& \Psi'(\phi_T) - \chi_c \phi_\sigma- \eps_T^2 \Delta \phi_T \\
	\pt \phi_\sigma +  \div(\phi_\sigma v)                     ={}& \div\Big(M_\sigma \nabla\big( D_{\sigma} \phi_\sigma -\chi_c\phi_T  \big) \Big) +S_{\sigma}( \phi_T,\phi_\sigma,\cmt) \\
	\pt \cmt ={}& M_\CMT D_\CMT  \Delta \cmt + S_\CMT(\phi_T,\phi_\sigma,\cmt).
	\end{aligned}\end{gathered} } $$
The mobility of chemotherapeutic agents  is given by $M_\CMT$ and the source $S_\CMT$ reads
$$S_\CMT(\phi_T,\phi_\sigma,\cmt)= -\lambda_\CMT^\deg \cmt - \lambda_\CMT^\kill \frac{\phi_T (1-\phi_T) \cmt}{K_\CMT+\cmt},$$
where $\lambda_\CMT^\deg$ is the degradation factor of chemotherapeutic agents and $\lambda_\CMT^\kill$ represents the rate at which chemotherapeutic agents act and are subsequently blocked by the death of tumor cells. 
The killing term includes a saturation effect, so that chemotherapy is most effective against cells in a certain growth phase.
The parameter $K_\CMT>0$ is the density of chemotherapeutic drugs at half-maximum concentration.
Similarly, the source term of the tumor volume fraction will contain a term of the kind $$-\lambda_T^\kill \frac{\phi_T (1-\phi_T) \cmt}{K_\CMT+\cmt},$$ that represents the chemotherapy's killing impact at some rate $\lambda_T^\kill$.
In our approach in \cite{fritz2021subdiffusive}, chemotherapeutic agents in cycles are represented by a Dirichlet border of the type X. \\[-1cm]

$$\cmt(t,x)\vert_{x \in \p\Omega}=\begin{cases} 1, &\text{for $t\leq 2$ or $6<t \leq 8$ or $12<t\leq 14$,} \\ 0, &\text{else}.\end{cases}$$
That is, during the time $t \in [0,2] \cup (6,8] \cup (12,14]$ chemotherapy treatment is provided and in between, the body is permitted to rest.

\subsection{Angiogenesis and mixed-dimensional coupling} \label{Mod:Mixed}
Hypoxic tumor cells not only release MDEs to degrade the ECM, but also TAFs, which stimulate endothelial cell proliferation and new vessel formation.
Angiogenesis is the process of blood vessels sprouting and elongating in order to supply the tumor with nutrients.
Unless sufficient nutrients and oxygen are available for proliferation, the volume of an isolated colony of tumor cells is often limited to 1\si{mm}$^3$, as shown in \cite{nishida2006angiogenesis}, unless adequate nutrients and oxygen are supplied.
In order to access these nutrients, cancerous cells drive angiogenesis \citep{carmeliet2011molecular, patsch2015generation}.
Regarding angiogenesis modeling and numerical simulations, we refer to \cite{cristini2009nonlinear,cristini2010multiscale,xu2016mathematical}.
In \cite{fritz2021analysis}, we studied angiogenesis in terms of the mathematical analysis of weak solutions in a Cahn--Hilliard-type model.
We are unaware of any subsequent works.
Due to mixed-dimensional couplings and the presence of hypoxic tumor cells that generate TAFs, these models are extremely complex. 

\cite{lima2014hybrid, xu2016mathematical, xu2017full, xu2020phase, wise2008three, santagiuliana2016simulation, santagiuliana2019coupling} presents the effect of angiogenesis on models of stratified tumor development.
In contrast to their prior techniques employing, for example, agent-based systems, we represent the network of blood arteries feeding a solid tumor mass as a network of 1D capillaries within a 3D tissue domain in our studies in \cite{fritz2021analysis,fritz2021modeling}.
In this perspective, tumor growth is viewed as a phase-field system with multiple cell species and other components.
The microvascular network in tumor-bearing tissue is modeled as a graph with 1D filaments through which nutrient-rich blood can flow.
This microvascular network is represented by $\Lambda$ and the individual edges by $\Lambda_i$, such that $\Lambda$ is given by the union $
\Lambda = \bigcup_{i=1}^N \Lambda_i.
$
An edge $\Lambda_i$ is parameterized with a curve parameter $s_i$ as follows:
$$
\Lambda_i = \left\{ x \in \Omega : x = \Lambda_i ( s_i ) = x_{i,1} + s_i \cdot ( x_{i,2}-x_{i,1} ),\;s_i \in (0,1 ) \right\}.
$$


We suggest $s$ as the global curve parameter for the entire 1D network $\Lambda$ by setting $s=s_i$ if $x = \Lambda(s) = \Lambda_i(s_i)$.
We search the domain $\Omega$ for 1D items that couple to their 3D counterparts for each value of the curve parameter $s$.
We suppose that the surface of a single vessel is a cylinder with a constant radius, and that the radius of a vessel attached to the edge $\Lambda_i$ is $R_i$.
We describe $\Gamma_i$ as the surface of the cylinder, with the edge $\Lambda_i$ as its center line, and the total surface $\Gamma$ is the union of the surfaces of the individual vessels $\Gamma_i$, see as well \cref{Fig:3d1d} for a depiction of the individual fields.


\begin{figure}[htb!]
    \centering
    \includegraphics[width=.24\textwidth]{Images/Domain1.png}
    \includegraphics[width=.24\textwidth]{Images/Domain2.png}
    \includegraphics[width=.24\textwidth]{Images/Domain3.png}
    \includegraphics[width=.24\textwidth]{Images/Domain4.png}
    \caption{Discretization of the vessels into $N$-many vessels and introduction of 1D lines $\Lambda_i$; further, the vessel surfaces $\Gamma_i$ are depicted; taken with permission from Figure 2 in \cite{fritz2021analysis}.}
    \label{Fig:3d1d}
\end{figure}

On the 1D network $\Lambda$, we take the constituents $\phi_v$, $v_v$ and $p_v$ into account, which reflect the 1D equivalents of the local nutrient concentration $\phi_\sigma$, the volume-averaged velocity $v$, and the pressure $p$.
We incorporate a new source term $S_{\sigma v}$ for coupling the 1D constituents $\phi_v$ and $p_v$ into the $\phi_\sigma$-equation.
Consequently, this source word is accountable for the relationship between the elements in $\Omega$ and $\Lambda$. 

To quantify the flux of nutrients across the vessel surface, we employ the Kedem--Katchalsky law \citep{ginzburg1963frictional} and write the flux $J_{\sigma v}$ between the nutrients on the network and tissue as \begin{equation} \label{Eq:Source3d1d}
J_{\sigma v} (\ov \phi_\sigma, \ov p, \phi_v, p_v ) = ( 1-r_{\sigma} ) f(\phi_{\sigma},\phi_v) L_p ( p_v - \overline{p} )  + L_\sigma ( \phi_v - \ov \phi_\sigma ),
\end{equation} 
where $r_\sigma>0$ is the reflection parameter, $L_\sigma,L_p>0$ represent the permeabilities of the vessel wall, and the function $f$ is either $\phi_\sigma$ or $\phi_v$ depending on the values of $p$ and $p_v$.
In addition, $\overline{p}$ represents the average circumferential pressure of cylinder cross-sections.
The averaging reflects the fact that the 3D-1D coupling is a reduced model from a physical standpoint, whereas the exchange occurs through the surface in a fully linked 3D-3D model.
The first portion of the Kedem--Katchalsky law measures the nutritional flux caused by the passage of blood plasma from arteries to tissues or vice versa.
It is defined by Starling's law, which is given by the pressure difference between $p_v$ and $p$ multiplied by a parameter $L_p$ representing the permeability of the vessel wall.
The second component of the law is a Fickian-type law that accounts for the tendency of nutrient concentrations to equalize. 

As the exchange activities between the vascular network and the tissue occur at the vessel surface $\Gamma$, we concentrate the flux $J_{\sigma v}$ using the Dirac measure $\delta_\Gamma$, i.e., by defining 
$$\langle \delta_\Gamma, \varphi \rangle_{C_c^\infty(\Omega)} = \int_\Gamma \varphi\vert_{\Gamma}(x) \, \dd S \quad \forall \varphi \in C_c^\infty(\Omega),$$
where $(C_c^\infty(\Omega))'$ is the space of distributions.
The resulting new source term in the nutrient equation is as follows: 
$$
S_{\sigma v}(\phi_\sigma,p,\phi_v,p_v) = J_{\sigma v} ( \phi_\sigma, p, \Pi_\Gamma \phi_v, \Pi_\Gamma p_v) \delta_\Gamma,
$$
where $\Pi_\Gamma \in \scrL(L^2(\Lambda);L^2(\Gamma))$ is the projection of the 1D quantities onto the cylindrical surface $\Gamma$ by extending the function value $\Pi_\Gamma \phi_v(s) = \phi_v(s_i)$ for all $s \in \p B_{R_i}(s_i)$. 

The 3D model reads:
$$  \boxed{\begin{gathered} 
\textbf{Angiogenesis model: 3D} \\
\begin{aligned}
\pt \phi_\alpha+ \div(\phi_\alpha v)            ={}& \div \big(m_\alpha( \phi_\A) \nabla \mu_\alpha\big)+ S_\alpha( \phi_\A)                                                      \\
\mu_\alpha                                 ={}&   \p_{\phi_\alpha} \Psi(\phi_\CH) - \eps^2_\alpha \Delta \phi_\alpha - \chi_c \phi_\sigma-\chi_h \ecm                                                      \\
\pt \phi_\beta                            ={}& S_\beta( \phi_\A)                                                                                                                                   \\
\pt \phi_\gamma    ={}& \div \big(m_{\gamma}( \phi_\A) D_{\gamma} \nabla \phi_\gamma\big)+S_{\gamma}( \phi_\A) \\
\pt \phi_\sigma + \div(\phi_\sigma v) ={}& \div \big(m_\sigma( \phi_\A) \nabla (D_\sigma  \phi_\sigma - \chi_c (\phi_P +\phi_H  )\big)+S_\sigma( \phi_\A)  \\ &+ J_{\sigma v} (\tr_\Gamma \phi_\sigma, \tr_\Gamma p, \Pi_\Gamma \phi_v,  \Pi_\Gamma p_v ) \delta_\Gamma \\
v  ={}& - K\big(\nabla p -S_p(\phi_\A,\mu_P,\mu_H)\big)  \\
\div\, v ={}&    L_p (\Pi_\Gamma p_v - p) \delta_\Gamma%
\end{aligned}\end{gathered}} $$
for $\alpha \in \{P,H\}$, $\beta \in \{N,\ECM\}$, $\gamma \in \{\MDE,\TAF\}$.

Since the vascular network is often composed of small inclusions, we averaged all the physical units across the cross-sections of the individual blood vessels and held them constant with regard to the angular and radial components.
In other words, the 1D variables $\phi_v$ and $p_v$ of a 1D vessel $\Lambda_i$ are entirely dependent on $s_i$.
\cite{koppl20203d} contains further information regarding the derivation of 1D pipe flow and transport models.
Consequently, the 1D model equations for vessel flow and transport are as follows: 
$$  \label{Eq:Model1D}
\boxed{\begin{gathered} 
	\textbf{Angiogenesis model: 1D} \\\begin{aligned}
\pt \phi_v + \p_{s_i} (v_v \phi_v) ={}&  \p_{s_i} (m_v(\phi_v)D_v \p_{s_i} \phi_v) -2\pi R_i J_{\sigma v} (\ov \phi_\sigma, \ov p, \phi_v,  p_v)\\
-  \;\p_{s_i} ( R_i^2 \pi K_{v,i} \; \p_{s_i} p_v )   ={}& -2 \pi R_i J_{pv}( \overline{p}, p_v ) \\
v_v ={}& - R_i^2 \pi K_{v,i} \p_{s_i} p_v
\end{aligned}\end{gathered}}
$$
In order to interconnect the multiple solutions on the vessel $\Lambda_i$ at inner network nodes at junctions $x \in \p \Lambda_i \setminus \p \Lambda$, we require the continuity of pressure and concentration as well as the conservation of mass, as shown in \cite{fritz2021modeling}.

We present a numerical simulation of the tumor evolution in the setting of a capillary network in \cref{Fig:Angiogenesis}. We can observe that the tumor is deprived of nutrients and therefore, it stratifies, and it becomes hypoxic. The hypoxic tumor phase releases TAFs, and it is clear to see that angiogenesis happens -- the capillary begins to grow towards the tumor and provides it with new nutrients living in the 1D vessel network.

\begin{figure}[htb!]
    \centering
    \includegraphics[width=.27\textwidth]{Images/complex_4.png} \!\!\!\!\!\!\!\!
    \includegraphics[width=.27\textwidth]{Images/complex_22.png}\!\!\!\!\!\!\!\!
    \includegraphics[width=.27\textwidth]{Images/complex_100.png}\!\!\!\!\!\!\!\!
    \includegraphics[width=.27\textwidth]{Images/complex_150.png} \\
    \includegraphics[width=.4\textwidth]{Images/color1.pdf}
    \caption{Stratification of tumor cells in its proliferative (green), hypoxic (red) and necrotic (black) phases over the time steps $t \in \{0,5,10,15\}$; growth of capillaries and movement of tumor cells to high-nutrient regions on the 1D lines that is expressed by the nutrients $\phi_v$.}
    \label{Fig:Angiogenesis}
\end{figure}


\section{Experiments}\label{sec:exp}

\iffalse
To summarize our experiments:

\begin{itemize}
    \item Data-triggered joins lead to faster reaction time and queueing time in real-time decision-making than time-triggered joins (\S\ref{sec:exp-joins}, \S\ref{sec:exp-nuscenes}).
    \item Lazy data routing significantly reduces communication time when data payloads are large (\S\ref{sec:exp-lazy-tradeoff}). It naturally supports parallelism and maintains its performance even in the presence of network contention (\S\ref{sec:exp-lazy-parallelism}, \S\ref{sec:exp-congestion}). Lazy data routing also benefits from reduced latency when certain data is skipped, as the rate of data arrival exceeds model throughput (\S\ref{sec:exp-lazy-data-skipping}).
    \item In real-world settings, decentralized model placement with effective data skipping gives us higher real-time accuracy and lower backlog latency (\S\ref{sec:exp-opportunity}).
    % \item Effective downsampling ensures the timeliness of incoming data, and thus significantly improves real-time accuracy while preserving near-zero backlog. (\S\ref{sec:exp-opportunity})
    \item \sys's lightweight queuing system offers more flexible data routing, supports more network topologies, and has much lower system overhead than Ray Serve (\S\ref{sec:exp-system-overhead}, \S\ref{sec:exp-network}).
\end{itemize}
\fi


\subsection{Experimental Setup}\label{sec:exp-setup}
All of our experiments are performed on a private ``edge cluster''.
Our hardware setup consists of 5 NVIDIA Jetson Nano Developer Kits, 4 Intel Skylake NUC computers, and a desktop PC. Each NUC is equipped with an Intel Core i3-6100U CPU, 16 GB RAM, and M.2 SSD.
The desktop PC features an Intel Xeon CPU E5-2603 v4 CPU, NVIDIA Quadro P6000 GPU, 64 GB RAM, and HDD.
Direct peer-to-peer connection is available between all nodes via 1Gbps Ethernet.
Throughout these experiments, we vary the network topology to test various scenarios.
In some experiments, only partial nodes are used.
These variations will be explained in respective sections, but one NUC is always used as the leader node.
% Generally, NUCs are used as the leader node and 3 data source nodes; Jetson Nanos are used as 1 other data source node and all prediction nodes.


As a primary baseline, we have configured PyTorch distributed~\cite{pytorch-distributed} on our edge cluster, with Gloo as the distributed communication backend.
% We run PyTorch in both a centralized mode (traditional model serving) as well as a decentralized mode when possible.
We have also implemented an eager data routing architecture similar to ROS~\cite{quigley2009ros} within our framework to understand the key design decisions.
ROS is widely used in the sensor and robotics communities and provides a centralized message broker service.
However, ROS does not support lazy data routing, distributed stream synchronization, and adaptive rate control.
Additionally, we implemented a time-triggered join strategy similar to Apache Flink~\cite{flink}.
% We excluded Tensorflow from this evaluation because we found that TensorFlow distributed did not offer fine-grained control over communication needed for fair experiments.
We also set up a local NTP server to make sure all nodes share a global wall clock time.

We borrow evaluation metrics from the streaming literature and a detailed description of these metrics is in Appendix~\ref{sec:exp-metrics}.


\subsection{Application: Human Activity Recognition}\label{sec:exp-opportunity}
\noindent \textbf{Description. } We use the Opportunity dataset for human activity recognition~\cite{opportunity-dataset,opportunity-challenge} as an example.
Data from multiple motion sensors were collected about every 33ms while users were executing typical daily activities. For each subject, there are five activity of daily living (ADL) runs, and each run lasts 15-30 minutes.
We take the first subject's first four ADL runs as the training set and the last ADL run as the test set. When played at 2x speed, the last ADL run takes 8 minutes and 22 seconds.
We partition the first 134 columns vertically into four disjoint subsets, each placed on one of four nodes (3 NUCs and 1 Jetson Nano) as data sources.
The subsets are distributed as follows: columns 1-37 (accelerometers), 38-76 (IMU back and right arm), 77-102 (IMU left arm), and 103-134 (IMU shoes).
We train an aggregated random forest model with scikit-learn~\cite{scikit-learn} for all 134 features as an early fusion baseline, and also four separate random forest models for each subset of features to evaluate an ensemble-based late fusion method.
We primarily evaluate \sys with the late fusion deployment: one RF model at each data source node and we ensemble local predictions at another node.
This simulates a scenario where there is a small amount of compute on each wearable sensor and that compute is used to reduce the data communicated to make a global prediction across those sensors.
Due to space limits, we defer the early vs. late fusion discussion to Appendix~\ref{sec:appendix-late-fusion}.

In our best-effort PyTorch implementation, we use the \texttt{gather()} API to aggregate data from multiple data source nodes.
PyTorch distributed requires all tensors to be the same size to be gathered, so we have to pad each local tensor to the maximum size with zeros.
% Since there is no message queue in PyTorch, it is not possible to compare the parallel (topology 2) strategy between \sys and PyTorch. However, we can still compare centralized and decentralized strategies across these two frameworks. 
The individual streams are fast enough that misalignments can occur due to queueing delays.
However, PyTorch enforces that multiple data sources are always perfectly synchronized, as it does not begin the actual computation until data from all data sources has been gathered, and it only gathers new data after finishing previous predictions.
Such strict requirement does not exist in \sys, as we are able to set a reasonable skew (\S\ref{sec:message-skew}) in \sys.



\vspace{0.25em} \noindent \textbf{Queueing in \sys is better suited for real-time applications. }
\begin{figure}[t]
    \centering
    \begin{minipage}{0.48\columnwidth}
        \centering
        \includegraphics[width=\textwidth]{figures/exp-opportunity-latency-only-decentralized.pdf}
        \caption{Measure of backlog in the activity recognition task. More frequent predictions are on the left side.}
        \label{fig:exp-opportunity-latency}
    \end{minipage}\hfill
    \begin{minipage}{0.48\columnwidth}
        \centering
        \includegraphics[width=\textwidth]{figures/exp-opportunity-stdev.pdf}
        \caption{Standard deviation of actual prediction frequency, where \sys maintains a lower variability.}
        \label{fig:exp-opportunity-stdev}
    \end{minipage}
\end{figure}
First, we evaluate the ability of the system to even issue real-time predictions by measuring the \textit{backlog} in the system, or the accumulated queuing time as defined in Appendix~\ref{sec:exp-metrics-backlog}.
Unlike PyTorch, \sys deployments have a prediction frequency target and can use this target to automatically downsample data to meet real-time requirements.
We illustrate the improvements in Figure~\ref{fig:exp-opportunity-latency}. The x-axis is the target prediction frequency (\S\ref{sec:target-pred-freq}) designated by the end-user, where a larger number means a lower frequency; the y-axis is the \textit{backlog} for each of the serving systems over this dataset.
The compute part of the task itself takes about 23ms to complete, and a near-zero number in backlog means the inference is processed in real time.
% When target prediction frequency $\geq 27$ms per prediction, \sys has near-zero backlog.
\sys offers a no-backlog queue for a wider range of prediction frequency targets ($\geq 27$ms/pred).
However, a long queue of unprocessed examples is quickly developed without proper rate control (e.g. when target prediction frequency $\leq 26$ms per prediction).
Since PyTorch lacks a message queue and rate control, it has to process each example individually and trigger joins in a strictly synchronous manner, leading to an unsatisfactory backlog.

\begin{figure}[t]
    \centering
    \begin{minipage}{0.48\columnwidth}
        \centering
        \includegraphics[width=\textwidth]{figures/exp-opportunity-accuracy-only-decentralized.pdf}
        \caption{Overall real-time accuracy for human activity recognition task measured in F-1 score. }
        \label{fig:exp-opportunity-accuracy}
    \end{minipage}\hfill
    \begin{minipage}{0.48\columnwidth}
        \centering
        \includegraphics[width=\textwidth]{figures/nuscenes-e2e-latency-cdf.pdf}
        \caption{CDF of end-to-end latency for eager data routing vs. lazy data routing.}
        \label{fig:exp-nuscenes-e2e-latency}
    \end{minipage}
\end{figure}

Even if PyTorch could meet real-time prediction targets, we find that the variability in prediction latencies is quite high.
In Figure~\ref{fig:exp-opportunity-stdev}, we see a much higher variability in actual prediction frequencies for PyTorch than \sys across all user-defined rates.
This is because PyTorch communicates in a synchronous fashion, and has to account for the variability of all 4 nodes making local predictions with local data streams.

\vspace{0.25em} \noindent \textbf{Queueing Delays Reduce ``Real-Time'' Accuracy. }
In real-time serving scenarios, the timeliness of predictions becomes a key concern. For latency-sensitive tasks, a delayed prediction equates to an incorrect one. To evaluate the timeliness of predictions, we introduce \textit{real-time accuracy} as a measure, which evaluates the accuracy of predictions against the most recent label at the time of prediction. For instance, if a prediction is made between two consecutive labels at times $t_1$ and $t_2$ ($t_1 \leq t_2$), its accuracy is compared with the label at $t_1$.
Since we assume adjacent examples are likely similar, we expect roughly correct prediction results when the examples arrive slightly late. However, if the examples arrive significantly late, they are likely outdated and yield incorrect predictions.

% define \textit{real-time accuracy} to be the accuracy of predictions compared with the last known label at prediction time.
% For example, suppose a prediction is made temporally between two consecutive labels with timestamps $t_1$ and $t_2$ ($t_1 \leq t_2$). In that case, we compare the prediction result to the ground truth at $t_1$ to calculate the real-time accuracy.


Figure~\ref{fig:exp-opportunity-accuracy} shows the real-time accuracy of \sys and PyTorch under various target prediction frequencies.
PyTorch distributed is not able to issue accurate predictions because data is communicated in a synchronous manner. It is unable to downsample the input stream even if the node is overloaded, making most of its predictions outdated.
In contrast, \sys, at a target prediction frequency of 25ms, experiences a greater backlog compared to PyTorch but achieves superior real-time accuracy. This advantage is primarily due to the experiment setup of 3 NUCs and 1 Jetson Nano for local model inference. The NUCs process the CPU model more efficiently than the Jetson Nano, resulting in a significant portion of the backlog being attributed to the Jetson Nano, as it completes local inference later than the NUCs. This situation leads to a notable message skew. To mitigate this, \sys selectively skips data that exceeds the maximum tolerable skew (\S\ref{sec:message-skew}). This strategy of skipping mostly inaccurate data significantly boosts \sys's real-time accuracy.
Furthermore, when the target prediction frequency is set above 26ms/pred, \sys sees less backlog and achieves even higher real-time accuracy.
This improvement results from \sys's capability to instantly process fresher data that, while not perfectly synchronized, falls within an acceptable time skew.


\subsection{Application: Autonomous Driving}\label{sec:exp-nuscenes}
We use a subset of the nuScenes self-driving dataset for autonomous driving~\cite{nuscenes} consisting of 6 cameras and a lidar sensor.
All cameras generate 10 frames per second and the lidar sensor emits at 2 Hz.
Each camera is connected to a separate NVIDIA Jetson Nano running pre-trained YOLOv5n model~\cite{yolov5} on GPU.
The lidar sensor is connected to a NUC node, which preprocesses the data and then transfers it to our desktop PC equipped with NVIDIA Quadro P6000 GPU running pre-trained CenterPoint model~\cite{yin2021center}.
Communication incurs considerable cost here as preprocessed lidar data is very large.
All predictions are sent to another NUC node, which triggers the join and yields synchronized predictions.
% There is also a separate NUC node acting as the message broker.




\iffalse
% Notes
A naive, centralized solution would be to aggregate all data to a single node and run a large model with all sources of data.
(1) model too large to deploy
(2) still have to time-align between sources
(3) single point of failure

Queueing time (at each node)
End-to-end time
Data-triggered/Time-triggered, Lazy/Eager
https://docs.google.com/spreadsheets/d/1NFnzf8NBrQ-yGY9WqbCMc7UKzbOlh1JVOLIg2MwEJwU/edit#gid=1795317306

Eager vs. Lazy (both data-triggered), end-to-end latency
Eager: Average 37808.74183ms, Median 37691.8551ms, p95 70834.40801ms
Lazy: Average 2102.902788ms, Median 2231.150146ms, p95 2481.405029ms

Time-triggered vs. Data-triggered (both lazy), queueing time

\begin{table}[]
\small
\begin{tabular}{ll|lll}

\end{tabular}
\caption{End-to-end latency for eager vs. lazy data routing.}
\label{tab:exp-nuscenes-e2e-latency}
\end{table}
\fi


\begin{table}[]
\small
\begin{tabular}{ll|lll}
\multicolumn{2}{l|}{\makecell{Queueing time\\ (ms)}} & \makecell{Time-triggered\\ (Flink-like)} & \makecell{Data-triggered\\ (Ours)} & Speedup \\ \hline
\multirow{2}{*}{Lidar}       & Med.    & 463.60         & 16.56          & 28.00x  \\
                             & P95       & 963.20         & 48.35          & 19.92x  \\ \hline
\multirow{2}{*}{Cam 1}    & Med.    & 51.04          & 5.64           & 9.06x   \\
                             & P95       & 126.06         & 13.56          & 9.30x   \\ \hline
\multirow{2}{*}{Cam 2}    & Med.    & 64.05          & 9.59           & 6.68x   \\
                             & P95       & 124.08         & 18.27          & 6.79x   \\ \hline
\multirow{2}{*}{Cam 3}    & Med.    & 56.62          & 9.76           & 5.80x   \\
                             & P95       & 143.45         & 17.84          & 8.04x   \\ \hline
\multirow{2}{*}{Cam 4}    & Med.    & 74.30          & 5.23           & 14.19x  \\
                             & P95       & 117.08         & 13.08          & 8.95x   \\ \hline
\multirow{2}{*}{Cam 5}    & Med.    & 45.32          & 7.63           & 5.94x   \\
                             & P95       & 88.87          & 15.90          & 5.59x   \\ \hline
\multirow{2}{*}{Cam 6}    & Med.    & 57.33          & 5.38           & 10.66x  \\
                             & P95       & 109.61         & 13.20          & 8.31x   \\
\end{tabular}
\caption{Queueing time for time- vs. data-triggered joins.}
\label{tab:exp-nuscenes-joins}
\end{table}

First, we compare the end-to-end latency between eager and lazy data routing, applying data-triggered join in both scenarios.
In the lazy data routing approach, we implemented a freshness threshold SLO (\S\ref{sec:freshness-threshold}) that discards data older than 500ms.
As depicted in Figure~\ref{fig:exp-nuscenes-e2e-latency}, the CDF of end-to-end latency demonstrates that lazy data routing significantly reduces latency by only pulling data with recent timestamps.
This reduction is particularly notable since communication is the primary bottleneck in this task.
Notably, lazy data routing led to the skipping of 72.5\% of predictions that failed to meet our freshness threshold SLO compared to eager data routing.


Second, we compare the queueing time (as defined in~\ref{sec:exp-metrics}) between time-triggered and data-triggered joins, applying lazy data routing in both scenarios.
For time-triggered join, we set the time interval of joins to be every 1 second.
For data-triggered join, we issue a join as soon as a new example that meets our freshness threshold SLO comes in.
Table~\ref{tab:exp-nuscenes-joins} shows the median and 95th percentile of queueing time for each data source.
Data-triggered join reduces the queueing time by up to 28x as it does not have to wait for fixed intervals.



\subsection{Application: Network Intrusion Detection}\label{sec:exp-network}
\sys natively allows multiple producers and multiple consumers to operate on the shared message queue at the same time, which is an essential communication paradigm in decentralized prediction but not currently supported by PyTorch or TensorFlow.
We use a public Network Intrusion Detection dataset from Canadian Institute for Cybersecurity (CIC-IDS2017)~\cite{cic-ids2017} and an existing model~\cite{kostas2018} to differentiate malicious traffic from benign network traffic.
Specifically, we partition the data horizontally into four disjoint subsets by ``Source IP'' for our four data source nodes. The underlying assumption is that network traffic from different source IP addresses may be collected separately.

If a web attack is detected, the related network packet needs to be sent to a specific destination node, but the actual computation can be done anywhere. 
We show that \sys can support three deployment strategies: (Early fusion, topology 1) transfer all data to the prediction node that does all computations in a centralized way; (Early fusion with parallelism, topology 2) transfer all data from data source nodes to an intermediate shared queue, where four prediction nodes can pull data from when they become available, and they need to inform the destination node if an attack is detected; or (Late fusion, topology 3) data source nodes do computations locally and only transfer data to the destination node if an attack is detected.  

\iffalse
\begin{table}[]
\centering
\begin{tabular}{l|l|l|l}
Method   & Nodes & \begin{tabular}[c]{@{}l@{}}PyTorch\\ samples/sec\end{tabular} & \begin{tabular}[c]{@{}l@{}}\sys\\ samples/sec\end{tabular} \\ \hline
a) Centralized           & 4+1+1  & 41.94    & 47.58 (+13.4\%)      \\
b) Parallel              & 4+1+4  & - & 182.57 (3.84x)      \\
c) Decentralized         & 4+1+0 & 181.33 (4.32x)   & 197.30 (+8.8\%, 4.15x)
\end{tabular}
\caption{Throughput for network intrusion detection task. (a)-(c) involve the same task with different model placement methods. The nodes column shows the number of nodes involved, with $x+y+z$ denoting $x$ number of data source nodes, $y$ number of leader nodes, and $z$ number of prediction nodes. Both parallel (b) and decentralized (c) methods utilize four prediction nodes, whereas centralized (a) utilizes only one. Percentages in parentheses are relative to PyTorch distributed; speedups in parentheses are relative to centralized (a).\todo{this experiment is really confusing because we partition the dataset horizontally, with no aggregation required. In the later experiment we partition the dataset vertically, so aggregation is required.}}
\label{tab:exp-network}
\end{table}
\fi

In an early fusion setting, PyTorch distributed is able to process 41.94 examples per second, while \sys can process 47.58 examples per second.
This is the baseline setting of both systems, and the performances of both systems are comparable.
In an early fusion with parallelism setting that is only supported by \sys, thanks to its queuing design, 182.57 examples are processed per second, which is almost a linear (3.84x) speedup compared to a centralized setting given that we now have 4 prediction nodes.
In a late fusion setting, we make all 4 data source nodes also local prediction nodes, and PyTorch achieves 181.33 examples per second while \sys takes 197.30 examples per second. For both systems, superlinear speedup (4.32x and 4.15x compared to centralized, respectively) is achieved by making the most of local computational resources and communicating only local prediction results instead of the entire dataset.
Since we use the same model/sub-models for both \sys and PyTorch without synchronization issues, the accuracies of predictions between both systems are the same.


\subsection{Micro-benchmarks}


\subsubsection{Data-triggered Joins Are More Responsive}\label{sec:exp-joins}
\iffalse
Ted's notes:
Join experiments: time window / data triggered
Bursty data. Suddenly a lot of data coming in and we need immediate reactions. 
Metric: average/min/max/95 percentile reaction time, communication in bytes
Baseline: fixed time window.
1. Transfer the latest raw data at the end of each time window. 
2. Only transfer if there’s an update (not the same as previous) at the end of each time window. Need state management at local node.

2 streams. One constant every 5 seconds. The other either none or very frequent (10Hz)
Also with lazy data routing, we don't have to resend duplicated lower-frequency stream.

Low frequency stream A: 5MB each file, arrives every 5 seconds. Lazy data routing.
High frequency stream B: 1Byte each file, arrives at 10 Hz.
% But very bursty. Only arrives during the first 5 seconds of the minute.
Ours: Data-triggered join with LDR.
Baseline 1: Data-triggered join without LDR.
Baseline 2: Time-triggered join with LDR. (time window == 1s, only transfer the raw data when an update is detected).
Baseline 3: Time-triggered join without LDR (time window == 1s, transfer the raw data regardless if it's the same as previous data).
Metrics: median/min/max/95 percentile reaction time / end-to-end latency; total communication bytes.
\fi
\begin{table}[t]
\begin{tabular}{l| l|l}
\multicolumn{1}{c|}{\multirow{2}{*}{Join strategy}}
    & \multicolumn{2}{c}{Reaction time} \\
    & Median &P95\\ \hline
Data-triggered & 9.02ms & 10.28ms\\
Time-triggered (time window: 1s) & 0.5s & 0.9s\\
Time-triggered (time window: 5s) & 2.5s & 4.7s\\
\end{tabular}
\caption{Reaction time for data- and time-triggered joins.}
\label{tab:exp-reaction-time}
\end{table}
This micro-benchmark evaluates the responsiveness between time-triggered join and data-triggered join.
We use two NUC nodes as data sources, one NUC node as the message broker, and one more NUC node performing the join.
We show how data-triggered joins significantly reduce reaction time for latency-sensitive applications.
We have a steady stream that arrives every 5 seconds (5 MB each) and a bursty stream that arrives at 10 Hz (1 Byte each) for a minute.
Real-time decision-making requires the latest information from both streams.
% We measure the time between the latest data generation and joint data arrival at another node as \textit{reaction time}.
The \textit{reaction time} (as defined in Appendix \ref{sec:exp-metrics-latency}) for both join strategies during that one minute is shown in Table~\ref{tab:exp-reaction-time}.
Data-triggered join achieves a much better reaction time without the difficulty of setting a reasonable time window.

\subsubsection{Benefits of Lazy Data Routing}\label{sec:exp-lazy}
In \S\ref{sec:lazy}, we described our lazy data routing model as an alternative to a ROS-like system that eagerly transfers raw data through a centralized broker.
Now, we evaluate the pros and cons of the lazy data routing model.
We employ one NUC node as the data source and one NUC node as the receiver in this subsection, except for the parallelism experiment where the number of receiver nodes is varied.

% \subsubsection{Evaluation Metrics}
% In this subsection, we use the following three metrics to measure the performance of data routing:

\begin{figure}[t]
    \centering
    \includegraphics[width=0.9\columnwidth]{figures/exp-lazy-data-routing/comm_latency_breakdown.pdf}
    \caption{Lazy data routing reduces latency on producer side but has fixed overhead on consumer side. Both axes are log-scaled.}
    \label{fig:exp-comm-latency-breakdown}
\end{figure}

\begin{figure}[t]
    \centering
    \includegraphics[width=0.9\columnwidth]{figures/exp-lazy-data-routing/comm_latency_eager_lazy.pdf}
    \caption{Lazy data routing reduces latency on producer side but has fixed overhead on consumer side. Both axes are log-scaled.}
    \label{fig:exp-comm-latency-eager-lazy}
\end{figure}

First, we send a series of messages of different sizes from a data source node to a receiver node, through the leader node. No actual computation is performed. We compare the communication latency between eager and lazy data routing.

\vspace{0.25em} \noindent \textbf{Lazy Data Routing Reduces Latency on Producer but Has Fixed Overhead on Consumer.}\label{sec:exp-lazy-tradeoff}
Since we only need to transfer the headers instead of raw data in our lazy data routing model, the latency on producer side remains a negligible number even if the message is huge.
As shown in Figure~\ref{fig:exp-comm-latency-breakdown}a and~\ref{fig:exp-comm-latency-eager-lazy}a, a ROS-like eager data routing model could result in very high latency when sending large messages, which itself could force subsequent messages to queue up and become outdated when they arrive.
In contrast, our lazy data routing model makes sure that message headers are sent in milliseconds, which never blocks the rest of messages. The consumers may choose to downsample some data and only fetch necessary data.

% \subsubsection{Lazy Data Routing Has Fixed Overhead on Consumer Side}\label{sec:exp-lazy-consumer}
Whenever the consumer needs to fetch raw data, there is a fixed overhead to establish P2P connections even if the actual data is just a few bytes. This fixed overhead can be amortized when the actual data is larger, as depicted in Figure~\ref{fig:exp-comm-latency-breakdown}b and~\ref{fig:exp-comm-latency-eager-lazy}b.

% \subsubsection{Break-Even Point Between Lazy and Eager Data Routing}
In summary, lazy data routing is more performant when the messages transferred are larger in size.
As shown in Figure~\ref{fig:exp-comm-latency-eager-lazy}c, eager data routing actually has a lower total communication latency when the messages are smaller than 512KB in size, and lazy data routing performs better when the messages are larger than 512KB in size.

\begin{figure}[t]
    \centering
    \begin{minipage}{0.48\columnwidth}
        \centering
        \includegraphics[width=\textwidth]{figures/exp-lazy-data-routing/speedup.pdf}
        \caption{Lazy data routing scales out well while eager data routing does not.}
        \label{fig:exp-speedup}
    \end{minipage}\hfill
    \begin{minipage}{0.48\columnwidth}
        \centering
        \includegraphics[width=\textwidth]{figures/exp-lazy-data-routing/skipped.pdf}
        \caption{Lazy data routing saves communication when some data is skipped.}
        \label{fig:exp-lazy-skipped}
    \end{minipage}\hfill
\end{figure}

\vspace{0.25em} \noindent \textbf{Lazy Data Routing Naturally Supports Parallelism.}\label{sec:exp-lazy-parallelism}
It is very common for multiple consumers to fetch data from one or more producers at the same time.
In our lazy data routing model, since messages are transferred in a peer-to-peer fashion, the leader node only has a very light workload to process tiny headers simultaneously, saving precious bandwidth at the leader node.
However, in the eager data routing model, the leader node can be blocked when a piece of large message is going through the leader node from a producer to a consumer. As a result, other producers and/or consumers running in parallel cannot send or receive messages at the same time.
% This is essentially a temporary downtime at the leader node, which could cause the entire system to be delayed.

To compare the scalability of communication between eager and lazy data routing models, we have one producer continuously sending the same 512KB message for a total of 100 times to a shared queue. We gradually increase the number of consumer nodes from 1 to 4 and see how it scales out. While no actual computation is done, we measure the total working duration and use the single-node setup for both the eager and lazy data routing as the 1.0x baseline.
Figure~\ref{fig:exp-speedup} shows how the eager data routing model fails to scale out with more consumer nodes, while our lazy data routing model achieves reasonable speedup. The line shadows represent the lower and upper bounds of repeated experiments.
% In this aspect, our lazy data routing model is also a way to alleviate the single point of failure in case of a centralized message broker system.

\vspace{0.25em} \noindent \textbf{Lazy Data Routing Performs Better with Network Contention.}\label{sec:exp-congestion}
Our lazy data routing model is especially beneficial when the leader node is busy with network requests.
% This experiment shows how valuable this contribution can be with the network topology 1 described in \S\ref{sec:exp-setup}.
We specifically construct a task where the message payload is large: real-time inference over video streams.
% In the previous set of experiments, we largely evaluated \sys in terms of its end-to-end benefits in model serving. In this experiment, we focus specifically on the queuing and messaging system and use ROS as the primary baseline.
In this experiment, two webcams capture the same moving QR code from different positions. 
Both videos are 150 frames long at 1920x1080 resolution.
Each camera is connected to a unique data source node on the network.
For multi-camera tracking, the QR code has to be detected in both streams and corresponded in time-aligned frames from both cameras.
So these two data streams need to be joined at the prediction node.
We simulate a congestion scenario where the network bandwidth at the leader node is limited.
Note that the rest of the network retains its full speed; the only congestion is at the leader node.
% We measure the total working duration of our system versus a ROS-like system that transfers raw data through a centralized broker (without lazy routing). 

Table~\ref{tab:congestion} shows the results.
With no congestion, the system can process roughly 0.8 frames per second in both lazy and eager data routing. With congestion, the story is very different. Our lazy data routing is tolerant, while transferring raw frames in a ROS-style eager communication pattern can be extremely slow when the network is congested. The total working duration increases by a factor of 7 simply due to congestion. Without care, distributed, multi-sensor deployments can easily lose real-time processing capabilities if the broker becomes a point of contention. These experiments illustrate the value of \sys in a controlled scenario, where we can isolate performance differences.

\begin{table}[t]
\centering
\begin{tabular}{l|l|l}
Data routing strategy & Rate limit (up/down) & Time    \\ \hline
Lazy (ours) & No limit                       & 3m 10s  \\
Lazy (ours) & 1 Mbps / 1 Mbps                & 3m 12s  \\
Eager (similar to ROS)       & No limit                       & 3m 16s  \\
Eager (similar to ROS)         & 20 Mbps / 20 Mbps              & 21m 32s
\end{tabular}
\caption{Total working duration with network bandwidth limits.}
\label{tab:congestion}
\end{table}

\vspace{0.25em} \noindent \textbf{Lazy Data Routing Performs Better with Data Skipping.}\label{sec:exp-lazy-data-skipping}
Apart from network congestion, lazy data routing is also valuable when data skipping is employed to ensure the timeliness of prediction results.
We take one of the two 150-frame videos mentioned earlier and transfer these frames from one node to another, via the leader node. Each frame is about 6 MB in its uncompressed form. No actual computation is performed as we are only interested in the communication cost.
% We measure the \textit{total working duration} from when the first frame is sent out until the last frame is delivered.
In Figure~\ref{fig:exp-lazy-skipped}, we illustrate how much communication cost can be saved by lazy data routing.
On the x-axis, we have a variable percentage of frames skipped due to adaptive rate control described in \S\ref{sec:rate-control}; on the y-axis, we measure the total working duration defined in \S\ref{sec:exp-metrics-latency}.
Even when no frames are skipped, our lazy data routing model performs better than the eager data routing model due to the eliminated overhead of transferring a large amount of data through the leader node.
When more frames are skipped, our lazy data routing saves communication time almost linearly to the number of frames skipped by the downstream node.
% That means the overhead of transferring headers through the leader node is negligible.
On the other hand, the ROS-style eager routing pattern spends roughly the same time on communication even if most of the frames are skipped by the downstream model, because it would transfer the entire data payload upfront anyway.















\iffalse
\begin{table}[ht]
\centering
\begin{tabular}{l|l|l}
\begin{tabular}[c]{@{}l@{}}\sys parameter\\ \texttt{predition\_freq}\end{tabular} & \begin{tabular}[c]{@{}l@{}}End-to-end\\ latency\end{tabular}  & \begin{tabular}[c]{@{}l@{}}F-1 score\\ (\# of samples)\end{tabular} \\ \hline
30ms             & 1x (real-time)      & 0.90 (7920, 26\%)      \\
10ms             & 1x (real-time)      & 0.90 (11857, 39\%)     \\
5ms              & 1.84x   & 0.22 (31631, 105\%)     \\
1ms              & 6.98x   & 0.20 (120490, 400\%)    \\
\end{tabular}
\caption{End-to-end latency and overall accuracy with the number of samples for the same human activity recognition task. Latency numbers are relative to the actual data incoming rate, where 1x means real-time processing of streaming data. The number of samples is relative to the PyTorch baseline without downsampling/upsampling. Different \texttt{predition\_freq} parameters are applied to demonstrate the effect of adaptive SLO control.}
\label{tab:exp-rate-matching}
\end{table}
\fi




% Numbers are reported as fractions of real-time. \sys is able to serve real-time predictions in both centralized and decentralized settings. This is a ceiling on performance as it does not need to predict any faster than the data arrival rate. On the other hand, PyTorch is unable to serve real-time predictions for this example (running at about 75\% of real-time). Note that decentralized prediction can still save communication costs to reduce latency for PyTorch distributed. This shows the value of such decentralized architectures.

\iffalse
\begin{table}[]
\centering
\begin{tabular}{l|l|l}
End-to-end latency           & PyTorch  & \sys \\ \hline
Centralized             & 1.46x    & 1x (real-time)      \\
Decentralized        & 1.35x    & 1x (real-time)
\end{tabular}
\caption{End-to-end latency for human activity recognition task, from generating the first piece of data until finishing the entire workload. \sys is able to process incoming data in real time while PyTorch does computation based on stale data. Both involve the same task with different model placement methods.}
\label{tab:exp-opportunity-latency}
\end{table}
\fi

\iffalse
\subsubsection{Compute Utilization}
\begin{figure}[t]
    \centering
    \includegraphics[width=0.8\columnwidth]{figures/exp-opportunity-compute-percentage.pdf}
    \caption{Compute utilization for different model placement strategies and user requested output frequencies. Higher target frequencies are on the left side and lower frequencies are on the right side.}
    \label{fig:exp-opportunity-compute-percentage}
\end{figure}
We have defined how we measure compute utilization in \S\ref{sec:exp-metrics}, and figure~\ref{fig:exp-opportunity-compute-percentage} shows how compute utilization changes with different user-defined output rates.
Unlike most training workloads, decentralized prediction tasks are usually not compute-bound, at least not on all nodes. A certain amount of time is always spent on communication, time-synchronization or aggregation of different data sources, rather than mostly on computation.
In the case of \sys centralized, the prediction node is compute-bound when user-defined output rate is too high, but gradually becomes relaxed when the user asks for a lower output rate.
Decentralized model placements usually have a higher compute utilization because local data source nodes also do actual computations.
Note that higher compute utilization is not necessarily a good thing, because we might be working more than we need to. Remember our goal is to give accurate predictions in real time, and that usually means a certain amount of buffer is required to ensure timeliness. In the case of PyTorch decentralized, it does a lot of computational work but still fails to achieve our requirements for end-to-end latency and real-time accuracy.
\fi


% Now, it is true that this is not a completely fair comparison between PyTorch and \sys, since message misalignments can affect accuracy in \sys.
% However, we find this effect negligible. Figure \ref{fig:exp-opportunity-accuracy} shows the F1-score for \sys and Pytorch in both centralized and decentralized deployments. It is worth noting that there is a subtle issue of ``temporal'' accuracy because PyTorch is slower than real-time.  We ignore the fact that most predictions made with PyTorch distributed are outdated and compare them with the ground truths sequentially anyway. In \sys, however, we do care about the timeliness of predictions and treat outdated results as inaccurate results. Even under this strict definition of accuracy, we find that the communication optimizations in \sys are highly beneficial with a negligible impact on results.

\iffalse
\begin{table}[]
\centering
\begin{tabular}{l|l|l}
F-1 score   & PyTorch  & \sys \\ \hline
Centralized             & 0.90   & 0.90     \\
Decentralized         & 0.91   & 0.91
\end{tabular}
\caption{Overall accuracy for human activity recognition task measured in F-1 score. Despite message misalignments and rate-matching, \sys achieves the same accuracy as a fully-synchronized deployment.}
\label{tab:exp-opportunity-accuracy}
\end{table}
\fi





% We set different \texttt{predition\_freq} parameters for the same task with (a) full transfer method. Table~\ref{tab:exp-rate-matching} shows the end-to-end latency and overall accuracy with the number of samples processed by \sys with different parameter settings. With a higher prediction frequency like 1ms, \sys automatically upsamples incoming data in order to make more frequent predictions. However, since the prediction process takes much longer than 1ms, upsampled data queues up and makes the end-to-end latency even worse. The prediction accuracy also gets worse because the data we use for predictions is outdated. With a lower prediction frequency closer to the actual model throughput, \sys automatically downsamples incoming data to match the given frequency. Since the queue is clear, we always make predictions based on the latest data to achieve better accuracy with lower end-to-end latency, only with a smaller number of samples.

\iffalse
\begin{table}[]
\centering
\begin{tabular}{l|l|l}
\# of samples processed    & PyTorch  & \sys \\ \hline
Centralized             & 30127   & 7920 (26\%)     \\
Decentralized         & 30127   & 11671 (39\%)
\end{tabular}
\caption{Overall number of samples for human activity recognition task. \sys is able to automatically downsample incoming data in order to catch up with data incoming rate.}
\label{tab:exp-opportunity-matching}
\end{table}
\fi

% \subsection{Micro-Benchmarks}
% First, we evaluate key design decisions of \sys as micro-benchmarks. Specifically, we show the benefits of (1) lazy data routing and (2) flexible data routing provided by a message broker.










% \subsection{Message Queue vs. RESTful API}
% First, we compare \sys with state-of-the-art model serving systems with RESTful APIs.


% We evaluate \sys based on two use cases: (1) network intrusion detection and (2) human activity recognition.
% For use case 1, we demonstrate how \sys is able to handle distributed prediction in a traditional workload that maximizes throughput as its goal.
% In this case, we ignore the complexity of data movement and assume the entire dataset is readily available upon request. Experiments show that \sys outperforms PyTorch distributed in such scenarios by 4x with a message broker and lazy data routing.
% For use case 2, we evaluate stream alignment and automatic rate matching offered by \sys. Data continuously comes in real time, and we need to aggregate data from multiple sources in order to make predictions in real time. \sys is able to align multiple streams of data in a time-synchronized way and produce predictions at the rate requested by the user.

\subsection{Comparison with Ray Serve}\label{sec:exp-system-overhead}
We conduct an object detection task with another serving system, Ray Serve~\cite{ray}, with a sample of nuScenes~\cite{nuscenes} camera data and pre-trained YOLOv5n model~\cite{yolov5} on a single NVIDIA Jetson Nano.

Single-node performance between Ray Serve, \sys, and an ideal case where the job runs locally without any communication is presented in Table~\ref{tab:exp-nuscenes-overhead}.
First, we enforce the freshness threshold SLO (\S\ref{sec:freshness-threshold}) of 1 second and see how many examples must be skipped in order to hit the SLO.
Since the data comes faster than the model's inference speed, 19.0\% of incoming data has to be skipped even in an ideal case.
\sys skips a bit more examples than ideal, but the overhead is reasonably small.
Ray Serve, however, skips 89.4\% of incoming data, which means the system consumes more computational resources than the task itself.
Second, we drop the SLO requirement and see how long it takes for each system to complete the task without downsampling.
In an ideal scenario, the model runs for 25 seconds to finish the dataset.
\sys spends 26 seconds, which presents negligible system overhead.
Ray Serve, on the other hand, spends 2m40s finishing the task, which is 6.4x slower compared to ideal due to its complex design.
In addition, we compare key design decisions between Ray Serve and \sys in Table~\ref{tab:rayserve-comp}.



\begin{table}[]
\small
\centering
\begin{tabular}{l|l|l|l}
    & Ray Serve  & \sys & Ideal \\ \hline
\makecell{\% examples skipped\\ (w/ SLO enforced)}             & 89.4\%   & 22.2\% & 19.0\%     \\ \hline
\makecell{Total working duration\\ (w/o SLO enforced)}         & 2m40s    & 26s    & 25s
\end{tabular}
\caption{Single-node performance of Ray Serve, \sys, and an ideal case.}
\label{tab:exp-nuscenes-overhead}
\end{table}








\iffalse

\section{Experiments: Multi-Camera Tracking}
% Full experiments: https://hackmd.io/@swjz/rJ1nfwi3i
% Spreadsheet and figures: https://docs.google.com/spreadsheets/d/1ebVHpCJEicbJ0IQzw15sqvJAuKCvkWpHMAvcoJAiSeQ/edit?usp=sharing
Our goal for this paper is to simulate a synchronization-sensitive task and demonstrate the trade-off between latency, accuracy, and communication. In order to achieve this goal, we set up a QR code detector where two webcams capture the same QR code from different positions. We move the physical position of the QR code along a horizontal axis and observe the QR code positions detected by both cameras. We compare the trajectory of positions to a centralized baseline where both cameras collect data on the same node to evaluate accuracy. 

\iffalse
\begin{figure}[t]
    \centering
    \includegraphics[width=1\columnwidth]{figures/offset.pdf}
    \caption{Illustration of QR code offset.}
    \label{fig:offset}
\end{figure}
\fi


\subsection{Hardware Setup}\label{sec:exp-setup}
Our hardware setup consists of two 1080p webcams and four Intel Skylake NUC computers, each equipped with an Intel Core i3-6100U CPU, 16 GB memory, and M.2 SSD. In a centralized setting (Sec.~\ref{sec:centralized}), only one NUC computer is used and it is connected to both webcams. In a distributed setting (Sec.~\ref{sec:distributed}), all four NUC computers are used: two of them are connected to two webcams respectively serving as data source nodes, one of them serves as a message broker and the other NUC serves as the compute node taking input from data source nodes. All four NUCs are interconnected via 100Mbps Ethernet.

\subsection{Software Setup}\label{sec:software}
We use Apache Pulsar~\cite{apachepulsar} as the message broker to transfer messages between NUC computers. For small messages such as a 2D array, we transfer them directly via Pulsar. For larger files such as images, we create FTP paths for them and transfer those paths in messages for the compute node to download, saving traffic on the message broker side. For QR code detection, we use an OpenCV resolution with two CNN-based Caffe models: an object detection model to detect the QR code with a bounding box and a super-resolution model to zoom in the QR code when it is small.
Videos are collected in advance to ensure reproducibility and we simulate real-time streaming of these videos.
All results present the average values of 3 experiments.
All videos used in the following experiments are of 1920x1080 resolution, 5 seconds long at 30 FPS unless otherwise specified. The size of QR code is about 200x200.

\subsection{Metrics and Ground Truth}
In this experiment, we define accuracy as pixel-level `error', which is the difference between ground truth trajectory and the experiments in both x and y axes in the unit of absolute pixels, averaged over all frames. The smaller the absolute number of `error', the more accurate the target experiment is. The ground truth of such offset is defined as the offset between two videos in a centralized setup without compression (Table~\ref{tab:centralized}(a)-(d)). If there is no QR code detected from a certain frame in the target experiment, we use the last known QR code position for that camera. For down-sample experiments, we up-sample the missing frames with the last known frame when measuring accuracy as well.
We also define latency as the time period from the timestamp when the first piece of data is transferred until the timestamp when the prediction for the last piece of data is issued. Since all of our videos have the same length of 5 seconds, this metric is a proxy for ``timeliness'' defined before.

\subsection{Centralized Compute}
\label{sec:centralized}
As a baseline, we consider a scenario where the computation is centralized. There is no communication or synchronization issue in this case. Therefore, we treat the result from this run as ground truth and running time as a baseline. We demonstrate the intrinsic characteristics between fast and slow movements of the QR code and explore the latency component of disk I/O to get a better understanding of the task.

\begin{table}[]
\centering
\begin{tabular}{l|l|l}
                 & Fast movement & Slow movement \\ \hline
a) Memory       & 9.92s                 & 10.52s        \\
%b) Write as BMP     & 11.60s (+1.68s, 17\%)       & 12.04s (+1.52s, 14\%)        \\
%c) Read from BMP    & 12.07s (+2.15s, 22\%)       & 12.42s (+1.90s, 18\%)        \\
b) Disk (uncompressed)  & 13.53s (+3.61s, 36\%)       & 13.75s (+3.23s, 31\%)       \\
c) Disk (jpeg) & 29.76s (+19.84s, 200\%)       & 31.16s (+20.64s, 196\%)
\end{tabular}
\caption{Latency from a centralized compute: (a)-(c) involve the same task with different level of disk access. Numbers and percentages in parentheses are relative to (a).}
\label{tab:centralized}
\end{table}

Table~\ref{tab:centralized} shows the latency from centralized multi-camera tracking example: (a) both streams are captured and passed to \sys in memory; (b) the camera streams are stored to disk in an uncompressed format and incrementally retrieved by \sys; and (c) the camera streams are stored to disk in a JPEG format and incrementally retrieved by \sys.

First, we look at the differences between fast movement and slow movement columns. When we move the QR code too fast, quite a few frames are too blurry for the detector to recognize anything so the decoding step is skipped. Therefore, it takes a little shorter time to finish the computation in fast movement cases. Second, we compare Table~\ref{tab:centralized}(b) against (a) to measure the latency of disk I/O. Specifically, reading and writing image files from/to disk takes about the same amount of time and they add up to about 30-40\% of compute time. We see from Table~\ref{tab:centralized}(c) that disk I/O with JPEG runs significantly longer because extra time is spent on compression and decompression.

It should be noted that images in BMP format are lossless and the size of each BMP file is about 6 MB. JPEG compression is lossy but could significantly reduce file size to 200-300 KB. The accuracy of JPEG compression is shown in Table~\ref{tab:jpeg-accuracy}. In both axes, we see an average error of less than 1 pixel, which is nearly perfect. Fast movement is worse because some blurry frames that are recognizable as BMP files are no longer recognizable after JPEG compression.

\begin{table}[]
\centering
\begin{tabular}{l|l|l}
JPEG accuracy & Fast movement & Slow movement \\ \hline
x-axis        & 0.5931px      & 0.0050px     \\
y-axis        & 0.3868px      & 0.0022px     
\end{tabular}
\caption{Errors introduced by JPEG compression.}
\label{tab:jpeg-accuracy}
\end{table}

\subsection{Distributed Edge Cluster}\label{sec:distributed}
As described in Sec.~\ref{sec:exp-setup}, we construct an Edge cluster and do the same QR code detection task, where data is collected on different nodes.
% Both data sources need to arrive at the compute node in an aligned manner.
We build a queue for each data source at the message broker and they are aggregated to make a prediction as soon as there is new data coming in from any data source. A data point may be reused to make a joint prediction if it is still the latest from an infrequent data source.
Jitter in the network, variability in processing times, and queuing delays can introduce extra errors. We compare these errors to the sub-pixel errors introduced by lossy compression above.

% Slow movement results
% \begin{table}[]
% \centering
% \begin{tabular}{l|l|l|l|l}
%               & Size   & Time                                                      & x-axis error & y-axis error \\ \hline
% BMP (30 FPS)  & 1.7 GB & \begin{tabular}[c]{@{}l@{}}3m 10s\\ (2m 41s)\end{tabular} & -31.216px    & 1.4522px     \\
% BMP (10 FPS)  & 593 MB & \begin{tabular}[c]{@{}l@{}}1m 4s\\ (53.8s)\end{tabular}   & -14.6671px   & 0.6375px     \\
% JPEG (30 FPS) & 72 MB  & \begin{tabular}[c]{@{}l@{}}52.7s\\ (8.35s)\end{tabular}   & 4.7356px     & -0.0940px    \\
% JPEG (10 FPS) & 27 MB  & \begin{tabular}[c]{@{}l@{}}17.9s\\ (3.2s)\end{tabular}    & 0.4415px     & -0.0150px   
% \end{tabular}
% \caption{BMP, JPEG and downsampling comparison. Numbers in parentheses are time spent purely on data download, measured by \texttt{wget}.}
% \label{tab:distributed}
% \end{table}

% Fast movement results
\begin{table}[]
\centering
\begin{tabular}{l|l|l|l|l}
              & Size   & Time                                                      & x-axis error & y-axis error \\ \hline
BMP (30 FPS)  & 1.7 GB & \begin{tabular}[c]{@{}l@{}}3m 11s\\ (2m 41s)\end{tabular} & 22.8402px   & 4.1812px     \\
BMP (10 FPS)  & 593 MB & \begin{tabular}[c]{@{}l@{}}1m 4s\\ (53.8s)\end{tabular}   & 7.3501px    & 1.7942px     \\
BMP (5 FPS)   & 297 MB & \begin{tabular}[c]{@{}l@{}}32.1s\\ (27.1s)\end{tabular}   & 29.4458px    & 1.5753px     \\
JPEG (30 FPS) & 72 MB  & \begin{tabular}[c]{@{}l@{}}49.8s\\ (8.3s)\end{tabular}    & 4.0714px     & 0.4754px     \\
JPEG (10 FPS) & 24 MB  & \begin{tabular}[c]{@{}l@{}}16.6s\\ (2.9s)\end{tabular}    & 6.6651px    & 2.7245px     \\
JPEG (5 FPS)  & 12 MB  & \begin{tabular}[c]{@{}l@{}}8.2s\\ (1.6s)\end{tabular}    & 29.4797px    & 1.6021px    
\end{tabular}
\caption{BMP, JPEG and down-sampling comparison. Numbers in parentheses are time spent purely on data download, measured by \texttt{wget}.}
\label{tab:distributed}
\end{table}

% \vspace{0.25em} \noindent \emph{Effect of Down-Sampling and Compression on Accuracy.}
\subsubsection{Effect of Down-sampling and Compression on Accuracy.}\label{sec:exp-downsample}
Counter-intuitively, degrading the quality of the data collected from the sensors can lead to better results in the distributed setting.
We compare compression and effects of down-sampling in Table~\ref{tab:distributed}. We find that down-sampling can almost linearly improve latency because it directly reduces data transfer. It also forces synchronization on the data source side to improve accuracy until it reaches a sweet spot, after which we lose so much information between frames that accuracy starts to decrease. JPEG compression has a similar effect of reducing data transfer significantly, which also improves latency compared to BMP counterparts. Since JPEG compression takes a while (as shown in Table~\ref{tab:centralized}), it acts as a rate limit at the data source, similar to the down-sampling method, which also forces synchronization. The accuracy `sweet spot' for JPEG is reached at the original sampling rate and further down-sampling would only decrease accuracy.
% Alternatively, we could also manually set a rate limit at compute node, which buffers input data but does not issue predictions until all data sources are fully synchronized.

% Slow movement results
% \begin{table}[]
% \centering
% \begin{tabular}{l|l|l|l}
% Ensemble method      & Time  & x-axis error & y-axis error \\ \hline
% Original 30 FPS   & 6.55s & 58.911px     & 0.2834px     \\
% Downsampled to 10 FPS & 6.35s & -11.273px    & 0.8657px
% \end{tabular}
% \caption{Latency and accuracy for Ensemble method.}
% \label{tab:ensemble}
% \end{table}

% Fast movement results
\begin{table}[]
\centering
\begin{tabular}{l|l|l|l}
Endpoint-Placement & Time  & x-axis error & y-axis error \\ \hline
Original 30 FPS       & 5.79s & 22.4691px    & 2.7398px    \\
Down-sampled to 10 FPS & 5.93s & 24.7646px   & 5.6231px     \\
Down-sampled to 5 FPS  & 5.55s & 29.5273px   & 1.7011px
\end{tabular}
\caption{Latency and accuracy for Endpoint-Placement.}
\label{tab:ensemble}
\end{table}

% \vspace{0.25em} \noindent \emph{Compute Placement and Synchronization Errors.}
\subsubsection{Compute Placement and Synchronization Errors.}\label{sec:exp-placement}
It is a natural thought that moving inference closer to the point of data collection is desirable in edge computing; this is not always the case in multimodal prediction.
Instead of transferring frames over the network, we could also run the model at data source nodes and only transfer coordinates over the network. This method makes use of model parallelism and spare resources on data source nodes to save communication between data source nodes and compute node(s). Such an ``Endpoint-Placement'' method is perfectly suitable for our use case because we have light models but heavy communication. Table~\ref{tab:ensemble} shows that this method improves latency by 33x compared to BMP (30 FPS) in Table~\ref{tab:distributed}.

However, Endpoint-Placement methods can be a source of misalignment as well, especially when a model runs faster for some data points but slower for other data points. The variability in model inference latency across nodes can add up to become a synchronization problem and reduces accuracy. In Table~\ref{tab:ensemble}, we see higher synchronization errors because of this variability.


% \vspace{0.25em} \noindent \emph{Effect of Queuing Strategy.}
\subsubsection{Effect of Lazy Data Routing.}\label{sec:exp-queuing}
In Sec.~\ref{sec:software} we described how we use FTP to transfer large data directly from data source to compute node and only pass pointers to data over the message broker. This is especially beneficial when the message broker is busy with network requests. We simulate a congestion scenario where the network bandwidth at the message broker is limited and measure the end-to-end latency of our system versus a ROS-like system that transfers raw data through a centralized broker. Table~\ref{tab:congestion} shows that our system is tolerant to slow network bandwidth while transferring raw frames can be extremely slow when the network is congested.

\begin{table}[]
\centering
\begin{tabular}{l|l|l}
                          & Rate limit (up /down) & Time    \\ \hline
Lazy (ours) & No limit                       & 3m 10s  \\
Lazy (ours) & 1 Mbps / 1 Mbps                & 3m 12s  \\
Eager (similar to ROS)       & No limit                       & 3m 16s  \\
Eager (similar to ROS)         & 20 Mbps / 20 Mbps              & 21m 32s
\end{tabular}
\caption{Latency with network bandwidth limits.}
\label{tab:congestion}
\end{table}

\fi
\section{Related work}
\noindent \textbf{Video foundation models.}
With sufficient computational power and an abundant source of data, there have been attempts to build a single large-scale foundation model that can be adapted to diverse downstream tasks.
Along with the success of foundations models in the natural language processing domain~\cite{brown2020language,chen2021evaluating,devlin2019bert} and in computer vision~\cite{bertasius2021space,jia2021scaling,radford2021learning}, video data has become another data type of interest, as it has grown in scale due to numerous internet video-sharing platforms.
Accordingly, several methods to train a video foundation model have been proposed.
Due to the innate multi-modality of video data, \textit{i.e.}, a combination of visual $\cdot$ vocal $\cdot$ textual context, most works have centered around the variations of the cross-modal attention mechanism \cite{akbari2021vatt,bertasius2021space,gabeur2020multi,luo2020univl,neimark2021video,tan2021look,wei2020multi,yang2021taco}.
In addition, as most video data lack proper labels or descriptions, contrastive learning methods were studied to learn meaningful feature representations or enhance video-text alignment in a self-supervised manner \cite{akbari2021vatt,kuang2021video,luo2020univl,yang2021taco}.

More specifically, MERLOT \cite{zellers2021merlot} proposed a multi-modal representation learning method for visual commonsense reasoning, which also performed well in twelve video reasoning tasks.
VATT \cite{akbari2021vatt} introduced a multi-modal learning method via contrastive learning. 
The pre-trained model performed well in a variety of vision tasks from image classification to video action recognition and zero-shot video retrieval.
Another representative work, UniVL \cite{luo2020univl} proposed a straightforward pre-training method with auxiliary loss functions. 
After fine-tuning on a specific task, the pre-trained model performed outstandingly in a wide range of tasks of text-to-video retrieval, action segmentation, action step localization, video sentiment analysis, and video captioning.
Other foundation models for multiple video tasks include \cite{li2020hero,sun2019learning,sun2019videobert,zhu2020actbert,fu2021violet,wang2022all}. 

\noindent \textbf{Auxiliary learning.}
In order to enhance the performance of one or a multitude of primary tasks, auxiliary learning methods can be incorporated.
\cite{ruder2017overview} introduced Multi-task learning (MTL) to the deep neural networks by training a single model with multiple task losses to assist learning on the main task.
Such a method is generally adapted to pre-train the foundation models in the self-supervised manner~\cite{li2020hero,sun2019learning,sun2019videobert,zhu2020actbert,fu2021violet,wang2022all}.
However, these various pretext task losses used in the pre-training phase are ignored in the fine-tuning phase, and only the primary task loss is minimized.

Recently, meta-learning methods have been introduced for auxiliary learning.
\cite{liu2019self,navon2020auxiliary,shu2019meta} proposed a meta-learning method in which the model learns auxiliary tasks to generalize well to unseen data. 
In these settings, a separate subset of data is held out as the primary task, while the others are used as auxiliary tasks that aid the primary task's performance.
Similar methods were adopted for computer vision tasks such as semantic segmentation \cite{xu2021leveraging}.
Other domain applications include navigation tasks with reinforcement learning \cite{ye2021auxiliary}, or self-supervised learning methods on graph data \cite{hwang2020self}.
% \section{Conclusion}\label{sec:conclusion}
In this work, we focus on addressing the fundamental challenge of OOD detection tasks, which is how to fully understand the semantic discrepancy between the ID/OOD samples. We reveal that the key to success in the realistic SCOOD task is to allocate as many ID samples in the unlabeled set correctly as possible. To this end, we propose a novel uncertainty-aware optimal transport scheme that introduces class-specific energy scores as guidance for effective label assignment. Experimental results show that our method achieves better performance than previous state-of-the-art methods on SCOOD benchmarks.

\textbf{Limitations.} In addition to temperature scaling, other techniques such as feature clipping applied in ReAct~\cite{sun2021react} also enhance the performance of energy score, so how to obtain an OOD score that best fits the SCOOD task can be further explored. Moreover, a setting highly related to SCOOD has been proposed in \cite{katz2022training} and formulated as a constrained optimization problem. We will also theoretically analyze these practical OOD settings in our feature work.

% \section*{Acknowledgments}
\textbf{Acknowledgments.} 
This work is supported by National Key R\&D Program of China under Grant 2020AAA0105701, National Natural Science Foundation of China (NSFC) under Grants 61872327, Major Special Science and Technology Project of Anhui, National Natural Science Foundation of China (62033012) and Ant Group through Ant Research Intern Program.

\balance

% \bibliographystyle{ACM-Reference-Format}
% \bibliography{reference}
\bibliographystyle{plain}
\bibliography{reference}
\newpage
\section{Appendix for Proofs}

\paragraph{Proof of Theorem \ref{thm:main}.}

\begin{proof}
\label{proof:main}
Our proof has two steps. In Step 1, we will show that SimCLR is equivalent to minimizing the cross entropy loss defined in Eqn.~(\ref{eqn:cross-entropy}). 
In Step 2, we will show  that minimizing the cross-entropy loss 
is equivalent to spectral clustering on $\bfpi$. 
Combining the two steps together, we have proved our theorem. 

\textbf{Step 1: } SimCLR is equivalent to minimizing the cross entropy loss.

The cross-entropy loss takes expectation over 
$\bfW_\bfX\sim \mathbb{P}(\cdot ; \bfpi)$, 
which means $\bfW_\bfX$ has exactly one non-zero entry in each row $i$. By Lemma~\ref{lem:multinomial}, we know every row $i$ of $\bfW_\bfX$ is independent of other rows. Moreover, 
$\bfW_{\bfX,i}\sim \mathcal{M}(1, \bfpi_i/\sum_j \bfpi_{i,j})=\mathcal{M}(1, \bfpi_i)$, because $\bfpi_i$ itself is a probability distribution.
Similarly, we know $\bfW_\bfZ$ also has the row-independent property by sampling over $\mathbb{P}(\cdot;\bfK_\bfZ)$.
Therefore, by Lemma~\ref{lem:cross_split}, we know Eqn.~(\ref{eqn:cross-entropy}) is equivalent to:
\[
 -\sum_{i=1}^n \mathbb{E}_{\bfW_{\bfX,i}}[\log \mathbb{P}(\bfW_{\bfZ,i}=\bfW_{\bfX,i};\bfK_\bfZ)],
\]

This expression takes expectation over $\bfW_{\bfX,i}$ for the given row $i$. Notice that 
$\bfW_{\bfX,i}$ has exactly one non-zero entry, which equals $1$ (same for $\bfW_{\bfZ,i}$). 
As a result
we expand the above expression to be:
\begin{equation}
 -\sum_{i=1}^n \sum_{j\neq i} \Pr(\bfW_{\bfX,i,j}=1)\log \Pr(\bfW_{\bfZ,i,j}=1).
\label{eqn:detailed-expansion}    
\end{equation}


By Lemma~\ref{lem:multinomial}, $\Pr(\bfW_{\bfZ,i,j}=1)=\bfK_{\bfZ,i,j}/\|\bfK_{\bfZ,i}\|_1$ for $j\neq i$. Recall that $\bfK_\bfZ=(k(\bfZ_i-\bfZ_j))_{(i,j)\in[n]^2}$, which means 
$\bfK_{\bfZ,i,j}/\|\bfK_{\bfZ,i}\|_1=\frac{\exp(-\|\bfZ_i-\bfZ_j\|^2/{2\tau})}{\sum_{k\neq i}
\exp(-\|\bfZ_i-\bfZ_k\|^2/{2\tau})
}$ for $j\neq i$, when $k$ is the Gaussian kernel with variance $\tau$. 

Notice that $\bfZ_i=f(\bfX_i)$, so we know
\begin{equation}
-\log \Pr(\bfW_{\bfZ,i,j}=1)=
-\log \frac{\exp(-\|f(\bfX_i)-f(\bfX_j)\|^2/{2\tau})}{\sum_{k\neq i}
\exp(-\|f(\bfX_i)-f(\bfX_k)\|^2/{2\tau}),
}
\label{eqn:infonce-equivalence}    
\end{equation}


The right hand side is exactly the InfoNCE loss defined in Eqn.~(\ref{eqn:infonce}).
Inserting Eqn.~(\ref{eqn:infonce-equivalence}) into Eqn.~(\ref{eqn:detailed-expansion}), we get the SimCLR algorithm, which first samples augmentation pairs $(i,j)$ with $\Pr(\bfW_{\bfX,i,j}=1)$ for each row $i$, and then optimize the InfoNCE loss. 

\textbf{Step 2: } minimizing the cross entropy loss 
is equivalent to spectral clustering on $\bfpi$.


By Lemma~\ref{lem:convert_to_spectral}, we may further convert the loss to 
\begin{equation}
\label{eqn:main-theorem-repul-attr}
\min_{\bfZ}
-\sum_{(i,j)\in [n]^2} \mathbf{P}_{i,j}
\log k (\bfZ_i-\bfZ_j)+\log \mathbf{R}(\bfZ).
\end{equation}
Since $k$ is the Gaussian kernel, this reduces to \[
\min_\bfZ \mathrm{tr}(\bfZ^\top \mathbf{L}(\bfpi) \bfZ)
+\log \mathbf{R}(\bfZ),
\]

where we use the fact that $\mathbb{E}_{\bfW_\bfX\sim \mathbb{P}(\cdot; \bfpi)}[\mathbf{L}(\bfW_\bfX)]
=\mathbf{L}(\bfpi)
$, because the Laplacian operator is linear and $
\mathbb{E}_{\bfW_\bfX\sim \mathbb{P}(\cdot; \bfpi)}(\bfW_\bfX)=\bfpi
$.
\end{proof}

\paragraph{Proof of Theorem \ref{thm:clip}.}
\begin{proof}
Since $\bfW_\bfX\sim \mathbb{P}(\cdot;\bfpi_{\mathbf{A}, \mathbf{B}})$, we know 
$\bfW_\bfX$ has exactly one non-zero entry in each row, denoting the pair that got sampled. 
A notable difference compared to the previous proof is we now have $n_\mathcal{A}+n_\mathcal{B}$ objects in our graph. CLIP deals with this by taking a mini-batch of size $2N$, 
such that $n_\mathcal{A}=n_\mathcal{B}=N$, and adding the $2N$ InfoNCE losses together. We label the objects in $\mathcal{A}$ as $[n_\mathcal{A}]$, and the objects in $\mathcal{B}$ as $\{n_\mathcal{A}+1, \cdots, n_\mathcal{A}+n_\mathcal{B}\}$. 

Notice that $\bfpi_{\mathbf{A}, \mathbf{B}}$ is a bipartite graph, so the edges of objects in $\mathcal{A}$ will only connect to object in $\mathcal{B}$ and vice versa. We can define the similarity matrix in $\cZ$ as $\bfK_\bfZ$, 
where $\bfK_\bfZ(i, j+n_\mathcal{A})=\bfK_\bfZ(j+n_\mathcal{A},i)= k(\bfZ_i-\bfZ_j)$ for $i\in [n_\mathcal{A}], j\in [n_\mathcal{B}]$, and otherwise we set $\bfK_\bfZ(i,j)=0$. 
The rest is same as the previous proof. 
\end{proof}

\paragraph{Proof of Theorem \ref{thm:exponential}.}

\begin{proof}
\label{proof:exponential}
Since the objective function consists of a linear term combined with an entropy regularization, which is a strongly concave function, the maximization problem is a convex optimization problem. Owing to the implicit constraints provided by the entropy function, the problem is equivalent to having only the equality constraint. We then introduce the Lagrangian multiplier $\lambda$ and obtain the following relaxed problem:

$$
\widetilde{E}(\boldsymbol{\alpha})=\psi_{1}-\sum_{i=1}^n \alpha_{i} \psi_{i}+\tau \sum_{i=1}^n \alpha_{i}\log \alpha_{i}+\lambda\left(\boldsymbol{\alpha}^{\top} \mathbf{1}_n-1\right).
$$

As the relaxed problem is unconstrained, taking the derivative with respect to $\alpha_{i}$ yields

$$
\frac{\partial \widetilde{E}(\boldsymbol{\alpha})}{\partial \alpha_{i}}=-\psi_{i}+\tau\left(\log \alpha_{i}+\alpha_{i} \frac{1}{\alpha_{i}}\right)+\lambda=0.
$$

Solving the above equation implies that $\alpha_{i}$ takes the form
$
\alpha_{i}=\exp \left(\frac{1}{\tau} \psi_{i}\right) \exp \left(\frac{-\lambda}{\tau}-1\right).
$ Since $\alpha_{i}$ lies on the probability simplex, the optimal $\alpha_{i}$ is explicitly given by
$
\alpha^{*}_{i}=\frac{\exp \left(\frac{1}{\tau} \psi_{i}\right)}{\sum_{i^{\prime}=1}^n \exp \left(\frac{1}{\tau} \psi_{i^{\prime}}\right)} .
$ Substituting the optimal point into the objective function, we obtain
$$
\begin{aligned}
E\left(\boldsymbol{\alpha}^*\right)  &=\psi_1-\sum_{i=1}^n \frac{\exp \left(\frac{1}{\tau} \psi_{i}\right)}{\sum_{i^{\prime}=1}^n \exp \left(\frac{1}{\tau} \psi_{i^{\prime}}\right)} \psi_{i}+\tau \sum_{i=1}^n \frac{\exp \left(\frac{1}{\tau} \psi_{i}\right)}{\sum_{i^{\prime}=1}^n \exp \left(\frac{1}{\tau} \psi_{i^{\prime}}\right)}\log \frac{\exp \left(\frac{1}{\tau} \psi_{i}\right)}{\sum_{i^{\prime}=1}^n \exp \left(\frac{1}{\tau} \psi_{i^{\prime}}\right)} \\
& =\psi_1 - \tau \log \left(\sum_{i=1}^n \exp \left(\frac{1}{\tau} \psi_{i}\right)\right).
\end{aligned}
$$
Thus, the Lagrangian dual function is given by
\begin{equation*}
-E\left(\boldsymbol{\alpha}^*\right)= -\tau \log \frac{\exp \left(\frac{1}{\tau} \psi_{1}\right)}{\sum_{i=1}^n \exp \left(\frac{1}{\tau} \psi_{i}\right)}.\qedhere
\end{equation*}
\end{proof}



\section{More on Experiments} \label{section: experiment_details}

\paragraph{CIFAR-10 and CIFAR-100} CIFAR-10 ~\citep{krizhevsky2009learning} and CIFAR-100 ~\citep{krizhevsky2009learning} are well-known classic image classification datasets. Both CIFAR-10 and CIFAR-100 contain a total of 60k $32 \times 32$ labeled images of different classes, with 50k for training and 10k for testing. CIFAR-10 is similar to CIFAR-100, except there are 10 different classes in CIFAR-10 and 100 classes in CIFAR-100.

\paragraph{TinyImageNet} TinyImageNet ~\citep{le2015tiny} is a subset of ImageNet ~\citep{deng2009imagenet}. There are 200 different object classes in TinyImageNet, with 500 training images, 50 validation images, and 50 test images for each class. All the images in TinyImageNet are colored and labeled with a size of $64 \times 64$.

\textbf{Pseudo-code.} Algorithm \ref{alg:Training Procedure} presents the pseudo-code for our empirical training procedure.

\begin{algorithm}[!htbp]
\caption{Training Procedure}
\label{alg:Training Procedure}
\begin{algorithmic}[1]
\REQUIRE trainable encoder network $f$, batch size $N$, augmentation strategy \textit{aug}, loss function $L$ with hyperparameters \textit{args}
\FOR {sampled minibatch ${x_i}_{i=1}^N$}
\FORALL{$i \in { 1, ..., N }$}
\STATE draw two augmentations $t_i = \textit{aug}\left(x_i\right) $, $t_i' = \textit{aug}\left(x_i\right) $
\STATE $z_i = f\left(t_i\right)$, $z_i' = f\left(t_i'\right)$
\ENDFOR
\STATE compute loss $\mathcal{L} = L(N, z, z', \textit{args})$
\STATE update encoder network $f$ to minimize $\mathcal{L}$
\ENDFOR
\STATE \textbf{Return} encoder network $f$
\end{algorithmic}
\end{algorithm}

We also provide the pseudo-code for our core loss function used in the training procedure in Algorithm \ref{alg:Core loss}. The pseudo-code is almost identical to SimCLR's loss function, with the exception of an extra parameter $\gamma$.

\begin{algorithm}[!htbp]
\caption{Core loss function $\mathcal{C}$}
\label{alg:Core loss}
\begin{algorithmic}[1]
\REQUIRE batch size $N$, two encoded minibatches $z_1, z_2$, $\gamma$, temperature $\tau$
\STATE $z = \textit{concat}\left(z_1, z_2\right)$
\FOR {$i \in {1, ..., 2N }, j \in {1, ..., 2N}$ }
\STATE $s_{i,j} = \Vert z_i - z_j \Vert_2^{\gamma}$
\ENDFOR
\STATE \textbf{define} $l(i, j)$ \textbf{as} $l(i, j) = - \log \frac{exp\left(s_{i,j}/\tau \right)}{\sum_{k=1}^{2N} \mathbf{1}{[k \ne i]} exp\left(s{i, j} / \tau \right)} $
\STATE \textbf{Return} $\frac{1}{2N} \sum_{k=1}^N\left[l(i, i+N) + l(i+N, i)\right]$
\end{algorithmic}
\end{algorithm}

Utilizing the core loss function $\mathcal{C}$, we can define all kernel loss functions used in our experiments in Table \ref{table: loss definition}. For all $z_i \in z$ with even dimensions $n$, we define $z_{L_i} = z_i\left[0:n/2\right]$ and $z_{R_i} = z_i\left[n/2:n\right]$.

\begin{table}[ht]
\centering
\begin{tabular}{{@{}l|l@{}}}
Kernel  &  Loss function \\ \midrule
Laplacian & $\mathcal{C}\left(N, z, z', \gamma=1, \tau\right)$\\ \midrule
Sum       & $\lambda * \mathcal{C}\left(N, z, z', \gamma=1, \tau_1\right) + (1-\lambda) * \mathcal{C}\left(N, z, z', \gamma=2, \tau_2\right)$  \\ \midrule
Concatenation Sum&$\lambda * \mathcal{C}\left(N, z_L, z'_L, \gamma=1, \tau_1\right) + (1-\lambda) * \mathcal{C}\left(N, z_R, z'_R, \gamma=2, \tau_2\right)$\\ \midrule
$\gamma = 0.5$ & $\mathcal{C}\left(N, z, z', \gamma=0.5, \tau\right)$          \\ 

\end{tabular}

\caption{Definition of kernel loss functions in our experiments}
\label {table: loss definition}
\end{table}

\textbf{Baselines.} We reproduce the SimCLR algorithm using PyTorch Lightning~\citep{PytorchLightning}.

\textbf{Encoder details.}
The encoder $f$ consists of a backbone network and a projection network. We employ ResNet50~\citep{ResNet} as the backbone and a 2-layer MLP (connected by a batch normalization~\citep{ioffe2015batch} layer and a ReLU \cite{nair2010rectified} layer) with hidden dimensions 2048 and output dimensions 128 (or 256 in the concatenation kernel case).

\textbf{Encoder hyperparameter tuning.}
For each encoder training case, we randomly sample 500 hyperparameter groups (sample details are shown in Table \ref{table: Hyperparameter sample}) and train these samples simultaneously using Ray Tune ~\citep{RayTune}, with the ASHA scheduler~\citep{li2018massively}. Ultimately, the hyperparameter group that maximizes the online validation accuracy (integrated in PyTorch Lightning) within 5000 validation steps is chosen for the given encoder training case.

\begin{table}[ht]
\centering

\begin{tabular}{@{}l|l|l@{}}
\midrule
Hyperparameter  & Sample Range & Sample Strategy \\ \midrule
start learning rate & $\left[10^{-2}, 10\right]$ & log uniform \\ \midrule
$\lambda$       & $\left[0, 1\right]$ & uniform \\ \midrule
$\tau$, $\tau_1$, $\tau_2$ & $\left[0, 1\right]$ & log uniform \\ \midrule
\end{tabular}

\caption{Hyperparameters sample strategy}
\label {table: Hyperparameter sample}
\end{table}

\textbf{Encoder training.} 
We train each encoder using the LARS optimizer~\citep{LARSOptimizer}, LambdaLR Scheduler in PyTorch, momentum 0.9, weight decay $10^{-6}$, batch size 256, and the aforementioned hyperparameters for 400 epochs on a single A-100 GPU.

\textbf{Image transformation.} The image transformation strategy, including augmentation, is identical to the default transformation strategy provided by PyTorch Lightning.

\textbf{Linear evaluation.}
The linear head is trained using the SGD optimizer with a cosine learning rate scheduler, batch size 64, and weight decay $10^{-6}$ for 100 epochs. The learning rate starts at $0.3$ and ends at $0$.

\textbf{Moco Experiments.} We also tested our method based on MoCo~\citep{he2019moco}. The results are summarized in Table \ref{tab:results-moco}. Here we choose ResNet18~\citep{ResNet} as the backbone and set a temperature of $0.1$ as default. For our simple sum kernel, we set $\lambda=0.8$. The results show that our method outperforms the original MoCo method.

\begin{table}[thb]
\centering
\caption{MoCo Experiment Results on CIFAR-10 and CIFAR-100.}
\label{tab:results-moco}
\resizebox{\textwidth}{!}{%
\begin{tabular}{@{}c|ccc|ccc@{}}
\toprule
\multirow{3}{*}{Method} & \multicolumn{3}{c|}{CIFAR-10} & \multicolumn{3}{c}{CIFAR-100} \\ \cmidrule(lr){2-4} \cmidrule(lr){5-7} 
                        & 200 epochs & 400 epochs    & 1000 epochs   & 200 epochs & 400 epochs & 1000 epochs         \\ \midrule
MoCo (repro.)         & $76.41 \pm 0.12$    & $80.01 \pm 0.15$          & $84.45 \pm 0.08$    & $\mathbf{47.02 \pm 0.11}$ & $52.50 \pm 0.07$ & $57.62 \pm 0.15$            \\
\midrule
Laplacian Kernel        & ${78.09 \pm 0.10}$    & $\mathbf{83.85 \pm 0.09}$          & $\mathbf{88.34 \pm 0.16}$    & $46.12 \pm 0.22$   & $53.44 \pm 0.17$ & $59.10 \pm 0.14$        \\
Simple Sum Kernel & $\mathbf{78.12 \pm 0.15}$   & $83.23 \pm 0.18$ & $87.50 \pm 0.20$ & $46.65 \pm 0.06$ & $\mathbf{53.62 \pm 0.19}$ & $\mathbf{59.83 \pm 0.12}$\\
\bottomrule
\end{tabular}
}
\end{table}



\section{More Experiments on Synthetic Data}


Consider a scenario with $n$ clusters, each containing $k$ vertices. Let the probability of vertices $u$ and $v$ from the same cluster belonging to $\bfpi$ be $p$. Conversely, for vertices $u$ and $v$ from different clusters, let the probability of belonging to $\pi$ be $q$. We generate the graph $\bfpi$ randomly, based on $p$ and $q$. We experiment with values of $k=100$ and $n=6$ for ease of visualization, embedding all points in a two-dimensional space. Each vertex's initial position originates from a normal distribution. In each iteration, we sample a subgraph of $\bfpi$ uniformly, ensuring each vertex has an out-degree of $1$. We then optimize the corresponding vectors using InfoNCE loss with an SGD optimizer and iterate until convergence. Our experimental setup consists of an SGD learning rate of $1$, an InfoNCE loss temperature of $0.5$, and a batch size of $50$. We evaluate two scenarios with different $p$ and $q$ values: $p=1$, $q=0$, and $p=0.75$, $q=0.2$. The results of these experiments are visualized in Figure \ref{fig:vis-spectral-cluster}. The obtained embeddings exhibit the hallmark pattern of spectral clustering of graph $\bfpi$.

\begin{figure}[!tb]
\centering
\subfigure{
\includegraphics[width=1\textwidth]{Figures/cluster_pi.png}
\label{fig:vis-cluster}
}
\subfigure{
\includegraphics[width=1\textwidth]{Figures/noised_cluster_pi.png}
\label{fig:vis-noised-cluster}
}
\caption{Visualizations of the optimization process using InfoNCE Loss on the vectors corresponding to $\bfpi$. Points of identical color belong to the same cluster within $\bfpi$. To showcase the internal structure of $\bfpi$, we randomly select 10 vertices from each cluster to display the edge distribution of $\bfpi$.}
\label{fig:vis-spectral-cluster}
\end{figure}



\end{document}
\endinput