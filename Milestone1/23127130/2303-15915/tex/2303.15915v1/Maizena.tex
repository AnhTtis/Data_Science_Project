% ****** Start of file apssamp.tex ******
%
% This file is part of the APS files in the REVTeX 4.2 distribution.
% Version 4.2a of REVTeX, December 2014
%
% Copyright (c) 2014 The American Physical Society.
%
% See the REVTeX 4 README file for restrictions and more information.
%
% TeX'ing this file requires that you have AMS-LaTeX 2.0 installed
% as well as the rest of the prerequisites for REVTeX 4.2
%
% See the REVTeX 4 README file
% It also requires running BibTeX. The commands are as follows:
%
% 1) latex apssamp.tex
% 2) bibtex apssamp
% 3) latex apssamp.tex
% 4) latex apssamp.tex
%
\documentclass[%
% choose between aip or aps
%aip,
 reprint,
superscriptaddress,
%groupedaddress,
%unsortedaddress,
%runinaddress,
%frontmatterverbose, 
%preprint,
%preprintnumbers,
%nofootinbib,
%nobibnotes,
%bibnotes,
 amsmath,amssymb,
aps,
%pra,
%prb,
%rmp,
%prstab,
%prstper,
%floatfix,
]{revtex4-2}

\usepackage{graphicx}% Include figure files
\usepackage{dcolumn}% Align table columns on decimal point
\usepackage{bm}% bold math
\usepackage{textgreek}%pour mu droit
\usepackage{xcolor}
%\usepackage{hyperref}% add hypertext capabilities
%\usepackage[mathlines]{lineno}% Enable numbering of text and display math
%\linenumbers\relax % Commence numbering lines

%\usepackage[showframe,%Uncomment any one of the following lines to test 
%%scale=0.7, marginratio={1:1, 2:3}, ignoreall,% default settings
%%text={7in,10in},centering,
%%margin=1.5in,
%%total={6.5in,8.75in}, top=1.2in, left=0.9in, includefoot,
%%height=10in,a5paper,hmargin={3cm,0.8in},
%]{geometry}

\newcommand{\anais}[1]{\textcolor{teal}{#1}}

\begin{document}

%This file may be formatted in either the \texttt{preprint} or \texttt{reprint} style. \texttt{reprint} format mimics final journal output. Either format may be used for submission purposes. \texttt{letter} sized paper should be used when submitting to APS journals.

\title{Shear-thickening in presence of adhesive contact forces: the singularity of cornstarch}
%Are cornstarch suspensions a canonical shear-thickening system? 

\author{Anaïs Gauthier}
 \email{anais.gauthier@univ-rennes.fr}
\affiliation{%
 Univ Rennes, CNRS, IPR (Institut de Physique de Rennes) - UMR 6251, F-35000 Rennes, France}%
 
 
 \author{Guillaume Ovarlez}
\affiliation{%
 Univ. Bordeaux, CNRS, Solvay, LOF, UMR 5258, F-33608 Pessac, France}%

\author{Annie Colin}%
\affiliation{%
MIE – Chemistry, Biology and Innovation (CBI) UMR 8231, ESPCI Paris, CNRS, PSL Research University, 10 rue Vauquelin, Paris, France}%

\date{\today}

\begin{abstract}
A number of dense particle suspensions experience a dramatic increase in viscosity with the shear stress, up to a solid-like response. This shear-thickening process is understood as a frictionless/frictional transition between particles that repel each other. Here, we show that the emblematic shear-thickening fluid, cornstarch, does not fit into this framework. Using an original pressure sensor array, we map the normal stresses in the flow. We evidence in cornstarch a unique, stable stress wave which does not appear in a purely repulsive system (calcium carbonate). At high solid fraction and small gap, the stress signal shows a rolling aggregate, which presence is explained by adhesive contacts (associated with a hysteresis in the force profile) between two particles. Our findings offer a new perspective on the shear-thickening process of cornstarch, which is deeply connected to adhesion. As a result, experiments on cornstarch may not be generalized to other shear-thickening fluids.\\

%.
%\textbf{Point of the paper:}\\
%* showing that Cornstarch is not a canonical system for shear-thickening because of adhesion forces between the particles\\
%* evidencing the presence of normal stress waves - which are associated either with a steady band and a jammed zone or a propagating density wave (depending on phi and on the gap height)

\end{abstract}

%\keywords{shear thickening, cornstarch, normal stress measurement, steady stress waves}
 
\maketitle

%\tableofcontents

%\section*{\label{intro} Introduction}


Many industrial processes, such as waste water treatment, oil drilling or mixing pastes require the manipulation of liquids containing a high solid fraction in particles \cite{Blanco:2019, Feys:2008}. %In nature, such materials are also ubiquitous: water or oil saturated sediments, muds, crystal-bearing magmas \cite{}. 
Unfortunately, the flow of dense suspensions is particularly complex: 
%as it strongly depends on the microscopic particle-particle and particle-fluid interactions and dynamics \cite{Clavaud:2017, Comtet:2017, James:2018, Hsu:2021}. 
one of the its most striking features is shear thickening, where the viscosity can increase by more than one order of magnitude with the shear stress \cite{Barnes:1989,Brown:2014,Lootens:2003,Fall:2008,Guy:2015,Wagner:2009,Brown:2010}. In the most extremes case, the material undergoes a liquid-solid transition. It is sometimes detrimental, causing pipe clogging or the breaking of blades in mixing systems \cite{Ovarlez:2020}; but it can also be used to engineer smart fluids with a tunable rheology \cite{Ness:2018, Niu:2020} or able to resist impacts \cite{Mahfuz:2009, Majumdar:2014, deGoede:2019}. 

Understanding the mechanism at the origin of shear-thickening is thus essential. In this field, the work of Wyart and Cates \cite{Wyart:2014, Mari:2014, Lin:2015, Royer:2016, Clavaud:2017} is central. Shear-thickening is explained by the presence of short-ranged \textit{repulsive} forces between the particles. When the normal force exceeds the repulsive force \cite{Comtet:2017}, the particles switch from lubricated to frictional contacts, which leads to an increase of the viscosity. This framework explains many aspects of the average shear-thickening flow of a large number of shear-thickening fluids \cite{Guy:2015, Royer:2016, Lin:2015, Comtet:2017, Clavaud:2017}, even if it does not explain all details of the transition, in particular fluctuations \cite{Hermes:2016, Rathee:2022, Saint-Michel:2018, Ovarlez:2020, Richards:2019} and flow heterogeneities \cite{Boersma:1991, Lootens:2003, Pan:2015, Bossis:2017, Rathee:2017, Saint-Michel:2018, Ovarlez:2020, Rathee:2020} in the discontinuous shear-thickening regime. 

Interestingly, the most emblematic shear-thickening fluid -- a suspension of cornstarch particles-- does not really fit within this framework, despite being regularly used in experiments designed to study the shear-thickening process \cite{Fall:2008, Hermes:2016, Ovarlez:2020, Boyer:2016, Saint-Michel:2018, Richards:2019, Darbois:2020, Niu:2020}.
Indeed, there is no clear repulsion (or attraction) between two particles that are initially separated \cite{Comtet:2017, Galvez:2017}. However, after a forced contact, a strong microscopic \textit{adhesion} (associated with a hysteresis in the force profile between two particles) is measured \cite{Galvez:2017}. %In addition, cornstarch suspensions present a small but measurable yield stress \cite{Fall:2012}. 

While the macroscopic flow of cornstarch does not look so different from other shear-thickening fluids, we show that the flow at a smaller scale is very unusual. We evidence a single, remarkably stable normal stress band moving in the velocity direction. This phenomenon is not observed in stabilized calcium carbonate suspensions, which are purely repulsive as in \cite{Wyart:2014,Mari:2014}. At high solid fraction in particles and small gap widths, the signal switches from a stress wave to a stable phase separation, with a denser (solid) aggregate appearing and rolling. We link its origin and stability to the presence of adhesive forces between the particles at high normal stress. We finally discuss the implications of adhesion on the shear-thickening process of cornstarch, and propose a new mechanism which takes into account dynamic adhesion through the entanglement of the protein chains that cover the particles. 


%The work of Wyart and Cates \cite{Wyart:2014, Mari:2014, Lin:2015, Royer:2016, Clavaud:2017} has been a breakthrough in the understanding of shear thickening. A central point is that it assumes the presence of short-ranged repulsive forces between the particles. When the normal force exceeds the repulsive force \cite{Comtet:2017}, the particles switch from lubricated to frictional contacts, which leads to an increase of the viscosity. This framework explains many aspects of the average shear-thickening flow, but does match the details of the transition. In particular, large temporal fluctuations of the shear rate or the viscosity have been reported \cite{Hermes:2016, Rathee:2022, Saint-Michel:2018, Ovarlez:2020, Richards:2019, Boersma:1991, Lootens:2003, Pan:2015, Bossis:2017}, and are associated with an heterogeneous flow \cite{Rathee:2017, Saint-Michel:2018, Ovarlez:2020, Rathee:2020}. There is no consensus on the nature and the motion of the heterogeneity at this point.

%\smallskip

%Dense suspensions of cornstarch are an emblematic shear-thickening system, used both for scientific outreach \cite{Brown:2014} and to investigate shear thickening \cite{Fall:2008, Hermes:2016, Ovarlez:2020, Boyer:2016, Saint-Michel:2018, Richards:2019, Darbois:2020, Niu:2020}. However, the microscopic properties of these suspensions differ from those of the model shear-thickening systems, as found in the \citet{Wyart:2014} model and in the \citet{Mari:2014} simulations. Indeed, in cornstarch there is no clear repulsion (or attraction) between two particles that are initially separated \cite{Comtet:2017, Galvez:2017}. However, after a forced contact, a strong microscopic \textit{adhesion} (associated with a hysteresis in the force profile between two particles) is measured \cite{Galvez:2017}. In addition, cornstarch suspensions present a small but measurable yield stress \cite{Fall:2012}. Here, we show that the shear-thickening process of the cornstarch suspensions is quite unique, with the formation of a single, remarkably stable normal stress band moving in the velocity direction. This phenomenon is not observed in stabilized calcium carbonate suspensions, which are purely repulsive as in \cite{Wyart:2014,Mari:2014}. We study the nature of the signal, and show that it switches from a stress wave to a stable phase separation, with a denser (solid) aggregate apprearing and rolling. We link its origin and stability to the presence of adhesive forces between the particles at high normal stress, and discuss its implications on the shear-thickening process of cornstarch.


\section{Experiment\label{Experiment}}

We characterize the flow of cornstarch suspensions by combining classical rheology measurements with a mapping of the normal stresses at the millimeter scale, using a in-house sensor array. As shown in Figure~\ref{figure1}a, the sensor array is placed on the bottom plate of a conventional rheometer (DHR-2, TA instruments) in a parallel-plate geometry with a gap height $h$ varied between 0.5~mm and 3~mm. It consists of 25 regularly spaced piezo-capacitive sensors, built together in a 5 $\times$ 5 array of total width 4~cm. The details of the sensor array are shown in Figure \ref{figure1}a. It is a sandwich of three main layers: \textit{i)} an electrode array (the bottom part), which sets the position and the size of the measurement points, \textit{ii)} a 25~\textmugreek m thick Mylar grid which separates the top and the bottom parts and \textit{iii)} a soft piezo-capacitive layer (made from a solid polymeric foam developed previously in our lab \cite{Pruvost:2019}), which is covered with a soft silver electrode. When a pressure is applied on the foam, it partially collapses and bends onto the grid, which impacts its local capacitance. By measuring the capacitance between the top and the bottom electrodes, we can thus map the vertical component of the stress in the fluid, with a spatial resolution of 4.5 $\times$ 4.5~mm$^2$ (the size of the electrodes), a precision of 2~Pa and a temporal resolution of typically 50~Hz \cite{Gauthier:2021}.

%%%%%%%%%%%%%%%%%%%%%%%%% FIGURE 1 %%%%%%%%%%%%%%%%%%%%%%%%
\begin{figure}[!ht]
\centering
\includegraphics[width=0.99\columnwidth]{Figures/figure1.eps}
\caption{\label{figure1} 
\textbf{a.} Schematic of the setup for a normal stress measurement. A piezo-capacitive pressure sensor array is placed at the bottom of a parallel-plate rheometer. The geometry has a diameter of 4~cm and the gap thickness is typically 1~mm. The sensor is a sandwich of many layers (see legend). Its core is a soft house-made dielectric foam, which deforms when a pressure is applied. This impacts its local capacitance, which is measured in 25 different points. \textbf{b.} Top view of the experiment: each square corresponds to a different pressure sensor. The blue circle indicates the geometry, with a diameter of 4~cm. Here, the pressure is measured in the nine positions below the geometry as shown in red.}
\end{figure}
%%%%%%%%%%%%%%%%%%%%%%%%%%%%%%%%%%%%%%%%%%%%%%%%%%%%%%%%%%%%%%%

\smallskip

During an experiment, the pressure of the nine central sensors (in red in Figure \ref{figure1}b), which are fully covered by the geometry, is recorded. In a typical experiment, the shear stress $\sigma$ is imposed: it is increased by constant steps of duration 50 to 250~s, between 0.05 and 1000~Pa. The rheometer records the shear rate $\dot{\gamma}$ and the average vertical stress $P_{\rm m}$ through the built-in force sensor. Simultaneously, the sensor array on the bottom plate records the local normal stress $P$ as a function of time $t$. All these data are aggregated to obtain both a global and a local view of the shear-thickening process.

\smallskip

The suspension used here is a dispersion of cornstarch particles (Sigma Aldrich) in a density-matched liquid made of water and salt (Cesium Chloride, 55\% in weight). Cornstarch particles are polydisperse, with a mean diameter of 15 $\pm$ 7~\textmugreek m (see Supplementary Figure 1a). The particles being porous \cite{Han:2017}, we follow Refs. \cite{Saint-Michel:2018,Ovarlez:2020} and characterize the suspensions using the weight percentage in cornstarch $\phi_{\rm w}$ instead of a volume fraction. In our experiments, $\phi_{\rm w}$ is varied between 34\% and 42\%. 



\section{Results
\label{Results}}

The flow curve of a cornstarch suspension with weight fraction $\phi_{\rm w}$ = 40 \% is presented in Figure \ref{figure2}a. The suspension is strongly shear thinning at low shear rate $\dot{\gamma} <$ 3~s$^{-1}$, which is consistent with a small yield stress of order 0.1~Pa. Shear thickening is observed for $\dot{\gamma} > 5$~s$^{-1}$. It is first continuous, with a slow increase of the viscosity $\eta$ with $\dot{\gamma}$. It becomes discontinuous for $\dot{\gamma}$ = $\dot{\gamma}^* = 40$~s$^{-1}$ (corresponding to $\sigma$ = $\sigma^* = 20$~Pa): the suspension viscosity increases by more than an order of magnitude at a constant shear rate. Simultaneously, the mean vertical stress $P_{\rm m}$ measured by the force sensor of the rheometer (Figure \ref{figure2}b) switches from slightly negative to positive, and increases with the applied stress. The flow curves presented in Figs.~\ref{figure2}a and b are characteristic of all cornstarch suspensions we considered here, with 35\% $< \phi_{\rm w} < 42$ \% (Supplementary Figure 2a). At these solid fractions, fluctuations of the macroscopic normal stress also appear in the discontinuous shear thickening region (see Supplementary Figure 2b and c) and their amplitude increases with the applied stress.

%%%%%%%%%%%%%%%%%%%%%%%%% FIGURE 2 %%%%%%%%%%%%%%%%%%%%%%%%
\begin{figure*}[!ht]
\centering
\includegraphics[width=0.9\textwidth]{Figures/figure2.eps}
\caption{\label{figure2} 
\textbf{a.} Flow curve (viscosity $\eta$ as a function of the shear rate $\dot{\gamma}$) of a cornstarch suspension with weight fraction $\phi_{\rm w} = 40$\%. \textbf{b.} Mean vertical pressure $P_{\rm m}$ as measured by the force sensor of the rheometer. \textbf{c.} Local normal stress as a function of time $t$, for increasing shear stresses $\sigma$. The inset of b. shows the position of the two measurement points in the sensor array: the red and blue sensors are placed symmetrically with respect to the center of the geometry. The position of the measurements in the flow curves are indicated with colored triangles.}
\end{figure*}
%%%%%%%%%%%%%%%%%%%%%%%%%%%%%%%%%%%%%%%%%%%%%%%%%%%%%%%%%%%%%%%

Using the sensor array, the flow is probed at the millimeter scale. For clarity, we present in Figure \ref{figure2}c the signal of two sensors only, in red and blue, placed symmetrically under the geometry (see the schematic in the inset of Fig.~\ref{figure2}b) as a function of time. The signal measured by all nine sensors in the continuous shear thickening region (Supplementary Movie 1) and in the discontinuous shear thickening region (Supplementary Movie 2 and 3) are available in the Supplementary Materials. In the shear-thinning and in the continuous shear-thickening regions of the flow curves ($\sigma = 15$~Pa, top panel), the local normal stress matches the mean normal stress: the pressure measured by each of the sensors is constant and slightly negative. In the discontinuous shear thickening regime ($\sigma = 80$~Pa and $\sigma = 130$~Pa), however, the two sensors present pressure peaks of short duration in phase opposition. As evidenced in Supplementary Movie 2, it corresponds to a unique pressure wave that rotates within the geometry in the velocity direction. The peak pressure is typically 4 to 5 times higher than the mean normal stress measured by the rheometer, and, between two events, the pressure is close to zero and slightly negative. When the shear stress is increased, for example from $\sigma$ = 80~Pa (middle panel) to $\sigma$ = 130~Pa (bottom panel), the amplitude of the stress signal increases, but we still observe a unique wave.

It is interesting to note that the regular local normal stress wave is not evidenced by the force sensor of the rheometer. As evidenced in Supplementary Figure 2c and d, the mean normal stress measured by the rheometer is neither constant or periodic: it presents some noise around a constant value that is at least 2 times smaller than the local peak pressure. When increasing the applied stress, both the mean normal stress $P_m$ and the amplitude of the noise increase, while, at the small scale, the amplitude of the normal stress peak increases.

This single, stable and regular stress wave is observed in all cornstarch suspensions with a solid fraction $\phi_{\rm w} >$ 35\% and for gap widths varied between 0.5~mm and 2~mm. The regularity of the signal is lost in the most "extreme" situations of large gap width ($h \ge$ 2~mm) or small solid fraction (35\% $\leq \phi_{\rm w} <$ 37\%), as shown in Supplementary Figure 3. Finally, the signal completely disappears for $\phi_{\rm w} \leq $ 34\%, when the shear thickening is continuous (Supplementary Figure 4). 


The pressure wave being very regular, we now focus on its velocity. In Figure \ref{figure3}a we plot the angular velocity of the signal $\Omega_a$ (compared with the angular velocity $\Omega$ of the geometry) as a function of the applied shear stress $\sigma$. Interestingly, $\Omega_a/\Omega$ is independent of $\sigma$: it keeps a constant value $\Omega_a/\Omega = 1.6 \pm 0.1$ for $h$ = 1~mm (in red) and $\Omega_a/\Omega = 0.4 \pm 0.1$ for $h$ = 0.5~mm (in blue). What is remarkable here is that the two signals obtained for $h = 1$~mm and $h = 0.5$~mm, that could seem similar from a distance, surely correspond to very different physical situations. Indeed, the normal stress wave is faster than the geometry for $h$ = 1~mm ($\Omega_a > \Omega$): it is thus very unlikely that it originates from a solid object. However, for $h = 0.5$~mm, $\Omega_a \simeq 0.5\, \Omega$: the angular velocity of the signal is close to the mean velocity of the fluid within the geometry, which could indicate the presence of a rolling solid aggregate. The shape of the signal is also different in both situations: when $\Omega_a/\Omega > 1$, the signal is peaked as in Figure \ref{figure3}b. For $\Omega_a/\Omega < 1$ (Figure \ref{figure3}c, $h$ = 0.5~mm), the signal is more noisy and the width of the peak is larger.

%%%%%%%%%%%%%%%%%%%%%%%%% FIGURE 3 %%%%%%%%%%%%%%%%%%%%%%%%
\begin{figure}[!ht]
\centering
\includegraphics[width=0.99\columnwidth]{Figures/figure3.eps}
\caption{\label{figure3} \textbf{a.} Non-dimensional velocity $\Omega_a/\Omega$ of the stress signal in a 40\% cornstarch suspension, as a function of the applied shear stress $\sigma$. The green circles correspond to a gap height of $h$ = 0.5~mm  and the red triangles to $h$ = 1~mm. \textbf{b.} Normal stress signal $P$ measured by 2 sensors placed symmetrically as a function of time $t$ measured for $h$ = 1~mm, $\sigma$ = 100~Pa. \textbf{c.} Normal stress signal $P(t)$ for $h$ = 0.5~mm, $\sigma$ = 100~Pa. \textbf{d.} Non-dimensional velocity $\Omega_a/\Omega$ of the stress signal in a 40\% cornstarch suspension, as a function of $h$. \textbf{e.} Velocity of the stress signal as a function of the solid fraction $\phi_{\rm w}$ ($h$ = 1 mm).}
\end{figure}
%%%%%%%%%%%%%%%%%%%%%%%%%%%%%%%%%%%%%%%%%%%%%%%%%%%%%%%%%%%%%%%


To get a better picture of this phenomenon, we varied systematically the gap thickness $h$ (0.5~mm  $<h<$ 2~mm, figure \ref{figure3}d) and the solid weight fraction $\phi_{\rm w}$ of the suspensions (36\% $<\phi_{\rm w}<$ 42\%, figure \ref{figure3}e). In both cases, we evidence an abrupt transition between a slow pressure wave (with $\Omega_a/\Omega$ constant and close to 1/2 of the velocity of the geometry) and a fast wave (with a signal faster than the geometry: $\Omega_a/\Omega > 1$). The slow signal is measured when the suspension is confined ($h < $1~mm ) and dense ($\phi_{\rm w} > 40\%$), while the fast one is seen at lower solid fraction and large gap. The transition between the two regimes is sharp: for $\phi_{\rm w} = 40\%$, the velocity of the signal is multiplied by 3 when the gap width increases from 0.95~mm  to 1~mm  (figure \ref{figure3}d). The signal velocity also decreases by a factor $\simeq 3$ between $\phi_{\rm w} = 40$\% and $\phi_{\rm w} = 41$\% for $h$ = 1~mm. It is also associated with a visible change in the shape of the signal recorded by the sensors, as evidenced in Supplementary Figure 5: similarly to what is shown in Figure \ref{figure3}c, the "slow" pressure signal is more noisy and of longer peak extent than the fast one.


\smallskip


In order to know whether the presence of a stable normal stress wave is specific to cornstarch or not, we considered another shear-thickening fluid: a calcium carbonate suspension (Eskal 500, KSL Staubtechnik GmbH) in a water-glycerol mixture. We chose this system as it shares similarities with cornstarch: it is an industrial powder, the particles have a rhomboidal-like shape and they are slightly polydisperse with a mean size of 4 \textmugreek m (see picture in Supplementary Figure 1b). Here, following Richards~\textit{et al.} \cite{Richards:2021}, the particles are dispersed in a 50-50\% water-glycerol mixture containing 0.05 w/w\% of polyacrylic acid. 
The poly-acrylic acid (PAA) removes all adhesive contacts between the grains: in presence of PAA, they interact as purely frictional hard particles with a short range repulsive force \cite{Richards:2021}. The calcium carbonate suspensions are thus close to the model shear-thickening fluid, as defined in the \citet{Wyart:2014} model and in the \citet{Mari:2014} simulations. As shown in Figure \ref{figure4}a, they actually behave as model systems. Indeed, contrary to cornstarch, and as in simulations of an ideal hard sphere system \cite{Mari:2014}, calcium carbonate is not shear thinning and there is a clear low-viscosity plateau. We also evidence a two-branch shear thickening, with a high-viscosity plateau. In figure \ref{figure4}b we report the relative viscosity $\eta_r = \eta/\eta_0$ in the low (in blue) and high viscosity plateaus, which we fit to a Krieger-Dougherty model \cite{Krieger:1959} $\eta_r = \left(1-\frac{\phi}{\phi_{\rm m/rcp}}\right)^{-\beta}$, with $\beta$ = 2.5. From this, we can extract the low-shear jamming volume fraction ($\phi_m$ = 0.48) and the high-shear jamming volume fraction ($\phi_{\rm rcp}$ = 0.62). We expect that calcium carbonate is thus in the continuous shear-thickening regime for $\phi < \phi_m$ (filled circles in Figure \ref{figure4}a and c) and in the discontinuous shear-thickening regime at higher solid fractions (open circles). This is further confirmed in figure \ref{figure4}c, when plotting the mean normal stress. For $\phi <$ 47\%, $P_{\rm m}$ is positive and increases with $\dot{\gamma}$. For $\phi >$ 47\%, $P_{\rm m}$ is first positive and suddenly decreases (in the region of the apparent high viscosity plateau). Following \cite{Richards:2021} and \cite{Dhar:2019}, we interpret the drop away from the model in the high viscosity plateau in Figure \ref{figure4}b along with the sudden decrease of the normal stress as a fracturing of the jammed suspension for $\phi \geq 47$\% at high shear rate.


%%%%%%%%%%%%%%%%%%%%%%%%% FIGURE 4 %%%%%%%%%%%%%%%%%%%%%%%%
\begin{figure*}[!ht]
\centering
\includegraphics[width=0.99\textwidth]{Figures/figure4.eps}
\caption{\label{figure4} \textbf{a.} Flow curve of calcium carbonate suspensions, for volume fractions $\phi$ varied between 35\% and 52\%. The full circles correspond to the continuous shear-thickening regime, and the empty circles to the discontinuous shear-thickening regime. \textbf{b.} Relative viscosity $\eta_r = \eta/\eta_0$ in the low (in blue) or high (in red) plateau (with $\eta_0$ = 23 mPa\,s the viscosity of the suspending fluid). The blue and the red curves are fits of Krieger-Dougherty form \cite{Krieger:1959} $\eta_r = \left(1-\frac{\phi}{\phi_{\rm m/rcp}}\right)^{-\beta}$ with $\beta$ = 2.5, $\phi_m$ = 0.48, $\phi_{rcp}$ = 0.62. \textbf{c.} Mean vertical stress $P_{\rm m}$. The color code is the same as in $\textbf{a}$. \textbf{d.} Stress signal measured by the sensor array, of 2 sensors placed symmetrically below the geometry. The position of the measurements in the flow curves are indicated with the colored squares.}
\end{figure*}
%%%%%%%%%%%%%%%%%%%%%%%%%%%%%%%%%%%%%%%%%%%%%%%%%%%%%%%%%%%%%%%



Figure \ref{figure4}c shows the local normal stress measured by two sensors placed symmetrically below the geometry (see insert of Fig. \ref{figure4}c). Interestingly, contrary to cornstarch suspensions, in this system of purely repulsive particles, the local normal stress does not vary with time. At the millimeter scale, we do not detect the sign of any heterogeneous flow at all the solid fractions and shear stresses we considered. Figure \ref{figure4}d gives some examples of the signal that is recorded for $\phi = 45$\% in the continuous shear-thickening regime (pink square, top panel), and for $\phi = 50$\% in the discontinuous shear-thickening regime (light blue, middle panel). In both cases, the pressure is positive and of the order of 10~Pa, in good agreement with the mean normal stress $P_{\rm m}$. In the fracturing regime (bottom panel, dark blue), the local pressure $P$ is not uniform: the two sensors measure a different signal, which varies with a timescale of $\sim 10$~s. This is a bit larger than the period of rotation of the geometry, and could indicate slow rearrangements of the fracture within the jammed suspension.




\section{Discussion}

The absence of an heterogeneous flow in the shear thickening of calcium carbonate suspensions can fit within the current framework of shear-thickening models. Indeed, the Wyart and Cates model gives a general picture of the flow, but it does not describe what happens dynamically and at the small scale during the shear-thickening regime. Theory and simulations predict that homogeneous or inhomogeneous flows are possible \cite{Olmsted:2008, Hermes:2016, Chacko:2018} (with unstable bands moving in the vorticity direction). 

What is more curious is the pressure wave that we detect in cornstarch. 
In previous experiments, large temporal fluctuations of the shear rate or the viscosity have been reported in cornstarch \cite{Hermes:2016, Rathee:2022, Saint-Michel:2018, Ovarlez:2020, Richards:2019} or other shear-thickening fluids \cite{Boersma:1991, Lootens:2003, Pan:2015, Bossis:2017}. They have been associated with an heterogeneous flow \cite{Rathee:2017, Saint-Michel:2018, Ovarlez:2020, Rathee:2020}. However, the signal that we detect here combines many unusual properties: we observe a single wave in most situations, it appears systematically (and disappears out of the shear-thickening regime) and it moves at a constant and regular speed in the \textit{velocity} direction. Interestingly, the local flow that we evidence matches very nicely recent experiments by Ovarlez \textit{et al} \cite{Ovarlez:2020} and Rathee~\textit{et al.} \cite{Rathee:2017}. Using X-ray radiography in a 41\% cornstarch suspension, Ovarlez~\textit{et al.} evidenced a single and regular density wave spanning over the whole geometry and moving at a constant speed $\Omega_a \simeq 0.3 \Omega$ in the velocity direction. This is consistent with our experimental observations: using the same suspension and the same gap width ($h$ = 0.5~mm; in a parallel-plate geometry instead of a Couette cell), we measure a normal stress signal with a velocity $\Omega_a$ = 0.40 $\Omega$. Our observations are also consistent with the recent results of \citet{Rathee:2022}, who show that the local shear stress is inhomogeneous in cornstarch, with persistent regions of high shear stress propagating in the flow direction, with a velocity $\Omega_a \simeq \Omega$. Even if the cornstarch suspension used is different from ours, this could correspond to the normal stress wave that what we see in the low solid fraction and high gap regime.


\smallskip

By combining our results with Refs. \cite{Ovarlez:2020} and \cite{Rathee:2017}, we can get a better picture of the physical nature of the signal that we detect. Indeed, the "slow" wave of \cite{Ovarlez:2020} is associated with the motion of a high solid fraction object. From its velocity, we deduce that this denser object is most likely a log-rolling floc, associated with a local jamming of the suspension. Indeed, a rolling solid would naturally move at a velocity $\Omega_a = 0.5\,\Omega$ or more slowly if it partially slides: this is exactly what we observe in our experiments (Figure \ref{figure3}) and what is observed in \cite{Ovarlez:2020}. The presence of a rolling floc would explain why the signal is noisy, as the aggregate would push the sensors irregularly. From the width of the signal, we infer that the jammed zone would occupy typically 30\% of the sample volume in the geometry. In contrast, at low solid fraction $\phi_{\rm w} \leq 40$\% or large gap height, the signal is faster than the mean velocity of the fluid, which is not consistent with the presence of a solid aggregate. We infer that, in this region, the signal we record is a normal stress wave. These results are consistent with the signal detected by Rathee~\textit{et al.} \cite{Rathee:2022}: by using a clever combination of boundary stress microscopy and of particle tracking, they evidenced (at a smaller scale than our experiment) the presence of stress waves associated with propagating fronts moving in the velocity direction, at a velocity close to the one of the geometry.


\smallskip

The transition between a stress wave and a solid floc that we evidence here is highly unusual. To explain what is observed in cornstarch, we need to get a better picture of the suspension behavior, both at the microscopic and macroscopic scales.


% In addition, one might not expect a stable phase separation to appear in a non-brownian system. Indeed, the normal stress balance at the interface should make potential bands disappear \cite{Hermes:2016}; even if a recent theoretical and numerical article suggests the potential existence of long-lived unsteady bands, moving in the vorticity direction \cite{Chacko:2018}.}
% Our hypothesis is that the fluctuations are stabilized by adhesive forces between cornstarch particles: adhesion could maintain locally denser or jammed zones, resisting against local stresses. 

\smallskip

Cornstarch particles are made of two biopolymers (amylose and amylopectin) that are partially soluble in water \cite{Buleon:1998,Green:1975}. The microscopic interaction between two cornstarch particles has been probed using quartz-tuning fork atomic force microscopy \cite{Comtet:2017} and classical atomic force microscopy \cite{Galvez:2017}. Before contact, the interaction force between the starch grains is very small \cite{Comtet:2017,Galvez:2017}; repulsive forces, if any, are smaller than 2~nN. However, once the particles are put into contact (with an indentation force of the order of 100 nN), the force profile shows an hysteresis, corresponding to an adhesive force between the particles of the order of $10 - 20$~nN in water \cite{Galvez:2017}. In addition, a series of "pulling events" have been evidenced when separating the particles \cite{Galvez:2017}. Such events are seen when high density dangling polymers (in a poor solvent) disentangle, and corresponds to the breaking of bonds during the retraction \cite{Yu:2015}. These results indicate that cornstarch particles in water are most likely slightly swollen particles \cite{Belitz:2008} surrounded by a collapsed polymer brush.

\smallskip

Adhesion which has been evidenced for cornstarch in quasi-static experiments, can also promote a phase separation at higher shear rates. Indeed, as shown by Varga~\textit{et al.} \cite{Varga:2019}, dilute suspensions of weakly adhesive particles can form log-rolling flocs when they are sheared between parallel planes. The aggregates are only stable when the viscous drag force acting on them (which tends to break them) is smaller than the total attractive force between the particles within each floc. In concrete terms, a stable phase separation in a system of adhesive particles is only possible at low enough shear rate and gap height \cite{Varga:2019}. This is reminiscent of what we observe in the shear thickening of cornstarch (Figure \ref{figure3}d). To verify if a similar mechanism could  explain the apparition of solid aggregates in our system, we apply the criterion suggested in ref \cite{Varga:2019}: rolling flocs are present only when the Mason number $\text{Mn} = \frac{6\pi\eta\dot{\gamma}a^2}{F}$ (with $\eta$ the viscosity of the suspending fluid and $a$ the mean size of the particles), which compares the destabilizing viscous force to the stabilizing contact force is smaller than $2.5\left(\frac{h}{a}\right)^{-1.4}$.


%%%%%%%%%%%%%%%%%%%%%%%%% FIGURE 6 %%%%%%%%%%%%%%%%%%%%%%%%
\begin{figure}[!ht]
\centering
\includegraphics[width=0.9\columnwidth]{Figures/figure6.eps}
\caption{\label{figure6} Phase diagram showing the two different normal stress signals (in red, the signal associated with a potential aggregate and in green, with a normal stress wave) as a function of the shear rate $\dot{\gamma}$ and the gap width $h$ (compared with the particle size $a$ = 15 \textmugreek m. The dotted line corresponds to the prediction of Ref. \cite{Varga:2019} on the stability of log-rolling flocs, without any ajustable parameter.}
\end{figure}
%%%%%%%%%%%%%%%%%%%%%%%%%%%%%%%%%%%%%%%%%%%%%%%%%%%%%%%%%%%%%%%

To compare our results with the prediction of Varga \cite{Varga:2019}, we present in Figure \ref{figure6} a phase diagram of the normal stress signal in the shear-thickening regime as a function of dimensionless the gap height $h/a$ (with $a$ the particle size) and the shear rate $\dot{\gamma}$. In Figure \ref{figure6}, the red dots correspond to a slow and noisy signal (a potential aggregate) and the green dots to a fast and peaked pressure wave (a potential stress wave). The dotted gray line corresponds to the model (with $F$ = 20 nN and $a$ = 15 \textmugreek m). Remarkably, it is very well fitted to our experimental measurements, with the same prefactor as in Ref. \cite{Varga:2019}, \textit{i.e.}, without any fitting parameter. This is a strong indication that the adhesive contact forces between the cornstarch particles are also dominant at high stress, and that they are capable of stabilizing jammed aggregates in the shear-thickening regime. 

\smallskip

The manifestation of adhesion in cornstarch, which appears both in quasi-static experiments and in the shear-thickening regime seem to indicate that adhesive forces are present in every part of the flow curve. Thus, cornstarch appears as a singular shear-thickening fluid: it does not fit into the classical Wyart-Cates model, which needs repulsive short-range forces between the particles to explain the shear thickening. Another interpretation has thus to be found to explain the flow curve of cornstarch.

Let us first consider the yield stress: contrary to what is suggested by \citet{Galvez:2017}, the adhesion force measured by the AFM cannot explain the yield stress in cornstarch particles dispersed in water. Indeed, if the adhesion force $F_a \sim 10$ nN was at the origin of the yield stress, one would expect a yield stress $\tau_y\sim F_a/(\pi (a/2)^2) \simeq 60$~Pa (with $a=15$~\textmu m the typical diameter of the particles). This does not correspond to the case of particles dispersed in water, where $\tau_y$ is of order 0.1~Pa. This yield stress would be consistent with an attractive force of order 0.03~N, which is far below the sensitivity of the microscopic force measurements and is thus not inconsistent with the experimental measurements of \citet{Comtet:2017,Galvez:2017}.

Dealing with shear thickening, we note that the critical shear stress $\sigma^* = 20$~Pa cannot derive from a repulsive force: indeed, the maximum possible repulsive force, if any, is of order 2~nN in the experimental data~\cite{Comtet:2017}, which would imply a critical stress of order 1~Pa if it was associated to a shear-thickening transition as described by \citet{Wyart:2014}. 
Instead, we suggest that shear thickening occurs when the local force applied onto particles is sufficient to force adhesion between the particles; this would correspond to the force required for the interpenetration of the particle polymer brushes. In this regime, there is a competition between the time for the polymer brushes to disentangle and the contact time between the particles. When the first is larger than the latter, the system thickens. We expect the polymer brushes to be more and more entangled as the applied stress increases, so that they resist more when they are pulled apart: this would explain the discontinuous shear thickening observed experimentally.




%\color{black} If we reverse the analysis, an adhesion force (or an attractive force) of order 0.1~nN would be expected in order to explain the observed yield stress for cornstarch in water, which is much below what can be measured with the AFM. As stated above, the origin of adhesion is likely to be found in dangling polymers. In such case, by contrast with colloidal attractive forces, the dynamics of disentanglement is expected to play a role: when the particle separation velocity tends towards zero, the adhesion force should vanish (\color{blue} Un peu perturbant : on discute de l'adhesion à gamma point = 0 juste avant. Dire qu'il y a deux types d'adhésion ?) \color{red} This is why the material flows at low shear rate and low shear stress.


% \textbf{\color{blue} LA JE SUIS MAL A L'AISE : QUE SE PASSE-T-IL ENTRE 100 ET 1000 Pa ??? ON EST ENCORE EN DST, PQ L'ADHESION JOUE-T-ELLE ENCORE UN ROLE ? LA FORCE PEUT ELLE ETRE PLUS ELEVEE SI ON TIRE PLUS VITE ???  On the other hand, the discontinuous thickening transition is observed at stresses of order 20~Pa in Fig.~\ref{figure2}; this is thus in this regime that the observed adhesion force is expected to play a significant role. As an example, it is probably at the origin of the enhanced radial migration of particles \cite{Fall:2015} observed in this system.}

%At a larger scale, adhesion dramatically impacts the rheology of cornstarch suspensions. In particular, it enables shear localization in a Couette cell at low angular velocities, which is the sign of a small but measurable yield-stress of 0.3~Pa \cite{Fall:2012}. This is fully compatible with the shear-thinning behaviour of the suspension at intermediate shear rates (Figure \ref{figure2}). The contact forces also impact the fluid rheology, for example by enhacing the radial migration of particles \cite{Fall:2010}.}



\medskip
%\section*{Conclusion}


Cornstarch and calcium carbonate suspensions seem similar at first glance: both are made of non-Brownian, rhomboidal-shaped particles with similar size of the order of 10 \textmugreek m. At the scale of the rheometer, they are both shear-thickening. However, using the normal stress sensors developed in our laboratory, we show that shear thickening is associated with fundamentally different flows at the millimeter scale. One the one hand, calcium carbonate, which is a purely repulsive system remains homogeneous at all shear-stresses. On the other hand, the sensors evidence in cornstarch an regular normal stress signal, which corresponds either to a stress wave (at low solid fraction and large gap) or to a solid aggregate (at high solid fraction and small gap) rolling between the plates. The transition can be explained by taking into account the presence of adhesive forces between the particles, following ref. \cite{Varga:2019}.

\smallskip

Our experiments thus evidence the central role played by adhesive forces in cornstarch. Adhesion not only allows a partial phase transition and the stabilization of jammed aggregates, but must be at the origin of the shear-thickening process in this system. Our work shows that cornstarch is far from being a canonical system and it should be used with care in the study of shear thickening, especially when it is compared with theoretical models and simulations. 

Going further, we evidence that two fundamentally different microscopic mechanisms (the transition between lubricated and frictional contacts for repulsive particles, or adhesive forces) can both lead to discontinuous shear thickening and apparently similar macroscopic flow curves.
This highlights the considerable importance of the measurement of microscopic forces and dynamics between particles to understand the details of the flow at all scales.


\begin{acknowledgments}
 We thank Vikram Rathee, Jeffrey Urbach, Daniel Blair and Joia Miller for fruitful discussions, and Mickaël Pruvost for his help in the fabrication of the sensors and the acquisition system.
\end{acknowledgments}






\newpage

\nocite{*}
% The \nocite command causes all entries in a bibliography to be printed out
% whether or not they are actually referenced in the text. This is appropriate
% for the sample file to show the different styles of references, but authors
% most likely will not want to use it.

\bibliography{biblioMaizena}% Produces the bibliography via BibTeX.

\end{document}
%
% ****** End of file apssamp.tex ******
