%% ****** Start of file template.aps ****** %
%%
%%
%%   This file is part of the APS files in the REVTeX 4 distribution.
%%   Version 4.0 of REVTeX, August 2001
%%
%%
%%   Copyright (c) 2001 The American Physical Society.
%%
%%   See the REVTeX 4 README file for restrictions and more information.
%%
%
% This is a template for producing manuscripts for use with REVTEX 4.0
% Copy this file to another name and then work on that file.
% That way, you always have this original template file to use.
%
% Group addresses by affiliation; use superscriptaddress for long
% author lists, or if there are many overlapping affiliations.
% For Phys. Rev. appearance, change preprint to twocolumn.
% C\hoose pra, prb, prc, prd, pre, prl, prstab, or rmp for journal
%  Add 'draft' option to mark overfull boxes with black boxes
%  Add 'showpacs' option to make PACS codes appear

%\documentclass[aps,prl,twocolumn,showpacs,superscriptaddress,groupedaddress]{revtex4}  % for review and submission
\documentclass[aps,preprint,showpacs,superscriptaddress,groupedaddress]{revtex4}  % for double-spaced preprint
\usepackage[T1]{fontenc}
\usepackage[english]{babel}
\usepackage{graphicx}  % needed for figures
\usepackage{dcolumn}   % needed for some tables
\usepackage{bm}        % for math
\usepackage{amssymb}   % for math
\usepackage{xcolor}

%\usepackage{setspace}


% avoids incorrect hyphenation, added Nov/08 by SSR
\hyphenation{ALPGEN}
\hyphenation{EVTGEN}
\hyphenation{PYTHIA}

\begin{document}


\title{Influence of interparticle adhesion on the shear thickening of cornstarch\\
Supplementary materials}


\begin{center}
\large 
\textbf{Influence of interparticle adhesion on the shear thickening of cornstarch\\}
\textbf{Supplementary Materials}

\end{center}

\vspace{0.5 cm}

\section{Supplementary Movies}

\noindent \textbf{Supplementary Movie 1}. Normal stress signal recorded by the sensors in a 40\% cornstarch suspension, just before the shear-thickening regime (applied stress $\sigma$ = 10 Pa, gap width $h$ = 1 mm). The movie shows the pressure recorded by the 9 central sensors. In this region of the flow curve, only a residual noise is observed.

\smallskip

\noindent \textbf{Supplementary Movie 2}. Normal stress signal recorded by the sensors in a 40\% cornstarch suspension, in the shear-thickening regime ($\sigma$ = 100 Pa, $h$ = 1 mm). The movie evidences the presence of a unique normal stress wave that rotates in the velocity direction, and pushes the sensors alternatively with a pressure $P \simeq 120$ Pa. Here, the signal moves typically 1.7 times faster than the geometry.

\smallskip

\noindent \textbf{Supplementary Movie 3}. Local normal stress signal in a 41\% cornstarch suspension, in the shear-thickening regime (applied stress $\sigma$ = 150 Pa, gap width $h$ = 1 mm). In this region, the signal moves at a velocity $\Omega_a \simeq 0.5\,\Omega$, which is consistent with the flow of a solid aggregate.


\newpage 

\section{Supplementary Figures}


\begin{figure}[ht!]
\includegraphics[width=0.6\textwidth]{Supplementary/Supplementary_Figure1.eps}
\caption{\label{SuppFig1} Scanning electron microscopy images of the particules studied. \textbf{a.} Cornstarch particles (Sigma Aldrich). \textbf{b.} Calcium carbonate particles (Eskal 500, KSL Staubtechnik GmbH).}
\end{figure}


\begin{figure}[ht!]
\includegraphics[width=0.95\textwidth]{Supplementary/Supplementary_Figure2.eps}
\caption{\label{SuppFig2} \textbf{a.} Viscosity $\eta$ of cornstarch suspensions as a function of the shear rate $\dot{\gamma}$, for varying weight concentrations of cornstarch. \textbf{b.} Vertical stress $P_{\rm m}$ measured by the axis of the rheometer in a parallel plane geometry as a function of $\dot{\gamma}$. \textbf{c.} Mean normal stress $P_{\rm m}$ (measured by the force sensor of the rheometer) as a function of time $t$, for an applied stress $\sigma$ = 100 Pa. The data corresponds to the green square in panel b). \textbf{d.} Local normal stress $P$ measured at the same time by two sensors placed symmetrically below the geometry.}
\end{figure}



%\begin{figure}[ht!]
%\includegraphics[width=0.8\textwidth]{Supplementary/Supplementary_Figure3.eps}
%\caption{\label{SuppFig3} Shape of the normal stress signal $P$ as a function of time $t$ (compared with the period of rotation of the signal $T_{\rm a}$, for varying shear stress $\sigma$. The fluid is a cornstarch suspension with solid fraction $\phi_{\rm w} = 40\%$ \textbf{a.} For a gap height $h$ = 1~mm. \textbf{b.} For a gap height $h$ = 0.5~mm.}
%\end{figure}


\begin{figure}[ht!]
\includegraphics[width=0.6\textwidth]{Supplementary/Supplementary_Figure3.eps}
\caption{\label{SuppFig3} \textbf{a.} Normal stress signal $P$ as a function of time $t$ for a 40\% cornstarch suspension and a gap width of 2~mm. The inset indicates the position of the 4 sensors and the direction of motion of the geometry. \textbf{b}. Pressure signal obtained for a 35\% cornstarch suspension, sheared with a constant stress $\sigma$ = 100 Pa. The gap height is equal to 1~mm. The red and blue curves are associated to two sensors placed symmetrically under the geometry, as in \textbf{a}.}
\end{figure}

\begin{figure}[ht!]
\includegraphics[width=0.99\textwidth]{Supplementary/Supplementary_Figure4.eps}
\caption{\label{SuppFig4} Rheology of a 34\% cornstarch suspension \textbf{a.} Viscosity $\eta$ as a function of the shear rate $\dot{\gamma}$. \textbf{b.} Mean normal stress (measured by built-in sensor of the rheometer) as a function of the shear rate. \textbf{c.} Normal stress signal measured by two sensors placed symmetrically below the geometry, before the shear thickening regime. \textbf{d.} Normal stress signal measured during the shear thickening. The position of the measurement points of \textbf{c} and \textbf{d} in the flow curves are shown with green and blue circles in \textbf{a} and \textbf{b}.}
\end{figure}

\begin{figure}[ht!]
\includegraphics[width=0.95\textwidth]{Supplementary/Supplementary_Figure5.eps}
\caption{\label{SuppFig5} \textbf{a.} Phase diagram showing the dimensionless velocity $\Omega_a/\Omega$ of the stress signal as a function of the gap height $h$ and the weight fraction in particles $\phi_{\rm w}$. The size of the dot indicates the velocity, as shown in the legend. The points where $\Omega_a/\Omega < 1$ are plotted in red and in green for $\Omega_a / \Omega > 1$. \textbf{b.} Normal stress signal measured by two sensors placed symmetrically below the geometry, for $h$ = 0.8~mm, $\phi_{\rm w}$ = 40 \%. The pressure peak is noisy and large. \textbf{c.} For $h$ = 1.3~mm and $\phi_{\rm w}$ = 40\%, the normal stress signal is smoother and more peaked. \textbf{d.} Normal stress signal for $\phi$ = 39 \%, $h$ = 1~mm \textbf{e.} Normal stress signal for $\phi =$ 41 \%, $h$ = 1~mm.}
\end{figure}

\end{document}
