%\section{How to Ensure Fairness}
%\section{Tool-kit for Fairness}
\section{Requirements for Achieving Fairness}
\label{s:core}
%\im{Perhaps title: Mechanism(s) to achieve fairness?}
%\pg{See if the notations and math is formal enough? Can we include visualizations to help (e.g., for showing what does the inter-delivery conditions mean)? Do the titles make sense? Do we need to provide more context or is it easy to follow? Most importantly do we need to provide better justification for the kind of fairness we focus on? Should we use the term RTF?}

%\noindent
%\pg{Is this start OK?}
%In this section, we will look at  properties that the delivery processes ($\cup{D_i}$) must ensure to provide fairness.

%\attn{In this section, we will establish fundamental requirements for achieving fairness. We consider different variants of fairness. Our definitions of fairness are a formal extension of the temporal fairness first introduced in~\cite{libra}. Intuitively, temporal fairness argues that trade orders from different market participants generated based on the same market data information should be ordered based on the response time of the market participants (i.e., the faster market participants orders should be submitted to the ME first). We first introduce the notion of delivery (time) based ordering. }

\attn{In this section, we will establish fundamental requirements for achieving fairness. 
We consider different variants of fairness. 
%Our definitions of fairness are a formal extension of the temporal fairness first introduced in~\cite{libra}.
%Intuitively, each variant argues that trades from different market participants generated based on the same market data information should be ordered based on the response time of the market participants (i.e., faster MP's trades should be ordered ahead of other MPs).}
%
%\eg{nit: 
At a high level, each variant argues that trades generated by different market participants based on the same market data information should be ordered based on the response time of the market participants (i.e., faster MP's trades should be ordered ahead of other MPs).}
\attn{Note that, in this section our goal is not to provide exact schemes for the delivery and ordering processes; %we do provide an exact scheme for the ordering process
rather we only aim to establish the minimum constraints on them to achieve perfect fairness.
%to facilitate those. 
We will use these constraints as guiding principles to propose concrete schemes in the next section and show that these schemes can provide fairness with a high probability. We begin by introducing delivery-time-based ordering (DBO) and showing how it can be used to improve fairness.}
\pg{Maybe present DBO case by case, starting from a single data point and two MPs responding to it.}

%We will show that the necessary conditions we derive for the delivery processes are also sufficient. In particular, we will show that if the necessary conditions for the delivery processes are met then the delivery clock time based ordering ensures fairness in each case. %\pg{Can this be made better?}

%In this section, we will look at  properties that the delivery processes ($\cup{D_i}$) must ensure to provide fairness. We will consider different variants of fairness. Our definitions of fairness are a formal extension of the temporal fairness first introduced in~\cite{libra}. Intuitively, temporal fairness argues that trade orders from different market participants generated based on the same market data information should be ordered based on the response time of the market participants (i.e., the faster market participants orders should be submitted to the ME first). We will first introduce the notion of deliver clock time based ordering. We will show that the necessary conditions we derive for the delivery processes are also sufficient. In particular, we will show that if the necessary conditions for the delivery processes are met then the delivery clock time based ordering ensures fairness in each case.

%Before we begin, we n


\smallskip
\noindent
\textit{Delivery Clock:} %The high-level idea here is that 
\attn{%Competing MPs don't make trade decisions in isolation rather they place trades directly in response to same market data stream. 
Competing MPs make trade decisions directly in response to the same market data stream.}
%For trade decisions,} 
Events corresponding to the delivery of the same market data point to different MPs, thus, relate to the same conceptual event. %To capture this relationship, e
Each RB maintains a separate delivery clock to track these conceptual events. Delivery clock is represented by a lexicographical tuple and it increases monotonically with time. Formally, delivery clock at RB$_i$ at time $t$ is given by,
\begin{align}
    DC_i(t) = \langle x_l(t), t-D_i(x_l(t))\rangle.
\end{align}
where $x_l(t)$ is the latest data point that was delivered to MP$_i$ (i.e., $D_i(x_l(t)) \leq t < D_i(x_l(t)+1)$). Interval, $t-D_i(x_l(t))$, corresponds to the time that has elapsed since the latest delivery.\footnote{$t-D_i(x_l(t))$ can be computed based on the local clock of RB$_i$.}% as long as the clock drift rate is small.}.
This tuple tracks the progress of market data delivery to the corresponding MP. 

\begin{figure}[t]
\centering
    \includegraphics[trim={0 0 0 2mm},clip,width=0.9\columnwidth]{hotnets-images/time series visualization (3).pdf}
    \vspace{-3mm}
    \caption{\small{{\bf DBO can help correct for late delivery of data.} Delivery of market data to MP$_i$ is lagging behind MP$_j$. There are two trades $(i,k)$ and $(j,l)$ generated in response to the same market data $x$. $(j,l)$ was submitted before $(i,k)$ but
    %, i.e., $A_j(l) < A_i(k)$. 
    response time of $(i,k)$ is less than $(j,l)$.
    %, i.e., $rt_i(k) < rt_j(l)$. 
    With DBO, $O(i,k) (= \langle x, rt_i(k)\rangle) < O(j,l) (= \langle x, rt_j(l)\rangle)$ and trade $(i,k)$ is correctly ordered ahead of $(j,l)$. Ordering based on the submission time leads to incorrect ordering.}}
    \label{fig:dbo_correction}
    \vspace{-4mm}
\end{figure}
%\pg{Eashan see if you can redraw this for clarity.}} %\eg{$A_j(l)<A_i(k)$ and $rt_i(k)<rt_j(l)$. Even though trade order from $\text{RB}_j$ is submitted to the exchange network before $\text{RB}_i$, according to DBO, $\text{RB}_i$'s request should be executed first due to its lower response time i.e. $O(i,k) (= <x, rt_i(k)>) < O(j,l) (= <x, rt_j(l)>)$.}}\pg{Eashan redraw this.}}

\vspace{-1mm}
\begin{definition}
DBO satisfies the following condition,
\begin{align}
O(i,k) = DC_i(A_i(k)).
\end{align}
\vspace{-6mm}
\end{definition}

\attn{With DBO, trades are ordered based on the RB \emph{delivery clock time at trade submission}.\footnote{\attn{For DBO, each RB can tag this delivery clock time in trades before forwarding them to the CES.}} In other words, DBO is ordering trades from MPs relative to when they received the market data. }
Intuitively, DBO can be thought of as a post hoc way of correcting for time differences in delivery of market data to MPs. % (due to, for example,  high latency from the CES to a particular MP). 
For example, if market data delivery to a particular MP lags behind other MPs (e.g., due to a latency spike), then its delivery clock also lags behind. Compared to ordering trades based on the submission time, with DBO, trades from this MP receive a boost in ordering that can correct for the late delivery (example in \Fig{dbo_correction}). \pg{Another way of thinking about DBO is that (loosely speaking) it creates the perception that each MP has the same RTT to the CES. Roughly speaking, incoming trades from MPs are delayed in inverse proportion to the delay experienced in market data delivery (see Handling latency variation on the reverse path in \S\ref{s:exp}).}


%\Fig{dbo_correction} illustrates this in a simple scenario. 
% \eg{With DBO, trades are ordered based on the delivery clock time at trade submission to the corresponding RB. In other words, DBO is ordering trades from MPs relative to when they received the market data. For example, if market data delivery to a particular MP lags behind other MPs (say due to a latency spike), then its delivery clock compared to other MPs also lags behind. Compared to ordering trades based on the real time at submission, with DBO, trades from this MP receive a boost in ordering that can correct for the late delivery (example in \Fig{dbo_correction}).}

%\pg{Eashan a figure to illustrate this would be super useful.}
%
%(due to, for example,  high latency from the CES to a particular MP).
%

DBO alone is capable of partially correcting differences in market data delivery across MPs. Perfect correction, would require measuring the response time of a trade. The challenge, however, is that a trade could have been generated in response to any of the market data points delivered to the MP (and not just the latest data point $x_l$). The RB/OB cannot trust the MP to truthfully offer this information. We show that we can alleviate this issue by enforcing certain restrictions on how the delivery clocks advance across RBs (i.e., restrictions on the pace of market data delivery to the MPs). 

We will now derive the minimum requirements on the delivery processes for achieving fairness for arbitrary trade orders for \emph{any} ordering process.\footnote{We assume that $f_i$ is not known to the ordering process.} We will also show that if these delivery requirements are met,  DBO ensures perfect ordering for fairness. 

\subsection{Strong Fairness}
\label{ss:strong_fairness}
%\pg{Eashan verify if the theorems and are corollaries are indeed correct. Pay special attention to whether it should be $<$ ot $\leq$.}

\begin{definition}
An ordering process $O$ is strongly fair if it satisfies the following conditions,
\begin{align*}
    C1: &\text{ If } A_i(k) < A_i(l), \text{ then, } O(i,k) < O(i,l).\nonumber\\
    C2: &\text{ If } f_i(k) = f_j(l) \land rt_i(k) < rt_j(l), \text{ then, }
    O(i,k) < O(j,l).
\end{align*}
\label{def:strong}
\vspace{-5mm}
\end{definition}

%Condition $C1$ states that a MP is always better-off submitting the trade order as early as possible and waiting to submit trade orders does not offer any advantage.
Condition $C1$ states that a MP is always better-off submitting the trade order as early as possible. $C2$ states that trade orders generated based on the same market data point should be ordered based on the response time of the MPs.% (i.e., faster MP's trades should be ordered ahead).}
%$C2$ states that trade orders generated based on the same market data point should be ordered based on the response time of the MPs.} %.%forwarded to the ME first). 


\begin{theorem}
The \textit{necessary} and \textit{sufficient} conditions on the delivery processes for strongly fair ordering are given by,
\begin{align*}
    D_i(x+1) - D_i(x) &= D_j(x+1) - D_j(x), & \forall i,j,x.
\end{align*}
\label{thm:1}
\vspace{-6mm}
\end{theorem}

The theorem states that for strong fairness the inter-delivery times should be the same across all  MPs. In other words, the delivery clocks at all RBs (at any given delivery clock time) must advance at the same rate.

% eg{The theorem states that, for strong fairness, the intervals between consecutive market data deliveries -- inter-delivery times -- should be the same across all MPs. In other words, the delivery clocks at all RBs (at any given delivery clock time) must advance at the same rate.}

\begin{proof}
\textit{Necessary:} To prove that the above condition is necessary we will show that if this condition is not met then no ordering process exists which is strongly fair for arbitrary trade orders. %\attn{We will constuct a tra}%already said this %
%The reason for this is that the OB does not know the response time of trade orders ($\sum f_i$ is unknown). 

\begin{figure}[t]
\centering
    \includegraphics[trim={0 0 0 1mm},clip,width=0.8\columnwidth]{hotnets-images/delivery times.pdf}
    \vspace{-3mm}
    \caption{{\small{\bf Proof of Theorem 1.}}}% \pg{Eashan include as an arrow for progress of time.}}
    \label{fig:proof}
    \vspace{-5mm}
\end{figure}

Consider the following scenario (\Fig{proof}) where the above condition is not met. Let $D_i(x+1) - D_i(x) = c1$, $D_j(x+1) - D_j(x) = c2$. Without loss of generality we assume $c1<c2$. Consider hypothetical trades $(i,k)$ and $(j,l)$ s.t. $A_i(k) = D_i(x+1) + c3$ and $A_j(l) = D_j(x+1) + c4$. Further, we can pick $A_i(k)$ and $A_j(l)$ s.t. $c3>c4$ and $c1+c3<c2+c4$. Now we consider two scenarios for how these trades were generated. These two scenarios are indistinguishable at the OB.
%Consider a hypothetical trade order $k$ (from MP$_i$) and $l$ (from MP$_j$) s.t. $A_i(k) = D_i(x+1) + c3$ and $A_j(l) = D_j(x+1) + c4$. Further, we can pick $A_i(k)$ and $A_j(l)$ s.t. $c3>c4$ and $c1+c3<c2+c4$. Now we consider two scenarios for how these trades were generated. These two scenarios are indistinguishable at the OB.

\noindent
\text{Case 1:} $f_i(k) = f_j(l) = x+1$. Here,
\begin{align}
rt_i(k) = c3, rt_j(l) = c4.
\end{align}
Since $c3>c4$, a strongly fair ordering (condition $C2$) must satisfy,
%\begin{align}
$O(i,k) > O(j,l)$.
%\label{eq:theorem_1:necessary:case1}
%\end{align}

\noindent
\text{Case 2:} $f_i(k) = f_j(l) = x$. Here,
\begin{align}
rt_i(k) = c1+c3, rt_j(l) = c2+c4.
\end{align}
In this case, since $c1+c3<c2+c4$, a strongly fair ordering must instead satisfy the opposite, $O(i,k) < O(j,l)$. A contradiction! \attn{Thus, no ordering process can be strongly fair in both these scenarios.}
%\label{eq:theorem_2:necessary:case1}
%\end{align}


\smallskip
\noindent
\textit{Sufficient:} We will now show that if the inter-delivery times are same across MPs, then a strongly fair ordering exists. 

Assuming same inter-delivery times, DBO trivially satisfies $C1$. DBO also satisfies $C2$, i.e.,  if $f_i(k) = f_j(l) \land rt_i(k) < rt_j(l)$, then,
\begin{align}
    DC_i(A_i(k)) < DC_j(A_j(l)). 
\end{align}
Intuitively, this is because market data $f_i(k) (= f_j(l))$ is delivered to each MP at the same delivery clock time (by definition). Further, delivery clocks advance at the same rate across all MPs. When measured from the delivery of $f_i(k) (= f_j(l))$, delivery clock of RB$_j$ in duration $rt_j(l)$ will advance more than delivery clock of RB$_i$ in duration $rt_i(k)$.
Therefore, DBO is strongly fair.\footnote{\attn{Note that, DBO is not the only ordering process that can achieve strong fairness. Other ordering processes that also order trades from MPs based on when they received the market data can also achieve strong fairness.}} 
\end{proof}

\if 0
Assuming same inter-delivery times, DBO trivially satisfies $C1$. DBO also satisfies $C2$. If $f_i(k) = f_j(l) \land rt_i(k) < rt_j(l)$, then,
\begin{align}
    DC_i(A_i(k)) < DC_j(A_j(l)). 
\end{align}
% \eg{
% \begin{align}
%     DC_i(A_i(k)) = \langle x^i_l, rt_i(k)+(D_i(f_i(k))-D_i(x^i_l)) \rangle < DC_j(A_j(l)) = \langle x^j_l, rt_j(l)+(D_j(f_j(l))-D_j(x^j_l)) \rangle. 
% \end{align}}
Intuitively, DBO satisfies $C2$ because market data $f_i(k) (= f_j(l))$ is delivered to all the MPs at the same delivery clock time (by definition). Because delivery clocks advance at the same rate across all MPs, (measured from the delivery of $f_i(k)$) delivery clock of RB$_j$ in duration $rt_j(l)$ will have advanced more than delivery clock of RB$_i$ in duration $rt_i(k)$.
Therefore, DBO is strongly fair. 
%\end{proof}
\fi

\noindent
\textit{On impossibility of strong fairness:} It is possible to show that if communication latency is not bounded then RBs cannot %coordinate \im{co-ordinate implies talking to each other; another word perhaps?} to
achieve the same inter-delivery times always. %(if the market data stream is not known in advance \im{I don't understand this text in the parenthesis}). 
In the interest of space we skip this proof. The high-level idea is that if two RBs can achieve the same inter-delivery times then they can also agree to execute some task (not known at the start) simultaneously. 
%From earlier work on the folklore two-generals-problem~\cite{two_generals} it is well known that is impossible if the communication latency is not bounded.\eg{some rephrasing required: 
From earlier work on the folklore two-generals-problem~\cite{two_generals}, it is well known that such simultaneous execution is impossible if the communication latency is not bounded. While it might not be possible to achieve the same inter-delivery times all the time, we can achieve it with high probability at the cost of some additional latency at the RB~\cite{cloudex}.% \pg{check if the citation is correct, maybe a better one exists.}

Next, we consider two weaker variants of fairness. We give the necessary and sufficient conditions for the delivery processes in each case. The proofs in each case are similar to that of Theorem~\ref{thm:1}. We can construct a counter example to show that the conditions are necessary. For sufficiency, we can show that if the conditions are met then DBO satisfies the fairness properties. Depending on the needs of the application one can also consider alternate definitions of fairness, derive the desired properties and construct schemes that try to ensure these properties for fairness.

\subsection{Limited Fairness}
\label{ss:limited_fairness}
\vspace{-1mm}
\begin{definition}
An ordering process $O$ ensures limited fairness if it satisfies $C1$ and the following condition,
\begin{align*}
    C3: \text{If } f_i(k) &= f_j(l) \land rt_i(k) < rt_j(l) \land rt_i(k) < \delta, \text{ then},\nonumber\\
    O(i,k) &< O(j,l).
\end{align*}
\vspace{-5mm}
\end{definition}
%\im{that should be $rt_i(k) < rt_j(l)$ correct? If yes, please fix}
where $\delta$ is a positive constant.

Intuitively, $C3$ states that if the response time of a MP is bounded, then its trades will ordered ahead of corresponding trades from other MPs as long as it is faster than other MPs.
This definition is most relevant in the high frequency trading world where the response time is in the order of a few microseconds. 
%(\pg{Chaitanya can we motivate this definition better}). %\eg{can we say, "The reason for unfairness is the unequal transmission delays in the network, the variance of which in a high performance network is of the order of $100\mu s$. Following this, we may be only concerned with fairness for trades within an appropriate $\delta$ so as to provide all fast MPs a fair platform."}\pg{I dont understand your comment}

\vspace{-1mm}
\begin{corollary}
The \textit{necessary} and \textit{sufficient} conditions on the delivery processes for limited fairness are given by,
\begin{align*}
    \text{If }  D_i(x+1) - D_i(x) &< \delta, \text{ then},\nonumber\\
    D_j(x+1) - D_j(x) &= D_i(x+1) - D_i(x), & \forall j.
\end{align*}
\label{cor:1}
\vspace{-5mm}
\end{corollary}


%\im{leeway or headroom instead of slack? not sure}
Compared to strong fairness, the above condition offers some leeway for how the delivery clocks can advance. \attn{If for a certain MP, the inter-delivery time for two consecutive data points is greater than equal to $\delta$, then for any other MP, the inter-delivery for these points can differ as long as it is greater then $\delta$.}
%In particular, for certain market data points the inter-delivery times across MPs need not be the same as long as the inter-delivery time for such points at each MP $\geq \delta$. 
This leeway is useful for dealing with sudden latency spikes. In the next section, we will consider a simple scheme for the delivery processes that meets the above condition at all times %(for any value of $\delta$)
regardless of the fluctuations in latency.

%I like what you have, let me think more.
%Compared to strong fairness, the above condition offers some leeway for how the delivery clocks can advance. For example, in case of sudden latency spikes, as long as all MPs maintain inter-delivery times larger than $\delta$, the MPs can pace data delivery accordingly so as to handle the unexpected delays \sout{while ensuring the fairness property}. In the next section, we will consider a simple scheme for the delivery processes that meets the above \sout{fairness} condition at all times regardless of the fluctuations in latency.

\subsection{Approximate Fairness}
\label{ss:approximate_fairness}
\begin{definition}
An ordering process $O$ is approximately fair if it satisfies $C1$ and the following condition,
\begin{align*}
    C4: \text{If } f_i(k) &= f_j(l) \land rt_i(k) \cdot (1+\epsilon)< rt_j(l), \text{ then},\nonumber\\
    O(i,k) &< O(j,l),
\end{align*}
\vspace{-6mm}
\end{definition}
where $\epsilon$ is a positive constant.

Intuitively, $C4$ states that as long as a MP is faster than other MPs by a certain margin, it's trades will be ordered ahead.%submitted to the ME earlier.

\vspace{-1mm}
\begin{corollary}
The \textit{necessary} and \textit{sufficient} conditions on the delivery processes for approximately fair ordering are given by,
\begin{align*}
    D_i(x+1) - D_i(x) &\leq (D_j(x+1) - D_j(x)) \cdot (1 +\epsilon), & \forall i,j,x.
\end{align*}
\label{cor:2}
\vspace{-6mm}
\end{corollary}

Compared to strong fairness, the above condition also offers some leeway for inter-delivery times to differ. This leeway can be useful for masking fluctuations in latency (\S\ref{s:exp}). %\pg{Include a note on how this relates to clock drift rate?}

\smallskip
\noindent
\textit{Do we need clock-sync?} For fairness, RBs only need to ensure the specified constraints for inter-delivery times in each case. Ensuring these contraints does not require clock synchronization across RBs. A RB can use its own local clock for maintaining inter-delivery times as long as the clock drift rates are small.
%\footnote{It is possible to analyze the impact of high clock drift rates on fairness. Due to space constraints, we exclude it.} 
With perfect clock synchronization, RBs can additionally ensure that market data is released at the same time when not experiencing extreme fluctuations in network latencies. 
%Furthermore, with clock synchronization among RBs, DBO could additionally ensure fairness for `one-sided' trades (generated independently of the current market data -- e.g, limit orders) that ideally should be ordered based on the time when they were submitted.
Furthermore, simultaneous delivery of market data also syncs the delivery clocks at RBs. In such cases, DBO additionally provides fine-grained fairness (similar to that of CloudEx) for `one-sided' trades (generated independently of the current market data) % -- e.g, limit orders)
that ideally should be ordered based on the time when they were submitted.
%\pg{chaitanya please check this part about one sided orders, example of one side orders, how important is fairness for such one sided trades, not sure if this is needed}%\pg{does this make sense? should we say fairness for one-sided orders is less important or the time scales are less relavent (similar to our argument for trades based on non-real time data).}

%\pg{Ensuring these inter-delivery constraints does not require access to clock sync }