

\section{Results and Discussion}
Combining the quantitative results and feedback from post-study interviews informed the following insights and design recommendations. We performed a two-way ANOVA on the quantitative data where an align-ranked transform was performed as the scores of the questions do not have a normal distribution (Shapiro-Wilk normality test p<.001). 


\begin{figure}[b!]
    \centering
    \includegraphics[width=\linewidth]{Figures/cannon_eg.png}
    % \vspace{-10px}
    \caption{The user aligns the controller against the back of a chair to place the cannon (A). They can rest their hands on the chair, move it around and receive haptic feedback (B) as they rotate the virtual cannon to aim at the monsters (C).}
    \label{fig:demo-chair}
\end{figure}
\subsection{Haptic Feedback}
To understand whether the repurposed object indeed provided appropriate haptic feedback, we asked the following questions. Q1: The virtual object felt real when I touched it. Q2: The virtual object felt like it was there when I touched it. (1–strongly disagree, 7–strongly agree). For both questions, the haptic condition was rated significantly (p<.001) higher than the no-haptic condition (Q1 Haptic: M=3.7, SD=1.6; No-haptic: M=1.6, SD=1.2; Q2 Haptic: M=4.4, SD=1.8; No-haptic: M=2.1, SD=1.9). Evidently, the temporal coincidence of the hands interacting with physical objects helps reaffirm the virtual objects' realness. Most participants reported having found haptic feedback to be helpful, providing additional cues and feedback for their actions. For “whack-a-mole”, the surface of the table provides a stop to the hand’s downward motion when “mole” is hit, and participants reported that “slamming down on a table felt satisfying” (P5). Similarly, the back of the chair provided a cue for “where to rest the hands” (P4) on the virtual cannon. Participants also commented specifically on the similarity between the texture and compliance of the pillow and the cat's fur. Petting a “cushion” made the experience more immersive; participants were “surprised by how much more real it felt” (P8). There were variations in how well the haptic feedback matched the expectation and visual feedback. Both the questionnaire results and post-study interview reveal that participants preferred “Pet a cat” the most, followed by “Whack-a-mole” and “Shoot monsters”. The texture and compliance of a cushion coarsely match with those of a furry animal, thus the haptic feedback still positively contributes to the experience despite the difference in the shape of the cushion and the cat. Even though the table surface is able to cue where the hand stops for “Whack-a-mole”, the binary state of the button is not produced with a flat table. The virtual buttons "had no compliance" and felt more like "holograms" (P6). Finally, while the swivel mechanism of a chair affords the rotating motion of the cannon, P7 noted that “the cannon was too easy to spin when it appeared heavier”. To mitigate this, designers could use techniques like haptic retargeting \cite{azmandian2016haptic} or make the cannon appear lighter visually. To summarize, when considering an object or part of an object to repurpose as a haptic prop, designers should choose everyday objects that can complete the feedback loop of users' actions and use visual-audio cues to fill in the gaps.



\begin{figure}[t!]
        \centering
        \includegraphics[width=\linewidth]{Figures/q_haptics.png}
        \caption{Results of haptic feedback. Error bars: SE.}
        \label{fig:graph-1}
\end{figure}

\begin{figure}[b!]
   
        \centering
        \includegraphics[width=\linewidth]{Figures/q_real.png}
         \caption{Left: Results of naturalness. Right: Results of consistency. Error bars: SE.}
        \label{fig:graph-3}
\end{figure}





\subsection{Set Up and Configuration Process} \label{setup}
We followed the original 1–7 scale of the NASA TLX questionnaire \cite{hart1988development}, where a lower score means less workload and more desirable. The mean score for the haptic condition was 4.02 (SD=1.11), which is higher than that of the no-haptic condition (M=2.82, SD=1.82). This was expected, as the haptic condition required extra steps that are not part of a typical VR experience and hence require more workload. However, participants responded positively about the extra steps. In post-study interviews, participants reported that the setup process was "easy" (P2) and "the instructions were clear" (P4). They were able to "find the objects easily" (P1) because they felt "familiar with the room spatially" (P1), and they appreciate that the "pillow was adjustable" (P5). Though some participants expressed concerns regarding hitting the wall and damaging the controller. Some commented that including walls and large objects within the guardian boundary is counter-intuitive as typical VR experiences happen in an obstacle-free space. Participants responded well to soft mobile objects such as a pillow. These objects are lightweight thus easy to move, and they are soft so they cannot cause any harm during setup or interaction. Rigid chairs are accessible but require more effort to move. P1 commented that the chair had a "huge presence...and took physical effort to move around". In summary, we recommend designers take advantage of mobile objects that are soft and lightweight, such as books, boxes, and cushions. For static objects, we recommend indicating their presence to prevent the users from running into them. Given the current limitation of object tracking and to reduce the burden on the users, we recommend using a single physical object as a haptic prop throughout the gameplay.


\subsection{Controller and Hand Tracking}
To understand the effect of the input method (i.e., controller and hand tracking), we adapted the following two questions from \cite{witmer1998measuring}. Q3: How much did the controller tracking/hand tracking interfere with the performance of assigned tasks or with other activities? (1-not at all, 7-interfered) Q4: How well could you move or manipulate objects in the virtual environment? (1-not at all, 7-extensively) Although there was no statistically significant effect of tracking interference, "Shoot monsters" was rated higher (more inference) for both conditions. From participant feedback, having physical props allowed them to "know where to stop the hand" (P5) and can e.g., "pet [the cat] very naturally" (P8). They reported that the overall hand tracking worked well except for the "Shooting monsters" task, where the grasping gesture sometimes was not detected accurately.



\begin{figure}[t!]
    \centering
    \includegraphics[width=\linewidth]{Figures/q_control.png}
    \caption{Left: Results of tracking interference. Right: Results of object manipulation. Error bars: SE.}
    \label{fig:graph-2}
\end{figure}

\begin{figure} [b!]
        \centering
        \includegraphics[width=\linewidth]{Figures/q_fun.png}
        \caption{Left: Results of funness. Right: Results of involvement. Error bars: SE.}
        \label{fig:graph-4}
\end{figure}

\subsection{Immersion, Realism, and Fun}
To understand participants' experience, especially their sense of presence, in the virtual environment while engaging with real-world objects, we asked the following questions: Q5: How natural did your interactions with the environment seem? (1-artificial, 7-natural) Q6: How much did your experiences in the virtual environment seem consistent with your real-world experiences? (1-not consistent, 7-consistent) Q7: This task was fun. (1-strongly disagree, 7-strongly agree) Q8: How involved were you in the virtual environment experience? (1-not involved, 7-engrossed). Q5, Q6, and Q8 are from the Presence Questionnaire \cite{witmer1998measuring}. The haptic condition was rated as more natural (M=4.1, SD=1.6, p<.001) than the no-haptic condition (M=2.5, SD=1.5), and "Pet a cat" was rated as the most natural. The haptic condition was rated as statistically more consistent (M=3.7, SD=1.6, p<.001) than the no-haptic condition (M=2.0, SD=1.2), and likewise "Pet a cat" was rated as the most consistent. In the follow-up interview, most participants noted the order of the most to least realistic task is "Pet a cat", "Whack-a-mole", and "Shoot monsters". There was a novelty effect for "Shooting monsters" as some commented they "never projected a cannon on a chair before" (P1) but found it nevertheless an interesting experience. Thus, "Shoot monsters" is the least close to a real-world experience. Although there was no statistical significance for fun and involvement during post-study interviews, participants specifically commented on how they "didn't expect [petting] the cat to be so fun" (P6) with a pillow as a substitute, and a couple of participants reported “time went by faster; [they] didn’t notice the progress bar” (P8).

% \begin{figure}[t!]
% \includegraphics[width=\linewidth]{Figures/q_fun.png}
% \caption{Results of how fun and involved the experience in the virtual environment is. Error bars represent standard error (SE).}
% \label{fig:graph-4}
% \end{figure}

