

\section{Limitation and Future Work}

\subsubsection*{Object Tracking and Recognition} 
To reduce the manual effort for configuration, we are interested in exploring real-time object tracking using LiDAR and object recognition. In this case, once an object has been located and configured using a controller, we assume that the object is either stationary or follows the hands’ movement. Being able to adapt to the physical objects properties and features would further enhance the immersivity. For example, being able to recognize the texture of a cushion and adapt the cat's fur to the texture or change the curvature of the cannon's handle to match that of the back of the chair.

% , which would still be limiting when the objects are partially occluded

% \subsubsection*{Hand Tracking} For the most part, current hand tracking technologies work reliably well for interacting with static objects and interactions that does not require self-occluding hand gestures. The "Shoot monsters" scenario, on the other hand, suffered from the lack of accuracy in detecting grasps. The scenario involves moving of the object, and as discussed in the above section, our current approach relies on hand tracking technology to synchronize the movement between the virtual and physical object. At the moment, most state-of-the-art hand tracking focuses on the pinching gesture, whereas in reality, pinching is not suitable for a range of natural interactions (e.g., grabbing, squeezing, and clenching), but we have no doubt that the hand tracking technology will improve and allow for even more immersive experiences using our technique.

\subsubsection*{Variability in Objects} Even though we chose common objects that one can find in a room, we expected variations in the objects in our participants’ space. 
% The variation in the chair had the most distinct effect on the experience from participants’ feedback. 
For example, the back of the chair had variations in curvature and some participants’ chairs could not swivel, which limits how the virtual cannon could be manipulated. 
Also, from the participants’ feedback, the mismatch between the flat table surface and the binary state of the “mole” was prevalent and affected the experience more than other factors. 
Future work should investigate how to address these variations of physical objects through, for example, automated virtual shape deformation.
% Future work should investigate how to address these variations of the physical objects through, for example, an automated virtual shape deformation.
% There were variations in the texture of the cushion, and some are closer to the texture of a cat’s fur than others. 
% We did not receive any feedback about the table’s features so we could not make any conclusive insights, but 

% When asked about other scenarios that could use this technique, participants reported games that involve small, mobile objects such as the pillow or a book, modifying or substituting controllers to fit the context of the game, and sitting or standing on couches during the VR experience.

\subsubsection*{Fill the Gap of the Design Space} 
Finally, as discussed in the Design Space section, there is an interesting design challenge in leveraging \textit{functional affordances}, rather than the shape of an object itself. Moreover, our design space exploration is just an initial investigation, and we did not demonstrate or prototype most of the exemplary scenarios illustrated in Figure~\ref{fig:design-space}. Future work should further investigate and demonstrate other possible VR haptic applications by filling the gap in the current design space, and then examine how these affordances allow for richer haptic experiences in real-world environments. In addition, electric or motorized appliances such as an air-conditioner and Roomba as in MoveVR\cite{wang2020movevr}, could afford even more dynamic and diverse virtual applications. 


% \subsubsection*{Common Mechanisms} We only focused on mobility as one variable for our design space. Building on the findings in \cite{simeone2015substitutional} where the similarity in affordance allows for a more believable substitution, future work can leverage the functional affordance of everyday objects, and here we describe a few exemplary uses for different types of functional affordances. The compliance of a couch or a cushion could be used for CPR training where the user exerts a large downward force, or for a boxing game where the player hits the punching bag at a high speed. We can repurpose a doorknob and a light switch in the room to control elements in the VR game. Beyond using the mechanisms for their original purpose, we can focus on the constrained motion that they afford. A door handle affords a lever that can be repurposed as the handle of a lemon squeezer. The door affords a rotating motion constrained by the hinge that can only be moved along the edge of a circular plane at the height of the door handle. A drawer can only be pushed or pulled linearly. One may also think about the heat capacity of a material as an affordance to design experiences that involve temperature. User-generated actions can also help create a breeze at different temperatures. Using a hair dryer can generate warm air and using a fan or fanning oneself with a book can generate a cool breeze. Texture plays a key role in providing information about the material of an object yet challenging to recreate in VR. A fuzzy cushion is similar to an animal's fur, and a carpet is analogous to a grass field. With the help of visual cues, it is not difficult to imagine touching the curtain as feeling a willow tree or a giant's long hair.


% Many gameplay elements involve the use of pulling a drawer open and toggle on and off switches, which can be found in a physical room. Compared to static objects, these mechanisms are more difficult to track and configure, and interactions with these mechanisms require more precise, fine motor control. For future work, we hope to include these mechanisms to provide more dynamic kinesthetic haptic feedback. Furthermore, some of the everyday objects in our design space are less common and accessible than others. For example, walls, tables, chairs, and cushions are common household items, but a handheld vacuum is not guaranteed to be found in a home. Designers could consider multiple options of objects for repurposing. 


% {\it Shared experience} – We have only explored adapting to a single user's environment. Merging and sharing of the same VR experience but configured in different physical environments is another opportunity for remote collaborative gameplay in VR. Future work can explore whether bringing physical objects in multi-user experience could enhance collaboration and a feeling of shared presence.

% {\it Safety} – Conventionally, users are asked to clear up a space and set up a guardian boundary where no obstacles are included within. This safety measure is important to prevent harming the user during an immersive VR experience, and it should still be respected for our technique. For our implementation, we only included one physical object at a time in the guardian boundary, and the physical object is actively involved during the experience such that it is not left unattended and potentially forgotten. Future work should investigate how to provide visual guidance in such a virtual-physical hybrid environment to safe guard the users and to convey expected haptic feedback.






