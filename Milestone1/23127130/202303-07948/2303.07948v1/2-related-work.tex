


\section{Related Work}

\subsection{Passive Haptics}
Passive haptics is a technique that repurposes physical props or environments to create haptic sensations in VR and AR~\cite{insko2001passive, hoffman1998physically, daiber2021everyday}. Passive haptic devices allow users to control and manipulate 3D virtual models with greater flexibility and accuracy~\cite{hinckley1994passive, shapira2016tactilevr, zielasko2019passive, zhou2020gripmarks, henderson2008opportunistic}, without the need for special-purpose haptic devices, making them highly adaptable and deployable.
However, one challenge is the mismatch between virtual and physical objects~\cite{rock1964vision}. To address this problem, haptic retargeting~\cite{azmandian2016haptic, fang2021retargeted} is a technique that uses visual illusion to redirect the user’s hand when touching a virtual object. 
Similarly, other techniques, like Annexing Reality~\cite{hettiarachchi2016annexing} and Sparse Haptic Proxy~\cite{cheng2017sparse}, build on this approach by remapping the geometry of physical props and environments on-the-fly. 
Alternatively, encountered-type haptics~\cite{mcneely1993robotic} aim to provide dynamic passive haptics for whole-body haptic interaction (e.g., \textit{RoomShift}~\cite{suzuki2020roomshift}, \textit{MoveVR}~\cite{wang2020movevr}, \textit{ZoomWalls}~\cite{yixian2020zoomwalls}, \textit{CoVR}~\cite{bouzbib2020covr}). 
Instead of robotic systems, researchers have also explored human actuation to create dynamic haptic environments~\cite{cheng2015turkdeck,cheng2014haptic,cheng2017mutual}.
However, these active haptic systems are currently mostly limited to research lab use due to high costs, large size, and safety concerns.




\subsection{VR with Real-World Environment}
To address this problem, recent works have also started looking into how to further blend the virtual and physical environments for VR to provide more immersive experiences that go beyond mobile and hand-held haptic proxy.
Substitutional Reality~\cite{simeone2015substitutional, eckstein2019smart}, for example, reuses rather than replaces surrounding physical props as objects in the virtual space. 
Other systems use a mix of AR and VR to dynamically represent real-world objects and stimuli or reconstruct physical space ~\cite{yang2019dreamwalker, hartmann2019realitycheck,lindlbauer2018remixed,lin2020architect, tao2022integrating}.
Our work builds upon concepts presented in substitutional reality, but we contribute to the results and insights from the in-the-wild study to evaluate the idea in the real-world environment rather than in a controlled lab setting. We also contribute to the design space to expand the range of everyday objects that can be incorporated into VR experiences.
% Our work builds on top of the substitutional reality concept and repurpose physical objects and surfaces as proxies for virtual interactables. 
% However, unlike most of the previous work, we contribute to the results and insights from the in-the-wild study. We also contribute to possible design space to expand the range of everyday objects that can be incoporated into the VR experience.

% It investigates the limits of mismatches between physical and virtual objects. 
% in which VR environments are designed to match with surrounding physical environments. 

% However, these active haptic systems are still largely limited with the use in a research lab, mostly because of the high cost, large size, mechanically bulky systems, and safety concerns.

% These passive haptic devices allow the user to control and manipulate 3D virtual models with greater flexibility~\cite{hinckley1994passive, shapira2016tactilevr, zielasko2019passive, zhou2020gripmarks}, while passive haptics often do not require any special-purpose haptic devices, which enables the large adaptability of the system.


% Our system does not require large-scale instrumentation of the environment or additional human presence and labor. We focus on existing surfaces and objects in a home and asks the user to incorporate specific physical objects as props during the set up process. Our approach also takes advantage of availability and flexibility of passive haptic props, and we bring the use of passive props outside of research laboratories. Instead of configuring a prepared environment or requiring additional fabrication process, we repurpose everyday objects in people's homes to provide haptic feedback and evaluate this workflow with the in-the-wild user study.


% Instead of blending the virtual and physical environments, we preserve the virtual world and introduce a set of 

% \subsection{Encountered-Type Haptics}
% Encountered-type haptics aim to reconfigure physical environments with robotic devices to provide active haptic sensations with earth-grounded robotic systems~\cite{mcneely1993robotic}.
% Compared to the other handheld or mobile haptic systems, robotic encountered-type haptics allow the large-scale interaction that the user can touch and interact through their whole body~\cite{suzuki2020roomshift, wang2020movevr, yixian2020zoomwalls, bouzbib2020covr}.
% In fact, these active haptic systems are still largely limited with the use in a research lab, mostly because of the high cost, large size, mechanically bulky systems, and safety concerns.

% For larger-scale interaction, human actuation enables to reconfigure the environment with human's manual labors instead of robots. For example, TurkDeck~\cite{cheng2015turkdeck}, HapticTurk~\cite{cheng2014haptic}, and Mutual Human Actuation~\cite{cheng2017mutual} explore this concept for large-scale encountered-type haptics with manual or human-powered reconfiguration.

% Our system does not require large-scale instrumentation of the environment or additional human presence and labor. We focus on existing surfaces and objects in a home and asks the user to incorporate specific physical objects as props during the set up process.

%%%%%%%%%%%to-cite
%Visuo-haptic Illusions for Linear Translation and Stretching using Physical Proxies in VR
%https://dl.acm.org/doi/pdf/10.1145/3411764.3445456?casa_token=b9o0BF7jwyIAAAAA:QJfbXaKplR6FKTK90KcwUgl8QZp0LiCpHDogNeh-ifs-dTPV8SfYlzGk5KAIoctTvST3OrJlQEuG
