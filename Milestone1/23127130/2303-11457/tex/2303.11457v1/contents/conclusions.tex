\section{CONCLUSIONS}\label{sec:conclusion}

 We presented a team coordination problem in a graph environment, where high levels of coordination in the form of “support” allows agents to reduce the cost of traversal on some edges. As an alternative to solving this with a version of MDP, we presented a method of planning in the joint state space – the Joint State Graph (JSG). We showed that a multi-agent path planning problem can be reduced to a single-agent planning in JSG, since the actions taken by the agents are built in to the edges of the JSG. We addressed the issue of scalability in the graph size by presenting a hierarchical decomposition method to perform path planning in two levels. We provided complexity and statistical analysis which show that the construction time for both CJSG and JSG do not differ by much, but the CJSG is significantly more efficient with regards to shortest path planning. Our numerical results verify this.

For future work, there are many aspects of the problem we proposed that we are intrigued to build upon. For instance, we would like to integrate more sophisticated notions of risk by using concepts from game theory and incorporating stochasticity in the formulation, such as stochastic costs. We are also interested in addressing the issue of scalability in terms of number of agents.

%In this paper, we presented a method for transforming an environment graph into a Joint State graph in order to transform a cooperative multi-agent path planning problem into a single-agent problem. We presented the setup of the environment graph for two agents. We defined the JSG and provided an algorithm for constructing it. To simplify the problem further, we decomposed the environment graph into a base graph and a support graph to explicitly classify the different movements of the agents: coupled and decoupled. We stated that the decoupled path planning can be done on the base graph, while coupled planning can be done on a Critical Joint State Graph. We provided the algorithm for constructing a CJSG followed by an algorithm for performing path planning on CJSG. In all our calculations, we made the assumption that the number of supporting nodes are much less than the total nodes in the environment graph. We stated that the complexity of the CJSG can be no more complicated than the JSG, since the former is a subset of the latter. 


% \addtolength{\textheight}{-12cm}   % This command serves to balance the column lengths
                                  % on the last page of the document manually. It shortens
                                  % the textheight of the last page by a suitable amount.
                                  % This command does not take effect until the next page
                                  % so it should come on the page before the last. Make
                                  % sure that you do not shorten the textheight too much.

%%%%%%%%%%%%%%%%%%%%%%%%%%%%%%%%%%%%%%%%%%%%%%%%%%%%%%%%%%%%%%%%%%%%%%%%%%%%%%%%



%%%%%%%%%%%%%%%%%%%%%%%%%%%%%%%%%%%%%%%%%%%%%%%%%%%%%%%%%%%%%%%%%%%%%%%%%%%%%%%%



%%%%%%%%%%%%%%%%%%%%%%%%%%%%%%%%%%%%%%%%%%%%%%%%%%%%%%%%%%%%%%%%%%%%%%%%%%%%%%%%


%\section*{APPENDIX}

%Appendixes should appear before the acknowledgment.

%\section*{ACKNOWLEDGMENT}

%The preferred spelling of the word �acknowledgment� in America is without an �e� after the �g�. Avoid the stilted expression, �One of us (R. B. G.) thanks . . .�  Instead, try �R. B. G. thanks�. Put sponsor acknowledgments in the unnumbered footnote on the first page.



%%%%%%%%%%%%%%%%%%%%%%%%%%%%%%%%%%%%%%%%%%%%%%%%%%%%%%%%%%%%%%%%%%%%%%%%%%%%%%%%
