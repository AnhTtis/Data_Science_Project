\section{NUMERICAL RESULTS}\label{sec:numerical-results}
% \daigo{We can show results from the same graph, but with different edge costs.  This was the original motivation of the research: ``how do we get different coordinated behavior based on the level or risk (or the cost of loosing an agent)?''}\\
 In this section, we evaluate JSG and CJSG on the basis of graph construction and path planning under different conditions.\footnote{\url{https://github.com/RobotiXX/team-coordination}} Our experimental design allows us to gain insights through comparative analysis of JSG and CJSG in terms of scalability and performance. The experiments are carried out on a MacBook Pro with 2.8 GHz 8 core CPU and 8GB of RAM. 


%Agent B can also reduce the cost to $c_{41},_{44} = 3$ with support from agent A. However, in cases where supporting does not reduce the cost, such as in where edge $e_{1,4}$ has cost $c_{1,4}=3$, both agents traverse the edge together without supporting each other.
%\manshi{Sara: This needs to be updated based on illustrative example}

For both graph construction and path planning analyses, a random graph generator is used to generate environment graphs with varying number of nodes and edges. 
% where nodes $v \in \nodeset$ and edges $\edgeset \subset \nodeset \times \nodeset$. 
We control the ratio of risk edges to the total edges to be 1/5, 1/3, and 1/2. For different number of nodes and risk edges ratio in an environment graph, we calculate graph construction time and shortest path planning time for JSG and CJSG (Table~\ref{table:result_table}). We repeat every experiment trial five times for statistical significance.


%\begin{figure}[H]
%    \includegraphics[width=0.5\columnwidth]{figures/cost cases.pdf}
%    \caption{Illustrative example with two cases for different risk edge costs.}
%    \label{fig_illustrative_solution1}
%\end{figure}
%\begin{figure}[H]
%    \includegraphics[width=\columnwidth]{figures/Lower risk cost.pdf}
%    \caption{Illustrative example without support}
%    \label{fig_illustrative_solution2}
%\end{figure}

% \begin{figure}[H]
%     \includegraphics[width=\columnwidth]{figures/TT_JSGvsCJSG.png}
%     \caption{Comparison of total solution time taken by JSG and CJSG with respect to increasing number of nodes.}
%     \label{fig_wrt_node}
% \end{figure}

% \begin{figure}[H]
%     \includegraphics[width=\columnwidth]{figures/TT_wrt_riskyedgeratio.png}
%     \caption{Comparison of total solution time taken by JSG and CJSG with respect to increasing number of nodes.}
%     \label{fig_wrt_riskedgratio}
% \end{figure}

% \begin{figure}[H]
%     \includegraphics[width=\columnwidth]{figures/FinalJSGvsCJSGPlot.pdf}
%     \caption{Comparison of time taken by JSG and CJSG to generate total solution with respect to increasing number of nodes and risk edges ratio.}
%     \label{fig_wrt_riskedgratio}
% \end{figure}

\begin{figure}[t]
    \centering
    \includegraphics[width=0.5\textwidth]{figures/FinalFinalJSGvsCJSG.pdf}
    \caption{Comparison of time taken by JSG and CJSG to generate total solution with respect to increasing number of nodes and risk edges ratio.}
    \label{fig_wrt_riskedgratio}
\end{figure}


\begin{table*}[ht]
 \caption{Comparison of JSG and CJSG}\label{table:result_table}
\centering
\begin{tabularx}{\textwidth}{@{} l *{10}{C} c @{}}
\toprule
\multicolumn{1}{c}{} & \multicolumn{1}{c}{} & \multicolumn{2}{c}{\textbf{JSG}} & \multicolumn{2}{c}{\textbf{CJSG}} \\
\cmidrule(rl){3-4} \cmidrule(rl){5-6}
\textbf{Nodes} &  \textbf{Risk Edges Ratio} &  {Graph Construction(s)} & {Shortest Path(s)} & {Graph Construction(s)} & {Shortest Path(s)} \\

\midrule
%10 & 1/5 &  $0.4576 \pm 0.0997$ & $0.3978\pm0.1229$ & $0.0618\pm0.0314$ & $0.1428\pm$
10   & 1/5  & 0.2119$\pm$0.0410 & 0.1440$\pm$0.0176 & 0.0146$\pm$0.0021 & 0.0198$\pm$0.0152 \\
     &  1/3  &  0.1760$\pm$0.0519 & 0.1510$\pm$0.0040 & 0.0895$\pm$0.0151 & 0.1258$\pm$0.0207 \\
      & 1/2 & 0.1906$\pm$0.0273 & 0.1567$\pm$0.0091 & 0.1810$\pm$0.0143 & 0.1469$\pm$0.0478 \\

\midrule 
20 & 1/5 &  3.1662$\pm$0.0405 & 2.098$\pm$0.0445 & 0.6525$\pm$0.0931 & 0.4348$\pm$0.0651 \\
    & 1/3 & 3.3989$\pm$0.0603 & 2.1988$\pm$0.0660 & 1.7742$\pm$0.0497 & 0.7988$\pm$0.0094 \\
    & 1/2 & 3.8566$\pm$0.0906 & 2.3192$\pm$0.0265 & 3.6439$\pm$0.5942 & 1.2523$\pm$0.1181 \\
    
\midrule 
30 & 1/5 & 20.9363$\pm$0.8312 & 11.6431$\pm$0.12776 & 6.1126$\pm$0.5537 & 2.7973$\pm$0.1778 \\
    & 1/3 & 22.4891$\pm$1.7074 & 12.3330$\pm$0.4995 & 13.8171$\pm$0.7259 & 4.5996$\pm$0.2204 \\
    & 1/2 & 25.9774$\pm$0.4323 & 13.4440$\pm$0.14222 & 26.8156$\pm$1.49423 & 6.9094$\pm$0.2677 \\  
\bottomrule
\end{tabularx}
\end{table*} 

\subsection{Graph Construction Analysis} 
From Table~\ref{table:result_table}, we analyze the graph construction time for JSG and CJSG under different conditions. Given a fixed risk edges ratio, e.g., 1/3 of the total edges, the improvement in graph construction time by CJSG compared to JSG maintains as the number of nodes increases from 10 to 30. Similarly, if we fix the number of nodes, e.g., 10, and increase the risk edges ratio gradually from 1/5, then 1/3, and finally 1/2, CJSG still takes less time compared to JSG. We can also see such a pattern for node 20 and node 30. 
%(with an exception at ratio 1/2). 
These results provide empirical evidence that CJSG is more efficient in constructing graphs. Note that when the risk edge ratio reaches 1/2, nearly all joint states are critical joint states, i.e., $|\mathcal{M}|\to|\mathcal{V}|^2$, and the graph construction times for the two approaches are close to each other. This observation is in line with Remark \ref{Rm_complexity}.

 % We conduct two sets of experiments to analyze graph construction time. In the first set of experiments, where the number of risk edges is fixed to be 1/3 of the total edges, CJSG performs slightly better than JSG as the number of nodes increases, with JSG taking marginally more time to construct the graph (as shown in Table~\ref{table:result_table}). In the second set of experiments, we compare the construction time of of different ratio of risk edges to node but with a fixed size of nodes and edges. Table~\ref{table:result_table} illustrates that as the ratio of risk edges to the number of nodes increases, CJSG takes slightly less construction time than JSG.  


\subsection{Path Planning Analysis} 

From Table~\ref{table:result_table}, we also assess the path planning time for JSG and CJSG with varying nodes and risk edges ratio. Given a certain risk edges ratio, e.g., 1/3, we can see that CJSG takes less time than JSG. This is true even if we increase the nodes from 10 to 30. Similarly, if we fix the node size, e.g., 20, and gradually increase the risk edges ratio as 1/5, 1/3, and 1/2 of total edges, CJSG is still more efficient than JSG. We can see the same pattern for nodes 10, 20 and 30. These results indicate that CJSG is more efficient than JSG in terms of shortest path planning when the ratio of risk edges to nodes increases. 

 % Our path planning analysis involved comparing the shortest path planning time between JSG and CJSG using two experiments. For the first experiment, we assumed the size of risk edges to be one third of the total edges for any given node. We then evaluated the performance of JSG and CJSG with respect to an increasing number of nodes. Table~\ref{table:result_table} illustrates that as the number of nodes increases, the path planning time for JSG increases more rapidly in comparison to CJSG. In the second experiment, we compared the shortest path planning time of JSG and CJSG with varying ratios of risk edges per node. The experiment involved keeping the number of nodes and edges constant while gradually increasing the number of risk edges ratio. Table~\ref{table:result_table} shows that as the ratio increases, JSG's path planning time becomes significantly larger than that of CJSG. These results indicate that CJSG is more efficient than JSG in  terms of shortest path planning when the ratio of risk edges to nodes increases. 

Based on the experimental results shown in Table~\ref{table:result_table}, we compute the total time taken by both JSG and CJSG to find the final solution. The total time involves time taken for graph construction and shortest path planning. In Fig. \ref{fig_wrt_riskedgratio}, we show that as the number of nodes increases, the time to generate total solution for JSG increases more significantly than that of CJSG. Fig. \ref{fig_wrt_riskedgratio} also illustrates that as the risk edges ratio increases, the time to generate solution for JSG increases more significantly compared to CJSG. Thus, CJSG is more efficient than JSG for overall solution generation.  





