%%%%%%%%%%%%%%%%%%%%%%%%%%%%%%%%%%%%%%%%%%%%%%%%%%%%%%%%%%%%%%%%%%%%%%%%%%%%%%%%

\begin{abstract}
    %\xuesu{jumped too quickly into the technical details here, would recommend to start with the big picture here. }
    This paper studies a team coordination problem in a graph environment. 
    Specifically, we incorporate ``support'' action which an agent can take to reduce the cost for its teammate to traverse some edges that have higher costs otherwise.
    Due to this added feature, the graph traversal is no longer a standard multi-agent path planning problem.
    % -- rather it must be treated as a version of MDP.
    % We provide an algorithm for converting a graph environment into a Joint State Graph (JSG) which transforms a multi-agent coordination system into a single-agent system in order to solve its path planning problem. 
    % We define the problem formulation as a base graph with a number of nodes, edges, and constant edge costs. 
    % Instead of solving an MDP, 
    To solve this new problem, we propose a novel formulation that poses it as a planning problem in the joint state space: the \emph{joint state graph} (JSG).
    Since the edges of JSG implicitly incorporate the support actions taken by the agents, we are able to now optimize the joint actions by solving a standard single-agent path planning problem in JSG.
    %We then add to this definition the action set that each agent can take and the corresponding cost for each agent to traverse an edge on the graph. We define the JSG and provide and algorithm for constructing it. 
    One main drawback of this approach is the curse of dimensionality in both the number of agents and the size of the graph.
    To improve scalability in graph size, we further propose a hierarchical decomposition method to perform path planning in two levels.
    % We then simplify the problem by decomposing the environment graph into the base and support graphs, which allows us to construct a Critical Joint State Graph (CJSG). 
    We provide complexity analysis as well as a statistical analysis to demonstrate the efficiency of our algorithm.
    % \sara{once we have the numerical results, I can finalize the abstract and conclusion sections}
    
\end{abstract}

%%%%%%%%%%%%%%%%%%%%%%%%%%%%%%%%%%%%%%%%%%%%%%%%%%%%%%%%%%%%%%%%%%%%%%%%%%%%%%%%