\section{Numerical experiments}
\label{app:experiments}

In this section, we illustrate some of the examples from Sections~\ref{sec:natural-exp-fam}, \ref{sec:general-exp}, and \ref{sec:outside-exp-fam} through numerical simulations.
Specifically, we thin a Gamma($\alpha, \theta$) distribution using the three different approaches described in Examples~\ref{ex:dtgamma}, \ref{ex:scaled-normal}, and \ref{ex:weibull}, a Beta($\theta, \beta$) into two non-identical beta random variables as described in Example~\ref{ex:beta}, and a Unif($0, \theta$) into scaled betas as described in Example~\ref{ex:scaled-uniform}.  We take $K=2$ throughout for ease of presentation.

\begin{figure}[!h]
\begin{center}
\includegraphics[width=\textwidth]{figures/numerical_example_figure.pdf}
\end{center}
\caption{Numerical examples of data thinning. The left-hand column displays a sample from $P_\theta$, which we wish to thin.  The center-left and center-right columns display the empirical distributions of $\Xt{1}$ and $\Xt{2}$ that result from thinning, overlaid with  the theoretical distributions $\Qt{1}_\theta$ and $\Qt{2}_\theta$. The right-hand column displays the empirical joint distribution of $(\Qt{1}_\theta(\Xt{1}), \Qt{2}_\theta(\Xt{2}))$, providing visual evidence that they are independent. With a slight abuse of notation, $Q_\theta^1(\cdot)$ and $Q_\theta^2(\cdot)$ represent the CDFs of their respective distributions.}
\label{fig:numericalexamples}
\end{figure}

In Figure \ref{fig:numericalexamples}, each row corresponds to one of the examples mentioned above. In the left-hand column,  we display the empirical density of $B=100,000$ realizations, $x_b$ (for $b=1,\ldots,B$), of the $X\sim P_\theta$ that we wish to thin, overlaid with the true density of $P_\theta$. In the center-left and center-right columns, we display $B$ realizations of $\Xt{1}$ and $\Xt{2}$ respectively, where each realization $(\xt{1}_b,\xt{2}_b)$ is obtained by sampling from $G_{x_b}$, the conditional distribution of $(\Xt{1},\Xt{2})$ given $T(\Xt{1}, \Xt{2})=x_b$. (This sampling is done without knowledge of $\theta$.) We overlay the densities of the marginals $\Qt{1}_\theta$ and $\Qt{2}_\theta$. The right-hand column displays 
the empirical joint distribution of $(\Qt{1}_\theta(\Xt{1}), \Qt{2}_\theta(\Xt{2}))$.  

In each case, our empirical findings corroborate our theoretical results: we see that the empirical distribution of $X \sim P_\theta$ agrees with its theoretical density (left-hand column); that the empirical distributions of $\Xt{1}$ and $\Xt{2}$ sampled from $G_X$ coincide with $\Qt{1}_\theta$ and $\Qt{2}_\theta$ (even though the empirical distributions were obtained without knowledge of $\theta$; center-left and center-right columns); and that the joint distribution of $\Qt{1}_\theta(\Xt{1})$ and $\Qt{2}_\theta(\Xt{2})$ resembles the independence copula (right-hand column). 
