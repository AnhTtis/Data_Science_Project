\section{Changepoint detection simulations} \label{app:changepoint}
First, we generate data with a common variance, specifically $X_1,\ldots,X_{2000}\overset{\text{iid}}{\sim} N(0,1)$, and apply the two approaches to detecting and testing for a change in variance that were described in Section \ref{sec:changepoint}.  
We repeat this process 1000 times, and display the type 1 error rate in Figure \ref{fig:simulation}. The naive approach does not control the type 1 error rate, while the thinning-based approach does. 

\begin{figure}[!h]
\begin{center}
\includegraphics[width=0.8\textwidth]{figures/simulation_T1E_changepointGDT.pdf}
\end{center}
\caption{Type 1 error rate of naive and thinning-based approaches to testing for a change in variance, in a setting where the variance is truly constant.}
\label{fig:simulation}
\end{figure}

 Figure \ref{fig:null} displays the estimated changepoints, as well as those for which we rejected the null hypothesis of no change in variance, for a single realization of the simulated data. The naive approach resulted in a large number of false positives, while the data thinning approach did not. 

\begin{figure}[!h]
\begin{center}
\includegraphics[width=\textwidth]{figures/simulation_null_changepointGDT.pdf}
\end{center}
\caption{A simulated dataset with constant variance. \emph{Top row:} Red lines indicate estimated changepoints using all of the data. Asterisks indicate the estimated changepoints for which we rejected the null hypothesis at level $\alpha = 0.05$ using the naive approach that estimates and tests changepoints using all of the  data. We (falsely) rejected the null hypothesis for 20 out of 42 estimated changepoints. \emph{Bottom row:} We applied data thinning to the data shown in the top row, to obtain equally-sized training and test sets. We estimated changepoints on the training set (blue lines) and tested them using the test set. This led to zero  rejections of the null hypothesis at level  $\alpha=0.05$. }
\label{fig:null}
\end{figure}

Next, we generated data with two true changepoints: for $i = 1, \dots 500$, $X_i \overset{\text{iid}}{\sim} N(0,4)$; for $i = 501, \dots, 1500$, $X_i \overset{\text{iid}}{\sim} N(0,25)$; and for $i = 1501, \dots, 2000$, $X_i \overset{\text{iid}}{\sim} N(0,1)$. We again apply the two approaches to detecting and testing for a change in variance that were described in Section \ref{sec:changepoint}, and display the results in Figure \ref{fig:alt}. Data thinning rejects the null hypothesis of no change in variance at only three timepoints, which are located very close to the two true changepoints. However, the naive approach results in a much larger number of false rejections.

\begin{figure}[!h]
\begin{center}
\includegraphics[width=\textwidth]{figures/simulation_alt_changepointGDT.pdf}
\end{center}
\caption{A simulated dataset with a change in variance at timepoints 501 and 1501. \emph{Top row:} Red lines indicate estimated changepoints using all of the data. Asterisks indicate the estimated changepoints for which we rejected the null hypothesis at level $\alpha = 0.05$ using the naive approach that estimates and tests changepoints using all of the  data.  \emph{Bottom row:} We applied data thinning to the data shown in the top row, to obtain equally-sized training and test sets. We estimated changepoints on the training set (blue lines) and tested them using the test set. This led to only three  rejections of the null hypothesis at level  $\alpha=0.05$, which are located very close to the two true changepoints.}
\label{fig:alt}
\end{figure}