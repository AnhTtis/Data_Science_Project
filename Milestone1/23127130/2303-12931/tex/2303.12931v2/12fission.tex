\section{Relaxing the independence assumption} \label{app:fission}

We now consider how the generalized data thinning recipe changes if we relax the independence requirement for $\Xt{1}$ and $\Xt{2}$.

\begin{algorithm}[Finding distributions that can be decomposed into non-independent components]
\label{alg:recipe-fission}
\textcolor{white}{.}
\begin{enumerate}
    \item Choose a family of distributions $\mathcal Q=\{Q_\theta:\theta\in\Omega\}$ over $(\Xt{1},\Xt{2})$, 
     where $\Xt{1}$ and $\Xt{2}$ are not necessarily independent.
    \item Let $(\Xt{1},\Xt{2})\sim Q_\theta$, and let  $T(\Xt{1},\Xt{2})$ denote a sufficient statistic for $\theta$.
    \item Let $P_\theta$ denote the distribution of $T(\Xt{1},\Xt{2})$.  
\end{enumerate}
\end{algorithm}
Then, given $X \sim P_\theta$, we can generate $(\Xt{1},\Xt{2})$ by sampling from $G_X$, where $G_t$ is defined as the conditional distribution
    $$
    (\Xt{1},\Xt{2})|T(\Xt{1},\Xt{2})=t.
    $$

By sufficiency, the sampling mechanism $G_t$  can be performed without knowledge of $\theta$.
The key point here is that the main ideas in this paper apply even if $\Xt{1}$ and $\Xt{2}$ are dependent; however, we focused on independence in this paper to facilitate downstream application of the decompositions that we obtain. 



