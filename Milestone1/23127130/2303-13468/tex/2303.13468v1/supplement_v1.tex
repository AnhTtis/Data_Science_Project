\documentclass[aps,prl, onecolumn, superscriptaddress]{revtex4-2}

\usepackage{amsmath}
\usepackage{times}
\usepackage{amssymb}
\usepackage{graphicx}
\usepackage[caption=false, position=top, singlelinecheck=off, justification=raggedright]{subfig}
\usepackage{color}
\usepackage{pst-node}
\usepackage[unicode]{hyperref}

\def\black{\color{black}}
\def\clr{\color{red}}
\begin{document}
\title{Supplemental Material for \\ Quantum rotation sensor with real-time readout based on an atom-cavity system}

\author{Jim Skulte}
\affiliation{Zentrum f\"ur Optische Quantentechnologien and Institut f\"ur Laser-Physik, Universit\"at Hamburg, 22761 Hamburg, Germany}
\affiliation{The Hamburg Center for Ultrafast Imaging, Luruper Chaussee 149, 22761 Hamburg, Germany}

\author{Jayson G. Cosme}
\affiliation{National Institute of Physics, University of the Philippines, Diliman, Quezon City 1101, Philippines}

\author{Ludwig Mathey}
\affiliation{Zentrum f\"ur Optische Quantentechnologien and Institut f\"ur Laser-Physik, Universit\"at Hamburg, 22761 Hamburg, Germany}
\affiliation{The Hamburg Center for Ultrafast Imaging, Luruper Chaussee 149, 22761 Hamburg, Germany}
\date{\today}

\maketitle

\section{Particle number fluctuations}
To further show the applicability of our detector we assume to have a gaussian particle number distribution with $N=40000$ and $\sigma=0.1 \times N$. SFig.~\ref{Sfig:1} we use the same protocol and parameters as discussed in the main text Fig.3(b), but with the new particle number distribution. The change of the rotational frequency can still be distinguished from the fluctuations.


\begin{figure}[!htpb]
\centering
\includegraphics[width=0.7\columnwidth]{Sfig1.pdf}
\caption{(a) In blue the applied phase $\theta$ modulated using a sinusoidal drive with strength $\delta_\theta = \pi/20$ and frequency $\omega_\mathrm{dr} = 2\pi \times 0.5~\mathrm{kHz}$. The dark red line shows the light field intensity derived from $10^3$ TWA trajectories. The light red shaded area shows the corresponding standard deviation due to quantum fluctuations within TWA. The system is initialized at rest before the drive starts. (b) Same protocol as in (a) but the system is initialized with a phase of $\theta=\pi/4$. In both cases the amplitude change of $\Omega$ corresponds to $\omega_\mathrm{rec}/(20 \pi n_\mathrm{s})=350~\mathrm{Hz}/n_\mathrm{s}$ and we choose $g=1.09 g_\mathrm{0,crit}$.} 
\label{Sfig:1} 
\end{figure}

\black
\bibliography{references_sensor_supp}
\end{document}