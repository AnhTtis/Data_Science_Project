\section{Conclusion and Future Work}
\label{sec:conclusion}
In this work, we propose an API sequence recommendation technique, \method which takes a natural language description of a programming task as input. 
It leverages a bi-encoder to conduct contrastive learning and a cross-encoder to build a classification model for finding the most semantically similar \so given an annotation (i.e., code comment).
\textcolor{black}{\method internally utilizes the title and APIs from similar \so post as a query expansion to fine-tune a pre-trained model resulting in a better API sequence generation model.}
% We aim to retrieve a suggested sequence of Java API calls that can perform that task.
Based on our experimental results, we show that the title and APIs are beneficial to boost the effectiveness of API sequence recommendation.
Specifically, \method improved approximately 11\% of the BLEU-4 score for the sequence recommendation.


In the future, we plan to investigate whether \so posts can help boost the performance of the API recommendation in other programming languages, such as Python.
We also plan to utilize the information from \so in other ways beyond query expansion. 
\textcolor{black}{Furthermore, it is also interesting to explore the usage of other parts of \so post such as its body to boost performance further. It may be beneficial as \so body contains detailed information about the post.}

\textcolor{black}{
Moreover, we plan to investigate how to use Stack Overflow posts to help with tasks beyond API recommendation, particularly those that involve natural language queries, such as bug report related tasks~\cite{wang2014version}. We also plan to investigate the challenges that developers face when using API recommendation tools, following prior works for other automated solutions, e.g., ~\cite{kochhar2015understanding, zou2018practitioners}.}

\textcolor{black}{\textbf{Availability.} Our replication package is publicly available at \url{https://github.com/soarsmu/Picaso}.}