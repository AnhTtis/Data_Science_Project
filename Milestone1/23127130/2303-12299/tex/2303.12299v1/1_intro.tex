\section{Introduction}
\label{sec:intro}
% 1.5 pages
% what is API, why it is important and why we need API recommendation
In modern software development, Application Programming Interfaces (APIs) and libraries have been widely used by developers to improve development efficiency.
Considering the large volume of existing APIs, multiple APIs can be used to implement the same functionality.
For instance, to read a file in Java using the JDK library, a developer could use either FileReader, BufferedReader, Files, Scanner, etc., depending on the size of the file and type of the data that need to be read (i.e., byte or characters).
While having many options is liberating, the problem of choosing the right API for a certain functionality arises.
Given the enormous number of available APIs, it is challenging for developers to find and learn which API to use along with the complementary APIs needed in the \textcolor{black}{API invocation} sequence~\cite{peng2022revisiting}.
Therefore, it is desirable to have an API recommendation system that can recommend the correct APIs to use~\cite{deepapi,fowkes2016parameter}.


% query-based API recom?
To support developers in finding the correct APIs, several API recommendation methods have been proposed~\cite{huang2018api,rahman2016rack,wei2022clear,chen2021holistic}.
Depending on the input type, these methods can be categorized into query-based or code-based API recommendation~\cite{peng2022revisiting}.
In this work, we mainly focus on the former type of API recommendation. 
We are concerned with how to accurately translate a natural language query that describes the intention of the programming task to an API sequence that implements the requirements.
While many query-based API recommendation approaches have been proposed~\cite{wei2022clear,huang2018api,rahman2016rack}, only a limited number of them recommend API sequence~\cite{deepapi,martin2022deep}. 
The best of which is the work by Martin and Guo~\cite{martin2022deep} that leverages CodeBERT, a bimodal pre-trained model for programming language and natural language, to generate API sequences. 

% Motivation of Stack Overflow as an external resource
In practice, to search for suitable APIs, developers tend to resort to \so~\cite{huang2018api}.
% Nowadays, developers tend to resort to \so when they seek suitable APIs to fulfill the programming task~\cite{huang2018api}. 
Prior studies~\cite{wei2022clear,huang2018api,rahman2016rack} have shown that \so can be effectively utilized to recommend a set of APIs as it contains tremendous crowd-knowledge.  
Additionally, diversifying the resources for \textcolor{black}{training an API recommendation model} has been investigated by researchers~\cite{wang2018entagrec++,wei2022clear,ponzanelli2014prompter, zhou2021boosting} and verified to be beneficial.

Previous studies\cite{huang2018api,wei2022clear} utilize \so data by finding similar \so posts and recommending curated APIs mentioned in the answers.
Their objective is to correctly identify the similar \so posts for a question that a developer comes up with.
However, developers' need for good API recommendations goes beyond a question when they face difficulties.
On a daily basis, they implement functions to perform certain tasks that could be summarized in the form of natural language (NL), i.e., code comment/annotation.
Oftentimes, the NL representation of the code could be paired with a \so post (example presented in Section \ref{sec:motivating-example}). This motivates us to explore the potential of leveraging \so information in search of generating better API sequence recommendations.

% TODO IVANA: give example for this at introduction
% \kisub{Shall we give an example of a similar post? so that the readers can understand why it can be supportive for finding alternative APIs via expanding the user query.}

% However, it is a common case that the required APIs for a specific task are different from the ones suggested in the posts. 


% Therefore, \method takes the similar \so posts as one of the resources to expand the user query instead of directly retrieving the APIs from such posts.
\textcolor{black}{
With a goal to ease developers' daily tasks, we propose \method (Enhancing A\underline{PI} Re\underline{c}ommend\underline{a}tions with Relevant \underline{S}tack \underline{O}verflow Posts), an API recommendation technique that leverages multi-source (i.e., \so and GitHub) information via query expansion.
\method takes a user query as either code comment or annotation, then finds the \so post that is most similar to the query. 
Once it discovers a similar post, the post is delivered to the CodeBERT encoder as query expansion to generate the API sequence that implements the functionality described in the query.
To find the most similar \so post for a given annotation, we adapted a framework \textcolor{black}{proposed by Wei et al.}~\cite{wei2022clear}, who propose a state-of-the-art technique in finding similar \so posts for a \so query. The framework consists of a filtering and a re-ranking model.
The filtering model is built on top of the contrastive learning method, while the re-ranking model is trained by performing joint-embedding training.}
% Different from other studies, \method additionally leverages contrastive learning~\cite{oord2018representation} to build a model that embeds the query and the \so post, which are related to a similar set of APIs, to vectors that are near to each other.
% \kisub{The last sentence needs to be rephrased. will come back.}
% The contrastive learning model finally ranks the posts by calculating the vector similarities (i.e., between the query vector and post vectors). 
% It then re-ranks the top-$n$ most similar posts using a cross-encoder~\cite{} that has been trained to classify whether a query and a post have a similar set of APIs.
% \method utilizes only the top post to expand the query. 


% Inspired by the phenomenon, we propose \method, that aims to similarly utilize \so to better recommend a sequence of APIs.

% As such, instead of recommending the mentioned APIs, \method treats similar \so posts as the source of query expansion.



% In a nutshell, \method leverages multi-source information from Stack Overflow and GitHub to better recommend API sequences. \method takes a query (i.e., code comment/annotation) and finds the \so post that is most similar to the query. It then takes the title of the similar post to expand the query, and input it to CodeBERT, which has been trained on GitHub repositories, to generate the API sequence that implements the functionality described in the query. 

% To find the similar post, \method leverages contrastive learning to build a model that embeds the query and the SO post, which are related to a similar set of APIs, to vectors that are near to each other. Using this model, the \so posts are ranked based on the cosine similarity of their embeddings to the query embedding. \method reranks the top-$n$ most similar posts using a cross-encoder that has been trained to classify whether a query and a post has the similar set of APIs. \method uses the resultant top-1 post to expand the query. 

%Leveraging \so posts as the expansion of a query is under-explored.
%Instead of fully recommending all the APIs mentioned in the \so posts, in this work, we are curious about exploring an alternative way to utilize \so posts.
%Furthermore, since the easier access to a large amount of data, it would be 
% domain-knowledge?

We evaluated \method on a dataset derived from the work of Martin and Guo~\cite{martin2022deep}. \textcolor{black}{\method manages to outperform the state-of-the-art API sequence recommendation model by 10.8\% in terms of BLEU-4 score.}
% \textcolor{blue}{0.29} in terms of BLEU-4 score, which improves upon the state-of-the-art approach by 10.8\%. 

Our contributions can be summarized as follows:
\textcolor{black}{
\begin{itemize}
    \item{We are the first to show that \so posts can boost the effectiveness of the query-based API sequence recommendation, particularly a query expansion.}
    \item{We propose \method, a multi-source API sequence recommendation method that can outperform the state-of-the-art approach.}
    %We propose \method, which utilizes multi-source information from GitHub and Stack Overflow to recommend API sequences.}
    \item{We conduct experiments to demonstrate the effectiveness of \method, achieving a \textcolor{black}{10.8\% improvement measured in BLEU-4 score.}}
\end{itemize}
}

The remainder of the paper is organized as follows. 
Section~\ref{sec:background} introduces the problem formulation and a motivating example. 
We describe our approach in Section~\ref{sec:approach}. 
Section~\ref{sec:setting} \textcolor{black}{presents} our experimental setup and the research questions. 
Experimental results are \textcolor{black}{described} in Section~\ref{sec:result}. 
We discuss the results in Section~\ref{sec:discussion}. 
Section~\ref{sec:related} \textcolor{black}{presents} the related works. 
We conclude our work \textcolor{black}{and present} future work in Section~\ref{sec:conclusion}.

% Several have shown that similar \so posts contain similar APIs in their answers~\cite{wei2022clear,huang2018api}.
% In their setting, the query is the title of a \so post.
% Although RACK~\cite{rahman2016rack} show that \so posts are helpful, they only work in class-level API recommendation.
% However, there is no work on leveraging \so posts to boost the performance of query-based API recommendation at the method level.
% We fill this gap by exploring leveraging \so posts to recommend APIs.


% Query Expansion
% A recent survey by Peng et al.~\cite{peng2022revisiting} shows that query reformulation techniques, including query expansion and query modification, can boost the accuracy of API recommendations.

% when the title of \so posts are the query.
% It stays unknown whether it is true when the query is the annotation other than being the \so question itself.

% Contrastive learning

% API sequence generation model
