% revtex 4.2 APS template to start with
\documentclass[reprint,aps,superscriptaddress,footinbib,prl]{revtex4-2}
\usepackage{amsmath}
\usepackage{physics}
\usepackage{graphicx}
\usepackage{amssymb}
\usepackage{amsthm}
\usepackage{bm}
\usepackage{txfonts}
\usepackage{pstricks}
\usepackage[colorlinks=true,allcolors=blue]{hyperref} 
\usepackage[capitalise]{cleveref}
\allowdisplaybreaks

\newcommand{\vk}{{\bm{k}}}
\newcommand{\vq}{{\bm{q}}}
\newcommand{\vp}{{\bm{p}}}
\newcommand{\vm}{{\bm{m}}}
\newcommand{\kF}{{k_\mathrm{F}}}
\definecolor{mygreen}{rgb}{0.15, 0.6, 0.15}
\definecolor{mygrey}{rgb}{0.5, 0.5, 0.5}
\newcommand*\grey[1]{\textcolor{mygrey}{#1}}
\newcommand*\tcb[1]{\textcolor{red}{[TCB: #1]}}
\newcommand*\van[1]{\textcolor{mygreen}{[Verena: #1]}}
\newcommand*\gvv[1]{\textcolor{mygreen}{#1}}

\begin{document}
\title{Highly accurate electronic structure of metallic solids from coupled-cluster theory with nonperturbative triple excitations}
\author{Verena A. Neufeld}
\affiliation{Department of Chemistry, Columbia University, New York, New York 10027, USA}
\author{Timothy C. Berkelbach}
\affiliation{Department of Chemistry, Columbia University, New York, New York 10027, USA}
\date{\today}  % https://tex.stackexchange.com/questions/165991/how-to-use-date-today-without-using-maketitle

\begin{abstract}
Coupled-cluster theory with single, double, and perturbative triple excitations
(CCSD(T))---often considered the ``gold standard'' of main-group quantum
chemistry---is inapplicable to three-dimensional metals due to an infrared
divergence, preventing its application to many important problems in materials
science.
We study the full, nonperturbative inclusion of triple excitations (CCSDT) and
propose a new, iterative method, which we call ring-CCSDT, that resums the
essential triple excitations with the same $N^7$ run-time scaling as CCSD(T).
CCSDT and ring-CCSDT are used to calculate the correlation energy of the
uniform electron gas at metallic densities and the structural properties of
solid lithium.  Inclusion of connected triple excitations is shown to be
essential to achieving high accuracy.  We also investigate semiempirical CC
methods based on spin-component scaling and the distinguishable cluster
approximation and find that they enhance the accuracy of their parent
\textit{ab initio} methods.
\end{abstract}

\maketitle

%\section{Introduction}
\textit{Introduction.}
Accurately predicting energetic properties of metallic solids is crucial in
computational materials science, with applications in heterogeneous
catalysis, electrochemistry, and battery 
science~\cite{norskov_density_2011,calle-vallejo_first-principles_2012,he_density_2019}.
Coupled-cluster theory with single and double excitations
(CCSD)~\cite{shavitt_many-body_2009,bartlett_coupled-cluster_2007} has recently
been shown to provide reasonable energies for the uniform electron gas
(UEG)~\cite{shepherd_communication_2016,mihm_power_2021,callahan_dynamical_2021}
and for atomistic metallic solids, such as lithium and
aluminum~\cite{stoll_incremental_2009,mihm_shortcut_2021,neufeld_ground-state_2022,weiler_machine_2022},
but it does not reliably outperform density functional theory (DFT), which is
significantly cheaper---some inclusion of connected triple excitations is clearly required.
For non-metallic main-group solids, CCSD with perturbative triple excitations
(CCSD(T))~\cite{raghavachari_fifth-order_1989} is highly accurate
for bulk properties~\cite{schwerdtfeger_convergence_2010,booth_towards_2013,gruneis_coupled_2015,gruber_applying_2018,gruber_ab_2018} and surface
chemistry~\cite{tsatsoulis_comparison_2017,gruber_ab_2018,tsatsoulis_reaction_2018,brandenburg_physisorption_2019,lau_regional_2021},
mirroring its performance on molecules, where it commonly yields ``chemical accuracy'' of
about 1~kcal/mol~\cite{bartlett_coupled-cluster_2007}.
However, CCSD(T) is not expected to be applicable to three-dimensional metals: 
an approximate evaluation of the CCSD(T) energy of the UEG was shown to diverge in 
the thermodynamic limit~\cite{shepherd_many-body_2013}, similar to the textbook result of second-order
perturbation theory~\cite{macke_uber_1950,gell-mann_correlation_1957}.

Here, we investigate the accuracy of CC theory with nonperturbative triple
excitations (CCSDT)
to determine whether such a theory provides the desired accuracy for metals.
Because the high cost of CCSDT limits its routine application, we also design and test lower cost
alternatives.
Below, we first review diagrammatic results on the ground-state
energy of the UEG, including its high-density expansion, divergences and necessary
resummations, and connections with coupled-cluster theory including double and triple excitations.
An analysis of the (T) correction for the UEG motivates a new theory, which nonperturbatively
retains the triple excitations necessary to preclude a divergence and which has the same
$N^7$
computational scaling as CCSD(T).
We assess the performance of these methods with applications to the UEG at metallic densities
and to solid lithium.
Furthermore, we test several empirical modifications, including the
distinguishable cluster (DC)
approximation~\cite{kats_communication_2013,kats_distinguishable_2019,rishi_can_2019}
and spin-component-scaled (SCS) CC
theory~\cite{grimme_improved_2003,takatani_improvement_2008,kats_improving_2018},
which were designed to approximate the effect of higher excitations without
increasing the computational cost.

%\section{Diagrammatic results on the uniform electron gas}
\textit{Diagrammatic results on the uniform electron gas.}
%\label{sec:review}
The UEG, a model of interacting electrons in a uniform positive background, 
has been a famous testing ground for new developments in
nonperturbative many-body quantum field theory.  Specifically, the total energy
of the UEG with electron density $n$ has been evaluated to
leading orders in the Wigner-Seitz radius 
$r_s = (3/(4\pi n))^{1/3}$~\cite{gell-mann_correlation_1957,carr_ground-state_1964,Endo1999},
in the absence and presence of a spin polarization; in this work, we focus on the upolarized
case.
The kinetic energy and Hartree-Fock exchange energy produce terms
of $O(r_s^{-2})$ and $O(r_s^{-1})$, respectively, and
the remaining terms define the correlation energy.

From dimensionality arguments, it is expected that second-order perturbation theory
contributes all terms of $O(r_s^0)$, which is correct for the second-order
exchange energy~\cite{gell-mann_correlation_1957,onsager_integrals_1966}.
The second-order direct (ring) term, whose diagram is shown in
Fig.~\ref{fig:diagrams}(a), contributes a correlation energy
$E_{24} \propto r_s^0 \int_0^\infty dq f(q)/q^2$, where
\begin{equation}
f(q) = \int_{|\vk+\vq|>1}d^3k \int_{|\vp+\vq|>1}d^3p
    \frac{\theta(1-k)\theta(1-p)}{q^2 + (\vk+\vp)\cdot\vq}
\end{equation}
and all dimensionless momenta $\vk, \vp, \vq$ are normalized to the Fermi momentum;
we use the notation $E_{m,2n}$ from Refs.~\onlinecite{carr_ground-state_1964,Endo1999},
where $m$ is the order in perturbation theory and $n$ is the number of interactions with
the same momentum transfer.
It can be shown that $f(q) \propto q$ in the limit $q\rightarrow 0$, and thus
the second-order direct term famously diverges logarithmically.  All higher order terms
with the same ring structure ($n$ rings at order $n$ in
perturbation theory), such as the one shown in Fig.~\ref{fig:diagrams}(b) (i.e., $E_{36}$) 
exhibit the strongest divergences at each order, and their resummation to infinite order
defines the random-phase approximation
(RPA)~\cite{macke_uber_1950,bohm_collective_1951,pines_collective_1952,bohm_collective_1953,gell-mann_correlation_1957,carr_ground-state_1964,Endo1999},
$\varepsilon' = E_{24} + E_{36} + E_{48} + \cdots$.
The RPA provides a correlation energy that is correct to $O(\ln r_s)$ and is
therefore exact in the high-density $r_s\rightarrow 0$ limit (aside from a constant); 
the appearance of terms $O(\ln r_s)$ in the density expansion signals the non-analyticity of the
correlation energy.  As is well-known, the CCSD energy contains all terms
included in the
RPA~\cite{freeman_coupled-cluster_1977,bishop_electron_1978,scuseria_ground_2008},
providing a strong theoretical argument for the application of CC theories to
metallic solids---a research agenda started more than 40 years
ago~\cite{freeman_coupled-cluster_1977,bishop_electron_1978,bishop_electron_1982,emrich_electron_1984}.
\begin{figure}[t]
	\includegraphics[width=3.25in]{diagrams.pdf}
	\caption{
            Goldstone diagrams discussed in the text, which are included at various orders in perturbation
            theory and various flavors of CC theory. The dashed red box in (d) and (e) highlights the problematic
            feature responsible for the divergence of the CCSD(T) correlation energy.
        }
	\label{fig:diagrams}
\end{figure}


%\subsection{Third-order perturbation theory and CCSD}

Third-order perturbation theory produces 
convergent terms that are $O(r_s)$ (i.e., $E_{32}$),
strongly divergent terms with three rings that are included in the RPA [i.e., Fig.~\ref{fig:diagrams}(b)
or $E_{36}$],
and more weakly divergent terms whose diagrams have only one ring,
such as that shown in Fig.~\ref{fig:diagrams}(c), which define $E_{34}$.
These latter terms have to be resummed with higher-order divergent contributions
that have analogous structure ($n-2$ rings at order $n$ in perturbation theory), 
$\varepsilon'' = E_{34} + E_{48} + \cdots$,
which can be evaluated to identify a correlation energy that is exact
to $O(r_s, r_s\ln r_s)$~\cite{dubois_electron_1959,carr_ground-state_1964,Endo1999}.
Remarkably, all of these terms are included in the CCSD correlation energy.
Although it has long been appreciated that CCSD resums the most divergent
terms that define the RPA correlation energy $\varepsilon'$~\cite{freeman_coupled-cluster_1977,bishop_electron_1978,scuseria_ground_2008}, to the
best of our knowledge, it has not been noted that it also resums these
next most divergent terms that define $\varepsilon''$.
Therefore, CCSD is exact for the energy of the UEG to $O(r_s, r_s\ln r_s)$,
which is one order higher than the RPA, in addition to recovering the correct
constant term due to second-order exchange.
%\subsection{Fourth-order perturbation theory and CCSDT}

As expected, the CCSD energy is missing terms from fourth order in perturbation
theory, including those that yield finite values of $O(r_s^2)$ or that diverge
weakly and must be resummed with higher-order terms. CCSDT produces an energy that
is exact to fourth order in perturbation theory and includes resummations necessary to
eliminate fourth-order divergences, thus providing
a potentially powerful theory of the energy of metals.  However, CCSDT has a high
computational cost that scales as $N^8$, which precludes routine application to
atomistic materials.
Nonetheless, below we exploit the simplicity of the UEG and carefully
designed composite corrections to provide the first estimates of the
performance of CCSDT for the UEG in the thermodynamic limit and for solid lithium. 

The intermediate theory CCSD(T), with a reduced $N^7$
scaling, is very accurate for many molecules and insulating solids. However,
CCSD(T) yields a divergent energy for metals, which was demonstrated
numerically using an approximate form in
Ref.~\onlinecite{shepherd_many-body_2013}.  Here, we provide a diagrammatic analysis of
the same behavior to shed more light on the failures of CCSD(T).
Neglecting single excitations, which vanish for the UEG by symmetry, the energy
correction in CCSD(T) is shown by the diagram in Fig.~\ref{fig:diagrams}(d)
(plus permutations due to exchange),
where the double line indicates a converged CCSD $T_2$ amplitude.  To lowest order,
the (T) correction is that of bare fourth-order perturbation theory, shown in
Fig.~\ref{fig:diagrams}(e), whose analysis elucidates the (T) divergence.
Considering only the contribution without exchange, the problematic process
has four interactions with two pairs of identical momenta exchanged, $\vq$ and $\vq'$,
i.e., the correlation energy is
$E_\mathrm{c} \propto r_s^2 \int d^3 q \int d^3 q' f(\vq,\vq')/(q^4 q'^4)$,
where
\begin{widetext}
\begin{equation}
\label{eq:e4}
f(\vq,\vq') = \int_{|\vk+\vq|>1}d^3k \int_{|\vm+\vq'|>1}d^3m
	\int_{\substack{|\vp+\vq|>1\\|\vp-\vq'|<1}}d^3p 
	\frac{\theta(1-k)\theta(1-p)\theta(1-m)}
        {[q^2 + (\vk+\vp)\cdot\vq]^2 [q^2 + (\vk+\vp)\cdot\vq + (\vm+\vp)\cdot\vq']}.
\end{equation}
\end{widetext}
As usual, the correlation energy integral diverges due to the behavior of the 
integrand near $q,q'=0$.
Letting $q_c$ be an infrared cutoff on both momentum integrals, the integrated
result can be checked to diverge
as $O(q_c^{-2} \ln q_c)$, demanding resummation with higher-order terms.

By replacing the outer Coulomb interactions by CCSD $T_2$ amplitudes as in
Fig.~\ref{fig:diagrams}(d), CCSD(T) regularizes the integral over $\vq$, but
not $\vq'$.  This single ring diagram self-energy insertion, highlighted with a
red box in Figs.~\ref{fig:diagrams}(d) and (e), is responsible for the
divergence of the CCSD(T) energy for metals. By analytically performing this
regularization, the CCSD(T) energy can be shown to diverge as $O(\ln q_c)$,
which is naturally weaker than that of bare fourth-order perturbation theory,
but still useless for quantitative calculations. 
This rate of divergence is exactly the same as that of second-order
perturbation theory, which we exploit in the Supplemental
Material~\footnote{See Supplemental Material at [URL will be inserted by
publisher] for technical details of all calculations and a discussion
of finite-size errors in metals, including numerical demonstration of
convergent and divergent behaviors of the considered theories.} 
to numerically confirm the divergence of CCSD(T), along the lines of
other works~\cite{shepherd_range-separated_2014,shepherd_coupled_2014}.

Importantly, this analysis also identifies the minimal physics necessary to
regularize the CCSD(T) approximation for metals, which is an infinite-order
RPA-style resummation of ring diagrams in the self-energy insertion (like in
the GW approximation~\cite{hedin_new_1965}), as shown in
Fig.~\ref{fig:diagrams}(f).  This can be achieved approximately by removing
many of the terms from the CCSDT equations, analogous to the equivalence
between (direct) ring-CCD and the RPA.  This method, which we call ring-CCSDT,
is implemented as follows.  The singles and doubles amplitude equations are
exactly as in CCSDT.  The triples amplitude equation is the same as in the
CCSDT-1
approximation~\cite{lee_study_1984,lee_coupled_1984,urban_towards_1985,noga_towards_1987,shavitt_many-body_2009},
but is supplemented with direct ring diagrams,
$\epsilon_{ijk}^{abc} t_{ijk}^{abc}  = R_{\mathrm{CCSDT-1}} +  R_{\mathrm{dr}}$,
\begin{subequations}
\begin{align}
\begin{split}
&R_{\mathrm{CCSDT-1}} = 
\hat{P}(k/ij|a/bc) \langle bc || dk \rangle t_{ij}^{ad} 
- \hat{P}(i/jk|c/ab) \langle lc||jk \rangle t_{il}^{ab} \\
     &\hspace{1em}+ 
\hat{P}(c/ab) (f_{cd} - \epsilon_c \delta_{cd}) t_{ijk}^{abd} 
- \hat{P}(k/ij) (f_{lk} - \epsilon_k \delta_{lk}) t_{ijl}^{abc} 
\end{split} \\
\begin{split}
&R_{\mathrm{dr}} = \hat{P}(i/jk|a/bc) \bra{al}\ket{id} t_{ljk}^{dbc} 
+ \hat{P}(i/jk|abc) \bra{lb}\ket{de} t_{il}^{ad} t_{jk}^{ec} \\
    &\hspace{1em} 
- \hat{P}(ijk|a/bc) \bra{lm}\ket{dj} t_{il}^{ad}t_{mk}^{bc}
+ \hat{P}(i/jk|a/bc) \bra{lm} \ket{de} t_{il}^{ad} t_{mjk}^{ebc}
\end{split}
\end{align}
\end{subequations}
where
$\hat{P}(k/ij|a/bc) = [1 - \hat{P}(ik) - \hat{P}(jk)][1 - \hat{P}(ab) - \hat{P}(ac)]$
and $\hat{P}(ij)$ generates the permutation of $i$ and $j$
(as usual, $\varepsilon_{ijk}^{abc}$ is an orbital energy denominator, 
$f_{pq}$ is a Fock matrix element,
$i,j,k,l,m$ indicate occupied spin orbitals, $a,b,c,d,e$ unoccupied
spin orbitals, Coulomb integrals are in $\langle 12|12\rangle$ notation,
the double bar indicates antisymmetrized integrals, and summation over
repeated indices $l,m,d,e$ is implied).

Unfortunately, despite its iterative nature,
the CCSDT-1 approximation (without the ring diagrams)
is a divergent theory of metals, like CCSD(T),
because of the isolated ring diagram highlighted in Figs.~\ref{fig:diagrams}(d)
and (e).
In the ring-CCSDT approximation,
not all time-orderings of repeated ring diagrams are included: all forward (Tamm-Dancoff)
time-orderings are included, which is sufficient to preclude a divergence~\cite{freeman_coupled-cluster_1977},
and a subset of the non-Tamm-Dancoff time-orderings are included, but not
all those corresponding to the complete RPA; this is very similar to the diagrammatic
content of the coupled-cluster Green's function~\cite{lange_relation_2018,tolle_exact_2022}.
To include all time-orderings that define RPA screening would require inclusion
of connected quadruple excitations.

The first and last terms of $R_\mathrm{dr}$ exhibit $N^8$ computational scaling, like the
parent CCSDT method. However, the use of direct (non-antisymmetrized) ring 
diagrams enables a reduction in scaling with the use of density-fitting
(or Cholesky decomposition)
of the Coulomb integrals $\bra{pq}\ket{rs} = \sum_P L_{pr}^{P} L_{qs}^{P}$,
where $P$ is an auxiliary index.
For example, the last term can be constructed as
\begin{equation}
\sum_{lmde} \bra{lm} \ket{de} t_{il}^{ad} t_{mjk}^{ebc} 
= \sum_P \left[\sum_{ld} L_{ld}^P  t_{il}^{ad}\right] \left[ \sum_{me} L_{me}^P t_{mjk}^{ebc}\right].
\end{equation}
With such a compression of the Coulomb integrals, ring-CCSDT is an iterative
$N^7$ method, providing an appealing alternative to the CCSD(T) approximation
that is applicable to metals (although the storage of the $T_3$ amplitudes
is a separate bottleneck).

\begin{figure}[t]
	\includegraphics[width=3.25in]{ueg_rs.pdf}
        \caption{Ratio of the coupled-cluster correlation energy to the diffusion Monte
            Carlo (DMC) correlation
            energy~\cite{ceperley_ground_1980,perdew_self-interaction_1981} for
            the three-dimensional UEG with $r_s = 1$--5, as given by the
            methods indicated in the legend.
            The methods are separated into those that are purely 
            diagrammatic (left) and those that are semi-empirical (right).
            Range of chemical accuracy ($\pm$1~kcal/mol or $\pm$1.6~m$E_h$) is shown with a grey shaded area.
        }
	\label{fig:ueg_rs_frac}
\end{figure}

\textit{Results for the UEG.}
CC approximations are difficult to treat semi-analytically,
even for the UEG.  Therefore, we simulate a UEG of electron density $n$ via a cubic box
of $N$ electrons with volume $V=N/n=(4/3)\pi r_s^3N$ and a plane-wave orbital basis.  
Although several improved CC methods have been previously applied to UEG models
containing a finite number of
electrons~\cite{mcclain_spectral_2016,spencer_developments_2016,neufeld_study_2017,liao_towards_2021}, here
we are concerned with the critical question of their performance in the
thermodynamic limit, which we estimate via basis set corrections and
extrapolations to the thermodynamic limit. 
Specifically, we perform CCSD and DCSD calculations on systems
containing up to $N=1404$ electrons and estimate the complete basis set limit
using calculations on smaller system sizes. These results are then used to
extrapolate to the thermodynamic limit assuming that finite-size errors in the correlation
energy decay asymptotically as $N^{-2/3}$---a functional form that has also
been proposed in recent work~\cite{mihm_how_2023}. 
Our final CCSD correlation energies agree within about 1~m$E_h$ with previous
studies that targeted the thermodynamic
limit~\cite{shepherd_communication_2016,mihm_power_2021}, despite different
technical details, providing a validation of our methods.  
CCSDT, ring-CCSDT, and DCSDT calculations are performed on systems containing up to
$N=156$ electrons, and we calculate the energy difference with respect to DCSD.
The complete basis set limit of this energy difference is estimated based on smaller
values of $N$ and then extrapolated to the thermodynamic limit.
Additional technical details are given in the Supplemental Material~\cite{Note1}.

In Fig.~\ref{fig:ueg_rs_frac}, we present the correlation energy of the UEG at
metallic densities of $r_s=1$--5 from various CC theories 
as a fraction of the numerically exact result, estimated via the
Perdew-Zunger fit~\cite{perdew_self-interaction_1981} to diffusion Monte Carlo
(DMC) results~\cite{ceperley_ground_1980};
a table of all values is given in the Supplemental Material~\cite{Note1}.
The magnitude of the DMC correlation energy ranges from 60~m$E_h$ at $r_s=1$ to 28~m$E_h$
at $r_s=5$. 
As expected based on the density expansion discussed above, the relative
accuracy of diagrammatic methods shown in Fig.~\ref{fig:ueg_rs_frac}(a) (CCSD,
CCSDT, and ring-CCSDT) decreases with increasing $r_s$.
Compared to CCSD, which recovers only about 75--95\% of the DMC correlation energy,
CCSDT performs extremely well and recovers between 99\% (at $r_s=1$) and 92\% (at $r_s=5$),
corresponding to an absolute accuracy of 0.5--2.2~m$E_h$.
The good performance of ring-CCSDT, with errors of 0.9--3.0~m$E_h$, 
shows that the same ring diagram resummation responsible for curing the divergence
of CCSD(T) is also responsible for most of the correlation energy associated with
connected triple excitations.

The semi-empirical CC methods shown in Fig.~\ref{fig:ueg_rs_frac}(b) (SCS-CCSD, DCSD,
SCS-DCSD, and DCSDT) typically 
perform better than their parent diagrammatic method. SCS-CCSD~\cite{takatani_improvement_2008}
improves over CCSD, except at small $r_s$, demonstrating that semi-empirical modifications can spoil valuable
formal properties like the exactness of CC theories in the high-density limit.
DCSD~\cite{kats_communication_2013} is better behaved and roughly halves the error of CCSD over this density range.
SCS-DCSD~\cite{kats_improving_2018} is a further improvement and provides the best overall performance of the
$N^6$ scaling methods. Remarkably, DCSDT~\cite{kats_distinguishable_2019,rishi_can_2019}
yields results of extremely high accuracy,
recovering more than 94\% of the DMC correlation energy at all densities, which
corresponds to an error of less than 1.6~m$E_h$, i.e., under 1~kcal/mol.

%\subsection{Solid lithium}
\textit{Results on solid lithium.}
Next, we investigate the transferability of the above performance to a real
material.  We study solid lithium, which is a simple metal with a valence
electron density corresponding to $r_s \approx 3.2$.  We use CCSD, DSCD,
ring-CCSDT, CCSDT, and DCSDT to calculate the equilibrium lattice parameter,
bulk modulus, and cohesive energy.  All calculations were performed with a
development branch of
PySCF~\cite{sun_libcint_2015,sun_pyscf_2018,sun_recent_2020}, and all technical
details---such as pseudopotentials, basis sets (up to quadruple-zeta Gaussian
type orbitals), and Brillouin zone samplings (up to 64 $k$-points, plus
extrapolation)---are the same as in our previous
work~\cite{neufeld_ground-state_2022}; in that work, we found that CCSD
predictions had significant room for improvement (at the CCSD level, we find
that our updated finite-size extrapolations cause only small differences from
our previous work, e.g., about 0.8~m$E_h$ in the cohesive energy).
We estimate the ring-CCSDT, CCSDT, and DCSDT energies
using composite corrections, by again considering the differences to DCSD,
based on calculations with small supercells
(containing 8 and 16 Li atoms), frozen core orbitals, and frozen virtual
natural orbitals~\cite{Note1}.

Results are presented in Fig.~\ref{fig:structplotli}, where they are compared
to low temperature experimental results~\cite{zhang_performance_2018, berliner_effect_1986,felice_temperature_1977,kittel_intro_solid_2005}
that have been corrected for zero-point
vibrational effects based on HSE06 phonon calculations~\cite{zhang_performance_2018};
a table of all values is given in the Supplemental
Material~\cite{Note1}.  Consistent with our results on the UEG, we see relatively systematic
improvement with increasing sophistication of the theory. DCSD, ring-CCSDT,
CCSDT, and DCSDT are all improvements over CCSD and they achieve accuracies of
0.004--0.02~\AA, 0.1--0.2~GPa, and 4--6~m$E_h$ in the lattice constant, bulk
modulus, and cohesive energy, respectively.
It is hard to disentangle the
remaining discrepancies, which likely include some combination of pseudopotential,
basis set, and finite-size error, incomplete correlation, and experimental
uncertainty, including vibrational corrections.
We also compare to DFT results reported in Ref.~\cite{zhang_performance_2018} 
using the LDA~\cite{kohn_self-consistent_1965} and 
HSE06~\cite{heyd_hybrid_2003,heyd_erratum_2006,krukau_influence_2006} functionals.
While the LDA functional does not predict accurate structural properties
(despite its exactness for the UEG), the HSE06 functional performs very well.
Importantly, we see that the improved methods explored in this work clearly
outperform CCSD, bringing CC theory in line with the best performing DFT functionals.


\begin{figure}[t]
	\includegraphics[width=3.25in]{structplot_ccnew_flipped.pdf}
        \caption{Equilibrium lattice constant $a$, bulk modulus $B$, and
            cohesive energy $E_{\mathrm{coh}}$ for solid lithium.  Results are
            shown at the indicated levels of CC theory and compared to
            experimental results~\cite{zhang_performance_2018,
            berliner_effect_1986,felice_temperature_1977,kittel_intro_solid_2005}
            (solid horizontal lines), which have been corrected for zero-point
            vibrational energy using the HSE06 corrections from
            Ref.~\onlinecite{zhang_performance_2018}.  DFT results for the LDA
            and HSE06 functionals are shown for comparison (from
            Ref.~\onlinecite{zhang_performance_2018})
	}
	\label{fig:structplotli}
\end{figure}


%\section{Conclusion}
%\label{sec:conc}
\textit{Conclusion.}
Despite the apparent simplicity of simple metals, including the uniform
electron gas, achieving high accuracy for the electron correlation energy with
\textit{ab initio} wavefunction or diagrammatic methods is clearly a challenge.
We have shown that within the family of CC theories, the infinite-order
inclusion of connected triple excitations is essential, although semi-empirical
treatments of these effects are surprisingly effective.  We expect that the
methods explored here, which have been evaluated for their ability to predict
the properties of nearly uniform systems, will outperform DFT for more
heterogeneous systems, such as those arising in surface chemistry.  Before CC
methods are widely used in this context, their comparatively high
computational and storage costs must be addressed.  However, in the meantime,
they can be used to provide predictions of benchmark quality, especially in the
many situations where experimental values cannot be obtained to the required
precision.

\vspace{2em}

\textit{Acknowledgments.} 
We thank James Callahan and Xiao Wang for helpful discussions.  This work was
supported by the Columbia Center for Computational Electrochemistry and the
National Science Foundation under Grant No.~CHE-1848369.  We acknowledge
computing resources from Columbia University's Shared Research Computing
Facility project, which is supported by NIH Research Facility Improvement Grant
1G20RR030893-01, and associated funds from the New York State Empire State
Development, Division of Science Technology and Innovation (NYSTAR) Contract
C090171, both awarded April 15, 2010.
Data analysis and visualization were performed using 
NumPy~\cite{harris_array_2020},
SciPy~\cite{scipy_2020},
pandas~\cite{mckinney_data_2009}, 
Matplotlib~\cite{hunter_matplotlib_2007},
seaborn~\cite{Wseaborn2021}, and
JaxoDraw~\cite{binosi_jaxodraw_2004}. UEG calculations used
Julia~\cite{bezanson_julia_2017}, 
Fermi.jl~\cite{aroeira_fermijl_2022},
TensorOperations.jl~\cite{noauthor_tensoroperationsjl_nodate}, 
and Tullio.jl~\cite{noauthor_tulliojl_nodate}.

%apsrev4-2.bst 2019-01-14 (MD) hand-edited version of apsrev4-1.bst
%Control: key (0)
%Control: author (8) initials jnrlst
%Control: editor formatted (1) identically to author
%Control: production of article title (0) allowed
%Control: page (0) single
%Control: year (1) truncated
%Control: production of eprint (0) enabled
\begin{thebibliography}{81}%
\makeatletter
\providecommand \@ifxundefined [1]{%
 \@ifx{#1\undefined}
}%
\providecommand \@ifnum [1]{%
 \ifnum #1\expandafter \@firstoftwo
 \else \expandafter \@secondoftwo
 \fi
}%
\providecommand \@ifx [1]{%
 \ifx #1\expandafter \@firstoftwo
 \else \expandafter \@secondoftwo
 \fi
}%
\providecommand \natexlab [1]{#1}%
\providecommand \enquote  [1]{``#1''}%
\providecommand \bibnamefont  [1]{#1}%
\providecommand \bibfnamefont [1]{#1}%
\providecommand \citenamefont [1]{#1}%
\providecommand \href@noop [0]{\@secondoftwo}%
\providecommand \href [0]{\begingroup \@sanitize@url \@href}%
\providecommand \@href[1]{\@@startlink{#1}\@@href}%
\providecommand \@@href[1]{\endgroup#1\@@endlink}%
\providecommand \@sanitize@url [0]{\catcode `\\12\catcode `\$12\catcode
  `\&12\catcode `\#12\catcode `\^12\catcode `\_12\catcode `\%12\relax}%
\providecommand \@@startlink[1]{}%
\providecommand \@@endlink[0]{}%
\providecommand \url  [0]{\begingroup\@sanitize@url \@url }%
\providecommand \@url [1]{\endgroup\@href {#1}{\urlprefix }}%
\providecommand \urlprefix  [0]{URL }%
\providecommand \Eprint [0]{\href }%
\providecommand \doibase [0]{https://doi.org/}%
\providecommand \selectlanguage [0]{\@gobble}%
\providecommand \bibinfo  [0]{\@secondoftwo}%
\providecommand \bibfield  [0]{\@secondoftwo}%
\providecommand \translation [1]{[#1]}%
\providecommand \BibitemOpen [0]{}%
\providecommand \bibitemStop [0]{}%
\providecommand \bibitemNoStop [0]{.\EOS\space}%
\providecommand \EOS [0]{\spacefactor3000\relax}%
\providecommand \BibitemShut  [1]{\csname bibitem#1\endcsname}%
\let\auto@bib@innerbib\@empty
%</preamble>
\bibitem [{\citenamefont {Nørskov}\ \emph {et~al.}(2011)\citenamefont
  {Nørskov}, \citenamefont {Abild-Pedersen}, \citenamefont {Studt},\ and\
  \citenamefont {Bligaard}}]{norskov_density_2011}%
  \BibitemOpen
  \bibfield  {author} {\bibinfo {author} {\bibfnamefont {J.~K.}\ \bibnamefont
  {Nørskov}}, \bibinfo {author} {\bibfnamefont {F.}~\bibnamefont
  {Abild-Pedersen}}, \bibinfo {author} {\bibfnamefont {F.}~\bibnamefont
  {Studt}},\ and\ \bibinfo {author} {\bibfnamefont {T.}~\bibnamefont
  {Bligaard}},\ }\bibfield  {title} {\bibinfo {title} {Density functional
  theory in surface chemistry and catalysis},\ }\href
  {https://doi.org/10.1073/pnas.1006652108} {\bibfield  {journal} {\bibinfo
  {journal} {Proc. Natl. Acad. Sci.}\ }\textbf {\bibinfo {volume} {108}},\
  \bibinfo {pages} {937} (\bibinfo {year} {2011})}\BibitemShut {NoStop}%
\bibitem [{\citenamefont {Calle-Vallejo}\ and\ \citenamefont
  {Koper}(2012)}]{calle-vallejo_first-principles_2012}%
  \BibitemOpen
  \bibfield  {author} {\bibinfo {author} {\bibfnamefont {F.}~\bibnamefont
  {Calle-Vallejo}}\ and\ \bibinfo {author} {\bibfnamefont {M.~T.~M.}\
  \bibnamefont {Koper}},\ }\bibfield  {title} {\bibinfo {title}
  {First-principles computational electrochemistry: {Achievements} and
  challenges},\ }\href
  {https://doi.org/https://doi.org/10.1016/j.electacta.2012.04.062} {\bibfield
  {journal} {\bibinfo  {journal} {Electrochim. Acta}\ }\textbf {\bibinfo
  {volume} {84}},\ \bibinfo {pages} {3} (\bibinfo {year} {2012})}\BibitemShut
  {NoStop}%
\bibitem [{\citenamefont {He}\ \emph {et~al.}(2019)\citenamefont {He},
  \citenamefont {Yu}, \citenamefont {Li},\ and\ \citenamefont
  {Zhao}}]{he_density_2019}%
  \BibitemOpen
  \bibfield  {author} {\bibinfo {author} {\bibfnamefont {Q.}~\bibnamefont
  {He}}, \bibinfo {author} {\bibfnamefont {B.}~\bibnamefont {Yu}}, \bibinfo
  {author} {\bibfnamefont {Z.}~\bibnamefont {Li}},\ and\ \bibinfo {author}
  {\bibfnamefont {Y.}~\bibnamefont {Zhao}},\ }\bibfield  {title} {\bibinfo
  {title} {Density {Functional} {Theory} for {Battery} {Materials}},\ }\href
  {https://doi.org/10.1002/eem2.12056} {\bibfield  {journal} {\bibinfo
  {journal} {Energy Environ. Sci.}\ }\textbf {\bibinfo {volume} {2}},\ \bibinfo
  {pages} {264} (\bibinfo {year} {2019})}\BibitemShut {NoStop}%
\bibitem [{\citenamefont {Shavitt}\ and\ \citenamefont
  {Bartlett}(2009)}]{shavitt_many-body_2009}%
  \BibitemOpen
  \bibfield  {author} {\bibinfo {author} {\bibfnamefont {I.}~\bibnamefont
  {Shavitt}}\ and\ \bibinfo {author} {\bibfnamefont {R.~J.}\ \bibnamefont
  {Bartlett}},\ }\href {https://doi.org/10.1017/CBO9780511596834} {\emph
  {\bibinfo {title} {Many-{Body} {Methods} in {Chemistry} and {Physics}: {MBPT}
  and {Coupled}-{Cluster} {Theory}}}},\ Cambridge {Molecular} {Science}\
  (\bibinfo  {publisher} {Cambridge University Press},\ \bibinfo {address}
  {Cambridge},\ \bibinfo {year} {2009})\BibitemShut {NoStop}%
\bibitem [{\citenamefont {Bartlett}\ and\ \citenamefont
  {Musiał}(2007)}]{bartlett_coupled-cluster_2007}%
  \BibitemOpen
  \bibfield  {author} {\bibinfo {author} {\bibfnamefont {R.~J.}\ \bibnamefont
  {Bartlett}}\ and\ \bibinfo {author} {\bibfnamefont {M.}~\bibnamefont
  {Musiał}},\ }\bibfield  {title} {\bibinfo {title} {Coupled-cluster theory in
  quantum chemistry},\ }\href {https://doi.org/10.1103/RevModPhys.79.291}
  {\bibfield  {journal} {\bibinfo  {journal} {Rev. Mod. Phys.}\ }\textbf
  {\bibinfo {volume} {79}},\ \bibinfo {pages} {291} (\bibinfo {year}
  {2007})}\BibitemShut {NoStop}%
\bibitem [{\citenamefont {Shepherd}(2016)}]{shepherd_communication_2016}%
  \BibitemOpen
  \bibfield  {author} {\bibinfo {author} {\bibfnamefont {J.~J.}\ \bibnamefont
  {Shepherd}},\ }\bibfield  {title} {\bibinfo {title} {Communication:
  {Convergence} of many-body wave-function expansions using a plane-wave basis
  in the thermodynamic limit},\ }\href {https://doi.org/10.1063/1.4958461}
  {\bibfield  {journal} {\bibinfo  {journal} {J. Chem. Phys.}\ }\textbf
  {\bibinfo {volume} {145}},\ \bibinfo {pages} {031104} (\bibinfo {year}
  {2016})}\BibitemShut {NoStop}%
\bibitem [{\citenamefont {Mihm}\ \emph
  {et~al.}(2021{\natexlab{a}})\citenamefont {Mihm}, \citenamefont {Yang},\ and\
  \citenamefont {Shepherd}}]{mihm_power_2021}%
  \BibitemOpen
  \bibfield  {author} {\bibinfo {author} {\bibfnamefont {T.~N.}\ \bibnamefont
  {Mihm}}, \bibinfo {author} {\bibfnamefont {B.}~\bibnamefont {Yang}},\ and\
  \bibinfo {author} {\bibfnamefont {J.~J.}\ \bibnamefont {Shepherd}},\
  }\bibfield  {title} {\bibinfo {title} {Power {Laws} {Used} to {Extrapolate}
  the {Coupled} {Cluster} {Correlation} {Energy} to the {Thermodynamic}
  {Limit}},\ }\href {https://doi.org/10.1021/acs.jctc.0c01171} {\bibfield
  {journal} {\bibinfo  {journal} {J. Chem. Theor. Comput.}\ }\textbf {\bibinfo
  {volume} {17}},\ \bibinfo {pages} {2752} (\bibinfo {year}
  {2021}{\natexlab{a}})}\BibitemShut {NoStop}%
\bibitem [{\citenamefont {Callahan}\ \emph {et~al.}(2021)\citenamefont
  {Callahan}, \citenamefont {Lange},\ and\ \citenamefont
  {Berkelbach}}]{callahan_dynamical_2021}%
  \BibitemOpen
  \bibfield  {author} {\bibinfo {author} {\bibfnamefont {J.~M.}\ \bibnamefont
  {Callahan}}, \bibinfo {author} {\bibfnamefont {M.~F.}\ \bibnamefont
  {Lange}},\ and\ \bibinfo {author} {\bibfnamefont {T.~C.}\ \bibnamefont
  {Berkelbach}},\ }\bibfield  {title} {\bibinfo {title} {Dynamical correlation
  energy of metals in large basis sets from downfolding and composite
  approaches},\ }\href {https://doi.org/10.1063/5.0049890} {\bibfield
  {journal} {\bibinfo  {journal} {J. Chem. Phys.}\ }\textbf {\bibinfo {volume}
  {154}},\ \bibinfo {pages} {211105} (\bibinfo {year} {2021})}\BibitemShut
  {NoStop}%
\bibitem [{\citenamefont {Stoll}\ \emph {et~al.}(2009)\citenamefont {Stoll},
  \citenamefont {Paulus},\ and\ \citenamefont
  {Fulde}}]{stoll_incremental_2009}%
  \BibitemOpen
  \bibfield  {author} {\bibinfo {author} {\bibfnamefont {H.}~\bibnamefont
  {Stoll}}, \bibinfo {author} {\bibfnamefont {B.}~\bibnamefont {Paulus}},\ and\
  \bibinfo {author} {\bibfnamefont {P.}~\bibnamefont {Fulde}},\ }\bibfield
  {title} {\bibinfo {title} {An incremental coupled-cluster approach to
  metallic lithium},\ }\href {https://doi.org/10.1016/j.cplett.2008.12.042}
  {\bibfield  {journal} {\bibinfo  {journal} {Chem. Phys. Lett.}\ }\textbf
  {\bibinfo {volume} {469}},\ \bibinfo {pages} {90} (\bibinfo {year}
  {2009})}\BibitemShut {NoStop}%
\bibitem [{\citenamefont {Mihm}\ \emph
  {et~al.}(2021{\natexlab{b}})\citenamefont {Mihm}, \citenamefont {Schäfer},
  \citenamefont {Ramadugu}, \citenamefont {Weiler}, \citenamefont {Grüneis},\
  and\ \citenamefont {Shepherd}}]{mihm_shortcut_2021}%
  \BibitemOpen
  \bibfield  {author} {\bibinfo {author} {\bibfnamefont {T.~N.}\ \bibnamefont
  {Mihm}}, \bibinfo {author} {\bibfnamefont {T.}~\bibnamefont {Schäfer}},
  \bibinfo {author} {\bibfnamefont {S.~K.}\ \bibnamefont {Ramadugu}}, \bibinfo
  {author} {\bibfnamefont {L.}~\bibnamefont {Weiler}}, \bibinfo {author}
  {\bibfnamefont {A.}~\bibnamefont {Grüneis}},\ and\ \bibinfo {author}
  {\bibfnamefont {J.~J.}\ \bibnamefont {Shepherd}},\ }\bibfield  {title}
  {\bibinfo {title} {A shortcut to the thermodynamic limit for quantum
  many-body calculations of metals},\ }\href
  {https://doi.org/10.1038/s43588-021-00165-1} {\bibfield  {journal} {\bibinfo
  {journal} {Nat. Comput. Sci.}\ }\textbf {\bibinfo {volume} {1}},\ \bibinfo
  {pages} {801} (\bibinfo {year} {2021}{\natexlab{b}})}\BibitemShut {NoStop}%
\bibitem [{\citenamefont {Neufeld}\ \emph {et~al.}(2022)\citenamefont
  {Neufeld}, \citenamefont {Ye},\ and\ \citenamefont
  {Berkelbach}}]{neufeld_ground-state_2022}%
  \BibitemOpen
  \bibfield  {author} {\bibinfo {author} {\bibfnamefont {V.~A.}\ \bibnamefont
  {Neufeld}}, \bibinfo {author} {\bibfnamefont {H.-Z.}\ \bibnamefont {Ye}},\
  and\ \bibinfo {author} {\bibfnamefont {T.~C.}\ \bibnamefont {Berkelbach}},\
  }\bibfield  {title} {\bibinfo {title} {Ground-{State} {Properties} of
  {Metallic} {Solids} from {Ab} {Initio} {Coupled}-{Cluster} {Theory}},\ }\href
  {https://doi.org/10.1021/acs.jpclett.2c01828} {\bibfield  {journal} {\bibinfo
   {journal} {J. Phys. Chem. Lett.}\ }\textbf {\bibinfo {volume} {13}},\
  \bibinfo {pages} {7497} (\bibinfo {year} {2022})}\BibitemShut {NoStop}%
\bibitem [{\citenamefont {Weiler}\ \emph {et~al.}(2022)\citenamefont {Weiler},
  \citenamefont {Mihm},\ and\ \citenamefont {Shepherd}}]{weiler_machine_2022}%
  \BibitemOpen
  \bibfield  {author} {\bibinfo {author} {\bibfnamefont {L.}~\bibnamefont
  {Weiler}}, \bibinfo {author} {\bibfnamefont {T.~N.}\ \bibnamefont {Mihm}},\
  and\ \bibinfo {author} {\bibfnamefont {J.~J.}\ \bibnamefont {Shepherd}},\
  }\bibfield  {title} {\bibinfo {title} {Machine learning for a finite size
  correction in periodic coupled cluster theory calculations},\ }\href
  {https://doi.org/10.1063/5.0086580} {\bibfield  {journal} {\bibinfo
  {journal} {J. Chem. Phys.}\ }\textbf {\bibinfo {volume} {156}},\ \bibinfo
  {pages} {204109} (\bibinfo {year} {2022})}\BibitemShut {NoStop}%
\bibitem [{\citenamefont {Raghavachari}\ \emph {et~al.}(1989)\citenamefont
  {Raghavachari}, \citenamefont {Trucks}, \citenamefont {Pople},\ and\
  \citenamefont {Head-Gordon}}]{raghavachari_fifth-order_1989}%
  \BibitemOpen
  \bibfield  {author} {\bibinfo {author} {\bibfnamefont {K.}~\bibnamefont
  {Raghavachari}}, \bibinfo {author} {\bibfnamefont {G.~W.}\ \bibnamefont
  {Trucks}}, \bibinfo {author} {\bibfnamefont {J.~A.}\ \bibnamefont {Pople}},\
  and\ \bibinfo {author} {\bibfnamefont {M.}~\bibnamefont {Head-Gordon}},\
  }\bibfield  {title} {\bibinfo {title} {A fifth-order perturbation comparison
  of electron correlation theories},\ }\href
  {https://doi.org/10.1016/S0009-2614(89)87395-6} {\bibfield  {journal}
  {\bibinfo  {journal} {Chem. Phys. Lett.}\ }\textbf {\bibinfo {volume}
  {157}},\ \bibinfo {pages} {479} (\bibinfo {year} {1989})}\BibitemShut
  {NoStop}%
\bibitem [{\citenamefont {Schwerdtfeger}\ \emph {et~al.}(2010)\citenamefont
  {Schwerdtfeger}, \citenamefont {Assadollahzadeh},\ and\ \citenamefont
  {Hermann}}]{schwerdtfeger_convergence_2010}%
  \BibitemOpen
  \bibfield  {author} {\bibinfo {author} {\bibfnamefont {P.}~\bibnamefont
  {Schwerdtfeger}}, \bibinfo {author} {\bibfnamefont {B.}~\bibnamefont
  {Assadollahzadeh}},\ and\ \bibinfo {author} {\bibfnamefont {A.}~\bibnamefont
  {Hermann}},\ }\bibfield  {title} {\bibinfo {title} {Convergence of the
  {Møller}-{Plesset} perturbation series for the fcc lattices of neon and
  argon},\ }\href {https://doi.org/10.1103/PhysRevB.82.205111} {\bibfield
  {journal} {\bibinfo  {journal} {Phys. Rev. B}\ }\textbf {\bibinfo {volume}
  {82}},\ \bibinfo {pages} {205111} (\bibinfo {year} {2010})}\BibitemShut
  {NoStop}%
\bibitem [{\citenamefont {Booth}\ \emph {et~al.}(2013)\citenamefont {Booth},
  \citenamefont {Grüneis}, \citenamefont {Kresse},\ and\ \citenamefont
  {Alavi}}]{booth_towards_2013}%
  \BibitemOpen
  \bibfield  {author} {\bibinfo {author} {\bibfnamefont {G.~H.}\ \bibnamefont
  {Booth}}, \bibinfo {author} {\bibfnamefont {A.}~\bibnamefont {Grüneis}},
  \bibinfo {author} {\bibfnamefont {G.}~\bibnamefont {Kresse}},\ and\ \bibinfo
  {author} {\bibfnamefont {A.}~\bibnamefont {Alavi}},\ }\bibfield  {title}
  {\bibinfo {title} {Towards an exact description of electronic wavefunctions
  in real solids},\ }\href {https://doi.org/10.1038/nature11770} {\bibfield
  {journal} {\bibinfo  {journal} {Nature}\ }\textbf {\bibinfo {volume} {493}},\
  \bibinfo {pages} {365} (\bibinfo {year} {2013})}\BibitemShut {NoStop}%
\bibitem [{\citenamefont {Grüneis}(2015)}]{gruneis_coupled_2015}%
  \BibitemOpen
  \bibfield  {author} {\bibinfo {author} {\bibfnamefont {A.}~\bibnamefont
  {Grüneis}},\ }\bibfield  {title} {\bibinfo {title} {A coupled cluster and
  {Møller}-{Plesset} perturbation theory study of the pressure induced phase
  transition in the {LiH} crystal},\ }\href {https://doi.org/10.1063/1.4928645}
  {\bibfield  {journal} {\bibinfo  {journal} {J. Chem. Phys.}\ }\textbf
  {\bibinfo {volume} {143}},\ \bibinfo {pages} {102817} (\bibinfo {year}
  {2015})}\BibitemShut {NoStop}%
\bibitem [{\citenamefont {Gruber}\ \emph {et~al.}(2018)\citenamefont {Gruber},
  \citenamefont {Liao}, \citenamefont {Tsatsoulis}, \citenamefont {Hummel},\
  and\ \citenamefont {Grüneis}}]{gruber_applying_2018}%
  \BibitemOpen
  \bibfield  {author} {\bibinfo {author} {\bibfnamefont {T.}~\bibnamefont
  {Gruber}}, \bibinfo {author} {\bibfnamefont {K.}~\bibnamefont {Liao}},
  \bibinfo {author} {\bibfnamefont {T.}~\bibnamefont {Tsatsoulis}}, \bibinfo
  {author} {\bibfnamefont {F.}~\bibnamefont {Hummel}},\ and\ \bibinfo {author}
  {\bibfnamefont {A.}~\bibnamefont {Grüneis}},\ }\bibfield  {title} {\bibinfo
  {title} {Applying the {Coupled}-{Cluster} {Ansatz} to {Solids} and {Surfaces}
  in the {Thermodynamic} {Limit}},\ }\href
  {https://doi.org/10.1103/PhysRevX.8.021043} {\bibfield  {journal} {\bibinfo
  {journal} {Phys. Rev. X}\ }\textbf {\bibinfo {volume} {8}},\ \bibinfo {pages}
  {021043} (\bibinfo {year} {2018})}\BibitemShut {NoStop}%
\bibitem [{\citenamefont {Gruber}\ and\ \citenamefont
  {Grüneis}(2018)}]{gruber_ab_2018}%
  \BibitemOpen
  \bibfield  {author} {\bibinfo {author} {\bibfnamefont {T.}~\bibnamefont
  {Gruber}}\ and\ \bibinfo {author} {\bibfnamefont {A.}~\bibnamefont
  {Grüneis}},\ }\bibfield  {title} {\bibinfo {title} {\textit{{Ab} initio}
  calculations of carbon and boron nitride allotropes and their structural
  phase transitions using periodic coupled cluster theory},\ }\href
  {https://doi.org/10.1103/PhysRevB.98.134108} {\bibfield  {journal} {\bibinfo
  {journal} {Phys. Rev. B}\ }\textbf {\bibinfo {volume} {98}},\ \bibinfo
  {pages} {134108} (\bibinfo {year} {2018})}\BibitemShut {NoStop}%
\bibitem [{\citenamefont {Tsatsoulis}\ \emph {et~al.}(2017)\citenamefont
  {Tsatsoulis}, \citenamefont {Hummel}, \citenamefont {Usvyat}, \citenamefont
  {Schütz}, \citenamefont {Booth}, \citenamefont {Binnie}, \citenamefont
  {Gillan}, \citenamefont {Alfè}, \citenamefont {Michaelides},\ and\
  \citenamefont {Grüneis}}]{tsatsoulis_comparison_2017}%
  \BibitemOpen
  \bibfield  {author} {\bibinfo {author} {\bibfnamefont {T.}~\bibnamefont
  {Tsatsoulis}}, \bibinfo {author} {\bibfnamefont {F.}~\bibnamefont {Hummel}},
  \bibinfo {author} {\bibfnamefont {D.}~\bibnamefont {Usvyat}}, \bibinfo
  {author} {\bibfnamefont {M.}~\bibnamefont {Schütz}}, \bibinfo {author}
  {\bibfnamefont {G.~H.}\ \bibnamefont {Booth}}, \bibinfo {author}
  {\bibfnamefont {S.~S.}\ \bibnamefont {Binnie}}, \bibinfo {author}
  {\bibfnamefont {M.~J.}\ \bibnamefont {Gillan}}, \bibinfo {author}
  {\bibfnamefont {D.}~\bibnamefont {Alfè}}, \bibinfo {author} {\bibfnamefont
  {A.}~\bibnamefont {Michaelides}},\ and\ \bibinfo {author} {\bibfnamefont
  {A.}~\bibnamefont {Grüneis}},\ }\bibfield  {title} {\bibinfo {title} {A
  comparison between quantum chemistry and quantum {Monte} {Carlo} techniques
  for the adsorption of water on the (001) {LiH} surface},\ }\href
  {https://doi.org/10.1063/1.4984048} {\bibfield  {journal} {\bibinfo
  {journal} {J. Chem. Phys.}\ }\textbf {\bibinfo {volume} {146}},\ \bibinfo
  {pages} {204108} (\bibinfo {year} {2017})}\BibitemShut {NoStop}%
\bibitem [{\citenamefont {Tsatsoulis}\ \emph {et~al.}(2018)\citenamefont
  {Tsatsoulis}, \citenamefont {Sakong}, \citenamefont {Groß},\ and\
  \citenamefont {Grüneis}}]{tsatsoulis_reaction_2018}%
  \BibitemOpen
  \bibfield  {author} {\bibinfo {author} {\bibfnamefont {T.}~\bibnamefont
  {Tsatsoulis}}, \bibinfo {author} {\bibfnamefont {S.}~\bibnamefont {Sakong}},
  \bibinfo {author} {\bibfnamefont {A.}~\bibnamefont {Groß}},\ and\ \bibinfo
  {author} {\bibfnamefont {A.}~\bibnamefont {Grüneis}},\ }\bibfield  {title}
  {\bibinfo {title} {Reaction energetics of hydrogen on {Si}(100) surface: {A}
  periodic many-electron theory study},\ }\href
  {https://doi.org/10.1063/1.5055706} {\bibfield  {journal} {\bibinfo
  {journal} {J. Chem. Phys.}\ }\textbf {\bibinfo {volume} {149}},\ \bibinfo
  {pages} {244105} (\bibinfo {year} {2018})}\BibitemShut {NoStop}%
\bibitem [{\citenamefont {Brandenburg}\ \emph {et~al.}(2019)\citenamefont
  {Brandenburg}, \citenamefont {Zen}, \citenamefont {Fitzner}, \citenamefont
  {Ramberger}, \citenamefont {Kresse}, \citenamefont {Tsatsoulis},
  \citenamefont {Grüneis}, \citenamefont {Michaelides},\ and\ \citenamefont
  {Alfè}}]{brandenburg_physisorption_2019}%
  \BibitemOpen
  \bibfield  {author} {\bibinfo {author} {\bibfnamefont {J.~G.}\ \bibnamefont
  {Brandenburg}}, \bibinfo {author} {\bibfnamefont {A.}~\bibnamefont {Zen}},
  \bibinfo {author} {\bibfnamefont {M.}~\bibnamefont {Fitzner}}, \bibinfo
  {author} {\bibfnamefont {B.}~\bibnamefont {Ramberger}}, \bibinfo {author}
  {\bibfnamefont {G.}~\bibnamefont {Kresse}}, \bibinfo {author} {\bibfnamefont
  {T.}~\bibnamefont {Tsatsoulis}}, \bibinfo {author} {\bibfnamefont
  {A.}~\bibnamefont {Grüneis}}, \bibinfo {author} {\bibfnamefont
  {A.}~\bibnamefont {Michaelides}},\ and\ \bibinfo {author} {\bibfnamefont
  {D.}~\bibnamefont {Alfè}},\ }\bibfield  {title} {\bibinfo {title}
  {Physisorption of {Water} on {Graphene}: {Subchemical} {Accuracy} from
  {Many}-{Body} {Electronic} {Structure} {Methods}},\ }\href
  {https://doi.org/10.1021/acs.jpclett.8b03679} {\bibfield  {journal} {\bibinfo
   {journal} {J. Phys. Chem. Lett.}\ }\textbf {\bibinfo {volume} {10}},\
  \bibinfo {pages} {358} (\bibinfo {year} {2019})}\BibitemShut {NoStop}%
\bibitem [{\citenamefont {Lau}\ \emph {et~al.}(2021)\citenamefont {Lau},
  \citenamefont {Knizia},\ and\ \citenamefont
  {Berkelbach}}]{lau_regional_2021}%
  \BibitemOpen
  \bibfield  {author} {\bibinfo {author} {\bibfnamefont {B.~T.~G.}\
  \bibnamefont {Lau}}, \bibinfo {author} {\bibfnamefont {G.}~\bibnamefont
  {Knizia}},\ and\ \bibinfo {author} {\bibfnamefont {T.~C.}\ \bibnamefont
  {Berkelbach}},\ }\bibfield  {title} {\bibinfo {title} {Regional {Embedding}
  {Enables} {High}-{Level} {Quantum} {Chemistry} for {Surface} {Science}},\
  }\href {https://doi.org/10.1021/acs.jpclett.0c03274} {\bibfield  {journal}
  {\bibinfo  {journal} {J. Phys. Chem. Lett.}\ }\textbf {\bibinfo {volume}
  {12}},\ \bibinfo {pages} {1104} (\bibinfo {year} {2021})}\BibitemShut
  {NoStop}%
\bibitem [{\citenamefont {Shepherd}\ and\ \citenamefont
  {Grüneis}(2013)}]{shepherd_many-body_2013}%
  \BibitemOpen
  \bibfield  {author} {\bibinfo {author} {\bibfnamefont {J.~J.}\ \bibnamefont
  {Shepherd}}\ and\ \bibinfo {author} {\bibfnamefont {A.}~\bibnamefont
  {Grüneis}},\ }\bibfield  {title} {\bibinfo {title} {Many-{Body} {Quantum}
  {Chemistry} for the {Electron} {Gas}: {Convergent} {Perturbative}
  {Theories}},\ }\href {https://doi.org/10.1103/PhysRevLett.110.226401}
  {\bibfield  {journal} {\bibinfo  {journal} {Phys. Rev. Lett.}\ }\textbf
  {\bibinfo {volume} {110}},\ \bibinfo {pages} {226401} (\bibinfo {year}
  {2013})}\BibitemShut {NoStop}%
\bibitem [{\citenamefont {Macke}(1950)}]{macke_uber_1950}%
  \BibitemOpen
  \bibfield  {author} {\bibinfo {author} {\bibfnamefont {W.}~\bibnamefont
  {Macke}},\ }\bibfield  {title} {\bibinfo {title} {Über die
  {Wechselwirkungen} im {Fermi}-{Gas}. {Polarisationserscheinungen},
  {Correlationsenergie}, {Elektronenkondensation}},\ }\href
  {https://doi.org/10.1515/zna-1950-0402} {\bibfield  {journal} {\bibinfo
  {journal} {Z. Naturforsch. A}\ }\textbf {\bibinfo {volume} {5}},\ \bibinfo
  {pages} {192} (\bibinfo {year} {1950})},\ \bibinfo {note} {publisher: De
  Gruyter}\BibitemShut {NoStop}%
\bibitem [{\citenamefont {Gell-Mann}\ and\ \citenamefont
  {Brueckner}(1957)}]{gell-mann_correlation_1957}%
  \BibitemOpen
  \bibfield  {author} {\bibinfo {author} {\bibfnamefont {M.}~\bibnamefont
  {Gell-Mann}}\ and\ \bibinfo {author} {\bibfnamefont {K.~A.}\ \bibnamefont
  {Brueckner}},\ }\bibfield  {title} {\bibinfo {title} {Correlation {Energy} of
  an {Electron} {Gas} at {High} {Density}},\ }\href
  {https://doi.org/10.1103/PhysRev.106.364} {\bibfield  {journal} {\bibinfo
  {journal} {Phys. Rev.}\ }\textbf {\bibinfo {volume} {106}},\ \bibinfo {pages}
  {364} (\bibinfo {year} {1957})}\BibitemShut {NoStop}%
\bibitem [{\citenamefont {Kats}\ and\ \citenamefont
  {Manby}(2013)}]{kats_communication_2013}%
  \BibitemOpen
  \bibfield  {author} {\bibinfo {author} {\bibfnamefont {D.}~\bibnamefont
  {Kats}}\ and\ \bibinfo {author} {\bibfnamefont {F.~R.}\ \bibnamefont
  {Manby}},\ }\bibfield  {title} {\bibinfo {title} {Communication: {The}
  distinguishable cluster approximation},\ }\href
  {https://doi.org/10.1063/1.4813481} {\bibfield  {journal} {\bibinfo
  {journal} {J. Chem. Phys.}\ }\textbf {\bibinfo {volume} {139}},\ \bibinfo
  {pages} {021102} (\bibinfo {year} {2013})}\BibitemShut {NoStop}%
\bibitem [{\citenamefont {Kats}\ and\ \citenamefont
  {Köhn}(2019)}]{kats_distinguishable_2019}%
  \BibitemOpen
  \bibfield  {author} {\bibinfo {author} {\bibfnamefont {D.}~\bibnamefont
  {Kats}}\ and\ \bibinfo {author} {\bibfnamefont {A.}~\bibnamefont {Köhn}},\
  }\bibfield  {title} {\bibinfo {title} {On the distinguishable cluster
  approximation for triple excitations},\ }\href
  {https://doi.org/10.1063/1.5096343} {\bibfield  {journal} {\bibinfo
  {journal} {J. Chem. Phys.}\ }\textbf {\bibinfo {volume} {150}},\ \bibinfo
  {pages} {151101} (\bibinfo {year} {2019})}\BibitemShut {NoStop}%
\bibitem [{\citenamefont {Rishi}\ and\ \citenamefont
  {Valeev}(2019)}]{rishi_can_2019}%
  \BibitemOpen
  \bibfield  {author} {\bibinfo {author} {\bibfnamefont {V.}~\bibnamefont
  {Rishi}}\ and\ \bibinfo {author} {\bibfnamefont {E.~F.}\ \bibnamefont
  {Valeev}},\ }\bibfield  {title} {\bibinfo {title} {Can the distinguishable
  cluster approximation be improved systematically by including connected
  triples?},\ }\href {https://doi.org/10.1063/1.5097150} {\bibfield  {journal}
  {\bibinfo  {journal} {J. Chem. Phys.}\ }\textbf {\bibinfo {volume} {151}},\
  \bibinfo {pages} {064102} (\bibinfo {year} {2019})}\BibitemShut {NoStop}%
\bibitem [{\citenamefont {Grimme}(2003)}]{grimme_improved_2003}%
  \BibitemOpen
  \bibfield  {author} {\bibinfo {author} {\bibfnamefont {S.}~\bibnamefont
  {Grimme}},\ }\bibfield  {title} {\bibinfo {title} {Improved second-order
  {Møller}–{Plesset} perturbation theory by separate scaling of parallel-
  and antiparallel-spin pair correlation energies},\ }\href
  {https://doi.org/10.1063/1.1569242} {\bibfield  {journal} {\bibinfo
  {journal} {J. Chem. Phys.}\ }\textbf {\bibinfo {volume} {118}},\ \bibinfo
  {pages} {9095} (\bibinfo {year} {2003})}\BibitemShut {NoStop}%
\bibitem [{\citenamefont {Takatani}\ \emph {et~al.}(2008)\citenamefont
  {Takatani}, \citenamefont {Hohenstein},\ and\ \citenamefont
  {Sherrill}}]{takatani_improvement_2008}%
  \BibitemOpen
  \bibfield  {author} {\bibinfo {author} {\bibfnamefont {T.}~\bibnamefont
  {Takatani}}, \bibinfo {author} {\bibfnamefont {E.~G.}\ \bibnamefont
  {Hohenstein}},\ and\ \bibinfo {author} {\bibfnamefont {C.~D.}\ \bibnamefont
  {Sherrill}},\ }\bibfield  {title} {\bibinfo {title} {Improvement of the
  coupled-cluster singles and doubles method via scaling same- and
  opposite-spin components of the double excitation correlation energy},\
  }\href {https://doi.org/10.1063/1.2883974} {\bibfield  {journal} {\bibinfo
  {journal} {J. Chem. Phys.}\ }\textbf {\bibinfo {volume} {128}},\ \bibinfo
  {pages} {124111} (\bibinfo {year} {2008})}\BibitemShut {NoStop}%
\bibitem [{\citenamefont {Kats}(2018)}]{kats_improving_2018}%
  \BibitemOpen
  \bibfield  {author} {\bibinfo {author} {\bibfnamefont {D.}~\bibnamefont
  {Kats}},\ }\bibfield  {title} {\bibinfo {title} {Improving the
  distinguishable cluster results: spin-component scaling},\ }\href
  {https://doi.org/10.1080/00268976.2017.1417646} {\bibfield  {journal}
  {\bibinfo  {journal} {Mol. Phys.}\ }\textbf {\bibinfo {volume} {116}},\
  \bibinfo {pages} {1435} (\bibinfo {year} {2018})}\BibitemShut {NoStop}%
\bibitem [{\citenamefont {Carr}\ and\ \citenamefont
  {Maradudin}(1964)}]{carr_ground-state_1964}%
  \BibitemOpen
  \bibfield  {author} {\bibinfo {author} {\bibfnamefont {W.~J.}\ \bibnamefont
  {Carr}}\ and\ \bibinfo {author} {\bibfnamefont {A.~A.}\ \bibnamefont
  {Maradudin}},\ }\bibfield  {title} {\bibinfo {title} {Ground-{State} {Energy}
  of a {High}-{Density} {Electron} {Gas}},\ }\href
  {https://doi.org/10.1103/PhysRev.133.A371} {\bibfield  {journal} {\bibinfo
  {journal} {Phys. Rev.}\ }\textbf {\bibinfo {volume} {133}},\ \bibinfo {pages}
  {A371} (\bibinfo {year} {1964})}\BibitemShut {NoStop}%
\bibitem [{\citenamefont {Endo}\ \emph {et~al.}(1999)\citenamefont {Endo},
  \citenamefont {Horiuchi}, \citenamefont {Takada},\ and\ \citenamefont
  {Yasuhara}}]{Endo1999}%
  \BibitemOpen
  \bibfield  {author} {\bibinfo {author} {\bibfnamefont {T.}~\bibnamefont
  {Endo}}, \bibinfo {author} {\bibfnamefont {M.}~\bibnamefont {Horiuchi}},
  \bibinfo {author} {\bibfnamefont {Y.}~\bibnamefont {Takada}},\ and\ \bibinfo
  {author} {\bibfnamefont {H.}~\bibnamefont {Yasuhara}},\ }\bibfield  {title}
  {\bibinfo {title} {High-density expansion of correlation energy and its
  extrapolation to the metallic density region},\ }\href
  {https://doi.org/10.1103/PhysRevB.59.7367} {\bibfield  {journal} {\bibinfo
  {journal} {Phys. Rev. B}\ }\textbf {\bibinfo {volume} {59}},\ \bibinfo
  {pages} {7367} (\bibinfo {year} {1999})}\BibitemShut {NoStop}%
\bibitem [{\citenamefont {Onsager}\ \emph {et~al.}(1966)\citenamefont
  {Onsager}, \citenamefont {Mittag},\ and\ \citenamefont
  {Stephen}}]{onsager_integrals_1966}%
  \BibitemOpen
  \bibfield  {author} {\bibinfo {author} {\bibfnamefont {L.}~\bibnamefont
  {Onsager}}, \bibinfo {author} {\bibfnamefont {L.}~\bibnamefont {Mittag}},\
  and\ \bibinfo {author} {\bibfnamefont {M.~J.}\ \bibnamefont {Stephen}},\
  }\bibfield  {title} {\bibinfo {title} {Integrals in the {Theory} of
  {Electron} {Correlations}},\ }\href
  {https://doi.org/10.1002/andp.19664730108} {\bibfield  {journal} {\bibinfo
  {journal} {Ann. Phys.}\ }\textbf {\bibinfo {volume} {473}},\ \bibinfo {pages}
  {71} (\bibinfo {year} {1966})}\BibitemShut {NoStop}%
\bibitem [{\citenamefont {Bohm}\ and\ \citenamefont
  {Pines}(1951)}]{bohm_collective_1951}%
  \BibitemOpen
  \bibfield  {author} {\bibinfo {author} {\bibfnamefont {D.}~\bibnamefont
  {Bohm}}\ and\ \bibinfo {author} {\bibfnamefont {D.}~\bibnamefont {Pines}},\
  }\bibfield  {title} {\bibinfo {title} {A {Collective} {Description} of
  {Electron} {Interactions}. {I}. {Magnetic} {Interactions}},\ }\href
  {https://doi.org/10.1103/PhysRev.82.625} {\bibfield  {journal} {\bibinfo
  {journal} {Phys. Rev.}\ }\textbf {\bibinfo {volume} {82}},\ \bibinfo {pages}
  {625} (\bibinfo {year} {1951})}\BibitemShut {NoStop}%
\bibitem [{\citenamefont {Pines}\ and\ \citenamefont
  {Bohm}(1952)}]{pines_collective_1952}%
  \BibitemOpen
  \bibfield  {author} {\bibinfo {author} {\bibfnamefont {D.}~\bibnamefont
  {Pines}}\ and\ \bibinfo {author} {\bibfnamefont {D.}~\bibnamefont {Bohm}},\
  }\bibfield  {title} {\bibinfo {title} {A {Collective} {Description} of
  {Electron} {Interactions}: {II}. {Collective} vs {Individual} {Particle}
  {Aspects} of the {Interactions}},\ }\href
  {https://doi.org/10.1103/PhysRev.85.338} {\bibfield  {journal} {\bibinfo
  {journal} {Phys. Rev.}\ }\textbf {\bibinfo {volume} {85}},\ \bibinfo {pages}
  {338} (\bibinfo {year} {1952})}\BibitemShut {NoStop}%
\bibitem [{\citenamefont {Bohm}\ and\ \citenamefont
  {Pines}(1953)}]{bohm_collective_1953}%
  \BibitemOpen
  \bibfield  {author} {\bibinfo {author} {\bibfnamefont {D.}~\bibnamefont
  {Bohm}}\ and\ \bibinfo {author} {\bibfnamefont {D.}~\bibnamefont {Pines}},\
  }\bibfield  {title} {\bibinfo {title} {A {Collective} {Description} of
  {Electron} {Interactions}: {III}. {Coulomb} {Interactions} in a {Degenerate}
  {Electron} {Gas}},\ }\href {https://doi.org/10.1103/PhysRev.92.609}
  {\bibfield  {journal} {\bibinfo  {journal} {Phys. Rev.}\ }\textbf {\bibinfo
  {volume} {92}},\ \bibinfo {pages} {609} (\bibinfo {year} {1953})}\BibitemShut
  {NoStop}%
\bibitem [{\citenamefont {Freeman}(1977)}]{freeman_coupled-cluster_1977}%
  \BibitemOpen
  \bibfield  {author} {\bibinfo {author} {\bibfnamefont {D.~L.}\ \bibnamefont
  {Freeman}},\ }\bibfield  {title} {\bibinfo {title} {Coupled-cluster expansion
  applied to the electron gas: {Inclusion} of ring and exchange effects},\
  }\href {https://doi.org/10.1103/PhysRevB.15.5512} {\bibfield  {journal}
  {\bibinfo  {journal} {Phys. Rev. B}\ }\textbf {\bibinfo {volume} {15}},\
  \bibinfo {pages} {5512} (\bibinfo {year} {1977})}\BibitemShut {NoStop}%
\bibitem [{\citenamefont {Bishop}\ and\ \citenamefont
  {Lührmann}(1978)}]{bishop_electron_1978}%
  \BibitemOpen
  \bibfield  {author} {\bibinfo {author} {\bibfnamefont {R.~F.}\ \bibnamefont
  {Bishop}}\ and\ \bibinfo {author} {\bibfnamefont {K.~H.}\ \bibnamefont
  {Lührmann}},\ }\bibfield  {title} {\bibinfo {title} {Electron correlations:
  {I}. {Ground}-state results in the high-density regime},\ }\href
  {https://doi.org/10.1103/PhysRevB.17.3757} {\bibfield  {journal} {\bibinfo
  {journal} {Phys. Rev. B}\ }\textbf {\bibinfo {volume} {17}},\ \bibinfo
  {pages} {3757} (\bibinfo {year} {1978})}\BibitemShut {NoStop}%
\bibitem [{\citenamefont {Scuseria}\ \emph {et~al.}(2008)\citenamefont
  {Scuseria}, \citenamefont {Henderson},\ and\ \citenamefont
  {Sorensen}}]{scuseria_ground_2008}%
  \BibitemOpen
  \bibfield  {author} {\bibinfo {author} {\bibfnamefont {G.~E.}\ \bibnamefont
  {Scuseria}}, \bibinfo {author} {\bibfnamefont {T.~M.}\ \bibnamefont
  {Henderson}},\ and\ \bibinfo {author} {\bibfnamefont {D.~C.}\ \bibnamefont
  {Sorensen}},\ }\bibfield  {title} {\bibinfo {title} {The ground state
  correlation energy of the random phase approximation from a ring coupled
  cluster doubles approach},\ }\href {https://doi.org/10.1063/1.3043729}
  {\bibfield  {journal} {\bibinfo  {journal} {J. Chem. Phys.}\ }\textbf
  {\bibinfo {volume} {129}},\ \bibinfo {pages} {231101} (\bibinfo {year}
  {2008})}\BibitemShut {NoStop}%
\bibitem [{\citenamefont {Bishop}\ and\ \citenamefont
  {Lührmann}(1982)}]{bishop_electron_1982}%
  \BibitemOpen
  \bibfield  {author} {\bibinfo {author} {\bibfnamefont {R.~F.}\ \bibnamefont
  {Bishop}}\ and\ \bibinfo {author} {\bibfnamefont {K.~H.}\ \bibnamefont
  {Lührmann}},\ }\bibfield  {title} {\bibinfo {title} {Electron correlations.
  {II}. {Ground}-state results at low and metallic densities},\ }\href
  {https://doi.org/10.1103/PhysRevB.26.5523} {\bibfield  {journal} {\bibinfo
  {journal} {Phys. Rev. B}\ }\textbf {\bibinfo {volume} {26}},\ \bibinfo
  {pages} {5523} (\bibinfo {year} {1982})}\BibitemShut {NoStop}%
\bibitem [{\citenamefont {Emrich}\ and\ \citenamefont
  {Zabolitzky}(1984)}]{emrich_electron_1984}%
  \BibitemOpen
  \bibfield  {author} {\bibinfo {author} {\bibfnamefont {K.}~\bibnamefont
  {Emrich}}\ and\ \bibinfo {author} {\bibfnamefont {J.~G.}\ \bibnamefont
  {Zabolitzky}},\ }\bibfield  {title} {\bibinfo {title} {Electron correlations
  in the {Bogoljubov} coupled-cluster formalism},\ }\href
  {https://doi.org/10.1103/PhysRevB.30.2049} {\bibfield  {journal} {\bibinfo
  {journal} {Phys. Rev. B}\ }\textbf {\bibinfo {volume} {30}},\ \bibinfo
  {pages} {2049} (\bibinfo {year} {1984})}\BibitemShut {NoStop}%
\bibitem [{\citenamefont {DuBois}(1959)}]{dubois_electron_1959}%
  \BibitemOpen
  \bibfield  {author} {\bibinfo {author} {\bibfnamefont {D.~F.}\ \bibnamefont
  {DuBois}},\ }\bibfield  {title} {\bibinfo {title} {Electron interactions:
  {Part} {II}. {Properties} of a dense electron gas},\ }\href
  {https://doi.org/https://doi.org/10.1016/0003-4916(59)90062-4} {\bibfield
  {journal} {\bibinfo  {journal} {Ann. Phys.}\ }\textbf {\bibinfo {volume}
  {8}},\ \bibinfo {pages} {24} (\bibinfo {year} {1959})}\BibitemShut {NoStop}%
\bibitem [{Note1()}]{Note1}%
  \BibitemOpen
  \bibinfo {note} {See Supplemental Material at [URL will be inserted by
  publisher] for technical details of all calculations and a discussion of
  finite-size errors in metals, including numerical demonstration of convergent
  and divergent behaviors of the considered theories.}\BibitemShut {Stop}%
\bibitem [{\citenamefont {Shepherd}\ \emph
  {et~al.}(2014{\natexlab{a}})\citenamefont {Shepherd}, \citenamefont
  {Henderson},\ and\ \citenamefont {Scuseria}}]{shepherd_range-separated_2014}%
  \BibitemOpen
  \bibfield  {author} {\bibinfo {author} {\bibfnamefont {J.~J.}\ \bibnamefont
  {Shepherd}}, \bibinfo {author} {\bibfnamefont {T.~M.}\ \bibnamefont
  {Henderson}},\ and\ \bibinfo {author} {\bibfnamefont {G.~E.}\ \bibnamefont
  {Scuseria}},\ }\bibfield  {title} {\bibinfo {title} {Range-{Separated}
  {Brueckner} {Coupled} {Cluster} {Doubles} {Theory}},\ }\href
  {https://doi.org/10.1103/PhysRevLett.112.133002} {\bibfield  {journal}
  {\bibinfo  {journal} {Phys. Rev. Lett.}\ }\textbf {\bibinfo {volume} {112}},\
  \bibinfo {pages} {133002} (\bibinfo {year} {2014}{\natexlab{a}})}\BibitemShut
  {NoStop}%
\bibitem [{\citenamefont {Shepherd}\ \emph
  {et~al.}(2014{\natexlab{b}})\citenamefont {Shepherd}, \citenamefont
  {Henderson},\ and\ \citenamefont {Scuseria}}]{shepherd_coupled_2014}%
  \BibitemOpen
  \bibfield  {author} {\bibinfo {author} {\bibfnamefont {J.~J.}\ \bibnamefont
  {Shepherd}}, \bibinfo {author} {\bibfnamefont {T.~M.}\ \bibnamefont
  {Henderson}},\ and\ \bibinfo {author} {\bibfnamefont {G.~E.}\ \bibnamefont
  {Scuseria}},\ }\bibfield  {title} {\bibinfo {title} {Coupled cluster channels
  in the homogeneous electron gas},\ }\href {https://doi.org/10.1063/1.4867783}
  {\bibfield  {journal} {\bibinfo  {journal} {J. Chem. Phys.}\ }\textbf
  {\bibinfo {volume} {140}},\ \bibinfo {pages} {124102} (\bibinfo {year}
  {2014}{\natexlab{b}})}\BibitemShut {NoStop}%
\bibitem [{\citenamefont {Hedin}(1965)}]{hedin_new_1965}%
  \BibitemOpen
  \bibfield  {author} {\bibinfo {author} {\bibfnamefont {L.}~\bibnamefont
  {Hedin}},\ }\bibfield  {title} {\bibinfo {title} {New {Method} for
  {Calculating} the {One}-{Particle} {Green}'s {Function} with {Application} to
  the {Electron}-{Gas} {Problem}},\ }\href
  {https://doi.org/10.1103/PhysRev.139.A796} {\bibfield  {journal} {\bibinfo
  {journal} {Phys. Rev.}\ }\textbf {\bibinfo {volume} {139}},\ \bibinfo {pages}
  {A796} (\bibinfo {year} {1965})}\BibitemShut {NoStop}%
\bibitem [{\citenamefont {Lee}\ and\ \citenamefont
  {Bartlett}(1984)}]{lee_study_1984}%
  \BibitemOpen
  \bibfield  {author} {\bibinfo {author} {\bibfnamefont {Y.~S.}\ \bibnamefont
  {Lee}}\ and\ \bibinfo {author} {\bibfnamefont {R.~J.}\ \bibnamefont
  {Bartlett}},\ }\bibfield  {title} {\bibinfo {title} {A study of {Be}
  $_{\textrm{2}}$ with many‐body perturbation theory and a coupled‐cluster
  method including triple excitations},\ }\href
  {https://doi.org/10.1063/1.447214} {\bibfield  {journal} {\bibinfo  {journal}
  {J. Chem. Phys.}\ }\textbf {\bibinfo {volume} {80}},\ \bibinfo {pages} {4371}
  (\bibinfo {year} {1984})}\BibitemShut {NoStop}%
\bibitem [{\citenamefont {Lee}\ \emph {et~al.}(1984)\citenamefont {Lee},
  \citenamefont {Kucharski},\ and\ \citenamefont
  {Bartlett}}]{lee_coupled_1984}%
  \BibitemOpen
  \bibfield  {author} {\bibinfo {author} {\bibfnamefont {Y.~S.}\ \bibnamefont
  {Lee}}, \bibinfo {author} {\bibfnamefont {S.~A.}\ \bibnamefont {Kucharski}},\
  and\ \bibinfo {author} {\bibfnamefont {R.~J.}\ \bibnamefont {Bartlett}},\
  }\bibfield  {title} {\bibinfo {title} {A coupled cluster approach with triple
  excitations},\ }\href {https://doi.org/10.1063/1.447591} {\bibfield
  {journal} {\bibinfo  {journal} {J. Chem. Phys.}\ }\textbf {\bibinfo {volume}
  {81}},\ \bibinfo {pages} {5906} (\bibinfo {year} {1984})}\BibitemShut
  {NoStop}%
\bibitem [{\citenamefont {Urban}\ \emph {et~al.}(1985)\citenamefont {Urban},
  \citenamefont {Noga}, \citenamefont {Cole},\ and\ \citenamefont
  {Bartlett}}]{urban_towards_1985}%
  \BibitemOpen
  \bibfield  {author} {\bibinfo {author} {\bibfnamefont {M.}~\bibnamefont
  {Urban}}, \bibinfo {author} {\bibfnamefont {J.}~\bibnamefont {Noga}},
  \bibinfo {author} {\bibfnamefont {S.~J.}\ \bibnamefont {Cole}},\ and\
  \bibinfo {author} {\bibfnamefont {R.~J.}\ \bibnamefont {Bartlett}},\
  }\bibfield  {title} {\bibinfo {title} {Towards a full {CCSDT} model for
  electron correlation},\ }\href {https://doi.org/10.1063/1.449067} {\bibfield
  {journal} {\bibinfo  {journal} {J. Chem. Phys.}\ }\textbf {\bibinfo {volume}
  {83}},\ \bibinfo {pages} {4041} (\bibinfo {year} {1985})}\BibitemShut
  {NoStop}%
\bibitem [{\citenamefont {Noga}\ \emph {et~al.}(1987)\citenamefont {Noga},
  \citenamefont {Bartlett},\ and\ \citenamefont {Urban}}]{noga_towards_1987}%
  \BibitemOpen
  \bibfield  {author} {\bibinfo {author} {\bibfnamefont {J.}~\bibnamefont
  {Noga}}, \bibinfo {author} {\bibfnamefont {R.~J.}\ \bibnamefont {Bartlett}},\
  and\ \bibinfo {author} {\bibfnamefont {M.}~\bibnamefont {Urban}},\ }\bibfield
   {title} {\bibinfo {title} {Towards a full {CCSDT} model for electron
  correlation. {CCSDT}-n models},\ }\href
  {https://doi.org/10.1016/0009-2614(87)87107-5} {\bibfield  {journal}
  {\bibinfo  {journal} {Chem. Phys. Lett.}\ }\textbf {\bibinfo {volume}
  {134}},\ \bibinfo {pages} {126} (\bibinfo {year} {1987})}\BibitemShut
  {NoStop}%
\bibitem [{\citenamefont {Lange}\ and\ \citenamefont
  {Berkelbach}(2018)}]{lange_relation_2018}%
  \BibitemOpen
  \bibfield  {author} {\bibinfo {author} {\bibfnamefont {M.~F.}\ \bibnamefont
  {Lange}}\ and\ \bibinfo {author} {\bibfnamefont {T.~C.}\ \bibnamefont
  {Berkelbach}},\ }\bibfield  {title} {\bibinfo {title} {On the {Relation}
  between {Equation}-of-{Motion} {Coupled}-{Cluster} {Theory} and the
  \textit{{GW}} {Approximation}},\ }\href
  {https://doi.org/10.1021/acs.jctc.8b00455} {\bibfield  {journal} {\bibinfo
  {journal} {J. Chem. Theor. Comput.}\ }\textbf {\bibinfo {volume} {14}},\
  \bibinfo {pages} {4224} (\bibinfo {year} {2018})}\BibitemShut {NoStop}%
\bibitem [{\citenamefont {Tölle}\ and\ \citenamefont
  {Chan}(2022)}]{tolle_exact_2022}%
  \BibitemOpen
  \bibfield  {author} {\bibinfo {author} {\bibfnamefont {J.}~\bibnamefont
  {Tölle}}\ and\ \bibinfo {author} {\bibfnamefont {G.~K.-L.}\ \bibnamefont
  {Chan}},\ }\bibfield  {title} {\bibinfo {title} {Exact relationships between
  the {GW} approximation and equation-of-motion coupled-cluster theories
  through the quasi-boson formalism},\ }\href
  {https://arxiv.org/abs/2212.08982} {\bibfield  {journal} {\bibinfo  {journal}
  {arXiv:2212.08982 [cond-mat, physics:physics]}\ } (\bibinfo {year}
  {2022})}\BibitemShut {NoStop}%
\bibitem [{\citenamefont {Ceperley}\ and\ \citenamefont
  {Alder}(1980)}]{ceperley_ground_1980}%
  \BibitemOpen
  \bibfield  {author} {\bibinfo {author} {\bibfnamefont {D.~M.}\ \bibnamefont
  {Ceperley}}\ and\ \bibinfo {author} {\bibfnamefont {B.~J.}\ \bibnamefont
  {Alder}},\ }\bibfield  {title} {\bibinfo {title} {Ground {State} of the
  {Electron} {Gas} by a {Stochastic} {Method}},\ }\href
  {https://doi.org/10.1103/PhysRevLett.45.566} {\bibfield  {journal} {\bibinfo
  {journal} {Phys. Rev. Lett.}\ }\textbf {\bibinfo {volume} {45}},\ \bibinfo
  {pages} {566} (\bibinfo {year} {1980})}\BibitemShut {NoStop}%
\bibitem [{\citenamefont {Perdew}\ and\ \citenamefont
  {Zunger}(1981)}]{perdew_self-interaction_1981}%
  \BibitemOpen
  \bibfield  {author} {\bibinfo {author} {\bibfnamefont {J.~P.}\ \bibnamefont
  {Perdew}}\ and\ \bibinfo {author} {\bibfnamefont {A.}~\bibnamefont
  {Zunger}},\ }\bibfield  {title} {\bibinfo {title} {Self-interaction
  correction to density-functional approximations for many-electron systems},\
  }\href {https://doi.org/10.1103/PhysRevB.23.5048} {\bibfield  {journal}
  {\bibinfo  {journal} {Phys. Rev. B}\ }\textbf {\bibinfo {volume} {23}},\
  \bibinfo {pages} {5048} (\bibinfo {year} {1981})}\BibitemShut {NoStop}%
\bibitem [{\citenamefont {McClain}\ \emph {et~al.}(2016)\citenamefont
  {McClain}, \citenamefont {Lischner}, \citenamefont {Watson}, \citenamefont
  {Matthews}, \citenamefont {Ronca}, \citenamefont {Louie}, \citenamefont
  {Berkelbach},\ and\ \citenamefont {Chan}}]{mcclain_spectral_2016}%
  \BibitemOpen
  \bibfield  {author} {\bibinfo {author} {\bibfnamefont {J.}~\bibnamefont
  {McClain}}, \bibinfo {author} {\bibfnamefont {J.}~\bibnamefont {Lischner}},
  \bibinfo {author} {\bibfnamefont {T.}~\bibnamefont {Watson}}, \bibinfo
  {author} {\bibfnamefont {D.~A.}\ \bibnamefont {Matthews}}, \bibinfo {author}
  {\bibfnamefont {E.}~\bibnamefont {Ronca}}, \bibinfo {author} {\bibfnamefont
  {S.~G.}\ \bibnamefont {Louie}}, \bibinfo {author} {\bibfnamefont {T.~C.}\
  \bibnamefont {Berkelbach}},\ and\ \bibinfo {author} {\bibfnamefont
  {G.~K.-L.}\ \bibnamefont {Chan}},\ }\bibfield  {title} {\bibinfo {title}
  {Spectral functions of the uniform electron gas via coupled-cluster theory
  and comparison to the {G} {W} and related approximations},\ }\href
  {https://doi.org/10.1103/PhysRevB.93.235139} {\bibfield  {journal} {\bibinfo
  {journal} {Phys. Rev. B}\ }\textbf {\bibinfo {volume} {93}},\ \bibinfo
  {pages} {235139} (\bibinfo {year} {2016})}\BibitemShut {NoStop}%
\bibitem [{\citenamefont {Spencer}\ and\ \citenamefont
  {Thom}(2016)}]{spencer_developments_2016}%
  \BibitemOpen
  \bibfield  {author} {\bibinfo {author} {\bibfnamefont {J.~S.}\ \bibnamefont
  {Spencer}}\ and\ \bibinfo {author} {\bibfnamefont {A.~J.~W.}\ \bibnamefont
  {Thom}},\ }\bibfield  {title} {\bibinfo {title} {Developments in stochastic
  coupled cluster theory: {The} initiator approximation and application to the
  uniform electron gas},\ }\href {https://doi.org/10.1063/1.4942173} {\bibfield
   {journal} {\bibinfo  {journal} {J. Chem. Phys.}\ }\textbf {\bibinfo {volume}
  {144}},\ \bibinfo {pages} {084108} (\bibinfo {year} {2016})}\BibitemShut
  {NoStop}%
\bibitem [{\citenamefont {Neufeld}\ and\ \citenamefont
  {Thom}(2017)}]{neufeld_study_2017}%
  \BibitemOpen
  \bibfield  {author} {\bibinfo {author} {\bibfnamefont {V.~A.}\ \bibnamefont
  {Neufeld}}\ and\ \bibinfo {author} {\bibfnamefont {A.~J.~W.}\ \bibnamefont
  {Thom}},\ }\bibfield  {title} {\bibinfo {title} {A study of the dense uniform
  electron gas with high orders of coupled cluster},\ }\href
  {https://doi.org/10.1063/1.5003794} {\bibfield  {journal} {\bibinfo
  {journal} {J. Chem. Phys.}\ }\textbf {\bibinfo {volume} {147}},\ \bibinfo
  {pages} {194105} (\bibinfo {year} {2017})}\BibitemShut {NoStop}%
\bibitem [{\citenamefont {Liao}\ \emph {et~al.}(2021)\citenamefont {Liao},
  \citenamefont {Schraivogel}, \citenamefont {Luo}, \citenamefont {Kats},\ and\
  \citenamefont {Alavi}}]{liao_towards_2021}%
  \BibitemOpen
  \bibfield  {author} {\bibinfo {author} {\bibfnamefont {K.}~\bibnamefont
  {Liao}}, \bibinfo {author} {\bibfnamefont {T.}~\bibnamefont {Schraivogel}},
  \bibinfo {author} {\bibfnamefont {H.}~\bibnamefont {Luo}}, \bibinfo {author}
  {\bibfnamefont {D.}~\bibnamefont {Kats}},\ and\ \bibinfo {author}
  {\bibfnamefont {A.}~\bibnamefont {Alavi}},\ }\bibfield  {title} {\bibinfo
  {title} {Towards efficient and accurate \textit{ab initio} solutions to
  periodic systems via transcorrelation and coupled cluster theory},\ }\href
  {https://doi.org/10.1103/PhysRevResearch.3.033072} {\bibfield  {journal}
  {\bibinfo  {journal} {Phys. Rev. Res.}\ }\textbf {\bibinfo {volume} {3}},\
  \bibinfo {pages} {033072} (\bibinfo {year} {2021})}\BibitemShut {NoStop}%
\bibitem [{\citenamefont {Mihm}\ \emph {et~al.}(2023)\citenamefont {Mihm},
  \citenamefont {Weiler},\ and\ \citenamefont {Shepherd}}]{mihm_how_2023}%
  \BibitemOpen
  \bibfield  {author} {\bibinfo {author} {\bibfnamefont {T.~N.}\ \bibnamefont
  {Mihm}}, \bibinfo {author} {\bibfnamefont {L.}~\bibnamefont {Weiler}},\ and\
  \bibinfo {author} {\bibfnamefont {J.~J.}\ \bibnamefont {Shepherd}},\
  }\bibfield  {title} {\bibinfo {title} {How the {Exchange} {Energy} {Can}
  {Affect} the {Power} {Laws} {Used} to {Extrapolate} the {Coupled} {Cluster}
  {Correlation} {Energy} to the {Thermodynamic} {Limit}},\ }\bibfield
  {journal} {\bibinfo  {journal} {J. Chem. Theor. Comput.}\ }\href
  {https://doi.org/10.1021/acs.jctc.2c00737} {10.1021/acs.jctc.2c00737}
  (\bibinfo {year} {2023})\BibitemShut {NoStop}%
\bibitem [{\citenamefont {Sun}(2015)}]{sun_libcint_2015}%
  \BibitemOpen
  \bibfield  {author} {\bibinfo {author} {\bibfnamefont {Q.}~\bibnamefont
  {Sun}},\ }\bibfield  {title} {\bibinfo {title} {Libcint: {An} efficient
  general integral library for {Gaussian} basis functions},\ }\href
  {https://doi.org/https://doi.org/10.1002/jcc.23981} {\bibfield  {journal}
  {\bibinfo  {journal} {J. Comput. Chem.}\ }\textbf {\bibinfo {volume} {36}},\
  \bibinfo {pages} {1664} (\bibinfo {year} {2015})}\BibitemShut {NoStop}%
\bibitem [{\citenamefont {Sun}\ \emph {et~al.}(2018)\citenamefont {Sun},
  \citenamefont {Berkelbach}, \citenamefont {Blunt}, \citenamefont {Booth},
  \citenamefont {Guo}, \citenamefont {Li}, \citenamefont {Liu}, \citenamefont
  {McClain}, \citenamefont {Sayfutyarova}, \citenamefont {Sharma},
  \citenamefont {Wouters},\ and\ \citenamefont {Chan}}]{sun_pyscf_2018}%
  \BibitemOpen
  \bibfield  {author} {\bibinfo {author} {\bibfnamefont {Q.}~\bibnamefont
  {Sun}}, \bibinfo {author} {\bibfnamefont {T.~C.}\ \bibnamefont {Berkelbach}},
  \bibinfo {author} {\bibfnamefont {N.~S.}\ \bibnamefont {Blunt}}, \bibinfo
  {author} {\bibfnamefont {G.~H.}\ \bibnamefont {Booth}}, \bibinfo {author}
  {\bibfnamefont {S.}~\bibnamefont {Guo}}, \bibinfo {author} {\bibfnamefont
  {Z.}~\bibnamefont {Li}}, \bibinfo {author} {\bibfnamefont {J.}~\bibnamefont
  {Liu}}, \bibinfo {author} {\bibfnamefont {J.~D.}\ \bibnamefont {McClain}},
  \bibinfo {author} {\bibfnamefont {E.~R.}\ \bibnamefont {Sayfutyarova}},
  \bibinfo {author} {\bibfnamefont {S.}~\bibnamefont {Sharma}}, \bibinfo
  {author} {\bibfnamefont {S.}~\bibnamefont {Wouters}},\ and\ \bibinfo {author}
  {\bibfnamefont {G.~K.-L.}\ \bibnamefont {Chan}},\ }\bibfield  {title}
  {\bibinfo {title} {{PySCF}: the {Python}-based simulations of chemistry
  framework},\ }\href {https://doi.org/https://doi.org/10.1002/wcms.1340}
  {\bibfield  {journal} {\bibinfo  {journal} {WIREs Comput. Mol. Sci.}\
  }\textbf {\bibinfo {volume} {8}},\ \bibinfo {pages} {e1340} (\bibinfo {year}
  {2018})}\BibitemShut {NoStop}%
\bibitem [{\citenamefont {Sun}\ \emph {et~al.}(2020)\citenamefont {Sun},
  \citenamefont {Zhang}, \citenamefont {Banerjee}, \citenamefont {Bao},
  \citenamefont {Barbry}, \citenamefont {Blunt}, \citenamefont {Bogdanov},
  \citenamefont {Booth}, \citenamefont {Chen}, \citenamefont {Cui},
  \citenamefont {Eriksen}, \citenamefont {Gao}, \citenamefont {Guo},
  \citenamefont {Hermann}, \citenamefont {Hermes}, \citenamefont {Koh},
  \citenamefont {Koval}, \citenamefont {Lehtola}, \citenamefont {Li},
  \citenamefont {Liu}, \citenamefont {Mardirossian}, \citenamefont {McClain},
  \citenamefont {Motta}, \citenamefont {Mussard}, \citenamefont {Pham},
  \citenamefont {Pulkin}, \citenamefont {Purwanto}, \citenamefont {Robinson},
  \citenamefont {Ronca}, \citenamefont {Sayfutyarova}, \citenamefont
  {Scheurer}, \citenamefont {Schurkus}, \citenamefont {Smith}, \citenamefont
  {Sun}, \citenamefont {Sun}, \citenamefont {Upadhyay}, \citenamefont {Wagner},
  \citenamefont {Wang}, \citenamefont {White}, \citenamefont {Whitfield},
  \citenamefont {Williamson}, \citenamefont {Wouters}, \citenamefont {Yang},
  \citenamefont {Yu}, \citenamefont {Zhu}, \citenamefont {Berkelbach},
  \citenamefont {Sharma}, \citenamefont {Sokolov},\ and\ \citenamefont
  {Chan}}]{sun_recent_2020}%
  \BibitemOpen
  \bibfield  {author} {\bibinfo {author} {\bibfnamefont {Q.}~\bibnamefont
  {Sun}}, \bibinfo {author} {\bibfnamefont {X.}~\bibnamefont {Zhang}}, \bibinfo
  {author} {\bibfnamefont {S.}~\bibnamefont {Banerjee}}, \bibinfo {author}
  {\bibfnamefont {P.}~\bibnamefont {Bao}}, \bibinfo {author} {\bibfnamefont
  {M.}~\bibnamefont {Barbry}}, \bibinfo {author} {\bibfnamefont {N.~S.}\
  \bibnamefont {Blunt}}, \bibinfo {author} {\bibfnamefont {N.~A.}\ \bibnamefont
  {Bogdanov}}, \bibinfo {author} {\bibfnamefont {G.~H.}\ \bibnamefont {Booth}},
  \bibinfo {author} {\bibfnamefont {J.}~\bibnamefont {Chen}}, \bibinfo {author}
  {\bibfnamefont {Z.-H.}\ \bibnamefont {Cui}}, \bibinfo {author} {\bibfnamefont
  {J.~J.}\ \bibnamefont {Eriksen}}, \bibinfo {author} {\bibfnamefont
  {Y.}~\bibnamefont {Gao}}, \bibinfo {author} {\bibfnamefont {S.}~\bibnamefont
  {Guo}}, \bibinfo {author} {\bibfnamefont {J.}~\bibnamefont {Hermann}},
  \bibinfo {author} {\bibfnamefont {M.~R.}\ \bibnamefont {Hermes}}, \bibinfo
  {author} {\bibfnamefont {K.}~\bibnamefont {Koh}}, \bibinfo {author}
  {\bibfnamefont {P.}~\bibnamefont {Koval}}, \bibinfo {author} {\bibfnamefont
  {S.}~\bibnamefont {Lehtola}}, \bibinfo {author} {\bibfnamefont
  {Z.}~\bibnamefont {Li}}, \bibinfo {author} {\bibfnamefont {J.}~\bibnamefont
  {Liu}}, \bibinfo {author} {\bibfnamefont {N.}~\bibnamefont {Mardirossian}},
  \bibinfo {author} {\bibfnamefont {J.~D.}\ \bibnamefont {McClain}}, \bibinfo
  {author} {\bibfnamefont {M.}~\bibnamefont {Motta}}, \bibinfo {author}
  {\bibfnamefont {B.}~\bibnamefont {Mussard}}, \bibinfo {author} {\bibfnamefont
  {H.~Q.}\ \bibnamefont {Pham}}, \bibinfo {author} {\bibfnamefont
  {A.}~\bibnamefont {Pulkin}}, \bibinfo {author} {\bibfnamefont
  {W.}~\bibnamefont {Purwanto}}, \bibinfo {author} {\bibfnamefont {P.~J.}\
  \bibnamefont {Robinson}}, \bibinfo {author} {\bibfnamefont {E.}~\bibnamefont
  {Ronca}}, \bibinfo {author} {\bibfnamefont {E.~R.}\ \bibnamefont
  {Sayfutyarova}}, \bibinfo {author} {\bibfnamefont {M.}~\bibnamefont
  {Scheurer}}, \bibinfo {author} {\bibfnamefont {H.~F.}\ \bibnamefont
  {Schurkus}}, \bibinfo {author} {\bibfnamefont {J.~E.~T.}\ \bibnamefont
  {Smith}}, \bibinfo {author} {\bibfnamefont {C.}~\bibnamefont {Sun}}, \bibinfo
  {author} {\bibfnamefont {S.-N.}\ \bibnamefont {Sun}}, \bibinfo {author}
  {\bibfnamefont {S.}~\bibnamefont {Upadhyay}}, \bibinfo {author}
  {\bibfnamefont {L.~K.}\ \bibnamefont {Wagner}}, \bibinfo {author}
  {\bibfnamefont {X.}~\bibnamefont {Wang}}, \bibinfo {author} {\bibfnamefont
  {A.}~\bibnamefont {White}}, \bibinfo {author} {\bibfnamefont {J.~D.}\
  \bibnamefont {Whitfield}}, \bibinfo {author} {\bibfnamefont {M.~J.}\
  \bibnamefont {Williamson}}, \bibinfo {author} {\bibfnamefont
  {S.}~\bibnamefont {Wouters}}, \bibinfo {author} {\bibfnamefont
  {J.}~\bibnamefont {Yang}}, \bibinfo {author} {\bibfnamefont {J.~M.}\
  \bibnamefont {Yu}}, \bibinfo {author} {\bibfnamefont {T.}~\bibnamefont
  {Zhu}}, \bibinfo {author} {\bibfnamefont {T.~C.}\ \bibnamefont {Berkelbach}},
  \bibinfo {author} {\bibfnamefont {S.}~\bibnamefont {Sharma}}, \bibinfo
  {author} {\bibfnamefont {A.~Y.}\ \bibnamefont {Sokolov}},\ and\ \bibinfo
  {author} {\bibfnamefont {G.~K.-L.}\ \bibnamefont {Chan}},\ }\bibfield
  {title} {\bibinfo {title} {Recent developments in the {PySCF} program
  package},\ }\href {https://doi.org/10.1063/5.0006074} {\bibfield  {journal}
  {\bibinfo  {journal} {J. Chem. Phys.}\ }\textbf {\bibinfo {volume} {153}},\
  \bibinfo {pages} {024109} (\bibinfo {year} {2020})}\BibitemShut {NoStop}%
\bibitem [{\citenamefont {Zhang}\ \emph {et~al.}(2018)\citenamefont {Zhang},
  \citenamefont {Reilly}, \citenamefont {Tkatchenko},\ and\ \citenamefont
  {Scheffler}}]{zhang_performance_2018}%
  \BibitemOpen
  \bibfield  {author} {\bibinfo {author} {\bibfnamefont {G.-X.}\ \bibnamefont
  {Zhang}}, \bibinfo {author} {\bibfnamefont {A.~M.}\ \bibnamefont {Reilly}},
  \bibinfo {author} {\bibfnamefont {A.}~\bibnamefont {Tkatchenko}},\ and\
  \bibinfo {author} {\bibfnamefont {M.}~\bibnamefont {Scheffler}},\ }\bibfield
  {title} {\bibinfo {title} {Performance of various density-functional
  approximations for cohesive properties of 64 bulk solids},\ }\href
  {https://doi.org/10.1088/1367-2630/aac7f0} {\bibfield  {journal} {\bibinfo
  {journal} {New J. Phys.}\ }\textbf {\bibinfo {volume} {20}},\ \bibinfo
  {pages} {063020} (\bibinfo {year} {2018})}\BibitemShut {NoStop}%
\bibitem [{\citenamefont {Berliner}\ and\ \citenamefont
  {Werner}(1986)}]{berliner_effect_1986}%
  \BibitemOpen
  \bibfield  {author} {\bibinfo {author} {\bibfnamefont {R.}~\bibnamefont
  {Berliner}}\ and\ \bibinfo {author} {\bibfnamefont {S.~A.}\ \bibnamefont
  {Werner}},\ }\bibfield  {title} {\bibinfo {title} {Effect of stacking faults
  on diffraction: {The} structure of lithium metal},\ }\href
  {https://doi.org/10.1103/PhysRevB.34.3586} {\bibfield  {journal} {\bibinfo
  {journal} {Phys. Rev. B}\ }\textbf {\bibinfo {volume} {34}},\ \bibinfo
  {pages} {3586} (\bibinfo {year} {1986})}\BibitemShut {NoStop}%
\bibitem [{\citenamefont {Felice}\ \emph {et~al.}(1977)\citenamefont {Felice},
  \citenamefont {Trivisonno},\ and\ \citenamefont
  {Schuele}}]{felice_temperature_1977}%
  \BibitemOpen
  \bibfield  {author} {\bibinfo {author} {\bibfnamefont {R.~A.}\ \bibnamefont
  {Felice}}, \bibinfo {author} {\bibfnamefont {J.}~\bibnamefont {Trivisonno}},\
  and\ \bibinfo {author} {\bibfnamefont {D.~E.}\ \bibnamefont {Schuele}},\
  }\bibfield  {title} {\bibinfo {title} {Temperature and pressure dependence of
  the single-crystal elastic constants of {Li} 6 and natural lithium},\ }\href
  {https://doi.org/10.1103/PhysRevB.16.5173} {\bibfield  {journal} {\bibinfo
  {journal} {Phys. Rev. B}\ }\textbf {\bibinfo {volume} {16}},\ \bibinfo
  {pages} {5173} (\bibinfo {year} {1977})}\BibitemShut {NoStop}%
\bibitem [{\citenamefont {Kittel}(2005)}]{kittel_intro_solid_2005}%
  \BibitemOpen
  \bibfield  {author} {\bibinfo {author} {\bibfnamefont {C.}~\bibnamefont
  {Kittel}},\ }\href@noop {} {\emph {\bibinfo {title} {{Introduction} to
  {solid} {state} {physics}}}},\ \bibinfo {edition} {8th}\ ed.\ (\bibinfo
  {publisher} {John Wiley \& Sons, Inc.},\ \bibinfo {year} {2005})\BibitemShut
  {NoStop}%
\bibitem [{\citenamefont {Kohn}\ and\ \citenamefont
  {Sham}(1965)}]{kohn_self-consistent_1965}%
  \BibitemOpen
  \bibfield  {author} {\bibinfo {author} {\bibfnamefont {W.}~\bibnamefont
  {Kohn}}\ and\ \bibinfo {author} {\bibfnamefont {L.~J.}\ \bibnamefont
  {Sham}},\ }\bibfield  {title} {\bibinfo {title} {Self-{consistent}
  {equations} {including} {exchange} and {correlation} {effects}},\ }\href
  {https://doi.org/10.1103/PhysRev.140.A1133} {\bibfield  {journal} {\bibinfo
  {journal} {Phys. Rev.}\ }\textbf {\bibinfo {volume} {140}},\ \bibinfo {pages}
  {A1133} (\bibinfo {year} {1965})}\BibitemShut {NoStop}%
\bibitem [{\citenamefont {Heyd}\ \emph {et~al.}(2003)\citenamefont {Heyd},
  \citenamefont {Scuseria},\ and\ \citenamefont
  {Ernzerhof}}]{heyd_hybrid_2003}%
  \BibitemOpen
  \bibfield  {author} {\bibinfo {author} {\bibfnamefont {J.}~\bibnamefont
  {Heyd}}, \bibinfo {author} {\bibfnamefont {G.~E.}\ \bibnamefont {Scuseria}},\
  and\ \bibinfo {author} {\bibfnamefont {M.}~\bibnamefont {Ernzerhof}},\
  }\bibfield  {title} {\bibinfo {title} {Hybrid functionals based on a screened
  {Coulomb} potential},\ }\href {https://doi.org/10.1063/1.1564060} {\bibfield
  {journal} {\bibinfo  {journal} {J. Chem. Phys.}\ }\textbf {\bibinfo {volume}
  {118}},\ \bibinfo {pages} {8207} (\bibinfo {year} {2003})}\BibitemShut
  {NoStop}%
\bibitem [{\citenamefont {Heyd}\ \emph {et~al.}(2006)\citenamefont {Heyd},
  \citenamefont {Scuseria},\ and\ \citenamefont
  {Ernzerhof}}]{heyd_erratum_2006}%
  \BibitemOpen
  \bibfield  {author} {\bibinfo {author} {\bibfnamefont {J.}~\bibnamefont
  {Heyd}}, \bibinfo {author} {\bibfnamefont {G.~E.}\ \bibnamefont {Scuseria}},\
  and\ \bibinfo {author} {\bibfnamefont {M.}~\bibnamefont {Ernzerhof}},\
  }\bibfield  {title} {\bibinfo {title} {Erratum: “{Hybrid} functionals based
  on a screened {Coulomb} potential” [{J}. {Chem}. {Phys}. 118, 8207
  (2003)]},\ }\href {https://doi.org/10.1063/1.2204597} {\bibfield  {journal}
  {\bibinfo  {journal} {J. Chem. Phys.}\ }\textbf {\bibinfo {volume} {124}},\
  \bibinfo {pages} {219906} (\bibinfo {year} {2006})}\BibitemShut {NoStop}%
\bibitem [{\citenamefont {Krukau}\ \emph {et~al.}(2006)\citenamefont {Krukau},
  \citenamefont {Vydrov}, \citenamefont {Izmaylov},\ and\ \citenamefont
  {Scuseria}}]{krukau_influence_2006}%
  \BibitemOpen
  \bibfield  {author} {\bibinfo {author} {\bibfnamefont {A.~V.}\ \bibnamefont
  {Krukau}}, \bibinfo {author} {\bibfnamefont {O.~A.}\ \bibnamefont {Vydrov}},
  \bibinfo {author} {\bibfnamefont {A.~F.}\ \bibnamefont {Izmaylov}},\ and\
  \bibinfo {author} {\bibfnamefont {G.~E.}\ \bibnamefont {Scuseria}},\
  }\bibfield  {title} {\bibinfo {title} {Influence of the exchange screening
  parameter on the performance of screened hybrid functionals},\ }\href
  {https://doi.org/10.1063/1.2404663} {\bibfield  {journal} {\bibinfo
  {journal} {J. Chem. Phys.}\ }\textbf {\bibinfo {volume} {125}},\ \bibinfo
  {pages} {224106} (\bibinfo {year} {2006})}\BibitemShut {NoStop}%
\bibitem [{\citenamefont {Harris}\ \emph {et~al.}(2020)\citenamefont {Harris},
  \citenamefont {Millman}, \citenamefont {van~der Walt}, \citenamefont
  {Gommers}, \citenamefont {Virtanen}, \citenamefont {Cournapeau},
  \citenamefont {Wieser}, \citenamefont {Taylor}, \citenamefont {Berg},
  \citenamefont {Smith}, \citenamefont {Kern}, \citenamefont {Picus},
  \citenamefont {Hoyer}, \citenamefont {van Kerkwijk}, \citenamefont {Brett},
  \citenamefont {Haldane}, \citenamefont {del Río}, \citenamefont {Wiebe},
  \citenamefont {Peterson}, \citenamefont {Gérard-Marchant}, \citenamefont
  {Sheppard}, \citenamefont {Reddy}, \citenamefont {Weckesser}, \citenamefont
  {Abbasi}, \citenamefont {Gohlke},\ and\ \citenamefont
  {Oliphant}}]{harris_array_2020}%
  \BibitemOpen
  \bibfield  {author} {\bibinfo {author} {\bibfnamefont {C.~R.}\ \bibnamefont
  {Harris}}, \bibinfo {author} {\bibfnamefont {K.~J.}\ \bibnamefont {Millman}},
  \bibinfo {author} {\bibfnamefont {S.~J.}\ \bibnamefont {van~der Walt}},
  \bibinfo {author} {\bibfnamefont {R.}~\bibnamefont {Gommers}}, \bibinfo
  {author} {\bibfnamefont {P.}~\bibnamefont {Virtanen}}, \bibinfo {author}
  {\bibfnamefont {D.}~\bibnamefont {Cournapeau}}, \bibinfo {author}
  {\bibfnamefont {E.}~\bibnamefont {Wieser}}, \bibinfo {author} {\bibfnamefont
  {J.}~\bibnamefont {Taylor}}, \bibinfo {author} {\bibfnamefont
  {S.}~\bibnamefont {Berg}}, \bibinfo {author} {\bibfnamefont {N.~J.}\
  \bibnamefont {Smith}}, \bibinfo {author} {\bibfnamefont {R.}~\bibnamefont
  {Kern}}, \bibinfo {author} {\bibfnamefont {M.}~\bibnamefont {Picus}},
  \bibinfo {author} {\bibfnamefont {S.}~\bibnamefont {Hoyer}}, \bibinfo
  {author} {\bibfnamefont {M.~H.}\ \bibnamefont {van Kerkwijk}}, \bibinfo
  {author} {\bibfnamefont {M.}~\bibnamefont {Brett}}, \bibinfo {author}
  {\bibfnamefont {A.}~\bibnamefont {Haldane}}, \bibinfo {author} {\bibfnamefont
  {J.~F.}\ \bibnamefont {del Río}}, \bibinfo {author} {\bibfnamefont
  {M.}~\bibnamefont {Wiebe}}, \bibinfo {author} {\bibfnamefont
  {P.}~\bibnamefont {Peterson}}, \bibinfo {author} {\bibfnamefont
  {P.}~\bibnamefont {Gérard-Marchant}}, \bibinfo {author} {\bibfnamefont
  {K.}~\bibnamefont {Sheppard}}, \bibinfo {author} {\bibfnamefont
  {T.}~\bibnamefont {Reddy}}, \bibinfo {author} {\bibfnamefont
  {W.}~\bibnamefont {Weckesser}}, \bibinfo {author} {\bibfnamefont
  {H.}~\bibnamefont {Abbasi}}, \bibinfo {author} {\bibfnamefont
  {C.}~\bibnamefont {Gohlke}},\ and\ \bibinfo {author} {\bibfnamefont {T.~E.}\
  \bibnamefont {Oliphant}},\ }\bibfield  {title} {\bibinfo {title} {Array
  programming with {NumPy}},\ }\href
  {https://doi.org/10.1038/s41586-020-2649-2} {\bibfield  {journal} {\bibinfo
  {journal} {Nature}\ }\textbf {\bibinfo {volume} {585}},\ \bibinfo {pages}
  {357} (\bibinfo {year} {2020})}\BibitemShut {NoStop}%
\bibitem [{\citenamefont {{SciPy 1.0 Contributors}}\ \emph
  {et~al.}(2020)\citenamefont {{SciPy 1.0 Contributors}}, \citenamefont
  {Virtanen}, \citenamefont {Gommers}, \citenamefont {Oliphant}, \citenamefont
  {Haberland}, \citenamefont {Reddy}, \citenamefont {Cournapeau}, \citenamefont
  {Burovski}, \citenamefont {Peterson}, \citenamefont {Weckesser},
  \citenamefont {Bright}, \citenamefont {van~der Walt}, \citenamefont {Brett},
  \citenamefont {Wilson}, \citenamefont {Millman}, \citenamefont {Mayorov},
  \citenamefont {Nelson}, \citenamefont {Jones}, \citenamefont {Kern},
  \citenamefont {Larson}, \citenamefont {Carey}, \citenamefont {Polat},
  \citenamefont {Feng}, \citenamefont {Moore}, \citenamefont {VanderPlas},
  \citenamefont {Laxalde}, \citenamefont {Perktold}, \citenamefont {Cimrman},
  \citenamefont {Henriksen}, \citenamefont {Quintero}, \citenamefont {Harris},
  \citenamefont {Archibald}, \citenamefont {Ribeiro}, \citenamefont
  {Pedregosa},\ and\ \citenamefont {van Mulbregt}}]{scipy_2020}%
  \BibitemOpen
  \bibfield  {author} {\bibinfo {author} {\bibnamefont {{SciPy 1.0
  Contributors}}}, \bibinfo {author} {\bibfnamefont {P.}~\bibnamefont
  {Virtanen}}, \bibinfo {author} {\bibfnamefont {R.}~\bibnamefont {Gommers}},
  \bibinfo {author} {\bibfnamefont {T.~E.}\ \bibnamefont {Oliphant}}, \bibinfo
  {author} {\bibfnamefont {M.}~\bibnamefont {Haberland}}, \bibinfo {author}
  {\bibfnamefont {T.}~\bibnamefont {Reddy}}, \bibinfo {author} {\bibfnamefont
  {D.}~\bibnamefont {Cournapeau}}, \bibinfo {author} {\bibfnamefont
  {E.}~\bibnamefont {Burovski}}, \bibinfo {author} {\bibfnamefont
  {P.}~\bibnamefont {Peterson}}, \bibinfo {author} {\bibfnamefont
  {W.}~\bibnamefont {Weckesser}}, \bibinfo {author} {\bibfnamefont
  {J.}~\bibnamefont {Bright}}, \bibinfo {author} {\bibfnamefont {S.~J.}\
  \bibnamefont {van~der Walt}}, \bibinfo {author} {\bibfnamefont
  {M.}~\bibnamefont {Brett}}, \bibinfo {author} {\bibfnamefont
  {J.}~\bibnamefont {Wilson}}, \bibinfo {author} {\bibfnamefont {K.~J.}\
  \bibnamefont {Millman}}, \bibinfo {author} {\bibfnamefont {N.}~\bibnamefont
  {Mayorov}}, \bibinfo {author} {\bibfnamefont {A.~R.~J.}\ \bibnamefont
  {Nelson}}, \bibinfo {author} {\bibfnamefont {E.}~\bibnamefont {Jones}},
  \bibinfo {author} {\bibfnamefont {R.}~\bibnamefont {Kern}}, \bibinfo {author}
  {\bibfnamefont {E.}~\bibnamefont {Larson}}, \bibinfo {author} {\bibfnamefont
  {C.~J.}\ \bibnamefont {Carey}}, \bibinfo {author} {\bibfnamefont
  {I.}~\bibnamefont {Polat}}, \bibinfo {author} {\bibfnamefont
  {Y.}~\bibnamefont {Feng}}, \bibinfo {author} {\bibfnamefont {E.~W.}\
  \bibnamefont {Moore}}, \bibinfo {author} {\bibfnamefont {J.}~\bibnamefont
  {VanderPlas}}, \bibinfo {author} {\bibfnamefont {D.}~\bibnamefont {Laxalde}},
  \bibinfo {author} {\bibfnamefont {J.}~\bibnamefont {Perktold}}, \bibinfo
  {author} {\bibfnamefont {R.}~\bibnamefont {Cimrman}}, \bibinfo {author}
  {\bibfnamefont {I.}~\bibnamefont {Henriksen}}, \bibinfo {author}
  {\bibfnamefont {E.~A.}\ \bibnamefont {Quintero}}, \bibinfo {author}
  {\bibfnamefont {C.~R.}\ \bibnamefont {Harris}}, \bibinfo {author}
  {\bibfnamefont {A.~M.}\ \bibnamefont {Archibald}}, \bibinfo {author}
  {\bibfnamefont {A.~H.}\ \bibnamefont {Ribeiro}}, \bibinfo {author}
  {\bibfnamefont {F.}~\bibnamefont {Pedregosa}},\ and\ \bibinfo {author}
  {\bibfnamefont {P.}~\bibnamefont {van Mulbregt}},\ }\bibfield  {title}
  {\bibinfo {title} {{SciPy} 1.0: fundamental algorithms for scientific
  computing in {Python}},\ }\href {https://doi.org/10.1038/s41592-019-0686-2}
  {\bibfield  {journal} {\bibinfo  {journal} {Nature Methods}\ }\textbf
  {\bibinfo {volume} {17}},\ \bibinfo {pages} {261} (\bibinfo {year}
  {2020})}\BibitemShut {NoStop}%
\bibitem [{\citenamefont {McKinney}(2009)}]{mckinney_data_2009}%
  \BibitemOpen
  \bibfield  {author} {\bibinfo {author} {\bibfnamefont {W.}~\bibnamefont
  {McKinney}},\ }\bibfield  {title} {\bibinfo {title} {{D}ata {s}tructures for
  {s}tatistical {c}omputing in {p}ython},\ }in\ \href
  {https://doi.org/9.25080/Majora-92bf1922-00a} {\emph {\bibinfo {booktitle}
  {{P}roceedings of the 8th {P}ython in {S}cience {C}onference}}},\ \bibinfo
  {editor} {edited by\ \bibinfo {editor} {\bibfnamefont {S.}~\bibnamefont
  {van~der {W}alt}}\ and\ \bibinfo {editor} {\bibfnamefont {J.}~\bibnamefont
  {{M}illman}}}\ (\bibinfo {year} {2009})\ pp.\ \bibinfo {pages} {55 --
  61}\BibitemShut {NoStop}%
\bibitem [{\citenamefont {Hunter}(2007)}]{hunter_matplotlib_2007}%
  \BibitemOpen
  \bibfield  {author} {\bibinfo {author} {\bibfnamefont {J.~D.}\ \bibnamefont
  {Hunter}},\ }\bibfield  {title} {\bibinfo {title} {Matplotlib: A 2d graphics
  environment},\ }\href {https://doi.org/10.1109/MCSE.2007.55} {\bibfield
  {journal} {\bibinfo  {journal} {Comput. Sci. Eng.}\ }\textbf {\bibinfo
  {volume} {9}},\ \bibinfo {pages} {90} (\bibinfo {year} {2007})}\BibitemShut
  {NoStop}%
\bibitem [{\citenamefont {Waskom}(2021)}]{Wseaborn2021}%
  \BibitemOpen
  \bibfield  {author} {\bibinfo {author} {\bibfnamefont {M.~L.}\ \bibnamefont
  {Waskom}},\ }\bibfield  {title} {\bibinfo {title} {seaborn: statistical data
  visualization},\ }\href {https://doi.org/10.21105/joss.03021} {\bibfield
  {journal} {\bibinfo  {journal} {J. Open Source Softw.}\ }\textbf {\bibinfo
  {volume} {6}},\ \bibinfo {pages} {3021} (\bibinfo {year} {2021})}\BibitemShut
  {NoStop}%
\bibitem [{\citenamefont {Binosi}\ and\ \citenamefont
  {Theußl}(2004)}]{binosi_jaxodraw_2004}%
  \BibitemOpen
  \bibfield  {author} {\bibinfo {author} {\bibfnamefont {D.}~\bibnamefont
  {Binosi}}\ and\ \bibinfo {author} {\bibfnamefont {L.}~\bibnamefont
  {Theußl}},\ }\bibfield  {title} {\bibinfo {title} {{JaxoDraw}: {A} graphical
  user interface for drawing {Feynman} diagrams},\ }\href
  {https://doi.org/https://doi.org/10.1016/j.cpc.2004.05.001} {\bibfield
  {journal} {\bibinfo  {journal} {Comput. Phys. Commun.}\ }\textbf {\bibinfo
  {volume} {161}},\ \bibinfo {pages} {76} (\bibinfo {year} {2004})}\BibitemShut
  {NoStop}%
\bibitem [{\citenamefont {Bezanson}\ \emph {et~al.}(2017)\citenamefont
  {Bezanson}, \citenamefont {Edelman}, \citenamefont {Karpinski},\ and\
  \citenamefont {Shah}}]{bezanson_julia_2017}%
  \BibitemOpen
  \bibfield  {author} {\bibinfo {author} {\bibfnamefont {J.}~\bibnamefont
  {Bezanson}}, \bibinfo {author} {\bibfnamefont {A.}~\bibnamefont {Edelman}},
  \bibinfo {author} {\bibfnamefont {S.}~\bibnamefont {Karpinski}},\ and\
  \bibinfo {author} {\bibfnamefont {V.~B.}\ \bibnamefont {Shah}},\ }\bibfield
  {title} {\bibinfo {title} {Julia: {A} {Fresh} {Approach} to {Numerical}
  {Computing}},\ }\href {https://doi.org/10.1137/141000671} {\bibfield
  {journal} {\bibinfo  {journal} {SIAM Rev.}\ }\textbf {\bibinfo {volume}
  {59}},\ \bibinfo {pages} {65} (\bibinfo {year} {2017})}\BibitemShut {NoStop}%
\bibitem [{\citenamefont {Aroeira}\ \emph {et~al.}(2022)\citenamefont
  {Aroeira}, \citenamefont {Davis}, \citenamefont {Turney},\ and\ \citenamefont
  {{Schaefer III}}}]{aroeira_fermijl_2022}%
  \BibitemOpen
  \bibfield  {author} {\bibinfo {author} {\bibfnamefont {G.~J.~R.}\
  \bibnamefont {Aroeira}}, \bibinfo {author} {\bibfnamefont {M.~M.}\
  \bibnamefont {Davis}}, \bibinfo {author} {\bibfnamefont {J.~M.}\ \bibnamefont
  {Turney}},\ and\ \bibinfo {author} {\bibfnamefont {H.~F.}\ \bibnamefont
  {{Schaefer III}}},\ }\bibfield  {title} {\bibinfo {title} {Fermi.jl: {A}
  {Modern} {Design} for {Quantum} {Chemistry}},\ }\href
  {https://doi.org/10.1021/acs.jctc.1c00719} {\bibfield  {journal} {\bibinfo
  {journal} {J. Chem. Theor. Comput.}\ }\textbf {\bibinfo {volume} {18}},\
  \bibinfo {pages} {677} (\bibinfo {year} {2022})}\BibitemShut {NoStop}%
\bibitem [{noa({\natexlab{a}})}]{noauthor_tensoroperationsjl_nodate}%
  \BibitemOpen
  \href@noop {} {\bibinfo {title} {{TensorOperations}.jl}} ({\natexlab{a}}),\
  \bibinfo {note} {https://github.com/Jutho/TensorOperations.jl, accessed
  2023-02-13}\BibitemShut {NoStop}%
\bibitem [{noa({\natexlab{b}})}]{noauthor_tulliojl_nodate}%
  \BibitemOpen
  \href@noop {} {\bibinfo {title} {{Tullio}.jl}} ({\natexlab{b}}),\ \bibinfo
  {note} {https://github.com/mcabbott/Tullio.jl, accessed
  2023-02-13}\BibitemShut {NoStop}%
\end{thebibliography}%


\end{document}
