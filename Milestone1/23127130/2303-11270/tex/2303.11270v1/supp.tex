% revtex 4.2 APS template to start with
\documentclass[reprint,onecolumn,aps,superscriptaddress,footinbib,prx]{revtex4-2}
\usepackage{amsmath}
\usepackage{physics}
\usepackage{graphicx}
\usepackage{amssymb}
\usepackage{amsthm}
\usepackage{bm}
\usepackage{txfonts}
\usepackage{pstricks}
\usepackage[colorlinks=true,allcolors=blue]{hyperref} 
\usepackage[capitalise]{cleveref}
\allowdisplaybreaks

\newcommand{\vk}{{\bm{k}}}
\newcommand{\vq}{{\bm{q}}}
\newcommand{\vp}{{\bm{p}}}
\newcommand{\vm}{{\bm{m}}}
\newcommand{\kF}{{k_\mathrm{F}}}
\definecolor{mygreen}{rgb}{0.15, 0.6, 0.15}
\definecolor{mygrey}{rgb}{0.5, 0.5, 0.5}
\newcommand*\grey[1]{\textcolor{mygrey}{#1}}
\newcommand*\tcb[1]{\textcolor{red}{[TCB: #1]}}
\newcommand*\van[1]{\textcolor{mygreen}{[Verena: #1]}}
\newcommand*\gvv[1]{\textcolor{mygreen}{#1}}

\begin{document}
\title{Supplemental Material for: Highly accurate electronic structure of metallic solids from coupled-cluster theory with nonperturbative triple excitations}
\author{Verena A. Neufeld}
\author{Timothy C. Berkelbach}
\affiliation{Department of Chemistry, Columbia University, New York, New York 10027, USA}
\date{\today}  % https://tex.stackexchange.com/questions/165991/how-to-use-date-today-without-using-maketitle

\maketitle

\renewcommand\thesection{S\arabic{section}}
\renewcommand\theequation{S\arabic{equation}}
\renewcommand\thefigure{S\arabic{figure}}
\renewcommand\thetable{S\arabic{table}}

\section{Uniform electron gas calculations}
\label{sec:ueg}

In a plane-wave basis, the Hamiltonian of the uniform electron gas (UEG) in a cubic box of
volume $V=L^3$ is
\begin{align}
    H &= 
    \sum_{\vk\sigma} \epsilon_k a^\dagger_{\vk,\sigma} a_{\vk,\sigma}
    + \frac{1}{2}\sum_{\vk_1\sigma_1,\vk_2\sigma_2}\sum_{\vq\neq 0} v(q) 
        a^\dagger_{\vk_1+\vq,\sigma_1} a^\dagger_{\vk_2-\vq,\sigma_2} a_{\vk_2,\sigma_2} a_{\vk_1,\sigma_1} \\
    v(q) &= \begin{cases}
        4\pi e^2/(Vq^2) & q \neq 0 \\
        v_\mathrm{M} & q = 0
    \end{cases}
\end{align}
where $v_\mathrm{M} \propto L^{-1}$ is the Madelung constant of the cell, 
$\epsilon_k = \hbar^2k^2/(2m)$, $\vk = (2\pi/L)[n_x, n_y, n_z]+\vk_\mathrm{s}$ ($n_i$ are integers),
and $\vk_\mathrm{s}$ is a shift of the $k$-point mesh consistent with twisted boundary conditions.
Here we choose $\vk_\mathrm{s}$ to be the Baldereschi point~\cite{baldereschi_mean-value_1973} of the cubic Brillouin zone,
i.e., $\vk_\mathrm{s} = (2\pi/L)[1/4, 1/4, 1/4]$,
which we find to smooth the convergence to the
thermodynamic limit (TDL).  We study only closed-shell systems with ``magic
numbers'' of electrons $N$ increasing by roughly a factor of 2, i.e., $N =
14,34,70,\ldots,1404$. CCSD-based and CCSDT-based calculations are performed
using up to 1404 and 156 electrons, respectively. 
The box volume is based on the target density $n$, i.e., $V = N/n$.

We note that the orbitals to be used in constructing \textit{any}
reference determinant without translational symmetry breaking, such as spin- or
charge-density wave ordering, are just the plane-wave orbitals with the lowest
kinetic energy. In ``full'' CC theories, such as CCSD or CCSDT, the final
results depend only on this determinant of single-particle orbitals and
\textit{not} on their orbital energies. This point is important because the HF
orbital energies are known to be problematic for metals. However, removing some
terms from CC theory, as done in the (direct) RPA and ring-CCSDT, reintroduces
a dependence on the orbital energies.  The traditional RPA theory uses the kinetic
energies only; for consistency with standard CC
implementations, we use HF orbital energies in our ring-CCSDT calculations.

As a single-particle basis, we use $M$ plane-wave spin-orbitals, where the largest value of 
$M$ depends on $N$. Ultimately, we seek the combined complete basis set (CBS) limit
and thermodynamic limit (TDL), which we describe in Sec.~\ref{ssec:cbs} and \ref{ssec:tdl}.
But first, we demonstrate the convergence and divergence behaviors with $N$ of the CC
theories discussed in the main text.

\subsection{Demonstration of convergence or divergence behaviors}

\begin{figure}[t!]
	\includegraphics[width=3.5in]{compare_with_mp2_diagonal.pdf}
	\caption{Thermodynamic limit convergence at $r_s = 4$ of the difference of the CCSD(T), CCSDT, CCSDT-1,
		and ring-CCSDT correlation energies to the CCSD correlation energy, plotted against the 
                MP2 correlation energy.
	        The ratio of the number of orbitals to number of
                electrons is $M/N\approx 1.5$, which is far from the complete basis set limit.
                Results are shown for systems ranging from $(N=34, M=52)$ to $(N=180, M=272)$ for
                CCSDT-1, ring-CCSDT, CCSDT, and up to $(N=628, M=960)$ for CCSD(T).}
	\label{fig:tdl_trip}
\end{figure}



In Fig.~\ref{fig:tdl_trip}, we follow Refs.~\onlinecite{shepherd_many-body_2013,shepherd_range-separated_2014,shepherd_coupled_2014} 
and plot the correlation energy of CCSD(T), CCSDT-1, ring-CCSDT, and CCSDT against
that of MP2 theory, with $M/N \approx 1.5$ (i.e.,
the results are not in the CBS, but are sufficient to demonstrate the convergence or divergence with $N$).
More specifically, we plot the difference with respect to the CCSD correlation energy, which is known
to converge with $N$.
For CCSD(T), this difference is exactly the (T) correction,
which we find to increase in linear proportion to the MP2 correlation energy, which is fully
consistent with our analytical conclusion that both diverge logarithmically.
We see that CCSDT-1 appears to diverge at least as fast as CCSD(T) and shows no indication
of convergence at the accessible values of $N$.
In contrast, ring-CCSDT and CCSDT show the onset of a plateau, providing strong
numerical support for their convergence in the TDL, which is consistent with our expectations. 

\subsection{Complete basis set limit estimation}
\label{ssec:cbs}

For each $N$, we estimate the $M\rightarrow \infty$ complete basis set (CBS) limit incrementally, 
using a composite correction based on the smaller value of $N$ at the same level of theory,
\begin{equation}
\label{eq:cbs}
E_{N}(\mathrm{CBS}) \approx E_{N}(M) + E_{N/2}(\mathrm{CBS}) - E_{N/2}(M/2),
\end{equation}
where $M$ is the largest value accessed for that value of $N$, ``$N/2$'' is
short-hand for the number of electrons that is \textit{roughly} a factor of 2
smaller than $N$, and $E_{N/2}(M/2)$ is found via interpolation of results with
varying $M$, since we are limited to ``magic numbers'' of basis functions.
Table~\ref{tab:maxsystemsforueg} shows, for each CC method,
the maximum number of spin-orbitals $M$ that we use for each number of electrons $N$.

\begin{table}[b]
	\centering
	\begin{tabular*}{0.7\textwidth}{@{\extracolsep{\fill}}lccccccc}
		\hline\hline
		& $N=14$ & $N=34$  & $N=70$ & $N=156$ & $N=332$ & $N=700$& $N=1404$ \\
		\hline
		CCSD & 47118& 23966  & 10332 & 5252 & 2488 & 2392 & 2392 \\
		DCSD & 52746 & 27154  & 19962 & 8646 & 4850 & 3996& 4140 \\
		CCSDT &1476--1538& 832  & 502 & 344 & - & -& - \\
		ring-CCSDT & 5720 & 1676  & 1104 & 302--410 & - & -& - \\
		DCSDT & 1476 & 796--832 & 464--502 & 272--344 & - & -& - \\
		\hline\hline
	\end{tabular*}
        \caption{Maximum number of spin-orbitals $M$ for given number of
        electrons $N$ for the CC methods.  The maximum number of spin-orbitals $M$ sometimes 
        depends on $r_s$, in which case a range is given.}
	\label{tab:maxsystemsforueg}
\end{table}


For CCSD-based methods, we use Eq.~(\ref{eq:cbs}) to
estimate the correlation energy $E_\mathrm{c}$. For the more expensive
CCSDT-based methods, we use Eq.~(\ref{eq:cbs}) to estimate
the \textit{difference} to the DCSD correlation energy, $\Delta
E_\mathrm{c}(X\text{SDT}-\text{DCSD}) \equiv E_\mathrm{c}(X\text{SDT}) -
E_\mathrm{c}(\text{DCSD})$. 
In Fig.~\ref{fig:ueg_cbs_tdl}(a), we show the
convergence of the incremental basis set corrections at $r_s=4$, which can be
seen to decrease in magnitude with increasing $N$. 
The final CBS energies and energy differences are then
used to estimate the TDL, as described next.

\subsection{Thermodynamic limit extrapolation}
\label{ssec:tdl}

Given CBS estimates at each value of $N$, we estimate the TDL by extrapolation.
As justified in Sec.~\ref{sec:finite}, we extrapolate the CCSD-based correlation energies 
by fitting to the functional form
\begin{equation}
\label{eq:extrap}
E_\mathrm{c}(N) = E_\mathrm{c}(N\rightarrow \infty) + aN^{-2/3} + bN^{-1},
\end{equation}
where $E_\mathrm{c}(N)$ is the correlation energy per particle. In practice,
we fit to the six data points with the largest $N$.
For CCSDT-based methods, we extrapolate the CBS estimate of the \textit{difference} to DCSD,
as described in the previous section.
Because the $N^{-2/3}$ term in Eq.~(\ref{eq:extrap}) is attributed to the HF exchange energy,
which should cancel in the difference of two correlated energies, we extrapolate these
differences using only $N^{-1}$,
i.e., with $a=0$; specifically, a two-point extrapolation is performed with $N=34$ and $N=70$.
We then add this difference to the combined CBS+TDL value of the DCSD correlation energy
to obtain our final estimate of the CBS+TDL value of the CCSDT-based correlation energies.
Figure~\ref{fig:ueg_cbs_tdl}(b) shows the TDL convergence at $r_s = 4$ for the correlation
energy of CCSD and DCSD and for the correlation energy difference of various CCSDT-based methods.

\begin{figure}[h!]
	\includegraphics[width=6.5in]{si_ueg.pdf}
	\caption{Convergence to (a) complete basis set (CBS) limit and (b) thermodynamic limit (TDL)
	for the uniform electron gas at $r_s = 4$. CCSD and DCSD correlation energies are shown, 
        as well as CCSDT, ring-CCSDT, and DCSDT energy differences to that of DCSD. 
        Estimated values with error bars are shown with shaded regions.}
	\label{fig:ueg_cbs_tdl}
\end{figure}

\subsection{Final Results}
Combining the extrapolations described above yields our final correlation energy estimates in
the CBS and TDL limits, which are given in Tab.~\ref{tab:uegabsresults} and were presented
graphically in Fig.~1 of the main text.
\begin{table}[h]
\centering
\begin{tabular*}{0.48\textwidth}{@{\extracolsep{\fill}}lccccc}
\hline\hline
& \multicolumn{5}{c}{$E_\mathrm{c}/N$ (m$E_h$)} \\
& $r_s = 1$ & $r_s = 2$ & $r_s = 3$ & $r_s = 4$ & $r_s=5$  \\
\hline
CCSD & $-$56  & $-$39 & $-$31 & $-$25 & $-$22 \\
CCSDT & $-$59 & $-$44 & $-$35 & $-$30 & $-$26 \\
ring-CCSDT & $-$59 & $-$43  & $-$35  & $-$29 & $-$25  \\
SCS-CCSD & $-$68 & $-$48 & $-$38 & $-$31 & $-$27 \\
DCSD &$-$58 & $-$42 & $-$33 & $-$28& $-$24 \\
SCS-DCSD &$-$63 & $-$45& $-$36& $-$30 & $-$26 \\
DCSDT & $-$59 & $-$44 & $-$36& $-$31 & $-$27 \\
DMC~\cite{perdew_self-interaction_1981,ceperley_ground_1980} & $-$60& $-$45& $-$37& $-$32& $-$28 \\
\hline\hline
\end{tabular*}
\caption{Uniform electron gas correlation energy per electron for $r_s = 1-5$
from various coupled cluster methods
extrapolated to the complete basis set and thermodynamic limit.
(Fitted) diffusion Monte Carlo (DMC)
results~\cite{perdew_self-interaction_1981,ceperley_ground_1980} 
are shown for comparison.}
\label{tab:uegabsresults}
\end{table}



\section{Many-body finite-size errors in metals}
\label{sec:finite}

Here, we provide an analysis of asymptotic finite-size errors in metals, based
on the uniform electron gas (UEG).
The potential energy (per electron) of the UEG can be expressed via the Coulomb interaction
$v(q)=4\pi e^2/q^2$ and the structure factor $S(q)$~\cite{giuliani_quantum_2005},
\begin{equation}
U = 
\frac{1}{2} \int \frac{d^3q}{(2\pi)^3} v(q) \left[ S(q)-1 \right]
    = \frac{e^2}{\pi} \int_0^\infty dq
    \left[ S(q)-1 \right].
\end{equation}
The exact long wavelength behavior of the structure factor is
$S(q) = \hbar q^2 / 2m\omega_p$, where $\omega_p = \sqrt{4\pi ne^2/m}$ is the
plasmon frequency; we expect all correlated methods that include the physics of
the RPA capture this long wavelength behavior.
In numerical calculations of finite systems with
periodic boundary conditions, momentum transfers near $q=0$ are neglected, and
so we can estimate the finite-size error as 
\begin{equation}
\Delta U = \frac{e^2}{\pi} \int_0^{q_c} dq S(q)
\approx \frac{\hbar e^2}{6m\pi\omega_p} q_c^3
\end{equation}
where
\begin{equation}
q_c = \left(\frac{3}{4\pi}\right)^{1/3} \frac{2\pi}{L}
= \left(6\pi^2n\right)^{1/3} N^{-1/3} = 2^{1/3} \kF N^{-1/3}
\end{equation}
is the radius of a sphere with volume $(2\pi)^3/V$.
Therefore, the finite-size error of the potential energy is $O(N^{-1})$.
This analysis assumes that the structure factor itself has no finite-size error.

However, the \textit{correlation} energy is the difference between
the interacting and noninteracting (Hartree-Fock) energies,
\begin{equation}
E_\mathrm{c} = \frac{e^2}{\pi} \int_0^{\infty} dq
\left[S(q)-S^{(0)}(q)\right].
\end{equation}
In contrast to the behavior of interacting theories, the long wavelength
behavior of the noninteracting
structure factor is $S^{(0)}(q) = 3q/4k_\mathrm{F}$, such that the
finite-size error of the mean-field potential energy is $O(q_c^2) \sim O(N^{-2/3})$.
Therefore, in principle, the correlation energy inherits this finite-size
error and it vanishes as $N^{-2/3}$ in the large $N$ limit.
Specifically, using these leading-order terms in both the exact interacting
and noninteracting structure factors, we can estimate the finite-size error as
\begin{equation}
\label{eq:leading}
\Delta E_\mathrm{c}
= \frac{\hbar e^2}{6m\pi\omega_p} q_c^3 - \frac{3e^2}{8\pi k_\mathrm{F}}q_c^2,
\end{equation}
which is functionally equivalent to Eq.~(\ref{eq:extrap}).

As a tractable demonstration of this behavior, we consider
the textbook RPA correlation energy,
\begin{equation}
\label{eq:rpa_qc}
E_\mathrm{c}^{(\mathrm{RPA})}
= \lim_{q_\mathrm{c}\rightarrow 0} E_\mathrm{c}^{(\mathrm{RPA})}(q_\mathrm{c}),
\ \ \ \ \ E_\mathrm{c}^{(\mathrm{RPA})}(q_\mathrm{c}) 
= \frac{\hbar}{2\pi n} \int_{q_\mathrm{c}}^{\infty} \frac{4\pi q^2dq}{(2\pi)^3}
\int_0^\infty d\omega
\Big\{
    \ln\left[1-v(q)\chi^{(0)}(q,i\omega)\right] 
     + 
    v(q)\chi^{(0)}(q,i\omega)
\Big\},
\end{equation}
which can be evaluated by simple two-dimensional quadrature
and $\chi^{(0)}(q,i\omega)$ is the Lindhard function.
Again, we emphasize that the first term (containing the logarithm)
is the RPA potential energy, and second first term only serves to remove
the Hartree-Fock exchange energy, as needed for the definition of the
correlation energy.
At small $q_\mathrm{c}$, the second (exchange) term dominates and upon integration
yields $E_\mathrm{c}^{(\mathrm{RPA})}(q_\mathrm{c}) \sim q_\mathrm{c}^2 \sim N^{-2/3}$.

This behavior is confirmed numerically, and the result is shown in Fig.~\ref{fig:tdl_drpa} (left)
for the UEG at $r_s=4$. We see that the analytic expression (evaluated by numerical integration)
converges to the TDL with the expected form. However, the asymptotic behavior (linear
in $N^{-2/3}$) is only observed for large system sizes with $N \gtrsim 100$. We show that the
same behavior persists for plane-wave based calculations performed in the manner of Sec.~\ref{sec:ueg}.
Fitting the correlation energy to Eq.~(\ref{eq:extrap}) using four data points (with $b=0$)
or with six data points (with $b\neq 0$) gives predictions that are in excellent agreement
with exact RPA result. We also show the CCSD TDL convergence in Fig.~\ref{fig:tdl_drpa} (right)
and similarly see that the fits with $b=0$ and $b\neq 0$ agree well with each other.
In principle, Eqs.~(\ref{eq:leading}) or (\ref{eq:rpa_qc}) could also be used effectively
as finite-size \textit{corrections}, although we do not do so here.
\begin{figure}[h]
	\includegraphics[width=7in]{tdl_drpa.pdf}
	\caption{Thermodynamic limit convergence of the RPA (left) and CCSD (right)
		correlation energy at $r_s = 4$. 
	We show the numerical correlation energy from plane-wave based calculations (orange stars and blue circles).
        Two fits are tested on the results. For RPA, we also show the analytic correlation energy from Eq.~(\ref{eq:rpa_qc})
        (dark grey solid curve) and a horizontal line at the exact
        RPA correlation energy in the TDL.
        }
	\label{fig:tdl_drpa}
\end{figure}

\section{Atomistic solid lithium calculations}

\subsection{Calculation details}
Except where otherwise stated, all calculation details, including basis sets and pseudopotentials,
are the same as in Ref.~\cite{neufeld_ground-state_2022}.
The CCSD data is taken from Ref.~\cite{neufeld_ground-state_2022}, but using $N^{-2/3}$ extrapolation
to the TDL, as discussed in Sec.~\ref{sec:finite}. As seen in Tab.~\ref{tab:libulk}, this difference
causes only small changes to our predicted properties.
Calculation details for bulk DCSD are similar to those from CCSD~\cite{neufeld_ground-state_2022},
except again for the TDL extrapolation. The single atom DCSD correlation energies were found using
two shells of ghost atoms in a molecular calculation.

As in our UEG study, the CC energies with triple excitations are estimated via their differences to DCSD energies
calculated with the same system size.
Specifically, calculations with triple excitations were performed on an 8-atom supercell
with $\Gamma$-point sampling (equivalent to sampling with four $k$ points on a two atom cubic unit cell)
and 1s core electrons frozen.
An additional calculation was performed
on a 16-atom cubic supercell at lattice parameter~3.5~\AA\ using frozen virtual natural orbitals. The difference
with respect to a calculation on the 8-atom supercell with the same frozen virtual treatment was used
as an additional finite-size correction.
At each basis set, the difference between the triple calculation and
DCSD at this system size was added to the DCSD energy at that basis set in the TDL.
Correlation energies were extrapolated to the CBS limit using a $X^{-3}$ 
form ($X=3,4$ for TZ,~QZ)~\cite{helgaker_basis-set_1997}.
After the TDL CBS results for each coupled cluster variant were estimated,
as in Ref.~\cite{neufeld_ground-state_2022}, the Birch-Murnaghan~\cite{birch_finite_1947,zhang_performance_2018}
equation-of-state was fit to extract the lattice parameter, bulk modulus, and cohesive energy.


\subsection{Final results}

Table~\ref{tab:libulk} shows our calculated structural and energetic properties of lithium, i.e.,
lattice parameter $a$, bulk modulus $B$, and cohesive energy $E_{\mathrm{coh}}$.
\begin{table}[h]
\centering
\begin{tabular*}{0.48\textwidth}{@{\extracolsep{\fill}}lccc}
\hline\hline
& $a$ (\AA) & $B$ (GPa) & $E_{\mathrm{coh}}$ (m$E_h$)  \\
\hline 
CCSD ($N^{-1}$)~\cite{neufeld_ground-state_2022}& 3.49 &12.8 & $-$51 \\
CCSD  & 3.48& 13.0&$-$52\\
DCSD & 3.47& 13.2&$-$55 \\ 
ring-CCSDT  & 3.46& 13.1 & $-$57\\
CCSDT  & 3.47& 13.3&$-$57 \\
DCSDT & 3.47 &13.2 & $-$57 \\
Experiment~\cite{zhang_performance_2018,berliner_effect_1986,felice_temperature_1977,kittel_intro_solid_2005}& 3.45& 13.3&$-$61 \\
\hline\hline
\end{tabular*}
\caption{Equilibrium lattice parameter $a$, bulk modulus $B$ and cohesive energy $E_{\mathrm{coh}}$ for bulk BCC Li. Shown
	are CCSD with a $N^{-1}$ TDL extrapolation from a previous study~\cite{neufeld_ground-state_2022}, CCSD (with a $N^{-2/3}$ extrapolation), DCSD,
	ring-CCSDT, CCSDT, and DCSDT as well as zero-point motion corrected experimental 
	results~\cite{zhang_performance_2018,berliner_effect_1986,felice_temperature_1977,kittel_intro_solid_2005}.}
\label{tab:libulk}
\end{table}


%apsrev4-2.bst 2019-01-14 (MD) hand-edited version of apsrev4-1.bst
%Control: key (0)
%Control: author (8) initials jnrlst
%Control: editor formatted (1) identically to author
%Control: production of article title (0) allowed
%Control: page (0) single
%Control: year (1) truncated
%Control: production of eprint (0) enabled
\begin{thebibliography}{14}%
\makeatletter
\providecommand \@ifxundefined [1]{%
 \@ifx{#1\undefined}
}%
\providecommand \@ifnum [1]{%
 \ifnum #1\expandafter \@firstoftwo
 \else \expandafter \@secondoftwo
 \fi
}%
\providecommand \@ifx [1]{%
 \ifx #1\expandafter \@firstoftwo
 \else \expandafter \@secondoftwo
 \fi
}%
\providecommand \natexlab [1]{#1}%
\providecommand \enquote  [1]{``#1''}%
\providecommand \bibnamefont  [1]{#1}%
\providecommand \bibfnamefont [1]{#1}%
\providecommand \citenamefont [1]{#1}%
\providecommand \href@noop [0]{\@secondoftwo}%
\providecommand \href [0]{\begingroup \@sanitize@url \@href}%
\providecommand \@href[1]{\@@startlink{#1}\@@href}%
\providecommand \@@href[1]{\endgroup#1\@@endlink}%
\providecommand \@sanitize@url [0]{\catcode `\\12\catcode `\$12\catcode
  `\&12\catcode `\#12\catcode `\^12\catcode `\_12\catcode `\%12\relax}%
\providecommand \@@startlink[1]{}%
\providecommand \@@endlink[0]{}%
\providecommand \url  [0]{\begingroup\@sanitize@url \@url }%
\providecommand \@url [1]{\endgroup\@href {#1}{\urlprefix }}%
\providecommand \urlprefix  [0]{URL }%
\providecommand \Eprint [0]{\href }%
\providecommand \doibase [0]{https://doi.org/}%
\providecommand \selectlanguage [0]{\@gobble}%
\providecommand \bibinfo  [0]{\@secondoftwo}%
\providecommand \bibfield  [0]{\@secondoftwo}%
\providecommand \translation [1]{[#1]}%
\providecommand \BibitemOpen [0]{}%
\providecommand \bibitemStop [0]{}%
\providecommand \bibitemNoStop [0]{.\EOS\space}%
\providecommand \EOS [0]{\spacefactor3000\relax}%
\providecommand \BibitemShut  [1]{\csname bibitem#1\endcsname}%
\let\auto@bib@innerbib\@empty
%</preamble>
\bibitem [{\citenamefont {Baldereschi}(1973)}]{baldereschi_mean-value_1973}%
  \BibitemOpen
  \bibfield  {author} {\bibinfo {author} {\bibfnamefont {A.}~\bibnamefont
  {Baldereschi}},\ }\bibfield  {title} {\bibinfo {title} {Mean-{Value} {Point}
  in the {Brillouin} {Zone}},\ }\href {https://doi.org/10.1103/PhysRevB.7.5212}
  {\bibfield  {journal} {\bibinfo  {journal} {Phys. Rev. B}\ }\textbf {\bibinfo
  {volume} {7}},\ \bibinfo {pages} {5212} (\bibinfo {year} {1973})}\BibitemShut
  {NoStop}%
\bibitem [{\citenamefont {Shepherd}\ and\ \citenamefont
  {Grüneis}(2013)}]{shepherd_many-body_2013}%
  \BibitemOpen
  \bibfield  {author} {\bibinfo {author} {\bibfnamefont {J.~J.}\ \bibnamefont
  {Shepherd}}\ and\ \bibinfo {author} {\bibfnamefont {A.}~\bibnamefont
  {Grüneis}},\ }\bibfield  {title} {\bibinfo {title} {Many-{Body} {Quantum}
  {Chemistry} for the {Electron} {Gas}: {Convergent} {Perturbative}
  {Theories}},\ }\href {https://doi.org/10.1103/PhysRevLett.110.226401}
  {\bibfield  {journal} {\bibinfo  {journal} {Phys. Rev. Lett.}\ }\textbf
  {\bibinfo {volume} {110}},\ \bibinfo {pages} {226401} (\bibinfo {year}
  {2013})}\BibitemShut {NoStop}%
\bibitem [{\citenamefont {Shepherd}\ \emph
  {et~al.}(2014{\natexlab{a}})\citenamefont {Shepherd}, \citenamefont
  {Henderson},\ and\ \citenamefont {Scuseria}}]{shepherd_range-separated_2014}%
  \BibitemOpen
  \bibfield  {author} {\bibinfo {author} {\bibfnamefont {J.~J.}\ \bibnamefont
  {Shepherd}}, \bibinfo {author} {\bibfnamefont {T.~M.}\ \bibnamefont
  {Henderson}},\ and\ \bibinfo {author} {\bibfnamefont {G.~E.}\ \bibnamefont
  {Scuseria}},\ }\bibfield  {title} {\bibinfo {title} {Range-{Separated}
  {Brueckner} {Coupled} {Cluster} {Doubles} {Theory}},\ }\href
  {https://doi.org/10.1103/PhysRevLett.112.133002} {\bibfield  {journal}
  {\bibinfo  {journal} {Phys. Rev. Lett.}\ }\textbf {\bibinfo {volume} {112}},\
  \bibinfo {pages} {133002} (\bibinfo {year} {2014}{\natexlab{a}})}\BibitemShut
  {NoStop}%
\bibitem [{\citenamefont {Shepherd}\ \emph
  {et~al.}(2014{\natexlab{b}})\citenamefont {Shepherd}, \citenamefont
  {Henderson},\ and\ \citenamefont {Scuseria}}]{shepherd_coupled_2014}%
  \BibitemOpen
  \bibfield  {author} {\bibinfo {author} {\bibfnamefont {J.~J.}\ \bibnamefont
  {Shepherd}}, \bibinfo {author} {\bibfnamefont {T.~M.}\ \bibnamefont
  {Henderson}},\ and\ \bibinfo {author} {\bibfnamefont {G.~E.}\ \bibnamefont
  {Scuseria}},\ }\bibfield  {title} {\bibinfo {title} {Coupled cluster channels
  in the homogeneous electron gas},\ }\href {https://doi.org/10.1063/1.4867783}
  {\bibfield  {journal} {\bibinfo  {journal} {J. Chem. Phys.}\ }\textbf
  {\bibinfo {volume} {140}},\ \bibinfo {pages} {124102} (\bibinfo {year}
  {2014}{\natexlab{b}})}\BibitemShut {NoStop}%
\bibitem [{\citenamefont {Perdew}\ and\ \citenamefont
  {Zunger}(1981)}]{perdew_self-interaction_1981}%
  \BibitemOpen
  \bibfield  {author} {\bibinfo {author} {\bibfnamefont {J.~P.}\ \bibnamefont
  {Perdew}}\ and\ \bibinfo {author} {\bibfnamefont {A.}~\bibnamefont
  {Zunger}},\ }\bibfield  {title} {\bibinfo {title} {Self-interaction
  correction to density-functional approximations for many-electron systems},\
  }\href {https://doi.org/10.1103/PhysRevB.23.5048} {\bibfield  {journal}
  {\bibinfo  {journal} {Phys. Rev. B}\ }\textbf {\bibinfo {volume} {23}},\
  \bibinfo {pages} {5048} (\bibinfo {year} {1981})}\BibitemShut {NoStop}%
\bibitem [{\citenamefont {Ceperley}\ and\ \citenamefont
  {Alder}(1980)}]{ceperley_ground_1980}%
  \BibitemOpen
  \bibfield  {author} {\bibinfo {author} {\bibfnamefont {D.~M.}\ \bibnamefont
  {Ceperley}}\ and\ \bibinfo {author} {\bibfnamefont {B.~J.}\ \bibnamefont
  {Alder}},\ }\bibfield  {title} {\bibinfo {title} {Ground {State} of the
  {Electron} {Gas} by a {Stochastic} {Method}},\ }\href
  {https://doi.org/10.1103/PhysRevLett.45.566} {\bibfield  {journal} {\bibinfo
  {journal} {Phys. Rev. Lett.}\ }\textbf {\bibinfo {volume} {45}},\ \bibinfo
  {pages} {566} (\bibinfo {year} {1980})}\BibitemShut {NoStop}%
\bibitem [{\citenamefont {Giuliani}\ and\ \citenamefont
  {Vignale}(2005)}]{giuliani_quantum_2005}%
  \BibitemOpen
  \bibfield  {author} {\bibinfo {author} {\bibfnamefont {G.~F.}\ \bibnamefont
  {Giuliani}}\ and\ \bibinfo {author} {\bibfnamefont {G.}~\bibnamefont
  {Vignale}},\ }\href@noop {} {\emph {\bibinfo {title} {Quantum {Theory} of the
  {Electron} {Liquid}}}},\ \bibinfo {edition} {1st}\ ed.\ (\bibinfo
  {publisher} {Cambridge University Press},\ \bibinfo {address} {Cambridge/New
  York},\ \bibinfo {year} {2005})\BibitemShut {NoStop}%
\bibitem [{\citenamefont {Neufeld}\ \emph {et~al.}(2022)\citenamefont
  {Neufeld}, \citenamefont {Ye},\ and\ \citenamefont
  {Berkelbach}}]{neufeld_ground-state_2022}%
  \BibitemOpen
  \bibfield  {author} {\bibinfo {author} {\bibfnamefont {V.~A.}\ \bibnamefont
  {Neufeld}}, \bibinfo {author} {\bibfnamefont {H.-Z.}\ \bibnamefont {Ye}},\
  and\ \bibinfo {author} {\bibfnamefont {T.~C.}\ \bibnamefont {Berkelbach}},\
  }\bibfield  {title} {\bibinfo {title} {Ground-{State} {Properties} of
  {Metallic} {Solids} from {Ab} {Initio} {Coupled}-{Cluster} {Theory}},\ }\href
  {https://doi.org/10.1021/acs.jpclett.2c01828} {\bibfield  {journal} {\bibinfo
   {journal} {J. Phys. Chem. Lett.}\ }\textbf {\bibinfo {volume} {13}},\
  \bibinfo {pages} {7497} (\bibinfo {year} {2022})}\BibitemShut {NoStop}%
\bibitem [{\citenamefont {Helgaker}\ \emph {et~al.}(1997)\citenamefont
  {Helgaker}, \citenamefont {Klopper}, \citenamefont {Koch},\ and\
  \citenamefont {Noga}}]{helgaker_basis-set_1997}%
  \BibitemOpen
  \bibfield  {author} {\bibinfo {author} {\bibfnamefont {T.}~\bibnamefont
  {Helgaker}}, \bibinfo {author} {\bibfnamefont {W.}~\bibnamefont {Klopper}},
  \bibinfo {author} {\bibfnamefont {H.}~\bibnamefont {Koch}},\ and\ \bibinfo
  {author} {\bibfnamefont {J.}~\bibnamefont {Noga}},\ }\bibfield  {title}
  {\bibinfo {title} {Basis-set convergence of correlated calculations on
  water},\ }\href {https://doi.org/10.1063/1.473863} {\bibfield  {journal}
  {\bibinfo  {journal} {J. Chem. Phys.}\ }\textbf {\bibinfo {volume} {106}},\
  \bibinfo {pages} {9639} (\bibinfo {year} {1997})}\BibitemShut {NoStop}%
\bibitem [{\citenamefont {Birch}(1947)}]{birch_finite_1947}%
  \BibitemOpen
  \bibfield  {author} {\bibinfo {author} {\bibfnamefont {F.}~\bibnamefont
  {Birch}},\ }\bibfield  {title} {\bibinfo {title} {Finite {elastic} {strain}
  of {cubic} {crystals}},\ }\href {https://doi.org/10.1103/PhysRev.71.809}
  {\bibfield  {journal} {\bibinfo  {journal} {Phys. Rev.}\ }\textbf {\bibinfo
  {volume} {71}},\ \bibinfo {pages} {809} (\bibinfo {year} {1947})}\BibitemShut
  {NoStop}%
\bibitem [{\citenamefont {Zhang}\ \emph {et~al.}(2018)\citenamefont {Zhang},
  \citenamefont {Reilly}, \citenamefont {Tkatchenko},\ and\ \citenamefont
  {Scheffler}}]{zhang_performance_2018}%
  \BibitemOpen
  \bibfield  {author} {\bibinfo {author} {\bibfnamefont {G.-X.}\ \bibnamefont
  {Zhang}}, \bibinfo {author} {\bibfnamefont {A.~M.}\ \bibnamefont {Reilly}},
  \bibinfo {author} {\bibfnamefont {A.}~\bibnamefont {Tkatchenko}},\ and\
  \bibinfo {author} {\bibfnamefont {M.}~\bibnamefont {Scheffler}},\ }\bibfield
  {title} {\bibinfo {title} {Performance of various density-functional
  approximations for cohesive properties of 64 bulk solids},\ }\href
  {https://doi.org/10.1088/1367-2630/aac7f0} {\bibfield  {journal} {\bibinfo
  {journal} {New J. Phys.}\ }\textbf {\bibinfo {volume} {20}},\ \bibinfo
  {pages} {063020} (\bibinfo {year} {2018})}\BibitemShut {NoStop}%
\bibitem [{\citenamefont {Berliner}\ and\ \citenamefont
  {Werner}(1986)}]{berliner_effect_1986}%
  \BibitemOpen
  \bibfield  {author} {\bibinfo {author} {\bibfnamefont {R.}~\bibnamefont
  {Berliner}}\ and\ \bibinfo {author} {\bibfnamefont {S.~A.}\ \bibnamefont
  {Werner}},\ }\bibfield  {title} {\bibinfo {title} {Effect of stacking faults
  on diffraction: {The} structure of lithium metal},\ }\href
  {https://doi.org/10.1103/PhysRevB.34.3586} {\bibfield  {journal} {\bibinfo
  {journal} {Phys. Rev. B}\ }\textbf {\bibinfo {volume} {34}},\ \bibinfo
  {pages} {3586} (\bibinfo {year} {1986})}\BibitemShut {NoStop}%
\bibitem [{\citenamefont {Felice}\ \emph {et~al.}(1977)\citenamefont {Felice},
  \citenamefont {Trivisonno},\ and\ \citenamefont
  {Schuele}}]{felice_temperature_1977}%
  \BibitemOpen
  \bibfield  {author} {\bibinfo {author} {\bibfnamefont {R.~A.}\ \bibnamefont
  {Felice}}, \bibinfo {author} {\bibfnamefont {J.}~\bibnamefont {Trivisonno}},\
  and\ \bibinfo {author} {\bibfnamefont {D.~E.}\ \bibnamefont {Schuele}},\
  }\bibfield  {title} {\bibinfo {title} {Temperature and pressure dependence of
  the single-crystal elastic constants of {Li} 6 and natural lithium},\ }\href
  {https://doi.org/10.1103/PhysRevB.16.5173} {\bibfield  {journal} {\bibinfo
  {journal} {Phys. Rev. B}\ }\textbf {\bibinfo {volume} {16}},\ \bibinfo
  {pages} {5173} (\bibinfo {year} {1977})}\BibitemShut {NoStop}%
\bibitem [{\citenamefont {Kittel}(2005)}]{kittel_intro_solid_2005}%
  \BibitemOpen
  \bibfield  {author} {\bibinfo {author} {\bibfnamefont {C.}~\bibnamefont
  {Kittel}},\ }\href@noop {} {\emph {\bibinfo {title} {{Introduction} to
  {solid} {state} {physics}}}},\ \bibinfo {edition} {8th}\ ed.\ (\bibinfo
  {publisher} {John Wiley \& Sons, Inc.},\ \bibinfo {year} {2005})\BibitemShut
  {NoStop}%
\end{thebibliography}%


\end{document}
