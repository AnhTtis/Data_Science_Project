\section{Conclusion}
\label{conclusion}

This paper addresses the problem of unsupervised cross-domain adaptation for 3D human pose estimation. To reduce the domain gap, we propose global position alignment and local pose augmentation. We argue that global position alignment is simple yet effective, and the local pose augmentation enhances the diversity of 2D-3D pose mapping. The proposed global position alignment module significantly boosts performance with no additional learnable parameters needed. We also show that adversarial pose augmentation based on Wasserstein distance can further obtain stable, diverse, and high-quality pose pairs. With extensive and convincing experiments and ablation studies, \textit{PoseDA} can be flexibly applied on any 2D-3D pose lifting network and make a significant step towards solving domain adaptation problems for 3D human pose estimation.

\paragraph{Limitations and Future Work} Although we show that generalization performs better than adaptation on domain gap over local pose, it is still possible to design a method that can adapt 3D pose without hurting the diversity of 2D-3D mappings.

\paragraph{Acknowledgement} This work is supported by the National Key R\&D Program of China No.2022ZD0162000, and National Natural Science Foundation of China No.62106219.