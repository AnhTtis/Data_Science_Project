\subsection{Discussion Compared to SOTA Methods}

% \begin{figure*}[!t]
%     \centering
%     \includegraphics[width=0.95\linewidth]{figures/dis.pdf}
%     \caption{Discussion of input selection of generator condition, real 3d pose, and real 2d pose in local pose augmentation (LPA). Real pose means the goal of the discriminator. We show that applying adaptation on local pose actually degrades the performance.}
%     \vspace{-1.0em}
%     \label{fig:dis}
% \end{figure*}


The most related SOTA works are PoseAug and AdaptPose. Our proposed GPA has significant differences from the two methods in several aspects. In addition, our design of the discriminator is also different from these two methods.
% , we made an important discovery, as shown in the discussion section. 
PoseAug is a simple domain generalization framework that uses two discriminators, called $D_{3D}$ and $D_{2D}$. Their goal is to regulate the poses generated by the generator, making them similar to the 3D and 2D poses of the source domain. AdaptPose is a domain adaptation framework that also uses these two discriminators, but the goal of $D_{2D}$ is to make the generated poses more similar to the 2D poses of the target domain. AdaptPose is an intuitive framework. However, according to the supplementary material B, more discussion on Local Pose Augmentation, the assumption of AdaptPose is not supported after careful and comprehensive experiments. In the human pose domain adaptation problem, we abandon $D_{2D}$ and only use $D_{3D}$ to regulate the quality of the generated poses to achieve better results. The reason we give is that: 1) Compared to PoseAug, making the generated 2D poses unnecessary similar to the source domain brings greater diversity. 2) Compared to AdaptPose, we believe that forcing the generated 2D poses to be similar to the target domain does not guarantee that the corresponding 3D poses are also similar. In other words, this is also because the 2D-3D mapping has large ambiguity. 
% Therefore, the title of the article Global Adaptation meets Local Generalization comes from this insight. 
As a result, with our carefully designed GPA and LPA modules, global adaptation and local generalization are well combined, which is also our major contribution.

As for why AdaptPose is better than PoseAug, we think the translation part in the generator plays a role, and forms a transfer on the projected 2D scale and position. While other works use camera view estimation or generation as an auxiliary task to address the global position adaptation problems, GPA achieves alignment explicitly and directly in this part.

We conduct extensive and convincing experiments on the input selection of local pose augmentation module. All the experiments in this section are based on pre-trained VideoPose3D and global position alignment. Following~\cite{gholami2022adaptpose, gong2021poseaug}, we design a 2D pose discriminator of 5-layer MLP. Note that the condition of generator is used only to generate transformation, we still apply those transformation on the 3D pose from the source dataset. The 3D pose of target dataset used in 3D discriminator is the prediction of corresponding 2D pose by the pre-trained lifting network. As shown in Table~\ref{tab:dis}, these experiments lead to two important conclusions: 1) the design of 2D pose discriminator is unnecessary, and 2) there is a performance drop no matter either the target domain information is involved in the generator or the discriminator to adapt the characteristics of local pose. 

We argue that the adaption on the local pose in 2D (\textit{e.g.}, bone vector) actually \textit{hurts} the diversity of the 2D-3D mapping. To be specific, the human with the same 2D pose may act totally different in 3D space. The alignment of 2D local pose actually means nothing in 3D. According to this viewpoint, PoseAug~\cite{gong2021poseaug} does data augmentation on both global position and local pose, and AdaptPose~\cite{gholami2022adaptpose} focuses on domain adaption on both of them. In our case, we do domain adaptation on global position and augmentation on local pose, achieving the best performance.

\begin{table}[!t]
    \small
    \centering
    \setlength{\tabcolsep}{1mm}
    \begin{tabular}{ccc|ccc}
        \specialrule{1pt}{1pt}{2pt}
        $\mathcal{G}_{cond}$ & $\mathcal{D}_{3D}$ & $\mathcal{D}_{2D}$ & MPJPE~($\downarrow$) & PCK~($\uparrow$) & AUC~($\uparrow$)\\
        \hline
        - & - & - & 66.07 & 90.87 & 60.07\\
        \hline
        $\mathcal{T}$ & $\mathcal{T}$ & $\mathcal{T}$ & 66.34 & 90.81 & 59.72\\
        $\mathcal{T}$ & $\mathcal{S}$ & $\mathcal{T}$ & 65.91 & 91.02 & 59.92\\
        $\mathcal{T}$ & $\mathcal{S}$ & - & 65.37 & 90.98 & 60.30\\
        \hline
        $\mathcal{S}$ & $\mathcal{T}$ & $\mathcal{T}$ & 64.20 & 91.64 & 60.85\\
        $\mathcal{S}$ & $\mathcal{T}$ & - & 63.66 & 91.57 & 61.34\\
        \hline
        $\mathcal{S}$ & $\mathcal{S}$ & $\mathcal{S}$ & 73.55 & 88.96 & 56.41\\
        $\mathcal{S}$ & $\mathcal{S}$ & $\mathcal{T}$ & 65.46 & 91.27 & 60.03\\
        $\mathcal{S}$ & $\mathcal{S}$ & - & \textbf{61.36}& \textbf{92.05} & \textbf{62.52}\\
        \specialrule{1pt}{1pt}{2pt}
    \end{tabular}
    \caption{\small The input selection of generator condition $\mathcal{G}_{cond}$, 3D pose discriminator $\mathcal{D}_{3D}$, and 2D pose discriminator $\mathcal{D}_{2D}$ in LPA. $S, T$ denote the pose from source or target domain. Source: H3.6M. Target: 3DHP.}
    \label{tab:dis}
\end{table}   
