\section{Related Work}
\label{sec:rel}
\subsection{Two-stage 3D Human Pose Estimation}
Inspired by the rapid development of 2D human pose estimation algorithms, many works have tried to utilize 2D pose estimation results for 3D human pose estimation to improve in-the-wild performance~\cite{wang2021deep}. Two-stage 3D human pose estimation approaches, which first estimate 2D poses and then lift 2D poses to 3D poses, have been developed. Chen~\etal~\cite{chen20173d} present a simple approach to 3D human pose estimation by performing 2D pose estimation, followed by 3D exemplar matching. Martinez~\etal~\cite{martinez2017simple} propose a baseline focusing on lifting 2D poses to 3D with a simple yet effective neural network, which popularizes the research on lifting 2D pose to 3D space. Recently, semi/self-supervised learning based on geometry constraint~\cite{pavllo20193d, chen2019unsupervised, drover2018can} has been used to train models without 3D annotations or by auxiliary losses.

\subsection{Data Augmentation on 3D Human Poses}
Data augmentation is widely used to improve deep model generalization ability by enhancing training data diversity. Some methods apply pose data augmentation on images~\cite{rogez2016mocap, mehta2017vnect} or generate 3D synthetic data using graphics engines~\cite{chen2016synthesizing, yang20183d, gong2022posetriplet}. Other methods directly generate 2D-3D pose pairs by applying transformations on 3D skeleton\cite{li2020cascaded, gong2021poseaug, gholami2022adaptpose}. Gong~\etal~\cite{gong2021poseaug} make this augmentation process differentiable and further learnable. Those transformations consist of bone angle, bone length, and rigid-body transformation. %We argue that rigid transformation is the most simple yet efficient.

\subsection{Unsupervised Domain Adaptation for 3D Human Pose Estimation} Unsupervised domain adaptation~\cite{wilson2020survey} aims to transfer models from a fully-labeled source domain to an unlabeled target domain. Kundu~\etal~\cite{kundu2022uncertainty} address unsupervised domain adaptation problem by modeling pose uncertainty based on RGB images. Li~\etal~\cite{li2020cascaded} are the first to generate corresponding 2D-3D pose pairs by applying skeleton transformations. Gong~\etal~\cite{gong2021poseaug} develop this data augmentation module with the differentiable form to jointly optimize the augmentation process with a end-to-end trained model. Gholami~\etal~\cite{gholami2022adaptpose} further utilize 2D pose in target domain to address the domain adaptation problem. 

However, all existing methods focus more on local pose augmentation or adaptation. In this paper, \textit{PoseDA} utilizes both local pose augmentation and global position alignment, and achieves state-of-the-art performance in cross-domain tasks.

%Many deep domain adaptation methods incorporate adversarial training. The Generative Adversarial Network (GAN) method~\cite{goodfellow2020generative} is a deep generative model that pits two networks generator and discriminator against one another. To close the domain gap, the discriminator is considered as a domain classifier to guide the generator to learn domain-invariant features~\cite{tzeng2017adversarial, long2018conditional}. This paradigm has proven to be fruitful in semantic segmentation~\cite{zou2018unsupervised, vu2019advent}, human pose estimation~\cite{yang20183d}, image de-raining~\cite{zhang2019image}, \textit{etc}.