% \begin{itemize}
%     \item 
% \end{itemize}




%\changeda{R1: I find the introduction a bit long and wordy, and would encourage the authors to work
%on improving this part in particular - especially since the introduction is very important to
%capture the interest of readers.}

%\abnote{check network vs. software after its damnatio memoriae}\bdsnote{I already did this for the intro, removing one sentence about the point of reference.}
In the \mate attack model, attackers have white-box access to the software. This means they have full control over the systems on which they identify successful attack vectors in their lab, for which they use all kinds of attacker tools such as simulators, debuggers, disassemblers, decompilers, etc. Their goal is to reverse engineer the software (e.g., to steal valuable secret algorithms or embedded cryptographic keys or to find vulnerabilities in the code), to tamper with the software (e.g., to bypass license checks or to cheat in games), or to execute it in unauthorized ways (e.g., run multiple copies in parallel). In general, \mate attacks target software to violate the security requirements of assets present in that software.

{\mate} {\softprot} then refers to \emph{protections deployed within that software} to mitigate \mate attacks.\footnote{For the sake of brevity, we will omit the {\mate} and simply use {\softprot} to mean {\mate} {\softprot} in the remainder of this paper.}  {\softprot} is hence much narrower than the broad umbrella of software security. The latter also includes scenarios in which software is exploited to violate \emph{security requirements of other system components}, e.g., infiltrating networks or escalating privileges. In such scenarios, attackers start with limited capabilities, such as having only unprivileged, remote access to a computer via a web server interface. 

Because \mate attackers have full control over the devices in their labs, \softprot needs to defend assets in the software \emph{without relying on external services} running on those devices. Instead, defenders can only rely upon protections deployed within the protected software or remote servers. 
Advances in cryptography have yielded techniques that provide strong security guarantees but also introduce orders of magnitude performance overhead~\cite{horvath2020cryptographic}.
They are hence rarely practical today. Then again, practical {\softprot} is still dominated by fuzzy concepts and techniques~\cite{collbergbook}. {\softprot}s such as remote attestation, obfuscation, and anti-debugging do not aim to mitigate \mate attacks completely.
Instead, they aim to delay attacks and put off potential attackers by increasing the expected cost of attacks and by decreasing the expected \roi. 

As observed during a recent Dagstuhl seminar on \softprot Decision Support and Evaluation Methodologies~\cite{Dagstuhl}, the \softprot field is facing severe challenges: \sto is omnipresent in the industry; 
{\softprot} tools and consultancy are expensive and opaque; 
there is no generally accepted method for evaluating {\softprot}s and {\softprot} tools. Moreover, {\softprot} tools are not deployed sufficiently~\cite{ceccato-new-one,BSA,GOP,arxan-report}; and expertise is largely missing in software vendors to deploy (third-party) {\softprot} tools~\cite{Gartner-report-online,Irdeto-report1,Mandiant}. 
Moreover, we lack standardization. 
The \nist SP800-39 IT systems risk management standard~\cite{nistSP800-39} or the ISO27k framework for information risk management~\cite{ISO27k}, which are deployed consistently in practice to secure corporate computer networks, have no counterpart or instance in the field of {\softprot}. 
Neither do we have concrete regulations to implement \gdpr compliance in applications.% sensitive to \mate attacks. 
We can summarize the status of the \softprot domain as an industry with information system business needs involving so-called wicked problems~\cite{hevner2004design}. The foundations and methodologies currently available in the \softprot knowledge base have not met those needs. 

%\changeda{R1:I am not convinced that all bottom-up approaches of this type can be characterised as DSR - as seems to be stated. Suggest to rephrase (e.g., initiatives)}\bdsnote{I think my novel text answers this already.}
To plug gaps in this knowledge base, most existing \softprot research focuses on piece-wise, bottom-up extensions to its foundations and methodologies by presenting ever more novel \softprot \changed{artifacts} and attack \changed{artifacts} in a \softprot arms race. \changed{That existing offensive and defensive research fits into the information systems \dsr paradigm. Hevner et al.\ define this paradigm as research seeking to ``extend the boundaries of human and organizational capabilities by creating new and innovative artifacts,'' and ``create innovations that define the ideas, practices, technical capabilities, and products through which the analysis, design, implementation, management, and use of information systems can be effectively and efficiently accomplished''~\cite{hevner2004design}}. 

However, to overcome the aforementioned shortcomings and to pave the road towards a standardized risk management approach and automated decision support for \softprot, we are of the opinion that such bottom-up \dsr needs to be complemented with a holistic, top-down design search process in which we study what an end-to-end \softprot risk management approach has to cover and what parts can and should ideally be automated. Our own research hence includes both the bottom-up and the top-down approach in the search for answers to the following research questions (RQs): %to pave the road for a standardized risk management approach and decision support for \softprot:
\begin{itemize}\itemsep 0pt 
    \item \textbf{RQ1}: %Is it possible to improve the deployment of {\softprot}s and the usability of \softprot tools by means of automated decision support tools? 
    \changed{Can automated decision support tools assist experts with the deployment of {\softprot}s and the use of \softprot tools? Can they also assist non-experts?}
    \item \textbf{RQ2}: To adopt a standardized risk management approach in the domain of SP, which
constructs, models, and methods does the adopted approach need to entail, and which ones thereof should ideally be automated? 
    \item \textbf{RQ3}: Which parts of such an approach can already be automated using decision support tools that instantiate the identified constructs, models, and methods?
\end{itemize}

RQ1 is formulated rather broadly, as usability covers many different aspects such as efficacy, efficiency, and user-friendliness. Later in the paper this RQ will be refined.
%\changeda{R1:  When it comes to fulfilment of RQ3, I do not follow the argumentation that what is possible (RQ3) necessarily is the same as what you have done in the tool. }\bdsnote{I think my novel text answers this already.}
For answering it, we developed a \poc decision support tool for \softprot bottom-up, based on concrete requirements and needs from industrial partners of a European research project. Towards RQ2, we explored top-down how a standardized risk management approach can benefit the domain of \softprot. With the birds-eye view of such an approach, we identified existing work to build on and aspects that need more research and/or collaboration in the community. To ensure the relevance of our proposed design, we build on our experience in our academic \softprot research and past collaborations with the industry. That experience allows us to formulate the domain-specific requirements, consider the relevant industrial \sdlc requirements and practices, and position existing domain-specific knowledge in the design. Finally, for formulating a \changed{partial, lower bound} answer to RQ3 \changed{we identified which artifacts from our answer to RQ2 are already instantiated and automated in our \poc tool.}



%\changeda{R1:  would also encourage the authors to explain why they mention network security specifically, as software security seems closer to their needs. One or two sentences on this would be very helpful.}

This paper reports our research findings and presents our answers to the RQs with the following contributions. 
First, we provide a rationale for adopting and standardizing risk management processes for {\softprot}. We discuss several observations on the failing {\softprot} market and we analyse why existing standards are not applicable as is for {\softprot}. 
%To that extent, we use network security as a point of reference because the process of standardization and the adoption of standards are much more mature in that domain. 
Where useful, we also highlight differences between \softprot and other security fields such as cryptography, network security, and software security.

Secondly, we discuss in depth how to adopt the \nist risk management approach. We identify which artifacts in the forms of constructs, models, methods, and instantiations, i.e., (semi-)automated tools, we consider necessary and feasible to introduce and deploy the \nist risk management approach for \softprot. For all the required processes, we highlight (i) the current status; (ii) \softprot-specific concepts/artifacts to be covered; (iii) what existing parts can be borrowed from other fields; (iv) open questions and challenges that require further research; (v) needs for the research community and industry to come together to define standards; and (vi) relevant aspects towards formalizing and automating the processes.  

Finally, we demonstrate that several aspects can already be formalized and automated by presenting a \poc decision support system that automates some of the major risk management activities.
Even if not completely automated, this demonstrates that the more abstract constructs, models, and methods we discuss can indeed be instantiated concretely. This \poc  provides a starting point for protecting applications and building a more advanced system that follows all the methodological aspects of a \nist 800-compliant standard with industrial-grade maturity. 
The first results obtained with the tool have been validated mostly positively by industry experts on Android mobile app case studies of real-world complexity and are presented according to the Framework for Evaluation of Design Science~\cite{FEDS-evaluation}. 

%\textcolor{blue}{The remainder of the paper is structured as follows. In Section~\ref{sec:approach}, we discuss how we arrived at our current vision. In Section~\ref{sec:} we discuss the rationale for a standardized and automated risk management approach to {\mate} {\softprot}. In Section~\ref{sec:requirements}, we then extensively discuss how to adopt such an approach, thus providing an answer to RQ1. In Section~\ref{sec:workflow} we present our \poc decision support system that automates many of the steps in the workflow of the approach. That artifact is evaluated qualitatively and technically in Section~\ref{sec:esp:results}. Together, Sections~\ref{sec:workflow} and~\ref{sec:esp:results} provide an answer to RQ2. Finally, we draw conclusions in Section~\ref{sec:conclusion} and discuss future work.\bdsnote{I think this overview can be dropped, given the almost complete overview in Section 2.}}


