\section{Conclusion and future work}
\label{sec:conclusion}

\subsection{Conclusions}
%\changeda{R1: he conclusion would be easier to read and understand if the RQs were
%repeated (not just referred to by their number) and the answer to the RQs be made more
%clear.}

%\changeda{R1: For RQ1 and RQ2, it would be nice if you added a summary of the answer to the RQ, not only saying that you evaluated it. E.g., for RQ1, to write "we developed and presented … PoC implementation, and we found that …"}\bdsnote{easy to do}

We discussed the necessity and potential benefits of a standardized, formalized, and automated approach for risk management in the context of software protection against man-at-the-end attacks. To that end, we discussed just such a risk management approach for software protections, which we based on the NIST SP800-39 standard for risk management for information security. 

To provide an answer to RQ1 on the feasibility of automated decision support tools to improve the useability of \softprot tools, we developed and presented the \esp design and an evaluation of its \poc implementation. \changed{We found that many human expert judgment tasks can already benefit from automated tools and the data they produce, which experts found sufficiently usable, acceptable, and efficient; and of which they assessed the results as sufficiently correct and comprehensible.}

As an answer to RQ2 on how standardized risk management approaches can be adopted for \softprot, we discussed in detail how the different aspects of software protection deployment decision making could and should be mapped onto risk framing, risk assessment, risk mitigation, and risk monitoring phases. \changed{For all phases combined, we identified 50 required constructs, 6 models, and 32 methods that the adopted approach should entail.}

We answered RQ3 on the feasibility of formalizing and automate parts of the adopted approach by providing a mapping of the abstract construct, model, and method artifacts identified for the adopted approach onto the concrete instantiation artifacts that make up the \esp. 


With these answers, we have provided convincing evidence that the proposed approach is feasible and can be automated to a large degree, and deserves the launch of a structured community effort that leads to future standardization and automation.


\subsection{Future Work}

It is clear that quite some future work is needed, however, for the standardization itself, as well as for improving, refining, extending, replacing, and complementing the rather embryonic instantiations of the necessary constructs, models and methods currently available in the presented \esp in support of the automation of tasks in the approach. The artifacts in Tables~\ref{tab:constructs},~\ref{tab:models}, and~\ref{tab:methods} which have no counterpart in the \esp yet are clear examples of where more research is needed.

%\changeda{I am confused as to why you refer to Section 5 for discussion - I would not consider the content of Section 5 as a discussion in any traditional sense. Please rephrase.}\bdsnote{Done by simply omitting the word "discussion".}
\changed{Section~\ref{sec:requirements} also highlighted} a number of topics for future work in the form of open issues and open research questions, research directions that we consider interesting, and development steps for which a community effort is needed. Tables~\ref{tab:openissues},~\ref{tab:researchdirs}, and~\ref{tab:futurework} provide an overview of that future work. 



\begin{table}[t]
\begin{center}

{\footnotesize
\begin{tabular}{r l}
No. & Open Issue / Research Question Subjects\\
\hline
\openissuelab{1}  & \hyperlink{openissue.1}{secondary asset (a.k.a. pivots, hooks, and mileposts) model design}  \\
\openissuelab{2}  & \hyperlink{openissue.2}{selection and models of protection policy requirements}     \\
\openissuelab{3}  & \hyperlink{openissue.3}{level of detail needed in models of attacker capabilities}   \\
\openissuelab{4}  & \hyperlink{openissue.4}{to what extent worst-case assumptions are useful}      \\
\openissuelab{5}  & \hyperlink{openissue.5}{best abstractions to model software features that impact the execution of attack steps} \\
\openissuelab{6}  & \hyperlink{openissue.6}{empirical validation of such models and metrics, in particular for manual human activities}    \\
\openissuelab{7}  & \hyperlink{openissue.7}{identification of viable attack paths on the basis of software analysis results}   \\
\openissuelab{8}  & \hyperlink{openissue.8}{estimation of attack step's required effort and likelihood of success}  \\
\openissuelab{9}  & \hyperlink{openissue.9}{extent to which automated techniques can replace human pen testing} \\
\openissuelab{10}  & \hyperlink{openissue.10}{required granularity of attack steps forming attack paths} \\
\openissuelab{11}  & \hyperlink{openissue.11}{incorporation of informal information obtained from experts (e.g., pen testers) in automated threat analysis}   \\
\openissuelab{12}  & \hyperlink{openissue.12}{incremental attack path enumeration}    \\
\openissuelab{13} &  \hyperlink{openissue.13}{required precision of pre-deployment \softprot impact estimation}   \\
\openissuelab{14}  & \hyperlink{openissue.14}{pre-deployment potency, resilience, and stealth estimation for layered {\softprot}s}  \\
\openissuelab{15}  & \hyperlink{openissue.15}{pre-deployment estimation of \softprot impact on attack success probability }   \\
\openissuelab{16}  & \hyperlink{openissue.16}{validation of deployed \softprot against assumptions made pre-deployment }\\
\end{tabular}
}

\caption{Open issues and topics of open research questions identified in the paper}
\label{tab:openissues}
\end{center}
\end{table}

\begin{table}[t]
\begin{center}

{\footnotesize
\begin{tabular}{r l}
No. & Research Direction\\
\hline
\researchdirlab{1}  & \hyperlink{researchdir.1}{the concept and use of protection policy requirements} \\
\researchdirlab{2}  & \hyperlink{researchdir.2}{machine learning techniques to identify and quantify feasible attack paths} \\
\researchdirlab{3}  & \hyperlink{researchdir.3}{adoption of exploit generation techniques to identify feasible attack paths}  \\
\researchdirlab{4}  & \hyperlink{researchdir.4}{gradual path from a mostly manual process to automated feasible attack path identification} \\
\researchdirlab{5}  & \hyperlink{researchdir.5}{adoption of risk monetisation to evaluate and prioritize actual risks} \\
\researchdirlab{6}  & \hyperlink{researchdir.6}{adoption of the OWASP risk rating methodology to evaluate and prioritize actual risks}  \\
\researchdirlab{7}  & \hyperlink{researchdir.7}{single-pass selection of layered {\softprot}s with accurate assessment of impact on threats and risks} \\
\researchdirlab{8}  & \hyperlink{researchdir.8}{multi-pass selection of layered {\softprot}s with accurate assessment of impact on threats and risks}\\
\researchdirlab{9}  & \hyperlink{researchdir.9}{machine learning techniques to select the most effective layered combinations of {\softprot}s} \\
\end{tabular}
}

\caption{Potentially interesting research directions identified in the paper}
\label{tab:researchdirs}
\end{center}
\end{table}

\begin{table}[t]
\begin{center}

{\footnotesize
\begin{tabular}{r l}
No. & Required community efforts \\
\hline
\futureworklab{1}  & \hyperlink{futurework.1}{provisioning a complete vocabulary and methodology to describe the risk frame} \\
\futureworklab{2}  & \hyperlink{futurework.2}{provisioning a standard taxonomy of assets and their relevant features}   \\
\futureworklab{3}  & \hyperlink{futurework.3}{provisioning a standard taxonomy of \softprot security requirements} \\
\futureworklab{4}  & \hyperlink{futurework.4}{provisioning and maintaining a living catalog of potential attack steps and their relevant features}  \\
\futureworklab{5}  & \hyperlink{futurework.5}{provisioning a standard taxonomy of SPs and their relevant features in support of decision support} \\
\futureworklab{6}  & \hyperlink{futurework.6}{standardizing methodology for defining the actual threat model, attack surface and attack vectors} \\
\end{tabular}
}

\caption{Topics requiring a collaborative development effort by various stakeholders in the \softprot community}
\label{tab:futurework}
\end{center}
\end{table}