\documentclass[10pt,twocolumn,letterpaper]{article}
\usepackage[rebuttal]{cvpr}

% Include other packages here, before hyperref.
\usepackage{graphicx}
\usepackage{amsmath}
\usepackage{amssymb}
\usepackage{booktabs}


% If you comment hyperref and then uncomment it, you should delete
% egpaper.aux before re-running latex.  (Or just hit 'q' on the first latex
% run, let it finish, and you should be clear).
\usepackage[pagebackref,breaklinks,colorlinks,bookmarks=false]{hyperref}

% Support for easy cross-referencing
\usepackage[capitalize]{cleveref}
\crefname{section}{Sec.}{Secs.}
\Crefname{section}{Section}{Sections}
\Crefname{table}{Table}{Tables}
\crefname{table}{Tab.}{Tabs.}

% If you wish to avoid re-using figure, table, and equation numbers from
% the main paper, please uncomment the following and change the numbers
% appropriately.
%\setcounter{figure}{2}
%\setcounter{table}{1}
%\setcounter{equation}{2}

% If you wish to avoid re-using reference numbers from the main paper,
% please uncomment the following and change the counter for `enumiv' to
% the number of references you have in the main paper (here, 6).
%\let\oldthebibliography=\thebibliography
%\let\oldendthebibliography=\endthebibliography
%\renewenvironment{thebibliography}[1]{%
%     \oldthebibliography{#1}%
%     \setcounter{enumiv}{6}%
%}{\oldendthebibliography}


%%%%%%%%% PAPER ID  - PLEASE UPDATE
\def\cvprPaperID{9933} % *** Enter the CVPR Paper ID here
\def\confName{CVPR}
\def\confYear{2023}

\begin{document}

%%%%%%%%% TITLE - PLEASE UPDATE
% \title{\LaTeX\ Guidelines for Author Response}  % **** Enter the paper title here

% \maketitle
\thispagestyle{empty}
\appendix

% We appreciate the valuable review comments 
% \noindent\textbf{Overview}
% Classical 3D reconstruction with photogrammetry is a complex process consisting of SfM, MVS, surface reconstruction, and texture mapping. Our method builds on this framework and improves it in various ways. Since we use a mobile device equipped with VIO system and low-resolution LiDAR, we proposed an RGBD-aided SfM which can estimate accurate poses better than ARKit poses and a COLMAP SfM. In the geometry reconstruction and texture optimization parts, our main focus is how learning-based techniques can actually improve classical 3D reconstruction performance. 
% Fig.5 => 포즈 추정이 Modeling에 주는 영향을 Figure 5에서 설명한다. 즉 더 좋은 방식의 모듈이 있으면 이 방식으로 교체하여 좋은 성능을 낼 수 있다 => 이거 굳이 안써도 되나? ablation study 가 부족하다는게 왠지 리뷰어들이 neural 쪽을 말하는거 같은 생각이 자꾸 들어서
% Fig.6, Table2 
% Fig.7, Table1 
% Thanks to your comment, we will add more ablation studies 

% \noindent\textbf{Motivation and ablation study} Recently, neural surface reconstruction and texture mapping with differentiable rendering have become popular. However, classical 3D reconstruction with photogrammetry is still widely used and has many strengths such as relatively cheap
% hardware or robustness to certain environments compared to neural methods. To the best of our knowledge, most neural methods do not fully utilize the information from classical 3D reconstruction and only focus on building an end-to-end learning framework. Instead, we study how to combine the advantages of learning-based methods and classical reconstruction in order to outperform both methods' performance. If we skip a step, we will not be able to achieve the best reconstruction performance. In the manuscript, Fig.6 and Table 2. show that our method outperforms both classical and neural geometry reconstruction methods. Fig.7 and Table 1. show that our method is better than both classical and neural texture reconstruction methods.
% Thanks to reviewer's comment, we will conduct more ablation studies in terms of not only performance but also either run-time or influence to the next step. We will describe the correlation between reconstruction performance and run-time, and provide more detailed analysis about the necessity of each step. 
% Furthermore, we adopt the module-based framework instead of end-to-end learning because of two reasons. First, we can explain the output of our system by observing the results of intermediate modules. Second, each module can be replaced by a better algorithm if it can improve the reconstruction performance.

% Thanks to the valuable comments, we will provide more ablation studies of our proposed loss functions in each step and add details why we mainly chose them. 
% our main focus is on 

\noindent\textbf{[Overview]} Classical 3D reconstruction with photogrammetry is widely used and has many strengths such as working well on relatively cheap hardware and having an explainable pipeline. 
% Recently, neural surface reconstruction and texture mapping with differentiable rendering have become popular. 
To the best of our knowledge, most neural methods do not fully utilize the information from classical 3D reconstruction and only focus on building an end-to-end learning framework. In contrast, our approach combines the advantage of learning-based methods and classical reconstruction to outperform each method's performance. 
% We summarize the ablation study and computation time. 

\noindent\textbf{[Ablation Study and Computation Time]} Our pipeline consists of three modules. First, \textbf{RGBD-aided SfM} (Sec.3.1) is imperative to refine initial poses due to noisy sensor data in Fig.5 (a). Without RGBD-aided SfM, our geometric reconstruction with initial poses shows poor quality in Fig.5 (b). 
% In the pipeline, our module is crucial. 
This process takes around 5 minutes on i9 CPU and Nvidia 2080 GPU to estimate accurate poses. In \textbf{Geometry Reconstruction} (Sec.3.2), both classical 3D reconstruction methods like MVS and depth Fusion (ACMP and TSDF-Fusion in Fig.6) fail to generate a decent mesh. Thus, we emphasize that applying neural geometry reconstruction after MVS is necessary to generate a high-quality mesh (our method in Fig.6). We propose a new training method to use prior geometry cues from MVS. Since we outperform the baseline NeuS by 8\% in Table 2., our reconstruction algorithm is useful to improve the reconstruction performance. Additionally, efficient sampling in our method can reduce the training speed by around 20\%. Overall, most of the time in our pipeline is spent on geometry reconstruction (a few hours on Nvidia V100 GPU).  
% However, if the MVS results are too noisy, it may have a negative effect on performance (e.g. scan37 in Table 2).
Then, the mesh simplification which only takes a few seconds. Lastly, in \textbf{Texture optimization} (Sec.3.3), 
%classical texture reconstruction [53] is more photorealistic and sharper than all different methods in Fig.7. 
the classical texture reconstruction [53] takes only a few seconds. Fig.7 and Table 1. show that our proposed texture fine-tuning solves the seams and texture misalignment issues often seen with this classical method. Our proposed module, which takes less than 15 minutes on i9 CPU and Nvidia 2080 GPU, can generate visually realistic textures and improve the performance by 3.8 and 8.9\% in Table 1. If we skip a step, we will not be able to achieve the best reconstruction performance. We will add the run-time analysis to the manuscript.     

\noindent\textbf{[Response to Reviewer 1]}  
% Classical 3D reconstruction with photogrammetry is a complex process consisting of SfM, MVS, surface reconstruction, and texture mapping. We adopt this complex pipeline and add more steps to improve its performance. 
Please refer to \textbf{Overview} and \textbf{Ablation Study and Computation Time} for W1. Regarding W2, just like image-based global texture mapping [53,69], our approach aims to produce a \textit{view-independent} texture map from multi-view images, allowing for various applications such as texture editing or 3D content generation. Since the view-dependent effects are not important for our task, we don't need to use a color network in NeuS that encodes appearance. Additionally, the texture generated by the color network in NeuS can be too blurry compared to classical texture reconstruction [53,69]. Hence, we start from Waechter et al. [53] which can generate visually realistic and \textit{view-independent} texture images. 

\noindent\textbf{[Response to Reviewer 2]} 
Our smartphone application is used for collecting various sensor data, whereas our whole reconstruction pipeline is performed on a server (Supplement. Sec.2). Due to the training time and hardware, our reconstruction cannot perform on a mobile device. Instead, we focus on generating fine-detailed geometry and realistic texture with high computational power. To achieve this goal, previous works [24] mostly use specialized hardware e.g. turn-table or an expensive 3D scanning device. In our method, RGBD-aided SfM and texture optimization do not take much time on i9 CPU and Nvidia 2080 GPU, but most of the time is spent on geometry reconstruction on Nvidia V100 GPU. Please refer to \textbf{Overview} and \textbf{Ablation Study and Computation Time} for other comments (1), (2), (3).          
% \noindent\textbf{Other comments} 
 
\noindent\textbf{[Response to Reviewer 3]}
\noindent\textbf{Novelty} Please refer to \textbf{Overview}.
% Although neural geometry and texture reconstruction have been researched, 
We have focused on combining neural approaches with classical 3D reconstruction pipeline with photogrammetry, which can be directly applied to real-world objects collected by a mobile device.  

\noindent\textbf{Holes} The mesh simplification generated small holes. We can change the decimation ratio from 0.4\% to 1.6\% in order to remove these holes. We will address these issues and present the new results in the final version. 
% We will adjust the decimation ratio to remove these holes. 

\noindent\textbf{non-Lambertian objects} Reconstructing non-Lambertian objects is a well-known challenging problem. Both Classical SfM and MVS heavily rely on feature matching, thus the scanned object requires Lambertian features which is a limitation to reflecting surfaces. Our method can overcome this limitation to some extent. First, compared to classical SfM, our RGBD-aided SfM utilizes VIO poses which are robust to non-Lambertian surfaces. Moreover, neural surface reconstruction considers viewing direction and uses volumetric rendering which can consider non-Lambertian surface compared to classical MVS. Thus, it shows high reconstruction quality (first row in Fig.6). Our method performs well on commonly-seen real-world objects. However, since the SfM, MVS, and texture reconstruction in our method basically assume Lambertian materials, recovering non-Lambertian surfaces results in reduced quality compared to Lambertian objects.     

% Our method performs well on commonly-seen real-world objects. However, since the SfM, MVS, and classical texture reconstruction in our method basically assume Lambertian materials, recovering non-Lambertian surfaces shows reduced quality compared to Lambertian objects.   

\noindent\textbf{Lighting} We focus on collecting real-world data under varying lighting conditions. It is different from in-the-lab environments where the light condition can be controlled. Instead, we collect various datasets including both indoor (e.g. delivery robot) and outdoor (e.g. bicycle). 
% to consider photogrammetry issues such as lighting conditions. 
Since we do not design explicit modules to handle extreme lighting, our method works better in uniform lighting conditions. Also, in Fig.8, we compare other recent methods for decomposing lighting and materials, which do not work well on our real-world datasets. We will add our discussion about non-Lambertian surfaces and lighting conditions. With existing graphics pipelines, e.g., Blender, our textured mesh can be used for relighting.   
% Refer to IRON paper 



%%%%%%%%% REFERENCES
% {\small
% \bibliographystyle{ieee_fullname}
% \bibliography{egbib}
% }

\end{document}
