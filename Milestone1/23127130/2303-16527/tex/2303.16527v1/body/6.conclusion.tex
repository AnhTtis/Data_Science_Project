\section{Conclusion \& Limitations}
\label{sec:conclusion}

In this work, we studied the information contained in the features learned in the deep functional map pipelines, an aspect that has been neglected by all previous works. For this, we established a theorem according to which, under certain mild conditions, these features allow for \textit{direct} computation of point-to-point maps. Inspired by our analysis, we introduced two simple modifications to the deep functional map pipeline. We have shown that applying these modifications improves the overall accuracy of the pipeline in different scenarios, including supervised and unsupervised matching of non-rigid shapes, and makes the features usable for other downstream tasks.

A limitation of our approach is that it is currently geared towards \textit{complete} shape correspondence and must be adapted for partial shape matching. Moreover, it will be interesting to investigate how feature smoothness can be imposed in the context of noisy point clouds or other domains such as graphs or images.

\vspace{-3mm}
\paragraph{Acknowledgements}
The authors acknowledge the anonymous reviewers for their valuable suggestions. 
Parts of this work were supported by the ERC Starting Grant No. 758800 (EXPROTEA) and the ANR AI Chair AIGRETTE.

%\souhaib{do we have a limitation}