% CVPR 2023 Paper Template
% based on the CVPR template provided by Ming-Ming Cheng (https://github.com/MCG-NKU/CVPR_Template)
% modified and extended by Stefan Roth (stefan.roth@NOSPAMtu-darmstadt.de)

\documentclass[10pt,twocolumn,letterpaper]{article}
\pdfoutput=1

%%%%%%%%% PAPER TYPE  - PLEASE UPDATE FOR FINAL VERSION
%\usepackage[review]{cvpr}      % To produce the REVIEW version
\usepackage{cvpr}              % To produce the CAMERA-READY version
\makeatletter
\@namedef{ver@everyshi.sty}{}
\makeatother
% \usepackage{tikz}

\usepackage[accsupp]{axessibility}
%\usepackage[pagenumbers]{cvpr} % To force page numbers, e.g. for an arXiv version

% Include other packages here, before hyperref.
% \usepackage{graphicx}
% \usepackage{amsmath}
% \usepackage{amssymb}
% \usepackage{amsthm}
% \usepackage{booktabs}
% \usepackage{color}
% \usepackage[dvipsnames]{xcolor}
% \usepackage{colortbl}
% \usepackage{textcomp}
% \usepackage{epsfig}
% \usepackage{multirow}
% \usepackage{bigdelim}
% \usepackage{wrapfig}
% \usepackage{caption}
% % \usepackage{subcaption}
% \usepackage{url}
% \usepackage{nicefrac}
% \usepackage{tabularx}
% \usepackage{enumitem}


\newcommand{\tightpara}[1]{\vspace*{-3mm}\paragraph{#1}}


%%%%%%%%%%%%%%%%%%%%%%%%%%%%%%%%%%%%%%%%%%%%%%%%%%%%%%%%%%%%%%%%
%%%%%%%%%%%%%%%%%%%%%%%%%%%%%%%%%%%%%%%%%%%%%%%%%%%%%%%%%%%%%%%%
%%%%%%%%%%%%%%%%%%%%%%%%%%%%%%%%%%%%%%%%%%%%%%%%%%%%%%%%%%%%%%%%
\newcommand{\final}{0}  % 0 comment activated, 1 no comments 
\usepackage{geometrycollective}

\def\PaperTitle{Understanding and Improving Features Learned in Deep Functional Maps}

%%%%%%%%%%%%%%%%%%%%%%%%%%%%%%%%%%%%%%%%%%%%%%%%%%%%%%%%%%%%%%%%
%%%%%%%%%%%%%%%%%%%%%%%%%%%%%%%%%%%%%%%%%%%%%%%%%%%%%%%%%%%%%%%%
%%%%%%%%%%%%%%%%%%%%%%%%%%%%%%%%%%%%%%%%%%%%%%%%%%%%%%%%%%%%%%%%


%\newtheorem{theorem}{Theorem}

\theoremstyle{definition}
\newtheorem{definition}{Definition}[section]

\newtheorem{theorem}{Theorem}[section]
\newtheorem{lemma}[theorem]{Lemma}

\newtheorem{innercustomthm}{Theorem}
\newenvironment{customthm}[1]
  {\renewcommand\theinnercustomthm{#1}\innercustomthm}
  {\endinnercustomthm}

\newcommand{\C}{\mathbf{C}}
\newcommand{\A}{\mathbf{A}}
\newcommand{\X}{\mathbf{X}}
\newcommand{\Id}{\mathit{I}_d}
\newcommand{\opt}{\rm{opt}}


% It is strongly recommended to use hyperref, especially for the review version.
% hyperref with option pagebackref eases the reviewers' job.
% Please disable hyperref *only* if you encounter grave issues, e.g. with the
% file validation for the camera-ready version.
%
% If you comment hyperref and then uncomment it, you should delete
% ReviewTempalte.aux before re-running LaTeX.
% (Or just hit 'q' on the first LaTeX run, let it finish, and you
%  should be clear).




%%%%%%%%% PAPER ID  - PLEASE UPDATE
\def\cvprPaperID{1770} % *** Enter the CVPR Paper ID here
\def\confName{CVPR}
\def\confYear{2023}


\begin{document}

%%%%%%%%% TITLE - PLEASE UPDATE
\title{\PaperTitle}

\author{Souhaib Attaiki\\
LIX, École Polytechnique, IP Paris\\
{\tt\small attaiki@lix.polytechnique.fr}
% For a paper whose authors are all at the same institution,
% omit the following lines up until the closing ``}''.
% Additional authors and addresses can be added with ``\and'',
% just like the second author.
% To save space, use either the email address or home page, not both
\and
Maks Ovsjanikov\\
LIX, École Polytechnique, IP Paris\\
{\tt\small maks@lix.polytechnique.fr}
}
\maketitle

%%%%%%%%% ABSTRACT

Answering first-order logical (FOL) queries over knowledge graphs (KG) remains a challenging task mainly due to KG incompleteness. 
Query embedding approaches this problem by computing the low-dimensional vector representations of entities, relations, and logical queries. 
KGs exhibit relational patterns such as symmetry and composition and modeling the patterns can further enhance the performance of query embedding models.
However, the role of such patterns in answering FOL queries by query embedding models has not been yet studied in the literature.
In this paper, we fill in this research gap and empower FOL queries reasoning with pattern inference by introducing an inductive bias that allows for learning relation patterns. 
To this end, we develop a novel query embedding method, RoConE, that defines query regions as geometric cones and algebraic query operators by rotations in complex space. RoConE combines the advantages of Cone as a well-specified geometric representation for query embedding, and also the rotation operator as a powerful algebraic operation for pattern inference. 
%Therefore, RoConE enables inferring patterns during the multi-hop reasoning process.
Our experimental results on several benchmark datasets confirm the advantage of relational patterns for enhancing logical query answering task.
\section{Introduction}
Deep learning~\cite{dl} has been highly successful in computer vision~\cite{sg1,od1,app-detection,zhou2024diffdet4sar,li2024predicting,yang2024saratr,LiSARATRX25}, largely due to the availability of large-scale labeled datasets. However, in many practical scenarios, obtaining such large amounts of labeled data is difficult or costly. To address this challenge, Few-shot learning (FSL) aims to enable models to learn new tasks with only a limited number of labeled samples. Consequently, this problem has garnered significant attention in both academia and industry due to its broad real-world applications. While humans can easily distinguish between objects after seeing only a few examples, machines struggle to achieve similar efficiency. In domains such as natural scene images, large datasets are readily available, but FSL is crucial in scenarios where collecting large amounts of data is difficult. Since the problem was first introduced in 2006~\cite{fsl-1}, numerous methods have been proposed to tackle the challenges of FSL~\cite{fslsurvey,fslsurvey22,fslsurvey20,fsl18,fslsurvey1}.

With the development of FSL, challenges such as limited training data, domain variations, and task modifications have led to the emergence of various FSL variants, including semi-supervised FSL~\cite{semifsl}, unsupervised FSL~\cite{ufsl1,ufsl2}, zero-shot learning (ZSL)\cite{zsl1}, and cross-domain FSL (CDFSL)~\cite{feature-wise,bscd-fsl}, among others. These variants represent distinctive cases of FSL in terms of sample availability and domain learning. This paper focuses specifically on CDFSL variants. The traditional FSL problem assumes that both prior knowledge and target tasks come from the same domain, which is often restrictive in real-world applications. CDFSL addresses this issue by overcoming the domain gap between auxiliary data (which provides prior knowledge) and the target data in FSL tasks, as show in Figure~\ref{int}. For instance, in art image recognition tasks involving scribble, cartoon, or sketch images, FSL could theoretically leverage prior knowledge from related domains like cartoons and sketches. However, such data is often scarce due to copyright restrictions and the high cost of collection. As a result, researchers have turned to data-rich domains, such as natural scene images, to address the challenges of few-shot image recognition in the field of art.
However, the significant domain gap between these domains often leads to performance degradation in FSL. CDFSL faces challenges from both transfer learning and FSL, including domain gaps, class shifts, and the scarcity of labeled samples in the target domain, making it a more complex task. Since its formal introduction in 2020~\cite{feature-wise}, CDFSL has garnered widespread attention, with numerous methods published in top venues~\cite{bscd-fsl,st,dynamic,hybrid_1,feature_reweight_6}. Figure~\ref{imaging} presents the milestones of CDFSL technologies from 2019 to the present, showcasing representative CDFSL methods and related benchmarks.
\begin{figure}%[b]
	\centering
  \vspace{-0.3cm}
 	\includegraphics[width=0.9\linewidth]{CDFSLProblem-10.pdf}
  \vspace{-0.3cm}
	\caption{\textcolor{black}{The difference of few-shot learning and cross-domain few-shot learning.}}
 \vspace{-0.4cm}
	\label{int}
\end{figure}


So far, several surveys have provided comprehensive overviews and future directions for FSL~\cite{fsl18,fslsurvey,fslsurvey1,fslsurvey20,fslsurvey22}. For example,\cite{fsl18} categorizes FSL into experiential and conceptual learning, while\cite{fslsurvey} focuses on empirical risk minimization and defines FSL by experience, task, and performance, introducing CDFSL as a branch of FSL. Both~\cite{fslsurvey1} and~\cite{fslsurvey20} highlight CDFSL as a variant of FSL, discussing meta-learning approaches and benchmarks. Lastly,~\cite{fslsurvey22} offers a taxonomy based on prior knowledge and emphasizes that current methods have yet to fully tackle cross-domain challenges. Collectively, these works point to cross-domain learning as a promising area for future FSL research. Currently, there are two elementary surveys on CDFSL~\cite{wang2023survey,deng2023survey}. \cite{wang2023survey} classifies methods into benchmark, single source, and multiple source categories, while~\cite{deng2023survey} categorizes algorithms into data augmentation and feature alignment paradigms. In contrast, to stimulate future research and help newcomers better understand this challenging problem, this paper offers the first classification grounded in theoretical analysis and provides a comprehensive review, offering deeper insights into the core principles of CDFSL. Firstly, this paper compiles and analyzes a broad range of literature on the topic. An analysis of the reference index reveals that even before the formal introduction of CDFSL, some works had already tried to solve cross-domain issues within the FSL framework~\cite{clc, rffl}. Following its formal introduction as a branch of FSL, CDFSL has garnered significant attention. Additionally, we define CDFSL using both machine learning theory~\cite{ml,erm1} and transfer learning principles~\cite{tltheory}. Secondly, our analysis highlights that the unique challenge in CDFSL lies in the unreliable nature of two-stage empirical risk minimization. The details are discussed in Section~\ref{background}. To address these challenges, the paper organizes CDFSL research into four categories: $\mathcal{D}$-Extension, $\mathcal{H}$-Constraint, $\Delta$-Adaptation, and hybrid approaches. We also compile relevant datasets and benchmarks to evaluate the methods, and analyze their performance, as discussed in Sections~\ref{methods} and~\ref{performance}. Finally, we explore future research directions for CDFSL by considering three perspectives, including problem set-ups, applications, and theories, which provide a comprehensive understanding of the field and its potential for future growth. Contributions of this survey can be summarized as follows:

\begin{itemize}
    \item We analyzed existing CDFSL papers and provided a comprehensive survey. We also defined CDFSL formally, connecting it to classic ML~\cite{ml,erm1} and transfer learning theory~\cite{tltheory}. This helps guide future research in the field.
    \item We listed relevant learning problems for CDFSL with examples, clarifying their relation and differences. This helps position CDFSL among various learning problems. We also analyzed unique issues and challenges of CDFSL, helping to explore a scientific taxonomy for CDFSL work.
    \item We conducted an extensive literature review, organizing it into a unified taxonomy based on $\mathcal{D}$-Extension, $\mathcal{H}$-Constraint, $\Delta$-Adaptation, and hybrid approaches. We introduced applicable scenarios for each taxonomy to help discuss its pros and cons. We also presented datasets and benchmarks for CDFSL, summarizing insights from performance results to improve understanding of CDFSL methods.
    \item We proposed promising future directions for CDFSL in problem set-ups, applications, and theories, based on current weaknesses and potential improvements.
\end{itemize}

\begin{figure}
	\centering
  \vspace{-0.3cm}
        \includegraphics[width=\linewidth]{response/crop_fig2.pdf}
 \vspace{-0.5cm}
	\caption{Chronological milestones of CDFSL from 2019 to the present, including representative CDFSL approaches and related benchmarks. Key events include the release of Meta-Dataset~\cite{meta-dataset} and BSCD-FSL~\cite{bscd-fsl} in 2020, the introduction of pioneering works such as~\cite{feature-wise}, and subsequent contributions like~\cite{feature_reweight_1,lscdfsl}. Later works~\cite{st,dynamic,hybrid_1,hybrid_4,hybrid_2} explored new setups, while~\cite{boosting,ata,data_target_1,feature_reweight_5,parameter_weight_2,confess,feature_reweight_9} focused on improving performance. Please see Section~\ref{methods} for details.}
 \vspace{-0.3cm}
	\label{imaging}
\end{figure}

The remainder of this survey is organized as follows: Section \ref{background} provides an overview of CDFSL, including its definition, challenges, and taxonomy. Section \ref{methods} covers approaches to CDFSL in detail, while Section \ref{performance} presents performance results and evaluates methods. Section \ref{future} explores future directions in set-ups, applications, and theories. Finally, Section \ref{conclusion} concludes the survey.
\section{Related Work}
\label{sec:related}

Non-rigid shape matching is a very rich and well-established research area, and a full overview is beyond
the scope of this paper.  Below, we review works that are close to our method and  refer the interested reader to recent surveys \cite{cao2020comprehensive,bronstein2017geometric,guo2016comprehensive,guo2020deep} for a more in-depth treatment.

\tightpara{Functional Maps}
Since its introduction \cite{Ovsjanikov2012}, the functional map pipeline has become a widely-used tool for non-rigid shape matching, being adopted and extended in many follow-up works \cite{ginzburg2019cyclic,Ren2019,eynard2016coupled,Melzi_2019,eynard2016coupled,Nogneng2017,rodola2017partial,poulenard_persistence,sharma2020weakly}. The advantage of this approach is that it reduces the optimization of pointwise maps, which are quadratic in the number of vertices, to optimizing small matrices, thus greatly simplifying computational problems. We refer to \cite{Ovsjanikov2017} for an overview. 

The original functional map pipeline relied on input feature (probe) functions, which were given a priori \cite{Salti2014,sun2009concise,aubry2011wave}.
% , to estimate the underlying functional maps. Most earlier works in this domain used hand-crafted features \cite{Salti2014,sun2009concise,aubry2011wave}. 
Subsequent research has improved the method by adapting it to partial shapes \cite{cosmo2016shrec,rodola2017partial,Litany2017} or using robust regularizers  %estimation pipeline by introducing robust regularizers and penalties 
\cite{Ren2019,Nogneng2017,kovnatsky2013coupled,burghard2017embedding} and proposed efficient refinement techniques \cite{Melzi_2019,Pai_2021_CVPR,jing_maptree}. %, and combined it with extrinsic shape alignment techniques and optimal transport tools \cite{aygun2020unsupervised,Eisenberger2020SmoothSM,pmf}. 
In all of these works, fmaps were computed using hand-crafted probe functions, and any information loss in these descriptors hinders downstream optimization.

More recent works have proposed to solve this problem by learning descriptors (probe functions) directly from the data, using deep neural networks. This line of research was initiated by FMNet \cite{litany2017deep}, and extended in many subsequent works \cite{halimi2019unsupervised,attaiki2023vader,roufosse2019unsupervised,sharma2020weakly,sharp2021diffusion,attaiki2021dpfm,donati2020deep,Eisenberger2021NeuroMorphUS,attaiki2022ncp,eisenberger2020deep,li2022srfeat}. These methods learn the probe features directly from the raw geometry, using a neural network, and supervise the learning with a loss on the functional map in the reduced basis.

% Litany \etal \cite{litany2017deep} proposed to refine the input SHOT descriptors \cite{Salti2014} using a pointwise MLP, and supervised the training with a loss with ground-truth pointwise map. Follow-up approaches, such as \cite{donati2020deep} improved this framework by learning the probe features directly from the raw geometry, using a neural network, and supervising the learning with a loss on the functional map in the reduced basis. Subsequent work further improved the method, making it robust to mesh discretization \cite{sharp2021diffusion} and adapting it to handle partial shapes, achieving state-of-the-art results in many non-rigid matching scenarios.

A parallel line of work has focused on making the learning \textit{unsupervised}, which can be convenient in the absence of ground truth correspondences. Approaches in \cite{halimi2019unsupervised} and \cite{Ginzburg2020} have proposed penalizing either the geodesic distortion of the pointwise map predicted by the network or using the cycle consistency loss.  %However, such approaches require the computation and storage of heavy geodesic matrices, and has shown little generalizability. 
Another line of work \cite{roufosse2019unsupervised,sharma2020weakly,sharp2021diffusion,donati-duo} considered unsupervised training by imposing structural properties on the functional maps in the reduced basis, such as bijectivity, orthonormality, and commutativity of the Laplacian. 
% Eisenberger et al. \cite{eisenberger2020deep} proposed to combine intrinsic and extrinsic alignment, in addition to a refinement of the maps in the network, at the cost of efficiency and computational time. 
The authors of \cite{sharma2020weakly} have shown that feature learning can be done starting from raw 3D geometry in the weakly supervised setting, where shapes are only approximately rigidly pre-aligned. 
% Finally, interpolation was also used a geometric prior to learning robust features in \cite{Eisenberger2021NeuroMorphUS}.

In all of these works, features extracted by neural networks have been used to formulate the optimization problem, from which a functional map is computed. Thus, %feature functions were only used algebraically, as part of an equation to solve the functional maps, and
so far no attention has been paid to the geometric nature or other utility of learned probe functions. In contrast, we aim to analyze the conditions under which probe functions can be used for direct pointwise map computation and use this analysis to design improved map estimation pipelines. 

We also note briefly a very recent work  \cite{li2022srfeat} which has advocated for imposing feature smoothness  when learning for non-rigid shape correspondence. However, that work does not use the in-network functional map estimation and moreover lacks any theoretical analysis or justification for its design choices. % their content in general. Thus, the exact nature of information learned within these 
%
%\souhaib{should we say that sharma requires the shape to be aligned, because we will be using this later}



\tightpara{Recent Advances in Axiomatic Functional Maps}
Our results are also related to recent axiomatic methods for functional map-based methods, which  \textit{couple} optimization for the functional map (fmap) with the associated point map (p2p map) \cite{ren2018continuous,Melzi_2019,Pai_2021_CVPR,Xiang_2021_CVPR}. These techniques typically propose to refine functional maps by \textit{iterating} the conversion from functional to pointwise maps and vice versa. This approach was recently summarized in \cite{discrete_Ren2021}, where the authors introduce the notion of functional map ``properness'' and describe a range of energies that can be optimized via this iterative conversion scheme. 

The common denominator between all these methods is the inclusion of pointwise maps in the process of functional map optimization. In this work, we propose a method and a loss that similarly incorporate pointwise map computation. However, we do so in a learning context and show that this leads to significant improvements in the overall accuracy of the deep functional map pipeline.



\tightpara{Learning on Surfaces}
Multiple methods for deep surface learning have been proposed to address the limitations of handcrafted features in downstream tasks. One type of method is point-based (extrinsic) methods, such as PointNet \cite{qi2017pointnet} and follow-up works \cite{qi2017pointnet++,rethink_ma_22,thomas2019KPConv,dgcnn,pcnn_2018,Wiersma2022DeltaConv}. These methods are simple, effective, and widely applicable. However, they often fail to generalize to new datasets or significant pose changes in deep shape matching. 

Another line of research \cite{poulenard2018multi,sharp2021diffusion,wiersma2020cnns,gong2019spiralnet++,masci2015geodesic,maron2017convolutional} (intrinsic methods) has focused on defining the convolution operator directly on the surface of the input shape. These methods are more suitable for deformable shapes and can leverage the structure of the surface encoded by the mesh, which is ignored by the extrinsic methods.


% To overcome the limitations of handcrafted features in downstream tasks, multiple methods for deep surface learning have been proposed. The first type of method is point-based (extrinsic) methods, pioneered by \cite{qi2017pointnet}, and extended by many works such as \cite{qi2017pointnet++,rethink_ma_22,thomas2019KPConv,dgcnn,pcnn_2018,Wiersma2022DeltaConv} to name a few. These methods are characterized by their simplicity, effectiveness, and applicability in a wide range of domains. Many of these methods have been adapted in the case of deep shape matching \cite{sharma2020weakly,donati2020deep}, and while they perform well in normal scenarios (the test dataset is similar to the training dataset), they often fail to generalize to new datasets or under significant pose changes.

% Another line of research \cite{poulenard2018multi,sharp2021diffusion,wiersma2020cnns,gong2019spiralnet++,masci2015geodesic,maron2017convolutional} (intrinsic methods) has focused on defining the convolution operator directly on the surface of the input shape. The motivation is that these methods are more suitable for deformable shapes, and can benefit from the structure of the surface encoded by the mesh, which is ignored by the first category of methods.

In previous works, intrinsic methods tend to perform better for non-rigid shape matching than extrinsic methods, with DiffusionNet \cite{sharp2021diffusion} being considered the state-of-the-art feature extractor for shape matching. In this work, we will show that a simple modification to extrinsic feature extractors improves their overall performance for shape matching, making them comparable to or better than DiffusionNet. 

\paragraph{Paper Organization}
The rest of the paper is structured as follows. In \cref{sec:background}, we first introduce the deep functional map pipeline and our key definitions. We then state a theorem that shows that features learned in the deep functional map pipeline have a geometric interpretation under certain conditions. In light of this discovery, we introduce in \cref{sec:method} two modifications to the functional map pipeline, by promoting structural properties suggested by our analysis. Finally, in \cref{sec:application}, we will show how these modifications improve overall performance in different shape matching scenarios.
\section{Notation, Background \& Motivation}
\label{sec:background}

\subsection{Notation}
\label{sec:notation}

Suppose we are given a pair of shapes $S_1, S_2$ represented as triangle meshes with respectively $n_1$ and $n_2$ vertices. We compute the cotangent Laplace-Beltrami operator \cite{Meyer2003} of each shape $S_i$ and collect the first k eigenfunctions as columns in a matrix denoted by $\Phi_i$, and the corresponding eigenvalues in a diagonal matrix denoted as $\Delta_i$. $\Phi_i$ is orthonormal with respect to the area (mass) matrix $\Phi_i^{T} A_i \Phi_i = \mathbb{I}$. We denote $\Phi_i^{\dagger} = \Phi_i^{T} A_i$, where $\cdot^{\dagger}$ is the (left) Moore–Penrose pseudo-inverse.

A pointwise map $T_{12}: S_1 \rightarrow S_2$ can be encoded in a matrix $\Pi_{12} \in \mathbb{R}^{n_1 \times n_2},$ where $\Pi_{12}(i,j) = 1$ if $T_{12}(i) = j$ and 0 otherwise. We will use the letter $\Pi$ to denote pointwise maps in general. To every pointwise map, one can associate a functional map, simply via projection: $\C_{21} = \Phi^{\dagger}_1 \Pi_{12} \Phi_{2}$.

Let $\mathcal{F}_{\Theta}$ be some feature extractor, which takes as input a shape and produces a set of feature functions (where $\Theta$ are some trainable parameters). We then have $F_1 = \mathcal{F}_{\Theta}(S_1)$. For simplicity we will omit explicitly stating $\mathcal{F}_{\Theta}$ and just denote the features associated with a shape $S_i$ by $F_i$, where the presence of some feature extractor is assumed implicitly. Finally, $\A_i = \Phi_i^{\dagger} F_i$ denotes the matrix of coefficients of the feature functions in the corresponding reduced basis.












\subsection{Deep Functional Map Pipeline}

The standard Deep Functional Map pipeline can be described as follows.
     \tightpara{Training:} Pick a pair of shapes $S_1, S_2$ from some training set $\{S_i\}$, and compute their feature functions $F_1 = \mathcal{F}_{\Theta}(S_1), F_2 = \mathcal{F}_{\Theta}(S_2)$. Let $\A_{1} = \Phi_1^{\dagger} F_1$, and $\A_{2} = \Phi_2^{\dagger} F_2$ denote the coefficients of the feature functions in the reduced basis.
       Compute the functional map $\C_{12}$ by solving a least squares system inside the network:
        \begin{align}
            \argmin_{\C} \| \C \A_1 - \A_2 \|_F^2.
            \label{eq:fmap_basic}
        \end{align}
    This system can be further regularized by incorporating commutativity with the Laplacian \cite{donati2020deep}. A \emph{training loss} is then imposed on this computed functional map (in the supervised setting) by comparing the computed functional map $\C_{12}$ with some ground truth:
        \begin{align}
            \mathcal{L}_{sup}(\C_{12}) = \| \C_{12} - \C_{\rm{gt}} \|_F^2. \label{eq:sup_loss}
        \end{align}
        Alternatively, (in the unsupervised setting) we can impose a loss on $\C_{12}$ by penalizing its deviation from some desirable properties, such as being an orthonormal matrix \cite{roufosse2019unsupervised}. Finally, the loss above is used to back-propagate through the feature extractor network $\mathcal{F}_{\Theta}$ and optimize its parameters $\Theta$.


   \tightpara{Test time:} Once the parameters of the feature extractor network are trained, we follow a similar pipeline at test time. Given a pair of \emph{test shapes} $S_1, S_2$, we compute their feature functions $F_1 = \mathcal{F}_{\Theta}(S_1), F_2 = \mathcal{F}_{\Theta}(S_2)$. Let $\A_{1} = \Phi_1^{\dagger} F_1$, and $\A_{2} = \Phi_2^{\dagger} F_2$ denote their coefficients in the reduced basis. We then compute the functional map $\C_{12}$ by solving the least squares system in \cref{eq:fmap_basic}. Given this functional map, a point-to-point (p2p) correspondence $\Pi_{21}$ can be obtained by solving the following problem:
        \begin{align}
            \Pi_{21} = \argmin_{\Pi} \| \Pi \Phi_1 - \Phi_2 \C_{12}\|.
            \label{eq:fmap_conversion}
        \end{align}
        This problem, which was proposed and justified in \cite{ezuz2017deblurring,Pai_2021_CVPR} is row-separable and reduces to the nearest neighbor search between the rows of $\Phi_1$ and the rows of $\Phi_2 \C_{12}$. This method of converting the fmap into a p2p map is referred to hereafter as the \textit{adjoint method} following \cite{Pai_2021_CVPR}.
  
\vspace{-0.7em}
\paragraph{Advantages}
%\souhaib{will be removed}
This pipeline has several advantages: first, the training loss is imposed on the entire map rather than individual point correspondences, which has been shown to improve accuracy \cite{litany2017deep,donati2020deep}. Second, by using a reduced spectral basis, this approach strongly regularizes the learning problem and thus can be used even in the presence of limited training data \cite{donati2020deep,sharp2021diffusion}. Finally, by accommodating both supervised and unsupervised losses, this pipeline allows to promote structural map properties without manipulating large and dense, e.g., geodesic distance, matrices.

%One advantage of this pipeline is that during training all computations are in the reduced basis. For example, there is no in-network computation of a soft correspondence matrix, which could require manipulating $O(n^2)$ in GPU memory. Thus, this standard pipeline is reasonably scalable (although it does require storing $k \times n$ basis matrices $\Phi_i$). 

\vspace{-0.7em}
\paragraph{Drawbacks \& Motivation}
% One possible weakness is that this pipeline relies on the Laplacian basis both during training and at test time. Furthermore, 
At the same time, conceptually, the relation between the learned feature functions and the computed correspondences is still unclear. Indeed, although the feature (probe) functions are learned, they are used solely to formulate the optimization problem in \cref{eq:fmap_basic}. Moreover, the final pointwise  correspondence in \cref{eq:fmap_conversion} is still computed from the Laplacian basis. Thus, it is not entirely clear what exactly is learned by the network $\mathcal{F}_\Theta$, and whether the learned feature functions can be used for any other task. Indeed, in practice, learned features can fail to highlight some specific repeatable regions on shapes, even if they tend to produce high-quality functional map matrices (\cref{fig:teaser}).











\subsection{Theoretical Analysis}
\label{sec:motivation}
In this subsection, we introduce some notions and state a theorem that will be helpful in our results below. We will use the same notation as in \cref{sec:notation}.

First, note that both during training (in \cref{eq:fmap_basic}) and at test time (in  \cref{eq:fmap_conversion}) the learned feature functions $F_i$, are used by projecting them onto the Laplacian basis. Thus, both from the perspective of the loss and also when computing the pointwise map, the deep functional map pipeline only uses the part of the functions that lies in the span of the Laplacian basis. Put differently, if we \textit{modified} feature functions by setting $\tilde{F}_i  = \Phi_i \Phi_i^{\dagger} F_i$ then both during training and at test time the behavior of the deep functional map pipeline will be identical when using either $F_i$ or $\tilde{F}_i$.

\begin{definition}[Complete feature functions]
Motivated by this observation, we call the feature extractor $F_{\Theta}$ \textbf{complete} if $F_{\Theta}$ produces feature (probe) functions that are contained in the corresponding LB eigenbasis. I.e.,
\begin{align}
	&\mathcal{F}_\Theta (S_i) = \Phi_i \Phi_i^{\dagger} \mathcal{F}_\Theta (S_i) , \text{ or equivalently} \\
	&(Id - \Phi_i \Phi_i^{\dagger}) \mathcal{F}_\Theta (S_i)  = 0.
\end{align}
\end{definition}
%
%if it satisfies two conditions: first, the descriptor for every point is unique (i.e., feature extractor is injective) and, second, $F_{\Theta}$ produces feature (probe) functions, that are contained in the corresponding Laplace-Beltrami eigenbasis. I.e.,
%    \begin{align}
%        &\mathcal{F}_\Theta (S_i) = \Phi_i \Phi_i^{+} \mathcal{F}_\Theta (S_i) , \text{ or equivalently} \\
%        &(Id - \Phi_i \Phi_i^{+}) \mathcal{F}_\Theta (S_i)  = 0.
%    \end{align}
%\end{definition}
%
%It should be noted that an injective feature extractor can be made complete by simply projecting its feature functions onto the Laplacian basis, i.e., by setting: $\tilde{\mathcal{F}_\Theta}(S_i) = \Phi_i \Phi_i^{+} \mathcal{F}_\Theta (S_i),$ assuming this procedure does not break injectivity.

% \begin{definition}[Alignable Laplacian bases]
% Given a pair of shapes, $S_1, S_2$ we will call their Laplacian eigenbases $\Phi_1, \Phi_2$ \textit{perfectly alignable} if there exists a functional map $\C_{12}$ and a point-to-point map $\Pi_{21}$ s.t. $\Pi_{21} \Phi_{1} = \Phi_{2} \C_{12}.$
% \end{definition}

% It is worth noting that that for sufficiently high dimensionality $k$, any pair of Laplacian bases is perfectly alignable. This is because if two shapes have $n$ vertices each then, since $\Phi_{2}$ is guaranteed to be invertible in the full basis, there is a unique $n^2$ matrix $\C_{12}$ that will satisfy $\Pi_{21} \Phi_{1} = \Phi_{2} \C_{12}$ exactly.


\begin{definition}[Basis aligning functional maps] We also introduce a notion that relates to the properties of functional maps. Namely, given a pair of shapes, $S_1, S_2$ and a functional map $\C_{12}$, we will call it \textit{basis-aligning} if the bases, when transformed by this functional map, align exactly. This can be summarized as follows: $\Phi_{2} \C_{12} = \Pi_{21} \Phi_{1}$ for some point-to-point map $\Pi_{21}$.
\end{definition}

For simplicity, for our result below we will also assume that all optimization problems have unique global minima. Thus, for the problem $\argmin_{\C} \|\C \A_{1} - \A_{2} \|$, this means that $\A_1$ must be full rank, whereas for the problem of type $\argmin_{\Pi} \| \Pi F_1 - F_2 \|$, this means that the rows of $F_1$ must be distinct (i.e., no two rows are identical, as vectors).

% \begin{theorem}
%     Suppose that the Laplacian eigenbasis is perfectly alignable, \ie, $\exists (\C_{12}, \Pi_{21}): \Pi_{21} \Phi_{1} = \Phi_{2} \C_{12}$. Suppose moreover that the feature extractor $F_{\Theta}$ is \emph{complete}, and that $\C_{12} \A_1 = \A_2$ is satisfied exactly.

% Then the point-to-point map computed by nearest neighbor in descriptor space $\min_{\Pi} \| \Pi F_1 - F_2 \|$ will recover the exact same map as $\Pi_{21}$.

% Conversely, suppose that the feature extractor is \emph{complete} and  $\C_{12} \A_1 = \A_2$ is satisfied exactly, if the rows of $\A_1$ are linearly independent (which is required for the linear system $X \A_1 = \A_2$ to be solvable), then $\Pi_{21} F_1 = F_2$ implies that the Laplacian eigenbasis is perfectly alignable and $\min_{\Pi} \| \Pi \Phi_1 - \Phi_2 \C \|$ will recover the same map $\Pi_{21}$.
% \end{theorem}

\begin{theorem}
\label{thm:equivalence}
    Suppose the feature extractor $F_{\Theta}$ is \emph{complete}. Let $\A_1 = \Phi_1^{\dagger} F_1$ and $\A_2 = \Phi_2^{\dagger} F_2$. Then, denoting $\C_{\rm{opt}} = \argmin_{\C} \|\C \A_{1} - \A_{2} \|$, we have the following results hold:
    
(1) If $\Pi F_1 = F_2$ for some point-to-point map $\Pi$ then $\C_{12} = \Phi_2^{\dagger} \Pi \Phi_1$ is basis-aligning. Moreover, $\C_{12} = \C_{\opt}$, and extracting the pointwise map from $\C_{\opt}$ via the adjoint method (see \cref{eq:fmap_conversion}) or via nearest neighbor search in the feature space $\min_{\Pi} \| \Pi F_1 - F_2 \|$ will give the same result.

(2) Conversely, suppose that $F_1, F_2$ are complete and $\C_{\opt}$ is basis aligning, then $\argmin_{\Pi} \| \Pi F_1 - F_2 \| = \argmin_{\Pi} \| \Pi \Phi_1 - \Phi_2 C_{\opt}\|$.
\end{theorem}

\begin{proof}
    See the supplementary materials.
\end{proof}

\tightpara{Discussion:} 
Note that in the theorem above, we used the notion of basis-aligning functional maps. Two questions arise: what are the necessary and sufficient conditions for a functional map to be basis-aligning? 

A necessary condition is that the functional map must be \textit{proper} as defined in \cite{ren2021discrete}. Namely, a functional map $\C_{12}$ is \textit{\textbf{proper}} if there exists a pointwise map $\Pi_{21}$ such that  $\C_{12} = \Phi_{2}^{\dagger} \Pi_{21} \Phi_{1}$.  Note that for sufficiently high dimensionality $k$, \textit{any} proper functional map must be basis aligning.  This is because, since $\Phi_{2}$ is guaranteed to be invertible in the full basis, there is a unique matrix $\C_{12}$ that will satisfy $\Pi_{21} \Phi_{1} = \Phi_{2} \C_{12}$ exactly.

Conversely, a \textit{sufficient} condition for a functional map to be basis aligning is that the underlying point-to-point map is an isometry. Indeed, it is well known that isometries must be represented as block-diagonal functional maps, and moreover, isometries preserve eigenfunctions (see Theorem 2.4.1 in \cite{Ovsjanikov2017} and Theorem 1 in \cite{rustamov2013map}). Thus, assuming that the functional map is of size $k$ and that the $k^{\text{th}}$ and $(k+1)^{\text{st}}$ eigenfunctions are distinct, we must have that the functional map must be basis-aligning.

Finally, note that a functional map is basis-aligning if it is \textit{proper} and if the image (by pull-back) of the $k$ first eigenfunctions of the source lies in the range of the $k$ first eigenfunctions of the target. This can therefore be considered as a measure of the \textit{smoothness} of the map.

%
%
%It is worth noting that that , for any pair of Laplacian bases, there is a basis aligning fmap. This is because if two shapes have $n$ vertices each then, since $\Phi_{2}$ is guaranteed to be invertible in the full basis, there is a unique $n^2$ matrix $\C_{12}$ that will satisfy $\Pi_{21} \Phi_{1} = \Phi_{2} \C_{12}$ exactly.
%
%Finally, another notion that will be helpful is the fmap properness, introduced in \cite{discrete_Ren2021}.

\section{Proposed Modification}
\label{sec:method}

In the previous section, we provided a theoretical analysis of the conditions under which computed ``probe'' functions within the deep functional map pipeline can be used as pointwise descriptors directly, and lead to the same point-to-point maps as computed by the functional maps. In \cref{sec:application}, we provide an extensive evaluation of using learned feature functions for pointwise map computation and thus affirm the validity of \cref{thm:equivalence} in practice.

% \maks{is this paragraph necessary?} 
Our main observation is that the two approaches for point-to-point map computation are indeed often equivalent in practice, especially in ``easy'' cases, where existing state-of-the-art approaches lead to highly accurate maps. In contrast, we found that in more challenging cases, where existing methods fail, the two approaches are not equivalent and can lead to significantly different results. 

% Inspired by our analysis above, we thus propose to use the structural properties suggested in \cref{thm:equivalence} as a way to bridge this gap and, possibly improve the overall accuracy. As we demonstrate in \cref{sec:application}, our proposed modifications while being relatively simple, significantly improve the quality of the computed correspondences, especially in ``difficult'' matching scenarios.

Motivated by our analysis, we propose to use the structural properties suggested in \cref{thm:equivalence} as a way to bridge this gap and improve the overall accuracy. Our proposed modifications are relatively simple, but they significantly improve the quality of computed correspondences, especially in ``difficult'' matching scenarios, as we demonstrate in \cref{sec:application}.

% The two key assumptions in \cref{thm:equivalence} are \textit{basis-aligning} functional maps and \textit{complete} feature extractors. We thus propose to modify the functional map pipeline so that the conditions of the theorem are satisfied. As mentioned above, the basis-aligning property is closely related to \textit{properness} and thus we propose to approach it by imposing that the predicted functional map arises from some pointwise correspondence. For feature completeness, we propose a simple modification of the feature extractor so that it produces \textit{smooth features}. In what follows, we will use the same notation as \cref{sec:notation}.

The two key assumptions in \cref{thm:equivalence} are \textit{basis-aligning} functional maps and \textit{complete} feature extractors. We propose modifying the functional map pipeline to satisfy the conditions of the theorem. Since the basis-aligning property is closely related to \textit{properness}, we propose to impose that the predicted functional map to be proper, \ie arises from some pointwise correspondence. For feature completeness, we suggest modifying the feature extractor to produce \textit{smooth features}. We use the same notation as in \cref{sec:notation}.


\subsection{Enforcing Properness}
\label{sec:proper_fmap}

In this section, we propose two ways to enforce functional map properness and associated losses  for both supervised and unsupervised training.

\paragraph{The adjoint method}
Given feature functions $F_1, F_2$, produced by a feature extractor, we compute the functional map $\C_{12-pred}$ as explained in \cref{sec:background}. To compute a proper functional from it, we first convert $\C_{12-pred}$ into a p2p map $\Pi_{21-pred}$ in a differentiable way and then compute the ``differentiable'' proper functional map $\C_{21-proper} = \Phi_{2}^{\dagger} \Pi_{21-pred} \Phi_{1}$. 

To compute $\Pi_{21-pred}$, denoting $G_1 = \Phi_{1}$ and $G_2 = \Phi_{2}\C_{21-pred}$, we use:
\begin{align}
& \Pi_{21-pred}^{i, j} = \dfrac{\exp\big(\langle G_2^{i}, G_1^{j}\rangle / \tau\big)}{\sum_{k=1}^{n_1}\exp\big(\langle G_2^{i}, G_1^{k} \rangle / \tau\big)}.\label{eq:diff_p2p}%\\
%&s(\mathbf{x}, \mathbf{y}) = \mathbf{x} \cdot \mathbf{y}.\label{eq:FeatureDistanceFunc}
\end{align}
%
Here $\langle \cdot,\cdot \rangle$ is the scalar product measuring the similarity between $G_1$ and $G_2$, and $\tau$ is a temperature hyper-parameter. $\Pi_{21-pred}$ can be seen as a soft point-to-point map, formulated based on the adjoint conversion method described in \cite{Pai_2021_CVPR}, and computed in a differentiable manner, hence it can be used inside a neural network.

\paragraph{The feature-based method}
The feature-based method is similar to the adjoint method in spirit, the only difference being that $\Pi_{21-pred}$ is computed using the predicted features instead of the fmap. For this, we use \cref{eq:diff_p2p}, with $G_1 = F_1$ and $G_2 = F_2$. The modified deep functional map pipeline is illustrated in \cref{fig:fmap-pipeline}.

\begin{figure}
    \centering
    \includegraphics[width=\columnwidth]{figures/fmap_pipeline.pdf}
    \caption{An overview of our revised deep functional map pipeline. The extracted features are used to compute the functional map and the proper functional map, as explained in \cref{sec:proper_fmap}}
    \label{fig:fmap-pipeline}
    \vspace{-1em}
\end{figure}

In addition to $C_{12-pred}$, the two previous methods allow to calculate $C_{21-proper}$. We adapt the functional map losses to take into account this modification.

In the supervised case, we modify the supervised loss (see \cref{eq:sup_loss}) by simply introducing an additional term into the loss: 
\begin{align}
\mathcal{L}_{proper} = \| \C_{12-pred} - \C_{12-proper} \|_F^2 \label{eq:sup_loss_proper}.
\end{align}

The motivation behind this loss is that we want the predicted functional map to be as close as possible to the ground truth and stay within the space of proper functional maps.% a proper one , the gradient is strong enough to force the features to produce a  using \cref{eq:fmap_basic}, that is close to a proper one.

In the unsupervised setting, we simply impose the standard unsupervised losses on the differentiable proper functional map $\C_{12-proper}$ instead of $\C_{12-pred}$. Specifically, in our experiments below, we use the following unsupervised losses: 
%
\begin{align}
\nonumber
\mathcal{L}_{unsup}(\C_{12}, \C_{21}) &= \| \C_{12} \C_{21} - \mathbb{I} \|_F^2 + \| \C_{21} \C_{12} - \mathbb{I} \|_F^2 \\
& + \| \C_{12}^{\top} \C_{12} - \mathbb{I} \|_F^2 + \| \C_{21}^{\top} \C_{21} - \mathbb{I} \|_F^2
\label{eq:unsup_loss_proper}
\end{align}

%\maks{add the losses here.}


% \souhaib{how to justify that the results obtained with NN are better than fmap}












%%%%%%%%%%%%%%%%%%%%%%%%%%%%%%%%%%%%%%%%%%%%%%%%%%%%
%%%%%%%%%%%%%%%%%%%%%%%%%%%%%%%%%%%%%%%%%%%%%%%%%%%%%%%%%%%%%%%%%%%%%%%%%%%%%%%%%%%%%%%%%%%%%%%%%%%%%%%%
%%%%%%%%%%%%%%%%%%%%%%%%%%%%%%%%%%%%%%%%%%%%%%%%%%%%%%%%%%%%%%%%%%%%%%%%%%%%%%%%%%%%%%%%%%%%%%%%%%%%%%%%
%%%%%%%%%%%%%%%%%%%%%%%%%%%%%%%%%%%%%%%%%%%%%%%%%%%%
%%%%%%%%%%%%%%%%%%%%%%%%%%%%%%%%%%%%%%%%%%%%%%%%%%%%

\subsection{As Smooth As Possible Feature Extractor}
Another fundamental assumption of \cref{thm:equivalence} is the completeness of the features produced by a neural network. 

We have experimented with several ways to impose it and have found that it is not easy to satisfy it exactly in general because it would require the network to always produce features in some target  subspace, which is not explicitly specified in advance. Moreover, we have found that explicitly projecting feature functions to a small reduced subspace can also hinder learning. 

To circumvent this, we propose instead to \textit{encourage} this property by promoting the feature extractor to produce smooth features. 

The motivation for this is as follows. If $F_i$ is complete, then there exist coefficients $a_1 ... a_k$ such that $F_i = \sum_{j=1}^k a_j \Phi_i^j$, where $k$ is the size of the functional map used in \cref{eq:fmap_basic}.
However, it's known that Fourier coefficients for smooth functions decay rapidly (faster than any polynomial, if $f$ is of class $C^l$, the coefficients are $o(n^{-l})$), which means that the smoother the function is, the closer it will be to being complete for some index $k$.

Inspired by this, we propose the following simple modification to feature extractors used for deep functional maps. Since feature extractors are made of multiple layers, we propose to project the output of each layer into the Laplacian basis, diffuse it over the surface following \cite{sharp2021diffusion}, and then project it back to the ambient space before feeding it to the next layer, see \cref{fig:feat-extract-modif}. Concretely, for shape $S$, if $f_i$ is the output of layer $i$, we feed to layer $i+1$ the function $f^{'}_i$, such that $f^{'}_i = \Phi_j e^{-t \Delta} \Phi_j^{\dagger} f_i$, where $\Phi_j$ denotes the first $j$ eigenfunctions of the Laplacian, $\Delta$ is a diagonal matrix containing the first j eigenvalues, and $t$ is a learnable parameter. Please note there is no need to do this operation for the final layer, since the features will be projected into the Laplacian basis anyway, for computing the functional map. In practice, we observed that it is beneficial to set $j$ to \textit{be larger} than the size of the functional maps in \cref{eq:fmap_basic}. This allows the network to impose smoothness, while still allowing degrees of freedom to enable optimization.

%Also note, that DiffusionNet \cite{sharp2021diffusion} does this operation by construction for each layer, which can in part explain its success.
%
%\souhaib{what about the receiptive field}


%
%
% - smooth the input before feeding them to the network (doesn't work practically)
% 
% - smooth the features at the end of each layer
% 
% - for better results, increase the receiptive field of the features using diffusion 
%
%
\vspace{-1em}

\paragraph{Implementation details} we provide implementation details, for all our experiments, in the supplementary. Our code and data will be released after publication.

\begin{figure}
    \centering
    \includegraphics[width=\columnwidth]{figures/feature_extractor.pdf}
    \caption{An overview of the feature extractor modification is shown here. The features are made smooth by projecting them into the Laplacian basis at the end of each layer.}
    \label{fig:feat-extract-modif}
    % \vspace{-1.5em}
\end{figure}
\section{Results \& Applications}
\label{sec:application}

In this section, we provide results in a wide range of challenging tasks, showing the efficiency and
applicability of our approach to different types of data and feature extractors.% In particular, we show in \maks{incomplete sentence?}

\paragraph{Datasets} 
Our method is evaluated on five datasets commonly used in literature, which include both human and animal datasets.

To evaluate our method's performance for shape matching with humans, we use three datasets: FAUST \cite{Bogo2014}, SCAPE \cite{Anguelov2005}, and SHREC'19 \cite{shrec19}. We utilize the remeshed versions of the first two datasets introduced in \cite{Ren2019} and used in various follow-up works such as \cite{donati2020deep,sharma2020weakly,sharp2021diffusion,eisenberger2020deep,Eisenberger2021NeuroMorphUS,Eisenberger2020SmoothSM}. We follow the same train/test splits as in prior works.
% The FAUST Remeshed (\textbf{FR}) dataset consists of 100 human meshes, divided into 80/20 train/test splits as in prior works. Similarly, the SCAPE Remeshed (\textbf{SR}) dataset comprises 71 human meshes split into 51/20 shapes for training/testing. We also use the more challenging SHREC'19 dataset (SH) consisting of 44 human shapes with variations in pose, identity, and connectivity. We use its remeshed version and split it into 440 pairs exclusively used for testing.
For unsupervised experiments, we utilize the datasets' oriented versions, as described in \cite{sharma2020weakly}, denoted as \textbf{FA}, \textbf{SA}, and \textbf{SHA} for FAUST remeshed aligned, SCAPE remeshed aligned, and SHREC aligned, respectively.

We also evaluate our method on human segmentation using the dataset introduced in \cite{maron2017convolutional}, which comprises segmented human models from various prior datasets. We use the same test split as in prior works.% but only use \textbf{one shape} for training, which constitutes 0.3\% of the total training set.

For animal datasets, we use the SMAL-based dataset \cite{Zuffi:CVPR:2017,marin22_why} (denoted as \textbf{SMAL} hereafter), which consists of 50 organic, non-isometric, and non-rigid shapes represented as 3D meshes. We divide them into 25/25 shapes for training and testing, and to test the generalization capacity of our models, the animals and their positions during testing are never seen during training.









% We test our method on five human and animal datasets, widely used in the literature. 

% Concerning the human dataset for shape matching, we test our method on FAUST \cite{Bogo2014}, SCAPE \cite{Anguelov2005} and SHREC'19 \cite{shrec19} datasets. For the first two datasets, we use their remeshed version, introduced in \cite{Ren2019}, and used in many follow-ups works such as \cite{donati2020deep,sharma2020weakly,sharp2021diffusion,eisenberger2020deep,Eisenberger2021NeuroMorphUS,Eisenberger2020SmoothSM} to name a few. FAUST Remeshed (\textbf{FR}) dataset is composed of 100 human meshes in different positions. Following prior work \cite{donati2020deep,eisenberger2020deep}, we divide the dataset into train/test splits using 80/20 shapes respectively. SCAPE Remeshed (\textbf{SR}) dataset is composed of 71 human meshes, split into 51/20 shapes for training/testing. Finally, we use the more recent and more challenging \textbf{SHREC'19} dataset \textbf{SH} \cite{shrec19, Ren2019}. This dataset is composed of 44 human shapes with great changes in pose, identity, and connectivity. This resulted in 440 pairs used exclusively for testing. In addition to these datasets, we use their oriented versions for the unsupervised experiments, following the protocol of \cite{sharma2020weakly}. In particular, the datasets are oriented in a way  that they have a consistent ``up'' direction (along, \eg, the y-axis) and an approximate forward-facing direction (along, \eg, the
% z-direction). The resulting datasets are denoted \textbf{FA}, \textbf{SA}, \textbf{SHA} for FAUST remeshed aligned, SCAPE remeshed aligned, and SHREC aligned respectively.

% In addition to the above human datasets used for matching, we also test our method on the scenario of human segmentation using the dataset introduced in \cite{maron2017convolutional}, which combines segmented human models taken from a variety of prior datasets. We use the same test split as in prior works \cite{maron2017convolutional,sharp2021diffusion}, however, as mentioned below, we only use \textbf{one shape} for training, which constitutes 0.3\% of the total training set.

% For the animal dataset, we use the organic, non-isometric non-rigid SMAL-based dataset \cite{Zuffi:CVPR:2017,marin22_why}. This dataset consists of 50 shapes represented as 3D meshes, divided into 25/25 shapes for training and testing. To test the generalization capacity of the different models, the animals and the position seen during the tests are never seen during the training.


%%%%%%%%%%%%%%%%%%%%%%%%%%%%%%%%%%%%%%%%%%%%%%%%%%%%%%%%%%%%%%
%%%%%%%%%%%%%%%%%%%%%%%%%%%%%%%%%%%%%%%%%%%%%%%%%%%%%%%%%%%%%%
%%%%%%%%%%%%%%%%%%%%%%%%%%%%%%%%%%%%%%%%%%%%%%%%%%%%%%%%%%%%%%
%%%%%%%%%%%%%%%%%%%%%%%%%%%%%%%%%%%%%%%%%%%%%%%%%%%%%%%%%%%%%%
%%%%%%%%%%%%%%%%%%%%%%%%%%%%%%%%%%%%%%%%%%%%%%%%%%%%%%%%%%%%%%









%%%%%%%%%%%%%%%%%%%%%%%%%%%%%%%%%%%%%%%%%%%%%%%%%%%%%%%%%%%%%%
%%%%%%%%%%%%%%%%%%%%%%%%%%%%%%%%%%%%%%%%%%%%%%%%%%%%%%%%%%%%%%
%%%%%%%%%%%%%%%%%%%%%%%%%%%%%%%%%%%%%%%%%%%%%%%%%%%%%%%%%%%%%%
%%%%%%%%%%%%%%%%%%%%%%%%%%%%%%%%%%%%%%%%%%%%%%%%%%%%%%%%%%%%%%
%%%%%%%%%%%%%%%%%%%%%%%%%%%%%%%%%%%%%%%%%%%%%%%%%%%%%%%%%%%%%%


\subsection{Non-Rigid Shape Matching}
In this section, we evaluate our modification to the deep functional map pipeline in three different and challenging settings, including near-isometric supervised matching in \cref{sec:supervised-human}, near-isometric unsupervised shape matching in \cref{sec:unsup-human}, and finally, non-isometric non-rigid matching, both in supervised and unsupervised settings, in \cref{sec:animals-matching}.

We evaluate our proposed modifications in the presence of three different feature extractors, an intrinsic feature extractor DiffusionNet \cite{sharp2021diffusion} which is considered as the state of the art for shape matching, and two extrinsic feature extractors, DGCNN \cite{dgcnn} and DeltaConv \cite{Wiersma2022DeltaConv}, which operate on point clouds. We will show that with our modifications, the latter feature extractors can surpass DiffusionNet in some scenarios.

% \paragraph{Evaluation protocol} We follow the protocol of \cite{Kim2011} to evaluate all correspondences, used in all recent works. Namely, we measure the geodesic error of predicted maps with respect to the given ground truth and normalize the scale by the square root of the total surface area. All the values are multiplied by $\times 100$ for clarity.

\paragraph{Evaluation protocol} Our evaluation protocol, following \cite{Kim2011}, measures the geodesic error between predicted maps and the given ground truth, normalized by the square root of the total surface area. All reported values are multiplied by $\times 100$ for clarity.

% In all experiments below, The notation ``X on Y'' means that we train on X and test on Y. Also, we will use \textbf{FM} to denote the point-to-point map extracted by converting the functional map, and \textbf{NN} to denote the point-to-point map extracted by the nearest neighbor in the feature space. Finally, the notation ``Feature Extractor - Ours'', means for the case of DGCNN and DeltaConv, that we train the network with both the smoothness block as well as the properness enforced during training, meanwhile, for DiffusionNet, we only enforce the properness, since the smoothness is enforced by construction. For properness, we will report results only for the feature-based method, and we will include results using the adjoint method in the supplementary.

In our experiments, we use the notation ``X on Y'' to indicate training on X and testing on Y. We use \textbf{FM} to represent the point-to-point map derived from the functional map and \textbf{NN} for the point-to-point map extracted by nearest neighbor in the feature space. We use ``Feature Extractor - Ours'' to denote our training approach for DGCNN and DeltaConv, which involves training the network with both smoothness and properness enforced, while for DiffusionNet, we only enforce properness since smoothness is enforced by construction. We report properness results only for the feature-based method and include adjoint method results in the supplementary materials.

%\souhaib{properness enforced == using loss from section above}


%%%%%%%%%%%%%%%%%%%%%%%%%%%%%%%%%%%%%%%%%%%%%%%%%%%%%%%%%%%%%%
%%%%%%%%%%%%%%%%%%%%%%%%%%%%%%%%%%%%%%%%%%%%%%%%%%%%%%%%%%%%%%
%%%%%%%%%%%%%%%%%%%%%%%%%%%%%%%%%%%%%%%%%%%%%%%%%%%%%%%%%%%%%%
%%%%%%%%%%%%%%%%%%%%%%%%%%%%%%%%%%%%%%%%%%%%%%%%%%%%%%%%%%%%%%
%%%%%%%%%%%%%%%%%%%%%%%%%%%%%%%%%%%%%%%%%%%%%%%%%%%%%%%%%%%%%%
\subsubsection{Supervised Shape Matching}
\label{sec:supervised-human}

\begin{table}[!t]
    \centering
     \ra{1.0}
          \resizebox{\columnwidth}{!}{%
               \begin{tabular}{@{}l cc cc cc cc cc cc@{}}

\rowcolor{Gray!50}
\textbf{Model / Dataset} & \multicolumn{2}{c}{\textbf{FR on FR}} & \multicolumn{2}{c}{\textbf{SR on SR}} & \multicolumn{2}{c}{\textbf{FR on SR}} & \multicolumn{2}{c}{\textbf{SR on FR}} & \multicolumn{2}{c}{\textbf{FR on SH}} & \multicolumn{2}{c}{\textbf{SR on SH}}\\

& FM & NN & FM & NN & FM & NN & FM & NN & FM & NN & FM & NN\\

\midrule
\rowcolor{JungleGreen!80}
DiffusionNet & 2.6 & 2.2 & \textbf{2.9} & 2.7 & \textbf{3.4} & \textbf{3.1} & 2.9 & 3.2 & 9.6 & 7.6 & 6.9 & 9.2 \\
\rowcolor{JungleGreen!80}
DiffusionNet - Ours & \textbf{2.6} & \textbf{2.0} & \textbf{2.9} & \textbf{2.4} & \textbf{3.4} & 3.2 & \textbf{2.7} & \textbf{2.3} & \textbf{5.7} & \textbf{5.7} & \textbf{5.0} & \textbf{5.6} \\

%\cmidrule[\heavyrulewidth]{1-13}
% \midrule
\addlinespace

\rowcolor{Apricot!80}
DGCNN & 2.9 & 6.0 & 3.6 & 8.0 & 13.6 & 16.8 & 3.5 & 10.2 & 11.5 & 17.1 & 10.4 & 19.5 \\
\rowcolor{Apricot!80}
DGCNN - Ours & \textbf{2.6} & 2.6 & 3.1 & 3.3 & 4.1 & 4.8 & 3.4 & 4.6 & 5.8 & 6.0 & 13.8 & 15.2\\

% \midrule
\addlinespace

\rowcolor{Salmon!80}
DeltaConv & 2.7 & 13.8 & 4.4 & 21.2 & 18.1 & 26.6 & 12.2 & 24.4 & 19.4 & 32.6 & 27.7 & 34.9\\
\rowcolor{Salmon!80}
DeltaConv - Ours & \textbf{2.6} & 2.5 & \textbf{2.9} & 2.9 & 3.6 & 4.2 & 5.8 & 6.0 & 10.7 & 10.0 & 22.1 & 21.1\\

\bottomrule


               \end{tabular}
          }
          \caption{Mean geodesic error of various feature extractors for supervised shape matching.}
          \label{tab:supervised_humans}
     % \vspace{-2.5em}
\end{table}

In this section, we test the quality of our proposed modifications in the supervised setting. For this, we follow the same pipeline described in \cref{sec:background}, and train our network with the supervised loss in \cref{eq:sup_loss_proper}. Also, to stress the generalization power of each network, we do multiple tests, by training on one dataset, and testing on another one, following multiple previous works \cite{donati2020deep,sharma2020weakly,eisenberger2020deep}. For this, we use the \textbf{FR}, \textbf{SR}, and \textbf{SH} datasets. The results are summarized in \cref{tab:supervised_humans}.

First of all, we can see that our approach improves the global result for all feature extractors, on all datasets. The performance increase can be up to 80\%, as is the case for DeltaConv and DGCNN on the scenario \textbf{FR} on \textbf{SR}, where \eg the error has decreased from 18.1 to 3.6.

Another interesting result to note is that using our modifications, the extrinsic feature extractors, which were failing in the generalization scenarios, now perform as well as DiffusionNet.
%
Finally, and perhaps more \textit{importantly}, after our modifications, the results obtained with the nearest neighbor are much better, and get closer to the maps extracted by converting the functional map, demonstrating, for the first time, that the features learned in the deep fmap framework do have a geometric meaning, and can be used directly for matching, even without the need to compute a functional map. We explain the slight discrepancy between the result obtained with fmap and NN by the fact that, in practice, the conditions of the theorem are not 100\% satisfied, since we only used approximations to obtain them. In \cref{fig:hum_matching_one}, we show the quality of the produced maps, before and after our modifications. It can be seen that our modifications produce visually plausible correspondences.

\begin{figure}
    \centering
    \includegraphics[width=\columnwidth]{figures/hum_matching_one.pdf}
    \caption{ Qualitative results on SCAPE Remeshed dataset using DeltaConv \cite{Wiersma2022DeltaConv}, before and after our modifications.}
    \label{fig:hum_matching_one}
\end{figure}




%%%%%%%%%%%%%%%%%%%%%%%%%%%%%%%%%%%%%%%%%%%%%%%%%%%%%%%%%%%%%%
%%%%%%%%%%%%%%%%%%%%%%%%%%%%%%%%%%%%%%%%%%%%%%%%%%%%%%%%%%%%%%
%%%%%%%%%%%%%%%%%%%%%%%%%%%%%%%%%%%%%%%%%%%%%%%%%%%%%%%%%%%%%%
%%%%%%%%%%%%%%%%%%%%%%%%%%%%%%%%%%%%%%%%%%%%%%%%%%%%%%%%%%%%%%
%%%%%%%%%%%%%%%%%%%%%%%%%%%%%%%%%%%%%%%%%%%%%%%%%%%%%%%%%%%%%%
\subsubsection{Unsupervised Shape Matching}
\label{sec:unsup-human}



In this section, we test the quality of our modifications in the case of unsupervised learning on near-isometric data. We follow the same method as explained in \cref{sec:background} with the loss in \cref{eq:unsup_loss_proper}, and use three feature extractors: DGCNN, DiffusionNet and DeltaConv, and three datasets: \textbf{FA}, \textbf{SA} and \textbf{SHA}. The results are summarized in \cref{tab:unsup_humans}.

%\souhaib{the unsupervised losses are not mentioned in the text, is it a problem}

As for the supervised case, our modifications improve the results for all scenarios and for all feature extractors. For example, the result improved by over 70 \% for DeltaConv on the \textbf{FRA} on \textbf{SHA} scenario.

In addition, the extrinsic feature extractors are now on par with DiffusionNet, and the NN results are now close to those obtained with fmap, reinforcing the conclusions obtained in the supervised case \cref{sec:supervised-human}.

\begin{table}[!t]
     \centering
     \ra{1.0}
          \resizebox{\columnwidth}{!}{%
               \begin{tabular}{@{}l cc cc cc cc cc cc@{}}

\rowcolor{Gray!50}
\textbf{Model / Dataset} & \multicolumn{2}{c}{\textbf{FA on FA}} & \multicolumn{2}{c}{\textbf{SA on SA}} & \multicolumn{2}{c}{\textbf{FA on SA}} & \multicolumn{2}{c}{\textbf{SA on FA}} & \multicolumn{2}{c}{\textbf{FA on SHA}} & \multicolumn{2}{c }{\textbf{SA on SHA}}\\

& FM & NN & FM & NN & FM & NN & FM & NN & FM & NN & FM & NN\\

\midrule
\rowcolor{JungleGreen!80}
DiffusionNet & 3.9 & 6.5 & 4.5 & 6.5 & 5.4 & 8.5 & 3.7 & 6.0 & 6.1 & 11.9 & 6.0 & 10.6 \\
\rowcolor{JungleGreen!80}
DiffusionNet - Ours & \textbf{3.3} & \textbf{2.6} & \textbf{3.9} & \textbf{3.4} & \textbf{4.2} & \textbf{4.0} & \textbf{3.3} & \textbf{2.7} & 6.2 & \textbf{5.7} & \textbf{5.3} & \textbf{5.3} \\

% \cmidrule[\heavyrulewidth]{1-13}
\addlinespace

\rowcolor{Apricot!80}
DGCNN & 3.9 & 9.3 & 5.0 & 11.2 & 7.0 & 13.6 & 4.1 & 11.9 & 6.7 & 17.1 & 6.5 & 16.7\\
\rowcolor{Apricot!80}
DGCNN - Ours & 3.9 & 2.8 & 4.6 & 3.8 & 5.2 & 5.0 & 4.0 & 3.4 & 6.5 & \textbf{5.7} & 6.2 & 6.0\\

% \cmidrule[\heavyrulewidth]{1-13}
\addlinespace

\rowcolor{Salmon!80}
DeltaConv & 3.8 & 12.9 & 4.7 & 15.5 & 5.1 & 17.4 & 4.0 & 16.5 & 7.0 & 23.6 & 6.7 & 25.0 \\
\rowcolor{Salmon!80}
DeltaConv - Ours & 3.6 & 3.5 & 4.4 & 4.0 & 4.7 & 4.7 & 4.0 & 3.5 & \textbf{6.0} & 6.1 & 6.7 & 7.7 \\

\bottomrule


               \end{tabular}
          }
          \caption{Mean geodesic error of various feature extractors for unsupervised shape matching.}
          \label{tab:unsup_humans}

     \vspace{-1em}
\end{table}

Note that the effect of enforcing properness is more visible in this context. This can be seen for example in the case of DiffusionNet in Table \cref{sec:supervised-human} since for this feature extractor the only change brought about by our method is via properness. In fact, in the supervised setting, since the training is done using the ground truth-functional maps, and these are proper, this forced the network to learn features that produce functional maps that are as proper as possible, whereas in the unsupervised setting, the network is trained with losses that impose structural properties on the functional maps such as orthogonality, and no properness is involved. We can see that after imposing properness, both functional map and NN results improve, while also getting closer to each other.





%%%%%%%%%%%%%%%%%%%%%%%%%%%%%%%%%%%%%%%%%%%%%%%%%%%%%%%%%%%%%%
%%%%%%%%%%%%%%%%%%%%%%%%%%%%%%%%%%%%%%%%%%%%%%%%%%%%%%%%%%%%%%
%%%%%%%%%%%%%%%%%%%%%%%%%%%%%%%%%%%%%%%%%%%%%%%%%%%%%%%%%%%%%%
%%%%%%%%%%%%%%%%%%%%%%%%%%%%%%%%%%%%%%%%%%%%%%%%%%%%%%%%%%%%%%
%%%%%%%%%%%%%%%%%%%%%%%%%%%%%%%%%%%%%%%%%%%%%%%%%%%%%%%%%%%%%%
\subsubsection{Non-Isometric Shape Matching}
\label{sec:animals-matching}
\begin{table}[!t]
     \centering
     \ra{1.0}
          \resizebox{0.7\columnwidth}{!}{%
               \begin{tabular}{@{}l cc cc@{}}

\rowcolor{Gray!50}
\textbf{Animals Dataset} & \multicolumn{2}{c}{\textbf{Supervised}} & \multicolumn{2}{c}{\textbf{Unsupervised}} \\

& FM & NN & FM & NN \\
\toprule

\rowcolor{JungleGreen!80}
DiffusionNet & 6.1 & 8.7 & 8.0 & 16.6 \\
\rowcolor{JungleGreen!80}
DiffusionNet - Ours & 5.7 & 7.8 & 6.4 & 10.1 \\

% \cmidrule[\heavyrulewidth]{1-5}
\addlinespace

\rowcolor{Apricot!80}
DGCNN & 8.3 & 18.4 & 10.9 & 19.1 \\
\rowcolor{Apricot!80}
DGCNN - Ours & \underline{5.0} & \underline{4.8} & \underline{5.3} & \textbf{5.5} \\

% \cmidrule[\heavyrulewidth]{1-5}
\addlinespace

\rowcolor{Salmon!80}
DeltaConv & 5.4 & 20.7 & 9.5 & 24.1 \\
\rowcolor{Salmon!80}
DeltaConv - Ours & \textbf{4.7} & \textbf{4.2} & \textbf{5.1} & \underline{5.6}\\


\bottomrule



               \end{tabular}
          }
          \caption{Mean geodesic error comparison of different feature extractors on the \textbf{SMAL} dataset, using supervised and unsupervised methods. The highest performing result in each column is denoted in \textbf{bold}, while the second best is denoted with \underline{underline}.}
          \label{tab:match_animals}
      \vspace{-0.5em}
\end{table}

In this section, we test the utility of our modifications in the case of non-rigid non-isometric shape matching, on the \textbf{SMAL} dataset. We test it both in the case of supervised and unsupervised learning, following the same procedure as in \cref{sec:supervised-human} and \cref{sec:unsup-human}. The results are summarized in \cref{tab:match_animals}. As in the previous sections, our modifications improve the results for all feature extractors for both supervised and unsupervised matching. For example, the results for supervised DeltaConv improve by approximately 80\%, from 20.7 to 4.2.

More surprisingly, using our modifications, in the non-isometric setting, DiffusionNet is surpassed as the best feature extractor for shape matching, as DGCNN and DeltaConv achieve better results, using either the functional map or NN method to extract the p2p map. %\souhaib{why this happens?}

%\souhaib{should we add another remark that I missed?}









%%%%%%%%%%%%%%%%%%%%%%%%%%%%%%%%%%%%%%%%%%%%%%%%%%%%%%%%%%%%%%
%%%%%%%%%%%%%%%%%%%%%%%%%%%%%%%%%%%%%%%%%%%%%%%%%%%%%%%%%%%%%%
%%%%%%%%%%%%%%%%%%%%%%%%%%%%%%%%%%%%%%%%%%%%%%%%%%%%%%%%%%%%%%
%%%%%%%%%%%%%%%%%%%%%%%%%%%%%%%%%%%%%%%%%%%%%%%%%%%%%%%%%%%%%%
%%%%%%%%%%%%%%%%%%%%%%%%%%%%%%%%%%%%%%%%%%%%%%%%%%%%%%%%%%%%%%

\subsection{Generalization Power of Geometric Features}

% \begin{table}[!t]
    \centering
    \ra{1.0}
          \resizebox{\columnwidth}{!}{%
               \begin{tabular}{@{}l r@{}}
               \toprule
                \textbf{Method} & \textbf{Accuracy} \\
                \midrule
                GCNN \cite{masci2015geodesic} & 86.4\% \\
                ACNN \cite{boscaini2016learning} & 83.7\% \\
                Pointwise MLP - XYZ - N=1 & 57.2 $\pm$ 2.5  \%\\
                Pointwise MLP - Pretrained Features - N=1& 86.7 $\pm$ 0.4 \% \\
                \bottomrule

               \end{tabular}
          }
          \caption{Human part segmentation on the dataset of \cite{maron2017convolutional}}% Our method is trained on pretrained features with a point wise MLP, using only N=1 shape for  supervision, achieves results comparable to complex 3D networks fully trained \textit{on the full dataset}.}
          \label{tab:generalisation}
     \vspace{-1em}
\end{table}

% We conclude our experiment section by testing the geometric nature of the features learned in the deep fmap pipeline. In this experiment, we aim to see how much information is included in the features themselves. If they are informative enough, they can be used for other tasks with little fine-tuning.

% To do this, we use the human segmentation dataset presented in \cite{maron2017convolutional}. We first extract features by a network pre-trained on shape matching, and then refine these features for the segmentation task. For this, we will use the pre-trained DiffusionNet using \textbf{SA} dataset and the unsupervised loss. 
% We will refine the features using a point-wise MLP, which allows the quality of the features to be reflected without resorting to complex convolution and pooling operations. 

We conclude our experiments by evaluating the generalization ability of the learned features in the deep functional map pipeline. Our goal is to investigate whether the extracted features contain sufficient information to be used for other tasks without the need for significant fine-tuning.

To conduct this experiment, we employ the human segmentation dataset presented in \cite{maron2017convolutional}. We first extract features from this dataset using a network that has been pre-trained on shape matching. We use the pre-trained DiffusionNet on the \textbf{SA} dataset with the unsupervised. Next, we refine the features for the segmentation task using only a small fraction of the training data. To accomplish this, we utilize a point-wise MLP to avoid complex convolution and pooling operations and to provide an accurate measure of the feature quality.

\begin{figure}
    \centering
    \includegraphics[width=\columnwidth]{figures/hum_seg_ev.pdf}
    \caption{Evolution of human segmentation accuracy as a function of the number of shapes used for the pointwise MLP training.}
    \label{fig:hum_seg_ev}
     \vspace{-0.5em}
\end{figure}

In \cref{fig:hum_seg_ev}, we plot the accuracy of the MLP 
%trained with the pre-trained features used previously,
as a function of the number of finetuning shapes. First note that the accuracy of the results improves with the number of training shapes (up to a certain point, where most likely the limited capacity of pointwise MLP is reached). Note also that our features, which are pre-trained in a completely unsupervised manner, produce \textit{significantly} higher accuracy than the raw XYZ coordinates. We attribute this feature utility in a downstream task to the fact that these features capture the geometric structure of the shapes in a compact and invariant manner.
To further demonstrate the generalization power of the pretrained features, we conducted a similar experiment on the RNA dataset \cite{poulenard2019effective} and present the results in the supplementary material.

%In addition, we use only \textbf{one shape} for training the point-wise MLP, which represents \textbf{0.3\% of the training set}. The results are summarized in \cref{tab:generalisation}. For robustness, we repeat the experiment 5 times by randomly selecting the training shape, and reporting the mean and standard deviation.
%\maks{we should promise more results (at least variance of results depending on the choice of the training shape) in the sup.mat. otherwise we might get hit because the numbers might not be reliable.}

% We find that the performance of our features is equal to or better than that of some complex 3D networks that are trained on the \textit{complete training set}. We attribute this result to the robust geometric information encoded in the pre-trained features. 
% We note that we do not claim SOTA results on this dataset, but rather aim to demonstrate the potential of using learned deep functional map features in other tasks, going beyond their current purely algebraic role in formulating linear systems for functional maps.



\tightpara{Ablation Studies} 
% Our method includes two modifications to the deep functional map pipeline that we consider essential for optimal performance. 
In the supplementary, we report an ablation study that demonstrates the effectiveness of the individual components we introduced, in addition to a comparison of our modified pipeline to other recent baseline methods for shape matching.


\section{Conclusion and Limitations}
\label{sec:conclusion}
In conclusion, we have shown that our method of pre-training local features on rigid 3D scenes can generalize well to new and unseen classes of deformable organic shapes, enabling effective performance in various shape analysis tasks. Our study has highlighted the importance of selecting the right receptive field size to ensure feature transferability, which has led to the \textit{first general-purpose local feature pre-training}  for deformable shape analysis tasks. This research also sheds light on the relationship between rigid and non-rigid processing tasks, providing a link between two fields that have traditionally used different tools.

One limitation of our method is its reliance on differentiable voxelization, which can be memory and time-consuming, particularly during pre-training. Nonetheless, our results outperform PointContrast \cite{xie2020pointcontrast}, a point-based method that requires \textit{more training data} and has limited generalizability. Another limitation is that our features rely on LRF estimation, which might lack robustness to thin structures or boundaries of partial shapes. Exploring alternative scalable and robust local feature pre-training strategies is an fascinating direction for future work.

\mypara{Acknowledgements}
The authors would like to thank the anonymous reviewers for their valuable suggestions. 
Parts of this work were supported by the ERC Starting Grant No. 758800 (EXPROTEA) and the ANR AI Chair AIGRETTE.









% In this work, we demonstrated that local features trained for rigid alignment of 3D scenes can generalize remarkably well to new unseen classes and especially deformable organic shapes in a wide range of shape analysis tasks. For this, we first showed the critical role that the receptive field size plays in the transferability of local features and proposed an optimization strategy to enable feature transfer across significantly different shape classes. 

% Our approach leads to the \textit{first general-purpose local feature pre-training} method that is applicable in deformable shape analysis tasks. Remarkably, our learned features enable tasks such as generalizable unsupervised shape matching without relying on shape pre-alignment. Our work also sheds light on the utility of low-level features in 3D transfer learning and creates an interesting link between rigid (3D scene or man-made object) shape analysis and non-rigid processing tasks -- two fields that have traditionally been considering very different tools.

% Perhaps the biggest limitation of our work is that it relies on differentiable voxelization and is thus relatively memory and time-consuming, especially during pre-training. Nevertheless, we obtain better results than PointContrast \cite{xie2020pointcontrast} that, despite being point-based, requires more training data and
% has limited generalizability across domains.

% \mypara{Acknowledgements}
% The authors would like to acknowledge the anonymous reviewers for their valuable suggestions. 
% Parts of this work were supported by the ERC Starting Grant No. 758800 (EXPROTEA) and the ANR AI Chair AIGRETTE.

% \paragraph{Societal impact}
% Efficient methods for non-rigid shape analysis have immediate impact in many
% areas of science and engineering from medical imaging
% (for instance for detecting anomalies, and performing follow-up analysis) to shape recognition and classification in areas such as computational biology, archaeology and paleontology to name a few. Our approach can immediately be adapted and tested in such diverse scenarios, due to the strong generalization power of the proposed descriptors. Our work also opens major avenues for future research as it can facilitate geometric deep learning methods without training, thus potentially enabling small labs to do research in this field without having big clusters or access to large-scale datasets. Finally we note that avoiding extensive training for each application also reduces the environmental impact of geometric deep learning, by significantly reducing the computation requirements for each independent application.



 %%%%% supp, for the arxiv version
\newpage
% \onecolumn
% \null

% \title{Supplementary Materials for:\\Generalizable Local Feature Pre-training for Deformable Shape Analysis}
% \author{ Souhaib Attaiki \hspace{1.5cm} Lei Li \hspace{1.5cm} Maks Ovsjanikov\\
% LIX, \'Ecole Polytechnique, IP Paris}

% \begin{center}
%       % smaller title font only for rebuttal
%       {\Large \bf \title \par}
%       % additional two empty lines at the end of the title
%       {\vspace*{24pt}}
%       {
%       \large
%       \lineskip .5em
%       \begin{tabular}[t]{c}
%         \author
%       \end{tabular}
%       \par
%       }
%       % additional small space at the end of the author name
%       \vskip .5em
%       % additional empty line at the end of the title block
%       \vspace*{12pt}
%    \end{center}

% \begin{multicols}{2}

\twocolumn[{%
 \centering
 {\Large \bf Supplementary Materials for:\\Generalizable Local Feature Pre-training for Deformable Shape Analysis \par}
 {\vspace*{24pt}}
      {
      \large
      \lineskip .5em
      \begin{tabular}[t]{c}
        Souhaib Attaiki \hspace{1.5cm} Lei Li \hspace{1.5cm} Maks Ovsjanikov\\
LIX, \'Ecole Polytechnique, IP Paris
      \end{tabular}
      \par
      }
      % additional small space at the end of the author name
      \vskip .5em
      % additional empty line at the end of the title block
      \vspace*{12pt}
}]
\appendix

In this document, we collect all the results and discussions, which, due to the page limit, could not find space in the main manuscript.
This supplementary material consists of two parts.
First, in \cref{suppsec:implementation_details}, we describe more implementation details mainly regarding our pilot study, local feature pre-training, and experiments on downstream deformable shape data.
Next, in \cref{suppsec:additional_results}, we present additional experimental results and analysis of our local feature pre-training strategy and its generalization in downstream tasks, including deformable shape matching and segmentation. 


\section{Implementation Details}
\label{suppsec:implementation_details}


\subsection{Feature Locality vs. Transferability}
\label{suppsubsec:feature_locality_vs_transferability}
In Sec.~3 of the main text, we conducted a pilot study on feature locality vs. transferability on deformable shapes.
We tested three different architectures for pre-training a \textit{local} feature extractor, and their details are as follows.

\mypara{SparseConv.}
We used the \texttt{ResNet14} architecture introduced in \cite{choy20194d}.
During pre-training, given a 3D point cloud $P$, a fixed-size local patch with a radius of 0.15 is cropped at point $\mathbf{p} \in P$ and then reoriented with a local reference frame (LRF) computed by the method in \cite{gojcic2019perfect} for rotation invariance.
The resulting local patch is fed to the sparse convolution network, which extracts a 32-dimensional feature vector for point $\mathbf{p}$.

\mypara{PCPNet.}
It is a variant of PointNet \cite{qi2017pointnet} endowed with a quaternion spatial transformer.
We used the single-scale architecture proposed by \cite{pcpnet2018}.
PCPNet is designed to be a local network requiring input patches to have a fixed number of points.
Thus during pre-training, a fixed-size local patch (radius = 0.15) is cropped at point $\mathbf{p}$ and reoriented by an LRF. 
The local patch is then resampled to 1,024 points and fed to the network, resulting in a 32-dimensional feature vector for point $\mathbf{p}$.

\mypara{3DCNN.}
We used the architecture from \cite{li2021updesc} with a learnable receptive field size and differentiable voxelization, the same as our \OurMethodName{} in Sec.~4.1 of the main text.
More details can be found in \cref{suppsubsec:local_feature_pretraining}.

\mypara{Dataset.}
We pre-trained the above local networks on the 3DMatch dataset, which is a collection of RGB-D scan datasets with 62 indoor scenes and 4,142 point cloud fragments. 
There are 13K points on average in a fragment after downsampling.

\mypara{Loss.}
We used the PointInfoNCE loss, in which 300 point correspondences were randomly sampled for a pair of point clouds for faster training and the temperature parameter $\tau$ was set to 0.07. 

We also used the cycle consistency loss $\mathcal{L}_c$. During pre-training, we use the extracted features to build correspondences for rigid alignment between shapes $P$ and $Q$.
The intuition for $\mathcal{L}_c$ is that the estimated transformation $(\mathbf{R}, \mathbf{t})$ aligning $P$ to $Q$ should be the inverse of the transformation $(\mathbf{R}', \mathbf{t}')$ aligning $Q$ to $P$.
Mathematically, this can be expressed as:

\begin{equation}
\begin{bmatrix}
\mathbf{R} & \mathbf{t}\\
\mathbf{0} & 1
\end{bmatrix}
\begin{bmatrix}
\mathbf{R'} & \mathbf{t'}\\
\mathbf{0} & 1 
\end{bmatrix}
=
\begin{bmatrix}
\mathbf{R}\mathbf{R'} & \mathbf{R}\mathbf{t'} + \mathbf{t}\\
\mathbf{0} & 1
\end{bmatrix}
= \mathbf{I}
\end{equation}

\mypara{Application to deformable shape matching.}
In Fig. 3 of the main text, we have shown the results of shape matching on the Faust Remeshed dataset, directly using the pre-trained feature extractors. Given two shapes $S_1$, and $S_2$, we compute their respective point-wise features $F_1$ and $F_2$ using a specific pre-trained model. We first produce an estimate of the point-to-point maps $T_{21}^{nn}$ and $T_{12}^{nn}$ using nearest neighbor search between $F_1$ and $F_2$. We then filter the correspondences by mutual check: a pair of points $x \in S_1, y \in S_2$ is considered to be in correspondence, if and only if in the feature space, $x$ is the nearest neighbor of $y$, and $y$ is the nearest neighbor of $x$. This results in two filtered maps $T_{21}^{mf}$ and $T_{12}^{mf}$. Finally, we further refine these two maps using the ZoomOut method \cite{Melzi_2019}, which is based on navigating between the spectral and spatial domains while progressively increasing the number of spectral basis functions. We emphasize that if the initial point-to-point map is noisy or contains strong ambiguities like symmetry ambiguities, ZoomOut is not able to remedy these errors, thus leading to final correspondences of bad quality. We perform 10 iterations of ZoomOut, starting from 30 eigenfunctions up to 100 eigenfunctions.



\subsection{Local Feature Pre-training}
\label{suppsubsec:local_feature_pretraining}
In Sec.~4.1 of the main text, we introduced our local feature pre-training strategy.

\mypara{Feature extraction.} We use $r_{\text{LRF}}=0.3$ and $\sigma=10^{-3}$ for differentiable voxelization \cite{li2021updesc}, and the voxel grid resolution is set to $16^3$.
We pre-trained on the 3DMatch dataset introduced in \cref{suppsubsec:feature_locality_vs_transferability}.

\mypara{Pre-training loss.} 
For the PointInfoNCE loss $\mathcal{L}_{\text{nce}}$, its settings are described in \cref{suppsubsec:feature_locality_vs_transferability}.
For the cycle consistency loss $\mathcal{L}_{\text{c}}$, 300 points were randomly sampled on each point cloud for feature extraction and alignment estimation.
A relaxation-based solver is used in $\mathcal{L}_{\text{c}}$ for estimating a 3D transformation between two point clouds, and its details can be found in \cite{li2021updesc}.
 
In the main text, we investigated the performance difference between the cycle consistency loss and PointInfoNCE loss w.r.t learned feature smoothness.
Suppose that $F \in \mathbb{R}^{m \times n}$ is the matrix of extracted $n$-dimensional point-wise features for a shape of $m$ vertices, we measure the Dirichlet energy as follows: 
\begin{equation}
    % E_{Dirichlet}(F) = \frac{1}{n} \sum_{i=1}^n \frac{F_i^{\top} W F_i}{F_i^{\top} A F_i},
    E_{Dirichlet}(F) = \frac{1}{n} \sum_{i=1}^n F_i^{\top} W F_i,
\end{equation}
where $F_i$ is the $i^{\text{th}}$ column of $F$, and $W$ is the standard stiffness matrix computed using the classical cotangent discretization scheme of the Laplace-Beltrami operator \cite{Pinkall1993}.



\subsection{Baselines}
In Sec.~5 of the main text, we tested our proposed \OurMethodName{} features against a wide spectrum of competitors, including both hand-crafted and learned features.

Specifically, the Heat Kernel Signature (HKS) and Wave Kernel Signature (WKS) features are both sampled at 100 values of energy \textit{t}, logarithmically spaced in the range proposed in their respective original papers.
SHOT descriptors are 352-dimensional, and we used the implementation from the PCL library \cite{Rusu_ICRA2011_PCL}.
PointContrast features are 32-dimensional, and we used the publicly available implementation and the pre-trained weights released by the authors\footnote{\url{https://github.com/facebookresearch/PointContrast}}.


\subsection{Downstream Shape Analysis Training}
In Sec.~5 of the main text, we used DiffusionNet on top of the baselines features and our \OurMethodName{} respectively, in both the shape matching and segmentation tasks. We employed the publicly available implementation of DiffusionNet released by the authors\footnote{\url{https://github.com/nmwsharp/diffusion-net}}.
Unless specified otherwise, in our experiments, we used four DiffusionNet blocks of width = 128. 
The DiffusionNet is trained by an ADAM optimizer \cite{kingma2017adam} with an initial learning rate of $10^{-3}$.

In Sec.~5.1 of the main text, we also used a point-wise MLP network on top of the baselines features and our \OurMethodName{} respectively for supervised shape matching.
For this, we use the same MLP architecture as in FMNet \cite{litany2017deep}. 
After computing the point features with the MLP, we use them to compute the predicted functional map $C_{pred}$ as in \cite{donati2020deep} and penalize its deviation from the ground-truth map $C_{gt}$ using the L2 loss: $L = \|C_{pred} - C_{gt}\|_2^2$.


\subsection{Computational Specifications}
All our experiments were executed using Pytorch \cite{NEURIPS2019_9015}, on a 64-bit machine, equipped with an Intel(R) Xeon(R) CPU E5-2630 v4 @ 2.20GHz and an RTX 2080 Ti Graphics Card.

In terms of computational time, pertaining our method takes about 12 hours on a single RTX 2080 Ti Graphics Card, in contrast to the 64 hours required for PointContrast. The receptive field optimization takes about 20 minutes per dataset. For feature extraction, our method takes 3 seconds to extract local features for a 5000-vertex shape, which is on par with other local features like SHOT~\cite{tombari2010unique}, but slower than PointContrast (0.1s). Finally, the forward pass using \OurMethodName{} takes the same time as for all baseline features, e.g., 0.2 seconds per iteration for the unsupervised shape-matching experiment in Sec 5.1 of the main text.


\section{Additional Results and Analysis}
\label{suppsec:additional_results}

\subsection{Size of the learned receptive field}
Fig.~5 of our paper provides an illustration of the optimized receptive field in downstream tasks. In \cref{fig:receptive_field}, we include more visualizations for shape \textit{pairs} for both humans and animals.
Observe that the optimized receptive field indeed corresponds to interpretable concepts, such as the head or foot of a human, and is consistent across shape pairs.

\begin{figure}[t]
  \centering
  \includegraphics[width=0.99\linewidth]{figures/patch_viz.pdf}
   \caption{Visualizing the optimized receptive field for shape pairs.}
   % \vspace{-0.6cm}
   \label{fig:receptive_field}
\end{figure}


\subsection{Human Shape Matching} 
\label{suppsubsec:human_matching}
In Sec.~5.1 of the main text, we performed unsupervised shape matching on the FAUST-Remeshed (FR), SCAPE-Remeshed (SR), and SHREC’19 datasets (SH) and reported the matching performance in Tab.~1.
We provide additional quantitative results of the FR-SR and SR-FR settings in \cref{tab:unaligned_unsup_supp}.
Compared with the baseline features, our \OurMethodName{} has the best and most consistent performance in both settings.

\begin{table}[t]
    \begin{center}
    \ra{1.0}
        \resizebox{0.8\columnwidth}{!}{%
            \begin{tabular}{@{} lrr @{}}
                \toprule
                \textbf{Method / Dataset}                & \textbf{FR}-\textbf{SR} & \textbf{SR}-\textbf{FR} \\
                \midrule
                SURFMNET                                 & 15.2                    & 9.5                     \\
                Cyclic FMaps                             & 23                      & 23.2                    \\
                WSupFMNet                                & 27.1                    & 14.2                    \\
                Deep Shells                              & 6.0                     & \textbf{3.4}            \\
                \midrule
                DiffusionNet - XYZ                       & 25.7                    & 8.4                     \\ % 8.1 & 23.1 &
                DiffusionNet - HKS                       & 7.9                     & 23                      \\ % 7.1 & 15.4 &
                DiffusionNet - WKS                       & 4.2                     & 24.1                    \\ % 3.8 & 4.4 &
                DiffusionNet - SHOT                      & 7.2                     & 4.1                     \\ % 3.8 & \textbf{4.2} &
                DiffusionNet - PCH                       & 11.4                    & 8.7                     \\ % 6.3 & 4.3 &
                DiffusionNet - PCN                       & 20.4                    & 9.1                     \\ % 8.7 & 4.8 &
                DiffusionNet - \OurMethodName{} (ours)   & \textbf{4.1}            & 3.9                     \\ % % 8.7 & 4.8 &
                \bottomrule
            \end{tabular}
        }
        \caption{Accuracy of various features for unsupervised shape matching on un-aligned data.  X-Y means train on X and test on Y. Values are mean geodesic error $\times 100$ on unit-area shapes.}
        \label{tab:unaligned_unsup_supp}
    \end{center}
\end{table}


% %\setlength{\tabcolsep}{4pt}
\begin{table}
    \begin{center}
        \begin{tabular}{lccc}
            \toprule
            \textbf{Method / Dataset}     & \textbf{FO}-\textbf{FR} & \textbf{FO}-\textbf{FQ} &
            \textbf{FO}-\textbf{SR}                                                                           \\
            \midrule
            MLP - XYZ                     & 13.8                    & 12.8                    & 26.9          \\
            MLP - HKS                     & 12.4                    & 21.5                    & 23.4          \\
            MLP - WKS                     & 12.7                     & 34.0                    & 27.2          \\
            MLP - SHOT                    & 10.0                    & 17.6                    & 10.6          \\
            MLP - PCH                     & 17.7                    & 41.5                    & 32.8          \\
            MLP - PCN                     & 15.7                    & 41.4                    & 28.4          \\
            MLP - \OurMethodName{} (ours) & \textbf{4.2}            & \textbf{4.9}            & \textbf{9.4} \\
            \bottomrule
        \end{tabular}
        \caption{Accuracy of various features for supervised shape matching when the connectivity changes from train to test.  X-Y means train on X and test on Y. Values are mean geodesic error $\times 100$ on unit-area shapes. \rev{Are the \OurMethodName{} results in this table up-to-date? If not, please remove the table. Corresponding discussions have been commented out.}}
        \label{tab:robust_connectivity}
    \end{center}
\end{table}
%\setlength{\tabcolsep}{1.4pt}



% \rev{In the main text, we also tested the robustness of our \OurMethodName{} features to the change of remeshing.
% To complement Fig.~6 of the main text, we provide additional experiments using the same training setup, by training on the FO dataset, and testing on the FR and SR datasets, respectively.
% \cref{tab:robust_connectivity} shows the evaluation results.
% We observe that our method consistently outperforms the competing features.
% This indicates that our \OurMethodName{} features are robust and descriptive under change of sampling and can generalize well across datasets (FO-SR setting).}

\subsection{Molecular Surface Segmentation} 
\label{suppsubsec:qualitative_evaluation}
In \cref{fig:rna_seg_qual}, we show qualitative results of RNA segmentation using DiffusionNet + \OurMethodName{}.
It can be seen that the challenging RNA molecules can be robustly segmented into functional components with our pre-trained features.

\begin{figure}[t]
    \centering
    \includegraphics[width=\linewidth]{figures/mol_seg.pdf}
    \caption{Qualitative evaluation of RNA segmentation on the dataset of \cite{poulenard2019effective}. Left: ground truth. Right: prediction by DiffusionNet + \OurMethodName{}.}
    \label{fig:rna_seg_qual}
\end{figure}



\subsection{Human Shape Segmentation}
\label{suppsubsec:human_segmentation}

We performed an additional experiment on the human shape segmentation task.
We used the dataset introduced in \cite{maron2017convolutional}, which combines segmented human models taken from a variety of existing datasets.
We used the same train/test split of 380 training and 18 test shapes as in prior works.
We compared our \OurMethodName{} only with methods that used the original evaluation protocol as in \cite{maron2017convolutional}, i.e., without using post-processing and evaluating the results on the full shape resolution (techniques such as Mesh Walker \cite{lahav2020meshwalker} are thus excluded). 

We ran each experiment five times and report the mean and standard deviation of the accuracy in \cref{tab:human-segmentation}.  
Our \OurMethodName{} features achieve an accuracy of $92.4 \pm 0.25\%$, the state-of-the-art result on this dataset.
In \cref{fig:human_seg_qual}, we present qualitative results of human segmentation using DiffusionNet + \OurMethodName{}.
Note that the segmentation results are simply the network predictions, and we do not perform any complex post-processing to the segmentation.

%\setlength{\tabcolsep}{4pt}
\begin{table}[t]
    \begin{center}
    \ra{1.0}
        \begin{tabular}{@{}lr@{}}
            % \cmidrule[\heavyrulewidth]{1-2}
            \toprule
            \textbf{Method}                           & \textbf{Accuracy $\pm$ s.d}                   \\
            % \cmidrule{1-2}
            \midrule
            GCNN \cite{masci2015geodesic}             & 86.4\%                                        \\
            ACNN \cite{boscaini2016learning}          & 83.7\%                                        \\
            Toric Cover \cite{maron2017convolutional} & 88.0\%                                        \\
            PointNet++ \cite{qi2017pointnet}          & 90.8\%                                        \\
            MDGCNN \cite{poulenard2019effective}      & 88.6\%                                        \\
            DGCNN \cite{wang2019dgcnn}                & 89.7\%                                        \\
            SNGC \cite{haim2019surface}               & 91.0\%                                        \\
            CGConv \cite{yang2021continuous}          & 89.9\%                                        \\
            \cmidrule{1-2}
            DiffusionNet - XYZ                        & 91.9 $\pm$ 0.27\%                             \\
            DiffusionNet - HKS                        & 91.5 $\pm$ 0.21\%                             \\
            DiffusionNet - WKS                        & 91.8 $\pm$ 0.33\%                             \\
            DiffusionNet - SHOT                       & 91.5 $\pm$ 0.77\%                             \\
            DiffusionNet - PCH                        & 85.6 $\pm$ 0.75\%                             \\
            DiffusionNet - PCN                        & 87.3 $\pm$ 0.57\%                             \\
            DiffusionNet - \OurMethodName{} (ours)    & \textbf{92.4 $\pm$ 0.25\%} \accuchange{+0.9}  \\
            % \cmidrule[\heavyrulewidth]{1-2}
            \bottomrule
        \end{tabular}
        \caption{Human shape segmentation on the dataset of \cite{maron2017convolutional}. Our \OurMethodName{} achieves the state-of-the-art performance among methods that do not perform post-processing and evaluate on the full shape resolution. The reported numbers are the mean and standard deviation of the accuracy over five runs initialized randomly.}
        \label{tab:human-segmentation}
    \end{center}
\end{table}
%\setlength{\tabcolsep}{1.4pt}


\begin{figure}[t]
    \centering
    \includegraphics[width=\linewidth]{figures/human_seg.pdf}
    \caption{Qualitative evaluation of human shape segmentation on the dataset of \cite{maron2017convolutional}. Left: ground truth. Right: prediction by DiffusionNet + \OurMethodName{}.}
    \label{fig:human_seg_qual}
\end{figure}



% \subsection{Training with Limited Data}
% Train on small dataset to show how informative the features are

% we can obtain better results by training on a small dataset
% see table

% not worth it, experiment dropped

\subsection{Robustness to Noise}
We performed an additional experiment to evaluate the robustness of our features to noise. For this, we followed the same setup as in Sec.~5.1 of the main text and in \cref{suppsubsec:human_matching}, by performing unsupervised learning on FR and testing on SR with an increasing amount of noise as input.
We compared our method to the best three competing features. The results are shown in \cref{fig:noise_robust} - left. It can be seen that our features are more robust to noise, i.e., the performance does not vary much with different noise levels (the intensity of the noise can be seen in \cref{fig:noise_robust} - right), which is not the case with other features, such as SHOT, whose performance degrades very quickly.


\begin{figure}[t]
    \centering
    %\includegraphics[width=0.9\columnwidth]{figures/noise_robust.pdf}
    \includegraphics[width=1.0\columnwidth]{figures/noise_levels.pdf}
    \caption{Left: Evolution of the geodetic error as a function of different
input noise levels. Right: Qualitative visualization of noise levels.}
    \label{fig:noise_robust}
\end{figure}



\subsection{Convergence Speed}

In our experiments, we observed that our \OurMethodName{} descriptors take less time to train and facilitate learning. To demonstrate this, we show in \cref{fig:convergence_speed} the evolution of validation accuracy during learning of the RNA segmentation task (Sec. 5.2 of the main text). It can be seen that compared to the other features, VADER requires far fewer training iterations to achieve similar performance. This clearly indicates the better descriptiveness and generalizability of our features.

\begin{figure}[t]
    \centering
    \includegraphics[width=0.9\columnwidth]{figures/eval_acc_seg_new.pdf}
    \caption{Evolution of the RNA segmentation accuracy on the validation set, during the training of DiffusionNet with different features.}
    \label{fig:convergence_speed}
\end{figure}





% \subsection{Ablation Study}
% \label{suppsubsec:ablation_study}

% In Sec.~5.5 of the main text, we investigated different 3D datasets for local feature pre-training and showed that local geometries in 3DMatch are richer than those in DFAUST.
% We performed PCA on the local patches of 3DMatch and DFAUST.
% For each dataset, we first randomly extracted 200K local patches. 
% We then encode each patch as a high dimensional vector by first orienting it using a local reference frame and then voxelizing it to a small 3D grid of resolution = $16^3$ using the method of \cite{gojcic2019perfect}. 
% The resulting vectors are 4096-dimensional and fed as input to PCA to analyze the internal richness and complexity of each dataset.
% In \cref{fig:pca_unexplained_supp}, we report the unexplained variance as a function of the number of principal components. It further confirms that 3DMatch is significantly more diverse than DFAUST, since more principal components are needed to explain its full variance. 

% \begin{figure}[t]
    \centering
    \includegraphics[width=0.9\columnwidth]{figures/unexplained_variance.pdf}
    \caption{Unexplained variance in PCA of local patches from DFAUST and 3DMatch.}
    \label{fig:pca_unexplained_supp}
\end{figure}







%%%%%%%%% REFERENCES
{\small
\bibliographystyle{ieee_fullname}
\bibliography{references}
}

\end{document}
