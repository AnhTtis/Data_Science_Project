\documentclass[aps,prl,notitlepage,superscriptaddress]{revtex4-1}
\usepackage[colorlinks=true, citecolor=magenta, linkcolor=blue, urlcolor=blue]{hyperref}
\usepackage{graphicx}
\usepackage{amsmath}
\usepackage{amssymb}
\usepackage{amsfonts}
\usepackage{hyperref}
\usepackage{mathtools}
\usepackage{xcolor}
\usepackage{tikz}
\usetikzlibrary{shapes.geometric, arrows}
\usetikzlibrary{decorations.markings}
\DeclareMathOperator{\tr}{Tr}
\DeclareMathOperator{\re}{Re}
\DeclareMathOperator{\im}{Im}
\DeclareMathOperator{\sgn}{sgn}
\newcommand{\red}{\color{red}}
\newcommand{\blue}{\color{blue}}
\newcommand{\bk}{{\bf k}}
\newcommand{\bA}{{\bf A}}
\newcommand{\bq}{{\bf q}}
\newcommand{\br}{{\bf r}}
\newcommand{\bd}{\boldsymbol\delta}
\allowdisplaybreaks

\begin{document}
 
\title{Optimizing Electronic and Optical Responses of Novel Spin Transfer Torque-based Magnetic Tunnel Junctions with High Tunnel Magnetoresistance and Low Critical Currents}

\author{Tahereh Sadat Parvini}
\email{tahereh.parvini@uni-greifswald.de}
\affiliation{Institut für Physik, Universität Greifswald, Greifswald, Germany}
\author{Elvira Paz}
\affiliation{INL - International Iberian Nanotechnology Laboratory, Avenida Mestre José Veiga, s/n, 4715-330 Braga, Portugal}
\author{Tim B\"ohnert}
\email{tim.bohnert@inl.int}
\affiliation{INL - International Iberian Nanotechnology Laboratory, Avenida Mestre José Veiga, s/n, 4715-330 Braga, Portugal}
\author{Alejandro Schulman}
\affiliation{INL - International Iberian Nanotechnology Laboratory, Avenida Mestre José Veiga, s/n, 4715-330 Braga, Portugal}
\author{Luana Benetti}
\affiliation{INL - International Iberian Nanotechnology Laboratory, Avenida Mestre José Veiga, s/n, 4715-330 Braga, Portugal}
\author{Jakob Walowski}
\email{jakob.walowski@uni-greifswald.de}
\affiliation{Institut für Physik, Universität Greifswald, Greifswald, Germany}
\author{Farshad Moradi}
\affiliation{ICELab, Aarhus University, Denmark}
\author{Ricardo Ferreira}
\affiliation{INL - International Iberian Nanotechnology Laboratory, Avenida Mestre José Veiga, s/n, 4715-330 Braga, Portugal}
\author{Markus M\"unzenberg}
\affiliation{Institut für Physik, Universität Greifswald, Greifswald, Germany}
\date{\today}

\maketitle

\subsection{Hysteresis loop of the A-MTJs with different capping layer material}

Fig.\ref{Fig1SM} shows the magnetic hysteresis loops of A-MTJs with different capping layers measured using a MicroSense EV9 vibrating sample magnetometer (VSM) with the magnetic field applied parallel and perpendicular to the easy axis. All three stacks show the same magnetic anisotropy field $H_k=30 Oe$. The superior TMR ratio of MTJs incorporating Ta capping layers, in comparison to those with Cu and Ru capping layers, can be attributed to the Ta layer's capacity to enhance interfacial quality and promote a more uniform interface between the ferromagnetic and capping layers. This leads to a reduction in interfacial scattering and an enhancement in spin-dependent tunneling across the MTJ.

\begin{figure}[h!]
    \centering
    \includegraphics[width=0.8\textwidth]{Fig1SM.pdf}
    \caption{(a) Easy axis and (b) hard axis VSM measurement of the magnetic moment per area of A-MTJ stacks with different capping layers.}
\label{Fig1SM}
\end{figure}

\subsection{Time-Resolved Magneto-Optic Kerr Effect (TRMOKE) Measurements}

Time-resolved magneto-optical Kerr effect (TRMOKE) signals of the free layer (FL) simplified stacks, 5 Ta / 5 Ru / MgO / 2 CoFe$_{40}$B$_{20}$ / X / 2 Ta / 4 Ru with X$\equiv$0.21 Ta / 6 CoFe$_{40}$SiB$_{20}$ for A-FL stack and X$\equiv$0.21 Ta / 7 Ni$_{80}$Fe$_{20}$ for B-FL stack as a function of applied magnetic field amplitude are shown in Fig. \ref{Fig2SM} (b) and (a), respectively. In order to obtain a correct fit of the precessional processes, without any contamination from the ultrafast demagnetization, the TRMOKE measurements were fitted, after a delay
time of 1 ps, to a damped-harmonic function superposed with an exponential decay background
\begin{equation}
\Delta\theta_k=A+Be^{-\upsilon t}+Csin(2\pi ft+\phi_{0})e^{-t/\tau},
\label{Fitt}
\end{equation}
Here, the first two terms represent the change due to the recovery from the demagnetization and are characterized by the amplitudes of A and B and the recovery rate $\upsilon$. The last term represents the change due to the damped magnetization precession. C, $f$, $\tau$, and $\phi_{0}$ denote the precession amplitude, frequency, lifetime, and initial phase, respectively.  First, the TRMOKE data were fitted with Eq. \ref{Fitt} to extract the values of the precession frequencies ($f$) and magnetization relaxation times ($\tau$).The precession frequency increases while the relaxation time decreases with a rise in the magnetic field, as illustrated in Fig. \ref{Fig2SM} (c). For every magnetic field value, the B-FL stack exhibits a frequency approximately one GHz higher than the A-FL stack. The intrinsic damping $\alpha_0$ is an important parameter of stacks that has garnered significant attention. An alternative method for evaluating it involves determining the effective Gilbert damping parameter, $\alpha_{eff}=(2\pi f\tau)^{-1}$, which can be calculated using the fitted values of $f$ and $\tau$ (refer to Fig. \ref{Fig2SM} (d)). The $\alpha_{eff}$ values under high fields suppress the extrinsic contributions such as the inhomogeneous anisotropy and thus approximately equals the intrinsic $\alpha_{0}$. However, in lower fields, $\alpha_{eff}$ is larger than $\alpha_{0}$. To obtain the Gilbert damping parameter intrinsic to the sample geometry (not intrinsic to the material), we fit Kittel's formula, with extracted frequencies and derive the gyromagnetic ratio of the stacks by treating that as a fit parameter after inputting the values of $4\pi M_s$, $H_k$ extracted from VSM measurements, as shown in Fig. \ref{Fig2SM} (e). It is important to note that the middle Ta layer has an extremely low thickness of 0.2 nm, rendering it nothing more than an impurity. Additionally, the FFT spectrum depicting a single precession frequency has led us to conclude that the free layer, comprising two magnetic layers, behaves like a macrospin. For this reason, in the calculations we used $M_{s}^{B-FL}=\frac{t_{NiFe}M_s^{NiFe}+t_{CoFeB}M_s^{CoFeB}}{t_{NiFe}+t_{CoFeB}}$ and $M_{s}^{A-FL}=\frac{t_{CoFeSiB}M_s^{CoFeSiB}+t_{CoFeB}M_s^{CoFeB}}{t_{CoFeSiB}+t_{CoFeB}}$ as magnetization saturation of the B-FL and A-FL stacks respectively. The obtained values of intrinsic damping are shown in Fig. \ref{Fig2SM} (f). According to these experiments, we consider that the B-FL stack has the advantage of achieving low damping for STT switching.



\begin{figure*}[htp]
    \centering
    \includegraphics[width=0.95\textwidth]{Fig2SM.pdf}
    \caption{Time-resolved magneto-optical Kerr effect (TRMOKE) signals measured at a pump laser power of 300mW for free layer stacks (a) B-FL and (b) A-FL under different external applied fields $\mu_0 H=$60, 90, 120, and 150 mT (25$^{\circ}$ out of plane). Theoretical curves (solid curves) are fit to the experimental data (filled circles) using Eq. \ref{Fitt}. (c) Precession frequency ($f$) and relaxation time ($\tau$) obtained using the fitting function. The solid lines are guides to the eyes. (d) Effective damping coefficient as a function of the applied magnetic field. The solid lines are guides to the eyes. (e) Frequency as a function of magnetic field bias magnetic field. Symbols represent experimental data points and a solid curve is fitted to the Kittel formula. (f) Gilbert damping constant as a function of the applied magnetic field. The symbols represent experimental data points and the solid lines guide the eyes.}
\label{Fig2SM}
\end{figure*}

Following this, we investigated the magnetization dynamics of the full stacks of A-MTJs and B-MTJs with different capping layer thicknesses, denoted as t$_{\rm Ta}+$t$_{\rm Ru}$. TRMOKE signal of the B-MTJ and A-MTJ stacks measured at a pump laser power of 300mW by applying varying external magnetic fields $\mu_0 H=$60, 90, 120, and 150 mT (25$^{\circ}$ out of plane) are shown in Fig. \ref{Fig3SM} and \ref{Fig4SM} respectively. It can be seen that as the magnetic field strength increases, there is an increase in the precession frequency and a decrease in the relaxation time of the magnetization vector across all stacks. In both categories of MTJs, there is a consistent reduction in the $M_s$ value observed as the thickness of the capping layer increases and the magnetic field amplitude decreases. The fast Fourier transform (FFT) of the TRMOKE data after subtracting the exponential background for A-MTJs and B-MTJs are respectively shown in Fig. \ref{Fig5SM} and Fig. \ref{Fig6SM}. The energy dissipation process from electron spins to the lattice and subsequently to the surroundings during remagnetization, followed by ultrafast demagnetization, is a key contributing factor to the observed background effect. According to the FFT spectra, all stacks show a single precession frequency which is attributed to the collective precession of the magnetization vector in the first ferromagnetic layer (CoFe$_{40}$SiB$_{20}$ (6 nm) or Ni$_{80}$Fe$_{20}$ (7 nm)) and second ferromagnetic layer (CoFe$_{40}$B$_{20}$) within the free layers.The study's results suggest that the entire stack can be modeled as a macrospin, allowing for the use of the Landau-Lifshitz-Gilbert (LLG) equation with a macrospin model. This enables the accurate determination of frequency and damping parameters.The observed shift in the frequency of the uniform precessional mode towards lower values, as the bias magnetic field is reduced, provides evidence of the magnetic nature of the modes. The experimental results reveal that the precession frequency of the samples remains unchanged upon increasing the thickness of the capping layers. In comparison to A-MTJ stacks, B-MTJs exhibit a higher precession frequency and reduced oscillation amplitude attributed to the smaller saturation magnetization of NiFe compared to CoFeSiB. Additionally, B-MTJs demonstrate lower effective damping than A-MTJs. The $\alpha_{eff}$ values under high fields suppress the extrinsic contributions such as the inhomogeneous anisotropy and thus approximately equals the intrinsic $\alpha_{0}$. The gyromagnetic ratio values for A-MTJs and B-MTJs, as presented in Fig. \ref{Fig7SM}, were obtained using a method akin to that employed for free layers. Similarly, the full stacks containing NiFe exhibit a higher gyromagnetic ratio compared to those containing CoFeSiB.

\begin{figure*}[htp]
    \centering
    \includegraphics[width=0.75\textwidth]{Fig3SM.pdf}
    \caption{Kerr rotation spectra for A-MTJs, measured for fields applied for 30, 60, 90, 120, and 150 mT (25$^{\circ}$ out of plane). Curves acquired by fitting function and experimental results are shown by solid lines and star points respectively.}
\label{Fig3SM}
\end{figure*}

\begin{figure*}[htp]
    \centering
    \includegraphics[width=0.75\textwidth]{Fig4SM.pdf}
    \caption{Kerr rotation spectra for B-MTJs, measured for fields applied for 30, 60, 90, 120, and 150 mT (25$^{\circ}$ out of plane). Curves acquired by fitting function and experimental results are shown by solid lines and star points respectively.}
\label{Fig4SM}
\end{figure*}

\begin{figure*}[htp]
    \centering
    \includegraphics[width=0.75\textwidth]{Fig5SM.pdf}
    \caption{Fourier power spectra of A-MTJs with different thicknesses of the capping layer, measured for fields applied for 30, 60, 90, 120, and 150 mT (25$^{\circ}$ out of plane).}
\label{Fig5SM}
\end{figure*}

\begin{figure*}[htp]
    \centering
    \includegraphics[width=0.75\textwidth]{Fig6SM.pdf}
    \caption{Fourier power spectra of B-MTJs with different thicknesses of the capping layer, measured for fields applied for 30, 60, 90, 120, and 150 mT (25$^{\circ}$ out of plane).}
\label{Fig6SM}
\end{figure*}



\begin{figure*}[htp]
    \centering
    \includegraphics[width=0.95\textwidth]{Fig7SM.pdf}
    \caption{Frequency as a function of magnetic field bias magnetic field. Symbols represent experimental data points and a solid curve is fitted to the Kittel formula.}
\label{Fig7SM}
\end{figure*}



\clearpage
%\bibliography{main.bib}

\end{document}
