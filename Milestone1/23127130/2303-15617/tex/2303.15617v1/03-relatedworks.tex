\subsection{Related Works}

There exists substantial literature on baseline estimation methods \cite{coughlin2008estimating, grimm2008evaluating, mathieu2011quantifying, wijaya2014bias, nolan2015challenges, weng2015probabilistic, zhang2016cluster, nolan2015challenges, zhou2016forecast, hatton2016statistical, wang2018synchronous, muthirayan2019mechanism, muthirayan2019minimal}. These can be broadly classified into four classes: (a) averaging, (b) regression, (c) control group methods and (d) baseline reporting approaches.

\emph{Averaging methods} determine baselines by averaging the consumption on past days that are similar (\emph{e.g.,} in weather conditions) to the event
day. A detailed comparison of  different averaging methods is offered in \cite{coughlin2008estimating, grimm2008evaluating, wijaya2014bias}. Averaging
methods are simple but they suffer from estimation biases \cite{wijaya2014bias, weng2015probabilistic, nolan2015challenges}. \emph{Regression methods} estimate a load prediction model based on historical data which is then used to predict the baseline \cite{zhou2016forecast, mathieu2011quantifying}. They can potentially overcome biases incurred by averaging methods \cite{nolan2015challenges}. %But they often require considerable historical data for acceptable accuracy, and the models may not capture the complex behavior of individual consumers.
\emph{Control group methods} have been suggested to have better accuracy than averaging or regression type methods and do not require large amounts of historical data \cite{hatton2016statistical}. However, these methods require the SO to recruit an additional set of consumers and install additional metering infrastructure. In addition, prior data based analysis might be required to identify the most appropriate control group depending on the control group method deployed. This can raise the costs of implementation \cite{hatton2016statistical}. 

%We also show later that the adverse incentives to inflate still persists in this method. Compared to all the above methods, the proposed {\it self-reported baseline} avoids all of these issues, \emph{i.e.} (i) bias and inflation, (ii) need for historical data, and (iii) high implementation cost.

\emph{Baseline reporting approaches} were proposed in \cite{muthirayan2016mechanism, muthirayan2018baseline, muthirayan2019minimal, satchidanandan2022two} as an alternative baseline estimation method. These approaches employ the framework of mechanism design to design payment and selection schemes to ensure that the consumers report the correct baseline values. While these methods can provably reduce the baseline error from that of averaging type methods \cite{muthirayan2019mechanism}, they violate privacy and are infeasible when the consumers can be unaware of their baselines. %Moreover, these approaches violate the privacy of the consumers by requiring them to report their baseline values.

In contrast to the above approaches, we propose an online approach that does not require large quantities of historical data, or a control group, or reporting private information.
%In contrast to the above approaches, we consider the problem of managing DR resources while having to estimate the baseline online. Thus, the problem we consider is characteristically different from the first three approaches, where it is assumed that there is sufficient historical data at the disposal of the System Operator (SO) or the DR program manager to estimate the baseline. Here, we consider the online learning setting, where the SO has to learn the baseline from the data it observes online without any prior knowledge or historical data. The key difference from the estimation approaches, which leads to the main technical challenge, lies in the problem that the SO has to simultaneously learn and optimize its costs. In contrast to the baseline reporting approaches, where the reported baseline and thus the operating cost can deviate from the optimal \cite{muthirayan2019minimal}, we propose a method that ensures that the operating costs of the SO converges to the optimal operating cost with DR and at the same time individually rational for the consumers.  

%Dynamic pricing based DR mechanisms \cite{jacquot2018,muratori2016,yoon2014} mechanisms essentially influence consumers by using time varying prices to alter their energy consumption so that the system objectives are met. In contrast, we consider the online learning problem in the incentive-based DR approach. This requires the estimation of consumer baseline because measuring load reduction requires a baseline. Hence, baseline estimation becomes a primary concern in our setting whereas such a requirement does not arise in the dynamic pricing DR setting.

{\bf Notation}: We denote the expectation over a probability distribution by $\mathbb{E}[\cdot]$. We use $\mathcal{O}(\cdot)$ for the standard big-O notation while $\widetilde{\mathcal{O}}(\cdot)$ denotes the big-O notation neglecting the poly-log terms. %We also use $o(\cdot)$ for the standard little-o notation. Further, when a function $g(n) = o_n(1)$, then $g(n) \rightarrow 0$ as $n \rightarrow \infty$. 
We denote the sequence $(x_{m_1}, x_{m_1+1}, \dots , x_{m_2})$ compactly by $x_{m_{1}:m_{2}}$ and the sequence $(x^{m_1}, x^{m_1+1}, \dots , x^{m_2})$ compactly by $x^{m_{1}:m_{2}}$.