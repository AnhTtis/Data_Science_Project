\section{Conclusion}
\label{sec:conclusion}

In this paper, we investigated the problem of aligning instructional videos with a high-level schematic representation of the task, depicted by abstract instructional diagrams showing the steps in the process.
We proposed a method based on contrastive learning to align video and diagram features using three novel losses designed specifically for this task.
Our focus is on Ikea furniture assembly where alignment is done between in-the-wild videos and the corresponding official assembly manuals.
To this end, we also collected a dataset of 183 hours of in-the-wild assembly videos and nearly 8,300 diagrams.
Two tasks are designed on this dataset to evaluate the performance of our method: (i) a nearest neighbor retrieval task between video clips and instructional diagrams, (ii) alignment of the instruction diagrams to their corresponding assembly video clips.
On both tasks, experimental results show that our proposed sinusoidal progress rate feature and optimal transport modules lead to better temporal alignment and each one of the proposed losses enables the model to learn better representations, compared with compelling alternatives that do not take into account the unique nature of the problem.

Our work suggests several directions for future work.
First, it would be interesting to consider including additional modalities such as video narrations into our framework.
Second, extending the task to unsupervised or weakly supervised settings would overcome our current limitation of requiring ground truth alignments for learning.
Last, an ambitious long-term goal is to develop applications, built on our alignment model, that automatically monitor and guide a user through an assembly process or facilitate robot-human collaboration on instructional tasks.
