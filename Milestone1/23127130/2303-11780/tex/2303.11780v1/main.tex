\documentclass[sigconf]{acmart}

%\settopmatter{printacmref=false} % Removes citation information below abstract
% \renewcommand\footnotetextcopyrightpermission[1]{} % removes footnote with conference information in first column
\pagestyle{plain} % removes running headers
\usepackage{booktabs} % To thicken table lines

\usepackage[english]{babel}
\usepackage{moresize}
\usepackage{amsmath}
\usepackage{algorithmic}
\usepackage{balance}
\usepackage{comment}
\usepackage{paralist}
\usepackage{bm}
\usepackage{pgfplots}
\usetikzlibrary{pgfplots.dateplot}

\usepackage{flushend}
\usepackage[english]{babel}
\usepackage[latin1]{inputenc}
\usepackage{mathrsfs}
\usepackage{graphicx}
\let\Bbbk\relax
\usepackage{amssymb}
\usepackage{amsfonts}
\usepackage{url}
\usepackage{longtable}
\usepackage{rotating}
\usepackage{multirow}
\usepackage{mathrsfs}
\usepackage{subfigure}
\usepackage{enumitem}
\usepackage[linesnumbered,algoruled,boxed,lined]{algorithm2e}
\usepackage{adjustbox}
\usepackage{hyperref}
\usepackage{pgfplots}
\usetikzlibrary{pgfplots.dateplot}
% \usepackage{filecontents}
% Tableau colors
\definecolor{tblue}{RGB}{31,119,180}
\definecolor{torange}{RGB}{255,127,14}
\definecolor{tgreen}{RGB}{44,160,44}
\definecolor{tred}{RGB}{214,39,40}
\definecolor{tpurple}{RGB}{148,103,189}

\newcommand{\hide}[1]{} %hide
\newcommand{\vpara}[1]{\vspace{0.05in}\noindent\textbf{#1 }}
\newcommand{\noind}[1]{\hspace{-0.05in}\noindent{#1}}
\newcommand{\beq}[1]{\vspace{-0.03in}\begin{equation}#1\end{equation}\vspace{-0.02in}}
\newcommand{\beqn}[1]{\vspace{-0.03in}\begin{eqnarray}#1\end{eqnarray}\vspace{-0.03in}}
\newcommand{\besp}[1]{\begin{split}#1\end{split}}
\newcommand{\red}[1]{{\color{red}#1}}
\newcommand{\ie}{\emph{i.e.,}\xspace}
\newcommand{\eg}{\emph{e.g.,}\xspace}
\newcommand{\etal}{\emph{et al.}\xspace}
\newcommand{\wrt}{\textit{w}.\textit{r}.\textit{t}} 
\newcommand{\trans}{{\mkern-1.5mu\mathsf{T}}}
\newtheorem{Dfn}{Definition}
\newtheorem{Exa}{Example}
% \newcommand{\model}{\textsc{CLICD}}
\newcommand{\model}{\text{DCRec}}
\newcommand{\paratitle}[1]{\noindent\textbf{#1}}

%\setcopyright{acmcopyright}
% \copyrightyear{2018}
% \acmYear{2018}
% \acmDOI{XXXXXXX.XXXXXXX}

% \acmConference[Conference acronym 'XX]{Make sure to enter the correct
%   conference title from your rights confirmation emai}{June 03--05,
%   2018}{Woodstock, NY}
% \acmPrice{15.00}
% \acmISBN{978-1-4503-XXXX-X/18/06}

% \settopmatter{printfolios=true}
% \pgfplotsset{compat=1.18}
\pgfplotsset{compat=1.17}

\copyrightyear{2023}
\acmYear{2023}
\setcopyright{acmlicensed}\acmConference[WWW'23]{Proceedings of the ACM Web Conference 2023}{April 30-May 4, 2023}{Austin, TX, USA}
\acmBooktitle{Proceedings of the ACM Web Conference 2023 (WWW'23), April 30-May 4, 2023, Austin, TX, USA}
\acmPrice{15.00}
\acmDOI{10.1145/3543507.3583361}
\acmISBN{978-1-4503-9416-1/23/04}

\begin{document}
% \fancyhead{}

% \title{Self-Supervised Graph Transformer via Adaptive Masked Autoencoding for Recommendation}

% \title{Adaptive Masked Graph Transformer for Recommendation}

% \title{Contrastive Learning with Interest and Conformity Disentanglement for Sequential Recommendation}

\begin{CCSXML}
<ccs2012>
<concept>
<concept_id>10002951.10003317.10003347.10003350</concept_id>
<concept_desc>Information systems~Recommender systems</concept_desc>
<concept_significance>500</concept_significance>
</concept>
</ccs2012>
\end{CCSXML}
\ccsdesc[500]{Information systems~Recommender systems}

\keywords{Sequential Recommendation, Contrastive Learning, Popularity Bias}

\title{Debiased Contrastive Learning for Sequential Recommendation}

\author{Yuhao Yang}
\affiliation{The University of Hong Kong}
% \affiliation{
%   \institution{University of Hong Kong}
%   \city{Hong Kong}
%   \country{China}
% }
\email{yuhao-yang@outlook.com}

\author{Chao Huang}
\authornote{Chao Huang is the corresponding author.}
\affiliation{The University of Hong Kong}
% \affiliation{
%   \institution{University of Hong Kong}
%   \city{Hong Kong}
%   \country{China}
% }
\email{chaohuang75@gmail.com}

\author{Lianghao Xia}
\affiliation{The University of Hong Kong}
% \affiliation{
%   \institution{University of Hong Kong}
%   \city{Hong Kong}
%   \country{China}
% }
\email{aka\_xia@foxmail.com}

\author{Chunzhen Huang}
\affiliation{Wechat, Tencent}
% \affiliation{
%   \institution{Wechat, Tencent}
%   \city{Guangzhou}
%   \country{China}
% }
\email{chunzhuang@tencent.com}

\author{Da Luo}
\affiliation{Wechat, Tencent}
% \affiliation{
%   \institution{Wechat, Tencent}
%   \city{Guangzhou}
%   \country{China}
% }
\email{lodaluo@tencent.com}

\author{Kangyi Lin}
\affiliation{Wechat, Tencent}
% \affiliation{
%   \institution{Wechat, Tencent}
%   \city{Guangzhou}
%   \country{China}
% }
\email{plancklin@tencent.com}
%%
%% The abstract is a short summary of the work to be presented in the
%% article.
\renewcommand{\shortauthors}{Yuhao Yang et al.}
\begin{abstract}
Current sequential recommender systems are proposed to tackle the dynamic user preference learning with various neural techniques, such as Transformer and Graph Neural Networks (GNNs). However, inference from the highly sparse user behavior data may hinder the representation ability of sequential pattern encoding. To address the label shortage issue, contrastive learning (CL) methods are proposed recently to perform data augmentation in two fashions: (i) randomly corrupting the sequence data (\eg stochastic masking, reordering); (ii) aligning representations across pre-defined contrastive views. Although effective, we argue that current CL-based methods have limitations in addressing popularity bias and disentangling of user conformity and real interest. In this paper, we propose a new \underline{D}ebiased \underline{C}ontrastive learning paradigm for \underline{Rec}ommendation (\model) that unifies sequential pattern encoding with global collaborative relation modeling through adaptive conformity-aware augmentation. This solution is designed to tackle the popularity bias issue in recommendation systems. Our debiased contrastive learning framework effectively captures both the patterns of item transitions within sequences and the dependencies between users across sequences. Our experiments on various real-world datasets have demonstrated that \model\ significantly outperforms state-of-the-art baselines, indicating its efficacy for recommendation. To facilitate reproducibility of our results, we make our implementation of \model\ publicly available at: \url{https://github.com/HKUDS/DCRec}.

% to tackle the popularity bias issue. In \model, both intra-sequence item transition patterns and inter-sequence user dependencies are well captured with our conformity and interest disentanglement. We perform thorough experiments to demonstrate that \model\ significantly outperforms the state-of-the-art baselines over multiple real-world datasets. To make our results more reproducible, we release our PyTorch implementation of \model\ at: \url{https://github.com/HKUDS/DCRec}. 
\end{abstract}

% \keywords{datasets, neural networks, gaze detection, text tagging}

\maketitle

\section{Introduction}

The increasing complexity of source code poses a key challenge to the reliability of large-scale software systems. Software bugs in these systems can lead to safety issues~\cite{bug_safety} for users around the world as well as cause non-negligible financial losses~\cite{bug_loss}. As such, developers have to spend a large amount of time and effort on bug fixing. Consequently, \aprfull (\apr), designed to automatically generate patches to fix software bugs, has attracted wide attention from both academia and industry~\cite{long2016prophet, legoues2012genprog, long2015spr, lou2020can, tufano2018empstudy}. 


To achieve \apr, one popular approach is known as Generate-and-Validate (G\&V)~\cite{qi2015gv, ghanbari2019prapr, lou2020can, le2016hdrepair, legoues2012genprog, wen2018capgen, hua2018sketchfix, martinez2016astor, koyuncu2020fixminder, liu2019tbar, liu2019avatar}, which is typically based on the following pipeline: First, fault localization techniques~\cite{wong2016fl, abreu2007ochiai, zhang2013injecting, papadakis2015metallaxis, li2019deepfl, li2017transforming} are applied to determine the suspicious locations in programs where bugs are likely to exist. Then, the buggy locations are used by the \apr tools to generate a list of patches that replace buggy lines with correct lines. Afterward, each patch is validated against the original test suite to identify any \emph{plausible patches} (i.e., passing all tests in the test suite). Finally, to determine the \emph{correct patches}, developers examine the list of plausible patches to see if any of them can correctly fix the bug. 

Traditional \apr tools can mainly be categorized into heuristic-based~\cite{legoues2012genprog, le2016hdrepair, wen2018capgen}, constraint-based~\cite{mechtaev2016angelix, le2017s3, demacro2014nopol, long2015spr} and \template~\cite{ghanbari2019prapr, hua2018sketchfix, martinez2016astor, liu2019tbar, liu2019avatar}. Among these traditional tools, \template \apr tools~\cite{ghanbari2019prapr, liu2019tbar, benton2020effectiveness} have been able to achieve state-of-the-art results. \Template \apr tools typically leverage pre-defined templates (e.g., adding a nullness check) for bug fixing. However, since these fix templates are typically handcrafted, the number and types of bugs they are able to fix can be limited. 



To address the limitations of traditional \apr, researchers have proposed various \learning \apr tools~\cite{li2020dlfix, chen2018sequencer, jiang2021cure, lutellier2020coconut, zhu2021recoder, ye2022rewardrepair} based on the \nmtfull (\nmt) architecture~\cite{sutskever2014mt} where the input is the buggy code snippets and the goal is to translate the buggy code snippets into a fixed version. To accomplish this, \learning \apr tools require supervised training datasets with pairs of both buggy and fixed code snippets in order to learn how to perform this translation step. These training data are usually obtained by mining historical bug fixes using heuristics/keywords~\cite{dallmeier2007benchmark}, which can be imprecise for identifying bug-fixing commits; even the actual bug-fixing commits can include irrelevant code changes, leading to further pollution in the dataset~\cite{xia2022alpharepair}.
% 
Moreover, it can be hard for such \apr tools to generalize and fix bug types unseen during training. 



To better leverage recent advances in \plmfull{s} (\plm{s}), researchers~\cite{xia2022alpharepair, xia2023repairstudy, kolak2022patch, prenner2021codexws} have directly applied \plm{s} to generate patches without bug-fixing datasets. These \llm-based \apr tools work by either directly generating a complete code function~\cite{prenner2021codexws, xia2023repairstudy} or predict/infill the correct code snippet given its surrounding context~\cite{xia2022alpharepair, xia2023repairstudy}. By directly using \llm{s} that are pre-trained on billions of open-source code snippets, \llm-based \apr tools can achieve state-of-the-art performance on many repair datasets~\cite{xia2022alpharepair}. 


% 
%
%

Traditional \apr tools have long used the insight of the \emph{plastic surgery hypothesis}~\cite{barr2014plastic} where it states that the code ingredients to fix a bug already exist within the same project. Traditional \apr tools have manually designed pattern-~\cite{ghanbari2019prapr, saha2017elixir} or heuristic-based~\cite{jiang2018simfix, legoues2012genprog} approaches to finding and using such relevant code ingredients to generate fixes for bugs. However, the plastic surgery hypothesis has been largely ignored in \llm-based \apr. In fact, \llm provides a unique opportunity to fully automate the plastic surgery hypothesis idea via fine-tuning (learning project-specific information via model updates from the buggy project) and prompting (directly providing relevant code ingredients to the model), and make it directly applicable to different languages (since the \llm{s} are typically multi-lingual).%
Moreover, despite the intensive manual efforts involved, traditional \apr tools still cannot fully leverage project-specific information due to large search space for leveraging/composing existing code ingredients. In contrast, the project-specific information can effectively leveraged by \llm{s} due to their power in code understanding/vectorization, e.g., even partial/imprecise information may still guide \llm{s} in correct patch generation!
 To this end, we ask the question: \emph{How useful is the plastic surgery hypothesis in the era of \plm{s}}?








\mypara{Our Work.} To answer the question, we present \ourtech{\xspace} -- a \llm-based approach that automatically utilizes the plastic surgery hypothesis by systematically combining multiple fine-tuning and prompting strategies for \apr. \ourtech fine-tunes \plm{s} using two novel domain-specific training strategies: \textbf{\epfinetune} -- we fine-tune using the original buggy project by aggressively masking out a high percentage of tokens, which allows \plm to learn project-specific code tokens and programming styles; and \textbf{\rofinetune} -- which only masks out a single continuous code sequence per training sample, allowing the model to get used to the final \csapr task of predicting a single continuous code sequence. Furthermore, we directly leverage the ability for \plm{s} to understand natural language instructions and introduce a novel prompting strategy, \textbf{\idprompting}, which uses information retrieval and static analysis to obtain a list of relevant identifiers for the buggy lines. While such relevant identifiers are critical for fixing some difficult bugs, they may not be seen by the \llm during inference due to limited context window size. Through the use of prompting, we directly tell the model to use these extracted identifiers (relevant code ingredients) to generate the correct code. Finally, to perform repair, we combine all four model variants (including the base model, both fine-tuned models and the base model with prompting) for the final repair.





While our insight of leveraging the plastic surgery hypothesis for \llm-based \apr is generalizable across different types of \plm{s}, to implement \ourtech, we choose a recent \plm{\xspace}, \ctfive~\cite{wang2021codet5}, which is pre-trained on millions of open-source code snippets. \ctfive is an encoder-decoder model trained using \mspfull (\msp) objective where a percentage of tokens are masked out and each continuous masked token sequence is referred to as a masked span. Also, although we only extract relevant identifiers from the current buggy project (since this paper focuses on the plastic surgery hypothesis), our work can be easily extended to obtain other code information (such as relevant statements or functions) from other sources, such as  the massive pre-training corpora~\cite{husain2020codesearchnet} or historical bug-fixing datasets~\cite{jiang2019infer}, which can provide more coding knowledge for \llm{s}. Besides, although we mainly focus on using traditional string comparison algorithms for information retrieval in this paper, these techniques can be easily replaced by other frequency-based retrieval~\cite{robertson2009probabilistic} and neural search (or embedding-based search)~\cite{reimers2019sentence}.
  In summary, this paper makes the following contributions:


%


\begin{itemize}[noitemsep, leftmargin=*, topsep=0pt]
    \item \textbf{Dimension.} This paper is the first to revisit the important plastic surgery hypothesis in the era of \llm{s}. It opens up a new dimension for \llm-based \apr to incorporate previously neglected information from the buggy project itself to boost \apr performance. Furthermore, it demonstrates the promising future of retrieval-based prompting for modern \llm-based \apr.
    \item \textbf{Implementation.} We implement \ourtech based on the recent \ctfive model. We augment the model using two novel fine-tuning strategies: \epfinetune and \rofinetune, along with a novel prompting strategy based on information retrieval and static analysis: \idprompting. We combine the patches generated by all four models together and perform patch ranking to speed up \apr.% 
    \item \textbf{Evaluation Study.} We conduct an extensive evaluation against state-of-the-art \apr tools. On the widely studied \dfj 1.2 and 2.0 datasets~\cite{just2014dfj}, \ourtech is able to achieve the new state-of-the-art results of 89 and 44 correct bug fixes (15 and 8 more than best baseline) respectively.  Furthermore, we perform a broad ablation study to justify our design. \ourtech demonstrates for the first time that the plastic surgery hypothesis can substantially boost \llm-based \apr and advance state-of-the-art \apr, while being fully automated and general. Moreover, even partial/imprecise code ingredients may still effectively guide \llm{s} for \apr!
\end{itemize}


% \section{Proposed Framework: {\ourmodel}}
\label{model}


In this section, we introduce a novel self-supervised co-training framework {\ourmodel}.
The proposed framework is illustrated in Figure~\ref{fig:intro_model} and works in three phases.
Phase one automatically generates two sets of pseudo labels.
We use a combination of off-the-shelf pre-trained POS and NER taggers, knowledge graph, and GPT-2 scorer for generating the first set of pseudo labels automatically without any hand-crafted rules for matching the slot values.
The other set of pseudo labels is acquired through a zero-shot slot filling model~\cite{liu2020coach}, trained on the out-of-domain dataset.
It is critical to emphasize that both sets of labels are noisy and incomplete which poses serious challenges to training effective models for the task of open-domain slot filling.
Phase two fine-tunes the pre-trained BERT to the slot filling task that effectively transfers the knowledge from the pre-trained language model~(LM) to overcome the issue of label incompleteness to some extent. 
Further, we employ the early stopping technique to minimize the noise in the labels.
The output of this phase is two BERT models that can generate soft labels for self-supervision during co-training in phase three.
Phase three leverages the fine-tuned models and further trains them in an iterative fashion.
Specifically, the proposed peer training approach facilitates high-confidence soft label selection for the other peer to perform training. This phase progressively reduces the noise in the labels and enables effective model fitting. 



\subsection{Phase One: Automatic Label Generation}
To acquire the first set of labels, we perform the following steps.
First of all, off-the-shelf trained POS and NER taggers are used to predict initial estimates of the slot values irrespective of the slot types. Then, the type information of the slot values is queried from the KG and the slot value is tagged for the most appropriate slot in the target domain.
This approach, however, produces low recall. 
To expand the candidate slot values, we generate n-grams of the natural language text and employ a partial matching scheme to query the KG for type information (e.g., \myspecial{Jason} \myspecial{Aldean} = \myspecial{American} \myspecial{singer}) of the n-grams if the entry exists.
This process generates multiple overlapping hypotheses about the slot values.
We replace a span of text that corresponds to a slot value by its type information and a GPT-2 based scorer (see Section~\ref{sec:nlpmodels}) is used to select the best candidate based on the fluency of the text.
Naturally, if a token (or span of tokens) is replaced by its type, the sentence should score higher as compared to the case where an inappropriate substitution is performed. 
We select the best hypothesis if the score is greater than the threshold.
Intuitively, the candidate selection threshold can automatically be searched based on a small validation set from the target domain, making the label generation process fully automatic. 
The other set of noisy labels is acquired by the zero-shot slot filling model~\cite{liu2020coach} that has been trained using an out-of-domain dataset. It is important to highlight that the zero-shot slot filling model does not require any labeled in-domain training example. 
To summarize the automatic label generation phase, both sets of labels are acquired in a fully automatic fashion without any hand-crafting.


In contrast to previous work in weak supervision~\cite{ren2015clustype,he2017autoentity,fries2017swellshark,giannakopoulos2017unsupervised} that obtains a single set of noisy labels and then propose techniques to overcome the challenge of fitting an effective model to the noisy labels, we acquire two sets of complementary labels.
The choice of these two sets of labels is guided by the intuition that they should be complementary and the models trained on these sets of labels should be able to share complementary information with the other to improve the performance in the later phases of the framework.
Essentially, the first set of labels carries information from external knowledge sources, whereas the labels generated through the pre-trained zero-shot slot filling model capture how the slot values are mentioned in other domains.
%
To further elaborate on the motivation and our process for the first set of labels (i.e., labels using KG and other NLP models), the pre-trained LMs have been shown to have a great deal of knowledge~\cite{petroni2019language}, thus should be capable of generating automatic labels with no need of external KG. 
To the best of our knowledge, there exists no work that shows that accurate token-level automatic labeling (e.g., slot filling task) is possible with pre-trained LMs. 
Moreover, such approaches would require heavy prompting in each new target domain, whereas our label generation process is fully automatic and only relies on the readily-available pre-trained NLP models and external KG.

\subsection{Phase Two: LM-assisted Weak Supervision}
Since we do not have access to dataset $\{(\mathbf{X}_n,\mathbf{Y}_n)\}_{n=1}^N$ with true ground-truth labels.
We use pseudo labels generated in phase one, $\{(\mathbf{X}_n,\mathbf{D}_n)\}_{n=1}^N$, to learn 
$f_{m,c}(\cdot; \cdot)$ that outputs the probability of the $m$-th token to take on class $c$. 
We learn $f_{m,c}(\cdot; \cdot)$ by minimizing the following loss over the noisy dataset $\{(\mathbf{X}_n,\mathbf{D}_n)\}_{n=1}^N$: 
$$
\hat\theta = \argmin_{\theta}\frac{1}{N}\sum_{n=1}^{N} \ell(\mathbf{D}_n, f(\mathbf{X}_{n}; \theta)),
\label{eq:stage1}
$$
where $\ell(\mathbf{D}_n, f(\mathbf{X}_{n}; \theta)) = \frac{1}{M} \sum_{m=1}^{M} -\log{f_{m,d_{n, m}}(\mathbf{X}_{n}; \theta)}$. 
We employ the pre-trained multilingual BERT with token-level classification head that uses Adam optimizer \cite{kingma2014adam,Liu2019} with early stopping and multiple random initializations. 


Since slot filling task is similar to the MLM training objective of the BERT, we employ pre-trained BERT as the backbone model.
That is, MLM's goal is to predict the masked tokens using bidirectional contexts. Similarly, slot filling tries to predict the label for a token leveraging both left and right contexts simultaneously, which makes the pre-trained BERT an ideal model of choice that greatly facilitates minimizing incomplete labels.
It is important to highlight that our automatically generated labels are not only incomplete but also potentially wrong.
The training strategies employed in this phase minimize the noise in the label to some extent. 
Specifically, early stopping can provide a strong regularization and would not let the model overfit to the noisy labels, especially wrong labels. 
Moreover, early stopping does not let the model forget the knowledge in the pre-trained model.
Similarly, multiple random initializations enforce robustness. 
Since the model is fine-tuned on the noisy labels, averaging the predictions of multiple models for each token ensures that wrong labels end up with low probabilities and true labels consistently achieve high probabilities.
Using the above-mentioned strategies, we train two slot filling models, which we call the peers. The peer one is trained on the first set of pseudo labels that were generated using POS and NER taggers, KG, and the GPT-2 scorer in phase one. Similarly, peer two is trained using the predictions of the zero-shot slot filling model~\cite{liu2020coach}.
Both models have the same architecture and follow the same training procedures.

\begin{table*}[t!]
\centering
\caption{Dataset statistics.}
\vspace{-7pt}
\label{tab:dataset}
\begin{tabular}{lccccc}
\toprule
\textbf{Dataset}  & \textbf{Dataset Size} & \textbf{Vocab. Size} & \textbf{Avg. Length} & \textbf{\# of Domains} & \textbf{\# of Slots} \\ \hline
\textbf{SGD}      & 188K                  & 33.6K                & 13.8                 & 20                     & 240                  \\
\textbf{MultiWoZ} & 67.4K                 & 10.5K                & 13.3                 & 8                      & 61 \\
\bottomrule
\end{tabular}
\vspace{-7pt}
\end{table*}

\subsection{Phase Three: Self-supervised Co-training}
We introduce an iterative peer training algorithm where both peers generate high-confidence soft labels for training the other peer in the next iteration. 
Theoretically, these peers can be anything, but in this work, 
we explore two of the most promising directions that have shown the promise to minimize the need for manual labeling for the task: zero-shot learning and distant supervision.
This phase uses a self-supervised co-training scheme to exploit the patterns of slot values from other domains through the labels generated by the zero-shot filling model (i.e., peer two)~\cite{liu2020coach} as well as utilize the knowledge in external KGs and pre-trained models via labels provided by the peer one.
Specifically, we initialize the peers trained in phase two and use their pseudo labels to kick-start training in this phase.
Specifically, peer one $f_{m,c}(\cdot; \theta_{\textrm{p1}})$ would generate labels $\{\tilde{\mathbf{Y}}^{(t)}_n = [\tilde{y}_{n,1}^{(t)}, ..., \tilde{y}_{n,m}^{(t)}]\}_{n=1}^{N}$ for peer two $f_{m,c}(\cdot; \theta_{\textrm{p2}})$ at the $t$-th iteration by:
$$
\tilde{y}_{n,m}^{(t)} = \argmax_{c}{f_{m,c}(\mathbf{X}_n; \theta_{\textrm{p1}}^{(t)})}. 
\label{eq:pseudo}
$$

Based on these labels, the peer two can be fine-tuned by: 
$$
\hat\theta_{\textrm{p2}}^{(t+1)} = \argmin_{\theta}\frac{1}{N}\sum_{n=1}^N \ell(\tilde{\mathbf{Y}}_n^{(t)}, f(\mathbf{X}_{n}; \theta)).
\label{eq:self_train1}
$$

Similarly, peer two $f_{m,c}(\cdot; \theta_{\textrm{p2}})$ would generate pseudo labels for peer one $f_{m,c}(\cdot; \theta_{\textrm{p1}})$ that are used to fine-tune peer one. 
We also notice that it is beneficial to stop early during this phase as well, to improve the model fitting and gradually reduce the noise associated with the automatically generated labels.
Since pseudo labels are refined gradually in an iterative way, both peers can benefit from the knowledge contained within the labels of the other while avoiding overfitting.
Furthermore, as an alternative to pseudo labels, we also generate soft labels that are used for confidence re-weighting. 
The high-confidence soft label selection strategy enables better model fitting and efficient learning via better quality of the automatic labels.
Specifically, for the given $m$-th token in the $n$-th training example, the probability for all classes $C$ is $[f_{m,1}(\mathbf{X}_n;\theta),...,f_{m,C}(\mathbf{X}_n;\theta)]$. 
Following ~\cite{xie2016unsupervised}, at $t$-th iteration, peer one generates soft labels, $\{\mathbf{S}_n^{(t)} = [\mathbf{s}_{n,m}^{(t)}]_{m=1}^M \}_{n=1}^N$, as given below:
$$
\mathbf{s}_{n,m}^{(t)} = [s_{n,m,c}^{(t)}]_{c=1}^{C} = \Bigg[  \frac{f_{m,c}^2(\mathbf{X}_n;\theta_{\textrm{peer1}}^{(t)})/p_{c}}{\sum_{c'=1}^C f_{m,c'}^2(\mathbf{X}_n;\theta_{\textrm{peer1}}^{(t)})/p_{c'}}\Bigg]_{c=1}^{C}
\label{eq:soft}
$$ 
where $p_{c} = \sum_{n=1}^N \sum_{m=1}^M f_{m,c}(\mathbf{X}_n;\theta_{\textrm{p1}}^{(t)})$ computes the frequency of the tokens for the $c$-th class. 
Then, peer two $f(\cdot; \theta_{\textrm{p2}}^{(t+1)})$ is fine-tuned by:
$$
\theta_{\textrm{p2}}^{(t+1)} = \argmin_{\theta} \frac{1}{N} \sum_{n=1}^{N} \ell_{\rm KL}(\mathbf{S}_n^{(t)}, f(\mathbf{X}_{n}; \theta)),
$$
where $\ell_{\rm KL}(\cdot,\cdot)$ is the KL-divergence-based loss:
$$
\ell_{\rm KL}(\mathbf{S}_n^{(t)}, f(\mathbf{X}_{n}; \theta))=\frac{1}{M}\sum_{m=1}^M\sum_{c=1}^C - s_{n,m,c}^{(t)} \log f_{m,c}(\mathbf{X}_{n}; \theta).
\label{eq:klloss}
$$

Moreover, we also investigate selecting tokens that have high confidence. 
For instance, we pick high-confidence tokens from the $m$-th input example at the $t$-th iteration by  
$
H^{(t)}_n = \{m : \max_{c} s_{n,m,c}^{(t)} > \epsilon \},
$
where $\epsilon\in [0,1]$ is a threshold that can be searched based on a small validation set. 
Then, peer two $f(\cdot; \theta_{\textrm{p2}}^{(t+1)})$ is fine-tuned by:
$$
\theta_{\textrm{p2}}^{(t+1)} %&= \argmin_{\theta} \frac{1}{N} \sum_{n=1}^{N} \ell_{\rm S-KL}(\bS_n^{(t)}, f(\bX_{n}; \theta)) \\
= \argmin_{\theta} \frac{1}{N|H^{(t)}_n|}\sum_{n=1}^{N} \sum_{m\in H^{(t)}_n}\sum_{c=1}^C - s_{n,m,c}^{(t)} \log f_{m,c}(\mathbf{X}_{n}; \theta).
$$

This phase improves the robustness to effectively fit the model for tokens with high confidence. 
Both peers keep sharing information and their confidence by producing soft labels for their counterparts until they approximate to the true labels while employing early stopping and scheduled learning rates.
It is important to remind that phase three is the most important phase that progressively reduces noise from the labels to a great extent and enables superior performance for the task of open-domain slot filling.
\vspace{-0.1in}
\section{Methodology}
\label{sec:solution}

The overall model architecture of our \model\ is shown in Figure~\ref{fig:arch}.

\begin{figure*}
    \centering
    \includegraphics[width=\linewidth]{material/model_arc_.pdf}
    \vspace{-0.2in}
    \caption{The overall framework of \model. $\mathcal{G}_c$ and $\mathcal{G}_t$ are built to encode the sequences from diversified views (left part). In addition, we generate reasonable interaction-level conformity weights $\omega$ from the rich structure of $\mathcal{G}_t$ (right part). The weights are restrained in normal distribution and empower the cross-view contrastive learning to be adaptive and aware of conformity.}
    \label{fig:arch}
    \vspace{-0.15in}
\end{figure*}

\subsection{Task Formulation}
\noindent \textbf{Notations}. We suppose a recommender with a set of users and items denoted by $\mathcal{U} (u\in \mathcal{U})$ and $\mathcal{V} (v\in \mathcal{V})$, respectively. For each user, his/her engaged subset of items in a temporal order is defined as $\boldsymbol{s}_u=\left(v_1, v_2, \cdots, v_T\right)$. Here, $T$ is the sequence length which varies by users, and indexed by $t$, \ie $1\leq t \leq T$. Following settings in~\cite{bert4rec,iclrec}, we conduct the padding operation over different item sequences ($\boldsymbol{s}_u\ \in \mathcal{S}$) to mitigate the variable length.\\\vspace{-0.12in}

\noindent \textbf{Task}. 
Our objective is to develop a personalized ranking function that takes into account the past item sequences of a user, and predicts the next item ($v_{T+1}$) that the user is most likely to adopt.

% Given the past item sequences, our goal is to learn a personalized ranking function over all candidate items and predict the next item (\ie $v_{T+1}$) that the user is likely to adopt at the future step.

\vspace{-0.1in}
\subsection{Sequential Pattern Encoding}
As of now, Transformer has emerged as the dominant method for encoding sequences, capable of mapping temporally-ordered tokens from different types of sequential data to latent representation space. Examples include textual data~\cite{devlin2018bert} and electronic health data~\cite{poulain2021transformer}. Our sequential pattern encoder is built upon the Transformer, inspired by the effectiveness of this approach in modeling item sequence in~\cite{bert4rec,wu2020sse,yuan2022multi}. This allows us to incorporate temporal context into embeddings, resulting in an effective representation of the user's sequential behavior.

% To date, Transformer has become the most prevalent sequence encoding solution to project temporally-ordered tokens into latent representation space from various sequential data, such as textual data~\cite{devlin2018bert}, electronic health data~\cite{poulain2021transformer}, and user behaviors~\cite{yang2022getnext}. Inspired by the effectiveness of Transformer in modeling item sequence in~\cite{bert4rec,wu2020sse,yuan2022multi}, our sequential pattern encoder is built upon the Transformer, to incorporate temporal context into embeddings.

We start by adding a positional embedding $\mathbf{p}_v$ to the initial item representation $\mathbf{v}_v$ using the operation $\mathbf{h}_v^0 = \mathbf{v}_v \oplus \mathbf{p}_v$, which serves as the input item embedding $\mathbf{h}_v^0$ for the first block of Transformer. We represent each user's item sequence with an embedding matrix $\mathbf{H}_u^0 \in \mathbb{R}^{T \times d}$, where $T$ is the length of the sequence and $d$ is the dimension of the item embedding. The embedding matrix corresponds to the padded item sequence $\boldsymbol{s}_u$ of the user. To capture the correlations between items, we apply a self-attention layer with multi-head ($N$) channels to the user's item embedding matrix:
% Specifically, we first inject the positional embedding $\mathbf{p}_v$ of item $i$ into the initialized item representation $\mathbf{v}_v$ with the operation $\mathbf{h}_v^0 = \mathbf{v}_v \oplus \mathbf{p}_v$ as the input item embedding $\mathbf{h}_v^0$ for the first block of Transformer. We associate each user with an embedding matrix $\mathbf{H}_u^0 \in \mathbb{R}^{T \times d}$ corresponding to the padded item sequence $\boldsymbol{s}_u$. To capture the item-wise correlations, a self-attention layer with multi-head ($N$) channels is applied to user's item embedding matrix:
\begin{align}
    \text{MH}\left({\textbf{H}_u^\ell}\right) &= \left(\text{head}_1 \mathbin\Vert \text{head}_2 \mathbin\Vert  \cdots \mathbin\Vert \text{head}_N\right)\mathbf{W}^D \\
	\text{head}_n &= \text{Attention}\left( \textbf{H}_u^\ell \mathbf{W}^Q_n, \textbf{H}_u^\ell \mathbf{W}^K_n,  \textbf{H}_u^\ell \mathbf{W}^V_n \right),
\end{align}
\noindent $\mathbf{W}^Q_n, \mathbf{W}^K_n, \mathbf{W}^V_n \in \mathbb{R}^{d \times d/N}$ represents the head-specific mapping matrices corresponding to the query, key, value dimension, respectively. $\mathbf{W}^D \in \mathbb{R}^{d \times d}$ is a learnable projection matrix, and $\textbf{H}_u^\ell$ is the embedding matrix of user $u$'s sequence $\boldsymbol{s}_u$ at the $\ell$-th block of Transformer. Here, the self-attention calculation is conducted as: $\text{Attention}\left( \mathbf{Q},\mathbf{K},\mathbf{V}  \right) = \text{softmax}\left( \frac{\mathbf{Q}\cdot \mathbf{K}^\trans}{\sqrt{d/N}} \right)\mathbf{V}$. $\frac{d}{N}$ is the scale factor.

% $\mathbf{W}^Q_n, \mathbf{W}^K_n, \mathbf{W}^V_n \in \mathbb{R}^{d \times d/N}$ represents the head-specific mapping matrices corresponding to the query, key, value dimension, respectively. $\mathbf{W}^D \in \mathbb{R}^{d \times d}$ is a learnable projection matrix. $\textbf{H}_u^\ell$ is the embedding matrix of user $u$'s sequence $\boldsymbol{s}_u$ at the $\ell$-th block of Transformer. Here, the self-attention calculation is conducted as: $\text{Attention}\left( \mathbf{Q},\mathbf{K},\mathbf{V}  \right) = \text{softmax}\left( \frac{\mathbf{Q}\cdot \mathbf{K}^\trans}{\sqrt{d/N}} \right)\mathbf{V}$. $\frac{d}{N}$ is the scale factor.

To inject non-linearity into the embedding generation, a point-wise feed-forward network (FFN) is used for representation transformation within the sequential pattern encoder, which is defined:
% To inject the non-linearity into the embedding generation, we equip our sequential pattern encoder with a point-wise feed-forward network for representation transformation, which is defined as:
\begin{align}
    \label{eq:transformer}
    \text{PFFN}\left(\mathbf{H}_u^{\ell}\right) &= [\text{FFN}\left(\mathbf{h}_1^{\ell}\right)^\trans, \cdots, \text{FFN}\left(\mathbf{h}_T^{\ell}\right)^\trans] \\
	\text{FFN}\left(\mathbf{x}\right) &= \text{GELU}\left(\mathbf{x}\mathbf{W}_1^{\ell} + \mathbf{b}_1^{\ell}\right)\mathbf{W}_2^{\ell}+\mathbf{b}_2^{\ell},
\end{align}
\noindent where $\mathbf{W}_1^{\ell}, \mathbf{W}_2^{\ell}, \mathbf{b}_1^{\ell}, \mathbf{b}_2^{\ell}$ are learnable model parameters as projection and bias terms. $\text{GELU}(\cdot)$ is the activation function.

\subsection{Unifying Sequential and CF Views}
In real-life applications, long-tail sequences with a limited number of items are prevalent in recommendation scenarios~\cite{liu2020long,cl4srec}. These sequences pose challenges to most existing solutions. In particular, short sequences with very few items can hardly provide sufficient contextual signals for neural sequence encoders. This issue affects various types of models, such as self-attention mechanisms~\cite{bert4rec, sasrec}, and graph neural networks~\cite{gcsan, srgnn, mbht, surge}. To tackle the challenge of short sequences with very few items in sequential recommenders, we propose to unify the sequential view of item transitions and the collaborative view of user-item interactions. This design aims to capture the implicit cross-sequence user dependencies, allowing user-wise knowledge transfer in sequential recommender systems. This aspect is largely overlooked in most current solutions.


% In real-life sequential recommenders, long-tail sequence with limited number of items is prevalent in recommendation scenario~\cite{liu2020long,cl4srec}, which poses challenges to most of existing solutions. In particular, short sequences with very few items can hardly provide sufficient contextual signals for neural sequence encoders, such as recurrent neural network~\cite{gru4rec, gru4rec2}, self-attention mechanisms~\cite{bert4rec, sasrec}, and graph neural networks~\cite{gcsan, srgnn, mbht, surge}. To tackle this challenge, we propose to unify the sequential view of item transitions and the collaborative view of user-item interactions. With such design, our model can capture the implicit cross-sequence user dependencies to allow the user-wise knowledge transfer in sequential recommender, which are largely overlooked in most current solutions~\cite{zhang2022enhancing}.

To achieve the goal of unifying the sequential view of item transitions and the collaborative view of user-item interactions, you can start by generating two graphs: \emph{item transition graph} $\mathcal{G}_t$ and \emph{item co-interaction graph} $\mathcal{G}_c$. To be specific, $\mathcal{G}_t$ and $\mathcal{G}_c$ over the item set $\mathcal{V}$ are constructed by following the instructions below:
% Towards this end, we first generate the \emph{item transition graph} $\mathcal{G}_t$ and \emph{item co-interaction graph} $\mathcal{G}_c$ corresponding to the sequential, collaborative views, respectively. To be specific, $\mathcal{G}_t$ and $\mathcal{G}_c$ over the item set $\mathcal{V}$ are constructed by following the instructions below:
\begin{itemize}[leftmargin=*]
\item \textbf{Item Transition Graph $\mathcal{G}_t$}. To capture the transitional relationships among items from the sequential pattern view, adjacent item pairs (\eg $v_{t-1}$, $v_{t}$) in each sequence $\boldsymbol{s}_u$ are connected with an edge in $\mathcal{G}_t$. Given the item sequences of all users $\mathcal{S} = \{ \boldsymbol{s}_1, \boldsymbol{s}_2, \cdots, \boldsymbol{s}_{|U|} \}$, the adjacency matrix $\mathbf{A}_{\mathcal{G}_t} \in \mathbb{R}^{|\mathcal{V}| \times |\mathcal{V}|}$ representing the item correlations in graph $\mathcal{G}_t$ is generated by:
\begin{align}
\label{eq:gt}
    \mathbf{A}^u_{\mathcal{G}_t}(v_p, v_q) = \begin{cases}
                        1, & |p-q|=1 \\
                        0, & \text{otherwise}
    \end{cases} ;\quad
    \mathbf{A}_{\mathcal{G}_t} = \sum_{u=1}^{|U|}\mathbf{A}^u_{\mathcal{G}_t},
\end{align}
\noindent where $\mathbf{A}^u_{\mathcal{G}_t}$ denotes the user-specific item transition connections over sequence $\boldsymbol{s}_u$. Here, $p$ and $q$ denotes the position index in sequence. We sum up $\mathbf{A}^u_{\mathcal{G}_t}$ of all users ($u\in \mathcal{U}$) to obtain $\mathbf{A}_{\mathcal{G}_t}$. The adjacency matrix $\mathbf{A}_{\mathcal{G}t}$ takes into account the transition frequency between items with edge weights in the item transition graph. \\\vspace{-0.12in}

% Hence, the transition frequency between items are considered in $\mathbf{A}_{\mathcal{G}_t}$ to reflect the edge weights in the item transition graph $\mathcal{G}_t$.\\\vspace{-0.12in}

\item \textbf{Item Co-Interaction Graph $\mathcal{G}_c$}. To incorporate collaborative signals to model the cross-user dependencies, we generate another graph $\mathcal{G}_c$ to maintain the item correlations based on their co-interaction patterns. To this end, we firstly construct the interaction matrix $\mathbf{R} \in \mathbb{R}^{|\mathbf{U}| \times |\mathbf{V}|}$ between users and items by setting the entry $\mathbf{R}_{u, v}=1$ if user $u$ has adopted item $v$ and $\mathbf{R}_{u, v}=0$ otherwise. With the operation $\mathbf{A}_{\mathcal{G}_c} = \mathbf{R}^\trans\mathbf{R}$, we obtain the initial correlation strength between items in $\mathbf{A}_{\mathcal{G}_c}$ based on their co-interaction frequency. To filter out less-relevant item-wise connections, we apply \emph{top}-$k(\cdot)$ function to keep highly-relevant connections among items in $\mathbf{A}_{\mathcal{G}_c}$ based on top-$k$ co-interaction frequency of each item. Here, $k$ determines the density of $\mathbf{A}_{\mathcal{G}_c}$.

% In particular, there exists an edge between two items if they are adopted by the same user before. By constructing the interaction matrix $\mathbf{R} \in \mathbb{R}^{|\mathbf{U}| \times |\mathbf{V}|}$ between users and items 

\end{itemize}

After generating the item transition graph $\mathcal{G}_t$ and co-interaction graph $\mathcal{G}_c$, we utilize the graph neural network to project individual item into latent embedding space. Formally, our graph convolution-based message passing is presented as follows:
\begin{equation}
    \label{eq:gcn}
    \mathbf{X}^{(l+1)} = \left(\mathbf{D}_t^{-\frac{1}{2}} \mathbf{A}_{\mathcal{G}_t} \mathbf{D}_t^{-\frac{1}{2}}\right)\mathbf{X}^{(l)};\ 
    \mathbf{Z}^{(l+1)} = \left(\mathbf{D}_c^{-\frac{1}{2}} \mathbf{A}_{\mathcal{G}_c} \mathbf{D}_c^{-\frac{1}{2}}\right)\mathbf{Z}^{(l)}
\end{equation}
\noindent We let $\mathbf{X}^{(l)}$ and $\mathbf{Z}^{(l)}$ respectively denote the embedding matrix of items over the item transition graph ($\mathcal{G}_t$) and the co-interaction graph ($\mathcal{G}_c$) under the $l$-th graph layer. $\mathbf{D}_a$ and $\mathbf{D}_i$ are degree matrices used for graph normalizing. To simplify the model with lightweight GNN architecture, we remove the redundant transformation and activation operations during the message propagation.

\subsection{Adaptive Cross-View Contrastive Learning}
\label{sec:adaptive}
Building on the success of contrastive data augmentation across various domains, including vision learning~\cite{he2020momentum}, text mining~\cite{rethmeier2021primer}, and graph modeling~\cite{zhu2021graph}, our \model\ method harnesses self-supervised signals through contrastive learning across different item semantic views. Nonetheless, the popularity bias is often overlooked, as conformity can entangle real interests and subsequently influence user behaviors~\cite{zheng2021disentangling,chen2021autodebias}. For instance, a user might be influenced by conformity to click on a product or watch a short video, following the actions of others, rather than being genuinely interested in the content. If user interest and conformity are not disentangled when generating augmented signals, contrastive learning methods may focus on incorrect positive pairs, thereby introducing biased information. This can lead to less-interested recommendation.

% With the success of contrastive data augmentation in various domains, vision learning~\cite{he2020momentum}, text mining~\cite{rethmeier2021primer}, and graph modeling~\cite{zhu2021graph}, our \model\ method distills self-supervised signals with the contrastive learning across different item semantic views. However, the popularity bias is ignored when the conformity often entangles the real interests to influence user behaviors~\cite{zheng2021disentangling,chen2021autodebias}. For example, a user may follow others to click a product or view a short-video due to his/her conformity, rather than driven by the real interest. Without disentangling user interest and conformity in generating augmented signals, contrastive learning method may concentrate on the incorrect positive pairs and introduce biased information. 

Intuitively, conformity may vary across users and interactions. For example, user conformity and real interest might be entangled in a complex manner, jointly driving interaction behaviors. This complexity makes it challenging to accurately disentangle conformity from genuine interest, which is essential for providing more helpful augmented SSL signals. To address this challenge, we propose a debiased cross-view contrastive learning approach with adaptive augmentation that incorporates interaction-level conformity. We develop a multi-channel conformity weighting network (CWNet) to calculate the conformity degree of an interaction. By incorporating the estimated conformity degrees into our contrastive learning paradigm, we can adaptively determine the regularization strength. This allows the model to more effectively disentangle user interests from conformity behaviors.

% Intuitively, conformity may vary by users and interactions, \eg user conformity and real interest may be entangled in a complex way and jointly drive the interaction behaviors. To tackle this challenge, we propose a debiased cross-view contrastive learning approach with adaptive augmentation which incorporates the interaction-level conformity. To achieve this goal, we devise a multi-channel conformity weighting network (CWNet) to derive the conformity degree of an interaction. Then, the estimated conformity degrees are incorporated into our contrastive learning paradigm to determine the regularization strength in an adaptive manner.

\subsubsection{\bf Multi-Channel Conformity Weighting Network}
In our CWNet module, we aim to learn the conformity degree of an interaction between user $u$ and item $v$ from three semantic channels.\\\vspace{-0.12in}

\begin{itemize}[leftmargin=*]
\item (1) \textbf{ User-Specific Conformity Influence}. First, we propose to infer the interaction-level (\eg $u-v$) conformity degree by considering the conformity of user $u$ based on his/her past interactions. Given a user with strong conformity, their interactions are more likely to be influenced by popularity bias compared to others who exhibit strong individuality. To obtain the conformity degree of user $u$, we perturb the item transition graph $\mathcal{G}_t$ by removing the edges generated from $u$'s sequence $\boldsymbol{s}u$. This results in the generation of an augmented adjacency matrix $\bar{\mathbf{A}}{\mathcal{G}c}$, where $\bar{\mathbf{A}}^u{\mathcal{G}_t}(v_p, v_q) = 0$ for any two adjacent items $v_p$ and $v_q$ in $\boldsymbol{s}_u$. Subsequently, both the original and augmented item transition graphs are fed into our graph encoder (as per Eq.~\ref{eq:gcn}) to generate two embeddings ($\mathbf{x}_v, \mathbf{x}_v^\prime$) for the target item $v$. The user-specific conformity influence, denoted as $\omega^1_{\left(u,v\right)}$, is estimated using the cosine similarity between the two embeddings ($\mathbf{x}_v, \mathbf{x}v^\prime$), calculated as $\omega^{\alpha}{\left(u,v\right)} = \cos\left( \mathbf{x}_v, \mathbf{x}_v^\prime \right)$. A larger $\omega^u$ score indicates that user $u$'s interactions have little influence over the item graph structures, suggesting that their interaction patterns are more likely to be observed from others, \ie strong user conformity.  \\\vspace{-0.12in} 

\item (2) \textbf{Consistency with Other Users}. We also propose to calculate the conformity from the perspective of considering the sequential behavior consistency between the target user and others. In particular, for a given $u-v$ interaction, we compare the learned transitional patterns of user $u$ with those of other relevant users. To be specific, given the target item $v$, we aggregate the intra-sequence neighboring information using mean-pooling among inner neighbors within the sequence $\boldsymbol{s}_u$. The overall transitional patterns of other correlated users are combined to obtain $\overline{\mathbf{x}}_{O_v}$, which is derived from $v$'s outer neighbors $O_v$ across different user sequences. After that, the transition consistency is measured by $\omega^{\beta}_{\left(u,v\right)} = \cos\left(\overline{\mathbf{x}}_{N_v}, \overline{\mathbf{x}}_{O_v}\right)$. This measure quantifies the degree of consistency between the target user's sequential behavior and that of other users, providing insights into conformity. \\\vspace{-0.12in}

\item (3) \textbf{Subgraph Isomorphic Property}. The isomorphic property of item subgraph is also an important factor in reflecting user conformity with similar interaction patterns. To incorporate this factor into our conformity estimation, we calculate the similarity between item $v$'s embedding $\mathbf{x}_v$ and the representation $\overline{\mathbf{x}}_{O_v}$ aggregated from its outer neighbors, \ie $\omega^{\gamma}_{\left(u,v\right)} = \cos\left(\mathbf{x}_v, \overline{\mathbf{x}}_{O_v}\right)$.
\end{itemize}

\noindent \textbf{Mixing Signals from Different Channels.} We derive the final interaction-level conformity degree by fusing the information from the above three channels. Here, we first adopt mean-pooling over channel-specific results as: $\omega_{\left(u,v\right)} = \frac{1}{3}\sum_{\lambda \in \left\{\alpha,\beta,\gamma \right\}}\omega_{\left(u,v\right)}^{\lambda}$. Following the mapping strategy in~\cite{kgcl,zhu2021graph}, we perform the transformation for $\omega$ values as follows:
\begin{equation}
\label{eq:omega}
    \omega^{(1)} = \text{sigmoid}\left(\omega\right);\ 
    \omega^{(2)} = \frac{\omega^{(1)} - \omega_{min}^{(1)}}{\omega_{max}^{(1)} - \omega_{min}^{(1)}};\ 
    \omega^{(3)} = \frac{\mu_c}{\overline{\omega}^{(2)}} \cdot \omega^{(2)}
\end{equation}
\noindent $\mu_c$ is the hyperparameter that adjusts the mean value $\overline{\omega}$ of $\omega$. We omit the subscript $\left(u,v\right)$ for simplicity and adopt $\omega = \omega^{(3)}$ as the output conformity. Furthermore, to approximate the conformity degrees with normal distribution, we adopt the KL-divergence over the derived conformity results of all interactions:
\begin{equation}
    \label{eq:kl}
    \mathcal{L}_w = \sum_{i=1}^{|\{(u,v)\}|} \phi_i\log\frac{\phi_i}{\omega_i},
\end{equation}
\noindent where $\phi_i$ is generated by random sampling from normal distribution with the hyperparameter $\mu_c$ for the mean and $\sigma$ for the standard deviation. $\omega_i$ is the conformity weighting result.

\subsubsection{\bf Conformity-aware Contrastive Augmentation}
To enhance our \model\ with adaptively debiased augmentation, we integrate the conformity factor into our embedding contrasting paradigm to determine the agreement regularization strength. As discussed before, both sequential and collaborative views are generated through different encoders, \ie Transformer and GNNs. Our \model\ employs contrastive learning (CL) to learn conformity-aware augmented representations from two key dimensions:\\\vspace{-0.12in}

\noindent \textbf{Contrasting from User Dimension}. The first stage of our CL paradigm aims to realize the knowledge transfer across different users. By contrasting user-specific preferences with cross-user global interaction patterns, the learned augmented representations can naturally preserve user-wise implicit dependencies. In this process, the conformity regularizer weakens the impacts of perturbations caused by popularity bias for SSL augmentation. Given the embedding $\mathbf{h}_v$ and $\mathbf{x}_v$ encoded generated by our sequential pattern encoder (Eq.~\ref{eq:transformer}) and transition graph encoder (Eq.~\ref{eq:gcn}), respectively, our debiased contrastive learning paradigm is given as follows:
\begin{equation}
\label{eq:cl1}
    \mathcal{L}_u = -\sum_{u\in \mathcal{U}}\sum_{v \in \boldsymbol{s}_u}  \omega_{\left(u,v\right)} \log \frac{\exp\left(\cos\left(\mathbf{h}_v, \mathbf{x}_v\right)/\tau\right)}{\sum_{v^\prime\in \mathcal{V}} \exp\left(\cos\left(\mathbf{h}_v, \mathbf{x}_{v^\prime}\right)/\tau\right)},
\end{equation}
\noindent In the SSL loss $\mathcal{L}_u$, InfoNCE~\cite{infonce} is adopted for embedding contrasting. By incorporating our learned conformity factor $\omega$, we allow representations $\mathbf{h}_v$ and $\mathbf{x}_v$ to supervise each other adaptively, that is, weighted by the interaction-level conformity. \\\vspace{-0.12in}

% With the control of our learned conformity $\omega$, we let representations $\mathbf{h}_v$ and $\mathbf{x}_v$ supervise each other in an adaptive manner, \ie weighed by the interaction-level $\left(u,v\right)$ conformity context.\\\vspace{-0.12in}

\noindent \textbf{Contrasting from Item Dimension}. The goal of our second stage CL is to extract self-supervision signals by contrasting the global item embedding $\mathbf{x}_v$ with the item semantic representation $\mathbf{z}_v$. Our conformity factor $\omega_{u,v}$ is incorporated into this contrasting process by estimating the uniformity $\psi_{\left(u,v\right)}=1-\omega_{\left(u,v\right)}$. Formally, our item dimension CL loss $\mathcal{L}_v$ is defined as follows:
\begin{equation}
\label{eq:cl2}
    \mathcal{L}_v = -\sum_{u\in \mathcal{U}}\sum_{v \in \boldsymbol{s}_u}  \psi_{\left(u,v\right)} \log \frac{\exp\left(\cos\left(\mathbf{x}_v, \mathbf{z}_v\right)/\tau\right)}{\sum_{v^\prime\in \mathcal{V}} \exp\left(\cos\left(\mathbf{x}_v, \mathbf{z}_{v^\prime}\right)/\tau\right)},
\end{equation}
\noindent In our CL paradigm, instance self-discrimination~\cite{sgl,xia2022hypergraph} is used for generating positive pairs. Representation of different samples are pushed apart as negative pairs to reflect embedding uniformity.

\subsection{Model Training and Prediction}
In the training phase, the last interacted item of each sequence $\boldsymbol{s}_u$ is regarded as the label for model optimization. In the prediction phase, to encourage the cooperation between sequence and collaborative views, we combine view-specific item embeddings into an aggregated representation $\mathbf{p}_v$ with the learnable attentive weights:
\begin{align}
\label{eq:fusion}
    f\left(\mathbf{e}\right) = \frac{\exp\left(\mathbf{a}^\trans \cdot \mathbf{W}_a \mathbf{e}\right)}{\sum_{i=1}^3 \exp\left(\mathbf{a}^\trans \cdot \mathbf{W}_a \mathbf{e}\right)};~~~ \mathbf{p}_v = \sum_{\mathbf{e}\in \left\{\mathbf{h},\mathbf{x},\mathbf{z} \right\}}f\left(\mathbf{e}\right)\mathbf{e}
\end{align}
\noindent where $\mathbf{a}\in\mathbb{R}^{d}$ and $\mathbf{W}_a\in\mathbb{R}^{d\times d}$ are trainable attention parameters. Input embedding $\mathbf{e}$ is selected from the set of view-specific representations, \ie $\mathbf{e} \in \left\{\mathbf{h}_v, \mathbf{x}_v, \mathbf{z}_v \right\}$. $\mathbf{p}_v$ is derived through attentive aggregation with the view-specific importance score $f\left(\mathbf{e}\right)$.

The next item interaction probability $\hat{y}_{u,v}$ is derived as $\hat{y}_{u,v} = \mathbf{p}_{|\boldsymbol{s}_u|}^\trans\mathbf{v}$, where we adopt hidden states of the last item on the sequence as the user embedding. For each user and the ground truth item $v_t$ pair, we utilize the cross-entropy as the loss:
\begin{equation}
    \mathcal{L}_{rec} = \sum_{(u,v_{T+1})\in \mathcal{D}^+} -\log\frac{\exp \hat{y}_{u,v_{T+1}}}{\sum_{v^\prime \in \mathcal{V}}\exp\hat{y}_{u,v^\prime}},
\end{equation}
\noindent where $\mathcal{D}^+$ is the training data set of positive interactions at the $T+1$ timesteps.
To supplement the recommendation loss $\mathcal{L}_{rec}$ with our augmented SSL tasks under a multi-task training framework, we define our joint optimized objective $\mathcal{L}$ as:
\begin{equation}
\label{eq:all}
    \mathcal{L} = \mathcal{L}_{rec} + \lambda_1\left(\mathcal{L}_u+\mathcal{L}_v\right) + \lambda_2\left(\mathcal{L}_w\right),
\end{equation}
\noindent where $\lambda_1$ and $\lambda_2$ are parameters to balance the tasks-specific loss. $\mathcal{L}_w$ is the regularization term with KL-divergence for mixing signals (Eq.~\ref{eq:kl}) in our multi-channel conformity weighting network.\\\vspace{-0.12in}

\noindent \textbf{Time Complexity Analysis}. In our sequential pattern encoder, the computational cost is $O\left(T^2d+Td^2\right)$ where the majority of the cost  is attributed to the item-wise self-attention operations. In our GNN encoder, the graph convolutional message passing and aggregation have a complexity of $O\left(|\mathcal{V}|d^2\right)$. In the cross-view representation aggregation, our \model\ requires a computational cost of $O(d^2)$ for attentional weighting. Owing to the independent nature of our sequential and collaborative relation encoders, the Transformer and GNN encoding can be performed in parallel using the CUDA infrastructure for speeding up computation. In summary, the time complexity of our \model\ is $O\left(\left(|\mathcal{V}|+1\right)d^2\right)$, making it comparable to the state-of-the-art GNN-based sequential recommenders. 

% which is largely dominated by the item-wise self-attention operations~\cite{sasrec}. In our GNN encoder, the graph convolutional message passing and aggregation takes $O\left(|\mathcal{V}|d^2\right)$ complexity. For the cross-view representation aggregation, our \model\ needs $O(d^2)$ cost for attentional weighting. Due to the independent property of our sequential and collaborative relation encoders, Transformer and GNN encoding can be performed in a parallel way using CUDA infrastructure. Overall, the time complexity of our \model\ is $O\left(\left(|\mathcal{V}|+1\right)d^2\right)$, which is comparable to GNN-based state-of-the-art sequential recommenders.

\vspace{-0.1in}
\subsection{Theoretical Analyzes of \model}
\begin{figure}
    \centering
    \includegraphics[width=0.9\linewidth]{material/theo_case.pdf}
    \vspace{-0.2in}
    \caption{Upper part: curve of $0.5f(p)$ and $0.5f(n)$ under $\tau=0.4$. Lower part: distribution area of potential values of $\omega \cdot f(p)$ and $\omega \cdot f(n)$ and random samples within a batch.}
    \label{fig:theo}
    \vspace{-0.2in}
\end{figure}

In this section, we provide an analysis of how the new conformity-aware contrastive learning paradigm benefits the recommendation task. We focus on how to bring theoretical interpretability for the conformity-aware adaptive contrastive learning in Equation~\ref{eq:cl1}-\ref{eq:cl2}. We take Equation~\ref{eq:cl1} for studying because of the symmetry of these two equations. Following the discussion in~\cite{kgcl, sgl, khosla2020supervised}, the gradient of the contrastive objective in Equation~\ref{eq:cl1} can be expressed as:
\begin{equation}
\label{eq:contrib1}
    \nabla \mathcal{L}_u^{(u,v)} = \frac{1}{\tau \|\mathbf{h}_v\|}\left(c(v)+\sum_{v^\prime\in V\setminus\{p\}}c(v^\prime)\right),
\end{equation}
\noindent where $\mathcal{L}_u^{(u,v)}$ is the contrastive loss $\mathcal{L}_u$ for an user-item pair $(u,v)$. $c(v)$ and $c(v^\prime)$ are the gradient contribution from the positive pair $(\mathbf{h}_v,\mathbf{x}_v)$ and the negative pair, respectively. Formally, $c(v)$ and $c(v^\prime)$ are derived using the following formulas:
\begin{equation}
\label{eq:contrib2}
    \begin{aligned}
    c(v) &= \left(\mathbf{\bar{x}}_v - \left(\mathbf{\bar{h}}_v^\trans\mathbf{\bar{x}}_v\right)\mathbf{\bar{h}}_v\right)^\trans\left(P_{vv}-1\right)\\
    c(v^\prime) &= \left(\mathbf{\bar{x}}_{v^\prime} - \left(\mathbf{\bar{h}}_v^\trans\mathbf{\bar{x}}_{v^\prime}\right)\mathbf{\bar{h}}_v\right)^\trans P_{vv^\prime},
    \end{aligned}
\end{equation}
\noindent where $P_{vi} = \exp\left(\mathbf{\bar{h}}_v^\trans\mathbf{\bar{x}}_i / \tau\right) \big/ \sum_{i\in V\setminus\{v\}}\exp\left(\mathbf{\bar{h}}_v^\trans\mathbf{\bar{x}}_i / \tau \right)$. $\mathbf{\bar{x}}, \mathbf{\bar{h}}$ are normalized representations. To this end, we can derive two functions $f(p)$ and $f(n)$ that are proportional to the $L_2$ norm of $c(v)$ and $c(v^\prime)$~\cite{sgl}. Specifically, we have the following derivations:
\begin{equation}
\label{eq:contrib3}
    f_1(p) = \sqrt{1-p^2}\left(\exp\left(\frac{p}{\tau}\right)-1\right);\ f_2(n) = \sqrt{1-n^2}\left(\exp\frac{n}{\tau}\right)
\end{equation}
\noindent where $p = \mathbf{h}_v^\trans\mathbf{x}_v$ is the agreement between the positive pair. $n = \mathbf{h}_v^\trans\mathbf{x}_v^\prime$ denotes the similarity between the negatives. To visualize the impact of $\mathcal{L}_u$ without adaptive weight $\omega$, we plot the curve of $0.5f_1(p)$ and $0.5f_2(n)$ in Figure~\ref{fig:theo}. Note that without $\omega$, the coefficient of $\mathcal{L}_u$ is 0.5 by default. From the curves, it is obvious that the contribution of positive and negative samples at different similarity levels are fixed. This means that the model has difficulty in discriminating among diverse samples. In the context of an interest-driven interaction, it is crucial to dynamically reduce the influence of samples from the conformity modeling view.

% and $n = \mathbf{h}_v^\trans\mathbf{x}_v^\prime$ are the agreement between the positive pair and the similarity between the positive and the negative. To visualize the impact of $\mathcal{L}_u$ without adaptive weight $\omega$, we plot the curve of $0.5f_1(p)$ and $0.5f_2(n)$ in Figure~\ref{fig:theo}. Note that without $\omega$, the coefficient of $\mathcal{L}_u$ is 0.5 by default. From the curves, it is obvious that the contribution of positive and negative samples at different similarity levels are fixed. That is, the model lacks the ability to discriminate among diverse samples. Specifically, for an interest-driven interaction, impact of samples from the conformity modeling view should be dynamically weighted down. 

At this stage, we investigate the advantages of introducing a conformity-aware weight (denoted by $\omega$) in contrastive learning. Specifically, the conformity-aware weight $\omega$ influences the learning process by directly scaling the gradient values. Recall that the distribution of $\omega$ is restrained by normal distribution in Equation~\ref{eq:kl}. The distribution range of $\omega\cdot f_1(p)$ and $\omega\cdot f_2(n)$ creates an area rather than a single curve. We further plot the distribution areas in the lower part of Figure~\ref{fig:theo}. The values are weighted by the interaction-level conformity, falling within the ranges of $(0,f_1(p))$ and $(0,f_2(n))$ following a normal distribution. We also plot the discrete distribution of $\omega\cdot f_1(p)$ and $\omega\cdot f_2(n)$ by sampling two batches of training data. As evident from the distributions, the effect of some samples is enhanced while the influence of others are weakened. This endows the learning process with richer semantics, allowing for a dynamic and adaptive contribution of samples to the contrastive learning gradients with data debiasing. The analyzes also apply to $\mathcal{L}_v$ in Equation \ref{eq:cl2}, since $\gamma = 1-\omega$ has similar properties.

% To this stage, we study what benefits does the introduction of the conformity-aware weight $\omega$ brings. Specifically, $\omega$ weights the contribution to the gradient by directly scaling the gradient values. Recall that the distribution of $\omega$ is restrained by normal distribution in Equation~\ref{eq:kl}. The distribution range of $\omega\cdot f_1(p)$ and $\omega\cdot f_2(n)$ forms an area instead of a curve. We further plot the distribution areas in the lower part of Figure~\ref{fig:theo}. The values are weighted by the interaction-level conformity, range in $(0,f_1(p))$ and $(0,f_2(n))$ following normal distribution. We also plot the discrete distribution of $\omega\cdot f_1(p)$ and $\omega\cdot f_2(n)$ by sampling two batches of training data. Evidently, the effect of some samples is enhanced while some others are weakened. 

\section{Evaluation}
\label{sec:eval}

We conduct experiments from different aspects to validate the efficacy of the propose \model\ framework. The implementation details for our \model\ and the baseline methods are presented in~\ref{sec:implement}. Our experiments aim to answer the following research questions:
\begin{itemize}[leftmargin=*]
    \item \textbf{RQ1}: How does the proposed \model\ perform on different experimental datasets in comparison to state-of-the-art baselines?
    \item \textbf{RQ2}: How does different sub-modules of the proposed \model\ framework contribute to the overall performance?
    \item \textbf{RQ3}: How scalabile is \model\ in handling large-scale data?
    \item \textbf{RQ4}: How does the model performance vary when tuning important hyperparameters of the proposed \model\ model?
    \item \textbf{RQ5}: How can our \model\ model address the over-smoothing issue compared with GNN-based recommendation methods?
\end{itemize}

\subsection{Experimental Settings}
\subsubsection{\bf Experimental Datasets}
\begin{table}[t]
    \centering
    \caption{Statistics of the experimental datasets.}
    \label{tab:datasets}
    \small
    \vspace{-0.18in}
    \begin{tabular}{ccccc}
        \toprule
        Dataset & \# Users & \# Items & \# Interactions & Interaction Density \\
        \midrule
        Gowalla & 25,557 & 19,747 & 294,983 & $5.85\times 10^{-4}$\\
        Yelp & 42,712 & 26,822 & 182,357 & $1.59\times 10^{-4}$\\
        Amazon & 76,469 & 83,761 & 966,680 & $1.51\times 10^{-4}$\\
        % Tmall & 805,506 & 584,050 & 39,183,700 & $8.33\times 10^{-5}$\\
        \bottomrule
    \end{tabular}
    \vspace{-0.15in}
    \Description{A table showing the statistics of the Gowalla data (25557 users, 19747 items, 294983 interactions), the Yelp data (42712 users, 26822 items, 182357 interactions), and the Amazon data (76469 users, 83761 items, 966680 interactions).}
\end{table}


\begin{table*}[h]
\vspace{-0.1in}
\caption{Performance comparison on Gowalla, Yelp, and Amazon datasets in terms of \textit{Recall} and \textit{NDCG}.}
\vspace{-0.15in}
\centering
%\ssmall
% \scriptsize
\footnotesize
%\small
\setlength{\tabcolsep}{1.2mm}
\begin{tabular}{|c|c|c|c|c|c|c|c|c|c|c|c|c|c|c|c|c|c|l|}
\hline
Data & Metric & BiasMF & NCF & AutoR & PinSage & STGCN & GCMC & NGCF & GCCF & LightGCN & DGCF & SLRec & NCL & SGL & HCCF & \emph{\model} & p-val.\\
\hline
\multirow{4}{*}{Gowalla}
&Recall@20 & 0.0867 & 0.1019 & 0.1477 & 0.1235 & 0.1574 & 0.1863 & 0.1757 & 0.2012 & 0.2230 & 0.2055 & 0.2001 & 0.2283 & 0.2332 & 0.2293 & \textbf{0.2434} & $2.1e^{-8}$\\
&NDCG@20 & 0.0579 & 0.0674 & 0.0690 & 0.0809 & 0.1042 & 0.1151 & 0.1135 & 0.1282 & 0.1433 & 0.1312 & 0.1298 & 0.1478 & 0.1509 & 0.1482 & \textbf{0.1592} & $1.2e^{-9}$\\
\cline{2-18}
&Recall@40 & 0.1269 & 0.1563 & 0.2511 & 0.1882 & 0.2318 & 0.2627 & 0.2586 & 0.2903 & 0.3181 & 0.2929 & 0.2863 & 0.3232 & 0.3251 & 0.3258 & \textbf{0.3399} & $2.4e^{-8}$\\
&NDCG@40 & 0.0695 & 0.0833 & 0.0985 & 0.0994 & 0.1252 & 0.1390 & 0.1367 & 0.1532 & 0.1670 & 0.1555 & 0.1540 & 0.1745 & 0.1780 & 0.1751 & \textbf{0.1865} & $1.7e^{-9}$\\
\hline

\multirow{4}{*}{Yelp}
&Recall@20 & 0.0198 & 0.0304 & 0.0491 & 0.0510 & 0.0562 & 0.0584 & 0.0681 & 0.0742 & 0.0761 & 0.0700 & 0.0665 & 0.0806 & 0.0803 & 0.0789 & \textbf{0.0823} & $3.7e^{-4}$\\
&NDCG@20 & 0.0094 & 0.0143 & 0.0222 & 0.0245 & 0.0282 & 0.0280 & 0.0336 & 0.0365 & 0.0373 & 0.0347 & 0.0327 & 0.0402 & 0.0398 & 0.0391 & \textbf{0.0414} & $3.8e^{-5}$\\
\cline{2-18}
&Recall@40 & 0.0307 & 0.0487 & 0.0692 & 0.0743 & 0.0856 & 0.0891 & 0.1019 & 0.1151 & 0.1175 & 0.1072 & 0.1032 & 0.1230 & 0.1226 & 0.1210 & \textbf{0.1251} & $4.8e^{-3}$\\
&NDCG@40 & 0.0120 & 0.0187 & 0.0268 & 0.0315 & 0.0355 & 0.0360 & 0.0419 & 0.0466 & 0.0474 & 0.0437 & 0.0418 & 0.0505 & 0.0502 & 0.0492 & \textbf{0.0519} & $2.4e^{-4}$\\
\hline

\multirow{4}{*}{Amazon}
&Recall@20 & 0.0324 & 0.0367 & 0.0525 & 0.0486 & 0.0583 & 0.0837 & 0.0551 & 0.0772 & 0.0868 & 0.0617 & 0.0742 & 0.0955 & 0.0874 & 0.0885 & \textbf{0.1067} & $1.1e^{-10}$\\
&NDCG@20 & 0.0211 & 0.0234 & 0.0318 & 0.0317 & 0.0377 & 0.0579 & 0.0353 & 0.0501 & 0.0571 & 0.0372 & 0.0480 & 0.0623 & 0.5690 & 0.0578 & \textbf{0.0734} & $7.0e^{-12}$\\
\cline{2-18}
&Recall@40 & 0.0578 & 0.0600 & 0.0826 & 0.0773 & 0.0908 & 0.1196 & 0.0876 & 0.1175 & 0.1285 &0.0912 & 0.1123 & 0.1409 & 0.1312 & 0.1335 & \textbf{0.1535} & $6.6e^{-10}$\\
&NDCG@40 & 0.0293 & 0.0306 & 0.0415 & 0.0402 & 0.0478 & 0.0692 & 0.0454 & 0.0625 & 0.0697 & 0.0468 & 0.0598 & 0.0764 & 0.0704 & 0.0716 & \textbf{0.0879} & $2.0e^{-12}$\\
\hline
\end{tabular}
\vspace{-0.1in}
\label{tab:overall_performance}
\Description{A table presenting the evaluated performance of the proposed \model\ model and the baselines, in which \model\ significantly outperforms the baseline methods.}
\end{table*}

% atings are transformed into binary implicit feedback following~\cite{he2020lightgcn}. We filter users and items with less than 3 interactions, 

Three benchmark datasets collected from real-world online services are used to evaluate the performance of \model. Data statistics are shown in Table~\ref{tab:datasets}. We split the interaction data into training set, validation set and test set with 70\%:5\%:25\%. Details of the experimental datasets are:
\begin{itemize}[leftmargin=*]
    \item \textbf{Gowalla}: This dataset is collected from Gowalla, including user check-in records at geographical locations, from Jan to Jun, 2010.
    \item \textbf{Yelp}: This dataset contains users' ratings on venues, collected from Yelp platform. The time range is from Jan to Jun, 2018.
    \item \textbf{Amazon}: This dataset is composed of users' rating behaviors over books collected from Amazon platform, during 2013.
\end{itemize}

\vspace{-0.1in}
\subsubsection{\bf Evaluation Protocols}
Following previous works on CF recommenders~\cite{wang2019neural, xia2022self}, we conduct all-rank evaluation, in which positive items from test set are ranked with all un-interacted items for each user. The widely-used \emph{Recall@N} and \emph{NDCG@N} metrics~\cite{wu2021self,2021knowledge} are used adopted for evaluation, where $N=20$ by default.

\vspace{-0.05in}
\subsubsection{\bf Baseline Models}
We compare \model\ with the following 14 baselines from 4 research lines for comprehensive validation.
\\\noindent\textbf{Traditional Collaborative Filtering Technique:}
\begin{itemize}[leftmargin=*]
    \item \textbf{BiasMF}~\cite{koren2009matrix}: It is a classic matrix factorization approach that combines user/item biases with learnable embedding vectors.
\end{itemize}
\textbf{Non-GNN Neural Collaborative Filtering}:
\begin{itemize}[leftmargin=*]
    \item \textbf{NCF}~\cite{he2017neural}: It is an early study of deep learning CF model that enhances the user-item interaction modeling with MLP networks.
    \item \textbf{AutoR}~\cite{sedhain2015autorec}: This method applies a three-layer autoencoder with fully-connected layers to encode user interaction vectors.
\end{itemize}
\textbf{Graph Neural Architectures for Collaborative Filtering}:
\begin{itemize}[leftmargin=*]
    \item \textbf{PinSage}~\cite{ying2018graph}: This method combines random walk with graph convolutions for web-scale graph in recommendation.
    \item \textbf{STGCN}~\cite{zhang2019star}: This method augments GCN with autoencoding sub-networks on hidden features for better inductive inference.
    \item \textbf{GCMC}~\cite{berg2017graph}: This is a representative work to introduce graph convolutional operations into the matrix completion task.
    \item \textbf{NGCF}~\cite{wang2019neural}: It is a GNN-based CF method which conducts graph convolutions on the user-item interaction graph for embeddings.
    \item \textbf{GCCF}~\cite{chen2020revisiting} and \textbf{LightGCN}\cite{he2020lightgcn}: These two methods propose to simplify conventional GCN structures by removing transformations and activations for improving performance.
\end{itemize}
\textbf{Disentangled GNN-based Collaborative Filtering}:
\begin{itemize}[leftmargin=*]
    \item \textbf{DGCF}\cite{wang2020disentangled}: This method disentangles user-item interactions into multiple hidden factors in the graph message passing process.
\end{itemize}
\textbf{Self-Supervised Learning Approaches for Recommendation}:
\begin{itemize}[leftmargin=*]
    \item \textbf{SLRec}~\cite{yao2021self}: This method applies contrastive learning to recommendation models with feature-level data augmentations.
    \item \textbf{NCL}~\cite{lin2022improving}: This approach enhances self-supervised graph CF models with enriched neighbor-wise contrastive learning.
    \item \textbf{SGL}~\cite{wu2021self}: It conducts various types of graph augmentations and feature augmentations with graph contrastive learning for CF.
    \item \textbf{HCCF}~\cite{xia2022hypergraph}: This method augments GNN-based CF with a global hypergraph GNN and conducts cross-view contrastive learning.
\end{itemize}

\subsection{Overall Performance Comparison (RQ1)}


The overall performance of \model\ and the baselines are shown in Table~\ref{tab:overall_performance}. From the results we have the following observations: \vspace{-0.05in}
\begin{itemize}[leftmargin=*]
    \item Our \model\ consistently achieves best performance compared to baselines methods. Also, we re-train \model\ and the best-performed baselines (\ie, SGL and NCL) for 5 times to calculate $p$-values. The experimental results validate the significance of the improvement by \model. Compared to the state-of-the-art GNN methods, the MLP-based inference model of our graph-less \model\ generates more accurate recommendation results, due to its adaptive contrastive knowledge distillation. Specifically, the dual-level KD in \model\ enables enriched and adaptive high-order smoothing, which not only distills the accurate dark knowledge in the well-trained GNN teacher, but also avoids being affected by the over-smoothing signals. Furthermore, the adaptive contrastive regularization automatically alleviates the over-smoothing effects, which further boosts the performance. \\\vspace{-0.12in}
    
    \item While the self-supervised learning schema greatly improves the performance of GNN-based CF, our graph-less \model\ model still significantly outperforms the SSL-enhanced graph models. We attribute the performance deficiency to the inherent incapability of existing SSL frameworks in filtering over-smoothing signals. For example, SGL augments model training by introducing random noises, which may even aggravate the inaccuracy in node embeddings when the noises are magnified through high-order graph propagation. As for NCL and HCCF, they seek to connect nodes based on global semantic relatedness, which may even over-smooth nodes distant from each other in the original graph. In comparison, our graph-less \model\ model abandons GNN architectures in the inference model, which fundamentally minimizes the possibility of over-smoothed node embeddings. Furthermore, our KD paradigm avoids distilling over-smoothed embeddings via the adaptive contrastive regularization. \\\vspace{-0.12in}
    
    \item We observe that non-GNN CF models (\ie, NCF and AutoR) present very bad performance, event though they have similar MLP-based network architectures as the inference model in \model. This sheds light on the deficiency of MLPs in modeling high-order graph connectivity into user/item embeddings. While sharing similar MLP structures, our \model\ is additionally supervised by knowledge distilled from advanced GNN models. This not only improves the optimization for MLP networks, but also makes it possible to adaptively filter the over-smoothing signals in parameter learning. The huge performance gap between NCF/AutoR and our \model\ strongly shows the effectiveness of our contrastive knowledge distillation.
\end{itemize}


\begin{table}[t]
    %\vspace{-0.05in}
    \caption{Ablation study on key components of \model.}
    \vspace{-0.15in}
    \centering
    %\small
    %\scriptsize
    %\ssmall
    \footnotesize
    %\small
    % \setlength{\tabcolsep}{1.2mm}
    \begin{tabular}{c|c|cc|cc|cc}
        \hline
        % \multirow{2}{*}{Category} & \multirow{2}{*}{Variant} 
        \multicolumn{2}{c|}{Data}& \multicolumn{2}{c|}{Gowalla} & \multicolumn{2}{c|}{Yelp} & \multicolumn{2}{c}{Amazon}\\
        % \cline{3-8}
        \hline
        \multicolumn{2}{c|}{Variant} & Recall & NDCG & Recall & NDCG & Recall & NDCG\\
        \hline
        % \hline
        % \multicolumn{8}{c}{Top-20}\\
        \hline
        \multicolumn{2}{c|}{-$\mathcal{L}_1$} & 0.2180 & 0.1415 & 0.0756 & 0.0377 & 0.1012 & 0.0692\\
        \hline
        \multirow{3}{*}{-$\mathcal{L}_2$} & User & 0.2292 & 0.1493 & 0.0806 & 0.0405 & 0.0998 & 0.0667 \\
        & Item & 0.2266 & 0.1477 & 0.0808 & 0.0406 & 0.0974 & 0.0649 \\
        & Both & 0.2222 & 0.1451 & 0.0787 & 0.0399 & 0.0938 & 0.0626 \\
        \hline
        \multirow{4}{*}{-$\mathcal{L}_3$} & U-I & 0.2330 & 0.1496 & 0.0814 & 0.0410 & 0.0939 & 0.0607 \\
        & U-U & 0.2349 & 0.1512 & 0.0811 & 0.0407 & 0.0965 & 0.0634 \\
        & I-I & 0.2331 & 0.1514 & 0.0813 & 0.0409 & 0.1009 & 0.0674\\
        & All & 0.2282 & 0.1480 & 0.0810 & 0.0407 & 0.0933 & 0.0605\\
        \hline
        \hline
        \multicolumn{2}{c|}{\emph{\model}} & \textbf{0.2434} & \textbf{0.1592} & \textbf{0.0823} & \textbf{0.0414} & \textbf{0.1067} & \textbf{0.0734}\\
        \hline
    \end{tabular}
    \vspace{-0.1in}
    \label{tab:module_ablation}
    \Description{A table presenting the results of module ablation study. The results are divided into three parts: loss $\mathcal{L}_1$ for the prediction-level distillation, loss $\mathcal{L}_2$ for the embedding level distillation, and loss $\mathcal{L}_3$ for the contrastive regularization. All ablated variants performs worse than the proposed \model.}
\end{table}

\vspace{-0.1in}
\subsection{Model Ablation Study (RQ2)}
We validate the effectiveness of the applied sub-modules in \model\ by ablating each module separately. The evaluated performance is shown in Table~\ref{tab:module_ablation}. We also show the performance change \wrt, training epochs in Figure~\ref{fig:ablation_lines}. We have the following observations:
\begin{itemize}[leftmargin=*]
    \item \textbf{Effect of Prediction-Level Distillation}: Our prediction-level distillation (\ie, $\mathcal{L}_1$) excavates deep dark knowledge in the teacher using the pair-wise ranking task with enriched KD samples. The variant -$\mathcal{L}_1$ removes this module, which leads to performance degradation on Gowalla and Yelp data. The results validate the effectiveness of learning from the predictive outputs of teacher model using our distillation loss $\mathcal{L}_1$.\\\vspace{-0.12in}
    % The prediction-level KD is removed to produce variant \textbf{-$\mathcal{L}_1$}. From the results we can observe that, removing $\mathcal{L}_1$ causes the most significant performance degradation compared to other variants on Gowalla data and Yelp data. This evidently reflects the importance of learning from the predictive outputs of teacher model. And it validates the effectiveness of excavating deep dark knowledge in the teacher using the pair-wise ranking task with enriched KD samples.
    \item \textbf{Effect of Embedding-Level Distillation}: We then test the effect of embedding-level KD with the variant -$\mathcal{L}_2$ by removing $\mathcal{L}_2$ on user/item embeddings. In some cases the alignment between users and the alignment between items have different effect on the performance. What's more, the results reveal not only the contribution of $\mathcal{L}_2$ to the final performance, but also its prominent accelerating effect in model training shown in Fig~\ref{fig:ablation_lines}. \\\vspace{-0.12in}
    \item \textbf{Effect of Contrastive Regularization}: We ablate \model\ without the contrastive regularization in variant -$\mathcal{L}_3$. The regularization for user-item, user-user, and item-item relatedness are individually ablated. We observe the importance of $\mathcal{L}_3$ for the superior performance, especially on Amazon data. We ascribe this to the larger scale of Amazon data which makes it more likely to over-smooth with irrelevant high-order neighbors. The incorporation of $\mathcal{L}_3$ can cancel out over-smoothing signals.\\\vspace{-0.12in}
    \item \textbf{Comparison to Student and Teacher Models}: From the learning curves in Fig~\ref{fig:ablation_lines}, we can observe the great performance gap between simple MLP student and advanced GNN teacher. The three augmented tasks greatly minimizes this gap by effectively distilling useful knowledge. Additionally, the distillation tasks accelerate the training to surpass the original teacher model.
\end{itemize}

\begin{figure}[t]
    \centering
    \includegraphics[width=0.43\columnwidth]{figs/ablation_converge_Gowalla_Recall.pdf}\quad
    \includegraphics[width=0.43\columnwidth]{figs/ablation_converge_Amazon_Recall.pdf}
    \vspace{-0.12in}
    \caption{Test performance in each epoch for ablated models.}
    \vspace{-0.1in}
    \label{fig:ablation_lines}
    \Description{A line figure showing the performance with respect to epochs for \model\ and some representative baselines. The figure shows that \model\ converges faster while training.}
\end{figure}

\begin{table}[t]
    \centering
    %\small
    \footnotesize
    \setlength{\tabcolsep}{1.4mm}
    % \caption{Model efficiency study on per-epoch training time and inference time on Gowalla, Yelp, and Amazon data.}
    \caption{Model performance and per-epoch model inference time of representative methods on large-scale Tmall dataset.}
    \label{tab:scalability}
    \vspace{-0.1in}
    \begin{tabular}{ccccccc}
        \hline
        Metric & \# Edges & DGCF & SGL & HCCF & NCL & \emph{\model}\\
        \hline
        \hline
        \multirow{2}{*}{R@20} & 1.6M & 0.0221 & 0.0258 & 0.0272 & 0.0286 & \multirow{2}{*}{\textbf{0.0308}}\\
        & 2.9M & 0.0253 & 0.0278 & 0.0283 & 0.0294 & \\
        \hline
        \multirow{2}{*}{N@20} & 1.6M & 0.0258 & 0.0296 & 0.0309 & 0.0337 & \multirow{2}{*}{\textbf{0.0366}}\\
        & 2.9M & 0.0279 & 0.0311 & 0.0319 & 0.0334 & \\
        \hline
        \multirow{2}{*}{Time} & 1.6M & 7190.2s & 1331.8s & 1342.5s & 1392.2s & \multirow{2}{*}{\textbf{785.1s}}\\
        & 2.9M & 11431.8s & 1456.3s & 1530.8s & 1693.8s & \\
        \hline
    \end{tabular}
    \vspace{-0.12in}
    \Description{A table showing the performance and the inference time of \model\ and baselines on the large-scale Tmall dataset. \model\ outperforms the baselines and consumes the least time for inference.}
\end{table}

\vspace{-0.1in}
\subsection{Model Scalability Study (RQ3)}
To validate the efficiency of our \model\ in handling large-scale real-world data, we compare \model\ with the best performed baselines on a e-commerce data collected from Tmall platform. The dataset contains around 40 million records of user clicks. To successfully run on this dataset, GNN-based methods have to sample subgraphs for information propagation. In contrast, graph sampling is not required by the MLP-based inference model of our \model. The performance and the inference time are shown in Table~\ref{tab:scalability}, where we run the baselines using graph sampling strategy~\cite{hu2020heterogeneous} with two scales (\ie, subgraphs contain 1.6M edges and 2.9M edges, respectively). We have mainly two key observations shown as follows:
\begin{itemize}[leftmargin=*]
    \item \textbf{More Accurate Recommendations}: \model\ achieves better recommendation performance in terms of Recall and NDCG. This reflects the higher probability of over-smoothing on the large but sparse interaction graph. Our \model\ avoids this problem without explicit graph message passing. Instead, informative knowledge is distilled from GNNs for model compression.
    \item \textbf{Much Higher Efficiency}: \model\ greatly reduces the inference time on the large Tmall data. \textit{Firstly}, the embedding process of our MLP predictor is agnostic to the holistic interaction graph, thus the large-scale graph does not increase much overhead for embedding processing. No graph sampling is required in comparison to GNNs. \textit{Secondly}, \model\ infers user-item relations based on simple MLPs. The computational costs of fully-connected layers in MLPs are much lower than the cost of GNNs.
\end{itemize}

\begin{figure}[t]
    \centering
    \includegraphics[width=0.3\columnwidth]{figs/hyper_gowalla_soft_Recall_20.pdf}\quad
    \includegraphics[width=0.3\columnwidth]{figs/hyper_gowalla_cd_Recall_20.pdf}\quad
    \includegraphics[width=0.3\columnwidth]{figs/hyper_gowalla_sc_Recall_20.pdf}\\
    \includegraphics[width=0.3\columnwidth]{figs/hyper_gowalla_soft_NDCG_20.pdf}\quad
    \includegraphics[width=0.3\columnwidth]{figs/hyper_gowalla_cd_NDCG_20.pdf}\quad
    \includegraphics[width=0.3\columnwidth]{figs/hyper_gowalla_sc_NDCG_20.pdf}\\
    \vspace{-0.12in}
    \caption{Hyperparameter study for our \model\ model on Gowalla dataset, in terms of \emph{Recall@20} and \emph{NDCG@20}.}
    \vspace{-0.1in}
    \label{fig:hyper2d}
    \Description{A line figure showing the performance change with respect to the weight of the prediction-level distillation, the embedding-level distillation, and the contrastive regularization.}
\end{figure}

\begin{figure}[t]
    \centering
    \subfigure[Pred. Distillation]{
        \includegraphics[width=0.3\columnwidth]{figs/hyper_soft_Recall.pdf}
        \label{fig:hyper3d_pred}
    }
    \subfigure[Embed. Distillation]{
        \includegraphics[width=0.3\columnwidth]{figs/hyper_cd_Recall.pdf}
        \label{fig:hyper3d_embed}
    }
    \subfigure[Contrastive Reg.]{
        \includegraphics[width=0.3\columnwidth]{figs/hyper_sc_Recall.pdf}
    }
    \vspace{-0.17in}
    \caption{Impact of weights and temperature in different learning objectives on Yelp, in terms of \emph{Recall@20}.}
    \vspace{-0.2in}
    \label{fig:hyper3d}
    \Description{A three-D figure showing the composite effect of the weight and the temperature coefficient on the performance, for the prediction-level distillation, the embedding-level distillation, and the contrastive regularization.}
\end{figure}
\subsection{Hyperparameter Study (RQ4)}
In this section, we examine the influence of different hyperparameters on the performance of \model. The effect of loss weights $\lambda_1, \lambda_2, \lambda_3$ are shown in Figure~\ref{fig:hyper2d}. The composite effect of loss weights and corresponding temperatures $\tau_1, \tau_2, \tau_3$ are shown in Figure~\ref{fig:hyper3d}. The effect of the size $|\mathcal{T}_1|$ for the prediction-level distillation is shown in Table~\ref{tab:batch_hyper}. Our observations are as follows:
\begin{itemize}[leftmargin=*]
    \item \textbf{Strength of Prediction-Level Distillation}. $\lambda_1, \tau_1$: This weight $\lambda_1$ and temperature $\tau_1$ jointly control the strength of the prediction-level KD $\lambda_1$. We first study the influence of $\lambda_1$ in Figure~\ref{fig:hyper2d} with $\tau_1$ fixed. When $\lambda_1$ is small, not enough knowledge is distilled to the student model which results in deficient performance. When $\lambda_1$ is too large, $\mathcal{L}_1$ cover up the optimization of main loss and yield degraded performance. Additionally, Figure~\ref{fig:hyper3d_pred} shows the positive effect of applying smaller $\tau_1$ to produce larger gradients.
    
    \item \textbf{Strength of Embedding-Level Distillation}. $\lambda_2, \tau_2$: The parameters control the strength of \model\ in restricting the embeddings in MLP to be close to embeddings in GNN. From Figure~\ref{fig:hyper3d_embed} it can be observed that $\lambda_2$ and $\tau_2$ jointly adjust the strength of embedding KD to have modest influence on optimization, to prevent from insufficient knowledge distillation and too-strict embedding regularization. Either large weight with low temperature or small weight with high temperature causes performance decay.
    
    \item \textbf{Strength of Contrastive Regularization} $\lambda_3, \tau_3$: These parameters determine the strength of push-away regularization for preventing over-smoothing. The results show that either too small weight $\lambda_3$ or too high temperature $\tau_3$ causes insufficient regularization and produces over-smoothed embeddings. Meanwhile, strong regularization may damage the modeling of node-wise affinity, and also yields worse performance.
    
    \item \textbf{Per-Batch Number of Samples to Distill} $|\mathcal{T}_1|$: This hyperparameter determines how many instances are sampled to conduct the prediction-level distillation in each training step. According to the results in Table~\ref{tab:batch_hyper}, increasing batch size brings better KD performance until the performance saturates. We ascribe this to the effect that larger batch size filters low-frequency noise in predictions made by the teacher model in \model.
\end{itemize}


\begin{table}[t]
    %\vspace{-0.05in}
    \caption{Investigation on the impact of batch size in the prediction-oriented distillation of the proposed \model.}
    \vspace{-0.15in}
    \centering
    %\small
    %\scriptsize
    %\ssmall
    \footnotesize
    %\small
    % \setlength{\tabcolsep}{1.2mm}
    \begin{tabular}{c|c|cccccc}
        \hline
        \multirow{2}{*}{Data} & \multirow{2}{*}{Metric} & \multicolumn{6}{c}{Batch Size $|\mathcal{T}_1|$ in Prediction-Level Distillation}\\
        \cline{3-8}
        & & $1e3$ & $5e3$ & $1e4$ & $5e4$ & $1e5$ & $5e5$\\
        \hline
        \hline
        \multirow{2}{*}{Gowalla} & Recall & 0.2208 & 0.2361 & 0.2399 & 0.2420 & 0.2434 & 0.2448\\
        & NDCG & 0.1441 & 0.1530 & 0.1554 & 0.1577 & 0.1592 & 0.1597\\
        \hline
        \multirow{2}{*}{Yelp} & Recall & 0.0443 & 0.0730 & 0.0773 & 0.0802 & 0.0823 & 0.0822\\
        & NDCG & 0.0210 & 0.0372 & 0.0392 & 0.0407 & 0.0414 & 0.0414\\
        \hline
    \end{tabular}
    \vspace{-0.2in}
    \label{tab:batch_hyper}
    \Description{A table recording the performance change of \model\ with respect to the }
\end{table}

\vspace{-0.1in}
\subsection{Over-Smoothing Investigation (RQ5)}
To investigate whether our graph-less \model\ framework is able to mitigate the over-smoothing effect in graph-structured relation learning for CF, we compare representative baselines and our \model\ model on the Mean Average Distance (MAD) values~\cite{chen2020measuring} over embeddings for the most popular users and items. The evaluation results are shown in Table~\ref{tab:mad}. Our \model\ has higher MAD values on both user and item embeddings for Gowalla and Yelp data, in comparison to not only GCN model GCCF, but also state-of-the-art SSL frameworks. It can be concluded that our \model\ framework better addresses the over-smoothing issue, by learning more uniform-distributed embeddings for users and items, to better characterize their unique interaction patterns. This should be attributed to the MLP-based inference framework, and the contrastive regularization that adaptively alleviates over-smoothing signals.


\begin{table}[t]
    %\vspace{-0.05in}
    \caption{Investigation on the ability to address the over-smoothing effect on Gowalla and Yelp data in terms of MAD.}
    \vspace{-0.15in}
    \centering
    % \small
    %\scriptsize
    %\ssmall
    \footnotesize
    %\small
    % \setlength{\tabcolsep}{1.2mm}
    \begin{tabular}{c|c|cccccc}
        \hline
        \multicolumn{2}{c|}{Data} & GCCF & LightGCN & SGL & NCL & HCCF & \emph{\model}\\
        \hline
        \hline
        \multirow{2}{*}{Gowalla} & User & 0.8276 & 0.8203 & 0.8412 & 0.8088 & 0.8394 & \textbf{0.8576}\\
        & Item & 0.7579 & 0.7614 & 0.7702 & 0.8169 & 0.7905 & \textbf{0.8335}\\
        \hline
        \multirow{2}{*}{Yelp} & User & 0.9226 & 0.9610 & 0.9755 & 0.9640 & 0.9749 & \textbf{0.9819}\\
        & Item & 0.6288 & 0.7095 & 0.7191 & 0.6953 & 0.6246 & \textbf{0.7662}\\
        \hline
    \end{tabular}
    \vspace{-0.1in}
    \label{tab:mad}
    \Description{A table presenting the evaluated MAD value of \model\ and baselines. The MAD value of \model\ is higher.}
\end{table}
\section{Related Work}
The key to multi-exposure HDR imaging is to exploit cross-exposure correlation. Early works assume that the scene is static, and synthesize the HDR image by inferring the light energy map and camera response function from exposure bracketing \cite{SIGGRAPH97,mitsunaga1999radiometric,pal2004probability} or by a weighted fusion of different exposures \cite{ward2003fast,gelfand2010multi}. The fusion approach is now more widely investigated. The advent of CNN recasts the fusion in image domain to fusion in feature space. Recently, research effort on multi-exposure fusion has been mainly devoted to addressing cross-exposure registration and misalignment, to overcome the difficulty caused by scene motion \cite{ma2017robust}. 

\textbf{CNN-Based Multi-Exposure Fusion for HDR.} Applying deep convolution to multi-exposure fusion allows establishing cross-exposure correlation by extracting high-level contextual features, yielding robust matching. The first CNN-based model estimates image motion by optical flow to align the image sequence before convolutional feature extraction \cite{Kalantari2017DeepHD}. This external registration step prevents the model from being end-to-end trainable. Moreover, accurate optical flow algorithms are generally too slow for real-time HDR imaging. \cite{wu2018end} propose an end-to-end image translation network to formulate multi-image HDR fusion. Generative Adversarial Network is investigated, implicitly aligns the images in the generator \cite{niu2021hdrgan}. Several works perform soft feature selection. For example, \cite{yan2019attention} designs a spatial attention module by filtering the concatenation of low-level features extracted from the reference and side images, whereas \cite{xiong2021hierarchical} trains a masking tensor to blend the exposures. These models show that feature selection is crucial for HDR imaging. Recently, \cite{liu2021adnet} and \cite{xiong2021hierarchical} demonstrate the benefits of pyramidal alignment and fusion.

Different from the CNN-based models, our network explores the potential of Swin Transformer for HDR imaging. The self-attention mechanism in Swin Transformer implicitly searches for global cross-exposure alignment. We modify Swin Transformer with the gating mechanism similar to \cite{yu2019free} to mask out outliers.

\textbf{Transformer Network for Single Image Restoration}
A related line of work is single image restoration, e.g., denoising, deblurring, inpainting, super-resolution, based on modified Transformer (e.g., \cite{zamir2021restormer}) or Swin Transformer (e.g., \cite{liang2021swinir}).  

Our work is closely related to Restormer. Briefly, Restormer attaches a gating unit to the feed-forward network in Transformer. To lower the computational cost inherent to Transformer, Restormer replaces the pairwise affinity computation in Transformer by cross-channel covariance. That is, it uses channel attention instead of spatial attention, therefore makes it unsuitable for inferring cross-exposure spatial correlation. Instead, we employ Swin Transformer, which replaces global self-attention by shifted window self-attention. Swin Transformer trades memory storage for computational efficiency, maintaining the capability of connecting spatially distant yet contextually close features. Inspired by the hierarchical architectures in \cite{zamir2021restormer,xiong2021hierarchical,liu2021adnet}, our Gated Swin Transformer units serve in a U-shape pyramidal encoder-decoder, which significantly saves the memory storage. 
\section{Conclusion}\label{sec:conclusion}
In this work, we focus on addressing the fundamental challenge of OOD detection tasks, which is how to fully understand the semantic discrepancy between the ID/OOD samples. We reveal that the key to success in the realistic SCOOD task is to allocate as many ID samples in the unlabeled set correctly as possible. To this end, we propose a novel uncertainty-aware optimal transport scheme that introduces class-specific energy scores as guidance for effective label assignment. Experimental results show that our method achieves better performance than previous state-of-the-art methods on SCOOD benchmarks.

\textbf{Limitations.} In addition to temperature scaling, other techniques such as feature clipping applied in ReAct~\cite{sun2021react} also enhance the performance of energy score, so how to obtain an OOD score that best fits the SCOOD task can be further explored. Moreover, a setting highly related to SCOOD has been proposed in \cite{katz2022training} and formulated as a constrained optimization problem. We will also theoretically analyze these practical OOD settings in our feature work.

% \section*{Acknowledgments}
\textbf{Acknowledgments.} 
This work is supported by National Key R\&D Program of China under Grant 2020AAA0105701, National Natural Science Foundation of China (NSFC) under Grants 61872327, Major Special Science and Technology Project of Anhui, National Natural Science Foundation of China (62033012) and Ant Group through Ant Research Intern Program.


\section*{Acknowledgments}
This project is partially supported by 2022 Tencent Wechat Rhino-Bird Focused Research Program Research and Weixin Open Platform. This research work is also supported by Department of Computer Science \& Musketeers Foundation Institute of Data Science at the University of Hong Kong.

\clearpage
\bibliographystyle{ACM-Reference-Format}
\balance
\bibliography{sample-base}

\clearpage
\section{Appendix for Proofs}

\paragraph{Proof of Theorem \ref{thm:main}.}

\begin{proof}
\label{proof:main}
Our proof has two steps. In Step 1, we will show that SimCLR is equivalent to minimizing the cross entropy loss defined in Eqn.~(\ref{eqn:cross-entropy}). 
In Step 2, we will show  that minimizing the cross-entropy loss 
is equivalent to spectral clustering on $\bfpi$. 
Combining the two steps together, we have proved our theorem. 

\textbf{Step 1: } SimCLR is equivalent to minimizing the cross entropy loss.

The cross-entropy loss takes expectation over 
$\bfW_\bfX\sim \mathbb{P}(\cdot ; \bfpi)$, 
which means $\bfW_\bfX$ has exactly one non-zero entry in each row $i$. By Lemma~\ref{lem:multinomial}, we know every row $i$ of $\bfW_\bfX$ is independent of other rows. Moreover, 
$\bfW_{\bfX,i}\sim \mathcal{M}(1, \bfpi_i/\sum_j \bfpi_{i,j})=\mathcal{M}(1, \bfpi_i)$, because $\bfpi_i$ itself is a probability distribution.
Similarly, we know $\bfW_\bfZ$ also has the row-independent property by sampling over $\mathbb{P}(\cdot;\bfK_\bfZ)$.
Therefore, by Lemma~\ref{lem:cross_split}, we know Eqn.~(\ref{eqn:cross-entropy}) is equivalent to:
\[
 -\sum_{i=1}^n \mathbb{E}_{\bfW_{\bfX,i}}[\log \mathbb{P}(\bfW_{\bfZ,i}=\bfW_{\bfX,i};\bfK_\bfZ)],
\]

This expression takes expectation over $\bfW_{\bfX,i}$ for the given row $i$. Notice that 
$\bfW_{\bfX,i}$ has exactly one non-zero entry, which equals $1$ (same for $\bfW_{\bfZ,i}$). 
As a result
we expand the above expression to be:
\begin{equation}
 -\sum_{i=1}^n \sum_{j\neq i} \Pr(\bfW_{\bfX,i,j}=1)\log \Pr(\bfW_{\bfZ,i,j}=1).
\label{eqn:detailed-expansion}    
\end{equation}


By Lemma~\ref{lem:multinomial}, $\Pr(\bfW_{\bfZ,i,j}=1)=\bfK_{\bfZ,i,j}/\|\bfK_{\bfZ,i}\|_1$ for $j\neq i$. Recall that $\bfK_\bfZ=(k(\bfZ_i-\bfZ_j))_{(i,j)\in[n]^2}$, which means 
$\bfK_{\bfZ,i,j}/\|\bfK_{\bfZ,i}\|_1=\frac{\exp(-\|\bfZ_i-\bfZ_j\|^2/{2\tau})}{\sum_{k\neq i}
\exp(-\|\bfZ_i-\bfZ_k\|^2/{2\tau})
}$ for $j\neq i$, when $k$ is the Gaussian kernel with variance $\tau$. 

Notice that $\bfZ_i=f(\bfX_i)$, so we know
\begin{equation}
-\log \Pr(\bfW_{\bfZ,i,j}=1)=
-\log \frac{\exp(-\|f(\bfX_i)-f(\bfX_j)\|^2/{2\tau})}{\sum_{k\neq i}
\exp(-\|f(\bfX_i)-f(\bfX_k)\|^2/{2\tau}),
}
\label{eqn:infonce-equivalence}    
\end{equation}


The right hand side is exactly the InfoNCE loss defined in Eqn.~(\ref{eqn:infonce}).
Inserting Eqn.~(\ref{eqn:infonce-equivalence}) into Eqn.~(\ref{eqn:detailed-expansion}), we get the SimCLR algorithm, which first samples augmentation pairs $(i,j)$ with $\Pr(\bfW_{\bfX,i,j}=1)$ for each row $i$, and then optimize the InfoNCE loss. 

\textbf{Step 2: } minimizing the cross entropy loss 
is equivalent to spectral clustering on $\bfpi$.


By Lemma~\ref{lem:convert_to_spectral}, we may further convert the loss to 
\begin{equation}
\label{eqn:main-theorem-repul-attr}
\min_{\bfZ}
-\sum_{(i,j)\in [n]^2} \mathbf{P}_{i,j}
\log k (\bfZ_i-\bfZ_j)+\log \mathbf{R}(\bfZ).
\end{equation}
Since $k$ is the Gaussian kernel, this reduces to \[
\min_\bfZ \mathrm{tr}(\bfZ^\top \mathbf{L}(\bfpi) \bfZ)
+\log \mathbf{R}(\bfZ),
\]

where we use the fact that $\mathbb{E}_{\bfW_\bfX\sim \mathbb{P}(\cdot; \bfpi)}[\mathbf{L}(\bfW_\bfX)]
=\mathbf{L}(\bfpi)
$, because the Laplacian operator is linear and $
\mathbb{E}_{\bfW_\bfX\sim \mathbb{P}(\cdot; \bfpi)}(\bfW_\bfX)=\bfpi
$.
\end{proof}

\paragraph{Proof of Theorem \ref{thm:clip}.}
\begin{proof}
Since $\bfW_\bfX\sim \mathbb{P}(\cdot;\bfpi_{\mathbf{A}, \mathbf{B}})$, we know 
$\bfW_\bfX$ has exactly one non-zero entry in each row, denoting the pair that got sampled. 
A notable difference compared to the previous proof is we now have $n_\mathcal{A}+n_\mathcal{B}$ objects in our graph. CLIP deals with this by taking a mini-batch of size $2N$, 
such that $n_\mathcal{A}=n_\mathcal{B}=N$, and adding the $2N$ InfoNCE losses together. We label the objects in $\mathcal{A}$ as $[n_\mathcal{A}]$, and the objects in $\mathcal{B}$ as $\{n_\mathcal{A}+1, \cdots, n_\mathcal{A}+n_\mathcal{B}\}$. 

Notice that $\bfpi_{\mathbf{A}, \mathbf{B}}$ is a bipartite graph, so the edges of objects in $\mathcal{A}$ will only connect to object in $\mathcal{B}$ and vice versa. We can define the similarity matrix in $\cZ$ as $\bfK_\bfZ$, 
where $\bfK_\bfZ(i, j+n_\mathcal{A})=\bfK_\bfZ(j+n_\mathcal{A},i)= k(\bfZ_i-\bfZ_j)$ for $i\in [n_\mathcal{A}], j\in [n_\mathcal{B}]$, and otherwise we set $\bfK_\bfZ(i,j)=0$. 
The rest is same as the previous proof. 
\end{proof}

\paragraph{Proof of Theorem \ref{thm:exponential}.}

\begin{proof}
\label{proof:exponential}
Since the objective function consists of a linear term combined with an entropy regularization, which is a strongly concave function, the maximization problem is a convex optimization problem. Owing to the implicit constraints provided by the entropy function, the problem is equivalent to having only the equality constraint. We then introduce the Lagrangian multiplier $\lambda$ and obtain the following relaxed problem:

$$
\widetilde{E}(\boldsymbol{\alpha})=\psi_{1}-\sum_{i=1}^n \alpha_{i} \psi_{i}+\tau \sum_{i=1}^n \alpha_{i}\log \alpha_{i}+\lambda\left(\boldsymbol{\alpha}^{\top} \mathbf{1}_n-1\right).
$$

As the relaxed problem is unconstrained, taking the derivative with respect to $\alpha_{i}$ yields

$$
\frac{\partial \widetilde{E}(\boldsymbol{\alpha})}{\partial \alpha_{i}}=-\psi_{i}+\tau\left(\log \alpha_{i}+\alpha_{i} \frac{1}{\alpha_{i}}\right)+\lambda=0.
$$

Solving the above equation implies that $\alpha_{i}$ takes the form
$
\alpha_{i}=\exp \left(\frac{1}{\tau} \psi_{i}\right) \exp \left(\frac{-\lambda}{\tau}-1\right).
$ Since $\alpha_{i}$ lies on the probability simplex, the optimal $\alpha_{i}$ is explicitly given by
$
\alpha^{*}_{i}=\frac{\exp \left(\frac{1}{\tau} \psi_{i}\right)}{\sum_{i^{\prime}=1}^n \exp \left(\frac{1}{\tau} \psi_{i^{\prime}}\right)} .
$ Substituting the optimal point into the objective function, we obtain
$$
\begin{aligned}
E\left(\boldsymbol{\alpha}^*\right)  &=\psi_1-\sum_{i=1}^n \frac{\exp \left(\frac{1}{\tau} \psi_{i}\right)}{\sum_{i^{\prime}=1}^n \exp \left(\frac{1}{\tau} \psi_{i^{\prime}}\right)} \psi_{i}+\tau \sum_{i=1}^n \frac{\exp \left(\frac{1}{\tau} \psi_{i}\right)}{\sum_{i^{\prime}=1}^n \exp \left(\frac{1}{\tau} \psi_{i^{\prime}}\right)}\log \frac{\exp \left(\frac{1}{\tau} \psi_{i}\right)}{\sum_{i^{\prime}=1}^n \exp \left(\frac{1}{\tau} \psi_{i^{\prime}}\right)} \\
& =\psi_1 - \tau \log \left(\sum_{i=1}^n \exp \left(\frac{1}{\tau} \psi_{i}\right)\right).
\end{aligned}
$$
Thus, the Lagrangian dual function is given by
\begin{equation*}
-E\left(\boldsymbol{\alpha}^*\right)= -\tau \log \frac{\exp \left(\frac{1}{\tau} \psi_{1}\right)}{\sum_{i=1}^n \exp \left(\frac{1}{\tau} \psi_{i}\right)}.\qedhere
\end{equation*}
\end{proof}



\section{More on Experiments} \label{section: experiment_details}

\paragraph{CIFAR-10 and CIFAR-100} CIFAR-10 ~\citep{krizhevsky2009learning} and CIFAR-100 ~\citep{krizhevsky2009learning} are well-known classic image classification datasets. Both CIFAR-10 and CIFAR-100 contain a total of 60k $32 \times 32$ labeled images of different classes, with 50k for training and 10k for testing. CIFAR-10 is similar to CIFAR-100, except there are 10 different classes in CIFAR-10 and 100 classes in CIFAR-100.

\paragraph{TinyImageNet} TinyImageNet ~\citep{le2015tiny} is a subset of ImageNet ~\citep{deng2009imagenet}. There are 200 different object classes in TinyImageNet, with 500 training images, 50 validation images, and 50 test images for each class. All the images in TinyImageNet are colored and labeled with a size of $64 \times 64$.

\textbf{Pseudo-code.} Algorithm \ref{alg:Training Procedure} presents the pseudo-code for our empirical training procedure.

\begin{algorithm}[!htbp]
\caption{Training Procedure}
\label{alg:Training Procedure}
\begin{algorithmic}[1]
\REQUIRE trainable encoder network $f$, batch size $N$, augmentation strategy \textit{aug}, loss function $L$ with hyperparameters \textit{args}
\FOR {sampled minibatch ${x_i}_{i=1}^N$}
\FORALL{$i \in { 1, ..., N }$}
\STATE draw two augmentations $t_i = \textit{aug}\left(x_i\right) $, $t_i' = \textit{aug}\left(x_i\right) $
\STATE $z_i = f\left(t_i\right)$, $z_i' = f\left(t_i'\right)$
\ENDFOR
\STATE compute loss $\mathcal{L} = L(N, z, z', \textit{args})$
\STATE update encoder network $f$ to minimize $\mathcal{L}$
\ENDFOR
\STATE \textbf{Return} encoder network $f$
\end{algorithmic}
\end{algorithm}

We also provide the pseudo-code for our core loss function used in the training procedure in Algorithm \ref{alg:Core loss}. The pseudo-code is almost identical to SimCLR's loss function, with the exception of an extra parameter $\gamma$.

\begin{algorithm}[!htbp]
\caption{Core loss function $\mathcal{C}$}
\label{alg:Core loss}
\begin{algorithmic}[1]
\REQUIRE batch size $N$, two encoded minibatches $z_1, z_2$, $\gamma$, temperature $\tau$
\STATE $z = \textit{concat}\left(z_1, z_2\right)$
\FOR {$i \in {1, ..., 2N }, j \in {1, ..., 2N}$ }
\STATE $s_{i,j} = \Vert z_i - z_j \Vert_2^{\gamma}$
\ENDFOR
\STATE \textbf{define} $l(i, j)$ \textbf{as} $l(i, j) = - \log \frac{exp\left(s_{i,j}/\tau \right)}{\sum_{k=1}^{2N} \mathbf{1}{[k \ne i]} exp\left(s{i, j} / \tau \right)} $
\STATE \textbf{Return} $\frac{1}{2N} \sum_{k=1}^N\left[l(i, i+N) + l(i+N, i)\right]$
\end{algorithmic}
\end{algorithm}

Utilizing the core loss function $\mathcal{C}$, we can define all kernel loss functions used in our experiments in Table \ref{table: loss definition}. For all $z_i \in z$ with even dimensions $n$, we define $z_{L_i} = z_i\left[0:n/2\right]$ and $z_{R_i} = z_i\left[n/2:n\right]$.

\begin{table}[ht]
\centering
\begin{tabular}{{@{}l|l@{}}}
Kernel  &  Loss function \\ \midrule
Laplacian & $\mathcal{C}\left(N, z, z', \gamma=1, \tau\right)$\\ \midrule
Sum       & $\lambda * \mathcal{C}\left(N, z, z', \gamma=1, \tau_1\right) + (1-\lambda) * \mathcal{C}\left(N, z, z', \gamma=2, \tau_2\right)$  \\ \midrule
Concatenation Sum&$\lambda * \mathcal{C}\left(N, z_L, z'_L, \gamma=1, \tau_1\right) + (1-\lambda) * \mathcal{C}\left(N, z_R, z'_R, \gamma=2, \tau_2\right)$\\ \midrule
$\gamma = 0.5$ & $\mathcal{C}\left(N, z, z', \gamma=0.5, \tau\right)$          \\ 

\end{tabular}

\caption{Definition of kernel loss functions in our experiments}
\label {table: loss definition}
\end{table}

\textbf{Baselines.} We reproduce the SimCLR algorithm using PyTorch Lightning~\citep{PytorchLightning}.

\textbf{Encoder details.}
The encoder $f$ consists of a backbone network and a projection network. We employ ResNet50~\citep{ResNet} as the backbone and a 2-layer MLP (connected by a batch normalization~\citep{ioffe2015batch} layer and a ReLU \cite{nair2010rectified} layer) with hidden dimensions 2048 and output dimensions 128 (or 256 in the concatenation kernel case).

\textbf{Encoder hyperparameter tuning.}
For each encoder training case, we randomly sample 500 hyperparameter groups (sample details are shown in Table \ref{table: Hyperparameter sample}) and train these samples simultaneously using Ray Tune ~\citep{RayTune}, with the ASHA scheduler~\citep{li2018massively}. Ultimately, the hyperparameter group that maximizes the online validation accuracy (integrated in PyTorch Lightning) within 5000 validation steps is chosen for the given encoder training case.

\begin{table}[ht]
\centering

\begin{tabular}{@{}l|l|l@{}}
\midrule
Hyperparameter  & Sample Range & Sample Strategy \\ \midrule
start learning rate & $\left[10^{-2}, 10\right]$ & log uniform \\ \midrule
$\lambda$       & $\left[0, 1\right]$ & uniform \\ \midrule
$\tau$, $\tau_1$, $\tau_2$ & $\left[0, 1\right]$ & log uniform \\ \midrule
\end{tabular}

\caption{Hyperparameters sample strategy}
\label {table: Hyperparameter sample}
\end{table}

\textbf{Encoder training.} 
We train each encoder using the LARS optimizer~\citep{LARSOptimizer}, LambdaLR Scheduler in PyTorch, momentum 0.9, weight decay $10^{-6}$, batch size 256, and the aforementioned hyperparameters for 400 epochs on a single A-100 GPU.

\textbf{Image transformation.} The image transformation strategy, including augmentation, is identical to the default transformation strategy provided by PyTorch Lightning.

\textbf{Linear evaluation.}
The linear head is trained using the SGD optimizer with a cosine learning rate scheduler, batch size 64, and weight decay $10^{-6}$ for 100 epochs. The learning rate starts at $0.3$ and ends at $0$.

\textbf{Moco Experiments.} We also tested our method based on MoCo~\citep{he2019moco}. The results are summarized in Table \ref{tab:results-moco}. Here we choose ResNet18~\citep{ResNet} as the backbone and set a temperature of $0.1$ as default. For our simple sum kernel, we set $\lambda=0.8$. The results show that our method outperforms the original MoCo method.

\begin{table}[thb]
\centering
\caption{MoCo Experiment Results on CIFAR-10 and CIFAR-100.}
\label{tab:results-moco}
\resizebox{\textwidth}{!}{%
\begin{tabular}{@{}c|ccc|ccc@{}}
\toprule
\multirow{3}{*}{Method} & \multicolumn{3}{c|}{CIFAR-10} & \multicolumn{3}{c}{CIFAR-100} \\ \cmidrule(lr){2-4} \cmidrule(lr){5-7} 
                        & 200 epochs & 400 epochs    & 1000 epochs   & 200 epochs & 400 epochs & 1000 epochs         \\ \midrule
MoCo (repro.)         & $76.41 \pm 0.12$    & $80.01 \pm 0.15$          & $84.45 \pm 0.08$    & $\mathbf{47.02 \pm 0.11}$ & $52.50 \pm 0.07$ & $57.62 \pm 0.15$            \\
\midrule
Laplacian Kernel        & ${78.09 \pm 0.10}$    & $\mathbf{83.85 \pm 0.09}$          & $\mathbf{88.34 \pm 0.16}$    & $46.12 \pm 0.22$   & $53.44 \pm 0.17$ & $59.10 \pm 0.14$        \\
Simple Sum Kernel & $\mathbf{78.12 \pm 0.15}$   & $83.23 \pm 0.18$ & $87.50 \pm 0.20$ & $46.65 \pm 0.06$ & $\mathbf{53.62 \pm 0.19}$ & $\mathbf{59.83 \pm 0.12}$\\
\bottomrule
\end{tabular}
}
\end{table}



\section{More Experiments on Synthetic Data}


Consider a scenario with $n$ clusters, each containing $k$ vertices. Let the probability of vertices $u$ and $v$ from the same cluster belonging to $\bfpi$ be $p$. Conversely, for vertices $u$ and $v$ from different clusters, let the probability of belonging to $\pi$ be $q$. We generate the graph $\bfpi$ randomly, based on $p$ and $q$. We experiment with values of $k=100$ and $n=6$ for ease of visualization, embedding all points in a two-dimensional space. Each vertex's initial position originates from a normal distribution. In each iteration, we sample a subgraph of $\bfpi$ uniformly, ensuring each vertex has an out-degree of $1$. We then optimize the corresponding vectors using InfoNCE loss with an SGD optimizer and iterate until convergence. Our experimental setup consists of an SGD learning rate of $1$, an InfoNCE loss temperature of $0.5$, and a batch size of $50$. We evaluate two scenarios with different $p$ and $q$ values: $p=1$, $q=0$, and $p=0.75$, $q=0.2$. The results of these experiments are visualized in Figure \ref{fig:vis-spectral-cluster}. The obtained embeddings exhibit the hallmark pattern of spectral clustering of graph $\bfpi$.

\begin{figure}[!tb]
\centering
\subfigure{
\includegraphics[width=1\textwidth]{Figures/cluster_pi.png}
\label{fig:vis-cluster}
}
\subfigure{
\includegraphics[width=1\textwidth]{Figures/noised_cluster_pi.png}
\label{fig:vis-noised-cluster}
}
\caption{Visualizations of the optimization process using InfoNCE Loss on the vectors corresponding to $\bfpi$. Points of identical color belong to the same cluster within $\bfpi$. To showcase the internal structure of $\bfpi$, we randomly select 10 vertices from each cluster to display the edge distribution of $\bfpi$.}
\label{fig:vis-spectral-cluster}
\end{figure}



\end{document}
\endinput
%%
%% End of file `sample-sigconf.tex'.
