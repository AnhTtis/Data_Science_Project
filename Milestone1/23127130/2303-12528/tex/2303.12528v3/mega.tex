% This must be in the first 5 lines to tell arXiv to use pdfLaTeX, which is strongly recommended.
\pdfoutput=1
% In particular, the hyperref package requires pdfLaTeX in order to break URLs across lines.

\documentclass[11pt]{article}

% Remove the "review" option to generate the final version.
\usepackage[]{EMNLP2022}
% Standard package includes
\usepackage{times}
\usepackage{latexsym}
\usepackage{caption}
\usepackage{subcaption}
\usepackage{graphicx}
\usepackage{subfloat}
\usepackage{booktabs} % To thicken table lines
\usepackage{multirow}
\usepackage{amsmath,amssymb}
\usepackage{cuted}
\usepackage{flushend}
\usepackage{tablefootnote}

\usepackage{caption}
\usepackage{subcaption}
\usepackage{graphicx}
\usepackage{subfloat}
\usepackage{wrapfig}

% For proper rendering and hyphenation of words containing Latin characters (including in bib files)
\usepackage[T1]{fontenc}
% For Vietnamese characters
% \usepackage[T5]{fontenc}
% See https://www.latex-project.org/help/documentation/encguide.pdf for other character sets

% This assumes your files are encoded as UTF8
\usepackage[utf8]{inputenc}
\usepackage{pifont}
% This is not strictly necessary, and may be commented out.
% However, it will improve the layout of the manuscript,
% and will typically save some space.
\usepackage{microtype}

% This is also not strictly necessary, and may be commented out.
% However, it will improve the aesthetics of text in
% the typewriter font.
\usepackage{inconsolata}
\usepackage{xcolor}
\newcommand\myworries[1]{\textcolor{red}{#1}}
\DeclareMathOperator*{\argmax}{arg\,max}
\DeclareMathOperator*{\argmin}{arg\,min}
\DeclareMathOperator*{\E}{\mathbb{E}}
\newcommand{\cmark}{\ding{51}}%
\newcommand{\xmark}{\ding{55}}%

% If the title and author information does not fit in the area allocated, uncomment the following
%
\setlength\titlebox{6cm}
%
% and set <dim> to something 5cm or larger.

\title{MEGA: Multilingual Evaluation of Generative AI}

% Author information can be set in various styles:
% For several authors from the same institution:
% \author{Author 1 \and ... \and Author n \\
%         Address line \\ ... \\ Address line}
% if the names do not fit well on one line use
%         Author 1 \\ {\bf Author 2} \\ ... \\ {\bf Author n} \\
% For authors from different institutions:
% \author{Author 1 \\ Address line \\  ... \\ Address line
%         \And  ... \And
%         Author n \\ Address line \\ ... \\ Address line}
% To start a seperate ``row'' of authors use \AND, as in
% \author{Author 1 \\ Address line \\  ... \\ Address line
%         \AND
%         Author 2 \\ Address line \\ ... \\ Address line \And
%         Author 3 \\ Address line \\ ... \\ Address line}

%will change author order later
\author{Kabir Ahuja \quad Harshita Diddee \quad Rishav Hada \quad Millicent Ochieng  
\AND  Krithika Ramesh \quad Prachi Jain \quad Akshay Nambi  \quad Tanuja Ganu \\
 \AND Sameer Segal \quad Maxamed Axmed \quad Kalika Bali \quad Sunayana Sitaram \\ 
 \\
 Microsoft Corporation \\ \\
 {\tt Contact: sunayana.sitaram@microsoft.com}}

%Outline
%1. Intro:
    % - Motivate the problem, how cross lingual transfer has been studied w.r.t to accuracy but not calibration
    % - why calibration is important
    % - some related work on calibration
    % - contributions
    %     - 1. We show that the models are badly calibrated for langauges other than english in zero-shot setting specially for low resource languages
    %     - 2. Investigate factors influencing the calibration, rersourcefulness, typological diversity, choice of model, choice of task, number of epochs, choice of pivot language.
    %     - 3. Standard Calibration techniques leads to substantial improvement in calibration errors, and collecting a few examples in the target can not only improve the performance but substantially improve calibration as well.
% 2 Calibration of Neural Network based Classifiers:
%     - Define what calibration is,
%     - Formulas for ECE and CACE
%     2.1 Experimental Setup
%         - Tasks and Datasets
%         - Models
%         - Calibration Methods
%     2.2 
        

\begin{document}
\maketitle
\begin{abstract}
Generative AI models have shown impressive performance on many Natural Language Processing tasks such as language understanding, reasoning and language generation. An important question being asked by the AI community today is about the capabilities and limits of these models, and it is clear that evaluating generative AI is very challenging. Most studies on generative LLMs have been restricted to English and it is unclear how capable these models are at understanding and generating text in other languages. We present the first comprehensive benchmarking of generative LLMs - MEGA, which evaluates models on standard NLP benchmarks, covering 16 NLP datasets across 70 typologically diverse languages. We compare the performance of generative LLMs including Chat-GPT and GPT4 to State of the Art (SOTA) non-autoregressive models on these tasks to determine how well generative models perform compared to the previous generation of LLMs. We present a thorough analysis of the performance of models across languages and tasks and discuss challenges in improving the performance of generative LLMs on low-resource languages. We create a framework for evaluating generative LLMs in the multilingual setting and provide directions for future progress in the field. 

% Next, we show that standard calibration methods like Temperature Scaling and Label Smoothing can be used to substantially improve calibration in the zero-shot scenario which can be even further improved by collecting few-shot examples in those languages. Overall, our work provides a step towards building more reliable multilingual models by taking into account their calibration in addition to their performance across languages. %Overall, our work contributes towards understanding and building more reliable multilingual models by highlighting calibration as an issue in these models, which should be considered along with performance while deploying them into production.


% Overall, our work provides a step towards building more reliable multilingual models by taking into account their calibration in addition to their performance across languages.

%Overall, our work provides a step towards building more reliable multilingual models and draw attention of the community to consider calibration of these models along with their performance while deploying them into production

\end{abstract}

\section{Introduction}

The increasing complexity of source code poses a key challenge to the reliability of large-scale software systems. Software bugs in these systems can lead to safety issues~\cite{bug_safety} for users around the world as well as cause non-negligible financial losses~\cite{bug_loss}. As such, developers have to spend a large amount of time and effort on bug fixing. Consequently, \aprfull (\apr), designed to automatically generate patches to fix software bugs, has attracted wide attention from both academia and industry~\cite{long2016prophet, legoues2012genprog, long2015spr, lou2020can, tufano2018empstudy}. 


To achieve \apr, one popular approach is known as Generate-and-Validate (G\&V)~\cite{qi2015gv, ghanbari2019prapr, lou2020can, le2016hdrepair, legoues2012genprog, wen2018capgen, hua2018sketchfix, martinez2016astor, koyuncu2020fixminder, liu2019tbar, liu2019avatar}, which is typically based on the following pipeline: First, fault localization techniques~\cite{wong2016fl, abreu2007ochiai, zhang2013injecting, papadakis2015metallaxis, li2019deepfl, li2017transforming} are applied to determine the suspicious locations in programs where bugs are likely to exist. Then, the buggy locations are used by the \apr tools to generate a list of patches that replace buggy lines with correct lines. Afterward, each patch is validated against the original test suite to identify any \emph{plausible patches} (i.e., passing all tests in the test suite). Finally, to determine the \emph{correct patches}, developers examine the list of plausible patches to see if any of them can correctly fix the bug. 

Traditional \apr tools can mainly be categorized into heuristic-based~\cite{legoues2012genprog, le2016hdrepair, wen2018capgen}, constraint-based~\cite{mechtaev2016angelix, le2017s3, demacro2014nopol, long2015spr} and \template~\cite{ghanbari2019prapr, hua2018sketchfix, martinez2016astor, liu2019tbar, liu2019avatar}. Among these traditional tools, \template \apr tools~\cite{ghanbari2019prapr, liu2019tbar, benton2020effectiveness} have been able to achieve state-of-the-art results. \Template \apr tools typically leverage pre-defined templates (e.g., adding a nullness check) for bug fixing. However, since these fix templates are typically handcrafted, the number and types of bugs they are able to fix can be limited. 



To address the limitations of traditional \apr, researchers have proposed various \learning \apr tools~\cite{li2020dlfix, chen2018sequencer, jiang2021cure, lutellier2020coconut, zhu2021recoder, ye2022rewardrepair} based on the \nmtfull (\nmt) architecture~\cite{sutskever2014mt} where the input is the buggy code snippets and the goal is to translate the buggy code snippets into a fixed version. To accomplish this, \learning \apr tools require supervised training datasets with pairs of both buggy and fixed code snippets in order to learn how to perform this translation step. These training data are usually obtained by mining historical bug fixes using heuristics/keywords~\cite{dallmeier2007benchmark}, which can be imprecise for identifying bug-fixing commits; even the actual bug-fixing commits can include irrelevant code changes, leading to further pollution in the dataset~\cite{xia2022alpharepair}.
% 
Moreover, it can be hard for such \apr tools to generalize and fix bug types unseen during training. 



To better leverage recent advances in \plmfull{s} (\plm{s}), researchers~\cite{xia2022alpharepair, xia2023repairstudy, kolak2022patch, prenner2021codexws} have directly applied \plm{s} to generate patches without bug-fixing datasets. These \llm-based \apr tools work by either directly generating a complete code function~\cite{prenner2021codexws, xia2023repairstudy} or predict/infill the correct code snippet given its surrounding context~\cite{xia2022alpharepair, xia2023repairstudy}. By directly using \llm{s} that are pre-trained on billions of open-source code snippets, \llm-based \apr tools can achieve state-of-the-art performance on many repair datasets~\cite{xia2022alpharepair}. 


% 
%
%

Traditional \apr tools have long used the insight of the \emph{plastic surgery hypothesis}~\cite{barr2014plastic} where it states that the code ingredients to fix a bug already exist within the same project. Traditional \apr tools have manually designed pattern-~\cite{ghanbari2019prapr, saha2017elixir} or heuristic-based~\cite{jiang2018simfix, legoues2012genprog} approaches to finding and using such relevant code ingredients to generate fixes for bugs. However, the plastic surgery hypothesis has been largely ignored in \llm-based \apr. In fact, \llm provides a unique opportunity to fully automate the plastic surgery hypothesis idea via fine-tuning (learning project-specific information via model updates from the buggy project) and prompting (directly providing relevant code ingredients to the model), and make it directly applicable to different languages (since the \llm{s} are typically multi-lingual).%
Moreover, despite the intensive manual efforts involved, traditional \apr tools still cannot fully leverage project-specific information due to large search space for leveraging/composing existing code ingredients. In contrast, the project-specific information can effectively leveraged by \llm{s} due to their power in code understanding/vectorization, e.g., even partial/imprecise information may still guide \llm{s} in correct patch generation!
 To this end, we ask the question: \emph{How useful is the plastic surgery hypothesis in the era of \plm{s}}?








\mypara{Our Work.} To answer the question, we present \ourtech{\xspace} -- a \llm-based approach that automatically utilizes the plastic surgery hypothesis by systematically combining multiple fine-tuning and prompting strategies for \apr. \ourtech fine-tunes \plm{s} using two novel domain-specific training strategies: \textbf{\epfinetune} -- we fine-tune using the original buggy project by aggressively masking out a high percentage of tokens, which allows \plm to learn project-specific code tokens and programming styles; and \textbf{\rofinetune} -- which only masks out a single continuous code sequence per training sample, allowing the model to get used to the final \csapr task of predicting a single continuous code sequence. Furthermore, we directly leverage the ability for \plm{s} to understand natural language instructions and introduce a novel prompting strategy, \textbf{\idprompting}, which uses information retrieval and static analysis to obtain a list of relevant identifiers for the buggy lines. While such relevant identifiers are critical for fixing some difficult bugs, they may not be seen by the \llm during inference due to limited context window size. Through the use of prompting, we directly tell the model to use these extracted identifiers (relevant code ingredients) to generate the correct code. Finally, to perform repair, we combine all four model variants (including the base model, both fine-tuned models and the base model with prompting) for the final repair.





While our insight of leveraging the plastic surgery hypothesis for \llm-based \apr is generalizable across different types of \plm{s}, to implement \ourtech, we choose a recent \plm{\xspace}, \ctfive~\cite{wang2021codet5}, which is pre-trained on millions of open-source code snippets. \ctfive is an encoder-decoder model trained using \mspfull (\msp) objective where a percentage of tokens are masked out and each continuous masked token sequence is referred to as a masked span. Also, although we only extract relevant identifiers from the current buggy project (since this paper focuses on the plastic surgery hypothesis), our work can be easily extended to obtain other code information (such as relevant statements or functions) from other sources, such as  the massive pre-training corpora~\cite{husain2020codesearchnet} or historical bug-fixing datasets~\cite{jiang2019infer}, which can provide more coding knowledge for \llm{s}. Besides, although we mainly focus on using traditional string comparison algorithms for information retrieval in this paper, these techniques can be easily replaced by other frequency-based retrieval~\cite{robertson2009probabilistic} and neural search (or embedding-based search)~\cite{reimers2019sentence}.
  In summary, this paper makes the following contributions:


%


\begin{itemize}[noitemsep, leftmargin=*, topsep=0pt]
    \item \textbf{Dimension.} This paper is the first to revisit the important plastic surgery hypothesis in the era of \llm{s}. It opens up a new dimension for \llm-based \apr to incorporate previously neglected information from the buggy project itself to boost \apr performance. Furthermore, it demonstrates the promising future of retrieval-based prompting for modern \llm-based \apr.
    \item \textbf{Implementation.} We implement \ourtech based on the recent \ctfive model. We augment the model using two novel fine-tuning strategies: \epfinetune and \rofinetune, along with a novel prompting strategy based on information retrieval and static analysis: \idprompting. We combine the patches generated by all four models together and perform patch ranking to speed up \apr.% 
    \item \textbf{Evaluation Study.} We conduct an extensive evaluation against state-of-the-art \apr tools. On the widely studied \dfj 1.2 and 2.0 datasets~\cite{just2014dfj}, \ourtech is able to achieve the new state-of-the-art results of 89 and 44 correct bug fixes (15 and 8 more than best baseline) respectively.  Furthermore, we perform a broad ablation study to justify our design. \ourtech demonstrates for the first time that the plastic surgery hypothesis can substantially boost \llm-based \apr and advance state-of-the-art \apr, while being fully automated and general. Moreover, even partial/imprecise code ingredients may still effectively guide \llm{s} for \apr!
\end{itemize}


\section{Experiments}

In this section, we describe how we adapt various NLP tasks to the in-context learning setting. We describe the prompting strategies we use for the benchmark and the models, tasks and datasets included in our initial study.

\subsection{Problem Formulation}

In order to solve different tasks via in-context learning we adopt the prompt-based few-shot learning strategy as defined in \cite{brown-etal-2020-language}. We define four main components of the prompts that we use in our experiments as follows: i) a \textbf{test example} $x_{test}$ for which the predictions are to be made; ii) $K$ \textbf{few-shot exemplars} $\{(x_i, y_i)\}_{i = 1}^{K}$, that are used to provide in-context supervision to the model; iii) a \textbf{prompt template} $f_{temp}(x)$ which turns a dataset input example into a text format that can 
 be used for prompting, containing the task description; and iv) an \textbf{answer verbalizer} $f_{verb}(y)$ that maps the label $y$ to a textual representation. In our evaluation framework we often consider the template and verbalizer as a single entity, and from now on will denote the template to encapsulate both the template and verbalizer unless specified separately. Some examples of $f_{temp}$ and $f_{verb}$ are given in table \ref{tab:promptsource}.



Given these components, the final prompt $f_{prompt}(x_{test}; \{(x_i, y_i)\}_{i = 1}^{K}, f_{temp}, f_{verb})$ or $f_{prompt}(x_{test})$ for short for a test input $x_{test}$ can be defined as:
\begin{align*}
f_{prompt}(x_{test}) =\mathbin\Vert_{i = 1}^{K} \big\{f_{temp}(x_i)&\mathbin\Vert f_{verb}(y_i)\big\}\\
&\mathbin\Vert f_{temp}(x_{test})
\end{align*}

where $\mathbin\Vert$ denotes the string concatenation operator.

The prompt can then be provided as input to the LLM $P(.;\theta)$ to obtain the prediction $z_{test}$

\begin{align*}
    z_{test} = \argmax_{z \in \mathcal{Z}} P(z | f_{prompt}(x_{test}); \theta)
\end{align*}

where $\mathcal{Z}$ is the space of possible answers, which in all of our experiments is taken to be the entirety of the language as modeled by the LLM. We approximate the $\argmax$ by sampling from the probability distribution predicted by the LLM.

The predicted answer $z_{test}$ is compared with the verbalized label using $f_{metric}(z_{test}, f_{verb}(y_{test})) \in [0, 1]$ that measures the extent of similarity between the ground truth and predicted answer. For our experiments, we use the exact-match score to determine accuracy for classification tasks and use the exact-match and F1-score for QA tasks. Formally, the evaluation score $s$ for an LLM $P(.;\theta)$ on a task $\mathcal{T}$ can be defined as:

\begin{equation*}
    s = {\E_{(x_{test}, y_{test}) \in \mathcal{T}}}[f_{metric}(z_{test}, f_{verb}(y_{test}))]
\end{equation*}


\subsection{Prompting Strategies}
\label{sec:prompt_strategies}
The choice of prompt significantly influences the performance of Large Language Models. Generative models have been shown to be brittle to simple prompting variations, such as ordering of examples, number of few-shot examples and the choice of words in the prompt. There are many variations to consider for our setup: the choice of the language of the prompt examples, the language of the prompt template, and the language of the test examples. In this work we evaluate all the models using three types of prompts:

\iffalse
The choice of prompt can greatly influence the performance of generative models, and models have been shown in the past to be brittle to prompting variations such as the words used in the prompt, number of few-shot examples, ordering of examples etc CITE. Our setup in particular involves the choice of the few-shot examples $\{(x_i, y_i)\}_{i = 1}^{K}$ as well as  choice of different template $f_{temp}$ and verbalizer $f_{verb}$ functions, for defining the prompt. For our evaluation framework we consider two higher level decisions which stem from the choice of language in which the few-shot examples and the text examples are represented and the language in which the templates are written. For the former in particular we consider three setups:

\fi
\noindent
\begin{itemize}
    \item \textbf{Monolingual Prompting}: In this setup, the k\footnote{k=8, unless specified} randomly selected examples are of the same language as the test examples. Figure \ref{fig:monoprompting} illustrates an example of monolingual prompting in hindi.
    \item \textbf{Zero-Shot Cross-Lingual Prompting}: Pre-trained multilingual models are effective at zero-shot cross-lingual transfer \cite{pires-etal-2019-multilingual, wu-dredze-2019-beto}, that is on fine-tuning them for a task in one language leads to reasonable performance on unseen languages. In this section, we evaluate the model's zero-shot cross-lingual transfer ability after in-context learning. In this experiment, we use k-shot examples from a pivot language\footnote{We use english as the pivot language in this paper} which is different from the language of the test example. Figure \ref{fig:zsprompting} illustrates the setup for a hindi test query.
    \item \textbf{Translate-Test Prompting}: This setup is similar to the Zero-Shot Cross-Lingual setup in the fact that the few-shot examples are sampled from English data. However, here we modify the test example itself by translating it to English.  Translate-test has been shown to be often better than cross-lingual transfer for both fine-tuning \cite{ponti-etal-2021-modelling} and in-context learning \cite{lin-etal-2022-shot, shi-etal-2022-language} and hence we explore its effectiveness for our benchmarking exercise as well. An example for this setup for Hindi is given in Figure \ref{fig:translate_test}, we use Bing Translator to translate the test examples to English.
\end{itemize}

\iffalse
\noindent
\textbf{2. Zero-Shot Cross-Lingual Prompting}: Pre-trained multilingual models have been shown to be surprisingly effective at zero-shot cross lingual transfer \cite{pires-etal-2019-multilingual, wu-dredze-2019-beto}, where fine-tuning them for a task in one language leads to reasonable performance on unseen languages. In this setup, we try to probe to what extent LLMs can exhibit this behavior via in-context learning. Hence, the few-shot examples here are selected from a language (we call it pivot language) different from the language of the test example and for the purposes of our experiments we use English as the pivot language. Refer to Figure \ref{fig:zsprompting} for an example of this setup for Hindi.

\noindent
\textbf{3. Translate-Test Prompting}: This setup is similar to the Zero-Shot Cross-Lingual setup in the fact that the few-shot examples are sampled from English data. However, here we modify the test example itself by translating it to English.  Translate-test has been shown to be often better than cross-lingual transfer for both fine-tuning \cite{ponti-etal-2021-modelling} and in-context learning \cite{lin-etal-2022-shot, shi-etal-2022-language} and hence we explore its effectiveness for our benchmarking exercise as well. An example for this setup for Hindi is given in Figure \ref{fig:translate_test}. We use Bing Translator to translate the test examples to English in our experiments.
\fi

For the choice of language for the prompt template, we consider the following two setups:
\noindent
\begin{itemize}
    \item \textbf{English-Template}: Here the prompt templates are written in English, irrespective of the language of the few-shot and test examples. As has been shown in \citet{shi-etal-2022-language}, English instructions can often perform on par or even better than providing them in the native language. 
    \item \textbf{Native-Language-Template}: Here the prompt templates are written in the language of the test example $(x_{test}, y_{test})$. For our experiments, we use Bing Translator to translate the prompts from English to the native language.
\end{itemize}

In our initial experiments, we used different languages for task templates in few-shot examples and the test example but it performs poorly. We speculate such a poor performance of the model, as it may be getting confused about which language to generate the predictions in. We also found English templates to perform better than native language prompts. Hence we use English prompts for all our experiments.

\iffalse
\begin{table}[]
    \centering
    \begin{tabular}{p{2.5cm}p{2cm}p{2cm}}
        \toprule
         &  \textbf{English-Template} & \textbf{Native-Language-Template}\\
         \midrule
         Monolingual & \checkmark & \checkmark\\
         Zero-Shot Cross-Lingual & \checkmark & \xmark \\
         Translate-Test & \checkmark & \xmark\\
         \bottomrule
    \end{tabular}
    \caption{Combinations of the two higher level prompting setups.}
    \label{tab:prompting_setups}
\end{table}

Table \ref{tab:prompting_setups} provides the possible combinations of the two classes of prompting strategies. We only allow Native-Language templates for Monolingual setup, as in the other two setups the few-shot examples are in English. In our initial experiments, we tried using different language templates for few-shot examples and the test example but found that it performs poorly, as it would often lead to the model being confused about which language to generate the predictions in.
\fi
\subsection{Models}

We conduct all benchmarking experiments on OpenAI's GPT text-davinci-003 \cite{brown-etal-2020-language} model, which is available via API access. We do not conduct any fine-tuning of the model for our benchmarking experiments or carry out hyperparameter tuning for temperature or other settings.

We compare the performance of DV003 with the following models: BLOOMZ \cite{muennighoff2022crosslingual}, a fine-tuned version of the BLOOM \cite{scao2022bloom} model, which is a 176 parameter model created by the BigScience community trained on 46 natural languages and 13 programming languages. We also compare DV003's performance against SOTA non-autoregressive models such as TULRv6 \cite{patra2022beyond} and MuRIL \cite{khanuja2021muril} for the Indic benchmarks. The TULRv6 model is trained with a novel sampling strategy and bitexts in multiple languages and is currently at the top position on the XTREME \cite{hu2020xtreme} benchmark as of writing this paper. MuRIL is multilingual BERT model trained on 16 Indic languages and obtains SOTA performance on some Indic benchmarks. 

\subsection{Tasks and Datasets}

In our experiments, we consider two broad families of NLU tasks, i) Classification and ii) Question Answering. Below we review the experimental setups and datasets used for benchmarking for these two tasks. A list of all the datasets with the languages covered by them can be found in Table \ref{tab:datasets}.

\subsubsection{Classification}
These tasks involve classifying a single sentence or a group of sentences into a finite number of discrete labels. For each dataset, we measure the performance of different models in terms of classification accuracy. For prompt-based models in particular, since we add no constraint on the output space of the LLM we compute the exact match between the generated output and a verbalized label to determine if the example was classified correctly. We run experiments for all the prompting strategies that we discussed in the previous sections for each dataset. The details of each dataset that we use for benchmarking are given below:

 \begin{table*}[h]
 \small
     \centering
     \begin{tabular}{ccc}
     \toprule
    Dataset&Task&Languages\\
    \midrule
    XNLI&Natural Language Inference&en, fr, es, de, el, bg, ru, tr, ar, vi, th, zh, hi, sw, ur\\
    Indic-XNLI&Natural Language Inference&as, bn, gu, hi, kn, ml, mr, or, pa, ta, te\\
    %Indic-WNLI&Natural Language Inference&gu, hi, mr\\
    %GLUECoS-NLI&Natural Language Inference&hi-en\\
    %GLUECoS-En-Es-Sentiment&Sentiment Analysis&es-en\\
    PAWS-X&Paraphrase Identification&zh, fr, de, ko, ja, es\\
    XCOPA&Commonsense Reasoning&et, ht, id, it, qu, sw, ta, th, tr, vi, zh\\
    TyDiQA&Question Answering&en, ar, bn, fi, id, ja, sw, ko, ru, te, th\\
    MLQA&Question Answering&de, es, ar, zh, vi, hi\\
    XQUAD&Question Answering&en, es, de, el, ru, tr, ar, vi, th, zh, hi\\
    IndicQA&Question Answering&as, bn, gu, hi, kn, ml, mr, or, pa, ta, te\\
  \bottomrule
     \end{tabular}
     \caption{Datasets and languages}
     \label{tab:datasets}
     \vspace{-0.4cm}
 \end{table*}

\noindent{\textbf{1. Natural Language Inference}}: XNLI \cite{Conneau2018xnli} is a dataset for cross-lingual Natural Language Inference, which consists of professional translations of the MNLI \cite{wang2018glue} corpus into 14 languages. We also consider IndicXNLI \cite{aggarwal2022indicxnli} that translates the XNLI dataset into 11 Indic languages by using Machine Translation, followed by validation by native speakers.

% Indic-WNLI \cite{doddapaneni2022indicxtreme} is a translation of the Winograd NLI dataset \cite{wang2018glue}, an NLI version of the Winograd Schema Challenge into three Indic languages. 

%remove gluecos-nli if not done

% GLUECoS-NLI \cite{khanuja2020new} is a code-mixed NLI dataset in Hindi-English, consisting of Bollywood (Hindi) movie conversations as premises, with manually created hypotheses. 

% \subsubsection{Sentiment Analysis}

% %remove section if not done

% The EN-ES-CS Sentiment Analysis dataset \cite{vilares2016cs}, part of the GLUECoS benchmark \cite{khanuja2020gluecos} is a code-mixed dataset consisting of English-Spanish Tweets annotated with SentiStrength \cite{thelwall2017heart} scores. 

\noindent{\textbf{2. Paraphrase Identification}}: PAWS-X \cite{yang2019paws} is a paraphrase identification dataset professionally translated from the PAWS \cite{zhang2019paws} dataset into six typologically diverse languages. 

\noindent{\textbf{3. Commonsense Reasoning}}: XCOPA \cite{ponti2020xcopa} is a commonsense reasoning dataset, which is a translation of the COPA \cite{roemmele2011choice} dataset into 11 typologically diverse languages, including very low-resource languages such as Eastern Apurímac Quechua and Haitian Creole.

\subsubsection{Question Answering}
We focus on the Span Prediction type of Question Answering (QA) tasks in our experiments, where given a context and a question the task is to predict the answer within the context. One major challenge that we come across for multilingual evaluation of QA tasks is that for many languages we often cannot fit the context and question pairs for the few-shot and text examples in the maximum context size of 4096 for the DV003 model.

To overcome this issue we follow two steps. First, for the few-shot examples we only provide the line within the paragraph containing the answer as the context. Second, for the test example, we index the chunks of the context using the embeddings from the \texttt{text-embedding-ada-002} model. Given the question, the closest chunk in the full context is retrieved and used in the prompt for the test example. We use a maximum chunk size of 100 in our experiments and use the implementation for retrieval provided in the \textbf{LangChain}\footnote{\url{https://github.com/hwchase17/langchain}} library. By doing this,we minimize the space taken by the context tokens in our prompt.

For each task, we calculate the Exact Match and F1 score as defined in \citet{rajpurkar-etal-2016-squad}.  For our experiments we 
 consider the following three tasks:

\noindent \textbf{1. TyDiQA} \cite{clark2020tydi} is a QA dataset covering 11 typologically diverse languages. The task consists of two sub-tasks - passage selection and minimum answer span (Gold-P). For our experiments, we consider the Gold-P task and evaluate Monolingual and Zero-Shot Cross-Lingual prompting strategies. Since the labels do not directly transfer one-to-one across translation for QA tasks as they do for classification and require the use of alignment algorithms, we skip translate-test prompting for this task.

\noindent \textbf{2. MLQA} \cite{lewis2020mlqa} is an extractive QA dataset translated into 7 languages by professional translators. The task has two variants, the first where the question, context, and answer are all in the same language; and the second, where the question is in a different language than the context and answer. We consider the former variant of the task in our experiments. For MLQA, translate-test splits are also available, where each language's test data has been translated into English with answers aligned using the attention scores. There is no training data available for MLQA, and we use SQuAD's\citet{rajpurkar-etal-2016-squad} training data for selecting few-shot examples in English and validation data for MLQA in other languages to get their few-shot examples. This way, we are able to evaluate for all three prompting setups.

\noindent \textbf{3. XQuAD} \cite{artetxe2020cross} consists of professional translations of a subset of the SQuaD dataset \cite{rajpurkar2016squad} into 10 languages. XQuAD only has validation datasets available publicly, hence we evaluate the models on them. Like MLQA we use English SQuAD data for few-shot examples and since we cannot use validation data in other languages for few-shot, we only evaluate for zero-shot cross-lingual setup for this task.


\noindent \textbf{4. IndicQA} \cite{doddapaneni2022indicxtreme} is a manually curated cloze-style reading comprehension dataset that can be used for evaluating question-answering models in 11 Indic languages. The context paragraphs are chosen from Wikipedia articles whose topics are closely related to Indic culture, history,etc. The publicly available test set has about 2000 sentences that we carry out our evaluation on. 

% IndicQA \cite{doddapaneni2022indicxtreme} is a cloze-style reading comprehension dataset with context taken from Wikipedia articles on Indian culture and history, manually created in 11 Indic languages. 


% the few-shot examples are selected to belong to the same language as the test example to be evaluated. An example of the same for Hindi is shown in Figure \ref{fig:monoprompting}.

% Figure \ref{fig:zsprompting} illustrates the zero-shot prompting technique, in which the few-shot examples are in English, while the test example is in the target language. In this case, the model learns how to perform the task using few-shot examples in English and generates a response for the task in the target language.

% As shown in Figure \ref{fig:monoprompting}, in the monolingual prompting strategy, the entire prompt is in the target language, with the few-shot examples also coming from the target language. 

% The choice of prompt can greatly influence the performance of generative models, and models have been shown in the past to be brittle to prompting variations such as the words used in the prompt, number of few-shot examples, ordering of examples etc CITE. We used four prompting strategies for all our experiments - translate-test, zero-shot prompting, cross-lingual translated prompting and monolingual prompting. To illustrate the differences between the prompting methods, we use the following terminology: the \textit{instructions} part of the prompt contains the instructions on how to perform the task, along with \textit{few-shot examples}. The \textit{test example}
 % part of the prompt contains the data point for which we want the response from the model. We used the Bing Translator to perform the automatic translation in all our experiments. 

% \subsubsection{Translate-test}

% As shown in Figure \ref{fig:translate_test}, the Translate-test setting translates test example into English and uses the English instructions along with few-shot examples from English data.

\begin{figure*}[h!]
    \centering
    \includegraphics[width=18cm]{figures/translate_test.jpg}
    \caption{Translate-test}
    \label{fig:translate_test}
\end{figure*}

% \subsubsection{Zero-shot prompting}

% Figure \ref{fig:zsprompting} illustrates the zero-shot prompting technique, in which the few-shot examples are in English, while the test example is in the target language. In this case, the model learns how to perform the task using few-shot examples in English and generates a response for the task in the target language.

\begin{figure*}[h!]
    \centering
    \includegraphics[width=18cm]{figures/zero_shot_prompting.jpg}
    \caption{Zero-shot prompting}
    \label{fig:zsprompting}
\end{figure*}

% \subsubsection{Monolingual translated prompting}

 

% \subsubsection{Monolingual prompting}

% As shown in Figure \ref{fig:monoprompting}, in the monolingual prompting strategy, the entire prompt is in the target language, with the few-shot examples also coming from the target language. 

\begin{figure*}[h!]
    \centering
    \includegraphics[width=18cm]{figures/monolingual_prompting.jpg}
    \caption{Monolingual prompting}
    \label{fig:monoprompting}
\end{figure*}


\subsection{Few-shot examples}

In all our experiments, we choose few-shot examples randomly from the development set available in the dataset, unless specified. Better choices of few-shot examples for the tasks can lead to higher performance, which we leave for future work.

% how we chose the few shot examples and the different design choices that can be made here

% Issues with prompt length - cannot stuff more few shot examples in some languages. just mention here and can go into more detail in the tokenizer discussion

\subsection{Choice of prompts}

% Task-specific choice of prompts

% how we went about choosing the prompt for each task, using promptsource, optimizing for english, can add the potential drawbacks of doing so here or in discussion.

For each task that we consider in our benchmarking study, we need to come up with a prompt that specifies the instruction that the model should follow. We use PromptScource from the BigScience community to use the existing prompts or to create new prompts for the tasks \footnote{Hosted version: \url{https://huggingface.co/spaces/bigscience/promptsource}}. PromptSource is a toolkit for creating, sharing, and using natural language prompts. Prompts are saved in standalone structured files and written in a simple templating language called Jinja. 

For all datasets, we evaluate the performance of all English prompts on 10\% of the English test set. The 10\% of the test set is sampled randomly. We select the prompt that gives best performance on English. This prompt is then used for the entire test set for all prompt strategies as described in \ref{sec:prompt_strategies}. The selected English prompt is also translated to the corresponding target language using the Bing translator. Table \ref{tab:promptsource} shows the final English prompt for each dataset. 

There is a possibility that the best prompt for English is not necessarily the best prompt for other or all languages. Additionally, translation errors may propagate in the form of incorrect syntax and semantics in the prompts, which may influence task performance negatively. To avoid this, we manually inspect and edit prompts for languages that we know (mainly Indian languages and Swahili). We recommend that all translated prompt templates should be verified by a native speaker and plan to do so in future work.

% \myworries{ToDo: Discuss drawbacks of English finetuning here?}

\begin{table*}[h]
\begin{tabular}{p{3cm}p{2cm}p{5cm}p{3cm}}
\toprule
Dataset                 & Prompt Name & Template $f_{temp}$ & Verbalizer $f_{verb}$ \\ \midrule
XNLI, Indic-XNLI                    & Based on previous passage & \{premise\} Based on previous passage, is it true that \{hypothesis\} ? Yes, No, or Maybe? & Entailment -> Yes, Contradiction -> No, Neutral -> Maybe \\ \midrule
PAWS-X                  & Concatenation                                          & Sentence 1: \{sentence1\} Sentence 2: \{sentence2\} Question: Does Sentence 1 paraphrase Sentence 2? Yes or No? & Positive -> Yes, Negative -> No  \\ \midrule
XCOPA                   & Discrete version of plausible alternatives prompt &
\{ premise \} \{\% if question == "cause" \%\} This happened  because \{\% else \%\} As a consequence... \{\% endif \%\} Help  me pick the more plausible option:- choice1: \{choice1\}, choice2: \{choice2\} & \{choice1\} -> choice1 , \{choice2\} -> choice2       \\ \midrule
TyDiQA, MLQA, XQUAD, IndicQA &
  Answer given context and question &
  \{context\} Q: \{question\} Referring to the passage above, the correct answer to the given question is & Identity\\
  \bottomrule
\end{tabular}
\caption{Prompt type and prompt used for each dataset.}
\label{tab:promptsource}
\end{table*}

\section{Results}
\label{results}

\begin{figure*}[ht]
    \centering
    \includegraphics[scale=0.15,trim={0 2.5cm 0 5cm},clip]{images/aoi-single_burst}
    \caption{The time average peak Age of Information with burst and \gls{soa} loss values against the dynamic reliability logic for different network topologies.}
    \label{fig:aoi_burst}\vspace{-0.4cm}
\end{figure*}


This paper focuses on both transport layer and application layer metrics to determine the feasibility of dynamic reliability. For this, we have selected the session packet volume, as transmitted, retransmitted, lost and backlogged packets as \glspl{kpi} for the transport layer; while focusing on the \gls{aoi} for the application layer. The \gls{aoi} was chosen as a crucial indicator for the freshness of packets in real-time applications. More specifically, this work adopts the time average peak \gls{aoi} equation \cite{aoi_equation} depicted in Eq. \ref{aoi}, where $\Delta(r_{i+1})$ is the $i$th update at the time it was received at the server, for a session time period of $\tau$.

\begin{equation}
    \label{aoi}
    \gls{aoi}_\tau = \frac{1}{n-1}\sum_{i=1}^{n-1} \Delta(r_{i+1})
\end{equation}

We include a comparison between the vanilla QUIC implementation which does not enjoy the dynamic reliability extension, with a number of dynamic reliability policies. The tests were run a number of times for statistical significance, with the mean value of vanilla implementation used as a baseline for comparison. The topology utilised both random loss and bursty loss to explore the bounds of dynamic reliability. The \gls{soa} loss in the figures correspond to the loss values presented in Table. \ref{tab:path_char}, for ease of comparison between bursty and random loss scenarios.

\subsection{Transport-Layer KPIs}

To analyse the performance gain at the transport layer due to dynamic reliability, the volume of transmitted and backlogged packets is examined. The figures are in the form of boxplots, which take the vanilla implementation as a benchmark, depicted as the red dashed line.

As seen in Fig. \ref{fig:sent_burst}, the loss plays a crucial role in the performance of the reliability policies. The policies under random loss did incredibly well for the networks with a larger capacity, namely \gls{mmwave} and Sub-6~GHz, whereas for burst loss, the lower network capacities had a larger packet reduction. With the increase in burst loss, the behaviour of the set split reliable policies became unpredictable, if a reliable assignment happened to coincide with a burst loss, the number of transmitted packets increases, and vice versa. On the other hand, in smarter policies, such as Loss-Aware, the performance lightly matched the vanilla baseline, as the reliable assignment dominated the session to compensate for a higher burst loss. Not only that but, the burst loss also impacted the variance of the transmitted packets for the policies.

Unsurprisingly, the unreliable focused policy, 80-20 split, outperformed other policies for all topologies in random and bursty loss scenarios, with an approximate reduction of 80\%. That being said, the majority of the policies reduced the transmitted packets on the link by approximately 70\% for random loss, while the reduction started at $\approx 15\%$ and decreased as the loss increased for the burst loss scenario.

The retransmitted and lost packets, not shown due to space limitations, followed the same trend as the transmitted packets for the random loss scenarios. However, for the burst loss scenarios, the larger capacity networks had a lower reduction in the retransmitted and lost packets. This can be seen as a favorable outcome since the lower capacity networks are scarce on resources. It is important to note that the Loss-Aware policy mimicked the vanilla approach as the burst loss increased, signifying the overwhelming appointment of reliable packets in adapting to the harsh burst loss conditions.
 
Alternatively, Fig. \ref{fig:backlog_burst} clearly shows a stark comparison between the policies and loss scenario in the reduction of the backlogged packets. The Loss-Aware policy for random loss scenario reduced the backlogged packets by up to 50\%, beating all other policies by approximately 30\%. Furthermore, it is clear that the unreliability focused policies resulted in the lowest backlog for the session. In comparison, we notice that the burst loss and the backlogged frequency have a positive correlation, where the maximum reduction of the backlogged packets for the policies is at most 20\%. Much like the transmitted packets, the probability of a burst loss occurrence plays a vital role in the number of retransmissions sent and by extension the number of backlogged packets. Thus, we can conclude that the stress placed on the buffer is a result of the reliable packets which is tightly coupled with the congestion on the session. Whereas, unreliable focused policies did not encounter such a phenomenon regardless if it was experiencing a burst loss.


\subsection{Application-Layer KPIs}

The feasibility of dynamic reliability for real-time applications can be determined by the \gls{aoi}, with comparison across different topologies and policies. If we take a strict approach and consider anything below $10$~ms is real-time \cite{real-time}, then all the reliability policies passed that requirement, which is attractive for real-time applications, as shown in Fig. \ref{fig:aoi_burst}. Utilising the median as an estimate of the runs, the policies in the WLAN and Sub-6~GHz topology with random loss floated around $4-5$~ms with negligible difference, while the \gls{aoi} for \gls{mmwave} was $\approx 2-3$~ms. It is clear that the \gls{aoi} and the network capacity have a negative correlation, as the network capacity decreases, the \gls{aoi} increases. The same correlation is extended to the bursty loss scenarios, where \gls{mmwave} dominated the other topologies. That being said, it is crucial to note that the \gls{aoi} for the reliability policies is often slightly better than or equal to the \gls{aoi} of the vanilla implementation, proving that dynamic reliability reduces the congestion of the session at no cost to the \gls{aoi}.

We provide some comments on the growth conditions which constituted the majority of our analysis in sections \ref{sec:Hmixing} and \ref{sec:Hsigma}. In the simplest cases of Lemma \ref{lemma:unstableGrowth}, growth was established in an analogous fashion to the old one-step expansion condition (\ref{eq:oldOneStepExpansion}), finding the relevant Jacobians $M_j$ and checking that their expansion factors $K(M_j)$ satisfy
\begin{equation}
    \label{eq:discussionOneStep}
    \sum_j \frac{1}{K(M_j)} <1.
\end{equation}
For the more complicated cases, the inductive method used to establish growth near the accumulation points in Lemma \ref{lemma:unstableGrowth} and the weakened one-step expansion condition (\ref{eq:oneStep}) both address the same fundamental issue: the splitting of unstable curves by singularities into an unbounded number of small components. They circumvent this obstacle in rather different ways, however. While (\ref{eq:oneStep}) generalises (\ref{eq:discussionOneStep}) to ensure an growth of unstable curves `on average' (see \cite{chernov_statistical_2009} for a precise statement), our inductive method is a more direct adaptation of (\ref{eq:discussionOneStep}), using it to generate contradictory geometric conditions which a hypothetical non-growing unstable curve must satisfy. It may be possible to prove Theorem \ref{sec:Hmixing} using (\ref{eq:oneStep}) as the basis for growth. Since we required (\ref{eq:oneStep}) anyway for proving Theorem \ref{thm:HsigmaExp}, this could potentially condense our analysis, but only to a minor extent. A convenience of the method used in section \ref{sec:Hmixing} is that, by way of the `simple intersection' property, it naturally gives geometric information on the images of manifolds, useful for proving the property \textbf{(M)} of Theorem \ref{thm:katok-strelcyn}.

We expect that essentially analogous analysis can be applied to establish mixing properties in a wide class of piecewise linear non-uniformly hyperbolic maps, including those (like the OTM) which sit on the boundary of ergodicity and beyond. While we have relied on the precise partition structure of $H_\sigma$, its fundamental feature (self-similar sequences of elements $A^k$, sharing boundaries with its neighbours $A^{k-1},A^{k+1}$ and accumulating onto some point $p$) is quite typical to return map systems. See, for example, those of various stadium billiards \cite{chernov_chaotic_2006,chernov_improved_2008,chernov_statistical_2009} and LTMs \cite{springham_polynomial_2014}. Indeed, the same method can be used to prove the Bernoulli property for non-monotonic LTMs \cite{myers_hill_mixing_2022}, where monotonicity of the manifold images cannot be assumed and the classical argument \cite{sturman_mathematical_2006} fails. The OTM is the pointwise limit of these maps as the boundary shrinks to null measure. It further has utility in proving growth conditions for maps which are uniformly hyperbolic but possess regions $A_j$ where the hyperbolicity is very weak, signified by $K(M_j) \approx 1$, so that (\ref{eq:discussionOneStep}) fails. Typically this leads to suboptimal bounds on mixing windows, see e.g. \cite{wojtkowski_model_1981,przytycki_ergodicity_1983,myers_hill_family_2022}. The map $H_{(\eta,\eta)}$ for $\eta \approx 1/2$ is another example, possessing weak hyperbolicity over $A_2, A_3$. Letting $\varepsilon = |\eta-1/2|>0$, there is an upper bound $N = N(\varepsilon)$ on escape times from the intersections $A_2\cap \sigma, A_3 \cap \sigma$. The growth lemma then follows by applying the inductive step roughly $N$ times and can be established for arbitrarily small $\varepsilon$, opening the door to establishing optimal mixing windows.

The above gives two examples of piecewise linear perturbations to $H$ where mixing with respect to Lebesgue is preserved and our methods can be applied. Nonlinear perturbations to the shear profiles complicate the analysis in several ways. Firstly as the map's Jacobians takes on a broader range of values, cone invariance becomes an increasingly harder condition to establish. Cones must be widened, giving looser bounds on expansion factors, which may already be weak due to new regions of weaker stretching. This, together with the change from polygonal to curvilinear return time partition elements and nonlinear local manifolds, adds some complexity to showing growth conditions. This does not rule out certain (small) nonlinear perturbations however. There is some leeway in the inequalities which govern cone invariance and growth of local manifolds, the latter of which is not too dissimilar from the piecewise linear setting (see Lemmas \ref{lemma:piecewiseApprox}, \ref{lemma:componentLength}). Certain small perturbations would not alter the \emph{topological} structure of the return time partition, i.e. which elements share boundaries, the key information needed for setting up the induction. Finally while the partition elements would no longer be polygonal, only coarse geometric information is required for verifying each inductive step. Following the above, a potential perturbation could be to replace the linear portions of each shear by a cubic, perturbing the tent profile
\[  f(t) = \begin{cases} 2t & 0 \leq t \leq 1/2, \\ 2(1-t) & 1/2 \leq t \leq 1 ,\end{cases} \]
of the OTM shears to
\[  f_a(t) = \begin{cases} \frac{1}{8} t \left(16 - a + 6at - 8at^{2} \right) & 0 \leq t \leq 1/2, \\ \frac{1}{8}\left(1-t\right)\left( 16 - a + 6a\left(1-t\right) - 8a\left(1-t\right)^{2}\right)  & 1/2 \leq t \leq 1, \end{cases}   \]
for $a>0$. For small enough $a$ the gradient range $f'(t)$ is restricted to small neighbourhoods of $\{ 2, -2\}$ and the escape time partition retains a similar structure. We illustrate this in Figure \ref{fig:perturbations}, showing escapes from the square $S_3$ under the map $G \circ F$, equivalent to escapes from the perturbed $A_3$ under the $G \circ F$, but with a cleaner geometry for comparison. When $a$ is too large the analogy to the OTM breaks down. At $a=16$ the map is twice differentiable everywhere and features a new source of slowed mixing, the Jacobian is the identity at the corner points $x,y \in \{  0, 1/2 \}$ giving locally parabolic behaviour (visible in the escape time partition). 

\begin{figure}
    \centering
    \includegraphics[width=0.24 \linewidth]{0.png}
    \includegraphics[width=0.24 \linewidth]{4.png}
    \includegraphics[width=0.24 \linewidth]{8.png}
    \includegraphics[width=0.24 \linewidth]{16.png}
    \caption{Partition of escape times from $S_3$ under the mapping $F \circ G$ for $a= 0,4,8,16$. }
    \label{fig:perturbations}
\end{figure}
% % \section{V2 dump}

% Dump all results, information etc. that does no t fit very clearly into the paper here. We will merge later

% Please add the following required packages to your document preamble:
% \usepackage{booktabs}
% \usepackage{graphicx}
% \begin{table*}[]
% \resizebox{\textwidth}{!}{%
% \begin{tabular}{@{}ccccccccccccc@{}}
% \toprule
%  & \multicolumn{3}{c}{Google}      & \multicolumn{3}{c}{Microsoft}   & \multicolumn{3}{c}{Amazon}      & \multicolumn{3}{c}{GPT Turbo 3.5} \\
%  & Acc & $\Delta_G$\ & $\Delta_S$\ & Acc & $\Delta_G$\ & $\Delta_S$\ & Acc & $\Delta_G$\ & $\Delta_S$\ & Acc  & $\Delta_G$\  & $\Delta_S$\ \\ \midrule
% ES & 50.9          & 23.2 & 20.9 & 45            & 36.5 & 22.9 & 57.2 & 15.3 & 21.7 & \textbf{60.6} & 13.4 & 22.5 \\
% FR & \textbf{61.6} & 6.1  & 22.3 & 44.5          & 34.2 & 15.8 & 54.2 & 16.4 & 15   & 55.3          & 16.6 & 23.4 \\
% IT & 38.6          & 32.9 & 18.6 & 38.8          & 41.8 & 10.5 & 40.2 & 26.8 & 14.7 & \textbf{48.9} & 17.6 & 23.1 \\ \midrule
% RU & 37.8          & 36.7 & 11.4 & 36.9          & 42   & 8.4  & 39.8 & 34.8 & 9.4  & \textbf{40.9} & 32.2 & 12.2 \\
% UK & 38.4          & 43.5 & 10.7 & 41.3          & 46.8 & 11.9 & -    & -    & -    & \textbf{43.1} & 35.4 & 11.3 \\ \midrule
% HE & 50.8          & 11.7 & 35.5 & 44            & 22   & 29.8 & 48   & 13.6 & 45.9 & \textbf{58}   & 8.6  & 35.9 \\
% AR & 45.8          & 42.5 & 16.2 & 45            & 47.1 & 14.2 & 48.3 & 37.8 & 18.8 & \textbf{59.7} & 17.2 & 25.6 \\ \midrule
% DE & 59.4          & 12.5 & 12.6 & \textbf{74.1} & 0    & 8.8  & 62.4 & 12   & 16.7 & 61.5          & 15.4 & 20.8 \\ \bottomrule
% \end{tabular}%
% }
% \caption{Performance of commercial MT systems and LLMs on the WinoMT corpus on all tested languages, categorized by their family: Spanish, French, Italian, Russian, Ukrainian, Hebrew, Arabic, and German. Acc indicates overall gender accuracy (\% of instances the translation had the correct gender), $\Delta_G$ denotes the difference in performance (F1 score) between masculine and feminine scores, and $\Delta_S$ is the difference in performance (F1 score) between pro-stereotypical and anti-stereotypical gender role assignments (higher numbers in the two latter metrics indicate stronger biases). Numbers in bold indicate best accuracy for the language across MT systems (row), and underlined numbers indicate best accuracy for the MT system across languages (column). ∗Amazon Translate does not have a trained model for English to Ukrainian.}
% \label{tab:wino-mt}
% \end{table*}

% Please add the following required packages to your document preamble:
% \usepackage{graphicx}
\begin{table*}[]
\resizebox{\textwidth}{!}{%
\begin{tabular}{ccccccccccccclccccc}
\hline
 &
  \multicolumn{3}{c}{Google} &
  \multicolumn{3}{c}{Microsoft} &
  \multicolumn{3}{c}{Amazon} &
  \multicolumn{3}{c}{Systran} &
  \multicolumn{3}{c}{GPT Turbo 3.5} &
  \multicolumn{3}{c}{Bloomz} \\
 &
  Acc &
  $\Delta_G$\ &
  $\Delta_S$\ &
  Acc &
  $\Delta_G$\ &
  $\Delta_S$\ &
  Acc &
  $\Delta_G$\ &
  $\Delta_S$\ &
  Acc &
  $\Delta_G$\ &
  $\Delta_S$\ &
  \multicolumn{1}{c}{Acc} &
  $\Delta_G$\ &
  $\Delta_S$\ &
  Acc &
  $\Delta_G$\ &
  $\Delta_S$\ \\ \hline
ES &
  50.9 &
  23.2 &
  20.9 &
  45 &
  36.5 &
  22.9 &
  \textbf{57.2} &
  15.3 &
  21.7 &
  42.5 &
  46.2 &
  15.6 &
  54.9 &
  22.7 &
  26.2 &
  55.6 &
  17.2 &
  32.5 \\
FR &
  \textbf{61.6} &
  6.1 &
  22.3 &
  44.5 &
  34.2 &
  15.8 &
  54.2 &
  16.4 &
  15 &
  43.4 &
  41.8 &
  -0.1 &
  52.7 &
  21.4 &
  26.1 &
  52 &
  17.8 &
  24.6 \\
IT &
  38.6 &
  32.9 &
  18.6 &
  38.8 &
  41.8 &
  10.5 &
  40.2 &
  26.8 &
  14.7 &
  38.1 &
  47.3 &
  6.3 &
  45.1 &
  21.9 &
  26.7 &
  \textbf{45.7} &
  9 &
  18.5 \\ \hline
RU &
  37.8 &
  36.7 &
  11.4 &
  36.9 &
  42 &
  8.4 &
  39.8 &
  34.8 &
  9.4 &
  37.3 &
  44.1 &
  9.2 &
  \textbf{41} &
  31.6 &
  10.2 &
  5.9 &
  INV &
  0 \\
UK &
  38.4 &
  43.5 &
  10.7 &
  41.3 &
  46.8 &
  11.9 &
  - &
  - &
  - &
  28.9 &
  22.4 &
  12.9 &
  \textbf{42.9} &
  34.2 &
  12.1 &
  16.8 &
  22.7 &
  2.2 \\ \hline
HE &
  50.8 &
  11.7 &
  35.5 &
  44 &
  22 &
  29.8 &
  48 &
  13.6 &
  45.9 &
  43.1 &
  26.9 &
  23.1 &
  \textbf{57.5} &
  7.6 &
  40.8 &
  27.5 &
  31.4 &
  5 \\
AR &
  45.8 &
  42.5 &
  16.2 &
  45 &
  47.1 &
  14.2 &
  48.3 &
  37.8 &
  18.8 &
  45.6 &
  49.4 &
  -4.1 &
  \textbf{61.1} &
  13.9 &
  27.9 &
  48.1 &
  23 &
  25.6 \\ \hline
DE &
  59.4 &
  12.5 &
  12.6 &
  \textbf{74.1} &
  0 &
  8.8 &
  62.4 &
  12 &
  16.7 &
  48.5 &
  34.5 &
  10 &
  57.5 &
  19.5 &
  14.2 &
  47.6 &
  56.2 &
  6.6 \\ \hline
\end{tabular}%
}
\caption{Performance of commercial MT systems and LLMs on the WinoMT corpus on all tested languages, categorized by their family: Spanish, French, Italian, Russian, Ukrainian, Hebrew, Arabic, and German. Acc indicates overall gender accuracy (\% of instances the translation had the correct gender), $\Delta_G$ denotes the difference in performance (F1 score) between masculine and feminine scores, and $\Delta_S$ is the difference in performance (F1 score) between pro-stereotypical and anti-stereotypical gender role assignments (higher numbers in the two latter metrics indicate stronger biases). Numbers in bold indicate best accuracy for the language across MT systems (row), and underlined numbers indicate best accuracy for the MT system across languages (column). \footnote{For Google, Microsoft, Amazon, and Systran we use the translations provided by XX. Some values differ from the original paper due to updated Spcay modules.} \footnote{Amazon Translate does not have a trained model for English to Ukrainian.} \footnote{For Ru in Bloomz, Precision in male predictions is 0 leading to Invalid (INV) in $\Delta_G$}}
\label{tab:wino-mt}
\end{table*}


% Please add the following required packages to your document preamble:
% \usepackage{booktabs}
% \usepackage{graphicx}
\begin{table*}[]
\resizebox{\textwidth}{!}{%
\begin{tabular}{@{}llll@{}}
\toprule
Dataset(s) &
  \multicolumn{2}{c}{Template} &
  Verbalizer \\
 &
  System &
  User &
   \\ \midrule
XNLI, IndicXNLI &
  \begin{tabular}[c]{@{}l@{}}You are an NLP assistant whose purpose is to solve Natural Language Inference (NLI) problems. \\ NLI is the task of determining the inference relation between two (short, ordered) texts: entailment, \\ contradiction, or neutral.  Answer as concisely as possible in the same format as the examples below:\end{tabular} &
  \textbackslash{}texttt\{\{premise\}\}\textbackslash{}nQuestion: \{hypothesis\} True, False, or Neither? &
  \begin{tabular}[c]{@{}l@{}}Entailment : True, \\ Contradiction: False,\\ Neutral: Neither\end{tabular} \\
PAWS-X &
  \begin{tabular}[c]{@{}l@{}}You are an NLP assistant whose purpose is to perform Paraphrase Identification. The goal of \\ Paraphrase Identification is to  determine whether a pair of sentences have the same meaning. \\ Answer as concisely as possible in the same format as the examples below:\end{tabular} &
  \{sentence1\} Question: \{sentence2\} True or False? &
  - \\
XCOPA &
  \begin{tabular}[c]{@{}l@{}}You are an AI assistant whose purpose is to perform open-domain commonsense causal reasoning. \\ You will be provided a premise and two alternatives, where the task is to select the alternative that more \\ plausibly has a causal relation with the premise. Answer as concisely as possible in the same format as\\  the examples below:\end{tabular} &
  \begin{tabular}[c]{@{}l@{}}\{ premise \} \\ \{\% if question == "cause" \%\} This happened because... \\ \{\% else \%\} As a consequence... \{\% endif \%\} \\ Help me pick the more plausible option: - \{choice1\} - \{choice2\}\end{tabular} &
  - \\
\begin{tabular}[c]{@{}l@{}}XQUAD, TyDiQA,\\ MLQA, IndicQA\end{tabular} &
  \begin{tabular}[c]{@{}l@{}}You are an NLP assistant whose purpose is to solve reading comprehension problems. \\ You will be provided questions on a set of passages and you will need to provide the answer as it\\  appears in the passage. The answer should be in the same language as the question and the passage.\end{tabular} &
  \begin{tabular}[c]{@{}l@{}}\{context\} Q: \{question\}Referring to the passage above, \\ the correct answer to the given question is: \{answer\}\end{tabular} &
  - \\
XStoryCloze &
  - &
  \begin{tabular}[c]{@{}l@{}}\{input\_sentence\_1\} \{input\_sentence\_2\} \{input\_sentence\_3\} \{input\_sentence\_4\}\\ What is a possible continuation for the story given the following options ?\\ Option1: \{sentence\_quiz1\}\textbackslash{}n-Option2: \{sentence\_quiz2\}\end{tabular} &
  \begin{tabular}[c]{@{}l@{}}\{sentence\_quiz1\}: Option1, \\ \{sentence\_quiz2\}: Option2\end{tabular} \\
PANX &
  \begin{tabular}[c]{@{}l@{}}You are an NLP assistant whose purpose is to perform Named Entity Recognition (NER). \\ NER involves identifying and classifying named entities in a text into predefined categories such as \\ person names, organizations, locations, and others. You will need to use the tags defined below:\\ \textbackslash{}nO means the word doesn’t correspond to any entity.\textbackslash{}nB-PER/I-PER means the word corresponds \\ to the beginning of/is inside a person entity.\textbackslash{}nB-ORG/I-ORG means the word corresponds to the \\ beginning of/is inside an organization entity.\textbackslash{}nB-LOC/I-LOC means the word corresponds to the\\  beginning of/is inside a location entity.\textbackslash{}nDo not try to answer the question! Just tag each token in the sentence.\end{tabular} &
  \{token\_1 token\_2 ... token\_n\} &
  \begin{tabular}[c]{@{}l@{}}\{tag\_1\} \{tag\_2\} ... \{tag\_n\}: \\ \{token\_1\}\_\{tag\_1\} \{token\_2\}\_\{tag\_2\}\\  ... \{token\_n\}\_\{tag\_n\}\end{tabular} \\
UDPOS &
  \begin{tabular}[c]{@{}l@{}}You are an NLP assistant whose purpose is to perform Part of Speech (PoS) Tagging. PoS tagging is the \\ process of marking up a word in a text (corpus) as corresponding to a particular part of speech, based on\\  both its definition and its context. You will need to use the tags defined below:\end{tabular} &
  \{token\_1 token\_2 ... token\_n\} &
  \begin{tabular}[c]{@{}l@{}}\{tag\_1\} \{tag\_2\} ... \{tag\_n\}: \\ \{token\_1\}\_\{tag\_1\} \{token\_2\}\_\{tag\_2\} ... \\ \{token\_n\}\_\{tag\_n\}\end{tabular} \\
GLUECoS &
  \begin{tabular}[c]{@{}l@{}}You are an NLP assistant whose purpose is to solve Sentiment Analysis problems. Sentiment Analysis \\ is the task of determining whether the sentiment, opinion or emotion expressed in a textual data is: \\ positive, negative, or neutral. Answer as concisely as possible in the same format as the examples below:\end{tabular} &
  Does the following sentence have a positive, negative or neutral sentiment? \{text\} &
  - \\
XLSum &
  \begin{tabular}[c]{@{}l@{}}You are an NLP assistant whose purpose is to summarize any given article. You should summarize all \\ important information concisely in the same language in which you have been provided the document. \\ Following the examples provided below:\end{tabular} &
  \begin{tabular}[c]{@{}l@{}}\{document\} \\ === \\ Write a summary of the text above :\end{tabular} &
  - \\
WikiANN &
   &
   &
   \\
HONEST &
   &
   &
   \\
Jigsaw &
   &
   &
   \\
WinoMT &
   &
   &
   \\ \bottomrule
\end{tabular}%
}
\caption{Prompt type and prompt used for each dataset.}
\label{tab:prompt_template}
\end{table*}
\section{Conclusion}\label{sec:conclusion}
In this work, we focus on addressing the fundamental challenge of OOD detection tasks, which is how to fully understand the semantic discrepancy between the ID/OOD samples. We reveal that the key to success in the realistic SCOOD task is to allocate as many ID samples in the unlabeled set correctly as possible. To this end, we propose a novel uncertainty-aware optimal transport scheme that introduces class-specific energy scores as guidance for effective label assignment. Experimental results show that our method achieves better performance than previous state-of-the-art methods on SCOOD benchmarks.

\textbf{Limitations.} In addition to temperature scaling, other techniques such as feature clipping applied in ReAct~\cite{sun2021react} also enhance the performance of energy score, so how to obtain an OOD score that best fits the SCOOD task can be further explored. Moreover, a setting highly related to SCOOD has been proposed in \cite{katz2022training} and formulated as a constrained optimization problem. We will also theoretically analyze these practical OOD settings in our feature work.

% \section*{Acknowledgments}
\textbf{Acknowledgments.} 
This work is supported by National Key R\&D Program of China under Grant 2020AAA0105701, National Natural Science Foundation of China (NSFC) under Grants 61872327, Major Special Science and Technology Project of Anhui, National Natural Science Foundation of China (62033012) and Ant Group through Ant Research Intern Program.

\section{Limitations and Future Work}

We summarize the limitations we have identified for our method and propose
future research directions.

\textbf{Parallel implementation:} 
With a focus on accuracy and algorithms, our implementation for this work is
serial. Some of the most time-consuming routines in our method can easily
benefit from a parallel implementation, while the same is not obvious for the
SAP solver and the Schur complement computation. Leveraging the power of
parallelization on modern hardware for these computations is an interesting area
for future investigation.

\textbf{Rotational invariance:} 
As with all other linear constitutive models, our linearized model with lagged
rotational component is not rotationally invariant. Thus it is not suitable for
simulation of extreme deformations using large time steps. For those scenarios,
we fall back to traditional nonlinear models with Hessian positive definite
corrections proposed in \cite{bib:teran2005robust}.

\textbf{Self-contact:} 
We do not consider self-contact at the moment due to the lack of support by our
geometry engine. Self-contact can be incorporated into our method by updating the
geometry engine to augment the set of contacts reported.

\textbf{Tunneling at high speeds:} Though our method has a lower computational
cost, it could benefit from continuous collision detection strategies
\cite{bib:li2020ipc} to provide constraints before contact is established. This
would allow to mitigate issues such as objects tunneling past each other at high
speeds. Efficient solution to mitigate this issue is a topic of active research
for the authors.

\textbf{Redundant constraints:} Our geometry engine often introduces a large
number of constraints to resolve contact. Similarly, welding a large number of
deformable mesh vertices to a rigid body (as done in Section
\ref{sec:bubble_gripper}) introduces many constraints. Even though our SAP
solver \cite{bib:castro2022unconstrained} provides existence and uniqueness
guarantees, a large number of constraints hurts performance as can be observed
in the \emph{Soft-bubble} example. We are currently investigating strategies to
significantly reduce the number of constraints without sacrificing accuracy.



% \section*{Ethics Statement}



% Scientific work published at EMNLP 2022 must comply with the \href{https://www.aclweb.org/portal/content/acl-code-ethics}{ACL Ethics Policy}. We encourage all authors to include an explicit ethics statement on the broader impact of the work, or other ethical considerations after the conclusion but before the references. The ethics statement will not count toward the page limit (8 pages for long, 4 pages for short papers).

% \section*{Acknowledgements}

% Entries for the entire Anthology, followed by custom entries
\bibliography{anthology,custom}
\bibliographystyle{acl_natbib}

\appendix
\section{Appendix for Proofs}

\paragraph{Proof of Theorem \ref{thm:main}.}

\begin{proof}
\label{proof:main}
Our proof has two steps. In Step 1, we will show that SimCLR is equivalent to minimizing the cross entropy loss defined in Eqn.~(\ref{eqn:cross-entropy}). 
In Step 2, we will show  that minimizing the cross-entropy loss 
is equivalent to spectral clustering on $\bfpi$. 
Combining the two steps together, we have proved our theorem. 

\textbf{Step 1: } SimCLR is equivalent to minimizing the cross entropy loss.

The cross-entropy loss takes expectation over 
$\bfW_\bfX\sim \mathbb{P}(\cdot ; \bfpi)$, 
which means $\bfW_\bfX$ has exactly one non-zero entry in each row $i$. By Lemma~\ref{lem:multinomial}, we know every row $i$ of $\bfW_\bfX$ is independent of other rows. Moreover, 
$\bfW_{\bfX,i}\sim \mathcal{M}(1, \bfpi_i/\sum_j \bfpi_{i,j})=\mathcal{M}(1, \bfpi_i)$, because $\bfpi_i$ itself is a probability distribution.
Similarly, we know $\bfW_\bfZ$ also has the row-independent property by sampling over $\mathbb{P}(\cdot;\bfK_\bfZ)$.
Therefore, by Lemma~\ref{lem:cross_split}, we know Eqn.~(\ref{eqn:cross-entropy}) is equivalent to:
\[
 -\sum_{i=1}^n \mathbb{E}_{\bfW_{\bfX,i}}[\log \mathbb{P}(\bfW_{\bfZ,i}=\bfW_{\bfX,i};\bfK_\bfZ)],
\]

This expression takes expectation over $\bfW_{\bfX,i}$ for the given row $i$. Notice that 
$\bfW_{\bfX,i}$ has exactly one non-zero entry, which equals $1$ (same for $\bfW_{\bfZ,i}$). 
As a result
we expand the above expression to be:
\begin{equation}
 -\sum_{i=1}^n \sum_{j\neq i} \Pr(\bfW_{\bfX,i,j}=1)\log \Pr(\bfW_{\bfZ,i,j}=1).
\label{eqn:detailed-expansion}    
\end{equation}


By Lemma~\ref{lem:multinomial}, $\Pr(\bfW_{\bfZ,i,j}=1)=\bfK_{\bfZ,i,j}/\|\bfK_{\bfZ,i}\|_1$ for $j\neq i$. Recall that $\bfK_\bfZ=(k(\bfZ_i-\bfZ_j))_{(i,j)\in[n]^2}$, which means 
$\bfK_{\bfZ,i,j}/\|\bfK_{\bfZ,i}\|_1=\frac{\exp(-\|\bfZ_i-\bfZ_j\|^2/{2\tau})}{\sum_{k\neq i}
\exp(-\|\bfZ_i-\bfZ_k\|^2/{2\tau})
}$ for $j\neq i$, when $k$ is the Gaussian kernel with variance $\tau$. 

Notice that $\bfZ_i=f(\bfX_i)$, so we know
\begin{equation}
-\log \Pr(\bfW_{\bfZ,i,j}=1)=
-\log \frac{\exp(-\|f(\bfX_i)-f(\bfX_j)\|^2/{2\tau})}{\sum_{k\neq i}
\exp(-\|f(\bfX_i)-f(\bfX_k)\|^2/{2\tau}),
}
\label{eqn:infonce-equivalence}    
\end{equation}


The right hand side is exactly the InfoNCE loss defined in Eqn.~(\ref{eqn:infonce}).
Inserting Eqn.~(\ref{eqn:infonce-equivalence}) into Eqn.~(\ref{eqn:detailed-expansion}), we get the SimCLR algorithm, which first samples augmentation pairs $(i,j)$ with $\Pr(\bfW_{\bfX,i,j}=1)$ for each row $i$, and then optimize the InfoNCE loss. 

\textbf{Step 2: } minimizing the cross entropy loss 
is equivalent to spectral clustering on $\bfpi$.


By Lemma~\ref{lem:convert_to_spectral}, we may further convert the loss to 
\begin{equation}
\label{eqn:main-theorem-repul-attr}
\min_{\bfZ}
-\sum_{(i,j)\in [n]^2} \mathbf{P}_{i,j}
\log k (\bfZ_i-\bfZ_j)+\log \mathbf{R}(\bfZ).
\end{equation}
Since $k$ is the Gaussian kernel, this reduces to \[
\min_\bfZ \mathrm{tr}(\bfZ^\top \mathbf{L}(\bfpi) \bfZ)
+\log \mathbf{R}(\bfZ),
\]

where we use the fact that $\mathbb{E}_{\bfW_\bfX\sim \mathbb{P}(\cdot; \bfpi)}[\mathbf{L}(\bfW_\bfX)]
=\mathbf{L}(\bfpi)
$, because the Laplacian operator is linear and $
\mathbb{E}_{\bfW_\bfX\sim \mathbb{P}(\cdot; \bfpi)}(\bfW_\bfX)=\bfpi
$.
\end{proof}

\paragraph{Proof of Theorem \ref{thm:clip}.}
\begin{proof}
Since $\bfW_\bfX\sim \mathbb{P}(\cdot;\bfpi_{\mathbf{A}, \mathbf{B}})$, we know 
$\bfW_\bfX$ has exactly one non-zero entry in each row, denoting the pair that got sampled. 
A notable difference compared to the previous proof is we now have $n_\mathcal{A}+n_\mathcal{B}$ objects in our graph. CLIP deals with this by taking a mini-batch of size $2N$, 
such that $n_\mathcal{A}=n_\mathcal{B}=N$, and adding the $2N$ InfoNCE losses together. We label the objects in $\mathcal{A}$ as $[n_\mathcal{A}]$, and the objects in $\mathcal{B}$ as $\{n_\mathcal{A}+1, \cdots, n_\mathcal{A}+n_\mathcal{B}\}$. 

Notice that $\bfpi_{\mathbf{A}, \mathbf{B}}$ is a bipartite graph, so the edges of objects in $\mathcal{A}$ will only connect to object in $\mathcal{B}$ and vice versa. We can define the similarity matrix in $\cZ$ as $\bfK_\bfZ$, 
where $\bfK_\bfZ(i, j+n_\mathcal{A})=\bfK_\bfZ(j+n_\mathcal{A},i)= k(\bfZ_i-\bfZ_j)$ for $i\in [n_\mathcal{A}], j\in [n_\mathcal{B}]$, and otherwise we set $\bfK_\bfZ(i,j)=0$. 
The rest is same as the previous proof. 
\end{proof}

\paragraph{Proof of Theorem \ref{thm:exponential}.}

\begin{proof}
\label{proof:exponential}
Since the objective function consists of a linear term combined with an entropy regularization, which is a strongly concave function, the maximization problem is a convex optimization problem. Owing to the implicit constraints provided by the entropy function, the problem is equivalent to having only the equality constraint. We then introduce the Lagrangian multiplier $\lambda$ and obtain the following relaxed problem:

$$
\widetilde{E}(\boldsymbol{\alpha})=\psi_{1}-\sum_{i=1}^n \alpha_{i} \psi_{i}+\tau \sum_{i=1}^n \alpha_{i}\log \alpha_{i}+\lambda\left(\boldsymbol{\alpha}^{\top} \mathbf{1}_n-1\right).
$$

As the relaxed problem is unconstrained, taking the derivative with respect to $\alpha_{i}$ yields

$$
\frac{\partial \widetilde{E}(\boldsymbol{\alpha})}{\partial \alpha_{i}}=-\psi_{i}+\tau\left(\log \alpha_{i}+\alpha_{i} \frac{1}{\alpha_{i}}\right)+\lambda=0.
$$

Solving the above equation implies that $\alpha_{i}$ takes the form
$
\alpha_{i}=\exp \left(\frac{1}{\tau} \psi_{i}\right) \exp \left(\frac{-\lambda}{\tau}-1\right).
$ Since $\alpha_{i}$ lies on the probability simplex, the optimal $\alpha_{i}$ is explicitly given by
$
\alpha^{*}_{i}=\frac{\exp \left(\frac{1}{\tau} \psi_{i}\right)}{\sum_{i^{\prime}=1}^n \exp \left(\frac{1}{\tau} \psi_{i^{\prime}}\right)} .
$ Substituting the optimal point into the objective function, we obtain
$$
\begin{aligned}
E\left(\boldsymbol{\alpha}^*\right)  &=\psi_1-\sum_{i=1}^n \frac{\exp \left(\frac{1}{\tau} \psi_{i}\right)}{\sum_{i^{\prime}=1}^n \exp \left(\frac{1}{\tau} \psi_{i^{\prime}}\right)} \psi_{i}+\tau \sum_{i=1}^n \frac{\exp \left(\frac{1}{\tau} \psi_{i}\right)}{\sum_{i^{\prime}=1}^n \exp \left(\frac{1}{\tau} \psi_{i^{\prime}}\right)}\log \frac{\exp \left(\frac{1}{\tau} \psi_{i}\right)}{\sum_{i^{\prime}=1}^n \exp \left(\frac{1}{\tau} \psi_{i^{\prime}}\right)} \\
& =\psi_1 - \tau \log \left(\sum_{i=1}^n \exp \left(\frac{1}{\tau} \psi_{i}\right)\right).
\end{aligned}
$$
Thus, the Lagrangian dual function is given by
\begin{equation*}
-E\left(\boldsymbol{\alpha}^*\right)= -\tau \log \frac{\exp \left(\frac{1}{\tau} \psi_{1}\right)}{\sum_{i=1}^n \exp \left(\frac{1}{\tau} \psi_{i}\right)}.\qedhere
\end{equation*}
\end{proof}



\section{More on Experiments} \label{section: experiment_details}

\paragraph{CIFAR-10 and CIFAR-100} CIFAR-10 ~\citep{krizhevsky2009learning} and CIFAR-100 ~\citep{krizhevsky2009learning} are well-known classic image classification datasets. Both CIFAR-10 and CIFAR-100 contain a total of 60k $32 \times 32$ labeled images of different classes, with 50k for training and 10k for testing. CIFAR-10 is similar to CIFAR-100, except there are 10 different classes in CIFAR-10 and 100 classes in CIFAR-100.

\paragraph{TinyImageNet} TinyImageNet ~\citep{le2015tiny} is a subset of ImageNet ~\citep{deng2009imagenet}. There are 200 different object classes in TinyImageNet, with 500 training images, 50 validation images, and 50 test images for each class. All the images in TinyImageNet are colored and labeled with a size of $64 \times 64$.

\textbf{Pseudo-code.} Algorithm \ref{alg:Training Procedure} presents the pseudo-code for our empirical training procedure.

\begin{algorithm}[!htbp]
\caption{Training Procedure}
\label{alg:Training Procedure}
\begin{algorithmic}[1]
\REQUIRE trainable encoder network $f$, batch size $N$, augmentation strategy \textit{aug}, loss function $L$ with hyperparameters \textit{args}
\FOR {sampled minibatch ${x_i}_{i=1}^N$}
\FORALL{$i \in { 1, ..., N }$}
\STATE draw two augmentations $t_i = \textit{aug}\left(x_i\right) $, $t_i' = \textit{aug}\left(x_i\right) $
\STATE $z_i = f\left(t_i\right)$, $z_i' = f\left(t_i'\right)$
\ENDFOR
\STATE compute loss $\mathcal{L} = L(N, z, z', \textit{args})$
\STATE update encoder network $f$ to minimize $\mathcal{L}$
\ENDFOR
\STATE \textbf{Return} encoder network $f$
\end{algorithmic}
\end{algorithm}

We also provide the pseudo-code for our core loss function used in the training procedure in Algorithm \ref{alg:Core loss}. The pseudo-code is almost identical to SimCLR's loss function, with the exception of an extra parameter $\gamma$.

\begin{algorithm}[!htbp]
\caption{Core loss function $\mathcal{C}$}
\label{alg:Core loss}
\begin{algorithmic}[1]
\REQUIRE batch size $N$, two encoded minibatches $z_1, z_2$, $\gamma$, temperature $\tau$
\STATE $z = \textit{concat}\left(z_1, z_2\right)$
\FOR {$i \in {1, ..., 2N }, j \in {1, ..., 2N}$ }
\STATE $s_{i,j} = \Vert z_i - z_j \Vert_2^{\gamma}$
\ENDFOR
\STATE \textbf{define} $l(i, j)$ \textbf{as} $l(i, j) = - \log \frac{exp\left(s_{i,j}/\tau \right)}{\sum_{k=1}^{2N} \mathbf{1}{[k \ne i]} exp\left(s{i, j} / \tau \right)} $
\STATE \textbf{Return} $\frac{1}{2N} \sum_{k=1}^N\left[l(i, i+N) + l(i+N, i)\right]$
\end{algorithmic}
\end{algorithm}

Utilizing the core loss function $\mathcal{C}$, we can define all kernel loss functions used in our experiments in Table \ref{table: loss definition}. For all $z_i \in z$ with even dimensions $n$, we define $z_{L_i} = z_i\left[0:n/2\right]$ and $z_{R_i} = z_i\left[n/2:n\right]$.

\begin{table}[ht]
\centering
\begin{tabular}{{@{}l|l@{}}}
Kernel  &  Loss function \\ \midrule
Laplacian & $\mathcal{C}\left(N, z, z', \gamma=1, \tau\right)$\\ \midrule
Sum       & $\lambda * \mathcal{C}\left(N, z, z', \gamma=1, \tau_1\right) + (1-\lambda) * \mathcal{C}\left(N, z, z', \gamma=2, \tau_2\right)$  \\ \midrule
Concatenation Sum&$\lambda * \mathcal{C}\left(N, z_L, z'_L, \gamma=1, \tau_1\right) + (1-\lambda) * \mathcal{C}\left(N, z_R, z'_R, \gamma=2, \tau_2\right)$\\ \midrule
$\gamma = 0.5$ & $\mathcal{C}\left(N, z, z', \gamma=0.5, \tau\right)$          \\ 

\end{tabular}

\caption{Definition of kernel loss functions in our experiments}
\label {table: loss definition}
\end{table}

\textbf{Baselines.} We reproduce the SimCLR algorithm using PyTorch Lightning~\citep{PytorchLightning}.

\textbf{Encoder details.}
The encoder $f$ consists of a backbone network and a projection network. We employ ResNet50~\citep{ResNet} as the backbone and a 2-layer MLP (connected by a batch normalization~\citep{ioffe2015batch} layer and a ReLU \cite{nair2010rectified} layer) with hidden dimensions 2048 and output dimensions 128 (or 256 in the concatenation kernel case).

\textbf{Encoder hyperparameter tuning.}
For each encoder training case, we randomly sample 500 hyperparameter groups (sample details are shown in Table \ref{table: Hyperparameter sample}) and train these samples simultaneously using Ray Tune ~\citep{RayTune}, with the ASHA scheduler~\citep{li2018massively}. Ultimately, the hyperparameter group that maximizes the online validation accuracy (integrated in PyTorch Lightning) within 5000 validation steps is chosen for the given encoder training case.

\begin{table}[ht]
\centering

\begin{tabular}{@{}l|l|l@{}}
\midrule
Hyperparameter  & Sample Range & Sample Strategy \\ \midrule
start learning rate & $\left[10^{-2}, 10\right]$ & log uniform \\ \midrule
$\lambda$       & $\left[0, 1\right]$ & uniform \\ \midrule
$\tau$, $\tau_1$, $\tau_2$ & $\left[0, 1\right]$ & log uniform \\ \midrule
\end{tabular}

\caption{Hyperparameters sample strategy}
\label {table: Hyperparameter sample}
\end{table}

\textbf{Encoder training.} 
We train each encoder using the LARS optimizer~\citep{LARSOptimizer}, LambdaLR Scheduler in PyTorch, momentum 0.9, weight decay $10^{-6}$, batch size 256, and the aforementioned hyperparameters for 400 epochs on a single A-100 GPU.

\textbf{Image transformation.} The image transformation strategy, including augmentation, is identical to the default transformation strategy provided by PyTorch Lightning.

\textbf{Linear evaluation.}
The linear head is trained using the SGD optimizer with a cosine learning rate scheduler, batch size 64, and weight decay $10^{-6}$ for 100 epochs. The learning rate starts at $0.3$ and ends at $0$.

\textbf{Moco Experiments.} We also tested our method based on MoCo~\citep{he2019moco}. The results are summarized in Table \ref{tab:results-moco}. Here we choose ResNet18~\citep{ResNet} as the backbone and set a temperature of $0.1$ as default. For our simple sum kernel, we set $\lambda=0.8$. The results show that our method outperforms the original MoCo method.

\begin{table}[thb]
\centering
\caption{MoCo Experiment Results on CIFAR-10 and CIFAR-100.}
\label{tab:results-moco}
\resizebox{\textwidth}{!}{%
\begin{tabular}{@{}c|ccc|ccc@{}}
\toprule
\multirow{3}{*}{Method} & \multicolumn{3}{c|}{CIFAR-10} & \multicolumn{3}{c}{CIFAR-100} \\ \cmidrule(lr){2-4} \cmidrule(lr){5-7} 
                        & 200 epochs & 400 epochs    & 1000 epochs   & 200 epochs & 400 epochs & 1000 epochs         \\ \midrule
MoCo (repro.)         & $76.41 \pm 0.12$    & $80.01 \pm 0.15$          & $84.45 \pm 0.08$    & $\mathbf{47.02 \pm 0.11}$ & $52.50 \pm 0.07$ & $57.62 \pm 0.15$            \\
\midrule
Laplacian Kernel        & ${78.09 \pm 0.10}$    & $\mathbf{83.85 \pm 0.09}$          & $\mathbf{88.34 \pm 0.16}$    & $46.12 \pm 0.22$   & $53.44 \pm 0.17$ & $59.10 \pm 0.14$        \\
Simple Sum Kernel & $\mathbf{78.12 \pm 0.15}$   & $83.23 \pm 0.18$ & $87.50 \pm 0.20$ & $46.65 \pm 0.06$ & $\mathbf{53.62 \pm 0.19}$ & $\mathbf{59.83 \pm 0.12}$\\
\bottomrule
\end{tabular}
}
\end{table}



\section{More Experiments on Synthetic Data}


Consider a scenario with $n$ clusters, each containing $k$ vertices. Let the probability of vertices $u$ and $v$ from the same cluster belonging to $\bfpi$ be $p$. Conversely, for vertices $u$ and $v$ from different clusters, let the probability of belonging to $\pi$ be $q$. We generate the graph $\bfpi$ randomly, based on $p$ and $q$. We experiment with values of $k=100$ and $n=6$ for ease of visualization, embedding all points in a two-dimensional space. Each vertex's initial position originates from a normal distribution. In each iteration, we sample a subgraph of $\bfpi$ uniformly, ensuring each vertex has an out-degree of $1$. We then optimize the corresponding vectors using InfoNCE loss with an SGD optimizer and iterate until convergence. Our experimental setup consists of an SGD learning rate of $1$, an InfoNCE loss temperature of $0.5$, and a batch size of $50$. We evaluate two scenarios with different $p$ and $q$ values: $p=1$, $q=0$, and $p=0.75$, $q=0.2$. The results of these experiments are visualized in Figure \ref{fig:vis-spectral-cluster}. The obtained embeddings exhibit the hallmark pattern of spectral clustering of graph $\bfpi$.

\begin{figure}[!tb]
\centering
\subfigure{
\includegraphics[width=1\textwidth]{Figures/cluster_pi.png}
\label{fig:vis-cluster}
}
\subfigure{
\includegraphics[width=1\textwidth]{Figures/noised_cluster_pi.png}
\label{fig:vis-noised-cluster}
}
\caption{Visualizations of the optimization process using InfoNCE Loss on the vectors corresponding to $\bfpi$. Points of identical color belong to the same cluster within $\bfpi$. To showcase the internal structure of $\bfpi$, we randomly select 10 vertices from each cluster to display the edge distribution of $\bfpi$.}
\label{fig:vis-spectral-cluster}
\end{figure}



\end{document}
