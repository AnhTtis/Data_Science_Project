%%%%%%%%%%%%%%%%%%%%%%%%%%%%%%%%%%%%%%%%%%%%%%%%%%%%%%%%%%%%%%%%%%%%%%%%%%%%%%%%
%2345678901234567890123456789012345678901234567890123456789012345678901234567890
%        1         2         3         4         5         6         7         8

\documentclass[letterpaper, 10 pt, conference]{ieeeconf}  % Comment this line out
% if you need a4paper
%\documentclass[a4paper, 10pt, conference]{ieeeconf}      % Use this line for a4
                                                          % paper

\IEEEoverridecommandlockouts                              % This command is only
%                                                           % needed if you want to
%                                                           % use the \thanks command
\overrideIEEEmargins
% See the \addtolength command later in the file to balance the column lengths
% on the last page of the document



% The following packages can be found on http:\\www.ctan.org
%\usepackage{graphics} % for pdf, bitmapped graphics files
%\usepackage{epsfig} % for postscript graphics files
% \usepackage{mathptmx} % assumes new font selection scheme installed
% \usepackage{times} % assumes new font selection scheme installed
\usepackage{amsmath} % assumes amsmath package installed
\usepackage{amssymb}  % assumes amsmath package installed

% Optional math commands from https://github.com/goodfeli/dlbook_notation.
\newcommand{\bbox}{\text{bbox}}
\newcommand{\alphapck}{\alpha_\bbox}
\newcommand{\kcycle}{\text{k-CyPCK}}
\newcommand{\cycle}{\text{-CyPCK}}

\newcommand{\I}{\mathbf{I}}
\newcommand{\Ia}{\I^\text{a}}
\newcommand{\Ib}{\I^\text{b}}
\newcommand{\Iatob}{\I^\text{a $\rightarrow$ b}}
\newcommand{\F}{\mathbf{F}}
\newcommand{\Fa}{\F^\text{a}}
\newcommand{\Fb}{\F^\text{b}}
\newcommand{\f}{\mathbf{f}}
\newcommand{\fa}{\f^\text{a}}
\newcommand{\fb}{\f^\text{b}}
\newcommand{\p}{\mathbf{p}}
\newcommand{\pa}{\p^\text{a}}
\newcommand{\pb}{\p^\text{b}}
\newcommand{\A}{\boldsymbol{\Phi}_\text{align}}
\newcommand{\G}{\mathbf{G}}
\newcommand{\C}{\mathbf{C}}
\newcommand{\Ca}{\C^\text{a}}
\newcommand{\Cb}{\C^\text{b}}
\newcommand{\cc}{\mathbf{c}}
\newcommand{\cca}{\cc^\text{a}}
\newcommand{\ccb}{\cc^\text{b}}
\newcommand{\Irec}{\I_\text{Recon}}
\newcommand{\M}{\mathbf{M}}
\newcommand{\Mrec}{\M_\text{Recon}}
\newcommand{\loss}{\mathcal{L}}
\newcommand{\T}{\mathcal{T}}
\newcommand{\W}{\mathcal{W}}
\newcommand{\Id}{\mathcal{I}}
 %from TMLR template, standard math notation


\usepackage[style=ieee]{biblatex}
\addbibresource{bib.bib}

\usepackage{hyperref}
% \usepackage{url}

% \usepackage{mathtools} % For \begin{bmatrix*}[l] 
% \usepackage{amsmath}
% \usepackage{amssymb}
\usepackage{graphicx}
\usepackage{xcolor}         % colors


\newcommand{\e}{\mathbf{e}}
\newcommand{\br}{\mathbf{r}}
\newcommand{\x}{\mathbf{x}}
\newcommand{\y}{\mathbf{y}}
\newcommand{\z}{\mathbf{z}}
\newcommand{\w}{\mathbf{w}}
\newcommand{\bv}{\mathbf{v}}
\newcommand{\reals}{\mathbb{R}}

\newcommand{\hyf}{\hat{\y}_{f}}
% \newcommand{\hyfi}{\hat{\y}_{f,i}}
\newcommand{\hyfi}{\hyf}

\newcommand{\bP}{\mathbf{P}}
\newcommand{\bE}{\mathbf{E}}
\newcommand{\F}{\mathbf{F}}
\newcommand{\bnu}{\bm{\nu}}
\newcommand{\etab}{\bm{\eta}}
% \usepackage{amsmath}
% \usepackage{amssymb}


\newtheorem{Assumption}{Assumption}[section]
\newtheorem{lemma}{Lemma}[section]
\newtheorem{Problem}{Problem}[section]
\newtheorem{Remark}{Remark}[section]
\newtheorem{Definition}{Definition}[section]
\newtheorem{Theorem}{Theorem}[section]
\newtheorem{Corollary}{Corollary}[section]
\newtheorem{Proof}{Proof}[section]

% \usepackage{bm}


\title{\LARGE \bf  PAC-Bayesian bounds for learning LTI-ss systems with input from empirical loss}

% Authors must not appear in the submitted version. They should be hidden
% as long as the tmlr package is used without the [accepted] or [preprint] options.
% Non-anonymous submissions will be rejected without review.

% \author{\name Deividas Eringis \email der@es.aau.dk\\
%   \addr Department of Electronic Systems\\
%   Aalborg University\\
%   \AND
%   \name John Leth \email jjl@es.aau.dk\\
%      \addr Department of Electronic Systems\\
%   Aalborg University\\
%      \AND
%   \name Zheng-Hua Tan \email zt@es.aau.dk\\
%      \addr Department of Electronic Systems\\
%   Aalborg University\\
%      \AND
%   \name Rafal Wisniewski \email raf@es.aau.dk\\
%      \addr Department of Electronic Systems\\
%   Aalborg University\\
%   \AND
%   \name Mihaly Petreczky \email mihaly.petreczky@centralelille.fr\\
%   \addr Laboratoire Signal et Automatique de Lille (CRIStAL) \\
% }
% \author{ \parbox{3 in}{\centering Huibert Kwakernaak*
%         \thanks{*Use the $\backslash$thanks command to put information here}\\
%         Faculty of Electrical Engineering, Mathematics and Computer Science\\
%         University of Twente\\
%         7500 AE Enschede, The Netherlands\\
%         {\tt\small h.kwakernaak@autsubmit.com}}
%         \hspace*{ 0.5 in}
%         \parbox{3 in}{ \centering Pradeep Misra**
%         \thanks{**The footnote marks may be inserted manually}\\
%        Department of Electrical Engineering \\
%         Wright State University\\
%         Dayton, OH 45435, USA\\
%         {\tt\small pmisra@cs.wright.edu}}
% }

\author{Deividas Eringis*, John Leth, Zheng-Hua Tan, Rafal Wisniewski, Mihaly Petreczky% <-this % stops a space
\thanks{This work was not supported by any organization}% <-this % stops a space
\thanks{D. Eringis, J. Leth, Z. Tan, R. Wisniewski is with Department of Electronic Systems,
        Aalborg University, Aalborg, Denmark
        {\tt\small \{der,jjl,zt,raf\}@es.aau.dk}}%
\thanks{Mihaly Petreczky is with Laboratoire Signal et Automatique de Lille (CRIStAL), Lille, France {\tt\small mihaly.petreczky@centralelille.fr}}%
}


\begin{document}


\maketitle

\begin{abstract}
In this paper we derive a Probably Approxilmately Correct(PAC)-Bayesian error  bound for linear time-invariant (LTI) stochastic dynamical systems with inputs. Such bounds
are widespread in machine learning, and they are useful for characterizing the predictive power of models learned from  
finitely many data points.  
In particular, with  the bound derived in this paper relates
future average prediction errors with the prediction error 
generated by the model on the data used for learning.
In turn, this allows us to provide finite-sample  error bounds for
a wide class of learning/system identification algorithms. 
Furthermore, as LTI systems are a sub-class of recurrent neural
networks (RNNs), these error bounds could be a first step towards 
PAC-Bayesian bounds for RNNs. 
\end{abstract}

\section{Introduction}
Linear time invariant (LTI) state-space models have been widely used in control and econometric applications to model time-series and 
have rich literature on learning (classically called identification)\cite{LjungBook}. 

In this paper, we present PAC-Bayesian type bounds on learning LTI systems from data generated by LTI system driven by zero-mean, i.i.d., Gaussian or sub-Gaussian noise. 

The Probably Approximately Correct (PAC)-Bayesian framework, provides theoretical guarantees (with arbitrary high probability) on the difference between learning from infinite amount of data, and learning from finite empirical data, see \cite{guedj2019primer,alquier2021userfriendly,zhang-06,grunwald-2012,alquier-15,nips-16,ShethK17}. 

\textbf{Motivation}
PAC and PAC-Bayesian bounds have been a major tool for analyzing
learning algorithms. 
They provide bounds on the generalization error in terms of the empirical error,
in a manner which is independent of the learning algorithm. Hence, these bounds can be used to analyze and explain a wide variety of learning algorithms. Moreover,
by minimizing the error bound, new, theoretically well-founded learning algorithms can be 
formulated. In particular, PAC-Bayesian error bounds turned out to be useful for 
providing non-vacuous error bounds for neural networks \cite{Dziugaite2017}.
\par
  While there is a wealth of literature on PAC \cite{shalev2014understanding} and PAC-Bayesian \cite{alquier2021userfriendly,guedj2019primer},
  bounds for static models, much less is known on  dynamical systems.
  \par

    Traditionally, the literature on LTI systems \cite{LjungBook} has focused on statistical consistency. More recently, several results have appeared on finite-sample bounds for learning LTI systems, but they are valid only for specific learning algorithms or for very limited subclasses
    \cite{simchowitz2021statistical,Ozay2018,oymak2021revisiting},


   \textbf{Contribution}
     In this paper we consider stochastic LTI state-space representations (LTI systems for short)
     in innovation form.
     In accordance with the standard
     practice in system identification, we view
     stochastic LTI systems as predictors, which take past inputs and
     outputs and generate predictions for the current
     output. We assume that the data used for learning (system identification) are generated by
     stochastic LTI systems in innvation form too.
     Learning/identifying
      an LTI system is then amounts to
     finding the best predictor, i.e., the predictor
     which results in the smallest prediction error
     for the training data, i.e., in the smalled
     \emph{empirical loss}
     However, for decision making (fault detection,control, etc.), the quality of the learned model is determined by the 
     \emph{generalization
     error}, i.e., the average prediction error
     for future, unseen data. The PAC-Bayesian
     bound of this paper says that with a high probability (probability $1-\delta$), the generalization error
     is smaller than the empirical loss plus a 
     an error term. The error term depends on the
     number of data points $N$ and on parameter
     (learning rate $\lambda$). 
     In this paper we provide explicit formulas for
     the error term. We show that the error
     term converges to a constant as $N \rightarrow \infty$. The constant depends on the confidence
     level $\delta$ and the distance  between
     prior and posterior densities on models. 
     If we assume that the data used for learning is generated by an LTI system with \emph{bounded noise}, we can show that the error term
     converges to $0$ as $N \rightarrow \infty$.
     The rate of convergence is $O(\frac{1}{\sqrt{N}})$, which is consistent with most of finite-sample bounds available in the literature for various, not necessarily LTI, models. This suggests that the obtained error
     bound is likely to be asymptotically sharp for bounded signals.
        
    \textbf{Related work}
%\label{related:work}
   The related literature can be divided into the following categories.
  \\
   \textbf{Generalization bounds for RNNs.}
      PAC bounds for RNN were developed in \cite{KOIRAN199863,sontag1998learning,pmlr-v108-chen20d} using VC dimension, and
      in  \cite{pmlr-v161-joukovsky21a, pmlr-v108-chen20d} using Rademacher complexity, and in 
      \cite{pmlr-v80-zhang18g} using PAC-Bayesian bounds approach.
      However, all the cited papers assume noiseless models,  
      a fixed number of time-steps, that the training data are i.i.d sampled time-series, and the signals are bounded.
      In contrast, we consider (1) noisy models, (2) prediction error defined on infinite time horizon, (3) only one single time series available for training data, and (4) unbounded signals. 
      Moreover, several
      papers \cite{KOIRAN199863,sontag1998learning,pmlr-v144-hanson21a} assume
      Lipschitz loss functions, while we use quadratic loss function.

\textbf{Finite-sample bounds for system identification of LTI systems.}
Guarantees for asymptotic convergence of learning algorithms is a classical
topic in system identification \cite{LjungBook}.
Recently, several publications on finite-sample  bounds for learning linear dynamical systems were derived, without claiming completeness \cite{simchowitz2018learning,simchowitz2019learning,simchowitz2021statistical,oymak2021revisiting,lale2020logarithmic,foster2020learning,NEURIPS2018_d6288499,Pappas1,SarkarRD21}. 
% \cite{simchowitz2018learning,simchowitz2019learning}
First, all the cited papers propose a bound which is valid only for models generated by a specific learning algorithm. In particular, these bounds do not relate the generalization loss with the empirical loss for arbitrary models, i.e., they are not PAC(-Bayesian) bounds. This means that in contrast to the results of this paper, the bounds of the cited papers cannot be use for analyzing algorithms others than for which they were derived.
Second, many of the cited papers
do not derive bounds on the infinite horizon prediction error.
More precisely, \cite{oymak2021revisiting,SarkarRD21,lale2020logarithmic,Pappas1,Simchowitz_Foster_2020} provided error bounds for the difference of the first $T$ Markov-parameters of the estimated and true system for a specific
identification algorithm. However, in order to characterize the infinite horizon prediction error,
we need to take $T=\infty$.
For $T=\infty$ the cited bounds become infinite, i.e., vacuous. 
In addition, in contrast to the present paper,  \cite{oymak2021revisiting,SarkarRD21,simchowitz2018learning} deals only with the deterministic part of the stochastic LTI, \cite{Pappas1} deals only with the stochastic part.
% Note that the error bounds of
% the cited papers
%\cite{simchowitz2019learning} ,lale2020logarithmic,NEURIPS2018_d6288499,Pappas1,SarkarRD21}
% converge to their limit at rate $O(\frac{\ln(N)}{\sqrt{N}})$, which is comparable to the rate $O(\frac{1}{\sqrt{N}})$ of this paper.
% This being said, the cited papers 
% provide bounds on the parameter estimation error, and many of them allow marginally stable systems.

\textbf{PAC-Bayesian bounds for state-space representation.}
In \cite{haussmann2021learning} learning of stochastic differential equations without inputs was considered and it was assumed that  several independently sampled time-series were available for learning. 
In contrast, in this paper we deal with discrete-time systems with inputs and the learning takes place from a single time-series.
In \cite{PACMarkov} learning of general Markov-chains was considered, but the state of the Markov-chain was assumed to be observable and no inputs were considered. The learning problem of \cite{PACMarkov} is thus different from the one considered in this paper.

In \cite{CDC21paper} %cdc paper
PAC-Bayesian error bounds were developed for autonomous LTI state-space systems without exogenous input. 
In contrast to \cite{CDC21paper}, in the current paper we consider
systems with exogenous inputs.
Moreover, the error bound of this
paper is much tighter than that of \cite{CDC21paper}: 
in contrast to \cite{CDC21paper}, with the growth of the number of observations,
the error bounds of this paper converge either to zero (in the case of bounded innovation noise) or to a constant involving KL-divergence. 
Finally, the proof technique is completely different from that of
\cite{CDC21paper}.

\textbf{Paper Outline}
We start by defining the problem formulation in Section \ref{sect:Problem_formulation}, where all the assumptions and important quantities are defined. Then we will discuss the PAC-Bayesian framework in Section \ref{sect:pac:gen}, then we will present the main results of the paper in Section \ref{sec:mainResults}, then we will present some auxiliary results for systems driven by bounded noise in Section \ref{sec:boundedResults}, We will finish off with a short numerical example in Section \ref{sec:numEx}. Finally, we will have the conclusion in Section \ref{sect:concl}.

\section{Problem formulation}
\label{sect:Problem_formulation}
\subsection*{ Notation and terminology} 
We occasionally use $\triangleq$ to denote ''defined by''. 
Let $\F$ denote a $\sigma$-algebra on the set $\Omega$ and $\bP$ be a probability measure on $\F$. Unless otherwise stated all probabilistic considerations will be with respect to the probability space $(\Omega,\F,\bP)$, and we let $\bE(\z)$ denote expectation of the stochastic variable $\z$.
We use bold face letters to indicate stochastic variables/processes. 
Each euclidean space is associated with the topology generated by the 2-norm $\|\cdot\|_2$, and the Borel $\sigma$-algebra generated by the open sets. The induced matrix 2-norm is also denoted $\|\cdot\|_2$. 
 We say that a random variable $\z$
taking values in $\mathbb{R}^n$ is essentially bounded, if for some constant $C > 0$, $\|\z\|_2 < C$
holds with probability one.

A \emph{stochastic linear-time invariant (LTI) systems with inputs in state-space form} \cite[Chapter 17]{LindquistBook} is a dynamical system
of the form
\begin{equation}\label{eq:assumedSys}
	\begin{split}
		\x(t+1)=A\x(t)+B\mathbf{u}(t)+\bnu(t),\\
		\y(t)=C\x(t)+D\mathbf{u}(t)+\etab(t)
	\end{split}
\end{equation}
defined for all $t \in \mathbb{Z}$, where 
$A,B,C,D$ are $n \times n$, $n \times n_u$, $n_y \times n$ and $n_y \times n_u$ matrices respectively, $A$ is a Schur matrix (a square matrix with all its eigenvalues inside the unit disk), $\bnu,\etab$ are zero-mean Gaussian i.i.d processes,
$\mathbf{u}$, $\x$, are zero-mean stationary Gaussian processes, $\mathbf{u}(t)$ and $\begin{bmatrix} \etab^T(t),\bnu^T(t) \end{bmatrix}^T$ are independent, and $\x(t)$ and $\begin{bmatrix} \bnu^T(t),\etab^T(t) \end{bmatrix}^T$ are independent. 
The process $\x$ is called the state process, $\bnu$ is called the  process noise and $\etab$ is the 
measurement noise. 
%Note that $\x$ is uniquely determined by the matrices $A,B,E$ and noise $\v$
%The unique stationary  process $\x$ satisfies $\x(t)=\sum_{k=0}^{\infty} A^kB\mathbf{u}(t-k)+\sum_{k=0}^{\infty} A^k\bnu(t-k)$.
If $B,D$ are absent from \eqref{eq:assumedSys}, then we say that \eqref{eq:assumedSys} is an \emph{autonomous stochastic LTI system} \par
Let us fix stochastic processes $\y(t)\in \reals^{n_\y}$, and $\rvu(t)\in  \reals^{n_\rvu}$, that share a time axis $t\in\sZ$, that is, for any $t\in\mathbb{Z}$, $\y(t):\Omega\to\sR^{n_\y};\omega\mapsto\y(t)(\omega)$, and $\mathbf{u}(t):\Omega\to\sR^{n_\rvu};\omega\mapsto\rvu(t)(\omega)$
%; hence  $y(t):\omega\to\y(t)(\omega)$, $w(t):\omega\to\w(t)(\omega)$ 
are random vectors on $(\Omega,\F,\bP)$. The goal is to estimate $\y(t)$ from current and past values of $\rvu(t)$, for this we need a structure connecting $\rvy(t)$ and $\rvu(t)$, thus we have 
\begin{Assumption}\label{as:generator}
	Let $\y(t)$ and $\mathbf{u}(t)$ be generated by an autonomous stochastic LTI system
	\begin{subequations}\label{eq:generator}
		\begin{align}
			\x(t+1) &=A_g\x(t)+K_g\e_g(t), \\
			\begin{bmatrix}\y(t)\\\mathbf{u}(t)\end{bmatrix}&=C_g\x(t)+\e_g(t)
		\end{align}
	\end{subequations}
	where $A_g \in \mathbb{R}^{n \times n},K_g \in \mathbb{R}^{n \times m},C_g \in \mathbb{R}^{m \times n}$ for $n > 0$, $m=n_y+n_u\geq2$ and $\x$, $\y$ and $\e_g$ are stationary, zero-mean, and jointly Gaussian stochastic processes.
	%The processes $\x$ and $\e_g$ are called state and noise process, respectively. 
	%Recall that stationarity and square-integrability imply constant expectation and the covariance matrix $Cov(\y(t),\y(s))\triangleq\bE[(\y(t)-\bE[\y(t)])(\y(s)-\bE[\y(s)])^T]$ only depends on time lag $(t-s)$. 
	% Processes $\y$ and $\w$ are markovian, therefore there exists a splitting subspace $\mathcal{X}_t$, with $x(t)\in\mathcal{X}_t$, such that, $(H_{t-1}^-(y)\vee H_t^-(w))\perp (H_{t-1}^+(y)\vee H_t^+(w)) | \mathcal{X}_t$ or in more standard notation $E[\{z(s)\}_{s=t+1}^\infty|\{z(s)\}_{s=-\infty}^t,x(t)]=E[\{z(s)\}_{s=t+1}^\infty|x(t)]$.
	Furthermore, we require that $A_g$ and $A_g-K_gC_g$ are Schur (all its eigenvalues are inside the open unit circle), that %for any $t,k \in \mathbb{Z}$, $k \geq 0$, $E[\e_g(t)\e_g^T(t\!-\! k\!-\! 1)]=0$, $E[\e_g(t)\x^T(t-k)]=0$, i.e., the stationary Gaussian process 
	$\e_g(t)$ is white noise uncorrelated with $\x(t-k)$, with covariance $\bE[\e_g(t)\e_g^T(t)]=Q_e$, and that $\e_g$ is the innovation process (see \cite{LindquistBook} for definition) of $\begin{bmatrix} \y^T & \mathbf{u}^T \end{bmatrix}^T$. 
	%see \cite{LindquistBook} for the definition of the innovation process.
	We identify the system \eqref{eq:generator} with the tuple
	$\Sigma_{gen}\triangleq(A_g,K_g,C_g,I)$;% note that the state process $\x$ is uniquely defined by
	%the infinite sum $\x(t)=\sum_{k=1}^{\infty} A_g^{k-1}K_g\e_g(t-k)$
\end{Assumption}
% Note that if there is no feedback from $\y$ to $\mathbf{u}$ (see \cite[Definition 17.1.1]{LindquistBook}), then, by \cite{LindquistBook,eringis2021optimal}, Assumption \ref{as:generator} is equivalent to the existence of a stochastic LTI  \eqref{eq:assumedSys} with output $\y$ and input $\mathbf{u}$.\par
\textbf{Note:} For learning, we assume to have the training data set $\train_N = \{\{\y(s),\rvu(s)\}\}_{s=0}^{N-1}$, i.e. a single trajectory of $[\y^T(t),\rvu^T(t)]^T$, but no knowledge of the matrices $A_g,K_g,C_g$ and noise process $\e_g$. The system \eqref{eq:generator} only defines the assumptions on the data generating process. \par
The goal is to use the past and present of $\rvu(t)$, or past of $\y(t)$, to estimate $\y(t)$. 
Note that $\y$ and $\rvu$ are stationary processes by
\cite[Theorem 1.4]{CainesBook}. Moreover,
from classical theory of LTI systems
it follows that $\y(t)$ and $\rvu(t)$, $t \in \mathbb{Z}$
are essentially bounded if the noise 
$\e_g(s)$ is essentially bounded for all $s \in \mathbb{Z}$

That is we wish to consider LTI predictors, 
	\begin{subequations}\label{eq:predictor}
		\begin{align}
			\hat{\x}(t+1)&=\hat{A}\hat{\x}(t)+\hat{B}\rvu(t)+\hat{L}\y(t), ~ \hat{\x}(0)=0 \\
			\hat{\y}(t)&=\hat{C}\hat{\x}(t)+\hat{D}\rvu(t)
			%\hat{\x}(0)&=E[\bar{x}(0)]=0
		\end{align}
	\end{subequations}
where matrices $\hat{A}, \hat{B}, \hat{L}, \hat{C}, \hat{D}$ are of appropriate size, and $\hat{A}$ is Schur (all its eigenvalues are inside the unit disk).

\textbf{Note:} In this paper, we will allow a more general form of predictors, where $\hat{L}$ can be set to 0, i.e. we may wish to estimate $\y(t)$ only from measurements $\rvu(t)$, when past values of the process $\y(t)$ is not available. In order to accommodate this let us define a stochastic process $\w(t)\in \sR^{n_\w}$, by two cases
\begin{itemize}
    \item $\w(t)=\begin{bmatrix} \y^T(t) & \rvu^T(t) \end{bmatrix}^T$, $n_\w=n_\y+n_\rvu$
    \item $\w(t)=\rvu(t)$, $n_\w=n_\rvu$
\end{itemize}
Note that, one can define $\w(t)$, to consist of some of the components of $\y(t)$, i.e. $\w(t)$ does not need to contain all of $\y$. 

% , consisting of $n_\w\leq n_\y+n_\rvu$ components of $\begin{bmatrix} \y^T(t) & \rvu^T(t) \end{bmatrix}^T$, that are used to estimate $\y(t)$. 


% whose components are the components of $\begin{bmatrix} \y^T(t) & \rvu^T(t) \end{bmatrix}^T$ used to estimate $\y(t)$. More explicitly, we will consider two cases

 
\textbf{Class of predictors (hypotheses)} In this paper, we will be interested in the following hypothesis class, consisting of predictors realizable by LTI systems.
\begin{Assumption}[Parameterised hypothesis class]\label{as:parameterisation}
	The hypothesis class $\mathcal{F}$ is a parametrized set of LTI predictors, with $\Sigma(\theta)=(\hat{A}(\theta),\hat{B}(\theta),\hat{C}(\theta),\hat{D}(\theta))$:
        \begin{subequations} \label{eq:formal_predictor}
            \begin{align}
                \hat{\rvx}(t+1)&=\hat{A}\hat{\rvx}(t)+\hat{B}\rvw(t), \;\hat{\rvx}(0)=0,\\
                f_{\Sigma(\theta)}(\{\rvw(s)\}_{s=0}^{t})&=\hat{C}\hat{\rvx}(t)+\hat{D}\rvw(t).
            \end{align}
        \end{subequations}
	$$\mathcal{F}=\{f_{\Sigma(\theta)} \mid \gamma(\hat{A}(\theta))<1,\;\theta\in\Theta\}$$
	with $\gamma(\hat{A}(\theta))$ the spectral radius of $\hat{A}(\theta)$, i.e. the largest modulus of eigenvalues of $\hat{A}(\theta)$. Set $\Theta\subset \sR^{n_\theta}$ is a compact set, and  
	%for any $\theta\in\Theta$, 
	%the tuple $$\mathbf{\Sigma}(\theta)=(\hat{A}(\theta),\hat{B}(\theta),\hat{C}(\theta),\hat{D}(\theta))$$
	%is such that 
	$\hat{A}(\theta)$,$\hat{B}(\theta),\hat{C}(\theta),\hat{D}(\theta)$ are continuous functions of 
	$\theta$ taking values in the sets of 
	$\hat{n}\times\hat{n}$, $\hat{n}\times n_w$,
	$n_y\times \hat{n}$ and $n_y\times n_w$
	matrices respectively.
        If $\w(t)=[\y^T(t),\rvu^T(t)]^T$, then  $\hat{D}=[0,\hat{D}_\rvu]$ for some $n_y \times n_u$ matrix $\hat{D}_\rvu$, i.e., $\hat{D}\w(t)$  depends only on $\mathbf{u}(t)$\footnote{The latter assumption is necessary, since otherwise we would be using the components of $\y(t)$ to predict $\y(t)$, which is not meaningful.}.
\end{Assumption}
We will identify the system \eqref{eq:formal_predictor}
with the tuple 
$(\hat{A},\hat{B},\hat{C},\hat{D})$.
%\emph{a LTI predictor realization}.
%\end{Definition}
% We will often denote the predictor 
% realizable by the LTI system $(\hat{A},\hat{B},\hat{C},\hat{D})$ by
% $f_{(\hat{A},\hat{B},\hat{C},\hat{D})}$.
For the sake of notation, throughout the paper we will use $f$, to denote $f_{\Sigma(\theta)}$, for some arbitrary $\theta\in\Theta$. 


Under assumption \ref{as:parameterisation}, we can use probability densities on the set of predictors $\mathcal{F}$. The latter will be essential for using the PAC-Bayesian framework. 
% Loosely speaking, the  learning problem for stochastic LTI systems is as follows:
% based on a sample of finite length $\{y(t),u(t)\}_{t=1}^{N}$  of $\{\y,\mathbf{u}\}$, estimate the matrices $A,B,C,D$
% of \eqref{eq:assumedSys}. In addition, often the variance of the noises $\bnu$,$\etab$ is also estimated. 

Next, we define the notions of empirical and generalization loss for 
predictors which are realized by LTI systems.
\begin{Assumption}[Quadratic loss function]\label{as:lossfunc} \;\\
	We will consider \emph{quadratic loss functions}
	$\ell : \reals^{n_y}\times \reals^{n_y} \ni (y,y') \mapsto \|y-y'\|_2^2=(y-y')^T(y-y') \in [0,\infty)$.
\end{Assumption}
The empirical loss  of a  predictor for the data $\train_N=\{\y(t),\w(t)\}_{t=0}^{N}$ is defined as follows:
we define the random variable
\[\hyf(t\mid s) \triangleq f(\w(s),\ldots,\w(t))\] 
%for the random variable 
which represents the estimate of $\y(t)$ based on random variables $\{\w(s),\ldots,\w(t)\}$ .
%The quantity $\bE[\ell(\hyf(t\mid s),\y(t))]$, measures the expected difference between the actual process $\y(t)$ and the predicted value $\hyf(t \mid s)$ based on $\{\w(\tau)\}_{\tau=s}^{t}$.
%Equipped with a loss function, we define empirical loss.
%in order to evaluate the predictor on some training set $S_N$.
%\begin{Definition}
	The \emph{empirical loss for a predictor $f$ }and processes $(\y,\w)$ is defined by
	\begin{equation}
		\hat{\mathcal{L}}_{N}(f)\triangleq\frac{1}{N}\sum_{i=0}^{N-1} \ell(\hyf(i \mid 0), \y(i)).\label{eq:EmpLoss}
	\end{equation}
%\end{Definition}
The definition of the generalization loss is a 
bit more involved. Namely, we are using varying number of inputs for predictions and 
hence 
%the size of the predictors, i.e., the number of inputs 
%used for prediction  will impact its quality:
the expectation $\bE[\ell(\hyf(t \mid 0),\y(t))]$ depends on $t$.
This will hold true even if the processes $\y$ and $\w$ are stationary.
Note that this issue is specific for state-space models: autoregressive models 
always use the same number of inputs to make a prediction, see Remark \ref{remark:ARMAdifference}.
%\\
In this paper we will opt for looking at the case when the size of the past used for the prediction is infinite.
%the generalization loss will be the limit of the expected loss as the number of data points used for prediction tends to infinity. 
%This is consistent with the usual practice in system identification
%\cite{LjungBook}. From a practical standpoint it is reasonable, because the longer the predictor
%runs, the more of the past dat
%a it has integrated.  
%We will define this quantity as generalised loss.
To this end, we need the following result from \cite{HannanBook}.
\begin{lemma}[\cite{HannanBook}]
%[Infinite past prediction]
\label{l:ihp}
    \; \\
    The limit 
	%begin{equation}
	%\label{model:inf:eq-1}
	\( \hyf(t)=\lim_{s \rightarrow -\infty} \hyf(t \mid s) \)
	%\end{equation}
	exists
	in the mean-square sense for all $t$, the process $\hyf(t)$ is stationary, and
	%\begin{equation}
	%	\label{model:inf:eq}
	\(	\bE[\ell(\hyf(t),\y(t))]=\lim_{s \rightarrow -\infty} \bE[\ell(\hyf(t \mid s),\y(t))] \).
	%end{equation} 
	%and $\bE[\ell(\hyf(t),\y(t))]$ does not depend on $t$. 
\end{lemma}
This motivates us to introduce
%\begin{Definition}[Generalization loss of a predictor]\;\\
the quantity
	\[ \mathcal{L}(f)=\bE[\ell(\hyf(t),\y(t))]=\lim_{s \rightarrow -\infty} \bE[\ell(\hyf(t \mid s),\y(t))]
	\]
	which is called the \emph{generalization loss} of the predictor $f$ when applied to
	process $(\y,\w)$.% or simply generalization loss, when $\y$ is clear from the context. 
%\end{Definition}
\par
Intuitively, $\hyf(t)$ can be interpreted as the prediction of
$\y(t)$ generated by the predictor $f$ based on all (infinite) past and present values of $\w$. As stated in Lemma~\ref{l:ihp} we consider the special case when $\hyf(t)$ is the mean-square limit of $\hyf(t \mid s)$ as
$s \rightarrow -\infty$. Clearly, for large enough $t-s$, the empirical loss, is close to the generalization loss. 
%More precisely, from the proof of Lemma II.2 of \cite{CDC21paper}
%it follows that $\lim_{N \rightarrow \infty} \bE[\hat{\mathcal{L}}_N(f)]=\mathcal{L}(f)$.
%Note that the generalised loss is independent of the training data, and as such is suitable to quantify the "quality" of a predictor.
In fact, it is standard practice in learning dynamical systems \cite{LjungBook} to use $\mathcal{L}(f)$ as the measure of fitness
of the predictor.
%In order to define the learning problem, we will fix the class of predictors (hypotheses).
%\textbf{Learning problem}
With these definitions in mind, the learning problem considered in this paper can be stated as follows.
\begin{Problem}[Learning problem]
\label{learn:prob}
Compute a predictor $f \in \mathcal{F}$ from a sample 
$\train_N=\{\y(t)(\omega),\rvw(t)(\omega)\}_{t=0}^{N}$ 
of the random variables $\{\y(t),\w(t)\}_{t=0}^{N}$ 
such that the generalization loss $\mathcal{L}(f)$ is small.
\end{Problem}


\begin{Remark} \label{remark:ARMAdifference}
%Note that autoregressive models use fixed number of past values of predictor %variables to predict the output, while for state-space representations this number increases with time. Indeed, 
%$\hyf(t|s)=\sum_{j=s}^{t-1} \hat{C}\hat{A}^{t-j-1}\hat{B}\w(j)+\hat{D}\w(t)$,
%i.e., the number of past values of $\w$ is $t-s$. 
It is known
\cite[Section 4.2]{LjungBook} that the LTI system 
\eqref{eq:predictor} can be
rewritten as an ARX model: %\text{LTI of order n predictors:}\quad
\begin{equation}
\label{lti:arx}
\hat{\y}_f(t|s)=\sum_{i=1}^{n}\hat{\gamma}_i \hat{\y}_f(t-i|s) + \sum_{i=0}^{n-1}\hat{\eta}_i \w(t-i)
\end{equation}
At a first glance this is similar to classical ARX predictors, where
\(\hat{\y}(t)=\sum_{k=1}^n\hat{\alpha}_k \y(t-k) + \sum_{i=0}^{n-1}\hat{\beta}_i \w(t-i) \)
where $\y$ is predicted based on the last $n$ values of $\y$ and $\w$.
However, in contrast to classical ARX models, in \eqref{lti:arx} we do not use the past values of $\y$, but the past values
of the prediction $\hat{\y}_f$.
This difference has significant consequences, in particular, it means that
the previous results \cite{shalaeva2019improved} do not apply.  Note that \cite{alquier2012pred,alquier2013prediction} studied autoregressive models
without inputs (nonlinear AR models), 
so those results are not applicable either.
%In the light of the relationship between LTI predictors and representations
%of $\y$ by stochastic LTI systems with inputs, this is hardly surprising. 
In fact, the problem of learning LTI systems with inputs, or, which is almost equivalent, learning LTI predictors,  is essentially equivalent to learning ARMA models, and the latter is much more involved than learning
ARX models.
%hile, state-space predictors more closely follow Output Error model structure \cite{LjungBook}, where, one now computes $\hat{\y}$, instead of measuring $\y$ %p. 85 (p. 108 in pdf)
%\
%ARMA predictors follow ARMA representation of the process $\y$, with estimates of the process parameters $\alpha_k,\beta_k$, one obtains the prediction of the process
\end{Remark}












\section{PAC-Bayesian Framework}
\label{sect:pac:gen}
Below we present the adaptation of the PAC-Bayesian framework for LTI systems.
%is closely related to the PAC-Bayesian framework for autonomous LTI system described in %\cite{CDC21paper}. 
To this end, let $B_{\Theta}$ be the $\sigma$-algebra of Lebesque-measurable subsets of the parameter set $\Theta\subseteq\mathbb{R}^{n_\theta}$, and $m$ denote the Lebesque measure on $\mathbb{R}^{n_\theta}$. We then define
\begin{equation}
	\underset{f \sim \rho}{E} g(f)\triangleq\int_{\theta \in \Theta} \rho(\theta)g(f_{\Sigma(\theta)})dm(\theta)\label{eq:Edef}
\end{equation}
with $\rho$ a probability density function on the measure space $(\Theta,B_{\theta},m)$, and $g:\mathcal{F} \rightarrow \mathbb{R}$ a map such that
$\Theta \ni \theta \mapsto g(f_{\theta})$ is measurable and absolutely integrable. 
The essence of the PAC-Bayesian approach is to prove 
that for any density $\pi$ on $\mathcal{F}$, and
any $\delta\in(0,1]$, 
% \begin{align} \label{T:pac:gen}
% 		\bP \Bigg( \Bigg\{ \omega \in& \Omega  \mid  
% 		E_{f\sim \hat{\rho}} \mathcal{L} (f) 
% 		 \nonumber \le E_{f\sim \hat{\rho}} \hat{\mathcal{L}}_{N}(f)(\omega) \\
% 		 &
% 		  + r_N 
% 		  %\dfrac{1}{\lambda}\!\left[ KL(\hat{\rho} \|\pi) +
% 		%\ln\dfrac{1}{\delta}
% 		%+ \Psi_{\ell,\pi}(\lambda,N)  \right]  
% 		\Bigg \}\Bigg) > 1-\delta \
% 	\end{align}
\begin{align} 
\label{T:pac:gen}
	\bP \Big( \Big\{ \omega \in \Omega  \mid  \forall \hat{\rho}\in \mathcal{M}_\pi : 
	\underset{f\sim \hat{\rho}}{E} \mathcal{L} (f) \le \kappa(\omega)&
	\Big \}\Big) > 1-\delta,
\end{align}
with
$$
\kappa(\omega)=\underset{f\sim \hat{\rho}}{E} \hat{\mathcal{L}}_{N}(f)(\omega) + r_N
$$
$\mathcal{M}_\pi$ the set of all absolutely continuous densities w.r.t $\pi$, and $r_N=r_N(\pi,\hat{\rho},\delta)$ an error term. That is, the PAC-Bayesian bound holds for every posterior $\hat{\rho}$ in $\mathcal{M}_\pi$, simultaneously.

We may think of $\pi$ as a prior distribution density function and $\hat{\rho}$ as any candidate to a posterior distribution on the space of predictors. 
The inequality \eqref{T:pac:gen} says that the average generalization loss for models sampled from the posterior distribution is smaller than the average empirical loss for the posterior distribution plus the error terms $r_N$. \par
%For a discussion on how to use PAC-Bayesian for learning, see \cite{alquier2021userfriendly}. In a nutshell, 
A learning algorithm
can be thought of as fixing a prior $\pi$ and then
choosing a posterior $\hat{\rho}$ for which
$\kappa(\omega)$ is small.
%i.e., the sum
%$E_{f\sim \hat{\rho}} \hat{\mathcal{L}}_{N}(f)(\omega)
%		 + r_N(\pi,\hat{\rho},\delta)$ is small.
Moreover, $\kappa(\omega)$ can be viewed as a cost function involving the
empirical loss and the regularization term $r_N$.
The learned model
is either sampled from the posterior density $\hat{\rho}$, or it is chosen
as the one with maximal likelihood w.r.t. $\hat{\rho}$. 
Inequality \eqref{T:pac:gen} then gives guarantees on the 
generalization loss of the learned model.
For more details on using PAC-Bayesian bounds see \cite{alquier2021userfriendly}
%As it was pointed out in %\cite{alquier2021userfriendly}, 
%\eqref{T:pac:gen} could be used to derive
%so called \emph{oracle inequalities}. An oracle inequality relates
%generalization error of the posterior with 
%the best possible generalization error achievable for the given hypothesis class.
For \eqref{T:pac:gen} to be useful, 
the term  $r_N$ 
%to be
%decreasing in $\delta$ and in $N$. Ideally, wewould like $r_N$ 
should converge to a small constant, preferably zero,  as $N \rightarrow \infty$, and to be decreasing in $\delta$. The most common way of expressing the error term $r_N$, is based on Donsker-Varadhan’s change of measure \cite[Theorem 3]{nips-16}:
%\cite{alquier2021userfriendly}, the error term $r_N$ is of the form
    \begin{align} 
	r_N=
		 \dfrac{1}{\lambda}\!\left[ \KL(\hat{\rho} \|\pi) +
	\ln\dfrac{1}{\delta}+ \Psi_{\pi}(\lambda,N)  \right] \label{T:pac},
	\end{align}
where $\lambda > 0$ and
	$\KL(\hat{\rho} \mid \pi)\triangleq E_{f\sim\hat{\rho}} \ln \frac{\hat{\rho}(f)}{\pi(f)}$ is the KL-divergence between
	$\pi$ and $\hat{\rho}$, and 
	\begin{equation}
	\label{T:pac:2}
	   \Psi_{\pi}(\lambda,N) \triangleq \ln E_{f\sim\pi} \bE[e^{\lambda(\mathcal{L}(f)-\hat{\mathcal{L}}_{N}(f))}]
	\end{equation}
That is, $r_N$ involves the KL-divergence and a free parameter $\lambda$.
The density which minimizes $\kappa(\omega)$, with $r_N$ from \eqref{T:pac} is known as the Gibbs-posterior
\cite{alquier2021userfriendly} and it can be explicitly computed, i.e. \begin{align}
\rho_{\text{Gibbs}}(f)&\triangleq Z^{-1}\pi(f)\exp(-\lambda \hat{\mathcal{L}}_N(f)),\label{eq:gibbs}\\ 
Z&\triangleq E_{f\sim\pi}\exp(-\lambda \hat{\mathcal{L}}_N(f)). \nonumber
\end{align}
%Moreover, $\lambda$ i. \\
The disadvantage of this approach is that it is difficult to bound
$\Psi_{\pi}(\lambda,N)$, since it involves bounding higher-order moments
\begin{align}
    \bE[|\mathcal{L}(f)-\hat{\mathcal{L}}_N(f)|^r],\quad  r\in\sN
\end{align}


% \textcolor{blue}{
% \begin{Problem}[Bayesian learning problem]
%     find the probability density function $\hat{\rho}(f_{\Sigma(\theta)}|\train_N):\mathcal{F}\to \sR_+$ from a sample $\train_N=\{y(t),w(t)\}_{t=0}^{N}$, s.t. $E_{f\sim\hat{\rho}}\mathcal{L}(f)$ is minimised, then either
%     \begin{itemize}
%         \item sample $\hat{f}$ according to $\hat{\rho}$, (more likely to sample predictors which yield smaller generalization loss)
%         \item take the maximum likelihood predictor $\hat{f}=\argmax_f \hat{\rho}(f)$
%     \end{itemize}
% \end{Problem}
% }
One can also use PAC-Bayesian bounds, in order to choose the prior $\pi$ or the hypothesis class $\mathcal{F}$, s.t. the difference between generalised loss and empirical loss is within some acceptable level, i.e.  
\begin{align}
    E_{f\sim\rho}\left ( \mathcal{L}(f)-\hat{\mathcal{L}}_N(f) \right ) \leq r_N(\lambda,\pi) \leq \epsilon
\end{align}
then it is only a matter of choosing $\pi,\lambda,\mathcal{F}$, s.t. $r_N(\lambda,\pi)\leq \epsilon$, after which one can proceed with more standard Bayesian learning approach on just the empirical loss $\hat{\mathcal{L}}_N(f)$.

In the next section, we will apply a simple trick, which will allow us to upper-bound higher-order moments. 
\section{Main Results} \label{sec:mainResults}
In this paper we derive PAC-Bayesian bounds  \eqref{T:pac:gen}
for LTI systems. The main idea is to use the change of measure inequality from \cite[Theorem 3]{nips-16}. The major
challenge is to bound the corresponding moment generating function/higher-order moments of $(\mathcal{L}(f)-\hat{\mathcal{L}}_N(f))$. However this brings some technical challenges. Namely, the processes involved are not i.i.d.. Moreover, they are not bounded, and the quadratic loss function is not Lipschitz.
In addition, the empirical loss $\hat{\mathcal{L}}_N(f)$ is not
an unbiased estimate of the generalization loss $\mathcal{L}(f)$. This 
is specific to state-space representations, for auto-regressive models 
considered in \cite{alquier2012pred,alquier2013prediction,alquier:hal-01385064} this problem does not occur.  All these issues make it impossible to directly apply existing techniques \cite{alquier2012pred,alquier2013prediction,alquier:hal-01385064}. \\
As the first step, temporarily we replace the empirical loss $\hat{\mathcal{L}}_{N}(f)$ by
\begin{equation}
	\label{inf:emp:pred}
	V_N(f)\triangleq\frac{1}{N}\sum_{i=0}^{N-1}(\y(i)-\hyf(i))^2
\end{equation}
where the finite-horizon prediction $\hyf(t\mid 0)$ is replaced by the 
infinite horizon prediction $\hyf(t)$ defined in Lemma \ref{l:ihp}.
%Intuitively, $V_N(f)$ represents the empirical prediction error
%if the whole past of $\w$ was used for prediction, as opposed
%to a finite portion of the past of $\w$. \\
The advantage of $V_N(f)$ over $\hat{\mathcal{L}}_{N}(f)$
is that $V_N(f)$ is an unbiased estimate of the generalization loss
$\mathcal{L}(f)$, i.e.,
$\bE[V_N(f)]=\mathcal{L}(f)$.
Indeed, since $\y(t)-\hat{\y}_f(t)$ is a stationary process, $E[\|\y(i)-\hat{\y}_f(i)\|^2_2]=\mathcal{L}(f)$ 
does not depend on $i$, and hence 
$\bE[V_N(f)]=\frac{1}{N} \sum_{i=0}^{N-1} E[\|\y(i)-\hat{\y}_f(i)\|^2_2]= \mathcal{L}(f)$.
hence, usual techniques for deriving error bounds are easier to
extend to $V_N(f)$ than to $\hat{\mathcal{L}}_{N}(f)$.
Moreover, , from Lemma B.7 in Appendix B of the supplementary
material, it follows that $\hat{\mathcal{L}}_{N}(f)-V_N(f)$ converges to zero as $N \rightarrow \infty$
%
%from the proof of (Lemma II.2) in \cite[Lemma II.2]{CDC21paper} it follows that
%$\lim_{N \rightarrow \infty} \hat{\mathcal{L}}_{N}(f)-V_N(f)=0$
%where the limit is understood 
in the mean sense. 
% This suggests that for large enough $N$, the empirical prediction loss $\hat{\mathcal{L}}_{N}(f)$ could be replaced by the infinite past prediction error $V_N(f)$\\
%, and we could use the latter in the PAC-Bayesian error bound. \\
In order to derive upper bounds on the errors of the type \eqref{T:pac}, we will first derive upper bounds of the type \eqref{T:pac}, for $\mathcal{L}(f)-V_N(f)$, secondly we will derive upper bounds for $V_N(f)-\hat{\mathcal{L}}_N(f)$, then we will combine them using union bound. 
%we decompose the moments $(\mathcal{L}(f)-\hat{\mathcal{L}}_N(f))^r=(\mathcal{L}(f)-V_N(f) + V_N(f)-\hat{\mathcal{L}}_N(f))^r$, which then by convexity of $\phi(x)=x^r$, is bounded by
% \begin{multline}
%     (\mathcal{L}(f)-\hat{\mathcal{L}}_N(f))^r\leq 2^{r-1} (\mathcal{L}(f)-V_N(f))^r\\
%     +2^{r-1}(V_N(f)-\hat{\mathcal{L}}_N(f))^r
% \end{multline}
Doing this might seem counter-productive, however it is significantly easier to bound moments, $\bE[(\mathcal{L}(f)-V_N(f))^r]$, and $\bE[(V_N(f)-\hat{\mathcal{L}}_N(f))^r]$

% we apply change of measures on $\lambda|\mathcal{L} (f) -  V_N(f)|$ instead of $\lambda|\mathcal{L} (f) -  \hat{\mathcal{L}}_{N}(f))|$. Then we apply another change of measure on $\lambda|V_N(f)-\hat{\mathcal{L}}_N(f)|$. By combining these two bounds we will finally derive a PAC-Bayesian error bound for $\hat{\mathcal{L}}_{N}(f)$.
For every predictor $f$ 
 we define the following constants. %% $G(f)$ and $G_e(f)$. 
	\begin{Definition}[Constants $\bar{G}_f(f),G_e(f)$]
	\label{def:constants}
	 Let $f=(\hat{A},\hat{B},\hat{C},\hat{D})$ be a predictor.
	 % The constants $G(f)$, $G_e(f)$ are define as follows.
     Let
	$A_g,K_g,C_g$ be the matrices of the data generator  from Assumption \ref{as:generator}. Define
	the matrices $(A_e,K_e,C_e,D_e)$ as
	  $ D_e= I-\hat{D}_w$,
	\begin{align*}
	   & A_e=
	   \begin{bmatrix} A_g & 0 \\ \hat{B}C_w & \hat{A}  \end{bmatrix}, 
	   ~
	   K_e=
	    \begin{bmatrix} K_g \\
	    \hat{B}_w
	    \end{bmatrix},
	     ~ C_e=
	         \begin{bmatrix} (C_1-\hat{D}C_w)^T \\ -\hat{C}^T \end{bmatrix}^T,
	    \end{align*}
	where $C_g=\begin{bmatrix} C_1^T & C_2^T \end{bmatrix}^T$ and $C_1$ has $n_y$ rows and $C_2$ has $n_u$ rows; and
	$(C_w,\hat{B}_w,\hat{D}_w)=(C_2,\begin{bmatrix} 0 & \hat{B} \end{bmatrix}, \begin{bmatrix} 0 & \hat{D} \end{bmatrix})$
	if $\w=\mathbf{u}$, and 
	$(C_w,\hat{B}_w,\hat{D}_w)=(C_g,\hat{B},\hat{D})$, if $\w=\begin{bmatrix} \y^T & \mathbf{u}^T \end{bmatrix}^T$.
 Choose for all $f\in\mathcal{F}$, $\hat{M}(f)>1$, and $\hat{\gamma}(f)\in[\hat{\gamma}^*(f),1)$, such that $\|\hat{A}^k\|_2\leq \hat{M}(f)\hat{\gamma}^k(f)$, with $\hat{\gamma}^*(\hat{A})$ the spectral radius of $\hat{A}$.
	With these definitions, 
  \begin{align*}
	    &G_e(f)\hspace{-2pt}=\hspace{-2pt}\|(A_e,K_e,C_e,D_e)\|_{\ell_1}\hspace{-2pt}\triangleq\hspace{-2pt}\|D_e\|_2 \hspace{-2pt}+\hspace{-2pt}\sum_{k=0}^{ \infty} \|C_eA_e^{k}K_e\|_2 \nonumber \\
        &\|\Sigma_{gen}\|_{\ell_1}=1+\sum_{k=0}^\infty \|C_gA_g^{k-1}K_g\|_2\\
        &\bar{G}_{gen}=\|\Sigma_{gen}\|_{\ell_1}^{2} \mu_{\max}(Q_e)\\
        &\bar{G}_f(f)=\left (1+\|\hat{D}\|+ \frac{\hat{M}\|\hat{B}\|\|\hat{C}\|}{1-\hat{\gamma}} \right ) \frac{\hat{M}\|\hat{C}\| \|\hat{B}\|}{(1-\hat{\gamma})^{1.5}}
   \end{align*}
\end{Definition}
The interpretation of the various terms appearing in  Definition \ref{def:constants} is as follows.
\begin{Remark}[Interpretation of constants]\text{}
\\
\textbf{The matrices $A_e,K_e,C_e,D_e$} represent the LTI system driven
  by the innovation process $\e_g$ of $(\y^T,\w^T)^T$, output
  of which is $\y-\hat{\y}_f$, i.e.,
\begin{equation}
\label{error-sys1}
\begin{split}
   \tilde{\x}(t+1)=A_e\tilde{\x}(t)+K_e\e_g(t), \\
   \y(t)-\hat{\y}_f(t)=C_e\tilde{\x}(t)+D_e\e_g(t)
\end{split}  
\end{equation}

\textbf{The term $\bar{G}_{\text{gen}}$} depends only on the data generator system \eqref{eq:generator}, and characterises the scaling of $\y,\rvu$  

\textbf{The term $\bar{G}_f(f)$} depends only the predictor $f$, and should be interpreted similarly to $\|(\hat{A},\hat{B},\hat{C},\hat{D})\|_{\ell_1}^2$.
\end{Remark}


% \begin{Theorem}\label{thm:unbounded} Let $\mathcal{M}_\pi$ denote the set of all absolutely continuous densities w.r.t $\pi$. Then for any density $\pi$ on hypothesis class $\mathcal{F}$, any $\delta\in(0,1]$, and
%     \begin{multline} 0<\lambda < \Big (\sup_{f\in\mathcal{F}} \max\{8 (n_u+n_y) \bar{G}_{gen}  \bar{G}_f(f),  \\ 
%     6(n_u+n_y+1) n_y \mu_{\max}(Q_e)G_e(f)^{2} \}\Big )^{-1} \label{eq:lambdaBound}
%     \end{multline}
%      the following inequality holds with probability at least $1-\delta$
%     \begin{multline}
%          \forall\rho\in\mathcal{M}_\pi:\quad E_{f\sim \hat{\rho}} \mathcal{L} (f) \le \E_{f\sim \hat{\rho}} \hat{\mathcal{L}}_N(f) \\
%          +\dfrac{1}{\lambda}\!\left[\KL(\hat{\rho} \|\pi) + \ln\dfrac{1}{\delta}	+ \widehat{\Psi}_{\pi}(\lambda,N) \right ], \label{eq:NewKLBound}
%     \end{multline}
%       with
%       \begin{multline}
%           \widehat{\Psi}_{\pi}(\lambda,N)\triangleq \ln \E_{f\sim\pi} \left [  1+  \frac{1}{2\sqrt{N}} C_1(f,\lambda) +\frac{1}{2N}C_2(f,\lambda)  \right ]\\
%           \widehat{\Psi}_{\pi}(\lambda,N) \geq \Psi_{\pi}(\lambda,N)= \ln E_{f\sim\pi} \bE[e^{\lambda(\mathcal{L}(f)-\hat{\mathcal{L}}_{N}(f))}]  \\
%       \end{multline}
%       and
%       \begin{align}
%           C_1(f,\lambda)&\triangleq \frac{8 (m!)\lambda \bar{G}_{gen}  \bar{G}_f(f)}{1-8 \lambda m\bar{G}_{gen}  \bar{G}_f(f)}\\
%           C_2(f,\lambda)&\triangleq \frac{2(m+1)! \left (6\lambda n_y\mu_{\max}(Q_e)G_e(f)^{2}\right )^2}{(1-6(m+1)\lambda n_y\mu_{\max}(Q_e)G_e(f)^{2})}
%       \end{align}
%     % \begin{align}
%     %     &\Psi_{\pi,1}(\lambda,N)=	\ln E_{f\sim\pi} \bE[e^{\lambda(\mathcal{L}(f)-V_N(f))}] \nonumber \\
%     %     &\leq  \ln E_{f\sim\pi} \left ( 1+\frac{2}{N}\frac{(m+1)! \left (6\lambda n_y\mu_{\max}(Q_e)G_e(f)^{2}\right )^2}{(1-6(m+1)\lambda n_y\mu_{\max}(Q_e)G_e(f)^{2})} \right )\\
%     %     &\Psi_{\pi,2}(\lambda,N)=	\ln E_{f\sim\pi} \bE[e^{\lambda(V_N(f)-\hat{\mathcal{L}}(f))}] \nonumber \\
%     %     &\leq  \ln E_{f\sim\pi} \left ( 1+ \frac{1}{\sqrt{N}}\frac{m!}{\sqrt{1-\hat{\gamma}^2}} \frac{8 \lambda \bar{G}_{gen}  \bar{G}_f(f)}{1-8 \lambda m\bar{G}_{gen}  \bar{G}_f(f)} \right )
%     % \end{align}
% \end{Theorem}
\begin{Theorem}\label{thm:unbounded} Let $\mathcal{M}_\pi$ denote the set of all absolutely continuous densities w.r.t $\pi$. Then for any density $\pi$ on hypothesis class $\mathcal{F}$, any $\delta\in(0,1]$, and
    \begin{multline} 0<\lambda < \Big (\sup_{f\in\mathcal{F}} \max\{8 (n_u+n_y) \bar{G}_{gen}  \bar{G}_f(f),  \\ 
    6(n_u+n_y+1) n_y \mu_{\max}(Q_e)G_e(f)^{2} \}\Big )^{-1} \label{eq:lambdaBound}
    \end{multline}
     the following inequality holds with probability at least $1-2\delta$
    \begin{multline}
         \forall\rho\in\mathcal{M}_\pi:\quad E_{f\sim \hat{\rho}} \mathcal{L} (f) \le \E_{f\sim \hat{\rho}} \hat{\mathcal{L}}_N(f) + r_N(\lambda,N), \label{eq:NewKLBound}
    \end{multline}
      with
      \begin{align}
          r_N(\lambda,N)&\triangleq \dfrac{1}{\lambda}\!\left[\KL(\hat{\rho} \|\pi) + \ln\dfrac{1}{\delta}	+ \widehat{\Psi}_{\pi}(\lambda,N) \right ]\\
          \widehat{\Psi}_{\pi}(\lambda,N)&\triangleq \frac{1}{2}\left (\widehat{\Psi}_{\pi,1}(\lambda,N)+\widehat{\Psi}_{\pi,2}(\lambda,N)\right ) \\
          \widehat{\Psi}_{\pi}(\lambda,N) &\geq \Psi_{\pi}(\lambda,N)= \ln E_{f\sim\pi} \bE[e^{\lambda(\mathcal{L}(f)-\hat{\mathcal{L}}_{N}(f))}] \nonumber
      \end{align}
      and
      \begin{align}
        \widehat{\Psi}_{\pi,1}(\tilde{\lambda},N) &\triangleq \ln E_{f\sim\pi} \left ( 1+\frac{1}{N}C_1(f,\lambda) \right )\\
        \widehat{\Psi}_{\pi,2}(\tilde{\lambda},N) &\triangleq \ln E_{f\sim\pi } \left ( 1+ \frac{1}{\sqrt{N}} C_2(f,\lambda) \right )\\
        C_1(f,\lambda)&\triangleq \frac{2(m+1)! \left (6\lambda n_y\mu_{\max}(Q_e)G_e(f)^{2}\right )^2}{(1-6(m+1)\lambda n_y\mu_{\max}(Q_e)G_e(f)^{2})} \\ 
        C_2(f,\lambda)&\triangleq \frac{8 (m!)\lambda \bar{G}_{gen}  \bar{G}_f(f)}{1-8 \lambda m\bar{G}_{gen}  \bar{G}_f(f)}
      \end{align}
\end{Theorem}
For proof of Theorem \ref{thm:unbounded}, see Proof \ref{proof:thm:unbounded}, in the Appendix. 

Note that, as $N\to \infty $ the PAC-Bayesian error $r_N\to \frac{1}{\lambda}\left (\KL(\rho|\pi)+\ln \left (\frac{1}{\delta}\right ) \right )$. That is, irrespective of $\rho,\pi$, the error $r_N\geq \frac{1}{\lambda}\ln \left (\frac{1}{\delta}\right )$. Usually, one chooses $\lambda=\lambda(N)$ as an increasing function of $N$, which then allows the PAC-Bayesian error to converge to 0. However, since by Theorem \ref{thm:unbounded}, $\lambda$ is bounded by a constant, we can not control the term $\frac{1}{\lambda}\ln \left (\frac{1}{\delta}\right )$, and $r_N>0$ always. 
\begin{Remark}
    Theorem \ref{thm:unbounded}, holds under assumption \ref{as:generator}, for any distribution of $\e_g(t)$, as long as
    \begin{itemize}
        \item $\e_g(t)\in\reals^m$ is zero-mean, i.i.d.,
        \item $\bE\left [ \|\e_g(t)\|^{2r} \right ]\leq 2^r\mu_{\max}(Q_e)^{r}(m+r-1)!$,
        \item $\sigma(r)\leq 3^r\mu_{\max}(Q_e)^{r}(m+r-1)!$,
    \end{itemize}
    with
    \begin{multline*}
        \sigma(r)=\sup_{t,k,j} \bE\left [\| \e(t,k,j) \|^r_2   \right ],\\
        \e(t,k,j)\triangleq\bE[\e_g(t-k)\e_g^T(t-j)] - \e_g(t-k)\e_g^T(t-j)
    \end{multline*}
    That is, Theorem \ref{thm:unbounded} holds, for $\e_g(t)$, zero mean, i.i.d. with any sub-gaussian distribution. 
\end{Remark}


\section{Bounded case} \label{sec:boundedResults}
If we drop the assumption that $\e_g(t)$ has a Gaussian distribution, and only assume that $\e_g(t)$ is bounded, we get quite straight-forward PAC-Bayesian bounds. 
\begin{Assumption}\label{assumption}
    $\e_g(t)$ is a zero mean i.i.d. stochastic process, with arbitrary distribution, but for all components $\e_{g,i}(t)$ of $\e_g(t)$
 %   \begin{align}
        $|\e_{g,i}(t)|\leq c_e$,
  %  \end{align}
     for some $c_e>0$.
\end{Assumption} 
\begin{Theorem}\label{thm:bounded} Let $\mathcal{M}_\pi$ denote the set of all absolutely continuous densities w.r.t $\pi$. Under assumption \ref{assumption} it holds true that for any density $\pi$ on hypothesis class $\mathcal{F}$, any $\delta\in(0,1]$, and $\lambda>0$ the following inequality holds with probability at least $1-2\delta$
    \begin{multline}
        \forall \rho\in\mathcal{M}_\pi:\quad E_{f\sim\rho} \mathcal{L}(f) \leq E_{f\sim\rho }\hat{\mathcal{L}}_N(f)+\bar{r}_N(\lambda,N) 
    \end{multline}
    with 
    \begin{align}
        \bar{r}_N(\lambda,N) &\triangleq \frac{1}{\lambda}\left [\KL(\rho||\pi)+\ln\frac{1}{\delta} + \widehat{\Psi}_{c_e,\pi}(\lambda,N)\right ]\\
        \widehat{\Psi}_{c_e,\pi}(\lambda,N) &\triangleq \frac{1}{2} \left ( \widehat{\Psi}_{c_e,\pi,1}(\lambda,N)+\widehat{\Psi}_{c_e,\pi,2}(\lambda,N)\right )\\
        \widehat{\Psi}_{c_e,\pi,1}(\lambda,N) &\triangleq \ln E_{f\sim\pi } \left ( 1+\frac{1}{N}e^{ \lambda G_{gen,1} G_e(f)^2} \right )\\
        \widehat{\Psi}_{c_e,\pi,2}(\lambda,N) &\triangleq \ln E_{f\sim\pi } \left (1+\frac{1}{\sqrt{N}}e^{\lambda G_{gen,2} \bar{G}_f(f) } \right )
    \end{align}
    and
    \begin{align}
        G_{gen,1}&\triangleq 8c_e^2n_y(n_y+n_u)\\
        G_{gen,2}& \triangleq 4\|\Sigma_{gen}\|_{\ell_1}^2 c_e^2 (n_y+n_u)
    \end{align}
% with probability at least $1-\delta$
%     \begin{multline}
%          \forall\rho\in\mathcal{M}_\pi :\quad E_{f\sim \hat{\rho}} \mathcal{L} (f) \leq   E_{f\sim \hat{\rho}} \hat{\mathcal{L}}_N(f) \\
%          +\dfrac{1}{\lambda}\!\left[\KL(\hat{\rho} \|\pi) + \ln\dfrac{1}{\delta}	+ \Psi_{\pi,c_e}(\lambda,N) \right ], \label{eq:NewKLBound}
%     \end{multline}
%       with
%     \begin{align}
%         &\Psi_{\pi,c_e}(\lambda,N)=	\ln E_{f\sim\pi} \bE[e^{\lambda(\mathcal{L}(f)-\hat{\mathcal{L}}_N(f))}] \nonumber \\
%         &\leq \ln  E_{f\sim\pi}\left ( 1+ \frac{1}{2N} e^{\lambda G_{gen,1} G_e(f)^{2}} + \frac{1}{2\sqrt{N}} e^{\lambda \bar{G}_f(f) G_{gen,2}} \right )
%     \end{align}
%     and
%     \begin{align}
%         G_{gen,1}&\triangleq 8c_e^2 n_y (n_y+n_u) \\
%         G_{gen,2}&\triangleq 4  c_e^2 (n_y+n_u) \|\Sigma_{gen}\|_{\ell_1}^2
%     \end{align}
\end{Theorem}
For proof of Theorem \ref{thm:bounded}, see Corollary \ref{cor:thm:bounded}, in the Appendix.
Note that, in this case $\lambda$ is not bounded, and as such we can choose $\lambda=\lambda(N)$ an increasing function of $N$, in order to control the term $\frac{1}{\lambda(N)}\ln \delta^{-1}$. More specifically one can choose 
\begin{align}
    \lambda(N)&=\frac{\ln\sqrt{N}}{\sup_{f\in\mathcal{F}} \max \{G_{gen,1} G_e(f)^2, G_{gen,2} \bar{G}_f(f) \}},
\end{align}
for which, it can be shown that $\lambda^{-1}(N)\Psi_{\pi,c_e}(\lambda(N),N)\to 0$, and $\lambda^{-1}(N)\ln\delta^{-1}\to 0$. If one considers $\rho$ independently of $\lambda$, then $\lambda^{-1}(N)\KL(\hat{\rho} \|\pi)\to 0$, however if one considers Gibbs posteriors \eqref{eq:gibbs}, which do depend on $\lambda$, then it is hard to say what will happen with  $\lambda^{-1}(N)\KL(\hat{\rho} \|\pi)$. Simulations seem to indicate that if $\lambda(N)$ is any reasonable increasing function of $N$, then $\lambda(N)$ will converge to some problem dependant constant.

The bound above has all the desired properties,
but its rate of convergence to zero as $N \rightarrow +\infty$ is very slow. In fact, using
\cite{alquier2013prediction}, the results of Theorem \ref{thm:bounded} can be sharpened as follows.
%\section{Alternative Bounded case, based on µ\cite{alquier2013prediction}}
%\textcolor{red}{TO DO: intro to alt case}
\begin{Theorem}\label{thm:bounded_alt} Let $\mathcal{M}_\pi$ denote the set of all absolutely continuous densities w.r.t $\pi$. Under assumption \ref{assumption} it holds true that for any density $\pi$ on hypothesis class $\mathcal{F}$, any $\delta\in(0,1]$, and $\lambda>0$ the following inequality holds with probability at least $1-2\delta$
    \begin{multline}
        \forall \rho\in\mathcal{M}_\pi:\quad E_{f\sim\rho} \mathcal{L}(f) \leq E_{f\sim\rho }\hat{\mathcal{L}}_N(f)+\tilde{r}_N(\lambda,N) 
    \end{multline}
    with 
    \begin{align}
        \tilde{r}_N(\lambda,N) &\triangleq \frac{1}{\lambda}\left [\KL(\rho||\pi)+\ln\frac{1}{\delta} + \tilde{\Psi}_{c_e,\pi}(\lambda,N)\right ]\\
        \tilde{\Psi}_{c_e,\pi}(\lambda,N) &\triangleq \frac{1}{2} \left ( \tilde{\Psi}_{1}(\lambda,N)+\tilde{\Psi}_{2}(\lambda,N)\right )\\
        \tilde{\Psi}_{1}(\lambda,N) &\triangleq \ln E_{f\sim\pi } \left ( 1-C_{1,2}(f)+ C_{1,2}(f) e^{\frac{\lambda}{N}C_{1,1}(f)}  \right )\\
        \tilde{\Psi}_{2}(\lambda,N) &\triangleq \ln E_{f\sim\pi} \left(
        e^{\frac{\lambda^2}{N}C_2(f) }\right) 
    \end{align}
    and, with $C\triangleq c_e\sqrt{n_u+n_y}$,
    \begin{align}
        C_{1,1}(f)&\triangleq 2\|\Sigma_{gen}\|_{\ell_1} C \bar{G}_{f,2}(f)\\
        C_{1,2}(f)&\triangleq \bar{G}_{f,1}(f) \|\Sigma_{gen}\|_{\ell_1} C \\
        C_2(f)&\triangleq 8(G_e(f)+G_{e,1}(f))^2C^2 (4G_e(f)C+1)^2\\
        G_{e,1}&\triangleq \|D_e\|_2 + \sum_{k=0}^\infty (k+1)\|C_eA_e^kK_e\|_2
    \end{align}
\end{Theorem}
For proof of Theorem \ref{thm:bounded_alt}, see Proof \ref{proof:thm:bounded_alt}, in the Appendix. 
If $\lambda_N=\sqrt{N}$ is chosen, then 
the error bound $\bar{r}_N(\lambda_N)$
above converges to zero as $N \rightarrow \infty$ 
at a rate $O(\frac{1}{\sqrt{N}})$. 






\section{Numerical example}\label{sec:numEx}
For the sake of illustration let us assume that data is generated by
\begin{align*}
    \x(t+1)&=\begin{bmatrix} 0.16 & -0.3 \\ 0 & -0.05 \end{bmatrix} \x(t) + \begin{bmatrix} 0.33 & -0.75 \\ 0 & -0.09 \end{bmatrix} \e_g(t)\\
    \begin{bmatrix} \y(t) \\ \rvu(t) \end{bmatrix} &= \begin{bmatrix}
        1 & 1 \\ 0 & 1 \end{bmatrix} \x(t) + \e_g(t),
\end{align*}
Following the two theorems in the paper, we will consider two cases
\begin{itemize}
    \item Unbounded innovation noise: $\e_g(t)\sim\mathcal{N}(0,Q_e)$, 
    \begin{align}
        Q_e=\begin{bmatrix} 0.054 &  0.018\\ 0.018 & 0.248\end{bmatrix}
    \end{align}
    \item Bounded innovation noise: $\e_g(t)$ is distributed according to zero-mean truncated gaussian, s.t. $c_e=1$, and 
    \begin{align}
        \bE[\e_g(t)\e_g^T(t)]\approx Q_e
    \end{align}
    
\end{itemize}
We will assume that the predictors are fully parameterised, i.e. for the case of $\w(t)=\rvu(t)$
\begin{align*}
    \hat{A}(\theta)=\begin{bmatrix} \theta_1&\theta_2\\ \theta_3&\theta_4\end{bmatrix} \;
    \hat{B}(\theta)=\begin{bmatrix} \theta_5\\ \theta_6\end{bmatrix} \\
    \hat{C}(\theta)=\begin{bmatrix} \theta_7&\theta_8\end{bmatrix} \;
    \hat{D}(\theta)=\begin{bmatrix} \theta_9\end{bmatrix}
\end{align*}
for the case of $\w(t)=[\y^T(t), \rvu^T(t)]^T$
\begin{align*}
    \hat{A}(\theta)=\begin{bmatrix} \theta_1&\theta_2\\ \theta_3&\theta_4\end{bmatrix} \;
    \hat{B}(\theta)=\begin{bmatrix} \theta_{10} & \theta_5\\ \theta_{11}& \theta_6\end{bmatrix} \\
    \hat{C}(\theta)=\begin{bmatrix} \theta_7&\theta_8\end{bmatrix} \;
    \hat{D}(\theta)=\begin{bmatrix} 0& \theta_9\end{bmatrix}
\end{align*}
Thus, with $\Sigma(\theta)=(\hat{A}(\theta),\hat{B}(\theta),\hat{C}(\theta),\hat{D}(\theta))$, we will define our hypothesis class to be 
$$\mathcal{F}=\{f_{\Sigma(\theta)}|\gamma(\hat{A}(\theta))<1, \bar{G}_f(f)<10, \theta\in\sR^{11}\}$$
The prior is given by
\begin{align}
    \pi(f)=Z_\pi \exp(-\bar{G}_f(f))
\end{align}
with $Z_\pi$ the normalisation term. This prior will act as regularisation, penalising predictors with high $\ell_1$ norms.
We will use the Gibbs posterior
\begin{align}
    \rho(f|N)=Z_\rho \pi(f) \exp(-\lambda(N) \hat{\mathcal{L}}_N(f))
\end{align}
In order to compute the numerical value of $r_N$, we can use Markov-Chain Monte-Carlo methods, which means that we only need to be able to evaluate
\begin{align}
    \hat{\pi}(f)=\exp(-\bar{G}_f(f)) \propto \pi(f)\\
    \hat{\rho}(f)=\hat{\pi}(f)\exp(-\lambda \hat{\mathcal{L}}_N(f))\propto \rho(f)
\end{align}
More precisely one can approximate $r_N$, by only being able to evaluate $\hat{\pi}(f)$ and 
$
    \beta(f)\triangleq\frac{\hat{\rho}(f)}{\hat{\pi}(f)} \propto \frac{\rho(f)}{\pi(f)}
$

\begin{figure}[ht]
    \centering
    \includegraphics[width=0.99\linewidth]{rN_plot_both.eps}
    \caption{Numerical simulation of both cases (bounded and unbounded noise), solid lines depict case of $\w=\rvu$, dashed lines show case of $\w=[\y^T,\rvu^T]^T$, $\lambda^*$ is found by numerical optimisation, i.e. $\lambda^*=\arg\min_\lambda r_N(\lambda,N)$, the black horizontal line denotes a vacuous bound for the bounded noise case, i.e. any bounds above that line are vacuous}
    \label{fig:rN}
\end{figure}
In Figure \ref{fig:rN} we see the convergence of the error term, for the case of bounded noise. Note that the proposed function $\lambda(N)$ is close to numerically optimal (blue line in Figure \ref{fig:rN}), asymptotically $\lambda(N)\propto \ln\sqrt{N}$, seem to be optimal, one could try to find a less conservative scaling $g(\mathcal{F})< \sup_{f\in\mathcal{F}} \max \{G_{gen,1} G_e(f)^2, G_{gen,2} \bar{G}_f(f) \}$. For the proposed PAC-Bayesian bounds to be useful, the bounds should convergence faster than $\mathcal{O}(\frac{1}{\ln\sqrt{N}})$, since in most applications collecting $N=10^{10}$ data points is not feasible. 
Note that for $N\leq 460$, for this system Theorem \ref{thm:bounded}, yields vacuous bounds, i.e. $\bar{r}_N\geq 2(C\sup_{f\in\mathcal{F}}G_e(f))^2$. However for Theorem \ref{thm:bounded_alt}, only for $N\leq 64$, is the bound vacuous. 


 For the case of unbounded innovation noise, as stated before we see in Figure \ref{fig:rN} that it converges to a constant. Unfortunately, since $\lambda$ is bounded not much can be done. However, since the noise is unbounded it is difficult to determine if the bound is vacuous.

% \begin{figure}[ht]
%     \centering
%     \includegraphics[width=0.99\linewidth]{Psi_plot_both.eps}
%     \caption{Numerical simulation of the bounded and unbounded case, solid lines depict case of $\w=\rvu$, dashed lines show case of $\w=[\y^T,\rvu^T]^T$, $\lambda^*$ is found by numerical optimisation, i.e. $\lambda^*=\arg\min_\lambda r_N(\lambda,N)$}
%     \label{fig:Psi}
% \end{figure}
% Focusing on the problem dependant terms $\widehat{\Psi}$, in Figure \ref{fig:Psi}, we see that they converge to $0$ at the rate of $\mathcal{O}(\frac{1}{\sqrt{N}})$, meaning that the slow convergence of $\mathcal{O}(\frac{1}{\ln\sqrt{N}})$, is induced more from the general framework of the PAC-Bayesian bounds, i.e. $\frac{1}{\lambda}\KL(\rho|\pi)$, and $\frac{1}{\lambda}\ln\frac{1}{\delta}$. 





\section{Conclusion}
\label{sect:concl}
In this paper we have 
derived two PAC-Bayesian error bounds
for stochastic LTI systems with inputs. For data generated by an LTI system with sub-gaussian noise, we see that the difference between empirical and generalised loss is bounded from below, which intuitively should not be the case. Thus, more work needs to be done, to obtain less conservative bounds, or use a difference approach, i.e. one can derive PAC-Bayesian type bounds based on different change of measure inequalities. 

For data generated by an LTI system with bounded innovation noise, we have that the difference between empirical and generalised loss will convergence to 0, slowly at the rate of $\mathcal{O}(\frac{1}{\ln\sqrt{N}})$. That is the problem of minimising the empirical loss, becomes equivalent to minimising the generalised loss, at the aforementioned rate. 

Future research will be directed towards extending these results to more general state-space representations and using the results of the paper for deriving oracle inequalities \cite{alquier2021userfriendly}.

% \bibliography{bib}
% \bibliographystyle{ieee}
\printbibliography

\newpage


\section{Appendix for Proofs}

\paragraph{Proof of Theorem \ref{thm:main}.}

\begin{proof}
\label{proof:main}
Our proof has two steps. In Step 1, we will show that SimCLR is equivalent to minimizing the cross entropy loss defined in Eqn.~(\ref{eqn:cross-entropy}). 
In Step 2, we will show  that minimizing the cross-entropy loss 
is equivalent to spectral clustering on $\bfpi$. 
Combining the two steps together, we have proved our theorem. 

\textbf{Step 1: } SimCLR is equivalent to minimizing the cross entropy loss.

The cross-entropy loss takes expectation over 
$\bfW_\bfX\sim \mathbb{P}(\cdot ; \bfpi)$, 
which means $\bfW_\bfX$ has exactly one non-zero entry in each row $i$. By Lemma~\ref{lem:multinomial}, we know every row $i$ of $\bfW_\bfX$ is independent of other rows. Moreover, 
$\bfW_{\bfX,i}\sim \mathcal{M}(1, \bfpi_i/\sum_j \bfpi_{i,j})=\mathcal{M}(1, \bfpi_i)$, because $\bfpi_i$ itself is a probability distribution.
Similarly, we know $\bfW_\bfZ$ also has the row-independent property by sampling over $\mathbb{P}(\cdot;\bfK_\bfZ)$.
Therefore, by Lemma~\ref{lem:cross_split}, we know Eqn.~(\ref{eqn:cross-entropy}) is equivalent to:
\[
 -\sum_{i=1}^n \mathbb{E}_{\bfW_{\bfX,i}}[\log \mathbb{P}(\bfW_{\bfZ,i}=\bfW_{\bfX,i};\bfK_\bfZ)],
\]

This expression takes expectation over $\bfW_{\bfX,i}$ for the given row $i$. Notice that 
$\bfW_{\bfX,i}$ has exactly one non-zero entry, which equals $1$ (same for $\bfW_{\bfZ,i}$). 
As a result
we expand the above expression to be:
\begin{equation}
 -\sum_{i=1}^n \sum_{j\neq i} \Pr(\bfW_{\bfX,i,j}=1)\log \Pr(\bfW_{\bfZ,i,j}=1).
\label{eqn:detailed-expansion}    
\end{equation}


By Lemma~\ref{lem:multinomial}, $\Pr(\bfW_{\bfZ,i,j}=1)=\bfK_{\bfZ,i,j}/\|\bfK_{\bfZ,i}\|_1$ for $j\neq i$. Recall that $\bfK_\bfZ=(k(\bfZ_i-\bfZ_j))_{(i,j)\in[n]^2}$, which means 
$\bfK_{\bfZ,i,j}/\|\bfK_{\bfZ,i}\|_1=\frac{\exp(-\|\bfZ_i-\bfZ_j\|^2/{2\tau})}{\sum_{k\neq i}
\exp(-\|\bfZ_i-\bfZ_k\|^2/{2\tau})
}$ for $j\neq i$, when $k$ is the Gaussian kernel with variance $\tau$. 

Notice that $\bfZ_i=f(\bfX_i)$, so we know
\begin{equation}
-\log \Pr(\bfW_{\bfZ,i,j}=1)=
-\log \frac{\exp(-\|f(\bfX_i)-f(\bfX_j)\|^2/{2\tau})}{\sum_{k\neq i}
\exp(-\|f(\bfX_i)-f(\bfX_k)\|^2/{2\tau}),
}
\label{eqn:infonce-equivalence}    
\end{equation}


The right hand side is exactly the InfoNCE loss defined in Eqn.~(\ref{eqn:infonce}).
Inserting Eqn.~(\ref{eqn:infonce-equivalence}) into Eqn.~(\ref{eqn:detailed-expansion}), we get the SimCLR algorithm, which first samples augmentation pairs $(i,j)$ with $\Pr(\bfW_{\bfX,i,j}=1)$ for each row $i$, and then optimize the InfoNCE loss. 

\textbf{Step 2: } minimizing the cross entropy loss 
is equivalent to spectral clustering on $\bfpi$.


By Lemma~\ref{lem:convert_to_spectral}, we may further convert the loss to 
\begin{equation}
\label{eqn:main-theorem-repul-attr}
\min_{\bfZ}
-\sum_{(i,j)\in [n]^2} \mathbf{P}_{i,j}
\log k (\bfZ_i-\bfZ_j)+\log \mathbf{R}(\bfZ).
\end{equation}
Since $k$ is the Gaussian kernel, this reduces to \[
\min_\bfZ \mathrm{tr}(\bfZ^\top \mathbf{L}(\bfpi) \bfZ)
+\log \mathbf{R}(\bfZ),
\]

where we use the fact that $\mathbb{E}_{\bfW_\bfX\sim \mathbb{P}(\cdot; \bfpi)}[\mathbf{L}(\bfW_\bfX)]
=\mathbf{L}(\bfpi)
$, because the Laplacian operator is linear and $
\mathbb{E}_{\bfW_\bfX\sim \mathbb{P}(\cdot; \bfpi)}(\bfW_\bfX)=\bfpi
$.
\end{proof}

\paragraph{Proof of Theorem \ref{thm:clip}.}
\begin{proof}
Since $\bfW_\bfX\sim \mathbb{P}(\cdot;\bfpi_{\mathbf{A}, \mathbf{B}})$, we know 
$\bfW_\bfX$ has exactly one non-zero entry in each row, denoting the pair that got sampled. 
A notable difference compared to the previous proof is we now have $n_\mathcal{A}+n_\mathcal{B}$ objects in our graph. CLIP deals with this by taking a mini-batch of size $2N$, 
such that $n_\mathcal{A}=n_\mathcal{B}=N$, and adding the $2N$ InfoNCE losses together. We label the objects in $\mathcal{A}$ as $[n_\mathcal{A}]$, and the objects in $\mathcal{B}$ as $\{n_\mathcal{A}+1, \cdots, n_\mathcal{A}+n_\mathcal{B}\}$. 

Notice that $\bfpi_{\mathbf{A}, \mathbf{B}}$ is a bipartite graph, so the edges of objects in $\mathcal{A}$ will only connect to object in $\mathcal{B}$ and vice versa. We can define the similarity matrix in $\cZ$ as $\bfK_\bfZ$, 
where $\bfK_\bfZ(i, j+n_\mathcal{A})=\bfK_\bfZ(j+n_\mathcal{A},i)= k(\bfZ_i-\bfZ_j)$ for $i\in [n_\mathcal{A}], j\in [n_\mathcal{B}]$, and otherwise we set $\bfK_\bfZ(i,j)=0$. 
The rest is same as the previous proof. 
\end{proof}

\paragraph{Proof of Theorem \ref{thm:exponential}.}

\begin{proof}
\label{proof:exponential}
Since the objective function consists of a linear term combined with an entropy regularization, which is a strongly concave function, the maximization problem is a convex optimization problem. Owing to the implicit constraints provided by the entropy function, the problem is equivalent to having only the equality constraint. We then introduce the Lagrangian multiplier $\lambda$ and obtain the following relaxed problem:

$$
\widetilde{E}(\boldsymbol{\alpha})=\psi_{1}-\sum_{i=1}^n \alpha_{i} \psi_{i}+\tau \sum_{i=1}^n \alpha_{i}\log \alpha_{i}+\lambda\left(\boldsymbol{\alpha}^{\top} \mathbf{1}_n-1\right).
$$

As the relaxed problem is unconstrained, taking the derivative with respect to $\alpha_{i}$ yields

$$
\frac{\partial \widetilde{E}(\boldsymbol{\alpha})}{\partial \alpha_{i}}=-\psi_{i}+\tau\left(\log \alpha_{i}+\alpha_{i} \frac{1}{\alpha_{i}}\right)+\lambda=0.
$$

Solving the above equation implies that $\alpha_{i}$ takes the form
$
\alpha_{i}=\exp \left(\frac{1}{\tau} \psi_{i}\right) \exp \left(\frac{-\lambda}{\tau}-1\right).
$ Since $\alpha_{i}$ lies on the probability simplex, the optimal $\alpha_{i}$ is explicitly given by
$
\alpha^{*}_{i}=\frac{\exp \left(\frac{1}{\tau} \psi_{i}\right)}{\sum_{i^{\prime}=1}^n \exp \left(\frac{1}{\tau} \psi_{i^{\prime}}\right)} .
$ Substituting the optimal point into the objective function, we obtain
$$
\begin{aligned}
E\left(\boldsymbol{\alpha}^*\right)  &=\psi_1-\sum_{i=1}^n \frac{\exp \left(\frac{1}{\tau} \psi_{i}\right)}{\sum_{i^{\prime}=1}^n \exp \left(\frac{1}{\tau} \psi_{i^{\prime}}\right)} \psi_{i}+\tau \sum_{i=1}^n \frac{\exp \left(\frac{1}{\tau} \psi_{i}\right)}{\sum_{i^{\prime}=1}^n \exp \left(\frac{1}{\tau} \psi_{i^{\prime}}\right)}\log \frac{\exp \left(\frac{1}{\tau} \psi_{i}\right)}{\sum_{i^{\prime}=1}^n \exp \left(\frac{1}{\tau} \psi_{i^{\prime}}\right)} \\
& =\psi_1 - \tau \log \left(\sum_{i=1}^n \exp \left(\frac{1}{\tau} \psi_{i}\right)\right).
\end{aligned}
$$
Thus, the Lagrangian dual function is given by
\begin{equation*}
-E\left(\boldsymbol{\alpha}^*\right)= -\tau \log \frac{\exp \left(\frac{1}{\tau} \psi_{1}\right)}{\sum_{i=1}^n \exp \left(\frac{1}{\tau} \psi_{i}\right)}.\qedhere
\end{equation*}
\end{proof}



\section{More on Experiments} \label{section: experiment_details}

\paragraph{CIFAR-10 and CIFAR-100} CIFAR-10 ~\citep{krizhevsky2009learning} and CIFAR-100 ~\citep{krizhevsky2009learning} are well-known classic image classification datasets. Both CIFAR-10 and CIFAR-100 contain a total of 60k $32 \times 32$ labeled images of different classes, with 50k for training and 10k for testing. CIFAR-10 is similar to CIFAR-100, except there are 10 different classes in CIFAR-10 and 100 classes in CIFAR-100.

\paragraph{TinyImageNet} TinyImageNet ~\citep{le2015tiny} is a subset of ImageNet ~\citep{deng2009imagenet}. There are 200 different object classes in TinyImageNet, with 500 training images, 50 validation images, and 50 test images for each class. All the images in TinyImageNet are colored and labeled with a size of $64 \times 64$.

\textbf{Pseudo-code.} Algorithm \ref{alg:Training Procedure} presents the pseudo-code for our empirical training procedure.

\begin{algorithm}[!htbp]
\caption{Training Procedure}
\label{alg:Training Procedure}
\begin{algorithmic}[1]
\REQUIRE trainable encoder network $f$, batch size $N$, augmentation strategy \textit{aug}, loss function $L$ with hyperparameters \textit{args}
\FOR {sampled minibatch ${x_i}_{i=1}^N$}
\FORALL{$i \in { 1, ..., N }$}
\STATE draw two augmentations $t_i = \textit{aug}\left(x_i\right) $, $t_i' = \textit{aug}\left(x_i\right) $
\STATE $z_i = f\left(t_i\right)$, $z_i' = f\left(t_i'\right)$
\ENDFOR
\STATE compute loss $\mathcal{L} = L(N, z, z', \textit{args})$
\STATE update encoder network $f$ to minimize $\mathcal{L}$
\ENDFOR
\STATE \textbf{Return} encoder network $f$
\end{algorithmic}
\end{algorithm}

We also provide the pseudo-code for our core loss function used in the training procedure in Algorithm \ref{alg:Core loss}. The pseudo-code is almost identical to SimCLR's loss function, with the exception of an extra parameter $\gamma$.

\begin{algorithm}[!htbp]
\caption{Core loss function $\mathcal{C}$}
\label{alg:Core loss}
\begin{algorithmic}[1]
\REQUIRE batch size $N$, two encoded minibatches $z_1, z_2$, $\gamma$, temperature $\tau$
\STATE $z = \textit{concat}\left(z_1, z_2\right)$
\FOR {$i \in {1, ..., 2N }, j \in {1, ..., 2N}$ }
\STATE $s_{i,j} = \Vert z_i - z_j \Vert_2^{\gamma}$
\ENDFOR
\STATE \textbf{define} $l(i, j)$ \textbf{as} $l(i, j) = - \log \frac{exp\left(s_{i,j}/\tau \right)}{\sum_{k=1}^{2N} \mathbf{1}{[k \ne i]} exp\left(s{i, j} / \tau \right)} $
\STATE \textbf{Return} $\frac{1}{2N} \sum_{k=1}^N\left[l(i, i+N) + l(i+N, i)\right]$
\end{algorithmic}
\end{algorithm}

Utilizing the core loss function $\mathcal{C}$, we can define all kernel loss functions used in our experiments in Table \ref{table: loss definition}. For all $z_i \in z$ with even dimensions $n$, we define $z_{L_i} = z_i\left[0:n/2\right]$ and $z_{R_i} = z_i\left[n/2:n\right]$.

\begin{table}[ht]
\centering
\begin{tabular}{{@{}l|l@{}}}
Kernel  &  Loss function \\ \midrule
Laplacian & $\mathcal{C}\left(N, z, z', \gamma=1, \tau\right)$\\ \midrule
Sum       & $\lambda * \mathcal{C}\left(N, z, z', \gamma=1, \tau_1\right) + (1-\lambda) * \mathcal{C}\left(N, z, z', \gamma=2, \tau_2\right)$  \\ \midrule
Concatenation Sum&$\lambda * \mathcal{C}\left(N, z_L, z'_L, \gamma=1, \tau_1\right) + (1-\lambda) * \mathcal{C}\left(N, z_R, z'_R, \gamma=2, \tau_2\right)$\\ \midrule
$\gamma = 0.5$ & $\mathcal{C}\left(N, z, z', \gamma=0.5, \tau\right)$          \\ 

\end{tabular}

\caption{Definition of kernel loss functions in our experiments}
\label {table: loss definition}
\end{table}

\textbf{Baselines.} We reproduce the SimCLR algorithm using PyTorch Lightning~\citep{PytorchLightning}.

\textbf{Encoder details.}
The encoder $f$ consists of a backbone network and a projection network. We employ ResNet50~\citep{ResNet} as the backbone and a 2-layer MLP (connected by a batch normalization~\citep{ioffe2015batch} layer and a ReLU \cite{nair2010rectified} layer) with hidden dimensions 2048 and output dimensions 128 (or 256 in the concatenation kernel case).

\textbf{Encoder hyperparameter tuning.}
For each encoder training case, we randomly sample 500 hyperparameter groups (sample details are shown in Table \ref{table: Hyperparameter sample}) and train these samples simultaneously using Ray Tune ~\citep{RayTune}, with the ASHA scheduler~\citep{li2018massively}. Ultimately, the hyperparameter group that maximizes the online validation accuracy (integrated in PyTorch Lightning) within 5000 validation steps is chosen for the given encoder training case.

\begin{table}[ht]
\centering

\begin{tabular}{@{}l|l|l@{}}
\midrule
Hyperparameter  & Sample Range & Sample Strategy \\ \midrule
start learning rate & $\left[10^{-2}, 10\right]$ & log uniform \\ \midrule
$\lambda$       & $\left[0, 1\right]$ & uniform \\ \midrule
$\tau$, $\tau_1$, $\tau_2$ & $\left[0, 1\right]$ & log uniform \\ \midrule
\end{tabular}

\caption{Hyperparameters sample strategy}
\label {table: Hyperparameter sample}
\end{table}

\textbf{Encoder training.} 
We train each encoder using the LARS optimizer~\citep{LARSOptimizer}, LambdaLR Scheduler in PyTorch, momentum 0.9, weight decay $10^{-6}$, batch size 256, and the aforementioned hyperparameters for 400 epochs on a single A-100 GPU.

\textbf{Image transformation.} The image transformation strategy, including augmentation, is identical to the default transformation strategy provided by PyTorch Lightning.

\textbf{Linear evaluation.}
The linear head is trained using the SGD optimizer with a cosine learning rate scheduler, batch size 64, and weight decay $10^{-6}$ for 100 epochs. The learning rate starts at $0.3$ and ends at $0$.

\textbf{Moco Experiments.} We also tested our method based on MoCo~\citep{he2019moco}. The results are summarized in Table \ref{tab:results-moco}. Here we choose ResNet18~\citep{ResNet} as the backbone and set a temperature of $0.1$ as default. For our simple sum kernel, we set $\lambda=0.8$. The results show that our method outperforms the original MoCo method.

\begin{table}[thb]
\centering
\caption{MoCo Experiment Results on CIFAR-10 and CIFAR-100.}
\label{tab:results-moco}
\resizebox{\textwidth}{!}{%
\begin{tabular}{@{}c|ccc|ccc@{}}
\toprule
\multirow{3}{*}{Method} & \multicolumn{3}{c|}{CIFAR-10} & \multicolumn{3}{c}{CIFAR-100} \\ \cmidrule(lr){2-4} \cmidrule(lr){5-7} 
                        & 200 epochs & 400 epochs    & 1000 epochs   & 200 epochs & 400 epochs & 1000 epochs         \\ \midrule
MoCo (repro.)         & $76.41 \pm 0.12$    & $80.01 \pm 0.15$          & $84.45 \pm 0.08$    & $\mathbf{47.02 \pm 0.11}$ & $52.50 \pm 0.07$ & $57.62 \pm 0.15$            \\
\midrule
Laplacian Kernel        & ${78.09 \pm 0.10}$    & $\mathbf{83.85 \pm 0.09}$          & $\mathbf{88.34 \pm 0.16}$    & $46.12 \pm 0.22$   & $53.44 \pm 0.17$ & $59.10 \pm 0.14$        \\
Simple Sum Kernel & $\mathbf{78.12 \pm 0.15}$   & $83.23 \pm 0.18$ & $87.50 \pm 0.20$ & $46.65 \pm 0.06$ & $\mathbf{53.62 \pm 0.19}$ & $\mathbf{59.83 \pm 0.12}$\\
\bottomrule
\end{tabular}
}
\end{table}



\section{More Experiments on Synthetic Data}


Consider a scenario with $n$ clusters, each containing $k$ vertices. Let the probability of vertices $u$ and $v$ from the same cluster belonging to $\bfpi$ be $p$. Conversely, for vertices $u$ and $v$ from different clusters, let the probability of belonging to $\pi$ be $q$. We generate the graph $\bfpi$ randomly, based on $p$ and $q$. We experiment with values of $k=100$ and $n=6$ for ease of visualization, embedding all points in a two-dimensional space. Each vertex's initial position originates from a normal distribution. In each iteration, we sample a subgraph of $\bfpi$ uniformly, ensuring each vertex has an out-degree of $1$. We then optimize the corresponding vectors using InfoNCE loss with an SGD optimizer and iterate until convergence. Our experimental setup consists of an SGD learning rate of $1$, an InfoNCE loss temperature of $0.5$, and a batch size of $50$. We evaluate two scenarios with different $p$ and $q$ values: $p=1$, $q=0$, and $p=0.75$, $q=0.2$. The results of these experiments are visualized in Figure \ref{fig:vis-spectral-cluster}. The obtained embeddings exhibit the hallmark pattern of spectral clustering of graph $\bfpi$.

\begin{figure}[!tb]
\centering
\subfigure{
\includegraphics[width=1\textwidth]{Figures/cluster_pi.png}
\label{fig:vis-cluster}
}
\subfigure{
\includegraphics[width=1\textwidth]{Figures/noised_cluster_pi.png}
\label{fig:vis-noised-cluster}
}
\caption{Visualizations of the optimization process using InfoNCE Loss on the vectors corresponding to $\bfpi$. Points of identical color belong to the same cluster within $\bfpi$. To showcase the internal structure of $\bfpi$, we randomly select 10 vertices from each cluster to display the edge distribution of $\bfpi$.}
\label{fig:vis-spectral-cluster}
\end{figure}





\end{document}
