\chapter{Modelling Black Hole Signals with Gaussian Processes}
\chapterimage[height=130pt]{Chapter4/Figs/black_hole.jpg}


\ifpdf
    \graphicspath{{Chapter3/Figs/Raster/}{Chapter3/Figs/PDF/}{Chapter3/Figs/}}
\else
    \graphicspath{{Chapter3/Figs/Vector/}{Chapter3/Figs/}}
\fi


\textbf{Status:} Published as Griffiths, RR., Jiang, J., Buisson, DJ., Wilkins, D., Gallo, LC., Ingram, A., Grupe, D., Kara, E., Parker, ML., Alston, W., Bourached, A. Cann, G., Young, A., Komossa, S., Modeling the Multiwavelength Variability of Mrk 335 Using Gaussian Processes. \textit{The Astrophysical Journal}, 2021.


\section{Background on High-Energy Astrophysics}

The chapter begins with a self-contained background on high-energy astrophysics to aid in contextualising the findings.

\subsection{Black Holes}


John Michell was the first to posit the existence of  black holes \citep{1784_Michell}, describing them as \say{dark stars} due to the fact that no light could escape from them. At the time, however, his work was largely ignored due to the absence of a theory of gravity describing the behaviour of light in a strong gravitational field. Following the introduction of the Einstein field equations from General Relativity \citep{1916_Einstein}, Schwarzchild was the first to calculate the radius of a black hole in the Schwarzschild metric \citep{1916_Schwarzschild}. The Schwarzchild radius is

\begin{equation}
    r_s = \frac{2GM}{c^2},
\end{equation}

\noindent where $G$ is the gravitational constant, $M$ is the mass of the object and $c$ is the speed of light. Black holes are characterised according to their mass and spin. When the mass of a black hole exceeds $10^5 M_{\odot}$, it is termed a supermassive black hole (SMBH), where $M_{\odot}$ is the solar mass unit, approximately equal to the mass of the Sun.

\subsection{Active Galactic Nuclei}

The term Active Galactic Nucleus (AGN) was coined by Viktor Ambartsumian in the early 1950s \citep{1997_Victor}. Ambartsumian argued that the nuclei of galaxies were subject to explosions which caused large amounts of mass to be expelled, and that for these explosions to occur, galactic nuclei must contain unknown bodies of huge mass. Moreover, AGN were observed to be highly luminous with unusual spectral properties, indicating that their power source could not be ordinary stars. In 1964, some insight on the nature of AGN was offered by Salpeter and Zeldovich  \citep{1964_Salpeter, 1964_Zeldovich}, who proposed accretion of gas onto a SMBH as the mechanism responsible for the power source of a powerful class of AGN known as quasars. \citet{1969_Lynden} later paid testament to the importance of the black hole accretion disc model by remarking that, 

\begin{displayquote}
\say{With different values of the black hole mass and accretion rate these discs are capable of providing an explanation for a large fraction of the incredible phenomena of high-energy astrophysics.}
\end{displayquote}

\noindent Lynden-Bell's statement is supported by the fact that AGN are one of the most persistent luminous sources of electromagnetic radiation in the universe and as such, may be leveraged to discover distant objects. Furthermore, the evolution of AGN in cosmic time may be used to inform theoretical models of the cosmos. It is estimated that one fifth of research astronomers work on AGNs \citep{1997_Peterson}. 

The observed properties of AGNs depend on the mass of the central SMBH, the extent that the nucleus is obscured by dust, the orientation of the accretion disc, the rate of gas accretion, as well as the presence or absence of outflows of ionised matter along the axis of rotation known as jets. Some subclasses of AGN include quasars, the most powerful form of AGN, blazars, which contain a jet pointed toward the Earth, and Seyfert galaxies which are characterised by broad emission lines in the optical band. It is the last of these categories of AGN that is the subject of this chapter and will be discussed next.






\subsection{Seyfert Galaxies}

In 1943, Carl Seyfert systematically studied a collection of bright AGN possessing broad emission lines in the optical band \citep{1943_Seyfert}. The eponymous Seyfert galaxies are further subdivided into Seyfert 1 (Sy1) and Seyfert 2 (Sy2) galaxies based on their emission line range, $1000-20,000 \text{kms}^{-1}$ and $300-1000 \text{kms}^{-1}$ respectively. The orientation-based unified model of is one of the most popular means of describing Seyfert galaxies and is based on the idea that classes of AGN are physically similar but are viewed at different orientations \citep{1993_Antonucci, 1995_Urry}. Some features of the model include:

\begin{itemize}
    \item In the narrow line region there is ionised, low-velocity and low-density gas extending to $100-1000$ parsecs (pc).
    \item In the broad line region there are high-density, dust-free gas clouds located at a distance of $0.01-1$ pc from the SMBH moving at Keplerian velocities.
    \item There is an antisymmetric dusty structure known as a torus located at a distance of $0.1=10$ pc from the SMBH.
    \item There is a sub-pc accretion disc located around the SMBH which may be optically thick or optically thin depending on the disc's state.
    \item There is an outflowing radio jet pointed in the general direction of the accretion disc.
\end{itemize}

\begin{figure}[!htbp]
    \begin{center}
        \includegraphics[width=0.75\textwidth]{Background/Figs/smbh_diagram.png}
    \end{center}
    \caption{The orientation-based unified model of AGN. Reprinted with permission from \cite{2019_Jiang}.}
    \label{smbh}
\end{figure}

A schematic for the orientation-based unified model is provided in \autoref{smbh}. In Sy2 galaxies, the narrow line region is viewable by an edge-on observer due to the fact that it is more extended relative to the broad line region. Both the broad line region and the accretion disc are obscured by the middle plane of the torus. In Sy1 galaxies, the observer is closer to the torus axis and has an unobscured line of sight towards the nuclear region of the AGN. While the orientation-based unified model can explain the spectral diversity of many AGN, it has recently been challenged. For example, a different torus shape is suggested by interferometry observations in the mid-infrared band of Sy1 galaxies which demonstrate that the majority of infrared emission originates from dust in the polar region as opposed to the disc plane \citep{2013_Honig}.


\subsection{Space Observatories}

High energy X-rays function as the primary tool for examining the innermost regions of AGN as they originate from the area closest to the central SMBH and can easily penetrate absorbing materials along the line of sight. Optical/UV emission is also useful in characterising the behaviour of AGN, for example, in verifying reprocessing models of X-ray emission by computing lags between X-ray and optical/UV lightcurves, where a lightcurve is a graph of the light intensity of a celestial object or region with respect to time.

All astronomical data in this thesis originates from the Neil Gehrels Swift observatory which was first launched by NASA in 2004 to detect and study Gamma-Ray Bursts (GRBs) using the Burst Alert Telescope (BAT) \citep{2005_Barthelmy}. While originally designed for the study of GRBs, Swift now also functions as a multiwavelength observatory containing an X-ray Telescope (XRT) \citep{2005_Burrows} and a UV/Optical Telescope (UVOT) \citep{2005_Roming}. Swift is also used to conduct long-term all sky surveys.



\subsection{Accretion Disc Models}



Theoretical models to explain accretion discs differ based on the physical processes considered. Four representative examples are the Polish doughnut (thick disc), Shakura-Sunyaev (thin disc), slim disc, and the advection-dominated accretion flow (ADAF) models \citep{2011_Abramowicz}. These theoretical models are not mutually exclusive in the sense that different aspects of real physical systems may be best described by different models.

\noindent \textbf{Polish Doughnut (Thick Disc) Model:} The \say{Polish Doughnut} introduced by Paczynski and collaborators in the 1970s and 80s \citep{1980_Jaroszynski, 1980_Paczynski, 1981_Paczynski, 1982_Paczynski} is the minimal analytic accretion disc model in so far as it only considers gravity and assumes a perfect fluid. The Polish Doughnut Model is predicated on a general method for constructing perfect fluid equilibria of matter orbiting an uncharged, rotating black hole known as a Kerr black hole \citep{1963_Kerr}.

\noindent \textbf{Thin Disc Models:} The majority of analytic accretion disc models assume a stationary and axially-symmetric state for matter undergoing accretion onto the black hole with all physical quantities depending only on two spatial coordinates, $r$, the radial distance from the disc centre, and $z$, the vertical distance from the equatorial plane of symmetry. Unlike the Polish Doughnut Model, which assumes vertically thick discs, in thin disc models, $\frac{z}{r} \ll 1$ applies at all points within the matter distribution. In 1973, Shakura and Sunyaev introduced the canonical thin disc model \citep{1973_Shakura} by specifying additional physically reasonable assumptions that allowed them to construct a set of algebraic equations from the standard set of thin disc equations. The relativistic extension of the Shakura-Sunyaev model was later proposed by \cite{1973_Novikov}.

\noindent \textbf{Slim Disc Models:} Slim discs are characterised by $\frac{z}{r} \leq 1$. Thin disc models such as the Shakura-Sunyaev and Novikov–Thorne models assume that viscous heating is balanced locally by radiative cooling i.e. the accretion process is radially efficient and as such, all viscosity-generated heat is radiated away. Although the assumption is valid if the accretion rate is small, at a luminosity $L = 0.3 L_{\text{Edd}}$ \footnote{$L_{\text{Edd}}$ is the Eddington limit, the maximum achievable luminosity of a body subject to the balance of an outward radiative force and an inward gravitational force.} the radial velocity is large and the disc is sufficiently thick to permit advection to function as a cooling mechanism. At the highest luminosities, thin disc models no longer apply as the cooling effect of advection becomes comparable to radiative cooling. The standard thin disc model equations become a two-dimensional system of ordinary differential equations with a critical point for the slim disc case. These equations were first solved by \cite{1988_Abramowicz} and extended to a fully relativistic treatment by \cite{1998_Beloborodov}.

\noindent \textbf{ADAF Models:} Advection-dominated accretion flow models, introduced first in a series of papers \citep{1995_Narayan, 1994_Narayan, 1995_Abramowicz, 1996_Abramowicz, 1998_Gammie}, assume that almost all viscously dissipated energy is not radiated but advected into the black hole and applies when the luminosity and mass accretion rate are low. As such, ADAF discs are typically far less luminous than thin discs. Fully relativistic solutions to such discs have been obtained numerically \citep{1997_Abramowicz, 1997_Beloborodov}. Further information on ADAFs is available in \cite{1997_Narayan}.













\subsection{Markarian 335}


The accretion disc of Markarian 335 (Mrk 335) is the focus of study in this chapter. Mrk 335 is a Sy1 galaxy located 324 million light-years from Earth in the constellation of Pegasus. The central SMBH of Mrk 335 is notable for the spinning rate of its corona at ca. 20\% the speed of light. Relativistic blurring of the reflection of the accretion disc has been used to infer the geometry of the corona \citep{2015_Wilkins_drive}. By using \textsc{gp}s to interpolate the unevenly-sampled lightcurves of Mrk 335 and performing a cross-correlation analysis, some insight into the structure of the accretion disc may be obtained, and subsequently used to inform future developments in accretion disc theories. Of the aforementioned disc theories, the Shakura-Sunyaev thin disc model is the most relevant for Mrk 335 as its predictions for the extent of UV emission match that from observation. The distance between the UV and X-ray emission regions however is shorter than the light travel time measured using \textsc{gp}-based inference on the observational data. The main contributions of this chapter are now introduced.

\section{Preface}

The optical and UV variability of the majority of AGN may be related to the reprocessing of rapidly-changing X-ray emission from a more compact region near the central black hole. Such a reprocessing model would be characterised by lags between X-ray and optical/UV emission due to differences in light travel time. Observationally, however, such lag features have been difficult to detect due to gaps in the lightcurves introduced through factors such as source visibility or limited telescope time. In this chapter, \textsc{gp} regression is employed to interpolate the gaps in the Swift X-ray and UV lightcurves of the narrow-line Seyfert 1 galaxy Mrk 335. In a simulation study of five commonly-employed analytic \textsc{gp} kernels, it is concluded that the Matérn $\frac{1}{2}$ and rational quadratic kernels yield the most well-specified models for the X-ray and UVW2 bands of Mrk 335. In analysing the structure functions of the \textsc{gp} lightcurves, a broken power law is obtained with a break point at 125 days in the UVW2 band. In the X-ray band, the structure function of the \textsc{gp} lightcurve is consistent with a power law in the case of the RQ kernel, whilst a broken power law with a break point at 66 days is obtained from the Matérn $\frac{1}{2}$ kernel. The subsequent cross-correlation analysis is consistent with previous studies and furthermore, shows tentative evidence for a broad X-ray-UV lag feature of up to 30 days in the lag-frequency spectrum. The significance of the lag depends on the choice of \textsc{gp} kernel. 

\section{Introduction} \label{intro}

AGN show strong and variable emission across multiple wavelengths. The UV emission from an AGN is believed to be dominated by thermal emission from an accretion disc close to the central SMBH \citep{pringle81}. The variability of optical and UV AGN \footnote{AGN with an UV and optical luminosity change of more than 1 magnitude such as changing-look AGN, are not discussed in this chapter cf. \cite{jiang21} for details.} emission is stochastic and described by random Gaussian fluctuations \citep{welsh11,gezari13,zhu16,sanchez18,smith18,xin20} with the autocorrelation functions of such fluctuations adhering to the `damped random walk' model. The X-ray emission from an AGN is often found to show faster variability relative to emission at longer wavelengths \citep{mushotzky93, gaskell03} and originates from a more compact region \citep{morgan08,chartas17}.


The relationship between the UV and X-ray emission has been well studied. For instance, correlations between the variability in two energy bands has been seen in some individual sources \citep{shemmer01,buisson17} while others do not show significant evidence for similar correlation \citep{smith07, 2018_Buisson}. In sources where correlation is found, lags that are related to the light travel time between two emission regions are frequently observed. These lags are often found to be on timescales of days and are longer than those predicted by classical disc theories \citep{1973_Shakura}. Such lag amplitudes indicate a disc of size a few times larger than expected \citep{edelson00,shappee14,troyer16, buisson17}. Alternatively, some modified models have been proposed for the underestimation of lags by the classical thin disc model, e.g. disc turbulence \citep{cai20}, additional varying FUV illumination \citep{gardner17}, a tilted or inhomogeneous inner disc \citep{dexter11,starkey17} or an extended coronal region \citep{kammoun20}. Much shorter lags, e.g. hundreds of seconds, in agreement with the Shakura-Sunyaev model \citep{1973_Shakura} have been rarely observed by comparison e.g. in NGC-4395 \citep{mchardy16}.

The Neil Gehrels \swift\ Observatory has been monitoring the X-ray sky in the past decade in tandem with simultaneous pointings in the optical and UV band. In this work, we focus on the X-ray and UVW2 ($\lambda=$212~nm) lightcurves of the narrow-line Seyfert 1 galaxy (NLS1) Mrk~335 obtained by XRT and UVOT, the soft X-ray and UV/optical telescopes on \swift.  Mrk~335 was one of the brightest X-ray sources prior to 2007, before its flux diminished by $10-50\times$ its original brightness \citep{grupe07}.  The X-ray brightness has not recovered since.  During this low X-ray flux period, the UV brightness remains relatively unchanged rendering Mrk~335 X-ray weak \citep{tripathi20}. The behavior has been explained as a possible collapse of the X-ray corona \citep{2013_Gallo_New, 2015_Gallo_New, 2014_Parker_New} and/or increased absorption in the X-ray emitting region \citep{grupe12, 2013_Longinotti, 2019_Longinotti, 2019_Parker}.


Mrk~335 has been continuously monitored since 2007 making it one of the best-studied AGN with \swift. Previous studies from the \swift\ monitoring program can be found in \citet{grupe07,grupe12, gallo18, tripathi20, 2020_Komossa}. The X-rays are constantly fluctuating and regularly display large amplitude flaring \citep{2015_Wilkins}.  The UV are significantly variable, but at a much smaller amplitude than the X-rays. \cite{gallo18} found tentative evidence for lags of $\approx20$ days based on cross-correlation analyses, suggesting a potential reprocessing mechanism of the more variable X-ray emission in the UV emitter of this source. One challenge faced by the \swift\ monitoring program is that the lightcurves are not continuously sampled and hence standard Fourier techniques cannot be applied. This uneven sampling of the lightcurves is imposed by limited telescope time.




In the context of cross-correlation analysis, methods have been developed to address the problem of unevenly-sampled lightcurves. In \citet{2000_Reynolds}, the method of \citet{1992_Press} is extended to interpolate the lightcurve gaps using a model of the covariance function, or equivalently the power spectrum, of the lightcurve. In \citet{1998_Bond, 2010_Miller, 2013_Zoghbi} a maximum likelihood approach is taken to fit models of the lightcurve power spectra which accounts for the correlation between the lightcurves. In this paper we focus on a relatively new approach to tackle unevenly-sampled lightcurves. 

Gaussian processes (\textsc{gp}s) confer a Bayesian nonparametric framework to model general time series data \citep{2013_Roberts, 2015_Tobar} and have proven effective in tasks such as periodicity detection \citep{2016_Durrande} and spectral density estimation \citep{2018_Tobar}. More broadly \textsc{gp}s have recently demonstrated modelling success across a wide range of spatial and temporal application domains including robotics \citep{2011_Deisenroth, 2020_Greeff}, Bayesian optimisation \citep{2015_Shahriari, 2020_Grosnit, 2020_Rivers} as well as areas of the natural sciences such as molecular machine learning \citep{2021_Nigam, 2020_Griffiths, 2020_flowmo, 2021_Griffiths, 2021_Hase} and genetics \citep{2020_Moss}. In the context of astrophysics there is a recent trend favouring nonparametric models such as \textsc{gp}s due to the flexiblity afforded when specifying the underlying data modelling assumptions. Applications have arisen in lightcurve modelling \citep{2021_Luger1, 2021_Luger_next}, continuous-time autoregressive moving average (CARMA) processes \citep{2021_Yu}, modelling stellar activity signals in radial velocity data \citep{2015_Rajpaul}, lightcurve detrending \citep{2016_Aigrain}, learning imbalances for variable star classification \citep{2020_Lyon}, inferring stellar rotation periods \citep{2018_Angus}, estimating the dayside temperatures of hot Jupiters \citep{2019_Pass}, exoplanet detection \citep{2017_Jones, 2017_Czekala, 2020_Gordon, 2020_Langellier}, spectral modelling \citep{2012_Gibson, 2018_Nikolov, 2020_Diamond}, as well as blazar variability studies \citep{2015_Karamanavis, 2017_Karamanavis, 2020_Covino, 2020_Yang}.

It has recently been demonstrated in lightcurve simulations by \citet{2019_Wilkins} that a \textsc{gp} framework can compute time lags associated with X-ray reverberation from the accretion disc that are longer and observed at lower frequencies than can be measured by applying standard Fourier transform techniques to the longest available continuous segments. It is for this principal reason that \textsc{gp}s are employed for the timing analysis in this chapter. Further desirable facets of \textsc{gp}s include the fact that, unlike parametric models, they do not make strong assumptions about the shape of the underlying light curve \citep{2012_Wang_light}. Additionally, Bayesian model selection may be performed at the level of the covariance function or kernel allowing the quantitative comparison of different models of the lightcurve power spectrum. Finally in the cross-correlation analysis, a weaker modelling assumption is made than in \citet{2013_Zoghbi} in treating the X-ray and UV lightcurves as being independent \citep{2019_Wilkins}.


The remainder of this chapter is outlined as follows: In Section~\ref{gp_mod} procedures used to fit \textsc{gp}s to the X-ray and UVW2 bands are described, including aspects such as identification of the flux distribution, consideration of measurement noise as well as a simulation study to determine the appropriate kernels. In Section~\ref{structure_analysis} the structure functions of the \textsc{gp}-interpolated lightcurves are compared with the observational structure functions from \cite{gallo18}. In Section \ref{lag_and_coherence} a cross-correlation analysis of the X-ray and UVW2 bands is presented using the \textsc{gp}-interpolated lightcurves. Finally, in Section~\ref{conc} concluding remarks are provided about the discrepancy between the observational and \textsc{gp}-derived structure functions as well as the implications of the cross-correlation analysis, namely that the broad lag features suggest an extended emission region of the disc in Mrk 335 during the reverberation process. All code for reproducing the analysis is available at \url{https://github.com/Ryan-Rhys/Mrk_335}.

\section{Modelling Markarian 335} 
\label{gp_mod}

\noindent This Chapter considers the Swift X-ray and UVW2 lightcurves in time bins of one day. The reader is referred to \cite{gallo18} for details of the data reduction processes. The observational measurements used in this work run from $54327-58626$ modified Julian days and comprise $509$ data points for the X-ray band and $498$ data points for the UVW2 band. The latest UVOT sensitivity calibration file (`swusenscorr20041120v006.fits') was considered so as to account for the sensitivity loss with time in the UVW2 band\footnote{The most up-to-date calibration files: \href{https://heasarc.gsfc.nasa.gov/docs/heasarc/caldb/swift}{https://heasarc.gsfc.nasa.gov/docs/heasarc/caldb/swift}. Only UVW2 data collected by UVOT is considered because the UVW2 filter was most frequently used in the archival observations.}.

\subsection{Identifying the Flux Distribution}


In order to assess the applicability of \textsc{gp}s in modelling the flux distribution of the X-ray and UVW2 bands of Mrk 335, a series of graphical distribution tests were performed to determine the sample distribution. The histograms of the log count rates for the X-ray, and flux for the UV bands, of Mrk 335 are shown in \autoref{Histograms}. The histograms show that the distribution of the UVW2 flux is approximately Gaussian-distributed whereas the X-ray count rate distribution appears to be log-Gaussian distributed in line with the general observation of \cite{2005_Uttley} that fluxes from accreting black holes tend to follow log-Gaussian distributions. Further graphical distribution tests based on probability-probability (PP) plots and empirical cumulative distribution functions (ECDFs) are provided in Appendix~\ref{dist_tests}. 

Furthermore, following \cite{2019_Wilkins} a Kolmogorov-Smirnov test for goodness-of-fit was performed, where the null hypothesis is that the sample was drawn from a Gaussian distribution. For the UVW2 flux values a p-value of $0.164$ was obtained. For the raw X-ray count rates a p-value of $1.017e^{-20}$ was obtained, and a p-value of $0.028$ for the log-transformed X-ray count rates. As such, the null hypothesis that either UVW2 flux or log-transformed X-ray count rates are drawn from a Gaussian distribution cannot be rejected at the $1\%$ level of significance. The null hypothesis may however be rejected in the case of the raw X-ray count rates, providing evidence that the raw X-ray count rates should be log-transformed in order to be well-modelled by a Gaussian distribution. As such, the raw X-ray count rates were log-transformed and the UVW2 flux values were left unchanged.

\begin{figure*}[ht]
\centering
\subfigure[X-Ray Log Count Rates]{\label{fig:4}\includegraphics[width=0.49\textwidth]{Chapter3/Figures/new_xray_histogram.png}}
\subfigure[UVW2 Flux]{\label{fig:3}\includegraphics[width=0.49\textwidth]{Chapter3/Figures/new_uv_histogram_mags_is_False.png}}
\caption{Histograms of the observed Swift X-ray log count rate and UVW2 flux overlaid with Gaussian kernel density estimates. The raw UVW2 flux values have been scaled by $1e^{14}$.}
\label{Histograms}
\end{figure*}

\subsection{Noise Considerations}
\label{noise}

As noted by \cite{2019_Wilkins} fitting a \textsc{gp} to the logarithm of the count rate is appropriate only in the limit of a large signal-to-noise ratio. In the case of Mrk 335, the Poisson (shot) noise intrinsic to the photon detectors used to obtain the flux measurements is over an order of magnitude smaller than the flux measurement itself. As such the choice of the log-Gaussian process would appear to be justified. 

\subsection{Lightcurve Simulations}
\label{sim_section}

\begin{figure*}[h]
    \centering
    \includegraphics[width=0.75\textwidth]{Chapter3/Figures/resid_plot.png}
    \caption{Residual plot. The normalised RSS metric is the sum of squared residuals divided by the total number of discretised points (4390) comprising the simulated lightcurve. A residual in this case represents the difference between the Gaussian process predictive mean and the ground truth value of the simulated lightcurve.}
    \label{rss_figure}
\end{figure*}

A simulation study was undertaken to quantitatively assess the abilities of different kernels to interpolate gapped simulated lightcurves. Observational power spectral densities (PSDs) of AGN are well-described by (broken) power laws \citep{2004_Mchardy}. As such, the simulations employed a power law PSD with index fit to the observational data. The goals with the study are twofold: Firstly, although one cannot be sure of the true PSD for the observational data, it is hoped that the simulations may afford a good proxy for identifying performant kernels based on the fact that AGN typically exhibit power law-like PSDs and secondly, it is desirable to asses the correlation between a kernel's ability to reconstruct the full simulated lightcurve and its marginal likelihood value for the gapped data on which it is trained. If there is a correlation, the marginal likelihood may be used as a metric for identifying the appropriate kernel on the observational data.

One thousand simulated light curves with gaps were generated for the \src \: X-ray and UV bands using the method of \cite{1987_Davies}, first applied in astrophysics by \cite{1995_Timmer}. For each lightcurve there is access to the ground truth functional form of the lightcurve before the introduction of gaps. Computationally, the ground truth lightcurve was evaluated on a fine, discrete grid of $4390$ time points whereas the gapped lightcurves were evaluated on a coarser, unevenly-spaced grid of $498$ time points for the UV simulations and $509$ time points for the X-ray simulations in line with the number of observational data points. How well each \textsc{gp} kernel performs in recovering the ground truth lightcurve was then quantified by measuring the normalised residual sum of squared errors,


\begin{equation}
    \text{RSS} = \frac{1}{N} \sum_{i=1}^{N} (f(t_i) - y_i)^2,
\end{equation}


\noindent where $f(t_i)$ is the \textsc{gp} prediction at grid point $t_i$ and $y_i$ is the true simulated count rate value. The RSS values were averaged over the one thousand simulated lightcurves. An illustration of the RSS metric is provided in \autoref{rss_figure}. In addition, the averaged negative log marginal likelihood (NLML) values are computed for each kernel. Kernel hyperparameters were selected via optimisation of the NLML using the SciPy optimiser of GPflow \citep{GPflow}. The jitter level was fixed at 0.001, a small positive number to ensure numerical stability. The output values (flux or the logarithm of the count rate) were standardised according to their empirical mean and standard deviation. A constant mean function set to the empirical mean of the data following standardisation was employed.

The results of the simulation study are reported in \autoref{sim_study}. The NLML values show correlation with RSS, thus providing evidence that NLML is an appropriate metric for determining the \textsc{gp} kernel for the real observational data (for which the ground truth lightcurve is of course not available). A paired t-test was conducted to determine whether the RSS results were significant in terms of identifying the best kernel. For the X-ray simulations, a t-statistic of $9$ was obtained corresponding to a two-sided p-value of $5^{-20}$. For the UVW2 simulations, a t-statistic of $-22$ was obtained corresponding to a two-sided p-value of $9^{-85}$. As such, the null hypothesis that the performance discrepancy between kernels on the RSS metric is due to chance variation across $1000$ simulations, may be rejected at the $1\%$ level of significance. Further rationalisation for why the top two performing kernels in the simulation study are the Matérn $\frac{1}{2}$ and RQ kernels is offered in Appendix~\ref{kern_rat} (plotted by Douglas Buisson).



\begin{table}[h]
\caption{Performance comparison of kernels based on the NLML on the simulated gapped X-ray and UV lightcurves and normalised residual sum of squared errors (RSS) on the ground truth simulated lightcurves. The mean NLML and RSS across 1000 simulations are reported with the standard error. UVW2 RSS values have an exponent of $-30$.}
\label{sim_study}
\begin{center}
\begin{tabular}{l|ll}
\toprule
Kernel & NLML & RSS \\ \midrule
\multicolumn{1}{c|}{\underline{\textbf{X-Ray}}} & & \\[5 pt]
Matérn$\frac{1}{2}$ & $\textbf{180.2} \pm \: \textbf{3.8}$ & $0.121 \pm \: 0.002$  \\
Matérn$\frac{3}{2}$ & $420.7 \pm \: 3.3$ & $0.309 \pm \: 0.003$ \\
Matérn$\frac{5}{2}$ & $523.5 \pm \: 2.9$ & $0.374 \pm \: 0.003$ \\
Rational Quadratic & $\textbf{184.2} \pm \: \textbf{3.6}$ & $\textbf{0.117} \pm \: \textbf{0.002}$ \\
Squared Exponential & $632.1 \pm \: 1.5$ & $0.554 \pm \: 0.004$ \\ \midrule
 \multicolumn{1}{c|}{\underline{\textbf{UVW2}}} & & \\[5 pt]
Matérn$\frac{1}{2}$ & $\textbf{-399.0} \pm \: \textbf{5.2}$ & $\textbf{2.9} \pm \: \textbf{0.08}$ \\
Matérn$\frac{3}{2}$ & $-298.3 \pm \: 6.0$ & $7.9 \pm \: 0.25$ \\
Matérn$\frac{5}{2}$ & $-219.6 \pm \: 6.5$ & $17.0 \pm \: 0.41$ \\
Rational Quadratic & $-349.2 \pm \: 5.4$ & $3.4 \pm \: 0.09$ \\
Squared Exponential & $-65.0 \pm \: 7.4$ & $32.8 \pm \: 0.55$ \\ \bottomrule
\end{tabular}
\end{center}
\end{table}


\subsection{Modelling Markarian 335 with Gaussian Processes}

The fits to the observational data for the UVW2 and X-ray bands are shown in \autoref{GP_uv_fits} and \autoref{GP_xray_fits} respectively. In an analogous fashion to the simulation experiments five stationary kernels were evaluated: Matérn $\frac{1}{2}$, Matérn $\frac{3}{2}$, Matérn $\frac{5}{2}$, rational quadratic and squared exponential. The two kernels, rational quadratic and Matérn $\frac{1}{2}$, which performed best in the simulation study in their abilities to model power law-like PSDs are displayed. These kernels also have the most favourable values under the NLML metric for the observational data. A constant mean function set to the empirical mean of the data following standardisation was again used. All kernel hyperparameters were optimised under the marginal likelihood save for the noise level which was fixed to a constant value in the standardised space. This constant noise value is computed by dividing the mean output value in the standardised space by the mean signal-to-noise ratio in the original space.

\begin{figure*}[]
\centering
\subfigure[UV Band | Matérn $\frac{1}{2}$ | Mean.]{\label{fig:4pp1}\includegraphics[width=0.49\textwidth]{Chapter3/Figures/Matern_12_Kernel_and_-211_9190066627031_log_lik_and_0_02735732327535883_noise_color_00b764_mean.png}}
\subfigure[UV Band | Matérn $\frac{1}{2}$ | Sample.]{\label{fig:4pp2}\includegraphics[width=0.49\textwidth]{Chapter3/Figures/Matern_12_Kernel_and_-211_9190066627031_log_lik_and_0_02735732327535883_noise_color_00b764_sample.png}}
\subfigure[UV Band | Rational Quadratic | Mean.]{\label{fig:4pp3}\includegraphics[width=0.49\textwidth]{Chapter3/Figures/Rational_Quadratic_Kernel_and_-216_41870246779_log_lik_and_0_02735732327535883_noise_color_0d9a00_mean.png}}
\subfigure[UV Band | Rational Quadratic | Sample]{\label{fig:4pp4}\includegraphics[width=0.49\textwidth]{Chapter3/Figures/Rational_Quadratic_Kernel_and_-216_41870246779_log_lik_and_0_02735732327535883_noise_color_0d9a00_sample.png}}  
\caption{\textsc{gp} lightcurves for the UVW2 band. The shaded regions denote the \textsc{gp} 95\% confidence interval. Both the \textsc{gp} mean and a sample from the \textsc{gp} posterior are shown in separate plots. The insets are included to highlight the variability of the fit.}
\label{GP_uv_fits}
\end{figure*}

\begin{figure*}[]
\centering
\subfigure[X-ray Band | Matérn $\frac{1}{2}$ | Mean.]{\label{fig:4pp}\includegraphics[width=0.49\textwidth]{Chapter3/Figures/Matern_12_Kernel_and_-523_6156902622217_log_lik_and_0_0001_noise_color_e65802_mean.png}}
\subfigure[X-ray Band | Matérn $\frac{1}{2}$ | Sample.]{\label{fig:4pp}\includegraphics[width=0.49\textwidth]{Chapter3/Figures/Matern_12_Kernel_and_-523_6156902622217_log_lik_and_0_0001_noise_color_e65802_sample.png}}
\subfigure[X-ray Band | Rational Quadratic | Mean.]{\label{fig:4pp}\includegraphics[width=0.49\textwidth]{Chapter3/Figures/Rational_Quadratic_Kernel_and_-450_8868024165312_log_lik_and_0_0001_noise_color_cd6002_mean.png}}
\subfigure[X-ray Band | Rational Quadratic | Sample]{\label{fig:4pp}\includegraphics[width=0.49\textwidth]{Chapter3/Figures/Rational_Quadratic_Kernel_and_-450_8868024165312_log_lik_and_0_0001_noise_color_cd6002_sample.png}}  
\caption{\textsc{gp} lightcurves for the X-ray band. The shaded regions denote the \textsc{gp} 95\% confidence interval. Both the \textsc{gp} mean and a sample from the \textsc{gp} posterior are shown in separate plots. The insets are included to highlight the variability of the fit.}
\label{GP_xray_fits}
\end{figure*}

\section{Structure Function Analysis}
\label{structure_analysis}

Ideally one would like to examine the PSD of the \textsc{gp} fits to the observational data. The PSD characterises the distribution of power over frequencies of a given emission band and properties of the PSD can be linked to underlying physical processes in the accretion disc. Computation of the PSD, while possible, can be complicated by the uneven sampling of the observational data, leading previous studies to instead perform a structure function analysis on the \src \: data \citep{gallo18}. While it is possible to extract the PSD from the learned kernel \citep{2019_Wilkins}, in this work a structure function analysis of the \textsc{gp} lightcurves was performed in order to compare directly against the results of \cite{gallo18}. The method described in \cite{1985_Simonetti, 1992_Hughes, 1996_di_Clemente, 2001_Collier, gallo18} was followed. The binned structure function is defined as:

\begin{equation}
    \text{SF}(\tau) = \frac{1}{N(\tau)} \sum_{i} \: [f(t_i) - f(t_i + \tau)]^2,
\end{equation}


\noindent where $\tau = t_j - t_i$ is the distance between pairs of points $i$ and $j$ such that $t_j > t_i$. The structure function is binned according to $\tau$ where the centres of each bin are given by $\tau_i = (i - \frac{1}{2})\delta$. In this instance, $\delta$ is the structure function resolution. The same $\delta$ as in \cite{gallo18} was used, namely 5.3 days for the structure function computation over both the X-ray and UVW2 bands. $f(t_i)$ gives the count rate value at time point $t_i$ and $N(\tau)$ is the number of structure function pairs in each bin $i$ with centre $\tau_i$. Accounting for measurement noise by subtracting twice the mean noise variance from each structure function bin, as performed in \cite{gallo18} was found to have negligible effect on the \textsc{gp} structure functions and so was ignored. As in \cite{gallo18}, the structure function values were normalised by the global lightcurve variance. 

The \textsc{gp} structure functions for the interpolated lightcurves are shown in \autoref{GP Structure Functions}. The $1\sigma$ \textsc{gp} error bars were obtained by computing the structure function over 50 samples from the \textsc{gp} posterior. Each sample gives rise to highly similar structure functions and so the errors are not visible on the plot. The structure functions computed from the observational data, 509 and 498 data points for the X-ray and UV bands of \src \: respectively, are included for reference. In contrast to the \textsc{gp} structure function errorbars, in the case of the observational data the error bars are computed as $\frac{\sigma_i}{\sqrt(\frac{N_i}{2})}$ where $\sigma_i$ is the noise standard deviation in bin $i$ and $N_i$ is the number of pairs in bin $i$.\\ 

The \textsc{gp} structure functions are compared against the observational structure functions in \autoref{GP Structure Functions}. In addition, the broken power law fits to the \textsc{gp} structure functions are plotted, the parameters of which are given in \autoref{params}. In the UVW2 band, both \textsc{gp} kernels yield structure functions possessing a consistent break point at ca. 125 days. In the X-ray band the Matérn $\frac{1}{2}$ kernel yields a break point at 66 days whereas the rational quadratic kernel fit yields an unbroken power law. Given the discrepancy between \textsc{gp} kernels, definite evidence for a break in the X-ray power law is not found. 


Of particular interest is whether the dip in the X-ray structure function is an expected feature of the latent lightcurve or a measurement artefact arising from uneven sampling. In order to assess the potential for the dip to be a sampling artefact, simulations were performed using the Timmer and König algorithm from \autoref{sim_section}. In this case structure functions of gapped lightcurves were computed and compared against structure functions derived from the ground truth lightcurves with no gaps. One representative simulation is depicted in \autoref{sf_sims}. In this instance a similar dip to that found in the observational data is observed in the X-ray band simulation. This highlights the possiblity that the dip seen in the observational X-ray structure function is a sampling artefact arising from gaps in the lightcurve.

\begin{figure*}[]
\centering
\subfigure[Matérn $\frac{1}{2}$ UVW2]{\label{fig:4pp}\includegraphics[width=0.49\textwidth]{Chapter3/Figures/gp_structure_function_uv_Matern.png}}
\subfigure[Rational Quadratic UVW2]{\label{fig:4pp}\includegraphics[width=0.49\textwidth]{Chapter3/Figures/gp_structure_function_uv_RQ.png}}  
\subfigure[Matérn $\frac{1}{2}$ X-ray]{\label{fig:4pp}\includegraphics[width=0.49\textwidth]{Chapter3/Figures/gp_structure_function_xray_Matern.png}}
\subfigure[Rational Quadratic X-ray]{\label{fig:4pp}\includegraphics[width=0.49\textwidth]{Chapter3/Figures/gp_structure_function_xray_RQ.png}}
\caption{Comparison of observational and \textsc{gp} structure functions. The \textsc{gp} structure functions are consistent with those calculated from the observational data in the non-noise dominated regions. The dip at ca. $200$ days in the observational X-ray structure function is potentially a sampling artefact as demonstrated by simulation in \autoref{sf_sims}.}
\label{GP Structure Functions}
\end{figure*}

\begin{table}[]
\caption{Parameters for the broken power law fits to the \textsc{gp} structure functions. $\alpha_1$ and $\alpha_2$ are the indices for the power law before and after the break point $\tau_{char}$. The break point $\tau_{char}$ is reported in days. Errors were computed using 200 bootstrap samples of the data points corresponding to the \textsc{gp} structure functions. The X-ray rational quadratic structure function was fit using a power law and as such only has a single index as a parameter. The Astropy library \citep{astropy_1, astropy_2} was used to compute the (broken) power law fits using the simplex algorithm and least squares statistic for optimisation with the \textsc{gp} structure function uncertainties used as weights in the fitting.}
\begin{center}
\begin{tabular}{lllll}
\toprule
Waveband & Kernel & $\alpha_1$ & $\alpha_2$ & $\tau_{char}$ \\ \midrule
UVW2 & Matérn $\frac{1}{2}$ & $-0.72 \pm \: 0.03$ & $-0.26 \pm \: 0.01$ & $127 \pm \: 8$ \\
UVW2 & Rational Quadratic & $-0.62 \pm \: 0.01$ & $-0.28 \pm \: 0.01$ & $125 \pm \: 5$ \\
X-ray & Matérn $\frac{1}{2}$ & $-0.29 \pm \: 0.03$ & $-0.07 \pm \: 0.004$ & $66 \pm \: 8$ \\
X-ray & Rational Quadratic & $-0.21 \pm \: 0.002$ & N/A & N/A \\ \bottomrule
\end{tabular}
\end{center}
\label{params}
\end{table}

\begin{figure*}[]
\centering
\subfigure[Observational]{\label{fig:sf_sim}\includegraphics[width=0.49\textwidth]{Chapter3/Figures/ground_truth_function_xray_on_same_axis_.png}}
\subfigure[\textsc{gp} | Matérn $\frac{1}{2}$]{\label{fig:sf_sim2}\includegraphics[width=0.49\textwidth]{Chapter3/Figures/gp_function_xray_Matern_on_same_axis_10.png}}
\caption{Structure function simulations. Pseudo-observational lightcurves are produced by introducing gaps into the simulated lightcurves. The structure function for the gapped lightcurve is shown in red in (a) whereas the structure function of the \textsc{gp} interpolation is shown in red in (b). Both structure functions are compared against the ground truth structure function obtained from the full simulated ground truth lightcurve (no gaps). The dips at $\tau = 200 \: \text{days}$ and $\tau = 400 \: \text{days}$ in the structure function derived from the gapped observational simulation in 3.7(a) are artefacts of the uneven sampling.}
\label{sf_sims}
\end{figure*}

\section{Lag and Coherence}
\label{lag_and_coherence}

\begin{figure*}
    \centering
    \includegraphics{Chapter3/Figures/coherence_jitter_0001.pdf}
    \includegraphics{Chapter3/Figures/lag_jitter_0001.pdf}
    \caption{The coherence and lag spectra for Mrk~335, calculated by using 1000 pairs of \textsc{gp} lightcurve samples fit to the observed lightcurves. The error bars are the standard errors of the corresponding measurements for the 1000 samples. Different panels are for different kernels. Positive lags imply that the X-ray band leads the UVW2 band. Spectra plotted by Jiachen Jiang.}
    \label{pic_coh_lag}
\end{figure*}

In this section, the coherence between the UVW2 and X-ray emission from Mrk 335 is computed in search of evidence of lag features in the Fourier frequency domain. The coherence and lag spectra were estimated from one thousand pairs of UVW2 and X-ray \textsc{gp} lightcurve samples drawn from the \textsc{gp} posterior for each kernel. The lags are defined as the phase lags divided by the corresponding Fourier frequency. {A similar approach has been used in other disciplines \citep[e.g.][]{fabian09, kara13}.} Both Matérn $\frac{1}{2}$ and rational quadratic kernels are considered. The results are shown in \autoref{pic_coh_lag}. These spectra were plotted by Jiachen Jiang. Positive lags imply that the X-ray variability leads the UVW2 variability. The error bars in the figure are the standard errors of the corresponding measurements for the one thousand samples.

The coherence between the UVW2 and X-ray emission decreases with frequency, suggesting more coherent variability at lower frequency. Positive lag features are shown at the low frequencies in the range $f=$ 0.005--0.025 d$^{-1}$. The absolute value of the lag at $f=0.0039\pm0.0014$\,d$^{-1}$ is estimated to be $19 \pm 22$ days for the Matern $\frac{1}{2}$ kernel applied to both lightcurves and $29 \pm 19$ days for the rational quadratic kernel, however both measurements are consistent with zero lag in the $2\sigma$ uncertainty range.

Tentative evidence of a shorter time lag at a higher frequency of $f=0.018\pm0.006$\,d$^{-1}$ is also found. The longer lag feature at a lower frequency would correspond to a more extended emission region while the shorter lag feature at a higher frequency would correspond to a more compact region. This could be explained by the presence of an extended UV emission region on the disc where reverberation happens.

Given that the lags are consistent with zero lag within $2\sigma$ uncertainty ranges, it is concluded that only tentative evidence for a broad lag feature is found by applying \textsc{gp}s to the UVW2 and X-ray lightcurves of \src. Previous attempts to identify lags between two wavelengths of \src\ based on cross-correlation analysis in the time domain suggests similar results \citep[e.g.][]{gallo18}.

\section{Conclusions}
\label{conc}


Following the interpolation of the unevenly-sampled lightcurves of \src\ using \textsc{gp}s, tentative evidence for broad lag features is found in the Fourier frequency domain. The magnitude of the lags is consistent with previous cross-correlation analyses. In addition, the broad lag features {might} suggest an extended emission region e.g. of the disc in \src \: during the reverberation processes. {If the corona is compact within 5 $R_{\rm g}$ in Mrk~335 \citep{2015_Wilkins}, our data suggest a possibly wide range of UVW2 emission radii.}

The structure functions computed from the \textsc{gp}-interpolated lightcurves are consistent with those derived from the observational data and furthermore, illicit potential insights into the properties of the latent lightcurves. In particular, it is shown through a simulation study that it is possible that dips in the X-ray structure function may be produced by sampling artefacts arising from gaps in the lightcurve. In contrast, the \textsc{gp} structure functions show no dips. While this is not proof that the dip in the observational X-ray structure function is due to a sampling artefact, it does allude to the possibility. The UVW2 \textsc{gp} structure functions do not exhibit strong dependence on the choice of kernel with both Matérn $\frac{1}{2}$ and rational quadratic showing up a broken power law with breaks at 139 and 155 days respectively. The X-ray structure functions however do show up differences between kernels with the rational quadratic kernel predicting a power law and the Matérn 1/2 kernel predicting a broken power law.

From the \textsc{gp} modelling perspective, the ability to carry out Bayesian model selection affords a quantitative means of comparing analytic kernels under the marginal likelihood. It may be possible to incorporate further flexibility into the fitting procedure by making use of more sophisticated methods of kernel design \citep{2014_Duvenaud} to allow the assessment of fits of sums and products of analytic kernels or by leveraging advances in transforming \textsc{gp} priors via Deep \textsc{gp}s \citep{2013_Damianou} or normalising flows \citep{2020_Maronas}. Such approaches could be validated using simulation studies. Additionally, modelling the cross-correlation using multioutput \textsc{gp}s \citep{2020_de_Wolff} may be an interesting avenue for comparison against the approach taken here. Lastly, Bayesian spectral density estimation \citep{2018_Tobar} may afford further flexibility through nonparametric modelling of the PSD in addition to nonparametric modelling of the lightcurve in the time domain. These improvements in Bayesian modelling machinery may help to minimise model misspecification and as such, enable more robust inferences to be made about the functional forms of the latent lightcurves.





\nomenclature[Z-SMBH]{SMBH}{Supermassive Black Hole}
\nomenclature[Z-AGN]{AGN}{Active Galactic Nucleus}
\nomenclature[Z-Sy1]{Sy1}{Seyfert 1}
\nomenclature[Z-Sy2]{Sy2}{Seyfert 2}
\nomenclature[Z-GRB]{GRB}{Gamma-Ray Burst}
\nomenclature[Z-BAT]{BAT}{Burst Alert Telescope}
\nomenclature[Z-XRT]{XRT}{X-Ray Telescope}
\nomenclature[Z-UVOT]{UVOT}{UV/Optical Telescope}
\nomenclature[Z-ADAF]{ADAF}{Advection-Dominated Accretion Flows}
\nomenclature[Z-NLS1]{NLS1}{Narrow-Line Seyfert 1}
\nomenclature[Z-ECDF]{ECDF}{Empirical Cumulative Distribution Function}
\nomenclature[Z-PSD]{PSD}{Power Spectral Density}
\nomenclature[Z-RSS]{RSS}{Residual Sum of Squares}