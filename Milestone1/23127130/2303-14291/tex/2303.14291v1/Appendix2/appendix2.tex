
\chapter{Modelling Black Hole Signals with Gaussian Processes}

\section{Additional Graphical Tests for Identifying the Flux Distribution}
\label{dist_tests}

In \autoref{PP Plots} probability-probability (PP) plots and empirical cumulative distributions functions (ECDFs) are shown as graphical distribution tests for Gaussianity. It may be observed qualitatively that both X-ray band log count rates and UVW2 flux are well-modelled by a Gaussian distribution.


\begin{figure}[h!]
\centering
\subfigure[PP plot for X-ray log count rates]{\label{fig:4pt1}\includegraphics[width=0.49\textwidth]{Chapter3/Figures/xray_prob_plot.png}}
\subfigure[PP plot for UVW2 flux]{\label{fig:4pt2}\includegraphics[width=0.49\textwidth]{Chapter3/Figures/uv_prob_plot_mags_is_False.png}}
\subfigure[ECDF for X-ray log count rates]{\label{fig:4pt3}\includegraphics[width=0.49\textwidth]{Chapter3/Figures/xray_ecdf_with_ref.png}}
\subfigure[ECDF for UVW2 flux]{\label{fig:4pt4}\includegraphics[width=0.49\textwidth]{Chapter3/Figures/uv_ecdf_mags_is_False_with_ref.png}}  
\caption{PP plots and ECDFs for X-ray log count rates and UVW2 flux, graphical tests of Gaussianity. In the case of the PP plots, proximity to the line is an indicator of Gaussianity. In the case of the ECDF plots, resemblance to the cumulative distribution function of a Gaussian is indicative of Gaussianity. The  figures above were generated by Douglas Buisson.}
\label{PP Plots}
\end{figure}




\section{Spectral Properties of the Examined Kernels}
\label{kern_rat}

The autocorrelation functions, log autocorrelation functions and PSDs are illustrated for the Matérn, squared exponential and rational quadratic kernels in \autoref{kern_acfs}. The figures were generated by Douglas Buisson.

\begin{figure}[h!]
\centering
\subfigure[Kernel autocorrelation functions]{\label{fig:4k}\includegraphics[width=0.4\textwidth]{Chapter3/Figures/gp_kernel_acf.pdf}}
\subfigure[Kernel log autocorrelation functions]{\label{fig:3k}\includegraphics[width=0.4\textwidth]{Chapter3/Figures/gp_kernel_acf_log.pdf}}
\subfigure[Kernel PSDs]{\label{fig:5k}\includegraphics[width=0.4\textwidth]{Chapter3/Figures/gp_kernel_psds.pdf}}
\caption{Kernel autocorrelation functions and PSDs. The rational quadratic kernel is plotted for different values of the $\alpha$ parameter. The Matérn kernel plots in the PSD figure are offset by a factor of 10 for clarity. A PSD of $f^{-2}$ will match the high frequency part of the Matérn $\frac{1}{2}$ kernel and the rational quadratic is endowed with additional flexibility to model PSDs by virtue of its $\alpha$ parameter. Such characteristics may explain why these kernels are preferred in the simulation study.}
\label{kern_acfs}
\end{figure}

