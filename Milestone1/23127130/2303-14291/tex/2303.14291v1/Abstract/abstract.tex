\begin{abstract}

In many areas of the observational and experimental sciences data is scarce. Observation in high-energy astrophysics is disrupted by celestial occlusions and limited telescope time while laboratory experiments in synthetic chemistry and materials science are both time and cost-intensive. On the other hand, knowledge about the data-generation mechanism is often available in the experimental sciences, such as the measurement error of a piece of laboratory apparatus. \\
\indent Both characteristics make Gaussian processes (\textsc{gp}s) ideal candidates for fitting such datasets. \textsc{gp}s can make predictions with consideration of uncertainty, for example in the virtual screening of molecules and materials, and can also make inferences about incomplete data such as the latent emission signature from a black hole accretion disc. Furthermore, \textsc{gp}s are currently the workhorse model for Bayesian optimisation, a methodology foreseen to be a vehicle for guiding laboratory experiments in scientific discovery campaigns. \\
\indent The first contribution of this thesis is to use \textsc{gp} modelling to reason about the latent emission signature from the Seyfert galaxy Markarian 335, and by extension, to reason about the applicability of various theoretical models of black hole accretion discs. The second contribution is to deliver on the promised applications of \textsc{gp}s in scientific data modelling by leveraging them to discover novel and performant molecules. The third contribution is to extend the \textsc{gp} framework to operate on molecular and chemical reaction representations and to provide an open-source software library to enable the framework to be used by scientists. The fourth contribution is to extend current \textsc{gp} and Bayesian optimisation methodology by introducing a Bayesian optimisation scheme capable of modelling aleatoric uncertainty, and hence theoretically capable of identifying molecules and materials that are robust to industrial scale fabrication processes.



\end{abstract}


