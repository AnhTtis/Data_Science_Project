
\chapter{Molecular Discovery with Gaussian Processes}  %
\chapterimage[height=100pt]{Chapter1/Figs/azo_photoswitching.png}


\ifpdf
    \graphicspath{{Chapter1/Figs/Raster/}{Chapter1/Figs/PDF/}{Chapter1/Figs/}}
\else
    \graphicspath{{Chapter1/Figs/Vector/}{Chapter1/Figs/}}
\fi



\textbf{Status:} Accepted as Griffiths, RR., Greenfield, JL., Thawani, AR., Jamasb, AR., Moss HB., Bourached A., Jones, P., McCorkindale W., Aldrick AA., Fuchter MJ., Lee, AA. Data-Driven Discovery of Molecular Photoswitches with Multioutput Gaussian
Processes, \textit{Chemical Science} 2022.

\section{Preface}


This chapter is focussed on leveraging the predictive capabilities of \textsc{gp}s for molecular discovery. The discovery campaign is focussed on photoswitches, a particular class of molecule defined by their ability to convert between two or more isomeric forms in response to light. Photoswitches may be employed for information transfer and photopharmacological applications. Key  photoswitch properties in these domains include separation of the electronic absorption bands of the isomers as well as red-shifting of the absorption bands. The former property is useful for addressing a specific isomer and achieving high photostationary states (PSS), while the latter limits material damage from UV exposure and serves to increase the penetration depth for photopharmacological applications. The ability to engineer these properties, however, is challenging. As such, a predictive model is highly desirable for identifying novel and performant molecules.

In this chapter, a data-driven discovery pipeline for molecular photoswitches is presented, underpinned by dataset curation and multitask learning with \textsc{gp}s. In the prediction of electronic transition wavelengths, it is demonstrated that a multioutput Gaussian process (\textsc{mogp}) trained using labels from four photoswitch transition wavelengths yields the strongest predictive performance relative to single-task models as well as operationally outperforming time-dependent density functional theory (TD-DFT). The proposed approach is validated experimentally, by screening a library of commercially-available photoswitch molecules. Through this screen, several motifs are identified that displayed separated electronic absorption bands of their isomers and exhibit red-shifted absorptions. The curated dataset and all models are made available at \url{https://github.com/Ryan-Rhys/The-Photoswitch-Dataset}.

\section{Introduction}
\label{intro}



Photoswitch molecules are capable of reversible structural isomerisation upon irradiation with light as depicted in \autoref{fig:photo}, a characteristic behaviour that has led to a broad range of molecular \citep{Eisenreich2018,Dorel2019,Neilson2013}, supramolecular \citep{Corra2022, Han2016, Lee2022}, and materials applications \citep{Wang2021, Garcia-Amoros2012a, Hou2019, GouletHanssens2020}. Efficient light addressability is key to many of these applications for which photophysical properties of the photoswitch are the core determinant.

\begin{figure*}[ht!]
\centering
{\includegraphics[width=0.68\textwidth]{Chapter1/Figs/azo_photoswitching.png}\label{fig:photo_mech}}
\caption{Azobenzene, an example of a photoswitch that undergoes a reversible structural change upon irradiation with light. }
\label{fig:photo}
\end{figure*}




Properties which govern the utility of a photoswitch include quantum yields of photoswitching, the steady-state distribution of a particular isomer at a given irradiation wavelength (known as the photostationary state - PSS) as well as the thermal half-life of the metastable isomer. The desired thermal half-life depends on the application. Information transfer applications benefit from short thermal half-life photoswitches \citep{Garcia-Amoros2012a} whilst, in contrast, photoswitches used in energy storage are serviced by long thermal half-lives \citep{Dong2018}. In contrast, the attainment of separated isomeric electronic absorption bands and a high PSS are uniformly favourable properties for photoswitches as they dictate the light addressability of the isomeric forms. Minimal spectral overlap for a set irradiation wavelength is made possible by modulating the $\pi-\pi^*$ and $n-\pi^*$ bands of the \emph{E} and \emph{Z} isomers. Low spectral overlap maximises the composition of a given isomer at a set PSS. Inducing red-shifted absorption spectra away from the UV region is also desirable given that the use of high wavelength light decreases photo-induced material degradation and simultaneously improves tissue penetration depth.




To date, laboratory synthesis or quantum chemical calculations such as TD-DFT have been the choice approaches for measuring ground truth and computing predicted estimates of photoswitch properties. Both approaches are cost-intensive in terms of synthesis or compute time, although it should be noted that high-throughput DFT approaches have potential to mitigate the wall-clock time to some extent in the future \citep{2017_Lopez, 2018_Wilbraham, 2020_Choudhary}. In light of this, human intuition remains the guide for candidate selection in many photoswitch chemistry laboratories. Advances in molecular machine learning, however, have taken great strides in recent years. In particular, machine learning property prediction has the potential to cut the attrition rate in the discovery of novel and impactful molecules by virtue of its short inference time. A rapid, accessible, and accurate machine learning prediction of a photoswitch's properties prior to synthesis would allow promising structures to be prioritised, facilitating photoswitch discovery as well as revealing new structure-property relationships.









Recently, work by Lopez and co-workers \citep{2021_Mukadum} employed machine learning to accelerate a quantum chemistry screening workflow for photoswitches. The screening library in this case is generated from 29 known azoarene photoswitches and their derivatives yielding a virtual library of 255,991 photoswitches in total. The authors observed that screening using active search tripled the discovery rate of photoswitches compared to random search according to a binary labelling system which assigns a positive label to a molecule possessing a $\lambda_{\text{max}} > 450 \text{nm}$ and a negative label otherwise. The approach highlights the potential for AL and \textsc{bo} methodology to accelerate DFT-based screening. Nonetheless, to the best of our knowledge, open questions remain in terms of the utility of machine learning-based predictive models for experimental photoswitch properties, in addition to experimental validation of machine learning approaches.

In this chapter an experimentally-validated framework for molecular photoswitch discovery is presented based on the curation of a large dataset of experimental photophysical data, and multitask learning with \textsc{mogp}s. This framework was designed with the goals of: (i) performing faster prediction relative to TD-DFT and directly training on experimental data; (ii) obtaining improved accuracy relative to human experts; (iii) operationalising model predictions in the context of laboratory synthesis. To achieve these goals, a dataset of the electronic absorption properties of 405 photoswitches in their \emph{E} and \emph{Z} isomeric forms was curated originally by Aditya Raymond Thawani, a full description of the dataset and collated properties is provided in \autoref{sec:data_description}. 

Following an extensive benchmark study, an appropriate machine learning model and molecular representation was identified for prediction, as detailed in \autoref{sec:model_choice}. A key feature of this model is that it is performant in the small data regime as photoswitch properties (data labels) obtained via laboratory measurement are expensive to collect in both financial cost and time. The chosen model uses a \textsc{mogp} approach due to its ability to operate in the multitask learning setting, amalgamating information obtained from molecules with multiple labels. In \autoref{sec:dft} it is shown that the \textsc{mogp} model trained on the curated dataset obtains comparable predictive accuracy to TD-DFT (at the CAM-B3LYP level of theory) and only suffers slight degradations in accuracy relative to TD-DFT methods with data-driven linear corrections whilst maintaining inference time on the order of seconds. A further benchmark against a cohort of human experts is included in \autoref{sec:human}. In \autoref{sec:valid} the approach is used to screen a set of commercially-available azoarene photoswitches, and in the process, identify several motifs displaying separated electronic absorption bands of their isomers as well as red-shifted absorptions, thus making them suitable for information transfer
and photopharmacological applications.






\section{Dataset Curation}
\label{sec:data_description}

Experimentally-determined properties of azobenzene-derived photoswitch molecules reported in the literature were curated initially by Aditya Raymond Thawani. Azobenzene derivatives in possession of diverse substitution patterns and functional groups were included to cover as large a fraction of chemical space as possible. Azoheteroarenes and cyclic azobenzenes were also included. The dataset includes properties for 405 photoswitches denoted using the SMILES syntax. A full list of references for the data sources is provided in Appendix~\ref{exp_source}. 

The following properties from the literature were collated, where available. (i) The rate of thermal isomerisation (units = $s^{-1}$), a solution-based measure of the thermal stability of the metastable isomer. For cyclic azophotoswitches, this corresponds to the \textit{E} isomer, whereas for non-cyclic azophotoswitches the rate is for the \textit{E} isomer. (ii) The PSS of each isomer at the set wavelength of photoirradiation. Such values are obtained through continuous, solution-based irradiation of a photoswitch until the point at which the steady-state distribution of the \emph{E} and \emph{Z} isomers is observed. The PSS values reported in the literature all correspond to solution-phase measurements. (iii) The irradiation wavelength (nm)  corresponds to the wavelength of light employed to irradiate samples, such that a PSS is attained, from \emph{E}-\emph{Z} or \emph{Z}-\emph{E}. (iv) Experimental transition wavelengths (nm) correspond to the wavelength at which the $\pi-\pi{^*}$/$\emph{n}-\pi{^*}$ electronic transition attains a maximum for the given isomer. This data was curated from solution-phase measurements. (v) DFT-computed transition wavelengths (nm), obtained using solvent continuum TD-DFT methods, correspond to the predicted $\pi-\pi{^*}$/$\emph{n}-\pi{^*}$ electronic transition maximum for a given isomer. (vi) The extinction coefficient (M$^{-1}$cm$^{-1}$), corresponds to the extent to which a molecule absorbs light, conditioned on the solvent. (vii) The theoretically-computed Wiberg Index \citep{1968_Wiberg} (through the analysis of the SCF density calculated at the PBE0/6-31G** level of theory), a measure of the bond order of the N=N bond in an azo-based photoswitch, provides an indication of the ‘strength’ of the azo bond.

Following the curation of the Photoswitch dataset, the goal is to use a machine learning model to predict the four experimentally-determined transition wavelengths. These four properties were chosen as they are core determinants of quantitative, bidirectional photoswitching \citep{2019_Crespi}. The wavelength properties include, the $\pi-\pi{^*}$ transition wavelength of the \emph{E} isomer (labels for 392 molecules), the \emph{n}$-\pi{^*}$ transition wavelength of the \emph{E} isomer (labels for 141 molecules), the $\pi-\pi{^*}$ transition wavelength of the \emph{Z} isomer (labels for 93 molecules), and the \emph{n}$-\pi{^*}$ transition wavelength of the \emph{Z} isomer (labels for 123 molecules). While other photophysical or thermal properties, such as the thermal half-life of the metastable state, could also be investigated using machine learning approaches, there are fewer reported measurements of thermal half-lives which significantly reduces the amount of data that may be used to train a model.


\section{Machine Learning Prediction Pipeline}\label{sec:model_choice}

There are three constituents to the prediction pipeline: A dataset, a model and a representation. The effects of the choice of dataset are examined in Appendix~\ref{sec:big_data}, where performance is compared between models trained on the curated dataset against those trained on a large out-of-domain dataset of 6,142 photoswitches \citep{2019_Beard}. In terms of the choice of model, a broad range of models are evaluated including Gaussian processes (\textsc{gp}), random forest (\textsc{rf}), Bayesian neural networks (BNNs), graph convolutional networks (GCNs), message-passing neural networks (MPNNs), graph attention networks (GATs), LSTMs with augmented SMILES, attentive neural processes (\textsc{anp}), as well as multioutput Gaussian processes (\textsc{mogp}), which aggregate information across prediction tasks to perform multitask learning \citep{1997_Caruana}.

Full model benchmark results, as well as all hyperparameter settings, are provided in Appendix~\ref{benchmark_ml}, where Wilcoxon signed rank tests \citep{1945_Wilcoxon} determine that there is weak evidence to support that multitask learning affords improvements over the single task setting in the case where auxiliary task labels (i.e. not the label being predicted) are available for test molecules. All subsequent experiments in this chapter assume that the \textsc{mogp} is not provided with auxiliary task labels for test molecules. All experiments may be reproduced via the scripts provided at \url{https://github.com/Ryan-Rhys/The-Photoswitch-Dataset}.
The \textsc{mogp} was chosen to take forward to the comparison against TD-DFT and experimental screening due to its predictive performance in the multitask setting as well as its ability to represent uncertainty estimates. Some use-cases for the \textsc{gp} uncertainty estimates with confidence-error curves are illustrated in Appendix~\ref{conf_error}.

\begin{figure}[!htbp]
    \begin{center}
        \includegraphics[width=0.98\textwidth]{Chapter1/Figs/new_box_nov.png}
    \end{center}
    \caption{Marginal boxplot showing the performance of representations aggregated over different models (\textsc{rf}, \textsc{gp}, \textsc{mogp} and \textsc{anp}). Performance is evaluated on 20 random train/test splits of the Photoswitch Dataset in a ratio of 80/20 using the mean absolute error (MAE) as the performance metric. An individual box is computed using the mean values of the MAE for the four models for the representation indicated by the associated colour and shows the range in addition to the upper and lower quartiles of the error distribution. The plot indicates that fragprints are the best representation on the \emph{E} isomer $\pi - \pi^*$ prediction task and RDKit fragments alone are disfavoured across all tasks.}
    \label{boxplot}
\end{figure}
 
In terms of the choice of representation, three commonly-used descriptors are evaluated: RDKit fragment features \citep{rdkit}, ECFP fingerprints \citep{2010_Rogers} as well as a hybrid 'fragprints' representation formed by concatenating the Morgan fingerprint and fragment feature vectors. The performance of the RDKit fragment, ECFP fingerprint, and fragprint representations on the wavelength prediction tasks is visualised in \autoref{boxplot} where aggregation is performed over the \textsc{rf}, \textsc{gp}, \textsc{mogp} and \textsc{anp} models. This analysis motivated the use of the fragprints representation in conjunction with the \textsc{mogp} to take forward to the TD-DFT comparison and experimental screening. The \textsc{mogp} with Tanimoto kernel employed for prediction will now be described.

\subsection{Multioutput Gaussian Processes (MOGPs)}

A \textsc{mogp} generalises the idea of the \textsc{gp} to multiple outputs and a common use case is multitask learning. In multitask learning, tasks are learned in parallel using a shared representation; the idea being that learning for one task may benefit from the training signals of related tasks. In the context of photoswitches, the tasks constitute the prediction of the four transition wavelengths. We wish to perform Bayesian inference over a stochastic function $f: \mathbb{R}^D \to \mathbb{R}^P$ where $P$ is the number of tasks and we possess observations $\{(\mathbf{x_{11}}, y_{11}), \dotsc , (\mathbf{x_{1N}}, y_{1N}), \dotsc , (\mathbf{x_{P1}}, y_{P1}), \dotsc , (\mathbf{x_{PN}}, y_{PN})\}$. We do not necessarily have property values for all tasks for a given molecule. 

To construct a \textsc{mogp} we compute a new kernel function $k(\mathbf{x}, \mathbf{x'}) \cdot B[i, j]$ where $B$ is a positive semi-definite $P \times P$ matrix, where the $(i, j)^{\text{th}}$ entry of the matrix $B$ multiplies the covariance of the $i$-th function at $\mathbf{x}$ and the $j$-th function at $\mathbf{x'}$. Such a \textsc{mogp} is termed the intrinsic coregionalisation model (ICM) \citep{2007_Williams}. Inference proceeds in the same manner as for vanilla \textsc{gp}s, substituting the new expression for the kernel into the equations for the predictive mean and variance. Positive semi-definiteness of $B$ may be guaranteed through parametrising the Cholesky decomposition $LL^{\top}$, where $L$ is a lower triangular matrix and the parameters may be learned alongside the kernel hyperparameters through maximising the marginal likelihood in \autoref{equation: log_lik_} substituting the appropriate kernel. In all our experiments we use bit/count vectors to represent molecules and hence we choose the Tanimoto kernel defined in \autoref{equation: tanimoto}.

While it has been widely cited that \textsc{gp}s scale poorly to large datasets due to the $O(N^3)$ cost of training, where $N$ is the number of datapoints \citep{2006_Rasmussen}, recent advances have seen \textsc{gp}s scale to millions of data points using multi GPU parallelisation \citep{2019_Pleiss}. Nonetheless, on CPU hardware, scaling \textsc{gp}s to datasets on the order of $10,000$ data points can prove challenging. For the applications considered in this chapter, however, we are unlikely to be fortunate enough to encounter datasets of relevant experimental measurements on the order of tens of thousands of data points and so CPU hardware is sufficient for this purpose.



\section{TD-DFT Performance Comparison}\label{sec:dft}


The \textsc{mogp}, Tanimoto kernel and fragprints combination are compared against two widely-utilised levels of TD-DFT: CAM-B3LYP \citep{2004_Yanai} and PBE0 \citep{1996_Perdew, 1999_Adamo}. While the CAM-B3LYP level of theory offers highly accurate predictions, its computational cost is high relative to that of machine learning methods. To obtain the predictions for a single photoswitch molecule one is required to perform a ground state energy minimisation followed by a TD-DFT calculation \citep{2015_Belostotskii}. In the case of photoswitches these calculations need to be performed for both molecular isomers and possibly multiple conformations which further increases the wall-clock time. When screening multiple molecules is desirable, this cost, in addition to the expertise required to perform the calculations may be prohibitive, and so in practice it is easier to screen candidates based on human chemical intuition. In contrast, inference in a data-driven model is on the order of seconds but may yield poor results if the training set is out-of-domain relative to the prediction task. 

In \autoref{tab_merge1} a performance comparison is presented against 99 molecules and 114 molecules for CAM-B3LYP and PBE0 respectively, both using the 6-31G** basis set taken from the results of a benchmark quantum chemistry study \citep{2011_Jacquemin}, to which the reader is referred for all information pertaining to the details of the calculations. \footnote[1]{The TD-DFT CPU runtime in \autoref{tab_merge1} estimates are taken from \citep{2015_Belostotskii} and hence represent a ballpark figure that is liable to decrease with advances in high performance computing.} An additional $15$ molecules are included in the test set for PBE0. These molecules are not featured in the study by \citet{2011_Jacquemin}, but are included from the other literature sources present in the Photoswitch Dataset which use the same basis set. It should also be noted that the data presented in \citet{2011_Jacquemin} contains measurements for the same molecules under different solvents. In this chapter, solvent effects are absorbed into the noise. Specifically, the solvent is not treated as part of the molecular representation. As such, for duplicated molecules a single solvent measurement is chosen at random. We report the mean absolute error (MAE) and the mean signed error (MSE), presented in Appendix~\ref{spearman_section}, to assess systematic deviations in predictive performance for the TD-DFT methods. For the \textsc{mogp} model, leave-one-out validation is performed, testing on a single molecule and training on the others as well as the experimentally-determined property values for molecules acquired from the Photoswitch Dataset. The prediction errors are then averaged and the standard error is reported.

\begin{table*}[h]
\caption{\textsc{mogp} against TD-DFT performance comparison on the PBE0 benchmark consisting of 114 molecules, and the CAM-B3LYP benchmark consisting of 99 molecules. Best metric values for each benchmark are highlighted in bold.}
\resizebox{0.98\textwidth}{!}{
\centering
\begin{tabular}{l l | c  c | c}
    \toprule
    \multicolumn{2}{c|}{{\bf Method}} & \multicolumn{2}{c|}{{\bf Accuracy Metric (nm)}} & \multicolumn{1}{c}{{\bf CPU Runtime ($\downarrow$)}}  \\
     &  & MAE ($\downarrow$) & MSE &  \\
    \hline
    \multicolumn{2}{c|}{{\bf \underline{PBE0 Benchmark}}} & 
    & 
    \\
    \textsc{mogp} & & $15.5 \pm 1.3$ & $\textbf{0.0} \pm \textbf{2.0}$ & $\textbf{<}$ \textbf{1 minute} \\
    PBE0 & uncorrected & $26.0 \pm 1.8$ & $-19.1 \pm 2.5$ &  \\
    & linear correction & $\textbf{12.4} \pm \textbf{1.3}$ & $-1.2 \pm 1.8$ & ca. 228 days \\ 
    \hline
    \multicolumn{2}{c|}{{\bf \underline{CAM-B3LYP Benchmark}}} & 
    & 
    \\
    
    \textsc{mogp} & & $15.3 \pm 1.4$ & $-0.2 \pm 2.1$ & $\textbf{<}$ \textbf{1 minute} \\
    
    CAM-B3LYP & uncorrected & $16.5 \pm 1.6$ & $6.7 \pm 2.2$ &  \\
    & linear correction & $\textbf{10.7} \pm \textbf{1.2}$ & $\textbf{0.0} \pm \textbf{1.6}$ & ca. 396 days \\
    \bottomrule
\end{tabular}}
\label{tab_merge1}
\end{table*}


The \textsc{mogp} model outperforms PBE0 by a large margin and provides comparable performance to CAM-B3LYP in terms of accuracy. The MSE values for the TD-DFT methods, however, indicate that there is systematic deviation in the TD-DFT predictions. This motivates the addition of a data-driven correction to the TD-DFT predictions. As such, a Lasso model, with an $L_1$ multiplier of $0.1$, is trained on the prediction errors of the TD-DFT methods and this correction is applied when evaluating the TD-DFT methods on the heldout set in leave-one-out validation. Lasso is chosen because it outperforms linear regression empirically in fitting the errors, likely due to inducing sparsity in the high-dimensional fragprint feature vectors. The Spearman rank-order correlation coefficients of all methods as well as the error distributions are provided in Appendix~\ref{spearman_section}. There, it is observed that an improvement is obtained in the correlation between TD-DFT predictions on applying the linear correction. Furthermore, the error distribution becomes more symmetric on applying the correction. 

\section{Human Performance Benchmark}\label{sec:human}

\begin{figure*}[!htbp]
\centering
\subfigure{\includegraphics[width=0.41\textwidth]{Chapter1/Figs/human_mols2.png}}  
\subfigure{\label{fig:4ppb}\includegraphics[width=0.55\textwidth]{Chapter1/Figs/human_performance_comparison_new.png}}
\caption{A performance comparison between human experts (orange) and the \textsc{mogp}-fragprints model (blue). MAEs are computed on a per molecule basis across all human participants.}
\label{human}
\end{figure*}

In practice, candidate screening is undertaken based on the opinion of a human chemist due to the speed at which predictions may be obtained. While inference in a data-driven model is comparable to the human approach in terms of speed, the aim in this section is to compare the predictive accuracy of the two approaches. To achieve this, a panel of 14 photoswitch chemists were assembled, comprising Postdoctoral Research Assistants and PhD students in photoswitch chemistry with a median research experience of 5 years. The assigned task was to predict the \emph{E} isomer $\pi-\pi{^*}$ transition wavelength for five molecules taken from the dataset. The study was designed by Aditya Raymond Thawani who was also responsible for recruiting human participants. All model predictions and plots were generated by Ryan-Rhys Griffiths.

All participants had prior knowledge of UV-vis spectroscopy. It should be noted that one of the limitations of this study is that the human chemists were not provided with the full dataset of 405 photoswitch molecules in advance of making their predictions. As such, the goal in constructing the study was to enable a comparison of the benefits of dataset curation, together with a machine learning model to internalise the information contained in the dataset, against the experience acquired over a photoswitch chemist's research career. Analysing the MAE across all humans per molecule \autoref{human}, it is observed that the human chemists perform worse than the \textsc{mogp} model in all instances. In going from molecule A to E, the number of point changes on the molecule increases steadily, thus, increasing the difficulty of prediction. Noticeably, the human performance is approximately five-fold worse on molecule E (three point changes) relative to molecule A (one point change). This highlights the fact that in instances of multiple functional group modifications, human experts are unable to reliably predict the impact on the \emph{E} isomer $\pi-\pi^*$ transition wavelength. The full results breakdown is provided in Appendix~\ref{sec:human_app}.

\section{Screening for Novel Photoswitches using the MOGP}
\label{sec:valid}

Having determined that the \textsc{mogp} approach does not suffer substantial degradation in accuracy relative to TD-DFT, the model was subsequently used to perform experimental screening. Diazo-containing compounds supplied by Molport and Mcule were identified. There were 7,265 commercially-available diazo molecules as of November 2020, when experiments were planned. The full list is made available at \href{https://github.com/Ryan-Rhys/The-Photoswitch-Dataset/tree/master/dataset}{https://github.com/Ryan-Rhys/The-Photoswitch-Dataset/tree/master/dataset}. The \textsc{mogp} was then used to score the list. A subset of 11 molecules were chosen to screen which satisfied the criteria detailed in the following section. The goal of the screening was to discover a novel azophotoswitch motif satisfying the performance criteria.

\subsection{Screening Criteria}

To demonstrate the utility of the machine learning prediction pipeline, commercially-available photoswitches were screened based on a set of performance criteria. The experimental properties of the screened photoswitches were subsequently measured and compared against the predictions made by the \textsc{mogp} model. The criteria were selected to demonstrate that beneficial properties for materials and photopharmacological applications, which are difficult to engineer, could be obtained using the \textsc{mogp} model. The criteria are:

\begin{enumerate}
    \item A $\pi-\pi^*$ maximum between 450-600 nm for the \textit{E} isomer.
    \item A separation in excess of 40 nm between the $\pi-\pi^*$ of the \textit{E} isomer and the $\pi-\pi^*$ of the \textit{Z} isomer.
\end{enumerate}

The first criterion was imposed to limit UV-included material damage and enhance tissue penetration depths. The second criterion was chosen to provide complete bidirectional photoswitching as the specified degree of separation between the $\pi-\pi^*$ bands of the isomers facilitates a given isomer to be selectively addressed using light emitting diodes (LEDs), commonly used for their low power consumption and ability to express broad emission profiles relative to laser diodes.

 \begin{figure*}[!htbp]
    \begin{center}
        \includegraphics[width=0.8\textwidth]{Chapter1/Figs/structures_5.png}
    \end{center}
    \caption{The chemical structures of the 11 commercially-available azo-based photoswitches that were predicted to meet the criteria. Figure produced by Jake Greenfield.}
    \label{structures2}
\end{figure*}

\subsection{Lead Candidates}

Based on the stated selection criteria, 11 commercially-available molecules were identified via the predictions of the \textsc{mogp} model. The molecular structures are shown in \autoref{structures2}. Solutions of the 11 photoswitches were prepared in the dark to a concentration of 25 $\mu$M in DMSO. The UV-vis spectra of the photoswitches were recorded using a photodiode array spectrometer where the photoswitches were in their thermodynamically stable \emph{E} isomeric form. Samples were continuously irradiated with wavelengths of light at an angle of 90$^{\circ}$ relative to the measurement path. UV-vis spectra were recorded during irradiation until no further change in the UV-vis trace was observed, indicating attainment of the PSS. The \emph{in situ} irradiation procedure was implemented such that compounds displaying short thermal half-lives could be reliably measured. Through repetition of the measurement process with one or more distinct irradiation wavelengths, the PSS could be quantified and subsequently used to predict the UV-vis spectrum of the pure \emph{Z} isomer using the method detailed by \citet{Fischer1967}. With spectra of the \emph{E} and \emph{Z} isomers in hand, the experimental wavelength of the $\pi-\pi^*$ band of each isomer was determined and compared with that predicted by the \textsc{mogp}. Full experimental details are made available in Appendix~\ref{exp_app}. 

Model predictions are compared against the experimentally-determined values in \autoref{tab_preds}. The \textsc{mogp} MAE on the \emph{E} isomer $\pi-\pi^*$ wavelength prediction task was $22.7$ nm and $21.6$ nm on the \emph{Z} isomer $\pi-\pi^*$ wavelength prediction task, comparable for the \emph{E} isomer $\pi-\pi^*$ and slightly higher for the \emph{Z} isomer $\pi-\pi^*$ relative to the benchmark study in Appendix~\ref{benchmark_ml}, reflecting the challenge of achieving strong generalisation performance when extrapolating to large regions of chemical space. The first criterion is a requirement on the absolute rather than the relative value of the $\pi-\pi^*$ transition wavelengths and so the experimental values may be subject to shifts depending on the solvent.

Molecules can display solvatochromism in so far as the dielectric of the solvent, as well as hydrogen-bonding interactions, can influence the electronic transitions giving rise to hypsochromic or bathochromic shifts in the absorption spectra. This can manifest as changes in the position, intensity and shape of the UV-vis absorption spectrum. As such, the 450 nm criterion could be considered a rough guide and candidates that are just short of the threshold may fulfill the criterion in a different solvent. Nonetheless, given that the \textsc{mogp} model is trained on just a few hundred data points and is required to extrapolate to several thousand structures, the accuracy is promising with the advent of further experimental data. In terms of satisfying the pre-specified criteria, 7 of the 11 molecules possessed an \emph{E} isomer $\pi-\pi^*$ wavelength greater than 450 nm, 10 of the 11 molecules possessed a separation between the \emph{E} and \emph{Z} isomer $\pi-\pi^*$ wavelengths of greater than 40 nm, and 6 of the 11 molecules satisfied both criteria. Compound 7 did not photoswitch under irradiation.

\definecolor{caribbeangreen}{rgb}{0.0, 0.8, 0.6}
\definecolor{carrotorange}{rgb}{0.93, 0.57, 0.13}
\definecolor{cinnabar}{rgb}{0.89, 0.26, 0.2}

\begin{table}[h]
\caption{\textsc{mogp} predictions compared against experimental values (nm). A traffic light system indicates whether the molecules satisfied the criteria. Both criteria are indicated by (\color{caribbeangreen}{green}\color{black}{) and one criterion is indicated by}(\color{carrotorange}{orange}\color{black}{). All molecules satisfied at least one criterion. The model MAE was 22.7 nm for the \emph{E} isomer $\pi - \pi^*$} and 21.6 nm for the \emph{Z} isomer $\pi - \pi^*$. Experimental measurements were taken by Jake Greenfield.}
\resizebox{0.983\textwidth}{!}{
\centering
\begin{tabular}{c|cc|cccc}
\toprule
& \multicolumn{2}{c|}{{ \bf \underline{Model}}} & \multicolumn{4}{c}{{ \bf \underline{Experimental}}} \\
Switch & \begin{tabular}[c]{@{}c@{}}\emph{E} $\pi - \pi^*$ \end{tabular} & \begin{tabular}[c]{@{}c@{}}\emph{Z} $\pi - \pi^*$. \end{tabular} & \begin{tabular}[c]{@{}c@{}}\emph{E} $\pi - \pi^*$ \end{tabular} & \begin{tabular}[c]{@{}c@{}}\emph{Z} $\pi - \pi^*$ \end{tabular} & \emph{Z} PSS (\%) & \begin{tabular}[c]{@{}c@{}}ca. t$\frac{1}{2}$  (s)\end{tabular}  \\
\midrule
\color{carrotorange}
\textbf{1}  & \color{carrotorange}456 & \color{carrotorange}368 & \color{carrotorange}446 &  \color{carrotorange}355 &  \color{carrotorange}90 (405 nm) & \color{carrotorange}<5 \\ 
\color{carrotorange}
\textbf{2}  & \color{carrotorange}459 & \color{carrotorange}377 & \color{carrotorange}441 &  \color{carrotorange}356 &  \color{carrotorange}96 (405 nm) &  \color{carrotorange}<1 \\
\color{carrotorange}
\textbf{3}  & \color{carrotorange}457 & \color{carrotorange}377 & \color{carrotorange}399 &  \color{carrotorange}331 &  \color{carrotorange}66 (405 nm) & \color{carrotorange}<10 \\
\color{carrotorange}
\textbf{4}  & \color{carrotorange}463 & \color{carrotorange}373 & \color{carrotorange}445 &  \color{carrotorange}357 &  \color{carrotorange}94 (405 nm) &  \color{carrotorange}<1 \\
\color{caribbeangreen}
\textbf{5}  & \color{caribbeangreen}471 &\color{caribbeangreen} 381 & \color{caribbeangreen}450 &  \color{caribbeangreen}370 &  \color{caribbeangreen}68 (450 nm) &  \color{caribbeangreen}<1 \\
\color{caribbeangreen}
\textbf{6}  & \color{caribbeangreen}460 & \color{caribbeangreen}368 & \color{caribbeangreen}451 &  \color{caribbeangreen}360 &  \color{caribbeangreen}92 (405 nm) & \color{caribbeangreen}<30 \\
\color{carrotorange}
\textbf{7}  & \color{carrotorange}467 & \color{carrotorange}369 & \color{carrotorange}534 &  \color{carrotorange}\textit{n/a} & \color{carrotorange}\textit{n/a} & \color{carrotorange}\textit{n/a} \\
\color{caribbeangreen}
\textbf{8}  & \color{caribbeangreen}450 & \color{caribbeangreen}359 & \color{caribbeangreen}465 &  \color{caribbeangreen}376 &  \color{caribbeangreen}87 (405 nm) & \color{caribbeangreen}<10 \\
\color{caribbeangreen}
\textbf{9}  & \color{caribbeangreen}453 & \color{caribbeangreen}369 & \color{caribbeangreen}468 &  \color{caribbeangreen}399 &  \color{caribbeangreen}60 (450 nm) & \color{caribbeangreen}<10 \\
\color{caribbeangreen}
\textbf{10} & \color{caribbeangreen}453 & \color{caribbeangreen}363 & \color{caribbeangreen}471 &  \color{caribbeangreen}398 &  \color{caribbeangreen}15 (450 nm) &  \color{caribbeangreen}<1 \\
\color{caribbeangreen}
\textbf{11} & \color{caribbeangreen}453 & \color{caribbeangreen}360 & \color{caribbeangreen}452 &  \color{caribbeangreen}379 & \color{caribbeangreen}88 (405 nm)  &  \color{caribbeangreen}<1 \\
\bottomrule
\end{tabular}}
\label{tab_preds}
\end{table}

The correlation between the ML-predicted electronic absorption bands and the experimental measurements provided in \autoref{tab_preds} highlights the utility of the \textsc{mogp} model in identifying photoswitches with red-shifted and separated $\pi-\pi^*$ transitions. However, it should be noted that several photoswitches exhibit low PSS compositions of the metastable isomer at the irradiation wavelengths employed. Low PSS values of the \emph{Z} isomer may be attributed to overlap of broad electronic transitions for the isomeric forms. It is envisage that the composition of the \emph{Z} isomer at the PSS may be enhanced by expanding the curated dataset to consider the full-width-at-half-maximum (FWHM) of the electronic absorption bands. Moreover, the thermal half-lives of the photoswitches in \autoref{tab_preds} are short (less than 1 minute). This rapid thermal relaxation is to be expected for the push-pull type photoswitches the \textsc{mogp} predicted. Despite showing some potential applications for information transfer, it is envisaged that consideration of the thermal half-life properties would be beneficial for future work. Prediction of the thermal half-lives would enable further selectivity in the choice of photoswitch for a given application. It is anticipated that machine learning-based prediction, using the \textsc{mogp} model or otherwise, will be of use for synthetic photoswitch chemists who aim to design photoswitches with red-shifted absorption bands.

 \begin{figure*}[!htbp]
    \begin{center}
        \includegraphics[width=\textwidth]{Chapter1/Figs/UV_vis3.png}
    \end{center}
    \caption{The experimental UV-vis absorption spectrum of photoswitches \textbf{1}-\textbf{11} measured at 25 $\mu$M in DMSO and shown as the molar extinction coefficient (M$^{-1}$ cm$^{-1}$). Distinct irradiation wavelengths were chosen to predict the "pure" \emph{Z} spectra by applying the procedure detailed by \citet{Fischer1967} The chemical structures of these photoswitches are shown in Figure \ref{structures2}. Spectra generated by Jake Greenfield.}
    \label{UV_vis}
\end{figure*}

\section{Conclusions}\label{sec:conc}



This chapter introduces a data-driven prediction pipeline underpinned by dataset curation and multioutput Gaussian processes. It is demonstrated that a \textsc{mogp} model trained on a small curated azophotoswitch dataset can achieve comparable predictive accuracy to TD-DFT, and only slightly degraded performance relative to TD-DFT with a data-driven linear correction, in near-instantaneous time. The methodology is applied to discover several motifs that displayed separated electronic absorption bands of their isomers and which exhibit red-shifted absorption. The discovered motifs are hence suited for information transfer materials and photopharmacological applications. Sources of future work include the curation of a dataset of the thermal reversion barriers to improve the predictive capabilities of machine learning models as well as investigating how synthetic chemists may use model uncertainty estimates in the decision process to screen molecules e.g. via active learning \citep{2021_Mukadum} and Bayesian optimisation. The confidence-error curves in Appendix~\ref{conf_error} show initial promise in this direction and indeed understanding how best to tailor calibrated Bayesian models to molecular representations \citep{2020_flowmo, 2022_Gauche} is an avenue worthy of pursuit. The curated dataset and all code to train models is released under an MIT licence at \url{https://github.com/Ryan-Rhys/The-Photoswitch-Dataset}.

\nomenclature[Z-RF]{RF}{Random Forest}
\nomenclature[Z-GCN]{GCN}{Graph Convolutional Network}
\nomenclature[Z-GAT]{GAT}{Graph Attention Network}
\nomenclature[Z-MPNN]{MPNN}{Message-Passing Neural Network}
\nomenclature[Z-MOGP]{MOGP}{Multioutput Gaussian Process}
\nomenclature[Z-ICM]{ICM}{Intrinsic Coregionalisation Model}
\nomenclature[Z-MAE]{MAE}{Mean Absolute Error}
\nomenclature[Z-MSE]{MSE}{Mean Signed Error}