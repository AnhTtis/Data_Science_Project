
\begin{acknowledgements}      

I would like to thank all those who contributed in some part, indirect or otherwise, to my research productivity over the past years. I would like to thank Alpha Lee, my supervisor, firstly for giving me an opportunity to return from the working world to pursue research at a time when it seemed like all doors had been closed, and secondly, for striking a balance between giving me the freedom to explore new topics and offering excellent guidance and support when needed.\\
\indent In rough chronological order, I would like to thank Philippe Schwaller who introduced me to Alpha following a chance encounter on the streets of Cambridge in \textbf{April 2018}. My life may have been very different save for that meeting! Additionally, I would like to thank Philippe for his ongoing collaboration and sharing his expertise in all things involving sequence data and chemical reactions. \\
\indent From my time at Prowler.io (now Secondmind Labs) from \textbf{2017-2018}, I would like to thank Alexis Boukouvalas for acting as a fantastic mentor and supporting me in all my endeavours. I learned a great deal about both machine learning and software engineering during our pair programming sessions, especially when developing the code for adaptive sensor placement \citep{2019_Grant}. I would like to thank James Hensman, Richard Turner and Carl Rasmussen for giving lectures on Gaussian processes which sparked my interest in the topic. I would also like to thank Adithya Devraj for the numerous interesting conversations about machine learning and the philosophy of science. \\
\indent I would like to thank my colleagues from Cambridge Spark for keeping me up to speed with machine learning in industry from \textbf{2017-2022}. In particular, Raoul Gabriel-Urma, Petar Velickovic, Tim Hillel, Patrick Short, Catalina Cangea, Sahan Bulathwela, Ilyes Khemakhen, Chris Davis, Fred Hallgren and Kevin Lemagnen. Acting as a mentor for the Schmidt Data for Science Residency program was a highlight where I had the opportunity to learn about areas ranging from synthetic biology to geophysics and climate modelling. \\
\indent I would like to thank my colleagues from the Lee group at TCM from \textbf{2018-2022}, namely, Philip Verpoort, Alex Aldrick, Penelope Jones, Yunwei Zhang, William McCorkindale, Felix Faber, Alwin Bucher, Rhys Goodall, Janosh Riebesell, Rokas Elijosius, David Kovacs and Emma King-Smith for social interactions and academic discussions when I was not absent due to the global pandemic or internships. \\
\indent I would like to thank Bingqing Cheng, whom I first met in \textbf{2019} for involving me in her work on the ASAP library \citep{2020_Cheng}, and for taking the time to introduce me to a broad network of researchers applying machine learning to problems in physics and materials science. \\
\indent At the Institute of Astronomy I would like to thank Jiachen Jiang for introducing me to the world of high-energy astrophysics in the summer of \textbf{2019} and for providing excellent guidance on the contents of Chapter 3, namely modelling the multiwavelength variability of Mrk-335 using Gaussian processes \citep{2021_Mrk}. I would also like to thank Douglas Buisson, Dan Wilkins and Luigi Gallo for their feedback as well as Andy Fabian and Christopher Reynolds for being kind enough to attend an astrophysics talk given by a PhD student (myself) with no background in astrophysics! \\
\indent At the Computer Lab I would like to thank Ben Day, Simon Mathis and Arian Jamasb, whom I first met in \textbf{2020}, for discussions on graphs, molecules, proteins and antibodies. In particular, I would like to thank Arian for our almost daily Slack conversations and for answering my endless lists of questions! \\
\indent I would like to thank my colleagues at Huawei Noah's Ark Lab whom I began working with in \textbf{October 2020}. I would like to thank Haitham Bou-Ammar for his mentorship as well as Rasul Tutunov, Vincent Moens, Alexander Cowen-Rivers, Alexander Maraval, Antoine Grosnit and Hang Ren whom I learned a great deal from during our joint work \citep{2020_Grosnit, 2020_Rivers, 2021_Grosnit}. \\
\indent I would like to thank Anthony Bourached, George Cann, Gregory Kell and David Stork for their collaboration applying machine learning to artwork starting in late \textbf{2020} \citep{2021_Bourached_art, 2021_Cann, 2021_Stork, 2022_Kell}. In particular David has been an excellent mentor on the subject of research practices. \\
\indent I would like to thank Miguel Garcia-Ortegon, Vidhi Lalchand, James Wilson and Luke Corcoran for informative discussions on heteroscedastic Bayesian optimisation \citep{2021_Griffiths}, the topic of Chapter 6. I would like to thank Ajmal Aziz and Edward Kosasih at the Institute for Manufacturing for introducing me to supply chain logistics in the summer of \textbf{2021}, and in particular to applications of graph neural networks for supply chain problems \citep{2021_Aziz}. \\
\indent I would like to thank Jian Tang for supervising me at MILA from \textbf{January 2022} as well as Bojana Rankovic, Sang Truong, Leo Klarner, Aditya Ravuri, Yuanqi Du, Julius Schwartz, Austin Tripp, Alex Chan, Jacob Moss, Felix Opolka and Chengzhi Guo for their contributions to the GAUCHE library. \\
\indent I would like to thank Jake Greenfield for providing his photoswitch expertise and in particular, for helping out in a tight spot by rediscovering an old batch of lost molecules during a laboratory cleanup after the new batch had been mislaid by courier following a 3-month journey through customs. \\ 
\indent I would like to thank Henry Moss, whom I met virtually during the pandemic in the summer of \textbf{2020} and with whom I began developing a Gaussian process library for chemistry in the form of FlowMO \citep{2020_flowmo}. The evolved version, GAUCHE, comprises the contents of Chapter 4. I also appreciate the daily Slack discussions about all things to do with Bayesian optimisation and Gaussian processes. \\
\indent I would like to thank David Ginsbourger for hosting myself and Henry Moss in Bern in \textbf{April 2022} to discuss extensions of the ideas comprising Chapters 4 and 5 with Athénaïs Gautier and Anna Broccard. I would like to thank Ekansh Verma and Souradip Chakraborty for involving me in their work on invariances in Bayesian optimisation \citep{2021_Verma} in the summer of \textbf{2022}. I would like to thank my long-time friend S. F. for rigorously inspecting the notation of the final thesis in \textbf{July 2022}. I would like to thank Victor Prokhorov for always providing interesting food for thought during our many conversations in Cambridge. \\
\indent On a personal note, I would like to thank Leandro Charanga and Monika Jankauskaite for teaching me how to dance, Subhankar, Thomas C and Dan for their advice on ethical dilemmas, Thomas M for discussions on mathematics and Bachata, Teja for trying to teach me some gymnastics and my parents for their ongoing support.

\end{acknowledgements}

