\chapter{Modelling Experimental Noise with Gaussian Processes} 

\section{Heteroscedasticity of the Soil Phosphorus Fraction Dataset}
\label{soil_het_demo}


\autoref{the_table} is used to demonstrate the efficacy of modelling the soil phosphorus fraction dataset using a heteroscedastic \textsc{gp}. The heteroscedastic \textsc{gp} outperforms the homoscedastic \textsc{gp} on prediction based on the metric of negative log predictive density (NLPD)

\begin{equation*}
    \text{NLPD} = \frac{1}{n} \sum_{i=1}^n - \log p(t_i | \boldsymbol{x_i}),
\end{equation*}

\noindent which penalises both over and under-confident predictions.

\begin{table}[ht]
\centering
\caption{Comparison of NLPD values on the soil phosphorus fraction dataset. Standard errors are reported for 10 independent train/test splits. Lower scores are better.}
\label{the_table}
\vspace{3mm}
\resizebox{0.75\textwidth}{!}{%
\begin{tabular}{@{}lll@{}}
\toprule
\textbf{Soil Phosphorus Fraction Dataset} & \multicolumn{1}{c}{\textbf{GP}} & \multicolumn{1}{c}{\textbf{Het GP}} \\ \midrule
NLPD & $1.35 \pm 1.33$ & $1.00 \pm 0.95$ \\ \bottomrule
\end{tabular}%
}
\end{table}

\section{Additional Ablation Experiments}\label{more_ablation}

In this section the ablation results are presented on noiseless, homoscedastic and heteroscedastic noise tasks in line with \autoref{first_ablation} of the main thesis.

\subsection{Goldstein-Price Function}

The form of the Goldstein-Price function is the same as in the main thesis with noise function in \autoref{eq_g-p_noise}. The function is visualised in \autoref{fig:g-p_function}. 9 data points are used for initialisation in the noiseless and homoscedastic noise cases whereas 100 data points are used for initialisation in the heteroscedastic noise case. $\beta$ is set to 0.5 for the noiseless and homoscedastic noise tasks and $\frac{1}{11}$ for the heteroscedastic noise task. $\gamma$ is set to 500 for all experiments.

\begin{figure*}
\centering
\subfigure[Latent Function]{\label{fig:gp_1}\includegraphics[width=0.32\textwidth]{Chapter2/Figs/goldstein/goldstein_function.png}}
\subfigure[Noise Function ]{\label{fig:gp_2}\includegraphics[width=0.32\textwidth]{Chapter2/Figs/goldstein/goldstein_noise_function.png}}
\subfigure[Objective Function]{\label{fig:gp_3}\includegraphics[width=0.32\textwidth]{Chapter2/Figs/goldstein/goldstein_composite_function.png}} 
\caption{(a) The latent Goldstein-Price Function $f(\mathbf{x})$ together with (b) its heteroscedastic noise function $g(\mathbf{x})$ and (c) the objective function $f(\mathbf{x}) + g(\mathbf{x})$.}
\label{fig:g-p_function}
\end{figure*}

\subsubsection{Noiseless Case}

The results of the noiseless case for Goldstein-Price are given in \autoref{noiseless_g-p}. All \textsc{bo} methods outperform random search with ANPEI best and HAEI second best.

\begin{figure*}[t]
\centering
    \includegraphics[width=.7\textwidth]{Chapter2/Figs/bayesopt_plot10_iters_noiseless_.png}
    \caption{Goldstein-Price function noiseless case. All \textsc{bo} methods outperform random search. ANPEI performs best and HAEI is runner-up.}
    \label{noiseless_g-p}
\end{figure*}

\pagestyle{fancy}
\fancyhf{}
\lhead{Appendix}
\rhead{\thepage}

\subsubsection{Homoscedastic Noise Case}

The results of the homoscedastic noise case for Goldstein-Price are shown in \autoref{homo_g-p}. In this instance HAEI performs best.

\begin{figure*}[t]
\centering
    \includegraphics[width=.7\textwidth]{Chapter2/Figs/bayesopt_plot10_iters_with_noise_.png}
    \caption{Goldstein-Price function homoscedastic noise case. HAEI performs best.}
    \label{homo_g-p}
\end{figure*}

\subsubsection{Heteroscedastic Noise}

The results of the heteroscedastic noise case for Goldstein-Price are shown in \autoref{fig:g-p_hetero}. ANPEI performs best whilst HAEI performs worse than random search.

\begin{figure*}
\centering
\subfigure[Best Objective Value Found so Far]{\label{fig:bo_g-p}\includegraphics[width=0.49\textwidth]{Chapter2/Figs/bayesopt_plot10_iters_heteroscedastic.png}}
\subfigure[Lowest Aleatoric Noise Found so Far ]{\label{fig:bo_2_g-p}\includegraphics[width=0.49\textwidth]{Chapter2/Figs/heteroscedastic_bayesopt_plot10_itersnoise_only.png}}
\caption{Comparison of heteroscedastic and homoscedastic \textsc{bo} on the heteroscedastic 2D Goldstein-Price function. (a) shows the optimisation of $h(\boldsymbol{x}) = f(\boldsymbol{x}) + g(\boldsymbol{x})$ (lower is better) where $g(\boldsymbol{x})$ is the aleatoric noise. (b) shows the values $g(\boldsymbol{x})$ obtained over the course of the optimisation of $h(\boldsymbol{x})$.}
\label{fig:g-p_hetero}
\end{figure*}


\subsection{Branin-Hoo Function}

The form of the Branin-Hoo function is given in \autoref{branin_eq} with noise function in \autoref{branin_noise_eq}. The function is visualised in \autoref{fig:double_bran}, a figure from the main thesis repeated here for clarity. 9 data points are used for initialisation in the noiseless and homoscedastic noise cases whereas 100 data points are used for initialisation in the heteroscedastic noise case. $\beta$ is set to 0.5 and $\gamma$ is set to 500 for all experiments.

\begin{figure*}
\centering
\subfigure[Latent Function]{\label{fig:double_bran_1}\includegraphics[width=0.32\textwidth]{Chapter2/Figs/branin/branin_function.png}}
\subfigure[Noise Function ]{\label{fig:double_bran_2}\includegraphics[width=0.32\textwidth]{Chapter2/Figs/branin/branin_noise_function.png}}
\subfigure[Objective Function]{\label{fig:double_bran_3}\includegraphics[width=0.32\textwidth]{Chapter2/Figs/branin/branin_composite_function.png}} 
\caption{Heteroscedastic Branin Function.}
\label{fig:double_bran}
\end{figure*}

\subsubsection{Noiseless Case}

The results of the noiseless case for the Branin-Hoo function are given in \autoref{noiseless_double_branin}. HAEI performs best in this case whereas ANPEI performs worst.

\begin{figure*}[t]
\centering
    \includegraphics[width=.7\textwidth]{Chapter2/Figs/branin_noiseless_rbf.png}
    \caption{Branin-Hoo function noiseless case. HAEI performs best. ANPEI performs worst.}
    \label{noiseless_double_branin}
\end{figure*}

\subsubsection{Homoscedastic Noise Case}

The results of the homoscedastic noise case for the Branin-Hoo function are given in \autoref{homoscedastic_braninhoo}. All \textsc{bo} methods outperform random search yet perform comparably against each other.

\begin{figure*}[t]
\centering
    \includegraphics[width=.7\textwidth]{Chapter2/Figs/branin_homoscedastic_noise_5_rbf.png}
    \caption{Branin-Hoo function homoscedastic noise case. All \textsc{bo} methods outperform random search.}
    \label{homoscedastic_braninhoo}
\end{figure*}

\subsubsection{Heteroscedastic Noise}

The results of the heteroscedastic noise case for the Branin-Hoo function are shown in \autoref{fig:bran_hetero2}. ANPEI performs best whilst HAEI performs worse than random search.

\begin{figure*}
\centering
\subfigure[Best Objective Value Found so Far]{\label{fig:bo_bran2}\includegraphics[width=0.495\textwidth]{Chapter2/Figs/branin_objective_hetero_500_weight_haei_rbf.png}}
\subfigure[Lowest Aleatoric Noise Found so Far ]{\label{fig:bo_2_bran2}\includegraphics[width=0.493\textwidth]{Chapter2/Figs/branin_noise_hetero_500_weight_haei_rbf.png}}
\caption{Comparison of heteroscedastic and homoscedastic \textsc{bo} on the heteroscedastic 2D Branin function. (a) shows the optimisation of $h(\boldsymbol{x}) = f(\boldsymbol{x}) + g(\boldsymbol{x})$ (lower is better) where $g(\boldsymbol{x})$ is the aleatoric noise. (b) shows the values $g(\boldsymbol{x})$ obtained over the course of the optimisation of $h(\boldsymbol{x})$.}
\label{fig:bran_hetero2}
\end{figure*}

\section{Performance Impact of the Kernel Choice}\label{kernel_exps}

In this section the impact that the choice of \textsc{gp} kernel has on \textsc{bo} performance is analysed. Three kernels are selected for this purpose: the RBF kernel

\begin{equation*}
    k_{\text{RBF}}(\boldsymbol{x}, \boldsymbol{x'}) = \sigma_{f}^{2}\cdot\text{exp}\Big(\frac{-\lVert\boldsymbol{x} - \boldsymbol{x'}\rVert^{2}}{2\ell^{2}}\Big),
\end{equation*}

\noindent used for all experiments in the main thesis, the exponential kernel (Exp)

\begin{equation*}
    k_{\text{exp}}(\boldsymbol{x}, \boldsymbol{x'}) = \sigma_{f}^{2}\cdot\text{exp}\Big(\frac{-\lVert\boldsymbol{x} - \boldsymbol{x'}\rVert}{\ell}\Big),
\end{equation*}

\noindent a special instance of the Mat\'{e}rn kernel for values of $\nu = \frac{1}{2}$ \citep{2006_Rasmussen}, as well as the Mat\'{e}rn 5/2 kernel

\begin{equation*}
    k_{\text{Mat\'{e}rn}(5/2)}(\boldsymbol{x}, \boldsymbol{x'}) = \sigma_{f}^{2}\cdot \Big(1 + \frac{\sqrt{5} \lVert\boldsymbol{x} - \boldsymbol{x'}\rVert}{\ell} + \frac{5 \lVert\boldsymbol{x} - \boldsymbol{x'}\rVert^2}{3\ell^2}\Big) \cdot \text{exp}\Big(\frac{-\sqrt{5} \lVert\boldsymbol{x} - \boldsymbol{x'}\rVert}{\ell}\Big),
\end{equation*}

\noindent which is one of the most popular kernels for large-scale empirical studies \citep{2018_Wilson, 2020_Grosnit}. It should be noted that while the equations are written assuming a single scalar lengthscale, in practice for the experiments in greater than 1D, each lengthscale is optimised per dimension under the marginal likelihood. For all experiments the same kernel is chosen for both \textsc{gp}s of the \textsc{mlhgp} model i.e. the \textsc{gp} modelling the objective as well as the \textsc{gp} modelling the noise. 100 points are used for initialisation in the Branin-Hoo and Goldstein-Price functions and 144 points are used for the Hosaki function. $\beta$ is set to 0.5 for the Branin-Hoo and Hosaki functions and $\frac{1}{11}$ for the Goldstein-Price function. $\gamma$ is set to 500 for all experiments. The results are shown in \autoref{fig:bran_kernel}, \autoref{fig:gold_kernel} and \autoref{fig:hos_kernel} for the Branin-Hoo function, Goldstein-Price function and Hosaki functions respectively. There is no significant difference in performance using each kernel save for the Branin-Hoo function where ANPEI underperforms using the somewhat rougher exponential kernel.

\begin{figure*}
\centering
\subfigure[ANPEI]{\label{fig:bo_brank}\includegraphics[width=0.495\textwidth]{Chapter2/Figs/kernel/bayesopt_plot10_iters_kernel_test_anpei_kernel_branin.png}}
\subfigure[HAEI]{\label{fig:bo_2_brank}\includegraphics[width=0.495\textwidth]{Chapter2/Figs/kernel/bayesopt_plot10_iters_kernel_test_haei_kernel_branin.png}}
\caption{Branin-Hoo function kernel comparison.}
\label{fig:bran_kernel}
\end{figure*}

\begin{figure*}
\centering
\subfigure[ANPEI]{\label{fig:bo_g-p}\includegraphics[width=0.486\textwidth]{Chapter2/Figs/kernel/bayesopt_plot10_iters_kernel_test_anpei_kernel_goldstein.png}}
\subfigure[HAEI]{\label{fig:bo_2_g-p}\includegraphics[width=0.495\textwidth]{Chapter2/Figs/kernel/bayesopt_plot10_iters_kernel_test_haei_kernel_goldstein.png}}
\caption{Goldstein-Price function kernel comparison.}
\label{fig:gold_kernel}
\end{figure*}

\begin{figure*}
\centering
\subfigure[ANPEI]{\label{fig:bo_hosk}\includegraphics[width=0.495\textwidth]{Chapter2/Figs/kernel/bayesopt_plot10_iters_kernel_test_anpei_kernel_hosaki.png}}
\subfigure[HAEI]{\label{fig:bo_2_hosk}\includegraphics[width=0.495\textwidth]{Chapter2/Figs/kernel/bayesopt_plot10_iters_kernel_test_haei_kernel_hosaki.png}}
\caption{Hosaki function kernel comparison.}
\label{fig:hos_kernel}
\end{figure*}
