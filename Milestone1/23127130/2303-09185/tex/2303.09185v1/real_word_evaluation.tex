\section{real world evaluation}
\label{deployment}

In order to measure the effectiveness of the six algorithms introduced in Section~\ref{relatedwork} and NLLS with real sensor network data and to be able to compare the results with the executed simulations, we recorded the data of a series of different test runs. The experiments were carried out using a modified version of the Modular Sensor Board (MSB) A2 \citep{TR-B-08-15} node which is equipped with a Nanotron nanoPAN 5375 \citep{nanopan} transceiver. This hardware enables the sensor nodes to measure inter-node ranges using \ac{TOF} in the 2.4 GHz frequency band. The experiments took place on the second floor of our Computer Science Department during daytime.

Figure \ref{fig:real_world_run2} shows one exemplary campaign of measurements following a route among offices, laboratories and with a few people walking around. For the reason of clarity, we plotted only the results of Min-Max and \ac{MD-Min-Max} using a Kalman filter. The starting point is denoted by ``S'', the endpoint is denoted by ``E'' and the total length of the path was about 100 meters. In each run, we used 17 anchors which were deployed throughout the building. Most of the anchors were placed in office rooms with doors closed. Only a small fraction of nodes was placed on the hallway, in case of Figure~\ref{fig:real_world_run2}, there were four nodes. Ground truth was measured with the aid of a robot system developed at our Department using a Microsoft Kinect. This reference system provides about 10~cm positioning accuracy. The robot carried the unlocalised node and followed a predefined path with a predefined speed. We used the maximum movement speed of the robot, which is 0.5~m/s. In total, we performed over 5300 localisations when adding up all test runs. The nanoPAN achieves ranging precision of around 2.85 m in average and the RMSE is 4.32~m. However, the ranging error can be as large as 20~m. We even encountered measurement errors up to 75~m in rare cases.

The quantitative results of the seven localisation algorithms are shown in Table \ref{table_loc_results}. The average anchor degree throughout all experiments was 7.48. Additionally, Table \ref{table_loc_results} contains the results of multilateration using NLLS to give a comparison to a well known general purpose algorithm. As it can be seen, \ac{MD-Min-Max} outperforms the other algorithms in terms of localisation accuracy with achieving an average error of 1.63~m. The basic Min-Max algorithm (2.05~m) is still more than twice as good as NLLS (4.43~m) which serves as a reference algorithm. The good performance of Min-Max (and therefore also the other Min-Max algorithms) is not surprising because the inter-anchor distances were relative short (between 5 and 10 meters) and the mobile node took mainly positions within the bounds of the network. As we know from Section \ref{section_spatial_evaluation} this is the optimal situation for Min-Max algorithm. This fact is also observed by \citet{savvides2002} and proved by \citet{langendoen2003}. All three enhanced Min-Max algorithms outperform the original one: \ac{E-Min-Max} (W2) (1.46\%), \ac{E-Min-Max} (W4) (4.39\%) and \ac{MD-Min-Max} (20.48\%). Furthermore, all Min-Max based algorithms show quite small maximum positioning error as they bound the estimate inside the \ac{IR}.

To make NLLS, MLE-$\mathcal{N}$ and MLE-$\Gamma$ comparable, they all use the same three initial starting points for their optimization procedure. MLE-$\mathcal{N}$ outperforms NLLS by 52.82\% and MLE-$\Gamma$ even by 56.43\%. NLLS neither considers the real distribution of the error nor bounds the estimate and thus can be easily misled by the \ac{NLOS} error. Both MLE algorithms perform significantly better due to the consideration of the positive bias of the real error. MLE-$\Gamma$ achieves even a better average error (1.93~m) than the best \ac{E-Min-Max} algorithm (1.96~m). However, the outstanding results shown in Section~\ref{section_spatial_evaluation} cannot be reproduced in real world experiments. The reason for this lies in the error model of the simulation. Even if the error model of the simulation and the measured error of the real world experiment nearly have the same distribution, the origin of this distribution is quite different.

In a real world experiment, the occurrence of a certain error is twofold: while the measurement error part could be described as random noise with a certain distribution, the reflection and multipath error part is not that random. This path error is linked to the position in the building and can be characterised by the real position and the properties of the physical structure of the building. If the experiment is a result of a walk through the building none of two parameters could be characterized as random, even not if we use a random walk movement model. If we are at a certain position $P_n$ the next measured position $P_{n+1}$ depends on $P_n$ if the movement speed is limited. The physical structure also directly depends on the position. The error model of the simulation does not take this into account and realizes the error distribution in a total random order without any position dependencies. This results in a very different behaviour from simulation and real world 
experiments. To overcome this, ray-tracing error models and structure maps can be used. Alternatively, if we adopt the algorithm to the dependencies in the distribution, we would end up in zero position error and hard-coding the whole experiment with its exact path into the algorithm, which would be a useless approach of course.

The fact that the RMSE of Min-Max, \ac{E-Min-Max} (W2), \ac{E-Min-Max} (W4), \ac{MD-Min-Max}, and both \ac{ML} estimators is much smaller than the RMSE of the distance measurements tells us that these algorithms performed very well relative to the quality of the distance measurements available. NLLS with having a RMSE only slightly larger than the RMSE of the distance measurements showed acceptable performance. The distribution of the localisation error of all algorithms is shown in Figure~\ref{fig:real_world_histogram_boxplot} where the vertical axis is the localisation error in meters and the horizontal axis is the corresponding algorithm. NLLS shows poor performance compared to the other algorithms. Also the RMSE is much larger than that of the other algorithms. \ac{MD-Min-Max} has the smallest spread among all algorithms when regarding only non-outliers and also the lowest median of the error. Furthermore, the interquartile
range of \ac{MD-Min-Max} is the smallest among all algorithms. This algorithm outperforms even \ac{E-Min-Max} (W4) by more than 16\%. This performance gain is mainly achieved by adjusting the parameters of the algorithm to the error distribution (see Figure~\ref{fig:histogram_ranging_and_fitting}) of the used distance measurement hardware as described in Section~\ref{section:md-min-max}. Only the \ac{ML} estimators achieve a comparable performance gain but they suffer from many large outliers compared to the Min-Max algorithms. \ac{E-Min-Max} (W2) and \ac{E-Min-Max} (W4) show nearly the same performance. \ac{E-Min-Max} (W4) is slightly better because its weighting function is optimized for locations inside the perimeter of the anchors as was mostly the case.

Note, that \ac{MD-Min-Max} is quite sensitive to the parameters of the membership function. When assuming a Gaussian error distribution on our statistical data and using the three-sigma rule, the membership function is characterized by $[-8.3; 2.4; 13.1]$. With this function, the average error regresses to 1.89 meters. \ac{MD-Min-Max} can even become the worst algorithm, when the membership function does not fit the data. For example, choosing $[6; 12; 18]$ for the membership function will result in an average error of 2.19 meters. A careful analysis of the statistical data is necessary for good results. The membership function is characteristic to a deployment, e.g. a building or one of its floors, and results are of similar quality for multiple runs in such a deployment.

Obviously, the position accuracy could be improved using some filtering techniques, such as Kalman or particle filters, but the aim of this paper is to show and compare the performance of the used localisation algorithms without using any of those filtering techniques.
