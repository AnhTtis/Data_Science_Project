\section{Distribution of the Spatial Position Error}
\label{section_spatial_evaluation}

The \emph{spatial distribution of the positioning error}, or shorter, the \emph{spatial error distribution}, measures the expected positioning error at each location in space. It characterises the performance of the algorithm for a specific anchor set-up and shows where an algorithm is expected to perform well and where it does not perform.

To evaluate the spatial distribution of the positioning error, we executed every algorithm on each position of a $1000\times 1000$ unit sized grid $1000$ times in our LS\textsuperscript{2} simulation engine \citep{ls2,will2012ls2}. LS\textsuperscript{2} calculates the position error for every discrete point in the simulated area using an error model and an algorithm selectable by the user. In the first scenario we chose a very basic anchor set-up with four anchors placed in the four corners of the playing field. The inter anchor distance is much higher than in most real world scenarios and shows the performance of the evaluated algorithms in borderline situations. The resulting image consists of up to three differently coloured areas. The grey area indicates a position error between 100\% and 500\% of the expected distance measurement error value; the darker the area, the higher is the error. The green area (if present) indicates a position error lower than the expected distance measurement error; the darker the area, the lower is the error. In the blue area the error 
is 
higher than 500\% of the distance error and is cropped to achieve a better image contrast. The anchors are represented by the small red squares.

The green area is very important for cooperative localisation strategies in WSNs, because the position error stays in a reasonable range as long as the node remains in the green area. Otherwise the position error tends to grow much faster than expected because for each step of the recursive cooperation strategy, the resulting position error is added to the average distance error. If the resulting position error is larger than the average ranging error, it grows very fast. 

For this simulation we chose a Gaussian distributed error for the general noise simulation and an exponential distributed error to simulate \ac{NLOS} situations. The expected value of the distance measurement error is 5\% of the playing field width, the standard deviation is 1.5\%. A \ac{NLOS} error occurs with a probability of 10\% and adds an exponential error with rate $2$. The membership function of the MD-Min-Max was set up like described in \ref{membership_function}. The inter-anchor distance is 15 times higher than the expected distance error.

We show the results of the first simulation run in Figure \ref{fig:data4}. The weaknesses of Min-Max are clearly visible. Min-Max performs very well only on the diagonal lines between the anchors and in the centre of the playing field. For similar setups in real world deployments Min-Max's performance is not really predictable because a mobile node will cross all areas. The \ac{E-Min-Max} (W2) algorithm performs slightly better in this setup but shows the same strengths and weaknesses. \ac{E-Min-Max} (W4) performs completely different in this scenario and shows a very homogeneous picture. It shows a slight performance drop close around the anchors but provides very good results for the rest of the area.  \ac{MD-Min-Max}'s results are comparable to Min-Max but with a slightly bigger area of high accuracy. Even if \ac{MD-Min-Max} has the highest accuracy inside the green area of all four algorithms one should choose \ac{E-Min-Max} (W4) for a random walk in such scenarios.

The second simulation is shown in Figure \ref{fig:data1}. In this scenario we simulated every algorithm with a uniform grid layout for the anchors. We chose nine anchors which convex hull covers 4\% of the simulation area. The inter-anchor distance is comparable to common indoor deployments. The focus in this scenario is to evaluate how the algorithms will perform outside the convex hull of the anchors. The main strengths and the main weaknesses of Min-Max are clearly visible in this image. Min-Max performs very good inside the convex hull of a dense anchor setup and fast lowers its performance outside the convex hull down to unusable values. The main design goal of \ac{E-Min-Max} was to dilute this behaviour of Min-Max. As shown in Figure \ref{fig:subfig_1eminmaxw4} \ac{E-Min-Max} (W4) greatly improves the performance of Min-Max outside the convex hull without lowering the performance inside very much. \ac{E-Min-Max} (W2) stretches the usable area even a bit more but has some disadvantages in areas where Min-
Max performed well. Even if the average error over the whole playing field is nearly the same for both \ac{E-Min-Max} algorithms one could gain a noticeable advantage over the other if closer limitations to the area can be made in real world deployments. \ac{MD-Min-Max} clearly shows its advantages and disadvantages in this scenario. The area of high accuracy is only slightly increased and it also shows a fast performance drop outside the diagonals of the anchor hull, but the results inside this area are much more accurate than those of the Min-Max algorithm. For real world indoor deployments this observation can be important because the anchors are usually wall mounted and because of this, a mobile node rarely leaves the anchor hull.

In Figure \ref{fig:data3} the results of a more challenging scenario are shown. We placed four anchors nearly on a line and a fifth anchor to form a flat triangle with the rest. For most lateration algorithms this scenario is a kind of worst case scenario and the performance is weaker than the average performance of real world experiments because the overall number of anchors is low and the average inter-anchor distance is on a medium level. Min-Max has strong performance drops even inside the convex hull and then drops very fast to unusable values. \ac{E-Min-Max}~(W4) noticeably increases the performance and provides very good results for a centre area that covers ~30\% of the whole simulation area. \ac{E-Min-Max}~(W2) increases the average performance again but the results are very heterogeneous, so it could be challenging to make use of this performance gain in real world usage. \ac{MD-Min-Max} shows a comparable but much smaller shape than \ac{E-Min-Max}~(W4) but the accuracy inside this shape is much 
higher.

To highlight the difference of the average performance shown in Figure~\ref{fig:data3} between those algorithms, we visualize the difference of average errors between two algorithms in Figure~\ref{fig:diff}. Areas coloured in shades of red are areas in which the first mentioned algorithm achieves a lower average position error than the second algorithm. Areas coloured in shades of blue to white indicate areas in which the second algorithm achieves a lower position error. Areas coloured in green mark the areas in which both algorithms perform within $1.6\%$ of the playing field, i.e.\ their position error can be considered to be equivalent.

Figures~\ref{fig:diff-ew2-mm},~\ref{fig:diff-ew4-mm} and~\ref{fig:diff-abs-mm} show that the \ac{E-Min-Max} algorithms and our \ac{MD-Min-Max} algorithm all improve on Min-Max, especially outside of the area in which Min-Max performs best. The \ac{MD-Min-Max} algorithm is able to maintain the good performance of Min-Max in its strongest area and shows its weaknesses in areas outside of the convex hull of the anchors. Figure~\ref{fig:diff-ew2-ew4} compares \ac{E-Min-Max}~(W2) to \ac{E-Min-Max}~(W4) and shows that both can complement each other well. In the inner, blue tinted area, \ac{E-Min-Max}~(W4) compares much better while outside of that area, \ac{E-Min-Max}~(W2) performs better. Interestingly, their performance is comparable in the convex hull of the anchors, and thus worse than the original Min-Max algorithm. Figures~\ref{fig:diff-abs-ew2} and~\ref{fig:diff-abs-ew4} compare our \ac{MD-Min-Max} algorithm to \ac{E-Min-Max}~(W2) and (W4). Outside of the convex hull of the anchors, the \ac{E-Min-Max} algorithms perform much better than \ac{MD-Min-Max} however, in the centre area, performance is comparable or \ac{MD-Min-Max} is able to reduce the position error significantly. These areas, however, are of interest in many indoor deployments, where the mobile node is usually inside of the hull of anchors.

Figure~\ref{fig:mle} shows outstanding simulation results for the MLE-$\Gamma$ algorithm. Unless otherwise noted, Figures~\ref{fig:mle91}-\ref{fig:mle44} use the same simulation parameters than in the preceding simulations. The simulation results are hardly to distinguish and show a very homogeneous spatial distribution for the whole simulation area. Even if the probability for the occurrence of NLOS errors is raised to 40\% (as shown in Figure~\ref{fig:mle44}), the average result and the spatial distribution of the position error is not changing very much. For the MLE-$\mathcal{N}$ algorithm the simulation results are quite similar to the MLE-$\Gamma$ results, only a little weaker and because of that not illustrated here.