\section{related work}
\label{relatedwork}

Several measurement techniques are used to track the positions for indoor systems~\citep{patwari2005locating,bahl2000radar,bahl2000enhancements,ranta2011accu}. Range based methods which measure the distance or range value between the target and anchor sensors are common and efficient tools, for instance, received signal strength (RSS) in RADAR system~\citep{bahl2000radar,bahl2000enhancements}, time-of-arrival (TOA) and its improved metrics: time-difference-of-arrival (TDOA)~\citep{patwari2005locating} and \ac{TOF}~\citep{bulusu2000gps}. \ac{TOF} measures the round-trip time of packet and averages the result together to reduce the impact of time-varying errors. It is a promising solution for its low cost and feasible for the capacity of real-time application.

Range based location algorithms are designed to reduce range errors such as the complicated indoor multi-path propagations, low signal-to-noise ratio (SNR), severe multi-path effects, reflection and link failures and improve the estimation accuracy~\citep{whitehouse2005effects,whitehouse2007practical,guvenc2007analysis,venkatraman2004novel,Venkatesh2006}. These algorithms include iterative methods, which use gradient descent or Newton method to calculate an estimated position. Grid-scan methods~\citep{lazos2005serloc,srinivasan2007survey} divide the target field into several cells and are using voting based methods to select a cell as an estimated position. Refined geometry relationship~\citep{venkatraman2004novel,moore2004robust} obtains the target relative position rather than actual position, and the method is still based on the range based measurements, in which the measurement noise still causes estimation errors. Least squares (LS) method~\citep{guvenc2007analysis,Venkatesh2006} can be classified into linear least squares (LLS) algorithm and nonlinear least squares (NLLS) algorithm. LS is a common and accurate way for localisation, however, the achieved solution is suboptimal in case the estimated distances contain outlier errors~\citep{beck2008exact}. Optimal range selection~\citep{kaplan2006global,li2010low} directly reduces the range error by adapting the range measurement and choosing effective anchors.

Most of the common algorithms do not perform very well in indoor scenarios. Indoor scenarios are commonly classified by a large number of anchors with a short inter-anchor distance. The error on the distance measurements is often biased for a subset of the overall anchor configuration due to multi-path effects and reflections of the received signals. The Min-Max~\citep{simic01:_distr} algorithm is an effective and simple method for localisation. Experiments show, that the Min-Max method performs very well in short-range scenarios~\citep{zanca08:_exper_rssi}.

