\pdfoutput=1

\documentclass[10pt]{article}
%\documentclass[11pt]{article}

\makeatletter
%%%%%%
\renewcommand{\maketitle}{\bgroup\setlength{\parindent}{0pt}
\begin{flushleft}
  \textbf{\@title}

  \@author
\end{flushleft}\egroup
} %%%%%
\def\footnoterule{\kern-3\p@
  \hrule \@width 2in \kern 2.6\p@} % the \hrule is .4pt high
  %%%%%%
\makeatother
\renewcommand{\thefootnote}{\fnsymbol{footnote}}

\usepackage{preamble}
\usepackage{courier}
\usepackage[square,numbers]{natbib}
%\usepackage{preamble}

\title{{\Large \textbf{On f-generic types}}}
%\title{{\large \textbf{A note on product-free sets in distal groups}}}
%\author{Atticus Stonestrom\footnote{Email: \texttt{atticusstonestrom@yahoo.com}}}
\author{Atticus Stonestrom\ \ \ \ \ \ \ \ \ \ \ \ \ \ \ \ \ \  {\footnotesize (Email: \texttt{atticusstonestrom@yahoo.com})}}
\date{}

\begin{document}
\maketitle

{\small \noindent\textbf{Abstract:} Let $G$ be a group definable in an NIP theory. We prove that, if $G$ admits a global f-generic type, then $G$ is definably amenable, answering a question of Chernikov and Simon. \newline %\textcolor{blue}{We prove moreover that such a type is never concentrated on a cut of an indiscernible sequence.} \newline

\noindent\textbf{Acknowledgements:} I would like to thank Artem Chernikov, Itay Kaplan, and Anand Pillay for all of their encouragement and feedback. \newline

\noindent\textbf{Notation:} Throughout $T$ will denote a complete theory in a first-order language $L$, with a monster model $\mathfrak{C}$. $G$ will denote a definable group in $T$, and for any model $M\preccurlyeq\mathfrak{C}$ we will write $G(M)$ for the realizations of $G$ whose coordinates lie in $M$. For a parameter set $C\subseteq\mathfrak{C}$, we will use $S_G(C)$ to denote the space of all complete types over $C$ that are concentrated on $G$. For tuples $a,b$, and a small parameter set $C\subset\mathfrak{C}$, we will write $a\equiv_C b$ to mean $\tp(a/C)=\tp(b/C)$. Finally, we will assume throughout the entire paper that $T$ is NIP.

\section{Introduction} Recall that $G$ is said to be `definably amenable' if it admits a translation-invariant Keisler measure. The study of definably amenable groups was instigated in the paper \citep{hrushovski_peterzil_pillay}, and we refer the reader to the introduction of \citep{chernikov_simon} or to the survey \citep{pillay_groups} for general history and motivation for the notion. Here we will focus on only one strand of the study of definably amenable groups, which originates in the paper \citep{hrushovski_pillay}. Hrushovski and Pillay showed there that, for $T$ countable and NIP, definable amenability of $G$ is equivalent to the existence of a \textit{strongly f-generic}\footnote[1]{In \citep{hrushovski_pillay} such a type is just called f-generic, but the notion was renamed in \citep{chernikov_simon}.} type: a global type $p(x)\in S_G(\mathfrak{C})$ such that, for some small model $M\prec\mathfrak{C}$, the translate $gp(x)$ does not fork over $M$ for all $g\in G(\mathfrak{C})$. In the paper \citep{chernikov_simon}, Chernikov and Simon introduced the weaker notion of an \textit{f-generic} type (over some parameter set $C\subseteq\mathfrak{C}$): a type $p(x)\in S_G(C)$ such that, for every formula $\theta(x,c)$ in $p$, there is some small model $M\prec\mathfrak{C}$ such that the translate $g\theta(x,c)$ does not fork over $M$ for all $g\in G(\mathfrak{C})$. Every strongly f-generic type is f-generic, but a global f-generic type need not be strongly f-generic; see Example 3.10 in \citep{chernikov_simon}. Thus a natural question, in analogy with Hrushovski and Pillay's above-mentioned theorem, is whether, for $T$ NIP, definable amenability of $G$ is equivalent to the existence of a global f-generic type; this is Question 3.18 in Chernikov and Simon's paper. In this paper we give a positive answer.
%Recall that $G$ is said to be definably amenable if there is a global Keisler measure concentrated on $G$ that is invariant under left translation by elements of $G$. This notion was introduced in the paper \citep{hrushovski_peterzil_pillay}, and we refer the reader to the introduction of \citep{chernikov_simon} for general history and motivation for the notion. Here we will focus on only one strand of the study of definably amenable groups, which originates in the paper \citep{hrushovski_pillay}. Hrushovski and Pillay showed there that, for $T$ NIP, definable amenability of $G$ is equivalent to the existence of a \textit{strongly f-generic}\footnote[1]{In \citep{hrushovski_pillay} such a type is just called f-generic, but the notion was renamed in \citep{chernikov_simon}.} type: a global type $p(x)\in S_G(\mathfrak{C})$ such that, for some small model $M\prec\mathfrak{C}$, the translate $gp(x)$ does not fork over $M$ for all $g\in G(\mathfrak{C})$. In the paper \citep{chernikov_simon}, which resolved a number of conjectures on definably amenable NIP groups, Chernikov and Simon introduced the weaker notion of an \textit{f-generic} type: a type $p(x)\in S_G(C)$, over some parameters $C\subseteq\mathfrak{C}$, such that, for every formula $\theta(x,c)$ in $p$, there is some small model $M\prec\mathfrak{C}$ such that the translate $g\theta(x,c)$ does not fork over $M$ for all $g\in G(\mathfrak{C})$. Every strongly f-generic type is f-generic, but the converse need not hold, even when $G$ is definably amenable; see Example 3.10 in \citep{chernikov_simon}. Thus a natural question, motivated by Hrushovski and Pillay's characterization of definably amenable NIP groups, is whether, again for $T$ NIP, definable amenability is equivalent to the existence of a global f-generic type.

\section{Preliminaries}
As in the rest of the paper, we assume throughout this section that $T$ is NIP.
%$G^{00}$ on its own, or in definably amenable groups section? $G^{000}_C$ always generated by $a^{-1}b$ with same Lascar strong type over $C$. \textcolor{red}{Is this right when $C$ is not a model?} Hrushovski and Pillay proved that, if $G$ is definably amenable and $M\prec\mathfrak{C}$ is any small model, then $G^{00}(\mathfrak{C})=\{a^{-1}b:a,b\in G(\mathfrak{C}),a\equiv_M b\}$.
\subsection{Forking in NIP Theories}
We recall here some properties of forking independence in NIP theories. In this paper we will always be working over models, so we state all the necessary facts only for that context. Due to Shelah \citep{shelah_forking} and Adler \citep{adler_nip} we first have the following:
\begin{fact}
    If $p(x)\in S(\mathfrak{C})$ is a global type and $M\prec\mathfrak{C}$ is a small model, then $p(x)$ forks over $M$ if and only if it is $M$-invariant.
\end{fact}

For partial types or complete types that are not global, the relationship of forking and dividing in NIP theories is greatly clarified by Chernikov and Kaplan's work \citep{chernikov_kaplan} on the more general class of NTP$_2$ theories. We will rely on a number of their results, which we record in the facts below.

\begin{fact}
    Let $M\prec\mathfrak{C}$ be a small model. Then an $L_\mathfrak{C}$-formula forks over $M$ if and only if it divides over $M$. In particular, $\ind^f_M=\ind^d_M$.
\end{fact}

\begin{fact}
    Let $M\prec\mathfrak{C}$ be a small model. Then $\ind^f$ has \textit{left extension} over $M$: if $A,C$ are sets with $A\ind^f_M C$, and $B\supseteq A$, then there is $B'$ with $B'\equiv_{M,A} B$ and $B'\ind^f_M C$.
\end{fact}

\begin{fact}
    Let $p(x)\in S(M)$ be a complete type over a small model $M$. Then there is $q(x)\in S(\mathfrak{C})$ a global extension of $p(x)$ strictly non-forking over $M$. (So, for any small parameter set $C\supseteq M$, if $a\models q|_C$ then $a\ind^f_M C$ and $C\ind^f_M a$.)
\end{fact}

\subsection{Connected Components} Recall that, for a small model $M\prec\mathfrak{C}$, $G^{00}_M$ denotes the smallest subgroup of $G$ type-definable over $M$ and of bounded index, and $G^{\infty}_M$ denotes the smallest subgroup of $G$ invariant over $M$ and of bounded index; $G^{\infty}_M(\mathfrak{C})$ is precisely the subgroup of $G(\mathfrak{C})$ generated by $\{a^{-1}b:a,b\in G(\mathfrak{C}),a\equiv_M b\}$. When $G^{00}_M$, respectively $G^{\infty}_M$, is independent of the choice of $M$, one says that $G^{00}$, respectively $G^{\infty}$, `exists' and drops the subscript. Due to Shelah \citep{shelah_g00} and Gismatullin \citep{gismatullin} respectively, it is known that $G^{00}$ and $G^{\infty}$ always exist if $T$ is NIP.

Note that, if $p(x)\in S_G(\mathfrak{C})$ is a global type invariant under left translation by elements of $G^{00}(\mathfrak{C})$, then the orbit of $p(x)$ under left translation by elements of $G(\mathfrak{C})$ is bounded. One wishes to conclude from Theorem 3.12 in \citep{chernikov_simon} that $G$ is definably amenable, but a slight modification is needed to account for the possibility of an uncountable language; see Proposition 3.16 in \citep{pillay_groups} for detailed discussion of this, and for the argument one needs to obtain the following:

\begin{fact}
    Suppose $p(x)\in S_G(\mathfrak{C})$ is a global $G^{00}(\mathfrak{C})$-invariant type. Then $G$ is definably amenable.
\end{fact}

\subsection{f-Genericity}
Here we record a few facts about f-generic formulas from \citep{chernikov_simon}. Recall that a $\mathfrak{C}$-definable set $D\subseteq G$ is called `f-generic' if, for any small model $M$ over which $D$ is defined, the formula $x\in gD$ does not fork over $M$ for all $g\in G$. Likewise, a partial type is called f-generic if it implies only f-generic formulas.

By Fact 2.2 and Ramsey's theorem, one has a purely combinatorial characterization of f-genericity: a $\mathfrak{C}$-definable set $D\subseteq G$ is not f-generic if and only if it `$G$-divides', ie if and only if there is some $k\in\omega$ and some sequence $(g_i)_{i\in\omega}$ from $G$ such that the family of translates $(g_iD)_{i\in\omega}$ is $k$-inconsistent. In particular, like genericity, f-genericity can be characterized purely in terms of the group structure on $G$. Throughout this paper we will use freely the equivalence between $G$-dividing and non-f-genericity.

The following fact is implicit in Proposition 3.4 of \citep{chernikov_simon}, but is not stated there as such, so we include the proof here.

\begin{fact}
    Suppose $D\subseteq G$ is definable over a small model $M$, and that $g\in G(\mathfrak{C})$ is such that $r(x):=\tp(g/M)$ is f-generic. If $g^{-1}D$ does not fork over $M$, then $D$ is f-generic.
\end{fact}
\begin{proof}
    Suppose $D$ is not f-generic. Then there is an $M$-indiscernible sequence $J=(h_i)_{i\in\omega}$ such that $\bigwedge_{i\in\omega}h_iD=\varnothing$; let $k$ be such that the translates $h_iD$ are $k$-inconsistent. Since $r(x)$ is f-generic and defined over $M$, the translate $h_0r(x)$ does not fork over $M$. In particular, since $J$ is $M$-indiscernible, the partial type $\bigwedge_{i\in\omega}h_ir(x)$ is consistent. Let $u\in G(\mathfrak{C})$ be any realization, and let $g_i=h_i^{-1}u$ for each $i\in\omega$. Then $g_i\models r(x)$, so that $g_i^{-1}\equiv_M g^{-1}$, for each $i\in\omega$. But the translates $g_i^{-1}D$ are $k$-inconsistent; indeed, for any $s\subset\omega$ of size $k$, we have $$\bigwedge_{i\in s}g_i^{-1}D=u^{-1}\bigwedge_{i\in s}h_iD=\varnothing.$$ Thus $g^{-1}D$ divides over $M$.
\end{proof}
%Corollary of f-generic formulas forming an ideal??

%Finally, the following is standard, but we include a proof for completeness.

%\begin{fact} Let $M$ be a small model. If $G$ admits a global f-generic type, then it admits a global $M$-invariant f-generic type.
    %Suppose $D\subseteq G$ is definable over a small model $M$. If there is a global f-generic type concentrated on $D$, then there is a global $M$-invariant type concentrated on $D$.
%\end{fact}
%\begin{proof}
 %   Let $X\subseteq S_G(\mathfrak{C})$ be the space of f-generic global types; this is a closed and $M$-invariant subspace of $S_G(\mathfrak{C})$.
%\end{proof}

\newpage
\section{Results}
Now we can begin proving the result. As always, we assume throughout that $T$ is NIP.
\subsection{Strict Morley Sequences}
First we need the following rather general observation.
\begin{lemma}
    Let $M$ be a small model, and suppose $a,b\in\mathfrak{C}$ are such that $a\equiv_M b$. Suppose also that $q(x,y)\in S(\mathfrak{C})$ is a global extension of $\tp(a,b/M)$ strictly non-forking over $M$, and that $(a_i,b_i)_{i\in\omega}\models q^{\otimes\omega}|_M$ is a Morley sequence of $q$ over $M$. Then, for every $n\in\omega$, there is a model $N\prec\mathfrak{C}$ containing $(M,a_{\neq n},b_{\neq n})$ and such that $a_n\equiv_N b_n$. %Moreover, if $a,b\in G(\mathfrak{C})$ and $a^{-1}b\notin G(M)$, then we may choose $N$ so that $a_n^{-1}b_n\notin G(N)$.
\end{lemma}
\begin{proof}
    Fix $n\in\omega$, and pick a small model $N_0$ containing $(M,a_{<n},b_{<n})$. Let $N_1$ be the preimage of $N_0$ under an automorphism fixing $(M,a_{<n},b_{<n})$ pointwise and taking $(a_i,b_i)_{i\geqslant n}$ to a Morley sequence of $q$ over $N_0$; then $N_1$ contains $(M,a_{<n},b_{<n})$ and $(a_i,b_i)_{i\geqslant n}$ realizes $q^{\otimes\omega}|_{N_1}$. In particular, $N_1\ind_M(a_n,b_n)$, so that $\tp(N_1/M,a_n,b_n)$ extends to a global $M$-invariant type. Since $a_n\equiv_M b_n$, it follows that $a_n\equiv_{N_1}b_n$.

    Now, $(a_i,b_i)_{i>n}$ realizes $q^{\otimes\omega}|_{N_1,a_n,b_n}$. Since $q$ is $M$-invariant, $q^{\otimes\omega}$ is $M$-invariant, and thus also $N_1$-invariant, so we have $(a_{>n},b_{>n})\ind_{N_1}(a_n,b_n)$. Now let $N_2$ be any small model containing $(N_1,a_{>n},b_{>n})$. By Fact 2.3, after conjugating by an automorphism, we may assume that $N_2\ind_{N_1}(a_n,b_n)$, so that $\tp(N_2/N_1,a_n,b_n)$ extends to a global $N_1$-invariant type. Since $a_n\equiv_{N_1}b_n$, it follows that $a_n\equiv_{N_2}b_n$, and the model $N_2$ contains $(M,a_{\neq n},b_{\neq n})$. %For the second claim, suppose that $a,b\in G(\mathfrak{C})$ and that $a^{-1}b\notin G(M)$. Then $a_n^{-1}b_n\notin G(M)$, and since $N_1\ind_M(a_n,b_n)$ this implies $a_n^{-1}b_n\notin G(N_1)$. Since $N_2\ind_{N_1}(a_n,b_n)$, this in turn implies $a_n^{-1}b_n\notin G(N_2)$, as needed.  %and so $a_n,b_n$ are at Lascar distance at most $2$ over that set.
\end{proof}

\subsection{f-Generic Types}
Now we record a few lemmas on f-genericity. First let us make an observation; suppose $D\subseteq G$ is an f-generic set definable over a small model $M$, and that $a,b\in G(\mathfrak{C})$ are elements of $G$ with $a\equiv_M b$. By f-genericity, the formula $x\in aD$ does not fork over $M$, and so is contained in some global $M$-invariant type; this type must then also contain the formula $x\in bD$, and so one concludes that the intersection $aD\wedge bD$ (and hence $D\wedge a^{-1}bD$) is non-empty. If there is a global f-generic type, then this set will in fact also be f-generic. To see this we need the following, which is standard:

\begin{lemma} Let $M$ be a small model. If $G$ admits a global f-generic type, then it admits a global $M$-invariant f-generic type.
    %Suppose $D\subseteq G$ is definable over a small model $M$. If there is a global f-generic type concentrated on $D$, then there is a global $M$-invariant type concentrated on $D$.
\end{lemma}
\begin{proof}
    Let $X\subseteq S_G(\mathfrak{C})$ be the space of f-generic global types; this is a closed (hence compact) and $M$-invariant subspace of $S_G(\mathfrak{C})$. By Fact 2.1, it suffices to show that $X$ contains a type which does not fork over $M$, so suppose otherwise for contradiction. Then, since $X$ is compact, there are formulas $\phi_1(x,b_1),\dots,\phi_n(x,b_n)$ which fork over $M$ and such that no element of $X$ is concentrated on $\bigwedge_{i\in[n]}\neg\phi_i(x,b_i)$. In particular, letting $\psi(x,b)=\bigvee_{i\in[n]}\phi_i(x,b_i)$, then $\psi(x,b)$ forks over $M$ and is contained in every element of $X$. By Fact 2.2, there is now an $M$-indiscernible sequence $(b^i)_{i\in\omega}$ with $b^i\equiv_M b$ and with $(\psi(x,b^i))_{i\in\omega}$ inconsistent. But $X$ is $M$-invariant, so $\psi(x,b^i)$ is contained in every element of $X$ for all $i\in\omega$, which forces $X$ to be empty.
\end{proof}

Now we can obtain the desired strengthening of the observation above.
\begin{lemma}
    Suppose there is a global f-generic type. Then, for any f-generic set $D\subseteq G$ definable over a small model $M$, and any $a,b\in G(\mathfrak{C})$ with $a\equiv_M b$, the intersection $D\wedge a^{-1}bD$ is f-generic.
\end{lemma}
\begin{proof} By Lemma 3.2, let $q(x)\in S_G(\mathfrak{C})$ be f-generic and $M$-invariant. Since f-genericity is translation-invariant, it suffices to show that $E:=aD\wedge bD$ is f-generic. Let $N$ be any small model containing $(M,a,b)$ and let $g\models q|_N$. By Fact 2.6 (and Fact 2.2) it suffices now to show that $g^{-1}E$ does not divide over $N$, so let $(g_i)_{i\in\omega}$ be any $N$-indiscernible sequence with $g_i\equiv_N g$; we wish to show that $\bigwedge_{i\in\omega}g_i^{-1}E=\bigwedge_{i\in\omega}g^{-1}_iaD\wedge\bigwedge_{j\in\omega}g^{-1}_jbD$ is consistent. Since $D$ is f-generic, the formula $x\in g^{-1}_iaD$ does not fork over $M$, and so is contained in some global $M$-invariant type; thus it suffices to show that all the $g^{-1}_ia$ and $g^{-1}_jb$ have the same type over $M$.

But $\tp(g_i/M,a,b)$ extends to the global $M$-invariant type $q(x)$, and by hypothesis $a\equiv_M b$, whence $a\equiv_{M,g_i}b$ and so also $g^{-1}_ia\equiv_M g^{-1}_ib$. Moreover, $(g_i)_{i\in\omega}$ is $N$-indiscernible, and in particular $(M,b)$-indiscernible, whence in turn $g_i\equiv_{M,b}g_j$ and so $g^{-1}_ib\equiv_M g^{-1}_jb$. Thus indeed $g^{-1}_ia\equiv_M g^{-1}_jb$ and the desired result follows.
\end{proof} As a consequence we get the following key lemma:

\begin{corollary}
    Let $M$ be a small model, and let $I=(c_i)_{i\in\omega}$ be an $M$-indiscernible sequence of elements of $G(\mathfrak{C})$ with the following property: for every $n\in\omega$, there is a model $N$, containing $(M,c_{\neq n})$, and a pair of elements $a,b\in G(\mathfrak{C})$ with $a\equiv_N b$ and $a^{-1}b=c_n$.

    Suppose also that there is a global f-generic type. Then, for any $M$-definable set $D\subseteq G$, the partial type $\{x\in D\}\cup\{x\notin c_iD:i\in\omega\}$ is not f-generic.
    
    %let $D\subseteq G$ be an $M$-definable set, and let $I=(c_i)_{i\in\omega}$ be a sequence of distinct elements of $G(\mathfrak{C})$ with the following property: for every $n\in\omega$, there is a model $N$, containing $(M,I\setminus\{c_n\})$, and a pair of elements $a,b\in G(\mathfrak{C})$ with $a\equiv_N b$ and $a^{-1}b=c_n$. Then, if $p(x)\in S_G(M,I)$ is a compressible f-generic type concentrated on $D$, we have $p\vdash x\in c_iD$ for all but finitely many $i$.
\end{corollary}
\begin{proof}
    Suppose otherwise for contradiction, and let $\pi(x)=\{x\in D\}\cup\{x\notin c_{2i}D:i\in\omega\}$; then $\pi(x)$ is f-generic. We claim that $\pi(x)\wedge\bigwedge_{i\in\omega}c_{2i+1}\pi(x)$ is f-generic; by compactness it suffices to show that $\sigma_n(x):=\pi(x)\wedge\bigwedge_{i\in[n]}c_{2i-1}\pi(x)$ is f-generic for each $n\in\omega$, and we prove this by induction on $n$. The base case $n=0$ is by hypothesis, and for the inductive step assume we have shown that $\sigma_n(x)$ is f-generic. Now, $\sigma_n$ is a partial type defined over $(M,(c_{2i})_{i\in\omega},(c_{2i-1})_{i\in[n]})$. In particular, by the hypothesis on $I$, there is a small model $N$ such that $\sigma_n(x)$ is defined over $N$ and such that $c_{2n+1}=a^{-1}b$ for some $a,b\in G(\mathfrak{C})$ with $a\equiv_N b$. Since $\sigma_n(x)$ is f-generic, by Lemma 3.3 and compactness the intersection $\sigma_n(x)\wedge a^{-1}b\sigma_n(x)$, ie $\sigma_n(x)\wedge c_{2n+1}\sigma_n(x)$, is thus f-generic. But this intersection implies $\sigma_n(x)\wedge c_{2n+1}\pi(x)=\sigma_{n+1}(x)$, so the result follows.

    So indeed $\pi(x)\wedge\bigwedge_{i\in\omega}c_{2i+1}\pi(x)$ is f-generic. In particular, it is consistent. But it contains the formulas $x\notin c_{2i}D$ and $x\in c_{2i+1}D$ for every $i\in\omega$; since $I$ is indiscernible this contradicts NIP.
\end{proof}
\subsection{Main Result}
Now we are ready to prove the main result. For the rest of the section, assume that $p(x)\in S_G(\mathfrak{C})$ is a global f-generic type. %We first give the main strengthening of Lemma 3.3.
\begin{lemma}
    Let $M\prec\mathfrak{C}$ be a small model, and let $D$ be an $M$-definable set with $p(x)\vdash x\in D$. Then the partial type $\{x\in a^{-1}bD:a,b\in G(\mathfrak{C}),a\equiv_M b\}$ is f-generic.
\end{lemma}
\begin{proof}
    %Suppose otherwise. Then there are finitely many $(a_1,b_1),\dots,(a_n,b_n)$ such that $a_i\equiv_M b_i$ for each $i$ and such that $\bigwedge_{i\in[n]}a_i^{-1}b_iD$ is not f-generic; assume we have picked $D$ so that $n$ is minimal possible. Then note in particular that $a_i^{-1}b_i\notin G(M)$ for each $i$; indeed, suppose otherwise. Then without loss of generality $a_1^{-1}b_1\in G(M)$. Let $D'$ be the $M$-definable set $D\wedge a_1^{-1}b_1 D$. call this fact $\star$.

    Suppose otherwise. Then there are $(a_1,b_1),\dots,(a_n,b_n)$ such that $a_i\equiv_M b_i$ for each $i$ and such that $\bigwedge_{i\in[n]}a_i^{-1}b_iD$ is not f-generic. By Fact 2.4, let $q(x_1,y_1,\dots,x_n,y_n)$ be a global extension of $\tp(a_1,b_1,\dots,a_n,b_n/M)$ strictly non-forking over $M$, and let $(a_{k1},b_{k1},\dots,a_{kn},b_{kn})_{k\in\omega}\models q^{\otimes\omega}|_M$ be a Morley sequence of $q$ over $M$. In particular, $(a_{k1},\dots,b_{kn})$ has the same type as $(a_1,\dots,b_n)$ over $M$ for every $k\in\omega$, so the set $\bigwedge_{i\in[n]}a_{ki}^{-1}b_{ki}D$ is not f-generic for every $k\in\omega$. Since the type $p$ is f-generic, by the pigeonhole principle there is hence some $i\in[n]$ and some infinite subset $s\subseteq\omega$ with $p(x)\vdash x\notin a_{ki}^{-1}b_{ki}D$ for all $k\in s$.

    Note that the restriction of $q$ to the variables $(x_i,y_i)$ is a global extension of $\tp(a_i,b_i/M)$ strictly non-forking over $M$. So, by Lemma 3.1, for every $n\in\omega$ there is a small model $N$ containing $(M,a_{ki},b_{ki}:k\neq n)$ and such that $a_{ni}\equiv_N b_{ni}$.
    
    In particular, if we define $c_n=a_{ni}^{-1}b_{ni}$, then $I=(c_n)_{n\in\omega}$ satisfies the hypotheses of Corollary 3.4. But now $p$ is f-generic, concentrated on $D$, and contains the formula $x\notin c_kD$ for every $k\in s$; this contradicts Corollary 3.4.
\end{proof}

\begin{corollary}
    Let $M$ be a small model. Then the set $\{a^{-1}b:a,b\in G(\mathfrak{C}),a\equiv_M b\}$ is a group, and by the facts from Section 2.2 it is thus $G^{00}(\mathfrak{C})$.
\end{corollary}
\begin{proof}
    It suffices to show this set is closed under multiplication, so fix any $a,b,c,d\in G(\mathfrak{C})$ with $a\equiv_M b$ and $c\equiv_M d$. Let $r(x)$ denote the restriction $p|_M(x)$. By Lemma 3.5 and compactness, the partial type $b^{-1}ar(x)\wedge c^{-1}dr(x)$ is f-generic, and hence in particular consistent. Let $e$ be any realization. Then $a^{-1}be$ and $d^{-1}ce$ each realize $r$, so $e^{-1}b^{-1}a$ and $e^{-1}c^{-1}d$ have the same type over $M$, and now $a^{-1}bc^{-1}d=(e^{-1}b^{-1}a)^{-1}(e^{-1}c^{-1}d)$.
\end{proof}
\begin{corollary}
The type $p$ is $G^{00}(\mathfrak{C})$-invariant.
\end{corollary}
\begin{proof}
    Suppose otherwise. Then there is a $\mathfrak{C}$-definable set $D\subseteq G$ and an element $g\in G^{00}(\mathfrak{C})$ with $p(x)$ concentrated on $E:=D\setminus gD$. Let $N$ be any small model over which $D$ and $g$ are both defined. Now, by Corollary 3.6, there are some $a,b\in G(\mathfrak{C})$ with $a\equiv_N b$ and $g=a^{-1}b$. Since $E$ is defined over $N$ and $p(x)\vdash x\in E$, by Lemma 3.3 we have that $E\wedge a^{-1}bE$, ie $E\wedge gE$, is f-generic. But this set is contained in $gD\setminus gD=\varnothing$, a contradiction.
\end{proof} Now from Corollary 3.7 and Fact 2.5 we obtain the main theorem.
\begin{corollary}
    The group $G$ is definably amenable.
\end{corollary}

\newpage


%\bibliographystyle{plain}
%\bibliography{references}
\begin{thebibliography}{9}
\bibitem{adler_nip}Hans Adler. Introduction to theories without the independence property. \textit{Archive of Mathematical Logic}, 2008.

\bibitem{adler_ntp2}Hans Adler. Kim's lemma for NTP$_2$ theories. Preprint. 2014.

%\bibitem{bays_kaplan_simon}Martin Bays, Itay Kaplan, Pierre Simon. Density of compressible types and some consequences. \textit{Journal of the EMS}, 2022.

\bibitem{chernikov_kaplan}Artem Chernikov, Itay Kaplan. Forking and dividing in NTP$_2$ theories. \textit{The Journal of Symbolic Logic}, 2012.

\bibitem{chernikov_simon}Artem Chernikov, Pierre Simon. Definably amenable NIP groups. \textit{Journal of the AMS}, 2018.

%\bibitem{chernikov_simon_compressible}Artem Chernikov, Pierre Simon. Externally definable sets and dependent pairs II. \textit{Transactions of the AMS}, 2015.

\bibitem{gismatullin}Jakub Gismatullin. Model-theoretic connected components of groups. \textit{Israel Journal of Mathematics}, 2011.

\bibitem{hrushovski_peterzil_pillay}Ehud Hrushovski, Ya'acov Peterzil, Anand Pillay. Groups, measures, and the NIP. \textit{Journal of the AMS}, 2008.

\bibitem{hrushovski_pillay}Ehud Hrushovski, Anand Pillay. On NIP and invariant measures. \textit{Journal of the EMS}, 2011.

\bibitem{pillay_groups}Anand Pillay. Model theory and groups. Preprint, 2021.

\bibitem{shelah_forking}Saharon Shelah. Dependent first-order theories, continued. \textit{Israel Journal of Mathematics}, 2009.

\bibitem{shelah_g00}Saharon Shelah. Minimal bounded index subgroup for dependent theories. \textit{Proceedings of the AMS}, 2008.

%\bibitem{shelah_g000}Saharon Shelah. Definable groups for dependent and 2-dependent theories. 2009.

\bibitem{simon_book}Pierre Simon. A guide to NIP theories. \textit{Cambridge University Press}, 2015.

\bibitem{simon_distal}Pierre Simon. Distal and non-distal NIP theories. \textit{Annals of Pure and Applied Logic}, 2013.

%\bibitem{simon_compressible}Pierre Simon. Type decomposition in NIP theories. \textit{Journal of the EMS}, 2016.

%\bibitem{usvyatsov}Alexander Usvyatsov. Morley sequences in dependent theories. Preprint. 2008.


\end{thebibliography}


\end{document}