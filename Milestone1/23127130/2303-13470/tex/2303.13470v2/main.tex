\pdfoutput=1

\documentclass[10pt]{article}
%\documentclass[11pt]{article}

\makeatletter
%%%%%%
\renewcommand{\maketitle}{\bgroup\setlength{\parindent}{0pt}
\begin{flushleft}
  \textbf{\@title}

  \@author
\end{flushleft}\egroup
} %%%%%
\def\footnoterule{\kern-3\p@
  \hrule \@width 2in \kern 2.6\p@} % the \hrule is .4pt high
  %%%%%%
\makeatother
\renewcommand{\thefootnote}{\fnsymbol{footnote}}

\usepackage{preamble}
\usepackage{courier}
\usepackage[square,numbers]{natbib}
%\usepackage{preamble}

\title{{\Large \textbf{On f-generic types in NIP groups}}}
%\title{{\large \textbf{A note on product-free sets in distal groups}}}
%\author{Atticus Stonestrom\footnote{Email: \texttt{atticusstonestrom@yahoo.com}}}
\author{Atticus Stonestrom\ \ \ \ \ \ \ \ \ \ \ \ \ \ \ \ \ \  {\footnotesize (Email: \texttt{atticusstonestrom@yahoo.com})}}
\date{}

\begin{document}
\maketitle

{\small \noindent\textbf{Abstract:} Let $G$ be a group definable in an NIP theory. We prove that, if $G$ admits a global f-generic type, then $G$ is definably amenable, answering a question of Chernikov and Simon. \newline %\textcolor{blue}{We prove moreover that such a type is never concentrated on a cut of an indiscernible sequence.} \newline

\noindent\textbf{Acknowledgements:} I would like to thank Artem Chernikov, Itay Kaplan, and Anand Pillay for all of their encouragement and feedback. I would like to thank Itay Kaplan in particular for pointing out a shorter proof of Lemma 3.1. \newline

\noindent\textbf{Notation:} Throughout $T$ will denote a complete theory in a first-order language $L$, with a monster model $\mathfrak{C}$. $G$ will denote a definable group in $T$, and for any model $M\preccurlyeq\mathfrak{C}$ we write $G(M)$ for the realizations of $G$ whose coordinates lie in $M$. For a parameter set $C\subseteq\mathfrak{C}$, we use $S_G(C)$ to denote the space of all complete types over $C$ that are concentrated on $G$. For tuples $a,b$, and a small parameter set $C\subset\mathfrak{C}$, we write $a\equiv_C b$ to mean $\tp(a/C)=\tp(b/C)$. We write $\ind$ for non-forking independence. Finally, we will assume throughout the entire paper that $T$ is NIP.

\section{Introduction} %\textcolor{red}{A type $p\in S_G(\mathfrak{C})$ is f-generic iff, for every small model $M\prec\mathfrak{C}$, the translate $gp|_M(x)$ does not fork over $M$ for all $g\in G(\mathfrak{C})$.}


Recall that $G$ is said to be `definably amenable' if it admits a translation-invariant Keisler measure. The study of definably amenable groups was instigated in the paper \citep{hrushovski_peterzil_pillay}, and we refer the reader to the introduction of \citep{chernikov_simon} or to the survey \citep{pillay_groups} for general history and motivation for the notion. Here we will focus on only one aspect of the theory. In \citep{hrushovski_pillay}, Hrushovski and Pillay showed that, for $T$ countable and NIP, definable amenability of $G$ is equivalent to the existence of a \textit{strongly f-generic}\footnote[1]{In \citep{hrushovski_pillay} such a type is just called f-generic, but the notion was renamed in \citep{chernikov_simon}.} type: a global type $p(x)\in S_G(\mathfrak{C})$ such that, for some small model $M\prec\mathfrak{C}$, the translate $gp(x)$ does not fork over $M$ for all $g\in G(\mathfrak{C})$. In the paper \citep{chernikov_simon}, Chernikov and Simon introduced the weaker notion of an \textit{f-generic} type (over some parameter set $C\subseteq\mathfrak{C}$): a type $p(x)\in S_G(C)$ such that, for every formula $\theta(x,c)$ in $p$, there is some small model $M\prec\mathfrak{C}$ such that the translate $g\theta(x,c)$ does not fork over $M$ for all $g\in G(\mathfrak{C})$. Every strongly f-generic type is f-generic, but a global f-generic type need not be strongly f-generic; see Example 3.10 in \citep{chernikov_simon}. So one can ask whether the analogue of Hrushovski and Pillay's theorem holds for f-generic types; this is Question 3.18 in \citep{chernikov_simon}. In this paper we give a positive answer. Thus, by the implications already known from \citep{hrushovski_pillay} and \citep{chernikov_simon}, the following are equivalent when $T$ is NIP:
\begin{enumerate}
    \item $G$ is definably amenable.
    \item $G$ admits a global f-generic type.
    \item The non-f-generic subsets of $G$ form an ideal.
\end{enumerate}

\section{Preliminaries}
As in the rest of the paper, we assume throughout this section that $T$ is NIP.

\subsection{Forking in NIP Theories}
We recall here some properties of non-forking independence in NIP theories. In this paper we will always be working over models, so we state all the necessary facts only for that context. Due to Shelah \citep{shelah_forking} and Adler \citep{adler_nip} we first have the following:
\begin{fact}
    If $p(x)\in S(\mathfrak{C})$ is a global type and $M\prec\mathfrak{C}$ is a small model, then $p(x)$ does not fork over $M$ if and only if it is $M$-invariant.
\end{fact}

For partial types or complete types that are not global, the relationship of forking and dividing in NIP theories is greatly clarified by Chernikov and Kaplan's work \citep{chernikov_kaplan} on the more general class of NTP$_2$ theories. We will rely on a number of their results, which we record in the facts below.

\begin{fact}
    Let $M\prec\mathfrak{C}$ be a small model. Then an $L_\mathfrak{C}$-formula forks over $M$ if and only if it divides over $M$. %In particular, $\ind^f_M=\ind^d_M$.
\end{fact}

\begin{fact}
    Let $M\prec\mathfrak{C}$ be a small model. Then $\ind$ has `left-extension' over $M$: if $A,C$ are sets with $A\ind_M C$, and $B\supseteq A$, then there is $B'$ with $B'\equiv_{M,A} B$ and $B'\ind_M C$.
\end{fact}

\begin{fact}
    Let $p(x)\in S(M)$ be a complete type over a small model $M$. Then there is $q(x)\in S(\mathfrak{C})$ a global extension of $p(x)$ strictly non-forking over $M$. (So, for any small parameter set $C\supseteq M$, if $a\models q|_C$ then $a\ind_M C$ and $C\ind_M a$.)
\end{fact}We also recall that $\ind$ in arbitrary theories has `left-transitivity' (see \citep{adler_forking}): for any small sets $A,B,C,D\subset\mathfrak{C}$, if $A\ind_{(B,C)} D$ and $B\ind_C D$ then $(A,B)\ind_C D$.

\subsection{Connected Components} Recall that, for a small model $M\prec\mathfrak{C}$, $G^{00}_M$ denotes the smallest subgroup of $G$ type-definable over $M$ and of bounded index, and $G^{\infty}_M$ denotes the smallest subgroup of $G$ invariant over $M$ and of bounded index; $G^{\infty}_M(\mathfrak{C})$ is precisely the subgroup of $G(\mathfrak{C})$ generated by $\{a^{-1}b:a,b\in G(\mathfrak{C}),a\equiv_M b\}$. When $G^{00}_M$, respectively $G^{\infty}_M$, is independent of the choice of $M$, one says that $G^{00}$, respectively $G^{\infty}$, `exists' and drops the subscript. Due to Shelah \citep{shelah_g00} and Gismatullin \citep{gismatullin} respectively, it is known that $G^{00}$ and $G^{\infty}$ always exist if $T$ is NIP.

Then groups $G^{00}$ and $G^\infty$ are normal subgroups of $G$, and moreover one can endow the quotient $G/G^{00}$ with the `logic topology', where a subset is closed if and only if its preimage under the projection map $G\to G/G^{00}$ is type-definable over some small set. This makes $G/G^{00}$ into a compact Hausdorff topological group, which is thus endowed with a translation-invariant Haar measure $h$ such that $h(G/G^{00})=1$.

For $T$ countable and NIP, Hrushovski and Pillay gave a construction in \citep{hrushovski_pillay} to obtain translation-invariant Keisler measures on $G$ from $G^{00}$-invariant types: suppose $p(x)\in S_G(\mathfrak{C})$ is invariant under left translation by elements of $G^{00}(\mathfrak{C})$. For a $\mathfrak{C}$-definable set $D\subseteq G$, define $S_{p,D}$ to be the subset of $G/G^{00}$ given by $\{\bar{g}:gp(x)\vdash x\in D\}$; this is well-defined by $G^{00}$-invariance of $p$. If the sets $S_{p,D}$ are all Borel, then one can obtain a Keisler measure $\mu_p$ by taking $\mu_p(D)=h(S_{p,D})$ for each $D$, and $\mu_p$ will be left-invariant by left-invariance of $h$.

In the case that $T$ is countable and NIP, the $S_{p,D}$ will indeed all be Borel. This is proved in \citep{hrushovski_pillay} under the additional hypothesis that $p(x)$ is $M$-invariant over some countable $M\prec\mathfrak{C}$, and a further argument of Chernikov and Simon shows that this hypothesis is not needed; see Definition 3.16 of \citep{chernikov_simon}.\footnote[1]{In \citep{chernikov_simon} this is not expressed that way; Chernikov and Simon assume there that $G$ is definably amenable and that $p(x)$ is f-generic. But all that one needs to apply their argument is that what they denote $p_M$ is $M$-invariant for every small $M\prec\mathfrak{C}$, and this is true whenever $p$ is $G^{00}$-invariant: if $p(x)\vdash x\in aD\setminus bD$ for some $M$-definable $D\subseteq G$ and some $a\equiv_M b$ in $G$, then $ab^{-1}\in G^{00}$ and $ab^{-1}p(x)\vdash x\notin aD$, so that $p$ is not $G^{00}$-invariant.} Thus one has the following fact:

\begin{fact}
    If $T$ is countable and $G$ admits a global $G^{00}$-invariant type, then $G$ is definably amenable.
\end{fact} Fact 2.5 is true even without the countability hypothesis, but a different argument is needed. One approach is via Theorem 3.12 in \citep{chernikov_simon}, as in Proposition 3.16 of \citep{pillay_groups}. However, we will need only the countable case in this paper.

\subsection{f-Genericity}
Here we record a few facts about f-generic formulas from \citep{chernikov_simon}. Recall that a $\mathfrak{C}$-definable set $D\subseteq G$ is called `f-generic' if, for any small model $M$ over which $D$ is defined, the formula $x\in gD$ does not fork over $M$ for all $g\in G$. Likewise, a partial type is called f-generic if it implies only f-generic formulas.

By Fact 2.2 and Ramsey's theorem, f-genericity (like genericity) can be characterized purely in terms of the group structure on $G$: a $\mathfrak{C}$-definable set $D\subseteq G$ is not f-generic if and only if it `$G$-divides', ie if and only if there is some $k\in\omega$ and some sequence $(g_i)_{i\in\omega}$ from $G$ such that the family of translates $(g_iD)_{i\in\omega}$ is $k$-inconsistent. Throughout this paper we will freely use the equivalence between $G$-dividing and non-f-genericity, 

Finally, the following fact is implicit in Proposition 3.4 of \citep{chernikov_simon}, but is not stated there as such, so we include the proof here for completeness.

\begin{fact}
    Suppose $D\subseteq G$ is definable over a small model $M$, and that $g\in G(\mathfrak{C})$ is such that $r(x):=\tp(g/M)$ is f-generic. If $g^{-1}D$ does not fork over $M$, then $D$ is f-generic.
\end{fact}
\begin{proof}
    Suppose $D$ is not f-generic. Then there is an $M$-indiscernible sequence $J=(h_i)_{i\in\omega}$ such that $\bigwedge_{i\in\omega}h_iD=\varnothing$; let $k$ be such that the translates $h_iD$ are $k$-inconsistent. Since $r(x)$ is f-generic and defined over $M$, the translate $h_0r(x)$ does not fork over $M$. In particular, since $J$ is $M$-indiscernible, the partial type $\bigwedge_{i\in\omega}h_ir(x)$ is consistent. Let $u\in G(\mathfrak{C})$ be any realization, and let $g_i=h_i^{-1}u$ for each $i\in\omega$. Then $g_i\models r(x)$, so that $g_i^{-1}\equiv_M g^{-1}$, for each $i\in\omega$. But the translates $g_i^{-1}D$ are $k$-inconsistent; indeed, for any $s\subset\omega$ of size $k$, we have $$\bigwedge_{i\in s}g_i^{-1}D=u^{-1}\bigwedge_{i\in s}h_iD=\varnothing.$$ Thus $g^{-1}D$ divides over $M$.
\end{proof} By (the argument of) Corollary 3.5 in \citep{chernikov_simon}, it follows that the existence of a global f-generic type is equivalent to the non-f-generic sets forming an ideal.

%\newpage
\section{Results}
Now we can begin proving the result. As always, we assume throughout that $T$ is NIP.
\subsection{Strict Morley Sequences}
First we need the following rather general observation; thank you to Itay Kaplan for pointing out a shorter argument than I originally had.
\begin{lemma}
    Let $M$ be a small model, and suppose $a,b\in\mathfrak{C}$ are such that $a\equiv_M b$. Suppose also that $q(x,y)\in S(\mathfrak{C})$ is a global extension of $\tp(a,b/M)$ strictly non-forking over $M$, and that $(a_i,b_i)_{i\in\omega}\models q^{\otimes\omega}|_M$ is a Morley sequence of $q$ over $M$. Then, for every $n\in\omega$, there is a model $N\prec\mathfrak{C}$ containing $(M,a_{\neq n},b_{\neq n})$ and such that $a_n\equiv_N b_n$. %Moreover, if $a,b\in G(\mathfrak{C})$ and $a^{-1}b\notin G(M)$, then we may choose $N$ so that $a_n^{-1}b_n\notin G(N)$.
\end{lemma}
\begin{proof} Fix $n\in\omega$. Since $(a_n,b_n)\models q|_{M,a_{<n},b_{<n}}$, we have $(a_{<n},b_{<n})\ind_M (a_n,b_n)$ by strict non-forking of $q$. Moreover, $q^{\otimes\omega}$ is an $M$-invariant type, and $(a_i,b_i)_{i>n}$ realizes its restriction to $(M,a_{\leqslant n},b_{\leqslant n})$, so that $(a_{>n},b_{>n})\ind_M (a_{\leqslant n},b_{\leqslant n})$; in particular $(a_{>n},b_{>n})\ind_{(M,a_{<n},b_{<n})}(a_n,b_n)$. By left-transitivity we thus have $(a_{\neq n},b_{\neq n})\ind_M(a_n,b_n)$, and by Fact 2.3 there is now a model $N$ containing $(M,a_{\neq n},b_{\neq n})$ and such that $N\ind_M(a_n,b_n)$. By Fact 2.1, $\tp(N/M,a_n,b_n)$ extends to a global $M$-invariant type, and since $a_n\equiv_M b_n$ this implies that $a_n\equiv_N b_n$, as needed.
\end{proof}

\subsection{f-Generic Types}
Now we record a few lemmas on f-genericity. First let us make an observation; suppose $D\subseteq G$ is an f-generic set definable over a small model $M$, and that $a,b\in G(\mathfrak{C})$ are elements of $G$ with $a\equiv_M b$. By f-genericity, the formula $x\in aD$ does not fork over $M$, and so is contained in some global $M$-invariant type; this type must then also contain the formula $x\in bD$, and so one concludes that the intersection $aD\wedge bD$ (and hence $D\wedge a^{-1}bD$) is non-empty. If there is a global f-generic type, then this set will in fact also be f-generic. To see this we need the following, which is standard:

\begin{lemma} Let $M$ be a small model. If $G$ admits a global f-generic type, then it admits a global $M$-invariant f-generic type.
    %Suppose $D\subseteq G$ is definable over a small model $M$. If there is a global f-generic type concentrated on $D$, then there is a global $M$-invariant type concentrated on $D$.
\end{lemma}
\begin{proof}
    Let $\pi(x)$ be the partial type containing the formula $x\notin D$ for every $\mathfrak{C}$-definable set $D\subseteq G$ that is not f-generic. By hypothesis, $\pi(x)$ is consistent. It is also $M$-invariant, since the property of being f-generic is preserved under automorphisms. Thus $\pi(x)$ does not divide over $M$. By Fact 2.2, this means $\pi(x)$ does not fork over $M$. By Fact 2.1, $\pi(x)$ thus extends to a global $M$-invariant type, which is then f-generic, as needed.
    %Let $X\subseteq S_G(\mathfrak{C})$ be the space of f-generic global types; this is a closed (hence compact) and $M$-invariant subspace of $S_G(\mathfrak{C})$. By Fact 2.1, it suffices to show that $X$ contains a type which does not fork over $M$, so suppose otherwise for contradiction. Then, since $X$ is compact, there are formulas $\phi_1(x,b_1),\dots,\phi_n(x,b_n)$ which fork over $M$ and such that no element of $X$ is concentrated on $\bigwedge_{i\in[n]}\neg\phi_i(x,b_i)$. In particular, letting $\psi(x,b)=\bigvee_{i\in[n]}\phi_i(x,b_i)$, then $\psi(x,b)$ forks over $M$ and is contained in every element of $X$. By Fact 2.2, there is now an $M$-indiscernible sequence $(b^i)_{i\in\omega}$ with $b^i\equiv_M b$ and with $(\psi(x,b^i))_{i\in\omega}$ inconsistent. But $X$ is $M$-invariant, so $\psi(x,b^i)$ is contained in every element of $X$ for all $i\in\omega$, which forces $X$ to be empty.
\end{proof}

Now we can obtain the desired strengthening of the observation above.
\begin{lemma}
    Suppose there is a global f-generic type. Then, for any f-generic set $D\subseteq G$ definable over a small model $M$, and any $a,b\in G(\mathfrak{C})$ with $a\equiv_M b$, the intersection $D\wedge a^{-1}bD$ is f-generic.
\end{lemma}
\begin{proof} By Lemma 3.2, let $q(x)\in S_G(\mathfrak{C})$ be f-generic and $M$-invariant. Since f-genericity is translation-invariant, it suffices to show that $E:=aD\wedge bD$ is f-generic. Let $N$ be any small model containing $(M,a,b)$ and let $g\models q|_N$. By Fact 2.6 (and Fact 2.2) it suffices now to show that $g^{-1}E$ does not divide over $N$, so let $(g_i)_{i\in\omega}$ be any $N$-indiscernible sequence with $g_i\equiv_N g$; we wish to show that $\bigwedge_{i\in\omega}g_i^{-1}E=\bigwedge_{i\in\omega}g^{-1}_iaD\wedge\bigwedge_{j\in\omega}g^{-1}_jbD$ is consistent. Since $D$ is f-generic, the formula $x\in g^{-1}_iaD$ does not fork over $M$, and so is contained in some global $M$-invariant type; thus it suffices to show that all the $g^{-1}_ia$ and $g^{-1}_jb$ have the same type over $M$.

But $\tp(g_i/M,a,b)$ extends to the global $M$-invariant type $q(x)$, and by hypothesis $a\equiv_M b$, whence $a\equiv_{M,g_i}b$ and so also $g^{-1}_ia\equiv_M g^{-1}_ib$. Moreover, $(g_i)_{i\in\omega}$ is $N$-indiscernible, and in particular $(M,b)$-indiscernible, whence in turn $g_i\equiv_{M,b}g_j$ and so $g^{-1}_ib\equiv_M g^{-1}_jb$. Thus indeed $g^{-1}_ia\equiv_M g^{-1}_jb$ and the desired result follows.
\end{proof} As a consequence we get the following key lemma:

\begin{corollary}
    Let $M$ be a small model, and let $I=(c_i)_{i\in\omega}$ be an $M$-indiscernible sequence of elements of $G(\mathfrak{C})$ with the following property: for every $n\in\omega$, there is a model $N$, containing $(M,c_{\neq n})$, and a pair of elements $a,b\in G(\mathfrak{C})$ with $a\equiv_N b$ and $a^{-1}b=c_n$.

    Suppose also that there is a global f-generic type. Then, for any $M$-definable set $D\subseteq G$, the partial type $\{x\in D\}\cup\{x\notin c_iD:i\in\omega\}$ is not f-generic.
    
    %let $D\subseteq G$ be an $M$-definable set, and let $I=(c_i)_{i\in\omega}$ be a sequence of distinct elements of $G(\mathfrak{C})$ with the following property: for every $n\in\omega$, there is a model $N$, containing $(M,I\setminus\{c_n\})$, and a pair of elements $a,b\in G(\mathfrak{C})$ with $a\equiv_N b$ and $a^{-1}b=c_n$. Then, if $p(x)\in S_G(M,I)$ is a compressible f-generic type concentrated on $D$, we have $p\vdash x\in c_iD$ for all but finitely many $i$.
\end{corollary}
\begin{proof}
    Suppose otherwise for contradiction, and let $\pi(x)=\{x\in D\}\cup\{x\notin c_{2i}D:i\in\omega\}$; then $\pi(x)$ is f-generic. We claim that $\pi(x)\wedge\bigwedge_{i\in\omega}c_{2i+1}\pi(x)$ is f-generic; by compactness it suffices to show that $\sigma_n(x):=\pi(x)\wedge\bigwedge_{i\in[n]}c_{2i-1}\pi(x)$ is f-generic for each $n\in\omega$, and we prove this by induction on $n$. The base case $n=0$ is by hypothesis, and for the inductive step assume we have shown that $\sigma_n(x)$ is f-generic. Now, $\sigma_n$ is a partial type defined over $(M,(c_{2i})_{i\in\omega},(c_{2i-1})_{i\in[n]})$. In particular, by the hypothesis on $I$, there is a small model $N$ such that $\sigma_n(x)$ is defined over $N$ and such that $c_{2n+1}=a^{-1}b$ for some $a,b\in G(\mathfrak{C})$ with $a\equiv_N b$. Since $\sigma_n(x)$ is f-generic, by Lemma 3.3 and compactness the intersection $\sigma_n(x)\wedge a^{-1}b\sigma_n(x)$, ie $\sigma_n(x)\wedge c_{2n+1}\sigma_n(x)$, is thus f-generic. But this intersection implies $\sigma_n(x)\wedge c_{2n+1}\pi(x)=\sigma_{n+1}(x)$, so the result follows.

    So indeed $\pi(x)\wedge\bigwedge_{i\in\omega}c_{2i+1}\pi(x)$ is f-generic. In particular, it is consistent. But it contains the formulas $x\notin c_{2i}D$ and $x\in c_{2i+1}D$ for every $i\in\omega$; since $I$ is indiscernible this contradicts NIP.
\end{proof}
\subsection{Main Result}
Now we are ready to prove the main result. %We first give the main strengthening of Lemma 3.3.
\begin{lemma}
    Let $p(x)\in S_G(\mathfrak{C})$ be a global f-generic type, $M$ a small model, and $D$ an $M$-definable set such that $p(x)\vdash x\in D$. Then the partial type $\{x\in a^{-1}bD:a,b\in G(\mathfrak{C}),a\equiv_M b\}$ is f-generic.
\end{lemma}
\begin{proof}
    %Suppose otherwise. Then there are finitely many $(a_1,b_1),\dots,(a_n,b_n)$ such that $a_i\equiv_M b_i$ for each $i$ and such that $\bigwedge_{i\in[n]}a_i^{-1}b_iD$ is not f-generic; assume we have picked $D$ so that $n$ is minimal possible. Then note in particular that $a_i^{-1}b_i\notin G(M)$ for each $i$; indeed, suppose otherwise. Then without loss of generality $a_1^{-1}b_1\in G(M)$. Let $D'$ be the $M$-definable set $D\wedge a_1^{-1}b_1 D$. call this fact $\star$.

    Suppose otherwise. Then there are $(a_1,b_1),\dots,(a_n,b_n)$ such that $a_i\equiv_M b_i$ for each $i$ and such that $\bigwedge_{i\in[n]}a_i^{-1}b_iD$ is not f-generic. By Fact 2.4, let $q(x_1,y_1,\dots,x_n,y_n)$ be a global extension of $\tp(a_1,b_1,\dots,a_n,b_n/M)$ strictly non-forking over $M$, and let $(a_{k1},b_{k1},\dots,a_{kn},b_{kn})_{k\in\omega}\models q^{\otimes\omega}|_M$ be a Morley sequence of $q$ over $M$. In particular, $(a_{k1},\dots,b_{kn})$ has the same type as $(a_1,\dots,b_n)$ over $M$ for every $k\in\omega$, so the set $\bigwedge_{i\in[n]}a_{ki}^{-1}b_{ki}D$ is not f-generic for every $k\in\omega$. Since the type $p$ is f-generic, by the pigeonhole principle there is hence some $i\in[n]$ and some infinite subset $s\subseteq\omega$ with $p(x)\vdash x\notin a_{ki}^{-1}b_{ki}D$ for all $k\in s$.

    Note that the restriction of $q$ to the variables $(x_i,y_i)$ is a global extension of $\tp(a_i,b_i/M)$ strictly non-forking over $M$. So, by Lemma 3.1, for every $n\in\omega$ there is a small model $N$ containing $(M,a_{ki},b_{ki}:k\neq n)$ and such that $a_{ni}\equiv_N b_{ni}$.
    
    In particular, if we define $c_n=a_{ni}^{-1}b_{ni}$, then $I=(c_n)_{n\in\omega}$ satisfies the hypotheses of Corollary 3.4. But now $p$ is f-generic, concentrated on $D$, and contains the formula $x\notin c_kD$ for every $k\in s$; this contradicts Corollary 3.4.
\end{proof}

\begin{corollary}
    Suppose there exists a global f-generic type, and let $M$ be a small model. Then the set $\{a^{-1}b:a,b\in G(\mathfrak{C}),a\equiv_M b\}$ is a group, and by the facts from Section 2.2 it is thus $G^{00}(\mathfrak{C})$.
\end{corollary}
\begin{proof}
    Let $p(x)\in S_G(\mathfrak{C})$ be f-generic. It suffices to show the set defined above is closed under multiplication, so fix any $a,b,c,d\in G(\mathfrak{C})$ with $a\equiv_M b$ and $c\equiv_M d$. Let $r(x)$ denote the restriction $p|_M(x)$. By Lemma 3.5 and compactness, the partial type $b^{-1}ar(x)\wedge c^{-1}dr(x)$ is f-generic, and hence in particular consistent. Let $e$ be any realization. Then $a^{-1}be$ and $d^{-1}ce$ each realize $r$, so $e^{-1}b^{-1}a$ and $e^{-1}c^{-1}d$ have the same type over $M$, and now $a^{-1}bc^{-1}d=(e^{-1}b^{-1}a)^{-1}(e^{-1}c^{-1}d)$.
\end{proof}
\begin{corollary}
Any global f-generic type is $G^{00}(\mathfrak{C})$-invariant.
\end{corollary}
\begin{proof}
    Suppose otherwise that $p(x)\in S_G(\mathfrak{C})$ is f-generic but not $G^{00}(\mathfrak{C})$-invariant. Then there is a $\mathfrak{C}$-definable set $D\subseteq G$ and an element $g\in G^{00}(\mathfrak{C})$ with $p(x)$ concentrated on $E:=D\setminus gD$. Let $N$ be any small model over which $D$ and $g$ are both defined. By Corollary 3.6, there are $a,b\in G(\mathfrak{C})$ with $a\equiv_N b$ and $g=a^{-1}b$. Since $E$ is defined over $N$ and $p(x)\vdash x\in E$, by Lemma 3.3 we have that $E\wedge a^{-1}bE$, ie $E\wedge gE$, is f-generic. But this set is contained in $gD\setminus gD=\varnothing$, a contradiction.
\end{proof} Now from Corollary 3.7 and Fact 2.5 we obtain the main theorem.
\begin{corollary}
    Suppose $G$ admits a global f-generic type. Then $G$ is definably amenable.
\end{corollary}
\begin{proof}
    By compactness in the space of Keisler measures on $G$, it suffices to show that the reduct of $G$ to any countable sublanguage of $L$ (over which $G$ remains defined) is definably amenable. Moreover, by the `$G$-dividing' characterization of non-f-genericity, the restriction of an f-generic type to the smaller language remains f-generic in the reduct. So the desired result follows from Corollary 3.7 and Fact 2.5.
\end{proof}

\newpage


%\bibliographystyle{plain}
%\bibliography{references}
\begin{thebibliography}{9}
\bibitem{adler_nip}Hans Adler. Introduction to theories without the independence property. \textit{Archive of Mathematical Logic}, 2008.

\bibitem{adler_forking}Hans Adler. A geometric introduction to forking and thorn-forking. \textit{Journal of Mathematical Logic}, 2009.

\bibitem{adler_ntp2}Hans Adler. Kim's lemma for NTP$_2$ theories. Preprint. 2014.

%\bibitem{bays_kaplan_simon}Martin Bays, Itay Kaplan, Pierre Simon. Density of compressible types and some consequences. \textit{Journal of the EMS}, 2022.

\bibitem{chernikov_kaplan}Artem Chernikov, Itay Kaplan. Forking and dividing in NTP$_2$ theories. \textit{The Journal of Symbolic Logic}, 2012.

\bibitem{chernikov_simon}Artem Chernikov, Pierre Simon. Definably amenable NIP groups. \textit{Journal of the AMS}, 2018.

%\bibitem{chernikov_simon_compressible}Artem Chernikov, Pierre Simon. Externally definable sets and dependent pairs II. \textit{Transactions of the AMS}, 2015.

\bibitem{gismatullin}Jakub Gismatullin. Model-theoretic connected components of groups. \textit{Israel Journal of Mathematics}, 2011.

\bibitem{hrushovski_peterzil_pillay}Ehud Hrushovski, Ya'acov Peterzil, Anand Pillay. Groups, measures, and the NIP. \textit{Journal of the AMS}, 2008.

\bibitem{hrushovski_pillay}Ehud Hrushovski, Anand Pillay. On NIP and invariant measures. \textit{Journal of the EMS}, 2011.

\bibitem{newelski}Ludomir Newelski. Topological dynamics of definable group actions. \textit{The Journal of Symbolic Logic}, 2009.

\bibitem{pillay_groups}Anand Pillay. Model theory and groups. Preprint, 2021.

\bibitem{shelah_forking}Saharon Shelah. Dependent first-order theories, continued. \textit{Israel Journal of Mathematics}, 2009.

\bibitem{shelah_g00}Saharon Shelah. Minimal bounded index subgroup for dependent theories. \textit{Proceedings of the AMS}, 2008.

%\bibitem{shelah_g000}Saharon Shelah. Definable groups for dependent and 2-dependent theories. 2009.

\bibitem{simon_book}Pierre Simon. A guide to NIP theories. \textit{Cambridge University Press}, 2015.

%\bibitem{simon_distal}Pierre Simon. Distal and non-distal NIP theories. \textit{Annals of Pure and Applied Logic}, 2013.

%\bibitem{simon_compressible}Pierre Simon. Type decomposition in NIP theories. \textit{Journal of the EMS}, 2016.

%\bibitem{usvyatsov}Alexander Usvyatsov. Morley sequences in dependent theories. Preprint. 2008.


\end{thebibliography}


\end{document}