\documentclass[aps,amsmath,amssymb,twocolumn,floatfix,prl,longbibliography,superscriptaddress]{revtex4-1}

%\usepackage{newtxtext,newtxmath}
%\usepackage[T1]{fontenc}

\usepackage{graphicx}
\usepackage{dcolumn}
\usepackage{bm}
\usepackage{latexsym}
\usepackage{color}
\usepackage{xcolor}
\usepackage[colorlinks=true,linkcolor=blue,citecolor=blue]{hyperref}
%\usepackage{blindtext}
%\usepackage{amssymb}
%\usepackage{amsmath}
%\usepackage[normalem]{ulem}

\newcommand{\VB}{\textcolor{orange}}
\newcommand{\LG}{\textcolor{cyan}}

\begin{document}

\title{Particle acceleration at magnetized, relativistic turbulent shock fronts}

\author{Virginia Bresci} 
\affiliation{Institut d'Astrophysique de Paris,
CNRS -- Sorbonne Universit\'e,
98 bis boulevard Arago, F-75014 Paris, France}
\affiliation{CEA, DAM, DIF, F-91297 Arpajon, France}
\affiliation{Leibniz-Institut f\"ur Astrophysik Potsdam (AIP), An der Sternwarte 16, 14482 Potsdam, Germany}
\author{Martin Lemoine} 
\affiliation{Institut d'Astrophysique de Paris,
CNRS -- Sorbonne Universit\'e,
98 bis boulevard Arago, F-75014 Paris, France}
\author{Laurent Gremillet} 
\affiliation{CEA, DAM, DIF, F-91297 Arpajon, France}
\affiliation{Universit\'e Paris-Saclay, CEA, LMCE, 91680 Bruy\`eres-le-Ch\^atel, France}

\date{\today}

\begin{abstract} 
The efficiency of particle acceleration at shock waves in relativistic, magnetized astrophysical outflows is a debated topic with far-reaching implications. Here, for the first time, we study the impact of turbulence in the pre-shock plasma. Our simulations demonstrate that, for a mildly relativistic, magnetized pair shock (Lorentz factor $\gamma_{\rm sh} \simeq 2.7$, magnetization level $\sigma \simeq 0.01$), strong turbulence can revive particle acceleration in a superluminal configuration that otherwise prohibits it. 
Depending on the initial plasma temperature and magnetization, stochastic-shock-drift or diffusive-type acceleration governs particle energization, producing powerlaw spectra $\mathrm{d}N/\mathrm{d}\gamma \propto \gamma^{-s}$ with $s \sim 2.5-3.5$. 
At larger magnetization levels, stochastic acceleration within the pre-shock turbulence becomes competitive and can even take over shock acceleration.
\end{abstract}

%\pacs{}
\maketitle

\textit{Motivation--}
The nonthermal radiative spectra observed from high-energy, relativistic astrophysical sources point to a bulk energy reservoir being dissipated into accelerated particles through, \textit{e.g.}, magnetic reconnection \citep{Guo_2014,2015SSRv..191..545K,Werner_2017, 2019ApJ...880...37P}, shock acceleration \cite{2008ApJ...682L...5S,Sironi_2015}, or turbulent Fermi processes \cite{Zhdankin+_17, CS_18,2019PhRvL.122e5101Z,Bresci+22}. As direct offsprings of the powerful outflows associated with those sources, collisionless shock waves emerge as natural dissipation agents \cite{2012SSRv..173..309B}.
Yet, in relativistic and magnetized plasmas, particle acceleration appears inhibited by the generic superluminal nature of the shock  \citep{Begelman_1990, Lemoine_2006, Niemiec+_06, 2009MNRAS.393..587P, 2009ApJ...698.1523S, Lemoine_2010, Sironi_2013}. Specifically, \emph{ab initio} particle-in-cell (PIC) numerical simulations have revealed a powerlaw tail of nonthermal particles increasingly shrinking as the magnetization parameter $\sigma$ \footnote{We define the magnetization level as $\sigma \equiv \langle B^2 \rangle/4\pi \langle \epsilon \rangle$, in terms of mean-squared magnetic field \unexpanded{$\langle B^2 \rangle$} and the energy density \unexpanded{$\langle \epsilon \rangle = n\langle \gamma \rangle m_{\rm e} c^2$} as measured in the simulation (downstream) frame; $n$ represents the total apparent density. When measured in the comoving turbulence frame, the resulting value of $\sigma$ may be a factor $\sim2$ larger.} rises above $\sim 10^{-4}$, until it vanishes at $\sigma \sim 10^{-2}$ \citep{Sironi_2013, Plotnikov+_18}. Given that a significant magnetization is expected in a wide class of high-energy astrophysical jets, {\it e.g.}  gamma-ray bursts, pulsar wind nebulae or blazars, this result challenges the role of shocks as sources of high-energy particles~\citep{Sironi+_15}.

One limitation of previous studies, though, is to systematically consider laminar in-flow conditions, \emph{i.e.}, nonturbulent, homogenous background plasmas of uniform magnetization. Now, the presence of a strong turbulence upstream of a fast shock may change the picture in various ways: it may preaccelerate the plasma particles via a stochastic Fermi process \citep{Zhdankin+_17,CS_18,2019PhRvL.122e5101Z,Bresci+22,2022PhRvL.129u5101L}, just as it may corrugate the shock front so that the turbulence does not transform trivially through the shock \citep{Zank+_02, Mizuno+_11, Lemoine_16, Lemoine+_16, Demidem+_18, Trotta+_21,2022ApJ...926..109N}, possibly unlocking particles from the field lines and enabling their acceleration. In light of these considerations, the paradigm of inefficient relativistic magnetized shocks as particle accelerators needs to be revisited in the likely common case of turbulent environments.

To this goal, we here report on the first PIC simulations of relativistic shocks propagating in turbulent, magnetized pair plasmas. We demonstrate that, despite a substantial magnetization ($\sigma \sim 0.01$), shock acceleration is manifest, and that the particle spectrum develops a powerlaw tail extending in time, which is absent without turbulence. This drastic change involves a significant ($\delta B>B_0$), but not too strong upstream turbulence, otherwise its own contribution to particle acceleration can supersede that of the shock. 

\textit{Numerical method--} We perform such simulations using the fully electromagnetic and relativistic \textsc{calder} code \citep{Calder, Lemoine_2019, Vanthieghem_22}.  Turbulence is excited close to the right-hand side of the domain, in a pair plasma
continually injected along $-\boldsymbol{\hat{x}}$ at a relativistic velocity $v_\infty=-0.87\,c$ (Lorentz factor $\gamma_\infty=2$). The flow is left to propagate across the domain until the turbulence hits its left-hand side.  Switching the local boundary condition from open to reflective at that time triggers a rightward propagating shock wave~\cite{2008ApJ...682L...5S}, which sweeps the incoming turbulent plasma. The simulation frame thus corresponds to the downstream rest frame of the shock. Due to physical constraints discussed thereafter, we restrict ourselves to 2D3V geometry (2D in space, 3D in momentum). A uniform magnetic guide field $\boldsymbol{B_0}$ is applied along the (out-of-plane) $z$ direction with corresponding magnetization level $\sigma_0$. 

\begin{figure*}
\includegraphics[width=0.9\textwidth]{complete_2D_modsigmahot_pcolor_0603.png}
        \caption{Spatial distributions of (a) apparent positron density $n_p$ (normalized to the apparent far-upstream positron density $n_0$), (b) mean kinetic energy per positron $\langle \gamma-1 \rangle$, and (c) squared turbulent magnetic field $\delta B_x^2 + \delta B_y^2$ (normalized to $4 \pi n_0 m_{\rm e} c^2$) for simulation (S2) at $t \simeq 12\,400\, \omega_p^{-1}$. In panel (d), the longitudinal profile of the turbulent magnetization $\sigma_{\delta B} = (\delta B_x^2+\delta B_y^2)/(4\pi \sum_\alpha n_\alpha \langle \gamma \rangle_\alpha m_{\rm e}c^2)$, with $n_\alpha$ the total (apparent) density of the plasma, averaged over the transverse dimension ($y$). The left and right vertical lines locate the shock front and the boundary of forced turbulence.
        }
        \label{fig:turb_shock}
\end{figure*}

The turbulence is driven in the rest frame of the drifting plasma, in a finite region covering a few stirring length scales $\ell_{\rm c}$. Elsewhere, the energy injection in the system is halted and the turbulence is let to develop freely, thereby initiating the cascade before it interacts with the shock. Following~\cite{Zhdankin+_17}, we aim at exciting turbulence on a scale $\ell_{\rm c}$ as large as possible compared to the kinetic scale $c/\omega_{\rm p}$, in order to simulate an inertial range under near-magnetohydrodynamics conditions where the turbulence cascades down to the dissipative range. Those simulations are demanding, because the interaction between the turbulence and the shock must be followed over a long enough timespan, the transverse dimension must accommodate $\gtrsim 1-2 \,\ell_{\rm c}$,  and also because the need to stir turbulence in the plasma rest frame brings in further constraints due to time dilation effects. 

In detail, the simulation domain contains $48\,000 \times 6\,000$ cells along the $x$ and $y$ axes and the simulation is run over up to $120\,000$ time steps. The mesh size is $\Delta x = \Delta y = 0.1 \,c/\omega_{\rm p}$~\footnote{$\omega_{\rm p} \equiv (4 \pi n_\infty e^2/m_{\rm e})^{1/2}$ denotes the nonrelativistic plasma frequency of the far upstream plasma, with $n_\infty = 2n_0/\gamma_\infty$ the (total) proper density, $n_0$ the  apparent density of one species.} and the time step is $\Delta t = 0.99\,\Delta x/c$. Periodic boundary conditions are used for particles and fields in the transverse direction. Turbulence is excited in the interval $4200 \leq \omega_{\rm p}x/c \leq 4800$ through external magnetic perturbations $(\delta B_x,\delta B_y)$ implemented as plane waves. Those are seeded following a Langevin antenna scheme~\citep{TenBarge_2014,Zhdankin+_17, Bresci+22}, which acts in the proper plasma frame~\citep{Supp_Mat}. The mean wavenumber $\langle k' \rangle \simeq 2.9 \times 2 \pi /L_y$ ($L_y = 600\,c/\omega_{\rm p}$ the transverse box size) implies a comoving coherence length  $\ell_{\rm c} = 2\pi/\langle k' \rangle \simeq 200\,c/\omega_{\rm p}$ (primed quantities are evaluated in the local plasma frame).  At $x\leq 4200\,c/\omega_{\rm p}$, stirring is halted, hence fluctuations evolve on a (proper) nonlinear timescale $\tau_{\rm nl}'\equiv\ell_{\rm c}/v_{\rm A} \sim 1\,200\, \omega_{\rm p}^{-1}$ ($v_{\rm A}\simeq c\sqrt{\sigma_{\delta B}} \simeq 0.17\,c$ is the Alfv\'en velocity), corresponding to $\tau_{\rm nl} =  \gamma_{\infty} \ell_{\rm c}/v_{\rm A} \sim 2\,400 \, \omega_{\rm p}^{-1}$ in the simulation frame.  

 We report here on three main simulations exploring different initial background magnetization levels $\sigma_0 \sim 10^{-4} \rightarrow 10^{-3}$, turbulent magnetization $\sigma_{\delta B} \sim 10^{-2}\rightarrow 10^{-1}$ and initial proper plasma temperatures $T$, from subrelativistic to relativistic~\citep{Supp_Mat}. In detail, $\left\{\sigma_0,\,\,\sigma_{\delta B},\,\,k_{\rm B}T/m_ec^2 \right\} = \left\{ 0.2\times10^{-3},\,\, 2 \times 10^{-2},\,\,0.1 \right\}$ [hereafter (S1)], $\left \{ 0.6 \times 10^{-3},\,\,1\times 10^{-2},\,\,4.\right\}$ [(S2)], $\left \{ 0.6 \times 10^{-3},\,\, 10^{-1},\,\,4.\right\}$ [(S3)]. 
%6\times 10^{-3}, 0.7\times10^{-3}, 
A relativistically hot initial plasma as in (S2) and (S3) could describe internal shocks inside a strongly turbulent jet. We have run ancillary simulations, in particular (S2a), similar to (S2) albeit deprived of turbulence, (S2b) which retains open boundary conditions and thus models drifting turbulence without a shock, and finally (S4), for which $k_{\rm B}T/m_ec^2=0.1$ as in (S1), but with larger $\sigma_0$ and $\sigma_{\delta B}$, as in (S2).

\textit{Results--} Figure~\ref{fig:turb_shock} displays (from top to bottom) the spatial distributions of plasma positron density, mean Lorentz factor and magnetization level at the final simulation time $t \simeq 12\,400\,\omega_{\rm p}^{-1}$ for simulation (S2). The rightward-moving shock front has then reached $x\simeq 2\,400\, c/\omega_{\rm p}$. Once swept up by the shock, the plasma is compressed by a factor of $\simeq 3.5$ and the mean kinetic energy per particle slightly increases, in good agreement with the shock-crossing conditions~\cite{Kirk-Duffy} [Figs.~\ref{fig:turb_shock}(a,b)]. Magnetic fluctuations are at their highest near the right boundary where turbulence is continually excited [Fig.~\ref{fig:turb_shock}(c)]. The $\sim 2\,000\, c/\omega_{\rm p}$ distance between the shock and the boundary of forced turbulence is then just below the minimum distance $c \tau_{\rm nl}$ needed for nonlinear evolution of the turbulence. This guarantees that the shock-turbulence interaction is not affected by the stirring procedure in the right part of the domain.

As shown in Fig.~\ref{fig:turb_shock}(d), the turbulence profile reaches an approximately steady state by the time it encounters the shock. We have checked that the spatial power spectrum of magnetic fluctuations in the transverse $y$ direction, which extends over three orders of magnitude, shows a general scaling $\propto k_y^{-5/3}$ at large scales $k_y \lesssim 15 \ell_c^{-1}$,
and a steeper behavior at kinetic scales, consistent with previous PIC studies of nondrifting decaying turbulence \citep{CS_18}.

Further upstream, corresponding to an earlier stage in the turbulence evolution, stronger fluctuations are observed, as expected. The eddies are compressed when transiting across the shock and continue interacting until the turbulence eventually relaxes further downstream. This general picture resembles that observed in MHD simulations of the interaction of a monochromatic, linear plasma eigenmode with a relativistic shock front~\citep{Demidem+_18}. 

\begin{figure}[t]
\includegraphics[width=0.47\textwidth]{allspectra_main_2.pdf}
\caption{Particle energy spectra $\gamma^2 {\rm d}N/{\rm d} \gamma$ at different times (from light to dark solid blue) in simulations (S1), (S2) and (S3), from top to bottom. Middle panel: in dotted line, spectrum from (S2a), {\it i.e.}, a shock interacting with a non-turbulent plasma, and in dashed line, spectrum from (S2b), {\it i.e.}, a drifting turbulence without a shock, both in conditions otherwise similar to (S2). In the lower panel, the light orange band delineates the range of spectra measured in a simulation similar to (S3) albeit without a shock, as extracted at various places and times.}
\label{fig:spectrum_evolution}
\end{figure} 

The shock moves at velocity $\simeq 0.4\,c$ in the simulation (downstream) frame ($0.36\,c$ is predicted by the shock-crossing conditions \citep{Kirk-Duffy}), and consequently at $v_{\rm sh} \simeq 0.93\,c$ in the upstream frame, corresponding to a shock Lorentz factor $\gamma_{\rm sh} \simeq 2.7$. Ahead of the shock, the transversely averaged magnetic field strength is $\delta B/(m_{\rm e} \omega_{\rm p} c/e) \simeq 0.8$, so that particles with Lorentz factor $\gamma \simeq 100-300$ have a gyroradius $r_{\rm g} \simeq 120-360\,c/\omega_{\rm p}$.

Figure~\ref{fig:spectrum_evolution} plots the time evolution of the particle energy spectra $\gamma^2 {\rm d}N/{\rm d} \gamma$ (per log-interval of energy) in each of the three simulations, as integrated over a moving window centered on the shock front position $x_{\rm sh}$ (located from the plasma density map) and with a $200 \,c/\omega_{\rm p}$ half width along $x$. Those spectra peak at energies consistent with shock-heating of the incoming plasma. Remarkably, a suprathermal tail develops in all cases, with approximate powerlaw index $s\simeq 2.5\rightarrow 3.5$ (as defined through ${\rm d}N/{\rm d}\gamma\propto\gamma^{-s}$), providing manifest evidence of particle acceleration. In (S1) and (S2), and unlike in (S3), the maximal energy is seen to increase with time.
 
\begin{figure}[t]
        \includegraphics[width=0.45\textwidth]{plot3_2812-crop.pdf}
        \caption{Position $x$ along the shock normal (a) and Lorentz factor $\gamma$ (b) versus time for $5$ particles with initial Lorentz factor $30<\gamma<100$ in simulation (S2). The dashed curve in (a) indicates the shock front position $x_{\rm sh}(t)$.}  
       \label{fig: ptest}
\end{figure} 
 
The middle panel of Fig.~\ref{fig:spectrum_evolution} presents additional spectra from the turbulence-free simulation (S2a), thus where the shock forms immediately in the external field $\boldsymbol{B_0}$ (dotted line), and from shock-free simulation (S2b), which contains only drifting turbulence (dashed line). Clearly, the suprathermal tail only arises when the shock interacts with the turbulence. The absence of particle acceleration in (S2a) can be attributed to the background magnetization level $\sigma \equiv \sigma_0 \simeq 10^{-3}$, large enough to inhibit Fermi cycles around the shock. In (S2b), the magnetic fluctuations are too slow to accelerate particles, as the characteristic acceleration timescale $t_{\rm acc} \sim \gamma_{\infty} \,c \ell_{\rm c}/v_{\rm A}^2 \sim 10^4 \,\omega_{\rm p}^{-1}$ indeed exceeds the time needed for the plasma to cross the domain. Note that the spectrum of (S2b) coincides with that of (S2) at $t=5940\,\omega_{\rm p}^{-1}$, because the plasma has then just hit the reflective wall, and the shock has not formed yet.

\textit{Particle acceleration--} Standard theory depicts acceleration at a magnetized shock front as the result of pitch-angle diffusion back and forth across the shock, or of shock-drift motion along the mean advected electric field \citep{87Blandford}. In the relativistic limit, shock acceleration is ineffective~\cite{Begelman_1990,Lemoine_2006} unless intense turbulence can unlock particles off the field lines~\cite{Lemoine_2010}, and thereby trigger diffusive-type acceleration~\cite{Kirk+_00, Achterberg+_01}, a form of stochastic-shock-drift process~\cite{Takamoto&Kirk_2015, Matsumoto17}, or a combination of both, meaning orbits in the regular field upstream of the shock, diffusive orbits downstream~\cite{2009MNRAS.393..587P}.

To probe the acceleration process at work, we have tracked a large number $\sim\mathcal{O}( 10^5)$ of particles sampled in various (initial) energy intervals. Figure~\ref{fig: ptest} shows the trajectories and energy histories of four particles in (S2), representative of the population able to circulate around the shock for an extended period of time. The Lorentz factor of some particles (e.g. orange and cyan in that figure) undergoes sizable oscillations before they encounter the shock; this results from their gyromotion along the fast-moving magnetic field lines~\cite{Wong+_20}, not from acceleration \emph{per se}. Notwithstanding this effect, the energization of the particles traveling in the vicinity of the shock is evident.  

We discriminate the acceleration process in simulations (S1) and (S2) using the following argument. In a shock-drift process, the energy gained by a particle of velocity $\boldsymbol{v}$ equals the amount of work performed by the mean motional electric field $\boldsymbol{E_0} = v_\infty B_0\,\boldsymbol{\hat y}$, \emph{i.e.}, $W(E_0)=q\int{\rm d}t\, \boldsymbol{v}\cdot\boldsymbol{E_0}$, while for diffusive-type acceleration, the energy gain is rather related to the work $W(\delta E_z)= q\int {\rm d}t\, v_z \delta E_z$ performed by the motional turbulent electric field. The latter is mostly directed along $\boldsymbol{\hat z}$ because the plasma, which flows along $-\boldsymbol{\hat x}$, carries essentially $(\delta B_x,\,\delta B_y)$ magnetic fluctuations; hence $\delta E_z \simeq - v_\infty \delta B_y$. For each tracked particle with initial Lorentz factor at the onset of the powerlaw tail, {\it i.e.} $\gamma \geq 20$ [in (S1)] and $\gamma\geq200$ [(S2)], we have thus recorded $W(E_0)$ and $W(\delta E_z)$ during the time interval $\Delta t_{\rm sh}$ between the first and last encounters of the particle with the shock front. 

\begin{figure}
        \includegraphics[width=0.47\textwidth]{kdescatter_log_ncycles_deltagamWE0ydeltaEz_hotANDcold_0603_noleg.png}
    \caption{Correlation between the measured variation in Lorentz factor $\Delta\gamma$ and that predicted by shock-drift acceleration [$W(E_0)/m_{\rm e} c^2$, top row (a) and (c)] or diffusive-type acceleration [$W(\delta E_z)/m_{\rm e} c^2$, bottom row (b) and (d)] in simulations (S1) [left column, (a) and (b)] and (S2) [right column, (c) and (d)]. Orange dots represent individual measurements which are interpolated by the pseudocolor density plot.)
    }
    \label{fig: dgammavsdy}
\end{figure} 

Figure~\ref{fig: dgammavsdy} shows the correlations between the observed variation in Lorentz factor $\Delta \gamma$ during $\Delta t_{\rm sh}$ and $W(E_0)/m_{\rm e} c^2$ [top row, (a) and (c)], as well as $W(\delta E_z)/m_e c^2$ [bottom row, (b) and (d)], for simulations (S1) [left column, (a) and (b)] and (S2) [right column, (c) and (d)]. In these plots, the dashed red line indicates the expected level of contribution from either shock-drift or diffusive-type acceleration. Figures~\ref{fig: dgammavsdy}~(c) and (d) reveal that the shock-drift process nicely accounts for particle energization in (S2), while in (S1), acceleration appears dominated by diffusive-type acceleration. 

 This general picture is further supported by the angular distribution of the suprathermal particle momenta~\citep{Supp_Mat}. In (S2), this angular map presents a clear asymmetry between positrons and electrons, roughly polarized along $\boldsymbol{E_0}$, such that positrons (resp. electrons) appear to drift with negative (resp. positive) $p_y$, as expected for $E_0 < 0$. By contrast, the angular map is significantly more isotropic in (S1), as expected if diffusive-type acceleration dominates. Additionally, simulation (S2) displays a net linear correlation between $\Delta \gamma$ and $\Delta t_{\rm sh}$, from which one can infer an acceleration timescale, $t_{\rm acc} \equiv \vert\langle\Delta\gamma/\gamma\rangle\vert ^{-1}\Delta t_{\rm sh} \sim 2-3\, p/(e E_0)$, again consistent with that expected for particles drifting along $\boldsymbol{E_0}$ at mildly relativistic speeds. 

The difference in spectral index observed between simulations (S1) ($s\sim 2.5$) and (S2) ($s\sim 3.5$) likely results from the distinct acceleration mechanisms at play. In particular, the spectral index for stochastic-shock-drift acceleration -- at least, in subrelativistic shocks -- depends sensitively on the shock speed and on the ratio of the scattering frequency of particles to their gyrofrequency in the mean field~\citep{Takamoto&Kirk_2015}. At large scattering frequencies, the index approaches the canonical value of $2$, associated with diffusive shock acceleration, while at small scattering frequencies, the spectrum steepens significantly, encompassing the value measured in (S2). 

Regarding the different acceleration processes in (S1) and (S2), we observe that the $r_{\rm g}$ vs $\ell_{\rm c}$ ordering, which controls the scattering rate of particles, varies between those two simulations because of different initial temperatures and magnetizations: $r_{\rm g}/ \ell_{\rm c} \sim 0.1$ at the onset of the powerlaw tail for (S1), while $r_{\rm g} /\ell_{\rm c} \sim 1$ for (S2). A detailed study of the ancillary simulation (S4), which shares the same initial temperature as (S1) and same magnetization as (S2), reveals, however, that %\LG
although a powerlaw index similar to that in (S2) is obtained,
shock-drift and diffusive-type processes now contribute in about equal amounts to particle energization; accordingly, the angular map is less anisotropic than in (S2), more than in (S1). Overall, this suggests that both plasma temperature and magnetization may influence the prevalence of one mechanism over the other. While a comprehensive explanation for this change of regime certainly deserves further investigation, we emphasize that the main result of the present work, namely, the formation of a nonthermal spectrum at relativistic, magnetized turbulent shocks, is a robust feature.

Let us finally address simulation (S3), characterized by a relativistically hot initial plasma and a substantial magnetization $\sigma_{\delta B}\sim 0.1$~\cite{Supp_Mat}. This simulation probes a new regime in which stochastic Fermi acceleration inside the pre-shock turbulence controls the acceleration process, because the (stochastic) acceleration timescale $t_{\rm acc} \sim \gamma_\infty \ell_{\rm c}/\sigma_{\delta B}c$ now becomes short enough ($\simeq 10^3\,\omega_{\rm p}^{-1}$) to energize the freshly injected particles before they attain the shock. Accordingly, the spectrum shown in Fig.~\ref{fig:spectrum_evolution} (lower panel) does not vary with time because the turbulence inside the simulation box is stationary, up to its fluctuations. This figure also reveals that the particle distribution has undergone significant heating beyond the simple shock-crossing conditions, as can be seen by direct comparison to simulation (S2) (middle panel). Furthermore, we have verified that the same simulation, albeit with open boundary conditions to prevent shock formation, yields a similar spectrum~\cite{Supp_Mat}. Finally, the spectral index $s \sim 3.5$ falls in line with that observed in PIC simulations of turbulence in the semi-relativistic regime~\cite{Zhdankin+_17, CS_18, Bresci+22}. 

\textit{Discussion--} Interestingly, the range of magnetizations that we consider here, $\sigma \sim 0.01\rightarrow 0.1$, and the range of spectral indices that we measure, $s\sim 2.5 \rightarrow 3.5$, appear rather typical of what is inferred from one-zone models of blazars~\citep{Celotti-Ghisellini_08} and gamma-ray bursts~\citep{Burgess+_2020}. This supports the idea that mildly relativistic shocks interacting with magnetized turbulence can play a leading role in dissipation and particle acceleration in a broad range of relativistic high-energy sources, up to moderate magnetizations. Our study thus significantly extends the realm where relativistic shock acceleration can operate without turbulence (\emph{i.e.} $\sigma\ll 10^{-4}$). As stochastic turbulent acceleration  is observed to take over shock acceleration at $\sigma\gtrsim 0.1$, one is tempted to sketch a picture in which, as the magnetization level rises, a shock, or a shock plus turbulence, then turbulence and eventually magnetic reconnection control dissipation and acceleration.

As noted, the present simulations are computationally expensive, which limits a broad parameter study. Future works should explore a larger parameter range, in particular larger dimensions (and dimensionalities) in order to examine how the particle spectrum changes with increasing $\ell_{\rm c}$, to make better contact with phenomenology.

\begin{acknowledgments}
This work has been supported by the Sorbonne Universit\'e DIWINE Emergence-2019 program and by the ANR (UnRIP project, Grant No.~ANR-20-CE30-0030). V.~B. acknowledges support by the European Research Council under ERC-AdG Grant PICOGAL-101019746. This work was granted access to the HPC resources of TGCC under the
allocations 2019-A0050407666, 2020-A0080411422, 2021-A0080411422 and 2022-A0130512993 made by GENCI. We thank X. Davoine for his assistance on particle-tracking diagnostics.
\end{acknowledgments}

After this manuscript was submitted, Ref.~\cite{2022arXiv221206053D} appeared, which shows that the interaction of a shock with a monochromatic linear eigenmode of the upstream plasma leads to particle acceleration in the resultant downstream turbulence.

%merlin.mbs apsrev4-1.bst 2010-07-25 4.21a (PWD, AO, DPC) hacked
%Control: key (0)
%Control: author (0) dotless jnrlst
%Control: editor formatted (1) identically to author
%Control: production of article title (0) allowed
%Control: page (1) range
%Control: year (0) verbatim
%Control: production of eprint (0) enabled
\bibliography{refs}





\end{document}