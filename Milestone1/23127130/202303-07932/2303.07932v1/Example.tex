\section{Example}
\label{sec:example}
In this section, the developed approach of feedforward for LPV systems is validated on an example.
%%%%%%%%%%%%%%%%%%%%%%%%%%%%%%%%%%%%%%%%%%%%%%%%%%%%%%%%%%%%%%%%%%%%%%%%%%%%%%%%%%%%%%%%%%%%%%%%%%%%%%%%%%%%%%%%%%%%%%%%%%%%%%%%%%%%%%
\subsection{Example Setup}
The two-mass-spring-damper in \figRef{fig:LPVMSD} is considered. The feedback controller $C$ is a lead filter. The system is seen in \eqref{eq:exampleIO} and \figRef{fig:LPVMSD}, with damper constants $c=1$ and $c_2=10^{-4}$ Ns/m, masses $m_1=1$ and $m_2=0.5$ kg. Stiffness $k(\rho)$ is 
\begin{equation}
	k(\rho) = \frac{EA}{\rho(L-\rho)},
\end{equation}
with length $L=1$ m, Young's modulus $E=0.24\cdot10^9$ Pa and area $A=1\cdot10^{-5}\text{ m}^2$. The reference $r$ is chosen as a fourth order point-to-point motion, as in \citet{Lambrechts2005}, consisting of 1810 samples. The scheduling sequence $\rho$ used is the reference itself, and ranges from 0.2 m to 0.8 m. The feedforward parameters $\theta_i(\rho)$ in \eqref{eq:FFModel} are identified according to Procedure~\ref{proc:1}.
%%%%%%%%%%%%%%%%%%%%%%%%%%%%%%%%%%%%%%%%%%%%%%%%%%%%%%%%%%%%%%%%%%%%%%%%%%%%%%%%%%%%%%%%%%%%%%%%%%%%%%%%%%%%%%%%%%%%%%%%%%%%%%%%%%%%%%
\subsection{Compared Approaches}
The following three approaches are compared in feedforward to evaluate the developed framework.
\begin{description}
	\item[LTI] Completely ignoring the LPV dynamics of the system and using static feedforward parameters as in \eqref{eq:FFPolLTI}, where the feedforward parameters are taken as the true parameters for $\rho=0.5$ m, i.e.
	\begin{equation}
		\label{eq:LTIFFsim}
		F_{LTI}: \quad u_{ff} = \theta_1 \dot{r} + \theta_2 \ddot{r} + \theta_3 \ddddot{r}.
	\end{equation}
	\item[Static LPV] Including the LPV effects in the standard polynomial snap feedforward, but ignoring the additional terms which arise due to the chain and product rule of integration \Citep{VanHaren2022a}, i.e.,
	\begin{equation}
		\label{eq:staticLPVFF}
		F_{SLPV}: \quad u_{ff}  = \theta_1 \dot{r} + \theta_2 \ddot{r} + \theta_3(\rho) \ddddot{r}.
	\end{equation}
	\item[Dynamic LPV] Application of the developed \mbox{feedforward} approach in \eqref{eq:FFModel} and \eqref{eq:FFCalc} with dynamic snap feedforward, i.e.,
	\begin{subnumcases}{F_{LPV}:}
		w_{ff}= \theta_1 \int {r} \, dt + \theta_2 {r} + \theta_3(\rho) \ddot{r}., \label{eq:FFModelSim}
		\\
		u_{ff} = \theta_1 \dot{r} + \theta_2 \ddot{r} + \theta_3(\rho) \ddddot{r}+u_{dyn}, \label{eq:FFCalcSim}
	\end{subnumcases}
with 
\begin{equation}
	\label{eq:udyn}
	u_{dyn} = \ddot{\rho}\theta^\prime_3(\rho)\ddot{r}+\dot{\rho}^2\theta_3^{\prime\prime}(\rho)\ddot{r}+2\dot{\rho}\theta_3^\prime(\rho)\dddot{r}.
\end{equation}
\end{description}
The parameters $\theta_i(\rho)$ are identified using the developed framework, where, for simplicity,  the kernel is chosen block-diagonal, i.e., $K_{ij}=0 \, \forall i\neq j$, meaning different feedforward parameters $\theta_i$ and $\theta_j \, \forall i\neq j$ do not correlate, but individual feedforward parameters $\theta_i$ on different values of $\rho$ do correlate. The kernels $K_{11}$ and $K_{22}$ are chosen to be identity matrices of appropriate size, i.e., parameters $\theta_1$ and $\theta_2$ are constant. The kernel $K_{33}$ is chosen as the squared exponential kernel from \eqref{eq:SEKernel}, where the $\sigma_{33}^2$ and $\ell_{33}$ are optimized using marginal likelihood optimization.
%%%%%%%%%%%%%%%%%%%%%%%%%%%%%%%%%%%%%%%%%%%%%%%%%%%%%%%%%%%%%%%%%%%%%%%%%%%%%%%%%%%%%%%%%%%%%%%%%%%%%%%%%%%%%%%%%%%%%%%%%%%%%%%%%%%%%%
%\subsection{Identification of the Feedforward Parameters}
%The feedforward parameters are identified using the developed approach for the two-mass-spring-damper. For simplicity, the kernel is chosen as a block-diagonal matrix, i.e., $K_{ij}=0 \, \forall i\neq j$, meaning different feedforward parameter do not correlate. The kernels $K_{11}$ and $K_{22}$ are chosen to be identity matrices of appropriate size, i.e., the resulting feedforward parameters are constant.  An example of the parameter fit $\theta_3(\rho)$ using the developed framework can be seen in \figRef{fig:paraFitExample}.

%%%%%%%%%%%%%%%%%%%%%%%%%%%%%%%%%%%%%%%%%%%%%%%%%%%%%%%%%%%%%%%%%%%%%%%%%%%%%%%%%%%%%%%%%%%%%%%%%%%%%%%%%%%%%%%%%%%%%%%%%%%%%%%%%%%%%%
\subsection{Results} 
In this section, the results of the example are shown. First, in \figRef{fig:timeDomainError}, an error plot is shown for the three feedforward approaches. 
%Second, in \figRef{fig:paraFitExample}, the parameter fit of $\theta_3(\rho)$ is shown. 
Second, in \figRef{fig:dynContribution}, the contribution of the developed feedforward approach due to the dynamic dependency is shown. Third, the contribution of snap feedforward for both static LPV and the developed feedforward approach is shown in \figRef{fig:staticvsdyamic}. Finally, a surface plot of the true and estimated dynamic dependent snap feedforward is shown in \figRef{fig:surface}. The following observations are made:
\begin{itemize}
	\item \figRef{fig:timeDomainError} shows that the best tracking performance is achieved by the developed approach, while the static LPV feedforward performs better than LTI feedforward. The 2-norm of the errors are respectively $5.8\cdot10^{-8}$ m,  $2.5\cdot10^{-6}$ m and $4.2\cdot10^{-6}$ m.
	%\item \figRef{fig:paraFitExample} shows that the snap feedforward parameter $\theta_3(\rho)$ is estimated accurately.
	\item The LPV snap parameter $\frac{m_1m_2}{k(\rho)}$ is estimated accurately, with a root-mean-square error of $4\cdot10^{-5}$ kgs$^2$.
	\item \figRef{fig:dynContribution} and \figRef{fig:staticvsdyamic} show that the contribution of dynamic feedforward is not zero, hence should be taken into account.
%	\item The contribution to the feedforward due to dynamic dependence on the scheduling sequence is not zero, hence should be taken into account.
%	\item There is a difference between static LPV feedforward and the developed approach, which includes the dynamic dependence on the scheduling sequence.
	\item For a high contribution of the dynamic feedforward in \figRef{fig:dynContribution}, the tracking error for static LPV in \figRef{fig:timeDomainError} increases, showing that the dynamic dependence has significant effect on the tracking error.
	\item \figRef{fig:surface} shows that, for the reference, the dynamic contribution to the feedforward is estimated accurately.
\end{itemize}
\begin{figure}[tb]
	\centering
	\hspace{-15mm}
	\includegraphics{pdf/ErrorTime2.pdf}
	\caption{Time-domain error for LTI \li{red}{densely dashed}[1.5], static LPV \li{yel}{densely dotted}[1.5] and developed \li{blue}{solid}[1.5] feedforward approaches.}
	\label{fig:timeDomainError}
\end{figure}
%\begin{figure}[tb]
%	\centering
%	\includegraphics{pdf/GPTheta22.pdf}
%	\caption{True parameter $\theta_3(\rho)$ \li{blue}{solid}[2] and estimated $\hat{\theta}_3(\rho)$ \li{red}{densely dashed}[2] with developed framework.}
%	\label{fig:paraFitExample}
%\end{figure}
\begin{figure}[tb]
	\centering
	\includegraphics{pdf/FFContribution.pdf}
	\caption{Left axis: Dynamic dependent feedforward $u_{dyn}$ from \eqref{eq:udyn} \li{blue}{solid}[1.5]\hspace{-1mm}. Right axis: Total feedforward \li{red}{densely dashed}[1.5]\hspace{-1mm}.}
	\label{fig:dynContribution}
\end{figure}
\begin{figure}[tb]
	\centering
	\hspace{-15mm}
	\includegraphics{pdf/AppliedFF2.pdf}
	\caption{Feedforward contribution of static LPV $\theta_3(\rho)\ddddot{r}$ \li{red}{densely dashed}[1.5] and developed approach $\frac{d^2}{dt^2}\left(\theta_3(\rho) \ddot{r}\right) $\li{blue}{solid}[1.5]. Note that the difference is solely caused by the dynamic scheduling dependency.}
	\label{fig:staticvsdyamic}
\end{figure}

\begin{figure}[tb]
	\centering
	\includegraphics{pdf/surfDynamic.pdf}
	\caption{True dynamic contribution $u_{dyn}$ from \eqref{eq:udyn} and estimated given the data \li{red}{solid}[2] for $\ddot{r}=1$ and $\dddot{r}=20$.}
	\label{fig:surface}
\end{figure}



%%%%%%%%%%%%%%%%%%%%%%%%%%%%%%%%%%%%%%%%%%%%%%%%%%%%%%%%%%%%%%%%%%%%%%%%%%%%%%%%%%%%%%%%%%%%%%%%%%%%%%%%%%%%%%%%%%%%%%%%%%%%%%%%%%%%%
%old
%The estimated parameters are compared to the true parameters by calculating the Normalized Root Mean Squared Error (NRMSE), which is calculated as
%\begin{equation}
%	\alert{\frac{1}{\theta_i\diamond \rho_0}\sqrt{\frac{1}{N}\left\|\overline{\theta}_i-\hat{\overline{\theta}}_i \right\|_2 }}
%\end{equation}
%where $\rho_0$ is chosen as the initial scheduling sequence, in this case 0.2 m. The NRMSE for 100 Monte Carlo simulations with different levels of output noise are seen in \tabRef{tab:RMSEExample}.
%\tabRef{tab:RMSEExample} shows that for larger levels of noise, the resulting performance in terms of parameter estimation deteriorate. This can be explained because taking a derivative of the output when there is noise present typically gives bad results, and additionally, the solution that is minimized is an open-loop one while the system is in closed-loop, which results in bias, as discussed in Remark~\ref{rem:bias}.
%\begin{table}[hb]
%	\begin{center}
%		\caption{Average of the NRMSE on the parameters $\theta_1$ and $\theta_2$ for different levels of output noise after 100 Monte Carlo Simulations. }
%		\label{tab:RMSEExample}
%		\begin{tabular}{llll}
%			& \multicolumn{3}{c}{NRMSE}                                                                                                                                                              \\ \cline{2-4} 
%			& $\sigma_\nu=0$ 	& $\sigma_\nu=10^-8$ 			& $\sigma_\nu=10^-6$ \\ 
%			\toprule
%			$\theta_1$ 	&   $4.2\cdot 10 ^{-6}$    	&   $4.4\cdot 10 ^{-5}$     	& $2.6\cdot 10 ^{-3}$      \\
%			$\theta_2$ 	&   $2.5\cdot10^{-7}$     &   $2.6\cdot10^{-7}$  &   $3.6\cdot 10 ^{-5}$       \\ 
%			$\theta_3$ 	&   $8.9\cdot 10 ^{-6}$     &   $9.2\cdot 10 ^{-6}$ &   $2.7\cdot 10 ^{-3}$       \\ 
%			\bottomrule
%		\end{tabular}
%	\end{center}
%\end{table}

%A histogram of the 2-norm of the tracking error for 250 Monte Carlo simulations for the developed approach and the two compared feedforward strategies can be seen in \figRef{fig:errorHisto}. 
%\begin{figure*}[tb]
%	\centering
%	\includegraphics{pdf/ErrorHisto.pdf}
%	\caption{Histogram of tracking error for 250 Monte Carlo simulations for LTI feedforward (\filledSquare{blue}[0.8]), static LPV feedforward (\filledSquare{red}[0.8]) and developed dynamic LPV feedforward (\filledSquare{yel}[0.8]). The output noise has variance $\sigma_\nu=10^{-8}$.}
%	\label{fig:errorHisto}
%\end{figure*}
%The error histogram in \figRef{fig:errorHisto} shows that the tracking error for static LPV feedforward already reduces the error 2-norm significantly, approximately by a factor 1.3. The error can be further reduced by applying the developed approach, showing an error 2-norm reduction of approximately factor 2 compared to LTI feedforward.