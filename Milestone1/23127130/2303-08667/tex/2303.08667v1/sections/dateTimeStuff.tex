%
%
%

\usepackage{xstring} %
\usepackage[calc,english]{datetime2}

%
%
\DTMnewdatestyle{US-chicago-manual-style-full-date}{%
\renewcommand*{\DTMdisplaydate}[4]{\DTMenglishordinal{##3} \DTMenglishshortmonthname{##2} ##1}%
\renewcommand*{\DTMDisplaydate}{\DTMdisplaydate}%
}

%
\DTMnewdatestyle{US-chicago-manual-style-year-month}{%
\renewcommand*{\DTMdisplaydate}[4]{\DTMenglishmonthname{##2} ##1}%
\renewcommand*{\DTMDisplaydate}{\DTMdisplaydate}%
}

%
\DTMnewdatestyle{year-comma-short-month}{%
\renewcommand*{\DTMdisplaydate}[4]{##1, \DTMenglishshortmonthname{##2}}%
\renewcommand*{\DTMDisplaydate}{\DTMdisplaydate}%
}

%
\newcommand{\setdefaultdatetimeformat}{%
\DTMsetdatestyle{US-chicago-manual-style-full-date}%
\DTMsettimestyle{iso}%
\DTMsetup{showseconds=true}%
}
\setdefaultdatetimeformat

%
\makeatletter
\DeclareRobustCommand{\datetime}[2]{%
\StrCount{#1}{-}[\@numberOfHyphens]%
\StrCount{#2}{:}[\@numberOfColons]%
%
\ifnum\@numberOfHyphens>0
\ifnum\@numberOfHyphens>1
\DTMdate{#1}%
\else
\DTMsetdatestyle{US-chicago-manual-style-year-month}%
\DTMdate{#1-01}%
\setdefaultdatetimeformat%
\fi
\ifx#2""\else %
\xspace%
\fi
\fi
\ifnum\@numberOfColons>0
\ifnum\@numberOfColons>1
\DTMsetup{showseconds=true}%
\DTMtime{#2}%
\setdefaultdatetimeformat%
\else
\DTMsettimestyle{englishampm}%
\DTMtime{#2:00}%
\setdefaultdatetimeformat%
\fi
\fi
}
\makeatother