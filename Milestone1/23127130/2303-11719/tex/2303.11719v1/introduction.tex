\section{Introduction}
%%%%%%%%%%%
%\FHO{Je prefererais de mettre nos resultats en forme de theoreme dans l'introduction et de juste les citer ailleurs. Qu'en pensez-vous?}
Notation not given below is consistent with \cite{bang2009}. We denote by $[k]$ the set $\{1,2, \dots, k\}$.


A {\bf feedback arc set}  in a digraph is a set of arcs whose reversal results in an acyclic digraph. 
Finding a mininum cardinality feedback arc set is an important and heavily studied problem. It is one of the first problems shown to be NP-hard listed by Karp in~\cite{karp1972}. 
Furthermore, it is hard to approximate. For arbitrary digraphs, the best
known ratio is $\bigO(\log n \log \log n)$~\cite{EvenNSS95}. It is APX-hard~\cite{KannThesis} and, unless P is
equal to NP, it cannot be approximated within a factor of better than 1.36~\cite{DiSa05}.
For tournaments, the problem remains NP-complete \cite{alonSJDM20,charbitCPC16}, but there is a $3$-approximation algorithm~\cite{ACN08} and a polynomial-time approximation scheme~\cite{HowSTOC07}.% that admits an efficient computation of a
%solution that is at most by a factor of $1 + \epsilon$ worse than the optimum, for any $\epsilon >0$.



\medskip

To make a digraph $D$ acyclic, one can use a different operation from arc reversal, called inversion.
The {\bf inversion} of a set $X$ of vertices consists in reversing the direction of all arcs of $D\langle X\rangle$.
We say that we {\bf invert} $X$ in $D$. The resulting digraph is denoted by $\Inv(D;X)$.  
If $(X_i)_{i\in I}$  is a family of subsets of $V(D)$, then $\Inv(D; (X_i)_{i\in I})$ is the digraph obtained after inverting the
$X_i$ one after another. Observe that this is independent of the order in which we invert the $X_i$~: $\Inv(D; (X_i)_{i\in I})$ is obtained from $D$ by reversing the arcs such that an odd number of the $X_i$ contain its two end-vertices.
A {\bf decycling family} of a digraph $D$ is a family $(X_i)_{i\in I}$ of subsets of $V(D)$ such that
 $\Inv(D; (X_i)_{i\in I})$ is acyclic.
A digraph admits a decycling family if and only if it does not contain a {\bf digon}, that is, a pair of arcs in opposite direction between the same two vertices. Indeed, observe that an inversion operation changes the orientation of either none or both of the arcs in a digon. Hence, a digraph containing a digon cannot be made acyclic by inversions.
 Conversely, in an oriented graph, the pairs of end-vertices of the arcs of a feedback arc-set form a decycling family.
 The {\bf inversion number} of an oriented graph $D$, denoted by $\inv(D)$,  is the minimum number of inversions needed to transform $D$ into an acyclic oriented graph, that is, the minimum cardinality of a decycling family.
It was first introduced by Belkhechine et al. in \cite{BBBP10} and then studied in several papers~\cite{BCH,PST,inversion,APSSW}.
In particular, Belkhechine et al. in \cite{BBBP10} proved that, for any fixed integer $k$, deciding whether a  given tournament has inversion number at most $k$ is polynomial-time solvable. In contrast, Bang-Jensen et al.~\cite{BCH} proved that deciding whether a given digraph has inversion number $1$ is NP-complete. This was generalized by Alon et al.~\cite{APSSW}:  for any fixed positive integer $k$, deciding whether a  given digraph has inversion number at most $k$ is NP-complete.
The maximum $\inv(n)$ over all inversion numbers of digraphs of order $n$ has also been investigated. Independently, Aubian et al.~\cite{inversion} and Alon et al.~\cite{APSSW} proved
$n - 2\sqrt{n\log n} \leq \inv(n) \leq n - \lceil \log (n+1) \rceil$.

\medskip
The main purpose of this article is to study the possibilities of applying the inversion operation to obtain a different objective than the obtained digraph being acyclic. Instead of making a digraph acyclic, we are interested in making it satisfy a prescribed connectivity property.
A digraph $D$ is {\bf strongly connected} or simply {\bf strong} (resp. {\bf $k$-arc-strong}) for some positive integer $k$, if for any partition $(V_1, V_2)$ of $V(D)$ with $V_1, V_2\neq \emptyset$ there is an arc (resp. at least $k$ arcs) with tail in $V_1$ and head in $V_2$.
For a given digraph $D$, we denote by $\UG(D)$ the undirected (multi)graph that we
obtain by suppressing the orientations of the arcs.  A digraph is {\bf $k$-connected} (resp. {\bf $k$-edge-connected}) if its underlying (multi)graph is.
Clearly, a digraph $D$ can be made $k$-arc-strong by reversing some arcs if and only if the edges of $\UG(D)$ can be oriented such that the resulting digraph
is $k$-arc-strong.
Robbins' Theorem~\cite{Rob1939} asserts that a graph admits a strong orientation if and only if it is 2-edge-connected, and more generally, Nash–Williams’ weak orientation theorem~\cite{NashW60}, asserts that a graph admits a $k$-arc-strong orientation 
if and only if it is $2k$-edge-connected.
%\begin{theorem}[Nash-Williams~\cite{NashW60}]
%A graph admits a $k$-arc-strong orientation if and only if it is $2k$-edge-connected.
%\end{theorem}
 It is well known that, by reducing to a minimum-cost submodular flow problem, one can determine, in polynomial time, a minimum set of arcs in $D$ whose reversal gives a $k$-arc-strong digraph or detect that such a set does not exists, see  Section 8.8.4 of \cite{bang2009} for details.
%For an arbitrary digraph $D$ the size of minimum set of arcs in $D$ whose reversal gives a $k$-arc-strong digraph may depend on $n$, the number
%of vertices of $D$ (for example for a digraph having a linear number of sinks, vertices with in-degree $0$).
It is easy to see that the number of necessary arc reversals to make a $2k$-edge-connected digraph $D$ $k$-arc-strong cannot be bounded by a function depending only on $k$. For example, one can consider digraphs that contain a number of sinks which is linear in the number of vertices of the graph, where a {\bf sink} is a vertex with no outgoing arc.
However, Bang-Jensen and Yeo~\cite{BaYe04} proved that for {\bf tournaments}, which are the orientations of complete graphs, the size of such a set is always bounded by a quadratic function of $k$. Precisely, they showed that every tournament on at least $2k+1$ vertices can be made $k$-arc-strong by reversing at most $\frac{1}{2}k(k-1)$ arcs. This result is tight for the transitive tournaments.

We are interested in the problem of using inversions to make a digraph $k$-arc-strong.
A {\bf $k$-arc-strengthening family} of a digraph $D$ is a family $(X_i)_{i\in I}$ of subsets of $V(D)$ such that
 $\Inv(D; (X_i)_{i\in I})$ is $k$-arc-strong.
The {\bf $k$-arc-strong inversion number} of a digraph $D$, denoted by $\sinv'_k(D)$, is the minimum number of inversions needed to transform $D$ into a $k$-arc-strong digraph, that is, the minimum cardinality of a $k$-strengthening family.
 We first deal with the extremal behaviour of $\sinv'_k(D)$ for some fixed $k$, that is, we deal with the question of finding the maximum number of necessary inversions to make a $2k$-edge-connected digraph on a fixed number of vertices $k$-arc-strong. To this end, we define $\sinv'_k(n) = \max\{\sinv'_k(D) \mid D~$\mbox{$2k$-edge-connected digraph of order $n$}$\}$. It turns out that $\sinv_k'(n)$ is an unbounded, but slowly growing function. We are able to determine $\sinv_k'(n)$ up to a constant multiplicative factor of roughly 2. More precisely, we show the following result:

%$$ \frac{1}{k} \log n - \log k \leq \sinv'_k(n) \leq \log n + 4k -3.$$
%\CR{Nouvelle borne (juste en collant une clique subdivisee a une clique bidirigee.}


\begin{restatable}{theorem}{extrem}\label{thm:extrem}
For any pair of positive integers $k,n$ with $n \geq 2k+1$, we have 
\begin{equation*}
\frac{1}{2} \log (n - k+1) \leq \sinv'_k(n) \leq \log n + 4k -3.
\end{equation*}
\end{restatable}


Observe that the condition $n \geq 2k+1$ is necessary for $\sinv_k'(n)$ to be well-defined.
\medskip


 Next, we consider the algorithmic complexity of computing $\sinv_k'(D)$ algorithmically for a given graph $D$ and a fixed integer $k$. We show that this problem is NP-hard. More precisely, we show the following result. 

%\FHO{After, we deal with the problem of computing $\sinv_k'(D)$ algorithmically for a given graph $D$ and a fixed integer $k$. We show that there is little hope to decide this problem in polynomial time even when fixing $k$ and the number of required inversions. More precisely, we show the following result.}

\begin{restatable}{theorem}{archard}\label{archard1}
Deciding whether a given oriented graph $\vec{G}$ satisfies $\sinv'_k(\vec{G}) \leq t$ is NP-complete for all pairs of positive integers $k$ and $t$.
\end{restatable}
%we prove that, for any fixed postive integers $k$ and $t$, deciding whether a given oriented graph $\vec{G}$ satisfies $\sinv'_k(\vec{G}) \leq t$ is NP-complete.

Furthermore we show that there is little hope to approximate $\sinv'_k(\vec{G})$ within a factor better than 2.

\begin{restatable}{theorem}{approxarc}\label{approx2}
Unless P=NP, for any positive integer $k$, there is no $(2-\epsilon)$-approximation algorithm for computing $\sinv'_k(D)$ for simple oriented graphs for any $\epsilon >0$.
\end{restatable}

\medskip



As a related problem, one may also want to make a digraph $k$-strong.
 A digraph $D$ is {\bf $k$-strong} if $|V(D)|\geq k+1$ and for any set $S$ with less than $k$ vertices $D-S$ is strong.
A digraph which can be made $k$-strong by reversing arcs
 is {\bf $k$-strengthenable}.
The $1$-strengthenable digraphs are the $2$-edge-connected ones, because
being $1$-strong is equivalent to be strong or $1$-arc-strong. 
Thomassen~\cite{Thomassen2015} proved that the $2$-strengthenable digraphs are the $4$-edge-connected digraphs $D$ such that $D-v$ is $2$-edge-connected for every vertex $v \in V(D)$, but it is NP-hard to compute the minimum number of arc reversals needed to make a give digraph 2-strong~\cite{BJHK} 
Furthermore, in contrast to the anologous for $k$-arc-strengthenable digraphs, for $k\geq 3$, it is NP-complete to decide whether a digraph is $k$-strengthenable. Indeed, Durand de Gevigney~\cite{Dur20} proved that it is NP-complete to decide whether an undirected graph has a $k$-strong orientation for any $k \geq 3$.

It is also natural to use inversions to make a digraph $k$-strong. 
A {\bf $k$-strengthening family} of a digraph $D$ is a family of subsets $(X_i)_{i\in I}$ of subsets of $V(D)$ such that
 $\Inv(D; (X_i)_{i\in I})$ is $k$-strong.
The {\bf $k$-strong inversion number} of a $k$-strengthenable digraph $D$, denoted by $\sinv_k(D)$, is the minimum number of inversions needed to transform $D$ into a $k$-strong digraph, that is, the minimum cardinality of a $k$-strengthening family.
In the light of the complexity result of Durand de Gevigney~\cite{Dur20}, it seems difficult to obtain an extremal result in the shape of Theorem \ref{thm:extrem} for $k$-strong digraphs. However, we give the following complexity results which are the natural analogues of Theorems \ref{archard1} and \ref{approx2}.
\begin{restatable}{theorem}{verhard}\label{verhard}
Deciding whether a given $k$-strengthenable oriented graph $\vec{G}$ satisfies $\sinv_k(\vec{G}) \leq t$ is NP-complete for all pairs of positive integers $k$ and $t$.
\end{restatable}
\begin{restatable}{theorem}{approxver}\label{approx1}
Unless P=NP, for any positive integer $k$, there is no $(2-\epsilon)$-approximation algorithm for computing $\sinv_k(D)$ for simple oriented graphs for any $\epsilon >0$.
\end{restatable}

We believe that it is possible to strengthen Theorems \ref{approx2} and \ref{approx1}.
\begin{conjecture}
    Unless P=NP, for any positive integer $k$ and any constant $\alpha$, there is no $\alpha$-approximation algorithm for computing $\sinv_k(D)$ or $\sinv_k'(D)$ for simple oriented graphs.
\end{conjecture}
%In Section~\ref{sec:complexity}, we show that for any positive integers $k$ and $t$, it is NP-complete to decide whether $\sinv_k(D)\leq t$ for a given $k$-strengthenable oriented graph. We also show that, unless P=NP,  $\sinv_k$ cannot be approximated within a factor better than 2.
\medskip

In the remainder of this article, we focus on a particular kind of oriented graphs, namely tournaments. A {\bf tournament} is an orientation of a complete graph.
It is not hard to show that every tournament of order at least $2k+1$ is $k$-strengthenable and that it can be made $k$-strong by reversing the orientation of at most $\frac{1}{4}(4k-2)(4k-3)$ arcs, see e.g. \cite{bang2009}, p. 379.

\begin{conjecture}[Bang-Jensen, 1994]
Every tournament on at least $2k+1$ vertices can be made $k$-strong by reversing at most $\frac{1}{2}k(k+1)$ arcs.
\end{conjecture}
Bang-Jensen, Johansen, and Yeo~\cite{BKY2020} proved this conjecture for tournaments of order at least $3k-1$.
It is tight as shown by the transitive tournaments. 

It is then natural to ask whether or not we can make a tournament $k$-strong or $k$-arc-strong in (a lot) less than $\frac{1}{2}k(k+1)$ inversions. This leads to consider $M_k= \max\{\sinv_k(T) \mid T~\mbox{tournament of order at least $2k+1$}\}$ and $M'_k= \max\{\sinv'_k(T) \mid T~\mbox{tournament of order at least $2k+1$}\}$.
We show that these numbers are indeed significantly smaller than $\frac{1}{2}k(k+1)$, but cannot be bounded by a constant independent of $k$. More precisely, we show the following result.
\begin{theorem}\label{thm:exttournoi}
    For every sufficiently large integer $k$, we have
$\frac{1}{2} \log(2k+1) \leq M'_k \leq M_k \leq 2k.$
\end{theorem}
The lower bound is obtained by considering a random tournament of order $2k+1$.
We also prove that $M_1=M'_1=1$ and $M_2=M'_2=2$ showing that the upper bound is not tight for $k=1,2$. We also believe that it is not tight for larger values of $k$.
\begin{problem}
Find better bounds on $M_k$ and $M'_k$.
\end{problem}


We further study the question of how the parameters $\sinv_k$ and $\sinv'_k$  behave when fixing $k$ and considering tournaments whose size is significantly larger then $2k+1$.  As a first result, we  prove that every sufficiently large tournament can be made $k$-strong (and thus also $k$-arc-strong) in one inversion. 

\begin{restatable}{theorem}{sone}\label{thm:s1}
If $n \geq (2k-1)2^{2k}$, then $\sinv_k(T) \leq 1$.
\end{restatable}

This leads us to the study of the functions
$m_k(n)= \max \{\sinv_k(T) \mid T~\mbox{tournament of order}~n\}$ and $m'_k(n)= \max \{\sinv_k(T) \mid T~\mbox{tournament of order}~n\}$ for all $n\geq 2k+1$. We believe that $m_k$ and $m_k'$ have the following monotonic behaviour. 
\begin{conjecture}

\begin{enumerate}
  \item[(i)] $m_k$ and $m'_k$ are non-increasing mappings.
  \item[(ii)] $M_k = M'_k = m_k(2k+1) = m'_k(2k+1)$.
\end{enumerate}
\end{conjecture}


Note  that (i) implies (ii).
This conjecture is motivated bys the fact that one can easily prove that $m_k$ and $m'_k$ are non-increasing for $n>4k-2$. Indeed, it is well-known that every tournament $T$ of order $n>4k-2$ has a vertex $v$ with $d_T^-(v)\geq k$ and $d_T^+v)\geq k$.
Moreover adding a vertex with in- and out-degree at least $k$ to a $k$-(arc-)strong digraph results in a $k$-(arc-)strong digraph.
Hence any $k$-(arc-)strenghtening family of $T-v$ is also a $k$-(arc-)strenghtening family of $T$. Thus $\sinv_k(T) \leq \sinv_k(T-v) \leq m_k(n-1) $ and $\sinv'_k(T) \leq \sinv'_k(T-v) \leq m'_k(n-1) $. Hence $m_k(n) \leq m_k(n-1)$ and $m'_k(n) \leq m'_k(n-1)$.
Therefore, to get bounds on $M_k$ and $M'_k$, it is sufficient to consider tournaments whose order is in the range from $2k+1$ to $4k+2$.

\medskip

%The fact that $m_k(n) =1$ when $n$ is sufficiently large in comparison to $k$
Theorem \ref{thm:s1} implies that, for every pair of positive integers $k$ and $i$, there is a smallest integer $N_k(i)$ such that
$m_k(n) \leq i$ for all $n\geq N_k(i)$.
Similarly, for every pair of positive integers $k$ and $i$, there is a smallest integer $N'_k(i)$ such that
$m'_k(n) \leq i$ for all $n\geq N'_k(i)$.
Since $m'_k(n) \leq m_k(n)$ for all $k$ and $n$, we have $N'_k(i) \leq N_k(i)$ for all $k$ and $i$.
It is natural to ask the following questions.
\begin{problem}\label{probnk}
What is the minimum integer $N_k(i)$ such that $\sinv_k(T) \leq i$ for every tournament $T$ of order at least $N_k(i)$ ?

What is the minimum integer $N'_k(i)$ such that $\sinv'_k(T) \leq i$ for every tournament $T$ of order at least $N'_k(i)$ ?

\end{problem}

Our most important result on Problem \ref{probnk} is the following significant improvement on Theorem \ref{thm:s1}.
\begin{restatable}{theorem}{nkone}\label{nk1}
    For any positive integer $k$, we have $N'_k(1)\leq N_k(1)\leq 28k-5$.
\end{restatable}
Concerning lower bounds for $N_k(1)$, we have the following result.
\begin{restatable}{proposition}{nkunter}\label{nk1unter}
    For any positive integer $k$, we have $N_k(1)\geq 5k-2$.
\end{restatable}
We further have the following result that that shows that a significantly smaller tournament can still be dealt with by a small constant number of inversions.
\begin{restatable}{theorem}{nksix}\label{nk6}
    For any positive integer $k$, we have $N'_k(6)\leq N_k(6)\leq 14k-3$.
\end{restatable}
%We prove $5k-2\leq N'_k(1)\leq N_k(1)\leq 94k -40$ (Proposition~\ref{prop:N1-inf} and Theorem~\ref{thm:N1-linear}) and $N_k(6)\leq 14k -3$ (Theorem~\ref{thm:s6}).
Finally, using probabilistic methods, we manage to prove that every tournament that is a constant factor bigger than $2k$ can be made $k$-strong by a constant number of inversions. More precisely, we prove the following result.
\begin{restatable}{theorem}{pluseps}\label{thm:2+eps}
    There exists a function $f: \mathbb{R}_{>0} \to \mathbb{N}$ such that for every $\epsilon>0$ and every positive integer $k$, 
    if $T$ is an $n$-vertex tournament with $n \geq 2k+1 +\epsilon k$, then $\sinv_k(T) \leq f(\epsilon)$.
\end{restatable}

%In addition we show in Theorem~\ref{thm:2+eps} the existence a function $f: \mathbb{R}_{>0} \to \mathbb{N}$ such that, for every $\epsilon>0$ and every positive integer $k$, $\sinv_k(T) \leq f(\epsilon)$ for all
 %tournament $T$ on at least $(2+\epsilon)k$ vertices.

\medskip

The fact that $m_k(n)=1$ for $n$ sufficiently large implies that the set ${\cal F}_k$ of tournaments $T$ such that $\sinv_k(T) >1$ is finite.
This implies that computing $\sinv_k$ and $\sinv'_k$ for fixed $k$ can be done in polynomial time for tournaments.

\begin{corollary}\label{cor:sinv_k-poly}
Let $k$ be a positive integer. We can compute $\sinv_k(T)$ and $\sinv'_k(T)$ for a given tournament $T$ in quadratic time.
\end{corollary}
\begin{proof}
We first check in constant time, whether $T$ is in ${\cal F}_k$. If yes, then we can return $\sinv_k(T)$ and $\sinv'_k(T)$ that may be stored in some precomputed table.
If not, then $\sinv'_k(T) \leq \sinv_k(T) \leq 1$. Then we check in quadratic time (using a flow algorithm for example) whether $T$ is $k$-strong or $k$-arc-strong to determine whether $\sinv_k(T)$ and $\sinv'_k(T)$ are equal to $0$ or $1$.
\end{proof}


\begin{problem}
What is the complexity of computing $\sinv_k(T)$ (resp. $\sinv'_k(T)$) for a given tournament $T$ if $k$ is part of the input ?
\end{problem}

This article is structured as follows: In Section \ref{sec:oriented}, we prove Theorem \ref{thm:extrem}. In Section \ref{sec:complexity}, we prove the complexity results, namely Theorems \ref{archard1} to \ref{approx1}. The results on tournaments are contained in Section \ref{sec:M}, in which we prove Theorem \ref{thm:exttournoi}, and in Section \ref{sec:upper_bound_Mkn}, in which we prove Theorems \ref{thm:s1}, \ref{nk1},\ref{nk6} and \ref{thm:2+eps} and Proposition \ref{nk1unter}.