
\section{Bounds on \texorpdfstring{$M'_k$}{M'k} and \texorpdfstring{$M_k$}{Mk}}\label{sec:M}
%%%%%%%%%%%%%%%%%%%%%



In this section, we prove Theorem~\ref{thm:exttournoi} which states $\frac{1}{2} \log(2k+1) \leq M_k \leq 2k$.
The left hand-side  inequality is proved in Theorem~\ref{thm:m(2k+1)}, and the right-hand side one in Theorem~\ref{thm:M<2k}. 





\subsection{Lower bound on \texorpdfstring{$m'_k(2k+1)$}{mk(2k+1)}}
%%%%%%%%%%%%%%%%%%%%%%%%%%%%%

We shall first show the lower bound $m'_k(2k+1) = \Omega (\log k)$. We need the following results.
A digraph is {\bf eulerian} if $d^-(v) = d^+(v)$ for all vertex $v$.


\begin{proposition}[Folklore]\label{eul}
Every $k$-arc-strong tournament on $2k+1$ vertices is eulerian.
\end{proposition}

\begin{theorem}[McKay~\cite{McK90}]\label{count}
The number of labelled eulerian tournaments on $n$ vertices is  $$\left(\frac{2^{n+1}}{\pi n}\right)^{\frac{n-1}{2}}\sqrt{\frac{n}{e}} (1 +o(1)).$$
\end{theorem}
We are now ready to prove the lower bound in Theorem \ref{thm:exttournoi}.
\begin{theorem}\label{thm:m(2k+1)}
For every sufficiently large $k$, $m'_k(2k+1)\geq \frac{1}{2}\log (2k+1)$.
\end{theorem}
\begin{proof}
    Let $n=2k+1$. By Proposition \ref{eul}, all the $k$-arc-strong tournaments on $n$ vertices are eulerian. Furthermore, by Theorem~\ref{count}, if $k$ is large enough, then the number of labelled eulerian tournaments on $n$ vertices is at most
    $\left(\frac{2^{n+1}}{\pi n}\right)^{\frac{n-1}{2}}\sqrt{n}$.
    For two tournaments $T,T'$ on the same vertex set, we say that $T'$ is {\bf reachable} from $T$ by $t$ inversions for some positive integer $t$, if there is a family of sets $(X_1,\ldots,X_t)$ such that $\Inv(T;X_1,\ldots,X_t)=T'$. Observe that there are $2^n$ possibilities to choose $X_i$ for $i=1,\ldots,t$, hence the number of tournaments reachable from $T$ by $t$ inversions is at most $(2^n)^t=2^{nt}$. Therefore,
    %Since at most $2^{nt}$ labelled tournaments are reachable from a fixed $n$-vertex tournament using at most $t$ inversions,
    the number of labelled $n$-vertex tournaments that are reachable from an eulerian  one is at most 
    \begin{eqnarray*}
    2^{nt}\left(\frac{2^{n+1}}{\pi n}\right)^{\frac{n-1}{2}}\sqrt{n}
    & = &2^{\binom{n}{2}}\cdot 2^{nt} \left(\frac{2}{\pi n}\right)^{\frac{n-1}{2}}\sqrt{n} \\
    & = &2^{\binom{n}{2}}\cdot 2^{nt +\frac{n-1}{2}(1 -\log(\pi)-\log n)+\frac{1}{2}\log n}\\
    & < &2^{\binom{n}{2}}\cdot 2^{nt-\frac{1}{2} n \log n}
    \end{eqnarray*}
    for $n$ sufficiently large. % (actually $k = \frac{n-1}{2} \geq 5$ will do)\FHO{J'effacerais la remarque en parentheses. Je ne sais pas si ca marche bien avec le $o(1)$}.
    If $t \leq \frac{1}{2}\log n$, then $nt - \frac{1}{2} n \log n \leq 1$, so the number of such tournaments is less than $2^{\binom{n}{2}}$. It follows that there is at least one tournament $T^*$ on $2k+1$ vertices which cannot be reached from an eulerian tournament by at most $t$ inversions. Thus $\sinv'_k(T^*)>t$.
\end{proof}


Theorem~\ref{thm:m(2k+1)} can be slightly generalised to tournaments with $2k+c$ vertices with $c$ small. To do so, we will need the following generalisation of Theorem \ref{count}.

For a positive integer $n$ and integer $\alpha_1,\dots ,\alpha_n$, we denote by
$NT(n;\alpha_1, \dots,\alpha_n)$ the number of labelled tournaments
of order $n$ whose vertex $i$ has $d^+(i)-d^-(i)=\alpha_i$.
\begin{theorem}[Spencer~\cite{Spencer74} and McKay~\cite{McK90}]\label{thm:spencer}
Let $n$ be a positive odd integer and let $\alpha_1, \dots, \alpha_n$ be integers.
Then
\[
NT(n,\alpha_1, \dots,\alpha_n) \leq
\left(\frac{2^{n+1}}{\pi n}\right)^{(n-1)/2} \sqrt{\frac{n}{\e}} \exp\left(-\frac{1+o(1)}{2n}\sum_{i=1}^n\alpha_i^2\right).
\]
\end{theorem}
%\FHO{Qc est bizarre. Ce resultat est beacoup plus fort que le precedent, mais 15 ans plus vieux.}\CR{Après quelques recherches, il semblerai que Spencer exprime l'estimation de $NT(N,(\alpha_i))$ en fonction de $NT(n,0,\dots,0)$, et c'est McKay qui a eu la premiere estimation de $NT(n,0)$ a un facteur $1+o(1)$ près (celle de Spencer etait à $(1+o(1))^n$ près.}
\begin{corollary}\label{corollary:number_tournaments_high_degree}
For all integers $k$ and $c$, the number of tournaments $T$ on $n=2k+c$ vertices
such that every vertex has in- and out-degree at least $k$ is at most
$2^{\log(2c)n}\left(\frac{2^{n+1}}{\pi n}\right)^{(n-1)/2} \sqrt{\frac{n}{\e}}$ for $k$ large enough.
\end{corollary}

\begin{proof}
By Theorem~\ref{thm:spencer},
the number of tournaments $T$ on $n=2k+c$ vertices
such that every vertex has in- and out-degree at least $k$ is at most

\[
\left(\frac{2^{n+1}}{\pi n}\right)^{(n-1)/2} \sqrt{\frac{n}{\e}}
\sum_{\alpha_1, \dots,\alpha_n \in \{-c+1,\dots c-1\}^n}\exp\left(-\frac{1+o(1)}{2n}\sum_{i=1}^n\alpha_i^2\right).
\]
%\FHO{je ne vois pas pourquoi c'est pas $\{-c,\ldots,c\}$ sous la somme.}
%\CR{On compte le sommet lui même, par exemple pour $c=1$, on a toujours $\alpha_i=0$.}
% Further, we have
% \[
% \begin{split}
% \sum_{\alpha_1, \dots,\alpha_n \in \{-c+1,\dots c-1\}^n}\exp\left(-\frac{1+o(1)}{2n}\sum_{i=1}^n\alpha_i^2\right)
% &= \prod_{i=1}^n\left(\sum_{\alpha=-c+1}^{c-1}\exp\left(-\frac{1+o(1)}{2n} \alpha^2\right)\right) \\
% &= \prod_{i=1}^n\left(1+o(1)+2\sum_{\alpha=1}^{c-1}\exp\left(-\frac{1+o(1)}{2n} \alpha^2\right)\right) \\
% &\leq \left(1+o(1)+2(c-1)\e^{-\frac{1+o(1)}{2n}}\right)^n \\
% &\leq \left(1+o(1)+2(c-1) +o(1)\right)^n \\
% &\leq 2^{\log(2c)n}
% \end{split}
% \]
Further, we have
\[
\begin{split}
\sum_{\alpha_1, \dots,\alpha_n \in \{-c+1,\dots c-1\}^n}\exp\left(-\frac{1+o(1)}{2n}\sum_{i=1}^n\alpha_i^2\right)
&= \prod_{i=1}^n\left(\sum_{\alpha=-c+1}^{c-1}\exp\left(-\frac{1+o(1)}{2n} \alpha^2\right)\right) \\
&\leq  \prod_{i=1}^n\left(\sum_{\alpha=-c+1}^{c-1}1 + o(1)\right) \\
&= (2c-1+o(1))^n \\
&\leq 2^{\log(2c)n}
\end{split}
\]
% \FHO{on ne peut pas simplifier ce calcul en estimant $\exp\left(-\frac{1+o(1)}{2n} \delta^2\right)$ par $1$?}
%\FHO{je ne vois pas du tout d'ou vient la premiere egualite}
%\CR{J'ai rajouté une etape. Il faut juste developper.}
for $n$ large enough, and the result follows.
\end{proof}

% \JD{On ne peut pas plutôt écrire $\exp{ 2 n \log(c)}$ pour la dernière ligne? Si c'est vrai, raffiner les constantes donnerai une borne inf pas si horrible pour le passage à $m_k \leq $ constante, de l'ordre de $c = k^{O(1)}$.}
% \CR{Oui tu as raison, j'ai fait la modif.}

\begin{theorem}\label{thm:2k+c}
    For every positive integer $c$ fixed, and for every positive integer $k$ large enough compared to $c$, there exists a tournament
    $T$ on $2k+c$ vertices such that $\sinv'_k(T) > \frac{1}{2}\log (2k+c)-\log(2c)$.
    In particular $m'_k(2k+c)$ is unbounded for every fixed $c$.
\end{theorem}

\begin{proof}
By Corollary~\ref{corollary:number_tournaments_high_degree},
the number of tournaments $T$ on $n=2k+c$ vertices with $\sinv'_k(T) \leq t$
is for $k$ large enough at most 
\[
2^{nt} 2^{\log(2c)n}\left(\frac{2^{n+1}}{\pi n}\right)^{(n-1)/2} \sqrt{\frac{n}{\e}}
\leq 2^{\binom{n}{2}} 2^{nt-\frac{1}{2}n\log n+\log(2c)n}
\]
which is smaller than $2^{\binom{n}{2}}$ if $t \leq \frac{1}{2}\log n -\log(2c)$.
Hence there exists such a tournament $T$ with $\sinv'_k(T) > \frac{1}{2}\log (2k+c)-\log(2c)$.
\end{proof}







\subsection{Upper bounds on \texorpdfstring{$M_k$}{Mk}}
%%%%%%%%%%%%%%%%%%%%%%%%%


%We shall need the following lemma.

%\begin{lemma}[Folklore]\label{lem:kstrong+}
%Let $k$ be a positive integer, let $D$ be a digraph and let $v$ a vertex such that $d^-(v)\geq k$ and $d^+(v)\geq k$.
%If $D-v$ is $k$-strong, then $D$ is also $k$-strong.
%\end{lemma}


\begin{theorem}\label{thm:M<2k}
$M_k \leq 2k$.
\end{theorem}

\begin{proof}
Let $D$ be a tournament with $V(D)=\{v_1,\ldots,v_n\}$ with $n \geq 2k+1$. Further, let $T$ be a $k$-strong tournament on $\{v_1,\ldots,v_{2k+1}\}$. We now define sets $X_1,\ldots,X_{2k}$. 
Suppose that the sets $X_1,\ldots,X_{i-1}$ have already been created and let $D_{i-1}$ be the graph obtained from $D$ by inverting $X_1,\ldots,X_{i-1}$.
Now let $X_i = \{v_i\} \cup A_i \cup B_i$, where
$A_i$ is the set of vertices $v_j$ with $j \in \{i+1,\ldots,2k+1\}$ for which the edge $v_iv_j$ has a different orientation in $T$ and $D_{i-1}$,
and $B_i$ is, when $i \leq k$ (resp. $i \geq k+1$), the set of vertices $v_j$ with $j \geq 2k+2$ for which $D_{i-1}$ contains the arc $v_iv_j$ (resp. $v_jv_i$).

We still need to show that $D_{2k}$ is $k$-strong. Observe that $D_{2k}\langle\{v_1,\ldots,v_{2k+1}\}\rangle=T$ which is $k$-strong by assumption. However, for any $j\geq 2k+2$, $D_{2k}$ contains the arcs $v_jv_i$ for $i=1,\ldots,k$ and the arcs $v_iv_j$ for $i=k+1,\ldots,2k$. Hence, by Lemma~\ref{lem:kstrong+}, $D_{2k}$ is $k$-strong.
\end{proof}


Theorems~\ref{thm:m(2k+1)} and~\ref{thm:M<2k} directly imply Theorem \ref{thm:exttournoi}.%M_k \leq 2k$.

\subsection{Values of \texorpdfstring{$M'_1$}{M'1}, \texorpdfstring{$M_1$}{M1}, \texorpdfstring{$M'_2$}{M'2} and \texorpdfstring{$M_2$}{M2}}
%%%%%%%%%%%%%%%%%%%%%%%%%%%%%%%%%%%%%%%%%%%%%%%%%%%%%%%
We here provide the exact values of $M_i$ and $M_i'$ for $i \in \{1,2\}$.
\begin{proposition}\label{prop:M1}
Let $T$ be a tournament of order $n \geq 3$.
We have $\sinv_1(T) = \sinv'_1(T) =0$ if $T$ is strong and $\sinv_1(T) = \sinv'_1(T) =1$ otherwise.
In particular,  $m_1(n) = m'_1(n) =1$ for all $n\geq 3$ and $M_1=M'_1=1$.
\end{proposition}
\begin{proof}
(i) Trivially, if $T$ is strong, then $\sinv_1(T)=0$. 
If $T$ is not strong, then $\sinv_1(T) \geq 1$. Now consider a hamiltonian path of $T$
which exists by Redei's Theorem (see e.g. Theorem 1.4.2 in \cite{bang2009}. Let $X$ be its initial vertex and $y$ its terminal vertex.
Then inverting $\{x,y\}$ yields a tournament with a directed hamiltonian cycle because $V(T) \setminus\{x,y\} \neq \emptyset$, so a strong tournament.
Hence $\sinv_1(T) = 1$.
\end{proof}








A non-strong tournament is said to be {\bf reducible}.
It is folklore that a reducible tournament has a {\bf reduction}  $T_1\Ra T_2$ that is two subtournaments $T_1, T_2$ such that $(V(T_1), V(T_2))$ is a partition of $V(T)$ and $V(T_1) \Ra V(T_2)$.





Let $S_4$ be the unique strong tournament of order $4$.
Its vertex set is $\{a,b,c,d\}$ and its arc set is $\{ab, bc, cd, da, ca, db\}$.


\begin{proposition}\label{prop:M2}
$M_2=M'_2=2$.
\end{proposition}
\begin{proof}
$R_5$ the rotative tournament of order $5$ is the only 2-arc-strong tournament of order $5$.
As observed in \cite{BBBP10}, $\inv(R_5)=2$, so $\sinv'_2(TT_5) = 2$.
Hence $M_2 \geq M'_2\geq 2$.

\medskip

Let us now prove that $M_2\leq 2$.
We shall prove by induction on $n$ that every tournament $T$ of order at least $5$ satisfies $\sinv_2(T) \leq 2$.


Assume first that $T$ is a tournament of order $5$.
If $T$ is strong, then, by Camion's theorem~\cite{camion1959}, it has a hamiltonian cycle
$v_1v_2v_3v_4v_5v_1$. Let $A^+=A(T)\cap \{v_1v_3, v_2v_4, v_3v_5, v_4v_1, v_5v_2\}$ and $A^-=A(T)\cap \{v_3v_1, v_4v_2, v_5v_3, v_1v_4, v_2v_5\}$.
We have $|A^+| + |A^-|=5$, so one of the two sets $A^+, A^-$ has at most two arcs. Reversing the arcs of this set, one after another, yields the 2-strong tournament $R_5$.

Assume now that $T$ is not strong. Then it must be one of the following tournaments or their converse. (The {\it converse} of a digraph is the digraph obtained by reversing all arcs.)
\begin{itemize}
\item $TT_5$ with hamiltonian path $v_1v_2v_3v_4v_5$. Then inverting $\{v_1, v_2, v_4, v_5\}$ and $\{v_1, v_5\}$ yields $R_5$.

\item $S_4\Ra \{x\}$. Then inverting $\{c, d, x\}$ and $\{c, d\}$ yields $R_5$.

\item $\{x\} \Ra \vec{C}_3 \Ra \{y\}$ with $ \vec{C}_3 = abca$. Then inverting $\{a,x,y\}$ yields $R_5$.

\item $\{x\} \Ra \{y\} \Ra \vec{C}_3 $ with $ \vec{C}_3 = abca$.
Then inverting $\{a,b,c,y\}$ and $\{a,x,y\}$ yields $R_5$.

\end{itemize}

Assume now that $T$ has at least $6$ vertices.

 Assume first $T$ has a vertex $v$ such that $\min\{d^+(v),d^-(v)\}\geq 2$.
 By the induction hypothesis, $\sinv_2(T-v) \leq 2$, so there is a family ${\cal X}$ of at most two subsets of $V(T-v)$ such that
 $\Inv(T-v; {\cal X})$ is $2$-strong.
 Now $\Inv(T; {\cal X})-v = \Inv(T-v; {\cal X})$ and $\min\{d^+(v),d^-(v)\}\geq 2$. Thus, by Lemma~\ref{lem:kstrong+},
 $\Inv(T; {\cal X})$ is $2$-strong, and it follows that $\sinv_2(T) \leq |{\cal X}| \leq 2$.

Assume now that for every vertex has either in-degree at most $1$ or out-degree at most $1$. Then necessarily $T$ must be the tournament $\vec{C}_3\Ra \vec{C}_3$.
Let $V(T) = \{a,b,c,d,e,f\}$ with $\{a,b,c\} \Ra \{d,e,f\}$.
Then inverting $\{a,b,c,d\}$ and $\{a, d,e,f\}$ transforms $T$ into a $2$-strong tournament.
\end{proof}


\section{Upper bounds on \texorpdfstring{$m_k(n)$}{mk(n)}\label{sec:upper_bound_Mkn}}
%%%%%%%%%%%%%%%%%%%%%%%%%%%%%%%%%%%%
In this section, we prove several results showing that tournaments on significantly more than $2k$ vertices can be made $k$-strong by a small number of inversions. More precisely, in Section \ref{first}, we prove Theorem \ref{thm:s1}, in Section \ref{nk10}, we prove Theorem \ref{nk1}, in Section \ref{nk6sec}, we prove Theorem \ref{nk6}, and in Section \ref{epsgross}, we prove Theorem \ref{thm:2+eps}.
 \subsection{First upper bounds on \texorpdfstring{$m_k(n)$}{mk(n)}}\label{first}
%%%%%%%%%%%%%%%%%%%%%%%%%%%%%

In subsection, we first establish that, for every fixed $k$, $m_k(n)$ tends to $1$ when $n$ tends to infinity. More precisely, we prove Theorem \ref{thm:s1}. While Theorem \ref{thm:s1} is clearly weaker than Theorem \ref{nk1}, this result justifies some of the notation used later on and may serve as a warm-up exercise of the more involved proof of Theorem \ref{nk1}.
%\sone*

\begin{proof}
Let $T$ be a tournament of order $n\geq (2k-1)2^{2k}$.

  It is easy and well-known that if $D$ is an acyclic digraph, $x$ a source in $D$, and $D-x$ is contained in every tournament of order $n$, then $D$ is contained in every tournament of order $2n$.  
An easy induction yields that $T$ contains three sets
  $A_1, A_2, A_3$ such that $A_1\Ra (A_2\cup A_3)$ and
  $A_2\Ra A_3$ with $|A_1|=|A_3| = k$ and $|A_2|=2k-1$.
Set $A=A_1\cup A_2 \cup A_3$.
Let $I$ be the set of vertices in $V(T)\setminus A$ that have either less than $k$ out-neighbours in $A$ or less than $k$ in-neighbours in $A$.
Let $X=A_1\cup A_3\cup I$.
Let us prove that $T'=\Inv(T; X)$ is $k$-strong.

In $T'$, we have $A_1 \Ra A_2 \Ra A_3\Ra A_1$. Since the three sets $A_1$, $A_2$, and $A_3$ have size at least $k$, the tournament $T'\langle A\rangle$ is $k$-strong.
Consider now a vertex $v$ in $V(T)\setminus A$.
If $v\notin I$, then not arcs incident to $v$ has been reversed so its in- and out-degree have been unchanged and are at least $k$ by definition of $I$.
If $v\in I$, then all the arcs between $v$ and $A_1\cup A_3$ have been reversed and those between $v$ and $A_2$ are unchanged.
If $v$ has less than $k$ out-neighbours in $A$ in $T$, then $|N^+_{T'}(v)\cap A| \geq |N^-_T(v) \cap (A_1\cup A_3)| \geq 2k - d^+_T(v) \geq k$, and $|N^-_{T'}(v)\cap A|\geq |N^-_T(v) \cap A_2| \geq 2k-1 - d^+_T(v) \geq k$.
Similarly, if $v$ has less than $k$ in-neighbours in $A$ in $T$, then 
$|N^+_{T'}(v)\cap A|\geq k$ and $|N^-_{T'}(v)\cap A|\geq k$.
Thus, by 
Lemma~\ref{lem:kstrong+}, $T'$ is $k$-strong.
Hence $\sinv_k(T) \leq 1$.
\end{proof}






\subsection{Linear upper bound on \texorpdfstring{$N_k(1)$}{Nk(1)}}\label{nk10}
%%%%%%%%%%%%%%%%%%%%%%%%%%%%%%%%%%%%%%%%%%

In this subsection, we shall prove  Theorem~\ref{nk1} which states that $N_k(1)\leq 28k -5$. To prove it we need some preliminaries.




For a digraph $D$, let $\sigma=(v_1,v_2, \ldots, v_n)$ be an ordering of the vertices of $D$. An arc $v_iv_j$ is {\bf forward} (according to $\sigma$) if $i<j$ and {\bf backward} (according to $\sigma$) if $j<i$.
A {\bf median order} of $D$ is an ordering of the vertices of $D$ with the maximum number of forward arcs, or equivalently the minimum number of backward arcs.

Let us note basic well-known properties of median orders of
tournaments (the feedback property in \cite{HaTh00}).

\begin{lemma}\label{lem:median}
 Let $T$ be a tournament and $(v_1,v_2, \ldots, v_n)$ a
median order of $T$. Then, for any two indices $i,j$ with $1 \leq i <
j \leq n$:
\medskip
\begin{enumerate}
\item[\rm (M1)] $(v_i,v_{i+1},\ldots,v_j)$ is a median order of the
  induced subtournament $T\langle \{v_i,v_{i+1},\ldots,v_j\}\rangle$.\\
  
\item[\rm (M2)] vertex $v_i$ dominates at least half of the vertices
  $v_{i+1},v_{i+2},\ldots,v_j$, and vertex $v_j$ is dominated by at least half of the vertices $v_i,v_{i+1},\ldots,v_{j-1}$.  In particular, each vertex $v_i$, $1 \leq i <n$, dominates its successor $v_{i+1}$. 

\end{enumerate}
\end{lemma}

Let $v$ be a vertex of a digraph $D$. We denote by $R^+_D(v)$ (resp. $R^-_D(v)$) the set of vertices which are {\bf reachable} from $v$ (resp. vertices that can reach $v$) in $D$, that are the vertices $w$ such that there is a directed $(v,w)$-path (resp.~$(w, v)$-path) in $D$. Note that $v\in R^+_D(v)$.


\begin{lemma}\label{lem:n-2F}
Let $T$ be a tournament with median order $(v_1, v_2, \ldots , v_n)$.
Let $F$ be a subset of vertices such that $v_1\notin F$. 
Then $|R^+_{T-F}(v_1)| \geq n - 2 |F|$.
\end{lemma}
\begin{proof}
We prove the result by induction on $n+|F|$, the result holding trivially by (M2) if $|F|=0$.

If all the out-neighbours of $v_1$ are in $|F|$, then by (M2), $|N^-(v_1)| \leq |N^+(v_1)|  \leq |F|$. Hence $n-1 \leq 2|F|$, and the result holds.
Henceforth we may assume that $v_1$ has an out-neighbour not in $F$.
Let $i_0$ be the smallest index of such a vertex.
Let $T_0= T\langle \{v_1, \dots , v_{i_0-1}\}\rangle$, $T_1= T\langle \{v_{i_0}, \dots , v_{n}\}\rangle$,  $F_0=F\cap V(T_0)$ and $F_1=F\cap V(T_1)$.
By (M1), $(v_1, v_2, \ldots , v_{i_0-1})$ is a median order of $T_0$ and $(v_{i_0},  \ldots , v_n)$ is a median order of $T_1$.
By definition of $i_0$, all out-neighbours of $v_1$ in $T_0$ are in $F_0$. Thus, as above, we have $i_0-2 \leq 2|F_0|$.
By the induction hypothesis, $|R^+_{T_1-F_1}(v_{i_0})| \geq n - i_0 + 1 - 2 |F_1|$.
Now $R^+_{T_1-F_1}(v_{i_0}) \cup \{v_1\} \subseteq R^+_{T-F}(v)$. Hence
 $$|R^+_{T-F}(v)| \geq |R^+_{T_1-F_1}(v_{i_0})| +1 \geq n - i_0 +2 - 2|F_1| \geq n - 2|F_0| - 2|F_1| = n - 2|F|.$$
\end{proof}

 
\begin{lemma}\label{median order bounds}
    Let $T$ be a tournament with median order $(v_1, v_2, \ldots , v_n)$. For any $i\in [n]$ and any $X \subseteq V(T)\setminus \{v_i\}$, $|R^+_{T-F}(v_i)| \geq n + 1 - i - 2 |F|$, and $R^-_{T-F}(v_i) \geq i - 2 |F|$.
\end{lemma}
\begin{proof}
    By (M1), $(v_i, \dots, v_n)$ is a median order of $T\langle \{v_i, \dots, v_n\}\rangle$ on which we can apply Lemma~\ref{lem:n-2F} to obtain the bound on $R^+_{T-F}(v_i)$. Symmetrically, $(v_i, v_{i-1}, \dots, v_1)$ is a median order of the converse of $T\langle \{v_1, \dots, v_i \} \rangle$, and the directional dual of Lemma~\ref{lem:n-2F} yields the bound on $R^-_{T-F}(v_i)$.
\end{proof}


We are now ready to prove a linear upper bound on $N_k(1)$.



% \JD{Vraiment pas mal ! On peut enlever peut être 8k au thm en remarquant qu'on a modifié exactement 2k (et pas les 6k) sommets dans A (resp. B) du coup au lieu d'appeler \ref{median order bounds} sur F, on l'applique sur $Y \cup A_0 \cup A_1$, puis on s'assure d'enlever les sommets en trop de $A$. Mais je sais pas si ça vous semble valoir le coup. Le résultat donne $|R_b| \geq n - 6k - (i-1)  - 2(k-1) - 2(2k) - (4k - i) = n - 16k $ si je ne me trompe pas}

% \JD{D'ailleurs pour $N_k(6)$ on peut appliquer exactement le 5.15 au lieu du 5.17, mais en laissant A et B inchangé. Ça donne une meilleure borne sur 5.15 (le cas 4 devient le même que le 3) en principe 4k pour $|A|$? et on peut appliquer la démo de 5.19 ensuite. On se retrouve avec $12k-2$ pour $N_k(3)$.}
% \FHO{Je crois que tu as raison, mais on verra plus tard.}
\begin{lemma}\label{6ab}
    Let $k$ be a positive integer, $T$ a tournament on $12k$ vertices and $(A,B)$ a bipartition of $V(T)$ such that $|A|=|B|=6k$. Then there is a set $X \subseteq V(T)$ such that for $T'=\Inv(T,X)$ and for any $Y \subseteq V(T)$ with $|Y|\leq k-1$, we have that $T'-Y$ contains a directed path from $a$ to $B\setminus Y$ for every $a \in A\setminus Y$ and $T'-Y$ contains a directed path from $A\setminus Y$ to $b$ for every $b \in B\setminus Y$.
\end{lemma}
\begin{proof}
    Let $(a_1,\ldots,a_{6k})$ be a median order of $T\langle A \rangle$ and let $(b_1,\ldots,b_{6k})$ be a median order of $T\langle B \rangle$. Let $A_0$ be the set of vertices in $\{a_{4k+1},\ldots,a_{6k}\}$ which have less than $k$ out-neighbours in $B$ in $T$. Further, let $A_1=\{a_1,\ldots,a_{2k-|A_0|}\}$. Observe that $|A_0 \cup A_1|=|A_0|+|A_1|=2k$. Similarly, let $B_0$ be the set of vertices in $\{b_1,\ldots,b_{2k}\}$ that have less than $k$ in-neighbours in $A$ and let $B_1=\{b_{4k+|B_0|+1},\ldots,b_{6k}\}$. Observe that $|B_0 \cup B_1|=|B_0|+|B_1|=2k$. Now let $X=A_0 \cup A_1 \cup B_0 \cup B_1$, let $T'=\Inv(T,X)$ and let $Y \subseteq V(T)$ with $|Y|\leq k-1$. In order show that $T'$ has the desired properties, by symmetry, it suffices to prove that $T'-Y$ contains a directed path from $a$ to $B\setminus Y$ for every $a \in A\setminus Y$. Suppose for the sake of a contradiction that this is not true. There is a largest integer $i \in [6k]$ such that $a_i \in A\setminus Y$ and $T'-Y$ does not contain a directed path from $a_i$ to $B\setminus Y$. We will distinguish several cases.

    \begin{case}
        $i \in \{4k+1,\ldots,6k\}$ and $a_i \in A\setminus (A_0 \cup Y)$.
    \end{case}
    In this case, by the choice of $A_0$, we have 
    \begin{align*}
        |(N_{T'}^+(a_i)\cap B)\setminus Y|&\geq |(N_{T'}^+(a_i)\cap B)|-|Y|\\
        &= |(N_{T}^+(a_i)\cap B)|-|Y|\\
        &\geq k-(k-1)\\
        &=1,
    \end{align*}
    so $a_i$ has an out-neighbour in $B\setminus Y$ in $T'-Y$, a contradiction.
%\end{proof}
    \begin{case}
        $i \in \{4k+1,\ldots,6k\}$ and $a_i \in A_0$.
    \end{case}
    In this case, by the choice of $A_0$, we have
    \begin{align*}
        |(N_{T'}^+(a_i)\cap B)\setminus Y|&\geq |N_{T'}^+(a_i)\cap B)|-|Y|\\
        &\geq|N_{T'}^+(a_i)\cap (B_0 \cup B_1)|-|Y|\\
        &=|B_0 \cup B_1|-|N_{T'}^-(a_i)\cap (B_0 \cup B_1)|-|Y|\\
        &=|B_0 \cup B_1|-|N_{T}^+(a_i)\cap (B_0 \cup B_1)|-|Y|\\
        &\geq|B_0 \cup B_1|-|N_{T'}^-(a_i)\cap B|-|Y|\\
        &\geq 2k-(k-1)-(k-1)\\
        &=2,
    \end{align*}
    so $a_i$ has an out-neighbour in $B\setminus Y$ in $T'-Y$, a contradiction.
    
    \begin{case}
        $i \in \{2k-|A_0|+1,\ldots,4k\}$.
    \end{case}
    As $(a_1,\ldots,a_{6k})$ is a median order of $T\langle A \rangle$ and by (M2) applied to $T\langle A \rangle$, we have 
     \begin{align*}
    |(N_{T'}^+(a_i)\cap \{a_{i+1},\ldots,a_{6k}\})\setminus Y|&\geq |N_{T'}^+(a_i)\cap \{a_{i+1},\ldots,a_{6k}\}|-|Y|\\
    &\geq |N_{T}^+(a_i)\cap \{a_{i+1},\ldots,a_{6k}\}|-|Y|\\
    &\geq \frac{1}{2} |\{a_{i+1},\ldots,a_{6k}\}|-|Y|\\
    & \geq \frac{1}{2} (6k-i)-(k-1)\\
    &=2k+1-\frac{i}{2}\\
    &\geq 2k+1-2k\\
    &=1.
    \end{align*}
    Hence there is some $j>i$ such that $a_j\in A\setminus Y$ and $T'-Y$ contains the arc $a_ia_j$. By the maximality of $i$, there is a directed path from $a_j$ to $B\setminus Y$ in $T'-Y$. Hence $T'-Y$ also contains a directed path from $a_i$ to $B\setminus Y$, a contradiction.
    \begin{case}
    $i \in \{1,\ldots,2k-|A_0|\}$.
    \end{case}
     As $(a_1,\ldots,a_{6k})$ is a median order of $T\langle A \rangle$ and by (M2) applied to $T\langle A \rangle$, we have 
     \begin{align*}
    |(N_{T'}^+(a_i)\cap \{a_{i+1},\ldots,a_{6k}\})-Y|&\geq |N_{T'}^+(a_i)\cap \{a_{i+1},\ldots,a_{6k}\}|-|Y|\\
    &\geq |N_{T}^+(a_i)\cap \{a_{i+1},\ldots,a_{6k}\}|-|(A_0 \cup A_1)\cap \{a_{i+1},\ldots,a_{6k}\}|-|Y|\\
    &\geq \frac{1}{2} |\{a_{i+1},\ldots,a_{6k}\}|-|(A_0 \cup A_1)\cap \{a_{i+1},\ldots,a_{6k}\}|-|Y|\\
    & \geq \frac{1}{2} (6k-i)-(2k-i)-(k-1)\\
    &=\frac{i}{2}+1\\
    &\geq 1.
    \end{align*}
    Hence there is some $j>i$ such that $a_j\in A\setminus Y$ and $T'-Y$ contains the arc $a_ia_j$. By the maximality of $i$, there is a directed path from $a_j$ to $B\setminus Y$ in $T'-Y$. Hence $T'-Y$ also contains a directed path from $a_i$ to $B\setminus Y$, a contradiction.
\end{proof}


%\nkone*
We are now ready to prove Theorem \ref{nk1}.
\begin{proof}
    Let $T$ be a tournament of order $n \geq 28k-5$ and let $(v_1,\ldots,v_n)$ be a median order of $V(T)$. Let $A=\{v_{n-6k+1},\ldots,v_n\}$ and $B=\{v_1,\ldots,v_{6k}\}$. Observe that $A$ and $B$ are disjoint as $n \geq 28k-5$. By Lemma~\ref{6ab}, there is a set $X \subseteq A \cup B$ such that in the tournament $T_0=\Inv(T\langle A \cup B \rangle,X)$, for any $Y \subseteq V(T_0)$ with $|Y|\leq k-1$, we have that $T_0-Y$ contains a directed path from $a$ to $B\setminus Y$ for every $a \in A\setminus Y$ and $T_0-Y$ contains a directed path from $A\setminus Y$ to $b$ for every $b \in B\setminus Y$. Let $T'=\Inv(T,X)$. We will show that $T'$ is $k$-strong. To this end, let $Y \subseteq V(T)$ with $|Y|\leq k-1$.

    \begin{claim}\label{nett}
        Let $b \in B\setminus Y$ and $a \in A\setminus Y$. Then $T'-Y$ contains a directed path from $b$ to $a$.
    \end{claim}
    \begin{proofclaim}
        Let $R_b$ be the set of vertices in $\{v_{6k+1},\ldots,v_{n-6k}\}$ which are reachable from $b$ in $T'-Y$ and let $R_a$ be the set of vertices in $\{v_{6k+1},\ldots,v_{n-6k}\}$ from which $a$ is reachable in $T'-Y$. By construction, there is some $i \in \{1,\ldots,6k\}$ such that $b=v_i$. Let $F=Y \cup \{v_{i+1},\ldots,v_{6k}\}$. 
        \begin{align*}
        |R_b|&= |R^+_{T'-Y}(b)\cap \{v_{6k+1},\ldots,v_{n-6k}\}|\\
             &\geq |R^+_{T'\langle\{v_i,\ldots,v_{n-6k}\}\rangle -Y} (b)\cap \{v_{6k+1},\ldots,v_{n-6k}\}|\\
             &\geq |R^+_{T'\langle\{v_i,\ldots,v_{n-6k}\}\rangle-F} (b) \setminus \{v_i\}|\\ 
            &\geq |R^+_{T'\langle\{v_i,\ldots,v_{n-6k}\}\rangle-F} (b)|-1 ~~~~~\mbox{(by Lemma \ref{median order bounds})}\\ 
            &\geq |\{v_i,\ldots,v_{n-6k}\}|-2|F|-1\\
            &\geq (n-6k-(i-1))-2((6k-i)+(k-1))-1\\
            &\geq n-20k+i+2\\
            &\geq n-20k+3.
        \end{align*}
        A similar argument shows that $|R_a|\geq n-20k+3$. As $R_a \cup R_b \subseteq \{v_{6k+1},\ldots,v_{n-6k}\}$, we obtain $|R_a \cap R_b|=|R_a|+|R_a|-|R_a\cup R_b|\geq 2(n-20k+3)-(n-12k)=n-28k+6\geq 1$. Hence there is a vertex $v^* \in \{v_{6k+1},\ldots,v_{n-6k}\}\cap R_a \cap R_b$. By definition, $T'-Y$ contains a directed path from $b$ to $v^*$ and a directed path from $v^*$ to $a$. Hence $T'-Y$ contains a directed path from $b$ to $a$.
    \end{proofclaim}
    We now show that $B\setminus Y$ is strong in $T'-Y$. Let $b,b' \in B\setminus Y$. By the definition of $X$, there is some $a \in A\setminus Y$ such that $T_0-Y$, and hence $T'-Y$, contains a directed path from $a$ to $b'$. Further, by Claim \ref{nett}, there is a path from $b$ to $a$ in $T'-Y$. Hence $T'-Y$ contains a path from $b$ to $b'$. This shows that $B\setminus Y$ is strong in $T'-Y$. Similarly, $A\setminus Y$ is strong in $T'-Y$. Next, by the choice of $X$, we have that $T_0-Y$ and hence $T'-Y$ contains a path from $A\setminus Y$ to $B\setminus Y$ and by Claim \ref{nett}, we have that $T'-Y$ contains a path  from $B\setminus Y$ to $A\setminus Y$. Hence $(A \cup B)\setminus Y$ is strong in $T'-Y$. 
    Now consider the vertex $v_i$ for some $i \in \{6k+1,n-6k\}$. By Lemma \ref{median order bounds}, we have $|R^-_{T\langle\{v_1,\ldots,v_i\}\rangle-Y}(v_i)|\geq i-2(k-1)$. As $R^-_{T\langle\{v_1,\ldots,v_i\}\rangle-Y}(v_i)\subseteq \{v_1,\ldots,v_i\}$, there is some $b \in B\setminus Y$ such that $b \in R^-_{T\langle\{v_1,\ldots,v_i\}\rangle-Y}(v_i)$. Hence $T\langle\{v_1,\ldots,v_i\}\rangle-Y$ contains a directed $(b,v_i)$-path $P$. Let $b'$ be the last vertex of this path which is contained in $B\setminus Y$. Then the $(b',v_i)$-path which is contained in $P$ also exists in $T'-Y$. Similarly, $T'-Y$ contains a directed path from $v_i$ to $A$. Hence $T'-Y$ is strongly connected. This finishes the proof.
\end{proof}





%\begin{proposition}\label{prop:N1-inf}
 % $N_k(1)\geq 5k-2$.  
%\end{proposition}
Finally, we give the proof of Proposition \ref{nk1unter}.
\begin{proof}
    Let $T$ be a tournament of order $5k-3$ whose vertex set has a partition $(A,B,C)$ such that $T\langle A \rangle$ and $T\langle C\rangle$ are $(k-1)$-diregular tournaments of order $2k-1$, and $A\Ra B\cup C$ and $B\Ra C$.
We shall prove that $\sinv'_k(T) >1$.

Assume for a contradiction that there is a set $X$ of vertices such that $T'=\Inv(T;X)$ is $k$-strong.
Every vertex of $A$ (resp. $C$) has in-degree (resp. out-degree) $k-1$ in $T$, and so belongs to $X$.
Thus $A\cup C\subseteq X$, and so $C\Ra A$ in $T'$. Hence $T'-B$ is not strong. Since $|B|=k-1$, $T'$ is not $k$-strong, a contradiction.
\end{proof}

\subsection{Better upper bound on \texorpdfstring{$N_k(6)$}{Nk(6)}}\label{nk6sec}
%%%%%%%%%%%%%%%%%%%%%%%%%%%%%%%%%%%%%%%

%\begin{lemma}\label{lem:end-path}
%Let $k$ be a positive integer, let $T$ a tournament or order $n > 4k-2$ with median order $(v_1, v_2, \ldots , v_n)$, and let 
%$F$ be set of cardinality $k-1$ in $\{v_2, \ldots , v_{n-1}\}$.
%In $T-F$, there is a directed $(v_1, v_n)$-path.
%\end{lemma}
%\begin{proof}
%\sloppy By Lemma~\ref{median order bounds}, $|R^+_{T-F}(v_1)|$ and $|R^-_{T-F}(v_n)|$ are at least $n - 2k+2$.
%Since~$n>4k-2$, we have $|R^+_{T-F}(v_1)| + |R^-_{T-F}(v_n)| > n$, so $R^+_{T-F}(v_1)$ and  $R^-_{T-F}(v_n)$ intersect, which implies that $v_n\in R^+_{T-F}(v_1)$.
%In other words, there is a directed $(v_1, v_n)$-path in $T-F$.
%\end{proof}

\begin{lemma}\label{lem:A-B}
Let $k$ be a positive integer.
Let $T$ be a tournament with $(A,B)$ a bipartition of $T$ with $|A| = |B| = 5k$.
There is a family ${\cal X}$ of at most six sets such that the following holds with $T_1= \Inv(T ; {\cal X})$, 
\begin{itemize}
\item[(i)] $T_1\langle A \rangle = T\langle A\rangle$ and $T_1\langle B \rangle = T\langle B\rangle$ ; 
\item[(ii)] in $T_1$, every vertex of $A$ has at least $k$ out-neighbours in $B$ ; 
\item[(iii)] in $T_1$, every vertex of $B$ has at least $k$ in-neighbours in $A$.
\end{itemize}
\end{lemma}
\begin{proof}
For every pair $S_1,S_2$ of set of vertices, we denote by $a(S_1,S_2)$ the number of arcs in $T$ with tail in $S_1$ and head in $S_2$.

Let $X$ be the set of vertices of $A$ having less than $\frac{5}{2}k$ out-neighbours in $B$.
Let $T_0= \Inv(T ; (X\cup B, X, B))$.
We have $T_0\langle A \rangle = T\langle A\rangle$, $T_0\langle B \rangle = T\langle B\rangle$, and in $T_0$ every vertex of $A$ has at least $\frac{5k}{2}$ out-neighbours in $B$. 
In particular $a(A,B) \geq \frac{25k^2}{2}$ and so $a(B,A) \leq \frac{25k^2}{2}$.
Let $B_0$ be the set of vertices having at least $k$ in-neighbours in $A$, and set $B_1=B\setminus B_0$.
Every vertex of $B_1$ has at least $4k$ out-neighbours in $A$.
Hence $|B_1| \leq \frac{a(B_1, A)}{4k} \leq  \frac{a(B, A)}{4k}\leq \frac{25k}{8}$.

Let $A_0$ be the set of vertices of $A$ having at least $k$ out-neighbours in $B_0$, and set $A_1=A\setminus A_0$.
Every vertex of $A_1$ has at least $\frac{5k}{2}-k = \frac{3k}{2}$ out-neighbours in $B_1$.
Now $a(A_1, B_1) \leq k|B_1|$ arcs between $A_1$ and $B_1$ so $|A_1| \leq \frac{2k|B_1|} {3k} \leq \frac{25k}{12}$.

Let $T_1= \Inv(T_0 ; (A_0\cup B_1, A_0, B_1)) = \Inv(T; (X\cup B, X, B, A_0\cup B_1, A_0, B_1))$.
Clearly, $T_1\langle A \rangle =T_0\langle A \rangle = T\langle A\rangle$, $T_1\langle B \rangle= T_0\langle B \rangle = T\langle B\rangle$.
So (i) holds.
Let us now prove that (ii) and (iii) also hold.

\begin{itemize}
\item Let $a$ be a vertex in $A$.
If $a\in A_0$, then $|N^+_{T_1}(a) \cap B_0| = |N^+_{T_0}(a) \cap B_0| \geq k$. 
If $a\in A_1$, then $|N^+_{T_1}(a) \cap B| = |N^+_{T_0}(a) \cap B| \geq \frac{5k}{2}$. 
This proves (ii).
\item Let $b$ be a vertex in $B$.
If $b\in B_0$, then $|N^-_{T_1}(b) \cap A| = |N^-_{T_0}(b) \cap A| \geq k$ by definition of $B_0$. 
If $b\in B_1$, then $|N^-_{T_1}(b) \cap A| \geq |N^+_{T_0}(b) \cap A_0| \geq  |N^+_{T_0}(b) \cap A| - |A_1| \geq 4k - 25k/12  > k$.
This proves (iii).
\end{itemize}
\end{proof}




\nksix*
\begin{proof}
Assume $n\geq 14 k-3$.
Let $(v_1, \ldots , v_n)$ be a median order of $T$.
Let $B=\{v_1, \dots , v_{5k}\}$, $A=\{v_{n-5k+1}, \ldots , v_{n}\}$ and $C=V(T)\setminus (A\cup B)$. We have $|C| \geq 4k-3$.
By Lemma~\ref{lem:A-B}, applied to $T\langle A\cup B\rangle$, there is a family  ${\cal X}$ of at most six subsets of $A\cup B$ such that
for $T_1 = \Inv(T;\mathcal{X})$ we have
\begin{itemize}
\item[(i)] $T_1\langle A \rangle = T\langle A\rangle$ and $T_1\langle B \rangle = T\langle B\rangle$ ; 
\item[(ii)] in $T_1$, every vertex of $A$ has at least $k$ out-neighbours in $B$ ; 
\item[(iii)] in $T_1$, every vertex of $B$ has at least $k$ in-neighbours in $A$.
\end{itemize}


Let us now prove that $T_1$ is $k$-strong, which implies $\sinv_k(T) \leq 6$.
Note that $T\langle A \cup C \rangle$, as well as $T\langle B \cup C \rangle$ are unchanged by the inversions.
Let $F$ be a set of $k-1$ vertices of $T_1$. Let us show that $T_1-F$ is strong.
Let $x=v_{i_1}$ and $y=v_{i_2}$ be two vertices in $T_1-F$.
As $|C| \geq 4k-3$, if $x \in B$, Lemma~\ref{lem:n-2F} asserts the existence of a path from $x$ to $c \in C$. Furthermore, Lemma~\ref{lem:n-2F} asserts the existence of a (possibly empty) path from any vertex of $A \cup C$ to a vertex of $A$ in $T_1$. Thus there exists $x'$, a vertex of $A$ reachable from $x$ by a path $P_x$. This vertex $x'$ has at least $k$ out-neighbours in $B$, so at least one of them, say $u$, is in $B\setminus F$.
Similarly, by directional duality, in $T_1-F$, there is a directed path $P_y$ (possibly empty) from a vertex $y'\in B$ to $y$.
This vertex $y'$ has at least $k$ in-neighbours in $A$, so at least one of them, say $w$, is in $A\setminus F$.
By Lemma~\ref{lem:n-2F}, $|R^+_{B \cup C - F}(u) \cap C| \geq |C|-2(k-1) > |C|/2$, and $|R^-_{A \cup C - F}(w) \cap C|  >|C|/2$. Thus there is a vertex of $C$ in $R^+_{B \cup C - F}(u)\cap R^-_{A \cup C - F}(w)$, and so there exists a path $P_{u, w}$ from $u$ to $w$ in $T_1 - F$.
$P_x P_{u, w} P_y$ is then a path from $x$ to $y$.
\end{proof}







All the results of the previous subsections imply the following.
\begin{corollary}\label{cor:m_k-upper}
$m_k(n) \leq \left\{ \begin{array}{ll}
      2k,&\text{if~ $ 2k+1 \leq n \leq 14k - 3$,}\\
      6,&\text{if~ $ 14k - 3 < n < 28k - 5$,}\\
      1,&\text{if~  $ n\geq 28k - 5$.}
  
    \end{array}\right.
    $
\end{corollary}


\begin{problem}
Find better upper bounds on $m_k(n)$. 
\end{problem}



\subsection{Upper bounds for \texorpdfstring{$k$}{k} large.}\label{epsgross}
%%%%%%%%%%%%%%%%%%%%%%%%%%%%%%%%%%
% \FHO{J'ai lu cette demonstration et je suis a peu pres convaincu que l'idee marche, mais c'est quand-meme tres dure a lire parce qu'il y a beaucoup de notations. Ce serait bien, si on arriverarit a couper la demonstration un peu. Je commencerais par un claim a peu pres de la forme suivante:
% \begin{claim}
%     If $T'$ is not $k$-strong, then one of the following holds:
% \begin{itemize}
%     \item there is a vector $z \in \mathbb{F}_2^t \setminus \{\vec{0}\}$ such that $|\{v \in V(T)|\vec{v} \neq z\}|\leq k$,
%     \item there are $u,v \in V(T)$ with $\vec{u} \neq\vec{v} $ such that $\min\{|N_{T'}^+(u)\cap N_{T'}^-(v)|,|N_{T'}^+(u)\cap N_{T'}^+(v)|,|N_{T'}^-(u)\cap N_{T'}^-(v)|\leq \frac{k+\epsilon}{2}$,
%     \item there are sets $A,B \subseteq V(T')$ with $|A|,|B|=\frac{k+\epsilon}{2}$ such that the underlying of $(A \cup B, \delta_{T'}(A,B))$ does not contain a matching of size $\frac{k}{2}$.
% \end{itemize}
% \end{claim}
% Apres, on pourrait mettre des bornes pour les 3 evenements et conclure.}
% \CR{J'ai implemente ta remarque}\FHO{Super, merci. Je regarderai demain.}
In this section, we show that if a tournament has at least $2k+1+\epsilon k$ vertices for some positive integer $k$ and some $\epsilon>0$, then it can be made $k$-strong by inversing a family of sets whose cardinality only depends on $\epsilon$.
The proof consists in drawing this family uniformly at random, under the constraint that every vertex is contained in at least one of the sets. To analyse this procedure we will need the celebrated Chernoff's bound.
%\FHO{On devrait peut-etre ajouter une referencea un livre qui contient des bases sur la methode probabiliste, y compris la borne de Chernoff.}


\
\begin{lemma}[Chernoff's bound]\label{chern}
    If $X$ is a random variable following a binomial law with parameters $p \in [0,1]$ and $n \geq 0$, then
    for every $\epsilon \in [0,1]$
    \[
    \Pr[X \geq (1+\epsilon)pn] \leq e^{-\frac{\epsilon^2 }{3} pn}.
    \]
    and
    \[
    \Pr[X \leq (1-\epsilon)pn] \leq e^{-\frac{\epsilon^2}{2} pn}.
    \]
\end{lemma}
We refer the reader to~\cite[Part II, Section 5]{molloy2002graph} for an introduction to the probabilistic method, including a proof of this bound.
We will also need the two following technical lemmas.

\begin{lemma}\label{lemma:proba_2}
Let $\vec{u} \neq \vec{v} \in \mathbb{F}_2^t \setminus \{\vec{0}\}$ and $x,y \in \mathbb{F}_2$ be fixed, and let $\vec{w} \in \mathbb{F}_2^t \setminus \{\vec{0}\}$ 
be drawn uniformly at random. Then $\Pr[\vec{u} \cdot \vec{w} = x, \vec{v} \cdot \vec{w}=y] \geq \frac{1}{4}-\frac{3}{4}\frac{1}{2^t-1}$.
\end{lemma}

\begin{proof}
As $\vec{u}\neq \vec{v}$, the mapping $\mathbb{F}_2^t \to \mathbb{F}_2^2, \vec{w} \mapsto (\vec{u} \cdot \vec{w}, \vec{v} \cdot \vec{w})$ is surjective and linear.
As a consequence, there are $\frac{1}{4}2^t$ vectors $\vec{w} \in \mathbb{F}_2^t$ which satisfy $\vec{u} \cdot \vec{w}=x$ and $\vec{v} \cdot \vec{w} = y$.
Thus by possibly removing the solution $\vec{w}=0$, we obtain $\Pr[\vec{u} \cdot \vec{w} = x, \vec{v} \cdot \vec{w}=y] \geq \frac{2^{t-2}-1}{2^t-1} =
\frac{1}{4}-\frac{3}{4}\frac{1}{2^t-1}$.
\end{proof}


\begin{lemma}\label{lemma:proba_3}
    Let $\epsilon >0$, let $t \geq 16$ be an integer, and let $k \geq \frac{8t}{\epsilon}$ be an integer.
    Let $U,V \in (\mathbb{F}_2^{t} \setminus\{\vec{0}\})^{\lceil\epsilon k/8\rceil}$ be drawn uniformly at random and 
    $W \in \mathbb{F}_2^{\lceil\epsilon k/8\rceil \times \lceil\epsilon k/8\rceil}$ be fixed.
    Then $\Pr[U^\top \cdot V = W] \leq 2^{-t\epsilon k/128}$.
\end{lemma}
% \FHO{Il me semble qu'il manque une condition dans ce lemme. Si $\lfloor \epsilon k/4\rfloor>t/2+1$, alors le premier coefficient binomial n'est pas defini et en plus on a clairement $\Pr[\rk(U) \leq t/2]=1$ si $U$ a moins que $t/2$ colonnes.}
% \CR{Oui tu as raison, on a un pb st $\lfloor \epsilon k/4\rfloor \leq t/2$.
% Je renforce l'hypothese en $k \geq \frac{8t}{\epsilon}$. Ecrites comme je viens de faire les conditions pourraient etre plus precises, mais ca change rien dans la suite.}
% \CR{J'ai du remplacer $\epsilon k/4$ par $\epsilon k /8$ pour pouvoir l'appliquer dans la suite. J'espère que je n'ai pas cree de nouvelles erreurs de calculs! Aussi il suffit d'avoir $\lceil$ au lieu de $\lfloor$ donc ca simplifie un peu certain calculs.}
\begin{proof}
    Note that since $k \geq \frac{8t}{\epsilon}$, we have 
    %$\lfloor\epsilon k/4\rfloor \geq \frac{\epsilon k}{8}$, and
    $\lceil \epsilon k/8 \rceil \geq t \geq t/2+1$.
    First we bound the probability that $\rk(U) < t/2+1$. 
    If $U$ has rank at most $t/2$, then there is a choice of $\lfloor t/2\rfloor$ columns of $U$ such that all the other ones are in the linear span of these selected columns.
    Since the linear span of $\lfloor t/2\rfloor$  vectors has dimension at most $t/2$, and so size at most $2^{t/2}$, we deduce the following.
    % \[
    % \begin{split}
    %     \Pr[\rk(U) \leq t/2] &\leq \binom{\lfloor \epsilon k/8\rfloor}{\lfloor t/2\rfloor}\left(\frac{2^{\lfloor t/2\rfloor}-1}{2^t -1} \right)^{\lfloor \epsilon k/4 \rfloor -t/2} \\
    %     &\leq \binom{\lfloor\epsilon k/4\rfloor}{\lfloor t/2\rfloor}\left(\frac{2^{t/2}-1}{2^t -1} \right)^{\epsilon k/8-t/2} \\
    %     &\leq 2^{\epsilon k/8}\left(\frac{2^{t/2}-1}{2^t -1} \right)^{\epsilon k/16} \\
    %     &\leq 2^{\epsilon k/8} (2^{t/2})^{-\epsilon k/16} \\
    %     &\leq 2^{\epsilon k/8} 2^{-\epsilon k t/32} \\
    %     &\leq 2^{- t \epsilon k/32} \\
    % \end{split}    
    % \]
    % since $t \geq 8$.
    
% \FHO{Je ne vois pas du tout comment tu arrives a la troisieme ligne a partir de la deuxieme. Aussi, d'ou vient le +1 dans le coefficient binomial?}
% \CR{Le $+1$ etait une erreur. Ah oui j'ai zappé le $-t/2$!!! pardon, c'est corrige bientot.
% Je viens aussi de rajouter les arrondis pour le cas $t$ impair.}
% \FHO{je crois que quand tu passes de la deuxieme a la troisieme ligne, il y a une estimation qui va dans le mauvais sense: genre: $\binom{\lfloor\epsilon k/4\rfloor}{\lfloor t/2\rfloor}\leq 2^{\lfloor\epsilon k/4\rfloor}\leq 2^{\epsilon k/8}$ et la derniere inegalite n'est pas juste}
% \JD{Je suis d'accord avec Flo, je propose la correction en dessous avec $t > 16$: . }\CR{Oui c'est vrai, merci pour la correction.}
% \CR{Finalement j'ai du encore changer pour arriver a $\lceil \epsilon k/8 \rceil$ au lieu de $\lfloor \epsilon k/4 \rfloor$.}\FHO{Je ne crois pas que l'estimation $\epsilon k/8-t/2\leq \epsilon k/16$ entre la deuxieme et la troisieme ligne est correcte. Aussi, ou utilise-t-on que $k \geq 16$?}
% \CR{Je crois que c'est bon: $k \geq 8t/\epsilon$ donc $t/2 \leq \epsilon k/16$. Le $t \geq 16$ est utilisé pour obtenir l'avant derniere ligne.}\FHO{Pardon, tu as raison. Il faudrait peut-etre mentionner ca explicitement en haut. Par contre, le probleme en fin de la demonstration persiste.}
    \[
    \everymath={\displaystyle}
    \renewcommand{\arraystretch}{2.5}
    \begin{array}{r l l}
        \Pr\left[\rk(U) \leq t/2\right] &\leq
        \binom{\lceil \epsilon k/8 \rceil}{\lfloor t/2\rfloor}\left(\frac{2^{\lfloor t/2\rfloor}-1}{2^t -1} \right)^{\lceil \epsilon k/8 \rceil -\lfloor t/2\rfloor } &\\
        &\leq \binom{\lceil\epsilon k/8\rceil}{\lfloor t/2\rfloor}\left(\frac{2^{t/2}-1}{2^t -1} \right)^{\epsilon k/8-t/2} & \\
        &\leq 2^{\epsilon k/8+1}\left(\frac{2^{t/2}-1}{2^t -1} \right)^{\epsilon k/16} & \text{ because $\frac{t}{2} \leq \frac{\epsilon k}{16}$ since } k\geq \frac{8t}{\epsilon} \\
        &\leq 2^{\epsilon k/4} (2^{t/2})^{-\epsilon k/16} & \\
        &\leq 2^{\epsilon k t/64} 2^{-\epsilon k t/32} & \text{ because } t \geq 16 \\
        &= 2^{- \epsilon k t/64} & \\
    \end{array}    
    \] 
    Now we assume that $\rk(U)> t/2$.
    Then for every column $v$ of $V$, $v$ must be chosen in an affine space of dimension at most $t - \lfloor t/2\rfloor -1 \leq t/2$.
    It follows that
    \[
    \begin{split}
        \Pr[U^\top \cdot V = W \mid \rk(U) > t/2] & \leq 
        \left(\frac{2^{t/2}-1}{2^{t}-1} \right)^{\lceil\epsilon k/8 \rceil} \\
        &\leq (2^{-t/2})^{\epsilon k/8} \\
        &\leq 2^{-t\epsilon k/16} \\
    \end{split}
    \]
    Therefore
    \[
    \begin{split}
        \Pr[U^\top \cdot V = W] & \leq \Pr\left[\rk(U) \leq t/2\right] + \Pr[U^\top \cdot V = W \mid \rk(U) > t/2]\\
        &\leq 2^{-t\epsilon k/16} + 2^{-t \epsilon k/64} \leq 2\cdot 2^{-t \epsilon k/64}%=2^{-t \epsilon k/64+1}.\\\leq 2^{-t\epsilon k/128}\\
    \end{split}
    \]
%     \FHO{as $t \epsilon k\geq 8 t^2 \geq 128$.}
% %    \FHO{ on a calcule $2^{-t\epsilon k/64}$ avant et on utilise $2^{-t\epsilon k/32}$  ici.}
%     as $2^{-t\epsilon k/128} \leq 2^{-t^2/16} \leq \frac{1}{2}$ \FHO{d'ou vient le $t^2$? A mon avis, on a juste $t\epsilon k \geq 128$ ce qui implique resultat}since $k \geq \frac{8t}{\epsilon}$ and $t \geq 4$, and because $2x^2 \leq x$ for every $0 \leq x \leq 1/2$.
%     \CR{J'ai juste remplacé $k$ par $8t/\epsilon$.}\FHO{et comment on utilse $2^{-t\epsilon k/128}\leq \ldots$ pour demontrer $2^{-t\epsilon k/128}\geq \ldots$?}\CR{Oui c'etait pas clair!! Mais ta façon de faire le calcul est surement plus facile.}
%     \FHO{Ca te va comme ca?}
%     \JD{Histoire de trancher, je vous propose:}\CR{J'achete.} 
    We know that $t\epsilon k \geq 8 t^2 \geq 2 \cdot 64$, thus $2^{-t \epsilon k / 64} \leq 1/4$. As $2x \leq \sqrt{x}$ for any $x \in [0, 1/4]$, we end with $\Pr[U^\top \cdot V = W]  \leq 2 \cdot 2^{-t\epsilon k/64} \leq 2^{-t\epsilon k/128}$.
\end{proof}
For technical reasons, we prove the following seemingly weaker restatement of Theorem \ref{thm:2+eps}.

%\begin{restatable}{theorem}{pluseps}
\begin{theorem}\label{thm:2+eps+2}
    There exists a function $f: \mathbb{R}_{>0} \to \mathbb{N}$ such that for every $\epsilon>0$ and every positive integer $k$, 
    if $T$ is an $n$-vertex tournament with $n \geq 2k+ 2 \epsilon k$, then $\sinv_k(T) \leq f(\epsilon)$.
%\end{restatable}
\end{theorem}
%\pluseps*

It is not difficult to see that Theorem~\ref{thm:2+eps+2} actually implies Theorem~\ref{thm:2+eps}. Indeed, given a function $f$ like in Theorem~\ref{thm:2+eps+2} at hand, define $f':\mathbb{R}_{>0} \to \mathbb{N}$ by $f'(\epsilon)=\max\{\frac{4}{\epsilon},f(\frac{\epsilon}{2})\}$. Let $T$ be a tournament with $|V(T)|\geq 2k+1 +\epsilon k$ for some positive integer $k$. If $k \leq \frac{2}{\epsilon}$, then Theorem \ref{thm:M<2k} yields $\sinv_k(T)\leq 2k\leq \frac{4}{\epsilon}\leq f'(\epsilon)$. Otherwise, we have $|V(T)|\geq 2k+1+\epsilon k\geq 2k+2 + \frac{\epsilon}{2}k$, so $\sinv_k(T)\leq f(\frac{\epsilon}{2})\leq f'(\epsilon)$ by Theorem~\ref{thm:2+eps+2}.



\begin{proof}
% \FHO{je crois que ce serait beaucoup plus claire si on arriverait de definir avant la demonstration. Pas forcement de maniere explicte, mais de la facon qu'on dit: Soit $f(e)$ un entier tel que les inegalites suivantes sont satisfaites:.... Apres, on peut supposer que $n \geq f(\epsilon)$ et ce serait beaucoup plus claire apres que le choix de f depend pas de k.}

Without loss of generality, we may assume $\epsilon \leq \frac{1}{3}$.
Let $C$ be a constant such that $\sinv_k(T) \leq 1$ if $n \geq Ck$, which exists by Theorem~\ref{nk1}. %Clearly, we can assume that $n \leq Ck$.
Let $t$ be the smallest integer such that 
\begin{itemize}
    \item $t\geq 16$, 
    \item $t \geq \log(1+\frac{48}{\epsilon})$, and
    \item $t \geq \frac{128}{\epsilon}(2C+2+\epsilon/{\color{blue}4}) +16$.
\end{itemize}
Clearly, $t$ is well defined and depends only on $\epsilon$.
Let $k_0(\epsilon)$ be the smallest integer such that for every $k' \geq k_0(\epsilon)$ 
\begin{equation}\label{eq:condition_k_large_enough}
(2^t-1) \exp\left(-\epsilon^2\frac{(2+\epsilon)k'}{24}\right) + 3(Ck')^2\exp\left(-\epsilon^2\frac{(2+\epsilon)k'}{4096}\right) + 2^{-k'} <1.
\end{equation}

% \FHO{Ici, ton choix de $k_0(\epsilon)$ depend de $t$. Je ne crois pas que c'est bon, parce que tu dois choisir $t$ dependant de $\epsilon$ et puis il faut que ca marche pour chaque $k$.}
% \CR{Ok, j'ai modifier un peu tout ca. Maintenant on defini $t$ tel que toutes les inegualite necessaire a la suite soient vraies. Puis on definit $k_0$ en utilisant $t$. Si $k<k_0$, on a $\sinv_k \leq 2k_0$ et on est bon, sinon on commence la preuve.}
% \FHO{ok, je crois que ca devrait marcher. Il faut donc enfin dire que $f(\epsilon)=\max\{t,2k_0(\epsilon)-2\}$.}

%If $k < k_0(\epsilon)$, then we conclude directly using Theorem~\ref{thm:M<2k} that $\sinv_k(T) \leq 2k_0-2$.
%Similarly, if $k \leq \frac{8t}{\epsilon}-1$, then we conclude by Theorem~\ref{thm:M<2k} that $\sinv_k(T) \leq \frac{16}{\epsilon}-2$.
%Now we assume $k \geq \frac{8t}{\epsilon} \geq \frac{8}{\epsilon} \geq 24$ and that \eqref{eq:condition_k_large_enough} holds, and we will show that $\sinv_k(T) \leq t$.
%This will implies the theorem by taking $f(\epsilon) = \max\{t,k_0(\epsilon),\frac{8t}{\epsilon}\}$.
%Note that $k \geq \frac{8t}{\epsilon} \geq \frac{8}{\epsilon}$.

We now prove the statement for $f(\epsilon)=\max\{t,2k_0(\epsilon)-2,\lceil\frac{16t}{\epsilon}\rceil-2\}$. If $k < k_0(\epsilon)$, then we conclude directly using Theorem~\ref{thm:M<2k} that $\sinv_k(T) \leq 2k\leq 2k_0(\epsilon)-2\leq f(\epsilon)$. Similarly, if $k \leq \frac{8t}{\epsilon}-1$, then we conclude by Theorem~\ref{thm:M<2k} that $\sinv_k(T) \leq \frac{16t}{\epsilon}-2\leq f(\epsilon)$. 
 Moreover, if $n \geq Ck$, we have $\sinv_k(T)\leq 1 \leq f(\epsilon)$.
Henceforth, we may assume $k\geq \max\{k_0(\epsilon), \frac{8t}{\epsilon}-1\}$ and $n \leq Ck$.
\medskip



%\FHO{If $n\leq 17$, then by Theorem \ref{thm:M<2k}, we have $\sinv_k(T)\leq 16 \leq f(\epsilon)$. We may hence assume that $n \geq 18$.}



For every vertex $u \in V(T)$, we choose uniformly and independently at random a vector $\vec{u} \in \mathbb{F}_2^t \setminus \{\vec{0}\}$.
For $i\in [t]$, let $X_i = \{u \in V(T) \mid \vec{u}_i =1\}$.
We will prove that with positive probability, the tournament $T' = \Inv(T; X_1, \dots, X_t)$ is $k$-strong.
Note that for every arc $uv \in A(T)$, we have $uv\in A(T')$ if and only if $\vec{u} \cdot \vec{v}=0 \mod 2$.
%and otherwise $vu$ is in $T'$.
For two subsets $A$ and $B$ of vertices of $T'$, a {\bf directed $(A,B)$-matching} is a set of arcs with tails in $A$, heads in $B$, and without common tail or common head.
\begin{claim}\label{claim:decompose_into_events_A_B_C_}
    If $T'$ is not $k$-strong, then at least one of the following events occurs:
    \begin{enumerate}[label=\Alph*]
    \item[$E_1$]: there is a vector $z \in \mathbb{F}_2^t \setminus \{\vec{0}\}$ such that $|\{v \in V(T)\mid \vec{v} \neq z\}|\leq k$,
    \item[$E_2$]: there are $u,v \in V(T)$ with $\vec{u} \neq\vec{v} $ such that $\min\{|N_{T'}^+(u)\cap N_{T'}^-(v)|,|N_{T'}^+(u)\cap N_{T'}^+(v)|,|N_{T'}^-(u)\cap N_{T'}^-(v)|\}\leq (1+\epsilon/4)\frac{k}{2}$,
    \item[$E_3$]: there are sets $A,B \subseteq V(T')$ with $|A|,|B| \geq (1+\epsilon/4)\frac{k}{2}$ 
    with no directed $(A,B)$-matching of size $\frac{k}{2}$.
    %such that the underlying graph of $(A \cup B, \delta_{T'}(A,B))$ does not contain a matching of size $\frac{k}{2}$.
    \end{enumerate}
\end{claim}

\begin{proofclaim}
    Assume that none of~$E_1$,~$E_2$ and~$E_3$ holds.
    Suppose for a contradiction that there is a set $X$ of at most $k-1$ vertices, and a partition $(V_1,V_2)$ of $V(T'-X)$ into nonempty sets such that $V_2 \Rightarrow V_1$ in $T'-X$.
    Since~$E_1$ does not hold, there exist $x,y\in V_1 \cup V_2$  with $\vec{x} \neq \vec{y}$. If both $x$ and $y$ are in $V_1$ (resp. $V_2$), consider $v \in V_2$ (resp. $u \in V_1$) and either $\vec{x}\neq\vec{v}$ or $\vec{y}\neq\vec{v}$ (resp. $\vec{x}\neq\vec{u}$ or $\vec{y}\neq\vec{u}$).
    If $x \in V_1$ and $y \in V_2$ we set $u=x$ and $v=y$,
    and if $y \in V_2$ and $x \in V_1$ we set $u=y$ and $v=x$.
    In all cases, there are $u\in V_1$ and $v \in V_2$ with $\vec{u} \neq \vec{v}$.

    Now, as ~$E_2$ does not hold, we have $|N_{T'}^+(u)\cap N_{T'}^-(v)|,|N_{T'}^+(u)\cap N_{T'}^+(v)|,|N_{T'}^-(u)\cap N_{T'}^-(v)|\geq(1+\epsilon/4)\frac{k}{2}$.
    Finally, as~$E_3$ does not hold, 
    there is a directed $(N_{T'}^+(u)\cap N_{T'}^+(v), N_{T'}^-(u)\cap N_{T'}^-(v))$-matching $M$ of size $k/2$ in $T'$.
    For every arc $e=xy\in M$, observe that $P_e=uxyv$ is a directed $(u,v)$-path in $T'$. 
    Furthermore, for every $x \in N_{T'}^+(u)\cap N_{T'}^-(v)$, observe that $P_x=uxv$ is a directed $(u,v)$-path in $T'$. 
    This yields a collection of at least $k$ internally vertex-disjoint $(u,v)$-paths in $T'$,
    a contradiction since every $(u,v)$-path meets $X$ which has size at most $k-1$.
\end{proofclaim}

We will show that with high probability none of the events ~$E_1$,~$E_2$ and~$E_3$ occurs.


\begin{claim}\label{claim:proba_A}
$\Pr(E_1) \leq (2^t-1) \exp(-\epsilon^2n/24)$
\end{claim}

\begin{proofclaim}
If $\vec{c} \in \mathbb{F}_2^t \setminus\{\vec{0}\}$ is fixed, then $Y_{\vec{c}} = |\{u \in V(T) \mid \vec{u} \neq \vec{c}\}|$ is a random variable 
having a binomial law with parameters $n$ and $1-\frac{1}{2^t-1}$. 
As
%$n \geq 2k \geq 48$\FHO{c'est utilise ou?} \JD{Pas utile je crois.}
$\epsilon  \leq \frac{1}{3}$ and $t\geq 2$, we have $k \leq \frac{1}{2}n\leq \frac{2}{3}\cdot\frac{5}{6}n\leq(1-\frac{1}{2^t-1})(1-\epsilon/2)n$. By Lemma~\ref{chern} (Chernoff's bound), and because $t \geq 2$, we have
\begin{align*}
%\Pr[\exists X \subseteq V(T), |X| < k, |\{\vec{u} \mid u \not\in X\}| \leq 1] &= 
\Pr[\exists \vec{c} \in \mathbb{F}_2^t\setminus\{\vec{0}\}, Y_{\vec{c}} < k] 
&\leq \sum_{\vec{c} \in \mathbb{F}_2^t\setminus\{0\}} \Pr[Y_{\vec{c}} < k] \\
&\leq \sum_{\vec{c} \in \mathbb{F}_2^t\setminus\{0\}} \Pr\left[Y_{\vec{c}} < \left(1-\frac{\epsilon}{2}\right)\left(1-\frac{1}{2^t-1}\right)n\right] \\
&\leq (2^t-1)\exp\left(-(\epsilon/2)^2\left(1-\frac{1}{2^t-1}\right)n/3\right) \\
&\leq (2^t-1)\exp(-\epsilon^2 n/24),
\end{align*}
as claimed.
\end{proofclaim}



% Let $s,t$ be two distinct vertices, and let $C = N^+_{T'}(s) \cap N^-_{T'}(t), A = N^+_{T'}(s) \setminus C, B = N^-_{T'}(t) \setminus C$.
% We assume that $\vec{s} \neq \vec{t}$ (otherwise Claim~\ref{claim:proba_1} applies).
% Observe that if there is a matching of size $k-|C|$ from $A$ to $B$, then together with vertices in $C$, this gives $k$ vertex-disjoint
% $(s,t)$-paths in $T'$. We will show that this happens with high probability.

% \begin{claim}
%     $\Pr[\min(|A|,|B|,|C|) \leq (1+\epsilon/2)k/2] \leq 3\exp\left(-\epsilon^2\frac{n-2}{192}\right)$
% \end{claim}

\begin{claim}\label{claim:proba_B}
    $\Pr(E_2) \leq 3\exp\left(-\epsilon^2\frac{n-2}{4096}\right)$
\end{claim}

\begin{proofclaim}
    Let $u,v$ be distinct vertices and let
    $A = N_{T'}^+(u)\cap N_{T'}^-(v), B = N_{T'}^+(u)\cap N_{T'}^+(v)$ and $C=N_{T'}^-(u)\cap N_{T'}^-(v)$.
    Let $X \in \{A,B,C\}$.
    Once $\vec{u}$ and $\vec{v}$ revealed, for every vertex $w \neq u,v$, let $Y_w$ be a random variable with $Y_w=1$ if $w \in X$, $Y_w=0$ otherwise.
    By Lemma~\ref{lemma:proba_2}, $Y_w$ is a random variable following a Bernoulli distribution whose parameter is at least $\frac{1}{4}(1-\frac{3}{2^t-1})$. Further, the $Y_w$ are mutually independent.
%    \FHO{Ici, je ne suis pas sure. Qu'est-ce qui est la definition exacte de $X_w$? Je crois qu'une fois $\vec{u}$ et $\vec{v}$ sont fixes, on connait la distribution exacte de $Y_w$ et les $Y_w$ sont effectivement independents. Apres, je vois deux solutions possibles: On utilise une forme plus generale de Chernoff qui prend en compte plusieurs variables, c'est probablement moche. Sinon, pour les $Y_w$ tel que la probabilite $p_w>\frac{1}{2}-\frac{3}{4}\frac{1}{2^t-1}$, on definit $X_w$ de maniere qu'on prend le resultat de $Y_w$ et le multiplie par 0 avec probabilite $\frac{p_w}{\frac{1}{2}-\frac{3}{4}\frac{1}{2^t-1}}$ independemment. Apres, je crois que tous les $X_w$ restent independents et ont la distribution souhaitee, donc on peut appliquer Chernoff.} 
%    \CR{Oui on peut faire ca. C'est pas très compliqué en fait.}

    We now define a random variable $X_w$ for every $w \in V(T)\setminus \{u,v\}$ as follows.
    If $Y_w=0$, set $X_w=0$, and otherwise set $X_w=1$ with probability $\frac{\frac{1}{4}(1-\frac{3}{2^t-1})}{\Pr[Y_w=1]}$ and $X_w=0$ otherwise, where the latter random experiments are executed independently. Observe that, as the $Y_w$ are mutually independent, so are the $X_w$. Moreover, $\Pr[X_w=1] = \frac{1}{4}(1-\frac{3}{2^t-1})$.
    
    We have $(1+\epsilon/4)k/2 \leq \frac{1+\epsilon/4}{2+\epsilon}(n-2)/2 = \frac{1}{4}(1-\frac{\epsilon/4}{1+\epsilon/2}) (n-2)/2 \leq \frac{1}{4}(1-\epsilon/8)(n-2)$ since $n \geq (2+\epsilon)k+2$.
    %\FHO{L'egalite du deuxieme et troisieme terme n'est pas correct}.
    Moreover, as $t \geq  \log(\frac{48}{\epsilon}+1)$
    %\FHO{je crois ca devrait etre $t \geq \log(1+\frac{24}{\epsilon})$}
    we have 
    $\frac{1}{4}(1-\epsilon/8)\leq \frac{1}{4}(1-\epsilon/16)^2 \leq \frac{1}{4}\left(1-\frac{3}{2^t-1}\right)(1-\frac{\epsilon}{16}) = \left(\frac{1}{4}-\frac{3}{4(2^t-1)}\right)(1-\epsilon/16)$.
    Hence $(1+\epsilon/4)k/2 \leq (1-\epsilon/16)\left(\frac{1}{4}-\frac{3}{4(2^t-1)}\right)(n-2)$, and by Chernoff's bound (Lemma~\ref{chern})
     \begin{align*}
        \Pr[|X| \leq (1+\epsilon/4)k/2]
        & \leq \Pr\left[|X| \leq \left(1-\frac{\epsilon}{16}\right)\left(\frac{1}{4}-\frac{3}{4(2^t-1)}\right)(n-2)\right]\\
        &\leq \exp\left(-(\epsilon/16)^2\left(\frac{1}{4}-\frac{3}{4(2^t-1)}\right)\frac{n-2}{2}\right)\\    
        &\leq \exp\left(-\epsilon^2\frac{n-2}{4096}\right) 
   \end{align*}
    since $t \geq 5$ implies $\frac{3}{2^t-1} \leq 1/8$. 
    Hence by the union bound $\Pr[\min\{|A|,|B|,|C|\} \leq (1+\epsilon/4)\frac{k}{2}] \leq \sum_{X \in \{A,B,C\}}\Pr[|X| \leq (1+\epsilon/4)\frac{k}{2}]\leq 3\exp\left(-\epsilon^2\frac{n-2}{4096}\right)$.
\end{proofclaim}

For two disjoint sets of vertices $X,Y$ in $T'$, we denote by $\mu(X,Y)$ the size of a largest directed $(X,Y)$-matching in $T'$.

\begin{claim}\label{claim:proba_C}
    $\Pr(E_3) \leq 2^{-k}$.
\end{claim}
    
\begin{proofclaim}
    Let $A, B \subseteq V(T')$ be disjoint sets of $\lceil (1+\epsilon/4)\frac{k}{2} \rceil$ vertices.
    We shall prove that with high probability there is a directed $(A,B)$-matching in $T'$ of size at least $\frac{k}{2}$.
    
    Let $M$ be a maximal directed $(A,B)$-matching and let $Y_A$ (resp. $Y_B$) the the set of vertices in $A$ (resp. in $B$) incident to no arc of $M$.
    Then $Y_B \Rightarrow Y_A$ in $T'$ since $M$ is maximal.
    Moreover, if $|M| \leq k/2$, then $|Y_A|,|Y_B| \geq (1+\epsilon/4)\frac{k}{2}-\frac{k}{2} = \frac{\epsilon}{8}k$.
    
    For every such $Y_A \subseteq A, Y_B \subseteq B$, we identify $Y_A$ and $Y_B$ with the matrices whose columns are the $\vec{u}$ for $u \in Y_A$ (resp. $u \in Y_B$).
    We also denote by $T(Y_A,Y_B)$ the $|A| \times |B|$ matrix whose cell $(u,v)$ equal $1$ if and only if $uv \in A(T)$.
    Then observe that $Y_B \Rightarrow Y_A$ in $T'$ if and only if $Y_B^\top \cdot Y_A = T(Y_B,Y_A)$.
    By these observations, we have
    \[
    \begin{array}{r c l}
       \Pr[\mu_{T'}(A,B)<k]  & \leq & %\Pr[\exists X_A\subseteq A, X_B \subseteq B', |X_A \cup X_B| \leq k/2, X_A \cup X_B \text{ vertex cover of } (A',B')] \\
       \Pr[\exists Y_A \subseteq A, |Y_A| \geq \frac{\epsilon}{8}k, \exists Y_B \subseteq B, |Y_B| \geq \frac{\epsilon}{8}k, Y_B \Rightarrow Y_A \text{ in } T']  \\
       & \leq & \Pr[\exists Y_A \subseteq A, |Y_A| \geq \frac{\epsilon}{8}k, \exists Y_B \subseteq B, |Y_B| \geq \frac{\epsilon}{8}k, Y_B^\top \cdot Y_A = T(Y_A,Y_B)]  \\
       & \leq & 2^{2\lceil (1+\epsilon/{\color{blue} 4})k/2 \rceil} 2^{-t\epsilon k/128} \hspace{1cm} \text{ by Lemma~\ref{lemma:proba_3} and the Union Bound} \\
       & \leq & 2^{(1+\epsilon/{\color{blue}4})k+2} 2^{-t\epsilon k/128} \hspace{1.325cm}  \\
       & \leq & 2^{-(2C+1)k} \\
    \end{array}
    \]
    
    as $t \geq \frac{128}{\epsilon}(2C+2+\epsilon/{\color{blue}4}) + 16$ and $\epsilon k/64 \geq \frac{1}{8}$.
    It follows from the union bound that $\Pr(E_3) \leq 2^{2n} 2^{-(2C+1)k} \leq 2^{-k}$ using the fact that $n \leq Ck$.
\end{proofclaim}

We can now conclude using Claims~\ref{claim:decompose_into_events_A_B_C_},~\ref{claim:proba_A},~\ref{claim:proba_B}
and~\ref{claim:proba_C} and the Union Bound: 
\[
\begin{split}
\Pr[T' \text{ not } k\text{-strong}] &\leq \Pr(E_1) + \Pr(E_2)+\Pr(E_3) \\
&\leq (2^t-1) \exp\left(-\epsilon^2\frac{n}{24}\right) + 3n^2\exp\left(-\epsilon^2\frac{n-2}{4096}\right) + 2^{-k} \\
&\leq (2^t-1) \exp\left(-\epsilon^2\frac{(2+\epsilon)k}{24}\right) + 3(Ck)^2\exp\left(-\epsilon^2\frac{(2+\epsilon)k}{4096}\right) + 2^{-k} \\
&<1, \\
\end{split}
\]
by \eqref{eq:condition_k_large_enough}.
This proves that there exist $X_1, \dots, X_t \subseteq V(T)$ such that $T'=\Inv(T;X_1,\dots, X_t)$ is $k$-strong.
\end{proof}







