
\section{Complexity of computing \texorpdfstring{$\sinv_k$}{sinvk} and \texorpdfstring{$\sinv'_k$}{sinv'k}\label{sec:complexity}}
%%%%%%%%%%%%%%%%%%%%%%%%%%%%%%%%%%%%%%%%%%%%%%%

In this section, we deal with the complexity of computing the parameters $\sinv_k(D)$ and $\sinv'_k(D)$ for a given digraph $D$. More concretely, we prove Theorems \ref{archard1},\ref{approx2},\ref{verhard}, and \ref{approx1}.





%The rest of this section consists of the proofs of Theorems \ref{verhard},\ref{archard1}, \ref{approx1} and \ref{approx2}. 
In Subsection \ref{eins}, we prove Theorems~\ref{archard1}  and \ref{verhard}, for $t=1$ and every positive integer $k$.   
In Subsection \ref{zwei} (resp. Subsection~\ref{drei}), we prove Theorem~\ref{verhard} (resp. Theorem~\ref{archard1}), in the remaining cases, that is for every $t\geq 2$ and every positive integer $k$.  
  Finally, in Subsection~\ref{vier} we prove Theorems~\ref{approx1} and~\ref{approx2}.
Note that all the studied decision problems are in NP since a $k$-strenghtening (resp. $k$-arc-strenghthening) family ${\cal X}$ is a certificate as checking whether $\Inv(D; {\cal X})$ is $k$-strong (resp. $k$-arc-strong) can be done in polynomial time. Therefore we just need to prove their hardness.


\subsection{One single inversion}\label{eins}
%%%%%%%%%%%%%%%%%%%%%%%%%%%%%%%%%%%%%%%%%%%%%%%

In this section, we show hardness results for the case that a connectivity property is supposed to be achieved by a single inversion.

We need the following problem. For some integer $k\geq 1$, an instance of the {\sc Monotone Equitable $k$-SAT} problem (ME$k$SAT) consists of a set of variables $X$ and a set of clauses $\mathcal{C}$ each of which contains exactly $2k+1$ nonnegated variables and the question is whether there is a truth assignment $\phi:X \rightarrow \{\true,\false\}$ such that every clause in $\mathcal{C}$ contains at least $k$ true and $k$ false variables with respect to $\phi$. %Observe that the monotone equitable $1$-SAT problem is exactly monotone NAE3SAT.

%I would appreciate a reference for the following result. In worst case, I know how to prove it by hand.
%\FH{NotAllEqual3SAT is proved NP-complete in \\
% S. Porschen, T. Schmidt, E. Speckenmeyer, and A. Wotzlaw. XSAT
%and NAE-SAT of linear CNF classes. Discrete Applied Mathematics,
%167:1?14, 2014. Maybe the others as well}
\begin{proposition}\label{equ}
ME$k$SAT is NP-hard for every $k \geq 1$.
\end{proposition}
\begin{proof}
    For $k=1$, we exactly have the MNAE3SAT problem, which is well-known to be NP-hard, see for example \cite{schaefer1978complexity}. We now proceed by induction. We fix some integer $k \geq 1$, assume that ME$k$SAT is NP-hard and show through a reduction from ME$k$SAT that so is ME$(k+1)$SAT.

    Let $(X,\mathcal{C})$ be an instance of ME$k$SAT. We now create an instance $(X',\mathcal{C}')$ of ME$(k+1)$SAT. First, let $Y=\{y_1,\ldots,y_{k+3}\}$ and $Z=\{z_1,\ldots,z_{k+3}\}$ be sets of new variables. Now let $\mathcal{C}_1$ the the set of all clauses $C \subseteq Y \cup Z$ with $|C|=2k+3$ and $k+1\leq |C \cap Y|\leq k+2$. Moreover, we define $\mathcal{C}_2 = \{C\cup \{y_1, z_1\} \mid C \in \mathcal{C}\}$. Finally, we set $X'=X\cup Y \cup Z$ and $\mathcal{C}'=\mathcal{C}_1\cup \mathcal{C}_2$. 

    Before giving the main proof that our reduction works indeed, we show the following intermediate result.  \begin{claim}\label{force}
        For any mapping $\phi: Y \cup Z \rightarrow \{\true,\false\}$ that satisfies $(Y \cup Z, \mathcal{C}_1)$, we have $\phi(y_1)=\ldots=\phi(y_{k+3})\neq \phi(z_1)=\ldots=\phi(z_{k+3})$.
    \end{claim}
    \begin{proofclaim}
        First suppose that both $Y$ and $Z$ contain both $\true$ and $\false$ variables with respect to $\phi$, say $\phi(y_1)=\phi(z_1)=\true$ and $\phi(y_2)=\phi(z_2)=\false. $ %\CR{It is a bit dangerous to say they are $y_1,z_1$, since $z_1$ is particular in the construction.}. 
        Let $C_1=Y \cup Z \setminus \{y_2,z_2,z_3\}$ and $C_2= Y \cup X \setminus \{y_1,z_1,z_3\}$. Further, let $\alpha_1,\alpha_2$ be the number of true variables with respect to $\phi$ in $C_1$ and $C_2$, respectively. As $C_1,C_2 \in \mathcal{C}_1$, we have $k+1\leq \alpha_i\leq k+2$ for $i=1,2$. On the other hand, by assumption, we have $\alpha_2=\alpha_1-2$. This yields $k+1 \leq \alpha_2=\alpha_1-2\leq (k+2)-2=k$, a contradiction.

        We may hence suppose by symmetry that $\phi(y_1)=\ldots=\phi(y_{k+3})=\true$. Suppose that $Z$ contains a true variable with respect to $\phi$, say $\phi(z_1)=\true$. Then consider the clause $C=Y \cup Z \setminus\{y_1,z_2,z_3\}$. We obtain that $C$ contains at most $k$ negative variables with respect to $\phi$, a contradiction, as $C \in \mathcal{C}_1$. This proves the claim.
    \end{proofclaim}
    
    We now show that $(X',\mathcal{C}')$ is a positive instance of ME$(k+1)$SAT if and only if $(X,\mathcal{C})$ is a positive instance of ME$k$SAT.

    First suppose that $(X,\mathcal{C})$ is a positive instance of ME$k$SAT, so there is an assignment $\phi:X \rightarrow \{\true,\false\}$ such that every clause in $\mathcal{C}$ contains at least $k$ true and at least $k$ false variables with respect to $\phi$. 
    Define $\phi':X' \rightarrow \{\true,\false\}$ by $\phi'(x)=\phi(x)$ for all $x \in X$ and $\phi'(y_i)=\true$ and $\phi'(z_i)=\false$ for all $i\in[k+3]$. 
    By construction, every clause in $\mathcal{C}_1$ contains at least $k+1$ true and at least $k+1$ false variables with respect to $\phi'$. 
    Further, for every clause $C\cup \{y_1, z_1\} \in \mathcal{C}_2$, the clause $C$ contains at least $k$ true and at least $k$ false variables with respect to $\phi$. 
    As $\phi'(y_1)=\true$ and $\phi'(z_1)=\false$, we obtain that $C\cup \{y_1, z_1\}$ contains at least $k+1$ true and at least $k+1$ false variables with respect to $\phi'$. 
    Hence $(X',\mathcal{C}')$ is a positive instance of ME$(k+1)$SAT.

    Now suppose that $(X',\mathcal{C}')$ is a positive instance of ME$(k+1)$SAT, so there is an assignment $\phi':X' \rightarrow \{\true,\false\}$ such that every clause in $\mathcal{C}'$ contains at least $k+1$ true and at least $k+1$ false variables with respect to $\phi'$. 
    By Claim \ref{force}, we have $\phi'(y_1)=\ldots=\phi'(y_{k+3})\neq \phi'(z_1)=\ldots=\phi'(z_{k+3})$.

   
By symmetry and Claim \ref{force}, we may suppose that $\phi'(y_i)=\true$ and $\phi'(z_i)=\false$ for all $i\in[k+3]$. 
Let $\phi$ be the restriction of $\phi'$ to $X$. For every $C \in \mathcal{C}$, the clause $C\cup \{y_1, z_1\}$ contains at least $k+1$ true and at least $k+1$ false variables with respect to $\phi'$. 
As $\phi'(y_1)=\true$ and $\phi'(z_1)=\false$, we obtain that $C$ contains at least $k$ true and at least $k$ false variables with respect to $\phi$. 
Hence $(X,\mathcal{C})$ is a positive instance of ME$k$SAT.
This proves the lemma.
\end{proof}

We can now prove the following results.


\begin{theorem}\label{arc1}
Deciding whether $\sinv'_k(D)\leq 1$ for a given oriented graph $D$ is NP-complete.
\end{theorem}


\begin{theorem}\label{ver1}
Deciding whether $\sinv_k(D)\leq 1$ for a given $k$-strengthenable oriented graph $D$ is NP-complete.
\end{theorem}




%To prove these theorems, we need the following definition and well-known result.
 
%Let $D$ be a digraph and $u,v$ two distinct vertices in $D$. The {\bf strong-connectivity} from $u$ to $v$ in $D$, denoted by $\kappa_D(u,v)$, is the maximal number $\alpha$ such that $D-X$ contains a $(u,v)$-path  for every $X \subseteq V(D)\setminus \{u,v\}$ with $|X|\leq \alpha-1$. For some $S \subseteq V(D)$ and positive integer $k$, we say that $S$ is {\bf $k$-strong in $D$} if $\kappa_D(u,v)\geq k$ for all $u,v \in S$.

%\begin{proposition}\label{lem:kstrong+}
%Let $D$ be a digraph, $S$ a $k$-strong set in $D$ and $v \in V(D)\setminus S$. If $v$ has $k$ in-neighbours in $S$ and $k$ out-neighbours in $S$, then $S \cup \{v\}$ is $k$-strong in $D$.
%\end{proposition}
%\begin{proof}
    %Suppose for contradiction that there is a set $X$ of size at most $k-1$ and a pair $(s,t)$ of vertices in $S \cup \{v\} \setminus X$ such that there is no $(s,t)$-path in $D-X$.
    %If $s,t \in S$, then this contradicts the fact that $D$ is $k$-strong in $S$.
    %Thus exactly one of $s$ and $t$ is $v$. Without loss of generality, suppose $s \in S$ and $t=v$.
    %Since $|N^-(v) \cap S| \geq k$, $v$ has an in-neighbour $t_0 \in S-X$, and as $D$ is $k$-strong in $D$, there is an $(s,t_0)$-path in $D-X$, and so there is an $(s,t)$-path in $D-X$, a contradiction.
%\end{proof}

The proof of Theorems~\ref{arc1} and~\ref{ver1}
is a reduction from ME$k$SAT which is NP-hard by Proposition \ref{equ}. Fix some $k \geq 2$ and let $(X,\mathcal{C})$ be an instance of ME$k$SAT. We construct an oriented graph $D(X,\mathcal{C})$ as follows. 
We start from the disjoint union of a $k$-strong tournament $T$ with vertex set $\{s_1,\ldots,s_{2k+1}\}$ and two stable sets 
$V=\{v_x \mid x\in X\}$ and $W=\{w_C \mid C \in \mathcal{C}\}$.
Next, for every $i\in [k]$ and every $x \in X$, we add an arc from $s_i$ to $v_x$ and for every $i=\{k+1,\ldots,2k\}$ and every $x \in X$, we add an arc from $v_x$ to $s_i$. Finally, for every $x \in X$ and $C \in \mathcal{C}$ with $x \in C$, we add an arc from $v_x$ to $w_C$.

Observe that the digraph $D(X,\mathcal{C})$ is $k$-strengthenable.
Indeed $D(X,\mathcal{C}) \langle S\rangle$ is $k$-strong and hence $S$ is $k$-strong in $D(X,\mathcal{C})$.
Hence by Proposition~\ref{lem:kstrong+}, $S\cup V$ is $k$-strong in $D(X,\mathcal{C})$.
Now, for every $C \in \mathcal{C}$, we reverse $k$ arcs between $w_C$ and its $2k+1$ neighbours in $V$. Then each $w_C$ has at least $k$ in-neighbours and $k$-out-neighbours in $S\cup V$. So, by Proposition~\ref{lem:kstrong+} again, the resulting digraph is $k$-strong.
 

\medskip

The main technical part of the reduction is contained in the proof of the following result.
\begin{claim}\label{haupt}
The following are equivalent:
\begin{enumerate}[label=(\roman*)]
    \item\label{item:haupt_i} $(X,\mathcal{C})$ is a positive instance of ME$k$SAT;
    \item\label{item:haupt_ii} $D(X,\mathcal{C})$ can be made $k$-strong by inverting a single set $Z$;
    \item\label{item:haupt_iii} $D(X,\mathcal{C})$ can be made $k$-arc-strong by inverting a single set $Z$.
\end{enumerate}

\end{claim}
\begin{proofclaim}
Set $D=D(X,\mathcal{C})$.

Suppose that \ref{item:haupt_i} holds, so there is an assignment $\phi:X \rightarrow \{\true,\false\}$ such that every clause in $\mathcal{C}$ contains at least $k$ true and at least $k$ false variables with respect to $\phi$. Let $Z=W \cup \{v_x \mid x \in X, \phi(x)=\true\}$ and let $D'$ be the oriented graph obtained from $D$ by inverting $Z$. Clearly, $S$ is $k$-strong in $D'$. Next, by Lemma~\ref{lem:kstrong+}, we obtain that $S \cup V$ is $k$-strong in $D'$. Now consider some $C \in \mathcal{C}$. As $\phi$ is equitable, $C$ contains at least $k$ variables $x$ which satisfy $\phi(x)=\true$. Hence, by construction, $w_C$ has at least $k$ out-neighbours in $S \cup V$ in $D'$. Further, as $\phi$ is equitable, $C$ contains at least $k$ variables $x$ which satisfy $\phi(x)=\false$. Hence, by construction, $w_C$ has at least $k$ in-neighbours in $S \cup V$ in $D'$. By Lemma~\ref{lem:kstrong+}, $D'$ is $k$-strong, so \ref{item:haupt_ii} is satisfied.

Clearly, \ref{item:haupt_ii} implies \ref{item:haupt_iii}. 

Now suppose that \ref{item:haupt_iii} holds, so $D$ can be transformed into a $k$-arc-strong oriented graph $D'$ by inverting a single set $Z$. We now define an assignment $\phi:X \rightarrow \{\true,\false\}$ in the following way: if $v_x \in Z$, we set $\phi(x)=\true$, otherwise we set $\phi(x)=\false$. In order to prove that $\phi$ has the desired properties, consider some clause $C \in \mathcal{C}$. As $D'$ is $k$-arc-strong, $w_C$ has at least $k$ outgoing arcs in $D'$. It follows that $w_C \in Z$ and there are at least $k$ variables in $X$ with $x \in C$ and $v_x \in Z$. Hence $C$ contains at least $k$ variables $x$ with $\phi(x)=\true$. Next, as $D'$ is $k$-arc-strong, $w_C$ has at least $k$ incoming arcs in $D'$.  As $w_C \in Z$, it follows that there are at least $k$ variables in $X$ with $x \in C$ and $v_x \notin Z$. Hence $C$ contains at least $k$ variables $x$ with $\phi(x)=\false$. We obtain that $\phi$ has the desired properties, so \ref{item:haupt_iii} is satisfied.
\end{proofclaim}

Clearly, the size of $D(X,\mathcal{C})$ is polynomial in the size of $(X,\mathcal{C})$ for fixed $k$, and $D(X,\mathcal{C})$ can be constructed in polynomial time from $(X,\mathcal{C})$. Hence Claim \ref{haupt} implies Theorems~\ref{arc1} and \ref{ver1}.


\subsection{Several inversions  to become \texorpdfstring{$k$}{k}-arc-strong}\label{drei}
%%%%%%%%%%%%%%%%%%%%%%%%%%%%%%%%%%

In order to prove Theorems~\ref{verhard} and \ref{archard1} for $t\geq 2$ and all $k$, we shall use another well-studied graph parameter. Let $G$ be a graph. 
A {\bf cut cover} of $G$ is a collection $X_1,\ldots,X_t$ of subsets of $V(G)$ such that $\cup_{i=1}^t \delta_G(X_i)=E(G)$.
The {\bf cut covering number} of $G$, denoted by $\cc(G)$, is the minimum integer $t$ such that there is a cut cover of size $t$. The following well-known result shows a close relationship between the cut covering number and the classical chromatic number.

\begin{proposition}\label{prop:cc-chi}
For any graph $G$, we have $\cc(G)=\lceil\log(\chi(G))\rceil$.
\end{proposition}

It is well-known that the problem of deciding whether a given graph can be coloured with $t$ colours is NP-hard for any $t\geq 3$, see \cite{schaefer1978complexity}. It hence follows from Proposition \ref{prop:cc-chi} that deciding whether the cut covering number of a given graph is at most $t$ is NP-hard for any integer $t\geq 2$.
We show show a reduction from this problem to the one of deciding whether $\sinv'_k(D) \leq t$  for a given oriented graph $D$.



Given a digraph $D$ and a set $X$ of vertices in $D$, we denote by $\partial_D(X)$ the set of edges of $\UG(D)$ with exactly one endvertex in $X$:
$\partial_D(X) = \delta_{\UG(D)}(X)$.
The main technical part of the reduction is contained in the following Lemma that will be reused in Section \ref{vier}.0
%We also denote by $\delta^+_D(X)$ (resp. $\delta^-_D(X)$) the set of arc in $D$ with tail (resp. head) in $X$ and head (resp. tail) out of $S$.
%For a vertex $x \in V(D)$, we simply write $\delta_D(x)$ for $\delta_D(\{x\})$.





\begin{lemma}\label{lem:reduc2}
Given a graph $G$ and a positive integer $k$, one can construct in polynomial time some oriented graph $D$ such that $\sinv'_k(D)=\cc(G)$ and $|V(D)|=|V(G)|+(2k+1)|E(G)|+2k+1$.
\end{lemma}

\begin{proof}
Let $<$ be an arbitrary ordering on $V(G)$.
We let $V(D)$ contain $V(G)$, a set of vertices  $Z_e=\{z^1_e,\ldots,z_e^{2k+1}\}$ for every $e \in E(G)$ and a set of vertices $X=\{x_1,\ldots,x_{2k+1}\}$. We add arcs to $D$ such that $D\langle X\rangle$ is a $k$-arc-strong tournament. We let $A(D)$ contain an arc from $v$ to $x_i$ for every $v \in V(G)$ and every $i=1,\ldots,k$ and an arc from $x_i$ to $v$ for every $v \in V(G)$ and every $i=k+1,\ldots,2k$. Next for every $e \in E(G)$, we add arcs to $D$ such that $D\langle Z_e\rangle$ is a $k$-arc-strong tournament. Finally, for every $e=uv \in E(G)$ with $u<v$, we add the arcs $uz_e^{i}$ for $i=1,\ldots,k$, the arcs ${z_e^{i}}v$ for $i=2,\ldots,k$ and the arc $v{z_e^{1}}$. This finishes the description of $D$. Observe that $|V(D)|=|V(G)|+(2k+1)|E(G)|+2k+1$. 

\medskip

We now show that $\sinv'_k(D)=\cc(G)$.
\smallskip

We first show that $\sinv'_k(D)\leq \cc(G)$. 
Let $(X_1,\ldots,X_{\cc(G)})$ be an optimal cut cover of $G$. 
Now for every $e=uv \in E(G)$, there is a smallest $\alpha(e)$ such that exactly one of $u$ and $v$ is contained in $X_{\alpha(e)}$. 
Now for $i\in [\cc(G)]$, let $X'_i=X_i\cup\{z_e^1:\alpha(e)=i\}$. 
Let $D'$ be the digraph obtained from $D$ by inverting $X'_1,\ldots,X'_{\cc(G)}$. We will show that $D'$ is $k$-arc-strong. 
Let $S \subseteq V(D)$ and suppose that $\min\{d_{D'}^-(S),d_{D'}^{+}(S)\}<k$ .
By symmetry, we may suppose that $S \cap X \neq \emptyset$. 
As $D'\langle X\rangle=D\langle X\rangle$ is $k$-arc-strong, we obtain $X \subseteq S$. 
Next observe that in $D'$ every $v \in V(G)$ is incident to at least $k$ arcs coming from $X$ and $k$ arcs going to $X$. Thus $V(G)\subseteq S$. Now consider some $e=uv \in E(G)$ with $u<v$. 
%As $D'\langle Z_e\rangle=D\langle Z_e\rangle$ is $k$-arc-strong, we have either $Z_e\subseteq S$ or $Z_e \cap S = \emptyset$. 
As $D'$ by construction contains $k$ arcs from $V(G)$ to $Z_e$ and $k$ arcs from $Z_e$ to $V(G)$, we obtain that $Z_e\subseteq S$. 
Observe that either $vz_e^1,z_e^1u \in A(D')$ (if $u \in X_{\alpha(e)}$) or $z_e^1v,uz_e^1 \in A(D')$ (if $v \in X_{\alpha(e)}$).
Thus $D'$ contains $k$ arcs from $V(G)$ to $Z_e$ and $k$ arcs from $Z_e$ to $V(G)$, and we obtain that $Z_e\subseteq S$.
This yields $S=V(D')$, so $D'$ is $k$-arc-strong. We hence have $\sinv'_k(D)\leq \cc(G)$.


\smallskip

We now show that $\cc(G)\leq\sinv'_k(D)$. Let $(X_1,\ldots, X_t)$ be a smallest collection of sets whose inversion makes $D$ $k$-arc-strong.  Let $D' =\Inv(D ; (X_i)_{i\in [t]})$ and for $i\in[t]$, let $X_i'=X_i \cap V(G)$.
\begin{claim}\label{par1}
Let $e=uv \in E(G)$. Then there is some $i \in [t]$ such that $X'_i$ contains exactly one of $u$ and $v$.
\end{claim}
\begin{proofclaim}
Suppose otherwise. As $D'$ is $k$-arc-strong and $d_{D}(Z_e)=2k$, we have $d_{D'}^-(Z_e)= d_{D'}^+(Z_e)=k$. For every $i \in[t]$ either both $u$ and $v$ or none of $u$ and $v$ are contained in $X_i$. Thus we deduce that for every $j \in [k]$, either both the arcs corresponding to edges in $\partial_{D}(Z_e)\cap \partial_{D}(z_e^{j})$ are inverted or none of them are. Hence, in $D'$,  there is exactly one arc entering $Z_e$ incident to $z_e^{j}$ for $j=2,\ldots,k$, and there are either zero or two arcs entering $Z_e$ incident to $z_e^{1}$. Thus $d_{D'}^-(Z_e)\in \{k-1,k+1\}$, a contradiction.
\end{proofclaim}
By Claim~\ref{par1}, we obtain that $(X_1',\ldots,X_t')$ is a cut cover for $G$. Hence $\cc(G)\leq t=\sinv'_k(D)$.
\end{proof}

Lemma~\ref{lem:reduc2} and the NP-hardness of deciding whether a given graph $G$ satisfies $\cc(G)\leq t$ for all $t\geq 2$ directly imply the following.
\begin{corollary}\label{hujil}
Let $t$ be an integer greater than $1$ and $k$ a positive integer.
Deciding whether $\sinv'_k(D) \leq t$  for a given oriented graph $D$ is NP-complete.
\end{corollary}
Clearly, Theorem \ref{arc1} and Corollary \ref{hujil} imply Theorem \ref{archard1}.

\subsection{Several inversions to become \texorpdfstring{$k$}{k}-strong}\label{zwei}
%%%%%%%%%%%%%%%%%%%%%%%%%%%%%%%%%%


In this subsection, we give a reduction from the cut covering problem to the problem of deciding whether $\sinv_k(D) \leq t$  for a given oriented graph $D$. The main technical part of the reduction is contained in the following lemma that will be reused in Section \ref{vier}.



\begin{lemma}\label{lem:reduc3}
Given a graph $G$ and a positive integer $k$, one can construct in polynomial time a $k$-strengthenable oriented graph $D$ such that $\sinv_k(D)=\cc(G)$ and $|V(D)|=|V(G)|+(2k+1)|E(G)|+2k+1$.
\end{lemma}

\begin{proof}
We let $V(D)$ contain $V(G)$, a set of vertices  $Z_e=\{z^1_e,\ldots,z_e^{2k+1}\}$ for every $e \in E(G)$ and a set of vertices $W=\{w_1,\ldots,w_{2k+1}\}$. 
We add arcs to $D$ such that $D\langle W\rangle$ is a $k$-strong tournament.
We let $A(D)$ contain an arc from $v$ to $w_i$ for every $v \in V(G)$ and every $i\in[k]$ and an arc from $w_i$ to $v$ for every $v \in V(G)$ and every $i\in \{k+1,\ldots,2k\}$.
Next for every $e \in E(G)$, we add arcs to $D$ such that $D\langle Z_e\rangle$ is a $k$-strong tournament. 
Further, for every $e \in E(G)$, $i\in [k]$ and $j\in [k-1]$, we add an arc from $z_e^{i}$ to $w_{j}$ and for every $e \in E(G)$, $i\in\{k+1,\ldots,2k\}$ and $j\in [k-1]$, we add an arc from  $w_{j}$ to $z_e^{i}$. 
Finally, for every $e=uv \in E(G)$, we add the arcs $u z_e^{1}$ and $v z_e^{1}$. This finishes the description of $D$. 
Observe that $|V(D)|=|V(G)|+(2k+1)|E(G)|+2k+1$. 


We now show that $\sinv_k(D)=\cc(G)$.
Note that this implies that $\sinv_k(D)$ is finite and thus $D$ is $k$-strengthenable.

We first show that $\sinv_k(D)\leq \cc(G)$. Let $(X_1,\ldots,X_t)$ be an optimal cut cover of $G$. 
For every $e=uv \in E(G)$, there is a smallest integer $\alpha(e)$ such that exactly one of $u$ and $v$ is contained in $X_{\alpha(e)}$. 
Now for every $i\in[t]$, let $X'_i=X_i\cup\{z_e^1:\alpha(e)=i\}$. Let $D'$ be the digraph obtained from $D$ by inverting $X'_1,\ldots,X'_t$. 
We will show that $D'$ is $k$-strong. Let $Y\subseteq V(D)$ with $|Y|\leq k-1$. We need to show that $D-Y$ is strongly connected. 
As $D\langle W\rangle$ is $k$-strong, we obtain that $W\setminus Y$ is contained in a single strongly connected component $S$ of $D'-Y$. Next observe that every $v \in V(G)\setminus Y$ has an in-neighbour and an out-neighbour in $S$, so $V(G)\setminus Y \subseteq S$. 
Now consider some $e=uv \in E(G)$. As $D'\langle Z_e\rangle=D\langle Z_e\rangle$ is $k$-strong, we obtain that $Z_e\setminus Y$ is contained in a single strongly connected component of $D'-Y$. 
If $Y\neq \{w_1,\ldots,w_{k-1}\}$, then there is at least one arc from $Z_e\setminus Y$ to $W \setminus Y$ and there is at least one arc from $W \setminus Y$ to $Z_e \setminus Y$, so $Z_e\setminus Y \subseteq S$. 
Finally, if $Y=\{w_1, \dots, w_{k-1}\}$, observe that, by construction, either $D'$ contains the arcs $uz_e^1$ and $z_e^1v$ or the arcs $vz_e^1$ and $z_e^1u$. 
Again, we obtain $Z_e\setminus Y \subseteq S$. This yields that $D'$ is $k$-strong. We hence have $\sinv_k(D)\leq t=\cc(G)$.


We now show that $\cc(G)\leq\sinv_k(D)$. Let $(X_1,\ldots, X_t)$ be a smallest collection of sets whose inversion makes $D$ $k$-strong. For $i=1,\ldots,t$, let $X_i'=X_i \cap V(G)$.
\begin{claim}\label{par2}
Let $e=uv \in E(G)$. Then there is some $i \in \{1,\ldots,t\}$ such that $X_i$ contains exactly one of $u$ and $v$.
\end{claim}
\begin{proofclaim}
Suppose otherwise, that is for every $i \in [t]$ either both $u$ and $v$ or none of $u$ and $v$ are in $X_i$.
Thus both the arcs corresponding to edges in $\partial_{D}(Z_e)\cap \partial_{D}(z_e^{1})$ are inverted or none of them. This yields that in $D'- \{w_1,\ldots,w_{k-1}\}$,  there are either zero or two arcs entering $Z_e$. Hence $D'-\{w_1,\ldots,w_{k-1}\}$ is not strongly connected, a contradiction.
\end{proofclaim}
By Claim \ref{par2}, we obtain that $(X_1',\ldots,X_t')$ is a cut cover for $G$. Thus $\cc(G)\leq t=\sinv_k(D)$, and we conclude that $\cc(G) = \sinv_k(D)$.
\end{proof}


Lemma~\ref{lem:reduc3} and the NP-hardness of deciding whether a given graph $G$ satisfies $\cc(G)\leq t$ for all $t\geq 2$ directly imply the following.
\begin{corollary}\label{rsedgzj}
Let $t$ be an integer greater than $1$ and $k$ a positive integer.
Deciding whether $\sinv_k(D) \leq t$  for a given oriented graph $D$ is NP-complete.
\end{corollary}
Clearly, Theorem \ref{ver1} and Corollary \ref{rsedgzj} imply Theorem \ref{archard1}.

\subsection{Unapproximability}\label{vier}
%%%%%%%%%%%%%%%%%%%%%%%%%%%%%%%%%%%%%%%%%%


In order to prove Theorems~\ref{approx2} and~\ref{approx1}, we need the following result of Zuckerman~\cite{v003a006}.
\begin{proposition}[\cite{v003a006}]\label{zuck}
Unless P=NP, there is no polynomial-time $\bigO(n^{1-\epsilon})$-approximation algorithm for computing $\chi(G)$ for any $\epsilon >0$.
\end{proposition}

As a consequence, we obtain a negative result concerning the approximation of the cut covering number.

\begin{proposition}\label{approcc}
Unless P=NP, there is no polynomial-time $(2-\epsilon)$-approximation algorithm for computing the cut covering number of a given graph for any $\epsilon >0$.
\end{proposition}
\begin{proof}
    Suppose that there exists a polynomial-time $(2-\epsilon)$-approximation algorithm $A$ for computing the cut covering number of a given graph for some $\epsilon >0$. Now choose $\epsilon'$ with $0<\epsilon'<\epsilon$ and consider a graph $G$ on $n\geq 4^{\frac{1}{\epsilon-\epsilon'}}$ vertices. Let $A'$ be the algorithm that takes $G$ as input and returns $2^{\alpha}$ where $\alpha$ is the output of $A$ applied to $G$. Clearly, as $A$ runs in polynomial time, so does $A'$. Further, by Proposition \ref{prop:cc-chi}, we have $2^{\alpha}\geq 2^{cc(G)}\geq 2^{\log(\chi(G))}=\chi(G)$. Finally, by Proposition \ref{prop:cc-chi}, we have 
    \begin{align*}
        2^{\alpha}&\leq 2^{(2-\epsilon)\lceil\log(\chi(G))\rceil} \\
        &\leq 2^{(2-\epsilon)(\log(\chi(G))+1)} \\
        &=2^{2-\epsilon}2^{\log(\chi(G))(1-\epsilon)}2^{\log(\chi(G))}\\
        &\leq 4\chi(G)^{1-\epsilon}\chi(G)\\
        &\leq 4 n^{1-\epsilon}\chi(G)\\
        & \leq n^{1-\epsilon'}\chi(G).
    \end{align*}
Hence $A'$ is a $n^{1-\epsilon'}$-approximation algorithm for the chromatic number, so the statement follows by Proposition~\ref{zuck}. \end{proof}

Now Proposition~\ref{approcc} and Lemma~\ref{lem:reduc2} imply Theorem~\ref{approx2}, 
and Proposition~\ref{approcc} and Lemma~\ref{lem:reduc3} imply Theorem~\ref{approx1}.
