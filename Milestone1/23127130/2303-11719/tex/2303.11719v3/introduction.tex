\section{Introduction}
%%%%%%%%%%%
%\FHO{Je prefererais de mettre nos resultats en forme de theoreme dans l'introduction et de juste les citer ailleurs. Qu'en pensez-vous?}
Notation not given below is consistent with \cite{bang2009}. For some positive integer $k$, we denote by $[k]$ the set $\{1,2, \dots, k\}$, and we use $\log$ to refer to the logarithm to the base 2. A {\bf digraph} may contain {\bf digons}, those are pairs of arcs in opposite direction between the same two vertices, but no parallel arcs or loops while an {\bf oriented graph} is a digraph without digons. A {\bf multigraph} may contain parallel edges, but no loops, while a multigraph with no parallel edges is called a {\bf graph}. When the graph or digraph is clear from the context, we use $n$ for its number of vertices.


A {\bf feedback arc set}  in a digraph is a set of arcs whose reversal results in an acyclic digraph. 
Considering an arbitrary ordering $(v_1, \dots , v_n)$ of the vertices of a digraph $D$ and a smallest set among the one containing the forward arcs (i.e. arcs $v_iv_j$ with $i<j$) and the one containing the backward arcs (i.e. arcs $v_iv_j$ with $i>j$), one gets that $D$ has a feedback arc set of size at most $|A(D)|/2$. 
In particular, every oriented graph of order $n$ has a feedback arc set of size at most $\frac{1}{2}\binom{n}{2}$. 
This bound is almost tight : de la Vega~\cite{dlV83} showed that there are oriented graphs with no feedback arc set of size less than  $\frac{1}{2}\binom{n}{2} - 1.73 n^{3/2}$.
Finding a minimum cardinality feedback arc set of a given digraph is an important and heavily studied algoritmic problem. It is one of the first problems shown to be NP-hard listed by Karp in~\cite{karp1972}. 
Furthermore, it is hard to approximate. For arbitrary digraphs, the best
known ratio is $\bigO(\log n \log \log n) $ due to Even et al.~\cite{EvenNSS95}. It was proven to be APX-hard by Kann~\cite{KannThesis}. This was later strengthened by Dinur and Safra who showed that, unless P is
equal to NP, it cannot be approximated within a factor of better than about $1.36$~\cite{DiSa05}.
For {\bf tournaments}, which are orientations of complete graphs, the problem was proven to be remain NP-complete, independently by Alon~\cite{alonSJDM20} and Charbit, Thomassé, and Yeo~\cite{charbitCPC16}. On the other hand, a polynomial-time approximation scheme was provided by Kenyon-Mathieu and Schudy ~\cite{HowSTOC07}, improving on an earlier  $3$-approximation algorithm by Ailon, Charikar, and Newman~\cite{ACN08}.% and a polynomial-time approximation scheme~\cite{HowSTOC07}.% that admits an efficient computation of a
%solution that is at most by a factor of $1 + \epsilon$ worse than the optimum, for any $\epsilon >0$.



\medskip

To make a digraph $D$ acyclic, one can use a different operation from arc reversal, called inversion.
The {\bf inversion} of a set $X$ of vertices consists in reversing the direction of all arcs of $D\langle X\rangle$.
We say that we {\bf invert} $X$ in $D$. The resulting digraph is denoted by $\Inv(D;X)$.  
If $(X_i)_{i\in I}$  is a family of subsets of $V(D)$, then $\Inv(D; (X_i)_{i\in I})$ is the digraph obtained after inverting the
$X_i$ one after another. Observe that this is independent of the order in which we invert the $X_i$~: $\Inv(D; (X_i)_{i\in I})$ is obtained from $D$ by reversing exactly those arcs for which an odd number of the $X_i$ contain its two endvertices.
A {\bf decycling family} of a digraph $D$ is a family $(X_i)_{i\in I}$ of subsets of $V(D)$ such that
 $\Inv(D; (X_i)_{i\in I})$ is acyclic.
A digraph admits a decycling family if and only if it does not contain any digon. %, that is, a pair of arcs in opposite direction between the same two vertices.
Indeed, observe that an inversion operation changes the orientation of either none or both of the arcs in a digon. Hence, a digraph containing a digon cannot be made acyclic by inversions.
Conversely, in an oriented graph, the pairs of endvertices of the arcs of a minimal feedback arc-set form a decycling family.
The {\bf inversion number} of an oriented graph $D$, denoted by $\inv(D)$, is the minimum number of inversions needed to transform $D$ into an acyclic oriented graph, that is, the minimum cardinality of a decycling family.
This parameter was first introduced by Belkhechine et al. in \cite{BBBP10} and then studied in several papers~\cite{BCH,PST,inversion,APSSW}.
In particular, Belkhechine et al. proved in \cite{BBBP10}  that, for any fixed integer $k$, deciding whether a  given tournament has inversion number at most $k$ is polynomial-time solvable. In contrast, Bang-Jensen et al.~\cite{BCH} proved that deciding whether a given digraph has inversion number~$1$ is NP-complete. This was generalized by Alon et al.~\cite{APSSW}:  for any fixed positive integer $k$, deciding whether a  given digraph has inversion number at most $k$ is NP-complete.
Further, the extremal question of determining the maximum integer $\inv(n)$ of the inversion numbers of all oriented graphs on $n$ vertices has been investigated.
Independently, Aubian et al.~\cite{inversion} and Alon et al.~\cite{APSSW} proved
$n - 2\sqrt{n\log n} \leq \inv(n) \leq n - \lceil \log (n+1) \rceil$.

\medskip
The main purpose of this article is to study the possibilities of applying the inversion operation to obtain a different objective than the resulting digraph being acyclic. Instead of making a digraph acyclic, we are interested in making it satisfy a prescribed connectivity property.
A digraph $D$ is {\bf strongly connected} or simply {\bf strong} (resp. {\bf $k$-arc-strong} for some positive integer $k$), if for any partition $(V_1, V_2)$ of $V(D)$ with $V_1, V_2\neq \emptyset$ there is an arc (resp. at least $k$ arcs) with tail in $V_1$ and head in $V_2$.
Similarly, a multigraph $G$ is {\bf connected} (resp. {\bf $k$-edge-connected} for some positive integer $k$), if for any partition $(V_1, V_2)$ of $V(G)$ with $V_1, V_2\neq \emptyset$ there is an edge (resp. at least $k$ edges) with one endvertex in $V_1$ and one endvertex in $V_2$. We further say that $G$ is {\bf $k$-connected} if $|V(G)|\geq k+1$ and $G-S$ is connected for every $S \subseteq V(G)$ with $|S|\leq k-1$.
For a given digraph $D$, we denote by $\UG(D)$ the undirected multigraph that we
obtain by suppressing the orientations of the arcs.  A digraph is {\bf $k$-connected} (resp. {\bf $k$-edge-connected}) if its underlying multigraph $\UG(D)$ is.
Clearly, a digraph $D$ can be made $k$-arc-strong by reversing some arcs if and only if the edges of $\UG(D)$ can be oriented such that the resulting digraph
is $k$-arc-strong.
Robbins' Theorem~\cite{Rob1939} asserts that a graph admits a strong orientation if and only if it is 2-edge-connected, and more generally, Nash–Williams’ weak orientation theorem~\cite{NashW60} asserts that a graph admits a $k$-arc-strong orientation 
if and only if it is $2k$-edge-connected.
%\begin{theorem}[Nash-Williams~\cite{NashW60}]
%A graph admits a $k$-arc-strong orientation if and only if it is $2k$-edge-connected.
%\end{theorem}

We can hence decide in polynomial time whether a given digraph can be transformed into a $k$-arc-strong digraph via arc reversals, applying standard flow algorithms to its underlying graph. Furthermore, it is well-known that, if this is the case, then, by reducing to a minimum-cost submodular flow problem, one can determine, in polynomial time, a minimum set of arcs in $D$ whose reversal gives a $k$-arc-strong digraph, see Section 8.8.4 of~\cite{bang2009} for details.
%For an arbitrary digraph $D$ the size of minimum set of arcs in $D$ whose reversal gives a $k$-arc-strong digraph may depend on $n$, the number
%of vertices of $D$ (for example for a digraph having a linear number of sinks, vertices with in-degree $0$).
It is easy to see that the number of necessary arc reversals to make a $2k$-edge-connected digraph $D$ $k$-arc-strong cannot be bounded by a function depending only on $k$. For example, one can consider digraphs that contain a number of sinks which is linear in the number of vertices of the graph, where a {\bf sink} is a vertex with no outgoing arc.
However, Bang-Jensen and Yeo~\cite{BaYe04} proved that for tournaments, the size of such a set is always bounded by a quadratic function of $k$. Precisely, they showed that every tournament on at least $2k+1$ vertices can be made $k$-arc-strong by reversing at most $\frac{1}{2}k(k+1)$ arcs. This result is tight for the transitive tournaments.

We are interested in the problem of using inversions to make a digraph $k$-arc-strong.
A {\bf $k$-arc-strengthening family} of a digraph $D$ is a family $(X_i)_{i\in I}$ of subsets of $V(D)$ such that
 $\Inv(D; (X_i)_{i\in I})$ is $k$-arc-strong.
The {\bf $k$-arc-strong inversion number} of a digraph $D$, denoted by $\sinv'_k(D)$, is the minimum number of inversions needed to transform $D$ into a $k$-arc-strong digraph, that is, the minimum cardinality of a $k$-strengthening family.
We first deal with the extremal behaviour of $\sinv'_k(D)$ for some fixed $k$, that is, we deal with the question of finding the maximum number of necessary inversions to make a $2k$-edge-connected digraph on a fixed number of vertices $k$-arc-strong. To this end, we define $\sinv'_k(n) = \max\{\sinv'_k(D) \mid D~$\mbox{is a $2k$-edge-connected digraph of order $n$}$\}$. It turns out that $\sinv_k'(n)$ is an unbounded, but slowly growing, function. We are able to determine $\sinv_k'(n)$ up to a constant multiplicative factor of roughly 2. More precisely, we show the following result:

%$$ \frac{1}{k} \log n - \log k \leq \sinv'_k(n) \leq \log n + 4k -3.$$
%\CR{Nouvelle borne (juste en collant une clique subdivisee a une clique bidirigee.}


\begin{restatable}{theorem}{extrem}\label{thm:extrem}
For any pair of positive integers $k,n$ with $n \geq k+2$, we have 
\begin{equation*}
\frac{1}{2} \log (n - k+1) \leq \sinv'_k(n) \leq \log n + 4k -3.
\end{equation*}
\end{restatable}


Observe that the condition $n \geq k+2$ is necessary for $\sinv_k'(n)$ to be well-defined.
\medskip


 Next, we consider the algorithmic complexity of computing $\sinv_k'(D)$ for a given digraph $D$ and a fixed integer $k$. We show that this problem is NP-hard. More precisely, we show the following, slightly stronger, result. 

%\FHO{After, we deal with the problem of computing $\sinv_k'(D)$ algorithmically for a given graph $D$ and a fixed integer $k$. We show that there is little hope to decide this problem in polynomial time even when fixing $k$ and the number of required inversions. More precisely, we show the following result.}

\begin{restatable}{theorem}{archard}\label{archard1}
Deciding whether a given oriented graph $D$ satisfies $\sinv'_k(D) \leq t$ is NP-complete for all pairs of positive integers $k$ and $t$.
\end{restatable}
%we prove that, for any fixed postive integers $k$ and $t$, deciding whether a given oriented graph $\vec{G}$ satisfies $\sinv'_k(\vec{G}) \leq t$ is NP-complete.

Furthermore, we show that there is little hope to approximate $\sinv'_k(D)$ within a factor better than 2.

\begin{restatable}{theorem}{approxarc}\label{approx2}
Unless P=NP, for any positive integer $k$, there is no polynomial time $(2-\epsilon)$-approximation algorithm for computing $\sinv'_k(D)$ for oriented graphs for any $\epsilon >0$.
\end{restatable}

\medskip



As a related problem, one may also want to make a digraph $k$-strong.
 A digraph $D$ is {\bf $k$-strong} if $|V(D)|\geq k+1$ and for any set $S \subseteq V(D)$ with fewer than $k$ vertices, $D-S$ is strong.
A digraph which can be made $k$-strong by reversing arcs
 is {\bf $k$-strengthenable}.
The $1$-strengthenable digraphs are the $2$-edge-connected ones, because
being $1$-strong is the same as being strong or $1$-arc-strong. 
Thomassen~\cite{Thomassen2015} proved that the $2$-strengthenable digraphs are the $4$-edge-connected digraphs $D$ such that $D-v$ is $2$-edge-connected for every vertex $v \in V(D)$. On the other hand, it is NP-hard to compute the minimum number of arc reversals needed to make a given digraph 2-strong, as shown by Bang-Jensen et al. in ~\cite{BJHK}.
Furthermore, in contrast to the anologous problem for $k$-arc-strengthenable digraphs, for $k\geq 3$, it is NP-complete to decide whether a digraph is $k$-strengthenable. Indeed, Durand de Gevigney~\cite{Dur20} proved that it is NP-complete to decide whether an undirected graph has a $k$-strong orientation for any $k \geq 3$.

It is also natural to use inversions to make a digraph $k$-strong. 
A {\bf $k$-strengthening family} of a digraph $D$ is a family $(X_i)_{i\in I}$ of subsets of $V(D)$ such that
 $\Inv(D; (X_i)_{i\in I})$ is $k$-strong.
The {\bf $k$-strong inversion number} of a $k$-strengthenable digraph $D$, denoted by $\sinv_k(D)$, is the minimum number of inversions needed to transform $D$ into a $k$-strong digraph, that is, the minimum cardinality of a $k$-strengthening family. 
It follows directly from the fact that every $k$-strong digraph is $k$-arc-strong that $\sinv_k(D)\geq \sinv'_k(D)$ holds for every $k$-strengthenable digraph $D$.
In the light of the complexity result of Durand de Gevigney~\cite{Dur20}, it seems difficult to obtain an extremal result in the shape of Theorem~\ref{thm:extrem} for $k$-strong digraphs. However, we give the following complexity results which are the natural analogues of Theorems~\ref{archard1} and~\ref{approx2}.

\begin{restatable}{theorem}{verhard}\label{verhard}
Deciding whether a given $k$-strengthenable oriented graph $D$ satisfies $\sinv_k(D) \leq t$ is NP-complete for all pairs of positive integers $k$ and $t$.
\end{restatable}
\begin{restatable}{theorem}{approxver}\label{approx1}
Unless P=NP, for any positive integer $k$, there is no polynomial time $(2-\epsilon)$-approximation algorithm for computing $\sinv_k(D)$ for $k$-stengthenable oriented graphs for any $\epsilon >0$.
\end{restatable}

%We believe that it is possible to strengthen Theorems~\ref{approx2} and~\ref{approx1}.
%\begin{conjecture}
 %   Unless P=NP, for any positive integer $k$ and any constant $\alpha$, there is no $\alpha$-approximation algorithm for computing $\sinv_k(D)$ or $\sinv_k'(D)$ for simple oriented graphs.
%\end{conjecture}
%In Section~\ref{sec:complexity}, we show that for any positive integers $k$ and $t$, it is NP-complete to decide whether $\sinv_k(D)\leq t$ for a given $k$-strengthenable oriented graph. We also show that, unless P=NP,  $\sinv_k$ cannot be approximated within a factor better than 2.
\medskip

In the remainder of this article, we focus on a particular kind of oriented graphs, namely tournaments. %A {\bf tournament} is an orientation of a complete graph.
No tournament on fewer than $2k+1$ vertices is $k$-arc-strong, and thus no tournament on fewer than $2k+1$ vertices is $k$-arc-strengthenable. On the other hand, it is not hard to show that every tournament of order at least $2k+1$ is $k$-strengthenable. Moreover, every tournament on at least $2k+1$ vertices can be made $k$-strong by reversing the orientation of at most $\frac{1}{4}(4k-2)(4k-3)$ arcs, see e.g. \cite{bang2009}, p. 379.


Further, the following improvement was conjectured by Bang-Jensen. (See \cite{bang2009}.)
\begin{conjecture}[Bang-Jensen, 1994]
Every tournament on at least $2k+1$ vertices can be made $k$-strong by reversing at most $\frac{1}{2}k(k+1)$ arcs.
\end{conjecture}
This would be tight as shown by the transitive tournaments. 
Bang-Jensen, Johansen, and Yeo~\cite{BKY2020} proved this conjecture for tournaments of order at least $3k-1$.

It is then natural to ask whether or not we can make a tournament $k$-strong or $k$-arc-strong using significantly fewer than $\frac{1}{2}k(k+1)$ inversions. This leads us to consider $M_k= \max\{\sinv_k(T) \mid T~\mbox{tournament of order at least $2k+1$}\}$ and $M'_k= \max\{\sinv'_k(T) \mid T~\mbox{tournament of order at least $2k+1$}\}$.
We show that these numbers are indeed significantly smaller than $\frac{1}{2}k(k+1)$, but cannot be bounded by a constant independent of $k$. More precisely, we show the following result.
\begin{theorem}\label{thm:exttournoi}
    For every sufficiently large integer $k$, we have
$\frac{1}{2} \log(2k+1) \leq M'_k \leq M_k \leq 2k.$
\end{theorem}
The lower bound is obtained by a counting argument.%considering a random tournament of order $2k+1$.

We also prove that $M_1=M'_1=1$ and $M_2=M'_2=2$ showing that the upper bound is not tight for $k=1,2$. 
We also prove a better upper for $M'_k$ for large values of $k$.

\begin{restatable}{theorem}{borneMprime}\label{thm:upperM'}
$M'_k \leq \frac{4}{3}k+o(k)$. 
\end{restatable}

However, we believe that this bound, as well as the others, is not tight.
\begin{problem}\label{rsedth}
Find better bounds on $M_k$ and $M'_k$.
\end{problem}


We further study the question of how the parameters $\sinv_k$ and $\sinv'_k$  behave when fixing $k$ and considering tournaments whose size is significantly larger than $2k+1$.  As a first result, we  prove that every sufficiently large tournament can be made $k$-strong (and thus also $k$-arc-strong) in one inversion. 



\begin{restatable}{theorem}{sone}\label{nk1}
Let $n$ and $k$ be positive integers with $n \geq 19k-2$. If $T$ is a tournament on $n$ vertices, then $\sinv'_k(T) \leq \sinv_k(T) \leq 1$.
\end{restatable}
In fact, it is
relatively straightforward to see that $\sinv_k(T) \leq 1$ for tournaments $T$ of even larger
order relative to $k$ (see Theorem~\ref{thm:s1}). 


This leads us to the study of the functions
$m_k(n)= \max \{\sinv_k(T) \mid T~\mbox{tournament of order}~n\}$ and $m'_k(n)= \max \{\sinv'_k(T) \mid T~\mbox{tournament of order}~n\}$ for all $n\geq 2k+1$. We believe that $m_k$ and $m_k'$ have the following monotonic behaviour. 

\begin{restatable}{conjecture}{theconj}\label{conjM}
\begin{enumerate}
\item[] 
  \item[(i)] $m_k$ and $m'_k$ are non-increasing mappings.
  \item[(ii)] $M_k =  m_k(2k+1)$ and $M'_k =m'_k(2k+1)$.
\end{enumerate}
\end{restatable}


Note  that (i) implies (ii).
This conjecture is motivated by the fact that one can easily prove that $m_k$ and $m'_k$ are non-increasing for $n>4k-2$. See Section~\ref{conclusion}.
Therefore, in order to approach Problem~\ref{rsedth}, % to get bounds on $M_k$ and $M'_k$,
it is sufficient to consider tournaments whose order is in the range from $2k+1$ to $4k-2$.

\medskip

%The fact that $m_k(n) =1$ when $n$ is sufficiently large in comparison to $k$
Theorem~\ref{nk1} implies that, for every pair of positive integers $k$ and $i$, there is a smallest integer $N_k(i)$ such that
$m_k(n) \leq i$ for all $n\geq N_k(i)$.
Similarly, for every pair of positive integers $k$ and $i$, there is a smallest integer $N'_k(i)$ such that
$m'_k(n) \leq i$ for all $n\geq N'_k(i)$.
Since $m'_k(n) \leq m_k(n)$ for all $k$ and $n$, we have $N'_k(i) \leq N_k(i)$ for all $k$ and $i$.
It is natural to ask the following questions.
\begin{problem}\label{probnk}
What is the minimum integer $N_k(i)$ such that $\sinv_k(T) \leq i$? % for every tournament $T$ of order at least $N_k(i)$?

What is the minimum integer $N'_k(i)$ such that $\sinv'_k(T) \leq i$? % for every tournament $T$ of order at least $N'_k(i)$?
\end{problem}

Theorem~\ref{nk1} states $N'_k(1)\leq N_k(1)\leq 19k-2$.
Since the transitive tournament on $3k-1$ vertices satisfies $\sinv'_k(TT_{3k+1})=\sinv_k(TT_{3k+1})=2$ (Theorem~\ref{thm:TTn}), we have $ N_k(1) \geq N'_k(1) \geq 3k$. We further establish the following better lower bounds for $N_k(1)$ and $N_k'(1)$ by giving simple examples.
%Concerning lower bounds for $N_k(1)$, we have the following result.
\begin{restatable}{proposition}{nkunter}\label{nk1unter}
    For any positive integer $k$, we have $N_k(1)\geq 5k-2$ and $N'_k(1)\geq 4k-2$.
\end{restatable}
We further show that a significantly smaller tournament can still be dealt by three inversions.



\begin{restatable}{theorem}{nkthree}\label{nk3}
    For any positive integer $k$, we have $N'_k(3)\leq N_k(3)\leq 11k-2$.
\end{restatable}

%We prove $5k-2\leq N'_k(1)\leq N_k(1)\leq 94k -40$ (Proposition~\ref{prop:N1-inf} and Theorem~\ref{thm:N1-linear}) and $N_k(6)\leq 14k -3$ (Theorem~\ref{thm:s6}).
Finally, using probabilistic methods, we prove that every tournament that is a constant factor bigger than $2k$ can be made $k$-strong by a constant number of inversions. More precisely, we prove the following result.
\begin{restatable}{theorem}{pluseps}\label{thm:2+eps}
    There exists a function $f\colon \mathbb{R}_{>0} \to \mathbb{N}$ such that for every $\epsilon>0$ and every positive integer $k$, 
    if $T$ is an $n$-vertex tournament with $n \geq 2k+1 +\epsilon k$, then $\sinv_k(T) \leq f(\epsilon)$.
\end{restatable}


%In addition we show in Theorem~\ref{thm:2+eps} the existence a function $f\colon \mathbb{R}_{>0} \to \mathbb{N}$ such that, for every $\epsilon>0$ and every positive integer $k$, $\sinv_k(T) \leq f(\epsilon)$ for all
 %tournament $T$ on at least $(2+\epsilon)k$ vertices.

\medskip

The fact that $m_k(n)=1$ for $n$ sufficiently large implies that the set ${\cal F}_k$ of tournaments $T$ such that $\sinv_k(T) >1$ is finite.
This implies that computing $\sinv_k$ and $\sinv'_k$ for fixed $k$ can be done in polynomial time for tournaments.

\begin{corollary}\label{cor:sinv_k-poly}
Let $k$ be a positive integer. We can compute $\sinv_k(T)$ (resp. $\sinv'_k(T)$) for a given tournament $T$ on $n$ vertices in $O(n^{7/2})$ time (resp. $O(n^2)$ time).
\end{corollary}
\begin{proof}
We first check in constant time, whether $T$ is in ${\cal F}_k$. If yes, then we can return $\sinv_k(T)$ and $\sinv'_k(T)$ that may be stored in some precomputed table.
If not, then $\sinv'_k(T) \leq \sinv_k(T) \leq 1$. We then check in $O(n^{7/2})$ time using algorithms due to Even and Tarjan~\cite{EvenTarjan75} or Galil~\cite{Galil80} based on flow (resp. $O(n^2)$ time using Mansour and  Schieber's~\cite{MANSOUR198976} algorithm based on flow or Gabow's algorithm~\cite{Gabow91} based on matroids) whether $T$ is $k$-strong (resp. $k$-arc-strong) to determine whether $\sinv_k(T)$ (resp. $\sinv'_k(T)$) is equal to $0$ or $1$. %\FHO{Les deuxpremiers references ne sont pas claires pour moi. Je ne trouve pas de $O(n^{\frac{5}{2}})$ dedans. Les deux derniers me semblent correctes, j'ai ajoute la reference.}
\end{proof}


%\begin{problem}
%What is the complexity of computing $\sinv_k(T)$ (resp. $\sinv'_k(T)$) for a given tournament $T$ if $k$ is part of the input?
%\end{problem}

This article is structured as follows. We give some notation and a collection of preliminary results in Section~\ref{prel}. In Section~\ref{sec:oriented}, we prove Theorem~\ref{thm:extrem}. In Section~\ref{sec:complexity}, we prove the complexity results, namely Theorems~\ref{archard1} to~\ref{approx1}. The results on tournaments are contained in Section~\ref{sec:M}, in which we prove Theorem~\ref{thm:exttournoi}, and in Section~\ref{sec:upper_bound_Mkn}, in which we prove Theorems~\ref{thm:s1},~\ref{nk1},~\ref{nk3} and~\ref{thm:2+eps} and Proposition~\ref{nk1unter}.

\section{Preliminaries}\label{prel}
%%%%%%%%%%%%%%%%%%%%%%%%%%%%%%%%%%%%%

In Section~\ref{not}, we introduce some formal notation and in Section~\ref{prelres}, we give a collection of auxiliary results.



\subsection{Notation}\label{not}
%%%%%%%%%%%%%%%%%%%%%%%%%%%%%%%

 A {\bf mixed graph} $G = (V, E, A)$ is a triple consisting of a set $V$ of elements called {\bf vertices}, a set $E$ of unordered pairs of vertices  called {\bf edges}, and a set $A$ of ordered pairs of vertices called {\bf arcs}.
Hence a multigraph can be seen as a mixed graph whose arc set is empty, while a digraph can be seen as a mixed graph whose edge set is empty and that does not contain parallel arcs.
Given a mixed graph $G$, its vertex set is denoted by $V(G)$, its edge set is denoted by $E(G)$ and its arc set by $A(G)$.
The {\bf underlying graph} $\UG(G)$ of a mixed graph $G$ is the undirected multigraph that we obtain by suppressing the orientations of the arcs (i.e. replacing each arc by an edge between its two endvertices). 
An {\bf orientation} of a mixed graph $G$ is a digraph obtained by giving an orientation to every edge, i.e.
replacing each edge by one of the two arcs between its endvertices.



Let $S,S'$ be disjoint sets of vertices in a mixed graph $G$. 
We denote by $\delta_G(S,S')$ the set of edges of $E(G)$ with exactly one endvertex in $S$ and one endvertex in $S'$. We use $\delta_G(S)$ for $\delta_G(S,V(G)\setminus S)$.
We also denote by $\delta^+_G(S,S')$ (resp. $\delta^-_G(S,S')$) the set of arcs in $A(G)$ with tail (resp. head) in $S$ and head (resp. tail) in $S'$. Again, we use $\delta^+_G(S)$ (resp. $\delta^-_G(S)$) for $\delta^+_G(S,V(G)\setminus S)$ (resp. $\delta^-_G(S,V(G)\setminus S)$).
The {\bf degree} (resp. {\bf out-degree}, {\bf in-degree}) of $S$ in $G$ is $d_G(S) = |\delta_G(S)|$ 
%\CR{not $|N_g(S) \setminus S|$? By the way we should define $N_G(S)$. Or maybe it should be $|\delta(S)|$, which is more consistent with Prop 2.4.} 
(resp.
$d^+_G(S) = |\delta^+_G(S)|$, $d^-_G(S) = |\delta^-_G(S)|$). We further use $d_G(S,S') = |\delta_G(S,S')|$, 
$d^+_G(S,S') = |\delta^+_G(S,S')|$, and 
$d^-_G(S,S') = |\delta^-_G(S,S')|$.
Let {$N_G(S)$ (resp. $N_G^+(S)$, $N_G^-(S)$)} denote the set of vertices in $V(G)\setminus S$ that are incident to at least one edge (arc) in $\delta_G(S)$ (resp. $\delta_G^+(S)$, $\delta_G^-(S)$).
The {\bf degree} (resp. {\bf out-degree}, {\bf in-degree}) of a vertex $v$ in $G$ is $d_G(v) = d_G(\{v\})$ (resp. 
$d^+_G(v) = d^+_G(\{v\})$, $d^-_G(v) = d^-_G(\{v\})$). Furthermore, for two disjoint sets $S_1,S_2$, let $d_G(S_1,S_2)$ be the number of edges in $E$ with one endvertex in $S_1$ and one endvertex in $S_2$, let $d_G^+(S_1,S_2)$ be the number of arcs in $A$ whose tail is in $S_1$ and whose head is in $S_2$, and let $d_{G}^-(S_1,S_2)$ be $d_G^+(S_2,S_1)$.
For simplicity, when the mixed graph is clear from the context, the subscript $G$ is omitted.

\medskip


Let $D$ be a digraph.
A {\bf sink} (resp. {\bf source}) in a digraph is a vertex with out-degree $0$ (resp. in-degree $0$).
We say that $D$ is {\bf eulerian} if $d_D^-(v) = d_D^+(v)$ for all vertices $v$.
Let $A$ and $B$ be two sets in $D$.
If $ab$ is an arc for all pairs $(a,b)$ in $A\times B$, then we write $A\Ra B$.

Let $u,v$ be two distinct vertices in $D$. The {\bf strong-connectivity} from $u$ to $v$ in $D$, denoted by $\kappa_D(u,v)$, is the maximal number $\alpha$ such that $D-X$ contains a $(u,v)$-path  for every $X \subseteq V(D)\setminus \{u,v\}$ with $|X|\leq \alpha-1$. For some $S \subseteq V(D)$ and positive integer $k$, we say that $S$ is {\bf $k$-strong in $D$} if $\kappa_D(u,v)\geq k$ for all $u,v \in S$.
For some $S \subseteq V(D)$ and positive integer $k$, we say that $S$ is {\bf $k$-strong in $D$} if for all distinct $u,v \in S$ and every $X \subseteq V(D)\setminus X$ with $|X|\leq k-1$, we have that $D-X$ contains a $uv$-path.

\subsection{Preliminary results}\label{prelres}
We use several times without explicitly mentioning it that for every positive integer $k$, there is a $k$-strong tournament on $2k+1$ vertices. One example for such a tournament is the socalled rotative tournament whose vertex set is $[2k+1]$ and where an arc is oriented from $i$ to $j$ if $i-j\in [k] ({\rm mod }~2k+1)$.

\medskip

We need the following simple result that allows us to extend a set which is $k$-strong in a digraph.
\begin{proposition}\label{lem:kstrong+}
Let $D$ be a digraph, let $S$ be a $k$-strong set in $D$ and let $v \in V(D)\setminus S$. If $v$ has at least $k$ in-neighbours in $S$ and at least $k$ out-neighbours in $S$, then $S \cup \{v\}$ is $k$-strong in $D$.
\end{proposition}
\begin{proof}
    Suppose for the sake of a contradiction that there is a set $X\subseteq V(D)$ of size at most $k-1$ and a pair $(s,t)$ of vertices in $(S \cup \{v\}) \setminus X$ such that there is no $(s,t)$-path in $D-X$.
    If $s,t \in S$, then this contradicts the fact that $S$ is $k$-strong in $D$.
    Thus exactly one of $s$ and $t$ is $v$. Without loss of generality, suppose $s \in S$ and $t=v$.
    Since $|N^-(v) \cap S| \geq k$, $v$ has an in-neighbour $t_0 \in S-X$, and as $S$ is $k$-strong in $D$, there is an $(s,t_0)$-path in $D-X$, and so there is an $(s,t)$-path in $D-X$, a contradiction.
\end{proof}
The well-known following result is helpful or applying Proposition~\ref{lem:kstrong+}. It follows directly from Proposition~2.2.2 of \cite{Tour-book}.
\begin{proposition}\label{4k-2}
    Let $T$ be a tournament on at least $4k-1$ vertices for some positive integer $k$. Then there exists some $v \in V(T)$ with $\min\{d_T^+(v),d_T^-(v)\}\geq k$.
\end{proposition}



%\CR{
%For sake of completeness, we include a short proof of this fact.
%\begin{proof}
%    Suppose that $T$ is an $n$-vertex tournament such that every vertex in $T$ has at most $k-1$ out-neighbours or at most $k-1$ in-neighbours.
%    Let $A$ be the set of vertices $v \in V(T)$ such that $d_T^-(v) \leq k-1$ and let $B = V(T) \setminus A$.
%    Then a counting of the arcs of $T$ yields 
%    \[
%        \binom{n}{2} \leq |A||B| + \sum_{v \in A} d_T^-(v) + \sum_{v \in B} d_T^+(v)
%    \]
%    and it follows that
%    \begin{align*}
%        \binom{n}{2} 
%        &\leq  |A||B| + |A| (k-1) + |B| (k-1) \\
 %       &\leq \left(\frac{n}{2}\right)^2 + n(k-1).
%    \end{align*}
%    We deduce that $n - 1 \leq \frac{n}{2} + 2(k-1)$ and it follows that $n \leq 4k - 2$.
%\end{proof}
%}

%A graph is {\bf eulerian} if every vertex has even degree.
%A digraph is {\bf eulerian} if $d^+(v)=d^-(v)$ for every vertex $v$.
We need the following simple characterization of $k$-arc-strong tournaments of order $2k+1$.
\begin{proposition}[Folklore]\label{eul}
A tournament on $2k+1$ vertices is $k$-arc-strong if and only if it is eulerian.
\end{proposition}


For sake of completeness, we give a short proof of this fact.

\begin{proof}
    Let $T$ be a tournament on $2k+1$ vertices.
    If $T$ is $k$-arc-strong, then every vertex $u$ in $T$ has in- and out-degree at least $k$, but since $d_T^+(u)+d_T^-(u) =2k$, we deduce that $d_T^+(u)=d_T^-(u)$ and so $T$ is eulerian.
    Reciprocally, if $T$ is eulerian, then $d_T^+(u)=d_T^-(u)=k$ for every vertex $u$ of $T$.
    Now consider a partition $(V_1,V_2)$ of $V(T)$ where $V_1$ and $V_2$ are non-empty.
    Then the number of arcs from $V_1$ to $V_2$ is at least 
    \begin{align*}
        \sum_{v \in V_1} d_T^+(v) - \binom{|V_1|}{2} 
        &\geq |V_1| \left( k - \frac{|V_1|-1}{2} \right) \\
        &\geq k
    \end{align*}
    and so $T$ is $k$-arc-strong.
\end{proof}



We need the following orientation property of mixed graphs that can be found in \cite{FF}.
\begin{proposition}\label{prop:eul-or}
    Let $H$ be a mixed graph whose underlying graph is eulerian. Then $H$ has an eulerian orientation if and only if $d_{H}(S)\geq d_{H}^+(S)-d_{H}^-(S)$ for all $S \subseteq V(H)$.
\end{proposition}



Finally, we state two basic tools from probability theory.
\begin{proposition}[Union Bound]\label{union}
    Let $E_1,\ldots,E_\ell$ be a set of events in a random experiment and $E$ the event that at least one of $E_1,\ldots,E_\ell$ occurs. Then $\Pr(E)\leq \sum_{i=1}^\ell \Pr(E_i)$.
\end{proposition}


\begin{proposition}[Chernoff's Bound]\label{chernoff}
    If $X$ is a random variable following a binomial law with parameters $p \in [0,1]$ and $n \geq 0$, then
    for every $\epsilon \in [0,1]$
    %\[
   % \Pr[X \geq (1+\epsilon)pn] \leq \exp \left(-\frac{\epsilon^2 }{3} pn\right).
    %\]
    %and
    \[
    \Pr[X \leq (1-\epsilon)pn] \leq \exp \left(-\frac{\epsilon^2}{2} pn\right).
    \]
\end{proposition}

We refer the reader to~\cite[Part II, Section 5]{molloy2002graph} for an introduction to the probabilistic method, including proofs of Propositions~\ref{union} and~\ref{chernoff}.

