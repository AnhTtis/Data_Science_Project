\section{ Bounds on \texorpdfstring{$\sinv'_k(n)$}{sinv'k(n)}}\label{sec:oriented}
%%%%%%%%%%%%%%%%%%
This section is dedicated to the extremal results we have on the $k$-arc-strong inversion number. In particular, we prove Theorem~\ref{thm:extrem}, which we restate.

\extrem*


Let $d$ be a positive integer.
A multigraph $G$ is said to be {\bf $d$-degenerate} if every submultigraph of $G$ has a vertex of degree at most $d$.
Every $d$-degenerate graph admits a {\bf $d$-degenerate ordering}, that is an ordering $(v_1, \dots, v_n)$ of the vertices of $G$ such that every vertex has at most $d$ neighbours with lower indices.


We shall need the following proposition, which, as observed in \cite{BJHK}, is a direct consequence of Corollary~2 in~\cite{zoltans06}.
\begin{proposition}\label{prop:nash_williams_more_precise}
Let $G$ be a multigraph that has a $k$-arc-connected orientation for some positive integer $k$
and let $(e_1, f_1),\dots ,(e_t, f_t)$ be a collection of pairwise disjoint pairs of parallel edges in G. Then G has
a $k$-arc-connected orientation in which $e_i$ and $f_i$ are oriented in opposite directions for $i = 1,\dots , t$.
\end{proposition}

We use the following result which has recently been proved in a related paper by the second, third, and fourth author \cite{havet2024diameter}.
%The following result is part of a draft of a forthcoming paper by the second, third, and fourth author.
\begin{theorem}[\cite{diameter}]\label{theorem:bound_diam_L_degenerate}
Let $G$ be an $n$-vertex $d$-degenerate graph.
For any two orientations $\vec{G}_1,\vec{G}_2$ of $G$, one can transform $\vec{G}_1$ into $\vec{G}_2$ by inverting at most $\log n + 2d-1$ sets.
\end{theorem}


In a multigraph $G$, for a pair $(u,v)$ of vertices, a {\bf $(u,v)$-cut} is 
a set $F$ of edges such that every $(u,v)$-path in $G$ intersects $F$. A {\bf cut} is a $(u,v)$-cut for some distinct $u,v$.
Note that a multigraph is $k$-edge-connected if and only if for every pair $u,v$ of vertices, there is no $(u,v)$-cut of size at most $k-1$ in $G$.

A multigraph $G$ is {\bf minimally $k$-edge-connected} if it is $k$-edge-connected and 
$G\setminus e$ is not $k$-edge-connected 
for any edge $e \in E(G)$.
Equivalently, every edge is in a cut of size exactly $k$.

The next result gives an upper bound on the number of edges of a multigraph each edge of which is contained in a small cut.
\begin{lemma}\label{lemma:min_k_connected_mad}
    Let $n$ and $k$ be positive integers with $n \geq k+1$, and let $G$ be an $n$-vertex multigraph.
    If for every edge $uv \in E(G)$ there is a $(u,v)$-cut of size at most $k$ in $G$, then $G$ has at most $k(n-1)$ edges.
\end{lemma}
\begin{proof}
    Let $F_1,\ldots,F_k$ be a set of $k$ edge-disjoint spanning forests in $G$ such that $\sum_{i=1}^k|E(F_i)|$ is maximized. If there is an edge $e=uv \in E(G)\setminus \bigcup_{i=1}^kE(F_i)$, then, as $e$ cannot be added to $E(F_i)$, we obtain that $F_i$ contains a $(u,v)$-path for $i\in [k]$. Hence $G\setminus e$ does not contain any $(u,v)$-cut whose size is smaller than $k$, a contradiction to the fact $e$ is contained in a cut of size at most $k$. This yields $|E(G)|=\sum_{i=1}^k|E(F_i)|\leq k(n-1)$.
\end{proof}

We are now ready to prove the upper bound in Theorem~\ref{thm:extrem}.
\begin{lemma}\label{extremsup}Let $n$ and $k$ be positive integers with $n \geq k+1$. Then,
    for every $2k$-edge-connected $n$-vertex digraph $D$, $\sinv'_k(D) \leq \log n + 4k-3$.
\end{lemma}

\begin{proof}
    Without loss of generality, suppose that $D$ is minimally $2k$-edge-connected.
    Then by Lemma~\ref{lemma:min_k_connected_mad}, every subgraph of $\UG(D)$ has average degree smaller than $2k$.
    This implies that $\UG(D)$ is $(2k-1)$-degenerate.
    Let $D_0$ be the subdigraph obtained from $D$ by removing all digons.
    By Proposition~\ref{prop:nash_williams_more_precise}, there is an orientation $D'_0$ of $\UG(D_0)$ such that together with the digons of $D$, this digraph is $k$-arc-strong.
    By Theorem~\ref{theorem:bound_diam_L_degenerate}, there is a set $\mathcal{X}$ of at most $\log n + 2(2k-1) -1$ inversions transforming $D_0$ into $D'_0$. Since digons are preserved by inversions, we deduce that $\Inv(D;\mathcal{X})$ is $k$-arc-strong, and so $\sinv'_k(D) \leq \log n + 4k-3$.
\end{proof}

In order to prove the lower bound in Theorem~\ref{thm:extrem}, we first need the following intermediate result.


\begin{lemma}\label{sizet}
    For every positive integer $t$, there exists a 2-edge-connected digraph $D$ on $2^{t-1}+1+\binom{2^{t-1}+1}{2}$ vertices with a specified vertex $s$ such that  each digraph obtained from $D$ by applying at most $t-1$ inversions contains a sink or source distinct from $s$.
\end{lemma}
\begin{proof}
    Let first $S$ be a set of $2^{t-1}+1$ vertices. Now we obtain the digraph $D$ by adding a vertex $v_{\{s_1,s_2\}}$ as well as the arcs $s_1v_{\{s_1,s_2\}}$ and $s_2v_{\{s_1,s_2\}}$ for every $\{s_1,s_2\}\subseteq S$ . It is easy to see that $D$ is 2-edge-connected and that the number of vertices of $D$ is $2^{t-1}+1+\binom{2^{t-1}+1}{2}$. Now let $D'$ be obtained from $D$ by inverting a collection of $t-1$ sets $X_1,\ldots,X_{t-1} \subseteq V(D)$. As $|S|=2^{t-1}+1$, there exist distinct vertices $s_1,s_2 \in S$  such that for $i=1,\ldots,t-1$, we have either $\{s_1,s_2\}\subseteq X_i$ or $\{s_1,s_2\}\cap X_i = \emptyset$.
    %Let $v$ be the unique vertex in $V(D)$ such that $A(D)$ contains the arcs $s_1v$ and $s_2v$.
    Then in each of the $t-1$ inversions, either both or none of the arcs incident to $v_{\{s_1,s_2\}}$ are inverted. We obtain that $v_{\{s_1,s_2\}}$ is either a source or a sink in $D'$. The statement hence follows for an arbitrary $s \in S$.
\end{proof}

The next lemma gives a construction of graphs of arbitrary size that need a significant amount of inversions to become strong.


\begin{lemma}\label{arbn1}
    For every positive integer $n\geq 3$, there is a 2-edge-connected digraph $D$ on $n$ vertices such that any digraph obtained from $D$ by applying at most $\frac{1}{2}\lceil\log n \rceil-1$ inversions contains a sink or a source.
\end{lemma}

\begin{proof}
For $n=3,4$, the statement is clearly true. We may therefore suppose that $n \geq 5$ and hence $\frac{3}{2 \sqrt{2}\sqrt{n}}\leq \frac{1}{2}$.
    Let $n'=2^{\frac{1}{2}\lceil\log n\rceil-1}+1+\binom{2^{\frac{1}{2}\lceil\log n\rceil-1}+1}{2}$. By Lemma~\ref{sizet}, there is a digraph $D'$ on $n'$ vertices together with a vertex $s \in V(D')$ such that every graph obtained from $D'$ by applying at most $\frac{1}{2}\lceil\log n \rceil-1$ inversions contains a sink or source distinct from $s$.

    Next observe that 
    \begin{align*}
        n'&=2^{\frac{1}{2}\lceil\log n\rceil-1}+1+\binom{2^{\frac{1}{2}\lceil\log n\rceil-1}+1}{2}\\
        &=2^{\frac{1}{2}\lceil\log n\rceil-1}+1+\frac{1}{2}(2^{\frac{1}{2}\lceil\log n\rceil-1}+1)2^{\frac{1}{2}\lceil\log n\rceil-1}\\
        &\leq 2^{\frac{1}{2}\log n-\frac{1}{2}}+1+\frac{1}{2}(2^{\frac{1}{2}\log n-\frac{1}{2}}+1)2^{\frac{1}{2}\log n-\frac{1}{2}}\\
        &= \frac{\sqrt{n}}{\sqrt{2}}+1+\frac{1}{2}(\frac{\sqrt{n}}{\sqrt{2}}+1)\frac{\sqrt{n}}{\sqrt{2}}\\
        &=\frac{n}{4}+\frac{3}{2 \sqrt{2}}\sqrt{n}+1\\
        &=\frac{n}{4}+\frac{3}{2 \sqrt{2}\sqrt{n}}n+1\\
        &\leq \frac{n}{4}+\frac{n}{2}+\frac{n}{4}\\
        &=n.
    \end{align*}

We now obtain $D$ from $D'$ by adding a new set of $n-n'$ vertices and for each of them, adding a digon linking it to $s$. By construction, $D$ has $n$ vertices. Further, as the same property holds for $D'$, in any graph obtained from $D$ by at most $\frac{1}{2}\lceil\log n \rceil-1$ inversions, one of the vertices in $V(D')\setminus \{s\}$ is a source or a sink.
\end{proof}

We are now ready to prove the lower bound in Theorem~\ref{thm:extrem}.

\begin{lemma}\label{extreminf}
    For every pair of positive integers $n,k$ with $n \geq k+2$, there is a $2k$-edge-connected digraph $D$ on $n$ vertices such that $\sinv'_k(D) \geq \frac{1}{2}\log(n -k+1)$.
\end{lemma}

\begin{proof}
By assumption, we have $n-k+1\geq 3$.
    Hence, by Lemma~\ref{arbn1}, there exists a 2-edge-connected digraph $D_0$ on $n-k+1$ vertices such that any digraph obtained from $D_0$ by inverting fewer than $\frac{1}{2} \log(n-k+1)$ sets has a sink or a source. %There are infinitely many such oriented graphs by Lemma~\ref{thm:inf_bound_sinv'_1}.
    Now let $D$ be the digraph obtained 
    from $D_0$ by adding a set $S$ of $k-1$ vertices, for any pair of vertices in $S$ a digon linking them, and for every vertex in $S$ and every vertex in $V(D_0)$ a digon linking these vertices. %and $\bid{K_{k-1}}$ by adding
    %a digon between $u$ and $v$ for every $u \in V(D_0)$ and $v \in V(\bid{K_{k-1}})$.
    %
    %for every vertex $v$ in $D$ the arcs from $D_0$ to $\bid{K_{k-1}}$ and from $\bid{K_{k-1}}$ to $D_0$.
    One can check that $D$ has $n$ vertices and is $2k$-edge-connected.
    
    Now consider a family of subsets $\cal X$ such that
    $D'=\Inv(D; {\cal X})$ is $k$-arc-strong. Any vertex $v\in V(D_0)$ is linked to the vertices of $S$ by digons in $D$. Since an inversion transforms a digon into a digon, it is also connected to the vertices of $S$ by digons in $D'$. Hence, in $D'$, $v$ has at least one in-neighbour and one out-neighbour in $V(D_0)$. In other words, the subdigraph of $D'$ induced by $V(D_0)$ has no source and no sink. Thus $|{\cal X}| \geq \frac{1}{2}\log(n-k+1)$.
    
     Therefore $\sinv'_k(D) \geq \frac{1}{2}\log(n-k+1)$.
\end{proof}

% {\color{purple}
% In order to prove the lower bound in Theorem~\ref{thm:extrem}, we need the following intermediate result. Another feature of the construction which is interesting in its own respect is that even when restricting to oriented graphs whose underlying graphs are highly connected, we cannot hope to be able to make them strong in a constant number of inversions.
% %One can wonder if strengthening the condition ``$G$ $2k$-edge-connected'' by ''$G$ is $Ck$-edge-connected'' for some $C>2$ makes possible to lose the dependency in $n$.
% %The answer is no because of the following construction.

% \begin{lemma}\label{thm:inf_bound_sinv'_1}
%     For every positive integer $k$, there are infinitely many $k$-vertex-connected $n$-vertex oriented graphs $D$ such that  any oriented graph obtained from $D$ by at most $\frac{1}{k}\log n - \log k$ inversions has a sink or a source. In particular,  $\sinv_1(D) > \frac{1}{k}\log n - \log k$.
% \end{lemma}

% \begin{proof}
%     Let $t$ be a positive integer.
%     Let $D$ be the oriented graph obtained from a set $X$ of $\ell= 2^t(k-1)+1$ vertices by adding for every subset $S \subseteq X$ of size $k$ a vertex $v_S$ with
%     $N_D^+(v_S)=S$ and $N_D^-(v_S)=\emptyset$. One can easily check that $D$ is $k$-vertex-connected. 
%     Observe that $n = |V(D)| = \ell + \binom{\ell}{k} \leq 2^{kt}k^k$ and so $t \geq \frac{1}{k}\log n - \log k$.
    
%     %We shall now show that $\sinv_1(D) > t$.
%    % {\color{gray}Suppose for a contradiction that $(X_i)_{i\in [t]}$ is a $k$-arc-strengthening family of $D$.}
   
% % Suppose for contradiction that there exists $X_1, \dots, X_t \subseteq V(D)$ such that $\Inv(D; (X_i)_{i\in[t]})$ has no source and no sink.
%   %  Since $\ell \geq 2^t (k-1)+1$, there is a set $S$ of $k$ vertices in $V(T)$ such that for every
%    % $i \in [t]$, either $S\subseteq X_i$ or $S\cap X_i = \emptyset$.
%     %Thus after inverting $(X_i)_{i\in [t]}$, the vertex $v_S$ is either a sink or a source in $\Inv(T;(X_i)_{i\in [t]})$, contradicting the assumption that 
%  %   $\Inv(D;(X_i)_{i\in [t]})$ has no source and no sink.


%      %We shall now show that $\sinv_1(D) > t$.
%    % {\color{gray}Suppose for a contradiction that $(X_i)_{i\in [t]}$ is a $k$-arc-strengthening family of $D$.}
   
% Let $(X_i)_{i\in[t]}$ be a family of $t$ subsets of $V(D)$.
%     Since $\ell = 2^t (k-1)+1$, there is a set $S$ of $k$ vertices in $V(T)$ such that for every
%     $i \in [t]$, either $S\subseteq X_i$ or $S\cap X_i = \emptyset$.
%     Thus after inverting $(X_i)_{i\in [t]}$, the vertex $v_S$ is either a sink or a source in $\Inv(T;(X_i)_{i\in [t]})$. 
% \end{proof}
% We are now ready to prove the lower bound in Theorem~\ref{thm:extrem}.

% \begin{lemma}\label{extreminf}
%     For every integer $k$, there are infinitely many $2k$-edge-connected digraphs $D$ on $n$ vertices such that $\sinv'_k(D) \geq \frac{1}{2}\log(n -k+1)$.
% \end{lemma}

% \begin{proof}
%     By Lemma~\ref{thm:inf_bound_sinv'_1}, there are infinitely many
%     $2$-edge-connected oriented graphs $D_0$ on $n_0$ vertices such that any oriented graph obtained from $D_0$ by inverting at most $\frac{1}{2} \log(n_0)-1$ sets has a sink or a source. 
%     %There are infinitely many such oriented graphs by Lemma~\ref{thm:inf_bound_sinv'_1}.
%     Let $D$ be the oriented graph obtained 
%     the disjoint union of $D_0$ and $\bid{K_{k-1}}$ by adding
%     a digon between $u$ and $v$ for every $u \in V(D_0)$ and $v \in V(\bid{K_{k-1}})$.
%     %
%     %for every vertex $v$ in $D$ the arcs from $D_0$ to $\bid{K_{k-1}}$ and from $\bid{K_{k-1}}$ to $D_0$.
%     One can check that $D$ is $2k$-edge-connected.
    
%     Consider now a family $\cal X$ of subsets of vertices of $D$ such that
%     $D'=\Inv(D; {\cal X})$ is $k$-arc-strong. Any vertex $v\in V(D_0)$ is linked to the vertices of $\bid{K_{k-1}}$ by digons in $D$. Since an inversion transforms a digon into a digon, it is also connected to the vertices of $\bid{K_{k-1}}$ by digons in $D'$. Hence, in $D'$, $v$ has at least one in-neighbour and one out-neighbour in $V(D_0)$. In other words, the subdigraph of $D'$ induced by $V(D_0)$ has no source and no sink. Thus, by Theorem~\ref{thm:inf_bound_sinv'_1}, $|{\cal X}| \geq  \frac{1}{2}\log(n_0) = \frac{1}{2}\log(n-k+1)$.
    
%      Therefore $\sinv'_k(D) \geq \frac{1}{2}\log(n-k+1)$.
% \end{proof}}
     
     
     Finally, Lemmas~\ref{extremsup} and~\ref{extreminf} imply Theorem~\ref{thm:extrem}.



