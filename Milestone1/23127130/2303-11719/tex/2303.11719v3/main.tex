\documentclass[a4paper,10pt]{article}
%\usepackage{times}
\usepackage{amsmath,amsthm,amssymb,dsfont,graphicx,xspace,epsfig,xcolor}
\usepackage[plain]{fullpage}
\usepackage{thmtools, thm-restate}
\usepackage{hyperref}

\usepackage{algorithm}\usepackage{algorithmic}
\usepackage{tikz}
\usepackage{color}
\usepackage{comment}
\usepackage{authblk,enumitem}

\newtheorem{theorem}{Theorem}[section]
\newtheorem{corollary}[theorem]{Corollary}
\newtheorem{proposition}[theorem]{Proposition}
\newtheorem{lemma}[theorem]{Lemma}
\newtheorem{claim}{Claim}[theorem]

\theoremstyle{definition}
\newtheorem{remark}[theorem]{Remark}
\newtheorem{question}[theorem]{Question}
\newtheorem{problem}[theorem]{Problem}
\newtheorem{conjecture}[theorem]{Conjecture}
\newtheorem{definition}[theorem]{Definition}
\newtheorem{exercise}{Exercise}
\newtheorem{case}{Case}


%%%%%%%%
% \newcommand{\rev}[1]{\textcolor{blue}{#1}}

% %\newcommand{\red}[1]{\textcolor{red}{#1}}
% %\newcommand{\jbj}[1]{{\color{blue}#1}}
% \newcommand{\FH}[1]{{\color{red}Fred: #1}}
% \newcommand{\FHO}[1]{{\color{orange}Flo: #1}}

% %\newcommand{\QV}[1]{\textcolor{blue}{QV: #1}}
% \newcommand{\CR}[1]{{\color{violet}{\bf CR:} #1}}
% %\newcommand{\jc}[1]{{\color{purple}#1}}
% %\definecolor{dark-green}{rgb}{0.2, 0.5, 0.2}
% \newcommand{\JD}[1]{{\color{dark-green}Julien: #1}}

% \newcommand{\reviewer}[1]{{\color{red}{\bf Reviewer:} #1}}

%%%%%%%%% Macros pour les noms

\newcommand{\ra}{\rightarrow}
\newcommand{\Ra}{\Rightarrow}
\newcommand{\la}{\leftarrow}
\newcommand{\rad}{\Rightarrow}
\newcommand{\dic}{\vec{\chi}}
\newcommand{\bid}{\overleftrightarrow}

\newcommand{\fact}[1]{{\color{blue}#1}}
\newcommand{\sfact}[1]{{\color{red}#1}}
\newcommand{\asm}[1]{{\color{orange}#1}}

\newcommand{\wreach}{\mathrm{WReach}}
\newcommand{\wcol}{\mathrm{wcol}}

\newcommand{\e}{\mathrm{e}}

\newenvironment{subproof}{\par\noindent {\it Proof}.\ }{\hfill$\lozenge$\par\vspace{11pt}}

\newcommand{\true}{\mathrm{true}}
\newcommand{\false}{\mathrm{false}}

\DeclareMathOperator{\Vect}{Vect}
\DeclareMathOperator{\rk}{rk}
\DeclareMathOperator{\Inv}{Inv}
\DeclareMathOperator{\Flp}{Flp}
\DeclareMathOperator{\sinv}{sinv}
\DeclareMathOperator{\inv}{inv}
\DeclareMathOperator{\flp}{flp}
\DeclareMathOperator{\minv}{minv}
%\DeclareMathOperator{\fas}{fas}
\DeclareMathOperator{\unvd}{unvd}
\DeclareMathOperator{\vc}{vc}
\DeclareMathOperator{\TT}{tt}
\DeclareMathOperator{\cc}{cc}
\DeclareMathOperator{\diam}{diam}
\DeclareMathOperator{\dist}{dist}
\DeclareMathOperator{\Ad}{Ad}
\DeclareMathOperator{\Mad}{Mad}
\DeclareMathOperator{\Bin}{Bin}
\DeclareMathOperator{\bigO}{\mathcal{O}}
\DeclareMathOperator{\UG}{UG}
%\let\def\relax
\DeclareMathOperator{\defect}{def}

\newenvironment{proofclaim}[1][]%
	{\par\noindent {\it Proof of claim}. }{ \hfill$\lozenge$\par\vspace{11pt}}


\sloppy

\title{On the minimum number of inversions to make a digraph $k$-(arc-)strong.}

\author[1]{Julien Duron}
\author[2]{Fr\'ed\'eric Havet}
\author[3]{Florian H\"orsch}
\author[2,4]{Cl\'ement Rambaud}

\affil[1]{\'Ecole normale sup\'erieure de Lyon, LIP, France}
\affil[2]{Universit\'e C\^ote d'Azur, CNRS, Inria, I3S, Sophia Antipolis, France}
\affil[3]{CISPA Saarbrücken, Germany}
\affil[4]{DIENS, \'Ecole normale sup\'erieure, CNRS, PSL University, Paris, France}


\date{}

\begin{document}
\maketitle
\begin{abstract}
The {\it inversion} of a set $X$ of vertices in a digraph $D$ consists of reversing the direction of all arcs of $D\langle X\rangle$. We study $\sinv'_k(D)$ (resp. $\sinv_k(D)$) which is (for some positive integer $k$) the minimum number of inversions needed to transform $D$ into a $k$-arc-strong (resp. $k$-strong) digraph and $\sinv'_k(n) = \max\{\sinv'_k(D) \mid D~\mbox{is a $2k$-edge-connected digraph of order $n$}\}$.
Note that $\sinv'_k(D) \leq \sinv_k(D)$.
We show: 
\begin{itemize}
\item[(i)] $\frac{1}{2} \log (n - k+1) \leq \sinv'_k(n) \leq \log n + 4k -3$;
\item[(ii)] for any fixed positive integers $k$ and $t$, deciding whether a given oriented graph $D$ with $\sinv'_k(D)<+\infty$ satisfies $\sinv'_k(D) \leq t$ is NP-complete; 
\item[(iii)] for any fixed positive integers $k$ and $t$, deciding whether a given oriented graph $D$ with $\sinv_k(D)<+\infty$ satisfies $\sinv_k(D) \leq t$ is NP-complete; 
\item[(iv)] if $T$ is a tournament of order at least $2k+1$, then
$\sinv_k(T) \leq 2k$, 
and $\sinv'_k(T) \leq \frac{4}{3}k+o(k)$;

\item[(v)] $\frac{1}{2}\log(2k+1) \leq \sinv'_k(T)$ for some tournament $T$ of order $2k+1$; 

\item[(vi)] if $T$ is a tournament of order at least $19k-2$ (resp. $11k-2$), then
$\sinv_k(T) \leq 1$ (resp. $\sinv_k(T) \leq 3$);
\item[(vii)] for every $\epsilon>0$, there exists $C$ such that $\sinv_k(T) \leq C$ for every tournament $T$ on at least $2k+1 + \epsilon k$ vertices.
\end{itemize}

%
% The {\it inversion} of a set $X$ of vertices in a digraph $D$ consists in reversing the direction of all arcs of $D\langle X\rangle$. We study $\sinv'_k(D)$ (resp. $\sinv_k(D)$) which is the minimum number of inversions needed to transform $D$ into a $k$-arc-strong (resp.$k$-strong) digraph and $\sinv'_k(n) = \max\{\sinv'_k(D) \mid D~\mbox{$2k$-edge-connected digraph of order $n$}\}$. We show : (i) $\frac{1}{2} \log (n - k+1) \leq \sinv'_k(n) \leq \log n + 4n -3$;
% (ii) for any fixed positive integers $k$ and $t$, deciding whether a given oriented graph $\vec{G}$ satisfies $\sinv'_k(\vec{G}) \leq t$ (resp. $\sinv_k(\vec{G}) \leq t$) is NP-complete; 
% (iii) if $T$ is a tournament of order at least $2k+1$, then
% $\frac{1}{2} \log(2k+1) \leq \sinv'_k(T) \leq \sinv_k(T) \leq 2k$ \CR{On peut pas ecrire ca, la borne inf est vraie que pour certains $T$}; 
% (iv) if $T$ is a tournament of order at least $28k-5$ (resp. $14k-3$), then
% $\sinv_k(T) \leq 1$ (resp. $\sinv_k(T) \leq 6$);
% (v) for every $\epsilon>0$, there exists $C$ such that $\sinv_k(T) \leq C$ for every tournament $T$ on at least $(2+\epsilon)k+2$ vertices.

\medskip

\noindent{}{\bf Keywords:}  inversion; tournament; $k$-strong; $k$-arc-strong.
\end{abstract}


\section{Introduction}
\label{sec:introduction}
% \begin{itemize}
%     % Diffusion of FL
%     \item {\st{Diffusion of FL}}
%     % Security threats to FL
%     \item {\st{Security threats to FL with particular focus on model poisoning}}
%     % Limitations of existing countermeasures
%     \item {\st{Current countermeasures (e.g., KRUM) and their limitations}}
%     % Proposed method and its advantages
%     \item {\st{Intuitive description of the proposed method and its difference (i.e., advantages) w.r.t. state of the art}}
%     % Main contributions
%     \item {\st{Summary of the main contributions of this work}}
%     % Paper's structure and organization
%     \item {\st{Paper's structure and organization}}
% \end{itemize}

% Diffusion of FL
Recently, {\em federated learning} (FL) has emerged as the leading paradigm for training distributed, large-scale, and privacy-preserving machine learning (ML) systems~\cite{mcmahan2017googleai,mcmahan2017aistats}. 
The core idea of FL is to allow multiple edge clients to collaboratively train a shared, global model without disclosing their local private training data.
%Specifically, an FL system consists of a central server and many edge clients; 
A typical FL round involves the following steps: {\em(i)} the server randomly picks some clients and sends them the current, global model; {\em(ii)} each selected client locally trains its model with its own private data; then, it sends the resulting local model to the server;\footnote{Whenever we refer to global/local model, we mean global/local model {\em parameters}.} {\em(iii)} the server updates the global model by computing an \emph{aggregation function}, usually the average (FedAvg), on the local models received from clients.
% \begin{enumerate}
%     \item[{\em(i)}] the server sends the current, global model to the clients and appoints some of them for training;
%     \item[{\em(ii)}] each selected client locally trains its copy of the global model with its own private data; then, it sends the resulting local model back to the server;\footnote{Whenever we refer to global/local model, we mean global/local model {\em parameters}.}
%     \item[{\em(iii)}] the server updates the global model by computing an \emph{aggregation function} on the local models received from clients (by default, the average, also referred to as FedAvg~\cite{mcmahan2017aistats}).
% \end{enumerate}
This process goes on until the global model converges. %(e.g., after a certain number of rounds or other similar stopping criteria).
%\\
% The advantages of FL over the traditional, centralized learning paradigm are undoubtedly clear in terms of flexibility/scalability (clients can join/disconnect from the FL network dynamically), network communications (only model weights\footnote{We will use \textit{parameters} and \textit{weights} interchangeably.} are exchanged between clients and server), and privacy (each client's private training data is kept local at the client's end and not uploaded to the server).
\\
% Security threats to FL
%However, the growing adoption of FL also raises security concerns~\cite{costa2022covert}, particularly about its confidentiality, integrity, and availability.
Although its advantages over standard ML, FL also raises security concerns~\cite{costa2022covert}. %, particularly about its confidentiality, integrity, and availability~\cite{costa2022covert}.
% OLD, LONG VERSION
% Indeed, some work deals with privacy leakage that may expose the local data of some clients~\cite{melis2019sp}. 
% A large body of work, instead, investigates attacks that usually aim to detriment the predictive accuracy of the learned global model. For instance, \emph{data poisoning} attacks achieve this goal by letting an adversary pollute the training set of some corrupt FL clients with maliciously crafted examples~\cite{jagielski2018sp}.
% Similarly, in \emph{model poisoning} the attacker attempts to tweak the global model weights~\cite{bhagoji2019pmlr} by directly perturbing the local model's weights of some infected FL clients before these are sent to the central server for aggregation, usually via so-called Byzantine attacks. 
% It turns out that Byzantine model poisoning attacks severely impact standard FedAvg; therefore, more robust aggregation functions must be designed to make FL systems secure.
Here, we focus on \emph{untargeted model poisoning} attacks~\cite{bhagoji2019pmlr}, where an adversary attempts to tweak the global model weights %\footnote{We will use the terms \textit{parameters} and \textit{weights} interchangeably.} 
by directly perturbing the local model's parameters of some infected clients before these are sent to the central server for aggregation.
In doing so, the adversary aims to jeopardize the global model \textit{indiscriminately} at inference time.
Such model poisoning attacks severely impact standard FedAvg; therefore, more robust aggregation functions must be designed to secure FL systems.
\\
% In this paper, we focus on designing a novel robust aggregation scheme at the server's end to contrast the effect of Byzantine model poisoning attacks.
%
% Current countermeasures and their limitations
%Several countermeasures have been proposed in the literature to combat model poisoning attacks on FL systems.
% Some methods use simple statistics more robust than plain average to smooth the impact of malicious updates (e.g., Trimmed Mean and FedMedian~\cite{yin2018icml}). 
% Other defenses implement outlier detection techniques to discard malicious updates from the aggregation performed at the server's end. Those are either based on heuristics (e.g., Krum/Multi-Krum~\cite{blanchard2017nips} and Bulyan~\cite{mhamdi2018pmlr}) or data-driven approaches (e.g., K-means clustering~\cite{shen2016acm} or DnC via spectral analysis~\cite{shejwalkar2021ndss}). 
% Finally, some strategies rely on a centralized ``source of trust'' to spot potential malicious updates (e.g., FLTrust~\cite{cao2020fltrust}).
% Several countermeasures have been proposed in the literature to combat model poisoning attacks on FL systems, i.e., to discard possible malicious local updates from the aggregation performed at the server's end. 
% These techniques range from simple statistics more robust than plain average (e.g., Trimmed Mean and FedMedian~\cite{yin2018icml}) to outlier detection heuristics (e.g., Krum/Multi-Krum~\cite{blanchard2017nips} and Bulyan~\cite{mhamdi2018pmlr}) or data-driven approaches (e.g., spectral analysis via K-means clustering~\cite{shen2016acm} or spectral analysis), or methods based on ``source of trust'' (e.g., FLTrust~\cite{cao2020fltrust}).
% OLD, LONG VERSION
%Several countermeasures have been proposed in the literature to combat Byzantine model poisoning attacks on FL systems.
% Descriptive statistics
% For example, Trimmed Mean and FedMedian aggregate local model updates using more robust statistics than standard average~\cite{yin2018icml}.
%
% % Heuristics for outlier detection
% Many existing Byzantine-resilient strategies implement some outlier detection heuristics to discard the model updates sent by potentially malicious clients from the input of the aggregation function.
% One of the most popular heuristics is Krum~\cite{blanchard2017nips}.
% This strategy tries to mitigate the impact of Byzantine attacks by selecting as a global model the local model with the smallest sum of Euclidean distances to {\em all} the other local models.
% Although powerful, Krum requires the server to know (or, at least, estimate) the number of malicious FL clients upfront, which is generally impossible in a realistic attack scenario. %
% Moreover, Krum may become ineffective for complex, high-dimensional model parameter spaces due to the curse of dimensionality.
% Bulyan~\cite{mhamdi2018pmlr} tries to overcome this issue by combining Krum with a variant of Trimmed Mean.
% % Data-driven outlier detection
% Other strategies use data-driven outlier detection techniques -- e.g., via K-means clustering~\cite{shen2016acm} -- to spot potential malicious local model updates. 
% %For instance, Shen et al. propose to cluster local model updates with K-means and thus identify outliers.
%
% % Other techniques
% As far as the server is concerned, any local model received can be from a potential malicious client. 
% FLTrust~\cite{cao2020fltrust} assumes the server acts as a client, i.e., trains a local model on an additional {\em trustworthy} dataset at the server's end and compares it against all the local models from other clients. 
% This way, the server can rely on some ``source of trust'' when discarding potentially malicious clients.
%\\
% Limitations of existing Byzantine-resilient strategies
Unfortunately, existing defense mechanisms either rely on simple heuristics (e.g., Trimmed Mean and FedMedian by~\cite{yin2018icml}) or need strong and unrealistic assumptions to work effectively (e.g., foreknowledge or estimation of the number of malicious clients in the FL system, as for Krum/Multi-Krum~\cite{blanchard2017nips} and Bulyan~\cite{mhamdi2018pmlr}, which, however, cannot exceed a fixed threshold).
Furthermore, outlier detection methods using K-means clustering~\cite{shen2016acm} or spectral analysis like DnC~\cite{shejwalkar2021ndss} do not directly consider the temporal evolution of local model updates received.
Finally, strategies like FLTrust~\cite{cao2020fltrust} require the server to collect its own dataset and act as a proper client, thereby altering the standard FL protocol.
\\
% OLD, LONG VERSION
% Overall, existing Byzantine-resilient strategies are either simple heuristics (e.g., FedMedian) or, if they are more complex, they rely on strong and unrealistic assumptions to work effectively (e.g., knowing the number of malicious clients in the FL system in advance, as for Krum and alike).
% Furthermore, data-driven outlier detection methods do not consider the temporary evolution of local model updates received (e.g., K-means clustering). 
% Finally, strategies like FLTrust requires the server to collect its own dataset and act as a proper client, thereby altering the standard FL protocol.
%
% Description of the proposed method
This work introduces a novel pre-aggregation \textit{filter} robust to untargeted model poisoning attacks. Notably, this filter $(i)$ operates without requiring prior knowledge or constraints on the number of malicious clients and $(ii)$ inherently integrates temporal dependencies. 
The FL server can employ this filter as a preprocessing step before applying \textit{any} aggregation function, be it standard like FedAvg or robust like Krum or Bulyan.
Specifically, we formulate the problem of identifying corrupted updates as a multidimensional (i.e., matrix-valued) time series anomaly detection task. 
The key idea is that legitimate local updates, resulting from well-calibrated iterative procedures like stochastic gradient descent (SGD) with an appropriate learning rate, show \textit{higher predictability} compared to malicious updates. This hypothesis stems from the fact that the sequence of gradients (thus, model parameters) observed during legitimate training exhibit regular patterns, as validated in Section~\ref{subsec:intuition}. %until convergence. 
%This regularity may be more pronounced for smooth convex loss functions, but it can still be captured within an appropriate time window, even for more complex and convoluted loss surfaces. 
%We provide evidence of this claim in Appendix~B, where we show that the average mutual information (i.e., ``predictability''), calculated over pairs of legitimate model updates sent at different FL rounds, is significantly higher than the corresponding computation for a malicious client.
\\
Inspired by the matrix autoregressive (MAR) framework for multidimensional time series forecasting~\cite{chen2021je}, we propose the FLANDERS ({\em \textbf{F}ederated \textbf{L}earning meets \textbf{AN}omaly \textbf{DE}tection for a \textbf{R}obust and \textbf{S}ecure}) filter.
The main advantages of FLANDERS over existing strategies like FLDetector~\cite{zhao2020multivariate} are its resilience to large-scale attacks, where $50\%$ or more FL participants are hostile, and the capability of working under realistic non-iid scenarios.
We attribute such a capability to two key factors: $(i)$ FLANDERS works without knowing a priori the ratio of corrupted clients, and $(ii)$ it embodies temporal dependencies between intra- and inter-client updates, quickly recognizing local model drifts caused by evil players. Below, we summarize our main contributions:

\begin{itemize}
\item[{\em(i)}]
We provide empirical evidence that the sequence of models sent by legitimate clients is more predictable than those of malicious participants performing untargeted model poisoning attacks.
\\
\item[{\em(ii)}] 
We introduce FLANDERS, the first pre-aggregation filter for FL robust to untargeted model poisoning based on multidimensional time series anomaly detection.
\\
\item[{\em(iii)}] 
We integrate FLANDERS into Flower,\footnote{\scriptsize{\url{https://flower.dev/}}} a popular FL simulation framework for reproducibility.
\\
\item[{\em(iv)}] 
We show that FLANDERS improves the robustness of the existing aggregation methods under multiple settings: different datasets, client's data distribution (non-iid), models, and attack scenarios.
\\
\item[{\em(v)}] 
We publicly release all the implementation code of FLANDERS along with our experiments.\footnote{\scriptsize{\url{https://anonymous.4open.science/r/flanders_exp-7EEB}}}
\end{itemize}

% Paper's structure and organization
The remainder of the paper is structured as follows. %some related work and the current state-of-the-art solutions to security issues that FL entails. 
Section~\ref{sec:background} covers background and preliminaries. 
In Section~\ref{sec:related}, we discuss related work.
Section~\ref{sec:problem} and Section~\ref{sec:method} describe the problem formulation and the method proposed. % to tackle it. 
Section~\ref{sec:experiments} gathers experimental results. %, and Section~\ref{sec:limitations} discusses some limitations of this work.
Finally, we conclude in Section~\ref{sec:conclusion}.
 %discusses the limitations of this work and draws future research directions.
%reports conclusions and draws perspectives for future research directions.

%%%%%%% OLD %%%%%%%
%to overcome the resilience of Byzantine failures in distributed Stochastic Gradient Descent computations. 
% The strength of Krum is its time complexity, which is linear in the gradient dimension. 
% However, the robustness of the approach is guaranteed for gradient-based learning applications only when the majority of the clients are not compromised. 
% Besides, the aggregation mechanism of Krum, as well as that of similar methods, is robust from a coarse-grained perspective and does not provide solutions to errors and perturbations that may occur at inference time.
%A related approach to~\cite{blanchard2017nips} is the work of Su et al.~\cite{su2016dc}. Here, the authors propose an iterated approximate agreement to tackle a multi-layer scenario attacked by Byzantine agents. 
%However, the method works efficiently on the sole discrete context and it is inapplicable to continuous state environments.
%\gabri{Maybe, we should just talk about the main limitations of existing countermeasures without digging into their details (or, we can just mention Krum as this is the most popular one). I will move the description of all these methods to the Related Work section.}

\section{ Bounds on \texorpdfstring{$\sinv'_k(n)$}{sinv'k(n)}}\label{sec:oriented}
%%%%%%%%%%%%%%%%%%
This section is dedicated to the extremal results we have on the $k$-arc-strong inversion number. In particular, we prove Theorem~\ref{thm:extrem}, which we restate.

\extrem*


Let $d$ be a positive integer.
A multigraph $G$ is said to be {\bf $d$-degenerate} if every submultigraph of $G$ has a vertex of degree at most $d$.
Every $d$-degenerate graph admits a {\bf $d$-degenerate ordering}, that is an ordering $(v_1, \dots, v_n)$ of the vertices of $G$ such that every vertex has at most $d$ neighbours with lower indices.


We shall need the following proposition, which, as observed in \cite{BJHK}, is a direct consequence of Corollary~2 in~\cite{zoltans06}.
\begin{proposition}\label{prop:nash_williams_more_precise}
Let $G$ be a multigraph that has a $k$-arc-connected orientation for some positive integer $k$
and let $(e_1, f_1),\dots ,(e_t, f_t)$ be a collection of pairwise disjoint pairs of parallel edges in G. Then G has
a $k$-arc-connected orientation in which $e_i$ and $f_i$ are oriented in opposite directions for $i = 1,\dots , t$.
\end{proposition}

We use the following result which has recently been proved in a related paper by the second, third, and fourth author \cite{havet2024diameter}.
%The following result is part of a draft of a forthcoming paper by the second, third, and fourth author.
\begin{theorem}[\cite{diameter}]\label{theorem:bound_diam_L_degenerate}
Let $G$ be an $n$-vertex $d$-degenerate graph.
For any two orientations $\vec{G}_1,\vec{G}_2$ of $G$, one can transform $\vec{G}_1$ into $\vec{G}_2$ by inverting at most $\log n + 2d-1$ sets.
\end{theorem}


In a multigraph $G$, for a pair $(u,v)$ of vertices, a {\bf $(u,v)$-cut} is 
a set $F$ of edges such that every $(u,v)$-path in $G$ intersects $F$. A {\bf cut} is a $(u,v)$-cut for some distinct $u,v$.
Note that a multigraph is $k$-edge-connected if and only if for every pair $u,v$ of vertices, there is no $(u,v)$-cut of size at most $k-1$ in $G$.

A multigraph $G$ is {\bf minimally $k$-edge-connected} if it is $k$-edge-connected and 
$G\setminus e$ is not $k$-edge-connected 
for any edge $e \in E(G)$.
Equivalently, every edge is in a cut of size exactly $k$.

The next result gives an upper bound on the number of edges of a multigraph each edge of which is contained in a small cut.
\begin{lemma}\label{lemma:min_k_connected_mad}
    Let $n$ and $k$ be positive integers with $n \geq k+1$, and let $G$ be an $n$-vertex multigraph.
    If for every edge $uv \in E(G)$ there is a $(u,v)$-cut of size at most $k$ in $G$, then $G$ has at most $k(n-1)$ edges.
\end{lemma}
\begin{proof}
    Let $F_1,\ldots,F_k$ be a set of $k$ edge-disjoint spanning forests in $G$ such that $\sum_{i=1}^k|E(F_i)|$ is maximized. If there is an edge $e=uv \in E(G)\setminus \bigcup_{i=1}^kE(F_i)$, then, as $e$ cannot be added to $E(F_i)$, we obtain that $F_i$ contains a $(u,v)$-path for $i\in [k]$. Hence $G\setminus e$ does not contain any $(u,v)$-cut whose size is smaller than $k$, a contradiction to the fact $e$ is contained in a cut of size at most $k$. This yields $|E(G)|=\sum_{i=1}^k|E(F_i)|\leq k(n-1)$.
\end{proof}

We are now ready to prove the upper bound in Theorem~\ref{thm:extrem}.
\begin{lemma}\label{extremsup}Let $n$ and $k$ be positive integers with $n \geq k+1$. Then,
    for every $2k$-edge-connected $n$-vertex digraph $D$, $\sinv'_k(D) \leq \log n + 4k-3$.
\end{lemma}

\begin{proof}
    Without loss of generality, suppose that $D$ is minimally $2k$-edge-connected.
    Then by Lemma~\ref{lemma:min_k_connected_mad}, every subgraph of $\UG(D)$ has average degree smaller than $2k$.
    This implies that $\UG(D)$ is $(2k-1)$-degenerate.
    Let $D_0$ be the subdigraph obtained from $D$ by removing all digons.
    By Proposition~\ref{prop:nash_williams_more_precise}, there is an orientation $D'_0$ of $\UG(D_0)$ such that together with the digons of $D$, this digraph is $k$-arc-strong.
    By Theorem~\ref{theorem:bound_diam_L_degenerate}, there is a set $\mathcal{X}$ of at most $\log n + 2(2k-1) -1$ inversions transforming $D_0$ into $D'_0$. Since digons are preserved by inversions, we deduce that $\Inv(D;\mathcal{X})$ is $k$-arc-strong, and so $\sinv'_k(D) \leq \log n + 4k-3$.
\end{proof}

In order to prove the lower bound in Theorem~\ref{thm:extrem}, we first need the following intermediate result.


\begin{lemma}\label{sizet}
    For every positive integer $t$, there exists a 2-edge-connected digraph $D$ on $2^{t-1}+1+\binom{2^{t-1}+1}{2}$ vertices with a specified vertex $s$ such that  each digraph obtained from $D$ by applying at most $t-1$ inversions contains a sink or source distinct from $s$.
\end{lemma}
\begin{proof}
    Let first $S$ be a set of $2^{t-1}+1$ vertices. Now we obtain the digraph $D$ by adding a vertex $v_{\{s_1,s_2\}}$ as well as the arcs $s_1v_{\{s_1,s_2\}}$ and $s_2v_{\{s_1,s_2\}}$ for every $\{s_1,s_2\}\subseteq S$ . It is easy to see that $D$ is 2-edge-connected and that the number of vertices of $D$ is $2^{t-1}+1+\binom{2^{t-1}+1}{2}$. Now let $D'$ be obtained from $D$ by inverting a collection of $t-1$ sets $X_1,\ldots,X_{t-1} \subseteq V(D)$. As $|S|=2^{t-1}+1$, there exist distinct vertices $s_1,s_2 \in S$  such that for $i=1,\ldots,t-1$, we have either $\{s_1,s_2\}\subseteq X_i$ or $\{s_1,s_2\}\cap X_i = \emptyset$.
    %Let $v$ be the unique vertex in $V(D)$ such that $A(D)$ contains the arcs $s_1v$ and $s_2v$.
    Then in each of the $t-1$ inversions, either both or none of the arcs incident to $v_{\{s_1,s_2\}}$ are inverted. We obtain that $v_{\{s_1,s_2\}}$ is either a source or a sink in $D'$. The statement hence follows for an arbitrary $s \in S$.
\end{proof}

The next lemma gives a construction of graphs of arbitrary size that need a significant amount of inversions to become strong.


\begin{lemma}\label{arbn1}
    For every positive integer $n\geq 3$, there is a 2-edge-connected digraph $D$ on $n$ vertices such that any digraph obtained from $D$ by applying at most $\frac{1}{2}\lceil\log n \rceil-1$ inversions contains a sink or a source.
\end{lemma}

\begin{proof}
For $n=3,4$, the statement is clearly true. We may therefore suppose that $n \geq 5$ and hence $\frac{3}{2 \sqrt{2}\sqrt{n}}\leq \frac{1}{2}$.
    Let $n'=2^{\frac{1}{2}\lceil\log n\rceil-1}+1+\binom{2^{\frac{1}{2}\lceil\log n\rceil-1}+1}{2}$. By Lemma~\ref{sizet}, there is a digraph $D'$ on $n'$ vertices together with a vertex $s \in V(D')$ such that every graph obtained from $D'$ by applying at most $\frac{1}{2}\lceil\log n \rceil-1$ inversions contains a sink or source distinct from $s$.

    Next observe that 
    \begin{align*}
        n'&=2^{\frac{1}{2}\lceil\log n\rceil-1}+1+\binom{2^{\frac{1}{2}\lceil\log n\rceil-1}+1}{2}\\
        &=2^{\frac{1}{2}\lceil\log n\rceil-1}+1+\frac{1}{2}(2^{\frac{1}{2}\lceil\log n\rceil-1}+1)2^{\frac{1}{2}\lceil\log n\rceil-1}\\
        &\leq 2^{\frac{1}{2}\log n-\frac{1}{2}}+1+\frac{1}{2}(2^{\frac{1}{2}\log n-\frac{1}{2}}+1)2^{\frac{1}{2}\log n-\frac{1}{2}}\\
        &= \frac{\sqrt{n}}{\sqrt{2}}+1+\frac{1}{2}(\frac{\sqrt{n}}{\sqrt{2}}+1)\frac{\sqrt{n}}{\sqrt{2}}\\
        &=\frac{n}{4}+\frac{3}{2 \sqrt{2}}\sqrt{n}+1\\
        &=\frac{n}{4}+\frac{3}{2 \sqrt{2}\sqrt{n}}n+1\\
        &\leq \frac{n}{4}+\frac{n}{2}+\frac{n}{4}\\
        &=n.
    \end{align*}

We now obtain $D$ from $D'$ by adding a new set of $n-n'$ vertices and for each of them, adding a digon linking it to $s$. By construction, $D$ has $n$ vertices. Further, as the same property holds for $D'$, in any graph obtained from $D$ by at most $\frac{1}{2}\lceil\log n \rceil-1$ inversions, one of the vertices in $V(D')\setminus \{s\}$ is a source or a sink.
\end{proof}

We are now ready to prove the lower bound in Theorem~\ref{thm:extrem}.

\begin{lemma}\label{extreminf}
    For every pair of positive integers $n,k$ with $n \geq k+2$, there is a $2k$-edge-connected digraph $D$ on $n$ vertices such that $\sinv'_k(D) \geq \frac{1}{2}\log(n -k+1)$.
\end{lemma}

\begin{proof}
By assumption, we have $n-k+1\geq 3$.
    Hence, by Lemma~\ref{arbn1}, there exists a 2-edge-connected digraph $D_0$ on $n-k+1$ vertices such that any digraph obtained from $D_0$ by inverting fewer than $\frac{1}{2} \log(n-k+1)$ sets has a sink or a source. %There are infinitely many such oriented graphs by Lemma~\ref{thm:inf_bound_sinv'_1}.
    Now let $D$ be the digraph obtained 
    from $D_0$ by adding a set $S$ of $k-1$ vertices, for any pair of vertices in $S$ a digon linking them, and for every vertex in $S$ and every vertex in $V(D_0)$ a digon linking these vertices. %and $\bid{K_{k-1}}$ by adding
    %a digon between $u$ and $v$ for every $u \in V(D_0)$ and $v \in V(\bid{K_{k-1}})$.
    %
    %for every vertex $v$ in $D$ the arcs from $D_0$ to $\bid{K_{k-1}}$ and from $\bid{K_{k-1}}$ to $D_0$.
    One can check that $D$ has $n$ vertices and is $2k$-edge-connected.
    
    Now consider a family of subsets $\cal X$ such that
    $D'=\Inv(D; {\cal X})$ is $k$-arc-strong. Any vertex $v\in V(D_0)$ is linked to the vertices of $S$ by digons in $D$. Since an inversion transforms a digon into a digon, it is also connected to the vertices of $S$ by digons in $D'$. Hence, in $D'$, $v$ has at least one in-neighbour and one out-neighbour in $V(D_0)$. In other words, the subdigraph of $D'$ induced by $V(D_0)$ has no source and no sink. Thus $|{\cal X}| \geq \frac{1}{2}\log(n-k+1)$.
    
     Therefore $\sinv'_k(D) \geq \frac{1}{2}\log(n-k+1)$.
\end{proof}

% {\color{purple}
% In order to prove the lower bound in Theorem~\ref{thm:extrem}, we need the following intermediate result. Another feature of the construction which is interesting in its own respect is that even when restricting to oriented graphs whose underlying graphs are highly connected, we cannot hope to be able to make them strong in a constant number of inversions.
% %One can wonder if strengthening the condition ``$G$ $2k$-edge-connected'' by ''$G$ is $Ck$-edge-connected'' for some $C>2$ makes possible to lose the dependency in $n$.
% %The answer is no because of the following construction.

% \begin{lemma}\label{thm:inf_bound_sinv'_1}
%     For every positive integer $k$, there are infinitely many $k$-vertex-connected $n$-vertex oriented graphs $D$ such that  any oriented graph obtained from $D$ by at most $\frac{1}{k}\log n - \log k$ inversions has a sink or a source. In particular,  $\sinv_1(D) > \frac{1}{k}\log n - \log k$.
% \end{lemma}

% \begin{proof}
%     Let $t$ be a positive integer.
%     Let $D$ be the oriented graph obtained from a set $X$ of $\ell= 2^t(k-1)+1$ vertices by adding for every subset $S \subseteq X$ of size $k$ a vertex $v_S$ with
%     $N_D^+(v_S)=S$ and $N_D^-(v_S)=\emptyset$. One can easily check that $D$ is $k$-vertex-connected. 
%     Observe that $n = |V(D)| = \ell + \binom{\ell}{k} \leq 2^{kt}k^k$ and so $t \geq \frac{1}{k}\log n - \log k$.
    
%     %We shall now show that $\sinv_1(D) > t$.
%    % {\color{gray}Suppose for a contradiction that $(X_i)_{i\in [t]}$ is a $k$-arc-strengthening family of $D$.}
   
% % Suppose for contradiction that there exists $X_1, \dots, X_t \subseteq V(D)$ such that $\Inv(D; (X_i)_{i\in[t]})$ has no source and no sink.
%   %  Since $\ell \geq 2^t (k-1)+1$, there is a set $S$ of $k$ vertices in $V(T)$ such that for every
%    % $i \in [t]$, either $S\subseteq X_i$ or $S\cap X_i = \emptyset$.
%     %Thus after inverting $(X_i)_{i\in [t]}$, the vertex $v_S$ is either a sink or a source in $\Inv(T;(X_i)_{i\in [t]})$, contradicting the assumption that 
%  %   $\Inv(D;(X_i)_{i\in [t]})$ has no source and no sink.


%      %We shall now show that $\sinv_1(D) > t$.
%    % {\color{gray}Suppose for a contradiction that $(X_i)_{i\in [t]}$ is a $k$-arc-strengthening family of $D$.}
   
% Let $(X_i)_{i\in[t]}$ be a family of $t$ subsets of $V(D)$.
%     Since $\ell = 2^t (k-1)+1$, there is a set $S$ of $k$ vertices in $V(T)$ such that for every
%     $i \in [t]$, either $S\subseteq X_i$ or $S\cap X_i = \emptyset$.
%     Thus after inverting $(X_i)_{i\in [t]}$, the vertex $v_S$ is either a sink or a source in $\Inv(T;(X_i)_{i\in [t]})$. 
% \end{proof}
% We are now ready to prove the lower bound in Theorem~\ref{thm:extrem}.

% \begin{lemma}\label{extreminf}
%     For every integer $k$, there are infinitely many $2k$-edge-connected digraphs $D$ on $n$ vertices such that $\sinv'_k(D) \geq \frac{1}{2}\log(n -k+1)$.
% \end{lemma}

% \begin{proof}
%     By Lemma~\ref{thm:inf_bound_sinv'_1}, there are infinitely many
%     $2$-edge-connected oriented graphs $D_0$ on $n_0$ vertices such that any oriented graph obtained from $D_0$ by inverting at most $\frac{1}{2} \log(n_0)-1$ sets has a sink or a source. 
%     %There are infinitely many such oriented graphs by Lemma~\ref{thm:inf_bound_sinv'_1}.
%     Let $D$ be the oriented graph obtained 
%     the disjoint union of $D_0$ and $\bid{K_{k-1}}$ by adding
%     a digon between $u$ and $v$ for every $u \in V(D_0)$ and $v \in V(\bid{K_{k-1}})$.
%     %
%     %for every vertex $v$ in $D$ the arcs from $D_0$ to $\bid{K_{k-1}}$ and from $\bid{K_{k-1}}$ to $D_0$.
%     One can check that $D$ is $2k$-edge-connected.
    
%     Consider now a family $\cal X$ of subsets of vertices of $D$ such that
%     $D'=\Inv(D; {\cal X})$ is $k$-arc-strong. Any vertex $v\in V(D_0)$ is linked to the vertices of $\bid{K_{k-1}}$ by digons in $D$. Since an inversion transforms a digon into a digon, it is also connected to the vertices of $\bid{K_{k-1}}$ by digons in $D'$. Hence, in $D'$, $v$ has at least one in-neighbour and one out-neighbour in $V(D_0)$. In other words, the subdigraph of $D'$ induced by $V(D_0)$ has no source and no sink. Thus, by Theorem~\ref{thm:inf_bound_sinv'_1}, $|{\cal X}| \geq  \frac{1}{2}\log(n_0) = \frac{1}{2}\log(n-k+1)$.
    
%      Therefore $\sinv'_k(D) \geq \frac{1}{2}\log(n-k+1)$.
% \end{proof}}
     
     
     Finally, Lemmas~\ref{extremsup} and~\ref{extreminf} imply Theorem~\ref{thm:extrem}.





\section{Complexity Analysis}
\label{sec:complexity_analysis}

{\bf Size bounds.} For a join query $Q$, its hypergraph $H(Q)$ has one node per variable in $Q$ and one hyperedge per relation in $Q$.  Figures~\ref{fig:example_intro_varorder} depicts a query hypergraph.

An edge cover is a subset of (hyper)edges of $H(Q)$ such that each node appears in at least one edge. Edge cover can be formulated as an integer programming problem by assigning to each edge $R_i$ a weight $w_{R_i}$ that can be $1$ if $R_i$ is part of the cover and $0$ otherwise. The size of an edge cover upper bounds the size of the query result, since the Cartesian product of the relations in the cover includes the
query result: $|Q(\db)| \leq |R_1|^{w_{R_1}}\cdot\ldots\cdot|R_n|^{w_{R_n}}$, where the database $\db$ is $(R_1,\ldots,R_n)$. By minimizing the size of the edge cover, we can obtain a lower upper bound on the size of the query result. This bound becomes tight for fractional weights~\cite{AGM:2013}.  Minimizing the sum of the weights thus becomes the objective of a linear program.

\begin{definition}[\cite{AGM:2013}]\label{def:agm}
Given a join query $Q$ over a database $(R_1,\ldots,R_n)$, the {\em fractional edge cover number} $\rho^*(Q)$ is the cost of an optimal solution to the linear program with variables $(w_{R_i})_{i\in[n]}$ representing weights of $(R_i)_{i\in[n]}$:
\begin{flalign*}
\textrm{minimize} &\prod_{i\in[n]} |R_i|^{w_{R_i}}\\
\textrm{subject to} &\sum_{R\textrm{ is relation of } X} w_R \geq 1~~\textrm{for each variable } X \\
&~~~\forall i\in[n]: \omega_{R_i}\geq 0.
\end{flalign*}
\end{definition}

\begin{example}
\em
Consider the triangle query:
\begin{align*}
Q_{\vartriangle} = R(A,B), S(B,C), T(C,A)
\end{align*}
Figure~\ref{fig:triangle_hypergraph_viewtree} gives the hypergraph of $Q_{\vartriangle}$. The linear program is: 
\begin{flalign*}
\textrm{\em minimize} & \quad |R|^{w_{R}} \cdot |S|^{w_{S}} \cdot |T|^{w_{T}} \\ 
\textrm{\em subject to} & \quad
\begin{tabular}[t]{@{}c@{\hspace*{.5em}}c@{\hspace*{.25em}}c@{\hspace*{.25em}}c@{\hspace*{.25em}}c@{\hspace*{.25em}}c@{\hspace*{.25em}}c}
$A:$ & $w_{R}$ & & & $+$& ${w_{T}}$& $\geq 1$ \\
$B:$ & $w_{R}$ & $+$ & ${w_{S}}$ & & & $\geq 1$ \\
$C:$ & & & $w_{S}$ & $+$ & ${w_{T}}$ & $\geq 1$
\end{tabular}
\end{flalign*}
% 
For $|R|=|S|=|T|=N$, setting $w_{R}=w_{S}=w_{T}=1/2$ gives the optimal solution $\rho^*(Q_{\vartriangle}) = N^{3/2}$. Consequently, the query result has $\bigO{N^{3/2}}$ tuples. This bound is tight in the sense that there exist classes of databases for which the result size is at least $\Omega(N^{3/2})$. For the acyclic query $Q$ in Section~\ref{sec:introduction}, setting the weights $1$ to each of the three relations gives $\rho^*(Q)=N^3$ if all relations have size $N$.
\punto
\end{example}

\nop{
Cardinality constraints can be used to lower the size bounds of query results. For instance, if the number of distinct $A$-values in $R(A,B)$ is $k \ll N$, then we can refine  $Q_\vartriangle$ as $R(A,B),S(B,C),T(C,A),U(A)$ with the new size bound $\rho^*(Q_\vartriangle) = N \cdot k$, where $w_{S}=1$ and $w_{U}=1$.

Join selectivities can also be incorporated to obtain a size {\em estimate} (in contrast to an upper bound). For instance, assume the selectivity of the join on $A$ between $R$ and $T$ is very low: $sel(A) = \frac{|R(A,B),T(C,A)|}{|R|\cdot|T|} = \frac{k}{N}$. Then, we consider a relation $U(A,B,C)=R(A,B),T(C,A)$ whose size estimate is $k \cdot N$ and use this as a cardinality constraint to obtain an estimate of $k \cdot N$ for $Q_\vartriangle$'s size since the size of the join of $S$ and $U$ cannot exceed the size of $U$.
}

Similarly to $\rho^*(Q)$, the {\em factorization width} $\fw(Q)$ governs the sizes of the factorized results of a join query $Q$~\cite{Olteanu:FactBounds:2015:TODS}. In a factorized join over a variable order $\omega$, the values of a variable $X$  depend on the tuples of values of its $\mathit{key}(X)$ variables and are independent of the values for other variables. A tight bound on this number is then given by the size of a join query that covers the variables in $\mathit{key}(X)\cup\{X\}$. We denote this restriction of $Q$ by $Q_{\mathit{key}(X)\cup\{X\}}$. An upper bound on the size of the factorization is then given by the maximum over all variables in $\omega$ of their number of values. This can be improved by going over all possible variable orders of $Q$ and taking the minimum upper bound. This is the factorization width of the query.

\begin{definition}\label{def:fw}
Given a join query $Q$, the {\em factorization width} of $Q$ is  $\fw(Q) = \min_{\omega\in\Omega(Q)} \max_{v\in\mathit{vars}(Q)} \rho^*(Q_{\mathit{key}(X)\cup\{X\}})$.
\end{definition}

\begin{example}\em
For acyclic queries $Q$ over relations $R_1,\ldots,R_n$, $\fw(Q)=\max_{i\in[n]}(|R_i|)$, while $\rho^*(Q)$ can be as much as $\prod_{i\in[n]}|R_i|$ as in our running example. Here are examples of restrictions of our natural join $Q$ in Section~\ref{sec:intro_example}: $\mathit{key}(D)\cup\{D\}=\{C,D\}$ is covered by the query restriction $Q_{\{C,D\}}$ that is the relation $T$; $\mathit{key}(C)\cup\{C\}=\{A,C\}$ is covered by the query restriction $Q_{\{A,C\}}$ that is the relation $S$. For the triangle query $Q_\vartriangle$ and variable order $A-B-C$: $\mathit{key}(C)\cup\{C\}=\{A,B,C\}$ is covered by $Q_\vartriangle$, while $\mathit{key}(B)\cup\{B\}=\{A,B\}$ is covered by relation $R$.
\punto
\end{example}

For any join query $Q$, its factorization width is the fractional hypertree width~\cite{Olteanu:FactBounds:2015:TODS}, a parameter that captures tracta\-bility for a host of computational problems~\cite{FAQ:PODS:2016}.

\begin{proposition}
\label{prop:factorization}
Given a join query $Q$, for every database $\db$, the result $Q(\db)$ admits:
\begin{itemize}
\item a flat representation of size $\bigO{\rho^*(Q)}$~{\em\cite{AGM:2013}};
\item a factorized representation of size $\bigO{\fw(Q)}$~{\em\cite{Olteanu:FactBounds:2015:TODS}}.
\end{itemize}

There are classes of databases $\db$ for which the above size bounds are tight. The flat and factorized representations of $Q(\db)$ can be computed worst-case optimally{\em~\cite{Ngo:SIGREC:2013,Olteanu:FactBounds:2015:TODS}}.
\end{proposition}


\subsection{Dynamic Factorization Width}
\label{sec:dynamic_width}

\milos{Doesn't consider other rings (only LR), indicator projections, and factorizable updates}

As in the non-incremental case, different variable orders may lead to wildly different performance of our IVM approach. In this section, we settle the question of which variable orders can best support IVM under updates to a given set of relations and thereby pinpoint the complexity of maintaining query results under updates. This is captured by a novel notion called {\em dynamic factorization width}, which is a refinement of the factorization width.

We first recall the complexities in the non-incremental case. There, we only materialize the root view of a view tree over a variable order with the smallest factorization width, and we thus have the time data complexity $\bigO{\fw(Q)}$ for computing factorized joins~\cite{Olteanu:FactBounds:2015:TODS} and aggregates over them~\cite{BKOZ:PVLDB:2013,FAQ:PODS:2016}; for cofactor matrices over factorized joins, there is an additional $\bigO{m^2}$ factor, since the sizes of these matrices can be quadratic in the number $m$ of variables (features)~\cite{SOC:SIGMOD:2016}. The space complexity is $\bigO{1}$ or $\bigO{m^2}$ to store the aggregate or cofactor matrix in addition to the database (modulo logarithmic factors in the data size for data iterators).

We next discuss the IVM case. Let $Q$ be any join query. For any variable order $\omega \in \Omega(Q)$, let $\tau(\omega)$ be the view tree inferred from $\omega$. This view tree has exactly one leaf for each relation symbol in $Q$.

We consider updates to relations whose relation symbols in $Q$ form a set ${\mathcal{U}}$; a relation may have several relation symbols  if it is involved in self-joins in $Q$, in which case all of them are in ${\mathcal{U}}$. For a relation symbol $R\in{\mathcal{U}}$, let $\Upsilon_{\tau(\omega)}(R)$ be the set of views that are ancestors of the leaf $R$ in $\tau(\omega)$, i.e., it consists of all the views (recursively) defined using $R$. 

The time needed to compute the delta for a view $\VIEW[keys]{V^{@X}_{rels}}$ is upper bounded by that of a join query $Q^{\sf rels}_{\sf keys \cup \{X\}-\sigma(R)}$ over relations in {\sf rels} that cover $X$ and the variables in {\sf keys} but excluding the variables in $R$. The reason for the exclusion is that a single-tuple update to $R$ binds the variables in $R$ to constants. The overall time to compute the deltas of all views in $\Upsilon_{\tau(\omega)}(R)$ is then
\begin{align*}
T(\omega,R) = \sum_{\VIEW[keys]{V^{@X}_{rels}}\in\Upsilon_{\tau(\omega)}(R)} \rho^*(Q^{\sf rels}_{{\sf keys\cup\{x\}}-\sigma(R)}).
\end{align*}

We are now ready to define the dynamic factorization width that captures the time complexity of incremental maintenance of $Q$ under updates to relations in ${\mathcal{U}}$.

\begin{definition}
Given a join query $Q$ and a set of relation symbols ${\mathcal{U}}$ in $Q$. Then, the {\em dynamic factorization width} of $Q$ and ${\mathcal{U}}$ is $\dfw(Q,{\mathcal{U}}) = \min_{\omega\in\Omega(Q)}\max_{R\in\mathcal{U}} T(\omega,R).$
\end{definition}

\begin{theorem}
Given a query $Q$ with $m$ variables, database $\db$, a payload ring $\RING$, and a set of relations ${\mathcal{U}}$ in $\db$. The time complexity of incrementally maintaining the result of $Q$ over the ring $\RING$ under single-tuple updates to relations in ${\mathcal{U}}$ is $\bigO{\dfw(Q,{\mathcal{U}})\cdot T_\RING}$, where $T_\RING$ is $\bigO{1}$ for rings of numbers and $\bigO{m^2}$ for the degree-$1$ matrix ring.
\end{theorem}

\begin{example}\label{ex:time-complexity}
\em
For our query $Q$ in Section~\ref{sec:intro_example} and database $\db$, the (static) factorization width is $\fw(Q)=O(|R|+|S|+|T|)$. Under single-tuple updates to relations in a set ${\mathcal{U}}_1\subseteq\{R,S\}$, the dynamic factorization width is $\dfw(Q,{\mathcal{U}}_1)=1$ since there are no free variables of the views over $R$ or $S$ in the variable order in Figure~\ref{fig:example_intro_varorder}. This means that we can maintain the result of a sum aggregate over $Q$ in $\bigO{1}$ time under ${\mathcal{U}}_1$ updates. The same holds for ${\mathcal{U}}_2\subseteq\{S,T\}$, i.e., $\dfw(Q,{\mathcal{U}}_2)=1$, as supported by the variable order $C-\{ D, A - \{ B, E \}\}$. However, $\dfw(Q,{\mathcal{U}}_3)=\bigO{|\db|}$ for ${\mathcal{U}}_3=\{R,S,T\}$ since there is no variable order without free variables above all three relations and some variable orders have one free variable above at least one of the three relations. Under the variable order in Figure~\ref{fig:example_intro_varorder}, $\dfw(Q,{\mathcal{U}}_3)=\min(|R|,|S|)$.

The triangle query $Q_\vartriangle$ has the (static) factorization width $\fw(Q_\vartriangle)= \rho^*(Q_\vartriangle)$. For any relation $U \in \{ R, S, T \}$, the dynamic factorization width is $\dfw(Q,\{ U \})=1$ as supported by a path variable order that has the variables in $U$ as prefix. We can thus maintain an aggregate over the triangle query in $\bigO{1}$ under single-tuple updates to exactly one of its three relations. For updates to at least two relations ${\mathcal{U}}_4$, $\dfw(Q,{\mathcal{U}}_4)=O(|\db|)$. For instance, assume a variable order $A-B-C$. We need to cover: no variable under updates to $R$; one of the variables $A$ or $B$ under updates to $S$ or $T$ respectively (the case for other permutations of this variable order is analog). Maintenance has thus lower time cost than recomputation.
\punto
\end{example}

We next analyze the space complexity $S(Q)$ of our approach. This is the sum of the sizes of the views in a view tree. The space needed by the keys of a view $\VIEW[keys]{V^{@X}_{rels}}$ is given by the fractional edge cover of a join query built using relation symbols {\sf rels} to cover the variables in {\sf keys}. To obtain the minimum size, we go over all variable orders of $Q$:
\begin{align*}
 S(Q) = \min_{\omega\in\Omega(Q)}\sum_{\VIEW[keys]{V^{@X}_{rels}}\in\tau(\omega)} \rho^*(Q^{\sf rels}_{\sf keys}).
\end{align*}

\begin{theorem}
Given a query $Q$ with $m$ variables, database $\db$, a payload ring $\RING$. The space complexity required by the materialization of a view tree for $Q$ over the ring $\RING$ is $\bigO{S(Q)\cdot T_\RING}$, where $T_\RING$ is $\bigO{1}$ for the sum ring and $\bigO{m^2}$ for the degree-$m$ matrix ring.
\end{theorem}

There are three differences between the formula $S(Q)$ and Definition~\ref{def:fw} of the factorization width $\fw(Q)$: (1) the use of summation vs. maximum, though the gap between them is linear in $m$ and thus independent of the database size; (2) the cover for $S(Q)$ can only use relation symbols of the view; (3) for $S(Q)$, we only need to cover $\sf keys$ and not also the variable at the view as in the case of $\fw(Q)$. The interplay of (2) and (3) can in fact make $S(Q)$ larger than $\fw(Q)$.
For acyclic queries, both complexities are linear if all relations have the same size and $S(Q)$ can be smaller than $\fw(Q)$ in case some relations are asymptotically smaller than others. 
For cyclic queries, however, $S(Q)$ can be larger than $\fw(Q)$. We show this for the triangle query $Q_\vartriangle$ and relations of the same size $N$. Under any variable order, there is a view of size $\bigO{N^2}$, whereas $\fw(Q_\vartriangle)=N^{3/2}$. For instance, for the variable order $A-B-C$, we materialize the view $\VIEW[A,B]{V^{@C}_{ST}} = \VSUM_{C} \VIEW[B,C]{S} \VPROD \VIEW[C,A]{T} \VPROD \VIEW[C]{\VLIFT_{C}}$, which may create $\bigO{N^2}$ pairs $(A,B)$ as we need both $S$ and $T$ to cover the variables $A$ and $B$. To avoid the large intermediate result, we join all three relations at the same time~\cite{Ngo:SIGREC:2013}, so as to cover $(A,B)$ using $R$. That would, however, require recomputation of this 3-way join for each update. This takes $\bigO{N}$ time since only two of the three variables are bound to constants. In contrast, our IVM approach trades off space for time: We need $\bigO{N^2}$ space but then support $\bigO{1}$ updates to one of the three relations (Example~\ref{ex:time-complexity}).

%%%%%%%%%%%%%%%%%%%%%%%%%%%%%%%%%%%%%%%%%%%%



\section{Bounds on \texorpdfstring{$M'_k$}{M'k} and \texorpdfstring{$M_k$}{Mk}}\label{sec:M}
%%%%%%%%%%%%%%%%%%%%%



In this section, we prove Theorem~\ref{thm:exttournoi} which states $\frac{1}{2} \log(2k+1) \leq M_k \leq 2k$.
The left hand-side  inequality is proved in Theorem~\ref{thm:m(2k+1)}, and the right-hand side one in Theorem~\ref{thm:M<2k}. 





\subsection{Lower bound on \texorpdfstring{$m'_k(2k+1)$}{mk(2k+1)}}
%%%%%%%%%%%%%%%%%%%%%%%%%%%%%

We shall first show the lower bound $m'_k(2k+1) = \Omega (\log k)$. We need the following results.
A digraph is {\bf eulerian} if $d^-(v) = d^+(v)$ for all vertex $v$.


\begin{proposition}[Folklore]\label{eul}
Every $k$-arc-strong tournament on $2k+1$ vertices is eulerian.
\end{proposition}

\begin{theorem}[McKay~\cite{McK90}]\label{count}
The number of labelled eulerian tournaments on $n$ vertices is  $$\left(\frac{2^{n+1}}{\pi n}\right)^{\frac{n-1}{2}}\sqrt{\frac{n}{e}} (1 +o(1)).$$
\end{theorem}
We are now ready to prove the lower bound in Theorem \ref{thm:exttournoi}.
\begin{theorem}\label{thm:m(2k+1)}
For every sufficiently large $k$, $m'_k(2k+1)\geq \frac{1}{2}\log (2k+1)$.
\end{theorem}
\begin{proof}
    Let $n=2k+1$. By Proposition \ref{eul}, all the $k$-arc-strong tournaments on $n$ vertices are eulerian. Furthermore, by Theorem~\ref{count}, if $k$ is large enough, then the number of labelled eulerian tournaments on $n$ vertices is at most
    $\left(\frac{2^{n+1}}{\pi n}\right)^{\frac{n-1}{2}}\sqrt{n}$.
    For two tournaments $T,T'$ on the same vertex set, we say that $T'$ is {\bf reachable} from $T$ by $t$ inversions for some positive integer $t$, if there is a family of sets $(X_1,\ldots,X_t)$ such that $\Inv(T;X_1,\ldots,X_t)=T'$. Observe that there are $2^n$ possibilities to choose $X_i$ for $i=1,\ldots,t$, hence the number of tournaments reachable from $T$ by $t$ inversions is at most $(2^n)^t=2^{nt}$. Therefore,
    %Since at most $2^{nt}$ labelled tournaments are reachable from a fixed $n$-vertex tournament using at most $t$ inversions,
    the number of labelled $n$-vertex tournaments that are reachable from an eulerian  one is at most 
    \begin{eqnarray*}
    2^{nt}\left(\frac{2^{n+1}}{\pi n}\right)^{\frac{n-1}{2}}\sqrt{n}
    & = &2^{\binom{n}{2}}\cdot 2^{nt} \left(\frac{2}{\pi n}\right)^{\frac{n-1}{2}}\sqrt{n} \\
    & = &2^{\binom{n}{2}}\cdot 2^{nt +\frac{n-1}{2}(1 -\log(\pi)-\log n)+\frac{1}{2}\log n}\\
    & < &2^{\binom{n}{2}}\cdot 2^{nt-\frac{1}{2} n \log n}
    \end{eqnarray*}
    for $n$ sufficiently large. % (actually $k = \frac{n-1}{2} \geq 5$ will do)\FHO{J'effacerais la remarque en parentheses. Je ne sais pas si ca marche bien avec le $o(1)$}.
    If $t \leq \frac{1}{2}\log n$, then $nt - \frac{1}{2} n \log n \leq 1$, so the number of such tournaments is less than $2^{\binom{n}{2}}$. It follows that there is at least one tournament $T^*$ on $2k+1$ vertices which cannot be reached from an eulerian tournament by at most $t$ inversions. Thus $\sinv'_k(T^*)>t$.
\end{proof}


Theorem~\ref{thm:m(2k+1)} can be slightly generalised to tournaments with $2k+c$ vertices with $c$ small. To do so, we will need the following generalisation of Theorem \ref{count}.

For a positive integer $n$ and integer $\alpha_1,\dots ,\alpha_n$, we denote by
$NT(n;\alpha_1, \dots,\alpha_n)$ the number of labelled tournaments
of order $n$ whose vertex $i$ has $d^+(i)-d^-(i)=\alpha_i$.
\begin{theorem}[Spencer~\cite{Spencer74} and McKay~\cite{McK90}]\label{thm:spencer}
Let $n$ be a positive odd integer and let $\alpha_1, \dots, \alpha_n$ be integers.
Then
\[
NT(n,\alpha_1, \dots,\alpha_n) \leq
\left(\frac{2^{n+1}}{\pi n}\right)^{(n-1)/2} \sqrt{\frac{n}{\e}} \exp\left(-\frac{1+o(1)}{2n}\sum_{i=1}^n\alpha_i^2\right).
\]
\end{theorem}
%\FHO{Qc est bizarre. Ce resultat est beacoup plus fort que le precedent, mais 15 ans plus vieux.}\CR{Après quelques recherches, il semblerai que Spencer exprime l'estimation de $NT(N,(\alpha_i))$ en fonction de $NT(n,0,\dots,0)$, et c'est McKay qui a eu la premiere estimation de $NT(n,0)$ a un facteur $1+o(1)$ près (celle de Spencer etait à $(1+o(1))^n$ près.}
\begin{corollary}\label{corollary:number_tournaments_high_degree}
For all integers $k$ and $c$, the number of tournaments $T$ on $n=2k+c$ vertices
such that every vertex has in- and out-degree at least $k$ is at most
$2^{\log(2c)n}\left(\frac{2^{n+1}}{\pi n}\right)^{(n-1)/2} \sqrt{\frac{n}{\e}}$ for $k$ large enough.
\end{corollary}

\begin{proof}
By Theorem~\ref{thm:spencer},
the number of tournaments $T$ on $n=2k+c$ vertices
such that every vertex has in- and out-degree at least $k$ is at most

\[
\left(\frac{2^{n+1}}{\pi n}\right)^{(n-1)/2} \sqrt{\frac{n}{\e}}
\sum_{\alpha_1, \dots,\alpha_n \in \{-c+1,\dots c-1\}^n}\exp\left(-\frac{1+o(1)}{2n}\sum_{i=1}^n\alpha_i^2\right).
\]
%\FHO{je ne vois pas pourquoi c'est pas $\{-c,\ldots,c\}$ sous la somme.}
%\CR{On compte le sommet lui même, par exemple pour $c=1$, on a toujours $\alpha_i=0$.}
% Further, we have
% \[
% \begin{split}
% \sum_{\alpha_1, \dots,\alpha_n \in \{-c+1,\dots c-1\}^n}\exp\left(-\frac{1+o(1)}{2n}\sum_{i=1}^n\alpha_i^2\right)
% &= \prod_{i=1}^n\left(\sum_{\alpha=-c+1}^{c-1}\exp\left(-\frac{1+o(1)}{2n} \alpha^2\right)\right) \\
% &= \prod_{i=1}^n\left(1+o(1)+2\sum_{\alpha=1}^{c-1}\exp\left(-\frac{1+o(1)}{2n} \alpha^2\right)\right) \\
% &\leq \left(1+o(1)+2(c-1)\e^{-\frac{1+o(1)}{2n}}\right)^n \\
% &\leq \left(1+o(1)+2(c-1) +o(1)\right)^n \\
% &\leq 2^{\log(2c)n}
% \end{split}
% \]
Further, we have
\[
\begin{split}
\sum_{\alpha_1, \dots,\alpha_n \in \{-c+1,\dots c-1\}^n}\exp\left(-\frac{1+o(1)}{2n}\sum_{i=1}^n\alpha_i^2\right)
&= \prod_{i=1}^n\left(\sum_{\alpha=-c+1}^{c-1}\exp\left(-\frac{1+o(1)}{2n} \alpha^2\right)\right) \\
&\leq  \prod_{i=1}^n\left(\sum_{\alpha=-c+1}^{c-1}1 + o(1)\right) \\
&= (2c-1+o(1))^n \\
&\leq 2^{\log(2c)n}
\end{split}
\]
% \FHO{on ne peut pas simplifier ce calcul en estimant $\exp\left(-\frac{1+o(1)}{2n} \delta^2\right)$ par $1$?}
%\FHO{je ne vois pas du tout d'ou vient la premiere egualite}
%\CR{J'ai rajouté une etape. Il faut juste developper.}
for $n$ large enough, and the result follows.
\end{proof}

% \JD{On ne peut pas plutôt écrire $\exp{ 2 n \log(c)}$ pour la dernière ligne? Si c'est vrai, raffiner les constantes donnerai une borne inf pas si horrible pour le passage à $m_k \leq $ constante, de l'ordre de $c = k^{O(1)}$.}
% \CR{Oui tu as raison, j'ai fait la modif.}

\begin{theorem}\label{thm:2k+c}
    For every positive integer $c$ fixed, and for every positive integer $k$ large enough compared to $c$, there exists a tournament
    $T$ on $2k+c$ vertices such that $\sinv'_k(T) > \frac{1}{2}\log (2k+c)-\log(2c)$.
    In particular $m'_k(2k+c)$ is unbounded for every fixed $c$.
\end{theorem}

\begin{proof}
By Corollary~\ref{corollary:number_tournaments_high_degree},
the number of tournaments $T$ on $n=2k+c$ vertices with $\sinv'_k(T) \leq t$
is for $k$ large enough at most 
\[
2^{nt} 2^{\log(2c)n}\left(\frac{2^{n+1}}{\pi n}\right)^{(n-1)/2} \sqrt{\frac{n}{\e}}
\leq 2^{\binom{n}{2}} 2^{nt-\frac{1}{2}n\log n+\log(2c)n}
\]
which is smaller than $2^{\binom{n}{2}}$ if $t \leq \frac{1}{2}\log n -\log(2c)$.
Hence there exists such a tournament $T$ with $\sinv'_k(T) > \frac{1}{2}\log (2k+c)-\log(2c)$.
\end{proof}







\subsection{Upper bounds on \texorpdfstring{$M_k$}{Mk}}
%%%%%%%%%%%%%%%%%%%%%%%%%


%We shall need the following lemma.

%\begin{lemma}[Folklore]\label{lem:kstrong+}
%Let $k$ be a positive integer, let $D$ be a digraph and let $v$ a vertex such that $d^-(v)\geq k$ and $d^+(v)\geq k$.
%If $D-v$ is $k$-strong, then $D$ is also $k$-strong.
%\end{lemma}


\begin{theorem}\label{thm:M<2k}
$M_k \leq 2k$.
\end{theorem}

\begin{proof}
Let $D$ be a tournament with $V(D)=\{v_1,\ldots,v_n\}$ with $n \geq 2k+1$. Further, let $T$ be a $k$-strong tournament on $\{v_1,\ldots,v_{2k+1}\}$. We now define sets $X_1,\ldots,X_{2k}$. 
Suppose that the sets $X_1,\ldots,X_{i-1}$ have already been created and let $D_{i-1}$ be the graph obtained from $D$ by inverting $X_1,\ldots,X_{i-1}$.
Now let $X_i = \{v_i\} \cup A_i \cup B_i$, where
$A_i$ is the set of vertices $v_j$ with $j \in \{i+1,\ldots,2k+1\}$ for which the edge $v_iv_j$ has a different orientation in $T$ and $D_{i-1}$,
and $B_i$ is, when $i \leq k$ (resp. $i \geq k+1$), the set of vertices $v_j$ with $j \geq 2k+2$ for which $D_{i-1}$ contains the arc $v_iv_j$ (resp. $v_jv_i$).

We still need to show that $D_{2k}$ is $k$-strong. Observe that $D_{2k}\langle\{v_1,\ldots,v_{2k+1}\}\rangle=T$ which is $k$-strong by assumption. However, for any $j\geq 2k+2$, $D_{2k}$ contains the arcs $v_jv_i$ for $i=1,\ldots,k$ and the arcs $v_iv_j$ for $i=k+1,\ldots,2k$. Hence, by Lemma~\ref{lem:kstrong+}, $D_{2k}$ is $k$-strong.
\end{proof}


Theorems~\ref{thm:m(2k+1)} and~\ref{thm:M<2k} directly imply Theorem \ref{thm:exttournoi}.%M_k \leq 2k$.

\subsection{Values of \texorpdfstring{$M'_1$}{M'1}, \texorpdfstring{$M_1$}{M1}, \texorpdfstring{$M'_2$}{M'2} and \texorpdfstring{$M_2$}{M2}}
%%%%%%%%%%%%%%%%%%%%%%%%%%%%%%%%%%%%%%%%%%%%%%%%%%%%%%%
We here provide the exact values of $M_i$ and $M_i'$ for $i \in \{1,2\}$.
\begin{proposition}\label{prop:M1}
Let $T$ be a tournament of order $n \geq 3$.
We have $\sinv_1(T) = \sinv'_1(T) =0$ if $T$ is strong and $\sinv_1(T) = \sinv'_1(T) =1$ otherwise.
In particular,  $m_1(n) = m'_1(n) =1$ for all $n\geq 3$ and $M_1=M'_1=1$.
\end{proposition}
\begin{proof}
(i) Trivially, if $T$ is strong, then $\sinv_1(T)=0$. 
If $T$ is not strong, then $\sinv_1(T) \geq 1$. Now consider a hamiltonian path of $T$
which exists by Redei's Theorem (see e.g. Theorem 1.4.2 in \cite{bang2009}. Let $X$ be its initial vertex and $y$ its terminal vertex.
Then inverting $\{x,y\}$ yields a tournament with a directed hamiltonian cycle because $V(T) \setminus\{x,y\} \neq \emptyset$, so a strong tournament.
Hence $\sinv_1(T) = 1$.
\end{proof}








A non-strong tournament is said to be {\bf reducible}.
It is folklore that a reducible tournament has a {\bf reduction}  $T_1\Ra T_2$ that is two subtournaments $T_1, T_2$ such that $(V(T_1), V(T_2))$ is a partition of $V(T)$ and $V(T_1) \Ra V(T_2)$.





Let $S_4$ be the unique strong tournament of order $4$.
Its vertex set is $\{a,b,c,d\}$ and its arc set is $\{ab, bc, cd, da, ca, db\}$.


\begin{proposition}\label{prop:M2}
$M_2=M'_2=2$.
\end{proposition}
\begin{proof}
$R_5$ the rotative tournament of order $5$ is the only 2-arc-strong tournament of order $5$.
As observed in \cite{BBBP10}, $\inv(R_5)=2$, so $\sinv'_2(TT_5) = 2$.
Hence $M_2 \geq M'_2\geq 2$.

\medskip

Let us now prove that $M_2\leq 2$.
We shall prove by induction on $n$ that every tournament $T$ of order at least $5$ satisfies $\sinv_2(T) \leq 2$.


Assume first that $T$ is a tournament of order $5$.
If $T$ is strong, then, by Camion's theorem~\cite{camion1959}, it has a hamiltonian cycle
$v_1v_2v_3v_4v_5v_1$. Let $A^+=A(T)\cap \{v_1v_3, v_2v_4, v_3v_5, v_4v_1, v_5v_2\}$ and $A^-=A(T)\cap \{v_3v_1, v_4v_2, v_5v_3, v_1v_4, v_2v_5\}$.
We have $|A^+| + |A^-|=5$, so one of the two sets $A^+, A^-$ has at most two arcs. Reversing the arcs of this set, one after another, yields the 2-strong tournament $R_5$.

Assume now that $T$ is not strong. Then it must be one of the following tournaments or their converse. (The {\it converse} of a digraph is the digraph obtained by reversing all arcs.)
\begin{itemize}
\item $TT_5$ with hamiltonian path $v_1v_2v_3v_4v_5$. Then inverting $\{v_1, v_2, v_4, v_5\}$ and $\{v_1, v_5\}$ yields $R_5$.

\item $S_4\Ra \{x\}$. Then inverting $\{c, d, x\}$ and $\{c, d\}$ yields $R_5$.

\item $\{x\} \Ra \vec{C}_3 \Ra \{y\}$ with $ \vec{C}_3 = abca$. Then inverting $\{a,x,y\}$ yields $R_5$.

\item $\{x\} \Ra \{y\} \Ra \vec{C}_3 $ with $ \vec{C}_3 = abca$.
Then inverting $\{a,b,c,y\}$ and $\{a,x,y\}$ yields $R_5$.

\end{itemize}

Assume now that $T$ has at least $6$ vertices.

 Assume first $T$ has a vertex $v$ such that $\min\{d^+(v),d^-(v)\}\geq 2$.
 By the induction hypothesis, $\sinv_2(T-v) \leq 2$, so there is a family ${\cal X}$ of at most two subsets of $V(T-v)$ such that
 $\Inv(T-v; {\cal X})$ is $2$-strong.
 Now $\Inv(T; {\cal X})-v = \Inv(T-v; {\cal X})$ and $\min\{d^+(v),d^-(v)\}\geq 2$. Thus, by Lemma~\ref{lem:kstrong+},
 $\Inv(T; {\cal X})$ is $2$-strong, and it follows that $\sinv_2(T) \leq |{\cal X}| \leq 2$.

Assume now that for every vertex has either in-degree at most $1$ or out-degree at most $1$. Then necessarily $T$ must be the tournament $\vec{C}_3\Ra \vec{C}_3$.
Let $V(T) = \{a,b,c,d,e,f\}$ with $\{a,b,c\} \Ra \{d,e,f\}$.
Then inverting $\{a,b,c,d\}$ and $\{a, d,e,f\}$ transforms $T$ into a $2$-strong tournament.
\end{proof}


\section{Upper bounds on \texorpdfstring{$m_k(n)$}{mk(n)}\label{sec:upper_bound_Mkn}}
%%%%%%%%%%%%%%%%%%%%%%%%%%%%%%%%%%%%
In this section, we prove several results showing that tournaments on significantly more than $2k$ vertices can be made $k$-strong by a small number of inversions. More precisely, in Section \ref{first}, we prove Theorem \ref{thm:s1}, in Section \ref{nk10}, we prove Theorem \ref{nk1}, in Section \ref{nk6sec}, we prove Theorem \ref{nk6}, and in Section \ref{epsgross}, we prove Theorem \ref{thm:2+eps}.
 \subsection{First upper bounds on \texorpdfstring{$m_k(n)$}{mk(n)}}\label{first}
%%%%%%%%%%%%%%%%%%%%%%%%%%%%%

In subsection, we first establish that, for every fixed $k$, $m_k(n)$ tends to $1$ when $n$ tends to infinity. More precisely, we prove Theorem \ref{thm:s1}. While Theorem \ref{thm:s1} is clearly weaker than Theorem \ref{nk1}, this result justifies some of the notation used later on and may serve as a warm-up exercise of the more involved proof of Theorem \ref{nk1}.
%\sone*

\begin{proof}
Let $T$ be a tournament of order $n\geq (2k-1)2^{2k}$.

  It is easy and well-known that if $D$ is an acyclic digraph, $x$ a source in $D$, and $D-x$ is contained in every tournament of order $n$, then $D$ is contained in every tournament of order $2n$.  
An easy induction yields that $T$ contains three sets
  $A_1, A_2, A_3$ such that $A_1\Ra (A_2\cup A_3)$ and
  $A_2\Ra A_3$ with $|A_1|=|A_3| = k$ and $|A_2|=2k-1$.
Set $A=A_1\cup A_2 \cup A_3$.
Let $I$ be the set of vertices in $V(T)\setminus A$ that have either less than $k$ out-neighbours in $A$ or less than $k$ in-neighbours in $A$.
Let $X=A_1\cup A_3\cup I$.
Let us prove that $T'=\Inv(T; X)$ is $k$-strong.

In $T'$, we have $A_1 \Ra A_2 \Ra A_3\Ra A_1$. Since the three sets $A_1$, $A_2$, and $A_3$ have size at least $k$, the tournament $T'\langle A\rangle$ is $k$-strong.
Consider now a vertex $v$ in $V(T)\setminus A$.
If $v\notin I$, then not arcs incident to $v$ has been reversed so its in- and out-degree have been unchanged and are at least $k$ by definition of $I$.
If $v\in I$, then all the arcs between $v$ and $A_1\cup A_3$ have been reversed and those between $v$ and $A_2$ are unchanged.
If $v$ has less than $k$ out-neighbours in $A$ in $T$, then $|N^+_{T'}(v)\cap A| \geq |N^-_T(v) \cap (A_1\cup A_3)| \geq 2k - d^+_T(v) \geq k$, and $|N^-_{T'}(v)\cap A|\geq |N^-_T(v) \cap A_2| \geq 2k-1 - d^+_T(v) \geq k$.
Similarly, if $v$ has less than $k$ in-neighbours in $A$ in $T$, then 
$|N^+_{T'}(v)\cap A|\geq k$ and $|N^-_{T'}(v)\cap A|\geq k$.
Thus, by 
Lemma~\ref{lem:kstrong+}, $T'$ is $k$-strong.
Hence $\sinv_k(T) \leq 1$.
\end{proof}






\subsection{Linear upper bound on \texorpdfstring{$N_k(1)$}{Nk(1)}}\label{nk10}
%%%%%%%%%%%%%%%%%%%%%%%%%%%%%%%%%%%%%%%%%%

In this subsection, we shall prove  Theorem~\ref{nk1} which states that $N_k(1)\leq 28k -5$. To prove it we need some preliminaries.




For a digraph $D$, let $\sigma=(v_1,v_2, \ldots, v_n)$ be an ordering of the vertices of $D$. An arc $v_iv_j$ is {\bf forward} (according to $\sigma$) if $i<j$ and {\bf backward} (according to $\sigma$) if $j<i$.
A {\bf median order} of $D$ is an ordering of the vertices of $D$ with the maximum number of forward arcs, or equivalently the minimum number of backward arcs.

Let us note basic well-known properties of median orders of
tournaments (the feedback property in \cite{HaTh00}).

\begin{lemma}\label{lem:median}
 Let $T$ be a tournament and $(v_1,v_2, \ldots, v_n)$ a
median order of $T$. Then, for any two indices $i,j$ with $1 \leq i <
j \leq n$:
\medskip
\begin{enumerate}
\item[\rm (M1)] $(v_i,v_{i+1},\ldots,v_j)$ is a median order of the
  induced subtournament $T\langle \{v_i,v_{i+1},\ldots,v_j\}\rangle$.\\
  
\item[\rm (M2)] vertex $v_i$ dominates at least half of the vertices
  $v_{i+1},v_{i+2},\ldots,v_j$, and vertex $v_j$ is dominated by at least half of the vertices $v_i,v_{i+1},\ldots,v_{j-1}$.  In particular, each vertex $v_i$, $1 \leq i <n$, dominates its successor $v_{i+1}$. 

\end{enumerate}
\end{lemma}

Let $v$ be a vertex of a digraph $D$. We denote by $R^+_D(v)$ (resp. $R^-_D(v)$) the set of vertices which are {\bf reachable} from $v$ (resp. vertices that can reach $v$) in $D$, that are the vertices $w$ such that there is a directed $(v,w)$-path (resp.~$(w, v)$-path) in $D$. Note that $v\in R^+_D(v)$.


\begin{lemma}\label{lem:n-2F}
Let $T$ be a tournament with median order $(v_1, v_2, \ldots , v_n)$.
Let $F$ be a subset of vertices such that $v_1\notin F$. 
Then $|R^+_{T-F}(v_1)| \geq n - 2 |F|$.
\end{lemma}
\begin{proof}
We prove the result by induction on $n+|F|$, the result holding trivially by (M2) if $|F|=0$.

If all the out-neighbours of $v_1$ are in $|F|$, then by (M2), $|N^-(v_1)| \leq |N^+(v_1)|  \leq |F|$. Hence $n-1 \leq 2|F|$, and the result holds.
Henceforth we may assume that $v_1$ has an out-neighbour not in $F$.
Let $i_0$ be the smallest index of such a vertex.
Let $T_0= T\langle \{v_1, \dots , v_{i_0-1}\}\rangle$, $T_1= T\langle \{v_{i_0}, \dots , v_{n}\}\rangle$,  $F_0=F\cap V(T_0)$ and $F_1=F\cap V(T_1)$.
By (M1), $(v_1, v_2, \ldots , v_{i_0-1})$ is a median order of $T_0$ and $(v_{i_0},  \ldots , v_n)$ is a median order of $T_1$.
By definition of $i_0$, all out-neighbours of $v_1$ in $T_0$ are in $F_0$. Thus, as above, we have $i_0-2 \leq 2|F_0|$.
By the induction hypothesis, $|R^+_{T_1-F_1}(v_{i_0})| \geq n - i_0 + 1 - 2 |F_1|$.
Now $R^+_{T_1-F_1}(v_{i_0}) \cup \{v_1\} \subseteq R^+_{T-F}(v)$. Hence
 $$|R^+_{T-F}(v)| \geq |R^+_{T_1-F_1}(v_{i_0})| +1 \geq n - i_0 +2 - 2|F_1| \geq n - 2|F_0| - 2|F_1| = n - 2|F|.$$
\end{proof}

 
\begin{lemma}\label{median order bounds}
    Let $T$ be a tournament with median order $(v_1, v_2, \ldots , v_n)$. For any $i\in [n]$ and any $X \subseteq V(T)\setminus \{v_i\}$, $|R^+_{T-F}(v_i)| \geq n + 1 - i - 2 |F|$, and $R^-_{T-F}(v_i) \geq i - 2 |F|$.
\end{lemma}
\begin{proof}
    By (M1), $(v_i, \dots, v_n)$ is a median order of $T\langle \{v_i, \dots, v_n\}\rangle$ on which we can apply Lemma~\ref{lem:n-2F} to obtain the bound on $R^+_{T-F}(v_i)$. Symmetrically, $(v_i, v_{i-1}, \dots, v_1)$ is a median order of the converse of $T\langle \{v_1, \dots, v_i \} \rangle$, and the directional dual of Lemma~\ref{lem:n-2F} yields the bound on $R^-_{T-F}(v_i)$.
\end{proof}


We are now ready to prove a linear upper bound on $N_k(1)$.



% \JD{Vraiment pas mal ! On peut enlever peut être 8k au thm en remarquant qu'on a modifié exactement 2k (et pas les 6k) sommets dans A (resp. B) du coup au lieu d'appeler \ref{median order bounds} sur F, on l'applique sur $Y \cup A_0 \cup A_1$, puis on s'assure d'enlever les sommets en trop de $A$. Mais je sais pas si ça vous semble valoir le coup. Le résultat donne $|R_b| \geq n - 6k - (i-1)  - 2(k-1) - 2(2k) - (4k - i) = n - 16k $ si je ne me trompe pas}

% \JD{D'ailleurs pour $N_k(6)$ on peut appliquer exactement le 5.15 au lieu du 5.17, mais en laissant A et B inchangé. Ça donne une meilleure borne sur 5.15 (le cas 4 devient le même que le 3) en principe 4k pour $|A|$? et on peut appliquer la démo de 5.19 ensuite. On se retrouve avec $12k-2$ pour $N_k(3)$.}
% \FHO{Je crois que tu as raison, mais on verra plus tard.}
\begin{lemma}\label{6ab}
    Let $k$ be a positive integer, $T$ a tournament on $12k$ vertices and $(A,B)$ a bipartition of $V(T)$ such that $|A|=|B|=6k$. Then there is a set $X \subseteq V(T)$ such that for $T'=\Inv(T,X)$ and for any $Y \subseteq V(T)$ with $|Y|\leq k-1$, we have that $T'-Y$ contains a directed path from $a$ to $B\setminus Y$ for every $a \in A\setminus Y$ and $T'-Y$ contains a directed path from $A\setminus Y$ to $b$ for every $b \in B\setminus Y$.
\end{lemma}
\begin{proof}
    Let $(a_1,\ldots,a_{6k})$ be a median order of $T\langle A \rangle$ and let $(b_1,\ldots,b_{6k})$ be a median order of $T\langle B \rangle$. Let $A_0$ be the set of vertices in $\{a_{4k+1},\ldots,a_{6k}\}$ which have less than $k$ out-neighbours in $B$ in $T$. Further, let $A_1=\{a_1,\ldots,a_{2k-|A_0|}\}$. Observe that $|A_0 \cup A_1|=|A_0|+|A_1|=2k$. Similarly, let $B_0$ be the set of vertices in $\{b_1,\ldots,b_{2k}\}$ that have less than $k$ in-neighbours in $A$ and let $B_1=\{b_{4k+|B_0|+1},\ldots,b_{6k}\}$. Observe that $|B_0 \cup B_1|=|B_0|+|B_1|=2k$. Now let $X=A_0 \cup A_1 \cup B_0 \cup B_1$, let $T'=\Inv(T,X)$ and let $Y \subseteq V(T)$ with $|Y|\leq k-1$. In order show that $T'$ has the desired properties, by symmetry, it suffices to prove that $T'-Y$ contains a directed path from $a$ to $B\setminus Y$ for every $a \in A\setminus Y$. Suppose for the sake of a contradiction that this is not true. There is a largest integer $i \in [6k]$ such that $a_i \in A\setminus Y$ and $T'-Y$ does not contain a directed path from $a_i$ to $B\setminus Y$. We will distinguish several cases.

    \begin{case}
        $i \in \{4k+1,\ldots,6k\}$ and $a_i \in A\setminus (A_0 \cup Y)$.
    \end{case}
    In this case, by the choice of $A_0$, we have 
    \begin{align*}
        |(N_{T'}^+(a_i)\cap B)\setminus Y|&\geq |(N_{T'}^+(a_i)\cap B)|-|Y|\\
        &= |(N_{T}^+(a_i)\cap B)|-|Y|\\
        &\geq k-(k-1)\\
        &=1,
    \end{align*}
    so $a_i$ has an out-neighbour in $B\setminus Y$ in $T'-Y$, a contradiction.
%\end{proof}
    \begin{case}
        $i \in \{4k+1,\ldots,6k\}$ and $a_i \in A_0$.
    \end{case}
    In this case, by the choice of $A_0$, we have
    \begin{align*}
        |(N_{T'}^+(a_i)\cap B)\setminus Y|&\geq |N_{T'}^+(a_i)\cap B)|-|Y|\\
        &\geq|N_{T'}^+(a_i)\cap (B_0 \cup B_1)|-|Y|\\
        &=|B_0 \cup B_1|-|N_{T'}^-(a_i)\cap (B_0 \cup B_1)|-|Y|\\
        &=|B_0 \cup B_1|-|N_{T}^+(a_i)\cap (B_0 \cup B_1)|-|Y|\\
        &\geq|B_0 \cup B_1|-|N_{T'}^-(a_i)\cap B|-|Y|\\
        &\geq 2k-(k-1)-(k-1)\\
        &=2,
    \end{align*}
    so $a_i$ has an out-neighbour in $B\setminus Y$ in $T'-Y$, a contradiction.
    
    \begin{case}
        $i \in \{2k-|A_0|+1,\ldots,4k\}$.
    \end{case}
    As $(a_1,\ldots,a_{6k})$ is a median order of $T\langle A \rangle$ and by (M2) applied to $T\langle A \rangle$, we have 
     \begin{align*}
    |(N_{T'}^+(a_i)\cap \{a_{i+1},\ldots,a_{6k}\})\setminus Y|&\geq |N_{T'}^+(a_i)\cap \{a_{i+1},\ldots,a_{6k}\}|-|Y|\\
    &\geq |N_{T}^+(a_i)\cap \{a_{i+1},\ldots,a_{6k}\}|-|Y|\\
    &\geq \frac{1}{2} |\{a_{i+1},\ldots,a_{6k}\}|-|Y|\\
    & \geq \frac{1}{2} (6k-i)-(k-1)\\
    &=2k+1-\frac{i}{2}\\
    &\geq 2k+1-2k\\
    &=1.
    \end{align*}
    Hence there is some $j>i$ such that $a_j\in A\setminus Y$ and $T'-Y$ contains the arc $a_ia_j$. By the maximality of $i$, there is a directed path from $a_j$ to $B\setminus Y$ in $T'-Y$. Hence $T'-Y$ also contains a directed path from $a_i$ to $B\setminus Y$, a contradiction.
    \begin{case}
    $i \in \{1,\ldots,2k-|A_0|\}$.
    \end{case}
     As $(a_1,\ldots,a_{6k})$ is a median order of $T\langle A \rangle$ and by (M2) applied to $T\langle A \rangle$, we have 
     \begin{align*}
    |(N_{T'}^+(a_i)\cap \{a_{i+1},\ldots,a_{6k}\})-Y|&\geq |N_{T'}^+(a_i)\cap \{a_{i+1},\ldots,a_{6k}\}|-|Y|\\
    &\geq |N_{T}^+(a_i)\cap \{a_{i+1},\ldots,a_{6k}\}|-|(A_0 \cup A_1)\cap \{a_{i+1},\ldots,a_{6k}\}|-|Y|\\
    &\geq \frac{1}{2} |\{a_{i+1},\ldots,a_{6k}\}|-|(A_0 \cup A_1)\cap \{a_{i+1},\ldots,a_{6k}\}|-|Y|\\
    & \geq \frac{1}{2} (6k-i)-(2k-i)-(k-1)\\
    &=\frac{i}{2}+1\\
    &\geq 1.
    \end{align*}
    Hence there is some $j>i$ such that $a_j\in A\setminus Y$ and $T'-Y$ contains the arc $a_ia_j$. By the maximality of $i$, there is a directed path from $a_j$ to $B\setminus Y$ in $T'-Y$. Hence $T'-Y$ also contains a directed path from $a_i$ to $B\setminus Y$, a contradiction.
\end{proof}


%\nkone*
We are now ready to prove Theorem \ref{nk1}.
\begin{proof}
    Let $T$ be a tournament of order $n \geq 28k-5$ and let $(v_1,\ldots,v_n)$ be a median order of $V(T)$. Let $A=\{v_{n-6k+1},\ldots,v_n\}$ and $B=\{v_1,\ldots,v_{6k}\}$. Observe that $A$ and $B$ are disjoint as $n \geq 28k-5$. By Lemma~\ref{6ab}, there is a set $X \subseteq A \cup B$ such that in the tournament $T_0=\Inv(T\langle A \cup B \rangle,X)$, for any $Y \subseteq V(T_0)$ with $|Y|\leq k-1$, we have that $T_0-Y$ contains a directed path from $a$ to $B\setminus Y$ for every $a \in A\setminus Y$ and $T_0-Y$ contains a directed path from $A\setminus Y$ to $b$ for every $b \in B\setminus Y$. Let $T'=\Inv(T,X)$. We will show that $T'$ is $k$-strong. To this end, let $Y \subseteq V(T)$ with $|Y|\leq k-1$.

    \begin{claim}\label{nett}
        Let $b \in B\setminus Y$ and $a \in A\setminus Y$. Then $T'-Y$ contains a directed path from $b$ to $a$.
    \end{claim}
    \begin{proofclaim}
        Let $R_b$ be the set of vertices in $\{v_{6k+1},\ldots,v_{n-6k}\}$ which are reachable from $b$ in $T'-Y$ and let $R_a$ be the set of vertices in $\{v_{6k+1},\ldots,v_{n-6k}\}$ from which $a$ is reachable in $T'-Y$. By construction, there is some $i \in \{1,\ldots,6k\}$ such that $b=v_i$. Let $F=Y \cup \{v_{i+1},\ldots,v_{6k}\}$. 
        \begin{align*}
        |R_b|&= |R^+_{T'-Y}(b)\cap \{v_{6k+1},\ldots,v_{n-6k}\}|\\
             &\geq |R^+_{T'\langle\{v_i,\ldots,v_{n-6k}\}\rangle -Y} (b)\cap \{v_{6k+1},\ldots,v_{n-6k}\}|\\
             &\geq |R^+_{T'\langle\{v_i,\ldots,v_{n-6k}\}\rangle-F} (b) \setminus \{v_i\}|\\ 
            &\geq |R^+_{T'\langle\{v_i,\ldots,v_{n-6k}\}\rangle-F} (b)|-1 ~~~~~\mbox{(by Lemma \ref{median order bounds})}\\ 
            &\geq |\{v_i,\ldots,v_{n-6k}\}|-2|F|-1\\
            &\geq (n-6k-(i-1))-2((6k-i)+(k-1))-1\\
            &\geq n-20k+i+2\\
            &\geq n-20k+3.
        \end{align*}
        A similar argument shows that $|R_a|\geq n-20k+3$. As $R_a \cup R_b \subseteq \{v_{6k+1},\ldots,v_{n-6k}\}$, we obtain $|R_a \cap R_b|=|R_a|+|R_a|-|R_a\cup R_b|\geq 2(n-20k+3)-(n-12k)=n-28k+6\geq 1$. Hence there is a vertex $v^* \in \{v_{6k+1},\ldots,v_{n-6k}\}\cap R_a \cap R_b$. By definition, $T'-Y$ contains a directed path from $b$ to $v^*$ and a directed path from $v^*$ to $a$. Hence $T'-Y$ contains a directed path from $b$ to $a$.
    \end{proofclaim}
    We now show that $B\setminus Y$ is strong in $T'-Y$. Let $b,b' \in B\setminus Y$. By the definition of $X$, there is some $a \in A\setminus Y$ such that $T_0-Y$, and hence $T'-Y$, contains a directed path from $a$ to $b'$. Further, by Claim \ref{nett}, there is a path from $b$ to $a$ in $T'-Y$. Hence $T'-Y$ contains a path from $b$ to $b'$. This shows that $B\setminus Y$ is strong in $T'-Y$. Similarly, $A\setminus Y$ is strong in $T'-Y$. Next, by the choice of $X$, we have that $T_0-Y$ and hence $T'-Y$ contains a path from $A\setminus Y$ to $B\setminus Y$ and by Claim \ref{nett}, we have that $T'-Y$ contains a path  from $B\setminus Y$ to $A\setminus Y$. Hence $(A \cup B)\setminus Y$ is strong in $T'-Y$. 
    Now consider the vertex $v_i$ for some $i \in \{6k+1,n-6k\}$. By Lemma \ref{median order bounds}, we have $|R^-_{T\langle\{v_1,\ldots,v_i\}\rangle-Y}(v_i)|\geq i-2(k-1)$. As $R^-_{T\langle\{v_1,\ldots,v_i\}\rangle-Y}(v_i)\subseteq \{v_1,\ldots,v_i\}$, there is some $b \in B\setminus Y$ such that $b \in R^-_{T\langle\{v_1,\ldots,v_i\}\rangle-Y}(v_i)$. Hence $T\langle\{v_1,\ldots,v_i\}\rangle-Y$ contains a directed $(b,v_i)$-path $P$. Let $b'$ be the last vertex of this path which is contained in $B\setminus Y$. Then the $(b',v_i)$-path which is contained in $P$ also exists in $T'-Y$. Similarly, $T'-Y$ contains a directed path from $v_i$ to $A$. Hence $T'-Y$ is strongly connected. This finishes the proof.
\end{proof}





%\begin{proposition}\label{prop:N1-inf}
 % $N_k(1)\geq 5k-2$.  
%\end{proposition}
Finally, we give the proof of Proposition \ref{nk1unter}.
\begin{proof}
    Let $T$ be a tournament of order $5k-3$ whose vertex set has a partition $(A,B,C)$ such that $T\langle A \rangle$ and $T\langle C\rangle$ are $(k-1)$-diregular tournaments of order $2k-1$, and $A\Ra B\cup C$ and $B\Ra C$.
We shall prove that $\sinv'_k(T) >1$.

Assume for a contradiction that there is a set $X$ of vertices such that $T'=\Inv(T;X)$ is $k$-strong.
Every vertex of $A$ (resp. $C$) has in-degree (resp. out-degree) $k-1$ in $T$, and so belongs to $X$.
Thus $A\cup C\subseteq X$, and so $C\Ra A$ in $T'$. Hence $T'-B$ is not strong. Since $|B|=k-1$, $T'$ is not $k$-strong, a contradiction.
\end{proof}

\subsection{Better upper bound on \texorpdfstring{$N_k(6)$}{Nk(6)}}\label{nk6sec}
%%%%%%%%%%%%%%%%%%%%%%%%%%%%%%%%%%%%%%%

%\begin{lemma}\label{lem:end-path}
%Let $k$ be a positive integer, let $T$ a tournament or order $n > 4k-2$ with median order $(v_1, v_2, \ldots , v_n)$, and let 
%$F$ be set of cardinality $k-1$ in $\{v_2, \ldots , v_{n-1}\}$.
%In $T-F$, there is a directed $(v_1, v_n)$-path.
%\end{lemma}
%\begin{proof}
%\sloppy By Lemma~\ref{median order bounds}, $|R^+_{T-F}(v_1)|$ and $|R^-_{T-F}(v_n)|$ are at least $n - 2k+2$.
%Since~$n>4k-2$, we have $|R^+_{T-F}(v_1)| + |R^-_{T-F}(v_n)| > n$, so $R^+_{T-F}(v_1)$ and  $R^-_{T-F}(v_n)$ intersect, which implies that $v_n\in R^+_{T-F}(v_1)$.
%In other words, there is a directed $(v_1, v_n)$-path in $T-F$.
%\end{proof}

\begin{lemma}\label{lem:A-B}
Let $k$ be a positive integer.
Let $T$ be a tournament with $(A,B)$ a bipartition of $T$ with $|A| = |B| = 5k$.
There is a family ${\cal X}$ of at most six sets such that the following holds with $T_1= \Inv(T ; {\cal X})$, 
\begin{itemize}
\item[(i)] $T_1\langle A \rangle = T\langle A\rangle$ and $T_1\langle B \rangle = T\langle B\rangle$ ; 
\item[(ii)] in $T_1$, every vertex of $A$ has at least $k$ out-neighbours in $B$ ; 
\item[(iii)] in $T_1$, every vertex of $B$ has at least $k$ in-neighbours in $A$.
\end{itemize}
\end{lemma}
\begin{proof}
For every pair $S_1,S_2$ of set of vertices, we denote by $a(S_1,S_2)$ the number of arcs in $T$ with tail in $S_1$ and head in $S_2$.

Let $X$ be the set of vertices of $A$ having less than $\frac{5}{2}k$ out-neighbours in $B$.
Let $T_0= \Inv(T ; (X\cup B, X, B))$.
We have $T_0\langle A \rangle = T\langle A\rangle$, $T_0\langle B \rangle = T\langle B\rangle$, and in $T_0$ every vertex of $A$ has at least $\frac{5k}{2}$ out-neighbours in $B$. 
In particular $a(A,B) \geq \frac{25k^2}{2}$ and so $a(B,A) \leq \frac{25k^2}{2}$.
Let $B_0$ be the set of vertices having at least $k$ in-neighbours in $A$, and set $B_1=B\setminus B_0$.
Every vertex of $B_1$ has at least $4k$ out-neighbours in $A$.
Hence $|B_1| \leq \frac{a(B_1, A)}{4k} \leq  \frac{a(B, A)}{4k}\leq \frac{25k}{8}$.

Let $A_0$ be the set of vertices of $A$ having at least $k$ out-neighbours in $B_0$, and set $A_1=A\setminus A_0$.
Every vertex of $A_1$ has at least $\frac{5k}{2}-k = \frac{3k}{2}$ out-neighbours in $B_1$.
Now $a(A_1, B_1) \leq k|B_1|$ arcs between $A_1$ and $B_1$ so $|A_1| \leq \frac{2k|B_1|} {3k} \leq \frac{25k}{12}$.

Let $T_1= \Inv(T_0 ; (A_0\cup B_1, A_0, B_1)) = \Inv(T; (X\cup B, X, B, A_0\cup B_1, A_0, B_1))$.
Clearly, $T_1\langle A \rangle =T_0\langle A \rangle = T\langle A\rangle$, $T_1\langle B \rangle= T_0\langle B \rangle = T\langle B\rangle$.
So (i) holds.
Let us now prove that (ii) and (iii) also hold.

\begin{itemize}
\item Let $a$ be a vertex in $A$.
If $a\in A_0$, then $|N^+_{T_1}(a) \cap B_0| = |N^+_{T_0}(a) \cap B_0| \geq k$. 
If $a\in A_1$, then $|N^+_{T_1}(a) \cap B| = |N^+_{T_0}(a) \cap B| \geq \frac{5k}{2}$. 
This proves (ii).
\item Let $b$ be a vertex in $B$.
If $b\in B_0$, then $|N^-_{T_1}(b) \cap A| = |N^-_{T_0}(b) \cap A| \geq k$ by definition of $B_0$. 
If $b\in B_1$, then $|N^-_{T_1}(b) \cap A| \geq |N^+_{T_0}(b) \cap A_0| \geq  |N^+_{T_0}(b) \cap A| - |A_1| \geq 4k - 25k/12  > k$.
This proves (iii).
\end{itemize}
\end{proof}




\nksix*
\begin{proof}
Assume $n\geq 14 k-3$.
Let $(v_1, \ldots , v_n)$ be a median order of $T$.
Let $B=\{v_1, \dots , v_{5k}\}$, $A=\{v_{n-5k+1}, \ldots , v_{n}\}$ and $C=V(T)\setminus (A\cup B)$. We have $|C| \geq 4k-3$.
By Lemma~\ref{lem:A-B}, applied to $T\langle A\cup B\rangle$, there is a family  ${\cal X}$ of at most six subsets of $A\cup B$ such that
for $T_1 = \Inv(T;\mathcal{X})$ we have
\begin{itemize}
\item[(i)] $T_1\langle A \rangle = T\langle A\rangle$ and $T_1\langle B \rangle = T\langle B\rangle$ ; 
\item[(ii)] in $T_1$, every vertex of $A$ has at least $k$ out-neighbours in $B$ ; 
\item[(iii)] in $T_1$, every vertex of $B$ has at least $k$ in-neighbours in $A$.
\end{itemize}


Let us now prove that $T_1$ is $k$-strong, which implies $\sinv_k(T) \leq 6$.
Note that $T\langle A \cup C \rangle$, as well as $T\langle B \cup C \rangle$ are unchanged by the inversions.
Let $F$ be a set of $k-1$ vertices of $T_1$. Let us show that $T_1-F$ is strong.
Let $x=v_{i_1}$ and $y=v_{i_2}$ be two vertices in $T_1-F$.
As $|C| \geq 4k-3$, if $x \in B$, Lemma~\ref{lem:n-2F} asserts the existence of a path from $x$ to $c \in C$. Furthermore, Lemma~\ref{lem:n-2F} asserts the existence of a (possibly empty) path from any vertex of $A \cup C$ to a vertex of $A$ in $T_1$. Thus there exists $x'$, a vertex of $A$ reachable from $x$ by a path $P_x$. This vertex $x'$ has at least $k$ out-neighbours in $B$, so at least one of them, say $u$, is in $B\setminus F$.
Similarly, by directional duality, in $T_1-F$, there is a directed path $P_y$ (possibly empty) from a vertex $y'\in B$ to $y$.
This vertex $y'$ has at least $k$ in-neighbours in $A$, so at least one of them, say $w$, is in $A\setminus F$.
By Lemma~\ref{lem:n-2F}, $|R^+_{B \cup C - F}(u) \cap C| \geq |C|-2(k-1) > |C|/2$, and $|R^-_{A \cup C - F}(w) \cap C|  >|C|/2$. Thus there is a vertex of $C$ in $R^+_{B \cup C - F}(u)\cap R^-_{A \cup C - F}(w)$, and so there exists a path $P_{u, w}$ from $u$ to $w$ in $T_1 - F$.
$P_x P_{u, w} P_y$ is then a path from $x$ to $y$.
\end{proof}







All the results of the previous subsections imply the following.
\begin{corollary}\label{cor:m_k-upper}
$m_k(n) \leq \left\{ \begin{array}{ll}
      2k,&\text{if~ $ 2k+1 \leq n \leq 14k - 3$,}\\
      6,&\text{if~ $ 14k - 3 < n < 28k - 5$,}\\
      1,&\text{if~  $ n\geq 28k - 5$.}
  
    \end{array}\right.
    $
\end{corollary}


\begin{problem}
Find better upper bounds on $m_k(n)$. 
\end{problem}



\subsection{Upper bounds for \texorpdfstring{$k$}{k} large.}\label{epsgross}
%%%%%%%%%%%%%%%%%%%%%%%%%%%%%%%%%%
% \FHO{J'ai lu cette demonstration et je suis a peu pres convaincu que l'idee marche, mais c'est quand-meme tres dure a lire parce qu'il y a beaucoup de notations. Ce serait bien, si on arriverarit a couper la demonstration un peu. Je commencerais par un claim a peu pres de la forme suivante:
% \begin{claim}
%     If $T'$ is not $k$-strong, then one of the following holds:
% \begin{itemize}
%     \item there is a vector $z \in \mathbb{F}_2^t \setminus \{\vec{0}\}$ such that $|\{v \in V(T)|\vec{v} \neq z\}|\leq k$,
%     \item there are $u,v \in V(T)$ with $\vec{u} \neq\vec{v} $ such that $\min\{|N_{T'}^+(u)\cap N_{T'}^-(v)|,|N_{T'}^+(u)\cap N_{T'}^+(v)|,|N_{T'}^-(u)\cap N_{T'}^-(v)|\leq \frac{k+\epsilon}{2}$,
%     \item there are sets $A,B \subseteq V(T')$ with $|A|,|B|=\frac{k+\epsilon}{2}$ such that the underlying of $(A \cup B, \delta_{T'}(A,B))$ does not contain a matching of size $\frac{k}{2}$.
% \end{itemize}
% \end{claim}
% Apres, on pourrait mettre des bornes pour les 3 evenements et conclure.}
% \CR{J'ai implemente ta remarque}\FHO{Super, merci. Je regarderai demain.}
In this section, we show that if a tournament has at least $2k+1+\epsilon k$ vertices for some positive integer $k$ and some $\epsilon>0$, then it can be made $k$-strong by inversing a family of sets whose cardinality only depends on $\epsilon$.
The proof consists in drawing this family uniformly at random, under the constraint that every vertex is contained in at least one of the sets. To analyse this procedure we will need the celebrated Chernoff's bound.
%\FHO{On devrait peut-etre ajouter une referencea un livre qui contient des bases sur la methode probabiliste, y compris la borne de Chernoff.}


\
\begin{lemma}[Chernoff's bound]\label{chern}
    If $X$ is a random variable following a binomial law with parameters $p \in [0,1]$ and $n \geq 0$, then
    for every $\epsilon \in [0,1]$
    \[
    \Pr[X \geq (1+\epsilon)pn] \leq e^{-\frac{\epsilon^2 }{3} pn}.
    \]
    and
    \[
    \Pr[X \leq (1-\epsilon)pn] \leq e^{-\frac{\epsilon^2}{2} pn}.
    \]
\end{lemma}
We refer the reader to~\cite[Part II, Section 5]{molloy2002graph} for an introduction to the probabilistic method, including a proof of this bound.
We will also need the two following technical lemmas.

\begin{lemma}\label{lemma:proba_2}
Let $\vec{u} \neq \vec{v} \in \mathbb{F}_2^t \setminus \{\vec{0}\}$ and $x,y \in \mathbb{F}_2$ be fixed, and let $\vec{w} \in \mathbb{F}_2^t \setminus \{\vec{0}\}$ 
be drawn uniformly at random. Then $\Pr[\vec{u} \cdot \vec{w} = x, \vec{v} \cdot \vec{w}=y] \geq \frac{1}{4}-\frac{3}{4}\frac{1}{2^t-1}$.
\end{lemma}

\begin{proof}
As $\vec{u}\neq \vec{v}$, the mapping $\mathbb{F}_2^t \to \mathbb{F}_2^2, \vec{w} \mapsto (\vec{u} \cdot \vec{w}, \vec{v} \cdot \vec{w})$ is surjective and linear.
As a consequence, there are $\frac{1}{4}2^t$ vectors $\vec{w} \in \mathbb{F}_2^t$ which satisfy $\vec{u} \cdot \vec{w}=x$ and $\vec{v} \cdot \vec{w} = y$.
Thus by possibly removing the solution $\vec{w}=0$, we obtain $\Pr[\vec{u} \cdot \vec{w} = x, \vec{v} \cdot \vec{w}=y] \geq \frac{2^{t-2}-1}{2^t-1} =
\frac{1}{4}-\frac{3}{4}\frac{1}{2^t-1}$.
\end{proof}


\begin{lemma}\label{lemma:proba_3}
    Let $\epsilon >0$, let $t \geq 16$ be an integer, and let $k \geq \frac{8t}{\epsilon}$ be an integer.
    Let $U,V \in (\mathbb{F}_2^{t} \setminus\{\vec{0}\})^{\lceil\epsilon k/8\rceil}$ be drawn uniformly at random and 
    $W \in \mathbb{F}_2^{\lceil\epsilon k/8\rceil \times \lceil\epsilon k/8\rceil}$ be fixed.
    Then $\Pr[U^\top \cdot V = W] \leq 2^{-t\epsilon k/128}$.
\end{lemma}
% \FHO{Il me semble qu'il manque une condition dans ce lemme. Si $\lfloor \epsilon k/4\rfloor>t/2+1$, alors le premier coefficient binomial n'est pas defini et en plus on a clairement $\Pr[\rk(U) \leq t/2]=1$ si $U$ a moins que $t/2$ colonnes.}
% \CR{Oui tu as raison, on a un pb st $\lfloor \epsilon k/4\rfloor \leq t/2$.
% Je renforce l'hypothese en $k \geq \frac{8t}{\epsilon}$. Ecrites comme je viens de faire les conditions pourraient etre plus precises, mais ca change rien dans la suite.}
% \CR{J'ai du remplacer $\epsilon k/4$ par $\epsilon k /8$ pour pouvoir l'appliquer dans la suite. J'espère que je n'ai pas cree de nouvelles erreurs de calculs! Aussi il suffit d'avoir $\lceil$ au lieu de $\lfloor$ donc ca simplifie un peu certain calculs.}
\begin{proof}
    Note that since $k \geq \frac{8t}{\epsilon}$, we have 
    %$\lfloor\epsilon k/4\rfloor \geq \frac{\epsilon k}{8}$, and
    $\lceil \epsilon k/8 \rceil \geq t \geq t/2+1$.
    First we bound the probability that $\rk(U) < t/2+1$. 
    If $U$ has rank at most $t/2$, then there is a choice of $\lfloor t/2\rfloor$ columns of $U$ such that all the other ones are in the linear span of these selected columns.
    Since the linear span of $\lfloor t/2\rfloor$  vectors has dimension at most $t/2$, and so size at most $2^{t/2}$, we deduce the following.
    % \[
    % \begin{split}
    %     \Pr[\rk(U) \leq t/2] &\leq \binom{\lfloor \epsilon k/8\rfloor}{\lfloor t/2\rfloor}\left(\frac{2^{\lfloor t/2\rfloor}-1}{2^t -1} \right)^{\lfloor \epsilon k/4 \rfloor -t/2} \\
    %     &\leq \binom{\lfloor\epsilon k/4\rfloor}{\lfloor t/2\rfloor}\left(\frac{2^{t/2}-1}{2^t -1} \right)^{\epsilon k/8-t/2} \\
    %     &\leq 2^{\epsilon k/8}\left(\frac{2^{t/2}-1}{2^t -1} \right)^{\epsilon k/16} \\
    %     &\leq 2^{\epsilon k/8} (2^{t/2})^{-\epsilon k/16} \\
    %     &\leq 2^{\epsilon k/8} 2^{-\epsilon k t/32} \\
    %     &\leq 2^{- t \epsilon k/32} \\
    % \end{split}    
    % \]
    % since $t \geq 8$.
    
% \FHO{Je ne vois pas du tout comment tu arrives a la troisieme ligne a partir de la deuxieme. Aussi, d'ou vient le +1 dans le coefficient binomial?}
% \CR{Le $+1$ etait une erreur. Ah oui j'ai zappé le $-t/2$!!! pardon, c'est corrige bientot.
% Je viens aussi de rajouter les arrondis pour le cas $t$ impair.}
% \FHO{je crois que quand tu passes de la deuxieme a la troisieme ligne, il y a une estimation qui va dans le mauvais sense: genre: $\binom{\lfloor\epsilon k/4\rfloor}{\lfloor t/2\rfloor}\leq 2^{\lfloor\epsilon k/4\rfloor}\leq 2^{\epsilon k/8}$ et la derniere inegalite n'est pas juste}
% \JD{Je suis d'accord avec Flo, je propose la correction en dessous avec $t > 16$: . }\CR{Oui c'est vrai, merci pour la correction.}
% \CR{Finalement j'ai du encore changer pour arriver a $\lceil \epsilon k/8 \rceil$ au lieu de $\lfloor \epsilon k/4 \rfloor$.}\FHO{Je ne crois pas que l'estimation $\epsilon k/8-t/2\leq \epsilon k/16$ entre la deuxieme et la troisieme ligne est correcte. Aussi, ou utilise-t-on que $k \geq 16$?}
% \CR{Je crois que c'est bon: $k \geq 8t/\epsilon$ donc $t/2 \leq \epsilon k/16$. Le $t \geq 16$ est utilisé pour obtenir l'avant derniere ligne.}\FHO{Pardon, tu as raison. Il faudrait peut-etre mentionner ca explicitement en haut. Par contre, le probleme en fin de la demonstration persiste.}
    \[
    \everymath={\displaystyle}
    \renewcommand{\arraystretch}{2.5}
    \begin{array}{r l l}
        \Pr\left[\rk(U) \leq t/2\right] &\leq
        \binom{\lceil \epsilon k/8 \rceil}{\lfloor t/2\rfloor}\left(\frac{2^{\lfloor t/2\rfloor}-1}{2^t -1} \right)^{\lceil \epsilon k/8 \rceil -\lfloor t/2\rfloor } &\\
        &\leq \binom{\lceil\epsilon k/8\rceil}{\lfloor t/2\rfloor}\left(\frac{2^{t/2}-1}{2^t -1} \right)^{\epsilon k/8-t/2} & \\
        &\leq 2^{\epsilon k/8+1}\left(\frac{2^{t/2}-1}{2^t -1} \right)^{\epsilon k/16} & \text{ because $\frac{t}{2} \leq \frac{\epsilon k}{16}$ since } k\geq \frac{8t}{\epsilon} \\
        &\leq 2^{\epsilon k/4} (2^{t/2})^{-\epsilon k/16} & \\
        &\leq 2^{\epsilon k t/64} 2^{-\epsilon k t/32} & \text{ because } t \geq 16 \\
        &= 2^{- \epsilon k t/64} & \\
    \end{array}    
    \] 
    Now we assume that $\rk(U)> t/2$.
    Then for every column $v$ of $V$, $v$ must be chosen in an affine space of dimension at most $t - \lfloor t/2\rfloor -1 \leq t/2$.
    It follows that
    \[
    \begin{split}
        \Pr[U^\top \cdot V = W \mid \rk(U) > t/2] & \leq 
        \left(\frac{2^{t/2}-1}{2^{t}-1} \right)^{\lceil\epsilon k/8 \rceil} \\
        &\leq (2^{-t/2})^{\epsilon k/8} \\
        &\leq 2^{-t\epsilon k/16} \\
    \end{split}
    \]
    Therefore
    \[
    \begin{split}
        \Pr[U^\top \cdot V = W] & \leq \Pr\left[\rk(U) \leq t/2\right] + \Pr[U^\top \cdot V = W \mid \rk(U) > t/2]\\
        &\leq 2^{-t\epsilon k/16} + 2^{-t \epsilon k/64} \leq 2\cdot 2^{-t \epsilon k/64}%=2^{-t \epsilon k/64+1}.\\\leq 2^{-t\epsilon k/128}\\
    \end{split}
    \]
%     \FHO{as $t \epsilon k\geq 8 t^2 \geq 128$.}
% %    \FHO{ on a calcule $2^{-t\epsilon k/64}$ avant et on utilise $2^{-t\epsilon k/32}$  ici.}
%     as $2^{-t\epsilon k/128} \leq 2^{-t^2/16} \leq \frac{1}{2}$ \FHO{d'ou vient le $t^2$? A mon avis, on a juste $t\epsilon k \geq 128$ ce qui implique resultat}since $k \geq \frac{8t}{\epsilon}$ and $t \geq 4$, and because $2x^2 \leq x$ for every $0 \leq x \leq 1/2$.
%     \CR{J'ai juste remplacé $k$ par $8t/\epsilon$.}\FHO{et comment on utilse $2^{-t\epsilon k/128}\leq \ldots$ pour demontrer $2^{-t\epsilon k/128}\geq \ldots$?}\CR{Oui c'etait pas clair!! Mais ta façon de faire le calcul est surement plus facile.}
%     \FHO{Ca te va comme ca?}
%     \JD{Histoire de trancher, je vous propose:}\CR{J'achete.} 
    We know that $t\epsilon k \geq 8 t^2 \geq 2 \cdot 64$, thus $2^{-t \epsilon k / 64} \leq 1/4$. As $2x \leq \sqrt{x}$ for any $x \in [0, 1/4]$, we end with $\Pr[U^\top \cdot V = W]  \leq 2 \cdot 2^{-t\epsilon k/64} \leq 2^{-t\epsilon k/128}$.
\end{proof}
For technical reasons, we prove the following seemingly weaker restatement of Theorem \ref{thm:2+eps}.

%\begin{restatable}{theorem}{pluseps}
\begin{theorem}\label{thm:2+eps+2}
    There exists a function $f: \mathbb{R}_{>0} \to \mathbb{N}$ such that for every $\epsilon>0$ and every positive integer $k$, 
    if $T$ is an $n$-vertex tournament with $n \geq 2k+ 2 \epsilon k$, then $\sinv_k(T) \leq f(\epsilon)$.
%\end{restatable}
\end{theorem}
%\pluseps*

It is not difficult to see that Theorem~\ref{thm:2+eps+2} actually implies Theorem~\ref{thm:2+eps}. Indeed, given a function $f$ like in Theorem~\ref{thm:2+eps+2} at hand, define $f':\mathbb{R}_{>0} \to \mathbb{N}$ by $f'(\epsilon)=\max\{\frac{4}{\epsilon},f(\frac{\epsilon}{2})\}$. Let $T$ be a tournament with $|V(T)|\geq 2k+1 +\epsilon k$ for some positive integer $k$. If $k \leq \frac{2}{\epsilon}$, then Theorem \ref{thm:M<2k} yields $\sinv_k(T)\leq 2k\leq \frac{4}{\epsilon}\leq f'(\epsilon)$. Otherwise, we have $|V(T)|\geq 2k+1+\epsilon k\geq 2k+2 + \frac{\epsilon}{2}k$, so $\sinv_k(T)\leq f(\frac{\epsilon}{2})\leq f'(\epsilon)$ by Theorem~\ref{thm:2+eps+2}.



\begin{proof}
% \FHO{je crois que ce serait beaucoup plus claire si on arriverait de definir avant la demonstration. Pas forcement de maniere explicte, mais de la facon qu'on dit: Soit $f(e)$ un entier tel que les inegalites suivantes sont satisfaites:.... Apres, on peut supposer que $n \geq f(\epsilon)$ et ce serait beaucoup plus claire apres que le choix de f depend pas de k.}

Without loss of generality, we may assume $\epsilon \leq \frac{1}{3}$.
Let $C$ be a constant such that $\sinv_k(T) \leq 1$ if $n \geq Ck$, which exists by Theorem~\ref{nk1}. %Clearly, we can assume that $n \leq Ck$.
Let $t$ be the smallest integer such that 
\begin{itemize}
    \item $t\geq 16$, 
    \item $t \geq \log(1+\frac{48}{\epsilon})$, and
    \item $t \geq \frac{128}{\epsilon}(2C+2+\epsilon/{\color{blue}4}) +16$.
\end{itemize}
Clearly, $t$ is well defined and depends only on $\epsilon$.
Let $k_0(\epsilon)$ be the smallest integer such that for every $k' \geq k_0(\epsilon)$ 
\begin{equation}\label{eq:condition_k_large_enough}
(2^t-1) \exp\left(-\epsilon^2\frac{(2+\epsilon)k'}{24}\right) + 3(Ck')^2\exp\left(-\epsilon^2\frac{(2+\epsilon)k'}{4096}\right) + 2^{-k'} <1.
\end{equation}

% \FHO{Ici, ton choix de $k_0(\epsilon)$ depend de $t$. Je ne crois pas que c'est bon, parce que tu dois choisir $t$ dependant de $\epsilon$ et puis il faut que ca marche pour chaque $k$.}
% \CR{Ok, j'ai modifier un peu tout ca. Maintenant on defini $t$ tel que toutes les inegualite necessaire a la suite soient vraies. Puis on definit $k_0$ en utilisant $t$. Si $k<k_0$, on a $\sinv_k \leq 2k_0$ et on est bon, sinon on commence la preuve.}
% \FHO{ok, je crois que ca devrait marcher. Il faut donc enfin dire que $f(\epsilon)=\max\{t,2k_0(\epsilon)-2\}$.}

%If $k < k_0(\epsilon)$, then we conclude directly using Theorem~\ref{thm:M<2k} that $\sinv_k(T) \leq 2k_0-2$.
%Similarly, if $k \leq \frac{8t}{\epsilon}-1$, then we conclude by Theorem~\ref{thm:M<2k} that $\sinv_k(T) \leq \frac{16}{\epsilon}-2$.
%Now we assume $k \geq \frac{8t}{\epsilon} \geq \frac{8}{\epsilon} \geq 24$ and that \eqref{eq:condition_k_large_enough} holds, and we will show that $\sinv_k(T) \leq t$.
%This will implies the theorem by taking $f(\epsilon) = \max\{t,k_0(\epsilon),\frac{8t}{\epsilon}\}$.
%Note that $k \geq \frac{8t}{\epsilon} \geq \frac{8}{\epsilon}$.

We now prove the statement for $f(\epsilon)=\max\{t,2k_0(\epsilon)-2,\lceil\frac{16t}{\epsilon}\rceil-2\}$. If $k < k_0(\epsilon)$, then we conclude directly using Theorem~\ref{thm:M<2k} that $\sinv_k(T) \leq 2k\leq 2k_0(\epsilon)-2\leq f(\epsilon)$. Similarly, if $k \leq \frac{8t}{\epsilon}-1$, then we conclude by Theorem~\ref{thm:M<2k} that $\sinv_k(T) \leq \frac{16t}{\epsilon}-2\leq f(\epsilon)$. 
 Moreover, if $n \geq Ck$, we have $\sinv_k(T)\leq 1 \leq f(\epsilon)$.
Henceforth, we may assume $k\geq \max\{k_0(\epsilon), \frac{8t}{\epsilon}-1\}$ and $n \leq Ck$.
\medskip



%\FHO{If $n\leq 17$, then by Theorem \ref{thm:M<2k}, we have $\sinv_k(T)\leq 16 \leq f(\epsilon)$. We may hence assume that $n \geq 18$.}



For every vertex $u \in V(T)$, we choose uniformly and independently at random a vector $\vec{u} \in \mathbb{F}_2^t \setminus \{\vec{0}\}$.
For $i\in [t]$, let $X_i = \{u \in V(T) \mid \vec{u}_i =1\}$.
We will prove that with positive probability, the tournament $T' = \Inv(T; X_1, \dots, X_t)$ is $k$-strong.
Note that for every arc $uv \in A(T)$, we have $uv\in A(T')$ if and only if $\vec{u} \cdot \vec{v}=0 \mod 2$.
%and otherwise $vu$ is in $T'$.
For two subsets $A$ and $B$ of vertices of $T'$, a {\bf directed $(A,B)$-matching} is a set of arcs with tails in $A$, heads in $B$, and without common tail or common head.
\begin{claim}\label{claim:decompose_into_events_A_B_C_}
    If $T'$ is not $k$-strong, then at least one of the following events occurs:
    \begin{enumerate}[label=\Alph*]
    \item[$E_1$]: there is a vector $z \in \mathbb{F}_2^t \setminus \{\vec{0}\}$ such that $|\{v \in V(T)\mid \vec{v} \neq z\}|\leq k$,
    \item[$E_2$]: there are $u,v \in V(T)$ with $\vec{u} \neq\vec{v} $ such that $\min\{|N_{T'}^+(u)\cap N_{T'}^-(v)|,|N_{T'}^+(u)\cap N_{T'}^+(v)|,|N_{T'}^-(u)\cap N_{T'}^-(v)|\}\leq (1+\epsilon/4)\frac{k}{2}$,
    \item[$E_3$]: there are sets $A,B \subseteq V(T')$ with $|A|,|B| \geq (1+\epsilon/4)\frac{k}{2}$ 
    with no directed $(A,B)$-matching of size $\frac{k}{2}$.
    %such that the underlying graph of $(A \cup B, \delta_{T'}(A,B))$ does not contain a matching of size $\frac{k}{2}$.
    \end{enumerate}
\end{claim}

\begin{proofclaim}
    Assume that none of~$E_1$,~$E_2$ and~$E_3$ holds.
    Suppose for a contradiction that there is a set $X$ of at most $k-1$ vertices, and a partition $(V_1,V_2)$ of $V(T'-X)$ into nonempty sets such that $V_2 \Rightarrow V_1$ in $T'-X$.
    Since~$E_1$ does not hold, there exist $x,y\in V_1 \cup V_2$  with $\vec{x} \neq \vec{y}$. If both $x$ and $y$ are in $V_1$ (resp. $V_2$), consider $v \in V_2$ (resp. $u \in V_1$) and either $\vec{x}\neq\vec{v}$ or $\vec{y}\neq\vec{v}$ (resp. $\vec{x}\neq\vec{u}$ or $\vec{y}\neq\vec{u}$).
    If $x \in V_1$ and $y \in V_2$ we set $u=x$ and $v=y$,
    and if $y \in V_2$ and $x \in V_1$ we set $u=y$ and $v=x$.
    In all cases, there are $u\in V_1$ and $v \in V_2$ with $\vec{u} \neq \vec{v}$.

    Now, as ~$E_2$ does not hold, we have $|N_{T'}^+(u)\cap N_{T'}^-(v)|,|N_{T'}^+(u)\cap N_{T'}^+(v)|,|N_{T'}^-(u)\cap N_{T'}^-(v)|\geq(1+\epsilon/4)\frac{k}{2}$.
    Finally, as~$E_3$ does not hold, 
    there is a directed $(N_{T'}^+(u)\cap N_{T'}^+(v), N_{T'}^-(u)\cap N_{T'}^-(v))$-matching $M$ of size $k/2$ in $T'$.
    For every arc $e=xy\in M$, observe that $P_e=uxyv$ is a directed $(u,v)$-path in $T'$. 
    Furthermore, for every $x \in N_{T'}^+(u)\cap N_{T'}^-(v)$, observe that $P_x=uxv$ is a directed $(u,v)$-path in $T'$. 
    This yields a collection of at least $k$ internally vertex-disjoint $(u,v)$-paths in $T'$,
    a contradiction since every $(u,v)$-path meets $X$ which has size at most $k-1$.
\end{proofclaim}

We will show that with high probability none of the events ~$E_1$,~$E_2$ and~$E_3$ occurs.


\begin{claim}\label{claim:proba_A}
$\Pr(E_1) \leq (2^t-1) \exp(-\epsilon^2n/24)$
\end{claim}

\begin{proofclaim}
If $\vec{c} \in \mathbb{F}_2^t \setminus\{\vec{0}\}$ is fixed, then $Y_{\vec{c}} = |\{u \in V(T) \mid \vec{u} \neq \vec{c}\}|$ is a random variable 
having a binomial law with parameters $n$ and $1-\frac{1}{2^t-1}$. 
As
%$n \geq 2k \geq 48$\FHO{c'est utilise ou?} \JD{Pas utile je crois.}
$\epsilon  \leq \frac{1}{3}$ and $t\geq 2$, we have $k \leq \frac{1}{2}n\leq \frac{2}{3}\cdot\frac{5}{6}n\leq(1-\frac{1}{2^t-1})(1-\epsilon/2)n$. By Lemma~\ref{chern} (Chernoff's bound), and because $t \geq 2$, we have
\begin{align*}
%\Pr[\exists X \subseteq V(T), |X| < k, |\{\vec{u} \mid u \not\in X\}| \leq 1] &= 
\Pr[\exists \vec{c} \in \mathbb{F}_2^t\setminus\{\vec{0}\}, Y_{\vec{c}} < k] 
&\leq \sum_{\vec{c} \in \mathbb{F}_2^t\setminus\{0\}} \Pr[Y_{\vec{c}} < k] \\
&\leq \sum_{\vec{c} \in \mathbb{F}_2^t\setminus\{0\}} \Pr\left[Y_{\vec{c}} < \left(1-\frac{\epsilon}{2}\right)\left(1-\frac{1}{2^t-1}\right)n\right] \\
&\leq (2^t-1)\exp\left(-(\epsilon/2)^2\left(1-\frac{1}{2^t-1}\right)n/3\right) \\
&\leq (2^t-1)\exp(-\epsilon^2 n/24),
\end{align*}
as claimed.
\end{proofclaim}



% Let $s,t$ be two distinct vertices, and let $C = N^+_{T'}(s) \cap N^-_{T'}(t), A = N^+_{T'}(s) \setminus C, B = N^-_{T'}(t) \setminus C$.
% We assume that $\vec{s} \neq \vec{t}$ (otherwise Claim~\ref{claim:proba_1} applies).
% Observe that if there is a matching of size $k-|C|$ from $A$ to $B$, then together with vertices in $C$, this gives $k$ vertex-disjoint
% $(s,t)$-paths in $T'$. We will show that this happens with high probability.

% \begin{claim}
%     $\Pr[\min(|A|,|B|,|C|) \leq (1+\epsilon/2)k/2] \leq 3\exp\left(-\epsilon^2\frac{n-2}{192}\right)$
% \end{claim}

\begin{claim}\label{claim:proba_B}
    $\Pr(E_2) \leq 3\exp\left(-\epsilon^2\frac{n-2}{4096}\right)$
\end{claim}

\begin{proofclaim}
    Let $u,v$ be distinct vertices and let
    $A = N_{T'}^+(u)\cap N_{T'}^-(v), B = N_{T'}^+(u)\cap N_{T'}^+(v)$ and $C=N_{T'}^-(u)\cap N_{T'}^-(v)$.
    Let $X \in \{A,B,C\}$.
    Once $\vec{u}$ and $\vec{v}$ revealed, for every vertex $w \neq u,v$, let $Y_w$ be a random variable with $Y_w=1$ if $w \in X$, $Y_w=0$ otherwise.
    By Lemma~\ref{lemma:proba_2}, $Y_w$ is a random variable following a Bernoulli distribution whose parameter is at least $\frac{1}{4}(1-\frac{3}{2^t-1})$. Further, the $Y_w$ are mutually independent.
%    \FHO{Ici, je ne suis pas sure. Qu'est-ce qui est la definition exacte de $X_w$? Je crois qu'une fois $\vec{u}$ et $\vec{v}$ sont fixes, on connait la distribution exacte de $Y_w$ et les $Y_w$ sont effectivement independents. Apres, je vois deux solutions possibles: On utilise une forme plus generale de Chernoff qui prend en compte plusieurs variables, c'est probablement moche. Sinon, pour les $Y_w$ tel que la probabilite $p_w>\frac{1}{2}-\frac{3}{4}\frac{1}{2^t-1}$, on definit $X_w$ de maniere qu'on prend le resultat de $Y_w$ et le multiplie par 0 avec probabilite $\frac{p_w}{\frac{1}{2}-\frac{3}{4}\frac{1}{2^t-1}}$ independemment. Apres, je crois que tous les $X_w$ restent independents et ont la distribution souhaitee, donc on peut appliquer Chernoff.} 
%    \CR{Oui on peut faire ca. C'est pas très compliqué en fait.}

    We now define a random variable $X_w$ for every $w \in V(T)\setminus \{u,v\}$ as follows.
    If $Y_w=0$, set $X_w=0$, and otherwise set $X_w=1$ with probability $\frac{\frac{1}{4}(1-\frac{3}{2^t-1})}{\Pr[Y_w=1]}$ and $X_w=0$ otherwise, where the latter random experiments are executed independently. Observe that, as the $Y_w$ are mutually independent, so are the $X_w$. Moreover, $\Pr[X_w=1] = \frac{1}{4}(1-\frac{3}{2^t-1})$.
    
    We have $(1+\epsilon/4)k/2 \leq \frac{1+\epsilon/4}{2+\epsilon}(n-2)/2 = \frac{1}{4}(1-\frac{\epsilon/4}{1+\epsilon/2}) (n-2)/2 \leq \frac{1}{4}(1-\epsilon/8)(n-2)$ since $n \geq (2+\epsilon)k+2$.
    %\FHO{L'egalite du deuxieme et troisieme terme n'est pas correct}.
    Moreover, as $t \geq  \log(\frac{48}{\epsilon}+1)$
    %\FHO{je crois ca devrait etre $t \geq \log(1+\frac{24}{\epsilon})$}
    we have 
    $\frac{1}{4}(1-\epsilon/8)\leq \frac{1}{4}(1-\epsilon/16)^2 \leq \frac{1}{4}\left(1-\frac{3}{2^t-1}\right)(1-\frac{\epsilon}{16}) = \left(\frac{1}{4}-\frac{3}{4(2^t-1)}\right)(1-\epsilon/16)$.
    Hence $(1+\epsilon/4)k/2 \leq (1-\epsilon/16)\left(\frac{1}{4}-\frac{3}{4(2^t-1)}\right)(n-2)$, and by Chernoff's bound (Lemma~\ref{chern})
     \begin{align*}
        \Pr[|X| \leq (1+\epsilon/4)k/2]
        & \leq \Pr\left[|X| \leq \left(1-\frac{\epsilon}{16}\right)\left(\frac{1}{4}-\frac{3}{4(2^t-1)}\right)(n-2)\right]\\
        &\leq \exp\left(-(\epsilon/16)^2\left(\frac{1}{4}-\frac{3}{4(2^t-1)}\right)\frac{n-2}{2}\right)\\    
        &\leq \exp\left(-\epsilon^2\frac{n-2}{4096}\right) 
   \end{align*}
    since $t \geq 5$ implies $\frac{3}{2^t-1} \leq 1/8$. 
    Hence by the union bound $\Pr[\min\{|A|,|B|,|C|\} \leq (1+\epsilon/4)\frac{k}{2}] \leq \sum_{X \in \{A,B,C\}}\Pr[|X| \leq (1+\epsilon/4)\frac{k}{2}]\leq 3\exp\left(-\epsilon^2\frac{n-2}{4096}\right)$.
\end{proofclaim}

For two disjoint sets of vertices $X,Y$ in $T'$, we denote by $\mu(X,Y)$ the size of a largest directed $(X,Y)$-matching in $T'$.

\begin{claim}\label{claim:proba_C}
    $\Pr(E_3) \leq 2^{-k}$.
\end{claim}
    
\begin{proofclaim}
    Let $A, B \subseteq V(T')$ be disjoint sets of $\lceil (1+\epsilon/4)\frac{k}{2} \rceil$ vertices.
    We shall prove that with high probability there is a directed $(A,B)$-matching in $T'$ of size at least $\frac{k}{2}$.
    
    Let $M$ be a maximal directed $(A,B)$-matching and let $Y_A$ (resp. $Y_B$) the the set of vertices in $A$ (resp. in $B$) incident to no arc of $M$.
    Then $Y_B \Rightarrow Y_A$ in $T'$ since $M$ is maximal.
    Moreover, if $|M| \leq k/2$, then $|Y_A|,|Y_B| \geq (1+\epsilon/4)\frac{k}{2}-\frac{k}{2} = \frac{\epsilon}{8}k$.
    
    For every such $Y_A \subseteq A, Y_B \subseteq B$, we identify $Y_A$ and $Y_B$ with the matrices whose columns are the $\vec{u}$ for $u \in Y_A$ (resp. $u \in Y_B$).
    We also denote by $T(Y_A,Y_B)$ the $|A| \times |B|$ matrix whose cell $(u,v)$ equal $1$ if and only if $uv \in A(T)$.
    Then observe that $Y_B \Rightarrow Y_A$ in $T'$ if and only if $Y_B^\top \cdot Y_A = T(Y_B,Y_A)$.
    By these observations, we have
    \[
    \begin{array}{r c l}
       \Pr[\mu_{T'}(A,B)<k]  & \leq & %\Pr[\exists X_A\subseteq A, X_B \subseteq B', |X_A \cup X_B| \leq k/2, X_A \cup X_B \text{ vertex cover of } (A',B')] \\
       \Pr[\exists Y_A \subseteq A, |Y_A| \geq \frac{\epsilon}{8}k, \exists Y_B \subseteq B, |Y_B| \geq \frac{\epsilon}{8}k, Y_B \Rightarrow Y_A \text{ in } T']  \\
       & \leq & \Pr[\exists Y_A \subseteq A, |Y_A| \geq \frac{\epsilon}{8}k, \exists Y_B \subseteq B, |Y_B| \geq \frac{\epsilon}{8}k, Y_B^\top \cdot Y_A = T(Y_A,Y_B)]  \\
       & \leq & 2^{2\lceil (1+\epsilon/{\color{blue} 4})k/2 \rceil} 2^{-t\epsilon k/128} \hspace{1cm} \text{ by Lemma~\ref{lemma:proba_3} and the Union Bound} \\
       & \leq & 2^{(1+\epsilon/{\color{blue}4})k+2} 2^{-t\epsilon k/128} \hspace{1.325cm}  \\
       & \leq & 2^{-(2C+1)k} \\
    \end{array}
    \]
    
    as $t \geq \frac{128}{\epsilon}(2C+2+\epsilon/{\color{blue}4}) + 16$ and $\epsilon k/64 \geq \frac{1}{8}$.
    It follows from the union bound that $\Pr(E_3) \leq 2^{2n} 2^{-(2C+1)k} \leq 2^{-k}$ using the fact that $n \leq Ck$.
\end{proofclaim}

We can now conclude using Claims~\ref{claim:decompose_into_events_A_B_C_},~\ref{claim:proba_A},~\ref{claim:proba_B}
and~\ref{claim:proba_C} and the Union Bound: 
\[
\begin{split}
\Pr[T' \text{ not } k\text{-strong}] &\leq \Pr(E_1) + \Pr(E_2)+\Pr(E_3) \\
&\leq (2^t-1) \exp\left(-\epsilon^2\frac{n}{24}\right) + 3n^2\exp\left(-\epsilon^2\frac{n-2}{4096}\right) + 2^{-k} \\
&\leq (2^t-1) \exp\left(-\epsilon^2\frac{(2+\epsilon)k}{24}\right) + 3(Ck)^2\exp\left(-\epsilon^2\frac{(2+\epsilon)k}{4096}\right) + 2^{-k} \\
&<1, \\
\end{split}
\]
by \eqref{eq:condition_k_large_enough}.
This proves that there exist $X_1, \dots, X_t \subseteq V(T)$ such that $T'=\Inv(T;X_1,\dots, X_t)$ is $k$-strong.
\end{proof}









% \section*{Acknowledgements}
% \CR{We are thankful to an anonymous reviewer whose numerous comments on a previous version \FHO{have led} to many improvements on the presentation of the paper.}
% \FH{Je ne sais pas si on doit le mettre dès maintenant. Ca pourrait influer un revoewer à refuser car le papier a déjà été refusé. On pourra le mettre à la fin}




\bibliographystyle{alpha}
\bibliography{biblio}
 
 
 
\end{document}
