
\section{Bounds on \texorpdfstring{$M'_k$}{M'k} and \texorpdfstring{$M_k$}{Mk}}\label{sec:M}
%%%%%%%%%%%%%%%%%%%%%

This section is dedicated to giving upper and lower bounds on $M_k$ and $M_k'$, in particular proving Theorems~\ref{thm:exttournoi} and~\ref{thm:upperM'}. First, in Section \ref{sec:52}, we completely determine $M_k$ and $M_k'$ for some small values of $k$. Next, in Section \ref{sec:51}, we prove Theorem \ref{thm:m(2k+1)}, which is the lower bound in Theorem \ref{thm:exttournoi} and a slight extension of this result.   In Section \ref{sec:53}, we prove Theorem \ref{thm:M<2k}, which is the upper bound in Theorem \ref{thm:exttournoi}. We further give a slight improvement of this result for tournaments on exactly $2k+1$ vertices. Finally, in Section \ref{sec:54}, we give the somewhat involved, probabilistic proof of Theorem \ref{thm:upperM'}.


\subsection{Values of \texorpdfstring{$M'_1$}{M'1}, \texorpdfstring{$M_1$}{M1}, \texorpdfstring{$M'_2$}{M'2} and \texorpdfstring{$M_2$}{M2}}\label{sec:52}
%%%%%%%%%%%%%%%%%%%%%%%%%%%%%%%%%%%%%%%%%%%%%%%%%%%%%%%
We here provide the exact values of $M_i$ and $M_i'$ for $i \in \{1,2\}$.
\begin{proposition}\label{prop:M1}
Let $T$ be a tournament of order $n \geq 3$.
We have $\sinv_1(T) = \sinv'_1(T) =0$ if $T$ is strong and $\sinv_1(T) = \sinv'_1(T) =1$ otherwise.
In particular,  $m_1(n) = m'_1(n) =1$ for all $n\geq 3$ and $M_1=M'_1=1$.
\end{proposition}
\begin{proof}
Trivially, if $T$ is strong, then $\sinv_1(T)=0$. 
If $T$ is not strong, then $\sinv_1(T) \geq 1$. Now consider a hamiltonian path of $T$.
Such a path does exist by Redei's Theorem (see e.g. Theorem~1.4.2 in \cite{bang2009}). Let $x$ be its initial vertex and $y$ its terminal vertex.
As $T$ is not strong, we have $xy \in A(T)$, and hence inverting $\{x,y\}$ yields a tournament with a directed hamiltonian cycle because $V(T) \setminus\{x,y\} \neq \emptyset$.
It follows that this tournament is strong, and so $\sinv_1(T) = 1$.
\end{proof}








%A non-strong tournament is said to be {\bf reducible}.
%By definition, a reducible tournament has a {\bf reduction}  $T_1\Ra T_2$ that is two subtournaments $T_1, T_2$ such that $(V(T_1), V(T_2))$ is a partition of $V(T)$ and $V(T_1) \Ra V(T_2)$.




For every integer $n$, let $TT_n$ be the unique (up to isomorphism) acyclic tournament of order $n$.
Let $\vec{C}_3$ be the directed triangle, and let $S_4$ be the unique (up to isomorphism) strong tournament of order $4$.
Its vertex set is $\{a,b,c,d\}$ and its arc set is $\{ab, bc, cd, da, ca, db\}$.


\begin{proposition}\label{prop:M2}
$M_2=M'_2=2$.
\end{proposition}
\begin{proof}
The rotative tournament $R_5$ of order $5$ is the only 2-arc-strong tournament of order $5$.
As observed in \cite{BBBP10}, we have $\inv(R_5)=2$, so $\sinv'_2(TT_5) = 2$.
Hence $M_2 \geq M'_2\geq 2$.

\medskip

Let us now prove that $M_2\leq 2$.
We shall prove by induction on $n$ that every tournament $T$ of order at least $5$ satisfies $\sinv_2(T) \leq 2$.


Assume first that $T$ is a tournament of order $5$.
If $T$ is strong, then, by Camion's theorem~\cite{camion1959}, it has a hamiltonian cycle
$v_1v_2v_3v_4v_5v_1$. Let $A^+=A(T)\cap \{v_1v_3, v_2v_4, v_3v_5, v_4v_1, v_5v_2\}$ and $A^-=A(T)\cap \{v_3v_1, v_4v_2, v_5v_3, v_1v_4, v_2v_5\}$.
We have $|A^+| + |A^-|=5$, so one of the two sets $A^+, A^-$ has at most two arcs. Reversing the arcs of this set, one after another, yields the 2-strong tournament $R_5$.

Assume now that $T$ is not strong. Then it must be isomorphic to one of the following tournaments or their converse, where the {\bf converse} of a digraph $D$ is defined to be $\Inv(D,V(D))$. 
\begin{itemize}
\item $TT_5$ with hamiltonian path $v_1v_2v_3v_4v_5$. Then inverting $\{v_1, v_2, v_4, v_5\}$ and $\{v_1, v_5\}$ yields $R_5$.

\item $S_4\Ra \{x\}$. Then inverting $\{c, d, x\}$ and $\{c, d\}$ yields $R_5$.

\item $(\{x\} \Ra \vec{C}_3) \Ra \{y\}$ with $ \vec{C}_3 = abca$. Then inverting $\{a,x,y\}$ yields $R_5$.

\item $(\{x\} \Ra \{y\}) \Ra \vec{C}_3 $ with $ \vec{C}_3 = abca$.
Then inverting $\{a,b,c,y\}$ and $\{a,x,y\}$ yields $R_5$.

\end{itemize}
As $\sinv_2(\Inv(D,V(D)))=\sinv_2(D)$ for digraph $D$, the statement follows.
Assume now that $T$ has at least $6$ vertices.

Assume first $T$ has a vertex $v$ such that $\min\{d^+(v),d^-(v)\}\geq 2$.
By the induction hypothesis, $\sinv_2(T-v) \leq 2$, so there is a family ${\cal X}$ of at most two subsets of $V(T-v)$ such that
$\Inv(T-v; {\cal X})$ is $2$-strong.
Now $\Inv(T; {\cal X})-v = \Inv(T-v; {\cal X})$ and $\min\{d^+(v),d^-(v)\}\geq 2$. Thus, by Lemma~\ref{lem:kstrong+},
$\Inv(T; {\cal X})$ is $2$-strong, and it follows that $\sinv_2(T) \leq |{\cal X}| \leq 2$.

Assume now that every vertex has either in-degree at most $1$ or out-degree at most $1$. Then necessarily $T$ must be the tournament $\vec{C}_3\Ra \vec{C}_3$.
Let $V(T) = \{a,b,c,d,e,f\}$ with $\{a,b,c\} \Ra \{d,e,f\}$.
Then inverting $\{a,b,c,d\}$ and $\{a, d,e,f\}$ transforms $T$ into a $2$-strong tournament.
\end{proof}



\subsection{Lower bound on \texorpdfstring{$m'_k(2k+1)$}{mk(2k+1)}}\label{sec:51}
%%%%%%%%%%%%%%%%%%%%%%%%%%%%%

We shall first show the lower bound $m'_k(2k+1) = \Omega (\log k)$. We need the following result.
%A digraph is {\bf eulerian} if $d^-(v) = d^+(v)$ for all vertex $v$.


%\begin{proposition}[Folklore]\label{eul}
%Every $k$-arc-strong tournament on $2k+1$ vertices is eulerian.
%\end{proposition}

\begin{theorem}[McKay~\cite{McK90}]\label{count}
Let $n$ be an odd integer.
The number of labelled eulerian tournaments on $n$ vertices is  $$\left(\frac{2^{n+1}}{\pi n}\right)^{\frac{n-1}{2}}\sqrt{\frac{n}{\e}} (1 +o(1)).$$
\end{theorem}
We are now ready to prove the lower bound in Theorem~\ref{thm:exttournoi}.
\begin{theorem}\label{thm:m(2k+1)}
For every sufficiently large $k$, $m'_k(2k+1)\geq \frac{1}{2}\log (2k+1)$.
\end{theorem}
\begin{proof}
    By Theorem ~\ref{count}, we may choose an integer $k_0$ such that for all $k \geq k_0$ and for $n=2k+1$, the number of labelled eulerian tournaments on $n$ vertices is at most $\left(\frac{2^{n+1}}{\pi n}\right)^{\frac{n-1}{2}}\sqrt{n}$ and $\frac{n-1}{2}(\log \pi -1)>\log n$. Now fix some $k\geq k_0$ and let $n=2k+1$.  By Proposition~\ref{eul}, all the $k$-arc-strong tournaments on $n$ vertices are eulerian. % Furthermore, by Theorem~\ref{count}, if $k$ is large enough, then the number of labelled eulerian tournaments on $n$ vertices is at most
    %$\left(\frac{2^{n+1}}{\pi n}\right)^{\frac{n-1}{2}}\sqrt{n}$.
    For two labelled tournaments $T,T'$ on the same vertex set, we say that $T'$ is {\bf reachable} from $T$ by $t$ inversions for some positive integer $t$, if there is a family of sets $(X_1,\ldots,X_t)$ such that $\Inv(T;X_1,\ldots,X_t)=T'$. Observe that there are $2^n$ possibilities to choose $X_i$ for $i=1,\ldots,t$, hence the number of tournaments reachable from $T$ by $t$ inversions is at most $(2^n)^t=2^{nt}$. Therefore,
    %Since at most $2^{nt}$ labelled tournaments are reachable from a fixed $n$-vertex tournament using at most $t$ inversions,
    the number of labelled $n$-vertex labelled tournaments that are reachable by $t$ inversions from an eulerian  one is at most 
    \begin{eqnarray*}
    2^{nt}\left(\frac{2^{n+1}}{\pi n}\right)^{\frac{n-1}{2}}\sqrt{n}
    & = &2^{\binom{n}{2}}\cdot 2^{nt} \left(\frac{2}{\pi n}\right)^{\frac{n-1}{2}}\sqrt{n} \\
    & = &2^{\binom{n}{2}}\cdot 2^{nt +\frac{n-1}{2}(1 -\log(\pi)-\log n)+\frac{1}{2}\log n}\\
    & < &2^{\binom{n}{2}}\cdot 2^{nt-\frac{1}{2} n \log n}
    \end{eqnarray*}
    %for $n$ sufficiently large. % (actually $k = \frac{n-1}{2} \geq 5$ will do)\FHO{J'effacerais la remarque en parentheses. Je ne sais pas si ca marche bien avec le $o(1)$}.
    If $t \leq \frac{1}{2}\log n$, then $nt - \frac{1}{2} n \log n \leq 1$, so the number of such tournaments is fewer than $2^{\binom{n}{2}}$. It follows that there is at least one tournament $T^*$ on $2k+1$ vertices which cannot be reached from an eulerian tournament by at most $t$ inversions. Thus $\sinv'_k(T^*)>t$.
\end{proof}


Theorem~\ref{thm:m(2k+1)} can be slightly generalised to tournaments on $2k+c$ vertices for small integers $c$. To do so, we will need the following generalisation of Theorem~\ref{count}.

For a positive integer $n$ and a collection of integers $\alpha_1,\dots ,\alpha_n$, we denote by
$NT(n;\alpha_1, \dots,\alpha_n)$ the number of labelled tournaments
on $[n]$ in which the vertex $i$ satisfies $d^+(i)-d^-(i)=\alpha_i$.
% \begin{theorem}[Spencer~\cite{Spencer74} and McKay~\cite{McK90}]\label{thm:spencer}
% Let $n$ be a positive odd integer and let $\alpha_1, \dots, \alpha_n$ be integers.
% Then
% \[
% NT(n,\alpha_1, \dots,\alpha_n) \leq
% \left(\frac{2^{n+1}}{\pi n}\right)^{(n-1)/2} \sqrt{\frac{n}{\e}} \exp\left(-\frac{1+o(1)}{2n}\sum_{i=1}^n\alpha_i^2\right).
% \]
% \end{theorem}

\begin{theorem}[Spencer~\cite{Spencer74} and McKay~\cite{McK90}]\label{thm:spencer}
Let $c$ be an integer and let $\epsilon > 0$.
There is an integer $n_0$ such that for every $n \geq n_0$,
for every $\alpha_1, \dots, \alpha_n \in [-c,c]$, we have
\[
NT(n,\alpha_1, \dots,\alpha_n) \leq
\left(\frac{2^{n+1}}{\pi n}\right)^{(n-1)/2} \sqrt{\frac{n}{\e}} \exp\left(-\frac{1-\epsilon}{2n}\sum_{i=1}^n\alpha_i^2\right).
\]
\end{theorem}

% \CR{Changer la suite en consequence.}

% \reviewer{when introducing k and c make it clear that c is fixed and k is large as in
% Theorem 5.5}

% \reviewer{as far as I can tell, Theorem 5.3 is only proved in these references for $\alpha_i$ "small".
% This is, of course, sufficient for your purposes, but unless I am missing something in
% the references, please add a suitable condition on the $\alpha_i$
% in the theorem statement.}
% \CR{Maybe say "fixed $\alpha_i$"?}\FHO{I would state something like 'For every $c$, there exists $n$ s.t. whenever $\alpha_1,\ldots,\alpha_n \leq c$, then $\ldots$'. I am also not quite sure how to interprete this $o(1)$. Is this a negative number? How does this depend on $c$?It might be clearer to suppose that $n$ is sufficiently large and replace it by some constant.}

%\FHO{Qc est bizarre. Ce resultat est beacoup plus fort que le precedent, mais 15 ans plus vieux.}\CR{Après quelques recherches, il semblerai que Spencer exprime l'estimation de $NT(N,(\alpha_i))$ en fonction de $NT(n,0,\dots,0)$, et c'est McKay qui a eu la premiere estimation de $NT(n,0)$ a un facteur $1+o(1)$ près (celle de Spencer etait à $(1+o(1))^n$ près.}
\begin{corollary}\label{corollary:number_tournaments_high_degree}
For all integers $k$ and $c$, the number of labelled tournaments $T$ on $n=2k+c$ vertices
such that every vertex has in- and out-degree at least $k$ is at most
$2^{\log(2c)n}\left(\frac{2^{n+1}}{\pi n}\right)^{(n-1)/2} \sqrt{\frac{n}{\e}}$ for $k$ large enough.
\end{corollary}

\begin{proof}
By Theorem~\ref{thm:spencer} for $\epsilon = 1$,
for $k$ large enough,
the number of labelled tournaments $T$ on $n=2k+c$ vertices
such that every vertex has in- and out-degree at least $k$ is at most

\[
\left(\frac{2^{n+1}}{\pi n}\right)^{(n-1)/2} \sqrt{\frac{n}{\e}}
\sum_{(\alpha_1, \dots,\alpha_n) \in \{-c+1, \dots, c-1\}^n} 1,
\]
which is
\[
\left(\frac{2^{n+1}}{\pi n}\right)^{(n-1)/2} \sqrt{\frac{n}{\e}}
(2c-1)^n
\]
and the corollary follows from the fact that $(2c-1)^n \leq 2^{\log(2c)n}$.
%\CR{Merci de verifier le calcul, mais en fait c'est carrement plus simple avec $\epsilon=1$.}
% Further, we have
% \[
% \begin{split}
% \sum_{\alpha_1, \dots,\alpha_n \in \{-c+1,\dots c-1\}^n}\exp\left(-\frac{1}{4n}\sum_{i=1}^n\alpha_i^2\right)
% &= \prod_{i=1}^n\left(\sum_{\alpha=-c+1}^{c-1}\exp\left(-\frac{1}{4n} \alpha^2\right)\right) \\
% &\leq  \prod_{i=1}^n\left(\sum_{\alpha=-c+1}^{c-1}1\right) \\
% &= (2c-1)^n \\
% &\leq 2^{\log(2c)n}
% \end{split}
% \]
% \FHO{on ne peut pas simplifier ce calcul en estimant $\exp\left(-\frac{1+o(1)}{2n} \delta^2\right)$ par $1$?}
%\FHO{je ne vois pas du tout d'ou vient la premiere egualite}
%\CR{J'ai rajouté une etape. Il faut juste developper.}
% for $n$ large enough, and the result follows.
\end{proof}

% \JD{On ne peut pas plutôt écrire $\exp{ 2 n \log(c)}$ pour la dernière ligne? Si c'est vrai, raffiner les constantes donnerai une borne inf pas si horrible pour le passage à $m_k \leq $ constante, de l'ordre de $c = k^{O(1)}$.}
% \CR{Oui tu as raison, j'ai fait la modif.}

\begin{theorem}\label{thm:2k+c}
    For every fixed positive integer $c$, and for every positive integer $k$ large enough compared to $c$, there exists a tournament
    $T$ on $2k+c$ vertices such that $\sinv'_k(T) > \frac{1}{2}\log (2k+c)-\log(2c)$.
    In particular, $m'_k(2k+c)$ is unbounded for every fixed $c$.
\end{theorem}



\begin{proof}
By Corollary~\ref{corollary:number_tournaments_high_degree},
the number of labelled tournaments $T$ on $n=2k+c$ vertices with $\sinv'_k(T) \leq t$
is for $k$ large enough at most 
\[
2^{nt} 2^{\log(2c)n}\left(\frac{2^{n+1}}{\pi n}\right)^{(n-1)/2} \sqrt{\frac{n}{\e}}
< 2^{\binom{n}{2}} 2^{nt-\frac{1}{2}n\log n+\log(2c)n}
\]
which is at most $2^{\binom{n}{2}}$ if $t \leq \frac{1}{2}\log n -\log(2c)$.
Hence there exists a tournament $T^*$ with $\sinv'_k(T^*) > \frac{1}{2}\log (2k+c)-\log(2c)$.
\end{proof}







\subsection{Upper bounds on \texorpdfstring{$M_k$}{Mk} and transitive tournaments}\label{sec:53}
%%%%%%%%%%%%%%%%%%%%%%%%%
In this section, we give some upper bounds on $M_k$ for all positive integers $k$. Most notably, we prove the upper bound in Theorem~\ref{thm:exttournoi}.
%\CR{Peut-être ajouter une phrase ici.}


\begin{theorem}\label{thm:M<2k}
$M_k \leq 2k$.
\end{theorem}

\begin{proof}
Let $T$ be a tournament with $V(T)=\{v_1,\ldots,v_n\}$ with $n \geq 2k+1$. Further, let $T'$ be a $k$-strong tournament on $\{v_1,\ldots,v_{2k+1}\}$. We now define sets $X_1,\ldots,X_{2k}$. 
Suppose that the sets $X_1,\ldots,X_{i-1}$ have already been created and let $T_{i-1}$ be the graph obtained from $T$ by inverting $X_1,\ldots,X_{i-1}$.
Now let $X_i = \{v_i\} \cup A_i \cup B_i$, where
$A_i$ is the set of vertices $v_j$ with $j \in \{i+1,\ldots,2k+1\}$ for which the edge $v_iv_j$ has a different orientation in $T'$ and $T_{i-1}$,
and $B_i$ is, when $i \leq k$ (resp. $i \geq k+1$), the set of vertices $v_j$ with $j \geq 2k+2$ for which $T_{i-1}$ contains the arc $v_iv_j$ (resp. $v_jv_i$).

We still need to show that $T_{2k}$ is $k$-strong. Observe that $T_{2k}\langle\{v_1,\ldots,v_{2k+1}\}\rangle=T'$ which is $k$-strong by assumption. Moreover, for any $j\geq 2k+2$, $T_{2k}$ contains the arcs $v_jv_i$ for $i=1,\ldots,k$ and the arcs $v_iv_j$ for $i=k+1,\ldots,2k$. Hence, by Lemma~\ref{lem:kstrong+}, $T_{2k}$ is $k$-strong.
\end{proof}


Theorems~\ref{thm:m(2k+1)} and~\ref{thm:M<2k} directly imply Theorem~\ref{thm:exttournoi}.%M_k \leq 2k$.

It is tempting to improve the upper bound in Theorem~\ref{thm:exttournoi}. While we are not able provide an improvement for the general case, we show in the following that a small improvement can be achieved in a seemingly critical case, namely when the size of the tournament is exactly $2k+1$. We first need the following result.

\begin{theorem}\label{thm:TTn}
    Let $n,k$ be integers with $n \geq 2k+1$.
    \[
    \sinv_k(TT_n) = \sinv'_k(TT_n) =
    \left\{
    \begin{array}{l l}
        2 & \text{ if~~} 2k+1 \leq n < 3k \\
        1 & \text{ if~~} 3k \leq n. \\
    \end{array}
    \right.
    \]
\end{theorem}

\begin{proof}
We first prove that $\sinv_k(n)\leq 2$.    Let $(v_1, \dots, v_n)$ be the unique ordering of $V(TT_n)$ such that
    $v_i v_j \in A(TT_n)$ for every $i<j$.
    Let $X_0 = \{v_i \mid i \text{ even}\}$ and $X_1 = \{v_i \mid i \text{ odd}\}$.
    We claim that $T' = \Inv(TT_n; X_0,X_1)$ is $k$-strong.
    Consider  a set $Y$ of at most $k-1$ vertices of $T'$.
    Let $i_0$ (resp. $j_0$) be the smallest (resp. largest) even integer $\ell$ such that $v_\ell \not\in Y$, and let
    $i_1$ (resp. $j_1$) be the smallest (resp. largest) odd integer $\ell$ such that $v_\ell \not\in Y$.
    Observe that $j_0 > i_1$ and $j_1 > i_0$ since $|Y|<k$ and $n \geq 2k+1$.
    In particular, $v_{i_1}v_{j_0}, v_{i_0}v_{j_1} \in A(T')$.
    Since $T'\langle X_0 \rangle-Y$ (resp. $T'\langle X_1\rangle-Y$) is a transitive tournament with source $v_{j_0}$ and sink $v_{i_0}$ (resp. source $v_{j_1}$ and sink $v_{i_1}$), there are hamiltonian paths $P_0$ from $v_{j_0}$ to $v_{i_0}$ in $T'\langle X_0 \rangle-Y$ and $P_1$ from $v_{j_1}$ to $v_{i_1}$ in $T\langle X_1 \rangle-Y$.
    Finally $P_0v_{i_0}v_{j_1}P_1v_{i_1}v_{j_0}$ is an hamiltonian directed cycle of $T'-Y$, and so $T'-Y$ is strong.
    Hence $T'$ is $k$-strong, and $\sinv_k(TT_n) \leq 2$.

    Let us now prove that $\sinv'_k(TT_n) \geq 2$ if $n \leq 3k-1$.
    Suppose for a contradiction that there exists $X \subseteq V(TT_n)$ such that $T'=\Inv(TT_n;X)$ is $k$-arc-strong.
    Then, for every $i \leq k$, since $v_i$ has in-degree $i-1$ in $TT_n$,  we have $v_i \in X$.
    Similarly, for every $i \geq n-k+1$, since $v_i$ has out-degree $n-i$ in $TT_n$, we have $v_i \in X$.
    But then every out-neighbour of $v_1$ in $T'$ is in $\{v_{k+1}, \dots, v_{n-k}\}$ and so $d^+_{T'}(v_1) \leq n-2k < k$.
    Hence $T'$ is not $k$-arc-strong, a contradiction.

    Finally, let us show that $\sinv_k(TT_n) \leq 1$ if $n \geq 3k$.
    Let $(v_1, \dots, v_n)$ be the unique ordering of $V(TT_n)$ such that
    $v_i v_j \in A(TT_n)$ for every $i<j$.
    Let $A = \{v_1, \dots, v_k\}, B= \{v_{k+1},\dots, v_{2k}\}$ and $C = \{v_{2k+1}, \dots, v_n\}$.
    In $T' = \Inv(TT_n; A\cup C)$, we have $A \Rightarrow B, B \Rightarrow C$ and $C \Rightarrow A$. Since $|A|,|B|,|C| \geq k$, $T'$ is $k$-strong.
\end{proof}


As mentioned in the introduction, it was proven independently by Alon et al. \cite{APSSW} and Aubian et al. \cite{inversion} that $\inv(n) \leq n - \lceil \log (n+1) \rceil$. Hence , every tournament of order $2k+1$ can be made acyclic in at most $2k - \lceil \log (k+1) \rceil$ inversions, and $k$-strong in two more inversions by Theorem~\ref{thm:TTn}. Thus we have the following.
\begin{corollary}
$m_k(2k+1) \leq 2k - \lceil \log (k+1) \rceil +2$.  
\end{corollary}






\subsection{Upper bound on \texorpdfstring{$M'_k$}{M'k}}\label{sec:54}
%%%%%%%%%%%%%%%%%%%%%%%%%%%%%%%%%%

This section is dedicated to proving Theorem~\ref{thm:upperM'}. First, we give some auxiliary results in Section~\ref{prelm'}. Next, in Section~\ref{small}, we show the result when restricting to tournaments of size exactly $2k+1$. We finally generalize this to bigger tournaments in Section~\ref{big}.

\subsubsection{Preliminaries}\label{prelm'}
%%%%%%%%%%%%%%%%%%%%%%%%%%%%%
We here collect some auxiliary results we need for the proof of Theorem~\ref{thm:upperM'}.
We first need a basic result combining probability and linear algebra.

\begin{proposition}\label{linalg}
    Let $A\in \mathbb{F}_2^{q \times q'}$ be a matrix whose rank is $q$ for some integers $q,q'$ with $q \leq q'$. Further, let $\vec{v}\in \mathbb{F}_2^q$ be a fixed vector and let another vector $\vec{w}\in \mathbb{F}_2^{q'}$ be drawn uniformly at random. Then $\Pr(A\vec{w}=\vec{v})=2^{-q}$.
\end{proposition}
The simple proof of the following result is similar to the one of Theorem~\ref{thm:M<2k}. It will appear fully in \cite{diameter}.
\begin{proposition}\label{fuzzu}
    Let $T_1,T_2$ be tournaments with $V(T_1)=V(T_2)$. Then there is a collection of $|V(T_1)|-1$ subsets $X_1,\ldots,X_{|V(T_1)|-1}$ of $V(T_1)$ such that $\Inv(T_1;X_1,\ldots,X_{|V(T_1)|-1})=T_2$.
\end{proposition}


We are now ready to give the last preliminary result.
\begin{proposition}\label{extend}
    Let $T$ be a tournament on $2k+1$ vertices and $X \subseteq V(T)$ with $|X|\leq \frac{2}{3}k$ such that $d_T^-(x)=d_T^+(x)=k$ for all $x \in X$. Then there exists a $k$-arc-strong tournament $T'$ on $V(T)$ such that all the edges in $E(\UG(T)\langle X \rangle) \cup \delta_{\UG(T)}(X)$ have the same orientation in $T$ and $T'$.
\end{proposition}

\begin{proof}
    Let $H$ be the mixed graph which is obtained from $\UG(T)$ by giving all edges in $E(\UG(T)\langle X \rangle) \cup \delta_{\UG(T)}(X)$ the orientation they have in $T$. Now consider some $S \subseteq V(T)$.
    Let $x \in S\cap X$. Since $d_H^+(x)=d_H^-(x)$, we have
    \begin{align*}
        d_H^+(S\setminus \{x\})-d_H^-(S\setminus \{x\})
        &=\left(d_H^+(S)-d_H^+(x,V(H)\setminus S)+d_H^+(S\setminus \{x\},x)\right)\\
        &\qquad -\left(d_H^-(S)-d_H^-(x,V(H)\setminus S)+d_H^-(S\setminus \{x\},x)\right)\\
        &=d_H^+(S)-d_H^+(S)\\
        &\qquad +d_H^+(S\setminus \{x\},x)+d_H^+(V(H)\setminus S,x)\\
        &\qquad -d_H^+(x,V(H)\setminus S)-d_H^+(x,S\setminus \{x\}))\\
        &=d_H^+(S)-d_H^+(S)+d_H^-(x)-d_H^+(x)\\
        &=d_H^+(S)-d_H^-(S).
    \end{align*}
    Repeatedly applying this argument, we obtain $d_H^+(S)-d_H^-(S)=d_H^+(S \setminus X)-d_H^-(S \setminus X) \leq d_H^+(S \setminus X) \leq |S \setminus X| \cdot |X|$ since every arc in $H$ has an extremity in $X$. Symmetrically, we obtain $d_H^+(S)-d_H^-(S)=d_H^-(V(H)\setminus S)-
    d_H^+(V(H)\setminus S)=d_H^-(V(H)\setminus (S\cup X))-
    d_H^+(V(H)\setminus (S\cup X))\leq |V(H)\setminus (S\cup X)| \cdot |X|$.
    
    As $|X|\leq \frac{2}{3}k\leq \frac{1}{2}|V(H)\setminus X| = \frac{1}{2}(|S\setminus X| + |V(H)\setminus(S \cup X)|)$, we obtain
    \begin{align*}
        d_H^+(S)-d_H^-(S)&\leq \min \{|S \setminus X|,\ |V(H)\setminus (S\cup X)|\} \cdot |X|\\
        & \leq \min \{|S \setminus X|,\ |V(H)\setminus (S\cup X)|\} \cdot \tfrac{1}{2}(|S\setminus X| + |V(H)\setminus(S \cup X)|)\\
        & \leq \min \{|S \setminus X|,\ |V(H)\setminus (S\cup X)|\} \cdot \max \{|S \setminus X|,\ |V(H)\setminus (S\cup X)|\}\\
        &=|S \setminus X| \cdot |V(H)\setminus (S\cup X)|\\
        &= d_H(S).
    \end{align*}
    %
    %By symmetry, we may suppose that $|V(H)\setminus (S\cup X)|\geq |S\setminus X|$. Observe that $|V(H)\setminus (S\cup X)|\geq \frac{1}{2}|V(H)\setminus X|\geq \frac{2}{3}k\geq |X|$. As $d_{H}^+(x)=d_{H}^-(x)$ for all $x \in X$, this yields
    %\begin{align*}
    %d_{H}^+(S)-d_{H}^-(S)&=\left(\sum_{x\in S \cap X}d_{H}^+(x)-|A(T\langle S\cap X \rangle)| -d^+_{H}(S\cap X,S\setminus X)\right)\\&\hspace{5mm}-\left(\sum_{x\in S\cap X}d_{H}^-(x)-|A(T\langle S\cap X \rangle)|-d^+_{H}(S\setminus X,S \cap X)\right)\\
   % & = \left(k|S \cap X| - |A(T\langle S\cap X \rangle)| - d^+_{H}(S\cap X,S\setminus X)\right)\\&\hspace{5mm}-\left(k|S \cap X| - |A(T\langle S\cap X \rangle)| - d^+_{H}(S\setminus X, S \cap X)\right) \\
    %& = d^+_{H}(S\setminus X, S \cap X) - d^+_{H}(S %\cap X, S\setminus X) \\
    %& \leq d^+_{H}(S\setminus X, S \cap X)\\
    %& \leq d^+_{H}(S\setminus X,X)\\
    %& \leq |S\setminus X|\;|X|\\
    %&\leq |S\setminus X|\;|V(H)\setminus (S\cup X)|\\
    %& = d_{H}(S).
    %\end{align*}
    Hence, by Proposition~\ref{prop:eul-or}, there is an eulerian orientation $T'$ of $T$ such that all the edges in $\UG(T)\langle X \rangle \cup \delta_{\UG(T)}(X)$ have the same orientation in $T$ and $T'$. As $|V(T)|=2k+1$, the tournament $T'$ is $k$-arc-strong by Proposition~\ref{eul}.
\end{proof}


\subsubsection{Main proof for tournaments of size \texorpdfstring{$2k+1$}{2k+1}}\label{small}
%%%%%%%%%%%%%%%%%%%%%%%
We here give the proof of Theorem~\ref{thm:upperM'} for tournaments of size $2k+1$. More precisely, we prove the following statement.
\begin{theorem}\label{mkstrich}
For every $\epsilon >0$, there is an integer $k_0$ such that for every $k \geq k_0$ and for every tournament $T$ on $2k+1$ vertices, we have $\sinv'_k(T)\leq (\frac{4}{3}+\epsilon)k$.
\end{theorem}

\begin{proof}
Let $\epsilon >0$. We may assume $\epsilon \leq 1/3$.
We choose $k_0$ large enough so that for all $k \geq k_0$, the  three following inequalities hold:

\begin{itemize}

\item $\lceil \log^2 k\rceil + \lceil\frac{1}{4}\log k\rceil +  \frac{2k}{3\lceil\frac{1}{4}\log k\rceil} \leq \epsilon k$

\item $16 \log (k) \cdot \lceil \frac{1}{4}\log k\rceil \leq \sqrt{k}$.

\item $10 \log k \leq k^{\frac{1}{8}}$, 

%\item$ 3 \lceil \frac{1}{4} \log k\rceil \leq \log k$

\end{itemize}

In particular, the later inequality implies that $k_0 > 2^{12}$. Thus the following inequalities also hold for all $k\geq k_0$: 
$\lceil\frac{1}{4}\log k\rceil\leq \frac{1}{3} k$; 
$\lceil\frac{1}{4}\log k\rceil\leq \sqrt{k}$; 
$k\geq 13$; 
$ 3 \lceil \frac{1}{4} \log k\rceil \leq \log k$.

\medskip

Let $T$ be a tournament on $2k+1$ vertices for some $k \geq k_0$. Let $A \subseteq V(T)$ be an arbitrary subset of $V(T)$ with $|A|=k$ and $B=V(T)\setminus A$. 
For some $b \in B$ and a tournament $\tilde{T}$ on $V(T)$, we denote $|k-d_{\tilde{T}}^-(b)|$ by {\boldmath $\defect_{\tilde{T}}(b)$}, the {\bf defect of $b$ in $\Tilde{T}$}. Let $q=\lceil\frac{1}{4}\log k\rceil$ and $q^* = \lceil \log^2 k\rceil$. We now choose $q^*$ sets $X_1,\ldots,X_{q^*}\subseteq V(T)$ independently and uniformly at random. 
Let $T'=\Inv(T;X_1,\ldots,X_{q^*})$. For some $b \in B$, we denote by $\vec{b}$ the vector in $\mathbb{F}_2^{q^*}$ whose $i$-th entry is 1 if $b \in X_i$ and 0 otherwise for $i\in[q^*]$.
Observe that by the choice of $X_1,\ldots,X_{q^*}$, the vectors $\{\vec{b} \mid b \in B\}$ follow a uniform distribution and are mutually independent.


We now define a list of possible events:

\begin{enumerate}[label=\Alph*]
    \item[$E_1$]: There is some $b \in B$ with $\defect_{T'}(b)\geq 2 \sqrt{k} \log k $.
    \item[$E_2$]: There are some $b_1,\ldots,b_q \in B$ such that $\vec{b}_1,\ldots,\vec{b}_q$ are linearly dependent.
    \item[$E_3$]: There are some $b_1,\ldots,b_q \in B$ such that $\vec{b}_1,\ldots,\vec{b}_q$ are linearly independent and a sequence $(\star_1,\ldots,\star_q)\in \{+,-\}^q$ such that $|\bigcap_{i=1}^q N_{T'}^{\star_i}(b_i)\cap A|\leq 5 \sqrt{k} \log k $.
\end{enumerate}

\begin{lemma}\label{erfolg}
    If none of $E_1$, $E_2$ and $E_3$ occur, then $\sinv_k'(T)\leq (\frac{4}{3}+\epsilon)k$.
\end{lemma}

\begin{proof}
Let $(B_1,\ldots,B_t)$ be a maximal collection of disjoint subsets of $B$ with the following properties:
\begin{itemize}
    \item $|B_i|=q$ for $i=1,\ldots,t$,
    \item $\defect_{T'}(b)\equiv \defect_{T'}(b') \mod 2$ for all $b,b' \in B_i$ for $i=1,\ldots,t$,
    \item $tq \leq \frac{2}{3}k$.
\end{itemize}

\begin{claim}\label{bgross}
    $|\bigcup_{i=1}^tB_i|\geq \frac{2}{3}k-q$.
\end{claim}
\begin{proofclaim}
    Suppose otherwise. Observe that, as $\lceil\frac{1}{4}\log k\rceil\leq \frac{1}{3} k$, we have $|B\setminus (\bigcup_{i=1}^tB_i)|=|B|-|\bigcup_{i=1}^tB_i|\geq (k+1)-(\frac{2}{3}k-q)\geq \frac{1}{3}k+q \geq 2q$. 
    By the Pigeonhole Principle, there exists a set $B_0\subseteq B\setminus (\bigcup_{i=1}^tB_i)$ with $|B_0|=q$ such that $\defect_{T'}(b)\equiv \defect_{T'}(b') \mod 2$ for all $b,b' \in B_0$. Hence $B_0$ can be added to $\mathcal{B}$ without violating any of the above conditions, a contradiction to the maximality of $(B_1,\ldots,B_t)$.
\end{proofclaim}

%We now consider some fixed $i \in \{1,\ldots,t\}$.
\begin{claim} Let $i\in [t]$.
    There is a set $Y_i\subseteq B_i \cup A$ such that $\defect_{\Inv(T';Y_i)}(b)=0$ for all $b \in B_i$.
\end{claim}
\begin{proofclaim}
    Let $p=\max\{\defect_{\Inv(T';B_i)}(b) \mid b \in B_i\}$. Observe that $p\leq  \max\{\defect_{T'}(b)+q-1 \mid b \in B_i\}$. Hence, as $E_1$ does not occur and $q=\lceil\frac{1}{4}\log k\rceil\leq \sqrt{k} \log k $, we have $p\leq 2 \sqrt{k} \log k + \sqrt{k} \log k = 3 \sqrt{k} \log k $. We shall now build a sequence of sets $Y_i^0,\ldots,Y_i^p$ with the following properties:
    
    \begin{itemize}
        \item $Y_i^0=B_i$,
        \item $Y_i^{j+1}$ is obtained from $Y_i^j$ by adding one vertex from $A\setminus Y_i^j$ for $j=0,\ldots,p-1$,
        \item $\max\{\defect_{\Inv(T';Y_i^j)}(b) \mid b \in B_i\}=p-j$ for $j=0,\ldots,p$.
    \end{itemize}
    The statement then follows for $Y_i=Y_i^p$. 
    
    Suppose that we have already created the sets $Y_i^0,\ldots,Y_i^j$ for some $j$ with $0\leq j \leq p-1$ and now want to create $Y_i^{j+1}$. 
    Let $(b_1,\ldots,b_q)$ be an arbitrary ordering of the vertices in $B_i$. 
    For every $\mu \in [q]$, if  $\max\{\defect_{\Inv(T';Y_i^j)}(b) \mid b \in B_i\}=d_{\Inv(T';Y_i^j)}^-(b_\mu)-k$, let $\star_\mu=-$, if $\max\{\defect_{\Inv(T';Y_i^j)}(b) \mid b \in B_i\}=k-d_{\Inv(T';Y_i^j)}^-(b_\mu)$, let $\star_\mu=+$, otherwise choose $\star_\mu \in \{+,-\}$ arbitrarily. 
    As $E_2$ does not occur, $\vec{b}_1,\ldots,\vec{b}_q$ are linearly independent. 
    Moreover, as $E_3$ does not occur, $|\bigcap^q_{\mu=1} N^{\star_\mu}_{T'}(b_\mu)\cap A|\geq 5 \sqrt{k} \log k$. 
    As $p \leq 3 \sqrt{k} \log k$, and $q \leq \sqrt{k} \log k $, we have $|Y_i^j|\leq p+q \leq 4 \sqrt{k} \log k$. 
    Thus $(\bigcap_{\mu=1}^q N_{T'}^{\star_\mu}(b_\mu)\cap A)\setminus Y_i^j\neq \emptyset$. 
    We can hence choose some $y \in (\bigcap_{\mu=1}^q N^{\star_\mu}_{T'}(b_\mu)\cap A)\setminus Y_i^j$ and define $Y_i^{j+1}=Y_i^j\cup \{y\}$. 
    By definition, $Y_i^{j+1}$ is obtained from $Y_i^j$ by adding one vertex in $A\setminus Y_i^j$. 
    In order to see that the last property holds, let $\mu \in [q]$. 
    If $\max\{\defect_{\Inv(T';Y_i^j)}(b)\mid b \in B_i\}=d_{\Inv(T';Y_i^j)}^-(b_\mu)-k$, we have $\star_\mu=-$ and hence $y \in N^-_{\Inv(T';Y_i^j)}(b_\mu)$. 
    As $d_{\Inv(T';Y_i^j)}^-(b_\mu)-k >0$, we obtain $\defect_{\Inv(T';Y_i^{j+1})}(b_\mu)=d_{\Inv(T';Y_i^{j+1})}^-(b_\mu)-k=d_{\Inv(T';Y_i^j)}^-(b_\mu)-k-1=p-j-1=p-(j+1)$. 
    Similarly, if $\max\{\defect_{\Inv(T';Y_i^j)}(b)\mid b \in B_i\}=k-d_{\Inv(T';Y_i^j)}^-(b_\mu)$, then $\defect_{\Inv(T';Y_i^{j+1})}(b)=p-(j+1).$ 
    Otherwise, let $b' \in B$ with $\defect_{\Inv(T';Y_i^j)}(b')=p-j$.
    As $\defect_{T'}(b_\mu) \equiv \defect_{T'}(b') \mod 2$, we have $\defect_{\Inv(T';Y_i^j)}(b_\mu)\equiv \defect_{T'}(b_\mu)+|Y_i^j|-1\equiv \defect_{T'}(b')+|Y_i^j|-1\equiv \defect_{\Inv(T';Y_i^j)}(b') \mod 2$, hence $\defect_{\Inv(T';Y_i^j)}(b_\mu)\leq \defect_{\Inv(T';Y_i^j)}(b')-2$. 
    This yields $\defect_{\Inv(T';Y_i^{j+1})}(b_\mu)\leq \defect_{\Inv(T';Y_i^j)}(b_\mu)+1\leq \defect_{\Inv(T';Y_i^j)}(b')-2+1=p-(j+1)$.
\end{proofclaim}



Let $T''=\Inv(T';(Y_1,\ldots,Y_t))$. Observe that by definition of the sets $Y_i$, we have $\defect_{T''}(b)=0$ for all $b \in\bigcup_{i=1}^t B_i$. 
As $|\bigcup_{i=1}^t B_i|=qt\leq\frac{2}{3}k$, we obtain by Proposition~\ref{extend} that there is a $k$-arc-strong tournament $T'''$ on $V(T)$ such that all the edges in $\UG(T)\langle \bigcup_{i=1}^t B_i\rangle \cup \delta_{\UG(T)}(\bigcup_{i=1}^t B_i)$ have the same orientation in $T''$ and $T'''$. 
Further, by Proposition~\ref{fuzzu}, there is a collection of $r=|V(T)\setminus \bigcup_{i=1}^t B_i|-1$ subsets $Z_1,\ldots,Z_r$ of $V(T)\setminus\bigcup_{i=1}^t B_i$ such that $\Inv(T''\langle V(T)\setminus\bigcup_{i=1}^t B_i \rangle; Z_1,\ldots,Z_r)=T'''\langle V(T)\setminus \bigcup_{i=1}^t B_i \rangle$. 
We obtain that $\Inv(T;X_1,\ldots,X_{q^*},\linebreak Y_1,\ldots,Y_t,Z_1,\ldots,Z_r)=\Inv(T';Y_1,\ldots,Y_t,Z_1,\ldots,Z_r)=\Inv(T'';Z_1,\ldots,Z_r)=T'''$. 
As $T'''$ is $k$-arc-strong, we have  $\sinv_k'(T)\leq q^* +t+r$.
Now $q^*= \lceil \log^2 k\rceil$, $t\leq \frac{2k}{3q}=\frac{2k}{3\lceil\frac{1}{4}\log k\rceil}$, and $r = |V(T)| - |\bigcup_{i=1}^t B_i| -1 \leq 2k -(\frac{2}{3}k -q) = \frac{4}{3}k + q =  \frac{4}{3}k + \lceil\frac{1}{4}\log k\rceil$ by Claim~\ref{bgross}.
Thus $\sinv_k'(T)\leq\frac{4}{3}k + \lceil \log^2 k\rceil + \lceil\frac{1}{4}\log k\rceil +  \frac{2k}{3\lceil\frac{1}{4}\log k\rceil} \leq \frac{4}{3}k + \epsilon k$.
\end{proof}
In the following lemmas, we show that the probability of each of $E_1,E_2$ and $E_3$ is small.
\begin{lemma}\label{prob1}
    $\Pr(E_1)<\frac{1}{3}$.
\end{lemma}
\begin{proof}
Let $b \in B$. We first bound the probability that $\vec{b}=\vec{0}$. Due to the independent and uniform choice of $(X_1,\ldots,X_{q^*})$, we have $\Pr[\vec{b}=\vec{0}]=2^{-q^*}\leq 2^{-2\log k}=\frac{1}{k^2}$.


Now suppose that $\vec{b}\neq \vec{0}$. Once $\vec{b}$ has been revealed, for every $v \in V(T)$, let $\Gamma_v$ be a random variable which is 1 if $T'$ contains the arc $bv$ and 0 otherwise and let $\Gamma=\sum_{v \in V(T-b)}\Gamma_v$. Observe that, due to the independent and uniform choice of $(X_1,\ldots,X_{q^*})$, we have $\Pr(\Gamma_v=1)=\frac{1}{2}$ for all $v \in V(T-b)$ and the $\Gamma_v$ are independent. This yields that $\Gamma\sim \Bin(2k,\frac{1}{2})$. Further, observe that $\defect_{T'}(b)\leq 2 \sqrt{k} \log k $ if and only if $k-2 \sqrt{k} \log k\leq \Gamma \leq k+2 \sqrt{k} \log k$. By Chernoff's Bound (Proposition~\ref{chernoff}), we obtain 

\begin{align*}
    \Pr\left[\Gamma<k-2 \sqrt{k} \log k\right]&\leq \Pr \left[\Gamma<\left(1-\frac{2 \sqrt{k} \log k}{k}\right)k\right]\\
    &\leq \exp\left(-\frac{4 k \log ^2 k}{2k^2}k\right)\\
    &\leq \exp(-2 \log k)\\
    &\leq \exp(-2 \ln k)\\
    & = \frac{1}{k^2}.
\end{align*} 
Similarly, since $2k-\Gamma \sim \Bin(2k,\frac{1}{2})$, we obtain that $\Pr[\Gamma>k+2 \sqrt{k} \log k]\leq \frac{1}{k^2}$.
This yields that 
\begin{align*}
    \Pr[\defect_{T'}(b)&>2 \sqrt{k} \log k]\\&=\Pr[\vec{b}=\vec{0}] \cdot \Pr[\defect_{T'}(b)>2 \sqrt{k} \log k\mid \vec{b}=\vec{0}]+\Pr[\vec{b}\neq 0] \cdot \Pr[\defect_{T'}(b)>2 \sqrt{k} \log k\mid \vec{b}\neq \vec{0}]\\
    &\leq \Pr[\vec{b}=\vec{0}]+\Pr[\Gamma<k-2 \sqrt{k} \log k\mid \vec{b}\neq \vec{0}]+\Pr[\Gamma>k+2 \sqrt{k} \log k \mid \vec{b}\neq \vec{0}]\\
    &\leq \frac{3}{k^2}.
\end{align*}

As there are $k+1$ vertices contained in $B$, by the Union Bound (Proposition~\ref{union}),
$\Pr(E_1)\leq (k+1)\frac{3}{k^2}< \frac{1}{3}$, as $k \geq 10$.
\end{proof}

\begin{lemma}\label{prob2}
    $\Pr(E_2)<\frac{1}{3}$.
\end{lemma}
\begin{proof}
Let $b_1,\ldots,b_q \in B$. Recall that $\vec{b}_1,\ldots,\vec{b}_q$ are independent and uniformly distributed. For $i\in [q]$, observe that if $\vec{b}_1,\ldots,\vec{b}_i$ are linearly independent, they span a vector space containing $2^{i}$ elements. Hence 

\begin{align*}
    \Pr[\vec{b}_1,\ldots,\vec{b}_q\text{ linearly dependent}]&=\sum_{i=1}^q \Pr[\vec{b}_1,\ldots,\vec{b}_i\text{ linearly dependent}\mid \vec{b}_1,\ldots,\vec{b}_{i-1}\text{ linearly independent }] \\
    & \qquad \qquad \times \Pr[\vec{b}_1,\ldots,\vec{b}_{i-1}\text{ linearly independent}]\\
    &\leq \sum_{i=1}^q \Pr[\vec{b}_1,\ldots,\vec{b}_i\text{ linearly dependent}\mid \vec{b}_1,\ldots,\vec{b}_{i-1}\text{ linearly independent}]\\
    &=\sum_{i=1}^q \frac{2^{i-1}}{2^{q^*}}\\
    &\leq \frac{2^q}{2^{q^*}}\\
    &\leq 2^{-2 \log (k) q} ~~~~\mbox{because $q^*\geq  \log^2 k \geq %3 \log k \lceil \frac{1}{4} \log k\rceil = 
    3\log(k) q\geq 2 \log(k) q+q$}\\
    &= k^{-2q}.
\end{align*}

Observe that there are $\binom{k+1}{q}$ possibilities to choose $b_1,\ldots,b_q$. Hence the Union Bound (Proposition~\ref{union}) yields $\Pr(E_2)\leq \binom{k+1}{q} k^{-2q}\leq (2k)^qk^{-2q}=\left(\frac{2}{k}\right)^q< \frac{1}{3}$ 
as  $q\geq 1$ and $k \geq 7$.
\end{proof}


\begin{lemma}\label{prob3}
    $\Pr(E_3)<\frac{1}{3}$.
\end{lemma}

\begin{proof}
Let $b_1,\ldots,b_q \in B$ such that $\vec{b}_1,\ldots,\vec{b}_q$ are linearly independent and let $(\star_1,\ldots,\star_q)\in \{+,-\}^q$. Once $\vec{b}_1,\ldots,\vec{b}_q$ revealed, for every $a \in A$, let $\Gamma_a$ be a random variable that is 1 if $a \in \bigcap_{i=1}^q N_{T'}^{\star_i}(b_i)$ and 0 otherwise and let $\Gamma=\sum_{a \in A}\Gamma_a$. 
Observe that, due to the independent and uniform choice of $(X_1,\ldots,X_{q^*})$ and by Proposition~\ref{linalg}, we have that $\Pr[\Gamma_a=1]=\frac{1}{2^q}$ for all $a \in A$ and that the $\Gamma_a$ are independent. This yields that $\Gamma\sim \Bin(k,\frac{1}{2^q})$. Further observe that $\Gamma=|\bigcap_{i=1}^q N_{T'}^{\star_i}(b_i)\cap A|$.
%Since $10 \sqrt{k} \log k\leq k^{\frac{2}{3}}$ and $\lceil\frac{1}{4}\log k\rceil\leq \frac{1}{3}\log k$, we have 
Now, since $10 \log k \leq k^{\frac{1}{8}}$ and $q = \lceil\frac{1}{4}\log k\rceil \leq \frac{3}{8}\log k$, we obtain $5 \sqrt{k} \log k \leq \frac{1}{2}k^{\frac{5}{8}}=\frac{1}{2}k2^{-\frac{3}{8}\log k} \leq \frac{1}{2}k2^{-q}$.
By Chernoff's Bound (Proposition~\ref{chernoff}), we obtain 

\begin{align*}
\Pr[\Gamma<5 \sqrt{k} \log k]&\leq \Pr\left[\Gamma<\left(1-\frac{1}{2}\right)k2^{-q}\right]\\
&\leq \exp\left(-\frac{1}{8}k2^{-q}\right)\\
&\leq \exp\left(-\frac{1}{8}\sqrt{k}\right)\\
&\leq \exp(-2 q\log k) ~~~~~ \mbox{because $\sqrt{k} \geq 16 \log k \lceil \frac{1}{4}\log k\rceil$}\\
&\leq \exp(-2 q\ln k)\\
& = k^{-2q}.
\end{align*} 

Observe that there are $\binom{k+1}{q}$ possibilities to choose $b_1,\ldots,b_q$ and for each of those, there are $2^q$ possibilities to choose $(\star_1,\ldots,\star_q)$. Hence the Union Bound (Proposition~\ref{union}) yields $\Pr(E_3)\leq \binom{k+1}{q}2^qk^{-2q}\leq (2k)^q2^qk^{-2q}= \left(\frac{4}{k}\right)^q<\frac{1}{3}$, as $q \geq 1$ and $k \geq 13$.
\end{proof}

We are now ready to conclude the proof of Theorem~\ref{mkstrich}. By Lemmas~\ref{prob1},~\ref{prob2} and~\ref{prob3} and  the Union Bound (Proposition~\ref{union}), we obtain that, with positive probability, none of $E_1$, $E_2$ and $E_3$ occur. Hence the statement follows from Lemma~\ref{erfolg}.
\end{proof}



\subsubsection{Bigger tournaments}\label{big}
%%%%%%%%%%%%%%%%%%%%%%%%%%%%%%%%%

We now show how to extend this result to larger tournaments and derive Theorem~\ref{thm:upperM'} from Theorem~\ref{mkstrich}.

\borneMprime*

%\begin{theorem}\label{thm:upperM'-2}
%For every $\epsilon >0$, there is an integer $k_0$ such that for every $k \geq k_0$ and for every tournament $T$ on at least $2k+1$ vertices, we have $\sinv'_k(T)\leq (\frac{4}{3}+\epsilon)k$. 
%In other words, $M'_k \leq \frac{4}{3}k+o(k)$.
%\end{theorem}
\begin{proof}
Let $\epsilon >0$. 
We shall prove that there is an integer $k^*$ such that for every $k \geq k^*$ and for every tournament $T$ on at least $2k+1$ vertices, we have $\sinv'_k(T)\leq (\frac{4}{3}+\epsilon)k$. 

By Theorem~\ref{mkstrich}, there is an integer $k_0$ such that for every $k \geq k_0$ and for every tournament $T$ on at least $2k+1$ vertices, we have $\sinv'_k(T)\leq (\frac{4}{3}+\frac{\epsilon}{2})k$. Further, let $k_1$ be an integer such that $\frac{\epsilon}{2}k \geq 6\log k\sqrt{k}+1$, $\sqrt{k}\geq 3 \log k$ and $\log k \geq 5$ hold for all $k \geq k_1$. 
Set $k^*= \max\{k_0,k_1\}$.


We now fix an integer $k \geq k^*$ and prove that for every tournament $T$ on $n \geq 2k+1$ vertices, we have $\sinv'_k(T)\leq (\frac{4}{3}+\frac{\epsilon}{2})k+\min\{n-(2k+1),6\log k\sqrt{k}+1\}$ from which the statement follows immediately. Suppose that $T$ is a tournament that does not satisfy this statement and whose number $n$ of vertices is minimum with respect to this property.
\begin{case}
$n > 4k-2$. 
\end{case}
By Proposition~\ref{4k-2}, %\FH{on a plus cette proposition} 
there is some $v \in V(T)$ with $\min\{d_T^+(v),d_T^-(v)\}\geq k$. By the minimality of $T$, there is a collection $\mathcal{X}$ of $(\frac{4}{3}+\frac{\epsilon}{2})k+\min\{(n-1)-(2k+1),6\log k\sqrt{k}+1\}$ subsets of $V(T-v)$ such that $\Inv(T-v;\mathcal{X})$ is $k$-arc-strong. By Proposition~\ref{lem:kstrong+}, we obtain that $\Inv(T;\mathcal{X})$ is $k$-arc-strong. As $(\frac{4}{3}+\frac{\epsilon}{2})k+\min\{(n-1)-(2k+1),6\log k\sqrt{k}+1\} \leq (\frac{4}{3}+\frac{\epsilon}{2})k+\min\{n-(2k+1),6\log k\sqrt{k}+1\}$, we obtain a contradiction.
\begin{case}
$2k+1+6\log k \sqrt{k}< n \leq 4k-2$. 
\end{case}
Let $A$ be an arbitrary subset of $V(T)$ of size $2k+1+\lfloor6\sqrt{k} \log k\rfloor$ and let $B=V(T)\setminus A$. We now choose a subset $X$ of $A$ uniformly at random and let $T'=\Inv(T;B \cup X)$. Now consider some $b \in B$. For every $a \in A$, let $\Gamma_a$ be the probability that $ab \in A(T')$ and observe that $\Gamma=\sum_{a \in A}\Gamma_a$ is exactly $|N^-_{T'}(b)\cap A|$.
Due to the uniform choice of $X$, we have $\Pr[ab \in A(T)]=\frac{1}{2}$ and that the $\Gamma_a$ are independent. This yields $\Gamma \sim \Bin(2k+1+\lfloor6\sqrt{k} \log k\rfloor,\frac{1}{2})$. By Chernoff's Bound (Proposition~\ref{chernoff}), we obtain 

\begin{align*}
\Pr[\Gamma<k]&\leq \Pr\left[\Gamma<\left(1-\frac{ \sqrt{k} \log k}{k}\right)\frac{1}{2}\left(2k+1+\lfloor6\sqrt{k} \log k\rfloor\right)\right]\\
&\leq \exp(-\log^2(k)/2)\\
&\leq \exp(-2 \log k) ~~~~~~~ \mbox{because $k\geq 16$}\\
&\leq \exp(-2 \ln k)\\
& = \frac{1}{k^2}.
\end{align*} 
Similarly, we obtain $\Pr[|N^+_{T'}(b)\cap A|<k]\leq \frac{1}{k^2}$. As $B$ contains at most $(4k-3)-(2k+1+\lfloor6\sqrt{k} \log k\rfloor)\leq 2k$ elements, the Union Bound (Proposition~\ref{union}) yields that the probability that there is at least one $b \in B$ with $\min\{|N^-_{T'}(b)\cap A|,|N^+_{T'}(b)\cap A|\}\leq k$ is at most $\frac{2}{k^2}2k=\frac{4}{k}<1$, as $k\geq 5$.
Hence, with positive probability, we have $\min\{|N^-_{T'}(b)\cap A|,|N^+_{T'}(b)\cap A|\}\geq k$ for all $b \in B$. 
Thus there is $X_0$ such that every vertex of $B$ has in- and out-degree at least $k$ in $T'_0= \Inv(T;B \cup X_0)$.
Further, by the minimality of $T$ and as $|V(T_0)|-2k+1=6 \sqrt{k}\log k$, there is a collection $\mathcal{X}$ of $\sinv'_k(T_0\langle A\rangle)\leq (\frac{4}{3}+\frac{\epsilon}{2})k+\lfloor6\log k\sqrt{k}\rfloor$ sets such that $\Inv(T'_0\langle A \rangle;\mathcal{X})$ is $k$-arc-strong. Hence, by Proposition~\ref{lem:kstrong+}, we obtain that $\Inv(T;\{X_0\} \cup \mathcal{X})$ is $k$-arc-strong, a contradiction.  

%\reviewer{On the penultimate line of Case 2, point out that we are using the $n-(2k+1)$ bound
%in the min function, and add an $X$ to $\Inv(T''\langle A_i \rangle)$.}

\begin{case}
$2k+2\leq n \leq 2k+6\sqrt{k} \log k$. 
\end{case}
Let $v \in V(T)$ be an arbitrary vertex. We can then find a set $X \subseteq V(T)$ such that for $T'=\Inv(T;X)$, we have $\min\{d_{T'}^+(v),d_{T'}^-(v)\}\geq k$. Further, by the minimality of $T$, there is a collection $\mathcal{X}$ of $\sinv'_k(T'-v)\leq (\frac{4}{3}+\frac{\epsilon}{2})k+n-(2k+1)-1$ sets such that $\Inv(T'-v; \mathcal{X})$ is $k$-arc-strong. Hence by Proposition~\ref{lem:kstrong+}, we obtain that $\Inv(T;\{X\}\cup \mathcal{X})$ is $k$-arc-strong, a contradiction.
\end{proof}

\section{Upper bounds on \texorpdfstring{$m_k(n)$}{mk(n)}\label{sec:upper_bound_Mkn}}
%%%%%%%%%%%%%%%%%%%%%%%%%%%%%%%%%%%%
In this section, we prove several results showing that tournaments on significantly more than $2k$ vertices can be made $k$-strong by a small number of inversions. More precisely, in Subsection~\ref{first} we prove Theorem~\ref{thm:s1} and Proposition \ref{nk1unter}, in Subsection~\ref{nk10} we prove Theorem~\ref{nk1}, in Subsection~\ref{nk3sec} we prove Theorem~\ref{nk3}, and in Subsection~\ref{epsgross}, we prove Theorem~\ref{thm:2+eps}.

 
 \subsection{First upper bounds on \texorpdfstring{$m_k(n)$}{mk(n)}}\label{first}
%%%%%%%%%%%%%%%%%%%%%%%%%%%%%

In this subsection, we first establish that, for every fixed $k$, every tournament which is sufficiently large in comparison to $k$ can be made $k$-strong by a single inversion. While this result ( Theorem~\ref{thm:s1}) is clearly weaker than Theorem~\ref{nk1}, it justifies some of the notation used later on, and may serve as a warm-up exercise of the more involved proof of Theorem~\ref{nk1}.

\begin{theorem}\label{thm:s1}
If $n \geq (2k-1)2^{2k}$, then $\sinv'_k(T) \leq \sinv_k(T) \leq 1$.
\end{theorem}
\begin{proof}
Let $T$ be a tournament of order $n\geq (2k-1)2^{2k}$.

  It is easy and well-known that if $D$ is an acyclic digraph, $x$ a source in $D$, and $D-x$ is contained (as a subdigraph) in every tournament of order $n$, then $D$ is contained (as a subdigraph) in every tournament of order $2n$.  
An easy induction yields that $T$ contains three sets
  $A_1, A_2, A_3$ such that $A_1\Ra (A_2\cup A_3)$ and
  $A_2\Ra A_3$ with $|A_1|=|A_3| = k$ and $|A_2|=2k-1$.
Set $A=A_1\cup A_2 \cup A_3$.
Let $I$ be the set of vertices in $V(T)\setminus A$ that have either fewer than $k$ out-neighbours in $A$ or fewer than $k$ in-neighbours in $A$.
Let $X=A_1\cup A_3\cup I$.
Let us prove that $T'=\Inv(T; X)$ is $k$-strong.

In $T'$, we have $A_1 \Ra A_2 \Ra A_3\Ra A_1$. Since the three sets $A_1$, $A_2$, and $A_3$ have size at least $k$, the tournament $T'\langle A\rangle$ is $k$-strong.
Now consider a vertex $v$ in $V(T)\setminus A$.
If $v\notin I$, then no arcs incident to $v$ have been reversed so its in- and out-degree have been unchanged and are at least $k$ by definition of $I$.
If $v\in I$, then all the arcs between $v$ and $A_1\cup A_3$ have been reversed and those between $v$ and $A_2$ are unchanged.
If $v$ has fewer than $k$ out-neighbours in $A$ in $T$, then $|N^+_{T'}(v)\cap A| \geq |N^-_T(v) \cap (A_1\cup A_3)| \geq 2k - d^+_T(v) \geq k$, and $|N^-_{T'}(v)\cap A|\geq |N^-_T(v) \cap A_2| \geq 2k-1 - d^+_T(v) \geq k$.
Similarly, if $v$ has fewer than $k$ in-neighbours in $A$ in $T$, then 
$|N^+_{T'}(v)\cap A|\geq k$ and $|N^-_{T'}(v)\cap A|\geq k$.
Thus, by 
Lemma~\ref{lem:kstrong+}, $T'$ is $k$-strong.
Hence $\sinv_k(T) \leq 1$.
\end{proof}






\subsection{Linear upper and lower bounds on \texorpdfstring{$N_k(1)$}{Nk(1)}}\label{nk10}
%%%%%%%%%%%%%%%%%%%%%%%%%%%%%%%%%%%%%%%%%%

In this subsection, we shall prove  Theorem~\ref{nk1} which states that $N_k(1)\leq 19k -2$. To prove it we need some preliminaries.




For a digraph $D$, let $\sigma=(v_1,v_2, \ldots, v_n)$ be an ordering of the vertices of $D$. An arc $v_iv_j$ is {\bf forward} (according to $\sigma$) if $i<j$ and {\bf backward} (according to $\sigma$) if $j<i$.
A {\bf median order} of $D$ is an ordering of the vertices of $D$ with the maximum number of forward arcs, or equivalently the minimum number of backward arcs.

Let us note basic well-known properties of median orders of
tournaments (the ``feedback property'' in \cite{HaTh00}).

\begin{lemma}\label{lem:median}
 Let $T$ be a tournament and $(v_1,v_2, \ldots, v_n)$ a
median order of $T$. Then, for any two indices $i,j$ with $1 \leq i <
j \leq n$:
\medskip
\begin{enumerate}
\item[\rm (M1)] $(v_i,v_{i+1},\ldots,v_j)$ is a median order of the
  induced subtournament $T\langle \{v_i,v_{i+1},\ldots,v_j\}\rangle$.\\
  
\item[\rm (M2)] the vertex $v_i$ is an in-neighbour of at least half of the vertices
  $v_{i+1},v_{i+2},\ldots,v_j$, and the vertex $v_j$ is  an out-neighbours of at least half of the vertices $v_i,v_{i+1},\ldots,v_{j-1}$.  In particular, for every $1 \leq i \leq n-1$, $v_i v_{i+1} \in A(T)$. 

\end{enumerate}
\end{lemma}

Given two vertices $u,v$ of some digraph $D$, we say that $v$ is {\bf reachable} from $u$ if $D$ contains a directed $uv$-path. For $v \in V(D)$, we use $R^+_D(v)$ (resp. $R^-_D(v)$) for the set of vertices which are reachable from $v$ in $D$ (resp. from which $v$ is reachable in $D$). %Let $v$ be a vertex of a digraph $D$. We denote by $R^+_D(v)$ (resp. $R^-_D(v)$) the set of vertices which are {\bf reachable} from $v$ (resp. vertices that can reach $v$) in $D$, those are the vertices $w$ such that there is a directed $(v,w)$-path (resp.~$(w, v)$-path) in $D$.
Note that $v\in R^+_D(v) \cap R^-_D(v)$.


\begin{lemma}\label{lem:n-2F}
Let $T$ be a tournament with median order $(v_1, v_2, \ldots , v_n)$.
Let $F$ be a subset of vertices such that $v_1\notin F$. 
Then $|R^+_{T-F}(v_1)| \geq n - 2 |F|$.
\end{lemma}
\begin{proof}
We prove the result by induction on $n+|F|$, the result holding trivially by (M2) if $|F|=0$.

If all the out-neighbours of $v_1$ in $T$ are in $|F|$, then by (M2), $|N_T^-(v_1)| \leq |N_T^+(v_1)|  \leq |F|$. Hence $n-1 \leq 2|F|$, and the result holds.
Henceforth we may assume that $v_1$ has an out-neighbour not in $F$.
Let $i_0$ be the smallest index of such a vertex.
Let $T_0= T\langle \{v_1, \dots , v_{i_0-1}\}\rangle$, $T_1= T\langle \{v_{i_0}, \dots , v_{n}\}\rangle$,  $F_0=F\cap V(T_0)$ and $F_1=F\cap V(T_1)$.
By (M1), $(v_1, v_2, \ldots , v_{i_0-1})$ is a median order of $T_0$ and $(v_{i_0},  \ldots , v_n)$ is a median order of $T_1$.
By definition of $i_0$, all out-neighbours of $v_1$ in $T_0$ are in $F_0$. Thus, as above, we have $i_0-2 \leq 2|F_0|$.
By the induction hypothesis, $|R^+_{T_1-F_1}(v_{i_0})| \geq n - i_0 + 1 - 2 |F_1|$.
Now $R^+_{T_1-F_1}(v_{i_0}) \cup \{v_1\} \subseteq R^+_{T-F}(v_1)$. Hence
 $$|R^+_{T-F}(v_1)| \geq |R^+_{T_1-F_1}(v_{i_0})| +1 \geq n - i_0 +2 - 2|F_1| \geq n - 2|F_0| - 2|F_1| = n - 2|F|.$$
\end{proof}

 
%\begin{lemma}\label{median order bounds}
 %   Let $T$ be a tournament with median order $(v_1, v_2, \ldots , v_n)$. For any $i\in [n]$ and any $F \subseteq V(T)\setminus \{v_i\}$, we have $|R^+_{T-F}(v_i)| \geq n + 1 - i - 2 |F|$, and $R^-_{T-F}(v_i) \geq i - 2 |F|$.
%\end{lemma}

%\reviewer{, I think Lemma 6.3 can be deleted (it is only used once, in the proof of Claim
%6.4.1, and I would suggest using Lemma 6.2 there). Perhaps mention property (M1)
%whenever you apply Lemma 6.2 though}

%\begin{proof}
 %   By (M1), $(v_i, \dots, v_n)$ is a median order of $T\langle \{v_i, \dots, v_n\}\rangle$ on which we can apply Lemma~\ref{lem:n-2F} to obtain the bound on $R^+_{T-F}(v_i)$. Symmetrically, $(v_i, v_{i-1}, \dots, v_1)$ is a median order of the converse of $T\langle \{v_1, \dots, v_i \} \rangle$, and applying Lemma~\ref{lem:n-2F} to the converse of $T$ yields the bound on $R^-_{T-F}(v_i)$.
%\end{proof}

We are now ready to prove the main lemma of the proof of Theorem~\ref{nk1}.



% \JD{Vraiment pas mal ! On peut enlever peut être 8k au thm en remarquant qu'on a modifié exactement 2k (et pas les 6k) sommets dans A (resp. B) du coup au lieu d'appeler~\ref{median order bounds} sur F, on l'applique sur $Y \cup A_0 \cup A_1$, puis on s'assure d'enlever les sommets en trop de $A$. Mais je sais pas si ça vous semble valoir le coup. Le résultat donne $|R_b| \geq n - 6k - (i-1)  - 2(k-1) - 2(2k) - (4k - i) = n - 16k $ si je ne me trompe pas}

% \JD{D'ailleurs pour $N_k(6)$ on peut appliquer exactement le 5.15 au lieu du 5.17, mais en laissant A et B inchangé. Ça donne une meilleure borne sur 5.15 (le cas 4 devient le même que le 3) en principe 4k pour $|A|$? et on peut appliquer la démo de 5.19 ensuite. On se retrouve avec $12k-2$ pour $N_k(3)$.}
% \FHO{Je crois que tu as raison, mais on verra plus tard.}
\begin{lemma}\label{6ab}
    Let $k$ be a positive integer, let $T$ be a tournament on $12k$ vertices and let $(A,B)$ be a bipartition of $V(T)$ such that $|A|=|B|=6k$. 
    Then there is a set $X \subseteq V(T)$ with $|X \cap A|=|X \cap B|=2k$ such that for $T'=\Inv(T;X)$ and for any $Y \subseteq V(T)$ with $|Y|\leq k-1$, we have that $T'-Y$ contains a directed path from $a$ to $B\setminus Y$ for every $a \in A\setminus Y$,
    and $T'-Y$ contains a directed path from $A\setminus Y$ to $b$ for every $b \in B\setminus Y$.
\end{lemma}
\begin{proof}
    Let $(a_1,\ldots,a_{6k})$ be a median order of $T\langle A \rangle$ and let $(b_1,\ldots,b_{6k})$ be a median order of $T\langle B \rangle$. 
    Let $A_0$ be the set of vertices in $\{a_{4k+1},\ldots,a_{6k}\}$ which have fewer than $k$ out-neighbours in $B$ in $T$. Further, let $A_1=\{a_1,\ldots,a_{2k-|A_0|}\}$. 
    Observe that $|A_0 \cup A_1|=|A_0|+|A_1|=2k$. 
    Similarly, let $B_0$ be the set of vertices in $\{b_1,\ldots,b_{2k}\}$ that have fewer than $k$ in-neighbours in $A$ and let $B_1=\{b_{4k+|B_0|+1},\ldots,b_{6k}\}$. 
    Observe that $|B_0 \cup B_1|=|B_0|+|B_1|=2k$. Let $X=A_0 \cup A_1 \cup B_0 \cup B_1$ and let $T'=\Inv(T;X)$. 
    Consider any $Y \subseteq V(T)$ with $|Y|\leq k-1$. 
    In order show that $T'$ has the desired properties, by symmetry, it suffices to prove that $T'-Y$ contains a directed path from $a$ to $B\setminus Y$ for every $a \in A\setminus Y$. 
    Suppose by contradiction that this is not true. Then there is a largest integer $i \in [6k]$ such that $a_i \in A\setminus Y$ and $T'-Y$ does not contain a directed path from $a_i$ to $B\setminus Y$. We will distinguish several cases.

\setcounter{case}{0}
    \begin{case}
        $i \in \{4k+1,\ldots,6k\}$ and $a_i \in A\setminus A_0 $.
    \end{case}
    In this case, by the choice of $A_0$, we have 
    \begin{align*}
        |(N_{T'}^+(a_i)\cap B)\setminus Y|&\geq |N_{T'}^+(a_i)\cap B|-|Y|\\
        &= |N_{T}^+(a_i)\cap B|-|Y|\\
        &\geq k-(k-1)\\
        &=1,
    \end{align*}
    so $a_i$ has an out-neighbour in $B\setminus Y$ in $T'-Y$, a contradiction.
%\end{proof}
    \begin{case}
        $i \in \{4k+1,\ldots,6k\}$ and $a_i \in A_0$.
    \end{case}
    In this case, by the choice of $A_0$, we have
    \begin{align*}
        |(N_{T'}^+(a_i)\cap B)\setminus Y|&\geq |N_{T'}^+(a_i)\cap B|-|Y|\\
        &\geq|N_{T'}^+(a_i)\cap (B_0 \cup B_1)|-|Y|\\
        &=|B_0 \cup B_1|-|N_{T'}^-(a_i)\cap (B_0 \cup B_1)|-|Y|\\
        &=|B_0 \cup B_1|-|N_{T}^+(a_i)\cap (B_0 \cup B_1)|-|Y|\\
        &\geq|B_0 \cup B_1|-|N_{T}^+(a_i)\cap B|-|Y|\\
        &\geq 2k-(k-1)-(k-1)\\
        &=2,
    \end{align*}
    so $a_i$ has an out-neighbour in $B\setminus Y$ in $T'-Y$, a contradiction.
    
    \begin{case}
        $i \in \{2k-|A_0|+1,\ldots,4k\}$.
    \end{case}
    As $(a_1,\ldots,a_{6k})$ is a median order of $T\langle A \rangle$, and by (M2) applied to $T\langle A \rangle$, we have 
     \begin{align*}
    |(N_{T'}^+(a_i)\cap \{a_{i+1},\ldots,a_{6k}\})\setminus Y|&\geq |N_{T'}^+(a_i)\cap \{a_{i+1},\ldots,a_{6k}\}|-|Y|\\
    &\geq |N_{T}^+(a_i)\cap \{a_{i+1},\ldots,a_{6k}\}|-|Y|\\
    &= \frac{1}{2} |\{a_{i+1},\ldots,a_{6k}\}|-|Y|\\
    & \geq \frac{1}{2} (6k-i)-(k-1)\\
    &=2k+1-\frac{i}{2}\\
    &\geq 2k+1-2k\\
    &=1.
    \end{align*}
    Hence there is some $j>i$ such that $a_j\in A\setminus Y$ and $T'-Y$ contains the arc $a_ia_j$. By the maximality of $i$, there is a directed path from $a_j$ to $B\setminus Y$ in $T'-Y$. Hence $T'-Y$ also contains a directed path from $a_i$ to $B\setminus Y$, a contradiction.
    \begin{case}
    $i \in \{1,\ldots,2k-|A_0|\}$.
    \end{case}
     As $(a_1,\ldots,a_{6k})$ is a median order of $T\langle A \rangle$ and by (M2) applied to $T\langle A \rangle$, we have 
     \begin{align*}
    |(N_{T'}^+(a_i)\cap \{a_{i+1},\ldots,a_{6k}\}) \setminus Y| & \geq |N_{T'}^+(a_i)\cap \{a_{i+1},\ldots,a_{6k}\}|-|Y|\\
    &\geq |N_{T}^+(a_i)\cap \{a_{i+1},\ldots,a_{6k}\}|-|(A_0 \cup A_1)\cap \{a_{i+1},\ldots,a_{6k}\}|-|Y|\\
    &\geq \frac{1}{2} |\{a_{i+1},\ldots,a_{6k}\}|-|(A_0 \cup A_1)\cap \{a_{i+1},\ldots,a_{6k}\}|-|Y|\\
    & \geq \frac{1}{2} (6k-i)-(2k-i)-(k-1)\\
    &=\frac{i}{2}+1\\
    &\geq 1.
    \end{align*}
    Hence there is some $j>i$ such that $a_j\in A\setminus Y$ and $T'-Y$ contains the arc $a_ia_j$. By the maximality of $i$, there is a directed path from $a_j$ to $B\setminus Y$ in $T'-Y$. Hence $T'-Y$ also contains a directed path from $a_i$ to $B\setminus Y$, a contradiction.
\end{proof}


%\nkone*
%We are now ready to prove Theorem~\ref{nk1}.
%\begin{proof}
    %Let $T$ be a tournament of order $n \geq 28k-5$ and let $(v_1,\ldots,v_n)$ be a median order of $V(T)$. Let $A=\{v_{n-6k+1},\ldots,v_n\}$ and $B=\{v_1,\ldots,v_{6k}\}$. Observe that $A$ and $B$ are disjoint as $n \geq 28k-5$. By Lemma~\ref{6ab}, there is a set $X \subseteq A \cup B$ such that in the tournament $T_0=\Inv(T\langle A \cup B \rangle;X)$, for any $Y \subseteq V(T_0)$ with $|Y|\leq k-1$, we have that $T_0-Y$ contains a directed path from $a$ to $B\setminus Y$ for every $a \in A\setminus Y$ and $T_0-Y$ contains a directed path from $A\setminus Y$ to $b$ for every $b \in B\setminus Y$. Let $T'=\Inv(T;X)$. We will show that $T'$ is $k$-strong. To this end, let $Y \subseteq V(T)$ with $|Y|\leq k-1$.

    %\begin{claim}\label{nett}
        %Let $b \in B\setminus Y$ and $a \in A\setminus Y$. Then $T'-Y$ contains a directed path from $b$ to $a$.
    %\end{claim}
    %\begin{proofclaim}
       % Let $R_b$ be the set of vertices in $\{v_{6k+1},\ldots,v_{n-6k}\}$ which are reachable from $b$ in $T'-Y$ and let $R_a$ be the set of vertices in $\{v_{6k+1},\ldots,v_{n-6k}\}$ from which $a$ is reachable in $T'-Y$. By construction, there is some $i \in \{1,\ldots,6k\}$ such that $b=v_i$. Let $F=Y \cup \{v_{i+1},\ldots,v_{6k}\}$. 
        %\begin{align*}
        %|R_b|&= |R^+_{T'-Y}(b)\cap \{v_{6k+1},\ldots,v_{n-6k}\}|\\
             %&\geq |R^+_{T'\langle\{v_i,\ldots,v_{n-6k}\}\rangle -Y} (b)\cap \{v_{6k+1},\ldots,v_{n-6k}\}|\\
            % &\geq |R^+_{T'\langle\{v_i,\ldots,v_{n-6k}\}\rangle-F} (b) \setminus \{v_i\}|\\ 
            %&\geq |R^+_{T'\langle\{v_i,\ldots,v_{n-6k}\}\rangle-F} (b)|-1 ~~~~~\mbox{(by Lemma~\ref{median order bounds})}\\ 
            %&\geq |\{v_i,\ldots,v_{n-6k}\}|-2|F|-1\\
           % &\geq (n-6k-(i-1))-2((6k-i)+(k-1))-1\\
            %&\geq n-20k+i+2\\
            %&\geq n-20k+3.
        %\end{align*}
        %A similar argument shows that $|R_a|\geq n-20k+3$. As $R_a \cup R_b \subseteq \{v_{6k+1},\ldots,v_{n-6k}\}$, we obtain $|R_a \cap R_b|=|R_a|+|R_a|-|R_a\cup R_b|\geq 2(n-20k+3)-(n-12k)=n-28k+6\geq 1$. Hence there is a vertex $v^* \in \{v_{6k+1},\ldots,v_{n-6k}\}\cap R_a \cap R_b$. By definition, $T'-Y$ contains a directed path from $b$ to $v^*$ and a directed path from $v^*$ to $a$. Hence $T'-Y$ contains a directed path from $b$ to $a$.
    %\end{proofclaim}
    %We now show that $B\setminus Y$ is strong in $T'-Y$. Let $b,b' \in B\setminus Y$. By the definition of $X$, there is some $a \in A\setminus Y$ such that $T_0-Y$, and hence $T'-Y$, contains a directed path from $a$ to $b'$. Further, by Claim~\ref{nett}, there is a path from $b$ to $a$ in $T'-Y$. Hence $T'-Y$ contains a path from $b$ to $b'$. This shows that $B\setminus Y$ is strong in $T'-Y$. Similarly, $A\setminus Y$ is strong in $T'-Y$. Next, by the choice of $X$, we have that $T_0-Y$ and hence $T'-Y$ contains a path from $A\setminus Y$ to $B\setminus Y$ and by Claim~\ref{nett}, we have that $T'-Y$ contains a path  from $B\setminus Y$ to $A\setminus Y$. Hence $(A \cup B)\setminus Y$ is strong in $T'-Y$. 
    %Now consider the vertex $v_i$ for some $i \in \{6k+1,n-6k\}$. By Lemma~\ref{median order bounds}, we have $|R^-_{T\langle\{v_1,\ldots,v_i\}\rangle-Y}(v_i)|\geq i-2(k-1)$. As $R^-_{T\langle\{v_1,\ldots,v_i\}\rangle-Y}(v_i)\subseteq \{v_1,\ldots,v_i\}$, there is some $b \in B\setminus Y$ such that $b \in R^-_{T\langle\{v_1,\ldots,v_i\}\rangle-Y}(v_i)$. Hence $T\langle\{v_1,\ldots,v_i\}\rangle-Y$ contains a directed $(b,v_i)$-path $P$. Let $b'$ be the last vertex of this path which is contained in $B\setminus Y$. Then the $(b',v_i)$-path which is contained in $P$ also exists in $T'-Y$. Similarly, $T'-Y$ contains a directed path from $v_i$ to $A$. Hence $T'-Y$ is strongly connected. This finishes the proof.
    We are now ready to prove Theorem~\ref{nk1}, which states $N_k(1)\leq 19k-2$.
\begin{proof}[Proof of Theorem~\ref{nk1}]
    Assume $n\geq 19k-2$.
    Let $(v_1, \ldots , v_n)$ be a median order of $T$.
    Let $B=\{v_1, \dots , v_{6k}\}$, $A=\{v_{n-6k+1}, \ldots , v_{n}\}$ and $C=V(T)\setminus (A\cup B)$. We have $|C| \geq 7k-2$.
    By Lemma~\ref{6ab}, applied to $T\langle A\cup B\rangle$, there is a subset $X$ of $A\cup B$ such that,
    for $T_1 = \Inv(T;X)$, we have
    \begin{itemize}
        \item[(i)] $|X \cap B|=|X \cap A|=2k$ ; 
        \item[(ii)] for any $Y \subseteq V(T)$ with $|Y|\leq k-1$, $T_1-Y$ contains a directed path from $a$ to $B\setminus Y$ for every $a \in A\setminus Y$; 
        \item[(iii)] for any $Y \subseteq V(T)$ with $|Y|\leq k-1$, $T_1-Y$ contains a directed path from $A\setminus Y$ to $b$ for every $b \in B\setminus Y$.
    \end{itemize}
    %Let $T_1=\Inv(T;X)$.
    Let us now prove that $T_1$ is $k$-strong, which implies $\sinv_k(T) \leq 1$.
    %Note that $T\langle A \cup C \rangle$, as well as $T\langle B \cup C \rangle$ are unchanged by the inversions.
    Let $F$ be a set of at most $k-1$ vertices of $T_1$.
    We first need the following intermediate result.
    \begin{claim}\label{nett}
        Let $b \in B\setminus F$ and $a \in A\setminus F$. Then $T_1-F$ contains a directed path from $b$ to $a$.
    \end{claim}
    
    
   
    \begin{proofclaim}
    %Let $R_b$ be the set of vertices in $\{v_{6k+1},\ldots,v_{n-6k}\}$ which are reachable from $b$ in $T'-Y$ and let $R_a$ be the set of vertices in $\{v_{6k+1},\ldots,v_{n-6k}\}$ from which $a$ is reachable in $T'-Y$.
    By construction, there is some $i \in \{1,\ldots,6k\}$ such that $b=v_i$. Let $S=((X \cap B)\cup F)\cap \{v_{i+1},\ldots,v_{n-6k}\}$ and let $T'=T\langle\{v_{i},\ldots,v_{n-6k}\}\rangle$. By (M2), every vertex in $S$ and every vertex in $C\setminus F$ is reachable from $b$ in $T'$. On the other hand, there is obviously no vertex in $S$ reachable from $b$ in $T'-S$. % Let $F=Y \cup \{v_{i+1},\ldots,v_{6k}\}$.
    Further, by (M1), $(v_i,\ldots,v_{n-6k})$ is a median order of $T'$.
    By Lemma~\ref{lem:n-2F} applied to $T'$ and $S$, this yields  
    \begin{align*}
        2|S|&\geq |R^+_{T'}(b)|-|R^+_{T'-S}(b)|\\
        &= |R^+_{T'}(b)\cap B|-|R^+_{T'-S}(b)\cap B|\\
        & \qquad +|R^+_{T'}(b)\cap (C \cap F)|-|R^+_{T'-S}(b)\cap (C \cap F)|\\
        & \qquad +|R^+_{T'}(b)\cap (C \setminus F)|-|R^+_{T'-S}(b)\cap (C \setminus F)|\\
        &\geq  |S| +|C \setminus F|-|R^+_{T'-S}(b)\cap (C \setminus F)|.
    \end{align*}
    For every $v \in C \setminus F$, if $v$ is reachable from $b$ in $T'-S$, then $v$ is clearly reachable from $b$ in $T_1-F$. Since $|X \cap B|\leq 2k$ and $|F|\leq k-1$, we obtain 
    \begin{align*}
        |R^+_{T_1-F}(b)\cap (C \setminus F)|&\geq  |R^+_{T'-S}(b)\cap (C \setminus F)|\\
        &\geq |C \setminus F|-|S|\\
        &\geq |C \setminus F|-(|X \cap B|+|F|)\\
        &\geq |C \setminus F|-\big(2k+(k-1)\big)\\
        &= |C \setminus F|-(3k-1).
    \end{align*}
    A similar argument shows that $ |R^-_{T_1-F}(a)\cap (C \setminus F)|\geq |C \setminus F|-(3k-1)$. As $|C|\geq n-12k\geq 7k-2$, we obtain
     \begin{align*}
        |(R^+_{T_1-F}(b)\cap (C \setminus F)) \cap (R^-_{T_1-F}(a)\cap (C \setminus F))|&
        =|R^+_{T_1-F}(b)\cap (C \setminus F)|+|R^-_{T_1-F}(a)\cap (C \setminus F)|\\
        &\qquad-|(R^+_{T_1-F}(b)\cap (C \setminus F)) \cup (R^-_{T_1-F}(a)\cap (C \setminus F))|\\
        & \geq |R^+_{T_1-F}(b)\cap (C \setminus F)|+|R^-_{T_1-F}(a)\cap (C \setminus F)|-|C \setminus F|\\
        &\geq 2\big(|C \setminus F|-(3k-1)\big)-|C \setminus F|\\
        &=|C \setminus F|-2(3k-1)\\
        &\geq |C|-|F|-(6k-2)\\
        & \geq (7k-2)-(k-1)-(6k-2)\\
        & \geq 1.
    \end{align*}
    Hence there is a vertex $v^* \in \{v_{6k+1},\ldots,v_{n-6k}\}\cap (R^+_{T_1-F}(b)\cap (C \setminus F)) \cap (R^-_{T_1-F}(a)\cap (C \setminus F))$. By definition, $T_1-F$ contains a directed path from $b$ to $v^*$ and a directed path from $v^*$ to $a$. Hence $T_1-F$ contains a directed path from $b$ to $a$.
    \end{proofclaim}
      
    We are now ready to show that $T_1-F$ is strong.
    Let $x$ and $y$ be two vertices in $T_1-F$. It suffices to show that $y$ is reachable from $x$ in $T_1-F$.
    
    %\FHO{We first show that there is a vertex $u \in B\setminus F$ that is reachable from $x$ in $T_1-F$. Clearly, we may suppose that $x \in A\cup C$}
    We first show that there is a path from $x$ to $B \setminus F$. Clearly, we may suppose that $x \in (A \cup C)\setminus F.$ %As $|C| \geq 3k-2 \geq 2|F|+1$, if $x \in B$, Lemma~\ref{lem:n-2F} asserts the existence of a vertex $c \in C\setminus F$ reachable from $x$ in $T_1$.
    Then, since $|A| > 2|F|$, Lemma~\ref{lem:n-2F} implies that
    there is a vertex $x' \in A\setminus F$ reachable from $x$ in $T-F-B$, and so in $T_1-F$.
    % Then, Lemma~\ref{lem:n-2F} asserts that from any vertex of $(A \cup C)\setminus F$ one can reach a vertex of $A\setminus F$ in $T-F-B$ and hence also in $T_1-F$. Thus there exists a vertex $x' \in A\setminus F$ reachable from $x$ in $T_1-F$.
    By (ii), there is a directed path from $x'$ to a vertex $u$ in $B\setminus F$ in $T_1-F$.
    Hence there is a directed path $P_x$ from $x$ to $u$ in $T_1-F$.
    Similarly, by directional duality, in $T_1-F$, there is a directed path $P_y$ from a vertex $w\in A\setminus F$ to $y$.
    
    Finally, by Claim~\ref{nett}, there exists a path $Q$ from $u$ to $w$ in $T_1 - F$. Then
    $P_x Q P_y$ is a path from $x$ to $y$ in $T_1-F$.
    This proves that $T_1-F$ is strong.
\end{proof}





Finally, we give the proof of Proposition~\ref{nk1unter}, which states that $N_k(1)\geq 5k-2$ and $N_k'(1)\geq 4k-1$.
\begin{proof}[Proof of Proposition~\ref{nk1unter}]
    Let $T_1$ be a tournament of order $5k-3$ whose vertex set has a partition $(A,B,C)$ such that $T_1\langle A \rangle$ and $T_1\langle C\rangle$ are eulerian tournaments of order $2k-1$, and $A\Ra B\cup C$ and $B\Ra C$.
We shall prove that $\sinv_k(T_1) >1$.

Assume for a contradiction that there is a set $X$ of vertices such that $T_1'=\Inv(T_1;X)$ is $k$-strong.
Every vertex of $A$ (resp. $C$) has in-degree (resp. out-degree) $k-1$ in $T_1$, and so belongs to $X$.
Thus $A\cup C\subseteq X$, and so $C\Ra A$ in $T_1'$. Hence $T_1'-B$ is not strong. Since $|B|=k-1$, $T_1'$ is not $k$-strong, a contradiction.

\medskip

Now let $T_2$ be a tournament of order $4k-2$ whose vertex set has a partition $(A,B)$ such that $T_2\langle A \rangle$ and $T_2\langle B\rangle$ are eulerian tournaments of order $2k-1$, and $A\Ra B$.
We shall prove that $\sinv_k'(T_1) >1$.

Assume for a contradiction that there is a set $X$ of vertices such that $T_2'=\Inv(T_2;X)$ is $k$-arc-strong.
Every vertex of $A$ (resp. $B$) has in-degree (resp. out-degree) $k-1$ in $T_2$, and so belongs to $X$.
Thus $X=V(T_2)$, and so $T_2'$ is isomorphic to $T_2$. Hence $T_2'$ is not $k$-arc-strong, a contradiction.
\end{proof}

%\reviewer{Consider noting after the proof that e.g. for $k = 2$, $T'$ is $k$-arc-strong}






%\begin{proposition}\label{prop:N1-inf}
 % $N_k(1)\geq 5k-2$.  
%\end{proposition}



\subsection{Better upper bound on \texorpdfstring{$N_k(3)$}{Nk(3)}}\label{nk3sec}
%%%%%%%%%%%%%%%%%%%%%%%%%%%%%%%%%%%%%%%
This section is dedicated to the proof of Theorem~\ref{nk3}. The structure is analogous to the proof of Theorem~\ref{nk1} in Section~\ref{nk10}.
First we show a result which is very similar to Lemma~\ref{6ab}.

\begin{lemma}\label{lem:4ab}
    Let $k$ be a positive integer, $T$ a tournament on $8k$ vertices and $(A,B)$ a bipartition of $V(T)$ such that $|A|=|B|=4k$. 
    There is a family ${\cal X}$ of three subsets of $V(T)$ such that the following hold with $T'= \Inv(T ; {\cal X})$. 
    \begin{itemize}
        \item[(i)] $T'\langle A \rangle = T\langle A\rangle$ and $T'\langle B \rangle = T\langle B\rangle$ ; 
        \item[(ii)] for any $Y \subseteq V(T)$ with $|Y|\leq k-1$, $T'-Y$ contains a directed path from $a$ to $B\setminus Y$ for every $a \in A\setminus Y$; 
        \item[(iii)] for any $Y \subseteq V(T)$ with $|Y|\leq k-1$, $T'-Y$ contains a directed path from $A\setminus Y$ to $b$ for every $b \in B\setminus Y$.
    \end{itemize}
\end{lemma}
\begin{proof}
    Let $(a_1,\ldots,a_{4k})$ be a median order of $T\langle A \rangle$ and let $(b_1,\ldots,b_{4k})$ be a median order of $T\langle B \rangle$. Let $A_0$ be the set of vertices in $\{a_{2k+1},\ldots,a_{4k}\}$ which have fewer than $k$ out-neighbours in $B$ in $T$. Further, let $A_1=\{a_1,\ldots,a_{2k-|A_0|}\}$. Observe that $|A_0 \cup A_1|=|A_0|+|A_1|=2k$. Similarly, let $B_0$ be the set of vertices in $\{b_1,\ldots,b_{2k}\}$ that have fewer than $k$ in-neighbours in $A$ and let $B_1=\{b_{2k+|B_0|+1},\ldots,b_{4k}\}$. Observe that $|B_0 \cup B_1|=|B_0|+|B_1|=2k$. Now let $X_1=A_0 \cup A_1$, $X_2= B_0 \cup B_1$,
    $X_3=X_1\cup X_2$, and ${\cal X} = (X_i)_{i\in [3]}$ Let $T'=\Inv(T;{\cal X})$. Observe that $T'$ is obtained from $T$ by reversing the arcs between $X_1$ and $X_2$. In particular, (i) holds. 

    We only show (ii) as (iii) follows symmetrically.
    Let $Y \subseteq V(T)$ with $|Y|\leq k-1$.  Suppose for the sake of a contradiction that (ii) does not hold. There is a largest integer $i \in [4k]$ such that $a_i \in A\setminus Y$ and $T'-Y$ does not contain a directed path from $a_i$ to $B\setminus Y$. We will distinguish several cases.

\setcounter{case}{0}
    \begin{case}
        $i \in \{2k+1,\ldots,4k\}$ and $a_i \in A\setminus A_0$.
    \end{case}
    In this case, by the choice of $A_0$, we have 
    \begin{align*}
        |(N_{T'}^+(a_i)\cap B)\setminus Y|&\geq |N_{T'}^+(a_i)\cap B|-|Y|\\
        &= |N_{T}^+(a_i)\cap B|-|Y|\\
        &\geq k-(k-1)\\
        &=1,
    \end{align*}
    so $a_i$ has an out-neighbour in $B\setminus Y$ in $T'-Y$, a contradiction.
%\end{proof}
    \begin{case}
        $i \in \{2k+1,\ldots,4k\}$ and $a_i \in A_0$.
    \end{case}
    In this case, by the choice of $A_0$, we have
    \begin{align*}
        |(N_{T'}^+(a_i)\cap B)\setminus Y|&\geq |N_{T'}^+(a_i)\cap B|-|Y|\\
        &\geq|N_{T'}^+(a_i)\cap (B_0 \cup B_1)|-|Y|\\
        &=|B_0 \cup B_1|-|N_{T'}^-(a_i)\cap (B_0 \cup B_1)|-|Y|\\
        &=|B_0 \cup B_1|-|N_{T}^+(a_i)\cap (B_0 \cup B_1)|-|Y|\\
        &\geq|B_0 \cup B_1|-|N_{T'}^+(a_i)\cap B|-|Y|\\
        &\geq 2k-(k-1)-(k-1)\\
        &=2,
    \end{align*}
    so $a_i$ has an out-neighbour in $B\setminus Y$ in $T'-Y$, a contradiction.
    
    \begin{case}
        $i \in [2k]$.
    \end{case}
    As $(a_1,\ldots,a_{4k})$ is a median order of $T\langle A \rangle=T'\langle A\rangle$, by (M2) we have 
     \begin{align*}
    |(N_{T'}^+(a_i)\cap \{a_{i+1},\ldots,a_{4k}\})\setminus Y|&\geq |N_{T'}^+(a_i)\cap \{a_{i+1},\ldots,a_{4k}\}|-|Y|\\
    & = |N_{T}^+(a_i)\cap \{a_{i+1},\ldots,a_{4k}\}|-|Y|\\
    & \geq \frac{1}{2} |\{a_{i+1},\ldots,a_{4k}\}|-|Y|\\
    & \geq \frac{1}{2} (4k-i)-(k-1)\\
    & = k+1-\frac{i}{2}\\
    & \geq k+1-k\\
    & = 1.
    \end{align*}
    Hence there is some $j>i$ such that $a_j\in A\setminus Y$ and $T'-Y$ contains the arc $a_ia_j$. By the maximality of $i$, there is a directed path from $a_j$ to $B\setminus Y$ in $T'-Y$. Hence $T'-Y$ also contains a directed path from $a_i$ to $B\setminus Y$, a contradiction.
\end{proof}

We now prove the upper bound on $N_k(3)$, which we restate here.


\nkthree*

\begin{proof}
    Let $T$ be a tournament on  $n\geq 11 k-2$ vertices.
    Let $(v_1, \ldots , v_n)$ be a median order of $T$.
    Let $B=\{v_1, \dots , v_{4k}\}$, $A=\{v_{n-4k+1}, \ldots , v_{n}\}$ and $C=V(T)\setminus (A\cup B)$. We have $|C| \geq 3k-2$.
    By Lemma~\ref{lem:4ab} applied to $T\langle A\cup B\rangle$, there is a family  ${\cal X}$ of three subsets of $A\cup B$ such that
    for $T_1 = \Inv(T;\mathcal{X})$ we have
    \begin{itemize}
        \item[(i)] $T_1 \langle A \rangle = T\langle A\rangle$ and $T_1\langle B \rangle = T\langle B\rangle$ ; 
        \item[(ii)] for any $Y \subseteq V(T)$ with $|Y|\leq k-1$, $T_1 -Y$ contains a directed path from $a$ to $B\setminus Y$ for every $a \in A\setminus Y$; 
        \item[(iii)] for any $Y \subseteq V(T)$ with $|Y|\leq k-1$, $T_1 -Y$ contains a directed path from $A\setminus Y$ to $b$ for every $b \in B\setminus Y$.
    \end{itemize}
    
    Let $T_1=\Inv(T; \mathcal{X})$.
    Let us now prove that $T_1$ is $k$-strong, which implies $\sinv_k(T) \leq 3$.
    Note that $T\langle A \cup C \rangle$, as well as $T\langle B \cup C \rangle$ are unchanged by the inversions.
    Let $F$ be a set of at most $k-1$ vertices of $T_1$. Let us show that $T_1-F$ is strong.
    Let $x$ and $y$ be two vertices in $T_1-F$. It suffices to show that $y$ is reachable from $x$ in $T_1-F$.
    
    Let us first show that there is a vertex $u \in B\setminus F$ that is reachable from $x$ in $T_1-F$. It is trivial if $x\in B$, so we may suppose that $x \in A\cup C$.
    %As $|C| \geq 3k-2 \geq 2|F|+1$, if $x \in B$, Lemma~\ref{lem:n-2F} asserts the existence of a vertex $c \in C\setminus F$ reachable from $x$ in $T_1$.
    Lemma~\ref{lem:n-2F} asserts that from any vertex of $(A \cup C)\setminus F$ one can reach a vertex of $A\setminus F$ in $T_1-F$. Thus there exists a vertex $x' \in A\setminus F$ reachable from $x$ in $T_1-F$. By (ii), there is a directed path from $x'$ to a vertex $u$ in $B\setminus F$ in $T_1 - F$.
    Hence there is a directed path $P_x$ from $x$ to $u$ in $T_1-F$.
    Similarly, by directional duality, in $T_1-F$, there is a directed path $P_y$ from a vertex $w\in A\setminus F$ to $y$.
    
    %By Lemma~\ref{lem:n-2F}, $|R^+_{T\langle B \cup C\rangle - F}(u) \cap C| \geq |C|-2(k-1) > |C|/2$, and $|R^-_{T\langle A \cup C\rangle - F}(w) \cap C|  >|C|/2$. Thus there is a vertex of $C$ in $R^+_{T\langle B \cup C\langle - F}(u)\cap R^-_{T\langle A \cup C\rangle - F}(w)$, and so there exists a path $Q$ from $u$ to $w$ in $T_1 - F$. Then $P_x Q P_y$ is a path from $x$ to $y$ in $T_1-F$.
    
    By Lemma~\ref{lem:n-2F}, we have $|R^+_{T\langle B \cup C\rangle - F}(u) \cap (C\setminus F)| \geq |C|-2|F\cap (B \cup C)| = |C\setminus F| -|F\cap C|-2|F \cap B|.$ Similarly, we obtain $|R^-_{T\langle A \cup C\rangle - F}(w) \cap (C\setminus F)|  \geq |C\setminus F| -|F\cap C|-2|F \cap A|.$ 
    
    This yields 
    
    \begin{align*}
    |R^+_{T\langle B \cup C\rangle - F}(u)\cap R^-_{T\langle A \cup C\rangle - F}(w)|&=|(R^+_{T\langle B \cup C\rangle - F}(u)\cap(C \setminus F))\cap (R^-_{T\langle A \cup C\rangle - F}(w)\cap(C \setminus F))|\\
    &=|R^+_{T\langle B \cup C\rangle - F}(u)\cap(C \setminus F)|+| R^-_{T\langle A \cup C\rangle - F}(w)\cap(C \setminus F)|\\ & \qquad-|(R^+_{T\langle B \cup C\rangle - F}(u)\cap(C \setminus F))\cap (R^-_{T\langle A \cup C\rangle - F}(w)\cap(C \setminus F))|\\
    &\geq (|C\setminus F| -|F\cap C|-2|F \cap B|)+(|C\setminus F| -|F\cap C|-2|F \cap A|)-|C\setminus F|\\
    &=|C\setminus F|-2|F|\\
    &\geq|C|-3|F|\\
    &\geq (3k-2)-3(k-1)\\
    &=1.
    \end{align*}
%    By Lemma~\ref{lem:n-2F}, $|R^+_{T\langle B \cup C\rangle - F}(u) \cap (C\setminus F)| \geq |C|-2|F| = |C\setminus F| -|F| >  |C\setminus F|/2$ because $|C|\geq 3k-2$. Similarly, we obtain $|R^-_{T\langle A \cup C\rangle - F}(w) \cap (C\setminus F)|  >|C\setminus F|/2$. 
    Thus there is a vertex of $C\setminus F$ in $R^+_{T\langle B \cup C\rangle - F}(u)\cap R^-_{T\langle A \cup C\rangle - F}(w)$, and so there exists a directed path $Q$ from $u$ to $w$ in $T_1 - F$. 
    Then $P_x Q P_y$ is a directed path from $x$ to $y$ in $T_1-F$.
\end{proof}




All the results of the previous subsections imply the following.
\begin{corollary}\label{cor:m_k-upper}
    $m_k(n) \leq
    \left\{ 
    \begin{array}{ll}
        2k &\text{if~ $ n\geq 2k+1$,}\\
        3  &\text{if~ $ n \geq 11k - 2$,}\\
        1  &\text{if~  $ n\geq 19k - 2$.}
    \end{array}\right.$
\end{corollary}


%\begin{problem}
%Find better upper bounds on $m_k(n)$. 
%\end{problem}


\subsection{Upper bounds for \texorpdfstring{$k$}{k} large}\label{epsgross}
%%%%%%%%%%%%%%%%%%%%%%%%%%%%%%%%%%
% \FHO{J'ai lu cette demonstration et je suis a peu pres convaincu que l'idee marche, mais c'est quand-meme tres dure a lire parce qu'il y a beaucoup de notations. Ce serait bien, si on arriverarit a couper la demonstration un peu. Je commencerais par un claim a peu pres de la forme suivante:
% \begin{claim}
%     If $T'$ is not $k$-strong, then one of the following holds:
% \begin{itemize}
%     \item there is a vector $z \in \mathbb{F}_2^t \setminus \{\vec{0}\}$ such that $|\{v \in V(T)|\vec{v} \neq z\}|\leq k$,
%     \item there are $u,v \in V(T)$ with $\vec{u} \neq\vec{v} $ such that $\min\{|N_{T'}^+(u)\cap N_{T'}^-(v)|,|N_{T'}^+(u)\cap N_{T'}^+(v)|,|N_{T'}^-(u)\cap N_{T'}^-(v)|\leq \frac{k+\epsilon}{2}$,
%     \item there are sets $A,B \subseteq V(T')$ with $|A|,|B|=\frac{k+\epsilon}{2}$ such that the underlying of $(A \cup B, \delta_{T'}(A,B))$ does not contain a matching of size $\frac{k}{2}$.
% \end{itemize}
% \end{claim}
% Apres, on pourrait mettre des bornes pour les 3 evenements et conclure.}
% \CR{J'ai implemente ta remarque}\FHO{Super, merci. Je regarderai demain.}
In this section, we show that if a tournament has at least $2k+1+\epsilon k$ vertices for some positive integer $k$ and some $\epsilon>0$, then it can be made $k$-strong by inversing a family of sets whose cardinality only depends on $\epsilon$.
The proof consists in drawing this family uniformly at random, under the constraint that every vertex is contained in at least one of the sets. 

To analyse this procedure we will need Chernoff's Bound (Proposition~\ref{chernoff}) as well as the two following technical lemmas.

\begin{lemma}\label{lemma:proba_2}
    Let $\vec{u} \neq \vec{v} \in \mathbb{F}_2^t \setminus \{\vec{0}\}$ and $x,y \in \mathbb{F}_2$ be fixed, and let $\vec{w} \in \mathbb{F}_2^t \setminus \{\vec{0}\}$ 
    be drawn uniformly at random. Then $\Pr[\vec{u} \cdot \vec{w} = x, \vec{v} \cdot \vec{w}=y] \geq \frac{1}{4}-\frac{3}{4}\frac{1}{2^t-1}$.
\end{lemma}

\begin{proof}
    As $\vec{u}\neq \vec{v}$, the mapping $\mathbb{F}_2^t \to \mathbb{F}_2^2, \vec{w} \mapsto (\vec{u} \cdot \vec{w}, \vec{v} \cdot \vec{w})$ is surjective and linear.
    As a consequence, there are $\frac{1}{4}2^t$ vectors $\vec{w} \in \mathbb{F}_2^t$ which satisfy $\vec{u} \cdot \vec{w}=x$ and $\vec{v} \cdot \vec{w} = y$.
    Thus by possibly removing the solution $\vec{w}=0$, we obtain $\Pr[\vec{u} \cdot \vec{w} = x, \vec{v} \cdot \vec{w}=y] \geq \frac{2^{t-2}-1}{2^t-1} =
    \frac{1}{4}-\frac{3}{4}\frac{1}{2^t-1}$.
\end{proof}


\begin{lemma}\label{lemma:proba_3}
    Let $\epsilon >0$, let $t \geq 16$ be an integer, and let $k \geq \frac{8t}{\epsilon}$ be an integer.
    Let $U,V \in (\mathbb{F}_2^{t} \setminus\{\vec{0}\})^{\lceil\epsilon k/8\rceil}$ be drawn uniformly at random and 
    $W \in \mathbb{F}_2^{\lceil\epsilon k/8\rceil \times \lceil\epsilon k/8\rceil}$ be fixed.
    Then $\Pr[U^\top \cdot V = W] \leq 2^{-\epsilon t k/128}$.
\end{lemma}

\begin{proof}
    Note that since $k \geq \frac{8t}{\epsilon}$, we have 
    $\lceil \epsilon k/8 \rceil \geq t \geq t/2+1$.
    First we bound the probability that $\rk(U) \leq t/2$. %\reviewer{Change to $\leq t/2$}. 
    If $U$ has rank at most $t/2$, then there is a choice of $\lfloor t/2\rfloor$ columns of $U$ such that all the other ones are in the linear span of these selected columns.
    Since the linear span of $\lfloor t/2\rfloor$  vectors has dimension at most $t/2$, and so size at most $2^{t/2}$, we deduce the following.
    \[
    \everymath={\displaystyle}
    \renewcommand{\arraystretch}{2.5}
    \begin{array}{r l l}
        \Pr\left[\rk(U) \leq t/2\right] &\leq
        \binom{\lceil \epsilon k/8 \rceil}{\lfloor t/2\rfloor}\left(\frac{2^{\lfloor t/2\rfloor}-1}{2^t -1} \right)^{\lceil \epsilon k/8 \rceil -\lfloor t/2\rfloor } &\\
        &\leq \binom{\lceil\epsilon k/8\rceil}{\lfloor t/2\rfloor}\left(\frac{2^{t/2}-1}{2^t -1} \right)^{\epsilon k/8-t/2} & \\
        &\leq 2^{\epsilon k/8+1}\left(\frac{2^{t/2}-1}{2^t -1} \right)^{\epsilon k/16} & \text{ because $\frac{t}{2} \leq \frac{\epsilon k}{16}$ since } k\geq \frac{8t}{\epsilon} \\
        &\leq 2^{\epsilon k/4} (2^{t/2})^{-\epsilon k/16} & \\
        &\leq 2^{\epsilon t k/64} 2^{-\epsilon t k/32} & \text{ because } t \geq 16 \\
        &= 2^{- \epsilon t k/64}. & \\
    \end{array}    
    \] 
    Now we assume that $\rk(U)> t/2$.
    Then for every column $v$ of $V$, $v$ must be chosen in an affine space of dimension at most $t - \lfloor t/2\rfloor -1 \leq t/2$.
    It follows that
    \[
    \begin{split}
        \Pr[U^\top \cdot V = W \mid \rk(U) > t/2] & \leq 
        \left(\frac{2^{t/2}-1}{2^{t}-1} \right)^{\lceil\epsilon k/8 \rceil} \\
        &\leq (2^{-t/2})^{\epsilon k/8} \\
        &\leq 2^{-\epsilon t k/16}. \\
    \end{split}
    \]
    Therefore
    \[
    \begin{split}
        \Pr[U^\top \cdot V = W] & \leq \Pr\left[\rk(U) \leq t/2\right] + \Pr[U^\top \cdot V = W \mid \rk(U) > t/2]\\
        &\leq 2^{- \epsilon t k/64} + 2^{-\epsilon t k/16}\leq 2\cdot 2^{-\epsilon t k/64}.%=2^{-t \epsilon k/64+1}.\\\leq 2^{-t\epsilon k/128}\\
    \end{split}
    \]
    We know that $\epsilon t k \geq 8 t^2 \geq 2 \cdot 64$, thus $2^{-\epsilon t k / 64} \leq 1/4$. As $2x \leq \sqrt{x}$ for any $x \in [0, 1/4]$, we end with $\Pr[U^\top \cdot V = W]  \leq 2 \cdot 2^{-\epsilon t k/64} \leq 2^{-\epsilon t k/128}$.
\end{proof}
For technical reasons, we prove the following seemingly weaker restatement of Theorem~\ref{thm:2+eps}.

%\begin{restatable}{theorem}{pluseps}
\begin{theorem}\label{thm:2+eps+2}
    There exists a function $f\colon \mathbb{R}_{>0} \to \mathbb{N}$ such that for every $\epsilon>0$ and every positive integer $k$, 
    if $T$ is an $n$-vertex tournament with $n \geq 2k+ 2 \epsilon k+2$, then $\sinv_k(T) \leq f(\epsilon)$.
%\end{restatable}
\end{theorem}
%\pluseps*

It is not difficult to see that Theorem~\ref{thm:2+eps+2} actually implies Theorem~\ref{thm:2+eps}. 
Indeed, given a function $f$ like in Theorem~\ref{thm:2+eps+2} at hand, define $f':\mathbb{R}_{>0} \to \mathbb{N}$ by $f'(\epsilon)=\max\{\frac{4}{\epsilon},f(\frac{\epsilon}{2})\}$. 
Let $T$ be a tournament with $|V(T)|\geq 2k+1 +\epsilon k$ for some positive integer $k$. 
If $k \leq \frac{2}{\epsilon}$, then Theorem~\ref{thm:M<2k} yields $\sinv_k(T)\leq 2k\leq \frac{4}{\epsilon}\leq f'(\epsilon)$. 
Otherwise, we have $|V(T)|\geq 2k+1+\epsilon k\geq 2k+2 + \frac{\epsilon}{2}k$, so $\sinv_k(T)\leq f(\frac{\epsilon}{2})\leq f'(\epsilon)$ by Theorem~\ref{thm:2+eps+2}.



\begin{proof}
% \FHO{je crois que ce serait beaucoup plus claire si on arriverait de definir avant la demonstration. Pas forcement de maniere explicte, mais de la facon qu'on dit: Soit $f(e)$ un entier tel que les inegalites suivantes sont satisfaites:.... Apres, on peut supposer que $n \geq f(\epsilon)$ et ce serait beaucoup plus claire apres que le choix de f depend pas de k.}

Without loss of generality, we may assume $\epsilon \leq \frac{1}{3}$.
Let $C$ be a constant such that $\sinv_k(T) \leq 1$ if $n \geq Ck$, which exists by Theorem~\ref{nk1}. %Clearly, we can assume that $n \leq Ck$.
Let $t$ be the smallest integer such that 
\begin{itemize}
    \item $t\geq 16$, 
    \item $t \geq \log(1+\frac{48}{\epsilon})$, and
    \item $t \geq \frac{128}{\epsilon}(2C+2+\epsilon/4) +16$.
\end{itemize}
Clearly, $t$ is well defined and depends only on $\epsilon$.
Let $k_0$ be the smallest integer such that for every $k' \geq k_0$,
\begin{equation}\label{eq:condition_k_large_enough}
    (2^t-1) \exp\left(-\epsilon^2\frac{(2+\epsilon)k'}{24}\right) + 3(Ck')^2\exp\left(-\epsilon^2\frac{(2+\epsilon)k'}{4096}\right) + 2^{-k'} <1.
\end{equation}

We now prove the statement for $f(\epsilon)=\max\{t,2k_0-2,\lceil\frac{16t}{\epsilon}\rceil-2\}$. If $k < k_0$, then we conclude directly using Theorem~\ref{thm:M<2k} that $\sinv_k(T) \leq 2k\leq 2k_0-2\leq f(\epsilon)$. Similarly, if $k \leq \frac{8t}{\epsilon}-1$, then we conclude by Theorem~\ref{thm:M<2k} that $\sinv_k(T) \leq \frac{16t}{\epsilon}-2\leq f(\epsilon)$. 
 Moreover, if $n \geq Ck$, we have $\sinv_k(T)\leq 1 \leq f(\epsilon)$.
Henceforth, we may assume $k\geq \max\{k_0, \frac{8t}{\epsilon}-1\}$ and $n \leq Ck$.
\medskip



%\FHO{If $n\leq 17$, then by Theorem~\ref{thm:M<2k}, we have $\sinv_k(T)\leq 16 \leq f(\epsilon)$. We may hence assume that $n \geq 18$.}



For every vertex $u \in V(T)$, we choose uniformly and independently at random a vector $\vec{u} \in \mathbb{F}_2^t \setminus \{\vec{0}\}$.
For $i\in [t]$, let $X_i = \{u \in V(T) \mid \vec{u}_i =1\}$.
We will prove that with positive probability, the tournament $T' = \Inv(T; X_1, \dots, X_t)$ is $k$-strong.
Note that for every arc $uv \in A(T)$, we have $uv\in A(T')$ if and only if $\vec{u} \cdot \vec{v}=0$.
%\reviewer{Change $=$ to $\equiv$.}\CR{I would prefer to remove the $\mod 2$, since the elements are in $\mathbb{F}_2$.}
%and otherwise $vu$ is in $T'$.
For two disjoint subsets $A$ and $B$ of vertices of $T'$, a {\bf directed $(A,B)$-matching} is a set of arcs with tails in $A$, heads in $B$, and without common tail or common head.
\begin{claim}\label{claim:decompose_into_events_A_B_C_}
    If $T'$ is not $k$-strong, then at least one of the following events occurs:
    \begin{enumerate}[label=\Alph*]
        \item[$E_1$]: there is a vector $\vec{z} \in \mathbb{F}_2^t \setminus \{\vec{0}\}$ such that $|\{v \in V(T)\mid \vec{v} \neq \vec{z}\}|< k$,
        \item[$E_2$]: there are $u,v \in V(T)$ with $\vec{u} \neq\vec{v} $ such that $\min\{|N_{T'}^+(u)\cap N_{T'}^-(v)|,|N_{T'}^+(u)\cap N_{T'}^+(v)|,|N_{T'}^-(u)\cap N_{T'}^-(v)|\}\leq \left(1+\frac{\epsilon}{4}\right)\frac{k}{2}$,
        \item[$E_3$]: there are disjoint sets $A,B \subseteq V(T')$ with $|A|,|B| \geq \left(1+\frac{\epsilon}{4}\right)\frac{k}{2}$ 
        with no directed $(A,B)$-matching of size at least $\frac{k}{2}$.
        %such that the underlying graph of $(A \cup B, \delta_{T'}(A,B))$ does not contain a matching of size $\frac{k}{2}$.
    \end{enumerate}
\end{claim}

\begin{proofclaim}
    Assume that none of~$E_1$,~$E_2$ and~$E_3$ holds.
    Suppose for a contradiction that there is a set $X$ of at most $k-1$ vertices, and a partition $(V_1,V_2)$ of $V(T'-X)$ into nonempty sets such that $V_2 \Rightarrow V_1$ in $T'-X$.
    Since~$E_1$ does not hold, there exist $x,y\in V_1 \cup V_2$  with $\vec{x} \neq \vec{y}$. If both $x$ and $y$ are in $V_1$ (resp. $V_2$), consider $v \in V_2$ (resp. $u \in V_1$) and either $\vec{x}\neq\vec{v}$ or $\vec{y}\neq\vec{v}$ (resp. $\vec{x}\neq\vec{u}$ or $\vec{y}\neq\vec{u}$).
    If $x \in V_1$ and $y \in V_2$ we set $u=x$ and $v=y$,
    and if $y \in V_2$ and $x \in V_1$ we set $u=y$ and $v=x$.
    In all cases, there are $u\in V_1$ and $v \in V_2$ with $\vec{u} \neq \vec{v}$.

    Now, as ~$E_2$ does not hold, we have $|N_{T'}^+(u)\cap N_{T'}^-(v)|,|N_{T'}^+(u)\cap N_{T'}^+(v)|,|N_{T'}^-(u)\cap N_{T'}^-(v)|\geq(1+\epsilon/4)\frac{k}{2}$.
    Finally, as~$E_3$ does not hold, 
    there is a directed $(N_{T'}^+(u)\cap N_{T'}^+(v), N_{T'}^-(u)\cap N_{T'}^-(v))$-matching $M$ of size at least $k/2$ in $T'$.
    For every arc $e=xy\in M$, observe that $P_e=uxyv$ is a directed $(u,v)$-path in $T'$. 
    Furthermore, for every $x \in N_{T'}^+(u)\cap N_{T'}^-(v)$, observe that $P_x=uxv$ is a directed $(u,v)$-path in $T'$. 
    This yields a collection of at least $k$ internally vertex-disjoint $(u,v)$-paths in $T'$,
    a contradiction since every $(u,v)$-path meets $X$ which has size at most $k-1$.
\end{proofclaim}

We will show that with positive probability none of the events ~$E_1$,~$E_2$ and~$E_3$ occurs.


\begin{claim}\label{claim:proba_A}
    $\Pr(E_1) \leq (2^t-1) \exp\left(-\epsilon^2\frac{n}{16}\right)$
\end{claim}

\begin{proofclaim}
    If $\vec{c} \in \mathbb{F}_2^t \setminus\{\vec{0}\}$ is fixed, then $Y_{\vec{c}} = |\{u \in V(T) \mid \vec{u} \neq \vec{c}\}|$ is a random variable 
    having a binomial law with parameters $n$ and $1-\frac{1}{2^t-1}$. 
    As
    %$n \geq 2k \geq 48$\FHO{c'est utilise ou?} \JD{Pas utile je crois.}
    $\epsilon  \leq \frac{1}{3}$ and $t\geq 2$, we have $k \leq \frac{1}{2}n\leq \frac{2}{3}\cdot\frac{5}{6}n\leq(1-\frac{1}{2^t-1})(1-\frac{\epsilon}{2})n$. By Chernoff's Bound (Proposition~\ref{chernoff}), and because $t \geq 2$, we have
    \begin{align*}
    %\Pr[\exists X \subseteq V(T), |X| < k, |\{\vec{u} \mid u \not\in X\}| \leq 1] &= 
    \Pr[\exists \vec{c} \in \mathbb{F}_2^t\setminus\{\vec{0}\}, Y_{\vec{c}} < k] 
    &\leq \sum_{\vec{c} \in \mathbb{F}_2^t\setminus\{0\}} \Pr[Y_{\vec{c}} < k] \\
    &\leq \sum_{\vec{c} \in \mathbb{F}_2^t\setminus\{0\}} \Pr\left[Y_{\vec{c}} < \left(1-\frac{\epsilon}{2}\right)\left(1-\frac{1}{2^t-1}\right)n\right] \\
    &\leq (2^t-1)\exp\left(-\left(\frac{\epsilon}{2}\right)^2\left(1-\frac{1}{2^t-1}\right)\frac{n}{2}\right) \\
    &\leq (2^t-1)\exp\left(-\epsilon^2 \frac{n}{16}\right),
    \end{align*}
    as claimed.
\end{proofclaim}
% \FHO{dans l'avant-premiere inegalite, je ne vois pas pourqoui il n'y a pas de 2 a la place de 3.}
% \CR{Oui on peut surement mettre 2. Je le ferai.}


% Let $s,t$ be two distinct vertices, and let $C = N^+_{T'}(s) \cap N^-_{T'}(t), A = N^+_{T'}(s) \setminus C, B = N^-_{T'}(t) \setminus C$.
% We assume that $\vec{s} \neq \vec{t}$ (otherwise Claim~\ref{claim:proba_1} applies).
% Observe that if there is a matching of size $k-|C|$ from $A$ to $B$, then together with vertices in $C$, this gives $k$ vertex-disjoint
% $(s,t)$-paths in $T'$. We will show that this happens with high probability.

% \begin{claim}
%     $\Pr[\min(|A|,|B|,|C|) \leq (1+\epsilon/2)k/2] \leq 3\exp\left(-\epsilon^2\frac{n-2}{192}\right)$
% \end{claim}

\begin{claim}\label{claim:proba_B}
    $\Pr(E_2) \leq 3n(n-1)\exp\left(-\epsilon^2\frac{n-2}{4096}\right)$
\end{claim}

\begin{proofclaim}
    Let $u,v$ be distinct vertices and let
    $A = N_{T'}^+(u)\cap N_{T'}^-(v), B = N_{T'}^+(u)\cap N_{T'}^+(v)$ and $C=N_{T'}^-(u)\cap N_{T'}^-(v)$.
    Let $X \in \{A,B,C\}$.
    Once $\vec{u}$ and $\vec{v}$ have been revealed, for every vertex $w \neq u,v$, let $Y_w$ be a random variable with $Y_w=1$ if $w \in X$, $Y_w=0$ otherwise.
    By Lemma~\ref{lemma:proba_2}, $Y_w$ is a random variable following a Bernoulli distribution whose parameter is at least $\frac{1}{4}(1-\frac{3}{2^t-1})$. Further, the $Y_w$ are mutually independent.


    We now define a random variable $X_w$ for every $w \in V(T)\setminus \{u,v\}$ as follows.
    If $Y_w=0$, set $X_w=0$, and otherwise set $X_w=1$ with probability $\frac{\frac{1}{4}(1-\frac{3}{2^t-1})}{\Pr[Y_w=1]}$ and $X_w=0$ otherwise, where the latter random experiments are executed independently. Observe that, as the $Y_w$ are mutually independent, so are the $X_w$. Moreover, $\Pr[X_w=1] = \frac{1}{4}(1-\frac{3}{2^t-1})$
    and $\sum_{w \in V(T) \setminus \{u,v\}} X_w \geq |X|$.
    
    We have $(1+\frac{\epsilon}{4})\frac{k}{2} \leq \frac{1+\frac{\epsilon}{4}}{2+\epsilon}(n-2)/2 = \frac{1}{2}\left(1-\frac{\frac{\epsilon}{4}}{1+\frac{\epsilon}{2}}\right) \frac{n-2}{2} \leq \frac{1}{4}\left(1-\frac{\epsilon}{8}\right)(n-2)$ since $n \geq (2+\epsilon)k+2$.
    Moreover, as $t \geq  \log(\frac{48}{\epsilon}+1)$ we have 
    $\frac{1}{4}(1-\epsilon/8)\leq \frac{1}{4}(1-\epsilon/16)^2 \leq \frac{1}{4}\left(1-\frac{3}{2^t-1}\right)(1-\frac{\epsilon}{16}) = \left(\frac{1}{4}-\frac{3}{4(2^t-1)}\right)\left(1-\frac{\epsilon}{16}\right)$.
    Hence $\left(1+\frac{\epsilon}{4}\right) \frac{k}{2} \leq \left(1-\frac{\epsilon}{16}\right)\left(\frac{1}{4}-\frac{3}{4(2^t-1)}\right)(n-2)$, and by Chernoff's Bound (Proposition~\ref{chernoff})
     \begin{align*}
        \Pr\left[|X| \leq \left(1+\frac{\epsilon}{4}\right)\frac{k}{2}\right]
        & \leq \Pr\left[|X| \leq \left(1-\frac{\epsilon}{16}\right)\left(\frac{1}{4}-\frac{3}{4(2^t-1)}\right)(n-2)\right]\\
        &\leq \exp\left(-\left(\frac{\epsilon}{16}\right)^2\left(\frac{1}{4}-\frac{3}{4(2^t-1)}\right)\frac{n-2}{2}\right)\\    
        &\leq \exp\left(-\epsilon^2\frac{n-2}{4096}\right) 
   \end{align*}
    since $t \geq 5$ implies $\frac{3}{2^t-1} \leq \frac{1}{8}$. 
    Hence, by the Union Bound (Proposition~\ref{union}), $\Pr\left[\min\{|A|,|B|,|C|\} \leq \left(1+\frac{\epsilon}{4}\right)\frac{k}{2}\right] \leq \sum_{X \in \{A,B,C\}}\Pr\left[|X| \leq \left(1+\frac{\epsilon}{4}\right)\frac{k}{2}\right]\leq 3\exp\left(-\epsilon^2\frac{n-2}{4096}\right)$.
    Finally, by the Union Bound over $u,v$, the probability that there exist $u,v$ distinct such that $\min\{|A|,|B|,|C|\} \leq \left(1+\frac{\epsilon}{4}\right)\frac{k}{2}$ is at most
     $3n(n-1)\exp\left(-\epsilon^2\frac{n-2}{4096}\right)$.
\end{proofclaim}

For two disjoint sets of vertices $X,Y$ in $T'$, we denote by $\mu_{T'}(X,Y)$ the size of a largest directed $(X,Y)$-matching in $T'$.

\begin{claim}\label{claim:proba_C}
    $\Pr(E_3) \leq 2^{-k}$.
\end{claim}
    
\begin{proofclaim}
    Let $A, B \subseteq V(T')$ be disjoint sets of $\lceil (1+\frac{\epsilon}{4})\frac{k}{2} \rceil$ vertices.
    We shall prove that with high probability there is a directed $(A,B)$-matching in $T'$ of size at least $\frac{k}{2}$.
    
    Let $M$ be a maximal directed $(A,B)$-matching and let $Y_A$ (resp. $Y_B$) the the set of vertices in $A$ (resp. in $B$) incident to no arc of $M$.
    Then $Y_B \Rightarrow Y_A$ in $T'$ since $M$ is maximal.
    Moreover, if $|M| \leq k/2$, then $|Y_A|,|Y_B| \geq (1+\frac{\epsilon}{4})\frac{k}{2}-\frac{k}{2} = \frac{\epsilon}{8}k$.
    
    For every such $Y_A \subseteq A, Y_B \subseteq B$, we identify $Y_A$ and $Y_B$ with the matrices whose columns are the $\vec{u}$ for $u \in Y_A$ (resp. $u \in Y_B$).    

    We also denote by $T(Y_B,Y_A)$ the $|Y_B| \times |Y_A|$ matrix over $\mathbb{F}_2$ whose cell $(b,a)$ equals $1$ if and only if $ab \in A(T)$.
    Then observe that $Y_B \Rightarrow Y_A$ in $T'$ if and only if $Y_B^\top \cdot Y_A = T(Y_B,Y_A)$.
    %\reviewer{I think the condition is the wrong way round. By my reckoning, it should be $Y^T_A Y_B = T(Y_A, Y_B)$.} \CR{I am not sure.}\FHO{I think he's right.}
    By these observations, we have
    \[
    %\everymath={\displaystyle}
    \renewcommand{\arraystretch}{1.5}
    \begin{array}{r c l}
       \Pr[\mu_{T'}(A,B)<k]  & \leq & %\Pr[\exists X_A\subseteq A, X_B \subseteq B', |X_A \cup X_B| \leq k/2, X_A \cup X_B \text{ vertex cover of } (A',B')] \\
       \Pr[\exists Y_A \subseteq A, |Y_A| \geq \frac{\epsilon}{8}k, \exists Y_B \subseteq B, |Y_B| \geq \frac{\epsilon}{8}k, Y_B \Rightarrow Y_A \text{ in } T']  \\
       & \leq & \Pr[\exists Y_A \subseteq A, |Y_A| \geq \frac{\epsilon}{8}k, \exists Y_B \subseteq B, |Y_B| \geq \frac{\epsilon}{8}k, Y_B^\top \cdot Y_A = T(Y_B,Y_A)]  \\
       & \leq & 2^{2\left\lceil (1+\frac{\epsilon}{4})\frac{k}{2} \right\rceil} 2^{-\frac{\epsilon t k}{128}} \hspace{1cm} \text{ by Lemma~\ref{lemma:proba_3} and the Union Bound} \\
       & \leq & 2^{\left(1+\frac{\epsilon}{4}\right)k+2} 2^{-\frac{\epsilon t k}{128}} \hspace{1.325cm}  \\
       & \leq & 2^{-(2C+1)k} \\
    \end{array}
    \]
    
    as $t \geq \frac{128}{\epsilon}\left(2C+2+\frac{\epsilon}{4}\right) + 16$ and $\frac{\epsilon k}{64} \geq \frac{1}{8}$.
    It follows from the Union Bound (Proposition~\ref{union}) that $\Pr(E_3) \leq 2^{2n} 2^{-(2C+1)k} \leq 2^{-k}$ using the fact that $n \leq Ck$.
\end{proofclaim}

We can now conclude using Claims~\ref{claim:decompose_into_events_A_B_C_},~\ref{claim:proba_A},~\ref{claim:proba_B}
and~\ref{claim:proba_C} and the Union Bound: 
\[
\begin{split}
    \Pr[T' \text{ not } k\text{-strong}] &\leq \Pr(E_1) + \Pr(E_2)+\Pr(E_3) \\
    &\leq (2^t-1) \exp\left(-\epsilon^2\frac{n}{16}\right) + 3n^2\exp\left(-\epsilon^2\frac{n-2}{4096}\right) + 2^{-k} \\
    &\leq (2^t-1) \exp\left(-\epsilon^2\frac{(2+\epsilon)k}{16}\right) + 3(Ck)^2\exp\left(-\epsilon^2\frac{(2+\epsilon)k}{4096}\right) + 2^{-k} \\
    &<1, \\
\end{split}
\]
by \eqref{eq:condition_k_large_enough}.
This proves that there exist $X_1, \dots, X_t \subseteq V(T)$ such that $T'=\Inv(T;X_1,\dots, X_t)$ is $k$-strong.
\end{proof}


\section{Conclusion}\label{conclusion}
%We have dealt with several aspects of making a given digraph satisfy some prescribed connectivity properties by applying inversion operations.\CR{J'aime pas trop cette phrase. Alternative:}
In this paper, we investigated problems of the form: given a digraph, what is the minimum number of inversions needed such that the resulting digraph has a prescribed connectivity property?


We posed Conjecture~\ref{conjM} which we restate. 
\theconj*
Conjecture~\ref{conjM} is motivated by the following consideration. By Proposition~\ref{4k-2}, every tournament $T$ of order $n>4k-2$ has a vertex $v$ with $d_T^-(v)\geq k$ and $d_T^+(v)\geq k$. 
Moreover, adding a vertex with in- and out-degree at least $k$ to a $k$-(arc-)strong digraph results in a $k$-(arc-)strong digraph. 
Hence any $k$(-arc)-strengthening family of $T-v$ is also a $k$-(arc-)strengthening family of $T$. Thus $\sinv_k(T) \leq \sinv_k(T-v) \leq m_k(n-1) $ and $\sinv'_k(T) \leq \sinv'_k(T-v) \leq m'_k(n-1) $. Hence $m_k(n) \leq m_k(n-1)$ and $m'_k(n) \leq m'_k(n-1)$.
Therefore, in order to approach Conjecture~\ref{conjM} and Problem~\ref{rsedth}, % to get bounds on $M_k$ and $M'_k$,
it is sufficient to consider tournaments whose order is in the range from $2k+1$ to $4k-2$.

Another intriguing question is whether $M_k = M'_k$.
\begin{problem}
    Is it true that $M_k = M'_k$ for every positive integer $k$ ?
\end{problem}
In Subsection~\ref{sec:52}, we proved that this is the case for $k=1,2$.


For the first part, it is tempting to understand better the asymptotic behaviour of $\sinv_k(n)$. In particular, it would be interesting to see if an analogue of Theorem~\ref{thm:extrem} exists for $\sinv_k$.
\begin{problem}
  %Let $\sinv_k(n) = \max\{\sinv_k(D)\mid D\ \mbox{$k$-strengtenable digraph of order $n$}\}$.
  Find good lower and upper bounds on $\sinv_k(n)$.
\end{problem}

\medskip
For the algorithmic part, one may wonder whether polynomial-time constant-factor-approximation algorithms for computing $\sinv_k$ or $\sinv'_k$ exist. Actually, we believe that this is not the case and that Theorems~\ref{approx2} and~\ref{approx1} can be strengthened in the following way:

\begin{conjecture}
    Unless P=NP, for any positive integer $k$ and any constant $\alpha$, there is no $\alpha$-approximation algorithm for computing $\sinv_k(D)$ or $\sinv_k'(D)$ given an oriented graph $D$.
\end{conjecture}

It would further be interesting to understand the complexity when restricting to tournaments. While the complexity for fixed $k$ is resolved by Corollary~\ref{cor:sinv_k-poly}, the following question remains open:

\begin{problem}
    What is the complexity of computing $\sinv_k(T)$ (resp. $\sinv'_k(T)$) for a given tournament $T$ if $k$ is part of the input ?
\end{problem}

Next, we established several bounds on $M_k$ and $M_k'$. There is still a significant gap between the logarithmic lower bounds and the linear upper bounds, so it would be good to improve these.
A first question is the following.
\begin{problem}
    Are $M'_k$ and $M_k$ sublinear functions of $k$ ? 
\end{problem}

Further, we proved a collection of bounds for $N_k(1)$ and $N_k(3)$. It would be interesting to have stronger bounds on $N_k(i)$ for any integers $i$. In particular, a tight bound for $N_k(1)$ would be satisfying.

Finally, Theorem~\ref{thm:2+eps} states that every tournament that is a constant factor bigger than $2k$ can be made $k$-strong by a constant number of inversions. It would be interesting to know if this result can be strengthened in the following way:

\begin{problem}
    Is there an integer $t$ such that for every $\epsilon>0$, there is an integer $k_0$ such that for every $k \geq k_0$, for every tournament $T$ on at least $(2+\epsilon)k$ vertices, we have $\sinv_k(T)\leq t$?
\end{problem}

Proposition~\ref{nk1unter} shows that such an integer would have to satisfy $t \geq 2$, so $t=2$ is the first open case. The analogous statement for $\sinv'_k$ is also open.


