\documentclass[aps,prl,preprint,tightenlines]{revtex4-2}

\usepackage{graphicx}% Include figure files
\usepackage{xcolor}
\usepackage{dcolumn}% Align table columns on decimal point
\usepackage{bm, siunitx, gensymb, amssymb, xfrac, physics}% bold math
% removed bbm
\usepackage{hyperref}
\usepackage{booktabs, multirow}
\usepackage{braket}
\renewcommand{\vec}[1]{\mathbf{#1}}
\renewcommand{\arraystretch}{1.4}
\setlength{\tabcolsep}{4.7pt}
\hypersetup{
    colorlinks=true,
    urlcolor= blue,
    citecolor=blue,
    linkcolor= blue,
    bookmarksopen=false,
    }
    
\newif\ifptitle
\newif\ifpnumber
\newcounter{para}
\newcommand\ptitle[1]{\par\refstepcounter{para}
{\ifpnumber{\noindent\textcolor{darkgray}{\textbf{\thepara}}\indent}\fi}
{\ifptitle{\textbf{[{#1}]}}\fi}}
% \ptitletrue  % comment this line to hide paragraph titles
% \pnumbertrue  % comment this line to hide paragraph numbers

\definecolor{sblue}{RGB}{33, 118, 199}
\definecolor{mblue}{RGB}{0, 118, 186}
\definecolor{hblue}{RGB}{36, 34, 111}
\newcommand{\blue}[1]{\textcolor{sblue}{#1}}
\definecolor{mgreen}{RGB}{29, 177, 0}
\newcommand{\green}[1]{\textcolor{mgreen}{#1}}
\definecolor{sred}{RGB}{209, 28, 36}
\newcommand{\red}[1]{\textcolor{sred}{#1}}
\definecolor{orang}{RGB}{255, 147, 0}
\newcommand{\orange}[1]{\textcolor{orang}{#1}}
\renewcommand{\figurename}{FIG.}
\renewcommand{\tablename}{{Table}}
\renewcommand{\theequation}{ S\arabic{equation}}
\renewcommand{\thefigure}{{S\arabic{figure}}}
\renewcommand{\thetable}{{S\Roman{table}}}

%\setcitestyle{round}


\begin{document}
\title{{\Large Supplementary Information} \\ \vspace{0.20cm} Visualizing the atomic-scale origin of metallic behavior in Kondo insulators}

\author{Harris Pirie}
\author{Eric Mascot}
\author{Christian E. Matt}
\author{Yu Liu}
\author{Pengcheng Chen}
\author{M. H. Hamidian}
\author{Shanta Saha}
\author{Xiangfeng Wang}
\author{Johnpierre Paglione}
\author{Graeme Luke}
\author{David Goldhaber-Gordon}
\author{Cyrus Hirjibehedin}
\author{J. C. S\'{e}amus Davis}
\author{Dirk K. Morr}
\author{Jennifer E. Hoffman}

\maketitle


\section{Materials and Methods}


\nocite{Doniach1977, Steglich1979, Andres1975, Fisk1986, Laurita2016, Flachbart2006, Fuhrman2018, Thomas2019, Hartstein2018, Tan2015, Millichamp2021, Eo2019, Knolle2015, Baskaran2015, Erten2017, Shen2018, Skinner2019, Fuhrman2020, Abele2020, Dagotto2005, Sollie1991, Figgins2011, Figgins2019, Hamidian2011, Urbano2007, Nefedova1999, Gabani2002, Orendac2017, Souza2020, Valentine2016, Eo2021, Koslowski1995, Wagner2015, Hapala2016, Nonnenmacher1991, GrossScience2009, Mohn2012, Zerweck2005, Sadewasser2009, AlbrechtPRL2015, AlbrechtPRB2015, Kotta2022, Madhavan1998, Maltseva2009, Figgins2010, Schmidt2010, Giannakis2019, Kohsaka2007, delaTorre1992, Santander-Syro2009, Matt2020, Sun2018, Pirie2020, Jiang2013, Jiao2018, Akintola2017, Liu2009, Okada2011, Jack2020, Phelan2016, Bork2011, Aishwarya2022, Zenodo}


\subsection{STM measurements}
\ptitle{} We studied single crystals of 1\% Th-doped URu$_2$Si$_2$ grown using the Czochralski method, and 0.1\% Gd-doped and 0.5\% Fe-doped SmB$_6$ grown using the Al-flux method \cite{Kim2013}. The measured doping concentration in the STM topographies was typically lower than the nominal doping, as shown in Table \ref{tab:doping}. Each sample was cleaved in cryogenic ultrahigh vacuum and immediately inserted into a variable-temperature scanning tunneling microscope (STM). Due to the cryogenic vacuum conditions, atomically flat terraces were stable for many months without noticeable degradation. The corresponding topographies for the temperature-dependent measurements on URu$_2$Si$_2$ are shown in Fig.~\ref{fig:URS-topo}. The d$I(\mathbf{r},V)$/d$V$ measurements were acquired with a standard alternating current lock-in amplifier technique. STM tips were cut from PtIr wire and cleaned by {\it in situ} field emission on Au foil. 


\subsection{Theory}
\ptitle{Eric's description of calculations}The calculations in Fig.~1 are based on the Kondo-Heisenberg Hamiltonian,
\begin{align}\label{eq:hamiltonian}
	\mathcal{H} = \sum_{\vec{k}, \sigma} \varepsilon_{\vec{k}} c_{\vec{k}, \sigma}^\dag c_{\vec{k}, \sigma}
	+ J \sum_{\vec{r}} \vec{S}_{\vec{r}} \cdot \vec{s}_{\vec{r}}
	+ I \sum_{\braket{\vec{r}, \vec{r}'}} \vec{S}_{\vec{r}} \cdot \vec{S}_{\vec{r}'},
\end{align}
where the conduction electrons have the dispersion \(\varepsilon_{\vec{k}} = -2t(\cos k_x + \cos k_y) - \mu\) with nearest-neighbor hopping, \(t\), and chemical potential, \(\mu\);
\(c_{\vec{k}, \sigma}^\dag\) (\(c_{\vec{k}, \sigma}\)) creates (annihilates) a conduction electron with momentum \(\vec{k}\) and spin \(\sigma\);
\(J\) is the Kondo coupling between the local \(f\) moment, \(\vec{S}_{\vec{r}}\), and the conduction electron's spin, \(\vec{s}_{\vec{r}} = \frac{1}{2} \sum_{\alpha, \beta} c_{\vec{r}, \alpha}^\dag \vec{\sigma}_{\alpha\beta} c_{\vec{r}, \beta}\); and
\(I\) is the antiferromagnetic Heisenberg exchange coupling.
We proceed on the interpretation that a Kondo hole at site $\mathbf{r}_0$ is described by removing all terms in Eq.~\ref{eq:hamiltonian} that include $\mathbf{S}_{\mathbf{r}_0}$ and adding an onsite electrostatic potential $U \sum_\sigma c_{\vec{r}_0 \sigma}^\dagger c_{\vec{r}_0 \sigma}$. We set $U = t$ for the calculations in Fig.~1, F to H.
We use a spin-1/2 Abrikosov fermion representation \cite{Abrikosov1965} for the \(f\) moments, \(\vec{S}_{\vec{r}} = \frac{1}{2} \sum_{\alpha, \beta} f_{\vec{r}, \alpha}^\dag \vec{\sigma}_{\alpha\beta} f_{\vec{r}, \beta}\) subject to the constraint \(n_f(\vec{r}) = \sum_\sigma f_{\vec{r}, \sigma}^\dag f_{\vec{r}, \sigma} = 1\).
We enforce the constraint using a Lagrange multiplier for each site, \(\tilde{\varepsilon}_f(\vec{r})\).
We introduce the mean-fields,
\begin{align}
	\nu(\vec{r}) =& -\frac{J}{2} \sum_\sigma \braket{f_{\vec{r}, \sigma}^\dag c_{\vec{r} \sigma}}, &
	\chi(\vec{r}, \vec{r}') =& \frac{I}{2} \sum_\sigma \braket{f_{\vec{r}', \sigma}^\dag f_{\vec{r}, \sigma}},
\end{align}
where $\nu(\vec{r})$ is the hybridization and $\chi(\vec{r}, \vec{r}')$ is the $f$-electron dispersion, to obtain the mean-field Hamiltonian,
\begin{align}\label{eq:Hfull}
	\mathcal{H} =
	\sum_{\vec{k}, \sigma} \varepsilon_{\vec{k}} c_{\vec{k}, \sigma}^\dag &c_{\vec{k}, \sigma}
	- \sum_{\vec{r}, \sigma} \tilde{\varepsilon}_f(\vec{r}) f_{\vec{r}, \sigma}^\dag f_{\vec{r}, \sigma} \nonumber 
	 -\sum_{\braket{\vec{r}, \vec{r}'}, \sigma} \left(
		\chi(\vec{r}, \vec{r}') f_{\vec{r}, \sigma}^\dag f_{\vec{r}', \sigma}
		+ \text{H.c.}
	\right) \nonumber \\
	& + \sum_{\vec{r}, \sigma} \left(
		\nu(\vec{r}) c_{\vec{r}, \sigma}^\dag f_{\vec{r}, \sigma}
		+ \text{H.c.}
	\right),
\end{align}
where H.c. is the Hermitian conjugate. 
The Hamiltonian in Eq.~\ref{eq:Hfull} was diagonalized in real space on a $64 \times 64$ square lattice with periodic boundary conditions, allowing a self-consistent calculation of the mean-field parameters $\tilde{\varepsilon}_f(\mathbf{r})$, $\nu(\mathbf{r})$, and $\chi(\mathbf{r}, \mathbf{r}^\prime)$ following the procedure described in Ref.~\cite{Johnson1988}.
To compute the tunneling conductance \(\mathrm{d}I/\mathrm{d}V\), we introduce the tunneling term
\begin{align}
	\mathcal{H}_T = \sum_{\sigma} \left(
		t_c d_{\vec{R}, \sigma}^\dag c_{\vec{R}, \sigma}
		+ t_f d_{\vec{R}, \sigma}^\dag f_{\vec{R}, \sigma}
		+ \mathrm{H.c.}
	\right),
\end{align}
where \(d_{\vec{R}, \sigma}^\dag\) creates an electron in the STM tip.
The tunneling conductance in the zero-temperature, wide-band, weak-tunneling limit is then given by
\begin{align}
	\frac{\mathrm{d}I(\vec{r}, V)}{\mathrm{d}V} \propto
	t_c^2 N_c(\vec{r}, eV) + t_f^2 N_f(\vec{r}, eV) + 2 t_c t_f N_{cf}(\vec{r}, eV),
\end{align}
where \(N_c, N_f\) are the local density of states of the conduction electrons and \(f\) electrons, respectively, and \(N_{cf} = -\mathrm{Im} G_{cf} / \pi\) \cite{Figgins2010}. The current is given by     
\begin{align}
        I(\mathbf{r}, V) = \int_0^V dV’ \frac{dI(\mathbf{r}, V’)}{dV’} .
\end{align}
Finally, the rectification $R(\mathbf{r}, V)$ is then computed from 
\begin{align}
    R(\mathbf{r}, V) = \qty|\frac{I(\mathbf{r}, V)}{I(\mathbf{r}, -V)}|.
\end{align}
The Kondo lattice parameters used in Fig.~1 are taken from Refs.~\cite{Figgins2010, Figgins2011} with a slightly different Fermi wavelength of $8a$ instead of $10a$ to increase the number of oscillations in a fixed system size.


\newpage
\section{Supplementary Text}


\subsection{Rectification in Kondo lattice systems}


\ptitle{} The dominant contribution to the local rectification in Kondo lattice systems arises from the large energy asymmetric d$I$/d$V$ peak caused by the Kondo resonance $\tilde{\varepsilon}_f$. As $\tilde{\varepsilon}_f$ moves in response to local charge, so does the energy position of the d$I$/d$V$ peak, which dramatically alters the odd component of d$I$/d$V$. The odd component of d$I$/d$V$ leads to an even component in $I(V)$, or an ``asymmetry'' since $I(V)$ is usually odd  (see Fig.~\ref{fig:chi}, A to D, for an illustration of this process):
    \begin{align}
        \frac{I(V)+I(-V)}{2} &= 
        \frac{1}{2} \int_0^V dV’ \frac{dI(V’)}{dV’} + \frac{1}{2} \int_0^{-V} dV’ \frac{dI(V’)}{dV’} \nonumber
        \\
        &= \frac{1}{2} \int_0^V dV’ \qty[\frac{dI(V’)}{dV’} - \frac{dI(-V’)}{dV’}]
    \end{align}
The rectification $R(V)$  can be written in terms of the $I(V)$ asymmetry as
    \begin{align}
        R(V) &= \frac{I(V)}{I(-V)} 
        = \frac{\int_0^V dV’ \frac{dI(V’)}{dV’}}{\int_0^V dV’ \frac{dI( -V’)}{dV’}}
        = \frac{1 + \alpha(V)}{1 - \alpha(V)}
    \end{align}
    where
    \begin{align}
       \alpha(V) &= \frac{\int_0^V dV’ \qty[\frac{dI(V’)}{dV’} - \frac{dI(-V’)}{dV’}]}{\int_0^V dV’\qty[\frac{dI(V’)}{dV’} + \frac{dI( -V’)}{dV’}]}
        = \frac{I(V) + I(-V)}{I(V) - I(-V)}
    \end{align}
Here $\alpha(V)$ is the ratio of the even and odd components of $I(V)$. As $\tilde{\varepsilon}_f$ moves towards the Fermi level, the odd component of d$I$/d$V$ grows quickly leading to an increase in $\alpha(V)$ and a large peak in $R(V)$. For pedagogical purposes, we isolated this mechanism in Fig.~\ref{fig:chi}, A to D, by artificially changing $\tilde{\varepsilon}_f$ and calculating the resulting d$I$/d$V$ with all other tight-binding parameters fixed. In a self-consistent solution to the model in Eq.~\ref{eq:Hfull}, we induce a change in $\tilde{\varepsilon}_f$ by adding additional charge $\Delta n_c$ to the system by adjusting $\mu$. In this case, we find that $\tilde{\varepsilon}_f$ depends linearly on $n_c$, and $R$ is a monotonic function of $\tilde{\varepsilon}_f$ that depends on the $f$-band dispersion $\chi(\vec{r}, \vec{r}')$, as shown in Fig.~\ref{fig:chi}, E and F.

    
\ptitle{}In simple Kondo-lattice systems, d$I$/d$V$ often empirically follows a Fano function, a model typically used to describe the interference between a discrete state (here, the Kondo resonance) and a continuum (the conduction band), given by \cite{Schmidt2010, Giannakis2019,Figgins2019}
\begin{align}\label{eqn:s1}
    \frac{\mathrm{d}I(\vec{r}, V)}{\mathrm{d}V} \propto \frac{(E^\prime + q)^2}{E^\prime + 1} 
    \quad \mathrm{where} \quad
    E^\prime = \frac{eV - \varepsilon_f}{\Gamma/2}.
\end{align}
Here $\Gamma$ is the energy width of the Kondo resonance, $\varepsilon_f$ is its energy position, and $q$ is its asymmetry \cite{Figgins2019}. Where valid, Eq.~\ref{eqn:s1} allows $\varepsilon_f(\mathbf{r})$ to be extracted directly from d$I(\mathbf{r},V)$/d$V$, providing an independent check of the metallic $R(\mathbf{r},V)$ oscillations discovered in the main text. In  URu$_2$Si$_2$, d$I(\mathbf{r},V)$/d$V$ is adequately described by Eq.~\ref{eqn:s1} plus a parabolic background for low biases $|V|< 5$ mV (see Fig.~\ref{fig:s1}A). Near a Th dopant, the extracted $\varepsilon_f(\mathbf{r})$ trace displays clear oscillations that match those in $R(\mathbf{r},V)$ for $\mathbf{r} > 1$ nm, with a correlation coefficient of $R^2=-0.95$ (Fig.~\ref{fig:s1}, B and C). These two quantities differ at the defect site (where $\mathbf{r} < 1$ nm), likely because the validity of Eq.~\ref{eqn:s1} breaks down around Kondo holes. Nevertheless, the map of $\varepsilon_f(\mathbf{r})$ contains prominent oscillations at the parent-state Fermi wavevector of $2k_\mathrm{F}^\mathrm{c} \approx 0.3\ (2\pi/a)$ (see Fig.~\ref{fig:s1}, D and F), corroborating the existence of charge oscillations in $R(\mathbf{r},V)$ (Fig.~\ref{fig:s1}, E and G). 


\subsection{Origin of $R(\mathbf{r}, V)$ oscillations}


\ptitle{}The $R(\mathbf{r}, V)$ oscillations in SmB$_6$ have a different origin to the simultaneously measured heavy Dirac surface states that appear in our d$I$/d$V$ measurements. First, the largest major-axis wavevector of the surface state (see Fig.~4B) is about 30\% smaller than the extrapolated major axis of the $R(\mathbf{q})$ ellipse (dashed line in Fig.~\ref{fig:s2}A). Second, the surface state contributes very little quasiparticle interference at the bias corresponding to the Dirac point $V_D = E_D/e \approx -5$ mV, whereas the $R(\mathbf{q})$ ellipse is clear and prominent at this bias (see Fig.~\ref{fig:s2}A). Third, the surface state disperses rapidly for biases within the hybridization gap $|V| < \Delta/e \approx 10$ mV (see Fig.~4B), whereas the $R(\mathbf{q})$ ellipse is non dispersive (Fig.~\ref{fig:s2}B). Fourth, the rectification $R(V)$ from a Dirac-like density of states $\rho(E) \propto v_F |E - E_D|$ depends only on the Dirac point energy $E_D$. It has the explicit form 
    \begin{align}\label{eq:rect}
        R(V) = \begin{cases} 
            \frac{V(V-2V_D)}{(V+V_D)^2+V_D^2} & V V_D < - V_D^2 \\
            \frac{V-2V_D}{V+2V_D} & |V|\leq |V_D| \\
            \frac{(V-V_D)^2+V_D^2}{V(V+2V_D)} & V V_D > V_D^2 
            \end{cases}
    \end{align}
where $V_D = E_D/e$ is the bias corresponding to the Dirac point. In SmB$_6$, the Dirac point occurs slightly below $E_F$ at $E_D \approx -5$ meV (see Fig.~4B and \cite{Pirie2020}). Therefore, from Eq.~\ref{eq:rect}, the heavy Dirac surface states contribute a \textit{forward-biased} rectification curve with a peak at positive $V$, whereas our measured $R(V)$ curves reveal a \textit{reverse-biased} junction with a peak at negative $V$ (e.g.~Fig.~3C).


\ptitle{}Although it cannot be explained by the surface state, the non-dispersive $R(\mathbf{q})$ ellipse matches expectations for the metallic $5d$ band, after accounting for its projection onto a $(2 \times 1)$ reconstructed surface. The surface projection of a bulk band typically traces the ($k_x, k_y$) contours of the corresponding bulk band for all $k_z$ points with zero group velocity in the $\hat{z}$ direction, $v_{\hat{z}} = \hbar^{-1}\grad_{\hat{k}_z}E(\mathbf{k})=0$. In SmB$_6$, this projection results in an ellipse at the $\bar{X}$ and $\bar{Y}$ points with $k_z=0$ and a circle around $\bar{\Gamma}$ with $k_z = \pi$. The $(2 \times 1)$ surface reconstruction reduces the size of the Brillouin zone in the $\hat{y}$ direction. This potential folds the $\bar{Y}$-point ellipse to the $\bar{\Gamma}$ point of the $(2 \times 1)$ surface Brillouin zone (see Fig.~\ref{fig:2x1}, B and E). The opposite effect occurs on $(1 \times 2)$ domains, which fold the $\bar{X}$-point ellipse to $\bar{\Gamma}$. We separated the field of view in Fig.~4A into regions of uniform $(2 \times 1)$ or $(1 \times 2)$ termination by inverse Fourier transforming the corresponding Bragg peak (Fig.~\ref{fig:2x1}, A and D). The $R(\mathbf{q})$ signal within each domain closely matches the folded $\bar{Y}$ or $\bar{X}$ ellipse (Fig.~\ref{fig:2x1}, C and F). In Fig.~3G, we averaged the $\bar{X}$ and the $\bar{Y}$ pockets by rotating the Fourier transform $R_{2\times 1}(\mathbf{q})$ by 90$^\circ$ before averaging with the Fourier transform $R_{1\times 2}(\mathbf{q})$.


\subsection{Metallic puddles around Gd dopants}


\ptitle{} To verify the discovery of metallic puddles in SmB$_6$, we measured $R(\mathbf{r},V)$ around a third common Sm-site defect: Gd dopants. Gd dopants are suspected to form spinful Kondo holes in SmB$_6$ because they are known to dope magnetically \cite{Fuhrman2018}. They also contribute strongly to the residual linear specific heat \cite{Fuhrman2018} and appear to generate local metallic puddles from electron-spin resonance measurements \cite{Souza2020}. Our $R(\mathbf{r},V)$ data confirms the existence of these metallic puddles, as we see the same unhybridized $5d$ wavevector around Gd dopants as we do around Sm vacancies (compare both dopants in Fig.~\ref{fig:s3}, E and F). The similarity between the $R(\mathbf{r})$ maps around each dopant type, despite their different topographies, implies that $R(\mathbf{r})$ couples primarily to the parent Fermi surface and not to a local property of the specific dopant (such as an impurity bound state). This similarity also indicates that the defect form factor does not significantly impact the $R(\mathbf{r})$ maps.


\subsection{Quantum oscillations from metallic puddles}


\ptitle{} Our discovery of light $d$ electrons around point defects suggests an alternative origin for the measured de Haas-van Alphen (dHvA) effect in SmB$_6$. These metallic puddles are large enough to host Landau levels at the magnetic field of 35 T where high-frequency (large-$k_F$) dHvA oscillations are first observed \cite{Hartstein2018}. Specifically, the real-space area enclosed by a Landau orbit is $S = \pi l_B^2$, where $l_B = \sqrt{\hbar/eB} \approx 4.3$ nm is the magnetic length at 35 T, whereas the puddle radius is typically a few times the $R(\mathbf{r})$ decay length of $\gamma = 2.6$ nm (see Fig.~\ref{fig:size}). 


\ptitle{} Our discovery also reveals the theoretical challenge to model the emergence of dHvA oscillations from disordered metallic puddles. For example, a confining potential is known to qualitatively modify the dHvA response \cite{Curnoe1998}, changing both the shape and amplitude of the oscillations \cite{Herzog2016}, and typically violating Lifshitz-Kosevich theory \cite{Champel2001}. In SmB$_6$, the dHvA amplitude was already shown to deviate from Lifshitz-Kosevich behavior at low temperatures \cite{Hartstein2018}. Additionally, the onset of the large-$k_F$ oscillations at 35 T violates the traditional low-scattering condition $\omega_c \tau  = l/r_c \gg 1$, as the estimated mean free path of $l \approx $ 10-50 nm is smaller than the cyclotron radius $r_c \approx 109$ nm \cite{Hartstein2018}. An interesting and open question is whether the Landau orbits are really confined to a single puddle, or whether the overlap between puddles is sufficient to generate orbits that involve multiple puddles. Bulk measurements of floating-zone-grown SmB$_6$ suggest a vacancy concentration of about $d=1\%$, translating to an average nearest-neighbor separation of $a d^{-\frac{1}{3}} \approx 2$ nm, where $a=0.413$ nm is the SmB$_6$ lattice constant. This separation is comparable to the measured $R(\mathbf{r})$ decay length of $\gamma = 2.6$ nm, suggesting some samples may contain regions of extended but non-percolating metallic clusters. Predicting the dHvA response of such a spatially heterogeneous and strongly interacting system is a pertinent and pressing theoretical task. 


\newpage
\begin{figure}[h!]
	\includegraphics[width=5in]{FIGS1.pdf}
		\caption{{\bf Topographies for URu$_2$Si$_2$ measurements.} \label{fig:URS-topo}
    	Topographies of the $50 \times 50$ nm$^2$ regions of the U-terminated surface of URu$_2$Si$_2$ where data was collected at 
        ({\bf A}) 5.9 K and
        ({\bf B}) 18.6 K. Dark spots correspond to surface Th atoms. White spots correspond to subsurface impurities.
        }
\end{figure}


\newpage
\begin{figure}[h!]
	\includegraphics[width=4.3in]{FIGS2.pdf}
		\caption{{\bf Rectification tracks variations in $\tilde{\varepsilon}_f$.}\label{fig:chi}
		({\bf A}) Tight-binding band structure of a Kondo lattice consisting of a parabolic conduction band with Fermi wavevector $k_F^c = 0.54\ \pi/a$ and minimum $-1.6$ eV (as measured by ARPES in SmB$_6$ \cite{Jiang2013}), and an $f$ band with dispersion $\chi=0$ meV, location $\tilde{\varepsilon}_f \in [-5, 0]$ meV and hybridization strength $\nu = 50$ meV. 
        ({\bf B}) Calculated differential tunneling conductance d$I$/d$V$ for different $\tilde{\varepsilon}_f$ keeping all other parameters fixed. For these calculations, we used STM tunneling amplitudes of $t_f /t_c = 0.03$ and an $f$-electron self energy of 3 meV. 
        ({\bf C}) Calculated current-voltage curve by integrating d$I$/d$V$ in panel (B). As the peak in d$I$/d$V$ shifts to lower biases (pink curves), the current at $+V$ becomes more similar to the current at $-V$. 
        ({\bf D}) Consequently, the peak in $R(V) = |I(+V)/I(-V)|$ at $V\approx -8$ mV decreases as $\tilde{\varepsilon}_f$ is reduced.
        ({\bf E}) In a self consistent solution to Eq.~\ref{eq:Hfull}, the local doping $n_c$ (set by adjusting $\mu$) is linearly related to the Kondo resonance position $\tilde{\varepsilon}_f$.
        ({\bf F}) The rectification tracks the changes in $\tilde{\varepsilon}_f$ with highest sensitivity (steepest slope) in the Kondo metal regime ($\chi > 0$).
	}
\end{figure}


\newpage
\begin{figure}[h!]
	\includegraphics[width=5in]{FIGS3.pdf}
	\caption{\label{fig:s1}{\bf $R(\mathbf{r}, V)$ tracks the position of the Kondo resonance in URu$_2$Si$_2$.}
    ({\bf A}) Measured d$I(\mathbf{r},V)$/d$V$ on the U termination of URu$_2$Si$_2$ (open circles, each curve offset for clarity) is well described by a Fano model plus a parabolic background (solid lines). As the tip approaches a Th dopant, the energy position of the Kondo resonance ($\varepsilon_f$, black triangles) shifts to maintain a constant $f$-electron occupancy. The spectra are averaged over the 18 well-isolated thorium dopants marked in (D). 
	({\bf B}) The Kondo resonance position extracted from the Fano model displays clear oscillations.
	({\bf C}) The simultaneously measured $R(\mathbf{r}> 1\ \mathrm{nm}, V=-3\ \mathrm{mV})$ accurately tracks the variations in $\varepsilon_f(\mathbf{r}> 1\ \mathrm{nm})$  [correlation coefficient: $R^2$ = –0.95, excluding gray points in (B)].
	({\bf D}) A map of $\varepsilon_f(\mathbf{r})$ extracted from fits to Eq.~\ref{eqn:s1} at each site contains clear ripples centered around Th sites, even though the fits fail in several small patches (e.g.~black arrow).
	({\bf E}) The measured $R(\mathbf{r}, V)$ (as shown in Fig.~2D) is highly sensitive to the Kondo resonance position.
	({\bf F-G}) Both measurements reveal oscillations at the parent Fermi wavevector $2k_\mathrm{F}^\mathrm{c} \approx 0.3\ (2\pi/a)$, indicating the presence of unhybridized electrons around Kondo holes. 
	}
\end{figure}


\newpage
\begin{figure}[h!]
	\includegraphics[width=5in]{FIGS4.pdf}
	\caption{\label{fig:s2}{\bf Non-dispersive wavevector in $R(\mathbf{q}, V)$.}
    ({\bf A}) Measured $R(\mathbf{q}, V=-5\ \mathrm{mV})$ on SmB$_6$ (as shown in Fig.~3G) displays a sharp elliptical feature that has maximum contrast along the $\hat{q}_x$ direction (white dashed line). Conversely, the surface reconstruction creates a sharp peak along the $\hat{q}_y$ direction at $Q_\mathrm{Bragg} = (0, \pi/a)$.
	({\bf B}) Linecut of $R(\mathbf{q}, V)$ along $\hat{q}_x$ contains a peak at $2k_F^c$. This peak is dominant and non-dispersive for biases within the hybridization gap $|V| < \Delta/e \approx 10$ mV, but it is gradually washed out at higher $|V|$.
	}
\end{figure}


\newpage
\begin{figure}[h!]
	\includegraphics[width=5in]{FIGS5.pdf}
		\caption{{\bf Possible wavevectors of the bulk $5d$ band on a half-Sm-terminated surface.}\label{fig:2x1}
        ({\bf A}) A $(2 \times 1)$ surface reconstruction doubles the size of the unit cell along the $\hat{y}$ direction, creating domains of uniform stripes as shown in this topography. We calculated the edges of the $(2 \times 1)$ domains from the inverse Fourier transform of the $(2 \times 1)$ Bragg peak. 
        ({\bf B}) Consequently, the Brillouin zone size is halved along $k_y$, folding the $\bar{Y}$ pocket to the $\bar{\Gamma}$ point.
        ({\bf C}) Measured $R(\mathbf{q})$ for the $(2 \times 1)$ domains shown in panel (B), overlaid with all possible $5d$ wavevectors. The signal we image most closely matches the $5d$ contour originating at the $\bar{Y}$ point and folded to $\bar{\Gamma}$ by the surface reconstruction (solid black line). 
        ({\bf D-F}) Same as (A-C) but for regions of the sample with a $(1 \times 2)$ surface reconstruction, which doubles the unit cell in the $\hat{x}$ direction. This reconstruction folds the $\bar{X}$ contour to the $\bar{\Gamma}$ point, such that the dominant $R(\mathbf{q})$ ellipse is oriented along $q_x$, perpendicular to that in panel (C).
	}
\end{figure}


\newpage
\begin{figure}[h!]
	\includegraphics[width=5in]{FIGS6.pdf}
	\caption{\label{fig:s3}{\bf Metallic puddles around Gd dopants in SmB$_6$.}
    ({\bf A-C}) Topography of the $(2\times 1)$-reconstructed Sm termination of an SmB$_6$ sample lightly doped with Gd dopants (e.g.~yellow circle), in addition to native defects like Sm vacancies (red circle).
	({\bf D-F}) Measured $R(\mathbf{r}, V)$ displays similar oscillations around Gd dopants as other Sm-site defects, such as Sm vacancies.
	({\bf G-H}) Linecut of $R(\mathbf{r}, V=-2\ \mathrm{mV})$ around ({\bf G}) a Gd dopant along the dashed line in (B) and ({\bf H}) a Sm vacancy along the dashed line in (C). 
	}
\end{figure}


\newpage
\begin{figure}[h!]
	\includegraphics[width=5in]{FIGS7.pdf}
		\caption{\label{fig:size}
    	{\bf Size of the metallic puddles in SmB$_6$.}
        ({\bf A}) The oscillations in $R(\mathbf{r})$ around an Fe dopant have an estimated decay length of $\gamma=2.3$ nm, calculated by fitting the $R(\mathbf{r})$ linecut (blue points) to a decaying oscillation $\propto e^{–x/\gamma} \cos(kx)$ plus a quadratic background (black line).
        ({\bf B}) In a larger field of view containing 15 well-isolated Kondo holes, the average decay length is $\gamma=2.6$ nm, calculated by fitting $|R(q_x)|^2$ data to a Lorentzian peak plus an exponential background. 
	}
\end{figure}

\newpage
\begin{table}[h!]
    \caption{Comparison of nominal and measured doping levels for the three samples analyzed in the main text.
        }\label{tab:doping}
    \begin{ruledtabular}
    \begin{tabular}[c]{lccc}
        Sample  & Nominal  & Measured  &  Sm vacancies
        \\ \hline
        Th-URu$_2$Si$_2$    &  1\%  & 0.5\%  &  -
        \\
        Gd-SmB$_6$          & 0.3\% & 0.15\% &  0.33\%
        \\
        Fe-SmB$_6$          & 0.5\% & 0.05\% &  0.09\%
        \\
    \end{tabular}
    \end{ruledtabular}
\end{table}


\newpage
%\section{References and Notes}
\bibliography{refs}


\end{document}

