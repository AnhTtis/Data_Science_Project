
In this study. we have analysed the form of the artificial diffusion at low Mach number for three convection-pressure flux-vector splittings: Liou-Steffen, Zha-Bilgen, and Toro-Vasquez.
The approximate diffusion form of the Liou-Steffen and Zha-Bilgen splittings bear a close resemblance to that of the Roe scheme, whereas the Toro-Vasquez form does not.
We identified two forms of the energy equation component of the pressure perturbation term for the Zha-Bilgen and Toro-Vasquez splittings - the first form results in diffusion terms with a close correspondence to the original flux-vector splitting, whereas the second results in diffusion terms closer to the Roe and Liou-Steffen forms.

We then transformed the artificial diffusion to the entropy variables to gain more insight into the differences between the splittings.
The limit equations of the Liou-Steffen splitting and the second forms of the Zha-Bilgen and Toro-Vasquez splittings are identical to those of the Roe scheme for both convective and acoustic flow with both the convective and mixed diffusion scalings.
One the other hand, the first forms of the Zha-Bilgen and Toro-Vasquez splittings were found to have erroneous pressure anti-diffusion terms on the entropy equation, which remain in the limit equations with the convective diffusion scaling at both the convective and acoustic limits.
The first form of the Toro-Vasquez splitting also completely lacks the usual pressure diffusion in the pressure equation.

Examining low Mach number convection-pressure flux-vector splitting schemes in the literature, we see that the Liou-Steffen splitting is clearly the most popular, and that almost all modern schemes of all splittings use the mixed diffusion scaling.
All of the Liou-Steffen and Zha-Bilgen schemes reviewed had a form matching the general form assumed in this paper.
The Toro-Vasquez schemes, while mostly matching, apply the velocity perturbation diffusion inconsistently by applying it only to the density and energy equations and not the momentum equations.
All Zha-Bilgen and Toro-Vasquez schemes reviewed used the second form of the diffusion which more closely resembles the Roe and Liou-Steffen forms, instead of the first forms that results in the erroneous entropy generation terms.

Three numerical examples were used to verify the findings of the analysis, showing excellent agreement between prediction and results for all the splittings at both the convective and acoustic limits, including the spurious entropy generation of the first Zha-Bilgen and Toro-Vasquez forms and the significant improvements when using the second forms.

The differences between the three flux-splitting schemes considered here show that the diffusion on the energy equation is an important factor in the design of numerical schemes for low Mach number.
In particular, it is crucial to obtain the correct form of the $\delta U$ diffusion term - the spurious entropy generation of the first Zha-Bilgen and Toro-Vasquez forms is entirely due to this term, irrespective of the form of the $\delta p$ diffusion term.
This is also relevant for future studies of low Mach number numerical schemes because it is not uncommon for the barotropic Euler equations to be used for such studies, which removes the energy equation by enforcing additional constraints on the equation of state.
The barotropic equations are simpler and allow for more in-depth mathematical analysis, but clearly care must be taken when extending the findings of such analysis to the full Euler equations where the energy equation must also be considered.
Future extensions of convection-pressure flux-vector splittings could include alternative methods for ensuring correct resolution of the entropy field, the effect of including $\delta U$ in deciding the upwind direction, and other forms for the $\delta(pU)$ term in the pressure perturbation.
