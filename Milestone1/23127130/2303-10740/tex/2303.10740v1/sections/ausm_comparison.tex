Work out Jacobian coefficients for each AUSM flux
\begin{itemize}
    \item contribution of Van Leer Mach/pressure splitting to interface velocity/pressure diffusion
    \item contribution of additional diffusion terms to Jacobian
    \item analyse: AUSM+up, LDFSS2001, SLAU, SLAU-WS, AUSM+up'
    \item for each flux:
    \begin{itemize}
        \item convective or acoustic asymptotic behaviour (or neither?)
        \item collateral diffusion from enforcing correct scaling at low mach
        \item spectral radius
    \end{itemize}
    \item compare fluxes with the same asymptotic behaviour (LSDF2001 / AUSM+up, SLAU-AS / AUSM+up')
    \item analyse existing diffusion reduction strategies: (flux2 \& HR-flux)
\end{itemize}

\subsection{LDFSS2001}
\textit{Convective limit found from analogy to preconditioned system}\\

\subsection{AUSM$^+$-up}
\textit{Convective limit found from discrete asymptotic analysis}\\
AUSM$^+$-up was designed by Liou in 2006 to combine the success of pressure-velocity coupling terms of previous AUSM fluxes with the low Mach number capabilities of LDFSS2001 **.
Liou considered only the convective low Mach limit, and used a discrete asymptotic analysis to rigorously derive the required scaling of the diffusion terms as $M\to0$.
At low Mach number, the interface velocity and pressure in AUSM$^+$+up are (Appendix **):
\begin{equation} \label{eq:ausmpup:interface_velocity_lm}
    u_{1/2} = \{u\} - \frac{\llbracket p \rrbracket}{8\rho a \overline{M}} + \mathcal{O}(M^3)
\end{equation}
\begin{equation} \label{eq:ausmpup:interface_pressure_lm}
    p_{1/2} \approx \{p\} - \bigg( \frac{3\rho a \overline{M}}{8} + \frac{15\rho a M^2}{4} \bigg)\llbracket u \rrbracket + \mathcal{O}(M^4)
\end{equation}
Where $\overline{M}$ is the local Mach number clamped between a lower cutoff $M_{co}$ and 1.
The lower cutoff prevents division by zero in (\ref{eq:ausmpup:interface_velocity_lm}), and the upper limit ensures the low Mach number scaling does not affect the scheme in supersonic flow.
\begin{equation} \label{eq:mach_number_clamped}
    \overline{M} = \texttt{min}(1,\texttt{max}(M,M_{co}))
\end{equation}
From (\ref{eq:ausmpup:interface_velocity_lm}) and (\ref{eq:ausmpup:interface_pressure_lm}) it can be see that:
\begin{equation} \label{eq:ausmpup_nu_convective}
    \nu_u \approx 0,
    \quad
    \nu_p = \frac{1}{8\rho a M} \sim \mathcal{O}(M^0)
\end{equation}
\begin{equation} \label{eq:ausmpup_mu_convective}
    \mu_p \approx 0,
    \quad
    \mu_u \approx -\frac{3\rho a M}{8} \sim \mathcal{O}(M^0)
\end{equation}
Where the second term in brackets in (\ref{eq:ausmpup:interface_pressure_lm}) is an order of $M$ smaller than the first, so has been neglected.
Collecting (\ref{eq:ausmpup_nu_convective}) and (\ref{eq:ausmpup_mu_convective}) together, the diffusion matrix of AUSM$^+$-up scales as:
\begin{equation} \label{eq:ausmpup_scaling_convective}
    \underline{\underline{A}} \sim \mathcal{O}
    \begin{matrix}
    \begin{pmatrix}
        M^{-2} & 0     \\
        0      & M^{0} \\
    \end{pmatrix}
    \end{matrix}
\end{equation}
This is a diagonal approximation to the convective limit diffusion matrix (\ref{eq:convective_diffusion_scaling}).
This result is unsurprising given the use of a discrete asymptotic analysis in the derivation of AUSM$^+$-up, and almost 15 years of evidence of AUSM$^+$-up producing satisfactory results for convective low Mach number problems, however it again verifies the correctness of our method.

In the original paper, Liou recognised that the asymptotic analysis relies on the assumptions of a uniform background pressure and that a convective limit solution is sought, and recommended that for situations where these assumptions do not hold, the low Mach number scaling factor $f_a$ should be fixed equal to one.
In this case, the diagonal coefficients $\nu_p$ and $\mu_u$ decrease and increase by an order of $M$ respectively, and the approximation $\alpha\approx-\frac{3}{4}$ in the pressure splitting is no longer valid.
The resulting diffusion coefficients are:
\begin{equation} \label{eq:ausmpup_nu_acoustic}
    \nu_u \approx 0,
    \quad
    \nu_p = \frac{1}{4\rho a} \sim \mathcal{O}(M^1)
\end{equation}
\begin{equation} \label{eq:ausmpup_mu_acoustic}
    \mu_p \approx 0,
    \quad
    \mu_p \approx -\frac{21\rho a}{16} \sim \mathcal{O}(M^{-1})
\end{equation}
and the diffusion matrix scales now as:
\begin{equation} \label{eq:ausmpup_scaling_acoustic}
    \underline{\underline{A}} \sim \mathcal{O}
    \begin{matrix}
    \begin{pmatrix}
        M^{-1} & 0      \\
        0      & M^{-1} \\
    \end{pmatrix}
    \end{matrix}
\end{equation}
Which is a diagonal approximation of the acoustic limit diffusion matrix (\ref{eq:acoustic_diffusion_scaling}), showing that with $f_a=1$, AUSM$^+-up$ is suitable for low Mach number acoustic problems.

\subsection{AUSM$^+$up' \& AUSM$^+$u'p'}
\textit{use timestep ratio to decide $f_a$ for pressure diffusion on mass flux and velocity diffusion on pressure - changes $A_{11}$, $A_{12}$ and $A_{22}$ elements of Jacobian.}\\
Sachdev et al ** investigated preconditioned Roe and AUSM type fluxes for low Mach numbers, specifically flow with both convective and acoustic phenomena.
They pointed out that the large pressure diffusion in convective schemes (\ref{eq:convective_diffusion_scaling}) quickly damps acoustic waves, while the large velocity diffusion in acoustic schemes (\ref{eq:acoustic_diffusion_scaling}) severely damps convective phenomena such as vortices.
Neither form is therefore suitable for flows where convective and acoustic phenomena coexist and interact.
They formulate a 'blended' preconditioning for flux-difference splitting schemes which takes the pressure diffusion from the acoustic matrix and the velocty diffusion from the convective matrix - i.e. the smallest diffusion on each field.
This blended scheme gave improved results over a range of problems, with one exception being a low Mach number shock tube problem we will study later.
To control this blending, they include an 'unsteady Mach number' $M_u$, in the choices for the preconditioning parameter $\beta$, which is the same as $\overline{M}$ in (\ref{eq:mach_number_clamped}).
\begin{equation} \label{eq:unsteady_mach_number}
    M_u = \frac{L}{\pi a \Delta t},
\end{equation}
This definition of $M_u$ was first used by Venkateswaran \& Merkle **, who derived it as a way of improving the condition number of preconditioned dual-timestepping schemes with very small timesteps at low Mach number.g in the bac
It is commonly called an unsteady Mach number because of its use in (\ref{eq:preconditioning_parameter}) with the local and cutoff (and sometime viscous **Weiss \& Smith) Mach numbers.
However, $L/a=\tau$ is the acoustic timescale used for the two-timescale asymptotic analysis in section **, meaning that $M_u$ can also be thought of as a ratio of timescales $\tau/\Delta t$ (except for a factor of $1/\pi$ which is a result of the Fourier analysis approach of V\&M).
$M_u$ says that if $\Delta t > \tau$, the acoustic timescale is unresolved, so the convective limit scheme should be used, whereas if $\Delta t < \tau$, the acoustic timescale is resolved, so the acoustic limit scheme should be used.
Sachdev et al extend the blended preconditioning approach to AUSM$^+$-up by using (\ref{eq:unsteady_mach_number}) in the calculation of the pressure diffusion in the interface velocity (\ref{eq:ausmpup:interface_velocity_lm}). For small timesteps $M_u=1$ and the pressure diffusion is decreased by an order of $M$, becoming equivalent to the acoustic scaling (\ref{eq:ausmpup_mu_acoustic}).
The diffusion matrix of this scheme - called AUSM$^+$-up' because of the modification to the pressure diffusion - scales as:
\begin{equation} \label{eq:ausmpup'_scaling_acoustic}
    \underline{\underline{A}} \sim \mathcal{O}
    \begin{matrix}
    \begin{pmatrix}
        M^{-1} & 0     \\
        0      & M^{0} \\
    \end{pmatrix}
    \end{matrix}
\end{equation}
Where we can see that the pressure field is damped according to the acoustic scaling, and the velocity field is damped according to the convective scaling - the lowest choice for each field.
This scheme performed well for almost all test cases, and in some cases outperformed the blended preconditioning flux-difference scheme.
They both failed for a low Mach shocktube case, however a solution was proposed, where $M_u$ is used in the calculation of both the velocity and pressure diffusion terms.
For small timesteps, this scheme, called AUSM$^+$-u'p', reverts to AUSM$^+$-up without the low Mach number scaling, whose diffusion matrix was shown in (\ref{eq:ausmpup_scaling_acoustic}) - i.e. the scheme suggested by Liou for shock tube problems.
The benefit of AUSM$^+$-up' and AUSM$^+$-u'p' is that they locally adapt between different scalings, so can be used in simulations where both very high and low speed flow exist simulataneously, or between different simulations with different requirements without modification.

\subsection{SLAU}
\textit{Acoustic limit found from SHUS mass flux and "reducing" dissipation compared to AUSM+-up}\\

\subsection{SLAU-WS}
\textit{Very ad-hoc solution to alter diffusive terms based on a pressure "wiggle-sensor". When switched on, wiggle sensor returns convective scaling?}\\

\subsection{AUSM-M}
\textit{Acoustic limit from following SLAU on "reduced" mass flux dissipation and some AUSMPW+ base}\\

\subsection{A minimal AUSM scheme for either limit}
\textit{AUSM scheme with absolutely minimal possible diffusive terms and adaptive to both limits}\\

\subsection{Reduced diffusion at convective limit}
\textit{What has lower diffusion - AUSM+up, LDSF2001, SLAU-AS (at large dt)? How do hr-flux and flux2 modifications change this?}\\

\subsection{Reduced diffusion at acoustic limit}
\textit{What has lower diffusion - SLAU, AUSM+up', AUSM+u'p' (both at large dt)? How do hr-flux and flux2 modifications change this?}\\