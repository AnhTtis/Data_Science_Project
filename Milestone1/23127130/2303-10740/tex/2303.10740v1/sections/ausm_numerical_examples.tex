numerical examples
\begin{itemize}
    \item D'Alembert's paradox (canonical convective test)
    \begin{itemize}
        \item drag
        \item Non-dimensional pressure fluctuations
        \item explain influence of pressure extrapolation to wall, run all fluxes with same wall treatment for fair comparison
    \end{itemize}
    \item 1D soundwave (canonical acoustic test)
    \begin{itemize}
        \item show acoustic fluxes all work ok, convective don't
        \item see if restoring the proper scale to $\mu_u$ in the acoustic limit affects the solution.
    \end{itemize}
    \item NACA 0015 airfoil at low Mach
    \begin{itemize}
        \item this case previously shown to produce pressure decoupling with SLAU
        \item SLAU-WS fixed this with in a very ad-hoc manner
        \item show you only need to restore the proper scale to $\mu_p$ to fix this
    \end{itemize}
    \item Low mach 1D shock tube
    \begin{itemize}
        \item very low speed (M=0.0001) contact discontinuity
        \item SLAU and AUSM+up' shown unstable (both have vanishing velocity diffusion as $M\to0$
        \item show can be fixed by restoring velocity diffusion. AUSM+u'p' already does this, SLAU-WS does this in very ad-hoc fashion (show how to do properly for SLAU)
    \end{itemize}
    \item circular cylinder
    \begin{itemize}
        \item inviscid
        \begin{itemize}
            \item Convective scheme is good solution but poor (explicit) convergence
            \item Acoustic scheme is bad
            \item Blended scheme allows small checkerboard but better (explicit) convergence
        \end{itemize}
        \item shedding (Re=150), JFM2002
        \begin{itemize}
            \item Convective scheme has vortex street but no acoustics
            \item Acoustic scheme damps shedding (but maybe has farfield acoustics?)
            \item Blended scheme allows shedding and acoustics (but maybe a bit unstable at v low Mach?)
        \end{itemize}
    \end{itemize}
    \item soundwave through linear fields at acoustic limit: a) entropy bump and b) gresho vortex
    \begin{itemize}
        \item show erroneous diffusion on entropy, but not for vorticity
        \item are there any classic combustion-y test cases we could use to show what effect this actually has on cases of interest?
    \end{itemize}
\end{itemize}

\subsection{1D test cases}
\graphicspath{{./figures/OneD/}}

\subsubsection{Isolated soundwave}
\textit{1) confirm that convective fluxes are unsuitable for acoustic simulations. 2) show acoustic fluxes have excellent resolution with unlimited reconstructions, and that restoring the velocity diffusion only minorly degrades the solutions. 3) show that using fluxes with vanishing velocity diffusion with limiters produces oscillatory solutions, but that these oscillations are removed by restoring the velocity diffusion. \textcolor{red}{This section probably too verbose, try be more curt.}}\\
The first test case we show is an isolated, isentropic, sinusoidal sound-wave in a periodic domain at various Mach numbers.
This test will show the ability (or lack thereof) of a scheme to resolve acoustic waves at low Mach number.
Shima and Kitamura ** previously estimated that SLAU is able to resolve sound-waves without visible decay if at least 40ppw (points per wavelength) are used.
We simulate a marginally resolved wave with 32ppw here, as this results in enough diffusion that the accuracy of different fluxes can be differentiated with the naked eye.
All cases are run for 2048 timesteps with a CFL of 0.5, equivalent to the soundwave travelling 16 wavelengths.
For fluxes with acoustic scaling, a 3 stage, 3rd order SSP explicit Runge-Kutta scheme with a CFL bound of 1 is used **, whereas for fluxes with convective scaling, dual-timestepping with the 2nd order implicit trapezium rule is used due to the restrictive CFL limit we proved in section **.

\begin{figure}[t]
    \centering
\begin{subfigure}[t]{0.49\textwidth}
    \centering
    \includegraphics[width=0.99\linewidth]{../place_holder.jpg}
    \caption{}
    \label{fig:soundwave_convective_fluxes}
\end{subfigure}
\begin{subfigure}[t]{0.49\textwidth}
    \centering
    \includegraphics[width=0.99\linewidth]{../place_holder.jpg}
    \caption{}
    \label{fig:soundwave_acoustic_fluxes}
\end{subfigure}
    \caption{Isolated sinusoidal soundwave at Mach numbers of $10^{-2}$ and $10^{-4}$ with various AUSM fluxes. \ref{fig:soundwave_convective_fluxes} shows fluxes which converge to the convective limit, and \ref{fig:soundwave_acoustic_fluxes} shows fluxes which converge to the acoustic limit.}
    \label{fig:soundwave_unlim_fluxes}
\end{figure}

First, we show the results for AUSM$^+$-up and LDFSS2001.
These are well known to be unsuitable for acoustic simulations, as we confirmed in section ** when we showed that the diffusion of these fluxes converges to the convective limit.
However, we include them here for completeness, to show the drastic differences which are possible between schemes designed for the convective and acoustic limits.
Figure \ref{fig:soundwave_convective_fluxes} shows that after 16 periods these fluxes have completely dissipated the sound-wave, and that it only takes ** periods for the wave amplitude to be halved compared to the initial conditions, confirming the unsuitability of these fluxes for acoustic simulations.

The results for the fluxes that are suitable for the acoustic limit are shown in figure \ref{fig:soundwave_acoustic_fluxes} at Mach numbers of $10^{-2}$ and $10^{-4}$, using an unlimited third order reconstruction.
All these fluxes resolve the sound-wave with very little diffusion, in contrast to those in figure \ref{fig:soundwave_convective_fluxes}.
We also ran Mach numbers down to $10^{-12}$, but the results were indistinguishable from those at $10^{-4}$ so are not shown.
We showed in section ** that SLAU, AUSM$^+$-up' and AUSM-M all have vanishing diffusion on the velocity field as $M\to0$, whereas the velocity in SLAU-u' and AUSM$^+$-u'p' remains well-balanced on all fields.
Unsurprisingly, the two fluxes with non-vanishing diffusion produce solutions which are slightly damped compared to those with vanishing diffusion - the difference in peak velocity between SLAU and SLAU-u' is 3.2\% at $M=10^{-2}$ and 3.3\% at $M=10^{-4}$.
What is perhaps surprising is just how little the solution is affected by the restoration of diffusion on the velocity field, considering that at $M=10^{-4}$, the velocity diffusion term in the interface pressure is $10^4$ times larger in SLAU-u' than in SLAU.

As the solution is smooth, the results shown so far all used an unlimited third order reconstruction.
However, even at low Mach number these fluxes may also be used with limiters if sharp enough gradients are present in the solution (for example in LES, where there is substantial energy at the grid scale).
The results for SLAU and SLAU-u' using Cada's third order limiter ** at Mach numbers from $10^{-1}$ to $10^{-6}$ are shown in figures \ref{fig:soundwave_slau_ca_cada3} and \ref{fig:soundwave_slau_aa_cada3} respectively.
The solutions using SLAU show oscillations in the velocity from a Mach number of $M^{-2}$ and below, although the solution remains stable and somewhat recognisable.
In contrast, the solutions using SLAU-u' are monotone but, compared to figure \ref{fig:soundwave_unlim_fluxes} there is a noticeable flattening of the sin wave, although Mach independence is achieved at the same rate (below $M=10^{-2}$).
For comparison, figure \ref{fig:soundwave_slau_m1e-4} shows results at $M=10^{-4}$ for SLAU and SLAU-u', with unlimited and limited 3rd order reconstructions.
It can be seen that the unlimited SLAU solution has the least diffusion, closely followed by unlimited SLAU-u' solution (3.3\% reduction in peak value), then the limited SLAU-u' solution (18.7\% reduction in peak value).
The limited SLAU solution has oscillations, although the damping is comparable to the limited SLAU-u' solution, and the phase is still mostly accurate.
As the velocity oscillations in the SLAU solutions are not present in the pressure field (not shown), and are removed when the velocity diffusion is restored in SLAU-u', we suspect they are a result of the vanishing diffusion on the velocity field, although we have so far found no satisfactory explanation as to why they only appear in solutions using limited reconstructions.
Second order reconstructions were also tested using a central gradient estimation for the unlimited solutions, and a variaty of methods for the limited solutions, including Minmod, Superbee, Van Albada and Monotonized Central.
Qualitatively the second order results agree with the third order results across the board, so are not shown. The only noteable difference was that the solutions with SLAU and 2nd order limiters were slightly more robust to the oscillations seen in figure \ref{fig:soundwave_slau_ca_cada3}, only appearing at $M=10^{-3}$ and below.

\begin{figure}[t]
    \centering
\begin{subfigure}[t]{0.49\textwidth}
    \centering
    \includegraphics[width=0.99\linewidth]{OneD_soundwave_slau_ca_unlim3.png}
    \caption{}
    \label{fig:soundwave_slau_ca_unlim3}
\end{subfigure}
\begin{subfigure}[t]{0.49\textwidth}
    \centering
    \includegraphics[width=0.99\linewidth]{OneD_soundwave_slau_aa_unlim3.png}
    \caption{}
    \label{fig:soundwave_slau_aa_unlim3}
\end{subfigure}
    \caption{Isolated sinusoidal sound-wave at various Mach numbers, calculated with an unlimited 3rd order reconstruction. SLAU has vanishing diffusion on the velocity field as $M\to0$, whereas the diffusion in SLAU-u' remains well balanced.}
    \label{fig:soundwave_slau_unlim}
\end{figure}

\begin{figure}[t]
    \centering
\begin{subfigure}[t]{0.49\textwidth}
    \centering
    \includegraphics[width=0.99\linewidth]{OneD_soundwave_slau_ca_cada3.png}
    \caption{}
    \label{fig:soundwave_slau_ca_cada3}
\end{subfigure}
\begin{subfigure}[t]{0.49\textwidth}
    \centering
    \includegraphics[width=0.99\linewidth]{OneD_soundwave_slau_aa_cada3.png}
    \caption{}
    \label{fig:soundwave_slau_aa_cada3}
\end{subfigure}
    \caption{Isolated sinusoidal sound-wave at various Mach numbers, calculated with Cada's 3rd order limiter **. SLAU has vanishing diffusion on the velocity field as $M\to0$, whereas the diffusion in SLAU-u' remains well balanced.}
    \label{fig:soundwave_slau_cada3}
\end{figure}

\begin{figure}
    \centering
    \includegraphics[width=0.6\linewidth]{OneD_soundwave_slau_m1e-4.png}
    \caption{Isolated sinusoidal soundwave at a Mach number of $10^{-4}$ with various schemes. SLAU has vanishing diffusion on the velocity field as $M\to0$, whereas the diffusion in SLAU-u' remains well balanced. For the unlimited solutions, the peak value for SLAU-u' is 3.3\% less than the peak value for SLAU.}
    \label{fig:soundwave_slau_m1e-4}
\end{figure}

\subsubsection{Low Mach number shocktube}
\textit{1) reproduce Sachdev et al's result that AUSM+up' produces oscillations around the low Mach contact, which are removed by AUSM+u'p'. 2) link this to our result that AUSM+up' has vanishing diffusion on the velocity field, but that AUSM+u'p' restores this diffusion. 3) show comparable results for SLAU and SLAU-u' (which scale like AUSM+up' and AUSM+u'p' respectively).}\\
This test case was introduced by Sachdev et al **, as a particularly stringent test for fluxes in the acoustic low Mach number regime.
It is a shocktube problem with a very small pressure difference, and the solution consists of two very weak shockwaves travelling in either direction, and an almost stationary contact discontinuity moving at a Mach number of $10^{-4}$.
Sachdev et al found that using AUSM$^+$-up' produced oscillations in the velocity and pressure, as seen in figure \ref{fig:shocktube_ausm}, even though they had designed it for the acoustic regime.
If the initial pressure difference is increased, the Mach number is also increased, and the oscillations are reduced, until they are barely visible at $M=0.1$ **.
In addition, Sachdev et al propose a solution to remove the oscillations completely at all Mach numbers, by using the unsteady Mach number parameter in the calculation of the velocity diffusion in the interface pressure (AUSM$^+$-u'p').
It can be seen from figure \ref{fig:shocktube_ausm} that AUSM$^+$-u'p' produces a monotone solution, albeit with slightly more smearing on the shocks.
We showed earlier that the AUSM$^+$-up' flux has vanishing diffusion on the velocity field as $M\to0$, whereas AUSM$^+$-u'p' has balanced diffusion that remains in the low Mach limit.
We believe that this provides a physical justification both for increasing the velocity diffusion on the interface pressure, and for why this results in a stable scheme for this test case.

\begin{figure}[t]
    \centering
\begin{subfigure}[t]{0.49\textwidth}
    \centering
    \includegraphics[width=0.99\linewidth]{OneD_shocktube_M00001_ausm.png}
    \caption{}
    \label{fig:shocktube_ausm}
\end{subfigure}
\begin{subfigure}[t]{0.49\textwidth}
    \centering
    \includegraphics[width=0.99\linewidth]{OneD_shocktube_M00001_slau.png}
    \caption{}
    \label{fig:shocktube_slau}
\end{subfigure}
    \caption{Weak shock tube flow with a contact discontinuity travelling at $M=0.0001$. SLAU and AUSM$^+$-up' have vanishing diffusion on the velocity field as $M\to0$, whereas the diffusion in SLAU-u' and AUSM$^+$-u'p' remains well balanced.}
    \label{fig:shocktube}
\end{figure}

SLAU was also found to suffer from this instability, shown in figure \ref{fig:shocktube_slau}.
As described in section **, a rather ad-hoc solution was proposed by Shima 2013 ** to rectify both this instability and the checkerboarding at the convective limit.
This solution prevents the velocity diffusion from dropping below a certain cut-off, and uses a pressure-wiggle sensor to locally switch the pressure flux to the convective scale.
Here we propose a more physically motivated solution, by modifying the velocity diffusion in the interface pressure of SLAU in the same manner as AUSM$^+$-u'p', thus resulting in a well-balanced scheme.
The coefficient $M$ in the velocity diffusion equation ** is instead calculated using the unsteady parameter **, so becomes equal to 1 at the acoustic limit.
We call this scheme SLAU-u', and it can be seen in figure \ref{fig:shocktube_slau} that it produces no oscillations for the low Mach number shock tube, and suffers from a much smaller increase in the smearing of the shocks compared to AUSM$^+$-u'p'.

\subsubsection{Interaction of acoustic waves with linear fields}
\textit{1D gaussian soundwave passing through gaussian vorticity / entropy waves. Should show zero modification of vorticity wave before/after but some modification of entropy wave before/after due to negative diffusion terms. Effect will be greater for -u' fluxes as $\mu_u$ is order of M larger.}

\subsection{2D test cases}