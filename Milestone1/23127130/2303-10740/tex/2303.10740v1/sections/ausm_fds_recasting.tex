Recast AUSM family of fluxes as flux difference splitting scheme to isolate the diffusive flux Jacobian.
\begin{itemize}
    \item immediate form of flux Jacobian - fluxes in conserved variables, jumps in u,p and convected variables
    \item transform to symmetry variables
    \item show it doesn't matter (for us) if you define AUSM with interface velocity or interface mass flow
    \item analysis of general ausm flux Jacobian
    \begin{itemize}
        \item which AUSM parameters affect which fields?
        \item which AUSM parameters decide the scaling necessary for low mach asymptotes?
        \item coupling of linear to nonlinear fields / vice-versa?
        \item any other obvious simplifications or desirable features?
        \item general expressions for eigenvalues of ausm fluxes
    \end{itemize}
\end{itemize}

\subsection{Flux formulations}
We briefly describe the formulation of AUSM schemes and flux-difference splitting schemes, as preparation for transforming AUSM into an equivalent flux-difference splitting form.
\subsubsection{AUSM family fluxes}
The AUSM family of fluxes have the following general form:
\begin{equation} \label{eq:ausm_form}
    \underline{\hat{f}} =  \frac{u_{1/2} + |u_{1/2}|}{2}\underline{\varphi}_{L}
                         + \frac{u_{1/2} - |u_{1/2}|}{2}\underline{\varphi}_{R}
                         + p_{1/2}\underline{N}
\end{equation}
\begin{equation}
\underline{\varphi} = 
    \begin{matrix}
    \begin{pmatrix}
        \rho \\ \rho u \\ \rho v \\ \rho h
    \end{pmatrix}
    \end{matrix},
    \quad
    \quad
\underline{N} = 
    \begin{matrix}
    \begin{pmatrix}
        0 \\ n_x \\ n_y \\ 0
    \end{pmatrix}
    \end{matrix}
\end{equation}
where underlining indicates vector valued quantities, $\underline{\hat{f}}$ is the interface flux, $\underline{\varphi}$ is the vector of convected quantities, $\underline{N}$ is the (zero padded) interface normal vector, and $u_{1/2}$ and  $p_{1/2}$ are interface velocity and pressure respectively.
A specific AUSM scheme is uniquely determined by the definition of interface velocity and pressure.
Note that this form differs slightly from the usual presentation, where the interface velocity is replaced by an interface mass flow $\dot{m}_{1/2}$, and $\underline{\varphi}=(1, u, v, h)$ in equation (\ref{eq:ausm_form}).
However, usually the interface mass flow is defined as:
\begin{equation} \label{eq:interface_mass_flow}
    \dot{m}_{1/2} = u_{1/2}
    \begin{cases}
        \rho_L, & \text{if}\ u_{1/2} > 0 \\
        \rho_R, & \text{otherwise}
    \end{cases}
\end{equation}
making the two forms equivalent.
One notable exception is SLAU **, where the interface mass flow is not calculated using equation (\ref{eq:interface_mass_flow}).
However, they show in their paper (equations (24b) and (24c)) that the interface mass flux function of SLAU reduces to (\ref{eq:interface_mass_flow}) if both the left and right velocities are of the same sign.
As we expect this to be the case at most cell interfaces, we will use this simplification when we consider SLAU later in the paper.
We show the conditions which must hold to make this simplification more generally in Appendix **.

\subsubsection{Flux difference splitting fluxes}
Flux difference splitting (FDS) schemes have the following general form:
\begin{equation} \label{eq:fds_form}
    \underline{\hat{f}} = \{\underline{f}\} - \frac{|\underline{\underline{A}}|}{2} \llbracket \underline{q} \rrbracket
\end{equation}
where $\{(\cdot)\} = \frac{(\cdot)_L + (\cdot)_R}{2}$ is the arithmetic average and $\llbracket (\cdot) \rrbracket = (\cdot)_R - (\cdot)_L$ is the jump over an interface. $\underline{f}$ is the exact inviscid flux function, $|\underline{\underline{A}}|$ is the diffusive flux Jacobian where $|(\cdot)|$ indicates positive (semi-)definiteness, and $\underline{q}$ is the vector of state variables.
The diffusive flux Jacobian $|\underline{\underline{A}}|$ uniquely determines an FDS scheme.
The FDS form is useful because it distinguishes between the non-diffusive central part of the flux $\{\underline{f}\}$, and the diffusive upwind correction $|\underline{\underline{A}}| \llbracket \underline{q} \rrbracket$.
If the diffusive flux Jacobian is in symmetrising form, we can make direct comparison to the artificial diffusion in equation (\ref{eq:euler_symmetric}), and make use of the results of the previous section for numerical schemes at low Mach number.

\subsection{AUSM fluxes in FDS form}
The $\frac{u_{1/2} \pm |u_{1/2}|}{2}$ coefficients in equation (\ref{eq:ausm_form}) simply act as a switch to upwind the convected variables $\underline{\varphi}$ so, without loss of generality, we assume $u_{1/2} > 0$ and reduce equation (\ref{eq:ausm_form}) to:
\begin{equation} \label{eq:ausm_form2}
    \underline{\hat{f}} = u_{1/2}\underline{\varphi}_L + p_{1/2}\underline{N}
\end{equation}
To rewrite equation (\ref{eq:ausm_form2}) in FDS form (equation (\ref{eq:fds_form})), we write the interface velocity and pressure as a central approximation plus some diffusive terms:
\begin{equation} \label{eq:interface_up}
    \begin{split}
        u_{1/2} & = \{u\} - u_d \\
        p_{1/2} & = \{p\} - p_d
    \end{split}
\end{equation}
By adding $\big(u_R\underline{\varphi}_R - u_R\underline{\varphi}_R\big)/2$ to the right hand side of equation (\ref{eq:ausm_form2}) and rearranging, we find:
\begin{equation} \label{eq:ausm_form3}
    \underline{\hat{f}} = \{\underline{f}\} - \frac{u_R}{2}\llbracket\underline{\varphi}\rrbracket - u_d\underline{\varphi}_L - p_d\underline{N}
\end{equation}
where we see the interface flux is now composed of a central approximation plus some diffusive terms.
For most AUSM fluxes, the diffusive terms of the interface velocity and pressure are functions only of the jumps in the pressure and face normal velocity, so $u_d$ and $p_d$ can be written as:
\begin{equation} \label{eq:interface_up_diffusion}
    \begin{split}
        u_d & = \nu_p \llbracket p \rrbracket + \nu_u \llbracket u \rrbracket \\
        p_d & = \mu_p \llbracket p \rrbracket + \mu_u \llbracket u \rrbracket
    \end{split}
\end{equation}
We can now write the diffusive terms of equation \ref{eq:ausm_form3} as:\\
\textit{\textcolor{red}{The velocity and pressure diffusion terms in this equation and the next can be written as matrices or as Cartesian products between two vectors. Which is better? The matrix form is more standard, but I like the Cartesian product form as it clearly shows the structure, and that these terms are rank-deficient (is the rank-deficiency even important? Does the rank deficiency show that there exists a basis where each term only affects one field, which would be useful?)}}
\begin{equation} \label{eq:ausm_diffusion_immediate}
    u\underline{\underline{I}}
    \begin{matrix}
        \begin{pmatrix}
            \llbracket \rho  \rrbracket \\
            \llbracket \rho u\rrbracket \\
            \llbracket \rho v\rrbracket \\
            \llbracket \rho h\rrbracket
        \end{pmatrix}
    \end{matrix}
    +
    \Bigg[
    \begin{matrix}
        \begin{pmatrix}
            \nu_p \rho    & \nu_u \rho   \\
            \nu_p \rho u  & \nu_u \rho u \\
            \nu_p \rho v  & \nu_u \rho v \\
            \nu_p \rho u  & \nu_u \rho h
        \end{pmatrix}
    \end{matrix}
    +
    \begin{matrix}
        \begin{pmatrix}
            0     & 0 \\
            \mu_p & \mu_u \\
            0     & 0 \\
            0     & 0
        \end{pmatrix}
    \end{matrix}
    \Bigg]
    \begin{matrix}
        \begin{pmatrix}
            \llbracket p \rrbracket \\
            \llbracket u \rrbracket
        \end{pmatrix}
    \end{matrix}
\end{equation}

\begin{equation}
    u\underline{\underline{I}}
    \begin{matrix}
        \begin{pmatrix}
            \llbracket \rho  \rrbracket \\
            \llbracket \rho u\rrbracket \\
            \llbracket \rho v\rrbracket \\
            \llbracket \rho h\rrbracket
        \end{pmatrix}
    \end{matrix}
    +
    \Bigg[
    \begin{matrix}
        \begin{pmatrix}
            \rho   \\
            \rho u \\
            \rho v \\
            \rho h 
        \end{pmatrix}
        \begin{pmatrix}
            \nu_p & \nu_u
        \end{pmatrix}
    \end{matrix}
    +
    \begin{matrix}
        \begin{pmatrix}
            0 \\
            1 \\
            0 \\
            0 
        \end{pmatrix}
        \begin{pmatrix}
            \mu_p & \mu_u
        \end{pmatrix}
    \end{matrix}
    \Bigg]
    \begin{matrix}
        \begin{pmatrix}
            \llbracket p \rrbracket \\
            \llbracket u \rrbracket
        \end{pmatrix}
    \end{matrix}
\end{equation}

Where we have rotated the coordinate system to be aligned with the interface so that the pressure flux appears only in the normal momentum equation, and dropped the $L/R$ subscripts on the state variables.
At low Mach number the flow should be smooth, meaning the relative difference between left and right states will be small.
As we are interested in general behaviour and scaling, this simplification should not be of consequence.
The first term in equation (\ref{eq:ausm_diffusion_immediate}) is due to the upwinding of the convected variables, while the second and third terms are due to the diffusive contributions to the interface velocity and pressure respectively.

We transform equation (\ref{eq:ausm_diffusion_immediate}) to entropy variables by pre-multiplying every term by the Jacobian matrix $\frac{\partial \underline{q}^s}{\partial \underline{q}^c}$, substituting $\llbracket \underline{\varphi} \rrbracket = \frac{\partial \underline{\varphi}}{\partial \underline{q}^s} \llbracket \underline{q}^s \rrbracket$ in the upwinding term, and substituting $\llbracket p \rrbracket = \rho a \llbracket \phi \rrbracket$ in the last two terms (see Appendix **).
After these manipulations, equation (\ref{eq:ausm_diffusion_immediate}) becomes:
\begin{equation} \label{eq:ausm_diffusion_symmetric}
    \Bigg[
    u
    \begin{matrix}
        \begin{pmatrix}
            \gamma & 0 & 0 & 0 \\
            0      & 1 & 0 & 0 \\
            0      & 0 & 1 & 0 \\
            \Gamma & 0 & 0 & 1
        \end{pmatrix}
    \end{matrix}
    + 
    \begin{matrix}
        \begin{pmatrix}
            1 \\
            0 \\
            0 \\
            0 
        \end{pmatrix}
        \begin{pmatrix}
            \rho a^2\nu_p & a\nu_u & 0 & 0
        \end{pmatrix} \\
    \end{matrix}
    +
    \begin{matrix}
        \begin{pmatrix}
            -\Gamma u/a \\
            1 \\
            0 \\
            -\Gamma u/a \\
        \end{pmatrix}
        \begin{pmatrix}
            a\mu_p & \mu_u/\rho & 0 & 0
        \end{pmatrix} \\
    \end{matrix}
    \Bigg]
    \begin{matrix}
        \begin{pmatrix}
            \llbracket \phi  \rrbracket \\
            \llbracket u     \rrbracket \\
            \llbracket v     \rrbracket \\
            \llbracket \zeta \rrbracket
        \end{pmatrix}
    \end{matrix}
\end{equation}

where $\Gamma=\gamma-1$.
First we look at the upwinding contribution.
The velocity and vorticity are upwinded by the correct magnitude ($u$), and are not affected by jumps in other variables in this term.
However, the pressure is over-upwinded by a factor of $\gamma$, and the entropy is erroneously upwinded by jumps in the pressure, which introduces coupling between the linear and acoustic subsystems which is not present in the physical flux Jacobian (\ref{eq:symmetric_jacobian}).
These "imperfections" in the upwinding are because, in the standard AUSM form (\ref{eq:ausm_form}), the upwinding is on the convected variables $\underline{\varphi}$ but fluxes are calculated in the conserved variables $\underline{q}^c$.
As a result, the matrix multiplying $u$ in the upwinding is equal to $\frac{\partial \underline{q}^s}{\partial \underline{q}^c} \frac{\partial \underline{\varphi}}{\partial \underline{q}^s}$.

The second term in (\ref{eq:ausm_diffusion_symmetric}), from the diffusive terms in the interface velocity, affects only the pressure equation.
This will prove useful later when we estimate the required scaling of these terms, as we can make these estimates based only on requirements on the pressure field, without affecting any other field.

Finally, we have the contribution from the diffusive terms in the interface pressure.
This provides diffusion only on the velocity field which, like for the interface velocity diffusion, we will use later to estimate the required scaling of these terms.
However, the diffusion on the interface pressure also results in anti-diffusion terms in the pressure and entropy equations, which can be seen from the $-\Gamma u/a$ terms with negative sign.
This is problematic only if these anti-diffusion terms do not disappear in the limit equations, i.e. if they are the same order of $M$, or lower, than the largest term in the pressure or entropy equations.
The anti-diffusion on the entropy from the interface pressure diffusion introduces additional coupling between the acoustic subsystem and the entropy field, on top of the coupling caused by the erroneous upwinding of the entropy by jumps in the pressure.

We now write out the diffusion on each equation separately: \\
\begin{subequations}
Symmetric pressure $\phi$:
\begin{equation}
    \big(u(\mu_p + \gamma(1 - \mu_p)) + \rho a^2 \nu_p \big) \llbracket \phi \rrbracket + 
    \big(a\nu_u -   \frac{(\gamma-1)u}{\rho a}\mu_u \big)   \llbracket u    \rrbracket
    \label{eq:sym_pressure_diffusion}
\end{equation}
Velocity $u$:
\begin{equation}
    a\mu_p\big                       \llbracket \phi \rrbracket + 
    \big(u + \frac{\mu_u}{\rho}\big) \llbracket u \rrbracket
    \label{eq:sym_velocity_diffusion}
\end{equation}
Vorticity $v$:
\begin{equation}
    u \llbracket v \rrbracket
    \label{eq:sym_vorticity_diffusion}
\end{equation}
Symmetric entropy $\zeta$:
\begin{equation}
    u(\gamma-1)(1-\mu_p)              \llbracket \phi \rrbracket - 
    \frac{(\gamma-1)u}{\rho a}\mu_u   \llbracket u    \rrbracket +
    u                                 \llbracket \zeta \rrbracket
    \label{eq:sym_entropy_diffusion}
\end{equation}
\end{subequations}

In the pressure and velocity equations, we see that $\nu_p$, $\nu_u$, $\mu_p$ and $\mu_u$ provide dissipation in the $A_{11}$, $A_{12}$, $A_{21}$, $A_{22}$ terms respectively, which will be used shortly to derive the required scaling of these coefficients to achieve accurate solutions in the convective or acoustic limits as $M\to0$.
We can also see the negative diffusion in the pressure equation with $\mu_p$ and $\mu_u$ as previously noted.
The terms $\nu_p$ and $\mu_u$ are often called the pressure-velocity coupling, as they add pressure diffusion to the interface velocity, and velocity diffusion to the interface pressure respectively.
We can see from (\ref{eq:sym_pressure_diffusion}) and (\ref{eq:sym_velocity_diffusion}) however that the role they play in the symmetrised variable form is to provide diffusion on the pressure field with jumps in the pressure, and diffusion on the velocity field with jumps in the velocity, which will help damp oscillations in these fields.

Interestingly, the diffusion on the vorticity equation is "ideal", in the sense that it reduces exactly to an upwinding with the scale of the local velocity, with no coupling to any other fields, which mirrors exactly the form of the physical flux Jacobian.
The upwinding on the entropy also scales with the local velocity, but as noted previously, there are some erroneous terms which couple the entropy to the acoustic subsystem.

\subsubsection{Requirements on the interface pressure and velocity diffusion}
\textit{Work out how $\nu_p$, $\nu_u$, $\mu_p$ and $\mu_u$ need to scale for the $A_{ij}$ coefficients to scale correctly in the convective and acoustic limits. Is there any freedom for choices or are you fully constrained?}

We determine the required scaling of the AUSM parameters $\nu_p$, $\nu_u$, $\mu_p$ and $\mu_u$ by comparing the scaling of the artificial dissipation coefficients (\ref{eq:convective_diffusion_scaling}) and (\ref{eq:acoustic_diffusion_scaling}) with the diffusion terms (\ref{eq:sym_pressure_diffusion}) and (\ref{eq:sym_velocity_diffusion}).
First, we consider the off-diagonal terms $A_{12}$ and $A_{21}$, as these have the same scaling $\mathcal{O}(M^{-1})$ at both the convective and acoustic limits.
Using $a\sim\mathcal{O}(M^{-1})$ and ignoring anti-diffusion terms, we find:
\begin{equation}
    A_{12} \sim \mathcal{O}(\nu_u M^{-1}),
    \quad
    A_{21} \sim \mathcal{O}(\mu_p M^{-1})
\end{equation}
So for correct solutions at either limit, we must have:
\begin{equation} \label{eq:offdiagonal_ausm_scaling}
    \nu_u \sim \mathcal{O}(M^0),
    \quad
    \mu_p \sim \mathcal{O}(M^0)
\end{equation}
These parameters can be asymptotically smaller, but if so then the off-diagonal terms disappear from the limit equations as $M\to0$.
Next, we look at the diagonal terms $A_{11}$ and $A_{22}$.
Again ignoring anti-diffusive terms, we find:
\begin{equation}
    A_{11} \sim \mathcal{O}(\nu_p M^{-2}),
    \quad
    A_{22} \sim \mathcal{O}(\mu_u M^{0})
\end{equation}
By comparing to the scaling in the convective limit (\ref{eq:convective_diffusion_scaling}), we must have:
\begin{equation} \label{eq:diagonal_ausm_convective_scaling}
    \nu^c_p \sim \mathcal{O}(M^0),
    \quad
    \mu^c_u \sim \mathcal{O}(M^0)
\end{equation}
and comparing to the scaling in the acoustic limit (\ref{eq:acoustic_diffusion_scaling}), we must have:
\begin{equation} \label{eq:diagonal_ausm_acoustic_scaling}
    \nu^a_p \sim \mathcal{O}(M^1),
    \quad
    \mu^a_u \sim \mathcal{O}(M^{-1})
\end{equation}
These scaling requirements are summarised in Table \ref{table:ausm_parameter_scaling}.
At the convective limit, all diffusion coefficients must become independent of the Mach number, which makes intuitive sense as in this limit we expect to approach the incompressible solution, where the Mach number loses any real meaning.
At the acoustic limit, the diffusion on the pressure must be decreased to allow a first order variation (compared to the second order variation at the convective limit).
As discussed in section **, the diffusion on the velocity must be increased, because the magnitude of the term $a\partial_x \phi$ on the left hand side of equation (\ref{eq:velocity_symmetric}) increases by an order of $M$ relative to the convective limit.
\begin{table}
\begin{center}
\begin{tabular}{|c|c|c|c|} \hline
             & Convective         & Acoustic              & Blended             \\ \hline
     $\nu_p$ & $\mathcal{O}(M^0)$ & $\mathcal{O}(M^1)$    & $\mathcal{O}(M^1)$  \\ \hline
     $\nu_u$ & $\mathcal{O}(M^0)$ & $\mathcal{O}(M^0)$    & $\mathcal{O}(M^0)$  \\ \hline
     $\mu_p$ & $\mathcal{O}(M^0)$ & $\mathcal{O}(M^0)$    & $\mathcal{O}(M^0)$  \\ \hline
     $\mu_u$ & $\mathcal{O}(M^0)$ & $\mathcal{O}(M^{-1})$ & $\mathcal{O}(M^0)$  \\ \hline
\end{tabular}
\caption{Required scaling of the interface velocity and pressure diffusion coefficients in AUSM schemes to achieve accurate solutions at the convective and acoustic low Mach number limits.}
\label{table:ausm_parameter_scaling}
\end{center}
\end{table}

Now we know the correct scaling of the diffusion coefficients at both limits, we consider the anti-diffusive terms on the pressure (\ref{eq:sym_pressure_diffusion}) and entropy (\ref{eq:sym_entropy_diffusion}) fields.
The anti-diffusive terms on the pressure in the convective limit are:
\begin{equation}
    -\Gamma u \mu_p \sim \mathcal{O}(M^0),
    \quad
    -\frac{\Gamma u}{\rho a} \mu_u \sim \mathcal{O}(M^1)
\end{equation}
and in the acoustic limit:
\begin{equation}
    -\Gamma u \mu_p \sim \mathcal{O}(M^0),
    \quad
    -\frac{\Gamma u}{\rho a} \mu_u \sim \mathcal{O}(M^0)
\end{equation}
Which means that the anti-diffusive terms in the pressure equation are always at least one order of $M$ smaller than the diffusive terms, in both limits, so should not cause any instabilities in the scheme.

\subsubsection{Diffusion on the entropy field}
\textit{The diffusion on entropy has contributions from the acoustic subsystem. This seems less than ideal (especially for combustion, which is something the JAXA guys want to do with SLAU). Can this diffusion be decreased while still conforming to the restrictions for proper scaling? You know what a sound wave looks like (from the acoustic eigenvectors), so you can get an estimate for the eronious diffusion on entropy from a single forward/backward travelling soundwave.}

For the entropy equation, recall that we do not know the scaling of the entropy field unless we have information on the entropy generation mechanisms in a particular case (usually viscous effects and/or heat addition either from an outside source or chemical reaction).
\textit{\textcolor{red}{Small comment on scaling of entropy in last paragraph of section 2.2. Definitely feels like conjecture, would be good to see if there are any proper estimations for entropy scaling, as it's a bit wooly at the moment.}}
Because of this, we consider the entire diffusion terms, not just the coefficients.

\begin{subequations} \label{eq:convective_entropy_diffusion_scaling}
Assuming the coefficients $\mu_p$ and $\mu_u$ have the correct scale, in the convective limit we have for the diffusive terms:
\begin{equation}
    u\Gamma\llbracket\phi\rrbracket \sim \mathcal{O}(M),
    \quad
    u\llbracket\zeta\rrbracket \sim \mathcal{O}(\zeta)
\end{equation}
and the anti-diffusive terms:
\begin{equation}
    -\Gamma u \mu_p \llbracket\phi\rrbracket \sim \mathcal{O}(M),
    \quad
    -\frac{\Gamma u}{\rho a} \mu_u \llbracket u \rrbracket \sim \mathcal{O}(M)
\end{equation}
\end{subequations}
All the unwanted terms (the terms coupling the entropy to the acoustic subsystem) are $\mathcal{O}(M)$.
If we say that, in the convective limit, we expect to converge to the incompressible solution, then it seems reasonable to assume that the entropy field becomes independent of the Mach number $\zeta\sim\mathcal{O}(M^0)$.
If this assumption holds, then the unwanted terms disappear from the limit equation, and the leading order diffusion on the entropy field is an upwinding term with the correct scale, $u$.

\begin{subequations} \label{eq:acoustic_entropy_diffusion_scaling}
Repeating for the acoustic limit, the diffusive terms in the entropy equation are:
\begin{equation}
    u\Gamma\llbracket\phi\rrbracket \sim \mathcal{O}(M^0),
    \quad
    u\llbracket\zeta\rrbracket \sim \mathcal{O}(\zeta)
\end{equation}
and the anti-diffusive terms:
\begin{equation}
    -\Gamma u \mu_p \llbracket\phi\rrbracket \sim \mathcal{O}(M^0),
    \quad
    -\frac{\Gamma u}{\rho a} \mu_u \llbracket u \rrbracket \sim \mathcal{O}(M^0)
\end{equation}
\end{subequations}
In the acoustic limit, the unwanted terms are an order of $M$ larger than in the convective limit, because both $\llbracket\phi\rrbracket$ and $\mu_u$ are an order of $M$ larger.
We do not attempt to estimate the scaling of the entropy field in the acoustic limit here, as this is a complicated topic in and of itself.
For example, we might expect that the viscous entropy generation still be independent of the Mach number, but other phenomena such as entropy noise, combustion and other chemistry may well have a dependence on the Mach number.
We do however explore the coupling of the entropy field to the acoustic subsystem with a numerical example later in the paper.


\subsubsection{Spectral radius of AUSM schemes}
In section ** we found that for schemes which converge to the acoustic limit, the spectral radius (and hence the stability bound) should grow at the same rate as that of the physical flux, $\mathcal{O}(M^{-1})$.
However, for schemes which converge to the convective limit, we estimated that the spectral radius should grow an order of the Mach number faster, $\mathcal{O}(M^{-2})$, due to the contribution of the large diffusion on the pressure field required to enforce a second order pressure variation.

Now we have an approximation of the diffusive flux Jacobian of AUSM schemes (\ref{eq:ausm_diffusion_symmetric}), we can find exact expressions for the eigenvalues of this matrix, and compare how these scale if we assume the diffusion coefficients have the scales we estimated in the previous section (Table \ref{table:ausm_parameter_scaling}).
The exact expressions have been calculated using the symbolic manipulation package Sympy ** (the code for this is included in the supplementary material).
There are two eigenvalues $\lambda_{0,1}$ which are exactly equal to the convective velocity $u$, identically to the physical flux Jacobian.
The corresponding eigenvectors $\mathcal{R}_{0,1}$ are the vorticity and entropy fields:
\begin{equation} \label{eq:linear_eigenvalues_vectors}
    \lambda_{0,1} = u,
    \quad
    \mathcal{R}_0 = 
    \begin{matrix}
    \begin{pmatrix}
    0 \\ 0 \\ 1 \\ 0
    \end{pmatrix}
    \end{matrix},
    \quad
    \mathcal{R}_1 = 
    \begin{matrix}
    \begin{pmatrix}
    0 \\ 0 \\ 0 \\ 1
    \end{pmatrix}
    \end{matrix},
\end{equation}
The expressions for the remaining two eigenvalue/eigenvector pairs, which relate to the acoustic subsystem, are much more complex and unwieldy, so are not included in the main text - the interested reader can refer to Appendix ** for the eigenvalues, and the supplementary material for the eigenvectors.
Instead, by using the scaling requirements in Table \ref{table:ausm_parameter_scaling}, we give leading order estimations to the the largest eigenvalues at the convective and acoustic limits.
At the convective limit we have:
\begin{equation} \label{eq:convective_spectral_radius}
\begin{split}
    \lambda^c_4 & \approx \frac{( \rho a^2 \nu_p + \mathcal{O}(M^{0}) ) + \sqrt{\rho^2 a^4 \nu_p^2 + \mathcal{O}(M^{-2})}}{2} \\
    \lambda^c_4 & \approx \frac{\rho a^2 \nu_p}{2} \\
    \lambda^c_4 & \sim \mathcal{O}(M^{-2})
\end{split}
\end{equation}
This confirms our estimate that AUSM schemes which converge to the convective limit as $M\to0$ have a spectral radius which grows severely as $\mathcal{O}(M^{-2})$.
Previous works in the literature ** have reported instabilities at very low Mach number ($M<0.1$) when using AUSM$^+$-up with explicit schemes.
AUSM$^+$-up was designed for the convective limit, and, as we will show later, the diffusion coefficients have the scaling we found in Table \ref{table:ausm_parameter_scaling}.
The 'fixes' used to alleviate this instability all reduce the value of $\nu_p$ (usually with a cut-off Mach number), which would imply that the instabilities were a result of using a timestep size according to the physical spectral radius, not the one in (\ref{eq:convective_spectral_radius}).
Because the scaling (\ref{eq:convective_spectral_radius}) comes from the requirements on the artificial diffusion, we suggest that it is not a deficiency of the scheme per-se, but rather the necessary price for achieving accurate solutions at the convective limit.
\textit{It's known that implicit schemes are needed for time-accurate preconditioned Roe schemes (convective limit). This shows the same is true for convective limit AUSM schemes. Possible heuristic is that the requirement that the pressure react "instantaneously" to the divergence means some information must travel faster than any physical wave. Incompressible schemes solve the pressure poisson implicity, filtered low mach schemes normally solve some implicit equation for the pressure, why should density based scheme be any different?}

The leading order expression for the largest eigenvalue at the acoustic limit is:
\begin{equation}
\begin{split}
    \lambda^a_4 & \approx \frac{( \mu_u + \mathcal{O}(M^{0}) ) + \sqrt{ (\rho^2 a^2 \nu_p - \mu_u)^2 + \mathcal{O}(M^{-1})}}{2\rho} \\
    \lambda^a_4 & \approx \frac{\rho a^2 \nu_p}{2} \\
    \lambda^a_4 & \sim \mathcal{O}(M^{-1})
\end{split}
\end{equation}
Now the spectral radius scales at the same rate as the physical one, and the stability bound should be well estimated with the usual estimates.
It is interesting that, although $\nu_p$ is asymptotically the smallest parameter in the acoustic limit - 2 orders of $M$ smaller than $\mu_u$ - the leading order approximation of the spectral radius still simplifies down to only contain this parameter.
It is also identical to the leading order approximation at the convective limit, which could be useful as a simple estimate for the spectral radius in a practical CFD code.

\textit{Reference to stability restriction for AUSM+up: Matsuyama, channel flow. HR-SLAU2 papers, 2D shear layer roll up. Iampietro, Daude, Galon, "A low-diffusion self-adaptive flux-vector splitting approach for compressible flows". Chen, Yan, Lin, 2018 - An improved entropy-consistent Euler flux in low Mach number.}