
The form (\ref{eq:fds-interface-flux},\ref{eq:fds-interface-delta-up}) is useful for analysing interface fluxes as it explicitly separates the non-diffusive central component from the stabilising artificial diffusion components, however the artificial diffusion in convection-pressure flux-vector splitting schemes is usually implicit in the formulation.
In this section, we construct approximate forms for three families of convection-pressure flux-vector splittings in the conserved variables that explicitly separate the artificial diffusion terms, so that we can make comparison to (\ref{eq:fds-interface-flux},\ref{eq:fds-interface-delta-up}).
We then transform the splittings to the entropy variables to make direct comparison to (\ref{eq:euler_modified}) and the results of \cite{hope-collins_artificial_2022}.

\subsection{Conserved variables form of convection-pressure flux-vector splittings}\label{sec:cp-splittings-conserved}

Convection-pressure flux-vector splittings split the physical flux vector $\underline{f}$ into two components:
\begin{equation}\label{eq:general-cp-flux-splitting}
    \underline{f} = U\underline{\varphi} + \underline{P}
\end{equation}
where $U$ is the local convecting velocity, $\underline{\varphi}$ is some vector of convected quantities, and $\underline{P}$ is some vector representing the pressure flux.
The choice of $\underline{\varphi}$ and $\underline{P}$ specify the flux-splitting.
For the three flux-splittings considered in this paper, the pressure flux can be written as $\underline{P}=p\underline{\Pi}$, where $p$ is the local pressure.
A numerical interface flux can be constructed from the flux-splitting (\ref{eq:general-cp-flux-splitting}) with the form:
\begin{equation}\label{eq:general-cp-interface-flux}
    \underline{\hat{f}} = U_{1/2}\underline{\varphi} + \underline{P}_{1/2}
\end{equation}
where $U_{1/2}$ is an interface velocity and $\underline{P}_{1/2}$ is an interface pressure flux.
We assume that the interface velocity and pressure can be rearranged (possibly approximately) into the form:
\begin{equation}\label{eq:general-interface-up}
    U_{1/2} = \{U\} - \delta U,
    \quad
    \underline{P}_{1/2} = \{\underline{P}\} - \delta\underline{P}
\end{equation}
where the interface velocity and pressure perturbations $\delta U$ and $\delta\underline{P}$ are diffusion terms similar to those in (\ref{eq:fds-interface-delta-up}).
The interface flux (\ref{eq:general-cp-interface-flux}) can then be rearranged into a form that separates the central and diffusive components by assuming $U_{1/2}>0$ and $\underline{\varphi}$ is taken as the upwind value based on $U_{1/2}$, as is usually the case.
Adding $(U_R\underline{\varphi}_R - U_R\underline{\varphi}_R)$ to the right hand side, we obtain:
\begin{equation}\label{eq:discrete-cp-diffusion}
    \underline{\hat{f}} = \{\underline{f}\} - \Big(\frac{U_R}{2}\Delta\underline{\varphi} + \delta U\underline{\varphi}_L + \delta\underline{P}\Big)
\end{equation}
The terms in brackets are all artificial diffusion terms.
Assuming smooth flow, the differences between left and right values are small, and we can approximate the flux as:
\begin{equation}\label{eq:general-cp-diffusion}
    \underline{\hat{f}} = \{\underline{f}\} - \Big(\frac{U}{2}\Delta\underline{\varphi} + \delta U\underline{\varphi} + \delta\underline{P}\Big)
\end{equation}
where $U$ and $\underline{\varphi}$ are some consistent approximation of the local values.
Comparing to equations (\ref{eq:fds-interface-flux},\ref{eq:fds-interface-delta-up}) in section \ref{sec:artificial-diffusion}, it can be seen that equation (\ref{eq:general-cp-diffusion}) closely approximates the Roe form of the artificial diffusion.
The first diffusion term is the convective upwinding, the second term is diffusion due to the interface velocity perturbation, and the third term is due to the interface pressure flux perturbation.
The interface velocity perturbation $\delta U$ is assumed to have the same form as in the Roe diffusion (\ref{eq:fds-interface-delta-up}), however the form of interface pressure flux perturbation $\delta\underline{P}$ can be different for each splitting.
The recasting of the convection term into central and upwinding components was shown for AUSM in the original paper from Liou \& Steffen \cite{liou_new_1993}, and the splitting of the pressure term was also used by Liou \cite{liou_sequel_1996} and Edwards \& Liou \cite{edwards_low-diffusion_1998,liou_ausm_1999,edwards_reflections_2019}.
The CUSP scheme \cite{jameson_artificial_1993} uses explicit diffusion terms with a similar form.
Now we have an explicit expression for the artificial diffusion we can substitute the specific forms of $\underline{\varphi}$ and $\underline{P}$ for each flux splitting.

\subsubsection{Liou-Steffen}
For the Liou-Steffen splitting we have:
\begin{equation}\label{eq:ls-splitting}
    \underline{\varphi} = 
    \begin{matrix}
    \begin{pmatrix}
        \rho \\ \rho u \\ \rho v \\ \rho H
    \end{pmatrix}
    \end{matrix},
    \quad
    \underline{P} = 
    p\underline{\Pi} =
    p
    \begin{matrix}
    \begin{pmatrix}
        0 \\ n_x \\ n_y \\ 0
    \end{pmatrix}
    \end{matrix},
    \quad
    \delta \underline{P} = \delta p \underline{\Pi}
\end{equation}
where $\delta p$ is the same interface pressure perturbation as the Roe diffusion (\ref{eq:fds-interface-delta-up}).
Substituting (\ref{eq:ls-splitting}) into (\ref{eq:general-cp-diffusion}), the approximate form of the diffusion is:
\begin{equation}\label{eq:ls-diffusion}
    \underline{\hat{f}} = \{\underline{f}\}
    - \frac{U}{2}
    \begin{matrix}
    \begin{pmatrix}
        \Delta\rho \\ \Delta\rho u \\ \Delta\rho v \\ \Delta\rho H
    \end{pmatrix}
    \end{matrix}
    - \delta U
    \begin{matrix}
    \begin{pmatrix}
        \rho \\ \rho u \\ \rho v \\ \rho H
    \end{pmatrix}
    \end{matrix}
    - \delta p
    \begin{matrix}
    \begin{pmatrix}
        0 \\ n_x \\ n_y \\ 0
    \end{pmatrix}
    \end{matrix}
\end{equation}
The first term, the convective upwinding, is almost exactly the same as the Roe diffusion, except that the energy equation is upwinded on jumps in the total enthalpy, not total energy.
For low Mach number flows, the zeroth order pressure is a factor of $\mathcalOM{\-2}$ larger than the kinetic energy, so to leading order the total energy and total enthalpy differ by a factor of $\gamma$.
The second term, the velocity perturbation diffusion, is exactly that of the Roe diffusion, which is particularly important for preventing pressure instabilities in purely convective flows.
The third term, the pressure perturbation diffusion, naturally has the form $\delta p \underline{\Pi}$ and matches the Roe form for the momentum equations but provides no diffusion on the energy equation, unlike the Roe form.

\subsubsection{Zha-Bilgen}
For the Zha-Bilgen splitting we have:
\begin{equation}\label{eq:zb-splitting}
    \underline{\varphi} = 
    \begin{matrix}
    \begin{pmatrix}
        \rho \\ \rho u \\ \rho v \\ \rho E
    \end{pmatrix}
    \end{matrix},
    \quad
    \underline{P} = 
    p\underline{\Pi} =
    p
    \begin{matrix}
    \begin{pmatrix}
        0 \\ n_x \\ n_y \\ U
    \end{pmatrix}
    \end{matrix},
    \quad
    \delta \underline{P} =
    \begin{matrix}
    \begin{pmatrix}
        0 \\ \delta p\,n_x \\ \delta p\,n_y \\ \delta(pU)
    \end{pmatrix}
    \end{matrix},
\end{equation}
leading to the approximate form:
\begin{equation}\label{eq:zb-diffusion1}
    \underline{\hat{f}} = \{\underline{f}\}
    - \frac{U}{2}
    \begin{matrix}
    \begin{pmatrix}
        \Delta\rho \\ \Delta\rho u \\ \Delta\rho v \\ \Delta\rho E
    \end{pmatrix}
    \end{matrix}
    - \delta U
    \begin{matrix}
    \begin{pmatrix}
        \rho \\ \rho u \\ \rho v \\ \rho E
    \end{pmatrix}
    \end{matrix}
    -
    \begin{matrix}
    \begin{pmatrix}
        0 \\ \delta p\,n_x \\ \delta p\,n_y \\ \delta(pU)
    \end{pmatrix}
    \end{matrix}
\end{equation}
The convective upwinding term is exactly that of the Roe form, so can be expected to give good resolution for the convective features.
The velocity perturbation term matches the Roe term in the mass and momentum equations, but uses the total energy scale in the energy equation instead of the total enthalpy.
The third term matches the Roe form except possibly in the energy equation, which depends on the specific expression for $\delta(pU)$.
If we use the expression:
\begin{equation}\label{eq:delta-pu-simplification}
    \delta(pU) = U\delta p
\end{equation}
then we can find a first form for the diffusion in terms of $\delta U$ and $\delta p$:
\begin{equation}\label{eq:zb-diffusion2}
    \underline{\hat{f}} = \{\underline{f}\}
    - \frac{U}{2}
    \begin{matrix}
    \begin{pmatrix}
        \Delta\rho \\ \Delta\rho u \\ \Delta\rho v \\ \Delta\rho E
    \end{pmatrix}
    \end{matrix}
    - \delta U
    \begin{matrix}
    \begin{pmatrix}
        \rho \\ \rho u \\ \rho v \\ \rho E
    \end{pmatrix}
    \end{matrix}
    - \delta p
    \begin{matrix}
    \begin{pmatrix}
        0 \\ n_x \\ n_y \\ U
    \end{pmatrix}
    \end{matrix}
\end{equation}
where the pressure perturbation term now has the form $\delta p \underline{\Pi}$, and is exactly that of the Roe scheme.
A second form is found if the $\delta(pU)$ term is instead written as:
\begin{equation}\label{eq:delta-pu-simplification-2}
    \delta(pU) = p\delta U + U\delta p
\end{equation}
Then the total diffusion on the energy equation is:
\begin{equation}\label{eq:zb-diffusion3}
\begin{split}
    \underline{\hat{f}}^d_{\rho E} & = U\Delta\rho E + (\rho E + p)\delta U + U\delta p \\
                                   & = U\Delta\rho E + \rho H \delta U + U\delta p
\end{split}
\end{equation}
which matches that in the Roe-type scheme (\ref{eq:fds-interface-flux}).

\subsubsection{Toro-Vasquez}
For the Toro-Vasquez splitting we have:
\begin{equation}\label{eq:tv-splitting}
\renewcommand{\arraystretch}{1.25}
    \underline{\varphi} = 
    \begin{matrix}
    \begin{pmatrix}
        \rho \\ \rho u \\ \rho v \\ \rho k
    \end{pmatrix}
    \end{matrix},
    \quad
    \underline{P} = 
    p\underline{\Pi} =
    p
    \begin{matrix}
    \begin{pmatrix}
        0 \\ n_x \\ n_y \\ \dfrac{\gamma U}{\gamma-1}
    \end{pmatrix}
    \end{matrix},
    \quad
    \delta \underline{P} =
    \begin{matrix}
    \begin{pmatrix}
        0 \\ \delta p\,n_x \\ \delta p\,n_y \\ \dfrac{\gamma\delta(pU)}{\gamma-1}
    \end{pmatrix}
    \end{matrix},
\end{equation}
leading to the approximate form:
\begin{equation}\label{eq:tv-diffusion1}
\renewcommand{\arraystretch}{1.25}
    \underline{\hat{f}} = \{\underline{f}\}
    - \frac{U}{2}
    \begin{matrix}
    \begin{pmatrix}
        \Delta\rho \\ \Delta\rho u \\ \Delta\rho v \\ \Delta\rho k
    \end{pmatrix}
    \end{matrix}
    - \delta U
    \begin{matrix}
    \begin{pmatrix}
        \rho \\ \rho u \\ \rho v \\ \rho k
    \end{pmatrix}
    \end{matrix}
    -
    \begin{matrix}
    \begin{pmatrix}
        0 \\ \delta p\,n_x \\ \delta p\,n_y \\ \dfrac{\gamma\delta(pU)}{\gamma-1}
    \end{pmatrix}
    \end{matrix}
\end{equation}
The Toro-Vasquez diffusion has the least similarity to the Roe form.
The diffusion on the continuity and momentum equations match, but the energy equation diffusion differs in all three terms.
As stated above, the ratio of the kinetic energy to the internal energy is $\mathcalOM{2}$, so the convective upwind diffusion and velocity perturbation diffusion on the energy equation in this form of the Toro-Vasquez splitting will be significantly smaller than in the Liou-Steffen and Zha-Bilgen splittings.
If we again use the first expression for $\delta(pU)$, equation (\ref{eq:delta-pu-simplification}), then the first form of the diffusion in terms of $\delta U$ and $\delta p$ is:
\begin{equation}\label{eq:tv-diffusion2}
\renewcommand{\arraystretch}{1.25}
    \underline{\hat{f}} = \{\underline{f}\}
    - \frac{U}{2}
    \begin{matrix}
    \begin{pmatrix}
        \Delta\rho \\ \Delta\rho u \\ \Delta\rho v \\ \Delta\rho k
    \end{pmatrix}
    \end{matrix}
    - \delta U
    \begin{matrix}
    \begin{pmatrix}
        \rho \\ \rho u \\ \rho v \\ \rho k
    \end{pmatrix}
    \end{matrix}
    - \delta p
    \begin{matrix}
    \begin{pmatrix}
        0 \\ n_x \\ n_y \\ \dfrac{\gamma U}{\gamma-1}
    \end{pmatrix}
    \end{matrix}
\end{equation}
The third diffusion term now has the form $\delta p\underline{\Pi}$, and matches the Roe form, up to a factor $\frac{\gamma}{\gamma-1}$ in the energy equation.
A second form of the Toro-Vasquez diffusion can be found similarly to the second Zha-Bilgen form by using (\ref{eq:delta-pu-simplification-2}).
This results in the total diffusion on the energy equation having the form:
\begin{equation}\label{eq:tv-diffusion3}
\begin{split}
    \underline{\hat{f}}^d_{\rho E} & = U\Delta\rho k + \Big(\rho k + \frac{\gamma p}{\gamma-1}\Big)\delta U + \frac{\gamma U\delta p}{\gamma-1} \\
                                   & = U\Delta\rho k + \rho H \delta U + \frac{\gamma U\delta p}{\gamma-1}
\end{split}
\end{equation}
where the velocity perturbation now matches the Roe form, but the convective upwinding and the pressure perturbation terms remain unchanged.\\

In this section it has been shown how convection-pressure flux-vector splittings can be approximately reformulated to make the artificial diffusion terms explicit, allowing comparison with the findings of the first paper.
This reformulation requires a number of assumptions which limit the scope of its applicability.
The first two assumptions are that the convection term in (\ref{eq:general-cp-interface-flux}) can be cast as $U_{1/2}\underline{\varphi}$, and that the interface velocity and pressure flux can be cast as the sum of a central approximation and a diffusive term (\ref{eq:general-interface-up}), which hold true for many schemes in the literature.
In this study we are interested in the asymptotic scaling of the artificial diffusion coefficients $\mu_{ij}$, and not their specific form.
This means that, for the purpose of our analysis, the reformulation into (\ref{eq:general-cp-interface-flux},\ref{eq:general-interface-up}) can be approximate and these first two assumptions are not particularly limiting even when they do not hold exactly.
The next assumptions are on the form of the pressure diffusion term $\delta\underline{P}$.
Firstly, that this term has the same sparsity as the pressure flux $\underline{P}$.
While this may seem like a logical assumption, we will see later that it does not apply to all schemes in the literature.
The second assumption is the simplification $\delta(pU)=U\delta p$ (\ref{eq:delta-pu-simplification}), which allows the pressure diffusion term of the Zha-Bilgen and Toro-Vasquez splittings to be written as $\delta\underline{P}=\delta p\underline{\Pi}$, as it naturally is for the Liou-Steffen splitting and the Roe flux.
However, (\ref{eq:delta-pu-simplification}) will not necessarily hold for all schemes even if the sparsity assumption holds, in which case the second forms of the Zha-Bilgen and Toro-Vasquez splittings may be more applicable.
Lastly, we note that the only assumption which is specific to low Mach number flow is that the flow is smooth, which we used in the step from (\ref{eq:discrete-cp-diffusion}) to (\ref{eq:general-cp-diffusion}).

The assumptions on the diffusion terms enforce only light requirements, and result in diffusion forms which closely match that of the Roe scheme so allow better comparison to the findings of \cite{hope-collins_artificial_2022}.
The approximate diffusion terms of all of these forms match the Roe form on the continuity and momentum equations, although they all differ to some extent on the energy equation.
The Liou-Steffen and both of the Zha-Bilgen forms match the Roe form of the artificial diffusion exactly in at least one of the terms (all three in the case of the second Zha-Bilgen form (\ref{eq:zb-diffusion3})), and approximately match in the others.
On the other hand, the Toro-Vasquez splitting shows little resemblance to the Roe form in the energy equation, other than in the second form (\ref{eq:tv-diffusion3}) where the velocity perturbation term matches.

\subsection{Entropy variables form}
The artificial diffusion fluxes (\ref{eq:ls-diffusion}), (\ref{eq:zb-diffusion2}), and (\ref{eq:tv-diffusion2}) are now transformed to the non-dimensional entropy variables, and the resulting modified equations are compared to (\ref{eq:euler_modified}) and the findings of the previous paper.
The Roe type diffusion (\ref{eq:fds-interface-flux},\ref{eq:fds-interface-delta-up}) in the entropy variables is:
\begin{equation}\label{eq:entropy-diffusion-fds}
%\renewcommand{\arraystretch}{1.25}
%    \underline{f}^d =
    \mu_u|u|
    \begin{matrix} \begin{pmatrix}
        \phantom{\rho}\partial_{xx}p \\
        \rho\partial_{xx}u \\
        \rho\partial_{xx}v \\
        \phantom{\rho}\partial_{xx}s
        \end{pmatrix} \end{matrix}
    +
    \delta U
    \begin{matrix} \begin{pmatrix}
        \gamma p \\
        0 \\
        0 \\
        0
    \end{pmatrix} \end{matrix}
    +
    \delta p
    \begin{matrix} \begin{pmatrix}
        0 \\
        1 \\
        0 \\
        0
    \end{pmatrix} \end{matrix}
\end{equation}
which is exactly the right hand side of (\ref{eq:euler_modified}) written in terms of $\delta U$ and $\delta p$.
The diffusion terms found from the Liou-Steffen splitting (\ref{eq:ls-diffusion}) are:
\begin{equation}\label{eq:entropy-diffusion-ls}
%\renewcommand{\arraystretch}{1.25}
%    \underline{f}^d =
    \mu_u|u|
    \begin{matrix} \begin{pmatrix}
        \gamma\partial_{xx}p \\
        \rho\partial_{xx}u \\
        \rho\partial_{xx}v \\
        \partial_{xx}s + (\gamma-1)\partial_{xx}p
%       \rho\gamma R\partial_{xx}T
    \end{pmatrix} \end{matrix}
    +
    \delta U
    \begin{matrix} \begin{pmatrix}
        \gamma p \\
        0 \\
        0 \\
        0
    \end{pmatrix} \end{matrix}
    +
    \delta p
    \begin{matrix} \begin{pmatrix}
        -M^2u(\gamma-1) \\
        1 \\
        0 \\
        -M^2u(\gamma-1)
    \end{pmatrix} \end{matrix}
\end{equation}
As expected, the second term in (\ref{eq:entropy-diffusion-ls}) exactly matches that of the Roe scheme.
The convective upwinding is slightly different to the Roe form - increased by a factor of $\gamma$ on the pressure, and the upwinding on the entropy field also has a contribution from the pressure variations.
Interestingly, the upwinding terms on the entropy field are equivalent to temperature diffusion: $ds + (\gamma-1)dp = \rho R\gamma dT$.
The pressure perturbation diffusion in the third term, which should act only on the velocity field, now also has anti-diffusion terms on the pressure and entropy fields, although these terms are reduced by a factor of $M^2$.

For the first form of the Zha-Bilgen splitting (\ref{eq:zb-diffusion2}) we have:
\begin{equation}\label{eq:entropy-diffusion-zb}
%\renewcommand{\arraystretch}{1.25}
%    \underline{f}^d =
    \mu_u|u|
    \begin{matrix} \begin{pmatrix}
        \phantom{\rho}\partial_{xx}p \\
        \rho\partial_{xx}u \\
        \rho\partial_{xx}v \\
        \phantom{\rho}\partial_{xx}s
    \end{pmatrix} \end{matrix}
    +
    \delta U
    \begin{matrix} \begin{pmatrix}
        p \\
        0 \\
        0 \\
        -(\gamma-1)p
    \end{pmatrix} \end{matrix}
    +
    \delta p
    \begin{matrix} \begin{pmatrix}
        0 \\
        1 \\
        0 \\
        0
    \end{pmatrix} \end{matrix}
\end{equation}
As expected, the first and third terms in (\ref{eq:entropy-diffusion-zb}) exactly match those in the Roe scheme.
However, the $\delta U$ diffusion is a factor of $\gamma$ smaller on the pressure field, and has introduced an anti-diffusion term on the entropy field.
By inspection, the second Zha-Bilgen form (\ref{eq:zb-diffusion3}) will appear the same as the Roe form (\ref{eq:entropy-diffusion-fds}), with the correct pressure diffusion $\delta U$ term.\\
Finally, for the first form of the Toro-Vasquez splitting (\ref{eq:tv-diffusion2}) we have:
\begin{equation}\label{eq:entropy-diffusion-tv}
%\renewcommand{\arraystretch}{1.25}
%    \underline{f}^d =
    \mu_u|u|
    \begin{matrix} \begin{pmatrix}
        0 \\
        \rho\partial_{xx}u \\
        \rho\partial_{xx}v \\
        \partial_{xx}s - \partial_{xx}p
%       -a^2\partial_{xx}\rho
    \end{pmatrix} \end{matrix}
    +
    \delta U
    \begin{matrix} \begin{pmatrix}
        0 \\
        0 \\
        0 \\
        -\gamma p
    \end{pmatrix} \end{matrix}
    +
    \delta p
    \begin{matrix} \begin{pmatrix}
        M^2u \\
        1 \\
        0 \\
        M^2u
    \end{pmatrix} \end{matrix}
\end{equation}
None of the three terms in (\ref{eq:entropy-diffusion-tv}) match those in (\ref{eq:entropy-diffusion-fds}).
The upwinding and $\delta U$ term provide no diffusion on the pressure field, which only has (erroneous) diffusion from the $\delta p$ term scaled by $M^2$.
The upwinding on the entropy field has a contribution from the pressure variations, similar to the Liou-Steffen form.
This time, the upwinding on the entropy field is equivalent to density diffusion: $ds - dp = -a^2d\rho$.\footnote{The negative sign does not indicate anti-diffusion because $\frac{\partial s}{\partial\rho}<0$.}
The only effect of the $\delta U$ term is an anti-diffusion term on the entropy equation, similar to that in first form of the Zha-Bilgen splitting.
By inspection, the second Toro-Vasquez form will appear as (\ref{eq:entropy-diffusion-tv}) but with the velocity perturbation form of the Roe scheme, which removes the pressure anti-diffusion term from the entropy equation and restores it to the pressure equation.\\
%By inspection, the second Toro-Vasquez form (\ref{eq:tv-diffusion3}) will appear as:
%\begin{equation}\label{eq:entropy-diffusion-tv-2}
%%\renewcommand{\arraystretch}{1.25}
%%    \underline{f}^d =
%    \mu_u|u|
%    \begin{matrix} \begin{pmatrix}
%        0 \\
%        \rho\partial_{xx}u \\
%        \rho\partial_{xx}v \\
%        \partial_{xx}s - \partial_{xx}p
%%       -a^2\partial_{xx}\rho
%    \end{pmatrix} \end{matrix}
%    +
%    \delta U
%    \begin{matrix} \begin{pmatrix}
%        \gamma p \\
%        0 \\
%        0 \\
%        0
%    \end{pmatrix} \end{matrix}
%    +
%    \delta p
%    \begin{matrix} \begin{pmatrix}
%        M^2u \\
%        1 \\
%        0 \\
%        M^2u
%    \end{pmatrix} \end{matrix}
%\end{equation}
%Where the pressure diffusion $\delta U$ term now matches the Roe form.

Because all forms are consistent with the Roe form on the mass and momentum equations in the conserved variables, they all have the correct diffusion on the velocity and vorticity equations in the entropy variables.
However, the diffusion on the pressure and entropy fields does not fully match the Roe form for any of the flux-splitting forms, apart from the second Zha-Bilgen form.

\subsection{Limit equations of flux-vector splittings}

In this section the limiting forms of the modified equations of each diffusion scheme are found for the convective and acoustic asymptotic regimes.
The limit equations are found with the following steps:
\begin{enumerate}
    \item Replace the right hand side of the modified equations (\ref{eq:euler_modified}) with one of the diffusion schemes (\ref{eq:entropy-diffusion-fds}-\ref{eq:entropy-diffusion-tv}).
    \item Set the scaling of $\mu_{11}$, $\mu_{22}$ according to either the convective or mixed diffusion scaling in table \ref{tab:diffusion-scaling}.
    \item Enforce the scaling of the physical quantities and their gradients according to either the convective or acoustic regime scaling in table \ref{tab:lowmach_scaling}.
    \item Retain only the leading order terms in each equation.
\end{enumerate}
The limit equations provide less detail than full asymptotic expansions of the modified equations would, but are significantly more compact and still give useful insight into the asymptotic behaviour of schemes at low Mach number, as shown in \cite{hope-collins_artificial_2022}.
The velocity equations for all three flux splittings match the Roe form which means that with the acoustic diffusion scaling they will all be inaccurate for convective features.
For this reason only the limit equations of convective and mixed diffusion scalings will considered.
To reduce the size of the relations, all schemes are taken to have asymptotically diagonal diffusion Jacobians i.e. $\mu_{12}\sim\mu_{21}\sim o(M^0)$, which is almost always the case for convection-pressure flux-vector splittings at low Mach number.
%The limit equations with the off-diagonal terms included can be found in appendix \ref{app:full-limit-eqs}, and discussion of their behaviour can be found in \cite{hope-collins_artificial_2022}.

\subsubsection{Roe form}
The limit equations for the `ideal' modified equations (\ref{eq:euler_modified}) are included here for the flux-vector-splitting schemes to be compared against.
These limit equations are equivalent to those presented in section 3 of \cite{hope-collins_artificial_2022}, except with the specific form of the diffusion coefficients from the present equation (\ref{eq:euler_modified}).
Because (\ref{eq:euler_modified}) are the modified equations for Roe-type schemes in the entropy variables, we shall refer to them as the Roe form of the modified equations.
Their main features are described here; see \cite{hope-collins_artificial_2022} for more in-depth discussion.\\
The limit equations for the Roe form (\ref{eq:entropy-diffusion-fds}) using the convective diffusion scaling and enforcing convective variations from table \ref{tab:lowmach_scaling} are:
\begin{equation} \label{eq:limit-fds-Ac-convective}
    \begin{aligned}
        \partial_t\ord{p}{0} + \gamma\ord{p}{0}\partial_x\ord{u}{0} & = M^{\-2}\frac{\gamma\ord{p}{0}}{\ord{\rho}{0}|v|}\mu_{11}\partial_{xx}\ord{p}{2} \\
        \ord{\rho}{0}\partial_t\ord{u}{0} + \partial_x\ord{p}{2} + \ord{\rho u}{0}\partial_x\ord{u}{0} & = \ord{\rho}{0}\mu_u|u|\partial_{xx}\ord{u}{0} + \ord{\rho}{0}|v|\mu_{22}\partial_{xx}\ord{u}{0} \\
        \partial_t\ord{v}{0}                           +      \ord{u}{0}\partial_x\ord{v}{0} & = \mu_u|u|\partial_{xx}\ord{v}{0} \\
        \partial_t\ord{s}{0}                           +      \ord{u}{0}\partial_x\ord{s}{0} & = \mu_u|u|\partial_{xx}\ord{s}{0}
    \end{aligned}
\end{equation}
The limit equations for the Roe form (\ref{eq:entropy-diffusion-fds}) using the convective diffusion scaling and enforcing acoustic variations from table \ref{tab:lowmach_scaling} are:
\begin{equation} \label{eq:limit-fds-Ac-acoustic}
    \begin{aligned}
        0 & = M^{\-2}\frac{\gamma\ord{p}{0}}{\ord{\rho}{0}|v|}\mu_{11}\partial_{xx}\ord{p}{1} \\
        \ord{\rho}{0}\partial_{\tau}\ord{u}{0} + \partial_x\ord{p}{1} & = 0 \\
        \partial_{\tau}\ord{v}{0} & = 0 \\
        \partial_{\tau}\ord{s}{0} & = 0
    \end{aligned}
\end{equation}
The left-hand-side of the convective limit equations match those of the single scale convective asymptotic expansion (\ref{eq:convective_timescale}) by construction.
The diffusion on the velocity equation is well balanced, and the vorticity and entropy equations appear as scalar advection-diffusion equations with upwind diffusion.
Note that at steady state the divergence does not disappear, but is equal to the pressure diffusion term.
As mentioned previously, this term damps pressure oscillations from the solution, preventing chequerboard modes.
In the acoustic limit, this term overwhelms the physical terms in the pressure equation, causing it to become a Poisson-like relation for the acoustic pressure $\ord{p}{1}$ which quickly damps out any acoustic waves from the solution.

The limit equations for the Roe form (\ref{eq:entropy-diffusion-fds}) using the mixed diffusion scaling and enforcing convective variations are:
\begin{equation} \label{eq:limit-fds-Am-convective}
    \begin{aligned}
        \partial_t\ord{p}{0} + \gamma\ord{p}{0}\partial_x\ord{u}{0} & = 0 \\
        \ord{\rho}{0}\partial_t\ord{u}{0} + \partial_x\ord{p}{2} + \ord{\rho u}{0}\partial_x\ord{u}{0} & = \ord{\rho}{0}\mu_u|u|\partial_{xx}\ord{u}{0} + \ord{\rho}{0}|v|\mu_{22}\partial_{xx}\ord{u}{0} \\
        \partial_t\ord{v}{0}                           +      \ord{u}{0}\partial_x\ord{v}{0} & = \mu_u|u|\partial_{xx}\ord{v}{0} \\
        \partial_t\ord{s}{0}                           +      \ord{u}{0}\partial_x\ord{s}{0} & = \mu_u|u|\partial_{xx}\ord{s}{0}
    \end{aligned}
\end{equation}
The limit equations for the Roe form (\ref{eq:entropy-diffusion-fds}) using the mixed diffusion scaling and enforcing acoustic variations are:
\begin{equation} \label{eq:limit-fds-Am-acoustic}
    \begin{aligned}
        \partial_{\tau}\ord{p}{1} + \gamma\ord{p}{0}\partial_x\ord{u}{0} & = M^{\-2}\frac{\gamma\ord{p}{0}}{\ord{\rho}{0}|v|}\mu_{11}\partial_{xx}\ord{p}{1} \\
        \ord{\rho}{0}\partial_{\tau}\ord{u}{0} + \partial_x\ord{p}{1} & = 0 \\
        \partial_{\tau}\ord{v}{0} & = 0 \\
        \partial_{\tau}\ord{s}{0} & = 0
    \end{aligned}
\end{equation}
Because the only difference between the convective and mixed diffusion scalings is in the pressure diffusion term, the limit equations for velocity, vorticity, and entropy with the mixed diffusion scaling are identical to those with the convective diffusion scaling, with only the pressure equation differing.
At the convective limit the pressure diffusion disappears from the continuity equation.
This allows for divergence-free solutions, consistent with the incompressible limit but, as stated previously, makes the scheme susceptible to pressure chequerboard instabilities.
At the acoustic limit, the pressure diffusion no longer overwhelms the physical terms.
The left-hand-side of the pressure and velocity equations are the equations for low Mach number acoustics.
The diffusion on the velocity vanishes asymptotically, making the mixed scaling susceptible to velocity instabilities at the acoustic limit.
The leading order entropy does not vary on the acoustic timescale, consistent with the approximation of isentropic acoustics.

Because the diffusion on the velocity and vorticity equations match the Roe scheme for all three flux-vector splittings, these limit equations will match those of the Roe form.
In the following sections, the limit equations for the vorticity will be omitted, but the limit equations for the velocity will be retained so that the pressure-velocity subsystem can be understood as a whole.

\subsubsection{Liou-Steffen splitting}

The limit equations for the Liou-Steffen form (\ref{eq:entropy-diffusion-ls}) using the convective diffusion scaling and enforcing convective variations are:
\begin{equation} \label{eq:limit-ls-Ac-convective}
    \begin{aligned}
        \partial_t\ord{p}{0} + \gamma\ord{p}{0}\partial_x\ord{u}{0} & = M^{\-2}\frac{\gamma\ord{p}{0}}{\ord{\rho}{0}|v|}\mu_{11}\partial_{xx}\ord{p}{2} \\
        \ord{\rho}{0}\partial_t\ord{u}{0} + \partial_x\ord{p}{2} + \ord{\rho u}{0}\partial_x\ord{u}{0} & = \ord{\rho}{0}\mu_u|u|\partial_{xx}\ord{u}{0} + \ord{\rho}{0}|v|\mu_{22}\partial_{xx}\ord{u}{0} \\
        \partial_t\ord{s}{0}                           +      \ord{u}{0}\partial_x\ord{s}{0} & = \mu_u|u|\partial_{xx}\ord{s}{0}
    \end{aligned}
\end{equation}
The limit equations for the Liou-Steffen form (\ref{eq:entropy-diffusion-ls}) using the convective diffusion scaling and enforcing acoustic variations are:
\begin{equation} \label{eq:limit-ls-Ac-acoustic}
    \begin{aligned}
        0 & = M^{\-2}\frac{\gamma\ord{p}{0}}{\ord{\rho}{0}|v|}\mu_{11}\partial_{xx}\ord{p}{1} \\
        \ord{\rho}{0}\partial_{\tau}\ord{u}{0} + \partial_x\ord{p}{1} & = 0 \\
        \partial_{\tau}\ord{s}{0} & = 0
    \end{aligned}
\end{equation}
The limit equations for the Liou-Steffen form (\ref{eq:entropy-diffusion-ls}) using the mixed diffusion scaling and enforcing convective variations are:
\begin{equation} \label{eq:limit-ls-Am-convective}
    \begin{aligned}
        \partial_t\ord{p}{0} + \gamma\ord{p}{0}\partial_x\ord{u}{0} & = 0 \\
        \ord{\rho}{0}\partial_t\ord{u}{0} + \partial_x\ord{p}{2} + \ord{\rho u}{0}\partial_x\ord{u}{0} & = \ord{\rho}{0}\mu_u|u|\partial_{xx}\ord{u}{0} + \ord{\rho}{0}|v|\mu_{22}\partial_{xx}\ord{u}{0} \\
        \partial_t\ord{s}{0}                           +      \ord{u}{0}\partial_x\ord{s}{0} & = \mu_u|u|\partial_{xx}\ord{s}{0}
    \end{aligned}
\end{equation}
The limit equations for the Liou-Steffen form (\ref{eq:entropy-diffusion-ls}) using the mixed diffusion scaling and enforcing acoustic variations are:
\begin{equation} \label{eq:limit-ls-Am-acoustic}
    \begin{aligned}
        \partial_{\tau}\ord{p}{1} + \gamma\ord{p}{0}\partial_x\ord{u}{0} & = M^{\-2}\frac{\gamma\ord{p}{0}}{\ord{\rho}{0}|v|}\mu_{11}\partial_{xx}\ord{p}{1} \\
        \ord{\rho}{0}\partial_{\tau}\ord{u}{0} + \partial_x\ord{p}{1} & = 0 \\
        \partial_{\tau}\ord{s}{0} & = 0
    \end{aligned}
\end{equation}
The erroneous diffusion from the $\delta p$ terms vanishes asymptotically from the pressure and entropy equations for both the convective and mixed diffusion scalings at both the convective and acoustic limits.
The pressure diffusion in the entropy equation from the convective upwinding term also vanishes at both limits.
As a result, the limit equations for the Liou-Steffen form exactly match those of the Roe form.
This should not be surprising given the success of low Mach number schemes based on this splitting with both convective and mixed diffusion scalings.

\subsubsection{Zha-Bilgen splitting}

The limit equations for the first Zha-Bilgen form (\ref{eq:entropy-diffusion-zb}) using the convective diffusion scaling and enforcing convective variations are:
\begin{equation} \label{eq:limit-zb-Ac-convective}
    \begin{aligned}
        \partial_t\ord{p}{0} + \gamma\ord{p}{0}\partial_x\ord{u}{0} & = M^{\-2}\frac{\ord{p}{0}}{\ord{\rho}{0}|v|}\mu_{11}\partial_{xx}\ord{p}{2} \\
        \ord{\rho}{0}\partial_t\ord{u}{0} + \partial_x\ord{p}{2} + \ord{\rho u}{0}\partial_x\ord{u}{0} & = \ord{\rho}{0}\mu_u|u|\partial_{xx}\ord{u}{0} + \ord{\rho}{0}|v|\mu_{22}\partial_{xx}\ord{u}{0} \\
        \partial_t\ord{s}{0}                           +      \ord{u}{0}\partial_x\ord{s}{0} & = \mu_u|u|\partial_{xx}\ord{s}{0} - M^{\-2}(\gamma-1)\frac{\ord{p}{0}}{\ord{\rho}{0}|v|}\mu_{11}\partial_{xx}\ord{p}{2} \\
    \end{aligned}
\end{equation}
The limit equations for the Zha-Bilgen form (\ref{eq:entropy-diffusion-zb}) using the convective diffusion scaling and enforcing acoustic variations are:
\begin{equation} \label{eq:limit-zb-Ac-acoustic}
    \begin{aligned}
    0 & = M^{\-2}\frac{\ord{p}{0}}{\ord{\rho}{0}|v|}\mu_{11}\partial_{xx}\ord{p}{1} \\
        \ord{\rho}{0}\partial_{\tau}\ord{u}{0} + \partial_x\ord{p}{1} & = 0 \\
        \partial_{\tau}\ord{s}{0} & = -M^{\-2}(\gamma-1)\frac{\ord{p}{0}}{\ord{\rho}{0}|v|}\mu_{11}\partial_{xx}\ord{p}{1}
    \end{aligned}
\end{equation}
The pressure diffusion is a factor of $\gamma$ smaller than for the Roe and Liou-Steffen forms, although if $\gamma\sim\mathcal{O}(1)$ then this should not have a significant effect, and can easily be compensated for in the definition of $\mu_{11}$.
Other than this difference, the pressure, velocity, and vorticity equations match the Roe and Liou-Steffen forms, so can be expected to perform similarly.
On the other hand, the entropy equation retains the pressure anti-diffusion term in both limits, which could cause erroneous entropy generation from leading order pressure variations.\\
The limit equations for the Zha-Bilgen form (\ref{eq:entropy-diffusion-zb}) using the mixed diffusion scaling and enforcing convective variations are:
\begin{equation} \label{eq:limit-zb-Am-convective}
    \begin{aligned}
        \partial_t\ord{p}{0} + \gamma\ord{p}{0}\partial_x\ord{u}{0} & = 0 \\
        \ord{\rho}{0}\partial_t\ord{u}{0} + \partial_x\ord{p}{2} + \ord{\rho u}{0}\partial_x\ord{u}{0} & = \ord{\rho}{0}\mu_u|u|\partial_{xx}\ord{u}{0} + \ord{\rho}{0}|v|\mu_{22}\partial_{xx}\ord{u}{0} \\
        \partial_t\ord{s}{0}                           +      \ord{u}{0}\partial_x\ord{s}{0} & = \mu_u|u|\partial_{xx}\ord{s}{0} \\
    \end{aligned}
\end{equation}
The limit equations for the Zha-Bilgen form (\ref{eq:entropy-diffusion-zb}) using the mixed diffusion scaling and enforcing acoustic variations are:
\begin{equation} \label{eq:limit-zb-Am-acoustic}
    \begin{aligned}
        \partial_{\tau}\ord{p}{1} + \gamma\ord{p}{0}\partial_x\ord{u}{0} & = M^{\-2}\frac{\ord{p}{0}}{\ord{\rho}{0}|v|}\mu_{11}\partial_{xx}\ord{p}{1} \\
        \ord{\rho}{0}\partial_{\tau}\ord{u}{0} + \partial_x\ord{p}{1} & = 0 \\
        \partial_{\tau}\ord{s}{0} & = 0
    \end{aligned}
\end{equation}
With the mixed diffusion scaling the pressure anti-diffusion term in the entropy equation vanishes asymptotically at both limits, so the Zha-Bilgen form matches the Roe form at the convective limit, and the only difference at the acoustic limit is the factor of $\gamma$ in the pressure equation diffusion.

The limit equations for the second Zha-Bilgen form (\ref{eq:zb-diffusion3}) will be identical to the limit equations for the Roe form, which will remove the erroneous pressure anti-diffusion term from the entropy equation at the convective limit.

\subsubsection{Toro-Vasquez splitting}

The limit equations for the first Toro-Vasquez form (\ref{eq:entropy-diffusion-tv}) using the convective diffusion scaling and enforcing convective variations are:
\begin{equation} \label{eq:limit-tv-Ac-convective}
    \begin{aligned}
        \partial_t\ord{p}{0} + \gamma\ord{p}{0}\partial_x\ord{u}{0} & = 0 \\
        \ord{\rho}{0}\partial_t\ord{u}{0} + \partial_x\ord{p}{2} + \ord{\rho u}{0}\partial_x\ord{u}{0} & = \ord{\rho}{0}\mu_u|u|\partial_{xx}\ord{u}{0} + \ord{\rho}{0}|v|\mu_{22}\partial_{xx}\ord{u}{0} \\
        \partial_t\ord{s}{0}                           +      \ord{u}{0}\partial_x\ord{s}{0} & = \mu_u|u|\partial_{xx}\ord{s}{0} - M^{\-2}\frac{\gamma\ord{p}{0}}{\ord{\rho}{0}|v|}\mu_{11}\partial_{xx}\ord{p}{2}
    \end{aligned}
\end{equation}
The limit equations for the Toro-Vasquez form (\ref{eq:entropy-diffusion-tv}) using the convective diffusion scaling and enforcing acoustic variations are:
\begin{equation} \label{eq:limit-tv-Ac-acoustic}
    \begin{aligned}
        \partial_{\tau}\ord{p}{1} + \gamma\ord{p}{0}\partial_x\ord{u}{0} & = 0 \\
        \ord{\rho}{0}\partial_{\tau}\ord{u}{0} + \partial_x\ord{p}{1} & = 0 \\
        \partial_{\tau}\ord{s}{0} & =  -M^{\-2}\frac{\gamma\ord{p}{0}}{\ord{\rho}{0}|v|}\mu_{11}\partial_{xx}\ord{p}{1}
    \end{aligned}
\end{equation}
The lack of pressure diffusion on the pressure equation means that, even with the convective diffusion scaling, the divergence relation is undamped and therefore susceptible to pressure chequerboard instabilities.
At the acoustic limit, the absence of this pressure diffusion means that the pressure equation does not become a Poisson-like relation for the acoustic pressure $\ord{p}{1}$ which would damp out any acoustic waves.
The erroneous $\delta p$ terms in the pressure and entropy equations vanish asymptotically at both limits, as does the pressure diffusion in the entropy equation from the convective upwinding term, leaving the correct entropy upwinding.
However, the pressure anti-diffusion term from $\delta U$ remains on the entropy equation, as for the Zha-Bilgen form.\\

The limit equations for the Toro-Vasquez form (\ref{eq:entropy-diffusion-tv}) using the mixed diffusion scaling and enforcing convective variations are:
\begin{equation} \label{eq:limit-tv-Am-convective}
    \begin{aligned}
        \partial_t\ord{p}{0} + \gamma\ord{p}{0}\partial_x\ord{u}{0} & = 0 \\
        \ord{\rho}{0}\partial_t\ord{u}{0} + \partial_x\ord{p}{2} + \ord{\rho u}{0}\partial_x\ord{u}{0} & = \ord{\rho}{0}\mu_u|u|\partial_{xx}\ord{u}{0} + \ord{\rho}{0}|v|\mu_{22}\partial_{xx}\ord{u}{0} \\
        \partial_t\ord{s}{0}                           +      \ord{u}{0}\partial_x\ord{s}{0} & = \mu_u|u|\partial_{xx}\ord{s}{0}
    \end{aligned}
\end{equation}
The limit equations for the Toro-Vasquez form (\ref{eq:entropy-diffusion-tv}) using the mixed diffusion scaling and enforcing acoustic variations are:
\begin{equation} \label{eq:limit-tv-Am-acoustic}
    \begin{aligned}
        \partial_{\tau}\ord{p}{1} + \gamma\ord{p}{0}\partial_x\ord{u}{0} & = 0 \\
        \ord{\rho}{0}\partial_{\tau}\ord{u}{0} + \partial_x\ord{p}{1} & = 0 \\
        \partial_{\tau}\ord{s}{0} & = 0
    \end{aligned}
\end{equation}
With the mixed diffusion scaling, the limit equations at the convective limit match the Roe form, however at the acoustic limit all diffusion terms vanish asymptotically, which leaves the scheme completely undamped.
It should be noted that the pressure diffusion in the pressure equation of the Roe form (\ref{eq:entropy-diffusion-fds}) with the mixed diffusion scaling still stabilises the zeroth and first order pressures $\ord{p}{0,1}$ even if it disappears from the limit equations at the convective limit (\ref{eq:limit-fds-Am-convective}).
This can be seen from an asymptotic expansion of the modified or discrete equations, even if it is not evident from the limit equations \cite{hope-collins_artificial_2022}.
The pressure diffusion in the pressure equation is completely absent from the first Toro-Vasquez form, which means that even though the limit equations (\ref{eq:limit-tv-Am-convective}) match those of the Roe scheme, it is potentially susceptible to instabilities which do not occur for the Roe scheme.

The limit equations for the second Toro-Vasquez form (\ref{eq:tv-diffusion3}) are, like those for the second Zha-Bilgen form, identical to the limit equations for the Roe form.
This not only removes the erroneous pressure anti-diffusion term from the entropy limit equations, it also restores the pressure diffusion terms to the pressure limit equations, which will stabilise the convective diffusion scaling at the convective limit, the mixed diffusion scaling at the acoustic limit, and will provide the proper damping on the acoustic variations for the convective diffusion scaling at the acoustic limit.

One last point to mention is that that erroneous $\delta p$ terms in pressure and entropy equations of the Liou-Steffen and Toro-Vasquez forms (\ref{eq:entropy-diffusion-ls}) and (\ref{eq:entropy-diffusion-tv}) vanish by a factor of $M^2$ from the limit equations, so these terms are unlikely to have any visible effect on the results found with these schemes.
On the other hand, the erroneous $\delta U$ terms in the entropy equations of the first Zha-Bilgen and Toro-Vasquez forms (\ref{eq:entropy-diffusion-zb}) and (\ref{eq:entropy-diffusion-tv}) only vanish by a factor of $M$ from the limit equations for the mixed diffusion scaling (\ref{eq:limit-zb-Am-convective},\ref{eq:limit-zb-Am-acoustic}) and (\ref{eq:limit-tv-Am-convective},\ref{eq:limit-tv-Am-acoustic}).
For moderately low Mach number, the pressure anti-diffusion terms in the entropy equations of the first Zha-Bilgen and Toro-Vasquez forms with mixed diffusion scaling may leave small but noticeable residues.\\

In this section we have shown that transforming to the entropy variables and examining the limiting forms of the modified equations provides additional insights into the structure of flux-vector-splittings at low Mach number.
The limit equations for the Liou-Steffen splitting exactly match those of the Roe form, despite only matching one of three diffusion terms exactly in the conserved variables.
The Zha-Bilgen splitting matches the Roe form on the pressure, velocity, and vorticity equations, up to a factor of $\gamma$ on the pressure diffusion.
However, with the convective diffusion scaling, a pressure anti-diffusion term remains in the entropy equations, which may lead to spurious entropy generation.
The Toro-Vasquez splitting has no diffusion on the pressure equation, which may lead to instabilities for both the convective and acoustic limits for any diffusion scaling.
Like for the Zha-Bilgen splitting, the Toro-Vasquez splitting with the convective diffusion scaling has a pressure anti-diffusion term on the entropy equation at both the convective and acoustic limits.
The second forms of the Zha-Bilgen and Toro-Vasquez splittings match the limit equations of the Roe scheme.