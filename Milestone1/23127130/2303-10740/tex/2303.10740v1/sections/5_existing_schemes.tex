
\begin{table}
    \centering
    \begin{tabularx}{\textwidth}{|p{0.25\textwidth}|c|c|c|X|} \hline
         \multirow{2}{*}{Scheme} & \multirow{2}{*}{Splitting} & \multicolumn{2}{c|}{Scaling}                                 & \multirow{2}{*}{Comments} \\ \cline{3-4}
                                 &                            & $\sigma_u\sim\mathcal{O}(1)$  & $\sigma_a\sim\mathcal{O}(1)$ &                           \\ \hline
         Zha-Bilgen 1993 \cite{zha_numerical_1993} & Zha-Bilgen & \multicolumn{2}{c|}{Acoustic}                      & Original Zha-Bilgen transonic scheme. \\ \hline
         AUSM/AUSM$^+$ 1993/1996 \cite{liou_new_1993,liou_sequel_1996} & Liou-Steffen & \multicolumn{2}{c|}{Acoustic}   & Original Liou-Steffen transonic scheme. \\ \hline
         LDFSS 1998/1999/2001 \cite{edwards_low-diffusion_1998,liou_numerical_1999,edwards_towards_2001} & Liou-Steffen & \multicolumn{2}{c|}{Convective} & AUSM with elements of flux-preconditioning. \\ \hline
         AUSM$^+$-up 2006 \cite{liou_sequel_2006} & Liou-Steffen & \multicolumn{2}{c|}{Convective}                   & Discrete asymptotic expansion to find scaling of diffusion coefficients. \\ \hline
         SLAU 2009/11 \cite{shima_parameter-free_2011} & Liou-Steffen & \multicolumn{2}{c|}{Mixed}                   & Numerically shows acoustic capability. \\ \hline
         Toro-Vasquez 2012 \cite{toro_flux_2012} & Toro-Vasquez & \multicolumn{2}{c|}{Acoustic}                      & Original Toro-Vasquez transonic scheme. \\ \hline
         Li \& Gu 2013 \cite{li_mechanism_2013} & Liou-Steffen & \multicolumn{2}{c|}{Mixed}                          & Modified AUSM$^+$-up to mixed scheme according to their guidelines. \\ \hline
         Sachdev et al. 2012 \cite{sachdev_improved_2012} & Liou-Steffen & Convective                    & Mixed     & Adaptive schemes for AUSM$^+$-up and SLAU. \\ \hline
         SLAU-WS 2013 \cite{shima_improvement_2013} & Liou-Steffen & \multicolumn{2}{c|}{Mixed/convective}                   & Modifies the pressure diffusion in SLAU to increase by a factor of $M$ in the presence of pressure `wiggles' independently of the timestep. \\ \hline
         TV-MAS 2017 \cite{sun_robust_2017} & Toro-Vasquez & \multicolumn{2}{c|}{Mixed}        & $U_{1/2}$ only adds upwind diffusion. $\underline{P}_{1/2}$ adds Zha-Bilgen $\delta U$ diffusion to the density and energy equations only, and adds Toro-Vasquez $\delta p$ diffusion to the momentum equations with mixed scaling and to the energy equation with the acoustic scaling. \\ \hline
         TV-MAS2 2018 \cite{lin_density_2018} & Toro-Vasquez & \multicolumn{2}{c|}{Mixed}        & Reduces $\delta p$ term in energy equation of TVMAS to the mixed scaling. \\ \hline
         E-AUSMPWAS 2018 \cite{qu_new_2018} & Zha-Bilgen & \multicolumn{2}{c|}{Mixed}                                & Uses $\delta(pU)=p\delta U$, resulting in second Zha-Bilgen form of $\delta U$ and Liou-Steffen form of $\delta p$. \\ \hline
         AUPM 2018 \cite{chen_novel_2018} & Zha-Bilgen & \multicolumn{2}{c|}{Mixed}                                  & Second Zha-Bilgen form. \\ \hline
         ZB-FVS-Corr 2020 \cite{iampietro_low-diffusion_2020} & Zha-Bilgen & \multicolumn{2}{c|}{Mixed}              & Second Zha-Bilgen form. \\ \hline
         AUSM-M 2020 \cite{chen_improved_2020} & Liou-Steffen & \multicolumn{2}{c|}{Mixed}                           & Reduces $\mu_{22}$ diffusion through shear layer. \\ \hline
         TVAP 2021 \cite{chen_low-diffusion_2021} & Toro-Vasquez & \multicolumn{2}{c|}{Mixed}                        & $U_{1/2}$ only adds upwind diffusion. $\underline{P}_{1/2}$ adds Liou-Steffen $\delta U$ diffusion to the density and energy equations only, and adds Liou-Steffen $\delta p$ diffusion to momentum equations.  \\ \hline
    \end{tabularx}
    \caption{Diffusion scaling of a number of existing low Mach number convection-pressure flux-vector splitting schemes. Some schemes have different scalings when the timestep is calculated using the convective CFL ($\sigma_u$) and $M_u\sim\mathcal{O}(M)$, or using the acoustic CFL ($\sigma_a$) and $M_u\sim\mathcal{O}(1)$.}
    \label{tab:existing-schemes}
\end{table}

In this section we compare the current analysis to previous studies and schemes, and identify the diffusion scaling of a number of existing schemes from the literature, which are shown in table \ref{tab:existing-schemes} along with the diffusion scaling they use.
Almost all of the studies on convection-pressure flux-vector splittings at low Mach number have focused on the Liou-Steffen splitting, which reflects its popularity over the last 30 years, and all of these studies were carried out in the conserved variables.\footnote{Liou 2017 \cite{liou_root_2017} swapped the energy equation for the entropy equation in a number of schemes including AUSM in the context of the overheating problem in high speed receding flow.}\\

Liou \& Edwards \cite{edwards_low-diffusion_1998,liou_ausm_1999,liou_numerical_1999,edwards_towards_2001} discuss the use of ideas from low Mach number flux preconditioning for improving the performance of Liou-Steffen splitting schemes at low Mach number, showing that the diffusion coefficients should become independent of the Mach number as the incompressible regime is approached.
This is equivalent to the convective diffusion scaling in table \ref{tab:diffusion-scaling}, which flux-difference splitting schemes using flux preconditioning also approach \cite{turkel_preconditioning_1999,guillard_behaviour_1999}.
An overview of these developments, and of the AUSM family of schemes more generally, is given by Edwards in \cite{edwards_towards_2001,edwards_reflections_2019}.
In the development of AUSM$^+$-up for all speeds \cite{liou_sequel_2006} Liou goes into detail - including discrete asymptotic analysis - on the design of an AUSM scheme for low Mach number independently from the preconditioned system (although inspired by the earlier work mentioned), and also arrives at the convective diffusion scaling.

Dellacherie 2010 and Dellacherie et al 2016 \cite{dellacherie_analysis_2010,dellacherie_construction_2016} analysed the requirements on Godunov schemes to accurately resolve convective flow features at low Mach number, showing that the reduction from $\mu_{22}\sim\mathcalOM{\-1}$ for the acoustic diffusion scaling to $\mu_{22}\sim\mathcalOM{0}$ for the mixed or convective diffusion scalings is essential.
It is also shown that AUSM/AUSM$^+$ \cite{liou_new_1993,liou_sequel_1996} are unsuitable for low Mach number flow, having the acoustic diffusion scaling on the pressure perturbation term, whereas AUSM$^+$-up has the correct scaling on $\mu_{22}$ so is suitable for low Mach number flow as designed.
By studying existing low Mach number Roe-type schemes, Li \& Gu 2013 \cite{li_mechanism_2013} draw out a set of guidelines for the design of such schemes.
By applying these guidelines to the AUSM$^+$-up scheme, they modified the velocity perturbation to change the diffusion scaling from convective to mixed.
More detail on how the approach used in the current paper and \cite{hope-collins_artificial_2022} relates to the works of Dellacherie and Li \& Gu can be found in \cite{hope-collins_artificial_2022}.

The works of Venkateswaran and co-workers \cite{venkateswaran_efficiency_2000,venkateswaran_artficial_2003,potsdam_unsteady_2007,sachdev_improved_2012} were, to the authors' knowledge, the earliest examples of the mixed diffusion scaling in the literature, and some of the only studies to explicitly identify the mixed diffusion scaling as a combination of the convective and acoustic diffusion scalings which is suitable for both limits.
In one of their more recent studies, Sachdev et al. \cite{sachdev_improved_2012} show the two instabilities that the mixed diffusion scaling is susceptible to: pressure chequerboards at the convective limit and a velocity grid mode at the acoustic limit.
They show that AUSM$^+$-up and SLAU \cite{shima_parameter-free_2011} have the convective and mixed diffusion scalings respectively, with the expected behaviour, and go on to show how they can both be modified into adaptive schemes i.e. approaching the convective diffusion scaling as $\Delta t\to\infty$ and approaching the mixed diffusion scaling as $\Delta t\to0$.
This adaptive behaviour is achieved using an ``unsteady Mach number'' $M_{u}=L_{\infty}/(\Delta t a_{\infty})$.
Shima 2013 \cite{shima_improvement_2013} takes a different approach to resolving this problem, damping out pressure instabilities by switching the diffusion scaling near pressure ``wiggles'' i.e. locations where the ratio of fourth to second derivatives of the pressure field is large.\\

From table \ref{tab:existing-schemes} it is clear that the Liou-Steffen splitting is the most commonly used, although the Zha-Bilgen and Toro-Vasquez splittings have recently gained more attention for low Mach number flow.
Just as for Roe-type schemes, the early Liou-Steffen schemes were rooted in the low Mach number preconditioning to obtain the correct scaling, which results in the convective diffusion scaling.
The almost total switch to the mixed diffusion scaling also happened around the same time as for Roe-type schemes \textit{c.}2008 - Postdam et al. 2007 \cite{potsdam_unsteady_2007}, Thornber et al. 2008 \cite{thornber_improved_2008,thornber_numerical_2008} and Li \& Gu 2008 \cite{li_all-speed_2008} for Roe schemes, and Shima \& Kitamura 2009/11 (SLAU) \cite{shima_new_2009,shima_parameter-free_2011} for Liou-Steffen schemes.
This means that, to the authors' knowledge, no Zha-Bilgen or Toro-Vasquez schemes using the convective diffusion scaling exist in the literature.
Of these earlier Roe-type fluxes with the mixed diffusion scaling, only Potsdam et al. had the explicit aim of creating a scheme capable of resolving both convective and acoustic low Mach number features, whereas aeroacoustic simulations were one of the original target applications for SLAU.\\

%Several studies implement spatial schemes with the mixed diffusion scaling within an implicit timestepping strategy \cite{potsdam_unsteady_2007,shima_cfd_2011,sachdev_improved_2012,shima_new_2013}.
%The mixed diffusion scaling asymptotically loses diagonal dominance for the pressure field at the convective limit and the velocity field at the acoustic limit, and if the diffusion matrix is diagonal (as is the case for almost all mixed schemes in the literature) then it is asymptotically rank deficient on these fields.
%Many traditional iterative methods for the linear equations arising from an implicit scheme require diagonal dominance of the left-hand-side matrix, so in \cite{potsdam_unsteady_2007,sachdev_improved_2012} the left-hand-side is constructed using the convective or acoustic diffusion scaling (whichever is most appropriate for the flow being solved), and in \cite{shima_new_2013} a cut-off Mach number is used to prevent the diffusion terms from becoming too small.
%While this restores diagonal dominance to the left-hand-side, the inconsistency between the left-hand-side and right-hand-side introduces additional errors into the iteration procedure and degrades the convergence of the Newton solve.
%A second approach is shown in \cite{shima_new_2013} to circumvent this difficulty by using a Newton-Krylov FGMRES strategy which does not require diagonal dominance, so a consistent left-hand-side can be used.\\

The AUSM$^+$-up scheme is one of the most popular convection-pressure flux-vector splitting schemes for low Mach number.
As stated earlier the diffusion coefficients become independent of the Mach number as $M\to0$, which means that it has the convective diffusion scaling.
As such, it can accurately resolve convective flow features in low Mach number flow \cite{liou_sequel_2006,dellacherie_analysis_2010} and is robust against the pressure chequerboard instabilities that can occur for collocated schemes for low Mach number or incompressible flow.
On the other hand, it damps acoustic variations very quickly \cite{moguen_pressurevelocity_2012,moguen_pressurevelocity_2013}, and suffers from a spectral radius which scales as $\mathcalOM{\-2}$ instead of the $\mathcalOM{\-1}$ scaling of the mixed and acoustic diffusion scalings and the Jacobian of the physical flux vector \cite{hope-collins_artificial_2022}.
The $\mathcalOM{\-2}$ spectral radius scaling of the convective diffusion scaling has long been known in the context of Roe-type schemes since it was first proven by Birken \& Meister 2005 \cite{birken_stability_2005}.
This spectral radius means that the stability bound on the timestep for an explicit scheme reduces prohibitively as $\Delta t\sim\mathcalOM{2}$, a factor of $M$ faster than usual.

Birken \& Meister noted that other schemes with the convective diffusion scaling would also suffer from these problems, and gave the specific example of the AUSMDV scheme \cite{wada_flux_1994}.
That these findings also apply to AUSM$^+$-up appears to have gone unrecognised (or at least unmentioned) in the more recent literature, although evidence of it can be seen in a number of studies.
In the original paper \cite{liou_sequel_2006} it can be seen from figure 22 that the convergence of the implicit scheme stalls after at least four orders of magnitude decrease in the residual, whereas convergence to machine zero is achieved when low Mach number preconditioning is used.
This is consistent with the expectation that the spectral radius and, more importantly, the condition number of the system is reduced from $\mathcalOM{\-2}$ to $\mathcalOM{0}$ by the preconditioning.
Several papers have reported stability issues with AUSM$^+$-up at very low Mach number when using an explicit timestepping strategy \cite{matsuyama_performance_2014,kitamura_reduced_2016,chen_improved_2018}.
The response to these issues has been to prevent the Mach number used in the calculation of the artificial diffusion falling below some cut-off Mach number $M_{co}>M_{\infty}$.
Keeping $M_{co}$ above the actual Mach number of the flow has two effects.
Firstly, it reduces the $\mu_{11}$ pressure diffusion, which reduces the severe scaling of the spectral radius and alleviates the stability restrictions as intended.
Secondly, it increases the $\mu_{22}$ velocity diffusion, which results in a more diffusive solution approaching that found with the acoustic diffusion scaling.
Li \& Gu \cite{li_mechanism_2013} mention that the AUSM$^+$-up scheme suffers from the cut-off Mach number problem just as the preconditioned Roe scheme does, but do not explicitly connect this back to the spectral radius scaling and the results of Birken \& Meister.
More recently, Chen et al. 2018 \cite{chen_improved_2018} showed empirically that the scaling of the maximum stable timestep of AUSM$^+$-up is what would be expected of the convective diffusion scaling,\footnote{This can be seen from table 1 and figure 5 in \cite{chen_improved_2018} which show linear scaling of the maximum timestep with the Mach number. This scaling is found by holding the speed of sound constant and varying the convective velocity to control the Mach number, so $\Delta t\sim\mathcalOM{}$ is equivalent to $\Delta t\sim\mathcalOM{2}$ when holding the convective velocity constant and varying the speed of sound.

Incidentally, the authors also empirically found this timestep scaling for AUSM$^+$-up, independently of \cite{chen_improved_2018}. Investigating if and how it was connected to the results of Birken \& Meister was one of the original motivations for \cite{hope-collins_artificial_2022} and the current study.} but again do not remark on the connection to the preconditioned Roe scheme or the results of Birken \& Meister.

The remedies to the stability issues of the convective scheme which do not sacrifice accuracy are the same as for the Roe-type schemes.
The most common solution is to use low Mach number preconditioning, which results in a $\mathcalOM{0}$ spectral radius on the convective timescale, but destroys the time accuracy of the scheme \cite{turkel_preconditioning_1999}.
When time accuracy is required, an implicit scheme can be used, often in conjunction with dual-time stepping \cite{jameson_time_1991}, which allows the use of low Mach number preconditioning while maintaining time accuracy \cite{venkateswaran_dual_1995,shima_cfd_2011,shima_new_2013}.
Using a scheme with the mixed diffusion scaling would also avoid this issue, but at the price of losing the robustness against pressure chequerboard instabilities.
It should be mentioned that preconditioning does not entirely remove the need for some cut-off Mach number because, although the condition number is reduced which improves the linear long-time convergence, the eigenvectors of the preconditioned system become increasingly non-normal as $M\to0$ which impedes the non-linear short-time convergence \cite{darmofal_importance_1996}.
In this case the cut-off Mach number should be applied only to the preconditioning matrix and $\mu_{11}$ to keep the correct scaling of $\mu_{22}$.\\

Turning now to the Zha-Bilgen and Toro-Vasquez splittings, the first thing to note is that none of the schemes in table \ref{tab:existing-schemes} use the first forms where $\delta(pU)=U\delta p$, which gives $\delta\underline{P}=\delta p\underline{\Pi}$ and maintains the separation of the $\delta p$ diffusion in the pressure perturbation term and the $\delta U$ diffusion in the velocity perturbation term.
Chen et al. 2018 (AUPM) \cite{chen_novel_2018} and Iampietro et al. 2020 \cite{iampietro_low-diffusion_2020} both use the second form of the Zha-Bilgen splitting with $\delta(pU)=p\delta U + U\delta p$, which gives diffusion equivalent to the Roe scheme on all three diffusion terms - convective upwinding, and the velocity and pressure perturbations.
Qu et al. 2018 (E-AUSMPWAS) \cite{qu_new_2018} use $\delta(pU)=p\delta U$, which is equivalent to using the Liou-Steffen form for both the velocity and pressure perturbation diffusion terms.
According to the findings of section \ref{sec:cp-splittings}, the differences between different forms of the $\delta p$ pressure perturbation terms are much less important than the form of the velocity perturbation term, so the difference in behaviour between these two expressions for $\delta(pU)$ should be small.

In the three Toro-Vasquez schemes in table \ref{tab:existing-schemes}, Sun et al. 2017 (TV-MAS) \cite{sun_robust_2017}, Lin et al. 2018 (TV-MAS2) \cite{lin_density_2018}, and Chen et al. 2021 (TVAP) \cite{chen_low-diffusion_2021}, the convection term $U_{1/2}\underline{\varphi}$ adds only convective upwind diffusion, and not the velocity perturbation $\delta U$ diffusion.
Instead, all of the $\delta U$ and $\delta p$ diffusion terms are included through the pressure perturbation term $\delta\underline{P}$.
$\delta U$ diffusion terms are added only to the density and energy equations, with the Liou-Steffen form used in the energy equation.
TVAP modifies this slightly, with the static enthalpy $(\rho h \delta U)$ used instead of the total enthalpy $(\rho E \delta U)$.
None of the TV schemes apply $\delta U$ diffusion to the momentum equations.
In their review of low Mach Roe-type schemes, Li \& Gu 2013 \cite{li_mechanism_2013} note that inconsistent application of the velocity perturbation diffusion across the different equations reduces the effectiveness of the convective diffusion scaling at suppressing pressure chequerboards.
Whether the same is true for the Toro-Vasquez flux with the mixed diffusion scaling has not yet been addressed.

TVAP adds a $\delta p$ diffusion term only to the momentum equations, so in effect uses the Liou-Steffen form for both the velocity and pressure perturbation diffusion terms, albeit with no $\delta U$ diffusion on the momentum equations.
TV-MAS keeps the Toro-Vasquez form of the $\delta p$ terms in the momentum and energy equations, but uses the mixed scaling for the $\mu_{22}$ coefficient in the momentum equations and the acoustic scaling in the energy equations.
Lin et al. \cite{lin_density_2018} show that having different $\mu_{22}$ scalings in each equation results in the density variations scaling as $\nabla\rho\sim\mathcalOM{}$, instead of the expected $\nabla\rho\sim\mathcalOM{2}$ for isentropic flow, even when the pressure variations are correct at $\nabla p\sim\mathcalOM{2}$.
They rectify this in TV-MAS2, and show that two low Mach number Roe-type schemes with the mixed diffusion scaling also require this correction \cite{thornber_numerical_2008,fillion_flica-ovap_2011}.\\

In this section we have discussed the diffusion scaling and form of a number of low Mach number convection-pressure flux-vector splitting schemes from the literature.
The Liou-Steffen splitting is the most popular, although schemes using the Zha-Bilgen and Toro-Vasquez splittings have also appeared recently.
All of the Liou-Steffen and Zha-Bilgen schemes surveyed match one of the general forms used here.
On the other hand, the Toro-Vasquez schemes do not match exactly, applying the velocity perturbation only to the density and energy equations, although the current analysis is presumed to still give a reasonable indication of their low Mach number behaviour.
Of the Zha-Bilgen and Toro-Vasquez schemes surveyed, every one has $\delta U$ diffusion equivalent to the Liou-Steffen and Roe forms on the energy equation.
The majority of schemes use the mixed diffusion scaling, especially in the more recent literature.
One notable exception to this is the AUSM$^+$-up scheme which uses the convective diffusion scaling, so should be treated as asymptotically equivalent to the preconditioned Roe scheme as $M\to0$.