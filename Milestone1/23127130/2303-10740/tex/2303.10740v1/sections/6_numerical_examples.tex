
Three numerical examples are presented to demonstrate the behaviour of the different splittings with both the convective and mixed diffusion scalings for convective and acoustic flows.
Results are shown for both the first and second forms of the Zha-Bilgen and Toro-Vasquez forms, with particular attention paid to the predicted entropy field for each example to show the effect of the erroneous diffusion terms found in section \ref{sec:cp-splittings}.

The same finite volume code used in \cite{hope-collins_artificial_2022} is used here, with the only difference being the interface flux functions.
These are constructed using (\ref{eq:general-cp-interface-flux}), (\ref{eq:general-interface-up}) and (\ref{eq:fds-interface-delta-up}).
$\underline{\varphi}$ is taken as the upwind value based on the sign of $U_{1/2}$, and the form of $\delta\underline{P}$ is as described for each splitting in section \ref{sec:cp-splittings-conserved}.
The scheme is first order in space, and uses first order Euler forward integration in time.
Note that this means that $1/M$ times as many timesteps are required for the convective scheme due to the increased spectral radius of the diffusion.
An ideal gas law is used for all cases, with $\gamma=1.4$ and $R=287.058$.

\subsection{One dimensional examples}

\subsubsection{Isolated soundwave}

\begin{figure}
    \centering
\begin{subfigure}[t]{0.49\textwidth}
    \centering
    \includegraphics[width=0.99\linewidth]{figures/OneD/soundwave/convective_pressure.png}
    \caption{}
    \label{fig:soundwave-convective}
\end{subfigure}
\begin{subfigure}[t]{0.49\textwidth}
    \centering
    \includegraphics[width=0.99\linewidth]{figures/OneD/soundwave/mixed_pressure.png}
    \caption{}
    \label{fig:soundwave-mixed}
\end{subfigure}
    \caption{Non-dimensional gauge pressure profiles for an isolated soundwave at $M=0.01$ after a single period using (a) convective diffusion scaling $\uuline{A}^c$ (b) mixed diffusion scaling $\uuline{A}^m$.}
    \label{fig:soundwave}
\end{figure}

The performance of the discrete schemes for purely acoustic flow is demonstrated using the isolated soundwave test case from \cite{hope-collins_artificial_2022}.
The initial profile, shown in figure \ref{fig:soundwave}, is a sinusoidal right travelling isentropic soundwave with a background flow at $M=0.01$ and 16 points per wavelength.
Pressure profiles after a single period are shown for each of the flux-splittings with the convective and mixed diffusion scaling in figures \ref{fig:soundwave-convective} and \ref{fig:soundwave-mixed} respectively, with the Roe scheme results shown for comparison.

With the convective diffusion scaling, all schemes except the first Toro-Vasquez form quickly damp the acoustic variations and closely match the results with the Roe scheme, as predicted by the limit equations (\ref{eq:limit-ls-Ac-acoustic}) and (\ref{eq:limit-zb-Ac-acoustic}).
The first form of the Toro-Vasquez scheme does not damp the acoustic wave at all due to the lack of pressure diffusion on the pressure equation.
The results with the mixed diffusion scaling, shown in figure \ref{fig:soundwave-mixed}, also agree with the findings of section \ref{sec:cp-splittings}.
The Liou-Steffen scheme and the second Zha-Bilgen and Toro-Vasquez forms match the Roe scheme almost exactly, and the first Zha-Bilgen form has similar resolution but slightly reduced diffusion.
The first Toro-Vasquez form predicts the soundwave amplitude growing, which is unsurprising given that the scheme asymptotically approaches a central scheme for acoustic flow, and a first order Euler forward discretisation is used for the time derivative.

Entropic profiles of $s=p/\rho^{\gamma}$ predicted by the different schemes are shown in figure \ref{fig:soundwave-entropy}.
The Roe and Liou-Steffen schemes preserve the isentropic nature of the soundwave for both convective and mixed diffusion scalings, as predicted by the limit equation analysis.
The second Zha-Bilgen and Toro-Vasquez forms preserve the isentropic profile to a similar degree, but have fractionally more entropy generation just visible in figure \ref{fig:soundwave-mixed-entropy}.
However, the first Zha-Bilgen and Toro-Vasquez forms both generate spurious entropy variations.
For the convective diffusion scaling, this can be seen from the limit equations (\ref{eq:limit-zb-Ac-acoustic}) and (\ref{eq:limit-tv-Ac-acoustic}) to be due to the erroneous pressure diffusion term in the acoustic entropy.
This term is a factor of $\gamma/(\gamma-1)$ larger in the Toro-Vasquez splitting and $\nabla^2p$ is larger because the soundwave is undamped, which results in a higher entropy generation for this scheme.
The variations are less than $1\%$ of the background value, but this is after the soundwave has travelled just a single wavelength.
With the mixed diffusion scaling, these erroneous pressure diffusion terms are one order of $M$ smaller than the leading order terms in the entropy equations of these two schemes, so disappear from the limit equations (\ref{eq:limit-zb-Am-acoustic}) and (\ref{eq:limit-tv-Am-acoustic}).
Figure \ref{fig:soundwave-mixed-entropy} shows that the entropy variations predicted with the mixed diffusion scaling are a factor of $M$ smaller than those predicted by the convective diffusion scaling, consistent with this scaling of the pressure diffusion terms.
The entropy profiles are essentially unchanged if the test case is rerun with the $\delta p$ terms in the interface pressure perturbation $\delta\underline{P}$ switched off, which further reinforces that the entropy generation of the first Zha-Bilgen and Toro-Vasquez forms is due to the form of the $\delta U$ diffusion terms in the energy equation.

\begin{figure}
    \centering
    \begin{subfigure}[t]{0.49\textwidth}
        \centering
        \includegraphics[width=0.99\linewidth]{figures/OneD/soundwave/convective_entropy.png}
        \caption{}
        \label{fig:soundwave-convective-entropy}
    \end{subfigure}
    \begin{subfigure}[t]{0.49\textwidth}
        \centering
        \includegraphics[width=0.99\linewidth]{figures/OneD/soundwave/mixed_entropy.png}
        \caption{}
        \label{fig:soundwave-mixed-entropy}
    \end{subfigure}
    \caption{Non-dimensional entropy profiles for an isolated soundwave at $M=0.01$ after a single period using (a) convective diffusion scaling $\uuline{A}^c$ (b) mixed diffusion scaling $\uuline{A}^m$.}
    \label{fig:soundwave-entropy}
\end{figure}

\subsection{Two dimensional examples}

\subsubsection{Circular cylinder}
The next test case is inviscid flow at $M=0.01$ around a circular cylinder, which will demonstrate the performance of the different schemes for purely convective stationary flow.
The setup is identical to that in \cite{hope-collins_artificial_2022}.
A stretched O-grid is used with the first cell height of $0.036r$, where $r$ is the cylinder radius, and the farfield boundary is $50r$ from the centre of the cylinder.
The farfield boundary conditions are enforced by setting ghost cell values using upstream velocity and entropy and downstream static pressure.
At the cylinder wall, curvature corrected boundary conditions are used which reduce spurious entropy production at the wall \cite{dadone_surface_1994}.
Convergence to steady state is accelerated using local timestepping and Weiss \& Smith's low Mach number preconditioner \cite{weiss_preconditioning_1995}.

The profiles of the gauge pressure, the pressure error relative to the analytical potential flow solution, the entropy, and the density using the Roe and Liou-Steffen schemes with the convective artificial diffusion scaling are shown in figures \ref{fig:cylinder-roe-convective} and \ref{fig:cylinder-ls-convective}.
As for the isolated soundwave, the Liou-Steffen results are almost indistinguishable from the Roe results.
The numerical results are close to the potential solution, with the pressure variations \ref{fig:cylinder-roe-convective-pressure} and \ref{fig:cylinder-ls-convective-pressure} on the correct order of magnitude.
The pressure loss downstream of the cylinder is due to the fact a first order scheme has been used, but the smooth error profiles \ref{fig:cylinder-roe-convective-pressure-error} and \ref{fig:cylinder-ls-convective-pressure-error} indicate that the pressure is free of chequerboard modes.
There is a very small amount of entropy generated at the wall visible in \ref{fig:cylinder-roe-convective-entropy} and \ref{fig:cylinder-ls-convective-entropy} - which is unavoidable with standard upwind schemes, although is reduced by the curvature corrected boundary conditions (see \cite{gouasmi_entropy-stable_2022} for a recent example of low Mach number entropy conserving/stable schemes) - and the density profiles \ref{fig:cylinder-roe-convective-density} and \ref{fig:cylinder-ls-convective-density} resemble the pressure profiles as expected.

The results for the first Zha-Bilgen splitting with the convective diffusion scaling are shown in figure \ref{fig:cylinder-zb-convective1}.
The pressure field \ref{fig:cylinder-zb-convective-pressure1} is essentially the same as that of the Roe and Liou-Steffen schemes, although the error plot \ref{fig:cylinder-zb-convective-pressure-error1} shows that the pressure loss downstream of the cylinder is marginally less due to the reduced diffusion in the  pressure equation.
However, the entropy profile \ref{fig:cylinder-zb-convective-entropy1} shows large unphysical variations due to the pressure anti-diffusion term in the entropy equation, which peak at around $30\%$ of the freestream values close to the wall where the pressure Poisson $\nabla^2p$ is largest.
The effect of this entropy generation can be seen in the density profile \ref{fig:cylinder-zb-convective-density1}, where large variations can also be seen close to the wall.
The results for the second Zha-Bilgen splitting with the convective diffusion scaling are shown in figure \ref{fig:cylinder-zb-convective2}.
The pressure and pressure error fields are essentially indistinguishable from the first form, but the entropy and density fields show clear improvement.
The entropy variations have been reduced to approximately the same level as the Roe and Liou-Steffen schemes, and the density profile is vastly improved.

The results for the first Toro-Vasquez form with the convective diffusion scaling are shown in figure \ref{fig:cylinder-tv-convective1}.
For this case, the previous schemes converged in around $25,000-30,000$ iterations, however the first Toro-Vasquez form did not converge, with NaNs occuring in the solution after around $42,000$ iterations.
Results are shown here after $30,000$ iterations.
Significant radial chequerboard modes can be seen in the pressure and pressure error profiles \ref{fig:cylinder-tv-convective-pressure1} and \ref{fig:cylinder-tv-convective-pressure-error1}.
The density and entropy profiles \ref{fig:cylinder-tv-convective-entropy1} and \ref{fig:cylinder-tv-convective-density1} also have radial chequerboard modes and large errors close to the wall due to the pressure anti-diffusion in the entropy equation.
These errors grow until the solution eventually diverges and breaks down.
The major difference between the limit equations of the first Zha-Bilgen and Toro-Vasquez forms with the convective diffusion scaling is the diffusion term in the pressure equation which, by comparing figures \ref{fig:cylinder-zb-convective1} and \ref{fig:cylinder-tv-convective1}, implies that this term somewhat damps the errors arising from the spurious entropy generation.
The results for the second Toro-Vasquez form with the convective diffusion scaling are shown in figure \ref{fig:cylinder-tv-convective2}.
As for the Zha-Bilgen scheme, the entropy and density fields show a marked improvement on the first form, although the entropy variations are about twice that seen for the Roe and Liou-Steffen schemes.\\


\begin{figure}
    \centering
    \begin{subfigure}[b]{0.475\textwidth}
        \centering
        \includegraphics[width=0.99\textwidth]{figures/TwoD/cylinder/roe.convective.pressure.png}
        \caption{Gauge pressure}
        \label{fig:cylinder-roe-convective-pressure}
    \end{subfigure}
    \hfill
    \begin{subfigure}[b]{0.475\textwidth}
        \centering
        \includegraphics[width=0.99\textwidth]{figures/TwoD/cylinder/roe.convective.pressure_error.png}
        \caption{Pressure error}
        \label{fig:cylinder-roe-convective-pressure-error}
    \end{subfigure}
    \vskip\baselineskip
    \begin{subfigure}[b]{0.475\textwidth}
        \centering
        \includegraphics[width=0.99\textwidth]{figures/TwoD/cylinder/roe.convective.entropy.png}
        \caption{Entropy}
        \label{fig:cylinder-roe-convective-entropy}
    \end{subfigure}
    \hfill
    \begin{subfigure}[b]{0.475\textwidth}
        \centering
        \includegraphics[width=0.99\textwidth]{figures/TwoD/cylinder/roe.convective.density.png}
        \caption{Density}
        \label{fig:cylinder-roe-convective-density}
    \end{subfigure}
    \caption{Flow around a circular cylinder at $M=0.01$ using the Roe scheme with convective diffusion scaling.}
    \label{fig:cylinder-roe-convective}
\end{figure}

\begin{figure}
    \centering
    \begin{subfigure}[b]{0.475\textwidth}
        \centering
        \includegraphics[width=0.99\textwidth]{figures/TwoD/cylinder/liou-steffen.convective.pressure.png}
        \caption{Gauge pressure}
        \label{fig:cylinder-ls-convective-pressure}
    \end{subfigure}
    \hfill
    \begin{subfigure}[b]{0.475\textwidth}
        \centering
        \includegraphics[width=0.99\textwidth]{figures/TwoD/cylinder/liou-steffen.convective.pressure_error.png}
        \caption{Pressure error}
        \label{fig:cylinder-ls-convective-pressure-error}
    \end{subfigure}
    \vskip\baselineskip
    \begin{subfigure}[b]{0.475\textwidth}
        \centering
        \includegraphics[width=0.99\textwidth]{figures/TwoD/cylinder/liou-steffen.convective.entropy.png}
        \caption{Entropy}
        \label{fig:cylinder-ls-convective-entropy}
    \end{subfigure}
    \hfill
    \begin{subfigure}[b]{0.475\textwidth}
        \centering
        \includegraphics[width=0.99\textwidth]{figures/TwoD/cylinder/liou-steffen.convective.density.png}
        \caption{Density}
        \label{fig:cylinder-ls-convective-density}
    \end{subfigure}
    \caption{Flow around a circular cylinder at $M=0.01$ using the Liou-Steffen scheme with convective diffusion scaling.}
    \label{fig:cylinder-ls-convective}
\end{figure}

\begin{figure}
    \centering
    \begin{subfigure}[b]{0.475\textwidth}
        \centering
        \includegraphics[width=0.99\textwidth]{figures/TwoD/cylinder/zha-bilgen1.convective.pressure.png}
        \caption{Gauge pressure}
        \label{fig:cylinder-zb-convective-pressure1}
    \end{subfigure}
    \hfill
    \begin{subfigure}[b]{0.475\textwidth}
        \centering
        \includegraphics[width=0.99\textwidth]{figures/TwoD/cylinder/zha-bilgen1.convective.pressure_error.png}
        \caption{Pressure error}
        \label{fig:cylinder-zb-convective-pressure-error1}
    \end{subfigure}
    \vskip\baselineskip
    \begin{subfigure}[b]{0.475\textwidth}
        \centering
        \includegraphics[width=0.99\textwidth]{figures/TwoD/cylinder/zha-bilgen1.convective.entropy.png}
        \caption{Entropy}
        \label{fig:cylinder-zb-convective-entropy1}
    \end{subfigure}
    \hfill
    \begin{subfigure}[b]{0.475\textwidth}
        \centering
        \includegraphics[width=0.99\textwidth]{figures/TwoD/cylinder/zha-bilgen1.convective.density.png}
        \caption{Density}
        \label{fig:cylinder-zb-convective-density1}
    \end{subfigure}
    \caption{Flow around a circular cylinder at $M=0.01$ using the first Zha-Bilgen scheme with convective diffusion scaling.}
    \label{fig:cylinder-zb-convective1}
\end{figure}

\begin{figure}
    \centering
    \begin{subfigure}[b]{0.475\textwidth}
        \centering
        \includegraphics[width=0.99\textwidth]{figures/TwoD/cylinder/zha-bilgen2.convective.pressure.png}
        \caption{Gauge pressure}
        \label{fig:cylinder-zb-convective-pressure2}
    \end{subfigure}
    \hfill
    \begin{subfigure}[b]{0.475\textwidth}
        \centering
        \includegraphics[width=0.99\textwidth]{figures/TwoD/cylinder/zha-bilgen2.convective.pressure_error.png}
        \caption{Pressure error}
        \label{fig:cylinder-zb-convective-pressure-error2}
    \end{subfigure}
    \vskip\baselineskip
    \begin{subfigure}[b]{0.475\textwidth}
        \centering
        \includegraphics[width=0.99\textwidth]{figures/TwoD/cylinder/zha-bilgen2.convective.entropy.png}
        \caption{Entropy}
        \label{fig:cylinder-zb-convective-entropy2}
    \end{subfigure}
    \hfill
    \begin{subfigure}[b]{0.475\textwidth}
        \centering
        \includegraphics[width=0.99\textwidth]{figures/TwoD/cylinder/zha-bilgen2.convective.density.png}
        \caption{Density}
        \label{fig:cylinder-zb-convective-density2}
    \end{subfigure}
    \caption{Flow around a circular cylinder at $M=0.01$ using the second Zha-Bilgen scheme with convective diffusion scaling.}
    \label{fig:cylinder-zb-convective2}
\end{figure}

\begin{figure}
    \centering
    \begin{subfigure}[b]{0.475\textwidth}
        \centering
        \includegraphics[width=0.99\textwidth]{figures/TwoD/cylinder/toro-vasquez1.convective.pressure.png}
        \caption{Gauge pressure}
        \label{fig:cylinder-tv-convective-pressure1}
    \end{subfigure}
    \hfill
    \begin{subfigure}[b]{0.475\textwidth}
        \centering
        \includegraphics[width=0.99\textwidth]{figures/TwoD/cylinder/toro-vasquez1.convective.pressure_error.png}
        \caption{Pressure error}
        \label{fig:cylinder-tv-convective-pressure-error1}
    \end{subfigure}
    \vskip\baselineskip
    \begin{subfigure}[b]{0.475\textwidth}
        \centering
        \includegraphics[width=0.99\textwidth]{figures/TwoD/cylinder/toro-vasquez1.convective.entropy.png}
        \caption{Entropy}
        \label{fig:cylinder-tv-convective-entropy1}
    \end{subfigure}
    \hfill
    \begin{subfigure}[b]{0.475\textwidth}
        \centering
        \includegraphics[width=0.99\textwidth]{figures/TwoD/cylinder/toro-vasquez1.convective.density.png}
        \caption{Density}
        \label{fig:cylinder-tv-convective-density1}
    \end{subfigure}
    \caption{Flow around a circular cylinder at $M=0.01$ using the first Toro-Vasquez scheme with convective diffusion scaling.}
    \label{fig:cylinder-tv-convective1}
\end{figure}

\begin{figure}
    \centering
    \begin{subfigure}[b]{0.475\textwidth}
        \centering
        \includegraphics[width=0.99\textwidth]{figures/TwoD/cylinder/toro-vasquez2.convective.pressure.png}
        \caption{Gauge pressure}
        \label{fig:cylinder-tv-convective-pressure2}
    \end{subfigure}
    \hfill
    \begin{subfigure}[b]{0.475\textwidth}
        \centering
        \includegraphics[width=0.99\textwidth]{figures/TwoD/cylinder/toro-vasquez2.convective.pressure_error.png}
        \caption{Pressure error}
        \label{fig:cylinder-tv-convective-pressure-error2}
    \end{subfigure}
    \vskip\baselineskip
    \begin{subfigure}[b]{0.475\textwidth}
        \centering
        \includegraphics[width=0.99\textwidth]{figures/TwoD/cylinder/toro-vasquez2.convective.entropy.png}
        \caption{Entropy}
        \label{fig:cylinder-tv-convective-entropy2}
    \end{subfigure}
    \hfill
    \begin{subfigure}[b]{0.475\textwidth}
        \centering
        \includegraphics[width=0.99\textwidth]{figures/TwoD/cylinder/toro-vasquez2.convective.density.png}
        \caption{Density}
        \label{fig:cylinder-tv-convective-density2}
    \end{subfigure}
    \caption{Flow around a circular cylinder at $M=0.01$ using the second Toro-Vasquez scheme with convective diffusion scaling.}
    \label{fig:cylinder-tv-convective2}
\end{figure}

%%%%%%%%%%%%%%%%%%%

The results for the Roe and Liou-Steffen schemes with the mixed diffusion scaling are shown in figures \ref{fig:cylinder-roe-mixed} and \ref{fig:cylinder-ls-mixed} respectively.
The pressure and density plots are very similar to the results found with the convective diffusion scaling, and the entropy generated at the walls is even lower.
Inspection of figures \ref{fig:cylinder-roe-mixed-pressure-error} and \ref{fig:cylinder-ls-mixed-pressure-error} reveals that the error profiles are no longer smooth, and a slight chequerboard exists in both solutions, as is expected for schemes with the mixed diffusion scaling.

The results for the first Zha-Bilgen form in figure \ref{fig:cylinder-zb-mixed1} with the mixed diffusion scaling also show that the pressure field is generally accurate, but a chequerboard mode also exists in the solution which is visible in \ref{fig:cylinder-zb-mixed-pressure-error1}.
The entropy and density variations \ref{fig:cylinder-zb-mixed-entropy1} and \ref{fig:cylinder-zb-mixed-density1} are approximately a factor of $M$ smaller than for the convective scaling \ref{fig:cylinder-zb-convective-entropy1} and \ref{fig:cylinder-zb-convective-density1},  although are still much larger than those found with the Roe and Liou-Steffen schemes.
The results for the second Zha-Bilgen form are shown in figure \ref{fig:cylinder-zb-mixed2} and closely resemble those of the Roe and Liou-Steffen schemes, including the pressure chequerboard and the level of the entropy variations.

The results for the first Toro-Vasquez form with mixed diffusion scaling in figure \ref{fig:cylinder-tv-mixed1} show a vast improvement on the results with the convective scaling in figure \ref{fig:cylinder-tv-convective1}.
The pressure field much more closely matches the potential solution, although still with a chequerboard mode visible in \ref{fig:cylinder-tv-mixed-pressure-error1}.
As for the Zha-Bilgen scheme, the entropy and density variations are much reduced compared to the convective scaling results - especially the entropy variations - while still being larger than the Roe and Liou-Steffen solutions.
The results for the second Toro-Vasquez form with mixed diffusion scaling are shown in figure \ref{fig:cylinder-tv-mixed2}.
As for with the convective diffusion scaling, the results are a significant improvement on the first form, although the entropy generation is still around twice that of the Roe and Liou-Steffen schemes.

\begin{figure}
    \centering
    \begin{subfigure}[b]{0.475\textwidth}
        \centering
        \includegraphics[width=0.99\textwidth]{figures/TwoD/cylinder/roe.mixed.pressure.png}
        \caption{Gauge pressure}
        \label{fig:cylinder-roe-mixed-pressure}
    \end{subfigure}
    \hfill
    \begin{subfigure}[b]{0.475\textwidth}
        \centering
        \includegraphics[width=0.99\textwidth]{figures/TwoD/cylinder/roe.mixed.pressure_error.png}
        \caption{Pressure error}
        \label{fig:cylinder-roe-mixed-pressure-error}
    \end{subfigure}
    \vskip\baselineskip
    \begin{subfigure}[b]{0.475\textwidth}
        \centering
        \includegraphics[width=0.99\textwidth]{figures/TwoD/cylinder/roe.mixed.entropy.png}
        \caption{Entropy}
        \label{fig:cylinder-roe-mixed-entropy}
    \end{subfigure}
    \hfill
    \begin{subfigure}[b]{0.475\textwidth}
        \centering
        \includegraphics[width=0.99\textwidth]{figures/TwoD/cylinder/roe.mixed.density.png}
        \caption{Density}
        \label{fig:cylinder-roe-mixed-density}
    \end{subfigure}
    \caption{Flow around a circular cylinder at $M=0.01$ using the Roe scheme with mixed diffusion scaling.}
    \label{fig:cylinder-roe-mixed}
\end{figure}

\begin{figure}
    \centering
    \begin{subfigure}[b]{0.475\textwidth}
        \centering
        \includegraphics[width=0.99\textwidth]{figures/TwoD/cylinder/liou-steffen.mixed.pressure.png}
        \caption{Gauge pressure}
        \label{fig:cylinder-ls-mixed-pressure}
    \end{subfigure}
    \hfill
    \begin{subfigure}[b]{0.475\textwidth}
        \centering
        \includegraphics[width=0.99\textwidth]{figures/TwoD/cylinder/liou-steffen.mixed.pressure_error.png}
        \caption{Pressure error}
        \label{fig:cylinder-ls-mixed-pressure-error}
    \end{subfigure}
    \vskip\baselineskip
    \begin{subfigure}[b]{0.475\textwidth}
        \centering
        \includegraphics[width=0.99\textwidth]{figures/TwoD/cylinder/liou-steffen.mixed.entropy.png}
        \caption{Entropy}
        \label{fig:cylinder-ls-mixed-entropy}
    \end{subfigure}
    \hfill
    \begin{subfigure}[b]{0.475\textwidth}
        \centering
        \includegraphics[width=0.99\textwidth]{figures/TwoD/cylinder/liou-steffen.mixed.density.png}
        \caption{Density}
        \label{fig:cylinder-ls-mixed-density}
    \end{subfigure}
    \caption{Flow around a circular cylinder at $M=0.01$ using the Liou-Steffen scheme with mixed diffusion scaling.}
    \label{fig:cylinder-ls-mixed}
\end{figure}

\begin{figure}
    \centering
    \begin{subfigure}[b]{0.475\textwidth}
        \centering
        \includegraphics[width=0.99\textwidth]{figures/TwoD/cylinder/zha-bilgen1.mixed.pressure.png}
        \caption{Gauge pressure}
        \label{fig:cylinder-zb-mixed-pressure1}
    \end{subfigure}
    \hfill
    \begin{subfigure}[b]{0.475\textwidth}
        \centering
        \includegraphics[width=0.99\textwidth]{figures/TwoD/cylinder/zha-bilgen1.mixed.pressure_error.png}
        \caption{Pressure error}
        \label{fig:cylinder-zb-mixed-pressure-error1}
    \end{subfigure}
    \vskip\baselineskip
    \begin{subfigure}[b]{0.475\textwidth}
        \centering
        \includegraphics[width=0.99\textwidth]{figures/TwoD/cylinder/zha-bilgen1.mixed.entropy.png}
        \caption{Entropy}
        \label{fig:cylinder-zb-mixed-entropy1}
    \end{subfigure}
    \hfill
    \begin{subfigure}[b]{0.475\textwidth}
        \centering
        \includegraphics[width=0.99\textwidth]{figures/TwoD/cylinder/zha-bilgen1.mixed.density.png}
        \caption{Density}
        \label{fig:cylinder-zb-mixed-density1}
    \end{subfigure}
    \caption{Flow around a circular cylinder at $M=0.01$ using the first Zha-Bilgen scheme with mixed diffusion scaling.}
    \label{fig:cylinder-zb-mixed1}
\end{figure}

\begin{figure}
    \centering
    \begin{subfigure}[b]{0.475\textwidth}
        \centering
        \includegraphics[width=0.99\textwidth]{figures/TwoD/cylinder/zha-bilgen2.mixed.pressure.png}
        \caption{Gauge pressure}
        \label{fig:cylinder-zb-mixed-pressure2}
    \end{subfigure}
    \hfill
    \begin{subfigure}[b]{0.475\textwidth}
        \centering
        \includegraphics[width=0.99\textwidth]{figures/TwoD/cylinder/zha-bilgen2.mixed.pressure_error.png}
        \caption{Pressure error}
        \label{fig:cylinder-zb-mixed-pressure-error2}
    \end{subfigure}
    \vskip\baselineskip
    \begin{subfigure}[b]{0.475\textwidth}
        \centering
        \includegraphics[width=0.99\textwidth]{figures/TwoD/cylinder/zha-bilgen2.mixed.entropy.png}
        \caption{Entropy}
        \label{fig:cylinder-zb-mixed-entropy2}
    \end{subfigure}
    \hfill
    \begin{subfigure}[b]{0.475\textwidth}
        \centering
        \includegraphics[width=0.99\textwidth]{figures/TwoD/cylinder/zha-bilgen2.mixed.density.png}
        \caption{Density}
        \label{fig:cylinder-zb-mixed-density2}
    \end{subfigure}
    \caption{Flow around a circular cylinder at $M=0.01$ using the second Zha-Bilgen scheme with mixed diffusion scaling.}
    \label{fig:cylinder-zb-mixed2}
\end{figure}

\begin{figure}
    \centering
    \begin{subfigure}[b]{0.475\textwidth}
        \centering
        \includegraphics[width=0.99\textwidth]{figures/TwoD/cylinder/toro-vasquez1.mixed.pressure.png}
        \caption{Gauge pressure}
        \label{fig:cylinder-tv-mixed-pressure1}
    \end{subfigure}
    \hfill
    \begin{subfigure}[b]{0.475\textwidth}
        \centering
        \includegraphics[width=0.99\textwidth]{figures/TwoD/cylinder/toro-vasquez1.mixed.pressure_error.png}
        \caption{Pressure error}
        \label{fig:cylinder-tv-mixed-pressure-error1}
    \end{subfigure}
    \vskip\baselineskip
    \begin{subfigure}[b]{0.475\textwidth}
        \centering
        \includegraphics[width=0.99\textwidth]{figures/TwoD/cylinder/toro-vasquez1.mixed.entropy.png}
        \caption{Entropy}
        \label{fig:cylinder-tv-mixed-entropy1}
    \end{subfigure}
    \hfill
    \begin{subfigure}[b]{0.475\textwidth}
        \centering
        \includegraphics[width=0.99\textwidth]{figures/TwoD/cylinder/toro-vasquez1.mixed.density.png}
        \caption{Density}
        \label{fig:cylinder-tv-mixed-density1}
    \end{subfigure}
    \caption{Flow around a circular cylinder at $M=0.01$ using the first Toro-Vasquez scheme with mixed diffusion scaling.}
    \label{fig:cylinder-tv-mixed1}
\end{figure}

\begin{figure}
    \centering
    \begin{subfigure}[b]{0.475\textwidth}
        \centering
        \includegraphics[width=0.99\textwidth]{figures/TwoD/cylinder/toro-vasquez2.mixed.pressure.png}
        \caption{Gauge pressure}
        \label{fig:cylinder-tv-mixed-pressure2}
    \end{subfigure}
    \hfill
    \begin{subfigure}[b]{0.475\textwidth}
        \centering
        \includegraphics[width=0.99\textwidth]{figures/TwoD/cylinder/toro-vasquez2.mixed.pressure_error.png}
        \caption{Pressure error}
        \label{fig:cylinder-tv-mixed-pressure-error2}
    \end{subfigure}
    \vskip\baselineskip
    \begin{subfigure}[b]{0.475\textwidth}
        \centering
        \includegraphics[width=0.99\textwidth]{figures/TwoD/cylinder/toro-vasquez2.mixed.entropy.png}
        \caption{Entropy}
        \label{fig:cylinder-tv-mixed-entropy2}
    \end{subfigure}
    \hfill
    \begin{subfigure}[b]{0.475\textwidth}
        \centering
        \includegraphics[width=0.99\textwidth]{figures/TwoD/cylinder/toro-vasquez2.mixed.density.png}
        \caption{Density}
        \label{fig:cylinder-tv-mixed-density2}
    \end{subfigure}
    \caption{Flow around a circular cylinder at $M=0.01$ using the second Toro-Vasquez scheme with mixed diffusion scaling.}
    \label{fig:cylinder-tv-mixed2}
\end{figure}

%%%%%%%%%%%%%%%%%%%

\subsubsection{Isentropic vortex}

The final test case is an inviscid stationary Gresho vortex \cite{gresho_theory_1990}, which will be used to further demonstrate the impact of the form of the entropy limit equations.
The maximum velocity of the vortex $u_m$ is at a radius $r$ and Mach number $M=0.01$.
The initial conditions can be found in the soundwave-vortex interaction test case of \cite{hope-collins_artificial_2022}, with the exception that in the present study the density profile is calculated using $d\rho=dp/a^2$ to enforce isentropic conditions instead of using a constant value.
The domain is $10r\times10r$, and is periodic in both $x$ and $y$.
Results are shown after a quarter turn of the vortex, i.e. $t=\frac{\pi r}{2u_m}$.

The pressure profile on a line through the centre of the vortex predicted by each of the schemes is shown in figure \ref{fig:vortex-pressure}, where it can be seen that all of the schemes diffuse the vortex to some extent.
With the convective diffusion scaling, the Roe, Liou-Steffen, and second Zha-Bilgen and Toro-Vasquez schemes match once again, the solution from the first Zha-Bilgen scheme is slightly less diffusive, and the first Toro-Vasquez scheme is less diffusive still, in agreement with the limit equation analysis.
The limit equations with the mixed diffusion scaling at the convective limit are the same for all four schemes, and figure \ref{fig:vortex-mixed-pressure} shows that the predicted pressure profiles are all almost identical to each other.
The solution with the first Toro-Vasquez scheme with the mixed diffusion scaling broke down and is not shown.

The entropy profiles found with the convective and mixed diffusion scalings are shown in figures \ref{fig:vortex-convective-entropy} and \ref{fig:vortex-mixed-entropy} respectively.
The Roe, Liou-Steffen, and second Zha-Bilgen schemes generate very little entropy, and the second Toro-Vasquez generates only a small amount more.
For the Roe and the second Zha-Bilgen and Toro-Vasquez schemes, the entropy generation is predominantly at the four corners where the flow is the most misaligned with the grid.
The Liou-Steffen splitting generates slightly more entropy than the Roe scheme, which is the opposite of the trend seen in the circular cylinder test case.
The profiles in \ref{fig:vortex-roe-convective-entropy} and \ref{fig:vortex-roe-mixed-entropy} are similar to the entropy production modes of an entropy stable Roe scheme shown in figure 11 of \cite{gouasmi_entropy-stable_2022} smeared by a quarter turn, and the entropy production after a single timestep closely resembles the production mode.
The entropy profiles found with the first Zha-Bilgen and Toro-Vasquez forms and the convective scaling have much larger variations, and their shape is consistent with what would be expected from the limit equation (\ref{eq:limit-zb-Ac-convective}) i.e. there appears to be an entropy sink where $\nabla^2p>0$ in the vortex core, and an entropy source where $\nabla^2p<0$ at the edge of the vortex.
The entropy profile for the first Zha-Bilgen form with the mixed diffusion scaling is qualitatively very similar to that with the convective diffusion scaling, except with the variations reduced by around an order of $M$.
However, the solution for the first Toro-Vasquez for with the mixed diffusion scaling has entirely broken down, as can be seen from figure \ref{fig:vortex-tv-mixed-entropy1}, with only very high wavenumber oblique modes remaining.
As with the previous examples, the second forms of Zha-Bilgen and Toro-Vasquez significantly improve the solution compared to the first forms, with the entropy field for the second Zha-Bilgen form almost the same as with the Roe scheme.
The benefits of the second form are especially clear for the Toro-Vasquez scheme with mixed diffusion scaling, preventing the total breakdown that was seen with the first form, with the entropy variations on the same order of magnitude as for the Roe scheme.\\

%\begin{figure}
%    \centering
%    \begin{subfigure}[t]{0.35\textwidth}
%        \centering
%        \includegraphics[width=0.99\linewidth]{figures/TwoD/vortex/pressure.initial.png}
%        \caption{}
%        \label{fig:vortex-initial-pressure}
%    \end{subfigure}
%    \begin{subfigure}[t]{0.35\textwidth}
%        \centering
%        \includegraphics[width=0.99\linewidth]{figures/TwoD/vortex/velocity.initial.png}
%        \caption{}
%        \label{fig:vortex-initial-velocity}
%    \end{subfigure}
%    \caption{Initial pressure (a) and velocity (b) profiles for the isentropic Gresho vortex}
%    \label{fig:vortex-initial}
%\end{figure}

\begin{figure}
    \centering
\begin{subfigure}[t]{0.49\textwidth}
    \centering
    \includegraphics[width=0.99\linewidth]{figures/TwoD/vortex/pressure_centreline_convective.png}
    \caption{}
    \label{fig:vortex-convective-pressure}
\end{subfigure}
\begin{subfigure}[t]{0.49\textwidth}
    \centering
    \includegraphics[width=0.99\linewidth]{figures/TwoD/vortex/pressure_centreline_mixed.png}
    \caption{}
    \label{fig:vortex-mixed-pressure}
\end{subfigure}
    \caption{Non-dimensional gauge pressure profiles for a Gresho vortex at $M=0.01$ after $1/4$ rotation using (a) convective diffusion scaling $\uuline{A}^c$ (b) mixed diffusion scaling $\uuline{A}^m$. $p_{\text{min}}$ is the minimum pressure at the centre of the vortex core at $t=0$.}
    \label{fig:vortex-pressure}
\end{figure}

\begin{figure}
    \centering
    \begin{subfigure}[b]{0.30\textwidth}
        \centering
        \includegraphics[width=0.99\textwidth]{figures/TwoD/vortex/entropy.roe.convective.png}
        \caption{Roe}
        \label{fig:vortex-roe-convective-entropy}
    \end{subfigure}
    \begin{subfigure}[b]{0.30\textwidth}
        \centering
        \includegraphics[width=0.99\textwidth]{figures/TwoD/vortex/entropy.liou-steffen.convective.png}
        \caption{Liou-Steffen}
        \label{fig:vortex-ls-convective-entropy}
    \end{subfigure}
    \vskip\baselineskip
    \begin{subfigure}[b]{0.30\textwidth}
        \centering
        \includegraphics[width=0.99\textwidth]{figures/TwoD/vortex/entropy.zha-bilgen1.convective.png}
        \caption{Zha-Bilgen 1}
        \label{fig:vortex-zb-convective-entropy1}
    \end{subfigure}
    \begin{subfigure}[b]{0.30\textwidth}
        \centering
        \includegraphics[width=0.99\textwidth]{figures/TwoD/vortex/entropy.zha-bilgen2.convective.png}
        \caption{Zha-Bilgen 2}
        \label{fig:vortex-zb-convective-entropy2}
    \end{subfigure}
    \vskip\baselineskip
    \begin{subfigure}[b]{0.30\textwidth}
        \centering
        \includegraphics[width=0.99\textwidth]{figures/TwoD/vortex/entropy.toro-vasquez1.convective.png}
        \caption{Toro-Vasquez 1}
        \label{fig:vortex-tv-convective-entropy1}
    \end{subfigure}
    \begin{subfigure}[b]{0.30\textwidth}
        \centering
        \includegraphics[width=0.99\textwidth]{figures/TwoD/vortex/entropy.toro-vasquez2.convective.png}
        \caption{Toro-Vasquez 2}
        \label{fig:vortex-tv-convective-entropy2}
    \end{subfigure}
    \caption{Non-dimensional entropy profiles for a Gresho vortex at $M=0.01$ after $1/4$ rotation using the convective diffusion scaling $\uuline{A}^c$.}
    \label{fig:vortex-convective-entropy}
\end{figure}

\begin{figure}
    \centering
    \begin{subfigure}[b]{0.30\textwidth}
        \centering
        \includegraphics[width=0.99\textwidth]{figures/TwoD/vortex/entropy.roe.mixed.png}
        \caption{Roe}
        \label{fig:vortex-roe-mixed-entropy}
    \end{subfigure}
    \begin{subfigure}[b]{0.30\textwidth}
        \centering
        \includegraphics[width=0.99\textwidth]{figures/TwoD/vortex/entropy.liou-steffen.mixed.png}
        \caption{Liou-Steffen}
        \label{fig:vortex-ls-mixed-entropy}
    \end{subfigure}
    \vskip\baselineskip
    \begin{subfigure}[b]{0.30\textwidth}
        \centering
        \includegraphics[width=0.99\textwidth]{figures/TwoD/vortex/entropy.zha-bilgen1.mixed.png}
        \caption{Zha-Bilgen 1}
        \label{fig:vortex-zb-mixed-entropy1}
    \end{subfigure}
    \begin{subfigure}[b]{0.30\textwidth}
        \centering
        \includegraphics[width=0.99\textwidth]{figures/TwoD/vortex/entropy.zha-bilgen2.mixed.png}
        \caption{Zha-Bilgen 2}
        \label{fig:vortex-zb-mixed-entropy2}
    \end{subfigure}
    \vskip\baselineskip
    \begin{subfigure}[b]{0.30\textwidth}
        \centering
        \includegraphics[width=0.99\textwidth]{figures/TwoD/vortex/entropy.toro-vasquez1.mixed.png}
        \caption{Toro-Vasquez 1}
        \label{fig:vortex-tv-mixed-entropy1}
    \end{subfigure}
    \begin{subfigure}[b]{0.30\textwidth}
        \centering
        \includegraphics[width=0.99\textwidth]{figures/TwoD/vortex/entropy.toro-vasquez2.mixed.png}
        \caption{Toro-Vasquez 2}
        \label{fig:vortex-tv-mixed-entropy2}
    \end{subfigure}
    \caption{Non-dimensional entropy profiles for a Gresho vortex at $M=0.01$ after $1/4$ rotation using the mixed diffusion scaling $\uuline{A}^m$.}
    \label{fig:vortex-mixed-entropy}
\end{figure}

We have shown three numerical examples which demonstrate the behaviour of the three flux-vector-splittings for both acoustic and convective flow.
Most schemes have the expected property that the mixed diffusion scaling is suitable for both convective and acoustic flow, whereas the convective diffusion scaling will only properly resolve convective flow and will quickly damp out acoustic waves.
However, the first Toro-Vasquez form will not damp out acoustic waves even with the convective scaling.
All three schemes suffer from pressure chequerboard modes with the mixed scaling, but the convective diffusion scaling resolves this issue for all schemes except the first Toro-Vasquez scheme.
The most significant result is the demonstration that the erroneous pressure anti-diffusion terms in the entropy variable limit equations of the Zha-Bilgen and Toro-Vasquez schemes do in fact degrade the solutions found with these schemes.
On the other hand, the Liou-Steffen results almost exactly match the results found with the Roe-type scheme for all cases and with both the convective and mixed diffusion scaling, as predicted by the limit equations analysis.
