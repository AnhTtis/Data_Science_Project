
Low Mach number flows occur in many applications of engineering and scientific interest, and the design of numerical schemes that can accurately simulate these flows has been an active research area for several decades \cite{turkel_review_1993,guillard_chapter_2017}.
It is well-known that collocated density-based schemes for compressible flow may fail to accurately resolve convective flow features at low Mach number due to the incorrect scaling of the artificial diffusion terms in the asymptotic limit $M\to0$ \cite{guillard_behaviour_1999,turkel_preconditioning_1999}.
A great many schemes which overcome this failing have been developed, including central-difference schemes \cite{turkel_preconditioned_1987,turkel_review_1993,venkateswaran_artficial_2003} and flux-difference splitting schemes \cite{guillard_behaviour_1999,guillard_behavior_2004,thornber_improved_2008,rieper_low-mach_2011}, as well as convection-pressure flux-vector splitting schemes \cite{liou_numerical_1999,liou_sequel_2006,shima_parameter-free_2011} which are the focus of the current work.

Convection-pressure flux-vector splittings separate the physical flux of the Euler equations into two terms: one describing the transport of some set of convected quantities, and the other containing the contribution of the pressure.
This is in contrast with other flux-vector splitting methods where the flux vector is split into a forward travelling wave and a backward travelling wave \cite{steger_flux_1981,van_leer_flux-vector_1982}.
A numerical flux function can be constructed from a convection-pressure flux-vector splitting by finding suitable discretisations for each term, and fluxes of this type are attractive for a number of reasons.
Separating the flux allows different modelling approaches to be used for each term, which is especially useful at low Mach number where the velocity and pressure play increasingly distinct roles \cite{muller_low_1999}.
Splitting in this manner also gives a balance of accuracy and computational efficiency: intermediate contact waves can be accurately resolved unlike standard flux-vector splitting schemes, but the computational expense scales with the size $n$ of the system being solved instead of $n^2$ for flux-difference splittings, which is especially useful for multi-component flows \cite{liou_new_1993,liou_sequel_1996}.

The most commonly used convection-pressure flux-vector splitting is the Liou-Steffen splitting, named advection upstream splitting method (AUSM) in the original papers from Liou \& Steffen 1993 and Liou 1996 \cite{liou_new_1993,liou_sequel_1996}.
This splitting has been widely applied, including to low Mach number \cite{liou_sequel_2006}, hypersonic \cite{kitamura_reduced_2016}, multicomponent and multiphase \cite{paillere_extension_2003,kitamura_extension_2014} flows.
The earliest low Mach number schemes were inspired by the ideas of low Mach number preconditioning \cite{edwards_low-diffusion_1998,liou_numerical_1999,edwards_towards_2001}, which can be combined with flux-difference splitting schemes to overcome the accuracy problem at low Mach number \cite{turkel_preconditioning_1999,guillard_behaviour_1999}.
Liou 2006 \cite{liou_sequel_2006} integrated these ideas into the AUSM$^+$-up for all speeds scheme, which is still in widespread use today.
Of the more recent Liou-Steffen splitting schemes suitable for low Mach number, SLAU and its descendants are particularly popular \cite{shima_parameter-free_2011,kitamura_towards_2013,kitamura_reduced_2016}.
Two other convection-pressure flux-vector splittings are the Zha-Bilgen \cite{zha_numerical_1993} and Toro-Vasquez \cite{toro_flux_2012} splittings, which differ from the Liou-Steffen splitting only in the treatment of the energy equation.
The Zha-Bilgen splitting was derived by splitting the exact flux Jacobian into convective and acoustic components in \cite{schiff_numerical_1980,steger_flux_1981}.
Zha \& Bilgen 1993 \cite{zha_numerical_1993} developed a numerical scheme for this splitting, and as such it is usually referred to by their names.
The Toro-Vasquez splitting was proposed by Toro \& Vasquez 2012 \cite{toro_flux_2012} in the context of developing a general framework for the discretisation of convection-pressure flux-vector splittings.
Over the last decade, the Zha-Bilgen and Toro-Vasquez splittings have gained some attention for low Mach number flow \cite{qu_new_2018,chen_novel_2018,iampietro_low-diffusion_2020,sun_robust_2017,lin_density_2018,chen_low-diffusion_2021}, and it has been shown that accurate schemes for this regime can be produced from these splittings.
We cover more detail on the literature for low Mach number schemes using convection-pressure flux-vector splittings in section \ref{sec:existing-schemes}.\\

In our previous paper \emph{``Artificial diffusion for convective and acoustic low Mach number flows I: Analysis of the modified equations, and application to Roe-type schemes''} \cite{hope-collins_artificial_2022} we analysed the form and behaviour of artificial diffusion in numerical schemes at low Mach number.
Three diffusion scalings naturally arise - one for purely convective flow, one for purely acoustic flow, and one for flow with either or both convective and acoustic features.
Our approach used the modified equations, which make the artificial diffusion terms of the discrete numerical scheme explicit in the continuous PDE and enable the behaviour of these terms to be studied independently of the specific discretisation they arose from.
The modified equations have previously been used to study low Mach number numerical schemes in \cite{turkel_preconditioning_1999,dellacherie_analysis_2010}.
By applying this analysis to Roe-type schemes, we showed that many existing low Mach number schemes from the literature adhere to one of the three diffusion scalings, and display the expected behaviour.
While the behaviours of these schemes have been shown and studied previously, this approach demonstrated that many properties of low Mach number schemes can be predicted and understood using the continuous equations, and that considering both convective and acoustic flow is necessary to provide a complete picture of this behaviour.

In this paper, we apply this approach to investigate the behaviour of convection-pressure flux-vector splitting schemes at low Mach number, specifically the Liou-Steffen, Zha-Bilgen, and Toro-Vasquez splittings.
By forming explicit diffusion matrices for each splitting, we can clearly see the differences in the forms of the diffusion, and by transforming to the entropy variables we can make direct comparison to the results of \cite{hope-collins_artificial_2022}.
Many previous studies of low Mach number numerical schemes have focused on Roe-type schemes - as we did in \cite{hope-collins_artificial_2022} - and the findings are extended to convection-pressure flux-vector splitting schemes by identifying the diffusion terms that these schemes have in common with the Roe-type schemes \cite{dellacherie_analysis_2010,sachdev_improved_2012,li_mechanism_2013}.
While we also do this here, working in the entropy variables means that we can additionally show the effect of the diffusion terms which are different between the convection-pressure flux-vector splitting schemes and the Roe-type schemes.
The particular form of these terms turns out to be crucial to the success of these schemes at low Mach number.

The structure of the remainder of the paper is as follows.
In section \ref{sec:lowmach-euler} we very briefly cover the necessary background for the low Mach number limits of the Euler equations, and in section \ref{sec:artificial-diffusion} we recap some of the key findings of \cite{hope-collins_artificial_2022} which will be used later.
Section \ref{sec:cp-splittings} contains the main content of this study.
The approximate diffusion matrix for each splitting is shown in the conserved variables, then transformed to the entropy variables.
The form in the entropy variables is discussed, and the limiting form of the modified equations for each splitting at both the convective and acoustic limits are found.
In section \ref{sec:existing-schemes} we look at a number of existing schemes from the literature, compare them to the generic forms used in section \ref{sec:cp-splittings}, and identify which of the three diffusion scalings they use.
The results of several numerical examples are shown in section \ref{sec:numerical-examples} and display excellent agreement with the analysis of section \ref{sec:cp-splittings} for all three splittings for both convective and acoustic flow.
