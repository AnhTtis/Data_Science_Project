
In \cite{hope-collins_artificial_2022}, the $x$-split form of the modified equations in the entropy variables was used to determine the appropriate scaling of the artificial diffusion at low Mach number.
Although convective and acoustic flow features can only properly be distinguished with a truly multi-dimensional analysis \cite{dellacherie_analysis_2010,barsukow_truly_2021}, many finite-volume/difference schemes use a dimension splitting procedure, and many low Mach number behaviours of these schemes can be demonstrated with the 1D-split equations - with the understanding that the final scheme will be the sum over all dimensions.
The modified equations make the artificial diffusion terms which are inherent in many numerical schemes explicit in the continuous equations \cite{warming_modified_1974}, which allows analysis of the behaviour of these terms without restriction to a specific discrete form.
The modified equations in the symmetry variables were used in \cite{turkel_preconditioning_1999} to demonstrate the behaviour of the artificial diffusion of the standard and preconditioned Roe scheme, and the modified equations of the barotropic Euler equations were used in \cite{dellacherie_analysis_2010} to investigate the requirements for accurate schemes at the convective limit.
The entropy variables are useful at low Mach number because in the 1D-split form the vorticity (transverse-velocity) and entropy equations are equivalent to scalar advection equations, which require only the usual upwind diffusion using the convective velocity scale.
This reduces the degrees of freedom we need to consider to only those in the pressure and (normal-)velocity equations, which are the distinguishing feature between the different low Mach number limits and schemes.

In the first part of this section, the form of the artificial diffusion in the entropy variables used in \cite{hope-collins_artificial_2022} will be introduced, and the required scaling at each of the three low Mach number limits will be recalled and briefly discussed.
In the second part of the section, the form of the artificial diffusion transformed to the conserved variables will be presented, which will be useful later for comparison with the diffusion of the convection-pressure flux-vector splitting schemes.

\subsection{Entropy variables form}
The form of the modified equations in the non-dimensional entropy variables used in \cite{hope-collins_artificial_2022} to investigate the behaviour of the artificial diffusion is:\footnote{The form written here is equivalent to equation (14) with the diffusion coefficients from equation (41) in \cite{hope-collins_artificial_2022}.}
\begin{equation} \label{eq:euler_modified}
    \begin{aligned}
%        \label{eq:pressure_modified}
        \partial_t p +           u\partial_x p +\gamma p\partial_x u & = \phantom{\rho}\mu_u|u|\partial_{xx}p + M^{\-2}\frac{\gamma p}{\rho|v|}\mu_{11}\partial_{xx}p \pm \gamma p\mu_{12}\partial_{xx}u \\
%        \label{eq:velocity_modified}
        \rho\partial_t u + M^{\-2}\partial_x p +  \rho u\partial_x u & = \rho\mu_u|u|\partial_{xx}u \phantom{\frac{\gamma p}{\rho|v|}}\pm M^{\-2}\mu_{21}\partial_{xx}p + \rho|v|\mu_{22}\partial_{xx}u \\
%        \label{eq:vorticity_modified}
        \partial_t v                           +       u\partial_x v & = \phantom{\rho}\mu_u|u|\partial_{xx}v \\
%        \label{eq:entropy_modified}
        \partial_t s                           +       u\partial_x s & = \phantom{\rho}\mu_u|u|\partial_{xx}s
    \end{aligned}
\end{equation}
The terms with the $\mu_u$ diffusion coefficient are the convective upwinding, so $\mu_u\sim\mathcalOM{0}$, and $|v|$ is some velocity scale.
The diffusion terms with the $\mu_{ij}$, $i,j=1,2$ coefficients are elements of the artificial diffusion Jacobian $\uuline{A}$.
The form of the Jacobian elements used here was chosen because it simplifies the diffusion coefficients once we transform to the conserved variables below.
The precise form of $\mu_{ij}$, $i,j=1,2$ is particular to each discrete scheme.
The sign of the off-diagonal terms may be negative so long as the diffusion Jacobian remains positive (semi-)definite.
See \cite{hope-collins_artificial_2022} for a discussion of the requirements on the form of the off-diagonal diffusion components.

In \cite{hope-collins_artificial_2022} the method of Turkel 1999 \cite{turkel_preconditioning_1999} was used to find the limiting forms of (\ref{eq:euler_modified}) for each of the convective and acoustic limits by enforcing the variations in table \ref{tab:lowmach_scaling}.
The Mach number scaling of the coefficients $\mu_{ij}$ was then chosen for each of the two limits such that the artificial diffusion terms had the same Mach number scaling as the dominant physical terms on the left hand side.
This ensures that the artificial diffusion terms do not vanish asymptotically - which would result in loss of stability - nor do they dominate asymptotically - which would result in loss of accuracy.
The $M^{\-2}$ coefficients in $A_{11}$ and $A_{21}$ are included so that at the convective limit all $\mu_{ij}$ are independent of the Mach number, as is appropriate for the incompressible regime, while keeping the correct scaling for the full Jacobian elements $A_{ij}$.
The convective and acoustic diffusion scalings were then combined into a mixed diffusion scaling which is suitable for both convective and acoustic flow features.
The scaling of the coefficients for the convective ($\uuline{A}^c$), acoustic ($\uuline{A}^a$) and mixed ($\uuline{A}^m)$ diffusion Jacobians are shown in table \ref{tab:diffusion-scaling}.
Each of these three diffusion scalings has its own strengths and weaknesses.
A brief description is given here, with more detail given in \cite{hope-collins_artificial_2022} and references therein.

\begin{table}
\def\-{\scalebox{0.4}[1.0]{\(-\)}}
    \centering
    \renewcommand{\arraystretch}{1.5}
    \begin{tabular}{|c|c|c|c|} \hline
        & $\uuline{A}^c$ & $\uuline{A}^a$ & $\uuline{A}^m$ \\ \hline
        {\renewcommand{\arraystretch}{1} $
        \begin{matrix} \begin{pmatrix}
        A_{11} & A_{12} \\
        A_{21} & A_{22}
        \end{pmatrix} \end{matrix}$ } &
        {\renewcommand{\arraystretch}{1} $\mathcal{O}
        \begin{matrix} \begin{pmatrix}
        M^{\-2} & M^{0} \\
        M^{\-2} & M^{0}
        \end{pmatrix} \end{matrix}$} &
        {\renewcommand{\arraystretch}{1} $\mathcal{O}
        \begin{matrix} \begin{pmatrix}
        M^{\-1} & M^{0} \\
        M^{\-2} & M^{\-1}
        \end{pmatrix} \end{matrix}$} &
        {\renewcommand{\arraystretch}{1} $\mathcal{O}
        \begin{matrix} \begin{pmatrix}
        M^{\-1} & M^{0} \\
        M^{\-2} & M^{0}
        \end{pmatrix} \end{matrix}$} \\ \hline
        {\renewcommand{\arraystretch}{1} $
        \begin{matrix} \begin{pmatrix}
        \mu_{11} & \mu_{12} \\
        \mu_{21} & \mu_{22}
        \end{pmatrix} \end{matrix}$ } &
        {\renewcommand{\arraystretch}{1} $\mathcal{O}
        \begin{matrix} \begin{pmatrix}
        M^{0} & M^{0} \\
        M^{0} & M^{0}
        \end{pmatrix} \end{matrix}$} &
        {\renewcommand{\arraystretch}{1} $\mathcal{O}
        \begin{matrix} \begin{pmatrix}
        M^{} & M^{0} \\
        M^{0} & M^{\-1}
        \end{pmatrix} \end{matrix}$} &
        {\renewcommand{\arraystretch}{1} $\mathcal{O}
        \begin{matrix} \begin{pmatrix}
        M^{} & M^{0} \\
        M^{0} & M^{0}
        \end{pmatrix} \end{matrix}$} \\ \hline
        Suitable for convective flow & Yes  & No & Yes$^{1}$  \\ \hline
        Suitable for acoustic flow   & No & Yes  & Yes$^{2}$  \\ \hline
        Timestep limit & \mathcalOM{2} & \mathcalOM{1} & \mathcalOM{1} \\ \hline
    \end{tabular}
    \caption{Details of the convective $\uuline{A}^c$, acoustic $\uuline{A}^a$, and mixed $\uuline{A}^m$ diffusion scalings. The mixed scheme is susceptible to $^{1}$pressure chequerboard instabilities at the convective limit and $^{2}$velocity instabilities at the acoustic limit.}
    \label{tab:diffusion-scaling}
\end{table}

The convective diffusion scaling has a large $\mathcalOM{\-2}$ pressure diffusion coefficient $A_{11}$ to compensate for the $\mathcalOM{2}$ scaling of $\partial_{x}p$ at the convective limit.
This term prevents pressure chequer-board instabilities, which result from the same pressure-velocity decoupling that occurs with collocated schemes for incompressible flow.
This increased pressure diffusion also damps out any acoustic features in the flow, making the convective scheme unsuitable for the acoustic or mixed limits.
The pressure diffusion also increases the spectral radius of the diffusion matrix by a factor of $\mathcalOM{\-1}$ compared to the physical flux Jacobian and the other two diffusion matrix scalings.
This means that an implicit time integration strategy is required for time-accurate simulations, as the timestep restriction scale as $\Delta t\sim\mathcalOM{2}$ instead of the usual $\Delta t\sim\mathcalOM{}$ for stability of the explicit scheme.

The acoustic diffusion scaling has a smaller pressure diffusion coefficient, so can resolve acoustic flow features.
However, the velocity diffusion $A_{22}$ is a factor of $\mathcalOM{\-1}$ larger than for the convective scaling, which destroys the accuracy for convective flow features and makes this scaling unsuitable for the convective or mixed limits.
Most schemes for compressible flow designed for higher Mach number regimes approach this scaling at low Mach number, and hence are inaccurate at low Mach number unless at least this term is modified.

The mixed scaling has neither the $\mathcalOM{\-2}$ pressure diffusion of the convective diffusion nor the $\mathcalOM{\-1}$ velocity diffusion of the acoustic scaling.
As such it does not overdamp acoustic nor convective flow features and is accurate for the convective, acoustic, and mixed limits, although this flexibility does not come without cost.
The pressure diffusion vanishes asymptotically at the purely convective limit, meaning that this scheme is susceptible to pressure chequerboard instabilities.
The velocity diffusion vanishes asymptotically at the purely acoustic limit, meaning that there is an undamped grid mode which can be excited by non-linearities, initial or boundary conditions, poor mesh quality, and acoustic discontinuities such as shockwaves.

\subsection{Conserved variables form}

Transforming the modified equations in the non-dimensional entropy variables (\ref{eq:euler_modified}) to the dimensional conserved variables, we can construct a numerical interface flux $\underline{\hat{f}}$ over a face with normal $\underline{n}$:
\begin{equation}\label{eq:fds-interface-flux}
\renewcommand{\arraystretch}{1.25}
    \underline{\hat{f}} = \{\underline{\tilde{f}}\}
    -\frac{1}{2}
    \Bigg[
    \mu_u|\tilde{U}|
    \begin{matrix} \begin{pmatrix}
        \Delta \tilde{\rho}  \\
        \Delta \tilde{\rho}\tilde{\underline{u}} \\
        \Delta \tilde{\rho}\tilde{E}
    \end{pmatrix} \end{matrix}
    +
    \delta \tilde{U}
    \begin{matrix} \begin{pmatrix}
        \tilde{\rho}   \\
        \tilde{\rho}\tilde{\underline{u}} \\
        \tilde{\rho}\tilde{H}
    \end{pmatrix} \end{matrix}
    +
    \delta \tilde{p}
    \begin{matrix} \begin{pmatrix}
        0 \\
        \underline{n} \\
        \tilde{U}
    \end{pmatrix} \end{matrix}
    \Bigg]
    = \{\underline{\tilde{f}}\} - \underline{\tilde{f}}^d
\end{equation}
\begin{equation} \label{eq:fds-interface-delta-up}
    \begin{split}
        \delta \tilde{U} & = \frac{\mu_{11}}{\tilde{\rho}|\tilde{v}|}\Delta\tilde{p} + \frac{\tilde{U}}{|\tilde{U}|}\mu_{12}\Delta\tilde{U} \\
        \delta \tilde{p} & = \frac{\tilde{U}}{|\tilde{U}|}\mu_{21}\Delta\tilde{p} + \tilde{\rho}|\tilde{v}|\mu_{22}\Delta\tilde{U} 
    \end{split}
\end{equation}
Where $\{(\dotp)\}=\frac{1}{2}((\dotp)_L+(\dotp)_R)$ and $\Delta(\dotp)=(\dotp)_R-(\dotp)_L$ are the central average and the jump between the left/right states of the interface and $U=\underline{u}\dotp\underline{n}$ is the interface-normal velocity.
The first term $\{\underline{\tilde{f}}\}$ is the central approximation to the physical flux, and the other terms are the artificial diffusion $\underline{\tilde{f}}^d$.
The first of these is the convective upwinding on the conserved variables.
The interface velocity perturbation $\delta \tilde{U}$ comes from the diffusion on the pressure in the entropy variables, and the interface pressure perturbation $\delta \tilde{p}$ comes from the diffusion on the velocity in the entropy variables.
The switch $\tilde{U}/|\tilde{U}|$ ensures that the off-diagonal terms maintain a diffusive character.
Matrix Rusanov fluxes such as the Roe scheme \cite{roe_approximate_1981} can be arranged in the form of (\ref{eq:fds-interface-flux}) \cite{liu_upwind_1989,weiss_preconditioning_1995} to compare existing Roe-type schemes at low Mach number to the scalings in table \ref{tab:diffusion-scaling}, as was done in \cite{li_mechanism_2013} and \cite{hope-collins_artificial_2022}.
This form will be referred to as the Roe form for the rest of the paper.
