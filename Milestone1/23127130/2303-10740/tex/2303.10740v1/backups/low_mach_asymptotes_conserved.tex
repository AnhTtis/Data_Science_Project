
\textit{What is the point of this section? \\ To introduce the necessary theoretical background for the rest of the paper. The asymptotic solutions section is about the physical behaviour, the symmetrising variables section is on what this means for designing accurate numerical schemes. No original content except the stability estimates at the end.}

\subsection{Convective and acoustic asymptotic solutions}
We review the theoretical results for the asymptotic solutions for the Euler equations at low Mach number, as preparation for analysing numerical schemes in this regime later.
We follow closely the exposition of Muller99 **, highlighting the results most relevant to the current study. \\
First, all variables are non-dimensionalised using:
\begin{equation}
\begin{gathered}
    \tilde{\rho} = \frac{\rho}{\rho_{\infty}},
    \quad
    \tilde{\underline{u}} = \frac{\underline{u}}{u_{\infty}},
    \quad
    \tilde{\underline{x}} = \frac{\underline{x}}{L_{\infty}},
    \quad
    \tilde{e} = \frac{e}{a^2_{\infty}},
    \quad
    \tilde{h} = \frac{h}{a^2_{\infty}},
    \quad
    \tilde{p} = \frac{p}{\rho a^2_{\infty}},
    \quad
    \tilde{t} = \frac{t}{L_{\infty}/u_{\infty}}
\end{gathered}
\end{equation}
The Euler equations in conserved variables become:
\begin{subequations} \label{eq:euler}
\begin{align}
    \label{eq:euler_continuity}
    \partial_{\tilde{t}} \tilde{\rho} + \underline{\tilde{\nabla}}\cdot(\tilde{\rho} \underline{\tilde{u}}) = 0 \\
    \label{eq:euler_momentum}
    \partial_{\tilde{t}} (\tilde{\rho} \tilde{u}) + \underline{\tilde{\nabla}}\cdot(\tilde{\rho} \underline{\tilde{u}}\,\underline{\tilde{u}}) + \frac{1}{\tilde{M}^2}\underline{\tilde{\nabla}}\,\tilde{p} = 0 \\
    \label{eq:euler_energy}
    \partial_{\tilde{t}} \tilde{\rho} \tilde{e} + \underline{\tilde{\nabla}}\cdot(\tilde{\rho} \underline{\tilde{u}} \tilde{h}) = 0
\end{align}
\end{subequations}
Where $M=u_{\infty}/a_{\infty}$ is the reference Mach number.
Using $M$ as a small parameter, we treat system (\ref{eq:euler}) as a perturbation problem and expand all variables as power series of $M$:
\begin{equation} \label{eq:power_expansion}
    \psi(\underline{x},t,M) =     \psi_0 (\underline{x},t)
                             + M  \psi_1 (\underline{x},t)
                             + M^2\psi_2 (\underline{x},t)
                             + \mathcal{O}(M^3)
\end{equation}
The expansions (\ref{eq:power_expansion}) are inserted into the system (\ref{eq:euler}) and terms are grouped by powers of the parameter $M$.
First we examine the relations from the velocity equations \ref{eq:euler_velocity}, where the $1/M^2$ coefficient on the pressure gradient term means the lowest order terms are $M^{-2}$, $M^{-1}$ and $M^{0}$, leading to the following relations:
\begin{subequations} \label{eq:momentum_convective_timescale}
\begin{align}
    \label{eq:momentum_convective_0}
    \underline{\tilde{\nabla}}\,\tilde{p}_0 = 0 \\
    \label{eq:momentum_convective_1}
    \underline{\tilde{\nabla}}\,\tilde{p}_1 = 0 \\
    \label{eq:momentum_convective_2}
    \partial_{\tilde{t}} (\tilde{\rho} \tilde{u})_0 + \underline{\tilde{\nabla}}\cdot(\tilde{\rho} \underline{\tilde{u}}\,\underline{\tilde{u}})_0 + \underline{\tilde{\nabla}}\,\tilde{p}_2 = 0
\end{align}
\end{subequations}
Relations (\ref{eq:momentum_convective_0}) and (\ref{eq:momentum_convective_1}) imply that both the zeroth and first order terms for the pressure vary only in time, so for the pressure, the expansion (\ref{eq:power_expansion}) can be replaced with:
\begin{equation}
    \tilde{p}(\underline{\tilde{x}},\tilde{t},\tilde{M}) = \tilde{p}_0(\tilde{t}) + \tilde{M}^2\tilde{p}_2(\underline{\tilde{x}},\tilde{t}) + \mathcal{O}(\tilde{M}^3)
\end{equation}
with a spatially uniform background pressure, and spatial variations on the order of $\mathcal{O}(\tilde{M}^2)$, which essentially react only to the dynamic pressure of the leading order momentum in a Bernoulli type manner.
The divergence of the leading order velocity is constrained by:
\begin{equation} \label{eq:divergence_convective_timescale}
    \underline{\tilde{\nabla}}\cdot\underline{\tilde{u}}_0 = \frac{1}{\gamma \tilde{p}_0}d_{\tilde{t}}\tilde{p}_0
\end{equation}
meaning that the divergence is also spatially uniform and reacts everywhere instantaneously to temporal variations in the background pressure.
Acoustics are removed from the solution, leaving only convective effects.
Heuristically, this should not be surprising, given our choice of the convective timescale $L_{\infty}/u_{\infty}$ to non-dimensionalise the time derivatives in equation (\ref{eq:euler}).

Muller uses a two-timescale, one space scale asymptotic analysis to include acoustic effects, defining an additional non-dimensional time, $\tau$, scaled by the acoustic speed:
\begin{equation}
    \tau = \frac{t}{L_{\infty}/a_{\infty}} = \frac{\tilde{t}}{\tilde{M}}
\end{equation}
leading to the power series expansion:
\begin{equation} \label{eq:power_expansion_acoustic}
    \psi(\underline{\tilde{x}},\tilde{t},\tilde{M}) =             \psi_0 (\underline{\tilde{x}},\tilde{t},\tau)
                                                     + \tilde{M}  \psi_1 (\underline{\tilde{x}},\tilde{t},\tau)
                                                     + \tilde{M}^2\psi_2 (\underline{\tilde{x}},\tilde{t},\tau) + \mathcal{O}(\tilde{M}^3)
\end{equation}
The time derivatives at constant $\underline{\tilde{x}}$ and $\tilde{M}$ are now:
\begin{equation}
    \partial_{\tilde{t}} \psi\Big|_{\underline{\tilde{x}},\tilde{M}} = \big( \partial_{\tilde{t}} + \frac{1}{\tilde{M}}\partial_{\tau} \big) \psi
\end{equation}
Similiarly to the convective limit, the main results for the acoustic limit stem from the three lowest order relations from the momentum equation:
\begin{subequations} \label{eq:momentum_acoustic_timescale}
\begin{align}
    \label{eq:momentum_acoustic_0}
    \underline{\tilde{\nabla}}\,\tilde{p}_0 = 0 \\
    \label{eq:momentum_acoustic_1}
    \partial_{\tau}(\tilde{\rho}\tilde{u})_0 + \underline{\tilde{\nabla}}\,\tilde{p}_1 = 0 \\
    \label{eq:momentum_acoustic_2}
    \partial_{\tau}(\tilde{\rho}\tilde{u})_1 + \partial_{\tilde{t}} (\tilde{\rho} \tilde{u})_0 + \underline{\tilde{\nabla}}\cdot(\tilde{\rho} \underline{\tilde{u}}\,\underline{\tilde{u}})_0 + \underline{\tilde{\nabla}}\,\tilde{p}_2 = 0
\end{align}
\end{subequations}
The zeroth order pressure term is again constant in space, however the same is no longer true for the first order pressure term, which now varies with fluctuations on the acoustic timescale of the leading order momentum.
The second order pressure is still the relevant pressure for leading order momentum fluctuations on the convective time scale, but now is also related to the fluctuations of the first order momentum on the acoustic timescale.
With the two-timescale asymptotic expansion, the divergence of the leading order velocity is constrained by:
\begin{equation} \label{eq:divergence_acoustic_timescale}
    \underline{\tilde{\nabla}}\cdot\underline{\tilde{u}}_0 = \frac{1}{\gamma p_0}\big( d_{\tilde{t}}\tilde{p}_0 + \partial_{\tau}\tilde{p}_1 \big)
\end{equation}
and is now allowed to vary spatially through the acoustic-time derivative of the first order pressure. \\

To summarise the important points of this section, we have two asymptotic expansions of the inviscid Euler equations at low Mach number, one including acoustic effects, one not.
At the convective limit:
\begin{itemize}
    \item the non-dimensional pressure $\frac{p}{\rho_{\infty}a^2_{\infty}}$ has second order spatial variations with the dynamic pressure,
    \item the divergence of the leading order velocity is spatially uniform and reacts everywhere instantaneously to changes in the background pressure.
\end{itemize}
At the acoustic limit:
\begin{itemize}
    \item the non-dimensional pressure $\frac{p}{\rho_{\infty}a^2_{\infty}}$ has spatial variations due to the first order acoustic pressure and the second order dynamic pressure.
    \item the divergence of the leading order velocity varies in space with fluctuations on the acoustic timescale of the first order acoustic pressure.
\end{itemize}

Only the most relevant points of the asymptotics analysis have been presented here. For a more complete picture, see Muller99 ** and Muller (habilitation thesis), including the effects of diffusion, conduction and heat addition, and some implications for numerical schemes for these regimes.

\subsection{Asymptotic notation}
Before we continue, we clarify our use of asymptotic notation.
Up until now we have used the traditional definition of big-O ($\mathcal{O}$) as an upper bound, i.e.
\[f(M)\sim\mathcal{O}(M^n) \;\Rightarrow\; |f(M)|<c_f M^n \; \forall \; M<c_0\]
where $c_f$ and $c_0$ are constants that may depend on any flow condition or material property excluding the Mach number, although $c_0$ will usually be somewhere slightly smaller than 0.2, where compressibility effects become small.
For the rest of the paper we are concerned with finding (and designing) limit equations, where only the largest terms remain as $M\to0$, for which a bound is an insufficient metric.
For example, both terms in $f(M) = M^3 + M^2$ are $\mathcal{O}(M^2)$ as $M\to0$ under the definition above, however only $M^2$ remains in the limit equation.
Instead, we will use a stricter definition:
\begin{equation} \label{eq:asymptotic_order}
    f(M) \sim \mathcal{O}(M^n) \; \Rightarrow \; \exists \; c_f \in \mathbb{R}\; \textrm{s.t.}\; \forall \varepsilon>0 \;\exists\; \delta>0 \;\textrm{s.t.}\; | f(M) - c_f M^n |<\varepsilon \; \forall \: M<\delta
\end{equation}
We will also (mis)use the \textit{less-than-approximately} and \textit{greater-than-approximately} symbols to mean asymptotically smaller or larger than.
As we are concerned with the limit as $M\to0$, a higher order of M will be asymptotically smaller than a lower order of M, and vice-versa:
\begin{equation}
    \begin{split}
        g(M) \lesssim \mathcal{O}(M^n) \; & \Rightarrow \; g(M) \sim \mathcal{O}(M^m), \: m>n \\
        h(M) \gtrsim  \mathcal{O}(M^n) \; & \Rightarrow \; h(M) \sim \mathcal{O}(M^l), \: l<n
    \end{split}
\end{equation}
Lastly, where terms are independent of $M$ we use $\mathcal{O}(M^0)$ instead of $\mathcal{O}(1)$, to be clear we do not mean that the order of magnitude is necessarily on the order of unity.

\subsection{Symmetrising variables and artificial diffusion}
In this section we examine the implications of the above asymptotic analysis on the symmetrised Euler equations.
In particular, we are interested in the appropriate scaling of the artificial diffusion of a numerical scheme to achieve accurate solutions at the convective and acoustic limits.
The results for the single timescale convective limit are reproduced from Turkel's Annual Review article on low Mach number preconditioning **. We extend the analysis to the two timescale mixed convective/acoustic limit, as well as estimating the scaling of the spectral radius of the artificial diffusion at each limit. \\
The symmetrising variables are:
\begin{equation} \label{eq:symmetry_variables}
    d\underline{q}^s = ( d\phi, du, dv, d\zeta ),
    \quad
    d\phi = \frac{dp}{\rho a},
    \quad
    d\zeta = \frac{ads}{C_p} = \frac{dp - a^2d\rho}{\rho a}
\end{equation}
where $ds$ is the entropy variation.
The corresponding form of the $x$-split 2D Euler equations, with artificial diffusion matrix $A_{ij}$ is:
\begin{subequations} \label{eq:euler_symmetric}
    \begin{align}
        \label{eq:pressure_symmetric}
        \partial_t \phi + u\partial_x \phi + a\partial_x u & = A_{11}\partial_{xx}\phi + A_{12}\partial_{xx}u \\
        \label{eq:velocity_symmetric}
        \partial_t u    + a\partial_x \phi + u\partial_x u & = A_{21}\partial_{xx}\phi + A_{22}\partial_{xx}u \\
        \label{eq:vorticity_symmetric}
        \partial_t v                       + u\partial_x v & = A_{33}\partial_{xx}v \\
        \label{eq:entropy_symmetric}
        \partial_t s                       + u\partial_x s & = A_{44}\partial_{xx}s
    \end{align}
\end{subequations}
where the form of the artificial diffusion mirrors the form of the physical ($x$-split) flux Jacobian:
\begin{equation} \label{eq:symmetric_jacobian}
\frac{\partial \underline{f}^s}{\partial \underline{q}^s} =
\begin{matrix}
\begin{pmatrix}
u & a & 0 & 0 \\
a & u & 0 & 0 \\
0 & 0 & u & 0 \\
0 & 0 & 0 & u
\end{pmatrix}
\end{matrix}
\end{equation}
The flux Jacobian shows one of the main advantages of the symmetry variables, which is that they decouple the linearly degenerate vorticity and entropy fields from the pressure and normal velocity fields, which constitute the genuinely non-linear acoustic subsystem.
For the majority of this section we will only consider the acoustic subsystem, as this is the distinguishing feature between the low Mach number limits.
We call the normal velocity $du$ the velocity field, as it is the convecting velocity in the $x$-split fluxes, and we call the transverse velocity $dv$ the vorticity field, as it corresponds to the gradient $\frac{\partial v}{\partial x}$.

Turkel ** estimates the required scaling of the coefficients $A^c_{ij}$, $i,j=1,2$ in the convective limit by finding the order of the limit equations, given that the variables in equations (\ref{eq:pressure_symmetric}) and (\ref{eq:velocity_symmetric}) scale as:
\begin{equation} \label{eq:symmetric_convective_scaling}
    d\phi \sim \mathcal{O}(M),
    \quad
    du \sim \mathcal{O}(M^0),
    \quad
    a \sim \mathcal{O}(M^{-1})
\end{equation}
where the symmetric pressure $d\phi=\frac{dp}{\rho a}$ scales one order of $M$ lower than the non-dimensional pressure $\tilde{p}=\frac{p}{\rho a^2}$ used in the previous section.
To properly derive the continuous limit equations, we would also need to state whether we are using the convective ($dt\sim\mathcal{O}(M^0)$) or acoustic ($d\tau\sim\mathcal{O}(M^{1})$) timescale for the time derivative term.
However, as we are concerned with numerical schemes we consider this term a derived value which, for both unsteady and steady solvers, is calculated from the spatial residual and so is always included in the limit equations.
Using the above scaling (\ref{eq:symmetric_convective_scaling}), the largest terms on the left hand side of equations (\ref{eq:pressure_symmetric}) and (\ref{eq:velocity_symmetric}) in the convective limit are $\mathcal{O}(M^{-1})$ and $\mathcal{O}(M^0)$ respectively, so the coefficients $A^c_{11}$ and $A^c_{22}$ should be $\mathcal{O}(M^{-2})$ and $\mathcal{O}(M^0)$ respectively for the diagonal artificial diffusion terms to remain in the limit equations.
By the same argument, the off-diagonal coefficients should be $\mathcal{O}(M^{-1})$\footnote[2]{
    The off-diagonal coefficients can be asymptotically smaller, although in this case they will disappear from the limit equations as $M\to0$ and only the diagonal coefficients will remain.}.
The limit equations for the convective limit are then:
\begin{subequations} \label{eq:convective_limit_equations}
    \begin{align}
        \label{eq:convective_limit_pressure}
        \partial_t \phi                    + a\partial_x u & = A^c_{11}\partial_{xx}\phi + A^c_{12}\partial_{xx}u \\
        \label{eq:convective_limit_velocity}
        \partial_t u    + a\partial_x \phi + u\partial_x u & = A^c_{21}\partial_{xx}\phi + A^c_{22}\partial_{xx}u
    \end{align}
\end{subequations}
With diffusion matrix scaling as:
\begin{equation} \label{eq:convective_diffusion_scaling}
    \underline{\underline{A}}^c \sim \mathcal{O}
    \begin{matrix}
    \begin{pmatrix}
    M^{-2} & M^{-1} \\
    M^{-1} & M^0    \\
    \end{pmatrix}
    \end{matrix}
\end{equation}
The limit equations (\ref{eq:convective_limit_equations}) are almost identical to the governing equations of the Artificial Compressibility Method (**Chorin), which was the basis for much of the early work on low Mach number preconditioning **.
If converged to a steady state, the pressure equations (\ref{eq:convective_limit_pressure}) shows that the velocity divergence will be driven to zero (up to the value of the diffusion terms) as expected in the incompressible limit.

We now repeat to obtain the limit equations and correct scaling of the artificial diffusion in the two timescale limit.
At this limit, there are both convective and acoustic components, with different Mach number scalings.
The convective component scales identically to in the single timescale limit, so will require the artificial diffusion in (\ref{eq:convective_diffusion_scaling}) to be well balanced.
According to the analysis in section **, the scaling for the acoustic component of the two timescale limit is:
\begin{equation} \label{eq:symmetric_acoustic_scaling}
    d\phi \sim \mathcal{O}(M^0),
    \quad
    du \sim \mathcal{O}(M^0),
    \quad
    a \sim \mathcal{O}(M^{-1})
\end{equation}
Resulting in the limit equations:
\begin{subequations} \label{eq:acoustic_limit_equations}
    \begin{align}
        \label{eq:acoustic_limit_pressure}
        \partial_t \phi + a\partial_x u    & = A^a_{11}\partial_{xx}\phi + A^a_{12}\partial_{xx}u \\
        \label{eq:acoustic_limit_velocity}
        \partial_t u    + a\partial_x \phi & = A^a_{21}\partial_{xx}\phi + A^a_{22}\partial_{xx}u
    \end{align}
\end{subequations}
which are the equations governing linear acoustics at low Mach number (see Muller ** for more details).
For the artificial diffusion terms to appear in the limit equations, the coefficients should be:
\begin{equation} \label{eq:acoustic_diffusion_scaling}
    \underline{\underline{A}}^a \sim \mathcal{O}
    \begin{matrix}
    \begin{pmatrix}
    M^{-1} & M^{-1} \\
    M^{-1} & M^{-1} \\
    \end{pmatrix}
    \end{matrix}
\end{equation}
We can see that at the acoustic limit, all coefficients of the artificial diffusion scale at the same rate, which is how standard upwind schemes for compressible flow scale at low Mach number.
The diagonal diffusion coefficients for the convective (\ref{eq:convective_diffusion_scaling}) and acoustic (\ref{eq:acoustic_diffusion_scaling}) components have different orders, which is problematic if we are to design a numerical scheme for the two-timescale limit.
We will show why by working out the limit equations assuming the flow variables scale with the convective component but the artificial diffusion scales with the acoustic component, and vice versa.
For convective flow scaling (\ref{eq:symmetric_convective_scaling}) but acoustic diffusion scaling (\ref{eq:acoustic_diffusion_scaling}), the limit equations are:
\begin{subequations} \label{eq:convective_limit_acoustic_diffusion}
    \begin{align}
        \label{eq:pressure_clim_adiff}
        \partial_t \phi                    + a\partial_x u & =                           A^a_{12}\partial_{xx}u \\
        \label{eq:velocity_clim_adiff}
        \partial_t u                                       & =                           A^a_{22}\partial_{xx}u
    \end{align}
\end{subequations}
While for acoustic flow scaling (\ref{eq:symmetric_acoustic_scaling}) but convective diffusion scaling (\ref{eq:convective_diffusion_scaling}), the limit equations are:
\begin{subequations} \label{eq:acoustic_limit_convective_diffusion}
    \begin{align}
        \label{eq:pressure_alim_cdiff}
        \partial_t \phi                                      = A^c_{11}\partial_{xx}\phi                        \\
        \label{eq:velocity_alim_cdiff}
        \partial_t u    + a\partial_x \phi                   = A^c_{21}\partial_{xx}\phi
    \end{align}
\end{subequations}

The acoustic velocity diffusion $A^a_{22}$ and the convective pressure diffusion $A^c_{11}$ are too large for flows at the opposite limit, and turn the limit equations (\ref{eq:velocity_clim_adiff}) and (\ref{eq:pressure_alim_cdiff}) into Poisson/elliptic type equations (depending on if a steady or unsteady solution is sought).
Overdamping of the velocity field (\ref{eq:velocity_clim_adiff}) is a well-known problem with using conventional compressible schemes at low Mach number.
Overdamping of the pressure field (\ref{eq:pressure_alim_cdiff}) will effectively filter out acoustics from the solution - a desired effect if we want solutions at the convective limit, but very undesirable if we want solutions at the two timescale limit.

On the other hand, the acoustic pressure diffusion $A^a_{11}$ and the convective velocity diffusion $A^c_{22}$ are too small for flows at the opposite limit, and disappear from the limit equations (\ref{eq:pressure_clim_adiff}) and (\ref{eq:velocity_alim_cdiff}), leaving only the off-diagonal artificial diffusion components.
Whilst this may provide enough stabilisation in some circumstances, problems may occur when the diagonal field ($\phi$ in (\ref{eq:pressure_clim_adiff}) and $u$ in (\ref{eq:velocity_alim_cdiff})) is oscillatory but the off-diagonal field ($u$ in (\ref{eq:pressure_clim_adiff}) and $\phi$ in (\ref{eq:velocity_alim_cdiff})) is smooth, in which case not enough diffusion will be generated to prevent instabilities.
In the pathological case that either $A^a_{12},A^c_{21} \lesssim \mathcal{O}(M^{-1})$, then the off-diagonal terms will also disappear from the limit equations, leaving no diffusion on either/both of the convective component of pressure (\ref{eq:pressure_clim_adiff}) or the acoustic component of velocity (\ref{eq:velocity_alim_cdiff}).

\textit{Sentence or two here saying what we want from a numerical scheme for the two-timescale limit: limit equations with all the lhs terms from (\ref{eq:convective_limit_equations}) and (\ref{eq:acoustic_limit_equations}), and non-vanishing diffusion on all field.}\\
To create a diffusion matrix suitable for the two timescale limit, we take the lowest order or either $\underline{\underline{A}}^c$ or $\underline{\underline{A}}^a$ for each component:
\begin{equation} \label{eq:blended_diffusion_scaling}
    \underline{\underline{A}}^b \sim \mathcal{O}
    \begin{matrix}
    \begin{pmatrix}
    M^{-1} & M^{-1} \\
    M^{-1} & M^{0} \\
    \end{pmatrix}
    \end{matrix}
\end{equation}
This will allow both convective and acoustic components of pressure and velocity, albeit at the risk of the convective pressure and the acoustic velocity having non-diagonal or vanishing diffusion.
We shall see later that most modern schemes use this scaling, which we shall refer to as the blended scheme, following Sachdev2012**.
While this scaling is widely used, to the authors knowledge Sachdev et al is one of the only papers to explicitly identify the use of acoustic scaling for the pressure and convective scaling for the velocity. They also numerically demonstrate the issue of vanishing diffusion on the convective pressure and acoustic velocity.

The limit equations for convective flow scaling and blended diffusion scaling are:
\begin{subequations} \label{eq:convective_limit_blended_diffusion}
    \begin{align}
        \label{eq:pressure_clim_bdiff}
        \partial_t \phi                    + a\partial_x u & = \phantom{A^b_{21}\partial_{xx}\phi +\,} A^b_{12}\partial_{xx}u \\
        \label{eq:velocity_clim_bdiff}
        \partial_t u    + a\partial_x \phi + u\partial_x u & = A^b_{21}\partial_{xx}\phi + A^b_{22}\partial_{xx}u
    \end{align}
\end{subequations}
The limit equations for acoustic flow scaling and blended diffusion scaling are:
\begin{subequations} \label{eq:acoustic_limit_blended_diffusion}
    \begin{align}
        \label{eq:pressure_alim_bdiff}
        \partial_t \phi + a\partial_x u    & = A^b_{11}\partial_{xx}\phi + A^b_{12}\partial_{xx}u \\
        \label{eq:velocity_alim_bdiff}
        \partial_t u    + a\partial_x \phi & = A^b_{21}\partial_{xx}\phi                          
    \end{align}
\end{subequations}
From which we see that, compared to the desired limit equations (\ref{eq:convective_limit_equations}) and (\ref{eq:acoustic_limit_equations}), all terms on the left-hand sides are present in (\ref{eq:convective_limit_blended_diffusion}) and (\ref{eq:acoustic_limit_blended_diffusion}), so all relevant flow variations can be admitted in the solution.
As stated above, the main limitations of the blended scheme is the vanishing diagonal diffusion terms on the convective pressure and the acoustic velocity, both of which will be explored with numerical examples later.

\subsubsection{Adaptive schemes}

We have identified three possible diffusion matrices - convective, acoustic, and blended - each suited to different flows.
However, it would be useful to have a single numerical method that could select the most appropriate scaling on the fly.
This would be useful either so the same method can be used for multiple problems, or because different flow regimes exist in a single problem \textit{** example from Sachdev?? Helicopter landing?? **}
The major difference between the different schemes is whether they allow acoustic solutions, so a sensible metric for which scheme to use would be whether acoustic waves can actually be resolved by a simulation.
This can be estimated by the ratio of the acoustic timescale $\tau$ and the simulation timestep $\Delta t$:
\begin{equation} \label{eq:unsteady_mach}
    \frac{\tau}{\Delta t} = \frac{L_{\infty}/a}{\Delta t} = \frac{L_{\infty}/\Delta t}{a} = M_u
\end{equation}
When $\Delta t > \tau$, acoustic waves cannot be temporally resolved, and $M_u$ is small.
When $\Delta t < \tau$, acoustic waves can be resolved, and $M_u$ is large.
In practice, $M_u$ is clamped $M$ and $1$ so $M_u\to M$ as $\Delta t\to0$.
If $A_{11}\sim\mathcal{O}(M^{-1}M_u^{-1})$ then the pressure diffusion will range from the convective scaling if $M_u$ is small to the acoustic scaling if $M_u=1$.
Equivalently, $A_{22}\sim\mathcal{O}(M^{-1}M_u)$ can be used to force the velocity diffusion to scale between the convective and acoustic scaling.

The parameter $\tau/\Delta t$ is normally presented as the unsteady Mach number $M_u$, so called because it can be arranged as a "speed" $L_{\infty}/\Delta t$ divided by $a$.
To the author's knowledge, this parameter was first introduced by Venkateswaran \& Merkle **1995** to improve the conditioning of the inner iterations of a preconditioned dual-time scheme.
The use of $M_u$ in preconditioned schemes is equivalent to its use in both $A_{11}$ and $A_{22}$, so the numerical method varies from the fully convective scheme to the fully acoustic scheme.
Venkateswaran and Merkle later identified that returning fully to the acoustic limit created undesirably high diffusion on the velocity field \textit{**Evaluation of artificial dissipation models and their relationship to the accuracy of Euler and Navier-Stokes computations**}.
Sachdev et al's blended preconditioning is essentially designed to resolve this by using $M_u$ for the pressure diffusion $A_{11}$ but not the velocity diffusion $A_{22}$, therefore their schemes (both preconditioned FDS and AUSM) vary between the fully convective scheme and the blended scheme.

\subsubsection{Spectral radius of low Mach number schemes}
We now estimate the growth of the spectral radii of $\underline{\underline{A}}^c$ and $\underline{\underline{A}}^a$ as $M\to0$.
The maximum spectral radius of the convective and diffusive flux components in a numerical scheme determines the CFL bound required for stability in an explicit timestepping scheme, and also affects the convergence behaviour of implicit schemes through the condition number.
The spectral radius of the physical flux Jacobian, $u+a$, grows as $\mathcal{O}(M^{-1})$ in the limit $M\to0$, so if the spectral radius of the artificial diffusion grows faster than this, it will induce a more stringent stability bound on the numerical scheme.
It can be seen from (\ref{eq:convective_diffusion_scaling}) and (\ref{eq:acoustic_diffusion_scaling}) that the traces of the two coefficient matrices will grow as:
\begin{equation} \label{eq:trace_growths}
    \text{tr}(\underline{\underline{A}}^c) \sim \mathcal{O}(M^{-2}),
    \quad
    \text{tr}(\underline{\underline{A}}^a) \sim \mathcal{O}(M^{-1})
\end{equation}
Given the trace is the sum of the eigenvalues, and is an invariant of the matrix, we can use (\ref{eq:trace_growths}) to estimate the growth of the spectral radii\footnote[2]{
    The trace of a matrix can grow more slowly than the spectral radius due to cancellation.
    For example, the trace of the Euler equations flux Jacobian in $N$ dimensions is $(N+2)u$, which is $\mathcal{O}(1)$, even though the spectral radius is $\mathcal{O}(M^{-1})$.
    However, a well designed diffusion matrix is positive-definite, which makes cancellation impossible.
    Additionally, the trace cannot grow faster than the spectral radius for any matrix.
    As the trace of a diffusion matrix cannot grow asymptotically faster or slower than the spectral radius, it is an appropriate estimate for the scaling of the spectral radius.}.
The trace in the acoustic limit grows at the same rate as the physical spectral radius, so it will not change the stability requirements - up to a constant factor.
On the other hand, the trace in the convective limit grows faster than the physical spectral radius by an order of $M$, so will require a CFL bound which decreases one order faster than the physical CFL bound in order to remain stable as $M\to0$.
This scaling in the convective limit was proven by Birken \& Meister for the specific case when the artificial diffusion is due to a flux Jacobian with low Mach number preconditioning **.
However, our estimate implies that this restriction will hold for any scheme whose diffusion converges to the convective limit as $M\to0$.
Later we will see that this is true for existing AUSM schemes that converge to this limit.\\
\textit{\textcolor{red}{What about the Thornber and Rieper fixes? Both of these proved the pressure field has 2nd order variation, but say the spectral radius was unchanged. Should check this. Gu and Li put these fluxes in almost the form that we put AUSM in. See if they increase the pressure flux or only reduce the velocity flux (pressure diffusion drives spectral radius estimate). NOTE: reducing velocity diffusion prevents anomalous creation of acoustic pressure, increase in pressure diffusion damps existing acoustic pressure. Also could put comment here about "instantaneous" divergence-pressure coupling just means faster than any physical wave, and all other schemes for convective limit solve some implicit problem, why should density based be different?}} \\

Finally, before moving on to the next section, we consider the coefficients $A_{33}$ and $A_{44}$ of the artificial diffusion on the linear fields (\ref{eq:vorticity_symmetric}) and (\ref{eq:entropy_symmetric}).
In both limits, the terms on the left hand side of these equations are $\mathcal{O}(v)$ and $\mathcal{O}(s)$ respectively.
The vorticity field is $\mathcal{O}(M^0)$ in both limits, so the coefficient $A_{33}$ should both be $\mathcal{O}(M^0)$ for the artificial diffusion to be well balanced.
The scaling of the entropy field however, depends on the specific case.
Entropy is generated by viscous effects ($\mathcal{O}(1/Re)$) and heat addition, so without knowing how these scale with, or compare to, the Mach number we cannot state the appropriate scaling for $A_{44}$, other than to assume that $\mathcal{O}(M^0)$ might be a good starting point, purely from examination of the physical flux Jacobian.
\textit{\textcolor{red}{Entropy generation is due to source terms, which are not included in our formulation}}.
For a discussion of the effects of viscosity and heat addition at low Mach number, see the VKI lectures by Muller ** and references therein.