\documentclass{elsarticle}

\begin{document}
    \begin{frontmatter}
        \title{Learning-Based Modeling of Human-Autonomous Vehicle Interaction for Improved Safety in Mixed-Vehicle Platooning Control}
        
        \author[label1]{Jie\ Wang\corref{cor1}}
        \ead{jwangjie@outlook.com}
        \cortext[cor1]{Corresponding author}
        \author[label1]{Yash Vardhan\ Pant}
        \ead{yash.pant@uwaterloo.ca}
        \author[label2]{Zhihao\ Jiang}
        \ead{jiangzhh@shanghaitech.edu.cn}
        
        \affiliation[label1]{organization={Electrical and Computer Engineering Department, University of Waterloo},%Department and Organization
                    % addressline={200 University Avenue West}, 
                    city={Waterloo},
                    % postcode={N2L~3G1}, 
                    state={ON},
                    country={Canada}}
        
        \affiliation[label2]{organization={School of Information Science and Technologies, ShanghaiTech University},%Department and Organization
                    % addressline={393 Middle Huaxia Road}, 
                    % city={Pudong},
                    % postcode={201210}, 
                    state={Shanghai},
                    country={China}}
        
        \begin{abstract}
        Given the increasing integration of autonomous vehicles (AVs) on public roads, it is crucial to develop effective and efficient control strategies for AVs that can accommodate the unpredictable behaviors of human-driven vehicles (HVs). In this research, we introduce a learning-centric methodology to model HVs. The proposed approach combines a first-principles model with a Gaussian process (GP) learning-based component, resulting in improved accuracy of velocity predictions and providing a quantifiable measure of uncertainty. Our novel model is then utilized to develop a GP-based model predictive control (GP-MPC) approach. The goal of this strategy is to augment the safety of mixed vehicle platoons by incorporating  uncertainty assessment into the distance constraints. Simulated studies are used to compare our GP-MPC strategy against a standard model predictive control (MPC) that relies solely on the first-principles model. The results reveal that our GP-MPC approach ensures more consistent safe distancing, promotes efficient travel behavior (including higher travel speeds) within the mixed platoon. Incorporating a sparse GP technique into the modeling of HVs and a dynamic GP prediction approach within the MPC, we have effectively minimized the average computation time for the GP-MPC to just 5\% longer than the standard MPC. This is a drastic improvement from our previous work, being roughly 100 times faster than models that didn't use these approximations. Our research highlights the potential of learning-based modeling of HVs to enhance both safety and efficiency in mixed traffic scenarios that involve AV-HV interactions. 
        \end{abstract}
        
        % %Graphical abstract
        % \begin{graphicalabstract}
        % %\includegraphics{grabs}
        % \includegraphics[width=\columnwidth]{figures/AVs_HVs_platoon.png}
        % \end{graphicalabstract}
        
        %Research highlights
        \begin{highlights}
        \item A Gaussian process (GP) based model corrects predictions and estimates uncertainty.
        \item An uncertainty-aware controller uses GP models to enhance vehicle interaction safety.
        \item The controller is 100x faster than previous methods, ensuring real-time control.
        \end{highlights}
        
        \begin{keyword}
        %% keywords here, in the form: keyword \sep keyword
        Human-Autonomous vehicle interaction \sep Modeling uncertainty \sep Mixed vehicle platoon \sep Gaussian process \sep Model predictive control
        \end{keyword}
    
    \end{frontmatter}

\end{document}

