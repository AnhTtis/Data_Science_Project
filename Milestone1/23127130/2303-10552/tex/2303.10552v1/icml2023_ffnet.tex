%%%%%%%% ICML 2023 EXAMPLE LATEX SUBMISSION FILE %%%%%%%%%%%%%%%%%

\documentclass{article}

% Recommended, but optional, packages for figures and better typesetting:
\usepackage{microtype}
\usepackage{graphicx}
\usepackage{subfigure}
\usepackage{booktabs} % for professional tables
% hyperref makes hyperlinks in the resulting PDF.
% If your build breaks (sometimes temporarily if a hyperlink spans a page)
% please comment out the following usepackage line and replace
% \usepackage{icml2023} with \usepackage[nohyperref]{icml2023} above.
\usepackage{hyperref}


% % Attempt to make hyperref and algorithmic work together better:
% \newcommand{\theHalgorithm}{\arabic{algorithm}}

% Use the following line for the initial blind version submitted for review:
% \usepackage{icml2023}

% If accepted, instead use the following line for the camera-ready submission:
\usepackage[accepted]{icml2023}

% For theorems and such
\usepackage{amsmath}
\usepackage{amssymb}
\usepackage{mathtools}
\usepackage{amsthm}
\usepackage{url}
\usepackage{mathrsfs,algpseudocode}

\usepackage{multirow}
\usepackage{graphicx}
\usepackage{wrapfig}

\usepackage{enumitem}
\setlist[itemize]{leftmargin=*}

% \usepackage{paralist} 
% \let\itemize\compactitem 
% \let\enditemize\endcompactitem 
% \let\enumerate\compactenum 
% \let\endenumerate\endcompactenum 
% \let\description\compactdesc 
% \let\enddescription\endcompactdesc
% \setlength{\pltopsep}{10pt}

% if you use cleveref..
\usepackage[capitalize,noabbrev]{cleveref}
\usepackage{balance}

%%%%%%%%%%%%%%%%%%%%%%%%%%%%%%%%
% THEOREMS
%%%%%%%%%%%%%%%%%%%%%%%%%%%%%%%%
\theoremstyle{plain}
\newtheorem{theorem}{Theorem}[section]
\newtheorem{proposition}[theorem]{Proposition}
\newtheorem{lemma}[theorem]{Lemma}
\newtheorem{corollary}[theorem]{Corollary}
\theoremstyle{definition}
\newtheorem{definition}[theorem]{Definition}
\newtheorem{assumption}[theorem]{Assumption}
\theoremstyle{remark}
\newtheorem{remark}[theorem]{Remark}

% Todonotes is useful during development; simply uncomment the next line
%    and comment out the line below the next line to turn off comments
%\usepackage[disable,textsize=tiny]{todonotes}
\usepackage[textsize=tiny]{todonotes}

\usepackage{setspace}
\renewcommand{\baselinestretch}{0.97} 


% The \icmltitle you define below is probably too long as a header.
% Therefore, a short form for the running title is supplied here:
\icmltitlerunning{Vehicle-Infrastructure Cooperative 3D Object Detection via Feature Flow Prediction}

\begin{document}

\twocolumn[
\icmltitle{Vehicle-Infrastructure Cooperative 3D Object Detection \\ via Feature Flow Prediction}

% It is OKAY to include author information, even for blind
% submissions: the style file will automatically remove it for you
% unless you've provided the [accepted] option to the icml2023
% package.

% List of affiliations: The first argument should be a (short)
% identifier you will use later to specify author affiliations
% Academic affiliations should list Department, University, City, Region, Country
% Industry affiliations should list Company, City, Region, Country

% You can specify symbols, otherwise they are numbered in order.
% Ideally, you should not use this facility. Affiliations will be numbered
% in order of appearance and this is the preferred way.
\icmlsetsymbol{equal}{*}

\begin{icmlauthorlist}
\icmlauthor{Haibao Yu}{lll,ttt}
\icmlauthor{Yingjuan Tang}{bbb,ttt}
\icmlauthor{Enze Xie}{lll}
\icmlauthor{Jilei Mao}{ttt}
\icmlauthor{Jirui Yuan}{ttt}
\icmlauthor{Ping Luo}{lll}
\icmlauthor{Zaiqing Nie}{ttt}
\end{icmlauthorlist}

\icmlaffiliation{lll}{University of Hong Kong}
\icmlaffiliation{ttt}{AIR, Tsinghua University}
\icmlaffiliation{bbb}{Beijing Institute of Technology}

\icmlcorrespondingauthor{Ping Luo}{pluo@cs.hku.hk}
\icmlcorrespondingauthor{Zaiqing Nie}{zaiqing@air.tsinghu.edu.cn}

% You may provide any keywords that you
% find helpful for describing your paper; these are used to populate
% the "keywords" metadata in the PDF but will not be shown in the document
\icmlkeywords{Machine Learning, ICML}

\vskip 0.3in
]

% this must go after the closing bracket ] following \twocolumn[ ...

% This command actually creates the footnote in the first column
% listing the affiliations and the copyright notice.
% The command takes one argument, which is text to display at the start of the footnote.
% The \icmlEqualContribution command is standard text for equal contribution.
% Remove it (just {}) if you do not need this facility.

\printAffiliationsAndNotice{}  % leave blank if no need to mention equal contribution
% \printAffiliationsAndNotice{\icmlEqualContribution} % otherwise use the standard text.

\begin{abstract}
Cooperatively utilizing both ego-vehicle and infrastructure sensor data can significantly enhance autonomous driving perception abilities. However, temporal asynchrony and limited wireless communication in traffic environments can lead to fusion misalignment and impact detection performance. This paper proposes Feature Flow Net (FFNet), a novel cooperative detection framework that uses a feature flow prediction module to address these issues in vehicle-infrastructure cooperative 3D object detection. Rather than transmitting feature maps extracted from still-images, FFNet transmits feature flow, which leverages the temporal coherence of sequential infrastructure frames to predict future features and compensate for asynchrony. Additionally, we introduce a self-supervised approach to enable FFNet to generate feature flow with feature prediction ability. Experimental results demonstrate that our proposed method outperforms existing cooperative detection methods while requiring no more than 1/10 transmission cost of raw data on the DAIR-V2X dataset when temporal asynchrony exceeds 200$ms$. 
The code is available at \href{https://github.com/haibao-yu/FFNet-VIC3D}{https://github.com/haibao-yu/FFNet-VIC3D}.
\end{abstract}

%%%%%%%%% Introduction
%%%%%%%%%%%%%%%%%%%%%%%%%%%%%%%%%%%%%%%%%%%%%%%%%%%%%%%%%%%%%%%%%%%%
%%%%%%%%%%%%%%%%%%%%%%%%%%%%%%%%%%%%%%%%%%%%%%%%%%%%%%%%%%%%%%%%%%%%
%-------------------------------------------------------------------------
%% the only exception to this rule is the \firstsection command
% \firstsection{Introduction}

\section{Introduction}
\label{sec:Intro}

Glyph-based visualization is a family of widely used techniques, which are often integrated with other families of visualization techniques, such as small multiples in geo-spatial visualization, multivariate vertices in network visualization, directional and multivariate feature depiction in volume, vector, and tensor field visualization, and dynamic characteristics of objects in event and video visualization. While there have been proposals and discourses on desirable properties of glyph designs in the literature (e.g., \cite{Bertin:2011:book,Maguire:2012:TVCG,Borgo:2013:STAR,chung2015glyph}), there is not yet a coherent methodology that visualization designers can use, consistently and methodically, in evaluating different design options in a process for designing and developing a visualization solution.

In glyph design processes, a designer may face many challenges (e.g., knowledge about the data, users, tasks, cognitive theories and experimental findings related to glyph-based visualization, and so on). This work focuses on one particular challenge, that is, there are many desirable properties of glyph designs, and likely a good design does not necessarily meet all criteria as one might desire, but embodies a relatively optimized set of trade-offs among the visual representations of different variables. In particular, we propose a methodology for evaluating different design options based on multiple-criteria decision analysis (MCDA) \cite{Ishizaka:2013:book,Azzabi:2020:book}, which is an established and widely adopted methodology in management science for evaluating multiple complementary and conflicting criteria explicitly in decision making. Its applications include business, governance, medicine, and engineering. 

We aim to introduce MCDA to glyph design as a systematic and cost-effective methodology and to bring together different desirable properties proposed in the literature into a typology of rateable criteria.
Attention has been paid to (1) providing a good coverage of all proposed criteria for static glyph designs, (2) defining each criterion to facilitate clear interpretation and unambiguous ratings, (3) reducing the overlapping among criteria, (4) enabling distinct considerations of conflicting criteria, and (5) recommending a weighted scoring mechanism that balances between overview vs. detail and precision vs. cost.

\revise{The proposed methodology is built on a comprehensive study of the literature on glyph-based visualization as summarized in Section \ref{sec:RelatedWork}. In particular, the 12 criteria in the proposed MCDA scheme are carefully selected following an in-depth analysis of four existing sets of criteria in the literature \cite{Bertin:2011:book,Maguire:2012:TVCG,Borgo:2013:STAR,chung2015glyph}, which is detailed in Appendix \ref{apx:FrameworkDesign} in the supplementary materials. The analysis enabled us to identify the overlapping or partially overlapping criteria as well as a few missing criteria, while allowing us to gain an appreciation about the scientific reasons behind their categorical arrangement of different criteria. The main contributions of this work include:
%
\begin{itemize}
    %\vspace{-1mm} 
    \item The introduction of the MCDA methodology in glyph design processes for evaluating glyph designs systematically;
    % \vspace{-1mm} 
    \item The integration of four existing sets of criteria for evaluating glyph designs, and the identification and addition of a few criteria absent in the existing sets;%
    \vspace{-1mm}
    \item The definition of five scales for rating each of the 12 criteria, featuring quantitative and meticulous specifications;%
    % \vspace{-1mm}
    \item The testing and demonstration of the proposed MCDA scheme through \reviseTVCG{scoring examples} where the scheme was applied to several existing glyph designs in the literature and their parody versions as well as several new designs in a biomechanical application.% 
\end{itemize}}

The proposed methodology represents a major step forward from the four existing sets of criteria. The definitions in the scheme are expected to be improved through the experience gained in its uses in practice as well as new findings in theoretical and empirical research.
Furthermore, the proposed methodology is not a replacement for user-centered requirements analysis and evaluation, but it can complement, strengthen, and reduce the frequencies of user-centered studies in individual design processes.
We will discuss these further in Section \ref{sec:Conclusions}.
% In particular, we anticipate that the rating schemes for a number of criteria will be enriched when more measurements and models in visualization psychology are becoming available.   
%





%%%%%%%% Related Work
%%%%%%%%%%%%%%%%%%%%%%%%%%%%%%%%%%%%%%%%%%%%%%%%%%%%%%%%%%%%%%%%%%%
%%%%%%%%%%%%%%%%%%%%%%%%%%%%%%%%%%%%%%%%%%%%%%%%%%%%%%%%%%%%%%%%%%%
\section{Related Work}
\label{sec:RalatedWork}
%
Glyph visualization has been applied in many fields, such as
biology, biomedical and medical research \cite{Maguire:2012:TVCG,somarakis2019imacyte,muller2014analysis,meuschke2017glyph,raidou2018bladder,lichtenberg2017concentric,kalamaras2022graph,khawatmi2021shapography,oeltze2008glyph,meyer2008glyph},
meteorology and environmental studies \cite{martin2008results,pilar2013representing,pfeiffer2021glyph,drocourt2011temporal,sanyal2010noodles},
human behavioral analysis \cite{el2016contovi,kovacevic2020glyph},
web and database searching \cite{chau2011visualizing,siva2014evaluation},
sports \cite{legg2016glyph,polk2014tennivis,wang2021tac,wu2021tacticflow,cava2013glyphs,parry2011hierarchical},
music and multimedia \cite{chan2009visualizing,lind2022visualizing,janicke2010soundriver,botchen2008action},
and business and industrial applications \cite{rees2020agentvis,surtola2005effect,suntinger2008event}.
A glyph object conveys multi-dimensional data in a concise way, which significantly reduces the perceptual and cognitive load for information comprehension. It has been widely adopted in applications that require a simultaneous view of multiple variables, including
the facilitation of visual search \cite{healey1999large,cai2015applying},
data comparison \cite{verma2004comparative,zhang2015glyph,meuschke2017glyph,koc2022peaglyph},
data ordering \cite{Chung:2016:CGF,miller2019evaluating},
and feature extraction \cite{keck2017towards}.
Glyphs have also been used in combination with spatial or temporal displays \cite{tominski20053d,ropinski2007surface,drocourt2011temporal,bleisch2017exploring,Legg:2012:CGF}
and other visualization methods \cite{lichtenberg2017concentric,kammer2020glyphboard,fernstad2020explore,borgo2012empirical}.
%
Algorithms for glyph placement \cite{ward2002taxonomy,lie2009critical,streeb2018design,rees2020agentvis,tong2016glyphlens,mcnabb2019multivariate,hlawitschka2007interactive}
and three-dimensional display \cite{lie2009critical,tong2016glyphlens,stevens2016hairy} were created to ensure effective presentation of glyph objects. Some studies developed techniques for specialized scenarios or requirements, such as
temporal summarization \cite{Duffy:2015:TVCG,el2016contovi,gerrits2017glyphs,tominski20053d,botchen2008action},
uncertainty depiction \cite{aigner2005planninglines,sanyal2010noodles,ribicic2012sketching,hlawatsch2011flow,wittenbrink1996glyphs},
visualizing data with special structure or inter-relations \cite{rees2020agentvis,dunne2013motif,cayli2013glyphlink,lee2021cluster,soares2020depicting,cao2011dicon,reda2019dynamic,kalamaras2022graph},
and display of trends and gradients of vectors and tensor fields \cite{dovey1995vector,wittenbrink1996glyphs,schultz2010superquadric,tong2016crystal,meuschke2017glyph,zhang2015glyph,gerrits2016glyphs,gerrits2017glyphs,hergl2019visualization,peeters2009fast,peng2011mesh,hashash2003glyph,hlawatsch2011flow}.
The recent appearance of accessible tools for glyph generation \cite{ribarsky1994glyphmaker,xia2018dataink,brehmer2021generative,ying2022metaglyph,cunha2018many} should lead to a broader application of this visualization technique.

With the seemingly endless design space and the wide variety of applications, guidance is required for glyph designers to evaluate and make selections among different designs. The development of a comprehensive list of criteria that allows systematic evaluations of glyph designs could provide designers with a framework to create designs of quality and to compare design options. The most common and straightforward evaluation is done via user studies \cite{aigner2005planninglines,surtola2005effect,weigle2005visualizing,chan2009visualizing,chau2011visualizing,siva2014evaluation,dunne2013motif}. A systematic review conducted by Fuchs et al.\cite{fuchs2016systematic} revealed that users' accuracy scores and task completion times are the most widely adopted measures to evaluate the effectiveness of visualisations. These measurements provide reliable statistical evidence for the evaluation; however, there are limitations. Firstly, it takes significant time and effort to design and carry out studies of users. This is especially true when multiple design options and objectives exist. Secondly, the performance outcomes are often the collective effect of various design factors, and they provide few indications of specific changes which could be made to improve the design. %\sout{The effects of individual factors can be finely evaluated by setting up experiments for each design factor while others are well controlled, which, however, further increases the costs of the evaluations.}

Normative rating is another way to evaluate visualizations, in which design qualities can be individually quantified. Subjective rating is commonly used in user studies to gauge design qualities from user experiences \cite{lee2003empirical,aigner2005planninglines,surtola2005effect,fuchs2013evaluation}. McDougall et al. \cite{mcdougall2000exploring} adopted subjective ratings to characterize cognitive features of icon designs and used the ratings in further analyses to investigate the correlations between the qualities. However, in most cases, the subjective rating is performed without a clear definition of each rating level. The only instruction given was how the two ends of the scale map to the extremities of the target quality, e.g. 1=very unfamiliar; 5=very familiar. The boundaries between the rating levels remained flexible to the raters. This casts doubts about the consistency of the scoring standards. It has also been shown that the users' preferences are not always consistent with the results from the statistical evidence \cite{fuchs2016systematic}.

Attempts have been made to develop standardized measures for qualities of glyph visualization. Garcia et al. \cite{garcia1994development} proposed a metric to evaluate glyph complexity.
Forsythe et al. \cite{forsythe2003measuring} proposed an automatic complexity measuring algorithm, aiming to remove subjective elements from the evaluation of complexity and to  prevent judgment bias. Up to now, however, the number of standardized measures of visual design qualities remains very limited.
%The subjective rating is still considered the method with more comprehensive coverage in quantifying visual design features \cite{ng2008visual}. 
%To extend the coverage, there is a need to identify more qualities of visual designs that the users may consider and provide clear criteria for the rating.

Fortunately, there have been abundant glyph design guidelines proposed in the literature. The guidelines concern various levels of glyph design: variable encoding, inter-channel interaction, and holistic glyph design. At the variable encoding level, Bertin \cite{bertin1983semiology} proposed basic semantic criteria for determining the suitability of channel encoding. Cleveland and McGill \cite{cleveland1984graphical} identified the accuracy of human perception in various visual variables and gave recommendations on the choices of visual channels for different tasks and purposes of visualization. Other considerations include the capacity, orderability, semantic closeness, visual pre-attentiveness, robustness, and normalizability of the channels \cite{lie2009critical,Legg:2012:CGF,ropinski2011survey,ward2008multivariate,chung2015glyph,yousef2001assessment}. At the levels of inter-channel interactions and holistic glyph design, guidelines were proposed for the integration and separability of channels, the balance of attention, searchability, visual hierarchy, and the learnability of glyph designs \cite{karve2007glyph,Maguire:2012:TVCG,lie2009critical,chung2015glyph}. Beyond the design for the glyph object \textit{per se}, advice was provided for the stages of data-mapping and glyph rendering \cite{ropinski2007surface,ward2008multivariate,meyer2008glyph,ropinski2008taxonomy,lie2009critical,ropinski2011survey}. A survey conducted by Borgo et al. \cite{Borgo:2013:STAR} collected glyph design guidelines and criteria from a wide range of previous studies. Consulting the three-stage glyph design framework by Lie et al. \cite{lie2009critical} and the perceptual-based glyph taxonomy by Ropinski et al. \cite{ropinski2008taxonomy,ropinski2011survey}, a set of fourteen design guidelines that aimed to provide comprehensive coverage of glyph-based visualization was proposed.

%It is manifested that although design criteria taken into account vary amongst studies, consensus on some universal criteria started to form. 
More recent studies\cite{Maguire:2012:TVCG,chung2015glyph,Borgo:2013:STAR} have integrated guidelines from various previous studies, and consensus on some universal design criteria has started to form. 
%As the outline of universal criteria becomes clearer and the coverage becomes comprehensive, 
With further development, a normative set of evaluation criteria might be expected to emerge through a process of collecting and comparing existing guidelines and selecting, modifying, or creating definitions of criteria to make them generally applicable to a vast range of glyph applications and data types. To make the criteria set succinct and equally weighted amongst different design aspects, higher orthogonality must be considered. Ideally, to make the evaluations consistent amongst raters, standards should be defined for each and every level of the evaluation criteria.



% \vfill\eject
% -\\
% \newpage


%%%%%%%%% Feature Flow Net
%%%%%%%%%%%%%%%%%%%%%%%%%%%%%%%%%%%%%%%%%%%%%%%%%%%%%%%%%%%%%%%%%%%%
%%%%%%%%%%%%%%%%%%%%%%%%%%%%%%%%%%%%%%%%%%%%%%%%%%%%%%%%%%%%%%%%%%%%
\section{Method}
\begin{figure*}[ht]
  \centering
  \includegraphics[width=1\linewidth]{images/FFNet-OVERVIEW.pdf}
  \vspace{-10pt}
  \caption{FFNet Overview.
  In the infrastructure system, the feature extractor and the first-order derivative generator make up the feature flow generator. 
  The feature flow is linearly represented with a feature and a first-order derivative as shown in Equation~\ref{eq: first-order of feature flow}, and can be used to predict the future feature.
  The compressors and decompressors focus on reducing the transmission cost while reserving valuable information for cooperative detection. Detailed configurations for FFNet can be found in the Appendix.
  }\label{fig: FFNet}
\end{figure*}

This paper aims to improve vehicle-infrastructure cooperative 3D (VIC3D) object detection while considering practical communication conditions to reduce transmission costs and overcome the temporal asynchrony challenge.
To achieve this goal, we first apply V2VNet~\cite{wang2020v2vnet} for cooperative detection, which is originally used to transmit intermediate features for multi-vehicle perception. We provide details on this basic solution in the Appendix. 
However, this approach ignores temporal asynchrony and may introduce serious fusion misalignment.
To address these issues, we propose a novel intermediate-level framework called Feature Flow Net (FFNet), which includes a feature flow prediction function that generates aligned infrastructure features with vehicle features to compensate for the temporal asynchrony.
Moreover, we introduce a self-supervised approach to train the feature flow generator.
In the following sections, we describe the VIC3D problem in Section~\ref{sec: vic3d detection task}, detail the inference of our solution in Section~\ref{sec: FFNet}, and explain the training in Section~\ref{sec: semi-supervised learning}.
We also compare different possible solutions in the Appendix.

\subsection{VIC3D Object Detection}\label{sec: vic3d detection task}
The VIC3D object detection aims to improve the performance of localizing and recognizing the surrounding objects by utilizing both the infrastructure and vehicle sensor data under limited communication conditions. 
This paper focuses on point clouds captured from LiDAR as inputs. 
The input of VIC3D consists of two parts:
\begin{itemize}[itemsep=2pt,topsep=0pt,parsep=0pt]
    \item Point cloud $P_{v}(t_{v})$ captured by the vehicle sensor with timestamp $t_{v}$ as well as its relative pose $M_{v}(t_{v})$, where $P_{v}(\cdot)$ denotes the capturing function of vehicle LiDAR.
    \item Point cloud $P_{i}(t_{i})$ captured by the infrastructure sensor with timestamp $t_{i}$ as well as its relative pose $M_{i}(t_{i})$, where $P_{i}(\cdot)$ denotes the capturing function of infrastructure LiDAR. Previous frames captured by the infrastructure sensor can also be utilized in cooperative detection.
\end{itemize}
Note that the timestamp $t_i$ should be earlier than timestamp $t_v$ since receiving the data through long-range communication from infrastructure devices to vehicle devices requires a significant amount of transmission time. 
Moreover, the latency $(t_v-t_i)$ should be uncertain before receiving the data, as the transmitted data could be obtained by various autonomous driving vehicles in different locations after infrastructure broadcasting.
More illustration of the uncertain latency is provided in the Appendix.

Compared with 3D object detection in single-vehicle autonomous driving, the VIC3D faces more challenges in temporal asynchrony and transmission costs. 
Directly fusing the infrastructure data can cause serious fusion errors and affect the detection performance due to the scene changing and dynamic objects moving. 
Asynchronous phenomena are also demonstrated in the experiment results in Section~\ref{sec: ablation study}. 
Additionally, reducing the amount of transmitted data could effectively shorten the overall latency because the transmission time is positively correlated with the amount of the transmitted data~\cite{fano1961transmission}. 
Therefore, in addition to the detection performance, the VIC3D has another goal: to reduce the transmission cost. 
Here, mean Average Precision (mAP) is used to measure the detection performance, while the Average Byte ($\mathcal{AB}$) is used to measure the transmission cost. 
We detail the two metrics in the Appendix.

\subsection{Feature Flow Net}\label{sec: FFNet}
Our Feature Flow Net (FFNet) provides a new intermediate-level solution for VIC3D. 
This approach transmits the compressed feature flow for cooperative detection, which reduces redundant information in raw data and eliminates the fusion misalignment caused by temporal asynchrony. 
The inference process of FFNet consists of following stages: (1) generating the feature flow, (2) compressing, transmitting, and decompressing the feature flow, and (3) predicting the aligned feature and fusing it with the vehicle feature to generate the detection results. The inference process is illustrated in Figure~\ref{fig: FFNet}.

\paragraph{Feature Flow Concept for VIC3D.}
The concept of flow is based on mathematical models that formalize the idea of the motion of points over time~\cite{deville2012mathematical}. 
Feature flow is a variant that characterizes the intermediate feature changes over time. 
In this paper, we adopt the feature flow as a prediction function to describe the infrastructure feature changes over time in the future. 
Given the current point cloud frame $P_i(t_i)$ and the infrastructure feature extractor $F_i(\cdot)$, the feature flow over the future time $t$ after $t_i$ is defined as:
\begin{equation}
\setlength\abovedisplayskip{0.05cm}
\setlength\belowdisplayskip{0.15cm}
\widetilde{F}_i(t) =
F_i(P_i(t)), t \geq t_i
\end{equation}

After an uncertain transmitting time, some autonomous driving vehicles receive the feature flow. 
Compared with the previous approach of transmitting feature $F_i(P_i(t_i))$ produced from per frames~\cite{wang2020v2vnet}, which lacks temporal and predictive information, feature flow enables the direct prediction of the aligned feature at the timestamp $t_v$ of the vehicle sensor data. 
We fuse this predicted infrastructure feature $\widetilde{F}_i(t_v)$ with the vehicle feature to compensate for the asynchronous time and eliminate the fusion error caused by temporal asynchrony. 
Note that we generate the feature flow to obtain temporal prediction ability on the infrastructure side rather than the vehicle side. 
This design is because we can extract richer temporal correlation from raw sequential frames than from information-reduction features. 
We also implement the comparison of the extraction on different sides in the Appendix. 
We will introduce how to implement the feature flow in the next part.

\paragraph{Feature Flow Generation.}
Two issues need to be addressed in order to apply the feature flow to transmission and cooperative detection: expressing and transmitting the continuous feature flow changes over time, and enabling the feature flow with prediction ability. 
Considering that the time interval $t_v\rightarrow t_i$ is generally short, we address the expressing issue by using the simplest first-order expansion to represent the continuous feature flow over time, which takes the form of Equation~\eqref{eq: first-order of feature flow}, 
\begin{equation}\label{eq: first-order of feature flow}
    \setlength\abovedisplayskip{0.05cm}
    \setlength\belowdisplayskip{0.15cm}
    \widetilde{F}_i(t_i+\Delta t) \approx F_{i}(P_i(t_i)) + \Delta t * \widetilde{F}_i^{'}(t_i),
\end{equation}
where $\widetilde{F}i^{'}(t_i)$ denotes the first-order derivative of the feature flow and $\Delta t$ denotes a short time period in the future. 
Thus, we only need to obtain the feature $F{i}(P_i(t_i))$ and the first-order derivative of the feature flow $\widetilde{F}_i^{'}(t_i)$ to obtain an approximate representation of the feature flow.
When an autonomous driving vehicle receives the feature $F_{i}(P_i(t_i))$ and the first-order derivative $\widetilde{F}_i^{'}(t_i)$ after an uncertain latency, we can generate the infrastructure feature aligned with the vehicle sensor data, and the prediction operation consumes minor computation because it only needs linear calculation. 
To enable the feature flow with prediction ability, we use a network to extract the first-order derivative of the feature flow $\widetilde{F}_i^{'}(t_i)$ from the historical infrastructure frames ${I_i(t_i-N+1), \cdots, I_i(t_i-1), I_i(t_i)}$. 
Generally, the larger $N$ will generate more accurate estimations.
In this paper, we take $N$ as two and use two consecutive infrastructure frames $P_i(t_i-1)$ and $P_i(t_i)$.

Specifically, we first use the Pillar Feature Net~\cite{lang2019pointpillars} to convert the two consecutive point clouds into two pseudo-images with a bird-eye view (BEV) and with the size of $[384, 288, 288]$. Then, we concatenate the two BEV pseudo-images into the size of $[768, 288, 288]$, and input the concatenated pseudo-images into a 13-layer Backbone and a 3-layer FPN (Feature Pyramid Network), as in SECOND~\cite{yan2018second}, to generate the estimated first-order derivative $\widetilde{F}_i^{'}(t_i)$ with the size of $[364, 288, 288]$. Finally, with the feature $F_{i}(P_i(t_i))$ and the estimated first-order derivative $\widetilde{F}_i^{'}(t_i)$, we can represent the feature flow with prediction ability. The detailed network configuration is provided in the Appendix.

\paragraph{Compression, Transmission and Decompression.}
The transmission cost of the feature flow in its original form is substantial, amounting to up to 243Mb~\footnote{This transmission involves 384$\times$288$\times$288$\times$2 floating point numbers, which equates to 243Mb.}. 
In order to eliminate redundant information and reduce the transmission cost further, we apply two compressors to the feature $F_{i}(P_i(t_i))$ and the derivative $\widetilde{F}_i^{'}(t_i)$, compressing them from size $[384, 288, 288]$ to $[12, 36, 36]$ using three Conv-Bn-ReLU blocks in each compressor.
The size of the timestamp and calibration file is small and cannot be ignored. 
Therefore, the compressed feature flow only requires 0.12Mb per transmission, which is $1/32$$\times$$1/8$$\times$$1/8$ of the original feature flow. 
We broadcast the compressed feature flow along with the corresponding timestamp and calibration file on the infrastructure side.
Upon receiving the compressed feature flow, the vehicle uses two decompressors, each composed of three Deconv-Bn-ReLU blocks, to decompress the compressed feature and compressed first-order derivatives from size $[384/32, 288/8, 288/8]$ to the original size $[384, 288, 288]$.

\paragraph{Vehicle-Infrastructure Feature Fusion.}
We employ a Pillar Feature Net~\cite{lang2019pointpillars} and a feature extractor to generate the vehicle feature from the vehicle point cloud $P_v(t_v)$, which is captured at timestamp $t_v$. The decompressed feature flow is then utilized to predict the infrastructure feature at timestamp $t_v$, aligned with the vehicle feature, as follows:
\begin{equation}
\setlength\abovedisplayskip{0.05cm}
\setlength\belowdisplayskip{0.15cm}
\widetilde{F}i(t_v) \approx F{i}(P_i(t_i)) + (t_v-t_i) * \widetilde{F}_i^{'}(t_i).
\end{equation}
This linear prediction operation effectively compensates for uncertain latency and requires minimal computation. 
The predicted feature $\widetilde{F}_i(t_v)$ is then transformed into the vehicle coordinate system using the corresponding calibration files. 
The bird's-eye view of the infrastructure and vehicle features are obtained, both at the vehicle coordinate system, while preserving spatial alignment. 
The feature located outside the vehicle's interest area is discarded for the infrastructure feature, and empty locations are padded with zero elements.

Subsequently, we concatenate the infrastructure feature with the vehicle feature and employ a Conv-Bn-Relu block to fuse the concatenated features to obtain the vehicle-infrastructure fused feature. 
Finally, we input the fused feature into a 3D detection head, utilizing the Single Shot Detector (SSD)~\cite{liu2016ssd} setup as the 3D object detection head, to generate 3D outputs for more accurate localization and recognition. 
The experimental results indicate that the infrastructure feature flow significantly enhances the detection ability.

\subsection{Training Feature Flow Net}\label{sec: semi-supervised learning}
The FFNet training consists of two stages: training a basic fusion framework in an end-to-end way and then using a self-supervised learning to train the feature flow generator.

\paragraph{Training Basic Fusion Framework.}
In the first stage, we train a basic fusion framework in an end-to-end manner without considering latency. 
The objective of this stage is to train FFNet to fuse the infrastructure feature with the vehicle feature to enhance detection performance. 
Specifically, we train FFNet using cooperative data and annotations obtained from both the vehicle and the infrastructure. 
The localization regression and object classification loss functions used in SECOND~\cite{yan2018second} are applied in this stage. 
To speed up training, we mask the first-order derivative generation and use only one infrastructure point cloud $P_i(t_i)$. 
At the end of this stage, all the modules in FFNet are trained except for the first-order derivative generator.

\paragraph{Training Feature Flow Generator.}
\begin{figure}
  \centering
  \includegraphics[width=0.45\textwidth]{images/training-ffgenerator.pdf}
  \vspace{-5pt}
  \caption{Illustration of training a feature flow generator using self-supervised learning and similarity loss. The upper red circle represents the first-order derivative generator, while the lower purple circles with solid and dashed lines share the same infrastructure feature extractor.
  Feature flow predicts the feature at $t_i+k$.
  }\label{fig: ffgenerator-training}
\end{figure}
In the second stage, we use self-supervised learning to train the feature flow generator by exploiting the temporal correlations in infrastructure sequences, as shown in Figure~\ref{fig: ffgenerator-training}.
The idea is to generate the feature flow, which contains information about the movement of objects over time, and then use it to predict the feature at a future time. 
We construct the ground truth feature by using nearby infrastructure frames that do not require any manual annotations. 
Specifically, we generate training frame pairs $\mathcal{D}={d_{t_i,k}=(P_i(t_i-1), P_i(t_i), P_i(t_i+k))}$, where $P_i(t_i-1)$ and $P_i(t_i)$ are two consecutive infrastructure point cloud frames, and $P_i(t_i+k)$ is the $(k+1)$-th frame after $P_i(t_i)$. 
For each pair $d_{t_i,k}$ in $\mathcal{D}$:
\vspace{-0.1cm}
\begin{itemize}[itemsep=2pt,topsep=0pt,parsep=0pt]
    \item we input $P_i(t_i-1)$ and $P_i(t_i)$ into the feature flow generator to generate the feature flow, which is composed of $F_{i}(P_i(t_i))$ and the estimated first-order derivative $\widetilde{F}_i^{'}(t_i)$. 
    \item We then use the feature flow to predict the feature at time $t_i+k$ as
    \begin{equation}
        \setlength\abovedisplayskip{0.05cm}
        \setlength\belowdisplayskip{0.15cm}
        \widetilde{F}_{i}(t_i+k) \approx F_{i}(P_i(t_i)) + |(t_i+k)-t_i| * \widetilde{F}_i^{'}(t_i).
    \end{equation}
    \item We use the infrastructure feature extractor $F_i(\cdot)$ to extract the feature $F_i(P_i(t_i+k))$ from $P_i(t_i+k)$ as the ground truth feature to supervise the feature flow prediction.
\end{itemize}

We construct the loss function to optimize the feature flow generator. 
The objective is to generate the feature flow and use it to predict $\widetilde{F}_i(t_i+k)$ as close as possible to $F_i(P_i(t_i+k))$. 
We use the cosine similarity to measure the similarity between the predicted feature and the ground truth feature as
\begin{equation}
    \setlength\abovedisplayskip{0.05cm}
    \setlength\belowdisplayskip{0.15cm}
    similarity = \frac{\widetilde{F}_i(t_i+k) \odot F_i(P_i(t_i+k))}{||\widetilde{F}_i(t_i+k)||_2*||F_i(P_i(t_i+k))||_2},
\end{equation}
where $\odot$ denotes the inner product, $*$ denotes the scalar multiplication, and $||\cdot||_2$ denotes the L2 norm. 
We use this similarity as the loss function to train the feature flow generator as
\begin{equation}\label{Eq: loss}
    \setlength\abovedisplayskip{0.05cm}
    \setlength\belowdisplayskip{0.15cm}
    \mathcal{L}(\mathcal{D},\theta) = \sum_{d_{t_i,k}\in \mathcal{D}} (1-\frac{\widetilde{F}_i(t_i+k) \odot F_i(P_i(t_i+k))}{||\widetilde{F}_i(t_i+k)||_2*||F_i(P_i(t_i+k))||_2}),
\end{equation}
where $\theta$ is the parameter of the feature flow generator, and we only update the parameters in first-order derivative generator $\widetilde{F}_i^{'}(\cdot)$ and frozen other parameters.
To train the feature flow generator, we use the Adam optimizer~\cite{kingma2014adam} to minimize the loss function with the batch size of 2, the learning rate of 0.001, and the total epoch of 10. 

\paragraph{Remark.} 
It is important to note that cosine similarity is unaffected by the magnitudes of the input tensors. For instance, the input tensors $[1, 2, 3]$ and $[2, 4, 6]$ achieve a maximal cosine similarity value of $1$, even though they have different values. 
Therefore, to adjust the magnitudes of $\widetilde{F}_i(t_i+k)$, we require a scale transformation. We propose using $||F_i(P_i(t_i))||_1 / ||\widetilde{F}_i(t_i+k)||_1$ as the scaling factor, where $||\cdot||_1$ denotes the L1 norm.

%%%%%%%%% Experiments
%%%%%%%%%%%%%%%%%%%%%%%%%%%%%%%%%%%%%%%%%%%%%%%%%%%%%%%%%%%%%%%%%%%%
%%%%%%%%%%%%%%%%%%%%%%%%%%%%%%%%%%%%%%%%%%%%%%%%%%%%%%%%%%%%%%%%%%%%
\section{Experiments}\label{sec: experiments}
\begin{table*}[ht]
\small
\centering
\caption{\textbf{VIC3D Object Detection Results with FFNet and Different Fusion Methods.} $\mathcal{AB}$ denotes the average byte used to measure the transmission cost. "/" denotes no latency for non-fusion methods. "-" denotes no information provided. FFNet outperforms non-fusion methods by up to 10.90\% mAP@BEV (IoU=0.5) and 10.96\% mAP@BEV (IoU=0.5) in 100$ms$ and 200$ms$ latency, respectively. And FFNet outperforms all other middle-fusion methods including DiscoNet and V2VNet while requiring the same transmission cost. FFNet also outperforms all other fusion methods when the latency reaches 200$ms$. Notably, FFNet achieves more than 2\% mAP@BEV (IoU=0.5) improvement over early fusion in 200$ms$ latency while requiring no more than $1/10$ transmission cost.
}\label{tab: experiment performance comparison}
\scalebox{1.0}{
\begin{tabular}{ccc|lcllc}
\hline
\hline
\multirow{2}{*}{Model} & \multirow{2}{*}{FusionType} & \multirow{2}{*}{Latency ($ms$)} & \multicolumn{2}{c}{mAP@3D~$\uparrow$}                & \multicolumn{2}{c}{mAP@BEV~$\uparrow$} & \multirow{2}{*}{$\mathcal{AB}$~(Byte)~$\downarrow$} \\ \cline{4-7} 
&   &  & IoU=0.5 & \multicolumn{1}{l}{IoU=0.7} & IoU=0.5    & IoU=0.7    &    \\
\hline \hline
PointPillars~\cite{lang2019pointpillars}  & non-fusion & /              & 48.06   & -                   & 52.24      & -      & 0    \\
\hline \hline
Early Fusion  & early & 100              & 57.35   & -                   & 64.06      & -      & 1.4$\times 10^6$    \\
TCLF~\cite{yu2022dairv2x}     & late     & 100              & 40.79   & -                   & 46.80      & -      & 5.4$\times 10^2$     \\
DiscoNet~\cite{li2021learning}  & middle    & 100              & 52.83   & 29.19                  & 61.25      & 50.11      & 1.2$\times 10^5$   \\
V2VNet~\cite{wang2020v2vnet}    & middle    & 100              & 52.02   & 28.54                   &  60.78     & 50.02      & 1.2$\times 10^5$   \\
\textbf{FFNet (Ours)}   & middle & 100   & 55.48   & 31.50                    & \textbf{63.14} (\textcolor{red}{+10.90})     & 54.28      & 1.2$\times 10^5$   
\\ \hline \hline                       
Early Fusion & early & 200              & 54.63   & -                   & 61.08      & -      & 1.4$\times 10^6$     \\
TCLF~\cite{yu2022dairv2x}     & late     & 200              & 36.72   & -                   & 41.67      & -      & 5.1$\times 10^2$     \\
DiscoNet~\cite{li2021learning}  & middle    & 200              & 50.76   & 28.57                  & 58.20      & 48.90      & 1.2$\times 10^5$   \\
V2VNet~\cite{wang2020v2vnet}    & middle    & 200              & 49.67   & 26.96                   & 56.02      & 46.32      & 1.2$\times 10^5$   \\
\textbf{FFNet (Ours)}   & middle & 200 & 55.37   & 31.66                   & \textbf{63.20} (\textcolor{red}{+10.96})      & 54.69      & 1.2$\times 10^5$       \\ \hline \hline
\end{tabular}
}
\label{table_xiaorong}
\end{table*}

In this section, we present our approach to solve VIC3D on the DAIR-V2X dataset~\cite{yu2022dairv2x} using the FFNet and various fusion methods. We compare the experimental results of different fusion methods on various latency. Our proposed FFNet outperforms all other methods, including early fusion, when the latency reaches 200$ms$, while only requiring a transmission cost of no more than 1/10 of that of early fusion.
Furthermore, we investigate how the FFNet overcomes the challenge of temporal asynchrony with our proposed feature flow. Our experimental results show that temporal asynchrony can significantly reduce the performance of the cooperative detection model, but the feature flow can effectively compensate for this drop.
Additionally, we evaluate the robustness of the FFNet to uncertain latency by testing it on different latency. The experimental results show that our FFNet can robustly solve the problem of uncertain latency.
In the Appendix, we compare the performance of the feature flow extraction on different sides, i.e., the infrastructure side and the ego vehicle side. We develop our models using MMDetection3D~\cite{mmdet3d2020}.

\subsection{DAIR-V2X Dataset}
DAIR-V2X~\cite{yu2022dairv2x} is a large-scale vehicle-infrastructure cooperative perception dataset consisting of real-world images and point clouds.
With over 100 scenes and 18,000 data pairs, this dataset captures the infrastructure and vehicle sensors simultaneously at an equipped intersection when an autonomous driving vehicle passes through. 
All frames in this dataset are annotated with 10 3D object classes. 
Furthermore, cooperative 3D annotations with vehicle-infrastructure cooperative view are provided for 9,311 pairs, where each object is labeled with its corresponding category (\textit{Car}, \textit{Bus}, \textit{Truck}, or \textit{Van}). 
The dataset is divided into \textit{train/val/test} sets in a 5:2:3 ratio, with all models evaluated on the \textit{val} set.

\subsection{Comparison with Different Fusion Methods}\label{sec: exp-FFNet}
We compare our proposed FFNet with four different fusion methods: non-fusion (e.g., PointPillars~\cite{lang2019pointpillars}), early fusion, late fusion (e.g., TCLF~\cite{yu2022dairv2x}), middle fusion (e.g., DiscoNet~\cite{li2021learning}) and V2VNet~\cite{li2022v2x}. 
The performance of FFNet and these fusion methods are evaluated under 100$ms$ and 200$ms$ latency, respectively, on the DAIR-V2X dataset~\cite{yu2022dairv2x}. 
The detection performance is measured using KITTI~\cite{geiger2012we} evaluation detection metrics: bird-eye view (BEV) mAP and 3D mAP with 0.5 IoU and 0.7 IoU, respectively. Only objects located in the rectangular area [0, -39.12, 100, 39.12] are considered in the metrics, and only the \textit{Car} class is taken into account. 
Implementation details of FFNet and the fusion methods are provided in the Appendix. Evaluation results are presented in Tab.~\ref{tab: experiment performance comparison}.

\paragraph{Result Analysis.}
Firstly, our proposed FFNet outperforms the non-fusion method PointPillars by 10.90\% BEV-mAP (IoU=0.5) and 10.96\% BEV-mAP (IoU=0.5) in 100$ms$ and 200$ms$ latency, respectively. This result indicates that utilizing infrastructure data can improve 3D detection performance.
Secondly, although late fusion requires little transmission cost, the BEV-mAP (IoU=0.5) of TCLF is much lower than that of FFNet, up to 21.53\% in 200$ms$ latency.
Thirdly, compared with early fusion methods, FFNet achieves similar detection performance in 100$ms$ latency and outperforms more than 2\% mAP in 200$ms$ latency, while it only requires no more than 1/10 transmission cost.
Fourthly, our FFNet achieves the best detection performance with the exact transmission cost as the middle fusion methods. For example, FFNet surpasses DiscoNet by 1.89\% BEV-mAP (IoU=0.5) and 5.0\% BEV-mAP (IoU=0.5) in 100$ms$ and 200$ms$ latency, respectively.
In summary, our proposed FFNet achieves new SOTA on DAIR-V2X when the latency reaches 200$ms$, while only requiring no more than 1/10 transmission cost of raw data. Moreover, it not only surpasses all other intermediate-level methods but also surpasses the early fusion.

\begin{table*}
\small
\centering
\caption{\textbf{Comparison between with and without Feature Prediction.} Compared with no prediction models, FFNet with feature prediction has a significantly lower performance drop when there is communication latency.}\label{tab: ablation study of feature prediction}
\scalebox{1.0}{
\begin{tabular}{cc|lcllc}
\hline
\hline
\multirow{2}{*}{Model}  & \multirow{2}{*}{Latency (ms)} & \multicolumn{2}{c}{mAP@3D~$\uparrow$}                & \multicolumn{2}{c}{mAP@BEV~$\uparrow$} & \multirow{2}{*}{AB~(Byte)~$\downarrow$} \\ \cline{3-6} 
 &  & IoU=0.5 & \multicolumn{1}{l}{IoU=0.7} & IoU=0.5    & IoU=0.7    &    \\
\hline \hline
FFNet   & 0              & 55.81   & 30.23                  & 63.54      & 54.16      & 1.2$\times 10^5$    \\
FFNet (without prediction)         & 0              & 55.81   & 30.23                  & 63.54      & 54.16     & 6.2$\times 10^4$     \\
FFNet-V2 (without prediction)      & 0              & 55.78   & 30.22                  & 64.23      & 55.00      & 1.2$\times 10^5$   \\
\hline \hline
FFNet   & 200              & 55.37   & 31.66                   & 63.20 (\textcolor{red}{-0.34})       & 54.69      & 1.2$\times 10^5$    \\
FFNet (without prediction)         & 200              & 50.27   & 27.57                  & 57.93 (\textcolor{red}{-5.61})      & 48.16     & 6.2$\times 10^4$     \\
FFNet-V2 (without prediction)      & 200              & 49.90   & 27.33                  & 58.00 (\textcolor{red}{-6.23})      & 48.22      & 1.2$\times 10^5$   \\
\hline \hline
\end{tabular}
}
\label{table_xiaorong}
\end{table*}

\subsection{Ablation Study}\label{sec: ablation study}
We conducted a series of experiments to demonstrate the effectiveness of the feature flow module in overcoming the temporal asynchrony challenge, as well as to show that FFNet performs robustly under various latencies.

\paragraph{Feature prediction can well solve temporal asynchrony.}
We first evaluated the performance of FFNet under 0$ms$ latency and 200$ms$ latency on DAIR-V2X, where 0$ms$ latency indicates no temporal asynchrony between infrastructure data and vehicle data. 
We then removed the prediction module from FFNet to directly fuse the infrastructure feature without temporal compensation, which we refer to as FFNet (without prediction). We evaluated FFNet (without prediction) under both 0$ms$ and 200$ms$ latency. 
Since FFNet (without prediction) does not require the transmission of the first-order derivative of the feature flow, it only requires half the transmission cost of FFNet. To make a fair comparison, we also trained another version of FFNet, called FFNet-V2, which compressed the feature flow from (384, 288, 288) to (384/16, 288/8, 288/8). Thus, FFNet-V2 (without prediction) has the same transmission cost as FFNet. We evaluated FFNet-V2 (without prediction) under both 0$ms$ and 200$ms$ latency as well. The evaluation results are presented in Table~\ref{tab: ablation study of feature prediction}.

As shown in Table~\ref{tab: ablation study of feature prediction}, FFNet (without prediction) and FFNet-V2 (without prediction) both exhibit a significant performance drop when there is a 200$ms$ latency. For example, FFNet (without prediction) experiences a 5.61\% BEV-mAP (IoU=0.5) drop in 200$ms$ latency compared to 0$ms$ latency. Although FFNet-V2 (without prediction) performs slightly better than FFNet and FFNet (without prediction) in 0$ms$ latency, FFNet significantly surpasses FFNet-V2 (without prediction) in 200$ms$ latency. These results demonstrate that temporal asynchrony can significantly impact performance if we directly fuse the infrastructure feature, and that our feature prediction module can effectively compensate for the performance drop caused by temporal asynchrony.
 
\paragraph{FFNet is robust to uncertain latency.}
\begin{figure}
\centering
\vspace{-0.1cm}
\includegraphics[width=0.40\textwidth]{images/Different_communication_delay.pdf}
\caption{\small Average latency vs. detection performance. FFNet-O denotes the FFNet (without prediction). FFNet-V2-O denotes the FFNet-V2 (without prediction). The yellow curve denotes the evaluation results of the FFNet, which behaves robustly across different latencies compared to models without prediction.
}\label{fig: different latency}
\end{figure}
We conducted further experiments to evaluate the performance of FFNet, FFNet (without prediction), and FFNet-V2 (without prediction) under various latency cases, ranging from 100$ms$ to 500$ms$. 
The experiment results are presented in Figure~\ref{fig: different latency}. 
As shown in the figure, both FFNet (without prediction) and FFNet-V2 (without prediction) exhibit continuous performance drops as the latency increases from 100$ms$ to 500$ms$. 
Specifically, in 500$ms$ latency, FFNet (without prediction) and FFNet-V2 (without prediction) exhibit a significant 9.38\% BEV-mAP (IoU=0.5) drop and 9.76\% BEV-mAP (IoU=0.5) drop, respectively. 
In contrast, FFNet shows little performance degradation within 200$ms$ latency and only has a 4.39\% BEV-mAP (IoU=0.5) drop.
These results indicate that FFNet is robust to different latency and can effectively handle the uncertain latency in VIC3D.

Robustness to uncertain latency is crucial because our feature flow could be received by different vehicles with varying latencies, as illustrated in Figure~\ref{fig: uncertain temporal asynchrony} in the Appendix. The feature flow must have the ability to make predictions at an arbitrary future time before being transmitted.
%%%%%%%%% conclusion
%%%%%%%%%%%%%%%%%%%%%%%%%%%%%%%%%%%%%%%%%%%%%%%%%%%%%%%%%%%%%%%%%%%%
%%%%%%%%%%%%%%%%%%%%%%%%%%%%%%%%%%%%%%%%%%%%%%%%%%%%%%%%%%%%%%%%%%%%
\section{Conclusion}
This paper proposes a novel intermediate-level cooperative framework called FFNet to address the object detection task in VIC3D. 
The proposed FFNet can effectively overcome the temporal asynchrony challenge and reduce the transmission cost by transmitting the compressed feature flow for cooperative detection. Additionally, a novel self-supervised learning approach is designed to train the feature flow generator.
Experimental results on the DAIR-V2X dataset demonstrate that the FFNet outperforms existing state-of-the-art methods. 
Notably, our approach emphasizes improving feature prediction quality and can complement existing intermediate-level methods. 
Furthermore, our approach can be extended to other modalities, including image and multi-modality data. 
The proposed method can also benefit multi-vehicle cooperative perception. 
Moreover, leveraging additional frames to generate feature flow can enhance the feature prediction ability of our method.
In summary, the proposed FFNet with feature prediction and self-supervised learning is a promising approach for VIC3D object detection, and can potentially be applied to various cooperative perception tasks in the future. 

% In the unusual situation where you want a paper to appear in the
% references without citing it in the main text, use \nocite
% \nocite{langley00}
\balance
\bibliography{reference}
\bibliographystyle{icml2023}


%%%%%%%%%%%%%%%%%%%%%%%%%%%%%%%%%%%%%%%%%%%%%%%%%%%%%%%%%%%%%%%%%%%%%%%%%%%%%%%
%%%%%%%%%%%%%%%%%%%%%%%%%%%%%%%%%%%%%%%%%%%%%%%%%%%%%%%%%%%%%%%%%%%%%%%%%%%%%%%
% APPENDIX
%%%%%%%%%%%%%%%%%%%%%%%%%%%%%%%%%%%%%%%%%%%%%%%%%%%%%%%%%%%%%%%%%%%%%%%%%%%%%%%
%%%%%%%%%%%%%%%%%%%%%%%%%%%%%%%%%%%%%%%%%%%%%%%%%%%%%%%%%%%%%%%%%%%%%%%%%%%%%%%
\newpage
\appendix
\onecolumn
\section{More Details about VIC3D Object Detection}
\subsection{Uncertain Temporal Asynchrony}
In this section, we elaborate on the challenge of uncertain temporal asynchrony in the context of the VIC3D object detection task, which was briefly introduced in Section~\ref{sec: vic3d detection task}. In practical traffic environments, autonomous driving vehicles encounter limited communication conditions and varying ranges, resulting in the reception of infrastructure data with varying latencies. This introduces uncertainty in the temporal synchronization of the received data. We present a visualization example in Figure~\ref{fig: uncertain temporal asynchrony} to illustrate this challenge.
This uncertainty in temporal asynchrony is characterized by a significant range, which necessitates a flexible prediction strategy. To address this issue, we propose the extraction of feature flow from sequential infrastructure frames. This feature flow can be utilized to predict future features at any arbitrary future time, thereby providing the necessary flexibility to account for the uncertain temporal asynchrony. The predicted feature can then be used for cooperative detection, which can mitigate the fusion errors due to the uncertain temporal asynchrony.
\begin{figure}[ht]
  \centering
  \includegraphics[width=1\linewidth]{images/v2i_main_challenges.pdf}
  \vspace{-10pt}
  \caption{Uncertain Temporal Asynchrony In Vehicle-Infrastructure Cooperative 3D Object Detection.
  (a) The illustration depicts the utilization of infrastructure sensors to enhance the perception ability of autonomous driving vehicles by providing sensor data from different views. For instance, the infrastructure sensor data can be used to perceive objects located in the long-range zone of the vehicle sensor.
  (b) The vehicle-infrastructure cooperative 3D object detection system encounters uncertainty in communication delay $\Delta t$ occurring before the vehicle devices receive the infrastructure data. Due to limited communication conditions and varying ranges, the infrastructure data received by each autonomous driving vehicle exhibits varying latencies ranging from $\Delta t_1$ to $\Delta t_N$.
  }\label{fig: uncertain temporal asynchrony}
\end{figure}

\subsection{Detailed Evaluation Metrics for VIC3D}\label{sec: evaluation metrics}
This part provides a detailed explanation of the evaluation metrics used to assess the performance of the VIC3D object detection algorithm, namely mean Average Precision (mAP) and Average Byte ($\mathcal{AB}$).

\paragraph{Mean Average Precision (mAP).} We use mAP as a metric to evaluate the detection performance, as commonly used in previous works such as~\cite{yu2022dairv2x, everingham2015pascal}. The AP is calculated based on the 11-points interpolated precision-recall curve and is defined as follows:
\begin{equation}
\begin{aligned}
    AP & = \frac{1}{11} \sum_{r\in {0.0,...,1.0}} AP_{r} \
        & = \frac{1}{11} \sum_{r\in {0.0,...,1.0}} P_{interp}(r),
\end{aligned}
\end{equation}
where $P_{interp}(r) = \mathop{max}\limits_{\tilde{r}\geq r}p(\tilde{r})$,
and a prediction is considered positive if the Intersection over Union (IoU) is greater than or equal to 0.5 or 0.7, respectively. 
Here, we remove objects outside the egocentric surroundings, which we define as a rectangular area with coordinates [0, -39.12, 100, 39.12]. Finally, we average the AP across all classes to obtain the mAP.


\paragraph{Average Byte ($\mathcal{AB}$).}
We use $\mathcal{AB}$ as a metric to evaluate the transmission cost. 
In our implementation, we represent each element in the transmitted data using a 32-bit float, and we ignore the transmission cost of calibration files and timestamps.
We now explain how to calculate the transmission cost for each of the three transmission forms:
\begin{itemize}
    \item Calculating the transmission cost of transmitting raw data for early fusion: We represent each point in the point clouds with $(x,y,z, \text{intensity})$, and each point requires four 32-bit floats, equivalent to 16 Bytes. If there are 100,000 points in the point clouds, the $\mathcal{AB}$ of the transmission cost is $1.6\times 10^6$ Bytes.
    \item Calculating the transmission cost of transmitting detection outputs for late fusion: We represent each 3D detection output with $(x, y, z, w, l, h, \theta, \text{confidence})$, and each detection output requires eight 32-bit floats, equivalent to 32 Bytes. If we transmit ten detection outputs per transmission on average, the $\mathcal{AB}$ of the transmission cost is $3.2\times 10^2$ Bytes.
    \item Calculating the transmission cost of transmitting intermediate data for middle fusion: The feature or feature flow is represented with a tensor, and each element is represented by a 32-bit float. If the size of the feature or feature flow is $(100, 100, 100)$, then the $\mathcal{AB}$ of the transmission cost is $100\times 100\times 100\times 4$ Bytes, which is equivalent to $4\times 10^6$ Bytes.
\end{itemize}

\clearpage
\section{Implementation Details of FFNet}
\paragraph{Architecture.}
In this part, we provide a detailed architecture for FFNet, which is composed of the following six main parts.
1) The feature flow generation module. The infrastructure PFNet (Pillar Feature Net) shares the same architecture as PointPillars~\cite{lang2019pointpillars}.
The x, y, and z ranges of the input point cloud are [(0, 92.16), (-46.08, 46.08), (-3, 1)] meters, respectively.
The voxel sizes for x, y, and z are [0.16, 0.16, 4] meters, respectively.
The output shape of the pseudo-images is (64, 576, 576).
Both the infrastructure feature extractor $F_i(\cdot)$ and the estimated first-order derivative generator $\widetilde{F}_i^{'}(\cdot)$ use the same architecture of Backbone and FPN as SECOND~\cite{yan2018second}, with output shapes of [384, 288, 288].
2) The compressor and decompressor. 
The compressor uses four convolutional blocks with strides (2, 1, 2, 2) to compress the features from (384, 288, 288) to (384/32, 288/8, 288/8).
The decompressor uses three deconvolutional blocks with strides (2, 2, 2) to decompress the features back to the original size.
3) Affine transformation module. 
The affine transformation is implemented with the $affine_grid$ function supported in Pytorch.
We ignore the rotation around the x-y plane.
4) Feature fusion module.
The fusion module is a $3\times3$ convolutional block with stride 1 to compress the concatenated feature from (768, 288, 288) to (384, 288, 288).
5) Vehicle feature extractor. 
This extractor has the same configuration as the infrastructure PFNet and the feature extractor.
6) 3D object detection head.
We use a Single Shot Detector (SSD)~\cite{liu2016ssd} to generate the 3D outputs.
The anchor has a width, length, and height of (1.6, 3.9, 1.56) meters, with a z-center of -1.78 meters.
Positive and negative thresholds of matching are 0.6 and 0.45, respectively.

In this part, we provide a detailed architecture for FFNet, which is composed of following six main parts.
(1) The feature flow generation module: The infrastructure PFNet (Pillar Feature Net) shares the same architecture as PointPillars~\cite{lang2019pointpillars}. 
The x, y, and z ranges of the input point cloud are [(0, 92.16), (-46.08, 46.08), (-3, 1)] meters, respectively.
The voxel size of x, y, and z are [0.16, 0.16, 4] meters, respectively.
The output shape of the pseudo-images is (64, 576, 576).
The feature extractor $F_i(\cdot)$ and the estimated first-order derivative generator $\widetilde{F}_i^{'}(\cdot)$ both use the same Backbone and FPN as SECOND~\cite{yan2018second}, with output shapes of [384, 288, 288].
(2) The compressor and decompressor: The compressor uses four convolutional blocks with strides (2, 1, 2, 2) to compress the features from (384, 288, 288) to (384/32, 288/8, 288/8). The decompressor uses three deconvolutional blocks with strides (2, 2, 2) to decompress the features back to the original size.
(3) Affine transformation module: The affine transform is implemented with the $affine_grid$ function supported in Pytorch. Rotation around the x-y plane is ignored.
(4) Feature fusion module: The fusion module is a $3\times3$ convolutional block with stride 1 to compress the concatenated feature from (768, 288, 288) to (384, 288, 288).
(5) Vehicle feature extractor: This extractor has the same configuration as the infrastructure PFNet and feature extractor.
(6) 3D object detection head: A Single Shot Detector (SSD)~\cite{liu2016ssd} is used to generate the 3D outputs. The anchor has a width, length, and height of (1.6, 3.9, 1.56) meters, with a z-center of -1.78 meters. The positive and negative thresholds of matching are 0.6 and 0.45, respectively.

\paragraph{Training Settings.}
We trained the feature fusion base model on the training set of DAIR-V2X for 40 epochs with a learning rate of 0.001 and weight decay of 0.01. 
We randomly selected 11037 frames from the training set of DAIR-V2X and randomly set $k$ from the range [1, 2] to form $\mathcal{D}$. The explanation of $k$ and $\mathcal{D}$ can be found in Sec.~\ref{sec:semi-supervised-learning}. The trained feature fusion base model was used to pretrain FFNet.
We trained the feature flow generator on $D_u$ for 10 epochs, with a learning rate of 0.001 and weight decay of 0.01. All training and evaluation were implemented on an NVIDIA GeForce RTX 3090 GPU.

\clearpage
\section{Implementation Details of V2VNet and DiscoNet for VIC3D}
\paragraph{V2VNet for VIC3D.}
V2VNet~\cite{wang2020v2vnet} is a pioneering work in multi-vehicle cooperative perception, which is the first to transmitting intermediate-level data for cooperative perception without utilizing sequential frames for extracting temporal correlations. In this paper, we apply this approach to solving the VIC3D problem, as shown in Figure~\ref{fig: V2Vnet}. We made two modifications to the V2VNet architecture: (1) we removed multi-vehicle selection and kept only one vehicle settings as the infrastructure settings, and (2) we compressed the features from (384, 288, 288) to (384, 288/8, 288/8) to keep the transmission cost comparable to FFNet. The other modules keep the same configurations as their corresponding modules in FFNet.
We trained the V2VNet model on the training part of the DAIR-V2X dataset for 40 epochs, with a learning rate of 0.001 and weight decay of 0.01. The other training configurations are the same as those used for training FFNet.

\begin{figure}[ht!]
  \centering
  \includegraphics[width=1\linewidth]{images/V2Vnet.pdf}\label{fig: V2Vnet}
  \caption{\small \textbf{Implementation of V2VNet for Solving VIC3D Problem.} The V2VNet directly transmit the feature generated from the single point cloud, and the fuse it with the vehicle feature. This feature fusion could cause serious fusion error by the temporal asynchrony.}\label{fig: V2Vnet}
\end{figure}

\paragraph{DiscoNet for VIC3D.}
DiscoNet~\cite{li2021learning} was originally developed for cooperative perception among multiple vehicles. 
This method employs a teacher-student paradigm, where cooperative perception with raw data is used as the teacher network to instruct cooperative perception with intermediate data as the student network. To adapt DiscoNet to the VIC3D task, we use an early-fusion network as the teacher network and V2VNet as the student network. We train both the teacher and student models on the training subset of DAIR-V2X for 40 epochs, with a learning rate of 0.001 and weight decay of 0.01. Next, we fine-tune the student network using soft labels generated by the early-fusion network for an additional 10 epochs, with a learning rate of 0.0001 and weight decay of 0.01


\clearpage
\section{Comparison of Feature Flow Extraction on Different Sides}~\label{sec: extraction on different side}
This section discusses the effect of extracting the feature flow on different sides (infrastructure side \textit{vs.} vehicle side).

\paragraph{Experiment Setting.}
To compare the effect of feature flow extraction on infrastructure side and vehicle side, we train a modified FFNet called FFNet-V. The FFNet-V inputs the features $F_i(P_i(t_i-1))$ and $F_i(P_i(t_i-1))$ produced from consecutive infrastructure frames to generate the feature flow on vehicle devices.
We first concatenates the two received features and feeds them into a first-order derivative generator to generate the estimated first-order derivative of the feature flow $\widetilde{F}_i^{'}(t_i)$.
Then we predict the future feature as Equation~\ref{eq: predict F(t_i+k)}.
In addition, FFNet-V shares the same architecture modules and training configuration as FFNet.
The FFNet-V implementation framework is shown in Figure~\ref{fig: FFNet-v}.
To ensure a fair comparison with FFNet, we also train another FFNet-V by compressing the feature from (384, 288, 288) to (384/16, 288/8, 288/8), which has the same transmission cost as FFNet. This version of FFNet-V is referred to as FFNet-V (Same-TC). We evaluate both FFNet-V and FFNet-V (Same-TC) under different latencies (100$ms$, 300$ms$, and 500$ms$).
\begin{figure}[ht!]
  \centering
  \includegraphics[width=1\linewidth]{images/FFNET-OVERVIEW-v.pdf}
  \vspace{-20pt}
  \caption{\small FFNet-V Overview. FFNet-V generates the feature flow on vehicle devices.}
\label{fig: FFNet-v}
\end{figure}

\paragraph{Result Analysis.}
Table~\ref{tab: flow extraction comparison} shows that FFNet-V, FFNet-V (Same-TC), and FFNet outperform FFNet-O under different latencies. This indicates that all these methods can reduce the detection performance drop caused by temporal asynchrony. 
However, FFNet can compensate for more performance drop and achieve better performance than FFNet-V and FFNet-V (Same-TC). 
Specifically, FFNet outperforms FFNet-V (Same-TC) by more than 3\% mAP@BEV (IoU=0.5) in 300$ms$ latency, while having the same transmission cost. The results demonstrate that extracting feature flow from raw sequential frames on infrastructure can improve the VIC3D detection performance more effectively than extracting feature flow from intermediate sequential feature frames on vehicle. 
Moreover, FFNet requires much fewer ego-vehicle computing resources, and the computing cost complexity (CCC) is only O(N), since the feature flow has already been produced on infrastructure devices and does not need to be generated on vehicle devices again. In contrast, FFNet-V consumes computation resources up to O(N) to extract flow from past features. Therefore, FFNet is more computation-friendly to resource-limited vehicle devices. Additionally, extracting feature flow on the vehicle requires much more storage because the feature flow extraction depends on the past frames that the vehicle received. Furthermore, FFNet-V relies heavily on past consecutive frames, so dropped frames can significantly affect the execution and performance. Therefore, FFNet is more storage-friendly to the ego vehicle and more robust to frame dropping.


\clearpage
\begin{table*}[h]
\small
\centering
\caption{\small
\textbf{Extracting Feature Flow on Infrastructure side \textit{vs.} on Vehicle Side.}
FFNet-O denotes the FFNet model without feature prediction.
FFNet-V denotes the model that extracts the feature flow on vehicle.
FFNet-V (Same-TC) denotes the FFNet-V which has the same transmission cost as FFNet.
``AB'' denotes the average byte used to measure the transmission cost. 
``SCC'' indicates the storage cost complexity for the vehicle devices to store the past frames, 
``CCC'' indicates the computing cost complexity for vehicle to extract the feature flow, 
``N'' indicates the number of historical structures to be used.
``/'' indicates the FFNet-O does not need infrastructure transmission and extra computation and storage.
The SCC of FFNet is O(1) because it does not need extra historical frames on vehicle devices. 
At the same time, the SCC of extracting feature flow on vehicle is O(N) because extracting feature flow on vehicle needs past frames received from infrastructure.
Moreover, FFNet achieves better detection performs, and this advantage becomes more pronounced~(+3\% mAP) when latency increases to 300$ms$.}\label{tab: flow extraction comparison}
\scalebox{0.97}{
\begin{tabular}{cc|lcllccc}
\hline
\multirow{2}{*}{Model} & \multirow{2}{*}{Latency (ms)} & \multicolumn{2}{c}{mAP@3D~$\uparrow$}                & \multicolumn{2}{c}{mAP@BEV~$\uparrow$} & \multirow{2}{*}{AB(Byte)~$\downarrow$} & \multirow{2}{*}{SCC~$\downarrow$} & \multirow{2}{*}{CCC~$\downarrow$}  \\ \cline{3-6} 
&    & IoU=0.5 & \multicolumn{1}{l}{IoU=0.7} & IoU=0.5    & IoU=0.7    \\
\hline \hline
FFNet-O            & 100              & 52.18   & 27.99                 & 60.39      & 49.14      & /  & /  & / \\
FFNet-V            & 100              & 53.21   & 28.43                 & 61.50      & 50.50      & 6.2$\times 10^4$  & O(N)  & O(N) \\
FFNet-V (Same-TC)  & 100              & 53.17   & 28.45                 & 62.44      & 51.68      & 1.2$\times 10^5$  & O(N) & O(N) \\
\textbf{FFNet (Ours)}        & 100              & 55.48   & 31.50                   & \textbf{63.14} (\textcolor{red}{+0.7})     & 54.28  & 1.2$\times 10^5$   & O(1) & O(1)
\\ \hline \hline   
FFNet-O            & 300              & 49.03   & 27.39                 & 55.81      & 47.28      & /  & /  & / \\
FFNet-V           & 300              & 50.81   & 28.45                  & 57.75      & 49.62      & 6.2$\times 10^4$  & O(N)  & O(N) \\
FFNet-V (Same-TC) & 300              & 50.5   & 28.25                  & 58.02      & 50.03      & 1.2$\times 10^5$  &  O(N) &  O(N) \\
\textbf{FFNet (Ours)}       & 300              & 53.46   & 30.42                   & \textbf{61.20} (\textcolor{red}{+3.18})    & 52.44  & 1.2$\times 10^5$     & O(1)  & O(1)  \\ \hline \hline
FFNet-O            & 500              & 47.49   & 27.01                 & 54.16      & 45.99      & /  & /  & / \\
FFNet-V            & 500              & 49.93   & 28.63                & 56.42      & 48.87      & 6.2$\times 10^4$ & O(N)  & O(N)  \\
FFNet-V (Same-TC)  & 500              & 49.98   & 27.7                 & 56.99     & 49.55      & 1.2$\times 10^5$ & O(N)  & O(N)  \\
\textbf{FFNet (Ours)}        & 500              & 52.08   & 30.11                   & \textbf{59.13} (\textcolor{red}{+2.14})      & 51.70  & 1.2$\times 10^5$  &  O(1)  &  O(1) \\ \hline \hline
\end{tabular}
}
\label{table_xiaorong}
\end{table*}


\clearpage
\section{Relationship to Other Existing Possible Solutions}\label{sec:relationship-discussion}
Compared with other possible existing solutions, FFNet provides a more applicable paradigm to implement the vehicle-infrastructure cooperative 3D object detection with the following advantages:
\begin{itemize}[itemsep=2pt,topsep=0pt,parsep=0pt]
    \item Better Performance-Bandwidth Balance: FFNet achieves a better performance-bandwidth balance compared with early fusion and late fusion methods. Unlike early fusion, FFNet transmits compressed intermediate data, thereby reducing transmission costs. Moreover, compared with late fusion methods, FFNet transmits more valuable information for the egocentric object detection.
    \item Overcoming Temporal Asynchrony Challenge: FFNet can address the temporal asynchrony challenge between the data captured from vehicle sensors and received from infrastructure sensors. Compared with middle fusion methods like V2VNet~\cite{wang2020v2vnet} and DiscoNet~\cite{li2021learning}, which only transmit features and do not consider temporal asynchrony problem, FFNet transmits the feature flow with the feature prediction ability. The feature flow can be used to generate future features that are aligned with the vehicle feature, which can help alleviate fusion errors caused by temporal asynchrony. Notably, the first-order derivative in the feature flow is an independent module, and it can also be applied to newer feature fusion methods to achieve lower transmission costs.
    \item Computing-friendly for Vehicles with Limited Computing Resources: Our feature flow is generated on the infrastructure side and can directly predict future features to compensate for uncertain latency with linear computation in ego vehicles. Another possible solution to solve the temporal asynchrony problem is proposed in~\cite{lei2022latency}, which generates future features with received historical features on vehicle devices. However, this solution requires significant computing resources to process the historical frames and extract temporal correlations to predict future features. Furthermore, extracting temporal information from compressed features could be challenging, as compressed features have lost much valuable information compared to raw sequential point clouds.
    \item Saving Annotation Costs: FFNet training saves much annotation costs. We use a self-supervised learning method to train the feature flow generator and extract temporal feature flow from sequential point clouds. Our training method does not rely on labeled data, and it is possible to exploit the massive unlabelled infrastructure-side sequential data in the future.
\end{itemize}

% \clearpage
% \section{Visualization Results}
% In this section, we provide more visualization results to illustrate the effectiveness of our FFNet.
% In this section, we present additional visualization results to further demonstrate the effectiveness of our proposed FFNet. 
% \begin{figure}[ht!]
%   \centering
%   \includegraphics[width=1\linewidth]{images/ffnet-effect.png}\label{fig: ffnet performance}
%   \caption{\small \textbf{Detection Performance of FFNet.} On the vehicle side, a car is occluded by other objects and cannot be seen from the vehicle view. On the infrastructure side, this occluded car can be seen from an infrastructure view. Our FFNet can detect this car.}
% \end{figure}

%%%%%%%%%%%%%%%%%%%%%%%%%%%%%%%%%%%%%%%%%%%%%%%%%%%%%%%%%%%%%%%%%%%%%%%%%%%%%%%
%%%%%%%%%%%%%%%%%%%%%%%%%%%%%%%%%%%%%%%%%%%%%%%%%%%%%%%%%%%%%%%%%%%%%%%%%%%%%%%


\end{document}


% This document was modified from the file originally made available by
% Pat Langley and Andrea Danyluk for ICML-2K. This version was created
% by Iain Murray in 2018, and modified by Alexandre Bouchard in
% 2019 and 2021 and by Csaba Szepesvari, Gang Niu and Sivan Sabato in 2022.
% Modified again in 2023 by Sivan Sabato and Jonathan Scarlett.
% Previous contributors include Dan Roy, Lise Getoor and Tobias
% Scheffer, which was slightly modified from the 2010 version by
% Thorsten Joachims & Johannes Fuernkranz, slightly modified from the
% 2009 version by Kiri Wagstaff and Sam Roweis's 2008 version, which is
% slightly modified from Prasad Tadepalli's 2007 version which is a
% lightly changed version of the previous year's version by Andrew
% Moore, which was in turn edited from those of Kristian Kersting and
% Codrina Lauth. Alex Smola contributed to the algorithmic style files.
