\documentclass[10pt, onecolumn]{extarticle}

%\usepackage{color}
%\usepackage{soul}
%\newcommand{\improve}[1]{{\sethlcolor{yellow}\hl{#1}}}
%\newcommand{\todo}[1]{{\sethlcolor{cyan}\hl{#1}}}
%\newcommand{\todo}[1]{{{#1}}}

\usepackage[bordercolor=white,backgroundcolor=gray!30,linecolor=black,colorinlistoftodos, textsize=10pt]{todonotes}
\newcommand{\rework}[1]{\todo[color=yellow,inline]{\textbf{Rework: }#1}}

%\newcommandx{\info}[2][1=]{\todo[linecolor=OliveGreen,backgroundcolor=OliveGreen!25,bordercolor=OliveGreen,#1]{#2}}

%This fixes many isses with too long words in a two-comumn layout
\setlength{\emergencystretch}{3em}

\usepackage{amssymb}
\usepackage{mathtools}
\usepackage{textcomp}
\usepackage{upgreek}
\usepackage{amsmath}
\usepackage[textstyle,amssymb]{SIunits}
%\usepackage{url}			% For \url
\usepackage{graphicx}
\usepackage[normalem]{ulem}  % for underline allowing linebreaks
\usepackage{soul}
\usepackage[onecolumn,textwidth=183mm,columnsep=6mm,top=20mm,bottom=25mm]{geometry}
%\usepackage[doublespacing]{setspace}
\usepackage{helvet}
\usepackage{titlesec}
\titleformat*{\section}{\bfseries\sffamily}
\titlespacing{\section}{0pt}{*4}{*0}
\titleformat{\subsection}[runin]{\normalfont\bfseries}{\thesubsection.}{3pt}{}
\usepackage{setspace}
\usepackage[breaklinks=true]{hyperref} %[breaklinks=true]
\usepackage[hang,flushmargin]{footmisc} 
\usepackage[font={footnotesize,sf},labelfont={sf,bf},labelsep=endash,justification=raggedright]{caption}
\renewcommand{\figurename}{Figure}
\DeclarePairedDelimiter\abs{\lvert}{\rvert}%
\usepackage{natbib}
\setlength{\bibsep}{0.05cm}
\citestyle{nature}
\usepackage{caption}
\usepackage{subcaption}

\newcommand\blfootnote[1]{%
  \begingroup
  \renewcommand\thefootnote{}\footnote{\noindent#1}%
  \addtocounter{footnote}{-1}%
  \endgroup
}

\renewcommand{\thefigure}{S\arabic{figure}}
\renewcommand\refname{Supplementary References}

\usepackage[labelfont=bf,justification=justified]{caption}
\hyphenation{wave-guide wave-guides output analyzed high brightness}

\title{\fontsize{22pt}{26pt} \sf \bfseries Supplementary Information for Sub-nominal resolution Fourier transform spectrometry with chip-based combs \par}
\date{}
\author{}

\begin{document}
%\keywords{lab-on-a-chip, quantum cascade laser, quantum cascade detector, surface plasmon polariton, dielectric loaded, monolithic}
\vspace{-2cm}
\maketitle

\section{Interferogram processing}
\vskip 0.2cm \noindent
Figure~\ref{fig:Schematic} visually depicts the evolution of the interferogram (IGM) at each processing step of the self-extracted sub-nominal resolution Fourier transform spectroscopy (FTS) routine. The left column shows a simulated IGM with a simplified temporal structure, while the right plots real data from a Fourier spectrometer. First, from a potentially longer IGM only the part between the peak of the centerburst ($\Delta=0$), and the satellite (round-trip) burst is kept (whose center position is retrieved from the measured repetition rate $f_\mathrm{r}$). The truncated part is plotted using a dark line. A non-zero offset frequency causes the round-trip burst to not have a maximum at the round-trip delay $T_\mathrm{r}=1/f_\mathrm{r}$. When the offset frequency is digitally canceled (through phase shifting of a Hilbert-transformed real IGM using a linear phase ramp), the IGM becomes harmonic with a local maximum at $T_\mathrm{r}$ (blue trace).

Such a trace is equivalent to an offset-free double-sided IGM acquired symmetrically around the zero-path-difference point (ZPD, $\Delta=0$). Its circular shift by $N_\mathrm{ss}$ samples (assuming the input is $N=2N_\mathrm{ss}+1$ samples long) yields the trace from the last row of Fig.~\ref{fig:Schematic}. Note that the two IGM sides match each other without any glitches or rapid jumps. This representation, however, is used only for visualization purposes. To calculate the spectrum, the non-center representation (asymmetric) is preferred.


\begin{figure}[!hp]
	\centering
	\includegraphics[width=1\textwidth]{SchematicVertical.pdf}
	\caption{\textbf{Visual depiction of interferogram processing steps.}\label{fig:Schematic} }
\end{figure}


\section{Tuning of the repetition rate and offset frequency -- diode combs at 3~$\upmu$m \label{sec:tuningf0frep}}
\vskip 0.2cm \noindent
The ability to precisely extract the offset frequency ($f_\mathrm{0}$) from the measured IGM coupled with a precise measurement of $f_\mathrm{r}$ using a microwave spectrum analyzer or frequency counter enables straightforward optical frequency axis retrieval. The high stability of the reference wavelength laser, which is used for measuring the optical displacement (or a precise optical encoder), can be directly leveraged for frequency axis calibration. This is particularly useful in more exotic spectral regions, where the availability of suitable single-mode lasers for anchoring the frequency axis is scarce. 

Using the offset frequency retrieval routine described in detail in the main manuscript, we have characterized the offset frequency tuning behavior for two semiconductor laser frequency combs operating in the mid-infrared. Figure~\ref{fig:DiodeTuning} shows the anti-correlation of the tuning direction between $f_\mathrm{r}$ and $f_\mathrm{0}$ for a 3~$\upmu$m diode laser frequency comb. Whereas $f_\mathrm{r}$ is \emph{measured} with a microwave spectrum analyzer, $f_\mathrm{0}$ is \emph{estimated}. The oscillatory shape of the two curves results most likely from external optical cavity effects. During the injection current scan, when comb lines were tuned by a full $f_\mathrm{r}$, both frequencies suffered from periodic oscillations in counter phase. A coarse estimate of 20 oscillation periods over $f_\mathrm{r}\approx10$~GHz yields a single oscillation period of $\sim$500~MHz. This corresponds to an external cavity round trip length of 60~cm, which can be attributed to the position of the optical isolator (sub-optimal for this wavelength), located $\sim$30~cm away from the laser facet.

\section{Offset frequency ambiguity}
\vskip 0.2cm \noindent
It is imperative to discuss the offset frequency tuning characteristics in the context of sign ambiguity ($\pm f_0$). For semiconductor lasers, one expects 
nearly linear optical frequency tuning with injection current. Figure~\ref{fig:DiodeTuningWrong} shows that such behavior is obtained only in one case -- when the offset frequency decreases with injection current (here corresponding to $+f_0$). In the opposite case ($-f_0$), or when it is not included in the frequency retrieval formula, incorrect and highly oscillatory comb line tuning characteristics are obtained. Experimental proof for line tuning linearity obtained with an independent optical instrument is provided in Section~\ref{sec:diodeTuningLinearity}.

\begin{figure}[!ht]
	\centering
	\includegraphics[width=0.5\textwidth]{OpticalTuningDiodes.pdf}
	\caption{\textbf{Tuning of the offset frequency $f_0$, repetition rate $f_\mathrm{rep}$, and one of the comb lines for a quantum well diode laser frequency comb retrieved using the algorithm}. Note the anti-correlation between $f_0$ and $f_\mathrm{rep}$. Mutual cancellation of the oscillations yields nearly linear comb line position tuning.\label{fig:DiodeTuning} }
\end{figure}


\begin{figure}[!ht]
	\centering
	\includegraphics[width=0.5\textwidth]{OpticalTuningDiodesWrongDir.pdf}
	\caption{\textbf{The problem of $f_0$ ambiguity can be easily solved for scanned measurements.} Prior knowledge about the nearly linear tuning of the optical wavelength/frequency can be incorporated to determine the sign of $f_0$.\label{fig:DiodeTuningWrong} }
\end{figure}

\newpage

\section{Tuning of optical spectra characterized using an optical spectrum analyzer \label{sec:diodeTuningLinearity}}
\vskip 0.2cm \noindent
Experimental evidence of comb line tuning linearity despite the oscillatory trajectories of $f_\mathrm{r}$ is shown in Fig.~\ref{fig:diodeTuningUp} and Fig.~\ref{fig:iclTuningUp}. Two comb platforms are characterized here. Figure~\ref{fig:diodeTuningUp} shows the tuning map for diode laser frequency combs, while Fig.~\ref{fig:iclTuningUp} shows one for an interband cascade laser frequency comb. It is clear that the diode comb is exhibits richer $f_\mathrm{r}$ tuning dynamics. 

%Diode tuning characteristics:
\begin{figure}[!h]
\centering
\begin{subfigure}{.37\textwidth}
\centering
  \includegraphics[height=5cm]{OSAvsCurrent_up_diode__zoom.pdf}  
  \caption{Full-span optical spectrum}
  \label{fig:diodeTuningUpA}
\end{subfigure}
%
\begin{subfigure}{.37\textwidth}
  \centering
  \includegraphics[height=5cm]{OSAvsCurrent_up_diode__farazoom.pdf}  
  \caption{Zoomed shorter-wavelength part}
  \label{fig:diodeTuningUpB}
\end{subfigure}
%
\begin{subfigure}{.18\textwidth}
\centering
  \includegraphics[height=5cm]{RFvsCurrent_diode__down_up.pdf}  
  \caption{Radio-frequency spectra}
  \label{fig:diodeTuningUpC}
\end{subfigure}
\caption{ \textbf{Tuning capabilities of the 3$~\upmu$m wavelength diode comb for gap-less high-resolution spectroscopy.} The spectra were measured for diagnostic purposes with an optical spectrum analyzer (a--b) and a microwave spectrum analyzer (c). The wavelength increases almost linearly with injection current, which corresponds to a decrease of the optical frequency (wavenumber). A full free spectral range (FSR) scan required increasing the injection current by 44.5~mA ($\sim$10\%). Note the oscillatory tuning of the RF beat note. \label{fig:diodeTuningUp}}
\end{figure}

%ICL tuning characteristics:
\begin{figure}[!h]
\centering
\begin{subfigure}{.37\textwidth}
\centering
  \includegraphics[height=5cm]{OSAvsCurrent_up_ICL__zoom.pdf}  
  \caption{Full-span optical spectrum}
  \label{fig:iclTuningUpA}
\end{subfigure}
%
\begin{subfigure}{.37\textwidth}
  \centering
  \includegraphics[height=5cm]{OSAvsCurrent_up_ICL__farazoom.pdf}  
  \caption{Zoomed shorter-wavelength part}
  \label{fig:iclTuningUpB}
\end{subfigure}
%
\begin{subfigure}{.18\textwidth}
\centering
  \includegraphics[height=5cm]{RFvsCurrent_ICL__down_up.pdf}  
  \caption{Radio-frequency spectra}
  \label{fig:iclTuningUpC}
\end{subfigure}
\caption{ \textbf{Tuning capabilities of the 3.25$~\upmu$m wavelength interband cascade laser frequency comb for gap-less high-resolution spectroscopy.} The spectra were measured for diagnostic purposes with a Fourier spectrometer (a--b) and a microwave spectrum analyzer (c). The wavelength increases almost linearly with the injection current, which corresponds to a decrease in the optical frequency (wavenumber). A full free spectral range (FSR) scan required increasing the injection current by 14.7~mA ($\sim$5\%). \label{fig:iclTuningUp}}
\end{figure}

\newpage

\section{Tuning of the repetition rate and offset frequency - ICLs at 3.25~$\upmu$m}
\vskip 0.2cm \noindent
Analogously to the data presented in Section~\ref{sec:tuningf0frep}, we have retrieved the frequency tuning characteristics for ICL combs operating with a center wavelength of 3.25~$\upmu$m. This is shown in Fig.~\ref{fig:ICLTuning}. Note the much lower amplitude of frequency oscillations of $f_0$ and $f_\mathrm{r}$ because the emission wavelengths almost match the optimal wavelength of the coating and one that provides the best optical isolation from downstream optical components (3.35~$\upmu$m). However, the periodicity of the oscillations is halved compared to the diode laser comb case ($\sim$36 periods yielding a period of$\sim$267~MHz). This corresponds to an external cavity roundtrip length of $\sim$1.12~m, which can be attributed to the position of another parasitic etalon in the system -- the windows of the gas cell.

\begin{figure}[!ht]
	\centering
	\includegraphics[width=0.5\textwidth]{OpticalTuningICLs.pdf}
	\caption{\textbf{Optical frequency axis retrieval for ICL combs}. Tuning of the offset frequency $f_0$, repetition rate $f_\mathrm{rep}$, and one of the comb lines for an interband cascade laser frequency comb retrieved using the algorithm. Note the anti-correlation between $f_0$ and $f_\mathrm{rep}$. Mutual cancellation of the oscillations yields nearly linear comb line position tuning. \label{fig:ICLTuning} }
\end{figure}




\newpage

\section{Raw subnominal-resolution optical spectra}
\vskip 0.2cm \noindent
For reader's convenience, and justification of the $\sim$20~dB dynamic range, Fig.~\ref{fig:combinedSpectrumC2H2} and Fig.~\ref{fig:combinedSpectrumCH4} plot raw interleaved spectra calculated using the subnominal resolution routine for acetylene (C$_2$H$_2$), and methane (CH$_4$), respectively. Panels (a) show full-span interleaved spectra with frequency axis calibration based on the estimated $f_0$, known $\lambda_\mathrm{ref}$ (temperature stabilized HeNe laser) and measured $f_\mathrm{r}$. Panels (b) are scatter-type plots where dots sharing the same color correspond to individual comb lines coexisting at a given injection current. They are separated by 0.32--0.33~cm$^{-1}$, which is the comb repetition rate expressed in wavenumbers. Injection current tuning responsible for spectral interleaving is analogous to having $\sim$100 single-mode lasers simultaneously tuned in frequency. Each tuning curve is uniquely defined by the comb line number $n$, $f_0$, and $f_\mathrm{r}$. Note that the saw-tooth-like shape of the spectral edges is caused by the appearance of new comb lines that do not exist at lower injection currents. In other words, with increased pumping, the spectrum gradually broadens and lines of the spectral edges reach higher intensities. In contrast, the central part of the spectrum is dominated by frequency tuning with minor intensity changes.

\begin{figure}[!h]
\centering
\begin{subfigure}{.55\textwidth}
  \centering
  \includegraphics[width=\textwidth]{C2H2_Cell_Spectrum.pdf}  
  \caption{}
  \label{fig:combinedSpectrumA}
\end{subfigure}

\begin{subfigure}{.55\linewidth}
\centering
  \includegraphics[width=\textwidth]{C2H2_Cell_Spectrum_Tuning.pdf}  
  \caption{}
  \label{fig:combinedSpectrumB}
\end{subfigure}

\begin{subfigure}{.55\textwidth}
\centering
  \includegraphics[width=\textwidth]{C2H2_Cell_Spectrum_TuningZoom.pdf}  
  \caption{}
  \label{fig:combinedSpectrumC}
\end{subfigure}

\begin{subfigure}{.55\textwidth}
\centering
  \includegraphics[width=\textwidth]{C2H2_Cell_Spectrum_TuningZoom2.pdf}  
  \caption{}
  \label{fig:combinedSpectrumD}
\end{subfigure}
\caption{ \textbf{Sub-nominal resolution FTS spectra showing the tuning range of individual comb lines. The analyte was low pressure acetylene (C$_2$H$_2$ at 2 Torr).} The sawtooth-like spectral shape on the edges of the spectrum results from its progressive broadening and appearance of comb lines that do not exist at lower injection currents. (b) Span of 1.8~cm$^{-1}$ (54~GHz), (c) Span of 0.3~cm$^{-1}$ (9~GHz), (d) Span of 0.033~cm$^{-1}$ (1000~MHz).\label{fig:combinedSpectrumC2H2}}
\end{figure}




\begin{figure}[!h]
\centering
\begin{subfigure}{.55\textwidth}
  \centering
  \includegraphics[width=\textwidth]{CH4_Cell_Spectrum.pdf}  
  \caption{}
  \label{fig:CH4combinedSpectrumA}
\end{subfigure}

\begin{subfigure}{.55\linewidth}
\centering
  \includegraphics[width=\textwidth]{CH4_Cell_Spectrum_Tuning.pdf}  
  \caption{}
  \label{fig:CH4combinedSpectrumB}
\end{subfigure}

\begin{subfigure}{.55\textwidth}
\centering
  \includegraphics[width=\textwidth]{CH4_Cell_Spectrum_TuningZoom.pdf}  
  \caption{}
  \label{fig:CH4combinedSpectrumC}
\end{subfigure}

\begin{subfigure}{.55\textwidth}
\centering
  \includegraphics[width=\textwidth]{CH4_Cell_Spectrum_TuningZoom2.pdf}  
  \caption{}
  \label{fig:CH4combinedSpectrumD}
\end{subfigure}
\caption{ \textbf{Sub-nominal resolution FTS spectra showing the tuning range of individual comb lines. The analyte was methane.} The sawtooth-like spectral shape on the edges of the spectrum results from its progressive broadening and appearance of comb lines that do not exist at lower injection currents. Panels (b--d) show individual modal intensities during the current scan. Dots sharing the same color are spaced by $f_\mathrm{r}$. (b) Span of 1.8~cm$^{-1}$ (54~GHz), (c) Span of 0.3~cm$^{-1}$ (9~GHz), (d) Span of 0.034~cm$^{-1}$ (1020~MHz). \label{fig:combinedSpectrumCH4}}
\end{figure}

\end{document}




