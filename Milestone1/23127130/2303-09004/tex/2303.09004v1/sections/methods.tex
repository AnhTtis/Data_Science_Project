\section{Data-Driven Safe Control} \label{sec:safe_no_w}


%%%-----------------------------
\subsection{Problem Statement}

% Consider the continuous-time control-affine system of the form
% \begin{equation} \label{eq:dynamics}
%     \dot{x} = f(x) + g(x) u
% \end{equation}
% where $x\in \R^n$ is the state, $u\in \R$ is the input, $f,g\colon \R^n \rightarrow \R^n$ are unknown polynomial functions, respectively. We assume the degrees of $f,g$ are known (or user-defined), then $f,g$ can be expressed by $f = F \phi$, $g = G \gamma$ where $F,G$ are unknown coefficient matrices and $\phi,\gamma$ are known monomial vectors in $x$, of proper dimensions. For example, for a two-state second-order polynomial system, $\phi,\gamma = [1,x_1,x_2,x_1^2,x_1x_2,x_2^2]^T$ and $F,G \in \R^{2\times 6}$.

The goal of this paper is to design a safe control law for a polynomial system where only system degrees and sample data are known. The problem is formally stated as:

\begin{problem} \label{pb:1}
Given an initial set $\mathcal{X}_0$ and an unsafe set $\mathcal{X}_u$ (both are basic semialgebraic), a noise bound $\epsilon > 0$, and noisy data $(y_s,x_s,u_s) = \dc$ generated by a system \eqref{eq:dynamics} under the relation that $\norm{y_s - f(x_s) - g(x_s) u_s}_\infty \leq \epsilon$, find a state-feedback control law $u(x)$ such that all systems consistent with data are safe w.r.t. $\mathcal{X}_0$ and $\mathcal{X}_u$.
\end{problem}

\begin{remark}
    The data $y_s$ is a noisy observation of the true derivative $\dot{x_s}$.
\end{remark}

\begin{remark}
The solved control law $u(x)$ is not necessarily consistent with sample data $\dc$.
\end{remark}

% In this section, we derive a data-driven algorithm to solve the problem \ref{pb:1}. 

The data-driven Problem \ref{pb:1} will be approached by  define two polytopic sets in parameter space: one for $\dc$-consistent systems ($\mathcal{P}_1$) and one for safe systems ($\mathcal{P}_2$). Applying the Extended Farkas' Lemma to enforce set-containment leads to a polynomial feasibility problem, which can be solved with \ac{SOS} method by \ac{SDP} solvers.

\subsection{Consistency Set}

Assume the sample data $\dc$ is corrupted by a sample (offline) noise bounded by $\epsilon$. The consistency set $\mathcal{P}_1$, 
which contains all systems that are consistent with the data, is defined as:
\begin{equation}
    \mathcal{P}_1 \doteq \left\{
    f,g\colon \|y_s-f(x_s)-g(x_s)u_s\|_\infty \leq \epsilon, t=1,\cdots,T \right\}.
\end{equation}
Recall that $f = F \phi$, $g = G \gamma$. By using the following property of the Kronecker product \cite{petersen2008matrix} $$\text{vec}(P^TXQ^T) = (Q\otimes P^T) \text{vec}(X),$$ an equivalent formulation is derived as
\begin{equation} \label{eq:p1}
    \mathcal{P}_1 = \left\{ 
    \mathbf{f},\mathbf{g} \colon 
    \begin{bmatrix}
        A & B \\ -A & -B
    \end{bmatrix} 
    \begin{bmatrix}
        \mathbf{f} \\ \mathbf{g}
    \end{bmatrix}
    \leq 
    \begin{bmatrix}
        \epsilon \mathbf{1} + \xi \\ \epsilon \mathbf{1} - \xi
    \end{bmatrix} \right\},
\end{equation}
where $\mathbf{f} = \text{vec} (F^T)$, $\mathbf{g} = \text{vec} (G^T)$ and
\begin{equation} \label{eq:abxi}
    A \doteq \begin{bmatrix}
    I \otimes \phi^T(1) \\ \vdots \\ I \otimes \phi^T(T)
    \end{bmatrix},
    B \doteq \begin{bmatrix}
    I \otimes u\gamma^T(1) \\ \vdots \\ I\otimes u\gamma^T(T)
    \end{bmatrix},
    \xi \doteq 
    \begin{bmatrix}
    y_s(1) \\ \vdots \\ y_s(T)
    \end{bmatrix}.
\end{equation}

\begin{assumption}
The consistency set $\mathcal{P}_1$ is compact. \label{assum:compact}
% ,thus enough data need to be collected until the matrix $ \begin{bmatrix} A & B \\ -A & -B \end{bmatrix}$ has full column rank. \urg{Reason?}
\end{assumption}
\begin{remark} A necessary condition for Assumption \ref{assum:compact} to hold is that $[A, B; -A, -B]$ has full column rank.
\end{remark}

\begin{remark}
The polytope $\mathcal{P}_1$ is defined by $2 nT$ linear inequality constraints. Redundant constraints may be identified and then eliminated using iterative linear programming \cite{caron1989degenerate}.
\end{remark}


\subsection{Safety Set}
The safety set $\mathcal{P}_2$, which contains all safe systems with respect to given $\mathcal{X}_0, \mathcal{X}_u$, is defined by \eqref{eq:den}. Introducing $\psi =\rho u$ to tackle the bilinearity, the safe control synthesis can be formulated as a convex optimization problem \cite{rantzer2004analysis}
\begin{subequations}\label{eq:safe_rhopsi}
\begin{align}
\nabla \cdot [\rho f + \psi g] (x)&> 0,\ \forall x\in \R^n \label{eq:safe_den} \\
|\psi(x)| &\leq c\rho(x),\ \forall x\in \R^n \label{eq:safe_u}\\
\rho (x) &> 0, \ \forall x\in \mathcal{X}_0\\
\rho (x)&\leq 0, \ \forall x\in \mathcal{X}_u,
\end{align}
\end{subequations}
where $c\geq 0$ is a given scalar.  Problem \eqref{eq:safe_rhopsi} is an infinite-dimensional \ac{LP} in the values of $(\rho, \psi)$ at each $x$, possessing both strict and non-strict inequality constraints.
\begin{remark}
Note that \eqref{eq:safe_u} is the big-M method to enforce $\psi=0$ when $\rho=0$, and it simultaneously leads to a bounded-control $|u|\leq c$ as $u=\psi/\rho$ by construction.
\end{remark}
\begin{remark}
    Constraint \eqref{eq:safe_u} may be modified to ensure that the control $u$ lies in the compact polytope $\{u \mid G u \leq h\}$ by forming the expression $G \psi \leq h (\1 \rho)$.
\end{remark}
% \begin{theorem} \label{thm:safety}
% Given an initial set $\mathcal{X}_0$ and an unsafe set $\mathcal{X}_u$, the system \eqref{eq:dynamics} is safe if there exists a scalar function $\rho(x)\in C^1$ such that
% \begin{subequations}
% \begin{align}
% \nabla \cdot [\rho (f + gu)] (x)&> 0, \ \forall x\in \R^n \label{eq:den0} \\
% \rho (x) &> 0, \ \forall x\in \mathcal{X}_0\\
% \rho (x)&\leq 0, \ \forall x\in \mathcal{X}_u.
% \end{align}
% \end{subequations}
% Further, introduce $\psi =\rho u$ to tackle the bilinearity in \eqref{eq:den0}, the safe control synthesis is formulated as a convex optimization problem
% \begin{subequations}\label{eq:safe}
% \begin{align}
% \nabla \cdot [\rho f + \psi g] (x)&> 0,\ \forall x\in \R^n \label{eq:safe_den} \\
% |\psi(x)| &\leq c\rho(x),\ \forall x\in \R^n \label{eq:safe_u}\\
% \rho (x) &> 0, \ \forall x\in \mathcal{X}_0\\
% \rho (x)&\leq 0, \ \forall x\in \mathcal{X}_u,
% \end{align}
% \end{subequations}
% \end{theorem}
% where $c$ is a scalar for the big-M relaxation to enforce $\psi=0$ if $\rho=0$. Note that \eqref{eq:safe_u} also leads to a bounded control $|u|\leq c$.
% \begin{proof}
% This follows from ... \urg{to be finished}
% \end{proof}
By the replacement $f=F\phi,g=G\gamma$ in \eqref{eq:safe_den}, the $u$-safe set $\mathcal{P}_2$ is may be described as
\begin{equation} \label{eq:p2}
    \mathcal{P}_2 \doteq \left\{\mathbf{f},\mathbf{g} \colon -
    \begin{bmatrix}
        \nabla\cdot(\rho\phi^T) & \nabla\cdot(\psi\gamma^T)
    \end{bmatrix}
    \begin{bmatrix}
    \mathbf{f} \\ \mathbf{g}
    \end{bmatrix} < 0\right\}.
\end{equation} 

\subsection{Data-Driven Safety Control}

% Applying the Extended Farkas' Lemma to enforce $\mathcal{P}_1 \subseteq \mathcal{P}_2$ (i.e., all consistent systems are safe), leads to the main theorem of this paper.

Applying the Extended Farkas' Lemma to enforce $\mathcal{P}_1 \subseteq \mathcal{P}_2$ leads to the following lemma:
\begin{lemma}\label{lem:farkas}  
$\mathcal{P}_1 \subseteq \mathcal{P}_2$ if and only if there exists a vector function $y(x)\geq 0$ (of dimension $1\times 2nT$ where $n$ is the number of states and $T$ is the number of data) such that the following functional set of affine constraints is feasible:
\begin{equation} \label{eq:lem1}
    y(x)N = d(x) \ \text{and} \ y(x)e  \leq 0,
\end{equation} 
where for simplicity we define
\begin{equation} \begin{aligned} \label{eq:lem2}  
    & N \doteq \begin{bmatrix} A & B\\-A & -B \end{bmatrix}, \; 
    e \doteq \begin{bmatrix}\epsilon \mathbf{1} + \xi \\ \epsilon \mathbf{1} - \xi \end{bmatrix}, \\
    & d(x) \doteq - \begin{bmatrix} \nabla\cdot(\rho\phi^T) & \nabla\cdot(\psi\gamma^T) \end{bmatrix}.
\end{aligned} \end{equation}
\end{lemma}
\begin{proof}
Directly apply the Extended Farkas' Lemma.
\end{proof}

The main result of this paper is the following theorem:
\begin{theorem}\label{thm:main}
% Given noisy data of a , a sufficient condition for there to  exist 
A sufficient condition for the existence of a state-feedback control law $u(x)$ such that all $\dc$-consistent systems following system \eqref{eq:dynamics} are safe is that there exists functions $\rho \in C^1$,  $\psi\in C^0$, and a (possibly discontinuous) vector function $y(x) \geq 0$ such that
\begin{subequations}\label{eq:thm_main} 
\begin{align}
y(x)N &= d(x), \ \forall x \in \R^n\label{eq:thm_main1}\\
y(x)e  &\leq 0, \ \forall x \in \R^n\\
|\psi(x)| &\leq c\rho(x),\ \forall x\in \R^n\\
\rho(x) &> 0,\ \forall x\in\mathcal{X}_0\\ 
\rho(x) &\leq 0, \ \forall x\in\mathcal{X}_u.
\end{align}
\end{subequations}
The control law $u$ can then be extracted by the division $u(x) = \psi(x)/\rho(x)$.
% And the control law can be extracted as $u=\psi/\rho$, which is a rational polynomial function. 
\end{theorem}
\begin{proof}
Apply Lemma \ref{lem:farkas} to Problem \eqref{eq:safe_rhopsi}.
\end{proof}

\subsection{Sum-of-Squares Safety Program}
\label{sec:sos_safety}
In order to solve the infinite-dimensional Problem \eqref{eq:thm_main} in a tractable manner, we restrict the variables $\rho, \psi, y$ to be polynomials. The extracted controller $u(x) = \psi(x)/\rho(x)$ is then a rational function.
% Note that \eqref{eq:thm_main} is a feasibility polynomial problem, which can be solved using SOS program. 
Compare the coefficients $k_l, k_r$ of \eqref{eq:thm_main1} on both sides and let $\mathcal{X}_0 \doteq \{x\colon f_0(x) \geq 0\}$, $\mathcal{X}_u \doteq \{x\colon f_u(x) \geq 0\}$, a detailed algorithm is given in Alg. \ref{alg:1}.
\begin{algorithm}[h]
    \caption{Data-Driven Safe Control Design} \label{alg:1}
    \begin{algorithmic}
        \State Input: sample data $\dc$, and degrees of $f,g,\rho,\psi$.
        \State Let $d_1=\text{max}\left\{d_f+d_\rho,d_g+d_\psi \right\}, d_2 = \text{max}\left\{d_\rho,d_\psi \right\}$.
        \State Solve: the feasibility problem
        \[ \begin{array}{l}
        k_l = k_r                                 \hfill(A.1)\\
        -Ye            \; \in\Sigma_{d_1} [x]    \hfill(A.2)\\
        c\rho - \psi   \; \in \Sigma_{d_2} [x]    \hfill(A.3)\\
        c\rho + \psi   \; \in \Sigma_{d_2} [x]    \hfill(A.4)\\
        \rho - s_1f_0  \; \in \Sigma_{d_\rho} [x] \hfill(A.5)\\
        -\rho - s_2f_u \; \in \Sigma_{d_\rho} [x] \hspace{1cm}\hfill(A.6)\\
        Y_i            \; \in \Sigma_{d_1} [x]    \hfill(A.7)\\
        s_1, s_2       \; \in \Sigma_{d_2} [x]    \hfill(A.8)\\
        
        \end{array} \]
        \State Output: the safe control law $u = \psi/\rho$ or a certificate of infeasibility at the current degree 
    \end{algorithmic}
\end{algorithm}
