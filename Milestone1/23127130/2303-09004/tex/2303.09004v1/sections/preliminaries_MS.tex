\section{Preliminaries} \label{sec:preliminaries}
\begin{acronym}
% \acro{BSA}{Basic Semialgebraic}
\acro{CBF}{Control Barrier Function}

\acro{DDC}{Data Driven Control}


% \acro{GAS}{Globally Asymptotically Stable}

% \acro{CSP}{Correlative Sparsity Pattern}

\acro{LMI}{Linear Matrix Inequality}
\acroplural{LMI}[LMIs]{Linear Matrix Inequalities}
\acroindefinite{LMI}{an}{a}
% \acp{LMI}: LMIs \iacf

\acro{LP}{Linear Program}
\acroindefinite{LP}{an}{a}
% \acro{OCP}{Optimal Control Problem}

% \acro{ODE}{Ordinary Differential Equation}

% \acro{POP}{Polynomial Optimization Problem}

% \acro{PSD}{Positive Semidefinite}

% \acro{PD}{Positive Definite}

% \acro{PDE}{Partial Differential Equation}

\acro{SDP}{Semidefinite Program}
\acroindefinite{SDP}{an}{a}

\acro{SOS}{Sum of Squares}
\acroindefinite{SOS}{an}{a}
% \iac{} an SOS \Iac An SOS
% \acp

% \acro{WSOS}{Weighted Sum of Squares}

\end{acronym}

\subsection{Notation} \label{subsec:notation}

\begin{tabular}{p{0.13\columnwidth}p{0.75 \columnwidth}}
$\R^n$ & Set of $n$-tuples of real numbers\\
$x,\vt{x},\mt{X}$ & Scalar, vector, matrix\\
$\mathbf{1,0},\mt{I}$ & Vector/matrix of all 1s, 0s, identity matrix\\
$\left\|\vt{x} \right\|_\infty$ & $L_\infty$-norm of vector $\vt{x}$\\
%$\left\|\mt{X} \right\|_p$ & Induced $\ell_p$-norm of matrix $\mt{X}$\\
$\mt{X}\succeq 0$ & $\mt{X}$ is positive semi-definite \\
$\otimes$ & Kronecker product\\
\text{vec}$(\mt{X})$ & Vectorized matrix along columns: \\ & $\text{vec}(\mt{X})= \left[ \mt{X}(\colon,1)^T,\ldots,\mt{X}(\colon,n)^T\right]^T$ \\
$\rho \in C^d$ & $\rho$ has a continuous $d^{th}$ derivative\\
$\nabla \rho$ & Gradient of scalar function $\rho$ \\
$\nabla \cdot f$ & Divergence of vector function $f$\\
% $\Sigma[x]$ & Cone of \ac{SOS} polynomials in $x \in \R^n$\\
% $\Sigma[x]_{\leq d}$ &  \ac{SOS} polynomials up to degree $d$\\
% \R[x]
\end{tabular}

\subsection{Sum-of-Squares}

We briefly review the concept of \ac{SOS} polynomials and proofs of nonnegativity \cite{parrilo2000structured}. A polynomial $p \in \R[\vt{x}]$ is \ac{SOS} (and hence nonnegative) if there exist polynomials $\{q_\ell \in \R[\vt{x}]\}_{\ell= 1}^L$ such that $p(\vt{x}) = \sum_{\ell=1}^L q_\ell(\vt{x})^2$.
%Given that each $q_\ell$ has real coefficients and $q_\ell(x)^2$ is nonnegative, it holds that the sum $\sum_{\ell=1}^L q_\ell(x)^2$ is also nonnegative. 

The cone of \ac{SOS} polynomials is $\Sigma[\vt{x}]$, and its up to degree $2d$ restriction is $\Sigma_{d}[\vt{x}]$. The cone $\Sigma_d[\vt{x}]$ is semidefinite representable as $p(\vt{x}) = \vt{v}(\vt{x})^T \mt{Q} \vt{v}(\vt{x})$ where $\vt{v}(\vt{x})$ is the monomial vector up to degree $d$ and $\mt{Q} \succeq 0$ is the Gram matrix. A sufficient condition for a polynomial $p$ to be nonnegative over the semialgebraic region $\{\vt{x} \mid h_i(\vt{x}) \geq 0, \ i = 1..N_c\}$ is that $p$ is contained in the quadratic module formed by $h_i$ (there exists $\sigma_{0..N_c} \in \Sigma[\vt{x}]$ such that $p(\vt{x}) = \sigma_0 + \sum_{i=1}^{N_c}\sigma_i h_i$).

% \begin{itemize}
%     \item[A1] Assumption 1
%     \item[A2]  Assumption 2
% \end{itemize}
% \subsection{SOS Polynomials}
% This paper shows that the data-driven safe control synthesis can be formulated as a polynomial nonnegativity  problem, which is generally NP-hard (for polynomials of even degree $d\geq 4$). This difficulty can be handled by relaxing the problem to establishing that a given polynomial is a Sum of Squares. Consider a polynomial $f$ of degree $2m$. A sufficient condition to $\forall x \in \R^n: f(x)\geq 0$ is that $f$ can be written as a \ac{SOS} 
% $f = \textstyle \sum_i g_i^2.$
% Let $v$ be a vector with all monomials of degree less than or equal to $m$. Then $f$ is \ac{SOS} if and only if there exists a positive semidefinite matrix $Q$ such that
% $f = v^T Qv$ \cite{parrilo2000structured}.
% Comparing term on the both sides leads to an \ac{SDP} program, which can be solved by off-the-shelf solvers. \textcolor{red}{do we need to mention the Psatz?}


\subsection{Level-Set-Based Safety Certification}
Consider a continuous-time  system of the form
\begin{equation} \label{eq:dynamics}
    \dot{\vt{x}} = f(\vt{x},\vt{w})
\end{equation}
where $\vt{x}\in \R^n$ is the state and $\vt{w} \in \mathcal{W} \subseteq \R^n$ is a disturbance. Further, assume that $\vt{w}(t)$ is such that the trajectories of \eqref{eq:dynamics} are well defined for any initial condition $\vt{x}_0 \in \mathcal{X}_0$. In the sequel, we will denote these trajectories as $\vt{x}(t,\vt{w},\vt{x}_0)$.
\begin{definition}
Given an initial condition set $\mathcal{X}_0 \subseteq \R^n$ and an unsafe set $\mathcal{X}_u \subseteq \R^n$, system \eqref{eq:dynamics} is robustly safe if, for all $t$, all initial conditions $\vt{x}_0 \in \mathcal{X}_0$ and all $\vt{w}(t) \in \mathcal{W}$,  $\vt{x}(t,\vt{w},\vt{x}_0) \not \in\mathcal{X}_u$.
 \end{definition}
Typically, safety is certified through the use of barrier functions, defined as:
\begin{definition}
A differentiable function $B(\vt{x}): \R^n \to \R$ is a robust barrier function for \eqref{eq:dynamics} with respect to $\mathcal{X}_0$ and $\mathcal{X}_u$ if 
\begin{align}
B(\vt{x}) & \leq 0,\; \forall \vt{x} \in \mathcal{X}_0, \; 
B(\vt{x})  >0, \; \forall \vt{x} \in\mathcal{X}_u \label{eq:B2} \\
\frac{\partial B}{\partial \vt{x}}f(\vt{x},\vt{w}) & < 0, \; \forall \vt{w} \in \mathcal{W} \quad \text{whenever $B(\vt{x})=0$.} \label{eq:B3}
\end{align}
\end{definition} 
As shown for instance in \cite{prajna2004safety}, existence of a barrier function is a sufficient condition to certify safety. Note however that the conditions above are non-convex, even when $\vt{w} \equiv  0$, due to the constraint \eqref{eq:B3}. For instance, in the case of polynomial dynamics and semialgebraic $\mathcal{X}_0$ and $\mathcal{X}_u$, if $B(\vt{x})$ is also polynomial,  this constraint can be enforced by introducing a polynomial multiplier $h(\vt{x})$ and imposing that
\begin{equation}\label{eq:CBF}
-\frac{\partial B}{\partial \vt{x}}f(\vt{x},\vt{w}) + h(\vt{x})B(\vt{x}) \in \Sigma[\vt{x}]. \end{equation}
The condition above cannot be written as a single semi-definite optimization due to the multiplication of the coefficients of the two unknown polynomials, $h$ and $B$. Possible relaxations include choosing a fixed multiplier $h$, or simply dropping the $B(\vt{x})=0$ quantifier \cite{ames2019control}.
An alternative, convex approach based on the use of densities was proposed in \cite{rantzer2004analysis}.
\begin{theorem}[\cite{rantzer2004analysis}] \label{thm:safety}
Given $\mathcal{X}_0$ and $\mathcal{X}_u$, system \eqref{eq:dynamics} is safe if there exists a scalar function $\rho(\vt{x})\in C^1$ such that
\begin{subequations} \label{eq:safe_rho}
\begin{align}
\nabla \cdot [\rho(\vt{x}) f (\vt{x})]&> 0, \ \forall \vt{x}\in \R^n \label{eq:den} \\
\rho (\vt{x}) & \geq 0, \ \forall \vt{x}\in \mathcal{X}_0, \;
\rho (\vt{x}) < 0, \ \forall \vt{x}\in \mathcal{X}_u.
\end{align}
\end{subequations}
\end{theorem}
The advantage of this approach is that it leads to a convex problem in $\rho$. On the other hand, imposing that the divergence condition holds everywhere can be unnecessarily conservative.  

The concepts above can be easily extended to the case where the goal is to synthesize a control action that keeps a system safe by introducing the concept of \acp{CBF}.
%{ Control Barrier Functions}. 
\begin{definition}  A function $B(\vt{x})$ is a \ac{CBF} for the system
%\begin{equation} \label{eq:dynamicsu}
  $  \dot{\vt{x}} = f(\vt{x},u,\vt{w})$
%\end{equation}
if there exists a control law $u(\vt{x})$ such that $B(\vt{x})$ is a barrier function for the closed loop dynamics $\dot{\vt{x}}=f(\vt{x},u(\vt{x}),\vt{w})$.
\end{definition}
In principle, a \ac{CBF} and associated control law can be found by modifying \eqref{eq:CBF} to
\beq \label{eq:h_relaxation} -\frac{\partial B}{\partial \vt{x}}f(\vt{x},u(\vt{x}),\vt{w}) + h(\vt{x})B(\vt{x}) \in \Sigma[\vt{x}]. \eeq
Problem \eqref{eq:h_relaxation} is bilinear in the coefficients of $B, u$ even when restricted to polynomial dynamics and control laws and a fixed multiplier $h$, necessitating the use of relaxations.
% However, even in the case of polynomial dynamics and control laws and a fixed multiplier $h$, the problem above is bilinear in the coefficients of $u,B$, necessitating the use of relaxations. 
On the other hand, as shown in \cite{rantzer2004analysis}, the density based formulation can be easily modified to lead to  problems that are jointly convex in $\rho$ and $ \psi \doteq \rho u$.

% Further, as shown in \cite{rantzer2004analysis}, it can be 
% \begin{theorem}
%     If the integrability condition $\int_{\R^n} \rho(x) dx < \infty$ holds in addition to \eqref{eq:safe_rhou}, then $x=0$ is asymptotically stable for all $x$ originating in $\mathcal{X}_0$ \urg{get the right source and denominator for this, if we use rational polynomials}
% \end{theorem}

% \begin{proof}
% This follows from ... \urg{to be finished}
% \end{proof}




%%%-------------------------------------------------
% \subsection{Farkas' Lemma (variant)}

% \begin{lemma}
% Let $A\in \R^{m\times n}$ and $b\in \R^n$. Then the following statements are strongly alternative, i.e., exactly one of them is true:
% \begin{enumerate}
%     \item There exists an $x\in \R^n$ such that $Ax \leq b$.
%     \item There exists a $y\geq 0, y\in\R^m$ such that $A^T y = 0$ and $b^T y <0$.
% \end{enumerate}
% \end{lemma}
%%%-------------------------------------------------

%\subsection{Extended Farkas' Lemma}
%The goal of this paper is to  find a control law %such that the set of all consistent systems is %contained in the set of all safe systems. Such set-%containment will be  enforced by using the %following variant of  Farkas' Lemma %\cite{henrion1999control}:
%\cite{dai2020semi}:
 %\begin{lemma}\label{lemma:Farkas} Given %$\mat{N}\in\mathbb{R}^{m\times n},
% \mat{d} \in \mathbb{R}^n, \mat{e} \in \mathbb{R}^{m}$, assume that $\mat{N}\mat{x} \leq \mat{e}$ is feasible. Then the inequalities:
%\beq\label{eq:alt1} \mat{Ax} \leq \mat{b} \; %\text{with $\mat{A}^T= 
%\begin{bmatrix}\mat{N}^T & - \textcolor{black}%{\mat{d}} \end{bmatrix}$ and
%$\mat{b}^T =\begin{bmatrix} \mat{e}^T & 0 %\end{bmatrix}$}
%\eeq
%and
%\beq \label{eq:alt2} 
%\mat{N}^T\mat{y} = \mat{d},\; \mat{e}^T\mat{y}< 0, \text{and $\mat{y} \geq 0$, $\mat{y} \in \mathbb{R}^m$} \eeq
%are strong alternatives, that is exactly only one %set is feasible.
%\end{lemma}
%\begin{lemma}
%Consider polytopes $\mathcal{P}_1 =\left\{ x| A_1 x \leq %b_1 \right\}$ and $\mathcal{P}_2 =\left\{ x| A_2 x \leq %b_2 \right\}$. Then $\mathcal{P}_1 \subseteq %\mathcal{P}_2$ if and only if there exists a nonnegative %matrix $Y$ of appropriate dimensions such that
%$$Y A_1= A_2\ \text{and} \ Y b_1 \leq b_2.$$
%\end{lemma}


% \subsection{Kronecker Product}
% Let $A\in \R^{m\times n}$ and $B\in \R^{p\times q}$, the Kronecker product $A\otimes B$ is the $\R^{mp\times nq}$ matrix:\\
% $$A\otimes B = \begin{bmatrix} 
% a_{11} B & \cdots & a_{1n} B \\
% \vdots & \ddots & \vdots \\
% a_{m1} B & \cdots & a_{mn} B 
% \end{bmatrix}, $$
% and the following property will be used in this paper
% \begin{equation} \label{eq:kron}
%     \text{vec}(B^TX^TA^T) = (A\otimes B^T) \text{vec}(X^T).
% \end{equation}



