

\section{Data-Driven Safe Control} \label{sec:safe_no_w}


%%%-----------------------------
\subsection{Problem Statement}

% Consider the continuous-time control-affine system of the form
% \begin{equation} \label{eq:dynamics}
%     \dot{x} = f(x) + g(x) u
% \end{equation}
% where $x\in \R^n$ is the state, $u\in \R$ is the input, $f,g\colon \R^n \rightarrow \R^n$ are unknown polynomial functions, respectively. We assume the degrees of $f,g$ are known (or user-defined), then $f,g$ can be expressed by $f = F \phi$, $g = G \gamma$ where $F,G$ are unknown coefficient matrices and $\phi,\gamma$ are known monomial vectors in $x$, of proper dimensions. For example, for a two-state second-order polynomial system, $\phi,\gamma = [1,x_1,x_2,x_1^2,x_1x_2,x_2^2]^T$ and $F,G \in \R^{2\times 6}$.

The goal of this paper is to design a safe control law based on (noisy) experimental measurements for unknown polynomial systems where only minimal a-priori information  is available. Specifically, we consider control affine nonlinear  systems of the form
\begin{equation}\label{eq:dynamics_noise}
\dot{x}(t) = f(x) + g(x)u(t) + w(t),
\end{equation}
where the input $w$ satisfying $\forall t \geq 0 \colon w \in \mathcal{W} \subseteq \R^n$ represents an unknown random but bounded disturbance.  The only information available about the dynamics is that they can be expressed in terms of known dictionaries $\vt{\phi}({x}) \in \mathbb{R}^{s_f}, \vt{\gamma}({x}) \in \mathbb{R}^{s_g}$, that is
\beq\begin{aligned}\label{eq:pro1}
f(x) = \mt{F} \vt{\phi}({x}); \; g(x)= \mt{G} \vt{\gamma}({x}) 
\end{aligned}\eeq
for some unknown  system parameter matrices $\mt{F} \in \mathbb{R}^{n\times n_f}$ and $\mt{G} \in \mathbb{R}^{n \times n_g}$.
In this context, the problem under consideration can be formally stated as:

\begin{problem} \label{pb:1} 
% Given $T$ measurements of the triple $(\dot{x}_s, x_s, u_s)$, $s=t_1,\ldots,t_T$, 
Given  a data-collection noise bound $\epsilon>0$, a process disturbance description $w \in \mathcal{W}$ (e.g. $L_\infty$-bounded input), noisy derivative-state-input data $\dc = \{(y_s, x_s, u_s)\}_{s=t_1..t_T}$ under the relation $\norm{y_s - f(x_s) - g(x_s)u_s}_\infty \leq \epsilon$, and basic semialgebraic sets  $\mathcal{X}_0$, $\mathcal{X}_u$, find a state-feedback control law $u(x)$  that renders all closed-loop  systems consistent with the observed data and priors   robustly safe with respect to $\mathcal{X}_0$ and $\mathcal{X}_u$, for all  $w \in \mathcal{W}$.
\end{problem}

%\begin{remark}
   % The data $y_s$ is a noisy observation of the true derivative $\dot{x_s}$.
%\end{remark}

%\begin{remark}
%The solved control law $u(x)$ is not necessarily consistent with sample data $\dc$.
%\end{remark}

% In this section, we derive a data-driven algorithm to solve the problem \ref{pb:1}. 

%The data-driven Problem \ref{pb:1} will be approached by  define two polytopic sets in parameter space: one for $\dc$-consistent systems ($\mathcal{P}_1$) and one for safe systems ($\mathcal{P}_2$). Applying the Extended Farkas' Lemma to enforce set-containment leads to a polynomial feasibility problem, which can be solved with \ac{SOS} method by \ac{SDP} solvers.

\subsection{Model Based Safety} \label{sec:MBS}

In order to solve Problem \ref{pb:1}, in this section we first develop a  convex condition, less conservative than \eqref{eq:safe_rho}, that guarantees robust controlled safety of a model of the form \eqref{eq:dynamics_noise} assuming that $f(.)$ and $g(.)$ are known.

\begin{lemma}\label{lem:lemma2} Assume that the set $\mathcal{X}_u$ has a description of the form:
\[ \mathcal{X}_u \doteq \left \{x \colon h_i(x) \geq 0,\ i=1,\ldots, N_c \right \}. \]
Then, if there exist scalar functions $\rho(x), \psi(x) \in C^1$ such that: (i) $u(x) \doteq \frac{\psi(x)}{\rho(x)}$ is well defined over the safe region $\rho(x)>0$, (ii) for all $w \in \mathcal{W}$ and initial condition $x_0 \in \mathcal{X}_0$, the trajectories of \eqref{eq:dynamics_noise} are well defined, and (iii)
the following conditions hold:
\begin{subequations}\label{eq:MBS}
\begin{align}
&\nabla \cdot [\rho(x) \left(f (x) +w\right ) + \psi(x) g(x)] - \rho(x)h(x) > 0 \label{eq:MBS1} \\
& \text{$\forall x\in \R^n$ and $w \in \mathcal{W}$} \nonumber \\
&\rho (x) \geq 0, \ \forall x\in \mathcal{X}_0, \;
\rho (x)< 0, \ \forall x\in \mathcal{X}_u. 
%&u(x) \doteq \frac{\psi(x)}{\rho(x)} \; \text{bounded for all finite $x$}
\end{align}
\end{subequations}
where $h\doteq \min_i \left \{ h_i(x) \right \}$,
then the control law $u(x)$ renders the closed loop system robustly safe with respect to $\mathcal{X}_u$.
\end{lemma}
\begin{proof} Since by assumption $\rho,\psi \in C^1$ and $u$ is well defined, \eqref{eq:MBS1} is equivalent to (omit $x$):
\beq 
\frac{\partial \rho}{\partial x}(f+gu+w) + \rho \left (\nabla \cdot (f+gu) -h \right ) > 0
\eeq
where we used the fact that $\psi = \rho u$. Hence, for all $w \in \mathcal{W}$,
\[ \frac{d\rho}{dt}+ \rho\left (\nabla \cdot (f+gu) -h \right ) > 0\]
along the closed loop trajectories, which implies that $ \frac{d\rho}{dt} > 0$ when $\rho[x(t)]=0$.  Assume that there exists  a trajectory  $x(t,x_0,w)$ that starts at $x_0 \in \mathcal{X}_0$ and such that $x(T,x_0,w) \in \mathcal{X}_u$. By continuity, there exists some $0<t_1<T$ and some $dt$ such that $\rho(t_1)=0$ and $\rho(t)<0$ for all $t\in [t_1,t_1+dt]$. However, this contradicts the fact that  $\frac{d\rho}{dt} \left |_{t=t_1} > 0 \right.$.
\end{proof}
\begin{remark} Since $\min_i\left \{ h_i(x) \right \}$ has a semialgebraic representation, finding polynomial functions $\rho$ and $\psi$ reduces to 
\iac{SOS} optimization via standard 
%Putinar Positivstellensatz 
arguments.
\end{remark}
\begin{remark}  Problem \eqref{eq:MBS} is an infinite-dimensional \ac{LP} in the values of $(\rho, \psi)$ at each $x$, possessing both strict and non-strict inequality constraints. When compared against \eqref{eq:h_relaxation}, this formulation  has two advantages: (i) it avoids using an arbitrary, fixed multiplier $h(x)$, and (ii) it leads to jointly convex (in $\rho$ and $\psi$) optimization problems when searching for a control barrier and associated control action. On the other hand, \eqref{eq:MBS}, while retaining the desirable convexity properties of   \eqref{eq:safe_rho}, is less conservative: since the second term in \eqref{eq:MBS1} is nonnegative over the safe region, it does not require the first term to be positive everywhere, as is the case with  \eqref{eq:safe_rho}. Note that any feasible solution to  \eqref{eq:safe_rho} is also feasible for \eqref{eq:MBS}.
\end{remark}

\subsection{Safe Data Driven Control}\label{sec:SDDC}

This section presents the main result of the paper: a tractable, convex reformulation of Problem \ref{pb:1}.  In order to present these results, we begin by presenting  a tractable characterization of all systems that could have generated the observed data. 

Assume the sample data $\dc \doteq \left \{(y_s,x_s,u_s)\right \}$ is corrupted by a sample (offline) noise bounded by $\epsilon$. The consistency set $\mathcal{C}$, 
which contains all systems that are consistent with the data, is defined as:
\begin{equation}
    \mathcal{C} \doteq \left\{
    f,g\colon \|y_s-f(x_s)-g(x_s)u_s\|_\infty \leq \epsilon, s=t_1,\ldots,t_T \right\}.
\end{equation}
Recall that $f = \mt{F} \vt{\phi}$, $g = \mt{G} \vt{\gamma}$. Exploiting the following property of the Kronecker product \cite{petersen2008matrix} $$\text{vec}(P^TXQ^T) = (Q\otimes P^T) \text{vec}(X),$$ 
leads to the equivalent representation
\begin{equation} \label{eq:p1}
    \mathcal{C} = \left\{ 
    \vt{f},\vt{g} \colon 
    \begin{bmatrix}
        \mt{A} & \mt{B} \\ -\mt{A} & -\mt{B}
    \end{bmatrix} 
    \begin{bmatrix}
        \vt{f} \\ \vt{g}
    \end{bmatrix}
    \leq 
    \begin{bmatrix}
        \epsilon \mathbf{1} + \vt{\xi} \\ \epsilon \mathbf{1} - \vt{\xi}
    \end{bmatrix} \right\},
\end{equation}
where $\vt{f} = \text{vec} (\mt{F}^T)$, $\vt{g} = \text{vec} (\mt{G}^T)$ and
\begin{equation} \label{eq:abxi}
    \mt{A} \doteq \begin{bmatrix}
    \mt{I} \otimes \vt{\phi}^T(t_1) \\ \vdots \\ \mt{I} \otimes \vt{\phi}^T(t_T)
    \end{bmatrix},
    \mt{B} \doteq \begin{bmatrix}
    \mt{I} \otimes u\vt{\gamma}^T(t_1) \\ \vdots \\ \mt{I} \otimes u\vt{\gamma}^T(t_T)
    \end{bmatrix},
    \vt{\xi} \doteq 
    \begin{bmatrix}
    y(t_1) \\ \vdots \\ y(t_T)
    \end{bmatrix}.
\end{equation}

%\begin{assumption}
%The consistency set $\mathcal{C}$ is compact. %\label{assum:compact}
% ,thus enough data need to be collected until the matrix $ \begin{bmatrix} A & B \\ -A & -B \end{bmatrix}$ has full column rank. \urg{Reason?}
%\end{assumption}
%\begin{remark} A necessary condition for Assumption \ref{assum:compact} to hold is that $[A, B; -A, -B]$ has full column rank.
%\end{remark}
In order to establish robust safety, we need to add to this representation a description of all admissible disturbances. In the sequel, we will assume that this set  has a polytopic description of the form $\mathcal{W} \doteq \left \{ \vt{w} \colon \mt{W} \vt{w} \leq \vt{d}_w \right \}$. Combining this description with the description of $\mathcal{C}$ leads to an augmented consistency set describing the set of all possible plants and disturbances:
\begin{equation}\label{eq:P1}
    \mathcal{P}_1 \doteq \left\{ \vt{f,g,w} \colon
    \begin{bmatrix}
        \mt{A} & \mt{B} & 0\\
        -\mt{A} & -\mt{B} & 0\\
        0 & 0 & \mt{W}
    \end{bmatrix}
    \begin{bmatrix}
        \vt{f}\\ \vt{g} \\ \vt{w}
    \end{bmatrix}
    \leq 
    \begin{bmatrix}
        \epsilon \mathbf{1} + \xi\\
        \epsilon \mathbf{1} - \xi\\
        \vt{d}_w\\
        \end{bmatrix} \right\}.
\end{equation}


%\begin{remark}
%The polytope $\mathcal{P}_1$ is defined by $2 nT + n_w$ linear %inequality constraints. Redundant constraints may be eliminated %using iterative linear programming \cite{caron1989degenerate}.
%\end{remark}

It follows that a pair $(\rho,\psi)$ solves Problem \ref{pb:1} if 
\beq \label{eq:allfgw}
\nabla \cdot [\rho f (x) + \psi g(x) + \rho w] - \rho(x)h(x)> 0 \eeq
holds for all $x$ and all $(\vt{f},\vt{g},\vt{w}) \in \mathcal{P}_1$. In principle, this condition can be reduced to an \ac{SOS} optimization over the coefficients of $\rho,\psi$ by a straight application of Putinar's Positivstellensatz. However, this approach quickly becomes intractable, since it requires considering polynomials in the indeterminates $(\vt{x},\vt{f},\vt{g},\vt{w})$ with a total dimension $n_p=n_f+n_g+2n$. Thus, for a relaxation of order $n_r$ the total number of variables in the optimization is
$\binom{n_r+n_p}{n_p}$. As an example,  for a second order system with polynomial dynamics of degree 2, we have $n_f =n_g=6$. If $\rho$ and $\psi$ are also limited to degree 2 polynomials, a relaxation of order $n_r=3$ involves 969 variables.  As we show next, computational complexity can be substantially reduced by exploiting duality. 

For a given pair $(\rho,\psi)$, consider the set of all systems of the form \eqref{eq:dynamics_noise} that are rendered safe by the control action $u=\frac{\psi}{\rho}$, along with the corresponding admissible perturbations, that is, the set of all
$(\vt{f},\vt{g},\vt{w})$ such that \eqref{eq:allfgw} holds for all $x \in \R^n$. For each $x$, this set is a polytope of the form:
\begin{equation}
    \mathcal{P}_2 \doteq \left\{\vt{f},\vt{g},\vt{w}\colon -
    \begin{bmatrix}
        (\nabla\cdot(\rho\phi^T))^T \\  
        (\nabla\cdot(\psi\gamma^T))^T\\
        (\nabla\rho)^T
    \end{bmatrix}^T
    \begin{bmatrix}
        \vt{f} \\ \vt{g} \\ \vt{w}
    \end{bmatrix} < -\rho h\right\}.
\end{equation}

It follows that \eqref{eq:allfgw} holds for all admissible disturbances $\vt{w} \in \mathcal{W}$ and all plants in the consistency $\mathcal{C}$ set if and only if $\mathcal{P}_1 \subseteq \mathcal{P}_2$.  This inclusion can be enforced through duality as follows:
%the extended Farkas' Lemma as follows:

\begin{lemma}\label{lem:farkas}   Assume that the data and priors are consistent (e.g. $\mathcal{C} \not = \emptyset$) and that  enough data has been collected so that $\mathcal{C}$ is compact. Then 
$\mathcal{P}_1 \subseteq \mathcal{P}_2$ if and only if there exists a vector function $\vt{y}(x)\geq 0, \vt{y}(x) \in \R^{2nT+2n}$ such that the following functional set of affine constraints is feasible:
\begin{equation}\label{eq:EFL}
    \vt{y}^T(x) \mt{N} = \vt{r}(x) \; \text{and} \;
    \vt{y}^T(x) \vt{e}  < -\rho(x)h(x)
\end{equation}
where
\begin{equation}
    \begin{aligned} \label{eq:Ndef}  & \mt{N} \doteq 
    \begin{bmatrix}
    \mt{A} & \mt{B} & 0\\
    -\mt{A} & -\mt{B} & 0 \\
    0 & 0 & \mt{W}\\
    \end{bmatrix}, 
    \; \vt{e} \doteq 
    \begin{bmatrix}
        \epsilon \mathbf{1} + \vt{\xi}\\
        \epsilon \mathbf{1} - \vt{\xi}\\
      \vt{d}_w
    \end{bmatrix}, \\
    & \vt{r}(x)\doteq -
    \begin{bmatrix}
        \nabla\cdot(\rho\vt{\phi}^T) & \nabla\cdot(\psi\vt{\gamma}^T) & \nabla\rho
    \end{bmatrix}.
    \end{aligned}
\end{equation}
\end{lemma}
\begin{proof} Since $\mathcal{C}$ is compact, from section 5.8.3 in \cite{boyd2004convex} it follows that the systems of inequalities
\beq \label{eq:alternatives}
  %  \begin{aligned}
       \begin{bmatrix} \mt{N} \\ -\vt{r} \end{bmatrix}  \begin{bmatrix}
        \vt{f} \\ \vt{g} \\ \vt{w}
    \end{bmatrix} \leq \begin{bmatrix} \vt{e} \\ \rho(x)h(x) \end{bmatrix} \; \textrm{and} \;\begin{array}{r}
    \vt{y}^T\mt{N}-\mu \vt{r} = 0 \\
     \vt{y}^T\vt{e} + \mu \rho(x)h(x) < 0 \\
    \vt{y} \geq 0, \; \mu \geq 0
    \end{array}
   % \end{aligned}
\eeq
are strong alternatives. Further, since $\mathcal{C} \not = \emptyset$ and $\mu >0$, we can take $\mu=1$ without loss of generality. Thus \eqref{eq:EFL} holds if and only if the left set of inequalities in \eqref{eq:alternatives} is infeasible. This implies that if \eqref{eq:EFL} holds,  a triple $(\vt{f,g,w}) \in \mathcal{P}_1$ if and only if 
$\begin{bmatrix}\vt{f}^T\; \vt{g}^T\;\vt{w}^T \end{bmatrix} \vt{r}^T < -\rho(x)h(x) $, that is  $(\vt{f,g,w}) \in \mathcal{P}_2$.
%Follows from direct application of the Extended Farkas' Lemma.
\end{proof}
\begin{remark} Proceeding as in Theorem 2 in \cite{dai2020semi}, it can be shown that  if $\vt{\phi}(\mat{x}), \vt{\gamma}(\mat{x})$ are continuous functions, then $\vt{y}({x})$ can be chosen to be continuous.
%\textcolor{red}{Further, if  $\vt{\phi},\vt{\gamma},\rho,\psi$ are polynomial and the trajectories of the closed-loop system stay in a compact region, then, without loss of generality $\vt{y}({x})$  can be taken to be polynomial. 
\end{remark}

Combining the observations above leads to
the main result of this paper:
\begin{theorem}\label{thm:main}
% Given noisy data of a , a sufficient condition for there to  exist 
A sufficient condition for the existence of a state-feedback control law $u(x)$ such that all systems in the consistency set $\mathcal{C}$ are rendered robustly safe, is that there exists a continuous vector function $\vt{y}(x) \geq 0$ and functions $\rho \in C^1$, $\psi\in C^1$ 
%with $\frac{\psi(x)}{\rho(x)} \in C^0$
such that
\begin{subequations}\label{eq:thm_main} 
\begin{align}
\vt{y}^T(x)\mt{N} &= \vt{r}(x), \ \forall x \in \R^n\label{eq:thm_main1}\\
\vt{y}^T(x)\vt{e}  & < -\rho(x)h(x), \ \forall x \in \R^n \label{eq:thm_main2}\\
|\psi(x)| &\leq -\rho(x)h(x),\ \forall x\in \R^n \label{eq:thm_main3}\\
\rho(x) & \geq 0,\ \forall x\in\mathcal{X}_0\\ 
\rho(x) & < 0, \ \forall x\in\mathcal{X}_u.
\end{align}
\end{subequations}
The control law $u$ can then be extracted by the division $u(x) = \psi(x)/\rho(x)$.
% And the control law can be extracted as $u=\psi/\rho$, which is a rational polynomial function. 
\end{theorem}
\begin{proof} The proof follows from the fact that from Lemma \ref{lem:farkas}, \eqref{eq:thm_main1} and \eqref{eq:thm_main2} guarantee that \eqref{eq:allfgw} holds for all plants in $\mathcal{C}$ and all admissible disturbances $\vt{w} \in \mathcal{W}$. Hence the conditions in Lemma \ref{lem:lemma2} hold for all plants that could have generated the observed data.
\end{proof}
    
\begin{remark}
Note that \eqref{eq:thm_main3} is a convex tightening of the condition that $\psi=0$ when $\rho=0$, %Simultaneously, it  leads to a bounded-control $|u|\leq c$ as $u=\psi/\rho$ by construction.
and for the safety concern, we focus on the region where it is safe.
\end{remark}

\subsection{Sum-of-Squares Safety Program}
\label{sec:sos_safety}
In order to solve the infinite-dimensional Problem \eqref{eq:thm_main} in a tractable manner, we restrict the variables $\rho, \psi, \vt{y}$ to be polynomials. Under this polynomial restriction, the extracted controller $u(x) = \psi(x)/\rho(x)$ is then a rational function.
% Note that \eqref{eq:thm_main} is a feasibility polynomial problem, which can be solved using SOS program. 



% . The set of \ac{SOS} polynomials is a strict subset of all nonnegative polynomials, but 

% \subsection{SOS Polynomials}
% This paper shows that the data-driven safe control synthesis can be formulated as a polynomial nonnegativity  problem, which is generally NP-hard (for polynomials of even degree $d\geq 4$). This difficulty can be handled by relaxing the problem to establishing that a given polynomial is a Sum of Squares. Consider a polynomial $f$ of degree $2m$. A sufficient condition to $\forall x \in \R^n: f(x)\geq 0$ is that $f$ can be written as a \ac{SOS} 
% $f = \textstyle \sum_i g_i^2.$
% Let $v$ be a vector with all monomials of degree less than or equal to $m$. Then $f$ is \ac{SOS} if and only if there exists a positive semidefinite matrix $Q$ such that
% $f = v^T Qv$ \cite{parrilo2000structured}.
% Comparing term on the both sides leads to an \ac{SDP} program, which can be solved by off-the-shelf solvers. \textcolor{red}{do we need to mention the Psatz?}


Let $\mathcal{X}_0 \doteq \{x\colon k(x) \geq 0\}$ and $\mathcal{X}_u \doteq \{x\colon h(x) \geq 0\}$ denote the initial condition and unsafe sets, respectively. Algorithm 1 is \iac{SOS}-based finite-degree tightening of \eqref{eq:thm_main} for robustly safe control. Successful execution of algorithm \ref{alg:1} is sufficient for finding a robustly safe control law. 
%\urg{note that (A.2) imposes non-negativity instead of positivity}
% a detailed algorithm for data-driven robustly safe control is given in Alg. \ref{alg:1}.
\begin{algorithm}[h]
    \caption{Data-Driven Safe Control Design} \label{alg:1}
    \begin{algorithmic}
        \State Input: sample data $\dc$, and degrees of $f,g,\rho,\psi$
        \State Let 2$d_1\geq\text{max}\left\{d_f+d_\rho,d_g+d_\psi \right\}, 2d_2 \geq \text{max}\left\{d_\rho,d_\psi \right\}$
        \State Solve: the feasibility problem
        \[ \begin{array}{rcl}
        \textrm{coeff}_x(\vt{y}^T(x) \mt{N} - \vt{r}(x))&=0                              \hfill(A.1) \\
        -\rho(x)h(x) - \vt{y}^T(x) \vt{e} - c_1\; &\in \Sigma_{d_1}[x] \hspace{1cm}   \hfill(A.2)\\
        -h\rho - \psi   \; &\in \Sigma_{d_2}[x]   \hfill(A.3)\\
        -h\rho + \psi   \; &\in \Sigma_{d_2}[x]   \hfill(A.4)\\
        \rho - s_1k  \; &\in \Sigma_{d_2} [x]  \hfill(A.5)\\
        -\rho - s_2h - c_2\; &\in \Sigma_{d_2} [x]  \hfill(A.6)\\
        \vt{y}_i            \; &\in \Sigma_{d_1}[x]    \hfill(A.7) \\
        c_1, c_2       \; &> 0    \hfill(A.8) \\
        s_1, s_2       \; &\in \Sigma_{d_2}[x]    \hfill(A.9)\\
        \end{array} \]
        \State Output: the safe control law $u = \psi/\rho$ or a certificate of infeasibility at degree $(d_1, d_2)$ 
    \end{algorithmic}
\end{algorithm}

% \urg{note about lack of convergence guarantee?}