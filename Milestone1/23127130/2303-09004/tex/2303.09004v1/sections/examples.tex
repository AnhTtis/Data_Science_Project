\section{Numerical Examples} \label{sec:examples}

The proposed algorithm is tested on a pair of examples. Both experiments are implemented in MATLAB 2020b with Yalmip \cite{lofberg2004yalmip} and solved by Mosek \cite{mosek92}. Code to generate experiments and plots is publicly available at \url{https://github.com/J-mzz/ddc-safety}. %All experiments have an input bound of $c=1$.

\begin{example} Consider the Flow system \cite{rantzer2004analysis} with
\begin{align}
\label{eq:expr_1_dyn}
    f &= \begin{bmatrix}
    x_2 \\ -x_1 + \frac{1}{3} x_1^3 - x_2
    \end{bmatrix}, &
    g &= \begin{bmatrix}
    0 \\ 1
    \end{bmatrix}.
\end{align}

The initial and unsafe sets are the (unions of) disks:
% $$
% \begin{aligned}
%     \mathcal{X}_0 & = \{x \mid 0.25-x_1^2-(x_2+3)^2 \geq 0\},\\
%     \mathcal{X}_{u_1} & = \{x \mid 0.16-(x_1+1)^2-(x_2+1)^2\geq 0\},\\
%     \mathcal{X}_{u_2} & = \{x \mid 0.16-(x_1+1)^2-(x_2-1)^2\geq 0\}.
% \end{aligned}
% $$
$$
\begin{aligned}
    \mathcal{X}_0 = \{x \mid \  &0.25-x_1^2-(x_2+3)^2 \geq 0\},\\
    \mathcal{X}_u = \{x \mid \ &h_1(x)=0.16-(x_1+1)^2-(x_2+1)^2\geq 0 \ , \\
    \textrm{OR} \ &h_2(x)=0.16-(x_1+1)^2-(x_2-1)^2\geq 0\}.
\end{aligned}
$$
% \begin{figure}[b]
%     \centering
%     \includegraphics[width=0.7\linewidth]{figures/noisy data.png}
%     \caption{Noisy and observed data for system \eqref{eq:expr_1_dyn} \urg{not a good example: noisy and GT are on top of each other. Reviewers will kill us}}
%     \label{fig:noisy_data} 
% \end{figure}




% \urg{

% Report the number of faces and vertices in the polytope $\mathcal{P}_1$.

% What is the rational control law that you discover? Write $\rho, \psi_1, \psi_2$

% Consider plotting the reference trajectory, or the clean/corrupted data that you used to learn. Look at \href{https://github.com/Jarmill/data_driven_occ/blob/27d9a2e28b7882f6dbf795c3792a2cde2a7f208b/data_class/data_generator/data_generator.m#L296}{data\_plot\_2} for a method to plot the data vectors.

% }
\begin{figure}
     \centering
          \begin{subfigure}[b]{0.49\columnwidth}
         \centering
         \includegraphics[width=\linewidth]{figures/open_loop.png}
         \caption{open-loop}
         \label{fig:ex1_open}
     \end{subfigure}
     \hfill
     \begin{subfigure}[b]{0.49\columnwidth}
         \centering
         \includegraphics[width=\linewidth]{figures/robust_with_noise.png}
         \caption{robust closed-loop}
         \label{fig:ex1_robust}
     \end{subfigure}
        \caption{Flow \eqref{eq:expr_1_dyn} simulations for Example 1}
        \label{fig:sim_ex1}
        \vspace{-1.5em}
\end{figure}

\begin{figure}
     \centering
     \begin{subfigure}[b]{0.49\columnwidth}
         \centering
         \includegraphics[width=\linewidth]{figures/nonrobust_without_noise.png}
         \caption{no process noise}
         \label{fig:ex1_nonrobust_no_noise}
     \end{subfigure}
     \hfill
     \begin{subfigure}[b]{0.49\columnwidth}
         \centering
         \includegraphics[width=\linewidth]{figures/nonrobust_with_noise.png}
         \caption{with process noise}
         \label{fig:ex1_nonrobust}
     \end{subfigure}
        \caption{Safe controllers synthesized without process noise may be unsafe when process noise is applied}
        \label{fig:sim_ex1_2}
        \vspace{-1.5em}
\end{figure}

% \begin{figure}[b]
%     \centering
%     \includegraphics[width=0.7\linewidth]{figures/nonrobust.png}
%     \caption{Example 1 designed without online noise}
%     \label{fig:ex1 w/o online noise} 
% \end{figure}

Results of the control design for Example 1 are shown in Fig. \ref{fig:sim_ex1} and \ref{fig:sim_ex1_2}. In each figure, 30 trajectories (blue curves) start from within the initial set $\mathcal{X}_0$ (black circle). The unsafe set $\mathcal{X}_u$ is the pair of red disks, implemented as $h(x) = -h_1(x)h_2(x) \geq 0$. Some of the open-loop trajectories Fig. \ref{fig:ex1_open} enter the unsafe set $\mathcal{X}_u$ when starting in $\mathcal{X}_0$.

The prior knowledge of the system model is that $f$ is a two-dimensional cubic polynomial vector with $f(\mathbf{0})=\mathbf{0}$ and that $g$ is a two-dimensional constant vector,
%$f = [ \textrm{cubic}_1(x_1, x_2); \textrm{cubic}_2(x_1, x_2)]$ with $f([0; 0]) = [0; 0]$ and $g = [\textrm{constant}_1; \textrm{constant}_2]$, 
where the cubic polynomials in $f$ and the constant terms in $g$ are both unknown. 80 datapoints were collected and used to design a robust safe controller under a sampling noise and a process noise bound of $\epsilon=\epsilon_w=2$, yielding a polytope $\mathcal{P}_2$ from \eqref{eq:P1} with 22 dimensions ($\text{dim}(\vt{f})=18, \ \text{dim}(\vt{g})=2, \ \text{dim}(\vt{w})=2$) and 324 faces (91 of the faces $\mathcal{P}_2$ are nonredundant \cite{caron1989degenerate}). Algorithm \ref{alg:1} was solved to find $\rho,\psi \in \R[x]_{\leq 4}$ yielding the rational control law $u=\psi/\rho$. Fig. \ref{fig:ex1_robust} plots trajectories associated with this safe control law, and also features the $\rho=0$ level set in green.


Fig. \ref{fig:sim_ex1_2} highlights the importance of robustness in execution as well as in data-collection. The controller in Fig. \ref{fig:sim_ex1_2} was computed with the same noisy observed data as in Fig. \ref{fig:sim_ex1} but with $\epsilon_w = 0$. The left plot in Fig. \ref{fig:ex1_nonrobust_no_noise} shows that the control is safe under noiseless trajectory execution. The right plot is zoomed into the lower red disk, and  demonstrates that some controlled trajectories pass through the $\rho=0$ contour and  enter $\mathcal{X}_u$ when process noise $\norm{w}_\infty\leq 2$ is applied in execution (trajectories are terminated when $u\geq 10^4$, which is caused numerically near the $\rho=0$ contour).

% Vertical lines in Fig. \ref{fig:ex1_nonrobust_no_noise} are caused by numerical issues involving actions $u \approx 10^{12}$ (rational controller).

% \old{and 80 samples are generated with sample noise bounded by $\epsilon = 2$. This experiment involves the design a robust safe control law with $\rho,\psi \in \R[x]_{\leq 4}$ under disturbances $w$ with $\|w\|_\infty \leq 2$. }

% The green curve is the $0$-level-set of $\rho=0$.

% \old{
% % The black and the red circles represent the initial and the unsafe set respectively, the green line is the contour of $\rho =0$.
% Trajectories % (closed-loop in Fig. \eqref{fig:ex1_robust} and open-loop in Fig. \eqref{fig:ex1_open}) 
% % are displayed in the 30 blue lines starting on the boundary of $\mathcal{X}_0$. % Different trajectories arise from the same initial point $x_0$ due to the influence of process noise $\omega(t)$ during execution.
% Fig. \eqref{fig:ex1_robust} shows the closed-loop trajectories of robust controller with process noise bound of $\epsilon_w = 2$, Fig. \eqref{fig:ex1_open} shows the open-loop , Fig. \eqref{fig:ex1_nonrobust_no_noise} and \eqref{fig:ex1_nonrobust} shows the closed-loop of nonrobust controller, which is designed without concerning the process noise, without and with the same bound process noise in executions.}

% Observations from Fig. \ref{fig:sim_ex1} and \ref{fig:sim_ex1_2} are:
% \begin{itemize}
%     \item $\rho=0$ separates the initial set $\mathcal{X}_0$ and the unsafe set $\mathcal{X}_u$.
%     \item $\rho\geq 0$ is an invariant set when the robustly-safe control $u$ is applied.
%     \item Uncontrolled (Fig. \ref{fig:ex1_open}) and  non-robustly-safe (Fig. \ref{fig:ex1_nonrobust}) trajectories may enter $\mathcal{X}_u$.
% \end{itemize}

% We briefly note about the savings in computational complexity due to using the Alternatives scheme. 
Using a Putinar Positivstellensatz on \eqref{eq:allfgw} with the 24 variables $(\vt{x}, \vt{f}, \vt{g}, \vt{w})$ at degree $d_1$ requires a Gram matrix of maximal size $\binom{24 + d_1}{d_1}$. In contrast, Algorithm \ref{alg:1} requires 99 Gram matrices of maximal size $\binom{2+d_1}{d_1}$ from (\textit{A}.2) and (\textit{A}.8) along with new equality constraints (\textit{A}.1). At the current value of $d_1 = 4$, the maximal Gram matrix size drops from $\binom{28}{4} =20475$ to $\binom{6}{4} = 15$.

To summarize this example, $\rho\geq 0$ is an invariant set for all consistent systems under online noise when the robust controller is applied. The level set $\rho=0$ will separates initial set $\mathcal{X}_0$ and the unsafe set $\mathcal{X}_u$. Uncontrolled (Fig. \ref{fig:ex1_open}) and  non-robustly-safe (Fig. \ref{fig:ex1_nonrobust}) trajectories may enter $\mathcal{X}_u$.
\end{example}

\begin{example} Consider the Twist system \cite{miller2022bounding} with:

\begin{align}
\label{eq:expr_2_dyn}
    f&= \begin{bmatrix}-2.5x_1 + x_2 - 0.5x_3 + 2x_1^3+2x_3^3 \\
    -x_1+1.5x_2+0.5x_3-2x_2^3-2x_3^3 \\
    1.5 x_1 + 2.5x_2 - 2 x_3 - 2x_1^3 - 2 x_2^3\end{bmatrix}, & g&= \begin{bmatrix} 0 \\ 0\\  1 \end{bmatrix}.
\end{align}

% \begin{align}
% \label{eq:expr_2_dyn}
%     f = \sum_j \mt{E_{ij}}x_j -\mt{F_{ij}}(4x_j^3-3x_j)/2, \ \
%     g =  [0, 0, 1]^T,\\
%     \mt{E} = \begin{bmatrix}
%     -1&1&1 \\ -1&0&-1\\ 0&1&-2
%     \end{bmatrix}, \ \ 
%     \mt{F} = \begin{bmatrix}
%     -1&0&-1 \\ 0&1&1\\ 1&1&0
%     \end{bmatrix}.
% \end{align}
The initial and unsafe sets are the spheres:
$$
\begin{aligned}
    \mathcal{X}_0 & = \{x \mid 0.01-(x_1+0.5)^2-x_2^2-x_3^2 \geq 0\},\\
    \mathcal{X}_u & = \{x \mid 0.01-(x_1+0.1)^2-x_2^2-x_3^2\geq 0\}.
\end{aligned}
$$

Results of Example 2 are are shown in Fig. \ref{fig:sim_ex2}. Trajectories start within the initial set $\mathcal{X}_0$ (black sphere), and some of the open-loop trajectories \ref{fig:ex2_wo_u} will enter the unsafe set $\mathcal{X}_u$ (red sphere). The prior knowledge of the system model is that $f$ is a three-dimensional cubic polynomial vector with $f(\mathbf{0})=\mathbf{0}$ and that $g$ is a three-dimensional constant vector. 80 datapoints were collected and used to design a robust safe controller under a sampling noise and a process noise bound of $\epsilon=\epsilon_w=1$, yielding a polytope $\mathcal{P}_2$ with 63 dimensions ($\text{dim}(\vt{f})=38, \ \text{dim}(\vt{g})=3, \ \text{dim}(\vt{w})=3$) and 304 faces (all of them are nonredundant). Solving Algorithm \ref{alg:1} to find $\rho,\psi \in \R[x]_{\leq 4}$ yields a rational control law $u=\psi/\rho$. Figure \ref{fig:ex2_w_u} features the $\rho=0$ level set surface in green.

\begin{figure}
     \centering
     \begin{subfigure}[b]{0.49\columnwidth}
         \centering
         \includegraphics[width=\linewidth]{figures/open_3d.png}
         \caption{open-loop}
         \label{fig:ex2_wo_u}
     \end{subfigure}
     \hfill
     \begin{subfigure}[b]{0.49\columnwidth}
         \centering
         \includegraphics[width=\linewidth]{figures/robust_3d.png}
         \caption{robust closed-loop}
         \label{fig:ex2_w_u}
     \end{subfigure}
        \caption{Twist \eqref{eq:expr_2_dyn} simulations for Example 2.}
        \label{fig:sim_ex2}
        \vspace{-1.5em}
\end{figure}
\end{example}


% \old{Results of the control design are shown in Fig. \ref{fig:sim_ex2}. The black and the red spheres represent the initial and the unsafe set respectively, the green surface is the level set of $\rho =0$.
% Trajectories (open-loop in Fig. \eqref{fig:ex2_wo_u} and closed-loop in Fig. \eqref{fig:ex2_w_u}) are displayed in the 30 blue lines starting on the boundary of $\mathcal{X}_0$.}