\section{Introduction} \ref{fig:ex1_nonrobust_no_noise}\label{sec:introduction}


This work finds safe full-state feedback control laws that are robust under $L_\infty$-bounded uncertainties in both data-collection and online dynamics (process noise). The \ac{DDC} problem over polynomial dynamics is posed as the feasibility of \iac{SOS} program. A density-function based control law may be derived from a feasible \ac{SDP} solution, thus certifying safety of all closed-loop trajectories with respect to a given initial and unsafe set. 

Trajectories starting from the initial set are safe if no point along the trajectories is ever inside the unsafe set. Level-set methods separate the initial and unsafe set by the $0$-contour of a solved function.
Barrier functions \cite{prajna2004safety} are a level set method to certify the safety of trajectories, given that the superlevel sets of the barrier function are invariant. This superlevel invariance can be relaxed through slack (class-K) conditions while ensuring that the $0$-level set is invariant \cite{ames2019control}. The level-set certificate of stability may be solved jointly with a safety-guaranteeing control policy $u(\cdot)$ (`control barrier function'). An alternative level-set certificate is  Density \cite{rantzer2004analysis} functions, which are based on Dual Lyapunov methods for stability \cite{rantzer2001dual}.   \urg{why choose Barrier vs Density? discuss here.} Other methods to ensure safety of controlled trajectories include funnels \cite{majumdar2013control}, Hamilton-Jacobi reachability \cite{bansal2017hamilton, chen2018hamilton}, Mixed Monotonicity \cite{coogan2020mixed}, and Reinforcement Learning \cite{brunke2022safe}.



% \urg{Literature review here.}


% \urg{Safe Control}

% \urg{Data-Driven Control, Data-Driven Safety}

\ac{DDC} is a methodology that formulates control laws directly from acquired system observations and skips a system-identification/robust-synthesis pipeline \cite{HOU20133, formentin2014comparison, hou2017datasurvey}. This paper utilizes a Set-Membership approach to \ac{DDC}: furnishing a control law $u(\cdot)$ with a certificate that the set of all data-consistent plants are contained within the set of all $u$-stabilized plants. The certificate in this work is expressed as \ac{SOS} nonnegativity proofs of linear inequalities over the basic-semialgebraic initial and unsafe sets \cite{parrilo2000structured}. Prior work that performs Set-Membership \ac{DDC} using \ac{SOS} proofs includes \cite{dai2020semi, martin2021data, miller2022eiv_short}. Other containment proofs used for Set-Membership \ac{DDC} include Farkas certificates for polytope containment \cite{cheng2015robust}, and a Matrix S-Lemma for Quadratic Matrix Inequalities \cite{waarde2020noisy}.
Other methods for \ac{DDC} include Iterative Feedback Tuning \cite{hjalmarsson1998iterative}, Virtual Reference Feedback Tuning \cite{campi2002virtual, formentin2012non}, Behavioral characterizations (Willem's Fundamental Lemma) \cite{willems2005note, depersis2020formulas, coulson2019data, berberich2021robustmpc}, and moment-based proofs for switching control  \cite{dai2018moments}.

Prior art for \ac{DDC} under safety constraints include \cite{rosolia2018learning, lopez2021robust, dacs2022robust}. To the best of our knowledge, our approach is the first \ac{DDC} method under safety constraints that simultaneously considers data-collection and online-dynamics noise.
% \vspace{-2cm}
% \urg{(outside of $H_2$ noise?)}.


% \urg{What makes our algorithm novel.}
\section{Introduction} \label{sec:introduction}


This work finds safe full-state feedback control laws that are robust under $L_\infty$-bounded uncertainties in both data-collection and online dynamics (process noise). The \ac{DDC} problem over polynomial dynamics is posed as the feasibility of \iac{SOS} program. A density-function based control law may be derived from a feasible \ac{SDP} solution, thus certifying safety of all closed-loop trajectories with respect to a given initial and unsafe set. 

Trajectories starting from the initial set are safe if no point along the trajectories is ever inside the unsafe set. Level-set methods separate the initial and unsafe set by the $0$-contour of a solved function.
Barrier functions \cite{prajna2004safety} are a level set method to certify the safety of trajectories, given that the superlevel sets of the barrier function are invariant. This superlevel invariance can be relaxed through slack (class-K) conditions while ensuring that the $0$-level set is invariant \cite{ames2019control}. The level-set certificate of stability may be solved jointly with a safety-guaranteeing control policy $u(\cdot)$ (`control barrier function'). An alternative level-set certificate is  Density \cite{rantzer2004analysis} functions, which are based on Dual Lyapunov methods for stability \cite{rantzer2001dual}.   \urg{why choose Barrier vs Density? discuss here.} Other methods to ensure safety of controlled trajectories include funnels \cite{majumdar2013control}, Hamilton-Jacobi reachability \cite{bansal2017hamilton, chen2018hamilton}, Mixed Monotonicity \cite{coogan2020mixed}, and Reinforcement Learning \cite{brunke2022safe}.



% \urg{Literature review here.}


% \urg{Safe Control}

% \urg{Data-Driven Control, Data-Driven Safety}

\ac{DDC} is a methodology that formulates control laws directly from acquired system observations and skips a system-identification/robust-synthesis pipeline \cite{HOU20133, formentin2014comparison, hou2017datasurvey}. This paper utilizes a Set-Membership approach to \ac{DDC}: furnishing a control law $u(\cdot)$ with a certificate that the set of all data-consistent plants are contained within the set of all $u$-stabilized plants. The certificate in this work is expressed as \ac{SOS} nonnegativity proofs of linear inequalities over the basic-semialgebraic initial and unsafe sets \cite{parrilo2000structured}. Prior work that performs Set-Membership \ac{DDC} using \ac{SOS} proofs includes \cite{dai2020semi, martin2021data, miller2022eiv_short}. Other containment proofs used for Set-Membership \ac{DDC} include Farkas certificates for polytope containment \cite{cheng2015robust}, and a Matrix S-Lemma for Quadratic Matrix Inequalities \cite{waarde2020noisy}.
Other methods for \ac{DDC} include Iterative Feedback Tuning \cite{hjalmarsson1998iterative}, Virtual Reference Feedback Tuning \cite{campi2002virtual, formentin2012non}, Behavioral characterizations (Willem's Fundamental Lemma) \cite{willems2005note, depersis2020formulas, coulson2019data, berberich2021robustmpc}, and moment-based proofs for switching control  \cite{dai2018moments}.

Prior art for \ac{DDC} under safety constraints include \cite{rosolia2018learning, lopez2021robust, dacs2022robust}. To the best of our knowledge, our approach is the first \ac{DDC} method under safety constraints that simultaneously considers data-collection and online-dynamics noise.
% \vspace{-2cm}
% \urg{(outside of $H_2$ noise?)}.


% \urg{What makes our algorithm novel.}
\begin{comment}
The goal of this paper is to develop a tractable framework for data-driven synthesis of safe control laws that are robust to $L_\infty$-bounded noise  in  data-collection and unknown but bounded disturbances during executions.
Trajectories starting from an initial set are safe if no point along the trajectories is ever inside the unsafe set. Level-set methods separate the initial and unsafe set by the $0$-contour of a solved function.
Barrier functions \cite{prajna2004safety} are a level set method to certify the safety of trajectories, given that the superlevel sets of the barrier function are invariant. This superlevel invariance can be relaxed through slack (class-K) conditions while ensuring that the $0$-level set is invariant \cite{ames2019control, xiao2019control}. The level-set certificate of stability may be solved jointly with a safety-guaranteeing control policy $u(\cdot)$ (`control barrier function'). When a barrier function is given, the min-norm controller will ensure safety of trajectories and can be found through quadratic programming \cite{freeman2008robust, ames2014control}. Robustness of given barrier functions to disturbances may be analyzed using input-to-state stability \cite{XU2015robustcbf}.
Barrier functions and funnels \cite{majumdar2013control} contain bilinearities when jointly synthesizing controllers and barriers.
An alternative level-set certificate is  Density \cite{rantzer2004analysis} functions, which are based on Dual Lyapunov methods for stability \cite{rantzer2001dual}. Controllers and density functions can be simultaneously solved for in a convex manner while also including input actuation constraints. Density functions may exist and provide improved performance as compared to  barrier functions in some systems \cite{chen2020densityvalue}.

We briefly compare against other methods of safety-constrained control.
Interval analyses such as Mixed Monotonicity \cite{coogan2020mixed} offer real-time performance at the expense of conservatism in safe generation.
Hamilton-Jacobi reachability \cite{bansal2017hamilton, chen2018hamilton} performs forward and backwards reachable set analysis based on level sets of a differential games' value function, whose computation could require solving PDEs or neural net approximations.
Reinforcement Learning necessitates training and prior information of safety properties (e.g. Lipschitz bounds on dynamics), and does not generally exploit physical principles and model structure \cite{brunke2022safe}. Koopman methods allow for predictive capabilities of nonlinear models \cite{korda2020optimal, otto2021koopman}, but they contain error bounds that can conflict against safety certification \cite{folkestad2020data}.

 \end{comment}