% !TEX TS-program = pdflatex
% !TEX encoding = UTF-8 Unicode

% This is a simple template for a LaTeX document using the "article" class.
% See "book", "report", "letter" for other types of document.

\documentclass[11pt]{article} % use larger type; default would be 10pt

\usepackage[utf8]{inputenc} % set input encoding (not needed with XeLaTeX)

%%% Examples of Article customizations
% These packages are optional, depending whether you want the features they provide.
% See the LaTeX Companion or other references for full information.

%%% PAGE DIMENSIONS
\usepackage{geometry} % to change the page dimensions
\geometry{a4paper} % or letterpaper (US) or a5paper or....
% \geometry{margin=2in} % for example, change the margins to 2 inches all round
% \geometry{landscape} % set up the page for landscape
%   read geometry.pdf for detailed page layout information

\usepackage{graphicx} % support the \includegraphics command and options
\usepackage[section]{placeins}
% \usepackage[parfill]{parskip} % Activate to begin paragraphs with an empty line rather than an indent

%%% PACKAGES
\usepackage{booktabs} % for much better looking tables
\usepackage{array} % for better arrays (eg matrices) in maths
\usepackage{paralist} % very flexible & customisable lists (eg. enumerate/itemize, etc.)
\usepackage{verbatim} % adds environment for commenting out blocks of text & for better verbatim
\usepackage{subfig} % make it possible to include more than one captioned figure/table in a single float
% These packages are all incorporated in the memoir class to one degree or another...
\usepackage{mathtools, natbib}
\usepackage{lineno}

%%% HEADERS & FOOTERS
\usepackage{fancyhdr} % This should be set AFTER setting up the page geometry
\pagestyle{fancy} % options: empty , plain , fancy
\renewcommand{\headrulewidth}{0pt} % customise the layout...
\lhead{}\chead{}\rhead{}
\lfoot{}\cfoot{\thepage}\rfoot{}

%%% SECTION TITLE APPEARANCE
\usepackage{sectsty}
\allsectionsfont{\sffamily\mdseries\upshape} % (See the fntguide.pdf for font help)
% (This matches ConTeXt defaults)

%%% ToC (table of contents) APPEARANCE
\usepackage[nottoc,notlof,notlot]{tocbibind} % Put the bibliography in the ToC
\usepackage[titles,subfigure]{tocloft} % Alter the style of the Table of Contents
\renewcommand{\cftsecfont}{\rmfamily\mdseries\upshape}
\renewcommand{\cftsecpagefont}{\rmfamily\mdseries\upshape} % No bold!

%%% END Article customizations

%%% The "real" document content comes below...

\title{Effects of extending residencies on the supply of family medicine practitioners; difference-in-differences evidence from the implementation of mandatory family medicine residencies in Canada.}

\author{Stephenson Strobel}
%\date{} % Activate to display a given date or no date (if empty),
         % otherwise the current date is printed 

\begin{document}
\maketitle

\begin{abstract}
\noindent Background: There is currently interest in extending medical residency training in several countries including the United States and Canada. There is little evidence on what impacts extending medical residency has on supply of primary care practitioners.\\

\noindent Methods: I leverage the province-by-province roll-out of mandatory family medicine residencies in Canada from 1976 to 1994. This mandated that practitioners had to complete a two year residency instead of a one year internship. I use annual Canadian Institute of Health Information data on supply of physicians by specialty and province. I employ a difference-in-differences estimation strategy comparing specialities impacted by the legislation to those that had no change in their residency length (first difference). I compare before and after legislation by province (second difference).\\

\noindent Results: I find reductions in the supply of family medicine practitioners in the range of 5-10\% of overall supply after implementation of a longer residency. This reduction is statistically significant lasting five years after mandate and point estimates of supply do not return to baseline until eight years after mandates. I find increases in the number of graduates of other programs that might plausibly substitute for family medicine suggestive that the policy drove medical students towards other residencies.\\

\noindent Conclusion: Extending residency length has the potential to cause declines in physician supply over the short to medium run. There are both direct effects on physician supply through delays in cohorts as well as indirect effects through substitution away from family medicine residencies. \\

\noindent CI Statement: There are no conflicts of interest or financial disclosures.\\

\vfill
\noindent JEL: I10, I18, I23, I28
\noindent Keywords: physician training; supply of physicians; medical residency choice

\end{abstract}

\newpage

\section{Introduction}

There is renewed policy interest in extending the length of primary care medical residencies by an additional year to increase the quality of candidates graduating into independent medical practice. In Canada, policy makers want to extend family medicine residencies to three years \citep{fowler_preparing_2022}. Debate on the length of American family medicine residencies also suggests that these programs be extended by an additional year \citep{carek_length_2013, douglass_case_2021, woolever_case_2021}. \\

However, little attention has been paid to the effects that residency extensions might have on number of graduating physicians.  If physicians need to spend an additional year in residency this could impact the supply of family physicians. Extending the length of family medicine may also have other effects like changing residency program choice.  Given the current concerns with primary care access this is of policy importance \citep{statscan_primary_2020, dall_complexities_2019}.  What impact does increasing the length of residency have on the supply of family physicians?\\

I am unaware of any evidence on what residency extensions might do to the supply of physicians \citep{fowler_optimal_2022}.  Most of the current debate revolves around how residents and directors view impacts to their well-being \citep{duane_length_2002,  smits_residency_2006,  gopal_internal_2007, sabey_views_2015} or opinion on how such changes might impact quality \citep{raiche_should_2009, carek_length_2013, douglass_case_2021, woolever_case_2021, glauser_longer_2022}.  Evidence suggests that they do not change resident knowledge or quality. \citep{hopson_program_2016, waller_impact_2017, eiff_comparison_2019}. \\


However, Canada has a historical example that might provide guidance on supply effects of residency extensions.  Over the period of 20 years, Canadian provincial governments mandated two year family medicine residencies in lieu of one year internships for primary care providers.  I examine the impact of lengthening primary care residencies using a difference in differences identification strategy. I compare the supply of family medicine practitioners relative to graduates of other specialty programs, like internal medicine, who did not have changes made to their residency lengths (first difference). I use the province-by-province roll out to assess pre-post differences across these groups (second difference).  I examine effects on substitute programs to see if there were changes in choice of residency. \\

I find a 5-10\% overall decline in overall numbers of family medicine practitioners after family medicine residencies become mandatory. This is a function of the the policy extending training by one year as well as changing residency preferences to those outside of primary care. This change occurs over 5-10 years suggesting that residency extensions may exacerbate supply issues over this period absent other interventions.   \\


\section{Methods}

\subsection{Policy Context}

Canadian medical students apply to residency programs through a centralized matching service which allows them to rank their location and specialty preferences \citep{lim_matchbook_2020}. Prior to 1994, medical students had two routes to become independent primary care practitioners.  The first was completing a rotating general internship of one year and then entering into practice.  Alternately, a physician could complete a family medicine residency lasting two years. Both routes allowed physicians to practice independently with the former route creating a general practitioner (GP) and the latter route creating a family medicine specialist (FMD). \\

However, both governments and the college of family physicians considered a one year period of training after medical school inadequate. Ending the rotating internship occurred on a provincial basis (Table 1) and by 1994, the rotating internship had ended country-wide \citep{levitt_demise_1991, banner_sandra_pgy-1_1995, chan_perceived_2002}. As this delays a GPs graduation into independent practice these interventions are coded as occurring the year after implementation.\\

\subsection{Data}

Information on the supply of physicians in Canada is collected by the Canadian Institute of Health Information. This data is recorded annually at a provincial level. I exclude data prior to 1970 due to documentation changes and data after 2003 because of large increases in residency positions \citep{turriff_carms_2020}. \\

I am interested in the number of physicians that practice family medicine after the implementation of the policy. I am able to distinguish between GPs who have completed the rotating general internship and FMDs who completed a family medicine residency. I consider the sum of these two categories as the total family practitioners. \\

\subsection{Econometric Strategy}

I employ the improved doubly robust difference-in-differences (DiD) estimator to assess effects of implementation \citep{santanna_doubly_2020, callaway_difference--differences_2021}. I compare the specialties affected, specifically GPs and FMDs, to all other physician specialties where training length did not change (first difference). I make this comparison for 10 years pre-post based on the province and the year the policy change occurred (second differences). This is specified as:

\begin{equation}
    y_{ijt}= \beta_1(D_{ijt}*Post_{ijt})+G_{ij}+T_{t}+\epsilon_{ij}
\end{equation}

\noindent where $y$ is the per-capita supply of physicians by specialty $i$ in province $j$ in year $t$. $D$ is equal to one for specialties that are treated which are total family practitioners, FMDs and GPs. $Post$ is equal to one for specialties in provinces after implementation. I am interested in the effects of $\beta_1$; this is the overall difference-in-differences effect on physician supply. $G$ is a fixed effect for the province-specialty group. $T$ is a fixed effect for the year.  $\epsilon$ is an error term clustered at the specialty-province level.\\

 DiD models assume that no other interventions occur simultaneously and that in the absence of an effect, outcomes would have trended similarly. The latter assumption can be tested by examining parallel trends in event analyses. Event regressions are specified as:

\begin{equation}
    y_{ijt}=\alpha \sum^{10}_{-10} (D_{ijt}*Post_{ijt})+G_{ij}+T_{t} +\epsilon_{ij}
\end{equation}

\noindent $\alpha$ estimates effects for each individual dummy variable. These dummies take a value of one for specialty group-provinces in years when there was implementation of mandatory family physician training and zero otherwise. All estimates are relative to the year immediately prior to implementation (ie. $t=-1$) and are plotted with 95\% confidence intervals. I limit the window to ten years before and after policy change.  DiDs are implemented via Rios-Avila et al. \citep{rios-avila_csdid_2022}.\\

\subsection{Substituting towards other specialties}

Debate has assumed that decreases in physician supply result from delayed entry into independent practice. This ignores substitution effects. Extending a residency increases its opportunity cost relative to other possible residencies. Two results might suggest that medical students are substituting towards other programs. First, if changes in primary care physician volumes occur more than two years after implementation (ie. the length of that residency), it suggests that delayed cohort effects are not the whole story. Second, specialties that attract similar candidates as family medicine should see increases in their graduating residents. \\

There is little evidence on the alternate preferences for individuals who are admitted to family medicine programs. However there is guidance from medical student groups which suggest specialties that are similar to family medicine in emphasis. These are internal medicine, pediatrics, neurology, psychiatry, and physiatry \citep{lim_matchbook_2020}. I repeat the above DiD and event exercise with these as the treatment groups.\\

\section{Results}

Figure 1 demonstrates the time-series of selected types of physicians over the period of 1970 to 2003. Around 1994 when rotating internships are phased out, the number of GPs begins to decline from 75 per 100,000 population. The number off FMDs increases and in 1994 is approximately 26 per 100,000 population. The overall number of family medicine practitioners in 1994 is approximately 100 per 100,000 population.\\ 

Table 1 demonstrates the DiD effects estimated by all family medicine practitioners, FMDs, and GPs. There is an decrease in the overall supply of family medicine practitioners but at statistically insignificant levels. This comes through an increase in family medicine specialists that are exactly offset by declines in general practitioners.\\

Figures 2 demonstrates the estimates from event analyses by type of physician. Pre-trends in all graphs are noisy but relatively flat and stable suggesting that the parallel trends assumption holds well. In the period after mandating training there are large declines in the number of GPs and large increases in the number of FMDs. However, there is a transient period, with a trough about five years after policy implementation, when the net effect is negative and statistically significant. That trough demonstrates a reduction in the number of total family practitioners of 5-6 per 100,000 population.\\

Figure 3 demonstrates the effects of the policy on substitute residencies. Both psychiatry and neurology demonstrate statistically significant increases after implementation of the policy. These effects peak about five years after the policy change which is the usual length of a specialist residency in Canada. Pediatrics has similar increases which are not statistically significant. The only outlier are internal medicine graduates which decline in number. Table 1 demonstrates the DiD estimates for these specialties. \\ 

\section{Discussion}

I demonstrate the effects of mandating a change in length of residency by leveraging cessation of the rotating internship in Canada. The impact on physician supply is in line with the mandate to increase the number of FMDs and decrease the number of GPs. There are declines in GPs that are nearly perfectly offset by increases in FMDs. \\

 The impact of the mandate is initially negative as one would expect by delaying a cohort of physicians. This effect is a reduction in the supply of family medicine practitioners by 5-10\% of the overall number of practitioners. This decline is statistically significant and negative for five years after mandates and the point estimates remain negative for 8 years after mandates. Current projections suggest that supply of primary care specialists will need to increase by 7 to 20\% by 2034 to keep up with demand. These results suggest that extending family medicine residencies will cause an additional decline in supply of physicians by the lower bound of these estimates.\\

However, these results do not just occur because of a delayed cohort effect which would likely impact physician supply over the first two or three years after implementation. There is also substitution towards other specialties outside of primary care. There is some suggestive evidence that this is occurring in historical CaRMS reports from 1995. Prior to the transition, medical school graduates could expect to match to one of their top 3 programs 80\% of the time; in 1993, this dropped to 70\% suggesting that there was poorer matching success around the time when the majority of provinces discontinued the rotating internship \citep{banner_sandra_pgy-1_1995}. I find further evidence for this by demonstrating increases in the number of practitioners in fields that might be considered substitutes for family medicine. \\

These results have several limitations. First, they say nothing about the impact on changes in quality of care. It is possible that improvements in quality may have made up for any possible negative consequences of reduced access to family physicians. Second, this result says nothing about access; it is possible that primary care practices were able to absorb patients who could not access new family practitioners. Third, from an estimation standpoint it is possible there are contaminated effects especially from treated provinces in the 1990s when investments were made to boost family physician supply. It is possible other interventions coinciding with legislation are driving the DiD results. \\

These results show that extending the length of a family residency program in Canada led to transient reductions in the number of primary care physicians. Policy makers should consider patient access as well as the quality implications of proposed extensions to family medicine training that are currently being debated.\\

\newpage
\section{Tables and Figures}
\begin{table}
\begin{center}

\begin{tabular}{l|c|c} \hline
Province                      & Year of policy &  Year of Effect \\  \hline \hline
Alberta        & 1976           & 1977                 \\ 
Quebec                        & 1988           & 1989                 \\
Saskatchewan                  & 1989           & 1990                 \\
British Columbia              & 1993           & 1994                 \\
Ontario                       & 1993           & 1994                 \\
New Brunswick                 & 1993           & 1994                 \\
Prince Edward Island          & 1993           & 1994                 \\
Nova Scotia                   & 1993           & 1994                 \\
Manitoba                      & 1994           & 1994                 \\
Newfoundland and Labrador     & 1994           & 1994    \\            \hline
\end{tabular}
\caption{Year of policy change by province. Note that Manitoba and Newfoundland and Labrador implemented policies abolishing rotating medical residencies in 1994 but these affected the residency class of 1992-1993.}
\end{center}

\end{table}

{ 
\begin{table}[h!] \scriptsize
\def\sym#1{\ifmmode^{#1}\else\(^{#1}\)\fi}
\begin{tabular}{l|c c c | c c c c c|}
\hline\hline
            &\multicolumn{3}{c|}{Family Medicine Practitioners} &\multicolumn{5}{c|}{Substitute Specialties}\\
            \hline
            &\multicolumn{1}{c}{Total}&\multicolumn{1}{c}{FMDs}&\multicolumn{1}{c|}{GPs}  &\multicolumn{1}{c}{Pediatrics}&\multicolumn{1}{c}{Neurology}&\multicolumn{1}{c}{Internal Medicine} &\multicolumn{1}{c}{Psychiatry} &\multicolumn{1}{c|}{Physiatry}\\
\hline
DiD Estimates         &      -0.301       &       3.604\sym{*}&      -4.420\sym{*}  &      0.0911       &      0.0578\sym{+}&      -0.284\sym{*}&       0.466\sym{*}&    -0.00916       \\
            &     (-0.31)       &      (7.33)       &     (-4.94)  &      (0.88)       &      (1.75)       &     (-2.44)       &      (4.65)       &     (-0.30)        \\
\hline
\(N\)       &           8690        &         8690   &       8690   & 8350 & 8010 & 7704 &7364 & 7024    \\
Mean        &       84.34       &       16.34       &       67.35  &  5.292       &       3.328       &       4.929       &       5.563       &       4.620      \\

\hline

\hline\hline
\multicolumn{9}{l}{\footnotesize \textit{t} statistics in parentheses}\\
\multicolumn{9}{l}{\footnotesize \sym{+} \(p<0.10\), \sym{*} \(p<0.05\)}\\
\end{tabular}
\caption{DiD Estimates by type of physician.}
\end{table}
}

\begin{figure}[h!]
\centering
\includegraphics[width=150mm]{figure1.pdf}
\caption{Time series of selected physician categories.}
\end{figure}

\begin{figure}[h!]
\centering
\includegraphics[width=150mm]{gr_fmd_final.pdf}
\caption{Event analysis for family medicine practitioners. Point estimates are displayed with 95\% CI.}
\end{figure}

\begin{figure}[h!]
\centering
\includegraphics[width=150mm]{gr_alt_final.pdf}
\caption{Event analysis for substitute specialties to family medicine. Point estimates are displayed with 95\% CI.}
\end{figure}





\newpage
\bibliographystyle{apalike}
\bibliography{main}

\end{document}