As shown in Sections \ref{sec:classical} and \ref{sec:bip-classical} many of the combinatorial structures on plane graphs that have been used in the literature to define  graph drawing algorithms are grand-Schnyder structures for special classes of plane graphs. Hence the grand-Schnyder framework developed in the present article offers the hope of extending known drawing algorithms to new classes of plane graphs, or to look for brand new algorithms in the spirit of the known ones. In this context, the algorithm established in the present article to compute grand-Schnyder structures in linear time is valuable.
%\begin{remark}\label{rem:bipartite_4} 
%Let us finally mention that 2-orientations were used by Barri\'ere and Huemer~\cite{Barriere-Huemer:4-Labelings-quadrangulation} to design a straight-line drawing algorithm for quadrangulations. These structures (in the form of, dual, even $4$-Schnyder structures) were also used in~\cite{OB-EF:Schnyder} to design a drawing algorithm for 4-regular plane maps. 
In a forthcoming article~\cite{OB-EF-SL:4-GS-drawing}, we will present extensions of several graph drawing algorithms related to $4$-GS structures \cite{Barriere-Huemer:4-Labelings-quadrangulation,OB-EF:Schnyder,Fu07b,He93:reg-edge-labeling}, and establish some connections between them. In another forthcoming article~\cite{OB-EF-SL:5QS-drawing} we will use $5$-QS structures on quasi 5-adapted triangulations of the pentagon to define a graph-drawing algorithm for these graphs. 

Several instances of grand-Schnyder structures have also been used to define bijections between classes of planar maps and classes of trees~\cite{albenque2013generic,Bernardi-Fusy:dangulations,Fu07b,FuPoScL,Poulalhon:triang-3connexe+boundary,Schaeffer:these}. Thus the general framework established in the present article offers the promise of extending these bijections (using the ``master-bijection'' approach developed in \cite{Bernardi-Fusy:dangulations,OB-EF:girth}). In particular, this approach seems well suited to tackle the $d$-irreducible maps enumerated in~\cite{bouttier2014irreducible}.


Lastly, the original impetus for the definition of Schnyder woods were results about the dimension of the incidence poset of planar graphs~\cite{Schnyder:wood1}. We wonder if some generalization of these results can be deduced from the existence of grand-Schnyder structures.
 
%\OB{We need to decide on the title of our future article using 4-GS structure. We had discussed ``A census of drawing algorithms based on Schnyder-type structures'', but that's not good since the article is really only about $d=4$. Before we had ``A census of drawing algorithms based on transversal structures'' which is a possibility. In emails we also had ``General grid-drawing algorithms for 3 and 4-valent planar graphs and their dual, based on Grand-Schnyder structures.'' I am not sure what is the best option.}
