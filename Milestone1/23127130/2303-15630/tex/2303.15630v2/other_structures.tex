%\documentclass{amsart}
%%%%%% GENERAL MATH COMMANDS
% Reals
\newcommand{\R}{{\mathbb R}}
% Integers
\newcommand{\Z}{{\mathbb Z}}
% Naturals
\newcommand{\N}{{\mathbb N}}
% Expectation
\DeclareMathOperator*{\E}{\mathbb{E}}
% ^th notation
\newcommand{\tth}{^{\text{th}}}
% Small dots for integer range [a .. b]
\newcommand{\sdots}{\,..\,}
% Vectorized version of matrix
\newcommand{\matvec}{\mbox{vec}}

% := sign
\newcommand{\defeq}{\vcentcolon=}
% Zero function
\newcommand{\zf}{\mathbf{0}}
% Vector of ones
\newcommand{\ones}{\mathbf{1}}

% Argmin and argmax definitions
\DeclareMathOperator*{\argmax}{arg\,max}
\DeclareMathOperator*{\argmin}{arg\,min}


%%%%% PROBLEM STATEMENT NOTATION 
% \newcommandtwoopt{\St}[2][t][]{{S_{#1}^{#2}}} % State
\newcommand{\task}[1][i]{{\mathcal{T}_{#1}}} % Task, optionally takes index
\newcommand{\tasks}{\{ \task \}_{i=1}^N}
\newcommand{\losst}[1][i]{{l_{#1}}}
\newcommand{\lossv}[1][i]{{l_{#1}^{\textrm{val}}}}
\newcommand{\tasktarget}{{\mathcal{T}_{\textrm{target}}}}
\newcommand{\lossttarget}{l_{\textrm{target}}}
\newcommand{\lossvtarget}{l_{\textrm{target}}^{\textrm{val}}}
\newcommand{\lossttargetit}{l_{\textrm{target}}^{(k)}}
\newcommand{\losstotal}{l^{\textrm{total}}}
\newcommand{\lossopt}{l^*}

\newcommand{\thetait}[2]{\theta_{#1}^{(#2)}}
\newcommand{\phit}[1]{\phi^{(#1)}}
\newcommand{\hist}[2]{S_{#1}^{(#2)}}
\newcommand{\grad}[2]{G_{#1}^{(#2)}}

\newcommand{\Alg}{\textup{\textbf{Opt}}}
\newcommand{\MetaAlg}{\textup{\textbf{MetaOpt}}}

%%%%% Theorems
\newtheoremstyle{mytheoremstyle} % name
    {\topsep}                    % Space above
    {\topsep}                    % Space below
    {\itshape}                   % Body font
    {}                           % Indent amount
    {\scshape}                   % Theorem head font
    {.}                          % Punctuation after theorem head
    {.5em}                       % Space after theorem head
    {}  % Theorem head spec (can be left empty, meaning ‘normal’)
\theoremstyle{mytheoremstyle}
\theoremstyle{plain}
\newtheorem{theorem}{Theorem}
\newtheorem{proposition}{Proposition}
\newtheorem{assumption}{Assumption}
\newtheorem{definition}{Definition}
\newtheorem{lemma}{Lemma}
\theoremstyle{remark}
\newtheorem{remark}{Remark}

%\begin{document}
%\section{Connection to previously known structures}\label{sec:classical}

In this section, we explain how the $d$-GS framework emcompasses previously known structures such as Schnyder woods (and more generally Schnyder decompositions) and transversal structures (a.k.a. regular edge labelings).

\subsection{Schnyder decompositions as GS-structures on $d$-angulations}\label{sec:Schnyder-decompositions}
We start by showing that Schnyder woods, as introduced in \cite{Schnyder:wood1,Schnyder:wood2}, can be identified with the grand-Schnyder structures on triangulations. Recall that a plane triangulation admits a Schnyder wood if and only if it it has no loop or multiple edges, that is, if its girth is $3$. A generalization of Schnyder woods, are the \emph{Schnyder decompositions}  defined in \cite{OB-EF:Schnyder} for all $d\geq 3$. Several incarnations of Schnyder decompositions were given in \cite{OB-EF:Schnyder}, and it was shown that a $d$-angulation admits a Schnyder decomposition if and only if its girth is $d$ (that is, every cycle has length at least $d$).

Let us now compare Schnyder decompositions to grand-Schnyder structures. The first observation is that a $d$-angulation is $d$-adapted if and only if its girth is $d$. Let $G$ be a  $d$-angulation of girth $d$. By definition the  $d$-GS marked orientations of $G$ have no mark at all, hence the definition of $d$-GS marked orientations simplifies: these are the weighted orientations where the outer vertices and edges have weight $0$, the inner edges have weight $d-2$, and the inner vertices have weight $d$. Such weighted orientations are called \emph{$d/(d-2)$-orientations} in~\cite{Bernardi-Fusy:dangulations,OB-EF:Schnyder}, and they are one of the incarntions of Schnyder decompositions. Hence, the notions of  Schnyder decompositions and grand-Schnyder structures coincide for $d$-angulations. Note that for the original case $d=3$, the $d/(d-2)$-orientations are classical orientations (weight 1 per inner edge) such that every inner vertex has outdegree 3.

Let us now discuss the other incarnations of $d$-GS structures for $d$-angulations, and how they compare to the definitions in \cite{OB-EF:Schnyder}
First, observe that for a $d$-angulation $G$, the  $d$-GS angular orientation have no weight on star edge, hence the notion of  $d$-GS angular orientation again simplifies and identifies with $d/(d-2)$-orientations.
%\subsection{Grand-Schnyder structures on $d$-angulations of girth $d$} 
%When all faces have degree~$d$, $d$-adapted maps are $d$-angulations of girth $d$ (i.e. such that every cycle has length at least $d$). In that case, $d$-GS marked orientations have no mark at corners, hence are weighted orientations where the outer vertices and edges have weight $0$, the inner edges have weight $d-2$, and the inner vertices have weight $d$. These are called $d/(d-2)$-orientations, and the fact that a $d$-angulation admits such an orientation if and only if it has girth $d$ was proved in~\cite{Bernardi-Fusy:dangulations}.  Note also that in that case, a $d$-GS angular orientation consists of a  $d/(d-2)$-orientation on the original edges, while all the star edges have weight~$0$.   

Next, consider $d$-GS labelings of $d$-angulations.
%Regarding $d$-GS corner labelings, 
In each inner face, Conditions~(L1)  and (L2) imply that 
the corner labels have to be $1,2,\ldots,d$ in clockwise order around the face (no label is missing), while
the total clockwise jump around any inner vertex is $d$. With Conditions~(L0),(L1),(L2) we thus 
recover the definition of corner labelings as defined in~\cite{OB-EF:Schnyder} for $d$-angulations of girth $d$. Condition~(L3) is actually redundant in that case since it is a consequence of (L2). 
%OLD version (here it states that two diagonally opposed corners around an inner edge do not have the same label), it is a consequence of  Lemma~\ref{lem:ccw-jumps-edges} whose proof does not require it (note that it just uses (L0), (L1), and (L2)). 
Moreover Lemma~\ref{lem:ccw-jumps-edges} ensures that for each inner edge $e=(u,v)$, the clockwise
jump accross $e$ at $u$ plus the clockwise jump across $e$ at $v$ add up to $d-2$, so that clockwise-jumps across edges at inner vertices are at most $d-2$. In particular, for $d=3$, 
at inner vertices all corner labels appear (the incident corners form three non-empty groups, of 1's, 2's and 3's in clockwise order), as in the definition of Schnyder labelings for triangulations~\cite{Schnyder:wood1}.  

Finally, let us consider the $d$-GS woods of $d$-angulations.
Conditions (W0), (W1), and (W2) in Definition~\ref{def:woods} give exactly the definition
of Schnyder woods for $d$-angulations of girth $d$ as defined in~\cite{OB-EF:Schnyder}. This includes the classical case $d=3$ of Schnyder woods of triangulations~\cite{Schnyder:wood1} (with a slight change on the rooting convention: in classical Schnyder woods, for $i\in\{1,2,3\}$ the tree $W_i$ would be naturally rooted at $v_{i-1}$, and the missing outer edge would be $(v_{i},v_{i+1})$). For $d$-angulations, the additional Condition~(W3) in Definition~\ref{def:woods} is redundant.  
Indeed, it only gives the constraint that no inner arc in $W_i$ is ending at $v_i$ or $v_{i+1}$ (which is already required by (W0)), 
and that, for every inner arc $a$ ending at an inner vertex $v$, if $a$ is in $W_i$ and the opposite arc is in none of $W_1,\ldots,W_d$, then $a$ appears strictly between  the outgoing arcs $a_{i+1}$ and $a_i$ in clockwise order around $v$ (which is already required by (W2)).  

We mention that, in the definition of Schnyder woods for $d$-angulations given in~\cite{OB-EF:Schnyder}, it is also required that every edge belongs to $d-2$ trees, but this requirement actually follows from (W0), (W1), (W2). Indeed, these conditions easily imply that every inner edge belongs to at most $d-2$ trees, while the total number of edges in the $d$ trees is $d$ times the number of inner vertices, which is also $(d-2)$ times the number of inner edges. 




%\OB{We should say somewhere that we have solved the algorithmic question of \cite{OB-EF:Schnyder} about computing a Schnyder structure in linear time.}

%\subsection{Grand-Schnyder structures on irreducible triangulations of the 4-gon}\label{sec:transversal} 
\subsection{Transversal structures as GS-structures on triangulations of the square}\label{sec:transversal} 
Recall that regular edge-labelings were introduced by He in~\cite{He93:reg-edge-labeling}. They were later rediscovered by the second author~\cite{Fu07b} who coined the term \emph{transversal structures}, which we will adopt here. Transversal structures are defined on \emph{triangulations of the square} (that is, 4-maps such that every inner face has degree 3). We now show that the notions of transversal structures and $4$-GS structures coincide for triangulations of the square.


\begin{figure}[h!]
\begin{center}
\includegraphics[width=12cm]{transversal_structure_example}
\end{center}
\caption{From left to right: a transversal structure on a triangulation of the square, the local condition at inner vertices, the red bipolar orientation, and the blue
bipolar orientation.}
\label{fig:transvers}
\end{figure}


First, let us recall that a triangulation of the square admits a transversal structure if and only if its non-facial cycles have length at least 4.\footnote{These are sometimes called  \emph{irreducible} triangulations of the square in the literature~\cite{Fu07b,bouttier2014irreducible}.} In other words, a triangulation of the square admits a transversal structure if and only if it is $4$-adapted. Note that triangulations of the 4-gon are also \emph{edge-tight} in the sense of Section \ref{sec:arc_labeling}. 


For a triangulation of the square $G$, a \emph{transversal structure} is
%Let us now discuss the case of \emph{transversal structures}, introduced in~\cite{He93:reg-edge-labeling} under the name of \emph{regular edge-labelings} (a further combinatorial study with other incarnations is given in~\cite{Fu07b}). A 4-adapted map $G$ with triangular inner faces is also called an \emph{irreducible triangulation of the 4-gon} in the literature~\cite{bouttier2014irreducible}, where irreducible means that every 3-cycle bounds a face (irreducibility is here equivalent to the fact that one obtains a 4-connected triangulation by adding a new vertex in the outer face connected to the 4 outer vertices).   A \emph{transversal structure} of $G$ is 
an orientation and partition  of the inner edges into red and blue oriented edges such that the following conditions hold (see Figure~\ref{fig:transvers}):
\begin{itemize}
\item[(T0):]
The inner edges at $v_1,v_2,v_3,v_4$ are respectively outgoing red, outgoing blue, ingoing red, and ingoing blue.
\item[(T1):]
Around each inner vertex, the incident edges in clockwise order form $4$ non-empty groups: ingoing red, ingoing blue, outgoing red, outgoing blue. 
\end{itemize}


There is a simple bijection $\al$ between the set $\bT_G$ of transversal structures of $G$ and the set $\bAL_G$ of $4$-GS arc labelings of $G$; see Figure \ref{fig:corresp_GS_transversal}. From a transversal structure $\cT\in \bT_G$, one defines an arc labeling $\al(\cT)$ by assigning the label 1 (resp. 2) to the red (resp. blue) arcs of $\cT$, and the label 3 (resp. 4) to the opposite of red (resp. blue) arcs of $\cT$. It is clear that  $\al(\cT)$ satisfy conditions (AL0-AL4) of arc labelings, hence $\al(\cT)\in \bAL_G$. Conversely, for $\cAL\in\bAL_G$, Condition (AL3) implies that opposite arcs have opposite label modulo 4, and Condition (AL2) implies that around every inner vertex there are 4 non-empty groups of outgoing arcs of label 1,2,3, and 4 respectively (in this order clockwise around $v$). Hence,  one can construct a transversal structure $\bal(\cAL)$ whose red (resp. blue) arcs are the arcs labeled 1 (resp. 2) in  $\cAL$. Obviously $\al$ and $\bal$ are inverse mappings, hence bijections, between $\bT_G$ and $\bAL_G$. 




It is well known~\cite{He93:reg-edge-labeling} that 
%OB removed: a triangulation of the 4-gon admits a transversal structure if and only if it is irreducible, and that   
 a transversal structure yields two plane bipolar orientations: the \emph{red bipolar orientation} is obtained by erasing the blue edges, and orienting the outer edges in the direction from $v_1$ to $v_3$; the \emph{blue bipolar orientation} is obtained by erasing the red edges, and orienting the outer edges in the direction from $v_2$ to $v_4$.  
Note that the red (resp. blue) bipolar orientation of the transversal structure $\cT$  is the plane bipolar orientation $B_4$ (resp. $B_1$) associated to $\cAL=\al(\cT)$ by the mapping $\beta$ of Definition \ref{def:Delta}. 


\begin{figure}
\begin{center}
\includegraphics[width=12cm]{from_corner_labeling_to_transversal}
\end{center}
\caption{On the left, a $4$-GS arc labeling on a triangulation of the 4-gon. On the right, the corresponding transversal structure (upon orienting the outer edges from $v_1$ toward $v_3$,  and coloring $(v_i,v_{i+1})$ red/blue for $i$ odd/even). The two bottom rows show the  local conditions at inner vertices and inner faces (of 4 possible types) when superimposing both structures. The top row shows the associated $4$-GS angular orientation. 
}
\label{fig:corresp_GS_transversal}
\end{figure}


 
%% We will now describe a bijection between the transversal structures and the $4$-GS arc labelings for a fixed triangulation of the square $G$. This bijection is  
%% illustrated in Figure~\ref{fig:corresp_GS_transversal}: from a $4$-GS arc labeling $\cAL$,  one gets a transversal structure $\cT$ by turning each inner edge with labels $\{1,3\}$ into a red edge oriented as the arc of label $1$, and turning each inner edge with labels $\{2,4\}$ into a blue edge oriented as the arc of label $2$. 



%% Note that the red (resp. blue) bipolar orientation of the transversal structure $\cT$  is the plane bipolar orientation $B_4$ (resp. $B_1$) associated to $\cAL$ by the mapping $\beta$ of Definition \ref{def:Delta}. 


%As illustrated in Figure~\ref{fig:corresp_GS_transversal}, for $G$ an irreducible triangulation of the 4-gon, there is a direct bijection between the transversal structures of $T$ and the $4$-GS arc labelings of $G$: for $\mathcal{AL}$ a $4$-GS arc labeling of $T$,  each inner edge with labels $\{1,3\}$ is turned into a red edge oriented as the arc of label $1$, and each inner edge with labels $\{2,4\}$ is turned into a blue edge oriented as the arc of label $2$. 

We now discuss other incarnations of $d$-GS structures for triangulations of the square, and  compare them to the known incarnations of  transversal structures  \cite{Fu07b}. We start with the incarnation in terms of $4$-GS woods. Let $\cT$ be a transversal structure, let  $\cAL=\al(\cT)$ be the corresponding arc labeling, and les  $\cW=(W_1,W_2,W_3,W_4)$ be the 4-GS wood associated to $\cAL$. By  Remark~\ref{rk:bipolarBi}, the tree $W_i$ (after changing its root from  $v_i$ to $v_{i-1}$) is obtained from $\cT$ as follows: the tree $W_1$ (resp. $W_3$) is the leftmost outgoing tree (resp. rightmost ingoing tree) of the blue bipolar orientation, while the tree $W_4$ (resp. $W_2$) is the leftmost outgoing tree (resp. rightmost ingoing tree) of the red bipolar orientation. 
%According to Remark~\ref{rk:bipolarBi} (or looking at the before last row in Figure~\ref{fig:corresp_GS_transversal}),  the $4$-GS wood $\cW=(W_1,W_2,W_3,W_4)$ associated to $\cAL$ is related to $\cT$ as follows (up to changing the tree-root of $W_i$ from $v_i$ to $v_{i-1}$): the tree $W_1$ (resp. $W_3$) is the leftmost outgoing tree (resp. rightmost ingoing tree) of the blue bipolar orientation, while the tree $W_4$ (resp. $W_2$) is the leftmost outgoing tree (resp. rightmost ingoing tree) of the red bipolar orientation.


Next, we consider the $4$-GS angular orientations for triangulations of the square. It was shown in~\cite{Fu07b} that transversal structures correspond to orientations of star edges in $G^+$, where $v_i$ has outdegree $2$ (resp. $0$) for $i\in\{1,3\}$ (resp. $i\in\{2,4\}$), original inner vertices have outdegree $4$, and star vertices have outdegree $1$. Letting $s_i$ be the star vertex in the inner face containing the outer edge $(v_{i-1},v_i)$, these coincide with the $4$-GS angular orientations for that case, upon returning the edges $(v_2,s_2), (v_1,s_2), (v_4,s_4), (v_3,s_4)$. Moreover, as shown in the top-part of Figure~\ref{fig:corresp_GS_transversal}, the correspondence in~\cite{Fu07b} commutes with our correspondence from $4$-GS arc labelings to $4$-GS angular orientations. 


%\bibliographystyle{plain}
%\bibliography{biblio}
%
%\end{document}
