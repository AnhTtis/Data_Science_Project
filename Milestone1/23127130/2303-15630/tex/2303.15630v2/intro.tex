%\documentclass[a4paper]{amsart}%[a4paper]
%%%%%% GENERAL MATH COMMANDS
% Reals
\newcommand{\R}{{\mathbb R}}
% Integers
\newcommand{\Z}{{\mathbb Z}}
% Naturals
\newcommand{\N}{{\mathbb N}}
% Expectation
\DeclareMathOperator*{\E}{\mathbb{E}}
% ^th notation
\newcommand{\tth}{^{\text{th}}}
% Small dots for integer range [a .. b]
\newcommand{\sdots}{\,..\,}
% Vectorized version of matrix
\newcommand{\matvec}{\mbox{vec}}

% := sign
\newcommand{\defeq}{\vcentcolon=}
% Zero function
\newcommand{\zf}{\mathbf{0}}
% Vector of ones
\newcommand{\ones}{\mathbf{1}}

% Argmin and argmax definitions
\DeclareMathOperator*{\argmax}{arg\,max}
\DeclareMathOperator*{\argmin}{arg\,min}


%%%%% PROBLEM STATEMENT NOTATION 
% \newcommandtwoopt{\St}[2][t][]{{S_{#1}^{#2}}} % State
\newcommand{\task}[1][i]{{\mathcal{T}_{#1}}} % Task, optionally takes index
\newcommand{\tasks}{\{ \task \}_{i=1}^N}
\newcommand{\losst}[1][i]{{l_{#1}}}
\newcommand{\lossv}[1][i]{{l_{#1}^{\textrm{val}}}}
\newcommand{\tasktarget}{{\mathcal{T}_{\textrm{target}}}}
\newcommand{\lossttarget}{l_{\textrm{target}}}
\newcommand{\lossvtarget}{l_{\textrm{target}}^{\textrm{val}}}
\newcommand{\lossttargetit}{l_{\textrm{target}}^{(k)}}
\newcommand{\losstotal}{l^{\textrm{total}}}
\newcommand{\lossopt}{l^*}

\newcommand{\thetait}[2]{\theta_{#1}^{(#2)}}
\newcommand{\phit}[1]{\phi^{(#1)}}
\newcommand{\hist}[2]{S_{#1}^{(#2)}}
\newcommand{\grad}[2]{G_{#1}^{(#2)}}

\newcommand{\Alg}{\textup{\textbf{Opt}}}
\newcommand{\MetaAlg}{\textup{\textbf{MetaOpt}}}

%%%%% Theorems
\newtheoremstyle{mytheoremstyle} % name
    {\topsep}                    % Space above
    {\topsep}                    % Space below
    {\itshape}                   % Body font
    {}                           % Indent amount
    {\scshape}                   % Theorem head font
    {.}                          % Punctuation after theorem head
    {.5em}                       % Space after theorem head
    {}  % Theorem head spec (can be left empty, meaning ‘normal’)
\theoremstyle{mytheoremstyle}
\theoremstyle{plain}
\newtheorem{theorem}{Theorem}
\newtheorem{proposition}{Proposition}
\newtheorem{assumption}{Assumption}
\newtheorem{definition}{Definition}
\newtheorem{lemma}{Lemma}
\theoremstyle{remark}
\newtheorem{remark}{Remark}

%\author{Olivier Bernardi$^{*}$ \and \'{E}ric Fusy$^{\dagger}$ \and Shizhe Liang$^{+}$}
%\title[Grand Schnyder Woods]{Grand Schnyder Woods}
%\begin{document}



In 1989, Walter Schnyder showed that planar triangulations can be endowed with remarkable combinatorial structures, which now go by the name of \emph{Schnyder woods}~\cite{Schnyder:wood1}. A Schnyder wood of a planar triangulation (drawn without crossing in the plane) is a partition of its inner edges into three trees, crossing each other in a specific manner. A Schnyder wood is represented in Figure~\ref{fig:triangulation2}, and more details about the definition are given in the caption.

\fig{width=\linewidth}{triangulation2}{(a) A Schnyder wood of a triangulation $G$, where the three trees are indicated by three colors.
A Schnyder wood of $G$ is a partition of the inner edges of $G$ into three trees $W_1$, $W_2$, $W_3$ satisfying two conditions: 
(1) %each tree spans all the inner vertices, as well as one outer vertex which is choosen as its root, with the root of $W_1$, $W_2$, $W_3$ appearing in clockwise order around the outer face of $G$
 for all $i\in\{1,2,3\}$, the tree $W_i$ spans all the inner vertices and the outer vertex $v_{i-1}$ which is chosen as its root,  
(2) if the trees are oriented toward their roots, then in clockwise direction around each inner vertex one has: the outgoing edge of $W_1$, the ingoing edges of $W_3$, the outgoing edge of $W_2$, the ingoing edges of $W_1$, the outgoing edge of $W_3$, and finally the ingoing edges of $W_2$. 
(b) Encoding the Schnyder wood by a corner labeling.} 

Schnyder used the existence of Schnyder woods to show that the incidence poset of any planar triangulation has dimension at most 3, thereby completing the proof that graphs are planar if and only if their incidence poset has dimension at most 3~\cite{Schnyder:wood1}. Another application explored by Schnyder is the possibility of drawing planar graphs with straight edges~\cite{Schnyder:wood1} (reproving a fact established by Wagner~\cite{Wagner:straght-line-drawing}), and he showed further that this can be done with all vertices on the lattice points of a $(n+2)\times(n+2)$ grid~\cite{Schnyder:wood2}. Since then, numerous other applications have been found for Schnyder woods~\cite{louigi2017scaling,bonichon2006planar,aleardi2018array,dhandapani2010greedy,F01,Felsner:posets,gonccalves2012triangle,li2017schnyder,Poulalhon:triang-3connexe+boundary}.\\


In 1993, Xin He showed that triangulations of the square without separating 3-cycles can also be endowed with remarkable combinatorial structures~\cite{He93:reg-edge-labeling} called \emph{regular edge-labelings}. These structures were later rediscovered by the second author~\cite{Fu07b} who named them \emph{transversal structures}, and we will adopt this name throughout the article. There are several ways to encode transversal structures, and one of the encodings given in~\cite{Fu07b} is as a partition of the inner edges of the triangulation into two graphs, crossing each other in a specific manner. A transversal structure is represented in Figure~\ref{fig:transversal}. 

\fig{width=\linewidth}{transversal}{(a) A transversal structure for a triangulation of the square $G$, where the two subgraphs are indicated using colors.
A transversal structure of $G$ is a partition of the inner edges into 2 subgraphs $S_1$ and $S_2$ satisfying 2 conditions (1) $S_1$ is incident to all inner vertices as well as the outer vertices $v_1$ and $v_3$, while $S_2$ is incident to all inner vertices as well as the outer vertices $v_2$ and $v_4$, (2) in clockwise order around each inner vertex of $G$ one has: a non-empty set of edges in $S_1$, a non-empty set of edges in $S_2$, a non-empty set of edges in $S_1$, and finally a non-empty set of edges in $S_2$. 
(b) Encoding the transversal structure by a corner labeling.}


Xin He~\cite{He93:reg-edge-labeling} used transversal structures to give an algorithm for realizing these triangulations as the contact graphs of rectangles (equivalently, drawing their dual in such a way that every face is a rectangle), and together with Goos Kant showed that the transversal structure and drawing can be computed in linear time~\cite{KantHe97:reg-edge-labeling-linear}. Since then, several other graph drawing algorithms based on transversal structures have been obtained~\cite{biedl2018embedding,eppstein2012area,felsner2013rectangle,felsnerFewLines,Fu07b}.\\

In 2012~\cite{OB-EF:Schnyder}, the first and second authors gave an analogue of Schnyder woods for \emph{$d$-angulations} (a planar graph drawn in the plane such that every face has degree $d$). This analogue, named \emph{$d$-Schnyder decomposition} is a $d$-tuple of spanning trees crossing each other in a specific manner and such that every inner edge belongs to exactly $d-2$ trees. A 5-Schnyder decomposition is shown in Figure~\ref{fig:pentagulation}. It is shown in~\cite{OB-EF:Schnyder} that a $d$-angulation admits a $d$-Schnyder decomposition if and only if its girth is $d$ (equivalently, it has no cycle of length less than $d$). The definition of 3-Schnyder decomposition (for triangulations) coincides with the classical definition of Schnyder woods. In~\cite{OB-EF:Schnyder} 4-Schnyder decompositions were also used to design a drawing algorithm for planar 4-valent graphs of min-cut 4. 
%This raised the question of similarly extending transversal structure to other classes of graphs ($d$-angulations of the $(d+1)$-gon). \\

\fig{width=\linewidth}{pentagulation}{(a) The five trees forming the 5-Schnyder decomposition. (b) Encoding the Schnyder decomposition by a corner labeling.}


At first sight, Schnyder woods and transversal structures may not appear to have much in common. However, we will show in this article that they can be given a common definition. One way to make the commonality more apparent, is to encode both structures by certain labelings of the corners. For Schnyder woods, this is a classical encoding, defined by Schnyder himself~\cite{Schnyder:wood1}, as a labeling of the corners of the triangulation with numbers in $\{1,2,3\}$ satisfying certain conditions. For transversal structures, we will define a similar encoding by a labeling of the corners of the triangulation with numbers in $\{1,2,3,4\}$. These corner labelings are indicated in Figures~\ref{fig:triangulation2}(b) and~\ref{fig:transversal}(b) respectively. With this ``labeling incarnation'' the conditions defining Schnyder woods and transversal structures look pretty similar.
\\

In this article we define a general combinatorial structure, the \emph{grand-Schnyder woods}, which put Schnyder woods and transversal structures under one roof.
%generalizes both Schnyder woods and transversal structures.
%, and we show that the classical Schnyder woods are the 3-grand-Schnyder woods for triangulations, while the transversal structures are the 4-grand-Schnyder woods for triangulations of the square. 
 For $d\geq 3$, we call \emph{$d$-map} a connected planar graph drawn in the plane without edge crossing such that the outer face has degree $d$ (and is incident to $d$ distinct vertices) and the inner faces have degree at most $d$.
%Recall that a \emph{plane map} is a connected planar graph drawn in the plane without edge crossing (and considered up to continuous deformation).
For a $d$-map $G$, a \emph{$d$-grand-Schnyder wood} is a $d$-tuple of spanning trees of $G$ crossing each other in a specific manner. An example is given in Figure~\ref{fig:4-grand-Schnyder}, and a precise definition is given in Section~\ref{sec:incarnations}.\\



When the $d$-map $G$ is a $d$-angulation, then the $d$-grand-Schnyder woods of $G$ coincide with the $d$-Schnyder decompositions of $G$ (up to minor differences in conventions). %Hence grand-Schnyder woods are a far reaching generalization of classical Schnyder woods.
When the map $G$ is a triangulation of the square (a 4-map), then the 4-grand-Schnyder woods of $G$ are in bijection with the transversal structures of $G$. 
%Hence the concept of grand-Schnyder woods put Schnyder woods and transversal structures under one roof. 
Hence grand-Schnyder woods are a far reaching generalization of both $d$-Schnyder decompositions and transversal structures. 
\\
%Our framework gives a common definition for both structures.\footnote{\EF{A different type of connection was established between transversal structures and Schnyder woods in~\cite[Sec.4]{KantHe97:reg-edge-labeling-linear}. Namely is was shown that transversal structures could be identified with a \emph{subclass} of Schnyder woods of 4-connected triangulations.} 



%% At first sight, Schnyder woods and transversal structures may not appear to have so much in common. However their commonality become more apparent if we encode these structure in a different manner. Indeed, both structures can be encoded by certain labelings of the corners. For Schnyder woods, this is a classical encoding, defined by Schnyder himself~\cite{Schnyder:wood1}, as a labeling of the corners of the triangulation with numbers in $\{1,2,3\}$ satisfying certain conditions. For transversal structures, we can define a similar encoding by a labeling of the corners of the triangulation with numbers in $\{1,2,3,4\}$. These corner labelings are indicated in Figures~\ref{fig:triangulation}(b) and~\ref{fig:transversal}(b) respectively.\\

%% In fact, we will show in this article that Schnyder woods and transversal structures are deeply related and can be given a common definition. Indeed, they are particular instances of a more general structure that we name \emph{grand-Schnyder woods}. For $d\geq 3$, let us call \emph{$d$-map} a connected planar graph drawn in the plane without edge crossing such that the outer face has degree $d$ and the inner faces have degree at most $d$.
%% %Recall that a \emph{plane map} is a connected planar graph drawn in the plane without edge crossing (and considered up to continuous deformation).
%% For a $d$-map $G$, a \emph{$d$-grand-Schnyder wood} is a $d$-tuple of spanning trees of $G$ crossing each other in a specific manner. An example is given in Figure~\ref{fig:4-grand-Schnyder}. Exact definitions are given in Section~\ref{sec:incarnations}.
%% One of our main results is that a $d$-map $G$ admits a $d$-grand-Schnyder wood if and only if all the non-facial cycles of $G$ have length at least $d$. \\

%% When the map $G$ is a $d$-angulation, then 3-grand-Schnyder woods of $G$ coincide with the $d$-Schnyder decompositions of~\cite{OB-EF:Schnyder} (up to minor differences in conventions). %Hence grand-Schnyder woods are a far reaching generalization of classical Schnyder woods.
%% When the map $G$ is a triangulation of the square (hence $G$ is a 4-map), then the 4-grand-Schnyder woods of $G$ are in bijection with the transversal structures of $G$. 
%% Hence grand-Schnyder woods are a far reaching generalization of both classical Schnyder woods, and transversal structures. Our framework gives a common definition for both structures.\footnote{\EF{A different type of connection was established between transversal structures and Schnyder woods in~\cite[Sec.4]{KantHe97:reg-edge-labeling-linear}. Namely is was shown that transversal structures could be identified with a \emph{subclass} of Schnyder woods of 4-connected triangulations.} 

%Hence the concept of grand-Schnyder woods put Schnyder woods and transversal structures under one roof. 
% OB: Maybe we should remove the following:
%Also, this concept gives a natural generalization of transversal structures to $d$-angulations of the $(d+1)$-gon for all $d\geq 3$.\\

One of our main results is that a $d$-map $G$ admits a $d$-grand-Schnyder wood if and only if all the non-facial cycles of $G$ have length at least $d$ (generalizing the existence results known for Schnyder decompositions and transversal structures~\cite{OB-EF:Schnyder,He93:reg-edge-labeling}). We call \emph{$d$-adapted} a $d$-map satisfying this condition.
In a forthcoming article~\cite{OB-EF-SL:4-GS-drawing}, we show that $4$-grand-Schnyder woods can be used for defining some graph-drawing algorithms. 
Schnyder decompositions and transversal structures have also been used to define bijections between classes of planar maps and classes of trees~\cite{albenque2013generic,Bernardi-Fusy:dangulations,Fu07b,FuPoScL,Poulalhon:triang-3connexe+boundary,Schaeffer:these}, and we plan to investigate whether these bijections can be extended thanks to the general framework of grand-Schnyder structures. %, so as to cover the $d$-irreducible maps addressed in~\cite{bouttier2014irreducible}.
%Regarding bijective enumeration, several known instances of grand-Schnyder woods have been exploited to design bijections for planar maps~\cite{FuPoScL,Fu07b,Bernardi-Fusy:dangulations,albenque2013generic}, and we plan to investigate whether these can be extended thanks to the general framework of grand-Schnyder structures, so as to cover the $d$-irreducible maps addressed in~\cite{bouttier2014irreducible}.


%The existence of grand-Schnyder woods also have applications for the bijective encoding of planar maps by tree-like structures~\cite{Bernardi-Fusy:dangulations}.\\
%\OB{Not sure if we want to cite a forthcoming article for bijective encoding of planar maps.}

As mentioned above, Schnyder woods and transversal structures have several incarnations. For instance, Schnyder woods can be encoded by either a triple of trees, or by a corner labeling. There are more incarnations, and this extends to the $d$-grand-Schnyder woods setting. More precisely, $d$-grand-Schnyder woods can be naturally encoded in four distinct ways.
\begin{compactitem}
\item As a $d$-tuple of trees crossing each other in a specific manner. We call such a structure \emph{$d$-grand-Schnyder wood}, or \emph{$d$-GS wood} for short.
\item As a labeling of the corners with values in $[d]:=\{1,2,\ldots,d\}$ satisfying certain local conditions. We call such a structure \emph{$d$-grand-Schnyder corner labeling}, or \emph{$d$-GS labeling} for short.
\item As a weighted orientation of $G$ together with marks at corners. We call such a structure \emph{$d$-grand-Schnyder marked orientation}, or \emph{$d$-GS marked orientation} for short.
\item As a weighted orientation of the \emph{angular map} of the $d$-map $G$ (the angular map of $G$ is obtained from $G$ by adding a vertex in each face of $G$ and joining that vertex to each vertex of $G$ incident to the face). We call such a structure a \emph{$d$-grand-Schnyder angular orientation}, or \emph{$d$-GS angular orientation} for short.
\end{compactitem}
These four incarnations are represented in Figure~\ref{fig:4-grand-Schnyder}.
We will define these three structures in Section~\ref{sec:incarnations}, and show that they are in bijection with each other in Section~\ref{sec:statements}. 

\fig{width=\linewidth}{4-grand-Schnyder}{Four incarnations of the same 4-grand-Schnyder structure.}

In Section~\ref{sec:dual}, we will also consider some incarnations of $d$-grand-Schnyder woods of a $d$-map $G$ as decorations of the dual graph $G^*$. This can be done either as a corner labeling of $G^*$ or as a $d$-tuple of spanning trees of $G^*$ crossing each other in a specific manner.\\

Schnyder woods and transversal structures are known to have two additional interesting properties. First, Schnyder woods and transversal structures are known to be computable in linear time~\cite{Schnyder:wood1,KantHe97:reg-edge-labeling-linear}. It was left as an open question in~\cite{OB-EF:Schnyder} to find a linear time algorithm for computing $d$-Schnyder decompositions for $d\geq 4$. We provide such an algorithm in Section~\ref{sec:proof-existence}. Precisely, for all $d\geq 3$, we give an algorithm for computing a $d$-GS wood for any $d$-map $G$ having all non-facial cycles of length at least $d$, in a number of operations which is linear in the number of vertices of $G$. 
Second, the set of Schnyder woods of any given triangulation is known to have a \emph{lattice structure} (in the sense of poset theory). This was first discovered in~\cite{Mendez:these} and reinterpreted in~\cite{Brehm:latticeSchnyder,Felsner:lattice}. Similarly, the set of transversal structures of a given triangulations of the square has a lattice structure~\cite{Fu07b}. In both cases, the covering relations in the lattice can be described in a simple local way. We generalize this lattice structure to the set of $d$-GS woods of a $d$-map and describe the covering relations in Section~\ref{sec:lattice}.\\


Schnyder woods and transversal structures are not the only combinatorial structures which can be captured by the grand-Schnyder framework. More precisely, there are additional structures which can be identified with \emph{bipartite grand-Schnyder woods}. For an even integer $d=2b$, the bipartite $d$-adapted maps admit a subclass of $d$-GS structures which we call \emph{even $d$-GS structures}. Even $d$-GS structures are a bit simpler than arbitrary $d$-GS structures, and after simplifications we arrive at the notion of \emph{$b$-bipartite grand-Schnyder structures} (or \emph{$b$-BGS} for short), which again have 4 different incarnations described in Section~\ref{sec:bipartite}. We will show that \emph{bipolar orientations} of 2-connected graphs, and \emph{Felsner woods} of 3-connected graphs~\cite{F01} can be identified with classes of bipartite grand-Schnyder structures (2-BGS and 3-BGS respectively).\\

%\OB{We can possibly move the following 2 paragraphs to section~\ref{sec:bipartite}.}


Recall that a \emph{bipolar orientation} of a graph is an acyclic orientation with a unique source (vertex with no ingoing edge) and a unique sink (vertex with no outgoing edge). Given a planar graph $G$ drawn in the plane with 2 marked outer vertices $s,t$, one can associate a 4-angulation $Q_G$ by the process indicated in Figure~\ref{fig:2-orientations}(a), and any planar 4-angulation arise in this way. It is known that the bipolar orientations of $G$ with source $s$ and sink $t$ are in bijection with the \emph{2-orientations of $Q_G$}, that is, the orientations of the inner edges of $Q_G$ such that every inner vertex has outdegree 2. The correspondence is shown in Figure ~\ref{fig:2-orientations}(b). As we will see in Section~\ref{sec:bipartite}, $2$-orientations are one of the incarnations of $2$-BGS. This gives a bijection between the set of plane bipolar orientations and the set of $2$-BGS of quadrangulations.\\

%First, recall that a \emph{2-orientation} for a 4-angulation $G$ is an orientation of the inner edges of $G$ such that every inner vertex has outdegree 2. It is known that $G$ admits a 2-orientation if and only if $G$ has girth~4. As we will see, the $2$-orientations of $G$ are exactly the $2$-BGS of $G$ (in one of its incarnations). Let us also recall that, given a connected planar graph $G$ drawn in the plane with 2 outer vertices marked as the \emph{source} and \emph{target} respectively, one can associate a 4-angulation $M_G$ by the process indicated in Figure~\ref{fig:2-orientations} (this process is actually a bijection). Moreover, the graph $G$ is 2-connected if and only if $M_G$ has girth~4; and the set of bipolar orientations of $G$ with specified source and target are in bijection with the set of 2-orientations of $M_G$, or equivalently the 2-BGS of $M_G$. This gives a bijection between the set of planar bipolar orientations and the set of $2$-BGS.\\
 
\fig{width=.9\linewidth}{2-orientations}{Left: A connected planar map $M$ with two distinguished outer vertices, and the associated 4-angulation $G$. Right: a bipolar orientation of $M$ and the associated 2-orientation of $G$ (an incarnation of 2-BGS structures).}

Second, recall that there exists a generalization of Schnyder woods defined by Felsner~\cite{F01} (and independently in~\cite{di1999output}) for so-called \emph{suspended 3-connected plane graphs}.  
Let us call \emph{Felsner woods} this generalization of Schnyder woods (triangulations are a special case of suspended 3-connected plane graphs, and Felsner woods correspond to Schnyder woods in the case of triangulations). By the process indicated in Figure~\ref{fig:Felsner_woods_intro_bis}, one can associate to each suspended 3-connected plane graph $G$ a 4-angulation of the hexagon $M_G$ for which all non-facial cycles have length at least 6 (this process is almost a bijection). We will show in Section~\ref{sec:bip-classical} that the Felsner woods of $G$ 
%\OB{Why say which incarnation if they are all in bijection?}\EF{I agree, it is just that we never really see Felsner woods later (apart from an illustration)} 
are in bijection with the 3-BGS of $M_G$ (as we will explain, the 3-BGS of $M_G$ are closely related to some tricoloring of the edges of $M_G$ which is an incarnation of Felsner woods~\cite{F01} which was also discovered independently by Miller \cite{Miller:FelsnerWoods}). This gives a bijection between the set of Felsner woods and the set of $3$-BGS. \\


\fig{width=\linewidth}{Felsner_woods_intro_bis}{Left: a suspended 3-connected plane graph $M$ (three outer vertices are marked), and the associated 4-angulation of the hexagon $G$. Right: a Felsner corner-labeling of $M$ (an incarnation of Felsner woods) 
and the corresponding tricoloration of inner edges of $G$ (an incarnation of 3-BGS structures).}




Lastly, in Section~\ref{sec:level2-GS} we will consider an extension of grand-Schnyder woods, called \emph{quasi-Schnyder structures}, which exist for $d$-maps which are not quite $d$-adapted. 
We say that a map is \emph{quasi $d$-adapted} if it is a $d$-map such that simple cycles of length less than $d$ are either facial cycles, or cycles of length $d-1$ containing a single edge and no vertex. \\ 
We show that a $d$-map with no face of degree less than $3$ admits a quasi-Schnyder wood if and only if it is quasi $d$-adapted. 
A quasi-Schnyder wood of a (quasi 5-adapted) triangulation of a 5-gon is represented in Figure~\ref{fig:5c_wood}. 
In this figure, the quasi-Schnyder structure is represented in terms of woods, whereas the other incarnations (in terms of orientations and labelings) 
will be discussed in Section~\ref{sec:level2-GS}. In \cite{OB-EF-SL:5QS-drawing} we focus on quasi-Schnyder woods of quasi $5$-adapted triangulations, discuss additional incarnations, and use these structures to define a graph-drawing algorithm (for triangulations of the 5-gon, the quasi 5-adapted condition is closely related to 5-connectedness).\\


Before we close this introduction, let us mention two links between Schnyder woods and transversal structures which have been previously established, but are not directly related to the present article. First, Kant and He showed in~\cite[Sec.4]{KantHe97:reg-edge-labeling-linear} that 4-connected triangulations admit a special kind of Schnyder woods, and that for $T$ any 4-adapted triangulation of the 4-gon, the transversal structures of $T$ can be mapped surjectively to the special Schnyder woods of the 4-connected triangulation obtained from $T$ by adding a diagonal in the outer 4-gon. 
%could be identified with a subclass of Schnyder woods of the triangulation obtained by adding a diagonal in the 4-gon. \OB{I already forgot the exact statement of the previous %sentence.} 
Second Felsner, Schrezenmaier and Steiner~\cite{felsner2018equiangular} have considered families of orientations for planar triangulations of the $d$-gon (for $d\geq 3$) that are inspired by representations of triangulations by contact-systems of equi-angular $d$-gons. The definition of these orientations depends on the parity of $d$. The family of orientations obtained for odd $d$ correspond to Schnyder woods in the case $d=3$, while the family of orientations for even $d$ correspond to transversal structures for $d=4$. These orientations are not closely related to grand-Schnyder woods for higher values of $d$ however, and they are defined on triangulations of the $d$-gon without girth constraint for $d\geq 5$.

%\footnote{\EF{A different type of connection was established between transversal structures and Schnyder woods in~\cite[Sec.4]{KantHe97:reg-edge-labeling-linear}. Namely, is was shown there that transversal structures could be identified with a \emph{subclass} of Schnyder woods of 4-connected triangulations.}} 
%\EF{We note that another unifying framework covering Schnyder woods and transversal structures has been recently developed~\cite{felsner2018equiangular} (with incarnations as certain flow-orientations and woods), associated to the representations of a map as a contact-system of equiangular $d$-gons; it does not seem to be closely related to our setting (for instance, in~\cite{felsner2018equiangular} the underlying maps are triangulations of the $d$-gon, with no non-facial triangle for the special case $d=4$; whereas we have girth constraints that increase with $d$, and allow for faces of any degree in $[2..d]$).} 


%TODO: Add the figure of Weak-Schnyder woods for triangulation of the pentagon. \\
\fig{width=\linewidth}{5c_wood}{(a) A 5-quasi-Schnyder wood of a (quasi 5-adapted) triangulation of a 5-gon. (b) Encoding the quasi-Schnyder structure by a corner labeling.}



This article is organized as follow. In Section~\ref{sec:notation} we set some notation and define the relevant classes of plane maps. 
In Section~\ref{sec:incarnations} we give four distinct incarnations of $d$-GS structures, as woods, corner labelings, marked orientations or angular orientations.
In Section~\ref{sec:statements} we state our main results: the existence result for $d$-GS structures, and the fact that the set of $d$-GS woods, labelings, marked orientations and angular orientations are in bijection.
We delay some of the proofs about these bijections to Section~\ref{sec:remaining-proofs}, and the proof of existence of $d$-GS structure to Section~\ref{sec:proof-existence} (which also contains the proof that these structures can be computed in linear time). 
In Section~\ref{sec:arc_labeling}, we provide a further incarnation of $d$-GS structure as arc-labelings for a restricted class of $d$-maps called \emph{edge-tight}. This further incarnation makes the connection to transversal structures and Felsner woods more straightforward and yield a decomposition of the map into plane bipolar orientations.
%in order to make the connection to transversal structures and Felsner woods more straightforward, we provide a further incarnation (arc-labelings, very close to the corner labelings incarnation) for a restricted class of $d$-maps called \emph{edge-tight}; we also show that (in the $d$-angulated case) these yield a decomposition into plane bipolar orientations.} 
In Section~\ref{sec:classical}, we explain in detail how $d$-GS structures specialize to the classical Schnyder woods and to transversal structures.
In Section~\ref{sec:bipartite}, we define bipartite $d$-GS structures and how these structures are related to 2-orientations and Felsner woods.
In Section~\ref{sec:lattice}, we study the lattice structure for the set of $d$-GS structures of a given $d$-map.
In Section~\ref{sec:dual}, we consider 
%define for the dual $G^*$ of $d$-map $G$, 
the dual $d$-GS structures (either as dual-corner labelings or as dual-woods), on the dual $G^*$ of a $d$-adapted map.
In Section~\ref{sec:level2-GS}, we discuss quasi-Schnyder structures.
We conclude in Section~\ref{sec:conclusion} with some perspectives and open questions. 
 
 
%\bibliography{biblio-Schnyder}
%\end{document}
