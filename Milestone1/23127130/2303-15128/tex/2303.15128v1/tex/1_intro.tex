%!TEX root = ../main.tex

\section{Introduction}%
\label{sec:introduction}

Future cars will  connect to the Internet as well as to other vehicles and infrastructure  (\ac{V2X}) for improving road safety, traffic efficiency, and driver comfort.
This opens a large attack surface across communication interfaces~\cite{cmkas-ceaas-11,mv-sraas-14,mv-reupv-15} and in-car software~\cite{kpk-essoe-19,xllzl-ssdai-22}.
Nevertheless, current automotive protocols and \acp{ECU} often lack security mechanisms~\cite{mrfmsvcJR21} since they were designed for a closed environment.
Industry standards (e.g., ISO/SAE 21434~\cite{iso-sae-21434}) and legislation (e.g., the European \acl{CRA}) demand automotive security throughout the entire supply chain for hardware and software.

\ac{SOA} for automotive software emerges as a paradigm that facilitates service provisioning by various suppliers of an \ac{OEM}. 
\ac{SOME/IP}~\cite{aspsJR21} -- standardized by AUTOSAR -- is the most widely deployed middleware tailored to the automotive environment and implements service-oriented communication via IP and Automotive Ethernet~\cite{mk-ae-15}. 
Paired with \ac{TSN}~\cite{ieee8021q-18}, Automotive Ethernet can meet real-time requirements. 
In this architecture, services are envisioned to be dynamically updated and orchestrated on the vehicle \acp{ECU}~\cite{kakww-dsosa-19}.
Therefor \ac{SOME/IP} provides a complementary \ac{SD}~\cite{assdJR21} that detects service availability and establishes sessions between producers and consumers.
\ac{SOME/IP}, however, does not verify the authenticity of service providers.

The problem of securing \ac{SD} is not unique to the automotive domain.
On the Internet, the endpoints of services are determined with the help of the \ac{DNS}. 
Its  \ac{DNSSEC}~\cite{RFC-2535} ensure data integrity and authenticity of the \ac{DNS} records. 
In addition, \ac{DANE}~\cite{RFC-6698} binds public certificates to names to ensure the authenticity of the connection endpoint unambiguously and without attestation collisions.
\ac{DNSSEC} and \ac{DANE} are well-established and widely deployed   Internet standards with almost eight million \ac{DNSSEC} verified zones and more than half a million \ac{DANE} enabled zones on the Internet\footnote{SecSpider Global DNSSEC deployment tracking [Online].
Available: \url{https://secspider.net/stats.html} (Accessed 28.11.2022)}.

In this paper, we leverage the  \ac{DNSSEC} protocol and its operational ecosystem to solve the problem of service authenticity and certificate management in vehicles.
We focus on \ac{SOME/IP} \ac{SD} for automotive service invocation, even though our approach could be transferred to other in-vehicle protocols.
Unlike earlier proposals, which manually pre-provisioned certificates for adding authentication during session establishment~\cite{irrsvssvJR20},  our approach manages security credentials dynamically and is capable of fully functional updates.

We model \ac{SOME/IP} service descriptions as a \ac{DNS} namespace and store  parameters of \ac{SOME/IP} service endpoints in the \ac{DNS}. This allows us to bind certificates to the service names using \ac{DANE}. 
Thereafter, \ac{DNSSEC} ensures authenticity and integrity of the records following a content object security model, which allows for seamless replication of records including caching, as well as credential updates.
In cars, we verify the authenticity of the publisher endpoints with the help of a lightweight challenge-response scheme anchored at the  \ac{DANE} certificates.
We demonstrate the feasibility of our approach by extending the \ac{SOME/IP} reference implementation with access to a local \ac{DNS} resolver for service parameters and certificates during the \ac{SD}.
We compare the performance of our scheme to the reference implementation.

The remainder of this paper is structured as follows.
Section~\ref{sec:background_and_related_work} recaps the \ac{SOME/IP} \ac{SD} and related work on secure discovery of services.
Section~\ref{sec:concept} presents our concept of \ac{DNSSEC}-based \ac{SD} for publisher authenticity.
We evaluate our concept in Section~\ref{sec:eval} and discuss performance results.
Section~\ref{sec:conclusion_and_outlook} concludes with an outlook.
