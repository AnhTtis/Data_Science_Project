%!TEX root = ../main.tex

\section{Conclusion and Outlook}%
\label{sec:conclusion_and_outlook}
In this paper, we designed and analyzed basic security elements for the rapidly evolving service-oriented software architecture in future cars. In provisioning service authentication and managing attestation credentials, we  addressed the urgent demand for securing a heterogeneous, distributed, and dynamically updatable software ecosystem that will drive the connected cars of the near future.

Our work was intentionally built on well-established standards. \ac{DNSSEC} and \ac{DANE}  enable certificate management and service authenticity while being a thoroughly validated, operationally stable Internet standard. \ac{SOME/IP} is a widely accepted service-oriented middleware standardized by AUTOSAR. We demonstrated how to combine  the \ac{SOME/IP} \ac{SD} with the Internet name system in design, implementation, and evaluation. Our findings indicated that \ac{SOME/IP} \ac{SD} can interact with the DNS without operational overhead, while  \ac{DNSSEC} with \ac{DANE} contribute not only a robust, reliable security solution but also a stable infrastructure for replication, (off-line) caching, and key management.

This basic solution to automotive service security opens three future research directions. First, the remaining \ac{SOME/IP} service primitives for onboard session establishment and migration need a detailed  security design and assessment. Second, operational guidelines for namespace management and service updates in the automotive ecosystem shall be developed. Third, we aim at configuring a full-featured production-grade vehicle with our security solution and evaluate its properties in macroscopic benchmarks. 

