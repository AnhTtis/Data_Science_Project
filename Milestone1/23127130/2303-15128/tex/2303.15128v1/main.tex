\documentclass[conference,letterpaper]{IEEEtran}
\IEEEoverridecommandlockouts%
% The preceding line is only needed to identify funding in the first footnote. If that is unneeded, please comment it out.
% \overrideIEEEmargins 
\usepackage{cite}
\usepackage[para]{footmisc}
\usepackage{amsmath,amssymb,amsfonts}
\usepackage{algorithmic}
%\usepackage{enumitem}
\usepackage{graphicx}
\graphicspath{{./img/}}
\usepackage{textcomp}
\usepackage[table,xcdraw]{xcolor}
\usepackage{diagbox}
\usepackage{colortbl}
\setlength{\marginparwidth}{2cm}
\usepackage{todonotes}
\usepackage{multirow}
\usepackage{url}
\usepackage{nohyperref} %%For export activate for no hyperlinks.
%\usepackage{hyperref} %%Activate for hyperlinks.
\usepackage[per-mode=symbol,group-separator={,},group-minimum-digits={3}]{siunitx}
\sisetup{detect-weight=true, detect-family=true}
%\usepackage{subfig}
\usepackage{caption}
\usepackage{subcaption}
\usepackage[nohyperlinks,nolist]{acronym} %\ac{} ausgabe, \acl{} langausgabe, \acf{} zweites erstes mal, \acp plural
\usepackage[]{algorithm2e}
\usepackage{tikz}
\usepackage{tikz-layers}
\usetikzlibrary{calc}
\usepackage{tabularx}
\newcolumntype{Y}{>{\centering\arraybackslash}X}
\newcolumntype{L}{>{\arraybackslash}X}
\newcolumntype{R}{>{\raggedleft\arraybackslash}X}
\newcolumntype{C}[1]{>{\centering\arraybackslash}p{#1}}
\usepackage{booktabs}
\usepackage[absolute,showboxes]{textpos}
\usetikzlibrary{patterns}
\usetikzlibrary{spy}
\usetikzlibrary{arrows.meta}
\usetikzlibrary{fit}
\usepackage{pgfplots}
\pgfplotsset{compat=newest}
\usepgfplotslibrary{units}
\usepgfplotslibrary{groupplots}
\usepgfplotslibrary{statistics}
\makeatletter
\pgfplotsset{
 unit code/.code 2 args=
   \begingroup
   \protected@edef\x{\endgroup\si{#2}}\x
}
\makeatother
\usetikzlibrary{matrix}
\usetikzlibrary{shapes.misc}

% colors from topology image.
\definecolor{CoreGray}{HTML}{BFBFBF}
\definecolor{CoreBlack}{HTML}{333333}
\definecolor{CoreBlue}{HTML}{002E7D}
\definecolor{CoreGreen}{HTML}{6AAC8E}
\definecolor{CoreRed}{HTML}{C80000}
\definecolor{CoreYellow}{HTML}{E6AC00}
\definecolor{CoreWhite}{HTML}{FFFFFF}
\definecolor{CoreMiddleGray}{HTML}{e5e5e5}
% Light colors from topology image.
\colorlet{LightCoreGray}{CoreGray!20}
\colorlet{LightCoreBlack}{CoreBlack!20}
\definecolor{CoreLightBlue}{HTML}{8BBDEB}
\colorlet{LightCoreGreen}{CoreGreen!30}
\colorlet{LightCoreRed}{CoreRed!20}
\colorlet{LightCoreYellow}{CoreYellow!20}
\colorlet{LightCoreWhite}{CoreWhite!20}

\newcommand\codeword[1]{\mbox{\texttt{\textcolor{CoreBlack}{#1}}}}

\usepackage[absolute,showboxes]{textpos}

\makeatletter
\newcommand\footnoteref[1]{\protected@xdef\@thefnmark{\ref{#1}}\@footnotemark}
\makeatother

\begin{document}

% Automatically shrink down the bib entries:
\bstctlcite{my:BSTcontrol}

\title{
    Authenticated and Secure Automotive Service Discovery with DNSSEC and DANE
    % \thanks{}
}
\author{\IEEEauthorblockN{Mehmet Mueller}\IEEEauthorblockA{\href{http://www.haw-hamburg.de/ti-i}{\textit{Dept. Computer Science}},
\href{http://www.haw-hamburg.de/ti-i}{\textit{Hamburg University of Applied Sciences}}, Germany \\
\{\href{mailto:mehmet.mueller@haw-hamburg.de}{mehmet.mueller}\}@haw-hamburg.de}
}
\author{\IEEEauthorblockN{Mehmet Mueller, Timo H\"ackel, Philipp Meyer, Franz Korf, and Thomas C. Schmidt}%
\IEEEauthorblockA{\href{http://www.haw-hamburg.de/ti-i}{\textit{Dept. Computer Science}},
\href{http://www.haw-hamburg.de/ti-i}{\textit{Hamburg University of Applied Sciences}}, Germany \\
\{\href{mailto:mehmet.mueller@haw-hamburg.de}{mehmet.mueller}, %
\href{mailto:timo.haeckel@haw-hamburg.de}{timo.haeckel}, %
\href{mailto:philipp.meyer@haw-hamburg.de}{philipp.meyer}, %
\href{mailto:franz.korf@haw-hamburg.de}{franz.korf}, %
\href{mailto:t.schmidt@haw-hamburg.de}{t.schmidt}\}@haw-hamburg.de} %
}

\maketitle

\setlength{\TPHorizModule}{\paperwidth}
\setlength{\TPVertModule}{\paperheight}
\TPMargin{5pt}
\begin{textblock}{0.8}(0.1,0.02)
     \noindent
     \footnotesize
     If you cite this paper, please use the original reference:
     Mehmet Mueller, Timo H\"ackel, Philipp Meyer, Franz Korf, and Thomas C. Schmidt. 
     ``Authenticated and Secure Automotive Service Discovery with DNSSEC and DANE,'' In: \emph{Proceedings of the 14th IEEE Vehicular Networking Conference (VNC)}. IEEE, April 2023.
\end{textblock}

\begin{abstract}
    Automotive softwarization is progressing and future cars are expected to operate a \acl{SOA} on multipurpose compute units, which are interconnected via a high-speed Ethernet backbone. The AUTOSAR architecture foresees a universal middleware called \acs{SOME/IP} that provides the service primitives, interfaces, and application  protocols on top of Ethernet and IP. \acs{SOME/IP} lacks a robust security architecture, even though security is an essential in future Internet-connected vehicles. 
	In this paper, we augment the \acs{SOME/IP} service discovery with an authentication and certificate management scheme based on DNSSEC and DANE. We argue that the deployment of well-proven, widely tested standard protocols should serve as an appropriate basis for a robust and reliable security infrastructure in cars. Our solution enables on-demand service authentication in offline scenarios, easy online updates, and remains free of attestation collisions. We evaluate our extension of the common \textit{vsomeip} stack and find performance values that fully comply with car operations. 
\end{abstract}

\begin{IEEEkeywords}
    Automotive security, authentication, attestation, service orientation, SOME/IP, AUTOSAR, standards
\end{IEEEkeywords}


%%%% 	Acronyms    %%%%
% !TEX root = ../main.tex
\begin{acronym}
	% A
	\acro{ACC}[ACC]{Adaptive Cruise Control}
	\acro{ACDC}[ACDC]{Automotive Cyber Defense Center}
	\acro{ACL}[ACL]{Access Control List}
	\acro{ADS}[ADS]{Anomaly Detection System}
	\acroplural{ADS}[ADSs]{Anomaly Detection Systems}
	\acro{ADAS}[ADAS]{Advanced Driver Assistance Systems}
	\acro{API}[API]{Application Programming Interface}
	\acro{AVB}[AVB]{Audio Video Bridging}
	\acro{ARP}[ARP]{Address Resolution Protocol}
	% B
	\acro{BE}[BE]{Best-Effort}
	% C
	\acro{CAN}[CAN]{Controller Area Network}
	\acro{CBM}[CBM]{Credit Based Metering}
	\acro{CBS}[CBS]{Credit Based Shaping}
	\acro{CNC}[CNC]{Central Network Controller}
	\acro{CMI}[CMI]{Class Measurement Interval}
	\acro{CoRE}[CoRE]{Communication over Realtime Ethernet}
	\acro{CT}[CT]{Cross Traffic}
	\acro{CM}[CM]{Communication Matrix}
	% D
	\acro{DoS}[DoS]{Denial of Service}
	\acro{DDoS}[DDoS]{Distributed Denial of Service}
	\acro{DDS}[DDS]{Data Distribution Service}
	\acro{DPI}[DPI]{Deep Packet Inspection}
	% E
	\acro{ECU}[ECU]{Electronic Control Unit}
	\acroplural{ECU}[ECUs]{Electronic Control Units}
	% F
	\acro{FDTI}[FDTI]{Fault Detection Time Interval}
	\acro{FHTI}[FHTI]{Fault Handling Time Interval}
	\acro{FRTI}[FRTI]{Fault Reaction Time Interval}
	\acro{FTTI}[FTTI]{Fault Tolerant Time Interval}
	% G
	\acro{GCL}[GCL]{Gate Control List}
	% H
	\acro{HTTP}[HTTP]{Hypertext Transfer Protocol}
	\acro{HMI}[HMI]{Human-Machine Interface}
	\acro{HPC}[HPC]{High-Performance Controller}
	% I
	\acro{IA}[IA]{Industrial Automation}
	\acro{IDS}[IDS]{Intrusion Detection System}
	\acroplural{IDS}[IDSs]{Intrusion Detection Systems}
	\acro{IEEE}[IEEE]{Institute of Electrical and Electronics Engineers}
	\acro{IGMP}[IGMP]{Internet Group Management Protocol}
	\acro{IoT}[IoT]{Internet of Things}
	\acro{IP}[IP]{Internet Protocol}
	\acro{ICT}[ICT]{Information and Communication Technology}
	\acro{IVNg}[IVN]{In-Vehicle Networking}
	\acro{IVN}[IVN]{In-Vehicle Network}
	\acroplural{IVN}[IVNs]{In-Vehicle Networks}
	%J
	% L
	\acro{LIN}[LIN]{Local Interconnect Network}
	% M
	\acro{MOST}[MOST]{Media Oriented System Transport}
	% N
	\acro{NADS}[NADS]{Network Anomaly Detection System}
	\acroplural{NADS}[NADSs]{Network Anomaly Detection Systems}
	% O
	\acro{OEM}[OEM]{Original Equipment Manufacturer}
	\acro{OTA}[OTA]{Over-the-Air}
	%P
	\acro{P4}[P4]{Programming Protocol-independent Packet Processors}
	\acro{PCP}[PCP]{Priority Code Point}
	% R
	\acro{RC}[RC]{Rate-Constrained}
	\acro{REST}[ReST]{Representational State Transfer}
	\acro{RPC}[RPC]{Remote Procedure Call}
	% S
	\acro{SD}[SD]{Service Discovery}
	\acro{SDN}[SDN]{Software-Defined Networking}
	\acro{SDN4CoRE}[SDN4CoRE]{Software-Defined Networking for Communication over Real-Time Ethernet}
	\acro{SIEM}[SIEM]{Security Information and Event Management}
	\acro{SOA}[SOA]{Service-Oriented Architecture}
	\acro{SOC}[SOC]{Security Operation Center}
	\acro{SOME/IP}[SOME/IP]{\textit{Scalable service-Oriented MiddlewarE over IP}}
	\acro{SR}[SR]{Stream Reservation}
	\acro{SRP}[SRP]{Stream Reservation Protocol}
	\acro{SW}[SW]{Switch}
	\acroplural{SW}[SW]{Switches}
	% T
	\acro{TAS}[TAS]{Time-Aware Shaping}
	\acro{TCP}[TCP]{Transmission Control Protocol}
	\acro{TDMA}[TDMA]{Time Division Multiple Access}
	\acro{TSN}[TSN]{Time-Sensitive Networking}
	\acro{TSSDN}[TSSDN]{Time-Sensitive Software-Defined Networking}
	\acro{TT}[TT]{Time-Triggered}
	\acro{TTE}[TTE]{Time-Triggered Ethernet}
	\acro{TTL}[TTL]{Time to Live}
	% U
	\acro{UDP}[UDP]{User Datagram Protocol}
	\acro{UN}[UN]{United Nations}
	% Q
	\acro{QoS}[QoS]{Quality-of-Service}
	% V
	\acro{V2X}[V2X]{Vehicle-to-X}
	%W
	\acro{WS}[WS]{Web Services}
	% Z
	\acro{ZC}[ZC]{Zone Controller}

\end{acronym}

%%%%	Document    %%%%
\section{Introduction}
\label{sec:intro}
\begin{figure}[t]
\begin{center}
    \includegraphics[width=1\linewidth]{figures/teaser.pdf}
\end{center}
\vspace{-0.1in}
\caption{\textbf{{\em Foggy} vs {\em Clear} NeRF.} Our \ournerf gets rid of reconstruction errors manifested as foggy ``floaters" in the density volume without additional input or significant computational overhead. 
%
Below are density profiles along a given ray before and after our geometry correction procedure, where we discard density peaks corresponding to floaters.
}
\label{fig:teaser}
\vspace{-0.2in}
\end{figure}



%The emergence of 
Neural Radiance Fields (NeRFs)~\cite{mildenhall2020nerf}  %and its variants 
have made revolutionary contributions in %photo-realistic 
novel view synthesis~\cite{barron2021mip,barron2022mip}, 
autonomous driving~\cite{rematas2022urban,tancik2022block}, digital human~\cite{hong2022headnerf,zhao2022humannerf}, and 3D content generation~\cite{eg3d,poole2022dreamfusion,lin2022magic3d}.
%by leveraging a multi-layer perceptron (MLP) to implicitly model the mapping from input 5D coordinates (i.e., 3D coordinates $\mathbf{x} = (x,y,z)$ and 2D viewing directions $\mathbf{d}=(\theta,\phi)$) to volume density $\sigma$ and view-dependent emitted radiance color $\mathbf{c} = (r,g,b)$. 
%
%They then use traditional volume rendering mechanisms on the obtained continuous 5D function (i.e., MLP) to generate novel views. 
To date, unfortunately, most NeRF-based methods encounter challenges when tackling large-scale cluttered scenes (e.g., Fig.~\ref{fig:teaser}):
\begin{enumerate}[leftmargin=0.16in, topsep=2pt,itemsep=-1ex,partopsep=1ex,parsep=1ex]
\item Input observations used for NeRF are often too sparse  compared to forward-facing or synthetic looking-inward scenes;
%\item Recovering fine-grained objects within a large volume is challenging for NeRF; %in capturing details accurately.
\item View-dependent visual effects give rise to ambiguity, resulting in a ``foggy" density field as shown in Fig.~\ref{fig:teaser}. 
%
Such artifacts are particularly pronounced in indoor scenes strewn with view-dependent appearances, such as specular highlights, glossy surface reflections from man-made objects. 
\end{enumerate}

Despite attempts to enhance NeRF's rendering quality given suboptimal input, such as using 3D conical frustums~\cite{barron2021mip,barron2022mip}, physically-grounded augmentations~\cite{chen2022aug}, and misalignment correction~\cite{jiang2022alignerf},  these challenges have yet to be fully resolved.
%
Depth supervision~\cite{deng2022depth, wei2021nerfingmvs} or proxy geometry~\cite{xu2021scalable,wu2022scalable} images can help alleviate the challenges in handling large-scale with sparse input, at the expense of %but they come at the cost of requiring 
expensive pre-processing or additional input.
%
Another line of work~\cite{wang2021neus, oechsle2021unisurf, wang2022neuris} achieves better reconstruction of surface geometry by using signed distances instead of volume density as scene representation. However, they sacrifice the ability to synthesize photo-realistic novel views.

%We observe that NeRF has been suffering from foggy ``floater" artifacts in large-scale cluttered scenes.
%
%Such artifacts are particularly pronounced in indoor scenes strewn with view-dependent appearances from man-made objects. 
%
To address the above issues, we propose an extension to NeRF, dubbed as {\bf \ournerf}, which enforces effective {\em appearance} and {\em geometry} constraints conducive to accurate colors and 3D densities estimation. We believe \ournerf can contribute beyond novel view synthesis, such as NeRF object detection~\cite{hu2022nerf}, NeRF object segmentation~\cite{zhi2021place, liu2022unsupervised, fan2022nerf,ren2022neural}, and NeRF registration~\cite{goli2022nerf2nerf}, where the rooms for improvement are substantial if more accurate color and density estimation are available.

Correspondingly, there are two steps in \ournerf. First, for appearance correction, the view-independent and view-dependent color components are predicted from the underlying 3D scene, which is combined to produce the final color estimation (Fig.~\ref{fig:toaster}).
%
The view-independent component (diffuse color and shading) captures the overall scene color, while the view-dependent component (highlights or reflections) captures color variations due to changes in viewing angle.
%
\ournerf then discards these view-dependent appearances in the training views to prevent them from interfering with the density estimation.
%
Second, a simple and effective geometry correction procedure will be performed to further eliminate the foggy ``floaters" or density errors. This geometry correction procedure is based on an assumption in line with traditional ray tracing in computer graphics.
\begin{comment}
% xh: basically copying method
On the other hand, ClearNeRF performs a geometric correction procedure performed on each traced ray during inference to refine the density estimation and better tackle the floater artifacts. 
%
The geometry correction procedure assumes that there should only be one salient peak along each traced ray during NeRF inference. 
Only the salient peak closest to the ray origin (the camera center) corresponds to  true geometry while the others will be manifested as foggy floaters hovering in the density volume. 
%
This assumption is in line with traditional ray tracing in computer graphics where in the absence of noise, only one intersection per ray should be returned to indicate the closest ray-object intersection.
%
\end{comment}
%%%%%%%%%%%
%As shown in Fig.~\ref{fig:teaser}, when reconstructing an indoor scene with sparse input and highly view-dependent objects, NeRF produces severe floating artifacts due to its attempt to explain view-dependent appearances.
%
Experiments verify that our proposed \ournerf can effectively get rid of floater artifacts without additional input.% or significant computational overhead. 


In summary, our contributions include the following:
\begin{itemize}[leftmargin=0.16in, topsep=2pt,itemsep=-1ex,partopsep=1ex,parsep=1ex]
    \item We propose a concise method for decomposing view-independent and view-dependent appearance during NeRF training and eliminate the interference of view-dependent appearance.
    \item We propose a geometric correction procedure performed on each traced ray during inference to refine the density estimation and better tackle the floater artifacts.
    \item Extensive experiments and ablations verify the effectiveness of our core designs and results in improvements over the vanilla NeRF and other state-of-the-art alternatives.
    %without additional computational resources or other inputs.
\end{itemize}




\section{Related Work}

 %Most MMLA studies so far have primarily focused on developing prototypes and testing the functionality of different combinations of sensors and analytics approaches \citep{shankar2018review, mu2020multimodal, noroozi2020multimodal}. 
In this section, we review the most recent systematic literature reviews on MMLA and related works that have identified several prominent logistical, privacy and ethical challenges that need to be addressed for this promising area to remain relevant and have an actual impact on educational practices.

%MMLA and multimodal data have received increased attention from the learning analytics and educational technology communities as a promising research direction that holds the potential to generate meaningful insights about teaching, and learning in partially and non-computer mediated educational contexts \citep{sharma2020multimodal, chango2022review}. 
% \citep{alwahaby2021evidence, yan2022scalability}. 

%\subsection{Practical Challenges}

%The practical challenges of MMLA are mainly associated with the lack of well-reported and large-scale studies that structurally assess the influence of MMLA innovations on actual educational practice. For example, \citet{alwahaby2021evidence} reviewed 100 MMLA articles and concluded that most of the empirical evidence presented in prior studies remains descriptive, correlational, or anecdotal, with little strong causal evidence regarding the impacts of MMLA innovations on real-world educational practices. 


%Consequently, although the alignment between MMLA innovations and learning design should be one of the foundations for developing MMLA innovations \citep{cukurova2020promise, ochoa_multimodal_2022}, such alignments are rarely considered or reported in the existing literature, as evidenced in recent reviews \citep{sharma2020multimodal, praharaj2021literature}. 
%Likewise, the lack of MMLA studies that closed the LA loop by providing feedback to students or insights to teachers during educational practices instead through post-hoc research-focused interviews or surveys also hindered the understanding of MMLA innovations' actual impacts on learning and teaching outcomes \citep{yan2022scalability}. %Therefore, in-depth insights on aligning MMLA innovations with learning designs, relevant theories, and educational stakeholders are urgently needed to ensure future MMLA studies do not deviate from the ultimate goals of learning analytics \citep{gavsevic2015let}.

\subsection{Logistical Challenges}
Most MMLA studies so far have primarily focused on developing prototypes and testing the functionality of different combinations of sensors and analytics approaches \citep{shankar2018review, mu2020multimodal, noroozi2020multimodal}. Yet, many concerns have been raised regarding the logistical challenges that can emerge when moving from controlled settings to in-the-wild MMLA deployments such as the added intrusiveness of sensing devices and complexity in their installation and orchestration \citep{chua2019technologies}. \citet{yan2022scalability} systematically reviewed these logistical issues and identified a relatively low level of technology readiness regarding existing MMLA innovations, resulting in heavy reliance on the onsite support of researchers or technicians. This  undermines the sustainability of these systems and unnecessarily increases the complexity of the learning situation from the teachers' perspective. While most of the sensing technologies used in MMLA research can be purchased off-the-shelf, implementing these technologies in authentic physical learning spaces often requires extensive technical background for tasks such as physical installation, system integration, and modalities synchronisation \citep{crescenzi2020multimodal, shankar2018review, mu2020multimodal}. There is also a trade-off between data quality and affordability as most of the MMLA innovations that rely on mature sensing technologies, such as location sensors, eye-trackers, and biometric sensors, can be financially unscalable due to the high unit prices \citep{yan2022scalability}. Although low-cost alternatives are emerging \citep[e.g.,][]{ochoa2018rap, saquib2018sensei}, these technologies remain in the prototype and validation stages and often sacrifice accuracy or portability for affordability. 

Likewise, the lack of MMLA studies that have closed the LA loop by providing some form of end-user interface to students or insights to teachers make it harder for educational stakeholders to weigh the benefits against the potential added complexity to their already rich educational ecologies \citep{yan2022scalability}. Although the alignment between MMLA innovations and learning design should be one of the foundations for developing MMLA innovations \citep{cukurova2020promise, ochoa_multimodal_2022}, such alignments are rarely considered or reported in the existing literature, as noted in recent literature reviews \citep{sharma2020multimodal, praharaj2021literature}. This can undermine teacher and student confidence, if they do not understand how the MMLA system aligns with their teaching practices or learning outcomes. 

All of these challenges are hallmarks of emerging HCI infrastructures that must be co-evolved with work practices. This in-the-wild MMLA deployment offered the opportunity to study how both educational and technical stakeholders learnt to work together to address the challenges.%Gaining insights regarding this trade-off between data quality and affordability could benefit educational researchers and practitioners when evaluating their budgets against the type of educational insights they are trying to capture. 

\subsection{Privacy Challenges}
As a research area that benefits from the data collection opportunities enabled by various sensing technologies, the privacy issues surrounding the adoption of MMLA innovations are the focus of critical debate. \citet{crescenzi2020multimodal} emphasised the need to consider the privacy implications of using sensing technologies to generate analytics about children's activity. Such implications have also been identified by students and teachers who have expressed concerns regarding the security of their data \citep{mangaroska2021challenges, kasepalu2021teachers}. These privacy implications of MMLA innovations have been under-investigated in the literature \citep{Alwahaby2022, yan2022scalability, Oviatt2018challenges}. Specifically, while most works published in MMLA  mention that informed consent was obtained from participants, none of the existing works has elaborated on the consenting strategies they adopted, which could contribute valuable insights regarding data security measures for protecting individual privacy and maximising data autonomy (e.g., individuals' autonomy of removing their data from the database) \citep{beardsley2020enhancing}. Additionally, while most of MMLA innovations endeavour to provide dashboards and visualisations for supporting educational practices, privacy issues regarding who has the right to see these visualisations  remain unclear, especially in the contexts of collaborative learning where, in most cases, individuals' personal trace data, even anonymised (e.g., masking students' identity with numbers or colours), could remain identifiable when used for provoking reflections at a group-level, since other students typically have the contextual knowledge to decode anonymised representations \citep{mangaroska2021challenges, Alwahaby2022}. Providing additional empirical evidence on educational stakeholders' perspectives of these privacy-related issues could potential benefit the on-going development of MMLA, and is a particular focus of this study.

\subsection{Ethical Challenges}
Beyond logistical and privcy issues, the potential biases in analytics, and cognitive dissonances that may be caused by the inconsistency between individuals' observations and generated insights, could also undermine the potential benefits of MMLA innovations \citep{ferguson2016guest,Oviatt2018challenges}. Such issues are vital as the accuracy of the existing MMLA-based predictive models and early-warning systems are far from suitable for practical deployment (e.g., rarely above 80\% accuracy), and these models have mostly been developed and evaluated based on relatively small sample sizes (i.e., with n < 50) \citep{yan2022scalability}. These small sample sizes combined with the poor reporting standards found in the existing MMLA literature could also mask potential algorithmic biases that may disadvantage certain minority groups of students as replicating these studies remain difficult without adequately reported methodologies \citep{luzardo2014estimation, yan2022scalability}. Additionally, \citet{Alwahaby2022} also highlighted the significant concerns regarding the need to enhance trust and data transparency within MMLA systems and \citet{yan2021footprints} suggested that more research needs to be done to assess the potential risk of making decisions with incomplete multimodal data.%Additionally, using unsupervised machine learning techniques to cluster and label students may also induce the potential risk of discrimination, where certain labels (e.g., at-risk)  could negatively impact learners' self-esteem and educators' expectations \cite{higgins2002stages}. 
Consequently, understanding the ethical practices of using these analytics is also essential but rarely considered in prior literature \citep{selwyn2019s} and requires the participation of key educational stakeholders such as students and educators \citep{Oviatt2018challenges}.  A large-scale in-the-wild study opens new opportunities to study approaches to these ethical challenges under more authentic conditions than has been reported to date. 

\subsection{Contribution to HCI and Research Question}
Against the literature reviewed above we formulate the following research question (RQ) that guided our study: 

\textit{\textbf{RQ:} What logistical, privacy and ethical challenges emerge from a complex MMLA, in-the-wild study that closes the analytics loop by providing direct feedback to students?}

In addressing this question, the contribution of this paper is a set of lessons learnt regarding how such challenges were, or could have been, addressed in the context of a two-year deployment of a MMLA system in an authentic educational scenario. The implications of this study should assist researchers, developers and designers in making informed decisions about the effective deployment of innovations that involve the use of ubiquitous computing technologies, sensing devices and artificial intelligence (AI) algorithms to augment teaching and learning in physical spaces. 

% Reviews I reckon you already cited in the Scalability paper and which I used in the intro
% \cite{sharma2020multimodal}
% \cite{crescenzi2020multimodal}
% \cite{chua2019technologies}
% \cite{alwahaby2021evidence}

% Note for Jimmie - Other SLR on MMLA reviews to be inlcuded: 
% \cite{noroozi2020multimodal}
% \cite{shankar2018review}
% \cite{praharaj2021literature}
% \cite{mu2020multimodal}
% \cite{yan2022scalability} %of course!
% \cite{chango2022review}

\begin{figure*}[ht]
    \centering
    %\begin{adjustbox}{width=0.5\textwidth}
    \setlength{\tabcolsep}{-0pt}{%
    \begin{tabular}{cccc}
    \hspace{0.0in}
    \includegraphics[height=4.3cm, trim={0.1cm 0.02cm 0.1cm 0.08cm},clip]{images/1_concept/1_concept_fixed_backbone_v1.pdf} &\hspace{0.2in}
    \includegraphics[height=4.3cm, trim={0.1cm 0.02cm 0.1cm 0.08cm},clip]{images/1_concept/1_concept_biased_transfer_v1.pdf} &\hspace{0.2in}
    \includegraphics[height=4.3cm, trim={0.1cm 0.02cm 0.1cm 0.08cm},clip]{images/1_concept/1_concept_slective_re_used_v1.pdf} &\hspace{0.2in}
    %\includegraphics[height=4.5cm, trim={0.1cm 0.02cm 0.1cm 0.08cm},clip]{images/1_concept/1_concept_wsns.pdf} \\ 
    \includegraphics[height=4.3cm, trim={0.1cm 0.02cm 0.1cm 0.08cm},clip]{images/1_concept/1_concept_soft_wsns_v1.pdf} \\ 
    
    \hspace{0.0in}
    \makecell{\small (a) Fixed Backbone \\ \small (Piggyback, SupSup) } &\hspace{0.2in}
    \makecell{\small (b) Biased \small Transfer \\ \small (PackNet, CLNP)} &\hspace{0.2in}
    \makecell{\small (c) Selective Reuse Expansion \\ \small beyond Dense Network (APD)} &\hspace{0.2in}
    \makecell{\small (d) Selective Reuse Expansion \\ \small within Network (WSN)} \\
    \end{tabular}
    %\end{adjustbox}
    }
    \vspace{-0.05in}
    %\caption{\textbf{Concept Comparison:} (a) Piggyback \cite{mallya2018piggyback}, and SupSup \cite{wortsman2020supermasks} find the optimal binary mask on a fixed backbone network a given task (b) PackNet \cite{mallya2018packnet} and CLNP \cite{golkar2019continual} forces the model to reuse all features and weights from previous subnetworks which causes bias in the transfer of knowledge (c) APD \cite{Yoon2020} selectively reuse and dynamically expand the dense network (d) Our WSN selectively reuse and dynamically expand subnetworks within a dense network. \textcolor{Green}{Green edges} are reused weights, and the softness of weights is represented in the line thickness.}
    \caption{\textbf{Concept Comparison:} (a) Piggyback \cite{mallya2018piggyback}, and SupSup \cite{wortsman2020supermasks} find the optimal binary mask on a fixed backbone network a given task (b) PackNet \cite{mallya2018packnet} and CLNP \cite{golkar2019continual} forces the model to reuse all features and weights from previous subnetworks which causes bias in the transfer of knowledge (c) APD \cite{Yoon2020} selectively reuse and dynamically expand the dense network (d) Our WSN selectively reuse and dynamically expand subnetworks within a dense network. \textcolor{Green}{Green edges} are reused weights.}
    \label{fig:concept_comparison}
    \vspace{-0.1in}
\end{figure*}
% !TEX root =  main.tex
%%%%%%%%%%%%%%%%%%%%%%%%%%%%%%%%%%%%%%%%%%%%%%%%%%%%%%%%%%
%%%%%%%%%%%%%%%%%%%%%%%%%%%%%%%%%%%%%%%%%%%%%%%%%%%%%%%%%%
%%%%%%%%%%%%%%%% input performance table
% !TEX root =  main.tex
%%%%%%%%%%%%%%%%%%%%%%%%%%%%%%%%%%%%%%%%%%%%%%%%%%%%%%%%%%


\begin{figure}[!t]
\renewcommand{\arraystretch}{0.9}
\centering
	%%%%%%%%%%%%%%%%%
	\scalebox{0.89}{\begin{tabular}{l|c|c|rr}
		    \multicolumn{1}{c|}{Name, source for ${T^{\text{in}}}$} &
		    \multicolumn{1}{c|}{
		    %$\mathbf{T^{\text{in}}}$\textbf{: } 
		  %  \textbf{ Source},
		    ${C^{\text{in,}\star}_\text{size}}$, ${S^{\text{in,}\star}}$ for ${T^{\text{in}}}$}
		    &
		  %  \multicolumn{1}{c}{$\mathbf{S^{\text{in,}\star}}$} &
		  %  \multicolumn{1}{c}{$\mathbf{C^{\text{in,}\star}_\text{size}}$} &
		  %  \multicolumn{1}{c|}{\textbf{Source}} &
		    \multicolumn{1}{c|}{${S^\text{quiz}}$ \scriptsize$\in \substructures({S^{\text{in,}\star}})$\normalsize} &
		 \multicolumn{1}{c}{${\#C^\text{quiz}}$} & \multicolumn{1}{c}{${\#T^\text{quiz}}$}
			 \\
			\toprule
			\multirow{3}{*}{
			\shortstack{
			T-1
			\\
		    \footnotesize{\textbf{HOC:Maze08}~\cite{hourofcode_maze}}
			}
			}
			& 
			\multirow{3}{*}{
			\shortstack[c]{
% 			Source = \footnotesize{HOC:Maze08}
% 		    \quad \quad \quad \quad \quad \ \hspace{0.25em}
% 			$C^{\text{in,}\star}_\text{size} = 6$ 
			$6$ 
			\\
% 			$S^{\text{in,}\star} = $ \text{\footnotesize{\{\DSLRun\{\DSLRepeat; \DSLRepeat{}\}\}}}
		   \text{\scriptsize{\{\DSLRun\{\DSLRepeat; \DSLRepeat{}\}\}}}
			}
			}
			& 
			\text{\scriptsize{\{\DSLRun{}}\}} 
			& 
           	$22$ & $220$
			\\
			&&
			\text{\scriptsize{\{\DSLRun\{\DSLRepeat{}}\}\}} 
			& $34$ & $340$
			\\
			&&
			\ $S^{\text{in,}\star}$
			& $179$ & $1790$
			\\
			\hline
			%%%%%%%%
			\multirow{3}{*}{\shortstack{
			T-2
			\\
			\footnotesize{\textbf{HOC:Maze16}~\cite{hourofcode_maze}}
			}}&
			\multirow{3}{*}{\shortstack[c]{
% 			Source = \text{\footnotesize{HOC:Maze16}}
% 			\quad \quad \quad \quad \quad \ \hspace{0.25em}
% 			$C^{\text{in,}\star}_\text{size} = 5$ 
			$5$ 
			\\
% 			$S^{\text{in,}\star} = $\text{\footnotesize{\{\DSLRun\{\DSLRUntil\{\DSLIf{}\}\}\}}}
		\text{\scriptsize{\{\DSLRun\{\DSLRUntil\{\DSLIf{}\}\}\}}}
			}
			}
			&
			\text{\scriptsize{\{\DSLRun{}}\}} 
			& 
			$10$ & $100$
			\\
            &&
            \text{\scriptsize{\{\DSLRun\{\DSLRUntil{}\}\}}} 
			& $6$ & $60$
			\\
			&& 
			\ $S^{\text{in,}\star}$
			& $19$ & $190$
			\\
			\hline
			%%%%%%%%%
			\multirow{3}{*}{\shortstack{
			T-3
			\\
			\footnotesize{\textbf{HOC:Maze18}~\cite{hourofcode_maze}}
			}}&
			\multirow{3}{*}{\shortstack[c]{
% 			Source = \text{\footnotesize{HOC:Maze18}}
% 			\quad \quad \quad \quad \quad \ \ \
		    $5$
			\\
		 \text{\scriptsize{\{\DSLRun\{\DSLRUntil\{\DSLIfElse{}\}\}}}
			}
			}
			& 
			\text{\scriptsize{\{\DSLRun{}\}}} 
			& 
			$10$ & $100$
			\\
			&&
			\text{\scriptsize{\{\DSLRun\{\DSLRUntil{}\}\}}} 
			& $6$ & $60$
			\\
			&&
			\ $S^{\text{in,}\star}$
			& $9$ & $90$
			\\
			\hline
			%%%%%%%%
			\multirow{4}{*}{\shortstack{
			T-4
			\\
		    \footnotesize{\textbf{HOC:Maze20}~\cite{hourofcode_maze}}
			}}
			&
			\multirow{4}{*}{
			\shortstack[c]{
		  %  Source = \footnotesize{HOC:Maze20}
		  %  \quad \quad \quad \quad \quad \ \
			$7$ 
			\\
			\scriptsize{\{\DSLRun\{\DSLRUntil\{\DSLIfElse\{\{\};\{\DSLIfElse{}\}\}\}\}\}}
			}
			}
			&
			\text{\scriptsize{\{\DSLRun{}}\}} 
			& $10$ & $100$
			\\
			&&
			\text{\scriptsize{\{\DSLRun\{\DSLRUntil{}\}\}}} 
			& $6$ & $60$
			\\
			&&
			\text{\scriptsize{\{\DSLRun\{\DSLRUntil\{\DSLIfElse{}\}\}\}}} 
			& $9$ & $90$
			\\
			&& 
			\ $S^{\text{in,}\star}$
			& $10$ & $100$
			\\
			\hline
			%%%%%%%
			\multirow{3}{*}{\shortstack{
			T-5
			\\
			\footnotesize{\textbf{Karel:Opposite}~\cite{intro_to_karel_codehs}}
			}}
			 &
			\multirow{3}{*}{
			\shortstack[c]{
% 			Source = \footnotesize{Karel:BallSpot}
% 			\quad \quad \quad \quad \quad
			$6$
			\\
			\text{\scriptsize{\{\DSLRun\{\DSLRepeat\{\DSLIfElse{}\}\}\}}}
			}
			}
			&
			\text{\scriptsize{\{\DSLRun{}}\}} 
			& $73$ & $730$
			\\
			&&
			\text{\scriptsize{\{\DSLRun\{\DSLRepeat{}\}\}}} 
			& $118$ & $1180$
			\\
			&& 
			\ $S^{\text{in,}\star}$
			& $343$ & $3430$
			\\
            \hline
			%%%%%%%%%%%%%
			\multirow{2}{*}{\shortstack{
			T-6
			\\
			\footnotesize{\textbf{Karel:Diagonal}~\cite{intro_to_karel_codehs}}
			}}
			&
			\multirow{2}{*}{
			\shortstack[c]{
% 			Source = \text{\footnotesize{Karel:Diagonal}}
% 			\quad \quad \quad \quad \quad
			$8$
			\\
			\text{\scriptsize{\{\DSLRun\{\DSLWhile{}\}\}}}
			}
			}
			&
			\text{\scriptsize{\{\DSLRun{}}\}} 
			& $447$ & $4470$
			\\
			&&
			\ $S^{\text{in,}\star}$
			& $579$ & $5790$
			\\
		\bottomrule
   \end{tabular}
   }
	\caption{\looseness-1\algmultihop~applied to six HOC and Karel reference tasks; see Section~\ref{sec:evaluation} for details. For brevity, sketches have been abbreviated, e.g., \DSLRepeatUntil{}(\DSLBoolGoal) as \DSLRUntil.
	%
    }
\label{fig:experiments.analysis}
%\vspace{-3mm}
\end{figure}
%reference

%%%%%%%%%%%%%%%%%%%%%%%%%%%%%%%%%%%%%%%%
\section{\algmultihop~on Real-World Tasks}\label{sec:evaluation}
\looseness-1In this section, we present the performance of \algmultihop~on six reference tasks taken from real-world block-based programming platforms: HOC~\cite{hourofcode_maze} and Karel~\cite{intro_to_karel_codehs}. The set of these tasks along with their sources are mentioned in Fig.~\ref{fig:experiments.analysis}. These tasks differ in complexity, measured in terms of the programming constructs of their solution code as illustrated by the diversity of their respective solution sketches ${S^{\text{in,}\star}}$. For the exhaustive set of substructures of $S^{\text{in,}\star}$, 
% of these tasks
Fig.~\ref{fig:experiments.analysis} lists the total number of pop quizzes, in the form of unique task-code pairs ($T^\text{quiz}, C^\text{quiz}$), generated by our algorithm.
As can be seen in the figure, our algorithm generates $50$ to $1000$s of pop quizzes for each substructure. For any potential student attempt on these tasks, Stage 1 of \algmultihop~ would generate one of these task-specific substructures by design -- hence, for every attempt we can present several unique yet adaptive pop quizzes to the student. Note that, our algorithm generates higher number of tasks than codes for each substructure. This is because the task synthesis methodology used in Stage 2(ii) can generate more than one task for a single code in Stage 2(ii) of Fig.~\ref{fig:pipeline.abstract}. In particular, for each new code, we obtain $10$ diverse tasks. For instance, Fig.~\ref{fig:intro} and Fig.~\ref{fig:karel.illustration} illustrate pop quizzes generated by \algmultihop{} for the specific student attempts on tasks T-4 and T-5, respectively.
%
































\section{Discussion and Limitations}

Although we can ablate concepts efficiently for a wide range of object instances, styles, and memorized images, our method is still limited in several ways. First, while our method overwrites a target concept, this does not guarantee that the target concept cannot be generated through a different, distant text prompt. We show an example in \reffig{limitation} (a), where after ablating {\menlo Van Gogh}, the model can still generate {\menlo starry night painting}. However, upon discovery, one can resolve this by explicitly ablating the target concept {\menlo starry night painting}. Secondly, when ablating a target concept, we still sometimes observe slight degradation in its surrounding concepts, as shown in \reffig{limitation} (c). 

\nupur{Our method does not prevent a downstream user with full access to model weights from re-introducing the ablated concept~\cite{ruiz2022dreambooth,kumari2022multi,gal2022image}. Even without access to the model weights, one may be able to iteratively optimize for a text prompt with a particular target concept. Though that may be much more difficult than optimizing the model weights, our work does not guarantee that this is impossible.}

Nevertheless, we believe every creator should have an ``opt-out'' capability. We take a small step towards this goal, creating a computational tool to remove copyrighted images and artworks from large-scale image generative models.

%\newpage

%%%% 	BibTeX		%%%%
%\bibliographystyle{plain}
\bibliographystyle{IEEEtran}
\bibliography{bibtex/HTML-Export/all_generated,bib/own,bib/security,bib/rfcs,bib/literature,bib/bibliography}

\end{document}
