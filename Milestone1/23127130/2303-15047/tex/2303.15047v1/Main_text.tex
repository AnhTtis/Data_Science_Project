%% ****** Start of file aiptemplate.tex ****** %
%%
%%   This file is part of the files in the distribution of AIP substyles for REVTeX4.
%%   Version 4.1 of 9 October 2009.
%%
%
% This is a template for producing documents for use with 
% the REVTEX 4.1 document class and the AIP substyles.
% 
% Copy this file to another name and then work on that file.
% That way, you always have this original template fi hile to use.

\documentclass[aip,graphicx]{revtex4-1}
\usepackage{graphicx}
%\usepackage{epstopdf,epsfig}
\usepackage{float}
\usepackage{newtxtext}
\usepackage{newtxmath}
\usepackage{natbib}
\usepackage{hyperref}
\usepackage{amsmath}
\usepackage{color}
\usepackage{nicefrac}
\usepackage{scalerel}
\usepackage{stackengine}
\usepackage{siunitx}
\usepackage{MnSymbol}
\stackMath

\usepackage{placeins}
\usepackage[dvipsnames]{xcolor}
\hypersetup{
	colorlinks = true,
	urlcolor   = blue,
	citecolor  = black,
}
\newtheorem{lemma}{Lemma}
\newtheorem{corollary}{Corollary}
\newcommand{\RomanNumeralCaps}[1]

\begin{document}
	
	\newcommand{\Upsilonbar}{\stackon[-7pt]{\Upsilon}{-}}
	\renewcommand{\um}{\si{\micro\metre}}
	\renewcommand{\us}{\si{\micro\second}}
	\renewcommand{\ul}{\si{\micro l}}

% Use the \preprint command to place your local institutional report number 
% on the title page in preprint mode.
% Multiple \preprint commands are allowed.
%\preprint{}

\title{Bubble nucleation and jetting inside a millimetric droplet} %Title of paper


% repeat the \author .. \affiliation etc. as needed
% \email, \thanks, \homepage, \altaffiliation all apply to the current author.
% Explanatory text should go in the []'s, 
% actual e-mai\textbf{}l address or url should go in the {}'s for \email and \homepage.
% Please use the appropriate macro for the type of information

% \affiliation command applies to all authors since the last \affiliation command. 
% The \affiliation command should follow the other information.

\author{Juan Manuel Rossell\'{o}}
\email{jrossello.research@gmail.com}
\affiliation{Otto von Guericke University Magdeburg, Institute of Physics, Universitätsplatz 2, 39106 Magdeburg, Germany.}
\affiliation{Faculty of Mechanical Engineering, University of Ljubljana, A\v{s}ker\v{c}eva 6, 1000 Ljubljana, Slovenia.}
\author{Hendrik Reese}
\author{K. Ashoke Raman}
\author{Claus-Dieter Ohl}
\affiliation{Otto von Guericke University Magdeburg, Institute of Physics, Universitätsplatz 2, 39106 Magdeburg, Germany.}

%\email[]{Your e-mail address}
%\homepage[]{Your web page}
%\thanks{}
%\altaffiliation{}


% Collaboration name, if desired (requires use of superscriptaddress option in \documentclass). 
% \noaffiliation is required (may also be used with the \author command).
%\collaboration{}
%\noaffiliation

\date{\today}

\begin{abstract}
In this work, we present experiments and simulations on the nucleation and successive dynamics of laser-induced bubbles inside liquid droplets in free-fall motion, i.e. a case where the bubbles are subjected to the influence of a free boundary in all directions. The droplets of a millimetric size are released from a height of around 20\,cm and acquire a velocity of around 2\,m/s at the moment the bubble is nucleated. Within this droplet, we have investigated the nucleation of secondary bubbles induced by the rarefaction wave that is produced when the shock wave emitted by the laser-induced plasma reflects at the drop surface. Interestingly, three-dimensional clusters of cavitation bubbles are observed. Their shape is compared with the negative pressure distribution computed with a CFD model and allows us to estimate a cavitation threshold value. In particular, we observed that the focusing of the waves in the vicinity of the free surface can give rise to explosive cavitation events that end up in fast liquid ejections. 
High-speed recordings of the drop/bubble dynamics are complemented by the velocity and pressure fields simulated for the same initial conditions. The effect of the proximity of a curved free surface on the jetting dynamics of the bubbles was qualitatively assessed by classifying the cavitation events using a non-dimensional stand-off parameter $\Upsilonbar$ which depends on the drop size, the bubble maximum radius and the relative position of the bubble inside the drop. Here, we found that the curvature of the free surface does not play a determinant role on the jet dynamics, being the distance to the surface the dominant parameter. 
The oscillation of the laser-induced bubbles promote the onset of Rayleigh-Taylor and Rayleigh-Plateau instabilities, observed on the drop's surface. This phenomenon was studied by varying the ratio of the maximum radii of the bubble and the drop. The specific mechanisms leading to the destabilisation of the droplet surface were identified through a careful inspection of the high speed images. 
\end{abstract}

\pacs{}% insert suggested PACS numbers in braces on next line

\maketitle %\maketitle must follow title, authors, abstract and \pacs

% Body of paper goes here. Use proper sectioning commands. 
% References should be done using the \cite, \ref, and \label commands

\section{Introduction}
\label{Intro}

Phase explosion in confined liquid volumes has recently gained interest because of its connection with thriving research areas like x-ray liquid crystallography\cite{Grunbein2021}, x-ray holography~\cite{Vassholz2021,Hagemann2021}, extreme UV light, and plasma generation~\cite{Favre2002}. A better understanding of the interaction of high-power lasers with small liquid particles is also relevant in laser-based atmospheric monitoring techniques~\cite{Rohwetter2010,Mei2015} or in optical atomisation techniques that can be applied to the production of airborne transported micro-drops used as drug carriers~\cite{Lee2022}. At the heart of all of these research fields is the injection of high-power photons into a small liquid sample, the initiation of phase transition from liquid to vapour, the rapid pressure fluctuations, and the successive complex fluid mechanics driven by this impulsive energy input. In this study, we want to shed light on the fundamental flows that can be induced in liquid samples once this phase transition has been initiated. In particular, we focus on the fluid dynamics within a spherically confined liquid sample after the violent phase explosion of the vapour bubble induced by a high-power laser pulse. We explore the non-spherical dynamics of vapour bubbles within a liquid droplet, i.e. surrounded by free boundaries only. Bubble dynamics in droplets have so far mostly been studied from the perspective of destabilisation of the liquid-gas interface of the droplet. Here, we explore the bubble dynamics within the droplet. 

Pulsed lasers can be focused into optically transparent media to induce explosive bubble nucleation by dielectric breakdown. This process is accompanied by the emission of an acoustic shock wave with an amplitude on the order of GigaPascals depending on the pulse energy, duration, and wavelength. For instance, the initial amplitude of the shock wave (i.e. at the edge of the plasma rim) in water can be in the range from $2.4\,$GPa to $11.8\,$GPa for a $6\,$ns laser pulse with an energy between $1\,$mJ to $10\,$mJ and a wavelength of $1064\,$nm focused with a numerical aperture (NA) of $22^{\circ}$ \cite{Vogel1996,Noack1998}. Recently, the initial shock wave amplitude produced by similar nanosecond laser pulses of $24\,$mJ (NA = $10^{\circ}$) was measured with a novel x-ray probing technique, obtaining peak values of around $20\,$GPa~\cite{Vassholz2021}.

When a laser-induced cavity is produced in a confined space with free boundaries, like a droplet, most of the sound wave energy reflects back from the interface with an inverted phase, meaning that the original shock wave is transformed into a rarefaction wave. If the negative pressure amplitude of the reflected wave is below the cavitation threshold of the liquid, a trail of bubbles is nucleated after the wave passage. This effect is commonly observed upon wave reflection on the free boundary of a flat surface~\cite{Heijnen2009}, nearby bubbles~\cite{Quinto-Su2013}, a liquid column~\cite{Sembian2016} or, as we already mentioned, a drop~\cite{Obreschkow2006,Gonzalez-Avila2016,Kyriazis2018,Biasiori-Poulanges2021}. Laser cavitation in some of these configurations was lately applied in studies involving x-ray holography or x-ray diffraction to investigate the propagation of shock waves in liquids~\cite{Stan2016,Ursescu2020,Hagemann2021}. The use of very small amounts of liquid prevents the x-rays from being fully absorbed by the sample, thus improving the contrast of the x-ray images. This technique is suitable to study the properties of opaque liquids without optical aberrations, it is less sensitive to distortions produced by wavy surfaces, and also allows retrieving information about the liquid density changes produced by the passage of the pressure waves~\cite{Vassholz2021,Vassholz2023}, which represents an advantage over traditional optical imaging.  

Another interesting aspect of the nucleation of bubbles in the proximity of a boundary resides in their jetting dynamics. Laser-induced bubbles produced under different boundary conditions have been widely studied, both experimentally and numerically. Perhaps the case that got the most attention is the one of a bubble collapsing in the proximity of a boundary of large extent, e.g. a solid boundary~\cite{Plesset1971,Lauterborn1975,Blake1999,Brujan2002,Lindau2003,Xiang2013,Lechner2017,gonzalez2021}, an elastic boundary~\cite{Brujan2001,Rossello2022b}, or a free surface~\cite{Koukouvinis2016a,Li2019,Bempedelis2021,Rossello2022}. In real-world conditions, the boundary is of finite extent and the cavity may be spuriously affected by more than a single boundary (for instance, the walls of a container or the liquid free surface), exerting a considerable influence the direction of the jetting\cite{Kiyama2021,Andrews2022}.

The jet dynamics are frequently characterised by a stand-off parameter~\cite{Lindau2003,Supponen2016,Lauterborn2018} computed as the ratio of the distance between the bubble nucleation position and the boundary ($d$) and the maximum radius attained by the bubble after its creation ($R_{max}$). If the cavity collapse occurs next to boundaries other than a plane, for instance, irregular or curved surfaces~\cite{TOMITA2002,Blake2015,Wu2018,Li2019c,Aganin2022} like pillars~\cite{Koch2021b,Kadivar2021}, fibres~\cite{Mur2023}, corners~\cite{Zhang2020,Mahmud2020}, crevices~\cite{Trummler2020,Andrews2020}, perforated plates~\cite{GonzalezAvila2015,Reese2022}, or spheres~\cite{Zhang2018,Li2019b,Zevnik2020,Ren2022}, the anisotropy does not have one predominant direction and thus the use of a single stand-off parameter (e.g. $d/R_{max}$) is no longer sufficient to fully characterise the system. The same situation arises in cases where the bubbles are produced in a constricted space, for example in narrow channels~\cite{Gonzalez-Avila2011,Wang2018,Brujan2022}, between two surfaces~\cite{Li2017a,Liu2017}, in a liquid column~\cite{Robert2007}, or inside a drop~\cite{Obreschkow2006,Thoroddsen2009,Gonzalez-Avila2016,Zeng2018}.

The dynamics of jetting bubbles inside drops or curved free surfaces have not been extensively explored. Recently, we have reported experimental and numerical results on the formation of a jetting bubble in the proximity of a curved free boundary, given by the hemispherical top of a water column or a drop sitting on a solid plate~\cite{Rossello2022}. As a natural extension of that work, we now present a study on the jet formation during the collapse of laser-induced bubbles inside a falling drop. This is a particularly interesting case as the bubble is surrounded entirely by a free boundary. From an experimental point, the intrinsic curvature of the liquid surface offers a very clear view into the bubble's interior.

The rapid acceleration induced by the bubble oscillations in the proximity of a free boundary also gives rise to surface instabilities, in particular Rayleigh-Taylor instabilities (RTI)\cite{Taylor1950,Keller1954,Zhou2017a,Zhou2017b}. This situation is more pronounced when the oscillating bubble wall gets close to the free surface, as commonly occurs in reduced volumes like a drop~\cite{Zeng2018,Klein2020}. The Rayleigh-Taylor instability produces corrugated patterns on the liquid surface that can grow and promote the onset of other instabilities like the Rayleigh-Plateau instability. Furthermore, the multiple pits and ripples produced by the RTI on the liquid surface can interact with the acoustic emissions of the oscillating bubble to generate a fluid focusing which results in a thin outgoing liquid jet~\cite{Tagawa2012,peters2013}.

This article is organised into different sections focusing on one of the above-discussed aspects, i.e. the shock wave dynamics and the nucleation of secondary cavitation bubbles, the jetting dynamics of the collapsing laser-induced bubbles, and the formation of instabilities on the drop surface as a consequence of the bubble oscillation.

\section{Experimental method}
\label{Exp_Method}

The experimental method used to achieve controlled laser bubble inception inside a millimetric drop is depicted in figure~\ref{fig:Exp_setup}(a). Individual drops were released from the tip of a blunt metallic needle with an internal diameter of 330\,\um~(and an external diameter of 600\,\um) by the action of an electronic syringe pump \emph{KD Scientific - Legato SPLG110}. This device pushed a fixed volume of $\sim9$\,\ul~of deionised water through the needle, producing single drops with a radius of ($1.42\pm0.01$)\,mm. After a drop was released it traveled a distance of $h=30\,$cm in free-fall motion. Just before it impacted a glass plate a pulsed laser was focused into the droplet to nucleate the cavitation bubble.

\begin{figure}
\centering\includegraphics[width=0.75\textwidth]{Figure_1.jpg}
\caption{\label{fig:Exp_setup} Description of the experimental setup. (a) A water drop with a volume of $\sim9$\,\ul~is detached from a cylindrical blunt needle (stainless steel, 600\,\um~of external diameter) by gravitational forces. When the drop reaches a velocity of ($1.7\pm0.1$)\,m/s a cavitation bubble is produced inside it by a laser pulse with a duration of 4\,ns and a typical energy of ($2.4\pm0.1$)\,mJ. (b) Once reflected from the drop surface, the shock waves emitted from the laser-induced bubble nucleate tiny bubbles inside the liquid drop. (c) The bubble undergoes an asymmetric collapse with jetting, whose shape depends on the position of the bubble inside the drop.}
\end{figure}

The pulse energy of the laser (Nd:YAG \emph{Q2-1064} series, pulse duration 4\,ns, wavelength 1064\,nm) could be varied between 1.9\,mJ and 20.3\,mJ and was focused with a microscope objective (\emph{Zeiss LD Achroplan $20\times$}, $\text{NA}=0.4$) see bottom of figure~\ref{fig:Exp_setup}(a). In the experiments, a standard microscope slide was placed on top of the laser focusing objective in order to prevent wetting of its outer lens, which would provoke a significant distortion of the laser beam. Accordingly, the protective glass was meticulously cleaned after each drop impact.

The fall distance $h$ was sufficient for the surface tension to stabilise the liquid into an approximately spherical shape, reaching a velocity of ($1.7\pm0.1$)\,m/s upon laser arrival. At the same time, the variation of the lateral position of the drop centre relative to the laser focus was typically below 200\,\um, which aids experimental repeatability. The vertical position where the bubble is created within the droplet is controlled with some precision by synchronizing the laser pulse with the passage of the drop through a light gate. This consists of a red laser diode paired with a photo-diode that triggers a digital delay generator \emph{Quantum 9520} which then fires the laser after a specified time.

The dynamics of the cavitation bubble within the droplet and the resulting surface instabilities were captured in high-speed videos using a \emph{Shimadzu XPV-X2} camera equipped with a photography macro lens \emph{Canon MP-E 65\,mm f/2.8 1-$5\times$}. A diffused back illumination from a continuous white LED lamp \emph{SMETec} (9000\,lm) in combination with the curved nature of the drops allowed us to obtain clear images of the droplet interior. Furthermore, the curvature of the liquid refracted the light in a way that reveals the internal structures of the jetting bubbles, knowing that it distorts the apparent position and shape~\cite{Koch2021}. For direct comparison of the experimental and the numerical results, an \emph{in-house} script was applied to the simulated results to compensate for such image distortions~\cite{Martins2018}. This correction (based on Snell's law) was also used to obtain the ``real'' nucleation position of the laser bubble.

Due to the limited number of recorded frames, the framing rate of the high-speed videos had to be adjusted to capture the important features of the phenomena under study. For instance, to visualise the shock wave propagation and the resulting nucleation of bubbles from the reflected rarefaction wave, see  figure~\ref{fig:Exp_setup}(b)) required a frame rate of 5\,Mfps (i.e. the maximum achievable by the camera), while the temporal evolution of the jets (depicted in figure~\ref{fig:Exp_setup}(c)) and the instabilities of the drop surface are captured already at 200\,kfps or 500\,kfps, respectively.

\subsection{Definition of a stand-off parameter for a curved boundary $\Upsilonbar$}
\label{Parameter_definition}

In order to consider the curvature of the surface in the characterisation of the jet dynamics, we defined a non-dimensional coefficient $\Upsilonbar$ that combines two non-dimensional numbers, each one representing a relevant dimension of the problem. First, we use the stand-off distance $D^*$ \cite{Lauterborn2018} as the ratio of the bubble ``seeding'' position ($d$) and the maximum radius achieved by the bubble when produced in the centre of the spherical drop ($R^*_{max}$). The second non-dimensional distance $\chi$ is given by the ratio of the drop radius ($R_d$) and the distance of the bubble from the drop centre ($r$). Summarizing,

\begin{align}
  D^* &= \frac{d}{R^*_{max}}\\
  \chi &= \frac{R_d}{r} = \frac{R_d}{R_d-d}\\
  \Upsilonbar &=D^*\,\chi=\frac{d}{R^*_{max}}\,\frac{R_d}{(R_d-d)}
\end{align}

\begin{figure}[ht]
\centering\includegraphics[width=0.5\textwidth]{Figure_2.jpg}
\caption{\label{fig:Dim_par} Schematic of the drop interior with relevant dimensional parameters.}
\end{figure}

A schematic representation of the aforementioned parameters is presented in figure~\ref{fig:Dim_par}. Here, $R^*_{max}$ is tightly related to the energy of the laser pulse~\cite{Lauterborn2018} and, as we explain later in Section~\ref{Bubble_Dynamics}, it also varies slightly with the drop size as $R_d\xrightarrow\,\infty$. In any case, the use of $\Upsilonbar$ should be limited to values of $R^*_{max}$ for which the bubble is contained inside the drop volume (i.e. $0\leq R^*_{max}<R_d$) and the drop shape is not significantly distorted by surface instabilities~\cite{Zeng2018}. Additionally, the bubble must be initially produced inside the drop, meaning that $d\leq R_d$.

In principle, the parameter $\Upsilonbar$ behaves in a similar way as the traditional stand-off distance (e.g. $d/R_{max}$), however, the addition of $\chi$ as a weighting factor represents a measure of the influence of the boundaries on all around the bubble, and not only its closest point. This means that the regions of the free surface in directions other than $\theta = 0º$ could also be relevant to the bubble dynamics as the separation from the bubble and the boundary in those angular directions gets smaller, i.e. when the radius $R_d$ is decreased and the bubble is located at a reduced $d$. The tight relation between the traditional stand-off distance and $\Upsilonbar$ is also evidenced by the following considerations and limiting cases: 

\begin{itemize} 
    \item $\Upsilonbar$ rises monotonically with $d$ for a fixed laser pulse energy (or $R^*_{max}$).
    \item In the limit $R_d\xrightarrow\,\infty$ the traditional stand-off distance is recovered.
    \item If the bubble is near the drop wall, $d \xrightarrow\,0$, then $\Upsilonbar\xrightarrow\,0$.
    \item If the bubble is near the drop centre, $d \xrightarrow\,R_d$, then $r=0,\,\Upsilonbar\,\xrightarrow\,\infty$, and we recover the traditional unbounded case, in which the bubble collapses spherically due to symmetry.
\end{itemize}

It is important to note that $\Upsilonbar$ can take the same value for different combinations of $d$, $R^*_{max}$, and $R_d$. Therefore, a comparison between cases could be made by fixing the value of one or two of the parameters. For example, the effect of the surface curvature $R_d$ on the bubble dynamics can be evaluated by maintaining $D^*$, or the influence of the ``seeding'' depth $d$ can be studied by fixing the drop size $R_d$ and the energy of the laser pulse. In this way, the parameter preserves the same functionality as the traditional stand-off parameter~\cite{Lauterborn2018}, but now includes the surface curvature dimension.

\subsection{Numerical simulations}
\label{Numerical}

Volume-of-Fluid simulations were carried out in \emph{OpenFOAM-v2006}~\cite{OpenFOAM-Foundation2021} using a modification of the solver \emph{compressibleMultiphaseInterFoam}.
This modified version is called \emph{MultiphaseCavBubbleFoam} and was already implemented in previous works to study the formation of the ``bullet jet"~\cite{Rossello2022} and micro-emulsification~\cite{AshokeRaman2022}. In those works, similar simulations of a single expanding and collapsing bubble in the vicinity of a liquid-gas and a liquid-liquid interface were performed, respectively.
Since the solver is explained in detail there, we will only give the information that is specific to the present case of a bubble created in a free-falling liquid drop.

Considering the approximate rotational symmetry of the experimental configuration, we carried out the simulations as quasi-two-dimensional.
The computational domain represents a slice of a cylindrical domain with a height of 3\,mm and a radius of 3\,mm, which is filled with a gas representing the surrounding air at ambient pressure.
The domain is divided into a square mesh of cells with a width of 40\,\um, which is then further refined to a cell width of 10\,\um~in the region occupied by the liquid drop.
The boundaries of the domain in the radial and axial directions are open, wave transmissive boundaries.\\
A slightly prolate ellipsoidal liquid drop representing a falling water drop is initiated in the centre of the cylinder with an axial radius of 1440\,\um~and a radial radius of 1400\,\um.
We neglect the relative motion of the drop through the air, and thus take the drop and the air to be initially at rest.
We also neglect any subsequent gravitational acceleration, since it is small on the time scales considered.
Inside the drop, a bubble is seeded on the symmetry axis with an initial over-pressure of 1.69\,GPa and an initial radius of 25.7\,\um.\\
The bubble contents are modelled with the same properties as the gas surrounding the liquid droplet but are calculated as a separate component. This allows us to apply a mass correction to the gas in the bubble only that accounts for the mass loss due to condensation during the bubble's first oscillation cycle. More details can be found in our previous work~\cite{Rossello2022}. The surface tension between the liquid and the gases is 0.07\,mN/m, and that between the gases is 0.
The Tait equation of state is used for all components,
\begin{equation}
    p=(p_0+B)\Big(\frac{\rho}{\rho_0}\Big)^\gamma-B~,
    \label{eq:tait}
\end{equation}
 with the parameters given in table~\ref{tab:simParams}. $\gamma$ is here the adiabatic exponent.\\
 The output of the numerical data was done in intervals of 10\,ns to capture shock wave propagation dynamics, and every 1\us~for the bubble and jetting dynamics.
 
\begin{table}
  \begin{center}
\def~{\hphantom{0}}
    \begin{tabular}{rrrrrr}
        & ~~~$B$ in MPa & ~~~$\rho_0$ in kg/m$^3$ & ~~~$p_0$ in Pa & $\gamma$ & ~~~$\mu$ in mPa\,s\\
        liquid & 303.6 & 998.2061 & 101325 & ~~~7.15 & 1\\
        gases & 0 & 0.12 & 10320 & 1.33 & 0.013
    \end{tabular}
    \caption{Tait equation of state parameters and dynamic viscosities $\mu$ of the simulated fluid components. Both gaseous components are treated as the same type of gas.}
    \label{tab:simParams}
  \end{center}
\end{table}

\section{Results and discussion}
\label{Results}

The inception of a laser-induced bubble inside a liquid drop gives rise to a rich and complex chain of events. We start with an overview of the fluid dynamics that are observed following the creation of the cavitation bubble by the dielectric rupture of the liquid, as shown in figure~\ref{fig:Stages}. Here, the bubble is nucleated off-centre and close to the upper interface of the droplet. The fluid dynamics can be divided into three stages, which are discussed in detail in the later sections. For now, we provide a brief description of these 3 stages: (1) The bubble is nucleated into a rapidly expanding vapour cavity that launches during its deceleration a shock wave into the droplet, not visible in figure~\ref{fig:Stages}. Upon reflection at the acoustic soft liquid-gas interface, the rarefaction wave propagates through the drop leaving behind a trail of cavitation bubbles in certain regions where the wave convergence produces sufficient tension to induce local acoustic cavitation, $\le 2\,\mu$s$\,t\,\le 6\,\mu$s in figure~\ref{fig:Stages}. Depending on the location of the laser bubble the rarefaction wave may focus in a reduced volume close to the interface, creating secondary cavitation and provoking the ejection of a single jet at the opposite side of the laser bubble nucleation site (e.g. $t>6\,\mu$s in figure~\ref{fig:Stages}). (2) In the second stage, the laser-induced bubble undergoes an asymmetrical collapse from its maximum size. Here, the anisotropy of the boundary conditions results in the formation of a jet, which starts as an indentation on one side of the cavity and grows to pierce the bubble at the opposite extreme. In cases where the laser cavity is created near the drop surface, we also observe the destabilisation of the liquid surface by a Rayleigh-Taylor instability. (3) In the third and last stage, the bubble re-expands after jetting, adopting a liquid-gas structure that depends mostly on the stand-off distance (i.e. $\Upsilonbar$). On its second collapse, the cavity fragments and later disperses due to the complex flow created by its first collapse.

In the following, the reported values of $\Upsilonbar$ are computed for a surface curvature of 1.42\,mm, which corresponds to the mean radius of the drops produced in this work.

\begin{figure}[ht]
\centering
\includegraphics[width=0.9\textwidth]{Figure_3.jpg}
\caption{\label{fig:Stages} Stages of the events developing inside the drop. In the first stage (framed in red) a rarefaction wave (i.e. the reflection of the shock wave) produces a trail of cavitation bubbles. For low values of $\Upsilonbar$ a liquid jet is ejected from the extreme of the drop opposite to the bubble inception. In the second stage (framed in blue) the bubble collapses after reaching its maximum size and a jet forms. In some cases, a Rayleigh-Taylor instability (RTI) is observed near the bubble. In the third stage (framed in green) the bubble re-expands after jetting and adopts a characteristic shape that depends mostly on $\Upsilonbar$. The width of each frame is $2.70\,$mm. The numbers indicate the time in \us~after the laser shot.}
\end{figure}

\subsection{Acoustic cavitation nucleation}
\label{Acoustic_cav}

The specific shape of the cavitation bubble clusters produced by the passage of the rarefaction wave is highly dependent on $\Upsilonbar$. This is because the negative pressure focuses differently when the original shock wave is emitted from a different location. As the acoustic nucleation only occurs below a certain pressure threshold, the resulting bubble clouds can assume complex three-dimensional structures. Figure~\ref{fig:Ac_cav_dif_gamma} presents experimental results showing the temporal evolution of bubble clouds generated for different values of $\Upsilonbar$. In this study, the bubble ``seeding'' position was varied by changing the delay between the drop release and the laser shot, thus shifting the laser focus position along the vertical symmetry axis of the drop.

\begin{figure}
\centering\includegraphics[width=0.86\textwidth]{Figure_4.jpg}
\caption{\label{fig:Ac_cav_dif_gamma} Acoustic cavitation inside the drop. The distribution of bubbles in the liquid changes significantly with the position of the laser-induced bubble. The frame width is 3.15\,mm. The time between consecutive frames is 600\,ns. (a) $\Upsilonbar=0.65$. (b) $\Upsilonbar=1.1$. (c) $\Upsilonbar=1.7$. (d) $\Upsilonbar=2.5$. (e) $\Upsilonbar=7.5$. (f) $\Upsilonbar=68$. (g) $\Upsilonbar=13$. (h) $\Upsilonbar=5.4$. (i) $\Upsilonbar=1.8$. (j) $\Upsilonbar=0.9$. Full videos of (b), (d) and (f) are available in the online supplementary movies 1-3.}
\end{figure}

Independently of the shock wave intensity, due to its spherical shape the drop acts as an acoustic chamber with two pressure anti-nodes of opposite pressure signs, one produced at the laser focal spot and the other one located over the symmetry axis at a distance $r$ from the centre. This means that after reflecting on the drop walls, the acoustic waves will focus on a spot located at the same distance from the drop centre but on the opposite side of the drop. 

For the case of the epicentre of the spherical shock wave being located near the surface (i.e. $\Upsilonbar$\,$\lesssim$\,1), the simulations show that the pressure minima are arranged as a ring (or a toroid) with its symmetry axis collinear with the drop's vertical axis, advancing from one extreme of the drop to the other, while changing its radius to match the lateral drop cross-section. In other words, the sound reflects multiple times on the drop walls and travels circumferentially near the liquid surface without a significant loss of intensity. This kind of negative pressure distribution dynamics generates the spherical shell of tiny cavitation bubbles close to the liquid surface displayed in figure~\ref{fig:Ac_cav_dif_gamma} in the panels (a) to (c) and also (i) to (j). As aforementioned, the sound wave reflects in a way such that the negative pressure focuses on the opposite drop extreme. This phenomenon usually referred to as ``whispering gallery effect''~\cite{Raman1922}, produces explosive cavitation events close to the free boundary and on the drop's vertical axis. The rapid expansion of those larger cavitation bubbles gives rise to the liquid jets shown in the first row of figure~\ref{fig:Stages}. A more detailed explanation of the formation and dynamics of this particular type of jet will be published elsewhere.

As the laser focusing depth $d$ is increased, the negative pressure is distributed in larger regions, but still, the nucleation of bubbles predominantly occurs on the side opposite to the laser focus. Additionally, the bubble clusters turn from having the structure of a shell (see panels (g), (h), and (i) of figure~\ref{fig:Ac_cav_dif_gamma}) into a volumetric cavitation cloud when the laser bubble is generated near the drop centre, as shown in the panels (e) and (f) of figure~\ref{fig:Ac_cav_dif_gamma}. This transition can be explained by analysing the pressure distribution dynamics with the numerical simulations~\cite{Ando2012,Quinto-Su2013,Gonzalez-Avila2016}. Figure~\ref{fig:Press_distri} demonstrates the clear correlation between the evolution of the acoustic pressure profile and the nucleation of secondary cavitation bubbles. Furthermore, this correlation can be used to determine the cavitation pressure threshold of the liquid by comparing the shape and the location of the negative pressure front with the shape of the bubble cloud within the drop. Such a comparison was only possible after applying a numerical algorithm to the simulated results to compensate for the image distortions induced by the drop curvature. The last frames in panels (a) and (b) of figure~\ref{fig:Press_distri} display an overlap of both the experimental video frames and the simulated pressure profiles. From the measurements, we found a consistent cavitation threshold of approximately 4.5\,MPa. Considering that we did not filter the water sample, we assume that the cavitation is most likely heterogeneous.

\begin{figure}[ht]
\centering\includegraphics[width=0.9\textwidth]{Figure_5.jpg}
\caption{\label{fig:Press_distri} Acoustic cavitation bubble clouds for laser-induced bubbles at different relative positions in the drop. The frames compare the advance of the shock/tension waves within the drop with the observed nucleation sites. The average drop diameter is ($2.84\pm0.05$)\,mm in all cases. The last frame of each series presents an overlay of the frames and the cumulative minimum pressure after the first reflection of the shock wave at the free boundary. The red line indicates the isobar of -4.5\,MPa, i.e. the approximate nucleation threshold pressure. (a) Here, the bubble is slightly off-centre (i.e. $d\,\simeq R_{d}$). (b) $\Upsilonbar$\,=\,5.4. (c) Change in the cluster dimensions with increasing laser pulse energy (indicated in mJ). (d) and (e) present evidence of the formation of complex hollow three-dimensional bubble structures. Here, $\Upsilonbar$ is 3.5 and 1.45, respectively.}
\end{figure}

The acoustic cavitation thresholds reported for water in the literature vary strongly, depending on the measurement method, water purity, gas saturation, and water temperature. Atchley \emph{et al.}~\cite{Atchley1988} used distilled, deionised, and filtered (0.2\,\um) tap water irradiated by pulsed ultrasound and found thresholds between 0.5 and 2.0\,MPa, depending on the pulse duration and frequency. Sembian \emph{et al.}~\cite{Sembian2016} subjected a water column to a single shock wave and found a cavitation threshold between 0.42 and 2.33\,MPa. Biasiori-Poulanges and Schmidmayer~\cite{Biasiori-Poulanges2021} compared numerical simulations and experiments of a liquid drop subjected to a planar shock wave and found a threshold between 0.37 and 2.4\,MPa. Assuming homogeneous nucleation, Ando \emph{et al.}~\cite{Ando2012} and later Quinto-Su \emph{et al.}~\cite{Quinto-Su2013} found a cavitation threshold of 60\,MPa and 20\,MPa, respectively, comparing experiments and simulations of a reflected shock wave at a free boundary. Therefore, the threshold value obtained in this work falls around the middle of the spectrum of values measured by other authors. Figure~\ref{fig:Press_distri}(c) evidences a growth in the secondary bubble cluster with increasing energy of the laser pulse, demonstrating the resulting shift in the location of the cavitation threshold isobar for higher amplitudes of the initial shockwave.   
It is relevant to point out that VoF simulations are notorious for numerical diffusion, which causes the shock wave to smear out over time. Because of this, the simulations may underestimate the pressures reached in the experiments. Another factor that might lower the precision of our method to determine the pressure inside the drop comes from the phase transition, which is not modelled in the VoF simulations. Finally, panels (d) and (e) of figure~\ref{fig:Press_distri} exemplify some of the hollow three-dimensional bubble structures observed in the experiments.
 
\subsection{Bubble jetting}
\label{Bub_Jet}

In the second stage presented in figure~\ref{fig:Stages} the laser-induced bubble reaches its maximum radius and then collapses. At this point, it becomes clear that a non-uniform distance between the bubble and the free surface produces an asymmetric collapse, which culminates in a liquid jet. In this section, we explore the effect of varying the parameter $\Upsilonbar$ (as performed in section \ref{Acoustic_cav}), but this time we lay focus on the development of the jets, as shown in figure~\ref{fig:Jets}. 

\begin{figure}[ht]
\centering\includegraphics[width=1\textwidth]{Figure_6.jpg}
\caption{\label{fig:Jets} Bubble jetting is produced by a laser-induced bubble generated at different relative positions inside the drop. The numbers indicate the time in \us. The length of the scale bars is 1\,mm. (a) Spherical oscillation case, $\Upsilonbar=203$. (b) Standard jet case, $\Upsilonbar=3.9$. (c) $\Upsilonbar=1.5$. (d) $\Upsilonbar=0.44$. (e) Bullet jet case, $\Upsilonbar=0.22$. Full videos are available in the online supplementary movies 4-8.}
\end{figure}

The experiments reveal that, as the position of the laser focus is varied between the centre and the surface of the drop, the characteristics of the jetting change smoothly: For large values of $\Upsilonbar$, a spherical rebound of the bubble without any jetting is observed. The values of $\Upsilonbar\gtrsim3.5$ are accompanied by the formation of a very thin liquid jet crossing through the centre of a weakly deformed bubble. In this ``standard jet'' case, the tip of the jet separates from the main cavity when it starts to collapse during its second oscillation cycle (see figure~\ref{fig:Jets}(b)). For $1.2\lesssim\Upsilonbar\lesssim3.5$, as in panel (c) of figure~\ref{fig:Jets}, the ``whispering gallery'' effect becomes relevant, causing the inception of larger acoustic bubbles on the side opposite to the laser cavity and the ejection of liquid driven by their expansion. The deformation of the bubble in its rebound phase is significantly stronger than in panel (b) of figure~\ref{fig:Jets}. As the laser is focused closer to the drop's surface, i.e. $0.3\lesssim\Upsilonbar\lesssim1.2$, the expansion of the bubble provokes the onset of a Rayleigh-Taylor instability. This can be seen in figure~\ref{fig:Jets} (d) by the formation of several ``spikes'' growing from the thin liquid film trapped between the cavity and the surrounding air. At the same time, the bubble collapse (from $t=66$\,\us) results in an elongated cavity, similarly as in the ``bullet jet'' case~\cite{Rossello2022}. This behaviour is more pronounced for even smaller stand-off distances, as presented in figure~\ref{fig:Jets} (e). The dynamics of this particular jet are described in detail in Ref.~\cite{Rossello2022} and correspond to the case where the laser cavity is generated almost directly on the surface of the drop (i.e. $0.01\lesssim\Upsilonbar\lesssim0.3$). Here, atmospheric gas is trapped after the closure of a conical ventilated splash and later dragged into the liquid by the liquid jet that grows from a stagnation point located on the top of a ``water bell'' (at the bottom of the frame at $t=36$\,\us). As a result, an elongated gas cavity is shaped and driven across the drop.

The combined effects of the curved shape of the drop in addition to the diffuse illumination lead to images of the interior of the gas cavity with remarkable clarity. A few examples of this are presented in figure~\ref{fig:Jet_interior}. 

\begin{figure}[ht]
\centering\includegraphics[width=1\textwidth]{Figure_7.jpg}
\caption{\label{fig:Jet_interior} Detailed view of the interior of a jetting bubble. The time between frames is 2\,\us. (a) Jet formation for $\Upsilonbar=1.6$. The frame width is 1.46\,mm. (b) Comparison between experimental data and a simulation performed for $\Upsilonbar=2.9$. The frame width is 1.38\,mm. (c) Spray produced by air entering the gas cavity (in which the pressure is lower than the atmospheric pressure) while the jet is formed. The frame width is 2.11\,mm}
\end{figure}

Panels (a) and (b) of figure~\ref{fig:Jet_interior}, reveal the temporal evolution of the liquid indentation into the bubble, as well as the toroidal shape acquired by the gas upon its collapse. Moreover, figure~\ref{fig:Jet_interior}(b) demonstrates the accuracy of the numerical simulations to reproduce the jetting process. In panel (c) we see how a perforation of the thin liquid sheet between the cavity and the atmosphere resulted in a spray of aerosol droplets ejected into the cavity during jetting. This event can be explained by the lower pressure inside the bubble compared to the atmospheric pressure and the disruption of the liquid on the upper side of the drop caused by the RTI. The spray front spreads into the cavity and collide with the lower wall of the bubble, disrupting the smoothness of the interface.

The bullet jet case of figure~\ref{fig:Jets}(e) distinguishes itself from the other cases by its unique features, i.e. its enhanced shape stability and its formation from an open splash, but also by the near robustness against the surrounding fluid and geometry. Bullet jets have been observed in shallow waters~\cite{Rossello2022b} and near flexible or rigid materials, without these conditions affecting their dynamics. Furthermore, in a previous work~\cite{Rossello2022}, we demonstrated that the bullet jet is scalable and independent of the orientation of the surface with respect to gravity. Here, we expand the list of remarkable robustness by showing it to exist of various sizes even within a highly curved and finite volume, see figure~\ref{fig:Bullet_jets}.

\begin{figure}[ht]
\centering\includegraphics[width=1\textwidth]{Figure_8.jpg}
\caption{\label{fig:Bullet_jets} Scalability of the bullet jet in a millimetric droplet. The measurements, organised in columns, show bullet jets formed from different splash sizes. In each column, the upper frame shows the time at which the water bell closes. In the lower frame, composed of two vertical stripes, the time at which the bullet jet is fully developed is shown on the left, and a frame illustrating the position of the jet tip at an advanced time indicated in \us~is shown on the right.}
\end{figure}

The images depict that the penetration depth of both the gas and the liquid conforming to the bullet jet is proportional to the initial splash size. For instance, in figure~\ref{fig:Bullet_jets}(a) the jet loses its momentum and stops around the middle of the drop, but it crosses the drop for the larger splashes shown in the panels (d) and (e). Remarkably, in those cases, the bullet jet occupies almost the entire drop while still preserving its characteristic features.

The physics behind the evolution of the bubble jetting cases classified in figure~\ref{fig:Jets} can be further explained with the aid of numerical simulations, as presented in figure~\ref{fig:Num_Sim_Jet}. 

\begin{figure}[ht]
\centering\includegraphics[width=1.0\textwidth]{Figure_9.jpg}
\caption{\label{fig:Num_Sim_Jet} Numerical simulations of the temporal evolution of jets produced inside the drop for different $\Upsilonbar$. The simulated drop has a height of 1.44\,mm and a width of 1.4\,mm as measured in the experiments. The plot shows the gas and liquid phases along with the velocity field. The time between frames is 26\,\us~for (a)-(c) and 30\,\us~for (d). (a) Spherical oscillation case, $\Upsilonbar\rightarrow\infty$. (b) Standard jet case, $\Upsilonbar=3.896$. (c) $\Upsilonbar=1.518$. (d) Bullet jet case, $\Upsilonbar=0.028$.}
\end{figure}

Figure~\ref{fig:Num_Sim_Jet}(a) depicts a purely radial oscillation of both the gas and liquid, found when the bubble is placed in the centre of the drop (i.e. $\Upsilonbar\rightarrow\infty$). The simulations shown in panels (b) and (c) of figure~\ref{fig:Num_Sim_Jet} were computed using the same $\Upsilonbar$ measured from the experimental cases displayed in the corresponding panels of figure~\ref{fig:Jets}. The agreement between the simulations and the experiments is remarkable. This can be seen in some of the morphological features that characterise the dynamics of each type of jet at different stages, like the width of the indentation formed during bubble piercing, the shape of the cavity after the first rebound, and the way in which the second collapse evolves in each case. The agreement is also found in the precise timings at which the collapse and the maximum expansion of the bubbles take place in the subsequent oscillation cycle. More details on noteworthy features are provided below in figure~\ref{fig:Jet_Second_collapse}.

In panels (a) to (c) of figure~\ref{fig:Num_Sim_Jet} the bubble is initiated with a much larger pressure than the atmospheric gas outside the drop. This pressure difference, which is constant in all directions, accelerates the liquid between the two gas domains. Since this force is proportional to the pressure gradient, the liquid gets accelerated more strongly between the bubble and the nearest part of the drop surface (where the liquid is thinner), causing the drop to bulge out in that location. Within the first few microseconds of the explosive bubble expansion, the pressure within the bubble decreases rapidly and reaches values much smaller than the atmospheric pressure. Thus, the pressure gradient changes its direction and now accelerates the liquid towards the bubble, which first slows down the bubble's expansion and afterward causes its collapse. In the same way as in the expansion phase, the thinnest part of the liquid experiences the strongest acceleration, which ultimately leads to a liquid jet indenting the bubble from the nearest part of the drop surface.

The case presented in figure~\ref{fig:Num_Sim_Jet}(d) differs greatly from the previous cases by the fact that now the bubble is close enough to the drop surface to generate an open cavity, allowing the ejection of the initially pressurised gas inside the cavity into the atmosphere, and later the flow of gas into the expanded cavity before the splash closes again. Once the cavity is closed, it remains with an approximate atmospheric pressure, which prevents it from undergoing a strong collapse as it occurs in the previously discussed cases (a) to (c). The radial sealing of the splash forms an axial jet directed toward the centre of the drop, which pierces the bubble and drags it through the drop. More details on the mechanisms behind the bullet jet formation can be found in Reference~\cite{Rossello2022}.

As a consequence of the conservation of momentum, the collapse of the gas cavity gives origin to a stagnation point, from which the liquid flows both inside the pierced bubble and away from it in opposite directions. In particular, the stagnation point is not stationary but moves along the axis of symmetry, following a different trajectory in each case. In the case of figure~\ref{fig:Num_Sim_Jet}(b) the stagnation point shifts towards the surface as the bubble moves deeper into the drop. For the case in figure~\ref{fig:Num_Sim_Jet}(c) the stagnation point does not reach the surface and its movement is less pronounced. In the bullet jet case, shown in figure~\ref{fig:Num_Sim_Jet}(d), the stagnation point forms on the apex of the water bell (i.e. the splash after its closure). It then trails the bell's collapse and remains very close to the drop surface afterward, moving slightly towards the drop centre while the bullet jet moves across the drop.

\subsubsection{Cavity dynamics on its second collapse}

After the jetting, the dynamics of the subsequent re-expansions and collapses of the cavities are characterised by the bubble's and the drop's distorted shapes and even more complicated flow fields. A good example of this can be found in the second collapse of the bubbles analysed in figure~\ref{fig:Jet_Second_collapse}, which shows a significant dependence on $\Upsilonbar$. 

\begin{figure}[ht]
\centering\includegraphics[width=1\textwidth]{Figure_10.jpg}
\caption{\label{fig:Jet_Second_collapse} Detailed collapse dynamics of the gas cavity immediately after the jetting of the laser bubble. Experimental (a) and simulated (b) view of the ``weak'' jet obtained when $\Upsilonbar=3.9$. The images were taken at 200\,kfps. (c) Ring formation after the necking of the cavity typically observed on cases with $\Upsilonbar\approx1.9$. The images were taken at 500\,kfps. (d) Direct comparison between experiment and simulation, revealing the precise flow pattern leading to the ring detachment (indicated by the white arrows). The time between frames is 2.5\,\us.}
\end{figure}

Figure~\ref{fig:Jet_Second_collapse} compares the shape taken by the bubble for two cases with $\Upsilonbar=3.9$ (panels (a) and (b)) and $\Upsilonbar=1.9$ (panels (c) and (d)). Interestingly, the flattened side of the ``teardrop'' shape acquired by the cavity after the re-expansion develops a curved indentation during its second collapse. The numerical simulations make clear that such an indentation is created by the flow produced by an uneven pressure gradient on the cavity surface. The shape of this ring-shaped indentation visibly changes with $\Upsilonbar$. For example, the case presented in figure~\ref{fig:Jet_Second_collapse}(c) displays an annular bubble necking with the detachment of two gaseous rings as the cavity shrinks. These concentric rings have two different diameters and are arranged in two distinct planes, as highlighted in figure~\ref{fig:Jet_Second_collapse}(d).

\subsubsection{Influence of $R_d$ on the jet dynamics}

The bubble dynamics observed in the falling drop case have many similarities with what is typically seen in bubbles collapsing near a planar rigid surface~\cite{Lauterborn2018} or a planar free surface~\cite{Supponen2016,Rossello2022}. Moreover, the analysis of the values of the stand-off parameter $D^*$ reveals that each type of jet (classified according to figure 7 in Ref.~\cite{Rossello2022}) occurs in a comparable range of values of $D^*$. One example of the latter can be found in figure~\ref{fig:Exp_curvature}. 

\begin{figure}[ht]
\centering\includegraphics[width=1\textwidth]{Figure_11.jpg}
\caption{\label{fig:Exp_curvature} Comparison of cases with similar bubble dynamics and a different curvature of the free surface $R_d$. The panels (a) and (c) showcases where the cavity is produced near a flat free surface, while the cases in (b) and (d) are generated inside a drop with a mean radius of 1.42\,mm. Here, the numbers represent the time normalised with the time of collapse of the cavities from each case. (a) Here, $D^*=0.85$. (b) $D^*=0.88$. (c) $D^*=1.6$. (d) $D^*=1.37$.}
\end{figure}

The parallel found between cases with dissimilar curvature of the liquid surface suggests that, contrarily to the reported observations for bubbles collapsing near concave solid surfaces~\cite{Aganin2022}, $R_d$ does not have a dominant role in the evolution of the jetting cavities when the bubbles are located near the free boundary. This statement was confirmed by the numerical simulations depicted in figure~\ref{fig:Sim_curvature}. There, the dynamics of identical bubbles expanding and collapsing near the surface of the drop, or the flat free surface of an ideally infinite pool, are compared for three stand-off distances $D^*$.  

\begin{figure}[ht]
\centering\includegraphics[width=1\textwidth]{Figure_12.jpg}
\caption{\label{fig:Sim_curvature} Jetting dynamics of identical bubbles produced near a flat surface or the curved surface of a droplet ($R_d = 1.42\,\um$). The non-dimensional time, indicated by the numbers, was normalised with the collapse time of each bubble. (a) Here, $D^*=0.61$. (b) $D^*=1.02$. (c) $D^*=1.23$.}
\end{figure}

The simulations show that the correspondence between the flat and the curved surface cases is gradually lost when the bubble is placed further away from the drop surface. The deviation between the two cases is already visible in figure~\ref{fig:Exp_curvature}(c) and (d). There, the jet dynamics are matched only when $D^*$ takes a higher value for the flat free surface measurement. The simulations indicate that this discrepancy starts at around $D^*=1.2$ (shown in figure~\ref{fig:Sim_curvature}(c)) and keeps growing for higher values. We can portrait these changes as being enclosed between two extreme scenarios: (1) the bubble is produced right on the liquid surface, generating a bullet jet, which is not affected by the characteristics of the boundaries and thus is independent on $R_d$. As the cavity is placed closer to the drop centre, the surface curvature becomes increasingly relevant to the jet dynamics. This is consistent with our definition of $\Upsilonbar$, since $D^*$ and $\Upsilonbar$ take similar values for lower values of $d$, and grow apart as the cavity is placed deeper in the drop. (2) When the bubble is almost at the drop centre (i.e. $\Upsilonbar\,\xrightarrow\,\infty$) there is no jetting for the curved case. However, in the flat surface case the jetting still occurs for comparable values of $D^*$ (e.g., $D^*\sim 2$), demonstrating how the curvature weighting factor $\chi$ becomes increasingly relevant. 

It is important to stress that the discrepancies found in the jetting dynamics of a bubble in the ``semi-infinite'' liquid pool when compared with the droplet case are mainly provoked by the surface curvature, and not by the dissimilar boundary conditions of the liquid below the gas cavity. This particular point is corroborated in the appendix by means of complementary measurements and numerical simulations of jetting bubbles in the proximity of a hemispherical top of a cylindrical water column.

\subsection{Radial bubble oscillations}
\label{Bubble_Dynamics}

In the previous section, we studied features found in the dynamics of an axisymmetric jetting bubble. Let us now take a closer look at the only case with spherical symmetry, i.e. where the laser cavity is placed in the centre of the drop ($\Upsilonbar\,\xrightarrow\,\infty$). In this scenario, the bubble undergoes several spherical oscillations with a decaying amplitude, as commonly observed in laser bubbles created in unbounded liquids~\cite{Liang2022}. Figure~\ref{fig:Rt_center}(a) presents a comparison between an experiment and simulated data computed using the VoF solver, finding an excellent agreement. Like for the previously simulated results, here we applied the correction script that accounts for the distortion induced by the drop curvature.

\begin{figure}
\centering\includegraphics[width=1.0\textwidth]{Figure_13.jpg}
\caption{\label{fig:Rt_center} Direct comparison between the experiment and a numerical simulation for a case where the laser bubble is placed at the centre of the drop. (a) The median diameter of the drop is 1.42\,mm. The simulated images showing the velocity field have been remapped to account for the distortion provoked by the curvature of the drop. The numbers indicate time in \us. (b) Radial dynamics of the experimental and simulated bubbles. The experimental radius was obtained by fitting a circle on the bubble. The radius in the simulations was estimated using the gas volume (i.e. the spherical equivalent radius). The results were compared with the unbounded case to find that the bubbles inside the drop have a shorter expansion/collapse cycle. (c) Bubble dynamics is obtained with a modified Rayleigh-Plesset model for different drop sizes. The radii of the larger drops remain almost unaltered during the bubble oscillation.}
\end{figure}

Figure~\ref{fig:Rt_center}(b) depicts the temporal evolution of the bubble radius $R(t)$ for the examples in panel (a). In addition, it presents $R(t)$ calculated for a case of a drop of an ideally infinite size, which corresponds to the case of an unbounded liquid domain. Notably, the bubble computed with the VoF model reaches a slightly larger maximum radius as the liquid layer thickness is increased to infinity (i.e. an unbounded bubble case) and thus also has a larger collapse time. This might be due to the effect produced by the consecutive (and alternating) tension and pressure waves interacting with the bubble during its expansion (see figure~\ref{fig:Rt_center}(a)). 

To shed some light on this matter, we implemented a spherical bubble model based on a modified Rayleigh-Plesset model\cite{Obreschkow2006,Zeng2018} that accounts for the finite droplet size, viscosity of the liquid and interfacial tension. The results, presented in figure~\ref{fig:Rt_center}(c), show that the bubble grows up to almost the same size independently of the drop size. For a constant initial condition ($p_g(t=0)=0.32\,$GPa and $R_b(t=0)=30\,\mu$m the work is almost completely done against the surrounding pressure ($p_\infty=1\,$bar) while surface energy and viscous dissipation is negligible. Yet, for smaller droplet volumes, the inertia is reduced and therefore the expansion time to maximum bubble radius and the almost symmetrical collapse reduce, too. 

For the particular initial conditions, both models agree on the elongation of the oscillation cycle, they also reveal a disparity. In the simple Rayleigh-Plesset model the maximum expansion of the bubble is nearly independent of the droplet size, while in the VoF simulations it does. The VoF model accounts for the reflected wave, thus the discrepancy suggests that upon the acoustic soft reflection of the shock wave momentum is imparted on the droplet interface. The importance of reflected waves on cavitation nucleation in confined liquid samples was recently also found for an acoustic hard reflection where the bubble expansion was lowered~\cite{Bao2023}.

\subsection{Drop surface instabilities}
\label{Surface_Intabilities}

In the previous sections, the formation of radial liquid jets growing from the drop surface in the shape of ``spikes'' was mentioned. As explained above, this phenomenon stems from an initial perturbation of the liquid interface and the posterior ejection of liquid produced by the Rayleigh-Taylor instability. This kind of instability occurs when the rapid expansion or the collapse of the bubble wall accelerates a thin liquid layer trapped between the cavity and the atmospheric gas, producing a pattern of ripples on the drop surface that grow further in the consecutive bubble oscillations. A clear example of the events leading to the onset of this kind of instability in this particular experiment is shown in figure~\ref{fig:Instabilities}(a). There, the acoustic emissions from the laser dielectric breakdown nucleate a cloud of bubbles within the drop. As the cavity expands, all these smaller bubbles are incorporated (by coalescence) into the main bubble, producing a series of dimples on the bubble surface, visible at $t=50\,\us$~of figure~\ref{fig:Instabilities}(a). These dimples may contribute to the later destabilisation of the drop surface, which is highly dependent on the ratio $R^*_{max}/R_d$. Additionally, $R^*_{max}/R_d$ determines the liquid layer thickness and its acceleration by the bubble/drop dynamics. The ripples in the drop surface become noticeable just after the first bubble collapse (i.e. $t=130\,\us$) and grow significantly during the bubble re-expansion, as shown at $t=190\,\us$. However, the most dramatic events take place after the second bubble collapse (i.e. at $t=230\,\us$). There, the ripples grow into liquid ``spikes'' which lead to the detachment of small droplets due to the action of the Rayleigh-Plateau instability, as shown in figure~\ref{fig:Instabilities}(b). At the same time, the second bubble collapse releases a strong shock wave in the radial direction. This shock wave interacts with the array of meniscus-shaped pits on the liquid surface to produce fast radial jets (see the frame at $t=420\,\us$). The later sequence is clearly captured in the frames of figure~\ref{fig:Instabilities}(c). It is important to note that this complex phenomenon not only depends on the shock wave strength but also requires certain conditions to be met~\cite{Tagawa2012,peters2013}, like a minimum depth and curvature of the pits, which may explain the absence of ``spikes'' during the first bubble collapse.

\begin{figure}[ht]
\centering\includegraphics[width=1\textwidth]{Figure_14.jpg}
\caption{\label{fig:Instabilities} Drop surface destabilisation mechanisms. The mean drop radius is 1.42\,mm and the numbers represent time in \us. (a) As the main bubble expands, the secondary, acoustic cavitation bubbles produce small dimples on the gas cavity surface (e.g. at 50\,\us). Those may promote the formation of a series of ripples during the bubble collapse (at 130\,\us). As the bubble re-expands, the Rayleigh-Taylor instability causes the growth of liquid ``spikes'' that later lead to the detachment of small drops due to the Rayleigh-Plateau instability, as indicated with a green arrow in panel (b). There, the frame width is 570\,\um. At the same time, the second collapse of the bubble enhances the surface irregularities and pits that appear in the areas between the ripples. The shock wave emitted during the second collapse gives origin to fast liquid jets ejected from the centre of the pits, as highlighted with a blue arrow in panel (c). The frame width in this sequence is 490\,\um. The full video is available in the online supplementary movie 9.}
\end{figure}

To further analyse the onset of these instabilities, we varied the energy of the laser pulse, hence producing bubbles with various sizes and thus with distinct ratios $R^*_{max}/R_d$. The results are presented in figure~\ref{fig:Instabilities_var_laser}. Even when the  extreme image distortion produced near the drop interface prevent us to obtain an accurate value of the bubble radius, these measurements make evident that the amplitude of the ripples increases with increasing $R^*_{max}$ and with each consecutive bubble oscillation. 

\begin{figure}[H]
\centering\includegraphics[width=0.9\textwidth]{Figure_15.jpg}
\caption{\label{fig:Instabilities_var_laser} Onset of the drop surface instabilities for bubbles produced with different laser pulse energies. The mean drop radius is 1.42\,mm and the numbers represent time in \us. (a) Here, the energy of the laser pulse is $L=1.9$\,mJ. (b) $L=3.1$\,mJ. (c) $L=3.9$\,mJ. For this energy, the RTI affects the drop surface enough to produce liquid ejection after the second bubble collapse. (d) $L=4.6$\,mJ. (e) $L=5.2$\,mJ. (f) $L=6.4$\,mJ. Note that panels (d)-(f) are shown in wider frames than the panels (a)-(c) to show the larger ``spikes''. A full video of panel (f) is available in the online supplementary movie 10.}
\end{figure}

In figure~\ref{fig:Instabilities_var_laser}(a) the expansion of the bubble is not sufficient to visibly disturb the drop's spherical surface. In the case shown in figure~\ref{fig:Instabilities_var_laser}(b) the bubble's first collapse does not break up the drop surface, however, a mild wave pattern is observed on the surface after the bubble re-expansion (at $t=420\,\us$). In spite of the presence of these low amplitude ripples, no radial jets are ejected from the drop upon the second bubble collapse. When the ratio $R^*_{max}/R_d$ is further increased, as shown in figure~\ref{fig:Instabilities_var_laser}(c), we find very similar dynamics of the bubble/drop system, but now the valleys between the ripples (and the acoustic pressure wave) are deep enough to trigger the radial jetting. This confirms the existence of threshold conditions for the ``spikes'' to be formed. In the remaining cases presented in panels (d) to (f) of figure~\ref{fig:Instabilities_var_laser}, the general dynamics of the bubble/drop system are very similar to the previous cases, although as the laser pulse energy is increased the instabilities become perceivable at an earlier time. For example, in figure~\ref{fig:Instabilities_var_laser}(f) liquid ``spikes'' are already formed after the first bubble collapse~\cite{Zeng2018}.

\section{Conclusion}

In this manuscript, we presented some of the complex fluid dynamics occurring once a vapour bubble expands within a water droplet. Specifically, we analysed the appearance of acoustic secondary cavitation, and the formation of liquid jets in the proximity of highly curved free surfaces, and finally, we provided detailed experimental images of the onset and the development of shape instabilities on the surface of the drop. The conclusions are based on the experimental results and the computational fluid dynamics simulations. 

The first part of the research highlights that acoustic waves emitted from the micro-explosion nucleate complex secondary cavitation clouds. Further, the study corroborates the existing relation between the evolution of the negative pressure profile and the shape of the bubble clusters inside the drop. A cavitation threshold pressure of around $-4.5$\,MPa was estimated by performing a direct comparison between the experiments and the simulations. The implementation of this experimental technique to other liquids, particularly in cases where large samples are not available, might contribute to achieving a deeper understanding of the nucleation of bubbles by sound waves. The present experimental setup may be modified to create a bubble within a superheated droplet to reveal in a well-defined system the coupling of fluid dynamics with thermodynamics, and also study how the liquid temperature affects the later fragmentation dynamics.

The secondary bubbles cluster and several types of jets, both caused by the generation of laser bubbles at different positions inside the droplet, were classified using a stand-off parameter $\Upsilonbar$. The use of a single quantity to characterise the system simplifies the direct comparison between cases. The optical lens effect linked to the spherical shape of the drops allowed us to obtain images of the bubble jet's interior with a remarkable level of detail.

The numerical simulations were crucial to explain the complex flow fields generating these jets, as well as to explain the shape acquired by the gas cavities during their second collapse phase, including many interesting features like the annular bubble necking and the detachment of multiple vapour rings.

Interestingly, the jetting dynamics of the bubble inside a droplet as compared to a bubble in a semi-infinite pool differs rather little. In this study, we have shown that for the droplet case the non-dimensional distance $D^*$ is the most determining quantity, while the curvature of the liquid does not have a dominant role in the evolution of the jetting cavities. This conclusion is based in the analysis of numerical simulations where only the parameter $R_d$ was modified, and also by comparing the current results with the previously reported for a flat surface. Of course, that is true as long as the liquid layer around the bubble is not thin enough to promote the onset of the RTI, as it happens in cases with a low $R_d/R^*_{max}$ ratio or a low value of $\chi$. 

The spherical bubble oscillations observed in the experiments where the laser was focused on the geometrical centre of the droplet were analysed using two different numerical models. Both models were in excellent agreement with the measured temporal bubble radius evolution. More importantly, both models predict a reduction in the expansion/collapse time when the drop size is decreased.

The radial oscillations of a central bubble were also used to study the onset of shape instabilities at the gas-liquid interfaces, given by the Rayleigh-Taylor and Rayleigh-Plateau instabilities. The destabilisation mechanism of each instability and its effect on the droplet surface was illustrated by detailed high-speed images. Here, we have demonstrated how the radial acceleration imposed by the bubble oscillation triggers the RTI, which in turn induces a pattern of superficial ripples on the drop. Those acquire a concave shape during the bubble collapse and give rise to liquid filaments due to the transfer of the momentum from the bubble shock wave emissions to the curved pits formed on the gas-liquid interface. The ejected filaments later break up by the action of the RPI causing the detachment of smaller droplets and thus the atomisation of the drop. 

The phase change from liquid to vapour within droplets is observed in a wide variety of applications, such as in flash boiling atomisation~\cite{Loureiro2021}, in spray-flame synthesis~\cite{Jungst2022}, spray cooling~\cite{Tran2012}, EUV light generation~\cite{Versolato2019}, and laser-induced breakdown spectroscopy of liquids~\cite{Lazic2014} to name a few. They all have in common that through a complex non-spherical symmetric process, a liquid is fragmented through a micro-explosion within. While Rayleigh-Taylor instabilities determine the growth of ripples on the surface of the droplet, the non-spherical bubble dynamics that leads to jetting out of the droplet affects the resulting size distribution of liquid particles, too. The high degree of control achieved in the current experiments opens up the possibility of studying RTI of more complex interfaces, e.g. the effect of particles covering the surface, surfactants, or complex fluids. 

\section*{Funding}
 J.M.R and K.A.R. acknowledge support by the Alexander von Humboldt Foundation (Germany) through the Georg Forster and Humboldt Research Fellowships. This project has received funding from the European Union's Horizon 2020 research and innovation programme under the Marie Skłodowska-Curie grant agreement No.\,101064097, as well as the Deutsche Forschungsgemeinschaft (DFG, German Research Foundation) under contract OH75/4-1.

\section*{Declaration of interests} The authors report no conflict of interest.

\section*{Author contributions} 
\textbf{Juan Manuel Rossell\'{o}}: Conceptualisation, Data curation, Formal Analysis, Funding acquisition, Investigation, Methodology, Visualisation and Writing – original draft. \textbf{Hendrik Reese}: Conceptualisation, Data curation, Formal Analysis, Software, Visualisation and Writing – review \& editing. \textbf{K. Ashoke Raman}: Conceptualisation, Funding acquisition and Investigation. \textbf{Claus-Dieter Ohl}: Conceptualisation, Formal Analysis, Funding acquisition, Resources and Writing – review \& editing. 

\section*{Appendix: Bubble jetting in a liquid pool with a curved free surface}
\label{Appendix}

In section~\ref{Bub_Jet}, the role of the curvature of the free surface was analysed by comparing the bubble jetting observed near a flat surface with the jetting dynamics of bubbles within the falling droplet. There, both the experimental and numerical results indicate that the effect of the curvature is almost negligible for low values of $D^*$, but the specific shape acquired by the cavity during and after the jetting are no longer similar as $D^*$ takes values larger than $1.2$. However, the curvature of the surface is not the only difference between these two cases, since in one case the liquid is confined (i.e. the droplet) and in the other the bubble is produced on top of an ideally "semi-infinite" liquid column (which has a length of 5\,cm in the experiments and was numerically infinite in the simulations). An intermediate step between those two experimental scenarios is given by the configuration described in figure~\ref{fig:Appendix}(a). Here, the bubbles are also produced close to the free surface of a liquid pool, but in this case the top of the liquid column presents a curved surface with the shape of a dome. Panels (b), (c) and (d) of figure~\ref{fig:Appendix} show three examples of jetting bubbles generated at different depths $d$. The bubbles were located away from the symmetry axis to make evident that the jets always point in the direction normal to the surface. As discussed in section~\ref{Bub_Jet}, we observed similar jetting dynamics for both curved surfaces at comparable values of the stand-off parameter. The example shown in panel (b) of figure~\ref{fig:Appendix} corresponds to the case (c) of figure~\ref{fig:Sim_curvature}, while the jet dynamics of the case (c) of figure~\ref{fig:Appendix} matches the one of figure~\ref{fig:Sim_curvature}(a).

\begin{figure}[ht]
\centering\includegraphics[width=1\textwidth]{Figure_A1.jpg}
\caption{\label{fig:Appendix} Bubble jetting on a curved surface. (a) A cylindrical tube with an external diameter of 3.6\,mm, an internal diameter of 2.7\,mm and a length of 4\,cm was overfilled with DI water to produce the curved top. Infra-red laser pulses were focused from the top at different depths and slightly away from the cylinder axis. (b) $\Upsilonbar=1.9$ ($D^*=1.3$). (c) $\Upsilonbar=0.7$ ($D^*=0.6$). (d) $\Upsilonbar=0.4$ ($D^*=0.3$). The time between frames is 10\,\us~for (b) and (c) and 20\,\us~in case (d).}
\end{figure}

Figure~\ref{fig:Appendix_2} compares the jetting of bubbles in the drop case with the jetting of bubbles in a configuration as the one shown in figure~\ref{fig:Appendix}(a). The results present almost identical bubble dynamics even in the case with $D^*=1.23$, meaning that the differences observed between the case with the flat surface and the drop case are indeed caused by the effect of the surface curvature and not by the difference in the boundary conditions below the bubble or in the liquid volume. As previously discussed for the flat surface case, the similarity found in the cases displayed in figure~\ref{fig:Appendix_2} will be eventually lost as the laser bubble is produced closer to the drop center.

\begin{figure}[ht]
\centering\includegraphics[width=1\textwidth]{Figure_A2.jpg}
\caption{\label{fig:Appendix_2} Comparison of the bubble jetting on the curved top of a long liquid column (left half) and in a drop of equal radius (right half). The numbers represent the time normalised with the time of collapse of the cavities from each case. (a) $D^*=1.02$. (b) $D^*=1.23$. The dynamics of the bubbles are almost identical.}
\end{figure}

\bibliography{References}

\end{document}