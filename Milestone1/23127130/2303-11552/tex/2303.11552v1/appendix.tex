
\noindent
\textbf{A.} \textbf{Calculation of Variance}

\begin{align}
	\label{variance_app}
	D(\mathbb{I}) & = E((\mathbb{I})^2) - E(\mathbb{I})^2 \notag\\
	& = E((\mathbb{I})^2) \notag\\
	& = \underbrace{(-1 + \frac{d}{2})^2 + (-1 + \frac{3d}{2}^2) + \ldots + (1 - \frac{d}{2})^2}_{2/d \enspace items} \notag\\
	& = 2 * \underbrace{((\frac{d}{2})^2 + \ldots + (1 - \frac{d}{2})^2)}_{1/d \enspace items} \\
	& \overset{\text{let n = 1/d}}{=} 2 * \underbrace{((\frac{1}{2n})^2 + \ldots + (\frac{2n-1}{2n})^2)}_{n \enspace items} \notag\\
	& = \frac{1^2 + 3^2 + \ldots + (2n-1)^2}{2n^2} * n \notag\\
%	& = \frac{\frac{n(2n-1)(2n+1)}{3}}{2n} \notag\\
	& = \frac{4n^2-1}{6} \notag
\end{align}

\noindent
\textbf{B.} \textbf{Effects of Abstraction Granularity}

\begin{figure}[h!]
% \vspace{-3mm}
    \centering
    \begin{subfigure}{0.49\linewidth}
        \centering
        \includegraphics[width=0.95\linewidth]{summaryMnist.jpg}
        \caption{MNIST}
    \end{subfigure}
    \begin{subfigure}{0.49\linewidth}
        \centering
        \includegraphics[width=0.95\linewidth]{summaryOther.jpg}
        \caption{CIFAR-10 and ImageNet}
    \end{subfigure}
    % \vspace{-2mm}
    \caption{Measuring verified errors with varying abstraction granularity $d$ ($x$-axis: the size of  granularity; $y$-axis: the verified error). 
    % Green curves are for displaying results on MNIST, and blue for CIFAR-10, red for ImageNet.
    }
    \label{fig:exp3}
\end{figure}

We had investigated the effects of abstraction granularity and obtained some preliminary results. As shown in Figure~\ref{fig:exp3}, 
when the abstraction granularity is relatively small (below the robustness bound), the verified errors are less affected. 
After it exceeds the bound, the verified errors become higher as the abstraction granularity increases. Hence, it is fair to say that  abstraction granularity is a key hyper-parameter for training robust models with low verified errors. We believe that this work would inspire more studies on investigating new mechanisms for finding optimal abstraction granularity.

