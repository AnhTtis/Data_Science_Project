\section{The Abstraction-Based Training  Approach}\label{sec:method}

In this section, we present our abstraction-based method for training neural networks. We define the training problem in Section \ref{3.1}. Section \ref{3.2} and Section \ref{3.3} give the abstraction procedure and the training procedure. 
Throughout these two sections, we also illustrate how varying
network inputs can affect the loss function,  which in turn contributes to the robustness of trained models.

\subsection{Formulating the Training Problem}
\label{3.1}

We solve the following training problem:

\begin{center}
	\begin{tcolorbox}[colback=gray!10,%gray background
		colframe=blue!40!white,% black frame colour
		% width=8cm,% Use 8cm total width,
		% arc=1mm, auto outer arc,
		boxrule=0.5pt,
		]
		\emph{Given a training set and a testing set of images and a perturbation distance $\epsilon$, our goal is to train an image classifier $f$ that is provably robust on $\mathbb{B}(x,\epsilon)$ for each image $x$ in the set while $f$ has a low verified error on the testing set. }
	\end{tcolorbox}
\end{center}

As the target image classifier $f$ must be guaranteed robust on $\mathbb{B}(x,\epsilon)$, $f$ returns the same classification results for all the perturbed images in $\mathbb{B}(x,\epsilon)$. Our idea is to (i) map $\mathbb{B}(x,\epsilon)$ to an interval vector $(I_1,I_2,\ldots I_n)$ by an abstraction function $\phi$ such that $\phi(\mathbb{B}(x,\epsilon))=(I_1,I_2,\ldots I_n)$, and (ii) train an image classifier $f'$ which takes as input the interval vector  and returns a classification result as $x$ is labeled, i.e., $\arg\max_{y\in Y}f'(I_1,I_2,\ldots,I_n)=y_{true}$ with $y_{true}$ the ground truth label of $x$.  The target image classifier $f$ is a composition of $f'$ and $\phi$. Apparently, $f$ is provably robust on  $\mathbb{B}(x,\epsilon)$ since all the images in $\mathbb{B}(x,\epsilon)$ are mapped to $(I_1,I_2,\ldots,I_n)$ which is classified to $y_{true}$ by $f'$. 


To feed the training intervals to the neural network, the number of neurons in the input layer is doubled. As shown in Figure \ref{training}, the upper bound and lower bound of each training interval are input to these neurons. 
Namely, the neurons in the input layer do not correspond to a pixel but the upper and lower bounds of each training interval. Any perturbation interval mapped to the same training interval will be fed into the neural network with the same upper and lower bounds. Other training parameters and settings are the same as the traditional training process.

\subsection{Abstraction of Perturbed Images}
\label{3.2}
\subsubsection{From Perturbed Interval to Training Interval}

We first introduce the abstraction function $\phi:\mathcal{I}^n\rightarrow \mathbb{I}^n$, where 
$\mathcal{I}^n$ is the set of all interval vectors,  
$\mathbb{I}^n$ is a finite set of interval vectors on which neural networks are trained, and $n$ denotes the size of the interval vectors. The 
norm ball $\mathbb{B}(x,\epsilon)$ of
$x$ under $\epsilon$ is essentially an interval vector in $\mathcal{I}^n$. We call the elements in $\mathcal{I}^n$ \textit{perturbation interval vectors}. 
 
Let $[-1, 1]$ be the range of the complete input space. We divide $[-1, 1]$ evenly into sub-intervals of size $d_i$ and denote the set of all the sub-intervals as $\mathbb{I}_i$. Then, we obtain $\mathbb{I}^n$ as the Cartesian product of all $\mathbb{I}_i$, i.e., $\Pi^{n}_{i=1} \mathbb{I}_i$. 
Because we train image classifiers on $\mathbb{I}^n$, we call the elements in $\mathbb{I}^n$ \textit{training interval vectors}. 
We use $\overline{d}$ to denote the integer vector $(d_1,d_2,\ldots,d_n)$ and call it \emph{abstraction granularity}. In our abstraction process, all values in $\overline{d}$ are the same, so we use a constant $d$ to represent it. 

Because perturbation interval vectors are infinite, it is impossible to enumerate all perturbation interval vectors for training. We abstract and map them onto a finite number of training interval vectors in $\mathbb{I}^n$. The purpose of the abstraction is to ease the follow-up robustness verification by restricting the infinite perturbation space to finite training space. In addition, the abstraction function is an element-wise operation. In such cases, we introduce our mapping function with an interval as an example. 

\begin{figure}
	\centering
	\includegraphics[width=0.48\textwidth]{train.pdf}
	\caption{Training on numerical intervals. 
	}
	\label{training}
 \vspace{-1mm}
\end{figure}


In order to ensure that the perturbation interval can only be mapped to a unique training interval, we make the following constraints on the mapping process: We first restrict the abstraction granularity $d$ to be greater than or equal to twice the perturbation size $\epsilon$, for the purpose of guaranteeing that a perturbation interval has an intersection with at most two training intervals. We divide the mapping into three cases: (1) If a training interval contains the perturbation interval, the perturbation interval will be mapped to the training interval; (2) If the perturbation interval has an intersection with two training intervals unequally, the perturbation interval will be mapped to the training interval with the larger coverage area; (3) If the perturbation interval has an intersection with two training intervals equally, the perturbation interval will be mapped to the training interval with a larger value. In this way, we map the perturbation interval on the unique training interval, and then we can train the neural network on these training intervals. 





\setlength{\textfloatsep}{10pt}
\begin{algorithm}[t]
	\renewcommand{\thealgocf}{1}
	\SetKwData{Left}{left}\SetKwData{This}{this}\SetKwData{Up}{up}
	\SetKwFunction{Union}{Union}\SetKwFunction{FindCompress}{FindCompress}
	\SetKwInOut{Input}{Input}\SetKwInOut{Output}{Output}
	\caption{ Abstraction of Perturbations $\phi$.}
	\label{abstraction_alg}
	\Input{~$\mathcal{I}$: perturbation interval vector;\\
		$d$: abstraction granularity}
	\Output{~$\mathbb{I}$: training interval vector}
	Initialize $I_{init}', \mathbb{I}$\tcp*{ $I_{init}'$ is [-1,1], and $\mathbb{I}$ is an empty vector}
	$I'$ $\leftarrow$ Divide($I_{init}'$, $d$)\tcp*{Get sub-intervals with size d}
		\For{i in len($\mathcal{I}$)}{ 
			\For{j in len($I'$)}{
				\If{min($\mathcal{I}_{i}$) $>$ max($I_j'$)}{
					Continue \tcp*{No intersection}
				}
				\If{$\mathcal{I}_{i} \in I_j'$}{
					$\mathbb{I}_{i} = I_j'$\tcp*{$I_j'$ contains $\mathcal{I}_{i}$}
				}
				\Else{
					$len_r$ = max($\mathcal{I}_{i}$) - min($I_{j+1}'$); \\
					$len_l$ = max($I_j'$) - min($\mathcal{I}_{i}$);\\
					\If{$len_r\geq len_l$}{
						$\mathbb{I}_{i} = I_{j+1}'$\tcp*{Map to numerically larger}
					}
					\Else{
						$\mathbb{I}_{i} = I_j'$\tcp*{Map to numerically smaller}
					}
				}
			}
		}
	Return ${\mathbb{I}}$;
\end{algorithm}



Algorithm \ref{abstraction_alg} shows our abstraction function. Firstly, we initialize the complete input interval and an training interval vector (Line 1). Then,
we obtain the sub-intervals which each size is $d$ (Line 2).
If a training interval contains the perturbation interval, the perturbation interval will be mapped to the training interval (Lines 7-8). The perturbation interval at most intersects with two training intervals at the same time as $d \geq 2*\epsilon$. Let $len_r$ be the size of the intersection of the perturbation interval and the numerically larger training interval (Line 10). Let $len_l$ be the size of the intersection of the perturbation interval and the numerically smaller training interval (Line 11). If $len_r\geq len_l$, the perturbation interval will be mapped to the numerically larger training interval (Lines 12-13); otherwise, it will be mapped to the numerically smaller training interval (Lines 14-15).

\vspace{-2mm}
\subsubsection{Effect of Abstraction on Input}
\vspace{-2mm}
We now explain that the abstraction process results in a smaller variance of training intervals. 

In our training method, the value of each pixel is normalized to $[-1, 1]$. Considering the arbitrariness of the input value distribution, we assume that the input values are evenly distributed in $[-1, 1]$. We calculate the variance of input, and in such cases, the value we get is the maximum likelihood estimation of the actual value.

We calculate the variance of the input values with $d$ representing abstraction granularity and $\mathbb{I}$ representing the training intervals of an image. Note that in conventional methods, it is equivalent to $d = 0$, while in our method, $d$ is a positive number. In this way, for an input image, the variance of it after the abstraction process is:
\begin{align}
	\label{variance}
	D(\mathbb{I}) & = E((\mathbb{I})^2) - E(\mathbb{I})^2 \notag\\
	% & = E((\mathbb{I})^2) \notag\\
	& = \underbrace{(-1 + \frac{d}{2})^2 + (-1 + \frac{3d}{2}^2) + \ldots + (1 - \frac{d}{2})^2}_{2/d \enspace items} \\
	% & = 2 * \underbrace{((\frac{d}{2})^2 + \ldots + (1 - \frac{d}{2})^2)}_{1/d \enspace items} \\
	% & \overset{\text{let n = 1/d}}{=} 2 * \underbrace{((\frac{1}{2n})^2 + \ldots + (\frac{2n-1}{2n})^2)}_{n \enspace items} \notag\\
	% & = \frac{1^2 + 3^2 + \ldots + (2n-1)^2}{2n^2} * n \notag\\
%	& = \frac{\frac{n(2n-1)(2n+1)}{3}}{2n} \notag\\
	& = \frac{4n^2-1}{6}. \notag
\end{align}
\noindent In Equation \ref{variance}, the variance of training interval is computed by the upper and lower bounds. For an abstraction-based trained neural network, a large abstraction granularity $d$ implies a small $n$, and consequently the variance of the training intervals becomes small. Apparently, intervals have a smaller variance than  concrete pixel values.
% We say that the training intervals $\mathbb{I}$ have \textit{small variance}.


\begin{algorithm}[t]
    % \setcounter{algorithm}{2}
    \renewcommand{\thealgocf}{2}
	\SetKwData{Left}{left}\SetKwData{This}{this}\SetKwData{Up}{up}
	\SetKwFunction{Union}{Union}\SetKwFunction{FindCompress}{FindCompress}
	\SetKwInOut{Input}{Input}\SetKwInOut{Output}{Output}
	\label{train_alg}
	\caption{\mbox{ Abstraction-based Training: \textsc{AbsTrain}}}
	\label{train_alg}
	\Input{~$\mathbf{X}$: training data; \\ ~$\mathbf{Y}$: ground-truth labels of training data; \\ ~$\epsilon$: perturbation radius; \\ ~$d$: abstraction granularity}
	\Output{~$f'$: a neural network, $\ell$: training loss}
	Initialize $f'$, $\ell$; \\ 
	\For{($X$, $Y$) in ($\bf X, \bf Y$)}{
    \For{($x$, $y$) in ($X$, $Y$)}{
		$\mathbb{I}$ $\leftarrow \phi(\mathbb{B}(x$, $\epsilon$), $d$) \tcp*{Get training intervals.} 
		$\ell \leftarrow \ell + $ $\mathcal{L}$($f'(\mathbb{I})$, $\mathbf{y}$) \tcp*{Accumulate the loss.}
	  }
        $f'\leftarrow$ Update($f'$, $\ell$) \tcp*{Update the parameters in $f'$.}
	}
	Return $f'$, $\ell$;
\end{algorithm}

\subsection{Training on Intervals}
\label{3.3}
\subsubsection{The Training Method}


When we get the training intervals, hyperparameter settings, such as the number of layers, the number of neurons in hidden layers, the loss function during training, etc., are all the same as the traditional training methods except for the number of neurons in the input layer.

Algorithm \ref{train_alg} shows the pseudo-code of the algorithm for training a neural network. The training dataset $\mathbf{X}$, the ground-truth labels $\mathbf{Y}$ corresponding to the dataset, the perturbation radius $\epsilon$, and the abstraction granularity $d$ are used as inputs. Firstly, a neural network is initialized. That is, the adjustable parameters of the neural network are initialized randomly (Line 1). 
For each image of training dataset $\mathbf{X}$,
the perturbation intervals are mapped to training intervals (Line 4). According to the current adjustable parameters, 
% the classification labels of the neural network for the training intervals are obtained, and 
the cross-entropy loss is calculated (Line 5). Finally, the backward propagation is performed according to the cross-entropy loss, and the values of the adjustable parameters are updated (Line 6). A trained neural network is returned when the loss no longer decreases.





% \section{Proof of the main theorem}

Let $p$ be an odd prime and let $V$ be an $n$-dimensional vector space over $\mathbb{F}_p$ with basis $v_1,v_2,\dots,v_n$. The groups $G$ in (a),(b) are precisely those in (\ref{Gpi}), with $n=3$, associated to the linear maps
\begin{enumerate}[label = (\alph*)]
\item $\pi : V\rightarrow \Lambda^2V;\,$ $v_2^\pi = v_3^\pi  =1 $ and $v_1^\pi = (v_1\wedge v_2)$,
\item $\pi : V\rightarrow \Lambda^2V;\,$ $v_2^\pi = v_3^\pi =1 $ and $v_1^\pi = (v_2\wedge v_3)$.
\end{enumerate}
Similarly, the groups $G$ in (c),(d),(e) are precisely those in  (\ref{Gpi}), with $n=4$, associated to the linear maps
\begin{enumerate}[label = (\alph*)]\setcounter{enumi}{+2}
\item $\pi : V\rightarrow \Lambda^2V;\,$ $v_2^\pi = v_3^\pi = v_4^\pi =1 $ and $v_1^\pi = (v_1\wedge v_2)$,
\item $\pi : V\rightarrow \Lambda^2V;\,$ $v_2^\pi = v_3^\pi = v_4^\pi =1 $ and $v_1^\pi = (v_3\wedge v_4)$,
\item $\pi : V\rightarrow \Lambda^2V;\,$ $v_2^\pi = v_3^\pi = v_4^\pi =1 $ and $v_1^\pi = (v_1\wedge v_2)(v_3\wedge v_4)$.
\end{enumerate}
By Propositions \ref{rank one prop'} and \ref{rank one prop}, for $n=3,4$ and up to a change of basis, these are the only linear maps $\pi$ of rank one.

In this section, let us take $n=3,4$ and the symbol $\pi$ denotes one of the five linear maps above. As explained in Section \ref{group section}, we may identify
\[ G/G' = V\mbox{ and }G'=\Lambda^2V.\]
Moreover, we have a natural isomorphism
\[ \Aut^c(G) \simeq \Aut^c(\pi).\]
With these identifications, we may rephrase (\ref{Delta2}) as
\begin{equation}\label{Delta3}
\Delta(u^\alpha,v^\alpha) = \Delta(u,v)^{\hat{\alpha}}
\end{equation}
for all $u,v\in V$ and $\alpha\in\Aut^c(\pi)$. The $S$ and $S'$ in Section \ref{bilinear form sec} become
\begin{align*}
S &=  \{\mbox{symmetric bilinear $\Delta :V\times V\rightarrow \Lambda^2V$ satisfying (\ref{Delta3})}\}\\
S' &= \{\mbox{anti-symmetric bilinear  $\Delta :V\times V\rightarrow \Lambda^2V$ satisfying (\ref{Delta3})}\}
\end{align*}
in the current setting. The group $\Aut^c(\pi)$ was computed in Section \ref{group section}. Let $P$ and $Q$ denote the subgroups defined there. Then, we have
\[\Aut^c(\pi) = P\rtimes Q.\]
We shall also make the following assumption.

\begin{assume}Assume that $p\geq 5$ in the cases (a),(c),(e).
\end{assume}

We first show that the groups $G$ in question satisfy Assumption \ref{assumption} so that the discussion thereafter applies.  
\begin{lemma}\label{gamma lemma}
Let $\gamma : V\rightarrow\Aut^c(\pi)$ be an $\Aut^c(\pi)$-equivariant homomorphism and let $1\leq i,j\leq n$. Suppose that
\begin{enumerate}[label = $(\arabic*)$]
\item $\gamma(v_i)=1$,
\item $v_i^\alpha = v_iv_j$ for some $\alpha\in \Aut^c(\pi)$.
\end{enumerate}
Then $\gamma(v_j)=1$ also holds.
\end{lemma}

\begin{proof}Indeed, we have
\[ 1 = \gamma(v_i)^\alpha = \gamma(v_i^\alpha) = \gamma(v_i)\gamma(v_j) = \gamma(v_j)\]
by the hypotheses.
\end{proof}

\begin{prop}\label{gamma prop}There is no non-trivial $\Aut^c(\pi)$-equivariant homomorphism from $V$ to $\Aut^c(\pi)$.
\end{prop}

\begin{proof}Let $\gamma : V\rightarrow\Aut^c(\pi)$ be an $\Aut^c(\pi)$-equivariant homomorphism and observe that $\gamma(V)$ must be a normal $p$-subgroup of $\Aut^c(\pi)$. But
 \[ Q \simeq \begin{cases}
\mathbb{F}_p^\times\times \mathbb{F}_p^\times &\mbox{in case (a)}\\
\GL_2(\mathbb{F}_p)&\mbox{in cases (b) and (e)}\\
\mathbb{F}_p^\times \times \GL_2(\mathbb{F}_p)&\mbox{in cases (c) and (d)}
\end{cases}\]
has no non-trivial normal $p$-subgroup. Since $\Aut^c(\pi) = P\rtimes Q$, we see that $\gamma(V)$ must lie inside $P$. We now deal with each case separately.
\begin{enumerate}[label=(\alph*), wide=0pt]
\item It is clear from Proposition \ref{auto1'} that
\[ v_1^{\alpha_{12}} = v_1v_2\]
for some $\alpha_{12} \in P$, and so it is enough to show that $\gamma(v_1)=\gamma(v_3)=1$ by Lemma \ref{gamma lemma}, it. Let us put
\[ \gamma(v_1) = \begin{bmatrix}
1 & b_1 & 0 \\
0 & 1 & 0 \\
0 & c_1 & 1
\end{bmatrix}\mbox{ and }\gamma(v_3)= \begin{bmatrix}
1 & b_3 & 0 \\
0 & 1 & 0 \\
0 & c_3 & 1
\end{bmatrix} \]
From $\gamma(v_1^\alpha) = \gamma(v_1)^\alpha$ for $\alpha\in Q$ of the shape
\[ \alpha = \begin{bmatrix}s & 0 & 0 \\
0 & 1 & 0\\
0 & 0 & s\end{bmatrix} \mbox{ with } s\in \mathbb{F}_p^\times,\]
we get that $\gamma(v_1^\alpha) = \gamma(v_1)^s$ and
\[  \begin{bmatrix}
1 & sb_1 & 0 \\
0 & 1 & 0 \\
0 & sc_1 & 1
\end{bmatrix}= \begin{bmatrix}
1 & s^{-1}b_1 & 0 \\
0 & 1 & 0 \\
0 & s^{-1}c_1 & 1
\end{bmatrix}.\]
Since $p\geq 5$, there exists $s\in \mathbb{F}_p^\times$ with $s^2\neq 1$, and so $b_1=c_1=0$. We may obtain $b_3 = c_3 =0$ by the exact same calculation.
\item It is clear from Proposition \ref{auto2'} that
\[ v_1^{\alpha_{12}} = v_1v_2\mbox{ and } v_1^{\alpha_{13}} = v_1v_3\]
for some $\alpha_{12},\alpha_{13} \in P$, so it suffices to show that $\gamma(v_1)=1$ by Lemma \ref{gamma lemma}. Let us put
\[ \gamma(v_1) = \begin{bmatrix}1 & b_1 & c_1\\0& 1 & 0 \\ 0 & 0 & 1\end{bmatrix}.\]
From $\gamma(v_1^\alpha) = \gamma(v_1)^\alpha$ for $\alpha\in Q$ of the shape
\[\begin{bmatrix}
1 & 0 & 0\\
0 & s & 0\\
0 & 0 &s^{-1}
\end{bmatrix}\mbox{ with }s\in\mathbb{F}_p^\times,\]
we get that $\gamma(v_1^\alpha) = \gamma(v_1)$ and
\[  \begin{bmatrix}1 & b_1 & c_1\\0& 1 & 0 \\ 0 & 0 & 1\end{bmatrix}
=  \begin{bmatrix}1 & sb_1 & s^{-1}c_1\\0& 1 & 0 \\ 0 & 0 & 1\end{bmatrix}.\]
This yields $b_1=c_1=0$.
\item It is clear from Proposition \ref{auto1} that
\[ v_1^{\alpha_{12}} = v_1v_2\mbox{ and }v_3^{\alpha_{34}} = v_3v_4\]
for some $\alpha_{12}\in P, \alpha_{34}\in Q$, so it suffices to show that $\gamma(v_1)=\gamma(v_3)=1$ by Lemma \ref{gamma lemma}. Let us put
\[ \gamma(v_1) = \begin{bmatrix}
1 & b_1 & 0 & 0\\
0 & 1 & 0 & 0\\
0 & c_1 & 1 & 0\\
0 & d_1 & 0 & 1
\end{bmatrix}\mbox{ and }
 \gamma(v_3) = \begin{bmatrix}
1 & b_3 & 0 & 0\\
0 & 1 & 0 & 0\\
0 & c_3 & 1 & 0\\
0 & d_3 & 0 & 1
\end{bmatrix}.\]
From $\gamma(v_1^\alpha) = \gamma(v_1)^\alpha$ for $\alpha\in  Q$ of the shape
\[ \alpha =\begin{bmatrix}
s & 0 & 0 &0\\
0 & 1 & 0 & 0\\
0 & 0 & s & 0\\
0 & 0 & 0 & s
\end{bmatrix} \mbox{ with } s\in \mathbb{F}_p^\times,\]
we get that $\gamma(v_1^\alpha) = \gamma(v_1)^s$ and
\[ \begin{bmatrix}
1 & sb_1 & 0 & 0\\
0 & 1 & 0 & 0\\
0 & sc_1 &1 & 0\\
0 & sd_1 & 0 & 1
\end{bmatrix} = \begin{bmatrix}
1 & s^{-1}b_1 & 0 & 0\\
0 & 1 & 0 & 0\\
0 & s^{-1}c_1 &1 & 0\\
0 & s^{-1}d_1 & 0 & 1
\end{bmatrix} .\]
Since $p\geq 5$, there exists $s\in \mathbb{F}_p^\times$ with $s^2\neq 1$, and so $b_1=c_1=d_1=0$. We may obtain $b_3=c_3=d_3=0$ by the exact same calculation.
\item It is clear from Proposition \ref{auto2} that
\[ v_1^{\alpha_{12}} = v_1v_2,\,\ v_1^{\alpha_{13}} = v_1v_3,\,\ v_1^{\alpha_{14}} = v_1v_4.\]
for some $\alpha_{12},\alpha_{13},\alpha_{14}\in P$, and so it suffices to show that $\gamma(v_1)=1$ by Lemma \ref{gamma lemma}. Let us put
\[ \gamma(v_1) = \begin{bmatrix}
1 & b_1 & c_1 & e_1\\
0 & 1 & d_1 & f_1\\
0 & 0 & 1 & 0\\
0 & 0 & 0 & 1
\end{bmatrix}.\]
From $\gamma(v_1^\alpha) = \gamma(v_1)^\alpha$ for $\alpha\in \Aut^c(\pi)$ of the shape
\[ \alpha =\begin{bmatrix}
1 & 0 & 0 & 0\\
0 & 1 & g & 0\\
0 & 0 & s & 0\\
0 & 0 & 0 & s^{-1}
\end{bmatrix} \mbox{ with  } s\in \mathbb{F}_p^\times\mbox{ and }g\in \mathbb{F}_p,\]
we get that $\gamma(v_1^\alpha) = \gamma(v_1)$ and
\[ \begin{bmatrix}
1 & b_1 & c_1 & e_1\\
0 & 1 & d_1 & f_1\\
0 & 0 &1 & 0\\
0 & 0 & 0 & 1
\end{bmatrix} = \begin{bmatrix}
1 & b_1 & gb_1 + sc_1 & s^{-1}e_1\\
0 & 1 & sd_1 & s^{-1}f_1\\
0 & 0 & 1 & 0\\
0 & 0 & 0 &1 
\end{bmatrix}.\]
This yields $b_1 = c_1 = d_1=e_1=f_1=0$. 
\item It is clear from Proposition \ref{auto3} that
\[v_1^{\alpha_{12}} = v_1v_2,\,\ v_1^{\alpha_{13}} = v_1v_3,\,\ v_1^{\alpha_{14}} = v_1v_4\]
for some $\alpha_{12},\alpha_{13},\alpha_{14}\in P$, and so it suffices to show that $\gamma(v_1)=1$ by Lemma \ref{gamma lemma}. Let us put
\[ \gamma(v_1) = \begin{bmatrix}
1 & b_1 & -d_1 & c_1\\
0 & 1 & 0 & 0\\
0 & c_1 & 1 & 0\\
0 & d_1 & 0 & 1
\end{bmatrix}.\]
From $\gamma(v_1^\alpha) = \gamma(v_1)^\alpha$ for $\alpha\in Q$ of the shape
\[ \alpha =\begin{bmatrix}
s& 0 & 0 & 0\\
0 & 1 & 0 & 0\\
0 & 0 & s & 0\\
0 & 0 & 0 &1
\end{bmatrix} \mbox{ with } s\in \mathbb{F}_p^\times,\]
we get that $\gamma(v_1^\alpha) = \gamma(v_1)^{s}$ and
\[\begin{bmatrix}
1 & sb_1 & -sd_1 & sc_1\\
0 & 1 & 0 & 0\\
0 & sc_1 & 1 & 0\\
0 & sd_1 & 0 & 1
\end{bmatrix}= \begin{bmatrix}
1 & s^{-1}b_1 & -d_1 & s^{-1}c_1\\
0 & 1 & 0 & 0\\
0 & s^{-1}c_1 & 1 & 0\\
0 & d_1 & 0 & 1\end{bmatrix}.\]
This implies that $d_1=0$. Since $p\geq 5$, there exists $s\in \mathbb{F}_p^\times$ with $s^2\neq 1$, and we see that $b_1=c_1=0$ as well.
\end{enumerate}
In all cases, we have shown that $\gamma$ is trivial.
 \end{proof}
 
 Therefore, we may apply Theorem \ref{pre thm} to obtain
 \begin{equation}\label{T(G)} T(G) \simeq S \rtimes \res(\mathcal{S}').\end{equation}
It remains to determine the structure of $S$ and $\res(\mathcal{S}')$.

 \subsection{A module-theoretic approach} 
 
Observe that by the universal property of $S^2V$, the symmetric square of $V$, there is a natural correspondence between
\begin{itemize}
\item symmetric bilinear forms $V\times V\rightarrow\Lambda^2V$,
\item linear maps $S^2V\rightarrow \Lambda^2V$.
\end{itemize}
Similarly, there is a natural correspondence between
\begin{itemize}
\item anti-symmetric bilinear forms $V\times V\rightarrow\Lambda^2V$,
\item linear maps $\Lambda^2V\rightarrow \Lambda^2V$.
\end{itemize}
Since we are writing addition in $V$ multiplicatively, let us denote multiplication in $S^2V$ by $*$ to avoid confusion. Then, both $S^2V$ and $\Lambda^2V$ are naturally $\Aut^c(\pi)$-modules via the action
\[ (u* v)^{\alpha} = u^\alpha * v^\alpha\mbox{ and }(u\wedge v)^\alpha = u^\alpha \wedge v^\alpha\]
for all $u,v\in V$ and $\alpha\in \Aut^c(\pi)$. Taking (\ref{Delta3}) into account, it follows that elements of $S$ and $S'$, respectively, correspond to $\Aut^c(\pi)$-module homomorphisms $S^2V\rightarrow \Lambda^2V$ and $\Lambda^2V\rightarrow\Lambda^2V$.

Let us first restrict the action to $Q$. An $\Aut^c(\pi)$-module homomorphism is in particular a $Q$-module homomorphism. The latter is easier to understand because matrices in $Q$ are all block diagonal, and so we easily see that both $S^2V$ and $\Lambda^2V$, as $Q$-modules, are decomposable as a direct sum of irreducible submodules. In the tables below, we list a basis for each irreducible component, and we indicate the action of an arbitrary $\alpha\in Q$ in matrix form with respect to the given basis. Here
\[ \alpha = \begin{bmatrix} s & 0 & 0 \\ 0 & 1 & 0 \\ 0 & 0 &t\end{bmatrix},\begin{bmatrix}
|A| &  \begin{matrix} 0 & 0 \end{matrix}\\
 \begin{matrix} 0 \\ 0 \end{matrix} & A
\end{bmatrix}\]
in cases (a),(b), respectively, while 
\[ \alpha = \begin{bmatrix}
s & 0 & \begin{matrix} 0 & 0 \end{matrix}\\
0 & 1 & \begin{matrix} 0 & 0 \end{matrix}\\
\begin{matrix} 0 \\ 0 \end{matrix} & \begin{matrix} 0 \\ 0 \end{matrix} & A
\end{bmatrix},\begin{bmatrix}
|A| & 0 & \begin{matrix} 0 & 0 \end{matrix}\\
0 & s & \begin{matrix} 0 & 0 \end{matrix}\\
\begin{matrix} 0 \\ 0 \end{matrix} & \begin{matrix} 0 \\ 0 \end{matrix} & A
\end{bmatrix},\begin{bmatrix}
|A| & 0 & \begin{matrix} 0 & 0 \end{matrix}\\
0 & 1 & \begin{matrix} 0 & 0 \end{matrix}\\
\begin{matrix} 0 \\ 0 \end{matrix} & \begin{matrix} 0 \\ 0 \end{matrix} & A
\end{bmatrix}\]
in cases (c),(d),(e), respectively. The variables $s,t$ here range over $\mathbb{F}_p^\times$, and $A$ ranges over $\GL_2(\mathbb{F}_p)$.

 \begingroup
\setlength{\tabcolsep}{10pt} % Default value: 6pt
\renewcommand{\arraystretch}{1.15}
%\captionof{table}{}
 \begin{center}
  \begin{longtable}{ |c|c|}
  \multicolumn{2}{c}{Case (a)}\\
 \hline
 \hline
\multicolumn{2}{|c|}{Components of $S^2V$} \\
\hline
 Basis & Action of $\alpha\in Q$\\ \hline
 $v_1*v_1 $ & $s^2$ \\ 
 $v_1*v_2$ & $s$ \\ 
 $v_1*v_3$ & $st$ \\ 
 $v_2*v_2$ & $1$\\
 $v_2*v_3$ & $t$ \\
 $v_3*v_3$ & $t^2$\\
\hline\hline
\multicolumn{2}{|c|}{Components of $\Lambda^2V$}\\
\hline
 Basis & Action of $\alpha\in Q$ \\ \hline
 $v_1\wedge v_2 $ & $s$\\ 
 $v_1\wedge v_3$ & $st$  \\ 
 $v_2\wedge v_3$ & $t$\\ 
\hline
\end{longtable} 
 \begin{longtable}{ |c|c|}
  \multicolumn{2}{c}{Case (b)}\\
 \hline
 \hline
\multicolumn{2}{|c|}{Components of $S^2V$}\\
\hline
 Basis & Action of $\alpha\in Q$  \\ \hline
 $v_1*v_1 $ & $|A|^2$ \\ 
 $v_1*v_2,v_1*v_3$ & $|A|A$  \\ 
 $v_2*v_2, v_2*v_3,v_3*v_3$ & omitted  \\ 
\hline
\hline
\multicolumn{2}{|c|}{Components of $\Lambda^2V$}\\
\hline
 Basis & Action of $\alpha\in Q$ \\ \hline
 $v_1\wedge v_2 ,v_1\wedge v_3$ & $|A|A$  \\ 
 $v_2\wedge v_3$ & $|A|$  \\ 
\hline
\end{longtable}
 \begin{longtable}{ |c|c| }
  \multicolumn{2}{c}{Case (c)}\\
 \hline
 \hline
\multicolumn{2}{|c|}{Components of $S^2V$}\\
\hline
 Basis & Action of $\alpha\in Q$ \\ \hline
 $v_1*v_1 $ & $s^2$ \\ 
 $v_1*v_2$ & $s$ \\ 
 $v_1*v_3,v_1*v_4$ & $sA$  \\ 
 $v_2*v_2$ & $1$ \\
 $v_2*v_3,v_2*v_4$ & $A$ \\
 $v_3*v_3,v_3*v_4,v_4*v_4$ & omitted \\
\hline
\hline
\multicolumn{2}{|c|}{Components of $\Lambda^2V$} \\
\hline
 Basis & Action of $\alpha\in Q$  \\ \hline
 $v_1\wedge v_2 $ & $s$  \\ 
 $v_1\wedge v_3,v_1\wedge v_4$ & $sA$  \\ 
 $v_2\wedge v_3,v_2 \wedge v_4$ & $A$  \\ 
 $v_3\wedge v_4$ & $|A|$ \\
\hline
\end{longtable}
%\captionof{table}{The case when $v_1^\pi = v_1\wedge v_2$}\label{a sym}
 \begin{longtable}{ |c|c| }
 \multicolumn{2}{c}{Case (d)}\\
 \hline
 \hline
\multicolumn{2}{|c|}{Components of $S^2V$}\\
\hline 
Basis & Action of $\alpha\in Q$ \\ \hline
 $v_1*v_1 $ & $|A|^2$ \\ 
 $v_1*v_2$ & $s|A|$  \\ 
 $v_1*v_3,v_1*v_4$ & $|A|A$  \\ 
 $v_2*v_2$ & $s^2$\\
 $v_2*v_3,v_2*v_4$ & $sA$  \\
 $v_3*v_3,v_3*v_4,v_4*v_4$ & omitted \\
\hline
\hline
\multicolumn{2}{|c|}{Components of $\Lambda^2V$}\\
\hline 
Basis & Action of $\alpha\in Q$  \\ \hline
 $v_1\wedge v_2 $ & $s|A|$ \\ 
 $v_1\wedge v_3,v_1\wedge v_4$ & $|A|A$  \\ 
 $v_2\wedge v_3,v_2 \wedge v_4$ & $sA$  \\ 
 $v_3\wedge v_4$ & $|A|$ \\
\hline
\end{longtable}
%\captionof{table}{The case when $v_1^\pi = v_3\wedge v_4$}\label{a sym}
 \begin{longtable}{ |c|c| }
 \multicolumn{2}{c}{Case (e)}\\
 \hline
 \hline
\multicolumn{2}{|c|}{Components of $S^2V$}\\
\hline 
Basis & Action of $\alpha\in Q$ \\ \hline
 $v_1*v_1 $ & $|A|^2$ \\ 
 $v_1*v_2$ & $|A|$  \\ 
 $v_1*v_3,v_1*v_4$ & $|A|A$  \\ 
 $v_2*v_2$ & $1$ \\
 $v_2*v_3,v_2*v_4$ & $A$  \\
 $v_3*v_3,v_3*v_4,v_4*v_4$ & omitted \\
\hline
\hline
\multicolumn{2}{|c|}{Components of $\Lambda^2V$}\\
\hline 
Basis & Action of $\alpha\in Q$  \\ \hline
 $v_1\wedge v_2 $ & $|A|$ \\ 
 $v_1\wedge v_3,v_1\wedge v_4$ & $|A|A$  \\ 
 $v_2\wedge v_3,v_2 \wedge v_4$ & $A$  \\ 
 $v_3\wedge v_4$ & $|A|$  \\
\hline
\end{longtable}
%\captionof{table}{The case when $v_1^\pi = v_3\wedge v_4$}\label{a sym}
\end{center} 
\endgroup

\vspace{-0.55cm}
 
Under a $Q$-module homomorphism, an irreducible component of the domain either lies in the kernel or gets mapped to an isomorphic irreducible component of the codomain. From the stated action of $Q$, we can easily compare the isomorphism classes of the irreducible components of $S^2V$ and $\Lambda^2V$. Note that the omitted action does not matter because $\Lambda^2V$ does not have any $3$-dimensional irreducible component. The next two propositions are then immediate. 
 
 \begin{prop}\label{prelim prop sym}For any $\Delta\in S$, the following holds.
 \begin{enumerate}[label= $(\arabic*)$]
 \item In case (a), we have
\begin{align*}\Delta(v_1,v_1)&=1,\\
\Delta(v_2,v_2) &=1,\\
 \Delta(v_3,v_3)&=1.
\end{align*}
\item In case (b), we have
\begin{align*}
\Delta(v_1,v_1)& = 1,\\
\Delta(v_2,v_2) &= \Delta(v_2,v_3) =\Delta(v_3,v_3)=1.
\end{align*}
\item In cases (c),(d), and (e), we have
\begin{align*}
\Delta(v_1,v_1) &=1,\\
 \Delta(v_2,v_2) &=1,\\
 \Delta(v_3,v_3) &= \Delta(v_3,v_4)=\Delta(v_4,v_4) =1.
 \end{align*}
 \end{enumerate}
 \end{prop}
 
 \begin{prop}\label{prelim prop anti}
For any $\Delta\in S'$, the following holds.
 \begin{enumerate}[label= $(\arabic*)$]
\item In case (a), we have
\begin{align*}
\Delta(v_1,v_2) & \in \langle v_1\wedge v_2\rangle,\\
\Delta(v_1,v_3) & \in \langle v_1\wedge v_3\rangle,\\
\Delta(v_2,v_3) & \in \langle v_2\wedge v_3\rangle.
\end{align*}
\item In case (b), we have
\begin{align*}
\Delta(v_1,v_2),\Delta(v_1,v_3)& \in \langle v_1\wedge v_2,v_1\wedge v_3\rangle,\\
\Delta(v_2,v_3) & \in \langle v_2\wedge v_3\rangle.
\end{align*}
\item In cases (c) and (d), we have
\begin{align*}
 \Delta(v_1,v_2) & \in \langle v_1\wedge v_2\rangle,\\\
 \Delta(v_1,v_3),\Delta(v_1,v_4) & \in \langle v_1\wedge v_3, v_1\wedge v_4 \rangle,\\
 \Delta(v_2,v_3),\Delta(v_2,v_4) & \in \langle v_2\wedge v_3, v_2\wedge v_4 \rangle, \\
 \Delta(v_3,v_4) & \in \langle v_3\wedge v_4\rangle.
\end{align*}
\item In case (e), we have
\begin{align*}
 \Delta(v_1,v_2),\Delta(v_3,v_4) & \in \langle v_1\wedge v_2, v_3\wedge v_4\rangle,\\
 \Delta(v_1,v_3),\Delta(v_1,v_4) & \in \langle v_1\wedge v_3, v_1\wedge v_4 \rangle,\\
 \Delta(v_2,v_3),\Delta(v_2,v_4) & \in \langle v_2\wedge v_3, v_2\wedge v_4 \rangle
\end{align*}
\end{enumerate}  
 \end{prop}
  
We may refine parts of Proposition \ref{prelim prop anti} as follows.

 \begin{prop}\label{scalar prop} For any $\Delta\in S'$, the following holds.
 \begin{enumerate}[label= $(\arabic*)$]
 \item In case (b), there exists $\lambda\in\mathbb{F}_p$ such that
 \[ \begin{cases}
 \Delta(v_1,v_2) = (v_1\wedge v_2)^\lambda,\\
 \Delta(v_1,v_3) = (v_1\wedge v_3)^\lambda.
 \end{cases}\]
 \item In cases (c),(d), and (e), there exist $\lambda_1,\lambda_2\in \mathbb{F}_p$ such that
\[\begin{cases}
\Delta(v_1,v_3) = (v_1\wedge v_3)^{\lambda_1} \\
\Delta(v_1,v_4) = (v_1\wedge v_4)^{\lambda_1}
\end{cases}\,\
\begin{cases}
\Delta(v_2,v_3) = (v_2\wedge v_3)^{\lambda_2},\\
\Delta(v_2,v_4) = (v_2\wedge v_4)^{\lambda_2}.
\end{cases}\]
 \end{enumerate}
  \end{prop}
 
 \begin{proof} Consider case (b). We know from Proposition \ref{prelim prop anti} that $\Delta$ has to induce a $Q$-module endomorphism 
 \[ \delta : \langle v_1\wedge v_2,v_1\wedge v_3\rangle \rightarrow \langle v_1\wedge v_2,v_1\wedge v_3 \rangle.\]
If $\delta$ is trivial, then simply take $\lambda=0$. If $\delta$ is non-trivial, then it has to be invertible because $\langle v_1\wedge v_2,v_1\wedge v_3\rangle$ is irreducible. Say $\delta$ is given by the matrix $M\in \GL_2(\mathbb{F}_p)$. But $M$ must commute with the action of $Q$ and observe that $Q$ restricts to an $\SL_2(\mathbb{F}_p)$-action on $\langle v_1\wedge v_2,v_1\wedge v_3\rangle$.
Since the only matrices that centralize $\SL_2(\mathbb{F}_p)$ are the scalar multiples of the identity, it follows that $M = \left[\begin{smallmatrix} \lambda& 0\\ 0 & \lambda\end{smallmatrix}\right]$
for some $\lambda\in\mathbb{F}_p^\times$. The proves (1), and the same argument may be applied to prove (2).\end{proof}

\subsection{Computation of $S$ and $S'$} We shall now compute $S$ and $S'$ by taking the action of $P$ into account.

First, notice that a symmetric bilinear form $\Delta : V\times V\rightarrow \Lambda^2V$ is uniquely determined by
  \[ \Delta(v_i,v_j)\mbox{ for }1\leq i \leq j \leq n.\]
The next observation shall also be useful.

\begin{lemma}\label{sym lemma}Let $\Delta \in S$ and let $1\leq i, j \leq n$. If
\begin{enumerate}[label = $(\arabic*)$]
\item $\Delta(v_i,v_i) = \Delta(v_j,v_j)=1$,
\item  $v_i^{\alpha }= v_iv_j$ for some $\alpha\in \Aut^c(\pi)$,
\end{enumerate}
then $\Delta(v_i,v_j) =\Delta(v_j,v_i)= 1$ also holds.
\end{lemma}

\begin{proof}By the hypothesis and the condition (\ref{Delta3}), we have
\begin{align*}
1 & = \Delta(v_i,v_i)^{\hat{\alpha}}\\
& = \Delta(v_i^{\alpha} ,v_i^{\alpha})\\
& = \Delta(v_iv_j,v_iv_j)\\
& = \Delta(v_i,v_i)\Delta(v_i,v_j)\Delta(v_j,v_i)\Delta(v_j,v_j)\\
&=\Delta(v_i,v_j)\Delta(v_j,v_i)\\
& = \Delta(v_i,v_j)^2,
\end{align*}
where the last equality holds because $\Delta$ is symmetric. Since $p$ is odd, we may take the square root and so $\Delta(v_i,v_j)=\Delta(v_j,v_i)=1$.
\end{proof}

\begin{prop}\label{S=1} We have $S=1$ in all cases (a),(b),(c),(d), and (e).
\end{prop}

\begin{proof}Let $\Delta\in S$ be arbitrary. We consider each case separately.
\begin{enumerate}[label=(\alph*),wide=0pt]
\item It is clear from Proposition \ref{auto1'} that
\[ v_1^{\alpha_{12}} = v_1v_2\mbox{ and } v_3^{\alpha_{23}} = v_2v_3\]
for some $\alpha_{12},\alpha_{23}\in P$. We then have
\[ \Delta(v_i,v_j) = 1\mbox{ for all }1\leq i \leq j\leq 3\mbox{ with }(i,j)\neq (1,3) \]
by Proposition \ref{prelim prop sym} and Lemma \ref{sym lemma}. Comparing the irreducible components of $S^2V$ and $\Lambda^2V$ as $Q$-modules, we also see that
\[ \Delta(v_1,v_3) = (v_1\wedge v_3)^\lambda\]
for some $\lambda\in\mathbb{F}_p$. But consider the action of $\alpha\in P$ given by
\[ \alpha = \begin{bmatrix} 1 & 1 & 0 \\ 0 & 1 & 0 \\ 0 & 1 & 1\end{bmatrix}.\]
By the condition (\ref{Delta3}), we have
\begin{align*}
\Delta(v_1,v_3)^{\hat{\alpha}}  & = \Delta(v_1^\alpha,v_3^\alpha)\\
& = \Delta(v_1v_2,v_2v_3)\\
& = \Delta(v_1,v_2)\Delta(v_1,v_3)\Delta(v_2,v_2)\Delta(v_2,v_3)\\
& = \Delta(v_1,v_3).
\end{align*}
But the left hand side is equal to
\[(v_1v_2\wedge v_2v_3)^\lambda =  (v_1\wedge v_2)^\lambda  (v_2\wedge v_3)^\lambda\Delta(v_1,v_3).\]
It follows that $\lambda=0$ and so $\Delta(v_1,v_3)=1$ also holds.
\item It is clear from Proposition \ref{auto2'} that
\[ v_1^{\alpha_{12}} = v_1v_2\mbox{ and } v_1^{\alpha_{13}} = v_1v_3\]
for some $\alpha_{12},\alpha_{13}\in P$. We then have
\[ \Delta(v_i,v_j) = 1\mbox{ for all }1\leq i \leq j\leq 3 \]
by Proposition \ref{prelim prop sym} and Lemma \ref{sym lemma}. 
\item It is clear from Proposition \ref{auto1} that
\[ v_1^{\alpha_{12}} = v_1v_2,\, 
v_3^{\alpha_{23}} = v_2v_3,\, v_4^{\alpha_{24}} = v_2v_4\]
for some $\alpha_{12},\alpha_{23},\alpha_{24}\in P$. We then have
 \[ \Delta(v_i,v_j)=1\mbox{ for all }1\leq i \leq j \leq 4 \mbox{ with }(i,j)\not\in\{(1,3),(1,4)\}\]
by Proposition \ref{prelim prop sym} and Lemma \ref{sym lemma}. Comparing the irreducible components of $S^2V$ and $\Lambda^2V$ as $Q$-modules, we also see that
\[ \Delta(v_1,v_3),\Delta(v_1,v_4)\in \langle v_1\wedge v_3,v_1\wedge v_4\rangle\]
has to hold. Let us write
\[ \Delta(v_1, v_3) = (v_1\wedge v_3)^{\lambda}(v_1\wedge v_4)^{\kappa},\]
and consider the action of $\alpha_1\in P$ defined by
\[\alpha_1 =  \begin{bmatrix}1 & 1 & 0 & 0 \\
0 & 1 & 0 & 0\\
 0& 1 & 1 & 0\\
 0 & 0 & 0 & 1\end{bmatrix}.
 \]
Since $\Delta$ satisfies the condition (\ref{Delta3}), we get that
\begin{align*}
\Delta(v_1,v_3)^{\hat{\alpha}_1}
& = \Delta(v_1^{\alpha_1},v_3^{\alpha_1})\\
& = \Delta(v_1v_2,v_2v_3)\\
& =\Delta(v_1,v_2)\Delta(v_1,v_3)\Delta(v_2,v_2)\Delta(v_2,v_3)\\
& = \Delta(v_1,v_3).\end{align*}
But explicitly, the left hand side is given by
\[(v_1v_2\wedge v_2v_3)^\lambda (v_1v_2\wedge v_4)^{\kappa}  = (v_1\wedge v_2)^{\lambda} (v_2\wedge v_3)^\lambda(v_2\wedge v_4)^\kappa\Delta(v_1,v_3).\]
This shows that $\lambda = \kappa = 0$ and hence $\Delta(v_1,v_3) =1$. Since there exists $\alpha_2\in Q$ for which $v_1^{\alpha_2} = v_1$ and $v_3^{\alpha_2} = v_4$, we have
\[ 1 = \Delta(v_1,v_3)^{\hat{\alpha}_2} = \Delta(v_1^{\alpha_2},v_3^{\alpha_2} ) = \Delta(v_1,v_4).\]
We have thus shown that $\Delta(v_1,v_3) = \Delta(v_1,v_4)=1$ also holds.
\item It is clear from Proposition \ref{auto2} that 
\[ v_1^{\alpha_{12}} = v_1v_2,\,
v_1^{\alpha_{13}} = v_1v_3,\,
v_1^{\alpha_{14}} = v_1v_4,\,
v_2^{\alpha_{23}} = v_2v_3,\,
v_2^{\alpha_{24}} = v_2v_4\]
for some $\alpha_{12},\alpha_{13},\alpha_{14},\alpha_{23},\alpha_{24}\in P$. We then have
\[ \Delta(v_i,v_j) = 1\mbox{ for all }1\leq i \leq j\leq 4 \]
by Proposition \ref{prelim prop sym} and Lemma \ref{sym lemma}.  
\item It is clear from Proposition \ref{auto3} that
\[ v_1^{\alpha_{12}} = v_1v_2,\,
v_1^{\alpha_{13}} = v_1v_3,\,
v_1^{\alpha_{14}} = v_1v_4,\,
v_3^{\alpha_{23}} = v_2v_3,\,
v_4^{\alpha_{24}}= v_2v_4\]
for some $\alpha_{12},\alpha_{13},\alpha_{14},\alpha_{23},\alpha_{24}\in P$. We then have
\[ \Delta(v_i,v_j) = 1\mbox{ for all }1\leq i \leq j\leq 4 \]
by Proposition \ref{prelim prop sym} and Lemma \ref{sym lemma}.  
\end{enumerate}
In all cases, we have shown that $\Delta=1$, and so indeed $S=1$.
  \end{proof}
 
Next, note that an anti-symmetric bilinear form $\Delta :V \times V \rightarrow \Lambda^2V$ is uniquely determined by
\[ \Delta(v_i,v_j) \mbox{ for }1\leq i < j\leq n.\]
We also make the following observation.

\begin{lemma}\label{anti lemma}
Let $\Delta\in S'$ and let $1\leq i,j,k\leq n$ with $i\neq j,k$. If
\begin{enumerate}[label = $(\arabic*)$]
\item $\Delta(v_i,v_j) = (v_i\wedge v_j)^{\lambda_1}$ or equivalently $\Delta(v_j,v_i) = (v_j\wedge v_i)^{\lambda_1}$,
\item $\Delta(v_i,v_k) = (v_i\wedge v_k)^{\lambda_2}$ or equivalently $\Delta(v_k,v_i) = (v_k\wedge v_i)^{\lambda_2}$,
\item $v_i^\alpha = v_i,\, v_j^\alpha =v_jv_k$ for some $\alpha\in \Aut^c(\pi)$,\end{enumerate}
then $\lambda_1 = \lambda_2$ has to hold.
\end{lemma}

%Note that the equivalence holds because $\Delta$ is anti-symmetric.

\begin{proof}By the condition (\ref{Delta3}), we have
\[
\Delta(v_i,v_j)^{\hat{\alpha}} = \Delta(v_i^\alpha,v_j^\alpha)
= \Delta(v_i,v_jv_k)
=\Delta(v_i,v_j)\Delta(v_i,v_k). \]
Using the hypothesis, we rewrite this as
\[ (v_i\wedge v_j)^{\lambda_1}(v_i\wedge v_k)^{\lambda_1} = (v_i\wedge v_j)^{\lambda_1}( v_i\wedge v_k)^{\lambda_2},\] 
which implies that $\lambda_1 =\lambda_2$, as claimed.
\end{proof}

For each $\lambda\in\mathbb{F}_p$, as noted in Remark \ref{remark}, clearly
\[ \Delta_{[\lambda]} : V \times V\rightarrow \Lambda^2V;\,\ \Delta_{[\lambda]}(u,v) = (u\wedge v)^\lambda\]
is an anti-symmetric bilinear form satisfying (\ref{Delta3}), namely $\Delta_{[\lambda]}\in S'$.

\begin{prop}\label{S' prop}We have
\[ S' =\begin{cases}
  \{ \Delta_{[\lambda]} : \lambda\in \mathbb{F}_p\}&\mbox{in cases (a),(b),(c), and (d)},\\
  \{ \Delta_{[\lambda]}\Delta_{[\kappa]}^* : \lambda,\kappa\in \mathbb{F}_p\}&\mbox{in case (e)},
  \end{cases} \] 
where $\Delta_{[\kappa]}^* : V\times V\rightarrow \Lambda^2V$ denotes the anti-symmetric form defined by
\begin{align*}
\Delta_{[\kappa]}^*(v_1,v_2) & = (v_3\wedge v_4)^\kappa,&\Delta_{[\kappa]}^*(v_2,v_3) & = (v_2\wedge v_3)^{-\kappa},\\
\Delta_{[\kappa]}^*(v_1,v_3) & = (v_1\wedge v_3)^{-\kappa},&\Delta_{[\kappa]}^*(v_2,v_4)& = (v_2\wedge v_4)^{-\kappa},\\
\Delta_{[\kappa]}^*(v_1,v_4)& = (v_1\wedge v_4)^{-\kappa},&\Delta_{[\kappa]}^*(v_3,v_4) & = (v_1\wedge v_2)^{\kappa}.\end{align*}
\end{prop}

\begin{proof}Let $\Delta\in S'$ be arbitrary. We consider each case separately.
\begin{enumerate}[label=(\alph*),wide=0pt]
\item[(a),(b)] By Propositions \ref{prelim prop anti} and \ref{scalar prop}, we know that
\begin{align*} \Delta(v_1,v_2) &= (v_1\wedge v_2)^{\lambda_1}\\ 
\Delta(v_1,v_3) &= (v_1\wedge v_3)^{\lambda_2}\\
\Delta(v_2,v_3) &= (v_2\wedge v_3)^{\lambda_3}
\end{align*}
for some $\lambda_1,\lambda_2,\lambda_3 \in \mathbb{F}_p$. In case (a), by Proposition \ref{auto1'}, we have
\[ \begin{cases}
v_1^{\alpha_{12}} = v_1\\
v_3^{\alpha_{12}} = v_2v_3
\end{cases}\,\ \begin{cases}
v_3^{\alpha_{23}} = v_3\\
v_1^{\alpha_{23}} = v_1v_2
\end{cases}\]
for some $\alpha_{12},\alpha_{23}\in P$. In case (b), we already know from Proposition  \ref{scalar prop} that $\lambda_1=\lambda_2$, and by Proposition \ref{auto2'}, we have
\[ \begin{cases}
v_3^{\alpha_{23}} = v_3\\
v_1^{\alpha_{23}} = v_1v_2
\end{cases}\]
for some $\alpha_{23}\in P$. In both cases, we get that
\[\lambda :=\lambda_1 = \lambda_2 = \lambda_3\]
by Lemma \ref{anti lemma}. This shows that $\Delta = \Delta_{[\lambda]}$, as claimed.
\item[(c),(d)] By Propositions \ref{prelim prop anti} and \ref{scalar prop}, we know that
\begin{align*} \Delta(v_1,v_2) &= (v_1\wedge v_2)^{\lambda_1}&\Delta(v_2,v_3) & = (v_2\wedge v_3)^{\lambda_3}\\ 
\Delta(v_1,v_3) &= (v_1\wedge v_3)^{\lambda_2} & \Delta(v_2,v_4) &= (v_2\wedge v_4)^{\lambda_3}\\
\Delta(v_1,v_4) &= (v_1\wedge v_4)^{\lambda_2}&\Delta(v_3,v_4) &= (v_3\wedge v_4)^{\lambda_4}
\end{align*}
for some $\lambda_1,\lambda_2,\lambda_3,\lambda_4 \in \mathbb{F}_p$. In case (c), by Proposition \ref{auto1}, we have
\[ \begin{cases}
v_1^{\alpha_{12}} = v_1\\
v_3^{\alpha_{12}} = v_2v_3
\end{cases}\,\
\begin{cases}
v_3^{\alpha_{23}} = v_3\\
v_1^{\alpha_{23}} = v_1v_2
\end{cases}
\,\
\begin{cases}
v_4^{\alpha_{34}} = v_4\\
v_3^{\alpha_{34}} = v_2v_3
\end{cases}\]
for some $\alpha_{12},\alpha_{23},\alpha_{34}\in P$. In case (d), by Proposition \ref{auto2}, we have
\[ \begin{cases}
v_1^{\alpha_{12}} = v_1\\
v_2^{\alpha_{12}} = v_2v_3
\end{cases}\,\
\begin{cases}
v_3^{\alpha_{23}} = v_3\\
v_1^{\alpha_{23}} = v_1v_2
\end{cases}
\,\
\begin{cases}
v_4^{\alpha_{34}} = v_4\\
v_2^{\alpha_{34}} = v_2v_3
\end{cases}\]
for some $\alpha_{12},\alpha_{23},\alpha_{34}\in P$. In both cases, we get that 
\[\lambda :=\lambda_1 = \lambda_2 = \lambda_3= \lambda_4\]
by Lemma \ref{anti lemma}. This shows that $\Delta = \Delta_{[\lambda]}$, as claimed.
\item[(e)] By Propositions \ref{prelim prop anti} and \ref{scalar prop}, we know that 
\begin{align*} \Delta(v_1,v_2) &= (v_1\wedge v_2)^{\lambda_1}(v_3\wedge v_4)^{\kappa_1}&\Delta(v_2,v_3) & = (v_2\wedge v_3)^{\lambda_3}\\ 
\Delta(v_1,v_3) &= (v_1\wedge v_3)^{\lambda_2} & \Delta(v_2,v_4) &= (v_2\wedge v_4)^{\lambda_3}\\
\Delta(v_1,v_4) &= (v_1\wedge v_4)^{\lambda_2}&\Delta(v_3,v_4) &= (v_1\wedge v_2)^{\kappa_4}(v_3\wedge v_4)^{\lambda_4}
\end{align*}
 for some $\lambda_1,\lambda_2,\lambda_3,\lambda_4,\kappa_1,\kappa_4\in \mathbb{F}_p$. Consider $\alpha\in P$ given by
\[ \alpha = \begin{bmatrix} 1 & 0 & 0 & 1\\
0 & 1 & 0 & 0\\
 0 & 1 & 1 & 0 \\ 
 0 & 0 & 0 & 1\end{bmatrix},\]
and we compute that
 \begin{align*}
\Delta(v_1,v_2)^{\hat{\alpha}}& = (v_1v_4\wedge v_2)^{\lambda_1}(v_2v_3\wedge v_4)^{\kappa_1} \\
&= \Delta(v_1,v_2)(v_4\wedge v_2)^{\lambda_1-\kappa_1},\\
\Delta(v_1^\alpha,v_2^\alpha) & = \Delta(v_1v_4,v_2) \\
&=\Delta(v_1,v_2)(v_4\wedge v_2)^{\lambda_3},\\
\Delta(v_1,v_3)^{\hat{\alpha}} & = (v_1v_4\wedge v_2v_3)^{\lambda_2} \\
&= \Delta(v_1,v_3)(v_1\wedge v_2)^{\lambda_2}(v_4\wedge v_2)^{\lambda_2}(v_4\wedge v_3)^{\lambda_2},\\
\Delta(v_1^\alpha,v_3^\alpha) & = \Delta(v_1v_4,v_2v_3) \\
&= \Delta(v_1,v_3)(v_1\wedge v_2)^{\lambda_1-\kappa_4}(v_4\wedge v_2)^{\lambda_3}(v_4\wedge v_3)^{\lambda_4-\kappa_1},\\
\Delta(v_3,v_4)^{\hat{\alpha}} & = (v_1v_4\wedge v_2)^{\kappa_4}(v_2v_3\wedge v_4)^{\lambda_4}\\
& = \Delta(v_3,v_4)(v_2\wedge v_4)^{\lambda_4-\kappa_4},\\
\Delta(v_3^\alpha,v_4^\alpha) & = \Delta(v_2v_3,v_4)\\
&= \Delta(v_3,v_4)(v_2\wedge v_4)^{\lambda_3}.
\end{align*}
Since the condition (\ref{Delta3}) has to hold, we deduce that
\[ \lambda_3= \lambda_1 - \kappa_1,\,\
\lambda_2 = \lambda_1-\kappa_4 =\lambda_3=\lambda_4-\kappa_1,\,\ \lambda_3 = \lambda_4-\kappa_4.\]
Solving this system of equations, we get that
\[ \lambda := \lambda_1 = \lambda_4 ,\,\ \kappa:=\kappa_1=\kappa_4,\,\ \lambda_2 =\lambda_3 = \lambda -\kappa.\]
This shows that $\Delta = \Delta_{[\lambda]}\Delta_{[\kappa]}^*$. Conversely, for any $\lambda,\kappa\in\mathbb{F}_p$, we know that $ \Delta_{[\lambda]}\in S'$ already and it is straightforward to check that $\Delta_{[\kappa]}^*$ also satisfies (\ref{Delta3}), so then $\Delta_{[\lambda]}\Delta_{[\kappa]}^*\in S'$.
 \end{enumerate}
 This completes the proof.
\end{proof}
   
\subsection{The structure of $T(G)$} We shall now prove Theorem \ref{thm1}. We already know from (\ref{T(G)}) and Proposition \ref{S=1} that
\[ T(G) \simeq \res(\mathcal{S}').\]
In cases (a),(b),(c), and (d), the theorem follows because we have
\[ \res(\mathcal{S}') \simeq \mathbb{F}_p^\times\]
by Remark \ref{remark} and Proposition \ref{S' prop}. In case (e), by Proposition \ref{S' prop}, the elements of $S'$  are precisely the bilinear forms
\[ \Delta_{[\sigma]}: V\times V\rightarrow\Lambda^2V ;\,\  \Delta_{[\sigma]}(u,v) = (u\wedge v)^\sigma.\]
Here $\sigma$ is any endomorphism on $\Lambda^2V$ of the form
\begin{equation}\label{tau}
 \begin{bmatrix}
\lambda & &  & &&\kappa\\
 & \lambda-\kappa & & &&\\
 & & \lambda-\kappa & & &\\
 & & & \lambda-\kappa & &\\
 & & & &\lambda-\kappa &\\
\kappa & & & &&\lambda
\end{bmatrix} \mbox{ with }\lambda,\kappa\in \mathbb{F}_p,\end{equation}
written with respect to the basis
\[ v_1\wedge v_2, v_1\wedge v_3, v_1\wedge v_4,v_2\wedge v_3,v_2\wedge v_4,v_3\wedge v_4\]
of $\Lambda^2V$. By \cite[Example 3.4]{LMH}, we know that $N_{\Delta_{[\sigma]}}\simeq G$ occurs only for $1+2\sigma\in \GL(\Lambda^2V)$. Let us make a change of variables $\tau = 1+2\sigma$, and consider $\tau_{\lambda,\kappa}\in \GL(\Lambda^2V)$ of the form (\ref{tau}) but with the restriction $\kappa\neq\pm\lambda$. Observe that then
\[
\eta_{\lambda,\kappa}   = \begin{bmatrix}
\lambda+\kappa &&&\\
&(\lambda+\kappa)^{-1} &&\\
&&1&\\
&&&1
\end{bmatrix},\]
written with respect to the basis $v_1,v_2,v_3,v_4$ of $V$, in which case
\[\hat{\eta}_{\lambda,\kappa} = \begin{bmatrix}
1 &&&&&\\
 &\lambda+\kappa&&&&\\
 &&\lambda+\kappa&&&\\
 &&&(\lambda+\kappa)^{-1}&&\\
 &&&&(\lambda+\kappa)^{-1}&\\
 &&&&&1
\end{bmatrix},\]
yields a solution to $\pi\hat{\eta}_{\lambda,\kappa}\tau_{\lambda,\kappa} = \eta_{\lambda,\kappa}\pi$. From (\ref{S'}), we deduce that
\[ \res(\mathcal{S}') \simeq \{(\eta_{\lambda,\kappa},\hat{\eta}_{\lambda,\kappa}\tau_{\lambda,\kappa}) : \lambda,\kappa\in\mathbb{F}_p\mbox{ with }\kappa\neq\pm\lambda\}.  \]
It is straightforward to verify that
\[ \eta_{\lambda_{1},\kappa_1} \eta_{\lambda_{2},\kappa_{2}} = \eta_{\lambda,\kappa},\,\
 \hat{\eta}_{\lambda_1,\kappa_1}\tau_{\lambda_1,\kappa_1}\hat{\eta}_{\lambda_2,\kappa_2}\tau_{\lambda_2,\kappa_2} = \hat{\eta}_{\lambda,\kappa}\tau_{\lambda,\kappa}\]
 for any $\lambda_1,\lambda_2,\kappa_1,\kappa_2\in \mathbb{F}_p$ with $\kappa_1\neq\pm\lambda_1$ and $\kappa_2\neq\pm\lambda_2$, where
 \[\begin{bmatrix} \lambda & \kappa\\ \kappa &\lambda \end{bmatrix}= \begin{bmatrix} \lambda_1 &\kappa_1\\\kappa_1&\lambda_1\end{bmatrix}
 \begin{bmatrix}\lambda_2& \kappa_2\\ \kappa_2 & \lambda_2\end{bmatrix}
 =\begin{bmatrix} \lambda_1\lambda_2 + \kappa_1\kappa_2 & \lambda_1\kappa_2 +\lambda_2\kappa_1\\\lambda_1\kappa_2 +\lambda_2\kappa_1&\lambda_1\lambda_2 + \kappa_1\kappa_2 \end{bmatrix}.\]
It follows that $\res(\mathcal{S}')$ is isomorphic to the subgroup
\[ \left\{ \begin{bmatrix}\lambda & \kappa \\ \kappa & \lambda\end{bmatrix}  : \lambda,\kappa\in\mathbb{F}_p\mbox{ with }\kappa\neq\pm\lambda\right\}\]
of $\GL_2(\mathbb{F}_p)$, or conjugating by $\left[\begin{smallmatrix}1 & -1\\ 1 & 1 \end{smallmatrix}\right]$, the subgroup
\[ \left\{ \begin{bmatrix}\lambda + \kappa& 0 \\  0 & \lambda- \kappa\end{bmatrix}  : \lambda,\kappa\in\mathbb{F}_p\mbox{ with }\kappa\neq\pm\lambda\right\}\]
of $\GL_2(\mathbb{F}_p)$. This decomposes as
\[ \left\{ \begin{bmatrix} \lambda & 0 \\ 0 & 1 \end{bmatrix}: \lambda\in \mathbb{F}_p^\times \right\}\times  \left\{ \begin{bmatrix} 1 & 0 \\ 0 & \kappa \end{bmatrix}: \kappa\in \mathbb{F}_p^\times \right\}\]
and so is isomorphic to $\mathbb{F}_p^\times \times \mathbb{F}_p^\times$, as claimed in (e).
%We then have
%\begin{align*}
% \res(\mathcal{S'}) = & \{ (\eta,\hat{\eta}\tau)\Gamma(G) : \eta\in \GL(V),\, \tau\in \GL(\Lambda^2V)\\
% &\hspace{2.5cm}\mbox{of the shape (\ref{tau}) with $\kappa \neq \lambda,\pm2\lambda$,}\\
% &\hspace{2.5cm}\mbox{and the equation }\pi\hat{\eta}\tau = \eta\pi\mbox{ holds}\}
% \end{align*}
%by (\ref{res(S')}). Let us solve $\pi\hat{\eta}\tau = \eta\pi$ for such $\eta\in \GL(V)$ and $\tau\in \GL(\Lambda^2V)$ in a manner very similar to the proof of Proposition \ref{auto3}. Write 
%\begin{align*}
%v_1^\eta & = v_1^{n_{11}} v_2^{n_{12}} v_3^{n_{13}} v_4^{n_{14}}, \\
%v_2^\eta & = v_1^{n_{21}} v_2^{n_{22}} v_3^{n_{23}} v_4^{n_{24}},\\
%v_3^\eta & = v_1^{n_{31}} v_2^{n_{32}} v_3^{n_{33}} v_4^{n_{34}},\\
%v_4^\eta & = v_1^{n_{41}} v_2^{n_{42}} v_3^{n_{43}} v_4^{n_{44}}.
%\end{align*}
%Since $v_2^\pi =v_3^\pi = v_4^\pi = 1$, necessarily $n_{21} = n_{31} = n_{41} = 0$ and $n_{11}\neq0$. We may then simplify $v_1^{\pi\hat{\eta}\tau} = v_1^{\eta\pi}$ as
%\begin{align*}
%&((v_1^{n_{11}} v_2^{n_{12}} v_3^{n_{13}} v_4^{n_{14}} \wedge v_2^{n_{22}} v_3^{n_{23}} v_4^{n_{24}})( v_2^{n_{32}} v_3^{n_{33}} v_4^{n_{34}}\wedge v_2^{n_{42}} v_3^{n_{43}} v_4^{n_{44}}))^{\tau} \\
%&\hspace{7.25cm}= (v_1\wedge v_2)^{n_{11}} (v_3\wedge v_4)^{n_{11}}.\end{align*}
%Since $\langle v_1\wedge v_3\rangle$ and $\langle v_1\wedge v_4\rangle$
%are eigenspaces of $\tau$, which is taken to be invertible here, by comparing exponents, we see that $n_{23} = n_{24}=0$. and $n_{22}\neq0$. The above equation then becomes
%\begin{align*}
%&((v_1^{n_{11}} v_2^{n_{12}} v_3^{n_{13}} v_4^{n_{14}} \wedge v_2^{n_{22}})( v_2^{n_{32}} v_3^{n_{33}} v_4^{n_{34}}\wedge v_2^{n_{42}} v_3^{n_{43}} v_4^{n_{44}}))^{\tau} \\
%&\hspace{7.25cm}= (v_1\wedge v_2)^{n_{11}} (v_3\wedge v_4)^{n_{11}}.\end{align*}
%Since $\langle v_2\wedge v_3\rangle$ and $\langle v_2\wedge v_4\rangle$
%are eigenspaces of $\tau$, by comparing exponents, we similarly deduce that
%\[ -n_{13}n_{22} + \begin{vmatrix} n_{32} & n_{33} \\ n_{42} & n_{43} \end{vmatrix}
%= -n_{14}n_{22} + \begin{vmatrix} n_{32} & n_{34} \\ n_{42} & n_{44} \end{vmatrix} = 0.\]
%Finally, by comparing the $v_1\wedge v_2$ and $v_3\wedge v_4$ terms, we obtain
%\[ \begin{bmatrix}\lambda & \kappa \\ \kappa &\lambda\end{bmatrix}
%\begin{bmatrix}n_{11}n_{22} \\[4pt] \lvert\begin{smallmatrix} n_{33}&n_{34}\\n_{43}&n_{44}\end{smallmatrix}\rvert\end{bmatrix} 
%=\begin{bmatrix} n_{11}\\n_{11}\end{bmatrix}.\]
%The matrix on the left is taken to be invertible, so equivalently
%\[ \begin{bmatrix}n_{22}\\[4pt] n_{11}^{-1}\lvert\begin{smallmatrix} n_{33}&n_{34}\\n_{43}&n_{44}\end{smallmatrix}\rvert\end{bmatrix}=
%\begin{bmatrix} \lambda & \kappa \\ \kappa & \lambda\end{bmatrix}^{-1}\begin{bmatrix}1\\1\end{bmatrix} = \begin{bmatrix}(\lambda+\kappa)^{-1} 
%\\ (\lambda + \kappa)^{-1} 
%\end{bmatrix}.\]
%Put $s_{\lambda,\kappa} =\lambda +\kappa$. Then $\pi\hat{\eta}\tau = \eta\pi$ holds if and only if
%\begin{align*}
%\eta &= \begin{bmatrix}
%s_{\lambda,\kappa} \lvert\begin{smallmatrix}n_{33} & n_{34}\\n_{43} & n_{44}\end{smallmatrix}\rvert & n_{12} & s_{\lambda,\kappa} \lvert\begin{smallmatrix}n_{32} & n_{33} \\ n_{42} & n_{43}\end{smallmatrix}\rvert & s_{\lambda,\kappa} \lvert\begin{smallmatrix} n_{32} & n_{34} \\ n_{42} & n _{44}\end{smallmatrix}\rvert\\
%0 & s_{\lambda,\kappa}^{-1} & 0 & 0\\
%0 & n_{32} & n_{33} & n_{34}\\
%0 & n_{42} & n_{43} & n_{44}
%\end{bmatrix}\\
%& = \begin{bmatrix} s_{\lambda,\kappa} & 0 & 0& 0 \\
%0 & s_{\lambda,\kappa}^{-1}& 0 & 0 \\
% 0 & 0 & 1 & 0\\
% 0 & 0 & 0 & 1\end{bmatrix} \begin{bmatrix}
%\lvert\begin{smallmatrix}n_{33} & n_{34}\\n_{43} & n_{44}\end{smallmatrix}\rvert & s_{\lambda,\kappa}^{-1}n_{12} &\lvert\begin{smallmatrix}n_{32} & n_{33} \\ n_{42} & n_{43}\end{smallmatrix}\rvert & \lvert\begin{smallmatrix} n_{32} & n_{34} \\ n_{42} & n _{44}\end{smallmatrix}\rvert \\
%0 & 1 & 0 & 0\\
%0 & n_{32} & n_{33} & n_{34}\\
%0 & n_{42} & n_{43} & n_{44}
%\end{bmatrix},
%\end{align*}
%where this last matrix lies in $\Aut^c(\pi)$ by Proposition \ref{auto3}. The class of $(\eta,\hat{\eta}\tau)$ modulo $\Gamma(G)$ is not affected when $\eta$ is multiplied by an element of $\Aut^c(\pi)$. Thus, we may take
%\[ \eta = \begin{bmatrix} s_{\lambda,\kappa} &  & &  \\
% & s_{\lambda,\kappa}^{-1}&  &  \\
%  &  & 1 & \\
%  &  &  & 1\end{bmatrix},\,\ \hat{\eta} =
%\begin{bmatrix}
%  1 & & & & & \\
%  & s_{\lambda,\kappa}& &  &\\
%  &  & s_{\lambda,\kappa} & & &\\
%  & & & s_{\lambda,\kappa}^{-1} & &\\
%   &  & & &s_{\lambda,\kappa}^{-1} & \\
%  & &  & & & 1
%  \end{bmatrix}.\]
%% in which case we have
%% \begin{align*}
%%  \hat{\eta}\tau  =
%%\begin{bmatrix}
%%  \lambda & 0 & 0 & 0 & 0 &\kappa\\
%%  0 & (\lambda-\kappa)s_{\lambda,\kappa}& 0 & 0 &0 & 0\\
%%  0 & 0 & (\lambda-\kappa)s_{\lambda,\kappa} & 0 & 0 &0\\
%%  0 & 0 & 0 & (\lambda-\kappa)s_{\lambda,\kappa}^{-1}   & 0 & 0\\
%%   0 & 0 & 0 & 0 &(\lambda-\kappa)s_{\lambda,\kappa}^{-1} & 0 \\
%%\kappa & 0 & 0 & 0 & 0 & \lambda
%%\end{bmatrix}.
%%   \end{align*}
%To simplify notation, let us put
%\begin{align*}
%M_{\lambda,\kappa} & =  \left[ \begin{smallmatrix} \lambda + \kappa&&&\\ & (\lambda+\kappa)^{-1} &&\\ &&1&&\\&&&1\end{smallmatrix}\right],\\
% N_{\lambda,\kappa} & =  \left[\begin{smallmatrix}
% \lambda &&&&&\kappa\\
% & (\lambda-\kappa)(\lambda+\kappa) &&&&\\
% &&(\lambda-\kappa)(\lambda+\kappa) &&&\\\
% &&&(\lambda-\kappa)(\lambda+\kappa)^{-1}&&\\
% &&&&(\lambda-\kappa)(\lambda+\kappa)^{-1}&\\
%\kappa &&&&&\lambda
%\end{smallmatrix} \right].
%\end{align*}
%We then deduce that
%\[ \res(\mathcal{S}') \simeq \{ (M_{\lambda,\kappa} ,N_{\lambda,\kappa}) : \lambda,\kappa\in \mathbb{F}_p\mbox{ with }\kappa\neq \lambda,\pm2\lambda\},\]
%and it is not hard to show that this is isomorphic to 

\vspace{-2mm}
\subsubsection{Smoothing Loss Landscape}

We illustrate that the small variance of input results in a smooth loss landscape during training. Loss landscape is the characterization of loss functions. For example, smooth loss landscape means that the size of 
the connected region around the minimum where the training loss remains low, showing convex contours in the center, 
while sharp loss landscape shows not convex but chaotic to that region~\cite{DBLP:conf/nips/Li0TSG18}. 

We investigate the smoothness of the loss landscape from both theoretical and experimental perspectives. 
For a classification problem, the loss function is usually cross-entropy loss. We use $y$ to represent 
the ground-truth label, and $\hat{y}$ to represent the prediction of the neural network. In such cases, we explore the relationship 
between $\mathbb{I}$ and loss function. 
\begin{equation}
	\begin{aligned}
		\mathcal{L}(y, \hat{y}) 
		& = CrossEntropyLoss(y, \hat{y}) \\
		& = \sum_{i=1}^{c}y_i \cdot (-log(\hat{y_i})) \\
		% & = -log(y_{true}) \\
		& = -log(max(A \cdot \mathbb{I} + b))
		\label{eq:lossfunction}
	\end{aligned}
\end{equation}
where $c$ represents the number of 
classes, $y_i$ is the one-hot encoding of labels, 
$y_{true}$ is the output value corresponding to the correct label, and $A$ and $b$ are the parameters of the neural network.
For parameter space, a smooth loss landscape means that when the value of trainable parameters of the neural network gradually deviates 
from the optimal value, the loss increases slowly with it. In other words, the first-order partial derivative of 
Equation \ref{eq:lossfunction} with respect to $A$ and $b$ should be a constant or a value with less variation. 
The two partial derivatives are:
\begin{equation}
	\begin{aligned}
		\frac{\partial \mathcal{L}(y, \hat{y})}{\partial A} 
		& = \frac{\partial (-log(max(A \cdot \mathbb{I} + b)))}{\partial A} \\
		& = -\frac{\mathbb{I}}{max(A \cdot \mathbb{I} + b)}
		\label{eq:derivative_A}
	\end{aligned}
\end{equation}
\begin{equation}
	\begin{aligned}
		\frac{\partial \mathcal{L}(y, \hat{y})}{\partial b} 
		& = \frac{\partial (-log(max(A \cdot \mathbb{I} + b)))}{\partial b} \\
		& = -\frac{1}{max(A \cdot \mathbb{I} + b)}
		\label{eq:derivative_b}
	\end{aligned}
\end{equation}

With fixed $A$ and $b$, we discuss the value of Equation \ref{eq:derivative_A} for different training intervals.
If $b = 0$, we get a constant $-\frac{1}{A}$. Obviously, the loss landscape is smooth in this case. If $b \neq 0$, we write the derivative as $-\frac{1}{A + b \cdot (\mathbb{I})^{-1}}$. In our method, the variance of $\mathbb{I}$ is small. 
That is, the value of $\mathbb{I}$ concentrates around a fixed value. Thus, the value of the derivative varies in a small range, and the loss landscape is smoother. For Equation \ref{eq:derivative_b}, $max(A \cdot \mathbb{I} + b)$ is large as it is the value corresponding to the ground-truth label.
Hence, Equation \ref{eq:derivative_b} is close to $0$, and the loss landscape is smooth. 

\begin{figure}
	\centering
	\begin{subfigure}{0.49\linewidth}
		\centering
		\includegraphics[width=\linewidth]{lossLandscapeD0.pdf}
		\caption{$d = 0$}
	\end{subfigure}
	\centering
	\begin{subfigure}{0.49\linewidth}
		\includegraphics[width=\linewidth]{lossLandscapeD025.pdf}
		\caption{$d = 0.025$}
	\end{subfigure}
	\caption{Visualization of DM-small's loss landscape ~\cite{DBLP:conf/iclr/ZhangCXGSLBH20} trained when $d = 0$ and $d = 0.025$. 
	}
	\label{fig:loss_landscape}
\end{figure}

In summary, the small variance of $\mathbb{I}$ results in a smooth loss landscape. 
As an example, we utilize the tool in \cite{DBLP:conf/nips/Li0TSG18} to visualize in Figure \ref{fig:loss_landscape} the loss landscape of two neural networks that are trained with $d = 0$ and $d = 0.025$, respectively. 
Figure \ref{fig:loss_landscape} depicts that the neural network trained with $d = 0.025$ has a broader loss landscape than the one trained with $d=0$, and the loss increases more smoothly in every direction.

\vspace{-3mm}
\subsubsection{The Robustness of Trained Neural Networks}

Training on numerical intervals guarantees that all the perturbed intervals that are mapped to the same numerical interval will have the same classification result. Intuitively, if a numerical interval represents more concrete values, even if the concrete value is slightly disturbed, these disturbed values will be mapped to the same numerical interval with a high probability. Consequently, the classification of the numerical interval according to the neural network still remains unchanged. In this sense, we say that the perturbation is \textit{dissolved} by the abstraction, and therefore the neural network's robustness is improved. 

Theoretically, we show that the smooth loss landscape due to the abstraction in  training contributes to the robustness of neural networks. 
As shown in Figure \ref{fig:loss_landscape}, 
% Figure \ref{fig:loss_landscape} shows that 
the loss of DM-small trained by \textsc{AbsCert} increases slowly and uniformly with the change of parameters, meaning that the growth rate of loss is slow in all directions of parameters' change. However, neural networks trained by conventional methods have a steep slope, which means that there is a direction where the loss increases rapidly. Obviously, smooth loss landscape is helpful for the optimization in training phase to find a global optimum. 

Moreover, in the process of our abstraction, an original pixel is first perturbed to a perturbed interval which is then mapped to a training interval for classification. This indicates that the loss is a constant because all the perturbed images of an image are mapped to a fixed set of training intervals, and the $\mathbb{I}^n$ in Equation \ref{eq:lossfunction} is never changed. 

Neural networks trained by our abstraction-based training method is more robust as the loss landscape is smooth in both parameter space and input space~\cite{DBLP:conf/ijcai/YuQLZWC19}. In particular, for parameter space, the model tends to find a global optimum in a reasonably efficient manner~\cite{DBLP:conf/nips/Li0TSG18, DBLP:conf/nips/WuX020}; for input space, the model is insensitive to the input perturbations ~\cite{DBLP:conf/cvpr/Moosavi-Dezfooli19}.