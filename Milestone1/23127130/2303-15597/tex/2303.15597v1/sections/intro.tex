
\section{Introduction}

Globally, the software industry and overall dependence on software grow faster than engineers and software developers graduate from primary higher education institutions (HEI) to fulfil the need \cite{devdemand}.
The knowledge gap between graduate capabilities and industry expectations is a long-time recognized challenge \cite{brechner2003things, garousi2019aligning, groeneveld2021identifying}. In response, there is a growing trend for companies to onboard graduates into trainee programs. Oguz and Oguz in \cite{oguz2019perspectives} discerned the gap problem from multiple angles: students, recent graduates, academics, and the software industry. They found that students were experiencing a knowledge gap when transitioning to real-life projects in the industry as they differed from school assignments. Preparation of soft skills such as teamwork and communication was mentioned as one factor. The authors analyzed job posts to understand what companies were looking for in new hires. Requirements that were mentioned in all job posts were skills in multiple programming languages and an undergraduate degree in software engineering (SE) --- which focuses on the methodological development of software --- or related fields. Other studies found that employers are more likely to hire candidates that know the company's tech stack \cite{stepanova2021hiring, lauvaas2021oneofus}. %

At the same time, technologies and collaboration methods change and evolve with time, and closing the educational gap is a moving target, making this a complicated problem to solve. Although previous studies attempted to find the gap between SE education and industrial needs \cite{garousi2019aligning,groeneveld2021identifying}, most of them are contextualized to a country and focus on a specific area. While acknowledging that a knowledge gap will always be expected to exist, we in this paper introduce a method and a tool  to assess and understand the existing gap. The collected trends and correlations will allow HE institutions and the software industry to learn and act. This study will contribute to this overarching aim by focusing on the following research question: What are the dominating technological knowledge gaps between higher education and industry? In this paper, we will use data from Sweden to test and validate our proposed tool.

As in all other European countries, reports predict that the number of software developers will need to increase also in Sweden. Figures show that the number of developers has to increase by 37\% between 2020 and 2024 to keep up with industry growth~\cite{swedish-compentence-shortage}. This makes Sweden a good starting point to collect data to be able to validate our proposed tool Job Market AnalyseR (JMAR). JMAR is a tool that imports large datasets of ads and syllabi to perform text analysis to compare Swedish SE education and demand from industry. Similar tools have successfully been used in other subject fields to highlight the research gaps and new emerging topics in the scientific literature \cite{westgate2015text}. A large amount of information can be processed with limited resources \cite{grandia2020assessing}, but previous studies have not made tools available for open use.


