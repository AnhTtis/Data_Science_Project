\section{Related Work}

\subsection{Education gap}
Garousi et al. \cite{garousi2019aligning} conducted a systematic literature review on the knowledge gap between SE education and industry based on 35 studies, where 8 of them assessed the knowledge gap quantitatively. The results were mapped against the knowledge areas defined in SWEBOK \cite{SWEBOK} and classified as small\slash large gaps and whether they are of low or high relevance. Nine out of 15 knowledge areas defined high relevance and large magnitude gaps. When including only reports made in the last five years, the areas showing large gaps and high relevance had grown to 11, indicating a widening gap. There are also quantitative studies focusing on examples of the knowledge gap between HE and software testing specifically~\cite{Cerioli20205million}. To get an up-to-date view of what employers look for when hiring CS graduates, Stepanova et al. conducted a study by sending surveys to recruiters \cite{stepanova2021hiring}. Software developers were highest in demand, and they listed experience, GPA, projects, and skills as the four most crucial areas employers looked for on a resume. The skills category is the only one directly related to the education syllabus, including programming languages and other technical skills. These are examples of studies focusing more on the content of SE education. Oguz and Oguz~\cite{oguz2019perspectives} study also shows that technical skills are important when hiring software engineers. There are also pedagogical methods, including active learning to reduce the knowledge gap \cite{metrolhoaligning2022}. 

General programming knowledge is the most desired technical skill in the Swedish software industry \cite{swedish-compentence-shortage}, but knowing a specific technology used by a hiring company increases the likelihood of getting hired \cite{stepanova2021hiring}. No scientific work has been published in the Swedish context, but some reports and agencies have collected related empirical data. The Swedish government authorities looked at how the skills supply of technical competence could be sustained, short- and long-term. The Swedish Agency for Economic and Regional Growth (SAERG), together with the Swedish Higher Education Authority (SHEA), created a report \cite{digital-spets} that collected and analyzed job posts from the Swedish Public Employment Service to map the demand for different technical competencies. Among other things, it listed data on what software technologies in the Swedish software industry were in demand and trending up to and including 2020. The report concluded that the most in-demand technologies in 2020 were Java, JavaScript, SQL, C\#, and .NET. They also examined how the demand changed between 2017-2020 to highlight the increasing and decreasing popularity. Results on trending technologies for 2020 can be seen in Figure \ref{fig:trend2017-2020}.

\begin{figure}[h!]
\centering
\includegraphics[width=8.8cm]{trend-languages-2020-01.jpg}
\caption{Programming languages trend 2017-2020 \cite{digital-spets}.}
\label{fig:trend2017-2020}
\end{figure}

A Swedish IT \& Telecom Industries report looked at what future competencies would be needed and sent surveys to recruiters and business owners~\cite{swedish-compentence-shortage}. One part of the survey asked what specific programming languages, database technologies, and other digital tools would be in demand for the next 3-5 years. The result can be seen in Figure \ref{fig:predictions2020}.

\begin{figure}[h!]
\centering
\includegraphics[width=8.85cm]{images/tech-predictions-01.png}
\caption{Predictions made by survey respondents 2020 looking 3-5 years into the future  \cite{swedish-compentence-shortage}.}
\label{fig:predictions2020}
\end{figure}

Both reports are from 2020, and although they used different methods, some common conclusions could be drawn. JavaScript, Java, and C\#/.NET were technologies in the top 5 in all diagrams showing that they were in demand during 2020, trending, and predicted to increase in demand.

\begin{figure}[b!]
\centering
\includegraphics[width=\columnwidth]{images/method-overview.png}
\caption{Method overview}
\label{fig:method-overview}
\end{figure}

Similar tools to JMAR have been successfully used in a previous study to highlight the research gaps and new emerging topics in the scientific literature~\cite{westgate2015text}. A definite benefit of such text analysis tools is that a large amount of information can be processed with limited resources \cite{grandia2020assessing}, which makes it possible to be more systematic in understanding the needs of the industry. This will systematically contribute to HEIs ability to make decisions about changes and the industry's ability to prepare, for example, onboarding and how to write job ads. 

  
 
