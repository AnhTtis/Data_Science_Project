\section{Conclusions and Outlook}
We collected job posts and program syllabi from HEIs in Sweden to capture the technological gap between Swedish SE education and the software industry. The results show that Swedish HEIs largely cover the technological skills requested in job posts. Here, C\# defines the exception as the third most sought-after skill, which was explicitly mentioned as part of only two program syllabi in this study. Other technologies, such as TypeScript, Kubernetes, and Docker, saw increased demand but were not covered by many SE programs. Simply analyzing demand and supply does further not say anything about the quality of education in and of itself, and this study limits itself to technology as the single skill variable, while others, such as soft skills, conceptual understanding, or regional demand of competency, have not been included. At the same time, even if the educational program’s primary goal is to teach a broader range of concepts and topics, there are clear benefits to carefully choosing the technology. Knowing the right technology from the start will benefit new hires and the organization - employees will require less time to become productive and be under less pressure in the early stages of their new job.

One key takeaway is that choosing appropriate terms on this interface between HEI and the job market is a prerequisite for a functioning market. This is where HEIs and companies can stand out from the crowd in their and the graduates' best interests. HEIs can clarify their technological stack in other ways than through program syllabi but also by regularly updating their knowledge of the landscape in support of tools such as JMAR. Employers can be explicit and inclusive in how they phrase their demand to ensure that graduates consider a job post, e.g., even writing about JavaScript in the ad that outlines knowledge about TypeScript as a requirement for a job.


Future work may consider using job posts to analyze existing knowledge gaps in other types of SE skills, such as soft skills, or investigate if general programming concepts could be represented better in SE job posts. JMAR could serve as a starting point for further development of the method used in this article, which may even expand to semantic analysis using Natural Language Processing. A more detailed and possibly qualitative follow-up study looking into course plans and or course instances could shed light on the false negative rate, i.e., programs that prepare students with the demanded technology but that do not mention this in their syllabi. This even leads to a valid question of when a technology can be considered \textit{coverer} by an education.