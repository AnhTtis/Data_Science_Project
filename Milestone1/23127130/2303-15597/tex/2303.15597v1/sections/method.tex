\newpage
\section{Research Methodology}
\label{method}

The following research questions are addressed:\\\\
{\textbf{RQ1} What technologies are taught in the SE program syllabi at HEIs?}\\
{\textbf{RQ2} What technologies are requested by the SE industry?}\\
{\textbf{RQ3} What are the strengths and limitations of the method/JMAR?}\\

The research methodology in response to the questions was divided into four steps: tool construction, data collection, defining skills and keywords, and measuring results. Figure \ref{fig:method-overview} shows an overview of the method.

\subsection{Tool Construction}
We created a software artefact called Job Market AnalyseR (JMAR)\footnote{\url{https://github.com/kristian-angelin/JMAR}} to import data and perform text analysis. We explain the four process steps (see Figure \ref{fig:JMAR}) with a dedicated section each.

\begin{figure*}[]
\centering
\includegraphics[width=0.7\textwidth]{images/JMAR.png}
\caption{Visual representation of JMAR functionality.}
\label{fig:JMAR}
\end{figure*}

\subsubsection{Data Import}
We connect to publicly available APIs created by JobTech to access job post data from the Swedish Public Employment Service and Historical ads API~\cite{API}). Secondly, by PDF import from a directory. The data from both approaches get integrated into a standard table scheme to allow for an analysis spanning multiple data sets.

\subsubsection{Text analysis}
A table with skill count got produced using terms from StackOverflow, as mentioned above. The final table got updated considering synonyms and cleared non-technology terms. The text analysis on the data set was performed by searching the text for any occurrences of each keyword specified in a keyword skill table. We used the same method as Saerg and Shea in \cite{digital-spets}, mapping synonymous keywords semantically. Thus, a table was implemented for specifying and mapping keywords and skill pairs. This enabled synonyms to identify a skill, e.g., both keywords HTML and HTML5 in the text should map to the skill HTML.


\subsection{Data collection}

\subsubsection{Program syllabi}
SE program syllabi from HEIs were collected manually since no centralized database or standard structure of their websites existed for easy access. Since Sweden's HE consists of around 50 institutions, each website was visited, and relevant program syllabi were selected based on the following criteria:

\begin{enumerate}
    \item Programs with majors in SE, computer science (CS), or computer engineering (CE)
    \item Removed any programs not a 3-year undergraduate degree program
    \item Removed CE programs containing courses in electronics and hardware topics
\end{enumerate}

The following section describes the reasoning behind the selection criteria. The CC2020\cite{CC2020} lists SE, CS, and CE as the three computing areas with SE capabilities; therefore, programs with majors in those fields were included. It could be argued that only majors within SE should qualify, but it would result in only a handful of programs. In reality, SE and CS graduates often compete for the same jobs in the industry, as well as some CE graduates. Then all programs not being 3-year undergraduate degree programs were removed to align with what Oguz and Oguz pointed out in their study, that all analyzed job posts required at least an undergraduate degree~\cite{oguz2019perspectives}. Programs such as civil engineering education were also removed since they usually last five years. The reasoning was that they could be compared to an undergraduate degree combined with a master's degree. The third criterion was removing any CE programs teaching electronics and hardware-related courses. While SE and CS fields are concerned with software, CE is generally about software and hardware. However, a few CE programs are only focused on software, and they were included in the study. Each HEI website was visited, and all of their IT-related undergraduate programs were manually examined and selected based on the following selection criteria: Firstly, all undergraduate programs related to CS, SE, or CE were selected as mentioned above. Then syllabi and curriculum were analyzed to select only programs that focused on software engineering. Most CS syllabi were included, but few of the CE programs since many of them contained several hardware or electronics-related courses.
Seventeen program syllabi were gathered to represent Swedish SE. A total of \educationHits{} program syllabi were gathered to represent the Swedish SE education (see Table \ref{tab:swe-programs}).

\begin{table}[ht]
\begin{center}
\renewcommand{\arraystretch}{1.5}
\footnotesize
\caption{program syllabi examined in this study}
\begin{tabular}{>{\raggedright}p{30mm} >{\raggedright\arraybackslash}p{40mm} }
\hline
Educational Institution & Educational program, 180 credits\\
\hline
Blekinge Institute of Technology & Software Engineering \\
Blekinge Institute of Technology & Web Programming \\
Jönköping University & Computer Engineering: Software Engineering and Mobile Platforms \\
Karlstad University & Bachelor Programme in Computer Science \\
Kristianstad University & Bachelor Programme in Software Development \\
Linnaeus University & Software Engineering Programme \\
Linnaeus University & Software Technology Programme \\
Linnaeus University & Web Development Programme \\
Malmö University & Computer Systems Developer \\
Mid Sweden University & Computer Science \\
Mid Sweden University & Software Engineering \\
Mälardalen University & Bachelor's programme in computer science \\
Stockholm University & Bachelor's Programme in Computer Science and Software Engineering \\
Umeå University & Bachelor Of Science Programme in Computing Science \\
University of Gothenburg & Software Engineering and Management \\
University of Gävle & Study Programme in Computer Science \\
Uppsala University & Bachelor's program in computer science \\
\hline

\end{tabular}
\label{tab:swe-programs}
\end{center}
\end{table}


\subsubsection{Job posts}

The Swedish Public Employment Service job posting site Platsbanken~\cite{platsbanken} was used as a source for collecting job posts since it is one of the largest in Sweden. It has several publicly available APIs allowing access to databases of currently listed job postings and historical ones \cite{API}. Sources such as LinkedIn\footnote{\url{https://www.linkedin.com/}}, Indeed\footnote{\url{https://se.indeed.com}}, and Carrerjet\footnote{\url{https://www.careerjet.se/}} were considered since they offered many job posts. However, none were selected due to insufficient APIs to extract the data. Since Platsbanken contains job posts written in Swedish and English, words and phrases from both languages were used to get the search results. The used search phrases can be found in Table \ref{Tab:search-words}\footnote{in English: \textit{system developer}, \textit{software developer}, \textit{programmer}}.%

\begin{table}[ht]
\begin{center}
\renewcommand{\arraystretch}{1.5}
\caption{Search phrases.}
\begin{tabular}{ p{35mm} }

\hline
software engineer \\
software developer \\
systemutvecklare \\
mjukvaruutvecklare \\
programmerare \\
\hline

\end{tabular}
\vspace*{2mm}
\label{Tab:search-words}
\end{center}
\end{table}


\begin{figure}[h!]
\centering
\begin{tikzpicture}
\begin{axis} [    
    axis lines = left,
    ybar,
    bar width = 2mm,
    xlabel = \(Year\),
    xmin=2015.5, xmax=2022.5,
    xtick={2016,2017,2018,2019,2020,2021,2022},
    x tick label style={/pgf/number format/1000 sep=,anchor=east,rotate=65},
    minor xtick={2016.5, 2017.5, ..., 2021.5},
    scaled ticks=false,
    ymin=0, ymax=14000,
    width = 80mm,
    ymajorgrids=true,
    grid style=dashed,
    ylabel = {\(Posts\)},
    legend pos=north west,
    ],
    \addplot [draw=none, fill=bardata] 
    coordinates{
        (2016,4429)(2016.5,4601)(2017,5193)(2017.5,5200)(2018,6112)
        (2018.5,6115)(2019,6446)(2019.5,5700)(2020,7558)(2020.5,6537)(2021,10457)(2021.5,12649)
    };
\end{axis}
\end{tikzpicture}
\caption{Number of job posts gathered between 2016-2022}
\label{fig:search-results}
\end{figure}

\subsection{Defining Skills and Keywords}
The Stack Overflow developer survey was used to identify technological skills as a base for skill and keyword pairs. One alternative to use TIOBE language popularity index\footnote{\url{https://www.tiobe.com}} was rejected as it is limited to programming languages only. All technologies in the Most popular technologies section listed in the following categories were included:
\begin{itemize}
    \item Programming, scripting, and markup languages
    \item Databases
    \item Web frameworks
    \item Other frameworks and libraries
    \item Other tools
\end{itemize}
The list was examined, and in the cases where alternative keywords could be identified, they were added to increase the chances of correctly identifying skills. Single-letter technologies such as C and R were excluded from the skill and keyword list since they resulted in many false positives. Go, Chef, Flow, and Julia were also removed since the names gave false positives on common words and names in English and Swedish languages.

\subsection{Evaluation measures}

To answer RQ1 and RQ2, we applied the same method to the respective data sets of program syllabi and job posts. Entries for each dataset were analyzed while skills were added and incremented. A skill ratio over all documents was calculated.

To answer RQ3, we grouped the skill count into six-month intervals between 2016-01-01 and 2021-12-31. This timeline was, after manual validation, model fitted by linear regression with slope $m$ and intercept $b$:
\[ y = mx + b \]

\[ m = \frac{N \sum(xy) - \sum x \sum y}{N \sum(x^2) - (\sum x)^2} \] %

\[ b = \frac{\sum y - m\sum x}{N} \] %
\\
The values were normalized into percentages of the total amount of job posts for the interval. The slope value was used to determine the trend of each skill, showing if it was increasing or decreasing in demand compared to the overall SE job market. A trend of 0 means that the skill was trending equally to the overall SE job market. A Positive trend meant a skill was trending more, and a negative trend was less than the overall job market. The magnitude of the trend quantifies the correlation. RQ3 was answered by analyzing the result from RQ1 and RQ2 to identify any variations in skills covered in HEIs program syllabi compared to industry demands. The balance of supply and demand for a particular skill can be derived from the difference between the percentage of syllabi coverage and job post mentions (0 means a perfect match, and the greater the value, the more significant the gap).
