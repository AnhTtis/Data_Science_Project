\section{Discussion}
We here answer the three research questions in order, followed by an outline of the threats to validity.

\subsection*{\textbf{RQ1} What technologies are taught in the SE program and course syllabi at HEIs?}
We look at the coverage in Figure \ref{fig:skills-education}. Few technologies are covered by more than half the syllabi (only SQL, Java, JavaScript and HTML\slash CSS, Python), suggesting a great diversity in the program's design. Databases, which are part of the computing foundations knowledge area in SWEBOK \cite{SWEBOK}, are highly covered overall, with SQL being covered by all syllabi and NoSQL by almost one in four. Web technologies and predominantly object-oriented languages follow this. Mobile development with Android, iOS\slash Swift gets mentioned in roughly a third of the syllabi. Further, despite its central role in the industry, container technology receives little attention, with 17.3\% for Docker and only a single program with Kubernetes mentions. While among the older web technologies, PHP is still covered in roughly one-third of the syllabi. While Git is the industry's default protocol for version control, it is only covered by 30\% of the syllabi. Our findings highlighted that most programs aim to cover a wide range of concepts and topics to prepare students for industry, but their technological implementations vary across HEIs.

Apart from universities not merely preparing students for the immediate job market needs, one practical aspect that may explain the limited appearance of technologies in program syllabi is that changing them demands a thorough and time-consuming collegiate process in Sweden. Syllabi are, therefore, often formulated such that changes and updates can be made on a course, if not course instance, level. Shorter and practical, also called vocational education, might therefore contain a more heavily technology-focused curriculum, and there the relevance could more easily be matched with industry needs. One idea in this scope could be a more frequently updated, entirely technology-focused document that an HEI makes available per program. Applicants could use this document to strengthen their justification of background knowledge.

\subsection*{\textbf{RQ2} What technologies are requested by the SE industry?}
The report by SAERG and SHEA \cite{digital-spets} showed Java, JavaScript, SQL, C\#, and .NET on top of the demanded technologies during 2018-2020. The technology coverage as of the last (see Figure \ref{fig:skills-jobs}) shows that Java, SQL, and C\# are the currently most requested technical skills in absolute numbers (followed by JavaScript and Python). More than every fourth SE job post asks for Java knowledge. The third most requested skill was SQL, which highlights that (relational) databases are a fundamental part of SE, with SQL as the industry standard. NoSQL does not receive as much attention in the job posts as in the curricula. Container knowledge with Docker and\slash or Kubernetes is independently listed in roughly one out of ten posts. 

As noted in the results section, most technologies saw an increasing trend in absolute numbers. The two technologies that saw the most uptrend during the period were container technologies, Docker, and Kubernetes. The move of many service providers into the cloud and the development of micro-service architectures have supported this movement. Comparing technologies predicted to increase demand by Swedish IT \& Telecom Industries \cite{swedish-compentence-shortage}, the greatest increase in technology skill demand was accounted by JavaScript, C\#/NET, HTML5/CSS, Java, and Android. As the results of this study show, Android is ranked as the 17th most demanded technology, with only 452 job postings, in the second half of 2021, compared to Python, which ranked fifth for the same period with 2307 job posts and with an actual slightly declining trend.

Looking at the SE job posts collected throughout 2016-2021, a staggering increase can be observed in 2021. This may partly be explained by the growth predictions made in the industry report by the Swedish IT \& Telecom Industries \cite{swedish-compentence-shortage}. It can not be ruled out that Covid-19 was a\slash the main growth catalyst putting greater stress on the digital transformation timelines of many if not most, institutions.

We also observe a decline in fundamental and older web technologies like JavaScript, HTML/CSS, and PHP, while modern web technologies such as Node.js are rising (see Figure \ref{fig:tech-trendlines}). In terms of languages, both TypeScript and Python are clearly on the rise, while C\# and C++ see a notable downward trend. The growing popularity of TypeScript happens simultaneously as JavaScript --- which TypeScript is based upon --- declines, which could be explained by a move towards a new norm with better language support. Lower-level concepts are exchanged in ads for higher-level frameworks and tools, even extending to newer continuous integration concepts and \textit{configuration as code} tools such as Terraform. Most of the highest in-demand technologies have all been around for more than 20 years, some much longer, which might affect how widely used the technologies are. %

Analyzing the supply and demand of technical skills can inform program syllabi design for SE students (see Figure \ref{fig:skills-education-vs-industry}). Teaching conceptual knowledge is crucial for preparing graduates for the industry long-term. C\# is the third most requested skill in the industry but is covered in only two SE programs, while Java is the most in-demand skill and is covered by many SE programs. Kubernetes is included in only one SE program despite its increased usage in the industry. Lesser-covered technologies like LISP, F\#, Erlang, and Hadoop are also low in demand. TypeScript is highly in demand but not explicitly mentioned in any SE programs. Overall, students might not be familiar with high-level or technology terms while they understand the concepts, which may complicate the job market matching, in-particular for junior positions.

\subsection*{\textbf{RQ3} What are the strengths and limitations of the method/JMAR?}
The strength of the JMAR tool is that it can be useful for both HE and the software industry since it uses empirical data that describe patterns and trends in HE and the industry.  This could bring a careful technology choice that can lead to smoother onboarding. The data can also be useful in dialogues between HE and the industry. Another strength is that it can be used to reveal trends. 

The JMAR tool also shows some limitations in this version. It does not consider that, likely, neither recruiters nor syllabi responsible at HEI follow the CC2020 classifications \cite{CC2020}. As discussed in the backgrounds section, non-conformity exists regarding terminology in education and the software industry. A commonly observed appearance was that subject area and skill level did not match, i.e., an SE position might, for instance, fall within the embedded systems, requiring knowledge of electronics more related to the CE field. As a consequence, relevant HEI programs could have been missed here. Another weak part of the method is that the keyword analysis does not further capture spelling errors. This is, however, somewhat mitigated in that texts might repeat keywords several times, and it only takes one correct match for a job post to count.

The tool has so far only been validated with data in a Swedish context, and therefore, we cannot claim the generalizability of the results to other demographics. For instance, the high rate of Java being covered may be affected by the demand from state agencies that all heavily build on Java in their technological stack. Further, limiting the objects of study to programs leading up to a BSc degree in SE may lead to excluding relevant MSc or civil engineering education. Still, since MSc and civil engineering educations prepare students also\slash more directly for a research career, and since BSc in SE prepares for a lucrative job market, enabling degrees with sufficiently many educations in Sweden, we decided to reduce the risk of introducing bias by broadening the boundaries too much.

HEIs may or may not use technologies in their education, irrespective of them being mentioned in the syllabi. While this varies, many HEIs focus on concepts such as functional programming (possible technologies LISP, Haskell) or working with containers (Docker, Kubernetes). This further extends to programs where technologies are covered but not explicitly mentioned in the syllabi. Thus, there is a small risk of true negatives in reporting technologies education cover as per this study. Also, Precision and Recall could have been applied to capture modelling errors. However, since terms were chosen from StackOverflow and verified to be relevant after non-ambiguity filtering of some terms (see Section \ref{method} on the method above), Precision can be expected to be high and close to 100\%. It is unlikely that an ad contains a term such as Java with a note that it is an undesired skill. The results may not be of high Recall but are neither meant to generalize to all possible technologies nor to predict\slash classify HEI programs as suitable or not for the job market. This is a negligible concern from the perspective of this study.

