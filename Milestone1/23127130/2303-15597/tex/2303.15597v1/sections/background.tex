\section{Background}

\subsection{The Computing Field and Software Engineering}
The Association for Computing Machinery (ACM) and the Institute of Electrical and Electronics Engineers Computer Society (IEEE-CS) have together created The Computing Curricula 2020 (CC2020)\cite{CC2020}. The report looks to give a clearer overview of the computing field and its knowledge areas and provide curricular guidelines for computing education. CC2020 defines~\cite[p.~28]{CC2020} SE as \say{an engineering discipline that focuses on developing and using rigorous methods for designing and constructing software artefacts that will reliably perform specified tasks}. The focus is on software construction, even though some of its knowledge areas overlap with other computing fields.

According to the Software Engineering Body of Knowledge\cite{SWEBOK} (SWEBOK), SE builds on top of computer science (CS) and has a dedicated knowledge area to cover the related parts called computer foundations. It covers some aspects of CS but not all its parts. The selected CS parts must be directly helpful to SE and software construction. Software Engineering Education Knowledge (SEEK) states that SE draws its foundation from CS and other fields like mathematics, engineering, and project management \cite{SEEK}.

\subsection{Swedish Software Engineering Education}
Undergraduate degrees in Sweden, and some parts of Europe, are obtained by completing three years of full-time studies. This differs from the American system, where an undergraduate degree is usually completed in four years. This is acknowledged in the CC2020 \cite[p.~71]{CC2020} explaining that \say{students do not begin a general degree and subsequently choose a specialization; they enrol from the outset in a specialist degree}. The SEEK also addresses this and includes a model for a three-year curriculum.

Another essential aspect of SE program syllabi at Swedish HEIs is ambiguity about what field a study program belongs to compared to CC2020. Only looking at the program major is insufficient since some programs have more than one field listed \cite{Curriculum-Electronics-CE}, and some list a different field than the program’s actual content \cite{programvaruteknik, Curriculum-datateknik-JU}. For example, a computer engineering (CE) computing major would involve hardware topics such as circuits and electronics, signal processing, or embedded systems. However, there are examples of programs with majors in CE that exclude hardware topics, and examining the content of their syllabi shows that they relate more to the CC2020 definition of SE than CE\cite{programvaruteknik, Curriculum-datateknik-JU}. The differences in Swedish SE education compared to the CC2020 is an exciting topic that is a study of itself and a subject too big to examine in this study thoroughly.

\subsection{Current Industry Software Technologies}
Since 2011 Stack Overflow - one of the most globally influential software developer Q\&A sites with 100 million monthly users \cite{StackOverFlowAbout} - has conducted a developer survey on software technology use with 80 000 developers' responses ( 58 000 by professionals) in 2021. The findings of this survey have, amongst others, been used to map IT industry roles to skills \cite{dada2022hidden} and to analyze directions of programming languages, databases, and developers' job-seeking statuses \cite{beeharry2018analysis}. According to the 2021 survey, Git, Python, SQL, Docker, HTML/CSS, and JavaScript are the most used tools and technologies.
