\section{Conclusion}
\label{sec:conclusion}

This paper presents the shift in the rewards system that ``The Merge" implied for the Ethereum platform. We have unveiled that in the new incentive program that the platform proposes, the consensus layer's rewards represent the 71.3\% of the total rewards, showcasing the importance of having well-behaved and active validators that achieve 61.4\% of their rewards from constant and stable attestations rewards. 

The presented study analyzes the merge of both chains by tracking the missed duties of validators during the process. Overall, the network keeps a healthy behavior that maintains a low ratio of missed blocks and missed attestation flags of 0.72\% and 4.9\%, respectively. We have identified several drops when comparing the obtained CL rewards with the MER during the merge, highlighting some instances where the network miss-performed, achieving 68\% of total possible rewards at some point. 
Moreover, the paper exposes that, currently, the platform faces a high ratio of consecutive blocks being proposed by large staking entities. We introduced that there were even several cases where a third of an epoch was proposed consecutively by a single entity. We have identified the hazard this represents due to these entities' control over which transactions get included and when. 

Ultimately, the paper introduces our rewards indexer tool, which can monitor the obtained rewards over the MER for any set of validators. We have observed that, even though big staking pools show a high commitment to the finalization of the network, they still miss 1,77\% of the rewards generally due to a non-optimal performance on their clients.

As future work to the presented study, we aim to explore the different causes that make validators miss the flags of their attestations and extrapolate the presented rewards study to compare the different existing software.

