\section{State of the Art}
\label{sec:state-of-art}

% The centralization of services in the IT field has become more present in the last decade with the apparition of big corporations such as Google or Amazon. With a wider presence of these ``big players" in many branches of IT services (i.e., cloud computing or cloud storage), the chances of smaller entities to compete with them get widely reduced. Without being willing to dig much into data privacy and the possible hazards that centralization could represent (i.e., content censorship, limitations based on users' location), peer-to-peer networks appeared at the beginning of the 2000s as a disruptive technology. In these networks, users could participate by being service providers, service consumers, or both at the same time, without intermediate entities and fully decentralized. 

Blockchain-based networks adopted the peer-to-peer philosophy to consolidate platforms able to store a public ledger. Generally, these public ledgers use incentive systems to promote honest participation in the consensus. This way, for most users, honest participation in the network is more rewarding than misbehaving. Ethereum also uses the same system to promote the chain's finalization (reach a consensus on which chain to follow).

The unknown nature of block rewards in Ethereum's execution layer has attracted many experiments that tried to modelize it. Authors in \cite{salimitari2017profit} already introduced the apparition of mining pools for the Bitcoin \cite{nakamoto2008bitcoin} network. Pools where individual miners unified efforts to split rewards and costs, making them more stable over time. Other works like \cite{simon2021block} presented an approach that, using linear and polynomial regressions, could predict the mining reward of the next Ethereum block to some degree. More focused on protocol optimizations, other works like \cite{albert2020gasol} introduced an intensive study highlighting the under-optimized storage patterns that most smart contracts use. Furthermore, they propose a tool that helps optimize gas usage when developing smart contracts.  

Showing a clear interest in predicting and maximizing the rewards extracted by each block, authors in \cite{daian2019flash} introduce the lack of fairness that decentralized exchanges and arbitrage bots introduced to the platform by using the usage of sandwich attacks \cite{qin2022quantifying} to front run users transactions. The authors present a novel idea of block-building audition or bid, where they introduce a set of techniques that optimize the reward extracted from each built block (also known as Miner-Extractable Value (MEV)), i.e., transaction ordering.  

The launch of the consensus layer and the shift of Ethereum to PoS, resulted in a way more sustainable network with over 400k validators distributed between 5k nodes in 90 different countries \cite{cortes2021discovering} running mostly five different software that adjusts to all kinds of users' need while increasing the overall resilience of the network ~\cite{cortes2021resource} to software bugs. However, despite the community's efforts to endorse a fully decentralized blockchain platform, authors in \cite{cortes2021discovering} identify a certain degree of centralization in terms of geographical location and used software. 

This paper aims to complete the literature surrounding Ethereum's rewards from the consensus layer perspective. In our study, we present a novel method to evaluate the performance of validators by \updates{comparing their on-chain rewards with the maximum theoretical} one they could obtain, identifying which are the most rewarding duties of the platform. We introduce the results of applying this method to Ethereum's transition from PoW to PoS during the merge. Furthermore, we present the decentralization degree that the current Ethereum network has by analyzing the distribution of validators among staking entities in the network.




