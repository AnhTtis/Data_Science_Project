\section{Discussion}
\label{sec:discussion}
In this section, we introduce the insights of the analysis we performed by indexing the rewards of all the validators active in the network between epochs $139875$ and $153875$, \updates{which are introduced in larger detail in our public dashboard at \cite{pandametrics}.} 
These dates represent precisely one month before the merge and one month after, which helps us understand the impact of the transition to PoS on validators' performance. 

\begin{figure}
    \centering
    \includegraphics[width=\linewidth]{figs/discussion/missed-flags.png}
    \caption{Time distribution of missed flags percentage from the total number of validators. Aggregating the total missed flags on the top chart and distinct missed flags at the bottom.}
    \label{fig:missed-flags}
\end{figure}
%%% Decentralization 
\subsection{Descentralization of the Consensus}
\label{subsec:descentralization}
Combining efforts to support the healthy performance of the network is a key point of the Web3 space. We have seen how during PoW, individuals were uniting efforts to be still competitive while supporting decentralization. PoS is not different in that sense. 
We analyzed the level of decentralization pre- and post-merge to determine if the platform improved or got worst in the aspect of decentralization. For that, we looked into the block proposal distribution for the 10 largest entities: mining pools for the pre-merge (determined by fee recipient and using Etherscan~\cite{etherscan}) and staking pools  and entities' validators for the post-merge.
Figure \ref{fig:staking-descentralization} summarizes that distribution, where overall, we observed a rather similar decentralization level. For instance, pre-merge, we see that Ethermine, F2Pool, and Hiveon Pool proposed $28.5\%$, $14.7\%$, and $10.2\%$ of the blocks, respectively. Post-merge, Lido, Coinbase, and Kraken proposed $29.9\%$, $14.2\%$, and $8.4\%$ of the blocks. Nonetheless, it is essential to keep in mind that Lido consists of 28 operators
% \footnote{Lido is governed by the Lido DAO and provides guidelines to node , such as increasing client diversity and using risk-management techniques; and a policy, such as how operators are expected to use MEV-Boost.} 
that are somewhat independent, which adds another level of decentralization. In addition, we observe notably more small-scale/solo staking (``Other” category) participation ($36\%$) in comparison to small-scale/solo mining ($20.9\%$), likely due to the reduced infrastructure and energy requirements that PoS represents over PoW.
 
%%% General Analysis
\subsection{Validators Overall Performance}
\label{subsec:val-performance}
%% Missed duties
As previously introduced, achieving consensus and the finalization of the chain now depends on the contributions of validators' duties. To rate the performance of validators and ultimately monitor the chain's performance, we looked at the behavior of the overall CL network (CL validators now propose EL blocks). Before making specific comparisons between staking entities, we analyzed the total missed duties of the active validators. 

%% Missed Attestation Flags
\subsubsection{Missed Attestation Flags}
\label{subsubsec:missed-flags}
Regarding the validator's attestations, Figure \ref{fig:missed-flags} displays the distribution of missed flags for the $14k$ measured epochs (note the red vertical line that marks the merge epoch). The upper chart in the figure represents the aggregation of all the missed flags normalized by total active validators per epoch. In contrast, the one at the bottom represents the same normalized distribution but is split between each of the flags present in the attestations.     
\begin{figure*}
    \minipage{0.5\textwidth}%
        \includegraphics[width=0.8\linewidth]{figs/discussion/missed-blocks.png}
        \caption{Number of missed blocks pre- and post-merge.}
        \label{fig:missed-blocks}  
    \endminipage\hfill
    \minipage{0.5\textwidth}%
          \includegraphics[width=0.95\linewidth]{figs/discussion/mer-per-epoch.png}
        \caption{Maximum Extractable Reward for validator network over the $14k$ epochs.}
        \label{fig:mer-ratios}
    \endminipage
\end{figure*}
We observe a slight increase in the total of missed attestation flags, going from a $3.9\%$ pre-merge to a $4.9\%$ post-merge. Considering the different flags included in the attestation, the numbers go from $0.5\%$, $0.5\%$, and $2.9\%$ pre-merge to $0.8\%$, $0.8\%$, and $3.3\%$ post-merge for \emph{Source}, \emph{Target}, and \emph{Head}, respectively.
In the upper graph of Figure \ref{fig:missed-flags}, we can also observe two distinct patterns: 
\begin{enumerate}
    \item A significant spike of $27.17\%$ of missed flags around epoch $145,000$, which we attribute to a large number of restarted clients that were updated to support the merge.
    \item A longer time range of total missed flags between epochs $146,875$ and $148000$ that we attribute to those validators underperforming or missing duties for not having upgraded their software.
\end{enumerate}

Moreover, we can appreciate a large difference in the number of missed flags inside the attestations. In the chart at the bottom of the figure, we can appreciate the validators tend to miss the \emph{Head} flag more often than the \emph{Source} and the \emph{Target} ones. In the chart, we also appreciate that the numbers between the \emph{Source} and the \emph{Target} often match. Meaning that if a peer is missing the \emph{Source} of the attestation (the hash of the justified checkpoint), it is quite likely to miss the \emph{Target} and the \emph{Head} of the attestation. In most cases, missing the three flags in the same attestation can also be interpreted as a ``not attested" sign, which a non-operative client or a non-fully synchronized one might cause. 

%% Missed blocks
\subsubsection{MER}
\label{subsubsec:mer}
The economic incentive promoted by most of the blockchain platforms directly bounds performance and financial rewards. Ethereum is not different in that aspect; validators are rewarded with ETH tokens for their successfully accomplished duties. Figure \ref{fig:mer-ratios} displays the percentage of the MER obtained by the network at each epoch (from attestations and sync committees rewards). We can observe in the figure that the mean for the MER achieved oscillates between $95\%$ and $99\%$, with some sporadic drops that can reach $68\%$ of the MER at some epochs. However, it also means that in each epoch, some validators miss the chance to earn more rewards. As previously introduced in Section \ref{subsubsec:missed-flags}, we attribute these drops to an offline or syncing CL client. 
%% Missed blocks

\subsubsection{Missed Block Proposals}
\label{subsubsec:missed-blocks}
When comparing the number of missed blocks for $7K$ epochs pre-merge to $7K$ epochs post-merge, Figure \ref{fig:missed-blocks} shows an absolute value reduction of $36.40\%$ in missed blocks. We relate this reduction to the fact that missing a block post-merge has a double economic penalty for not getting either CL proposal rewards or EL rewards. Thus, one could expect validators to make every effort to avoid missing block proposals. However, we have to remark that even in the pre-merge scenario, the number of block proposals missed represents $1.13\%$ of the total number of blocks. So the actual $36.40\%$ of reduction refers to a reduction from a $1.13\%$ to a $0.72\%$ ratio of missed blocks.

%%% Rewards Distribution
\subsection{Empirical Distribution of the Rewards in a post-Merge Scenario}
\label{subsec:rewards-empirical-distribution}
From the advantage of indexing all the possible rewards, we have been able to track and compute the different origins for all the validators in the network, and Figure \ref{fig:empirical-rewards-distribution} displays all of them. In the chart, we can appreciate $71.3\%$ of the rewards for a validator are generated through the new consensus layer. Digging a bit more into the CL rewards, the figure showcases that $61.4\%$ of those total rewards are coming attestations, which coincidentally are the most stable rewards a validator can have. Leaving the block proposals in second place with a $7.6\%$ of the total generated rewards, $28.7\%$ for the EL block proposals coming from transaction fees or MEV\footnote{Miner-Extractable Value (MEV) is a new mechanism to compose EL blocks. Block builders compete with each other to build the most rewarding block for the block proposers (maximizing the usage of gas per block, organizing transactions prioritizing tips, or through private transaction channels.)}, and $2.3\%$ for the sync committee rewards.

%%% bp-concatenation-hazards
\subsection{Concatenation of Multiple Block Proposals}
\label{subsec:bp-concatenation}
As long as the beacon chain keeps finalizing, more validators keep joining the network. With a continuous linear increase of $3.8\%$ validators over the two months, the chain went from $428K$ validators at the beginning of the pre-merge period and to $444K$ validators at the end of the post-merge period. 
With this increase in the total number of validators, the chances of being a block proposer decrease. However, the randomness of the RANDAO algorithm still provides some luck to a few validators that can propose multiple blocks in a time range of $2$ months. Table \ref{tab:proposals-per-validator} represents the frequency at which individual validators propose blocks, where we can observe that \updates{$90\%$ of validators proposed either one or no blocks. However, there were still a few lucky proposers that proposed $6$ or $7$ blocks in the same short period. These findings closely align with the fairness rewards study \cite{huang2021rich} performed over different consensus incentive systems, where authors showcase the steady fairness that compounded PoS incentive models, like Ethereum, provide among their participants.}
% table 1- number of consecutive block proposals from validators
\begin{table}
    \caption{Table with the number of validators proposing blocks consecutively. (Pre-merge and Post-merge comparison)}
    \label{tab:proposals-per-validator}
    \begin{center}
    \begin{tabular}{ cccccccccc } 
    \hline
         & 0 & 1 & 2 & 3 & 4 & 5 & 6 & 7 \\
    \hline 
        Pre & 255,983 & 131,192 & 34,145 & 6,088 & 805 & 90 & 5 & 1  \\
        Post & 270,251 & 134,458 & 33,682 & 5,747 & 739 & 62 & 5 & -  \\
    \hline
    \end{tabular}
    \end{center}
\end{table}


We have already introduced the origination of staking entities in the space, accumulating thousands of validators. As we already saw, individual validators have chances of proposing several blocks in a short period. And extrapolating those chances to staking entities' validators, there aren't only a higher number of block proposals in short periods but many consecutive block proposals for the entities. Table \ref{tab:consecutive-blocks} shows the number of consecutive blocks proposed for the three biggest staking entities (starting from the 4th consecutive proposal for space reasons). The table shows that Lido, with $27\%$ of the validators, proposed four consecutive blocks $3686$ times in only two months. Achieving even ten consecutive block proposals (a third of the epoch) on five occasions. \updates{Despite each validator eventually gains the ``same" reward as the rest, the uneven token distribution in the network still allows a certain selected group of individuals or entities to accumulate more validators. As works like \cite{kusmierz2022centralized} present, Ethereum's ETH coin shares a Gini coefficient of $0.499$ among the top 100 token holder addresses, outstanding all compared ERC20 token-based blockchains ($0.685$-$0.907$) and remaining close to the BTC Gini coefficient ($0.466$).}

Staking pools or entities have a legit place in the ecosystem; they allow staking for users that would not participate otherwise. However, since the merge happened, having this larger set of chances to propose consecutive blocks could mean a hazard for the overall network. If a large staking entity had some dishonest interests, it could be benefited by allocating users' transactions not only in the order that they want (i.e., performing a sandwich or similar attacks) but to allocate the transactions in the block they want, giving them a higher time window analyze the public transactions' pool. In the ultimate instance, the only thing that prevents large entities from adopting a dishonest position in the network is their public exposure to the same one, as validators are publicly identified in most cases. 

% table 2- number of consecutive block proposals for entities (pre and post merge)
\begin{table}
    \caption{Table with the number of consecutive blocks proposed by the larger staking entities.}
    \label{tab:consecutive-blocks}
    \begin{center}
    \begin{tabular}{ cccccccccc } 
    \hline
        Operator & Merge-Stage & 4 & 5 & 6 & 7 & 8 & 9 & 10 \\
    \hline 
        \multirow{2}{*}{Lido} 
           & Pre & 1849 & 529 & 169 & 51 & 23 & 2 & 5 \\
           & Post & 1837 & 524 & 168 & 48 & 23 & 2 & 4 \\
    \hline
        \multirow{2}{*}{Coinbase} 
           & Pre & 145 & 18 & 4 & - & - & - & - \\
           & Post & 144 & 17 & 4 & - & - & - & - \\
    \hline
        \multirow{2}{*}{Kraken} 
           & Pre & 20 & 3 & 1 & - & - & - & - \\
           & Post & 20 & 3 & 1 & - & - & - & - \\
    \hline
    \end{tabular}
    \end{center}
\end{table}

\subsection{MER per Staking Entities}
\label{subsec:meer-entities}
We have already analyzed the entire network's performance in previous sections. However, in a scenario where most users want to delegate their stake to others, Figure \ref{fig:mer-per-pool} displays the achieved MER percentage for the eight biggest pools in the network. 
Considering that MER (max attestations and sync committee rewards) represents $61.4\%$ of the total rewards of the two months we analyzed, we observe that some of the largest staking pools do not fully achieve a portion of the MER. From a minimum of $94,2\%$ for Dappnode to a maximum of $99.07\%$ for Houbi, an average loss of $1.77\%$ of the MER shows a high level of commitment from the respective entities. However, the absolute value of ETH tokens that $1.77\%$ represent for those entities is still relevant.

\begin{figure*}
    \minipage{0.5\textwidth}%
        \includegraphics[width=0.92\linewidth]{figs/discussion/empirical-rewards-distribution.png}
        \caption{Empirical distribution of rewards from different sources.}
        \label{fig:empirical-rewards-distribution}
    \endminipage\hfill
    \minipage{0.5\textwidth}%
        \includegraphics[width=0.92\linewidth]{figs/discussion/mer-per-pool.png}
        \caption{Maximum Extractable Reward for the eight biggest staking entities.}
        \label{fig:mer-per-pool}     
    \endminipage
\end{figure*}

