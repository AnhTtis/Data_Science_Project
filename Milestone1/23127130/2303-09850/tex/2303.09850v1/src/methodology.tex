\section{Methodology}
\label{sec:methodology}

Economical remuneration is a highly used incentive to recompense honest behaviors in blockchain scenarios. However, building a sustainable economic rewards system with a solid game theory \cite{liu2019survey} requires considerable effort. This section explains the different reward types that Ethereum miners (previously) and Ethereum validators can earn from their honest interaction with the network, \updates{describing in detail the evolution of rewards displayed in Figure \ref{fig:rewards-evolution}.}

\begin{figure}
    \centering
    \includegraphics[width=\linewidth]{figs/tool/reward-system.png}
    \caption{\updates{Evolution of rewards for different layers in Ethereum.}}
    \label{fig:rewards-evolution}
\end{figure}

\begin{figure}
    \centering
    \includegraphics[width=\linewidth]{figs/tool/tools_rewards_reading_schema.png}
    \caption{Rewards extracting schema followed by the state analyzer tool for Epoch $n-1$ (in red: actions of our indexer tool; in blue, actions at the protocol).}
    \label{fig:epoch-processing-schema}
\end{figure}

%%%%%%%%%%%%%%%%%%%%%%%%%%%%%%%%%%%%%%%%%%%%%%%%%%%%%%
\subsection{Execution Layer's Block Proposal Rewards}
\label{subsec:el-rewards}

As a method of preserving a fair usage of the EVM, interacting with the Ethereum EL requires paying for the needed resources. Following its close analogy to vehicles needing gas to move, users pay Gas for their interaction with the EVM to the network processing it. Thus, each interaction with the EVM requires paying Gas for block space for the user's transactions. With the introduction of EIP 1559~\cite{roughgarden2020transaction} in the \emph{London Hard Fork}\cite{london-hardfork} \updates{(see Figure\ref{fig:rewards-evolution})}, the fees that a user pays to include a transaction in a block are divided into three different values:
\begin{enumerate} 
    \item Gas Limit: Upper limit of Gas that the entire operation may consume, protection against malicious or broken smart contracts.
    \item Base Fee per Gas: Minimum price per gas unit to allocate a transaction in a block. Base Fee dynamically adjusts based on network congestion; the more demand for block space, the higher the Base Fee.
    \item Priority Fee per Gas: colloquially known as the tip users pay to prioritize their transactions.
\end{enumerate}
Leaving the total transaction as the product between the \emph{GasLimit} and \emph{BaseFee+Tip}.

In this new Gas system, block proposers in the execution layer no longer receive 100\% of the Gas paid by the user. Currently, the aggregation of base fees included in each block is burned. As a result, since EL block proposals do not create new ETH in a post-merge scenario, EL block proposers only get the aggregation of the \emph{Gas Tips} included in the block.  


%%%%%%%%%%%%%%%%%%%%%%%%%%%%%%%%%%%%%%%%%%%%%%%%%%%%%%
%% Basic Rewards in Ethereum CL
\subsection{Characterization of Attestation Rewards}
\label{subsec:attestation-rewards}

As part of each validator's duties in the network, once every epoch, all active validators are asked to participate in the consensus system by voting on whether a specific block is correct or not (understanding missing blocks as not correct ones). With this method, the network agrees and validates the data in the chain (finalization). As a result of performing this task, validators earn attestation rewards.

The whole set of validators is divided into beacon committees in each epoch of the beacon chain. However, each committee is assigned one slot per epoch. Thus, each validator is randomly assigned to attest to a single committee at a single slot, having a total of 32 slots to do it.
After each validator has shared its vote with its respective beacon committee, all votes in that committee are then aggregated into a single attestation. If some validators send their vote later, another attestation is created with these votes, so no votes are left behind if they were sent inside a window of 32 slots (one epoch) after the slot they are attesting to. Moreover, the attestation or vote that each validator produces contains three main flags that determine the ``quality" of the attestation:
\begin{itemize}
    \item Source: the hash of the justified checkpoint\footnote{Checkpoints in the CL represent the Beacon State root of the epoch's first slot, including the result of the state transition from the previous epoch.} at the moment the attested block was proposed.  
    \item Target: the hash of the first block at the epoch the attested block was proposed.
    \item Beacon block root: hash of the attested block.
\end{itemize}

After the \emph{Altair Hard Fork}\cite{altair-hardfork}, each flag has a respective weight, determining the final reward per flag. Ultimately, the summary of the product between these three flags and their weights and a \emph{BaseReward}\footnote{Specific value that depends on the validator's effective balance and total active balance at a given epoch.}, represent the attestation reward a validator will receive for its attestation to a given block.

From Ethereum's Consensus specs \cite{eth-cl-specs}, the rewards for each attestation are directly calculated using the following formulas. Please note that for the sake of clarity in the paper, we will summarize some of the aspects of the procedures to the basics (in the code implementation, for a better resolution between calculus, the formulas are calculated per units of the validator's effective balance).
% from code implementation: 
\begin{equation}
    BaseRwrd=\frac{EffBalance * BaseRewardFactor}{\sqrt{TotalActBalance}}
\end{equation}
\begin{equation}
    FlagRwd=\frac{FlagWeight}{WeightDenom}*BaseRwrd*\frac{AttBalance}{TotActBalance}   
\end{equation}
\begin{equation}
    AttestationReward=\sum{FlagRwrd}
\end{equation}
Where: \emph{EffBalance} is the effective balance of the attesting validator, \emph{BaseRewardFactor} is a constant of value $64$, \emph{FlagWeight} and \emph{WeightDenom} are constants that vary between flags, \emph{AttBalance} is the sum of effective balances of all attesting validators, and \emph{TotActBalance} is the sum of the effective balance of all active validators.


%%%%%%%%%%%%%%%%%%%%%%%%%%%%%%%%%%%%%%%%%%%%%%%%%%%%%%%%
\subsection{Characterization of Sync Committees Rewards}
\label{subsec:sync-rewards}

Since the \emph{Altair Hard Fork}~\cite{altair-hardfork}, the Ethereum network has introduced the idea of sync committees, a group of 512 randomly selected validators who sign new block headers every slot. Light clients can use these headers to trust which blocks have been validated without fully downloading and processing the beacon chain.
For a total of 256 epochs (8192 slots), each validator participating in a sync committee receives an extra reward: the sync committee reward. However, unlike attestation rewards, this reward is added at the state transition\footnote{The state transition corresponds to the act of including all the duties in the beacon chain in the parent Beacon State to produce the new Beacon State for the next slot. Includes balance modifications, slashing, etc.} of every slot, not at the epoch processing. After 256 epochs, the list of validators in the sync committee is refreshed.

For every slot that a validator correctly participates in the sync committee (by signing the new block headers and sharing this information with the other participants of the sync committee), its balance will be immediately updated at that slot. However, if the validator does not participate correctly, it will suffer a penalty for missing its duty.
The sync committee rewards per block can be summarized as follows:
\begin{equation}
TotSyncComRwrd=TotBaseReward*\frac{SyncRwrdWeight}{WeightDenom}
\end{equation}
\begin{equation}
    IndvSyncComRwrd=\frac{TotSyncComRwrd}{SlotsPerEpoch*SyncComSize}
\end{equation}

Where: \emph{TotSyncComRwrd} is the sync committee reward for the whole committee per epoch, \emph{TotBaseReward} is the sum of the \emph{BaseReward} of all the active validators, \emph{SyncRwrdWeight} and \emph{WeightDenom} are constants related to the sync committees, \emph{IndvSyncComRwrd} is the individual reward per validator per slot, \emph{SlotsPerEpoch} is a constant of 32 slots per epoch, and \emph{SyncComSize} a constant of 512 validators per sync committee. 

%%%%%%%%%%%%%%%%%%%%%%%%%%%%%%%%%%%%%%%%%%%%%%%%%%%%%%
\subsection{Characterization of Block Proposals}
\label{subsec:block-proposals}

For every epoch, a total of 32 validators are randomly chosen to be block proposers. Each block proposer will have the duty of proposing a block in one of the slots of the epoch. In the opposite direction to the constant and stable attestation rewards, block proposal rewards are way more sporadic due to the randomness of the block proposer election. With a current total of 470k active validators in the network, the chances of being elected as a block proposer are around one every two months. However, the rewards linked to proposing a beacon block are high. Those rewards are coming from:
\begin{itemize}
    \item Including the attestation aggregations from validator votes that are not yet included in the beacon chain.
    \item Including sync aggregates. The block proposer is in charge of including the sync aggregate from the 512 sync committee participants.
\end{itemize}
Similarly to sync committee rewards, proposer rewards are added to the validator’s balance when the slot (and, thus, the block) is processed, being the formulas to calculate the proposer reward per each block the following:
% proposer_reward_numerator += get_base_reward(attesting_index) * flag_weight
% proposer_reward_denominator = (WEIGHT_DENOMINATOR - PROPOSER_WEIGHT) * WEIGHT_DENOMINATOR / PROPOSER_WEIGHT
% proposer_reward = Gwei(proposer_reward_numerator / proposer_reward_denominator)
% increase_balance(state, get_beacon_proposer_index(state), proposer_reward)

% proposer_reward = Gwei(participant_reward * PROPOSER_WEIGHT) # WEIGHT_DENOMINATOR - PROPOSER_WEIGHT
% sync_aggregate = [bitList]
% # Apply participant and proposer rewards
% sync_committee_indices = get_current_sync_commitee_validator_indices()
%  if participation_bit:
%   increase_balance(state, get_beacon_proposer_index(state), proposer_reward)
\begin{equation}
    AttRewards=\frac{\sum{BaseReward*FlagWeight}}{\frac{(WeightDenom-PropWeight)*WeightDenom}{PropWeight}}
\end{equation}
% total_active_increments = get_total_active_balance(state) // EFFECTIVE_BALANCE_INCREMENT
% total_base_rewards = Gwei(get_base_reward_per_increment(state) * total_active_increments)
% max_participant_rewards = Gwei(total_base_rewards * SYNC_REWARD_WEIGHT // WEIGHT_DENOMINATOR // SLOTS_PER_EPOCH)
% participant_reward = Gwei(max_participant_rewards // SYNC_COMMITTEE_SIZE)
% proposer_reward = Gwei(participant_reward * PROPOSER_WEIGHT // (WEIGHT_DENOMINATOR - PROPOSER_WEIGHT))
\begin{equation}
    SyncCRwrds=\sum{\frac{IndvSyncComRwrd*PropWeight}{(WightDenom-PropWeight)}}
\end{equation}
\begin{equation}
    ProposerRewards=AttRewards+SyncCRwrds
\end{equation}
Where: \emph{BaseRewards} corresponds to every validator whose vote was included in the block, \emph{WeightDenom} and \emph{PropWeight} are constants related to the proposer rewards, and \emph{IndvSyncComRwrd} is the individual sync committee reward for each sync committee attestation included in the block.

\updates{
\subsection{Validators' Penalizations}
\label{subsec:penalization}
As an incentive to finalize epochs, Ethereum PoS validators receive rewards for their contributions when they are consistent with 66\% \cite{buterin2017casper} of the total validators (reaching consensus). However, they can also be penalized when they don’t comply with their duties or when their duties don’t match with the majority of validators. Penalties are the only protection mechanism that the platform has to defend the finality of the chain against bad actors or long time disconnections, ensuring that eventually, the network will be able to finalize again because the balance of those actors ends up below the minimum one needed to participate in the consensus.
In terms of quantifying the penalties, the PoS consensus mechanism penalizes validators when they miss-perform critical duties summarized as follows:
\begin{itemize}
    \item When a validator misses attestation flags, only the \emph{Source} and \emph{Target} flags will penalize the validator's balance. The quantity of balance that will be deducted equals the reward it would get for each flag but in negative. This ensures that one day of correct attestations means more rewards than the deducted ones for an inactivity day.
    \item Missing block proposals are bad enough for the validators as they represent a high reward in a single slot. The rest of the network is in charge of attesting for an empty slot. Thus, no extra penalty is applied to the validators that missed a block proposal.   
    \item The network doesn't have much mercy on validators when they miss a sync committee duty. Because they are essential for light clients to verify the validity of the latest epochs. The entire reward that the validator could have had, will be deducted from its balance when sync committee duty is missed. 
\end{itemize}
The previously introduced penalties refer to the duties of a validator that were, according to the consensus, wrongly done or missed. However, there are a set of actions a validator could easily do that could create conflicts inside the network. For this reason, the consensus protocol has a set of rules that, whenever any validator breaks them, it is subject to slashing. The slashing can be triggered when a validator performs two times the same duty, i.e., voting with two different attestations to the same slot, as it can divide the network in different forks depending on which attention was received first. The slashing of a validator presents the exit of the validator from the list of active ones and three concurrent balance modifications:
\begin{itemize}
    \item The penalization of the validator whose balance gets decreased:
    \begin{equation}
    SlashingPenalty=\frac{EffBalance}{32}
    \end{equation}
    \item The reward for the validator that proposed the block that includes the slashing, whose balance increases:
    \begin{equation}
    PropSlashRewrd=\frac{EffBalance*PropWeight}{512*WeightDenom}
    \end{equation}
    \item The reward of the \emph{Whistleblower} that reported the incident, whose balance increases: 
    \begin{equation}
    WhistleBReward=\frac{EffBalance}{512}-PropSlashRewrd
    \end{equation}
\end{itemize}
}
%%%%%%%%%%%%%%%%%%%%%%%%%%%%%%%%%%%%%%%%%%%%%%%%%%%%%%
\subsection{Maximum Extractable Reward (MER)}
\label{subsec:mer}

The presented formulas in the above sections describe Ethereum's current reward system at the time that we are writing this paper (post-Paris Hard Fork~\cite{paris-hardfork}). As a reminder, for a healthy performance of the beacon chain (keep finalizing states), only 66\% of the validators need to actively participate in the consensus on new blocks. This means that even though the network achieves resilience to validator churn\footnote{Validators that go offline for a period of time and come back afterward (due to client updates, internet shortage, among others).} the network is still able to finalize states in the chain. 
We understand as participating validators, the ones that are actively attesting in their committees; however, in the last instance, all the rewards that attesting validators, sync committee members, and block proposers receive also depend on the quality of their duties (i.e., an attestation might have 1/3 flags correct and still be considered as participation to the consensus).

This said, and as part of our contribution to this paper, we identified that we could theoretically qualify and compare the max reward each of the validators could achieve at every given epoch with the one they actually achieved. From now on, we will refer to this theoretical max reward as Maximum Extractable Reward (MER).  


%%%%%%%%%%%%%%%%%%%%%%%%%%%%%%%%%%%%%%%%%%%%%%%%%%%%%%
% tool
\subsection{Beacon State Analyzer}
\label{subsec:bstate-analyzer}

As a method to index the status of each validator in the network and the data included in the beacon chain, we developed an open source tool~\cite{state-analyzer} that can index the individual duties per validator per epoch, the quality of these duties (i.e., if validators missed a block proposal, the number of flags that were successfully voted, etc.), and the max attestation and sync committee (if the validator during that epoch was inside a sync committee) reward that each validator got in each epoch.

\updates{The tool can index in a PostgreSQL database metrics for any existing slot or epoch the indicated range, and it does it by requesting the last Beacon State\footnote{The Beacon State is the main data structure that represents the beacon chain’s status at every slot, including the entire list of validators, their balances, and duties.} of each epoch in the indicated time range from a trusted synchronized Beacon Node}. In the previous sections, we already introduced that there are two ways of applying rewards to the validators: during the state transition at the end of each slot (block proposal and sync committee rewards) or during the epoch transition that defines the first state of the next epoch (the state transition in the last slot of the epoch, where attestation rewards are applied). 

Since attestations can be included 31 slots after the block they are referring to, the tool processes the rewards following the schema in Figure \ref{fig:epoch-processing-schema}. The tool chronologically analyzes the states, tracking the proposer and sync committee rewards applied during epoch $n-1$ at the last epoch slot (slot $31$ in the figure). Once we have already distinguished the rewards, we process the next state, which is the state of the last slot in epoch $n$ (slot $63$ in the figure). From the second state that we analyze until the end of the given time range, we can index attestation rewards along the sync committees and block proposals. In the last slot of epoch $n$, we can already measure which are the duties that were accomplished in relation to epoch $n-1$ and which ones were not, which let us rate how well each validator performed and which would be its max reward for that particular attestation. Although it might be a bit misleading, these attestation rewards we can already compute won't appear in the validator's balance until the epoch transition is applied.

%%%%%%%%%%%%%%%%%%%%%%%%%%%%%%%%%%%%%%%%%%%%%%%%%%%%%%
% Eth CL Pools
\subsection{Validator to Staking Pools Mapping}
\label{subsec:staking-pools}
We have already introduced the steps and requirements to become a validator in Ethereum's PoS. However, collecting 32 ETH (for optimal performance) and running the required software 24-7 at home is not possible for every user. Either the lack of funds or technical knowledge to set up and maintain the validator online creates a large opportunity for delegated staking and the generation of staking pools. As it originally was for PoW with mining pools, the concept of staking pools consists of aggregating multiple users' funds until you can activate a validator. This type of service is generally driven by entities or organizations that are in charge of activating the validator, setting up the clients, and maintaining it.

In the Ethereum ecosystem, there are many entities. There are centralized organizations such as Coinbase, Kraken, or Binance, and decentralized entities like Lido Finance or Rocket Pool, which allow users to delegate their staking. In order to differentiate the validator performance of the wide set of staking pools, we mapped the whole set of validators to their respective staking pool (only for those ones linked to a staking pool) based on the transaction that activated the validator using the PoS smart contract. Since activating the validator in the Beaconchain requires interacting with smart contracts from a \emph{deposit address}, we inspected the whole list of PoS deposits, mapping (if possible) the \emph{deposit addresses} to any of the Ethereum addresses publicly identified from the entities.

\begin{figure}[hb!]
    \centering
    \includegraphics[width=1.05\linewidth]{figs/discussion/descentralization-per-and-post-merge.png}
    \caption{Block proposal's distribution per mining and staking entities for the two months of study.}
    \label{fig:staking-descentralization}
\end{figure}