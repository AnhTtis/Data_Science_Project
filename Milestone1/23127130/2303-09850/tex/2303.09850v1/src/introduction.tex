\section{Introduction}
\label{sec:introduction}

Ethereum~\cite{eth-whitepaper} has soundly consolidated in the Web3 and blockchain ecosystem for its widely decentralized network with more than $5.000$ active nodes~\cite{cortes2021discovering}, and its multipurpose Ethereum Virtual Machine (EVM)~\cite{hildenbrandt2018kevm}. Commonly popularized for being the pioneer blockchain able to process \emph{Smart Contracts}~\cite{wohrer2018smart}, it currently handles above $1$ million transactions per day from $600$ thousand active accounts~\cite{oliveira2022analysis}. As with any other blockchain in the Web3 space, to support this decentralized platform, Ethereum relies on an incentive program to reward honest behaviors between the participants. 

Starting its journey as an operative network in 2015, Ethereum chose Proof of Work (PoW) as its consensus mechanism. In this PoW-powered platform, participating nodes (commonly known as miners) competed with each other to become the next block proposer. Nodes in the network built block candidates by adding user transactions from the public \emph{Mem-pool}. Awarded with a generous tip in Ethereum's token Ether (ETH) for the first one publishing a valid block, miners would parallelly iterate over multiple nonce values\footnote{Nonce values in PoW-based blockchains is a field in the head of the block structure that miners can indistinctly modify to satisfy the hash target requirements of the chain.} until the hash value of the entire block satisfies the target required by the network.

Initially, as part of the incentives that the block proposal had until EIP1559~\cite{roughgarden2020transaction} was applied, block proposers were also granted with the fees coming from processing the transactions included in the block plus a dynamically adjusting base reward. 
However, as long as the network has proved to be resilient, powering the consensus through PoW required huge and unnecessary power consumption. As one of the biggest upgrades in the platform's lifetime, Ethereum embraced a long journey to shift its consensus mechanism from PoW to a more sustainable Proof of Stake (PoS) \cite{saleh2021blockchain}. The transition started with the launch of a side chain with the single purpose of achieving consensus. To participate in the Beacon Chain or Consensus Layer (CL), users now have to deposit in a publicly available smart contract 32 ETH to activate a validator. The beacon chain decides who is the next block proposer (among many other things) using a randomness generator algorithm called RANDAO~\cite{randao-code}. The beacon chain is organized in epochs, where each epoch contains 32 slots of 12 seconds, making a whole epoch last 6.4 minutes. Each slot will serve as a time range where the RNG algorithm decides which validator can propose the block. Removing the competition between nodes to become block proposers reduced by 99.95\% the total power consumption of the network, with the significant drawback of adding extra complexity to reaching consensus. 

The beacon chain stores all the information regarding the status of the validators, from when they are activated, to their interaction with the network when performing their duties, including their balance. To reach a consensus and maintain the finalization of the chain, these validators now have to perform a set of tasks at every epoch of the chain. Also known as duties, validators are grouped by committees, which have the major task of voting on the correctness of the proposed block (if proposed) from the slot they are assigned to. These votes are also known as attestations and are included in the beacon blocks. Validators are also grouped by sync committee, a small set of 512 validators that help sign headers of blocks to prove their validity to light clients that won't need to sync the entire chain. 
Based on the correctness of all these duties, validators get rewarded as an incentive to well-behave in the consensus, making honest actions in the network more economically lucrative than dishonest ones. 

% The merge
Both parallel chains had co-existed since December 2020, when the beacon chain was launched, until September 2022, when the chain in charge of processing transactions became consensus-dependent from the beacon chain. In this event called ``The Merge"\cite{paris-hardfork}, active validators in the consensus layer will also be the ones proposing blocks in the EVM chain (from now on called Execution Layer (EL)). With the successfully accomplished transition from PoW to PoS, there are some significant changes in the new reward mechanism of the multipurpose platform. Since validators now propose both blocks, they get the CL and the EL rewards. However, from the EL rewards, validators only earn the part coming from the transaction fees, which means that there is no base reward anymore. The fact that block proposers are chosen beforehand and the mining process has been removed increases the time to include transactions in a block. Which ultimately increases the chances of a proposer gaining more rewards
by filtering and ordering the transactions. 

%Since Validators now know in advance that they will be the ones proposing the next block, they have a small time window to include the transactions they want, with the final intention of increasing the total income that they perceive from the EL. And ultimately, this completely changes the reward system in the platform. 

This paper presents a previously unseen approach to measure the transition from PoW to PoS in a live-working blockchain network. Our approach uses both empirical and theoretical methods to compare and score how well validators accomplish their duties. Moreover, we present a comparison in performance based on rewards for entities that represent a large set of validators in the network.

The paper is organized as follows: Section \ref{sec:state-of-art} introduces the state of the art of the paper, going through previously done work on the topic, Section \ref{sec:methodology} introduces all the rewards-related terms, theoretical methods, and the methodology that we followed, Section \ref{sec:discussion} presents the discussion over the results we gathered from the study, and Section \ref{sec:conclusion} summarizes all the highlights of the study. 