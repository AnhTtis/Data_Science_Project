\documentclass[aps,pra,twocolumn,showpacs,floatfix,superscriptaddress,nofootinbib]{revtex4-2}


\usepackage{graphicx}%
\usepackage{dcolumn}%
\usepackage{bm}%
\usepackage{physics}
%\usepackage{hyperref}%
\usepackage{color}
\usepackage{amsmath}
\usepackage{amssymb}
\usepackage[normalem]{ulem}
\usepackage{multirow}
\usepackage{booktabs}
\usepackage{blkarray}


% please change to your favorite color
%
\newcommand{\ts}[1]{{\color{cyan}{{#1}}}}
\newcommand{\mc}[1]{{\color{red}{{#1}}}}
\newcommand{\nw}[1]{{\color{red}{{#1}}}}
\newcommand{\phb}[1]{{\color{red}{{#1}}}}
\newcommand{\ml}[1]{{\color{red}{{#1}}}}
\newcommand{\jz}[1]{{\color{orange}{{#1}}}}
\newcommand{\as}[1]{{\color{red}{{#1}}}}

\newcommand{\old}[1]{\textcolor{blue}{\sout{#1}}}

%SF Math defs
\newcommand{\figref}[1]{\figurename~\ref{#1}}
\newcommand{\secref}[1]{Sec.~\ref{#1}}
\renewcommand{\eqref}[1]{Eq.~(\ref{#1})}
\newcommand{\Schr}{Schr\"odinger }
\renewcommand{\ip}{I_\mathrm{p}}
\newcommand{\up}{U_\mathrm{p}}
\renewcommand{\pb}{\mathbf{p}}
\newcommand{\xb}{\mathbf{x}}
\newcommand{\kb}{\mathbf{k}}
\renewcommand{\dd}{\mathrm{d}}
\newcommand{\rb}{\mathbf{r}}
\newcommand{\rbh}{\hat{\mathbf{r}}}
\newcommand{\Ab}{\mathbf{A}}
\newcommand{\Eb}{\mathbf{E}}
\renewcommand{\grad}{\mathbf{\nabla}}
\renewcommand\Re{\mathrm{Re}}
\renewcommand\Im{\mathrm{Im}}
\renewcommand{\erf}{\;\mathrm{erf}}
\newcommand{\xhat}{\hat{\bf x}}
\newcommand{\yhat}{\hat{\bf y}}
\newcommand*\conj[1]{\overline{#1}}

\newcommand{\Qprop}{Q{\sc{prop}}}
%\usepackage{lineno}
%\linenumbers

\begin{document}
	
\title{Femtosecond pulse parameter estimation from photoelectron momenta using machine learning}

\author{Tomasz Szołdra}
\email{tomasz.szoldra@doctoral.uj.edu.pl}
\affiliation{Doctoral School of Exact and Natural Sciences, Jagiellonian University, \L{}ojasiewicza 11, PL-30-348 Krak\'ow, Poland}
\affiliation{Instytut Fizyki Teoretycznej, Wydzia\l{} Fizyki, Astronomii i Informatyki Stosowanej, Uniwersytet Jagiello\'nski, \L{}ojasiewicza 11, PL-30-348 Krak\'ow, Poland}

\author{Marcelo F. Ciappina}
\affiliation{Department of Physics, Guangdong Technion - Israel Institute of Technology, 241 Daxue Road, Shantou, Guangdong, China, 515063}
\affiliation{Technion - Israel Institute of Technology, Haifa, 32000, Israel}
\affiliation{Guangdong Provincial Key Laboratory of Materials and Technologies for Energy Conversion, Guangdong Technion - Israel Institute of Technology, 241 Daxue Road, Shantou, Guangdong, China, 515063}
\author{Nicholas Werby}
\affiliation{Stanford PULSE Institute, SLAC National Accelerator Laboratory 2575 Sand Hill Road, Menlo Park, CA 94025, USA}
\affiliation{Department of Physics, Stanford University, Stanford, CA 94305, USA}
\author{Philip H. Bucksbaum}
\affiliation{Stanford PULSE Institute, SLAC National Accelerator Laboratory 2575 Sand Hill Road, Menlo Park, CA 94025, USA}
\affiliation{Department of Physics, Stanford University, Stanford, CA 94305, USA}
\affiliation{Department of Applied Physics, Stanford University, Stanford, CA 94305, USA}
\author{Maciej Lewenstein}
\affiliation{ICFO-Institut de Ciencies Fotoniques, The Barcelona Institute of Science and Technology, Av. Carl Friedrich Gauss 3, 08860 Castelldefels (Barcelona), Spain}
\affiliation{ICREA, Pg. Lluís Companys 23, 08010 Barcelona, Spain}
\author{Jakub Zakrzewski}
\affiliation{Instytut Fizyki Teoretycznej, Wydzia\l{} Fizyki, Astronomii i Informatyki Stosowanej, Uniwersytet Jagiello\'nski, \L{}ojasiewicza 11, PL-30-348 Krak\'ow, Poland}
\affiliation{Mark Kac Complex Systems Research Center, Jagiellonian University, \L{}ojasiewicza 11, PL-30-348 Krak\'ow, Poland}
\author{Andrew S. Maxwell}
\affiliation{Department of Physics and Astronomy, Aarhus University, DK-8000 Aarhus C, Denmark}
\date{\today}


\begin{abstract}
Deep learning models have provided huge interpretation power for image-like data. Specifically, convolutional neural networks (CNNs) have demonstrated incredible acuity for tasks such as feature extraction or parameter estimation. Here we test CNNs on strong-field ionization photoelectron spectra, training on theoretical data sets to `invert' experimental data. Pulse characterization is used as a `testing ground', specifically we retrieve the laser intensity, where `traditional' measurements typically leads to 20\% uncertainty.  We report on crucial data augmentation techniques required to successfully train on theoretical data and return consistent results from experiments, including accounting for detector saturation. The same procedure can be repeated to apply CNNs in a range of scenarios for strong-field ionization. Using a predictive uncertainty estimation, reliable laser intensity uncertainties of a few percent can be extracted, which are consistently lower than those given by traditional techniques. Using interpretability methods can reveal parts of the distribution that are most sensitive to laser intensity, which can be directly associated to holographic interferences. The CNNs employed provide an accurate and convenient ways to extract parameters, and represent a novel interpretational tool for strong-field ionization spectra.
\end{abstract}

\maketitle

\section{Introduction}
Machine learning (ML) has been transformative for science over the last two decades, providing a huge range of new analytical tools. This has affected nearly every avenue of research, with major use across the physical sciences, particularly in particle physics \cite{karagiorgi_machine_2022}, astrophysics \cite{vanderplas_introduction_2012}, and condensed matter physics \cite{carleo_machine_2019, dawid_modern_2022}. Convolutional neural networks (CNNs) have enabled major leaps in computer vision and language processing \cite{rawat_deep_2017, li_survey_2022}. This makes CNNs well-suited for pattern recognition and parameter estimation in scientific data.
For example, CNNs have been used for determining crystal symmetries in electron diffraction \cite{kaufmann_crystal_2020}, and estimating parameters related to gravitational lensing \cite{hezaveh_fast_2017}. However, in the field of strong field physics and attoscience the high interpretability power of CNNs has not been fully explored.

Strong-field physics and attoscience exploit recent advances for producing intense and short laser pulses to image and control matter over attosecond ($10^{-18}$s) timescales \cite{krausz_attosecond_2009, salieres_study_1999, lewenstein_principles_2008, ciappina_attosecond_2017}. These capabilities have led to a wide range of atomic and molecular imaging procedures, e.g., high-order harmonic spectroscopy \cite{itatani_tomographic_2004}, laser-induced electron diffraction \cite{zuo_laserinduced_1996}, photoelectron holography \cite{huismans_timeresolved_2011,figueirademorissonfaria_it_2020}, attosecond streaking \cite{hentschel_attosecond_2001,itatani_attosecond_2002}, and reconstruction of attosecond harmonic beating by interference of two-photon transitions \cite{paul_observation_2001,muller_reconstruction_2002}. However, despite ever more accuracy from experiment and theory, due to the nonlinear nature of the interactions, interpretation of the data is often very challenging. This provides an opportunity for ML methods to be used to extract parameters and physical trends from experimental data sets.

A growing number of studies have begun to use ML techniques for strong-field physics. For example, studies using neural networks to classify semi-classical trajectories \cite{liu_deep_2020}, deep learning to predict in spectra of high-harmonic generation \cite{lytova_deep_2022}, and optimization of ``quantum pathways'' in enhanced ionization of diatomic molecules \cite{chomet_controlling_2022}. In terms of parameter estimation, in a recent theoretical study, CNNs were used to extract internuclear distances and laser intensities, using data generated solving the time-dependent \Schr equation (TDSE) \cite{shvetsov-shilovski_deep_2022}. The power of CNNs to extract useful information from  experimental images has big implications for strong-field physics and attoscience.
Most studies, however, focus on a proof of principle, using only theoretical data. Notable exceptions are, Ref.~\cite{liu_machine_2021}, where CNNs were used to extract molecular structure parameters from experimental laser-induced electron diffraction images, and Ref.~\cite{brunner_deep_2022}, where deep neural networks were applied to streaking traces for parameter extraction and prediction of uncertainties.
Unfortunately, the analytical power of machine learning-assisted imaging is limited if the laser pulse parameters can not be accurately measured.

The characterization of laser pulses in strong-field and attosecond physics has posed a persistent problem. The high intensity of the laser pulse means that a direct measurement (see e.g. \cite{trebino_measuring_1997,kielpinski_benchmarking_2014}) of the intensity leads to significant uncertainties, in the range of $10$--$20\%$ for the strong-field regime \cite{pullen_measurement_2013,kielpinski_benchmarking_2014}. An alternative approach is to use the high sensitivity of the non-linear phenomena in question, to estimate the laser pulse parameters, known as an in situ measurement. Using this approach, laser intensity uncertainties as low as $1\%$ have been reached, by matching experimental results to TDSE theory, under highly controlled experimental conditions for atomic hydrogen \cite{pullen_measurement_2013,kielpinski_benchmarking_2014}. Despite this success, such low uncertainties are not common, and would be more difficult to achieve routinely in standard experimental conditions. Simply fitting photoelectron spectra is less effective and more advanced methods, using all the available information in photoelectron momentum distributions (PMDs), are called for. Recent results, using quantum metrology tools, suggest the uncertainty from in situ measurements could be significantly reduced by exploring quantum interferences present in the PMDs in strong-field ionization \cite{maxwell_quantum_2021}. There are a variety of in situ methods for determining laser parameters that are implemented by hand, whose performances vary across parameter regimes.
The most consistent and powerful method is to use the whole PMD, matching it to that obtained with accurate theoretical methods. As such, ML schemes, and in particular CNNs, are an ideal tool for in situ extraction of laser parameters from experimental data.


The task of using ML for laser pulse characterization has been addressed in the relativistic regime where, a theoretical study used CNNs to predict laser intensities by using proton dynamics \cite{bukharskii_restoration_2023}. In FELs, neural networks have been used to accurately reconstruct pulses, by training a model on a small set of fully diagnosed pulses \cite{sanchez-gonzalez_accurate_2017}. CNNs were also used to characterize the FEL pulses by training on simulated data \cite{ren_temporal_2020}. Recently, ML tools have been used for pulse characterization for strong-field ionization \cite{geffert_situ_2022} using purely theoretical data. Here, the autocorrelation function of the ionization yield from two identical pulse was used to extract the pulse duration, spectral width and relative CEP, but the method was insensitive to laser intensity.

In this work, we investigate the power of a CNN as an analysis tool for strong-field physics. We use laser pulse characterization in strong-field ionization as a testing ground, retrieving parameters from PMDs. We train the CNNs on a TDSE model, and test this on large experimental datasets, focusing on retrieving laser intensities, over a larger parameter regime than has previously been considered, for an argon target. The CNNs trained may be used on any experimental data within the parameter range, without special requirements, and the CNN models are available online for testing. Important modifications to theoretical training data, are presented, that ensure the CNN models are insensitive to common experimental imperfections. We also include predictive uncertainty estimation, which goes beyond previous methods to extract uncertainty in strong-field studies. 
We produce so-called `explainability' figures that are able to highlight regions of the PMD that contribute the most to a laser intensity prediction, and connect these to the physical interference process that are most sensitive to changes in laser intensity. As such, CNNs represent the easiest way to extract laser parameters in strong-field ionization, making a key step towards producing a general tool for parameters estimation, while also providing more interpretational power. Atomic units are used unless otherwise stated.


\section{Datasets}
\label{sec:datasets}
    \begin{figure*}
        \centering
        \includegraphics[width=\linewidth]{data_samples3_inferno_histogram.pdf}
        \caption{Example PMDs (upper halves in the first row and lower halves in the second row) in the initially preprocessed a) \Qprop, b) SFA ($N=40$ cycles and $U_p=0.3575$) and f) experimental E (single) ($U_p=0.35$) datasets. Color scale corresponds to rescaled and shifted log-probability density and includes the leading 6 orders of magnitude of the original calculated/measured PMD. Panel c) shows the focal-averaged \Qprop\ PMD - minor features are smeared out, panel d) the focal- and CEP-averaged \Qprop\ PMD that does not show qualitative differences from c). Panel e) shows data from d) as seen during training of the model, i.e. at a random detector saturation level $SL=0.4$ (see text for details), contrast $0.8$, brightness $-0.2$. Histograms of pixel brightnesses in panel g) are calculated over a range of pulse ponderomotive potentials $0.15 < U_p < 0.5$. Clearly, direct comparison between theoretical and experimental PMDs is a hard task, as the distributions differ substantially due to experimental limitations.
        }
        \label{fig:ExampleDataSets}
    \end{figure*}
    \subsection{QProp}
        The main workhorse for generating our theoretical data set is the single-active-electron (SAE) TDSE solver \Qprop. 
        The latest version of \Qprop~\cite{tulsky_qprop_2019}, implements a fast and accurate method for the calculation of photoelectron momentum distributions (PMDs). \Qprop\ is a velocity gauge 3-dimensional (3D) TDSE solver in the dipole approximation that allows studies within the SAE  approximation using model pseudopotentials\footnote{There is also an implementation of many-electron systems via the solution of the time-dependent Kohn–Sham equations.}. For our SAE model of the Ar atom, we have employed the model potential of Ref.~\cite{tong_empirical_2005}, which has the form
	
	\begin{equation}
	    V(r)=-\frac{Z+f(r)}{r}
	\end{equation}
    with $f(r)=a_1 e^{-a_2 r}+a_3 r e^{-a_4 r} + a_5 e^{-a_6 r}$,
	where ${Z=1}$. For argon the coefficients $a_i$ are $a_1=16.039$, $a_2=2.007$, $a_3=-25.543$, $a_4=4.525$, $a_5=0.961$ and $a_6=0.443$~\cite{tong_empirical_2005}, which gives the correct ionization potential of $\ip=0.579$~a.u. In this computation, we considered angular momenta up to $l=55$. Total data set consists of $15712$ PMDs with ponderomotive potentials ranging from $\up=0.0075$ to $\up=0.95$, laser cycles numbers $N=2$ up to $N=61$. Number of CEP values depends on the pulse length, i.e. for under $N=13$ cycles we cover interval of length $\pi$ with $10$ values of CEP, and for longer cycles we have $5$ values of CEP spanning a shifted interval of the same length.
 
    \subsection{Strong Field Approximation}
     An extensive review of the SFA can be found in Ref~\cite{amini_symphony_2019}. Here, we use the transition amplitude for direct ATI from an initial bound state $|\psi_0\rangle$ to a final Volkov state with drift momentum $\pb$ given by \cite{becker_abovethreshold_2002,figueirademorissonfaria_highorder_2002,keldysh_ionization_1965,faisal_multiple_1973,reiss_effect_1980,maxwell_quantum_2021}
    \begin{equation}
    M(\pb)=-i\lim\limits_{t\to\infty}  e^{i S(\pb,t)} \int_{-\infty}^{t}dt' d(\pb,t') e^{i S(\pb,t')},
    \label{eq:SFA-transition-general}
    \end{equation}
    where $d(\pb,t')=\left\langle \pb+\Ab(t') \middle| \rb\cdot\Eb(t') \middle| \Psi_{0}\right\rangle$ is the dipole prefactor, which in this study we will neglect as we retain only terms correct to exponential accuracy.  The action is given by
    \begin{equation}
        S(\mathbf{p},t)=\ip t +\frac{1}{2}\int^t d t' (\pb+\Ab(t'))^2.
        \label{eq:SFA-action-general}
    \end{equation}
    Here, $\ip$ is the ionization potential of our target.
    We employ the saddle point approximation, seeking the stationary action for the integration variable $t'$, ${2\ip+\left(\pb+\Ab(t_s)\right)^2=0.}$ Now the probability distribution can be computed from \eqref{eq:SFA-transition-general} as
    \begin{align}
        M(\pb)=&
        \sum_s 
        c(\pb,t_s,t)	d(\pb,t_s) e^{i S(\pb,t_s)},
        \intertext{where the prefactor $c(\pb,t_s, t)$, derived from the application of the saddle point approximation, includes the $t'$ independent phase from \eqref{eq:SFA-transition-general},  is given by}
        c(\pb,t_s,t)&=-i e^{i S(\pb,t)}\sqrt{\frac{2\pi i}{\partial^2 S(\pb,t_s)/\partial t_s^2}}.
    \end{align}
    In both \Qprop\ and SFA calculations we use the vector potential:
    \begin{equation}
        \Ab(t)=2\sqrt{\up}\sin^2\left( \frac{\omega t}{2 N}\right)\cos(\omega t + \phi),
    \end{equation}
    where $N$ is the number of laser cycles, while $\up$ is the ponderomotive energy or quiver energy of a free electron in the laser field, which is proportional to the peak laser intensity $I_0=2\up c \epsilon_0 \omega^2$, where $\epsilon_0$ and $c$ are the vacuum permittivity and the speed of light, respectively. The angular frequency is given by $\omega$ and the carrier-envelope phase (CEP) is given by $\phi$. We also perform focal (FA) and CEP averaging (CA), to account for variations of the intensity across the focal volume and CEP fluctuations between laser pulses, respectively. 
    \subsection{Experimental methods}
    \label{sec:experimental_method}

    Our experimental data sets consist of PMDs of argon gas photoionized by intense, linearly polarized, 800~nm laser pulses generated by a 1~kHz commercial Ti:sapphire laser system. These laser pulses are focused in ultra-high vacuum and intersect a pulsed beam of argon gas delivered by an Even-Lavie valve \cite{even_even-lavie_2015}. Photoelectrons are collected in a velocity map imaging spectrometer (VMI), and impact a microchannel plate (MCP) and a fast phosphor detector system. A camera records the intensified phosphor, and on-the-fly peak finding is applied to the live camera feed to extract individual electron impacts.

    Six experimental data sets with $125$ PMDs in total are analyzed here. Four of these sets, labeled E1-E4, were collected using an experimental schema in which the ionizing pulse energy was the variable parameter. The pulse energy is controlled using a motorized rotation mount which manipulates a half-wave plate to rotate the pulse polarization. It then passes through a polarizing beamsplitter cube, which transmits only the component of the laser pulse polarized parallel to the optical table. 
    Another data set, “E HS”, was collected for intensities over a larger range and include the highest values of intensity.
    The final set, labeled E (single), is a single-intensity PMD, which is included as it has the highest signal-to-noise ratio.
    Each dataset was collected on the timescale of approximately two days, and contain in total $O(10^9)$ electron counts, distributed between their intensity slices. 

    The data sets presented have all been Abel inverted using the standard technique of polar onion peeling \cite{Roberts_2009, werby_dissecting_2021} to extract their cylindrical momentum cross sections. The ponderomotive potential, $U_p$, of the laser fields generating each slice of each data set is computed in a two-step process. First, it is roughly calculated by examining the direct electrons, which form a disk of radius $2U_p$. Then, that rough value is refined by comparing nodes found along the  ``spider-leg" holographic feature to those predicted by the Coulomb quantum orbit strong-field approximation, see Refs. \cite{maxwell_coulombcorrected_2017,maxwell_analytic_2018,maxwell_coulombfree_2018,figueirademorissonfaria_it_2020}, at nearby intensities. This procedure is described in more detail in Ref. \cite{werby_dissecting_2021}. This has an error of approximately $\pm 10\%$. 


\section{Deep neural network approach} 
	Our task is the following: given an experimental PMD $X$, find a physical parameter $y$, for example, the intensity of the laser pulse, that has been used to produce such PMD. We assume an underlying theoretical model of the strong-field process that allows us to generate PMDs expected at given physical parameters. The richness of the features present in the PMDs makes the task very challenging. Moreover, various imperfections are present in the experimental data, complicating the comparison with theory further, cf. histograms in Fig.~\ref{fig:ExampleDataSets}(g) showing a dramatic quantitative difference between the PMD values in both datasets.
	
    To address this demanding work, we then use a deep learning approach. By generating a dataset of PMDs labeled by many different physical parameters, we reformulate the problem as a standard supervised regression task. In Section~\ref{sec:cnn}, we describe our choice of deep neural network architecture. 


    \subsection{Convolutional Neural Networks and Transfer Learning}
    \label{sec:cnn}
    CNNs \cite{lecun_backpropagation_1989} are designed to work with data incorporating spatial correlations such as pictures. The main building block of CNN is a convolution matrix with trainable parameters that slides over the input image and produces its filtered representation. The composition of many such filters gives a feature map of the image, ready to be used for further processing in the final fully connected part of the network.
    
    Although deep CNNs achieve state-of-the-art in image recognition \cite{krizhevsky_imagenet_2012, russakovsky_imagenet_2014, chen_symbolic_2023}, this requires the use of prohibitively large training datasets and computing power. However, one can train deep models through the transfer learning paradigm \cite{tan_survey_2018, plested_deep_2022}: one takes a pre-trained deep model and fine-tunes it on a smaller dataset. The first few layers of the network extract general features such as edges, so often retraining only the last few layers of the model may be sufficient to achieve good performance on the new dataset. In this work we benchmark four pre-trained architectures called VGG16 \cite{simonyan_very_2015}, Xception \cite{chollet_xception_2017}, EfficientNetB7 \cite{tan_efficientnet_2020}, and EfficientNetV2L \cite{tan_efficientnetv2_2021} that achieved state-of-the-art accuracy in classification tasks on the Imagenet dataset \cite{russakovsky_imagenet_2014}. Models are ordered from least to most sophisticated. We do not describe the substantial innovations introduced in each of them but concentrate on the comparison of their general performance. Because the models were originally designed for classification and not regression, we remove the last classification layer, and replace it with a fully connected layer with a linear (identity) activation function, see Fig.~\ref{fig:schematic}. Models are implemented using Keras library \cite{chollet_keras_2015}. 
    
    \begin{figure}
             \centering
             \includegraphics[width=\linewidth]{scheme_v2.pdf}
             \caption{Schematic representation of the Deep Convolutional Neural Network regression problem. For given input $X$ network predicts the value of the parameter $\mu(X)$ and its uncertainty $\sigma(X)$. Adapted from \cite{iqbal_plotneuralnet_2018}.}
             \label{fig:schematic}
         \end{figure}
         
    \subsection{Predictive uncertainty estimation}
    \label{sec:Uncertainty}
    Along with the predicted label, we aim to provide an estimate of the model uncertainty for a given input \cite{gawlikowski_survey_2022, abdar_review_2021}. We slightly modify the model and the loss function \cite{nix_estimating_1994, lakshminarayanan_simple_2017}: instead of predicting a single value of the label $y_{pred}(X)$, we assume the label comes from a normal distribution $p(y_{true}|X) = \mathcal{N}(\mu(X), \sigma(X))$ and the model predicts its parameters $\mu(X), \sigma^2(X)$ for a given input $X$. $\sigma^2(X)$ is the output of an additional fully connected layer with a Softplus$(x) = \ln(\exp(x) + 1)$ activation function, see Fig.~\ref{fig:schematic}. The loss function to minimize is the negative log-likelihood, 
    \begin{equation}
    	\text{NLL}\!=\! - \!\left\langle \ln p(y_{true}|X)\right\rangle\!=\!\frac{1}{2}\!\left\langle\! \ln\sigma^2(X) \!+\!\frac{(y_{true}\!-\!\mu(X))^2}{\sigma^2(X)}\!\right\rangle,
     \label{eq:NLL}
    \end{equation}
    where $\langle . \rangle$ denotes the mean over the training dataset. To further increase the reliability of predictions and predictive uncertainties, we combine $M$ models trained on random subsets of the training dataset into a deep ensemble \cite{lakshminarayanan_simple_2017, ovadia_can_2019}. Further assuming that the ensemble prediction is a Gaussian, the ensemble mean and variance read
    \begin{equation}
        \mu_*(X) = \frac{1}{M} \sum_{m=1}^M \mu_{m}(X),
    \end{equation}
    \begin{equation}
        \sigma^2_{*}(X) = \frac{1}{M} \sum_{m=1}^M \left( \sigma^2_{m}(X) + \mu^2_{m}(X) \right) -  \mu^2_{*}(X),
        \label{eq:sigma}
    \end{equation}
    where $\mu_m(X),~\sigma_m(X)$ are the mean and variance predicted by the $m$-th model in the ensemble \cite{lakshminarayanan_simple_2017}. This goes beyond studies such as \cite{brunner_deep_2022}, where an ensemble of predictions is used to produce the uncertainty.

    
        \begin{table*}
\centering
\footnotesize
\begin{tabular}{l|rrrr|rrrr|rrrr|rrrr}
\toprule
 & \multicolumn{4}{c}{VGG16 @ SL=0.0} & \multicolumn{4}{c}{Xception @ SL=-0.5} & \multicolumn{4}{c}{EfficientNetB7 @ SL=0.0} & \multicolumn{4}{c}{EfficientNetV2L @ SL=-0.5} \\
 Dataset& NLL & RMSE & $\langle\sigma_*\rangle$ & MAPE & NLL & RMSE & $\langle\sigma_*\rangle$ & MAPE & NLL & RMSE & $\langle\sigma_*\rangle$ & MAPE & NLL & RMSE & $\langle\sigma_*\rangle$ & MAPE \\\midrule
train & -4.2 & 0.0042 & 0.018 & \textbf{0.77} & -2.0 & 0.11 & 0.12 & 14. & -4.4 & 0.0098 & 0.0097 & 1.3 & -4.1 & 0.015 & 0.012 & 1.7 \\
test & -4.1 & 0.0051 & 0.019 & \textbf{0.87} & -1.9 & 0.12 & 0.12 & 14. & -4.4 & 0.011 & 0.010 & 1.4 & -4.1 & 0.016 & 0.013 & 1.9 \\
E1 & -2.2 & 0.039 & 0.020 & 15. & -1.9 & 0.083 & 0.11 & 33. & 1.7 & 0.038 & 0.010 & \textbf{15}. & -0.99 & 0.040 & 0.016 & 17. \\
E2 & -0.93 & 0.046 & 0.020 & 15. & -2.5 & 0.0099 & 0.083 & \textbf{3.1} & 5.9 & 0.032 & 0.0077 & 12. & -0.70 & 0.034 & 0.012 & 12. \\
E3 & 0.61 & 0.12 & 0.050 & 29. & -2.5 & 0.035 & 0.077 & \textbf{7.2} & -0.89 & 0.037 & 0.021 & 8.5 & -2.3 & 0.043 & 0.029 & 9.8 \\
E4 & -3.4 & 0.021 & 0.015 & 6.2 & -2.2 & 0.061 & 0.089 & 22. & -3.4 & 0.014 & 0.0075 & 4.0 & -4.1 & 0.0091 & 0.014 & \textbf{2.9} \\
E (single) & -3.8 & 0.010 & 0.020 & 2.9 & -2.4 & 0.0072 & 0.087 & 2.0 & -3.1 & 0.015 & 0.0083 & 4.3 & -4.3 & 0.00067 & 0.014 & \textbf{0.19} \\
E HS & 0.70 & 0.11 & 0.041 & 19. & -1.9 & 0.11 & 0.11 & 15. & -2.1 & 0.035 & 0.023 & 4.5 & -3.1 & 0.025 & 0.025 & \textbf{4.5} \\\bottomrule
\end{tabular}


    
    \caption{Loss metrics for the training dataset QProp+FA+CA. For each CNN architecture, we show only saturation level $SL$ used in the image augmentation for which NLL on dataset E4 is the lowest. On E4, the best accuracy is obtained for EfficientNetV2L which simultaneously gives acceptable errors for other experimental data. Error metrics are similar on test/train subsets of QProp+FA+CA (first two rows) which is a signature of good model generalization. Lowest values of MAPE for each dataset are in bold.}
    \label{tab:models}
\end{table*}

    \subsection{Data preprocessing and augmentation}
    \label{sec:preprocessing}
    In this section, we describe a few technical steps that were necessary to preprocess the PMDs to form a viable image input for the CNNs.
    As a first step, we identified a common range of momenta accessible in all datasets to form a rectangle with $p_\perp \in [0.001 \mathrm{~a.u.}, 1.15 \mathrm{~a.u.}]$, $p_{||}\in [-1.15\mathrm{~a.u.}, 1.15\mathrm{~a.u.}]$, where the discretization is given by the resolution of the experimental dataset E1 $\Delta p_\perp=\Delta p_{||}=0.0049$~a.u. Then, we interpolated the theoretical data on the same 2-dimensional momentum grid. 
    
	
	The natural scale for features contained in the PMD is logarithmic and the signal has to be transformed accordingly, see eg. \cite{zimmermann_deep_2019} for CNNs applied to diffraction images with similar properties. 
    Thus, we take the logarithm of the PMDs, rescale and apply an offset to the pixel values so that in the end they fill the interval $[-1, 1]$. Thus, the pixel value $1$ represents the largest peak probability density in each PMD and $-1$ is the value smaller by a factor of $10^{-6}$. Each pixel is clipped according to $X_{ij} \rightarrow \max (-1, X_{ij})$. Six orders of magnitude are selected based on heuristics that will capture all features in the experimental data, at the same time not misguiding the network by showing extremely precise, low values in the theoretical data. All images are then resized to $224$ by $224$ pixels with $3$ (repeated) color channels to match the standard of the Imagenet dataset expected by the pre-trained models. 
	
	We split the theoretical datasets, see Section \ref{sec:datasets}, into training ($80\%$), validation ($10\%$), and test ($10\%$) subsets. For each model in the ensemble, the training/validation split is different and random, while the test dataset is constructed once by a random selection from the full dataset.
	
	
    During the training phase, we perform image augmentation
	\cite{shorten_survey_2019}. Each input image is randomly reflected in the up-down and left-right axes, and its contrast and brightness are randomly set from the interval $(0.1, 1.0)$ and $(-1, 1)$, respectively, using the builtin Keras \cite{chollet_keras_2015} functionalities, see Fig.~\ref{fig:ExampleDataSets}e). The final image is clipped to a fixed range $[-1, 1]$. While the testing data does not include reflected images, during training we apply reflections to help the network learn the same  ``shapes" in four different settings with the aim of reducing overfitting. On the other hand, the testing data has inherently varying levels of contrast and background signal (``brightness") and we deliberately make the network insensitive to them. 


    In our efforts to make the models useful for experimentalists, we encountered an obstacle: while the models performed well on theoretical data (see next Section \ref{sec:results}), they failed for experiments. We fixed it by adding a single extra step in the augmentation pipeline, motivated by the histogram in Fig.~\ref{fig:ExampleDataSets}g), that shows significant differences between distributions of pixel values in theory and experiment, especially for the brightest pixels. This suggests there was some uncontrolled detector saturation effect in the experiment, at a level not necessarily fixed between experiments. Thus, prior to all augmentations described earlier, we simulate a random detector saturation level. For each sample PMD, we draw a random  ``saturation" value $x$ from the interval $[SL, 1]$ and transform each pixel according to $X_{ij} \rightarrow \min(X_{i,j}, x) + 1 - x$. The lower bound on the random saturation value $SL$ is a parameter that has to be found by checking the performance of the model on part of the experimental data. By making the saturation level random, we increase the training difficulty, but at the same time make the model insensitive to detector saturation that occurs in a real experiment. \footnote{Since pixel brightness values are limited to the range $[-1,1]$, $SL=-1$ corresponds to a fully random saturation level, whereas $SL=1$ to no detector saturation effect present at all (case of  ``perfect detector").}
	

\subsection{Training}
	\label{sec:training}
    We train an ensemble of size $M=5$ of pretrained models VGG16, Xception, EfficientNetB7, EfficientNetV2L on four datasets: QProp, QProp+CA, QProp+FA, QProp+CA+FA, for a set of random detector saturation level lower bounds ${SL\in \lbrace -1, -0.5, 0, 0.5, 1 \rbrace}$, yielding $400$ models in total. We choose the Adam optimizer~\cite{kingma_adam_2014} and a batch size of $32$. During the first $50$ training epochs, the base model has fixed weights and only the added, two randomly initialized dense layers each with $1$-dimensional output $\mu(X)$ and $\sigma(X)$ are being updated at a learning rate of $10^{-3}$. This roughly sets up the last layer while not destroying pretrained filters. In the next $150$ iterations the model is fine-tuned: all weights are updated at a learning rate $10^{-4}$, which decreases by a factor of $0.5$ every $50$ iterations. We stop the training if the loss on the evaluation dataset does not decrease for more than $100$ epochs to save on computing time. All models can be trained in parallel. 
    \footnote{The training effectiveness may be improved by a recent training scheme \cite{sluijterman_optimal_2023} not applied here: in the first few iterations one should optimize the mean, while keeping the variance fixed.}

    We were unable to train any models if they were initialized with random weights. It demonstrates the power of transfer learning from real-world images to physical experiments. The models are already capable of extracting basic shapes from images and need fine-tuning only.
    

\section{Results}
	\label{sec:results}
    The quality of all $400$ trained models is measured in terms of the NLL (see \eqref{eq:NLL}), root mean squared error (RMSE) and the mean absolute percentage error (MAPE) achieved on the test datasets. For theoretical data sets, the `true' intensity is known exactly, so RMSE and MAPE give the error on model prediction. For experimental data sets, the `true' intensity carries 10\% uncertainty, so the MAPE only needs to be within this bound.    
    The mean predictive uncertainty $\sigma_*(X)$ is the models' prediction of the uncertainty. This can be compared with the RMSE to assess if the models  ``know when they're wrong". We give a general overview of these results and describe a method of model post-selection that allows us to find the best model candidates for experimental data presented in Table~\ref{tab:models}. 
    
    As a standard practice, no augmentation techniques are applied in the testing phase unless explicitly noted. While this can decrease performance of some models on theoretical testing data due to input distribution shifts, we concentrate more on the performance on experimental data which naturally includes imperfections. 
    In the “perfect detector” augmentation scenario, $SL=1$, all models are trainable on all theoretical \Qprop\ datasets with a testing MAPE$~<1\%$ (except for the VGG16 model and QProp+CA dataset where MAPE$~=4.9\%$). 
    
    Testing on experimental data, we notice that including focal averaging and CEP-averaging in the training results in a smaller error. This is supported by a visual inspection of the PMDs in Figs.~\ref{fig:ExampleDataSets}a), d), f), which unveils greater resemblance between experimental and QProp+FA+CA rather than QProp data with more small-sized features. Thus, we restrict our further discussion to the QProp+FA+CA training dataset. Moreover, we observe that the quality of the prediction is always improved if detector saturation effects are included ($SL \leq 0.5$) than if they are not ($SL=1$).  


    \begin{figure}[t]
        \centering
            \includegraphics[width=\columnwidth]{fig3_BayesianEfficientNetV2L_SL-0d5_1column_TWcm2.pdf}
            \caption{Performance of the EfficientNetV2L model for training dataset QProp+FA+CA, saturation level $SL=-0.5$, for different test datasets. Top plot shows the value of $U_p$ predicted by the model as a function of the true value. Perfect predictions would lie on the black diagonal line. Including focal- and CEP-averaging in the training dataset was necessary to achieve results in agreement with the experimental value, up to the estimated experimental uncertainty level of $10\%$, as shown in the middle plot where most experimental points lie below $10\%$ absolute error line. Bottom plot shows a measure of the model confidence, standard deviation $\sigma_*(U_p)$, as percentage of the true $U_p$ value.}
            \label{fig:pred_vs_true}
        \end{figure}
    




   Aiming for the highest-quality models for experiments, we post-select the saturation level based on the best NLL on a single evaluation dataset E4. 
   Obtained metrics are presented in Table~\ref{tab:models}. We find that the fittest model is the most sophisticated EfficientNetV2L trained with a saturation level $SL=-0.5$, reaching a MAPE of $2.9\%$ on E4, which is well below the experimental error. The predictive uncertainty is $\sigma_*(X)=0.014$~a.u., corresponding to $6\times 10^{12}$~W/cm$^2$ in typical intensity units. This is close to the reported RMSE, signaling a good calibration of the model confidence. Almost all errors on other datasets for this model also fall within the error bars of the experimental label. In Fig.~\ref{fig:pred_vs_true}, we plot the predicted value of ponderomotive energy $\up$, proportional to the laser intensity, as a function of the true $\up$, for experimental and theoretical (\Qprop\ and SFA) inputs. 
   All predictions are presented with uncertainties computed using \eqref{eq:sigma}, given by the vertical error bars in Fig.~\ref{fig:pred_vs_true} (upper panel) and as percentage (lower panel).


   Predictions on the test/train QProp+FA+CA dataset in Fig.~\ref{fig:pred_vs_true} lie within the uncertainty estimate around the true value up to $U_p\approx 0.5$. For larger $U_p$ the model slightly deviates, finishing with an error of around $7\%$ at $U_p=0.95$. We expect this drop in performance is associated with a limited range of momenta present in the training dataset. The $p=2\sqrt{\up}$ peak, used for labeling PMDs manually, is located at the border of accessible momenta at around $\up\sim 0.66$. Thus, some other, possibly less expressed features in the PMD have to be used by the CNN.

   Before we proceed to experimental data, we quickly cross-check the output of the model on the SFA dataset. Looking at the strong qualitative difference between sample images in Figs.~\ref{fig:ExampleDataSets} a) and b), not to mention the dramatic dissimilarity of the histograms of both datasets in Fig.~\ref{fig:ExampleDataSets}g), it is surprising that the model finds any structure at all in the SFA+FA+CA data, i.e., there is significant positive correlation coefficient between the true value and the prediction. 
   
   However, the uncertainty of SFA+FA+CA is strongly underestimated, particularly for the largest values. This is a warning that on the strongly out-of-distribution data\footnote{Out-of-distribution data describes data that is far from the kind of data that was used for training, where the model is likely to give unpredictable results.}, the model may fail to predict the true value and report a relatively high confidence.


   Testing the same model on experimental datasets we notice that errors generally stay equal or lower than the experimental uncertainty of $10\%$ (middle panel). Note that the model was selected based on its performance on evaluation dataset E4, yet its predictions are consistent for other datasets. 
   Overall, there is good agreement with the vast majority of intensity prediction carrying an uncertainty below that of attainable through traditional methods, getting a low as 4\% in a number of cases.
   In particular, the high-statistic dataset E HS gives a very good agreement over a wide range of intensities. It was crucial to consider focal averaging here, otherwise the “true” labels deviated from the predicted value, with an absolute percentage error up to $\sim50\%$ for large intensities.
   Notably, a large relative error is observed for a few points of the dataset E1 at low intensities. We believe this is primarily caused by a low contrast in the input image due to a lower number of electron counts in this setting, since this deviation can be manually removed by increasing the contrast of the input images.

    \section{Explanations}
    \label{sec:explainability}
    
    \subsection{Methods}

    High accuracy of the models presented above makes them a readily useful tool for parameter estimation. On the other hand, due to a rather complex flow of information in image regression, the understanding of why a certain output is produced is lacking, i.e., we deal with a  ``black box”. This hinders the progress in the development of new, more accurate models, and, more importantly, does not give any insight into the underlying physical reasoning. These issues are addressed in the following section using so-called explainability techniques that quantify how certain features of the input contribute to the output, see recent review \cite{linardatos_explainable_2020}, or \cite{mohseni_multidisciplinary_2021} for a more general survey.

    The most popular explainability techniques were designed for classifiers, and special care needs to be taken when using them for regression, see \cite{letzgus_toward_2021}. Here we adopt the simple yet powerful strategy of Ref.~\cite{zhang_explainability_2020}, where explanations of deep regression models are obtained directly using methods for classification. 

    Three basic approaches to explainable regression have been developed to date and a variety of algorithms can be found in each category \cite{letzgus_toward_2021}. The most straightforward are removal-based explanations \cite{covert_explaining_2021}, measuring the importance of a given subset of input features by hiding it from the model. Because there is an exponential number of subsets to check, usually these methods are limited to at most 15-20 features before the analysis of images becomes infeasible. Another set of methods are gradient-based explanations that rely on the computation of the gradient of the input in a single forward/backward propagation of the signal. They are built on the intuition that if some region of the image is important for the prediction, a small change in this region will noticeably change the output. Finally, propagation-based explanations aim to leverage the neural network structure to produce the feature attribution map. In particular, the layer-wise relevance propagation (LRP) algorithm \cite{bach_pixel-wise_2015} assigns a relevance score $R_i$ to each neuron $i$ based on the activations of the neurons in the next layer. Scores in a single layer are conserved, i.e., sum up to the final prediction. The relevance scores are calculated layer by layer in the backward pass from the output, until the input is reached.

    We apply 10 different explainability algorithms available in the iNNvestigate toolbox \cite{alber_innvestigate_2019}. Our analysis is restricted to the VGG16 model at $SL=0.0$ instead of the  ``best performing” EfficientNetV2L at $SL=-0.5$, presented in Fig.~\ref{fig:pred_vs_true}, due to the large size of the latter, making most methods intractable due to memory requirements, and a  ``swish” activation function which is not compatible with many algorithms. Out of all tested algorithms, for presentation we post-select four most relevant ones by scoring them following \cite{samek_evaluating_2017, zhang_explainability_2020}. Each explanation image is divided into $8$ by $8$ regions and perturbed one region at a time, from most to least important, according to the output of a given explanation algorithm. If the regions marked by the algorithm are indeed relevant for predictions, accuracy of the model drops faster than when the perturbations are applied in a random order. In our case, 9 out of 10 tested methods perform better than a random one, and the four presented in Fig.~\ref{fig:explanations} are noticeably better than the rest.

    
    \subsection{What the model learns.}
    \begin{figure}
        \centering
            \includegraphics[width=1.0\linewidth]{expl_QProp_FA_CA_Esingle_5.pdf}
            \caption{Explanations for the VGG16 model, obtained with 4 most informative methods (upper labels), from left (best) to right (worst). For QProp+FA+CA, true $\up=0.3535$ and predicted $\up=0.3550$, for experiment true $\up=0.351$ and predicted $\up=0.3344$. Red/white/blue colors correspond to positive/neutral/negative attribution in each explainability method.
}
            \label{fig:explanations}
    \end{figure}

    In Fig.~\ref{fig:explanations}, we used four explainability techniques: two variants of the LRP \cite{bach_pixel-wise_2015}, Guided Backpropagation \cite{springenberg_striving_2015}, and DeepTaylor \cite{montavon_explaining_2017} to highlight regions on the PMD that have the most effect on the predicted value. This can also be interpreted as highlighting the features that are the most sensitive to changes in the ponderomotive energy, and thus it is a unique way to extract physical meaning. The highlighted regions can be identified as known interference features that are used in photoelectron holography. The guided backpropagation technique picks out two regions. The first region is at the end of the `legs' of the so-called spider like structure \cite{huismans_timeresolved_2011,hickstein_direct_2012}, see green dashed rectangles in Fig.~\ref{fig:explanations}. This is formed via the interference between pairs of electronic wavepackets that are forward scattered/deflected off by the residual ion and have differing degree of interaction with the core.

    The explainability diagrams specifically pick out modulations along the spider legs above the direct boundary ($p=2\sqrt{\up}$). These modulations were already discussed in \cite{werby_dissecting_2021}. 
    Crucially, these modulations have been shown previously \cite{maxwell_coulombcorrected_2017} to be described approximately by circles with their centers determined by $\up$. Thus, explaining the sensitivity to the ponderomotive energy.

    Another region highlighted, with high perpendicular momentum, is the so-called carpet-like \cite{korneev_interference_2012, kang_holographic_2020} or spiral-like \cite{maxwell_spirallike_2020} structure, see blue dotted lines in  Fig.~\ref{fig:explanations}. Around $p_{||}=0$, the interference maxima can be described by $\ip+\up+E=2n\omega$ (with $n\in \mathbb{Z}$), which clearly encodes the ponderomotive energy. Away from $p_{||}=0$, in the DeepTaylor method, we see the strongest contribution. Here, the above equation will not hold exactly, but the fringes will be dependent on the interplay of two rescattered wavepackets, that undergo different rescattering angles, which will be heavily dependent on the laser intensity/ponderomotive energy, as it determines the tunnel exit and initial scattering velocity.
    
    Across many regions, and particularly in the DeepTaylor method, we can see the above-threshold ionization rings, which are ring-shaped interferences due to nearly identical wavepackets released at an integer number of laser cycles apart, see black dot-dashed circles in Fig.~\ref{fig:explanations}. The maxima may be described by a similar equation to the carpet-like structure $\ip+\up+E=n\omega$, which is clearly sensitive to the ponderomotive energy. In previous work, this has been shown to provide important sensitivity for determining laser intensity \cite{maxwell_quantum_2021}. 
    
    
    
    \section{Conclusions}
    We have proposed and tested deep learning models, powerful enough to detect objects in real world images, as a versatile analysis tool for strong-field ionization processes, adopting deep CNNs as our main workhorse. We have found they are capable of extracting physical parameters of interest---the laser peak intensity---and connect the parameter back to a specific feature, such as the interferences observed in the PMDs. 
    We have overcome key difficulties using theoretically-trained CNNs with experimental data, which paves the way for CNNs to be used in a variety of settings for strong-field ionization, particularly when characterizing the laser field or target. The CNNs have been tested via pulse characterization, in particular determining the laser field intensity, on a large experimental data set, consistently yielding lower uncertainties than are achievable in traditional methods. For the prediction of uncertainty, we have used a reliable predictive uncertainty approach, that provides additional evaluation of the experimental conditions. Deep CNNs can utilize information present in the picture to its full extent, while a human expert would typically be limited to using only a small subset of physical effects that are most sensitive to a change in parameter. As such, we have developed a novel tool that can be applied to strong-field ionization photoelectron momentum spectra without any special requirements from the experimental data. We have also verified that other laser field parameters such as the pulse length could easily be extracted. Beyond this we show-cased the “explainability” capability of CNNs, which highlighted the most relevant features in the PMDs, which could be associated directly to holographic interferences that display considerable sensitivity to changes in the ponderomotive energy. Thus, directly connecting the CNNs predictions to fundamental physical processes.

    This study paves the way for further exploitation of CNNs to analyze strong-field ionization data, yielding new physical insights or confirming existing understanding. The recipe we developed, training the neural networks to be insensitive to various types of imperfections through the use of data augmentation techniques, made them ideal candidates for robust parameter extraction. We emphasize that the same procedure can be repeated and used to develop a range of analysis tools, which, for instance, could be highly useful for extracting atomic targets and/or pulse shapes, or further developing photoelectron holographic imaging, where inversion of experimental data is very difficult. Using these techniques, universal extraction of physical parameters is possible from existing and future experimental data, regardless of whether all details of the physical processes at play are fully understood.
    
    
\acknowledgments 
We gratefully acknowledge Poland’s high-performance computing infrastructure PLGrid (HPC Centers: ACK Cyfronet AGH) for providing computer facilities and support within computational grant no. PLG/2022/015830. T.S. is supported by National Science Centre (Poland) under grant 2019/35/B/ST2/00034. This research was
also funded by National Science Centre (Poland) under the OPUS call within the WEAVE programme
2021/43/I/ST3/01142 (J.Z.) A
partial support by the Strategic Programme Excellence
Initiative at Jagiellonian University as well as that within the QuantEra II Programme that has received funding from the European Union's Horizon 2020 research and innovation programme under Grant Agreement No 101017733 DYNAMITE (M.L. and J.Z.). M.F.C. acknowledges financial support from the Guangdong Province Science and Technology Major Project (Future functional materials under extreme conditions, No. 2021B0301030005) and the Guangdong Natural Science Foundation (General Program project No. 2023A1515010871). A.S.M. acknowledges funding support from: The European Union’s Horizon 2020 research and innovation programme under the Marie Sk\l odowska-Curie grant agreement, SSFI No.\ 887153.

N.W. and P.H.B. are supported by the U.S. Department of Energy, Office of Science, Basic Energy Sciences (BES), Chemical Sciences, Geosciences, and Biosciences Division, AMOS Program.

M.L. acknowledges support from: ERC AdG NOQIA; Ministerio de Ciencia y Innovation Agencia Estatal de Investigaciones (PGC2018-097027-B-I00/10.13039/501100011033, CEX2019-000910-S/10.13039/501100011033, Plan National FIDEUA PID2019-106901GB-I00, FPI, QUANTERA MAQS PCI2019-111828-2, QUANTERA DYNAMITE PCI2022-132919, Proyectos de I+D+I “Retos Colaboración” QUSPIN RTC2019-007196-7); MICIIN with funding from European Union NextGenerationEU(PRTR-C17.I1) and by Generalitat de Catalunya; Fundació Cellex; Fundació Mir-Puig; Generalitat de Catalunya (European Social Fund FEDER and CERCA program, AGAUR Grant No. 2021 SGR 01452, QuantumCAT \ U16-011424, co-funded by ERDF Operational Program of Catalonia 2014-2020); Barcelona Supercomputing Center MareNostrum (FI-2022-1-0042); EU Horizon 2020 FET-OPEN OPTOlogic (Grant No 899794); EU Horizon Europe Program (Grant Agreement 101080086 — NeQST), National Science Centre, Poland (Symfonia Grant No. 2016/20/W/ST4/00314); ICFO Internal “QuantumGaudi” project; European Union’s Horizon 2020 research and innovation program under the Marie-Skłodowska-Curie grant agreement No 101029393 (STREDCH) and No 847648 (“La Caixa” Junior Leaders fellowships ID100010434: LCF/BQ/PI19/11690013, LCF/BQ/PI20/11760031, LCF/BQ/PR20/11770012, LCF/BQ/PR21/11840013). Views and opinions expressed in this work are, however, those of the author(s) only and do not necessarily reflect those of the European Union, European Climate, Infrastructure and Environment Executive Agency (CINEA), nor any other granting authority. Neither the European Union nor any granting authority can be held responsible for them.

% Bibliography
\documentclass{article}


% if you need to pass options to natbib, use, e.g.:
%     \PassOptionsToPackage{numbers, compress}{natbib}
\PassOptionsToPackage{numbers}{natbib}
% before loading neurips_2022


% ready for submission
% \usepackage{neurips_2022_arxiv}


% to compile a preprint version, e.g., for submission to arXiv, add add the
% [preprint] option:
    % \usepackage[preprint]{neurips_2022}


% to compile a camera-ready version, add the [final] option, e.g.:
    % \usepackage[final]{neurips_2022}
    \usepackage[final]{neurips_2022_arxiv}

% \nolinenumbers

% to avoid loading the natbib package, add option nonatbib:
%    \usepackage[nonatbib]{neurips_2022}


\usepackage[utf8]{inputenc} % allow utf-8 input
\usepackage[T1]{fontenc}    % use 8-bit T1 fonts
\usepackage{hyperref}       % hyperlinks
\usepackage{url}            % simple URL typesetting
\usepackage{booktabs}       % professional-quality tables
\usepackage{amsfonts}       % blackboard math symbols
\usepackage{nicefrac}       % compact symbols for 1/2, etc.
\usepackage{microtype}      % microtypography
\usepackage{xcolor}         % colors

% For theorems and such
\usepackage{amsmath}
\usepackage{amssymb}
\usepackage{mathtools}
\usepackage{amsthm}
\usepackage{subcaption}

% if you use cleveref..
\usepackage[capitalize,noabbrev]{cleveref}

\usepackage[textsize=tiny]{todonotes}

% Jindong
\usepackage{multicol}
\usepackage{defs}
% \usepackage{adjustbox}
\usepackage{siunitx}
% \usepackage{lineno}

% Packages added by Gautam.
\usepackage{lipsum}



\title{Object-Centric Slot Diffusion}

\author{%
   Jindong Jiang\thanks{Correspondence to \texttt{jindong.jiang@rutgers.edu} and \texttt{sungjin.ahn@kaist.ac.kr}.} \\
%   Department of Computer Science\\
   Rutgers University \\
   \texttt{jindong.jiang@rutgers.edu}\\
   \And
  Fei Deng\\
%   Department of Computer Science\\
   Rutgers University \\
   \texttt{fei.deng@rutgers.edu}\\
   \And
  Gautam Singh\\
%   Department of Computer Science\\
   Rutgers University \\
   \texttt{singh.gautam@rutgers.edu}\\
   \And
%    Yi-Fu Wu \\
% %   Department of Computer Science\\
%    Rutgers University \\
%    \texttt{yifu.wu@gmail.com}\\
%    \And
   Sungjin Ahn \\
%   School of Computing\\
   KAIST \\
   \texttt{sungjin.ahn@kaist.ac.kr}
}


\begin{document}


\maketitle


\begin{abstract}
% The goal of object-centric learning is to discover representations of compositional knowledge pieces in an unsupervised way. However, 
Despite remarkable recent advances, making object-centric learning work for complex natural scenes remains the main challenge. The recent success of adopting the transformer-based image generative model in object-centric learning suggests that having a highly expressive image generator is crucial for dealing with complex scenes. In this paper, inspired by this observation, we aim to answer the following question: \textit{can we benefit from the other pillar of modern deep generative models, i.e., the diffusion models, for object-centric learning} and \textit{what are the pros and cons of such a model}? To this end, we propose a new object-centric learning model, Latent Slot Diffusion (LSD). LSD can be seen from two perspectives. From the perspective of object-centric learning, it replaces the conventional slot decoders with a latent diffusion model conditioned on the object slots. Conversely, from the perspective of
diffusion models, it is the first unsupervised compositional conditional
diffusion model which, unlike traditional diffusion models, does not require supervised annotation such as a text description to learn to compose. In experiments on various object-centric tasks, including the FFHQ dataset for the first time in this line of research, we demonstrate that LSD significantly outperforms the state-of-the-art transformer-based decoder, particularly when the scene is more complex. We also show a superior quality in unsupervised compositional generation.
% because, despite the success and unique merits brought by the diffusion models in image generation, it has not been studied in the context of unsupervised object-centric learning.
\end{abstract}

\section{Introduction}

The underlying fundamental structure of the physical world is compositional and modular. While in some data modalities like language, this compositional structure is naturally revealed in the form of tokens or words, in general, this structure is hidden in modalities such as images and it is quite elusive how one may discover it.
% still elusive how to discover this  compositional structure when the modality is unstructured e.g., images. 
Yet, such representational compositionality and modularity are considered crucial for many applications that require systematically manipulating such modular token-like pieces, \emph{e.g.}, reasoning \cite{lake2017building,bottou2014machine}, causal inference \cite{scholkopf2021toward}, and out-of-distribution generalization \cite{bahdanau2018systematic, fodor88}.
% Many recent works have demonstrated the benefits of object-centric representations in applications such as ...

Object-centric learning \cite{binding} aims to discover this hidden compositional structure from unstructured observation by learning to bind relevant features and forming useful tokens in an unsupervised way. For images, one of the most popular approaches is to auto-encode the image via Slot Attention~\cite{slotattention} as the encoder. Slot Attention applies competitive spatial attention
to partition the image into separate local areas and represents each area as a slot.
% among the slots and making each slot represent an object in the image independently of other objects. 
Then, a decoder generates the image from the slots with the aim of minimizing the reconstruction error. 
Due to limited capacity per slot, the representation is encouraged to become compositional by making each slot capture a reusable entity, \emph{e.g.}, an object.

The most important challenge remaining in this framework of unsupervised object-centric learning is to make it work for complex naturalistic scene images. 
% And, in achieving this, it recently turned out that how to generate the reconstruction image is a crucial factor. 
Specifically, until recently, a special type of decoder, often called the \textit{mixture decoder}, has been used dominantly in most object-centric models \cite{slotattention,genesis,monet}. 
While the mixture decoder was originally designed with heavy priors to make it work on toy synthetic images, \emph{e.g.}, by decoding each slot with a weak decoder \cite{sbd}, subsequent results have shown that such priors eventually make it struggle when dealing with complex naturalistic scene images. Contrary to the conventional belief, Singh \emph{et al.}~\cite{slate} recently proposed to depart from the low-capacity mixture-decoder approach and use an expressive transformer-based autoregressive image generator in object-centric learning~\cite{slate,steve}. It was shown that increasing the decoder capacity is the key to dealing with complex and naturalistic scenes in this framework. Following this, several works have demonstrated the effectiveness of this transformer-based slot decoding approach in various settings~\citep{chang2022object, Chang2022HierarchicalAF, dinosaur, slotformer, sysbinder}.

\begin{figure}[t]
\centering
\includegraphics[width=0.98\textwidth]{fig/first_page_figure_horizontal.pdf}
\vskip -0.1in
\caption{\textbf{Overview.} In this paper, we propose and investigate a novel model called \textit{Latent Slot Diffusion} or \textit{LSD}. \textit{Left:} From the perspective of object-centric learning, LSD can effectively decompose complex and naturalistic scene images into objects in a fully unsupervised manner. \textit{Right:} From another perspective, LSD provides the ability to synthesize novel high-fidelity scenes by composing visual concepts acquired without any supervision.}
\label{fig:qualitative_overall}
\end{figure}

This success of transformer-based image generative modeling in object-centric learning naturally raises a question: \textit{can the other pillar of modern deep generative models, the diffusion models, also be beneficial for object-centric learning}? The diffusion models \cite{sohl2015deep,ddpm}, based on the stochastic denoising process, have achieved remarkable success in various image generation tasks \cite{dalle2,glide,imagen,adm,ldm,srdm,sdedit}, and can sometimes outperform the transformer-based autoregressive models. It also comes with modeling abilities that the transformer-based autoregressive models cannot provide \cite{sohl2015deep}. However, it has not yet been studied in the context of unsupervised object-centric learning and thus it is of crucial interest to investigate if this is realizable at all and what the pros and cons would be. 

In this paper, our aim is to answer this question. For this, we propose a novel model called Latent Slot Diffusion (LSD). The LSD model can be understood from two perspectives. On one hand, we have the perspective of object-centric learning. From this perspective, LSD can be seen as replacing the conventional slot decoders with a conditional latent diffusion model where the conditioning is on object-centric slots provided by Slot Attention. 
On the other hand, we have the perspective of diffusion models. From this perspective, ours is the first \textit{unsupervised} compositional conditional diffusion model. While conventional conditional diffusion models~\cite{dalle2,ldm,imagen,liu2022compositional} require providing a supervised annotation such as the text description of an image to perform compositional generation, LSD is a diffusion model that supports constructing such a compositional description in terms of visual concepts extracted from images through unsupervised object-centric learning. 


In experiments, we evaluate the proposed model in various object-centric tasks, including unsupervised object segmentation, downstream property prediction, compositional generation, and image editing. We show that the LSD model provides significantly better performance than the state-of-the-art model, \emph{i.e.}, the transformer-based autoregressive generative model. A remarkable property of the proposed model is that LSD outperforms the autoregressive transformers more significantly as the scene becomes more complex. Particularly, LSD allows exploring for the first time the applicability of object-centric models to FFHQ \cite{stylegan}, a dataset of high-resolution and high-quality face images that is beyond the generative capability of existing object-centric models.  Notably, we also find, however, that for very simple scene images like CLEVR \citep{clevr} that have limited visual diversity, LSD performs worse than the autoregressive transformer, presumably due to overfitting of the model.
% by memorizing spurious visual regularities.

\section{Latent Slot Diffusion}

\begin{figure}[t]
    \centering
    \includegraphics[width=0.98\textwidth]{fig/method.png}
    \caption{\textbf{Method.} \textit{Left:} In training, we encode the given image as a VQGAN latent and as slots. We then add noise to the VQGAN latent and we train a denoising network to predict the noise given the noisy latent and the slots. \textit{Right:} Given the trained model, we can generate novel images by composing a slot-based concept prompt and decoding it using the trained latent slot diffusion decoder.}
    \label{fig:method}
\end{figure}

Latent Slot Diffusion or LSD is a novel object-centric learning method that we propose and study in this work. Like conventional object-centric learning approaches, it seeks to achieve object-centric decomposition of a given scene image via auto-encoding. However, different from the conventional approaches, LSD incorporates the recent advances in diffusion modeling into the design of the decoder and investigates their pros and cons for the first time in this line of research. In the rest of this section, we describe the auto-encoding framework of LSD by first describing our object-centric encoder and then our proposed decoder.

\subsection{Object-Centric Encoder}
Given an input image $\bx \in \eR^{H \times W \times C}$, our object-centric encoder seeks to decompose and represent it as a collection of $N$ vectors or slots $\bS \in \eR^{N \times D}$ where each slot (denoted as $\bs_n \in \eR^{D}$) should represent one object in the image. For this, we adopt Slot Attention, an architecture that is also used in the current state-of-the-art object-centric learning approaches \citep{slotattention, slate, savi}. We now describe how Slot Attention works in our model.


In Slot Attention, we first encode the input image $\bx$ as a set of $M$ input features $\bE \in \eR^{M \times D_\text{input}}$ via a backbone network $\smash{f_\phi^\text{backbone}}$, \emph{i.e.}, $\smash{\bE = f_\phi^\text{backbone}(\bx)}$. The network $\smash{f_\phi^\text{backbone}}$ is implemented as a CNN
% in which positional embeddings are added in the second-last layer. 
whose final output feature map is flattened to form a set. 
% UNet ....
% The $N$ slots together provide a representation $\bS$ of the full image. The slots are expected to provide a modular representation of the full image and thus each spatial grouping should capture a meaningful entity e.g.,~an object.
Next, the features in $\bE$ are grouped into $N$ spatial groupings and the information in each grouping is aggregated to produce a \textit{slot}. 
The grouping is achieved via an iterative slot refinement procedure. At the start of the refinement procedure, the slots $\bS$ are filled with random Gaussian noise. Then, they are refined via \textit{competitive attention} over the input features, where the $N$ slots act as the queries and the $M$ input features act as the keys and values.  The  queries and  keys undergo a dot-product to produce $N\times M$ attention proportions. Next, on these attention proportions, softmax is applied along the axis $N$ to produce attention weights $\bA$ that capture the soft assignment of each input feature to a slot. Then, for each $n$, all input features are sum-pooled weighted by their attention weights $\bA_{n,1}, \ldots, \bA_{n,M}$ to produce an attention readout $\bu_n \in \eR^{D}$. These steps can be formally described as follows:
\begin{align*}
    \bA = \underset{N}{\texttt{softmax}}\left(\frac{q(\bS) \cdot k(\bE)^T}{\sqrt{D}}\right) 
    && \Longrightarrow && 
    \bA_{n,m} = \dfrac{\bA_{n,m}}{\sum_{m=1}^M \bA_{n,m}}
    && \Longrightarrow && 
    \bu_n = \sum_{m=1}^M  v(\bE_m) \bA_{n,m},
\end{align*}
where, $q, k, v$ are linear projections that map the slots and input features to a common dimension $D$. Using the bottom-up information captured by the readout $\bu_n$, the slots are updated by an RNN as $\smash{\bs_n = f_\phi^\text{RNN}(\bs_n, \bu_n)}$. In practice, competitive attention and RNN update are performed iteratively several times and slots from the last iteration are considered the final slot representation $\bS$.


\subsection{Latent Slot Diffusion Decoder}

In this section, we describe our proposed decoding approach called \textit{Latent Slot Diffusion Decoder} or \textit{LSD decoder} for reconstructing the image given the slot representation $\bS$. The design of the LSD decoder takes advantage of the recent advances in generative modeling based on diffusion \cite{ldm, ddpm}. In Figure~\ref{fig:method} we provide an overview of our decoding approach.

\subsubsection{VQGAN}
One of the key design components of the LSD decoder is VQGAN \cite{vqgan}. VQGAN  provides a way to map an image $\bx$ to a lower dimensional \textit{latent representation} $\bz_0$ via an encoder $\smash{f_\phi^\text{VQ}}$. This helps LSD reduce the computational burden for reconstructing high-resolution images by allowing it to use the lower dimensional latent $\bz_0$ as an intermediate reconstruction target. VQGAN also allows us to later obtain the original resolution image without significantly losing the image contents and fidelity by decoding the latent $\bz_0$ using the VQGAN decoder $\smash{g_\ta^\text{VQ}}$. These can be summarized as:
\begin{align*}
    \bz_0 = f_\phi^\text{VQ}(\bx), && \hat{\bx} = g_\ta^\text{VQ}(\bz_0), && \text{where } \bz_0 \in \eR^{H_\text{VQ} \times W_\text{VQ} \times D_\text{VQ}}, \text{ and } \hat{\bx} \in \eR^{H \times W \times C} .
    %, \quad \hat{\bx} = g_\text{AE}(\bz)
\end{align*}
In our model, we use a VQGAN pre-trained on OpenImages \footnote{We use the pre-trained weights from https://ommer-lab.com/files/latent-diffusion/kl-f8.zip}.
% In our model, we use VQGAN pre-trained on OpenImages.
% Remarkably, this can not only provide a significant reduction in the dimensionality compared to the original image dimensions but can do so without losing the image contents and the image fidelity. 
% However, in the object-centric learning context, previous methods have not leveraged VQGAN and ours is the first model that investigates its potential.

\subsubsection{Slot-Conditioned Diffusion}

In LSD, we leverage diffusion modeling to reconstruct the VQGAN latent $\bz_0$ conditioned on the slots $\bS$. This modeling approach has been primarily explored in supervised contexts, \emph{e.g.}, for text-to-image generation in Latent Diffusion Models (LDM) \citep{ldm}. However, different from LDM, in this work, instead of conditioning the decoder on embeddings of supervised labels, we condition it on slots where the process of obtaining the slots themselves, \emph{i.e.}, Slot Attention, is jointly trained with the decoder without supervision. 
Following LDM, our decoder works by training a decoding distribution $p_\ta(\bz_0|\bS)$ to maximize the log-likelihood $\log p_\ta(\bz_0|\bS)$ of the VQGAN latent $\bz_0$ given the slots $\bS$. This decoding distribution $p_\ta(\bz_0|\bS)$ is modeled as a $T$-step denoising process:
\begin{align*}
    p_\ta(\bz_0|\bS) = \int p(\bz_T) \prod_{t=T, \ldots, 1} p_\ta(\bz_{t-1} | \bz_{t}, t, \bS) \,\mathrm{d} \bz_{1:T},
\end{align*}
where $p(\bz_T) = \cN(\mathbf{0}, \mathbf{I})$, $p_\ta(\bz_{t-1} | \bz_{t},  t, \bS)$ is a one-step denoising distribution, and $\bz_T, \bz_{T-1}, \ldots, \bz_0$ is a sequence of progressively denoised latents.
The one-step denoising distribution $p_\ta(\bz_{t-1} | \bz_{t}, t, \bS)$ is parametrized via a neural network $g_\ta^\text{LSD}$ in the following manner:
\begin{align*}
    p_\ta(\bz_{t-1} | \bz_{t}, t, \bS) = \cN\left(\frac{1}{\sqrt{\alpha_t}}\left(\bz_t -\frac{\beta_t}{\sqrt{1-\bar{\alpha}_t}}\hat{\boldsymbol{\epsilon}}_t \right), \beta_t\mathbf{I}  \right), && \text{where } \hat{\boldsymbol{\epsilon}}_t = g^\text{LSD}_\ta(\bz_t, t, \bS),
\end{align*}
$\beta_1, \ldots, \beta_T$ is a linearly increasing variance schedule, $\alpha_t = 1 - \beta_t$, and $\bar{\alpha}_t = \prod_{i=1}^t (1 - \beta_i)$. 

\textbf{Sampling Procedure.} To sample a $\bz_0 \sim p_\ta(\bz_0|\bS)$, we adopt an iterative denoising procedure as in \citep{ldm, ddpm}. The sampling process starts with a latent representation $\bz_T\sim \cN(\mathbf{0}, \mathbf{I})$ filled with random Gaussian noise. Next, conditioned on the slots, we denoise it $T$ times by sampling sequentially from the one-step denoising distribution $\bz_{t-1} \sim p_\ta(\bz_{t-1} | \bz_t, t, \bS)$ for $t=T, \ldots, 1$.
% \begin{align*}
%     \hat{\boldsymbol{\epsilon}}_t = g^\text{LSD}_\ta(\bz_t, t, \bS) && \Longrightarrow && \bz_{t-1} \sim \cN\left(\frac{1}{\sqrt{\alpha_t}}\left(\bz_t -\frac{\beta_t}{\sqrt{1-\bar{\alpha}_t}} \hat{\boldsymbol{\epsilon}}_t\right), \beta_t  \right), 
% \end{align*}
% where $\alpha_t = 1 - \beta_t$ and $\bar{\alpha}_t = \prod_{i=1}^t 1 - \beta_i$. 
This produces a sequence of latents $\bz_T, \bz_{T-1}, \ldots, \bz_0$ that become progressively cleaner. Finally, $\bz_0$ can be considered as the reconstructed latent representation.

\textbf{Training Procedure.} Following LDM \cite{ldm}, the training of $p_\ta(\bz_0|\bS)$ can be cast to a simple procedure for training $g^\text{LSD}_\ta$ as follows. Given an image $\bx$, its slot representation $\bS$, and its VQGAN latent $\bz_0$, we first randomly choose a noise level $t \in \{1, \ldots, T\}$ from a uniform distribution. Given the $t$, we corrupt the clean latent $\bz_0$ and obtain a noised latent $\bz_t$ as follows:
\begin{align*}
    \bz_t = \sqrt{\bar{\alpha}_t} \bz_0 + \sqrt{1 - \bar{\alpha}_t} \boldsymbol{\epsilon}_t, && \text{ where } \boldsymbol{\epsilon}_t \sim \cN(\mathbf{0}, \mathbf{I}), \quad \bar{\alpha}_t = \prod_{i=1}^t (1 - \beta_i).
\end{align*}
The noised latent $\bz_t$ is then given as input to $g^\text{LSD}_\ta$ along with the slots $\bS$ and denoising time-step $t$ to predict the noise $\boldsymbol{\epsilon}_t$.
The network $g^\text{LSD}_\ta$ is trained by minimizing the mean squared error between the predicted noise $\hat{\boldsymbol{\epsilon}}_t$ and the true noise $\boldsymbol{\epsilon}_t$:
\begin{align*}
    \hat{\boldsymbol{\epsilon}}_t = g^\text{LSD}_\ta(\bz_t, t, \bS) && \Longrightarrow && \cL(\phi, \ta) = %\mathbb{E}_{t, \bz, \boldsymbol{\epsilon}}  
    || \hat{\boldsymbol{\epsilon}}_t - \boldsymbol{\epsilon}_t ||^2.
\end{align*}



\subsubsection{Denoising Network}
We implement the denoising network $g^\text{LSD}_\ta$ as a variant of the conventional UNet architecture adapted to incorporate slot-conditioning. Our denoising network consists of a stack of $L$ layers where each layer $l$ is a UNet-style CNN layer followed by a slot-conditioned transformer:
% \begin{align*}
%    \tilde{\bh}_{l} = \begin{cases}\texttt{CNN}_{\ta}^l(\bh_{l-1}, t) , & \text{if } l \leq \frac{L}{2} \\ \texttt{CNN}_{\ta}^l([\bh_{l-1}, \bh_{L-l + 1}], t), & \text{otherwise} \end{cases} && \Longrightarrow && \bh_{l} = \texttt{Transformer}_{\ta}^l(\tilde{\bh}_{l} + \mathbf{p}_l, \texttt{cond=}\bS),
% \end{align*}
\begin{align*}
   \tilde{\bh}_{l} = \texttt{CNN}_{\ta}^l([\bh_{l-1}, \bh_{\text{skip}(l)}], t), && \Longrightarrow && \bh_{l} = \texttt{Transformer}_{\ta}^l(\tilde{\bh}_{l} + \mathbf{p}_l, \texttt{cond=}\bS),
\end{align*}
where $\bh_0=\bz_t$ is the input, $\bh_1, \ldots, \bh_{L-1}$ are the hidden states and $\hat{\boldsymbol{\epsilon}}_t = \bh_L$ is the output. 

\textbf{CNN Layers.} Following UNet \cite{unet}, the convolutional layers $\smash{\texttt{CNN}_{\ta}^1, \ldots, \texttt{CNN}_{\ta}^L}$ downsample the feature map via the first $\smash{\frac{L}{2}}$ layers and then upsample it back to the original resolution via the remaining layers. Following standard UNet, these CNN layers also receive inputs via skip connection from an earlier layer denoted by $\text{skip}(l)$. This network design has also been explored in LDM \citep{ldm}.


% The CNN implements the conditioning on the noise level $t$ by a

\textbf{Slot-Conditioned Transformer.} The role of the transformer is to incorporate the information from the slots into the UNet-based denoising process. For this, in each layer $l$, the intermediate feature map $\smash{\tilde\bh_l}$ produced by the CNN layer is flattened into a set of features. To this, positional embeddings $\bp_l$ are added and the resulting features are provided as input to the transformer. Within the transformer, these features interact with each other and with the slots $\bS$, thus incorporating the information from the slots into the denoising process. The transformer output is then reshaped back to a feature map $\bh_l$.

% \begin{figure}[!t]
%     \centering
%     \includegraphics[width=0.98\textwidth]{fig/method.png}
%     \caption{\textbf{Method.} \textit{Left:} In training, we encode the given image as a VQGAN latent and as slots. We then add noise to the VQGAN latent and we train a denoising network to predict the noise given the noisy latent and the slots. \textit{Right:} Given the trained model, we can generate novel images by composing a slot-based concept prompt and decoding it using the trained latent slot diffusion decoder.}
%     \label{fig:method}
% \end{figure}

% \textcolor{red}{This allows....}. We shall see in Section \ref{sec:experiments} that this leads to an improved performance. Similarly to \cite{slotattention}, each $\mathbf{P}_l$ is obtained by mapping the 2D grid coordinates to embedding vectors of dimension $D$ using a learned linear projection.

% \textbf{Conditioning on $t$.}
% \begin{align*}
%     \psi(\bh_l, t) = \mathbf{t}_s(t) \texttt{GN}(\bh_l) + \mathbf{t}_b(t)
% \end{align*}

% $\mathbf{t}_s$ and $\mathbf{t}_b$ are implemented as MLP layers.


% \subsubsection{Training Objective}
% \begin{align*}
%     \cL = || \hat{\boldsymbol{\epsilon}}_t - \boldsymbol{\epsilon} ||^2
% \end{align*}

\section{Compositional Image Synthesis}
\label{sec:visual_concept_lib}
In this section, we describe how a trained LSD model can be used to compose and synthesize novel images. The conventional approach to composing novel images is commonly supervised and is based on text prompts. That is, using words from a vocabulary, a sentence is composed which is given to a text-to-image model to synthesize the desired image \citep{dalle, dalle2, imagen, ldm, glide}. However, in a fully unsupervised setting like ours, we first need to build a library of visual concepts by simply observing a large set of unlabelled images. Then, similarly to composing a sentence prompt using words, we compose a \textit{concept prompt} by picking concepts from the visual concept library. By providing this concept prompt to the LSD decoder, we can then synthesize a desired novel image. This approach of unsupervised compositional image synthesis was also explored in \cite{slate}.

\textbf{Unsupervised Visual Concept Library.} To build a library of visual concepts from unlabelled images, we first take a large batch of $B$ images $\bx_1, \ldots, \bx_B$. To this, we apply slot attention to obtain slots for these images $\bS_1, \ldots, \bS_B$. We then collect all these slots as a single set $\cS$ and apply $K$-means on it. The $K$-means procedure assigns each slot in $\cS$ to one of the $K$ clusters. We consider the set of slots that are assigned to a $k$-th cluster as a visual concept library $\cV_k$. With $K$ as the number of clusters, this procedure provides $K$ visual concept libraries $\cV_1, \ldots, \cV_K$. Our experiments shall show that this simple $K$-means procedure can produce semantically meaningful concept libraries. For instance, on a dataset of human face images such as FFHQ \cite{stylegan}, the $K$ libraries correspond to useful concept classes such as hair style, face, clothing, and background.

\textbf{Novel Image Synthesis.} Given libraries $\cV_1, \ldots, \cV_K$, we can compose a concept prompt $\bS_\text{compose}$ by picking $K$ slots, each from the corresponding $k$-th library, and stacking them together:
\begin{align*}
    \bS_\text{compose} = (\bs_1, \ldots, \bs_K), && \text{where } \bs_k \sim \text{Uniform}(\cV_k)
\end{align*}
We then give the composed prompt $\bS_\text{compose}$ to the LSD decoder to generate the VQGAN latent: $\smash{\bz_\text{compose} \sim p_\ta(\bz_0 | \bS_\text{compose})}$. Applying the VQGAN decoder on $\smash{\bz_\text{compose}}$ generates the desired novel scene image $\smash{\bx_\text{compose} = g_\ta^\text{VQ}(\bz_\text{compose})}$. For instance, in the FFHQ dataset, the concept prompt can be a collection of chosen hair style, face, clothing, and background. Decoding this prompt would generate a face image that conforms to this prompt.


% \begin{align*}
%     \bz_\text{compose} \sim p_\ta(\bz_0 | \bS_\text{compose}) && \Longrightarrow && \bx_\text{compose} = g_\ta^\text{VQ}(\bz_\text{compose})
% \end{align*}


% \clearpage
\section{Related Work}
\textbf{Unsupervised Object-Centric Learning.} Object-centric representation learning aims to decompose multi-object scenes into meaningful object entities. A common approach to this is by auto-encoding. In this line, the focus has been to design an appropriate decoder that supports good decomposition. The most widely used decoders include the mixture-decoder \cite{monet, iodine, slotattention, nem, genesis, von2020towards, anciukevicius2020object, du2020unsupervised, engelcke2021genesis, simone, Zhang2022RobustAC}, spatial transformer decoder \cite{air, spair, space, gnm, deng2020generative, roots}, Neural Radiance Fields (NeRF) \cite{stelzner2021decomposing,Wu2022ObPoseLC,Smith2022UnsupervisedDA}, transformer decoder \cite{slate, steve, sysbinder, slotformer, Chang2022HierarchicalAF, sajjadi2022osrt, Gopalakrishnan2022UnsupervisedLO}, energy-based models \cite{du2021unsupervised} and complex-valued functions \citep{Lowe2022ComplexValuedAF}. 
Among these, the mixture decoder has been shown to struggle on complex scene images \cite{slate, steve} while the spatial transformer decoder requires careful tuning of hyperparameters which limits their applicability in realistic scenes. Neural Radiance Fields require camera poses, and the learning setting involves 3D scenes unlike ours. 
% While Transformer decoders have been shown to improve object factorization and generation quality on simple synthetic scenes \cite{slate, slotformer}, they have yet to be tested on complex datasets and high-resolution images. 
Another line of work seeks object-centric scene decomposition without reconstruction. This includes works such as \cite{dinosaur, Wen2022SelfSupervisedVR, Henaff2022ObjectDA, cutler, sslslot, contrastslot}. However, unlike ours, these approaches lack the ability to generate images. Preceding object-centric learning, several works pursued disentanglement within a single-vector representation of the scene and use it for compositional image generation \citep{infogan, higgins2017beta, kim2018disentangling, kumar2017variational, chen2018isolating}. However, lacking spatial binding mechanisms, these methods struggle in multi-object scenes \citep{iodine} unlike ours.


\textbf{Diffusion Models.} 
Diffusion models (DMs) are a recent class of generative models that can produce high-quality images by reversing a stochastic process that gradually adds noise to an image \citep{ddpm,scoredm}. DMs have been applied to various tasks in computer vision, such as class-conditioned generation, text-to-image generation, image editing, super-resolution, and inpainting \citep{adm,cfdm,dalle2,glide,imagen,sdedit,cascadedm,srdm}. Recently, \citep{ldm} proposed Latent Diffusion Models (LDM). By virtue of operating on a low-dimensional latent space, LDM reduces the computational demands of DMs significantly.
% It also proposes to use a cross-attention for flexible conditional generations for various conditioning types. 
\citep{composdm} introduced a method for multi-object scene generation by combining signals from multiple text-conditioned denoising networks. However, to achieve controllable generation, many of these existing DMs require additional labels such as text descriptions to train and control the generation process. In contrast, our model can generate images compositionally using object concepts directly extracted from images. Moreover, DMs have also been used for representation learning. \citep{labeleffdm} demonstrated that the intermediate features of a learned DM can be used as useful image representations for label-efficient semantic segmentation. \citep{dmae} introduced an image encoder that compresses an image into a latent representation, jointly trained with the denoising objective. Nevertheless, these representations are unstructured and not modular like ours. 
% Our work differs from these approaches as we learn to segment an image into object-aware regions and obtain structured representations for each object.


% \clearpage
\section{Experiments}

% ########################################################################
\begin{figure*}[t]
\centering
\includegraphics[width=1.\textwidth]{fig/segmentation.pdf}
\vskip -0.1in
\caption{\textbf{Visualization of Unsupervised Object Segmentation.} We show visualizations of predicted segments on CLEVR, CLEVRTex, MOVi-E, and FFHQ datasets. 
% LSD enjoys a significant advantage on datasets with higher scene complexity.
}
\label{fig:seg}
\end{figure*}
% ########################################################################


We extensively evaluate our proposed Latent Slot Diffusion (LSD) model on various object-centric tasks, including unsupervised object segmentation, downstream property prediction, compositional generation, and image editing. As will be shown, our model significantly outperforms the state-of-the-art on multiple challenging datasets with complex texture and background, including FFHQ~\cite{stylegan} which has been beyond the generative capability of object-centric models.

\textbf{Datasets.} We evaluate our model on five datasets. Four of them are synthetic multi-object datasets---CLEVR~\citep{clevr}, CLEVRTex~\citep{clevrtex}, MOVi-C, MOVi-E~\citep{kubric}. They present increasing levels of difficulty---CLEVRTex adds texture to objects and backgrounds, MOVi-C uses more complex objects and natural backgrounds, and MOVi-E contains large numbers of objects (up to 23) per scene. Furthermore, we explore for the first time the applicability of object-centric models to FFHQ~\citep{stylegan}, a dataset of high-quality face images that is beyond the generative capability of current object-centric models. Unlike previous works in this line that only investigate low-resolution images, \emph{e.g.}, $128 \times 128$, we use a resolution of $256 \times 256$ for all datasets in our experiments.

\textbf{Baselines.} We compare our model against SLATE, the state-of-the-art object-centric learning and unsupervised compositional image generation approach. We use its improved version~\cite{steve}, which is more robust in complex scenes. For a fair comparison with LSD that leverages VQGAN, we also develop a VQGAN-based variant of SLATE denoted as SLATE$^+$, where its low-capacity dVAE~\cite{slate} is replaced with VQGAN. For all models in this work, we use an OpenImages-pretrained VQGAN~\cite{vqgan}.

\subsection{Object-Centric Representation Learning}

In line with previous work~\cite{slotattention}, we use unsupervised object segmentation and downstream property prediction to evaluate the object-centric learning capability and representation quality. Within each dataset, all models we compare use the same number of slots.
% We focus on the four multi-object datasets due to availability of groundtruth labels for evaluation. % For all models, we set the number of slots to be the maximum number of objects per image plus one (intended for background).




\begin{table*}[t]
\begin{small}
    \caption{\textbf{Segmentation Performance and Representation Quality.} \textit{Left:} We evaluate the segmentation quality and report mBO, mIoU, and FG-ARI scores across various datasets and baselines. \textit{Right:} We measure the representation quality by learning a probe to predict the object property given frozen slots. For position and 3D bounding box, we report MSE. For shape, material, and category, we report accuracy.
    %We report the position error in pixel units.
    }
    % \vskip 0.1in

    % CLEVRTex
    \begin{subtable}{1.0\linewidth}
    \caption{CLEVRTex}
    \vspace{-1mm}
    \centering
    \begin{tabular}{lccc}
    \toprule
    \scriptsize{\textbf{Segmentation}}          & SLATE        & SLATE$^+$      & LSD (Ours)   \\
    \midrule
    mBO ($\uparrow$)         & 51.24 & 56.22 & \textbf{66.56} \\
    mIoU ($\uparrow$)        & 50.04 & 54.93 & \textbf{65.02} \\
    FG-ARI ($\uparrow$)      & 43.59 & \textbf{73.42} & 61.74 \\
    \bottomrule
    \end{tabular}
        \begin{tabular}{lccc}
    \toprule
    \scriptsize{\textbf{Representation}}           & SLATE        & SLATE$^+$      & LSD (Ours)   \\
    \midrule
    Shape ($\uparrow$)       & 77.62 & 76.20 & \textbf{81.60} \\
    Material ($\uparrow$)   & 72.59 & 66.04 & \textbf{77.77} \\
    Position ($\downarrow$)  & 1.16 & 1.13 & \textbf{1.10}    \\
    \bottomrule
    \end{tabular}
    \vspace{2mm}
    \end{subtable}


    % MOVi-C
    \begin{subtable}{1.0\linewidth}
    \caption{MOVi-C}
    \vspace{-1mm}
    \centering
    \begin{tabular}{lccc}
    \toprule
    \scriptsize{\textbf{Segmentation}}          & SLATE        & SLATE$^+$      & LSD (Ours)   \\
    \midrule
    mBO ($\uparrow$)         & 38.60 & 39.24  & \textbf{46.29} \\
    mIoU ($\uparrow$)        & 37.11 & 37.59  & \textbf{44.99} \\
    FG-ARI ($\uparrow$)      & 48.00 & \textbf{51.53}  & 50.53 \\
    \bottomrule
    \end{tabular}
        \begin{tabular}{lccc}
    \toprule
    \scriptsize{\textbf{Representation}}           & SLATE        & SLATE$^+$      & LSD (Ours)   \\
    \midrule
    Position ($\downarrow$)  & 1.36 & 1.22   & \textbf{1.14}   \\
    3D B-Box ($\downarrow$)   & 1.46 & \textbf{1.32}  & 1.44    \\
    Category ($\uparrow$)    & 42.49 & 46.46 & \textbf{46.71} \\
    \bottomrule
    \end{tabular}
    \vspace{2mm}    
    \end{subtable}


    % MOVi-E
    \begin{subtable}{1.0\linewidth}
    \caption{MOVi-E}
    \vspace{-1mm}
    \centering
    \begin{tabular}{lccc}
    \toprule
    \scriptsize{\textbf{Segmentation}}          & SLATE        & SLATE$^+$      & LSD (Ours)   \\
    \midrule
    mBO ($\uparrow$)         & 28.66 & 22.69  & \textbf{39.63} \\
    mIoU ($\uparrow$)        & 27.20 & 21.10  & \textbf{38.28} \\
    FG-ARI ($\uparrow$)     & 41.37 & 46.34  & \textbf{53.40} \\
    \bottomrule
    \end{tabular}
        \begin{tabular}{lccc}
    \toprule
    \scriptsize{\textbf{Representation}}           & SLATE        & SLATE$^+$      & LSD (Ours)   \\
    \midrule
    Position ($\downarrow$)  & 2.25 & 2.07  & \textbf{1.92}    \\
    3D  B-Box ($\downarrow$)   & 3.24 & 3.11  & \textbf{2.94}    \\
    Category ($\uparrow$)    & 38.89 & 38.55  & \textbf{43.15} \\
    \bottomrule
    \end{tabular}
    \end{subtable}

    \label{tab:complex_quant} 
    % \vspace{-1em}
\end{small}

\end{table*}


\textbf{Unsupervised Object Segmentation.}
Following~\cite{slotattention, dinosaur}, to measure segmentation quality, we report the foreground adjusted rand index (FG-ARI), the mean intersection over union (mIoU), and the mean best overlap (mBO), computed using the attention masks of Slot Attention.
% whose resolution is $64 \times 64$ in all models.

Our results in Table \ref{tab:complex_quant} suggest that LSD is particularly strong in visually complex scenes. For example, on the most challenging MOVi-E dataset, LSD outperforms the strongest baseline by ${>} 10\%$ in mBO and mIoU, and ${\sim} 7\%$ in FG-ARI. On CLEVRTex and MOVi-C featuring complex textures, LSD also achieves ${>} 10\%$ and ${\sim} 7\%$ gain respectively, in both mBO and mIoU. We visually show the superior segmentation quality of LSD in Figure \ref{fig:seg}. LSD obtains tighter object boundaries, less object splitting, and cleaner background segmentation. The advantages are most noticeable on CLEVRTex and MOVi-E, where the baselines over-segment the objects more frequently, or exhibit a common failure mode that divides images into approximately uniformly distributed block masks.

We also note that the FG-ARI score is sometimes not an ideal metric for evaluating segmentation quality, because it only considers the foreground pixels and disregards the correctness of the mask shape, as also highlighted by \cite{genesis,clevrtex}. In the MOVi-E experiment, even though SLATE$^+$ tends to partition the images arbitrarily into uniform patches, as shown in Figure \ref{fig:seg}, it still shows ${\sim} 5\%$ an improvement gain on FG-ARI over SLATE, which, in fact, produces more reasonable object masks. This highlights the need for additional metrics, such as the mIoU score that we also measure for evaluating the segmentation quality in this work.

\textbf{Downstream Property Prediction.}
Following~\cite{dittadi2021generalization, slotattention}, we evaluate the quality of the learned object-centric representations through downstream property prediction. Specifically, for each property, we train a network to predict the property given a frozen slot representation as input. The correspondence between the slot and its true label is determined by Hungarian matching~\cite{hungarian} using masks. We use linear heads and 2-layer MLP as the prediction networks for discrete and continuous values, respectively. We report prediction accuracy for discrete properties (\emph{e.g.}, shape and material), and mean squared error for continuous properties (\emph{e.g.}, position).

As shown in Table \ref{tab:complex_quant}, LSD is competitive with or better than the strongest baseline on CLEVRTex, MOVi-C, and MOVi-E, which is consistent with our finding in unsupervised object segmentation that LSD shines in visually complex scenes. For example, LSD obtains ${>} 4\%$ gain in predicting object category on MOVi-E and object material on CLEVRTex. 


% ########################################################################
\begin{table}[t]
\caption{\textbf{Compositional Image Generation.} On various datasets, we generate images using object slots randomly sampled from the dataset. We compare the fidelity of the generated images via the FID score. The lower the FID score, the better the fidelity. We find that our model produces the best fidelity scores compared to the baselines. }
\begin{center}
\begin{small}
    \begin{tabular}{lccc}
    \toprule
    FID $\downarrow$     & SLATE        & SLATE$^+$       & LSD (Ours)       \\
    \midrule
    CLEVR         & 52.96  & 20.93  & \textbf{16.22}  \\
    CLEVRTex      & 105.83 & 69.23 & \textbf{29.53}  \\
    MOVi-C        & 170.83 & 148.27 & \textbf{69.12}  \\
    MOVi-E        & 169.32 & 126.51 & \textbf{64.76}  \\
    FFHQ          & 112.38 & 98.76  & \textbf{27.83} \\
    \bottomrule
    \end{tabular}
    \vspace{.7em}
    \label{tab:fid} 
    % \vspace{-1em}
\end{small}
\end{center}
\end{table}


\subsection{Compositional Generation with Visual Concept Library}

% ########################################################################
\begin{figure*}[t]
\centering
\includegraphics[width=1.\textwidth]{fig/compsitional_generation.pdf}
\vskip -0.1in
\caption{\textbf{Compositional Generation Samples.} We qualitatively compare the quality of image samples generated by our model with the baselines. We observe that LSD provides significantly higher fidelity and more clear details compared to the other methods.}
\label{fig:compose}
\end{figure*}
% ########################################################################

% ########################################################################
\begin{figure*}[t]
\centering
\vskip -0.1in
\includegraphics[width=1.\textwidth]{fig/component_composition.pdf}
\caption{\textbf{Compositional Image Generation with Concept Prompts.} In this visualization, we show a concept prompt constructed by composing arbitrary slots from our visual concept library and the corresponding generated image by LSD.}
\label{fig:component}
\end{figure*}
% ######################################################################## 


Like text-to-image generative models, LSD is able to take unseen slot-based prompts at test time and compose new images. Unlike text-to-image models, however, LSD obtains the compositional generation ability solely from images without relying on additional supervision like text inputs.


As described in Section \ref{sec:visual_concept_lib}, we first build a concept library. Then, we sample one slot representation from each visual concept library and concatenate them into a sequence to form a slot-based prompt. We then feed the slot-based prompts to the diffusion decoder to generate the images. This produces scenes with novel object layouts and faces with unseen attribute combinations (see Figure~\ref{fig:component} for two examples).


We report in Table~\ref{tab:fid} the FID score~\cite{fid} as a measure of the compositional generation quality. Following standard practice~\cite{adm}, we compute the FID score using 2K generated images and the full training dataset. Across all datasets, LSD achieves significantly better FID scores than SLATE and SLATE$^+$. We further demonstrate the superior compositional generation quality of LSD in Figure~\ref{fig:compose}. We observe that on the more challenging MOVi-E and FFHQ datasets, LSD generates images with substantially more clear details and better coherence. In contrast, SLATE is limited by its decoder and produces images with missing details in the objects or human faces. While the details can be improved by the pre-trained VQGAN in SLATE$^+$, some samples still exhibit severe distortions.

\subsection{Slot-Based Image Editing}


% ########################################################################
\begin{figure}[t]
\centering
\vskip -0.1in
\includegraphics[width=1.0\textwidth]{fig/object_based_editing_horizontal.pdf}
% \caption{\textbf{Slot-Based Image Editing.} On the left, we show the results on the CLEVRTex and MOVi-E datasets, including the gradual removal of objects and the replacement of the background while maintaining the original objects. On the right, we demonstrate the face-swapping task, where the new faces are composed by combining the face slots from source B inputs with the hair, dress, and background slots from source A.}
\caption{\textbf{Slot-Based Image Editing.} \textit{Left:} We show the slot-based image editing ability of our model. In particular, we show edit operations such as object removal, object extraction, object insertion, background extraction, and background swapping. \textit{Right:} We show face replacement in the FFHQ dataset, where we compose new images by combining the face slots from Source-B images with the hairstyle, clothing, and background slots from Source-A images.}
\label{fig:edit_multi_obj}
\end{figure}
% ########################################################################


In addition to generating new images from randomly sampled slot-based prompts, LSD also allows editing existing images by directly modifying their slot representations. We demonstrate the potential of LSD for slot-based image editing through object manipulation tasks such as object removal, single object segmentation, object insertion, background extraction, and background swapping. Notably, we demonstrate, for the first time in object-centric generative models, the ability to perform face editing in real-world images. The results are shown in Figure~\ref{fig:edit_multi_obj}.


We conduct slot manipulation on the CLEVRTex dataset, focusing on object removal, single-object extraction, object insertion, background extraction, and background swapping. For object removal and background extraction, we remove an object by discarding its corresponding slot, while background extraction is achieved by discarding all object slots, leaving only the background slot. Our results show that when object decomposition is almost perfect (\emph{e.g.}, on CLEVRTex), an object can be removed simply by removing its corresponding slot. Moreover, the background component can be rendered from a single background slot, despite the model not encountering single-slot conditioning during training. In the single object extraction task, we render an individual object with the same background utilizing the corresponding object slot and the background slot. To demonstrate object insertion and background swapping tasks, we split the image into top and bottom pairs in Figure~\ref{fig:edit_multi_obj} \textit{left}. In object insertion, we introduce the slot extracted from the other image into the target image and show that the object is rendered in the image coherently. For performing the background swap, we interchange the background slots of two images. We show that the entire background of an image can be changed while correctly preserving the original objects.

We further explore face replacement on the FFHQ dataset. LSD decomposes each image into four slots, corresponding to face, hairstyle, clothing, and background. Our results show that by replacing the face slots of the images, we are able to coherently change the image while maintaining the hairstyle, clothing, and background. The resulting images look realistic, suggesting that LSD can effectively blend various attributes even when given novel combinations.


\subsection{Discussion: LSD on Simple Images}

% NEW TABLE
\begin{table*}[t]
\begin{small}
    \caption{
    % \textbf{Segmentation Performance.} We evaluate the segmentation quality and report mBO, mIoU and FG-ARI scores across various datasets and baselines. 
    \textbf{Segmentation Performance and Representation Quality in CLEVR.} \textit{Left:} We evaluate the segmentation quality and report mBO, mIoU and FG-ARI scores across various datasets and baselines. \textit{Right:} We measure the representation quality by learning a probe to predict the object property given frozen slots. For position, we report MSE. For shape and material, we report the accuracy.
    %We report the position error in pixel units.
    }
    % \vskip 0.1in
    \begin{tabular}{lccc}
    \toprule
    \scriptsize{\textbf{Segmentation}}                 & SLATE        & SLATE$^+$       & LSD (Ours) \\
    \midrule
    mBO ($\uparrow$)         & 64.86 & \textbf{67.42}  & 38.49  \\
    mIoU ($\uparrow$)        & 63.96 & \textbf{66.62}  & 37.49  \\
    FG-ARI ($\uparrow$)      & 69.20 & \textbf{88.56}  & 76.32  \\
    \bottomrule
    \end{tabular}
    \hspace{0.2em}
    \begin{tabular}{lccc}
    \toprule
    \scriptsize{\textbf{Representation}}                 & SLATE        & SLATE$^+$       & LSD (Ours) \\
    \midrule
    Shape ($\uparrow$)       & 95.66 & \textbf{95.70}  & 75.16  \\
    Material ($\uparrow$)    & 97.62 & \textbf{97.96}  & 94.21  \\
    Position ($\downarrow$)  & 0.59  & \textbf{0.51}   & 0.86   \\
    \bottomrule
    \end{tabular}
    \label{tab:simple_quant} 
    % \vspace{-1em}
\end{small}

\end{table*}


While LSD shows significant gains in complex naturalistic scenes, we also note in Table \ref{tab:simple_quant} and in Figure \ref{fig:seg} that LSD shows a somewhat inferior performance on visually simple datasets like CLEVR
in terms of segmentation and representation. 
% We show these results . 
LSD splits certain parts of the background \emph{e.g.}, in the upper region of the image, which contributes to its sub-optimal performance on this simple dataset. 
We conjecture that this is because the diffusion model can ignore the slot-conditioning when generating these regions, giving no learning signal to the Slot Attention encoder to group them into meaningful slots. 
This may be caused by the simplicity of the background and the fact that this dataset does not contain any object in these regions, allowing our strong decoder, to memorize the rendering process for that region independently of the input from the encoder. 



\section{Conclusion}
In this work, we proposed the Latent Slot Diffusion model which can be seen in two ways: (1) the first model combining the diffusion models in unsupervised object-centric learning and (2) the first unsupervised compositional diffusion model which does not require supervised annotation like text. The main lessons of the proposed model are that it outperforms the state-of-the-art transformer-based object-centric models in various object-centric tasks. Importantly, this superiority becomes more significant as the image becomes more complex. We also show that the quality of the unsupervised compositional generation is also superior to the transformer-based object-centric models. Therefore, we believe that this is a step forward toward object-centric learning that can handle complex naturalistic images, the current main challenge. However, we also found that for simple images, the diffusion-based model tends to underperform the transformer-based model, and thus would like to investigate this in the future.

\section*{Acknowledgements}

This work is supported by Brain Pool PlusProgram (No. 2021H1D3A2A03103645) and Young Researcher Program (No. 2022R1C1C1009443) through the National Research Foundation of Korea (NRF) funded by the Ministry of Science and ICT.

% \textbf{Do not} include acknowledgements in the initial version of
% the paper submitted for blind review.

% If a paper is accepted, the final camera-ready version can (and
% probably should) include acknowledgements. In this case, please
% place such acknowledgements in an unnumbered section at the
% end of the paper. Typically, this will include thanks to reviewers
% who gave useful comments, to colleagues who contributed to the ideas,
% and to funding agencies and corporate sponsors that provided financial
% support.

\bibliography{refs, refs_gs, refs_ahn, refs_jd}
\bibliographystyle{plain}


\end{document}
%\bibliography{SF-ML}


\end{document}
