%%%%%%%%%%%%%%%%%%%%%%%%%%%%%%%%%%%%%%%%%%%%%%%%%%%%%%%%%%%%%%%%%%%%%%%%%%%%%%%%
%2345678901234567890123456789012345678901234567890123456789012345678901234567890
%        1         2         3         4         5         6         7         8

\documentclass[letterpaper, 10 pt, conference]{ieeeconf}  % Comment this line out if you need a4paper

%\documentclass[a4paper, 10pt, conference]{ieeeconf}      % Use this line for a4 paper

\IEEEoverridecommandlockouts                              % This command is only needed if 
                                                          % you want to use the \thanks command

\overrideIEEEmargins                                      % Needed to meet printer requirements.

%In case you encounter the following error:
%Error 1010 The PDF file may be corrupt (unable to open PDF file) OR
%Error 1000 An error occurred while parsing a contents stream. Unable to analyze the PDF file.
%This is a known problem with pdfLaTeX conversion filter. The file cannot be opened with acrobat reader
%Please use one of the alternatives below to circumvent this error by uncommenting one or the other
%\pdfobjcompresslevel=0
%\pdfminorversion=4

% See the \addtolength command later in the file to balance the column lengths
% on the last page of the document

% The following packages can be found on http:\\www.ctan.org
\usepackage[T1]{fontenc}
\usepackage{graphics} % for pdf, bitmapped graphics files
\usepackage{epsfig} % for postscript graphics files
% \usepackage{mathptmx} % assumes new font selection scheme installed
\usepackage{times} % assumes new font selection scheme installed
\usepackage{amsmath} % assumes amsmath package installed
\usepackage{amssymb}  % assumes amsmath package installed 
\usepackage{url}
\usepackage{threeparttable}
\usepackage{multirow}
\usepackage{makecell}
\usepackage{bbold}
\usepackage{bbm}
\usepackage{hyperref}
\usepackage{tikz}
\usetikzlibrary{calc}
\usetikzlibrary{shapes.geometric, arrows, decorations.pathreplacing}
\usetikzlibrary{backgrounds}
\usepackage{xcolor}
\usepackage{adjustbox}
\usepackage{fancyhdr}

\usepackage{xcolor}



%
%

%
\definecolor{jnSUCardinalRed}{HTML}{8c1515}
\definecolor{jnSUCardinalRedLight}{HTML}{B83A4B}
\definecolor{jnSUCardinalRedDark}{HTML}{820000}
\definecolor{jnSUWhite}{HTML}{ffffff}
\definecolor{jnSUCoolGrey}{HTML}{53565A}
\definecolor{jnSUBlack}{HTML}{2e2d29}
\definecolor{jnSUBlack100}{HTML}{2e2d29}
\definecolor{jnSUBlack90}{HTML}{43423E}
\definecolor{jnSUBlack80}{HTML}{585754}
\definecolor{jnSUBlack70}{HTML}{6D6C69}
\definecolor{jnSUBlack60}{HTML}{767674}
\definecolor{jnSUBlack50}{HTML}{979694}
\definecolor{jnSUBlack40}{HTML}{ABABA9}
\definecolor{jnSUBlack30}{HTML}{C0C0BF}
\definecolor{jnSUBlack20}{HTML}{D5D5D4}
\definecolor{jnSUBlack10}{HTML}{EAEAEA}

%
\definecolor{jnSUPaloAlto}{HTML}{175E54}
\definecolor{jnSUPaloAltoLight}{HTML}{2D716F}
\definecolor{jnSUPaloAltoDark}{HTML}{014240}
\definecolor{jnSUPaloVerde}{HTML}{279989}
\definecolor{jnSUPaloVerdeLight}{HTML}{59B3A9}
\definecolor{jnSUPaloVerdeDark}{HTML}{017E7C}
\definecolor{jnSUOlive}{HTML}{8F993E}
\definecolor{jnSUOliveLight}{HTML}{A6B168}
\definecolor{jnSUOliveDark}{HTML}{7A863B}
\definecolor{jnSUBay}{HTML}{6FA287}
\definecolor{jnSUBayLight}{HTML}{8AB8A7}
\definecolor{jnSUBayDark}{HTML}{417865}
\definecolor{jnSUSky}{HTML}{4298B5}
\definecolor{jnSUSkyLight}{HTML}{67AFD2}
\definecolor{jnSUSkyDark}{HTML}{016895}
\definecolor{jnSULagunita}{HTML}{007C92}
\definecolor{jnSULagunitaLight}{HTML}{009AB4}
\definecolor{jnSULagunitaDark}{HTML}{006B81}
\definecolor{jnSUPoppy}{HTML}{E98300}
\definecolor{jnSUPoppyLight}{HTML}{F9A44A}
\definecolor{jnSUPoppyDark}{HTML}{D1660F}
\definecolor{jnSUSpirited}{HTML}{E04F39}
\definecolor{jnSUSpiritedLight}{HTML}{F4795B}
\definecolor{jnSUSpiritedDark}{HTML}{C74632}
\definecolor{jnSUIlluminating}{HTML}{FEDD5C}
\definecolor{jnSUIlluminatingLight}{HTML}{FFE781}
\definecolor{jnSUIlluminatingDark}{HTML}{FEC51D}
\definecolor{jnSUPlum}{HTML}{620059}
\definecolor{jnSUPlumLight}{HTML}{734675}
\definecolor{jnSUPlumDark}{HTML}{350D36}
\definecolor{jnSUBrick}{HTML}{651C32}
\definecolor{jnSUBrickLight}{HTML}{7F2D48}
\definecolor{jnSUBrickDark}{HTML}{42081B}
\definecolor{jnSUArchway}{HTML}{5D4B3C}
\definecolor{jnSUArchwayLight}{HTML}{766253}
\definecolor{jnSUArchwayDark}{HTML}{2F2424}
\definecolor{jnSUStone}{HTML}{7F7776}
\definecolor{jnSUStoneLight}{HTML}{D4D1D1}
\definecolor{jnSUStoneDark}{HTML}{544948}
\definecolor{jnSUFog}{HTML}{DAD7CB}
\definecolor{jnSUFogLight}{HTML}{F4F4F4}
\definecolor{jnSUFogDark}{HTML}{B6B1A9}

%
\definecolor{jnSUDigitalRed}{HTML}{B1040E}
\definecolor{jnSUDigitalRedLight}{HTML}{E50808}
\definecolor{jnSUDigitalRedDark}{HTML}{820000}
\definecolor{jnSUDigitalBlue}{HTML}{006CB8}
\definecolor{jnSUDigitalBlueLight}{HTML}{6FC3FF}
\definecolor{jnSUDigitalBlueDark}{HTML}{00548f}
\definecolor{jnSUDigitalGreen}{HTML}{008566}
\definecolor{jnSUDigitalGreenLight}{HTML}{1AECBA}
\definecolor{jnSUDigitalGreenDark}{HTML}{006F54}



%

%
%
%
%
%

%
%
%
%
%
%
%
%
%
%
%
%
%
%
%
%
%
%
%
%
%
%
%
%
%

%
%
%
%



%

\definecolor{myParula01Blue}{RGB}{0,114,189}
\definecolor{myParula02Orange}{RGB}{217,83,25}
\definecolor{myParula03Yellow}{RGB}{237,177,32}
\definecolor{myParula04Purple}{RGB}{126,47,142}
\definecolor{myParula05Green}{RGB}{119,172,48}
\definecolor{myParula06LightBlue}{RGB}{77,190,238}
\definecolor{myParula07Red}{RGB}{162,20,47}



\title{\LARGE \bf
Penalty-Based Imitation Learning With Cross Semantics Generation Sensor Fusion for Autonomous Driving
}

\author{Hongkuan Zhou$^{2\, *}$, Aifen Sui $^{1\, \dagger}$, Letian Shi$^{2\, *}$, and Yinxian Li$^{2\,}$% <-this % stops a space
\thanks{$^{1}$A. Sui  is at the Trustworthy Technology and Engineering Laboratory, Huawei Munich Research Center.}%
\thanks{$^{2}$H. Zhou, L. Shi and Y. Li are students at TUM School of Computation, Information and Technology, Technical University of Munich. This work was done during their internship at Huawei Munich Research Center.}%
\thanks{$^{\dagger}$Corresponding author: A. Sui {\tt\small aifen.sui@huawei.com}}
\thanks{$^{*}$Equally contribution.}
}

\begin{document}



\maketitle
\thispagestyle{fancy}
\lhead{Accepted by ITSC 2023}
\pagestyle{empty}


%%%%%%%%%%%%%%%%%%%%%%%%%%%%%%%%%%%%%%%%%%%%%%%%%%%%%%%%%%%%%%%%%%%%%%%%%%%%%%%%

\begin{abstract}

What will Wi-Fi 8 be? 
Driven by the strict requirements of emerging applications, next-generation Wi-Fi is set to prioritize Ultra High Reliability (UHR) above all. 
In this paper, we explore the journey towards IEEE 802.11bn UHR, the amendment that will form the basis of Wi-Fi\,8. 
%[Lorenzo: \footnote{At the time of writing, 802.11bn is the name most likely to be adopted for the new Wi-Fi amendment. This will be confirmed prior to final publication.}]
%After providing an overview of the nearly completed Wi-Fi\,7 standard, 
We first present new use cases calling for further Wi-Fi evolution and also outline current standardization, certification, and spectrum allocation activities, sharing updates from the newly formed UHR Study Group.
We then introduce the disruptive new features envisioned for Wi-Fi\,8 %802.11bn
and discuss the associated research challenges. Among those, we focus on access point coordination and demonstrate that it could build upon 802.11be multi-link operation to make Ultra High Reliability a reality in Wi-Fi\,8.
\end{abstract}

%%%%%%%%%%%%%%%%%%%%%%%%%%%%%%%%%%%%%%%%%%%%%%%%%%%%%%%%%%%%%%%%%%%%%%%%%%%%%%%%
\section{Introduction}\label{section:Introduction}
\glspl{cps} integrate real-time computing and communication capabilities with monitoring and control actions over components in the physical world~\cite{shi_survey_2011}. To face the harshness of the space environment, modern space systems such as satellites and spacecraft require tight coupling between onboard processing, communication (cyber), sensing, and actuation (physical) elements~\cite{klesh_cyber-physical_2012}. The orbit determination and control subsystems on a small spacecraft or in satellites' constellations provide a clear link between onboard processing and sensing elements of the spacecraft's physical environment~\cite{di_mascio_-board_2021}, becoming increasingly critical as small spacecrafts become ever more capable~\cite{klesh_cyber-physical_2012}. In this scenario, digital computing systems representing the decisional part of a \gls{scps} must be designed to be reliable and tolerate faults induced by cosmic radiation. Radiation-induced soft errors such as \glspl{set} and \glspl{seu} can occur more frequently in space than at ground level, creating the need for additional hardware to mitigate detrimental effects on the system~\cite{wachter_survey_2019}.

Various solutions exist to protect electronics from the adverse effects of radiation~\cite{wachter_survey_2019}. Costly radiation-hardened technologies, insulating techniques~\cite{alles_radiation_2011}, and polymer shielding~\cite{shahzad_views_2022} help mitigate soft errors. It is also possible to enhance the fault tolerance capabilities of digital systems by introducing redundancy at different levels in their design flow. Temporal redundancy techniques rely on repeated executions of the same work to determine the correct result~\cite{feng_shoestring_2010}. Spatial or modular redundancy techniques rely on multiple hardware blocks executing the same task and comparing the results~\cite{ginosar_survey_2012}. These approaches rely on rigid schemes for repetition in space and time of redundant blocks or tasks, hence they can severely impact the \gls{ppa} of computing platforms.

The increasing demand for strong processing capabilities in space~\cite{xie_resource-cost-aware_2018} is pushing researchers toward lower-overhead solutions. In recent years, the advent of RISC-V and open-source hardware has encouraged the development of high-performance \glspl{soc} for various domains without licensing or other restrictions. This includes the space domain, where custom modifications to improve properties such as reliability~\cite{di_mascio_open-source_2021} and fault tolerance are often required. Among proposed architectures, heterogeneous systems with multi-core computing clusters have gained traction in the space industry~\cite{ginosar_rc64_2016} due to increased performance and flexibility for computation and \gls{dsp} workloads~\cite{kurth_hero_2018}.
While multiple processors offer increased performance for parallelizable tasks, they also provide a unique opportunity for reliability enhancements: multiple cores can execute identical tasks, comparing their results to detect and react to faults.

In this paper, we introduce a space-ready multi-core RISC-V-based computing system featuring a \gls{hmr} approach. We leverage the independent cores available in a multi-core RISC-V cluster for redundant execution in a dynamically runtime configurable manner and introduce \gls{dcls} and \gls{tcls} modes, extending the On-Demand Redundancy Grouping with \gls{tcls} configurable under reset presented in~\cite{rogenmoser_-demand_2022}.
Our design allows each application to configure its reliability setting according to its requirements, possibly decided at runtime.
% Without sacrificing performance in the general case, this architecture allows for the safe execution of a safety-critical section at a 2.3\% area overhead.
Furthermore, we implement two recovery alternatives, software and hardware-assisted, comparing their impact on the hardware resources and performance in case of a fault. The checking, voting, and switching hardware in the implemented design does not affect the internals of the processor core, allowing for the use of verified RISC-V processor cores without requiring any internal (potentially erroneous) modifications to rapidly build a reliable system.

% The key contributions of this paper are:
% \begin{itemize}
%     \item Combined Dual Core Lockstep and Triple Core Lockstep extensions within a RISC-V-based multi-core cluster.
%     \item Design of a split-lock mechanism to enable runtime-selectable redundant configurations switching in just 550 clock cycles between the available redundant modes. 
%     % \item Exploring the trade-offs for \gls{dmr} and \gls{tmr} on a core-level perspective of the implemented cluster and relating this to classical redundancy mechanisms.
%     \item Design of hardware extensions for fast fault recovery in just 24 clock cycles with only $\sim9\%$ area impact and no timing and performance impacts over the original architecture.
% \end{itemize}
% We propose the first system integrating these functionalities on an open-source RISC-V-based multi-core cluster for fine-tunable reliability vs. performance trade-offs. The key contributions of this paper are:
In summary, we introduce the following key contributions:
\begin{itemize}
    \item A re-configurable computing cluster for Dual-Core Lockstep and Triple-Core Lockstep execution capable of tackling compute-intensive and safety-critical applications. The proposed cluster can be configured so that the computing cores can operate independently if the application requires high-performance capabilities or in Dual/Triple-Core Lockstep mode, depending on the criticality of the executed task.
    \item Robust hardware support for fast fault recovery execution, featuring dedicated Error-Correcting Codes-protected registers to restore the state of the computing cores to the closest reliable state in time. This feature allows the cores to perform cycle-by-cycle backups of their internal state in the protected registers, reducing by $15\times$ the required time to recover from a fault  over the software-based approach.
    \item A runtime-programmable split-lock mechanism allowing for fast switching and re-configuration between the available redundant modes. With these features, it is possible to explicitly define portions of code within a \textit{safety-critical section}, configuring the cores for safe lockstep execution with minimum configuration switching overhead.
\end{itemize}

To validate our proposed approach, we implemented the RISC-V cluster in Global Foundries 22~\si{\nano\meter} technology, achieving up to \SI{430}{\mega\hertz} operating frequency and \SI{1160}{\mega OPS} when configured in independent mode and 617 and 414 MOPS in \gls{dmr} and \gls{tmr} mode, respectively. With only software-based recovery features, the proposed cluster occupies \SI{0.612}{\milli\meter\squared} with just 1.3\% area overhead over the non-redundant configuration, featuring 363 clock cycles time-to-recovery in triple mode. When enhanced with hardware-based recovery features, it provides rapid fault recovery in just 24 clock cycles occupying \SI{0.660}{\milli\meter\squared}, $\sim$9.4\% area overhead over the baseline RISC-V cluster. The proposed split-lock mechanism allows for entering and exiting a redundant mode in just 390 clock cycles for safety-critical code execution.
To foster future research in space-ready computer architecture, we release the proposed architecture as the first fully open-source RISC-V-based multi-core cluster with a finely tunable trade-off between reliability and performance.

\section{related works}
\subsection{End-to-End Autonomous Driving}
Today's autonomous driving technologies have two main branches, modular and end-to-end approaches. Modular approaches apply a fine-grained pipeline of software modules working together to control the vehicle. In contrast, the entire pipeline of end-to-end driving is treated as one single learning task. End-to-end approaches have shown great success in computer vision tasks, such as object detection \cite{7112511} \cite{Fast_RCNN} \cite{Faster_RCNN} \cite{redmon2016you}, object tracking \cite{brasó2020learning}, and semantic segmentation \cite{ronneberger2015u}\cite{DBLP:journals/corr/CicekALBR16}. The success of these tasks builds a solid foundation for end-to-end autonomous driving. It is reasonable to believe end-to-end approaches are capable of solving autonomous driving problems in the near future. 
%A remarkable advantage of end-to-end autonomous driving is its sustainable learning capability. The well-designed model can achieve better performance by feeding more data in different scenarios. Imagine millions of vehicles on the road every day, tons of data can be collected every second. After processing, these data can be used as training material for the model to improve its performance. 
The most common learning methods for end-to-end autonomous driving are imitation learning \cite{TransFuser}, \cite{latefusion}, \cite{Filos2020CanAV}, \cite{Chitta2021ICCV(NEAT)}, \cite{chen2022lav}, \cite{8813900} and reinforcement learning  \cite{DBLP:journals/corr/abs-2001-08726} \cite{CPRL}.

\subsection{Safety Mechanism in End-to-End Autonomous Driving}
In the realm of autonomous driving, a key challenge is implementing safety mechanisms that can prevent accidents and protect passengers, pedestrians, and other road users. Within the framework of imitation learning, the agent learns driving skills by emulating expert demonstrations. The quality of these demonstrations has a significant impact on the agent's ability to drive safely in traffic. To improve the safety of the autonomous driving agent, researchers in \cite{TransFuser+} focus on enhancing the quality of the expert agent, while those in \cite{shao2022interfuser} introduce an additional safety module that filters out potentially dangerous driving behaviors generated by the network. Our contribution is to introduce the ``Penalty'' concept to the imitation learning framework, which incentivizes the trained agent to adopt safer driving behaviors.

\subsection{Multi-sensor Fusion Technologies}
Sensor Fusion technologies are commonly employed for 3D object detection and motion forecasting. Among the various types of sensors that can be integrated, the fusion of LiDAR and camera sensors is most frequently employed, where LiDAR data serves as a supplement to image data, providing additional information about the surrounding environment and improving data reliability due to its consistency in various environments. There are three branches of sensor fusion: early fusion, middle fusion, and late fusion. In early fusion, the data is fused before being fed into the learnable system, which is the most efficient approach. In middle fusion, the information is merged in the middle of the network, and the fused features are used to produce task-specific outputs. Late fusion is an ensemble learning method that combines the outputs generated by each modality into a final result. 

Multi-sensor fusion has received much research attention in the field of end-to-end autonomous driving. Prior works such as LateFusion\cite{latefusion} used a large Multi-Layer Perception (MLP) network to process the features extracted by the perception networks of LiDAR and RGB inputs. This MLP layer takes the tasks of features weighting, selection, and fusion which makes it hard to 
capture a global context of multi-modality inputs. TransFuser\cite{TransFuser} provides an approach that leverages the attention mechanism to fuse the LiDAR and RGB information. They used the transformer architecture to achieve the multi-modality global context. The Transformer-based fusion model is applied to different resolutions between the LiDAR and RGB perception networks. TransFuser+ \cite{TransFuser+}, as an extension of TransFuser, introduced more headers in the neural networks which incorporate four auxiliary tasks: depth prediction and semantic segmentation from the image branch; HD map prediction, and vehicle object detection from the BEV branch. These auxiliary tasks help to visualize the black box of the whole network. In addition, this approach also guarantees important information flow in the latent space because the information contained in the latent space should not only be able to complete the navigation task but also manually pre-defined auxiliary tasks.

\section{Methodologies}

In this section, we propose a novel multi-sensor fusion approach and a penalty-based Imitation Learning paradigm for end-to-end autonomous driving.
\subsection{Problem Setting}
The task we concentrate on is point-to-point navigation in an urban setting where the goal is to complete a route with safe reactions to dynamic agents such as moving vehicles and pedestrians. The traffic rules should also be followed. 

\textbf{Imitation Learning (IL):} Imitation Learning can learn a policy $\pi$ that clone the behavior of an expert policy $\pi^*$. In our setup, the policy is conditioned on the multi-modalities inputs of current observations. We used the Behavior Clone (BC) approach of IL. An expert policy is applied in the environment to collect a large dataset $\mathcal{D}=\{(\textbf{x}^i, \textbf{w}^i)\}_{i=1}^{Z}$ with the size of $Z$, which contains the observation of the environment $\textbf{x}^i$ and a set of waypoints $\textbf{w}^i$ in the future timesteps. The objective function is defined as:
\begin{equation}
    \label{eq:objective function}
    \mathcal{F} = \mathbb{E}_{(\mathcal{X},\mathcal{W})\sim \mathcal{D}}[\mathcal{L}(\mathcal{W}, \pi(\mathcal{X}))]
\end{equation}
where $\mathcal{L}$ is the loss function. 

In our setting, the observation $\mathcal{X}$ consists of one RGB image and one LiDAR point cloud from the current time step. We used only one single frame since other works since \cite{DBLP:journals/corr/abs-1905-06937}, \cite{DBLP:journals/corr/abs-1812-03079} have shown that using multiple frames does not improve the information gain much. A PID controller $\mathcal{I}$ is applied to perform low-level control, i.e. steer, throttle, and brake based on these predicted future waypoints. 

\textbf{Global Planner:} According to CARLA \cite{Dosovitskiy17} 0.9.10's protocol, the high-level goal locations $G$ is provided as GPS coordinates. This goal location $G$ is sparse (hundreds of meters apart) which can only be used as guidance. In contrast, those to be predicted waypoints are dense, only a few meters way away from each other. 

\begin{figure*}[ht]
	\centering
	\includegraphics[width=1.0\linewidth]{figures/Graph2.pdf}
	\caption{\textbf{Architecture.} The top-down LiDAR pseudo image and front camera image go through two residual networks to extract 512 dimension feature vectors. We use four different MLPs to extract the shared features and the unique features. The unique features of RGB input are used to generate stop signs and traffic light indicators. The shared features of LiDAR are used to reconstruct the segmentation of RGB input while the shared features of RGB are used to reconstruct the segmentation of top-down LiDAR input. An alignment loss is used to align the shared features from LiDAR and camera inputs into the same space. These shared features and unique features are concatenated along with the measurements (velocity, throttle, steer, brake from the last frame) and then go through one MLP to reduce the size. Finally, they will be fed into one GRU decoder to predict short-term waypoints.}
	\label{fig:Architecture}
\end{figure*}

\subsection{Cross Semantics Generation}

The motivation of our approach is based on the fact that multi-modal inputs have shared semantic information and also unique information. For instance, the geometric attributes and spatial coordinates of both vehicles and pedestrians are shared information that can be extracted from both LiDAR and RGB input. Figure \ref{fig:Shared semantic features} demonstrates the shared information of LiDAR and RGB input. The unique information refers to the complementary information that other inputs do not have. In the case of RGB input, unique information often pertains to features such as the color of traffic lights, patterns on traffic signs, and similar attributes. On the other hand, in the context of LiDAR input, unique information pertains to spatial relationships of objects. Our multi-sensor fusion approach aims to extract and align the shared features from LiDAR and RGB input sources so that the later decision network can leverage the organized features to achieve better performance.

To extract the shared information from LiDAR and RGB inputs, we propose cross semantics generation sensor fusion. As Figure \ref{fig:Architecture} demonstrates, the front RGB and top-down pre-processed LiDAR pseudo images will first be fed into two residual networks \cite{DBLP:journals/corr/HeZRS15} to extract the corresponding RGB and LiDAR features. Note that the LiDAR point cloud is pre-processed into the bird's eye view pseudo images which is the same setting as \cite{TransFuser}. We use four different linear layers to extract the shared features and unique features of LiDAR and RGB. The shared features of  RGB are used to generate the top-down semantic segmentation align with LiDAR input; The shared features of LiDAR are used to generate the semantic segmentation of corresponding RGB input. We refer this approach as cross semantics generation since the information from one modality is utilized to generate semantic representations of the other modality. In this way, the information flow is said to be 'crossed', as each modality contributes to the understanding of the other. The extracted shared features will maximized since the information derived from one modality should strive to generate an accurate semantic segmentation of the other modality to the best of its ability. An extra L2 loss is introduced to align the shared features of RGB and LiDAR into the same latent space. In our setup, the semantic segmentation contains 4 channels, the drivable area, the non-drivable area, the object in the drivable areas like vehicles and pedestrians, and others. In terms of the unique features from RGB input, we mainly concentrate on the traffic lights and stop signs. As we can see from the figure, the unique features from RGB input are used to train the traffic light and stop sign indicator which ensures the important information flows of traffic lights and stop signs in the neural network. These headers are also critical for later penalty-based Imitation which we will discuss in the following sections.

\subsection{Waypoint Prediction Network}
As shown in Figure \ref{fig:Architecture}, all the unique and shared features are concatenated into a 512-dimensional feature vector. This vector is fed into an MLP to reduce the dimension to 64 for computational efficiency reasons. The hidden layer of GRU is initialized with a 64-dimensional feature vector. GRU’s update gate controls the information flow from the hidden layer to the output. In each timestep, it also takes the current location and goal location as input. We follow the approach of \cite{Filos2020CanAV} that a single GRU layer is followed by a linear layer which takes the state of the hidden layer and predicts the relative position of the waypoint compared to the previous waypoint for $T=4$ time-steps. Hence, the predicted future waypoints are formed as $\{w_t = w_{t-1} + \delta w_t\}_{t=1}^T$. The start symbol for GRU is given by (0,0). 

\textbf{Controller}: Based on the predicted waypoints, we use two PID controllers for lateral and longitudinal directions respectively. We follow the settings of \cite{chen2019lbc}.

\subsection{Loss Functions}
Similar to previous works \cite{TransFuser}, \cite{chen2019lbc}, we also use $L_1$ loss as our reconstruction loss. For each input, the loss function can be formalized as:
\begin{equation}
    \mathcal{L} = \sum_{t=1}^{T} ||w_t - w_t^{gt}||_1
\end{equation}
where $w_t$ is the t-th predicted waypoints and $w_t^{gt}$ is the t-th ground truth waypoint produced by the expert policy. 

\textbf{Auxiliary Tasks}: In our cross semantics generation approach, we have four extra auxiliary tasks along with the main imitation learning task. As we explained in the above section, two of the auxiliary tasks are semantic segmentation. In order to ensure some important information flow in the network, we introduce two extra classification headers, namely traffic light classification and stop sign classification. These two headers help the neural network to capture traffic light and stop sign information which is significant for later penalty-based Imitation learning.

\textbf{Front View Semantics}. Front-view semantic segmentation has four different channels. We define $y_f$ as the ground truth 3D tensor with the dimension $H_f \times W_f \times 4$ and $\hat{y}_f$ as the output of the front view decoder with the same shape.  

\textbf{Top-down View Semantics}. Like front-view semantic segmentation, top-down-view semantic segmentation also has four channels. We define $y_{td}$ as the ground truth 3D tensor with the dimension $H_{td} \times H_{td} \times 4$ and $\hat{y}_{td}$ as the output of the top-down view decoder with the same shape. 

\textbf{Image-LiDAR Alignment Loss}. This loss aims to align the shared semantic features of Image and LiDAR into the same latent space. We use an L2-loss to align these features. 

\textbf{Traffic Light Classification}. The output of the traffic light decoder should be a vector of 4 which indicates these four states red light, yellow light, green light, and none in the current frame. We then define $y_l$ as the ground truth traffic light vector of length 4 and $\hat{y}_l$ as the output of the traffic light decoder with the same shape. 

\textbf{Stop Sign Classification}. The output of the stop sign decoder should have a vector of 2 which indicates if a stop sign exists in the current frame. The ground truth stop sign vector of length 2 and the output of the stop sign decoder with the same shape are defined as $y_s$ and $\hat{y}_s$, respectively. 
Based on what we defined above, the new loss function is given by:
\begin{equation}
\begin{aligned}
        \mathcal{L} = & \sum_{t=1}^{T} ||w_t - w_t^{gt}||_1 + \omega_f \mathcal{L}_{\rm CE}(y_{f},\hat{y}_{f}) + \\
        & \omega_{td} \mathcal{L}_{\rm CE}(y_{td},\hat{y}_{td}) + \omega_{l} \mathcal{L}_{\rm CE}(y_{l}, \hat{y}_{l}) +  \\ 
        & \omega_{s} \mathcal{L}_{\rm CE}(y_{s}, \hat{y}_{s}) + \omega_{a} \mathcal{L}_{2}(y_{s}, \hat{y}_{s}) 
\end{aligned}
\end{equation}
 where $\mathcal{L}_{CE}$ and $\mathcal{L}_2$ are the cross entropy loss and L2 loss, respectively. $\omega_f$, $\omega_{td}$, $\omega_{l}$,  $\omega_s$, $\omega_a$ are the weights for these auxiliary losses.


\subsection{Penalty-based Imitation Learning}
\begin{figure*}[ht]
	\centering
	\includegraphics[width=0.8\linewidth]{figures/three_penalties.pdf}
	\caption{\textbf{Penalty Illustration.} To ensure compliance with red light and stop penalty rules, as well as promoting deceleration during turning maneuvers, our approach incorporates three distinct penalty types. The first column of the figures exemplifies the red light penalty, wherein waypoints situated beyond the stop line receive a penalty when the traffic light is red. In the second column, we demonstrate the stop sign penalty, wherein predicted waypoints within the vicinity of a stop sign are penalized if the agent fails to decelerate adequately. The speed penalty is enforced during turning actions as shown in last two figures. Specifically, if the predicted waypoints indicate an excessive speed, a speed penalty is imposed. }
	\label{fig:penalties}
\end{figure*}
We found that the objective function design for imitation learning and the autonomous driving metric are not unified which means a low loss of the objective function does not guarantee a high driving score and high route completion. After careful study, we figure out there exist two potential reasons.

\begin{itemize}
    \item The expert agent still makes mistakes when generating the dataset. Sometimes, the expert agent runs a red light and violates the stop sign rule.
    \item The objective function is not sensitive to serious violations of the traffic rules, i.e. the violation of red lights and stop signs. The average objective function loss may not increase too much when violating the traffic rules despite that this violation may cause serious consequences which result in a huge drop in driving score and route completion.
\end{itemize}

Behavior Cloning (BC), as an imitation learning method, aims to clone the behavior of the expert agent. In such a way, the performance of the trained agent can no longer be better than the expert agent. If the expert agent makes a mistake, the trained agent will learn how to make that mistake instead of getting rid of that mistake.

Our aim is to reformulate the objective function of imitation learning in line with traffic rules, whereby the agent is penalized (higher loss) when it generates short-term future waypoints that violate the traffic rules during the training process.

The traffic rules can be modeled as constrained functions which refer to conditions of the optimization problem that the solution must satisfy. In our setting, we concentrate on two kinds of traffic rule violations and one common driving experience, namely red light violations, stop sign violations, and slowing down when turning, because these are the main problems we found in our vanilla imitation learning approach. We first define three corresponding penalties to quantify these violations. Figure \ref{fig:penalties} illustrates these penalties. 

\subsubsection{Red Light Penalty}
For the red light violation, we design a red light penalty as follows:
\begin{equation}
    {\mathcal{P}}_{\rm tl} = \mathbb{E}_{\mathcal{X}\sim \mathcal{D}}[\mathbb{1}_{\rm red}\cdot\sum_{i=1}^{t}c_i\cdot {\rm max}\{0, w_{i} - \overline{p}\} ]
\end{equation}
where $w_{i}$ is the $i$-th predicted waypoints of the trained agent; $\overline{p}$ is the position of the stop line at the intersection. Both $w_{i}, \overline{p}$ are in the coordinate system of the ego car. $c_i$ is the weight parameter and $\sum_{i} c_i =1$.$\mathbb{1}_{\rm red}$ indicates if a red light that influences the agent exists in the current frame. $\mathcal{X}$ is the input of the current frame and $\mathcal{D}$ is the whole data set. 

In the scenarios of red lights, an extra red light penalty is defined by the distances of the predicted waypoints beyond the stop line at the intersection. If the predicted waypoints are within the stop line, then the penalty remains zero. On the other hand, if the predicted waypoints are beyond the stop line, the sum of distances between those waypoints and the stop line will be calculated as the red light penalty. The additional information for the red light penalty calculation like traffic light information and stop line location is pre-processed and saved in each frame of our dataset.

\subsubsection{Stop Sign Penalty}
Similar to the red light penalty, a stop sign penalty is given when the predicted waypoints violate the stop sign rule. The penalty is formalized as follows:
% \begin{equation}
%     {\mathcal{P}}_{\rm ss}=\mathbb{E}_{X \sim D}[\mathbb{1}_{stopsign \land v > \epsilon}\cdot \sum_{i=1}^{n} c_i \cdot min\{0, p_{i, pred}-p_{stopsign}\}]
% \end{equation}

\begin{equation}
    {\mathcal{P}}_{\rm ss}=\mathbb{E}_{\mathcal{X} \sim \mathcal{D}}[\mathbb{1}_{\rm stopsign}\cdot  {\rm max}\{v - \epsilon, 0\}]
\end{equation}
where $v$ is the desired speed calculated by 
\begin{equation}
    v = \frac{w_{0} - w_{1}}{\Delta t}
    \label{eq: desired speed}
\end{equation}
$w_{0}$ and $w_{1}$ is the first and second predicted waypoint, and $\Delta t$ is the time interval between each frame when collecting the data. $\mathbb{1}_{\rm stopsign}$ is an indicator for stop sign checking. If the vehicle drives into the area that a stop influences, this indicator turns to 1 otherwise it remains zero. $\epsilon$ is the maximum speed required to pass stop sign tests. 

\subsubsection{Speed Penalty}
A speed penalty will be applied if the agent attempts to turn at excessive speed. The motivation to introduce this speed is based on the common driving experience of human beings. Also, we observe the agent sometimes can not avoid hitting pedestrians when turning at high speed since it has less time to react. The penalty is formalized as:
\begin{equation}
    {\mathcal{P}}_{\rm sp}=\mathbb{E}_{\mathcal{X} \sim \mathcal{D}} [{\rm sin}(d\theta) \cdot {\rm max} \{v - v_{\rm lb}, 0\}]
\end{equation}
where $d\theta$ is the direction deviation between the current frame and the next frame. Like stop sign penalty, $v$ is defined in \eqref{eq: desired speed}. $v_{\rm lb}$ is the speed lower bound. Speed under the lower bound will not be imposed by speed punishment. 
% where $p_{i,pred}$ is the i-th predicted waypoints; $\mathcal{S}$ is the drivable street area. $d(\cdot, \cdot)$ is the distance between the waypoint and the street. $c_i$ is the weight factor. 

% As the above formula demonstrated, an additional penalty is added if the predicted waypoints are out of the street. The penalty is calculated as the weighted sum of the distances between those out of the street waypoints and the street. 

With the help of these penalties, the constrained optimization can be formalized as:

\begin{equation}
\label{eq:constraint objective function}
\begin{aligned}
\min \quad & \mathcal{F} \\
\textrm{s.t.} \quad & \mathcal{P}_{\rm tl}, \mathcal{P}_{\rm ss},\mathcal{P}_{\rm sp} = 0\\
\end{aligned}
\end{equation}
where $\mathcal{F}$ is the objective function defined in \eqref{eq:objective function}.

The Lagrange multiplier strategy can be applied here. We introduce three Lagrange Multiplier $\lambda_1$, $\lambda_2$, $\lambda_3$ and the Lagrange function is defined by:
\begin{equation}
\begin{aligned}
\min \quad \mathcal{F} + \lambda_1 \mathcal{P}_{\rm tl}+ \lambda_2 \mathcal{P}_{\rm ss} +\lambda_3 \mathcal{P}_{\rm sp}
\end{aligned}
\end{equation}
This is the final objective function to optimize. For simplicity, these Lagrange multipliers $\lambda_1$, $\lambda_2$, $\lambda_3$ are considered fixed hyper-parameters. Well-chosen $\lambda_1$, $\lambda_2$, $\lambda_3$ are important for optimization. According to our experiments, too large $\lambda$ influences the behaviors in other scenarios while too smaller $\lambda$ is not powerful enough for the agent to obey the corresponding traffic rules. 

The red light indicator and stop sign indicator headers are important for the agent to learn from the stop sign and red light penalty because the information flow of the stop sign and red light helps the agent to build the logistic connection between behavior, observation, and punishment. 
\section{Experiments}
\subsection{Experiment Setup}

\paragraph{Tasks and Datasets.}
We validate \modelName~using three types of experiments. 
We first conduct multi-view reconstruction for real-world watertight objects to ensure that \modelName~achieves comparable reconstruction quality on watertight surfaces. We conduct this experiment on 10 scenes from the \textit{DTU Dataset}~\cite{dtu}. Each scene contains $49$ or $64$ RGB images and masks with a resolution of $1600\times 1200$.
Second, we reconstruct open surfaces from multi-view images. We run this experiment on eight categories from the \DFD~\cite{zhu2020deep} and five categories from the \MGN~\cite{bhatnagar2019mgn}, which contain clothes with a wide variety of materials, appearance, and geometry, including challenging cases for reconstruction algorithms, such as camisoles.
Finally, we construct an autoencoder, which takes a single image as the input and provides validation on the challenging task of single-view reconstruction on open surfaces. We conduct this experiment on the \textit{dress} category from the \DFD~\cite{zhu2020deep}. We randomly select 116 objects as the training set and 25 objects as the test set.
All experiments are compared with the SOTA methods for better verification.
{To avoid thin closed reconstructions during the training process, we employ a smaller learning rate for the SDF-Net and a larger learning rate for the Validity-Net.}
Please refer to the implementation of \netName{} in the supplementary.
\vspace{-3mm}

\begin{figure*}[htb]
	\vspace{0.0mm}
	\begin{minipage}[t]{0.13\textwidth}
		\centering
		\includegraphics[width=0.9\textwidth]{Figure/comparison_open_d3d/long_sleeve_upper/252/crop/gt_marked.png}
	\end{minipage}
	\begin{minipage}[t]{0.06\textwidth}
		\centering
		\includegraphics[width=1.0\textwidth]{Figure/comparison_open_d3d/long_sleeve_upper/252/crop/gt.png}
	\end{minipage}
	\begin{minipage}[t]{0.13\textwidth}
		\centering
		\includegraphics[width=0.9\textwidth]{Figure/comparison_open_d3d/long_sleeve_upper/252/crop/ours_marked.png}
	\end{minipage}
	\begin{minipage}[t]{0.06\textwidth}
		\centering
		\includegraphics[width=1.0\textwidth]{Figure/comparison_open_d3d/long_sleeve_upper/252/crop/ours.png}
	\end{minipage}
	\begin{minipage}[t]{0.13\textwidth}
		\centering
		\includegraphics[width=0.9\textwidth]{Figure/comparison_open_d3d/long_sleeve_upper/252/crop/neus_marked.png}
	\end{minipage}
	\begin{minipage}[t]{0.06\textwidth}
		\centering
		\includegraphics[width=1.0\textwidth]{Figure/comparison_open_d3d/long_sleeve_upper/252/crop/neus.png}
	\end{minipage}
	\begin{minipage}[t]{0.13\textwidth}
		\centering
		\includegraphics[width=0.9\textwidth]{Figure/comparison_open_d3d/long_sleeve_upper/252/crop/idr_marked.png}
	\end{minipage}
	\begin{minipage}[t]{0.06\textwidth}
		\centering
		\includegraphics[width=1.0\textwidth]{Figure/comparison_open_d3d/long_sleeve_upper/252/crop/idr.png}
	\end{minipage}
	\begin{minipage}[t]{0.13\textwidth}
		\centering
		\includegraphics[width=0.9\textwidth]{Figure/comparison_open_d3d/long_sleeve_upper/252/crop/HFS_marked.png}
	\end{minipage}
	\begin{minipage}[t]{0.06\textwidth}
		\centering
		\includegraphics[width=1.0\textwidth]{Figure/comparison_open_d3d/long_sleeve_upper/252/crop/HFS.png}
	\end{minipage}
	\\
	\vspace{0.0mm}
	\begin{minipage}[t]{0.13\textwidth}
		\centering
		\includegraphics[width=0.9\textwidth]{Figure/comparison_open_d3d/no_sleeve_upper/323/crop/gt_marked.png}
	\end{minipage}
	\begin{minipage}[t]{0.06\textwidth}
		\centering
		\includegraphics[width=1.0\textwidth]{Figure/comparison_open_d3d/no_sleeve_upper/323/crop/gt.png}
	\end{minipage}
	\begin{minipage}[t]{0.13\textwidth}
		\centering
		\includegraphics[width=0.9\textwidth]{Figure/comparison_open_d3d/no_sleeve_upper/323/crop/ours_marked.png}
	\end{minipage}
	\begin{minipage}[t]{0.06\textwidth}
		\centering
		\includegraphics[width=1.0\textwidth]{Figure/comparison_open_d3d/no_sleeve_upper/323/crop/ours.png}
	\end{minipage}
	\begin{minipage}[t]{0.13\textwidth}
		\centering
		\includegraphics[width=0.9\textwidth]{Figure/comparison_open_d3d/no_sleeve_upper/323/crop/neus_marked.png}
	\end{minipage}
	\begin{minipage}[t]{0.06\textwidth}
		\centering
		\includegraphics[width=1.0\textwidth]{Figure/comparison_open_d3d/no_sleeve_upper/323/crop/neus.png}
	\end{minipage}
	\begin{minipage}[t]{0.13\textwidth}
		\centering
		\includegraphics[width=0.9\textwidth]{Figure/comparison_open_d3d/no_sleeve_upper/323/crop/idr_marked.png}
	\end{minipage}
	\begin{minipage}[t]{0.06\textwidth}
		\centering
		\includegraphics[width=1.0\textwidth]{Figure/comparison_open_d3d/no_sleeve_upper/323/crop/idr.png}
	\end{minipage}
	\begin{minipage}[t]{0.13\textwidth}
		\centering
		\includegraphics[width=0.9\textwidth]{Figure/comparison_open_d3d/no_sleeve_upper/323/crop/HFS_marked.png}
	\end{minipage}
	\begin{minipage}[t]{0.06\textwidth}
		\centering
		\includegraphics[width=1.0\textwidth]{Figure/comparison_open_d3d/no_sleeve_upper/323/crop/HFS.png}
	\end{minipage}
	\\
	\vspace{0.0mm}
	\begin{minipage}[t]{0.13\textwidth}
		\centering
		\includegraphics[width=0.9\textwidth]{Figure/comparison_open_d3d/short_sleeve_dress/63/crop/gt_marked.png}
	\end{minipage}
	\begin{minipage}[t]{0.06\textwidth}
		\centering
		\includegraphics[width=1.0\textwidth]{Figure/comparison_open_d3d/short_sleeve_dress/63/crop/gt.png}
	\end{minipage}
	\begin{minipage}[t]{0.13\textwidth}
		\centering
		\includegraphics[width=0.9\textwidth]{Figure/comparison_open_d3d/short_sleeve_dress/63/crop/ours_marked.png}
	\end{minipage}
	\begin{minipage}[t]{0.06\textwidth}
		\centering
		\includegraphics[width=1.0\textwidth]{Figure/comparison_open_d3d/short_sleeve_dress/63/crop/ours.png}
	\end{minipage}
	\begin{minipage}[t]{0.13\textwidth}
		\centering
		\includegraphics[width=0.9\textwidth]{Figure/comparison_open_d3d/short_sleeve_dress/63/crop/neus_marked.png}
	\end{minipage}
	\begin{minipage}[t]{0.06\textwidth}
		\centering
		\includegraphics[width=1.0\textwidth]{Figure/comparison_open_d3d/short_sleeve_dress/63/crop/neus.png}
	\end{minipage}
	\begin{minipage}[t]{0.13\textwidth}
		\centering
		\includegraphics[width=0.9\textwidth]{Figure/comparison_open_d3d/short_sleeve_dress/63/crop/idr_marked.png}
	\end{minipage}
	\begin{minipage}[t]{0.06\textwidth}
		\centering
		\includegraphics[width=1.0\textwidth]{Figure/comparison_open_d3d/short_sleeve_dress/63/crop/idr.png}
	\end{minipage}
	\begin{minipage}[t]{0.13\textwidth}
		\centering
		\includegraphics[width=0.9\textwidth]{Figure/comparison_open_d3d/short_sleeve_dress/63/crop/HFS_marked.png}
	\end{minipage}
	\begin{minipage}[t]{0.06\textwidth}
		\centering
		\includegraphics[width=1.0\textwidth]{Figure/comparison_open_d3d/short_sleeve_dress/63/crop/HFS.png}
	\end{minipage}
	\\
	\vspace{0.0mm}
	\begin{minipage}[t]{0.13\textwidth}
		\centering
		\includegraphics[width=0.9\textwidth]{Figure/comparison_open_d3d/pants/315/crop/gt_marked.png}
	\end{minipage}
	\begin{minipage}[t]{0.06\textwidth}
		\centering
		\includegraphics[width=1.0\textwidth]{Figure/comparison_open_d3d/pants/315/crop/gt.png}
	\end{minipage}
	\begin{minipage}[t]{0.13\textwidth}
		\centering
		\includegraphics[width=0.9\textwidth]{Figure/comparison_open_d3d/pants/315/crop/ours_marked.png}
	\end{minipage}
	\begin{minipage}[t]{0.06\textwidth}
		\centering
		\includegraphics[width=1.0\textwidth]{Figure/comparison_open_d3d/pants/315/crop/ours.png}
	\end{minipage}
	\begin{minipage}[t]{0.13\textwidth}
		\centering
		\includegraphics[width=0.9\textwidth]{Figure/comparison_open_d3d/pants/315/crop/neus_marked.png}
	\end{minipage}
	\begin{minipage}[t]{0.06\textwidth}
		\centering
		\includegraphics[width=1.0\textwidth]{Figure/comparison_open_d3d/pants/315/crop/neus.png}
	\end{minipage}
	\begin{minipage}[t]{0.13\textwidth}
		\centering
		\includegraphics[width=0.9\textwidth]{Figure/comparison_open_d3d/pants/315/crop/idr_marked.png}
	\end{minipage}
	\begin{minipage}[t]{0.06\textwidth}
		\centering
		\includegraphics[width=1.0\textwidth]{Figure/comparison_open_d3d/pants/315/crop/idr.png}
	\end{minipage}
	\begin{minipage}[t]{0.13\textwidth}
		\centering
		\includegraphics[width=0.9\textwidth]{Figure/comparison_open_d3d/pants/315/crop/HFS_marked.png}
	\end{minipage}
	\begin{minipage}[t]{0.06\textwidth}
		\centering
		\includegraphics[width=1.0\textwidth]{Figure/comparison_open_d3d/pants/315/crop/HFS.png}
	\end{minipage}
	\\
	\vspace{0.0mm}
	\begin{minipage}[t]{0.13\textwidth}
		\centering
		\subfloat[GT]{\includegraphics[width=0.9\textwidth]{Figure/comparison_open_mgn/TShirtNoCoat/125611500935128/crop/gt_marked.png}}
	\end{minipage}
	\begin{minipage}[t]{0.06\textwidth}
		\centering
		\includegraphics[width=1.0\textwidth]{Figure/comparison_open_mgn/TShirtNoCoat/125611500935128/crop/gt.png}
	\end{minipage}
	\begin{minipage}[t]{0.13\textwidth}
		\centering
		\subfloat[Ours]{\includegraphics[width=0.9\textwidth]{Figure/comparison_open_mgn/TShirtNoCoat/125611500935128/crop/ours_marked.png}}
	\end{minipage}
	\begin{minipage}[t]{0.06\textwidth}
		\centering
		\includegraphics[width=1.0\textwidth]{Figure/comparison_open_mgn/TShirtNoCoat/125611500935128/crop/ours.png}
	\end{minipage}
	\begin{minipage}[t]{0.13\textwidth}
		\centering
		\subfloat[NeuS]{\includegraphics[width=0.9\textwidth]{Figure/comparison_open_mgn/TShirtNoCoat/125611500935128/crop/neus_marked.png}}
	\end{minipage}
	\begin{minipage}[t]{0.06\textwidth}
		\centering
		\includegraphics[width=1.0\textwidth]{Figure/comparison_open_mgn/TShirtNoCoat/125611500935128/crop/neus.png}
	\end{minipage}
	\begin{minipage}[t]{0.13\textwidth}
		\centering
		\subfloat[IDR]{\includegraphics[width=0.9\textwidth]{Figure/comparison_open_mgn/TShirtNoCoat/125611500935128/crop/idr_marked.png}}
	\end{minipage}
	\begin{minipage}[t]{0.06\textwidth}
		\centering
		\includegraphics[width=1.0\textwidth]{Figure/comparison_open_mgn/TShirtNoCoat/125611500935128/crop/idr.png}
	\end{minipage}
	\begin{minipage}[t]{0.13\textwidth}
		\centering
		\subfloat[HFS]{\includegraphics[width=0.9\textwidth]{Figure/comparison_open_mgn/TShirtNoCoat/125611500935128/crop/HFS_marked.png}}
	\end{minipage}
	\begin{minipage}[t]{0.06\textwidth}
		\centering
		\includegraphics[width=1.0\textwidth]{Figure/comparison_open_mgn/TShirtNoCoat/125611500935128/crop/HFS.png}
	\end{minipage}
\vspace{-1em}
\caption{Comparisons on open surface reconstruction. Row 1 -- 4 are evaluated on \DFD~\cite{zhu2020deep} and Row 5 is evaluated on \MGN~\cite{bhatnagar2019mgn}. \modelName~is able to reconstruct high-fidelity open surfaces while NeuS~\cite{wang2021neus}, IDR~\cite{yariv2020idr} and HFS~\cite{wang2022hfneus} fail to recover the correct topologies.}
\vspace{-1em}
\label{fig:comparison_open}
\end{figure*}

\vspace{-0.5em}


% \paragraph{Implementation of \netName.} 
% We implement \netName~as follows.
% \textbf{SDF-Net}: We borrowed the implementation of the SDF network in DeepSDF~\cite{deepsdf}, which consists of 8 layers
% with hidden layers of width 512, and a single skip connection from the input to the middle layer. We initialize the parameters of the MLP with geometric initialization~\cite{igr}.
% \noindent\textbf{Color-Net}: We borrowed the implementation of the renderer MLP in IDR~\cite{yariv2020idr}, which consists of 4 layers, with hidden layers of
% width $512$. We apply positional encoding~\cite{mildenhall2020nerf} to improve the learning of high-frequencies.
% \noindent \textbf{Validity-Net}: The MLP consists of 8 layers with Xavier initialization. We used the \textit{ReLU} activation between hidden layers and \textit{Sigmoid} for the output.
% \weikai{Move this paragraph to supplemental.}

\vspace{-0.5mm}


\paragraph{Implementation details.}
For the reconstruction experiments on open surfaces, we render the  ground truth point clouds from \DFD~\cite{zhu2020deep} with Pytorch3D~\cite{ravi2020pytorch3d} at a resolution of $256^2$. To get diverse supervision data, we uniformly sample 648 and 64 viewpoints on the unit sphere for \DFD~and \MGN~(MGN), respectively.
For the single view reconstruction experiment, we uniformly sample 64 viewpoints on the unit sphere as the camera positions.
We use an ResNet-18~\cite{He_2016_CVPR} encoder to predict a latent code $\textbf{z}$ describing the surface's geometry and color. We then use the concatenation of $\{\textbf{z}, \pt\}$ as the input to \netName~(decoder) to evaluate the SDF, validity, and color at the query positions. We optimize the autoencoder by comparing the 2D rendering and the ground truth image. 
In the evaluation stage, we accept a single image as the input and directly export the evaluated SDF and validity as 3D mesh.

\vspace{-1em}
\paragraph{Evaluations.} For multiview reconstruction on watertight surfaces, we measure the Chamfer Distance (CD) with \textit{DTU MVS 2014 evaluation toolkit}~\cite{dtu}. For the reconstruction experiments on open surfaces, we measure the CD with the \textit{PCU} Library~\cite{point-cloud-utils}. For all the experiments, we evaluate the result meshes at resolution $512^3$. 
\vspace{-1.5mm}
\subsection{Multiview Reconstruction on Closed Surfaces}
\vspace{-0.5em}
% We investigate if our method is applicable to multi-view reconstruction in real-world scenarios. For this experiment, we do not condition our model and train one model per object.

% \weikai{We only have quantitative comparisons with NeRF?}\xiaoxu{Yes}
% \xiaoxu{NeuS compared with NerF in their paper. I copied the results.}

We compare our approach with the state-of-the-art volume and surface rendering based methods - HFS~\cite{wang2022hfneus}, NeuS~\cite{wang2021neus} and IDR~\cite{yariv2020idr}, and a classic mesh reconstruction and novel view synthesis method -- NeRF~\cite{mildenhall2020nerf}. We report the quantitative results in Table~\ref{table:comparison_watertight}.

We also show visual comparison with a widely-used MVS method: COLMAP~\cite{schoenberger2016sfm, schoenberger2016mvs}. 
We show qualitative results in Fig.~\ref{fig:comparison_watertight}. The results reconstructed with the proposed method show comparable quality compared with the state-of-the-art. 


% To evaluate our approach and baseline methods, we use 8 categories from the DeepFashion3D Dataset~\cite{zhu2020deep}: long sleeve upper, short-sleeve upper, long sleeve dress, short sleeve dress, no sleeve dress, long pants, short pants, and skirt. And we use TODO categories from the MGN Dataset~\cite{bhatnagar2019mgn}: TShirtNoCoat, ShirtNoCoat, LongCoat, Pants, ShortPants. The dataset contain clothes a wide variety of materials, appearance and geometry, including challenging cases for reconstruction algorithms. We render the meshes to generate 216 images with the image resolution of $256 \times 256$. 

% To validate that our approach could represent closed surfaces. We use 5 scenes from the DTU dataset~\cite{jensen2014large}. Each scene contains 49 or 64 images with the image resolution of $1600 \times 1200$.

% 122: neus: 0.53
\begin{table}[h]
    \small
    \centering
    \begin{tabular}{c|c|c|c|c|c}
        \hline
        CD$\downarrow$ & Ours & NeuS & IDR & NeRF & HFS\\
        \hline
        \hline
        % scan 37 & $1.86$ & $\textbf{0.98}$ & $1.87$ & $2.39$\\
        % scan 24 & $1.75$ & $\textbf{0.83}$ & $1.63$ & $1.15$\\
        scan 55 & $0.47$ & $0.38$ & $0.48$ & $0.66$ & $\textbf{0.37}$\\
        % scan 65 & $1.08$ & $\textbf{0.60}$ & $0.79$ & $1.44$\\
        scan 69 & $0.84$ & $\textbf{0.60}$ & $0.77$ & $1.50$ & $0.66$\\
        scan 83 & $1.28$ & $1.43$ & $1.33$ & $\textbf{1.20}$ & $1.27$\\
        scan 97 & $1.09$ & $\textbf{0.96}$ & $1.16$ & $1.96$ & $1.00$\\
        scan 105 & $\textbf{0.75}$ & $0.78$ & $0.76$ & $1.27$ & $0.86$\\
        scan 106 & $0.76$ & $\textbf{0.52}$ & $0.67$ & $0.66$ & $0.57$\\
        scan 110 & $\textbf{0.80}$ & $1.44$ & $0.90$ & $2.61$ & $1.24$\\
        scan 114 & $0.38$ & $\textbf{0.36}$ & $0.42$ & $1.04$ & $0.41$\\
        scan 118 & $0.56$ & $\textbf{0.46}$ & $0.51$ & $1.13$ & $0.52$\\
        scan 122 & $0.55$ & $\textbf{0.49}$ & $0.53$ & $0.99$ & $\textbf{0.49}$\\
        \hline
        \hline
        average & $0.749$ & $0.742$ & $0.753$ & $1.302$ & $\textbf{0.741}$\\
        \hline
    \end{tabular}
    \caption{Quantitative evals on real-world object reconstruction.
    % \brandon{we have the same performance as Neus in the last row; only bolding ours are OK or not? Also, I think we should add average for both tables.}
    }
    % \vspace{-1.5em}
    \label{table:comparison_watertight}
\end{table}


\begin{table}[htbp]
    \small
    \centering
    \begin{tabular}{c|c|c|c|c|c}
        \hline
        & CD ($\times 10^{-3}$) $\downarrow$ & Ours & NeuS & IDR & HFS \\
        \hline
        \hline
        \multirow{9}{*}{D3D} 
        &long slv upper & \textbf{4.483} & $6.864$ & $11.494$ & $9.695$\\
        &short slv upper & \textbf{4.517} & $6.048 $ & $9.043$ & $7.800$\\
        &no slv upper & \textbf{3.418} & $4.856 $ & $17.710$ & $8.576$\\
        &long slv dress & \textbf{4.843} & $6.135$ & $9.203$ & $8.235$\\
        &short slv dress & \textbf{4.276} & $7.951$ & $8.506$ & $7.705$\\
        &no slv dress & \textbf{3.706} & $5.406$ & $6.785$ & $7.565$\\
        &pants & \textbf{5.391} & $11.847 $ & $10.880$ & $16.205$\\
        &dress & \textbf{3.889} & $5.673 $ & $6.983$ & $11.644$\\
        \cline{2-6}
        &average & \textbf{4.315} & $6.847$ & $10.075$ & $9.678$\\
        \hline
        \hline
        \multirow{6}{*}{MGN}
        &LongCoat & \textbf{7.601} & $8.038$ & $12.058$ & $10.398$\\
        &TShirtNoCoat & \textbf{8.481} & $9.910$ & $15.709$ & $13.128$\\
        &ShirtNoCoat & \textbf{5.281} & $8.084$ & $9.509$ & $11.299$\\
        &ShortPants & \textbf{15.324} & $15.480$ & $16.329$ & $18.332$\\
        &Pants & \textbf{9.191} & $12.188$ & $19.931$ & $19.414$\\
        \cline{2-6}
        &average & \textbf{9.176} & $10.740$ & $14.707$ & $14.514$\\
        \hline
    \end{tabular}
    \vspace{-0.2em}
    \caption{Quantitative evaluation on \textit{Deep Fashion 3D
    Dataset}~(D3D)~\cite{zhu2020deep}~with chamfer distance averaged over five examples per category, and \MGN~(MGN)~\cite{bhatnagar2019mgn} with chamfer distance averaged on two examples per category.}
    \vspace{-1em}
    \label{table:comparison_open_d3d}
\end{table}
% \begin{table}[htbp]
    \centering
    \begin{tabular}{|c|c|c|c|}
        \hline
        CD ($\times 10^{-3}$) & Ours & NeuS~\cite{wang2021neus} & IDR~\cite{yariv2020idr} \\
        \hline
        LongCoat & \textbf{5.561} & $7.361$ & $10.348$\\
        TShirtNoCoat & \textbf{TODO} & $TODO$ & $14.784$\\
        ShirtNoCoat & \textbf{4.843} & $6.135$ & $10.091$\\
        ShortPants & \textbf{TODO} & $TODO$ & $14.203$\\
        Pants & \textbf{TODO} & $TODO$ & $15.574$\\
        \hline
    \end{tabular}
    \caption{Quantitative evaluation on \MGN.}
    \label{table:comparison_open_mgn}
\end{table}
\vspace{-0.5em}
\subsection{Multiview Reconstruction on Open Surfaces}
\vspace{-0.5em}
% We investigate if our method is applicable to multi-view reconstruction of open surfaces. For this experiment, we do not condition our model and train one model per object.

We conduct this experiment on eight categories from Deep Fashion 3D~\cite{zhu2020deep} and five categories from the MGN dataset~\cite{bhatnagar2019mgn}. {We compare our approach with two state-of-the-art volume rendering based methods -- NeuS~\cite{wang2021neus} and HFS~\cite{wang2022hfneus}, and a surface rendering based method -- IDR~\cite{yariv2020idr}.}

We report the Chamfer Distance averaged on five examples for each category from \DFD~\cite{zhu2020deep} and report the Chamfer Distance averaged on two examples for each category from \MGN~in Table~\ref{table:comparison_open_d3d}. \modelName~generally provides lower numerical errors compared with the state-of-the-arts.
We show qualitative results in Fig.~\ref{fig:comparison_open}. {\modelName~also provides lower numerical errors in F-score. Please refer to the supplemental for the comparisons.}

% \xiaoxu{
In most cases, NeuS~\cite{wang2021neus} and IDR~\cite{yariv2020idr} are able to reconstruct the geometry with thick, watertight surfaces. While, for the pants in Figure~\ref{fig:comparison_open}, NeuS fails to recover the shape of the waist. \modelName{} is able to reconstruct high-fidelity open surfaces with consistent normals, including the thin straps of the camisoles and dresses.
% }
% \brandon{Add some discussions here.}
\begin{figure}[htbp]
\centering
    \includegraphics[width=0.95\linewidth]{Figure/comparison_singleview_reconstruction/image.pdf}
\vspace{-0.5em}
\caption{With given single-view images, ours predicts
accurate 3D geometry of arbitrary shapes with the autoencoder. \modelName{} achieves CD = $0.0771$ averaged on the 25
objects from the test set, which outperforms NeuS~\cite{wang2021neus} (CD =
$0.0778$) and DVR~\cite{dvr} (CD = $0.0789$). }
\vspace{-1.5em}
\label{fig:comparison_singleview_reconstruction}
\end{figure}

\vspace{-0.5em}
\subsection{Single View Reconstruction on Open Surfaces}
\vspace{-0.5em}
% We investigate if our method is applicable to arbitrary 3D shape reconstruction from single-view images. 
We construct an autoencoder, which accepts a single image as the input, and exports the 3D mesh as the output. For this experiment, we compare our approach against the state-of-the-art single-view reconstruction method: DVR~\cite{dvr} and the volume rendering based method: NeuS~\cite{wang2021neus}.

% We use the dress subset from \textit{Deep Fashion 3D Dataset}~\cite{zhu2020deep} for this experiment. We randomly select $116$ meshes and $28$ meshes as the training and testing sets, respectively. We render $648$ RGB images and the corresponding masks of resolution $256\times256$ per object as the supervision.
% We randomly sample the viewpoint on the unit-sphere to get diverse supervision data.

The qualitative results is shown in Fig~\ref{fig:comparison_singleview_reconstruction}.
% , and the quantitative results in reported in Table~\ref{}.
Our method is able to infer accurate 3D shape representations from single-view images when only using 2D multi-view images and object masks as supervision.
Qualitatively, in contrast to the DVR~\cite{dvr} and NeuS~\cite{wang2021neus} autoencoder, our method is able to reconstruct open surfaces.
Quantitatively, our method achieves CD = $0.0771$, which outperforms NeuS (CD = $0.0778$) and DVR (CD = $0.0789$) averaged on all the 25 objects from the test set.
% Our results rivals the quality of the conventional multi-view reconstruction approach.

% \vspace{-0.5em}
\subsection{Ablation Studies}
\vspace{-0.5em}
\begin{figure}[htb]
    \centering
        \includegraphics[width=1.0\linewidth]{Figure/ablation_validity_regularization/image.pdf}
    \vspace{-1.5em}
    \caption{Ablation study on the regularizations about validity.
    }
\vspace{-2em}
\label{fig:ablation_validity}
\end{figure}

\paragraph{Regularizations on validity.}
We conduct an ablation study on the regularizations about validity, i.e. $\mathcal{L}_{bce}$ and $\mathcal{L}_{sparse}$.
As shown in Figure~\ref{fig:ablation_validity} (c), by setting $\mathcal{L}_{bce}=0$, the renderer tends to generate rendering probability between 0 and 1, thus resulting in noisy faces in the output mesh; as shown in Figure~\ref{fig:ablation_validity} (d), by setting $\mathcal{L}_{sparse}=0$, the renderer will keep the redundant surfaces, instead of learning a validity space as sparse as possible.


% Xiaoxu: sigmoid_factor doesn't affect that much
% \paragraph{Sigmoid factor (enforce consistency between meshing and rendering)}
% Goal: Consistency between the result from regression and from classification
% \input{Figure/ablation_sigmoid_factor/ablation_sigmoid_factor}

\begin{figure}[htb]
    \begin{minipage}[t]{.09\textwidth}
        \centering
        \subfloat[GT]{\includegraphics[width=\textwidth]{Figure/ablation_n_views/320/gt.png}}
    \end{minipage}
    \begin{minipage}[t]{.09\textwidth}
        \centering
        \subfloat[64 views\\CD = 0.00568]{\includegraphics[width=\textwidth]{Figure/ablation_n_views/320/66views.png}}
    \end{minipage}
    \begin{minipage}[t]{.09\textwidth}
        \centering
        \subfloat[32 views\\CD = 0.00632]{\includegraphics[width=\textwidth]{Figure/ablation_n_views/320/34views.png}}
    \end{minipage}
    \begin{minipage}[t]{.09\textwidth}
        \centering
        \subfloat[16 views\\CD = 0.00763]{\includegraphics[width=\textwidth]{Figure/ablation_n_views/320/18views.png}}
    \end{minipage}
    \begin{minipage}[t]{.09\textwidth}
        \centering
        \subfloat[8 views\\CD = 0.01326]{\includegraphics[width=\textwidth]{Figure/ablation_n_views/320/10views.png}}
    \end{minipage}
    \vspace{-1em}
    \caption{Ablation study on multi-view reconstruction with different number of views.}
\vspace{-1em}
\label{fig:ablation_n_views}
\end{figure}

\paragraph{Reconstruct with different number of views.}
We additionally show results on reconstruction with different number of views. As shown in Figure~\ref{fig:ablation_n_views}, our method is able to reconstruct open surfaces even with sparse viewpoints. The reconstruction quality improves with the increase of views, quantitively and qualitatively. 



% Not that important
% \paragraph{Reconstructing thin structures (w. vs. w/o mask regularization)}
% As described in Section~\ref{sec:method_training}, if a pixel is sensitive to dilation or erosion, we consider this pixel as part of the sensitive region. We use higher mask weight for the sensitive regions.
% \begin{figure}[htb]
    \begin{minipage}[t]{.2\textwidth}
        \centering
        \includegraphics[width=\textwidth]{Figure/placeholder.pdf}
        % \subcaption{Image 1.}
    \end{minipage}
    \hfill
    \begin{minipage}[t]{.2\textwidth}
        \centering
        \includegraphics[width=\textwidth]{Figure/placeholder.pdf}
        % \subcaption{Image 2.}
    \end{minipage}  
    \label{fig:ablation_mask}
    \caption{Ablation study on mask regularization.}
\end{figure}

\section{Discussion}
In this work, we introduce novel approaches for both Multi-sensor Fusion and Imitation Learning objective function design. The Cross Semantic Generation approach aims to extract and enhance the shared semantic information from LiDAR and RGB inputs. We used some auxiliary losses to regularize the feature space, ensuring the information flow of the features which are important for driving decisions according to human experience. Penalty-based Imitation Learning further increases the level of compliance of the agent with traffic rules. Some other approaches use an extra module to ensure the agent obeys traffic rules.  NEAT \cite{Chitta2021ICCV(NEAT)}, LAV \cite{chen2022lav} introduce some low-level control strategies in the PID controller to force braking at red lights. InterFuser uses a safety module to avoid dangerous actions such as collisions with other vehicles. These strategies largely increase the performance of the agent. However, these extra modules also make the network no longer end-to-end which may potentially cause more tuning effort and suboptimal solutions \cite{DBLP:journals/corr/abs-2003-06404}. With penalty-based Imitation Learning, we aim to avoid those decisions detached from the network. We use the penalty to make the agent more sensitive to traffic rules. The end-to-end nature of the network is guaranteed while constraining the agent to comply with traffic regulations. 

% To the best of our knowledge, it is the first time to introduce the penalty concept in imitation learning. Given that our proposed method is highly scalable, it would be interesting to extend the penalties based on new traffic rules such as collisions with pedestrians and vehicles. 

% It is also interesting to leverage multi-frame inputs to identify the violations of traffic regulations in the time dimension and introduce the corresponding penalty. For instance, the agent may drive on the line for long periods. 

Our study also has several limitations. First of all, we only use front-view images and 180-degree LiDAR data as input. The information from the rear of the vehicle is missing which may cause collisions when changing the lane. Besides, we only tried to integrate the RGB and LiDAR input, more sensors like radar input and depth-map can be taken into consideration. Additionally, we only test the performance of our model in the simulation environment. Real-world data can be more complex and contain more noise. Finally, we believe that further research should be dedicated to introducing additional penalties into the objective functions, as this approach holds promise in the development of human-level end-to-end autonomous driving agents.
\section{Conclusion}
The key points for the performance of end-to-end autonomous driving are improving fusion technologies and policy learning methods. These two points turn into two important questions. How to efficiently extract and integrate the features from different modalities? How to effectively use these features to learn a stable and well-performing policy approaching or even surpassing the human level. In this paper,  we contribute to the abovementioned aspects and achieve state-of-the-art performance. Compared to modular autonomous driving technologies, end-to-end autonomous driving has lower hardware costs and less expensive maintenance. It is also adaptable to different scenarios simply by feeding data. We believe end-to-end autonomous driving can be deployed in actual vehicles in the near future.

M.J. is supported by the Ministry of Trade, Industry, and Energy in Korea, under Human Resource Development Program for Industrial Innovation (Global) (P0017311) supervised by the Korea Institute for Advancement of Technology. H.C. is supported by NIH DP5 OD029574-01 and by the Schmidt Fellows Program at Broad Institute.


%%%%%%%%%%%%%%%%%%%%%%%%%%%%%%%%%%%%%%%%%%%%%%%%%%%%%%%%%%%%%%%%%%%%%%%%%%%%%%%%
{
\small
\bibliographystyle{IEEEtran}
\bibliography{reference}
}



\end{document}
