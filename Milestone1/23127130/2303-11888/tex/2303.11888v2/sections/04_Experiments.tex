\section{Experiments}
\begin{table*}
    \begin{center}
        \caption{}
        \label{table: Experiment_Long}
        \resizebox{\linewidth}{!}{
        \begin{tabular}{ c | c  c  c | c  c  c  c  c  } 
             \hline
                Model & \makecell[c]{Driving \\ Score \\ \%, $\uparrow$} & \makecell[c]{Route \\ Complication \\ \%, $\uparrow$} & \makecell[c]{Infraction \\ Score \\ $\uparrow$} &  \makecell[c]{Collision \\ Pedestrian \\ \#/km, $ \downarrow$}  & \makecell[c]{Collision \\ Vehicle \\ \#/km, $ \downarrow$} & \makecell[c]{Collision \\ Static \\ \#/km, $ \downarrow$} & \makecell[c]{Red light \\ Infraction \\ \#/km, $ \downarrow$} & \makecell[c]{Stop Sign Infraction \\ Infraction \\ \#/km, $ \downarrow$}\\
            \hline
            \hline
                P-CSG (ours) & $\textbf{56.38} \pm 4.18$ & $\textbf{94.00} \pm 1.75$ & $\textbf{0.61} \pm 0.05$ & $\textbf{0.00} \pm 0.00$ & $\textbf{0.08} \pm 0.02$ & $\textbf{0.00} \pm 0.00$ & $0.03 \pm 0.01$ & $0.01 \pm 0.01$ \\
                TransFuser & $34.50 \pm 2.54$ & $61.16 \pm 4.75$ & $0.56 \pm 0.06$ & $0.01 \pm 0.01$ & $0.58 \pm 0.07$ & $0.38 \pm 0.05$ & $0.12 \pm 0.03$ & $0.05 \pm 0.02$\\
                TransFuser+ & $36.19 \pm 0.90$ & $70.13 \pm 6.80$ & $0.51 \pm 0.03$ & $\textbf{0.00} \pm 0.00$ & $0.40 \pm 0.13$ & $0.04 \pm 0.03$&$0.11 \pm 0.10$ & $0.04 \pm 0.01$ \\
                InterFuser & $50.64 \pm 3.51 $ & $89.13 \pm 4.12$ & $0.57 \pm 0.05$  & $\textbf{0.00} \pm 0.00$ & $0.09 \pm 0.04$ & $0.01 \pm 0.01$ & $\textbf{0.02} \pm 0.01$ & $0.04 \pm 0.01$\\
                Geometric Fusion & $31.30 \pm 5.2$  & $57.17 \pm 11.6$ & $0.54 \pm 0.04$ & $0.01 \pm 0.01$ & $0.43 \pm 0.08$ & $0.02 \pm 0.01$ & $0.11 \pm 0.02$ & $0.08 \pm 0.04$\\
                LateFusion & $32.43 \pm 6.72$ & $60.41 \pm 4.23$ & $\textbf{0.61}\pm 0.06 $ & $0.03 \pm 0.02$ & $0.12 \pm 0.03$ & $0.02 \pm 0.01$ & $0.06 \pm 0.02$ & $0.06 \pm 0.02$ \\
                % Expert & $34.47 \pm 3.25$ & $79.631 \pm 11.29$ & $0.46 \pm 0.08$ & $0.02 \pm 0.01$ & $0.40 \pm 0.22$ & $\textbf{0.00} \pm 0.00$ & $0.13 \pm 0.07$ & $\textbf{0.00} \pm 0.00$ \\

             \hline
        \end{tabular}}
        \vspace{1mm}
        \begin{tablenotes}
            \footnotesize
            \item \textbf{Driving Performance.} We evaluate the Driving Score, Route Complication, Infraction Score, collisions, red light violations, and stop sign violations in the Town05 long benchmark. These evaluation metrics are defined by carla leaderboard \cite{Dosovitskiy17}. Note that all other baselines are retrained and tested in our local CARLA environment. Each model is evaluated three times with random seeds. 
        \end{tablenotes}
    \end{center}
    
\end{table*}

\begin{table*}
    \begin{center}
        \caption{}
        \label{table: Ablation Study}
        \resizebox{\linewidth}{!}{
        \begin{tabular}{ c | c  c  c | c  c  c  c  c } 
             \hline
                Model & \makecell[c]{Driving \\ Score \\ \%, $\uparrow$} & \makecell[c]{Route \\ Complication \\ \%, $\uparrow$} & \makecell[c]{Infraction \\ Score \\ $\uparrow$} &  \makecell[c]{Collision \\ Pedestrian \\ \#/km, $ \downarrow$}  & \makecell[c]{Collision \\ Vehicle \\ \#/km, $ \downarrow$} & \makecell[c]{Collision \\ Static \\ \#/km, $ \downarrow$} & \makecell[c]{Red light \\ Infraction \\ \#/km, $ \downarrow$} & \makecell[c]{Stop Sign Infraction \\ Infraction \\ \#/km, $ \downarrow$}\\
            \hline
            \hline
                P-CSG (ours) & $\textbf{56.38} \pm 4.18$ & $\textbf{94.00} \pm 1.75$ & $\textbf{0.61} \pm 0.05$ & $\textbf{0.00} \pm 0.00$ & $\textbf{0.08} \pm 0.02$ & $\textbf{0.00} \pm 0.00$ & $\textbf{0.03} \pm 0.01$ & $\textbf{0.01} \pm 0.01$ \\
                No CSG & $42.767 \pm 7.78$ & $83.23 \pm 1.86$ & $0.51 \pm 0.11$ & $0.03 \pm 0.01$ & $0.15 \pm 0.04$ & $0.04 \pm 0.03$ & $0.04 \pm 0.00 $ & $0.02 \pm 0.00$\\
                No penalty & $34.98 \pm 5.64$ & $76.20 \pm 13.43$ & $0.51 \pm 0.18 $ & $0.01 \pm 0.00$ & $0.15 \pm 0.11$ & $0.03 \pm 0.01$ & $0.05 \pm 0.02$ & $0.05 \pm 0.01$ \\
                $\lambda_1 = 0.3$ & $37.19 \pm 8.87$ & $82.40 \pm 1.60$ & $0.48 \pm 0.09$ & $0.01 \pm 0.00$ & $0.11 \pm 0.03$ & $0.04 \pm 0.01$ & $0.06 \pm 0.01$ & $\textbf{0.01} \pm 0.00$\\
                $\lambda_1 = 0.1$ & $45.02 \pm 5.54$ & $69.35 \pm 1.90$ & $0.70 \pm 0.07$ & $0.01 \pm 0.01$ & $0.09 \pm 0.07$ & $\textbf{0.00} \pm 0.00$ & $0.04 \pm 0.01$ & $\textbf{0.01} \pm 0.00$\\
                $\lambda_2 = 0.005$ & $53.20 \pm 7.02$ & $87.55 \pm 5.73$ & $0.62 \pm 0.10 $ & $0.03 \pm 0.01$ & $0.07 \pm 0.01$ & $\textbf{0.00} \pm 0.00$ & $0.05 \pm 0.02$ & $0.02 \pm 0.02$\\
                $\lambda_2 = 0.05$ & $49.43 \pm 6.73$ & $93.87 \pm 5.51$ & $0.53 \pm 0.02$ & $0.01 \pm 0.01$ & $0.09 \pm 0.04$ & $\textbf{0.00} \pm 0.00$ & $0.05 \pm 0.02$ & $\textbf{0.01} \pm 0.00 $ \\
                $\lambda_3 = 0.3$  & $47.30 \pm 3.58$ & $92.21 \pm 4.52$ & $0.51 \pm 0.07$ & $\textbf{0.00} \pm 0.00$ & $0.11 \pm 0.02$ & $\textbf{0.00} \pm 0.00$ & $0.05 \pm 0.02$ & $0.03 \pm 0.01$ \\
                $\lambda_3 = 0.7$  & $47.35 \pm 4.97$ & $94.45 \pm 3.95$ & $0.50 \pm 0.06$ & $0.01 \pm 0.01$ & $0.13 \pm 0.03$ & $\textbf{0.00} \pm 0.00$ & $0.05 \pm 0.00$ & $\textbf{0.01} \pm 0.01$ \\
             \hline
        \end{tabular}}
        \vspace{1mm}
        \begin{tablenotes}
            \footnotesize
            \item \textbf{Ablation study.} The ablation study for important hyper-parameters. $\lambda_1, \lambda_2, \lambda_3$ are the penalty weight for the red light, speed, and stop sign respectively. The model with no cross-semantics generation structure (No CSG) and no penalty are also given for comparison. Each model is evaluated three times with random seeds. 
        \end{tablenotes}
    \end{center}
    
\end{table*}
In this section, our experiment setup will first be described. Then we compare our model against other baselines. We also provide ablation studies to show the improvements from penalty-based imitation learning and cross semantics generation. 

\subsection{Task Description}
The task we concentrate on is a navigation task along the predefined routes in different scenarios and areas. There exists GPS signals guiding the vehicle. Low signal or no signal situations are not taken into consideration. Some predefined scenarios will appear in each route to test the agent's ability to avoid the emergencies, such as obstacle avoidance, other vehicles running a red light, and the sudden appearance of pedestrians on the road. There exists a time limit for the agent to complete the route. Time exceeding is considered a failure in terms of route completion. 
\subsection{Training Dataset}
Realistic driving data is hard to achieve. Alternatively, we use the Carla simulator \cite{Dosovitskiy17} to collect the training data processed by the expert policy. We use the same training dataset as TransFuser \cite{TransFuser}. It includes 8 towns and around 2500 routes through junctions with an average length of 100m and about 1000 routes along curved highways with an average length of 400m. We used the expert agent same as TransFuser to generate these training data. 

% \subsection{Metrics}
% The metrics we use to evaluate the behavior of each are Route completion and Infraction Scores.
% \subsubsection{Route Completion}
% Route completion(RC) refers to the proportion of the completed route out of the whole route. Suppose $R_i$ means the route completion proportion in route $i$. The RC can be defined by:
% \begin{equation}
%     RC = \frac{1}{N} \sum_{i}^{N} R_i
% \end{equation}
 
% \subsubsection{Infraction Score}
% Infraction Score (IS) is used to measure the driving behavior of the agent. We define $p_j$ as the penalty for an infraction instance, $j$ is incurred by the agent and $n_j$ is the number of occurrences of infraction instance $j$. Then the Infraction Score can be defined by:
% \begin{equation}
%     IS = \prod_{j}^{{\rm Ped, Veh, Stat, Red, Stop}} p_j^{n_j}
% \end{equation}
% The infraction instances include collision with a pedestrian, collision with a vehicle, collision with static layout, and red light violations. The penalties for them are 0.5, 0.60, 0.65, 0.7, and 0.8, respectively. 

% \subsubsection{Driving Score}
% The driving Score (DS) aims to measure the overall driving performance. It is defined by the weighted average of the route completion with an infraction multiplier:
% \begin{equation}
%     DS = \frac{1}{N} \sum_{i} ^{N} R_i P_i
% \end{equation}
% where $R_i$ and $P_i$ are the route completion and infraction score for $i$-th test instance.

% \subsubsection{Infractions per km}
% Infractions per km refer to the average infraction numbers per kilometer. The infractions include collisions with pedestrians, vehicles, and static elements, running a red light, off-road infractions, route deviations, timeouts, and vehicle blocks. 
% \begin{equation}
%     {\rm Infractions\ per\ km} = \frac{\sum_{i}^{N} \# \ {\rm infractions}_i}{\sum_{i}^{N} k_i}
% \end{equation}
% where $k_i$ is the driven distance (in km) for route $i$. Note that the Off-road infraction is handled differently. The total km driven out of the road is used instead of the number of infractions. 

\subsection{Test Result}
\subsubsection{Benchmark}
We use Town05 long benchmarks to evaluate our model. Town05 long benchmark contains 10 routes and all of these routes are over 2.5km. This benchmark is also used by InterFuser \cite{shao2022interfuser} and TransFuser.
\subsubsection{Baseline}
The other baselines we chose to compare with our model are TransFuser+, TransFuser, Geometric Fusion, and LateFusion. \textbf{TransFuser} \cite{TransFuser} introduces the Transformer into the multi-sensor fusion architecture to achieve better end-to-end autonomous driving results. \textbf{TransFuser+} \cite{TransFuser+}, as an extension of TransFuser, leverages several auxiliary losses to ensure important information flows such as traffic light and road line information in the network. \textbf{InterFuser} \cite{shao2022interfuser} developed a safety control module to regulate the behaviors of the agent, preventing the agent violate the traffic rules. \textbf{LateFusion} \cite{latefusion} uses a simple Multi-Layer Perception Network to integrate multi-modal information. \textbf{Geometric Fusion} \cite{TransFuser} implements both LiDAR-to-image and Image-to-LiDAR fusion to aggregate the information from LiDAR and image to increase the end-to-end autonomous driving ability. 

As Table \ref{table: Experiment_Long} shows, our model achieves the highest route completion and driving scores among all baselines. Compared to TransFuser, Transfuser+, and LateFusion, our model has a huge increase in driving scores and route complications. InterFuser, the current state-of-the-art model, performs well because its safety module avoids dangerous behavior inferred by the neural networks. However, this structure modularizes the decision-making process and these conflicted acts of the safety module and the neural network may have potential risks. Another disadvantage of modular approaches is that the predefined inputs and outputs of individual sub-systems might not be optimal for the driving task in different scenarios. \cite{DBLP:journals/corr/abs-2003-06404} analyses the end-to-end approaches and modular approaches of autonomous driving in detail. 
In contrast to InterFuser, we intend to restrict the behaviors of the agent by introducing penalties to the objective function so that the whole autonomous driving process remains end-to-end. As the results demonstrate, our penalty-based imitation learning can also avoid dangerous behaviors of the agent and make the agent more sensitive to the traffic rules. It achieves even better performance than InterFuser. 

\subsection{Ablation Study}
In this subsection, we will analyze the influences of different penalty weights for corresponding traffic rules. As Table \ref{table: Ablation Study} demonstrates, two extra weights for each penalty are selected for comparison. We also provide the result of models without CSG and penalties for comprehensive analysis. 
We notice that the infractions of traffic lights and stop signs are largely reduced by adding penalties. Our proposed multi-sensor fusion technology (CSG) also decreases the possibility of hitting obstacles such as vehicles, pedestrians, and other statics. 
The results of different penalty weights are also listed in the table for comparison. Note that the default weights $\lambda_1$, $\lambda_2$, and $\lambda_3$ we choose for our best model are 0.5, 0.01, and 0.5 respectively. We found that assigning greater weight to more severe violations will increase the performance of our model. For instance, we apply greater penalties for the red light and the stop sign violations compared to overspeeding by turning, since those two violations cause more serious consequences.

\subsection{Inference and Training Efficiency}
 In this subsection, we aim to compare training time, inference time and parameter number across three SOTA models in the field of end-to-end autonomous driving to gain insights into their computational characteristics. As illustrated in Table \ref{table: efficiency}, our model stands out by having the fewest parameters and the shortest training and inference times when compared to the other two state-of-the-art models. These findings provide compelling evidence of our model's reduced complexity. With fewer parameters and faster processing times, our model showcases an efficient  design, delivering comparable performance while minimizing computational demands. With the shorter inference time, our model also guarantee a safer and more efficient navigation, since the autonomous driving system can quickly detect and react to changes in the environment.
 \begin{table}[htbp]
     \centering
     \caption{}
     \resizebox{\linewidth}{!}{
     \begin{tabular}{c|c c c c}
        \hline
        Model &  \makecell[c]{Parameters \\ Number (M)} & \makecell[c]{Total \\ Training \\ Frames (K)} & \makecell[c]{Training \\ Time  \\ (min. / epoch) } & \makecell[c]{Inference \\ Time  \\ (s / frame)}\\
        \hline
        P-CSG(ours)&  36 & 209 & 33 & 0.043\\
        Interfuser&  53 & 232 & 112 & 0.312\\
        Transfuser+&  168 & 164 & 199 & 0.071\\
        \hline
     \end{tabular}
    }
     \label{table: efficiency}
    \vspace{1mm}
    \begin{tablenotes}
        \footnotesize
        \item \textbf{Complexity.} All experiments are tested with one NVIDIA RTX 6000 GPU and 30 process units under Ubuntu 20.04.6 LTS (Focal Fossa) operating system. The training time per epoch is derived by computing the mean duration across a set of 20 consecutive epochs. The inference time per frame is determined by averaging the elapsed time for 100 individual frames.
    \end{tablenotes}
 \end{table}