\documentclass[journal,review]{vgtc} 
\usepackage[utf8]{inputenc}
\usepackage{amsmath,amssymb,amsfonts}
\usepackage{algorithmic}
\usepackage{graphicx}
\usepackage{textcomp}
\newcommand{\bbox}{\text{bbox}}
\newcommand{\alphapck}{\alpha_\bbox}
\newcommand{\kcycle}{\text{k-CyPCK}}
\newcommand{\cycle}{\text{-CyPCK}}

\newcommand{\I}{\mathbf{I}}
\newcommand{\Ia}{\I^\text{a}}
\newcommand{\Ib}{\I^\text{b}}
\newcommand{\Iatob}{\I^\text{a $\rightarrow$ b}}
\newcommand{\F}{\mathbf{F}}
\newcommand{\Fa}{\F^\text{a}}
\newcommand{\Fb}{\F^\text{b}}
\newcommand{\f}{\mathbf{f}}
\newcommand{\fa}{\f^\text{a}}
\newcommand{\fb}{\f^\text{b}}
\newcommand{\p}{\mathbf{p}}
\newcommand{\pa}{\p^\text{a}}
\newcommand{\pb}{\p^\text{b}}
\newcommand{\A}{\boldsymbol{\Phi}_\text{align}}
\newcommand{\G}{\mathbf{G}}
\newcommand{\C}{\mathbf{C}}
\newcommand{\Ca}{\C^\text{a}}
\newcommand{\Cb}{\C^\text{b}}
\newcommand{\cc}{\mathbf{c}}
\newcommand{\cca}{\cc^\text{a}}
\newcommand{\ccb}{\cc^\text{b}}
\newcommand{\Irec}{\I_\text{Recon}}
\newcommand{\M}{\mathbf{M}}
\newcommand{\Mrec}{\M_\text{Recon}}
\newcommand{\loss}{\mathcal{L}}
\newcommand{\T}{\mathcal{T}}
\newcommand{\W}{\mathcal{W}}
\newcommand{\Id}{\mathcal{I}}

\usepackage[superscript,biblabel]{cite}

\usepackage{hyperref}
\usepackage{url}
\ifpdf%                                % if we use pdflatex
  \pdfoutput=1\relax                   % create PDFs from pdfLaTeX
  \pdfcompresslevel=9                  % PDF Compression
  \pdfoptionpdfminorversion=7          % create PDF 1.7
  \ExecuteOptions{pdftex}
  \usepackage{graphicx}                % allow us to embed graphics files
  \DeclareGraphicsExtensions{.pdf,.png,.jpg,.jpeg} % for pdflatex we expect .pdf, .png, or .jpg files
\else%                                 % else we use pure latex
  \ExecuteOptions{dvips}
  \usepackage{graphicx}                % allow us to embed graphics files
  \DeclareGraphicsExtensions{.eps}     % for pure latex we expect eps files
\fi%



\title{Unleashing ChatGPT on the Metaverse: \\Savior or Destroyer?}




\begin{document}

\maketitle

\textbf{
Incorporating artificial intelligence (AI) technology, particularly natural language processing (NLP), is becoming increasingly vital for developing immersive and interactive metaverse experiences. ChatGPT-like large language models can provide better descriptions and understand human reactions, thus having the potential to effectively create a more immersive and engaging virtual environment. The article delves into the pros and cons of utilizing ChatGPT for the metaverse and provides a number of suggestions.} %Dynamic and personalized experiences are possible with this technology, but there are also legitimate privacy, bias, and ethical issues to consider. This article aims to help readers understand the possible influence of ChatGPT on the metaverse and how it may be used to effectively create a more immersive and engaging virtual environment by evaluating these opportunities and obstacles.
\\\\

The metaverse~\cite{lee2021all}, a shared digital place where users may communicate and collaborate using a variety of virtual tools, has the potential to revolutionize how we study, socialize, and pass the time. Although researchers have made great strides, making a fascinating and engaging tailored metaverse experience is still a work in progress.
%
For instance, learning to accommodate a wide variety of user interactions using natural language processing (NLP) is a significant obstacle on the path to generating a unique metaverse experience. The OpenAI-developed state-of-the-art language model ChatGPT offers a potent answer in this situation. ChatGPT's natural language processing features allow programmers to make a wide variety of useful applications, such as digital assistants, educational companions, unique forms of entertainment, and individualized itineraries.

While ChatGPT shows promise as a means to tailor and enliven metaverse experiences for individual users, researchers will still need to overcome a number of obstacles. Keeping consumers interested in the metaverse requires, for example, providing a fluid and engaging user experience. For data to be used in an ethical manner, it is important to address concerns about data privacy and security.
%
ChatGPT is discussed in this study to see if it can be used to make interesting and useful metaverse apps for users. While doing so, we take a look at the numerous obstacles that researchers must surmount in order to create metaverse experiences, such as the limitations of current NLP technology, privacy and security concerns, and providing an immersive user experience. We hope that our analysis will help readers better understand how to use ChatGPT to create engaging and unique metaverse experiences, as well as highlight its potential pitfalls.
%\begin{figure}
%    \centering
%    \includegraphics[width=\columnwidth]{figs/overall.png}
%    \caption{ChatGPT as a ``Savior'' for Metaverse}
%    \label{fig:savior}
%\end{figure}
%\section{Related Work}
%Before ChatGPT, the exploration on the overlap between NLP and the metaverse was limited by the power of language models. Most exiting works focus on either simple one-after-another combination, e.g. using NLP to analyze user's comment for XR products or apps, or educational apps such as language learning in virtual environment.
%Fagernas et al.~\cite{fagernas2021users} proposed a semi-automated NLP technique to study 1379 published reviews of 30 different VR relaxation applications available for the Oculus Go and Gear VR.
%Experts in the field of education have investigated the possibility of using the metaverse to create more dynamic and engaging classroom environments~\cite{lin2015language}. For instance, back to 2009, Deutschmann et al.~\cite{deutschmann2009designing} proposed to host oral proficiency courses in the virtual world Second Life. Uppoor et al.~\cite{uppoor2022interactive} proposed a virtual environment to provide a more engaging and rewarding experience with the help of NLP. Mark Peterson~\cite{peterson2011towards} also proposed a research agenda for using virtual worlds in language learning in 2011. Overall, as ChatGPT-like large language models have just started to be released, the potential/impact of ChatGPT for the metaverse is at a very early stage waiting for extensive exploration.

\section*{ChatGPT as a savior}
%With ChatGPT, researchers can create a wide range of engaging metaverse applications. We outline 19 examples under 4 categories according to their purposes as summarized in  Figure~\ref{fig:savior}. We illustrate one case in each category for better understanding in Figure~\ref{fig:companion} to Figure~\ref{fig:mental}. 


%\begin{figure*}
%    \centering
%    \includegraphics[angle=90,origin=c,scale=0.8]{figs/learningcompanion.png}
%    \caption{Interactive educational companions.}
%    \label{fig:companion}
%\end{figure*}

In education, ChatGPT enables programmers to make engaging learning tools that may be used to teach new knowledge and abilities. Students can also augment their classroom learning with interactive virtual educational companions, which provide them the chance to practice their abilities and receive instant feedback~\cite{kok2022virtual}. ChatGPT can be used with mixed reality to learn a new skill or hobby, like gardening or cooking, or in-house employee training purposes~\cite{10042870}, making the process of learning new procedures or processes more interesting and engaging for the trainee. With ChatGPT, researchers can make language-learning assistants that can engage with learners in the virtual world~\cite{uppoor2022interactive,parmaxi2023virtual}. Depending on the learner's progress, language-learning companions can provide constructive criticism. Gamification features can be incorporated into language-learning companions to keep students interested and motivated. Language learning companions powered by ChatGPT make studying a new language interesting and engaging, which in turn improves learners' persistence.
%
Helping customers find their way through online learning platforms is just the beginning; ChatGPT also offers individualized support in the shape of reminders, study guides, and the like. Having this kind of support can keep pupils on track and ultimately help them do better in school. ChatGPT-enabled virtual assistants can also help instructors manage their online classrooms by responding to students' questions and delivering grades and comments in an automatic fashion. Because of this, educators may have more time to focus on guiding students.

%\subsection{Entertainment}
%\subsubsection{Interactive Storytelling. (Figure~\ref{fig:story})}

%\begin{figure*}
%    \centering
%    \includegraphics[width=.8\textwidth]{figs/storytelling.png}
%    \caption{Interactive storytelling.}
%    \label{fig:story}
%\end{figure*}

In entertainment, ChatGPT can be used to automatically produce text for use in interactive narratives~\cite{yuan2022wordcraft}. By using ChatGPT, a user may create vivid descriptions of one's own virtual world and its inhabitants, which will enhance the narratives. Creators of massively multiplayer online role-playing games (MMORPGs) can use this information to make games that stand out from the crowd. ChatGPT allows programmers to implement nuanced, interactive conversations between AI and players. These exchanges can be made to be branching and variable, giving players a one-of-a-kind and tailor-made adventure each time they log in. ChatGPT can also be used to produce random events, missions, and encounters, making the game world more dynamic and unpredictable. If used properly, ChatGPT has the potential to greatly improve the overall quality of interactive role-playing games (RPGs).
Another example is to use ChatGPT to build interactive museum exhibits and exhibitions in the metaverse with customized user experiences~\cite{li2019distance}. ChatGPT may produce exhibits and information that are catered to a user's particular tastes by assessing their interests and preferences, making the experience more interesting and pertinent to them.

%\subsection{Personalization}
%\subsubsection{Personalized Fitness Coaches. (Figure~\ref{fig:fitness})}

%\begin{figure*}
%    \centering
%    \includegraphics[width=.8\textwidth]{figs/fitness.png}
%    \caption{Personalized fitness coaches.}
%    \label{fig:fitness}
%\end{figure*}

For personalized services, ChatGPT also plays an important role. For instance, it can be integrated into virtual fitness apps~\cite{nair2019endure} to evaluate a user's fitness development and modify their exercise regimen accordingly, offering them individualized feedback and motivation along the way. Furthermore, ChatGPT may tailor the coaching experience by producing content that is in line with a user's unique fitness objectives, tastes, and incentives. Users in the metaverse can also enjoy customized news and media experiences thanks to ChatGPT, which can produce customized news recommendations and summaries that are catered to each user's tastes inferred from users' interests and browsing history. Moreover, ChatGPT may produce real-time news updates and notifications on AR glasses based on user choices and interests~\cite{lazaro2021interaction}, keeping users up to date on the most recent developments. ChatGPT may also create customized music playlists and recommendations that are catered to each user's tastes by examining users' listening habits and preferences. Moreover, ChatGPT can produce explanations in natural language about why particular songs or artists are suggested, giving users a more interesting experience.


ChatGPT together with metaverse can also make significant contributions to providing important life supports. For instance, virtual shopping assistants~\cite{billewar2022rise} offer tailored styling and fashion guidance depending on user preferences and body shape. These helpers may digitally try on clothing and accessories for consumers, offer feedback, and make suggestions using computer vision technology. By producing natural language responses that speak to users' emotions and preferences, ChatGPT can assist in making these assistants more convincing and engaging. Virtual assistants for mental health aides~\cite{ma2022effectiveness} can track users' actions and emotions using data mining, then offer tailored suggestions and interventions accordingly. Say if the assistant notices that a user is feeling stressed or concerned, it may provide relaxation techniques or advise practicing mindfulness. By producing empathy and encouraging natural language replies, ChatGPT can aid in making these assistants more sympathetic and understanding.
In general, virtual personal assistants can perform increasingly complex tasks and offer users in the metaverse more individualized and contextual support. These assistants' interactions with users could become more natural and effective if ChatGPT is used to help make them more conversational and human-like. Virtual concierge services, virtual tour guides, and AR language translation are a few other daily support services in the metaverse that can benefit from ChatGPT, to name a few.

%\begin{figure*}
%    \centering
%    \includegraphics[width=.8\textwidth]{figs/mental.png}
%    \caption{Virtual mental health assistants.}
%    \label{fig:mental}
%\end{figure*}
%\subsubsection{AI-powered customer service.}
%AI-powered customer service representatives can be built using ChatGPT to manage a variety of consumer inquiries and grievances in the metaverse. These agents can deliver effective and interesting customer service by assessing consumer inquiries and giving pertinent, natural-language responses. For instance, if a user encounters a problem with a virtual product, an AI-powered customer support assistant can instantly identify the issue, offer a fix, or, if necessary, connect the user with a human representative. This can increase customer satisfaction while lightening the workload of real people who provide customer support.



%\subsubsection{Virtual concierge services.}
%In virtual hotels and resorts, ChatGPT can be utilized to establish virtual concierge services that can offer users individualized recommendations and help. These virtual assistants can aid users with a variety of tasks, including scheduling activities, making reservations at restaurants, and arranging transportation. These virtual assistants can learn more about each user's preferences over time and offer increasingly tailored recommendations by utilizing natural language processing and machine learning. In addition to enhancing the entire visitor experience, this can make hotels and resorts stand out in a congested online market.

%\subsubsection{Virtual tour guides.}
%ChatGPT may be used to build virtual tour guides that can give users exploring virtual sites and attractions individualized recommendations and information. These virtual assistants can comprehend user inquiries using natural language processing and respond with insightful and interesting information. Based on their likes and interests, they can also assist consumers in finding new attractions and undiscovered gems. This can improve the user experience and motivate visitors to go further into virtual worlds.

%\subsubsection{Language translation.}
%Real-time language translation services can be developed using ChatGPT to enable users to converse with others in virtual worlds who speak different languages. These translation services can deliver precise and nuanced translations in real-time using natural language processing and machine learning. This can increase the inclusivity and accessibility of virtual environments, facilitating global user collaboration and communication. Also, it can aid in removing language barriers and promote intercultural communication and comprehension.




\section*{ChatGPT as a destroyer}
%It's also possible that new technologies and services will emerge to compete with or even replace the metaverse industry. Here are a few ways that ChatGPT or other language models could have a negative impact on the metaverse market share:

%\begin{figure*}
%    \centering
 %   \includegraphics[angle=90,origin=c,scale=0.65]{figs/concerns.png}
%    \caption{ChatGPT as a ``Destroyer'' for the metaverse.}
%    \label{fig:concern}
%\end{figure*}


%\subsection{Engagement Detraction}
Chatbots and virtual assistants powered by language models could potentially replace human interaction in the metaverse, resulting in lower user engagement and socialization. If users become overly reliant on AI-powered interactions, the immersive experience that the metaverse is supposed to provide may suffer.
Users may become so accustomed to communicating with chatbots and virtual assistants and miss out on human interaction nuances such as body language, facial expressions, and other nonverbal cues.
%
Furthermore, Chatbots and virtual assistants may be programmed to respond to specific types of queries or commands in specific ways, which may result in a more limited range of interaction and less spontaneity in user-to-user communication. This could eventually undermine the immersive experience that the metaverse is supposed to provide. It could potentially pave the way for the creation of alternative types of virtual reality experiences that do not necessitate the same level of immersion as the metaverse. ChatGPT, for example, could be used to create virtual chat rooms that do not necessitate the same level of 3D graphics and virtual environment design that the metaverse does. If users find that virtual chat rooms powered by ChatGPT meet their needs, they may be less likely to explore the more immersive virtual worlds available in the metaverse. This could result in a shift away from more immersive virtual worlds and toward simpler virtual chat rooms in terms of market share.


Contents created by ChatGPT-like language models may become formulaic and unoriginal because, in the end, language models are trained on massive amounts of existing text which may limit their ability to create truly unique content. For example, if a virtual world within the metaverse is generated by a language model using pre-written scripts, the resulting world may feel sterile and uninspired. The virtual world's content may lack the depth and complexity that comes from human creativity and imagination, failing to engage users and keep them coming back.
%
Training with existing data also makes them more likely to produce content that reflects existing biases and trends in the Internet. As a result, there may be a lack of diversity and creativity in the metaverse, as content creators may rely too heavily on language models rather than exploring new ideas and taking creative risks.


\section*{Moving forward}
A number of important problems are worth addressing for the better integration of ChatGPT and Metaverse:
\begin{itemize}
    \item Privacy and data security. The user data on which ChatGPT relies heavily may contain private information such as browsing history, location information, and conversations. This raises privacy and data security concerns because hackers or other bad actors may gain access to or steal this information. researchers must take strong security precautions to protect user data, and consumers must be made aware of the collection and use of their data.
    \item Defamation and fake news. ChatGPT can generate intelligent and persuasive responses, but it is not always accurate or reliable, which means it has the potential to spread false information and fake news. Users should be taught how to critically evaluate the presented information, and researchers should take care to validate the veracity of ChatGPT's responses.
    \item Stereotyping and negative behavior reinforcement. The language creation features of ChatGPT have the potential to support unfavorable attitudes and prejudices. For example, programming a virtual assistant to respond to users in a passive or subservient manner can perpetuate abusive or sexist behavior. Virtual assistants and other metaverse apps must be designed in a way that promotes good habits and values. researchers must be aware of these risks.
    \item Unrealistic expectations. Users' expectations may be inflated as a result of ChatGPT's personalized and targeted responses, particularly when it comes to education or mental health care. Users may expect virtual assistants to respond quickly and completely to complex queries or issues, but this may not always be the case. Researchers must manage user expectations and ensure that virtual assistants provide the appropriate levels of assistance.
    \item Dependence on technology. Using ChatGPT and other virtual assistants in the metaverse may lead to a greater reliance on technology and less on interpersonal contact. As a result, socialization, communication, and emotional health may all suffer. Researchers must consider these risks when creating metaverse apps, balancing the benefits of technology with the importance of human connection and interaction.
\end{itemize}


As the metaverse develops, so will opportunities for ChatGPT and other forms of NLP in MMORPGs. The need for interactive and unique experiences will grow as more people join the metaverse. ChatGPT is just one example of a natural language processing and generation technology that may be used to build a variety of useful and entertaining virtual assistants and companion applications. ChatGPT's potential for facilitating the development of sophisticated conversational agents in the metaverse is an intriguing avenue for potential future study. With reinforcement learning and other methods, these agents could acquire a deeper comprehension of user actions and reactions in order to provide more tailored and interesting interactions. 
While researchers have an amazing chance to create individualized and engaging virtual experiences by combining ChatGPT with the metaverse, they need to be aware of the obstacles so as to make the virtual world a safe and ethical place for all users.


\bibliographystyle{unsrt}
\bibliography{iclr2022_conference}

\end{document}
