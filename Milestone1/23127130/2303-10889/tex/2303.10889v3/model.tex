\documentclass[a4paper,twoside,12pt]{article}
%%%%%%%%%%%%%%%%%%%%%%%%%%%%%%%%%%%%%%%%%%%%%%%%%%%%%%%%%%%%%%%%%%%%%%%%%%%%%%%%%%%%%%%%%%%%%%%%%%%%%%%%%%%%%%%%%%%%%%%%%%%%%%%%%%%%%%%%%%%%%%%%%%%%%%%%%%%%%%%%%%%%%%%%%%%%%%%%%%%%%%%%%%%%%%%%%%%%%%%%%%%%%%%%%%%%%%%%%%%%%%%%%%%%%%%%%%%%%%%%%%%%%%%%%%%%
\usepackage{amsmath,bm}
\usepackage{amsfonts}
\usepackage{amssymb}
\usepackage{mathrsfs}
\usepackage{colortbl}
\usepackage{geometry}
\usepackage{times}
\usepackage{enumerate}
\usepackage[pdfborder=false]{hyperref}
\usepackage{fullpage}
\usepackage[T1]{fontenc}
\usepackage{ae,aecompl}
\usepackage[round,authoryear]{natbib}
\usepackage{tikz}
\usepackage{color}
\usepackage{array}
\usepackage{graphicx}
\usepackage{epsfig}
\usepackage{slashbox}
\usepackage{lscape}
\usepackage{epic}
\usepackage{fixltx2e}
\usepackage{extarrows}
\usepackage{arcs}
\usepackage{ntheorem}
\usepackage{graphics}
\usepackage{tabularx}
\usepackage{makecell}
\usepackage{mathtools}
\usepackage{stmaryrd}



\makeatletter
\def\overrightharpoonfill@{\arrowfill@\relbar\relbar\rightharpoonup}
\DeclareRobustCommand{\overrightharpoon}{\mathpalette{\overarrow@\overrightharpoonfill@}}

\def\downrightharpoonfill@{\arrowfill@\relbar\relbar\rightharpoondown}
\DeclareRobustCommand{\downrightharpoon}{\mathpalette{\raise0.2em\underarrow@\downrightharpoonfill@}}
\makeatother


\setcounter{MaxMatrixCols}{10}
%TCIDATA{OutputFilter=LATEX.DLL}
%TCIDATA{Version=5.50.0.2953}
%TCIDATA{Codepage=1252}
%TCIDATA{<META NAME="SaveForMode" CONTENT="1">}
%TCIDATA{BibliographyScheme=BibTeX}
%TCIDATA{LastRevised=Monday, January 02, 2017 15:47:57}
%TCIDATA{<META NAME="GraphicsSave" CONTENT="32">}

\definecolor{webgreen}{rgb}{0,0.4,0}
\definecolor{webbrown}{rgb}{0.6,0,0}
\definecolor{purple}{rgb}{0.5,0,0.25}
\definecolor{darkblue}{rgb}{0,0,0.7}
\definecolor{darkred}{rgb}{0.7,0,0}
\hypersetup{colorlinks,citecolor=darkred,filecolor=black,linkcolor=darkblue,urlcolor=webgreen,pdfpagemode=None,pdfstartview=FitH}
\citestyle{authordate}
\geometry{left=2.4cm,right=2.4cm,top=3cm,bottom=3.1cm}
\renewcommand{\cite}{\citet}
\makeatletter
\newcommand{\ignore}[1]{}
\newtheorem{assumption}{{\sc Assumption}}
\newtheorem{clarification}{{\sc Clarification}}
\newtheorem{fact}{{\sc Fact}}
\newtheorem{implication}{{\sc Implication}}
\newtheorem{lemma}{{\sc Lemma}}
\newtheorem{lemmacontinued}{{\sc Lemma}}
\newtheorem{proposition}{{\sc Proposition}}
\newtheorem{corollary}{{\sc Corollary}}
\newtheorem{notation}{{\sc Notation}}
\newtheorem{theorem}{{\sc Theorem}}
\newtheorem{definition}{{\sc Definition}}
\newtheorem{observation}{{\sc Observation}}
\newtheorem{claim}{{\sc Claim}}
\newtheorem{conjecture}{{\sc Conjecture}}
\newtheorem{example}{{\sc Example}}
\newtheorem{examplecontinued}{{\sc Example}}
\newtheorem{remark}{{\sc Remark}}
\newtheorem{result}{{\sc Result}}
\newenvironment{proof}{\noindent {\bf \sl Proof\/}:\enspace}
{\hfill $\Box$ \vspace{0.5em}}

\renewcommand\section{\@startsection {section}{1}{\z@}{-3.5ex \@plus -1ex \@minus -.2ex}                                   {2.3ex \@plus.2ex}{\centering\large\scshape}}

\renewcommand\subsection{\@startsection {subsection}{1}{\z@}{-3.5ex \@plus -1ex \@minus -.2ex}                                   {2.3ex \@plus.2ex}{\raggedright\large\scshape}}

\newcommand\xxrightharpoondown[1]{\xrightharpoondown{\textstyle  #1 }}

\linespread{1.2}





\begin{document}

\title{\sc Decomposability and Strategy-proofness\\
in Multidimensional Models\thanks{Huaxia Zeng acknowledges that his work was supported by the
Program for Professor of Special Appointment (Eastern Scholar) at Shanghai Institutions of Higher Learning
(No.~2019140015) and the Fundamental Research Funds for the Central Universities (No.~2018110153).}}
\author{Shurojit Chatterji\thanks{%
School of Economics, Singapore Management University, Singapore.}~ and
Huaxia Zeng\thanks{%
School of Economics, Shanghai University of Finance and Economics, and the Key Laboratory of Mathematical Economics (SUFE), Ministry of Education, Shanghai 200433, China. }}
\date{\today }
\maketitle

\vspace{-1em}
\begin{abstract}
\noindent
We introduce the notion of a multidimensional hybrid preference domain on a (finite) set of alternatives that is a Cartesian product of finitely many components.
We demonstrate that in a model of public goods provision, multidimensional hybrid preferences arise naturally through assembling marginal preferences under the condition of semi-separability - a weakening of separability.
We then study strategy-proof rules on this domain and show that every such rule can be decomposed into component-wise strategy-proof rules.
More importantly, we show that under a suitable ``richness'' condition, every domain of preferences that reconciles decomposability with strategy-proofness must be a multidimensional hybrid domain.

\medskip \noindent \textit{Keywords}:
Decomposability; strategy-proofness; multidimensional hybrid preference

\noindent \textit{JEL Classification}: D71.
\end{abstract}


\section{Introduction}\label{sec:Introduction}

Public decisions entail vast expenditures on a variety of components such as defence, education, health.
A mechanism design approach would base the decisions on the set of alternatives,  formulated as the Cartesian product of these multiple components (denoted $A\coloneqq\times_{s \in M} A^s$),  on the preferences of agents over $A$.
Decision making in such multidimensional settings is considerably more tractable if it can be ``decomposed'', that is, if the social planner is able to take the decisions on each of the components  independently based on ``marginal'' preferences in each component that are derived from the ``overall'' preferences of agents over $A$, and then piece these component-wise decisions into a final social decision.
Of course, one would also like this decomposed decision making process to have nice incentive properties.
Thus we seek to study ``straightforward''  mechanisms in multidimensional settings, that is, mechanisms that are decomposable and strategy-proof.
%Furthermore, the mechanisms we propose will turn out to be fully determined by the peaks of agents preferences, so that for each component, the profile of the top ranked alternative across agents on that component is fed into a marginal social choice function to ascertain the choice on that component, and finally the social alternative is constructed by collecting these component wise choices.

If overall preferences satisfy \emph{separability}\footnote{Intuitively speaking, an overall preference preserves the property of separability if
a marginal preference on each component can be induced such that
between any two distinct alternatives, the one endowed with a better element at each disagreed component is preferred.
Separability is an important preference restriction widely investigated in both the literature on strategic voting \citep[e.g.,][]{LS1999} and
on mechanism design with monetary compensations \citep[e.g.,][]{R1979}.},
then a straightforward mechanism for social decisions can be simply constructed by assembling independent component-wise mechanisms that are also strategy-proof.
However, separability is too demanding; for instance, in a model of club member recruitment  \citep[see][]{BSZ1991}, one might imagine that while the appointment of exactly one candidate is preferred to nobody being appointed, it may be less desirable  to recruit all candidates, while in an auction model with non-quasilinear preferences \citep[see][]{MS2015},
large-scale payments might influence an agent's ability to utilize objects.
With a view to broadening the scope of straightforward mechanism design in multidimensional settings, we seek to develop here a methodology that accommodates non-separable preferences that go beyond multidimensional single-peakedness of \citet{BGS1993}, and allows us to answer the  following question:
What sort of preference domain over alternatives (formulated as a Cartesian product of multiple components)  reconciles decomposability with strategy-proofness, that is, predicated on some way of deriving marginal preferences on each component,
(i) every rule\footnote{We focus on strategy-proofness, wherein a direct mechanism reduces to a social choice function that picks an outcome from $A$ for each preference profile. The social choice function will be assumed to satisfy the mild requirement of unanimity and we use the term ``rule'' to refer to a unanimous social choice function.} that is strategy-proof turns out to be decomposable into strategy-proof component-wise rules over marginal preferences, and
(ii) conversely, arbitrary strategy-proof component-wise rules can be assembled into a strategy-proof rule?

Once non-separable preferences are involved, the derivation of marginal preferences on a component may vary with the specification of elements on the remaining components \citep[see][]{LW1999}.
This would affect the scope for designing strategy-proof component-wise rules, which would in turn
affect the class of decomposable, strategy-proof rules.\footnote{More specifically, if ``too many'' marginal preferences are derived, only dictatorships on each component survive strategy-proofness. This implies that any strategy-proof rule other than a \emph{generalized dictatorship} (intuitively speaking, a combination of dictatorships on all components) fails to be decomposable, and the scope for assembling strategy-proof component-wise rules is limited to the class of generalized dictatorships.\label{footnote:generalizeddictatorship}}
We propose a natural way of deriving marginal preferences from overall preferences as follows.
Given an overall preference over $A$, simply refer to the top-ranked alternative ($a \coloneqq (a^s, a^{-s})$) and derive the marginal preference over a pair of elements $x^s$ and $y^s$ by comparing $(x^s,a^{-s})$ with $(y^s,a^{-s})$ in the overall preference.\footnote{This of course works for separable preferences as well, because according to a separable overall preference, one can derive a unique marginal preference on each component by fixing an arbitrary vector on the remaining components.}
We show that this way of deriving marginal preferences allows us to reconcile decomposability with   strategy-proofness on a general class of multidimensional models where preference domains satisfy a condition called \emph{multidimensional hybridness}.
Thus while our way of deriving marginal preferences may appear ad hoc\footnote{That said, the top-ranked alternative does play the role of an important benchmark in the study of strategy-proof rules. Indeed in many models, in particular ours, all strategy-proof rules are completely and endogenously determined by the profiles of top-ranked alternatives, i.e., satisfy \emph{the tops-only property}.},
it has the merit of precipitating the decomposability property on all strategy-proof rules defined on the domain we identify.
Hence, it allows social decisions to be made component wise and then assembled, thereby simplifying the task confronting the social planner.



We make the following claims on domains of multidimensional hybrid preferences. First,
the notion of multidimensional hybridness allows for more flexible descriptions than separable and multidimensional single-peaked preferences respectively (see condition (\ref{condition:MH}) and the illustration in Section \ref{sec:heuristicexample}).
Hence, multidimensional hybrid domains variously contain separable preferences, multidimensional single-peaked preferences and top-separable preferences of \citet{LW1999}.
Next, we demonstrate in a heuristic example of public goods provision in Section \ref{sec:heuristicexample},
that requiring ``semi-separability'' - a weakening of separability, in the procedure
wherein the information of hybridness restriction \citep[introduced by][]{CRSSZ2022} is extracted from the domains of marginal preferences and embedded into the overall preferences, provides an intuitive route to the generation of multidimensional hybrid preferences.
Finally, we show that on a class of rich domains\footnote{The richness condition requires a domain to satisfy \emph{minimal richness} (i.e., each alternative is top-ranked in some preference), \emph{diversty\textsuperscript{+}} (i.e., inclusion of two separable preferences that are complete reversals), and \emph{the Interior\textsuperscript{+} and Exterior\textsuperscript{+} properties} which extend the no-restoration condition of \citet{S2013} to a graph over all preferences formulated by the localness notions of \emph{adjacency} and \emph{adjacency\textsuperscript{+}} (see Section \ref{sec:theorem}), so that the difference between any two preferences can be reconciled via a sufficiently short path in the graph.}, multidimensional hybrid domains are the unique ones that reconcile decomposability with strategy-proofness (see Theorem \ref{thm}).
This in return enables us to fully characterize strategy-proof rules on a rich multidimensional hybrid domain (see Corollary \ref{cor:characterization}),
so that earlier characterization results on the separable domain, the multidimensional single-peaked domain and the top-separable domain emerge as special cases of our analysis.
A key step in our analysis is establishing endogenously the tops-only property for all strategy-proof rules on a rich domain, and for marginal rules on domains of marginal preferences\footnote{In most cases, the tops-only property is necessary for the decomposability of strategy-proof rules \citep[see for instance,][]{BSZ1991,BGS1993,LW1999}.},
which clearly further simplifies the task of the designer since the social decision on each component is determined by the profile of peaks on that component.
%Next, we consider a model of public goods provision and demonstrate how multidimensional hybrid preferences may arise in such a setting. We contend that requiring ``semi-separability''  in the procedure wherein marginal preferences are combined to generate overall preferences provides an intuitive route to multidimensional hybridness.\footnote{Semi-separability weakens the restriction of separability. It extracts pervasive preference information embedded in the domains of marginal preferences, and requires that the overall preference respects such information. Specifically, fixing a domain of marginal preferences in the component $A^s$, it may turn out that in all marginal preferences with the peak $a^s$, $b^s$ is ranked above $c^s$ (such a preference restriction always exists, since it is possible that $a^s = b^s$).
%Then, semi-separability requires each overall preference that includes $a^s$ in its top-ranked alternative,  ranks $(b^s, x^{-s})$ above $(c^s, x^{-s})$ for all $x^{-s} \in A^{-s}$.}

This paper is organized as follows.
Section \ref{sec:heuristicexample} provides a heuristic example of public goods provision to illustrate how multidimensional hybrid preferences arise under the condition of semi-separability.
Section \ref{sec:preliminaries} sets out the model and preliminaries.
In Section \ref{sec:result}, we formally introduce multidimensional hybrid domains and establish the characterization results, while Section \ref{sec:conclusion} contains some final remark and a review of the literature.
All omitted proofs are gathered in an Appendix.


\section{A heuristic example}\label{sec:heuristicexample}


Imagine a railway running within a region, which contains several stations $\Omega = \{l_1, \dots, l_v\}$, $v \geq 2$.
A set of multiple good public facilities $M = \{1, \dots, m\}$, $m \geq 2$, like a sports complex, a hospital, etc, needs to be allocated on stations.
For each public facility $s$, let a subset $A^s \subseteq \Omega$ collect all multiple locations available for construction.
Correspondingly, an $m$-tuple $(a^1, \dots, a^m) \in \times_{s \in M}A^s$ represents a feasible allocation of these $m$ public facilities, and the Cartesian product $A \coloneqq \times_{s \in M}A^s$ denotes the set of all feasible allocations.

We first briefly introduce multidimensional hybrid preferences.
For each public facility $s$, according to the railway,
we endow a prior linear order $\prec^s$ on the feasible locations of $A^s$
such that for all distinct locations $a^s = l_p\in A^s$ and $b^s = l_q\in A^s$, $[a^s \prec^s b^s] \Leftrightarrow [p< q]$.
Then, let
$\textrm{Int}\langle x^s, y^s\rangle \coloneqq \big\{a^s \in A^s: x^s \prec^s a^s \prec^s y^s,\; \textrm{or}\; y^s \prec^s a^s \prec^s x^s\big\}$ denote a set collecting all locations that are strictly between $x^s$ and $y^s$.
Furthermore, we fix two \emph{threshold locations} $\underline{x}^s, \overline{x}^s \in A^s$ such that
either $\underline{x}^s = \overline{x}^s$, or $\underline{x}^s \prec^s \overline{x}^s$ and $\textrm{Int}\langle \underline{x}^s, \overline{x}^s\rangle \neq \emptyset$.
Thus, we assemble a grid $\prec\,\coloneqq \times_{s \in M}\prec^s$ and
two \emph{threshold allocations} $\underline{x}\coloneqq (\underline{x}^1, \dots, \underline{x}^m)$ and
$\overline{x} \coloneqq (\overline{x}^1, \dots, \overline{x}^m)$ on $\prec$.
An overall preference $P_i$ (indeed, a linear order) over $A$ where the allocation $x$ is top-ranked, is \textbf{multidimensional hybrid} on $\prec$ w.r.t.~$\underline{x}$ and $\overline{x}$ if
for any allocations $a, b \in A$ that disagree on exactly one dimension, say $a^s \neq b^s$ and $a^{-s} = b^{-s}$,
$a$ is strictly preferred to $b$ in $P_i$ (denoted $a\mathrel{P_i}b$) whenever
either $a^s = x^s$,
or $a^s$ is located strictly between $x^s$ and $b^s$ and
$a^s$ is \emph{not} located strictly between $\underline{x}^s$ and $\overline{x}^s$, i.e.,
\begin{align}\label{condition:MH}
	\big[\textrm{either}\; a^s = x^s,\;
	\textrm{or}\; a^s \in \textrm{Int}\langle x^s, b^s\rangle\;\textrm{and}\; a^s \notin \textrm{Int}\langle \underline{x}^s, \overline{x}^s\rangle
	\big]
	\Rightarrow \big[a\mathrel{P_i}b\big].
\end{align}
Note that in the extreme case that $\underline{x} = \overline{x}$, multidimensional hybridness is strengthened to the conventional restriction of multidimensional single-peakedness, while
 the choice of distinct threshold allocations provides freedom to rank alternatives in ways that go beyond the requirement of multidimensional single-peakedness; for instance,
in a multidimensional hybrid preference $P_i$ where $x$ is top-ranked, given
$a^s \in \textrm{Int}\langle x^s, b^s\rangle$ and $a^s \in \textrm{Int}\langle \underline{x}^s, \overline{x}^s\rangle$,
we can have $(a^s, y^{-s})\mathrel{P_i}(b^s, y^{-s})$ and $(b^s, z^{-s})\mathrel{P_i}(a^s, z^{-s})$ simultaneously, which of course also indicate a violation of separability.

We now start to discuss how such preferences may arise in this model of public goods provision
given the underlying transportation network described above.
For each public facility $s$,
an individual $i$ living in this region has a marginal preference $P_i^s$ (indeed, a linear order) over the feasible locations $A^s$.
Each individual's marginal preference is private information, and
the social planner only knows that a domain of marginal preferences $\mathbb{D}^s$ contains all individuals' marginal preferences.
Given locations $x^s, a^s, b^s \in A^s$ such that $a^s \neq b^s$,
according to the marginal domain $\mathbb{D}^s$,
let $\overrightharpoon{(x^s, a^s, b^s)}$ specify a preference restriction indicating
that $a^s$ is always ranked above $b^s$ in a marginal preference where $x^s$ is top-ranked.
Accordingly, let $\mathcal{R}(\mathbb{D}^s) \coloneqq \big\{\overrightharpoon{(x^s, a^s, b^s)}:
[x^s\; \textrm{is top-ranked in}\; P_i^s\in \mathbb{D}^s]\Rightarrow [a^s\mathrel{P_i^s}b^s]\big\}$ collect all such preference restrictions embedded in $\mathbb{D}^s$.
Each individual $i$ also has an overall preference $P_i$ over all feasible allocations $A$, which is assumed to be private information as well.
The social planner of course will make some inference on individuals' overall preferences based on the known information.
Specifically, we assume that the social planner believes that each individual's overall preference is formulated according to the condition of \emph{semi-separability} which requires the rankings of allocations to fully respect the preference restrictions embedded in all marginal domains $\mathbb{D}^1, \dots, \mathbb{D}^m$.
Formally,
according to $\mathcal{R}(\mathbb{D}^1), \dots, \mathcal{R}(\mathbb{D}^m)$,
an overall preference $P_i$, where the allocation $x$ is top-ranked,
is \textbf{semi-separable} if for all allocations $a,b \in A$,
we have
\begin{align}\label{def:SS}
\left[\overrightharpoon{(x^s, a^s, b^s)} \in \mathcal{R}(\mathbb{D}^s)\; \textrm{for all}\; s \in M\; \textrm{such that}\; a^s \neq b^s\right] \Rightarrow \big[a\mathrel{P_i}b\big].\footnotemark
\end{align}

\footnotetext{Indeed, semi-separability weakens the restriction of separability. Given a marginal preference $P_i^s \in \mathbb{D}^s$ for each component $s \in M$, we assemble an overall preference $P_i$ that satisfies separability. It is evident that $P_i$ also satisfies the requirement of semi-separability. Conversely, a semi-separable may violates the requirement of separability. For instance, given $s \in M$ and $x^s, a^s, b^s \in A^s$ such that $a^s \neq b^s$, $\overrightharpoon{(x^s, a^s, b^s)} \notin \mathcal{R}(\mathbb{D}^s)$ and $\overrightharpoon{(x^s, a^s, b^s)} \notin \mathcal{R}(\mathbb{D}^s)$,
a semi-separable preference $P_i$ that includes $x^s$ in its top-ranked allocation, can contain $(a^s, y^{-s})\mathrel{P_i} (b^s, y^{-s})$ and
$(b^s, z^{-s})\mathrel{P_i} (a^s, z^{-s})$ simultaneously.}

%A domain $\mathbb{D}$ is called \textbf{the semi-separable domain} w.r.t.~$\mathcal{R}(\mathbb{D}^1), \dots, \mathcal{R}(\mathbb{D}^m)$ if
%we have
%\begin{align*}
%[P_i \in \mathbb{D}] \Leftrightarrow
%[P_i\; \textrm{is semi-separable w.r.t.}~\mathcal{R}(\mathbb{D}^1), \dots, \mathcal{R}(\mathbb{D}^m)].
%\end{align*}


\begin{figure}[t]
\begin{center}
\vspace{-2em}
\includegraphics[width=0.75\textwidth]{fig1.jpg}
\end{center}
\vspace{-1em}
\caption{\small The transportation system in the region}\label{fig:region}
\end{figure}

Furthermore, we assume that an urban area stands in the center of the region, surrounded by a large rural area,
the railway goes through the urban area such that all multiple urban stations/locations (more than three) cluster in the middle of the railway, and
the urban area is postulated to possess a modern metro transportation system that fully connects all urban stations (see for instance Figure \ref{fig:region}).\footnote{In Figure \ref{fig:region}, the bold line in the bigger oval represents the railway, while the dashed lines denote the metro transportation system in the urban area that complements the railway.}
Then, we can identify two particular urban stations $l_{\underline{k}}, l_{\overline{k}} \in \Omega$, $\underline{k}< \overline{k}$, that separate the railway into three parts $\mathcal{L} = \{l_1, \dots, l_{\underline{k}}\}$,
$\mathcal{M} = \{l_{\underline{k}}, \dots, l_{\overline{k}}\}$ and
$\mathcal{R} = \{l_{\overline{k}}, \dots, l_v\}$
where all stations of $\mathcal{M}$ are urban stations, and
the rest are all rural stations.
As the station $l_{\underline{k}}$ is directly connected to all urban stations and is the gate to the rural stations left to it (see for instance $l_4$ in Figure \ref{fig:region}), it can be viewed as the left transportation hub. Symmetrically, $l_{\overline{k}}$ is called the right transportation hub (see for instance $l_9$ in Figure \ref{fig:region}).
Moreover, for simplicity, we assume that the two transportation hubs are available for the construction of a public facility $s$ whenever at least two urban locations are included in $A^s$, i.e., $\big[|A^s\cap \mathcal{M}|\geq 2\big]\Rightarrow \big[l_{\underline{k}}, l_{\overline{k}} \in A^s\big]$.
Thus, we can identify threshold locations $\underline{x}^s$ and $\overline{x}^s$ on $\prec^s$ such that
$\big[|A^s\cap \mathcal{M}| \leq 2\big] \Rightarrow \big[\underline{x}^s=\overline{x}^s \in A^s\; \textrm{is arbitrary}\big]$ and
$\big[|A^s\cap \mathcal{M}| > 2\big] \Rightarrow \big[\underline{x}^s= l_{\underline{k}}\; \textrm{and}\;\overline{x}^s =l_{\overline{k}}\big]$.

It is natural to assume here that
the formulation of an individual's marginal preference over locations of $A^s$ is determined by the
distance between locations measured by both the railway and the metro transportation system.
Specifically,
if $|A^s\cap \mathcal{M}| \leq 2$,
an individual would prefer an available location that is closer to his/her own location along the railway, and therefore has \emph{single-peaked} preferences on $A^s$ w.r.t.~$\prec^s$; if $|A^s\cap \mathcal{M}| > 2$,
an individual living around a rural station would have single-peaked preferences on available locations that lie at the two sides of the transportation hubs along the railway, and
prefer the proximate transportation hub to all other available locations in the urban area,
while an individual living in the urban area would have arbitrary preferences on all urban available locations attributed to the complementary metro transportation, and prefer an rural available location that is closer to its proximate transportation hub.
In conclusion, by adopting the notion of hybridness introduced by \citet{CRSSZ2022}, we summarize that each marginal preference $P_i^s \in \mathbb{D}^s$ is \textbf{hybrid} on $\prec^s$ w.r.t.~$\underline{x}^s$ and $\overline{x}^s$,
i.e., say that $x^s$ is top-ranked in $P_i^s$, given distinct $a^s, b^s \in A^s$,
\begin{align}\label{def:h}
\big[a^s \in \textrm{Int}\langle x^s, b^s\rangle\;\textrm{and}\; a^s \notin \textrm{Int}\langle \underline{x}^s, \overline{x}^s\rangle\big]
\Rightarrow \big[a^s\mathrel{P_i^s}b^s\big].
\end{align}
Immediately, after assembling the threshold allocations $\underline{x} \coloneqq (\underline{x}^1, \dots, \underline{x}^m)$ and $\overline{x} \coloneqq (\overline{x}^1, \dots, \overline{x}^m)$,
we claim that the hybridness restrictions contained in $\mathcal{R}(\mathbb{D}^1), \dots, \mathcal{R}(\mathbb{D}^m)$ are syncretized via semi-separability into the multidimensional hybridness of each overall preference on $\prec$ w.r.t.~$\underline{x}$ and $\overline{x}$ (recall condition (\ref{condition:MH})).\footnote{Let $x$ be top-ranked in a semi-separable preference $P_i$.
Fix two allocations $a, b \in A$ such that $a^s \neq b^s$ and $a^{-s} = b^{-s}$ for some $s \in M$.
If $a^s = x^s$, it is evident that $a^s$ is ranked above $b^s$ in every marginal preference where $x^s$ is top-ranked.
If $a^s \neq x^s$, let $a^s \in \textrm{Int}\langle x^s, b^s\rangle$ and $a^s \notin \textrm{Int}\langle \underline{x}^s, \overline{x}^s\rangle$.
Then, by hybridness, $a^s$ is ranked above $b^s$ in every marginal preference where $x^s$ is top-ranked.
Therefore, we have $\overrightharpoon{(x^s, a^s, b^s)} \in \mathcal{R}(\mathbb{D}^s)$, and hence
$a\mathrel{P_i}b$ by semi-separability, which meets the requirement of multidimensional hybridness.}
Moreover, if each marginal domain contains \emph{all} hybrid marginal preferences,
then for each public facility $s$, it is true that
$\overrightharpoon{(x^s, a^s, b^s)} \in \mathcal{R}(\mathbb{D}^s)$ if and only if
either $x^s = a^s$, or $a^s \in \textrm{Int}\langle x^s, b^s\rangle$ and $a^s \notin \textrm{Int}\langle \underline{x}^s, \overline{x}^s\rangle$.
Hence, an overall preference is multidimensional hybrid on $\prec$ w.r.t.~$\underline{x}$ and $\overline{x}$ if and only if it is semi-separable.

In the remainder of the paper, we establish a general multidimensional model,
where multidimensional hybrid preferences are introduced without any imposition of additional condition like semi-separability in their formulation, and explore the salience of multidimensional domains by characterizing
that under some mild richness condition, multidimensional hybridness is necessary and sufficient for a  preference domain to reconcile decomposability with strategy-proofness.


%\begin{proposition}\label{prop:semi-separability}
%For each $s \in M$, let $\mathbb{D}^s$ contain all hybrid marginal preferences on $\prec^s$ w.r.t.~$\underline{x}^s$ and $\overline{x}^s$.
%Then, the semi-separable domain $\mathbb{D}$ w.r.t.~$\mathcal{R}(\mathbb{D}^1), \dots, \mathcal{R}(\mathbb{D}^m)$ equals the multidimensional hybrid domain $\mathbb{D}_{\emph{MH}}(\prec, \underline{x}, \overline{x})$, and therefore is a decomposable domain.
%\end{proposition}

%\begin{proof}
%Given $P_i \in \mathbb{D}$, let $r_1(P_i) = x$.
%In order to prove $P_i \in \mathbb{D}_{\textrm{MH}}(\prec, \underline{x}, \overline{x})$,
%we fix similar $a, b \in A$, say $M(a,b) = \{s\}$, such that either $a^s = x^s$, or $a^s \in \textrm{Int}\langle x^s, b^s\rangle$ and $a^s \notin \textrm{Int}\langle \underline{x}^s, \overline{x}^s\rangle$, and show $a\mathrel{P_i}b$.
%If $a^s = x^s$, it is evident that $a^s\mathrel{P_i^s}b^s$ for all $P_i^s \in \mathbb{D}^s$ with $r_1(P_i^s) = x^s$.
%If $a^s \neq x^s$, by hybridness on $\prec^s$ w.r.t.~$\underline{x}^s$ and $\overline{x}^s$,
%$a^s \in \textrm{Int}\langle x^s, b^s\rangle$ and $a^s \notin \textrm{Int}\langle \underline{x}^s, \overline{x}^s\rangle$ imply $a^s\mathrel{P_i^s}b^s$ for all $P_i^s \in \mathbb{D}^s$ with $r_1(P_i^s) = x^s$.
%Thus, we have $\overrightharpoon{(x^s, a^s, b^s)} \in \mathcal{R}(\mathbb{D}^s)$.
%Then, Definition \ref{def:semiseparability} implies $a\mathrel{P_i}b$, as required.
%Conversely, given $P_i \in \mathbb{D}_{\textrm{MH}}(\prec, \underline{x}, \overline{x})$, we show $P_i \in \mathbb{D}$.
%Let $r_1(P_i) = x$ and $a,b \in A$ be such that $\overrightharpoon{(x^s, a^s, b^s)} \in \mathcal{R}(\mathbb{D}^s)$ for all $s \in M(a,b)$.
%We show $a \mathrel{P_i} b$.
%For each $s \in M(a,b)$, since $\mathbb{D}^s$ contains all hybrid marginal preferences on $\prec^s$ w.r.t.~$\underline{x}^s$ and $\overline{x}^s$,
%we know that whenever $a^s \neq x^s$,
%if $a^s \notin \textrm{Int}\langle x^s, b^s\rangle$ or $a^s \in \textrm{Int}\langle \underline{x}^s, \overline{x}^s\rangle$,
%there exists a marginal preference of $\mathbb{D}^s$ such that $x^s$ is top-ranked, and $b^s$ is ranked above $a^s$.
%Therefore, to ensure $\overrightharpoon{(x^s, a^s, b^s)} \in \mathcal{R}(\mathbb{D}^s)$,
%it must be the case that either $a^s = x^s$, or $a^s \in \textrm{Int}\langle x^s, b^s\rangle$ and $a^s \notin \textrm{Int}\langle \underline{x}^s, \overline{x}^s\rangle$.
%Then, by Definition \ref{def:MH} and transitivity of $P_i$, we have $a \mathrel{P_i} b$, as required.
%In conclusion, $\mathbb{D} = \mathbb{D}_{\textrm{MH}}(\prec, \underline{x}, \overline{x})$.
%\end{proof}



\section{Preliminaries}
\label{sec:preliminaries}

Let $A$ be a finite set of alternatives with $|A| \geq 3$.
We throughout the paper assume
that the alternative set is represented by a \textbf{Cartesian product}
of a finite number of sets, each of which contains finitely many elements.
Formally, we fix $A = \times_{s \in M}A^{s}$ where $M= \{1, \dots, m\}$, $m \geq 2$ is an integer, and $2 \leq |A^{s}| < \infty$ for each $s \in M$.\footnote{Note that the condition $|A^s| \geq 2$ for all $s \in M$, ensures indispensability of all components.}
Here, each $s \in M$ is called a \textbf{component}; $A^{s}$ is referred to as a \textbf{%
component set}, and an element in $A^{s}$ is denoted as $a^{s}$.
Accordingly, an alternative is represented by an $m$-tuple,
i.e., $a \coloneqq (a^1, \dots, a^{m})$.
To highlight a component $s \in M$, we also write $a = (a^{s}, a^{-s})$.
Given two alternatives $a,b\in A$, let $M(a,b) \coloneqq \{s \in M: a^s \neq b^s\}$ denote the set of components on which $a$ and $b$ disagree.
In particular, two alternatives $a$ and $b$ are said \textbf{similar}
if $|M(a,b)| = 1$.
Given $s \in M$ and $x^{-s}\in A^{-s}$,
let $(A^s, x^{-s}) \coloneqq \{a \in A: a^{-s} = x^{-s}\}$.


%Given two alternatives $x,y \in A$ and a subset $M' \subseteq M$,
%let $[x/y]_{M'} = (y^{M'}, x^{-M'})$ denote the alternative generated by
%using the elements $y^{M'}$ to replace $x^{M'}$ in $x$.\footnote{Clearly, if $M' = \emptyset$, then
%$[x/y]_{M'} = x$, and if $M' = M$, then
%$[x/y]_{M'} = y$.}


Let $N \coloneqq \{1, \dots, n\}$ be a finite set of voters with $n \geq 2$.
Each voter $i$ has a preference order $P_{i}$ over $A$ which is complete,
antisymmetric and transitive, i.e., a \emph{linear order}.
For any $a, b \in A$,
$a\mathrel{P_{i}}b$ is interpreted as ``$a$ is strictly preferred to $b$
according to $P_{i}$".
%\footnote{%
%In a table, we specify a preference ``vertically", while in a sentence, we specify
%a preference ``horizontally". For instance, preference$P_i$: $a_{\rightharpoonup}b_{%
%\rightharpoonup}c_{\rightharpoonup}\cdots$ is one where $a$ is at the
%top, $b$ is the second best, $c$ is the third ranked alternative while the
%remaining alternatives are arbitrarily ranked.}
Given a preference $P_{i}$, let $r_{k}(P_{i})$, where $1 \leq k \leq |A|$, denote the $k$th ranked
alternative in $P_{i}$.
Moreover, given a nonempty subset $B \subseteq A$, let $\max^{P_i}(B)$ and $\min^{P_i}(B)$ be the best and worst alternatives in $B$ according to $P_i$ respectively.
Two preferences $P_i$ and $P_i'$ are called \textbf{complete reversals} if for all distinct $a,b \in A$, we have
$[a\mathrel{P_i}b] \Leftrightarrow [b\mathrel{P_i'}a]$.
Let $\mathbb{P}$
denote the set containing \emph{all} linear orders over $A$. The set of all
admissible preferences is a set $\mathbb{D} \subseteq \mathbb{P}$, referred
to as a \textbf{preference domain}.\footnote{In this paper, $\subseteq$ and $\subset$ denote the weak and strict inclusion relations respectively.}
We call $\mathbb{P}$ \textbf{the universal domain}.
Given $a \in A$, let $%
\mathbb{D}^{a} \coloneqq \{P_{i} \in \mathbb{D}: r_{1}(P_{i}) =a\}$.
Correspondingly, a domain $\mathbb{D}$
is \textbf{minimally rich} if $\mathbb{D}^{a} \neq \emptyset$ for every $a
\in A$. Throughout the paper, all domains under investigation are assumed to be minimally rich.
A \textbf{preference profile} is an $n$-tuple $P \coloneqq (P_{1}, \dots, P_{n})
= (P_{i}, P_{-i}) \in \mathbb{D}^{n}$.
Analogously, for each $s \in M$,
let $P_i^s$ denote a \textbf{marginal preference} over $A^s$, $\mathbb{P}^s$ denote  \textbf{the marginal universal domain}, and
$\mathbb{D}^s \subseteq \mathbb{P}^s$ denote an admissible \textbf{marginal domain}.


A \textbf{Social Choice Function} (or \textbf{SCF}) is a map $f: \mathbb{D}^{n} \rightarrow A$,
which associates to each preference profile $P \in \mathbb{D}^{n}$, a
``socially desirable'' outcome $f(P)$.
First, an SCF $f: \mathbb{D}^{n}
\rightarrow A$ is required to be \textbf{unanimous}, i.e.,
for all $a \in A$ and $P \in \mathbb{D}^n$,
we have
$[r_{1}(P_{i}) = a$ for all $i \in N] \Rightarrow [f(P) = a]$.
For ease of presentation, a unanimous SCF henceforth is called a \textbf{rule}.
Next, an SCF $f: \mathbb{D}^n \rightarrow A$ satisfies the \textbf{tops-only property} if
for all $P,P' \in \mathbb{D}^n$, we have $[r_1(P_i) = r_1(P_i')\; \textrm{for all}\; i \in N] \Rightarrow
[f(P) = f(P')]$.
Last, an SCF $f: \mathbb{D}^n \rightarrow A$ is \textbf{%
strategy-proof} if for all $i \in N$, $P_{i}, P_{i}^{\prime } \in \mathbb{D}$
and $P_{-i} \in \mathbb{D}^{n-1}$, we have $[f(P_i,P_{-i}) \neq f(P_i', P_{-i})] \Rightarrow [f(P_i,P_{-i}) \mathrel{P_i} f(P_i', P_{-i})]$.
Analogously, given $s \in M$, a \textbf{marginal SCF} is a map $f^s: [\mathbb{D}^s]^n \rightarrow A^s$.
The three axioms above also apply to marginal SCFs.
A unanimous marginal SCF is henceforth called a \textbf{marginal rule}.

%In particular, given a domain $\mathbb{D}$, if all strategy-proof rules $f: \mathbb{D}^n \rightarrow A$, $n \geq 2$, satisfy the tops-only property,
%we call $\mathbb{D}$ a \textbf{tops-only domain}. The same definition applies to a marginal domain, called a \textbf{tops-only marginal domain}.

%It is worth mentioning that if two separable preferences have distinct marginal preferences at the component $s$, then their \emph{minimum} Kemeny distance is $\Pi_{t \neq s}\tau^t$.\footnote{For instance, given two separable preferences $\bar{P}_i$ and $\hat{P}_i$,
%assume $a^s\bar{P}_i^sb^s$ and $b^s\hat{P}_i^sa^s$.
%Then, by separability, $\bar{P}_i$ and $\hat{P}_i$ must at least disagree on the relative ranking
%between $(a^s, x^{-s})$ and $(b^s, x^{-s})$ for all $x^{-s} \in A^{-s}$,
%which involves $\Pi_{t \neq s}\tau^t$ pairs of alternatives.
%By adopting the notion of \textit{lexicographical separability},
%we can construct a pair of adjacent\textsuperscript{$\ast$} preferences in the following three steps.
%First, we first fix marginal preferences $P_i^1, \dots, P_i^s, \dots, P_i^m$ and $P_i^{s\,'}$ such that $P_i^s \sim P_i^{s\,'}$.
%Second, we fix a linear order $\rhd$ over $M$ such that $[t \neq s] \Rightarrow [t \rhd s]$.
%Last, we assemble a \textbf{lexicographically separable preference} $P_i$, using the marginal preferences $P_i^1, \dots, P_i^s, \dots, P_i^m$ and the linear order $\rhd$, i.e., $[xP_iy] \Leftrightarrow \big[x^q P_i^q y^q\;\textrm{for some}\; q \in M\; \textrm{and}\; x^p = y^p\; \textrm{for all}\; p \in M\; \textrm{with}\; p \rhd q\big]$. Symmetrically, we assemble a lexicographically separable preference $P_i'$, using the marginal preferences $P_i^1, \dots, P_i^{s\,'}, \dots, P_i^m$ and the linear order $\rhd$.
%Then, it can be easily verified that $P_i \sim^{\ast} P_i'$.}

%\vspace{-1em}
%\subsection{\rm Graphs}
%
%In this section, we introduce some standard concepts from graph theory that will be repeatedly used throughout the paper.
%An (undirected) \textbf{graph}, denoted $G \coloneqq \langle V, \mathcal{E}\rangle$, is a combination of a ``vertex set'' $V$, and an ``edge set'' $\mathcal{E} \subseteq V\times V$ such that
%$\big[(\alpha, \beta) \in \mathcal{E}\big] \Rightarrow \big[\alpha \neq \beta\; \textrm{and}\;  (\beta, \alpha) \in \mathcal{E}\big]$.
%A vertex $\alpha \in V$ is called a \textbf{leaf} if there exists a unique vertex $\beta \in V$ such that $(\alpha, \beta) \in \mathcal{E}$.
%Given two distinct vertices $\alpha, \beta \in V$, a \textbf{path} in $G =\langle V, \mathcal{E}\rangle $ connecting $\alpha$ and $\beta$ is a sequence of ``non-repeated'' vertices $(\alpha_1, \dots, \alpha_v)$, $v \geq 2$, such that
%$\alpha_1 = \alpha$, $\alpha_v = \beta$, $\alpha_k \in V$ for all $k = 1, \dots, v$, and
%$(\alpha_k, \alpha_{k+1}) \in \mathcal{E}$ for all $k = 1, \dots, v-1$.
%A graph $G = \langle V, \mathcal{E}\rangle$ is called a \textbf{connected graph} if
%for each pair of distinct vertices, there exists a path in $G$ connecting them.
%%In particular, a graph $G = \langle V, \mathcal{E}\rangle$ is called a \textbf{complete graph} if any two distinct vertices form an edge.
%
%Note that the vertex set $V$ here can represent a subset of alternatives,
%elements in one component set, preferences, or marginal preferences.
%We provide one important illustration.
%Given a marginal domain $\mathbb{D}^s$, two distinct elements $a^s, b^s \in A$ are said \textbf{strongly connected},
%denoted $a^s \approx b^s$,
%if there exist $P_i^s, P_i^{s\,'} \in \mathbb{D}^s$ such that
%$r_1(P_i^s) = r_2(P_i^{s\,'}) = a^s$, $r_1(P_i^{s\,'}) = r_2(P_i^s) = b^s$ and
%$r_k(P_i^s) = r_k(P_i^{s\,'})$ for all $k = 3, \dots, |A^s|$.\footnote{The notion of strong connectedness between alternatives was introduced by \citet{CSS2013}.}
%Accordingly, given a nonempty subset $B^s \subseteq A^s$,
%we can induce a graph $G_{\approx}^{B^s} \coloneqq \langle B^s, \mathcal{E}_{\approx}^{B^s}\rangle$ such that two elements of $B^s$ form an edge if and only if they are strongly connected, i.e., $\mathcal{E}_{\approx}^{B^s}= \{(a^s, b^s): a^s, b^s \in B^s\; \textrm{and}\; a^s \approx b^s\}$.
%
%
%\subsection{\rm Hybrid marginal preferences}
%
%In this section, we introduce hybrid marginal preferences.
%Fix a component $s \in M$ and a linear order $\prec^s$ over $A^s$.\footnote{Let $a^s \preccurlyeq^s b^s$ denote either $a^s \neq b^s$ and $a^s \prec^s b^s$, or $a^s = b^s$.
%%Throughout the paper, the linear order $\prec^s$ is fixed.
%}
%Given $a^s, b^s \in A^s$, let $\langle a^s, b^s\rangle \coloneqq \big\{x^s \in A^s: a^s \preccurlyeq^s x^s \preccurlyeq^s b^s,\; \textrm{or}\;\, b^s \preccurlyeq^s x^s \preccurlyeq^s a^s\big\}$ denote the \textbf{interval} that contains elements located between $a^s$ and $b^s$ on $\prec^s$, and
%let $\textrm{Int}\langle a^s, b^s\rangle \coloneqq \big\{z^s \in A^s: a^s \prec^s z^s \prec^s b^s,\; \textrm{or}\;\, b^s \prec^s z^s \prec^s a^s\big\}$
%denote the \textbf{interior} of the interval $\langle a^s, b^s\rangle$.
%Two elements $\underline{x}^s, \overline{x}^s \in A^s$ are called \textbf{marginal thresholds} on $\prec^s$
%if $\underline{x}^s \preccurlyeq^s \overline{x}^s$ and $[\underline{x}^s \neq \overline{x}^s]\Rightarrow \big[|\langle \underline{x}^s, \overline{x}^s\rangle| \geq 3\big]$.
%
%Given marginal thresholds $\underline{x}^s, \overline{x}^s \in A^s$,
%and elements $x^s, a^s, b^s \in A^s$ such that $a^s \neq b^s$,
%note that one of the following three cases must occur:
%(1) $x^s = a^s$, (2) $x^s \neq a^s$ and $\underline{x}^s = \overline{x}^s$, or
%(3) $x^s \neq a^s$ and $\underline{x}^s \neq \overline{x}^s$.
%We adopt the following terminology: according to $\prec^s$,
%in Case (1), $a^s$ is \textbf{trivially between} $x^s$ and $b^s$;
%in Case (2), $a^s$ is said \textbf{naturally between} $x^s$ and $b^s$ if $a^s \in \langle x^s, b^s\rangle$, while
%in Case (3), $a^s$ is \textbf{conditionally between} $x^s$ and $b^s$ if
%$a^s \in \langle x^s, b^s\rangle$ and $a^s \notin \textrm{Int}\langle \underline{x}^s, \overline{x}^s\rangle$.\footnote{
%Note that if $|\langle \underline{x}^s, \overline{x}^s\rangle| =2$, then $\textrm{Int}\langle \underline{x}^s, \overline{x}^s\rangle = \emptyset$. Consequently, natural betweeness and conditional betweeness become identical. This is why
%we intentionally rule out the case $\underline{x}^s \neq \overline{x}^s$ and $|\langle \underline{x}^s, \overline{x}^s\rangle| =2$ in the definition of marginal thresholds.}
%In summary, let $\overrightharpoon{(x^s, a^s, b^s)}$ denote a relation of
%trivial betweeness, or natural betweeness, or conditional betweeness, and
%$\mathcal{B}(\prec^s, \underline{x}^s, \overline{x}^s)$ denote the set collecting all such betweeness relations.
%
%%\begin{remark}\label{rem:tenaryrelation}\rm
%%Each betweenness relation in $\mathcal{B}(\prec^s, \underline{x}^s, \overline{x}^s)$ is \textbf{weakly symmetric}, i.e.,
%%given pairwise distinct $x^s, a^s, b^s \in A^s$,
%%$\left[\overrightharpoon{(x^s, a^s, b^s)} \in \mathcal{B}(\prec^s, \underline{x}^s, \overline{x}^s)\right] \Rightarrow \left[\overrightharpoon{(b^s, a^s, x^s)} \in \mathcal{B}(\prec^s, \underline{x}^s, \overline{x}^s)\right]$,
%%and \textbf{transitive}, i.e.,
%%$\left[\overrightharpoon{(x^s, a^s, b^s)}, \overrightharpoon{(x^s, b^s, c^s)} \in \mathcal{B}(\prec^s, \underline{x}^s, \overline{x}^s)\right] \Rightarrow \left[\overrightharpoon{(x^s, a^s, c^s)} \in \mathcal{B}(\prec^s, \underline{x}^s, \overline{x}^s)\right]$.
%%The detailed verification is put in the Supplementary Material.
%%\end{remark}
%
%\begin{definition}
%Fixing two marginal thresholds $\underline{x}^s, \overline{x}^s \in A^s$,
%a marginal preference $P_i^s$, say $r_1(P_i^s) = x^s$, is \textbf{hybrid} on $\prec^s$ w.r.t.~$\underline{x}^s$ and $\overline{x}^s$ if for all distinct $a^s, b^s \in A^s$, we have
%\begin{align*}
%\Big[\,\overrightharpoon{(x^s, a^s, b^s)} \in \mathcal{B}(\prec^s, \underline{x}^s, \overline{x}^s)\Big]
%\Rightarrow \big[a^s\mathrel{P_i^s}b^s\big].
%\end{align*}
%Let $\mathbb{D}_{\emph{H}}(\prec^s, \underline{x}^s, \overline{x}^s)$ denote \textbf{the hybrid marginal domain} that contains all hybrid marginal preferences on $\prec^s$ w.r.t.~$\underline{x}^s$ and $\overline{x}^s$.
%A marginal domain $\mathbb{D}^s$ is called \textbf{a} \textbf{hybrid marginal domain} on $\prec^s$ w.r.t.~$\underline{x}^s$ and $\overline{x}^s$ if $\mathbb{D}^s \subseteq \mathbb{D}_{\emph{H}}(\prec^s, \underline{x}^s, \overline{x}^s)$,
%$G_{\approx}^{A^s}$ is a connected graph, and $[\underline{x}^s \neq \overline{x}^s] \Rightarrow \big[G_{\approx}^{\langle \underline{x}^s, \,\overline{x}^s\rangle}\; \textrm{has no leaf}\,\big]$.
%\end{definition}
%
%
%Note that if the marginal thresholds $\underline{x}^s$ and $\overline{x}^s$ are identical,
%the conditional betweeness relation vanishes, and
%the trivial and natural betweeness relations can be merged, i.e.,
%$\mathcal{B}(\prec^s, \underline{x}^s, \overline{x}^s) = \left\{\overrightharpoon{(x^s, a^s, b^s)}: a^s \neq b^s\; \textrm{and}\; a^s \in \langle x^s, b^s\rangle\right\}$.
%Consequently, hybridness degenerates to the seminal preference restriction of single-peakedness and the hybrid marginal domain $\mathbb{D}_{\textrm{H}}(\prec^s, \underline{x}^s, \overline{x}^s)$ equals \textbf{the single-peaked marginal domain}.
%In the other extreme, if $\underline{x}^s = \min^{\prec^s}(A^s)$ and $\overline{x}^s = \max^{\prec^s}(A^s)$,
%then only the trivial betweeness relations survive, i.e., $\mathcal{B}(\prec^s, \underline{x}^s, \overline{x}^s) = \left\{\overrightharpoon{(x^s, a^s, b^s)}: a^s \neq b^s\; \textrm{and}\; x^s =a^s\right\}$.
%Consequently, no preference restriction is imposed, and $\mathbb{D}_{\textrm{H}}(\prec^s, \underline{x}^s, \overline{x}^s)$ expands to the universal marginal domain $\mathbb{P}^s$.
%By Theorem 2 of \citet{CRSSZ2022},
%all strategy-proof marginal rules on the hybrid marginal domain $\mathbb{D}_{\textrm{H}}(\prec^s, \underline{x}^s, \overline{x}^s)$ are characterized to be a subset of \emph{fixed ballot rules} introduced by \citet{M1980}.


%The seminal Gibbard-Satterthwaite Theorem \citep{G1973,S1975} implies that certain preference restriction has to be imposed on the preference domain so as to admit some strategy-proof rule other than dictatorships.
%Note that the Cartesian product structure embedded in the alternative set would be redundant, for instance it could be simply viewed as a relabeling of alternatives,
%in determining the existence of non-dictatorial, strategy-proof rules,
%if it is not involved in establishing preference restrictions.
%Here, we formally introduce three specific preference restrictions, \textit{top-separability} of \citet{LW1999}, \textit{separability} of \citet{BSZ1991} and \citet{LS1999},
%and \textit{multidimensional single-peakedness} of \citet{BJ1983} and \citet{BGS1993}, which are widely adopted in the literature of strategic voting and allow for the design of strategy-proof rules that are significantly more flexible than dictatorships \citep[see for instance, \emph{generalized dictatorships} introduced by][]{LS1999}.
%Then, we explain the relation of our semi-separable domains to these there domains.


\subsection{\rm Separable preferences and non-separable preferences}\label{sec:separable}


%A particular class of tops-only and strategy-proof rules is the class of dictatorships.
%Formally, an SCF $f: \mathbb{D}^{n} \rightarrow A$ is a \textbf{dictatorship} if there exists a ``dictator'' $i \in N$ such that $f(P) = r_1(P_i)$ for all $P \in \mathbb{D}^n$.
%Indeed, a dictatorship is strategy-proof on any arbitrary domain.



%Intuitively speaking, at each component, a marginal preference is fixed;
%separability then requires that between any two similar alternatives, the one endowed with a better element at the disagreed component be preferred.
Formally,
a preference $P_i$ is \textbf{separable} if there exists a marginal preference $P_i^s$ for each $s \in M$ such that
for each pair of similar alternatives $a, b \in A$,
say $M(a,b) = \{s\}$, we have
$\big[a^{s}\mathrel{P_{i}^{s}} b^{s}\big] \Rightarrow \big[a\mathrel{P_i}b\big]$.
Let $\mathbb{D}_{\textrm{S}}$ denote \textbf{the separable domain} that contains all separable preferences.
Clearly, $\mathbb{D}_{\textrm{S}} \subset \mathbb{P}$.
Henceforth, any preference that is not separable is called a non-separable preference, and a domain is said to satisfy \textbf{diversity\textsuperscript{$+$}} if it contains two
separable preferences that are complete reversals.\footnote{\citet{CRSSZ2022} say that a domain satisfies \emph{diversity} if it contains two complete reversals. The term ``diversity\textsuperscript{$+$}'' emphasizes that the complete reversals are further required to be separable preferences.}

More importantly, we introduce a particular way of deriving marginal preferences from both separable and non-separable preferences.
Given a preference $P_i$ (separable or non-separable), say $r_1(P_i) = a$,
for each $s \in M$, according to $a^{-s}$ which are contained in the peak of $P_i$,
we induce a marginal preference over $A^s$, denoted $[P_i]^s$, such that
for all $x^s, y^s \in A^s$, $\big[x^s\mathrel{[P_i]^s}y^s\big] \Leftrightarrow \big[(x^s, a^{-s})\mathrel{P_i}(y^s, a^{-s})\big]$.
Accordingly, let $[\mathbb{D}]^s \coloneqq \big\{[P_i]^s: P_i \in \mathbb{D}\big\}$ denote the set of marginal preferences over $A^s$ induced from all preferences of a domain $\mathbb{D}$.
To avoid confusion with  a marginal preference $P_i^s$ and a marginal domain $\mathbb{D}^s$,
 we henceforth call $[P_i]^s$ an \textbf{induced marginal preference} and
$[\mathbb{D}]^s$ an \textbf{induced marginal domain}.\footnote{Indeed, both $P_i^s$ and $[P_i]^s$ refer to linear orders over $A^s$.
For the sake of notation, $[P_i]^s$ emphasizes that it is induced from a given preference $P_i$.
Similarly, $\mathbb{D}^s$ represents an arbitrary marginal domain, while
$[\mathbb{D}]^s$ emphasizes that it is induced from a given domain $\mathbb{D}$.}


\subsection{\rm Decomposable SCF and decomposable domain}
%Once a domain of separable preferences is fixed,
%marginal preferences and marginal domains are immediately derived, and then marginal SCFs emerge naturally.
%Then, one can investigate the relation between SCFs and marginal SCFs.
%More specifically, in one direction,
%one would ask whether it is able to \emph{decompose} a strategy-proof SCF into several strategy-proof marginal SCFs, while in the inverse direction, due to the preference restriction of separability, an SCF constructed via \emph{assembling} strategy-proof marginal SCFs at all components, is automatically endowed with strategy-proofness.
%%We throughout the paper investigate domains that also contain non-separable preferences.
%%This consequently invalidates the same investigations on domains of separable preferences.
%%However, from the perspective of decoding collective choice decision and the perspective of simplifying the design of mechanisms, even provided the presence of non-separable preferences,
%%we still insist to investigate the issues of decomposing strategy-proof SCFs and assembling strategy-proof marginal strategy-proof SCFs.
%%To put this analysis in perspective,
%Our objective in this paper is to extend the decomposition property and more importantly the assembling property to preference domains that contain non-separable preferences. To this end we propose a new approach to the problem of eliciting marginal preferences from a possibly non-separable preference.




An SCF $f: \mathbb{D}^n \rightarrow A$ is said \textbf{decomposable} if for each $s \in M$,
there exists a marginal SCF $f^s: \big[[\mathbb{D}]^s\big]^n \rightarrow A^s$  such that
for all $(P_1, \dots, P_n) \in \mathbb{D}^n$, we have
\begin{align*}
\big[f(P_1, \dots, P_n) = a\big] \Leftrightarrow \big[f^s([P_1]^s, \dots, [P_n]^s) = a^s\; \textrm{for all}\;\, s \in M\big].
\end{align*}
%Conversely, given a domain $\mathbb{D}$ and a marginal SCF $f^s: \big[[\mathbb{D}]^s\big]^n \rightarrow A^s$ for each $s \in M$,
%let $f \coloneqq (f^1, \dots, f^m)$ denote an SCF on $\mathbb{D}$ assembled by all marginal SCFs.
%We call $f$ the \textbf{assembled SCF}.

In order to investigate the relation between strategy-proof SCFs and marginal SCFs,
we focus on preference domains that reconcile decomposability with strategy-proofness.

\begin{definition}\label{def:decomposabledomain}
A domain $\mathbb{D}$ is a \textbf{decomposable domain} if for every SCF $f: \mathbb{D}^n \rightarrow A$,
$n \geq 2$, we have
\begin{align*}
\Big[f \; \textrm{is a strategy-proof rule}\Big]
\Leftrightarrow
\left[
\begin{array}{l}
\!\!f\; \textrm{is decomposable, and}\\
\!\!\textrm{all marginal SCFs}\; f^1, \dots, f^m\; \textrm{are strategy-proof marginal rules}
\end{array}
\!\!\right].
\end{align*}
\end{definition}

%\begin{definition}
%A domain $\mathbb{D}$ is a \textbf{decomposable domain} if the following two properties are satisfied: for all $n \geq 2$,
%\begin{description}
%\item[\rm (\textbf{The decomposition property})] each strategy-proof rule $f: \mathbb{D}^n \rightarrow A$ is decomposable, and all marginal SCFs are strategy-proof rules, and
%
%\item[\rm (\textbf{The assembling property})] given an arbitrary strategy-proof marginal rule
%$f^s: \big[[\mathbb{D}]^s\big]^n \rightarrow A^s$ for each $s \in M$, the assembled SCF $f = (f^1, \dots, f^m)$ is a strategy-proof rule.
%\end{description}
%\end{definition}


On the one hand,
since dictatorships are strategy-proof marginal rules on arbitrary induced marginal domains,
by the requirement of the direction ``$\Leftarrow$'' in Definition \ref{def:decomposabledomain},
all generalized dictatorships (see footnote \ref{footnote:generalizeddictatorship}) are entitled with strategy-proofness,
which implies that a decomposable domain must be embedded with some preference restriction.
For instance, if all preferences of the domain are separable,
an SCF constructed by assembling strategy-proof marginal rules immediately turns out to be a strategy-proof rule.
On the other hand, to meet the requirement of the direction ``$\Rightarrow$'' in Definition \ref{def:decomposabledomain}, a decomposable domain is required to contain sufficiently many preferences.
For instance, on the universal domain $\mathbb{P}$, by the Gibbard-Satterthwaite Impossibility Theorem \citep{G1973,S1975},
each strategy-proof rule is a dictatorship, and hence can be decomposed into $m$ marginal dictatorships that share the same dictator.
Therefore, a decomposable domain must be a restricted preference domain that satisfies some richness condition, e.g., domains of separable preferences that satisfy the richness condition of  \citet{LS1999}.


\section{Results}\label{sec:result}
In this section, we introduce multidimensional hybrid domains, and adopt it to establish a complete characterization of decomposable domains under some mild richness condition.

\subsection{\rm Multidimensional hybrid preferences}\label{sec:MHDomain}


Fixing a linear order $\prec^s$ over $A^s$ for each component $s \in M$,
let $\prec \,\coloneqq \times_{s \in M}\prec^s$ denote the Cartesian product of $\prec^1, \dots, \prec^m$.
Given $s \in M$ and $a^s, b^s \in A^s$, let $\langle a^s, b^s\rangle \coloneqq \{x^s \in A^s: a^s \preccurlyeq^s x^s \preccurlyeq^s b^s,\; \textrm{or}\; b^s \preccurlyeq^s x^s \preccurlyeq^s a^s\}$ denote the set of elements that located between $a^s$ and $b^s$ on the linear order $\prec^s$, and
let $\textrm{Int}\langle a^s, b^s\rangle \coloneqq \{x^s \in A^s: a^s \prec^s x^s \prec^s b^s\; \textrm{or}\; b^s \prec^s x^s \prec^s a^s\}$ denote the set of elements that are located strictly between $a^s$ and $b^s$.
Given $s \in M$, two elements $\underline{x}^s$ and $\overline{x}^s$ are called \textbf{marginal thresholds} on $\prec^s$ if either $\underline{x}^s = \overline{x}^s$,
or $\underline{x}^s \neq \overline{x}^s$ and $|\langle \underline{x}^s, \overline{x}^s\rangle|\geq 3$.
Correspondingly, two alternatives $\underline{x}$ and $\overline{x}$
are called \textbf{thresholds} on $\prec$ if
for each $s \in M$, $\underline{x}^s$ and $\overline{x}^s$ are marginal thresholds on $\prec^s$.

\begin{definition}\label{def:MH}
A preference $P_i$, say $r_1(P_i) = x$, is \textbf{multidimensional hybrid} on $\prec$ w.r.t.~thresholds $\underline{x}$ and $\overline{x}$ if for all similar $a, b \in A$, say $M(a,b) = \{s\}$, we have
\begin{itemize}
\item[\rm (i)] $\big[a^s = x^s\big]
\Rightarrow \big[a\mathrel{P_i}b\big]$, and

\item[\rm (ii)] $\big[a^s \in \emph{Int}\langle x^s, b^s\rangle\; \textrm{and}\;\, a^s \notin \emph{Int}\langle \underline{x}^s, \overline{x}^s\rangle\big]
\Rightarrow \big[a\mathrel{P_i}b\big]$.\footnote{By transitivity, given distinct $a,b \in A$ (not necessarily similar alternatives),
if $a^s = x^s$, or $a^s \in \textrm{Int}\langle x^s, b^s\rangle\; \textrm{and}\; a^s \notin \textrm{Int}\langle \underline{x}^s, \overline{x}^s\rangle$ hold for all $s \in M(a, b)$, we have $a\mathrel{P_i}b$.\label{footnote}}
\end{itemize}
Let $\mathbb{D}_{\emph{MH}}(\prec, \underline{x}, \overline{x})$ denote \textbf{the multidimensional hybrid domain} that contains all multidimensional hybrid preferences on $\prec$ w.r.t.~$\underline{x}$ and $\overline{x}$.
\end{definition}


We provide one example to explain multidimensional hybrid preferences.

\begin{figure}[t]
\begin{center}
  \includegraphics[width=0.5\textwidth]{fig2.jpg}
\end{center}
\vspace{-1.5em}
\caption{The Cartesian product of linear orders $\prec= \prec^1 \times \prec^2$}\label{fig:grid}
\end{figure}

\begin{example}\label{exm:MH}\rm
Let $A = A^1 \times A^2$ where $A^1 = \{1,2,3,4\}$ and $A^2 = \{0,1\}$.
Let $\prec^1$ and $\prec^2$ be the natural linear orders over $A^1$ and $A^2$ respectively and
$\prec = \prec^1 \times \prec^2$ (see Figure \ref{fig:grid}).
We fix two thresholds $\underline{x} = (2,0)$ and $\overline{x} = (4,0)$.
For instance, we specify the restriction of multidimensional hybridness on a preference with the peak $(1,0)$.
First, to meet condition (i) of Definition \ref{def:MH},
it must be the case that $(1,1)$ is ranked above $(2,1), (3,1)$ and $(4,1)$, and
symmetrically, $(k, 0)$ is ranked above $(k, 1)$ for all $k \in \{2, 3, 4\}$.
Second, notice that on the linear ordering $\prec^1$, $2 \in \textrm{Int}\langle 1, 3\rangle$, $2 \in \textrm{Int}\langle 1, 4\rangle$, and $2 \notin
\textrm{Int}\langle \underline{x}^1, \overline{x}^1\rangle = \textrm{Int}\langle 2, 4\rangle$.
Then, to meet condition (ii) of Definition \ref{def:MH},
we need to ensure that $(2,0)$ is ranked above both $(3,0)$ and $(4,0)$, and symmetrically
$(2,1)$ is ranked above both $(3,1)$ and $(4,1)$.
It is worth mentioning that multidimensional hybridness here provides a room for the arbitrary relative rankings between $(3,0)$ and $(4,0)$, and also between $(3,1)$ and $(4,1)$.
Specifically, the two preferences below are multidimensional hybrid on $\prec$ w.r.t.~$\underline{x}$ and $\overline{x}$:
\begin{align*}
P_i = &~ (1,0)_{\rightharpoonup}(2,0)_{\rightharpoonup}(3,0)_{\rightharpoonup}(4,0)_{\rightharpoonup}(1,1)_{\rightharpoonup}(2,1)_{\rightharpoonup}(3,1)_{\rightharpoonup}(4,1),\; \textrm{and}\\
P_i' = &~
(1,0)_{\rightharpoonup}(1,1)_{\rightharpoonup}(2,0)_{\rightharpoonup}(2,1)_{\rightharpoonup}(3,0)_{\rightharpoonup}(4,0)_{\rightharpoonup}(4,1)_{\rightharpoonup}(3,1).\protect \footnotemark
\end{align*}
Note that $P_i$ is a separable preference, and $\hat{P}_i$ is a non-separable preference.
\footnotetext{To save space, we specify the two preferences here horizontally. For instance,
the notation ``$(1,0)_{\rightharpoonup}(2,0)$'' in the specification of the preference $P_i$ represents ``$(1,0)\mathrel{P_i}(2,0)$''.}
\hfill$\Box$
\end{example}

\begin{remark}\label{rem:relation1}\rm
We consider the multidimensional hybrid domain $\mathbb{D}_{\textrm{MH}}(\prec, \underline{x}, \overline{x})$ in two extreme cases:
(1) $\underline{x}^s = \min^{\prec^s}(A^s)$ and $\overline{x}^s = \max^{\prec^s}(A^s)$ for each $s \in M$, and
(2) $\underline{x} = \overline{x}$.
In the first case, condition (ii) of Definition \ref{def:MH} becomes redundant\footnote{The two conditions $a^s \in \textrm{Int}\langle x^s, b^s\rangle$ and $a^s \notin \textrm{Int}\langle \underline{x}^s, \overline{x}^s\rangle$ cannot hold simultaneously in the first case.}, and
only condition (i) remains, which equals the preference restriction of top-separability introduced by \citet{LW1999}.
Correspondingly, the multidimensional hybrid domain expands to \textbf{the top-separable domain} $\mathbb{D}_{\textrm{TS}}$.
In the second case,
since the hypothesis $a^s \notin \textrm{Int}\langle \underline{x}^s, \overline{x}^s\rangle = \emptyset$ is vacuously satisfied,
the two conditions of Definition \ref{def:MH} can be merged:
$\big[a^s \in \langle x^s, b^s\rangle\big] \Rightarrow [a\mathrel{P_i}b]$,
which equals the preference restriction of multidimensional single-peakedness introduced by \citet{BGS1993}.
Correspondingly,
the multidimensional hybrid domain shrinks to \textbf{the multidimensional single-peaked domain} $\mathbb{D}_{\textrm{MSP}}(\prec)$.
For any thresholds $\underline{x}, \overline{x} \in A$ other than the two extreme cases,
we have $\mathbb{D}_{\textrm{TS}} \supset \mathbb{D}_{\textrm{MH}}(\prec,\underline{x}, \overline{x}) \supset \mathbb{D}_{\textrm{MSP}}(\prec)$.
\end{remark}

\subsection{Multidimensional hybrid domains and fixed ballot rules}

We notice that $\mathbb{D}_{\textrm{S}} \neq \mathbb{D}_{\textrm{MH}}(\prec, \underline{x}, \overline{x})$ for any thresholds $\underline{x}, \overline{x} \in A$ as $\mathbb{D}_{\textrm{MH}}(\prec, \underline{x}, \overline{x})$ admits non-separable preferences, and
$\mathbb{D}_{\textrm{S}}$ is strictly included in the top-separable domain $\mathbb{D}_{\textrm{TS}}$
which equals the multidimensional hybrid domain $\mathbb{D}_{\textrm{MH}}(\prec, \underline{x}, \overline{x})$ in the extreme case that $\underline{x}^s = \min^{\prec^s}(A^s)$ and $\overline{x}^s = \max^{\prec^s}(A^s)$ for each $s \in M$.
In order to include the separable domain $\mathbb{D}_{\textrm{S}}$ into our consideration,
we introduce a particular family of subdomains of the multidimensional domain.

We first introduce some standard concepts from graph theory.
An (undirected) \textbf{graph}, denoted $G \coloneqq \langle V, \mathcal{E}\rangle$, is a combination of a ``vertex set'' $V$ and an ``edge set'' $\mathcal{E} \subseteq V\times V$ such that
$\big[(\alpha, \beta) \in \mathcal{E}\big] \Rightarrow \big[\alpha \neq \beta\; \textrm{and}\;  (\beta, \alpha) \in \mathcal{E}\big]$.
A vertex $\alpha \in V$ is called a \textbf{leaf} if there exists a unique $\beta \in V$ such that $(\alpha, \beta) \in \mathcal{E}$.
Given distinct vertices $\alpha, \beta \in V$, a \textbf{path} in $G =\langle V, \mathcal{E}\rangle $ connecting $\alpha$ and $\beta$ is a sequence of ``non-repeated'' vertices $(\alpha_1, \dots, \alpha_v)$, $v \geq 2$, such that
$\alpha_1 = \alpha$, $\alpha_v = \beta$, and
$(\alpha_k, \alpha_{k+1}) \in \mathcal{E}$ for all $k = 1, \dots, v-1$.
A graph $G = \langle V, \mathcal{E}\rangle$ is called a \textbf{connected graph} if
for each pair of distinct vertices, there exists a path in $G$ connecting them.
%In particular, a graph $G = \langle V, \mathcal{E}\rangle$ is called a \textbf{complete graph} if any two distinct vertices form an edge.
Note that the vertex set $V$ here can be
a subset of elements in one component set,
a subset of alternatives,
a subset of marginal preferences, or
a subset of preferences.
For instance, following Definition 1 of \citet{CSS2013},
given a domain $\mathbb{D}$ and a component $s\in M$, two elements $a^s, b^s \in A^s$ are said \textbf{strongly connected},
denoted $a^s \approx b^s$,
if there exist $[P_i]^s, [P_i']^{s} \in [\mathbb{D}]^s$ such that
$r_1([P_i]^s) = r_2([P_i']^s) = a^s$, $r_1([P_i']^s) = r_2([P_i]^s) = b^s$ and
$r_k([P_i]^s) = r_k([P_i']^s)$ for all $k = 3, \dots, |A^s|$.
Accordingly, given a nonempty subset $B^s \subseteq A^s$,
we can induce a graph $G_{\approx}^{B^s} \coloneqq \langle B^s, \mathcal{E}_{\approx}^{B^s}\rangle$
where two elements of $B^s$ form an edge if and only if they are strongly connected.


\begin{definition}\label{def:AMH}
A domain $\mathbb{D}$ is called \textbf{a multidimensional hybrid domain} if
there exist thresholds $\underline{x}, \overline{x} \in A$ such that
\begin{itemize}
\item[\rm (i)] $\mathbb{D} \subseteq \mathbb{D}_{\emph{MH}}(\prec, \underline{x}, \overline{x})$, and

\item[\rm (ii)] for each $s \in M$,  $G_{\approx}^{A^s}$ is
a connected graph, and
$\big[s \in M(\underline{x}, \overline{x})\big] \Rightarrow \big[G_{\approx}^{\langle \underline{x}^s, \, \overline{x}^s\rangle}\; \textrm{has no leaf}\,\big]$.
\end{itemize}
\end{definition}


Of course, the multidimensional hybrid domain $\mathbb{D}_{\textrm{MH}}(\prec, \underline{x}, \overline{x})$ satisfies the two conditions of Definition \ref{def:AMH}.
It is easy to verify that the separable domain $\mathbb{D}_{\textrm{S}}$, where $|A^s| \geq 3$ for all $s\in M$,
is a multidimensional hybrid domain:
fixing thresholds $\underline{x}$ and $\overline{x}$ such that
$\underline{x}^s = \min^{\prec^s}(A^s)$ and $\overline{x}^s = \max^{\prec^s}(A^s)$ for each $s \in M$,
we have (i) $\mathbb{D}_{\textrm{S}} \subset \mathbb{D}_{\textrm{MH}}(\prec, \underline{x}, \overline{x})$, and
(ii) for each $s \in M$, since $[\mathbb{D}]^s = \mathbb{P}^s$,
in the graph $G_{\approx}^{A^s} = G_{\approx}^{\langle \underline{x}^s,\, \overline{x}^s\rangle}$,
any two distinct vertices form an edge.

Fixing a multidimensional hybrid domain $\mathbb{D}$ on $\prec$ w.r.t.~two thresholds $\underline{x}$ and $\overline{x}$,
given a component $s\in M$,
by the first condition of Definition \ref{def:AMH},
note that each induced marginal preference  satisfies the preference restriction of hybridness on $\prec^s$ w.r.t.~$\underline{x}^s$ and $\overline{x}^s$ (recall condition (\ref{def:h}) in Section \ref{sec:heuristicexample}).
It is clear that $[\mathbb{D}_{\textrm{MH}}(\prec, \underline{x}, \overline{x})]^s$ contains all marginal preferences that are hybrid on $\prec^s$ w.r.t.~$\underline{x}^s$ and $\overline{x}^s$.
It is worth mentioning that if $[\mathbb{D}]^s$ contains some preference restriction beyond the hybridness, e.g., $a^s \in \textrm{Int}\langle x^s, b^s\rangle$, $a^s \in \textrm{Int}\langle \underline{x}^s, \overline{x}^s\rangle$ and $\overrightharpoon{(x^s, a^s, b^s)} \in \mathcal{R}([\mathbb{D}]^s)$,
then $\mathbb{D}$ may include a multidimensional hybrid preference $P_i$ such that $(x^s, x^{-s})$ is top-ranked and $(b^s, z^{-s})\mathrel{P_i}(a^s, z^{-s})$, which indicates a violation of semi-separability (recall condition (\ref{def:SS}) in Section \ref{sec:heuristicexample}).
More importantly, under the second condition of Definition \ref{def:AMH},
sufficiently many hybrid marginal preferences are contained in each induced marginal domain so that
we can provide a complete characterization of strategy-proof marginal rules,
using the class of fixed ballot rules introduced by \citet{M1980},
which plays an important role in our investigation of decomposable domains in the next section.
Note that under both conditions of Definition \ref{def:AMH}, for each component $s \in M$,
if $\underline{x}^s = \overline{x}^s$, the graph $G_{\approx}^{A^s}$ is a line over $A^s$, while if $\underline{x}^s \neq \overline{x}^s$,
$G_{\approx}^{A^s}$ is simply a combination of an interval between $\min^{\prec^s}(A^s)$ and $\underline{x}^s$, a connected subgraph $G_{\approx}^{\langle \underline{x}^s, \, \overline{x}^s\rangle}$ that has no leaf and an interval between $\overline{x}^s$ and $\max^{\prec^s}(A^s)$.

According to $\prec^s$,
a marginal SCF $f^s: \big[[\mathbb{D}]^s\big]^n \rightarrow A^s$ is a \textbf{Fixed Ballot Rule} (or \textbf{FBR}) if there exists $b_J^s \in A^s$, called a ``fixed ballot'', for each coalition $J \subseteq N$, satisfying ``ballot unanimity'', i.e., $b_{\emptyset}^s = \min^{\prec^s}(A^s)$ and $b_{N}^s = \max^{\prec^s}(A^s)$, and ``monotonicity'', i.e., $[J \subset J' \subseteq N] \Rightarrow [b_J^s \preccurlyeq^s b_{J'}^s]$,
such that for all $([P_1]^s, \dots, [P_n]^s) \in [\mathbb{D}^s]^n$, we have
\begin{align*}
f^s([P_1]^s, \dots, [P_n]^s)  = \mathop{\mathop{\max}\nolimits^{\prec^s}}\limits_{J \subseteq N~~~\,}
\Big(\mathop{\mathop{\min}\nolimits^{\prec^s}}\limits_{i \in J~~~}\big(r_1([P_i]^s), b_J^s\big)\Big).
\end{align*}
Furthermore, given $\underline{x}^s, \overline{x}^s \in A^s$ with $\underline{x}^s \prec^s \overline{x}^s$,
the FBR $f^s$ is called an \textbf{$\bm{(\underline{x}^s, \overline{x}^s)}$-FBR} introduced by \citet{CRSSZ2022}, if it in addition satisfies ``the constrained-dictatorship condition'', i.e., there exists $i \in N$ such that $[i \in J] \Rightarrow \big[\overline{x}^s \preccurlyeq^s b_J^s \big]$ and $[i \notin J] \Rightarrow \big[b_J^s \preccurlyeq^s \underline{x}^s\big]$.\footnote{The constrained-dictatorship condition ensures that the FBR $f^s: \big[[\mathbb{D}]^s\big]^n \rightarrow A^s$ \emph{behaves like a dictatorship} on the interval $\langle \underline{x}^s, \overline{x}^s\rangle$, i.e., there exists $i \in N$ such that for all $[P_1]^s, \dots, [P_n]^s \in [\mathbb{D}]^s$, $[r_1([P_1]^s), \dots, r_1([P_n]^s) \in B^s] \Rightarrow \big[f^s([P_1]^s, \dots, [P_n]^s) = r_1([P_i]^s)\big]$.\label{footnote:behavelikeadictatorship}}
The proposition below characterizes that on the induced marginal domain $[\mathbb{D}]^s$,
given $\underline{x}^s = \overline{x}^s$,
a marginal SCF is a strategy-proof marginal rule if and only if it is an FBR,
while given $\underline{x}^s \neq \overline{x}^s$,
a marginal SCF is a strategy-proof marginal rule if and only if it is an $(\underline{x}^s,\overline{x})$-FBR.
The proof of the proposition is contained in Appendix \ref{app:FBR}.

\begin{proposition}\label{prop:FBR}
Fix a multidimensional hybrid domain $\mathbb{D}$ on $\prec$ w.r.t.~two thresholds $\underline{x}$ and $\overline{x}$.
Given a component $s\in M$, the following two statements hold:
\begin{itemize}
\item[\rm (i)] Let $\underline{x}^s = \overline{x}^s$.
An marginal SCF $f^s : \big[[\mathbb{D}]^s\big]^n \rightarrow A^s$ is a strategy-proof marginal rule if and only if it is an FBR.

\item[\rm (ii)] Let $\underline{x}^s \neq \overline{x}^s$.
An marginal SCF $f^s : \big[[\mathbb{D}]^s\big]^n \rightarrow A^s$ is a strategy-proof marginal rule if and only if it is an $(\underline{x}^s, \overline{x}^s)$-FBR.
\end{itemize}
\end{proposition}



\subsection{The Theorem}\label{sec:theorem}


In this section, we provide a complete characterization of decomposable domains under a mild richness condition.
We first introduce some necessary notions and notation for establishing the richness condition.
Two preferences $P_i$ and $P_i'$ over $A$ are \textbf{adjacent}, denoted $P_i \sim P_i'$,
if there exist distinct $a, b \in A$ such that $r_k(P_i) = r_{k+1}(P_i') = a$ and $r_k(P_i') = r_{k+1}(P_i) = b$ for some $1\leq k < |A|$, and $r_{\ell}(P_i) = r_{\ell}(P_i')$ for all $\ell \notin \{k,k+1\}$.
As a natural extension of the notion of adjacency, two preferences $P_i$ and $P_i'$ are said \textbf{adjacent\textsuperscript{$+$}}, denoted $P_i \sim^{+} P_i'$, if they are separable preferences, and there exist $s \in M$ and distinct $a^s, b^s \in A^s$ such that
the following two conditions are satisfied:
\begin{itemize}
\item[\rm (i)] given arbitrary $z^{-s}\in A^{-s}$,
$(a^s, z^{-s}) = r_k(P_i) = r_{k+1}(P_i') $ and $(b^s, z^{-s})=r_k(P_i') = r_{k+1}(P_i) $ for some $1\leq k < |A|$, and
\item[\rm (ii)] given arbitrary $c \in A$, $\big[c^s \notin \{a^s, b^s\}\big] \Rightarrow
\big[c = r_{\ell}(P_i) = r_{\ell}(P_i')\;\textrm{for some} \; 1 \leq \ell \leq |A|\big]$.
\end{itemize}
Given a domain $\mathbb{D}$,
we construct a graph $G_{\sim/\sim^+}^{\mathbb{D}}\coloneqq \langle \mathbb{D}, \mathcal{E}_{\sim/\sim^+}\rangle$
such that two preferences form an edge if and only if they are adjacent or adjacent\textsuperscript{+}.
Fixing a path $\pi = (P_{i|1}, \dots, P_{i|v})$ in $G_{\sim/\sim^+}^{\mathbb{D}}$, given distinct $a, b \in A$,
the path $\pi$ has \textbf{$\bm{\{a,b\}}$-restoration} if
the relative ranking of $a$ and $b$ has been flipped for more than once, i.e.,
there exist $1 \leq o< p < q \leq v$ such that
either $a\mathrel{P_{i|o}}b$, $b\mathrel{P_{i|p}}a$ and $a\mathrel{P_{i|q}}b$,
or $b\mathrel{P_{i|o}}a$, $a\mathrel{P_{i|p}}b$ and $b\mathrel{P_{i|q}}a$ hold.
\citet{S2013} restricted attention to the graph $G_{\sim}^{\mathbb{D}}$,
and introduced \textbf{the no-restoration condition}:
given distinct $P_i, P_i' \in \mathbb{D}$ and distinct $a,b \in A$,
there exists a path in $G_{\sim}^{\mathbb{D}}$ connecting $P_i$ and $P_i'$ that has no $\{a,b\}$-restoration.\footnote{Proposition 3.2 of \citet{S2013} has shown that the no-restoration condition is necessary for the equivalence between strategy-proofness and the notion of \emph{local strategy-proofness} which only prevents voter's manipulation via misreporting preferences adjacent to the sincere one.
\citet{KRSYZ2021a} introduce \textbf{Property $L$} - a strengthening of the no-restoration condition:
fixing a general graph $G \coloneqq \langle \mathbb{D}, \mathcal{E}\rangle$: given distinct $P_i, P_i' \in \mathbb{D}$ and $a \in A$,
there exists a path in $G$ connecting $P_i$ and $P_i'$ that has no $\{a,b\}$-restoration for any $b \in L(a, P_i)\coloneqq \{x \in A: a\mathrel{P_i} x\}$, and characterize that Property $L$ is necessary and sufficient for achieving the local-global equivalence for all SCFs.\label{footnote:PropertyL}}
Intuitively speaking, the no-restoration condition can be viewed as an ordinal counterpart of the convex-set assumption imposed on the valuation space in a cardinal model,
which reconciles the difference of two preferences via a sufficiently short path in the graph $G_{\sim}^{\mathbb{D}}$.
The following two properties adopted from \citet{CZ2019} extend Sato's no-restoration condition to the graph $G_{\sim/\sim^+}^{\mathbb{D}}$ which involves not only the edge of adjacency, but the edge of adjacency\textsuperscript{+} that is customized for separable preferences, and impose some additional requirements on some subgraphs of $G_{\sim/\sim^+}^{\mathbb{D}}$.

\begin{definition}
A domain $\mathbb{D}$ satisfies \textbf{the Interior\textsuperscript{$\bm{+}$} property}
if given distinct $P_i, P_i' \in \mathbb{D}$ such that
$r_1(P_i) = r_1(P_i')\coloneqq x$,
there exists a path $\pi = (P_{i|1}, \dots, P_{i|v})$ in $G_{\sim/\sim^+}^{\mathbb{D}}$ connecting $P_i$ and $P_i'$ such that $r_1(P_{i|k}) = x$ for all $k = 1, \dots, v$.
\end{definition}

By the Interior\textsuperscript{+} property, we know that for each $x \in A$, the subgraph $G_{\sim/\sim^+}^{\mathbb{D}^x}$ is a connected graph,
which implies that any two distinct preferences $P_i, P_i' \in \mathbb{D}$ that share the same peak, say $r_1(P_i) = r_1(P_i')\coloneqq x$, are connected by a path in $G_{\sim/\sim^+}^{\mathbb{D}}$ that has no $\{x,a\}$-restoration for any $a \in A\backslash \{x\}$.

\begin{definition}
A domain $\mathbb{D}$ satisfies \textbf{the Exterior\textsuperscript{$\bm{+}$} property}
if given $P_i, P_i' \in \mathbb{D}$ such that $r_1(P_i) \neq r_1(P_i')$ and distinct $a,b \in A$,
there exists a path $\pi=(P_{i|1}, \dots, P_{i|v})$ in $G_{\sim/\sim^+}^{\mathbb{D}}$ connecting $P_i$ and $P_i'$ such that $\pi$ has no $\{a,b\}$-restoration\footnote{The Exterior\textsuperscript{+} property of \citet{CZ2019} is slightly weaker, as it only requires no $\{a,b\}$-restoration on a path connecting $P_i$ and $P_i'$ when $a$ and $b$ are identically ranked in both $P_i$ and $P_i'$.} and in addition satisfies \textbf{the no-detour condition}: ~~\,$\big[r_1(P_i), r_1(P_i'), a, b \in (A^s, x^{-s})\;\,\textrm{for some}\; \,s \in M\;\textrm{and}\;\, x^{-s} \in A^{-s}\big]$\\
$\Rightarrow [r_1(P_{i|k}) \in (A^s, x^{-s})\; \textrm{for all}\; k = 1, \dots, v]$.
\end{definition}

The Exterior\textsuperscript{+} property not only ensures the difference of two preferences that differ in peaks, to be reconciled via a sufficiently short path in $G_{\sim/\sim^+}^{\mathbb{D}}$, but requires via the no-detour condition that in a subdomain of preferences that have similar peaks, say $\mathbb{D}^{(A^s,\, x^{-s})} \coloneqq \{P_i \in \mathbb{D}: r_1(P_i) \in (A^s, x^{-s})\}$,
for each pair $a,b \in (A^s, x^{-s})$,
any two preferences differing in peaks are connected by a path in the subgraph $G_{\sim/\sim^+}^{\mathbb{D}^{(A^s,\, x^{-s})}}$ that has no $\{a,b\}$-restoration.
This in conjunction with the Interior\textsuperscript{+} property implies that
$G_{\sim/\sim^+}^{\mathbb{D}^{(A^s,\, x^{-s})}}$ is a connected graph.

Henceforth, a domain $\mathbb{D}$ is called a \textbf{rich domain}
if it satisfies minimal richness, diversity\textsuperscript{$+$} and the Interior\textsuperscript{+} and Exterior\textsuperscript{+} properties.
It is evident that the universal domain $\mathbb{P}$ is a rich domain.
We provide an example of a rich multidimensional hybrid domain in Appendix \ref{app:anexample} and
explain how to verify the Interior\textsuperscript{+} and Exterior\textsuperscript{+} properties,
while in Clarifications \ref{cla:S} and \ref{cla:MH} of Appendix \ref{app:clarification},
we respectively show that
the separable domain $\mathbb{D}_{\textrm{S}}$ and the multidimensional hybrid domain $\mathbb{D}_{\textrm{MH}}(\prec, \underline{x}, \overline{x})$ are rich domains.

The main theorem below shows that under the richness condition,
multidimensional hybridness is necessary and sufficient for a domain to be a decomposable domain.

\begin{theorem}\label{thm}
Let $\mathbb{D}$ be a rich domain.
Then, $\mathbb{D}$ is a decomposable domain if and only if it is a multidimensional hybrid domain.
\end{theorem}

The proof of the Theorem is contained in Appendix \ref{app:theorem}.
Here, we provide a comprehensive sketch of the proof.
The proof of the Theorem consists of three parts.
\begin{description}
\item[\bf Part 1.]
We explore the rich domain $\mathbb{D}$, and establish three important results that will be applied for the proof of both the sufficiency and necessity parts of the Theorem.
First, we adopt Proposition 2 of \citet{CZ2019} to show that every strategy-proof rule on $\mathbb{D}$ satisfies the tops-only property (see Lemma \ref{lem:tops-onlydomain}).
Next, according to $G_{\sim/\sim^+}^{\mathbb{D}}$, we induce a connected graph on the alternatives of $(A^s, x^{-s})$ for each $s \in M$ and $x^{-s} \in A^{-s}$, and reveal an important implication of the Exterior\textsuperscript{+} property according to the graph on $(A^s, x^{-s})$ (see Lemma \ref{lem:path-connectedness}).
Last, by referring to the graph on $(A^s, x^{-s})$,
we partially characterize strategy-proof rules on $\mathbb{D}$ (see Lemmas \ref{lem:step1} and \ref{lem:step2}).

\item[\bf Part 2.]
We prove the sufficiency part of the Theorem: a rich multidimensional hybrid domain $\mathbb{D}$ is a decomposable domain.
First,
by Lemmas \ref{lem:tops-onlydomain}, \ref{lem:path-connectedness} and \ref{lem:step2},
we show that every strategy-proof rule on $\mathbb{D}$ is decomposable, and all marginal SCFs are strategy-proof marginal rules (see Lemma \ref{lem:decomposition}).
Conversely, by applying Proposition \ref{prop:FBR},
we show that if an SCF on $\mathbb{D}$ is assembled by strategy-proof marginal rules, it is a strategy-proof rule (see Lemma \ref{lem:assembling}).

\item[\bf Part 3.]
We verify the necessity part of the Theorem:
if a rich domain $\mathbb{D}$ is a decomposable domain, it is a multidimensional hybrid domain.
First, we explore the induced marginal domain $[\mathbb{D}]^s$ and graph $G_{\approx}^{A^s}$ for each $s \in M$.
In Lemma \ref{lem:connectedgraph}, we show that $G_{\approx}^{A^s}$ is a connected graph, and reveal a restriction, as an implication of Lemma \ref{lem:path-connectedness}, on induced marginal preferences according to $G_{\approx}^{A^s}$.
Second, by applying Lemmas \ref{lem:tops-onlydomain} and \ref{lem:step2} and the decomposable-domain hypothesis,
we show that every strategy-proof marginal rule on $[\mathbb{D}]^s$ satisfies the tops-only property (see Lemma \ref{lem:tops-onlymarginaldomain}).
Then, by applying Corollary 2 of \citet{CZ2023}, we show that all induced marginal preferences of $[\mathbb{D}]^s$ are hybrid on $\prec^s$ w.r.t.~some marginal thresholds $\underline{x}^s$ and $\overline{x}^s$.
Furthermore, using the preference restriction revealed in Lemma \ref{lem:connectedgraph},
we show that when $\underline{x}^s \neq \overline{x}^s$, the subgraph $G_{\approx}^{\langle \underline{x}^s,\,\overline{x}^s\rangle}$ has no leaf, which meets condition (ii) of Definition \ref{def:AMH} (see Lemma \ref{lem:noleaf}).
Last, by the decomposable-domain hypothesis and Proposition \ref{prop:FBR}, we construct various strategy-proof rules on $\mathbb{D}$ via assembling different FBRs on the induced marginal domains.
We then show that as an implication of strategy-proofness of these assembled rules, all preferences of $\mathbb{D}$ are multidimensional hybrid on $\prec$ w.r.t.~the thresholds $\underline{x}=(\underline{x}^1, \dots, \underline{x}^m)$ and $\overline{x} =(\overline{x}^1, \dots, \overline{x}^m)$ (see Lemma \ref{lem:MH}), which meets condition (ii) of Definition \ref{def:AMH} and hence concludes the whole proof.

\end{description}



By combining Proposition \ref{prop:FBR} and Theorem \ref{thm},
we obtain the Corollary below that provides a full characterization of strategy-proof rules on a rich multidimensional hybrid domain.

\begin{corollary}\label{cor:characterization}
Let $\mathbb{D}$ be a rich multidimensional hybrid domain on $\prec$ w.r.t.~thresholds $\underline{x}$ and $\overline{x}$.
An SCF $f: \mathbb{D}^n \rightarrow A$ is a strategy-proof rule if and only if
$f$ is decomposable,
$f^s$ is an FBR for each $s \in M\backslash M(\underline{x}, \overline{x})$, and
$f^t$ is an $(\underline{x}^t, \overline{x}^t)$-FBR for each $t \in M(\underline{x}, \overline{x})$.
\end{corollary}

%\begin{proof}
%Clearly, by Theorem \ref{thm}, $\mathbb{D}$ is a decomposable domain,
%which immediately by the decomposition and assembling properties implies that
%an SCF $f: \mathbb{D}^n \rightarrow A$ is a strategy-proof rule if and only if
%$f$ is decomposable and all marginal SCFs $f^1, \dots, f^m$ are strategy-proof marginal rules.
%Furthermore, by Lemmas \ref{lem:FBR} and \ref{lem:constrainedFBR} in the proof of Theorem \ref{thm}, we know that for each $s \in M\backslash M(\underline{x}, \overline{x})$, the strategy-proof marginal rule $f^s$ is an FBR, and
%for each $t \in M(\underline{x}, \overline{x})$, the strategy-proof marginal rule $f^t$ is an $(\underline{x}^t, \overline{x}^t)$-FBR.
%\end{proof}












\section{Final remark and literature review}\label{sec:conclusion}

In a class of rich preference domains, multidimensional hybrid domains are shown to be the unique decomposable domains, which enables us to provide a characterization of strategy-proof rules on these domains.

\citet{BJ1983} initiated the study of strategy-proof SCFs in a multidimensional setting where preferences over the real space $\mathbb{R}^m$ are required to be separable and star-shaped (which intuitively speaking, can be viewed as a variant of single-peakedness over $\mathbb{R}^m$).
They characterized that a rule is strategy-proof if only if it can be decomposed into component-wise strategy-proof marginal rules, and further justified the salience of separability by showing that their decomposability result degenerates back to an impossibility result of the Gibbard-Satterthwaite Theorem as soon as separability is slightly tampered with.
Followed by \citet{BSZ1991}, \citet{BGS1993} and \citet{LW1999}, the preference restrictions of inclusion/exclusion separability (separability in the special case that each component contains exactly two elements), multidimensional single-peakedness and top-separability have been introduced respectively, and then characterizations of strategy-proof rules on the corresponding restricted domains have been established, which indicate that all these domains are decomposable domains.
Our class of multidimensional hybrid domains contains these important domains, and our characterization of strategy-proof rules on the multidimensional hybrid domain covers their characterization results as special cases. More discussion on strategy-proof social choice functions in multidimensional settings can be found in the two comprehensive survey papers of \citet{S1995} and \citet{B2011}.
It is worth mentioning that \citet{BGS1993} and \citet{LW1999} were only able to derive decomposability for \emph{voting schemes}\footnote{A voting scheme is a function $g: \underset{n}{\underbrace{A\times \dots \times A}} \rightarrow A$ that associates to each profile of top-ranked alternatives, an alternative. Formally, a voting scheme $g: \underset{n}{\underbrace{A\times \dots \times A}} \rightarrow A$ is \emph{decomposable} if there exists a marginal voting scheme $g^s:  \underset{n}{\underbrace{A^s\times \dots \times A^s}} \rightarrow A^s $ for each $s \in M$ such that for all profile $(x_1, \dots, x_n) \in \underset{n}{\underbrace{A\times \dots \times A}}$, we have
$[g(x_1, \dots, x_n) = a] \Leftrightarrow [g^s(x_1^s, \dots, x_n^s) = a^s\; \textrm{for all}\; s \in M]$.
Note that a tops-only SCF degenerates to a voting scheme naturally.} since their domains involve non-separable preferences and they did not derive marginal preferences as we do here. Therefore, their characterizations of strategy-proof rules require a combination of the tops-only property endogenously and separately established on rules, with the decomposition of the corresponding voting schemes. Our way of deriving marginal preferences provides a unified approach for analyzing decomposability of strategy-proof rules in a multidimensional model involving both separable and non-separable preferences.
\citet{LS1999} restricted attention to separable preferences and introduced an elegant richness condition (loosely speaking, sufficiently many lexicographically separable preferences are included) on the preference domain that ensures its decomposability.
They then applied their decomposability result to fully characterize strategy-proof rules on the domain of preferences that are both separable and multidimensional single-peaked, and furthermore extended their characterization result to the multidimensional single-peaked domain where non-separable preferences are involved, via the top-only property endogenously embedded in all strategy-proof rules.
Our richness condition is different, and mainly related to the no-restoration condition widely explored in the recent literature investigating the equivalence between strategy-proofness and local strategy-proofness  \citep[e.g.,][]{S2013,KRSYZ2021a}.
More importantly, our theorem not only shows that multidimensional hybridness under our richness condition is sufficient for the domain to be a decomposable domain, but also characterizes the necessity of
multidimensional hybridness for decomposable domains.
Recently, \citet{GMS2020} study a multidimensional model where the preference over alternatives of $\mathbb{R}^m$ is measured by a norm towards the preference peak;
their main result shows that \emph{the marginal median mechanism} (i.e., an assembling of median marginal rules at all components) is strategy-proof if and only if the norm satisfies \emph{orthant monotonicity}, which implies the star-shape preference restriction of \citet{BJ1983} and generalizes the requirement of separability.
\citet{CZ2019} also investigate the domain implication of strategy-proof rules in a multidimensional setting, and have characterized that under a mild richness condition which is adopted by this paper, multidimensional single-peakedness is necessary and sufficient for the existence of an anonymous and strategy-proof rule.\footnote{This domain characterization result provides an evidence in favor of the Gul Conjecture \citep[see Section 6.5.2 of][]{B2011} attributed to Faruk Gul in a multidimensional setting.}
Their investigation however cannot be used to detect decomposability of all strategy-proof rules.
Our paper does not concentrate on specific SCFs like the marginal median mechanism, or exogenously require the SCF to be anonymous, but focuses on an environment that ensures decomposability of all strategy-proof rules.




\setlength{\bibsep}{0ex}
\begin{thebibliography}{39}
\newcommand{\enquote}[1]{``#1''}
\expandafter\ifx\csname natexlab\endcsname\relax\def\natexlab#1{#1}\fi

\bibitem[\protect\citeauthoryear{Barber{\`a}}{Barber{\`a}}{2011}]{B2011}
\textsc{Barber{\`a}, S.} (2011): \enquote{Strategy-proof social choice,} in
  \emph{Handbook of Social Choice and Welfare}, vol.~2, 731--831.

\bibitem[\protect\citeauthoryear{Barber{\`a}, Gul, and Stacchetti}{Barber{\`a}
  et~al.}{1993}]{BGS1993}
\textsc{Barber{\`a}, S., F.~Gul, and E.~Stacchetti} (1993):
  \enquote{Generalized median voter schemes and committees,} \emph{Journal of
  Economic Theory}, 61, 262--289.

\bibitem[\protect\citeauthoryear{Barber{\`a}, Sonnenschein, and
  Zhou}{Barber{\`a} et~al.}{1991}]{BSZ1991}
\textsc{Barber{\`a}, S., H.~Sonnenschein, and L.~Zhou} (1991): \enquote{Voting
  by committees,} \emph{Econometrica}, 59, 595--609.

\bibitem[\protect\citeauthoryear{Border and
  Jordan}{Border and Jordan}{1983}]{BJ1983}
\textsc{Border, K., and J.~Jordan} (1983): \enquote{Straightforward elections, unanimity and phantom
voters,} \emph{Review of Economic Studies}, 50(1), 153--170.

\bibitem[\protect\citeauthoryear{Chatterji, Roy, Sadhukhan, Sen, and Zeng}{Chatterji et~al.}{2022}]{CRSSZ2022}
\textsc{Chatterji, S., S. Roy, S. Sadhukhan, A. Sen,  and H. Zeng} (2022): \enquote{Probabilistic fixed ballot rules and hybrid domains,}
\emph{Journal of Mathematical Economics}, 100, 102656.

\bibitem[\protect\citeauthoryear{Chatterji, Sanver, and Sen}{Chatterji et~al.}{2013}]{CSS2013}
\textsc{Chatterji, S., R. Sanver, and A.~Sen} (2013): \enquote{On domains that admit well-behaved strategy-proof social choice functions,}
\emph{Journal of Economic Theory}, 148, 1050--1073.

\bibitem[\protect\citeauthoryear{Chatterji and Zeng}{Chatterji
  and Zeng}{2019}]{CZ2019}
\textsc{Chatterji, S. and H.~Zeng} (2019): \enquote{Random mechanism design on multidimensional domains,}
\emph{Journal of Economic Theory}, 182, 25--105.

\bibitem[\protect\citeauthoryear{Chatterji and Zeng}{Chatterji
  and Zeng}{2023}]{CZ2023}
\textsc{Chatterji, S. and H.~Zeng} (2023): \enquote{A taxonomy of non-dictatorial unidimensional domains,}
\emph{Games and Economic Behavior}, 137, 228--269.

%\bibitem[\protect\citeauthoryear{Chatterji and Zeng}{Chatterji and
%  Zeng}{2022}]{CZ2023}
%\textsc{Chatterji, S. and H.~Zeng} (2022): \enquote{
%\href{https://arxiv.org/abs/2201.00496}{\textcolor[rgb]{0.00,0.00,1.00}{A taxonomy of non-dictatorial unidimensional domains}},} \emph{mimeo}.

\bibitem[\protect\citeauthoryear{Gershkov, Moldovanu, and Shi}{Gershkov et~al.}{2020}]{GMS2020}
\textsc{Gershkov, A., B. Moldovanu, and X. Shi} (2020): \enquote{Monotonic norms and othogonal issues in multidimensional voting,} \emph{Journal of Economic Theory}, 129, 105103.

\bibitem[\protect\citeauthoryear{Gibbard}{Gibbard}{1973}]{G1973}
\textsc{Gibbard, A.} (1973): \enquote{Manipulation of voting schemes: A general result,} \emph{Econometrica}, 587--601.


%\bibitem[\protect\citeauthoryear{Kemeny}{Kemeny}{1959}]{K1959}
%\textsc{Kemeny, J. G.} (1959): \enquote{Mathematics without numbers,} \emph{Daedalus}, 88, 577--591.


\bibitem[\protect\citeauthoryear{Le~Breton and Sen}{Le~Breton and
  Sen}{1999}]{LS1999}
\textsc{Le~Breton, M. and A.~Sen} (1999): \enquote{Separable preferences,
  strategyproofness and decomposability,} \emph{Econometrica}, 67, 605--628.

\bibitem[\protect\citeauthoryear{Le~Breton and Weymark}{Le~Breton and
  Weymark}{1999}]{LW1999}
\textsc{Le~Breton, M. and J.~Weymark} (1999): \enquote{Strategy-proof social
  choice with continuous separable preferences,} \emph{Journal of Mathematical
  Economics}, 32, 47--85.

\bibitem[\protect\citeauthoryear{Kumar, Roy, Sen, Yadav, and Zeng}{Kumar et~al.}{2021}]{KRSYZ2021a}
\textsc{Kumar, U., S.~Roy, A.~Sen, S.~Yadav, and H.~Zeng} (2021): \enquote{Local-global equivalence in voting models: A characterization and applications,} \emph{Theoretical Economics}, 16, 1195--1220.

%\bibitem[\protect\citeauthoryear{Kumar, Roy, Sen, Yadav, and Zeng}{Kumar et~al.}{2021}]{KRSYZ2021b}
%\textsc{Kumar, U., S.~Roy, A.~Sen, S.~Yadav, and H.~Zeng} (2021): \enquote{Local global equivalence for unanimous social choice functions,} \emph{Games and Economic Behavior}, 130, 299--308.

\bibitem[\protect\citeauthoryear{Morimoto and Serizawa}{Morimoto and
  Serizawa}{2015}]{MS2015}
\textsc{Morimoto, S. and S.~Serizawa} (2015): \enquote{Strategy-proofness and
  efficiency with non-quasi-linear preferences: A characterization of minimum
  price Walrasian rule,} \emph{Theoretical Economics}, 10, 445--487.

\bibitem[\protect\citeauthoryear{Moulin}{Moulin}{1980}]{M1980}
\textsc{Moulin, H.} (1980): \enquote{On strategy-proofness and single peakedness,} \emph{Public Choice}, 35(4), 437--455.

%\bibitem[\protect\citeauthoryear{Reffgen and Svensson}{Reffgen and
%  Svensson}{2012}]{RS2012}
%\textsc{Reffgen, A. and L.-G. Svensson} (2012): \enquote{Strategy-proof voting
%  for multiple public goods,} \emph{Theoretical Economics}, 7, 663--688.

\bibitem[\protect\citeauthoryear{Roberts}{Roberts}{1979}]{R1979}
\textsc{Roberts, K.} (1979): \enquote{The characterization of implementable choice rules,} \emph{Aggregation and Revelation of Preferences}, 321--349, J-J. Laffont (ed.), North Holland Publishing Company.

%\bibitem[\protect\citeauthoryear{Rochet}{Rochet}{1987}]{R1987}
%\textsc{Rochet, J.-C.} (1987): \enquote{A necessary and sufficient condition for rationalizability in a quasi-linear context,} \emph{Journal of Mathematical Economics}, 16, 191--200.

%\bibitem[\protect\citeauthoryear{Roy and Sadhukhan}{Roy and Sadhukhan}{2019}]{RS2019}
%\textsc{Roy, S. and S. Sadhukhan} (2019): \enquote{A characterization of random min-max domains and its applications,} \emph{Economic Theory}, 68, 887--906.

%\bibitem[\protect\citeauthoryear{Sen}{Sen}{2001}]{S2001}
%\textsc{Sen, A.} (2001): \enquote{ Another direct proof of the Gibbard-Satterthwaite theorem,} \emph{Economic Letters}, 70, 381--385.

\bibitem[\protect\citeauthoryear{Sato}{Sato}{2013}]{S2013}
\textsc{Sato, S.} (2013): \enquote{A sufficient condition for the equivalence of strategy-proofness and nonmanipulability by preferences adjacent to the sincere one,} \emph{Journal of Economic Theory}, 148, 259--278.

\bibitem[\protect\citeauthoryear{Satterthwaite}{Satterthwaite}{1975}]{S1975}
\textsc{Satterthwaite, M.~A.} (1975): \enquote{Strategy-proofness and Arrow's conditions:
Existence and correspondence theorems for voting procedures and social welfare
functions,} \emph{Journal of Economic Theory}, 10, 187--217.

\bibitem[\protect\citeauthoryear{Sen}{Sen}{2001}]{Sen2001}
\textsc{Sen, A.} (2001): \enquote{Another direct proof of the Gibbard-Satterthwaite Theorem,} \emph{Economic Letters}, 70, 381--385.

\bibitem[\protect\citeauthoryear{Sprumont}{Sprumont}{1995}]{S1995}
\textsc{Sprumont, Y.} (1995): \enquote{Strategyproof collective choice in economic and political environments,} \emph{Canadian Journal of Economics}, 68--107.

\end{thebibliography}

\newpage

\appendix

\section*{Appendix}

\section{Proof of Proposition \ref{prop:FBR}}\label{app:FBR}

For ease of presentation, for each $s \in M$, we assume w.l.o.g.~that $A^s \coloneqq \big\{a_1^s,\dots, a_{|A^s|}^s\big\}$ and
$a_k^s \prec^s a_{k+1}^s$ for all $k = 1, \dots, |A^s|-1$.

First, given a component $s \in M$ such that $\underline{x}^s = \overline{x}^s$,
note that hybridness embedded in $[\mathbb{D}]^s$ degenerates to single-peakedness.
Hence, $[\mathbb{D}]^s$ is a single-peaked marginal domain on $\prec^s$.
Moreover, since $G_{\approx}^{A^s}$ is a connected graph,
it is true that $G_{\approx}^{A^s}$ equals the line $(a_1^s, \dots, a_k^s, a_{k+1}^s, \dots, a_{|A^s|}^s)$.
Then, by the proof of Theorem 2 of \citet{CRSSZ2022}, we know that a marginal SCF $f^s: \big[[\mathbb{D}]^s\big]^n \rightarrow A^s$ is a strategy-proof marginal rule if and only if it is an FBR.
This proves statement (i) of Proposition \ref{prop:FBR}.

Henceforth, we focus on showing statement (ii).
For the next lemma, we introduce some notation to simplify the presentation.
Given $s \in M$ and two voters $1$ and $2$,
let $\big((a^s \cdots), (b^s\cdots)\big)$ denote a two-voter preference profile such that
voter 1 reports an arbitrary induced marginal preference with the peak $a^s$, and
voter 2 reports an arbitrary induced marginal preference with the peak $b^s$.
More importantly, given a two-voter SCF $h^s: \big[[\mathbb{D}]^s\big]^2 \rightarrow A^s$,
let $h^s\big((a^s \cdots), (b^s\cdots)\big) = c^s$ denote ``$h^s\big([P_1]^s, [P_2]^s\big) = c^s$ for all $[P_1]^s, [P_2]^s \in [\mathbb{D}]^s$ such that $r_1([P_1]^s) = a^s$ and $r_1([P_2]^s) = b^s$''.
Symmetrically, let $\big((a^s \cdots), [P_{-i}]^s\big)$,
where $[P_{-i}]^s \coloneqq ([P_1]^s, \dots, [P_{i-1}]^s, [P_{i+1}]^s, \dots, [P_n]^s)$,
denote an $n$-voter preference profile such that
voter $i$ reports an arbitrary induced marginal preference with the peak $a^s$, and
all other voters report the induced marginal preferences $[P_1]^s, \dots, [P_{i-1}]^s, [P_{i+1}]^s, \dots, [P_n]^s$ respectively, and let $f^s\big((a^s \cdots), [P_{-i}]^s\big) = c^s$ denote ``$f^s\big([P_i]^s, [P_{-i}]^s\big) = c^s$ for all $[P_i]^s \in [\mathbb{D}]^s$ such that $r_1([P_i]^s) = a^s$''.

\begin{lemma}\label{lem:fundamental}
Given $s \in M$,
fix a two-voter strategy-proof marginal rule $h^s: \big[[\mathbb{D}]^s\big]^2 \rightarrow A^s$.
Given a path $(x_1^s, \dots, x_v^s)$ in $G_{\approx}^{A^s}$, the following two statements hold:
\begin{itemize}
\item[\rm (i)] \makebox{If $h^s\big((x_1^s\cdots), (x_2^s\cdots)\big) = x_1^s$,
then $h^s\big((x_k^s\cdots), (x_{k'}^s\cdots)\big) = x_k^s$ for all $1 \leq k \leq k' \leq v$.}

\vspace{-0.5em}
\item[\rm (ii)] \makebox{If $h^s\big((x_2^s\cdots), (x_1^s\cdots)\big) = x_1^s$,
then $h^s\big((x_{k'}^s\cdots), (x_k^s\cdots)\big) = x_k^s$ for all $1 \leq k \leq k' \leq v$.}
\end{itemize}
\end{lemma}

\begin{proof}
The two statements are symmetric. We focus on verifying statement (i).
We first claim $h^s\big((x_k^s\cdots), (x_{k+1}^s\cdots)\big) = x_k^s$ for all $1 \leq k < v$.
Since $x_2^s \approx x_3^s$, by Claims A and B of \citet{Sen2001} and their proofs,
we have either $h^s\big((x_2^s\cdots), (x_3^s\cdots)\big) = x_2^s$
or $h^s\big((x_2^s\cdots), (x_3^s\cdots)\big) =x_3^s$.
Suppose by contradiction that $h^s\big((x_2^s\cdots), (x_3^s\cdots)\big) = x_3^s$.
Since $x_1^s \approx x_2^s$, we have $[P_1]^s, [P_1']^s\in [\mathbb{D}]^s$ such that
$r_1([P_1]^s) = r_2([P_1']^s) = x_1^s$, $r_1([P_1']^s) = r_2([P_1]^s) = x_2^s$ and $r_{\ell}([P_1]^s) = r_{\ell}([P_1']^s)$ for all $3 \leq \ell \leq |A^s|$.
Similarly, by $x_2^s \approx x_3^s$, we have  $[P_2]^s, [P_2']^s \in [\mathbb{D}]^s$ such that
$r_1([P_2]^s) = r_2([P_2']^s) = x_2^s$, $r_1([P_2']^s) = r_2([P_2]^s) = x_3^s$ and $r_{\ell}([P_2]^s) = r_{\ell}([P_2']^s)$ for all $3 \leq \ell \leq |A^s|$.
Thus, $h^s([P_1]^s, [P_2]^s) = x_1^s$ and $h^s([P_1']^s, [P_2']^s) = x_3^s$.
On the one hand,
according to $[P_2]^s$ and $[P_2']^s$,
by strategy-proofness, $h^s([P_1]^s, [P_2]^s) = x_1^s$ first implies $h^s([P_1]^s, [P_2']^s) = x_1^s$.
On the other hand, according to $[P_1']^s$ and $[P_1]^s$,
by strategy-proofness, $h^s([P_1']^s, [P_2']^s) = x_3^s$ implies $h^s([P_1]^s, [P_2']^s) = x_3^s$ - a contradiction.
Therefore, $h^s\big((x_2^s\cdots), (x_3^s\cdots)\big) = x_2^s$.
Following the path $(x_1^s, \dots, x_v^s)$, by repeatedly applying the argument above, we eventually have
$h^s\big((x_k^s\cdots), (x_{k+1}^s\cdots)\big) = x_k^s$ for all $1 \leq k < v$.

Now, in conjunction with unanimity,
we have $h^s\big((x_k^s\cdots), (x_{k'}^s\cdots)\big) = x_k^s$ for all $1 \leq k \leq k' \leq v$ with $k'-k \leq 1$.
Next, we fix $2 \leq l \leq v-1$ and introduce an induction hypothesis: $h^s\big((x_k^s\cdots), (x_{k'}^s\cdots)\big) = x_k^s$ for all $1 \leq k \leq k' \leq v$ with $k'-k <l$.
We show $h^s\big((x_k^s\cdots), (x_{k'}^s\cdots)\big) = x_k^s$ for all $1 \leq k \leq k' \leq v$ with $k'-k = l$.
Fixing arbitrary $1 \leq k \leq k' \leq v$ with $k'-k = l$
and arbitrary $[P_2]^s \in [\mathbb{D}]^s$ with $r_1([P_2]^s) = x_{k'}^s$,
we show $h^s\big((x_k^s\cdots), [P_2]^s\big) = x_k^s$.
Since $k'-k = l \geq 2$, we have $1 \leq k'-(k+1) < l$.
Then, the induction hypothesis implies $h^s\big((x_{k+1}^s\cdots), [P_2]^s\big) = x_{k+1}^s$.
Since $x_k^s \approx x_{k+1}^s$, we have $[P_1]^s, [P_1']^s \in [\mathbb{D}]^s$ such that
$r_1([P_1]^s) = r_2([P_1']^s) = x_{k+1}^s$, $r_1([P_1']^s) = r_2([P_1]^s) = x_{k}^s$ and
$r_{\ell}([P_1]^s) = r_{\ell}([P_1']^s)$ for all $3 \leq \ell \leq |A^s|$.
Thus, $h^s([P_1]^s, [P_2]^s) = x_{k+1}^s$.
Furthermore, by strategy-proofness, $h^s([P_1]^s, [P_2]^s) = x_{k+1}^s$ implies $h^s([P_1']^s, [P_2]^s) \in \{x_k^s, x_{k+1}^s\}$.
Note that $h^s([P_1']^s, [P_2]^s) = x_{k+1}^s$ contradicts the induction hypothesis.
Therefore, $h^s([P_1']^s, [P_2]^s) = x_k^s$, and hence $h^s\big((x_k^s\cdots), [P_2]^s\big) = x_k^s$ by strategy-proofness, as required.
This completes the verification of the induction hypothesis, and hence proves statement (i).
\end{proof}


For the next three lemmas, we fix a component $s \in M$ such that $\underline{x}^s \neq \overline{x}^s$, and investigate strategy-proof marginal rules on $[\mathbb{D}]^s$.
First, note that every marginal preference of $[\mathbb{D}]^s$ is hybrid on $\prec^s$ w.r.t.~$\underline{x}^s$ and $\overline{x}^s$, $G_{\approx}^{A^s}$ is a connected graph, and
$G_{\approx}^{\langle \underline{x}^s,\, \overline{x}^s\rangle}$ has no leaf.
We assume w.l.o.g.~that $\underline{x}^s = a_{\underline{k}}^s$ and $\overline{x}^s = a_{\overline{k}}^s$ for some $1 \leq \underline{k} < \overline{k} \leq |A^s|$ with $\overline{k}-\underline{k}>1$.
Consequently, $G_{\approx}^{A^s}$ must be a combination of the line $(a_1^s, \dots, a_{\underline{k}}^s)$,
the connected subgraph $G_{\approx}^{\langle a_{\underline{k}}^s,\, a_{\overline{k}}^s\rangle}$ that has no leaf, and
the line $(a_{\overline{k}}^s, \dots, a_{|A^s|}^s)$.
By statement (i) of Lemma 4 of \citet{CZ2023},
we first know that every two-voter strategy-proof marginal rules  behaves like a dictatorship on
$\langle a_{\underline{k}}^s, a_{\overline{k}}^s\rangle$ (recall footnote \ref{footnote:behavelikeadictatorship}).
The lemma below generalizes the result to the case of $n$ voters, where $n \geq 2$.

\begin{lemma}\label{lem:behavelikeadictatorship}
For all $n \geq 2$, every strategy-proof marginal rule $f^s: \big[[\mathbb{D}]^s\big]^n \rightarrow A^s$ behaves like a dictatorship on $\langle a_{\underline{k}}^s,\, a_{\overline{k}}^s\rangle$, i.e.,
there exists $i \in N$ such that for all $[P_1]^s, \dots, [P_n]^s \in [\mathbb{D}]^s$,
$\big[r_1([P_1]^s), \dots, r_1([P_n]^s) \in \langle a_{\underline{k}}^s,\, a_{\overline{k}}^s\rangle\big]
\Rightarrow \big[f^s([P_1]^s, \dots, [P_n]^s) = r_1([P_i]^s)\big]$.
\end{lemma}

\begin{proof}
We first know that the lemma holds for the case $n=2$.
Next, we fix an integer $n >2$ and introduce an induction argument.\medskip

\noindent
\textsc{Induction Hypothesis}: For all $2 \leq n' < n$, every strategy-proof marginal rule $f^{s\,'}: \big[[\mathbb{D}]^s\big]^{n'} \rightarrow A^s$ behaves like a dictatorship on $\langle a_{\underline{k}}^s, a_{\overline{k}}^s\rangle$.\medskip

We henceforth fix a strategy-proof marginal rule $f^s: \big[[\mathbb{D}]^s\big]^n \rightarrow A^s$, and show that $f^s$ behaves like a dictatorship on $\langle a_{\underline{k}}^s, a_{\overline{k}}^s\rangle$.
Given arbitrary $i \in N$, we combine all voters other than $i$ as one, say voter $j$, and then induce a two-voter marginal SCF $h_i^s([P_i]^s, [P_j]^s)
= f^s([P_i]^s,[P_j]^s, \dots, [P_j]^s)$ for all $[P_i]^s, [P_j]^s \in [\mathbb{D}]^s$.
It is clear that $h_i^s$ inherits unanimity and strategy-proofness from $f^s$, and hence by statement (a) behaves like a dictatorship on $\langle a_{\underline{k}}^s, a_{\overline{k}}^s\rangle$:
either voter $i$ or $j$ dictates on $\langle a_{\underline{k}}^s, a_{\overline{k}}^s\rangle$ at $h_i^s$.
The claim below addresses a deeper relation between $h_i^s$ and $f^s$.
\medskip

\noindent
\textsc{Claim 1}: If voter $i$ dictates on $\langle a_{\underline{k}}^s, a_{\overline{k}}^s\rangle$ at $h_i^s$, then voter $i$ dictates on $\langle a_{\underline{k}}^s, a_{\overline{k}}^s\rangle$ at $f^s$.
\medskip

Suppose not, i.e., there exist $[P_1]^s, \dots, [P_n]^s \in [\mathbb{D}]^s$ such that
$r_1([P_1]^s), \dots, r_1([P_n]^s) \in \langle a_{\underline{k}}^s, a_{\overline{k}}^s\rangle$ and
$f^s([P_1]^s, \dots, [P_n]^s) \neq r_1([P_i]^s)$.
For notational convenience, let $r_1([P_i]^s) = a_k^s$ and $f^s([P_1]^s, \dots, [P_n]^s) = a_v^s$.
By minimal richness of $[\mathbb{D}]^s$, we fix a marginal preference $[\hat{P}_j]^s \in [\mathbb{D}]^s$ such that $r_1([\hat{P}_j]^s) = a_v^s$.
Immediately, by strategy-proofness of $f^s$, we know $h_i^s([P_i]^s, [\hat{P}_j]^s) = f^s([P_i]^s, [\hat{P}_j]^s, \dots, [\hat{P}_j]^s) = a_v^s \neq a_k^s$.
Then, to be consistent with voter $i$'s dictatorship on $\langle a_{\underline{k}}^s, a_{\overline{k}}^s\rangle$ at $h_i^s$, it must be the case that $a_v^s \notin \langle a_{\underline{k}}^s, a_{\overline{k}}^s\rangle$.
Recall that $G_{\approx}^{A^s}$ is a combination of the line $(a_1^s, \dots, a_{\underline{k}}^s)$,
the connected subgraph $G_{\approx}^{\langle a_{\underline{k}}^s, a_{\overline{k}}^s\rangle}$ that has no leaf, and the line $(a_{\overline{k}}^s, \dots, a_{|A^s|}^s)$.
Hence, there must exist a path $(x_1^s, \dots, x_{\ell}^s)$ in $G_{\approx}^{A^s}$ such that
$\ell \geq 3$,
$x_1^s, x_2^s \in \langle a_{\underline{k}}^s, a_{\overline{k}}^s\rangle$,
$a_k^s \in \{x_1^s, x_2^s\}$ and $a_v^s = x_{\ell}^s$.
By voter $i$'s dictatorship on $\langle a_{\underline{k}}^s, a_{\overline{k}}^s\rangle$ at $h_i^s$,
we first have $h_i^s\big((x_1^s\cdots), (x_2^s\cdots)\big)= x_1^s$.
Then, along the path $(x_1^s, \dots, x_{\ell}^s)$, by statement (i) of Lemma \ref{lem:fundamental},
we have $h_i^s\big((x_1^s\cdots), (x_{\ell}^s\cdots)\big)= x_1^s$ and $h_i^s\big((x_2^s\cdots), (x_{\ell}^s\cdots)\big)= x_2^s$, which imply $h_i^s([P_i]^s, [\hat{P}_j]^s) = a_k^s$ - a contradiction.
This completes the verification of the claim.
\medskip

Next, we fix voters $1$ and $2$, and clone them to construct a marginal SCF:
for all $[P_1]^s ,[P_3]^s, \dots, [P_n]^s\in [\mathbb{D}]^s$,
$g^s([P_1]^s, [P_3]^s, \dots, [P_n]^s) = f^s([P_1]^s, [P_1]^s, [P_3]^s, \dots, [P_n]^s)$.
It is evident that $g^s$ inherits unanimity and strategy-proofness from $f^s$, and hence by the induction hypothesis behaves like a dictatorship on $\langle a_{\underline{k}}^s, a_{\overline{k}}^s\rangle$.
Thus, there are two cases at $g^s$: (1) some voter $o \in \{3, \dots, n\}$ dictates on $\langle a_{\underline{k}}^s, a_{\overline{k}}^s\rangle$, and (2) voter $1$ dictates on $\langle a_{\underline{k}}^s, a_{\overline{k}}^s\rangle$.\medskip

\noindent
\textsc{Claim 2}: In case (1), voter $o$ dictates on $\langle a_{\underline{k}}^s, a_{\overline{k}}^s\rangle$ at $f^s$.\medskip

By voter $o$'s dictatorship on $\langle a_{\underline{k}}^s, a_{\overline{k}}^s\rangle$ at $g^s$,
we know $h_o^s([P_o]^s, [P_j]^s) = f^s([P_o]^s, [P_j]^s, \dots, [P_j]^s) = g^s\left(\frac{[P_j]^s}{\textrm{voter 1}},\frac{[P_j]^s}{\textrm{voters $3, \dots, o-1$}}, [P_o]^s, \frac{[P_j]^s}{\textrm{voters $o+1, \dots, n$}}\right)$ for all $[P_o]^s, [P_j]^s \in [\mathbb{D}]^s$ with $r_1([P_o]^s), r_1([P_j]^s) \in \langle a_{\underline{k}}^s, a_{\overline{k}}^s\rangle$.
This indicates that voter $o$ dictates on $\langle a_{\underline{k}}^s, a_{\overline{k}}^s\rangle$ at $h_o^s$.
Then, by Claim 1, voter $o$ dictates on $\langle a_{\underline{k}}^s, a_{\overline{k}}^s\rangle$ at $f^s$.
This completes the verification of the claim.
\medskip

Henceforth, we assume that case (2) occurs.
Thus, voter $1$ dictates on $\langle a_{\underline{k}}^s, a_{\overline{k}}^s\rangle$ at $g^s$.
Recall that $G_{\approx}^{\langle a_{\underline{k}}^s,\, a_{\overline{k}}^s\rangle}$ is connected subgraph and has no leaf.
Then, there must exist a cycle $\mathcal{C} \coloneqq (x_1^s, \dots, x_v^s)$ in $G_{\approx}^{\langle a_{\underline{k}}^s, \,a_{\overline{k}}^s\rangle}$, i.e., $v \geq 3$, $x_k^s \approx x_{k+1}^s$ for all $k = 1, \dots, v-1$, and $x_v^s \approx x_1^s$.
For the next two claims, we fix a marginal preference $[P_3]^s \in [\mathbb{D}]^s$ such that $r_1([P_3]^s) = x_v^s$.\medskip

\noindent
\textsc{Claim 3}:
For all $k=1, \dots, v-1$, we have either $f^s\big((x_k^s\cdots),(x_{k+1}^s\cdots), [P_3]^s, \dots, [P_3]^s\big) = x_k^s$, or
$f^s\big((x_k^s\cdots),(x_{k+1}^s\cdots), [P_3]^s, \dots, [P_3]^s\big) = x_{k+1}^s$.\medskip

Since $x_k^s \approx x_{k+1}^s$,
we have $[\hat{P}_1]^s, [\hat{P}_2]^s \in [\mathbb{D}]^s$ such that
$r_1([\hat{P}_1]^s) = r_2([\hat{P}_2]^s) = x_k^s$,
$r_1([\hat{P}_2]^s) = r_2([\hat{P}_1]^s) = x_{k+1}^s$ and
$r_{\ell}([\hat{P}_1]^s) = r_{\ell}([\hat{P}_2]^s)$ for all $\ell \in \{3, \dots, |A^s|\}$.
Immediately,
by voter $1$'s dictatorship on $\langle a_{\underline{k}}^s, a_{\overline{k}}^s\rangle$ at $g^s$,
we have $f^s([\hat{P}_1]^s, [\hat{P}_1]^s, [P_3]^s, \dots, [P_3]^s) = g^s([\hat{P}_1]^s, [P_3]^s, \dots, [P_3]^s) =x_k^s$ and $f^s([\hat{P}_2]^s, [\hat{P}_2]^s, [P_3]^s, \dots, [P_3]^s) = g^s([\hat{P}_2]^s, [P_3]^s, \dots, [P_3]^s) =x_{k+1}^s$.
It is clear that by strategy-proofness, $f^s([\hat{P}_1]^s, [\hat{P}_1]^s, [P_3]^s, \dots, [P_3]^s) =x_k^s$ implies $f^s([\hat{P}_1]^s, [\hat{P}_2]^s, [P_3]^s, \dots, [P_3]^s) \in \{x_k^s, x_{k+1}^s\}$.
We first assume $f^s([\hat{P}_1]^s, [\hat{P}_2]^s, [P_3]^s, \dots, [P_3]^s) =x_k^s$, and show
$f^s\big((x_k^s\cdots),(x_{k+1}^s\cdots), [P_3]^s, \dots, [P_3]^s\big) =x_k^s$.
Given arbitrary $[P_1]^s, [P_2]^s \in [\mathbb{D}]^s$ such that $r_1([P_1]^s) =x_k^s$ and $r_1([P_2]^s) =x_{k+1}^s$,
we claim $f^s([\hat{P}_1]^s, [P_2]^s, [P_3]^s, \dots, [P_3]^s) \in \{x_k^s, x_{k+1}^s\}$.
Suppose not, i.e., $f^s([\hat{P}_1]^s, [P_2]^s, [P_3]^s, \dots, [P_3]^s) \coloneqq z^s \notin \{x_k^s, x_{k+1}^s\}$.
According to voter $1$'s dictatorship on $\langle a_{\underline{k}}^s, a_{\overline{k}}^s\rangle$ at $g^s$,
we have $f^s([P_2]^s, [P_2]^s, [P_3]^s, \dots, [P_3]^s) = g^s([P_2]^s, [P_3]^s, \dots, [P_3]^s) = x_{k+1}^s$.
Consequently, since $x_{k+1}^s\mathrel{[\hat{P}_1]^s}z^s$,
voter $1$ will manipulate at $([\hat{P}_1]^s, [P_2]^s, [P_3]^s, \dots, [P_3]^s)$ via $[P_2]^s$.
Hence, $f^s([\hat{P}_1]^s, [P_2]^s, [P_3]^s, \dots, [P_3]^s) \in \{x_k^s, x_{k+1}^s\}$.
Furthermore, if $f^s([\hat{P}_1]^s, [P_2]^s, [P_3]^s, \dots, [P_3]^s) =x_{k+1}^s$,
voter $2$ will manipulate at $([\hat{P}_1]^s, [\hat{P}_2]^s, [P_3]^s, \dots, [P_3]^s)$ via $[P_2]^s$.
Hence, we have $f^s([\hat{P}_1]^s, [P_2]^s, [P_3]^s, \dots, [P_3]^s) =x_k^s$.
Then, by strategy-proofness, $f^s([P_1]^s, [P_2]^s, [P_3]^s, \dots, [P_3]^s) =x_k^s$, as required.
Symmetrically, if $f^s([\hat{P}_1]^s, [\hat{P}_2]^s, [P_3]^s, \dots, [P_3]^s) =x_{k+1}^s$,
we have $f^s\big((x_1^s\cdots),(x_2^s\cdots), [P_3]^s, \dots, [P_3]^s\big) = x_{k+1}^s$.
This completes the verification of the claim.
\medskip


\noindent
\textsc{Claim 4}: Either voter $1$ or $2$ dictates on $\langle a_{\underline{k}}^s, a_{\overline{k}}^s\rangle$ at $f^s$.\medskip

By Claim 3, we assume $f^s\big((x_1^s\cdots),(x_2^s\cdots), [P_3]^s, \dots, [P_3]^s\big) = x_2^s$.
We claim that voter $2$ dictates on $\langle a_{\underline{k}}^s, a_{\overline{k}}^s\rangle$ at $f^s$.
Since $x_1^s \approx x_v^s$, we have $[\hat{P}_1]^s, [\tilde{P}_1]^s \in [\mathbb{D}]^s$ such that $r_1([\hat{P}_1]^s) = r_2([\tilde{P}_1]^s) = x_1^s$, $r_1([\tilde{P}_1]^s) = r_2([\hat{P}_1]^s) = x_v^s$ and
$r_{\ell}([\hat{P}_1]^s) = r_{\ell}([\tilde{P}_1]^s)$ for all $\ell \in \{3, \dots, |A^s|\}$.
Clearly, $f^s\big([\hat{P}_1]^s,(x_2^s\cdots), [P_3]^s, \dots, [P_3]^s\big) = x_2^s$.
Then, by strategy-proofness, we have $f^s\big([\tilde{P}_1]^s,(x_2^s\cdots), [P_3]^s, \dots, [P_3]^s\big) =f^s\big([\hat{P}_1]^s,(x_2^s\cdots), [P_3]^s, \dots, [P_3]^s\big) =x_2^s$.
Now, we refer to the strategy-proof marginal rule $h_2^s$
which behaves like a dictatorship on $\langle a_{\underline{k}}^s, a_{\overline{k}}^s\rangle$.
We claim that voter $2$ dictates on $\langle a_{\underline{k}}^s, a_{\overline{k}}^s\rangle$ at $h_2^s$.
Otherwise, we have $f^s\big([P_3]^s, (x_2^s\cdots), [P_3]^s, \dots, [P_3]^s\big) = h_2^s\big((x_2^s\cdots), [P_3]^s\big) = x_v^s$, which by strategy-proofness implies $f^s\big([\tilde{P}_1]^s,(x_2^s\cdots), [P_3]^s, \dots, [P_3]^s\big) = x_v^s$ - a contradiction.
Hence, voter $2$ dictates on $\langle a_{\underline{k}}^s, a_{\overline{k}}^s\rangle$ at $h_2^s$, and furthermore by Claim 1, also dictates on $\langle a_{\underline{k}}^s, a_{\overline{k}}^s\rangle$ at $f^s$.

Next, by Claim 3, we assume $f^s\big((x_1^s\cdots),(x_2^s\cdots), [P_3]^s, \dots, [P_3]^s\big) = x_1^s$.
We first claim $f^s\big((x_2^s\cdots),(x_3^s\cdots), [P_3]^s, \dots, [P_3]^s\big) = x_2^s$.
Suppose that it is not true. Then, by Claim 3, we have $f^s\big((x_2^s\cdots),(x_3^s\cdots), [P_3]^s, \dots, [P_3]^s\big) = x_3^s$.
Since $x_1^s \approx x_2^s$ and $x_2^s \approx x_3^s$,
we have $[\hat{P}_1]^s, [\tilde{P}_1]^s\in [\mathbb{D}]^s$ such that
$r_1([\hat{P}_1]^s) = r_2([\tilde{P}_1]^s) = x_1^s$, $r_1([\tilde{P}_1]^s) = r_2([\hat{P}_1]^s) = x_2^s$ and
$r_{\ell}([\hat{P}_1]^s) = r_{\ell}([\tilde{P}_1]^s)$ for all $\ell \in \{3, \dots, |A^s|\}$, and
$[\hat{P}_2]^s, [\tilde{P}_2]^s \in [\mathbb{D}]^s$ such that
$r_1([\hat{P}_2]^s) = r_2([\tilde{P}_2]^s) = x_2^s$,
$r_1([\tilde{P}_2]^s) = r_2([\hat{P}_2]^s) = x_3^s$ and
$r_{\ell}([\hat{P}_2]^s) = r_{\ell}([\tilde{P}_2]^s)$ for all $\ell \in \{3, \dots, |A^s|\}$.
Clearly, $f^s([\hat{P}_1]^s,[\hat{P}_2]^s, [P_3]^s, \dots, [P_3]^s) = x_1^s$ and
$f^s([\tilde{P}_1]^s,[\tilde{P}_2]^s, [P_3]^s, \dots, [P_3]^s) = x_3^s$. Then, strategy-proofness implies
$f^s([\hat{P}_1]^s,[\tilde{P}_2]^s, [P_3]^s, \dots, [P_3]^s) = f^s([\hat{P}_1]^s,[\hat{P}_2]^s, [P_3]^s, \dots, [P_3]^s) = x_1^s$ and
$f^s([\hat{P}_1]^s,[\tilde{P}_2]^s, [P_3]^s, \dots, [P_3]^s) =f^s([\tilde{P}_1]^s,[\tilde{P}_2]^s, [P_3]^s, \dots, [P_3]^s) = x_3^s$ - a contradiction.
Therefore, it is true that $f^s\big((x_2^s\cdots),(x_3^s\cdots), [P_3]^s, \dots, [P_3]^s\big) = x_2^s$.
Following the path $(x_2^s, \dots, x_v^s)$ and applying a symmetric argument step by step,
we eventually have $f^s\big((x_{v-1}^s\cdots),(x_v^s\cdots), [P_3]^s, \dots, [P_3]^s\big) = x_{v-1}^s$.
Thus, since $r_1([P_3]^s) = x_v^s$, we know $f^s\big((x_{v-1}^s\cdots), [P_3]^s, [P_3]^s, \dots, [P_3]^s\big) = x_{v-1}^s$.
Consequently, by referring to the strategy-proof marginal rule $h_1^s$
which behaves like a dictatorship on $\langle a_{\underline{k}}^s, a_{\overline{k}}^s\rangle$,
we have $h_1^s\big((x_{v-1}^s\cdots), [P_3]^s\big) = f^s\big((x_{v-1}^s\cdots), [P_3]^s, [P_3]^s, \dots, [P_3]^s\big) = x_{v-1}^s$, which indicates voter $1$'s dictatorship on $\langle a_{\underline{k}}^s, a_{\overline{k}}^s\rangle$ at $h_1^s$.
Hence, by Claim 1, voter $1$ also dictates on $\langle a_{\underline{k}}^s, a_{\overline{k}}^s\rangle$ at $f^s$.
This completes the verification of the claim, and hence proves the Lemma
\end{proof}


\begin{lemma}\label{lem:tops-only}
For all $n \geq 1$, every strategy-proof marginal rule $f^s: \big[[\mathbb{D}]^s\big]^n \rightarrow A^s$ satisfies the tops-only property.\footnote{To simplify the proof, we bring the case of a single voter into consideration.}
\end{lemma}

\begin{proof}
It is clear that by unanimity, every one-voter marginal SCF on $[\mathbb{D}]^s$ satisfies the tops-only property.
Next, we fix $n \geq 2$ and introduce the following induction hypothesis.\medskip

\noindent
\textsc{Induction Hypothesis}: For all $1 \leq n' < n$, every strategy-proof marginal rule $f^{s\,'}: \big[[\mathbb{D}]^s\big]^{n'} \rightarrow A^s$ satisfies the tops-only property.\medskip

\noindent
Henceforth, we fix an $n$-voter strategy-proof marginal rule $f^s: \big[[\mathbb{D}]^s\big]^n \rightarrow A^s$, and show that $f^s$ satisfies the tops-only property.


First, by Lemma \ref{lem:behavelikeadictatorship}, we know that $f^s$ behaves like a dictatorship on $\langle a_{\underline{k}}^s, a_{\overline{k}}^s\rangle$.
Moreover, we can identify $1 \leq p \leq \underline{k}$ and $\overline{k} \leq q \leq |A^s|$ such that
$f^s$ behaves like a dictatorship on $\langle a_p^s, a_q^s\rangle$, and $f^s$ does not behave like a dictatorship on any interval $\langle a_{p'}^s, a_{q'}^s\rangle$ such that $p' \leq p$, $q' \geq q$ and $q'-p'> q-p$.
Henceforth, we assume w.l.o.g.~that voter $1$ dictates on $\langle a_{p}^s, a_{q}^s\rangle$ at $f^s$.
We induce a two-voter marginal rule:
for all $[P_1]^s, [P_2]^s \in [\mathbb{D}]^s$,
$h^s([P_1]^s, [P_2]^s) = f^s([P_1]^s, [P_2]^s, [P_2]^s, \dots, [P_2]^s)$.
Note that if $n = 2$, then $h^s = f^s$.
It is clear that $h^s$ inherits unanimity and strategy-proofness from $f^s$, and hence is a strategy-proof marginal rule.
Moreover, by construction, it is also true that voter $1$ dictates on $\langle a_p^s, a_q^s\rangle$ at $h^s$.\medskip


\noindent
\textsc{Claim 1}: For all $[P]^s \coloneqq ([P_1]^s, \dots, [P_n]^s) \in \big[[\mathbb{D}]^s\big]^n$,
we have $\big[r_1([P_1]^s) \in \langle a_p^s, a_q^s\rangle\big] \Rightarrow \big[f^s([P]^s) = r_1([P_1]^s)\big]$.
\medskip

Suppose not, i.e., there exists $[P]^s \in \big[[\mathbb{D}]^s\big]^n$ such that $r_1([P_1]^s) = a_r^s \in \langle a_p^s, a_q^s\rangle$ and $f^s([P]^s) = a_k^s \neq a_r^s$.
Clearly, by strategy-proofness, we have $h^s\big([P_1]^s, (a_k^s\cdots)\big) = f^s\big([P_1]^s, (a_k^s\cdots), \dots, (a_k^s\cdots)\big) = a_k^s$.
There are three cases: (i) $a_k^s \in \langle a_p^s, a_q^s\rangle$,
(ii) $a_k^s \in \langle a_1^s, a_{p-1}^s\rangle$ and
(iii) $a_k^s \in \langle a_{q+1}^s, a_{|A^s|}^s\rangle$.
In each case, we induce a contradiction.

In case (i), by voter 1's dictatorship on $\langle a_p^s, a_q^s\rangle$,
we have $h^s\big([P_1]^s, (a_k^s\cdots)\big) = a_r^s \neq a_k^s$ - a contradiction.
Cases (ii) and (iii) are symmetric; we focus on case (ii).
Recall that $|\langle a_p^s, a_q^s\rangle| \geq 3$, and
$G_{\approx}^{A^s}$ is a combination of the line $(a_1^s, \dots, a_{\underline{k}}^s)$,
the connected subgraph $G_{\approx}^{\langle a_{\underline{k}}^s,\, a_{\overline{k}}^s\rangle}$ and
the line $(a_{\overline{k}}^s, \dots, a_{|A^s|}^s)$.
Then, we have a path $(x_1^s, \dots, x_v^s)$ in $G_{\approx}^{A^s}$ such that
$x_1^s, x_2^s \in \langle a_p^s, a_q^s\rangle$, $a_r^s \in \{x_1^s, x_2^s\}$ and $x_v^s = a_k^s$.
By voter 1's dictatorship on $\langle a_p^s, a_q^s\rangle$, we first have $h^s\big((x_1^s\cdots), (x_2^s\cdots)\big) = x_1^s$.
Then, according to the path $(x_1^s, \dots, x_v^s)$, by statement (i) of Lemma \ref{lem:fundamental},
we have $h^s\big((x_1^s\cdots), (x_v^s\cdots)\big) = x_1^s$ and $h^s\big((x_2^s\cdots), (x_v^s\cdots)\big) = x_2^s$,
which imply $h^s\big((a_r^s\cdots), (a_k^s\cdots)\big) = a_r^s$ and hence
$h^s\big([P_1]^s, (a_k^s\cdots)\big) = a_r^s \neq a_k^s$ - a contradiction.
This completes the verification of the claim.
\medskip

\noindent
\textsc{Claim 2}: Given $[P]^s \in \big[[\mathbb{D}]^s\big]^n$,
let $r_1([P_1]^s) = a_k^s$.
The following two statements hold:
\begin{itemize}
\item[\rm (i)] if $a_k^s \in \langle a_1^s, a_{p-1}^s\rangle$, then $f^s([P]^s) \in \langle a_k^s, a_p^s\rangle$, and

\item[\rm (ii)] if $a_k^s \in \langle a_{q+1}^s, a_{|A^s|}^s\rangle$, then $f^s([P]^s) \in \langle a_q^s, a_k^s\rangle$.
\end{itemize}


The two statements are symmetric, and we focus on verifying statement (i).
Suppose not, i.e.,
either $f^s([P]^s) \coloneqq a_r^s \in \langle a_1^s, a_{k-1}^s\rangle$, or $f^s([P]^s) \coloneqq a_r^s \in \langle a_{p+1}^s, a_{|A^s|}^s\rangle$ holds.
If $a_r^s \in \langle a_{p+1}^s, a_{|A^s|}^s\rangle$,
by hybridness on $\prec^s$ w.r.t.~$a_{\underline{k}}^s$ and $a_{\overline{k}}^s$,
it is true that $a_p^s\mathrel{[P_1]^s}a_r^s$.
Consequently, since $f^s\big((a_p^s\cdots), [P_{-1}]^s\big) = a_p^s$ by Claim 1, voter $1$ will manipulate at $[P]^s$ via some induced marginal preference with the peak $a_p^s$.

Thus, we have $f^s([P_1]^s, [P_{-1}]^s)=a_r^s \in \langle a_1^s, a_{k-1}^s\rangle$.
Immediately, strategy-proofness implies $f^s\big((a_r^s\cdots), [P_{-1}]^s\big) = a_r^s$.
Since $a_r^s \approx a_{r+1}^s$, we have $[\hat{P}_1]^s, [\hat{P}_1']^s \in [\mathbb{D}]^s$ such that $r_1([\hat{P}_1]^s) = r_2([\hat{P}_1']^s) = a_r^s$, $r_1([\hat{P}_1']^s) = r_2([\hat{P}_1]^s) = a_{r+1}^s$ and $r_{\ell}([\hat{P}_1]^s) = r_1([\hat{P}_1']^s)$ for all $\ell = 3, \dots, |A^s|$.
Clearly, $f^s([\hat{P}_1]^s, [P_{-1}]^s) = a_r^s$.
Then, strategy-proofness implies $f^s([\hat{P}_1']^s, [P_{-1}]^s) \in \{a_r^s, a_{r+1}^s\}$.
Suppose $f^s([\hat{P}_1']^s, [P_{-1}]^s) = a_r^s$.
Immediately, strategy-proofness implies
$h^s\big([\hat{P}_1']^s, (a_r^s\cdots)\big)
 = f^s\big([\hat{P}_1']^s, (a_r^s\cdots), \dots, (a_r^s\cdots)\big)=a_r^s$,
which by $a_r^s \approx a_{r+1}^s$ further implies $h^s\big((a_{r+1}^s\cdots), (a_r^s\cdots)\big) =a_r^s$.
Then,
by statement (ii) of Lemma \ref{lem:fundamental},
according to the path $(a_r^s, \dots, a_k^s, \dots, a_p^s)$ in $G_{\approx}^{A^s}$,
we have $f^s\big((a_p^s\cdots), (a_r^s\cdots), \dots, (a_r^s\cdots)\big)=h^s\big((a_p^s\cdots), (a_r^s\cdots)\big) =a_r^s$,
which contradicts Claim 1.
Therefore, $f^s([\hat{P}_1']^s, [P_{-1}]^s) = a_{r+1}^s$, and
hence $f^s\big((a_{r+1}^s\cdots), [P_{-1}]^s\big) = a_{r+1}^s$ by strategy-proofness.
Following the path $(a_{r+1}^s, \dots, a_k^s)$ in $G_{\approx}^{A^s}$ step by step, by repeatedly applying the argument above,
we eventually have $f^s\big((a_k^s\cdots), [P_{-1}]^s\big) = a_k^s$, which contradicts the hypothesis $f^s([P_1]^s, [P_{-1}]^s)=a_r^s$.
Therefore, we have $f^s([P]^s) \in \langle a_k^s, a_p^s\rangle$.
This completes the verification of the claim.
\medskip



\noindent
\textsc{Claim 3}: Given $([P_1]^s, [P_{-1}]^s), ([P_1']^s, [P_{-1}]^s) \in \big[[\mathbb{D}]^s\big]^n$,
if $r_1([P_1]^s) = r_1([P_1']^s)$,
then $f^s([P_1]^s, [P_{-1}]^s)=f^s([P_1']^s, [P_{-1}]^s)$.\medskip

Let $r_1([P_1]^s) = r_1([P_1']^s) = a_k^s$.
Clearly, either $a_k^s \in \langle a_1^s, a_{p-1}^s\rangle$, or $a_k^s \in \langle a_p^s, a_q^s\rangle$, or $a_k^s \in \langle a_{q+1}^s, a_{|A^s|}^s\rangle$ holds.
This claim follows from Claim 1 if $a_k^s \in \langle a_p^s, a_q^s\rangle$.
The rest of two cases are symmetric, and we hence focus on the case $a_k^s \in \langle a_1^s, a_{p-1}^s\rangle$.
Suppose by contradiction $f^s\big([P_1]^s, [P_{-1}]^s\big) \neq f^s\big([P_1']^s, [P_{-1}]^s\big)$.
For notational convenience, let $f^s\big([P_1]^s, [P_{-1}]^s\big) = a_o^s$ and
$f^s\big([P_1']^s, [P_{-1}]^s\big) = a_r^s$.
Clearly, by strategy-proofness, $a_k^s \notin \{a_o^s, a_r^s\}$.
Since $a_k^s \in \langle a_1^s, a_{p-1}^s\rangle$, statement (i) of Claim 2 implies
$a_o^s, a_r^s \in \langle a_k^s, a_p^s\rangle$.
Thus, we have $a_k^s \prec^s a_o^s \prec^s a_r^s \preccurlyeq^s a_p^s$ or $a_k^s \prec^s a_r^s \prec^s a_o^s \preccurlyeq^s a_p^s$.
If $a_k^s \prec^s a_o^s \prec^s a_r^s \preccurlyeq^s a_p^s$,
by hybridness on $\prec^s$ w.r.t.~$a_{\underline{k}}^s$ and $a_{\overline{k}}^s$,
we know $a_o^s \mathrel{[P_1']^s} a_r^s$.
Consequently, voter $1$ will manipulate at $([P_1']^s, [P_{-1}]^s)$ via $[P_1]^s$.
Symmetrically, if $a_k^s \prec^s a_r^s \prec^s a_o^s \preccurlyeq^s a_p^s$,
we know $a_r^s \mathrel{[P_1]^s} a_o^s$.
Consequently, voter $1$ will manipulate at $([P_1]^s, [P_{-1}]^s)$ via $[P_1']^s$.
This completes the verification of the claim.\medskip

\noindent
\textsc{Claim 4}: Given $i \in N\backslash \{1\}$ and $([P_i]^s, [P_{-i}]^s),
([P_i']^s, [P_{-i}]^s) \in \big[[\mathbb{D}]^s\big]^n$,
if $r_1([P_i]^s) = r_1([P_i']^s)$, then $f^s([P_i]^s, [P_{-i}]^s)=f^s([P_i']^s, [P_{-i}]^s)$.\medskip

Let $r_1([P_i]^s) = r_1([P_i']^s) = a_k^s$ and $r_1([P_1]^s) = a_r^s$.
Clearly, either $a_r^s \in \langle a_1^s, a_{p-1}^s\rangle$, or $a_r^s \in \langle a_p^s, a_q^s\rangle$, or $a_r^s \in \langle a_{q+1}^s, a_{|A^s|}^s\rangle$ holds.
This claim follows from Claim 1 if $a_r^s \in \langle a_p^s, a_q^s\rangle$.
The rest of two cases are symmetric, and we hence focus on the case $a_r^s \in \langle a_1^s, a_{p-1}^s\rangle$.
Suppose by contradiction that $f^s([P_i]^s, [P_{-i}]^s) \neq f^s([P_i']^s, [P_{-i}]^s)$.
For notational convenience, let $f^s([P_i]^s, [P_{-i}]^s) = a_u^s$ and
$f^s([P_i']^s, [P_{-i}]^s) = a_v^s$.
Since $a_r^s \in \langle a_1^s, a_{p-1}^s\rangle$, statement (i) of Claim 2 implies $a_u^s, a_v^s \in \langle a_r^s, a_p^s\rangle$.
We assume w.l.o.g.~that $a_r^s \preccurlyeq^s a_u^s \prec^s a_v^s \preccurlyeq^s a_p^s$.
Accordingly, there are three subcases: (i) $a_k^s \preccurlyeq^s a_u^s$, (ii) $a_v^s \preccurlyeq^s a_k^s$ and (iii) $a_u^s \prec^s a_k^s \prec^s a_v^s$.
In each case, we induce a manipulation.
In subcase (i), by hybridness on $\prec^s$ w.r.t.~$a_{\underline{k}}^s$ and $a_{\overline{k}}^s$,
we know $a_u^s\mathrel{[P_i']^s} a_v^s$.
Consequently, voter $i$ will manipulate at $([P_i']^s, [P_{-i}]^s)$ via $[P_i]^s$.
In subcase (ii), by hybridness on $\prec^s$ w.r.t.~$a_{\underline{k}}^s$ and $a_{\overline{k}}^s$,
we know $a_v^s\mathrel{[P_i]^s} a_u^s$.
Consequently, voter $i$ will manipulate at $([P_i]^s, [P_{-i}]^s)$ via $[P_i']^s$.

For subcase (iii), we consider two situations: $n = 2$ and $n>2$.
First, let $n=2$. Thus, $i =2$ and $f^s([P_1]^s, [P_2']^s) = a_v^s$.
Since $a_r^s\preccurlyeq^s a_u^s \prec^s a_k^s \prec^s a_v^s \preccurlyeq a_p^s$,
by hybridness on $\prec^s$ w.r.t.~$a_{\underline{k}}^s$ and $a_{\overline{k}}^s$,
we have $a_k^s\mathrel{[P_1]^s}a_v^s$.
Consequently, since $f^s\big((a_k^s\cdots), [P_2']^s\big) = a_k^s$ by unanimity,
voter $1$ will manipulate at $([P_1]^s, [P_2']^s)$ via some induced marginal preference with the peak $a_k^s$.
Last, let $n>2$.
We first combine voters $1$ and $i$ as one and induce an $(n-1)$-voter marginal SCF:
$g^s([\hat{P}_1]^s, [\hat{P}_{-\{1, i\}}]^s) = f^s([\hat{P}_1]^s,[\hat{P}_1]^s, [\hat{P}_{-\{1, i\}}]^s)$
for all $[\hat{P}_1]^s \in [\mathbb{D}]^s$ and $[\hat{P}_{-\{1, i\}}]^s \in \big[[\mathbb{D}]^s\big]^{n-2}$.
It is evident that $g^s$ inherits unanimity and strategy-proofness from $f^s$, and hence by the induction hypothesis satisfies the tops-only property.
Hence, $f^s([P_i]^s, [P_i]^s, [P_{-\{1,i\}}]^s) = g^s([P_i]^s, [P_{-\{1,i\}}]^s)
= g^s([P_i']^s, [P_{-\{1,i\}}]^s) = f^s([P_i']^s, [P_i']^s, [P_{-\{1,i\}}]^s)$.
Henceforth, for notational convenience, let $f^s([P_i]^s, [P_i]^s, [P_{-\{1,i\}}]^s)
=f^s([P_i']^s, [P_i']^s, [P_{-\{1,i\}}]^s) =a_w^s$.
At $([P_i]^s, [P_i]^s, [P_{-\{1,i\}}]^s)$,
referring to voter $1$,\footnote{At $([P_i]^s, [P_i]^s, [P_{-\{1,i\}}]^s)$,
the first induced marginal preference $[P_i]^s$ is reported by voter $i$, while the second $[P_i]^s$ is reported by voter $1$.}
since $r_1([P_i]^s) = a_k^s \prec^s a_v^s \preccurlyeq^s a_p^s$,
statement (i) of Claim 2 implies
$a_w^s \in \langle a_k^s, a_p^s\rangle$.
Thus, we have $a_k^s \preccurlyeq^s a_w^s \prec^s a_v^s$,
or $a_w^s = a_v^s$, or $a_v^s \prec^s a_w^s \preccurlyeq^s a_p^s$.
If $a_k^s \preccurlyeq^s a_w^s \prec^s a_v^s$, we have $a_r^s \prec^s a_k^s \preccurlyeq^s a_w^s \prec^s a_v^s \preccurlyeq^s a_p^s$, and
then by hybridness on $\prec^s$ w.r.t.~$a_{\underline{k}}^s$ and $a_{\overline{k}}^s$, $a_w^s\mathrel{[P_1]^s}a_v^s$.
Consequently, voter $1$ will manipulate at $([P_i']^s, [P_1]^s, [P_{-\{1,i\}}]^s)$ via $[P_i']^s$.
If $a_w^s= a_v^s$, by strategy-proofness, $f^s([P_i]^s, [P_1]^s, [P_{-\{1,i\}}]^s) = a_u^s$ and $f^s([P_i']^s, [P_1]^s, [P_{-\{1,i\}}]^s) = a_v^s = a_w^s$ imply $a_u^s\mathrel{[P_i]^s}a_w^s$.
Consequently, voter $1$ will manipulate at $([P_i]^s, [P_i]^s, [P_{-\{1,i\}}]^s)$ via $[P_1]^s$.
If $a_v^s \prec^s a_w^s \preccurlyeq^s a_p^s$, we have $a_k^s \prec^s a_v^s \prec^s a_w^s  \preccurlyeq^s a_p^s$, and
then by hybridness on $\prec^s$ w.r.t.~$a_{\underline{k}}^s$ and $a_{\overline{k}}^s$, $a_v^s\mathrel{[P_i']^s}a_w^s$.
Consequently, voter $1$ will manipulate at $([P_i']^s, [P_i']^s, [P_{-\{1,i\}}]^s)$ via $[P_1]^s$.
This completes the verification of the claim.\medskip

Overall, by Claims 1, 3 and 4, we know $f^s([P_i]^s, [P_{-i}]^s) = f^s([P_i']^s, [P_{-i}]^s)$
 for all $i \in N$, $[P_i]^s, [P_i']^s \in [\mathbb{D}]^s$ with $r_1([P_i]^s) = r_1([P_i']^s)$ and
$[P_{-i}]^s \in \big[[\mathbb{D}]^s\big]^{n-1}$.
This implies that $f^s$ satisfies the tops-only property.
This completes the verification of the induction hypothesis, and hence proves the Lemma.
\end{proof}

\begin{lemma}\label{lem:constrainedFBR}
A marginal SCF $f^s: \big[[\mathbb{D}]^s\big]^n \rightarrow A^s$ is a strategy-proof marginal rule
if and only if it is an $(\underline{x}^s, \overline{x}^s)$-FBR.
\end{lemma}

\begin{proof}
The sufficiency part of the Lemma follows from the sufficiency part of Theorem 2 of \citet{CRSSZ2022}.
By the proof of the necessity part of Theorem 2 of \citet{CRSSZ2022} (specifically, their Lemmas 8, 9, 12 and 13), we first know that on $[\mathbb{D}]^s$,
every tops-only and strategy-proof marginal rule that behaves like a dictatorship on $\langle \underline{x}^s, \overline{x}^s\rangle$,
is an $(\underline{x}^s, \overline{x}^s)$-FBR.
Recall that in Lemma \ref{lem:tops-only} and its proof, we have shown that every strategy-proof marginal rule defined on $[\mathbb{D}]^s$ satisfies the tops-only property and behaves like a dictatorship on $\langle \underline{x}^s, \overline{x}^s\rangle$.
Then, we can directly apply their result to complete the proof.
This proves statement (ii) of Proposition \ref{prop:FBR}.
\end{proof}



\section{An example of a rich multidimensional domain}\label{app:anexample}

Let $A = A^1 \times A^2$ where $A^1 = \{1,2,3\}$ and $A^2 = \{0,1\}$.
Let $\prec^1$ and $\prec^2$ be the natural linear orders over $A^1$ and $A^2$ respectively and
$\prec = \prec^1 \times \prec^2$ (see Figure \ref{fig:grid2}).
We fix two thresholds $\underline{x} = (1,0)$ and $\overline{x} = (3,0)$.
A domain $\mathbb{D}$ of 30 preferences that are multidimensional hybrid on $\prec$ w.r.t.~$\underline{x}$ and $\overline{x}$ is specified in Table \ref{tab:AMH}.
The graph $G_{\sim/\sim^+}^{\mathbb{D}}$ is specified in Figure \ref{fig:adjacency+}.



\begin{figure}[t]
\begin{center}
  \includegraphics[width=0.35\textwidth]{fig3.jpg}
\end{center}
\vspace{-2em}
\caption{The Cartesian product of linear orders $\prec= \prec^1 \times \prec^2$}\label{fig:grid2}
\end{figure}


\begin{table}[h]
{\scriptsize
\begin{tabular}{cccccccccccccccc}
  $P_1$   & $P_2$   & $P_3$   & $P_4$   & $P_5$   & $P_6$   & $P_7$   & $P_8$   & $P_9$   & $P_{10}$& $P_{11}$& $P_{12}$& $P_{13}$& $P_{14}$\\
  $(1,0)$ & $(1,0)$ & $(1,0)$ & $(1,0)$ & $(1,0)$ & $(1,0)$ & $(2,0)$ & $(2,0)$ & $(2,0)$ & $(2,0)$ & $(3,0)$ & $(3,0)$ & $(3,0)$ & $(3,0)$ \\
  $(2,0)$ & $(3,0)$ & $(3,0)$ & $(3,0)$ & $(1,1)$ & $(1,1)$ & $(1,0)$ & $(1,0)$ & $(2,1)$ & $(2,1)$ & $(1,0)$ & $(1,0)$ & $(3,1)$ & $(3,1)$ \\
  $(3,0)$ & $(2,0)$ & $(2,0)$ & $(1,1)$ & $(3,0)$ & $(3,0)$ & $(3,0)$ & $(2,1)$ & $(1,0)$ & $(1,0)$ & $(2,0)$ & $(3,1)$ & $(1,0)$ & $(1,0)$ \\
  $(1,1)$ & $(1,1)$ & $(1,1)$ & $(2,0)$ & $(2,0)$ & $(3,1)$ & $(2,1)$ & $(3,0)$ & $(3,0)$ & $(1,1)$ & $(3,1)$ & $(2,0)$ & $(2,0)$ & $(1,1)$ \\
  $(2,1)$ & $(2,1)$ & $(3,1)$ & $(3,1)$ & $(3,1)$ & $(2,0)$ & $(1,1)$ & $(1,1)$ & $(1,1)$ & $(3,0)$ & $(1,1)$ & $(1,1)$ & $(1,1)$ & $(2,0)$ \\
  $(3,1)$ & $(3,1)$ & $(2,1)$ & $(2,1)$ & $(2,1)$ & $(2,1)$ & $(3,1)$ & $(3,1)$ & $(3,1)$ & $(3,1)$ & $(2,1)$ & $(2,1)$ & $(2,1)$ & $(2,1)$ \\
  &&&&&&&&&&&&&\\
  $P_{15}$& $P_{16}$& $P_{17}$& $P_{18}$& $P_{19}$& $P_{20}$& $P_{21}$& $P_{22}$& $P_{23}$& $P_{24}$& $P_{25}$& $P_{26}$& $P_{27}$& $P_{28}$& $P_{29}$& $P_{30}$ \\
  $(1,1)$ & $(1,1)$ & $(1,1)$ & $(1,1)$ & $(2,1)$ & $(2,1)$ & $(2,1)$ & $(2,1)$ & $(2,1)$ & $(2,1)$ & $(3,1)$ & $(3,1)$ & $(3,1)$ & $(3,1)$ & $(3,1)$ & $(3,1)$  \\
  $(1,0)$ & $(1,0)$ & $(3,1)$ & $(3,1)$ & $(2,0)$ & $(2,0)$ & $(2,0)$ & $(2,0)$ & $(3,1)$ & $(3,1)$ & $(3,0)$ & $(3,0)$ & $(1,1)$ & $(1,1)$ & $(1,1)$ & $(2,1)$  \\
  $(3,1)$ & $(3,1)$ & $(1,0)$ & $(2,1)$ & $(1,1)$ & $(1,1)$ & $(3,1)$ & $(3,1)$ & $(2,0)$ & $(1,1)$ & $(1,1)$ & $(1,1)$ & $(3,0)$ & $(2,1)$ & $(2,1)$ & $(1,1)$  \\
  $(3,0)$ & $(2,1)$ & $(2,1)$ & $(1,0)$ & $(1,0)$ & $(3,1)$ & $(1,1)$ & $(1,1)$ & $(1,1)$ & $(2,0)$ & $(1,0)$ & $(2,1)$ & $(2,1)$ & $(3,0)$ & $(3,0)$ & $(3,0)$  \\
  $(2,1)$ & $(3,0)$ & $(3,0)$ & $(3,0)$ & $(3,1)$ & $(1,0)$ & $(1,0)$ & $(3,0)$ & $(3,0)$ & $(3,0)$ & $(2,1)$ & $(1,0)$ & $(1,0)$ & $(1,0)$ & $(2,0)$ & $(2,0)$  \\
  $(2,0)$ & $(2,0)$ & $(2,0)$ & $(2,0)$ & $(3,0)$ & $(3,0)$ & $(3,0)$ & $(1,0)$ & $(1,0)$ & $(1,0)$ & $(2,0)$ & $(2,0)$ & $(2,0)$ & $(2,0)$ & $(1,0)$ & $(1,0)$
\end{tabular}
\caption{Domain $\mathbb{D}$}\label{tab:AMH}
}
\end{table}

Note that $\mathbb{D}$ contains both separable preferences (e.g., $P_1$) and non-separable preferences (e.g., $P_2$).
Indeed, $\mathbb{D}$ does not contain all multidimensional hybrid preferences on $\prec$ w.r.t.~$\underline{x} =(1,0)$ and $\overline{x} =(3,0)$, e.g.,
$P_i =  (1,1)_{\rightharpoonup}(2,1)_{\rightharpoonup}(3,1)_{\rightharpoonup}(1,0)_{\rightharpoonup}(2,0)_{\rightharpoonup}(3,0)$
is multidimensional hybrid on $\prec$ w.r.t.~$\underline{x}$ and $\overline{x}$,
but is not included in $\mathbb{D}$.
It is true that $\mathbb{D}$ is a multidimensional hybrid domain:
(i) all preferences of $\mathbb{D}$ are multidimensional hybrid on $\prec$ w.r.t.~$\underline{x}$ and $\overline{x}$, and
(ii) for each $s\in M$, the induced marginal domain $[\mathbb{D}]^s=\mathbb{P}^s$.


\begin{figure}[t]
\includegraphics[width=1.1\textwidth]{fig4.jpg}
\vspace{-2em}
\caption{The graph $G_{\sim/\sim^+}^{\mathbb{D}}$\protect\footnotemark}\label{fig:adjacency+}
\end{figure}

\footnotetext{In the graph $G_{\sim/\sim^+}^{\mathbb{D}}$,
for instance, the symbol ``$P_1 \frac{\overline{(2,0)}}{~~\underline{(3,0)}~~} P_2 \hspace{-1.25cm}\rule[-2.2mm]{0.11mm}{0.62cm} \hspace{0.57cm}\rule[-2.2mm]{0.11mm}{0.62cm}$~~~~~\;''
represents that \vspace{0.4em}
$P_1 \sim P_2$, $(2,0)\mathrel{P_1}(3,0)$ and $(3,0)\mathrel{P_2}(2,0)$,
while the symbol ``$P_6 \frac{\underline{~~\overline{(1,0)}~\overline{(3,0)}~\overline{(2,0)}~~}}{~~\underline{(1,1)}~\underline{(3,1)}~\underline{(2,1)}~~} P_{15} \hspace{-2.785cm}\rule[-2.2mm]{0.11mm}{0.67cm} \hspace{0.57cm}\rule[-2.2mm]{0.11mm}{0.67cm}
\hspace{0.11cm}\rule[-2.2mm]{0.11mm}{0.67cm} \hspace{0.56cm}\rule[-2.2mm]{0.11mm}{0.67cm}
\hspace{0.12cm}\rule[-2.2mm]{0.11mm}{0.67cm} \hspace{0.565cm}\rule[-2.2mm]{0.11mm}{0.67cm}
$~~~~~~\;''
represents that $P_6 \sim^+ P_{15}$, \vspace{0.4em}
$(1,0)\mathrel{P_6}(1,1)$, $(1,1)\mathrel{P_{15}}(1,0)$,
$(3,0)\mathrel{P_6}(3,1)$, $(3,1)\mathrel{P_{15}}(3,0)$,
$(2,0)\mathrel{P_6}(2,1)$ and $(2,1)\mathrel{P_{15}}(2,0)$.
}



Next, we show that $\mathbb{D}$ is a rich domain.
It is evident that $\mathbb{D}$ is minimally rich and satisfies diversity\textsuperscript{+} (see $P_1$ and $P_{30}$).
Notice that the following six paths in $G_{\sim/\sim^+}^{\mathbb{D}}$
indicate the Interior\textsuperscript{+} property:
\begin{itemize}
\item $(P_1, P_2,P_3,P_4,P_5,P_6)$, where each preference has the peak $(1,0)$,
\item $(P_7, P_8,P_9,P_{10})$, where each preference has the peak $(2,0)$,
\item $(P_{11}, P_{12},P_{13},P_{14})$, where each preference has the peak $(3,0)$,
\item $(P_{15}, P_{16},P_{17},P_{18})$, where each preference has the peak $(1,1)$,
\item $(P_{19}, P_{20},P_{21},P_{22},P_{23},P_{24})$, where each preference has the peak $(2,1)$, and
\item $(P_{25}, P_{26},P_{27},P_{28}, P_{29}, P_{30})$, where each preference has the peak $(3,1)$.
\end{itemize}
To explain the Exterior\textsuperscript{+} property, we make an important observation on Figure \ref{fig:adjacency+}:
each pair of distinct preferences is connected by three distinct paths in $G_{\sim/\sim^+}^{\mathbb{D}}$, and
whenever we spot a restoration on a path connecting two preferences,
we can immediately identify another path connecting these two preferences that has no such a restoration.
For instance, between $P_1$ and $P_{23}$,
the clockwise path $(P_1, P_2,P_3,P_4,P_5,P_6,P_{15}, P_{16},P_{17},P_{18},P_{28}, P_{29}, P_{30},P_{24},P_{23})$
has $\{(2,0), (1,1)\}$-restoration, i.e., $(2,0)\mathrel{P_3}(1,1)$, $(1,1)\mathrel{P_4}(2,0)$ and $(2,0)\mathrel{P_{23}}(1,1)$,
while on the counter-clockwise path $(P_1, P_7,P_8,P_9,P_{10},P_{19}, P_{20},P_{21},P_{22},P_{23})$,
$(2,0)$ and $(1,1)$ have not been locally switched.
Therefore, we can conclude that given arbitrary $P_i, P_i' \in \mathbb{D}$ such that $r_1(P_i) \neq r_1(P_i')$ and distinct $a,b\in A$,
there exists a path in $G_{\sim/\sim^+}^{\mathbb{D}}$ connecting $P_i$ and $P_i'$ that has no $\{a,b\}$-restoration.
To complete the verification,
we show the no-detour condition of the Exterior\textsuperscript{+} property:
given $P_i, P_i' \in \mathbb{D}$ and distinct $a,b \in A$ such that
$r_1(P_i) \neq r_1(P_i')$ and $r_1(P_i), r_1(P_i') , a,b \in (A^s,x^{-s})$ for some $s \in M$ and $x^{-s} \in A^{-s}$,
there exists a path $\pi = (P_{i|1}, \dots, P_{i|v})$ in $G_{\sim/\sim^+}^{\mathbb{D}}$
connecting $P_i$ and $P_i'$ such that $\pi$ has no $\{a,b\}$-restoration, and $r_1(P_{i|k}) \in (A^s,x^{-s})$ for all $k=1, \dots, v$.
First, by removing the edge between $P_1$ and $P_7$, the edge between $P_3$ and $P_{11}$, the edge between $P_{18}$ and $P_{28}$ and the edge between $P_{24}$ and $P_{30}$ in Figure \ref{fig:adjacency+},
we make the following three observations on the three remaining independent paths:
\begin{itemize}
\item[\rm (i)] the path $(P_1, P_2,P_3,P_4,P_5,P_6,P_{15}, P_{16},P_{17},P_{18})$ has no $\{(1,0),(1,1)\}$-restoration, and all preferences on this path have peaks in $(1, A^2)=\{(1,0),(1,1)\}$,

\item[\rm (ii)] the path $(P_7, P_8,P_9,P_{10},P_{15}, P_{20},P_{21},P_{22},P_{23}, P_{24})$ has no $\{(2,0),(2,1)\}$-restoration, and all preferences on this path have peaks in $(2, A^2) = \{(2,0),(2,1)\}$, and

\item[\rm (iii)] the path $(P_{11}, P_{12}, P_{13}, P_{14},P_{25}, P_{26},P_{27},P_{28},P_{29}, P_{30})$ has no $\{(3,0),(3,1)\}$-restoration, and
    all preferences on this path have peaks in $(3, A^2) = \{(3,0),(3,1)\}$.
\end{itemize}
This implies that for each $q \in \{1,2,3\} = A^1$,
given $P_i, P_i' \in \mathbb{D}$ and distinct $a,b \in A$
such that $r_1(P_i) \neq r_1(P_i')$ and $r_1(P_i), r_1(P_i') , a,b \in (q,A^2) = \{(q, 0), (q,1)\}$,
there exists a path $\pi = (P_{i|1}, \dots, P_{i|v})$ in $G_{\sim/\sim^+}^{\mathbb{D}}$
connecting $P_i$ and $P_i'$ such that $\pi$ has no $\{a,b\}$-restoration, and $r_1(P_{i|k}) \in (q,A^2)$ for all $k=1, \dots, v$, as required.
Similarly, we turn to the preferences that have peaks in $(A^1, 0) = \{(1,0),(2,0), (3,0)\}$,
which are $P_1, \dots, P_{14}$, and the preferences that have peaks in $(A^1, 1) = \{(1,1),(2,1), (3,1)\}$,
which are $P_{15}, \dots, P_{30}$.
By removing the edge between $P_{14}$ and $P_{15}$,
the edge between $P_{6}$ and $P_{15}$ and the edge between $P_{10}$ and $P_{19}$ in Figure \ref{fig:adjacency+}, we cut $G_{\sim/\sim^+}^{\mathbb{D}}$ into two disjoint connected subgraphs: the left subgraph is $G_{\sim/\sim^+}^{\{P_1, \dots, P_{14}\}}$, while
the right one is $G_{\sim/\sim^+}^{\{P_{15}, \dots, P_{30}\}}$.
Note that each path in $G_{\sim/\sim^+}^{\{P_1, \dots, P_{14}\}}$ has no restoration for any pair of alternatives $a,b \in (A^1, 0)$,
and similarly each path in $G_{\sim/\sim^+}^{\{P_{15}, \dots, P_{30}\}}$ has no restoration for any pair of alternatives $a,b \in (A^1, 1)$.
This meets the requirement of the no-detour condition as well.
In conclusion, $\mathbb{D}$ is a rich domain.



\section{Proof of Theorem \ref{thm}}\label{app:theorem}

Let $\mathbb{D}$ be a rich domain.
We first show that on $\mathbb{D}$ investigated in both the sufficiency and necessity parts of the Theorem, every strategy-proof rule satisfies the tops-only property.

\begin{lemma}\label{lem:tops-onlydomain}
For all $n \geq 2$, every strategy-proof rule $f: \mathbb{D}^n \rightarrow A$ satisfies the tops-only property.
\end{lemma}

\begin{proof}
To show the Lemma, we adopt Proposition 2 of \citet{CZ2019},
which says that on a domain of top-separable preferences
satisfying their Interior\textsuperscript{+} and Exterior\textsuperscript{+} properties,
every strategy-proof rule satisfies the tops-only property.
Hence, by the richness of $\mathbb{D}$, to complete the verification, it suffices to show that all preferences are top-separable.

First, let $\mathbb{D}$ be a rich domain investigated in the sufficiency part of the Theorem.
Thus, $\mathbb{D}$ is also a multidimensional hybrid domain on $\prec$ w.r.t.~some thresholds $\underline{x}$ and $\overline{x}$.
Hence, we have $\mathbb{D} \subseteq \mathbb{D}_{\textrm{MH}}(\prec, \underline{x}, \overline{x}) \subseteq \mathbb{D}_{\textrm{TS}}$, as required.

Next, let $\mathbb{D}$ be a rich domain investigated in the necessity part of the Theorem.
Thus, $\mathbb{D}$ is a decomposable domain.
Given arbitrary $P_i \in \mathbb{D}$, say $r_1(P_i) = x$, and similar $a,b \in A$, say $M(a,b) = \{s\}$, let $a^s = x^s$. We show $a\mathrel{P_i}b$ via strategy-proofness of some constructed SCF.
Let $N = \{1,2\}$.
At the component $s$, we construct a marginal SCF $f^s: \big[[\mathbb{D}]^s\big]^2 \rightarrow A^s$ called
\emph{the voter-$1$ marginal dictatorship},\footnote{Formally, a marginal SCF $f^s: \big[[\mathbb{D}]^s\big]^n \rightarrow A^s$ is \textbf{the
voter-$i$ marginal dictatorship} if
$f^s([P]^s) = r_1([P_i]^s)$ for all $[P]^s \in \big[[\mathbb{D}]^s\big]^n$.
Clearly, the voter-$i$ marginal dictatorship is a strategy-proof marginal rule.\label{footnote:marginaldictatorship}}
while at each $t \in M\backslash \{s\}$, we refer to the voter-$2$ marginal dictatorship $f^t: \big[[\mathbb{D}]^t\big]^2 \rightarrow A^t$.
Clearly, every marginal SCF here is a strategy-proof marginal rule.
Accordingly, we assemble an SCF $f: \mathbb{D}^n \rightarrow A$ such that for all $P \in \mathbb{D}^n$,
$f(P_1, P_2) = \big(f^1([P_1]^1, [P_2]^1), \dots, f^m([P_1]^m, [P_2]^m)\big)$.
Since $\mathbb{D}$ is a decomposable domain,
the assembled SCF $f$ is a strategy-proof rule.
Given $P_1,P_1',P_2 \in \mathbb{D}$ such that $P_1 = P_i$, $P_2 \in \mathbb{D}^b$ and $P_1'=P_2$, we by construction have
$f(P_1,P_2) = (x^s, b^{-s}) = a$ and $f(P_1',P_2) = b$.
Then, strategy-proofness of $f$ implies $a\mathrel{P_1}b$, as required.
This hence proves the Lemma.
\end{proof}

For the next lemma, we introduce the notion of strong connectedness\textsuperscript{+} between alternatives.
Intuitively, given two adjacent\textsuperscript{+} preferences that disagree on peaks,
we say that the two peaks are strongly connected\textsuperscript{+}.
Formally, $a,b \in A$ are said \textbf{strongly connected\textsuperscript{+}}, denoted $a \approx^+ b$,
if there exist $P_i \in \mathbb{D}^a$ and $P_i' \in \mathbb{D}^b$ such that $P_i \sim^+ P_i'$.
Accordingly, given a nonempty subset $B \subseteq A$,
we can induce a graph $G_{\approx^+}^B \coloneqq \langle B, \mathcal{E}_{\approx^+}^B\rangle$
such that two alternatives of $B$ form an edge if and only if they are strongly connected\textsuperscript{+}.
For instance, according to the domain of Appendix \ref{app:anexample},
the graph $G_{\approx^+}^A$ is specified in Figure \ref{fig:strongconnectedness+}.


\begin{figure}[t]
\begin{center}
  \includegraphics[width=0.45\textwidth]{fig5.jpg}
\end{center}
\vspace{-2em}
\caption{The graph $G_{\approx^+}^A$}\label{fig:strongconnectedness+}
\end{figure}

The next lemma explores the structure of $G_{\approx^+}^{(A^s,\, x^{-s})}$ for each $s \in M$ and
$x^{-s} \in A^{-s}$, and reveals some important implication of the Exterior\textsuperscript{+} property according to $G_{\approx^+}^{(A^s,\, x^{-s})}$.

\begin{lemma}\label{lem:path-connectedness}
Given $s \in M$ and $x^{-s} \in A^{-s}$, the following two statements hold:
\begin{itemize}
\item[\rm (i)] Graph $G_{\approx^+}^{(A^s,\, x^{-s})}$ is a connected graph.

\item[\rm (ii)] Given $a, b, c \in (A^s, x^{-s})$ such that $a \neq c$ and $b \neq c$,
if $b$ is included in all paths in $G_{\approx^+}^{(A^s,\, x^{-s})}$ connecting $a$ and $c$,
we have $b\mathrel{P_i}c$ for all $P_i \in \mathbb{D}^a$.
\end{itemize}
\end{lemma}

\begin{proof}
We first show statement (i).
Given distinct $a, b\in (A^s, x^{-s})$, we construct a path in $G_{\approx^+}^{(A^s,\, x^{-s})}$ connecting $a$ and $b$. First, by minimal richness, we have $P_i \in \mathbb{D}^a$ and $P_i' \in \mathbb{D}^b$.
Next, by the Exterior\textsuperscript{+} property, there exists a path $\pi =  (P_{i|1}, \dots, P_{i|v})$ in $G_{\sim/\sim^+}^{\mathbb{D}}$ connecting $P_i$ and $P_i'$ such that
$r_1(P_{i|k}) \in (A^s, x^{-s})$ for all $k = 1, \dots, v$.
Furthermore, we partition the path $\pi$ according to preference peaks (without changing the orders of preferences in $\pi$):
\begin{align*}
\left(\frac{~P_{i|1}, \dots, P_{i|k_1}~}{\textrm{peak $x_1$}}, \dots,
\frac{~P_{i|k_{p-1}+1}, \dots, P_{i|k_p}~}{\textrm{peak $x_p$}},
\frac{~P_{i|k_p+1}, \dots, P_{i|k_{p+1}}~}{\textrm{peak $x_{p+1}$}},\dots,
\frac{~P_{i|k_{q-1}+1}, \dots, P_{i|k_q}~}{\textrm{peak $x_q$}}\right),
\end{align*}
where $q \geq 2$, $k_0 = 0$, $k_q = v$,
$r_1(P_{i|k_{p-1}+1}) = \dots = r_1(P_{i|k_p}) = x_p$ for all $p =1, \dots, q$, and
$x_p \neq x_{p+1}$ for all $p=1, \dots, q-1$.
Thus, we have a sequence of alternatives $(x_1, \dots, x_q)$.
\medskip

\noindent
\textsc{Claim 1}:
We have $x_p \approx^+ x_{p+1}$ for all $1\leq p < q$.\medskip

Given $1\leq p < q$, we have the preferences $P_{i|k_p}$ and $P_{i|k_p+1}$ and the peaks $r_1(P_{i|k_p}) = x_p$ and $r_1(P_{i|k_p+1}) = x_{p+1}$.
Since $x_p, x_{p+1} \in (A^s, x^{-s})$, we write $x_p = (x^s, x^{-s})$ and $x_{p+1} = (y^s, x^{-s})$ where $x^s \neq y^s$.
Clearly, either $P_{i|k_p} \sim P_{i|k_p+1}$ or $P_{i|k_p} \sim^+ P_{i|k_p+1}$ holds.
If $P_{i|k_p} \sim^+ P_{i|k_p+1}$, it is evident that $x_p \approx^+ x_{p+1}$.
We complete the verification by ruling out the possibility that $P_{i|k_p} \sim P_{i|k_p+1}$.
Suppose by contradiction that $P_{i|k_p} \sim P_{i|k_p+1}$.
Thus, $(x^s, x^{-s})$ and $(y^s, x^{-s})$ are the unique two alternatives that are oppositely ranked across $P_{i|k_p}$ and $P_{i|k_p+1}$.
However, since $P_{i|k_p}$ and $P_{i|k_p+1}$ are shown to be top-separable in the proof of Lemma \ref{lem:tops-onlydomain}, we have
$(x^s, z^{-s})\mathrel{P_{i|k_p}}(y^s, z^{-s})$ and $(y^s, z^{-s})\mathrel{P_{i|k_p+1}}(x^s, z^{-s})$ for all $z^{-s} \in A^{-s}$ - a contradiction.
This completes the verification of the claim.\medskip

Note that some alternatives may appear multiple times in the sequence $(x_1, \dots, x_q)$.
For instance, let $x_p = x_{p'}$ where $1 \leq p < p' \leq q$.
Since $x_p\neq x_{p+1}$, it is clear that $p+1< p'$.
We then refine the sequence $(x_1, \dots, x_p, x_{p+1}, \dots, x_{p'}, x_{p'+1}, \dots, x_q)$
to $(x_1, \dots, x_p, x_{p'+1}, \dots, x_q)$,
where any consecutive alternatives in the refined sequence remain to be strongly connected\textsuperscript{+}.
By repeatedly refining the sequence, we eventually eliminate repetitions of alternatives, and construct a path in $G_{\approx^+}^{(A^s,\, x^{-s})}$ connecting $a$ and $b$.
This proves statement (i).

Next, we show statement (ii).
Suppose that it is not true. Thus, there exists $P_i \in \mathbb{D}^a$ such that $c\mathrel{P_i}b$.
Fixing a preference $P_i' \in \mathbb{D}^c$ by minimal richness,
by the Exterior\textsuperscript{+} property,
we have a path $\pi =  (P_{i|1}, \dots, P_{i|v})$ connecting $P_i$ and $P_i'$ such that $\pi$ has no $\{c,b\}$-restoration, and
$r_1(P_{i|k}) \in (A^s, x^{-s})$ for all $k = 1, \dots, v$.
Since $c\mathrel{P_i}b$ and $c\mathrel{P_i'}b$,
no $\{c,b\}$-restoration on $\pi$ implies $c\mathrel{P_{i|k}}b$ for all $k = 1, \dots, v$, and hence $r_1(P_{i|k}) \neq b$ for all $k = 1, \dots, v$.
Then, according to $\pi$, by the proof of statement (i),
we can elicit a path in $G_{\approx^+}^{(A^s,\, x^{-s})}$ connecting $a$ and $c$ that excludes $b$ - a contradiction.
\end{proof}


Furthermore, we fix an arbitrary strategy-proof rule $f: \mathbb{D}^n \rightarrow A$, and partially characterize $f$ in the next two lemmas using the graphs $G_{\approx^+}^{(A^s,\,x^{-s})}$ for all $s\in M$ and $x^{-s} \in A^{-s}$.
Clearly, by Lemma \ref{lem:tops-onlydomain}, $f$ satisfies the tops-only property.
For notational convenience, we henceforth write $(a, P_{-i})$ to denote a preference profile where voter $i$ reports an arbitrary preference with the peak $a$, and all other voters report the preferences $P_1, \dots, P_{i-1}, P_{i+1}, \dots, P_n$ respectively.
Moreover, for ease of presentation, given $a \in A$, $P_i \in \mathbb{D}^a$ and $s \in M$,
let $r_1(P_i)^s \coloneqq a^s$;
given $P \in \mathbb{D}$, $f(P) = a$ and $s \in M$, let $f(P)^s \coloneqq a^s$.

%Note that since the adjacent\textsuperscript{+} preferences $P_i$ and $P_i'$ here are separable preferences, it is true that $a$ and $b$ must be similar alternatives.



\begin{lemma}\label{lem:step1}
Given $i \in N$, $P_i, P_i' \in \mathbb{D}$ and $P_{-i} \in \mathbb{D}^{n-1}$,
let $M\big(r_1(P_i), r_1(P_i')\big) = \{s\}$.
We have $f(P_i, P_{-i})^t = f(P_i', P_{-i})^t$ for all $t \in M\backslash \{s\}$.
\end{lemma}

\begin{proof}
For notational convenience,
let $r_1(P_i) = (a^s, z^{-s})$ and $r_1(P_i') = (b^s, z^{-s})$,
where $a^s \neq b^s$.
By statement (i) of Lemma \ref{lem:path-connectedness},
we have a path $(a_1, \dots, a_v)$
in $G_{\approx^+}^{(A^s,\, z^{-s})}$ connecting $(a^s, z^{-s})$ and $(b^s, z^{-s})$.
For ease of presentation, let $f(a_k, P_{-i}) = x_k$ for all $k = 1, \dots, v$.
Clearly, $f(P_i, P_{-i}) = f(a_1, P_{-i}) = x_1$ and $f(P_i', P_{-i})=f(a_v, P_{-i})=x_v$.
To complete the verification, it suffices to show $x_k^{-s} = x_{k+1}^{-s}$ for all $k = 1, \dots, v-1$.

Given $1 \leq k < v$, we have $f(a_k, P_{-i}) = x_k$ and $f(a_{k+1}, P_{-i}) = x_{k+1}$.
The result holds evidently if $x_k = x_{k+1}$.
Next, assume $x_k \neq x_{k+1}$.
Since $a_k \approx^+ a_{k+1}$, there exist $\hat{P}_i \in \mathbb{D}^{a_k}$ and $\hat{P}_i' \in \mathbb{D}^{a_{k+1}}$ such that $\hat{P}_i \sim^{+} \hat{P}_i'$.
Since $f(\hat{P}_i, P_{-i}) = f(a_k, P_{-i}) = x_k$ and
$f(\hat{P}_i', P_{-i}) = f(a_{k+1}, P_{-i}) = x_{k+1}$,
strategy-proofness implies $x_k\mathrel{\hat{P}_i}x_{k+1}$ and $x_{k+1}\mathrel{\hat{P}_i'}x_k$.
Since $\hat{P}_i \sim^{+} \hat{P}_i'$, note that any two alternatives that are oppositely ranked across $\hat{P}_i$ and $\hat{P}_i'$, agree on all components other than $s$.
Hence, it must be the case that $x_k^{-s} = x_{k+1}^{-s}$, as required.
\end{proof}

\begin{lemma}\label{lem:step2}
Given $i \in N$, $P_i, P_i' \in \mathbb{D}$, $P_{-i} \in \mathbb{D}^{n-1}$ and $t \in M$,
let $r_1(P_i)^t= r_1(P_i')^t$.
We have $f(P_i, P_{-i})^t = f(P_i', P_{-i})^t$.
\end{lemma}

\begin{proof}
The Lemma immediately follows from the tops-only property if $r_1(P_i) = r_1(P_i')$.
Henceforth, let $r_1(P_i) \neq r_1(P_i')$.
Assume w.l.o.g.~that $M\big(r_1(P_i), r_1(P_i')\big) = \{1, \dots, s\}$, where $1\leq s<m$.
We write $r_1(P_i) = (a^1, \dots, a^s, z^{\{s+1, \dots, m\}})$ and
$r_1(P_i') = (b^1, \dots, b^s, z^{\{s+1, \dots, m\}})$ where $a^k \neq b^k$ for all $k=1, \dots, s$.
Clearly, $r_1(P_i)^t= r_1(P_i')^t$ implies $s< t \leq m$.
%Clearly, if $t =0$, then $r_1(P_i) = r_1(P_i')$ and hence the tops-only property implies $x^s = y^s$ for all $s \in M$.
We construct the alternative $x_{k} = (b^1, \dots, b^k, a^{k+1}, \dots, a^s, z^{\{s+1, \dots, m\}})$ for each $k = 0, 1, \dots, s$.
Thus, $x_0 = r_1(P_i)$ and
$x_s = r_1(P_i')$.
Furthermore, by minimal richness,
we fix a preference $P_{i|k} \in \mathbb{D}^{x_k}$ for each $k = 0, 1, \dots, s$.
For each $k=0,1, \dots, s-1$, since $t \notin M\big(r_1(P_i^k),r_1(P_i^{k+1})\big)$,
Lemma \ref{lem:step1} implies $f(P_{i|k}, P_{-i})^t = f(P_{i|k+1}, P_{-i})^t$.
Therefore, we have $f(P_i, P_{-i})^t = f(P_{i|0}, P_{-i})^t =\dots  =f(P_{i|s}, P_{-i})^t = f(P_i', P_{-i})^t$.
\end{proof}


Now, we are ready to prove the Theorem.\medskip


\noindent
\textbf{(Sufficiency Part)}~
Let $\mathbb{D}$ be a rich multidimensional domain on $\prec$ w.r.t.~the thresholds $\underline{x}$ and $\overline{x}$.
We show that $\mathbb{D}$ is a decomposable domain.
First, we fix a strategy-proof rule $f: \mathbb{D}^n \rightarrow A$, and show that $f$ is decomposable and all marginal SCFs are strategy-proof marginal rules.
Of course, $f$ satisfies the tops-only property by Lemma \ref{lem:tops-onlydomain} and triggers Lemma \ref{lem:step2}.

%We first introduce some new notation that would simplify the proof of the Theorem.
%A \textbf{voting scheme} is a mapping $g: A^n \rightarrow A$ that associates to each $n$-tuple profile of alternatives $(a_1, \dots, a_n) \in A^n$, an alternative $g(a_1, \dots, a_n) \in A$ (it is \emph{not} necessary that $g(a_1, \dots, a_n) = a_i$ for some $i \in N$), and satisfies an additional mild condition that is analogous to the axiom of unanimity imposed on an SCF: $g(a, \dots, a) = a$ for all $a \in A$.\footnote{Analogously, given $s \in M$, a \textbf{marginal voting scheme}
%is a mapping $g^s: [A^s]^n \rightarrow A^s$ that satisfies an additional condition: $g^s(a^s, \dots, a^s) = a^s$ for all $a^s \in A^s$.}
%%A voting scheme $g: A^n \rightarrow A$ is decomposable if there exists a marginal voting scheme $g^s: [A^s]^n \rightarrow A^s$ for each component $s \in M$ such that for all profile $(a_1, \dots, a_n) \in A^n$, we have $\big[g(a_1, \dots, a_n) = a\big] \Leftrightarrow \big[g^s(a_1^s, \dots, a_n^s) = a^s\; \textrm{for all}\; s \in M\big]$.
%Given a domain $\mathbb{D}$, a voting scheme $g: A^n \rightarrow A$ is said strategy-proof on $\mathbb{D}$ if for all $i \in N$, $P_i, P_i' \in \mathbb{D}$ and $a_{-i} \in A^{n-1}$, we have
%$\big[g\big(r_1(P_i), a_{-i}\big) \neq g\big(r_1(P_i'), a_{-i}\big)\big] \Rightarrow
%\big[g\big(r_1(P_i), a_{-i}\big) P_i g\big(r_1(P_i'), a_{-i}\big)\big]$.\footnote{Analogously,
%given $s \in M$, a marginal voting scheme $g^s: [A^s]^n \rightarrow A^s$ is said strategy-proof on $[\mathbb{D}]^s$ if for all $i \in N$, $[P_i]^s, [P_i']^s \in [\mathbb{D}]^s$ and $a_{-i}^s \coloneqq (a_1^s, \dots, a_{i-1}^s, a_{i+1}^s, \dots, a_n^s) \in [A^s]^{n-1}$, we have
%$\Big[g^s\big(r_1([P_i]^s), a_{-i}^s\big) \neq g^s\big(r_1([P_i']^s), a_{-i}^s\big)\Big] \Rightarrow
%\Big[g^s\big(r_1([P_i]^s), a_{-i}^s\big) [P_i]^s g^s\big(r_1([P_i']^s), a_{-i}^s\big)\Big]$.}
%It is clear that a tops-only and strategy-proof rule $f: \mathbb{D}^n \rightarrow A$ is a voting scheme that is strategy-proof on the domain $\mathbb{D}$.\footnote{Analogously, given $s \in M$, a tops-only and strategy-proof marginal rule $f^s: \big[[\mathbb{D}]^s\big]^n \rightarrow A^s$ is a marginal voting scheme that is strategy-proof on the induced marginal domain $[\mathbb{D}]^s$.}
%Conversely, given a domain $\mathbb{D}$ and a voting scheme $g: A^n \rightarrow A$ which is strategy-proof on $\mathbb{D}$, we can induce an SCF $f: \mathbb{D}^n \rightarrow A$ such that for all $(P_1, \dots, P_n) \in \mathbb{D}^n$, $f(P_1, \dots, P_n) = g\big(r_1(P_1), \dots, r_1(P_n)\big)$, which of course is a tops-only and strategy-proof rule.\footnote{Analogous, given $s \in M$, an induced marginal domain $[\mathbb{D}]^s$ and a marginal voting scheme $g^s: [A^s]^n \rightarrow A^s$ which is strategy-proof on $[\mathbb{D}]^s$, we can induce a marginal SCF $f^s: \big[[\mathbb{D}]^s\big]^n \rightarrow A^s$ such that for all $([P_1]^s, \dots, [P_n]^s) \in [\mathbb{D}^s]^n$, $f^s([P_1]^s, \dots, [P_n]^s) = g^s\big(r_1([P_1]^s), \dots, r_1([P_n]^s)\big)$, which of course is a tops-only and strategy-proof marginal rule.}


%For notational convenience, we write $(a, P_{-i})$ to denote a preference profile where voter $i$ reports an arbitrary preference with the peak $a$, and all other voters report the preferences $P_1, \dots, P_{i-1}, P_{i+1}, \dots, P_n$ respectively.
%Moreover, for ease of presentation, given $a \in A$, $P_i \in \mathbb{D}^a$ and $s \in M$,
%let $r_1(P_i)^s \coloneqq a^s$; given $P \in \mathbb{D}$, $f(P) = a$ and $s \in M$, let $f(P)^s \coloneqq a^s$.
%The proof consists of the following five lemmas.
%
%
%\medskip
%
%\begin{lemma}\label{lem:step1}
%Given $i \in N$, $P_i, P_i' \in \mathbb{D}$ and $P_{-i} \in \mathbb{D}^{n-1}$,
%let $M\big(r_1(P_i), r_1(P_i')\big) \coloneqq \{s\}$.
%We have $f(P_i, P_{-i})^t = f(P_i', P_{-i})^t$ for all $t \in M\backslash \{s\}$.
%\end{lemma}
%
%\begin{proof}
%Let $r_1(P_i) \coloneqq (a^s, z^{-s})$ and $r_1(P_i') \coloneqq (b^s, z^{-s})$,
%where $a^s \neq b^s$.
%Then, by path-connectedness\textsuperscript{$\ast$}, we have a path $(a_1, \dots, a_v)$
%in $G_{\approx^+}^{(A^s, z^{-s})}$ connecting $(a^s, z^{-s})$ and $(b^s, z^{-s})$.
%For notational convenience, let $f(a_k, P_{-i}) \coloneqq x_k$ for all $k = 1, \dots, v$.
%Clearly, $f(P_i, P_{-i}) = f(a_1, P_{-i}) = x_1$ and $f(P_i', P_{-i})=f(a_v, P_{-i})=x_v$.
%Thus, to complete the verification, it suffices to show $x_k^{-s} = x_{k+1}^{-s}$ for all $k = 1, \dots, v-1$.
%
%Given $1 \leq k < v$, we have $f(a_k, P_{-i}) = x_k$ and $f(a_{k+1}, P_{-i}) = x_{k+1}$.
%The result holds evidently if $x_k = x_{k+1}$.
%Next, assume $x_k \neq x_{k+1}$.
%Since $a_k \approx^+ a_{k+1}$, there exist $\hat{P}_i \in \mathbb{D}^{a_k}$ and $\hat{P}_i' \in \mathbb{D}^{a_{k+1}}$ such that $\hat{P}_i \sim^{\ast} \hat{P}_i'$.
%By the definition of adjacency\textsuperscript{+},
%note that any two alternatives that are oppositely ranked across $\hat{P}_i$ and $\hat{P}_i'$, must agree on all components other than $s$.
%Now, given $f(\hat{P}_i, P_{-i}) = f(a_k, P_{-i}) = x_k$ and
%$f(\hat{P}_i', P_{-i}) = f(a_{k+1}, P_{-i}) = x_{k+1}$,
%by strategy-proofness,
%we know that $x_k$ and $x_{k+1}$ are oppositely ranked across $\hat{P}_i$ and $\hat{P}_i'$, i.e., $x_k\hat{P}_ix_{k+1}$ and $x_{k+1}\hat{P}_i'x_k$.
%Hence, we have $x_k^{-s} = x_{k+1}^{-s}$, as required.
%\end{proof}
%
%\begin{lemma}\label{lem:step2}
%Given $i \in N$, $P_i, P_i' \in \mathbb{D}$, $P_{-i} \in \mathbb{D}^{n-1}$ and $t \in M$,
%let $r_1(P_i)^t= r_1(P_i')^t$.
%We have $f(P_i, P_{-i})^t = f(P_i', P_{-i})^t$.
%\end{lemma}
%
%\begin{proof}
%Note that if $M\big(r_1(P_i), r_1(P_i')\big) = \emptyset$, then $r_1(P_i) = r_1(P_i')$, and the result immediately follows from the tops-only property.
%Henceforth, let $M\big(r_1(P_i), r_1(P_i')\big) \neq \emptyset$.
%Assume w.l.o.g.~that $M\big(r_1(P_i), r_1(P_i')\big) \coloneqq \{1, \dots, s\}$, where $1\leq s<m$.
%Thus, let $r_1(P_i) \coloneqq (a^1, \dots, a^s, z^{\{s+1, \dots, m\}})$ and
%$r_1(P_i') \coloneqq (b^1, \dots, b^s, z^{\{s+1, \dots, m\}})$ where $a^k \neq b^k$ for all $k=1, \dots, s$.
%Clearly, since $r_1(P_i)^t= r_1(P_i')^t$, it is true that $s< t \leq m$.
%%Clearly, if $t =0$, then $r_1(P_i) = r_1(P_i')$ and hence the tops-only property implies $x^s = y^s$ for all $s \in M$.
%We construct the alternative $x_{k} \coloneqq (b^1, \dots, b^k, a^{k+1}, \dots, a^s, z^{\{s+1, \dots, m\}})$ for each $k = 0, 1, \dots, s$.
%Thus, $x_0 = r_1(P_i)$ and
%$x_s = r_1(P_i')$.
%Furthermore, by minimal richness,
%we fix a preference $P_{i|k} \in \mathbb{D}^{x_k}$ for each $k = 0, 1, \dots, s$.
%Note that for each $k=0,1, \dots, s-1$, we have $t \notin M\big(r_1(P_i^k),r_1(P_i^{k+1})\big)$, and hence $f(P_{i|k}, P_{-i})^t = f(P_{i|k+1}, P_{-i})_{k+1}^t$ by Lemma \ref{lem:step1}.
%Therefore, we have $f(P_i, P_{-i})^t = f(P_{i|0}, P_{-i})^t =\dots  =f(P_{i|s}, P_{-i})^t = f(P_i', P_{-i})^t$,
%where the first and the last equalities follow from the tops-only property.
%\end{proof}



\begin{lemma}\label{lem:decomposition}
SCF $f$ is decomposable, and each marginal SCF is a strategy-proof marginal rule.
\end{lemma}

\begin{proof}
The proof consists of three claims.\medskip

\noindent
\textsc{Claim 1}: Given $P, P' \in \mathbb{D}^n$ and $t \in M$,
let $r_1(P_i)^t= r_1(P_i')^t$ for all $i \in N$.
We have $f(P)^t = f(P')^t$.\medskip

We first construct the profile
$P(k) = (P_1', \dots, P_k', P_{k+1}, \dots, P_n)$ for each $k = 0, 1, \dots, n$.
Clearly, $P(0) = P$ and $P(n) = P'$.
For each $i = 1, \dots, n$, note that $P(i-1) = (P_1', \dots, P_{i-1}',P_i, P_{i+1}, \dots, P_n)$ and
$P(i)= (P_1', \dots, P_{i-1}',P_i', P_{i+1}, \dots, P_n)$ agree on preferences of all voters other than $i$, and $r_1(P_i)^t= r_1(P_i')^t$.
Then, Lemma \ref{lem:step2} implies $f(P(i-1))^t = f(P(i))^t$.
Hence,  we have $f(P)^t =f(P(0))^t = \dots =f(P(n))^t = f(P')^t$.
This completes the verification of the claim.\medskip

\noindent
\textsc{Claim 2}: SCF $f$ is decomposable.\medskip

Given a component $s \in M$,
by Claim 1 and minimal richness of $\mathbb{D}$,
we can construct a function $g^s: [A^s]^n \rightarrow A^s$
such that given arbitrary profile $(x_1^s, \dots, x_n^s) \in [A^s]^n$,
$g^s(x_1^s, \dots, x_n^s) = f(P_1, \dots, P_n)^s$
for all $(P_1, \dots, P_n) \in \mathbb{D}^n$ with $\big(r_1(P_1)^s, \dots, r_1(P_n)^s\big) = (x_1^s, \dots,  x_n^s)$.
Next,
according to $g^s$,
we construct a marginal SCF $f^s : \big[[\mathbb{D}]^s\big]^n \rightarrow A^s$ such that for all $([P_1]^s, \dots, [P_n]^s) \in \big[[\mathbb{D}]^s\big]^n$,
$f^s([P_1]^s, \dots, [P_n]^s) = g^s\big(r_1([P_1]^s), \dots, r_1([P_n]^s)\big)$.
Now, by construction,
it is clear that for all $(P_1, \dots, P_n) \in \mathbb{D}^n$,
$\big[f(P_1, \dots, P_n) = a\big] \Leftrightarrow \big[f^s([P_1]^s, \dots, [P_n]^s) = a^s\; \textrm{for all}\; s \in M\big]$.
Therefore, $f$ is decomposable.
This completes the verification of the claim.\medskip

\noindent
\textsc{Claim 3}: Given $s \in M$, the marginal SCF $f^s$ constructed in Claim 2 is a strategy-proof marginal rule.\medskip

First, we claim that $f^s$ is unanimous.
Given a profile $([P_1]^s, \dots, [P_n]^s) \in \big[[\mathbb{D}]^s\big]^n$,
let $r_1([P_1]^s) = \dots = r_1([P_n]^s) = a^s$.
We show $f^s([P_1]^s, \dots, [P_n]^s) =a^s$.
Given $z^{-s} \in A^{-s}$, by minimal richness, we have a profile $(P_1', \dots, P_n') \in \mathbb{D}^n$ such that $r_1(P_1') = \dots = r_1(P_n') = (a^s, z^{-s})$.
By unanimity of $f$,
it is clear that $f(P_1', \dots, P_n') = (a^s, z^{-s})$.
Then, by the decomposition of $f$,
we have $f^s([P_1]^s, \dots, [P_n]^s) = f(P_1', \dots, P_n')^s = a^s$, as required.

Last, we show strategy-proofness of $f^s$.
Given $i \in N$, $[P_i]^s ,[P_i']^s \in [\mathbb{D}]^s$ and $[P_{-i}]^s \in \big[[\mathbb{D}]^s\big]^{n-1}$,
let $f^s([P_i]^s, [P_{-i}]^s) = a^s$, $f^s([P_i']^s, [P_{-i}]^s) = b^s$ and $a^s \neq b^s$.
We show $a^s\mathrel{[P_i]^s}b^s$.
Since $f^s$ by construction satisfies the tops-only property, $f^s([P_i]^s, [P_{-i}]^s) \neq f^s([P_i']^s, [P_{-i}]^s)$ implies $r_1([P_i]^s) \neq r_1([P_i']^s)$.
For notational convenience, let $r_1(P_i) = (z^s, z^{-s})$ and $r_1([P_i']^s) = \hat{z}^s$.
By minimal richness, we fix $\hat{P}_i \in \mathbb{D}$ such that $r_1(\hat{P}_i)=(\hat{z}^s, z^{-s})$, and $\hat{P}_j \in \mathbb{D}$ such that $r_1(\hat{P}_j) = \big(r_1([P_j]^s), z^{-s}\big)$ for each $j \in N\backslash \{i\}$.
Thus, by the decomposition of $f$ and the tops-only property of $f^s$, we have
$f(P_i, \hat{P}_{-i})^s = f^s([P_i]^s, [\hat{P}_{-i}]^s)
= f^s([P_i]^s, [P_{-i}]^s) = a^s$ and
$f(\hat{P}_i, \hat{P}_{-i})^s = f^s([\hat{P}_i]^s, [\hat{P}_{-i}]^s)
= f^s([P_i']^s, [P_{-i}]^s) = b^s$, while
by the decomposition of $f$ and unanimity of marginal rules, we have
$f(P_i, \hat{P}_{-i})^t = f^t([P_i]^t, [\hat{P}_{-i}]^t)= z^t$ and
$f(\hat{P}_i, \hat{P}_{-i})^t = f^t([\hat{P}_i]^t, [\hat{P}_{-i}]^t)
= z^t$ for all $t \in M\backslash \{s\}$.
Thus,
$f(P_i, \hat{P}_{-i}) = (a^s, z^{-s})$ and $f(\hat{P}_i, \hat{P}_{-i}) = (b^s, z^{-s})$,
which by strategy-proofness imply $(a^s, z^{-s})\mathrel{P_i}(b^s, z^{-s})$, and hence $a^s\mathrel{[P_i]^s}b^s$, as required.
This completes the verification of the claim, and hence proves the Lemma.
\end{proof}

Next, we fix a strategy-proof marginal rule $f^s: \big[\mathbb{D}]^s\big]^n \rightarrow A^s$
for each $s \in M$,
assemble an SCF $f: \mathbb{D}^n \rightarrow A$: for all $P \in \mathbb{D}^n$,
$f(P) = \big(f^1([P]^1), \dots, f^m([P]^m)\big)$,
and show that $f$ is a strategy-proof rule.


\begin{lemma}\label{lem:assembling}
The assembled SCF $f$ is a strategy-proof rule.
\end{lemma}

\begin{proof}
Since all marginal SCFs $f^1, \dots, f^m$ satisfy unanimity,
by construction, it is evident that the assembled SCF $f$ is unanimous, and hence is a rule.
We henceforth focus on showing strategy-proofness of $f$.

Since $\mathbb{D}$ is a multidimensional domain on $\prec$ w.r.t.~the thresholds $\underline{x}$ and $\overline{x}$, we know that $\mathbb{D} \subseteq \mathbb{D}_{\textrm{MH}}(\prec, \underline{x}, \overline{x})$, and for each $s \in M$,
every induced marginal preference $[P_i]^s \in [\mathbb{D}]^s$ is hybrid on $\prec^s$ w.r.t.~$\underline{x}^s$ and $\overline{x}^s$.
Note that for each $s \in M$, $[\mathbb{D}_{\textrm{MH}}(\prec, \underline{x}, \overline{x})]^s$
contains all hybrid marginal preferences on $\prec^s$ w.r.t.~$\underline{x}^s$ and $\overline{x}^s$.
Thus, $[\mathbb{D}]^s\subseteq [\mathbb{D}_{\textrm{MH}}(\prec, \underline{x}, \overline{x})]^s$ for all $s\in M$.
By the necessity part of Proposition \ref{prop:FBR}, we first know that the marginal SCF $f^s: \big[[\mathbb{D}]^s\big]^n \rightarrow A^s$ is an FBR for each $s \in M\backslash M(\underline{x}, \overline{x})$, and the marginal SCF $f^t: \big[[\mathbb{D}]^t\big]^n \rightarrow A^t$ is an $(\underline{x}^t, \overline{x}^t)$-FBR for each $t\in M(\underline{x}, \overline{x})$.
For each $s \in M$, according to the fixed ballots $(b_J^s)_{J \subseteq N}$ of $f^s$,
we construct an FBR $\hat{f}^s: \big[[\mathbb{D}_{\textrm{MH}}(\prec, \underline{x}, \overline{x})]^s\big]^n \rightarrow A^s$, which by the sufficiency part of Proposition \ref{prop:FBR}
is clearly a strategy-proof marginal rule.
Then, we can assemble an SCF $\hat{f}: [\mathbb{D}_{\textrm{MH}}(\prec, \underline{x}, \overline{x})]^n \rightarrow A$: for all $P \in [\mathbb{D}_{\textrm{MH}}(\prec, \underline{x}, \overline{x})]^n$,
$\hat{f}(P) = \big(\hat{f}^1([P]^1), \dots, \hat{f}^m([P]^m)\big)$.
Since $\mathbb{D} \subseteq \mathbb{D}_{\textrm{MH}}(\prec, \underline{x}, \overline{x})$
and $[\mathbb{D}]^s \subseteq \big[\mathbb{D}_{\textrm{MH}}(\prec, \underline{x}, \overline{x})\big]^s$ for each $s \in M$, we know that
for all $(P_1, \dots, P_n) \in \mathbb{D}$,
\begin{align*}
f(P_1, \dots, P_n) = & ~\big(f^1([P_1]^1, \dots, [P_n]^1), \dots, f^m([P_1]^m, \dots, [P_n]^m)\big)\\
=&\left(\mathop{\mathop{\max}\nolimits^{\prec^1}}\limits_{J \subseteq N~~~\,}
\Big(\mathop{\mathop{\min}\nolimits^{\prec^1}}\limits_{i \in J~~~}\big(r_1([P_i]^1), b_J^1\big)\Big), \dots, \mathop{\mathop{\max}\nolimits^{\prec^m}}\limits_{J \subseteq N~~~\,}
\Big(\mathop{\mathop{\min}\nolimits^{\prec^m}}\limits_{i \in J~~~}\big(r_1([P_i]^m), b_J^m\big)\Big)\right)\\
= &~\big(\hat{f}^1([P_1]^1, \dots, [P_n]^1), \dots, \hat{f}^m([P_1]^m, \dots, [P_n]^m)\big) \\
= & ~\,\hat{f}(P_1, \dots, P_n).
\end{align*}
Therefore, if $\hat{f}$ is strategy-proof, $f$ immediately inherits strategy-proofness from $\widehat{f}$.

To complete the verification, we last show strategy-proofness of $\hat{f}$.
Given $i \in N$, $P_i, P_i' \in \mathbb{D}_{\textrm{MH}}(\prec, \underline{x}, \overline{x})$ and $P_{-i} \in [\mathbb{D}_{\textrm{MH}}(\prec, \underline{x}, \overline{x})]^{n-1}$,
let $\hat{f}(P_i,P_{-i}) = a$, $\hat{f}(P_i',P_{-i}) = b$ and $a \neq b$.
We show $a\mathrel{P_i}b$.
For notational convenience, let $r_1(P_i) = x$.
By the definition of multidimensional hybridness on $\prec$ w.r.t.~$\underline{x}$ and $\overline{x}$,
to show $a\mathrel{P_i}b$, it suffices to show that for all $s \in M(a,b)$, either $a^s = x^s$, or
$a^s \in \textrm{Int}\langle x^s, b^s\rangle$ and $a^s \notin \textrm{Int}\langle \underline{x}^s, \overline{x}^s\rangle$.
Given $s \in M(a,b)$,
we know either $a^s = x^s$ or $a^s \neq x^s$ holds.
Henceforth, we fix $a^s \neq x^s$, and
show $a^s \in \textrm{Int}\langle x^s, b^s\rangle$ and $a^s \notin \textrm{Int}\langle \underline{x}^s, \overline{x}^s\rangle$.
By the decomposition of $\hat{f}$ and strategy-proofness of $\hat{f}^s$,
we know $\hat{f}([P_i]^s, [P_{-i}]^s) = a^s$, $\hat{f}([P_i']^s, [P_{-i}]^s) = b^s$ and $a^s\mathrel{[P_i]^s} b^s$.
Since $\hat{f}^s$, as an FBR, satisfies the tops-only property,
we have $\hat{f}([\hat{P}_i]^s, [P_{-i}]^s) = a^s$ for all $[\hat{P}_i]^s \in [\mathbb{D}_{\textrm{MH}}(\prec, \underline{x}, \overline{x})]^s$ such that $r_1([\hat{P}_i]^s) = r_1([P_i]^s) = x^s$.
Consequently, strategy-proofness of $\hat{f}^s$ implies that
for all $[\hat{P}_i]^s \in [\mathbb{D}_{\textrm{MH}}(\prec, \underline{x}, \overline{x})]^s$,
$\big[r_1([\hat{P}_i]^s) = x^s\big] \Rightarrow \big[a^s\mathrel{[\hat{P}_i]^s} b^s\big]$.
Since $[\mathbb{D}_{\textrm{MH}}(\prec, \underline{x}, \overline{x})]^s$ contains all
hybrid marginal preferences on $\prec^s$ w.r.t.~$\underline{x}^s$ and $\overline{x}^s$,
we know that given $a^s \neq x^s$, if $a^s \notin \textrm{Int}\langle x^s, b^s\rangle$
or $a^s \in \textrm{Int}\langle \underline{x}^s, \overline{x}^s\rangle$,
there exists an induced marginal preference of $[\mathbb{D}_{\textrm{MH}}(\prec, \underline{x}, \overline{x})]^s$ such that $x^s$ is top-ranked, and $b^s$ is ranked above $a^s$.
Therefore,  it must be the case that $a^s \in \textrm{Int}\langle x^s, b^s\rangle$ and $a^s \notin \textrm{Int}\langle \underline{x}^s, \overline{x}^s\rangle$, as required.
This completes the verification of the Lemma, and hence proves the sufficiency part of the Theorem.
\end{proof}


\noindent
\textbf{(Necessity Part)}~
Let a rich domain $\mathbb{D}$ be a decomposable domain.
We identify two thresholds $\underline{x}$ and $\overline{x}$, and show that $\mathbb{D}$ is a multidimensional hybrid domain on $\prec$ w.r.t.~$\underline{x}$ and $\overline{x}$ (recall two conditions of Definition \ref{def:AMH}).
By diversity\textsuperscript{+},
we have two separable preferences $\underline{P}_i,\overline{P}_i \in \mathbb{D}$ that are complete reversals.
For each $s \in M$, according to the linear order $\prec^s$,
by relabeling elements as necessary, we can assume w.l.o.g.~that for all $a^s, b^s \in A^s$, $[a^s \prec^s b^s] \Rightarrow \big[a^s\mathrel{[\underline{P}_i]^s}b^s\; \textrm{and}\; b^s\mathrel{[\overline{P}_i]^s}a^s\big]$.


\begin{lemma}\label{lem:connectedgraph}
Given $s \in M$, according to $[\mathbb{D}]^s$,
the following two statements hold:
\begin{itemize}
\item[\rm (i)] Graph $G_{\approx}^{A^s}$ is a connected graph.

\item[\rm (ii)] Given $a^s, b^s, c^s \in A^s$ such that $a^s \neq c^s$ and $b^s \neq c^s$,
 if $b^s$ is included in all paths in $G_{\approx}^{A^s}$ connecting $a^s$ and $c^s$,
we have $b^s\mathrel{[P_i]^s}c^s$ for all $[P_i]^s \in [\mathbb{D}]^s$ with $r_1([P_i]^s) = a^s$.
\end{itemize}
\end{lemma}

\begin{proof}
We first show statement (i).
Given distinct $a^s, b^s \in A^s$, we construct a path in $G_{\approx}^{A^s}$ connecting $a^s$ and $b^s$.
Fixing arbitrary $x^{-s} \in A^s$, we have the alternatives $(a^s, x^{-s})$ and $(b^s, x^{-s})$.
Since $G_{\approx^+}^{(A^s, \, x^{-s})}$ is a connected graph by statement (i) of Lemma \ref{lem:path-connectedness},
we have a path $(x_1, \dots, x_v)$ in $G_{\approx^+}^{(A^s, \, x^{-s})}$ connecting
$(a^s, x^{-s})$ and $(b^s, x^{-s})$.
Clearly, $x_1^{-s} = \dots = x_v^{-s} = x^{-s}$ and $x_k^s \neq x_{k'}^s$ for all $1 \leq k<k' \leq v$.
Given arbitrary $1 \leq k < v$, since $x_k \approx^+ x_{k+1}$,
we have preferences $P_i \in \mathbb{D}^{x_k}$ and $P_i' \in \mathbb{D}^{x_{k+1}}$ such that $P_i \sim^+ P_i'$.
Immediately, we have $r_1([P_i]^s) = r_2([P_i']^s) = x_k^s$, $r_1([P_i']^s) = r_2([P_i]^s) = x_{k+1}^s$ and
$r_{\ell}([P_i]^s) = r_{\ell}([P_i']^s)$ for all $\ell \in \{3, \dots, |A^s|\}$,
which imply $x_k^s \approx x_{k+1}^s$. Then, we have a path $(x_1^s, \dots, x_v^s)$ in $G_{\approx}^{A^s}$ connecting $a^s$ and $b^s$, as required.
This proves statement (i).

Next, we show statement (ii).
Fixing arbitrary $P_i \in \mathbb{D}$ with $r_1([P_i]^s) = a^s$, we show $b^s\mathrel{[P_i]^s}c^s$.
Let $r_1(P_i) = (a^s, x^{-s})$.
Note that if $(b^s, x^{-s})\mathrel{P_i}(c^s,x^{-s})$, we immediately have $b^s\mathrel{[P_i]^s}c^s$.
Then, by statement (ii) of Lemma \ref{lem:path-connectedness}, it suffices to show that $(b^s, x^{-s})$ is included in all paths in $G_{\approx^+}^{(A^s,\,x^{-s})}$ connecting $(a^s, x^{-s})$ and $(c^s, x^{-s})$.
Suppose not, i.e., there exists a path $(x_1, \dots, x_v)$ in $G_{\approx^+}^{(A^s,\,x^{-s})}$ connecting $(a^s, x^{-s})$ and $(c^s, x^{-s})$ such that $x_k \neq (b^s, x^{-s})$ for all $k = 1, \dots, v$.
Clearly, $x_k^s \neq b^s$ for all $k =1, \dots, v$.
Consequently,
by the proof of statement (i),
we have the path $(x_1^s, \dots, x_v^s)$ in $G_{\approx}^{A^s}$ connecting $a^s$ and $c^s$ that
excludes $b^s$ - a contradiction.
\end{proof}

\begin{lemma}\label{lem:tops-onlymarginaldomain}
Given $s \in M$,
for all $n \geq 2$,
every strategy-proof marginal rule $f^s: \big[[\mathbb{D}]^s\big]^n \rightarrow A^s$ satisfies the tops-only property.
\end{lemma}

\begin{proof}
Suppose not, i.e., we have a non-tops-only and strategy-proof marginal rule $f^s: \big[[\mathbb{D}]^s\big]^n \rightarrow A^s$ for some $n \geq 2$.
Then, there must exist $i \in N$,
$[P_i]^s,[P_i']^s \in [\mathbb{D}]^s$ with $r_1([P_i]^s) = r_1([P_i']^s)$ and
$[P_{-i}]^s \in \big[[\mathbb{D}]^s\big]^{n-1}$ such that
$f^s([P_i]^s, [P_{-i}]^{s}) \neq f^s([P_i']^s, [P_{-i}]^{s})$.
For each component $t \in M\backslash \{s\}$, we fix a voter $j \in N\backslash \{i\}$ and construct the voter-$j$ marginal dictatorship $f^t: \big[[\mathbb{D}]^t\big]^n \rightarrow A^t$ (recall the formal definition in footnote \ref{footnote:marginaldictatorship}).
Then, we assemble an SCF $f: \mathbb{D}^n \rightarrow A$:
for all $P \in \mathbb{D}^n$, $f(P)=\big(f^1([P]^1), \dots, f^m([P]^m)\big)$.
Clearly, all marginal SCFs are strategy-proof marginal rules.
Since $\mathbb{D}$ is a decomposable domain, the assembled SCF $f$ is a strategy-proof rule.
Immediately, $f$ satisfies the tops-only property by Lemma \ref{lem:tops-onlydomain}, and then triggers Lemma \ref{lem:step2}.
On the one hand, by construction, we have $f(P_i, P_{-i})^s
= f^s\big([P_i]^s, [P_{-i}]^{s}\big) \neq f^s\big([P_i']^s, [P_{-i}]^{s}\big) = f(P_i', P_{-i})^s$.
On the other hand,
by Lemma \ref{lem:step2},
since $r_1(P_i)^s = r_1([P_i]^s) = r_1([P_i']^s) = r_1(P_i')^s$,
we have $f(P_i, P_{-i})^s = f(P_i', P_{-i})^s$ - a contradiction.
\end{proof}


Now, for each $s \in M$, we know that $G_{\approx}^{A^s}$ is a connected graph,
$[\mathbb{D}]^s$ contains $[\underline{P}_i]^s$ and $[\overline{P}_i]^s$ that are complete reversals, and $[\mathbb{D}]^s$ is a top-only marginal domain
(i.e., every strategy-proof marginal rule satisfies the tops-only property).
Then, by applying Corollary 2 of \citet{CZ2023},  for each $s \in M$, we can infer the following two conditions on $[\mathbb{D}]^s$:
\begin{description}
\item[\rm \textbf{Condition (i)}] there exist marginal thresholds $\underline{x}^s, \overline{x}^s \in A^s$ (either identical or not) such that all induced marginal preferences of $[\mathbb{D}]^s$ are hybrid on $\prec^s$ w.r.t.~$\underline{x}^s$ and $\overline{x}^s$, and

\item[\rm \textbf{Condition (ii)}] moreover if $\underline{x}^s \neq \overline{x}^s$,
there exist no linear order $\lhd^s$ over $A^s$ and no marginal thresholds $\hat{\underline{x}}^s,\hat{\overline{x}}^s \in A^s$ (either identical or not) such that
all induced marginal preferences of $[\mathbb{D}]^s$ are hybrid on $\lhd^s$ w.r.t.~$\hat{\underline{x}}^s$ and $\hat{\overline{x}}^s$, and
$\big\langle \hat{\underline{x}}^s, \hat{\overline{x}}^s\big\rangle^{\lhd^s} \coloneqq
\big\{x^s \in A^s: \hat{\underline{x}}^s\, \lhd^s\, x^s\, \lhd^s\, \hat{\overline{x}}^s \big\}\cup \big\{\hat{\underline{x}}^s, \hat{\overline{x}}^s\big\} \subset \langle \underline{x}^s, \overline{x}^s\rangle$.\footnote{\citet{CZ2023} investigate \emph{non-dictatorial domains} (i.e., a domain which admits a non-dictatorial, strategy-proof rule). Their Theorem 1 provides a classification of non-dictatorial, \emph{unidimensional domains} (analogous to a marginal domain $\mathbb{D}^s$ that induces a connected graph $G_{\approx}^{A^s}$, contains two completely reversed marginal preferences, and satisfies an additional condition called \emph{leaf symmetry}) according to the existence and non-existence of anonymous, tops-only and strategy-proof rules. Furthermore, their Corollary 2 refines the classification by further requiring the non-dictatorial, unidimensional domains to be \emph{tops-only domains} (i.e., a domain where every strategy-proof rule endogenously satisfies the tops-only property).
Leaf symmetry plays a trivial role in the classification - it is only adopted to ensure that all non-dictatorial domains under investigate are fully characterized by \emph{the unique seconds property} (see their footnote 38).
If we enlarge the classification to cover both non-dictatorial and dictatorial domains, the condition of leaf symmetry can be removed from the definition of unidimensional domains, and their proof still works to establish a classification result (analogous to \textbf{Conditions (i)} and \textbf{(ii)}).}
\end{description}


The next lemma completes the verification of condition (ii) of Definition \ref{def:AMH}.


\begin{lemma}\label{lem:noleaf}
Given $s \in M$, let $\underline{x}^s \neq \overline{x}^s$.
The subgraph $G_{\approx}^{\langle \underline{x}^s,\, \overline{x}^s\rangle}$ has no leaf.
\end{lemma}

\begin{proof}
For ease of presentation, we assume $A^s = \big\{a_1^s,\dots, a_{|A^s|}^s\big\}$ and
$a_k^s \prec^s a_{k+1}^s$ for all $k = 1, \dots, |A^s|-1$.
Clearly, $a_k^s\mathrel{[\underline{P}_i]^s}a_{k+1}^s$ and
$a_{k+1}^s\mathrel{[\overline{P}_i]^s}a_k^s$ for all $k = 1, \dots, |A^s|-1$.
\textbf{Condition (ii)} implies $\underline{x}^s = a_{\underline{k}}^s$ and $\overline{x}^s = a_{\overline{k}}^s$ for some $1 \leq \underline{k}< \overline{k}\leq |A^s|$ with $\overline{k}-\underline{k}>1$.
Moreover, since all induced marginal preferences of $[\mathbb{D}]^s$ are hybrid on $\prec^s$ w.r.t.~$\underline{x}^s$ and $\overline{x}^s$ by \textbf{Condition (i)},
the connected graph $G_{\approx}^{A^s}$ must be a combination of
the line $(a_1^s, \dots, a_{\underline{k}}^s)$,
the connected subgraph $G_{\approx}^{\langle a_{\underline{k}}^s,\, a_{\overline{k}}^s\rangle}$, and
the line $(a_{\overline{k}}^s, \dots, a_{|A^s|}^s)$.

Suppose by contradiction that $G_{\approx}^{\langle a_{\underline{k}}^s,\, a_{\overline{k}}^s\rangle}$ has a leaf.
Thus, we have $a_k^s, a_{k'}^s \in \langle a_{\underline{k}}^s,\, a_{\overline{k}}^s\rangle$
such that $a_k^s$ is a leaf of $G_{\approx}^{\langle a_{\underline{k}}^s,\, a_{\overline{k}}^s\rangle}$ and
$a_k^s \approx a_{k'}^s$.
There are two cases to consider: $\underline{k}<k< \overline{k}$ and
$k \in \{\underline{k}, \overline{k}\}$.
In each case, we induce a contradiction.
\medskip

First, let $\underline{k}<k< \overline{k}$.
Recall that $G_{\approx}^{A^s}$ is a combination of the line $(a_1^s, \dots, a_{\underline{k}}^s)$, the connected subgraph $G_{\approx}^{\langle a_{\underline{k}}^s,\, a_{\overline{k}}^s\rangle}$, and
the line $(a_{\overline{k}}^s, \dots, a_{|A^s|}^s)$.
Since $a_k^s$ is a leaf of $G_{\approx}^{\langle a_{\underline{k}}^s,\, a_{\overline{k}}^s\rangle}$, $\underline{k}< k < \overline{k}$ and $a_k^s \approx a_{k'}^s$,
it is true that $a_{k'}^s$ is included in all paths in $G_{\approx}^{A^s}$ connecting $a_1^s$ and $a_k^s$, and $a_{k'}^s$ is also included in all paths in $G_{\approx}^{A^s}$ connecting $a_{|A^s|}^s$ and $a_k^s$.
Immediately, statement (ii) of Lemma \ref{lem:connectedgraph} implies $a_{k'}^s\mathrel{[\underline{P}_i]^s}a_k^s$ and
$a_{k'}^s\mathrel{[\overline{P}_i]^s}a_k^s$, which contradict the fact that $[\underline{P}_i]^s$ and $[\overline{P}_i]^s$ are complete reversals.

In the rest of the prove, let $k \in \{\underline{k}, \overline{k}\}$.
We assume w.l.o.g.~that $k = \underline{k}$.
The verification related to $k = \overline{k}$ is symmetric.\medskip

\noindent
\textsc{Claim 1}:
We have $\underline{k}<k'< \overline{k}$.
\medskip

It is clear that $\underline{k}<k' \leq \overline{k}$.
Suppose by contradiction that $k' = \overline{k}$.
Since $\overline{k}-\underline{k}>1$,
we fix an arbitrary element $a_p^s$ such that $\underline{k}<p< \overline{k}$.
Since $G_{\approx}^{\langle a_{\underline{k}}^s,\, a_{\overline{k}}^s\rangle}$ is a connected subgraph,
$a_{\underline{k}}^s$ is a leaf and $a_{\underline{k}}^s \approx a_{\overline{k}}^s$,
we know that $a_{\overline{k}}^s$ is included in all paths in $G_{\approx}^{\langle a_{\underline{k}}^s,\, a_{\overline{k}}^s\rangle}$ connecting $a_{\underline{k}}^s$ and $a_p^s$.
Hence, in $G_{\approx}^{A^s}$, $a_{\overline{k}}^s$ is included in all paths in $G_{\approx}^{A^s}$ connecting $a_1^s$ and $a_p^s$.
Consequently, statement (ii) of Lemma \ref{lem:connectedgraph} implies $a_{\overline{k}}^s\mathrel{[P_i]^s}a_p^s$ for all
$[P_i]^s \in [\mathbb{D}]^s$ with $r_1([P_i]^s) = a_1^s$.
However, we have $r_1([\underline{P}_i]^s) = a_1^s$ and $a_p^s\mathrel{[\underline{P}_i]^s}a_{\overline{k}}^s$ - a contradiction.
This completes the verification of the claim.
\medskip


Now, since $G_{\approx}^{\langle a_{\underline{k}}^s,\, a_{\overline{k}}^s\rangle}$ is a connected subgraph,
$a_{\underline{k}}$ is a leaf and $a_{\underline{k}}^s \approx a_{k'}^s$,
it is evident that $G_{\approx}^{\langle a_{\underline{k}}^s,\, a_{\overline{k}}^s\rangle\backslash \{a_{\underline{k}}^s\}}$ is also connected subgraph.
Consequently, $G_{\approx}^{A^s}$ is a combination of
the line $(a_1^s, \dots, a_{\underline{k}}^s,a_{k'}^s)$, the connected subgraph $G_{\approx}^{\langle a_{\underline{k}}^s,\, a_{\overline{k}}^s\rangle\backslash \{a_{\underline{k}}^s\}}$ and
the line $(a_{\overline{k}}^s, \dots, a_{|A^s|}^s)$.
Now, we construct a new linear order $\lhd^s$ over $A^s$:
\begin{itemize}
\item if $k' = \underline{k}+1$, let $\lhd^s = \,\prec^s$, and

\item if $k' > \underline{k}+1$, let $a_1^s \,\lhd^s\, \cdot\cdot\cdot \,\lhd^s\, a_{\underline{k}}^s\,\lhd^s\,
a_{k'}^s\, \lhd^s\, a_{\underline{k}+1}^s\, \lhd^s\, \cdot\cdot\cdot \, \lhd^s\, a_{k'-1}^s \, \lhd^s\, a_{k'+1}^s \, \lhd^s\,\cdot\cdot\cdot \,
\lhd^s\, a_{\overline{k}}^s\, \lhd^s\, \cdot\cdot\cdot\, \lhd^s\, a_{|A^s|}^s$.
\end{itemize}
There are two subcases to consider: (1) $|\langle a_{\underline{k}}^s, a_{\overline{k}}^s\rangle| = 3$, and
(2) $|\langle a_{\underline{k}}^s, a_{\overline{k}}^s\rangle| > 3$.
In subcase (1), we fix the identical marginal thresholds $a_{\underline{k}}^s$ and $a_{\underline{k}}^s$ on $\lhd^s$, which clearly implies
$\langle a_{\underline{k}}^s, a_{\underline{k}}^s\rangle^{\lhd^s} = \{a_{\underline{k}}^s\} \subset \langle a_{\underline{k}}^s, a_{\overline{k}}^s\rangle$.
In subcase (2),  we fix the distinct marginal thresholds $a_{k'}^s$ and $a_{\overline{k}}^s$ on $\lhd^s$, and have
$\langle a_{k'}^s, a_{\overline{k}}^s\rangle^{\lhd^s} = \langle a_{\underline{k}}^s, a_{\overline{k}}^s\rangle\backslash \{a_{\underline{k}}^s\} \subset \langle a_{\underline{k}}^s, a_{\overline{k}}^s\rangle$.
\medskip

\noindent
\textsc{Claim 2}: In subcase (1),
all induced marginal preferences of $[\mathbb{D}]^s$ are hybrid on $\lhd^s$ w.r.t.~$a_{\underline{k}}^s$ and $a_{\underline{k}}^s$, which contradicts \textbf{Condition (ii)}.\medskip

In subcase (1), it is true that $\overline{k} = \underline{k}+2$ and $k'=\underline{k}+1$ which imply $\langle a_{\underline{k}}^s, a_{\overline{k}}^s\rangle\backslash \{a_{\underline{k}}^s\}
= \{a_{\underline{k}+1}^s, a_{\underline{k}+2}^s\}$.
Since $G_{\approx}^{A^s}$ is a combination of
the line $(a_1^s, \dots, a_{\underline{k}}^s,a_{\underline{k}+1}^s)$, the connected subgraph $G_{\approx}^{\langle a_{\underline{k}}^s,\, a_{\overline{k}}^s\rangle\backslash \{a_{\underline{k}}^s\}} = G_{\approx}^{\{a_{\underline{k}+1}^s,\, a_{\underline{k}+2}^s\}}$ which of course is a graph of two vertices and one singleton edge, and
the line $(a_{\underline{k}+2}^s, \dots, a_{|A^s|}^s)$,
$G_{\approx}^{A^s}$ equals the line $\big(a_1^s, \dots, a_{\underline{k}}^s,a_{\underline{k}+1}^s,a_{\underline{k}+2}^s, \dots, a_{|A^s|}^s\big)$.
Then, statement (ii) of Lemma \ref{lem:connectedgraph} implies that
all induced marginal preferences of $[\mathbb{D}]^s$ are single-peaked on $\lhd^s$ (equivalently, hybrid on $\lhd^s$ w.r.t.~$a_{\underline{k}}^s$ and $a_{\underline{k}}^s$).
This completes the verification of the claim.
\medskip

\noindent
\textsc{Claim 3}: In subcase (2), all induced marginal preferences of $[\mathbb{D}]^s$ are hybrid on $\lhd^s$ w.r.t.~$a_{k'}^s$ and $a_{\overline{k}}^s$, which contradicts \textbf{Condition (ii)}.\medskip

In subcase (2),
$G_{\approx}^{A^s}$ is a combination of
the line $(a_1^s, \dots, a_{\underline{k}}^s,a_{k'}^s) = \langle a_1^s, a_{k'}^s\rangle^{\lhd^s}$, the connected subgraph $G_{\approx}^{\langle a_{\underline{k}}^s,\, a_{\overline{k}}^s\rangle\backslash \{a_{\underline{k}}^s\}} = G_{\approx}^{\langle a_{k'}^s,\, a_{\overline{k}}^s\rangle^{\lhd^s}}$ and
the line $(a_{\overline{k}}^s, \dots, a_{|A^s|}^s) = \langle a_{\overline{k}}^s, a_{|A^s|}^s\rangle^{\lhd^s}$.
Then, statement (ii) of Lemma \ref{lem:connectedgraph} implies that
all induced marginal preferences of $[\mathbb{D}]^s$ are hybrid on $\lhd^s$ w.r.t.~$a_{k'}^s$ and $a_{\underline{k}}^s$.
This completes the verification of the claim, and hence proves the Lemma.
\end{proof}


Now, we know that for each $s \in M$, all induced marginal preferences of $[\mathbb{D}]^s$ are hybrid on $\prec^s$ w.r.t.~marginal thresholds $\underline{x}^s$ and $\overline{x}^s$, and correspondingly
assemble two thresholds $\underline{x} = (\underline{x}^1, \dots, \underline{x}^m)$ and $\overline{x} = (\overline{x}^1, \dots, \overline{x}^m)$ on $\prec = \times_{s \in M}\prec^s$.
According to condition (i) of Definition \ref{def:AMH}, to complete the verification,
we show $\mathbb{D} \subseteq \mathbb{D}_{\textrm{MH}}(\prec, \underline{x}, \overline{x})$.
As a preparation, we first highlight the following two specific two-voter FBRs that will be repeatedly adopted in the proof.

Fix $N = \{1,2\}$ and a component $s\in M$.
Given fixed ballots $(b_{J}^s)_{J \subseteq N}$,
the FBR $f^s: \big[[\mathbb{D}]^s\big]^2 \rightarrow A^s$ is called a \textbf{projection rule} if there exists $\hat{x}^s \in A^s$ such that $b_{J}^s = \hat{x}^s$ for all nonempty proper coalitions $J \subset N$.
Then, $f^s$ can be explicitly simplified as follows: for all $[P_1]^s, [P_2]^s \in [\mathbb{D}]^s$,
\begin{align*}
f^s([P_1]^s, [P_2]^s)
=  \mathop{\textrm{med}}\nolimits^{\prec^s}\big(r_1([P_1]^s), r_1([P_2]^s), \hat{x}^s\big).
\end{align*}
We henceforth call the projection rule $f^s$ above \textbf{the $\bm{\hat{x}^s}$-projection rule}.\footnote{We call $f^s$ a projection rule as it can be reinterpreted in the term of ``projection''.
Given $[P_1]^s, [P_2]^s \in [\mathbb{D}]^s$,
the projection of $\hat{x}^s$ on the interval $\langle r_1([P_1]^s), r_1([P_2]^s)\rangle$ equals $\mathop{\textrm{med}}\nolimits^{\prec^s}\big(r_1([P_1]^s), r_1([P_2]^s), \hat{x}^s\big)$.}
By Proposition 2 of \citet{M1980},
it is clear that if $\underline{x}^s = \overline{x}^s$, then
all induced marginal preferences of $[\mathbb{D}]^s$ are single-peaked on $\prec^s$, and
all projection rules are strategy-proof marginal rules.



Given fixed ballots $(b_{J}^s)_{J \subseteq N}$, the FBR $f^s: \big[[\mathbb{D}]^s\big]^2 \rightarrow A^s$ is called a \textbf{hybrid rule} if there exist $i \in N$ and $\underline{a}^s, \overline{a}^s \in A^s$
with $\underline{a}^s \prec^s \overline{a}^s$ such that for all nonempty coalition $J \subseteq N$,
$[i \notin J ] \Rightarrow [b_J^s = \underline{a}^s]$
and $[i \in J] \Rightarrow [b_J^s = \overline{a}^s]$.
Then, $f^s$ can be explicitly simplified as follows: for all $[P_1]^s, [P_2]^s \in [\mathbb{D}]^s$,
\begin{align*}
f^s([P_1]^s, [P_2]^s)
%= & \mathop{\mathop{\max}\nolimits^{\prec^s}}_{J \subseteq N~~~\,}
%\Big(\mathop{\mathop{\min}\nolimits^{\prec^s}}_{i \in J~~~}\big(a_i^s, b_J^s\big)\Big)\\
= \left\{
\begin{array}{ll}
r_1([P_i]^s) & \textrm{if}\; r_1([P_i]^s) \in \langle \underline{a}^s, \overline{a}^s\rangle,\\[0.3em]
\mathop{\textrm{med}}\nolimits^{\prec^s}\big(r_1([P_1]^s), r_1([P_2]^s), \underline{a}^s\big)& \textrm{if}\; r_1([P_i]^s) \prec^s \underline{a}^s,\; \textrm{and}\\[0.3em]
\mathop{\textrm{med}}\nolimits^{\prec^s}\big(r_1([P_1]^s), r_1([P_2]^s), \overline{a}^s\big)& \textrm{if}\; r_1([P_i]^s) \succ^s \overline{a}^s.
\end{array}
\right.
\end{align*}
We henceforth call the hybrid rule $f^s$ here \textbf{the voter-$\bm{i}$-$\bm{(\underline{a}^s, \overline{a}^s)}$-hybrid rule} to highlight that it is a hybrid of voter $i$'s dictatorship on the interval $\langle \underline{a}^s, \overline{a}^s\rangle$ and two projection rules.
By Theorem 2 of \citet{CRSSZ2022}, we know that
as all induced marginal preferences are hybrid on $\prec^s$ w.r.t.~$\underline{x}^s$ and $\overline{x}^s$,
if $\underline{a}^s \preccurlyeq \underline{x}^s$ and $\overline{x}^s \preccurlyeq^s \overline{a}^s$,
then the voter-$i$-$(\underline{a}^s, \overline{a}^s)$-hybrid rule is a strategy-proof marginal rule.


\begin{lemma}\label{lem:MH}
We have $\mathbb{D} \subseteq \mathbb{D}_{\emph{MH}}(\prec, \underline{x}, \overline{x})$.
\end{lemma}

\begin{proof}
Fixing a preference $P_i^{\ast} \in \mathbb{D}$ and two similar alternatives $a,b \in A$,
let $r_1(P_i^{\ast}) = x$, $M(a,b) = \{s\}$,
and either $a^s = x^s$, or $a^s \in \textrm{Int}\langle x^s, b^s\rangle$ and $a^s \notin \textrm{Int}\langle \underline{x}^s, \overline{x}^s\rangle$.
We show $a\mathrel{P_i^{\ast}}b$.
We consider 4 cases:
(1) $a^s = x^s$,
(2) $a^s \neq x^s$ and $\underline{x}^s = \overline{x}^s$,
(3) $a^s \neq x^s$, $\underline{x}^s \neq \overline{x}^s$ and $x^s \in \langle \underline{x}^s, \overline{x}^s\rangle$, and
(4) $a^s \neq x^s$, $\underline{x}^s \neq \overline{x}^s$ and $x^s \notin \langle \underline{x}^s, \overline{x}^s\rangle$.
In Case (1), by the proof of Lemma \ref{lem:tops-onlydomain}, we know that $P_i^{\ast}$ is a top-separable preference, and hence $a\mathrel{P_i^{\ast}}b$, as required.
In the rest of three cases, we fix $N = \{1,2\}$, assemble projection rules and hybrid rules to construct various strategy-proof rules, and show $a\mathrel{P_i^{\ast}}b$ via their strategy-proofness.


\noindent
\textbf{Case (2)}.~
Since $\underline{x}^s = \overline{x}^s$,
we know that $a^s \in \textrm{Int}\langle x^s, b^s\rangle$, and the $a^s$-projection rule $f^s: \big[[\mathbb{D}]^s\big]^2 \rightarrow A^s$ is a strategy-proof marginal rule.
At each $t \in M\backslash \{s\}$,
we refer to the voter-2 marginal dictatorship $f^t: \big[[\mathbb{D}]^t\big]^2 \rightarrow A^t$,
which of course is a strategy-proof marginal rule.
Then, we assemble the SCF $f: \mathbb{D}^2 \rightarrow A$ such that for all $(P_1, P_2) \in \mathbb{D}^2$,
$f(P_1, P_2) = \big(f^1([P_1]^1, [P_2]^1), \dots, f^m([P_1]^m, [P_2]^m)\big)$.
Since $\mathbb{D}$ is a decomposable domain, the assembled SCF $f$ is a strategy-proof rule.
Now, given $P_1 = P_i^{\ast}$, $P_1' \in \mathbb{D}^{b}$ and $P_2 \in \mathbb{D}^b$,
we have $f(P_1, P_2) = (a^s, b^{-s}) = (a^s, a^{-s}) = a$ and $f(P_1', P_2) = b$,
which by strategy-proofness imply $a\mathrel{P_1}b$, as required.

\noindent
\textbf{Case (3)}.~
We know $a^s \in \textrm{Int}\langle x^s, b^s\rangle$, $a^s \notin \textrm{Int}\langle \underline{x}^s, \overline{x}^s\rangle$ and $x^s \in \langle \underline{x}^s, \overline{x}^s\rangle$.
There are two symmetric subcases: $a^s \preccurlyeq^s \underline{x}^s$ and $\overline{x}^s \preccurlyeq^s a^s$.
We assume w.l.o.g.~that $a^s \preccurlyeq^s \underline{x}^s$.
Thus, it must be the case that $b^s \prec^s a^s \preccurlyeq^s \underline{x}^s \preccurlyeq^s x^s \preccurlyeq^s \overline{x}^s$.
Correspondingly, the voter-$2$-$(a^s, \overline{x}^s)$-hybrid rule $f^s: \big[[\mathbb{D}]^s\big]^2 \rightarrow A^s$ is a strategy-proof marginal rule.
At each $t \in M\backslash \{s\}$,
we refer to the voter-2 marginal dictatorship $f^t: \big[[\mathbb{D}]^t\big]^2 \rightarrow A^t$.
Then, we assemble the SCF $f: \mathbb{D}^2 \rightarrow A$ such that for all $(P_1, P_2) \in \mathbb{D}^2$,
$f(P_1, P_2) = \big(f^1([P_1]^1, [P_2]^1), \dots, f^m([P_1]^m, [P_2]^m)\big)$.
Since $\mathbb{D}$ is a decomposable domain, the assembled SCF $f$ is a strategy-proof rule.
Now, given $P_1 = P_i^{\ast}$, $P_1' \in \mathbb{D}^{b}$ and $P_2 \in \mathbb{D}^b$,
we have $f(P_1, P_2) = (a^s, b^{-s}) = (a^s, a^{-s}) = a$ and $f(P_1', P_2) = b$,
which by strategy-proofness imply $a\mathrel{P_1}b$, as required.

\noindent
\textbf{Case (4)}.~
We know $a^s \in \textrm{Int}\langle x^s, b^s\rangle$ and $a^s \notin \textrm{Int}\langle \underline{x}^s, \overline{x}^s\rangle$.
Moreover, since $x^s \notin \langle \underline{x}^s, \overline{x}^s\rangle$,
either $x^s \prec^s \underline{x}^s$ or $\overline{x}^s \prec^s x^s$ holds.
We assume w.l.o.g.~that $x^s \prec^s \underline{x}^s$.
Thus, there are three subcases: (i) $a^s \prec^s x^s$,
(ii) $x^s \prec^s a^s \prec^s \underline{x}^s$, and
(iii) $\overline{x}^s \prec^s a^s$.

In subcase (i), it must be the case that $b^s \prec^s a^s \prec^s x^s \prec^s \underline{x}^s \prec^s \overline{x}^s$.
Correspondingly, the voter-$2$-$(a^s, \overline{x}^s)$-hybrid rule $f^s: \big[[\mathbb{D}]^s\big]^2 \rightarrow A^s$ is a strategy-proof marginal rule.
At each $t \in M\backslash \{s\}$,
we refer to the voter-2 marginal dictatorship $f^t: \big[[\mathbb{D}]^t\big]^2 \rightarrow A^t$.
Then, we assemble the SCF $f: \mathbb{D}^2 \rightarrow A$ such that for all $(P_1, P_2) \in \mathbb{D}^2$,
$f(P_1, P_2) = \big(f^1([P_1]^1, [P_2]^1), \dots, f^m([P_1]^m, [P_2]^m)\big)$.
Since $\mathbb{D}$ is a decomposable domain, the assembled SCF $f$ is a strategy-proof rule.
Now, given $P_1 = P_i^{\ast}$, $P_1' \in \mathbb{D}^{b}$ and $P_2 \in \mathbb{D}^b$,
we have $f(P_1, P_2) = (a^s, b^{-s}) = (a^s, a^{-s}) = a$ and $f(P_1', P_2) = b$,
which by strategy-proofness imply $a\mathrel{P_1}b$, as required.

In subcase (ii), we know $x^s \prec^s a^s \prec^s \underline{x}^s \prec^s \overline{x}^s$ and
$x^s \prec^s a^s \prec^s b^s$.
Correspondingly, the voter-$1$-$(a^s, \overline{x}^s)$-hybrid rule $f^s: \big[[\mathbb{D}]^s\big]^2 \rightarrow A^s$ is a strategy-proof marginal rule.
At each $t \in M\backslash \{s\}$,
we refer to the voter-2 marginal dictatorship $f^t: \big[[\mathbb{D}]^t\big]^2 \rightarrow A^t$.
Then, we assemble the SCF $f: \mathbb{D}^2 \rightarrow A$ such that for all $(P_1, P_2) \in \mathbb{D}^2$,
$f(P_1, P_2) = \big(f^1([P_1]^1, [P_2]^1), \dots, f^m([P_1]^m, [P_2]^m)\big)$.
Since $\mathbb{D}$ is a decomposable domain, the assembled SCF $f$ is a strategy-proof rule.
Now, given $P_1 = P_i^{\ast}$, $P_1' \in \mathbb{D}^{b}$ and $P_2 \in \mathbb{D}^b$,
we have $f(P_1, P_2) = (a^s, b^{-s}) = (a^s, a^{-s}) = a$ and $f(P_1', P_2) = b$,
which by strategy-proofness imply $a\mathrel{P_1}b$, as required.

In subcase (iii), it must be the case that $x^s \prec^s\underline{x}^s \prec^s \overline{x}^s \prec^s a^s \prec^s b^s$.
Correspondingly, the voter-$2$-$(\underline{x}^s, a^s)$-hybrid rule $f^s: \big[[\mathbb{D}]^s\big]^2 \rightarrow A^s$ is a strategy-proof marginal rule.
At each $t \in M\backslash \{s\}$,
we refer to the voter-2 marginal dictatorship $f^t: \big[[\mathbb{D}]^t\big]^2 \rightarrow A^t$.
Then, we assemble the SCF $f: \mathbb{D}^2 \rightarrow A$ such that for all $(P_1, P_2) \in \mathbb{D}^2$,
$f(P_1, P_2) = \big(f^1([P_1]^1, [P_2]^1), \dots, f^m([P_1]^m, [P_2]^m)\big)$.
Since $\mathbb{D}$ is a decomposable domain, the assembled SCF $f$ is a strategy-proof rule.
Now, given $P_1 = P_i^{\ast}$, $P_1' \in \mathbb{D}^{b}$ and $P_2 \in \mathbb{D}^b$,
we have $f(P_1, P_2) = (a^s, b^{-s})= (a^s, a^{-s}) = a$ and $f(P_1', P_2) = b$,
which by strategy-proofness imply $a\mathrel{P_1}b$, as required.

In conclusion, $P_i^{\ast}$ is multidimensional hybrid on $\prec$ w.r.t.~$\underline{x}$ and $\overline{x}$.
This completes the verification of the Lemma, and hence proves the necessity part of the Theorem.
\end{proof}




\section{Two clarifications}\label{app:clarification}

We first introduce some necessary notation.
Given a preference $P_i$ and two alternatives $a, b \in A$,
let $a\mathrel{P_i!}b$ denote that $a$ is consecutively ranked above $b$ in $P_i$, i.e.,
$a\mathrel{P_i}b$ and
there exists no $c \in A$ such that $a\mathrel{P_i}c$ and $c\mathrel{P_i}b$.
Given a preference $P_i$ and an alternative $a \in A$, let $L(a, P_i) \coloneqq \{x \in A: a\mathrel{P_i}x\}$ denote the \textbf{strict lower contour set} of $a$ at $P_i$, and
$\overline{U}(a, P_i) \coloneqq \{x \in A: x\mathrel{P_i}a\}\cup \{a\}$ denote the \textbf{weak upper contour set} of $a$ at $P_i$.
The same notation works for marginal preferences as well.

\begin{clarification}\label{cla:S}
The separable domain $\mathbb{D}_{\emph{S}}$ is a rich domain.
\end{clarification}

\begin{proof}
It is evident that $\mathbb{D}_{\textrm{S}}$ is minimally rich and satisfies diversity\textsuperscript{+}.
Next, Proposition 2 of \citet{KRSYZ2021a} has shown that the separable domain $\mathbb{D}_{\textrm{S}}$ satisfies Property $L$ on the graph $G_{\sim/\sim^+}^{\mathbb{D}_{\textrm{S}}}$:
given distinct $P_i, P_i' \in \mathbb{D}_{\textrm{S}}$ and $a \in A$, there exists a path in $G_{\sim/\sim^+}^{\mathbb{D}_{\textrm{S}}}$ connecting $P_i$ and $P_i'$ that has no $\{a,b\}$-restoration for any $b \in L(a, P_i)$ (recall the definition in footnote \ref{footnote:PropertyL}).
It is clear that Property $L$ implies the Interior\textsuperscript{+} property and
the Exterior\textsuperscript{+} property without the fulfillment of the no-detour condition.
We henceforth focus on showing the no-detour condition of the Exterior\textsuperscript{+} property.
We first provide three facts below.

\begin{fact}\label{fact:construction1}
Given distinct $P_i,P_i' \in \mathbb{D}_{\emph{S}}$ such that $P_i^s = P_i^{s\,\prime}$ for all $s \in M$,
there exists a path in $G_{\sim}^{\mathbb{D}_{\emph{S}}}$ connecting $P_i$ and $P_i'$ that has no restoration for any pair of alternatives.
\end{fact}

\begin{fact}\label{fact:construction2}
For each $t \in M$, fix a marginal preference $P_i^t$.
Fix an alternative $a \in A$ and a component $s\in M$.
There exists $\hat{P}_i \in \mathbb{D}_{\emph{S}}$ satisfying the following three conditions:
\begin{enumerate}
\item $\hat{P}_i^t = P_i^t$ for all $t \in M$,

\item $[x \mathrel{\hat{P}_i} a] \Leftrightarrow [x^t \in \overline{U}(a^t, P_i^t)\; \textrm{for all}\; t \in M]$.

\item for all $b^s, c^s \in \overline{U}(a^s, P_i^s)$ or $b^s, c^s \in L(a^s, P_i^s)$,\\
$[b^s \mathrel{P_i^s!} c^s] \Rightarrow [(b^s, y^{-s})\mathrel{\hat{P}_i!}(c^s, y^{-s})\; \textrm{for all}\; y^{-s} \in A^{-s}]$.
\end{enumerate}
\end{fact}

Fact \ref{fact:construction1} is implied by the proof of Fact 5 of \citet{CZ2019},
while Fact \ref{fact:construction2} follows from Lemma 3 of \citet{KRSYZ2021a}.

\begin{fact}\label{fact:construction3}
Given $P_i, P_i' \in \mathbb{D}_{\emph{S}}$ and $s \in M$,
let $P_i^s \neq P_i^{s\,\prime}$, $P_i^t = P_i^{t\,\prime}$ for all $t \in M\backslash \{s\}$,
$r_1(P_i) = (x^s, x^{-s})$ and $r_1(P_i') =(y^s, x^{-s})$ where either $x^s = y^s$ or $x^s \neq y^s$.
There exists a path $(P_{i|1}, \dots, P_{i|v})$ in $G_{\sim/\sim^+}^{\mathbb{D}_{\emph{S}}}$ connecting $P_i$ and $P_i'$ that has no restoration for any pair of alternatives $a,b \in (A^s, x^{-s})$, and satisfies the following two additional conditions:
\begin{itemize}
\item[\rm (i)] if $x^s = y^s$, then $r_1(P_{i|k}) = (x^s, x^{-s})$ for all $k=1, \dots, v$, and

\item[\rm (ii)] if $x^s \neq y^s$, then $r_1(P_{i|k}) \in (A^s, x^{-s})$ for all $k=1, \dots, v$.
\end{itemize}
\end{fact}

\begin{proof}
For notational convenience, let $P_i^t = P_i^{t\,\prime} = \hat{P}_i^t$ for all $t \in M\backslash \{s\}$.
First, since $P_i^s, P_i^{s\,\prime} \in \mathbb{P}^s$,
where $\mathbb{P}^s$ is the universal marginal domain,
there exists a path $\pi^s=(P_{i|1}^s, \dots, P_{i|q}^s)$ in $G_{\sim}^{\mathbb{P}^s}$ connecting $P_i^s$ and $P_i^{s\,\prime}$ that has no restoration for any pair of elements.
Second, for each $k \in \{1, \dots, q-1\}$,
we identify the unique pair of elements $x_k^s, \hat{x}_k^s \in A^s$ such that $\hat{x}_k^s\mathrel{P_{i|k}^s!} x_k^s$ and
$x_k^s\mathrel{P_{i|k+1}^s!} \hat{x}_k^s$, and assemble the alternative $(x_k^s, x^{-s})$.
Thus, according to the marginal preferences $P_{i|k}^s$ and $\hat{P}_{i}^t$ for each $t \in M\backslash \{s\}$, the alternative $(x_k^s, x^{-s})$ and the component $s$,
by Fact \ref{fact:construction2}, we have a preference $\hat{P}_{i|k} \in \mathbb{D}_{\textrm{S}}$ satisfying conditions 1 - 3 of Fact \ref{fact:construction2}.
Third, for each $k \in \{1, \dots, q-1\}$,
since $(\hat{x}_k^s, y^{-s})\mathrel{\hat{P}_i!}(x_k^s, y^{-s})$ for all $y^{-s} \in A^{-s}$ by condition 3 of Fact \ref{fact:construction2},
we can simultaneously locally switch $(\hat{x}_k^s, y^{-s})$ and $(x_k^s, y^{-s})$ for all $y^{-s} \in A^{-s}$ in $\hat{P}_{i|k}$ to generate a preference $\check{P}_{i|k}$.
It is evident that $\check{P}_{i|k}$ is also a separable preference.
Thus, we have $\hat{P}_{i|k} \sim^+ \check{P}_{i|k}$,
$(\hat{x}_k^{s}, y^{-s})\mathrel{\hat{P}_{i|k}!} (x_k^{s}, y^{-s})$ and
$(x_k^{s}, y^{-s})\mathrel{\check{P}_{i|k}!} (\hat{x}_k^{s}, y^{-s})$ for all $y^{-s} \in A^{-s}$.
It is clear that $\check{P}_{i|k}^s = P_{i|k+1}^s$ for all $k = 1, \dots, q-1$.
Between $P_i$ and $\hat{P}_{i|1}$, since
$P_i^s = P_{i|1}^s = \hat{P}_{i|1}^s$ and $P_i^t  = \hat{P}_i^t = \hat{P}_{i|1}^t $ for all $t \in M\backslash \{s\}$,
by Fact \ref{fact:construction1},
we have a path $\pi_1$ in $G_{\sim}^{\mathbb{D}_{\textrm{S}}}$ connecting $P_i$ and $\hat{P}_{i|1}$
that has no restoration for any pair of alternatives.
For each $k \in \{1, \dots, q-2\}$,
between $\check{P}_{i|k}$ and $\hat{P}_{i|k+1}$, since
$\check{P}_{i|k}^s = \hat{P}_{i|k+1}^s = P_{i|k+1}^s$ and $\check{P}_{i|k}^t = \hat{P}_{i|k+1}^t = \hat{P}_i^t$ for all $t \in M\backslash \{s\}$,
by Fact \ref{fact:construction1},
we have a path $\pi_{k+1}$ in $G_{\sim}^{\mathbb{D}_{\textrm{S}}}$ connecting $\check{P}_{i|k}$ and $\hat{P}_{i|k+1}$ that has no restoration for any pair of alternatives.
Between $\check{P}_{i|q-1}$ and $P_i'$, since
$\check{P}_{i|q-1}^s = P_{i|q}^s = P_i^{s\,\prime}$ and $\check{P}_{i|q-1}^t  = \hat{P}_i^t = P_i^{t\,\prime}$ for all $t \in M\backslash \{s\}$,
by Fact \ref{fact:construction1},
we have a path $\pi_q $ in $G_{\sim}^{\mathbb{D}_{\textrm{S}}}$ connecting $\check{P}_{i|q-1}$ and $P_i'$
that has no restoration for any pair of alternatives.
Last, since $\hat{P}_{i|k} \sim^+ \check{P}_{i|k}$ for all $k=1, \dots, q-1$,
we have a concatenated path in $G_{\sim/\sim^+}^{\mathbb{D}_{\textrm{S}}}$ connecting $P_i$ and $P_i'$:
\begin{align*}
\pi = &~ \big(\,\underset{\pi_1}{\underbrace{(P_i, \dots, \hat{P}_{i|1})}},
\underset{\pi_2}{\underbrace{(\check{P}_{i|1}, \dots, \hat{P}_{i|2})}}, \dots,
\underset{\pi_{q-1}}{\underbrace{(\check{P}_{i|q-2}, \dots, \hat{P}_{i|q-1})}},
\underset{\pi_{q}}{\underbrace{(\check{P}_{i|q-1}, \dots, P_i')}}\,\big).
\end{align*}

For notational convenience, we label $\pi \coloneqq (P_{i|1}, \dots, P_{i|v})$.
Note that for each $k \in \{1, \dots, q\}$,
all preferences in $\pi_k$ have the same marginal preferences.
Therefore, for all $k = 1, \dots, v$, we have $P_{i|k}^t = \hat{P}_i^t$ for all $t \in M\backslash \{s\}$,
which implies $r_1(P_{i|k}) \in (A^s, x^{-s})$.
In particular, if $x^s = y^s$,
as $\pi^s$ has no restoration for any pair of elements, it is true that $r_1(P_{i|k}^s) = x^s$ for all $k = 1, \dots, q$. Consequently, we have $r_1(P_{i|k}) = (x^s, x^{-s})$ for all $k=1, \dots, v$.
This proves the two additional conditions of the Fact.
Last, we show that $\pi$ has no restoration for any pair $a,b \in (A^s, x^{-s})$.
Suppose not, i.e., we have $(a^s, x^{-s})\mathrel{P_{i|k_1}}(b^s, x^{-s})$,
$(b^s, x^{-s})\mathrel{P_{i|k_2}}(a^s, x^{-s})$ and
$(a^s, x^{-s})\mathrel{P_{i|k_3}}(b^s, x^{-s})$ for some $1 \leq k_1 < k_2 < k_3 \leq v$.
Since $P_{i|k_1}, P_{i|k_2}$ and $P_{i|k_3}$ are separable preferences,
we have $a^s\mathrel{P_{i|k_1}^s}b^s$, $b^s\mathrel{P_{i|k_2}^s}a^s$ and
$a^s \mathrel{P_{i|k_3}^s}b^s$.
Since for each $k \in \{1, \dots, q\}$, all preferences of $\pi_k$ have the same marginal preference on the component $s$, which is $P_{i|k}^s$,
it must be the case that $P_{i|k_1} \in \pi_{k}$, $P_{i|k_2} \in \pi_{k'}$ and
$P_{i|k_3} \in \pi_{k''}$ for some $1 \leq k < k' < k'' \leq q$.
Consequently, we have $a^s\mathrel{P_{i|k}^s}b^s$, $b^s\mathrel{P_{i|k'}^s}a^s$ and
$a^s\mathrel{P_{i|k''}^s}b^s$, which indicate an $\{a^s, b^s\}$-restoration on $\pi^s=(P_{i|1}^s, \dots, P_{i|q}^s)$ - a contradiction.
\end{proof}

%\begin{fact}\label{fact:construction4'}
%Given distinct $P_i, P_i' \in \mathbb{D}_{\emph{S}}$,
%let $r_1(P_i) = r_1(P_i') = x$.
%There exists a path $(P_{i|1}, \dots, P_{i|v})$ in $G_{\sim/\sim^+}^{\mathbb{D}_{\emph{S}}}$ connecting $P_i$ and $P_i'$ such that $r_1(P_{i|k}) = x$ for all $k=1, \dots, v$.
%\end{fact}
%
%\begin{proof}
%Let $M^{\ast} \subseteq M$ be such that $P_i^s \neq P_i^{s\,\prime}$ for all $s \in M^{\ast}$ and
%$P_i^t = P_i^{t\,\prime}$ for all $t \in M\backslash M^{\ast}$.
%If $M^{\ast} = \emptyset$, the Fact follows from Fact \ref{fact:construction1}.
%If $M^{\ast}$ is a singleton set, the Fact follows from Fact \ref{fact:construction3}.
%We assume w.l.o.g.~that $M^{\ast} = \{1, \dots, t\}$, where $2 \leq t \leq m$.
%For each $s \in \{1, \dots, t-1\}$,
%we identify a profile of marginal preferences $(P_i^{1\, \prime}, \dots, P_i^{s\,\prime}, P_i^{s+1}, \dots, P_i^m)$, and assemble a separable preference $P_{i|s}$ such that
%$[1\leq \tau \leq s] \Rightarrow [P_{i|s}^{\tau} = P_i^{\tau\, \prime}]$ and
%$[s<\tau\leq m] \Rightarrow [P_{i|s}^{\tau} = P_i^{\tau}]$.
%Note that between $P_i$ and $P_{i|1}$, since
%$P_i^1 \neq P_i^{1\,\prime} = P_{i|1}^1$, $P_i^{\tau} = P_{i|1}^{\tau}$ for all $\tau \in M\backslash \{s\}$ and $r_1(P_i) = r_1(P_{i|1}) = x$,
%by condition (ii) of Fact \ref{fact:construction3}, we have a path $\pi_1$ in $G_{\sim/\sim^+}^{\mathbb{D}_{\textrm{S}}}$ connecting $P_i$ and $P_{i|1}$ such that every preference of $\pi_1$ has the peak $x$.
%Next, for each $s \in \{1, \dots, t-2\}$,
%between $P_{i|s}$ and $P_{i|s+1}$, since
%$P_{i|s}^{s+1} =P_i^{s+1}\neq P_i^{s+1\,\prime} = P_{i|s+1}^{s+1}$,
%$P_{i|s}^{\tau} = P_{i|s+1}^{\tau}$ for all $\tau \in M\backslash \{s+1\}$ and $r_1(P_{i|s}) = r_1(P_{i|s+1}) = x$,
%by condition (ii) of Fact \ref{fact:construction3}, we have a path $\pi_{s+1}$ in $G_{\sim/\sim^+}^{\mathbb{D}_{\textrm{S}}}$ connecting $P_{i|s}$ and $P_{i|s+1}$ such that every preference of $\pi_{s+1}$ has the peak $x$.
%Symmetrically, between $P_{i|t-1}$ and $P_i'$, since
%$P_{i|t-1}^{t} =P_i^{t}\neq P_i^{t\,\prime}$,
%$P_{i|t-1}^{\tau} = P_i^{\tau\,\prime}$ for all $\tau \in M\backslash \{t\}$ and $r_1(P_{i|t-1}) = r_1(P_i') = x$,
%by condition (ii) of Fact \ref{fact:construction3}, we have a path $\pi_{t}$ in $G_{\sim/\sim^+}^{\mathbb{D}_{\textrm{S}}}$ connecting $P_{i|t-1}$ and $P_i'$ such that every preference of $\pi_{t}$ has the peak $x$.
%Thus, we have a concatenated path in $G_{\sim/\sim^+}^{\mathbb{D}_{\textrm{S}}}$ connecting $P_i$ and $P_i'$:
%\begin{align*}
%\pi = &~ \big(P_i, \dots, P_{i|1},\dots, P_{i|2},\dots \dots \dots, P_{i|t-2},\dots, P_{i|t-1},\dots, P_i'\big),\\[-1.4em]
%& ~~~\,\underset{\pi_1}{\underbrace{\rule[0mm]{1.5cm}{0mm}}}~\underset{\pi_2}{\underbrace{\rule[0mm]{1.5cm}{0mm}}}
%~~~~~~~~~~~~~~~~~~~\underset{\pi_{t-1}}{\underbrace{\rule[0mm]{1.85cm}{0mm}}}\;
%\underset{\pi_{t}}{\underbrace{\rule[0mm]{1.7cm}{0mm}}}
%\end{align*}
%such that every preference of $\pi$ has the peak $x$.
%\end{proof}

Now, we are ready to show the no-detour condition on $\mathbb{D}_{\textrm{S}}$.
Given $P_i, P_i' \in \mathbb{D}_{\textrm{S}}$ and $a,b \in A$, let
$r_1(P_i)=(x^s, x^{-s})$, $r_1(P_i') = (y^s, x^{-s})$, $a = (a^s, x^{-s})$ and $b=(b^s, x^{-s})$, where
$x^s \neq y^s$ and $a^s \neq b^s$.
We construct a path $\pi = (P_{i|1}, \dots, P_{i|v})$ in $G_{\sim/\sim^+}^{\mathbb{D}}$ connecting $P_i$ and $P_i'$ such that $\pi$ has no $\{a,b\}$-restoration, and $r_1(P_{i|k}) \in (A^s, x^{-s})$ for all $k = 1, \dots, v$.
We assume w.l.o.g.~that $a\mathrel{P_i}b$.

First, we identify an arbitrary marginal preference $\hat{P}_i^s$ such that $r_1(\hat{P}_i^s) = a^s$.
Second, according to the separable preferences $P_i$ and $P_i'$,
we have the marginal preferences $P_i^t$ and $P_i^{t\,\prime}$ for all $t\in M\backslash \{s\}$.
Third, according to $\hat{P}_i^s$ and $P_i^t$ for all $t\in M\backslash \{s\}$,
we assemble a separable preference $P_{i|1}$ such that $P_{i|1}^s = \hat{P}_i^s$ and $P_{i|1}^t = P_i^t$ for all $t\in M\backslash \{s\}$.
Symmetrically, according to $\hat{P}_i^s$ and $P_i^{t\,\prime}$ for all $t\in M\backslash \{s\}$,
we assemble a separable preference $P_{i|2}$ such that $P_{i|2}^s = \hat{P}_i^s$ and $P_{i|2}^t = P_i^{t\,\prime}$ for all $t\in M\backslash \{s\}$.
Between $P_i$ and $P_{i|1}$,
note that either $P_i^s = \hat{P}_i^s = P_{i|1}^s$ or $P_i^s \neq P_i^{1\,\prime} = P_{i|1}^1$, and
$P_i^t = P_{i|1}^t$ for all $t \in M\backslash \{s\}$.
Then, by either Fact \ref{fact:construction1} (if $P_i^s = P_{i|1}^s$), or
Fact \ref{fact:construction3} (if $P_i^s \neq P_{i|1}^s$), we have a path $\hat{\pi}$ in $G_{\sim/\sim^+}^{\mathbb{D}_{\textrm{S}}}$ connecting $P_i$ and $P_{i|1}$ that has no $\{a,b\}$-restoration.
Moreover, if $P_i^s = P_{i|1}^s$, Fact \ref{fact:construction1} implies that every preference of $\hat{\pi}$ has the peak $(x^s, x^{-s})$, while
if $P_i^s \neq P_{i|1}^s$, Fact \ref{fact:construction3} implies that every preference of $\hat{\pi}$ has a peak in $(A^s, x^{-s})$.
Overall, every preference of $\hat{\pi}$ has a peak in $(A^s, x^{-s})$.
Similarly, between $P_{i|2}$ and $P_i'$,
since either $P_{i|2}^s = \hat{P}_i^s = P_i^{s\,\prime}$ or $P_{i|2}^s = \hat{P}_i^s \neq P_i^{s\,\prime}$, and
$P_{i|2}^t = P_i^{t\,\prime}$ for all $t \in M\backslash \{s\}$,
by either Fact \ref{fact:construction1} (if $P_{i|2}^s = P_i^{s\,\prime}$), or
Fact \ref{fact:construction3} (if $P_{i|2}^s \neq P_i^{s\,\prime}$), we have a path $\check{\pi}$ in $G_{\sim/\sim^+}^{\mathbb{D}_{\textrm{S}}}$ connecting $P_{i|2}$ and $P_i'$ such that $\check{\pi}$ has no $\{a,b\}$-restoration, and every preference of $\check{\pi}$ has a peak in $(A^s, x^{-s})$.
Between $P_{i|1}$ and $P_{i|2}$,
since $r_1(P_{i|1}) = r_1(P_{i|2}) = (a^s,x^{-s}) = a$,
by the Interior\textsuperscript{+} property, we have a path $\tilde{\pi}$ in $G_{\sim/\sim^+}^{\mathbb{D}_{\textrm{S}}}$ connecting $P_{i|1}$ and $P_{i|2}$ such that every preference has the peak $a$.
Thus, we have a concatenated path connecting $P_i$ and $P_i'$:
\begin{align*}
\pi = &~ \big(P_i, \dots, P_{i|1},\dots, P_{i|2},\dots, P_i'\big).\\[-1.4em]
& ~~~\,\underset{\hat{\pi}}{\underbrace{\rule[0mm]{1.5cm}{0mm}}}~\underset{\tilde{\pi}}{\underbrace{\rule[0mm]{1.4cm}{0mm}}}
~\underset{\check{\pi}}{\underbrace{\rule[0mm]{1.4cm}{0mm}}}
\end{align*}
It is clear that every preference of $\pi$ has a peak in $(A^s, x^{-s})$.
To complete the verification, we last show that $\pi$ has no $\{a,b\}$-restoration.
Since $a\mathrel{P_i}b$ and $r_1(P_{i|1}) = a$,
by no $\{a,b\}$-restoration on $\hat{\pi}$, we know that $a$ is ranked above $b$ in every preference of $\hat{\pi}$.
Furthermore, since every preference of $\tilde{\pi}$ has the peak $a$, it is evident that $a$ is ranked above $b$ in every preference of $\tilde{\pi}$.
Last, since $r_1(P_{i|2})=a$ and $\check{\pi}$ has no $\{a,b\}$-restoration,
we know that either $a$ is ranked above $b$ in every preference of $\check{\pi}$ (if $a\mathrel{P_i'}b$),
or the relative ranking of $a$ and $b$ has been switched exactly once on $\check{\pi}$ (if $b\mathrel{P_i'}a$).
Therefore, it is true that $\pi$ has no $\{a,b\}$-restoration.
This completes the verification of the clarification.
\end{proof}


\begin{clarification}\label{cla:MH}
The multidimensional hybrid domain $\mathbb{D}_{\emph{MH}}(\prec, \underline{x}, \overline{x})$ is a rich domain.
\end{clarification}

\begin{proof}
Let $\mathbb{D}_{\textrm{MH}}^{\ast}(\prec, \underline{x}, \overline{x})
= \mathbb{D}_{\textrm{MH}}(\prec, \underline{x}, \overline{x})\cap \mathbb{D}_{\textrm{S}}$
denote the domain of separable preferences that are also multidimensional hybrid on $\prec$ w.r.t.~$\underline{x}$ and $\overline{x}$.
Clearly,
$G_{\sim/\sim^+}^{\mathbb{D}_{\textrm{MH}}^{\ast}(\prec, \,\underline{x},\, \overline{x})} \subseteq
G_{\sim/\sim^+}^{\mathbb{D}_{\textrm{MH}}(\prec,\, \underline{x}, \,\overline{x})}$, and
both $\mathbb{D}_{\textrm{MH}}^{\ast}(\prec, \underline{x}, \overline{x})$ and $\mathbb{D}_{\textrm{MH}}(\prec, \underline{x}, \overline{x})$ satisfy minimal richness and diversity\textsuperscript{+}.
We first show that $\mathbb{D}_{\textrm{MH}}^{\ast}(\prec, \underline{x}, \overline{x})$ satisfies the Interior\textsuperscript{+} and Exterior\textsuperscript{+} properties.
The following three facts on $\mathbb{D}_{\textrm{MH}}^{\ast}(\prec, \underline{x}, \overline{x})$
are analogous to Facts \ref{fact:construction1}, \ref{fact:construction2} and \ref{fact:construction3} respectively.


\begin{fact}\label{fact:construction4}
Given distinct $P_i,P_i' \in \mathbb{D}_{\emph{MH}}^{\ast}(\prec, \underline{x}, \overline{x})$ such that $P_i^s = P_i^{s\,\prime}$ for all $s\in M$,
there exists a path in $G_{\sim}^{\mathbb{D}_{\emph{MH}}^{\ast}(\prec,\, \underline{x},\, \overline{x})}$ connecting $P_i$ and $P_i'$ that has no restoration for any pair of alternatives.
\end{fact}

\begin{proof}
Let $r_1(P_i) = r_1(P_i') = z$.
Searching from $z$ down to the bottom of $P_i$ and $P_i'$, since $P_i \neq P_i'$,
we can identify $k \in \{2, \dots, |A|\}$ such that $r_k(P_i) \neq r_k(P_i')$ and
$r_{\ell}(P_i) = r_{\ell}(P_i')$ for all $1 \leq \ell < k$.
For notational convenience, let $r_k(P_i') = y$ and $r_q(P_i) = y$.
Clearly, $q> k$. Moreover, let $r_{q-1}(P_i) = x$.
Since $P_i$ and $P_i'$ agree on the top-$(k-1)$ ranked alternatives, it is true that $y\mathrel{P_i'}x$.
Now, by locally switching the relative ranking of $x$ and $y$ in $P_i$, we generate a preference $P_i''$.
It is clear that $P_i \sim P_i''$, $x\mathrel{P_i!} y$ and $y\mathrel{P_i''!}x$.\medskip

\noindent
\textsc{Claim 1}: We have $P_i'' \in \mathbb{D}_{\textrm{MH}}(\prec, \underline{x}, \overline{x})$.
\medskip

Suppose by contradiction that $P_i'' \notin \mathbb{D}_{\textrm{MH}}(\prec, \underline{x}, \overline{x})$.
Since $P_i \in \mathbb{D}_{\textrm{MH}}(\prec, \underline{x}, \overline{x})$, by the construction of $P_i''$, it must be the case that $y\mathrel{P_i''}x$ violates the definition of multidimensional hybridness on $\prec$ w.r.t.~$\underline{x}$ and $\overline{x}$.
However, since $P_i'$ has the same peak as $P_i''$ and $y\mathrel{P_i'}x$,
this indicates that $P_i'$ violates the requirement of multidimensional hybridness on $\prec$ w.r.t.~$\underline{x}$ and $\overline{x}$ as well - a contradiction.
Therefore, $P_i'' \in \mathbb{D}_{\textrm{MH}}(\prec, \underline{x}, \overline{x})$.
This completes the verification of the claim.\medskip


\noindent
\textsc{Claim 2}: We have $P_i'' \in \mathbb{D}_{\textrm{MH}}^{\ast}(\prec, \underline{x}, \overline{x})$.
\medskip

By Claim 1, it suffices to show that $P_i''$ is a separable preference.
We first claim that $x$ and $y$ disagree on at least two components.
Otherwise, we have $x^s \neq y^s$ and $x^{-s} = y^{-s}$ for some $s \in M$.
Consequently, $x\mathrel{P_i}y$ and $y\mathrel{P_i'}x$ imply $x^s\mathrel{P_i^s}y^s$ and
$y^s\mathrel{P_i^{s\,\prime}}x^s$ - a contradiction.
Since $x\mathrel{P_i}y$, there must exist $s \in M$ such that $x^s\mathrel{P_i^s}y^s$.
Next, we claim that there exists $t \in M\backslash \{s\}$ such that $y^t\mathrel{P_i^t}x^t$.
Otherwise, we have $x^t\mathrel{P_i^t}y^t$ for all $t \in M(x,y)$.
Fixing two components $s,t \in M(x,y)$,
we assemble another alternative $\hat{x} \coloneqq (x^s, y^t, x^{-\{s,t\}})$.
Clearly, $\hat{x} \notin \{x,y\}$.
However, by separability, we have $x\mathrel{P_i} \hat{x}$ and $\hat{x}\mathrel{P_i}y$,
which contradicts the fact $x\mathrel{P_i!}y$.
Thus, we can claim that the local switching of $x$ and $y$ in $P_i$ does not generate any violation of the separability requirement.
Therefore, $P_i''$ is a separable preference.
This completes the verification of the claim.\medskip

Now, notice that (i) $P_i''$ and $P_i'$ agree on the relative ranking of $x$ and $y$,
(ii) for any pair of alternatives other than $x$ and $y$ that are oppositely ranked across $P_i$ and $P_i'$, they are still oppositely ranked across $P_i''$ and $P_i'$, and
(iii) for any pair of alternatives that are identically ranked across $P_i$ and $P_i'$, they are still identically ranked across $P_i''$ and $P_i'$.
Therefore, we can say that $P_i''$ is closer to $P_i'$ than $P_i$.
By repeatedly applying the procedure above, we eventually can construct a path $\pi$ in $G_{\sim}^{\mathbb{D}_{\textrm{MH}}^{\ast}(\prec,\, \underline{x},\, \overline{x})}$ connecting $P_i$ and $P_i'$.
More importantly, by construction, it is evident that $\pi$ has no restoration for any pair of alternatives.
\end{proof}


\begin{fact}\label{fact:construction5}
For each $t \in M$, fix a marginal preference $P_i^t$ that is hybrid on $\prec^t$ w.r.t.~$\underline{x}^t$ and $\overline{x}^t$.
Fix an alternative $a \in A$ and a component $s\in M$.
There exists $\hat{P}_i \in \mathbb{D}_{\emph{MH}}^{\ast}(\prec, \underline{x}, \overline{x})$ satisfying the following three conditions:
\begin{enumerate}
\item $\hat{P}_i^t = P_i^t$ for all $t \in M$,

\item $[x \mathrel{\hat{P}_i} a] \Leftrightarrow [x^t \in \overline{U}(a^t, P_i^t)\; \textrm{for all}\; t \in M]$.

\item for all $b^s, c^s \in \overline{U}(a^s, P_i^s)$ or $b^s, c^s \in L(a^s, P_i^s)$,\\
$[b^s \mathrel{P_i^s!} c^s] \Rightarrow [(b^s, y^{-s})\mathrel{\hat{P}_i!}(c^s, y^{-s})\; \textrm{for all}\; y^{-s} \in A^{-s}]$.
\end{enumerate}
\end{fact}

\begin{proof}
First, by Fact \ref{fact:construction2}, we have a separable preference $\hat{P}_i$ satisfying conditions 1 - 3.
Next, since $\hat{P}_i^t = P_i^t$ is a hybrid marginal preference on $\prec^s$ w.r.t.~$\underline{x}^t$ and $\overline{x}^t$ for all $t \in M$, it is evident that the separable preference $\hat{P}_i$ is multidimensional hybrid on $\prec$ w.r.t.~$\underline{x}$ and $\overline{x}$.
Therefore, $\hat{P}_i \in \mathbb{D}_{\textrm{MH}}^{\ast}(\prec, \underline{x}, \overline{x})$.
\end{proof}

\begin{fact}\label{fact:construction3'}
Given $P_i, P_i' \in \mathbb{D}_{\emph{MH}}^{\ast}(\prec, \underline{x}, \overline{x})$ and $s \in M$,
let $P_i^s \neq P_i^{s\,\prime}$, $P_i^t = P_i^{t\,\prime}$ for all $t \in M\backslash \{s\}$,
$r_1(P_i) = (x^s, x^{-s})$ and $r_1(P_i') =(y^s, x^{-s})$ where either $x^s = y^s$ or $x^s \neq y^s$.
There exists a path $(P_{i|1}, \dots, P_{i|v})$ in $G_{\sim/\sim^+}^{\mathbb{D}_{\emph{MH}}^{\ast}(\prec,\, \underline{x},\, \overline{x})}$ connecting $P_i$ and $P_i'$ that has no restoration for any pair of alternatives $a,b \in (A^s, x^{-s})$, and satisfies the following two additional conditions:
\begin{itemize}
\item[\rm (i)] if $x^s = y^s$, then $r_1(P_{i|k}) = (x^s, x^{-s})$ for all $k=1, \dots, v$, and

\item[\rm (ii)] if $x^s \neq y^s$, then $r_1(P_{i|k}) \in (A^s, x^{-s})$ for all $k=1, \dots, v$.
\end{itemize}
\end{fact}

\begin{proof}
We can still apply the proof of Fact \ref{fact:construction3} to verify this Fact.
We only need to make the following three changes in the proof of Fact \ref{fact:construction3}:\\
(i) Change the reference of the universal marginal domain $\mathbb{P}^s$ to $[\mathbb{D}_{\textrm{MH}}^{\ast}(\prec, \underline{x}, \overline{x})]^s$ which contains all hybrid marginal preferences on $\prec^s$ w.r.t.~$\underline{x}^s$ and $\overline{x}^s$.
Also change the notation $G_{\sim}^{\mathbb{P}^s}$ to $G_{\sim}^{[\mathbb{D}_{\textrm{MH}}^{\ast}(\prec,\, \underline{x},\, \overline{x})]^s}$.
By the proof of Clarification 1 of \citet{CRSSZ2022}, we know that every pair of distinct marginal preferences of $[\mathbb{D}_{\textrm{MH}}^{\ast}(\prec, \underline{x}, \overline{x})]^s$ are connected by a path in $G_{\sim}^{[\mathbb{D}_{\textrm{MH}}^{\ast}(\prec,\, \underline{x},\, \overline{x})]^s}$ that has no restoration for any pair of elements.\\
(ii) Change the reference of Fact \ref{fact:construction1} and \ref{fact:construction2} to Fact \ref{fact:construction4} and \ref{fact:construction5} respectively.\\
(iii) Change the notation $G_{\sim}^{\mathbb{D}_{\textrm{S}}}$ and $G_{\sim/\sim^+}^{\mathbb{D}_{\textrm{S}}}$ to $G_{\sim}^{\mathbb{D}_{\textrm{MH}}^{\ast}(\prec,\, \underline{x},\, \overline{x})}$ and
    $G_{\sim/\sim^+}^{\mathbb{D}_{\textrm{MH}}^{\ast}(\prec,\, \underline{x},\, \overline{x})}$ respectively.
\end{proof}

Now, we are ready to show that $\mathbb{D}_{\textrm{MH}}^{\ast}(\prec, \underline{x}, \overline{x})$ satisfies the Interior\textsuperscript{+} and Exterior\textsuperscript{+} properties.
First, based on Facts \ref{fact:construction4} and \ref{fact:construction5}
which are analogous to Lemmas 2 and 3 of \citet{KRSYZ2021a}, by applying the same proof of Proposition 2 of \citet{KRSYZ2021a},
we can claim that $\mathbb{D}_{\textrm{MH}}^{\ast}(\prec, \underline{x}, \overline{x})$ satisfies Property $L$ on $G_{\sim/\sim^+}^{\mathbb{D}_{\textrm{MH}}^{\ast}(\prec,\, \underline{x},\, \overline{x})}$:
given distinct $P_i, P_i' \in \mathbb{D}_{\textrm{MH}}^{\ast}(\prec, \underline{x}, \overline{x})$ and $a \in A$, there exists a path in $G_{\sim/\sim^+}^{\mathbb{D}_{\textrm{MH}}^{\ast}(\prec,\, \underline{x},\, \overline{x})}$ connecting $P_i$ and $P_i'$ that has no $\{a,b\}$-restoration for any $b \in L(a, P_i)$.
Of course, Property $L$ also implies the Interior\textsuperscript{+} property and the Exterior\textsuperscript{+} property without the fulfillment of the no-detour condition.
Furthermore, by applying the proof of Clarification \ref{cla:S}, we can also claim that $\mathbb{D}_{\textrm{MH}}^{\ast}(\prec, \underline{x}, \overline{x})$ the no-detour condition of Exterior\textsuperscript{+} property.

Next, we turn to the multidimensional hybrid domain $\mathbb{D}_{\textrm{MH}}(\prec, \underline{x}, \overline{x})$.
We first provide a fact on $\mathbb{D}_{\textrm{MH}}(\prec, \underline{x}, \overline{x})$ that is analogous to Fact \ref{fact:construction4}.

\begin{fact}\label{fact:construction6}
Given distinct $P_i,P_i' \in \mathbb{D}_{\emph{MH}}(\prec, \underline{x}, \overline{x})$ such that $r_1(P_i) = r_1(P_i')$,
there exists a path in $G_{\sim}^{\mathbb{D}_{\emph{MH}}(\prec,\, \underline{x},\, \overline{x})}$  connecting $P_i$ and $P_i'$ that has no restoration for any pair of alternatives.
\end{fact}

\begin{proof}
After removing Claim 2 from the proof of Fact \ref{fact:construction4},
the remaining proof of Fact \ref{fact:construction4} can be adopted for the verification.
\end{proof}

It is clear that Fact \ref{fact:construction6} implies the Interior\textsuperscript{+} property.

Next, given $P_i, P_i' \in \mathbb{D}_{\textrm{MH}}(\prec, \underline{x}, \overline{x})$ such that $r_1(P_i) \neq r_1(P_i')$ and distinct $a, b \in A$, we construct a path in $G_{\sim/\sim^+}^{\mathbb{D}_{\textrm{MH}}(\prec,\, \underline{x}, \, \overline{x})}$ that connects $P_i$ and $P_i'$, and show that it has no $\{a,b\}$-restoration.
For notational convenience, let $r_1(P_i) = x$ and $r_1(P_i') = y$.
The construction of the path consists of three steps.
\begin{description}
\item[\sc Step 1.]
For each $t \in M$,
we identify a hybrid marginal preference $\hat{P}_i^t$ on $\prec^t$ w.r.t.~$\underline{x}^t$ and $\overline{x}^t$ such that
$r_1(\hat{P}_i^t) = x^t$ and for all $z^t \in A^t\backslash \{x^t\}$, $\big[z^t\mathrel{\hat{P}_i^t} a^t\big]
\Leftrightarrow
\big[z^t \in \textrm{Int}\langle x^t, a^t\rangle\;\textrm{and}\; z^t\notin \textrm{Int}\langle \underline{x}^t, \overline{x}^t\rangle\big]$.
Then, according to the marginal preferences $\hat{P}_i^1, \dots, \hat{P}_i^m$,
the alternative $a$ and an arbitrary component $s \in M$,
we have a preference $\hat{P}_i \in \mathbb{D}_{\textrm{MH}}^{\ast}(\prec, \underline{x}, \overline{x})$ satisfying conditions 1 - 3 of Fact \ref{fact:construction5}.
It is clear that $r_1(\hat{P}_i) = r_1(P_i)$.
Then, by Fact \ref{fact:construction6}, we have a path $\hat{\pi}$ in $G_{\sim/\sim^+}^{\mathbb{D}_{\textrm{MH}}(\prec,\, \underline{x},\, \overline{x})}$ connecting $P_i$ and $\hat{P}_i$ that has no restoration for any pair of alternatives.

\item[\sc Step 2.]
For each $t \in M$,
we identify a hybrid marginal preference $\check{P}_i^t$ on $\prec^t$ w.r.t.~$\underline{x}^t$ and $\overline{x}^t$ such that
$r_1(\check{P}_i^t) = y^t$ and for all $z^t \in A^t\backslash \{y^t\}$, $\big[z^t\mathrel{\check{P}_i^t} a^t\big]\Leftrightarrow \big[z^t \in \textrm{Int}\langle y^t, a^t\rangle\;\textrm{and}\; z^t\notin \textrm{Int}\langle \underline{x}^t, \overline{x}^t\rangle\big]$.
Then, according to the marginal preferences $\check{P}_i^1, \dots, \check{P}_i^m$,
the alternative $a$ and an arbitrary component $s \in M$,
we have a preference $\check{P}_i \in \mathbb{D}_{\textrm{MH}}^{\ast}(\prec, \underline{x}, \overline{x})$ satisfying conditions 1 - 3 of Fact \ref{fact:construction5}.
It is clear that $r_1(\check{P}_i) = r_1(P_i')$.
Then, by Fact \ref{fact:construction6}, we have a path $\check{\pi}$ in $G_{\sim/\sim^+}^{\mathbb{D}_{\textrm{MH}}(\prec,\, \underline{x},\, \overline{x})}$ connecting $\check{P}_i$ and $P_i'$ that has no restoration for any pair of alternatives.

\item[\sc Step 3.]
Since $\hat{P}_i,\check{P}_i \in \mathbb{D}_{\textrm{MH}}^{\ast}(\prec, \underline{x}, \overline{x})$ and $r_1(\hat{P}_i) = x \neq y = r_1(\check{P}_i)$, by the Exterior\textsuperscript{+} property on $\mathbb{D}_{\textrm{MH}}^{\ast}(\prec, \underline{x}, \overline{x})$,
we have a path $\tilde{\pi}$ in
$G_{\sim/\sim^+}^{\mathbb{D}_{\textrm{MH}}^{\ast}(\prec, \,\underline{x}, \,\overline{x})} \subseteq
G_{\sim/\sim^+}^{\mathbb{D}_{\textrm{MH}}(\prec, \, \underline{x}, \,\overline{x})}$
that connects $\hat{P}_i$ and $\check{P}_i$, and has no $\{a,b\}$-restoration.
In particular, if $r_1(\hat{P}_i) = (x^s, x^{-s})$, $r_1(\check{P}_i) = (y^s, x^{-s})$, $a = (a^s, x^{-s})$ and $b =(b^s, x^{-s})$ for some $s \in M$ and $x^{-s}\in A^{-s}$,
where $x^s \neq y^s$ and $a^s \neq b^s$,
by the Exterior\textsuperscript{+} property on $\mathbb{D}_{\textrm{MH}}^{\ast}(\prec, \underline{x}, \overline{x})$,
we further require that every preference of $\tilde{\pi}$ has a peak in $(A^s, x^{-s})$.
Thus, we have a concatenated path in $G_{\sim/\sim^+}^{\mathbb{D}_{\textrm{MH}}(\prec, \, \underline{x}, \,\overline{x})}$ that connects $P_i$ and $P_i'$:
\begin{align*}
\begin{array}{ll}
\pi = & \big(P_i, \dots, \hat{P}_i,\dots, \check{P}_i,\dots, P_i'\big).\\[-1.2em]
& ~~\,\underset{\hat{\pi}}{\underbrace{\rule[0mm]{1.3cm}{0mm}}}\;\underset{\tilde{\pi}}{\underbrace{\rule[0mm]{1.2cm}{0mm}}}
\;\underset{\check{\pi}}{\underbrace{\rule[0mm]{1.25cm}{0mm}}}
\end{array}
\end{align*}
\end{description}

We assume w.l.o.g.~that $a\mathrel{P_i}b$.
We show that $\pi$ has no $\{a,b\}$-restoration.
First, since $a\mathrel{P_i}b$, by multidimensional hybridness on $\prec$ w.r.t.~$\underline{x}$ and $\overline{x}$, we know that there exists $t \in M(a,b)$ such that $b^t \neq x^t$ and either $b^t \notin \textrm{Int}\langle x^t, a^s\rangle$ or $b^t \in \textrm{Int}\langle \underline{x}^t, \overline{x}^t\rangle$.
This implies $b^t \notin \overline{U}(a^t, \hat{P}_i^t)$.
Consequently, by the construction of $\hat{P}_i$ in Step 1, we know $a\mathrel{\hat{P}_i}b$.
Then, by no restoration on $\hat{\pi}$, we know that $a$ is ranked above $b$ in every preference of $\hat{\pi}$.
Second, there are two cases for $\check{P}_i$: (i) $a\mathrel{\check{P}_i}b$ and
(ii) $b\mathrel{\check{P}_i}a$.
In case (i), since $a\mathrel{\hat{P}_i}b$ and $a\mathrel{\check{P}_i}b$,
by no $\{a,b\}$-restoration on $\tilde{\pi}$, we know that $a$ is ranked above $b$ in every preference of $\tilde{\pi}$.
Furthermore, since $a\mathrel{\check{P}_i}b$, by no restoration on $\check{\pi}$, we know that either $a$ is ranked above $b$ in every preference of $\check{\pi}$ (if $a\mathrel{P_i'}b$),
or the relative ranking of $a$ and $b$ has been switched exactly once on $\check{\pi}$ (if $b \mathrel{P_i'} a$).
Therefore, $\pi$ has no $\{a,b\}$-restoration.
In case (ii), since $a\mathrel{\hat{P}_i}b$ and $b\mathrel{\check{P}_i}a$,
by no $\{a,b\}$-restoration on $\tilde{\pi}$, we first know that the relative ranking of $a$ and $b$ has been switched exactly once on $\tilde{\pi}$.
Furthermore, since $b\mathrel{\check{P}_i}a$,
by the construction of $\check{P}_i$ is Step 2,
it must be the case that for each $t \in M(a,b)$, either $b^t = y^t$, or $b^t \in \textrm{Int}\langle y^t, a^t\rangle$ and $b^t\notin \textrm{Int}\langle \underline{x}^t, \overline{x}^t\rangle$.
Consequently, since $r_1(P_i') = y$, by multidimensional hybridness on $\prec$ w.r.t.~$\underline{x}$ and $\overline{x}$, it is true that $b\mathrel{P_i'}a$.
Correspondingly, by no restoration on $\check{\pi}$, we know that $b$ is ranked above $a$ in every preference of $\check{\pi}$.
Therefore, $\pi$ has no $\{a,b\}$-restoration, as required.

Last, we show the no-detour condition of the Exterior\textsuperscript{+} property on $\mathbb{D}_{\textrm{MH}}(\prec, \underline{x}, \overline{x})$.
Given $P_i, P_i' \in \mathbb{D}_{\textrm{MH}}(\prec, \underline{x}, \overline{x})$ and $a,b \in A$,
let $r_1(P_i) = (x^s, x^{-s})$, $r_1(P_i') = (y^s, x^{-s})$, $a=(a^s, x^{-s})$ and $b=(b^s, x^{-s})$, where $x^s \neq y^s$ and $a^s \neq b^s$.
By the three steps above, we have a path $\pi\coloneqq (\hat{\pi}, \tilde{\pi}, \check{\pi})$ in
$G_{\sim/\sim^+}^{\mathbb{D}_{\textrm{MH}}(\prec,\,\underline{x},\,\overline{x})}$ connecting $P_i$ and $P_i'$ that has no $\{a,b\}$-restoration.
To complete the verification, we show that every preference of $\pi$ here has a peak in $(A^s, x^{-s})$.
In Step 1, since $r_1(P_i) = r_1(\hat{P}_i) = (x^s, x^{-s})$,
by Fact \ref{fact:construction6},
we know that every preference of $\hat{\pi}$ has the peak $(x^s, x^{-s})$.
Symmetrically, in Step 2, since $r_1(\check{P}_i) = r_1(P_i') = (y^s, x^{-s})$,
by Fact \ref{fact:construction6}, every preference of $\check{\pi}$ has the peak $(y^s, x^{-s})$.
Last, in Step 3, we know that every preference of $\tilde{\pi}$ has a peak in $(A^s, x^{-s})$.
Therefore, every preference of $\pi$ has a peak in $(A^s, x^{-s})$, as required.
This completes the verification of the Clarification.
\end{proof}






\end{document}
