\documentclass[12 pt, article]{article}
\usepackage[left=1in, right=1in, top=0.75in, bottom=0.75in]{geometry}
\usepackage[utf8]{inputenc}
\usepackage{setspace}
\usepackage{graphicx}
\usepackage{authblk}
\usepackage{sectsty}
\usepackage{gensymb}
\usepackage{parskip}
\usepackage{caption}
\usepackage{titlesec}
\usepackage{wrapfig}
\usepackage{xcolor}
\usepackage[utf8]{inputenc}
\graphicspath{{images/}}
\sectionfont{\fontsize{13}{16}\selectfont}
\renewcommand\Authfont{\fontsize{12}{14.4}\selectfont}
\renewcommand\Affilfont{\fontsize{9}{10.8}\itshape}
\captionsetup{justification=justified,singlelinecheck=false}
\parskip = 12pt
\titlespacing\section{0pt}{12pt plus 2pt minus 2pt}{12pt plus 2pt minus 2pt}
\usepackage[backend=biber, style=phys]{biblatex}
\addbibresource{references.bib}
\captionsetup[figure]{justification=justified,singlelinecheck=false,labelfont={bf},name={Figure}}

\newcommand{\ringonemacro}{$I_{0\overline{1}10} + I_{\overline{1}010} + I_{\overline{1}100}\,$ }
\newcommand{\ringtwomacro}{$I_{2\overline{1}10} + I_{\overline{2}110} + I_{1\overline{2}10}\,$ }

\title{\Large\textbf {Local atomic stacking and symmetry in twisted graphene trilayers}}

\author[1]{Isaac M. Craig}
\author[1]{Madeline Van Winkle}
\author[1]{Catherine Groschner}
\author[1]{Kaidi Zhang}
\author[1]{Nikita Dowlatshahi}
\author[2]{Ziyan Zhu}
\author[3]{Takashi Taniguchi}
\author[4]{Kenji Watanabe}
\author[5,6]{Sinéad M. Griffin}
\author[1,7,*]{D. Kwabena Bediako}
\affil[1]{\textit{Department of Chemistry, University of California, Berkeley, CA 94720, USA}}
\affil[2]{\textit{SLAC National Accelerator Laboratory, Stanford, CA, USA}}
\affil[3]{\textit{International Center for Materials Nanoarchitectonics, National Institute for Materials Science, 1-1 Namiki, Tsukuba 305-0044, Japan}}
\affil[4]{\textit{Research for Functional Materials, National Institute for Materials Science, 1-1 Namiki, Tsukuba 305-0044, Japan}}
\affil[5]{\textit{Molecular Foundry, Lawrence Berkeley National Laboratory, Berkeley, CA 94720, USA}}
\affil[6]{\textit{Materials Sciences Division, Lawrence Berkeley National Laboratory, Berkeley, CA 94720, USA}}
\affil[7]{\textit{Chemical Sciences Division, Lawrence Berkeley National Laboratory, Berkeley, CA 94720, USA}}
\affil[*]{Correspondence to: bediako@berkeley.edu}

\date{}
\begin{document}
\maketitle

\doublespacing
\textbf{Abstract}

Moiré superlattices formed from twisting trilayers of graphene are an ideal model for studying electronic correlation, and offer several advantages over bilayer analogues, including more robust and tunable superconductivity and a wide range of twist angles associated with flat band formation. Atomic reconstruction, which strongly impacts the electronic structure of twisted graphene structures, has been suggested to play a major role in the relative versatility of superconductivity in trilayers. Here, we exploit an inteferometric 4D-STEM approach to image a wide range of trilayer graphene structures. Our results unveil a considerably different model for moiré lattice relaxation in trilayers than that proposed from previous measurements, informing a thorough understanding of how reconstruction modulates the atomic stacking symmetries crucial for establishing superconductivity and other correlated phases in twisted graphene trilayers. 

\newpage
Since their discovery, graphene-based moiré superlattices have proven extraordinarily useful for exploring the interplay between crucial parameters in strongly correlated materials \cite{balents2020superconductivity, cao2018correlated,cao2018unconventional,lau2022reproducibility,tian2023evidence}. The advantage of these systems lies in the ability to realize a disparate set of  physical phenomena through adjusting carrier density, layer number, and interlayer stacking. Trilayer graphene in particular, displays profoundly different physics with different local stacking arrangements \cite{bao2011stacking, kumar2011integer, lui2011observation, yuan2011landau}. Bernal trilayer graphene (ABA stacking) is a semi-metal with a tunable band gap \cite{craciun2009trilayer}, and contains a band structure containing poorly coupled bands resembling a superimposition of that of monolayer and bilayer graphene. Rhombohedral trilayer graphene (ABC stacking), which differs from the Bernal analogue by only a uniform translation of one layer, instead displays hybridization between all three layers and properties such as metallic behavior \cite{zhou2021half}, Mott insulating states \cite{chen2019evidence}, and superconductivity \cite{zhou2021superconductivity}. Differences in lattice symmetry have been proposed to play a large role in the disparate correlated physics of few-layer graphene systems, both in these high symmetry Bernal and rhombodedral structures, and within the locally ordered domains of twisted structures more generally \cite{koshino2009gate, morell2013electronic}.

In particular, previous work has suggested that the flat-bands in twisted bilayer graphene (TBLG) are a fragile topological phase protected by the space-time $C_{2z}T$ inversion symmetry \cite{ahn2019failure, song2019all}. This protection is maintained although the atomic lattice is only locally inversion symmetric where the carbon atoms of each layer are vertically aligned (AA stacking) \cite{zou2018band}. Further, the overlap between flatband states localized within these AA regions (and therefore also the size of AA stacking regions) is thought to play a large role in the superconducting current and transition temperature of moiré systems \cite{torma2021superfluidity}. The flat-bands responsible for correlated phenomena in twisted graphene multi-layers also appear to rely on this same $C_{2z}T$ symmetry \cite{mora2019flatbands}. Like their bilayer analogues, twisted trilayer graphene (TTLG) structures are only locally inversion symmetric about pockets of AAA-type stacking. Results showing that superconductivity in magic angle twisted trilayer graphene (MATTLG) is more robust than in bilayers \cite{park2021tunable, hao2021electric, kim2022evidence, zhang2021correlated} have therefore prompted questions regarding the different roles spontaneous lattice relaxation (also termed atomic reconstruction) may play in these two systems. In particular, prior imaging has shown that atomic reconstruction in TBLG decreases the relative portion of AA stacking \cite{yoo2019atomic,kazmierczak2021strain}. It has been proposed that atomic relaxation in trilayers may instead increase the portion of inversion symmetric stackings \cite{turkel2022orderly}, motivating precise characterization of twisted graphene trilayers structures and the general mechanism by which they reconstruct.

\begin{figure*}[bthp]
    \centerline{\includegraphics[width=\textwidth]{Main_Figures/M1.png}}
    \caption{\textbf{Interferometric 4D-STEM dark field imaging of selected interfaces.} (\textbf{A}) Schematic of the 4D-STEM approach, wherein beam interference is used to extract stacking order. (\textbf{B}) Schematic illustrating the twist angle, $\theta$, and layer numbering conventions used to label the graphene trilayers. (\textbf{C}) Illustrations of various high-symmetry stacking configurations realized within twisted trilayer moirés. (\textbf{D,E}) Average convergent beam electron diffraction patterns for trilayers with $\theta_{13} \approx 0\degree$ \textbf{(D)} and $\theta_{13} = 0.22^{\degree}$ \textbf{(E)}. Overlapping TTLG Bragg disks are highlighted in the insets. \textbf{(F,H)} Virtual dark field images corresponding to the overlap of layers 1 \& 3. \textbf{(G,I)} Virtual dark field images corresponding to the overlap of all three layers. Scale bars are 1 nm$^{-1}$ and 25 nm for reciprocal (\textbf{D,E}) and real space (\textbf{F--I}), respectively.}
\end{figure*}

Here, we use a four-dimensional scanning transmission electron microscopy, 4D-STEM \cite{ophus2019four}, methodology termed Bragg interferometry (Fig. 1A) that exploits local interference patterns from diffracted electron beams to precisely determine the stacking orientation of atomic layers \cite{kazmierczak2021strain, van2022quantitative}. This technique not only allows us to probe moiré patterns within encapsulated materials, difficult in scanning tunneling microscopy (STM) as well as other STEM approaches, but further allows us to selectively map the stacking orientation of individual bilayer interfaces within complex multilayered materials. These results provide a direct measure of the local stacking within the material. We find that the results of this 4D-STEM measurement suggest a picture of reconstruction that is markedly different from that proposed by previous STM work \cite{turkel2022orderly}, and one that is consequential for understanding the correlated electron physics in these materials.

The interferometric 4D-STEM technique we use involves scanning a converged electron beam across the sample of interest and collecting individual diffraction patterns for each real space position of the probe (Fig. 1A). Throughout this work, we use the notation shown in (Fig. 1B) to label the twist angles, $\theta$, within the trilayer sample. Here, $\theta_{12}$ and $\theta_{23}$ denote the twist angles between layers 1 \& 2, and layers 2 \& 3 respectively such that $\theta_{13}$ = $\theta_{12} + \theta_{23}$. We further use the labels shown in Fig. 1C to denote the various high symmetry stacking configurations realized within the moiré. The converged beam electron diffraction (CBED) patterns collected at each probe location then appear as shown in Figs. 1D,E, where each layer of the material generates a set of Bragg disks. The overlap between Bragg disks originating from different layers (inset Figs. 1D,E) is then used to determine the stacking orientation of those two layers. As an example, Figs. 1F--H show how the intensity of the overlap between layers 1 and 3—denoted as `(1,3) overlap'—varies across the sample. This modulation in intensity directly manifests the moiré pattern between layers 1 and 3 but is insensitive to the orientation of the second layer. Similarly the variation in the intensity of the (1,2,3) overlap region (Figs. 1I,J) reveals the modulation in stacking order between all three graphene layers. 

Therefore, by exploiting the relationship between stacking order and overlap region intensity, we map the variation in atomic stacking and hence reconstruction within trilayer graphene samples. Results of these analyses are shown in Fig. 2 for a structure that we call `AtA' in which the top and bottom graphene layers are perfectly aligned to each-other and twisted with respect to the middle layer (Fig. 2A). In this structure, the average intensity of  the overlap regions in the first ring of Bragg reflections and the average overlap region intensity in the second ring of reflections can be used to determine the local stacking configuration. Using the bi-variate color-legend shown in Fig. 2B (in which the high symmetry stacking configurations associated with each color are overlaid), we create a map of local atomic stacking within an AtA sample (Fig. 2B). 

\begin{figure*}[btp]
    \centerline{\includegraphics[width=\textwidth]{Main_Figures/M2.png}}
    \caption{\textbf{Reconstruction in AtA and tAB trilayers. }(\textbf{A, G}) schematic illustrating the layer alignment in a AtA and a tAB trilayer. (\textbf{B}) Legends illustrating how color correlates with the average first and second order Bragg disks intensities. Overlaid points are the intensities of high symmetry stacking orders obtained via multislice. (\textbf{C, H}) Maps of local stacking order for AtA and tAB trilayers. (\textbf{D, I}) Histograms illustrating the relative prevalence of each stacking configuration. Note that since the intensity does not depend linearly on the stacking order, a rigid bilayer will not display a uniform distribution of intensities. (\textbf{E, J}) Schematics illustrating the anticipated variation in local stacking order expected for a rigid structure. (\textbf{F, K}) Intensity line-cuts corresponding to the dotted lines shown in (\textbf{C, H}) are given as solid lines. Intensity variation expected for a rigid structure are given as dotted lines. Domain sizes are calculated from the full width at half max of \ringonemacro (red) and \ringtwomacro (black) as highlighted. All scale bars are 25 nm. }
    
\end{figure*}


The local atomic stacking shown in Fig. 2C indicates that this particular AtA sample for which ($\theta_{12}=1.06\degree$) relaxes to decrease the total amount of AAA stacking (white) when compared to the stacking order distribution in a rigid AtA trilayer (Fig. 2E). This is expected as AAA stacking is roughly 29.5 and 36.5 meV/unit cell higher in energy than A-SP-A and ABA stacking respectively. \cite{zhu2020twisted} Further, the histogram shown (Fig. 2D) illustrates that this sample contains considerably more ABA, BAB, and SP type stacking than AAA type stacking. This reconstruction also manifests in the line-cut shown in Fig. 2F, which corresponds to the dotted line in Fig. 2C. From this profile, it is evident that the widths of the AAA (BAB/ABA) regions are notably smaller (larger) than expected for a rigid trilayer, even when accounting for noise-driven differences in normalization.

These results illustrate that AtA trilayers show an observably reconstructed atomic stacking distribution up to a at least a twist angle of $1.81\degree$, with few differences seen between the $1.81\degree$ and $1.0\degree$ samples. We also observe that the AtA trilayers show a pattern of reconstruction similar to that of a twisted bilayer within this twist angle range \cite{yoo2019atomic,kazmierczak2021strain}, although a more quantitative comparison between the bilayer and AtA trilayer reconstruction necessitates more detailed intensity fitting approach to map strain tensor fields \cite{kazmierczak2021strain,van2022quantitative} and will be addressed in future work. 

A similar analysis is carried out for samples which we refer to as `tAB', in which the bottom two graphene layers are aligned AB and the top layer is twisted (Fig. 2G), creating a structure sometimes referred to as a `monolayer-on-bilayer graphene', which also exhibits correlated electron physics \cite{tong2022spectroscopic,park2020gate}. Unlike the AtA trilayers, these tAB structures show atomic reconstruction patterns driven by a preference to increase the relative portion of ABC and ABA stacking, as seen by comparing Fig. 2H to the rigid stacking order distribution (Fig. 2J), the histogram in Fig. 2I, and the corresponding line-cut in Fig. 2K. This increase in ABC and ABA stacking is expected, as they are 17.9 meV/unit cell lower in energy than ABB. In addition, Fig. 2K illustrates the tAB structure reconstructs such that the portions of ABA and ABB stacking are no longer equivalent as they would be in a rigid moiré. We note here that this manifests in the ABB regions appearing to have a lower peak \ringonemacro intensity to the ABA regions, while these regions are expected to appear sharper but with similar maximum intensity. This is likely due to broadening from a number of factors associated with measurement acquisition and post-processing, especially the beam-width biasing and the Gaussian filter used. 

The observation that ABA and ABB regions are observably distinct even at a twist angle of $1.4\degree$ is nonetheless notable. This effect is more dramatic at smaller twist angles as seen in the stacking order percent area trends and maps gathered within the $0.1\degree$ -- $1.5\degree$ twist angle regime. While the approach used in this work prohibits a quantitative comparison of the ABB and ABA domain sizes (to be addressed in future work), these results still clearly establish that the size of ABC and ABA domains are comparable to each-other and much larger than the ABB domains. Moreover, the stacking order distributions seen for tAB appear similar to those observed in twisted bilayer graphene \cite{kazmierczak2021strain}, suggesting that atomic reconstruction in tAB trilayer samples can be explained primarily from consideration of the twisted interface. 

\begin{figure*}[tbhp]
    \centerline{\includegraphics[width=175mm]{Main_Figures/M3.png}}
    \caption{\textbf{Atomic Stacking in slightly misaligned TTLG.} (\textbf{A}) Maps of local atomic stacking from the larger moiré pattern only, corresponding to the local in-plane offset between between layers 1 and 3 in panels 1-3, and the local in-plane offset between layers 1 and 2 in panel 4. (\textbf{B}) Local atomic stacking obtained from considering all three graphene layers. (\textbf{C}) Simulated stacking order maps for rigid moiré superlattice analogues of \textbf{B}. All scale bars are 25 nm. (\textbf{D,F}) Schematics of layer alignment in TTLG with slightly misaligned layers. (\textbf{E}) Zoom-ins of the maps above illustrating the finer local modulation of stacking order within (\textbf{1}) AtA, (\textbf{2}) AtSP, (\textbf{3}) AtB, (\textbf{4}) tAA, (\textbf{5-6}) tAB regions.}   
\end{figure*}

The results discussed thus far concern a limiting case of graphene trilayers wherein two of the layers are perfectly aligned. While these materials are conceptually simpler and display rich physical properties which merit their investigation, this interferometric 4D-STEM technique also permits us to study a broader array of twisted trilayer structures with two independent twist angles. In these more complex multilayered samples, the ability to selectively probe buried bilayer interfaces allows us to independently image the larger scale moiré pattern and evaluate its effect on local stacking order. 

Following this approach, we extract stacking order maps associated with the larger moiré pattern from double overlap (Fig. 3A) and triple overlap (Fig. 3B,E). These are compared to the maps calculated for rigid moirés in Fig. 3C. For the the double overlap case, Fig. 3A reveals the presence of large local regions in which two layers are aligned directly atop each-other (AtA or tAA, white) and regions in which two layers are aligned AB (AtB or tAB, blue). From comparing the stacking distributions of samples with $0 < \theta_{13} \ll \theta_{23}$ (three leftmost panels in Fig. 3A, illustrated in Fig. 3D) and $0 < \theta_{23} \ll \theta_{13}$ (rightmost panel in Fig. 3A, illustrated in Fig. 3F), we find that the two regimes display distinct reconstruction patterns. When $\theta_{13} \ll \theta_{23}$, the observed atomic reconstruction is driven by a slight preference for AtA type stacking (white) over AtB (blue) and soliton-type (grey) regions. This result is somewhat unexpected as the energetic difference between rigid AtA and AtB domains (driven only by inter-layer coupling between the top and bottom layers) has been previously presumed to play a minor role in reconstruction \cite{zhu2020twisted}. Moreover, previous STM studies \cite{turkel2022orderly} concluded that trilayers with $0 < \theta_{13} \ll \theta_{23}$ relaxed to effectively eliminate AtB domains.

Fig. 3A also shows (rightmost panel) that the atomic reconstruction pattern for $\theta_{23} \ll \theta_{13}$ is instead driven by a preference to minimize the high energy tAA (white) domains, within which every possible stacking configuration must place two carbon atoms from neighboring layers directly atop each other — an arrangement that is sterically unfavorable \cite{yoo2019atomic,kazmierczak2021strain}. The extent of reconstruction in these $\theta_{23} \ll \theta_{13}$ samples is therefore much larger, since the energy difference between rigid tAA vs tAB domains ($\approx 18.2$ meV/unit when considering only adjacent interfaces) is much larger than that between rigid AtA vs AtB domains ($\approx 0$ meV/unit when considering only adjacent interfaces) \cite{zhu2020twisted, kim2022evidence}. This is reflected in the difference between the structures shown in the second and fourth columns of Fig. 3, in which both structures have comparable twist angles, but the structure in the fourth panel is observably more reconstructed, with the tAA domains appearing as a highly contracted spot. This spot appears orange rather than white due to beam-width and data processing effects. Although the weaker higher frequency texture observed within the white and blue domains in Fig. 3A might arise from the smaller scale moiré pattern imparting a modulation in these stacking distributions, this pattern likely predominantly results from a small bleed-in of the (1,2,3) interference pattern, which is hard to completely exclude with virtual apertures while retaining sufficient signal-to-noise ratios.

After extracting the local AtA/tAA and AtB/tAB domains as shown in Fig. 3A, we now examine the (1,2,3) overlap region, which is associated with all three graphene layers (Figs. 3B), to understand how the smaller scale moiré pattern modulates local stacking order within these larger domains (representative regions of these maps are magnified in Fig. 3E). We note that, as noted in previous work \cite{turkel2022orderly}, we see only two clear periodicities in our data despite the presence of three moire wavelengths from each twisted bilayer interface. However, this does not necessarily imply that only two moire wavelengths are present; inspection of the atomic stacking maps expected from even rigid structures (Fig. 3C) reveals that the smaller and larger periodicities observed reflect only the largest and smallest twist angles, respectively.

Taken together, the data in Fig. 3 allow a quantification of the area fractions in TTLG samples and the development of a qualitative model for reconstruction in the limit of slight misalignment (Fig. 4). Fig. 4A shows that for the larger moiré pattern, the proportion of AtA/tAA and AtB/tAB stacking domains inverts across the regimes illustrated in Figs. 3D,F. Associated area fractions from our measurements and those from continuum relaxation simulations as a function of $\theta_{13}-\theta_{23}$ are shown in Fig. 4A. Experiment and simulation show good agreement in the overall trends, though the measurements show a steeper drop-off in area fraction of AtA/tAA (and corresponding rapid rise in that of AtB/tAB/SP) than the simulation with increasing $\theta_{13}-\theta_{23}$. This slight discrepancy may arise because the simulations consider only interactions between adjacent interfaces, and stronger reconstruction may occur in reality owing to couplings between layers 1 and 3. 

\begin{figure*}[tbhp]
    \centerline{\includegraphics[width=120mm]{Main_Figures/M4.png}}
    \caption{ \textbf{Reconstruction patterns and trends in TTLG.} (\textbf{A}) Area fraction of atomic stacking domains from the larger moiré pattern only as a function of $\theta_{13}-\theta_{23}$. Experimental (\emph{exp}) data (corresponding to the maps shown in Fig. 3) are compared with relaxation simulations (\emph{sim}). (\textbf{B}) Local twist angle associated with the smaller moiré within AtA and AtB domains. All points correspond to the regime where $\theta_{13} \ll \theta_{23}$. (\textbf{C}) Qualitative schematic illustrating the atomic reconstruction patterns (large moiré) observed for $\theta_{13} \ll \theta_{23}$ and $\theta_{23} \ll \theta_{13}$. }
\end{figure*}

For the smaller scale moiré superlattice, we find that this pattern appears relatively invariant within the AtA, AtSP, AtB, and tAB domains. Indeed, the measured proportion of stacking orders within the AtA regions of Fig. 3  is very similar to the pure AtA sample seen in Fig. 2C, suggesting that the larger scale moiré plays a relatively minor role in the reconstruction of the smaller scale moiré. We do however observe some differences between the smaller-scale moiré within different domains. As shown in Fig. 4B, measurements of local $\theta_{12}$ values within the AtA and AtB domains of $\theta_{13} \ll \theta_{23}$ samples display a slightly smaller $\theta_{12}$ angle within AtA regions as compared to the values in adjacent AtB domains. This tightening of the smaller-scale moiré within AtB regions might help facilitate the overall minimization of these AtB regions. 

\begin{figure*}[tbhp]
    \centerline{\includegraphics[width=\textwidth]{Main_Figures/M5.png}}
    \caption{\textbf{Heterostrain Effects} (\textbf{A}) Maps of the modulation in local stacking order between layers 1 and 3 only for samples with an increasingly large percent of extrinsic heterostrain. (\textbf{B}) Corresponding maps of the local stacking order modulation obtained when considering all three graphene layers. Twist angles and percent heterostrains values and bounds were determined from fitting the size and asymmetry of the moire triangles. All scale bars are 25 nm. }
\end{figure*}

Lastly, we investigate the maps of local atomic stacking order in regions with an increasingly large extent of extrinsic heterostrain, $\epsilon$. From the maps shown in Fig. 5, we find that extrinsic heterostrain acts predominantly on the larger scale moire pattern and has a diminishing effect on the smaller scale moiré superlattice, consistent with previous work on bilayer moiré systems \cite{kazmierczak2021strain,van2022quantitative}.

Notably, in the most heterostrained sample of Fig. 5, despite similar $\theta_{13}$, the islands of AtA are deformed into stripes. These features have also been previously visualized in STM studies and attributed to heterostrain between the top and bottom layers \cite{kim2022evidence}. Heterostrain is therefore a powerful tuning knob for manipulating the contiguity of AtA domains (from islands to stripes) at the expense of AtB regions, potentially modulating the emergence of correlated phases that rely on the $C_{2z}T$-symmetric AtA domains. 


In conclusion, the nature of atomic reconstruction unveiled here for twisted trilayers is markedly different than that proposed in previous work, wherein it was suggested that slightly misaligned MATTLG samples relax to almost exclusively AtA regions, with the AtB and SP regions stretched into thin domain boundaries and/or topological defects that contribute insignificantly to the STM measurements \cite{turkel2022orderly}. While Fig. 2 shows that at length scales where only one moiré wavelength is apparent (when $\theta_{13} \approx 0\degree$), trilayers do favor the formation of large AtA domains, and in that case the local structure of trilayers is driven primarily by consideration of the smaller moiré, we see a clear presence of considerable AtB type stacking down to $\theta_{13} = 0.20\degree$ (Fig. 3). This observation contrasts previous claims of trilayer samples containing contributions from only AtA regions at a $\theta_{13}$ of $\approx 0.25\degree$. 

Further, we see domains of either AtA and AtB (when $\theta_{13} \ll \theta_{23}$) or tAB and tAA (when $\theta_{23} \ll \theta_{13}$) containing distinct distributions of twist angles associated with the small-moiré pattern. These domains appear strikingly similar to the clusters of stacking features, termed `plaquettes' (P) and `twistons' (T), presented previously as consisting of AAA domains \cite{turkel2022orderly}. Our measurements imply that the plaquettes and twistons might better be thought of as domains of local AtB (P) and AtA (T) or tAB (P) and tAA (T) symmetry, and that the bright spots in STM measurements (large local density of states, DOS) do not all correspond to AAA-type stacking, instead reflecting a richer distribution of stacking orientations. It is possible that the sample presented in \cite{turkel2022orderly} is structurally similar to the samples we present in Fig. 3 panels 2 and 3. In this case, the DOS maxima measured by STM would reflect both AAA and AAB symmetry (in the AtA and AtB regions, respectively) as flatbands are expected and observed in AAB regions as well \cite{tong2022spectroscopic}. In AtB regions, a lower local DOS of adjacent ABB regions would be measured from an STM measurement since the STM tip would be probing the Bernal bilayer side, as reported previously \cite{tong2022spectroscopic}. This effect may then result in the visualization, by STM, of a triangular lattice of the most prominent AAB symmetry. Taken together, our measurements highlight the particular utility of interferometric 4D-STEM imaging alongside other scanning probe techniques like STM for characterizing complex multi-layered moirés, as the ability to apply a direct structural probe selectively to separate interfaces can uncover the complex picture of atomic reconstruction. 

The extent of AtB stacking observed could have major implications for understanding superconductivity in misaligned MATTLG \cite{hao2021electric,kim2022evidence} and recently discovered moiré quasicrystal systems \cite{uri2023quasicrystal}. For instance, if $\theta_{13} \ll \theta_{23}$ configurations such as MATTLG favored entirely AtA stacking as previously proposed, their correlated behaviors could be predominantly understood by consideration of the $C_{2z}T$ inversion symmetric AAA stacking regions much like TBLG. However the presence of significant AtB domains may instead suggest that the ABC, AAB, and ABB regions, which have been shown to host correlated electronic phases \cite{zhou2021superconductivity,chen2020tunable,xu2021tunable,li2022imaging} despite a lack of inversion symmetry, may play an important role in understanding correlated electron physics in some twisted trilayers. 

\section*{Acknowledgements}
We thank P. Jarillo-Herrero, P. Kim, J. Ciston, K. Bustillo, and C. Ophus for vis-a-vis and epistolary discussions. This material is based upon work supported by the US National Science Foundation Early Career Development Program (CAREER), under award no. 2238196. I.M.C. acknowledges the NDSEG program for a pre-doctoral fellowship under contract FA9550-21-F-0003. C.G. was supported by a grant from the W.M. Keck Foundation (Award no. 993922). Experimental work at the Molecular Foundry, LBNL was supported by the Office of Science, Office of Basic Energy Sciences, the U.S. Department of Energy under Contract no. DE-AC02-05CH11231. Confocal Raman spectroscopy was supported by a Defense University Research Instrumentation Program grant through the Office of Naval Research under award no. N00014-20-1-2599 (D.K.B.). Other instrumentation used in this work was supported by grants from the Canadian Institute for Advanced Research (CIFAR–Azrieli Global Scholar, Award no. GS21-011), the Gordon and Betty Moore Foundation EPiQS Initiative (Award no. 10637), and the 3M Foundation through the 3M Non-Tenured Faculty Award (no. 67507585). K.W. and T.T. acknowledge support from JSPS KAKENHI (Grant Numbers 19H05790, 20H00354 and 21H05233).

\section*{Author Contributions}
I.M.C., M.V., C.G., and D.K.B. conceived the study. M.V., C.G., K.Z., and N.D. fabricated the samples. M.V., C.G., and I.M.C. performed the experiments.
I.M.C. and Z.Z. performed the continuum simulations using code developed by Z.Z. I.M.C. developed and implemented the interferometry code (with assistance from C.G.) and analyzed the data. T.T. and K.W. provided the hBN crystals. D.K.B. and S.M.G. supervised the work. I.M.C. and D.K.B. wrote the manuscript with input from all co-authors.

\section*{Competing Interests}
The authors declare no competing interests.

\section*{Data Availability}
The data supporting the findings of this study are available within the Article and its Supplementary Information files.


\section*{Additional Information}

Correspondence and requests for materials should be emailed to DKB (email: bediako@berkeley.edu).

\printbibliography
\end{document}
