% ****** Start of file apssamp.tex ******
%
%   This file is part of the APS files in the REVTeX 4.2 distribution.
%   Version 4.2a of REVTeX, December 2014
%
%   Copyright (c) 2014 The American Physical Society.
%
%   See the REVTeX 4 README file for restrictions and more information.
%
% TeX'ing this file requires that you have AMS-LaTeX 2.0 installed
% as well as the rest of the prerequisites for REVTeX 4.2
%
% See the REVTeX 4 README file
% It also requires running BibTeX. The commands are as follows:
%
%  1)  latex apssamp.tex
%  2)  bibtex apssamp
%  3)  latex apssamp.tex
%  4)  latex apssamp.tex
%
\documentclass[%
 reprint,
%superscriptaddress,
%groupedaddress,
%unsortedaddress,
%runinaddress,
%frontmatterverbose, 
%preprint,
%preprintnumbers,
%nofootinbib,
%nobibnotes,
%bibnotes,
 amsmath,amssymb,
 aps,
prl,
%prb,
%rmp,
%prstab,
%prstper,
%floatfix,
]{revtex4-2}

\usepackage{graphicx}% Include figure files
\usepackage{dcolumn}% Align table columns on decimal point
\usepackage{braket}
\usepackage{bm}% bold math
%\usepackage{hyperref}% add hypertext capabilities
%\usepackage[mathlines]{lineno}% Enable numbering of text and display math
%\linenumbers\relax % Commence numbering lines

%\usepackage[showframe,%Uncomment any one of the following lines to test 
%%scale=0.7, marginratio={1:1, 2:3}, ignoreall,% default settings
%%text={7in,10in},centering,
%%margin=1.5in,
%%total={6.5in,8.75in}, top=1.2in, left=0.9in, includefoot,
%%height=10in,a5paper,hmargin={3cm,0.8in},
%]{geometry}
\usepackage{amsthm}
\newtheorem*{definition}{Definition}
\newcommand{\bs}[1]{\boldsymbol{\mathbf{#1}}}
\begin{document}

%\preprint{APS/123-QED}

\title{%No eigenstate of a noninteracting local spin-chain Hamiltonian satisfies the area law \\
%if the ground state has half-integer central charge
No eigenstate of the critical transverse-field Ising chain satisfies the area law
}
\author{Saverio Bocini}
\affiliation{%
 Universit\'e Paris-Saclay, CNRS, LPTMS, 91405, Orsay, France
}%
\author{Maurizio Fagotti}
 \email{maurizio.fagotti@universite-paris-saclay.fr}
\affiliation{%
 Universit\'e Paris-Saclay, CNRS, LPTMS, 91405, Orsay, France
}%

\begin{abstract}
We argue that no eigenstate of a noninteracting local spin-$\frac{1}{2}$ chain Hamiltonian satisfies the area law if the ground state has half-integer central charge.
Conversely, if its ground state satisfies the area law, no eigenstate of a noninteracting local spin-chain Hamiltonian %with sub-extensive entropy 
breaks the area law as the ground state of a conformal system with half-integer central charge.

\noindent We consider only one-site shift invariant spin systems that are mapped to free fermions by a Jordan-Wigner transformation and restrict ourselves to the standard basis of Slater determinants.
\end{abstract}

%\keywords{Suggested keywords}%Use showkeys class option if keyword
                              %display desired
\maketitle

%\tableofcontents

The ground state of spin-chain systems with gapped local Hamiltonians are low entangled: the entropy of a block of spins has only a subleading dependence on the block's length. This is known as ``area law'' and applies to systems in higher dimensions as well. In 1D the area law generally breaks down at quantum phase transitions, where the entropy of a spin block can develop a logarithmic dependence on its length. In contrast, in generic systems a volume law is expected in every excited state: the entropy of a block of spins is proportional to the block's length. Exceptions to this rule define the so called quantum scars, which are excited states with anomalously low bipartite entanglement. Integrable systems are quite exceptional in this respect, as they exhibit infinitely many excited states with sub-extensive entropies and energy extensively higher than the ground state one. This is a direct consequence of the existence of infinitely many local conservation laws, indeed the ground state of any conserved charge
with fast enough decaying interactions is an eigenstate of the Hamiltonian with sub-extensive bipartite entanglement. How many locally different excited states satisfy however the area law? This is a tricky, somewhat ill-defined question. On one hand, the answer could depend on the basis chosen to diagonalise the Hamiltonian and degeneracies could also be sensitive to the system's size; on the other hand, in integrable systems there are privileged bases that allow for a systematic description, more or less independent of  chain's length and accidental degeneracies. 
In noninteracting spin chains that are mapped to free fermions by a Jordan-Wigner transformation, the natural basis consists of Slater determinants for the Jordan-Wigner fermions. 
%Ref.~\cite{Jafarizadeh2019Bipartite} has recently suggested that, in free fermionic systems, one should generally expect infinitely many locally different excited states satisfying the area law, as well as infinitely many states breaking the area law as ground states of conformal systems with half-integer and/or integer central charges. While the physical arguments provided sound convincing, we show here that the picture could be quite different. 
In any such excited state, the bipartite entanglement entropies can be easily computed numerically, and there are also mathematical theorems and conjectures on the asymptotic behaviour of Toepltz and block-Toeplitz matrices that allow one to predict their asymptotic behaviour analytically, at least in the thermodynamic limit. This opportunity is definitely rare and has been already exploited to quantify the picture summarised above. 
In particular, it was shown that there are infinitely many excited states that can be described by zero-temperature conformal field theories with half-integer or integer central charges. More vague are the statements about excited states satisfying the area law. For example, a recent investigation~\cite{Jafarizadeh2019Bipartite} into the bipartite entropies of excited states in noninteracting spin chains gives the impression that one should expect infinitely many states of that kind. We show here that the picture could be quite different. Specifically, only a finite number of locally different excited states seem to satisfy the area law, and there are notable cases, such as the critical Ising model, where there are none. 
%Before expanding on that, we need an effective definition of area law for excited states. Though generally convenient, taking the thermodynamic 
%limit in this case could be unfeasible: while the definition of ground state is independent of the chain's length $L$,  the number of excited states increases exponentially with $L$, hence the sequence of excited states (labelled by $L$) that one would like to associate with a given macrostate in the thermodynamic limit is ambiguous. In these notes we use the following definitions (for spin chains):
%\begin{definition}
%There is at least one excited state satisfying the area law if and only if there are sequences of excited states  (enumerated by $L$) in which the entropy of any spin block has an upper bound independent of both the block's length and $L$.  
%\end{definition}
%\begin{definition}
%Two sequences of excited states satisfying the area law are locally different if they can be distinguished by the expectation value of a local observable for arbitrarily large $L$. 
%\end{definition}
In particular, we are going to show that 
\begin{quote}
\emph{there are no sequences of excited states  (enumerated by $L$) in which the entropy of any spin block has an upper bound independent of both the block's length and $L$.  }
\end{quote}
If we focus on one-site shift invariant models with periodic boundary conditions, the excited states split in two sectors, usually called Ramond and Neveu-Schwarz, differing only in the quantization conditions satisfied by the momenta. Each excited state of the noninteracting model is fully characterised by the fermionic two-point correlations, which are customarily organised in a matrix $\Gamma$, known as ``correlation matrix''. 
The block-Fourier transform (aka symbol) $\Gamma(k)$ of the correlation matrix of an excited state can be written in terms of the block-Fourier transform $\mathcal H(k)$ of the matrix $\mathcal H$ describing the couplings between Majorana fermions $\bs a_\ell$ in the Hamiltonian $\bs H=\frac{1}{4}\sum_{\ell,n}\bs a_\ell\mathcal H_{\ell n}\bs a_n$. Specifically, the symbol $\Gamma(k)$ in the ground state (or the Slater determinant with the lowest energy in one of the sectors with boundary conditions  $e^{iL k}=\pm 1$, where $L$ is the chain's length) takes the form 
\begin{equation}
\Gamma_{GS}(k)=-\mathcal U(k)\mathrm{sgn}(\mathcal E(k))\mathcal U^\dag (k)\, ,
\end{equation}
where $\mathcal E(k)$ is the 2-by-2 diagonal matrix with non-increasing diagonal elements such that
\begin{equation}
\mathcal H(k)=\mathcal U(k)\mathcal E(k)\mathcal U^\dag (k)\, .
\end{equation}
We point out that the size ($2$) of the blocks is related to considering only one-site shift invariant Hamiltonians. 
An excitation with momentum $k$ has a double effect:
\begin{equation}
\begin{aligned}
\mathcal E(k)\rightarrow& -\sigma^z \mathcal E(k)\\
\mathcal E(-k)\rightarrow& \sigma^z \mathcal E(-k)
\end{aligned}
\end{equation}
In the asymptotic limit of large $L$, only a number of excitations proportional to $L$ can have effects on the entanglement entropies of a spin block with a given length (independent of $L$). 
This is because the expectation values of the observables in the subsystem are completely characterised by the correlation matrix in which the indices have been restricted to the values associated with the subsystem; each excitation has then an $O(L^{-1})$ effect, approaching zero in the thermodynamic limit. 
\paragraph{R\'enyi entropies.} 
The R\'enyi entropies of a spin block can be expressed as follows 
\begin{equation}\label{eq:Salpharep}
S_\alpha(A)=\sum_{j=0}^{\alpha-1}\frac{\log\det\Bigl|\sin(\frac{\pi(j+\frac{1}{2})}{\alpha})\mathrm I-i \cos(\frac{\pi(j+\frac{1}{2})}{\alpha})\Gamma_A^{(L)}\Bigr|}{2(1-\alpha)}
\end{equation}
hence the entanglement of spin blocks is completely characterised by the family of symbols
\begin{equation}
S(k;\phi)=\sin(\phi)\mathrm I-i \cos(\phi)\Gamma(k)\qquad
\end{equation}
satisfying in general the boundary conditions
\begin{equation}
e^{2i\alpha\phi}=-1\qquad e^{i L k}=\pm 1\, .
\end{equation}

\paragraph{Critical Ising model.}
The critical Ising model is special as $\mathcal H$ is not only block-(anti-)circulant, but it is a purely imaginary antisymmetric (anti-)circulant matrix. In particular, the ground-state correlation matrix has the following symbol
\begin{equation}
\Gamma(k)=-\mathrm{sgn}(\sin(k))\, .
\end{equation}
As a(n anti-)circulant matrix, it commutes with any other (anti-)circulant matrix, and a complete basis of excited states have purely imaginary antisymmetric (anti-)circulant correlation matrices with $|\Gamma(k)|=1$. 

\paragraph{Excited states with a thermodynamic limit.}
In the thermodynamic limit, since $\Gamma^{(\infty)}(k)$ is real, we can apply the result proved by Szeg\"o in Ref.~\cite{Szego1920Beitrage}, and obtain
\begin{multline}
\lim_{|A|\rightarrow\infty}\frac{S_\alpha(A)}{|A|}=\\
\sum_{\genfrac{}{}{0pt}{2}{\phi}{ e^{2i\alpha\phi}=-1}}\int_{-\pi}^\pi\frac{\mathrm d k}{2\pi}\frac{\log\Bigl|\sin^2(\phi)+\cos^2(\phi)[\Gamma^{(\infty)}(k)]^2\Bigr|}{4(1-\alpha)}
\end{multline}
The excited states with minimal entropy are sub-extensive, therefore, in order for $\Gamma^{(\infty)}(k)$ to describe the limit of a sequence of excited states satisfying the area law,  it should also satisfy
\begin{equation}
\Gamma^{(\infty)}(k)=\pm 1\, .
\end{equation}
Note that, while this is a trivial identity for every excited state with a finite value of $L$, it becomes a nontrivial requirement in the thermodynamic limit, in which $\Gamma^{(\infty)}(k)$ is supposed to be a function of the continuous variable $k$. All sequences of excited states satisfying the area law and with a definite thermodynamic limit $\Gamma^{(\infty)}(k)$ are therefore characterised by a piece-wise constant symbol with a given number of discontinuities. These are simple examples of Fisher-Hartwig singularities that give logarithmic contributions. The area law can be satisfied only if there are no discontinuities at all. This is however impossible since the symbol is an odd piece-wise constant function of $k$ equal to $\pm 1$, hence, the smallest number of discontinuities is $2$ and corresponds to the ground state or to the state with maximal energy. 

From~\eqref{eq:Salpharep} it readily follows
\begin{equation}
S_\alpha=\sum_{\phi|e^{2i\alpha\phi}=-1}\frac{\log\det\Bigl|\sin^2(\phi)\mathrm I+ \cos^2(\phi)\Gamma^2\Bigr|}{4(1-\alpha)}
\end{equation}
\begin{equation}
e^{-i\pi\epsilon}\frac{1+\epsilon}{-2\epsilon \pi}\int\limits_{C(-\frac{\pi}{2},\frac{\pi}{2}]}\mathrm d k\frac{\log\det\Bigl|\cos(k)\mathrm I-i \sin(k)\Gamma\Bigr|}{-e^{2i p}e^{i \epsilon 2p}-1}
\end{equation}

\paragraph{Critical Ising model.}
The critical Ising model is special as $\mathcal H$ is not only block-(anti-)circulant, but it is a purely imaginary antisymmetric (anti-)circulant matrix. In particular, the ground-state correlation matrix has the following symbol
\begin{equation}
\Gamma(k)=-\mathrm{sgn}(\sin(k))\, .
\end{equation}
As a(n anti-)circulant matrix, it commutes with any other (anti-)circulant matrix, and a complete basis of excited states have purely imaginary antisymmetric (anti-)circulant correlation matrices. 
Such an antisymmetry is technically what prevents the entropy of any excited state to satisfy the area law. We can easily understand it analytically for the sequences $\{\ket{\{k\}_L}\}$ of excited states that, in the thermodynamic limit $L\rightarrow\infty$, are described by symbols with Fisher-Hartwig singularities. If their symbol is smooth in an open interval around $k=0,\pi$, it should vanish at $k=0,\pi$ and be smaller than $1$ in modulus in a finite momentum interval. The latter condition is however sufficient to make the determinant of $\frac{\mathrm I_A+i\Gamma_A}{\sqrt{2}}$ decay to zero exponentially in $L$. 

An excitation cannot compromise the anti-symmetry of the symbol, and indeed it changes the sign of $\Gamma(k)$ at a given momentum $k$ and its opposite $-k$. 





\begin{equation}
\min_{\{k\}}\left\{-\mathrm{tr}\left[\frac{\mathrm I_A+\Gamma_A(\{k\})}{2}\log \frac{\mathrm I_A+\Gamma_A(\{k\})}{2}\right]\right\}
\end{equation}
Let us then fix the size of the subsystem to $\ell$. The entropy is completely determined by the first $2\ell+1$ Fourier coefficients of the symbol (taken symmetrically with respect to $0$)

Imagine then to add $\eta L$ excitations with momenta in some small shell $\Delta_\eta K\in (-\pi,\pi)$. Let us then construct an effective correlation matrix obtained by averaging $\Gamma(k)$ in $\Delta_\eta K$ The expectation value of local observables won't be affected if we replace $\Gamma(k)$ by an effective  correlation matrix by a 
\begin{equation}
\Gamma_{\ell n}=\frac{1}{L}\sum_{k} e^{i(\ell-n) k}\Gamma(k)=
\end{equation}
Generally, the resulting symbol looses the property of being an involution. When this happens, the entropy becomes proportional to the subsystem's length. Since we're looking for excited states satisfying the area law, we can focus on the exceptions to that rule. They arise when, up to a set of zero measure, all the excitations with momentum in a multiply connected domain are present in the excited state. Let $\vartheta(k)=\vartheta_e(k)+\vartheta_o(k)$ be the characteristic function of that domain, where $\vartheta_e(k)$ is even and $\vartheta_o(k)=1-\vartheta_o(-k)$. 

\begin{equation}
\mathcal H(k)=h_0(k) \mathrm I+h_1(k)\sigma^x+h_2(k)\sigma^y+h_3(k)\sigma^z
\end{equation}
where $h_2(k)$ is even whereas the other functions are odd. Since the symbol commutes with the identity, we can always assume $h_0(k)=0$ (a nonzero term does not affect the excited states), hence we have
\begin{multline}
(1-2\vartheta(k))\frac{\mathrm I+\frac{h_1(k)\sigma^x+h_2(k)\sigma^y+h_3(k)\sigma^z}{\sqrt{h_1^2(k)+h_2^2(k)+h_3^2(k)}}}{2}
+\\-
(1-2\vartheta(-k))\frac{\mathrm I-\frac{h_1(k)\sigma^x+h_2(k)\sigma^y+h_3(k)\sigma^z}{\sqrt{h_1^2(k)+h_2^2(k)+h_3^2(k)}}}{2}
\end{multline}
\begin{equation}
-2\vartheta_o(k)\mathrm I+(1-2\vartheta_e(k))\frac{h_1(k)\sigma^x+h_2(k)\sigma^y+h_3(k)\sigma^z}{\sqrt{h_1^2(k)+h_2^2(k)+h_3^2(k)}}
\end{equation}

Just as the expectation values of observables, so the entanglement entropies of a block of spins can be expressed in terms of the correlation matrix. The latter turns out to be a block-Toeplitz matrix with $2$-by-$2$ blocks. 




\paragraph{Numerical analysis}
By working directly in the thermodynamic limit, we are leaving out from our discussion those eigenstates that are not well-defined in such a limit.
In principle, one of those states may have an entropy that does not diverge with system's size even in the critical Ising chain.
Our claim is that, even if we consider those states, the eigenstate showing the smallest entropy is the ground state. This implies that the entropy of any eigenstate in the critical Ising grows at least logarithmically. To verify our claim in finite size, we performed a numerical analysis using an algorithm known as simulated annealing.

Before performing our numerical analysis, we start by recasting the spin Hamiltonian of a given length $L$ into a fermionic framework. This maps our original Hamiltonian into two sectors, each of which characterised by a quadratic fermionic Hamiltonian.
Considering the fermionic Hamiltonian linked to a given sector, we can specify a complete set of eigenstates in the momentum basis. For each fermionic Hamiltonian we have $L$ momenta: $\{\frac{2\pi k}{L}\}_{k\in\{-L/2+1,..., L/2\}}$ for the Ramond sector (with periodic boundary conditions) and $\{\frac{\pi (2k-1)}{L}\}_{k\in\{-L/2+1,..., L/2\}}$ for the Neveu-Schwartz sector (with anti-periodic boundary conditions). An eigenstate is identified by which momenta are occupied: this gives the $2^L$ eigenstates of each fermionic Hamiltonian.
Those eigenstates are shared by the original spin Hamiltonian if they have the right number of excitation. This can be checked by evaluating $\braket{\prod_{j=1}^L\sigma^z_j}=\operatorname{Pf}(\Gamma)$ over each eigenstate: the eigenstates of the Ramond sector have to give $+1$ to be also eigenstates of the spin Hamiltonian and those of the Neveu-Schwartz sector have to give $-1$.

Thanks to the fermionic picture, given $L$, we can efficiently compute the energy and entropy of each eigenstate, as reported in Fig.~\ref{fig:scar_plot}. Our goal is to find the minimum in such an entropy landscape. As long as $L$ is relatively small, we can do that by the means of a brute-force search, i.e. computing the entropy of each eigenstate. However, as the number of eigenstates diverges exponentially with $L$, making the computation off the entropy of all of them unfeasible for large $L$, so we have to resort to some approximating algorithm, such as simulated annealing.

\begin{figure}
    \centering
    \includegraphics[width=\linewidth]{scar_plot.png}
    \caption{Entropy of all the eigenstates of the critical Ising spin chain in the momentum basis in function of their energy for $L=16$. Energies and entropies were computed using the underlying fermionic theory and properly accounting for the distinction in sectors.}
    \label{fig:scar_plot}
\end{figure}

Simulated annealing is an algorithm to approximate the global minimum of a function defined in a large search space. Since it does not rely on evaluating any gradient, it is especially useful to us because our search space, made by all the eigenstates, is discrete.
To start off the algorithm we restrict to one fermionic sector and we initialise the system in a random eigenstate, obtained by deciding if its momenta are occupied or not by sampling from a uniform distribution. Then, at each step, we change the occupation-state of a random momentum (from occupied to empty if the momentum was originally occupied and vice versa) and we accept the new configuration with probability $\min\{\exp((S_{old}-S_{new})/T_{eff}), 1\}$ in analogy with the Metropolis–Hastings algorithm.
Here $T_{eff}$ is a parameter that is \emph{not} constant during the algorithm, but depends on the number of iteration: it starts very large and it goes to $0$ towards the maximum number of iterations. 
In this way, the algorithm initially explores the full landscape, ignoring small features of the entropy, then it drifts towards low-entropy regions and, once one of those regions has been chosen, it finally moves to its local minimum in a (quasi)deterministic way; the point reached in this way is candidate to be the global minimum. The algorithm is run several times, varying the how the temperature goes to zero among power laws of the kind
\begin{equation}
    T_{eff}= T_0\left(\frac{iteration_{tot} - iteration_{current}}{iteration_{tot}}\right)^\alpha.
\end{equation}
In any case, the minimum value of the entropy that is ever reached is the one of the ground state (which equals the entropy of the maximally-excited state), as shown in Fig.~\ref{fig:simulated_annealing}.
For a given run, what happens to the occupation numbers is that they stay completely disordered at first and then they start to clusterise, until every domain disappears, as shown in Fig.~\ref{fig:percolation_plot}.
The minimum is searched for in both the fermionic sectors: if we combine their eigenstates, we are minimising on a larger landscape than the eigenstates of the spin chain. This is enough for us since it lower bounds the minimum of the entropy of the spin chain.

\begin{figure}
    \centering
    \includegraphics[width=\linewidth]{simulated_annealing.png}
    \caption{Entropy history of several runs of the simulated annealing algorithm for the critical Ising fermionic chain with apbc and $L=100$. Here we also vary the exponent $\alpha$ that describes how $T_{eff}$ goes to zero in the interval $[1,4]$. The algorithm was run also for other choices of parameters that are not shown here.}
    \label{fig:simulated_annealing}
\end{figure}

\begin{figure}
    \centering
    \includegraphics[width=\linewidth]{percolation_plot.png}
    \caption{Focus on a single run of simulated annealing that converges to the ground state in the critical Ising fermionic chain with apbc and $L=100$. Each horizontal line is composed by $L$ points, corresponding to the momenta $\{-L/2+1,...,L/2\}$. Each point is coloured yellow if the corresponding momentum is occupied and black if is empty. We see that, as $T_{eff}$ decreases, the occupied momenta undergo a sort of clustering phenomenon to minimise the entropy. Eventually, the minimum of entropy is attained when no domain walls of momenta are present.
    }
    \label{fig:percolation_plot}
\end{figure}

 

\bibliography{apssamp}% Produces the bibliography via BibTeX.


\end{document}
%
% ****** End of file apssamp.tex ******
