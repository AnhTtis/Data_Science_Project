\section{Conclusion}
\label{sec:conclusion}
There are myriad applications for detailed, personalized, and animatable 3D human models, and they become increasingly practical as generation times, data acquisition, and hardware requirements decrease. 
One promising direction is to bootstrap the training process of human-specific NeRFs ~\cite{weng:cvpr2022:humannerf,chen:arxiv2021:animner} with the geometry and texture learnt from our model.
Since our model directly provides 3D coordinates and RGBs in the canonical space, volume rendering is not needed during this pretraining step.
% One promising direction is to use our system to generate synthetic data inexpensively at scale.
% Another promising direction is to use our system to generate synthetic data inexpensively at scale. 
% For example, in one month, our system is capable of generating approx.~1k unique avatars from videos on a single GPU.
% Once generated, these avatars can be reposed and rendered from different viewpoints %with different backgrounds 
% to efficiently scale up to many thousands of examples, which can be used to train, \eg, human pose detection, action recognition, and semantic segmentation models.
While NeRFs have demonstrated great versatility, \ie, they are scene-agnostic, %the same system can be used to reconstruct different types of scenes and objects), 
they are prohibitive for reconstructing well-defined objects with strong priors (\eg faces, human avatars) given their steep training requirements. 
For these applications, we argue that our mesh-based system is competitive and, in some cases, favorable. 
We have demonstrated that our approach is capable of generating results with 
very competitive performance 
% significantly better accuracy and detail compared to existing approaches, and 
% is comparable
% with comparable accuracy and detail 
to SOTA human-specific NeRFs \cite{chen:arxiv2021:animner,weng:cvpr2022:humannerf}, \brandonedit{but in a tiny} fraction of the training time and compute. 

\clearpage