\textbf{Dataset}. The MultiTHUMOS dataset \cite{yeung2018every} is a densely labeled extension of THUMOS14, consisting of 413 sports videos with 65 distinct action classes. The dataset presents a significant increase in the average number of distinctive action categories per video, compared to THUMOS14. As such, it poses a greater challenge for TAL than THUMOS. While MultiTHUMOS are being used in action detection benchmark \cite{dai2021pdan, dai2022ms}, a novel approach for action detection, PointTAD \cite{tan2022pointtad}, utilizes the TAL evaluation metric to assess the completeness of predicted action instances. Given that TAL and action detection share the same setting in terms of input features and annotations, we evaluate the performance of our model on MultiTHUMOS and compare it against the state-of-the-art action detection methods \cite{dai2021pdan, tirupattur2021modeling, dai2022ms, tan2022pointtad}, and strong baseline ActionFormer \cite{zhang2022actionformer}.

% \textbf{Dataset}. The MultiTHUMOS dataset \cite{yeung2018every} is a densely labeled extension of the THUMOS14 dataset, comprising a collection of 413 sports videos with 65 distinct classes. Notably, MultiTHUMOS has roughly 10.5 distinctive action categories per video, a significant increase compared to THUMOS14's average of 1.1 categories per video. Hence, the MultiTHUMOS dataset can be considered significantly more challenging than the THUMOS dataset. MultiTHUMOS is commonly used in action detection benchmarks, as demonstrated in recent works \cite{ dai2021pdan, dai2022ms}. However, a novel approach for action detection, PointTAD \cite{tan2022pointtad}, uses mAP at different tIoU thresholds to assess the completeness of predicted action instances, similar to the TAL evaluation metric. Moreover, TAL and action detection task share the same setting such as input features and annotations. Consequently, we validate our model on MultiTHUMOS and compare its performance to the state-of-the-art action detection methods \cite{dai2021pdan, tirupattur2021modeling, dai2022ms, tan2022pointtad}, and strong baseline ActionFormer \cite{zhang2022actionformer}.

\textbf{Feature Extraction}. We only utilize RGB stream as input for I3D network \cite{carreira2017quo} pre-trained on Kinetics \cite{kay2017kinetics}, following \cite{tan2022pointtad}, to extract features for MultiTHUMOS. The I3D pre-trained network is fed with 16 sequential frames through a sliding window with a stride of 4. The feature is extracted from the final fully connected layer, resulting in a 1024-D feature space that serves as input for the model.

\textbf{Result}. Tab. \ref{table:sota_multihumos} provides a comparison of our performance on the MultiTHUMOS dataset \cite{yeung2018every} with recent state-of-the-art methods. Specifically, our method surpasses the prior work \cite{tan2022pointtad} that utilizes TransFormer as feature encoding by a large margin, 6.4\% mAP on average. Moreover, TemporalMaxer improves the robust baseline, ActionFormer \cite{zhang2022actionformer}, by 2.4\% mAP at tIoU=0.7, and 1.3\% mAP on average. It should be noted that we utilized the code provided by the authors to implement ActionFormer \cite{zhang2022actionformer} on the MultiTHUMOS dataset, and the results \cite{dai2021pdan, tirupattur2021modeling, dai2022ms} are taken from \cite{tan2022pointtad}.

\begin{table}[]
\centering
\resizebox{.43\textwidth}{!}{%
\begin{tabular}{ccccc}
\hline
\multirow{2}{*}{Method}                                               & \multicolumn{4}{c}{tIoU}                                      \\ \cline{2-5}
                                                                      & 0.2           & 0.5           & 0.7           & Avg           \\ \hline
PDAN \cite{dai2021pdan}                              & ---           & ---           & ---           & 17.3          \\
MLAD \cite{tirupattur2021modeling}                   & ---           & ---           & ---           & 14.2          \\
MS-TCT \cite{dai2022ms}                              & ---           & ---           & ---           & 16.2          \\
PointTAD \cite{tan2022pointtad}                      & 39.7          & 24.9          & 12.0          & 23.5          \\
ActionFormer \cite{zhang2022actionformer} & 46.4          & 32.4          & 15.0          & 28.6          \\ \hline
Our (TemporalMaxer)                                                   & \textbf{47.5} & \textbf{33.4} & \textbf{17.4} & \textbf{29.9} \\ \hline
\end{tabular}}%
\caption{Comparison with the state-of-the-art methods on the MultiTHUMOS dataset. We report the results at different tIoU thresholds [0.2, 0.5, 0.7] and average mAP in [0.1:0.9:0.1].}
\label{table:sota_multihumos}
\end{table}
