%%%%%%%%%%%%%%%%%%%%%%%%%%%%%%%%%%%%%%%%%%%%%%%%%%%%%%%%%%%%%%%%%%%%%%%%%%%%%%%%
%2345678901234567890123456789012345678901234567890123456789012345678901234567890
%        1         2         3         4         5         6         7         8

%\documentclass[letterpaper, 10 pt, conference]{ieeeconf}  % Comment this line out if you need a4paper

 %\pagestyle{empty}
%\documentclass[a4paper, 10pt, conference]{ieeeconf}      % Use this line for a4 paper

%\IEEEoverridecommandlockouts                              % This command is only needed if 
%\documentclass[letterpaper, 10 pt, conference]{ieeeconf}  % Comment this line out if you need a4paper

%\documentclass[a4paper, 10pt, conference]{ieeeconf}      % Use this line for a4 paper

%\IEEEoverridecommandlockouts                              % This command is only needed if 
                                                          % you want to use the \thanks command

%\overrideIEEEmargins                            % you want to use the \thanks command

%\overrideIEEEmargins                                      % Needed to meet printer requirements.

%In case you encounter the following error:
%Error 1010 The PDF file may be corrupt (unable to open PDF file) OR
%Error 1000 An error occurred while parsing a contents stream. Unable to analyze the PDF file.
%This is a known problem with pdfLaTeX conversion filter. The file cannot be opened with acrobat reader
%Please use one of the alternatives below to circumvent this error by uncommenting one or the other
%\pdfobjcompresslevel=0
%\pdfminorversion=4

% See the \addtolength command later in the file to balance the column lengths
% on the last page of the document

% The following packages can be found on http:\\www.ctan.org
%\usepackage{graphics} % for pdf, bitmapped graphics files
%\usepackage{epsfig} % for postscript graphics files
%\usepackage{mathptmx} % assumes new font selection scheme installed
%\usepackage{times} % assumes new font selection scheme installed
%\usepackage{amsmath} % assumes amsmath package installed
%\usepackage{amssymb}  % assumes amsmath package installed
%\documentclass{article}
\documentclass[a4paper,11pt]{article}
\usepackage[utf8]{inputenc}
\usepackage[a4paper]{geometry}
\usepackage{authblk}
\title{\LARGE \bf
An invitation to quantum mean-field filtering and control}%\thanksref{footnoteinfo}}

%\thanks[footnoteinfo]{This work is supported by the Agence Nationale de la Recherche projects Q-COAST ANR- 19-CE48-0003 and IGNITION ANR-21-CE47- 0015.}

\date{\vspace{-5ex}}


\author[1]{Sofiane Chalal}

\author[1]{Nina H. Amini}
\author[2]{Gaoyue Guo}
\affil[1]{L2S, CNRS, CentraleSup\'elec, Universit\'e Paris-Saclay
        {\tt\small firstname.lastname@centralesupelec.fr}}
\affil[2]{MICS, CentraleSupélec, Université Paris-Saclay
        {\tt\small firstname.lastname@centralesupelec.fr}}

\iffalse
\author{Sofiane Chalal, 
\thanks{
{\small Laboratoire des Signaux et Syst\`{e}mes (L2S), CNRS-CentraleSup\'{e}lec-Universit\'{e} Paris-Saclay, 3, Rue Joliot Curie, 91190 Gif-sur-Yvette, France, (nina.amini@l2s.centralesupelec.fr).}}
\and 
Nina H. Amini
\thanks{
{\small Department of Information Engineering, University of Padua, 6B, Via Gradenigo, 35131 Padova, Italy, (weichao.liang@dei.unipd.it).}}%
\and 
Nina H. Amini,
\thanks{
{\small Laboratoire des Signaux et Syst\`{e}mes (L2S), CNRS-CentraleSup\'{e}lec-Universit\'{e} Paris-Sud, Universit\'{e} Paris-Saclay, 3, Rue Joliot Curie, 91190 Gif-sur-Yvette, France, (nina.amini@l2s.centralesupelec.fr).}}
}
\fi

\iffalse
\author{Sofiane Chalal$^{1}$,  Nina H. Amini$^{1}$, Gaoyue Guo $^{2}$ 
\thanks{This work is supported by the Agence Nationale de la Recherche projects Q-COAST ANR- 19-CE48-0003 and IGNITION ANR-21-CE47- 0015. }% <-this % stops a space
\thanks{$^{1}$ L2S,
CNRS, CentraleSup\'elec, Universit\'e Paris-Sud, Universit\'e Paris-Saclay
        {\tt\small firstname.lastname@centralesupelec.fr}}%
\thanks{$^{2}$ MICS, CentraleSupélec, Université Paris-Saclay
        {\tt\small firstname.lastname@centralesupelec.fr}}%
}
\fi
\usepackage{xcolor} % To use color :) 
%\usepackage{amssymb, amsfonts, amsmath, amsthm}
\usepackage{amssymb} %To write mathematics.
\usepackage{amsmath} % Write mathematics, include matrix environnement 
\usepackage{bigints} %To use big integral symbol :) 
\usepackage{braket} %Dirac Notations 
\usepackage{physics} %Physics notations
\usepackage{graphicx} % To add figures
\usepackage[nottoc]{tocbibind} % Bibliography
\usepackage{url}
%%%%%%%%%%%%%%%%%%%%%%
\newtheorem{theorem}{Theorem}
\newtheorem{proposition}{Proposition}
\newtheorem{lemma}{Lemma}
\newtheorem{corollary}{Corollary}
\newtheorem{definition}{Definition}
\newtheorem{question}{Questions}
\newtheorem{example}{Example}
\newtheorem{examples}{Examples}
\newtheorem{hypothesis}{Hypothesis}
\newtheorem{context}{Context}
\newtheorem{explication}{Explication}
\newtheorem{remark}{Remark}
\newtheorem{recall}{Recall}
\newtheorem{consequence}{Consequence}
\newcommand{\bproof}{\bigskip {\bf Proof. }}
\newcommand{\eproof}{\hfill\qedsymbol}
\newcommand{\Disp}{\displaystyle}
\newcommand{\qe}{\hfill\(\bigtriangledown\)}
%\setlength{\columnseprule}{1 pt}
\newcommand\mycommfont[1]{\footnotesize\ttfamily\textcolor{red}{#1}}

%%%%%%%%%%%%%%%%%%%%%%
%A bunch of definitions that make my life easier :)

\newcommand{\z}{\mathbf{z}} % z en gras pour les processus à sauts
\newcommand{\iu}{\mathrm{i}\mkern1mu} % Imaginary number i 
\newcommand{\uc}{\boldsymbol{\mathrm{u}}} %commande/control u 
\newcommand{\indep}{\perp \!\!\! \perp}
\newcommand{\prob}{\small \mathbf{p}}
\newcommand{\W}{\boldsymbol{\bar{W}}}
%%%%%%%%%%%%%%%%%%%%%%%%%%%%%%%%%%%

%%%% mathbb%%%%%%%%%%
\def \A{\mathbb{A}}
\def \B{\mathbb{B}}
\def \C{\mathbb{C}}
\def \D{\mathbb{D}}
\def \E{\mathbb{E}}
\def \F{\mathbb{F}}
\def \G{\mathbb{G}}
\def \H{\mathbb{H}}
\def \I{\mathbb{I}}
\def \J{\mathbb{J}}
\def \K{\mathbb{K}}
\def \L{\mathbb{L}}
\def \M{\mathbb{M}}
\def \N{\mathbb{N}}
\def \O{\mathbb{O}}
\def \P{\mathbb{P}}
\def \Q{\mathbb{Q}}
\def \R{\mathbb{R}}
\def \S{\mathbb{S}}
\def \T{\mathbb{T}}
\def \U{\mathbb{U}}
\def \V{\mathbb{V}}
\def \W{\mathbb{W}}
\def \X{\mathbb{X}}
\def \Y{\mathbb{Y}}
\def \Z{\mathbb{Z}}

%%% cali %%%%
\def\Ac{{\cal A}}
\def\Bc{{\cal B}}
\def\Cc{{\cal C}}
\def\Dc{{\cal D}}
\def\Ec{{\cal E}}
\def\Fc{{\cal F}}
\def\Gc{{\cal G}}
\def\Hc{{\cal H}}
\def\Ic{{\cal I}}
\def\Jc{{\cal J}}
\def\Kc{{\cal K}}
\def\Lc{{\cal L}}
\def\Mc{{\cal M}}
\def\Nc{{\cal N}}
\def\Oc{{\cal O}}
\def\Pc{{\cal P}}
\def\Qc{{\cal Q}}
\def\Rc{{\cal R}}
\def\Sc{{\cal S}}
\def\Tc{{\cal T}}
\def\Uc{{\cal U}}
\def\Vc{{\cal V}}
\def\Wc{{\cal W}}
\def\Xc{{\cal X}}
\def\Yc{{\cal Y}}
\def\Zc{{\cal Z}}

% greek letters
\def \a {\alpha}
\def \bt {\beta}
\def\e{{\epsilon}}
\def \g {\gamma}
\def \d {\delta}
\def \eps {\varepsilon}
\def \z {\zeta}
\def \et {\eta}
\def \th {\theta}
\def \la {\lambda}
\def \m {\mu}
\def \n {\nu}
\def \r {\rho}
\def \s {\sigma}
\def \t {\tau}
\def \p {\phi}
\def \vp {\varphi}
\def \ps {\psi}
\def \ch {\chi}
\def\q{\quad}
\def\k{\kappa}
\def \om {\omega}
\def \Gm {\Gamma}
\def \Lam {\Lambda}
\def \Sg {\Sigma}
\def \Ps {\Psi}
\def \Dl {\Delta}
\def \Om {\Omega}
\def \Th {\Theta}
\def \Ph {\Phi}

%relation symbols
\def \ev {\equiv}                    
\def \sb {\subset}              
\def \sp {\supset}                   
\def \sbe {\subseteq}          
\def \spe {\supseteq}        
\def \ax {\approx}
\def \pq {\preceq}
\def\pa{{\partial}}
\def\cd{{\cdot}}
\def\cds{{\cdots}}

% arrow
\def \ro {\Rightarrow}
\def \lro {\longrightarrow}
\def \La {\Longleftrightarrow}
\def \mt {\mapsto}
\def \sa {\searrow}      
\def \na {\nwarrow}

%%% divers %%%%
\def \pp {\mathsf{P}}
\def \dd {\mathsf{D}}
\def \hpsi {\hat{\ps}}
\def \hax {\hat{X}}
\def \hm {\hat{\mu}}
\def \haf {\hat{\F}}
\def \hfc{\hat{\Fc}}
\def \ho {\hat{\Omega}}
\def \hta {\hat{\tau}}
\def \no {\noindent}
\def \l {\rm leb}
\def \proof {\no \textbf{Proof.} }
\def \mut {\tilde{\mu}}
\def \d{{\rm d}}
\def \1{\mathds{1}}
\def \hP{\hat{\P}}
\def \dal{\dot{\alpha}}
\def \tp {\tilde{g}}
\def \tq {\tilde{h}}
\def \oB {\overline{B}}
\def \oS {\overline{\Sigma}}
\def \nd{\theta}
\def \nbd{\Theta}
\def \ox {\overline{X}}
\def \ux {\underline{X}}
\def \ol {\overline{\ell}}
\def \ul {\underline{\ell}}
\def \ot {\overline{\tau}}
\def \ut {\underline{\tau}}
\def \erf {{\rm Erf}}


\begin{document}

\maketitle
\thispagestyle{empty}
\pagestyle{empty}


%%%%%%%%%%%%%%%%%%%%%%%%%%%%%%%%%%%%%%%%%%%%%%%%%%%%%%%%%%%%%%%%%%%%%%%%%%%%%%%%
\begin{abstract}
Following the Kolokoltsov's work \cite{kolokoltsov2022qmfg}, we will present an extension of mean-field control theory in quantum framework. In particular such an extension is done naturally by considering the Belavkin quantum filtering and control theory in a mean-field setting. The state dynamics is described by a controlled Belavkin equation of McKean-Vlasov type, and we prove the well-posedness of the equation under imperfect measurements records and also the propagation of chaos for perfect measurements. Also,  we  apply particle methods to  simulate the mean-field equation and we suggest its application in a stabilizing feedback control.


\textit{Keywords : }Quantum filtering, Stochastic Control, Mean-field Belavkin equation,  McKean-Vlasov Equation, Quantum state reduction, Stabilization in mean-field 

\end{abstract}


%%%%%%%%%%%%%%%%%%%%%%%%%%%%%%%%%%%%%%%%%%%%%%%%%%%%%%%%%%%%%%%%%%%%%%%%%%%%%%%%
\section{Introduction}

\noindent Mean-field game (MFG) theory, lying at  the intersection of game theory and stochastic control theory, is the study of strategic decision made by interacting indistinguishable agents in very large populations. This class of problems was considered in the engineering literature by Huang,  Malhame and Caines \cite{huang2006large} and independently and around the same time by mathematicians Lasry and  Lions \cite{lasry2006mfg1,lasry2007mfg}. Namely, consider $N$ agents whose states evolve according to the stochastic differential equations below: for $j=1,\ldots, N$, 
\begin{align*}
\mathrm{d}X^{\uc,j}_t &= b(X^{\uc,j}_t,u^{j}_t,\hm_t^{\uc, N})\mathrm{d}t + a(X^{\uc,j}_t,u^{j}_t,\hm_t^{\uc, N})\mathrm{d}W_t^{j} \\
\hm_t^{\uc, N}&:=\frac{1}{N}\sum_{k=1}^{N}\delta_{X^{\uc,k}_t},\\
\hm^{\uc, N}&=(\hm_t^{\uc, N})_{0\leq t\leq T},
\end{align*}
where $T$ is supposed as the final time, and $b, a$ are suitable  functions and $W^1,\ldots, W^N$ are independent Brownian motions. Here $X^{\uc,j}_t$ stands for the state of agent $j$ at time $t$ subject to strategy profile $(u^1,\ldots, u^N)=:\uc$, and each agent interacts with the others through the empirical measure $\hm_t^{\uc, N}$. Provided some set $\Uc$ of admissible  strategies and a time horizon $T$, agent $j$ aims to minimize  its cost $\Uc\ni u\mapsto \mathcal{J}_j(u)\in\R$ with $\mathcal{J}_j(u)\equiv \mathcal{J}(u, \hm^{\uc_j,N})$:  
$$\mathcal{J}_j(u):= \mathbb{E}\left[\int_{0}^{T} f(X^{\uc_j,j}_t,u_t^j,\hm_t^{\uc_j,N})\d t + g(X_T^{\uc_j, j},\hm_T^{\uc_j,N})\right],$$
where $\uc_j:= (u^1,\ldots, u^{j-1}, u, u^{j+1},\ldots, u^N)$  and $f, g$ are some cost  functions. Nash equilibrium, where no player can do better by unilaterally changing their strategy,  is the most common way to define the solution of such a non-cooperative game. Namely, $\uc^*:=(u^{*,1},\ldots, u^{*,N})\in\Uc$ is said to achieve a Nash equilibrium if 
$$\mathcal{J}(u^*,\hm^{\uc^{*},N})= \inf_{u\in\Uc} \mathcal{J}(u,\hm^{\uc^*_j,N}),$$ where $\uc_j^*:= (u^{*,1},\ldots, u^{*,j-1}, u, u^{*,j+1},\ldots, u^{*,N}).$ 
Generally there is no explicit expression for the Nash equilibrium, and its numerical computation is quit costly. Given  the importance for applications, as well as its active theoretical interest, it becomes increasingly important to consider the mean-field limit as $N\to\infty$. Hence, the corresponding MFG consists of finding a pair  $(\hat u, \hat X)$ satisfying
\begin{align*}
&\mathrm{d}\hat{X}_t = b(\hat{X}_t,u(\hat{X}_t),\Lc(\hat{X}_t))\mathrm{d}t + a(\hat{X}_t,u(\hat{X}_t),\Lc(\hat{X}_t))\mathrm{d}W_t\\
&\mathcal{J}(\hat u,\hm) \leq \mathcal{J}(u,\hm),
\end{align*}
where $\hm:=(\Lc (\hat X_t))_{0\leq t\leq T}.$ 

The empirical measure playing a major role in classical MFG, does not have an analogue in quantum setting, since $N-$particle quantum evolution particles are not separated in individual dynamics, for instance, due to entanglement between particles. The other difficulty is related to quantum measurements which perturb the state of the system, which is known as a back-action effect. Moreover, measuring continuously freezes the  dynamics of the system \cite{sudarshan77}. Hence a new methodology is required to build a quantum MFG theory. 





In a remarkable series of papers \cite{kolokoltsov2021law,kolokoltsov2022dynamic,kolokoltsov2022qmfg,kolokoltsov2021qmfgcounting},  Kolokoltsov has developed a  methodology for quantum MFG, where indirect measurements are considered to conserve the system's dynamics. In this framework, the dynamics is described by matrix-valued stochastic differential equations. As same as for the classical case, the propagation of chaos has been derived by adopting the approach of Pickl \cite{pickl11simple} to a stochastic version. 
It should also be noted that this new framework allows us to deal with a measurement-based feedback control problem of quantum systems with high dimensionality. Feedback control of  quantum systems plays a major role in controlling quantum systems in a robust fashion, see e.g.,  \cite{serafini12feedback,gough13,van2005quantum,wiseman2009quantum}. Due to high dimensionality of the system, realization of a feedback control in real-time is time-consuming and not practical in a real experiment. 

%\textcolor{purple}{
%Similar to the classical case, the behavior of typical particles in MFG problems can be justified through the propagation of chaos concept. Kolokoltsov extended this idea by incorporating a stochastic version of the Pickl approach \cite{pickl11simple} to obtain mean-field limits for interacting quantum systems.}


In the following, we recall Belavkin quantum filtering equation \cite{boutenhandel07,gough22,ohki18,belavkin05QuantumNB} and we discuss the extension of  classical mean-field games and control characteristics in the quantum filtering framework, which is proposed in \cite{kolokoltsov2022qmfg}. The quantum filtering framework represents a natural one to construct a quantum mean field game theory.
Later, motivated by games with incomplete information, we extend the mean-field Belavkin equation in case of imperfect measurement records, and give a proof of the well-possedness of such the equation.
Finally,  we use particles methods algorithm to simulate the mean-field equation.  We suggest the use of quantum mean-field filtering as a method to reduce the complexity representation of open quantum systems, which is usually high. In case of quantum non-demolition measurement,  simulations illustrate a quantum state reduction for mean-field particle. This is encouraging to apply such a method, for instance, in feedback stabilization based on such a mean field theory. Inspired by \cite{liang2018exponential}, we construct a control law depending on the mean field equation, through simulations, we observe stabilization of the system toward the target state.
\subsection{Preliminaries}
%\noindent For a generic Polish space $E$, denote by $\Pc(E)$ the set of Borel probability distributions on $E$
%\textcolor{purple}{and by $\Pc_2(E)\subset \Pc(E)$ its subset of measures admitting finite second moment. We endow $\Pc_2(E)$ with the Wasserstein metric of order $2$, denoted by $\Wc_2$.}
%For each $\lambda \in \Pc(E)$, $L^2(E,\lambda)$ stands for the Hilbert space of measurable functions $k:E\to \C$ such that
%$$\int_E |k(x)|^2\lambda(\d x)<\infty.$$ 
We fix throughout the paper a finite set $\Xc = \{1,\dots,d\}$, and set $\mathbb{H} := \C^{d}.$
Let $\Mc_d(\C)$ be the set of $d\times d$ complex matrices. For every $A\in \Mc_d(\C)$, denote by $A^\dag$ its conjugate transpose. Define further the set of density matrices ${S}_{d}:=\{ \rho \in \mathcal{M}_{d}(\C) :~ \rho = \rho^{\dag}, ~ \rho \geq 0,~   
 tr(\rho) = 1 \}$. For any   $A,B\in \Mc_{d}(\C)$, set $[A,B] := AB - BA$  and $\{A,B\} := AB + BA$.
 

For $N\in\N$, let $\mathbb{H}^{\otimes {N}}$ be the $N-$tensor product  of $\H$. For any operator $B:\H\to\H$. For $j=1,\ldots,N,$ denote by ${\bf B}_j :\mathbb{H}^{\otimes {N}} \to\mathbb{H}^{\otimes {N}}$ the operator acting only on the sub-system living on the $j$-th Hilbert space $\H$, i.e. ${\bf B}_j(h_1,\ldots,h_j,\ldots, h_N):=(h_1,\ldots,B(h_j),\ldots, h_N)$. Similarly for any operator $O:\H^2\to\H^2$ and for $j\neq k\in \{1,\ldots,N\}$ denote by ${\bf O}_{j,k} :\mathbb{H}^{\otimes {N}} \to\mathbb{H}^{\otimes {N}}$ the operator acting only on the sub-systems living on the product of $j$-th and $k$-th Hilbert spaces $\H$, i.e. 

${\bf O}_{j,k}(h_1,\ldots,h_j,\ldots,h_k,\ldots, h_N):=(h_1,\ldots,O_1(h_j,h_k)\ldots,O_2(h_j,h_k),\ldots, h_N)$, where $O(.,.)=(O_1(.,.),O_2(.,.)).$
\iffalse
\begin{align*}\tiny
\mathbf{\sigma}_x = \begin{pmatrix}0 & 1 \\ 1 & 0\end{pmatrix}, \quad
\mathbf{\sigma}_y = \begin{pmatrix}0 & -\iu \\ \iu & 0\end{pmatrix}, \quad
\mathbf{\sigma}_z = \begin{pmatrix}1 & 0 \\ 0 & -1\end{pmatrix},\quad
\mathbf{I} = \begin{pmatrix}1 & 0 \\ 0 & 1\end{pmatrix}
\end{align*}
as,
\begin{align*}
\small \rho = \frac{1}{2}\big( \mathbf{I} + x\sigma_x + y\sigma_y + z\sigma_z\big) =\frac{1}{2}\begin{pmatrix} 1 + z & x - \iu y \\ x + \iu y & 1 - z \end{pmatrix}
\end{align*}
\fi

\section{Quantum filtering and control}
Now that we have examined the characteristics in the classical case, we want to extend them to the case where particles obey to the principles of quantum mechanics.

In a dynamic game situation with $N$-players. The strategies are made in real-time, and therefore the system must be measured continuously.  The quantum system to be considered is therefore necessarily open, and to be able to observe the evolution of the state and avoid quantum Zenon effect we have to pass through indirect measurements \cite[4.Watchdog effect]{belavkin92}. The control induced by each player is done via a controlled electromagnetic field.

%The irreversible and stochastic behaviour of the continuously observed quantum system expressed by the so–called collapse or reduction of the wave function has no analogue in the classical deterministic mechanics. The Hamiltonequations do not depend (for a nondemolition observation) on whether the dynamical object is observed during its motion along its trajectory. Ignoring that difference in the behaviour of classical and quantum observed objects leads to various quantum paradoxes of Zeno kind [5–11] which can be explained only in the way of a consistent investigation of the disturbed stochastic dynamics of the quantum system undergoing an observation.
%Firstly, due to entanglement, a composite system is described by a tensor product. Therefore, it no longer makes sense to treat the individual trajectories of particles in the system in the phase space.
%So in this case the dynamic evolution of a system described by the Schrodinger-Von Neumann equation\footnote{We write directly the equation in terms of density matrix because it's more general and take into account the mixed states.} $$ \frac{\mathrm{d}\rho_t}{\mathrm{d}t} = -\iu[\mathbf{H},\rho_t]$$

A mathematical model to describe, an open quantum system undergoing continuous-time measurements 
observed through a quantum channel is given by the matrix-valued stochastic differential equation called Belavkin equation:
\begin{align*}
\mathrm{d}\rho_t&= \left(-\iu[{H} + u(\rho_t)\hat{H},\rho_t] + \big( L\rho_tL^{\dag} - \frac{1}{2}\big\{L^{\dag}L,\rho_t\big\}\right)\mathrm{d}t \\
+& \sqrt{\eta}\left(L\rho_t + \rho_t L^{\dag} - tr\big((L + L^{\dag})\rho_t\big)\rho_t\right)\mathrm{d}W_t.
\end{align*}
Here ${H}$ is the free Hamiltonian and $\hat{H} $ corresponds to the controlled Hamiltonian. The matrix $L$ is measurement operator associated to the probe.
The observation process of the probe $Y$ is a continuous semimartingale with $\mathrm{d}Y_t = \mathrm{d}W_t + \sqrt{\eta}\,tr\big((L+ L^{\dag})\rho_t\big)\mathrm{d}t,$ where $dW$ is a classical Wiener process.  Here $u $ denotes the feedback controller adapted to $\mathcal{F}^{Y_t}$ and
$\eta \in (0,1)$ represents the efficiency of the detector.
%A low efficiency means that not all the photons emitted by the system are detected, leading to a lack of information about the true state of the system.

\begin{remark}
In absence of control input and in case of $\eta = 0$, the dynamics is described by a deterministic linear master equation, called Lindblad master equation.
\end{remark}

\section{Quantum 
$N$-particle system and mean-field limit}
\subsection{Belavkin equation for a controlled $N$-particle system}

\noindent Now we consider a system of $N$-quantum particles, where each particle interacts with the  others through an interaction Hamiltonian denoted by ${A}$. Each particle is measured indirectly through an appropriate observable, i.e., $N$-quantum channels are considered. To each particle, a feedback control is applied to achieve certain goals like minimizing a  cost function, maximizing a reward, stabilizing the system, etc. Under our setting, 
${A}$ is given as a symmetric self-adjoint integral operator with Hilbert-Schmidt kernel, i.e. ${A}:\Xc^4\to\C$ is so that ${A}(l,l';k,k') = {A}(l',l;k',k)$,  
${A}(l,l';k,k') = \overline{{{A}(l,l';k,k')}}.$ %and 
%$$||{A}|| :=  \sum_{\Xc^4} |{A}(l,l';k,k')|^2.$$
For $\psi:\Xc^2\to\C$, ${A}\psi:\Xc^2\to\C$ is defined by
$${A}\psi(l,l'):=\sum_{\Xc^2} {A}(l,l';k,k')\psi(k,k').$$

Hence, the dynamics of the system, identified by the density matrix $\boldsymbol{\rho}^N$, satisfies the Belavkin equation
\iffalse
\begin{align}
\mathrm{d}&\boldsymbol{\rho}_t^{N} = -\iu[\mathbf{H},\boldsymbol{\rho}_t^{N}]\mathrm{d}t + \sum_{j=1}^N \Big({\bf L}_j\boldsymbol{\rho}_t^{N}{\bf L}_j^{\dag} - \frac{1}{2}\big\{{\bf L}_j^{\dag}{\bf L}_j,\boldsymbol{\rho}_t^{N}\big\}\Big)\mathrm{d}t\nonumber\\
&+\sqrt{\eta}\sum_{j=1}^N\Big(\boldsymbol{\rho}_t^{N}{\bf L}^\dag_j+{\bf L}_j\boldsymbol{\rho}_t^{N}\big - tr\big(({\bf L}_j+{\bf L}_j^{\dag})\boldsymbol\rho_t^{N}\big)\boldsymbol\rho_t^{N}\Big)\mathrm{d}W_t^{j},
\label{eq:particle}
\end{align}
\fi
where $\boldsymbol{\rho}_0^{N}=\rho_0^{\otimes N}$, $\mathbf{H}:=\sum_{j} (\mathbf{H}_{j} + u(\rho_t^{j})\mathbf{\hat{H}}_j )+ \sum_{j<k}\mathbf{A}_{jk}/N$, where $\rho_t^{j}$ represents the state of the particle $j$ (for $j=1,\cdots,N$), which can be obtained by taking a partial trace over the other particles.  The corresponding observation process $Y^j$ for particle $j$ is given by
$$\mathrm{d}Y_t^{j} = \mathrm{d}W_t^{j} + \sqrt{\eta}tr\big( ({\bf L}_j + {\bf L}_j^{\dag})\rho_t^{j}\big)\mathrm{d}t.$$
Here we note that the above equation is well posed by using similar arguments as in \cite[Propositions 3.3 and 3.5]{mirrahimiHandel07}.
\subsection{Mean-field limit }

As in classical case, we expect that for an appropriate interaction Hamiltonian, when $N$ is large, each particle interacts with a mean-field instead of interacting individually with the others, and typical behavior for particle emerges.
For a closed quantum system described by Schr\"odinger equation, the mean field dynamics is given by the well-known Schr\"odinger-Hartee equation \cite{lewin2014derivation,pickl11simple}, and Lindblad-Hartee equation for open quantum systems \cite{merkil2012}. Later, this equation is extended by Kolokoltsov \cite{kolokoltsov2021law,kolokoltsov2022qmfg,kolokoltsov2021qmfgcounting} to treat the case of an open quantum system undergoing continuous-time measurements.

In the following, we consider the later treatment  and we assume in addition that measurements are not perfect, inspired by the previous treatment, we recover formally the following Belavkin equation of mean-field type: 
\begin{align}
\mathrm{d}\gamma_t&=  
(-\iu[ {H} + u(\gamma_t){\hat{H}} + {A}^{{m}_t}, \gamma_t] )\mathrm{d}t\nonumber\\
&+ \left(L\gamma_tL^{\dag} - \frac{1}{2}
\{L^{\dag}L,\gamma_t\}\right)\mathrm{d}t\nonumber\\ 
&+\sqrt{\eta}\Big(\gamma_tL^{\dag} + L\gamma_t - tr\big((L + L^{\dag})\gamma_t\big)\gamma_t\Big)\mathrm{d}W_t, 
\label{mfbn}
\end{align}
where  $m_t := \mathbb{E}[\gamma_t]$, $\gamma_0 =\rho_0 \in {S}_{d},$ and $A^m=\sum_{\Xc^2}A(l,l';k,k')\overline{m(k,k')}.$
\begin{remark}
   In the absence of control, it follows by taking  expectation that satisfies a new nonlinear version of Lindblad equation:
\begin{align*}
    \mathrm{d}{m}_t = &-\iu[ {H} + {A}^{m_t} , {m}_t]\mathrm{d}t + \Big(L{m}_tL^{\dag} - \frac{1}{2}\{LL^{\dag},m_t\}\Big)\mathrm{d}t.
\end{align*} 
\end{remark}
\section{Main result }

To justify the above approximation in mean-field limit, we have to show that $\boldsymbol\rho^N$ asymptotically becomes close to ${\boldsymbol\gamma^j}:=I\otimes\cdots\otimes \gamma\otimes\cdots\otimes I,$ where $j$-th component of ${\boldsymbol\gamma^j}$ is identified with $\gamma$ and the other components are all identity operator $I.$ This deviation can be calculated for instance through the following quantity
\begin{align}
\alpha_{N,j}(t) = 1 - tr\big(\gamma_t{\rho}_t^j\big)
= 1 - tr(\boldsymbol\gamma^j_t\boldsymbol\rho^N_t\big)
\label{deviation}
\end{align}
which is calculated only for the particle $j$ and we recall that  
${\rho}_t^{j}$ corresponds to the partial trace of $\boldsymbol\rho^N$ with respect to the particles other than the particle $j.$
By an inequality obtained in \cite[Proposition A.1]{kolokoltsov2022qmfg}, it is sufficient to control $\mathbb{E}[\alpha_N(t)]$ by $\alpha_N(0)$ to state a propagation of chaos result, with $\alpha_N := \alpha_{N,1}$.

In the following theorem, we state the main result of this paper concerning the wellposedness of Equation \eqref{mfbn} and propagation of chaos.

\begin{theorem}[Wellposedness and propagation of chaos]
Let $T > 0,\; U > 0$, and let $u: {S}_{d} \to [-U,U]$ be bounded and Lipschitz, i.e. 
$|u(\rho) - u(\rho') | \leq \kappa\, tr(|\rho-\rho'|)$, with $\kappa > 0.$
Then \eqref{mfbn} is well posed and valued in ${S}_{d}$. 


Furthermore for $\eta = 1$,  there exists $c$ such that
\begin{align*} \mathbb{E}\big[ \alpha_{N}(t) \big] &\leq \exp(c( ||{A}|| + \kappa||{\hat{H}}|| + ||L||^2 )t)\alpha_N(0)\\ &+ \big(\exp (c||{A}||t) - 1 \big)\frac{1}{\sqrt{N}},
\end{align*}
where $||.||$ corresponds to any matrix norm.
\end{theorem}
\begin{proof}
\paragraph{Well-posedness} It is sufficient to show that Equation \eqref{mfbn} is well posed. The proof will be a combination of arguments in \cite{mirrahimiHandel07} and  \cite[Pages 235-237]{carmona2018mfg1}, and can be done through the three steps.



 For each $\xi\in C\big([0,T], {S}_{d}),$ we consider the following equation 
\begin{align}\label{NBol}
    \mathrm{d}{\gamma}_t^{\xi} &= -\iu[ {F}_t^\xi ,{\gamma}_t^{\xi}]\mathrm{d}t + \Big(L{\gamma}_t^{\xi}L^{\dag} - \frac{1}{2}\big\{L^{\dag}L,{\gamma}_t^{\xi}\big\}\Big)\mathrm{d}t\nonumber\\ &+ \sqrt{\eta}\Big(\gamma_t^{\xi}L^{\dag} + L\gamma_t^{\xi} - tr\big((L + L^{\dag})\gamma_t^{\xi}\big)\gamma_t^{\xi}\Big)\mathrm{d}W^{}_t,
\end{align}
where $F_t^\xi:={H} + u(\gamma_t^\xi){\hat{H}} + {A}^{{\xi_t}}.$ This is well posed by similar arguments applied in \cite[Propositions 3.3 and 3.5]{mirrahimiHandel07}.

From the existence of the family of equations parametrized by $\xi$, we define the following mapping $\Xi:C\big([0,T], {S}_{d})\to C\big([0,T], {S}_{d})$ by $\Xi(\xi):=(\E[\gamma_t^{\xi}])_{0 \leq t \leq T}$,
%\begin{remark}
%\textcolor{purple}{As in common except for the dependence upon $T$ which we keep track of, we use the same notation $C_T$ even though the value of this constant can change from line to line.}
%\end{remark}
%In the second step, we show that MF-Belavkin equation with an open-loop control is well-posed with fixed point arguments \cite[Pages 235-237]{carmona2018mfg1}. In the last step we use a truncation argument to extend the results to the case of a closed-loop control as in \cite[Proposition 3.5] {mirrahimiHandel07}.   %This can be proved through the following lemma.
 %\begin{lemma}(\cite[Proposition 3.3]{mirrahimiHandel07})\end{lemma}
%\begin{lemma}$\gamma_t^{\xi}$ lives in  $\mathbf{S}_{d+1}$ is and it's solution to the following stochastic differential equation called Normalize open-loop control equation:\end{lemma}

%In the following, we show that the second step holds true. 
Therefore the process $\gamma^m$ corresponds to the  solution of \eqref{mfbn} if and only if $ m=\Xi({m})$. So we should prove the existence and uniqueness result of the theorem by showing  that the mapping $\Xi$ has a unique fixed point.

To show this, we prove that the map $\Xi$ is a contraction with respect to a uniform norm on $C\big([0,T], {S}_{d}).$


Let us pick two arbitrary elements $\xi^1$ and $\xi^2$ in $C\big([0,T], {S}_{d}).$ Set $\Delta\gamma_t := \gamma_t^{\xi^1} - \gamma_t^{\xi^2},$
     $\Delta{\xi}_t := \xi^1_t-\xi^2_t,$ and 
$K:=L + L^{\dag}.$
    Then it follows that 
%\begin{itemize}
   % \item $L + L^{\dag} = K$
   % \item $\Delta^{\xi}\gamma_t = \gamma_t^{\xi^1} - \gamma_t^{\xi^2}$
   % \item $\mathbf{A}^{\xi^1 - \xi^2} = \mathbf{A}^{\Delta\xi}$
%\end{itemize}
{\small{\begin{align*}
    &\Delta\gamma_t = \int_{0}^{t}\Bigg(-\iu[ {F}^{\xi^1}_s , \Delta \gamma_s] + \Big(L\Delta \gamma_sL^{\dag} - \frac{1}{2}\big\{L^{\dag}L,\Delta\gamma_s\big\}\Big)\Bigg)\mathrm{d}s\\ &+\sqrt{\eta}\int_{0}^{t}\Big(\Delta\gamma_sL^{\dag} + L\Delta\gamma_s\Big)\mathrm{d}W_s -\int_{0}^{t}\big(\iu[{F}^{\xi^{1}}_s-{F}^{\xi^{1}}_s,\gamma_s^{\xi^2}] \big)\mathrm{d}s\\
    &-\sqrt{\eta}\int_{0}^{t}\Big(tr\big(K\gamma_s^{\xi^1}\big)\gamma_s^{\xi^1} - (tr\big(K\gamma_s^{\xi^2}\big)\gamma_s^{\xi^2}\Big)\mathrm{d}W_s,
\end{align*}}}
which yields to the following by H\"older inequality and Itô's isometry
\begin{align*}
   &|\Xi(\xi^1)_t-  \Xi(\xi^2)_t| \le  \mathbb{E}\big[|\Delta\gamma_t|\big]| \\ &\leq\int_{0}^{t}\E\left[\big|[ {F}^{\xi^1}_s , \Delta \gamma_s]\big| + \Big|L\Delta \gamma_sL^{\dag}\Big| + \frac{1}{2}\Big|\big\{L^{\dag}L,\Delta\gamma_s\big\}\Big|\right]\mathrm{d}s\\  
   &+\int_{0}^{t}\E\Big[\Big|[{F}^{\xi^{1}}_s-{F}^{\xi^{1}}_s,\gamma_s^{\xi^2}]\Big| \Big]\mathrm{d}s \\
   &+\sqrt{\eta}\int_{0}^{t}\left(\E[\Big |\Delta\gamma_sL^{\dag} + L\Delta\gamma_s\Big|^2]\right)^{1/2}\mathrm{d}s\\
   & +\sqrt{\eta}\int_{0}^{t}\E\Big[\Big |tr\big(K\gamma_s^{\xi^1}\big)\gamma_s^{\xi^1} - (tr\big(K\gamma_s^{\xi^2}\big)\gamma_s^{\xi^2}\Big |\Big]^{1/2}\mathrm{d}s\\
   &\le \int_{0}^{t} C\mathbb{E}\big[|\Delta\gamma_s|+|\Delta\xi_s|\big] \mathrm{d}s,  
   \end{align*}
   where  $C>0$ is some constant depending on $T, ||{H}||, ||{A}||, \kappa, \eta, ||{\hat{H}}||, ||L||$. %and further by assumptions and straightforward computation \textcolor{purple}{The three first lines are linear under the square with respect to $\Delta^{\xi}\gamma$, so we have easily a constant $C_T$, such that } :
%\begin{align*}\mathbb{E}\Bigg[ &\sup_{0 \leq r \leq t}\Big|\int_{0}^{r}[ \mathbf{F}^{}_s , \Delta^{\xi}\gamma_s]\mathrm{d}s\Big|^2 \Big]+\\
 %   &\sup_{0 \leq r \leq t}\Big|\int_{0}^{r}\Big(L\Delta^{\xi}\gamma_sL^{\dag} - \frac{1}{2}\big\{L^{\dag}L,\Delta^{\xi}\gamma_s\big\}\Big)\mathrm{d}s\Big|^2 \Big]+\\
  %  & \eta\sup_{0 \leq r \leq t}\Big|\int_{0}^{r}\Big(\Delta^{\xi}\gamma_sL^{\dag} + L\Delta^{\xi}\gamma_s\Big)\mathrm{d}W^{}_s\Big|^2\Big]\Bigg ] \leq \\&\quad\quad C_T\Bigg(\int_{0}^{t} \mathbb{E}\Big[ \sup_{0 \leq r\leq s}|\Delta^{\xi}\gamma_r|^2\Big]\mathrm{d}s\Bigg)
%\end{align*}
%So now we control the non-linear drift part. We add and subtract $A^{\xi^2}.$ The triangular inequality and the fact that $\gamma_s^{\xi^2}\in \mathbf{S}_{d},$ allows us to  conclude $\Big|[\mathbf{A}^{\Delta\xi},\gamma^{\xi^2}]\Big| \leq  C\Big|\mathbf{A}^{\Delta\xi}\Big|.$
%More precisely, we have 
%\begin{align*}
%&\mathbb{E}\Bigg[\sup_{0 \leq r \leq t}\Big|\int_{0}^{r}\big([\mathbf{A}^{\xi_s^{1}},\gamma_s^{\xi^1}] - [\mathbf{A}^{\xi_s^{2}},\gamma_s^{\xi^2}]\big)\mathrm{d}s\Big|^2\Bigg] \leq \\
%&2\mathbb{E}\Bigg[ \sup_{0 \leq r \leq t}\Bigg|\int_{0}^{s}[\mathbf{A}^{\xi_r^{1} - \xi_r^{2}},\gamma_r^{\xi^1}]\mathrm{d}r\Bigg|^2\Bigg]\\
%&+2\mathbb{E}\Bigg[\sup_{0 \leq r \leq t} \Bigg|\int_{0}^{s}[\mathbf{A}^{\xi^2_r},\Delta^{\xi}\gamma_r]\mathrm{d}r\Bigg|^2\Bigg]\\
\iffalse 
&\leq C_T\Bigg(\int_{0}^{t} \mathbb{E}\Big[ \sup_{0 \leq r \leq s}|\Delta^{\xi}\gamma_r|^2\Big] + \mathbb{E}\Big[ \sup_{0 \leq r \leq s} |\mathbf{A}^{\Delta\xi}|^2\Big]\mathrm{d}s\Bigg)\\
&\leq C_T\Bigg(\int_{0}^{t} \mathbb{E}\Big[ \sup_{0 \leq r \leq s}|\Delta^{\xi}\gamma_r|^2\Big] + \mathcal{W}_2(\xi_s^{1},\xi_s^{2})^2\mathrm{d}s\Bigg)
\end{align*}
\textcolor{purple}{The last term can be bounded in the same manner as follows}
\begin{align*}
\mathbb{E}\Bigg[ &\sup_{0 \leq s \leq t }\Big| \int_{0}^{s}tr\big(K\gamma_s^{\xi^1}\big)\gamma_s^{\xi^1} - tr\big(K\gamma_s^{\xi^2}\big)\gamma_s^{\xi^2} \big)\mathrm{d}W_s\Big|^2 \Bigg]\\
&\leq \mathbb{E}\Bigg[ \int_{0}^{t} \Big|tr\big(K\gamma_s^{\xi^1}\big)\gamma_s^{\xi^1} - tr\big(K\gamma_s^{\xi^2}\big)\gamma_s^{\xi^2}\Big|^2\mathrm{d}s \Bigg] \leq \\
&2\int_{0}^{t}\mathbb{E}\Bigg[ \Big|tr(K\Delta^{\xi}\gamma_s)\gamma_s^{\xi^2}\Big|^2 + \Big|tr(K\gamma_s^{\xi^1})\Delta^{\xi}\gamma_s\Big|^2 \Bigg]\mathrm{d}s\\
&\leq C_T\Bigg(\int_{0}^{t} \mathbb{E}\Big[ \sup_{0 \leq r \leq s}|\Delta^{\xi}\gamma_r|^2\Big]\mathrm{d}s\Bigg)
\end{align*}
\fi
In view of Gronwall's inequality, one concludes the existence of some constant, still denoted by $C$ without any danger of confusion
$$\max_{0\le t\le r}|\Xi(\xi^1)_t-  \Xi(\xi^2)_t|\le C\|\Delta \xi\|_r,\quad \forall r\le T$$
with $\|\Delta \xi\|_r:=\max_{0\le t\le r} |\Delta \xi_t|$. Iterating this procedure, one has by induction
\iffalse
Therefore,
$$\sup_{0 \leq r \leq T}\mathcal{W}_2(\Xi(\boldsymbol\xi^1)_r,\Xi(\boldsymbol\xi^2)_r)^2 \leq C_T\int_{0}^{T}\mathcal{W}_2(\xi_r^1,\xi_r^2)^2\mathrm{d}r$$
By re-applying $\Xi$, 
\begin{align*}
\sup_{0 \leq r \leq T}&\mathcal{W}_2\Big( \Xi^2(\boldsymbol\xi^1)_r,\Xi^2(\boldsymbol\xi^2)_r\Big)^2
\\ &\leq C_T^2\int_{0}^{T}(T-r)\mathcal{W}_2(\xi^1_r,\xi^2_r)^2\mathrm{d}r. 
\end{align*}
\fi 
$$\max_{0\le t\le r}|\Xi^{(k)}(\xi^1)_t-  \Xi^{(k)}(\xi^2)_t|\le \frac{C^k}{k!}\|\Delta \xi\|_r,\quad \forall r\le T,$$
where $\Xi^{(k)}$ denotes the $k-$composition of $\Xi$. So for $k$ large enough $\Xi^{(k)}$ is a  contraction and hence, admits a unique
fixed point.
\paragraph{Propagation of chaos }
Here our objective is to estimate   the mean of deviation defined in Equation \eqref{deviation} by  an inequality depending on the deviation in initial time. 

In order to do this, it is sufficient to estimate this quantity for one of the particles, for instance here we consider the first particle. For the sake of simplicity, we adapt our notations as follows: 
$\boldsymbol{\gamma} := \boldsymbol{\gamma}^{1}$, $\mathbf{L} := \mathbf{L}_1.$





%We calculate the differential of $\alpha_N$ and %use Gronwall lemma. 



By  Itô formula,  we get 
    \begin{align*} \mathrm{d}\alpha_N^{}(t) &= -tr\big(\mathrm{d}\boldsymbol\rho_t^{N}\mathbf\gamma_t^{}\big) -tr\big(\boldsymbol{\rho}_t^{N}\mathrm{d}\mathbf\gamma_t^{}\big) - tr\big(\mathrm{d}\boldsymbol\rho_t^{N}\mathrm{d}\mathbf\gamma_t^{}\big).\\  &= \big(P_t^{(1)} + 
     P_t^{(2)})\mathrm{d}t + \sum_k P_t^{(3,k)}\mathrm{d}W_t^{k},
    \end{align*} 
where,
\begin{align*}
&P_t^{(1)} =\\
&\iu tr\Big([\frac{1}{N}\sum_{k \neq 1}\mathbf{A}_{k1} - \mathbf{A}_1^{m_t} + \big(u(\rho_t^{1}) - u(\gamma_t)\big)\mathbf{\hat{H}}, \mathbf{I}_1 - \boldsymbol{\gamma}_t\big]\boldsymbol\rho_t^{N}\Big)\\
P_t^{(2)} &\!=\! - tr\big(\boldsymbol{\gamma}_t \boldsymbol{L}\boldsymbol\rho^N_t\boldsymbol{L}^{\dag} \!+ \boldsymbol{\gamma}_t \boldsymbol{L}^{\dag}\boldsymbol{\rho}^N_t \boldsymbol{L} + \boldsymbol{\gamma}_t \boldsymbol{L}^{\dag}\boldsymbol{\rho}^{N}_t\boldsymbol{L}^{\dag} + \boldsymbol{\gamma}_t \boldsymbol{L} + \boldsymbol{\rho}^{N}_t\boldsymbol{L}\big) \\ 
&+\Big[tr\big(\boldsymbol{\gamma}_t\boldsymbol\rho^N_t\boldsymbol{L}^{\dag} + \boldsymbol{\gamma}_t \boldsymbol{L}\boldsymbol\rho^N_t\big)tr\big(\boldsymbol{\gamma}_t(\boldsymbol{L}^{\dag}+\boldsymbol{L})\big) + \\
&tr\big(\boldsymbol{\gamma}_t\boldsymbol\rho^N_t\boldsymbol{L}^{\dag} + \boldsymbol{\gamma}_t L\boldsymbol\rho^N_t\big)tr\big(\boldsymbol\rho^N_t(\boldsymbol{L}^{\dag}+\boldsymbol{L})\big) - \\
&tr\big(\boldsymbol\rho^N_t\boldsymbol{\gamma}_t\big)tr\big(\boldsymbol\rho^N_t(\boldsymbol{L}^{\dag} + \boldsymbol{L}\big)\Big)tr\big(\boldsymbol{\gamma}_t(\boldsymbol{L}+\boldsymbol{L}^{\dag})\big)\Big],
\end{align*}
and $P_t^{(3,k)}$ are bounded quantities. Taking expectation of the above equation, it follows from the proof of \cite[Theorem 3.1]{kolokoltsov2022qmfg} that there exists $C>0$ such that
\begin{align*}
\frac{\d\E[\alpha_N(t)]}{\d t}
&=\mathbb{E}\big[|P_t^{(1)}|\big]+\mathbb{E}\big[|P_t^{(2)}|\big]  \\
&\le \big(C ||{A}||+\kappa||{\hat{H}}||\big)\mathbb{E}[\alpha_N(t)] + \frac{C}{\sqrt{N}} +\mathbb{E}\big[|P_t^{(2)}|\big].
\end{align*}
As for $P^{(2)}_t,$ we combine Lemma \ref{lem1} and \cite[Inequality (44) of Lemma 6.1]{kolokoltsov2021law}, and obtain $|P_t^{(2)}| \leq C'||L||^2\alpha_N(t)$ for some $C'>0$. Therefore, the proof is fulfilled by Gronwall inequality. 
\end{proof}
It remains to prove Lemma \ref{lem1} which 
 generalizes \cite[Inequality (43) of Lemma 6.1]{kolokoltsov2021law} and proves 
 \eqref{deviation} without assuming that  $\gamma$ is supposed to be an one-dimensional projector. 
\begin{lemma}\label{lem1}
Let $A, B$ be in $\mathbf{S}_d,$ and $L$ be a hermitian matrix. Then 
\begin{align} &\Bigg|tr(L A LB) - \frac{1}{2}tr(B(L A + A L))tr(BL + A L) \label{inemt}\\
&\!\!\!\!+ tr(B A)tr(BL)tr(A L) \Bigg| \leq 18||L||^2tr\big((I-A)B\big)\nonumber.
\end{align}
\label{deviation}
\end{lemma}
\begin{proof}
To prove the lemma we decompose $A$ as $ A= \sum_{k}A_{kk} I_{kk},$\text{where } $\sum_{k}A_{k} = 1$ and $I_{kk}$ is the matrix whose only non-zero element is one on the $k$-th diagonal element.
$$ \alpha:= tr\big( (I -A)B\big) = tr\big( (I-B)A \big) = \sum_{k}A_{kk}\alpha_k.$$

%\vspace{-1em}

By the positivity of $B$, it follows that (see \cite{kolokoltsov2021law} for further details) 
\begin{equation} |B_{jr}| \leq \alpha_k, \; j,r \neq k \quad \max\big(|B_{jk}|,|B_{kj}|\big) \leq \sqrt{\alpha_k}, \; j \neq k 
\label{inl}\end{equation}

The left hand side of the inequality \eqref{inemt} can be rewritten as follows
\begin{align*}
&\Bigg| \sum_{{k}}A_{{kk}}tr(LI_{{kk}}LB) +tr(BL)tr(B A)tr(A L) \\ &-\frac{1}{2}\sum_{{k}}A_{{kk}}tr\Big(B(LI_{{kk}} + I_{{kk}}L)\Big)tr\Big(BL + \sum_{{j}}A_{{jj}}I_{{jj}}L\Big)\Bigg|\\
%&\text{By triangular inequality}\\
&\leq \sum_{{k,j}}A_{{kk}}A_{{jj}}\Bigg|B_{{kk}}L_{{jj}}tr(BL) +(LBL)_{{kk}} \\
&- \frac{1}{2}\Big((BL)_{{kk}} + (LB)_{{kk}} \Big)\Big( tr(BL) + L_{{jj}} \Big) \Bigg|\\
&{\leq \sum_{{k}}5||L||^2A_{kk}\alpha_{k}}+\sum_{{k\neq j}}A_{{kk}}A_{{jj}}\Bigg|B_{{kk}}L_{{jj}}tr(BL) +(LBL)_{{kk}} \\
&- \frac{1}{2}\Big((BL)_{{kk}} + (LB)_{{kk}} \Big)\Big( tr(BL) + L_{{jj}} \Big) \Bigg|,
\end{align*}
where the first term in the right hand side is obtained from Equation \eqref{inl}. By adding and subtracting $B_{jj}, L_{kk},$ we obtain the following inequality 
\begin{align*}
&\leq 5||L||^2\alpha+ \sum_{{j} \neq {k}}A_{{kk}}A_{{jj}} 5||L||^2\alpha_k \\&+ \sum_{{j} \neq {k}}A_{{kk}}A_{{jj}}\Bigg|[(B_{{kk}} - B_{{jj}})]L_{{kk}}tr(BL)\Bigg|\\&+   \Bigg|\sum_{{j} \neq {k}}(L_{kk} - L_{jj})A_{kk}A_{{jj}}\Big((BL)_{{kk}} + (LB)_{{kk}}\Big)  \Bigg| \\
 &\leq 12||L||^2 \alpha +   \Bigg|\sum_{{j} \neq {k}}(L_{kk} - L_{jj})A_{kk}A_{{jj}}\Big((BL)_{{kk}} + (LB)_{{kk}}\Big)  \Bigg|
\\&\leq  12||L||^2 \alpha\\
&+\Bigg|\sum_{k}\sum_{j \neq k}A_{jj}A_{kk}(L_{kk} - L_{jj})\sum_{r \neq j,k}\Big[B_{kr}L_{rk} + L_{kr}B_{rk}\Big]\Bigg|\\ &+ \Bigg|\sum_{k}\sum_{j \neq 
 k}(L_{kk} - L_{jj})A_{jj}A_{kk}\Big[B_{kk}L_{kk} + L_{kk}B_{kk}\Big]\Bigg|\\ &+ \Bigg|\sum_{{k}}\sum_{{j} \neq {k}}A_{{jj}}A_{{kk}}(L_{kk} - L_{jj})\Big[B_{{k}{j}}L_{{j}{k}} + L_{{k}{j}}B_{{j}{k}}\Big]\Bigg|\\
& \leq 12||L||^2\alpha+4||L||^2\alpha+2||L||^2\alpha,
 \end{align*}
where we apply Fubini and triangular inequality for the second and third terms and use the fact that the last term is equal to zero by symmetry. 
 

So the lemma is proved. 
\end{proof}


%\begin{align*}\mathrm{d}\tilde{\gamma}_t^{(j)} = &-\iu[ \mathbf{H}_f, \tilde{\gamma}_t^{(j)}]\mathrm{d}t\\ &- \iu[\mathbf{A}^{\mathrm{m}_t},\tilde{\gamma}_t^{(j)}]\mathrm{d}t\\  &+ \Big(L\tilde{\gamma}_t^{(j)}L^{\dag} - \frac{1}{2}\big\{L^{\dag}L,\tilde{\gamma}_t^{(j)}\big\}\Big)\mathrm{d}t\\ &+ \Big(\tilde{\gamma}_t^{(j)}L^{\dag} + L\tilde{\gamma}_t^{(j)}\Big)\mathrm{d}Y^{(j)}_t.\end{align*}


\section{Numerical illustration and applications}
%\subsection*{N-body system}
 Here we focus on measurement-based feedback control strategies stabilizing quantum systems toward a chosen target state, see e.g., \cite{mirrahimiHandel07,qibo10} for a mathematical model description.

%Suppose that initial state of the system is 
%$\boldsymbol\rho_0^{N} = \sum_{j}p_j\boldsymbol{\rho_{0,j}}$,
%where the states $\boldsymbol{\rho_{0,j}}$ has corresponding probability  $\sum_{j} p_j = 1$ and our target is to prepare a desired states says $\boldsymbol{\rho}_e = \underbrace{{\rho}_e\otimes\dots\otimes{\rho}_e}_{N \text{times}}$ with high fidelity.

In the following section, we consider $N$-quantum particles and consider stabilization of such a system.  Here the key point is to make the feedback control depend on   the mean-field limit equation $\gamma$ instead of the original filter equation $\boldsymbol\rho^N$ and to control $\gamma$ in situation where the interaction between particles is of mean-field type, by doing this the complexity of the problem can be reduced notably as it is sufficient to  control only the mean-field particle toward a target state. Our analysis here is only numerical.

\subsection{$N$-quantum particles}
We consider the case of $N$-qubit system (i.e $\mathcal{X} = \{1,2\}$), interacting through a Hamiltonian of mean-field type $\mathbf{A}.$

Suppose that the interaction operator between qubits is given by an operator describing the  exchange of photons \cite[Discussion.6
]{kolokoltsov2021qmfgcounting}, \cite[Remark 8]{kolokoltsov2022dynamic}, $A = a_1^{\dag}a_2 + a_2^{\dag}a_1$. This operator represents the exchange of a single photon between two qubits, where $a_{j}^{\dag}$ and $a_{j}$ are the creation and annihilation  operators respectively for the $j$-th qubit. The first term describes the process where a photon is absorbed by the first qubit, while the second qubit emits a photon. The second term describes the opposite process. This interaction is given by the tensor $A(l, l'; k, k')$ such that $A(2, 1; 1, 2) = A(1, 2; 2, 1) = 1$ and zeros otherwise.

For each particle we associate a free Hamiltonian $\mathbf{H}_j = \boldsymbol{\sigma_z}^{j}$, the observation channel $\boldsymbol{L}_j = \boldsymbol{\sigma_z}^{j} $, and controlled Hamiltonian $\mathbf{\hat{H}}_j = \boldsymbol{\sigma_x}^{j}$.

The evolution of the $N$-particle state is given by the following equation :
{\small\begin{align*}
\mathrm{d}\boldsymbol{\rho}_t^{N} =& -\iu\sum_{j=1}^{N}[\mathbf{H}_t^{j},\boldsymbol{\rho}_t^{N}]\mathrm{d}t + \sum_{j=1}^N \Big(\boldsymbol{\sigma_z}^{j}\boldsymbol{\rho}_t^{N}\boldsymbol{\sigma_z}^{j} - \boldsymbol{\rho}_t^{N}\Big)\mathrm{d}t\\
&+\sqrt{\eta}\sum_{j=1}^N\Big(\boldsymbol{\rho}_t^{N}\boldsymbol{\sigma_z}^{j}+\boldsymbol{\sigma_z}^{j}\boldsymbol{\rho}_t^{N} - 2tr\big(\boldsymbol{\sigma_z}^{j}\boldsymbol\rho_t^{N}\big)\boldsymbol\rho_t^{N}\Big)\mathrm{d}W_t^{j},
\end{align*}}
where the $\mathbf H_t^{j} = \boldsymbol{\sigma_z}^{j} + u(\rho_t^j)\boldsymbol{\sigma_x}^{j} + \frac{1}{N}\sum_{k<j}\mathbf{A}_{kj}.$

To describe the evolution of $\boldsymbol\rho^{N},$ we need to solve $(2^N)^2 - 1$ real stochastic differential equations so the complexity is of order $O(2^N)$. In the following section, we aim to reduce the complexity of such the system by considering mean field type equation and study numerically a stabilizing feedback control of such a reduced dynamics. 
\subsection{Stabilization of a mean-field equation}
To study stabilization problem, we obtain first the mean field type equation where we need to solve only $2^2 - 1 = 3$ real stochastic differential equations, as the evolution is described by the Bloch sphere coordinates,  however the drawback is that these are of Mckean-Vlasov type, so we require to simulate the interacting system of $N$-particles, therefore the complexity is of order $O(N).$ 

Straightforward calculations in Pauli basis give us 
$$\scriptsize A^{m} = \begin{pmatrix}
0 & \mathbb{E}[x] - \iu \mathbb{E}[y] \\ \mathbb{E}[x] + \iu \mathbb{E}[y] & 0 
\end{pmatrix}.$$
Here the free Hamiltonian is $H = {\sigma_z}$, the observation channel is $L = {\sigma_z} $, and controlled Hamiltonian is $\hat{H} = {\sigma_x}$.

The mean-field Belavkin equation projected in Pauli basis is represented as follows  :
{\small\begin{align*}
\small\mathrm{d}x_t &= \small \Big( - y_t - x_t + z_t\mathbb{E}[y_t]\Big)\mathrm{d}t - \sqrt{\eta}x_tz_t\mathrm{d}W_t,\\
\small\mathrm{d}y_t &= \small\Big( x_t - {y_t} + u(\gamma_t)z_t -z_t\mathbb{E}[x_t]\Big)\mathrm{d}t + \sqrt{\eta}y_tz_t\mathrm{d}W_t,\\
\small\mathrm{d}z_t &= \small\Big(-u(\gamma_t)x_t + y_t\mathbb{E}[x_t] + x_t\mathbb{E}[y_t]\Big)\mathrm{d}t + \sqrt{\eta}\big(1 - z_t^2)\mathrm{d}W_t. 
\end{align*}}

The equation is finite-dimensional Mckean-Vlasov equation on the Bloch sphere, where we can approximate the equation by a system of interacting particles, whose interactions are determined by the empirical measure of the system as follows
{\small\begin{align*}
\mathrm{d}x_t^{j} &= \small \Big( - y^{j}_t - x^{j}_t + z^{j}_t\frac{1}{N}\sum_{k=1}^{N}\delta_{y^{k}_t}\Big)\mathrm{d}t - \sqrt{\eta}x^j_tz^j_t\mathrm{d}W^{j}_t,\\
\mathrm{d}y^{j}_t &= \small\Big( x^{j}_t - {y^{j}_t} + u(\gamma_t^{j})z^{j}_t -z^{j}_t\frac{1}{N}\sum_{k=1}^{N}\delta_{x^{k}_t}\Big)\mathrm{d}t + \sqrt{\eta}y^{j}_tz^{j}_t\mathrm{d}W^{j}_t,\\
\mathrm{d}z^{j}_t &\!\!=\!\! \Big(-u(\gamma_t^j)x^{j}_t + \sum_{k=1}^{N}\!\Big(\frac{y^{j}_t}{N}\delta_{x^{k}_t} \!\!+ \!\!\frac{x^{j}_t}{N}\delta_{y^{k}_t}\Big)\Big)\mathrm{d}t \!\!+ \!\!\sqrt{\eta}(1 - z^{2}_t)\mathrm{d}W^{j}_t.
\end{align*}}%

After that we use Euler schemes discretization on system of particles, propagation of chaos guarantees the convergence, as the number of particles $N$ increases \cite[Pages 129-130]{liu19phd}, see \cite{simulation} for more details.


\subsubsection*{Quantum state reduction }
Before studying the stabilization problem, we need to know the asymptotic behavior of such the equation when the feedback control is turned off, i.e $(u\equiv  0)$, almost surely. In the case of quantum non-demilition measurements, such the asymptotic behavior is known as quantum state reduction and is treated in different papers, see e.g., \cite{handel05red,bauer2013repeated,bauer2011convergence,liang2019exponential}. Here we observe such the property holds true through numerical simulations for the mean field type equation,  %(\textcolor{purple}{We can prove using analytical methods converges but we don't have enough place}) 
where $(\gamma_t)_{t \geq 0}$ converges to one of the following states $ \{\rho_e,\rho_g\}$ with \begin{align*}\rho_g = \begin{pmatrix}1 & 0 \\ 0 & 0\end{pmatrix}, \; \rho_e = \begin{pmatrix}0 & 0 \\ 0 & 1\end{pmatrix}.\end{align*} These are the equilibrium points of the mean-field stochastic differential equation (see Fig. \ref{fig:QSRf}). 
%Recovering a well-known result  called  quantum state reduction in Quantum non-demolition regime \cite{van2005quantum}.  

\subsubsection*{Stabilization by feedback}
To ensure that the system attains a prescribed target, saying $\rho_e$, we adapt a feedback law proposed in \cite{liang2018exponential} for a qubit system. Our proposed feedback has the following form  $u(\gamma) = -7.6\iu tr\big([\sigma_x,\gamma]\rho_e\big) + 5\big( 1 - tr(\gamma\rho_e)\big).$ We observe numerically that such the stabilization is ensured (see Fig. \ref{fig:Stabilization}).
%\begin{minipage}{.5\textwidth}
 % \centering
   \begin{figure}[h!]
  \includegraphics[width=0.7\linewidth]{Quantum_States_Reduction.pdf}
  \caption{\footnotesize{Evolution of $z$-component for the mean-field equation. The red curve represent mean trajectories for arbitrary $100$ samples.}}
    \label{fig:QSRf}
  \end{figure}
%\end{minipage}
%\centering
%\begin{minipage}{.5\textwidth}
 % \centering
  \begin{figure}
  \includegraphics[width=0.7\linewidth]{Fidelity_with_feedback.pdf}
  \caption{{\footnotesize{Convergence of the fidelity $\mathfrak{F}(\gamma,\rho_e) = \Big(tr\sqrt{\sqrt{\gamma}\rho_e\sqrt{\gamma}} \Big)^2$ toward one with initial state $(x_0,y_0,z_0) = (1/4,-1/4,0) $ and $\eta = 0.8$. The red curve represents the mean value of 100 arbitrary samples.}}}
  \label{fig:Stabilization}
  \end{figure}
%\end{minipage}
\section{CONCLUSIONS}
In this paper we have considered  the works established in \cite{kolokoltsov2022qmfg,kolokoltsov2021law}. We show how this framework can be extended to the case of imperfect measurements. Also, we provide an alternative prove for the wellposedness of the mean field equation and we prove  a propagation  of chaos in a more general case where the solution to the mean field equation is not a rank one projector. We illustrate numerically this approximation is helpful to study a stabilizing control problem. In future work, we will prove rigorously such stabilization analysis. 
\paragraph{Acknowledgments.} This work is supported by the Agence Nationale de la Recherche projects Q-COAST ANR- 19-CE48-0003 and IGNITION ANR-21-CE47- 0015.


 
%\addtolength{\textheight}{-12cm}   % This command serves to balance the column lengths
                                  % on the last page of the document manually. It shortens
                                  % the textheight of the last page by a suitable amount.
                                  % This command does not take effect until the next page
                                  % so it should come on the page before the last. Make
                                  % sure that you do not shorten the textheight too much.

%%%%%%%%%%%%%%%%%%%%%%%%%%%%%%%%%%%%%%%%%%%%%%%%%%%%%%%%%%%%%%%%%%%%%%%%%%%%%%%%



%%%%%%%%%%%%%%%%%%%%%%%%%%%%%%%%%%%%%%%%%%%%%%%%%%%%%%%%%%%%%%%%%%%%%%%%%%%%%%%%



%%%%%%%%%%%%%%%%%%%%%%%%%%%%%%%%%%%%%%%%%%%%%%%%%%%%%%%%%%%%%%%%%%%%%%%%%%%%%%%%
%%%%%%%%%%%%%%%%%%%%%%%%%%%%%%%%%%%%
%%%%%%%%%%%%%%%%%%%%%%%%%%%%%%%%%%%%%%%%%%%%%%%%%%%%%%


%\bibliographystyle{unsrt}
\bibliographystyle{plain}  
\bibliography{ref}

\end{document}



%\begin{align}\label{eq1}\gamma_t^{\xi} &= \gamma_0 + \int_{0}^{t}-\iu[ \mathbf{H}_f + \uc_s\mathbf{\hat{H}} , \gamma_s^{\xi}]\mathrm{d}s\nonumber\\ &- \int_{0}^{t}\iu[\mathbf{A}^{\mathrm{\xi}_s},\gamma_s^{\xi}]\mathrm{d}s\nonumber\\  &+ \int_{0}^{t}\Big(L\gamma_t^{\xi}L^{\dag} - \frac{1}{2}\big\{L^{\dag}L,\gamma_s^{\xi}\big\}\Big)\mathrm{d}s\nonumber\\ &+ \sqrt{\eta}\int_{0}^{t}\Big(\gamma_s^{\xi}L^{\dag} + L\gamma_s^{\xi} - tr\big((L + L^{\dag})\gamma_s^{\xi}\big)\gamma_s^{\xi}\Big)\mathrm{d}W^{}_s\end{align}



%After taking these steps, if for optimal control $\uc$\subsubsection{Mean-field game problem}The Mean-field games problems is defined as follows :Find a pair $(\hat{\uc},\mu)$ such that denoting by $\hat{X}(\bullet) $ the solution of :\begin{align*}\mathrm{d}\hat{X}_t &= a\big(\hat{X}_t,\uc(\hat{X}_t),\mu_t\big)\mathrm{d}t + b(\hat{X}_t)\mathrm{d}W_t.\\\hat{X}_0 &= x_0.\end{align*}Then $\mu_{\bullet}$is the probability distribution of $\hat{X}_{\bullet}$ and $ \forall t \in [0,T]$.$$ \mathcal{J}(\hat{\uc}(\bullet),\mu_{\bullet}) \leq \mathcal{J}(\uc(\bullet),\mu_{\bullet}), \; \forall \uc(\bullet).$$
%\subsubsection{Mean-field type control problem}The Mean-field type control problem is defined as follows :For any feedback $\uc(\bullet)$, let $X_t = X^{\uc}_t$ be solution with $\mu_t^{\uc} = \mathcal{L}_{aw}(X^{\uc}_t)$. So becomes a McKean-Vlasov equation :\begin{align*}\mathrm{d}X^{\uc}_t &= a\big(X^{\uc}_t,\uc(X^{\uc}_t),\mu^{\uc}_t\big)\mathrm{d}t + b(X^{\uc}_t)\mathrm{d}W_t.\\X^{\uc}_0 &= x_0.\end{align*}Then we want to find $\hat{\uc}(\bullet)$ such that :$$ \mathcal{J}(\hat{\uc}(\bullet),\mu^{\hat{\uc}}_\bullet) \leq \mathcal{J}(\uc(\bullet),\mu^{\uc}_\bullet), \forall \uc(\bullet). $$So if we put $\hat{\mu}_{\bullet} = \mu^{\hat{\uc}}_{\bullet}$ and $X_t^{\hat{\uc}} = \hat{X}_t$, then we can say$\mu_{\bullet}$ is the probability distribution of $\hat{X}_{\bullet}, \; \forall t \in [0,T]$, and$$ \mathcal{J}(\hat{\uc}(\bullet),\hat{\mu}_\bullet) \leq \mathcal{J}(\uc(\bullet),\mu^{\uc}_\bullet), \; \forall \uc(\bullet).$$
%\begin{remark}\begin{multline*}-\frac{\partial S(t,x)}{\partial t}\Delta_xS(t,x) + H\big(x,\nabla_xS(t,x)\big) = f(x,\mu(t,x)).\\\frac{\partial \mu(t,x)}{\partial t} - \Delta_x\mu(t,x) - \text{div}\Big[\frac{\partial H}{\partial p}\big(x,\nabla_x S(t,x)\big)\mu(t,x)\Big] = 0\\\int_{\mathfrak{X}} \mu(t,\mathrm{d}x) = 1, \mu_t \geq 0.\; \;\mu(0,\bullet)= \mu_0(\bullet), \; \; S(T,\bullet) = G\big(\mu(T,\bullet),\bullet\big)\end{multline*}\end{remark}