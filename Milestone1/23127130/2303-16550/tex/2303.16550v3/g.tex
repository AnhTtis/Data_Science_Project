\paragraph{Dissipation parameter $g$}

The dissipation parameter $g$ depends on the forcing of turbulence. For decaying homogeneous turbulence with $b(t) = 0$, the decay of the kinetic energy of the flow can be described by a power law 
\begin{equation}
  \label{eq:u2}
u^2 = K \chi  \tilde t^{-6/5} 
\end{equation}
based on the Kolmogorov theory, where $K$ is the dimensionless Kolmogorov constant that depends on the dimension of the flow and $\chi$ is a constant with a dimension of $[L^2T^{-4/5}]$ \citep{Saffman67, Skrbek20}. 
The corresponding decay of the norm of the velocity field can be obtained from \Eq{eq:u2} as
\begin{equation}
  \label{eq:u-norm}
  \norm{\vec{u}}= \sqrt{K \chi} \tilde t^{-3/5},
\end{equation}
which leads to the following scaling of the dissipation parameter in terms of the flow velocity 
\begin{equation}
\label{eq:gu}
  g_u (\tilde t) = \frac{\norm{\vec{u}_{\rm in}}}{\norm{\vec{u}}} = \frac{\norm{\vec{u}_{\rm in}}}{\sqrt{K \chi}}  \tilde t^{3/5}. 
\end{equation}
At $\tilde t = \tilde T$, we insert \Eq{eq:evolutionT} into \Eq{eq:gu} and obtain
\begin{equation}
\label{eq:gu-T}
  g_u (T \left(\frac{L}{e_r}\right)) = \frac{\norm{\vec{u}_{\rm in}}}{\sqrt{K \chi}} \left(\frac{L}{e_r}\right)^{3/5} T^{3/5}. 
\end{equation}
For decaying turbulence, we note that $\underset{\tilde t\in[0,\tilde T]}{\max} \norm{\vec{u}(\tilde t)} = \norm{\vec{u}_{\rm in}}$ and a decaying Maxwell distribution function $f_m^{\rm eq}$ in \Eq{eq:feq} that eventually becomes stationary with $\rho \approx 1$ as $\underset{t\rightarrow \infty}{\lim} \vec{u}=0$,
\begin{equation}
\feq_m(T) \approx w_m a \, .
\label{eq:feqT}
\end{equation}
This leads to a stationary $f_m$ and as they relax to a stationary $f_m^{\rm eq}$ following \Eq{eq:lbm_origin} at a physically meaningful evolution time $T$. 
As a result, the dissipation parameter in terms of $g_f$ also reaches a stationary state,
\begin{equation}
\label{eq:gf}
g_f=\frac{\underset{t\in[0,T]}{\max} \norm{
f(t)}}{\norm{f(T)}} \le \frac{1}{\norm{w_m a}}.
\end{equation}

With the physical reasoning in hand, we now prove \Eq{eq:gf}.
We first upper bound $\underset{t\in[0,T]}{\max} \norm{
f(t)}$. For flow field with periodic boundary condition considered here, the variance of the density field can be expressed as \citep{Li_2020}
\begin{equation}
\label{eq:rhov}
\frac{\langle \rho^2 \rangle}{\langle \rho \rangle^2} = 1 + \sigma_\rho,
\end{equation}
where $\sigma_\rho$ is the standard deviation of the density field.
We recall that the flow field considered here is weakly impressible, i.e., $\rho\approx 1$ without losing generality, which leads to $\langle \rho \rangle^2\approx 1$ and $\sigma_\rho \approx 0$. Inserting them into \Eq{eq:rhov}, we obtain
\begin{equation}
\label{eq:rho-norm}
\langle \rho^2 \rangle = \norm{\rho}^2 \approx 1.
\end{equation}
Combine \Eq{eq:rho} and \Eq{eq:rho-norm} and observe the fact that $f_m > 0$, we get
\begin{equation}
\label{eq:fm-norm}
\norm{f_m(t)} \le 1.
\end{equation}
The mass and energy conservation during the collision ensures that
$|f_m (t) - \feq_m (t)| \propto u^2$ \citep{Chen1998LatticeFlows}. 
For weakly compressible flow, we observe $u^2 \ll 1$ and \Eq{eq:feqT}, and therefore, 
\begin{equation}
  \label{eq:nfeq}
|f_m (T)| \approx |\feq_m (T)| \approx w_m a
\end{equation}
for physically meaningful dissipation time scales $T$.
The conclusion of $g_f \approx O(1)$ (\Eq{eq:gf}) is reached by inserting \Eq{eq:fm-norm} and \Eq{eq:nfeq} 
into $\frac{\underset{t\in[0,T]}{\max} \norm{
f(t)}}{\norm{f(T)}}$ and considering the fact that $f(T) \approx \feq(T)$ at the stationary state of $f(T)$. 
\Eq{eq:gf} offers a mathematically upper bound of $g_f$. For weakly compressible decaying turbulence, \Eq{eq:nfeq} would hold for any simulation time $t$, i.e., $\underset{t\in[0,T]}{\max} \norm{
f(t)} \approx w_m a$ holds, which leads to a more realistic estimate of $g_f$ 
\begin{equation}
\label{eq:gf-phy}
g_f \approx 1. 
\end{equation}
\Eq{eq:gf-phy} is also confirmed by our numerical simulation of \Eq{eq:lbm_origin} (\Fig{fig:ts_f-norm-D2Q9}), which yields $g_f = 1 \pm 0.003$. 
Even though the macroscopic flow field $\vec{u}$ decays as $\norm{u}\propto \tilde t^{3/5}$ (\Eq{eq:gu}) following the Kolmogorov theory, the particle distribution function $f_m$ ($\phi$ for $n$-point LBE) relaxes to its stationary equilibrium state. This feature offers a more favourable $g_f \propto O(1)$ for the QLSA and is again due to the inherent nature of LBE. The only assumptions to arrive at \Eq{eq:gf} and \Eq{eq:gf-phy} are periodic boundary conditions (homogeneity), weakly compressible and decaying turbulence. \citet{Liu21} showed that the dissipation parameter $g$ decays exponentially for a homogeneous quadratic nonlinear (weak nonlinearity with $R < 1$) equation. The energy cascade of NSE and the geometric nature of its LBE form refute this exponential decay.   

To bound $\|\mathcal V (T)\|$, we first revisit the solution error.
Entries of $\mathcal V(T)$, $\mathcal V_j(T)$ as the truncated Taylor series of the solution of \Eq{eq:LBMs} at time $T$, $f(T)$, must satisfy
the bound,
\begin{equation}
  \label{eq:g-nn2}
    \|f^{\otimes j}(T)- \mathcal V_j(T)\|\leq \delta\|f^{\otimes j}(T)\|\,
\end{equation}
which leads to
\begin{align}
  \label{eq:g-nn3a}
  \norm{\mathcal V_j(T)} \ge (1-\delta)\norm{f^{\otimes j}(T)} \\
  \label{eq:g-nn3b}
  \norm{\mathcal V_j(T)} \le (1+\delta)\norm{f^{\otimes j}(T)}.  
\end{align}
Therefore, for all $t$, the following holds \citep{Krovi2023improvedquantum}
\begin{align}
  \|\mathcal V(t)\|^2
        &= \sum_{j=1}^k\|\mathcal V_j(t)\|^2
        \leq \sum_{j=1}^k(1+\delta)^2\|f^{\otimes j}(t)\|^2
        \le (1+\delta)^2 k \|f(t)\|^2                         \label{eq:parallel_inequality-a}
\\
  \|\mathcal V(t)\|^2 &= \sum_{j=1}^k\|\mathcal V_j(t)\|^2 \ge \sum_{j=1}^k (1-\delta)^2\|f^{\otimes j}(t)\|^2\ge (1-\delta)^2\|f(t)\|^{2}\,.
                        \label{eq:parallel_inequality-b}
\end{align}
Combining \Eq{eq:g} and (\ref{eq:parallel_inequality-a})--(\ref{eq:parallel_inequality-b}) yields
\begin{equation}
\label{eq:g-delta}
g \le \frac{1+\delta}{1-\delta}\sqrt{k}\frac{\underset{t\in[0,T]}{\max} \norm{
f(t)}}{\norm{f(T)}}.
\end{equation}
We insert \Eq{eq:gf-phy} to \Eq{eq:g-delta} and obtain,
\begin{equation}
\label{eq:g-delta2}
g \le \frac{1+\delta}{1-\delta}\sqrt{k} g_f \approx \frac{1+\delta}{1-\delta}\sqrt{k}
\end{equation}
with an error of $O(10^{-3})$.
Taking $\delta \le 1/3$ and an even larger-than-one $g_f=3/2$, \Eq{eq:g-delta2} can be reduced to
\begin{equation}
\label{eq:g-decay}
g \le 3\sqrt{k}.
\end{equation}

