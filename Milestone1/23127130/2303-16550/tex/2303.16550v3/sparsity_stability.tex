\paragraph{Sparsity}

The coefficient matrices $F^{(1)}$, $F^{(2)}$, and $F^{(3)}$ and resulting transfer matrices $A^i_j$
for a single point contain $O(Q^3\times Q^3)$ elements.  In the $n$-point
Carleman matrix, locality is enforced by \Eq{eq:F_i_alpha}, so that the
nonzero elements of the collision Carleman matrix (\Eq{eq:npt-coll}) are
diagonal in $\vec x$.

The streaming operator $S$ involves the $O(Q)$ nearest (or potentially
next-to-nearest, next-to-next-to-nearest, etc., if higher-order accuracy for the
gradient operator is desired) neighbors in space. This results in bands in $S$
that are not diagonal in $\vec x$, yielding an additional $O(Q^3\times Q^3)$ off-diagonal
elements in the streaming Carleman matrix, i.e., the first term in \Eq{eq:n_point_Carleman}.

The sparsity, i.e., the total number of nonzero elements in each row or column of
$\mathcal C$, is bounded by the sum of the sparsity of the collision Carleman
matrix and the sparsity of the streaming Carleman matrix, both of
which are $O(Q^3)$; thus,
\begin{equation}
  s=O(1),\label{eq:sparsity}
\end{equation}
i.e, sparsity is independent of the number of grid points $n$ and other free parameters of the
system for the fixed discretization schemes considered here.

\paragraph{Matrix norm}
We now determine the norm of the Carleman matrix $\mathcal C$ (\Eq{eq:n_point_Carleman}).
The matrix 2-norm of a matrix $M$ is bounded by its 1-norm and its $\infty$-norm: 
\begin{equation}
  \label{eq:specnorm}
  \|M\| \leq \sqrt{\|M\|_1 \|M\|_\infty}.
\end{equation}
The 1-norm and $\infty$-norm of $\mathcal C$ can both be directly bounded, since
they equal the maximum absolute column sum and maximum absolute row sum of the
matrix, respectively.  By \Eq{eq:sparsity}, there are $O(Q^3)$ nonzero
elements in each row.  From \Eq{eq:lbm-grouped}, these elements are each
$O((\kn\tau)^{-1})$, so that $\|\mathcal C\|_\infty = O(Q^3(\kn\tau)^{-1})$.  We
note that \Eq{eq:sparsity} also gives the number of nonzero elements in each
column due to the diagonality of $\mathcal C$ in $\vec x$, so that
$\|\mathcal C\|_1$ has the same scaling as $\|\mathcal C\|_\infty$.  Therefore,
the overall scaling of the matrix norm with the free parameters of the
system is
\begin{equation}
  \label{eq:normc}
  \|\mathcal C\| = O((\kn\tau)^{-1}).
\end{equation}
In practice, $\tau = O(1)$, because $\tau > 0.5$ is required for the stability of the
LBM \cite{Sterling1996StabilityMethods}.   Thus we have
\begin{equation}
    \|\mathcal{C} \| = O(\kn^{-1}).
\end{equation}

\paragraph{Stability}

Matrix stability refers to $\mathcal{R}(\lambda)\leq 0$ and is a necessary
condition for the \citet{Berry2017QuantumPrecision} algorithm. We now show that
$\mathcal C$ satisfies that condition if the underlying physical system is linearly
stable.

Because of its block triangular structure (in Kronecker degree, not $\vec x$),
the eigenvalue problem for $\mathcal C$ reduces to finding the eigenvalues of the block-diagonal
(in Kronecker degree) elements: the transfer matrices
\begin{align}
  \mathcal A_1^1 - \mathcal B_1^1  =& \mathcal F^{(1)} - S \\
  \mathcal A_2^2 - \mathcal B_2^2 =& \IdnQ\otimes (\mathcal F^{(1)} - S) + 
                                    (\mathcal F^{(1)} - S) \otimes\IdnQ \\
  \mathcal A_3^3 - \mathcal B_3^3 =& \IdnQ^{(2)}\otimes (\mathcal F^{(1)} - S)
                                    \nonumber \\
  &+ \IdnQ\otimes (\mathcal F^{(1)} - S)\otimes\IdnQ \nonumber \\
  &+ (\mathcal F^{(1)} - S)\otimes\IdnQ^{(2)}.
\end{align}

Denote the eigenvectors and eigenvalues of $\mathcal F^{(1)} - S$ as $v_i$ and $\mu_i$
($i\in 1,\dots, nQ$).  It is trivial to show that the eigenvectors and eigenvalues
of $(\mathcal F^{(1)} -  S) \otimes \IdnQ$ and $\IdnQ \otimes (\mathcal F^{(1)} - S)$ are simply
$v^{(2)} = v \otimes v$ and $\mu^{(2)} = \mu_i + \mu_j$
($i,j\in 1,\dots, nQ$); similarly for $(\mathcal F^{(1)} - S) \otimes \IdnQ^{(2)}$, etc.
Therefore, if all eigenvalues of $(\mathcal F^{(1)} - S)$ are nonpositive, the eigenvalues of
$\mathcal C$ will also be nonpositive.

The stability of $\mathcal C$ is thus determined by the stability of
$\mathcal F^{(1)} - S$. This is equivalent to a linear stability analysis on
the lattice Boltzmann system \cite{Sterling1996StabilityMethods}.
