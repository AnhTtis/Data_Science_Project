\subsubsection*{Contributions to algorithm complexity of \citet{Krovi2023improvedquantum}}
  
We now analyse the complexity of solving \Eq{eq:clbm} using the algorithm of \citet{Krovi2023improvedquantum}, which does not require the diagonalizability of matrices compared to the algorithm of \citet{Berry2017QuantumPrecision}. The algorithm complexity of \citet{Krovi2023improvedquantum} is restated here: 
\begin{equation}
\label{eq:ca_Krovi}
\text{query complexity}=O ( g T \Vert \mathcal C\Vert E(\mathcal C)
\text{poly}(s, \log k, \log (1 + \frac{T e^2 \Vert b \Vert}{\Vert \mathcal V_T \Vert}), \log (\frac{1}{\epsilon}), \log (T \Vert \mathcal C\Vert E(\mathcal C)) )),
\end{equation}
with the gate complexity no greater than a factor of
\begin{equation}
\label{eq:ca_Krovi_gate}
O (\poly \log(1 + \frac{T e^2 \Vert b \Vert}{\Vert \mathcal V_T \Vert}, 1/\epsilon, T \Vert \mathcal C \Vert) ),
\end{equation}
where   
\begin{equation}
  \label{eq:ec-def}
  E(\mathcal C) = \underset{t\in[0,T]}{\text{sup}} \Vert \exp(\mathcal C t) \Vert,
\end{equation}
is adopted to bound $E(\mathcal C)$ without additional assumptions on $\mathcal C$, e.g., normality or diagonalizability \citep{Krovi2023improvedquantum}. $e$ is the Euler constant.

\paragraph{Norm of the matrix exponential $E(\mathcal C)$}

We now bound $E(\mathcal C)$ following the method of
\citet{Krovi2023improvedquantum} after the stability of $\mathcal C$ is ensured
as discussed in the paragraph above. 
We start with the single-point collision-only Carleman matrix $C$ of \Eq{eq:Cdef} as the collision dominates LBM.  
Similar nomenclature of
\citet{Krovi2023improvedquantum} is adopted in the following discussion. The
following quantities
\begin{align}
  \label{eq:ln-nn1a}
    &\sigma(C)=\{\lambda\,|\, \lambda \text{ is an eigenvalue of } C\}&\text{Spectrum}\\
  \label{eq:ln-nn1b}
    &\alpha(C)=\max \{\Re(\lambda) \,|\, \lambda\in \sigma(C)\}&\text{Spectral abscissa} \\
  \label{eq:ln-nn1c}
    &\mu(C)=\max\{\lambda\,|\, \lambda\in \sigma((C+ C^\dag)/2)\}&\text{Log-norm}
\end{align}
and bounds \citep{Dahlquist63} 
\begin{equation}
  \label{eq:ln-nn2}
    \exp(\alpha t)\leq \|\exp(C t)\|\leq \exp(\mu t)\leq \exp(\|C \|t)\,.
\end{equation}
are defined to facilitate our discussions.

We first reduce the 3-order LBE, \Eq{eq:lbm-carleman}, to a quadratic equation,
\begin{equation}
  \label{eq:lbm-carleman-reduce}
  \ddt{\hat f} = -S \hat f + F^{(0)} +  \hat F^{(1)} \hat f +  \hat F^{(2)}  \hat f^{[2]},
\end{equation}
where $\hat f_i = f^{[i]}$ (not to be confused with the subscript of $f_m$), $\hat f =(\hat f_1, \hat f_2, \hat f_3)^T$, 
\begin{equation}
  \label{eq:hatF1}
\hat F^{(1)} 
 = \begin{pmatrix}
A_1^1 & A_2^1 \\ 
0 & A_2^2
\end{pmatrix},
\end{equation}
and
\begin{equation}
  \label{eq:hatF2}
\hat F^{(2)} 
 = \begin{pmatrix}
   0, & 0, A_3^1 & 0 \\ 
   0, & 0, A_3^2 & A_4^2 
\end{pmatrix} .
\end{equation}
We denote $\hat{F}^{(0)} = F^0$ for consistency and ease of the following discussion.

  Following the proof of \citet{Krovi2023improvedquantum}, we first split $C = H_0+H_1+H_2$, where
\begin{align}
  \label{eq:ln-nn3a}
    H_0 &= \sum_{j=2}^k \ket{j}\bra{j-1}\otimes A^j_{j-1}\\
  \label{eq:ln-nn3b}
    H_1 &= \sum_{j=1}^k \ket{j}\bra{j}\otimes A^j_j\\
  \label{eq:ln-nn3c}
    H_2 &= \sum_{j=2}^k \ket{j}\bra{j+1}\otimes A^j_{j+1}\,.
\end{align}
The exponential can be bounded as follows.
\begin{equation}
  \label{eq:ln-nn4}
    \|e^{Ct}\|\leq e^{\mu(C)t}\,.
\end{equation}
We have
\begin{equation}
  \label{eq:ln-nn5}
    \mu(C)=\sup_{ V:\| V\|=1}\bra{ V} C\ket{ V} = \sup_{V:\| V\|=1}\bra{ V}H_1\ket{ V} + \sup_{ V:\|V\|=1}\bra{ V}(H_0+H_2)\ket{ V}\,.
\end{equation}
This gives us
\begin{equation}
  \label{eq:ln-nn6}
    \mu(C)\leq \mu(H_1) + \mu(H_0)+\mu(H_2),.
\end{equation}
The Proposition 3.3 of \citet{Forets2017} states that 
for all $i\geq 1$, $0\leq j \leq k-1$, the estimate $\|A^i_{i+j\|} \leq i \|F_{j+1}\|$ holds. 
Therefore, 
we can write $\norm{H_{0, 1, 2}}=k\norm{\hat{F}_{0, 1, 2}}$ and 
\begin{equation}
  \label{eq:ln-nn7}
    \|e^{H_{0, 1, 2}}\|=\|e^{\hat{F}_{0, 1, 2}}\|^k\,.
\end{equation}
This gives us
\begin{equation}
  \label{eq:emu}
    \|e^{C}\|\leq e^{(\mu(\hat{F}_1)+\mu(\hat{F}_0)+\mu(\hat{F}_2))k}\,,
\end{equation}
In LBM, both $\hat{F}_1$ and $\hat{F}_2$ are constant matrices determined by the LBM constants for a given relaxation time $\tau$. Therefore, $\mu(\hat{F}_{1})$ and $\mu(\hat{F}_{2})$ are also constants and can be obtained explicitly. We show that $\mu(\hat{F}_1) \le 0$.
Since $\hat{F}_2$ in \Eq{eq:hatF2} is not square, we use the largest singuler value of $\hat{F}_2$ to bound $\mu(\hat{F}_2)$, which leads to $\mu(\hat{F}_2) = O(1)$ as shown in \Fig{fig:svd_norm_hat_F2_tau}.
For decaying turbulence ($b=0$), $\hat{F}_0 = 0$, therefore, $\mu(\hat{F}_0) = 0$. \Eq{eq:emu} can be bounded as 

\begin{equation}
  \label{eq:ec}
    E(C)\leq \exp{(\varrho t)}\,,
\end{equation}
where $\varrho =O(1)$.


\subsubsection*{Final quantum gate complexity to solve \Eq{eq:clbm}}
Finally, inserting \Eq{eq:epsilon}, \Eq{eq:evolutionT}, \Eq{eq:g-decay}, \Eq{eq:sparsity}, \Eq{eq:normc}, and \Eq{eq:ec},    into
\Eq{eq:ca_Krovi} yields the following result for the quantum algorithm complexity for the decaying turbulence 
\begin{equation}
  \label{eq:final-result-Krovi}
  \text{complexity} = O(t_c^{-1}\tilde{T}3\sqrt{k} \exp{(\varrho \tilde{T} / \left(\frac{L}{e_r}\right))}  \poly\log(n/\epsilon))
\end{equation}
where $\tilde T$ is the physical evolution time, and $L$ and
$e_r$ are characteristic scales chosen for the LBE formalism.
The scaling of each term in \Eq{eq:ca_Krovi} is summarized in
Table~\ref{tab:scaling_Krovi}. For any arbitrary simulation
time. The term $\tilde T/L$ in \Eq{eq:final-result-Krovi} can be reduced to
\begin{equation}
  \label{eq:T_L}
  \tilde T/L = 1/\norm{\vec{u}} = \frac{\tilde T^{3/5}}{\sqrt{K \chi}} .    
\end{equation}
The final complexity of applying \Eq{eq:ca_Krovi} to \Eq{eq:clbm} is
\begin{equation}
  \label{eq:final-result-Krovi2}
  \text{complexity} = O(t_c^{-1}\tilde{T}\sqrt{k}  \exp{(\varrho \frac{\tilde T^{3/5}}{e_r \sqrt{K \chi}})} \poly\log(n/\epsilon)),
\end{equation}
which yields an exponential growth in evolution time $\tilde T$. There could be a more strict bound such that $\norm{\mu (C)} \le 0$, which warrants further inquiry. Nevertheless, our analysis suggest urgent needs of domain driven QLSA development to achieve quantum speedup in solving nonlinear transport problems.
More importantly, our analysis sheds light on exploring the fundamental limit of applying quantum computing to nonlinear transport problems.       

\begin{table}
  \centering
  \begin{tabular}{lll}
    \hline\hline
    Term & Scaling & Reference \\ \hline
    $g$        & $3\sqrt{k}$                &  \Eq{eq:g-decay}        \\
    $s$        & $O(1)$              &  \Eq{eq:sparsity} \\
    $\|\mathcal C\|$    & $O(\kn^{-1})$ & \Eq{eq:normc}    \\
    $E(C)$        & $\exp{(\varrho t)}$                &  \Eq{eq:ec}        \\
    \hline\hline
  \end{tabular}
  \caption{Scaling of each term in \Eq{eq:ca} with lattice Boltzmann equation parameters leading to the
    result of \Eq{eq:final-result-Krovi}. 
    }
  \label{tab:scaling_Krovi}
\end{table}
\ifarXiv
\else
In the main text, \Eq{eq:complong}
is discussed in terms of the physical parameters of the turbulence
simulation problem.  
\fi



\begin{figure}[t!]\begin{center}
\includegraphics[width=\textwidth]{svd_norm_hat_F2_tau}
\end{center}
\caption{$\| \hat{F}_2 \|$ and maximum SVD of $\hat{F}_2$ for the D1Q3 LBM.}
\label{fig:svd_norm_hat_F2_tau}
\end{figure}


