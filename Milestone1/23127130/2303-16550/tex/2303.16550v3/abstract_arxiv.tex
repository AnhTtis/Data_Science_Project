\ifarXiv
Numerical
simulation of turbulent fluid dynamics
\else
Complex systems exhibiting multiscale fluid dynamics are ubiquitous in
  nature. Developing the ability to understand, predict, or modify these systems
  ranks among the ``grand challenges'' of science and engineering.
  The fluid dynamics of these systems admits no
  analytic solution, but numerical simulation
\fi
needs to either parameterize turbulence---which
introduces large uncertainties---or explicitly resolve the smallest
scales---which is prohibitively expensive.
Here we provide evidence through analytic bounds and numerical studies that a
potential quantum exponential speedup can be achieved to
simulate the Navier--Stokes equations governing turbulence using quantum computing. Specifically, we provide a
formulation of the lattice Boltzmann equation for which we give evidence that
low-order Carleman linearization is much more accurate than previously believed
for these systems and that for computationally interesting examples.  This is
achieved via a combination of reformulating the nonlinearity and accurately
linearizing the dynamical equations, effectively trading nonlinearity for
additional degrees of freedom that add negligible expense in the quantum solver.
Based on this we apply a quantum algorithm for simulating the Carleman-linearized
lattice Boltzmann equation and provide evidence that its cost scales
logarithmically with system size, compared to polynomial scaling
in the best known classical algorithms.  This work suggests that an exponential
quantum advantage may exist for simulating fluid dynamics,
paving the way for simulating nonlinear multiscale transport phenomena in a wide range of disciplines using quantum computing.  
