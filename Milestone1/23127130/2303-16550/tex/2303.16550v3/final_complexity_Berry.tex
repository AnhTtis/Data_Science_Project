\subsubsection*{Final quantum gate complexity to solve \Eq{eq:clbm}}
Finally, inserting  \Eq{eq:epsilon}, \Eq{eq:evolutionT}, \Eq{eq:g-decay},  \Eq{eq:sparsity}, \Eq{eq:normc}, and \Eq{eq:kj-value} 
into \Eq{eq:ca} yields the following result for the quantum algorithm complexity as quantified by the number of two-qubit gates needed in the simulation:
\begin{equation}
  \label{eq:complong}
  \text{gate complexity} = O(t_c^{-1}\tilde{T}\poly\log(n/\epsilon)),
\end{equation}
where $\tilde T$ is the physical evolution time.
The scaling of each term in \Eq{eq:ca} is summarized in Table~\ref{tab:scaling}.
\begin{table}
  \centering
  \begin{tabular}{lll}
    \hline\hline
    Term & Scaling & Reference \\ \hline
    $\beta$    & $O(1)$                &  \Eq{eq:beta}     \\
    $g$        & $3\sqrt{k}$                &  \Eq{eq:g-decay}        \\
    $s$        & $O(1)$              &  \Eq{eq:sparsity} \\
    $\|\mathcal C\|$    & $O(\kn^{-1})$ & \Eq{eq:normc}    \\
    $\kappa_J$ & $\le 1.152\times 10^5$ & \Eq{eq:kj-value}        \\
    \hline\hline
  \end{tabular}
  \caption{Scaling of each term in \Eq{eq:ca} with lattice Boltzmann equation parameters leading to the
    result of \Eq{eq:complong}. Note that the dissipation parameter $g$ is a constant because the Carleman linearized LBE converges at the truncation order $k=3$.  
    }
  \label{tab:scaling}
\end{table}
\ifarXiv
\else
In the main text, \Eq{eq:complong}
is discussed in terms of the physical parameters of the turbulence
simulation problem.  
\fi

The solution error of using the quantum
linear system algorithm (QLSA) of
\citet{Berry2017QuantumPrecision} to solve
\Eq{eq:LBMs} come from two sources: the
Carleman truncation error and the QLSA
error. Since the solution error from QLSA
contributes $\text{poly}(\log(\epsilon)$ to
the gate complexity expressed in \Eq{eq:ca},
we focus on bounding the Carleman truncation
error, which is $O(\text{Ma}^2)$. For weakly
compressible flow, $\text{Ma}\ll1$,
therefore, the Carleman truncation error is
very small. In \citet{Liu21}, a QLSA with
gate complexity scales with $T^2$ was used
due to their time dependent Carleman matrix.
A time dependent Carleman matrix requires
discretizing time using the forward Euler
method, which contributes to the solution
error. In contrast, our Carleman matrix
$\mathcal{C}$ is constant, which affords us
to use the QLSA of
\citet{Berry2017QuantumPrecision} with a
gate complexity scales with $T$
(\Eq{eq:ca}). Overall, our method of using
QLSA to solve \Eq{turb} has yields a gate
complexity scaling with $\poly(\log{N})$ and
$T$ for any arbitrary Reynolds number
comparing to \citet{Liu21}'s complexity
scaling with $T^2$ for $R<1$.   
