\paragraph{Norm of the matrix exponential $E(\mathcal C)$}

We now bound $E(\mathcal C)$ following the method of
\citet{Krovi2023improvedquantum} after the stability of $\mathcal C$ is ensured
as discussed in the paragraph above. 
We start with the single-point collision-only Carleman matrix $C$ of \Eq{eq:Cdef} as the collision dominates LBM.  
Similar nomenclature of
\citet{Krovi2023improvedquantum} is adopted in the following discussion. The
following quantities
\begin{align}
  \label{eq:ln-nn1a}
    &\sigma(C)=\{\lambda\,|\, \lambda \text{ is an eigenvalue of } C\}&\text{Spectrum}\\
  \label{eq:ln-nn1b}
    &\alpha(C)=\max \{\Re(\lambda) \,|\, \lambda\in \sigma(C)\}&\text{Spectral abscissa} \\
  \label{eq:ln-nn1c}
    &\mu(C)=\max\{\lambda\,|\, \lambda\in \sigma((C+ C^\dag)/2)\}&\text{Log-norm}
\end{align}
and bounds \citep{Dahlquist63} 
\begin{equation}
  \label{eq:ln-nn2}
    \exp(\alpha t)\leq \|\exp(C t)\|\leq \exp(\mu t)\leq \exp(\|C \|t)\,.
\end{equation}
are defined to facilitate our discussions.

We first reduce the 3-order LBE, \Eq{eq:lbm-carleman}, to a quadratic equation,
\begin{equation}
  \label{eq:lbm-carleman-reduce}
  \ddt{\hat f} = -S \hat f + F^{(0)} +  \hat F^{(1)} \hat f +  \hat F^{(2)}  \hat f^{[2]},
\end{equation}
where $\hat f_i = f^{[i]}$ (not to be confused with the subscript of $f_m$), $\hat f =(\hat f_1, \hat f_2, \hat f_3)^T$, 
\begin{equation}
  \label{eq:hatF1}
\hat F^{(1)} 
 = \begin{pmatrix}
A_1^1 & A_2^1 \\ 
0 & A_2^2
\end{pmatrix},
\end{equation}
and
\begin{equation}
  \label{eq:hatF2}
\hat F^{(2)} 
 = \begin{pmatrix}
   0, & 0, A_3^1 & 0 \\ 
   0, & 0, A_3^2 & A_4^2 
\end{pmatrix} .
\end{equation}
We denote $\hat{F}^{(0)} = F^0$ for consistency and ease of the following discussion.

  Following the proof of \citet{Krovi2023improvedquantum}, we first split $C = H_0+H_1+H_2$, where
\begin{align}
  \label{eq:ln-nn3a}
    H_0 &= \sum_{j=2}^k \ket{j}\bra{j-1}\otimes A^j_{j-1}\\
  \label{eq:ln-nn3b}
    H_1 &= \sum_{j=1}^k \ket{j}\bra{j}\otimes A^j_j\\
  \label{eq:ln-nn3c}
    H_2 &= \sum_{j=2}^k \ket{j}\bra{j+1}\otimes A^j_{j+1}\,.
\end{align}
The exponential can be bounded as follows.
\begin{equation}
  \label{eq:ln-nn4}
    \|e^{Ct}\|\leq e^{\mu(C)t}\,.
\end{equation}
We have
\begin{equation}
  \label{eq:ln-nn5}
    \mu(C)=\sup_{ V:\| V\|=1}\bra{ V} C\ket{ V} = \sup_{V:\| V\|=1}\bra{ V}H_1\ket{ V} + \sup_{ V:\|V\|=1}\bra{ V}(H_0+H_2)\ket{ V}\,.
\end{equation}
This gives us
\begin{equation}
  \label{eq:ln-nn6}
    \mu(C)\leq \mu(H_1) + \mu(H_0)+\mu(H_2),.
\end{equation}
The Proposition 3.3 of \citet{Forets2017} states that 
for all $i\geq 1$, $0\leq j \leq k-1$, the estimate $\|A^i_{i+j\|} \leq i \|F_{j+1}\|$ holds. 
Therefore, 
we can write $\norm{H_{0, 1, 2}}=k\norm{\hat{F}_{0, 1, 2}}$ and 
\begin{equation}
  \label{eq:ln-nn7}
    \|e^{H_{0, 1, 2}}\|=\|e^{\hat{F}_{0, 1, 2}}\|^k\,.
\end{equation}
This gives us
\begin{equation}
  \label{eq:emu}
    \|e^{C}\|\leq e^{(\mu(\hat{F}_1)+\mu(\hat{F}_0)+\mu(\hat{F}_2))k}\,,
\end{equation}
In LBM, both $\hat{F}_1$ and $\hat{F}_2$ are constant matrices determined by the LBM constants for a given relaxation time $\tau$. Therefore, $\mu(\hat{F}_{1})$ and $\mu(\hat{F}_{2})$ are also constants and can be obtained explicitly. We show that $\mu(\hat{F}_1) \le 0$.
Since $\hat{F}_2$ in \Eq{eq:hatF2} is not square, we use the largest singuler value of $\hat{F}_2$ to bound $\mu(\hat{F}_2)$, which leads to $\mu(\hat{F}_2) = O(1)$ as shown in \Fig{fig:svd_norm_hat_F2_tau}.
For decaying turbulence ($b=0$), $\hat{F}_0 = 0$, therefore, $\mu(\hat{F}_0) = 0$. \Eq{eq:emu} can be bounded as 

\begin{equation}
  \label{eq:ec}
    E(C)\leq \exp{(\varrho t)}\,,
\end{equation}
where $\varrho =O(1)$.
