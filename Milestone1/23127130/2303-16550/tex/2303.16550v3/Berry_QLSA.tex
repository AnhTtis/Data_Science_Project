\subsubsection*{Contributions to algorithm complexity of \citet{Berry2017QuantumPrecision}}
  
The complexity of solving \Eq{eq:clbm} is given by \Eq{eq:ca} in terms
of the properties of the coefficient matrix $\mathcal C$ and the solution vector
$\mathcal V$; for ease of reference, the equation is
reproduced here:
\begin{equation}
\tag{\ref{eq:ca}}
\text{gate complexity}=O(\Vert \mathcal C\Vert \kappa_\mathcal{J} g T s\cdot
\text{poly}(\log{(\kappa_\mathcal{J} s g \beta T \Vert \mathcal C \Vert N/\epsilon)})),
\end{equation}
where $\mathcal C=\mathcal J \mathcal D \mathcal J^{-1}$ is an $N\times N$ diagonalizable matrix with
eigenvalues $\mathcal D=\text{diag}(\lambda_i)$ ($i \in 1,\dots, N$) whose real part
$\mathcal{R}(\lambda_i)\leq 0\ \forall i$. The norm $\|\mathcal C\|$ is the
2-norm, the largest singular value of $\mathcal C$.  
The initial condition norm enters through
\begin{equation}
\label{eq:beta}
\beta = (\| \vert{\mathcal V}_\text{in}\rangle \| + T \norm{\ket{b}}) / \| \vert{\mathcal V}(T)\rangle \| = O(1), 
\end{equation}
which is dependent on the physical problem under consideration but independent
of other system parameters.
Here we consider the gate complexity to be the number of one and two-qubit gates needed for the simulation.  
The condition number in the above expression,
\begin{equation}
    \kappa_\mathcal{J} = \|\mathcal{J}\|\cdot\|\mathcal{J}^{-1}\|,
\end{equation}
is the condition number of the matrix $\mathcal{J}$ of
eigenvectors of $\mathcal C$.
The dissipation parameter $g$ is defined as
\begin{equation}
\label{eq:g}
g = \underset{t\in[0,T]}{\max} \Vert
\mathcal V(t)\Vert/\Vert \mathcal V(T)\Vert,
\end{equation}
and $s$ is the sparsity (number of nonzero
entries per row) of $\mathcal C$. The number of degrees of freedom $N$, which is the dimension of $\mathcal C$ and
$\mathcal V$, is given in terms of the number of grid points $n$ and discrete
velocities $Q$ by
\begin{equation}
  \label{eq:ndf}
  N = (n^3Q^3 +  n^2Q^2 + nQ).
\end{equation}
The term-by-term analysis of the factors that determine the algorithm’s complexity (\Eq{eq:ca}) is
conducted below.
\paragraph{Evolution time}

The physical evolution time $\tilde T$ is imposed by the nature of the system
being simulated.  Physical and lattice evolution time are related by the
\citet{Sterling1996StabilityMethods} factor (\Eq{eq:scale-t}),
\begin{equation}
  \label{eq:evolutionT}
  T=\tilde T/\left(\frac{L}{e_r}\right).
\end{equation}


\paragraph{Dissipation parameter $g$}

The dissipation parameter $g$ depends on the forcing of turbulence. For decaying homogeneous turbulence with $b(t) = 0$, the decay of the kinetic energy of the flow can be described by a power law 
\begin{equation}
  \label{eq:u2}
u^2 = K \chi  \tilde t^{-6/5} 
\end{equation}
based on the Kolmogorov theory, where $K$ is the dimensionless Kolmogorov constant that depends on the dimension of the flow and $\chi$ is a constant with a dimension of $[L^2T^{-4/5}]$ \citep{Saffman67, Skrbek20}. 
The corresponding decay of the norm of the velocity field can be obtained from \Eq{eq:u2} as
\begin{equation}
  \label{eq:u-norm}
  \norm{\vec{u}}= \sqrt{K \chi} \tilde t^{-3/5},
\end{equation}
which leads to the following scaling of the dissipation parameter in terms of the flow velocity 
\begin{equation}
\label{eq:gu}
  g_u (\tilde t) = \frac{\norm{\vec{u}_{\rm in}}}{\norm{\vec{u}}} = \frac{\norm{\vec{u}_{\rm in}}}{\sqrt{K \chi}}  \tilde t^{3/5}. 
\end{equation}
At $\tilde t = \tilde T$, we insert \Eq{eq:evolutionT} into \Eq{eq:gu} and obtain
\begin{equation}
\label{eq:gu-T}
  g_u (T \left(\frac{L}{e_r}\right)) = \frac{\norm{\vec{u}_{\rm in}}}{\sqrt{K \chi}} \left(\frac{L}{e_r}\right)^{3/5} T^{3/5}. 
\end{equation}
For decaying turbulence, we note that $\underset{\tilde t\in[0,\tilde T]}{\max} \norm{\vec{u}(\tilde t)} = \norm{\vec{u}_{\rm in}}$ and a decaying Maxwell distribution function $f_m^{\rm eq}$ in \Eq{eq:feq} that eventually becomes stationary with $\rho \approx 1$ as $\underset{t\rightarrow \infty}{\lim} \vec{u}=0$,
\begin{equation}
\feq_m(T) \approx w_m a \, .
\label{eq:feqT}
\end{equation}
This leads to a stationary $f_m$ and as they relax to a stationary $f_m^{\rm eq}$ following \Eq{eq:lbm_origin} at a physically meaningful evolution time $T$. 
As a result, the dissipation parameter in terms of $g_f$ also reaches a stationary state,
\begin{equation}
\label{eq:gf}
g_f=\frac{\underset{t\in[0,T]}{\max} \norm{
f(t)}}{\norm{f(T)}} \le \frac{1}{\norm{w_m a}}.
\end{equation}

With the physical reasoning in hand, we now prove \Eq{eq:gf}.
We first upper bound $\underset{t\in[0,T]}{\max} \norm{
f(t)}$. For flow field with periodic boundary condition considered here, the variance of the density field can be expressed as \citep{Li_2020}
\begin{equation}
\label{eq:rhov}
\frac{\langle \rho^2 \rangle}{\langle \rho \rangle^2} = 1 + \sigma_\rho,
\end{equation}
where $\sigma_\rho$ is the standard deviation of the density field.
We recall that the flow field considered here is weakly impressible, i.e., $\rho\approx 1$ without losing generality, which leads to $\langle \rho \rangle^2\approx 1$ and $\sigma_\rho \approx 0$. Inserting them into \Eq{eq:rhov}, we obtain
\begin{equation}
\label{eq:rho-norm}
\langle \rho^2 \rangle = \norm{\rho}^2 \approx 1.
\end{equation}
Combine \Eq{eq:rho} and \Eq{eq:rho-norm} and observe the fact that $f_m > 0$, we get
\begin{equation}
\label{eq:fm-norm}
\norm{f_m(t)} \le 1.
\end{equation}
The mass and energy conservation during the collision ensures that
$|f_m (t) - \feq_m (t)| \propto u^2$ \citep{Chen1998LatticeFlows}. 
For weakly compressible flow, we observe $u^2 \ll 1$ and \Eq{eq:feqT}, and therefore, 
\begin{equation}
  \label{eq:nfeq}
|f_m (T)| \approx |\feq_m (T)| \approx w_m a
\end{equation}
for physically meaningful dissipation time scales $T$.
The conclusion of $g_f \approx O(1)$ (\Eq{eq:gf}) is reached by inserting \Eq{eq:fm-norm} and \Eq{eq:nfeq} 
into $\frac{\underset{t\in[0,T]}{\max} \norm{
f(t)}}{\norm{f(T)}}$ and considering the fact that $f(T) \approx \feq(T)$ at the stationary state of $f(T)$. 
\Eq{eq:gf} offers a mathematically upper bound of $g_f$. For weakly compressible decaying turbulence, \Eq{eq:nfeq} would hold for any simulation time $t$, i.e., $\underset{t\in[0,T]}{\max} \norm{
f(t)} \approx w_m a$ holds, which leads to a more realistic estimate of $g_f$ 
\begin{equation}
\label{eq:gf-phy}
g_f \approx 1. 
\end{equation}
\Eq{eq:gf-phy} is also confirmed by our numerical simulation of \Eq{eq:lbm_origin} (\Fig{fig:ts_f-norm-D2Q9}), which yields $g_f = 1 \pm 0.003$. 
Even though the macroscopic flow field $\vec{u}$ decays as $\norm{u}\propto \tilde t^{3/5}$ (\Eq{eq:gu}) following the Kolmogorov theory, the particle distribution function $f_m$ ($\phi$ for $n$-point LBE) relaxes to its stationary equilibrium state. This feature offers a more favourable $g_f \propto O(1)$ for the QLSA and is again due to the inherent nature of LBE. The only assumptions to arrive at \Eq{eq:gf} and \Eq{eq:gf-phy} are periodic boundary conditions (homogeneity), weakly compressible and decaying turbulence. \citet{Liu21} showed that the dissipation parameter $g$ decays exponentially for a homogeneous quadratic nonlinear (weak nonlinearity with $R < 1$) equation. The energy cascade of NSE and the geometric nature of its LBE form refute this exponential decay.   

To bound $\|\mathcal V (T)\|$, we first revisit the solution error.
Entries of $\mathcal V(T)$, $\mathcal V_j(T)$ as the truncated Taylor series of the solution of \Eq{eq:LBMs} at time $T$, $f(T)$, must satisfy
the bound,
\begin{equation}
  \label{eq:g-nn2}
    \|f^{\otimes j}(T)- \mathcal V_j(T)\|\leq \delta\|f^{\otimes j}(T)\|\,
\end{equation}
which leads to
\begin{align}
  \label{eq:g-nn3a}
  \norm{\mathcal V_j(T)} \ge (1-\delta)\norm{f^{\otimes j}(T)} \\
  \label{eq:g-nn3b}
  \norm{\mathcal V_j(T)} \le (1+\delta)\norm{f^{\otimes j}(T)}.  
\end{align}
Therefore, for all $t$, the following holds \citep{Krovi2023improvedquantum}
\begin{align}
  \|\mathcal V(t)\|^2
        &= \sum_{j=1}^k\|\mathcal V_j(t)\|^2
        \leq \sum_{j=1}^k(1+\delta)^2\|f^{\otimes j}(t)\|^2
        \le (1+\delta)^2 k \|f(t)\|^2                         \label{eq:parallel_inequality-a}
\\
  \|\mathcal V(t)\|^2 &= \sum_{j=1}^k\|\mathcal V_j(t)\|^2 \ge \sum_{j=1}^k (1-\delta)^2\|f^{\otimes j}(t)\|^2\ge (1-\delta)^2\|f(t)\|^{2}\,.
                        \label{eq:parallel_inequality-b}
\end{align}
Combining \Eq{eq:g} and (\ref{eq:parallel_inequality-a})--(\ref{eq:parallel_inequality-b}) yields
\begin{equation}
\label{eq:g-delta}
g \le \frac{1+\delta}{1-\delta}\sqrt{k}\frac{\underset{t\in[0,T]}{\max} \norm{
f(t)}}{\norm{f(T)}}.
\end{equation}
We insert \Eq{eq:gf-phy} to \Eq{eq:g-delta} and obtain,
\begin{equation}
\label{eq:g-delta2}
g \le \frac{1+\delta}{1-\delta}\sqrt{k} g_f \approx \frac{1+\delta}{1-\delta}\sqrt{k}
\end{equation}
with an error of $O(10^{-3})$.
Taking $\delta \le 1/3$ and an even larger-than-one $g_f=3/2$, \Eq{eq:g-delta2} can be reduced to
\begin{equation}
\label{eq:g-decay}
g \le 3\sqrt{k}.
\end{equation}


\paragraph{Sparsity}

The coefficient matrices $F^{(1)}$, $F^{(2)}$, and $F^{(3)}$ and resulting transfer matrices $A^i_j$
for a single point contain $O(Q^3\times Q^3)$ elements.  In the $n$-point
Carleman matrix, locality is enforced by \Eq{eq:F_i_alpha}, so that the
nonzero elements of the collision Carleman matrix (\Eq{eq:npt-coll}) are
diagonal in $\vec x$.

The streaming operator $S$ involves the $O(Q)$ nearest (or potentially
next-to-nearest, next-to-next-to-nearest, etc., if higher-order accuracy for the
gradient operator is desired) neighbors in space. This results in bands in $S$
that are not diagonal in $\vec x$, yielding an additional $O(Q^3\times Q^3)$ off-diagonal
elements in the streaming Carleman matrix, i.e., the first term in \Eq{eq:n_point_Carleman}.

The sparsity, i.e., the total number of nonzero elements in each row or column of
$\mathcal C$, is bounded by the sum of the sparsity of the collision Carleman
matrix and the sparsity of the streaming Carleman matrix, both of
which are $O(Q^3)$; thus,
\begin{equation}
  s=O(1),\label{eq:sparsity}
\end{equation}
i.e, sparsity is independent of the number of grid points $n$ and other free parameters of the
system for the fixed discretization schemes considered here.

\paragraph{Matrix norm}
We now determine the norm of the Carleman matrix $\mathcal C$ (\Eq{eq:n_point_Carleman}).
The matrix 2-norm of a matrix $M$ is bounded by its 1-norm and its $\infty$-norm: 
\begin{equation}
  \label{eq:specnorm}
  \|M\| \leq \sqrt{\|M\|_1 \|M\|_\infty}.
\end{equation}
The 1-norm and $\infty$-norm of $\mathcal C$ can both be directly bounded, since
they equal the maximum absolute column sum and maximum absolute row sum of the
matrix, respectively.  By \Eq{eq:sparsity}, there are $O(Q^3)$ nonzero
elements in each row.  From \Eq{eq:lbm-grouped}, these elements are each
$O((\kn\tau)^{-1})$, so that $\|\mathcal C\|_\infty = O(Q^3(\kn\tau)^{-1})$.  We
note that \Eq{eq:sparsity} also gives the number of nonzero elements in each
column due to the diagonality of $\mathcal C$ in $\vec x$, so that
$\|\mathcal C\|_1$ has the same scaling as $\|\mathcal C\|_\infty$.  Therefore,
the overall scaling of the matrix norm with the free parameters of the
system is
\begin{equation}
  \label{eq:normc}
  \|\mathcal C\| = O((\kn\tau)^{-1}).
\end{equation}
In practice, $\tau = O(1)$, because $\tau > 0.5$ is required for the stability of the
LBM \cite{Sterling1996StabilityMethods}.   Thus we have
\begin{equation}
    \|\mathcal{C} \| = O(\kn^{-1}).
\end{equation}

\paragraph{Stability}

Matrix stability refers to $\mathcal{R}(\lambda)\leq 0$ and is a necessary
condition for the \citet{Berry2017QuantumPrecision} algorithm. We now show that
$\mathcal C$ satisfies that condition if the underlying physical system is linearly
stable.

Because of its block triangular structure (in Kronecker degree, not $\vec x$),
the eigenvalue problem for $\mathcal C$ reduces to finding the eigenvalues of the block-diagonal
(in Kronecker degree) elements: the transfer matrices
\begin{align}
  \mathcal A_1^1 - \mathcal B_1^1  =& \mathcal F^{(1)} - S \\
  \mathcal A_2^2 - \mathcal B_2^2 =& \IdnQ\otimes (\mathcal F^{(1)} - S) + 
                                    (\mathcal F^{(1)} - S) \otimes\IdnQ \\
  \mathcal A_3^3 - \mathcal B_3^3 =& \IdnQ^{(2)}\otimes (\mathcal F^{(1)} - S)
                                    \nonumber \\
  &+ \IdnQ\otimes (\mathcal F^{(1)} - S)\otimes\IdnQ \nonumber \\
  &+ (\mathcal F^{(1)} - S)\otimes\IdnQ^{(2)}.
\end{align}

Denote the eigenvectors and eigenvalues of $\mathcal F^{(1)} - S$ as $v_i$ and $\mu_i$
($i\in 1,\dots, nQ$).  It is trivial to show that the eigenvectors and eigenvalues
of $(\mathcal F^{(1)} -  S) \otimes \IdnQ$ and $\IdnQ \otimes (\mathcal F^{(1)} - S)$ are simply
$v^{(2)} = v \otimes v$ and $\mu^{(2)} = \mu_i + \mu_j$
($i,j\in 1,\dots, nQ$); similarly for $(\mathcal F^{(1)} - S) \otimes \IdnQ^{(2)}$, etc.
Therefore, if all eigenvalues of $(\mathcal F^{(1)} - S)$ are nonpositive, the eigenvalues of
$\mathcal C$ will also be nonpositive.

The stability of $\mathcal C$ is thus determined by the stability of
$\mathcal F^{(1)} - S$. This is equivalent to a linear stability analysis on
the lattice Boltzmann system \cite{Sterling1996StabilityMethods}.


\paragraph{Condition number}

To bound  $\kappa_{\mathcal{J}}$, we need to find the eigenvectors of $\mathcal C$.  This is
not possible analytically for arbitrary $n$; if an analytic expression for the
eigenvectors could be obtained, a quantum algorithm, or any numerical
simulation, would no longer be required to calculate the dynamics of the system
described by $\mathcal C$.  We will derive a bound on $\kappa_\mathcal{J}$ in three steps.
First, we will numerically diagonalize the single-point collision Carleman
matrix $C$ at Carleman order $k=3$ (\Eq{eq:Cdef}) for lattice structures used
to recover the NSE from the LBE; these are specific matrices with
fixed, explicitly known $a$, $b$, $c$, $d$, $w_m$, and $Q$ parameters
in \Eq{eq:lbm-grouped}, but they can solve
general flows \cite{Chen1998LatticeFlows}.  Second, we will show that the
eigenvectors of the $n$-point collision Carleman matrix $\mathcal C_c(\vec x)$,
defined in \Eq{eq:npt-coll}, can be constructed from the eigenvectors of the
single-point $C$.  Third, we will perform a perturbative expansion to include
the effect of the streaming Carleman matrix $\mathcal C_s(\vec x)$, defined in
\Eq{eq:npt-stream}, and derive an approximate expression for $\kappa_\mathcal{J}$ that
holds as long as the Knudsen number of the flow is small; this condition needs
to be satisfied for the LBE to describe the flow.  The numbered paragraphs below
provide the details of these three steps.

\begin{enumerate}
\item In one, two, and three spatial dimensions, the D1Q3, D2Q9, and D3Q27
  lattice topologies are frequently used and are known to recover the NSE in the
  $\ma\ll 1$ limit \cite{Chen1998LatticeFlows} on which the present work
  focuses.  The single-point collision Carleman matrices $C$ and their
  spectral decompositions $C=JDJ^{-1}$ are provided as a data
  supplement \cite{url}. 
  By inspection of these numerical results,
  all eigenvalues of $C$ are 
  $-(\kn\tau)^{-1} \times \{0, 1, 2, 3\}$, that is, they are either degenerate or separated by gaps of
  $(\kn\tau)^{-1}$, as shown by the histogram of the eigenvalue spectra
  $\mathcal{R}(\lambda)$ in Fig.~\ref{fig:spectral}.  The eigenvector matrices
  $J$ and $J^{-1}$  are similarly constant for each choice of $Q\in\{3,9,27\}$.
  The corresponding condition numbers are $\kappa_J = 12.94$ for D1Q3, $\kappa_J
  = 85.66$ for D2Q9, and $\kappa_J = 2269$ for D3Q27.  Bounds based on
  \Eq{eq:specnorm} are $\kappa_J \le 62.41$ for D1Q3, $\kappa_J
  \le 2252$ for D2Q9, and $\kappa_J \le 1.152\times 10^5$ for D3Q27; these bounds will be used
  to generalize to $n$ grid points below.
\item Denote the eigenvectors of the single-point $C$ as $\xi_i$, $i\in
  1,\dots,Q$, and decompose
  \begin{equation}
    \xi_i = (\xi_i^{(1)}, \xi_i^{(2)}, \xi_i^{(3)})^\trans
  \end{equation}
  according to whether its elements represent 1-, 2-, or 3-form elements (powers of $f_m$ polynomials defined in \Eq{eq:lbm-carleman}) of the
  single-point Carleman vectors $V$.  Using the $\delta_\alpha$ vector from
  \Eq{eq:delta-alpha}, we can construct localized eigenvectors $\Xi_i(\vec
  x_\alpha)$ of the $n$-point collision Carleman matrix (\Eq{eq:npt-coll}):
  \begin{equation}
    \label{eq:Xi}
    \Xi_i(\vec x_\alpha) =
    (\delta_\alpha \otimes \xi_i^{(1)},
    \delta_\alpha^{[2]} \otimes \xi_i^{(2)},
    \delta_\alpha^{[3]} \otimes \xi_i^{(3)})^\trans.
  \end{equation}
  By \Eq{eq:diagonality-A}, $\Xi_i(\vec x_\alpha)$ is an eigenvector of the
  $n$-point collision matrix $\mathcal C_c^{(3)}(\vec x)$ for any
  $\alpha\in 1,\dots,n$, as the diagonality of the $n$-point transfer matrices
  in $\vec x$ ensures that the nonzero blocks of $\mathcal C_c$ are aligned with
  the nonzero blocks of $\Xi_i(\vec x_\alpha)$.  The eigenvectors of the
  $n$-point collision Carleman matrix $\mathcal C_c$ are therefore a simple
  blockwise repetition of the eigenvectors of the single-point $C$.
  Furthermore, it follows from \Eq{eq:Xi} that
  \begin{equation}
    \label{eq:Xi-ortho}
    \Xi_i(\vec x_\alpha)^\trans \Xi_j(\vec x_\beta) = \xi_i^\trans \xi_j \delta_{\alpha\beta},
  \end{equation}
  i.e., the eigenvectors at different grid points are mutually orthogonal.  The
  corresponding eigenvalues are equal for all $\alpha \in 1,\dots,n$, i.e., they
  are repetitions of the eigenvalues of $C$.

  The $\Xi_i(\vec x_\alpha)$ are stacked column-wise or row-wise to
  form the eigenvector matrices of $\mathcal C_c$.  Denote these eigenvector
  matrices as $\mathcal J_0$ and $\mathcal J_0^{-1}$.  (The subscript 0 notation
  is used in anticipation of the perturbation analysis carried out in the next
  paragraph.)  By \Eq{eq:Xi-ortho}, the eigenvectors localized at different grid points
  form blocks in $\mathcal J_0$ and $\mathcal J_0^{-1}$ that ensure that row and
  column sums on $\mathcal J_0$ and $\mathcal J_0^{-1}$ consist of the
  same $O(Q^3)$
  nonzero elements as row and column sums on the \textit{single-point constant}
  eigenvector matrices $J$ and $J^{-1}$, rather than the full
  $O(n^3Q^3)$ dimension of the $n$-point 
  matrices.  Thus, the same spectral-norm bound from
  \Eq{eq:specnorm} applies to $\|J\|$ and $\|\mathcal J_0\|$; and
  the same bound applies to $\|J^{-1}\|$ and $\|\mathcal J_0^{-1}\|$.
  Thus, $\kappa_{\mathcal{J}_0}$ is fixed (for fixed $Q$), independent
  of $n$.
  
\item Finally, we bound the effect of the streaming contribution $\mathcal C_s$
  to the eigenvectors of $\mathcal C=\mathcal C_s + \mathcal C_c$ and thence the
  effect on $\mathcal J$, $\mathcal J^{-1}$, and $\kappa_{\mathcal J}$.
  Inspection of \Eq{eq:lbm-grouped} shows that the streaming term in the LBE
  is suppressed by $\kn\ll 1$ relative to the collision term; this suggests that
  we can treat streaming using well established principles of (degenerate)
  perturbation theory \cite{Sakurai1994, bamieh2020tutorial}, with the collision Carleman matrix in
  \Eq{eq:n_point_Carleman} constituting the base matrix (with known
  eigenvectors) and the streaming term constituting the $O(\kn)\ll 1$
  perturbation.

  Item 2.\ above derived the properties of the $n$-point collision eigenvector
  matrices $\mathcal J_0$ and $\mathcal J_0^{-1}$ in terms of the single-point
  eigenvector matrices $J$ and $J^{-1}$ that had to be computed numerically.
  When the perturbation due to $\mathcal C_s$ is included, the eigenvectors are
  perturbed \cite{Sakurai1994} in proportion to the perturbation strength
  ($\kn\tau$) and the gap size of the eigenvalue spectrum of $\mathcal C_c$,
  neglecting until the next paragraph the issue of degenerate eigenvalues.  For
  notational convenience, we suppress the $\vec x_a$ notation on the
  eigenvectors of $\mathcal C_c$; the perturbed eigenvectors to first order are
  then given by $\Xi_i + \Xi_i'$, where
  \begin{equation}
    \label{eq:pertnondeg}
    \Xi_i' =  \kn\tau \sum_{k\neq i} \frac{\Xi_i^{-1} \mathcal C_s \Xi_k}{\lambda_i -
      \lambda_k} \Xi_k.
  \end{equation}
  Numerical calculation of the eigenvalue spectrum of the single-point Carleman
  matrices in item 1.\ above showed the gap between nondegenerate eigenvalues
  to be integer multiples of $(\kn\tau)^{-1} \gg 1$, so the perturbation to the eigenvectors is
  proportional to $O((\kn\tau)^{2}) \ll 1$ times the expectation values $\sum\Xi_i^{-1}
  \mathcal C_s \Xi_k$.  In the nondegenerate case, the eigenvectors
  form an orthonormal basis, and the vector norm of the $\Xi_i'$ can be bounded
  as follows:
  \begin{align}
    \vert \Xi_i'\vert^2 &= (\kn\tau)^2 \sum\limits_{k\neq i}
    \frac{\Xi_i^{-1}\mathcal C_s \Xi_k
    \Xi_k^{-1}\mathcal C_s^\dagger \Xi_i}{(\lambda_k -
                          \lambda_i)^2} \nonumber \\
    &\leq \frac{(\kn\tau)^2}{\min_k (\lambda_k - \lambda_i)^2} \sum\limits_{k\neq i}
    \Xi_i^{-1}\mathcal C_s \Xi_k \Xi_k^{-1}\mathcal C_s^\dagger
      \Xi_i \nonumber \\
    \label{eq:temp1}
    &\leq \frac{(\kn\tau)^2}{\min_k (\lambda_k - \lambda_i)^2} \left[\sum\limits_k
      \Xi_i^{-1}\mathcal C_s \Xi_k \Xi_k^{-1}\mathcal C_s^\dagger \Xi_i
    - \left\vert \Xi_i^{-1} \mathcal C_s \Xi_i\right\vert^2\right].
  \end{align}
  Using completeness and noting that the final term in \Eq{eq:temp1} is
  nonpositive-definite,
  \begin{equation}
    \vert \Xi_i'\vert^2 \leq \frac{(\kn\tau)^2}{\min_k (\lambda_k -
    \lambda_i)^2} \Xi_i^{-1} \mathcal C_s \mathcal
    C_s^\dagger \Xi_i,
  \end{equation}
  and therefore
  \begin{equation}
    \vert \Xi_i'\vert \leq \frac{\kn\tau}{\min_k (\lambda_k -
      \lambda_i)} \|\mathcal C_s \|.
  \end{equation}
  The eigenvalues of the streaming operator correspond to the spectrum of
  wavenumbers supported by the spatial discretization.  On the lattice, these
  wavenumbers are rescaled according to \Eq{eq:scale-x} so that the maximum
  wavenumber is $O(1)$, and hence
  \begin{equation}
    \label{eq:xi}
    \vert \Xi_i'\vert = O\left(\frac{\kn\tau}{\min_k (\lambda_k -
      \lambda_i)}\right).
  \end{equation}
  As we determined numerically in
  Fig.~\ref{fig:spectral}, there are up to four (only three in
  the one-dimensional LBE) discrete eigenvalues $-\{0, 1, 2, 3\}
  \times (\kn\tau)^{-1}$, so 
  \begin{equation}
    \label{eq:vecnorm-xiprime}
    \vert \Xi_i'\vert = O((\kn\tau)^2).
  \end{equation}
  
  The repetition of the single-point Carleman eigenvalues in the $n$-point
  Carleman eigenvalue spectrum means that each eigenvalue is highly degenerate.  We
  therefore need to treat the case of eigenvalue degeneracy in
  \Eq{eq:pertnondeg}.  This, too, is a textbook application of perturbation
  theory \cite{Sakurai1994}.  The procedure treats each degenerate
  subspace of $\mathcal J_0$ and chooses the eigenbasis such that it diagonalizes the perturbation within each of the subspaces.  There are up to four (only three in
  the one-dimensional LBE) such degenerate subspaces
  corresponding to the four discrete eigenvalues $-\{0, 1, 2, 3\}
  \times (\kn\tau)^{-1}$; we label these spaces $D_0$ through $D_3$.
  For each space, we define a projection operator
  \begin{equation}
    \label{eq:projection}
    P_l = \sum\limits_{\Xi_i \in D_l} \frac{\Xi_i \Xi_i^\trans}{\|\Xi_i\|^2}.
  \end{equation}
  We then diagonalize the perturbation operator $\mathcal C_s$ in each
  degenerate subspace in turn.  
  For the $l$th subspace, the
  first-order eigenvector perturbation is given by
  \begin{equation}
    \label{eq:pertdeg}
    P_l \Xi_i' = (\kn\tau)^2
    P_l \sum\limits_{j\neq i} \frac{\Xi_i}{\mu_j - \mu_i}
    \sum\limits_{k\notin D_l} \Xi_i^{-1} \mathcal C_s \Xi_k
    \frac{1}{\lambda_{D_l} - \lambda_k} \Xi_k^{-1} \mathcal C_s \Xi_j \ ,
  \end{equation}
  where the projection operator ensures that the degeneracy-breaking
  diagonalization of $\mathcal C_s$ is only performed within each
  degenerate subspace and $\mu_{i(j)}$ are the eigenvectors of
  $\mathcal C_s$; between subspaces, \Eq{eq:pertnondeg} continues
  to apply.  The scaling of the terms in \Eq{eq:pertdeg} is
  familiar except for $\sum (\mu_j - \mu_i)^{-1}$.  This term
  needs careful examination because the smallest difference in
  eigenvalues of $\mathcal C_s$ is set by the smallest difference in
  scaled wavenumbers, which scales with the number of grid points as
  $n^{-1/D}$ in $D\in\{1,2,3\}$ spatial dimensions.  In the $n\gg 1$ limit, we can
  bound this factor by
  \begin{equation}
    \label{eq:mu}
    \sum\limits_{j\neq i} \frac{1}{\mu_j - \mu_i} \propto
    \int\limits_{\vec{k}_i \neq \vec{k}_j} \frac{d^D\vec{k}_j}
    {\vert\vec{k}_j - \vec{k}_i\vert} \propto  
    \int\limits_{k =
    O\left(n^{-\frac 1 D}\right)}^{O(1)} \frac{k^{D-1}dk}{k} = O\left(\log n\right),
  \end{equation}
for the most restrictive case, $D=1$.
For $D=2$ and $D=3$, \Eq{eq:mu} is reduced to $1-n^{-1/2}$ and $\frac{1}{2} (1-n^{-1/3})$, respectively, which becomes $0$ in the $n\gg 1$ limit. 

  Combining the various factors, the degenerate perturbations to $\Xi_i$ are suppressed
  relative to the unperturbed eigenvectors (treating $D$ as a fixed
  value rather than a parameter of the problem) by $O((\kn\tau)^3\log
  n)$.  The nondegenerate perturbations are suppressed
  by $O((\kn\tau)^2)$.  The singular values of $\mathcal J$ are not
  necessarily smooth but can be evaluated from the eigenvalues of the
  square of $\mathcal J$, i.e., $\Sigma = {\mathcal{J}}^\dagger {\mathcal{J}}$.  We then have that \begin{equation}
        \|{\mathcal{J}}^\dagger {\mathcal{J}} - {\mathcal{J}_0}^\dagger {\mathcal{J}_0}\| \le (\|\mathcal{J}\| +\|\mathcal{J}_0\|)\|\mathcal{J} - \mathcal{J}_0\|.
  \end{equation}  This shows that the singular values remain smooth provided that $\mathcal{J}_0$ is non-singular and further  places
  the restrictions
  \begin{align}
    (\kn\tau)^2 \ll 1 \\
    \label{eq:kn-logn}
    (\kn\tau)^3 \log n \ll 1 
  \end{align}
  on the free parameters of the system.  The first condition is less
  restrictive than the $\kn\tau \ll 1$ condition the system must
  already satisfy, while the second is unlikely to be a practical
  impediment for reasonable values of $n$.

  Lastly, we need to bound the effect of the perturbation on
  $\|\mathcal J^{-1}\|$.  To accomplish this, we use the approximation
  \begin{equation}
    \label{eq:J}
    \mathcal J^{-1} = (\mathcal J_0 + \mathcal J')^{-1} \approx  \mathcal
    J_0^{-1} - \mathcal J_0^{-1} \mathcal J' \mathcal J_0^{-1}
  \end{equation}
  where $\mathcal J' = \mathcal J - \mathcal J_0$ is the perturbation
  to the eigenvectors; since this perturbation scales as
  $O((\kn\tau)^2)$, the neglected terms in the expansion of the
  inverse scale as $O((\kn\tau)^4)$.  By the submultiplicative property
  of the spectral norm,
  \begin{equation}
    \|\mathcal J^{-1} - \mathcal J_0^{-1}\| \le \|\mathcal J_0^{-1}\|^2 \|\mathcal J'\|.
  \end{equation}
  Using $\|\mathcal J_0^{-1}\|$
  from the block structure
  of $\mathcal C_c$ and $\max_i \vert \Xi_i'\vert = O((\kn\tau)^2)$ from
  \Eq{eq:vecnorm-xiprime} to bound $\|\mathcal J'\|$, we conclude that the
  $n$-point $\|\mathcal J^{-1}\|$ is also independent of $n$.
  Finally, we numerically bound $\kappa_{\mathcal J}$ as
  \begin{equation}
    \label{eq:kj-value}
    \kappa_{\mathcal J} \le 1.152 \times 10^5.
  \end{equation}
\end{enumerate}

\begin{figure}
  \centering
  \includegraphics[width=12cm]{spectral-plot-1} 
    \caption{Histograms of eigenvalue spectra $\mathcal{R}(\lambda)$ of 3rd-degree Carleman-linearized D1Q3, D2Q9, and
    D3Q27 LBM matrices.  Up to floating point error $<10^{-12}$ from the numerical
    diagonalization, the eigenvalues are all discrete multiples $-\{0, 1, 2, 3\}
  \times (\kn\tau)^{-1}$ of the Boltzmann relaxation scale $(\kn\tau)^{-1}$.
  Again up to numerical fuzz $<10^{-13}$, all eigenvalues are purely real.}
  \label{fig:spectral}
\end{figure}


\subsubsection*{Final quantum gate complexity to solve \Eq{eq:clbm}}
Finally, inserting  \Eq{eq:epsilon}, \Eq{eq:evolutionT}, \Eq{eq:g-decay},  \Eq{eq:sparsity}, \Eq{eq:normc}, and \Eq{eq:kj-value} 
into \Eq{eq:ca} yields the following result for the quantum algorithm complexity as quantified by the number of two-qubit gates needed in the simulation:
\begin{equation}
  \label{eq:complong}
  \text{gate complexity} = O(t_c^{-1}\tilde{T}\poly\log(n/\epsilon)),
\end{equation}
where $\tilde T$ is the physical evolution time.
The scaling of each term in \Eq{eq:ca} is summarized in Table~\ref{tab:scaling}.
\begin{table}
  \centering
  \begin{tabular}{lll}
    \hline\hline
    Term & Scaling & Reference \\ \hline
    $\beta$    & $O(1)$                &  \Eq{eq:beta}     \\
    $g$        & $3\sqrt{k}$                &  \Eq{eq:g-decay}        \\
    $s$        & $O(1)$              &  \Eq{eq:sparsity} \\
    $\|\mathcal C\|$    & $O(\kn^{-1})$ & \Eq{eq:normc}    \\
    $\kappa_J$ & $\le 1.152\times 10^5$ & \Eq{eq:kj-value}        \\
    \hline\hline
  \end{tabular}
  \caption{Scaling of each term in \Eq{eq:ca} with lattice Boltzmann equation parameters leading to the
    result of \Eq{eq:complong}. Note that the dissipation parameter $g$ is a constant because the Carleman linearized LBE converges at the truncation order $k=3$.  
    }
  \label{tab:scaling}
\end{table}
\ifarXiv
\else
In the main text, \Eq{eq:complong}
is discussed in terms of the physical parameters of the turbulence
simulation problem.  
\fi

The solution error of using the quantum
linear system algorithm (QLSA) of
\citet{Berry2017QuantumPrecision} to solve
\Eq{eq:LBMs} come from two sources: the
Carleman truncation error and the QLSA
error. Since the solution error from QLSA
contributes $\text{poly}(\log(\epsilon)$ to
the gate complexity expressed in \Eq{eq:ca},
we focus on bounding the Carleman truncation
error, which is $O(\text{Ma}^2)$. For weakly
compressible flow, $\text{Ma}\ll1$,
therefore, the Carleman truncation error is
very small. In \citet{Liu21}, a QLSA with
gate complexity scales with $T^2$ was used
due to their time dependent Carleman matrix.
A time dependent Carleman matrix requires
discretizing time using the forward Euler
method, which contributes to the solution
error. In contrast, our Carleman matrix
$\mathcal{C}$ is constant, which affords us
to use the QLSA of
\citet{Berry2017QuantumPrecision} with a
gate complexity scales with $T$
(\Eq{eq:ca}). Overall, our method of using
QLSA to solve \Eq{turb} has yields a gate
complexity scaling with $\poly(\log{N})$ and
$T$ for any arbitrary Reynolds number
comparing to \citet{Liu21}'s complexity
scaling with $T^2$ for $R<1$.   

