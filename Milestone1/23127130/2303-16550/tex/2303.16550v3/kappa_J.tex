\paragraph{Condition number}

To bound  $\kappa_{\mathcal{J}}$, we need to find the eigenvectors of $\mathcal C$.  This is
not possible analytically for arbitrary $n$; if an analytic expression for the
eigenvectors could be obtained, a quantum algorithm, or any numerical
simulation, would no longer be required to calculate the dynamics of the system
described by $\mathcal C$.  We will derive a bound on $\kappa_\mathcal{J}$ in three steps.
First, we will numerically diagonalize the single-point collision Carleman
matrix $C$ at Carleman order $k=3$ (\Eq{eq:Cdef}) for lattice structures used
to recover the NSE from the LBE; these are specific matrices with
fixed, explicitly known $a$, $b$, $c$, $d$, $w_m$, and $Q$ parameters
in \Eq{eq:lbm-grouped}, but they can solve
general flows \cite{Chen1998LatticeFlows}.  Second, we will show that the
eigenvectors of the $n$-point collision Carleman matrix $\mathcal C_c(\vec x)$,
defined in \Eq{eq:npt-coll}, can be constructed from the eigenvectors of the
single-point $C$.  Third, we will perform a perturbative expansion to include
the effect of the streaming Carleman matrix $\mathcal C_s(\vec x)$, defined in
\Eq{eq:npt-stream}, and derive an approximate expression for $\kappa_\mathcal{J}$ that
holds as long as the Knudsen number of the flow is small; this condition needs
to be satisfied for the LBE to describe the flow.  The numbered paragraphs below
provide the details of these three steps.

\begin{enumerate}
\item In one, two, and three spatial dimensions, the D1Q3, D2Q9, and D3Q27
  lattice topologies are frequently used and are known to recover the NSE in the
  $\ma\ll 1$ limit \cite{Chen1998LatticeFlows} on which the present work
  focuses.  The single-point collision Carleman matrices $C$ and their
  spectral decompositions $C=JDJ^{-1}$ are provided as a data
  supplement \cite{url}. 
  By inspection of these numerical results,
  all eigenvalues of $C$ are 
  $-(\kn\tau)^{-1} \times \{0, 1, 2, 3\}$, that is, they are either degenerate or separated by gaps of
  $(\kn\tau)^{-1}$, as shown by the histogram of the eigenvalue spectra
  $\mathcal{R}(\lambda)$ in Fig.~\ref{fig:spectral}.  The eigenvector matrices
  $J$ and $J^{-1}$  are similarly constant for each choice of $Q\in\{3,9,27\}$.
  The corresponding condition numbers are $\kappa_J = 12.94$ for D1Q3, $\kappa_J
  = 85.66$ for D2Q9, and $\kappa_J = 2269$ for D3Q27.  Bounds based on
  \Eq{eq:specnorm} are $\kappa_J \le 62.41$ for D1Q3, $\kappa_J
  \le 2252$ for D2Q9, and $\kappa_J \le 1.152\times 10^5$ for D3Q27; these bounds will be used
  to generalize to $n$ grid points below.
\item Denote the eigenvectors of the single-point $C$ as $\xi_i$, $i\in
  1,\dots,Q$, and decompose
  \begin{equation}
    \xi_i = (\xi_i^{(1)}, \xi_i^{(2)}, \xi_i^{(3)})^\trans
  \end{equation}
  according to whether its elements represent 1-, 2-, or 3-form elements (powers of $f_m$ polynomials defined in \Eq{eq:lbm-carleman}) of the
  single-point Carleman vectors $V$.  Using the $\delta_\alpha$ vector from
  \Eq{eq:delta-alpha}, we can construct localized eigenvectors $\Xi_i(\vec
  x_\alpha)$ of the $n$-point collision Carleman matrix (\Eq{eq:npt-coll}):
  \begin{equation}
    \label{eq:Xi}
    \Xi_i(\vec x_\alpha) =
    (\delta_\alpha \otimes \xi_i^{(1)},
    \delta_\alpha^{[2]} \otimes \xi_i^{(2)},
    \delta_\alpha^{[3]} \otimes \xi_i^{(3)})^\trans.
  \end{equation}
  By \Eq{eq:diagonality-A}, $\Xi_i(\vec x_\alpha)$ is an eigenvector of the
  $n$-point collision matrix $\mathcal C_c^{(3)}(\vec x)$ for any
  $\alpha\in 1,\dots,n$, as the diagonality of the $n$-point transfer matrices
  in $\vec x$ ensures that the nonzero blocks of $\mathcal C_c$ are aligned with
  the nonzero blocks of $\Xi_i(\vec x_\alpha)$.  The eigenvectors of the
  $n$-point collision Carleman matrix $\mathcal C_c$ are therefore a simple
  blockwise repetition of the eigenvectors of the single-point $C$.
  Furthermore, it follows from \Eq{eq:Xi} that
  \begin{equation}
    \label{eq:Xi-ortho}
    \Xi_i(\vec x_\alpha)^\trans \Xi_j(\vec x_\beta) = \xi_i^\trans \xi_j \delta_{\alpha\beta},
  \end{equation}
  i.e., the eigenvectors at different grid points are mutually orthogonal.  The
  corresponding eigenvalues are equal for all $\alpha \in 1,\dots,n$, i.e., they
  are repetitions of the eigenvalues of $C$.

  The $\Xi_i(\vec x_\alpha)$ are stacked column-wise or row-wise to
  form the eigenvector matrices of $\mathcal C_c$.  Denote these eigenvector
  matrices as $\mathcal J_0$ and $\mathcal J_0^{-1}$.  (The subscript 0 notation
  is used in anticipation of the perturbation analysis carried out in the next
  paragraph.)  By \Eq{eq:Xi-ortho}, the eigenvectors localized at different grid points
  form blocks in $\mathcal J_0$ and $\mathcal J_0^{-1}$ that ensure that row and
  column sums on $\mathcal J_0$ and $\mathcal J_0^{-1}$ consist of the
  same $O(Q^3)$
  nonzero elements as row and column sums on the \textit{single-point constant}
  eigenvector matrices $J$ and $J^{-1}$, rather than the full
  $O(n^3Q^3)$ dimension of the $n$-point 
  matrices.  Thus, the same spectral-norm bound from
  \Eq{eq:specnorm} applies to $\|J\|$ and $\|\mathcal J_0\|$; and
  the same bound applies to $\|J^{-1}\|$ and $\|\mathcal J_0^{-1}\|$.
  Thus, $\kappa_{\mathcal{J}_0}$ is fixed (for fixed $Q$), independent
  of $n$.
  
\item Finally, we bound the effect of the streaming contribution $\mathcal C_s$
  to the eigenvectors of $\mathcal C=\mathcal C_s + \mathcal C_c$ and thence the
  effect on $\mathcal J$, $\mathcal J^{-1}$, and $\kappa_{\mathcal J}$.
  Inspection of \Eq{eq:lbm-grouped} shows that the streaming term in the LBE
  is suppressed by $\kn\ll 1$ relative to the collision term; this suggests that
  we can treat streaming using well established principles of (degenerate)
  perturbation theory \cite{Sakurai1994, bamieh2020tutorial}, with the collision Carleman matrix in
  \Eq{eq:n_point_Carleman} constituting the base matrix (with known
  eigenvectors) and the streaming term constituting the $O(\kn)\ll 1$
  perturbation.

  Item 2.\ above derived the properties of the $n$-point collision eigenvector
  matrices $\mathcal J_0$ and $\mathcal J_0^{-1}$ in terms of the single-point
  eigenvector matrices $J$ and $J^{-1}$ that had to be computed numerically.
  When the perturbation due to $\mathcal C_s$ is included, the eigenvectors are
  perturbed \cite{Sakurai1994} in proportion to the perturbation strength
  ($\kn\tau$) and the gap size of the eigenvalue spectrum of $\mathcal C_c$,
  neglecting until the next paragraph the issue of degenerate eigenvalues.  For
  notational convenience, we suppress the $\vec x_a$ notation on the
  eigenvectors of $\mathcal C_c$; the perturbed eigenvectors to first order are
  then given by $\Xi_i + \Xi_i'$, where
  \begin{equation}
    \label{eq:pertnondeg}
    \Xi_i' =  \kn\tau \sum_{k\neq i} \frac{\Xi_i^{-1} \mathcal C_s \Xi_k}{\lambda_i -
      \lambda_k} \Xi_k.
  \end{equation}
  Numerical calculation of the eigenvalue spectrum of the single-point Carleman
  matrices in item 1.\ above showed the gap between nondegenerate eigenvalues
  to be integer multiples of $(\kn\tau)^{-1} \gg 1$, so the perturbation to the eigenvectors is
  proportional to $O((\kn\tau)^{2}) \ll 1$ times the expectation values $\sum\Xi_i^{-1}
  \mathcal C_s \Xi_k$.  In the nondegenerate case, the eigenvectors
  form an orthonormal basis, and the vector norm of the $\Xi_i'$ can be bounded
  as follows:
  \begin{align}
    \vert \Xi_i'\vert^2 &= (\kn\tau)^2 \sum\limits_{k\neq i}
    \frac{\Xi_i^{-1}\mathcal C_s \Xi_k
    \Xi_k^{-1}\mathcal C_s^\dagger \Xi_i}{(\lambda_k -
                          \lambda_i)^2} \nonumber \\
    &\leq \frac{(\kn\tau)^2}{\min_k (\lambda_k - \lambda_i)^2} \sum\limits_{k\neq i}
    \Xi_i^{-1}\mathcal C_s \Xi_k \Xi_k^{-1}\mathcal C_s^\dagger
      \Xi_i \nonumber \\
    \label{eq:temp1}
    &\leq \frac{(\kn\tau)^2}{\min_k (\lambda_k - \lambda_i)^2} \left[\sum\limits_k
      \Xi_i^{-1}\mathcal C_s \Xi_k \Xi_k^{-1}\mathcal C_s^\dagger \Xi_i
    - \left\vert \Xi_i^{-1} \mathcal C_s \Xi_i\right\vert^2\right].
  \end{align}
  Using completeness and noting that the final term in \Eq{eq:temp1} is
  nonpositive-definite,
  \begin{equation}
    \vert \Xi_i'\vert^2 \leq \frac{(\kn\tau)^2}{\min_k (\lambda_k -
    \lambda_i)^2} \Xi_i^{-1} \mathcal C_s \mathcal
    C_s^\dagger \Xi_i,
  \end{equation}
  and therefore
  \begin{equation}
    \vert \Xi_i'\vert \leq \frac{\kn\tau}{\min_k (\lambda_k -
      \lambda_i)} \|\mathcal C_s \|.
  \end{equation}
  The eigenvalues of the streaming operator correspond to the spectrum of
  wavenumbers supported by the spatial discretization.  On the lattice, these
  wavenumbers are rescaled according to \Eq{eq:scale-x} so that the maximum
  wavenumber is $O(1)$, and hence
  \begin{equation}
    \label{eq:xi}
    \vert \Xi_i'\vert = O\left(\frac{\kn\tau}{\min_k (\lambda_k -
      \lambda_i)}\right).
  \end{equation}
  As we determined numerically in
  Fig.~\ref{fig:spectral}, there are up to four (only three in
  the one-dimensional LBE) discrete eigenvalues $-\{0, 1, 2, 3\}
  \times (\kn\tau)^{-1}$, so 
  \begin{equation}
    \label{eq:vecnorm-xiprime}
    \vert \Xi_i'\vert = O((\kn\tau)^2).
  \end{equation}
  
  The repetition of the single-point Carleman eigenvalues in the $n$-point
  Carleman eigenvalue spectrum means that each eigenvalue is highly degenerate.  We
  therefore need to treat the case of eigenvalue degeneracy in
  \Eq{eq:pertnondeg}.  This, too, is a textbook application of perturbation
  theory \cite{Sakurai1994}.  The procedure treats each degenerate
  subspace of $\mathcal J_0$ and chooses the eigenbasis such that it diagonalizes the perturbation within each of the subspaces.  There are up to four (only three in
  the one-dimensional LBE) such degenerate subspaces
  corresponding to the four discrete eigenvalues $-\{0, 1, 2, 3\}
  \times (\kn\tau)^{-1}$; we label these spaces $D_0$ through $D_3$.
  For each space, we define a projection operator
  \begin{equation}
    \label{eq:projection}
    P_l = \sum\limits_{\Xi_i \in D_l} \frac{\Xi_i \Xi_i^\trans}{\|\Xi_i\|^2}.
  \end{equation}
  We then diagonalize the perturbation operator $\mathcal C_s$ in each
  degenerate subspace in turn.  
  For the $l$th subspace, the
  first-order eigenvector perturbation is given by
  \begin{equation}
    \label{eq:pertdeg}
    P_l \Xi_i' = (\kn\tau)^2
    P_l \sum\limits_{j\neq i} \frac{\Xi_i}{\mu_j - \mu_i}
    \sum\limits_{k\notin D_l} \Xi_i^{-1} \mathcal C_s \Xi_k
    \frac{1}{\lambda_{D_l} - \lambda_k} \Xi_k^{-1} \mathcal C_s \Xi_j \ ,
  \end{equation}
  where the projection operator ensures that the degeneracy-breaking
  diagonalization of $\mathcal C_s$ is only performed within each
  degenerate subspace and $\mu_{i(j)}$ are the eigenvectors of
  $\mathcal C_s$; between subspaces, \Eq{eq:pertnondeg} continues
  to apply.  The scaling of the terms in \Eq{eq:pertdeg} is
  familiar except for $\sum (\mu_j - \mu_i)^{-1}$.  This term
  needs careful examination because the smallest difference in
  eigenvalues of $\mathcal C_s$ is set by the smallest difference in
  scaled wavenumbers, which scales with the number of grid points as
  $n^{-1/D}$ in $D\in\{1,2,3\}$ spatial dimensions.  In the $n\gg 1$ limit, we can
  bound this factor by
  \begin{equation}
    \label{eq:mu}
    \sum\limits_{j\neq i} \frac{1}{\mu_j - \mu_i} \propto
    \int\limits_{\vec{k}_i \neq \vec{k}_j} \frac{d^D\vec{k}_j}
    {\vert\vec{k}_j - \vec{k}_i\vert} \propto  
    \int\limits_{k =
    O\left(n^{-\frac 1 D}\right)}^{O(1)} \frac{k^{D-1}dk}{k} = O\left(\log n\right),
  \end{equation}
for the most restrictive case, $D=1$.
For $D=2$ and $D=3$, \Eq{eq:mu} is reduced to $1-n^{-1/2}$ and $\frac{1}{2} (1-n^{-1/3})$, respectively, which becomes $0$ in the $n\gg 1$ limit. 

  Combining the various factors, the degenerate perturbations to $\Xi_i$ are suppressed
  relative to the unperturbed eigenvectors (treating $D$ as a fixed
  value rather than a parameter of the problem) by $O((\kn\tau)^3\log
  n)$.  The nondegenerate perturbations are suppressed
  by $O((\kn\tau)^2)$.  The singular values of $\mathcal J$ are not
  necessarily smooth but can be evaluated from the eigenvalues of the
  square of $\mathcal J$, i.e., $\Sigma = {\mathcal{J}}^\dagger {\mathcal{J}}$.  We then have that \begin{equation}
        \|{\mathcal{J}}^\dagger {\mathcal{J}} - {\mathcal{J}_0}^\dagger {\mathcal{J}_0}\| \le (\|\mathcal{J}\| +\|\mathcal{J}_0\|)\|\mathcal{J} - \mathcal{J}_0\|.
  \end{equation}  This shows that the singular values remain smooth provided that $\mathcal{J}_0$ is non-singular and further  places
  the restrictions
  \begin{align}
    (\kn\tau)^2 \ll 1 \\
    \label{eq:kn-logn}
    (\kn\tau)^3 \log n \ll 1 
  \end{align}
  on the free parameters of the system.  The first condition is less
  restrictive than the $\kn\tau \ll 1$ condition the system must
  already satisfy, while the second is unlikely to be a practical
  impediment for reasonable values of $n$.

  Lastly, we need to bound the effect of the perturbation on
  $\|\mathcal J^{-1}\|$.  To accomplish this, we use the approximation
  \begin{equation}
    \label{eq:J}
    \mathcal J^{-1} = (\mathcal J_0 + \mathcal J')^{-1} \approx  \mathcal
    J_0^{-1} - \mathcal J_0^{-1} \mathcal J' \mathcal J_0^{-1}
  \end{equation}
  where $\mathcal J' = \mathcal J - \mathcal J_0$ is the perturbation
  to the eigenvectors; since this perturbation scales as
  $O((\kn\tau)^2)$, the neglected terms in the expansion of the
  inverse scale as $O((\kn\tau)^4)$.  By the submultiplicative property
  of the spectral norm,
  \begin{equation}
    \|\mathcal J^{-1} - \mathcal J_0^{-1}\| \le \|\mathcal J_0^{-1}\|^2 \|\mathcal J'\|.
  \end{equation}
  Using $\|\mathcal J_0^{-1}\|$
  from the block structure
  of $\mathcal C_c$ and $\max_i \vert \Xi_i'\vert = O((\kn\tau)^2)$ from
  \Eq{eq:vecnorm-xiprime} to bound $\|\mathcal J'\|$, we conclude that the
  $n$-point $\|\mathcal J^{-1}\|$ is also independent of $n$.
  Finally, we numerically bound $\kappa_{\mathcal J}$ as
  \begin{equation}
    \label{eq:kj-value}
    \kappa_{\mathcal J} \le 1.152 \times 10^5.
  \end{equation}
\end{enumerate}

\begin{figure}
  \centering
  \includegraphics[width=12cm]{spectral-plot-1} 
    \caption{Histograms of eigenvalue spectra $\mathcal{R}(\lambda)$ of 3rd-degree Carleman-linearized D1Q3, D2Q9, and
    D3Q27 LBM matrices.  Up to floating point error $<10^{-12}$ from the numerical
    diagonalization, the eigenvalues are all discrete multiples $-\{0, 1, 2, 3\}
  \times (\kn\tau)^{-1}$ of the Boltzmann relaxation scale $(\kn\tau)^{-1}$.
  Again up to numerical fuzz $<10^{-13}$, all eigenvalues are purely real.}
  \label{fig:spectral}
\end{figure}
