\vspace{\rmwhitebfsec}\section{Introduction}\label{sec:introduction}\vspace{\rmwhiteafsec}
Due to the ever-increasing performance requirements, control problems in engineering are getting increasingly more complex, with the need to precisely address \emph{nonlinear}~(NL) aspects of the behavior of the underlying systems. This in turn also 
requires %
accurate modeling of such NL behaviors, which often becomes cumbersome or even impossible with first-principle modeling techniques. While data-driven methods provide an alternative, in the absence of a mature NL identification for control theory, it is often  difficult to decide which part of the behavior is crucial to be captured for control design and how the uncertainty of the estimated model influences the subsequent control synthesis.
For this reason, data-driven control methods have been developed to design controllers \emph{directly} from data, eliminating the need of a modeling step. In the \emph{linear time-invariant} (LTI) case, the \emph{Fundamental Lemma}~\cite{WillemsRapisardaMarkovskyMoor2005} has proven to be a key result, allowing for direct data-driven analysis and control synthesis with stability and performance guarantees, see~\cite{MarkovskyDorfler2021Review}. %
Besides of promising approaches based on feedback and online linearizations, or polynomial bases~\cite{de2023learning, berberich2022linear, markovsky2021data}, an analogous result for general NL systems has not been achieved yet.

In this paper, we propose a novel extension of the Fundamental Lemma to a wide class of  \emph{discrete-time} (DT) NL systems that can be described in a state-space form with differentiable state transition and output functions. Our result is based on the use of  
the velocity-form of the NL system, which describes the time-difference
dynamics of the system and it has two important properties: %
\textit{(i)}~stability and performance of the velocity-form imply \emph{universal shifted}, i.e., equilibrium-independent,  stability and performance of the original NL system~\cite{Koelewijn2023,Koelewijn2023a,Simpson-Porco2019}, %
\textit{(ii)}~the velocity-form naturally results in a \emph{linear parameter-varying} (LPV) system. By calculating time-differences of the data from the underlying NL system, which characterizes the velocity form, our first contribution (C1) is to show that the resulting data-equations allow for convex data-driven analysis and controller synthesis by the use of the recently introduced LPV Fundamental Lemma \cite{VerhoekTothHaesaertKoch2021} due to property \emph{(i)}. Then, by exploiting \emph{(ii)}, our second main contribution (C2) is to show that the data-driven controller for the velocity-form, obtained in the previous step, exhibits a computable realization, and to prove that this realization provides {universal shifted} guarantees for  closed-loop control of the original NL system.

 




In Section~\ref{sec:problemstatement}{,} {we} formalize the NL data-driven control problem that we intend to solve. Section~\ref{sec:preliminaries} introduces the data-based representation of the velocity-form of the NL system using {an} LPV embedding and the {LPV} Fundamental Lemma. Section~\ref{sec:mainresults} uses the data-driven representation to synthesize a state-feedback controller for the velocity-form, which by realization to a NL state-feedback law provides {equilibrium independent} guarantees. Section~\ref{sec:examples} demonstrates the applicability of the results in a simulation example based on an unbalanced disc system, {while} the conclusions on the achieved results are given in Section~\ref{sec:conclusion}. 

\subsubsection*{Notation}
The set of integers is denoted by $\Z$, while the set of real numbers is denoted by $\R$. Moreover, $\R_0^+=[0,\infty)\subset\R$. A function $f:\R^{p}\to\R^q$ is in $\mc{C}^n$ if it is $n$-times continuously differentiable, while $f:\R^p\to\R$ belongs to the class $\mc{Q}_{x_\ast}$ if it is positive definite and decrescent w.r.t. $x_\ast\in\R^p$ (see \cite{SchererWeiland2021}). \new{$\col(x_1, \dots, x_n)$ denotes $[x_1^\top \hspace{0.2mm} \cdots\hspace{0.7mm} x_n^\top]^\top$.}