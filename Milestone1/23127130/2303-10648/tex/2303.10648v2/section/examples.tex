\vspace{\rmwhitebfsec}\section{Simulation study}\label{sec:examples}\vspace{\rmwhiteafsec}
We demonstrate the applicability of our results on a simulator of an unbalanced disc system, for which we synthesize a universal shifted data-driven state-feedback controller and compare it with a data-driven state-feedback LPV controller that uses a direct LPV embedding of the NL system, \new{cf.~\cite{Verhoek2022_DDLPVstatefb}}. For {comparison,} %
we also synthesize an LTI {data-driven} controller. %
The CT dynamics of the unbalanced disc system mimic those of an inverted pendulum and are thus described by the following ordinary differential equation
\begin{equation}\label{eq:unbalanced-disc}
    \ddot{\theta}(t)=-\tfrac{mgl}{J}\sin(\theta(t))-\tfrac{1}{\tau}\dot{\theta}(t)+\tfrac{K_\mr{m}}{\tau}u(t),
\end{equation}
where $\theta$ is the angular position of the disc in radians, $u$ is the input voltage to the system, which is its control input, and $m,g,l,J,\tau,K_\mr{m}$ are the physical parameters of the system that we take from \cite[Tab.~I]{Verhoek2022_DDLPVstatefb}. Discretizing the dynamics using a first-order Euler method and writing {them} in the form of \eqref{eq:NL_sys} gives
\begin{subequations}\label{eq:DT-unbaldisc}
\begin{align}
    \hspace{-1mm}x_{1,k+1} & = x_{1,k} + T_\mr{s} x_{2,k},\\
    \hspace{-1mm}x_{2,k+1} & = (\tfrac{T_\mr{s}}{\tau}-1)x_{2,k} -\tfrac{T_\mr{s}mgl}{J}\sin(x_{1,k})+\tfrac{T_\mr{s}K_\mr{m}}{\tau}u_k,
\end{align}
where $x_k = \begin{bsmallmatrix} \theta_k & \dot\theta_k \end{bsmallmatrix}^\top$.
\end{subequations}
We choose the sampling-time as $T_\mr{s}=0.01$ [s], which gives a negligible discretization error through the Euler scheme. The control objective is to design a controller that tracks a reference for $\theta_k$ with zero steady-state error, which requires integrator action. We introduce the integrator behavior with the tuning parameter $\alpha$, see~\cite[Cor.~8.2]{Koelewijn2023}. For the direct LPV design, we introduce integrator behavior by adding an augmented state $x_{\mr{aug},k+1}=\alpha x_{\mr{aug},k} + \theta_{\mr{ref},k} - x_{1,k}$. Note that with the extra state, we require a larger data-dictionary for the construction of the direct data-driven LPV representation.

\begin{figure*}
\deflen{removewhitespaceforfig}{-5mm}
	\centering
	\begin{minipage}[t]{0.49\linewidth}
	   \vspace{2mm}
	   \includegraphics[scale=1, trim=0mm 0.95mm 0mm 0.7mm, clip]{figures/datadictionary_ref}
	   \vspace{\removewhitespaceforfig}
	   \vspace{-2mm}
	   \caption{Data-dictionary $\mc{D}_{N\!+\!1}^{\textsc{nl}}$ used for the NL and LPV control synthesis {with} $N=8$. The {extra} gray (\legendline{cgray}) {data-points} are required  for the direct LPV representation because of the added state for the {integrator} behavior.}
	   \label{fig:datadictionary}
	\end{minipage}\hfill
	\begin{minipage}[t]{0.49 \linewidth}
       \vspace{2mm}
	   \includegraphics[scale=1, trim=0mm 0.95mm 0mm 0.7mm, clip]{figures/resexample_ref}
	   \vspace{\removewhitespaceforfig}
	   \vspace{-2mm}
	   \caption{Response of the unbalanced disc with the universal shifted controller (\legendline{mblue}) and the LPV controller (\legendline{morange}) in closed-loop for a step reference (\legendline{black}). An LTI controller (\legendline{myellow}) designed with the same specifications diverges directly.}%
	   \label{fig:resexample}
	\end{minipage}\vspace{\removewhitespaceforfig}
\end{figure*}

The velocity-form of \eqref{eq:DT-unbaldisc}
can be computed analytically:
\begin{subequations}\label{eq:DT-unbaldisc:velocity}
\begin{align}
    \deltax_{k+1} & = A_\mr{v}(x_{1,k}, x_{1,k-1})\deltax_k + B_\mr{v}\deltau_k, \\ 
    \deltay_k & = \deltax_k,
\end{align}
\end{subequations}
where $B_\mr{v}= \begin{bmatrix} 0 & \tfrac{T_\mr{s}K_\mr{m}}{\tau} \end{bmatrix}^\top$ and 
\begin{equation*}
    A_\mr{v}(x_{1,k}, x_{1,k-1}) = \begin{bmatrix} 1 & T_\mr{s} \\ -\tfrac{T_\mr{s}mgl}{J}\sind(x_{1,k}, x_{1,k-1}) & 1-\tfrac{T_\mr{s}}{\tau} \end{bmatrix},
\end{equation*}
{with} $\sind(a,b):=\tfrac{\sin(a)-\sin(b)}{a-b}$\new{, which is obtained by solving the integral in \eqref{eq:NL_sys_velocity}}. For the data-driven design of the NL universal shifted controller, we choose\footnote{{This basis is used} for simplicity and comparison purposes with the direct LPV design, but one could alternatively choose a polynomial basis.} $p_k := \psi(x_{1,k}, x_{1,k-1})=\sind(x_{1,k}, x_{1,k-1})$, which allows for an LPV embedding of the velocity-form \eqref{eq:DT-unbaldisc:velocity}. Note that $\lim_{x_{1,k}\rightleftarrows x_{1,k-1}}\psi(x_{1,k}, x_{1,k-1})$ exists and for all {trajectories of \eqref{eq:DT-unbaldisc}} $\psi(x_{1,k}, x_{1,k-1})\in[-1, 1]$. Hence, we take this interval as $\mb{P}$. For the {direct data-driven} LPV design, we follow \cite{Verhoek2022_DDLPVstatefb} {to formulate an LPV embedding of \eqref{eq:DT-unbaldisc}} where we choose $p_k = \tfrac{\sin(x_{1,k})}{x_{1,k}}$, {which is well-defined for $x_{1,k}=0$}. 

We are now ready to construct the LPV data-driven representations and synthesize controllers for the velocity-form and the {original} system. To construct well-posed data-driven representations for both approaches, while using the same data-set, we need $\mr{rank}(\mc{G}) = (1+\dnp)(\dnx+1+\dnu)=8$, i.e., we need $N\ge8$. The data-dictionary $\mc{D}_{N+1}^{\textsc{nl}}$ is obtained by applying {white noise} $u^\mathrm{d}_k\sim\mc{N}(0,3)$ to \eqref{eq:DT-unbaldisc} under an initial condition $x^\mathrm{d}_1\sim\mc{U}(0,1)$. {The resulting $\mc{D}_{N+1}^{\textsc{nl}}$} is shown in Fig.~\ref{fig:datadictionary}, where the additional data-points required for the augmented LPV representation are {given in} gray.
{Using} $\mc{D}_{N+1}^{\textsc{nl}}$, we construct the direct data-driven LPV representation as in~\cite{Verhoek2022_DDLPVstatefb} and for the velocity-form we construct \eqref{eq:datamatrices} and verify that indeed $\mr{rank}(\mG) = (1+\dnp)(\dnx+\dnu)=6$, %
{giving a well-posed} data-driven representation of \eqref{eq:DT-unbaldisc:velocity}. 

Using the constructed representations, we design an LPV controller and a universal shifted controller (with integral action) using \cite[Thm.~4]{Verhoek2022_DDLPVstatefb} and Corollary~\ref{cor:synthesis}, respectively, with the tuning parameters $Q=I$, $R=2$, $\alpha=0.9$. 
Running the synthesis algorithms yield \new{the parameters of} %
$K^\mr{v}$ and $K^\textsc{lpv}$.
We want to highlight here that  \eqref{eq:DT-unbaldisc} in closed-loop with  the universal shifted controller with $K^\mr{v}$ as above implies USAS of the closed-loop system, %
while the LPV controller only guarantees stability %
of the origin of the NL closed-loop system. {The} latter {is} problematic {for} reference tracking \cite{Koelewijn2020_pitfalls}, which we showcase in the following simulation study.

We simulate\footnote{See \texttt{youtu.be/NeOC9PBipMY} for an animation of the simulations.} \eqref{eq:DT-unbaldisc} in closed-loop with the LPV and universal shifted controller for the initial condition $x_{1}=\begin{bmatrix} \tfrac{\pi}{4} & 5 \end{bmatrix}^\top$. The system must follow a step-reference of magnitude $\tfrac{\pi}{2}$, which pushes the closed-loop away from the origin. The simulated responses of the closed-loops are plotted in Fig.~\ref{fig:resexample}, which shows that both controllers can regulate the system back to the origin. However, when the step reference is applied, only the universal shifted controller can drive the system to the reference, while the LPV controller ends up in a limit cycle. \new{We also design a data-driven LTI state-feedback controller using \cite[Thm.~4]{dePersisTesi2020}} under the same performance specifications and data. \new{Note that the LTI data-driven design spans an LTI behavior based on $\mc{D}_{N+1}^{\textsc{nl}}$, which results in a local approximation of the NL system. %
As outside of this local range, the LTI behavior is not valid anymore, the stability guarantee fails and the closed-loop system quickly diverges with the LTI controller}, see
Fig.~\ref{fig:resexample}.

This example shows that we can synthesize state-feedback controllers for general NL systems of the form \eqref{eq:NL_sys} that are \new{universally shifted} stabilizing and performing while using \emph{only} measured data from the system and a given a set of basis functions $\psi$ that is assumed to span the nonlinearities.













