\vspace{\rmwhitebfsec}\section{Data-based velocity representations}\label{sec:preliminaries}\vspace{\rmwhiteafsec}
To realize our objective, we first show that the velocity-form admits an LPV embedding and that we can obtain an LPV
data-driven representation of %
$\mf{B}_\Delta$ purely based on $\nldataset{N}$.


\vspace{\rmwhitebfssec}\subsection{LPV embedding of the velocity-form}\vspace{\rmwhiteafssec}
To apply the embedding principle, we will need to start with the following assumption:
\begin{assumption}\label{ass:schedulingmap}
    We are given a set of basis functions $\psi_1, \dots, \psi_{\dnp}$ with $\psi: (\mb{X}  \times \mb{U})^2\to\R^{\dnp}$, such that there exist  $A_0, \dots, A_{\dnp}\in\R^{\dnx\times\dnx}$ and $B_0, \dots, B_{\dnp}\in\R^{\dnx\times\dnu}$ for which
    \begin{subequations}\label{eq:decompab}
    \begin{align}
        A_\mr{v}(\xi_{k}, \xi_{k-1}) & = A_0 + {\textstyle\sum_{i=1}^{\dnp}}A_i\psi_i(\xi_{k}, \xi_{k-1}), \\
        B_\mr{v}(\xi_{k}, \xi_{k-1}) & = B_0 + {\textstyle\sum_{i=1}^{\dnp}}B_i\psi_i(\xi_{k}, \xi_{k-1}),
    \end{align}
    \end{subequations}
    with $\xi_k=\col(x_k, u_k)$. 
\end{assumption}
\begin{remark}
    {By} %
    a polynomial basis {set} in $x_k,u_k,x_{k-1},u_{k-1}$, one can \emph{approximate} any $A_\mr{v}$ and $B_\mr{v}$ in the velocity form~\eqref{eq:NL_sys_velocity} under the condition that $f\!\in\!\mc{C}^\infty$, which can be easily shown based on the Taylor series of $f$. Alternatively, one can use kernel-based methods to learn $\psi$ from data.
    Hence, explicit prior knowledge of $f$ is not {necessary} for {choosing an effective} %
   $\psi$. Only the number of basis $\dnp$, governing the approximation error, is required to be determined in advance.%
\end{remark}
Based on Assumption~\ref{ass:schedulingmap}, we use the given set of functions to define a so-called \emph{scheduling variable}, a signal that can represent all the variation of the nonlinearities in \eqref{eq:decompab}: %
\begin{equation}\label{eq:computation_p}
   p_k=\psi(x_k, u_{k}, x_{k-1}, u_{k-1})\in\mb{P} \subseteq \R^{\dnp}. 
\end{equation}
Note that $p_k$ can be computed from measurements of $y_k=x_k$ and $u_k$ through $\psi$, hence using %
the available data set $\nldataset{N}$. Here, $\mb{P}$ can be constructed as the convex hull of the image of $\mb{X} \times \mb{U}$ or $\nldataset{N}$ through $\psi$, where convexity is required by the analysis and synthesis tools we will use in Section~\ref{sec:mainresults}. %


 With \eqref{eq:computation_p}, the LPV embedding of \eqref{eq:NL_sys_velocity} is {formulated as}
\begin{subequations}\label{eq:NL_sys_LPV}
\begin{align}
    \deltax_{k+1} & = \bar{A}_\mr{v}(p_k)\deltax_k + \bar{B}_\mr{v}(p_k)\deltau_k, \label{eq:NL_sys_LPV:state}\\ 
    \deltay_k & = \deltax_k,
\end{align}
\end{subequations}
with $\bar{A}_\mr{v}(p_k)  = A_0+\sum_{i=1}^{\dnp}A_i p_{i,k}$ and $\bar{B}_\mr{v}(p_k)  = B_0+\sum_{i=1}^{\dnp}B_i p_{i,k}$. {To make \eqref{eq:NL_sys_LPV}  a linear surrogate representation of \eqref{eq:NL_sys_diff}, in the LPV framework, $p_k\in\mb{P}$ in \eqref{eq:NL_sys_LPV} is assumed to vary independently from $(\Delta x_k, \Delta u_k)$.} The resulting behavior of \eqref{eq:NL_sys_LPV} is defined as
\begin{multline*}
    \mf{B}_{\textsc{lpv}}\!=\! \{(\deltax,\deltau,\deltay)\in(\mb{X}\times\mb{U}\times\mb{Y})^\mb{Z} \mid \exists p \in \mb{P}^\mb{Z} \text{ s.t.}  \\\text{\eqref{eq:NL_sys_LPV} holds } \forall k\in\mathbb{Z}\}.
\end{multline*}
{The} assumption {of} the independent variation of $p_k$ {implies that} $\mf{B}_\Delta\subset\mf{B}_{\textsc{lpv}}$, {resulting in an embedding of the %
velocity behavior $\mf{B}_\Delta$ into a solution set of a linear representation. {While} the price for this linearity is payed in the conservatism %
of the resulting LPV representation,} %
linearity {in itself enables the derivation of} %
{a} data-driven representation concept {through the LPV extension of the Fundamental Lemma}. 
 \begin{remark}{The velocity-form is key to accomplish the LPV embedding, because \textit{(i)} \eqref{eq:NL_sys_velocity:state} naturally appears in an LPV form compared to the required non-unique factorization of $f$ and $h$ for the \emph{direct} LPV embedding of \eqref{eq:NL_sys} (as is used in, e.g., \cite{Verhoek2022_DDLPVstatefb_experiment}), and \textit{(ii)} ensuring (asymptotic) stability and dissipativity guarantees on \eqref{eq:NL_sys_LPV} results in \new{equilibrium independent} guarantees on \eqref{eq:NL_sys}, while this is not the case with a direct LPV embedding and LPV analysis of \eqref{eq:NL_sys}, see~\cite{Koelewijn2020_pitfalls} for further details.}
 \end{remark}


\vspace{\rmwhitebfssec}\subsection{Data-driven \new{closed-loop} velocity representations}\vspace{\rmwhiteafssec}

To make a data-driven synthesis for the velocity form and a subsequent realization of the controller for the original NL system \eqref{eq:NL_sys} possible, we require as a first step a data-driven representation of \eqref{eq:NL_sys_diff} in closed-loop with the to-be-designed controller. By exploiting the LPV embedding concept  \eqref{eq:NL_sys_LPV} of \eqref{eq:NL_sys}, we can derive such a closed-loop representation  based on \cite{Verhoek2022_DDLPVstatefb}
using $\nldataset{N+1}=\{u^\mr{d}_k, x^\mr{d}_k\}_{k=1}^{N+1}$, measured from %
\eqref{eq:NL_sys}. %

Based on $\mc{D}_{N+1}^{\textsc{nl}}$, we can construct the signals that constitute \eqref{eq:NL_sys_LPV}, %
{resulting in the} %
data-dictionary $\mc{D}_N^\Delta = \{\deltax^\mr{d}_k, p^\mr{d}_k, \deltau^\mr{d}_k \}_{k=2}^{N+1}$ {and} %
the data matrices
\begin{subequations}\label{eq:datamatrices}\deflen{savespaceeqeight}{-1.5mm}%
\begin{align}
	\hspace{\savespaceeqeight}U_\Delta   & = \begin{bmatrix} \deltau^\mr{d}_2 & \cdots  & \deltau^\mr{d}_{N} \end{bmatrix}\in\R^{\dnu\times N-1}, \\
	\hspace{\savespaceeqeight}\Up_\Delta & = \begin{bmatrix} p^\mr{d}_2\otimes \deltau^\mr{d}_2 & \cdots &  p^\mr{d}_{N}\otimes \deltau^\mr{d}_{N} \end{bmatrix}\in\R^{\dnu\dnp\times N-1},\hspace{\savespaceeqeight}\\
	\hspace{\savespaceeqeight}X_\Delta   & = \begin{bmatrix} \deltax^\mr{d}_2 & \cdots  & \deltax^\mr{d}_{N} \end{bmatrix}\in\R^{\dnx\times N-1},\\
	\hspace{\savespaceeqeight}\Xp_\Delta & = \begin{bmatrix} p^\mr{d}_2\otimes \deltax^\mr{d}_2 & \cdots &  p^\mr{d}_{N}\otimes \deltax^\mr{d}_{N} \end{bmatrix}\in\R^{\dnx\dnp\times N-1}, \hspace{\savespaceeqeight}\\
	\hspace{\savespaceeqeight}\Xsdelta   & = \begin{bmatrix} \deltax^\mr{d}_3 & \cdots  & \deltax^\mr{d}_{N+1} \end{bmatrix}\in\R^{\dnx\times N-1},
\end{align}
\end{subequations}
where `$\kron$' denotes the Kronecker product. \new{Moreover, for $\mG:=\begin{bmatrix} X_\Delta^\top & {\Xp_\Delta}^\top & U_\Delta^\top & {\Up_\Delta}^\top \end{bmatrix}^\top$, define $\ddataset{N}$ being \emph{persistently exciting} (PE) if $\mG$ has full row-rank, i.e., $\mr{rank}(\mG) = (1+\dnp)(\dnx+\dnu)$.}

Consider the  \emph{velocity controller} in terms of the LPV control law}
\begin{equation}\label{eq:fblaw}
    \deltau_k = K^\mr{v}(p_k)\deltax_k = \begin{bmatrix} K_0^\mr{v} & \bar{K}^\mr{v}\end{bmatrix}\begin{bmatrix} \deltax_k \\ p_k\otimes \deltax_k\end{bmatrix},
\end{equation}
with  $K^\mr{v}(p_k)=K^\mr{v}_0 + \sum_{i=1}^{\dnp}K^\mr{v}_i p_{i,k}$ and $\bar{K}^\mr{v}=\begin{bmatrix} K^\mr{v}_1 & \cdots & K^\mr{v}_{\dnp} \end{bmatrix}$. \new{Interconnection of this controller with the embedded velocity-form \eqref{eq:NL_sys_LPV}, can be formulated as a fully data-driven closed-loop representation. This is summarized in the following Corollary, derived from \cite[Thm.~1]{Verhoek2022_DDLPVstatefb}.}
\begin{corollary}\label{cor:closed-loop-data-based-general}
	Given a PE $\mc{D}_N^\Delta$ generated by \eqref{eq:NL_sys} based on which $\Xsdelta$ and $\mG$ are constructed. %
	Then, the interconnection of \eqref{eq:NL_sys_LPV}, i.e., \eqref{eq:NL_sys_velocity}, and {a given} $K^\mr{v}(p_k)$ under the feedback law \eqref{eq:fblaw} is represented equivalently as
	\begin{equation}\label{e:data-based-CLLPV-state-feedback-general}
		\deltax_{k+1} = \Xsdelta \mc{V} \begin{bmatrix} \deltax_k \\ p_k\otimes \deltax_k \\ p_k\otimes p_k\otimes \deltax_k \end{bmatrix},
	\end{equation} 
	where $\mc{V}\in\mathbb{R}^{N-1 \times \dnx(1+\dnp+\dnp^2) }$ is any matrix that satisfies
	\begin{equation}\label{e:consist-cond-general}
		\begin{bmatrix}
    		I_{\dnx} & 0& 0\\
    		0& I_{\dnp}\otimes I_{\dnx} & 0\\
    		K^\mr{v}_0 & \bar{K}^\mr{v} & 0\\
    		0 & I_{\dnp}\otimes K^\mr{v}_0 &  I_{\dnp}\otimes \bar{K}^\mr{v}
		\end{bmatrix} = \mG \mc{V}.
	\end{equation}
\end{corollary}
\begin{proof}
    The proof follows directly from \cite[Thm~1]{Verhoek2022_DDLPVstatefb}.
\end{proof}
We can now apply the direct data-driven LPV state-feedback controller synthesis methods from \cite{Verhoek2022_DDLPVstatefb} to \new{\emph{synthesize} the LPV velocity controller $K^\mr{v}(p_k)$}
for the system \eqref{eq:NL_sys_LPV}, i.e., a controller for the velocity-form of \eqref{eq:NL_sys}.
