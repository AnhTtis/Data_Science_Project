\vspace{\rmwhitebfsec}\section{Data-driven state-feedback control of NL systems with guarantees}\label{sec:mainresults}\vspace{\rmwhiteafsec}
\vspace{-0.1mm}\subsection{Data-driven velocity state-feedback synthesis}\vspace{-0.1mm}
Using the closed-loop data-driven representation of the velocity-form of \eqref{eq:NL_sys}, %
{we now formulate %
synthesis of $K^\mr{v}(p_k)$ with the objectives of  %
stabilization of \eqref{eq:NL_sys_velocity} and optimal performance in terms of the quadratic infinite-time horizon cost}
\begin{equation}\label{eq:infhorcost}
    J(\deltax,\deltau) = {\textstyle\sum_{k=1}^\infty} \deltax_k^\top Q \deltax_k + \deltau_k^\top R \deltau_k,
\end{equation}
where $Q,R\posdef0$ are user-defined matrices that encode the performance expectations. \chris{\emph{Velocity-dissipativity} of the closed-loop system (the velocity-form \eqref{eq:NL_sys_velocity} driven by the feedback law \eqref{eq:fblaw}) w.r.t. the supply function $\meu{S}_\mr{v}(\deltau_k, \deltax_k) = -(\deltax_k^\top Q \deltax_k + \deltau_k^\top R \deltau_k)$ implies that \eqref{eq:infhorcost} is finite. The following Corollary derived from \cite[Thm.~4]{Verhoek2022_DDLPVstatefb} gives a fully data-based algorithm for the synthesis of a velocity controller $K^\mr{v}(p_k)$ that ensures this and even minimizes \eqref{eq:infhorcost}.}
\begin{corollary}\label{cor:synthesis}
    Given a PE $\ddataset{N}$ generated by \eqref{eq:NL_sys}.
    \chris{Let $Z=Z^\top\in\mathbb{R}^{n_\mathrm{x}\times n_\mathrm{x}}$, with $Z\posdef0$, be the minimizer of $\sup_{p\in\mb{P}}\mr{trace}(Z)$, {such that there exist} multipliers $\mc{F}\in\mb{R}^{N-1\times\dnx(1+\dnp+\dnp^2)}$, $F_Q\in\mb{R}^{(N-1)(1+\dnp)\times \dnx (1+\dnp)}$, $\Xi\in\R^{4\dnx\dnp\times4\dnx\dnp}$, $Y_0\in\R^{\dnu\times\dnx}$, {and} $\bar{Y}\in\R^{\dnu\times\dnx\dnp}$}
    \begin{subequations}\label{eq:synthesis_conditions}
    satisfying
    \allowdisplaybreaks
	\begin{gather}
		\left[\begin{array}{c} * \\ \hline  * \end{array}\right]^\top \left[\begin{array}{c|c} \Xi & 0 \\ \hline  0 & W \end{array}\right]^\top	 \left[\begin{array}{c c} L_{11} & L_{12} \\ I & 0 \\ \hline  L_{21} & L_{22} \end{array}\right]\prec 0, \\
		\left[\begin{array}{c} * \\ \hline * \end{array}\right]^\top \underbrace{\begin{bsmallmatrix} \Xi_{11} & \Xi_{12} \\ \Xi_{12}^\top & \Xi_{22} \end{bsmallmatrix}}_{\Xi} \left[\begin{array}{c} I \\ \hline \mc{P} \end{array}\right]   \preceq 0, \quad \Xi_{22} \succ 0,\\
		\begin{bmatrix} Z & 0 & 0 \\ 0 & I_{n_\mr{p}}\otimes Z & 0 \\ Y_0 & \bar{Y} & 0 \\ 0 & I_{n_\mr{p}}\otimes Y_0 &  I_{n_\mr{p}}\otimes \bar{Y} \end{bmatrix} = \mG \mc{F}, \\
		\chris{\mc{F} \begin{bsmallmatrix} I_{\dnx}\\ p\kron I_{\dnx} \\ p\kron p\kron I_{\dnx} \end{bsmallmatrix} = \begin{bsmallmatrix} I_{N-1} \\ p\otimes I_{N-1} \end{bsmallmatrix}^{\!\top}\!\! F_Q \begin{bsmallmatrix} I_{n_\mr{x}} \\  p\otimes I_{n_\mr{x}} \end{bsmallmatrix},} \label{eq:syncond:equalFFQ}
	\end{gather}
    \end{subequations}
    for all $p\in\mb{P}$,
	\new{where %
	$\mc{P} = \diag(p)\otimes I_{2n_\mr{x}}$, and
    \begingroup\allowdisplaybreaks
    \begin{align}
        W & =\begin{bmatrix} Z_0 & F_Q^\top \overrightarrow{\mc{X}}_{\!\Delta}^\top  & \begin{bmatrix}  Q^{\frac{1}{2}}Z  & 0\end{bmatrix}^\top & \mc{Y}^\top R^{\frac{1}{2}} \\ \overrightarrow{\mc{X}}_{\!\Delta} F_Q & Z_0 & 0 & 0 \\ \begin{bmatrix}  Q^{\frac{1}{2}}Z  & 0\end{bmatrix} & 0 & I_{n_\mr{x}} & 0 \\ R^{\frac{1}{2}}\mc{Y} & 0 & 0 &  I_{n_\mr{u}} \end{bmatrix}, \notag\\
        Z_0 & = \mr{blkdiag}(Z, \ 0_{\dnx\dnp\times\dnx\dnp}),\quad\mc{Y} =[\,Y_0 \ \ \bar{Y}\,],  \notag\\
        \overrightarrow{\mc{X}}_{\!\Delta} & = \mr{blkdiag}(\Xsdelta, \ I_{\dnp}\kron\Xsdelta), \notag\\
        L_{11} & = 0_{2n_\mr{xp}\times 2n_\mr{xp}}, \hspace{3.88mm}\ L_{12}  =  \begin{bmatrix} 1_{\dnp}\otimes I_{2\dnx} & 0_{2n_\mr{xp}\times n_\mr{xu}} \end{bmatrix}, \notag\\
	    L_{21} & = \begin{bmatrix} 
	              0_{\dnx\times 2n_\mr{xp}}\\ 
	              I_{\dnp}\otimes \Gamma_1\\ 
	              0_{\dnx\times 2n_\mr{xp}}\\ 
	              I_{\dnp} \otimes \Gamma_2\\ 
	              0_{n_\mr{xu}\times 2n_\mr{xp}}
	           \end{bmatrix}, \ L_{22}  =%
	           \begin{bmatrix}
	              \Gamma_1 & 0 \\
	              1_{\dnp}\otimes 0_{\dnx\times 2n_\mr{x}} & 0\\
	              \Gamma_2 & 0 \\
	              1_{\dnp}\otimes 0_{\dnx\times 2n_\mr{x}} & 0 \\
	              0 & I_{n_\mr{xu}}
	           \end{bmatrix}, \notag\\ 
	    \Gamma_1 & = \begin{bmatrix} I_{\dnx} & 0 \end{bmatrix}, \hspace{8.9mm}\Gamma_2 = \begin{bmatrix} 0 & I_{\dnx} \end{bmatrix}, \label{eq:LFT-LQR}
    \end{align}%
    \endgroup%
    with $n_\mr{xp}=n_\mr{x}n_\mr{p}$, $n_\mr{xu}=n_\mr{x}+n_\mr{u}$.}
	Then, the state-feedback controller $K^\mr{v}(p_k)$ with $K^\mr{v}_0 = Y_0 Z^{-1}$, and $\bar{K}^\mr{v} = \bar{Y} (I_{n_\mr{p}}\otimes {Z} )^{-1}$ is a stabilizing controller for \eqref{eq:NL_sys_LPV}, and achieves the minimum of \eqref{eq:infhorcost} over all initial conditions $\deltax_1\in\mathbb{R}^{n_\mathrm{x}}$ and scheduling trajectories $p\in\mathbb{P}^{\mathbb{N}}$.
\end{corollary}
\begin{proof}
    See \cite[Thm.~4]{Verhoek2022_DDLPVstatefb}.
\end{proof}
\chris{Note that \eqref{eq:syncond:equalFFQ} can be easily satisfied by defining $\mc{F}=[\mc{F}_1 \,\mc{F}_2\,\mc{F}_3]$ in terms of a permutation of $F_Q=\begin{bsmallmatrix} F_{11} & F_{12} \\ F_{21} & F_{22} \end{bsmallmatrix}$, where $\mc{F}_1 = F_{11}$, $\mc{F}_2$ is constructed from the rows and columns of $F_{21}$ and $F_{12}$, respectively, and $\mc{F}_3$ is a permutation of $F_{22}$.}
By reformulation of \eqref{eq:synthesis_conditions} and assuming that $\mb{P}$ is compact, the synthesis algorithm of Corollary~\ref{cor:synthesis} corresponds to a \emph{semi-definite program} (SDP) with a finite set of \emph{linear matrix inequality} (LMI) constraints. The resulting %
controller $K^\mr{v}(p_k)$ provides stability and performance guarantees for the LPV surrogate form under all possible variations of $p$. This --through the embedding principle-- implies stability and performance in terms of Definition~\ref{def:velocity_stab},~\ref{def:velocity_diss} of the closed-loop velocity-form \eqref{eq:NL_sys_velocity} with {$K^\mr{v}(\psi(x_k, u_{k}, x_{k-1}, u_{k-1}))$} where $p_k$ {is substituted by} \eqref{eq:computation_p}. Hence, using \emph{only} the data-dictionary $\mc{D}_N^\Delta$ from the NL system \eqref{eq:NL_sys}, we synthesized a NL controller for the velocity-form, which corresponds to our contribution~C1. The problem that remains is to show that there exists a NL controller $K^\textsc{nl}$ for which $K^\mr{v}$ is its velocity-form, enabling to prove that that applying $K^\textsc{nl}$ on the unknown system \eqref{eq:NL_sys} will imply USAS and USD guarantees of the closed-loop operation.

\vspace{\rmwhitebfssec}\subsection{Realization of the NL controller}\vspace{\rmwhiteafssec}
{For the controller realization, we use the time-difference and summing operators $\difop$ and $\sumop$ on signals, such that $\sumop\deltax_k=x_k$, $\difop x_k = \deltax_k$ and $\difop(\sumop\deltax_k)=\deltax_k$. Note that these are the DT equivalents of the time-integration and differentiation operators in \emph{continuous-time} (CT). Hence, if we apply these to the closed-loop as depicted in Fig.~\ref{fig:realization}, we can define the NL controller {as}} %
\begin{figure}
    \centering
    \vspace{2mm}
    \includegraphics[width=0.8\linewidth]{figures/realization-alt}
    \vspace{-0mm}
    \vspace{-2mm}
    \caption{Realization of the controller.}\label{fig:realization}
    \vspace{-7mm}
\end{figure}
\begin{equation}\label{eq:primalcontroller}
    \hspace{-1mm}K^\textsc{NL}:\left\{\begin{aligned}
        \chi_{k+1} & = \begin{bmatrix} 0 & 0 \\ - K^\mr{v}(p_k)  & I \end{bmatrix} \chi_{k} + \begin{bmatrix} I \\ K^\mr{v}(p_k) \end{bmatrix} x_k, \\
        u_k & = \begin{bmatrix} - K^\mr{v}(p_k)  & I \end{bmatrix} \hspace{0.8mm}\chi_{k} + \hspace{1.8mm} K^\mr{v}(p_k) \hspace{2mm}x_k, \\
        p_k & = \psi(x_k, u_{k}, \chi_{k}),
    \end{aligned}\right.
\end{equation}
where $\chi_k = \begin{bmatrix} x_{k-1}^\top & u_{k-1}^\top \end{bmatrix}^\top$. {This is easily derived by noting that $u_k = \deltau_k+u_{k-1}$, i.e., }
\begin{subequations}\label{eq:controlrealization}
\allowdisplaybreaks
\begin{align}
    \deltau_k & = K^\mr{v}(p_k)\deltax_k, \\
    (u_{k}-u_{k-1}) & = K^\mr{v}(p_k) (x_{k}-x_{k-1}), \\
    u_{k} & = K^\mr{v}(p_k) (x_{k}-x_{k-1}) + u_{k-1}.\label{eq:controlrealization:c}
\end{align}
\end{subequations}
{Hence, the interconnection of $K^\mr{v}(p_k)$ with \eqref{eq:NL_sys_velocity} is in fact the velocity-form of the interconnection of \eqref{eq:primalcontroller} with \eqref{eq:NL_sys}.}
Note that to compute $u_k$ in the output equation of $K^\textsc{NL}$, $p_k$ is also dependent on $u_k$. This means that computation of $u_k$ requires the solution of a \emph{fixed point problem}, for which many reliable solvers exist, or one can use
$u_{k-1}$ instead of $u_k$ in the computation of $p_k$ as an approximative solution.

\vspace{\rmwhitebfssec}\subsection{Stability and performance guarantees}\label{ss:stabandperf}\vspace{\rmwhiteafssec}
With the realization of the controller for the original form of the NL system established, we are ready to present the main {result of the paper:} %
\begin{theorem}\label{thm:inducedstab}
    Given a PE $\mc{D}_{N}^{\Delta}$ from \eqref{eq:NL_sys}, based on which a stabilizing controller $K^\mr{v}$ is synthesized via Corollary~\ref{cor:synthesis}. Then, the interconnection of the realized controller $K^\textsc{nl}$ \eqref{eq:primalcontroller}  and the NL system \eqref{eq:NL_sys} is guaranteed to be USAS.
\end{theorem}
\begin{proof}
    With the synthesis of $K^\mr{v}$, we know that the velocity-form \eqref{eq:NL_sys_velocity} in closed-loop with $K^\mr{v}$ is asymptotically stable. {Realization of the controller $K^\textsc{nl}$ ensures that  its velocity-form is $K^\mr{v}$ and \eqref{eq:NL_sys} in closed-loop with $K^\textsc{nl}$ has a velocity-form that is the interconnection of \eqref{eq:NL_sys_velocity} with $K^\mr{v}$. Under these conditions, asymptotic stability of the velocity-interconnection implies USAS of the closed-loop interconnection of \eqref{eq:NL_sys} with $K^\textsc{nl}$ based on \cite[Thm.~8.3]{Koelewijn2023}.}
\end{proof}
\begin{conjecture}
    Given a PE $\mc{D}_{N}^{\Delta}$ from \eqref{eq:NL_sys} based on which a stabilizing controller $K^\mr{v}$ is synthesized via Corollary~\ref{cor:synthesis}. Then, the interconnection of the realized controller $K^\textsc{nl}$ \eqref{eq:primalcontroller} and the NL system \eqref{eq:NL_sys} is USD w.r.t. the supply function $\meu{S}_\mr{s}(u_k, u_\ast, x_k, x_\ast) = -(x_k-x_\ast)^\top Q (x_k-x_\ast) - (u_k-u_\ast)^\top R (u_k-u_\ast)$.
\end{conjecture}
We introduced the implication of performance as a conjecture, because the link between velocity-dissipativity and general USD has not been formally proven -- only under certain technical conditions, see \cite[Sec.~8.3]{Koelewijn2023}. {However, the analysis of USD through the velocity-form shares strong similarities with analysis of a stronger dissipativity notion called \emph{incremental dissipativity} \cite{Koelewijn2023}. %
Hence, there are strong indications that velocity-dissipativity w.r.t. a quadratic supply function implies USD w.r.t. a quadratic supply function}.

Finally, we want to note that the universal shifted controller {guarantees convergence} to an equilibrium point $(x_\ast,u_\ast)\in\ms{E}$. To ensure that the system is driven to a desired equilibrium point, {we can add integrators to {$K^\mr{v}$}, see \cite[Cor.~8.2]{Koelewijn2023} for the description of the approach and the example in Section \ref{sec:examples}}.

