\vspace{\rmwhitebfsec}\section{Simulation study}\label{sec:examples}\vspace{\rmwhiteafsec}
We demonstrate the applicability of our results on a simulator of an unbalanced disc system, for which we synthesize a universal shifted data-driven state-feedback controller and compare it with a data-driven state-feedback LPV controller that uses a direct LPV embedding of the NL system. For \TR{comparison,} %illustration, 
we also synthesize an LTI \TR{data-driven} controller. % \TR{using} the same settings. 
The \TR{\emph{continuous-time} (CT)} dynamics of the unbalanced disc system mimic those of an inverted pendulum and are thus described by the following ordinary differential equation
%
\begin{equation}\label{eq:unbalanced-disc}
    \ddot{\theta}(t)=-\tfrac{mgl}{J}\sin(\theta(t))-\tfrac{1}{\tau}\dot{\theta}(t)+\tfrac{K_\mr{m}}{\tau}u(t),
\end{equation}
%
where $\theta$ is the angular position of the disc in radians, $u$ is the input voltage to the system, which is its control input, and $m,g,l,J,\tau,K_\mr{m}$ are the physical parameters of the system that we take from \cite[Table~I]{Verhoek2022_DDLPVstatefb}. Discretizing the dynamics using a first-order Euler method and writing \TR{them} in the form of \eqref{eq:NL_sys} gives
%with $x_k = \begin{bmatrix} \theta_k & \dot\theta_k \end{bmatrix}^\top$ yields
\begin{subequations}\label{eq:DT-unbaldisc}
\begin{align}
    \hspace{-1mm}x_{1,k+1} & = x_{1,k} + T_\mr{s} x_{2,k},\\
    \hspace{-1mm}x_{2,k+1} & = (\tfrac{T_\mr{s}}{\tau}-1)x_{2,k} -\tfrac{T_\mr{s}mgl}{J}\sin(x_{1,k})+\tfrac{T_\mr{s}K_\mr{m}}{\tau}u_k,
\end{align}
where $x_k = \begin{bsmallmatrix} \theta_k & \dot\theta_k \end{bsmallmatrix}^\top$.
\end{subequations}
We choose the sampling-time as $T_\mr{s}=0.01$ [s], which gives a negligible discretization error through the Euler scheme. The control objective is to design a controller that tracks a reference for $\theta_k$ with zero steady-state error, which requires integrator action. We introduce the integrator behavior \chris{with the tuning parameter $\alpha$, see \cite[Corollary~8.2]{Koelewijn2023}. For the direct LPV design, we introduce integrator behavior by adding} an augmented state $x_{\mr{aug},k+1}=\alpha x_{\mr{aug},k} + \theta_{\mr{ref},k} - x_{1,k}$. Note that with the extra state, we require a larger data-dictionary for the construction of the direct data-driven LPV representation.

\begin{figure*}
\deflen{removewhitespaceforfig}{-5mm}
	\centering
	\begin{minipage}[t]{0.49\linewidth}
	   \vspace{2mm}
	   \includegraphics[scale=1, trim=0mm 0.95mm 0mm 0.7mm, clip]{figures/datadictionary_ref}
	   \vspace{\removewhitespaceforfig}
	   \vspace{-2mm}
	   \caption{Data-dictionary $\mc{D}_{N\!+\!1}^{\textsc{nl}}$ used for the NL and LPV control synthesis \TR{with} \chris{$N=8$. The \TR{extra} gray (\legendline{cgray}) \TR{data-points} are required  for the direct LPV representation, because of the added state for the \TR{integrator} behavior.}}
	   \label{fig:datadictionary}
	\end{minipage}\hfill
	\begin{minipage}[t]{0.49 \linewidth}
       \vspace{2mm}
	   \includegraphics[scale=1, trim=0mm 0.95mm 0mm 0.7mm, clip]{figures/resexample_ref}
	   \vspace{\removewhitespaceforfig}
	   \vspace{-2mm}
	   \caption{Response of the unbalanced disc with the universal shifted controller (\legendline{mblue}) and the LPV controller (\legendline{morange}) in closed-loop for a step reference (\legendline{black}). An LTI controller designed with the same specifications diverges directly (\legendline{myellow}).}% In case of reference tracking, the LPV controller ends up in a limitcycle.}
	   \label{fig:resexample}
	\end{minipage}\vspace{\removewhitespaceforfig}
\end{figure*}

The velocity-form of \eqref{eq:DT-unbaldisc}
%For the data-driven design of the nonlinear universal-shifted controller, we first need to calculate the velocity-form of \eqref{eq:DT-unbaldisc}, which 
can be computed analytically:
\begin{subequations}\label{eq:DT-unbaldisc:velocity}
\begin{align}
    \deltax_{k+1} & = A_\mr{v}(x_{1,k}, x_{1,k-1})\deltax_k + B_\mr{v}\deltau_k, \\ 
    \deltay_k & = \deltax_k,
\end{align}
\end{subequations}
where $B_\mr{v}= \begin{bmatrix} 0 & \tfrac{T_\mr{s}K_\mr{m}}{\tau} \end{bmatrix}^\top$ and 
\begin{equation*}
    A_\mr{v}(x_{1,k}, x_{1,k-1}) = \begin{bmatrix} 1 & T_\mr{s} \\ -\tfrac{T_\mr{s}mgl}{J}\sind(x_{1,k}, x_{1,k-1}) & 1-\tfrac{T_\mr{s}}{\tau} \end{bmatrix},
\end{equation*}
\TR{with} $\sind(a,b):=\tfrac{\sin(a)-\sin(b)}{a-b}$. For the data-driven design of the NL universal shifted controller, we choose\footnote{\TR{This basis is used} for simplicity and comparison purposes with the direct LPV design, but one could alternatively choose a polynomial basis.} $p_k := \psi(x_{1,k}, x_{1,k-1})=\sind(x_{1,k}, x_{1,k-1})$, which allows for an LPV embedding of the velocity-form \eqref{eq:DT-unbaldisc:velocity}. Note that $\lim_{x_{1,k}\rightleftarrows x_{1,k-1}}\psi(x_{1,k}, x_{1,k-1})$ exists and for all \TR{trajectories of \eqref{eq:DT-unbaldisc}} $\psi(x_{1,k}, x_{1,k-1})\in[-1, 1]$ \TR{that we take as} $\mb{P}$. For the \chris{direct data-driven} LPV design, we follow \cite{Verhoek2022_DDLPVstatefb} \TR{to formulate an LPV embedding of \eqref{eq:DT-unbaldisc}} where we choose $p_k = \tfrac{\sin(x_{1,k})}{x_{1,k}}$, \TR{which is well-defined for $x_{1,k}=0$}. 

We are now ready to construct the LPV data-driven representations and synthesize controllers for the velocity-form and the \TR{original} system. To construct well-posed data-driven representations for both approaches, while using the same data-set, we need $\mr{rank}(\mc{G}) = (1+\dnp)(\dnx+1+\dnu)=8$, i.e., we need $N\ge8$. The data-dictionary $\mc{D}_{N+1}^{\textsc{nl}}$ is obtained by applying \TR{white noise} $u^\mathrm{d}_k\sim\mc{N}(0,3)$ to \eqref{eq:DT-unbaldisc} \TR{under initial condition} $x^\mathrm{d}_1\sim\mc{U}(0,1)$. \TR{The resulting $\mc{D}_{N+1}^{\textsc{nl}}$} is shown in Fig.~\ref{fig:datadictionary}, where the additional data-points required for the augmented LPV representation are \TR{given in} gray.
\TR{Using} $\mc{D}_{N+1}^{\textsc{nl}}$, we construct the direct data-driven LPV representation as in \cite{Verhoek2022_DDLPVstatefb} and for the velocity-form we construct \eqref{eq:datamatrices} and verify that indeed $\mr{rank}(\mG) = (1+\dnp)(\dnx+\dnu)=6$, which yields a well-posed \TR{LPV velocity} data-driven representation of \eqref{eq:DT-unbaldisc:velocity}. 

Using the constructed representations, we design an LPV controller and a universal shifted controller (with integral action) using \cite[Theorem~4]{Verhoek2022_DDLPVstatefb} and Theorem~\ref{thm:synthesis}, respectively. The tuning parameters are chosen as $Q=I$, $R=2$, $\alpha=0.9$. 
Running the synthesis algorithms yield $K^\mr{v}$ and $K^\textsc{lpv}$, with 
\begin{align*}
 K^\mr{v}_0 &= \begin{bsmallmatrix} -9.85 \,&\, -1.08 \end{bsmallmatrix}, &&& K^\textsc{lpv}_0 &= \begin{bsmallmatrix} -9.11 \,&\, -0.99 \,&\, -8.88 \end{bsmallmatrix}, \\
 K^\mr{v}_1 &= \begin{bsmallmatrix} -1.26 \,&\, -0.005 \end{bsmallmatrix}, &&& K^\textsc{lpv}_1 &= \begin{bsmallmatrix} 0.12 \,&\, 0.03 \,&\, 15.2 \end{bsmallmatrix}. 
\end{align*}
We want to highlight here that  \eqref{eq:DT-unbaldisc} in closed-loop with  the universal shifted controller with $K^\mr{v}$ as above implies that the closed-loop system is USAS, %i.e., stable w.r.t. any forced equilibrium point in $\ms{E}$, 
while the LPV controller can only ensure stability %and performance 
of the origin of the NL closed-loop system. \TR{The} latter \TR{is} problematic \TR{for} reference tracking \cite{Koelewijn2020_pitfalls}, which we showcase in the following simulation study.

We simulate\footnote{See \texttt{youtu.be/NeOC9PBipMY} for an animation of the simulations.} \eqref{eq:DT-unbaldisc} in closed-loop with the LPV and universal shifted controller for the initial condition $x_{1}=\begin{bmatrix} \tfrac{\pi}{4} & 5 \end{bmatrix}^\top$. The system must follow a step-reference of magnitude $\tfrac{\pi}{2}$, which pushes the closed-loop away from the origin. The simulated responses of the closed-loops are plotted in Fig.~\ref{fig:resexample}, which shows that both controllers can regulate the system back to the origin. However, when the step reference is applied, only the universal shifted controller can drive the system to the reference, while the LPV controller ends up in a limit cycle. \chris{\TR{A} data-driven LTI state-feedback controller \TR{is also designed under} the same data and specifications. \TR{With} this controller, \TR{the system directly diverges, see} %however directly results in unstable behavior, as shown in 
Fig.~\ref{fig:resexample}.}

This example shows that we can synthesize state-feedback controllers for general NL systems of the form \eqref{eq:NL_sys} that are globally stabilizing and performing while using \emph{only} a given a basis set $\psi$ that is assumed to span the nonlinearities and measured data from the system.













