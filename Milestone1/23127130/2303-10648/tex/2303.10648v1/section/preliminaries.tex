\vspace{\rmwhitebfsec}\section{Data-based velocity representations}\label{sec:preliminaries}\vspace{\rmwhiteafsec}
%{As aforementioned, the velocity-form \eqref{eq:NL_sys_velocity} is similar to an LPV representation, which is a surrogate type of representation that has been used successfully in the model-based setting for analysis and control with global guarantees \cite{Koelewijn2023}. 

\TR{In order to realize our objective, we will first show that the velocity-form admits an LPV embedding and that we can obtain an LPV} %consider an LPV embedding of the velocity-form, with which we will define an LPV surrogate 
data-driven representation of the original velocity behavior, $\mf{B}_\Delta$, \TR{purely based on $\nldataset{N}$}.


%Before we derive our main results, we will first need to introduce some key concepts that are essential for building up the results in this paper.
\vspace{\rmwhitebfssec}\subsection{LPV embedding of the velocity-form}\vspace{\rmwhiteafssec}
%The analysis of \eqref{eq:NL_sys_velocity} is -- despite the linear relation with $(\deltax_k, \deltau_k, \deltay_k)$ -- still rather difficult and maybe even more complex than the analysis of \eqref{eq:NL_sys}. The LPV framework allows to make the analysis and synthesis of nonlinear systems computationally feasible and efficient by means of \emph{embedding} the nonlinear system in an LPV representation. 
To apply the embedding principle, we will need to start with the following assumption:
%
\begin{assumption}\label{ass:schedulingmap}
    We are given a set of basis functions $\psi_1, \dots, \psi_{\dnp}$ with $\psi: (\mb{X}  \times \mb{U})^2\to \mb{P}$, where $\mb{P}$ is a %compact, 
    convex subset of $\R^{\dnp}$, such that there exist  $A_0, \dots, A_{\dnp}\in\R^{\dnx\times\dnx}$ and $B_0, \dots, B_{\dnp}\in\R^{\dnx\times\dnu}$ for which
    \begin{subequations}\label{eq:decompab}
    \begin{align}
        A_\mr{v}(\xi_{k}, \xi_{k-1}) & = A_0 + {\textstyle\sum_{i=1}^{\dnp}}A_i\psi_i(\xi_{k}, \xi_{k-1}), \\
        B_\mr{v}(\xi_{k}, \xi_{k-1}) & = B_0 + {\textstyle\sum_{i=1}^{\dnp}}B_i\psi_i(\xi_{k}, \xi_{k-1}),
    \end{align}
    \end{subequations}
    with $\xi_k=\col(x_k, u_k)$. 
\end{assumption}
\begin{remark}
    Note that Assumption~\ref{ass:schedulingmap} is not restrictive, as taking a polynomial basis in $x_k,u_k,x_{k-1},u_{k-1}$ \emph{approximates} $\psi$ in the `true' decomposition \eqref{eq:decompab} arbitrarily well, because it forms a basis for the Taylor approximation of \TR{$f$}. \TR{Hence, explicit prior knowledge of $f$ is not required for an appropriate choice of $\psi$.} 
%   \chris{Therefore, we do not explicitly need to know $f(x,u)$.}
\end{remark}
%Note that the decomposition \eqref{eq:decompab} is always possible and can be chosen rather trivial. However, 
%We want to highlight that Assumption~\ref{ass:schedulingmap} can be rather easy to satisfy for simple nonlinear systems, while for more complex systems determining $\psi$ can be more involved. Note however that is this in fact \emph{always} possible \todo{hmm, do we then need an assumption?}. 
Based on Assumption~\ref{ass:schedulingmap}, we use the given set of functions to define a so-called \emph{scheduling variable}, a signal that collects all the variation, i.e., nonlinearities, of \eqref{eq:decompab} into an $\dnp$-dimensional signal
\begin{equation}\label{eq:computation_p}
   p_k=\psi(x_k, u_{k}, x_{k-1}, u_{k-1})\in\mb{P}. 
\end{equation}
Note that $p_k$ can be computed from measurements of $y_k=x_k$ and $u_k$ through $\psi$, hence this operation can be also accomplished for the available data set $\nldataset{N}$. With \eqref{eq:computation_p}, the LPV embedding of \eqref{eq:NL_sys_velocity} is \TR{formulated as}
%Using the definition of the scheduling variable, we can write \eqref{eq:NL_sys_velocity} \todo{\emph{equivalently}} as 
\begin{subequations}\label{eq:NL_sys_LPV}
\begin{align}
    \deltax_{k+1} & = \bar{A}_\mr{v}(p_k)\deltax_k + \bar{B}_\mr{v}(p_k)\deltau_k, \label{eq:NL_sys_LPV:state}\\ 
    \deltay_k & = \deltax_k,
\end{align}
%\begin{align}
%    \bar{A}_\mr{v}(p_k)  = A_0+\sum_{i=1}^{\dnp}A_i p_{k,i} , \quad
%    \bar{B}_\mr{v}(p_k)  = B_0+\sum_{i=1}^{\dnp}B_i p_{k,i} .
%\end{align}
\end{subequations}
with $\bar{A}_\mr{v}(p_k)  = A_0+\sum_{i=1}^{\dnp}A_i p_{i,k}$ and $\bar{B}_\mr{v}(p_k)  = B_0+\sum_{i=1}^{\dnp}B_i p_{i,k}$. \TR{To use \eqref{eq:NL_sys_LPV} as an attractive surrogate representation of \eqref{eq:NL_sys} \chris{in the LPV framework}, $p_k\in\mb{P}$ in \eqref{eq:NL_sys_LPV} is assumed to vary independently from $(\Delta x_k, \Delta u_k)$,} which results in the behavior of \eqref{eq:NL_sys_LPV} defined as
\begin{equation*}
    \mf{B}_{\textsc{lpv}}\!=\! \{(\deltax,\deltau,\deltay,p)\in\mf{B}_{\!\Delta}\times\mb{P}^\mb{Z} \mid \text{\eqref{eq:NL_sys_LPV} holds } \forall k\in\mathbb{Z}\}.
\end{equation*}
\TR{The} assumption \TR{of} the independent variation of $p_k$ \TR{implies that} $\mf{B}_\Delta\subset\mf{B}_{\textsc{lpv}}$, \TR{resulting in an embedding of the %original NL 
\chris{velocity} behavior $\mf{B}_\Delta$ into a solution set of a linear representation. However, the price for this linearity is payed in the conservatism %of the allowed variations $p$ 
of the resulting LPV representation.} % for linearity % of the representation 
 The \chris{linearity property} %\TR{former property} 
 is of key importance in the proposed data-driven representation concept, as we are now able to derive an LPV data-driven representation of the velocity-form {using only measurements of \eqref{eq:NL_sys}, where $p_k$ is constructed using the \emph{given} set of basis functions $\psi$ and the data.} 
 \begin{remark}\TR{The velocity-form is key to accomplish the LPV embedding, \chris{because} (i) \eqref{eq:NL_sys_velocity:state} naturally appears in an LPV form compared to the \chris{required} non-unique factorization of $f$ and $h$ for the \emph{direct} LPV embedding of \eqref{eq:NL_sys}, and (ii) ensuring (asymptotic) stability and dissipativity guarantees on \eqref{eq:NL_sys_LPV} results in global guarantees on \eqref{eq:NL_sys}, while this is not the case with a direct LPV embedding and LPV analysis of \eqref{eq:NL_sys}, see \cite{Koelewijn2020_pitfalls} for further details.}
 \end{remark}
%
%This system representation allows us to use the recently developed direct data-driven control theory for the \emph{Linear Parameter-Varying} framework to propose a direct data-driven state feedback control synthesis method for nonlinear systems of the form \eqref{eq:NL_sys}.

\vspace{\rmwhitebfssec}\subsection{Data-driven velocity representations}\vspace{\rmwhiteafssec}
%Based on the LPV embedding of the velocity-form \eqref{eq:NL_sys_LPV}, we propose to construct open-loop and closed-loop data-driven representations of its behavior. We construct these representations using only measurements of \eqref{eq:NL_sys}, where $p_k$ is constructed using the \emph{given} set of basis functions $\psi$.
\subsubsection{Open-loop representation} 
Based on \cite{Verhoek2022_DDLPVstatefb}, we now derive a data-driven representation of the velocity-form via the LPV embedding concept of \eqref{eq:NL_sys_LPV}, using only the data-dictionary  
%In order to formulate the data-driven representation, we need to make the following assumption on the measurability of the scheduling variable:
%\begin{assumption}\label{ass:schedulingmeasurability}
%    $p_k$ can either be:
%    \begin{itemize}
%        \item Directly measured, or
%        \item Constructed using current and past input-state measurements via a \emph{known} $\psi$.
%    \end{itemize}
%\end{assumption}
%
%Consider the data-set 
$\nldataset{N+1}=\{u^\mr{d}_k, x^\mr{d}_k\}_{k=1}^{N+1}$, measured from %the system 
\eqref{eq:NL_sys}. %For the purpose of this paper, we will only consider noise-free data. %\todo{Extensions to data-driven representation of NL systems with noise-affected data could emerge from combining the theory in this paper with existing (LTI) results \cite{breschi2022role, guo2021data}, however, this is an objective of future work on the topic.}
% So if this is removed, we have 6 pages... Sorry Valentina

\TR{Using} $\mc{D}_{N+1}^{\textsc{nl}}$, we construct the signals that constitute \eqref{eq:NL_sys_LPV} via \eqref{eq:deltasignals} and Assumption~\ref{ass:schedulingmap}, \TR{resulting in the} %Hence, we obtain the 
data-dictionary $\mc{D}_N^\Delta = \{\deltax^\mr{d}_k, p^\mr{d}_k, \deltau^\mr{d}_k \}_{k=2}^{N+1}$ \TR{and} %, based on which we construct 
the data matrices
\begin{subequations}\label{eq:datamatrices}\deflen{savespaceeqeight}{-1.5mm}% for saving space
\begin{align}
	\hspace{\savespaceeqeight}U_\Delta   & = \begin{bmatrix} \deltau^\mr{d}_2 & \cdots  & \deltau^\mr{d}_{N} \end{bmatrix}\in\R^{\dnu\times N-1}, \\
	\hspace{\savespaceeqeight}\Up_\Delta & = \begin{bmatrix} p^\mr{d}_2\otimes \deltau^\mr{d}_2 & \cdots &  p^\mr{d}_{N}\otimes \deltau^\mr{d}_{N} \end{bmatrix}\in\R^{\dnu\dnp\times N-1},\hspace{\savespaceeqeight}\\
	\hspace{\savespaceeqeight}X_\Delta   & = \begin{bmatrix} \deltax^\mr{d}_2 & \cdots  & \deltax^\mr{d}_{N} \end{bmatrix}\in\R^{\dnx\times N-1},\\
	\hspace{\savespaceeqeight}\Xp_\Delta & = \begin{bmatrix} p^\mr{d}_2\otimes \deltax^\mr{d}_2 & \cdots &  p^\mr{d}_{N}\otimes \deltax^\mr{d}_{N} \end{bmatrix}\in\R^{\dnx\dnp\times N-1}, \hspace{\savespaceeqeight}\\
	\hspace{\savespaceeqeight}\Xsdelta   & = \begin{bmatrix} \deltax^\mr{d}_3 & \cdots  & \deltax^\mr{d}_{N+1} \end{bmatrix}\in\R^{\dnx\times N-1},
\end{align}
where `$\kron$' denotes the Kronecker product.
\end{subequations}
Then, following \cite%[Sec.~III.A]
{Verhoek2022_DDLPVstatefb}, we can write \eqref{eq:NL_sys_LPV:state} w.r.t. the measured data as
%
\begin{equation}\label{eq:ddrep0}
    \Xsdelta = \mc{A}\begin{bmatrix} X_\Delta \\ \Xp_\Delta \end{bmatrix} + \mc{B}\begin{bmatrix} U_\Delta \\ \Up_\Delta \end{bmatrix} = \begin{bmatrix} \mc{A} & \mc{B} \end{bmatrix}\underbrace{\begin{bmatrix} X_\Delta \\ \Xp_\Delta \\ U_\Delta \\ \Up_\Delta \end{bmatrix}}_{\mG},
\end{equation}
%
where $\mc{A}=\begin{bmatrix} A_0 & \cdots & A_{\dnp} \end{bmatrix}$ and $\mc{B}=\begin{bmatrix} B_0 & \cdots & B_{\dnp} \end{bmatrix}$. With this, we see that $\begin{bmatrix} \mc{A} & \mc{B} \end{bmatrix}$ can be represented using only the data matrices \eqref{eq:datamatrices} in case $\mG$ has full row-rank, i.e., $\mr{rank}(\mG) = (1+\dnp)(\dnx+\dnu)$, which we will refer to as the \emph{persistence of excitation} (PE) condition on $\ddataset{N}$. %Under the PE condition, 
We can now formulate the \emph{data-driven} representation of \eqref{eq:NL_sys_LPV}.
\begin{definition}
    Given the PE data-dictionary $\ddataset{N}$. The  data-driven representation of $\mf{B}_{\textsc{lpv}}$, i.e., the data-driven representation of \eqref{eq:NL_sys_LPV} is given by 
    \begin{equation}\label{eq:ddrep}
    \deltax_{k+1} = \Xsdelta \mG^\dagger \begin{bmatrix} \deltax_k \\ p_k\otimes \deltax_k \\ \deltau_k \\ p_k\otimes \deltau_k \end{bmatrix},
\end{equation}
with $\Xsdelta$ and $\mG$ as in \eqref{eq:datamatrices} and \eqref{eq:ddrep0}, respectively.
\end{definition}
This representation allows us to fully characterize the LPV embedding of the velocity-form using only \TR{the data set $\mc{D}_{N+1}^{\textsc{nl}}$} and the set of basis functions $\psi$.
%Based on this, the velocity-form, represented by \eqref{eq:NL_sys_LPV}, can be fully characterized in terms of $\mc{D}_{N}^{\Delta}$ via the construction of the data matrices \eqref{eq:datamatrices} as

%Note that this representation is well-posed if the matrix $\mG$ has full row-rank, i.e., $\mr{rank}(\mG) = (1+\dnp)(\dnx+\dnu)$, which can be considered as the \emph{persistence of excitation} (PE) property of $\ddataset{N}$.

\subsubsection{Closed-loop representation}
In this paper, we first synthesize an \TR{LPV} state-feedback controller for the velocity-form, followed by a realization step \TR{compute a NL controller} for \eqref{eq:NL_sys}. Hence, we first require a data-driven characterization of the closed-loop velocity-form.
%To apply the direct data-driven synthesis methods on the data-driven representation, we first need to introduce the closed-loop data-driven representation in velocity-form. Hence, 
Consider the controller $K^\mr{v}(p_k)=K^\mr{v}_0 + \sum_{i=1}^{\dnp}K^\mr{v}_i p_{i,k}$, and control-law
\begin{equation}\label{eq:fblaw}
    \deltau_k = K^\mr{v}(p_k)\deltax_k = \mc{K}^\mr{v}\begin{bmatrix} \deltax_k \\ p_k\otimes \deltax_k\end{bmatrix},
\end{equation}
with $\mc{K}^\mr{v} = \begin{bmatrix} K_0^\mr{v} & \cdots & K_{\dnp}^\mr{v} \end{bmatrix}$. Interconnecting the controller with the embedded velocity-form \eqref{eq:NL_sys_LPV} allows to formulate a data-driven closed-loop representation using \cite[Thm.~1]{Verhoek2022_DDLPVstatefb}.
\begin{theorem}\label{th:closed-loop-data-based-general}
	Given a PE $\mc{D}_N^\Delta$ generated by \eqref{eq:NL_sys} based on which $\Xsdelta$ and $\mG$ are constructed according to \eqref{eq:datamatrices} and \eqref{eq:ddrep}. Then, the interconnection of \eqref{eq:NL_sys_LPV}, i.e., \eqref{eq:NL_sys_velocity}, and $K^\mr{v}(p_k)$ under the feedback law \eqref{eq:fblaw} is represented equivalently as
	%
	\begin{equation}\label{e:data-based-CLLPV-state-feedback-general}
		\deltax_{k+1} = \Xsdelta \mc{V} \begin{bmatrix} \deltax_k \\ p_k\otimes \deltax_k \\ p_k\otimes p_k\otimes \deltax_k \end{bmatrix},
	\end{equation} 
	%
	where $\mc{V}\in\mathbb{R}^{N-1 \times \dnx(1+\dnp+\dnp^2) }$ is any matrix that satisfies
	%
	\begin{equation}\label{e:consist-cond-general}
		\underbrace{\begin{bmatrix}
		I_{\dnx} & 0& 0\\
		0& I_{\dnp}\otimes I_{\dnx} & 0\\
		K^\mr{v}_0 & \bar{K}^\mr{v} & 0\\
		0 & I_{\dnp}\otimes K^\mr{v}_0 &  I_{\dnp}\otimes \bar{K}^\mr{v} 
		\end{bmatrix}}_{\mc{M}_\textsc{cl}}=
		\underbrace{\begin{bmatrix}
		X_\Delta\\ \Xp_\Delta \\ U_\Delta\\ \Up_\Delta
		\end{bmatrix}}_{\mG} \mc{V},
	\end{equation}
	%
	where $\bar{K}^\mr{v}=\begin{bmatrix} K^\mr{v}_1 & \cdots & K^\mr{v}_{\dnp} \end{bmatrix}$. % which we will refer to as the \emph{consistency condition}.
\end{theorem}
\begin{proof}
    The proof follows directly from \cite[Theorem~1]{Verhoek2022_DDLPVstatefb}.
\end{proof}
We can now apply the direct data-driven LPV state-feedback controller synthesis methods from \cite{Verhoek2022_DDLPVstatefb} to design an LPV state-feedback controller for the system \eqref{eq:NL_sys_LPV}, i.e., a controller for the velocity-form of \eqref{eq:NL_sys}.
