\vspace{\rmwhitebfsec}\section{Problem statement}\label{sec:problemstatement}\vspace{\rmwhiteafsec}
Consider a DT NL system\footnote{As we intent to establish the core concepts on data-driven control of general NL systems \TR{via the Fundamental Lemma}, in this work, we do not consider disturbance or noise signals in \eqref{eq:NL_sys}. The extensions towards noise-affected systems are objective of future \TR{research}, e.g., based on \cite{guo2021data, breschi2022role}.}, defined in terms of the state-space representation
%
\begin{equation}\label{eq:NL_sys}
    x_{k+1}  = f(x_k,u_k), \quad   y_k  = x_k, 
\end{equation}
%
where $x_k\in\mb{X}\subseteq \mb{R}^{n_\mr{x}}$ is the state, $u_k\in\mb{U} \subseteq \mb{R}^{n_\mr{u}}$ is the input and $y_k\in\mb{Y}=\mb{X}$ is the observed output at time moment $k\in\mathbb{Z}$. Here, $y_k$ is assumed to provide full state observation. %The sets 
$\mb{X}$ and $\mb{U}$ are considered to be open sets containing the origin, and $f: \mb{X} \times \mb{U} \to \mb{X}$ is assumed to be a $\mc{C}^1$ function. 

The \emph{behavior}, i.e., the set of all valid solution trajectories of \eqref{eq:NL_sys}, is defined as
\begin{equation}
    \hspace{-1mm}\mf{B}=\left\{ (x,u, y)\in ( \mb{X}\times\mb{U}\times\mb{Y})^{\mathbb{Z}}\mid\text{\eqref{eq:NL_sys} holds } \forall k\in\mathbb{Z}  \right\}.
\end{equation}
%Note that we trivially can exclude the output $y$ from $\mf{B}$. 
The set of all (forced) equilibrium points \TR{of} \eqref{eq:NL_sys}  \TR{are} %is defined as
\begin{equation*}
    \ms{E} = \{ (x_\ast,u_\ast, y_\ast)\in \mb{X}\times\mb{U}\times\mb{Y} \mid x_\ast  = f(x_\ast,u_\ast), \, y_\ast=x_\ast \}.
\end{equation*}
Furthermore, let $\mb{X}^\ast=\pi_{x_\ast}\ms{E}$, $\mb{U}^\ast=\pi_{u_\ast}\ms{E}$, $\mb{Y}^\ast=\pi_{y_\ast}\ms{E}$.

As highlighted in Section~\ref{sec:introduction}, analyzing the time-difference dynamics of \eqref{eq:NL_sys} allows for giving global guarantees on \eqref{eq:NL_sys}. \TR{For this purpose,} we introduce the {so-called} \emph{velocity-form} of \eqref{eq:NL_sys} \TR{that will be an important ingredient in our proposed method.}
%describes these time-difference dynamics \pk{\cite{Koelewijn2023}, on which we build our results.} 
For the increments
\begin{equation}\label{eq:deltasignals}
    \deltau_k = u_{k}-u_{k-1}, \ \deltax_k= x_{k}-x_{k-1}, \ \deltay_k= y_{k}-y_{k-1},
\end{equation}
%and $\deltay_k = \deltax_k$, 
we obtain the time-difference dynamics as
\begin{equation}\label{eq:NL_sys_diff}
    \deltax_{k+1}  = f(x_{k},u_{k})-f(x_{k-1},u_{k-1}), \quad \deltay_k  = \deltax_k.
\end{equation}
By the use of the \emph{Fundamental Theorem of Calculus} (FTC), e.g., {see} \cite[Lemma~C.1.1]{Koelewijn2023}, \cite{KoelewijnToth2021AutomaticEmbedding}, \eqref{eq:NL_sys_diff} can be rewritten in the equivalent \emph{velocity-form}:
\begin{subequations}\label{eq:NL_sys_velocity}
\begin{align}
    \deltax_{k+1} & = A_\mr{v}(\xi_{k}, \xi_{k-1})\deltax_k + B_\mr{v}(\xi_{k}, \xi_{k-1})\deltau_k, \label{eq:NL_sys_velocity:state}\\ 
    \deltay_k & = \deltax_k,
\end{align}
where $\xi_k=\col(x_k, u_k)$, and
\begin{align}
    %A_\mr{v}(\xi_{k+1}, \xi_k) = 
    A_\mr{v}(x_{k}, u_{k}, x_{k-1}, u_{k-1}) &= \int_0^1\tPartial{}{f}{x}\big(\bar{x}_k(\lambda), \bar{u}_k(\lambda)\big)\dif \lambda, \\
    %A_\mr{v}(\xi_{k+1}, \xi_k) = 
    B_\mr{v}(x_{k}, u_{k}, x_{k-1}, u_{k-1}) &= \int_0^1\tPartial{}{f}{u}\big(\bar{x}_k(\lambda), \bar{u}_k(\lambda)\big)\dif \lambda,
\end{align}
with $\bar{x}_k(\lambda)=x_{k-1} + \lambda(x_{k}-x_{k-1})$ and $\bar{u}_k(\lambda)=u_{k-1} + \lambda(u_{k}-u_{k-1})$.
\end{subequations}
The solutions of \eqref{eq:NL_sys_velocity} are collected in the \emph{velocity behavior} $\mf{B}_\Delta$, which is defined as
\begin{multline}
    \hspace{-1mm}\mf{B}_\Delta=\big\{ (\deltax,\deltau,\deltay)\in (\R^{\dnx}\times\R^{\dnu}\times\R^{\dny})^{\mathbb{Z}} \mid \text{the}\hspace{1mm} \\ \text{relations in \eqref{eq:deltasignals} hold }\forall k\in\mathbb{Z}, (x,u,y)\in\mf{B} \big\}.\hspace{-1mm}
\end{multline}
%
Analyzing stability and performance of the velocity-form \eqref{eq:NL_sys_velocity} by means of the concept of dissipativity yields universal guarantees on \eqref{eq:NL_sys}. Hence, consider the following definitions:
%
\begin{definition}\label{def:velocity_stab}
    The system \eqref{eq:NL_sys} is velocity-stable, if 
    \eqref{eq:NL_sys_velocity} is stable with $\deltau=0$, i.e., for each $\epsilon>0$ there exists a $\delta(\epsilon)$ such that $\|\deltax_{k_0}\|\le\delta(\epsilon) \Rightarrow \|\deltax_{k}\|\le\epsilon,\, \forall k\ge k_0$. \chris{It is asymptotic velocity-stable if it is velocity-stable and for $\deltau=0$ we have $\lim_{k\to\infty}\|\deltax_{k}\|=0$.}
    %
    %it is (asymptotically) stable at each $(x_\ast,u_\ast)\in\ms{E}$. 
\end{definition}
%
\begin{definition}\label{def:velocity_diss}
    The system \eqref{eq:NL_sys} is velocity-dissipative w.r.t. the supply function $\meu{S}_\mr{v}:\mb{R}^{\dnu}\times\mb{R}^{\dny}\to\mb{R}$, if there exists a storage function $\meu{V}_\mr{v}:\R^{\dnx}\to\mb{R}_0^+$ with $\meu{V}_\mr{v}\in\mc{C}_0$, $\meu{V}_\mr{v}\in\mc{Q}_{0}$, such that 
    \begin{equation}\label{eq:VD}
        \meu{V}_\mr{v}(\deltax_{k_1+1}) - \meu{V}_\mr{v}(\deltax_{k_0})  \chris{<}%\le 
        \sum_{k=k_0}^{k_1}\meu{S}_\mr{v}(\deltau_k, \deltay_k),
    \end{equation}
    for all $k_0, k_1\in\mb{Z}$, $k_0\le k_1$ and $(\deltax,\deltau,\deltay)\in\mf{B}_\Delta$, \chris{except when $\deltax_k=0$}. 
\end{definition}
%
{It is well-known that dissipativity implies asymptotic stability if the supply function is negative for trajectories \TR{with zero input}, i.e., $\meu{S}_\mr{v}(0, \deltay)<0$ for all $\deltay\in\R^{\dny}$ and $\deltay\neq0$ \cite{Koelewijn2023}.}

\TR{It has been shown in \cite{Koelewijn2023} that velocity-stability and velocity-dissipativity implies strong global notions in terms of}
%The link between concluding velocity-dissipativity and concluding global stability and performance of \eqref{eq:NL_sys} is in the relationship with 
\emph{universal shifted asymptotic stability} (US(A)S) and \emph{universal shifted dissipativity} (USD), which are defined as:
\begin{definition}\label{def:USS}
    The system \eqref{eq:NL_sys} is USS if it is stable w.r.t. all $(x_\ast,u_\ast, y_\ast)\in\ms{E}$, i.e., if for each $\epsilon>0$ there exists a $\delta(\epsilon)$ such that $\|x_{k_0}-x_\ast\|\le\delta(\epsilon) \Rightarrow \|x_{k}-x_\ast\|\le\epsilon$, $\forall k\ge k_0$. \chris{It is USAS if it is USS and for all $(x_\ast,u_\ast, y_\ast)\in\ms{E}$ we have $\lim_{k\to\infty}\|x_{k}-x_\ast\|=0$ with $(x,u,y)\in\mf{B}$ for which $u\equiv u_\ast$.} % and $\forall(x_\ast,u_\ast, y_\ast)\in\ms{E}$.
    %The system \eqref{eq:NL_sys} is universally shifted (asymptotically) stable, if it is (asymptotically) stable at each $(x_\ast,u_\ast)\in\ms{E}$. 
\end{definition}
\begin{definition}\label{def:USD}
    The system \eqref{eq:NL_sys} is USD w.r.t. the supply function $\meu{S}_\mr{s}:\mb{U}\times\mb{U}^*\times\mb{Y}\times\mb{Y}^\ast\to\mb{R}$, if there exists a storage function $\meu{V}_\mr{s}:\mb{X}\times\mb{U}^*\to\mb{R}_0^+$, \TR{which $\forall (x_\ast,u_\ast)\in\pk{\pi_{x_\ast,u_\ast}\ms{E}}$ satisfies} 
 $\meu{V}_\mr{s}(\cdot,u_\ast)\in\mc{C}_0$, $\meu{V}_\mr{s}(\cdot,u_\ast)\in\mc{Q}_{x_\ast}$%\chris{for every $(x_\ast,u_\ast)\in\pk{\pi_{x_\ast,u_\ast}\ms{E}}$} such that 
 \TR{, and}
    \begin{equation}\label{eq:USD}
        \meu{V}_\mr{s}({x_{k_1+1}}, u_\ast) -\meu{V}_\mr{s}(x_{k_0}, u_\ast) \chris{<}%\le 
        \sum_{k=k_0}^{k_1}\meu{S}_\mr{s}(u_k, u_\ast, y_k, y_\ast),
    \end{equation}
    for all $k_0, k_1\in\mb{Z}$, $k_0\le k_1$ and $(x,u,y)\in\mf{B}$, \chris{except when $x_k=x_\ast$}.
\end{definition}
%
%\begin{remark}
%    \chris{We can also define strict notions of velocity-dissipativity and USD, i.e., where \eqref{eq:VD} and \eqref{eq:USD} are strict inequalities. These imply in turn \emph{asymptotic} stability notions% analogous to Definitions~\ref{def:velocity_stab}~and~\ref{def:USS}
%    , i.e., asymptotic velocity-stability and \emph{universal shifted asymptotic stability} (USAS) in case the corresponding supply function is negative for trajectories with zero input.}
%%    Note that we can define \emph{asymptotic} velocity-stability and \emph{asymptotic} USS (USAS) analogous to Definitions~\ref{def:velocity_stab}~and~\ref{def:USS}.
%\end{remark}
The key-observation is that in \cite{Koelewijn2023} it is proven that velocity-dissipativity \emph{implies} USAS, i.e., asymptotic stability of \eqref{eq:NL_sys} w.r.t. any equilibrium point in $\ms{E}$. Furthermore, under certain assumptions\footnote{See \cite[Sec.~8.3.3]{Koelewijn2023} for the exact conditions and also the discussion in Section~\ref{ss:stabandperf}. %Although, analysis through the differential and incremental framework \cite{Verhoek2023_ConvexIncremental} suggests that the implication also must hold true without these conditions.
} velocity-dissipativity implies USD, i.e., performance of~\eqref{eq:NL_sys} w.r.t. any equilibrium point in $\ms{E}$. This inherent implication allows for the design and synthesis of  controllers for \eqref{eq:NL_sys} \TR{through the velocity\pk{-}form} that guarantee \emph{global} stability and performance of the NL system. In \cite{Koelewijn2023}, this has been \TR{accomplished} in the model-based setting using \TR{an} LPV surrogate form of \eqref{eq:NL_sys_velocity}. \TR{We aim to extend this result} to the data-based setting by solving the following problem:


%As highlighted in Section~\ref{sec:introduction}, we can establish \emph{global} stability and performance analysis of \eqref{eq:NL_sys} by means of the incremental dissipativity \cite{Verhoek2023_ConvexIncremental} and \emph{Universal Shifted Dissipativity} (USD) concept. For extension to the data-driven framework, we will consider USD, although the two concepts are closely related \cite{Koelewijn2023}. Hence, consider the following definitions:




%

%
%For USD analysis, we use the \emph{velocity-form} of \eqref{eq:NL_sys} by taking the difference of the dynamics in time, i.e., f




%Analogous to Definition~\ref{def:dissip}, we define the velocity dissipativity property of \eqref{eq:NL_sys} as in \cite{Koelewijn2023}:



% Stability for the velocity-form can be trivially extended from Definition~\ref{def:stab}. As shown in \cite{Koelewijn2023}, dissipativity analysis of \eqref{eq:NL_sys_velocity} in the model-based setting \emph{implies} USD of \eqref{eq:NL_sys} under certain conditions, similarly for the stability case. This inherent implication, allows for the design and synthesis of  controllers for \eqref{eq:NL_sys} that guarantee \emph{global} stability and performance of the nonlinear system. In this letter, we extend this to the direct data-driven setting, which allows for direct data-driven analysis and control of general nonlinear systems with global stability and performance guarantees.

\subsubsection*{Problem statement} 
Consider a system represented by \eqref{eq:NL_sys} \TR{from which} $N$ samples of input-state data \TR{have been obtained and} collected in the \emph{data-dictionary} $\nldataset{N}= \{u^\mr{d}_k, x^\mr{d}_k\}_{k=1}^N$. How to synthesize a state-feedback controller for \eqref{eq:NL_sys} \TR{purely} based on $\nldataset{N}$ that guarantees global stability and performance of the closed-loop system?



































