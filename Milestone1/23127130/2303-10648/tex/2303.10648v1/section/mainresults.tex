\vspace{\rmwhitebfsec}\section{Data-driven state-feedback control of nonlinear systems with guarantees}\label{sec:mainresults}\vspace{\rmwhiteafsec}
\vspace{\rmwhitebfssec}\subsection{Data-driven velocity state-feedback synthesis}\vspace{\rmwhiteafssec}
Using the closed-loop data-driven representation of the velocity-form of \eqref{eq:NL_sys}, %that we can construct using the matrices \eqref{eq:datamatrices} and $K(p_k)$ via Theorem~\ref{th:closed-loop-data-based-general}, 
\TR{we now formulate %the state-feedback controller 
synthesis of $K^\mr{v}(p_k)$ with the objectives of  %$K^\mr{v}(p_k)$ 
stabilization of \eqref{eq:NL_sys_velocity} and optimal performance in terms of the quadratic infinite-time horizon cost}
\begin{equation}\label{eq:infhorcost}
    J(\deltax,\deltau) = {\textstyle\sum_{k=1}^\infty} \deltax_k^\top Q \deltax_k + \deltau_k^\top R \deltau_k,
\end{equation}
where $Q\possemidef0,R\posdef0$ are user-defined matrices that encode the performance expectations. This objective corresponds to \emph{velocity-dissipativity} of the closed-loop system (the velocity-form \eqref{eq:NL_sys_velocity} driven by the feedback law \eqref{eq:fblaw}) w.r.t. the supply function $\meu{S}_\mr{v}(\deltau_k, \deltax_k) = -(\deltax_k^\top Q \deltax_k + \deltau_k^\top R \deltau_k)$.
Due to space restrictions, we give \TR{only} a short summary of the synthesis algorithm.
%, and refer to \cite[Theorems~3--6]{Verhoek2022_DDLPVstatefb} for the details and further extensions.
\begin{theorem}\label{thm:synthesis}
    Given a PE $\ddataset{N}$ generated by \eqref{eq:NL_sys}, let $Z\in \mathbb{R}^{n_\mathrm{x}\times n_\mathrm{x}}$ \TR{with $Z \succ 0$  be} the minimizer of $\sup_{p\in\mb{P}}\mr{trace}(Z)$, \TR{such that there exist} multipliers $\Xi\in\R^{4\dnx\dnp\times4\dnx\dnp}$, $Y_0\in\R^{\dnu\times\dnx}$, \TR{and} $\bar{Y}\in\R^{\dnu\times\dnx\dnp}$ \TR{satisfying} 
    \begin{subequations}\label{eq:synthesis_conditions}
    \allowdisplaybreaks
	\begin{gather}
		\left[\begin{array}{c} * \\ \hline  * \end{array}\right]^\top \left[\begin{array}{c|c} \Xi & 0 \\ \hline  0 & W \end{array}\right]^\top	 \left[\begin{array}{c c} L_{11} & L_{12} \\ I & 0 \\ \hline  L_{21} & L_{22} \end{array}\right]\prec 0, \\
		\left[\begin{array}{c} * \\ \hline * \end{array}\right]^\top \underbrace{\begin{bmatrix} \Xi_{11} & \Xi_{12} \\ \Xi_{12}^\top & \Xi_{22} \end{bmatrix}}_{\Xi} \left[\begin{array}{c} I \\ \hline \mc{P} \end{array}\right]   \preceq 0, \quad \Xi_{22} \succ 0,\\
		\begin{bmatrix} Z & 0 & 0 \\ 0 & I_{n_\mr{p}}\otimes Z & 0 \\ Y_0 & \bar{Y} & 0 \\ 0 & I_{n_\mr{p}}\otimes Y_0 &  I_{n_\mr{p}}\otimes \bar{Y} \end{bmatrix} = \mG \mc{F},
	\end{gather}
	for all $p\in\mb{P}$, where\footnote{See \extver{\cite[Thm.~4]{Verhoek2022_DDLPVstatefb} or the extended version of this paper on arXiv}{Appendix~\ref{appendix}} for a more detailed specification of these matrices.} $L_{ij}$ are selection matrices containing 1's and 0's; $W$ is a matrix dependent on $Q,R,Z,\Xsdelta,$ and $\mc{V}$; $\mc{F}$ is a matrix dependent on $\mc{V},Z$; and $\mc{P}=\mr{diag}(\TR{p})\kron I_{2\dnx}$. 
	\end{subequations}
	Then, the state-feedback controller $K^\mr{v}(p_k)$ with $K^\mr{v}_0 = Y_0 Z^{-1}$, \TR{and} $\bar{K}^\mr{v} = \bar{Y} (I_{n_\mr{p}}\otimes {Z} )^{-1}$ is a stabilizing controller for \TR{\eqref{eq:NL_sys_LPV}}, and achieves the minimum of \eqref{eq:infhorcost} over all \TR{initial conditions $\deltax_1\in\mathbb{R}^{n_\mathrm{x}}$ and scheduling trajectories $p\in\mathbb{P}^{\mathbb{N}}$}.
\end{theorem}
\begin{proof}
    See \cite[Theorem~4]{Verhoek2022_DDLPVstatefb}.
\end{proof}
%
%Although it might not directly evident from the shortened synthesis algorithm, 
{By reformulation of \eqref{eq:synthesis_conditions} and assuming that $\mb{P}$ is compact, the synthesis algorithm of Theorem~\ref{thm:synthesis} corresponds to a Semi-Definite Program (SDP) with a finite set of \emph{Linear Matrix Inequality} (LMI) constraints. The resulting %\emph{nonlinear} 
controller $K^\mr{v}(p_k)$ provides stability and performance guarantees for the LPV surrogate form under all possible variations of $p$. This -- through the embedding principle -- implies stability and performance in terms of Definition~\ref{def:velocity_stab},~\ref{def:velocity_diss} of the closed-loop velocity-form \eqref{eq:NL_sys_velocity} with \TR{$K^\mr{v}(\psi(x_k, u_{k}, x_{k-1}, u_{k-1}))$} where $p_k$ \TR{is substituted by} \eqref{eq:computation_p}. Hence, using \emph{only} the data-dictionary $\mc{D}_N^\Delta$ from the NL system \eqref{eq:NL_sys}, we synthesized a NL controller for the velocity-form. The problem that remains is to show that there exists a NL controller $K^\textsc{nl}$ for which $K^\mr{v}$ is its velocity-form. In that case, we can show that \TR{applying $K^\textsc{nl}$ on the unknown system \eqref{eq:NL_sys} will imply US\chris{A}S and USD guarantees of the closed-loop operation.}}
%velocity-stability/velocity-dissipativity of the closed-loop will imply USS and USD of \eqref{eq:NL_sys} in closed-loop with $K^\textsc{nl}$.}
%Hence, once we realize the obtained controller on the so-called \emph{primal} system, i.e., \eqref{eq:NL_sys}, we derived a methodology for data-driven state-feedback control of general nonlinear systems.

\vspace{\rmwhitebfssec}\subsection{Realization of the nonlinear controller}\vspace{\rmwhiteafssec}
{For the controller realization, we use the time-difference and summing operators $\difop$ and $\sumop$ on signals, such that $\sumop\deltax_k=x_k$, $\difop x_k = \deltax_k$ and $\difop(\sumop\deltax_k)=\deltax_k$. Note that these are the DT equivalents of the time-integration and differentiation operators in CT. Hence, if we apply these to the closed-loop as depicted in Fig.~\ref{fig:realization}, we can define the nonlinear controller \TR{as}} %highlighted in gray,
\begin{figure}
    \centering
    \vspace{2mm}
    \includegraphics[width=0.8\linewidth]{figures/realization-alt}
    \vspace{-0mm}
    \vspace{-2mm}
    \caption{Realization of the controller.}\label{fig:realization}
    \vspace{-7mm}
\end{figure}
%which yields}
\begin{equation}\label{eq:primalcontroller}
    \hspace{-1mm}K^\textsc{NL}:\left\{\begin{aligned}
        \chi_{k+1} & = \begin{bmatrix} 0 & 0 \\ - K^\mr{v}(p_k)  & I \end{bmatrix} \chi_{k} + \begin{bmatrix} I \\ K^\mr{v}(p_k) \end{bmatrix} x_k, \\
        u_k & = \begin{bmatrix} - K^\mr{v}(p_k)  & I \end{bmatrix} \hspace{0.8mm}\chi_{k} + \hspace{1.8mm} K^\mr{v}(p_k) \hspace{2mm}x_k, \\
        p_k & = \psi(x_k, u_{k}, \chi_{k}),
    \end{aligned}\right.
\end{equation}
where $\chi_k = \begin{bmatrix} x_{k-1}^\top & u_{k-1}^\top \end{bmatrix}^\top$. {This is easily derived by noting that $u_k = \deltau_k+u_{k-1}$, i.e., }
\begin{subequations}\label{eq:controlrealization}
\allowdisplaybreaks
\begin{align}
    \deltau_k & = K^\mr{v}(p_k)\deltax_k, \\
    (u_{k}-u_{k-1}) & = K^\mr{v}(p_k) (x_{k}-x_{k-1}), \\
    u_{k} & = K^\mr{v}(p_k) (x_{k}-x_{k-1}) + u_{k-1}.\label{eq:controlrealization:c}
\end{align}
\end{subequations}
%From the synthesis procedure we obtained a controller $K^\mr{v}(p_k)$ for the velocity-form of \eqref{eq:NL_sys}, i.e., \eqref{eq:NL_sys_velocity}, while we will need to implement it on the \emph{primal} form, i.e., \eqref{eq:NL_sys}. Therefore, we now present a realization step to obtain the \emph{universal shifted controller} that can be implemented on \eqref{eq:NL_sys}. Writing out \eqref{eq:fblaw} gives 
{Hence, the interconnection of $K^\mr{v}(p_k)$ with \eqref{eq:NL_sys_velocity} is in fact the velocity-form of the interconnection of \eqref{eq:primalcontroller} with \eqref{eq:NL_sys}.}
%Hence, for implementation on \eqref{eq:NL_sys}, we need past input and state measurements, i.e., we obtain a \emph{dynamic} state-feedback controller. Therefore, we introduce 
%\begin{equation}
%    \chi_k = \begin{bmatrix} x_{k} \\ u_{k} \end{bmatrix}
%\end{equation}
%as the state of the universal shifted state-feedback controller. Then the NL universal shifted controller realization is given by
Note that $K^\textsc{NL}$ must satisfy a well-posedness condition in the output equation, due to the dependency of $p_k$ on $u_k$. This can either be enforced within the given set of $\psi$ functions, solved as an optimization problem, e.g., $\min \|u_k\|$ subject to \eqref{eq:controlrealization:c}, {or using $u_{k-1}$ for $u_k$ as an approximative solution}.

\vspace{\rmwhitebfssec}\subsection{Stability and performance guarantees}\label{ss:stabandperf}\vspace{\rmwhiteafssec}
With the realization of the controller for the original form of the NL system established, we are ready to present the main \TR{result of the paper:} %conclusion on the results presented in this paper}.
\begin{theorem}\label{thm:inducedstab}
    Given a PE $\mc{D}_{N}^{\Delta}$ from \eqref{eq:NL_sys}, with which a stabilizing controller $K^\mr{v}$ is synthesized via Theorem~\ref{thm:synthesis}. Then, the interconnection of the realized controller $K^\textsc{nl}$ \eqref{eq:primalcontroller}  and the NL system \eqref{eq:NL_sys} is USAS.
\end{theorem}
\begin{proof}
    With the synthesis of $K^\mr{v}$, we know that the velocity-form \eqref{eq:NL_sys_velocity} in closed-loop with $K^\mr{v}$ is asymptotically stable. \TR{Realization of the controller $K^\textsc{nl}$ ensures that (i) its velocity-form is $K^\mr{v}$ and (ii) \eqref{eq:NL_sys} in closed-loop with $K^\textsc{nl}$ has a velocity-form that is the interconnection of \eqref{eq:NL_sys_velocity} with $K^\mr{v}$. Based on \cite[Theorem~8.3]{Koelewijn2023}, asymptotic stability of the velocity-form implies USAS of the original system, i.e., the closed-loop interconnection of \eqref{eq:NL_sys} with $K^\textsc{nl}$.}
    %Hence, by means of the realization of the controller $K^\textsc{nl}$, i.e., \eqref{eq:primalcontroller}, in closed-loop with \eqref{eq:NL_sys}, whose velocity-form is equal to \eqref{eq:NL_sys_velocity} in closed-loop with $K^\mr{v}$, we see that the synthesized controller yields the closed-loop USAS.
\end{proof}
\begin{conjecture}
    Given a PE $\mc{D}_{N}^{\Delta}$ from \eqref{eq:NL_sys} with which a stabilizing controller $K^\mr{v}$ is synthesized via Theorem~\ref{thm:synthesis}. Then, the interconnection of the realized controller $K^\textsc{nl}$ \eqref{eq:primalcontroller} and the NL system \eqref{eq:NL_sys} is USD w.r.t. the supply function $\meu{S}_\mr{s}(u_k, u_\ast, x_k, x_\ast) = -(x_k-x_\ast)^\top Q (x_k-x_\ast) - (u_k-u_\ast)^\top R (u_k-u_\ast)$.
\end{conjecture}
We introduced the implication of performance as a conjecture, because the link between velocity-dissipativity and USD has not been formally proven -- only under certain technical conditions, see \cite[Sec.~8.3]{Koelewijn2023}. {However, the analysis of USD through the velocity-form shares strong similarities with analysis of a stronger dissipativity notion called \emph{incremental dissipativity} \cite{Koelewijn2023}. %as the verification conditions for differential and incremental dissipativity are equivalent to those of velocity dissipativity, and differential and incremental dissipativity imply USD. 
Hence, there are strong indications that velocity-dissipativity w.r.t. a quadratic supply function implies USD w.r.t. a quadratic supply function}.

Finally, we want to note that the universal shifted controller {guarantees convergence} to an equilibrium point $(x_\ast,u_\ast)\in\ms{E}$. To ensure that the system is driven to a desired equilibrium point, {we can add integrators to \TR{$K^\mr{v}$} that can be tuned with a parameter $\alpha$, see e.g., \cite[Corollary~8.2]{Koelewijn2023} and the example in Section \ref{sec:examples}}.

%Hence, via the velocity-form of \eqref{eq:NL_sys}, we can design controllers purely based on data that \emph{guarantee} global stability and performance of the NL system in closed-loop with the designed controller.