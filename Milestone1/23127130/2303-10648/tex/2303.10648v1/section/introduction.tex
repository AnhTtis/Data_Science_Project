\vspace{\rmwhitebfsec}\section{Introduction}\label{sec:introduction}\vspace{\rmwhiteafsec}
Accurate system models allow us to give guarantees on stability and performance of the system. As systems are getting increasingly more complex, exhibit more \emph{nonlinear} (NL) behavior, and their performance requirements are ever-increasing, it is getting more difficult to obtain an accurate model  that can be used for control and \TR{provide performance} guarantees \TR{for the overall model-based control design toolchain}. 
%Hence, in the creation of a model that is suitable for control (usually a finite, low-order dynamic model), we do not want to loose information or introduce model mismatch that can lead to loss of the guarantees. Robust control techniques take this mismatch into account to ensure these guarantees, often at the price of performance loss. Moreover, these methods are mainly restricted to relatively low-order, \emph{Linear Time-Invariant} (LTI) models. As systems are becoming more complex and  (NL), this issue is becoming more pressing in control design.
%Eliminating the need to have an accurate mathematical model of a system available in order to design a controller for it, is an attractive wish in a world that is getting increasingly more complex. Especially the ever-growing demand in terms of higher accuracy, throughput and efficiency make that the accuracy of the model is becoming a bottleneck in the controller design. 
For this reason, data-driven control methodologies have been developed to design controllers \emph{directly} from data, which eliminates the need of a model. The Fundamental Lemma for LTI systems \cite{WillemsRapisardaMarkovskyMoor2005} is one of the key results in this field and allows for direct data-driven analysis and control with stability and performance guarantees, see \cite{MarkovskyDorfler2021Review} for an overview. However, analogous results for general NL systems is  missing with some exceptions for classes of (non-)linear systems using, e.g., %\emph{linear parameter-varying} (LPV) embeddings \cite{Verhoek2022_DDLPVstatefb_experiment},
 nonlinearity cancellations \cite{de2023learning} online linearizations \cite{berberich2022linear} or polynomial bases \cite{markovsky2021data}. Compared to these existing NL data-driven methodologies, we propose a novel concept based on the \emph{velocity-form} of a NL system that makes it possible to extend the Fundamental Lemma to a \emph{general} class of NL systems using some mild assumptions.

The velocity-form describes in \emph{discrete-time} (DT) the time-difference (in continuous-time %(CT)
 \TR{the time-differentiated})  dynamics of a NL system. Two key aspects of the velocity-form allow for the results in this paper; 1) the velocity-form admits a direct  \TR{\emph{linear parameter-varying} (LPV)} form, which allows for convex analysis and synthesis, and 2) stability and performance of the velocity-form imply \emph{universal shifted} stability and performance of the original NL system \cite{Koelewijn2023}, which is stability and performance w.r.t. any equilibrium point of the NL system. Recently, the velocity-form is successfully used to provide model-based output feedback control design for NL systems with global guarantees, i.e., guarantees on universal shifted stability and performance, via the LPV framework \cite{Koelewijn2023}. In this \TR{paper}, we develop a data-driven extension of this \TR{result} by the \TR{help} of the LPV Fundamental Lemma \cite{VerhoekTothHaesaertKoch2021}. We show that the design of a data-driven state-feedback controller for the velocity-form can be realized on the original NL system. With this realization step, the guarantees for the velocity-form imply global \TR{universal shifted} guarantees of the closed-loop NL system.
%
%A key observation that allows for the consideration of general NL systems, is that analysis and control of NL systems w.r.t. any arbitrary equilibrium point provides global stability and performance guarantees of the (closed-loop) NL system \cite{Verhoek2023_ConvexIncremental, Koelewijn2023, simpson2018equilibrium}. Moreover, the `transformed' analysis can be embedded as an LPV analysis/control problem, allowing for convex analysis and controller design with global stability and performance guarantees for general NL systems \cite{Koelewijn2023, Verhoek2023_ConvexIncremental} . This concept of `shifted' analysis is one link in a chain of implications, and is in fact implied by so-called \emph{velocity dissipativity}, where the velocity-form of the NL system is analyzed. This concept can directly be carried over to the data-driven domain, as we will show as our main contribution in this work for the state-feedback case, which allows for direct data-driven analysis of NL systems with global stability and performance guarantees.

In Section~\ref{sec:problemstatement}\TR{,} {we} formalize the NL data-driven control problem that we intend to solve. Section~\ref{sec:preliminaries} introduces the data-based representation of the velocity-form of the NL system using the LPV embedding and the Fundamental Lemma. Section~\ref{sec:mainresults} uses the data-driven representation to synthesize a state-feedback controller for the velocity-form, which by realization to a NL state-feedback law provides universal shifted guarantees. Section~\ref{sec:examples} demonstrates the applicability of the results in a simulation example based on an unbalanced disc system, \TR{while} the conclusions on the achieved results are given in Section~\ref{sec:conclusion}. 

\subsubsection*{Notation}
The set of integers is denoted by $\Z$, while the set of real numbers is denoted by $\R$. Moreover, $\R_0^+=[0,\infty)\subset\R$. A function $f:\R^{p}\to\R^q$ is in $\mc{C}^n$ if it is $n$-times continuously differentiable, while $f:\R^p\to\R$ belongs to the class $\mc{Q}_{x_\ast}$ if it is positive definite and decrescent w.r.t. $x_\ast\in\R^p$ (see \cite[Def.~3.31]{SchererWeiland2021}). %For a discrete-time signal $w:\Z\to\R^{\dnw}$, we denote its value at sample $k$ by $w_k\in\R^\dnw$, and the value of its $i$\tss{th} element at sample $k$ by $w_{i,k}\in\R$. 