\section{Introduction}
Graph spanners are sparse subgraphs that approximately preserve pairwise shortest-path distances. Let $G = (V, E)$ be an undirected unweighted graph on $n$ vertices and let $f:\mathbb{R}_+\rightarrow\mathbb{R}_+$ be a function. We say that a subgraph $H\subseteq G$ is a spanner with \emph{stretch function} $f$ iff for every pair $s, t\in V$, $\dist_H(s, t)\leq f(\dist_G(s, t))$. The research on spanners focuses on the optimal trade-offs between the stretch function $f$ and the sparsity (the number of edges) of the spanner $H$.

One extreme case is that we allow $f(d)$ to be significantly greater than $d$, and such spanners are known as \emph{multiplicative} spanners. It was shown that for every integer $k\ge 1$, there always exists a subgraph $H$ with $O(n^{1+1/k})$ edges and stretch function $f(d) = (2k-1)d$ \cite{althofer1993sparse}. Furthermore, this sparsity bound is tight under the Girth Conjecture of Erd{\"o}s \cite{erdos1963extremal}.

Another extreme case is that we restrict $f$ to be very close to $d$. In particular, $f(d) = d + O(1)$. There has been a line of previous work studying the sparsity of spanners with such stretch functions. When $f(d) = d+2$, it was shown that graph $G$ always has a spanner with $O(n^{3/2})$ edges \cite{aingworth1999fast}; when $f(d) = d+4$, a construction of spanner with $\tilde O(n^{7/5})$ edges was proposed in \cite{chechik2013new}; when $f(d) = d+6$, spanners with $O(n^{4/3})$ edges were known to exist by \cite{baswana2010additive}. For the lower bound side, in a recent breakthrough \cite{abboud20174}, it was proved that, for any constant $\epsilon>0$, there are graphs such that any spanner with $O(n^{4/3-\eps})$ edges has stretch $n^{\Omega(1)}$. Hence, we already have an almost complete understanding of the spanner sparsity when $f(d) = d + O(1)$.

Besides the two extreme cases mentioned above, much less is known when $f$ lies in intermediate regimes. Two notable regimes studied in previous works are the \emph{sublinear additive} regime where $f(d) = d + o(d)$, and the \emph{additive} regime where $f(d) = d + o(n)$.

As for sublinear additive spanners, Thorup and Zwick  \cite{thorup2006spanners} were the first to design a nontrivial construction of sublinear additive spanners when $f(d) = d + O(d^{1-1/k})$ where $k\geq 2$ is a constant integer, and the number of edges in the spanner is bounded by $O(n^{1 + 1/k})$. This sparsity bound was later improved to $O\bigg(n^{1+\frac{(3/4)^{k-2}}{7-2\cdot(3/4)^{k-2}}}\bigg)$ in \cite{pettie2009low}. The sparsity bound for the special case where $k=2$ was subsequently improved to $\tilde O(n^{20/17})$ by \cite{chechik2013new}. These algorithms also work for a non-constant $k$, but here we only focus on the case where $k$ is a constant, as we are mainly interested in the stretch/size dependency on $d$ and $n$. On the lower bound side, Abboud, Bodwin, and Pettie \cite{abboud2018hierarchy} proved that there exists hard instances where any spanner of stretch $f(d) = d + O(d^{1-1/k})$ must contain at least $\Omega\bigg(n^{1+\frac{1}{2^{k+1}-1}-o(1)}\bigg)$ edges for each constant $k\ge 2$. Thus, there still exists a large gap between sparsity upper and lower bounds for sublinear additive spanners.

For the additive regime where $f(d) = d + o(n)$, the tail term $f(d) - d$ is usually called the \emph{additive stretch}. A natural question is to study the best additive stretch that can be achieved by spanners with $\tilde{O}(n)$ edges. The first nontrivial construction was given by \cite{pettie2009low} with an additive stretch of $O(n^{9/16})$, which was improved subsequently by \cite{bodwin2015very, bodwin2016better} to $O(n^{3/7+\eps})$ for any constant $\epsilon>0$. On the negative side, the first stretch lower bound of $\Omega(n^{1/22})$ was proved in \cite{abboud20174}, and later on raised to $\Omega(n^{1/7})$ by a sequence of works \cite{huang2021lower,lu2022better,bodwin2022new}.

\subsection{Our results}
As our primary result, we construct sublinear additive spanners that almost match the %unconditional 
lower bound from \cite{abboud2018hierarchy}.

\begin{theorem}\label{sublinear}
For any undirected unweighted graph $G$ on $n$ vertices, any constant $\delta>0$ and any integer $k\ge 2$, there is a sublinear spanner $H\subseteq G$ with stretch function $f(d)=d+O_{\delta,k}(d^{1-1/k})$ and  $O\bigg(n^{1+\frac{1+\delta}{2^{k+1}-1}}\bigg)$ edges\footnote{$O_{\delta, k}(\cdot)$ hides factors only dependent on constants $\delta, k$.}.
\end{theorem}


%In other words, the correct constant $c$ for exponential base is $2$. Our bound significantly improves upon the previous bounds \cite{pettie2009low,chechik2013new}, and almost matches the lower bound proved in \cite{abboud2018hierarchy}\footnote{We note that our upper bound does not contradict the lower bound $\Omega\bigg(n^{1+\frac{1}{2^k-1}-o(1)}\bigg)$ proved in \cite{abboud2018hierarchy}, since the lower bounds are for stretch with small factors hidden in $O_{k,\delta}(\cdot)$. In fact, our bound should be compared with their lower bound for $d^{1-\frac{1}{k+1}}$ spanners, which is $\Omega\bigg(n^{1+\frac{1}{2^{k+1}-1}-o(1)}\bigg)$, and is almost optimal.}.

%Note that, for any parameter $\delta>0$, if we let $\eps=\frac{\delta}{10k(2^{k+1}-1)}$, then \Cref{sublinear} gives a sublinear spanner with stretch function $f(d)=d+2^{O(k^2 2^k/\delta)}\cdot d^{1-1/k}$ on $O(n^{1+\frac{1+\delta}{2^{k+1}-1}})$ edges. In other words, for any constant $\delta>0$, there is a sublinear additive spanner of stretch function $f(d)=d+O_{\delta,k}(d^{1-1/k})$ and size $O(n^{1+\frac{1+\delta}{2^{k+1}-1}})$.


%Next, we study fast algorithms for constructing additive spanners. Suppose now we want to find a sparse subgraph $H\subseteq G$ such that $\dist_H(s, t)\leq \dist_G(s, t) + \err(n)$ for some small error term $e$. It is known that when $|E(H)| = 2^{O(1/\epsilon)}n$, $\err(n)$ can be made $O(n^{3/7+\epsilon})$ \cite{bodwin2016better}. However, its construction time is $\tilde{O}(mn)$ since it requires computing all-pairs shortest paths which is at least quadratic. As our secondary result, we show a subquadratic time construction algorithm.

Our second result is a slightly improved bound on linear size additive spanners upon the bound $O(n^{3/7+\eps})$ of \cite{bodwin2016better}. In addition, we show that such a spanner can be computed in subquadratic time, while previous constructions in \cite{bodwin2016better,bodwin2015very} need the computation of all-pairs shortest paths in $G$ which takes time $O(\min\{mn, n^{2.373} \})$.

\begin{theorem}\label{subquad}
For any undirected unweighted graph $G$ on $n$ vertices and $m$ egdges, there exists a spanner with $O(n^{0.403})$ additive stretch and $O(n)$ edges. Moreover, such a spanner can be computed in time $\tilde{O}\big(m + n^{13/7}\big)$.
\end{theorem}

\section{Applications}
\label{sec:apps}
To demonstrate the wide range of usagages of our model, we implement a series of applications:
\begin{enumerate}
	\item Incremental surface \& color reconstruction
	\item 3D saliency detection
	\item Open vocabulary scene understanding
	\item Surface infrared field
	\item 3D style transfer
\end{enumerate}
Originating from our motivation in inspection and service robotics, we implement 1) Incremental surface \& color reconstruction for visualization of robot surroundings.
For robot exploration, we implement 2) 3D saliency detection to indicate the salient regions in maps.
For recovering object-level semantic information in environments, we implement 3) open vocabulary scene understanding to yield the regions containing the objects..
Furthermore, to demonstrate the flexibility, we implement 4) surface infrared fields and 5) 3D style transfer for artistic purposes. 

In~\cref{fig:latent_diff}, we classify those 3 applications into 3 categories: (a) directly obtaining the properties from sensor observation, such as application 1) and 4). (b) processing on sensor data and predict properties, such as application 2), 5). (c) extending (b) to operating beyond latent features, such as application 3).
%Thus, in the following, we discuss about those categories of applications.
% we mainly describe the application 1) (\cref{sec:incremental_reconstruction}) and 3) (\cref{sec:openvoc}).

Application 1) and 4) are in the first one category. Thus, we mainly describe 1) incremental surface \& color reconstruction (\cref{sec:incremental_reconstruction}), while for 4) we can easily exchange color with infrared.
%
For the second with 2) and 5) in~\cref{sec:fabircated_prop}, we mainly describe the usage of fabricated properties.
As the mapping part is redundant to previous category, it will not be detailed.
%
The third category is the application 3) that maps a LIM for high dimensional latent fields.
We demonstrate that this application provides a flexible inference in \cref{sec:openvoc}.


%Afterwards, we evaluate application 1) and 3) in~\cref{sec:exp} and extensively show demonstration for all application in~\cref{sec:exp:extensive_app}.


\subsection{Organization}
We start with preliminaries in \Cref{sec: prelim}. We provide a construction for pairwise additive spanners in \Cref{sec: pairwise}, which will be a building block in the proof of \Cref{sublinear} which appears in \Cref{sec: sublinear spanner}.
Next, we show the construction of subset spanners in \Cref{sec: subset}, which will be a crucial subroutine used in the algorithm of \Cref{subquad}. Lastly, we provide the proof of \Cref{subquad} in \Cref{sec: subq}.





