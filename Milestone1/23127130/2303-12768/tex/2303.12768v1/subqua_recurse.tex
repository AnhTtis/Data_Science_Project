
\subsection{Completing the proof of \Cref{subquad}}
\label{sec: proof of subquad}

In this subsection, we completing the proof of \Cref{subquad} by slightly modifying the algorithm in \Cref{sec: subquad for 3/7} and apply them recursively. Specifically, we first prove the following lemma.

\begin{lemma}
\label{lem: reduction}
Let $f(\rho)=\frac{2/3-\rho}{4-(19/6)\rho}$ be a function.
If there is an algorithm $\alg$, that given any graph $G$ on $n$ vertices and $m$ edges, in time $\tilde O(m+n^{\gamma})$ computes an $+O(n^{\rho})$ additive spanner of $G$ with at most $Cn$ edges, such that $\gamma\ge 1+\frac{(3/2)f(\rho)(1-\rho)}{3/2-\rho}$; then for any parameter $\epsilon>0$, there is an algorithm $\alg'$, that given any graph $G'$ on $n$ vertices and $m$ edges, in time $\tilde O(m+2^{O(\gamma/\eps)}n^{\gamma})$ computes an $+O(n^{\epsilon+f(\rho)})$ additive spanner of $G'$ with $2^{O(1/\eps)}\cdot Cn$ edges.
\end{lemma}

We now use \Cref{lem: reduction} to prove
\Cref{subquad}. 
We set $\eps>0$ as a small enough constant.
%, so $2^{O(1/\eps)}$ is a constant.
Note that \Cref{subquad for 3/7} in fact gives an algorithm with parameter $(\gamma=13/7,\rho=3/7+0.1, C=O(1))$ and we denote it by $\alg_0$. We then apply \Cref{lem: reduction} with $\alg=\alg_0$, and denote by $\alg_1$ the algorithm that it produces, so $\alg_1$ has produce an $+O(n^{\epsilon+f(\rho)})$ additive spanner.
%
Note that the invariant point $\rho^*$ of the mapping $f$ (i.e., the value of $0<\rho^*<1$ such that $f(\rho^*)=\rho^*$) is $\rho^*=\frac{15-\sqrt{54}}{19}=0.4027...$.
%
We then iteratively apply \Cref{lem: reduction} for $K$ times (where $K$ is a large enough constant such that $f(f(\cdots f(3/7+0.1+\eps)\cdots)+\eps)+\eps<0.403$), and get algorithms $\alg_2,\ldots,\alg_K$. It is not hard to verify that the property $\gamma\ge 1+\frac{(3/2)f(\rho)(1-\rho)}{3/2-\rho}$ always holds.
Eventually, $\alg_K$ is the algorithm that we return. Note that the additive error of the spanner it produces is $+O(n^{0.403})$, the running time is $\tilde O(m+n^{13/7}\cdot 2^{O((13/7)\cdot (K/\eps))})$, which is $\tilde O(m+n^{13/7})$ as $1/\eps$ and $N$ are both constants, and the size of the smaller it produces is $2^{O((13/7)\cdot (K/\eps))}\cdot Cn=O(n)$.



We now sketch the proof of \Cref{lem: reduction}, highlighting the difference between the algorithm here and the algorithm in \Cref{sec: subquad for 3/7}.

\begin{proof}[Proof Sketch of \Cref{lem: reduction}]
The algorithm for \Cref{lem: reduction} is very similar to the algorithm for \Cref{subquad for 3/7}, except for (i) an extra Step 4 below, which is a recursive call of an additive spanner algorithm; and (ii) more fine-grained tuning of parameters. We define the function $g(\rho)=\frac{(3/2)\cdot f(\rho)}{(3/2-\rho)}$.

\paragraph{Step 1.} Sparsify $G$ to get $G'\subseteq G$ using \Cref{preproc} with $d=O(n^{1-f(\rho)})$ and $|E(G')|=O(n^{2-f(\rho)})$. 

\paragraph{Step 2.} Compute a set $\balls$ of balls using \Cref{clustering} with parameters $R=\ceil{n^{f(\rho)}}$ and $\frac{\eps}{10\log n}$. We say that a ball is \emph{small} iff $|\ball(c,r)|\le n^{g(\rho)}$, otherwise we say it is \emph{large}.

\paragraph{Step 3.} For each small ball $\ball(c,r)$, we apply \Cref{bottleneck} to compute an integer $d\in [r, 2r]$ such that $|\ball^=(c, d)\cup\ball^=(c, d+1)|\leq 2|\ball(c, 4r)| / r \leq 2|\ball(c, 4r)| / R$, and then apply \Cref{subset} to compute a subset spanner $L_c$ of $G'[\ball(c, 4r)]$ on the set $\ball^=(c, d)\cup\ball^=(c, d+1)$.

\paragraph{Step 4.} For each large ball $\ball(c,r)$, we apply the algorithm \alg to compute a spanner of $G'[\ball(c, 4r)]$ with error $+O(|\ball(c,r)|^{\rho})$, that we denote by $L_c$.

\paragraph{Step 5.} Sample a random subset $S\subseteq V(G)$ of $\ceil{10R^{2/3}\log n}$ vertices, and apply \Cref{subset} to compute a subset spanner $\hat H$ of $G'$ on $S$ with additive error $O(|S|^{3/2}\cdot n^{\eps})$.

$\ $

The output graph $H$ is simply defined to be the union of 
\begin{itemize}
	\item for each ball $\ball(c,r)\in \balls$, a BFS tree $T_c$ that is rooted at $c$ and spans all vertices in $\ball(c, 4r)$;
	\item for each small or large ball $\ball(c,r)\in \balls$, graph $L_c$;
	\item graph $\hat H$.
\end{itemize}


The analysis is almost identical to the analysis in \Cref{sec: subquad for 3/7}, with the following changes.

\paragraph{Stretch analysis.} In \Cref{clm: beta_i}, the analysis would be changed to
$$\begin{aligned}
\dist_H(v_{i-1}, u_i) &\leq \dist_{G}(v_{i-1}, u_i) + \tilde{O}\left(\max\set{\left(\frac{|\ball(c_i, 4r_{i})|}{n^{f(\rho)}}\right)^{3/2}\cdot n^\epsilon, |\ball(c_i, 4r_{i})|^{\rho}}\right)\\
&\leq \dist_{G}(v_{i-1}, u_i) + \tilde{O}\left( \frac{|\ball(c_i, r_{i})|}{n^{(1-\rho)\cdot g(\rho)}}\cdot 2^{O(1/\epsilon)}\cdot n^\epsilon\right).
\end{aligned}$$
Consequently, the analysis in \Cref{clm: additive error} would be changed to
$$
\begin{aligned}
\dist_H(s, u_x) & \leq \dist_G(s, u_x) + O(n^{\epsilon+f(\rho)}) + O\left(2^{O(1/\epsilon)}\cdot n^\epsilon\cdot\sum_{i=2}^{x-1}\frac{|\ball(c_i, r_{i})|}{n^{(1-\rho)\cdot g(\rho)}}\right)\\
& \le \dist_G(s, u_x) + O(n^{\epsilon+f(\rho)}) + O\left(2^{O(1/\epsilon)}\cdot n^\epsilon\cdot\frac{n^{1-\frac{2}{3}f(\rho)}}{n^{(1-\rho)\cdot g(\rho)}}\right)\\
& \le \dist_G(s, u_x) + 2^{O(1/\epsilon)}\cdot n^{\epsilon+f(\rho)}.
\end{aligned}
$$
and similarly we get that $\dist_H(u_y, t)\leq \dist_G(u_y, t) + 2^{O(1/\epsilon)}\cdot n^{\epsilon+f(\rho)}$. Therefore, the additive error of the algorithm is $2^{O(1/\epsilon)}\cdot n^{\epsilon+f(\rho)}$.

\paragraph{Size and runtime analysis.} Via similar arguments as in the proof of \Cref{clm: spanner size}, we can show that $|E(H)|=2^{O(1/\eps)}\cdot n$.
We now analyze the runtime of the algorithm. Ignoring the new Step 4, we can show via similar arguments as in the proof of \Cref{clm: runtime} that the runtime is $\tilde O(m+2^{O(1/\eps)}\cdot n^{1+\frac{(3/2)f(\rho)(1-\rho)}{3/2-\rho}})$. The runtime of Step 4 is
$$\sum_{c: \text{ }\ball(c,r) \text{ large}}\bigg(\tilde O(\vol_{G}(\ball(c,4r)))+|\ball(c,4r)|^{\gamma}\bigg)=\tilde O(m+2^{O(\gamma/\eps)}\cdot n^{\gamma}).$$
As $\gamma\ge 1+\frac{(3/2)f(\rho)(1-\rho)}{3/2-\rho}$, the runtime of the whole algorithm is 
$\tilde O(m+2^{O(\gamma/\eps)}\cdot n^{\gamma})$.
\end{proof}



