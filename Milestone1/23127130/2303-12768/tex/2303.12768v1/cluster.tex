\section{Proof of \Cref{clustering}}
\label{apd: Proof of clustering}

\iffalse
In this section we show an almost-linear time algorithm that computes a clustering of $G$; this clustering scheme was originally proposed in \cite{bodwin2016better}, but by their algorithm description it has a very large runtime. This section is devoted to the following statement.
\begin{lemma}
	Given any undirected unweighted graph $G = (V,  E)$ on $m$ edges and $n$ vertices, and any possible value $\eps > 0$, for any integer $R$, there is an algorithm with runtime $\tilde{O}(mn^{\eps})$ that computes set of balls $\balls = \{\ball_{G}(c, r)\}$ where $R\leq r \leq 2^{10/\eps}R$, with the following two properties.
	\begin{itemize}
		\item (Coverage) For each $v\in V$, there is some ball such that $v\in \ball_{G}(c, r)\in \balls$.
		\item (Disjointness) We have $\sum_{\ball(c, r)\in\balls}|\ball(c, r)| = O(n\log n)$,  and $\sum_{\ball(c, r)\in\balls}|\ball(c, 4r)| = O(n^{1+\eps}\log n)$, and $\sum_{\ball(c, r)\in\balls}\vol(\ball(c, 4r)) = O(m\cdot n^{\eps}\log n)$.
	\end{itemize}
\end{lemma}
\fi


%\paragraph{Algorithm description.} 

%Without loss of generality, we can assume that $R$ is an integral power of $4$, and we will ensure that, over the course of the algorithm, all values $r_i$ will be integral powers of $4$, too. 

Throughout, we use the parameter $\beta = n^\eps$. 

The algorithm iteratively builds the collection $\balls$. Initially, $\balls = \emptyset$. The algorithm continues to be executed as long as there exists a vertex that is not covered by the collection $\balls$.
%Initially, $\balls = \emptyset$. 
We now describe an iteration. First, we pick an arbitrary vertex $c$ that is not covered by the current collection $\balls$,
%, namely $c\in V\setminus \big(\bigcup_{B\in \balls}B\big)$, 
and add a new ball $\ball(c,r)$ to $\bset$ centered at $c$. Its radius $r$ is determined by the following process: %initialized as $R$. 
%We will enlarge $\ball(c, r)$ by exponentially increasing $r$. The iterative process conducts the following steps.
\begin{enumerate}[(1)]
	\item Start with $r=R$.
	\item Perform breath-first search from $c$ in $G$ to compute the ball $\ball(c, 4r)$.
	\item If $|\ball(c, 4r)|\leq \beta\cdot |\ball(c, r/2)|$ and $\vol(\ball(c, 4r))\leq \beta\cdot \vol(\ball(c, r/2))$, then add the ball $\ball(c, r)$ to $\balls$, and terminate the iteration. Otherwise, update $r\leftarrow 4r$ and repeat Steps (2) and (3).
\end{enumerate}

%\paragraph{Correctness \& runtime.} 
We now proceed to analyze the algorithm. 
%Consider the resulting collection $\bset$ of balls we get.
First, it is easy to see that the radius of every ball in $\bset$ at the end of the algorithm is at least $R$, as the process of determining the radius of each new ball start with $r=R$ and only increases $r$ afterwards.
Second, from the algorithm, when a new ball is added to $\bset$, its radius $r$ is determined by an iterative process, where in each round, $r$ is increased by a factor of $4$ whenever $|\ball(c, 4r)|> \beta\cdot |\ball(c, r/2)|$ or $\vol(\ball(c, 4r)) > \beta\cdot |\ball(c, r/2)|$. As $|\ball(c, 4r)|$ and $\vol(\ball(c, 4r))$ are bounded by $n$ and $m$ respectively, the number of times that the radius $r$ is increased is at most $\ceil{\log_\beta n} + \ceil{\log_\beta m} < 5/\eps$. Therefore, in the end, $r\le  R\cdot  4^{5/\eps}= R\cdot  2^{10/\eps}$.

We next prove the following observation.

\begin{observation}\label{grow}
At the end of the algorithm, for every vertex $v\in V$, there are at most $5/\eps$ balls $\ball(c, r)$ in $\bset$, such that $\dist_{G}(c,v)\le r/2$.
\end{observation}
\begin{proof}
We first prove the following observation.
\begin{observation}\label{cover}
	When a new ball $\ball(c, r)$ is added to the collection $\bset$, for any other ball $\ball(c', r')$ in $\balls$ with $\ball(c', r'/2)\cap \ball(c, r/2)\neq\emptyset$, $r' \leq r/4$ must hold.
\end{observation}
\begin{proof}
	Suppose otherwise that $r'> r/4$. As all radius are integral powers of $4$, $r' \geq r$. Therefore, $\dist_{G}(c, c')\leq r'/2 + r/2 \leq r'$, and so $c\in\ball(c', r')$, which means that $c$ was covered by the collection $\bset$ before the ball $\ball(c, r)$ is added, a contradiction.
\end{proof}

We say that a vertex $v$ is \emph{captured} by a ball $\ball(c, r)$ iff $\dist_{G}(c,v)\le r/2$.
%
From \Cref{cover}, when $v$ is captured by a new ball $\ball(c, r)$, its radius $r$ is at least $4$ times the radius of any other ball in $\bset$ that captures $v$. As we have shown that the radius of every ball in $\bset$ is at least $R$ and at most $R\cdot 4^{5/\eps}$, the number of balls  in $\bset$ that captures $v$ is at most $5/\eps$.
\end{proof}

%We now complete the proof of \Cref{clustering}.
\iffalse
\begin{observation}
	At the end of the algorithm, any radius $r_i$ satisfies $R\leq r_i \leq 2^{10/\eps}\cdot R$, plus that $\sum|\ball(c, 4r)|= O(n^{1+\eps})$, $\sum\vol(\ball(c, 4r))= O(mn^{1+\eps})$.
\end{observation}
\begin{proof}
\fi


From \Cref{grow}, at the end of the algorithm, each vertex in $G$ is occupied by at most $O(1/\eps)$ balls in $\balls$. Therefore, $\sum|\ball(c, r/2)|\leq  O(n/\eps)$; and
$\sum|\ball(c, 4r)|\leq \beta\cdot\sum |\ball(c, r/2)|\leq O(n^{1+\eps}/\eps)$.
Similarly, $\sum\vol(\ball(c, 4r))\leq \beta\cdot\sum \vol(\ball(c, r/2))\leq \beta\cdot O(m/\eps) = O(m\cdot n^{\eps}/\eps)$.
%\end{proof}


Finally, note that when we add a ball $\ball(c, r)$ to $\bset$, the running time of the algorithm in that iteration is $O(\vol(\ball(c, 4r)))$. Therefore, algorithm terminates in time $O\big(\sum\vol(\ball(c, 4r))\big)\leq O(m\cdot n^{\eps}/\eps)$.

\iffalse
\begin{corollary}
	The runtime of the greedy algorithm is bounded by $\tilde{O}(m\cdot n^\eps)$.
\end{corollary}
\fi