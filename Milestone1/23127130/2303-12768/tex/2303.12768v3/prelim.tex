\section{Preliminaries}
\label{sec: prelim}

By default, all logarithms are to the base of $2$, and all graphs are undirected and unweighted.

Let $G = (V, E)$ be a graph. 
%For a vertex $v\in V$, we denote by 
For a subset $S\subseteq V$ of vertices, we define $\vol_G(S)=\sum_{v\in S}\deg_G(v)$, where $\deg_G(v)$ is the degree of $v$ in $G$.
For any pair $u, v\in V(G)$, let $\dist_G(u, v)$ be the shortest-path distance between $u$ and $v$ in $G$. For any $c\in V$ and $r>0$, we define the ball $\ball_G(c, r)$ as the set of vertices in $G$ that are at distance at most $r$ from $c$, namely $\ball_G(c, r)=\set{v\in V\mid \dist_G(c,v)\le r}$; $r$ is called the \emph{radius} of the ball, and we define the \emph{boundary} of the ball $\ball_G(c, r)$ as $\ball_G^=(c, r)=\set{v\in V\mid \dist_G(c,v)= r}$. We will sometimes omit the subscript $G$ in the above notations when it is clear from context. 
Let $\balls$ be a collection of balls. We say that a vertex $v$ is \emph{covered} by $\balls$ if $v$ belongs to some ball in $\balls$.
%and we say that $v$ is \emph{covered at boundary} by $\balls$, iff $v$ belongs to the boundary of \underline{every} ball in $\balls$ that contains $v$. 
We will use the following lemma, whose proof is a simple implementation of the breadth-first search (BFS) algorithm, and is omitted.


\iffalse
\begin{definition}\label{ball-cover}
	Given a set of balls $\clusters = \{\ball(c, d_c), c\in V\}$, a vertex that belongs to some balls $\ball(c, d_c)\in\clusters$ are called \textbf{covered} by $\clusters$. Moreover, a covered vertex $v$ is called \textbf{covered at rim} by $\clusters$, if $v\in\ball^=(c, d_c)$ for every ball $\ball(c, d_c)\in \clusters$ containing $v$. 
\end{definition}
\fi

\begin{lemma}\label{bottleneck}
Given a graph $G$, a vertex $c\in V(G)$, and two integers $0<r_1 < r_2<r$, we can find an integer $r_1< d\le r_2$, such that $|\ball^=(c, d)\cup\ball^=(c, d+1)|\leq \frac{2\cdot |\ball(c, r)|}{r_2-r_1}$, in time $O(\vol(\ball(c, r)))$.
%Suppose $\ball^=(c, R)\neq\emptyset$, then there exists an integer $r\in (R_1, R_2]$ such that $|\ball^=(c, d)\cup\ball^=(c, d+1)|\leq \frac{2|\ball(c, R)|}{R_2-R_1}$. Plus, this integer $d$ can be found in time $\sum_{v\in }\vol(\ball(c, R))$.
\end{lemma}



\paragraph{Clustering in almost-linear time.}
One of the building blocks in the construction of $+O(n^{3/7+\eps})$ spanner in \cite{bodwin2021better} is a clustering procedure, which takes the input graph and computes a collections of balls with certain coverage and disjointness properties.
We will also use this clustering procedure. However, the running time of algorithm for computing such a clustering in \cite{bodwin2021better} is a large polynomial in $n$. In order to design a sub-quadratic algorithm for additive spanners, we provide an almost-linear time algorithm for computing a clustering with slightly different parameters. 



%In this section we show an almost-linear time algorithm that computes a clustering of $G$; this clustering scheme was originally proposed in \cite{bodwin2021better}, but by their algorithm description it has a very large runtime. This section is devoted to the following statement.
\begin{lemma}[Almost-linear time algorithm for Lemma 13 in \cite{bodwin2021better}]
\label{clustering}
There is an algorithm, that, given any undirected unweighted graph $G = (V,  E)$ on $n$ vertices and $m$ edges, any parameter $\eps > 0$, and any integer $R>0$, computes in time $O(mn^{\eps}/\eps)$ a collection $\balls$ of balls in $G$, such that
\begin{itemize}
\item the radius of each ball $\ball(c,r)\in \balls$ satisfies that $R\leq r \leq 2^{10/\eps}\cdot R$; 
\item all vertices in $V$ are covered by $\balls$;
\item the following coverage properties hold:
\begin{itemize}
\item $\sum_{\ball(c, r)\in\balls}|\ball(c, r/2)| = O(n/\eps)$;
\item for each $\ball(c, r)\in\balls$, $|\ball(c, 4r)|\le n^{\eps}\cdot |\ball(c, r/2)|$, so $\sum_{\ball(c, r)\in\balls}|\ball(c, 4r)| = O(n^{1+\eps}/\eps)$;
\item $\sum_{\ball(c, r)\in\balls}\vol(\ball(c, 4r)) = O(m\cdot n^{\eps}/\eps)$.
\end{itemize}
\end{itemize}
\end{lemma}

\begin{remark}
	Compared to the original statement in \cite{bodwin2021better}, their runtime is $O(mn)$, but they have a better bound on sizes. More specifically, they have bounds $\sum_{\ball(c, r)\in\balls}|\ball(c, r/2)| = O(n)$ and $\sum_{\ball(c, r)\in\balls}|\ball(c, 4r)| = O(n^{1+\eps})$ which are smaller than our bounds by a factor of $\epsilon$.
\end{remark}

The proof of \Cref{clustering} is deferred to \Cref{apd: Proof of clustering}.



\paragraph{Consistent paths and distance preservers.} Let $\pi$ be a path and let $x, y$ be two vertices of $\pi$. We denote by $\pi[x, y]$ the subpath of $\pi$ between $x$ and $y$, and we denote by $|\pi|$ the number of edges in $\pi$; we can also define notations $\pi(x, y), \pi(x, y], \pi[x, y)$ in the natural way. Let $\Pi$ be a collection of paths. We say that $\Pi$ is \emph{consistent}, if for any pair $\pi_1, \pi_2\in \Pi$, the intersection between $\pi_1$ and $\pi_2$ is a (possibly empty) subpath of both $\pi_1$ and $\pi_2$.


%Our algorithm will use distance preservers as a building block.

Let $\pset\subseteq V\times V$ be a set of pairs of vertices in $G$. A subgraph $H\subseteq G$ is a \emph{distance preserver of $G$ with respect to $\pset$}, if $\dist_H(s, t) = \dist_{G}(s, t)$ holds for every $(s, t)\in \pset$. 
We will use the following previous results on distance preservers and $+6$ pairwise additive spanners.


\iffalse
\begin{definition}[\cite{coppersmith2006sparse}]
	Given a graph $G = (V, E)$ and a set of pairs $P\subseteq V\times V$, a subgraph $H\subseteq G$ is a distance preserver of $P$, if $\dist_H(s, t) = \dist_{G}(s, t)$ for any $(s, t)\in P$.
\end{definition}


\begin{definition}[consistency, \cite{coppersmith2006sparse,bodwin2021better}]
	A set of paths $\paths$ is consistent, if for any two paths $\pi_1, \pi_2\in \paths$, their intersection is (possibly empty) a continuous sub-path of both of them.
\end{definition}
\fi


\begin{lemma}[\cite{coppersmith2006sparse}]\label{consist}
	Let $G = (V, E)$ be a graph on $n$ vertices, let $\pset\subseteq V\times V$ be a set of pairs of its vertices, and let $\Pi$ be any consistent collection of paths in $G$ that contains, for each pair $(s,t)\in \pset$, a path $\pi_{s, t}$ connecting $s$ to $t$. Then $|E(\Pi)|=|E(\bigcup_{(s, t)\in \pset}\pi_{s, t})|=O(n+\sqrt{n}\cdot |\pset|)$.
\end{lemma}

It is also implicit in \cite{coppersmith2006sparse} that $|E(\Pi)| = O(n\sqrt{|\pset|})$ which will not be used in in our algorithm.

We use the following corollary of \Cref{consist}, whose proof is a straightforward implementation of the BFS algorithm and the standard edge-weight perturbation technique, and is omitted.

\begin{corollary}
\label{cor: BFS consistent}
There is an algorithm, that given a graph $G$ on $n$ vertices and $m$ edges, and a subset $S$ of its vertices, in time $O(m\cdot |S|)$, computes a consistent collection $\Pi$ of paths that contains, for each pair $s, t\in S$, a shortest path $\pi_{s,t}$ connecting $s$ to $t$ in $G$, such that $|E(\Pi)|=|E(\bigcup_{s, t\in S}\pi_{s, t})|=O(n+\sqrt{n}\cdot|S|^2)$.
\end{corollary}

%\begin{lemma}[\cite{bodwin2021better}]\label{dist-preserve}
%	Given a graph $G = (V, E)$ and a set of pairs $P\subseteq V\times V$, there exists a distance preserver for $P$ in $G$ with $O(n|P|^{1/3} + n^{2/3}|P|^{2/3})$ edges.
%\end{lemma}

\begin{lemma}[\cite{kavitha2017new}]\label{pairwise-spanner}
	There is an efficient algorithm, that, given any graph $G = (V, E)$ and any set $\pset\subseteq V\times V$ of pairs, computes a subgraph $H \subseteq G$ with $|E(H)|=O(n|\pset|^{1/4})$, such that for every pair $(s,t)\in \pset$, $\dist_H(s,t)\le \dist_G(s,t)+6$.
\end{lemma}


