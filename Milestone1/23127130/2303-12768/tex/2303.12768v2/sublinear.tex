\section{Almost Optimal Sublinear Additive Spanners}
\label{sec: sublinear spanner}

In this section, we provide the proof of \Cref{sublinear}.
In fact, we will prove the following theorem.

\begin{theorem}
\label{sublinear_real}
For any undirected unweighted graph $G = (V, E)$ on $n$ vertices, any parameter $\epsilon\in (0, 0.1)$ and any integer $k\ge 1$, there is a subgraph $H\subseteq G$ with $|E(H)|\le O\left(n^{1 + 10k\eps +\frac{1}{2^{k+1} - 1}}\right)$, such that for every pair $u,v\in V(G)$, $\dist_{H}(u,v)\le \dist_{G}(u,v)+2^{30k/\epsilon}\cdot \dist_{G}(u,v)^{1-1/k}$.
\end{theorem}

Note that, for any constant parameter $\delta>0$, if we let $\eps=\frac{\delta}{10k(2^{k+1}-1)}$, then \Cref{sublinear_real} gives a sublinear spanner with stretch function $f(d)=d+2^{O(k^2 2^k/\delta)}\cdot d^{1-1/k}$ on $O(n^{1+\frac{1+\delta}{2^{k+1}-1}})$ edges, which implies \Cref{sublinear}.


In the remainder of this section, we provide the proof of \Cref{sublinear_real} by induction on $k$, with the help of \Cref{pairwise-sublinear}. 
The base case is when $k=1$. We note that it was shown by \cite{baswana2010additive} that any undirected unweighted graph admits an $+6$-additive spanner of size $O(n^{4/3})$, and the base case follows from this result. Assume from now on that  \Cref{sublinear_real} is true for $1,\ldots,(k-1)$.
Similar to \Cref{sec: pairwise}, we denote $f_{k, \epsilon}(d) = d + 2^{30k/\epsilon}\cdot d^{1-1/k}$ for brevity.






%For any undirected graph $G = (V, E)$ with $n$ vertices, suppose we can always find a subgraph $H\subseteq G$ of  $g_k(n)$ edges such that for any $s, t\in V$ we have $\dist_{H}(s, t)\leq f_{k, \epsilon}(\dist_{G}(s, t))$; when $k = 1$, it is known that $g_1(n) = O(n^{4/3})$ \cite{baswana2010additive}. To prove Theorem \ref{sublinear} when $k\geq 2$, we will inductively find a sparse subgraph with stretch function $f_{k, \epsilon}(\cdot)$.
 
Similar to the algorithm in \Cref{sec: pairwise}, for each $D\in \set{1,2,2^2,\ldots,2^{\floor{\log n}}}$, we will construct a subgraph $H_D\subseteq G$, such that for all pairs $s,t\in V(G)$ with $D\le \dist_G(s, t) <2D$, $\dist_H(s, t)\leq \dist_G(s, t) + O(D^{1-1/k})$ holds.
We will then let $H=\bigcup_{0\le i\le \floor{\log n}}H_{2^i}$ to finish the construction.
%Fix any integral power of $2$ denoted by $D$, we will find a sparse subgraph $H\subseteq G$, such that for any $s, t\in V$ satisfying $D\leq \dist_{G}(s, t) < 2D$, we have $\dist_H(s, t)\leq \dist_{G}(s, t) + O(D^{1-1/k})$. 
For convenience, we assume that $D^{1-1/k}$ is an integer\footnote{Our computation will in fact use $\floor{D^{1-1/k}}$. For notational convenience we omit it and just write $D^{1-1/k}$ instead.}.

\subsection{Algorithm description}

We now describe the construction of graph $H_D$. We first apply the algorithm of \Cref{clustering} to $G$ with parameters $R = D^{1-1/k}$ and $\epsilon$. Let $\clusters$ be the set of balls we obtain. Let $L_k$ be a threshold value to be determined later. We say that a ball $\ball(c, r)\in\clusters$ is \emph{small} if $|\ball(c, r)| \leq L_k$, otherwise we say it is \emph{large}.

Similar to the algorithm in \Cref{sec: pairwise}, the spanner $H_D$ is the union of the following graphs:
\begin{enumerate}[(i)]
	\item for each ball $\ball(c, r)\in\clusters$, a BFS tree that is rooted at $c$ and spans all vertices in $G[\ball(c, 4r)]$;
	
	\item for each small ball $\ball(c, r)\in \clusters$, a spanner of the induced subgraph $G[\ball(c, 4r)]$, with stretch function $f_{k-1, \epsilon}$ and size $O\left(|\ball(c, 4r)|^{1 + 10(k-1)\eps +\frac{1}{2^{k} - 1}}\right)$, whose existence is guaranteed by the inductive hypothesis;
	
	\item for each ball $\ball(c, r)\in\clusters$, a pairwise spanner with respect to a collection $\pset_c$ of pairs in $\ball(c, 2r)$, with stretch function $f_{k-1, \epsilon}$ and size $O(2^{2(k-1)/\epsilon}|\ball(c, 4r)|^{1+10(k-1)\epsilon}|\pset_c|^{1 / 2^{k}})$, whose existence is guaranteed by the inductive hypothesis; we will guarantee that for each demand pair $(s, t)\in \pset_c$, any $s$-$t$ shortest path in $G$ lies in the induced subgraph $G[\ball(c, 4r)]$. The construction of sets $\pset_c$ is iterative and described next.
\end{enumerate}

%For any ball $\ball(c, r)\in \clusters$, add a breath-first search tree at $c$ that spans $\ball(c, 4r)$ in $G$ to $H$. T


Let $S$ be a random subset of $V$ of size $\ceil{\frac{10n}{L_k}\log n}$, so with high probability, $S$ intersects all large balls in $\balls$.
%For large balls, find a hitting set $S\subseteq V$ of size $\ceil{\frac{10n}{L_k}\log n}$ such that $\ball(c, r)\cap S\neq \emptyset$ for each large ball $\ball(c, r)\in \clusters$. Then, we apply a path-buying procedure to build a subset spanner on the hitting set $S$, which is similar to previous algorithms. Define a set of shortest paths between vertices in $S$:
We then compute $\Pi = \{\pi_{s, t}\mid s, t\in S, \dist_{G}(s, t) < 2D + 4\cdot 2^{10/\epsilon}D^{1-1/k}\}$, where $\pi_{s,t}$ is an $s$-$t$ shortest path in $G$.
%Also, each large ball $\ball(c, r)\in\clusters$ is associated with a vertex set $U_c\subseteq S$ and a pair set $\pset_c\subseteq \ball(c, 2r)\times\ball(c, 2r)$.
We now proceed to iteratively construct the sets $\set{\pset_c}$ of pairs. Throughout, we maintain, for each large  ball $\ball(c,r)\in \bset$, a set $\pset_c$ of pairs of vertices in $\ball(c,2r)$, and another set $U_c$ of vertices in $V$ which intuitively contains all vertices that are ``settled with the ball $\ball(c, r)$''.

We then process all paths in $\Pi$ sequentially in an arbitrary order. For each path $\pi_{s, t}\in \Pi$, we first apply the subroutine \pathpart to it and the collection $\bset$ of balls, and obtain a partitioning $\pi_{s,t} = \alpha_1\circ\alpha_2\circ\cdots \circ\alpha_l$. 
For each $1\le i\le l$, we denote by $\ball(c_i, r_i)$ the ball that hosts the subpath $\alpha_i$.

%Enumerate all paths $\pi_{s, t}$ from $\Pi$ in an arbitrary order. When a path $\pi = \pi_{s, t} \in \Pi$ between $s, t\in S$ is enumerated, apply the \emph{path partition subroutine} \ref{partition} from the previous section to decompose $\pi = \alpha_1\circ\alpha_2\cdots\circ \alpha_l$ along with balls $\ball(c_1, r_{1}), \ball(c_2, r_{2}), \cdots, \ball(c_l, r_{l})\in\clusters$.

If either all the balls $\ball(c_1, r_{1}), \ball(c_2, r_{2}), \ldots, \ball(c_l, r_{l})$ are small, or there exists a large ball $\ball(c_i, r_{i})$ such that $s, t\in U_{c_i}$, then we do nothing and move on to the next path in $\Pi$. Otherwise, for each large ball $\ball(c_i, r_{i})$, we add vertices $s,t$ to set $U_{c_i}$, and add the pair $(s_i, t_i)$ to set $\pset_c$.
% where we assume $\alpha_i = \pi[s_i, t_i]$.
This completes the description of the construction of sets $\set{\pset_c}$, and also finishes the description of the algorithm.

Before we proceed to the size and stretch analysis, we prove the following simple observation.

\begin{observation}\label{demand-size}
In the end, for each ball $\ball(c, r)\in \clusters$, $|\pset_c|\leq |S| = \ceil{\frac{10n}{L_k}\log n}$.
\end{observation}
\begin{proof}
	By the algorithm, each time a new pair is added to $\pset_c$, $|U_c|$ also increases by at least one. Therefore $|\pset_c|\leq |U_c|\leq |S| = \ceil{\frac{10n}{L_k}\log n}$.
\end{proof}

%After we have processed all shortest paths in $\Pi$, go over each large ball $\ball(c, r)\in\clusters$ and build a pairwise spanner on $G[\ball(c, 4r)]$ with demand pairs $\pset_c$; here we are applying the upper bound from Lemma \ref{pairwise-sublinear}. Finally, set the output spanner $H$ to be $\bigcup_{c\in V}H_c$.

\subsection{Stretch analysis}

The stretch analysis of the algorithm in \Cref{sec: sublinear spanner} is quite similar to the Step 1 stretch analysis in \Cref{sec: pairwise} (\Cref{C0} and \Cref{C0-ineq}). We start with the following claims.

\begin{claim}\label{path-buy}
In the end, for each large ball $\ball(c, r)\in \balls$ and each vertex $s\in U_c$,
$$\dist_{H_D}(s, c)\leq \dist_{G}(s, c) + 50\cdot 2^{(30k-10)/\epsilon}D^{1-1/k}.$$
\end{claim}
\begin{proof}
Assume that the shortest path $\pi_{s,t}$ was being processed when $s$ was added to $U_c$, and that $\ball(c, r)$ was $\ball(c_i, r_i)$ under the notation of the subroutine \pathpart. 
%Consider the moment when $s$ was added to $U_c$. Assume by the time the algorithm was enumerating a shortest path $\pi_{s, t}\in \Pi$, and $\ball(c, r)$ is equal to some ball $\ball(c_i, r_{i})$ after we had decomposed $\pi_{s, t} = \alpha_1\circ\alpha_2\circ\cdots \circ\alpha_l$. 
Next, we will construct a short path from $s$ to $c$ in $H_D$. For each $1\leq j<i$, consider the shortest path $\alpha_j = \pi[s_j, t_j]$ and the ball $\ball(c_j, r_{j})$. By \Cref{obs}, $\alpha_j$ lies entirely within $G[\ball(c_j, 4r_{j})]$. Let $\phi_j$ be a shortest path from $s_j$ to $t_j$ in $H_D$. 
%To upper bound the length of $\phi_j$, there are two possible cases.
We distinguish between the following cases.
\begin{itemize}[leftmargin=*]
	\item $\ball(c_j, r_{j})$ is small.
	
	Recall that graph $H_D$ contains a sublinear spanner of the induced subgraph $G[\ball(c_j, 4r_{j})]$ with stretch function $f_{k-1,\eps}$. Therefore,  
	$$\begin{aligned}
		|\phi_j| \leq |\alpha_j| + 2^{30(k-1)/\epsilon}\cdot |\alpha_j|^{1-1/(k-1)} &\leq |\alpha_j| + 2^{30(k-1)/\epsilon}\cdot \left(4\cdot 2^{10/\epsilon}\cdot D^{1-1/k}\right)^{1-1/(k-1)}\\
		&\leq |\alpha_j| + 4\cdot 2^{(30k-20)/\epsilon}D^{1-2/k}.
	\end{aligned}$$

	\item $\ball(c_j, r_{j})$ is large.
	
	From the algorithm description, the pair $(s_j, t_j)$ was added to set $\pset_c$ in this iteration, and $\alpha_j$ is contained entirely within $G[\ball(c_j, 4r_{j})]$ (from \Cref{obs}). Therefore, 
	%the pairwise spanner of $G[\ball(c_j, 4r_{j})]$ contains a short path from $s_j$ to $t_j$. Thus, 
	$$|\phi_j| \leq f_{k-1, \epsilon}(|\alpha_j|) \leq |\alpha_j| + 4\cdot 2^{(30k-20)/\epsilon}D^{1-2/k}.$$
\end{itemize}

As $t_{i-1} = s_i\in \ball(c_i, r_{i})$ holds for every $1\leq j<i$, we have $\dist_G(c, t_{i-1})\leq r \leq 2^{10/\epsilon}D^{1-1/k}$, and therefore
%Taking a summation over all indices $1\leq j<i$ and using triangle inequality, we have: 
	$$\begin{aligned}
		\sum_{j=1}^{i-1}|\phi_j| &\leq \sum_{j=1}^{i-1}\left(|\alpha_j| + 4\cdot 2^{(30k-20)/\epsilon}D^{1-2/k}\right)\leq \sum_{j=1}^{i-1}|\alpha_j| + O_\epsilon(D^{1-1/k})\\
		&\leq \dist_G(s, c) + \dist_G(c, t_{i-1}) + i\cdot 4\cdot 2^{(30k-20)/\epsilon}D^{1-2/k}\\
		&< \dist_G(s, c) + 49\cdot 2^{(30k-10)/\epsilon}D^{1-1/k}.
	\end{aligned}$$
%
Finally, we let $\phi_i$ be an arbitrary shortest path connecting $t_{i-1}$ to $c$ in $H_D$. Since $H_D$ contains a breadth-first search tree rooted at $c$ that spans all vertices in $\ball(c, 4r)$, $|\phi_i|\leq 4r\leq 4\cdot 2^{10/\epsilon}D^{1-1/k}$. Therefore, $\rho = \phi_1\circ\phi_2\circ\cdots\circ\phi_i$ is a path in $H_D$ that connects $s$ and $c$, and moreover,
%by the above inequality, we can prove:
	$$\dist_{H_D}(s, c)\leq \sum_{j=1}^{i}|\phi_j| \leq \dist_G(s, c) + 50\cdot 2^{(30k-10)/\epsilon}D^{1-1/k}.$$
\end{proof}

\begin{claim}\label{hitset}
	For any pair of vertices $s, t\in S$ such that $\dist_G(s, t) < 2D +  4\cdot 2^{10/\epsilon}D^{1-1/k}$, 
	$$\dist_{H_D}(s, t)\leq \dist_G(s, t) + 101\cdot 2^{(30k-10)/\epsilon}D^{1-1/k}.$$
\end{claim}
\begin{proof}
For any such pair of vertices $s, t\in S$, consider the moment when the shortest path $\pi_{s, t}\in \Pi$ was processed and partitioned into $\alpha_1,\ldots,\alpha_l$. We distinguish between the following two cases.
%Consider the time when the shortest path $\pi_{s, t}\in \Pi$ was enumerated by our algorithm. Suppose back then $\pi_{s, t}$ was decomposed as a sequence of sub-paths $\alpha_1\circ\alpha_2\circ\cdots\circ\alpha_l$ along with balls $\ball(c_1, r_{1}), \ball(c_2, r_{2}), \cdots, \ball(c_l, r_{l})\in \clusters$. There are two possible cases.
	\begin{itemize}[leftmargin=*]
		\item There existed an index $1\le i\le l$ such that $s, t\in U_{c_i}$ at the moment.
		%There exists a large ball $\ball(c_i, r_{i})$ such that $s, t\in U_c$ when $\pi_{s, t}$ was being processed by the algorithm.
		
		In this case, by \Cref{path-buy}, $\dist_{H_D}(s, c_i)\leq \dist_G(s, c_i) + 50\cdot 2^{(30k-10)/\epsilon}D^{1-1/k}$, and $\dist_{H_D}(c_i, t)\leq \dist_G(c_i, t) + 50\cdot 2^{(30k-10)/\epsilon}D^{1-1/k}$. By triangle inequality,
		$$\begin{aligned}
			\dist_{G}(s, c_i) + \dist_{G}(c_i, t)&\leq (\dist_G(s, s_i) + \dist_G(s_i, c_i)) + (\dist_G(s_i, t) + \dist_G(s_i, c_i))\\
			&\leq \dist_G(s, t) + 2\cdot 2^{10/\epsilon}D^{1-1/k}.
		\end{aligned}$$
		Therefore, $\dist_{H_D}(s, t)\leq \dist_{H_D}(s, c_i) + \dist_{H_D}(c_i, t)\leq \dist_G(s, t) +101\cdot 2^{(30k-10)/\epsilon}D^{1-1/k}$.
		
		\item There did not exist any $i$ such that $s, t\in U_{c_i}$ at the moment.
		
		In this case, for each $1\leq j\leq l$, the pair $(s_j, t_j)$ is added to the set $\pset_{c_j}$ after this iteration. According to the algorithm, in the resulting graph $H_D$, for each $1\leq i\leq l$, there is a path $\phi_i$ in ${H_D}$ between $s_i, t_i$ such that $|\phi_i|\leq |\alpha_i| + 4\cdot 2^{(30k-20)/\epsilon}D^{1-2/k}$. By  \Cref{obs},
		$$\begin{aligned}
			\dist_{H_D}(s, t)&\leq \sum_{i=1}^l|\phi_i|\leq \sum_{i=1}^l\left(|\alpha_i| + 4\cdot 2^{(30k-20)/\epsilon}D^{1-2/k}\right)\\
			&\leq \dist_G(s, t) + 48\cdot 2^{(30k-10)/\epsilon}D^{1-1/k}.
		\end{aligned}$$
	\end{itemize}
\end{proof}

%Finally, we are able to bound the additive stretch between any pair of vertices.
In the following lemma, we complete the analysis of stretch of the graph $H_D$ on pairs of vertices in $G$ at distance at most $2D$.

\begin{lemma}
	For any pair of vertices $s, t\in V$ such that $\dist_G(s, t) < 2D$, we have: 
	$$\dist_{H_D}(s, t)\leq \dist_G(s, t) + 2^{30k/\epsilon}D^{1-1/k}.$$
\end{lemma}
\begin{proof}
	Let $\pi$ be an $s$-$t$ shortest path. We apply the subroutine \pathpart to path $\pi$ and the set $\balls$  of balls and obtain a partition $\pi = \alpha_1\circ\alpha_2\circ\cdots\circ\alpha_l$ and balls $\ball(c_1, r_{1}), \ball(c_2, r_{2}), \ldots, \ball(c_l, r_{l})$. If all these balls are small, note that for each  $1\leq i\leq l$, the subpath $\alpha_i$ lies entirely in $G[\ball(r_i, r_{i})]$, so there exists a path $\phi_i$ in $H_D$, such that
	$$\begin{aligned}
		|\phi_i| \leq |\alpha_i| + 2^{30(k-1)/\epsilon}\cdot |\alpha_i|^{1-1/(k-1)} &\leq |\alpha_i| + 2^{30(k-1)/\epsilon}\cdot \left(4\cdot 2^{10/\epsilon}\cdot D^{1-1/k}\right)^{1-1/(k-1)}\\
		&\leq |\alpha_i| + 4\cdot 2^{(30k-20)/\epsilon}D^{1-2/k}.
	\end{aligned}$$
	Since $l \leq \ceil{\frac{|\pi|}{D^{1-1/k}}} < 2D^{1/k} + 1 < 3D^{1/k}$, we get that $\dist_{H_D}(s, t)\leq \dist_G(s, t) + 12\cdot 2^{(30k-20)\epsilon}D^{1-1/k}$.
	
	Assume from now on that, some ball among $\ball(c_1, r_{1}), \ball(c_2, r_{2}), \ldots, \ball(c_l, r_{l})$ is large. Let $1\le x\le l$ be the smallest index of a large ball, and let $1\le y\le l$ be the largest index of a large ball. By the hitting set property, there exist $u, v\in S$ such that $u\in \ball(c_x, r_{x}), v\in \ball(c_y, r_{y})$. By triangle inequality,
	$$\dist_G(u, v)\leq \dist_G(c_x, c_y) + \dist_G(c_x, u) + \dist_G(c_y, v) < 2D + 4\cdot 2^{10/\epsilon}D^{1-1/k}.$$
	From \Cref{hitset}, $\dist_{H_D}(u, v)\leq \dist_G(u, v) + 101\cdot 2^{(30k-10)/\epsilon}D^{1-1/k}$.
	
	For each $j\in [1, x)\cup (y, l]$, since $\ball(c_j, r_{j})$ is small, there is a path $\phi_j$ in $H_D$ connecting $s_j$ to $t_j$, such that
	$$|\phi_j|\leq |\alpha_j| + 2^{30(k-1)/\epsilon}\cdot |\alpha_j|^{1-1/(k-1)} \leq |\alpha_j| + 4\cdot 2^{(30k-20)/\epsilon}D^{1-2/k}.$$
	%We define $\phi_x$ ($\phi_y$, resp.) to be the shortest path in $H_D$ connecting $s_x$ to $u$ ($s_y$ and $v$, resp.). 
	Then,
	$$\dist_{H_D}(s, u) + \dist_{H_D}(v, t)\leq \dist_G(s, u) + \dist_G(v, t) + 12\cdot 2^{(30k-10)/\epsilon}D^{1-1/k}.$$
	Finally, by triangle inequality,
	$$\begin{aligned}
		\dist_{H_D}(s, t) &\leq \dist_G(s, u) + \dist_G(u, v) + \dist_G(v, t) + 113\cdot 2^{(30k-10)/\epsilon}D^{1-1/k}\\
		&\leq (\dist_G(s, s_x) + \dist_G(s_x, u))\\
		&+(\dist_G(s_x, s_y) + \dist_G(s_x, u) + \dist_G(s_y, v))\\
		&+(\dist_G(s_y, t) + \dist_G(v, s_y)) + 113\cdot 2^{(30k-10)/\epsilon}D^{1-1/k}\\
		&\leq \dist_G(s, t) + 8\cdot 2^{10/\epsilon}D^{1-1/k} + 113\cdot 2^{(30k-10)/\epsilon}D^{1-1/k}\\
		&\leq \dist_G(s, t) + 2^{30k/\epsilon}D^{1-1/k}.
	\end{aligned}$$
\end{proof}

\subsection{Size analysis}

From the algorithm, the edge set of $H_D$ contains three types of edges that are calculated below.
\begin{enumerate}[(1),leftmargin=*]
	\item For each ball $\ball(c, r)\in \clusters$, graph $H_D$ contains a BFS tree $T_c$ that is rooted at $c$ and spans all vertices in $\ball(c, 4r)$. From \Cref{clustering}, $\sum |E(T_c)|=\sum |\ball(c, 4r)|=O(n^{1+\epsilon}/\eps)$.
	\item For each small ball $\ball(c, r)\in \clusters$, graph $H_D$ contains a sublinear additive spanner of the subgraph $G[\ball(c, 4r)]$ with stretch function $f_{k-1, \epsilon}$, which contains $O(|\ball(c, 4r)|^{1+10(k-1)\epsilon + \frac{1}{2^k-1}})$ edges by inductive hypothesis. Summing over all small balls, the number of edges in these spanners is at most (ignoring constant factors):
	$$\sum_{\ball(c, r)\text{ is small}}|\ball(c, 4r)|^{1+10(k-1)\epsilon + \frac{1}{2^k-1}}\leq n^{1+(10k-5)\epsilon}\cdot L_k^{\frac{1}{2^k-1}}.$$
	\item For each large ball $\ball(c, r)\in \clusters$, graph $H_D$ contains a pairwise sublinear additive spanner of the induced subgraph $G[\ball(c, 4r)]$ with respect to the set $\pset_c$ of pairs, with stretch function $f_{k-1, \epsilon}$. By \Cref{pairwise-sublinear} and  \Cref{demand-size}, the number of edges in such a spanner is at most  $$\tilde{O}\left(2^{2k/\epsilon}|\ball(c, 4r)|^{1+10(k-1)\epsilon}\cdot \left(\frac{n}{L_k}\right)^{\frac{1}{2^k}}\right).$$
	Summing over all large balls, the number of edges in all these spanners is at most
	$$\tilde{O}\left(\sum_{\ball(c, r)\text{ is large}}2^{2k/\epsilon}|\ball(c, 4r)|^{1+10(k-1)\epsilon}\cdot \left(\frac{n}{L_k}\right)^{\frac{1}{2^k}}\right)\leq 2^{2k/\epsilon}n^{1+(10k-5)\epsilon}\cdot \left(\frac{n}{L_k}\right)^{\frac{1}{2^k}}.$$
\end{enumerate}

%\begin{proof}[Proof of Theorem \ref{sublinear}]
Setting $L_k = n^{\frac{2^k-1}{2^{k+1}-1}}$, the total number of edges over all the above types is at most  $O(2^{2k/\epsilon}\cdot n^{1+(10k-5)\epsilon + \frac{1}{2^{k+1}-1}})$. Summing over all $D\in \set{1,2,\ldots,2^{\ceil{\log n}}}$, we can conclude that $|E(H)|\le O(n^{1+10k\epsilon + \frac{1}{2^{k+1}-1}})$.
%\end{proof}
