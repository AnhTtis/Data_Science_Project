
%For the construction of almost optimal sublinear additive spanners, we employ the same framework and reduce the task to constructing pairwise additive spanners. Observe that it suffices to construct, for each scale $D\in \set{1,2,2^2,\ldots,2^{\floor{\log n}}}$, a spanner $H_D$ of $G$ approximately preserving (to within an $O(D^{1-1/k})$ factor) all distances between pairs in $G$ that are at distance between $D$ and $2D$. We set $R=D^{1-1/k}$ and compute a clustering. Consider an unsettled pair $u,v$ of vertices at distance $O(D)$, and let $\pi$ be a shortest path connecting them. For simplification, we further assume that $\pi$ passes through $O(D^{1/k})$ other balls, and each ball contains a subpath of $\pi$ of length $O(D^{1-1/k})$. Buying this path gives each ball a demand pair. Note that in order to preserve the distance between $u,v$ to within $O(D^{1-1/k})$ factor, it suffices to preserve each demand pair to within $O(D^{1-2/k})$ factor. And since these demand pairs are at distance $O(R)=O(D^{1-1/k})$, within a ball, it suffices to preserve all pairs at the scale of $R$ to within $O(D^{1-2/k})=O(R^{1-1/(k-1)})$ factor. In other words, it suffices to construct, within each ball, a pairwise sublinear additive spanner of stretch $f(d)=d+O(d^{1-1/(k-1)})$ with respect to its demand pairs. We again use a path-buying approach similar to the one in \cite{kavitha2017new} to complete the task.

\begin{comment}
	A prominent technique used in previous work for computing additive spanners is the \emph{clustering and path buying} method \cite{baswana2010additive,kavitha2013small,cygan2013pairwise,knudsen2014additive,parter2014bypassing,kavitha2017new}.
	In the clustering phase, we compute a set of clusters (usually in some simple way) and add their edges to the spanner. Then in the path buying phase, we repeatedly identify ``cost-effective'' paths and ``buy'' them (add their edges) to complete the construction.
	Here the cost of a path is normally defined as the number of (new) edges that it contains, and the value of a path intuitively measures the progress towards the desired spanner by taking this path, for example, how many (new) pairs of vertices whose distance is strictly decreased or ``settled'' to the required range. 
	
	Previous work for constant stretch additive spanners used low-degree-vertex centered clusters. However, such clusters usually contain too many edges for linear-size spanners.  For constructing polynomial stretch additive spanners, Bodwin and Williams \cite{bodwin2016better} instead used sparse clusters with large diameter (close to the stretch) in the clustering phase, and then aim to settle pairs of clusters instead of pairs of vertices in the path buying phase. Our approach builds upon their method and yields improved bounds.
	
	In what follows we provide an overview of our approach.
	For convenience, we consider the following oversimplified setting. Let $G$ be the input graph. Assume that for a target radius $R$, we can compute a collection $\balls$ of balls in $G$, such that (i) every ball $\ball\in \balls$ contains all the vertices of some breath-first search (BFS) tree rooted at some center, and depth of the BFS tree is approximately $R$; (ii) the balls only share \emph{boundary vertices}, that are defined to be the leaves of the corresponding BFS tree, and the number of boundary vertices is at most $O(|\ball|/R)$; and (iii) the shortest paths in $G$ ``behave well'' with respect to the balls: specifically, the intersection between any shortest path $\pi$ and any ball in $\balls$ is a subpath of $\pi$.
	See Figure ?? for an illustration. Clearly, $\sum_{\ball\in \balls}|\ball|=O(n)$, so the union of the BFS trees from all balls contains $O(n)$ edges, and we add them all to the candidate spanner.
	
	With this clustering in hand, the task of constructing an additive spanner of the whole graph can be broken down into constructing (pairwise/subsetwise) additive spanners within balls in $\balls$.
	To see why this is true, assume we want to compute an $+O(n^{\alpha})$ additive spanner. 
	We just let $R=O(n^{\alpha})$, so for pairs of vertices lying in the same ball, their distance is at most $O(R)$ and the BFS tree inside the ball already spans them well. On the other hand, if a pair $v,v'\in V(G)$ with $v\in \ball, v'\in \ball'$ is well-spanned, or \emph{settled} (meaning that their distance in the current candidate spanner is already within $O(R)$ from their distance in $G$), then from triangle inequality, all other pairs $u,u'\in V(G)$ with $u\in \ball, u'\in \ball'$ are also well-spanned. Therefore, we just need to deal with pairs of balls that are not well-spanned. 
	Let $\ball,\ball'$ be such a pair and let $\pi$ be a shortest path from a vertex in $\ball$ to a vertex in $\ball'$. 
	Let $\ball_1,\ldots,\ball_{s-1}$ be all other balls that path $\pi$ goes through, that appear on $\pi$ in this order. Then if we decide to buy $\pi$ by adding edges of $\pi$ into our candidate spanner, then all pairs between $\set{\ball,\ball_1,\ldots,\ball_{s-1},\ball'}$ are settled.
	However, we do not really need to add the edges of $\pi$.
	In fact, we denote $\ball=\ball_0, \ball'=\ball_s$, and for each $0\le i< s$, by $v_i$ the last vertex of $\pi$ that lies in $\ball_i$, then it is easy to verify that, in the end, as long as we add to our spanner, for each $0\le i< s$, a shortest path $\phi_i$ in $\ball_i$ connecting $v_{i-1}$ to $v_{i}$, then the balls $\ball_0, \ldots,\ball_s$ are still guaranteed to be settled.
	In fact, $\phi_i$ does not need to be the shortest path, but can be any path in $\ball_i$ that preserves the distance $\dist_{\ball_i}(v_{i-1},v_{i})$ well enough.
	We view the pair $(v_{i-1},v_{i})$ as a \emph{demand pair}. So in other words, we keep buying paths until all pairs of balls are settled, but instead of directly taking their edges, we create demand pairs, and it is now sufficient to handle these demand pairs by constructing pairwise/subsetwise spanners/preservers with respect to them in corresponding balls.
	
	Previous results showed that $O(\sqrt{n})$ demand pairs can be preserved exactly with $O(n)$ edges \cite{coppersmith2006sparse}, and the bound $O(\sqrt{n})$ is tight, so we need to ensure that a ball $\ball$ collects at most $O(\sqrt{|\ball|})$ demand pairs. Note that we may ignore balls $\ball$ with $|\ball|\le R^{4/3}$. This is since the demand pairs must come from boundary vertices, and (we have assumed that) the number of them is at most $|\ball|/R\le |\ball|^{1/4}$, which gives at most $|\ball|^{1/2}$ pairs. For large balls $\ball$ with $|\ball|\ge R^{4/3}$, Bodwin and Williams \cite{bodwin2016better} placed an $+O(\sqrt{|\ball|})$ additive spanner \cite{bodwin2015very} inside them. This eventually implies that, whenever a new path $\pi$ is bought and a ball $\ball$ was passed through by $\pi$, along with the demand pair added to $\ball$, at least $R\cdot (R^{4/3}/R^{2/3})=R^{5/3}$ 
	new vertices\footnote{Say $\pi$ connects vertex $s$ to vertex $t$. Note that the $+O(\sqrt{n})$ spanners planted inside the balls settled a total of at least $R\cdot (R^{4/3}/R^{2/3})$ vertices with $s$ in the balls passed through by $\pi$, which we call the prefix of $\pi$; and similarly they also settled at least $R\cdot (R^{4/3}/R^{2/3})$ vertices with $s$ in the balls passed through by $\pi$, which we call the prefix of $\pi$. Also note that either the prefix or the suffix must be previously not settled with $\ball$ and now settled with $\ball$, as otherwise the pair $(s,t)$ was already settled and we should not take $\pi$.}
	are settled with $\ball$, and so a ball $\ball$ may collect at most $O(n/R^{5/3})$ pairs. When $R=n^{3/7}$, $O(n/R^{5/3})=O(n^{2/7})\le (R^{4/3})^{1/2}\le O(\sqrt{|\ball|})$.
	
	We proceed differently. We start by showing via the above framework that any graph $G$ and any subset $S\subseteq V(G)$ has a $+O(|S|^{3/2})$ subset spanner with respect to $S$, which is a subgraph $H\subseteq G$, such that for every pair $s,s'\in S$, $\dist_{H}(s,s')\le \dist_{G}(s,s')+O(|S|^{3/2})$. The algorithm is simple. Set $R=|S|^{3/2}$ and compute a clustering (and we can ignore balls $\ball$ with $|\ball|\le (|S|^{3/2})^{4/3}=|S|^2$); iteratively going through all shortest paths connecting pairs of vertices in $S$ and buy them whenever the endpoints are unsettled. Since whenever a cluster collects a new demand pair, it is settled with a new vertex in $S$, a ball $\ball$ gets at most $|S|=O(\sqrt{|\ball|})$ demand pairs. Now instead of planting an $+O(\sqrt{|\ball|})$ additive spanner inside each ball, we plant a $+O(|S|^{3/2})$ subset spanner inside each ball with respect to its boundary vertices. The stretch between them is now $O((|\ball|/R)^{3/2})$, which is less than $\sqrt{|\ball|}$ when $R^{4/3}<|\ball|<R^{3/2}$, this small advantage, together with some further utilization of the subset spanner, allows us to eventually improve their bound $O(n^{3/7+\eps})$ to $O(n^{0.403})$.
\end{comment}


\item Then, for each path $\rho\in \Sigma'$, we wish to apply the subroutine \pathpart to $\rho$ and the collection $\balls$ of balls, and obtain a partition of $\rho$ into subpaths and balls that host them. However, a direct implementation might not be time-efficient. Let us discuss how to efficiently implement a similar path partition subroutine in this scenario.
% Note that all these subpaths are also subpaths of $\pi_{s, t}$, and we call them \emph{segments}.

Assume the two endpoints of $\rho$ are $u, v$, respectively, and let $u_1$ be the second vertex of $\rho$ from $u$ to $v$. Then, let us describe an iterative procedure that partitions $\rho$ into sub-paths which are added to some buffer sets $\Delta\paths_c$. Denote by $x$ a vertex variable that moves along the path $\rho$ in the following way.

\begin{enumerate}[(a),leftmargin=*]
	\item Initially, set $x\leftarrow u_1$.
	
	\item In each iteration, find a ball $\ball(c, r)\in\bset$ that contains $x$. By definition of $\rho\in \Sigma^\prime$, we know that $\ball(c, r)$ must be large. Then, find the vertex $y$ closest to $v$ on $\pi_{s, t}$ such that $\pi_{s, t}[x, y]\subseteq \ball(c, 2r-1)$; this can be done greedily by a  sequential scan along $\pi_{s, t}$ starting with $x$ toward $v$.
	
	Then, if $x = u_1$, then add path $\pi_{s, t}[u, y]$ to $\Delta\paths_c$; otherwise, add path $\pi_{s, t}[x, y]$ to $\Delta\paths_c$.
	
	\item If $y = v$, then we are done with this sub-path $\pi_{s, t}$. Otherwise, reassign $x\leftarrow y$, and go back to Step (b).
\end{enumerate}