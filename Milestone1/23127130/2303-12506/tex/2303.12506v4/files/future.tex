
\section{Conclusion and Future Work}
\label{sec:future}
% Whenever a new solution concept is defined, one of the most intuitive questions is how it can be related to other properties? 
We presented a practical solution to the problem of leximin optimization when only an approximate single-objective solver is available. 
The algorithm is guaranteed to terminate in polynomial time, and its approximation ratio degrades gracefully as a function of the approximation ratio of the single-objective solver.

Currently, our algorithm handles two main settings. First, when inaccuracies in the single-objective solver stem from numeric errors.
Second, when the problem is convex and satisfy several assumptions.
It may be interesting to study more settings in which the inaccuracies stem from computational hardness of the single-objective problem.
% Currently, our algorithm handles settings in which the inaccuracies in the single-objective solver stem from numeric errors.
% It may be interesting to study settings in which the inaccuracies stem from computational hardness of the single-objective problem.
%
%

% identify problems in which an appropriate approximate solver can be designed. 
In particular, to approximate the egalitarian welfare, it is common to model the problem as an integer program or as an exponential sized linear program (e.g., \cite{bansal2006santa, kawase_max-min_2020}) and then approximate the program using different techniques.
% rounding techniques or methods for convex optimization (such as the ellipsoid method).
Can these algorithms be generalized to consider the additional constraints described in Section \ref{sec:algo-short}? This will allow approximating leximin using the approach in this paper.
% In particular, in the problem of stochastic allocations (in Section \ref{sec:app}), to extend the approximation algorithm for the egalitarian welfare, we had to change some steps within.
% What if an algorithm for egalitarian welfare is provided as a black box --- could it be used to design the appropriate procedure to approximate leximin?

% In the context of fair division, this study assumes that there is an access to the true valuations of the agents involved. 
% In reality, people may lie about their valuations.
% Can our definition of approximate-leximin be related to some approximate version of truthfulness?

Another question is whether it is possible to obtain a better approximation factor for leximin, given an $(\multApprox, \additiveApprox)$-approximation algorithm for the single-objective problem.
Specifically, can an $(\multApprox, \additiveApprox)$-approximation to leximin can be obtained in polynomial time? 
If not, what would be the best possible approximation in this case?
% \erel{Mention the tightness of our results}


\iffalse % EREL: removed for the submission. To clarify later
Further, the algorithm suggested in Section \ref{sec:algo-short} tend to work very well if the single-objective optimization problems are convex, and in particular if they are linear programs. 
\erel{Why? the algorithm of \textcite{Ogryczak2004TelecommunicationsND} works even for non-convex programs.}
Can we find an algorithm that works with general approximation algorithms? For example, naive algorithms such as \emph{Next-fit}?
\fi

% \begin{itemize}
%     \item Meaning of approximately leximin in the context of other characteristics like truthfulness.
    
%     \item Solving more problems.
    
%     \item Is it possible to obtained a better approximation factor for leximin maximization in polynomial time? given that $(1-\beta)$ is the best possible for the egalitarian maximization, is it possible to obtain a $(1-\beta)$ approximately leximin optimal solution? what is the best possible approximation  in this case?
    
%     \item The algorithm works very well if the single-objective problem is convex, in particular if it is a linear programs, but can it be modify to work with general approximation algorithms? such as algorithms for makespan minimization?
% \end{itemize}