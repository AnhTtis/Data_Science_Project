

\section{Introduction}
% title: real life motivation
Many real life scenarios involve more than one objective. 
These situations are often modeled as \emph{multi-objective optimization problems}. These are defined by the set of possible decisions, along with functions that describe the different objectives.
As a concrete example, we use the context of social choice, in which the objective functions represent people's utilities.
Different criteria can be used to determine optimality when considering multi-objectives. 
For example, the \emph{utilitarian} criterium aims to maximize the sum of utilities, whereas the \emph{egalitarian} criterium aims to maximize the least utility.
This paper focuses on the \emph{leximin} criterium, according to which one aims to maximize the least utility, and, subject to this, maximize the second-smallest utility, and so on.
In the context of social choice, 
the leximin criterium is considered a criterium of fairness,
as strives to benefit, as much as possible, the least fortunate in society.

Common algorithms for finding a leximin optimal solution are iterative, optimizing one or more single-objective optimization problems at each iteration (for example~\cite{airiau_portioning_2019,arbiv_fair_2022,bei_truthful_2022, nace_max-min_2008,Ogryczak2004TelecommunicationsND,willson}).
Often, these single-objective problems cannot be solved exactly (e.g. when they are computationally hard, or when there are numeric inaccuracies in the solver), but can be solved approximately.
In this work, we define an approximate variant of leximin and show how such an approximation can be computed, given approximate single-objective solvers.

\paragraph{The Challenge}
When single-objective solvers only \emph{approximate} the optimal value, existing methods for extending the solvers to leximin optimally may fail, as we illustrate next.

A common algorithm,
independently proposed many times, e.g.~\cite{airiau_portioning_2019,nace_max-min_2008,Ogryczak2004TelecommunicationsND, willson}, is based on the notion of \textit{saturation}. It operates roughly  as follows.
In the first iteration, the algorithm looks for the maximum value that all objective functions can achieve simultaneously, denoted $z_1$. Then, it determines which of the objective functions are saturated --- that is, cannot achieve more than $z_1$ given that the others achieve at least $z_1$.
Afterwards, in each iteration $t$, given that for any $i<t$ the objective-functions that were determined saturated in the $i$'th iteration achieve at least $z_i$, it looks for the maximum value that all other objective-functions can achieve simultaneously, denoted $z_t$, and then determines which of those functions are saturated.
When all functions become saturated, the algorithm ends.

%As an example of how the algorithm works, consider the following problem:
%\begin{align*}
%    \lexmaxmin \quad &\{f_1(x) = x_1, f_2(x)=x_2, f_3(x) = x_3\}\\
%        s.t. \quad &(1) \quad x_1 \leq 1, \text{ }\Lquad \Hquad\Hquad\quad 
 %       (3) \quad x_2 + x_3 \leq 5 \\
  %      &(2) \quad x_1 + x_2 \leq 3, \Lquad      (4) \quad x \in \mathbb{R}^3_{+} 
%\end{align*}
  
%In the first iteration, the algorithm finds that the maximum value all objective functions can achieve simultaneously is $1$.
%The optimal value of $f_1$ given that the others ($f_2$ and $f_3$) achieve at least $1$ is $1$, where the optimal values of $f_2$ and $f_3$ are $2$ and $4$ respectively. 
%Therefore, only $f_1$ is determined saturated with a saturation value $z_1 =1$.
%In the second iteration, we find that when $f_1$ achieves at least $1$, the maximum value that $f_2$ and $f_3$ can achieve simultaneously is $2$.
%The optimal value of $f_2$ given that $f_1$ achieves at least $1$ and $f_3$ achieve at least $2$ is $2$, and the optimal value of $f_3$ given that $f_1$ achieves at least $1$ and $f_2$ achieve at least $2$ is $3$.
%And so, $f_2$ is determined saturated with saturation value $z_2 = 2$.
%At the third iteration, we find that when $f_1$ achieves at least $1$ and $f_2$ achieves at least $2$, the maximum value that $f_3$ can achieve is $3$, and as it is also its optimal value, $f_3$ is determined saturated with saturation value $z_3 = 3$.
%We get that the leximin optimal solution to this problem is $x = (1,2,3)$.

The following simple example  demonstrates the problem that may arise when the individual solver may return sub-optimal results. Consider the following problem:
\begin{align*}
    \lexmaxmin \quad &\{f_1(x) = x_1, f_2(x)=x_2\} \\
    s.t. \quad  &(1) \Hquad x_1 + x_2 \leq 1, \quad (2) \Hquad x \in \mathbb{R}^2_{+}
\end{align*}
As $f_1$ and $f_2$ are symmetric, the leximin optimal solution in this case is $(0.5,0.5)$.
Now suppose that rather than finding the exact value $0.5$, the solver returns the value $0.49$. 
The optimal value of $f_1$ given that $f_2$ achieves at least $0.49$ is $0.51$, and vice versa for $f_2$.
As a consequence, none of the objective functions would be determined saturated, and the algorithm may not terminate. 
One could perhaps define an objective as ''saturated'' if its maximum attainable value is close to the maximum $z_t$, but there is no guarantee that this would lead to a good approximation\footnote{An example will be given on request.}
% \eden{ (for more details see Appendix \ref{}\fullVer)}.
%
% EREL: In fact, we have examples showing that it will not lead to a good approximation. It may be good to mention them here.
% \eden{to add an example and reference to it}

% The algorithm suggested in this paper returns an appropriate approximation even in the presence of errors.
% EREL: It is not suggested in this paper.
% EDEN: maybe: suggested for porpuse?

\paragraph{Contributions}
This paper studies the problem of leximin optimization in multi-objective optimization problems, 
focusing on problems for which even the single-objective problems cannot be solved exactly in polynomial time.
\eden{Was: 'Our contribution is threefold.', should it be rewritten differently for two? maybe: 'We present two key contributions in this research.'}
Our contribution is twofold.


First, a new definition of leximin approximation is presented.
It captures both multiplicative and additive errors.
The definition has several desirable properties, including that a leximin optimal solution is always approximately-optimal,
% (for any approximation factor),
and that the definition is equivalent to the original one in the absence of errors.

Second, an algorithm is provided that, given an approximation algorithm for a single-objective problem, computes a leximin approximation to the multi-objective problem.
The algorithm was first presented by Ogryczak
and {\'{S}}liwi{\'{n}}ski~\cite{Ogryczak_2006} for exact leximin-optimization. In contrast to the saturation-based algorithm described in the Introduction, this algorithm always terminates, even when the single-objective solver is inaccurate. 
Moreover, the accuracy of the returned solution is closely correlated with the accuracy of the single-objective solver --- given an $(\multApprox,\additiveApprox)$-approximation algorithm for the single-objective problem (where $\multApprox$ and $\additiveApprox$ describe the multiplicative and additive factors respectively), the returned solution is an $\left(\frac{\multApprox^2}{1-\multApprox + \multApprox^2}, \frac{\additiveApprox}{1-\multApprox +\multApprox^2}\right)$-approximation of leximin.
Importantly, this holds for any number of objectives.
% It should also be noted that the algorithm is applicable in many cases, for example, when the single-objective solver is inaccurate due to numeric issues, e.g. floating-point rounding errors, or iterative methods that are stopped before complete convergence due to time constraints. 

Our algorithm is applicable whenever a single-objective solver is inaccurate due to numerical issues, such as floating-point rounding errors, or stopping an iterative solution procedure before complete convergence.
% Another application is when the feasible region is convex and all the objectives are concave (and polynomially computable) --- where under several assumptions it can be approximated to any wanted factor.
%
%
% \eden{I remove it for now....... but maybe it would be better to keep it and just change it a bit....\\
% It should also be noted that the algorithm is applicable in many cases, for example, when the feasible region is convex and all the objectives are concave (and polynomially computable).}


\paragraph{Organization} 
Section \ref{sec:preliminaries} gives preliminary knowledge and basic definitions. 
Section \ref{sec:approx-leximin-def} presents the definition of leximin approximation.
An algorithm for computing such an approximation is presented in Section \ref{sec:algo-short}.
Section \ref{sec:future} concludes with some future work directions.



\section{Related Work}\label{sec:related-work}



Over the last few years, several benchmarks for stream processing frameworks have been proposed and stream processing benchmarking studies have been conducted. The differentiation between benchmarks and experimental studies applying them is sometimes blurry. Many publications that present benchmarks perform also an experimental study with them. On the other hand, many experimental studies utilize existing benchmarks, but modify them.
Nevertheless, we structure this section into two parts: First, we give an overview of stream processing benchmarks to justify our benchmark selection for this study. Second, we discuss related stream processing benchmarking studies.

\subsection{Related Work on Stream Processing Benchmarks}

Besides the Theodolite benchmarks for event-driven microservices used in this study, several other benchmarks for stream processing frameworks have been proposed.
\cref{tab:related-benchmarks} summarizes characteristics of the discussed benchmarks. 


\begin{table*}
	\begin{threeparttable}[b]
		\caption{Overview of the characteristics and implementations of stream processing benchmarks.}
		\label{tab:related-benchmarks}
		\footnotesize
		\newcommand{\cmark}{\ding{51}}%
		\newcommand{\xmark}{\ding{55}}%
		\newcommand{\qmark}{\makebox[0pt][l]{\textbf{\textit{?}}}\phantom{\cmark}}%
		
		\newcommand{\txnote}[1]{\makebox[0pt][l]{\tnote{#1}}}
		
		\newcommand\undefcolumntype[1]{\expandafter\let\csname NC@find@#1\endcsname\relax}
		\newcommand\forcenewcolumntype[1]{\undefcolumntype{#1}\newcolumntype{#1}}
		
		\newcommand*\rot{\rotatebox{90}}
		\newcolumntype{L}{>{\raggedright\arraybackslash}X}
		\newcolumntype{R}{>{\raggedleft\arraybackslash}X}
		\newcolumntype{C}{>{\centering\arraybackslash}X}
		\newcolumntype{o}{p{0pt}}
		\renewcommand{\arraystretch}{1.2}
		\newcommand{\fnoptional}{a}
		\newcommand{\fnbeam}{b}
		\newcommand{\fnriottasksamples}{d}
		\newcommand{\fnbeamnexmark}{c}
		\begin{tabularx}{\textwidth}{ll o C o C o CCC o CCCCCCC o C o CCC}
			\toprule
			&&& && && \multicolumn{3}{c}{Messaging} && \multicolumn{7}{c}{Stream processing framework} && && \multicolumn{3}{c}{Cloud-native} \\
			\cmidrule{8-10} \cmidrule{12-18} \cmidrule{22-24}
			Benchmark & Published && \rot{Task samples} && \rot{Open source} && \rot{Kafka} & \rot{Others} & \rot{None} && \rot{Flink} & \rot{Spark} & \rot{Storm} & \rot{Samza} & \rot{Kafka Streams} & \rot{Hazelcast Jet} & \rot{Others} && \rot{Database} && \rot{Containers} & \rot{Kubernetes} & \rot{Others} \\
			\midrule
			Theodolite \cite{BDR2021} & \citeyear{BDR2021}
			& %
			& 4
			& %
			& \cmark %
			& %
			& \cmark %
			& %
			& %
			& %
			& \cmark %
			& %
			& %
			& \cmark\txnote{\fnbeam} %
			& \cmark %
			& \cmark %
			& \cmark\txnote{\fnbeam} %
			& %
			& \phantom{\cmark}\txnote{\fnoptional} %
			& %
			& \cmark %
			& \cmark %
			& %
			\\
			Beam Nexmark \cite{BeamNexmark2022} & \citeyear{BeamNexmark2022}\txnote{\fnbeamnexmark}
			& %
			& 13
			& %
			& \cmark %
			& %
			& \cmark %
			& \cmark %
			& %
			& %
			& \cmark\txnote{\fnbeam} %
			& \cmark\txnote{\fnbeam} %
			& %
			& \qmark\txnote{\fnbeam} %
			& %
			& \qmark\txnote{\fnbeam} %
			& \cmark\txnote{\fnbeam} %
			& %
			& %
			& %
			& %
			& %
			& %
			\\
			ESPBench \cite{Hesse2021} & \citeyear{Hesse2021}
			& %
			& 5
			& %
			& \cmark %
			& %
			& \cmark %
			& %
			& %
			& %
			& \cmark\txnote{\fnbeam} %
			& \cmark\txnote{\fnbeam} %
			& %
			& \qmark\txnote{\fnbeam} %
			& %
			& \cmark\txnote{\fnbeam} %
			& \qmark\txnote{\fnbeam} %
			& %
			& \cmark %
			& %
			& %
			& %
			& %
			\\
			OSPBench \cite{vanDongen2020} & \citeyear{vanDongen2020}
			& %
			& 5
			& %
			& \cmark %
			& %
			& \cmark %
			& %
			& %
			& %
			& \cmark %
			& \cmark %
			& %
			& %
			& \cmark %
			& %
			& %
			& %
			& %
			& %
			& \cmark %
			& %
			& \cmark %
			\\
			DSPBench \cite{Bordin2020} & \citeyear{Bordin2020}
			& %
			& 5
			& %
			& \cmark %
			& %
			& \cmark %
			& %
			& %
			& %
			& %
			& \cmark %
			& \cmark %
			& %
			& %
			& %
			& %
			& %
			& \cmark %
			& %
			& %
			& %
			& %
			\\
			\citet{Shahverdi2019} & \citeyear{Shahverdi2019}
			& %
			& 1
			& %
			& \cmark %
			& %
			& \cmark %
			& %
			& %
			& %
			& \cmark %
			& \cmark %
			& \cmark %
			& %
			& \cmark %
			& \cmark %
			& %
			& %
			& \cmark %
			& %
			& %
			& %
			& %
			\\
			\citet{Karimov2018} & \citeyear{Karimov2018}
			& %
			& 2
			& %
			& %
			& %
			& %
			& %
			& \cmark %
			& %
			& \cmark %
			& \cmark %
			& \cmark %
			& %
			& %
			& %
			& %
			& %
			& %
			& %
			& %
			& %
			& %
			\\
			RIoTBench \cite{Shukla2017} & \citeyear{Shukla2017}
			& %
			& 4\txnote{\fnriottasksamples} %
			& %
			& \cmark %
			& %
			& %
			& \cmark %
			& %
			& %
			& %
			& %
			& \cmark %
			& %
			& %
			& %
			& %
			& %
			& \cmark %
			& %
			& %
			& %
			& %
			\\
			YSB \cite{Chintapalli2016} & \citeyear{Chintapalli2016}
			& %
			& 1
			& %
			& \cmark %
			& %
			& \cmark %
			& %
			& %
			& %
			& \cmark %
			& \cmark %
			& \cmark %
			& %
			& %
			& %
			& %
			& %
			& \cmark %
			& %
			& %
			& %
			& %
			\\
			SparkBench \cite{Li2015} & \citeyear{Li2015}
			& %
			& 10
			& %
			& \cmark %
			& %
			& %
			& %
			& \cmark %
			& %
			& %
			& \cmark %
			& %
			& %
			& %
			& %
			& %
			& %
			& %
			& %
			& %
			& %
			& %
			\\
			StreamBench \cite{Lu2014} & \citeyear{Lu2014}
			& %
			& 7
			& %
			& %
			& %
			& \cmark %
			& %
			& %
			& %
			& %
			& \cmark %
			& \cmark %
			& %
			& %
			& %
			& %
			& %
			& %
			& %
			& %
			& %
			& %
			\\
			Linear Road \cite{Arasu2004} & \citeyear{Arasu2004}
			& %
			& 5
			& %
			& %
			& %
			& %
			& %
			& \cmark %
			& %
			& %
			& %
			& %
			& %
			& %
			& %
			& \cmark %
			& %
			& %
			& %
			& %
			& %
			& %
			\\
			\bottomrule
		\end{tabularx}
		\begin{tablenotes}\footnotesize
			\item[\fnoptional] optional
			\item[\fnbeam] using Apache Beam
			\item[\fnbeamnexmark] the Beam Nexmark benchmarks are based on the Nexmark paper \cite{Tucker2010} published in \citeyear{Tucker2010}
			\item[\fnriottasksamples] RIoTBench's 4 application benchmarks are composed of 27 microbenchmarks
		\end{tablenotes}
	\end{threeparttable}
\end{table*}



StreamBench~\cite{Lu2014} is one of the earliest benchmarks for modern stream processing frameworks. While originally only implemented for Spark and Storm, it has later been used to benchmark Apache Apex, Beam, Flink, and Samza as well \cite{Hesse2019, Qian2016}.
As its name suggests, SparkBench~\cite{Li2015} is a benchmark tailored to Apache Spark.
The Yahoo Streaming Benchmark (YSB) \cite{Chintapalli2016} is frequently used and adapted in research \cite{Lopez2016, Yang2017, Karakaya2017, Nasiri2019, Zeuch2019, Chu2020, vanDongen2020}.
Worth highlighting is the work of \citet{Shahverdi2019}, who extend YSB with implementations for the frameworks Kafka Streams and Hazelcast Jet. As discussed in \cref{sec:frameworks}, these frameworks are particularly suited for building event-driven microservices.
RIoTBench \cite{Shukla2017} provides four application benchmarks for Storm composed of 27~small task samples. \citet{Nasiri2019} adopt RIoTBench for Flink and Spark.
\citet{Karimov2018} present a benchmark with two task samples, derived from a real industrial context, yet without providing open-source implementations.

More recently, DSPBench \cite{Bordin2020}, OSPBench~\cite{vanDongen2020}, and ESPBench \cite{Hesse2021} have been proposed.
DSPBench contains 15~benchmarks, which resample typical stream processing applications, derived from reviewing the literature.
OSPBench provides benchmarks for analyzing traffic sensor data. Besides evaluations of latency, throughput, and resource usage, \citeauthor{vanDongen2020} used OSPBench to also evaluate scalability~\cite{vanDongen2021b} and fault recovery~\cite{vanDongen2021a}.
In contrast to most other benchmarks, OSPBench provides implementations for the rather new framework Kafka Streams, which is also evaluated in this study.
The Enterprise Stream Processing Benchmark (ESPBench) builds upon the Senska benchmark \cite{Hesse2018}.
It is special in the sense that it integrates a relational database management system.
In contrast to most other benchmarks, ESPBench's task samples are implemented with Apache Beam. While \citet{Hesse2021} only perform evaluations with Spark, Flink, and Hazelcast Jet, we expect that also other Beam runners can be used to run the benchmark.

The Nexmark benchmark \cite{Tucker2010} has originally been proposed as the \textit{Niagara Extension to the XMark benchmark} addressed to first-generation stream processing systems (see the survey of \citet{Fragkoulis2023} for a discussion of first and second-generation stream processing systems).
The Apache Beam community adapted and extended Nexmark with implementations for Beam to benchmark the performance of different runners~\cite{BeamNexmark2022}.
Documentation and benchmark results are provided for the direct runner as well as for the Flink, the Spark, and the Google Cloud Dataflow runners.
However, running the benchmark with other runners should be possible as well.
Recently, there seems to be an effort to implement the Nexmark task samples with other frameworks in an open-source project.\footnote{\url{https://github.com/nexmark/nexmark}}
However, currently this project only provides implementations for Apache Flink.
Moreover, \citet{Gencer2021} implemented the Nexmark benchmark for their performance evaluation of Hazelcast Jet.

Worth mentioning is also the Linear Road benchmark presented by \citet{Arasu2004}. Although published years before all modern stream processing frameworks considered in this work have been released, it is still used in research \cite{Zhang2017,Zeuch2019,Sax2020} and compared to newer benchmarks \cite{Bordin2020,Hesse2021}.
\citet{Pagliari2020} and \citet{Garcia2022a, Garcia2022b} present approaches to generate benchmarks.






From \cref{tab:related-benchmarks}, we can see that a lot of open-source benchmarks have been proposed. Apart from the Theodolite benchmarks, none of these benchmarks is particularly addressed to scalability.
Often originating in data management research, many benchmarks are defined as ``queries'' over data streams~\cite{Tucker2010,Karimov2018,Hesse2021}.
Most benchmarks include a messaging system as a middleware component between workload generation and stream processing framework. In the vast majority of cases, this is Apache Kafka.
\citet{Karimov2018} exclude such a system to not let it become the benchmark's bottleneck. Our Theodolite benchmarks purposely include Kafka to represent more realistic event-driven microservice deployments~\cite{BDR2021}.
Flink, Spark, and Storm are by far the most supported frameworks. Only a few benchmarks exist for Samza, Kafka Streams, and Hazelcast Jet, which are frameworks particularly suited for implementing event-driven microservice. Our Theodolite benchmarks are the only ones providing implementations for all of them.
While some benchmarks include an interaction with a database in their setup, others do not.
With the Theodolite benchmarks, a database can optionally be used as we did in a previous study~\cite{IC2E2022FaaSStreaming}.
Besides our Theodolite benchmarks, there is only one other benchmark (OSPBench) that is provided as container images to be used in a cloud-native setting. No other benchmark provides Kubernetes manifests.





\subsection{Related Work on Stream Processing Benchmarking}


\begin{table*}
	\begin{threeparttable}[b]
		\caption{Overview of employed benchmarks, frameworks, and experimental setup of stream processing benchmarking studies.}
		\label{tab:related-experiments}
		\footnotesize
		\newcommand{\cmark}{\ding{51}}%
		\newcommand{\xmark}{\ding{55}}%
		\newcommand{\qmark}{\makebox[0pt][l]{\textbf{\textit{?}}}\phantom{\cmark}}%
		
		\newcommand{\txnote}[1]{\makebox[0pt][l]{\tnote{#1}}}
		
		\newcommand\undefcolumntype[1]{\expandafter\let\csname NC@find@#1\endcsname\relax}
		\newcommand\forcenewcolumntype[1]{\undefcolumntype{#1}\newcolumntype{#1}}
		
		\newcommand*\rot{\rotatebox{90}}
		\newcolumntype{L}{>{\raggedright\arraybackslash}X}
		\newcolumntype{R}{>{\raggedleft\arraybackslash}X}
		\newcolumntype{C}{>{\centering\arraybackslash}X}
		\newcolumntype{o}{p{0pt}}
		\renewcommand{\arraystretch}{1.2}
		\newcommand{\fnvandenpoel}{a}
		\newcommand{\fnbeam}{b}
		\begin{tabularx}{\textwidth}{ll o CCCCCCCCCCCCC o CCCCCCC o CCCCCC}
			\toprule
			&&& \multicolumn{13}{c}{Benchmark} && \multicolumn{7}{c}{Framework} && \multicolumn{6}{c}{Execution} \\
			\cmidrule{4-16} \cmidrule{18-24} \cmidrule{26-31}
			Publication & Year &&
			\rot{Theodolite \cite{BDR2021}} &
			\rot{Beam Nexmark \cite{BeamNexmark2022}} &
			\rot{ESPBench \cite{Hesse2021}} &
			\rot{OSPBench \cite{vanDongen2020}} &
			\rot{DSPBench \cite{Bordin2020}} &
			\rot{\citet{Shahverdi2019}} &
			\rot{\citet{Karimov2018}} &
			\rot{RIoTBench \cite{Shukla2017}} &
			\rot{YSB \cite{Chintapalli2016}} &
			\rot{SparkBench \cite{Li2015}} &
			\rot{StreamBench \cite{Lu2014}} &
			\rot{Linear Road \cite{Arasu2004}} &
			\rot{Others}
			&&
			\rot{Flink} &
			\rot{Spark} &
			\rot{Storm} &
			\rot{Samza} &
			\rot{Kafka Streams} &
			\rot{Hazelcast Jet} &
			\rot{Others}
			&&
			\rot{Cloud environment} &
			\rot{Distributed} &
			\rot{Different resource amounts} &
			\rot{\dots in isolated experiments} &
			\rot{Different load intensities} &
			\rot{\dots in isolated experiments}
			\\
			\midrule
			This work &
				& %
				& \cmark %
				& %
				& %
				& %
				& %
				& %
				& %
				& %
				& %
				& %
				& %
				& %
				& %
				& %
				& \cmark %
				& %
				& %
				& \cmark\txnote{\fnbeam} %
				& \cmark %
				& \cmark %
				& %
				& %
				& \cmark %
				& \cmark %
				& \cmark %
				& \cmark %
				& \cmark %
				& \cmark %
			\\
			\citet{IC2E2022FaaSStreaming} & \citeyear{IC2E2022FaaSStreaming}
				& %
				& \cmark %
				& %
				& %
				& %
				& %
				& %
				& %
				& %
				& %
				& %
				& %
				& %
				& %
				& %
				& \cmark\txnote{\fnbeam} %
				& %
				& %
				& \cmark\txnote{\fnbeam} %
				& %
				& %
				& \cmark\txnote{\fnbeam} %
				& %
				& \cmark %
				& \cmark %
				& \cmark %
				& \cmark %
				& \cmark %
				& \cmark %
			\\
			\citet{Hesse2021} & \citeyear{Hesse2021}
				& %
				& %
				& %
				& \cmark %
				& %
				& %
				& %
				& %
				& %
				& %
				& %
				& %
				& %
				& %
				& %
				& \cmark\txnote{\fnbeam} %
				& \cmark\txnote{\fnbeam} %
				& %
				& %
				& %
				& \cmark\txnote{\fnbeam} %
				& %
				& %
				& %
				& \cmark %
				& \cmark %
				& \cmark %
				& %
				& %
			\\
			van Dongen\tnote{\fnvandenpoel} \cite{vanDongen2021b} & \citeyear{vanDongen2021b}
				& %
				& %
				& %
				& %
				& \cmark %
				& %
				& %
				& %
				& %
				& %
				& %
				& %
				& %
				& %
				& %
				& \cmark %
				& \cmark %
				& %
				& %
				& \cmark %
				& %
				& %
				& %
				& \cmark %
				& \cmark %
				& \cmark %
				& %
				& \cmark %
				& \cmark %
			\\
			van Dongen\tnote{\fnvandenpoel} \cite{vanDongen2021a} & \citeyear{vanDongen2021a}
				& %
				& %
				& %
				& %
				& \cmark %
				& %
				& %
				& %
				& %
				& %
				& %
				& %
				& %
				& %
				& %
				& \cmark %
				& \cmark %
				& %
				& %
				& \cmark %
				& %
				& %
				& %
				& \cmark %
				& \cmark %
				& %
				& %
				& \cmark %
				& %
			\\
			\citet{Bordin2020} & \citeyear{Bordin2020}
				& %
				& %
				& %
				& %
				& %
				& \cmark %
				& %
				& %
				& %
				& %
				& %
				& %
				& %
				& %
				& %
				& %
				& \cmark %
				& \cmark %
				& %
				& %
				& %
				& %
				& %
				& \cmark %
				& \cmark %
				& %
				& %
				& \cmark %
				& \cmark %
			\\
			\citet{Chu2020} & \citeyear{Chu2020}
				& %
				& %
				& %
				& %
				& %
				& %
				& %
				& %
				& %
				& \cmark %
				& %
				& %
				& %
				& \cmark %
				& %
				& \cmark %
				& %
				& \cmark %
				& %
				& %
				& %
				& \cmark %
				& %
				& %
				& \cmark %
				& \cmark %
				& %
				& %
				& %
			\\
			\citet{Vikash2020} & \citeyear{Vikash2020}
				& %
				& %
				& %
				& %
				& %
				& %
				& %
				& %
				& %
				& %
				& %
				& %
				& %
				& \cmark %
				& %
				& \cmark %
				& \cmark %
				& \cmark %
				& %
				& %
				& %
				& \cmark %
				& %
				& %
				& \cmark %
				& %
				& %
				& \cmark %
				& \cmark %
			\\
			van Dongen\tnote{\fnvandenpoel} \cite{vanDongen2020} & \citeyear{vanDongen2020}
				& %
				& %
				& %
				& %
				& \cmark %
				& %
				& %
				& %
				& %
				& %
				& %
				& %
				& %
				& %
				& %
				& \cmark %
				& \cmark %
				& %
				& %
				& \cmark %
				& %
				& %
				& %
				& \cmark %
				& \cmark %
				& \cmark %
				& %
				& %
				& %
			\\
			\citet{Nasiri2019} & \citeyear{Nasiri2019}
				& %
				& %
				& %
				& %
				& %
				& %
				& %
				& %
				& \cmark %
				& \cmark %
				& %
				& %
				& %
				& %
				& %
				& \cmark %
				& \cmark %
				& \cmark %
				& %
				& %
				& %
				& %
				& %
				& %
				& \cmark %
				& \cmark %
				& \cmark %
				& \cmark %
				& \cmark %
			\\
			\citet{Shahverdi2019} & \citeyear{Shahverdi2019}
				& %
				& %
				& %
				& %
				& %
				& %
				& \cmark %
				& %
				& %
				& %
				& %
				& %
				& %
				& %
				& %
				& \cmark %
				& \cmark %
				& \cmark %
				& %
				& \cmark %
				& \cmark %
				& %
				& %
				& \cmark %
				& \cmark %
				& \cmark %
				& \cmark %
				& %
				& %
			\\
			\citet{Zeuch2019} & \citeyear{Zeuch2019}
				& %
				& %
				& %
				& %
				& %
				& %
				& %
				& %
				& %
				& \cmark %
				& %
				& %
				& \cmark %
				& \cmark %
				& %
				& \cmark %
				& \cmark %
				& \cmark %
				& %
				& %
				& %
				& \cmark %
				& %
				& %
				& \cmark %
				& %
				& %
				& \cmark %
				& \cmark %
			\\
			\citet{Karimov2018} & \citeyear{Karimov2018}
				& %
				& %
				& %
				& %
				& %
				& %
				& %
				& \cmark %
				& %
				& %
				& %
				& %
				& %
				& %
				& %
				& \cmark %
				& \cmark %
				& \cmark %
				& %
				& %
				& %
				& %
				& %
				& %
				& \cmark %
				& \cmark %
				& %
				& \cmark %
				& \cmark %
			\\
			\citet{Truong2018} & \citeyear{Truong2018}
				& %
				& %
				& %
				& %
				& %
				& %
				& %
				& %
				& %
				& %
				& %
				& %
				& %
				& \cmark %
				& %
				& %
				& %
				& %
				& %
				& %
				& %
				& \cmark %
				& %
				& \cmark %
				& \cmark %
				& %
				& %
				& \cmark %
				& \cmark %
			\\
			\citet{Karakaya2017} & \citeyear{Karakaya2017}
				& %
				& %
				& %
				& %
				& %
				& %
				& %
				& %
				& %
				& \cmark %
				& %
				& %
				& %
				& %
				& %
				& \cmark %
				& \cmark %
				& \cmark %
				& %
				& %
				& %
				& %
				& %
				& %
				& \cmark %
				& %
				& %
				& \cmark %
				& \cmark %
			\\
			\citet{Shukla2017} & \citeyear{Shukla2017}
				& %
				& %
				& %
				& %
				& %
				& %
				& %
				& %
				& \cmark %
				& %
				& %
				& %
				& %
				& %
				& %
				& %
				& %
				& \cmark %
				& %
				& %
				& %
				& %
				& %
				& \cmark %
				& \cmark %
				& \cmark %
				& %
				& %
				& %
			\\
			\citet{Yang2017} & \citeyear{Yang2017}
				& %
				& %
				& %
				& %
				& %
				& %
				& %
				& %
				& %
				& \cmark %
				& %
				& %
				& %
				& \cmark %
				& %
				& \cmark %
				& \cmark %
				& \cmark %
				& %
				& %
				& %
				& %
				& %
				& \cmark %
				& \cmark %
				& %
				& %
				& %
				& %
			\\
			\citet{Chintapalli2016} & \citeyear{Chintapalli2016}
				& %
				& %
				& %
				& %
				& %
				& %
				& %
				& %
				& %
				& \cmark %
				& %
				& %
				& %
				& %
				& %
				& \cmark %
				& \cmark %
				& \cmark %
				& %
				& %
				& %
				& %
				& %
				& %
				& \cmark %
				& \cmark %
				& \cmark %
				& %
				& %
			\\
			\citet{Lopez2016} & \citeyear{Lopez2016}
				& %
				& %
				& %
				& %
				& %
				& %
				& %
				& %
				& %
				& %
				& %
				& %
				& %
				& \cmark %
				& %
				& \cmark %
				& \cmark %
				& \cmark %
				& %
				& %
				& %
				& %
				& %
				& %
				& \cmark %
				& %
				& %
				& \cmark %
				& \cmark %
			\\
			\citet{Qian2016} & \citeyear{Qian2016}
				& %
				& %
				& %
				& %
				& %
				& %
				& %
				& %
				& %
				& %
				& %
				& \cmark %
				& %
				& %
				& %
				& %
				& \cmark %
				& \cmark %
				& \cmark %
				& %
				& %
				& %
				& %
				& %
				& \cmark %
				& \cmark %
				& \cmark %
				& %
				& %
			\\
			\citet{Lu2014} & \citeyear{Lu2014}
				& %
				& %
				& %
				& %
				& %
				& %
				& %
				& %
				& %
				& %
				& %
				& \cmark %
				& %
				& %
				& %
				& %
				& \cmark %
				& \cmark %
				& %
				& %
				& %
				& %
				& %
				& %
				& \cmark %
				& \cmark %
				& \cmark %
				& %
				& %
			\\
			\bottomrule
		\end{tabularx}
		\begin{tablenotes}\footnotesize
			\item[\fnvandenpoel] and van den Poel
			\item[\fnbeam] using Apache Beam
		\end{tablenotes}
	\end{threeparttable}
\end{table*}

\cref{tab:related-experiments} provides an overview of experimental performance evaluation and benchmarking studies. It indicates the applied benchmark, the evaluated stream processing, and information regarding the experiment setup and method. The latter includes whether the respective study was performed in a cloud environment, in a distributed fashion with multiple instances of the framework deployed. Moreover, it shows whether the benchmarks have been executed with different resource amounts and different load intensities and whether different resource amounts and load intensities are evaluated in isolated experiments. In previous work, we argued that scalability should be evaluated with isolated experiments for different combinations of load and resources~\cite{LTB2021,EMSE2022}.

We can observe that there is no established stream processing benchmark. Only YSB is used in several studies. However, YSB can be considered a micro-benchmark~\cite{Bermbach2017} and, hence, is less suited to benchmark entire microservices.
Except for the preliminary evaluation of our Theodolite benchmarks~\cite{BDR2021}, there is no benchmarking study addressed to stream processing frameworks employed within microservice architectures.

Flink, Spark, and Strom are by far the most frequently benchmarked frameworks. Kafka Stream, Hazelcast Jet, and Samza, which are particularly suited for implementing event-driven microservices, are only benchmarked in a few studies and there is no study benchmarking all of them.

9 out of 20 studies report on experiments in public or private clouds.
Except for this and our previous study~\cite{IC2E2022FaaSStreaming}, there are no evaluations in Kubernetes.
Likewise, there are no further studies evaluating scalability with a systematic approach as we do in this study. \citet{Vikash2020}, \citet{Nasiri2019}, \citet{Karakaya2017}, and \citet{vanDongen2021b} explicitly evaluate scalability, however, without testing different load intensities against different resource amounts in isolated experiments. \citet{Nasiri2019} conduct independent evaluations of scaling load and computing resources and, thus, address another aspect than our study.
Our previous study~\cite{IC2E2022FaaSStreaming} applies our Theodolite method as well, but benchmarks scalability with respect to costs and is addressed to comparing stream processing deployments against Function-as-a-Service offerings.


