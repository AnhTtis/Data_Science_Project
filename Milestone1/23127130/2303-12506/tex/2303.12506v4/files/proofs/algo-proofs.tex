\section{Proofs Omitted From Section \ref{sec:algo-short}}\label{sec:algo-sec-proofs}


\subsection{Using an approximate solver for \eqref{eq:basic-OP}}
\label{sub:approximate-P1}
Recall that \eqref{eq:basic-OP} is described as follows:
\begin{align}
 \max &&&\ztVar{x}   \tag{\progBasic}\\
        s.t. &&& (\text{\progBasic.1}) \Hquad x \in S \nonumber\\
              &&& (\text{\progBasic.2}) \Hquad \valBy{\ell}{x}\geq z_{\ell} & \forall \ell \in [t-1] \nonumber \\
               &&& (\text{\progBasic.3}) \Hquad \valBy{t}{x} \geq \ztVar{x} \nonumber   
\end{align} 
This section proves the following lemma:
\begin{lemma}
    Let $\multApprox\in (0,1]$, $\additiveApprox \geq 0$, and \textsf{OP} be an $(\multApprox,\additiveApprox)$-approximation procedure to \eqref{eq:basic-OP}. Then Algorithm \ref{alg:basic-ordered-Outcomes} outputs an $\left(\multApprox, \additiveApprox\right)$-leximin-approximation.  
\end{lemma}

\begin{proof}
    Let $x^*$ be the returned solution and assume by contradiction that it is \emph{not} $\left(\multApprox, \additiveApprox\right)$-approximately leximin-optimal.
    This means that there exists a $y \in S$ that is $(\multApprox, \additiveApprox)$-leximin-preferred over it. 
    That is, there exists an integer $k \in [n]$ such that:
    \begin{align*}
        \forall i < k \colon \Hquad &\valBy{i}{y} \geq \valBy{i}{x^*}\\
        & \valBy{k}{y} > \frac{1}{\multApprox}\left(\valBy{k}{x^*}+\additiveApprox\right)
    \end{align*}
    
    Since $x^*$ is a solution to  \eqref{eq:basic-OP} that was solved in the iteration $t=n$, it must satisfy all its constraints, and therefore:
    \begin{align}\label{eq:p1-x-to-z}
        \forall i \in [n] \colon \Hquad \valBy{i}{x^*} \geq z_i
    \end{align}
    by constraint (\progBasic.2) for $i<n$ and by constraint (\progBasic.3) for $i=n$.
    
    As $y$'s smallest $k$ values are at least as those of $x^*$, we can conclude that for each $i\leq k$ the $i$-th smallest value of $y$ is at least $z_i$.
    Therefore, $y$ is feasible to  \eqref{eq:basic-OP} that was solved in the iteration $t=k$.
    
    During the algorithm run, $z_k$ was obtained as an $(\multApprox, \additiveApprox)$-approximation to \eqref{eq:basic-OP} that was solved in the iteration $t=k$ , and therefore, the optimal value of this problem
     is at most $\frac{1}{\multApprox}(z_k+\additiveApprox)$.
    But, the objective value $y$ yields in this problem is $\valBy{k}{y}$, which is higher than this value:
    \begin{align*}
        \valBy{k}{y} &> \frac{1}{\multApprox}\left(\valBy{k}{x^*}+\additiveApprox\right)\\
        &\geq \frac{1}{\multApprox}\left(z_k+\additiveApprox\right) && \text{(By Equation \eqref{eq:p1-x-to-z} for $i=k$)}
    \end{align*}
    This is a contradiction.
\end{proof}


% --------------------------
\subsection{Equivalence of \eqref{eq:sums-OP} and \eqref{eq:vsums-OP}}
\label{sub:equivalence-P2-P3}
Recall the problems' descriptions:
 
\begin{align*}
\max &&&\ztVar{x} \tag{\progSums}\\
s.t. &&& (\text{\progSums.1}) \quad x \in S \\
&&& (\text{\progSums.2}) \quad \sum_{i \in F'} f_i(x) \geq \sum_{i=1}^{|F'|}  z_i && \forall F' \subseteq [n], \Hquad |F'| < t \\
&&& (\text{\progSums.3}) \quad \sum_{i \in F'} f_i(x) \geq \sum_{i=1}^{t-1}  z_i + z_t && \forall F' \subseteq [n], \Hquad |F'| = t
\end{align*}

\begin{align}
\max &&& \ztVar{x} \tag{\progLinear}\\
s.t. &&& (\text{\progLinear.1}) \Hquad x \in S \nonumber  \\
                    &&& (\text{\progLinear.2}) \Hquad \ell y_{\ell} - \sum_{j=1}^n m_{\ell,j}\geq \sum_{i=1}^{\ell}  z_i && \forXinY{\ell}{t-1} \nonumber \\
                    &&& (\text{\progLinear.3}) \Hquad t y_t - \sum_{j=1}^{n} m_{t,j} \geq \sum_{i=1}^{t-1}  z_i + z_t \nonumber \\
                    &&& (\text{\progLinear.4}) \Hquad m_{\ell,j} \geq y_{\ell} - f_j(x)  && \forXinY{\ell}{t},\Hquad \forXinY{j}{n} \nonumber \\
                    &&& (\text{\progLinear.5}) \Hquad m_{\ell,j} \geq 0  &&  \forXinY{\ell}{t},\Hquad \forXinY{j}{n} \nonumber
\end{align}

We use another equivalent representation of \eqref{eq:sums-OP}, which is more compact and will simplify the proofs, also introduced by \cite{Ogryczak_2006}:
\begin{align*}
    \max &&& z_t \tag{\progCompact}\label{eq:compact-OP}\\
    s.t. &&& (\text{\progCompact.1}) \Hquad x \in S\nonumber\\
                    &&& (\text{\progCompact.2}) \Hquad \sum_{i=1}^{\ell} \valBy{i}{x} \geq \sum_{i=1}^{\ell}  z_i && \forXinY{\ell}{t-1} \nonumber\\
                    &&& (\text{\progCompact.3}) \Hquad \sum_{i=1}^{t} \valBy{i}{x} \geq \sum_{i=1}^{t-1}  z_i + z_t
\end{align*}
In this problem, constraints (\progSums.2) and (\progSums.3) are replaced by (\progCompact.2) and (\progCompact.3), respectively.  
(\progSums.2) gives, for each $\ell$, a lower bound on the sum for \emph{any} set of $\ell$ objective functions; whereas (\progCompact.2) only considers the sum of the $\ell$ \emph{smallest} such values,
and similarly for (\progSums.3) and (\progCompact.3). 

% Our next step is proving Lemma \ref{lemma:op3-to-comp} that will allow us, while proving the theorem, to assume that \textsf{OP} is an approximation procedure for\eqref{eq:compact-OP}. 

This section proves that these \emph{three} problems are \emph{equivalent} in the following sense: 
\begin{lemma}\label{lem:equivalence-of-all-three}
    Let $t \in [n]$ and let $z_1, \ldots z_{t-1} \in \mathbb{R}$.
    Then, $(x, z_t)$ is feasible for \eqref{eq:sums-OP} if and only if $(x, z_t)$ is feasible for \eqref{eq:compact-OP} if and only if there exist $y_{\ell}$ and $m_{\ell,j}$ for $\ell \in [t]$ and $j \in [n]$ such that $\left(x, z_t, (y_1, \ldots, y_t), (m_{1,1}, \ldots m_{t,n})\right)$ is feasible for \eqref{eq:vsums-OP}.
\end{lemma} 
It is clear that this lemma implies Lemma \ref{lem:equivalence}, which only claims a part of it.

We start by proving that \eqref{eq:sums-OP} and \eqref{eq:compact-OP} are equivalent. That is, $(x, z_t)$ is feasible for \eqref{eq:sums-OP} if and only if $(x, z_t)$ is feasible for \eqref{eq:compact-OP}.
First, it is clear that $x$ satisfies constraint (\progSums.1) if and only if it satisfies constraint (\progCompact.1) (as both constraints are the same, $x \in S$).
To prove the other requirements, we start with the following lemma:
\begin{lemma}\label{lemma:sums-to-comp-constrants}
    For any $x \in S$, any $\ell \in [n]$ and a constant $c \in \mathbb{R}$ the following two conditions are equivalent:
    \begin{align}\label{eq:sums-to-comp-constrants}
         \forall F' \subseteq [n], |F'| = \ell \colon \sum_{i \in F'} f_i(x) &\geq c 
         \\
         \sum_{i=1}^{\ell} \valBy{i}{x}&\geq c 
    \end{align}
\end{lemma}

\begin{proof}
    For the first direction, recall that the values $ \valBy{1}{x}, \dots,  \valBy{\ell}{x}$ were obtained from $\ell$ objective functions (those who yield the smallest value).
    By the assumption, the sum of any set of function with size $\ell$ is at least $c$; therefore, it is true in particular for the functions corresponding to the values $ (\valBy{1}{x})_{i=1}^{\ell}$.
    For the second direction, assume that $\sum_{i=1}^{\ell} \valBy{i}{x}\geq c$.
    Since $ \valBy{1}{x}, \dots,  \valBy{\ell}{x}$ are the $\ell$ smallest objective values, we get that:
    \begin{align*}
       \forall F' \subseteq [n],\Hquad |F'| = \ell \colon \quad \sum_{i \in F'}f_i(x) \geq \sum_{i=1}^s \valBy{i}{x}\geq c.
    \end{align*}
\end{proof}

Accordingly, $x$ satisfies constraint (\progSums.2) --- for any $\ell \in [t-1]$, 
    \begin{align*}
        \forall F' \subseteq [n], |F'| = \ell \colon \sum_{i \in F'} f_i(x) \geq \sum_{i=1}^{\ell} z_i
    \end{align*}
    if and only if it satisfies $\sum_{i=1}^{\ell} \valBy{i}{x} \geq \sum_{i=1}^{\ell} z_i$, which is constraint (\progCompact.2).
    Similarly, $x$ and $z_t$ satisfy constraint (\progSums.3), 
    \begin{align*}
        \forall F' \subseteq [n], |F'| = t \colon \sum_{i \in F'} f_i(x) \geq \sum_{i=1}^{t} z_i
    \end{align*}
    if and only if $\sum_{i=1}^{t} \valBy{i}{x} \geq \sum_{i=1}^{t} z_i$, which is constraint (\progCompact.3).
    That is, $x$ ans $z_t$ satisfy all the constraints of \eqref{eq:sums-OP} if and only if they satisfy all the constraints of \eqref{eq:compact-OP}.


%------------------------------------------
Now, we will prove that that \eqref{eq:compact-OP} and \eqref{eq:vsums-OP} are equivalent, that is, $(x, z_t)$ is feasible for \eqref{eq:compact-OP} if and only if there exist $y_{\ell}$ and $m_{\ell,j}$ for $\ell \in [t]$ and $j \in [n]$ such that $\left(x, z_t, (y_1, \ldots, y_t), (m_{1,1}, \ldots m_{t,n})\right)$ is feasible for \eqref{eq:vsums-OP}.

We start with the following lemma:
\begin{lemma}\label{lemma:comp-to-p3-m-sums}
    For any $x \in S$ and any constant $c \in \mathbb{R}$ (where $c$ does not depend on $j$),
    \begin{align*}
        \sum_{j=1}^n \max(0, c - f_j(x) ) = \sum_{j=1}^n \max(0, c - \valBy{j}{x} ).
    \end{align*}
\end{lemma}
\eden{TODO:
maybe to change to observation}
\begin{proof}
     Let $(\pi_1, \ldots, \pi_n)$ be a permutation of $\{1,\ldots,n\}$ such that $f_{\pi_i}(x) = \valBy{i}{x}$ for any $i \in [n]$ (notice that such permutation exists by the definition of $\valBy{}{}$).
     That is, the value that $f_{\pi_i}$ obtains is the $i$-th smallest one.
    Since each element in the sum $\sum_{j=1}^n \max(0, c - f_j(x))$ is affected by $j$ only through $f_j(x)$, the permutation $\pi$ allows us to conclude the following:
    \begin{align*}
        &\sum_{j=1}^n \max(0, c -f_j(x)) = \sum_{j=\pi_1}^{\pi_n} \max(0,c -f_j(x))\\ 
        &= \sum_{j=1}^n \max(0,c -f_{\pi_i}(x))  = \sum_{j=1}^{n} \max(0,c -\valBy{j}{x}).
    \end{align*}
\end{proof}



Lemma \ref{lemma:comp-to-p3-mapping} below proves the first direction of the equivalence between \eqref{eq:compact-OP} and \eqref{eq:vsums-OP}:
\begin{lemma}\label{lemma:comp-to-p3-mapping}
    Let $(x, z_t)$ be a feasible solution to \eqref{eq:compact-OP}.
    Then there exist $y_{\ell}$ and $m_{\ell,j}$ for $\ell \in [t]$ and $j \in [n]$ such that $\left(x, z_t, (y_1, \ldots, y_t), (m_{1,1}, \ldots m_{t,n})\right)$ is feasible for \eqref{eq:vsums-OP}.
\end{lemma}
% \begin{lemma}\label{lemma:comp-to-p3-mapping}
%     Let $(x,z_t)$ be a feasible solution  to \eqref{eq:compact-OP}. Then $(x, z_t, (y_1,\ldots,y_n), (m_{1,1},\ldots,m_{n,n}))$ is a feasible solution to \eqref{eq:vsums-OP}, where
%     \begin{align*}
%         \quad y_{\ell} &:= \valBy{\ell}{x} \Hquad\forall \ell \in [n], 
%         \\
%         m_{\ell,j} &:= \max(0,\valBy{\ell}{x} -f_j(x)) \Hquad \forall \ell \in [n], \Hquad \forall 1 \leq j \leq n 
%     \end{align*}
% \end{lemma}

\begin{proof}
    For any $\ell \in [t]$ and $j \in [n]$ define $y_{\ell}$ and $m_{\ell,j}$ as follows:
    \begin{align*}
        \quad y_{\ell} &:= \valBy{\ell}{x}
        \\
        m_{\ell,j} &:= \max(0,\valBy{\ell}{x} -f_j(x))
    \end{align*}
    
    First, since $x$ satisfies constraint (\progCompact.1), it is also satisfies constraint (\progLinear.1) of (as both constraints are the same and include only $x$).
    
    In addition, based on the choice of $y$ and $m$, it is clear that  $m_{\ell,j} \geq 0$ and $m_{\ell,j} \geq \valBy{\ell}{x} - f_j(x) = y_{\ell} - f_j(x)$ for any $\ell \in [n]$ and $j \in [n]$.
    Therefore, this assignment satisfies constraints (\progLinear.4) and (\progLinear.5).
    
    To show that this assignment also satisfies constraints (\progLinear.2) and (\progLinear.3), we first prove that for any $\ell \in [n]$ this assignment satisfies the following equation:
    \begin{align}\label{eq:comp-to-p3}
        \ell y_{\ell} - \sum_{j=1}^n m_{\ell,j} = \sum_{i=1}^{\ell} \valBy{i}{x} 
    \end{align}
    By the choice of $m$,  $\sum_{j=1}^n m_{\ell,j} = \sum_{j=1}^n \max(0, \valBy{\ell}{x} - f_j(x))$, and therefore, by Lemma \ref{lemma:comp-to-p3-m-sums}, it equals to $\sum_{j=1}^{n} \max(0,\valBy{\ell}{x} -\valBy{j}{x})$.
    \eden{\eden{TODO:
to try to add: as $\valBy{\ell}{x}$ does not depend on $j$}}
    Since $\valBy{\ell}{x}$ is the $\ell$-th smallest objective, it is clear that $\valBy{\ell}{x} - \valBy{j}{x} \leq 0$ for any $j > \ell$, and $\valBy{\ell}{x} - \valBy{j}{x} \geq 0$ for any $j \leq \ell$.
    And so, $\sum_{j=1}^n m_{\ell,j} = \ell\cdot \valBy{\ell}{x} - \sum_{i=1}^{\ell} \valBy{i}{x}$:
    \begin{align*}
        &\sum_{j=1}^n m_{\ell,j} = \sum_{j=1}^{n} \max(0,\valBy{\ell}{x} -\valBy{j}{x})\\
        &=\sum_{j=1}^{\ell} \max(0,\valBy{\ell}{x} -\valBy{j}{x}) + \sum_{j=\ell+1}^n \max(0,\valBy{\ell}{x} -\valBy{j}{x})\\
        &=\sum_{j=1}^{\ell} (\valBy{\ell}{x} -\valBy{j}{x}) + \sum_{j=\ell+1}^n 0 = \ell\cdot \valBy{\ell}{x} - \sum_{i=1}^{\ell} \valBy{i}{x}
    \end{align*}
    We can now conclude Equation \eqref{eq:comp-to-p3}:
    \begin{align*}
        \ell y_{\ell} - \sum_{j=1}^n m_{\ell,j} &= \ell \cdot \valBy{\ell}{x} - \ell\cdot \valBy{\ell}{x} + \sum_{i=1}^{\ell} \valBy{i}{x}\\
        &=\sum_{i=1}^{\ell} \valBy{i}{x}.
    \end{align*}

    Now, since $x$ satisfies constraint (\progCompact.2),  $\sum_{i=1}^{\ell} \valBy{i}{x} \geq \sum_{i=1}^{\ell} z_i$ for any $\ell \in [t-1]$.
    Therefore, by Equation \eqref{eq:comp-to-p3}, $\ell\cdot y_{\ell} - \sum_{j=1}^n m_{\ell,j}\geq  \sum_{i=1}^{\ell} z_i$ for any $\ell \in [t-1]$ and this assignment satisfies constraint (\progLinear.2).
    Similarly, as $x$ and $z_t$ satisfy constraint (\progCompact.3), $\sum_{i=1}^{t} \valBy{i}{x} \geq \sum_{i=1}^{t} z_i$, and so by Equation \eqref{eq:comp-to-p3}, $t \cdot y_{t} - \sum_{j=1}^n m_{t,j}\geq  \sum_{i=1}^{t} z_i$.
    This means that it also satisfies constraints (\progLinear.3).
\end{proof}

Finally, Lemma \ref{lemma:comp-to-p3-is-bij} below proves the second direction of the equivalence:

\begin{lemma}\label{lemma:comp-to-p3-is-bij}
    Let $\left(x, z_t, (y_1, \ldots, y_t), (m_{1,1}, \ldots m_{t,n})\right)$ be a feasible solution to \eqref{eq:vsums-OP}.
   Then, $(x, z_t)$ is feasible for \eqref{eq:compact-OP}.
\end{lemma}

\begin{proof}
    It is easy to see that since $x$ satisfies constraint (\progLinear.1), it is also satisfies constraint (\progCompact.1) (as both are the same).
    To show that it also satisfies constraints (\progCompact.2) and (\progCompact.3), we start by proving that for any $\ell \in [n]$:
    \begin{align}\label{eq:p3-to-comp}
          \sum_{i=1}^{\ell} \valBy{i}{x} \geq \ell y_{\ell} - \sum_{j=1}^n m_{\ell,j}
    \end{align}
    Suppose by contradiction that $ \sum_{i=1}^{\ell} \valBy{i}{x} < \ell y_{\ell} - \sum_{j=1}^n m_{\ell,j}$.

    
    For any $j\in [n]$ and any $\ell \in [n]$, $m_{\ell,j} \geq y_{\ell} - f_j(x)$ by constraint (\progLinear.4), and also $m_{\ell,j} \geq 0$ by constraint (\progLinear.5).
    Therefore, $m_{\ell,j} \geq \max(0,y_{\ell} -f_j(x))$.
    And so, by Lemma \ref{lemma:comp-to-p3-m-sums}, for any $\ell \in [t]$, the sum of $m_{\ell,j}$ over all $j \in [n]$ can be described as follows:
    \eden{TODO:
to try to add: as $\valBy{\ell}{x}$ does not depend on $j$}
    \begin{align}\label{eq:p3-to-conp-m-sum}
        \sum_{j=1}^n m_{\ell,j} \geq  \sum_{j=1}^n \max(0,y_{\ell} -f_j(x)) = \sum_{j=1}^n \max(0,y_{\ell} -\valBy{j}{x})
    \end{align}
    Therefore, $\sum_{i=1}^{\ell} \valBy{i}{x} <  \ell y_{\ell} - \sum_{j=1}^n \max(0,y_{\ell} -\valBy{j}{x})$ .
    Which means that:
\begin{align*}
    & \ell y_{\ell} - \sum_{i=1}^{\ell} \valBy{i}{x} - \sum_{j=1}^n \max(0,y_{\ell} -\valBy{j} {x}) > 0
\end{align*}
However, we will now see that the value of this expression is at most $0$, which is a contradiction:
\begin{align*}        
    &\ell y_{\ell} - \sum_{i=1}^{\ell} \valBy{i}{x} - \sum_{j=1}^n \max(0,y_{\ell} -\valBy{j} {x}) \\
    &= \sum_{i=1}^{\ell} y_{\ell} - \sum_{i=1}^{\ell} \valBy{i}{x} - \sum_{j=1}^n \max(0,y_{\ell} -\valBy{j} {x})\\
    & \equWithExp{\text{Since max with $0$ is at least $0$}}{\leq \sum_{i=1}^{\ell}( y_{\ell} - \valBy{i}{x} ) - \sum_{j=1}^{\ell} \max(0,y_{\ell} -\valBy{j} {x}) - \sum_{j=\ell+1}^n 0}\\
        &=  \sum_{j=1}^{\ell} \left((y_{\ell} - \valBy{j}{x}) - \max(0,y_{\ell} -\valBy{j}{x})\right)\\
        &\equWithExp{\text{Since each element\displayEcai{ of this sum} is at most $0$}}{\leq 0}
   \end{align*}
This is a contradiction; so \eqref{eq:p3-to-comp} is proved.

    Now, by constraint (\progLinear.2), $\ell y_{\ell} - \sum_{j=1}^n m_{\ell,j}\geq  \sum_{i=1}^{\ell} z_i$  for any $\ell \in [t-1]$. Therefore, by \eqref{eq:p3-to-comp}, also $\sum_{i=1}^{\ell} \valBy{i}{x} \geq \sum_{i=1}^{\ell} z_i$, which means that $x$ satisfies constraint (\progCompact.2).
    Similarly, by constraint (\progLinear.3),  $t y_{t} - \sum_{j=1}^n m_{t,j}\geq  \sum_{i=1}^{t} z_i$, and so by \eqref{eq:p3-to-comp}, also $\sum_{i=1}^{t} \valBy{i}{x} \geq \sum_{i=1}^{t} z_i$.
    This means that $x$ and $z_t$ satisfy constraint (\progCompact.3).
\end{proof}


\subsection{Proof of Theorem \ref{th:main}}
\label{sub:th:main}
This section is dedicated to proving Theorem \ref{th:main}:
let $\multApprox\in (0,1]$, $\additiveApprox \geq 0$, and \textsf{OP} be an $(\multApprox,\additiveApprox)$-approximation procedure to \eqref{eq:sums-OP} or \eqref{eq:vsums-OP}. Then Algorithm \ref{alg:basic-ordered-Outcomes} outputs an $\left(\frac{\multApprox^2}{1-\multApprox + \multApprox^2}, \frac{\additiveApprox}{1-\multApprox +\multApprox^2}\right)$-leximin-approximation. 

Based on Lemma \ref{lem:equivalence-of-all-three}, it is sufficient to prove the theorem for \eqref{eq:compact-OP}.

We start by observing that the value of the variable $z_t$ is completely determined by the variable $x$. This is because $z_t$ only appears in the last constraint, which is equivalent to 
$z_t \leq \sum_{i=1}^{t} \valBy{i}{x} - \sum_{i=1}^{t-1} z_i$. 
Therefore, for every $x$ that satisfies the first two constraints, 
it is possible to satisfy the last constraint by setting $z_t$ to any value which is at most
$\sum_{i=1}^{t} \valBy{i}{x} - \sum_{i=1}^{t-1} z_i$.
Moreover, as the program aims to maximize $z_t$, it will necessarily 
set $z_t$ to be equal to that expression, since $z_t$ is maximized when the constraint holds with equality. This is summarized in the observation below:

% For any constants $z_1,\ldots, z_{t-1}$,
% any vector $x \in S$ that satisfies constraint $(\Tilde{2})$ of \eqref{eq:compact-OP} 
% is feasible to this problem.
% This is because any solution $x \in S$ can satisfy constraint $(\Tilde{3})$ with a small enough assignment to the variable $z_t$. \eden{I'm not sure how to explain it....}
\begin{observation}
\label{obs:feasi-and-constraint2}
For any $t\geq 1$ and any constants $z_1,\ldots, z_{t-1}$,
every vector $x$ that satisfies constraints (\progCompact.1)  and (\progCompact.2)
is a part of a feasible solution $(x,z_t)$ for $z_t = \sum_{i=1}^{t} \valBy{i}{x} - \sum_{i=1}^{t-1} z_i$.
Moreover, 
\label{obs:obj-value}
the objective value  obtained by a feasible solution $x$ to the problem \eqref{eq:compact-OP} solved in iteration $t$ is $z_t = \sum_{i=1}^{t} \valBy{i}{x} - \sum_{i=1}^{t-1}  z_i$.
\end{observation}

% Now, consider the problem \eqref{eq:compact-OP} that was solved in iteration $t$.
% Here, $z_t$ is a \emph{variable} and $z_1, \ldots z_{t-1}$ are constants.
% The objective of this problem is $\max z_t$, and the only constraint that includes the variable $z_t$ is  (\progCompact.3).
% Therefore, rearranging it to $\sum_{i=1}^{t} \valBy{i}{x} - \sum_{i=1}^{t-1}  z_i\geq z_t$, allows us to conclude that the objective value is determined by the left side of this inequality (as $z_t$ is maximized when the inequality turns to equality).

Based on Observation \ref{obs:obj-value}, we can now slightly abuse the terminology and say that a solution $x$ is ``feasible`` in iteration $t$ if it satisfies constraints (\progCompact.1)  and (\progCompact.2) of the program solved in iteration $t$.


We denote $\retSol := x_n$ = the solution $x$ attained at the last iteration ($t=n$) of the algorithm. 
Since $\retSol$ is a feasible solution of \eqref{eq:compact-OP} in iteration $n$, and as each
iteration only adds new constraints to (\progCompact.2), it follows that $\retSol$ is also a feasible solution of \eqref{eq:compact-OP} in any iteration $1 \leq t\leq n$. 
\begin{observation}\label{obs:retSol-solves-any-t}
$\retSol$ is a feasible solution of \eqref{eq:compact-OP} in any iteration $1 \leq t\leq n$.
\end{observation}


Lastly, as the value obtained as $(\multApprox, \additiveApprox)$-approximation for this problem is the \emph{constant} $z_t$, the optimal value is at most $\frac{1}{\multApprox} (z_t+\additiveError)$. 
Consequently, the objective value of any feasible solution is at most this value.
Since $\retSol$ is feasible for any iteration $t$ (Observation \ref{obs:retSol-solves-any-t}) and since the objective value corresponding to $\retSol$ is $\sum_{i=1}^t \valBy{i}{\retSol} - \sum_{i=1}^{t-1} z_i$ (Observation \ref{obs:obj-value}), we can conclude:

\begin{observation}\label{obs:obj-xt-to-zt}
    The objective value obtained by $\retSol$ to the problem \eqref{eq:compact-OP} that was solved in iteration $t$ is at most $\frac{1}{\multApprox} (z_t+\additiveError)$. That is:
    \begin{align*}
        \sum_{i=1}^t \valBy{i}{\retSol} - \sum_{i=1}^{t-1} z_i \leq \frac{1}{\multApprox} \left(z_t+\additiveError \right).
    \end{align*}
\end{observation}

% This conclusion also implies that for any $1 \leq t \leq n$, the solution $(x_t, z_t)$ that that was outputted for \eqref{eq:compact-OP} in iteration $t$, satisfies constraint $(\Tilde{3})$ as equality. That is:
% \begin{observation}\label{obs:equality-xt-zt}
% For any $1 \leq t \leq n$,  $\sum_{i=1}^{t} \valBy{i}{x_t} = \sum_{i=1}^{t}  z_i$.
% \end{observation}



%%%
% OVERALL EXPLANATION 
We start with Lemmas \ref{lemma:beta-vk}-\ref{lemma:fk-to-all}, which establish a relationship between the $k$-th least objective value obtained by $\retSol$ 
% ($\valBy{k}{\retSol}$) 
and the difference between the sum of the $(k-1)$ least objective values obtained by $\retSol$ and the sum of the $(k-1)$ first $z_i$ values.
% ($\sum_{i=1}^{k-1}\valBy{k}{\retSol} - \sum_{i=1}^{k-1}z_i$). 
Theorem \ref{th:main} then uses this relation to prove that the existence of another solution that would be $\left(\frac{\multApprox^2}{1-\multApprox + \multApprox^2}, \frac{\additiveApprox}{1-\multApprox +\multApprox^2}\right)$-leximin-preferred over $\retSol$ would lead to a contradiction.

For clarity, throughout the proofs, we denote the multiplicative \emph{error} factor by $\multError = 1-\multApprox$.

% LEMMAS.
% BLAH BLAH.

\begin{lemma}\label{lemma:beta-vk}
    For any $1 \leq k\leq n$, 
    \begin{align*}
        &\multError \sum_{i=1}^{k} \valBy{i}{\retSol} - \multError \sum_{i=1}^{k-1} z_i \geq \sum_{i=1}^k \valBy{i}{\retSol} - \sum_{i=1}^k z_i - \additiveError
        % \\
        % &\multError \valBy{k}{\retSol} \geq \left(\sum_{i=1}^k \valBy{i}{\retSol} - \sum_{i=1}^k z_i\right) -\multError \left(\sum_{i=1}^{k-1} \valBy{i}{\retSol} - \sum_{i=1}^{k-1} z_i\right) -\additiveError
    \end{align*}
\end{lemma}

\begin{proof}
By Observation \ref{obs:obj-xt-to-zt},
    \begin{align*}
         &\sum_{i=1}^k \valBy{i}{\retSol} - \sum_{i=1}^{k-1} z_i \leq \frac{1}{\multApprox} \left(z_k + \additiveError \right) = \frac{1}{1-\multError} \left(z_k + \additiveError \right)\\
         &\Rightarrow (1-\multError) \left(\sum_{i=1}^{k} \valBy{i}{\retSol} - \sum_{i=1}^{k-1}  z_i\right) \leq z_k +\additiveError\\
         &\Rightarrow \sum_{i=1}^{k} \valBy{i}{\retSol} - \sum_{i=1}^{k-1}  z_i - \multError \sum_{i=1}^{k} \valBy{i}{\retSol} + \multError \sum_{i=1}^{k-1}  z_i \leq z_k +\additiveError\\
         &\Rightarrow \sum_{i=1}^{k} \valBy{i}{\retSol} - \sum_{i=1}^{k}  z_i - \additiveError \leq \multError \sum_{i=1}^{k} \valBy{i}{\retSol} - \multError \sum_{i=1}^{k-1}  z_i. 
    \end{align*}
        \qedhere
\end{proof}


\begin{lemma}\label{lemma:beta-sums-to-diff}
    For any $1 \leq k \leq n$, 
    \begin{align*}
        \sum_{i=1}^k \multError^{i} \valBy{k-i+1}{\retSol} \geq \sum_{i=1}^k \valBy{i}{\retSol} - \sum_{i=1}^{k} z_i -\frac{1}{1 - \multError}\additiveError
    \end{align*}
\end{lemma}

\begin{proof}
    The proof is by induction on $k$.
    For $k=1$ the claim follows directly from Lemma \ref{lemma:beta-vk} as $\frac{1}{1-\multError} \geq 1$ for any $\multError \in [0,1)$.
    Assuming the claim is true for $1,\ldots k-1$, we show it is true for $k$:
    \begin{align*}
        &\sum_{i=1}^k \multError^{i} \valBy{k-i+1}{\retSol} = \multError \valBy{k}{\retSol} + \sum_{i=2}^k \multError^{i} \valBy{k-i+1}{\retSol}\\
        &= \multError \valBy{k}{\retSol} + \sum_{i=1}^{k-1} \multError^{i+1} \valBy{k-(i+1)+1}{\retSol} \\
        &= \multError \valBy{k}{\retSol} + \multError \sum_{i=1}^{k-1} \multError^{i} \valBy{(k-1) -i+1}{\retSol}\\
    & \equWithExp{\text{By induction assumption}}{\geq \multError \valBy{k}{\retSol} + \multError \left(\sum_{i=1}^{k-1} \valBy{i}{\retSol} - \sum_{i=1}^{k-1} z_i -\frac{1}{1-\multError} \additiveError\right)}\\
        &= \multError \sum_{i=1}^{k} \valBy{i}{\retSol} - \multError\sum_{i=1}^{k-1} z_i -\frac{\multError}{1 - \multError}  \additiveError\\
    & \equWithExp{\text{By Lemma \ref{lemma:beta-vk}}}{\geq \sum_{i=1}^{k} \valBy{i}{\retSol} -  \sum_{i=1}^{k} z_i - \additiveError -\frac{\multError}{1 - \multError}  \additiveError}\\
         &=\sum_{i=1}^k \valBy{i}{\retSol} - \sum_{i=1}^{k} z_i -\frac{1}{1 - \multError}\additiveError
    \end{align*}
        \qedhere
\end{proof}


\begin{lemma}\label{lemma:fk-to-all}
    For all $1<k \leq n$, 
    \begin{align*}
        \frac{\multError}{1-\multError} \valBy{k}{\retSol} \geq \sum_{i=1}^{k-1}\valBy{i}{\retSol} - \sum_{i=1}^{k-1}z_i - \frac{1}{1 - \multError}\additiveError
    \end{align*}
\end{lemma}

\begin{proof}
    First, notice that since $k \geq (k-1)-i+1$ for any $1\leq i \leq k$ and as the function $\valBy{i}$ represents the $i$-th smallest objective value, also:
    \begin{align}\label{eq:increase-by-obj-size}
        \forall 1\leq i \leq k \colon \Hquad \valBy{k}{\retSol} \geq \valBy{(k-1)-i+1}{\retSol}
    \end{align}
    In addition, consider the geometric series with a first element $1$, a ratio $\multError$, and a length $(k-1)$. 
    As $\multError < 1$, its sum can be bounded in the following way:
    \begin{align}\label{eq:geometric-series-beta}
        \sum_{i=1}^{k-1} \multError^{i-1} = \frac{1-\multError^{k-1}}{1-\multError} < \lim_{k \to \infty}\frac{1-\multError^{k-1}}{1-\multError} = \frac{1}{1-\multError}
    \end{align}
    
    Now, the claim can be concluded as follows:
    \begin{align*}
        & \frac{\multError}{1-\multError}\valBy{k}{\retSol} = \multError \left(\frac{1}{1-\multError} \valBy{k}{\retSol} \right)\\
        & \equWithExp{\text{By Equation \eqref{eq:geometric-series-beta}}}{ > \multError \left(\sum_{i=1}^{k-1} \multError^{i-1} \valBy{k}{\retSol} \right)}\\
        & \equWithExp{\text{By Equation \eqref{eq:increase-by-obj-size}}}{\geq  \multError \left(\sum_{i=1}^{k-1} \multError^{i-1} \valBy{(k-1)-i+1}{\retSol} \right)}\\
        &= \sum_{i=1}^{k-1} \multError^{i} \valBy{(k-1)-i+1}{\retSol} \\ % in ecai it was on the previous line
        & \equWithExp{\text{By Lemma \ref{lemma:beta-sums-to-diff} for $(k-1)\geq 1$}}{\geq \sum_{i=1}^{k-1}\valBy{i}{\retSol} - \sum_{i=1}^{k-1}z_i -\frac{1}{1 - \multError} \additiveError}
\end{align*}
\end{proof}



%------
% thm.

We are now ready to prove the Theorem \ref{th:main}.
\begin{proof}[Proof of Theorem \ref{th:main}]
    % \eden{I'm not sure if we should write again about the claim with $\multApprox$}
    Recall that the claim is that $\retSol$ is a $\left(\frac{\multApprox^2}{1-\multApprox + \multApprox^2}, \frac{\additiveApprox}{1-\multApprox +\multApprox^2}\right)$-leximin-approximation.
    
    For brevity, we define the following constant:
    \begin{align*}
        \Delta(\multApprox) = \frac{1}{1-\multApprox + \multApprox^2}
    \end{align*}
    Accordingly, we need to prove that $\retSol$ is a $\left(\multApprox^2\cdot\Delta(\multApprox), \additiveApprox \cdot\Delta(\multApprox)\right)$-approximation.
    
   As $\multApprox = 1 - \multError$, it is easy to verify that:
   \begin{align}\label{eq:delta-alpha-with-beta}
       \Delta(\multApprox) = \frac{1}{1-\multError + \multError^2}
   \end{align}
    % Which implies that:
    % \begin{align}\label{equ:mu}
    % \frac{\multApprox^2}{\Delta(\multApprox)} = \frac{(1-\multError)^2}{1-\multError +\multError^2}
    % \end{align}
    % And also that:
    % \begin{align}\label{eq:additive-error}
    %     \frac{\Delta^{add}}{ \multApprox\cdot\Delta^{mult}}  = \emark{\frac{1}{(1-\multError)^2}}.
    % \end{align}
    % Both of these equations will be helpful later.

    Now, suppose by contradiction that $\retSol$ is \emph{not} $\left(\multApprox^2\cdot\Delta(\multApprox), \additiveApprox \cdot\Delta(\multApprox)\right)$-approximately leximin-optimal.
    By definition, this means there exists a solution $y \in S$  that is $\left(\multApprox^2\cdot\Delta(\multApprox), \additiveApprox \cdot\Delta(\multApprox)\right)$-leximin-preferred over it.
    That is, there exists an integer $1 \leq k \leq n$ such that:
    \begin{align*}
        \forall j < k \colon &\valBy{j}{y} \geq \valBy{j}{\retSol};\\
        & \valBy{k}{y} > \frac{1}{\multApprox^2\cdot\Delta(\multApprox)} \left(\valBy{k}{\retSol} + \additiveApprox \cdot\Delta(\multApprox) \right).
    \end{align*}

    Since $\retSol$ was obtained in \eqref{eq:compact-OP} that was solved in the last iteration $n$, it is clear that $\sum_{i=1}^k \valBy{i}{\retSol} \geq \sum_{i=1}^{k} z_i$ (by constraint (\progCompact.2) if $k<n$ and (\progCompact.3) otherwise).
    Which implies:
    \begin{align}\label{eq:fk-to-zk}
        \sum_{i=1}^k \valBy{i}{\retSol} - \sum_{i=1}^{k-1} z_i \geq z_k
    \end{align}

    Now, consider \eqref{eq:compact-OP} that was solved in iteration $k$.
    By Observation \ref{obs:retSol-solves-any-t}, $\retSol$ is feasible to this problem.
    As the $(k-1)$ smallest objective values of $y$ are at least as high as those of $\retSol$, it is easy to conclude that $y$ also satisfies constraints (\progCompact.2) of this problem; since, for any $\ell < k$:
    \begin{align*}
        \sum_{i=1}^{\ell} \valBy{i}{y} \geq\sum_{i=1}^{\ell} \valBy{i}{\retSol} \geq \sum_{i=1}^{\ell} z_i
    \end{align*}
    Therefore, by Observation \ref{obs:feasi-and-constraint2}, $y$ is also feasible to this problem. 

    The value obtained during the algorithm run as an approximation for this problem is $z_k$.
    This means that the optimal value is at most $\frac{1}{\multApprox}\left(z_k + \additiveError \right)$.
    As $y$ is feasible in this problem, and since the objective value obtained by $y$ in this problem is $\sum_{i=1}^k \valBy{i}{y} - \sum_{i=1}^{k-1} z_i$ (by Observation \ref{obs:obj-value}), this implies the following:
    \begin{align}\label{eq:y-upper bound} 
        \sum_{i=1}^k \valBy{i}{y} - \sum_{i=1}^{k-1} z_i \leq \frac{1}{(1-\multError)}\left(z_k+\additiveError\right)
    \end{align}

    If $k=1$, we get that the objective value obtained by $y$ is $\valBy{1}{y}$.
    In addition, $\valBy{1}{\retSol} \geq z_1$ by Equation \eqref{eq:fk-to-zk}. 
    However, as $0<\multApprox \leq 1$, then $\Delta(\multApprox) \geq 1$ but $\multApprox \cdot\Delta(\multApprox) \leq 1$.
    It follows that:
    \begin{align*}
        \valBy{1}{y}> \frac{1}{\multApprox^2 \cdot \Delta(\multApprox)} \left(\valBy{1}{\retSol} + \additiveError \cdot \Delta(\multApprox) \right)\geq \frac{1}{\multApprox} \left(z_1 + \additiveError \right)
    \end{align*}
    In contradiction to Equation \eqref{eq:y-upper bound}.

    
    Therefore, $k>1$.  
    We start by showing that the following holds:
    \begin{align}\label{eq:yk-to-sum}
        \valBy{k}{y} > \frac{1}{1-\multError} \left( \valBy{k}{\retSol} + \multError\sum_{i=1}^{k-1}\valBy{i}{\retSol} - \multError \sum_{i=1}^{k-1}z_i  +\additiveError \right)
    \end{align}
    Consider $\valBy{k}{y}$, by the definition of $y$ for $k$ we get that:
    \begin{align*}
        &\valBy{k}{y} > \frac{1}{ \multApprox^2 \cdot \Delta(\multApprox)} \left(\valBy{k}{\retSol} +  \additiveError \cdot \Delta(\multApprox) \right)\\
        &\equWithExp{\text{\displayComsoc{By Equ. \eqref{eq:delta-alpha-with-beta} and $\multError$'s def.}\displayEcai{Since $\multApprox = 1 - \multError$ and by Equation \eqref{eq:delta-alpha-with-beta}, this equals to}}}{= \frac{1}{\multApprox}\left(\frac{1-\multError +\multError^2}{1-\multError} \valBy{k}{\retSol}+ \frac{1}{\multApprox}\additiveError\right)}\\
        &= \frac{1}{\multApprox}\left(\valBy{k}{\retSol} + \frac{\multError^2}{1-\multError} \valBy{k}{\retSol}+ \frac{1}{\multApprox}\additiveError\right)\\
        &\equWithExp{\text{By Lemma \ref{lemma:fk-to-all}\displayEcai{ we can conclude that}}}{
         % and as $-\frac{\multError}{1-\multError} + \frac{1}{1-\multError} = 1$
        \geq \frac{1}{\multApprox}\left(\valBy{k}{\retSol} + \multError\Bigr[\sum_{i=1}^{k-1}\valBy{i}{\retSol} - \sum_{i=1}^{k-1}z_i -\frac{1}{1-\multError}\additiveError\Bigr]+ \frac{1}{\multApprox}\additiveError\right)}\\
        & \equWithExp{\text{\displayComsoc{By $\multError$'s def.}\displayEcai{As $\multApprox = 1-\multError$}}}{\frac{1}{1-\multError}\left(\valBy{k}{\retSol} + \multError\sum_{i=1}^{k-1}\valBy{i}{\retSol} -\multError \sum_{i=1}^{k-1}z_i +\additiveError\right)}
    \end{align*}    

    But, we shall now prove that this means, once again, that the objective value of $y$, which is $\sum_{i=1}^k \valBy{i}{y} - \sum_{i=1}^{k-1} z_i$, is higher than $\frac{1}{1-\multError} \left(z_k +\additiveError\right)$, in contradiction to Equation \eqref{eq:y-upper bound}:
    \begin{align*}
        &\displayComsoc{\quad}\sum_{i=1}^k \valBy{i}{y} - \sum_{i=1}^{k-1} z_i=\sum_{i=1}^{k-1} \valBy{i}{y} - \sum_{i=1}^{k-1} z_i + \valBy{k}{y}\\
        &\text{By the definition of $y$ for $i<k$:}\\
        &\displayComsoc{\quad}\geq \sum_{i=1}^{k-1} \valBy{i}{\retSol} - \sum_{i=1}^{k-1} z_i + \valBy{k}{y}\\
        &\text{By Equation \eqref{eq:yk-to-sum}:}\\
        &\displayComsoc{\quad}> \sum_{i=1}^{k-1} \valBy{i}{\retSol} - \sum_{i=1}^{k-1} z_i + \frac{1}{1-\multError} \valBy{k}{\retSol} \displayEcai{ \\
        & \quad\quad\quad} + \frac{\multError}{1-\multError}\sum_{i=1}^{k-1}\valBy{i}{\retSol} - \frac{\multError}{1-\multError}\sum_{i=1}^{k-1}z_i +\frac{1}{1-\multError}\cdot\additiveError\\
        &\text{Since  $1+\frac{\multError}{1-\multError} = \frac{1}{1-\multError}$, this equals to:}\\
        &\displayComsoc{\quad} = \frac{1}{1-\multError}\sum_{i=1}^{k-1} \valBy{i}{\retSol} - \frac{1}{1-\multError}\sum_{i=1}^{k-1} z_i + \frac{1}{1-\multError} \valBy{k}{\retSol} + \frac{1}{1-\multError} \additiveError\\
        & \displayComsoc{\quad}= \frac{1}{1-\multError} \left(\sum_{i=1}^k \valBy{k}{\retSol} - \sum_{i=1}^{k-1}z_i + \additiveError\right)\\
        &\text{By Equation \eqref{eq:fk-to-zk}:}\\
        &\displayComsoc{\quad}\geq \frac{1}{1-\multError} \left(z_k +\additiveError\right).
    \end{align*}
    This is a contradiction, so Theorem \ref{th:main} is proved.
\end{proof}
