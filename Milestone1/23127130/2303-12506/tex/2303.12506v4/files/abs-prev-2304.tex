
\begin{abstract}
Leximin is a common approach to multi-objective optimization,
frequently employed in fair division applications.
In leximin optimization, one first aims to maximize the smallest objective value; subject to this, one maximizes the second-smallest objective; and so on. 
Often, even the single-objective problem of maximizing the smallest value cannot be solved accurately, e.g. due to computational hardness. 
What can we hope to accomplish for leximin optimization in this situation?
Recently, Henzinger et al (2022) defined a notion of \emph{approximate} leximin optimality, and showed how it can be computed for the problem of representative cohort selection. 
However, their definition considers only an additive approximation in the single-objective problem.

In this work, we define approximate leximin optimality allowing approximations that have multiplicative and additive errors. 
We show how to compute a solution that approximates leximin in polynomial time, using an oracle that finds an approximation to the single-objective problem.  
The approximation factors of the algorithms are closely related: a factor of $(\multApprox,\additiveApprox)$ for the single-objective problem (where $\multApprox$ and $\additiveApprox$ describe the multiplicative and additive factors respectively) translates into a factor of $\left(\frac{\multApprox^2}{1-\multApprox + \multApprox^2}, \frac{\multApprox(2-\multApprox)\additiveApprox}{1-\multApprox +\multApprox^2}\right)$ for the multi-objective leximin problem, regardless of the number of objectives. 
\eden{I combined between the paragraphs because APPROX requires 2-paragraphs}
As a usage example, we apply our algorithm to the problem of \emph{stochastic allocations of indivisible goods}.  
For this problem, assuming sub-modular objectives functions, the single-objective egalitarian welfare can be approximated, with only a multiplicative error, to an optimal $1-\frac{1}{e}\approx 0.632$ factor w.h.p.
We show how to extend the approximation to leximin, over all the objective functions, to a multiplicative factor of $\frac{(e-1)^2}{e^2-e+1} \approx 0.52$ 
w.h.p or $\frac{1}{3}$ deterministically. 
\end{abstract}