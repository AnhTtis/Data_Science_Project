\subsection{Randomized Approximation Algorithms}
\eden{TO myself: maybe to add motivation...}
Another question is what happens when we consider approximation procedures that are also randomized?
We saw that if \textsf{OP} is a $(1-\beta)$ approximation algorithm to \eqref{eq:vsums-OP}, then Algorithm \ref{alg:basic-ordered-Outcomes} outputs a $(1-1-\frac{\beta}{1-\beta +\beta^2})$ approximately-optimal solution.
If \textsf{OP} is a $(1-\beta)$ approximation algorithm to \eqref{eq:vsums-OP} that s

So, using an approximate solver to \eqref{eq:vsums-OP}, what guarantee can we provide for the overall resulting solution? 
\begin{theorem}
\label{th:main}
Let $\beta<1/2$, and \textsf{OP} be a procedure that outputs a $(1-\beta)$ approximation to \eqref{eq:vsums-OP}. Then Algorithm \ref{alg:basic-ordered-Outcomes} outputs a $(1-\frac{\beta}{1-\beta +\beta^2})$-approximate Leximin solution.  
\end{theorem}