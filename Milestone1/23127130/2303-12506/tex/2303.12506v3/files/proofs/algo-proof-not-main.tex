
\subsection{Using an approximate solver for \eqref{eq:basic-OP}}
\label{sub:approximate-P1}
Recall that \eqref{eq:basic-OP} is described as follows:
\begin{align}
 \max &&&\ztVar{x}   \tag{\progBasic}\\
        s.t. &&& (\text{\progBasic.1}) \Hquad x \in S \nonumber\\
              &&& (\text{\progBasic.2}) \Hquad \valBy{\ell}{x}\geq z_{\ell} & \forall \ell \in [t-1] \nonumber \\
               &&& (\text{\progBasic.3}) \Hquad \valBy{t}{x} \geq \ztVar{x} \nonumber   
\end{align} 
This section proves the following lemma:
\begin{lemma}
    Let $\multApprox\in (0,1]$, $\additiveApprox \geq 0$, and \textsf{OP} be an $(\multApprox,\additiveApprox)$-approximation procedure to \eqref{eq:basic-OP}. Then Algorithm \ref{alg:basic-ordered-Outcomes} outputs an $\left(\multApprox, \additiveApprox\right)$-leximin-approximation.  
\end{lemma}

\begin{proof}
    Let $x^*$ be the returned solution and assume by contradiction that it is \emph{not} $\left(\multApprox, \additiveApprox\right)$-approximately leximin-optimal.
    This means that there exists a $y \in S$ that is $(\multApprox, \additiveApprox)$-leximin-preferred over it. 
    That is, there exists an integer $k \in [n]$ such that:
    \begin{align*}
        \forall i < k \colon \Hquad &\valBy{i}{y} \geq \valBy{i}{x^*}\\
        & \valBy{k}{y} > \frac{1}{\multApprox}\left(\valBy{k}{x^*}+\additiveApprox\right)
    \end{align*}
    
    Since $x^*$ is a solution to  \eqref{eq:basic-OP} that was solved in the iteration $t=n$, it must satisfy all its constraints, and therefore:
    \begin{align}\label{eq:p1-x-to-z}
        \forall i \in [n] \colon \Hquad \valBy{i}{x^*} \geq z_i
    \end{align}
    by constraint (\progBasic.2) for $i<n$ and by constraint (\progBasic.3) for $i=n$.
    
    As $y$'s smallest $k$ values are at least as those of $x^*$, we can conclude that for each $i\leq k$ the $i$-th smallest value of $y$ is at least $z_i$.
    Therefore, $y$ is feasible to  \eqref{eq:basic-OP} that was solved in the iteration $t=k$.
    
    During the algorithm run, $z_k$ was obtained as an $(\multApprox, \additiveApprox)$-approximation to \eqref{eq:basic-OP} that was solved in the iteration $t=k$ , and therefore, the optimal value of this problem
     is at most $\frac{1}{\multApprox}(z_k+\additiveApprox)$.
    But, the objective value $y$ yields in this problem is $\valBy{k}{y}$, which is higher than this value:
    \begin{align*}
        \valBy{k}{y} &> \frac{1}{\multApprox}\left(\valBy{k}{x^*}+\additiveApprox\right)\\
        &\geq \frac{1}{\multApprox}\left(z_k+\additiveApprox\right) && \text{(By Equation \eqref{eq:p1-x-to-z} for $i=k$)}
    \end{align*}
    This is a contradiction.
\end{proof}


% --------------------------
\subsection{Equivalence of \eqref{eq:sums-OP} and \eqref{eq:vsums-OP}}
\label{sub:equivalence-P2-P3}
Recall the problems' descriptions:
 
\begin{align*}
\max &&&\ztVar{x} \tag{\progSums}\\
s.t. &&& (\text{\progSums.1}) \quad x \in S \\
&&& (\text{\progSums.2}) \quad \sum_{i \in F'} f_i(x) \geq \sum_{i=1}^{|F'|}  z_i && \forall F' \subseteq [n], \Hquad |F'| < t \\
&&& (\text{\progSums.3}) \quad \sum_{i \in F'} f_i(x) \geq \sum_{i=1}^{t-1}  z_i + z_t && \forall F' \subseteq [n], \Hquad |F'| = t
\end{align*}

\begin{align}
\max &&& \ztVar{x} \tag{\progLinear}\\
s.t. &&& (\text{\progLinear.1}) \Hquad x \in S \nonumber  \\
                    &&& (\text{\progLinear.2}) \Hquad \ell y_{\ell} - \sum_{j=1}^n m_{\ell,j}\geq \sum_{i=1}^{\ell}  z_i && \forXinY{\ell}{t-1} \nonumber \\
                    &&& (\text{\progLinear.3}) \Hquad t y_t - \sum_{j=1}^{n} m_{t,j} \geq \sum_{i=1}^{t-1}  z_i + z_t \nonumber \\
                    &&& (\text{\progLinear.4}) \Hquad m_{\ell,j} \geq y_{\ell} - f_j(x)  && \forXinY{\ell}{t},\Hquad \forXinY{j}{n} \nonumber \\
                    &&& (\text{\progLinear.5}) \Hquad m_{\ell,j} \geq 0  &&  \forXinY{\ell}{t},\Hquad \forXinY{j}{n} \nonumber
\end{align}

We use another equivalent representation of \eqref{eq:sums-OP}, which is more compact and will simplify the proofs, also introduced by \cite{Ogryczak_2006}:
\begin{align*}
    \max &&& z_t \tag{\progCompact}\label{eq:compact-OP}\\
    s.t. &&& (\text{\progCompact.1}) \Hquad x \in S\nonumber\\
                    &&& (\text{\progCompact.2}) \Hquad \sum_{i=1}^{\ell} \valBy{i}{x} \geq \sum_{i=1}^{\ell}  z_i && \forXinY{\ell}{t-1} \nonumber\\
                    &&& (\text{\progCompact.3}) \Hquad \sum_{i=1}^{t} \valBy{i}{x} \geq \sum_{i=1}^{t-1}  z_i + z_t
\end{align*}
In this problem, constraints (\progSums.2) and (\progSums.3) are replaced by (\progCompact.2) and (\progCompact.3), respectively.  
(\progSums.2) gives, for each $\ell$, a lower bound on the sum for \emph{any} set of $\ell$ objective functions; whereas (\progCompact.2) only considers the sum of the $\ell$ \emph{smallest} such values,
and similarly for (\progSums.3) and (\progCompact.3). 

% Our next step is proving Lemma \ref{lemma:op3-to-comp} that will allow us, while proving the theorem, to assume that \textsf{OP} is an approximation procedure for\eqref{eq:compact-OP}. 

This section proves that these \emph{three} problems are \emph{equivalent} in the following sense: 
\begin{lemma}\label{lem:equivalence-of-all-three}
    Let $t \in [n]$ and let $z_1, \ldots z_{t-1} \in \mathbb{R}$.
    Then, $(x, z_t)$ is feasible for \eqref{eq:sums-OP} if and only if $(x, z_t)$ is feasible for \eqref{eq:compact-OP} if and only if there exist $y_{\ell}$ and $m_{\ell,j}$ for $\ell \in [t]$ and $j \in [n]$ such that $\left(x, z_t, (y_1, \ldots, y_t), (m_{1,1}, \ldots m_{t,n})\right)$ is feasible for \eqref{eq:vsums-OP}.
\end{lemma} 
It is clear that this lemma implies Lemma \ref{lem:equivalence}, which only claims a part of it.

We start by proving that \eqref{eq:sums-OP} and \eqref{eq:compact-OP} are equivalent. That is, $(x, z_t)$ is feasible for \eqref{eq:sums-OP} if and only if $(x, z_t)$ is feasible for \eqref{eq:compact-OP}.
First, it is clear that $x$ satisfies constraint (\progSums.1) if and only if it satisfies constraint (\progCompact.1) (as both constraints are the same, $x \in S$).
To prove the other requirements, we start with the following lemma:
\begin{lemma}\label{lemma:sums-to-comp-constrants}
    For any $x \in S$, any $\ell \in [n]$ and a constant $c \in \mathbb{R}$ the following two conditions are equivalent:
    \begin{align}\label{eq:sums-to-comp-constrants}
         \forall F' \subseteq [n], |F'| = \ell \colon \sum_{i \in F'} f_i(x) &\geq c 
         \\
         \sum_{i=1}^{\ell} \valBy{i}{x}&\geq c 
    \end{align}
\end{lemma}

\begin{proof}
    For the first direction, recall that the values $ \valBy{1}{x}, \dots,  \valBy{\ell}{x}$ were obtained from $\ell$ objective functions (those who yield the smallest value).
    By the assumption, the sum of any set of function with size $\ell$ is at least $c$; therefore, it is true in particular for the functions corresponding to the values $ (\valBy{1}{x})_{i=1}^{\ell}$.
    For the second direction, assume that $\sum_{i=1}^{\ell} \valBy{i}{x}\geq c$.
    Since $ \valBy{1}{x}, \dots,  \valBy{\ell}{x}$ are the $\ell$ smallest objective values, we get that:
    \begin{align*}
       \forall F' \subseteq [n],\Hquad |F'| = \ell \colon \quad \sum_{i \in F'}f_i(x) \geq \sum_{i=1}^s \valBy{i}{x}\geq c.
    \end{align*}
\end{proof}

Accordingly, $x$ satisfies constraint (\progSums.2) --- for any $\ell \in [t-1]$, 
    \begin{align*}
        \forall F' \subseteq [n], |F'| = \ell \colon \sum_{i \in F'} f_i(x) \geq \sum_{i=1}^{\ell} z_i
    \end{align*}
    if and only if it satisfies $\sum_{i=1}^{\ell} \valBy{i}{x} \geq \sum_{i=1}^{\ell} z_i$, which is constraint (\progCompact.2).
    Similarly, $x$ and $z_t$ satisfy constraint (\progSums.3), 
    \begin{align*}
        \forall F' \subseteq [n], |F'| = t \colon \sum_{i \in F'} f_i(x) \geq \sum_{i=1}^{t} z_i
    \end{align*}
    if and only if $\sum_{i=1}^{t} \valBy{i}{x} \geq \sum_{i=1}^{t} z_i$, which is constraint (\progCompact.3).
    That is, $x$ ans $z_t$ satisfy all the constraints of \eqref{eq:sums-OP} if and only if they satisfy all the constraints of \eqref{eq:compact-OP}.


%------------------------------------------
Now, we will prove that that \eqref{eq:compact-OP} and \eqref{eq:vsums-OP} are equivalent, that is, $(x, z_t)$ is feasible for \eqref{eq:compact-OP} if and only if there exist $y_{\ell}$ and $m_{\ell,j}$ for $\ell \in [t]$ and $j \in [n]$ such that $\left(x, z_t, (y_1, \ldots, y_t), (m_{1,1}, \ldots m_{t,n})\right)$ is feasible for \eqref{eq:vsums-OP}.

We start with the following lemma:
\begin{lemma}\label{lemma:comp-to-p3-m-sums}
    For any $x \in S$ and any constant $c \in \mathbb{R}$ (where $c$ does not depend on $j$),
    \begin{align*}
        \sum_{j=1}^n \max(0, c - f_j(x) ) = \sum_{j=1}^n \max(0, c - \valBy{j}{x} ).
    \end{align*}
\end{lemma}
\eden{TODO:
maybe to change to observation}
\begin{proof}
     Let $(\pi_1, \ldots, \pi_n)$ be a permutation of $\{1,\ldots,n\}$ such that $f_{\pi_i}(x) = \valBy{i}{x}$ for any $i \in [n]$ (notice that such permutation exists by the definition of $\valBy{}{}$).
     That is, the value that $f_{\pi_i}$ obtains is the $i$-th smallest one.
    Since each element in the sum $\sum_{j=1}^n \max(0, c - f_j(x))$ is affected by $j$ only through $f_j(x)$, the permutation $\pi$ allows us to conclude the following:
    \begin{align*}
        &\sum_{j=1}^n \max(0, c -f_j(x)) = \sum_{j=\pi_1}^{\pi_n} \max(0,c -f_j(x))\\ 
        &= \sum_{j=1}^n \max(0,c -f_{\pi_i}(x))  = \sum_{j=1}^{n} \max(0,c -\valBy{j}{x}).
    \end{align*}
\end{proof}



Lemma \ref{lemma:comp-to-p3-mapping} below proves the first direction of the equivalence between \eqref{eq:compact-OP} and \eqref{eq:vsums-OP}:
\begin{lemma}\label{lemma:comp-to-p3-mapping}
    Let $(x, z_t)$ be a feasible solution to \eqref{eq:compact-OP}.
    Then there exist $y_{\ell}$ and $m_{\ell,j}$ for $\ell \in [t]$ and $j \in [n]$ such that $\left(x, z_t, (y_1, \ldots, y_t), (m_{1,1}, \ldots m_{t,n})\right)$ is feasible for \eqref{eq:vsums-OP}.
\end{lemma}
% \begin{lemma}\label{lemma:comp-to-p3-mapping}
%     Let $(x,z_t)$ be a feasible solution  to \eqref{eq:compact-OP}. Then $(x, z_t, (y_1,\ldots,y_n), (m_{1,1},\ldots,m_{n,n}))$ is a feasible solution to \eqref{eq:vsums-OP}, where
%     \begin{align*}
%         \quad y_{\ell} &:= \valBy{\ell}{x} \Hquad\forall \ell \in [n], 
%         \\
%         m_{\ell,j} &:= \max(0,\valBy{\ell}{x} -f_j(x)) \Hquad \forall \ell \in [n], \Hquad \forall 1 \leq j \leq n 
%     \end{align*}
% \end{lemma}

\begin{proof}
    For any $\ell \in [t]$ and $j \in [n]$ define $y_{\ell}$ and $m_{\ell,j}$ as follows:
    \begin{align*}
        \quad y_{\ell} &:= \valBy{\ell}{x}
        \\
        m_{\ell,j} &:= \max(0,\valBy{\ell}{x} -f_j(x))
    \end{align*}
    
    First, since $x$ satisfies constraint (\progCompact.1), it is also satisfies constraint (\progLinear.1) of (as both constraints are the same and include only $x$).
    
    In addition, based on the choice of $y$ and $m$, it is clear that  $m_{\ell,j} \geq 0$ and $m_{\ell,j} \geq \valBy{\ell}{x} - f_j(x) = y_{\ell} - f_j(x)$ for any $\ell \in [n]$ and $j \in [n]$.
    Therefore, this assignment satisfies constraints (\progLinear.4) and (\progLinear.5).
    
    To show that this assignment also satisfies constraints (\progLinear.2) and (\progLinear.3), we first prove that for any $\ell \in [n]$ this assignment satisfies the following equation:
    \begin{align}\label{eq:comp-to-p3}
        \ell y_{\ell} - \sum_{j=1}^n m_{\ell,j} = \sum_{i=1}^{\ell} \valBy{i}{x} 
    \end{align}
    By the choice of $m$,  $\sum_{j=1}^n m_{\ell,j} = \sum_{j=1}^n \max(0, \valBy{\ell}{x} - f_j(x))$, and therefore, by Lemma \ref{lemma:comp-to-p3-m-sums}, it equals to $\sum_{j=1}^{n} \max(0,\valBy{\ell}{x} -\valBy{j}{x})$.
    \eden{\eden{TODO:
to try to add: as $\valBy{\ell}{x}$ does not depend on $j$}}
    Since $\valBy{\ell}{x}$ is the $\ell$-th smallest objective, it is clear that $\valBy{\ell}{x} - \valBy{j}{x} \leq 0$ for any $j > \ell$, and $\valBy{\ell}{x} - \valBy{j}{x} \geq 0$ for any $j \leq \ell$.
    And so, $\sum_{j=1}^n m_{\ell,j} = \ell\cdot \valBy{\ell}{x} - \sum_{i=1}^{\ell} \valBy{i}{x}$:
    \begin{align*}
        &\sum_{j=1}^n m_{\ell,j} = \sum_{j=1}^{n} \max(0,\valBy{\ell}{x} -\valBy{j}{x})\\
        &=\sum_{j=1}^{\ell} \max(0,\valBy{\ell}{x} -\valBy{j}{x}) + \sum_{j=\ell+1}^n \max(0,\valBy{\ell}{x} -\valBy{j}{x})\\
        &=\sum_{j=1}^{\ell} (\valBy{\ell}{x} -\valBy{j}{x}) + \sum_{j=\ell+1}^n 0 = \ell\cdot \valBy{\ell}{x} - \sum_{i=1}^{\ell} \valBy{i}{x}
    \end{align*}
    We can now conclude Equation \eqref{eq:comp-to-p3}:
    \begin{align*}
        \ell y_{\ell} - \sum_{j=1}^n m_{\ell,j} &= \ell \cdot \valBy{\ell}{x} - \ell\cdot \valBy{\ell}{x} + \sum_{i=1}^{\ell} \valBy{i}{x}\\
        &=\sum_{i=1}^{\ell} \valBy{i}{x}.
    \end{align*}

    Now, since $x$ satisfies constraint (\progCompact.2),  $\sum_{i=1}^{\ell} \valBy{i}{x} \geq \sum_{i=1}^{\ell} z_i$ for any $\ell \in [t-1]$.
    Therefore, by Equation \eqref{eq:comp-to-p3}, $\ell\cdot y_{\ell} - \sum_{j=1}^n m_{\ell,j}\geq  \sum_{i=1}^{\ell} z_i$ for any $\ell \in [t-1]$ and this assignment satisfies constraint (\progLinear.2).
    Similarly, as $x$ and $z_t$ satisfy constraint (\progCompact.3), $\sum_{i=1}^{t} \valBy{i}{x} \geq \sum_{i=1}^{t} z_i$, and so by Equation \eqref{eq:comp-to-p3}, $t \cdot y_{t} - \sum_{j=1}^n m_{t,j}\geq  \sum_{i=1}^{t} z_i$.
    This means that it also satisfies constraints (\progLinear.3).
\end{proof}

Finally, Lemma \ref{lemma:comp-to-p3-is-bij} below proves the second direction of the equivalence:

\begin{lemma}\label{lemma:comp-to-p3-is-bij}
    Let $\left(x, z_t, (y_1, \ldots, y_t), (m_{1,1}, \ldots m_{t,n})\right)$ be a feasible solution to \eqref{eq:vsums-OP}.
   Then, $(x, z_t)$ is feasible for \eqref{eq:compact-OP}.
\end{lemma}

\begin{proof}
    It is easy to see that since $x$ satisfies constraint (\progLinear.1), it is also satisfies constraint (\progCompact.1) (as both are the same).
    To show that it also satisfies constraints (\progCompact.2) and (\progCompact.3), we start by proving that for any $\ell \in [n]$:
    \begin{align}\label{eq:p3-to-comp}
          \sum_{i=1}^{\ell} \valBy{i}{x} \geq \ell y_{\ell} - \sum_{j=1}^n m_{\ell,j}
    \end{align}
    Suppose by contradiction that $ \sum_{i=1}^{\ell} \valBy{i}{x} < \ell y_{\ell} - \sum_{j=1}^n m_{\ell,j}$.

    
    For any $j\in [n]$ and any $\ell \in [n]$, $m_{\ell,j} \geq y_{\ell} - f_j(x)$ by constraint (\progLinear.4), and also $m_{\ell,j} \geq 0$ by constraint (\progLinear.5).
    Therefore, $m_{\ell,j} \geq \max(0,y_{\ell} -f_j(x))$.
    And so, by Lemma \ref{lemma:comp-to-p3-m-sums}, for any $\ell \in [t]$, the sum of $m_{\ell,j}$ over all $j \in [n]$ can be described as follows:
    \eden{TODO:
to try to add: as $\valBy{\ell}{x}$ does not depend on $j$}
    \begin{align}\label{eq:p3-to-conp-m-sum}
        \sum_{j=1}^n m_{\ell,j} \geq  \sum_{j=1}^n \max(0,y_{\ell} -f_j(x)) = \sum_{j=1}^n \max(0,y_{\ell} -\valBy{j}{x})
    \end{align}
    Therefore, $\sum_{i=1}^{\ell} \valBy{i}{x} <  \ell y_{\ell} - \sum_{j=1}^n \max(0,y_{\ell} -\valBy{j}{x})$ .
    Which means that:
\begin{align*}
    & \ell y_{\ell} - \sum_{i=1}^{\ell} \valBy{i}{x} - \sum_{j=1}^n \max(0,y_{\ell} -\valBy{j} {x}) > 0
\end{align*}
However, we will now see that the value of this expression is at most $0$, which is a contradiction:
\begin{align*}        
    &\ell y_{\ell} - \sum_{i=1}^{\ell} \valBy{i}{x} - \sum_{j=1}^n \max(0,y_{\ell} -\valBy{j} {x}) \\
    &= \sum_{i=1}^{\ell} y_{\ell} - \sum_{i=1}^{\ell} \valBy{i}{x} - \sum_{j=1}^n \max(0,y_{\ell} -\valBy{j} {x})\\
    & \equWithExp{\text{Since max with $0$ is at least $0$}}{\leq \sum_{i=1}^{\ell}( y_{\ell} - \valBy{i}{x} ) - \sum_{j=1}^{\ell} \max(0,y_{\ell} -\valBy{j} {x}) - \sum_{j=\ell+1}^n 0}\\
        &=  \sum_{j=1}^{\ell} \left((y_{\ell} - \valBy{j}{x}) - \max(0,y_{\ell} -\valBy{j}{x})\right)\\
        &\equWithExp{\text{Since each element\displayEcai{ of this sum} is at most $0$}}{\leq 0}
   \end{align*}
This is a contradiction; so \eqref{eq:p3-to-comp} is proved.

    Now, by constraint (\progLinear.2), $\ell y_{\ell} - \sum_{j=1}^n m_{\ell,j}\geq  \sum_{i=1}^{\ell} z_i$  for any $\ell \in [t-1]$. Therefore, by \eqref{eq:p3-to-comp}, also $\sum_{i=1}^{\ell} \valBy{i}{x} \geq \sum_{i=1}^{\ell} z_i$, which means that $x$ satisfies constraint (\progCompact.2).
    Similarly, by constraint (\progLinear.3),  $t y_{t} - \sum_{j=1}^n m_{t,j}\geq  \sum_{i=1}^{t} z_i$, and so by \eqref{eq:p3-to-comp}, also $\sum_{i=1}^{t} \valBy{i}{x} \geq \sum_{i=1}^{t} z_i$.
    This means that $x$ and $z_t$ satisfy constraint (\progCompact.3).
\end{proof}
