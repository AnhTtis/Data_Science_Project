\usepackage[utf8]{inputenc}
%--------------------------
% includes + macros
% ---------- comment the two following lines for ECAI 
% \usepackage{amsmath,amsthm,amssymb}
\usepackage{amsmath,amssymb}
% \newtheorem{theorem}{Theorem}[section]
% -------
\newtheorem{lemma}[theorem]{Lemma}
\newtheorem*{theorem*}{Theorem}
\newtheorem*{lemma*}{Lemma}
\newtheorem{corollary}[theorem]{Corollary}
\newtheorem{observation}[theorem]{Observation}
% \newtheorem{potentialDefinition}[theorem]{Potential Definition}
\theoremstyle{definition}
\newtheorem{definition}{Definition}[section]

\usepackage{algcompatible}
%\usepackage{algorithmic}
\usepackage{algorithm}

\usepackage{mathtools}
\usepackage{mathrsfs}
\DeclareMathAlphabet{\mathpzc}{OT1}{pzc}{m}{it}
%\usepackage{stmaryrd} % for obj


% to separate the algorithm into 2 pages with \algrestore{myalg}
% \usepackage{algcompatible}
% \usepackage{algorithm}


% \usepackage{algorithm,algpseudocode,caption}



\newcommand*{\RR}{%
  \textsf{I\kern-.3ex R}%
}
% \usepackage{xcolor}
\usepackage[dvipsnames]{xcolor}

\usepackage{blindtext}
\usepackage{hyperref}

\usepackage{csquotes}
\usepackage{array}

\usepackage[normalem]{ulem}


\ifdefined\DRAFT
	\newcommand{\er}[1]{\textcolor{blue}{#1}}
	\newcommand{\erel}[1]{\er{(Erel says: #1)}}
	\newcommand{\eden}[1]{\textcolor{red}{(Eden says: #1)}}
	\newcommand{\emark}[1]{\textcolor{red}{ #1}}
	\newcommand{\yon}[1]{\textcolor{Green}{(Yonatan says: #1)}}
     \newcommand{\yonatan}[1]{\textcolor{Green}{(Yonatan says: #1)}}
\else
	\newcommand{\er}[1]{#1}
	\newcommand{\erel}[1]{}
	\newcommand{\eden}[1]{}
	\newcommand{\emark}[1]{#1}
	\newcommand{\yon}[1]{}
\fi


% for def section 
\newcommand{\multApprox}{\alpha}
\newcommand{\additiveApprox}{\epsilon}
\newcommand{\DEFmultApprox}{\alpha}
\newcommand{\DEFadditiveApprox}{\epsilon}
\newcommand{\DEFmultError}{\overline{\multApprox}}
% \newcommand{\DEFmultErrorOf}[1]{\theta(#1)}
\newcommand{\DEFmultErrorOf}[1]{\overline{#1}}
\newcommand{\DEFadditiveError}{\epsilon}


%----------------------
% small quad 
\newcommand{\Hquad}{\hspace{0.5em}}
% long quad
\newcommand{\Lquad}{\hspace{5em}}

%----------------------
% leximin
\newcommand{\multError}{\beta}
\newcommand{\additiveError}{\epsilon}


\newcommand{\leximinPreferred}{\succ}
\newcommand{\nLeximinPreferred}{\nsucc}

\newcommand{\alphaBetaPreferred}{\succ_{(\DEFmultApprox,\DEFadditiveApprox)}}
\newcommand{\alphaBetaPreferredParams}[2]{\succ_{(#1,#2)}}
\newcommand{\nAlphaBetaPreferred}{\nsucc_{(\DEFmultApprox,\DEFadditiveApprox)}}
\newcommand{\nAlphaBetaPreferredParams}[2]{\nsucc_{(#1,#2)}}


\newcommand{\relationSetAlphaBeta}{\mathcal{R}_{(\multApprox,\additiveApprox)}}
\newcommand{\relationSetParams}[2]{\mathcal{R}_{(#1,#2)}}
\newcommand{\relationSetApproxParams}[2]{\mathcal{R}_{(#1,#2)}}


% \newcommand{\deltaPreferred}{\succ_{\delta}}
% \newcommand{\xPreferred}[1]{\succ_{#1}}
% \newcommand{\nDeltaPreferred}{\nsucc_{\delta}}
% \newcommand{\nxPreferred}[1]{\nsucc_{#1}}
% \newcommand{\epsilonLeximinPreferred}{\succ_{\epsilon}}

%----------------------
% lexical
\newcommand{\lexicalPreferred}{\succ_{lexical}}
\newcommand{\epsilonLexicalPreferred}{\succ_{\epsilon-lexical}}

%----------------------
% objective-functions
\newcommand{\allObjFunc}{\mathbf{F}}
% \newcommand{\allValues}[1]{\mathbf{F}(#1)}
\newcommand{\allValues}[1]{\mathbf{V}(#1)}
\newcommand{\sortedValues}[1]{\mathbf{V}^{\uparrow}(#1)}
\newcommand{\funcBy}[2]{\mathbf{F}_{[#1]}^{#2}}
% \newcommand{\valBy}[2]{\mathbf{F}_{[#1]}^{#2}(#2)}
% \newcommand{\valBy}[2]{\mathbf{F}(#2,#1)}
% \newcommand{\valBy}[2]{\mathbf{F}(#2)_{[#1]}}
% \newcommand{\valBy}[2]{\mathcal{O}(#2, #1)}
% \newcommand{\valBy}[2]{\mathpzc{obj}_{#1}^{\uparrow}(#2)}
 \newcommand{\valBy}[2]{\mathpzc{V}_{#1}^{\uparrow}(#2)}
% \newcommand{\valBy}[2]{\mathcal{OBJ}_{#1}^{\uparrow}(#2)} 
% \newcommand{\valBy}[2]{\mathcal{obj}_{#1}^{\shortuparrow}(#2)} % stmaryrdd
% \newcommand{\valBy}[2]{\mathcal{V}_{#1}^{\uparrow}(#2)} 
%\newcommand{\valBy}[2]{\mathpzc{V}_{#1}^{\uparrow}(#2)} 

%-----------------------
% optimization problems
\newcommand{\OPt}{OP3}
\newcommand{\OPtE}{\OPt_e} % explicit representation
\newcommand{\OPtP}{\OPt_p} % polynomial-size representation
\newcommand{\OPtC}{\OPt_c} % compact representation

%-----------------
% alg proofs
\newcommand{\retSol}{x^*}
% \newcommand{\ztVar}[1]{z_t^{#1}}
\newcommand{\ztVar}[1]{z_t}
\newcommand{\sOptT}[1]{\mathpzc{OPT}_{#1}}

% \newcommand{\multError}{\alpha}
% \newcommand{\additiveError}{\beta}

%-----------------------
% power set
\newcommand{\powerSet}[1]{P(#1)}
\newcommand{\powerSetFromTo}[3]{P(#1,#2,#3)}
\newcommand{\setPlus}[2]{{#1}_{#2}^{+}}


% alternatives:
% \newcommand{\allObjFunc}{\mathcal{F}}
% \newcommand{\allValues}[1]{\mathcal{F}(#1)}
% \newcommand{\funcBy}[2]{\mathcal{F}_{[#1]}^{#2}}
% \newcommand{\valBy}[2]{\mathcal{F}_{[#1]}^{#2}(#2)}
% \newcommand{\funcBy}[2]{\mathcal{F}_{#1}^{#2}}
% \newcommand{\valBy}[2]{\mathcal{F}_{#1}^{#2}(#2)}


%----------------------
% priority-sets
\newcommand{\priority}[1]{\mathbf{Q}_{#1}}


\newcommand{\lexmaxmin} {\operatorname{lex} \max \min}
\newcommand{\maxlexmin} {\max \operatorname{lex} \min}


% smaller quote
\renewenvironment{quote}
  {\list{}{\rightmargin=0.6cm \leftmargin=0.6cm}%
   \item\relax}
  {\endlist}

% tables
\usepackage{diagbox}
\usepackage{multirow, tabularray}
\usepackage{rotating}
\usepackage{booktabs}

% names of the single objective optimization problems
\newcommand{\progBasic}{P1}
\newcommand{\progSums}{P2}
\newcommand{\progCompact}{P2-Comp}
\newcommand{\progLinear}{P3}
\newcommand{\progAppFirst}{C1}
\newcommand{\progAppSecond}{C2}
\newcommand{\progAppDual}{D2}


% for table caption centring 
\usepackage[justification=centering, font=footnotesize]{caption}

\ifdefined\ONECOL
	\newcommand{\displayComsoc}[1]{#1}
        \newcommand{\displayEcai}[1]{}
        \newcommand{\forXinY}[2]{ #1 = 1, \ldots, #2}
        \newcommand{\equWithExp}[2]{#2 && (#1)}
\else
	\newcommand{\displayComsoc}[1]{}
    \newcommand{\displayEcai}[1]{#1}
    \newcommand{\forXinY}[2]{\forall #1 \in [#2]}
    \newcommand{\equWithExp}[2]{#1\text{:}\\ & #2}
\fi

\ifdefined\FULLVER
    \newcommand{\fullVer}{}
    \newcommand{\appendixName}[2]{#1}
\else
    \newcommand{\fullVer}{ of the full version}
    \newcommand{\appendixName}[2]{#2}
\fi

% \ifdefined\ECAI
% 	\newcommand{\displayEcai}[1]{#1}
% \else
% 	\newcommand{\displayEcai}[1]{}
% \fi