
\section{Stochastic Allocations of Indivisible Goods}\label{sec:app}
% title: what is the purpose of this section
In this section, we consider a particular application of our results, for the problem of \textit{stochastic allocations of indivisible goods}. 
We prove that, under the setting described bellow, a leximin approximation with \emph{only} a multiplicative error can be obtained in polynomial time.
Specifically, we prove that a $\frac{1}{3}$-approximation\footnote{Throughout this section, we only discuss multiplicative approximations; so, for brevity, we use the term "$\multApprox$-approximation" to refer to "$(\multApprox,0)$-approximation".} can be obtained deterministically, whereas a $\frac{(e-1)^2}{e^2-e+1} \approx 0.52$-approximation can be obtained w.h.p.
As a reference point, it is worth noting that the problem of maximizing the egalitarian welfare in the same settings has been shown to be NP-hard to approximate to a (multiplicative) factor better than than $1-\frac{1}{e} \approx 0.632$ \cite{kawase_max-min_2020}.
% \eden{should we say, as \textcite{kawase_max-min_2020}, that given an approximation algorithm with a multiplicative error of $\multError$ for welfare maximization, we obtain a approximate leximin with a multiplicative error of $\frac{\multError}{1-\multError +\multError^2}$?}.
However, as an $\multApprox$-approximation to leximin is first and foremost an $\multApprox$-approximation to the egalitarian welfare\displayEcai{\footnote{It can be easily verified that if a solution is an $\multApprox$-leximin-approximation, then it is also an $\multApprox$-approximation to the egalitarian welfare.}}, the same hardness result applies to our problem as well.


% title: basic model  
The setting postulates a set of $n$ agents $1,\ldots,n,$ and $m$ items, $1,\ldots,m,$ to be distributed amongst the agents.
A \emph{deterministic allocation} of the items to the agents is a mapping $A:[m]\rightarrow [n]$, determining which agent gets each item.
Note that as the term "deterministic" is used in this section also when discussing algorithms, we will use the term \emph{simple allocation} from now on.
% Given an allocation $A$, we denote by $A_i=A^{-1}(i)$ - the set of items allocated to $i$ in $A$. 
We denote by $\mathcal{A}$ the set of simple allocations. 
Each agent $j$ is associated with a function $u_j \colon \mathcal{A} \to \mathbb{R}_{\geq 0}$ that describes its utility from a simple allocation.

A \emph{stochastic allocation}, $d$, is a distribution over the simple allocations.  The set of all possible stochastic allocations is: 
\begin{align*}
    \mathcal{D} = \{d \mid p_d \colon \mathcal{A} \to [0,1], \sum_{A \in \mathcal{A}} p_d(A) = 1\}
\end{align*}  
\displayComsoc{
\eden{I'm not sure whether to move the formal definitions to the appendix (if so, where?) or to omit them.}
Agents are assumed to assign a positive utility to the set of all items and to care only about their own share (allowing us to use the following abuse of notation in which $u_j$ takes a bundle $b$ of items). 
Their utilities are assumed to be normalized ($u_j(\emptyset) = 0$), monotone ($u_j(b_1) \leq u_j(b_2)$ if $b_1 \subseteq b_2$), and submodular ($u_j(b_1) + u_j(b_2) \geq u_j(b_1 \cup b_2) + u_j(b_1 \cap b_2)$ for any bundles $b_1,b_2$); and to be given in the \emph{value oracle model} --- that is, we do not have a direct access to them, but only to an oracle that indicates the value of an agent from a given simple allocation.
}
\displayEcai{
Agents are assumed to care only about their own share (allowing us to use the following abuse of notation in which $u_j$ takes a bundle $b$ of items), their utilities are assumed to be normalized ($u_j(\emptyset) = 0$), monotone ($u_j(b_1) \leq u_j(b_2)$ if $b_1 \subseteq b_2$), and submodular ($u_j(b_1) + u_j(b_2) \geq u_j(b_1 \cup b_2) + u_j(b_1 \cap b_2)$ for any bundles $b_1,b_2$).
It is assumed that each agent assigns a positive utility to the set of all items.
The utilities $(u_i)_{i=1}^n$ are assumed to be given in the \emph{value oracle model}, meaning that we do not have a direct access to them, but only to an oracle that indicates the value of an agent from a given simple allocation.
}
% \eden{z1 > 0}
Lastly, the agents are assumed to be risk-neutral. This means that, given a stochastic allocation $d$, the utility of each agent $j$ is given by the expected value:
\begin{align*}
	E_j(d) = \sum_{A\in \mathcal{A}}p_d(A)\cdot u_j(A).
\end{align*}
The goal is to find a stochastic allocation that maximizes the set of functions $E_1,\ldots,E_n$. 
Formally, we consider the following problem:
\begin{align*}
	\lexmaxmin \quad &\{E_1(d), E_2(x), \dots E_n(d)\} \\
	s.t. \quad  & d \in \mathcal{D}
\end{align*}
% \eden{maybe $\maxlexmin$?} 
That is, the feasible region is the set of stochastic allocations ($S = \mathcal{D}$) and the objective functions are the expected utilities ($f_i = E_i$ for any $i\in [N]$).


% \begin{lemma}
%     If a solution is an $\multApprox$-approximation to leximin, then it is also an $\multApprox$-approximation to the egalitarian welfare.
% \end{lemma}

% \eden{I removed the proof for now due to lack of space}
% \begin{proof}
%     Let $d \in \mathcal{D}$ be a stochastic allocation, and assume that it is an $\multApprox$-approximation to leximin. 
%     By definition, there is no solution that is $(\multApprox,0)$-preferred over it --- $d' \nAlphaBetaPreferredParams{\multApprox}{0} d$ for any $d' \in \mathcal{D}$.
%     Suppose by contradiction that $d$ is \emph{not} an $\multApprox$-approximation to the egalitarian welfare, and let $d^*$ be the optimal solution to this problem.
%     This means that the smallest objective value of $d'$ is less than smallest objective value of $d^*$ times $\multApprox$ --- $\valBy{1}{d} < \multApprox \cdot\valBy{1}{d^*}$.
%     But it follows that $d^*\alphaBetaPreferredParams{\multApprox}{0} d $;
%     for $k=1$, the required for $j<k$ is vacuously true 
%     and $\valBy{1}{d^*} > \frac{1}{\multApprox}\valBy{1}{d}$.
%     However, we know that the $d' \nAlphaBetaPreferredParams{\multApprox}{0} d$ for any $d' \in \mathcal{D}$, so it is true in particular for $d^*$. This is a contradiction.
% \end{proof}
% As the proof is straightforward, it is omitted.


Kawase and Sumita \cite{kawase_max-min_2020} present an approximation algorithm, which relates the problem of finding a stochastic allocation that approximates the egalitarian welfare, to the problem of finding a \emph{simple} allocation that approximates the \emph{utilitarian welfare} 
(i.e., the sum of utilities):
\begin{align*}
 \max \quad &\sum_{i=1}^n u_i(A)   \;\;
        \quad s.t. \quad   A \in \mathcal{A}.
        \tag{U1}\label{eq:utilitarian}
\end{align*}
% \erel{Can you write the maximization problem for utilitarian welfare?}

We adapt their algorithm to find an approximately leximin-optimal allocation as follows:
\begin{theorem}
\label{th:app-main}
Given a randomized algorithm that returns a simple allocation that $\multError$-approximates the utilitarian welfare (with success probability $p$).
Then, Algorithm \ref{alg:basic-ordered-Outcomes} can be used to obtain a stochastic allocation that approximates leximin with a multiplicative error of at most $\frac{\multError}{1-\multError +\multError^2}$ (with the same probability).
\end{theorem}

A complete proof is given in Appendix \ref{sec:app-sec-proofs}.
Here we provide an outline.
We start by taking \eqref{eq:vsums-OP} and replacing the constraint (\progLinear.1) with the constraints describing a feasible stochastic allocation. 
Here we face a computational challenge: the number of variables describing a stochastic allocation is exponential in the input size, as we need a variable for each simple allocation. 
We address this challenge by moving to the dual of a closely related program.
The dual has polynomially-many variables but exponentially-many constraints.
However, we prove that a randomized approximate separation-oracle for this program can be designed and used within a variant of the ellipsoid method to approximate \eqref{eq:vsums-OP}.

Theorem \ref{th:app-main} yield two immediate corollaries, using  known algorithms to approximate the utilitarian welfare when the agents' utility functions are monotone and submodular.

First, there are deterministic $\frac{1}{2}$-approximation algorithms\cite{Fisher1978, Buchbinder2019}, and therefore:
\begin{corollary}
    Algorithm \ref{alg:basic-ordered-Outcomes} can be used to obtain a stochastic allocation that approximates leximin with a multiplicative error at most 
    $
    \frac{0.5}{1-0.5+0.5^2} = 
    \frac{2}{3}$.
\end{corollary}
\noindent Second, there is a randomized $(1-\frac{1}{e})$-approximation algorithm w.h.p \cite{vondrak_optimal_2008}, and therefore:
\begin{corollary}
    Algorithm \ref{alg:basic-ordered-Outcomes} can be used to obtain a stochastic allocation that approximates leximin with a multiplicative error at most $\frac{e}{e^2-e+1} \approx 0.48$ w.h.p.
\end{corollary}

% The proof of Theorem \ref{th:app-main} is provided in Appendix \ref{sec:app-sec-proofs}. 