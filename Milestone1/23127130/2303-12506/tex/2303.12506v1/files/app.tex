
\section{Stochastic Allocations of Indivisible Goods}\label{sec:app}
% title: what is the purpose of this section

In this section we consider a particular application of our results, for the problem of \textit{stochastic allocations of indivisible goods}. 
% title: basic model  
The setting postulates a set of $n$ agents $1,\ldots,n,$ and $m$ items, $1,\ldots,m,$ to be distributed amongst the agents.
A \emph{deterministic allocation} of the items to the agents is a mapping $A:[m]\rightarrow [n]$, determining which agent gets each item. 
% Given an allocation $A$, we denote by $A_i=A^{-1}(i)$ - the set of items allocated to $i$ in $A$. 
We denote by $\mathcal{A}$ the set of deterministic allocations. 
A \emph{stochastic allocation}, $d$, is a distribution over the deterministic allocations.  The set of all possible stochastic allocations is: 
\begin{align*}
    \mathcal{D} = \{d \mid p_d \colon \mathcal{A} \to [0,1], \sum_{A \in \mathcal{A}} p_d(A) = 1\}
\end{align*}   
Each agent $j$ is associated with a function $u_j \colon \mathcal{A} \to \mathbb{R}_{\geq 0}$ that describes its utility from a deterministic allocation.
Agents are assumed to care only about their own share (allowing us to use the following abuse of notation in which $u_j$ takes a bundle $b$ of items), their utilities are assumed to be normalized ($u_j(\emptyset) = 0$), monotone ($u_j(b_1) \leq u_j(b_2)$ if $b_1 \subseteq b_2$), and submodular ($u_j(b_1) + u_j(b_2) \geq u_j(b_1 \cup b_2) + u_j(b_1 \cap b_2)$ for any bundles $b_1,b_2$).
We also assume that utilities $(u_i)_{i=1}^n$ are given in the \emph{value oracle model}, meaning that we do not have a direct access to them, but only to an oracle that indicates the value of an agent from a given deterministic allocation.
Lastly, it is assumed that each agent has a positive utility from the set of all items.
% \eden{z1 > 0}
% \eden{should we explain somewhere that submodularity in the context of people's utilities makes a lot of sense? Maybe to explain what it is with diminishing returns?}
% \eden{should we say something about the value-oracle model? if so, where?}

% \eden{maybe to change "when agents have submodular utilities" to: "under these settings"?}
We prove that, in this setting, an approximately-optimal leximin solution with \emph{only} a multiplicative error can be obtained in polynomial time.
Specifically, we prove that a $\frac{1}{3}$-approximation\footnote{Throughout this section, we only discuss multiplicative approximations; so, for brevity, we use the term "$\multApprox$-approximation" to refer to "$(\multApprox,0)$-approximation".} can be obtained deterministically, whereas a $\frac{(e-1)^2}{e^2-e+1} \approx 0.52$-approximation can be obtained w.h.p.
As a reference point, it is worth noting that the problem of maximizing the egalitarian welfare in these settings has been shown to be NP-hard to approximate to a (multiplicative) factor better than than $1-\frac{1}{e} \approx 0.632$ \cite{kawase_max-min_2020}.
% \eden{should we say, as \textcite{kawase_max-min_2020}, that given an approximation algorithm with a multiplicative error of $\multError$ for welfare maximization, we obtain a approximate leximin with a multiplicative error of $\frac{\multError}{1-\multError +\multError^2}$?}.

% \footnote{Recall that a $(\multError,\additiveError)$-approximation has at most a multiplicative error of $\multError$ and at most an additive error of $\additiveError$. Therefore, the lower the parameters, the higher the accuracy.}




Given a stochastic allocation $d$, the expected utility of agent $j$ is given by
\begin{align*}
	E_j(d) = \sum_{A\in \mathcal{A}}p_d(A)\cdot u_j(A)
\end{align*}
The goal is to find a stochastic allocation that maximizes the set of functions $E_1,\ldots,E_n$. 
Formally, we consider the following problem:
\begin{align*}
	\lexmaxmin \quad &\{E_1(d), E_2(x), \dots E_n(d)\} \\
	s.t. \quad  & d \in \mathcal{D}
\end{align*}
% \eden{maybe $\maxlexmin$?} 
\eden{should write something about the output size, as \textcite{kawase_max-min_2020}}
That is, the feasible region is the set of stochastic allocations ($S = \mathcal{D}$) and the objective functions are the expected utilities ($f_i = E_i$ for any $i\in [N]$).

However, we shall see that an $\multApprox$-approximation to leximin is first and foremost an $\multApprox$-approximation to the egalitarian welfare. Therefore, the same hardness result applies to our problem as well.

\begin{lemma}
    If a solution is an $\multApprox$-approximation to leximin, then it is also an $\multApprox$-approximation to the egalitarian welfare.
\end{lemma}

% \eden{I removed the proof for now due to lack of space}
% \begin{proof}
%     Let $d \in \mathcal{D}$ be a stochastic allocation, and assume that it is an $\multApprox$-approximation to leximin. 
%     By definition, there is no solution that is $(\multApprox,0)$-preferred over it --- $d' \nAlphaBetaPreferredParams{\multApprox}{0} d$ for any $d' \in \mathcal{D}$.
%     Suppose by contradiction that $d$ is \emph{not} an $\multApprox$-approximation to the egalitarian welfare, and let $d^*$ be the optimal solution to this problem.
%     This means that the smallest objective value of $d'$ is less than smallest objective value of $d^*$ times $\multApprox$ --- $\valBy{1}{d} < \multApprox \cdot\valBy{1}{d^*}$.
%     But it follows that $d^*\alphaBetaPreferredParams{\multApprox}{0} d $;
%     for $k=1$, the required for $j<k$ is vacuously true 
%     and $\valBy{1}{d^*} > \frac{1}{\multApprox}\valBy{1}{d}$.
%     However, we know that the $d' \nAlphaBetaPreferredParams{\multApprox}{0} d$ for any $d' \in \mathcal{D}$, so it is true in particular for $d^*$. This is a contradiction.
% \end{proof}
As the proof is straightforward, it is omitted.


\textcite{kawase_max-min_2020} present an approximation algorithm that relates the problem of finding a stochastic allocation that approximates the egalitarian welfare, to the problem of finding a \emph{deterministic} allocation that approximates the \emph{utilitarian welfare} (i.e., the sum of utilities):
\begin{align*}
 \max \quad &\sum_{i=1}^n u_i(A)   \;\;
        \quad s.t. \quad   A \in \mathcal{A}  \tag{U1}\label{eq:utilitarian}
\end{align*}
% \erel{Can you write the maximization problem for utilitarian welfare?}

In this paper, we adapt their algorithm and prove the following relation to leximin:
\begin{theorem}
\label{th:app-main}
Suppose we are given a randomized algorithm that returns a deterministic allocation that approximates the utilitarian welfare with multiplicative error $\multError$ (with success probability $p$).
Then, Algorithm \ref{alg:basic-ordered-Outcomes} can be used to obtain a stochastic allocation that approximates leximin with a multiplicative error of at most $\frac{\multError}{1-\multError +\multError^2}$ (with the same probability).
\end{theorem}

\noindent Proving this theorem will yield two immediate results since there are known algorithms to approximate the utilitarian welfare when the agents' utility functions are monotone and submodular.
First, there are deterministic approximation algorithms with a multiplicative error $\frac{1}{2}$ \cite{Fisher1978, Buchbinder2019}, and therefore:
\begin{corollary}
    Algorithm \ref{alg:basic-ordered-Outcomes} can be used to obtain a stochastic allocation that approximates leximin with a multiplicative error at most 
    $
    \frac{0.5}{1-0.5+0.5^2} = 
    \frac{2}{3}$.
\end{corollary}
\noindent Second, there is a randomized approximation algorithm with a multiplicative error of  $\frac{1}{e}$ w.h.p \cite{vondrak_optimal_2008}, and therefore:
\begin{corollary}
    Algorithm \ref{alg:basic-ordered-Outcomes} can be used to obtain a stochastic allocation that approximates leximin with a multiplicative error at most $\frac{e}{e^2-e+1} \approx 0.48$ w.h.p.
\end{corollary}

The proof of Theorem \ref{th:app-main} is provided in Appendix \ref{sec:app-sec-proofs}. 