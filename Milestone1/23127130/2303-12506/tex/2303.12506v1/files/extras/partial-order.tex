\section{The Approximate Leximin Order}\label{sec:approx-order-is-strict-partial}

Unlike the leximin order, $\leximinPreferred$, which is a strict \textbf{total} order, the approximate leximin order, $\alphaBetaPreferred$ for $\DEFmultApprox\in (0,1]$ and $\DEFadditiveApprox \geq 0$ is a strict \textbf{partial} order.
The difference is that in partial orders, not all vectors are comparable.
Consider for example the sorted vectors $(1,2)$ and $(1, 3)$. 
According to the leximin order, $(1,3)$ is clearly preferred (as $3>2$), but according to many approximate leximin orders neither one is preferred over the other, for example according to the orders $\alphaBetaPreferredParams{0.6}{0}$,$ \alphaBetaPreferredParams{1}{1}$ or $\alphaBetaPreferredParams{0.8}{0.5}$.
% (irreflexive, asymmetric and transitive).

An order is a strict partial order if it is irreflexive, transitive and asymmetric.
Lemma \ref{lemma:order-is-irreflexive} proves that the order is irreflexive, Lemma \ref{lemma:order-is-transitive} proves it is transitive, and Lemma \ref{lemma:order-is-asymmetric} proves that it is asymmetric.
% \erel{It would be good to show an example why this is not a total order.}

% need to prove irreflexive, asymmetric (we have already proved that it is transitive).

% ***[I thought it would be better to prove it on vectors (rather than "solutions") to make it as general as possible]\\

Let $\DEFmultApprox\in (0,1]$ and $\DEFadditiveApprox \geq 0$. 

\begin{lemma}\label{lemma:order-is-irreflexive}
    The approximate leximin order $\alphaBetaPreferred$ is irreflexive.
\end{lemma}

\begin{proof}
    % \eden{I used $x$ only to remind the reader what irreflexive is, maybe it should simply be in the lemma description}
    Let $x$ be a solution. We will show that $x \nAlphaBetaPreferred x$.
    As the definition requires that one component be \emph{strictly greater} than the other, it is trivial.
\end{proof}

\begin{lemma}\label{lemma:order-is-transitive}
    The approximate leximin order $\alphaBetaPreferred$ is transitive.
\end{lemma}

\begin{proof}
    Let $x,y$ and $z$ be solutions such that $x \alphaBetaPreferred y$ and $y \alphaBetaPreferred z$.
    We will prove that $x \alphaBetaPreferred z$.

    
    Since $x \alphaBetaPreferred y$, there exists an integer $ k_1 \in [n]$ such that:
    \begin{align*}
        \forall j<k_1 \colon &  \valBy{j}{x} \geq \valBy{j}{y}\\
            & \valBy{k_1}{x} > \frac{1}{\DEFmultApprox} \left( \valBy{k_1}{y} + \DEFadditiveApprox \right)
    \end{align*}
    And since $y \alphaBetaPreferred z$, there exists an integer $k_2 \in [n]$ such that:
    \begin{align*}
        \forall j<k_2 \colon &  \valBy{j}{y} \geq \valBy{j}{z}\\
            & \valBy{k_2}{y} > \frac{1}{\DEFmultApprox} \left( \valBy{k_2}{z} + \DEFadditiveApprox \right) 
    \end{align*}

    As $\DEFmultApprox \in (0,1]$ and $\DEFadditiveApprox \geq 0$, it follows that:
    \begin{align}\label{eq:trans-k-s}
        \valBy{k_1}{x} > \valBy{k_1}{y}, \Hquad \valBy{k_2}{y} >  \valBy{k_2}{z}
    \end{align}

    % Accordingly, if $k_1=k_2$, then this integer, denoted by $k$, allows us to conclude that $x \alphaBetaPreferred z$. 
    % By the definitions of $k_1$ and $k_2$, for any $j<k_1=k_2$ the required holds as $\valBy{j}{x} \geq \valBy{j}{y} \geq \valBy{j}{z}$.
    % In addition, $\valBy{k_1}{x}> \valBy{k_1}{y}$ by equation \ref{eq:trans-k-s}
    % $ > \frac{1}{\DEFmultApprox} \left( \valBy{k_1}{y} + \DEFadditiveApprox \right)$ and nd  and 
    
    Let $k = \min\{k_1,k_2\}$.
    
    If $k = k_1$, by the definition of $k_1$, $\valBy{k}{x} > \frac{1}{\DEFmultApprox} \left( \valBy{k}{y} + \DEFadditiveApprox \right)$.
    However, $\valBy{k}{y} \geq \valBy{k}{z}$, by definition if $k<k_2$ and by equation \ref{eq:trans-k-s} if $k=k_2$. \ref{eq:transitive-k}
    Therefore, $\valBy{k}{x} > \frac{1}{\DEFmultApprox} \left( \valBy{k}{z} + \DEFadditiveApprox \right)$.
    
    Otherwise, if $k=k_2$, by the definition of $k_2$, $\valBy{k}{y} > \frac{1}{\DEFmultApprox} \left( \valBy{k}{z} + \DEFadditiveApprox \right)$. But, $\valBy{k}{x} \geq \valBy{k}{y}$, by definition if $k<k_1$ and by equation \ref{eq:trans-k-s} if $k=k_1$. Again, we can conclude that $\valBy{k}{x} > \frac{1}{\DEFmultApprox} \left( \valBy{k}{z} + \DEFadditiveApprox \right)$.

     In addition, for each $j<k$, since $j< k_1$ and $j < k_2$, by definition the following holds:
    \begin{align}\label{eq:transitive-k}
        \valBy{j}{x} \geq \valBy{j}{y} \geq \valBy{j}{z}
    \end{align}
    So, $k$ is an integer that satisfy all the requirements, and so, $x \alphaBetaPreferred z$.
    \end{proof}
    

    
    \begin{lemma}\label{lemma:order-is-asymmetric}
        The approximate leximin order $\alphaBetaPreferred$ is asymmetric.
    \end{lemma}
    
    \begin{proof}
        Let $x$ and $y$ be solutions such that $x \alphaBetaPreferred y$. We will show that $y \nAlphaBetaPreferred x$. 
        Assume by contradiction that $y \alphaBetaPreferred x$. 
        From Lemma \ref{lemma:order-is-transitive}, this relation is transitive. Therefore, since $x \alphaBetaPreferred y$ and $y \alphaBetaPreferred x$, also $x \alphaBetaPreferred x$.
        But, from Lemma \ref{lemma:order-is-irreflexive}, this relation is irreflexive --- a contradiction.
    \end{proof}