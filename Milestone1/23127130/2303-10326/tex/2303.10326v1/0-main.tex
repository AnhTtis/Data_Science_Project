 
\documentclass[runningheads]{llncs}
%
\usepackage[T1]{fontenc} 
\usepackage{graphicx}

\usepackage{epstopdf}
\usepackage{color}
\usepackage[table]{xcolor}
\usepackage{makecell}
\usepackage{cite}
% \usepackage{caption} 
\usepackage{multirow}
\usepackage{amsfonts,amssymb,amsmath}
\usepackage{longtable}
\usepackage{booktabs}
\usepackage{xcolor}
\usepackage{floatrow}
\usepackage{subcaption} 
\floatsetup[table]{capposition=top}
\newfloatcommand{capbtabbox}{table}[][\FBwidth]
\usepackage{marvosym}
% \usepackage{subfigure} 
 \usepackage[colorlinks]{hyperref}
\begin{document}
%
\title{Diff-UNet: A Diffusion Embedded Network\\
for Volumetric Segmentation}

\author{Zhaohu Xing\inst{1} \and
Liang Wan\inst{1}\and
Huazhu Fu\inst{2} \and
Guang Yang\inst{3} \and
Lei Zhu\inst{4,5} \textsuperscript{(\Letter)}
}
%index{Xing, Zhaohu} 
%index{Wan, Liang} 
%index{Fu, Huazhu}
%index{Yang, Guang}
%index{Zhu, Lei}

\institute{Medical College of Tianjin University, 
Tianjin, China \\
\email{xingzhaohu@tju.edu.cn} \and
 Institute of High Performance Computing, A*STAR  \and
Imperial College London \and
The Hong Kong University of Science and Technology (Guangzhou), Guangzhou, China \and
The Hong Kong University of Science and Technology, Hong Kong, China \
}
%
\maketitle              % typeset the header of the contribution
%
\begin{abstract}
%% 在3D医学图像分割任务中,许多方法通过定制的结构(例如Transformer建模全局特征)和大量无标注数据(例如自监督学习提升语义理解)实现了state-of-the-art分割结果。

%% 在这个工作中,我们使用一个结构简单的扩散模型和常规的损失函数,设计了一个新奇的端到端的训练框架,命名为Diffusion-UNet。

%% Diff-UNet的核心为一个去噪UNet网络,同时我们还添加了一个独立的多尺度图像编码器来提升去噪UNet网络的特征表达能力。

%% 此外,为了提升扩散模型预测结果的鲁棒性,我们还提出了一个step-uncertainty based Fusion(S-U Fusion)模块融合扩散模型每一步的输出。


%% 广泛的实验在三个数据集有着不同模态的展示了Diff-UNet明显的超越了state-of-the-arts.

% In 3D medical image segmentation, many approaches achieve state-of-the-art segmentation results with customized structures (e.g., Transformer modeling global features) and large amounts of unlabeled data (e.g., self-supervised learning to improve semantic understanding).
In recent years, Denoising Diffusion Models have demonstrated remarkable success in generating semantically valuable pixel-wise representations for image generative modeling. In this study, we propose a novel end-to-end framework, called Diff-UNet, for medical volumetric segmentation. Our approach integrates the diffusion model into a standard U-shaped architecture to extract semantic information from the input volume effectively, resulting in excellent pixel-level representations for medical volumetric segmentation.
To enhance the robustness of the diffusion model's prediction results, we also introduce a Step-Uncertainty based Fusion (SUF) module during inference to combine the outputs of the diffusion models at each step. We evaluate our method on three datasets, including multimodal brain tumors in MRI, liver tumors, and multi-organ CT volumes, and demonstrate that Diff-UNet outperforms other state-of-the-art methods significantly. Our experimental results also indicate the universality and effectiveness of the proposed model.
The proposed framework has the potential to facilitate the accurate diagnosis and treatment of medical conditions by enabling more precise segmentation of anatomical structures. 
The codes of Diff-UNet are available at 
\href{https://github.com/ge-xing/Diff-UNet}{https://github.com/ge-xing/Diff-UNet}. 

%% 然而,扩散模型不能直接用于医学图像分割问题由于它应用于连续域而不是离散域。
%% However, the diffusion model cannot directly solve the medical image segmentation problem because it is applied to the continuous rather than the discrete domains.
\keywords{Diffusion model  \and Medical Segmentation \and Volumetric Data.}
\end{abstract}

% \begin{figure}[t]
%     % \begin{subfigure}{1\linewidth}
%     %   \centering
%     % %   \includegraphics[width=1\linewidth]{figs/fig_1_moti_textattn.pdf}  
%     % %   \includegraphics[width=1\linewidth]{figs/fig_1_moti_textattn_v2.pdf}  
%     %   \includegraphics[width=1\linewidth]{figs/fig_1_moti_textattn_v5.pdf}  
%     %   \vspace{-0.5cm}
%     %     \caption{Amount of attention added to each video clip from the source video and query text in the self-attention layers of Moment-DETR encoder.}
%     %     % \caption{Distribution of attention for source and query in Moment-DETR encoder}
%     %     % Visualization of video clip's self-attention score in Moment-DETR encoder.
%     %   \label{fig:fig1_text_attn_ex}
%     % \end{subfigure}%\hfill% or  or \hspace{0.3\textwidth}
%     \vspace{0.2cm}
%     % \begin{subfigure}{1\linewidth}
%       \centering
%     %   \includegraphics[width=1\linewidth]{figs/fig1_moti_negattn.pdf}  
%       \includegraphics[width=1\linewidth]{figs/fig1_moti_negattn_v3.pdf}  
%       \vspace{-0.4cm}
%     %   \caption{Correspondence of saliency scores on the relevance between video clips and the text query.}
%     % \caption{Predicted saliency scores against the video relevant positive query and video irrelevant negative query}
%       \label{fig:fig1_neg_attn_ex}
%     % \end{subfigure}%\hfill% or  or \hspace{0.3\textwidth}
%     \caption{
%     % 원준 원본
%     % (a) Comparison between attention scores of source and query for each video clip~(We sum the attention scores from video and text). 
%     % We observe that the attention scores are dominated by other clips in the source video. 
%     % Text queries do not account for much attention regardless of the relevance to the video clips.
%     % \textbf{(a)} Inspection of the query dependency in Moment-DETR encoder.
%     % % We visualize the attention score of video tokens in the transformer encoder and observe that text query accounts for only a low portion of attention.
%     % % This tendency occurs regardless of the relevance between the text query and video clips. 
%     % We visualize the attention score of video tokens in the transformer encoder and observe 1) text query only accounts for a low portion of attention, and 2) relevance between video-query pair does not affect the attention scores ratio of text.
%     \textbf{(b)} Comparison of highlight-ness when relevant and non-relevant queries are input.
%     As observed in , existing work only uses queries to play an insignificant role, thereby may not be capable of detecting false queries and considering the video-query relevance even when the problem in (a) is resolved. 
%     % \SE{} % 이 부분이 "not capable of" 란 용어가 세다는 피드백이 있는 듯 합니다. 이러한 능력이 없다는 것은 굉장히 강한 어조인거 같기는 하고, 이러한 경우들이 종종 있다거나 좀 약화시킬 필요가 있어보이긴 하네요.
%     On the other hand, our QD-DETR yields a query-dependent representation that the relevance between the source video and query text is updated in the saliency scores.
%     There is a large gap between positive and negative saliency scores, and scores are consistent since the clips are all highly correlated to others.
%     }
%     \label{fig:motivation_ex}
%     % \captionsetup{belowskip=13pt}
%     % \setlength{\belowcaptionskip}{-10pt}
% \end{figure}
\begin{figure}
    \centering
    \includegraphics[width=1\linewidth]{figs/fig1_moti_negattn_1111.pdf}
    % \includegraphics[width=1\linewidth]{figs/fig1_moti_negattn_1109.pdf}
    % \includegraphics[width=1\linewidth]{figs/fig1_moti_negattn_stat.pdf}
    \vspace{-0.6cm}
    \caption{
        % \SE{} % 수정 필요
        Comparison of highlight-ness~(saliency score) when relevant and non-relevant queries are given.
        We found that the existing work only uses queries to play an insignificant role, thereby may not be capable of detecting negative queries and video-query relevance; saliency scores for clips in ground-truth~(GT) moments are low and equivalent for positive and negative queries.
        % This also results in mispredicted moments when ground-truth~(GT) moment is dominated by clips unrelated to GT since their prediction is highly focused on the video.
        % \SE{} % 여기 한번 더 보면 좋을 듯 합니다. GT moment에 unrelated한 clip이 많으면? label이 틀렷을 경우를 말씀하시는건지?
        % As observed in saliency graph, existing work only uses queries to play an insignificant role, thereby may not be capable of detecting false queries and considering the video-query relevance.
        On the other hand, query-dependent representations of QD-DETR result in corresponding saliency scores to the video-query relevance and precisely localized moments.
        % On the other hand, our QD-DETR yields a query-dependent representation that the
        % saliency scores are in accordance with the relevance between the video and query.
        % text is in accordance with the saliency scores.
        % There is a large gap between positive and negative saliency scores, and scores are consistent since the clips are all highly correlated to others.
}
    \label{fig:motivation_ex}
\end{figure}


\section{Introduction}
% 원준 원본
% Along with the advance of digital devices and platforms, video is now one of the most desired data type for consumers. However, although the large information capacity of videos may be beneficial in many aspects, e.g., informative and entertaining, on the contrary perspective, videos are time-consuming, and hard to search for desirable moments. 
% This has led many creators to use extra manpower to crop and edit the video to generate highlight clips to gain the consumer’s attention.
Along with the advance of digital devices and platforms, video is now one of the most desired data types for consumers~\cite{apostolidis2021video,wu2017deep}.
% SE: Video aware deep learning application & survey papers?
Although the large information capacity of videos might be beneficial in many aspects, e.g., informative and entertaining, inspecting the videos is time-consuming, so that it is hard to capture the desired moments~\cite{anne2017localizing,apostolidis2021video}. 
% This has led many creators to use extra manpower to crop and edit the video to generate highlight clips to gain the consumer’s attention.


% On the other side, 
Indeed, the need to retrieve user-requested or highlight moments within videos is greatly raised.
Numerous research efforts were put into the search for the requested moments in the video~\cite{anne2017localizing, gao2017tall, liu2015multi, escorcia2019temporal} and summarizing the video highlights~\cite{zhang2016video, mahasseni2017unsupervised, badamdorj2022contrastive, wei2022learning}.
% Numerous research efforts were put into the search for the requested moments in the video~\cite{anne2017localizing, gao2017tall, liu2015multi, escorcia2019temporal}, summarizing the video to generate highlights was another popular topic~\cite{zhang2016video, mahasseni2017unsupervised, badamdorj2022contrastive, wei2022learning}.
Recently, Moment-DETR~\cite{momentdetr} further spotlighted the topic by proposing a QVHighlights dataset that enables the model to perform both tasks, retrieving the moments with their highlight-ness, simultaneously.

% 원준 원본
% To detect the desired moments, previous works employed transformer encoder-decoder architectural designs to fuse the text query into the video representations. Moment-DETR~\cite{mDETR} modified detection transformer to process capture the moment as a set, and UMT~\cite{umt} implemented transformer decoder as to output clip-wise saliency. 
% Yet to their outstanding breakthroughs in the literature of moment retrieval with the seminal architectures, their limitation is that the role of the given text query is insignificant in representing the query-conditioned video representation; the attention mechanism of moment DETR is not explicitly conditioned on the text query, and the text query is conditioned on multi-modal clips where the differences between the clips are smoothed after encoding process in UMT.



% \begin{figure}[t]
% \centering
%     \begin{subfigure}[l]{0.37\linewidth}
%       \centering
%       \vspace{0.20cm}
%     %   \includegraphics[width=1\linewidth]{figs/fig_1_moti_textattn.pdf}  
%     %   \includegraphics[width=1\linewidth]{figs/fig_1_moti_textattn_v2.pdf}  
%       \includegraphics[width=1\linewidth]{figs/fig1_moti_violin_a.pdf}  
%       \vspace{-0.60cm}
%     %   \caption{text attention}
%         \caption{Importance of queries in video representation}
%       \label{fig:fig1_text_attn}
%     \end{subfigure}%\hfill% or  or \hspace{0.3\textwidth}
%     \vspace{0.2cm}
%     \begin{subfigure}[r]{0.61\linewidth}
%       \centering
%     %   \includegraphics[width=1\linewidth]{figs/fig1_moti_negattn.pdf}  
%       \includegraphics[width=1\linewidth]{figs/fig1_moti_violin_b.pdf}  
%     %   \caption{neg attention}
%         % \caption{Relation between the highlight-ness and the relevance between videos and query texts.}
%         \caption{Highlight-ness~(saliency) histogram of positive and negative video-query pairs\SE{}}
%       \label{fig:fig1_neg_attn}
%     \end{subfigure}%\hfill% or  or \hspace{0.3\textwidth}
%     % \vspace{-0.2cm}
%     \caption{Overall statistics for attention scores in Fig.~\ref{fig:motivation_ex} in QVHighlights dataset. 
%     (a) For the attention scores that measure how much the text query is generally involved in video representation, we use violin plots to show the probability density. We plot the score for each layer in the encoder.
%     % (b) Using the histogram, we compare how the baseline and QD-DETR yield different salient scores given the positive and negative video-text pairs.
%     (b) Saliency histogram shows the distributional gap between positive and negative video-text query pairs of baseline~(Moment-DETR) and proposed QD-DETR.\SE{}
%     }
%     \label{fig:motivation}
%     % \captionsetup{belowskip=13pt}
%     % \setlength{\belowcaptionskip}{-10pt}
% \end{figure}

% \begin{figure}[t]
% \centering

%     \begin{subfigure}[r]{1\linewidth}
%       \centering
%       \hspace{-0.2cm}
%     %   \includegraphics[width=1\linewidth]{figs/fig1_moti_negattn.pdf}  
%       \includegraphics[width=1.1\linewidth]{figs/fig1_moti_violin_a_v2.pdf}  
%     %   \caption{neg attention}
%         % \caption{Relation between the highlight-ness and the relevance between videos and query texts.}
%         \vspace{-0.5cm}
%         % \caption{Saliency histogram of positive and negative video-query pairs}
%         \caption{We plot the histograms and its average value~(dotted line) to compare saliency scores when true and false text queries are given for each method. (left) Since the video representations do not include much textual information, both the true and false queries yield similar saliency scores. (Middle) Even when the video representation is enforced to be updated with the textual information, the issue is not much resolved. (Right) By extracting discriminative features in the text query, distributions are differentiated.
%         % \SE{} % R1@0.5 설명
%         Also, R1@0.5 indicates evaluation metric, Recall at 1 with IoU 0.5 threshold on QVhighlight \textit{val} set.
%         }
%       \label{fig:fig1_neg_attn}
%     \end{subfigure}%\hfill% or  or \hspace{0.3\textwidth}
%     \\
%     \begin{tabular}{cc}
%     \hspace{-0.2cm}
%         \begin{minipage}{.4\linewidth}
%             \begin{subfigure}[l]{1\linewidth}
%               \centering
%             %   \vspace{0.20cm}
%             %   \includegraphics[width=1\linewidth]{figs/fig_1_moti_textattn.pdf}  
%             %   \includegraphics[width=1\linewidth]{figs/fig_1_moti_textattn_v2.pdf}  
%               \includegraphics[width=1\linewidth]{figs/fig1_moti_violin_a.pdf}  
%               \vspace{-0.60cm}
%             %   \caption{text attention}
%                 \caption{Importance of queries in video representation}
%               \label{fig:fig1_text_attn}
%             \end{subfigure}%\hfill% or  or \hspace{0.3\textwidth}
%         \end{minipage}
        
%         \begin{minipage}{.6\linewidth}
%             \vspace{-0.2cm}
%             \caption{Overall statistics of Fig.~\ref{fig:motivation_ex} in QVHighlights dataset. 
%             (a) Saliency histogram shows the distributional gap between positive and negative video-text query pairs.
%             % (a) For the attention scores that measure how much the text query is generally involved in video representation, we use violin plots to show the probability density. We plot the score for each layer in the encoder.
%             % (b) Using the histogram, we compare how the baseline and QD-DETR yield different salient scores given the positive and negative video-text pairs.
%             % (b) Text ratio in self-attention layer to  of Moment-DETR
%             % (b) Ratio of text when representing video tokens in self-attention of Moment-DETR.
%             % (b) Magnitude of attention text query involved.
%             % (b) Attention score of video tokens
%             % (b) Magnitude of text query to refine the video tokens in self-attention layer of Moment-DETR.
%             (b) Probability density depicting the weight of the text query in attention score for video clips. Scores are from the self-attention layers in Moment-DETR encoder.
%             % (b) The text query ratio in attention score of video clips (Self-attention layer in Moment-DETR encoder). We use violin plots to show probability density.
%             % 텍스트 쿼리가, 비디오 피쳐에 얼만큼 attend 하는지
%             }
%         \end{minipage}
    
%     \end{tabular}
%     \vspace{-0.5cm}
%     \label{fig:moti}
%     % \captionsetup{belowskip=13pt}
%     % \setlength{\belowcaptionskip}{-10pt}
% \end{figure}


% \begin{figure}
%     \centering
%     % \includegraphics[width=1\linewidth]{figs/fig1_moti_negattn_1109.pdf}
%     \includegraphics[width=1\linewidth]{figs/fig1_moti_negattn_stat_v2.pdf}
%     \vspace{-0.8cm}
%     \caption{
%         Histogram of saliency when the positive and negative queries are given. We plot the histograms and its average value~(dotted line) to compare saliency scores when relevant~(positive) and irrelevant~(negative) text queries are given for each method. (Left) Since the video representations do not properly reflect textual information, both the positive and negative queries yield similar saliency scores. 
%         % (Middle) Even when the video representation is enforced to be updated with the textual information, the issue is not much resolved. 
%         (Right) By representing video clips in query-dependent manner, distributions are differentiated.
%     }
%     \vspace{-0.6cm}
%     \label{fig:motivation}
% \end{figure}


% One of the demanding task is moment retrieval task, which is detecting the desired moments from the given query, typically the text query.
When describing the moment, one of the most favored types of query is the natural language sentence~(text)\cite{anne2017localizing}. 
While early methods utilized convolution networks~\cite{zhang2020learning, gao2021fast, wang2020temporally}, recent approaches have shown that deploying the attention mechanism of transformer architecture is more effective to fuse the text query into the video representation.
% To handle these modalities, previous works simply employed the attention mechanism of transformer architecture to fuse the text query into the video representation.
For example, Moment-DETR~\cite{momentdetr} introduced the transformer architecture which processes both text and video tokens as input by modifying the detection transformer~(DETR), and UMT~\cite{umt} proposed transformer architectures to take multi-modal sources, e.g., video and audio. 
Also, they utilized the text queries in the transformer decoder.
Although they brought breakthroughs in the field of MR/HD with seminal architectures, they overlooked the role of the text query.
To validate our claim, we investigate the Moment-DETR~\cite{momentdetr} in terms of the impact of text query in MR/HD~(Fig.\ref{fig:motivation_ex}).
Given the video clips with a relevant positive query and an irrelevant negative query, we observe that the baseline often neglects the given text query when estimating the query-relevance scores, i.e., saliency scores, for each video clip.
% the output saliency score, i.e. query-relevance scores.
% Based on the observation, we traced the actual saliency prediction of the model against both the video-relevant query and the irrelevant dummy one where we find that the baseline often neglects the given text query when estimating the query-relevance scores of video clips.
% For example, in Fig.~\ref{fig:motivation_ex}, saliency scores are not affected even when the query is substituted with the dummy.
% % General statistics for Fig.~\ref{fig:motivation_ex} is shown in Fig.~\ref{fig:motivation}. 
% General statistics corresponding to Fig.~\ref{fig:motivation_ex} are also shown in Fig.~\ref{fig:motivation}.



% The limitation of the concrete baseline~\cite{momentdetr} is inspected in two different aspects; 1) Utilization of text-query in the encoding process and 2) the output saliency score, i.e. query-relevance scores.
% Firstly, we visualize the attention score when video clips are given as a query in self-attention. 
% We observe that the text queries have relatively small impacts compared to other video features, as shown in Fig.~\ref{fig:fig1_text_attn_ex}.
% That is, the text does not account for much in representing every video clip, although the goal of MR/HD is to detect query-relevant moments.
% Based on the observation, we traced the actual saliency prediction of the model against both the video-relevant query and the irrelevant dummy one where we find that the baseline often neglects the given text query when estimating the query-relevance scores of video clips.
% For example, in Fig.~\ref{fig:motivation_ex}, saliency scores are not affected even when the query is substituted with the dummy.
% % General statistics for Fig.~\ref{fig:motivation_ex} is shown in Fig.~\ref{fig:motivation}. 
% General statistics are also shown in Fig.~\ref{fig:motivation}.

% Consequently, in Fig.~\ref{fig:fig1_neg_attn_ex}~(b), we found that the baseline often neglects the given text query when estimating the query-relevance scores of video clips; 
% For example, 


% We validate the previous work sometimes neglects the given query when estimating the saliency of video clips.
% For example, there is an example that the saliency scores from positive and negative queries cannot be distinguishable, as shown in Fig.~\ref{fig:fig1_neg_attn_ex}.
% % 우리는 추가로 text attention을 추가도 해봤지만, 효과가 있긴 했으나, still 이슈가 있는 것을 확인하였다?
% % Still, we observe that assuring the high attendance of text queries does not resolve the overlap which motivates us to question the quality of the naive use of task-agnostic text representation~\cite{momentdetr, umt}.
% We found that introducing the text-attention for ensuring the high attendance of text queries relieve the overlap, but there still be a severe overlap.


% To validate their limitations, we inspect the impacts of text queries in the concrete baseline~\cite{momentdetr} with the two different aspects, 1) tendency of attention in self-attention layer and 2) saliency score, i.e. query-relevance scores. \SE{} % attention 이 갑자기 등장하는가?
% Firstly, we visualize the attention score when video clips are given as a query in self-attention. We observe the text queries have relatively low attention scores compared to the video features, as shown in Fig.~\ref{fig:fig1_text_attn_ex}.
% That is, the text does not account for much in representing every video clip, although the goal of MR/HD is to detect query-relevant moments.
% Based on this observation, we trace the actual saliency prediction of the model against both positive and negative text queries.
% We validate the previous work sometimes neglects the given query when estimating the saliency of video clips.
% For example, there is an example that the saliency scores from positive and negative queries cannot be distinguishable, as shown in Fig.~\ref{fig:fig1_neg_attn_ex}.
% % 우리는 추가로 text attention을 추가도 해봤지만, 효과가 있긴 했으나, still 이슈가 있는 것을 확인하였다?
% % Still, we observe that assuring the high attendance of text queries does not resolve the overlap which motivates us to question the quality of the naive use of task-agnostic text representation~\cite{momentdetr, umt}.
% We found that introducing the text-attention for ensuring the high attendance of text queries relieve the overlap, but there still be a severe overlap.



% Thus, we 
% query dependency를 높이기 위해 
% Cross-attention? text-attention? detailed explanation on text-attention should be needed?
% By handling these two issues, we find that more precise retrieval can be achieved.
% 
% 
%
% By projecting video-discriminative text features with high text attendance to source video, we f 
% We also find the need to improve the quality of query features since assuring high text attendance also results in...
% pairs are not finetuned to be discriminative that even the similarity within the pairs does not reflect the relevance between the query and the video clips.
% General statistics for Fig.~\ref{fig:motivation_ex} is shown in Fig.~\ref{fig:motivation}. 
% \SE{} % 이거 ??로 뜨는데, 위처럼 figure 그리면 label이 안되는걸까요
% \SE{}
% 형님 아래 사항 생각 좀 해보는게 좋을 거 같아요.
% fig 1. (a) 그림만 봤을 때 모든 clip에 대해 text attention이 일정이상 존재하긴 하니까, 뭔가 not assured to be conditioned가 와닿지 않는거 같아요.
% + 왜 text가 항상 attend 해야하나?
% not assured to be conditioned --> text shows relatively low affects compared to video 같이 실제 나타난 현상까지 같이 적으면 어떨까 싶어요.
% fig 1. (b) 덜 반영한다?

% \SU{}
% 일단 text가 attend 잘 되어야 한다는 것에 좀 궁금점이 생깁니다. 결국에는 text와 관련있는 frame들을 attend해서 higlight를 찾아야 하는게 아닐까요? 그리고, 현제 저희의 모델 구조상 text query가 Key와 Value로 거의 활용되고 있는데 그렇다면 결국에는 해당 모델은 text에 대한 attention이 전혀 없다고 봐도 무방하지 않을까요? 그런 면에서 text attention을 강조하는게 좀 걸리긴 합니다.

% Specifically, the text query is not assured to be explicitly conditioned on every clip of the video, and as the query texts are evenly treated, discriminative keywords may not be spotlighted.
% attention mechanism of Moment-DETR is not explicitly conditioned on the text query as shown in Fig~\ref{}(d), and in UMT, the text are only used for conditioning the queries while the video representation are refined itself by self-attention.

% \begin{figure}[t]
%     \begin{subfigure}{1\linewidth}
%       \centering
%     %   \includegraphics[width=1\linewidth]{figs/fig_1_moti_textattn.pdf}  
%     %   \includegraphics[width=1\linewidth]{figs/fig_1_moti_textattn_v2.pdf}  
%       \includegraphics[width=1\linewidth]{figs/fig_1_moti_textattn_v4.pdf}  
%       \vspace{-0.5cm}
%     %   \caption{text attention}
%         \caption{Distribution of attention scores in Moment-DETR encoder}
%       \label{fig:fig1_text_attn}
%     \end{subfigure}%\hfill% or  or \hspace{0.3\textwidth}
%     \vspace{0.2cm}
%     \begin{subfigure}{1\linewidth}
%       \centering
%     %   \includegraphics[width=1\linewidth]{figs/fig1_moti_negattn.pdf}  
%       \includegraphics[width=1\linewidth]{figs/fig1_moti_negattn_v2.pdf}  
%       \vspace{-0.5cm}
%     %   \caption{neg attention}
%         \caption{Saliency score against positive and negative text queries}
%       \label{fig:fig1_neg_attn}
%     \end{subfigure}%\hfill% or  or \hspace{0.3\textwidth}
%     \vspace{0.2cm}
%     \begin{subfigure}{1\linewidth}
%       \centering
%     %   \includegraphics[width=1\linewidth]{figs/fig1_moti_violin.pdf}  
%       \includegraphics[width=1\linewidth]{figs/fig1_moti_violin_v2.pdf}  
%       \vspace{-0.5cm}
%       \caption{violin}
%       \label{fig:fig1_violin}
%     \end{subfigure}%\hfill% or  or \hspace{0.3\textwidth}
%     \vspace{-0.2cm}
%     \caption{(a) 1. portion of text attention vs. video attention 2. relation with text query and content (e.g. fg, bg) of clip seems not to affect the attention score
%     (b) 1. high variability even though entire clips are highly correlated with the given text query 2. positive and negative query makes overlaps on saliency score distribution
%     (3) actual distribution on validation dataset.}
%     \label{fig:motivation}
%     % \captionsetup{belowskip=13pt}
%     % \setlength{\belowcaptionskip}{-10pt}
% \end{figure}

To this end, we propose Query-Dependent DETR~(QD-DETR) that produces query-dependent video representation.
% Our key focus is to ensure each clip in predicted moments is explicitly conditioned by the query, particularly on the video-descriptive portion of the text query.
% Our key focus is to ensure that query-relevant clips are predicted by enforcing each clip to be explicitly conditioned by the query.
%Our key focus is to ensure that the model prediction for each clip is highly relevant to the query.
Our key focus is to ensure that the model's prediction for each clip is highly dependent on the query.
% by enforcing each clip to be explicitly conditioned by the query. :)
% hmm...
% \SE {} % "query-relevant clips are predicted" 이 문장이 좀 애매한거 같습니다. relevant 클립을 놓지지 않고 찾는 것을 보장한다? 이런 느낌인지 아니면 높은 saliency 를 주는게 목적이다? model prediction이 query-relevance를 반영하는 것을 보장한다?
% Our key focus is to ensure that the model prediction reflects query-relevance of clips by enforcing each clip to be explicitly conditioned by the query.
First, to fully utilize the contextual information in the query, we revise the transformer encoder to be equipped with cross-attention layers at the very first layers.
% 상익's thought :  single video - query간의 관계만 고려 - 같은 word가 더 많이 쓰이는 것을 보고 
% 교수님's thought : neg pair 를 쓰면 쿼리를 보지 않고서는 video clip간만 고려하는 것이 사라짐. 왜냐면 0으로 내보내야 하기 때문. --> SE: relative difference 만 고려하다가, 
By inserting a video as the query and a text as the key and value of the cross-attention layers, our encoder enforces the engagement of the text query in extracting video representation.
% 원준 교수님 코멘트 반영해서 다시
Then, in order to not only inject a lot of textual information into the video feature but also make it fully exploited, we leverage the negative video-query pairs generated by mixing the original pairs.
Specifically, the model is learned to suppress the saliency scores of such  negative~(irrelevant) pairs.
Our expectation is the increased contribution of the text query in prediction since the videos will be sometimes required to yield high saliency scores and sometimes low ones depending on whether the text query is relevant or not.
% \SE{}
% learns to?
% By suppressing the saliency scores of the irrelevant video-query pairs, the model learns to spotlight only the video-specific discriminative words in the query.
% % \SE{} % ====================== 상익 수정 ========================
% However, this architectural design still lacks the capability of identifying the video-descriptive keywords in the query.
% % However, this architectural design still lacks in identifying proper query relevance.
% This is because the current training scheme only focuses on the interactions of video and clips within a single video while neglecting information shared throughout the entire video.
% % We argue the problem of the current training scheme that only focuses on distinguishing the clips in a single video while neglecting information shared throughout the entire video.
% Therefore, we leverage the negative video-query relationships to enhance the capability of identifying the contextual similarity of query and video clips.
% 
% 원준 원본 
% However, this architectural design heavily relies on the quality of the text query.
% Therefore, we leverage the negative video-query relationships to enable the model to emphasize key corresponding query features.
% By suppressing the saliency scores of the irrelevant video-query pairs, the model learns to spotlight only the video-specific discriminative words in the query.
% =========================================================
Lastly, to apply the dynamic criterion to mark highlights for each instance, we deploy a saliency token to represent the entire video and utilize it as an input-adaptive saliency criterion. 
With all components combined, our QD-DETR produces query-dependent video representation by integrating source and query modalities.
This further allows the use of positional queries~\cite{dabdetr} in the transformer decoder.
% Furthermore, we can exploit the advanced DETR decoder architectures using the positional information, e.g., DAB-DETR, since our encoded tokens consist of identical position representations from a single modality.
% \SE{} % ====================== 상익 수정 ========================
% Furthermore, we can exploit the advanced DETR decoder architectures using the positional information, e.g., DAB-DETR, since our video clip tokens consist of identical position representations from a single modality.
% 원준 원본
% It also enables the use of advanced DETR decoder architectures, e.g., DAB-DETR, for the first time, as these works exploit the position information within a single modality.
% =========================================================
Overall, our superior performances over the existing approaches validate the significance of the role of text query for MR/HD.
% Our extensive experiments on QVHighlights, TVSum, and Charades-STA datasets validate the significance of considering the role and the quality of text query.

% All components combined with dynamic anchor moments for the query of decoder, our FOQUE fosters the query-dependent video representation, thereby making the 
% All components combined, our modified transformer encoding process fosters the query-dependent video representation thereby achieving the state-of-the-art results on various benchmarks of moment-retrieval and highlight detection.
	
% -	Video Platform & Streamer & Consumer의 증가. 
% Video는 다른 데이터 타입보다 정보가 많아 유용하지만, 이는 다른 말로 해석하면 video를 보는 것은 time-consuming 하고, 원하는 것을 찾아보기에는 힘들 수 있음.
% 따라서, 많은 매체에서는 사람들의 더 많은 이목을 끌기 위해 highlight 비디오라는 것을 편집하여 공유도 함.
% 하지만, highlight video를 만들기 위해 사람의 노력이 필요한 현 시점에서, This spotlights the need to retrieve the user-requested / Highlight moments in the video.

% -	이전에도 이러한 문제를 해결하기 위해 (asdfasdf) for moment retrieval, (asdfasdf) for highlight detection 등이 제안 되었지만, 이들은 비디오의 특정 영역을 찾는다는 공통된 목적을 가지고 있으면서도, 데이터 셋의 한계로 인해 따로 연구되었음. 이를 문제 삼으며, 최근에는 두 task를 동시에 학습할 수 있는 dataset이 소개 되었는데, 컴퓨터비전에서 최근 각광을 받고 있는 Transformer 모델 도입과 함께 큰 발전을 거듭하고 있음.

% -	구체적으로, 이 두가지 task를 수행하기 위해서는 transformer를 두가지 방법으로 이용할 수 있는데, moment-DETR 처럼 moment 를 clip의 set 단위로 예측할 수 있고, UMT 처럼 clip-wise prediction을 할 수 있음. 하지만, 이들은 query를 condition이 아닌 video와 동등한 레벨로 취급하거나 [mDETR], 매 클립이 self-attention으로 mixing 된 후에 condition을 걸어주어 clip간의 차이를 확실하지 이용하지 못하였고, 또한, 확실하게 condition으로 주지 못하였고, video와 query 사이의 관계를 한정적으로만 이용하였다.

% -	따라서, we explore three different ways to fully exploit query information. First, we design one-way cross-attention layer to condition every clip with the query features. Then, we utilized the negative video-text pairs to better model the relationships between the video and the text embeddings. Lastly, we define the saliency token to be the video-query dependent saliency estimator.


















% ===================== neg pair 부분 ===========================
% Nevertheless, the current training scheme, only considering the given video-query pair, still disturbs the model from identifying proper query-relevance prediction.
% In detail, the model focus on learning the fine-grained discrepancy between video clips, while neglecting the information they share, which contains significant clues to understand the context of video.
% Therefore, we leverage the negative video-query relationships to enhance the capability of identifying the contextual similarity of query and video clips.
% Therefore, we leverage the negative video-query relationships by suppressing those pairs, so that enhance the capability of identifying the contextual similarity of query and video clips.
% We hypothsize the diversity in query-video pairs are insufficient to learn the general relationship between text query and video.
% Therefore, we leverage the negative video-query relationships by suppressing the saliency scores of the irrelevant video-query pairs.
% However, this architectural design still lacks in identifying proper query relevance.
% We argue that the current training scheme only focuses on learning the fine-grained discrepancy between clips in a single video, while neglecting the information they share, which contains significant clues to understand the context of the video.
% Therefore, we leverage the negative video-query relationships to enhance the capability of identifying the contextual similarity of query and video clips.
% However, this architectural design still lacks in identifying proper query relevance.
% We argue the problem of the current training scheme that only focuses on learning the fine-grained discrepancy between clips in a single video.
% That is, the current design neglects the information shared throughout the video, although it contains significant clues to understand the context of the video.
%%%%%%%%%%%%%%%%%%%%%%%%%%%%%%%%%%%%%%%%%%%%%%%
%%%%%%%        2. Proposed Approach         %%%%%%%
%%%%%%%%%%%%%%%%%%%%%%%%%%%%%%%%%%%%%%%%%%%%%%%

\section{Proposed Approach}
\label{sec:methods}

A depiction of our approach is shown in Figure \ref{fig:model}. The model consists of a MOS prediction model (shown left) and a speech enhancement model (shown right). Our MOS prediction model is tailored to provide estimates for subjective-MOS (as rated by humans), and going forward, we will use MOS to refer to subjective-MOS unless explicitly stated otherwise, for ease of understanding. We next will provide notation and then describe each of these sub-modules.

\subsection{Notation}

We define a clean speech signal as $s_t$ and background noise as $n_t$ at time $t$. The mixture of clean speech and noise is denoted as $m_t=s_t+n_t$. We aim to extract the speech from the mixture by removing the unwanted noise. The short-time Fourier transform (STFT) converts the time-domain mixture into a T-F representation, $M_{t,f}$, that is defined at time $t$ and frequency $f$. The complex-valued STFT matrix, $\bm{M}$, can be written as $\bm{M}=|\bm{M}|e^{i\bm{\theta}^M}$ with magnitude $|\bm{M}|\in \bm{\Re}^{T\times F}_+$ and phase $\bm{\theta}^M \in \bm{\Re}^{T\times F}$, where $T$ is the length of speech in time and $F$ is the total number of frequency channels.

Enhancing the magnitude response of noisy speech results in an estimate of the clean speech magnitude response, $|\hat{\bm{S}}|$, using an enhancement function $\mathcal{F}_\delta$ such that $|\hat{\bm{S}}| =\mathcal{F}_\delta(|\bm{M}|)$. The enhancement function is modeled with a deep neural network which is trained to maximize the conditional log-likelihood of the training dataset, 
\begin{align*}
    &\max \frac{1}{N} \sum^N \log P\Big( |{\bm{S}}| \, \Big| \, |\bm{M}|\Big) \\
    \Rightarrow &\max_\delta \frac{1}{N} \sum^N \log P\Big( \mathcal{F}_\delta(|\bm{M}|) \, \Big| \, |\bm{M}|\Big)
\end{align*}
% $$\max \frac{1}{N} \sum^N \log P\Big( |{\bm{S}}| \, \Big| \, |\bm{M}|\Big) \Rightarrow \max_\delta \frac{1}{N} \sum^N \log P\Big( |\hat{\bm{S}}| \, \Big| \, |\bm{M}|\Big) $$
where $\delta$ denotes the set of tunable parameters and $N$ is the number of training examples. The estimated magnitude response $|\hat{\bm{S}}|$ is then combined with the noisy phase, $\bm{\theta}^M$, where the inverse STFT produces an enhanced speech signal in the time domain, $\hat{s}_t$. 

\subsection{Speech quality assessment model}
\label{subsec:mos_model}

A MOS prediction model proposed by \cite{dong2020pyramid} is adapted to estimate the MOS from noisy speech. This model has been developed with real-world captured data and it has been shown to outperform comparison approaches~\cite{fu2018quality, avila2019non, mittag2019non}, according to multiple metrics. The MOS prediction model consists of an attention-based encoder-decoder structure that uses stacked pyramid bi-directional long-short term memory (pBLSTM)~\cite{chan2016listen} networks in the encoder. We denote this model as Pyramid-MOS (PMOS). A pBLSTM architecture gives the advantages of processing sequences at multiple time resolutions, which effectively captures short- and long-term dependencies. Speech has spectral and temporal dependencies over short and long durations, and a multi-resolution framework is effective in learning these complex relations. 


A single T-F frame of the noisy-speech mixture, $|\bm{M}_t|$, is the input to the PMOS encoder. In a pyramid structure, the lower layer outputs from $\Upsilon$ consecutive time frames are concatenated and used as inputs to the next pBLSTM layer, along with the recurrent hidden states from the previous time step. The output of a pBLSTM node is an embedding vector, $h^l_t$, that is as defined below:
\begin{align}
    h^l_t &= pBLSTM\Big( h^l_{t-1}, \big[ h^{l-1}_{\Upsilon\times t -\Upsilon+1}, h^{l-1}_{\Upsilon\times t}\big] \Big)
\end{align}
where $\Upsilon$ is the reduction factor (e.g., number of concatenated frames) between successive pBLSTM layers and $l$ is the layer number. A pBLSTM reduces the time resolution from the input speech to the final latent representation $\bm{H}$. Figure~\ref{fig:pBLSTM} shows the internal structure of pBLSTM module.
This compressed vector accumulates the useful features for measuring speech perceptual quality that resides in a range of time-frames and ignores the least important features.
The encoder outputs a concatenated version of the hidden states of the last pBLSTM layer as vector $\bm{H}=\{\bm{h}_1, \dotsb, \bm{h}_\tau, \dotsb, \bm{h}_\wp\}$, where $\wp$ is the total number of final embedding vectors with time index $\tau$.

The output of the PMOS encoder becomes the input to the PMOS decoder unit. This decoder is implemented as an attention layer followed by a fully-connected (FC) layer and it outputs an estimated MOS of the input speech utterance. Attention models learn key attributes of a latent sequence, since adjacent time frames can provide important information, which is particularly crucial for our task.  
The attention mechanism~\cite{luong2015effective} uses the pyramid encoder output at the $i$-th and $k$-th time steps to compute the attention weights, $\alpha^{PMOS}_{i,k}$. Attention weights are used to compute a context vector, $c^{PMOS}_i$, using the following equations:
\begin{align}
    \alpha^{PMOS}_{i,k} &= \frac{\exp{(\bm{h}_i^\top \bm{Q} \bm{h}_k)}}{\sum^{\wp}_{\phi=1} \exp{(\bm{h}_i^\top \bm{Q} \bm{h}_\phi)}}\\
    % \alpha^{PMOS}_{i,k} &= Attention(\bm{h}_i, \bm{h}_k)\\
    c^{PMOS}_i &= \sum^\wp_{k=1} \alpha^{PMOS}_{i,k} \cdot \bm{h}_k
\end{align}
$\bm{Q}^{\wp\times\wp}$ is the trainable PMOS attention weight matrix. We learn $\bm{Q}$ using a feed-forward neural network that attempts to capture the alignment between the embeddings $\bm{h}_i$ and $\bm{h}_k$. 

The context vector is provided to a fully-connected layer to estimate the MOS. Note that the pyramid structure of the encoder results in a shorter sequence of latent representations than the original input sequence, and it leads to fewer encoding states for attention calculation at the decoding stage. Therefore, strictly  $\wp<T$, and in our case $\wp = \lceil T/\Upsilon^L \rceil$, where $L$ is the number of pBLSTM layers.
We train the PMOS model separately with the parameters defined in~\cite{nayem2019incorporating}. After training, this model is held frozen during inference.

\begin{figure}[t!]
    \centering
    \includegraphics[width = 0.95\linewidth]{figs/pBLSTM.png}
    \caption{Illustration of pBLSTM structure with reduction factor $\Upsilon=2$ and number of layer $L=2$.}
    % \vspace{-2em}
    \label{fig:pBLSTM}
    % \vspace{-0.4cm}
\end{figure}

\subsection{Proposed speech enhancement model}
\label{subsec:se_model}
Our proposed speech-enhancement (SE) model follows an encoder-decoder structure, and it is shown in Figure \ref{fig:model} (right). The SE encoder takes a single T-F frame of a noisy-speech mixture, $|\bm{M}_t|$, as input and multiple BLSTM layers, are stacked together to create a hidden representation of the frame, $\bm{g}_t$. In our SE encoder, we utilize BLSTM layers instead of pBLSTM layers since we aim to estimate an embedding frame for each time frame and pBLSTM layers reduce the number of output frames. 
An attention mechanism is applied using the mixture encoding from the SE model, $\bm{G}=\{\bm{g}_1, \bm{g}_2, \dotsb, \bm{g}_T\}$, and the PMOS encoding, $\bm{H}$, from the MOS prediction model. This allows the SE model to exploit the MOS estimator's encoding and utilize the important perceptual feature embedding that correlates with human assessment. Considering that the pBLSTM structure of the PMOS encoder condenses the final encoding vector $\bm{H}$ along time, PMOS yields a smaller time resolution than the encoding from the SE encoder, so we compute a score for each embedding vector $\bm{h}_{\tau}$ using an alignment  weight matrix, $\bm{W}^{T\times\wp}$. Then the attention weights for the SE model, $\alpha_{t,\tau}$, are obtained using a softmax operation over the scores of all $\bm{h}_\tau$. Now, the PMOS encoding is summarized in a context vector $\bm{c}_t$ for each mixture frame $\bm{g}_t$. Prior to computing $\bm{c}_t$, $\bm{h}_\tau$ passes through a linear layer $\ell$, so that we learn a different representation for the SE task. The computations are below:
\begin{align}
    \alpha_{t,\tau} &= \frac{\exp{(\bm{g}_t^\top \bm{W} \bm{h}_\tau})}{\sum^{\wp}_{\phi=1} \exp{(\bm{g}_t^\top \bm{W} \bm{h}_\phi)}} \\
    \bm{c}_t &= \sum_{\tau=1}^\wp \alpha_{t,\tau} \cdot \ell (\bm{h}_\tau)
\end{align}
\noindent
Then, the context vector and SE-model embedding vector are concatenated (e.g., $[\bm{c}_t, \bm{g}_t]$) and passed to the decoder module. The SE-decoder module follows the network structure from \cite{schulze2020joint}. It consists of a linear layer with a $tanh(\cdot)$ activation function, two BLSTM layers, and a linear layer with ReLU activation. It outputs the estimated enhanced speech $|\hat{\bm{S}}|$. This estimated speech magnitude with noisy phase produce the estimated clean speech, i.e. $\hat{\bm{S}} = |\hat{\bm{S}}|e^{i\bm{\theta}^M}$. Since we are estimating two targets MOS and enhanced speech simultaneously, the unified model will learn different representations for these tasks. Thus both PMOS and SE models will learn their corresponding targets with perceptual feature sharing. We freeze the PMOS model while training this SE model.


\subsection{Joint-learning of PMOS and SE model}
\label{subsec:joint_model}
We also develop an approach that allows the PMOS and SE models to be jointly trained. Our joint-learning objective function uses a weighted average of a {time-domain} signal-approximation loss $\mathcal{L}_{sa}$ (from the SE model), the MSE of the magnitude spectrum $\mathcal{L}_{mse}$ (from the SE model) and the MSE of the MOS estimation $\mathcal{L}_{mos}$ (from the PMOS model). We compute the signal-approximation loss from the time-domain signal difference between the reference speech $s$ and enhanced speech $\hat{s}$. The overall loss function of our network is defined as below, using hyper-parameters $\lambda_1$ and $\lambda_2$ that control the impact of individual loss terms:
\begin{align}
    \mathcal{L} &= \lambda_1\left[\lambda_2\mathcal{L}_{mse} + (1-\lambda_2)\mathcal{L}_{sa}\right] + (1-\lambda_1)\mathcal{L}_{mos}
    \label{eq:loss}
\end{align}
\noindent
The model training order is as such. First, we train the PMOS model using $\mathcal{L}_{mos}$ (e.g. $\lambda_1 = 0$). Then we train the SE model using $\lambda_1 = 1$, while running the PMOS model in inference mode (e.g. it is held fixed). This is done to ensure that the trained PMOS model effectively encodes the key features in the embedding vector that are important to perceptual speech quality. Finally, we train both the models jointly (e.g. $0 < \lambda_1 < 1$) using $\mathcal{L}$ to further reduce any correctional differences between the true and estimated MOS in the PMOS model, and to increase the perceptual quality of the enhanced speech.
\begin{figure}[t!]
    \centering
    \includegraphics[width = 0.8\linewidth]{figs/quant_fig2.png}
    \caption{Quantization of a clean magnitude spectrum.}
    % \vspace{-2em}
    \label{fig:quant}
    % \vspace{-0.4cm}
\end{figure}
\subsection{Quantized Spectral Model}
\label{subsec:QSM}
% An external language model can integrate additional information regarding speech correlation which is helpful for improving enhancement performance. Typical LM is applied at the phoneme or word level and the performance of the LM depends on the text and its vocabulary. Additionally, parallel corpus of speech and text is a requirement for training which rules out a huge number of corpus from usage. 
%A language model (LM) serves as prior knowledge on acoustic input that constrains the alternative word (or phonetic) hypothesis during speech recognition by learning which sequences of words (or phonetics) are most likely to be spoken. LM predicts which words will follow on from the current words and with what probability. $\mathbb{N}$-gram LM is a widely used approach which estimates the probability of a given sequence of words $w_{1\cdots\Omega}$ within the assumption that the probability of word $w_\delta$ depends only on previous $(\mathbb{N}-1)$ words, and the probability can be expressed as: 
%\begin{align}
%    P(w_{1\cdots\Omega})=\prod_{w_\delta} P(w_\delta|w_{\delta-1}, w_{\delta-2},\cdots,w_{\delta-\mathbb{N}+1})
%\end{align}
%Compared to conventional ASR approaches, deep ASR systems model learn in-house LM \cite{yu2016automatic}; and they can be coupled with SE task~\cite{weninger2015speech, wang2020complex}. LM helps a SE model by predicting probability of next utterance, which is otherwise will be any utterance in the whole speech spectrum. However, deep LM typically require more data to achieve comparable results. Additionally, parallel corpus of speech and text is a requirement for training which rules out a huge number of raw audio collections from usage.
%Therefore, we adapt an alternative view of a LM from \cite{nayem2021towards}, where quantized t-f values are considered as word. 
From written and spoken language, we can determine the sequences of words that are most likely to occur. This knowledge is captured by a language model (LM) of an automatic speech recognition system which we can expressed as,
\begin{align}
    \hat{words}=\argmax_{words\in Language} \overbrace{P(input|words)}^{acoustic\;model} \overbrace{P(words)}^{language\;model}
\end{align}
%Here, the most likely word sequence, $\hat{words}$, is estimated by an acoustic model that calculates the probability of the input audio given the word sequence $words$, and by a language model that gives the likelihood of the word sequence. Hence, the LM predicts the probability of a sequence of words. 
The LM is useful in eliminating rare and grammatically incorrect word sequences, and it enhances the performance of ASR systems. In the case of speech enhancement, models learn spectral information within frames over time, but they often neglect the temporal correlations. Our approach, as proposed in \cite{nayem2021towards}, suggests incorporating a ``LM" to fuse temporal correlations and overcome this limitation. Therefore, we construct a bi-gram Quantized Spectral Model (QSM), which functions in a similar way to a language model (LM), in order to produce more realistic spectra. The QSM estimates the probability of spectral magnitudes along time for each frequency channel conditioned on its previous T-F spectral magnitude. %Range of T-F unit values is constrained in a signal approximating SE system and is far smaller than typical spoken language vocabulary size. As a result, the training time and computational resource requirement are quite small fo spectral LM.
On a reference speech corpora, we apply a normalization scaling function, $\mathcal{N}_{[o,r]}(\cdot)$, that normalizes the magnitude spectrogram and re-scales the range to $[0,r]$. Then a quantization function, $\mathcal{Q}_\chi(\cdot)$, converts the range constrained magnitude spectrogram into $\mathcal{D}$ number of bins that are $\chi$ steps apart. This produces quantized speech, i.e. $|S|^q = \mathcal{Q}_\chi\big(\mathcal{N}_{[0,r]}(|S|)\big)$. Fig.~\ref{fig:quant} shows an example of the original clean and quantized clean magnitude spectra, where $\chi=2$ for display purposes. Our proposed QSM has $\mathcal{D}$ spectral levels. We construct the QSM using quantized speech magnitudes from the clean speech corpora. The QSM is less likely to suffer from the out of vocabulary problem when the model parameters, $\chi$ and $r$, are adequately defined.

%\begin{figure}[tbh!]
%    \centering
%    \includegraphics[width = \linewidth]{IEEEtran/figs/fQSM.png}
%    \caption{Proposed Quantized Spectral Models (QSMs) for per-frequency-channel.}
%    % \vspace{-2em}
%    \label{fig:fQSM}
%    % \vspace{-0.4cm}
%\end{figure}

We compute per-frequency-channel QSMs along the time axis where each entry, $d$, refers to a quantization attenuation level. We then compute the transition probability between two time consecutive T-F units, $fQSM_f = P(d_{t+1,f}|d_{t,f})$. The probabilities are calculated by counting the level transitions, and then normalizing by the appropriate scalar. These probabilities are stored in the per-frequency-channel QSM resulting in a $F\times \mathcal{D}\times \mathcal{D}$ probability matrix. %Figure~\ref{fig:fQSM} shows proposed QSMs along per-frequency-channel. 
We re-evaluate the transition probabilities using Good-Turing smoothing~\cite{jurafskyMartin2009} to overcome the zero-probability problem in N-grams. Shallow fusion~\cite{gulcehre2015using} is a simple method to incorporate an external LM into an encoder-decoder model, and it produces better results compared to others. Hence, we use shallow fushion to combine our QSM and SE model based on log-linear interpolations at inference time. This is shown in the below equations:

\begin{align}
    P^{QSM}_f(|\hat{\bm{S}}_{:,f}|) &= \prod^T_{i=1} P(d_{i,f}|d_{i-1,f}) \\
    |\hat{\bm{S}}_{:,f}|^* = \argmax_{|\hat{\bm{S}}_{:,f}|} &\log P\big(|\hat{\bm{S}}_{:,f}| \big| |\bm{M}|\big) + \mu \log P^{QSM}_f\big(|\hat{\bm{S}}_{:,f}| \big)
    \label{eq:S_hat}
\end{align}
\noindent
Here $P^{QSM}_f$ denotes the transitional probability of QSM at frequency $f$, $P\big(|\hat{\bm{S}}_{:,f}| \big| |\bm{M}|\big)$ represents the estimated magnitude output of the LSTM layers of the SE decoder, and $\mu$ is a hyper-parameter that is tuned to maximize the performance on a development set. Note that we train our QSM in advance on a clean speech corpus and use it in inference mode during enhancement. The tunable parameter $\mu$ of (\ref{eq:S_hat}) is set to zero when we do not have a trained QSM. 





\section{Experiments}



\begin{table*}[t]
\small
\centering
\caption{\textbf{Evaluation of panel-prediction quality} on seen and unseen garment classes. M-L2: Mask L2 ; P-L2 : Panel L2; R-L2: Rotation L2; T-L2: Translation L2 . $\dagger$ represents orderless-LSTM.} %$*$ denotes the models without data-filtering.}
\setlength{\tabcolsep}{8pt}
\begin{tabular}{@{}c|ccccc|ccccc@{}}
\toprule
                                   & \multicolumn{5}{c|}{\textbf{Seen classes}}                                                                                                                                                                                                      & \multicolumn{5}{c}{\textbf{Unseen classes}}                                                                                                                                                                                                     \\ \cmidrule(l){2-11} 
\multirow{-2}{*}{\textbf{Methods}}& \textbf{P-L2}                     & \textbf{\# Panels}                    & \textbf{\# Edges}                     & \textbf{R-L2}                       & \textbf{T-L2}           & \textbf{P-L2}                     & \textbf{\# Panels}                    & \textbf{\# Edges}                     & \textbf{R-L2}                       & \textbf{T-L2}                     \\ \midrule
Baseline-I                 & 3.92                                    &  \bf 99.9\%                                    &\bf 100.0 \%                                    & 0.06                             & 0.117                                       & 6.61                                    & 94.6\%                                     &  95.4\%                                     & 0.09                                    & 0.21                                    \\
Baseline-II                                                   & 4.3                                   & 99.4\%                                   &   99.7\%                                        & 0.08                                   & 1.46                                                             & 8.1                                   & 89.3\%                                   & 90.3\%                                   & 121                                   & 1.25                                   \\
%Baseline-III                                                 & 3.91                                   & 99.9 \%                                  & 99.9 \%                                    & 0.06                                   & 0.05                                                            & 6.3                                   & 93.9  \%                                    & 94.2 \%                                    & 0.07                                   & 0.18                                   \\
LSTM                                                       & 2.71                                  & 99.8\%                                & 99.9\%                                & \bf 0.004                                 & 0.32                                                                 & 14.7                                  & 6.5\%                                 & 53.2\%                                & 0.17                                  & 6.75                                  \\
LSTM$^{\dagger}$                                                    & 2.87                                  & 99.4\%                                & 99.9\%                                & \textbf{0.004}                                 & 0.33                                                                   & 12.94                                 & 2.7\%                                 & 59.0\%                                & 0.16                                  & 7.18                                  \\
Neural-Tailor                                                      & \textbf{1.5}                                   & 99.7\%                                & 99.7\%                                & 0.04                                  & 1.46                                                         & 5.2                                   & 83.6\%                                & 87.3\%                                & 0.07                                  & 3.22                                  \\
% Neural-Tailor*                                                    & 1.53                                  & 98.8\%                                & 99.6\%                                & 0.04                                  & 1.45                                                        & 7.96                                  & 73.1\%                                & 80.5\%                                & 0.08                                  & 3.57                                  \\
% Neural-Tailor*                                                    & 1.6/1.95                              & 98.6/97.5\%                           & 99.8/99.2\%                           & 0.07/0.07                             & 2.2/2.5                                                             & 6.2/6.4                               & 81.6/75.2\%                           & 88.5/88.2\%                           & 0.08/0.10                             & 3.9/4.5                               \\ \midrule
\midrule
\textbf{Ours w/ Text}                      & \cellcolor[HTML]{FFFFDB}2.80  & \cellcolor[HTML]{FFFFDB}\textbf{99.9\%}  & \cellcolor[HTML]{FFFFDB}{99.9\%}  & \cellcolor[HTML]{FFFFDB}0.04  & \cellcolor[HTML]{FFFFDB}\textbf{0.04}    & \cellcolor[HTML]{FFFFDB}\textbf{4.20}  & \cellcolor[HTML]{FFFFDB}\textbf{99.9\%}  & \cellcolor[HTML]{FFFFDB}\textbf{99.8\%}  & \cellcolor[HTML]{FFFFDB}\textbf{0.05}  & \cellcolor[HTML]{FFFFDB}\textbf{0.05}  \\
\textbf{Ours w/ Sketch}                      & \cellcolor[HTML]{FFFFDB}2.91  & \cellcolor[HTML]{FFFFDB}\textbf{99.9\%}  & \cellcolor[HTML]{FFFFDB}{99.9\%}  & \cellcolor[HTML]{FFFFDB}0.05  & \cellcolor[HTML]{FFFFDB}{0.06}    & \cellcolor[HTML]{FFFFDB}\textbf{4.33}  & \cellcolor[HTML]{FFFFDB}\textbf{99.9\%}  & \cellcolor[HTML]{FFFFDB}\textbf{99.9\%}  & \cellcolor[HTML]{FFFFDB}\textbf{0.06}  & \cellcolor[HTML]{FFFFDB}\textbf{0.07}  \\
%\textbf{Ours w/ Overlap}                      & \cellcolor[HTML]{FFFFDB}2.80  & \cellcolor[HTML]{FFFFDB}\textbf{99.9\%}  & \cellcolor[HTML]{FFFFDB}\textbf{99.9\%}  & \cellcolor[HTML]{FFFFDB}0.04  & \cellcolor[HTML]{FFFFDB}\textbf{0.04}    & \cellcolor[HTML]{FFFFDB}\textbf{4.20}  & \cellcolor[HTML]{FFFFDB}\textbf{99.9\%}  & \cellcolor[HTML]{FFFFDB}\textbf{99.8\%}  & \cellcolor[HTML]{FFFFDB}\textbf{0.05}  & \cellcolor[HTML]{FFFFDB}\textbf{0.05}  \\

 \bottomrule 
\end{tabular}

\label{tab:main_tab}
\end{table*}

\begin{figure*}[t]
    \centering
    \includegraphics [width=\linewidth]{img/fig4_v2.pdf}
    \caption{
    Comparing our method with NeuralTailor (\texttt{NT})
    on the unseen garment classes: ‘jacket sleeveless’, ‘skirt waistband’, ‘wb jumpsuit sleeveless’ and ‘dress’.
    {\em Metric}: the average Vertex L2 error.
    {\em Ground-truth}: dash thin lines.
    % Comparison of our method with NeuralTailor (\texttt{NT}) \cite{korosteleva2022neuraltailor} on the unseen garments from ‘jacket sleeveless’, ‘skirt waistband’, ‘wb jumpsuit sleeveless’ and ‘dress’ categories of the dataset \cite{korosteleva2021generating}. The numbers show the average Vertex L2 for the shown exemplars. The colored panels indicate predicted panels, and the dash thin lines indicate the ground-truth panels.
    }
    \label{fig:main_viz}
    % % \vspace{-0.2in}
\end{figure*}
\noindent \textbf{Dataset}
We evaluate the PersonalTailor on the 3D garments dataset with sewing patterns from \cite{korosteleva2021generating}. It contains 19 garment classes with $22,000$ 3D garment-sewing pattern pairs in total, covering the variations in t-shirts, jackets, pants, skirts, jumpsuits and dresses. 
There are 10627/722/729 samples for train/val/test
% The number of samples in train/val/test is 10627/722/729 
in the filtered version. 
Following NeuralTailor \cite{korosteleva2022neuraltailor}, the classes of panels are designed based on the panel's role and location around the body across all garment classes. For example, panels located around the back of human body are grouped in the ``back panels'' class. We follow the standard panel labels, data filtering and train/test splits of garment classes. There are 7 garment classes unseen to training and used for evaluation. 



\noindent \textbf{Evaluation metrics}
We use the same evaluation metrics as in \cite{korosteleva2022neuraltailor}. 
We evaluate the accuracy in predicting the number of panels within
every pattern (\ie, \#Panels) and the number of edges within every panel
(\ie, \#Edges). To estimate the quality of panel shape predictions, we use the average distance (L2 norm) between the vertices of predicted and ground
truth panels with curvature coordinates converted to panel space,
acting as panel masks in this comparison (Panel L2). Similarly,
we report L2 normalized differences of predicted panel rotations
(Rot L2) and translations (Transl L2) with the ground truth. The
quality of predicted stitching information is measured by a mean
precision (Precision) and recall (Recall) of predicted stitches.




\noindent \textbf{Implementation details}
For language encoding, we use CLIP \cite{radford2021learning} pretrained encoder. 
For sketch encoding, we use SketchRNN \cite{yang2021sketchgnn}. 
We follow the training scheme as \cite{korosteleva2022neuraltailor}.
We set the maximum number of panels $M=23$.
There are $g=12/8$ garment classes in training/testing set. We set the feature dimension for text and the global embedding $D = 512$. % is set as 512. 
% The number of codes $K$ in codebook is set as 2000. 
% For Stage-1 training, 
Our model is trained for 250 epochs using Adam optimizer with learning rate of $10e-5$ and batch size of 15. 
% For Stage-2 training, our model
The stitching GNN is trained for 50 epochs using SGD optimizer with learning rate of $10e-4$.
%
% Specifically, %We train our stitch prediction network
% it is trained by the edge vectors from ground truth panels and edges outputted by the prediction module under the text prompt scenario. 
%
Specifically, %We train our stitch prediction network
it is trained by the predicted edges.
%
The inference threshold for panel mask head is set as 0.5 and top-$k$ is set as 14. The code will be made publicly available upon acceptance.
% Our model is implemented in Pytorch and trained with batch size of 15 on a single NVIDIA 2080GTX GPU.

% \begin{figure}[t]
%     \centering
%     \includegraphics[scale=0.35]{img/final_model.png}
%     \caption{\textbf{Examples of garment personalization} 
%     % Based on user input via sketch/text prompt, we illustrate the customization 
%     %
%     % Garment personalization 
%     from (a) Pant Straight sides to Skirt 4 Panels, (b) Skirt 4 Panels to Pant Straight Sides, (c) Dress Sleeveless to Dress Waistband Sleeveless, (d) Dress Waistband Sleeveless to Dress Sleeveless respectively. }
%     \label{fig:customized}
% \end{figure}


\subsection{Personalized pattern design evaluation}
\noindent \textbf{Setting}  To quantitatively evaluate the performance of personalization,
% based on the user input prompts (\ie, text and sketch), 
we conduct 6 garment class transfer cases (case 1\&2: Tee $\leftrightarrow$ Jacket, case 3\&4: Jumpsuit$\leftrightarrow$ Dress, case 5\&6: Jacket $\leftrightarrow$ Jacket Sleeveless)
under both text and sketch prompt. We define the \textit{Panel IOU}  metric as the mean of panel-wise IOUs between predicted panels of the source garment class and the average panels of the target garment class. Formally, we use the desired input prompts to transfer the source garment class to the target garment class. Then we compare the \textit{Panel IOU} before and after personalization against the target class panel attributes. % in personalized query. 

\noindent \textbf{Baseline} Due to lacking of competing works or open-source alternatives, % in the literature, 
% we created our own baselines. More specifically, 
we created a personalization baseline by removing the prompt embedding and cross-modal embedding module (referred as \texttt{baseline}) from our PersonalTailor. 

\noindent \textbf{Quantitative results} 
The personalization results are reported in Tab.~\ref{tab:personalization}. It can be observed that (1) our method can achieve an average panel IOU of $53\%$ over 6 cases by text and $52\%$ by sketch, outperforming the baseline method by $13\%$/$16\%$ respectively. This is because the decoder of the baseline is randomly initialized lacking the semantic and structural information of the panel attributes. Thus, it has less personalization ability. (2) Our method yields a larger gain over the baseline before and after personalization under both text ($22\%$ \vs $16\%$) and sketch prompts  $21\%$ \vs $17\%$). 
This verifies our superior personalization ability.
% of our model design.
% This showcase that our method has better personalization ability.



\noindent \textbf{Qualitative/visual results}
We show the personalized garment transfer process of case 1\&2 by text prompt (target garment’s panel classes) in Fig.~\ref{fig:editing} (a,b), case 3\&4 by sketch prompt (target garment’s average panel silhouettes) in Fig.~\ref{fig:editing} (c,d).  Overall, it is shown that our method can support panel shape editing with complex topology changes from one garment class to another using personalized prompts, even for those unseen during training, \eg, Jumpsuit and Dress. Beyond topology change, it also supports adding 
new panels (Fig.~\ref{fig:teaser} (b)), removing panels (Fig.~\ref{fig:teaser} (c)), and creating a
new design % that is not included in the dataset 
(Fig.~\ref{fig:teaser} (e)).
We also observe that our method can achieve fine-grained panel shape editing by using sketch prompts. As shown in Fig.~\ref{fig:sketch_edit}, given a 3D jacket and different users' sketch prompts, our method can produce the panels aligned with the sketch's shape while preserving the intrinsic structure of the 3D shape. 
% We provide more illustration of personalized garment transfer in Fig.~\ref{fig:customized}. 
%\paragraph{Controllable garment personalization} \annie{figure 4 and 1} We demonstrate our framework is capable of controllable garment personalization. Given a source 3D garment, our PersonalTailor can accurately edit the 2D sewing panel shapes. Additionally, our model can also perform personalization by design choice during inference. From the results in Fig~\ref{fig:editing}, we can observe that PersonalTailor supports controllable editing on the 3D garment shape and topology with
%preserved intrinsic structure. And PerosnalTailor can edit garments with significant shape variations or transfer garments from one category to another by editing on the 2D panels via mask instructions, even for some categories that are not shown in the training set. As an added benefit, our network facilitates both text and sketch based editing to give more expressive power to the users.








% \begin{figure*}
%     \centering
%     \includegraphics[scale=0.43]{img/PTailor_main.png}
%     \caption{\textbf{Examples of PersonalTailor's output (unrefined)}
%     It is shown that our method works similarly well with text and sketch/visual prompts.
%     % As seen from the figure, both text and sketch prompt predicts similar mask prediction with textual prompt marginally better. 
%     }
%     \label{fig:ptailor_main}
% \end{figure*}

\begin{figure}[t]

    \centering
    \includegraphics[scale=0.32]{img/new_fig_6.png}
    % % \vspace{-0.1in}
    \caption{\ann{Illustration of fine-grained panel editing by sketch. Given a 3D garment and different users' sketches, our method can support fine-grained panel shape editing while preserving the intrinsic structure of the 3D garment.}}
  \label{fig:sketch_edit}
  % % \vspace{-0.2in}
\end{figure}
\subsection{Standard pattern design evaluation}

\noindent \textbf{Setting} In this setting, we evaluate the standard (non-personalized) pattern design.  
% We test the open-set scenario where the training and testing garment classes do not overlap, \ie $g_{train} \cap g_{test} = \phi$, whilst the panel classes may overlap $p_{train} \cap p_{test} \neq \phi$. 
We follow the same setting and dataset splits as proposed in NeuralTailor \cite{korosteleva2022neuraltailor}. More specifically, we evaluate on two settings:
\ann{(1) Training with seen classes and evaluating on unseen data of those seen classes, \ie closed-set setting; 
(2) Training with seen classes and evaluating on unseen classes, \ie open-set setting.} 

\noindent \textbf{Competitors} We considered the following competitors for comparison: 
(a) a competitive garment pattern prediction method Neural Tailor on filtered data \cite{korosteleva2021generating}, 
(c) an LSTM \cite{graves2012long} based garment pattern prediction ,
(d) an orderless LSTM \cite{yazici2020orderless} based garment pattern prediction, 
(e) \textit{Baseline-I} we created using GCN encoder and CNN decoder, 
(f) \textit{Baseline-II} we created using PointTransformer encoder \cite{zhao2021point} and Transformer decoder \cite{vaswani2017attention} with random initialized queries. %(g) a baseline termed as \textit{Baseline-III} created using GCN encoder and Transformer decoder without language. We evaluate all of the above techniques in a similar setup. 

\noindent \textbf{Results} The results are reported in Tab.~\ref{tab:main_tab}. 
\textbf{(I) Closed-set settings:} 
% Under this setting, t
The performance of some metrics (\eg, Edges/Panels) has almost saturated.
% by different methods.
In particular, NeuralTailor has the best Panel L2 result, indicating that learning vertex is better in the closed set than mask prediction.
However, our PersonalTailor achieves the best translation prediction, suggesting the importance of global information.
%
% In this settings, our PersonalTailor has near-to 100\% edge and panel accuracy indicating the superior model design. It is interesting to note that PersonalTailor has almost 8x better translation metric indicating the importance of global information. NeuralTailor has the best Panel L2 indicating that learning vertex has better generalization in the closed set than mask prediction.
\textbf{(II) Open-set settings:} %In contrast to the closed-set setting, 
Our method achieves the state-of-the-art in all the metrics, surpassing the competitors by a large margin. This indicates the superiority of \ann{personalized prompts} in open-set generalization.
% of class agnostic masks. %Additionally, we have tested our method on the non-overlapping setting which can be found in the supplementary file.\annie{check}.



% % \vspace{-0.1in}
\paragraph{Qualitative results} 
% We first present qualitative results for our PersonalTailor in Fig.~\ref{fig:main_viz},\ref{fig:ptailor_main} whilst comparing to our closest competitor NeuralTailor \cite{korosteleva2022neuraltailor} with unseen garments. 
We present qualitative results on unseen garments. 
It can be observed from Fig.~\ref{fig:main_viz} that our PersonalTailor predicts more accurate panels over the prior art NeuralTailor \cite{korosteleva2022neuraltailor}, due to our multi-modal embedding-based design enabling the prompt bring in additional semantic information about the garment's shape.  
We also show in Fig. \ref{fig:ptailor_main} that PersonalTailor works well with both text and sketch prompts.
% Our findings are also reflected in the Panel L2 metric for each garment where our model is superior in open-set scenario. %Additionaly, our model can also handle overlapping garment problem as seen in Fig~\ref{fig:overlap} where our model has significant better mask prediction than NeuralTailor.

\begin{figure}[t]
    \centering
    \includegraphics[scale=0.35]{img/final_model.png}
    \caption{\textbf{Examples of garment personalization} 
    % Based on user input via sketch/text prompt, we illustrate the customization 
    %
    % Garment personalization 
    from (a) Pant Straight sides to Skirt 4 Panels, (b) Skirt 4 Panels to Pant Straight Sides, (c) Dress Sleeveless to Dress Waistband Sleeveless, (d) Dress Waistband Sleeveless to Dress Sleeveless respectively. }
    \label{fig:customized}
\end{figure}





\subsection{Stitching prediction} 

We evaluate the stitching module design
by comparing with NeuralTailor \cite{korosteleva2022neuraltailor}.
As shown in Tab.~\ref{tab:stitch}, 
our GNN based design is clearly superior particularly
for unseen garment classes.
This validates the efficacy of our exploiting the structural information of panels.
%
We also show that using the edge vectors of ground-truth panels
for training is inferior than using the predicted for both methods,
as the former introduces some inconsistency with model inference.


% our stitch prediction network trained on the edges predictions outperforms the second best $3.3/0.5$ and $9.0/0.3$ points on Precision and Recall for Seen Type and Unseen Type respectively.} And the models  trained by edge prediction data (OurPred, NeuralTailorPred) perform better than that of ground truth edge vectors  (OurGT, NeuralTailorGT). The noised prediction data empowers the generalization ability of the networks.


\begin{table}[h]
\caption{\textbf{Evaluation of stitching prediction} on both seen and unseen garment classes.
$^*$: Trained by the edges of GT panels.
}
\label{tab:stitch}
\resizebox{\columnwidth}{!}{
\begin{tabular}{c|cc|cc}
\hline
\multirow{2}{*}{\textbf{Method}}                 & \multicolumn{2}{c|}{\textbf{Seen classes}}     & \multicolumn{2}{c}{\textbf{Unseen classes}}   \\ \cline{2-5}
                & \textbf{Precision}       & \textbf{Recall}          & \textbf{Precision}       & \textbf{Recall}          \\ \hline
NeuralTailor$^*$ \cite{korosteleva2022neuraltailor}  & 96.6\%          & 88.6\%          & 75.3\%          & 60.6\%          \\ \hline
NeuralTailor \cite{korosteleva2022neuraltailor} & 96.3\%          & 99.4\%          & 74.7\%          & 83.9\%          \\ \hline
Ours$^*$            & 74.9\%          & 65.0\%          & 76.8\%          & 73.0\%          \\ \hline
Ours          & \cellcolor[HTML]{F6DDCC}\textbf{99.9\%} & \cellcolor[HTML]{F6DDCC}\textbf{99.9\%} & \cellcolor[HTML]{F6DDCC}\textbf{85.8\%} & \cellcolor[HTML]{F6DDCC}\textbf{84.2\%} \\ \hline
\end{tabular}
}
\label{stitchingres}
\end{table}






\begin{figure*}
    \centering
    \includegraphics[scale=0.43]{img/PTailor_main.png}
    \caption{\textbf{Examples of PersonalTailor's output (unrefined)}
    It is shown that our method works similarly well with text and sketch/visual prompts.
    % As seen from the figure, both text and sketch prompt predicts similar mask prediction with textual prompt marginally better. 
    }
    \label{fig:ptailor_main}
\end{figure*}
\subsection{Ablation studies}
We conduct ablation studies to provide insights into each
component with text prompt. \saura{More in-depth ablations are provided in the \texttt{Supplementary}.}

\noindent \textbf{Impact of cross-modal alignment}
We evaluate the importance of cross-modal alignment.
To that end, we consider two down-stripped designs:
{\bf(1)} Removing the cross modal local association block (CMLA);
{\bf(2)} Removing the multi-modal transformer (MMT).
As shown in Tab.~\ref{tab:mmemb},
we find that without CMLA, a significant drop in the Panel L2 metric 
occurs, suggesting the importance of resolving the domain gap between the point cloud and semantic information from the prompt.
It is also shown that MMT is useful in terms of 
fusing information.

\begin{table}[h]
\centering
\small
\caption{Impact of multi-modal embedding.
CMLA: Cross modal local association;
MMT: Multi-modal transformer.
}
\label{tab:mmemb}
\setlength{\tabcolsep}{3pt}
\begin{tabular}{c|cc|cc}
\hline
\multirow{2}{*}{\textbf{Model}} & \multicolumn{2}{c|}{\textbf{Seen}}     & \multicolumn{2}{c}{\textbf{Unseen}}    \\ \cline{2-5} 
                                & \textbf{Panel L2} & \textbf{\# panels} & \textbf{Panel L2} & \textbf{\# panels} \\ \hline
 Ours                            & \cellcolor[HTML]{F6DDCC}\textbf{2.80}                & \cellcolor[HTML]{F6DDCC} \textbf{99.9\%}                 & \cellcolor[HTML]{F6DDCC}\textbf{4.20}               & \cellcolor[HTML]{F6DDCC}\textbf{99.9\%}  \\ \hline
w/o CMLA                         & 5.51               & 99.8\%                & 6.5                & 99.2\%                   \\
w/o MMT                         & 4.42                &  99.9\%                  & 5.32     
& 99.6\%                  \\ 
\hline


% \hline
 % \\ 
\hline
% w/o both                         & 41                & 42                 & 43   
% & 44 \\

\end{tabular}
% % \vspace{-0.2in}
\end{table}
% as the raw prompt features lack the power of expressively to predict the panel mask.

% in the pattern prediction performance in Tab.~\ref{tab:mmemb}. We first removed the cross modal local association block (CMLA) from our model and observed a significant drop of 2.7\%/2.3\% in the Panel L2 metric for Seen and Unseen setting respectively. This signifies the importance of resolving the domain gap between the point cloud and semantic information from the prompt. Once we remove the multi-modal transformer (MMT) we observe a drop of 1.6\%/1.1\% in the Panel L2 metric for Seen and Unseen setting respectively. This indicates that the raw prompt features lack the power of expressivity to predict the panel mask.



\noindent \textbf{Impact of point-cloud encoders} We evaluate our point-cloud encoder design including 
global encoder (GE) and local encoder (LE).
As shown in Tab.~\ref{tab:ptc}, we see that 
{\bf(1)} GE is useful particularly for unseen garment classes.
This is because global information plays an important role in estimating the cutting pattern and guiding the panel-mask decoder prediction.
% during cross-attention.
{\bf (2)} LE brings in further performance gain
due to extra part-level information introduced.
%
% ({\bf \color{red} W/O LE giving better Panel L2 on Unseen? Double check!})
% experimentally prove the necessity of our point cloud encoders in Tab.~\ref{tab:ptc}. Our variant without global encoder (GE) performs the worst by a margin of 0.6\%/1.3\% in Panel L2 metric for Seen and Unseen setting respectively. This is expected as global information plays an important role in estimating the cutting pattern and also guides the panel-mask decoder prediction using cross-attention. It additionally also affects the panel mask accuracy shown by a performance drop of 4.7\% in \# panels metric for Unseen setting  without global encoder (GE). The variant without local encoder (LE) however observes a slight drop of 0.5\%/0.7\% indicating fine-grained local structure is also important for the overall performance improvement. 






\begin{table}[t]
\centering
\small
\setlength{\tabcolsep}{3pt}
\caption{Impact of global and local point cloud encoders.
LE: Local Encoder; GE: Global Encoder.}
\label{tab:ptc}
\begin{tabular}{cc|cc|cc}
\hline
\multicolumn{2}{c|}{\textbf{Model} }      & \multicolumn{2}{c|}{\textbf{Seen}} & \multicolumn{2}{c}{\textbf{Unseen}} \\
\hline
 \textbf{LE}     & \textbf{GE}                       & \textbf{Panel L2}    & \textbf{\# Panels}      & \textbf{Panel L2}     & \textbf{\# Panels}     \\ \hline
\xmark          & \cmark                         & 3.20          & 99.4\%            & 4.90               & 95.2\%            \\
\cmark         & \xmark         & 3.30          & 99.9\%            & 5.51               & 96.3\%      \\
\hline
 \cmark        & \cmark    & \cellcolor[HTML]{F6DDCC}\textbf{2.80} & \cellcolor[HTML]{F6DDCC}\textbf{99.9\%} & \cellcolor[HTML]{F6DDCC}\textbf{4.20}  & \cellcolor[HTML]{F6DDCC}\textbf{99.9\%} \\ \hline
\end{tabular}
\end{table}



\noindent \textbf{Design choice of panel decoder}
We evaluate more choices of panel mask decoder
including (1) a CNN and (2) a Transformer decoder without positional embedding (Trans. w/o PE). 
We observe in Tab.~\ref{tab:dec} that (1) CNN is least performing for instruction based mask prediction as it loses out on the panel prediction performance due to lack of interaction among the panels. 
(2) Positional encoding is important as it predicts position specific masks.

\begin{table}[h]
\centering
\setlength{\tabcolsep}{3pt}
\caption{Ablation on the design choice of decoder.
Trans.: Transformer; PE: Positional Encoding.
}
\label{tab:dec}
\begin{tabular}{c|cc|cc}
\hline
\multirow{2}{*}{\textbf{Model}}       & \multicolumn{2}{c|}{\textbf{Seen}} & \multicolumn{2}{c}{\textbf{Unseen}} \\ \cline{2-5} 
                             & \textbf{Panel L2}    & \textbf{\# Panels}      & \textbf{Panel L2}     & \textbf{\# Panels}     \\ \hline
CNN                          & 5.49          & 93.2\%          & 6.92           & 91.7\%          \\
Trans. w/o PE        & 3.30          & 99.9\%            & 5.61           & 96.1\%    \\
\hline
Ours & \cellcolor[HTML]{F6DDCC}\textbf{2.80} & \cellcolor[HTML]{F6DDCC}\textbf{99.9\%} & \cellcolor[HTML]{F6DDCC}\textbf{4.20}  & \cellcolor[HTML]{F6DDCC}\textbf{99.9\%} \\ \hline
\end{tabular}
\end{table}

% To evaluate the expressive power of our panel mask decoder, we compare it against a CNN based baseline (CNN) and a vanilla transformer decoder without positional embedding (Trans. w/o PE). From the results in Tab.~\ref{tab:dec}, it is clear that the CNN based baseline  is not suitable for instruction based mask prediction as it loses out on the panel prediction performance due to lack of self-attention among the panels. This is improved by 2.7\%/2.7\% higher Panel L2 metric with the transformer decoder under Seen and Unseen settings respectively.  Besides, positional encoding is important as it predicts position specific masks which is reflected in the drop of 1.4\%/0.5\% in overall performance without positional encoding under Seen and Unseen settings.

\subsection{In-the-wild garment pattern design}
For more extensive evaluation, we qualitatively test on the garment captures from DeepFashion3D dataset \cite{zhu2020deep}.
We observe in Fig.~\ref{fig:wild} that our model makes better panel predictions than NeuralTailor \cite{korosteleva2022neuraltailor}. For example, NeuralTailor fails to perceive the sleeves
with the T-shirt (unseen to model training), whilst our model succeeds.
In the case of jeans, our model gives better panel uniformity than \cite{korosteleva2022neuraltailor}.
% We observe in Fig.~\ref{fig:wild} that our model makes more correct panel predictions and good guesses about the panel structure than NeuralTailor \cite{korosteleva2022neuraltailor}. For example, for garments not seen in the dataset, like the t-shirt which has sleeves, was not predicted by \cite{korosteleva2022neuraltailor} but succesfully predicted by our approach. In the case of jeans, our model has better panel uniformity than \cite{korosteleva2022neuraltailor} highlighting the effectivity of our model design. The quality of prediction on the in-the-wild garment scans can be improved further and thus bridging this sim-to-real gap is part of our future work.
\begin{figure}[h]
    \centering
    \includegraphics[scale=0.28]{img/3rd_party.png}
    \caption{\textbf{In-the-wild garment evaluation} Sewing patterns predicted by NeuralTailor \cite{korosteleva2022neuraltailor} and our PersonalTailor on examples from DeepFashion3D \cite{zhu2020deep}.}
    \label{fig:wild}
\end{figure}



\section{Human Evaluation}

In additional to benchmark based assessment,
we further provide human evaluation with a thoughtful user study.
In particular, we approached 20 professional tailors to request their 
preference on the significance of personalizing garment pattern design.
In particular, we asked them two questions: 
(1) If garment personalization is necessary? 
(2) Which prompt (text or sketch) is preferred?
As shown in Fig.~\ref{fig:hstud1}, 80\% of tailors consider 
automated personalization to be useful, as it saves them time in production and the cost of business.
Besides, 40\% prefer sketch over text for instruction,
whilst 5\% are concerned with sketch to be only for professional designers.
We also collected the tailors feedback on the functions of (a) adding new garment panels, (b) removing garment panels, (c) changing dress topology, and (d) creating new designs. As shown in Fig.~\ref{fig:hstud2}, removing garment panels is most popular, which is supported by our proposed method. 

% To evaluate the performance in a human perceptive level, we conduct thoughtful user studies in this section. Human subjects \ie professional Tailors evaluation is conducted to investigate the usefullness of customized garment pattern design. We have conducted this experiments on twenty human subjects and reported the scores in percentages out of hundred. As observed from Fig~\ref{fig:hstud1}(a), majority of the tailors prefer automated personalization to save time of production and cost of business. It is also interesting to note from Fig~\ref{fig:hstud1}(b) that majority of the tailors prefer sketch as a medium of customer input for the ease of production. However, 5\% of the tailors also raised concerns on the fact that sketch is only for professional designers and not so customer friendly. We also collected tailors feedback on the most requested mode of personalization among (a) adding new garment panels (b) removing panels (c) changing dress topology and (d) creating new designs. From the results in Fig~\ref{fig:hstud2}, it can be observed that majority of customers prefer removing garment panels which is also supported by our proposed solution. 

\begin{figure}[h]
    \centering
    \includegraphics[scale=0.38]{img/human_eval.png}
    % % \vspace{-0.1in}
    \caption{\textbf{Votes on garment personalization.}}
    \label{fig:hstud1}
    % % \vspace{-0.2in}
\end{figure}

\begin{figure}
    \centering
    \includegraphics[scale=0.23]{img/human_study_2.png}
    \caption{\textbf{Votes on garment personalization functions.}}
    \label{fig:hstud2}
\end{figure}

\section{Limitations and Future Work}
We presented a personalized 2D pattern design method for 3D garments,
featuring editing capabilities. Our model is 2D panel-aware, requires no panel annotation, and leverages the Transformer architecture to
form globally coherent 2D patterns of varied topology. The network was designed to allow editing at an interactive rate, where, as demonstrated, the user can interact with the model using a simple instruction (refer to Fig 2). However, one limitation is a relatively small set of panels and
garments, lacking multiple test domains for evaluation. Our mask based panel design cannot model complicated 2D patterns like pleats or darts which is another limitation.  As this is an under-studied area, many challenges (3D optimization, fitting to different body structures etc.) still exist for future work.

\section{Conclusion}
%In this paper, we propose a novel 3D medical segmentation method, dubbed as Diff-UNet, based on the diffusion probabilistic model.   
%Compared with other diffusion-based segmentation methods, this method achieves multi-target segmentation by modeling medical image segmentation as a discrete data generation task and realizes the first 3D medical image segmentation based on the Diffusion model.
In this paper, we propose the first 3D medical image segmentation method, named Diff-UNet, based on the diffusion model, which models medical image segmentation as a discrete data generation task. The proposed algorithm introduces a generic end-to-end 3D medical image segmentation approach, leveraging the advantages of the Diffusion model to improve segmentation robustness. Experimental results on different benchmark datasets demonstrate the superiority of our Diff-UNet over state-of-the-art approaches.
Overall, our work presents a significant contribution to the field of medical image segmentation, demonstrating the effectiveness of the Diffusion model in the 3D medical image segmentation task. The proposed method has the potential to facilitate more precise and accurate diagnosis and treatment of medical conditions, ultimately leading to improved patient outcomes.
 
% First, Diff-UNet uses one-hot encoding to convert multi-target segmentation labels into multiple one-hot labels for the multi-target segmentation problem and adds random noise to the converted labels.
% After that, we concatenate the original 3D medical images with noisy one-hot tags in the channel dimension and input them to Denoising-UNet's encoder for feature extraction. Meanwhile, to enhance the feature representation capability of Denoising-UNet, we add an image encoder to extract the multiscale features of the original image and fuse them with the encoder of Denoising-UNet.
% The output of the Denoising-UNet decoder is supervised by DICE loss, BCE loss, and MSE loss to learn the denoising process.
% Finally, we propose a novel SUF module that computes fusion weights based on prediction steps and model output uncertainty to achieve more robust segmentation results.

% \begin{table*}[tp]
% \floatsetup{floatrowsep=qquad,captionskip=5pt} \tabcolsep=5pt 
\small %此处写字体大小控制命令

\begin{floatrow}
\resizebox{0.58\textwidth}{!}{
\centering
\setlength\tabcolsep{3pt}%调列距
\renewcommand\arraystretch{1.0}
\ttabbox{\caption{Ablation study for different modules on BraTS2020 dataset. FE denotes the separate Feature Encoder. SF denotes a simple fusion. SUF denotes the Step-Uncertainty based Fusion module.}}{%
\vspace{-2mm}
\label{tab:ablation_module}
\begin{tabular}{c c c c c c} 
    \hline
    Module & & WT & TC & ET & Average \\
    \hline
    basic & &  91.62 & 85.02 & 75.10 & 83.91 \\
    basic+FE & & 91.52 & 85.85 & 75.59 & 84.32 \\
    basic+FE+SF & &  92.02 & 86.58 & 75.67 & 84.76 \\
    basic+FE+$\mathrm{SUF}$ (Ours) & & \textbf{92.23} & \textbf{86.94} & \textbf{76.87} & \textbf{85.35}\\
    \hline
    \end{tabular}
    }
%\vspace{-3mm}
}


%\caption{Ablation study for different prediction number of each DDIM step on BraTS2020 dataset. $S$ is the prediction number of each DDIM step to compute the uncertainty. We obtain the best averaged Dice score when $S = 4$.}

\resizebox{0.42\textwidth}{!}{
\centering
\setlength\tabcolsep{3pt}%调列距
\renewcommand\arraystretch{1.0}
\begin{floatrow}
\ttabbox{\caption{Ablation study for the number (S) of predictions to compute the uncertainty in each DDIM step on BraTS2020 dataset. }
\vspace{-2mm}
\label{tab:ablation_S}}{%
    \begin{tabular}{c c c c c c} 
    \hline
    $S$ & & WT & TC & ET & Average \\
    \hline
    3 & &  92.19 & 86.18 & 76.82 & 85.06 \\
    4 (Ours) & & \textbf{92.23} & 86.94 & \textbf{76.87} & \textbf{85.35} \\
    5 & &  92.17 & \textbf{86.96} & 76.84 & 85.32 \\
    6 & & 92.22 & 86.92 & 76.84 & 85.33 \\
    \hline

    \end{tabular}
    }
\end{floatrow}
}
\vspace{-5mm}
\end{floatrow}
\end{table*}

%
%
% \section{First Section}
% \subsection{A Subsection Sample}
% Please note that the first paragraph of a section or subsection is
% not indented. The first paragraph that follows a table, figure,
% equation etc. does not need an indent, either.

% Subsequent paragraphs, however, are indented.

% \subsubsection{Sample Heading (Third Level)} Only two levels of
% headings should be numbered. Lower level headings remain unnumbered;
% they are formatted as run-in headings.

% \paragraph{Sample Heading (Fourth Level)}
% The contribution should contain no more than four levels of
% headings. Table~\ref{tab1} gives a summary of all heading levels.

% \begin{table}
% \caption{Table captions should be placed above the
% tables.}\label{tab1}
% \begin{tabular}{|l|l|l|}
% \hline
% Heading level &  Example & Font size and style\\
% \hline
% Title (centered) &  {\Large\bfseries Lecture Notes} & 14 point, bold\\
% 1st-level heading &  {\large\bfseries 1 Introduction} & 12 point, bold\\
% 2nd-level heading & {\bfseries 2.1 Printing Area} & 10 point, bold\\
% 3rd-level heading & {\bfseries Run-in Heading in Bold.} Text follows & 10 point, bold\\
% 4th-level heading & {\itshape Lowest Level Heading.} Text follows & 10 point, italic\\
% \hline
% \end{tabular}
% \end{table}


% \noindent Displayed equations are centered and set on a separate
% line.
% \begin{equation}
% x + y = z
% \end{equation}
% Please try to avoid rasterized images for line-art diagrams and
% schemas. Whenever possible, use vector graphics instead (see
% Fig.~\ref{fig1}).

% \begin{figure}
% \includegraphics[width=\textwidth]{fig1.eps}
% \caption{A figure caption is always placed below the illustration.
% Please note that short captions are centered, while long ones are
% justified by the macro package automatically.} \label{fig1}
% \end{figure}

% \begin{theorem}
% This is a sample theorem. The run-in heading is set in bold, while
% the following text appears in italics. Definitions, lemmas,
% propositions, and corollaries are styled the same way.
% \end{theorem}
% %
% % the environments 'definition', 'lemma', 'proposition', 'corollary',
% % 'remark', and 'example' are defined in the LLNCS documentclass as well.
% %
% \begin{proof}
% Proofs, examples, and remarks have the initial word in italics,
% while the following text appears in normal font.
% \end{proof}
% For citations of references, we prefer the use of square brackets
% and consecutive numbers. Citations using labels or the author/year
% convention are also acceptable. The following bibliography provides
% a sample reference list with entries for journal
% articles~\cite{ref_article1}, an LNCS chapter~\cite{ref_lncs1}, a
% book~\cite{ref_book1}, proceedings without editors~\cite{ref_proc1},
% and a homepage~\cite{ref_url1}. Multiple citations are grouped
% \cite{ref_article1,ref_lncs1,ref_book1},
% \cite{ref_article1,ref_book1,ref_proc1,ref_url1}.

% \subsubsection{Acknowledgements} Please place your acknowledgments at
% the end of the paper, preceded by an unnumbered run-in heading (i.e.
% 3rd-level heading).

%
% ---- Bibliography ----
%
% BibTeX users should specify bibliography style 'splncs04'.
% References will then be sorted and formatted in the correct style.
%
\bibliographystyle{splncs04}
\bibliography{references}
%
% \begin{thebibliography}{8}
% \bibitem{ref_article1}
% Author, F.: Article title. Journal \textbf{2}(5), 99--110 (2016)

% \bibitem{ref_lncs1}
% Author, F., Author, S.: Title of a proceedings paper. In: Editor,
% F., Editor, S. (eds.) CONFERENCE 2016, LNCS, vol. 9999, pp. 1--13.
% Springer, Heidelberg (2016). \doi{10.10007/1234567890}

% \bibitem{ref_book1}
% Author, F., Author, S., Author, T.: Book title. 2nd edn. Publisher,
% Location (1999)

% \bibitem{ref_proc1}
% Author, A.-B.: Contribution title. In: 9th International Proceedings
% on Proceedings, pp. 1--2. Publisher, Location (2010)

% \bibitem{ref_url1}
% LNCS Homepage, \url{http://www.springer.com/lncs}. Last accessed 4
% Oct 2017
% \end{thebibliography}
\end{document}
