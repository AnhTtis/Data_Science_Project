\documentclass[10pt,journal,compsoc]{IEEEtran}
\usepackage{multirow}
\usepackage{xcolor}
\usepackage{fixltx2e}
\usepackage[T1]{fontenc}
\usepackage{microtype}
\usepackage{booktabs}
\usepackage{tabularx}
\newcommand{\CLASSINPUTbaselinestretch}{1.0} % baselinestretch
\newcommand{\CLASSINPUTinnersidemargin}{1in} % inner side margin
\newcommand{\CLASSINPUToutersidemargin}{1in} % outer side margin
\newcommand{\CLASSINPUTtoptextmargin}{1in}   % top text margin
\newcommand{\CLASSINPUTbottomtextmargin}{1in}% bottom text margin
\usepackage{ifpdf}
\ifCLASSOPTIONcompsoc
  \usepackage[nocompress]{cite}
\else
  \usepackage{cite}
\fi
\ifCLASSINFOpdf
  \usepackage[pdftex]{graphicx}
\else
   \usepackage[dvips]{graphicx}
\fi

\usepackage{amsmath}
\usepackage{acronym}
\usepackage{algorithmic}
\usepackage{array}
\usepackage{mdwmath}
\usepackage{mdwtab}

\usepackage{eqparbox}
\ifCLASSOPTIONcompsoc
  \usepackage[caption=false,font=footnotesize,labelfont=sf,textfont=sf]{subfig}
\else
  \usepackage[caption=false,font=footnotesize]{subfig}
\fi
\usepackage{fixltx2e}
\ifCLASSOPTIONcaptionsoff
  \usepackage[nomarkers]{endfloat}
 \let\MYoriglatexcaption\caption
 \renewcommand{\caption}[2][\relax]{\MYoriglatexcaption[#2]{#2}}
\fi
\usepackage{url}
\ifCLASSINFOpdf
  \usepackage[pdftex]{thumbpdf}
\else
  \usepackage[dvips]{thumbpdf}
\fi
\newcommand\MYhyperrefoptions{bookmarks=true,bookmarksnumbered=true,
pdfpagemode={UseOutlines},plainpages=false,pdfpagelabels=true,
colorlinks=true,linkcolor={black},citecolor={black},urlcolor={black},
pdftitle={Bare Demo of IEEEtran.cls for Computer Society Journals},%<!CHANGE!
pdfsubject={Typesetting},%<!CHANGE!
pdfauthor={Michael D. Shell},%<!CHANGE!
pdfkeywords={Computer Society, IEEEtran, journal, LaTeX, paper,
             template}}%<^!CHANGE!
\hyphenation{op-tical net-works semi-conduc-tor}


\begin{document}
\title{ System and Design Technology Co-optimization of SOT-MRAM for High-Performance AI Accelerator Memory System}
\author{Kaniz~Mishty~\IEEEmembership{}
        and~Mehdi~Sadi,~\IEEEmembership{Member,~IEEE,}% <-this % stops a space
\IEEEcompsocitemizethanks{\IEEEcompsocthanksitem The authors are with the Department of Electrical and Computer Engineering, Auburn University, Auburn, AL 36849 USA \protect\\
% note need leading \protect in front of \\ to get a newline within \thanks as
% \\ is fragile and will error, could use \hfil\break instead.
E-mail: kzm0114@auburn.edu; mehdi.sadi@auburn.edu
%\IEEEcompsocthanksitem J. Doe and J. Doe are with Anonymous University.
}% <-this % stops a space
\thanks{This work was supported in part by the National Science Foundation under Grant Number CRII-2153394.}
\thanks{Manuscript received November XX, 2022; revised month XX, 2023.}}
\markboth{IEEE Transactions on Computers,~Vol.~XX, No.~X, August~202X}%
{Shell \MakeLowercase{\textit{et al.}}: Bare Advanced Demo of IEEEtran.cls for IEEE Computer Society Journals}
\IEEEtitleabstractindextext{%
\begin{abstract}
SoCs are now designed with their own AI accelerator segment to accommodate  the ever-increasing demand of Deep Learning (DL)  applications. With powerful MAC engines for matrix multiplications, these accelerators show high computing performance. However, because of limited memory resources (i.e., bandwidth and capacity), they fail to achieve optimum system performance during large batch training and inference. In this work, we propose a memory system with high on-chip capacity and bandwidth to shift the gear of AI accelerators from memory-bound to achieving system-level peak performance. We develop the memory system with DTCO-enabled customized SOT-MRAM as large on-chip memory through STCO and detailed characterization of the DL workloads. 
%We evaluate our workload-aware memory system on the CV and NLP benchmarks and observe significant PPA improvement compared to an SRAM-based in both inference and training modes. 
Our workload-aware memory system achieves 8$\times$ energy and 9$\times$ latency improvement on Computer Vision (CV) benchmarks in training and 8$\times$ energy and 4.5$\times$ latency improvement on Natural Language Processing (NLP) benchmarks in training while consuming only around 50\% of SRAM area at iso-capacity. 
\end{abstract}

% Note that keywords are not normally used for peerreview papers.
\begin{IEEEkeywords}
DTCO, STCO, AI/Deep Learning Accelerator, SOT-MRAM.
\end{IEEEkeywords}}


% make the title area
\maketitle

\IEEEdisplaynontitleabstractindextext

\IEEEpeerreviewmaketitle
\section{Introduction}
\label{intro}
%\IEEEPARstart{T}{he} proliferation of Artificial Intelligence (AI) and Deep Learning (DL) has precipitated the microprocessor industry to experience a large amount of data processing and data storage. With a significant amount of research effort on Design Specific Architecture (DSA) for Deep Learning domains, large-scale data processing is no more overwhelming for computing devices \cite{dnn_survey}\cite{tpu}\cite{amp100_gpu}. However, the lack of efficient and high-performance data communication between the computing and memory element (known as memory wall or memory bottleneck) masks the improvement coming from the efficient compute system \cite{cao2021mobile}. One promising solution to address the memory bottleneck of AI-specific workload is to increase the on-chip memory capacity\cite{park2018deep}. The area inefficiency, leaky nature, and process variation of existing CMOS-based memory (i.e., SRAM) at advanced technology nodes, combined with an increasing demand for on-chip memory capacity, have led researchers to explore potential alternatives of SRAM as on-chip memory.



\IEEEPARstart{T}{he} proliferation of Artificial Intelligence (AI) and Deep Learning (DL) has precipitated the computing hardware community to continually design innovative AI/DL accelerators with  large data processing capabilities. Research shows that the AI/DL model accuracy improves as training data set size grows \cite{DL_data1}. With increasing data set, model size also grows. Consequently, memory demand in AI/DL accelerators will also grow asymptotically linearly with model and data size \cite{DL_data1} \cite{dnn_survey}. As a result, the bottleneck for state-of-the-art AI/DL models in the accelerator hardware is now memory rather than data and compute availability, and we expect this trend to worsen in the future \cite{dnn_survey}\cite{tpu}\cite{amp100_gpu}.

The lack of efficient and high-performance data flow between the computing and memory element (i.e., the memory wall or memory bottleneck) masks the improvement coming from the efficient compute system \cite{cao2021mobile}. One promising solution to the memory bottleneck of AI-specific workload is to increase the on-chip memory capacity\cite{park2018deep}. For both training and inference, the on-chip memory capacity in the accelerator needs to be increased to ensure that the intermediate activations, as well as the weights of the current layer, can be loaded. Moreover, significantly more memory is required during training to store the gradients and optimizer states. Inadequate on-chip memory capacity will cause frequent DRAM accesses which will exacerbate energy costs, as well as stall the compute cores of AI/DL accelerator until the data is fetched. Because of this large capacity demand, an SRAM-based on-chip memory system can be detrimental due to leakage energy and area inefficiency. 



\begin{figure*}[ht]
	\centering
	\includegraphics[width=1.0\textwidth]{Fig_1_Paper_concept_new.pdf}
	%\vspace{-20pt}
	\caption{Workflow of closed-loop analysis for system and device level optimization for AI/Deep Learning Accelerator Design}
	\label{fig:paper_concept}
	%\vspace{-10pt}
\end{figure*}

% The promising features, such as high density, near-zero leakage power, immunity against radiation-induced soft errors, and CMOS compatibility of non-volatile memory (NVM) technologies, attracted researchers from academia and industry to study different NVM technologies actively \cite{memory_trend}. In addition to retaining all NVM-inherent properties, Spin Transfer Torque (STT) MRAM exhibits superiority as an embedded storage element in the cache over the other NVM variants, e.g., Phase Change Memory (PCM), and Resistive RAM (ReRAM) \cite{dac_19}\cite{memory_trend} because of its higher endurance (number of times data can be written to the memory) and recent improvement in read-write access latency and energy \cite{stt_ai_us}. As a result, STT-MRAM has already shifted its gear from the R\&D phase to commercialization as the NAND-based embedded flash replacement \cite{dac_19} \cite{recent-progress-in-SOT_fab3}.


% The promising features, such as high density, near-zero leakage power, immunity against radiation-induced soft errors, and CMOS compatibility of  emerging Spin-based non-volatile (NVM)  magnetic memory (i.e., MRAM)  technologies, attracted researchers from academia and industry \cite{memory_trend}.  However, MRAM in its regular form cannot be used in AI accelerators due to its slow write speed and high write energy \cite{ recent-progress-in-SOT_fab3}\cite{optimized_SOT_imec}.

% Spin Transfer Torque (STT) MRAM, has already shifted its gear from the R\&D phase to commercialization as the NAND-based embedded flash replacement \cite{dac_19} \cite{recent-progress-in-SOT_fab3}. STT-MRAM, a two-terminal magnetic memory with Magnetic Tunnel Junction (MTJ) as the storing element, flows a bidirectional spin-polarized current through the MTJ for read-write operation \cite{stt_eqn1}. The major challenges of STT-MRAM, such as poor write performance, reliability issues, e.g., Read Disturbance (RD), retention failure, \cite{recent-progress-in-SOT_fab3}\cite{dualport_fieldfree_fab2}, stem from two main reasons. First, the high write current flowing through the MTJ accounts for almost $10\times$ energy consumption as SRAM. Large write delay (> ns range) resulting from spin injection symmetry in switching the magnetic orientation of free layer belittles STT-MRAM's feasibility as an on-chip cache \cite{ultrafast_embedded_mem_fab4}. The stress on the dielectric oxide of the MTJ due to the large write current accelerates the time-dependent wear out of the cell \cite{tahoori_1}. Second, its shared read-write  path makes it vulnerable to RD.
%that makes it vulnerable to RD because it shares the same electrical path for read-write access. 
%Furthermore, an identical read-write path never allows it to be suitable for dual-port memory that can simultaneously afford a read-write operation, thus limiting memory bandwidth \cite{dualport_fieldfree_fab2}. 

The promising features, such as high density, near-zero leakage power, immunity against radiation-induced soft errors, and CMOS compatibility of  emerging Spin-based non-volatile (NVM)  magnetic memory (i.e., MRAM)  technologies, attracted researchers from academia and industry \cite{memory_trend}. Spin Transfer Torque (STT) MRAM, has already shifted its gear from the R\&D phase to commercialization as the NAND-based embedded flash replacement \cite{dac_19} \cite{recent-progress-in-SOT_fab3}. However, MRAM in its regular form cannot be used in AI accelerators due to its slow write speed and high write energy \cite{ recent-progress-in-SOT_fab3}\cite{optimized_SOT_imec}.

STT-MRAM, a two-terminal magnetic memory with Magnetic Tunnel Junction (MTJ) as the storing element, flows a bidirectional spin-polarized current through the MTJ for read-write operation \cite{stt_eqn1}. The major challenges of STT-MRAM, such as poor write performance, reliability issues, e.g., Read Disturbance (RD), retention failure, \cite{recent-progress-in-SOT_fab3}\cite{dualport_fieldfree_fab2}, stem from two main reasons. First, the high write current flowing through the MTJ accounts for almost $10\times$ energy consumption as SRAM. Large write delay (> ns range) resulting from spin injection symmetry in switching the magnetic orientation of free layer belittles STT-MRAM's feasibility as an on-chip cache \cite{ultrafast_embedded_mem_fab4}. The stress on the dielectric oxide of the MTJ due to the large write current accelerates the time-dependent wear out of the cell \cite{tahoori_1}. Second, its shared read-write  path makes it vulnerable to RD.



Spin-Orbit Torque (SOT) MRAM, considered the next generation of STT-MRAM, offers high performance without compromising reliability issues such as RD. SOT-MRAM is a four-terminal memory cell that uses MTJ as the storing element \cite{sot_model_kazemi}. By splitting the read-write path and using a different switching scheme, SOT-MRAM resolves all the challenges of STT-MRAM while retaining its every benefit \cite{recent-progress-in-SOT_fab3} \cite{dualport_fieldfree_fab2} \cite{ultrafast_embedded_mem_fab4} \cite{tahoori_1} \cite{size_dependent_switching_fab1}   . Isolate read and write path allows the designer to optimize the read and write path independently, decreasing the write current and increasing the read-write operating margin, thus solving the RD-induced reliability issues. %Moreover, a separate read-write path allows straightforward implementation of the dual-port memory cell, resulting in high bandwidth \cite{dualport_fieldfree_fab2}. 
%The absence of a large write current flowing through the MTJ cell ensures a longer lifetime. The non-volatility property of SOT-MRAM, along with contributing to the easy adoption of power gating, also initiates normally-off/instant-on computing facilities \cite{tahoori_2}. 
Though lacking mass-scale production from foundries due to early-stage manufacturing challenges, \cite{recent-progress-in-SOT_fab3} \cite{optimized_SOT_imec} 
\cite{dualport_fieldfree_fab2} \cite{ultrafast_embedded_mem_fab4} \cite{size_dependent_switching_fab1} \cite{sot_0.35ns_write} have demonstrated the successful fabrication of SOT-MRAM with attractive specifications. Its attractive features, such as high density, reliability and endurance, zero leakage, read-write latency comparable to SRAM, and research effort to enable mass production make it one of the best candidates for AI accelerator memory system where large on-chip memory is a must for training and inference.

The performance of an AI accelerator depends on both the compute and memory throughput of the device. While most accelerators have enough compute throughput, their performance is limited by memory throughput operating in the \emph{memory bound} region. To address the \emph{memory bound} problem of the AI hardware, in this paper, we perform a closed-loop STCO on AI workloads and DTCO on SOT-MRAM to present a hybrid memory system. To our knowledge, this is the first work that analyzes and evaluates the performance of SOT-MRAM as the on-chip memory of AI accelerators targeting both inference and training. The STCO-DTCO methodology is shown in Fig. \ref{fig:paper_concept}, and the key contributions of the paper are highlighted as follows.

%The proposed memory system can provide up to xxxGHz off-chip, xxx GHz on-chip read bandwidth, and xxxGHz on-chip write bandwidth. The key contributions of this paper are as follows.
\begin{itemize}
    \item We present a power and performance-optimized hybrid memory system for Deep Learning (DL) accelerators through a workload-aware STCO and DTCO. Comprised of off-chip HBM3 DRAM, on-chip SRAMs, and DTCO-enabled SOT-MRAM, the hybrid memory system can support the training and inference of DL workloads. We perform a closed-loop STCO and DTCO by taking into account the (i) System performance attributes (e.g., throughput and energy cost); (ii) Architectural and micro-architectural attributes (e.g., compute resources utilization, memory bandwidth) (iii) Workload attributes at both training and inference (e.g., runtime action counts, dataflow and data reuse) to reach the Pareto optimal solution.
    
    \item Using the Deep Learning models’ execution profiles,  DTCO enables device and circuit level customization of read/write bandwidth, retention time, and capacity of SOT-MRAM memory banks to meet the bandwidth and capacity demands of DL workloads. Moreover, to achieve dynamic runtime optimization of the power and performance of the accelerator hardware for diverse workloads, memory banks are individually optimized with various bandwidths and capacities.

    \item Finally, using various DNN benchmarks,  we provide a comparative analysis of the existing SRAM-based memory system and the proposed DTCO-STCO optimized hybrid memory system for AI accelerators.
\end{itemize}


The rest of the article is organized as follows. Section \ref{background} discusses the background. In Section \ref{workload_profiling}, we present the analytical model for DNN workload profiling, followed by the DTCO of SOT-MRAM in Section \ref{dtco_MRAM}. Sections \ref{result_analysis} and \ref{rltd_work} present the results \& analysis, and related works, respectively, following the conclusion in Section \ref{conclusion}.

\section{Background}
\label{background}
\subsection{AI/DL Applications}

\begin{figure*}[ht]
    
    \centering
    \includegraphics[scale=0.8]{Fig_5_Convolution_rebuttal.pdf}
    
    \caption{CV model (CNN/DNN) abstract architecture. Deep convolution (Conv) layers with residual/skip connection followed by fully connected (FC) layer/s. For symbol meaning please see Table \ref{sys_param}.}
    \label{fig:conv_op}
       
\end{figure*}
% \vspace{-15pt}
\subsubsection{Computer Vision (CV) and Pattern Recognition}

% CV is one of the earliest applications of DL. 
CV models, also called Convolutional/Deep Neural Networks (CNN/DNN),  are the stacks of convolution layers connected straight and/or through residual connection \cite{resnet} to extract the objects’ features, and a few Fully Connected (FC) layers at the end to classify the objects. Image classification,  captioning, reconstruction and object/instance segmentation are the scopes of CV models.  Deep Residual Networks, having convolutional layers at their core, dominate the CV domain. The input images are convolved with the filter weights to produce the output feature map (\emph{OFMAP}). The \emph{OFMAP} goes through the pooling and normalization layers to act as input (\emph{IFMAP}) to the next layer. The linear and softmax layer at the end finally recognizes the image (Fig. \ref{fig:conv_op}). The size of each data entity (IFMAP, OFMAP, and Weights) depend on the model architecture. 


\subsubsection{Natural Language Processing (NLP)}
Language modeling deals with processing sequential data. Recurrent Neural Networks (RNN), Long Short Term Memory (LSTM), and Gated Recurrent Unit (GRU) have been used in language modeling until the state-of-the-art Transformer \cite{vaswani2017attention} model is introduced. NLP models are used in machine translation, text summarization, speech recognition, syntactic and semantic parsing, question answering, dialog system etc. In Transformer-based models \cite{vaswani2017attention}, the input sequence propagates through the embedding layer and different sublayers of the encoder stacks to extract different linguistic features and inter-token dependency of the input sequence. The decoder stacks then generate the output sequence by taking the encoded input sequence from the encoder stack and the output sequence generated by itself in the previous timesteps (Fig \ref{fig:transformer}). The input sequence multiplied by different layer weights takes different activation names and shapes throughout the model operation.



\begin{figure*}[ht]
    \centering
    \includegraphics[scale=0.65]{Fig_4_Transformer.pdf}
    
    \caption{Transformer model workflow breakdown}
   
    \label{fig:transformer}
\end{figure*}

\subsection{AI/DL Accelerators}

At the core of AI/DLs is the matrix-matrix/vector multiplication (GEMM) with massive parallelism. Exploiting this parallelism,  Systolic Array (SA) based architecture \cite{tpu} have been used to accelerate the computations. Different dataflows, such as row stationary, output stationary, weight stationary, have been evolved to maximize the reuse and reduce the data movement. Off-chip DRAM access being 100-200 times more energy and latency expensive than any ALU operation or on-chip access \cite{eyeriss} plays a crucial role in determining the overall system performance. Another non-conventioanl type of architecture, In-Memory Computing (IMC) \cite{imc_survey} has recently  evolved to address the data communication cost for DNN accelerators. However, in this work, we focus on reducing the off-chip memory access for conventional DNN accelerator architectures \cite{tpu}, \cite{eyeriss}\cite{nvdla} by increasing the on-chip Global Buffer (GLB) size with SOT-MRAM. 

\begin{figure}[ht]
	\centering
	\includegraphics[scale=0.72]{Fig_3_SOT_bitcell_new.pdf}
	
	\caption{Physical structure of a SOT-MRAM bit cell highlighting separate read (along blue line) and write (along red line) path}
	\label{fig:sot_bitcell}	
	
\end{figure}


\subsection{SOT-MRAM}
\subsubsection{Physical Structure} 
With MTJ \cite{stt_eqn1} as storing element, the SOT-MRAM is a 
three terminal device. Depending on the type of bit cell, there are three to four lines to control the read-write operation. In this work, we consider a two transistor one SOT (2T1SOT) bit cell architecture that requires two access transistors, (i) \emph{Read Wordline (RWL)}, (ii) \emph{Write Wordline (WWL)}, (iii) \emph{Bit Line (BL)}, and (iv) \emph{source Line (SL)} to accommodate separate read-write access path \cite{sot_model_kazemi} \cite{2T1SOT} (Fig. \ref{fig:sot_bitcell}). The MTJ stack, with its free layer at the interface, is placed on top of a SOT layer (i.e., channel) to ensure SOT-induced switching. The SOT layer is composed of heavy metals or topological insulators \cite{manchon2019current}. 



\subsubsection{Read-Write Operation}
Upon the activation of RWL, a small amount of current is passed through BL and grounded SL. The resistive state of the MTJ is captured by sensing the voltage across it and comparing the voltage with a reference value \cite{dualport_fieldfree_fab2}. Low resistive state ($R_{P}$) and high resistive state ($R_{AP}$) represents bit 0 and 1 respectively.
%\subsubsection{Write operation} 
The write operation of MTJ-based MRAM involves switching the resistive status of MTJ. In SOT-MRAM, switching occurs due to Spin Orbit Torque (SOT) effect. Unlike STT-MRAM, a current is passed through the SOT layer to change the MTJ resistive state by switching the magnetic orientation of the free layer. 
A bidirectional write current flows through BL and SL during write operation. The potential of BL and SL changes depending on the bit value written in the cell. For example, to write `1', current flows from BL to SL and vice versa to write `0' \cite{dualport_fieldfree_fab2} \cite{tahoori_1}.



%\section{WORKLOAD AWARE HYBRID MEMORY MODEL}
\section{DNN WORKLOAD PROFILING}
\label{workload_profiling}
Profiling the target workload is a prerequisite for designing an accelerator for the target workload. Assuming that we have a powerful computing system to handle the exhaustive computations of the DL workload, we focus on providing efficient data movement between the compute and memory system to ensure 100\% utilization of computing resources by introducing the workload-aware hybrid memory system. We propose the hybrid memory system by analyzing the Deep Learning model workloads from Computer Vision (CV) and Natural Language Processing (NLP) domain.
We analytically model the on-chip bandwidth requirement and memory access patterns of different parts of the workload during inference and training, \emph {Memory and Compute Model}, to develop the memory system.
%The model is developed considering the memory bandwidth requirement (
%both on-chip to DRAM and %
%compute unit to on-chip), memory access count, and memory size requirement of different parts of the workload. In this section, we explain each consideration in detail.


\begin{figure}[ht]
    \centering
    \includegraphics[scale=0.7]{Fig_2_Accelerator_system_2.pdf}
    %\vspace{-10pt}
    \caption{Block diagram of Accelerator architecture}
    \label{fig:system_architecture}
\end{figure}


\begin{figure*}[ht]
    \centering
    \includegraphics[scale=0.9]{Fig_5_Convolution_new.pdf}
    %\vspace{-12pt}
    \caption{CV models workflow breakdown}
    %\vspace{-5pt}
    \label{fig:conv_op}
\end{figure*}


\begin{figure*}[ht]
    \centering
    \includegraphics[scale=0.7]{Fig_6_Training_comp_graph.pdf}
    %\vspace{-10pt}
    \caption{Computational graph of DNN training}
    %\vspace{-12pt}
    \label{fig:comp_graph}
\end{figure*}


\subsection{Memory Bandwidth Expression}
We express the required bandwidth (BW) as a function of compute resources and workload. $BW$ (bytes/sec)  is defined as the rate at which data needs to be transferred to/from memory by a processor to fully utilize the computation resources of the processor. Mathematically,
\begin{align}
\label{req_mem_bw}
   BW= \frac{F_p}{OI}
\end{align}
Where $F_p$ = Theoretical peak performance of accelerator (ops/sec) = number of operations the accelerator performs per sec. The $F_{p}$ of a $H_{A} \times W_{A}$  Processing Element (PE) array (Fig. \ref{fig:system_architecture}):

\begin{align}
    \label{peak_fp}
        F_{p} = H_A*W_A*F_{acc}
\end{align}
$F_{acc}$ = Operating frequency of the accelerator. $OI$ = Operational Intensity of Workload (ops/byte) = number of operations performed by the accelerator per byte. It is a workload-dependent parameter and a measure of parallelism of the workload. In the subsequent subsections, we will formulate the $OI$ of Conv. and Fully-Connected (FC) layer to find their BW, respectively. Note that the read and write bandwidth will not be the same for these workloads. 
%For determining write bandwidth, we define $OI$ as the number of outputs generated per byte.




\subsubsection{Read Bandwidth ($BW_{RD}$) of Conv. layer}

To formulate an expression for \emph{OI} of convolution workload: First, we determine the total number of MAC operations, $T_{MAC}$, performed by a $H_{A} \times W_{A}$ PE array per clock cycle
\begin{align}
    T_{MAC}\;=H_{A}\;*\;W_{A}
\end{align}
Second, we figure out how many bytes should be read from memory to utilize all PEs of the accelerator in one clock cycle. In a row stationary dataflow \cite{eyeriss}, it takes ($k_{h}*k_{w}+of_{h}*of_{w})*d_{w}$ bytes of data ($d_{w}$ = data type in bytes, i.e., FP32, BF16 etc.) and \#$(of_{h} * of_{w} * k_{h} *k_{w})$ PEs to generate the partial ofmaps corresponding to one input channel. Depending on the size of the PE array, in each iteration (one complete use of accelerator), multiple input channels can be fit. The input channels (i.e., no. of partial ofmaps) computed by the PE array in each iteration:

\begin{align}
    N_{ich\_per\_stp}\;=\frac{H_A*W_A}{of_{h}*of_{w}*k_h*k_w} 
\end{align}
Total bytes read from memory to utilize all PEs:
\begin{equation}
\begin{aligned}
    T_{byte}=\frac{H_A*W_A}{k_h*k_w*of_{h}*of_{w}}* \\
    {(k_h*k_w+if_{h}*if_{w})*d_{w}}
\end{aligned}
\end{equation}
We divide the total number of MAC operations, $T_{MAC}$, by the total bytes accessed, $T_{byte}$, to find $OI$:
\begin{equation}
    \begin{aligned}
    \label{oi}
        OI=\frac{k_h*k_w*of_{h}*of_{w}}{d_{w}*(k_h*k_w+if_{h}*if_{w})}
    \end{aligned}
\end{equation}
Substituting the expression of $OI$ in equation  \eqref{req_mem_bw} gives $BW_{RD}$ as a function of array size and workload:
\begin{equation}
    \begin{aligned}
        BW_{RD} & = \frac{(k_h*k_w\;+if_{h}*if_{w})*d_{w}}{k_w*k_h*of_{h}*of_{w}}*\\
& \;\;\;\;\;\;\;H_{A}*W_{A}*F_{acc}
    \end{aligned}
\end{equation}
For the symbol meanings please see Fig. \ref{fig:conv_op}.
\subsubsection{Write Bandwidth ($BW_{WR}$) of Conv. Layer}
Partial ofmap of a single input channel requires \#($of_{h}*of_{w}*k_{h}*k_{w}$) PEs. Therefore, $H_{A}\times W_{A}$ PEs generate $(H_{A}*W_{A})/(of_{h}*of_{w}*k_{h}*k_{w})$ ofmaps in each iteration. Each partial ofmap contains $of_{h}*of_{w}$ elements. The total output bytes generated by the PE array in one iteration is, equivalently, the write bandwidth is:
% \begin{equation}
% \begin{aligned}
%     \label{wr_bw}
%     BW_{WR} = \frac{H_{A}*W_{A}}{N_{ofmap\_rw}*N_{ofmap\_cl}*k_{h}*k_{w}}\;*\\ N_{ofmap\_rw}*N_{ofmap\_cl}*d_{width}*F_{acc}
% \end{aligned}
% \end{equation}





\begin{equation}
\begin{aligned}
    \label{wr_bw}
    BW_{WR} = \frac{H_{A}*W_{A} * F_{acc} * d_{w}}{k_{h}*k_{w}}\;
\end{aligned}
\end{equation}


\begin{table}[ht]
\setlength{\tabcolsep}{3pt} 
\centering
\caption{RD/WR bandwidth expression of FC layer for different cases}
\label{table:bw_expression}
\vspace{-8pt}
\begin{tabular}{|cc|c|c|}
\hline
\multicolumn{2}{|c|}{Cases}& $BW_{RD}$& $BW_{WR}$\\ \hline
\multicolumn{1}{|c|}{\multirow{4}{*}{$M<H_{A}$; $N<W_{A}$}}& \multirow{2}{*}{$K<W_{A}$}& \multirow{2}{*}{$\frac{M*N+K*M}{N+K}$}& \multirow{2}{*}{$\frac{K*N}{2*N+K-1}$}\\
\multicolumn{1}{|c|}{}&&&\\ \cline{2-4} 
\multicolumn{1}{|c|}{}& \multirow{2}{*}{$K\ge W_{A}$} & \multirow{2}{*}{$\frac{M*N+W_{A}*M}{N+W_{A}}$}& \multirow{2}{*}{$\frac{W_{A}*N}{2*N+K-1}$}\\
\multicolumn{1}{|c|}{}&&&\\ \hline
\multicolumn{1}{|c|}{\multirow{4}{*}{$M<H_{A}; N\ge W_{A}$}}  & \multirow{2}{*}{$K<W_{A}$}& \multirow{2}{*}{$\frac{M*W_{A}+K*M}{N+K}$}& \multirow{2}{*}{$\frac{K*W_{A}}{2*W_{A}+K-1}$}\\
\multicolumn{1}{|c|}{}&&&\\ \cline{2-4} 
\multicolumn{1}{|c|}{}& \multirow{2}{*}{$K\ge W_{A}$} & \multirow{2}{*}{$\frac{M*W_{A}+W_{A}*M}{2*W_{A}}$}& \multirow{2}{*}{$\frac{{W_{A}}^{2}}{2*W_{A}+K-1}$} \\
\multicolumn{1}{|c|}{}&&&\\ \hline
\multicolumn{1}{|c|}{\multirow{4}{*}{$M\ge H_{A}; N<W_{A}$}}  & \multirow{2}{*}{$K<W_{A}$}& \multirow{2}{*}{$\frac{H_{A}*N+K*H_{A}}{N+K}$}& \multirow{2}{*}{$\frac{K*N}{2*N+K-1}$}\\
\multicolumn{1}{|c|}{}&&&\\ \cline{2-4} 
\multicolumn{1}{|c|}{}& \multirow{2}{*}{$K\ge W_{A}$} & \multirow{2}{*}{$\frac{H_{A}*N+W_{A}*H_{A}}{W_{A}+N}$}& \multirow{2}{*}{$\frac{W_{A}*N}{2*N+K-1}$}\\
\multicolumn{1}{|c|}{}&&&\\ \hline
\multicolumn{1}{|c|}{\multirow{4}{*}{$M\ge H_{A};N\ge W_{A}$}} & \multirow{2}{*}{$K<W_{A}$}    & \multirow{2}{*}{$\frac{H_{A}*W_{A}+W_{A}*H_{A}}{W_{A}+K}$} & \multirow{2}{*}{$\frac{W_{A}*N}{2*N+K-1}$}\\
\multicolumn{1}{|c|}{}&&&\\ \cline{2-4} 
\multicolumn{1}{|c|}{}& \multirow{2}{*}{$K\ge W_{A}$} & \multirow{2}{*}{$\frac{H_{A}*W_{A}+W_{A}*H_{A}}{2*W_{A}}$} & \multirow{2}{*}{$\frac{{W_{A}^2}}{2*W_{A}+K-1}$}   \\
\multicolumn{1}{|c|}{}&&&\\ \hline
\end{tabular}
%\vspace{-8pt}
\end{table}




%  \setlength{\tabcolsep}{-3.5pt} 
 \setlength{\tabcolsep}{3.2pt}
\begin{table*}[ht]
{\tiny
\caption{Architecture-and-platform-agnostic Memory access and Dataflow of CV model workload during inference and training}
\vspace{-10pt}
\label{conv_analysis}
\begin{tabular}{|l|l|ll|ll|}
\hline
\multicolumn{1}{|c|} {\multirow{2}{*}{\textbf{Data Entity}}} & \multicolumn{1}{c|} {\multirow{2}{*}{\textbf{Data Dimension}}} & \multicolumn{2}{c|}{\textbf{Memory Access}} & \multicolumn{2}{c|}{\textbf{Dataflow \& Reuse Status}} \\ \cline{3-6} 
 &  & \multicolumn{1}{c|}{\textbf{Inference}} & \multicolumn{1}{c|}{\textbf{Training}} & \multicolumn{1}{c|}{\textbf{Inference}} & \multicolumn{1}{c|}{\textbf{Training}} \\ \hline
\multirow{11}{*}{\begin{tabular}{@{}l@{}}{Initial Input (I)}
\end{tabular}} & \multirow{11}{*}{\begin{tabular}[c]{@{}l@{}}{$\begin{aligned}
I_{h}\times I_{w}\times \\ N_{ich}\times N_{bt}
\end{aligned}$}\end{tabular}}
& \multicolumn{1}{l|}{\multirow{11}{*}{\begin{tabular}[c]{@{}l@{}}\textbullet{} {\textbf{HBM3 DRAM}}\\ \underline{Read}: Once*, if UB $\ge$ I, else, multiple* \\ \underline{Write}: No write back\\ \textbullet{} {\textbf{Unified Buffer (UB)}}\\ \underline{Read}:\\ UB $\rightarrow$ PE core: Once*\\ UB $\rightarrow$ HBM3: None\\ \underline{Write}:\\ PE core $\rightarrow$ UB: None\\ HBM3 $\rightarrow$ UB: Once*, if UB $\ge$ I, \\ else multiple*\end{tabular}}} & \multirow{11}{*}{\begin{tabular}[c]{@{}l@{}}\textbullet{} {\textbf{HBM3 DRAM}}\\ \underline{Read}: Once*, if UB $\ge$ I, else multiple*\\ \underline{Write}: No write back\\ \textbullet{} {\textbf{Unified Buffer (UB)}}\\ \underline{Read}:\\ UB $\rightarrow$ PE core: At least twice* (one during forward \\ pass, one during backward pass to calculate gradient)
\\ UB $\rightarrow$ HBM3: None\\ \underline{Write}:\\ HBM3 $\rightarrow$ UB: Once*, if UB $\ge$ I, else multiple*\\ PE core $\rightarrow$ UB: None\end{tabular}} & \multicolumn{1}{l|}{\multirow{11}{*}{\begin{tabular}[c]{@{}l@{}}\textbullet{} {\textbf{Dataflow}}\\ $HBM3 \rightarrow UB \rightarrow PE\; core$\\ \textbullet{} {\textbf{Reuse in PE core}}\\ Multiple times for convolutional \\ resue and over different filters\end{tabular}}} & \multirow{11}{*}{\begin{tabular}[c]{@{}l@{}}\textbullet{} {\textbf{Dataflow}}\\ \underline{Forward}: \\ $HBM3 \rightarrow UB \rightarrow  PE\; core$\\ \underline{Backward}: $UB \rightarrow PE\;core$\\ \textbullet{} {\textbf{Reuse in PE core}}\\ \underline{Forward}:\\ Multiple times for convolutional \\ resue and over different filters\\ \underline{Backward}:\\ Multiple times for gradient calculation \\ of  different filters\end{tabular}} \\
 &  & \multicolumn{1}{l|}{} &  & \multicolumn{1}{l|}{} &  \\
 &  & \multicolumn{1}{l|}{} &  & \multicolumn{1}{l|}{} &  \\
 &  & \multicolumn{1}{l|}{} &  & \multicolumn{1}{l|}{} &  \\
 &  & \multicolumn{1}{l|}{} &  & \multicolumn{1}{l|}{} &  \\
 &  & \multicolumn{1}{l|}{} &  & \multicolumn{1}{l|}{} &  \\
 &  & \multicolumn{1}{l|}{} &  & \multicolumn{1}{l|}{} &  \\
 &  & \multicolumn{1}{l|}{} &  & \multicolumn{1}{l|}{} &  \\
 &  & \multicolumn{1}{l|}{} &  & \multicolumn{1}{l|}{} &  \\
 &  & \multicolumn{1}{l|}{} &  & \multicolumn{1}{l|}{} &  \\
%  &  & \multicolumn{1}{l|}{} &  & \multicolumn{1}{l|}{} &  \\
%  &  & \multicolumn{1}{l|}{} &  & \multicolumn{1}{l|}{} &  \\
%  &  & \multicolumn{1}{l|}{} &  & \multicolumn{1}{l|}{} &  \\
 &  & \multicolumn{1}{l|}{} &  & \multicolumn{1}{l|}{} &  \\ \hline
\multirow{14}{*}{\begin{tabular}[c]{@{}l@{}}Output feature\\ map (OFMAP)\end{tabular}} & \multirow{7}{*}{\begin{tabular}{@{}l@{}}Conv. Layer:\\
$\begin{aligned} of_{h} \times of_{w} \times \\ N_{och} \times N_{bt}\end{aligned}$\end{tabular}} &  \multicolumn{1}{l|}{\multirow{14}{*}{\begin{tabular}[c]{@{}l@{}}\textbullet{} {\textbf{HBM3 DRAM}}\\ \underline{Read \& Write}: None, if UB $\ge$ OFMAP, \\ else multiple*\\ \textbullet{} {\textbf{Unified Buffer (UB)}}\\ \underline{Read}:\\ UB $\rightarrow$ PE core: None\\ UB $\rightarrow$ HBM3: None, if UB $\ge$ OFMAP, \\else multiple*\\ \underline{Write}:\\ PE core $\rightarrow$ UB: Once*\\ HBM3 $\rightarrow$ UB: None\\ \textbullet{} {\textbf{SRAM}}\\ Multiple* R/W for partial OFMAP \\ storage and local reuse\end{tabular}}} & \multirow{14}{*}{\begin{tabular}[c]{@{}l@{}}\textbullet{} {\textbf{HBM3 DRAM}}\\ \underline{Read \& Write}: None, if UB $\ge$ OFMAP, else multiple*\\ \textbullet{} {\textbf{Unified Buffer (UB)}}\\ \underline{Read}:\\ UB $\rightarrow$ PE core: None\\ UB $\rightarrow$ HBM3: None, if UB $\ge$ OFMAP, else multiple*\\ \underline{Write}:\\ PE core $\rightarrow$ UB: Once*\\ HBM3 $\rightarrow$ UB: None\\ \textbullet{} {\textbf{SRAM}}\\ Multiple* R/W for partial OFMAP storage and \\ local reuse\end{tabular}} & \multicolumn{1}{l|}{\multirow{14}{*}{\begin{tabular}[c]{@{}l@{}}\textbullet{} {\textbf{Dataflow}}\\ $PE\; core \rightarrow SRAM \rightarrow UB$\\ \textbullet{} {\textbf{Reuse in PE core}}\\ Multiple times for partial sum \\generation\end{tabular}}} & \multirow{14}{*}{\begin{tabular}[c]{@{}l@{}}\textbullet{} {\textbf{Dataflow}}\\ \underline{Forward:} \\ $PE\; core \rightarrow SRAM \rightarrow UB$\\  \underline{Backward:} Not Applicable (N/A) \\ \textbullet{} {\textbf{Reuse in PE core}}\\ \underline{Forward:} Multiple times for partial \\sum  generation \\  \underline{Backward:} Not Applicable (N/A) \end{tabular}} \\ 
 &  & \multicolumn{1}{l|}{} &  & \multicolumn{1}{l|}{} &  \\
 &  & \multicolumn{1}{l|}{} &  & \multicolumn{1}{l|}{} &  \\
 &  & \multicolumn{1}{l|}{} &  & \multicolumn{1}{l|}{} &  \\
 &  & \multicolumn{1}{l|}{} &  & \multicolumn{1}{l|}{} &  \\
 &  & \multicolumn{1}{l|}{} &  & \multicolumn{1}{l|}{} &  \\
 &  & \multicolumn{1}{l|}{} &  & \multicolumn{1}{l|}{} &  \\ \cline{2}
 & \multirow{7}{*}{\begin{tabular}{@{}l@{}}FC Layer:\\
$\begin{aligned} N_{bt} \times m_{fc} \end{aligned}$\end{tabular}} & \multicolumn{1}{l|}{} &  & \multicolumn{1}{l|}{} &  \\
 &  & \multicolumn{1}{l|}{} &  & \multicolumn{1}{l|}{} &  \\
 &  & \multicolumn{1}{l|}{} &  & \multicolumn{1}{l|}{} &  \\
 &  & \multicolumn{1}{l|}{} &  & \multicolumn{1}{l|}{} &  \\
 &  & \multicolumn{1}{l|}{} &  & \multicolumn{1}{l|}{} &  \\
 &  & \multicolumn{1}{l|}{} &  & \multicolumn{1}{l|}{} &  \\
%  &  & \multicolumn{1}{l|}{} &  & \multicolumn{1}{l|}{} &  \\
 &  & \multicolumn{1}{l|}{} &  & \multicolumn{1}{l|}{} &  \\ \hline
\multirow{11}{*}{\begin{tabular}[c]{@{}l@{}}Intput feature\\ map (IFMAP)\end{tabular}}& \multirow{6}{*}{\begin{tabular}{@{}l@{}}Conv. Layer: \\{$\begin{aligned}if_{h} \times if_{w} \times \\ N_{ich} \times N_{bt}\end{aligned}$} \end{tabular}}&  \multicolumn{1}{l|}{\multirow{11}{*}{\begin{tabular}[c]{@{}l@{}}\textbullet{} {\textbf{HBM3 DRAM}}\\ \underline{Read \& Write}: None, if UB $\ge$ IFMAP,\\ else multiple*\\ \textbullet{} {\textbf{Unified Buffer (UB)}}\\ \underline{Read}:\\ UB $\rightarrow$ PE core: Once*\\ UB $\rightarrow$ HBM3: None\\ \underline{Write}:\\ PE core $\rightarrow$ UB: None\\ HBM3 $\rightarrow$ UB: None, if UB $\ge$ IFMAP, \\ else multiple*\end{tabular}}} & \multirow{11}{*}{\begin{tabular}[c]{@{}l@{}}\textbullet{} {\textbf{HBM3 DRAM}}\\ \underline{Read \& Write}: None, if UB $\ge$ IFMAP, else multiple*\\ \textbullet{} {\textbf{Unified Buffer (UB)}}\\ \underline{Read}: \\ UB $\rightarrow$ PE core: At least twice* (one during forward \\ pass, one during backward pass to calculate gradient)
\\ UB $\rightarrow$ HBM3: None\\ \underline{Write}:\\ PE core $\rightarrow$ UB: None \\ HBM3 $\rightarrow$ UB: None, if UB $\ge$IFMAP, else multiple*\end{tabular}} & \multicolumn{1}{l|}{\multirow{11}{*}{\begin{tabular}[c]{@{}l@{}}\textbullet{} {\textbf{Dataflow}}\\ $UB \rightarrow PE\;core$\\ \textbullet{} {\textbf{Reuse in PE core}}\\ Multiple times for convolutional \\ reuse and over multiple filters\end{tabular}}} & \multirow{11}{*}{\begin{tabular}[c]{@{}l@{}}\textbullet{} {\textbf{Dataflow}}\\ \underline{Forward \& Backward}: \\ $UB \rightarrow PE\;core$\\ \textbullet{} {\textbf{Reuse in PE core}}\\ \underline{Forward}:\\ Multiple times for convolutional reuse \\ and over multiple filters\\ \underline{Backward}:\\ Multiple times for gradient calculation \\ of multiple filters\end{tabular}} \\ 
 &  & \multicolumn{1}{l|}{} &  & \multicolumn{1}{l|}{} &  \\
 &  & \multicolumn{1}{l|}{} &  & \multicolumn{1}{l|}{} &  \\
 &  & \multicolumn{1}{l|}{} &  & \multicolumn{1}{l|}{} &  \\
 &  & \multicolumn{1}{l|}{} &  & \multicolumn{1}{l|}{} &  \\ 
 &  & \multicolumn{1}{l|}{} &  & \multicolumn{1}{l|}{} &  \\\cline{2}
 & \multirow{6}{*}{\begin{tabular}{@{}l@{}}FC Layer: \\{$\begin{aligned} N_{bt} \times n_{fc}\end{aligned}$} \end{tabular}} & \multicolumn{1}{l|}{} &  & \multicolumn{1}{l|}{} &  \\
 &  & \multicolumn{1}{l|}{} &  & \multicolumn{1}{l|}{} &  \\
 &  & \multicolumn{1}{l|}{} &  & \multicolumn{1}{l|}{} &  \\
 &  & \multicolumn{1}{l|}{} &  & \multicolumn{1}{l|}{} &  \\
%  &  & \multicolumn{1}{l|}{} &  & \multicolumn{1}{l|}{} &  \\
%  &  & \multicolumn{1}{l|}{} &  & \multicolumn{1}{l|}{} &  \\
 &  & \multicolumn{1}{l|}{} &  & \multicolumn{1}{l|}{} &  \\ \hline
\multirow{11}{*}{Weights (W)} & \multirow{6}{*}{\begin{tabular}{@{}l@{}}Conv. Layer: \\ {$\begin{aligned}k_{h} \times k_{w} \times \\ N_{ich} \times N_{och}\end{aligned}$}\end{tabular}} & \multicolumn{1}{l|}{\multirow{11}{*}{\begin{tabular}[c]{@{}l@{}}\textbullet{} {\textbf{HBM3 DRAM}}\\ \underline{Read}: Once, if $H_A \times W_A$ $\ge$ W,\\ else multiple*\\ \underline{Write}: No write back\\ \textbullet{} {\textbf{Unified Buffer (UB)}}\\ \underline{Read \& Write}: None\end{tabular}}} & \multirow{11}{*}{\begin{tabular}[c]{@{}l@{}}\textbullet{} {\textbf{HBM3 DRAM}}\\ \underline{Read}: Once*, if UB $\ge$ W, else multiple*\\ \underline{Write}: Once* (to write the trained weights)\\ \textbullet{} {\textbf{Unified Buffer (UB)}}\\ \underline{Read}:\\ UB $\rightarrow$ PE core: At least thrice* (for forward pass,\\input gradient calculation, and trained filter weights)
\\ UB $\rightarrow$ HBM3: Once* %(to write the trained weights)
\\ \underline{Write}:\\ PE core $\rightarrow$ UB: At least once* %(to write trained \\ weights)
\\ HBM3 $\rightarrow$ UB: Once*, if UB $\ge $W, else multiple*\end{tabular}} & \multicolumn{1}{l|}{\multirow{11}{*}{\begin{tabular}[c]{@{}l@{}}\textbullet{} {\textbf{Dataflow}}\\ $HBM3 \rightarrow PE\;core/SARM$\\ \textbullet{} {\textbf{Reuse in PE core}}\\ Multiple times for convolutional reuse \\ and for different samples of minibatch\\ \\ \\ Weights are double-buffered to PE \\ core's SRAM to hide DRAM access \\ latency\end{tabular}}} & \multirow{11}{*}{\begin{tabular}[c]{@{}l@{}}\textbullet{} {\textbf{Dataflow}}\\ \underline{Forward}: \\ $HBM3 \rightarrow UB \rightarrow PE\;core$\\ \underline{Backward}: $UB \rightarrow PE\;core$\\ \textbullet{} {\textbf{Reuse in PE core}}\\ \underline{Forward}: \\ Multiple times for convolutional reuse \\ and for different samples of minibatch\\ \underline{Forward}: \\ Multiple times to calculate the gradient \\ of input/IFMAP\end{tabular}} \\
 &  & \multicolumn{1}{l|}{} &  & \multicolumn{1}{l|}{} &  \\
 &  & \multicolumn{1}{l|}{} &  & \multicolumn{1}{l|}{} &  \\
 &  & \multicolumn{1}{l|}{} &  & \multicolumn{1}{l|}{} &  \\
 &  & \multicolumn{1}{l|}{} &  & \multicolumn{1}{l|}{} &  \\
 &  & \multicolumn{1}{l|}{} &  & \multicolumn{1}{l|}{} &  \\\cline{2}
 & \multirow{6}{*}{\begin{tabular}{@{}l@{}}FC Layer: \\ {$\begin{aligned} n_{fc} \times m_{fc}\end{aligned}$}\end{tabular}} & \multicolumn{1}{l|}{} &  & \multicolumn{1}{l|}{} &  \\
 &  & \multicolumn{1}{l|}{} &  & \multicolumn{1}{l|}{} &  \\
 &  & \multicolumn{1}{l|}{} &  & \multicolumn{1}{l|}{} &  \\
 &  & \multicolumn{1}{l|}{} &  & \multicolumn{1}{l|}{} &  \\
%  &  & \multicolumn{1}{l|}{} &  & \multicolumn{1}{l|}{} &  \\
%  &  & \multicolumn{1}{l|}{} &  & \multicolumn{1}{l|}{} &  \\
%  &  & \multicolumn{1}{l|}{} &  & \multicolumn{1}{l|}{} &  \\
%  &  & \multicolumn{1}{l|}{} &  & \multicolumn{1}{l|}{} &  \\
 &  & \multicolumn{1}{l|}{} &  & \multicolumn{1}{l|}{} &  \\ \hline
\multirow{9}{*}{\begin{tabular}[c]{@{}l@{}}Upstream gradient,\\ gradient of loss\\ function, gradient \\ of activation function, \\ optimizer states\end{tabular}} & \multirow{9}{*}{\begin{tabular}[c]{@{}l@{}}Same as current \\ layer's IFMAP\\ and filter weights\end{tabular}} & \multicolumn{1}{l|}{\multirow{9}{*}{Not Applicable (N/A)}} & \multirow{9}{*}{\begin{tabular}[c]{@{}l@{}}\textbullet{} {\textbf{HBM3 DRAM}}\\ \underline{Read \& Write}: None, if UB $\ge$ all grad., else multiple*\\ \textbullet{} {\textbf{Unified Buffer (UB}}\\ \underline{Read}:\\ UB $\rightarrow$ PE core: Once*\\ UB $\rightarrow$ HBM3: None, if UB $\ge$ all grad., else multiple* \\ \underline{Write}:\\ PE $\rightarrow$ UB: Once*\\ HBM3 $\rightarrow$ UB: None, if UB $\ge$ all grad., else multiple* \end{tabular}} & \multicolumn{1}{l|}{\multirow{9}{*}{Not Applicable (N/A)}} & \multirow{9}{*}{\begin{tabular}[c]{@{}l@{}}\textbullet{} {\textbf{Dataflow}}\\ $PE\;core \rightarrow SRAM \rightarrow UB$\\ \textbullet{} {\textbf{Reuse in PE core}}\\ For previous layer's gradient calculation\end{tabular}} \\
%  &  & \multicolumn{1}{l|}{} &  & \multicolumn{1}{l|}{} &  \\
%  &  & \multicolumn{1}{l|}{} &  & \multicolumn{1}{l|}{} &  \\
%  &  & \multicolumn{1}{l|}{} &  & \multicolumn{1}{l|}{} &  \\
%  &  & \multicolumn{1}{l|}{} &  & \multicolumn{1}{l|}{} &  \\
 &  & \multicolumn{1}{l|}{} &  & \multicolumn{1}{l|}{} &  \\
 &  & \multicolumn{1}{l|}{} &  & \multicolumn{1}{l|}{} &  \\
 &  & \multicolumn{1}{l|}{} &  & \multicolumn{1}{l|}{} &  \\
 &  & \multicolumn{1}{l|}{} &  & \multicolumn{1}{l|}{} &  \\
  &  & \multicolumn{1}{l|}{} &  & \multicolumn{1}{l|}{} &  \\
   &  & \multicolumn{1}{l|}{} &  & \multicolumn{1}{l|}{} &  \\
      &  & \multicolumn{1}{l|}{} &  & \multicolumn{1}{l|}{} &  \\
 &  & \multicolumn{1}{l|}{} &  & \multicolumn{1}{l|}{} &  \\ \hline 
\multicolumn{6}{l}{\multirow{2}{*}{*Exact access count depends on UB size, batch-size, IFMAP \& OFMAP size, PE core dimension, kernel size, number of filters, and dataflow mapping}}
%\multicolumn{6}{l}{} 
\end{tabular}}
%\vspace{-8pt}
\end{table*}





\subsubsection{$BW_{RD}$ \& $BW_{WR}$ of FC layer}
\label{bw_fc_layer}
The systolic array is a widely used architecture to perform GEMM operation \cite{tpu}. %The required Global buffer bandwidth of a systolic array-based architecture depends on the array dimensions, operand matrix dimensions, and the operational intensity of the workloads. 
Depending on the array dimension ($H_{A}\times W_{A}$) and operand matrix dimension (input matrix: $K\times M$, weight matrix: $M\times N$, and output matrix: $K\times N$), we formulate required Read and Write GLB bandwidth for four different cases: (i) Weight matrix dimensions (both) are less than the systolic array dimensions ($M<H_{A}, N<W_{A}$), (ii) Height of weight matrix is less than the height of systolic array, but the width of weight matrix is larger than or equal to the width of the systolic array ($M<H_{A}, N \ge W_{A}$), (iii) Height of weight matrix is larger than or equal to the height of systolic array, but width of the weight matrix is less than the width of the systolic array  ($M \ge H_{A}, N<W_{A}$), and (iv) Both height and width of weight matrix are larger than or equal to the height and width of systolic array respectively ($M \ge H_{A}, N\ge W_{A}$). 


%(see Fig. \ref{bw_calc_cases} for visualization).

In a weight stationary dataflow, it takes $N$ clock cycles to load the weight matrix into the systolic array. Once the weights are loaded, the input matrix is streamed from left to right and the outputs are collected downward. %It takes $K$ clock cycles to load the input matrix into the systolic array. 
The input matrix's first column reaches the weight matrix's last column at $2N$ clock cycles. The last (or $K^{th}$) column of the input matrix reaches the last column of weight matrix after $2N+K-1$ clock cycles and generates the output matrix, $K\times N$. %(fig \ref{bw_calc_all}). 
Based on the above dataflow and mapping, the peak read-write bandwidth per clock cycle for different cases is summarized in Table \ref{table:bw_expression}. The expressions are shown for weight stationary dataflow. However, the above expressions hold for the output and input stationary dataflow, except the fixed and streamed matrix getting swapped.




\subsection{Memory Access Patterns}
\label{mem_access}
Our proposed memory system consists of HMB3 (off-chip memory), a large Unified Buffer (UB) with multiple SOT-MRAM banks, a smaller double-buffered SRAM, and PE reg file specific to each PE unit (Fig. \ref{fig:system_architecture}). The banks inside SOT-MRAM are optimized through a DTCO between the SOT-MRAM parameters and the workload requirements. The double-buffered SRAM holds the weights (during inference) and partial outputs. Its size is determined by the PE array size and the largest partial output size. In this subsection, we analyze the memory access patterns of CV and NLP models for the proposed memory system. The terms GLB and UB are used interchangeably in the paper to represent the large on-chip memory.



\setlength{\tabcolsep}{1pt} 
\begin{table*}[!ht]
\tiny
\caption {Architecture-and-platform-agnostic Memory access and Dataflow of Transformer-based NLP model workload during inference and training}\vspace{-10pt}
\label{Transformer_analysis}
%\resizebox{\textwidth}{!}{%
\begin{tabular}{llllll}
\hline
\multicolumn{1}{|c|}{\multirow{2}{*}{\textbf{Data Entity}}} & \multicolumn{1}{c|}{\multirow{2}{*}{\textbf{Data Dimension}}} & \multicolumn{2}{c|}{\textbf{Memory Accesses}} & \multicolumn{2}{c|}{\textbf{Dataflow \& Reuse Status}} \\ \cline{3-6} 
\multicolumn{1}{|c|}{} & \multicolumn{1}{c|}{} & \multicolumn{1}{c|}{\textbf{Inference}} & \multicolumn{1}{c|}{\textbf{Training}} & \multicolumn{1}{c|}{\textbf{Inference}} & \multicolumn{1}{c|}{\textbf{Training}} \\ \hline
\multicolumn{6}{c}{\emph{Encoder Layer}} \\ \hline
\multicolumn{1}{|l|}{\multirow{11}{*}{Input Sequence (I)}} & \multicolumn{1}{l|}{\multirow{11}{*}{$N_{sql}\times N_{bt}$}} & \multicolumn{1}{l|}{\multirow{11}{*}{\begin{tabular}[c]{@{}l@{}}\textbullet{} {\textbf{HBM3 DRAM}}\\ \underline{Read}: Once*, if UB $\ge$ I, else multiple*\\ \underline{Write}: No write back\\ \textbullet{} {\textbf{Unified Buffer (UB)}:}\\ \underline{Read}:\\ UB$\rightarrow$ PE core: Once*\\ UB$\rightarrow$ HBM3: None\\ \underline{Write}:\\ PE core $\rightarrow$ UB: None\\ HBM3 $\rightarrow$ UB: Once*, if UB $\ge$ I,\\ else multiple*\end{tabular}}} & \multicolumn{1}{l|}{\multirow{11}{*}{\begin{tabular}[c]{@{}l@{}}\textbullet{} {\textbf{HBM3 DRAM}}\\ \underline{Read}: Once*, if UB $\ge$ I, else multiple*\\ \underline{Write}: No write back\\ \textbullet{} {\textbf{Unified Buffer (UB)}:}\\ \underline{Read}:\\ UB$\rightarrow$ PE core: Twice*(one during forward pass, \\one during  backward pass to calculate gradient)\\ UB$\rightarrow$ HBM3: None\\ \underline{Write}:\\ PE core$\rightarrow$ UB: None\\ HBM3$\rightarrow$ UB: Once*, if UB $\ge$ I, else multiple\end{tabular}}} & \multicolumn{1}{l|}{\multirow{11}{*}{\begin{tabular}[c]{@{}l@{}}\textbullet{} {\textbf{Dataflow}}\\ $HBM3\rightarrow UB \rightarrow PE\; core$\\ \textbullet{} {\textbf{Reuse in PE core}}\\ No resue\end{tabular}}} & \multicolumn{1}{l|}{\multirow{11}{*}{\begin{tabular}[c]{@{}l@{}}\textbullet{} {\textbf{Dataflow}}\\ \underline{Forward}: \\ $HBM3\rightarrow UB \rightarrow PE\; core$\\ \underline{Backward}: $UB\rightarrow PE\; core$\\ \textbullet{} {\textbf{Reuse in PE core}}\\ \underline{Forward \& backward pass}: No reuse\end{tabular}}} \\
\multicolumn{1}{|c|}{} & \multicolumn{1}{c|}{} & \multicolumn{1}{l|}{} & \multicolumn{1}{l|}{} & \multicolumn{1}{l|}{} & \multicolumn{1}{l|}{} \\
\multicolumn{1}{|c|}{} & \multicolumn{1}{c|}{} & \multicolumn{1}{l|}{} & \multicolumn{1}{l|}{} & \multicolumn{1}{l|}{} & \multicolumn{1}{l|}{} \\
\multicolumn{1}{|c|}{} & \multicolumn{1}{c|}{} & \multicolumn{1}{l|}{} & \multicolumn{1}{l|}{} & \multicolumn{1}{l|}{} & \multicolumn{1}{l|}{} \\
\multicolumn{1}{|c|}{} & \multicolumn{1}{c|}{} & \multicolumn{1}{l|}{} & \multicolumn{1}{l|}{} & \multicolumn{1}{l|}{} & \multicolumn{1}{l|}{} \\
\multicolumn{1}{|c|}{} & \multicolumn{1}{c|}{} & \multicolumn{1}{l|}{} & \multicolumn{1}{l|}{} & \multicolumn{1}{l|}{} & \multicolumn{1}{l|}{} \\
\multicolumn{1}{|c|}{} & \multicolumn{1}{c|}{} & \multicolumn{1}{l|}{} & \multicolumn{1}{l|}{} & \multicolumn{1}{l|}{} & \multicolumn{1}{l|}{} \\
\multicolumn{1}{|c|}{} & \multicolumn{1}{c|}{} & \multicolumn{1}{l|}{} & \multicolumn{1}{l|}{} & \multicolumn{1}{l|}{} & \multicolumn{1}{l|}{} \\
\multicolumn{1}{|c|}{} & \multicolumn{1}{c|}{} & \multicolumn{1}{l|}{} & \multicolumn{1}{l|}{} & \multicolumn{1}{l|}{} & \multicolumn{1}{l|}{} \\
\multicolumn{1}{|c|}{} & \multicolumn{1}{c|}{} & \multicolumn{1}{l|}{} & \multicolumn{1}{l|}{} & \multicolumn{1}{l|}{} & \multicolumn{1}{l|}{} \\
% \multicolumn{1}{|c|}{} & \multicolumn{1}{c|}{} & \multicolumn{1}{l|}{} & \multicolumn{1}{l|}{} & \multicolumn{1}{l|}{} & \multicolumn{1}{l|}{} \\
% \multicolumn{1}{|c|}{} & \multicolumn{1}{c|}{} & \multicolumn{1}{l|}{} & \multicolumn{1}{l|}{} & \multicolumn{1}{l|}{} & \multicolumn{1}{l|}{} \\
\multicolumn{1}{|c|}{} & \multicolumn{1}{c|}{} & \multicolumn{1}{l|}{} & \multicolumn{1}{l|}{} & \multicolumn{1}{l|}{} & \multicolumn{1}{l|}{} \\ \hline
\multicolumn{1}{|l|}{\multirow{6}{*}{Embedded Input}} & \multicolumn{1}{l|}{\multirow{7}{*}{\begin{tabular}[c]{@{}l@{}}{$\begin{aligned}N_{sql}\times N_{em}\times\\ N_{bt}\end{aligned}$}\end{tabular}}} & \multicolumn{1}{l|}{\multirow{19}{*}{\begin{tabular}[c]{@{}l@{}}\textbullet{} {\textbf{HBM3 DRAM}}\\ \underline{Read \& Write}: None, if UB can hold full \\batch, else multiple*\\ \textbullet{} {\textbf{Unified Buffer (UB)}:}\\ \underline{Read}: \\ UB$\rightarrow$ PE core: Once*, if PE array can \\ produce complete output in one iteration, \\ else multiple*\\ UB$\rightarrow$ HBM3: None, if UB can hold full\\ batch,  else multiple*\\ \underline{Write}:\\ PE core $\rightarrow$ UB: Once*, if PE array can \\ produce complete output  in one iteration,\\ else multiple*\\ HBM3 $\rightarrow$ UB: None, if UB can hold full\\ batch,  else multiple*\\ \textbullet{} {\textbf{SRAM:}}\\ Multiple R/W for partial sums and local reuse\end{tabular}}} & \multicolumn{1}{l|}{\multirow{19}{*}{\begin{tabular}[c]{@{}l@{}}\textbullet{} {\textbf{HBM3 DRAM}}\\ \underline{Read \& Write}: None, if UB can hold full batch, \\ else multiple*\\ \textbullet{} {\textbf{Unified Buffer (UB)}:}\\ \underline{Read}: \\ UB$\rightarrow$ PE core: twice* (once at forward pass \\ \& once at backward pass for local gradient \\ calculation)\\ UB$\rightarrow$ HBM3: None, if UB can hold full batch,\\ else multiple*\\ \underline{Write}:\\ PE core $\rightarrow$ UB: Once* (during forward pass \\ for next layer \& for future access in backprop)\\ HBM3 $\rightarrow$ UB: None, if UB can hold full batch,\\ else multiple*\\ \textbullet{} {\textbf{SRAM:}}\\ Multiple R/W for partial sums and local reuse\end{tabular}}} & \multicolumn{1}{l|}{\multirow{6}{*}{\begin{tabular}[c]{@{}l@{}}\textbullet{} {\textbf{Dataflow}}\\ $PE\;core \rightarrow SRAM/UB$\\ \textbullet{} {\textbf{Reuse in PE core}}\\ As input to $Q$, $K$, and $V$ layer\\ for  next stage\end{tabular}}} & \multicolumn{1}{l|}{\multirow{6}{*}{\begin{tabular}[c]{@{}l@{}}\textbullet{} {\textbf{Dataflow}}\\ \underline{Forward}: $PE\; core\rightarrow SRAM \rightarrow UB$\\ \underline{Backward}: $UB\rightarrow PE\; core$\\ \textbullet{} {\textbf{Reuse in PE core}}\\ \underline{Forward \& Backward pass}: To calculate $Q$, $K$, \\ and $V$ layer activation and weight gradient\end{tabular}}} \\
% \multicolumn{1}{|c|}{} & \multicolumn{1}{l|}{} & \multicolumn{1}{l|}{} & \multicolumn{1}{l|}{} & \multicolumn{1}{l|}{} & \multicolumn{1}{l|}{} \\
% \multicolumn{1}{|c|}{} & \multicolumn{1}{l|}{} & \multicolumn{1}{l|}{} & \multicolumn{1}{l|}{} & \multicolumn{1}{l|}{} & \multicolumn{1}{l|}{} \\
\multicolumn{1}{|c|}{} & \multicolumn{1}{l|}{} & \multicolumn{1}{l|}{} & \multicolumn{1}{l|}{} & \multicolumn{1}{l|}{} & \multicolumn{1}{l|}{} \\
\multicolumn{1}{|c|}{} & \multicolumn{1}{l|}{} & \multicolumn{1}{l|}{} & \multicolumn{1}{l|}{} & \multicolumn{1}{l|}{} & \multicolumn{1}{l|}{} \\
\multicolumn{1}{|c|}{} & \multicolumn{1}{l|}{} & \multicolumn{1}{l|}{} & \multicolumn{1}{l|}{} & \multicolumn{1}{l|}{} & \multicolumn{1}{l|}{} \\
\multicolumn{1}{|c|}{} & \multicolumn{1}{l|}{} & \multicolumn{1}{l|}{} &
\multicolumn{1}{l|}{} & \multicolumn{1}{l|}{} & \multicolumn{1}{l|}{} \\
\multicolumn{1}{|c|}{} & \multicolumn{1}{l|}{} & \multicolumn{1}{l|}{} &
\multicolumn{1}{l|}{} & \multicolumn{1}{l|}{} & \multicolumn{1}{l|}{} \\
\cline{1-2} \cline{5-6} 
\multicolumn{1}{|l|}{\multirow{6}{*}{\begin{tabular}[c]{@{}l@{}}Query(Q), Key(K),\\ Value(V), Output of \\ Concat. Layer (Z),\\ Output of Multi-head\\ attention Layer (E)\end{tabular}}} & \multicolumn{1}{l|}{\multirow{6}{*}{\begin{tabular}[c]{@{}l@{}}$\begin{aligned} N_{sql}\times N_{em}\times \\ N_{bt} \end{aligned}$\end{tabular}}} & \multicolumn{1}{l|}{} & \multicolumn{1}{l|}{} & \multicolumn{1}{l|}{\multirow{6}{*}{\begin{tabular}[c]{@{}l@{}}\textbullet{} {\textbf{Dataflow}}\\ $PE\;core \rightarrow SRAM/UB$\\ \textbullet{} {\textbf{Reuse in PE core}}\\ For next stage activation \\ calculation\end{tabular}}} & \multicolumn{1}{l|}{\multirow{6}{*}{\begin{tabular}[c]{@{}l@{}}\textbullet{} {\textbf{Dataflow}}\\ \underline{Forward}: $PE\;core\rightarrow SRAM \rightarrow UB$\\ \underline{Backward}: $UB\rightarrow PE\;core$\\ \textbullet{} {\textbf{Reuse in PE core}}\\ \underline{Forward \& Backward pass}: For next stage \\ activation \& gradient calculation\end{tabular}}} \\
\multicolumn{1}{|l|}{} & \multicolumn{1}{l|}{} & \multicolumn{1}{l|}{} & \multicolumn{1}{l|}{} & \multicolumn{1}{l|}{} & \multicolumn{1}{l|}{} \\
\multicolumn{1}{|l|}{} & \multicolumn{1}{l|}{} & \multicolumn{1}{l|}{} & \multicolumn{1}{l|}{} & \multicolumn{1}{l|}{} & \multicolumn{1}{l|}{} \\
\multicolumn{1}{|l|}{} & \multicolumn{1}{l|}{} & \multicolumn{1}{l|}{} & \multicolumn{1}{l|}{} & \multicolumn{1}{l|}{} & \multicolumn{1}{l|}{} \\
\multicolumn{1}{|l|}{} & \multicolumn{1}{l|}{} & \multicolumn{1}{l|}{} & \multicolumn{1}{l|}{} & \multicolumn{1}{l|}{} & \multicolumn{1}{l|}{} \\
\multicolumn{1}{|l|}{} & \multicolumn{1}{l|}{} & \multicolumn{1}{l|}{} & \multicolumn{1}{l|}{} & \multicolumn{1}{l|}{} & \multicolumn{1}{l|}{} \\  \cline{1-2} \cline{5-6}
\multicolumn{1}{|l|}{\multirow{2}{*}{Attention Filter (AF)}} & \multicolumn{1}{l|}{\multirow{2}{*}{\begin{tabular}[c]{@{}l@{}}$\begin{aligned} N_{sql}\times N_{sql}\times \\ N_{bt}\end{aligned}$ \end{tabular} }} & \multicolumn{1}{l|}{} & \multicolumn{1}{l|}{} & \multicolumn{1}{l|}{{\multirow{6}{*}{\begin{tabular}[c]{@{}l@{}}\textbullet{} {\textbf{Dataflow}}\\ $PE\;core \rightarrow SRAM/UB$\\ \textbullet{} {\textbf{Reuse in PE core}}\\ As input to $K$ \& $V$ of Encoder-\\Decoder  SelfAttention Layer\end{tabular}}} } & \multicolumn{1}{l|}{\multirow{5}{*}{\begin{tabular}[c]{@{}l@{}}\textbullet{} {\textbf{Dataflow}}\\ \underline{Forward}: $PE\; core\rightarrow SRAM \rightarrow UB$\\ \underline{Backward}: $UB\rightarrow PE\; core$\\ \textbullet{} {\textbf{Reuse in PE core}}\\ \underline{Forward \& Backward pass}: To find $K$, $V$  and \\ weight gradient of Encoder-Decoder Self-\\Attention Layer\end{tabular}}} \\
\multicolumn{1}{|l|}{} & \multicolumn{1}{l|}{} & \multicolumn{1}{l|}{} & \multicolumn{1}{l|}{} & \multicolumn{1}{l|}{} & \multicolumn{1}{l|}{} \\ 
\multicolumn{1}{|l|}{\multirow{5}{*}{\begin{tabular}[c]{@{}l@{}}Final Encoder \\ Output Layer\end{tabular}}} &  \multicolumn{1}{l|}{\multirow{5}{*}{\begin{tabular}[c]{@{}l@{}} $\begin{aligned}  N_{sql}\times N_{em}\times \\ N_{bt}\end{aligned}$\end{tabular}}} & \multicolumn{1}{l|}{} & \multicolumn{1}{l|}{} & \multicolumn{1}{l|}{\multirow{5}{*}{}} & \multicolumn{1}{l|}{} \\ \cline{1-2}
\multicolumn{1}{|l|}{} & \multicolumn{1}{l|}{} & \multicolumn{1}{l|}{} & \multicolumn{1}{l|}{} & \multicolumn{1}{l|}{} & \multicolumn{1}{l|}{} \\
\multicolumn{1}{|l|}{} & \multicolumn{1}{l|}{} & \multicolumn{1}{l|}{} & \multicolumn{1}{l|}{} & \multicolumn{1}{l|}{} & \multicolumn{1}{l|}{} \\
\multicolumn{1}{|l|}{} & \multicolumn{1}{l|}{} & \multicolumn{1}{l|}{} & \multicolumn{1}{l|}{} & \multicolumn{1}{l|}{} & \multicolumn{1}{l|}{} \\
% \multicolumn{1}{|l|}{} & \multicolumn{1}{l|}{} & \multicolumn{1}{l|}{} & \multicolumn{1}{l|}{} & \multicolumn{1}{l|}{} & \multicolumn{1}{l|}{} \\
% \multicolumn{1}{|l|}{} & \multicolumn{1}{l|}{} & \multicolumn{1}{l|}{} & \multicolumn{1}{l|}{} & \multicolumn{1}{l|}{} & \multicolumn{1}{l|}{} \\
% \multicolumn{1}{|l|}{} & \multicolumn{1}{l|}{} & \multicolumn{1}{l|}{} & \multicolumn{1}{l|}{} & \multicolumn{1}{l|}{} & \multicolumn{1}{l|}{} \\
\multicolumn{1}{|l|}{} & \multicolumn{1}{l|}{} & \multicolumn{1}{l|}{} & \multicolumn{1}{l|}{} & \multicolumn{1}{l|}{} & \multicolumn{1}{l|}{} \\ \hline
\multicolumn{1}{|l|}{\multirow{5}{*}{\begin{tabular}[c]{@{}l@{}}Upstream Gradient\\ ($\delta_{x}$)\\ x = 1,2,3,...,N\end{tabular}}} & \multicolumn{1}{l|}{\multirow{15}{*}{\begin{tabular}[c]{@{}l@{}}Same as current \\ layer activation\end{tabular}}}  & \multicolumn{1}{l|}{\multirow{15}{*}{Not Applicable (N/A)}} & \multicolumn{1}{l|}{\multirow{15}{*}{\begin{tabular}[c]{@{}l@{}}\textbullet{} {\textbf{HBM3 DRAM}}\\ \underline{Read \& Write}: None, if UB $\ge$ all grad.,\\ else multiple*\\ \textbullet{} {\textbf{Unified Buffer (UB)}:}\\ \underline{Read}: \\ UB$\rightarrow$ PE core: Once*, if PE core $\ge$ all grad.,\\ else multiple*\\ UB$\rightarrow$ HBM3: None, if UB $\ge$ all grad.,\\   else multiple*\\ \underline{Write}:\\ PE core $\rightarrow$ UB: Once* (For previous layer's $\delta$ )\\ HBM3 $\rightarrow$ UB: None, if UB $\ge$ all grad.,\\  else multiple*\\ \textbullet{} {\textbf{SRAM:}}\\ Multiple R/W for partial sums and local reuse  \end{tabular}}} & \multicolumn{1}{l|}{\multirow{15}{*}{Not Applicable (N/A)}} & \multicolumn{1}{l|}{\multirow{15}{*}{\begin{tabular}[c]{@{}l@{}}\textbullet{} {\textbf{Dataflow}}\\ $PE\; core\rightarrow SRAM \rightarrow UB$\\ \textbullet{} {\textbf{Reuse in PE core}}\\ For previous layer's $\delta$ calculation\end{tabular}}} \\
\multicolumn{1}{|l|}{} & \multicolumn{1}{l|}{} & \multicolumn{1}{l|}{} & \multicolumn{1}{l|}{} & \multicolumn{1}{l|}{} & \multicolumn{1}{l|}{} \\
\multicolumn{1}{|l|}{} & \multicolumn{1}{l|}{} & \multicolumn{1}{l|}{} & \multicolumn{1}{l|}{} & \multicolumn{1}{l|}{} & \multicolumn{1}{l|}{} \\
\multicolumn{1}{|l|}{} & \multicolumn{1}{l|}{} & \multicolumn{1}{l|}{} & \multicolumn{1}{l|}{} & \multicolumn{1}{l|}{} & \multicolumn{1}{l|}{} \\
\multicolumn{1}{|l|}{} & \multicolumn{1}{l|}{} & \multicolumn{1}{l|}{} & \multicolumn{1}{l|}{} &
\multicolumn{1}{l|}{} & \multicolumn{1}{l|}{} \\ \cline{1-1}
\multicolumn{1}{|l|}{\multirow{5}{*}{\begin{tabular}[c]{@{}l@{}}Gradient of Loss \\ ($\Delta$L) \\ (only for final \\ output layer)\end{tabular}}} & \multicolumn{1}{l|}{} & \multicolumn{1}{l|}{} & \multicolumn{1}{l|}{} & \multicolumn{1}{l|}{} & \multicolumn{1}{l|}{} \\
\multicolumn{1}{|l|}{} & \multicolumn{1}{l|}{} & \multicolumn{1}{l|}{} & \multicolumn{1}{l|}{} & \multicolumn{1}{l|}{} & \multicolumn{1}{l|}{} \\
\multicolumn{1}{|l|}{} & \multicolumn{1}{l|}{} & \multicolumn{1}{l|}{} & \multicolumn{1}{l|}{} & \multicolumn{1}{l|}{} & \multicolumn{1}{l|}{} \\
\multicolumn{1}{|l|}{} & \multicolumn{1}{l|}{} & \multicolumn{1}{l|}{} & \multicolumn{1}{l|}{} & \multicolumn{1}{l|}{} & \multicolumn{1}{l|}{} \\
\multicolumn{1}{|l|}{} & \multicolumn{1}{l|}{} & \multicolumn{1}{l|}{} & \multicolumn{1}{l|}{} & \multicolumn{1}{l|}{} & \multicolumn{1}{l|}{} \\
\multicolumn{1}{|l|}{} & \multicolumn{1}{l|}{} & \multicolumn{1}{l|}{} & \multicolumn{1}{l|}{} & \multicolumn{1}{l|}{} & \multicolumn{1}{l|}{} \\ \cline{1-1}
\multicolumn{1}{|l|}{\multirow{5}{*}{\begin{tabular}[c]{@{}l@{}}Gradient of \\ activations\end{tabular}}} & \multicolumn{1}{l|}{} & \multicolumn{1}{l|}{} & \multicolumn{1}{l|}{} & \multicolumn{1}{l|}{} & \multicolumn{1}{l|}{} \\
\multicolumn{1}{|l|}{} & \multicolumn{1}{l|}{} & \multicolumn{1}{l|}{} & \multicolumn{1}{l|}{} & \multicolumn{1}{l|}{} & \multicolumn{1}{l|}{} \\
% \multicolumn{1}{|l|}{} & \multicolumn{1}{l|}{} & \multicolumn{1}{l|}{} & \multicolumn{1}{l|}{} & \multicolumn{1}{l|}{} & \multicolumn{1}{l|}{} \\
% \multicolumn{1}{|l|}{} & \multicolumn{1}{l|}{} & \multicolumn{1}{l|}{} & \multicolumn{1}{l|}{} & \multicolumn{1}{l|}{} & \multicolumn{1}{l|}{} \\
\multicolumn{1}{|l|}{} & \multicolumn{1}{l|}{} & \multicolumn{1}{l|}{} & \multicolumn{1}{l|}{} & \multicolumn{1}{l|}{} & \multicolumn{1}{l|}{} \\
\multicolumn{1}{|l|}{} & \multicolumn{1}{l|}{} & \multicolumn{1}{l|}{} & \multicolumn{1}{l|}{} &
\multicolumn{1}{l|}{} & \multicolumn{1}{l|}{} \\ \hline

\multicolumn{1}{|l|}{\multirow{2}{*}{\begin{tabular}[c]{@{}l@{}}Embedding \\ Weights ($W^{em}$)\end{tabular}}} & \multicolumn{1}{l|}{\multirow{2}{*}{$N_{vocab}\times N_{em}$}} & \multicolumn{1}{l|}{\multirow{18}{*}{\begin{tabular}[c]{@{}l@{}}\textbullet{} {\textbf{HBM3 DRAM}}\\ \underline{Read}: \\ HBM3 $\rightarrow$ PE core: Once*, if PE array can\\ hold full weight matrix, else multiple*\\ \underline{Write}:\\ PE core $\rightarrow$ HBM3: No write back\\ \textbullet{} {\textbf{Unified Buffer (UB)}}\\ No read and write\\ \end{tabular}}} & \multicolumn{1}{l|}{\multirow{18}{*}{\begin{tabular}[c]{@{}l@{}}\textbullet{} {\textbf{HBM3 DRAM}}\\ \underline{Read}: Once*, if UB $\ge$ W, else multiple*\\ \underline{Write}: At least once* (to write back updated \\ weights after training)\\ \textbullet{} {\textbf{Unified Buffer (UB)}:}\\ \underline{Read}: \\ UB$\rightarrow$ PE core: at least thrice* (one at forward \\ pass, two at backward pass to calculate $\Delta$W \\ and updated W)\\ UB$\rightarrow$ HBM3: At least once* (to write back \\ updated weights after training)\\ \underline{Write}:\\ PE core $\rightarrow$ UB: At least once* (to update \\ weights)\\ HBM3 $\rightarrow$ UB: Once*, if UB $\ge$ W, else multiple*\\ \textbullet{} {\textbf{SRAM:}}\\ Multiple R/W for partial W, $\Delta$ W \\per minibatch iteration to facilitate reuse\end{tabular}}} & \multicolumn{1}{l|}{\multirow{18}{*}{\begin{tabular}[c]{@{}l@{}}\textbullet{} {\textbf{Dataflow}}\\ $HBM3 \rightarrow PE\;core /SRAM$\\ \textbullet{} {\textbf{Reuse in PE core}}\\ Multiple times for each sample \\per minibatch \\ \\ Weights are doubled buffered to \\ PE core's SRAM to hide DRAM \\ access latency.\end{tabular}}} & \multicolumn{1}{l|}{\multirow{18}{*}{\begin{tabular}[c]{@{}l@{}}\textbullet{} {\textbf{Dataflow}}\\ \underline{Forward}: $HBM3 \rightarrow UB \rightarrow PE\; core$\\ \underline{Backward}: $UB \rightarrow PE\; core$\\ \textbullet{} {\textbf{Reuse in PE core}}\\ \underline{Forward}: Multiple times for each sample\\ per minibatch\\ \underline{Backward}: For next stage gradient \\ calculation\end{tabular}}} \\
%\multicolumn{1}{|l|}{} & \multicolumn{1}{l|}{} & \multicolumn{1}{l|}{} & \multicolumn{1}{l|}{} & \multicolumn{1}{l|}{} & \multicolumn{1}{l|}{} \\
\multicolumn{1}{|l|}{} & \multicolumn{1}{l|}{} & \multicolumn{1}{l|}{} & \multicolumn{1}{l|}{} & \multicolumn{1}{l|}{} & \multicolumn{1}{l|}{} \\ \cline{1-2}
\multicolumn{1}{|l|}{\multirow{2}{*}{\begin{tabular}[c]{@{}l@{}}{Query Weights ($W^{Q}$)}\end{tabular}}} & \multicolumn{1}{l|}{\multirow{2}{*}{$(N_{em} \times d_{q})*h$}} & \multicolumn{1}{l|}{} & \multicolumn{1}{l|}{} & \multicolumn{1}{l|}{} & \multicolumn{1}{l|}{}\\
\multicolumn{1}{|l|}{} & \multicolumn{1}{l|}{} & \multicolumn{1}{l|}{} & \multicolumn{1}{l|}{} & \multicolumn{1}{l|}{} & \multicolumn{1}{l|}{} \\ \cline{1-2}
\multicolumn{1}{|l|}{\multirow{2}{*}{\begin{tabular}{@{}l@{}}{Key Weights ($W^{K}$)}\end{tabular}}} & \multicolumn{1}{l|}{\multirow{2}{*}{$(N_{em} \times d_{k})*h$}} & \multicolumn{1}{l|}{} & \multicolumn{1}{l|}{} & \multicolumn{1}{l|}{} & \multicolumn{1}{l|}{}\\ 
\multicolumn{1}{|l|}{} & \multicolumn{1}{l|}{} & \multicolumn{1}{l|}{} & \multicolumn{1}{l|}{} & \multicolumn{1}{l|}{} & \multicolumn{1}{l|}{} \\ \cline{1-2}
\multicolumn{1}{|l|}{\multirow{2}{*}{\begin{tabular}{@{}l@{}}{Value Weights ($W^{V}$)}\end{tabular}}} & \multicolumn{1}{l|}{\multirow{2}{*}{$(N_{em} \times d_{v})*h$}} & \multicolumn{1}{l|}{} & \multicolumn{1}{l|}{} & \multicolumn{1}{l|}{} & \multicolumn{1}{l|}{} \\ 
\multicolumn{1}{|l|}{} & \multicolumn{1}{l|}{} & \multicolumn{1}{l|}{} & \multicolumn{1}{l|}{} & \multicolumn{1}{l|}{} & \multicolumn{1}{l|}{} \\ \cline{1-2}
\multicolumn{1}{|l|}{\multirow{2}{*}{\begin{tabular}{@{}l@{}}{Concat. Weights ($W^{O}$)}\end{tabular}}} & \multicolumn{1}{l|}{\multirow{2}{*}{$N_{em} \times N_{em}$}} & \multicolumn{1}{l|}{} & \multicolumn{1}{l|}{} & \multicolumn{1}{l|}{} & \multicolumn{1}{l|}{} \\
\multicolumn{1}{|l|}{} & \multicolumn{1}{l|}{} & \multicolumn{1}{l|}{} & \multicolumn{1}{l|}{} &  \multicolumn{1}{l|}{} & \multicolumn{1}{l|}{} \\ \cline{1-2}
\multicolumn{1}{|l|}{\multirow{2}{*}{\begin{tabular}[c]{@{}l@{}}FFN FC1 Weights \\($W^{FC1}$)\end{tabular}}} & \multicolumn{1}{l|}{\multirow{2}{*}{$N_{em} \times d_{ff}$}} & \multicolumn{1}{l|}{} & \multicolumn{1}{l|}{} & \multicolumn{1}{l|}{} & \multicolumn{1}{l|}{} \\ 
%\multicolumn{1}{|l|}{} & \multicolumn{1}{l|}{} & \multicolumn{1}{l|}{} & \multicolumn{1}{l|}{} & \multicolumn{1}{l|}{} & \multicolumn{1}{l|}{} \\
\multicolumn{1}{|l|}{} & \multicolumn{1}{l|}{} & \multicolumn{1}{l|}{} & \multicolumn{1}{l|}{} & \multicolumn{1}{l|}{} & \multicolumn{1}{l|}{} \\ \cline{1-2}
\multicolumn{1}{|l|}{\multirow{3}{*}{\begin{tabular}[c]{@{}l@{}}FFN FC2 Weights \\($W^{FC2}$)\end{tabular}}} & \multicolumn{1}{l|}{\multirow{3}{*}{$d_{ff} \times N_{em}$}} & \multicolumn{1}{l|}{} & \multicolumn{1}{l|}{} & \multicolumn{1}{l|}{} & \multicolumn{1}{l|}{} \\
\multicolumn{1}{|l|}{} & \multicolumn{1}{l|}{} & \multicolumn{1}{l|}{} & \multicolumn{1}{l|}{} & \multicolumn{1}{l|}{} & \multicolumn{1}{l|}{} \\
\multicolumn{1}{|l|}{} & \multicolumn{1}{l|}{} & \multicolumn{1}{l|}{} & \multicolumn{1}{l|}{} & \multicolumn{1}{l|}{} & \multicolumn{1}{l|}{} \\ \cline{1-2}
\multicolumn{1}{|l|}{\multirow{3}{*}{\begin{tabular}[c]{@{}l@{}}Output classifier\\ weights ($W^{cl}$)\end{tabular}}} & \multicolumn{1}{l|}{\multirow{3}{*}{\begin{tabular}{@{}l@{}}{$\begin{aligned}(N_{sql}*N_{em})\times \\ N_{vocab}\end{aligned}$}\end{tabular}}} & \multicolumn{1}{l|}{} & \multicolumn{1}{l|}{} & \multicolumn{1}{l|}{} & \multicolumn{1}{l|}{} \\
\multicolumn{1}{|l|}{} & \multicolumn{1}{l|}{} & \multicolumn{1}{l|}{} & \multicolumn{1}{l|}{} & \multicolumn{1}{l|}{} & \multicolumn{1}{l|}{} \\
\multicolumn{1}{|l|}{} & \multicolumn{1}{l|}{} & \multicolumn{1}{l|}{} & \multicolumn{1}{l|}{} & \multicolumn{1}{l|}{} & \multicolumn{1}{l|}{} \\ \hline
\multicolumn{1}{|l|}{\multirow{9}{*}{Optimizer states}} & \multicolumn{1}{l|}{\multirow{9}{*}{}} & \multicolumn{1}{l|}{\multirow{9}{*}{Not Applicable (N/A)}} & \multicolumn{1}{l|}{\multirow{9}{*}{\begin{tabular}[c]{@{}l@{}}\textbullet{} {\textbf{HBM3 DRAM}}\\ \underline{Read \& Write}: None\\ \textbullet{} {\textbf{Unified Buffer (UB)}}\\ \underline{Read}:\\ $UB \rightarrow PE\;core$: Once* \\ \underline{Write}:\\ $PE\;core \rightarrow UB$: Once*\\ \textbullet{} {\textbf{SRAM}}\\ \underline{Read \& Write}: None\end{tabular}}} & \multicolumn{1}{l|}{\multirow{9}{*}{Not Applicable (N/A)}} & \multicolumn{1}{l|}{\multirow{9}{*}{\begin{tabular}[c]{@{}l@{}}\textbullet{} {\textbf{Dataflow}}\\ $UB \rightarrow PE\; core$\\ \textbullet{} {\textbf{Reuse in PE core}}\\ No reuse\end{tabular}}} \\
\multicolumn{1}{|l|}{} & \multicolumn{1}{l|}{} & \multicolumn{1}{l|}{} & \multicolumn{1}{l|}{} & \multicolumn{1}{l|}{} & \multicolumn{1}{l|}{} \\
\multicolumn{1}{|l|}{} & \multicolumn{1}{l|}{} & \multicolumn{1}{l|}{} & \multicolumn{1}{l|}{} & \multicolumn{1}{l|}{} & \multicolumn{1}{l|}{} \\
\multicolumn{1}{|l|}{} & \multicolumn{1}{l|}{} & \multicolumn{1}{l|}{} & \multicolumn{1}{l|}{} & \multicolumn{1}{l|}{} & \multicolumn{1}{l|}{} \\
\multicolumn{1}{|l|}{} & \multicolumn{1}{l|}{} & \multicolumn{1}{l|}{} & \multicolumn{1}{l|}{} & \multicolumn{1}{l|}{} & \multicolumn{1}{l|}{} \\
\multicolumn{1}{|l|}{} & \multicolumn{1}{l|}{} & \multicolumn{1}{l|}{} & \multicolumn{1}{l|}{} & \multicolumn{1}{l|}{} & \multicolumn{1}{l|}{} \\
\multicolumn{1}{|l|}{} & \multicolumn{1}{l|}{} & \multicolumn{1}{l|}{} & \multicolumn{1}{l|}{} & \multicolumn{1}{l|}{} & \multicolumn{1}{l|}{} \\
\multicolumn{1}{|l|}{} & \multicolumn{1}{l|}{} & \multicolumn{1}{l|}{} & \multicolumn{1}{l|}{} & \multicolumn{1}{l|}{} & \multicolumn{1}{l|}{} \\
\multicolumn{1}{|l|}{} & \multicolumn{1}{l|}{} & \multicolumn{1}{l|}{} & \multicolumn{1}{l|}{} & \multicolumn{1}{l|}{} & \multicolumn{1}{l|}{} \\
% \multicolumn{1}{|l|}{} & \multicolumn{1}{l|}{} & \multicolumn{1}{l|}{} & \multicolumn{1}{l|}{} & \multicolumn{1}{l|}{} & \multicolumn{1}{l|}{} \\ 
% \multicolumn{1}{|l|}{} & \multicolumn{1}{l|}{} & \multicolumn{1}{l|}{} & \multicolumn{1}{l|}{} & \multicolumn{1}{l|}{} & \multicolumn{1}{l|}{} \\
\hline
\multicolumn{6}{c}{\emph{Decoder Layer}} \\ \hline
\multicolumn{6}{|l|}{\multirow{3}{*}{\begin{tabular}[c]{@{}l@{}}   The decoder layer has two multi-head attention layers; (i) Decoder self-attention layer and (ii) Encoder-Decoder attention layer. Each of them has similar data entities ($Q,\;K,\;V, \;Z,\;E,\;AF,\;W^{Q},\;W^{K},\; W^{V},\; W^{O}$) \\ and \; $W^{FC1},\;W^{FC2}$ with same dimension. However, the sequence length might vary from that of the Encoder's one.\end{tabular}}} \\
\multicolumn{6}{|l|}{} \\
\multicolumn{6}{|l|}{} \\ \hline
\multicolumn{6}{l}{\multirow{2}{*}{*Exact access count depends on UB size, weight matrix, PE core dimension, and dataflow mapping}}\\
\end{tabular}
%\vspace{-10pt}
\end{table*}




\subsubsection{CV models}

%Unlike the conventional GPU-based on-chip memory hierarchy where there are multiple levels of cache memory (such as Register file, L0, L1 instruction cache, L1, L2 data cache, etc.) we propose a memory system of large unified buffer composed of multiple banks of SOT-MRAM, a smaller SRAM buffer and PE reg file specific to each PE unit (fig. \ref{fig:system_architecture}). The banks inside SOT-MRAM are again optimized through a DTCO between the SOT-MRAM parameters and the workload requirements. 
%Some operand data are directly loaded from (cite nvidia ample gpu white paper)


Deep Residual Networks, having convolutional layers at their core, dominate the Computer Vision domain. The input images are convolved with the layer-specific filter weights to produce the output feature map (\emph{OFMAP}). The \emph{OFMAP} goes through the pooling and normalization layers to act as input (\emph{IFMAP}) to the next layer. The linear and softmax layer at the end finally recognizes the image (Fig. \ref{fig:conv_op}). The dimensionality of the key components of such models depends on the model architecture (Table \ref{conv_analysis} column 1 \& 2). Depending on the UB size, activation \& weight size, the minimum memory accesses requirement of each layer’s activations and weights in different levels of memory hierarchy during inference and training are listed in col. 3 \& 4 of Table \ref{conv_analysis}, respectively. During inference, the read-only weights are directly loaded from HBM3 to the register file of each PE unit, bypassing the UB. However, we use double-buffered SRAM to hide the off-chip access latency. While the array is computing with loaded weights, the next set of weights is temporarily written to the SRAM buffer to hide the off-chip access latency behind the PE array computation latency. \emph{Read: Once*} under heading HBM3 DRAM means the whole chunk of data, for example, \emph{Initial Input(I)}, is read once from DRAM given that the Unified Buffer (UB) is large enough to hold all samples in the minibatch. $UB\; \rightarrow PE\;core$ means the entity is read from $UB$ and written into or operated inside \emph{PE core}. We also provide a high-level dataflow and reuse scope for each entity during inference and training (col. 5 \& 6 of Table \ref{conv_analysis}, respectively). To illustrate, the last column of the first row indicates input images' dataflow and reuse status during training. During the forward pass, input images are read from HBM3, written to UB, and read from UB to be operated inside PEs core. For weight gradient calculation, the inputs are read from UB to PE core during backpropagation, assuming that the UB is large enough to hold the input images along with the generated ofmap of the current layer, thus avoiding the DRAM accesses during backward pass. \emph{Reuse in PE core} represents at forward pass, the inputs can be reused multiple times for convolutional and filter reuse. It can also be reused multiple times during backpropagation to calculate the gradients of different filters. Row 5 of Table \ref{conv_analysis} shows the training-specific ephemeral entities with their dimension, memory access count, and reuse status.
% such as Upstream gradients, the gradient of loss and activation functions, and optimizer states with their dimension, memory access count, and reuse status. 
The training workflow is complicated and requires many more memory accesses (both off-chip and on-chip) compared to inference. For example, to calculate the weight gradients of Layer 1, it requires the current layer's activation gradient $\frac{da_1}{dz_1}$, input ($a_0$), next layer's weight ($W_{2}$) and the upstream gradient from Layer 2 ($\delta_{1}$) (as shown in  Fig. \ref{fig:comp_graph}). 



% We summarize the memory access requirements and the dataflow of the CNN-based DNN workload for both inference and training in the rest columns of Table  \ref{conv_analysis}. We follow the conventions similar to Table \ref{Transformer_analysis}. The analysis is based on the most optimized row-stationary dataflow. However, it is also applicable to all other dataflow, such as weight stationary, input, and output stationary dataflow as we generalized the analysis in the workload level. 
% %In a row stationary dataflow, the kernel rows are loaded into the PE array and kept stationary, ifmaps are shifted by stride size, and partial sums are accumulated vertically. 
% The last row of the table contains the memory access and the dataflow requirements for the training-only entities, such as upstream gradients, gradient of loss function, and gradient of activation functions.
% To start off, the input images are loaded from DRAM to UB and the weights are directly loaded from DRAM to PE array. 



\begin{figure*}[ht]
    \centering
    \includegraphics[scale=0.7]{Fig_4_Transformer.pdf}
    %\vspace{-8pt}
    \caption{Transformer model workflow breakdown}
    %\vspace{-15pt}
    \label{fig:transformer}
\end{figure*}

%%%%%%%%%%%%%%%%%%%%%

\subsubsection{Natural Language Processing (NLP)} 




We analyze the state-of-the-art Transformer model \cite{vaswani2017attention} as the representative of the NLP domain. The input sequence propagates through the embedding layer and different sublayers of the encoder stacks to extract different linguistic features and inter-token dependency of the input sequence. The decoder stacks then generate the output sequence by taking the encoded input sequence from the encoder stack and the output sequence generated by itself in the previous timesteps (Fig \ref{fig:transformer}). The input sequence multiplied by different layer weights takes different activation names and shapes (Table \ref{Transformer_analysis} col. 1 \& 2) throughout the model operation. We summarize the minimum memory accesses requirement, dataflow and reuse scopes of each layer’s (sub-layer) activations and weights in different levels of memory hierarchy during inference and training are in Table \ref{Transformer_analysis}. We follow the same convention as Table \ref{conv_analysis}. We illustrate one cell of Table \ref{Transformer_analysis} to help the readers naviagate the table. The last column of the second row indicates the dataflow and reuse status of Embedded input during training. During forward pass, the embedded inputs are generated inside the PE core, partial sums are accumulated in the SRAM, and finally, the output is written to Unified buffer (UB). For weight gradient calculation, the embedded inputs are read from UB to PE core during backpropagation. \emph{Reuse inside PE core} represents at forward pass, the embedded input can be reused thrice as input to Key, Query, and Value linear layer. It can also be reused thrice during backpropagation to calculate the weight gradient of the Key, Query, and Value linear layer. 



We consider the memory access counts in terms of how many times the data entities operate throughout the workflow. However, the exact access count depends on the model architecture, hyperparameters, hardware platforms, dataflow \& mapping such as IFMAP, OFMAP, word embedding, input sequence length, weight size, batch size, UB size, SRAM size, PE core dimension, row-stationary dataflow, etc. 
%The analysis is based on the row-stationary dataflow. However, it is also applicable to all other dataflows, such as weight stationary, input, and output stationary dataflow as we generalized the analysis in the workload level. 

% \vspace{-5pt}
% \vspace{-5pt}
\section{DTCO of SOT-MRAM}
\label{dtco_MRAM}
To ensure  overall system performance for AI workloads, the memory system should have large on-chip memory to avoid frequent DRAM accesses, and the on-chip memory should have high bandwidth to prevent the system from being memory-bound while being energy efficient. In this section, we perform a DTCO of SOT-MRAM in bit-cell level based on the workload profiling done in section \ref{workload_profiling}.

\subsection{Optimizing critical switching current $I_{c}$}
%Writing to any MRAM involves switching the magnetic orientation of the free layer of the MTJ stack. 
In SOT-MRAM, the magnetic orientation of the free layer is switched by Spin-Orbit Torque induced by spin Hall and interfacial effects between the channel (i.e., SOT layer) and free layer (FL) of MTJ. An in-plane charge current is flown through the channel to generate a spin current that exerts a spin torque on the free layer, which rotates the free layer's magnetic orientation. The critical current density required to switch the magnetic orientation of FL is expressed as \cite{lee2013threshold}

\begin{equation}
    j_{c} = \frac{2e\mu_0 M_{s,FL}t_{FL}}{\hbar\theta_{SH}} (\frac{H_{k,eff}}{2} -\frac{H_x}{\sqrt{2}})
    \label{Ic3}
\end{equation}
Where $H_{k,eff}$ is the effective anisotropy field, $H_{x}$ is the applied field, $M_{s,FL}$ is the saturation magnetization of free layer, and $t_{FL}$ is its thickness. Our interest is in lowering the switching current to achieve low write energy.
Here, the free layer thickness $t_{FL}$ and spin Hall efficiency $\theta_{SH}$ act as a control knob for critical switching current. $\theta_{SH}$ is a material-specific parameter and its higher value is expected to reduce the switching current. The typical value of $\theta_{SH}$ in heavy metal alloys ranges between 0.1 to 0.5 \cite{manchon2019current}. However, recent topological insulators as SOT layer can have a very large $\theta_{SH}$. \cite{khang2018conductive} demonstrated $\theta_{SH} = 152$ with $BiSb$ thin films.



\subsection{Optimizing read-write pulse width}
\subsubsection{Read pulse width}
The reading of SOT-MRAM involves sensing the resistance of the MTJ. A small amount of current is passed through the MTJ stack and the voltage across the stack $V_{data+}$ or $V_{data-}$ is compared against a reference voltage $V_{ref} = \frac{1}{2}(V_{data+}+ V_{data-})$ to read out the stored bit. The read Sensing Margin $SM = |V_{ref} - V_{data}|$ is typically very small. Sensing and amplifying this small difference requires a strong and complex Sense Amplifier  that contributes to most of the read latency and energy. The SM is determined by the Tunnel Magneto Resistance ratio ($TMR\;ratio = \frac{R_{AP}-R_{P}}{R_{P}}$) of MTJ. A higher TMR ratio produces a larger SM by making $V_{data+}$ higher and $V_{data-}$ lower. Thus the TMR ratio is inversely proportional to the read latency\cite{3T-2SOT}. The higher the TMR window, the higher the read speed and the less effort required on the periphery. The typical range of the TMR ratio is between $100$ to $300\%$. The TMR is tunable by oxide thickness \cite{tsunekawa2005giant} as shown in Fig. \ref{fig:tmr_vs_oxide_rd_ltncy} (a). In SOT-MRAM, we can increase the oxide thickness, thanks to the decoupled read-write path of SOT-MRAM, to achieve a high TMR and increase the read speed without worrying about the large incubation time \cite{wang2013low}. 

%\vspace{-10pt}
\begin{table}[ht]
\setlength{\tabcolsep}{3pt}
\centering
\caption{DTCO control parameters \& their impact on Power, Performance and  Area (PPA)}
\label{dtco_table}
% \vspace{-10pt}
\begin{tabular}{|l|l|}
\hline
\textbf{DTCO Parameters}               & \textbf{Impact on PPA}                                                                             \\ \hline
Spin Hall angle $\theta_{SH}$ & $\theta_{SH} \uparrow$, $j_{c} \downarrow$,  Switching energy $\downarrow$                \\ \hline
Free layer thickness $t_{FL}$ & $t_{FL} \downarrow$, $j_{c} \downarrow$, Switching energy $\downarrow$, Area $\downarrow$           \\ \hline
\multirow{2}{*}{\begin{tabular}{@{}l@{}}{SOT layer dimension} \\ {$A_{SOT}$}\end{tabular}} & 
\multirow{2}{*}{\begin{tabular}{@{}l@{}}{$A_{SOT} \downarrow$, $\tau_{p} \downarrow$, Area $\downarrow$, Write Bandwidth $\uparrow$}\end{tabular}}\\
 & \\ \hline
Oxide thickness $t_{MgO}$ & $t_{MgO} \uparrow$, TMR $ \uparrow$, Read Bandwidth $ \uparrow$\\ \hline
\end{tabular}
%\vspace{-10pt}
\end{table}

\subsubsection{Write pulse width $\tau_{p}$}
The width of the write current pulse for switching is inversely proportional to the magnitude of the applied current density in the SOT layer $j_{sw}$ \cite{manchon2019current}
\begin{equation}
    \tau_{p} \propto \frac{1}{j_{sw}}
\label{tau}
\end{equation}

As the area of the SOT layer ($A_{SOT}$) is scaled down, the effective current density increases,  $j_{sw} \propto 1/(A_{SOT})$.  Successful switching should take place when $j_{sw} > j_{c}$. We can increase $j_{sw}$ by reducing the SOT layer dimension and decrease $j_{c}$ by increasing $\theta_{SH}$ or by decreasing $t_{FL}$. Thus we can achieve successful switching in much shorter pulse width (equation \ref{tau}). \cite{garello2014ultrafast} demonstrated the switching at 180ps, \cite{wu2021voltage} at 400ps, and \cite{garello2018sot} at 210ps. Switching in shorter pulse width ensures larger write bandwidth which is essential for memory systems used in AI/Deep Learning hardware. The key DTCO parameters of SOT-MRAM and their impact on Power, Performance and  Area (PPA) are listed in Table \ref{dtco_table}. 





\section{Results and Analysis}
\label{result_analysis}
% We developed a MATLAB-based framework to implement our analytical \emph {Memory and Compute Model}. Unlike ScaleSim \cite{samajdar2020systematic} and Timelooop \cite{parashar2019timeloop} simulator, which only support profiling DNN workloads in inference mode to date, our model captures both training and inference behavior of CV and NLP models. In this section, we provide the result and analysis of the CV and NLP workloads during inference and training from our analytical model and present the optimum Power, Performance, and Area results by performing the DTCO of SOT-MRAM.

In this section, we provide the results and analysis of the STCO on the CV and NLP workloads during inference and training and present the optimum Power, Performance, and Area results by performing the DTCO of SOT-MRAM. We developed a MATLAB-based framework to implement our analytical \emph {Memory and Compute Model} to capture the relationship between the memory access counts and the memory hierarchy sizes in typical systolic array based AI accelerators. 
Unlike ScaleSim \cite{samajdar2020systematic} and Timelooop \cite{parashar2019timeloop} simulator, which only support profiling DNN workloads in inference mode to date, our model captures both training and inference behavior of CV and NLP models. We also verified our model's results with Timeloop in inference mode.



\subsection{Bandwidth Demand}
\label{bw_demand}


% We implement our analytical bandwidth characterization model to calculate the bandwidth demand for 18 widely used state-of-the-art CV models for both training and inferene cases. Our analytical model differs from the 
% ScaleSim\cite{samajdar2020systematic} simulator. Given the PE array dimension, on-chip memory size, and dataflow, ScaleSim reports the DRAM bandwidth for IFMAP, OFMAP, and weight buffer for stall-free execution of the PE array, and the PE array utilization depends on the array dimension and mapping of the dataflow. However, we estimate the required on-chip bandwidth to ensure that the array utilization can reach 100\%, assuming that the mapping efficiency is 100\%.

%%parameter table for NLP models
\begin{table*}[]
\centering
\caption{Parameters of NLP models}
\label{nlp_param}

\begin{tabular}{|c|c|c|c|c|c|c|c|}
\hline
Model & Enc. layer & Dec. layer & Attn. head & \begin{tabular}[c]{@{}c@{}}Word Embedding\\ ($N_{em}$)\end{tabular} & \begin{tabular}[c]{@{}c@{}}Intermediate dimension\\ ($d_{ff}$)\end{tabular} & \begin{tabular}[c]{@{}c@{}}Seq. length \\ ($N_{sql}$)\end{tabular} & \begin{tabular}[c]{@{}c@{}}Vocab. size\\ ($N_{vocab}$)\end{tabular} \\ \hline
Transformer  & 12 & 6 & 8 & 512 & 2048 & 1024 & 37000 \\ \hline
BERT & 12 & - & 12 & 768 & 3072 & 512 & 30522 \\ \hline
Distil BERT  & 6 & - & 12 & 768 & 3072 & 512 & 30522 \\ \hline
Mobile BERT  & 24 & - & 4 & 128 & 512 & 512 & 30522 \\ \hline
Squeeze BERT & 12 & - & 12 & 768 & 3072 & 512 & 30522 \\ \hline
Visual BERT  & 12 & - & 12 & 512 & 3072 & 512 & 30522 \\ \hline
GPT & - & 12 & 12 & 768 & 2048 & 512 & 40478 \\ \hline
GPT-2 & - & 12 & 12 & 768 & 2048 & 1024 & 50257 \\ \hline
GPT-3 & - & 96 & 96 & 12288 & 49152 & 2048 & 50257 \\ \hline
GPT-Neo & - & 24 & 16 & 2048 & 8192 & 2048 & 50257 \\ \hline
GPT-J & - & 28 & 16 & 4096 & 16384 & 2048 & 50400 \\ \hline
\end{tabular}
\end{table*}


\begin{figure}[ht]
    \centering
    \includegraphics[scale=0.6]{Bandwidth_CV.pdf}

    \caption{Bandwidth requirement of CV models for different PE array size. (a) Read Bandwidth, (b) Write Bandwidth.}

    \label{bw_cv}
\end{figure}

\begin{figure}[ht]
    \centering
    \includegraphics[scale=0.6]{Bandwidth_NLP.pdf}
  
    \caption{Bandwidth requirement of NLP models for different PE array size. (a) Read Bandwidth (for GEMM and softmax operation), (b) Write Bandwidth.}.
  
    \label{bw_nlp}
\end{figure}


\begin{figure*}[ht]
    \centering
    \includegraphics[width = \textwidth]{FIG_CV_Vary_GLB.pdf}
   
    \caption{Impact of larger GLB memories on performance and energy efficiency for CV models at inference and training. Percentage reduction in DRAM accesses at inference (a) and training (d). Performance Speedup  from DRAM access reductions at inference (b) and training (e). Energy savings from reduced DRAM accesses at inference (c) and training (f). Both cases compare results to a baseline of 2MB GLB running 16 samples.}
   
    \label{d_fig1_cv_infr}
\end{figure*}

\begin{figure*}[ht]
    \centering
    \includegraphics[scale = 1.15]{FIG_CV_Vary_BATCH.pdf}
    
    \caption{Impact of batch size on performance and energy efficiency for CV models at inference and training. Percentage increase in DRAM accesses at inference (a), at training (d). Performance slowdown (latency increase) from extra DRAM accesses at inference (b), at training (e). Energy increase from extra DRAM accesses at inference (c), at training (f). In both cases, results are compared to a baseline of 16 samples running with 4MB GLB.}
    
    \label{d_fig2_cv}
\end{figure*}

In Fig. \ref{bw_cv} (a), (b), we plot the read-write on-chip bandwidth demand in $bytes/cycle$ of 18 widely used CV models. Resenet101 and Resnet50 running on a 256$\times$256 PE array will demand the highest read bandwidth, 4017 bytes/cycle, from GLB, whereas Squeezenet will demand the lowest bandwidth, 1028 bytes/cycle. Naturally, as the PE array size increases, the computation capacity per cycle $T_{MAC}$ increases which demands more data from memory to keep all PEs active. From the workload perspective, we observe that the most contributing factor to the read bandwidth demand is its inverse relationship with the filter and ofmap size. We explain the inverse relationship of filter and ofmap size with the read bandwidth using the convolutional reuse concept. As the filter size decreases, the scope of convolutional reuse decreases. The ofmap again depends on the filter and ifmap size. With the decrease of filter size and ofmap size, the convolutional reuse decreases, giving rise to more bandwidth demand. The layer of Resnet101 that requires the most bandwidth (4017 bytes/cycle) has the ofmap dimension (7$\times$7) and filter dimension (1$\times$1). On the other hand, the most demanding (1028 bytes/cycle) layer of Squeezenet has the ofmap dimension (18$\times$18) and filter dimension (1$\times$1). Another observation is that though 1$\times$1 convolution reduces the computation complexity, it requires more bandwidth from memory, i.e., becomes memory intensive. The write bandwidth is also inversely proportional to the filter size. However, in 1$\times $1 convolutions, it depends on the number of outputs generated by the PE array. The write bandwidth is always smaller than the read bandwidth (Fig. \ref{bw_cv} (b)) as it takes more than one operand to generate one output. For example, in a 3$\times$3 convolution, it takes 18 operands to generate a single output; in a 1$\times$1 convolution, it takes two operands. 


\begin{figure*}[ht]
    \centering
    \includegraphics[width = \textwidth]{FIG_NLP_Vary_GLB.pdf}
    
    \caption{Impact of larger GLB memories on performance and energy efficiency for NLP models at inference and training. Percentage reduction in DRAM accesses at inference(a), at training (d). Performance Speedup  from DRAM access reductions at inference (b), at training (e). Energy savings from reduced DRAM accesses at inference (c), at training (e). In both cases, results are compared to a baseline of 2MB GLB running 16 samples}
    \label{d_fig1_nlp}
    
\end{figure*}

\begin{figure*}[ht]
    \centering
    \includegraphics[scale = 1.11]{FIG_NLP_VARY_Batch.pdf}
    
    \caption{Impact of batch size on performance and energy efficiency for NLP models at inference and training. Percentage increase in DRAM accesses, inference (a), and training (d). Performance slowdown (latency increase) from extra DRAM accesses at inference (b), at training (e). Energy increase from extra DRAM accesses at inference (c), at training (f). Results are compared to a baseline of 16 samples running with 4MB GLB.}
    \label{d_fig2_nlp}
\end{figure*}




As mentioned in section \ref{bw_fc_layer}, the bandwidth requirement for transformer-basded model are calculated using the expressions of Table \ref{table:bw_expression}. The dimension of the operand matrices is larger than the PE array dimension, hence following Case IV (Table \ref{table:bw_expression}, Section \ref{bw_fc_layer}), the read bandwidth of all models depends on the PE array size (Fig. \ref{bw_nlp} (a)). The write bandwidth depends on the PE array dimension and the input sequence length. The softmax read bandwidth depends on the SFU width, and matches with the GEMM read bandwidth. 
As different models are trained with different input sequence lengths\cite{hugging_face}, their write bandwidth demand is not the same across all models. The parameter sizes and settings of the models used in this work are shown in Table \ref{nlp_param}.
The models having the highest sequence length (2048) have the lower write bandwidth demand 102 bytes/cycle running on a 256$\times$256 PE array (Fig. \ref{bw_nlp} (b)).


\subsection{Impact of on-chip memory}
\label{on-chip impact}

\begin{figure*}[ht]
    \centering
    \includegraphics[width=\textwidth]{dtco_fig_1.pdf}
    
    \caption{Critical current vs $\theta_{SH}$(a), $w_{SOT}$(b), $t_{SOT}$(c), and $t_{FL}$(d).}
    \label{fig:dtco_ic}
    
\end{figure*}

%To realize the impact of on-chip memory on the accelerator performance, we implement our analytical model of memory access counts (Table \ref{conv_analysis} for CV models and Table \ref{Transformer_analysis} for NLP models, see section \ref{mem_access}). 
%Timeloop \cite{parashar2019timeloop} also reports the scalar DRAM read-write counts for DNN models, but we did not use it in our analysis as it does not map the DRAM access counts as a function of GLB size.
%rather a function of mapping and loop nests.

\begin{figure}[ht]
    \centering
    \includegraphics[scale=0.48]{dtco_fig2_ic_vs_tp_delta_t_ret.pdf}
   
    \caption{(a) Switching pulse width $\tau_{p}$ vs applied switching current $I_{sw}$.  (b) Thermal stability factor $\Delta$ (left Y-axis) and retention time $t_{ret}$ (right Y-axis) vs MTJ dimension for a fixed retention failure rate, $P_{RF} = 10^{-9}$. At $\Delta = 70$, MTJ dimension = 88nm, retention time is $>$ 10 years \cite{imce2019}.}
    \label{fig:dtco_pulse_width}
\end{figure}

\begin{figure}
    \centering
    \includegraphics[scale=0.48]{dtco_fig3_tmr_t_MgO_rd_latency.pdf}
    
    \caption{Impact of, (a) oxide thickness on TMR. (b) TMR on read latency.}
    \label{fig:tmr_vs_oxide_rd_ltncy}
    
\end{figure}

Compared to a GLB size of 2MB, the DRAM access counts for all CV models decrease significantly if we increase the GLB size. In inference, reaching the 100\% reduction in access means it only needs to read the initial inputs, weights for each layer, and write the final layer output, no DRAM access is needed for the intra and inter-layer operations. Further increase in GLB size will not improve the performance in these cases. 
%We did this analysis running a batch of 16 samples, which is quite a large batch size for inference. 
For 16 samples, DRAM access is reduced by 100\% for 14 models at 128MB, and most models experience a reduction of >80\% at 64MB (Fig. \ref{d_fig1_cv_infr} (a)). Fig. \ref{d_fig1_cv_infr} (b), (c) show the performance speed up and energy saving coming from these DRAM access reductions. 


We observe a slower improvement in the DRAM access reduction during training unless the GLB size is large enough, at least 256MB for most models (Fig. \ref{d_fig1_cv_infr} (d)). However, even the smaller percent reduction in DRAM access results in significant performance and energy improvement %compared to a larger percent reduction during inference% 
(Fig. \ref{d_fig1_cv_infr} (e), (f)). This is because training requires at least 2$\times$ DRAM accesses as inference. The smaller percent reduction of a large number of DRAM accesses translates to a significant energy and latency improvement. 
A similar trend is observed for NLP models. Transformer-based NLP models are usually larger than the CV models. This is the reason we achieve more performance speedup and energy reduction even at smaller DRAM access reduction rate (Fig. \ref{d_fig1_nlp}).We also observe that DNN models learn faster if we increase the batch size. However, for a fixed GLB size, the DRAM access count increases significantly at larger batch size, causing performance slowdown and more energy consumption. Fig. \ref{d_fig2_cv} and Fig. \ref{d_fig2_nlp} (a, b, c for inference and d, e, f for training) show the increase in DRAM access count and its associated impact on performance and energy for CV and NLP models respectively at different batch sizes.  

The key takeaway from this analysis is that we can reduce the energy and latency associated with DRAM accesses if we increase the GLB size. For larger batch sizes, the energy and latency improvement is even more. Because at large batch sizes, throughput increases at the cost of DRAM accesses. As we increase the GLB size, DRAM accesses reduce, and we achieve latency and energy reduction.

\subsection{DTCO of SOT for PPA Optimization}


From section \ref{on-chip impact} we see that the GLB size of 64MB (for inference) and 256MB (for training) offer significant energy and performance improvement. However, it is not feasible and efficient to use such large SRAMs because of its area and leakage power, even if the low-power techniques are employed. Section \ref{bw_demand} implies that we need approximately 4000bytes/cycle bandwidth between GLB and PE array for larger array size (256$\times$256). In this subsection, we provide the SOT-MRAM DTCO results and observation meeting the requirements stated in the above two subsections. We perform the DTCO in \textit{Cadence Virtuoso}  tool using the compact SOT-MRAM model from \cite{sot_model_kazemi}, and use \textit{Synopsys} 14nm library \cite{saed14} for  the CMOS transistors and peripheral circuits.

\subsubsection{$I_{C}$ optimization}

%The critical switching current depends on the device's physical and geometrical structure and its magnetic property. The critical switching current can be controlled for a given external field by adjusting the device dimension and using the appropriate material. 
To realize the impact of SOT efficiency $\theta_{SH}$ on $I_{c}$, we sweep $\theta_{SH}$ from 0.1 to 100 (Fig. \ref{fig:dtco_ic} (a)). With $\theta_{SH} \ge 100$,  $I_{c}$ goes as low as 0.5uA. Even though the widely used SOT layers are made of heavy metal alloys having smaller $\theta_{SH}$ (e.g., 0.1 to 0.5), recent advancement in material engineering demonstrates that in topological insulator $\theta_{SH}$ can go as high as 152 \cite{khang2018conductive}. We recommend using topological insulators as the SOT layer to achieve a lower switching current.


Next, we analyze the impact of SOT layer geometry on the switching current (Fig. \ref{fig:dtco_ic} (b), (c)). $I_{c}$ scales down linearly with the decrease of SOT layer width, and $w_{SOT}$ can be set to desired value based on the performance and reliability requirement (Fig. \ref{fig:dtco_ic} (b)). While $I_{c}$ scales linearly with the width of the SOT layer, the thickness of the SOT layer has an interesting effect on the switching current. The SOT layer should be relatively thin but bulk enough for heavy metal layers to experience the bulk effect to achieve high SOT efficiency. Once it crosses optimum thickness, which is ~3nm (Fig. \ref{fig:dtco_ic} (c)), many of the charges that are injected into the metal do not contribute to the switching, and $I_{c}$ increases. 

The smaller the free layer thickness, $t_{FL}$, the smaller the switching current (Fig. \ref{fig:dtco_ic} (d)). We also scale the diameter of MTJ, $d_{MTJ}$, to reduce the MTJ area. However, with the scaling down of $d_{MTJ}$ together with $t_{FL}$, the thermal stability factor $\Delta$ also scales down, reducing the memory's data retention time $t_{ret}$. Non-volatility is a great feature of MRAM, but it can be compromised to achieve higher density, higher bandwidth, and lower energy when the target application is a cache. Because, in the cache even for AI workloads, the data lifetime is much shorter, typically in the seconds range \cite{stt_ai_us}. Fig. \ref{fig:dtco_pulse_width}(b) shows $\Delta$ and $t_{ret}$ as functions of free layer volume. While scaling down $t_{FL}$ to optimize $I_{c}$, and $d_{MTJ}$ to optimize area, we keep an eye on the reliability of the stored data. We consider a retention failure rate of $10^{-9}$ (i.e., 1 bit flip per billion).

\subsubsection{Bandwidth optimization}
As shown in Fig. \ref{fig:tmr_vs_oxide_rd_ltncy} (a), TMR ratio of the MTJ device can be increased by increasing the oxide thickness \cite{tsunekawa2005giant}. We increase the oxide thickness to decrease the  read latency (Fig. \ref{fig:tmr_vs_oxide_rd_ltncy} (b)). 
The write pulse width is inversely proportional to the applied switching current. While we want to lower the applied current to achieve low energy, the higher amplitude of the applied current is required for faster magnetization reversal. However, switching occurs at smaller pulse width at the iso-current if we scale down the SOT layer width. This is because of the smaller critical current at smaller geometry (Fig. \ref{fig:dtco_ic} (b,d)).
Fig. \ref{fig:dtco_pulse_width}(a) shows that switching pulse width can be reduced significantly by scaling down the SOT layer width. Thus, we can achieve higher write bandwidth by scaling down the SOT layer width to meet the high BW demand from AI workloads. 

%The DTCO optimized parameters of SOT-MRAM used in this study are listed in Table \ref{optimized_dtco_param}. Since process variations can impact precise control of the device dimensions, we add 30\% guard-band to the device dimensions in Table \ref{optimized_dtco_param}.

\begin{table}[ht]
\centering
\caption{SOT-MRAM DTCO optimized parameters. 30\% guard-band are added with thickness and width for process variations.}
\label{optimized_dtco_param}

\setlength\tabcolsep{5pt}
\begin{tabular}{|l|l|l|l|}
\hline
\multicolumn{1}{|c|}{Parameter} & \multicolumn{1}{c|}{Value} & \multicolumn{1}{c|}{Parameter} & \multicolumn{1}{c|}{Value} \\ \hline
Spin Hall angle $\theta_{SH}$   & 1                          & TMR                            & 240\%                      \\ \hline
Free layer thickness $t_{FL}$   & 0.5nm                      & MTJ diameter $d_{MTJ}$               & 55nm                       \\ \hline
SOT width $w_{SOT}$             & 130nm                      & SOT thickness $t_{SOT}$        & 3nm                        \\ \hline
Oxide thickness $t_{MgO}$       & 3nm                        & Thermal stability factor $\Delta$                       & 45                         \\ \hline
\end{tabular}

\end{table}


\subsection{Process \& Temperature Variation and Bitcell Simulation}
In this subsection, we perform Process and Temperature variation on the DTCO-optimized parameters, design the peripheral circuits, and test the read-write operation on the bit cell at scaled parameters.

\subsubsection{Process and Temperature variation}
To incorporate process variations, we model MTJ diameter, free layer thickness, and SOT layer width as Gaussian variables in the Verilog A model of SOT-MTJ \cite{sot_model_kazemi}. We assume standard deviations ($\sigma$) as 5\% of their respective means ($\mu$) and perform Monte Carlo simulations with 5000 samples within 4$\sigma$ variation. We also consider the temperature variations. The extreme point at the right side of the scaled target parameter is $\mu + 4\sigma, T_{cold}$ (Fig. \ref{fig:process_var}). From equations \ref{Ic3} and \ref{tau}, $I_{sw}$ and $\tau_{p}$ are independent of Temperature. As a result, the worst case for write operation (highest $I_{sw}$ and longest $\tau_{p}$) is at $\mu + 4 \sigma$. This point is, however, benign to the read operation and retention failure.
As we scale down $d_{MTJ}$ and $t_{FL}$, $\Delta$ also reduces, reducing $t_{ret}$, and $I_{data}$. $\Delta$ reduces further as temperature increases \cite{stt_eqn1}. Thus, the worst case for read operation (smallest $I_{data}$) and retention failure (smallest $t_{ret}$) is at $\mu - 4 \sigma, T_{hot}$ (see Fig. \ref{fig:process_var}). As $I_{data}$ reduces, the difference between $I_{data1}$ and $I_{data0}$ becomes even smaller and difficult to sense.
%We optimize and test our read circuitry at the worst-case variation.

To ensure the reliability of the SOT-MRAM bit cell, we add a 30\% guard band on the scaled SOT device parameters: 20\% for process variation and 10\% for temperature variation. The optimized DTCO parameters after adding the PT induced 30\% guard-band are shown in Table \ref{optimized_dtco_param}.


\begin{figure}[h]
    \centering
\includegraphics[width=0.43\textwidth]{process_variation_new.pdf}
    \caption{Impact and distribution of Process and Temperature variation on scaled parameters.}
    \label{fig:process_var}
\end{figure}

\subsubsection{Write operation}
To write SOT-MRAM bitcell, we bias BL with the data-to-be-written and SL with the complement of data-to-be-written. Assuming that the magnetization state of the Reference layer is -1, to write 1 into the MTJ bitcell, we switch the magnetic orientation of the Free layer to +1 state resulting in a high resistive state. To achieve this state, we turn on the WWL, connect BL to VDD and SL to the ground. The resultant current switches the free layer's magnetic orientation from -1 to +1. The opposite bias is applied to write 0. We do not need any additional peripheral circuits for the write operation of SOT-MTJ.

\subsubsection{Read operation} Read operation involves sensing the current passing through MTJ at P and AP states. For our SOT-MRAM bitcell, with the parameters shown in Table \ref{optimized_dtco_param}, $I_{data0} = 20uA$ and $I_{data1} = 33uA$. We design and optimize the read circuitry to sense this small differential current, as shown in Fig. \ref{fig:peripheral_circuits}. 
Our proposed read sensing circuit only contains an additional current mirror block (to amplify current), and it does not require the precharge circuits compared to SRAM. Hence, there is no additional area overhead in the periphery compared to SRAM. The dynamic power consumption are shown in Table \ref{tab:power}

To capture the stochastic nature of MTJ switching, we simulate the bit cell for 1000 bitstream. We achieve a read and write yield of 100\%, and at 250ps and 520ps, respectively.  
% With DTCO we achieve a read latency of 250ps and a write latency of 500ps for SOT-MRAM bit-cell. 
This results in  read bandwidth of 4 Gbps and a write bandwidth of 1.9 Gbps. We then dynamically allocate the memory bus width on-demand to satisfy the bandwidth requirement for different workloads and PE array size stated in section \ref{bw_demand}.
\begin{table}[h]
    \centering
    \caption{Dynamic Power consumption (in uW) of SRAM and SOT-MRAM. (1/0) means the corresponding power to access bit 1 and 0.}
    \begin{tabular}{|c|c|c|}
        \hline
         & Read(1/0) & Write(1/0) \\ \hline
        SRAM & 426 & 373 \\ \hline
        SOT-MRAM & 150/368 & 325/300 \\ \hline
        \end{tabular}
    \label{tab:power}
\end{table}

\begin{figure}
    \centering    \includegraphics[scale=0.43]{mtj-read-write.pdf}
    \caption{SOT-MTJ bitcell with read sensing circuitry.}
\label{fig:peripheral_circuits}
\end{figure}




\subsection{System level performance evaluation of SOT-MRAM based Memory}
\begin{figure*}[ht]
    \centering
    \includegraphics[width = \textwidth]{SYS_LVL_PRF_ALL.pdf}

    \caption{System level energy improvement with SOT-MRAM and DTCO-optimized-SOT-MRAM over SRAM at the same size for CV (a-d) and NLP (e-h) models. The top plots show energy (a, e) and latency (b, f) for inference, and the bottom plots show energy (c, g) and latency (d,h) for training.}

    \label{fig_sys_lvl_perf_cv}
\end{figure*}

% \begin{figure*}[ht]
%     \centering
%     \includegraphics[width = \textwidth]{Fig_sys_lvl_perf_nlp.pdf}
%     \caption{System level energy improvement with SOT-MRAM and DTCO optimized SOT-MRAM over SRAM at the same size for NLP models: (a) Inference, (b) Training. System level latency improvement with SOT-MRAM and DTCO optimized SOT-MRAM over SRAM at the same size: (c) Inference, (d) Training.}
%     \label{fig_sys_lvl_perf_nlp}
% \end{figure*}

% \textcolor{blue}{In this subsection, we evaluate the performance at the system level on the DNN benchmarks with SRAM, SOT-MRAM, and DTCO-optimized-SOT-MRAM. }

In this subsection, we analyze the PPA (Power, Performance, and Area) metrics at the system level on the DNN/CNN benchmarks with SRAM, SOT-MRAM, and DTCO-optimized-SOT-MRAM. 
We use the Destiny \cite{destiny} memory simulator to find the array-level data for both SRAM and SOT-MRAM. We modify Destiny source code to reflect: (i) SOT switching mechanism, (ii) special read sensing circuit for SOT-MRAM, and (iii) 14nm CMOS technode. Then, we feed the extracted bitcell-level data of SOT-MRAM in the $.cell$ file to find the PPA at the desired memory capacity.

%We modify Destiny\cite{destiny} source code to reflect the SOT switching mechanism and special peripheral circuits for SOT-MRAM. With the extracted bitcell-level data (including peripheral circuits), we design the SOT-MRAM.cell in Destiny \cite{destiny} to find the array-level PPA of SOT-MRAM. We also modify the destiny Technology file to reflect the 14nm process node for CMOS technology, as we }  
% To find the bit-cell level read/write latency and energy, we simulate Synopsys 14nm low-power SRAM \cite{saed14} bit cell, and SOT-MRAM \cite{sot_model_kazemi} bit-cell in Cadence Virtuoso. We then feed this data to Destiny \cite{destiny} to find the array-level PPA. To reflect the separate read-write path and SOT switching mechanism, we designed the peripheral and sensing circuits in the Cadence Virtuoso and used the extracted latency and energy parameters in Destiny's config. file.
 Based on the array-level results from Destiny, and DRAM \& GLB access counts from Algorithms \ref{alg:dram_acc_cnt_infr}, and \ref{alg:dram_acc_cnt_trng}, we estimate the system-level power and performance. Finally, we analyze the area of the memory modules of different technologies (14nm SRAM, SOT-MRAM, and DTCO-opt-SOT-MRAM) at iso-capacity. This analysis only incorporates the PPA metrics from the memory system (DRAM and GLB), assuming that the PPA of the compute unit is constant. With SOT-MRAM as GLB, we see significant energy and latency improvement over SRAM at 64MB (for inference) and 256MB (for training) (see Fig. \ref{fig_sys_lvl_perf_cv} (a-d) for DNN benchmarks and (e-h) for NLP benchmarks). On average, the 64MB SOT-MRAM offers 5$\times$ energy reduction and 2$\times$ latency reduction over 64MB SRAM across all CNN models at inference. Our DTCO-optimized-SOT-MRAM offers further improvement, 7$\times$ energy, and 8$\times$ latency reduction over SRAM at iso-capacity. For latency improvement, the most contributing factor is the DRAM access reduction with large GLB and the smaller read/write latency of SOT-MRAM at larger capacity compared to SRAM. At smaller capacity, SRAM is way faster than SOT-MRAM \cite{tahoori_1,optimized_SOT_imec}. We observe that the most contributing factor in energy reduction (>50\%) is the near-zero leakage power of SOT-MRAM compared to high leakage power of SRAM. The improvement is even more in training mode; 6$\times$ (8$\times$ with SOT-opt.) energy reduction and 2$\times$ (9$\times$ with SOT-opt.) latency reduction. With 64MB SOT-MRAM, NLP models in inference mode experience 2$\times$ (3$\times$ with SOT-opt.) energy reduction and 2$\times$ (4$\times$ with SOT-opt.) latency reduction than 64MB SRAM. Like CV benchmarks, with 256MB SOT-MRAM, NLP benchmarks also experience more energy improvement, 6$\times$ (8$\times$ with SOT-opt.), and latency improvement, 2.5$\times$ (4.5$\times$ with SOT-opt.), in training mode. The more improvement in training mode is because of two reasons: (1) GLB size increases from 64MB to 256MB, and (ii) GLB access counts are significantly large (at least 5$\times$) in training. Our DTCO-opt-SOT-MRAM further adds value to PPA by its smaller silicon area, 0.54$\times$ at 64MB and 0.52$\times$ at 256MB of 14nm SRAM at iso-capacity (Fig. \ref{sys_area_comp}).

\begin{figure}[ht]
    \centering
    \includegraphics[scale = 0.55]{Fig_Area_comparison.pdf}

    \caption{Area improvement of SOT-MRAM and SOT-MRAM-OPT}
    \label{sys_area_comp}
   
\end{figure}





\section{Related Work}
\label{rltd_work}
SOT-MRAMs have been widely studied as the next generation of STT-MRAM to leverage all benefits of MRAMs as embedded memory \cite{recent-progress-in-SOT_fab3}\cite{optimized_SOT_imec}\cite{dualport_fieldfree_fab2}\cite{ultrafast_embedded_mem_fab4}\cite{size_dependent_switching_fab1}\cite{sot_0.35ns_write}. However, very few studies have evaluated the performance of SOT-MRAM as on-chip memory in system-level. \cite{tahoori_1} and \cite{sys_lvl_eval_sot_rltd_wrk} demonstrated the performance improvement of SOT-MRAM as L2 data cache compared to SRAM L2 cache on MiBench, SPEC2000 and SPEC2006 benchmarks. SOT-MRAMs have also been explored in the context of DL accelerators as a promising technology for In-Memory Computing (IMC) \cite{IMC1_for_RW}\cite{IMC2_for_RW}\cite{IMC4_for_RW}. IMC has pros and cons, and our work where we use SOT-MRAM as the cache storage element differs from IMC.
While the scope of SOT-MRAM has been explored both as regular CPU cache and IMC for DL accelerator to some extent, to the best of our knowledge, unlike IMC, 
%our work is the first to evaluate SOT-MRAM-based on-chip memory for DL applications. 
this is the first work that presents a comprehensive analysis of SOT-MRAM as on-chip memory for application in AI/DL accelerators.

\section{Conclusion}
\label{conclusion}
We proposed an efficient and high-performance memory system with SOT-MRAM for AI accelerators in this work. Guided by detailed target workload characterization, our memory system comprises of HBM3 DRAM, a DTCO-enabled SOT-MRAM GLB and a small SRAM buffer. Our large SOT-MRAM GLB significantly reduces the energy and latency by reducing expensive DRAM accesses while still having acceptable on-chip access energy and latency, achieving overall system-level high performance. We finally demonstrate that our memory system performs 8$\times$ and 9$\times$ better in terms of energy and latency respectively on CV benchmarks in training (7 and 8 times better in inference) and 8$\times$ and 4.5$\times$ better in terms of energy and latency respectively on NLP benchmarks in training (3 and 4 times better in inference) while consuming only around 50\% of SRAM area at iso-capacity.
\ifCLASSOPTIONcaptionsoff
  \newpage
\fi

\bibliographystyle{IEEEtran}
\bibliography{REFERENCE_NEW.bib}

\begin{IEEEbiography}[{\includegraphics[width=1\linewidth]{head_shot_edited}}]{Kaniz Mishty} received the B.S. degree in Electronics and Communication Engineering from Khulna University of Engineering and Technology, Bangladesh, in 2018. She is currently working towards her Ph.D. degree in ECE at Auburn University, AL, USA. Her current research interests are energy and area efficient VLSI system design, AI/Neuromorphic hardware design and AI/ML in CAD. She interned with Apple Inc. in Summer '22 and Qualcomm Tech. in Summer '21. During her internship with Apple she worked on AI application in custom circuit design flow to improve PPA.
\end{IEEEbiography}

\begin{IEEEbiography}[{\includegraphics[width=\linewidth]{sadi}}]{Mehdi Sadi} (S'12-M'17) is an Assistant Professor at the Department of Electrical and Computer Engineering at Auburn University, Auburn, AL.  Dr. Sadi  earned his PhD in ECE from  University of Florida, Gainesville, in 2017, MS from University of California at Riverside, USA in 2011 and BS from Bangladesh University of Engineering and Technology in 2010.   Prior to joining Auburn University, he was a Senior R\&D SoC Design Engineer at Intel Corporation in Oregon. Dr. Sadi`s research focus is on developing algorithms and CAD techniques for implementation, design, reliability, and security of AI hardware. His research also spans into developing Machine Learning/AI enabled design flows for  System-on-Chip, and Design-for-Robustness. He has published more than 25 peer-reviewed research papers. He was the recipient of Semiconductor Research Corporation best in session award, Intel Xeon Design Group recognition awards, and National Science Foundation CRII award.
\end{IEEEbiography}
\end{document}


