\section{Related Work}
\label{rltd_work}
SOT-MRAMs have been widely studied as the next generation of STT-MRAM to leverage all benefits of MRAMs as embedded memory \cite{recent-progress-in-SOT_fab3}\cite{optimized_SOT_imec}\cite{dualport_fieldfree_fab2}\cite{ultrafast_embedded_mem_fab4}\cite{size_dependent_switching_fab1}\cite{sot_0.35ns_write}. However, very few studies have evaluated the performance of SOT-MRAM as on-chip memory in system-level. \cite{tahoori_1} and \cite{sys_lvl_eval_sot_rltd_wrk} demonstrated the performance improvement of SOT-MRAM as L2 data cache compared to SRAM L2 cache on MiBench, SPEC2000 and SPEC2006 benchmarks. SOT-MRAMs have also been explored in the context of DL accelerators as a promising technology for In-Memory Computing (IMC) \cite{IMC1_for_RW}\cite{IMC2_for_RW}\cite{IMC4_for_RW}. IMC has pros and cons, and our work where we use SOT-MRAM as the cache storage element differs from IMC.
While the scope of SOT-MRAM has been explored both as regular CPU cache and IMC for DL accelerator to some extent, to the best of our knowledge, unlike IMC, 
%our work is the first to evaluate SOT-MRAM-based on-chip memory for DL applications. 
this is the first work that presents a comprehensive analysis of SOT-MRAM as on-chip memory for application in AI/DL accelerators.
