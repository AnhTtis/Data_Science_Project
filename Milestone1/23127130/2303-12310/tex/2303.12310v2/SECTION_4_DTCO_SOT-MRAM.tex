% \vspace{-5pt}
% \vspace{-5pt}
\section{DTCO of SOT-MRAM}
\label{dtco_MRAM}
To ensure  overall system performance for AI workloads, the memory system should have large on-chip memory to avoid frequent DRAM accesses, and the on-chip memory should have high bandwidth to prevent the system from being memory-bound while being energy efficient. In this section, we perform a DTCO of SOT-MRAM in bit-cell level based on the workload profiling done in section \ref{workload_profiling}.

\subsection{Optimizing critical switching current $I_{c}$}
%Writing to any MRAM involves switching the magnetic orientation of the free layer of the MTJ stack. 
In SOT-MRAM, the magnetic orientation of the free layer is switched by Spin-Orbit Torque induced by spin Hall and interfacial effects between the channel (i.e., SOT layer) and free layer (FL) of MTJ. An in-plane charge current is flown through the channel to generate a spin current that exerts a spin torque on the free layer, which rotates the free layer's magnetic orientation. The critical current density required to switch the magnetic orientation of FL is expressed as \cite{lee2013threshold}

\begin{equation}
    j_{c} = \frac{2e\mu_0 M_{s,FL}t_{FL}}{\hbar\theta_{SH}} (\frac{H_{k,eff}}{2} -\frac{H_x}{\sqrt{2}})
    \label{Ic3}
\end{equation}
Where $H_{k,eff}$ is the effective anisotropy field, $H_{x}$ is the applied field, $M_{s,FL}$ is the saturation magnetization of free layer, and $t_{FL}$ is its thickness. Our interest is in lowering the switching current to achieve low write energy.
Here, the free layer thickness $t_{FL}$ and spin Hall efficiency $\theta_{SH}$ act as a control knob for critical switching current. $\theta_{SH}$ is a material-specific parameter and its higher value is expected to reduce the switching current. The typical value of $\theta_{SH}$ in heavy metal alloys ranges between 0.1 to 0.5 \cite{manchon2019current}. However, recent topological insulators as SOT layer can have a very large $\theta_{SH}$. \cite{khang2018conductive} demonstrated $\theta_{SH} = 152$ with $BiSb$ thin films.



\subsection{Optimizing read-write pulse width}
\subsubsection{Read pulse width}
The reading of SOT-MRAM involves sensing the resistance of the MTJ. A small amount of current is passed through the MTJ stack and the voltage across the stack $V_{data+}$ or $V_{data-}$ is compared against a reference voltage $V_{ref} = \frac{1}{2}(V_{data+}+ V_{data-})$ to read out the stored bit. The read Sensing Margin $SM = |V_{ref} - V_{data}|$ is typically very small. Sensing and amplifying this small difference requires a strong and complex Sense Amplifier  that contributes to most of the read latency and energy. The SM is determined by the Tunnel Magneto Resistance ratio ($TMR\;ratio = \frac{R_{AP}-R_{P}}{R_{P}}$) of MTJ. A higher TMR ratio produces a larger SM by making $V_{data+}$ higher and $V_{data-}$ lower. Thus the TMR ratio is inversely proportional to the read latency\cite{3T-2SOT}. The higher the TMR window, the higher the read speed and the less effort required on the periphery. The typical range of the TMR ratio is between $100$ to $300\%$. The TMR is tunable by oxide thickness \cite{tsunekawa2005giant} as shown in Fig. \ref{fig:tmr_vs_oxide_rd_ltncy} (a). In SOT-MRAM, we can increase the oxide thickness, thanks to the decoupled read-write path of SOT-MRAM, to achieve a high TMR and increase the read speed without worrying about the large incubation time \cite{wang2013low}. 

%\vspace{-10pt}
\begin{table}[ht]
\setlength{\tabcolsep}{3pt}
\centering
\caption{DTCO control parameters \& their impact on Power, Performance and  Area (PPA)}
\label{dtco_table}
% \vspace{-10pt}
\begin{tabular}{|l|l|}
\hline
\textbf{DTCO Parameters}               & \textbf{Impact on PPA}                                                                             \\ \hline
Spin Hall angle $\theta_{SH}$ & $\theta_{SH} \uparrow$, $j_{c} \downarrow$,  Switching energy $\downarrow$                \\ \hline
Free layer thickness $t_{FL}$ & $t_{FL} \downarrow$, $j_{c} \downarrow$, Switching energy $\downarrow$, Area $\downarrow$           \\ \hline
\multirow{2}{*}{\begin{tabular}{@{}l@{}}{SOT layer dimension} \\ {$A_{SOT}$}\end{tabular}} & 
\multirow{2}{*}{\begin{tabular}{@{}l@{}}{$A_{SOT} \downarrow$, $\tau_{p} \downarrow$, Area $\downarrow$, Write Bandwidth $\uparrow$}\end{tabular}}\\
 & \\ \hline
Oxide thickness $t_{MgO}$ & $t_{MgO} \uparrow$, TMR $ \uparrow$, Read Bandwidth $ \uparrow$\\ \hline
\end{tabular}
%\vspace{-10pt}
\end{table}

\subsubsection{Write pulse width $\tau_{p}$}
The width of the write current pulse for switching is inversely proportional to the magnitude of the applied current density in the SOT layer $j_{sw}$ \cite{manchon2019current}
\begin{equation}
    \tau_{p} \propto \frac{1}{j_{sw}}
\label{tau}
\end{equation}

As the area of the SOT layer ($A_{SOT}$) is scaled down, the effective current density increases,  $j_{sw} \propto 1/(A_{SOT})$.  Successful switching should take place when $j_{sw} > j_{c}$. We can increase $j_{sw}$ by reducing the SOT layer dimension and decrease $j_{c}$ by increasing $\theta_{SH}$ or by decreasing $t_{FL}$. Thus we can achieve successful switching in much shorter pulse width (equation \ref{tau}). \cite{garello2014ultrafast} demonstrated the switching at 180ps, \cite{wu2021voltage} at 400ps, and \cite{garello2018sot} at 210ps. Switching in shorter pulse width ensures larger write bandwidth which is essential for memory systems used in AI/Deep Learning hardware. The key DTCO parameters of SOT-MRAM and their impact on Power, Performance and  Area (PPA) are listed in Table \ref{dtco_table}. 




