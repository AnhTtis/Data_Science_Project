\begin{table}[!t]
  \centering
  \resizebox{0.45\textwidth}{!}{
  % \begin{tabular}{@{} >{\raggedright}p{5.2cm} >{\raggedright}p{2.2cm} >{\raggedright}p{2.2cm} >{\raggedright}p{2cm} p{2cm}}
  \begin{tabular}{cccccc}
    \toprule  \\[-1em]
    % && \multicolumn{2}{c}{Unsupervised Adaptation} && Universal Adaptation \\
    % && \multicolumn{2}{c}{UDA} && UniDA \\
    % \cline{3-4} \cline{6-6}
    & Split &                               \GeoP & \GeoI && \GeoU{}  \\
    % \cline{3-6}
    \midrule
    \multirow{2}{*}{USA}    & Train        & 178110 & 154908 && 100136  \\
                            & Test         & 17234  & 16784 &&  25034  \\
                            % & \#Classes    & 205    & - && 186     \\
    \midrule
    \multirow{2}{*}{Asia}   & Train        & 187426 & 68722 && 33912   \\
                            & Test         & 26923  & 9636 && 8478     \\
                            % & \#Classes    & 205    & - && 185      \\
    \midrule
    \multicolumn{2}{l}{classes-shared}     & 205    & 600 &&  62     \\
    \multicolumn{2}{l}{classes-private}    & -    & - &&  138     \\
    \bottomrule \\
  \end{tabular}
  }
  \vspace{-1em}
  \captionsetup{width=0.45\textwidth}
  \caption{\label{tab:dataset-stats} {\bf Summary of \Ours{}} \tk{Number of images in train and test splits in each of our benchmarks. While \GeoP{} and \GeoI{} are developed for unsupervised adaptation, \GeoU{} is developed for universal domain adaptation across geographies.}}
  \vspace{-12pt}
\end{table}