\begin{table}[!t]
  \centering
  \resizebox{0.46\textwidth}{!}{
  % \begin{tabular}{@{} >{\raggedright}p{5.2cm} >{\raggedright}p{2.2cm} >{\raggedright}p{2.2cm} >{\raggedright}p{2cm} p{2cm}}
  \begin{tabular}{l cc c|cc}
    Method & closed-set & open-set & H-Score && Target Sup.  \\
    \midrule
    UniDA~\cite{you2019universal} & 27.64 & 43.93 & 33.93 && \multirow{3}{*}{\textcolor{gray}{70.70\%}}  \\
    DANCE~\cite{saito2020dance} & 38.54 & 78.73 & 51.75 &&  \\
    OVANet~\cite{saito2021ovanet} & 36.54 & 66.89 & 47.26 &&  \\
    \bottomrule
  \end{tabular}
  }
  \vspace{-1em}
  \captionsetup{width=0.45\textwidth}
  \caption{\label{tab:uniDA-benchmarking} {\bf Universal domain adaptation methods on \GeoU{}}. \textit{closed-set} and \textit{open-set} refer to the closed set and open set accuracies, and \textit{H-Score} is the harmonic-mean of the two. Note the significant gap that still exists with target supervised accuracy on closed-set labels with the best adaptation method DANCE~\cite{saito2020dance}.}
  \vspace{-1em}
\end{table}