\vspace{-0.1in}
\section{Physical Experiments}
\vspace{-0.1in}
We evaluate \peralgabbr{} on 1) tracing cables unseen during training, 2) cable inspection in multi-cable settings, 3) learning knot tying from demonstrations, 4) knot detection, and 5) untangling. 

The workspace has a 1 m $\times$ 0.75 m foam-padded, black surface, a bimanual ABB YuMi robot, and an overhead Photoneo PhoXi camera with $773 \times 1032 \times 4$ RGB-D observations. \peralgabbr{} is fed only the RGB image, but the depth data is used for grasping. The PhoXi outputs the same values across all 3, making the observations grayscale and depth. See Appendix, Section \ref{sec:failures} for details on failure modes for each experiment.

\subsection{Using \peralgabbr{} for Tracing Cables Unseen During Training}
\label{sec:gen_appearance}
For this perception experiment, the workspace contains a single cable with one of the following knots: overhand, figure-eight, overhand honda, or bowline. Here, we provide \peralgabbr{} with the two endpoints to test the tracer in isolation, independent of endpoint detection. We report a success if the progression of the trace correctly follows the path of the cable without deviating.
% If, at any point, the trace leaves the cable, we report that as a failure.

\begin{table}[!t]
    \centering
    \scriptsize
    \newcolumntype{?}{!{\vrule width 1.75pt}}
    \caption{Generalization of \peralgabbr{} to Different Cable Types}
    \setlength\tabcolsep{4pt}
    \begin{tabular}{|c|c|c|c|c|c|} \hline 
    % \multicolumn{1}{|c?}{} & \multicolumn{2}{|c|}{{\textbf{Tier B1}}} \\ \hline
    \cellcolor{gray!15}{Cable Reference} & \cellcolor{gray!15}{Length (m)} & \cellcolor{gray!15}{Color} & \cellcolor{gray!15}{Texture} & \cellcolor{gray!15}{Physical Properties} & \cellcolor{gray!15}{\peralgabbr{} Succ. Rate}\\ \hline
    TR (trained with) & 2.74 & White/gray & Braided & Slightly stiff & 6/8\\ \hline
    1 & 2.09 & Gray & Rubbery & Slightly thicker than TR & 7/8\\ \hline
    2 & 4.68 & Yellow with black text & Rubbery and plastic & Very stiff & 8/8 \\ \hline
    3 & 2.08 & Tan & Rubbery & Highly elastic & 7/8 \\ \hline
    4 & 1.79 & Bright red & Braided & Flimsy & 6/8\\ \hline
    5 & 4.61 & White & Braided & Flimsy & 6/8\\ \hline
    \end{tabular}
    \label{tab:gen-trace-cables}
    \vspace*{-0.18in}
\end{table}

Results in Table \ref{tab:gen-trace-cables} show \peralgabbr{} can generalize to cables with varying appearances, textures, lengths, and physical properties. Cables can also be seen in Appendix Figure \ref{fig:generalize_tracer}. \peralgabbr{} performs comparably on cable TR (the cable it was trained with) as it does on the other cables. 

\subsection{Using \peralgabbr{} for Cable Inspection in Multi-Cable Settings}
\label{subsec:multicablesetup}
\begin{figure}[!ht]
    \vspace*{-0.12in}
    \centering
    \includegraphics[width=1.0\linewidth]{figures/multicable_tracing.pdf}
    % \vspace*{-0.19in}
    \caption{\textbf{Multi-cable tracing}: here are 3 pairs of images. The left of each pair illustrates an example from the tier and the right is the successful trace. The traces from left to right encounter 24, 13, and 28 crossings.}
    \label{fig:multicable}
    \vspace*{-0.12in}
\end{figure}

For cable inspection, the workspace contains a power strip. Attached to the power strip are three white MacBook adapters with two 3 m USB C-to-C cables and one 2 m USB-C to MagSafe 3 cable (shown in Figure \ref{fig:multicable}). The goal is to provide the trace of a cable and identify the relevant adaptor given the endpoint. We evaluate perception in multi-cable settings across 3 tiers of difficulty.
\begin{enumerate}
    \item \textbf{Tier A1}: No knots; cables are dropped onto the workspace, one at a time. 
    \item \textbf{Tier A2}: Each cable is tied with a single knot (figure-eight, overhand, overhand honda, bowline, linked overhand, or figure-eight honda) measuring 5-10 cm in diameter, and subsequently dropped onto the workspace one by one.
    \item \textbf{Tier A3}: Similar to tier A2 but contains the following 2-cable knots (square, carrick bend, and sheet bend) with up to three knots in the scene.
\end{enumerate}

Across all 3 tiers, we assume the cable of interest cannot exit and re-enter the workspace and that crossings must be semi-planar. Additionally, we pass in the locations of all three adapters to the tracer and an endpoint to start tracing from. To account for noise in the input images, we take 3 images of each configuration. We count a success if a majority (2 of the 3 images) have the correct trace (reaching their corresponding adapter); otherwise, we report a failure.

% We evaluate the performance of the learned tracer from \peralgabbr{} against an analytic tracer from \citet{shivakumar2022sgtm} as a baseline with scoring rules inspired by the work of \citet{lui2013tangled} and \citet{schaal2022}. The analytic tracer explores all potential paths and determines the most correct trace through a scoring metric from~\cite{shivakumar2022sgtm}. The scoring metric prefers paths that reach an endpoint, discarding traces that do not reach adapters. This is because the scoring metric sees reaching an endpoint as indicative of completing a trace. Of the paths that reach an endpoint, the trace returned is the one with the least sharp angle deviations and the highest coverage score.

We compare the performance of the learned tracer from \peralgabbr{} against an analytic tracer from \citet{shivakumar2022sgtm} as a baseline, using scoring rules inspired by \citet{lui2013tangled} and \citet{schaal2022}. The analytic tracer explores potential paths and selects the most likely trace based on a scoring metric \cite{shivakumar2022sgtm}, prioritizing paths that reach an endpoint, have minimal angle deviations, and have high coverage scores. Table \ref{tab:multi-cable} shows that the learned tracer significantly outperforms the baseline analytic tracer on all 3 tiers of difficulty with a total of 80\% success across the tiers. 

\subsection{Using \peralgabbr{} for Physical Robot Knot Tying from Demonstrations}
\vspace{-0.05in}
\label{sec:demos_setup}

Knot-tying demonstrations are conducted on a nylon rope of length 147 cm and a diameter of 7 mm. We tune our demonstrations to work on plain backgrounds; however, during rollouts, we add distractor cables that intersect the cable to be tied. These distractor cables are of identical types to the cable of interest; thus, manipulating the correct points requires accurate cable state estimation from \peralgabbr{}. Although these cables are thicker, more twisted, and less stiff than the cable \peralgabbr{} is trained with, \peralgabbr{} is able to generalize, tracing the cable successfully in 13 out of 15 cases. We evaluate the policy executed by the YuMi bimanual robot on the following: \textbf{Tier B1:} 0 distractors, \textbf{Tier B2:} 1 distractor, and \textbf{Tier B3:} 2 distractors. We count a success when a knot has been tied (i.e. lifting the endpoints results in a knot's presence).

Table \ref{tab:demo_results} show 86\% tracing success and 80\% robotic knot tying success across the 3 tiers, suggesting \peralgabbr{} can apply policies learned from real world demonstrations, even with distractors.

\subsection{Using \peralgabbr{} for Knot Detection and Physical Robot Untangling}
\vspace{-0.1in}
To test \peralgabbr{} applied to the knot detection and physical untangling task, we use a single 3 m white, braided USB-A to micro-USB cable. 

\vspace{-0.1in}
\subsubsection{Knot Detection}
\label{sec:knot_detect_exp}
\vspace{-0.1in}
The details on the state-based knot detection method, which uses the sequence of over- and under-crossings to identify knots, can be found in the Appendix Section \ref{sec:analytic-knot-detect}.

\begin{figure}[!ht]
    \vspace*{-0.15in}
    \centering
    \includegraphics[width=1.0\linewidth]{figures/start_states.pdf}
    % \vspace*{-0.19in}
    \caption{\textbf{Starting configurations} for the tiers for \peralgabbr{} experiments and robot untangling experiments. Here is the crossing count for these examples from left to right: 5, 10, 6, 4, 9, and 14.}
    \label{fig:start-configs}
    \vspace*{-0.12in}
\end{figure}

We evaluate \peralgabbr{} on 3 tiers of cable configurations, shown in Figure \ref{fig:start-configs}. The ordering of the categories for these experiments does not indicate varying difficulty. Rather, they are 3 categories of knot configurations to test \peralgabbr{} on.
\begin{enumerate}
    \item \textbf{Tier C1}: Loose (35-40 cm in diameter) figure-eight, overhand, overhand honda, bowline, linked overhand, and figure-eight honda knots. 
    \item \textbf{Tier C2}: Dense (5-10 cm in diameter) figure-eight, overhand, overhand honda, bowline, linked overhand, and figure-eight honda knots. 
    \item \textbf{Tier C3}: Fake knots (trivial configurations positioned to appear knot-like from afar). 
\end{enumerate}

We evaluate \peralgabbr{} on the following 3 baselines on the 3 tiers.
\begin{enumerate}
    \item SGTM 2.0 \cite{shivakumar2022sgtm} perception system: using a Mask R-CNN model trained on overhand and figure-eight knots for knot detection.
    \item \peralgabbr{} (-LT): replacing the \underline{L}earned \underline{T}racer with the same analytic tracer from \citet{shivakumar2022sgtm} as described in Section \ref{subsec:multicablesetup} combined with the crossing identification.
    \item \peralgabbr{} (-CC): using the learned tracer and crossing identification scheme to do knot detection without \underline{C}rossing \underline{C}ancellation, covered in the appendix (Section \ref{sec:cc_appendix}).
    % \item \peralgabbr{}: the full perception system applied to knot detection.
\end{enumerate}

We report the success rate of each of these algorithms as follows: if a knot is present, the algorithm is successful if it correctly detects the first knot (i.e. labels its first undercrossing); if there are no knots, the algorithm is successful if it correctly detects that there are no knots. 

\begin{figure}[!t]
\begin{minipage}{0.35\textwidth}
\centering
\scriptsize
\newcolumntype{?}{!{\vrule width 1.25pt}}
\setlength\tabcolsep{4pt}
\captionsetup{justification=centering}
\captionof{table}{Multi-Cable Tracing}
\begin{tabular}{|c?c|c|} \hline 
\cellcolor{gray!15}{Tier} & \cellcolor{gray!15}{Analytic} & \cellcolor{gray!15}{Learned} \\ \hline
A1 & 3/30 & \textbf{26/30} \\ \hline
A2 & 2/30 & \textbf{23/30} \\ \hline
A3 & 1/30 & \textbf{23/30} \\ \hline
\end{tabular}
\caption*{\begin{scriptsize} 
Learned (\peralgabbr{}) compared to an analytic tracer from \citet{shivakumar2022sgtm}.
\end{scriptsize}}
\label{tab:multi-cable}
% \vspace*{-0.12in}
\end{minipage}
\begin{minipage}{0.48\textwidth}
\centering
\scriptsize
\newcolumntype{?}{!{\vrule width 1.0pt}}
\captionsetup{justification=centering}
\captionof{table}{Knot Detection Experiments}
 \label{tab:tusk_exp_no_fail}
\begin{tabular}{|c?c|c|c|c|} \hline 
% \multicolumn{1}{|c?}{} & \multicolumn{4}{|c|}{{\textbf{Tier A1}}} \\ \hline
\cellcolor{gray!15}{Tier} & \cellcolor{gray!15}{SGTM 2.0} & \cellcolor{gray!15}{\peralgabbrabbr{} (-LT)} & \cellcolor{gray!15}{\peralgabbrabbr{} (-CC)} & \cellcolor{gray!15}{\peralgabbrabbr{}} \\ \hline
C1 & 2/30 & {14/30} & 20/30 & \textbf{24/30} \\ \hline
C2 & \textbf{28/30} & 8/30 & 21/30 & 26/30 \\ \hline
C3 & 12/30 & 14/30 & 0/30 & \textbf{19/30} \\ \hline
\end{tabular}
\caption*{\begin{scriptsize} 
\peralgabbrabbr{} = \peralgabbr{}. \peralgabbr{} outperforms the baseline and ablations on all tiers except tier C2. Explanation provided in Section \ref{sec:knot_detect_exp}.
\end{scriptsize}}
\end{minipage}
\end{figure}

As summarized in Table \ref{tab:tusk_exp_no_fail}, knot detection using \peralgabbr{} considerably outperforms SGTM 2.0, \peralgabbr{} (-LT), and \peralgabbr{} (-CC) on tiers C1 and C3. SGTM 2.0 marginally outperforms \peralgabbr{} in tier C2. This is because SGTM 2.0's Mask R-CNN is trained on dense overhand and figure-eight knots, which are visually similar to the knots in tier C2. 

\begin{figure}[!t]
\begin{minipage}{0.3\textwidth}
% \begin{table}
\centering
\scriptsize
\newcolumntype{?}{!{\vrule width 1.25pt}}
\captionsetup{justification=centering}
\captionof{table}{Learning From Demos}
 \label{tab:demo_results}
\setlength\tabcolsep{4pt}
\begin{tabular}{|c?c|c| c} \hline 
\cellcolor{gray!15}{Tier} & \cellcolor{gray!15}{Corr. Trace} & \cellcolor{gray!15}{Succ.} \\ \hline
B1 & 5/5 & 5/5\\  \hline
B2 & 4/5 & 4/5\\  \hline
B3 & 4/5 & 3/5 \\   \hline
\end{tabular}
% \end{table}
\caption*{\begin{scriptsize} 
Corr. Trace = correct trace, or perception result success. Succ. = perception and manipulation success.
\end{scriptsize}}
\end{minipage}
\begin{minipage}{0.65\textwidth}
% \begin{table*}[!t]
    \centering
    \scriptsize
    \newcolumntype{?}{!{\vrule width 1.75pt}}
    \captionsetup{justification=centering}
    \captionof{table}{Robot Untangling Experiments (90 total trials)}
    \setlength\tabcolsep{4pt}
    \begin{tabular}{|c?c|c?c|c?c|c|}
    \hline \multicolumn{1}{|c?}{} & \multicolumn{2}{|c?}{\cellcolor{gray!20}{{\textbf{Tier D1}}}} & \multicolumn{2}{|c?}{\cellcolor{gray!20}{\textbf{Tier D2}}} & \multicolumn{2}{|c|}{\cellcolor{gray!20}{\textbf{Tier D3}}} \\ 
    \hline
    \multicolumn{1}{|c?}{} & {SGTM 2.0} & {\peralgabbrabbr{}} & {SGTM 2.0} & \peralgabbrabbr{} & {SGTM 2.0} & {\peralgabbrabbr{}} \\ \hline
    Knot 1 Succ. & {11/15} & {\textbf{12/15}} & {6/15} & {\textbf{11/15}} & {9/15} & {\textbf{14/15}} \\ \hline
    Knot 2 Succ. & {-} & {-} & {-} & {-} & 2/15 & \textbf{6/15} \\ 
    \hline
    Verif. Rate & {\textbf{11/11}} & 8/12 & \textbf{6/6} & 6/11 & \textbf{1/2} & 2/6 \\
    \hline
    Knot 1 Time (min) & 1.1$\pm$0.1 & 2.1$\pm$0.3 & 3.5$\pm$0.7 & 3.9$\pm$1.1 & 1.8$\pm$0.4 & 2.0$\pm$0.4 \\
    \hline
    Knot 2 Time (min) & {-} & {-} & - & - & 3.1$\pm$1.2 & 7.5$\pm$1.6 \\ 
    \hline
    Verif. Time (min) & 5.7$\pm$0.9 & 6.1$\pm$1.4 & 6.4$\pm$1.8 & 10.1$\pm$0.7 & 5.4 & 9.6$\pm$1.5 \\
    \hline
    \end{tabular}
    \label{tab:results_all}
% \end{table*}
\caption*{\begin{scriptsize} 
\peralgabbrabbr{} = \peralgabbr{}. Across all 3 tiers, \peralgabbr{} outperforms SGTM 2.0 on untangling success. SGTM 2.0, however, outperforms \peralgabbr{} on verification. Details provided in Section \ref{sec:untanging_setup}
\end{scriptsize}}
\end{minipage}
\vspace{-0.2in}
\end{figure}

\vspace{-0.1in}
\subsubsection{Physical Robot Untangling}
\vspace{-0.1in}
\label{sec:untanging_setup}
% We develop an untangling system based on \peralgabbr{}, which composes other perception elements such as endpoint detection with manipulation primitives to enable robust multi-step untangling of cables (details in Appendix Section \ref{sec:untangling_deets}).
 For the untangling system based on \peralgabbr{}, we compare performance against SGTM 2.0 \cite{shivakumar2022sgtm}, the current state-of-the-art algorithm for untangling long cables, using the same 15-minute timeout on each rollout and the same metrics for comparison. We evaluate \peralgabbr{} deployed on the ABB YuMi bimanual robot in untangling performance on the following 3 levels of difficulty (Figure \ref{fig:start-configs}), where all knots are upward of 10 cm in diameter:

\textbf{Tier D1:} Cable with overhand, figure-eight, or overhand honda knot; total crossings $\leq 6$.

\textbf{Tier D2:} Cable with bowline, linked overhand, or figure-eight honda knot; total crossings $\in [6, 10)$.

\textbf{Tier D3:} Cable with 2 knots (1 each from tiers D1 and D2); total crossings $\in [10, 15)$.

Table \ref{tab:results_all} shows that our \peralgabbr{}-based untangling system achieves a higher untangling success rate (29/45) than SGTM 2.0 (19/45) across 3 tiers of difficulty, although SGTM 2.0 is faster. The slower speed of \peralgabbr{} is attributed to the requirement of a full cable trace for knot detection which is inhibited by the fact that the cable is 3$\times$ as long as the width of the workspace, leading to additional reveal moves before performing an action or verifying termination. On the other hand, SGTM 2.0 does not account for the cable exiting the workspace, benefiting speed, but failing to detect off-workspace knots, leading to premature endings of rollouts without fully untangling.
