\documentclass{article}

% \usepackage{corl_2023} % Use this for the initial submission.
\usepackage[final]{corl_2023} % Uncomment for the camera-ready ``final'' version.
% \usepackage[preprint]{corl_2023} % Uncomment for pre-prints (e.g., arxiv); This is like ``final'', but will remove the CORL footnote.

\title{\peralgabbr{}: Learned Tracing of One-Dimensional Objects for Inspection and Manipulation}

\usepackage[usenames,dvipsnames,table,xcdraw]{xcolor}
\usepackage{xspace}
\usepackage{cancel}
\usepackage{soul}

\newcommand{\mallika}[1]{{\color{purple}Mallika: {#1}}}
\newcommand{\jainil}[1]{{\color{green}Jainil: {#1}}}
\newcommand{\vainavi}[1]{{\color{red} {#1}}}
\newcommand{\kaushik}[1]{{\color{red} {#1}}}
\newcommand{\jeff}[1]{{\color{Maroon}Jeff: {#1}}}
\newcommand{\justin}[1]{{\color{magenta}Justin: {#1}}}

% \newcommand{\algname}{Semi-planar Untangling with trace PErception and Robust MANipulation}
% \newcommand{\algabbr}{SUPERMAN}

% \newcommand{\algname}{Tracing for Robotic UNtangling of semi-planar Knots}
% \newcommand{\algabbr}{TRUNK}

\newcommand{\peralgname}{Heterogeneous Autoregressive Learned Deformable Linear Object Observation and Manipulation\xspace}
\newcommand{\peralgabbr}{HANDLOOM\xspace}
\newcommand{\peralgabbrabbr}{HL\xspace}

% \newcommand{\algname}{Semi-planar cable Understanding eNabled through State Estimation via Tracing}
% \newcommand{\algabbr}{SUNSET}

% DR STRANGE: Determining Rope State via TRAcing and crossiNG EXtraction
% TETRIS: Topology Estimation via Tracing and cRossing ISolation
% FALCON: FAst Learned Cable state estimatiON
% SECANT: State Estimation of Cables via Autoregressively learNed Tracing

% TLDR: Topological Learning to Disentangle Rope% TLDR: Topological Learning to Disentangle Rope

\newcommand{\todo}[1]{}
\renewcommand{\todo}[1]{{\color{red} TODO: {#1}}}
% numbers option provides compact numerical references in the text. 
\usepackage[numbers]{natbib}
% \usepackage{multicol}
\usepackage{mathtools}
% \usepackage{enumitem}
\let\labelindent\relax
\usepackage[shortlabels]{enumitem}
% \usepackage{bbm}
% \usepackage{bm}
% \usepackage[left]{caption}
\usepackage{caption}
\captionsetup[figure]{font=footnotesize,labelfont=footnotesize}
\captionsetup[table]{font=footnotesize,labelfont=footnotesize}

\usepackage{amsmath, amssymb, amscd}
\usepackage{algorithm}
\usepackage{algpseudocode}
% \usepackage[bookmarks=true]{hyperref}
\DeclareMathOperator*{\argmax}{argmax}


% The \author macro works with any number of authors. There are two
% commands used to separate the names and addresses of multiple
% authors: \And and \AND.
%
% Using \And between authors leaves it to LaTeX to determine where to
% break the lines. Using \AND forces a line break at that point. So,
% if LaTeX puts 3 of 4 authors names on the first line, and the last
% on the second line, try using \AND instead of \And before the third
% author name.

% NOTE: authors will be visible only in the camera-ready and preprint versions (i.e., when using the option 'final' or 'preprint'). 
% 	For the initial submission the authors will be anonymized.

% \author{
%   Vainavi Viswanath$^{*}$
%   The AUTOLAB\\
%   University of California Berkeley\\
%   United States\\
%   \texttt{vainaviv@berkeley.edu}\\
%   %% examples of more authors
%   \And
%   Kaushik Shivakumar$^{*}$\\
%   The AUTOLAB\\
%   University of California Berkeley\\
%   United States\\
%   \texttt{kaushiks@berkeley.edu}\\
%   \And
%   Mallika Parulekar\\
%   The AUTOLAB\\
%   University of California Berkeley\\
%   United States\\
%   \texttt{mallika.parulekar@berkeley.edu}\\
%   %% examples of more authors
%   \And
%   Jainil Ajmera\\
%   The AUTOLAB\\
%   University of California Berkeley\\
%   United States\\
%   \texttt{jainil@berkeley.edu}\\
%   \And
%   Justin Kerr\\
%   The AUTOLAB\\
%   University of California Berkeley\\
%   United States\\
%   \texttt{justin\_kerr@berkeley.edu}\\
%   \And
%   Jeffrey Ichnowski \\
%   Carnegie Mellon University \\
%   United States \\
%   \texttt{jeffi@cmu.edu} \\
%   \And
%   Richard Cheng \\
%   Toyota Research Institute \\
%   United States \\
%   \texttt{richard.cheng@tri.global} \\
%   \And
%   Thomas Kollar \\
%   Toyota Research Institute \\
%   United States \\
%   \texttt{thomas.kollar@tri.global} \\
%   \And
%   Ken Goldberg \\
%   The AUTOLAB \\
%   University of California Berkeley \\
%   United States \\
%   \texttt{goldberg@berkeley.edu} \\
% }

\author{
 Vainavi Viswanath*$^{1}$, Kaushik Shivakumar*$^{1}$, Mallika Parulekar$^{\dag1}$, Jainil Ajmera$^{\dag1}$, \\
 \textbf{Justin Kerr$^{1}$, Jeffrey Ichnowski$^2$, Richard Cheng$^{3}$, Thomas Kollar$^{3}$,} \\
 \textbf{Ken Goldberg$^{1}$} \\
%   $^{1}$Department of Electrical Engineering and Computer Sciences\\
%   University of California Berkeley 
%   United States\\
%   $^{2}$Toyota Research Institute (TRI) \\
  \scriptsize{* equal contribution}, \scriptsize{$\dag$ equal contribution} \\
  \texttt{vainaviv@berkeley.edu}, \texttt{kaushiks@berkeley.edu}
}

\makeatletter
\def\thanks#1{\protected@xdef\@thanks{\@thanks
        \protect\footnotetext{#1}}}
\makeatother
\thanks{$^{1}$University of California, Berkeley. $^{2}$Carnegie Mellon University. $^{3}$Toyota Research Institute. }

% \thanks{equal contribution}


\begin{document}
\maketitle

%===============================================================================

\begin{abstract}
    Tracing -- estimating the spatial state of -- long deformable linear objects such as cables, threads, hoses, or ropes, is useful for a broad range of tasks in homes, retail, factories, construction, transportation, and healthcare. For long deformable linear objects (DLOs or simply cables) with many (over 25) crossings, we present \peralgabbr\ (\peralgname) a learning-based algorithm that fits a trace to a greyscale image of cables.
    % , hoses, or ropes and resolves over- vs under-crossings to disambiguate cable state for tasks such as inspection, diagnostics, state-based imitation, and untangling.
    We evaluate \peralgabbr on semi-planar DLO configurations where each crossing involves at most 2 segments. \peralgabbr{} makes use of neural networks trained with 30,000 simulated examples and 568 real examples to autoregressively estimate traces of cables and classify crossings. Experiments find that in settings with multiple identical cables, \peralgabbr{} can trace each cable with $80\%$ accuracy. In single-cable images, \peralgabbr{} can trace and identify knots with $77\%$ accuracy. When \peralgabbr{} is incorporated into a bimanual robot system, it enables state-based imitation of knot tying with 80\% accuracy, and it successfully untangles $64\%$ of cable configurations across 3 levels of difficulty. Additionally, \peralgabbr{} demonstrates generalization to knot types and materials (rubber, cloth rope) not present in the training dataset with 85\% accuracy. Supplementary material, including all code and an annotated dataset of RGB-D images of cables along with ground-truth traces, is at \href{https://sites.google.com/view/cable-tracing}{https://sites.google.com/view/cable-tracing}.
    % This work demonstrates the applicability of learned autoregressive models for 1D deformable state estimation and its potential downstream applications.
\end{abstract}

% Two or three meaningful keywords should be added here
\keywords{state estimation, deformable manipulation}

%===============================================================================
\section{Introduction}


Recent years have witnessed the rise of human digitization~\cite{habermannDeepCapMonocularHuman2020,alexanderCREATINGPHOTOREALDIGITAL,pengNeuralBodyImplicit2021,alldieckDetailedHumanAvatars2018, rajANRArticulatedNeural2020}. This technology greatly impacts the entertainment, education, design, and engineering industry.
There is a well-developed industry solution for this task.
High-fidelity reconstruction of humans can be achieved either with full-body laser scans~\cite{saitoSCANimateWeaklySupervised2021}, dense synchronized multi-view cameras~\cite{xiangModelingClothingSeparate2021a,xiangDressingAvatarsDeep2022a}, or light stages~\cite{alexanderCREATINGPHOTOREALDIGITAL}.
However, these settings are expensive and tedious to deploy and consist of a complex processing pipeline, preventing the technology's democratization.

Another solution is to view the problem as inverse rendering and learn digital humans directly from custom-collected data.
Traditional approaches directly optimize explicit mesh representation~\cite{loperSMPLSkinnedMultiperson2015, fangRMPERegionalMultiperson2018, pavlakosExpressiveBodyCapture2019} which suffers from the problems of smooth geometry and coarse textures~\cite{prokudinSMPLpixNeuralAvatars2020,alldieckVideoBasedReconstruction2018}. Besides, they require professional artists to design human templates, rigging, and unwrapped UV coordinates.
Recently, with the help of volumetric-based implicit representations~\cite{mildenhallNeRFRepresentingScenes2020, parkDeepSDFLearningContinuous2019, meschederOccupancyNetworksLearning2019} and neural rendering~\cite{laineModularPrimitivesHighPerformance2020, liuSoftRasterizerDifferentiable2019, thiesDeferredNeuralRendering2019}, 
one can easily digitize a quality-plausible human avatar from video footage~\cite{jiangNeuManNeuralHuman2022,wengHumanNeRFFreeviewpointRendering}.
Particularly, volumetric-based implicit representations~\cite{mildenhallNeRFRepresentingScenes2020, pengNeuralBodyImplicit2021} can reconstruct scenes or objects with much higher fidelity against previous neural renderer~\cite{thiesDeferredNeuralRendering2019,prokudinSMPLpixNeuralAvatars2020}, and is more user-friendly as it does not need any human templates, pre-set rigging, or UV coordinates.
Captured visual footage and corresponding skeleton tracking are enough for training.
However, better reconstructions and more friendly usability are at the expense of the following factors.
1) \textbf{Inefficiency:}
They require longer optimization times (typically tens of hours or days) and inference slowly.
Volume rendering~\cite{mildenhallNeRFRepresentingScenes2020,lombardiNeuralVolumesLearning2019} formulates images by querying the densities and colors of millions of spatial coordinates. 
In the training stage, due to memory constraints, only a small fraction of points are sampled which leads to slow convergence speed.
2) \textbf{Entangled representations}:
The geometry, materials, and motion dynamics are entangled in the neural networks. 
Due to the implicit nature of neural nets, one can hardly edit one property without touching the others~\cite{yuanNeRFEditingGeometryEditing2022a,liuEditingConditionalRadiance2021}.
3) \textbf{Graphics incompatibility}:
Volume rendering is incompatible with the current popular graphic pipeline,
which renders triangular/quadrilateral meshes efficiently with the rasterization technique.
Many downstream applications require mesh rasterization in their workflow (\eg, editing~\cite{foundationBlenderOrgHome}, simulation~\cite{benderPositionBasedSimulationMethods2015}, real-time rendering~\cite{akenine2019real}, ray-tracing~\cite{waldRTXRayTracing}).
Although there are approaches~\cite{lorensenMarchingCubesHigh,labelleIsosurfaceStuffingFast2007} can convert volumetric fields into meshes, the gaps from discrete sampling degrade the output quality in terms of both meshes and textures.


To address these issues, we present \textbf{EMA}, a method based on \textbf{E}fficient \textbf{M}eshy neural fields to reconstruct animatable human \textbf{A}vatars.
Our method enjoys flexibility from implicit representations and efficiency from explicit meshes, yet still maintains high-fidelity reconstruction quality.
Given video sequences and the corresponding pose tracking, our method digitizes humans in terms of canonical triangular meshes, physically-based rendering (PBR) materials, and skinning weights \textit{w.r.t.} skeletons.
We jointly learn the above components via inverse rendering~\cite{laineModularPrimitivesHighPerformance2020,chenDIBRLearningPredict2021,chenLearningPredict3D2019} in an end-to-end manner.
Each of them is derived from a separate neural field, which relaxes the requirements of a preset human template, rigging, or UV coordinates.
Specifically, we predict a canonical mesh out of a signed distance field (SDF) by differentiable marching tetrahedra~\cite{shenDeepMarchingTetrahedra2021,gaoGET3DGenerativeModel,gaoLearningDeformableTetrahedral2020,munkbergExtractingTriangular3D2022}, then we extend the marching tetrahedra~\cite{shenDeepMarchingTetrahedra2021} for spatial-varying materials by utilizing a neural field to predict PBR materials \textit{on the mesh surfaces} after rasterization~\cite{munkbergExtractingTriangular3D2022,hasselgrenShapeLightMaterial2022,laineModularPrimitivesHighPerformance2020}.
To make the canonical mesh animatable, we take another neural field to model the forward linear blend skinning for the meshes. 
Given a posed skeleton, the canonical mesh is then transformed into the corresponding poses.
Finally, we shade the mesh with a rasterization-based differentiable renderer~\cite{laineModularPrimitivesHighPerformance2020} and train our models with a photo-metric loss.
After training, we export the mesh with materials and discard the neural fields.

\looseness=-1
There are several merits of our method design.
1) \textbf{Efficiency}:
Powered by efficient mesh rendering, our method can render in real-time.
Besides, the training speed is boosted as well, 
since we compute loss holistically on the whole image and the gradients only flow on the mesh surface. In contrast, volume rendering takes limited pixels for loss computation and back-propagates the gradients in the whole space.
Our method only needs about an hour of training and minutes of optimization are enough for plausible avatar reconstruction.
2) \textbf{Disentangled representations}:
Our shape, materials, and motion modules are disentangled naturally by design, which facilitates editing. 
Besides, Canonical meshes with forward skinning modeling handle the out-of-distribution poses better.
3) \textbf{Graphics compatibility}:
Our derived mesh representation is compatible with 
the prominent graphic pipeline, which leads to instant downstream applications (\eg, the shape and materials can be edited directly in design software~\cite{foundationBlenderOrgHome}).
To further improve reconstruction quality, we additionally optimize image-based environment lights and non-rigid motions.


We conduct extensive experiments on standards benchmarks H36M~\cite{ionescuHuman36MLarge2014b} and ZJU-MoCap~\cite{pengNeuralBodyImplicit2021}.
Our method achieves very competitive performance for novel view synthesis, generalizes better for novel poses, 
and significantly improves both training time and inference speed against previous arts.
Our research-oriented code reaches real-time inference speed (100+ FPS for rendering $512\times512$ images).
We in addition showcase applications including novel pose synthesis, material editing, and relighting.
\section{Related Work} \label{sec:related work}
\vspace{-0.2cm}
{\noindent \bf Vision-Language Pre-training.} In the early literature, \cite{Mori99,Frome13,Weston11} explore jointly training image-text embeddings using paired text documents. Recently, some studies have further scaled up the training with large-scale web data to form ``the \textbf{foundation} models'', {\em e.g.}, CLIP~\cite{Radford21}, ALIGN~\cite{Jia21}, Florence~\cite{yuan2021florence}, FILIP~\cite{yao2021filip}, VideoCLIP~\cite{xu2021videoclip}, and LiT~\cite{zhai2022lit}. These foundation models usually contain one visual encoder and one textual encoder, which are trained using simple noise contrastive learning for powerful cross-modal representations. They have shown promising potential in many tasks, such as image classification and detection, action recognition, and retrieval. In this paper, we use CLIP for low-shot temporal action localization, but the same technique should be applicable to other foundation models as well.



\vspace{0.1cm}
{\noindent \bf Prompting} refers to leveraging input instructions to steer foundation models for desired outputs. In the NLP domain, early papers~\cite{Gao21,Jiang20,Timo21,Shin20} focus on handcrafted prompt templates. To avoid labor and increase flexibility, some studies~\cite{Lester21,li21-prefixtuning,li2021prefix} propose learnable prompt tuning at the textual stream, showing strong low-shot generalization. In the CV domain, some recent papers~\cite{zhou2019learn,zhou2022conditional,ju2022prompting} introduce such randomly initialized prompt tuning to handle visual tasks, {\em e.g.}, image understanding~\cite{zhu2022prompt,lu2022prompt,yang2022learning,ma2023diffusionseg} and video understanding~\cite{jia2022visual,nag2022zero,ni2022expanding}. However, these studies ignore lexical ambiguity of category names, and cases that are not easy to describe in text. This paper designs novel conditional prompt tuning and language descriptions from LLMs, to solve these issues. 



\vspace{0.1cm}
{\noindent \bf Closed-set Temporal Action Localization} considers to detect and classify action instances from one pre-defined category list. Specifically, existing methods can be divided into two popular supervisions, {\em i.e.}, strong~\cite{zeng2019graph,lin2021learning,qing2021temporal} and weak~\cite{wang2017untrimmednets,ju2023constraint,ju2020point,yudistira2022weakly}. Strong supervision gives precise boundary labels and category labels for training. There are two detailed pipelines: the top-down framework~\cite{shou2016temporal,shou2017cdc,gao2017turn,chao2018rethinking,lin2017single,xu2017r,tan2021relaxed,zhu2021enriching,wang2022rcl,xu2020g} pre-defines extensive anchors, adopts fixed-length sliding windows to produce initial proposals, then regresses to refine boundaries; the bottom-up framework~\cite{zhao2017temporal,lin2018bsn,lin2019bmn,vo2023aoe,zhao2020bottom,bai2020boundary} learns frame-wise boundary detectors for the boundary frames, then groups extreme frames or estimates action lengths for proposal generation. In addition, several works~\cite{gao2018ctap,liu2019multi,yang2020revisiting} used various fusion strategies to complement these frameworks. On the other hand, weak supervision trains without boundary labels to alleviate annotation costs. The video-level setting learns from category labels~\cite{paul2018w,ju2022distilling}, the CAS-based framework~\cite{liu2019completeness,ju2021adaptive,min2020adversarial,narayan2021d2,lee2019background,lee2021weakly,zhao2021soda} and attention-based framework~\cite{nguyen2018weakly,luo2021action,nguyen2019weakly,shi2020weakly,gao2022fine,he2022asm,huang2021foreground,luo2020weakly,ma2022weakly} have been well studied. To generate better results from CAS or attention, some studies~\cite{shou2018autoloc,liu2019weakly} improved post-processing. To balance cost and performance, some papers introduced single-frame annotations~\cite{ju2021divide,ma2020sf,lee2021learning,yang2021background,mettes2019pointly} or instance-number annotations~\cite{narayan20193c,xu2019segregated}. 

Nevertheless, all the above methods assume that action categories remain identical for training and testing, which is an over-simplification of real application scenarios, limiting practical uses of the vision system.



\vspace{0.1cm} 
{\noindent \bf Low-Shot Temporal Action Localization} considers more realistic scenarios: generalize TAL towards action categories that are unseen (zero-shot) or with several support samples (few-shot). Existing methods~\cite{ju2022prompting,nag2022zero,zhang2022ow,bao2022opental} most rely on foundational models pre-trained on large-scale image-caption pairs for help. Typically, E-Prompt~\cite{ju2022prompting} is the first to construct wide baselines with popular prompt tuning~\cite{Lester21,li21-prefixtuning} and vanilla temporal modeling. STALE~\cite{nag2022zero} explores the one-stage framework to further simplify usage. Although promising, all above methods meet two main challenges: (1) For category semantics, the definition may be vague, inaccurate, or incomplete. (2) For visual motions, temporal modeling may be insufficient. In this paper, for detailed category understanding, we design novel language descriptions from LLMs and vision-conditional prompt tuning; for clearer motion understanding, we introduce optical flows to provide explicit motion inputs. 




\section{Problem Statement}
\label{sec:ps}

The objective is to bring a long (3 m) cable containing semi-planar knots into an untangled configuration, where no knots remain (knots defined in Section \ref{sec:knot_def}).

% The objective for the robot is to untangle a long (3 m) cable consisting of any semi-planar knot from overhead RGB image observations. We use a bimanual robot to execute manipulation primitives until the cable reaches a fully untangled state with no knots.

The workspace is defined by an $(x, y, z)$ coordinate system and consists of a bilateral robot and a foam-padded manipulation surface, which lies in the $(x, y)$ plane. The workspace also contains a fixed overhead RGB-D camera that faces the manipulation surface and outputs grayscale images and depth data. However, depth data is not used in \peralgabbr{}. Rather, it is only used during manipulation. We work with a 300 cm cable. We assume the cable is visually distinguishable from the manipulation surface, its initial configuration has at least one endpoint visible, and is semi-planar as assumed in \citet{grannen2020untangling}, meaning each crossing in the knot has at most 2 intersecting cable segments. For perception experiments, we work with knots as tight as 5\,cm in diameter. For physical experiments, due to robot graspability constraints, we work with knots of varying density, or approximate diameter, upwards of 10 cm in diameter. We define cable state to be $\theta(s) = \{(x(s), y(s), z(s))\}$ where $s$ is an arc-length parameter that ranges $[0, 1]$, representing the normalized length of the cable. Here, $(x(s), y(s), z(s))$ is the location of a cable point at a normalized arc length of $s$ from the cable's first endpoint. We also define the range of $\theta(s)$---that is, the set of all points on the cable at time $t$---to be $\mathcal{C}_t$.  

% \todo{I don't think we should define things like this. We should be looking at the cable configuration as a whole} Note that although the knot itself is semi-planar, crossings in the cable can consist of more than 2 cable segments as slack can fall on top of the knot or form loops elsewhere in the cable such that crossings with more than 2 cable segments appear. \todo{Check note above + definitions below. Also, highlight in methods? That is, the fact that the tracer can handle more than 2 crossings}.

% \subsection{Cable Segment Definition (For Semi-Planar Knots)}
% Suppose there exists a crossing on the cable path at time $t$. Owing to the semi-planar nature of the knot, there will be two sightings of this crossing (once as an undercrossing, once as an overcrossing) on the cable path $P_t(s)$. Assume that the two indices at which this crossing is encountered on $P_t(s)$ are $a$ and $b$ ($a < b$). 

% We define the cable segments intersecting at this crossing to be $s_1$ and $s_2$. Here, $s_1$ and $s_2 \subset C_t$ such that $s_1 = \{P_t(s) \mid s \in [0, a]\}$ and $s_2 = \{P_t(s) \mid s \in [b, 1]\}$.

\subsection{Knot Definition}
\label{sec:knot_def}
Consider a pair of points $p_1$ and $p_2$ on the cable path at time $t$ with ($p_1, p_2 \in \mathcal{C}_t)$. Knot theory strictly operates with closed loops, so to form a loop with the current setup, we construct an imaginary cable segment with no crossings joining $p_1$ to $p_2$ \cite{reidemeister1983knot}. This imaginary cable segment passes above the manipulation surface to complete the loop between $p_1$ and $p_2$ (``$p_1\rightarrow p_2$ loop").
A knot exists between $p_1$ and $p_2$ at time $t$ if no combination of Reidemeister moves I, II (both shown in Figure \ref{fig:reid_cc}), and III can simplify the $p_1 \rightarrow p_2$ loop to an unknot, i.e. a crossing-free loop. In this paper, we aim to untangle semi-planar knots. For convenience, we define an indicator function $k(s):[0,1]\rightarrow\{0,1\}$ which is 1 if the point $\theta(s)$ lies between any such points $p_1$ and $p_2$, and 0 otherwise.

\begin{figure}[!t]
    \centering
    \includegraphics[width=1.0\linewidth]{figures/crossing_cancellation.pdf}
    \caption{\textbf{Reidemeister Moves and Crossing Cancellation}: Top left depicts Reidemeister Move II. Top right depicts Reidemeister Move I. The bottom row shows that by algorithmically applying Reidemeister Moves II and I, we can cancel trivial loops, even if they visually appear as knots.}
    \label{fig:reid_cc}
    % \vspace*{-0.25in}
\end{figure}

 
Based on the above knot definition, this objective is to remove all knots, such that $\int k(s)_0^1=0$. In other words, the cable, if treated as a closed loop from the endpoints, can be deformed into an unknot. We measure the success rate of the system at removing knots, as well as the time taken to remove these knots. 
\section{Methodology}
\label{method}
Our goal is to recover decomposed geometry surface of the objects and background within a scene from the images and semantic masks inputs. To this end, we first review the SDF-based neural implicit representation and how to use the semantic logits for compositional reconstruction in Section~\ref{method-background}. Next, we propose two types of regularizations on unobserved regions to address the partial observation problem: patch-based background smoothness~(Section~\ref{method-bgloss}) and object-background relation~(Section~\ref{method-obj-scene}). Finally, we introduce the overall optimization procedure in Section~\ref{method-train}. An overview of our method is provided in Fig~\ref{fig:method}.

\subsection{Background}
\label{method-background}

\boldparagraph{Volume Rendering of SDF-based Implicit Surface}
For implicit reconstruction, the geometry of the scene is represented as the signed distance function~(SDF) $s(\bp)$ of each spatial point $\bp$, which is the point's distance to the closest surface. In practice, the SDF function is implemented as a multi-layer perceptron~(MLP) network $f(\cdot)$. The appearance of the scene is also defined as an MLP $g(\cdot)$:
\begin{equation}
    \begin{aligned}
        f & : \bp\in\nR^3 \mapsto (s\in\nR,\bff\in\nR^{256}) \\
        g & : (\bp\in\nR^3, \bn\in\nR^3, \bv\in\nS^2, \bff\in\nR^{256}) \mapsto \bc\in\nR^3
    \end{aligned}
\end{equation}
where $\bff$ is a geometry feature vector, $\bn$ is the normal at $\bp$, $\bv$ is the viewing direction and $\bc$ is the view-dependent color.
We adopt the unbiased rendering proposed in \cite{wang2021neus} to render the image. 
For each camera ray $\br=(\bo, \bv)$ with $\bo$ as the ray origin, $n$ points $\{\bp(t_i) = \bo + t_i\bv|i=0,1,\dots,n-1\}$ are sampled, and the pixel color can be approximated as: 
\begin{equation}
    \hat{\bC}(\br) = \sum_{i=0}^{n-1}T_i\alpha_i\bc_i.
    \label{eq-volume-rendering}
\end{equation}
The $T_i$ is the discrete accumulated transmittance derived from $T_i=\prod_{j=0}^{i-1}(1-\alpha_j)$, and $\alpha_i$ is the discrete density value defined as 
\begin{equation}
    \alpha_i = \max\left(\frac{\Phi_u(s(\bp(t_i))) - \Phi_u(s(\bp(t_{i+1})))}{\Phi_u(s(\bp(t_i)))}, 0\right),
\end{equation}
where $\Phi_u(x) = (1+e^{-ux})^{-1}$  and $u$ is a learnable parameter. By minimizing the difference between predicted and ground-truth pixel colors, we can learn the SDF and appearance function of the desired scene.

\boldparagraph{Learning SDF with Semantic Logits}
In this work, we consider compositional reconstruction of $k$ objects given their masks. Note that we also consider the background as an instantiated object for brevity as in \cite{wu2022object} and follow their network structure.
In detail, for a scene with $k$ objects, the SDF MLP $f(\cdot)$ now outputs $k$ SDFs at each point, and the $j$-th~($1\leq j\leq k$) SDF represents the geometry of $j$-th object. Without loss of generality, we set $j=1$ as the background category and others for objects in Fig.~\ref{fig:method} and the rest of the paper.
The \textit{scene} SDF is the minimum of $\{s_j\}$, which is used for sampling points along the ray and aforementioned volume rendering~(Eq.~\ref{eq-volume-rendering}). Moreover, each point's $k$ SDFs can be transformed into semantic logits $\bh(\bp)$ as
\begin{equation}
    \begin{aligned}
        h_j(\bp) &= \gamma / (1 + \exp(\gamma\cdot s_j(\bp))), \\
        \bh(\bp) &= [h_1(\bp), h_2(\bp),...,h_k(\bp)],
    \end{aligned}
\end{equation}
where $\gamma$ is a fixed parameter. Using volume rendering to accumulate the semantic logits of all the points along a ray, we can get the semantic logits $\bH(\br)\in \nR^k$ of each pixel. During training, the cross-entropy loss applied to $\bH(\br)$ is backpropagated to the SDF values, allowing for learning the compositional geometry.

\subsection{Patch-based Background Smoothness}
\label{method-bgloss}

Although the volume rendering can propagate gradients along the entire ray, the optimization mainly focuses on the surface-hitting point, as its accumulated weight can be much larger than the others. As a result, the geometry of the points behind the first-hit surface can not be optimized correctly. In the indoor scenes, the occluded part of the background surface is invisible in all the images, 
% and this part of the background surface 
which
can be with holes and random artifacts~(see Fig.~\ref{fig:intro-objsdf-edit}).

Since we cannot tell the exact color, depth or normal of the occluded part, it is intractable to optimize this region \wrt its ground truth. Therefore, we propose to regularize the geometry of the occluded background to be \textit{smooth}, thus preventing some clearly wrong artifacts.

In detail, we regularize the smoothness of rendered depth and normal of background surface within a small patch region. To save the computation budget, we randomly sample a $\cP\times \cP$ size patch every $\cT_{\cP}$ iterations in the given image and sample points along the patch rays using the \textit{background SDF} only. 
We compute depth map $\hat{D}(\br)$ and normal map $\hat{N}(\br)$ of the sampled patch following \cite{yu2022monosdf}, $\br$ denotes the sampled ray in patch. The semantic map of the patch is also computed and transformed into a patch Mask $\hat{M}(\br)$:
\begin{equation}
    \hat{M}(\br) = \mathbbm{1}[\arg\max (\bH(\br)) \neq 1],
\end{equation}
which means the mask value is $1$ if the rendered class is not the background, so that only the occluded background is regularized. Taking rendered depth as an example, the patch-based background smoothness loss is
\begin{equation}
\label{eq-smooth-depth-loss}
    \begin{aligned}
        \cL(\hat{D}) =& \sum_{d=0} ^3 \sum_{m,n=0}^{\cP-1-2^d} \hat{M}(\br_{m,n}) \odot (|\hat{D}(\br_{m,n}) - \\
        &\hat{D}(\br_{m,n+2^d})| + |\hat{D}(\br_{m,n}) - \hat{D}(\br_{m+2^d,n})| ).
    \end{aligned}
\end{equation}
Here the smoothness is applied on different intervals controlled by $d$. $m$ and $n$ are the pixel indices within the patch and the mask is multiplied at each position with hadamard product $\odot$. The normal smoothness loss $\cL(\hat{N})$ can be obtained similarly. We define the overall background smoothness loss $\cL_{bs}$ as:
\begin{equation}
    \cL_{\text{bs}} = \cL(\hat{D}) + \cL(\hat{N}).
\end{equation}
Here in contrast to \cite{niemeyer2022regnerf} which applies a patch-based regularization to visible regions in other views, we instead regularize the \textit{occluded} regions of the background.

\begin{figure}[!htbp]
\begin{center}
\includegraphics[width=\linewidth]{contents/figs/fig4_new.pdf}
\end{center}
   \caption{\textbf{Toy-case analysis.} A bird-eye view of an object against the background is shown here, the SDF and surface of object is shown in red and background's in blue. (a)~In \cite{wu2022object} the minimum SDF is used for volume rendering and semantic loss, so on object's visible surface where object SDF is optimized to $0$ and the background SDF here is positive, and similar for the visible background surface. Since the scene is partially observed from left, the right part of object surface is open and unobserved background region is unconstraint. (b)~With the smoothness prior the background surface can be plausible, (c)~and object point SDF loss close the object surface but still have intersections, (d)~finally the reversed depth loss optimizes the entire ray thus the object can be within the background.}
% \vspace{-1em}
\label{fig:obj-case-vis}
\end{figure}

\subsection{Object-background Relation as Regularization}
\label{method-obj-scene}

With the help of the patch-based background smoothness loss, most artifacts of the background are resolved, yielding a smooth surface. We further leverage this smooth background surface to regularize the SDF fields of other objects, leveraging a key prior knowledge that \textit{all the other objects are confined to the room}, \ie, the background surface.

In the original framework of \cite{wu2022object}, if an object is partially observed, \eg against the background, the object's reconstructed surface won't be a ``closed" surface~(see Fig.~\ref{fig:intro} and Fig.~\ref{fig:intro-objsdf-edit}).
We refer to the toy-case analysis shown in Fig.~\ref{fig:obj-case-vis} for the reason of the current unsatisfactory reconstruction.
To encourage the reconstruction to be \textit{watertight}, we design two types of regularization on the SDF fields of objects.

\boldparagraph{Object Point-SDF Loss}
A straightforward solution is to directly regularize the objects' SDFs on every sampled point behind the background surface. To be more specific, the regular volume rendering only guarantees a single change of the sign (positive SDF to negative) when a ray hits a visible surface. For watertight objects, the ray should hit another occluded surface (negative SDF to positive). With the prior that the object is confined within the room, the occluded object surface should be closer than the background surface, meaning that the object SDFs of points behind the background surface should all be positive~(Fig.~\ref{fig:obj-case-vis}~(c)).

We implement an object point-SDF loss based on the above analysis. For the sampled points along the rays, we first utilize the root-finding algorithm among the background SDF of these points and find the zero-SDF ray depth $t'$. Then the object point-SDF loss can be formulated as
\begin{equation}
    \cL_{\text{op}} = \frac{1}{k-1}\sum_{j=2}^{k}\max\left(0, \epsilon - \bs_j(\bp(t_i))\right)\cdot\mathbbm{1}[t_i > t'] ,
\end{equation}
which pushes the objects' SDFs at points behind the surface larger than a positive threshold $\epsilon$.

\boldparagraph{Reversed Depth Loss}
Although the $\cL_{\text{op}}$ can effectively regularize the SDF fields of each object, in practice we find the reconstructed object surface can still have intersections with the background surface~(Fig.~\ref{fig:obj-case-vis}~(c)). The reason is that the sampled points are discrete and in most cases, the background surface is between two sampled points. Therefore, the sign change of the occluded surface may still occur after hitting the background surface.

Since per-point optimization can not effectively propagate to the distribution of the entire ray. To optimize the entire ray's SDF distribution for better regularization, we compute a \textit{reversed depth} along each ray. With the help of $\cL_{\text{op}}$, the sign of the object SDF along one ray now is positive-negative-positive, which enables rendering a depth value backward. We first transform the ray depth $\{t_i|i=0,1,\dots,n-1\}$ into the reversed ray depth named $\{\hat{t}_i|i=0,1,\dots,n-1\}$, where
\begin{equation}
    \hat{t}_i = (t_0 + t_{n-1}) - t_{n-1-i}.
\end{equation}
With the reversed ray depth values, we use the background and object SDF both in reversed order to compute the accumulated weight and get the reversed depth respectively. Remarkably, in order to compute the exact correct depth, the points should be re-sampled along the reverse direction. 
Here we directly use the sampled points to avoid computation overhead, and empirical results prove its effectiveness. We only compute the reverse depth of one pixel if satisfying two conditions: 1)~this pixel's $\hat{M}(\br)=1$; 2)~the SDF value of the rendered object at the furthest point is positive. Note that the second condition is usually satisfied when the object point-SDF loss is applied. By computing the reversed depth $d_o$ of the hitting object~(determined by the pixel's rendered semantic) and $d_b$ of the background, we can get the reversed depth loss:
\begin{equation}
    \cL_{\text{rd}} = \max(0, d_b - d_o),
\end{equation}
which pushes the object surface within the background as illustrated in Fig.~\ref{fig:obj-case-vis}~(d).

\subsection{Training Objectives Details}
\label{method-train}

Monocular geometric cues are essential for indoor scene reconstruction as proved in~\cite{yu2022monosdf}. Following \cite{yu2022monosdf}, we add the depth and normal consistency loss~($\cL_{\text{D}}$,$\cL_{\text{N}}$) with pseudo ground truth from Omnidata~\cite{eftekhar2021omnidata} model. In experiments, we also add the monocular cues on \cite{wu2022object} to get a stronger baseline for a fair comparison.

The SDF network is also regularized by an Eikonal~\cite{gropp2020implicit} loss item $\cL_{\text{E}}$. We further use the semantic loss $\cL_{\text{S}}$ proposed in \cite{wu2022object} to learn the compositional geometry. The overall loss function for compositional reconstruction is:
\begin{equation}
    \begin{aligned}
        \cL =& \cL_{\text{RGB}} + \lambda_{\text{D}}\cL_{\text{D}} + \lambda_{\text{N}}\cL_{\text{N}} + \lambda_{\text{E}}\cL_{\text{E}} + \lambda_{\text{S}}\cL_{\text{S}} \\
    &+ \lambda_{\text{bs}}\cL_{\text{bs}} + \lambda_{\text{op}}\cL_{\text{op}} + \lambda_{\text{rd}}\cL_{\text{rd}}.
    \end{aligned}
\end{equation}
We set $\lambda_{\text{bs}},\lambda_{\text{op}},\lambda_{\text{rd}}=0.1$ in our experiments. 
% The detailed weight setting and calculation of other losses can be found in supplementary.
For other loss terms, given the weight of RGB reconstruction loss as $1$, we set $\lambda_{\text{D}} = 0.1$ and $\lambda_{\text{N}} = \lambda_{\text{E}} = 0.05$ following \cite{yu2022monosdf}. For the semantic loss, we follow the implementation in \cite{wu2022object} and set $\lambda_{\text{D}} = 0.04$.
The detailed calculation of other losses can be found in supplementary.

% \boldparagraph{Implementation Details}
% We implement our method in PyTorch~\cite{paszke2019pytorch}. We use the Adam~\cite{kingma2014adam} optimizer with a learning rate of 5e-4 for 50k iterations and sample 1024 rays per iteration. The weight initialization scheme for SDF MLP is identical to \cite{yariv2021volume,wang2021neus,wu2022object}. The $u$ is initialized as $0.05$ and we set $\gamma$=$20$ as proposed in \cite{wu2022object}. $\cP$, $\cT_{\cP}$ and $\epsilon$ are set as $32$, $10$ and $0.05$ respectively. All the reconstructions are acquired by using marching cube~\cite{lorensen1987marching} at the resolution of $512$. 
% % More implementation details can be found in the supplementary.

% Moreover, we adopt the geometric initialization~\cite{gropp2020implicit} for the geometry network, which initializes the reconstruction with a unit sphere and the surface normals are facing inside at the beginning of the optimization. For the ScanNet dataset, we follow the protocol in \cite{yu2022monosdf} to crop the input image to $384\times 384$ and adjust the intrinsics accordingly. For the synthetic dataset, we directly render the image at the same resolution. Since we focus on the indoor scenes in this paper, we adopt the common practice to process the ``bounded'' scene. For each scene, we normalize the camera poses so that all the cameras lie in a unit sphere. The rays intersection, \ie the furthest sampling location, is computed based on this sphere and we also conduct Marching Cubes~\cite{lorensen1987marching} within the same area for the final reconstruction.

\section{EXPERIMENTS AND ANALYSIS}
\label{results}
\subsection{Experiment Settings}
The ConvS2S model has 512 hidden units for both encoders and decoders. All embeddings, including the output produced by the decoder before the final linear layer, have a dimensionality of 768. This setup allows the encoders to concatenate with patch embeddings from ViT model. To avoid overfitting, dropout is applied on the embeddings, decoder output, and the input of the convolutional blocks with a retaining probability of 0.5.


% We train the convolutional model using Adam optimizer with a fixed learning rate 2.50e-4.
Many experiments are carried out in order to evaluate the proposed approach toward the VLSP-EVJVQA challenge. We begin by initializing the baseline result of ConvS2S without using any image information. This mean that the generated answers are completely based on the answer-question dependencies learned by the model during the training phase. We then sequentially add hint and image features to the input sequence and study their effect on the overall performance. Because of the limitation in computational resources as well as the strict timeline of the competition, we only deploy the fine-tuned ViLT-B/32 with 200K pretraining steps and pre-trained OFA$_{\mathrm{large}}$ with 472M parameters for hints inference given the question and image.
To have the comparative result, we set up the same hyperparameters for all experiments. The models are trained in 30 epochs using Adam optimizer with a fixed learning rate of 2.50e-4 and batch size of 128. After each epoch, the performance loss on the train and development sets is calculated using the Cross-Entropy Loss function.

The proposed architecture and SOTA vision and language models are implemented in PyTorch and trained on the Kaggle platform with hardware specifications: Intel(R) Xeon(R) CPU @ 2.00GHz; GPU Tesla P100 16 GB with CUDA 11.4.

\subsection{Experimental Results}
\begin{table}[H]

    \centering
    \resizebox{\columnwidth}{!}{%
    \setlength{\tabcolsep}{5pt}
    \renewcommand{\arraystretch}{1.2}
    \begin{tabular}{lcccccc}
    \toprule
        \textbf{Model} & \textbf{F1} & \textbf{BLEU-1} & \textbf{BLEU-2} & \textbf{BLEU-3} & \textbf{BLEU-4} & \textbf{BLEU (Avg.)}  \\ \midrule
        ConvS2S (no image features) & 0.3005 &0.2592	&0.2034	&0.1677	&0.1425& 0.1932  \\ \midrule
        ConvS2S + ViLT-B/32 & 0.3294 &0.2692	&0.2109	&0.1723	&0.1446& 0.1993  \\ 
        ConvS2S + OFA$_{\mathrm{large}}$ & 0.3331 &0.2858	&0.2269	&0.1876	&0.1598 & 0.2150  \\ 
        \textbf{ConvS2S + ViLT-B/32 + OFA$_{\mathbf{large}}$}
        % \tablefootnote{This model is not yet evaluated on the private test set \label{note1}}
        & \textbf{0.3442} &0.2797	&0.2205	&0.1808	&0.1529& \textbf{0.2085}  \\ 
        \midrule
                ConvS2S + ViT-B/16 & 0.3109 &0.2683	&0.2119	&0.1747	&0.1480 & 0.2007  \\ %\midrule
        ConvS2S + ViT-B/16 + ViLT-B/32 & 0.3361 &0.2833	&0.2243	&0.1845	&0.1564 & 0.2122  \\ 
        ConvS2S + ViT-B/16 + OFA$_{\mathrm{large}}$ & 0.3390 &0.2877	&0.2276	&0.1877	&0.1593 & 0.2156  \\
        \textbf{ConvS2S + ViT-B/16 + ViLT-B/32 + OFA$_{\mathbf{large}}$}
        % \textsuperscript{\ref{note1}}
        & \textbf{0.3442} & 0.2747	&0.2148	&0.1747	& 0.1465& \textbf{0.2027} \\ \bottomrule
    \end{tabular}}
    \caption{Performance of ConvS2S with different combinations of pre-trained models on the public test set.}
    \label{result_public}
\end{table}

\begin{figure}[ht]
\centering
% \subfloat[ConvS2S training loss per epoch]{%
%   \includegraphics[width=0.495\textwidth]{figure/train_loss1.pdf}%
% }
% \hspace{-0.2em}
% \subfloat[ConvS2S testing loss per epoch]{%
%   \includegraphics[width=0.495\textwidth]{figure/test_loss1.pdf}%
% }
\includegraphics[width=\textwidth]{figure/all_loss.pdf}
\caption{Training loss and public testing loss comparison of ConvS2S model with different combinations of hint and image features}
\label{loss}
\end{figure}


The two metrics: F1 and BLEU, are used in the challenge to evaluate the results. The BLEU score is the average of BLEU-1, BLEU-2, BLEU-3, and BLEU-4. F1 is used for ranking the final results. Table \ref{result_public} presents the performance of the proposed ConvS2S model with different combinations of pre-trained models on the UIT-EVJVQA public test set.

% First, with only question as input, ConvS2
According to Table \ref{result_public}, the original ConvS2S model without image features but using only question obtained 0.3005 by F1 and 0.1932 by BLEU. When integrating hint features from images, the F1 score improved at least 2.89\% and achieve highest result with 0.3442 by F1 and 0.2085 by BLEU when using both ViLT and OFA hints. After adding image feature from ViT-B/16, the performance of previous models tend to improve. However the final ensemble does not surpass the ConvS2S{\tiny~}+{\tiny~}ViLT-B/32{\tiny~}+{\tiny~}OFA$_{\mathrm{large}}$ ensemble on F1 metrics and even give lower BLEU score. Based on F1, these two ensembles are considered as our best models on the public test set. 
Figure \ref{loss} depicts the gradual improvement in both training loss and testing loss as more image features are added to the ConvS2S model. Memory-based ConvS2S does not catch the image context and thus have the highest loss. Though ConvS2S with ViT+VILT features does not obtained a competitive result on F1 and BLEU score, it has the best loss among methods in the public test phase. In general, the optimal testing loss of methods is achieved between 14th and 20th epoch, then the models tend to be overfitting.


% \begin{figure}
%     \centering
%     \includegraphics[width=\textwidth]{figure/train_loss.pdf}
%     \caption{tmp}
%     \label{100score}
% \end{figure}
% \subsubsection{Qualitative analysis}
% \label{quali_analysis}

% \begin{figure}
%     \centering
%     \includegraphics[width=\textwidth]{figure/test_loss.pdf}
%     \caption{tmp}
%     \label{100score}
% \end{figure}
% \subsubsection{Qualitative analysis}
% \label{quali_analysis}

We manage to deploy two ensembles of ConvS2S using features from ViT-B/16 combined with hints from {\tiny~}ViLT-B/32 and {\tiny~}OFA$_{\mathrm{large}}$, respectively, for the final evaluation on private test set. As shown in Table \ref{result_private}, the ConvS2S{\tiny~}+{\tiny~}ViT-B/16{\tiny~}+{\tiny~}OFA$_{\mathrm{large}}$ model obtained the better result, which is 0.4210 by F1 and 0.3482 by BLEU, and ranked $3^{rd}$ in the challenge. Table \ref{ranking} shows the final standing at the EVLSP-EVJVQA competition, in which our best model perform poorer 1.82\% and 1.39\% by F1 compared with the first and second place solutions. Overall, there is a gap between F1 and BLEU scores.



\begin{table}[H]
    \centering
    \small
    %\resizebox{\columnwidth}{!}{%
    \setlength{\tabcolsep}{5pt}
    \renewcommand{\arraystretch}{1.2}
    \begin{tabular}{lcc}
    \toprule
        \textbf{Model} & \textbf{F1} & \textbf{BLEU}  \\ \midrule
        ConvS2S + ViT-B/16 + ViLT-B/32 &0.4053  &0.3228  \\
        \textbf{ConvS2S + ViT-B/16 + OFA$_{\mathbf{large}}$} & \textbf{0.4210}  & \textbf{0.3482}
  \\ \bottomrule
    \end{tabular}
    \caption{Performance on the private test set.}
    \label{result_private}
\end{table}

\begin{table}[!htbp]
\small
%\resizebox{\columnwidth}{!}{%
\centering
\begin{tabular}{clccccc}
\toprule
\multirow{2}{*}{\textbf{No.}} & \multirow{2}{*}{\textbf{Team name}} & \multicolumn{2}{c}{\textbf{Public Test}} && \multicolumn{2}{c}{\textbf{Private Test}} \\\cmidrule{3-4} \cmidrule{6-7}
                             &                                     & \textbf{F1}         & \textbf{BLEU}      && \textbf{F1}         & \textbf{BLEU}       \\\midrule
1                            & CIST AI                             & 0.3491              & 0.2508             && 0.4392              & 0.4009              \\
2                            & OhYeah                              & 0.5755              & 0.4866             && 0.4349              & 0.3868              \\
3                            & \textbf{DS\_STBFL}                  & \textbf{0.3390}     & \textbf{0.2156}    && \textbf{0.4210}     & \textbf{0.3482}     \\
4                            & FCoin                               & 0.3355              & 0.2437             && 0.4103              & 0.3549              \\
5                            & VL-UIT                              & 0.3053              & 0.1878             && 0.3663              & 0.2743              \\
6                            & BDboi                               & 0.3023              & 0.2183             && 0.3164              & 0.2649              \\
7                            & UIT\_squad                          & 0.3224              & 0.2238             && 0.3024              & 0.1667              \\
8                            & VC\_Internship                      & 0.3017              & 0.1639             && 0.3007              & 0.1337
\\\bottomrule       
\end{tabular}
\caption{Our performance compared with other teams at VLSP2022-EVJVQA}
\label{ranking}
\end{table}

\subsection{Performance Analysis}

According to the final result in the private test phase, the generated output from ConvS2S
+ViT-B/16+OFA$_{\mathrm{large}}$ model are chosen for further analysis. Generally, the model manages to generate answers with correct language with the input question.
\subsubsection{Quantitative analysis}
We randomly choose 100 samples from the generated result to perform quantitative analysis. The average length, vocabulary size, and the number of POS tags in the ground truth and generated answers are calculated for each language. Table \ref{quanti} shows the statistics of the ground truth answer compared with the predicted answer by the model.

% \begin{table}[ht]
% \centering
% %\resizebox{\columnwidth}{!}{%
% \begin{tabular}{llrr}
% \toprule
% &Language&Ground Truth&Predicted\\\midrule

% \multirow{ 4}{*}{Avg. length} & English & 3.74 & 6.18 \\
% & Vietnamese & 4.42 & 5.97\\
% & Japanese & 4.67 & 8.43\\
% & All &4.26&6.78\\\midrule

% \multirow{ 4}{*}{Vocab. size} & English & 78 & 72 \\
% & Vietnamese & 97 & 101\\
% & Japanese & 77 & 83\\
% & All &252&256\\\midrule

% \multirow{ 4}{*}{\# POS tag} & English & 12 & 9 \\
% & Vietnamese &10  &9 \\
% & Japanese & 10 & 11\\
% & All &14 &14\\
% \bottomrule
% \end{tabular}
% \caption{The quantitative statistic of 100 generated samples compared with the ground truth}
% \label{quanti}
% \end{table}

\begin{table}[ht]
\centering
%\resizebox{\columnwidth}{!}{%
\begin{tabular}{llrr}
\toprule
Language&Stats.&Ground Truth&Predicted\\\midrule
\multirow{ 3}{*}{English} & Avg.length  & 3.74 & 6.18 \\
& Vocab. size & 78 & 72 \\
& \# POS tag  & 12 & 9 \\\midrule

\multirow{ 3}{*}{Vietnamese} & Avg.length  & 4.42 & 5.97 \\
& Vocab. size  & 97 & 101 \\
& \# POS tag &10  &9 \\\midrule

\multirow{ 3}{*}{Japanese} & Avg.length   & 4.67 & 8.43 \\
& Vocab. size & 77 & 83 \\
& \# POS tag  & 10 & 11 \\\midrule\midrule

\multirow{ 3}{*}{All} & Avg.length  &4.26 &6.78 \\
& Vocab. size &252 &256 \\
& \# POS tag  &14 &14 \\

\bottomrule
\end{tabular}
\caption{The quantitative statistic of 100 generated samples compared with the ground truth}
\label{quanti}
\end{table}

From Table \ref{quanti}, it can be seen that although the model gave the answers longer than the ground truth answers, the semantics is not as much as the ground truth. It can be seen from Table \ref{quanti} that the predicted answers in English have an average length higher than the ground truth answers. Also, the vocabulary in the generated answers is more than the original. In contrast, the number of POS tag components in the predicted answers is lower than the ground truth. This is similar to the answers in Vietnamese. For the Japanese, the characteristics of the predicted answers in average length and vocabulary size are the same as the two remaining languages. However, the number of POS tags in the predicted answers is more than in the ground truth answers. To make it clear, we propose three types of error on our model in Section \ref{quali_analysis}.

In addition, Figure \ref{100score} illustrates the distributions of F1 and BLEU scores for each language. Generally, the histograms skewed to the right and the model  performs inconsistently across languages. The proportion of samples with F1 and BLEU scores less than 0.2 dominates the overall result across all three languages. In Vietnamese, the number of generated samples with F1 and BLEU scores greater than 0.4 is significantly higher than in other languages. Meanwhile, English and Japanese responses rarely score greater than 0.6 on both metrics, furthermore, no Japanese samples scoring greater than 0.8 in BLEU. This illustrates that our model faces numerous challenges in producing the desired responses, with specific limitations on each language.

\begin{figure}[!ht]
    \centering
    \includegraphics[width=\textwidth]{figure/hist.pdf}
    \caption{Distributions of F1 and BLEU scores for each language from 100 generated samples}
    \label{100score}
\end{figure}

\begin{figure}[!htbp]
\centering
\subfloat[]{%
  \includegraphics[width=0.8\textwidth]{figure/attns1.pdf}%
  \label{attn1}
}

\subfloat[]{%
  \includegraphics[width=0.8\textwidth]{figure/attns2.pdf}%
  \label{attn2}
}

\subfloat[]{%
  \includegraphics[width=0.8\textwidth]{figure/attns3.pdf}%
  \label{attn3}
}

\subfloat[]{%
  \includegraphics[width=0.8\textwidth]{figure/attns4.pdf}%
  \label{attn4}
}

\subfloat[]{%
  \includegraphics[width=0.8\textwidth]{figure/attns5.pdf}%
  \label{attn5}
}
\caption{Numerous samples of attention alignment from ConvS2S and the changes in attention when adding features from ViT-B/16 and OFA$_{\mathrm{large}}$. The x-axis and y-axis of each plot correspond to the words in the question and the generated answer, respectively, while each pixel illustrates the weight $w_{ij}$ of the assignment of the j-th question word for the i-th
answer word.}
\label{attn}
\end{figure}

\subsubsection{Qualitative analysis}
\label{quali_analysis}
\paragraph{Attention visualization}

Figure \ref{attn} shows several samples of attention weights between each element from the generated answer with those in the input sequence that contains no image features, OFA hints, and OFA+ViT features, respectively. The visualization provided an intuitive way to discover which positions in the input sequence were considered more important when generating the target answer word. The brighter a pixel's color, the more important the word in the input sequence is in producing the respect answer word. Through this, we study that OFA hint is importance feature to model's attention as it provide the near-correct insight for the question and reduce the reliance on question words when generating the answer. However, in some cases, the model focuses too much on a specific element of the hint, which may lead to bias. ViT features has shown to control the affection of OFA hint, neutralizing it with other elements from question if hint appears to be off-topic. It may enhance the attention, making the model focus stronger on specific parts of the provided hint, for instance, the hint token ``nhà hàng'' (\textit{restaurant}) in Figure \ref{attn3} is given more attention when adding ViT image features. These features can also reduce the attention in one element and distributes concentration on other parts of the sequence. Figures \ref{attn1} and \ref{attn2} depict the reduction in hint attention into question elements, while Figures \ref{attn4} and \ref{attn5} show the changes in attention weight distribution among hint tokens.

\paragraph{Error analysis}
\begin{figure}[!ht]
\centering
\subfloat[Error Case I]{%
  \includegraphics[width=\textwidth]{figure/err1.pdf}%
\label{fig:1a}}
\vspace{1em}
\subfloat[Error Case II]{%
  \includegraphics[width=\textwidth]{figure/err2.pdf}%
    \label{fig:1b}
}
\vspace{1em}
\subfloat[Error Case III]{%
  \includegraphics[width=\textwidth]{figure/err3.pdf}%
\label{fig:1c}}
\caption{Three typical error cases from generated results.}
\label{fig:1}
\end{figure}

% \begin{figure}[H]
% \centering
% \resizebox{\textwidth}{!}{
%     \begin{subfigure}[b]{.3\linewidth}
%     \centering
%     \includegraphics[width=0.99\textwidth]{figure/00000001682.jpg}
%     \raggedright
%     { \scriptsize \textbf{Question}: what hat does the narrator of the 
%     historical site wear?}\\
%     {\scriptsize \textbf{Groundtruth}: non la}\\
%     {\scriptsize \textbf{Predicted}: the boy wears a white shirt and white and white}\\
%     {\scriptsize \textbf{F1:}  0.0000}\\
%     {\scriptsize \textbf{BLEU:} 0.0000
%     ~~~~~~~~~~~~~~~~~~~~~~~~~~~~~~~~~~~~~~~~~~~~~~~~~~~~~~~~~~~~~~~~~~~~~~~~~~~~~~~~~~~~~~~~~~~~~~~~~~ }
%     \caption{Error Type I}
%     \label{fig:1a}
%   \end{subfigure}%
%   \hspace{0.5em}
  
%  %\hspace*{\fill}
%   \begin{subfigure}[b]{.35\linewidth}
%     \centering
%     \includegraphics[width=0.99\textwidth]{figure/00000004737.jpg}
%     \raggedright
    
%     {\scriptsize \textbf{Question}: có bao nhiêu người đứng bên phải chàng trai? (\textit{English: How many people on the right of the man?})}\\
    
%     {\scriptsize \textbf{Groundtruth}: có ba người đứng bên phải chàng trai (\textit{English: There are three people on the right of the man})}\\
    
%     {\scriptsize \textbf{Predicted}: có hai người đứng bên phải chàng trai (\textit{English: There are two people on the right of the man})}\\
    
%     {\scriptsize F1:  0.8750}\\
    
%     {\scriptsize BLEU: 0.7799}
%     \caption{Error Type II}
%     \label{fig:1b}
%   \end{subfigure}%
%   \hspace{0.5em}
%   %\hspace*{\fill}
%   \begin{subfigure}[b]{0.35\linewidth}
%      \centering
%     \includegraphics[width=0.99\textwidth]{figure/00000000111.jpg}
    
%     \raggedright {\scriptsize \textbf{Question}:}
%     {\tiny
%     \begin{CJK*}{UTF8}{min}
%     {\CJKfamily{goth}小船手は何本のオールを使っていますか? (\scriptsize \textit{English: How many paddles does the boatman use?})}
%     \end{CJK*}}\\
%     {\scriptsize \textbf{Groundtruth}: 2}\\
%     {\scriptsize \textbf{Predicted}:}
%     {\tiny
%     \begin{CJK*}{UTF8}{min}
%     {\CJKfamily{goth}2本の船を使っています (\scriptsize \textit{ English: using two boats})}
%     \end{CJK*}}\\
%     {\scriptsize \textbf{F1:} 0.0000}\\
%     {\scriptsize \textbf{BLEU:} 0.0000 ~~~~~~~~~~~~~~~~~~~~~~~~~~~~~~~~~~~~~~~~~~~~~~~~~~~~~~~~~~~~~~~~~~~~~~~~~~~~~~~~~~~~~~~~~~~~~~~~~~ }\\
%     \caption{Error Type III}
%     \label{fig:1c}
%   \end{subfigure}%  
% }
%   \caption{Example of generated answers that contains errors.(b) the keyword 'hai người' (two people) is given  instead of 'ba người' (three people). Coincidentally, the question and groundtruth in this case both share the same phrase "đứng bên phải chàng trai" ("on the right of the man"), }\label{fig:1}
% \end{figure}


For better understand the generation performance on the VQA task, we examine the generated answers of our best ensemble, ConvS2S
+ViT-B/16+OFA$_{\mathrm{large}}$, to identify the limitations and analyze factors that may cause the model to perform poorly.
Through the error analysis process, various errors and mistakes have been pointed out in the outputs of the model. The typical examples of various types of errors are illustrated in Figure \ref{fig:1}. In summary, we divide these errors into three groups:

\begin{itemize}
    \item The generated answer does not match the question and has no correct tokens compared with the ground truth answer, as shown in Figure \ref{fig:1a}. This error case sometimes accompanied by text degeneration.
    \item The generated response gives the wrong answer to the question but share some insignificant tokens with the ground truth answer, as shown in Figure \ref{fig:1b}, which significantly improves the evaluation score. This incorrect scenario exemplifies the limitation of the evaluation measures.
    \item The model managed to generate the correct key answer while also adding unnecessary information compared to the ground truth, which may lead to the response's meaning being distorted. 
    As shown in Figure \ref{fig:1c}, the model correctly predicted quantity but then added unnecessary tokens afterward, resulting in a low score on both evaluation metrics.
\end{itemize}


% \begin{figure*}[h]
% \centering
%   \begin{tabular}{@{}ccc@{}}
%     \includegraphics[width=0.3\textwidth]{figure/5.3_ex/00000000111.jpg}
%     \includegraphics[width=0.3\textwidth]{example-image-b} &
%     \includegraphics[width=0.3\textwidth]{example-image-b} \\
%   \end{tabular}
%   \caption{This is some figure side by side}
% \end{figure*}


\section{Conclusion and Future Work}
\label{sec:conclusion}

We have presented a novel neural network that successively learns shape sketch and extrusion without any expensive annotations of shape segmentation and labels as the supervision.
%Without the guidance of sketch labels, 
Our approach is able to learn smooth sketches, followed by the differentiable extrusion to reconstruct CAD models that are close to the ground truth. 
We evaluate SECAD-Net using diverse CAD datasets and demonstrate the advantages of our approach by ablation studies and comparing it to the state-of-the-art methods. 
We further demonstrate our method’s applicability in single-image CAD reconstruction. 
Additionally, the CAD shapes generated by our approach can be directly fed into off-the-shelf CAD software for sketch-level or cylinder primitive-level editing. 

% We tested SE-Net on ABC dataset and Fusion 360 dataset. Quantitative results demonstrate that SE-Net can efficiently reconstruct 3D CAD shapes. Qualitative results show that our model can learn fine 2d sketches without any associated ground-truth.


% We propose SE-Net, a network that successively learns shape sketch and extrusion in an unsupervised manner. The CAD shapes generated by the network can be directly sent to off-the-shelf CAD software for sketch-level or cylinder primitive-level editing. SE-Net can be reconstructed to generate smooth sketches and the reconstruction effect is due to the current state-of-the-art, including supervised methods. Additionally, our method is the first to learn sketches from raw shapes without the guidance of sketch labels.

In future work, we plan to extend our approach to learn more CAD-related operations such as \emph{revolve, bevel, and sweep}. %using neural methods. 
Besides, we find that current deep learning models perform poorly on datasets with large differences in shape geometry and structure. %structural and topological variations
Therefore, another promising direction is to explore how to improve the generalization of neural networks and enhance the realism of the generated shapes by learning structural and topological information.
% \section*{Acknowledgement}
% \footnotesize
% This research was performed at the AUTOLAB at UC Berkeley in affiliation with the Berkeley AI Research (BAIR) Lab, the CITRIS “People and Robots” (CPAR) Initiative, and the RealTime Intelligent Secure Execution (RISE) Lab. The authors were supported in part by donations from Toyota Research Institute and by equipment grants from PhotoNeo, and Nvidia.
%===============================================================================

\clearpage
% The acknowledgments are automatically included only in the final and preprint versions of the paper.
\acknowledgments{This research was performed at the AUTOLAB at UC Berkeley in
affiliation with the Berkeley AI Research (BAIR) Lab. The authors were supported in part by donations from Toyota Research Institute.}

%===============================================================================

% no \bibliographystyle is required, since the corl style is automatically used.
\bibliography{references}  % .bib
% \newpage
% \newpage
\section{Appendix}\label{sec:append}
\subsection{Glossary of notations}
\label{appendix:notations}
In Table ~\ref{tab:notations}, we summarize the notations used in our work. 

\begin{table}[ht]
    \setlength{\abovecaptionskip}{0.25cm}
    \setlength{\belowcaptionskip}{-0.25cm}
    \caption{Glossary of Notations.}\label{tab:notations}
    % \vspace{-2.5mm}
    \centering
    \scalebox{1.0}{
        \begin{tabular}{l|l}
        \hline
        Notation & Description \\
        \hline
        $G;A;S$ & Graph; Adjacency matrix; Similarity matrix. \\
        % $A^{G}_{ij}$ & The weight of the edge between $v_i$ and $v_j$ in $G$.\\
        $v;e;x$ & Vertex; Edge; Vertex attribute. \\
        $V;E;X$ & Vertex set; Edge set; Attribute set. \\
        $|V|;|E|$ & The number of vertices and edges. \\
        $\mathcal{P};P_i$ & The partition of $V$; A community.\\
        $D;d(v_i)$ & The degree matrix; The degree of vertex $v_i$. \\
        $e_{ij}$ & The edge between $v_i$ and $v_j$. \\
        $w_{ij}$ & The weight of edge $e_{ij}$. \\
        $vol(G)$ & The volume of graph $G$, i.e., degree sum in $G$. \\
        $G^{(k)}_{knn}$ & The $k$-NN graph with parameter $k$.\\
        $G_{f}$  & Fusion graph.\\
        $G^{(k)}_{f}$ & The fusion graph with parameter $k$.\\
        \hline
        $\mathcal{T}$ & Encoding tree. \\
        $\mathcal{T}^*$ & The optimal encoding tree. \\
        % $\mathcal{T}^{(K)}$ & The encoding tree with height $K$. \\
        $\lambda$ & The root node of an encoding tree. \\
        $\alpha$ & A non-root node of an encoding tree. \\
        $\alpha^-$ & The parent node of $\alpha$. \\
        $\alpha^{\left \langle i \right \rangle}$ & the $i$-th child of $\alpha$.\\
        $T_\lambda$ & The label of $\lambda$, i.e., node set $V$. \\
        $T_\alpha$ & The label of $\alpha$, i.e., a subset of $V$.\\
        $\mathcal{V}_\alpha$ & Volume of graph $G$. \\
        $g_a$ & the sum weights of cut edge set $[T_\alpha,T_\alpha/T_\lambda]$. \\
        $N(\mathcal{T})$ & The number of non-root node in $\mathcal{T}$.\\
        \hline
        $H^\mathcal{T}(G)$ & Structural entropy of $G$ under $\mathcal{T}$.\\
        $H^K(G)$ & $K$-dimensional structural entropy.\\
        $H^1(G)$ & One-dimensional structural entropy.\\
        $H^\mathcal{T}(G;\alpha)$ & Structural entropy of node $\alpha$ in $\mathcal{T}$.\\
        $H^{\mathcal{T}}(G;(\lambda,\alpha])$ & Structural entropy of a deduction from $\lambda$ to $\alpha$.\\
        \hline
        \end{tabular}
    }
% \vspace{-6.5mm}
\end{table}

% \vspace{-0.8em}
\subsection{Dataset and Time Costs of \ \framework{}}
\label{appendix:dataset}
Our framework \framework~ is evaluated on nine graph datasets. the statistics of these datasets are shown in Table~\ref{tab:statistics}. The time costs of \ \framework{} on all datasets are shown in Table~\ref{tab:time comparasion}.
% \vspace{-0.2em}
\begin{table}[ht]
    \setlength{\abovecaptionskip}{0.25cm}
    \setlength{\belowcaptionskip}{-0.25cm}
    \caption{Statistics of benchmark datasets.}\label{tab:statistics}
    \centering
    % \vspace{-0.2em}
    \scalebox{1.0}{
        \begin{tabular}{c|ccccc}
        \hline
        Dataset & Nodes & Edges & Classes & Features & homophily\\
        \hline
        Cora & 2708 & 5429 & 7 & 1433 & 0.83\\
        Citeseer & 3327 & 4732 & 6 & 3703 & 0.71\\
        Pubmed & 19717 & 44338 & 3 & 500 & 0.79\\
        \hline
        % Chameleon & 2277 & 36101 & 5 & 2325 & 0.25\\
        % Squirrel & 5201 & 217073 & 5 & 2089 & 0.22\\
        PT & 1912 & 31299 & 2 & 3169 & 0.59\\
        TW & 2772 & 63462 & 2 & 3169 & 0.55\\
        \hline
        Actor & 7600 & 33544 & 5 & 931 & 0.24\\
        \hline
        Cornell & 183 & 295 & 5 & 1703 & 0.30\\
        Texas & 183 & 309 & 5 & 1703 & 0.11\\
        Wisconsin & 251 & 499 & 5 & 1703 & 0.21\\
        \hline
        \end{tabular}
    }
% \vspace{-4.5mm}
\end{table}

\begin{itemize}[leftmargin=*]
    \item \textbf{Citation networks}~\cite{yang2016revisiting,welling2016semi}. Cora, Citeseer, and Pubmed are benchmark datasets of citation networks. Nodes represent paper, and edges represent citation relationships in these networks. The features are bag-of-words representations of papers, and labels denote their academic fields.
    % \item \textbf{Wikipedia networks}~\cite{rozemberczki2021multi,pei2019geom}. Wikipedia dataset contain three page to page networks, Chameleon, Squirrel and Crocodile, which are originally designed for website traffic regression. In ~\cite{pei2019geom}, the author package monthly traffic into 5 categories, providing Chameleon and Squirrel for node classification task. In both networks, vertices are web pages, edges are hyper-links and features are nouns that characterize the page.
    \item \textbf{Social networks}~\cite{rozemberczki2021multi}.
    TW and PT are two subsets of Twitch Gamers dataset~\cite{rozemberczki2021twitch}, designed for binary node classification tasks, where nodes correspond to users and links to mutual friendships. 
    The features are liked games, location, and streaming habits of the users. 
    The labels denote whether a streamer uses explicit language (Taiwanese and Portuguese).
    \item \textbf{WebKB networks}~\cite{getoor2005link}. 
    Cornell, Texas, and Wisconsin are three subsets of WebKB, where nodes are web pages, and edges are hyperlinks. The features are the bag-of-words representation of pages. The labels denote categories of pages, including student, project, course, staff, and faculty.
    % The WebKB dataset consists of 877 scientific publications classified into one of five classes. The citation network consists of 1608 links. Each publication in the dataset is described by a 0/1-valued word vector indicating the absence/presence of the corresponding word from the dictionary. The dictionary consists of 1703 unique words. 
    \item \textbf{Actor co-occurrence network}~\cite{tang2009social}. This dataset is a subgraph of the film-director-actor-writer network, in which nodes represent actors, edges represent co-occurrence relation, node features are keywords of the actor, and labels are the types of actors.
    % \item \textbf{Karateclub dataset}
\end{itemize}
% \vspace{-0.5em}

% \vspace{-1.0em}
\subsection{Baselines}
\label{appendix:baseline}
% Extensive baselines are used for comparison, which is briefly described as follows\footnote{For GCN, GAT, GraphSAGE, and APPNP layers, we adopt implementation from DGL library~\cite{wang2019deep}:https://github.com/dmlc/dgl }:
Baselines are briefly described as follows\footnote{For GCN, GAT, GraphSAGE, and APPNP layers, we adopt implementation from DGL library~\cite{wang2019deep}:https://github.com/dmlc/dgl }:
\begin{itemize}[leftmargin=*]
% \vspace{-1.5mm}
    \item \textbf{GCN}~\cite{welling2016semi} 
    is the most popular GNN, which defines the first-order approximation of a localized spectral filter on graphs.
    \item \textbf{GAT}~\cite{velivckovic2017graph}  
    introduces a self-attention mechanism to important scores for different neighbor nodes.
    \item \textbf{GraphSAGE}~\cite{hamilton2017inductive} 
    is an inductive framework that leverages node features to generate embeddings by sampling and aggregating features from the local neighborhood. 
    \item \textbf{APPNP}~\cite{gasteiger2019predict} 
    combines GCN with personalized PageRank. 
    \item \textbf{GCNII}\footnote{https://github.com/chennnM/GCNII}~\cite{chen2020simple}
    employs residual connection and identity mapping.
    \item \textbf{Grand}\footnote{https://github.com/THUDM/GRAND}~\cite{feng2020graph}
    purposes a random propagation strategy for data augmentation, and uses consistency regularization to optimize.
    \item \textbf{Mixhop}\footnote{https://github.com/samihaija/mixhop}~\cite{abu2019mixhop} aggregates mixing neighborhood information.
    \item \textbf{Geom-GCN}\footnote{https://github.com/graphdml-uiuc-jlu/geom-gcn}~\cite{pei2019geom}
    exploits geometric relationships to capture long-range dependencies within structural neighborhoods. Three variant of Geom-GCN is used for comparison.
    \item \textbf{GDC}\footnote{https://github.com/gasteigerjo/gdc}~\cite{gasteiger_diffusion_2019}
    refines graph structure based on diffusion kernels.
    % \item \textbf{Pro-GNN}\footnote{https://github.com/DSE-MSU/DeepRobust}~\cite{jin2020graph}
    \item \textbf{GEN}\footnote{https://github.com/BUPT-GAMMA/Graph-Structure-Estimation-Neural-Networks}~\cite{wang2021graph} 
    estimates underlying meaningful graph structures.
    \item \textbf{H$_2$GCN}\footnote{https://github.com/GemsLab/H2GCN}~\cite{zhu2020beyond} combine multi-hop neighbor-embeddings for adapting to both heterophily and homophily graph settings.
    \item \textbf{DropEdge}\footnote{https://github.com/DropEdge/DropEdge}~\cite{rong2019dropedge}
    randomly removes edges from the input graph for over-fitting prevention. 
    \item \textbf{Jaccard}\footnote{https://github.com/DSE-MSU/DeepRobust}~\cite{wu2019adversarial}
    prunes the edges connecting nodes with small Jaccard similarity.
\end{itemize} 

% \footnote{The implementation provided by CogDL~\cite{cen2021cogdl} is adopted for GCNII, GDC and Grand: https://github.com/thudm/cogdl}
% For SGC, GCN~\cite{welling2016semi}, Chebnet, GAT~\cite{velivckovic2017graph},GraphSAGE and APPNP, we adopt the implementations from the Deep Graph Learning library ~\cite{wang2019deep}.
% For the remaining baselines

% we utilize two-layer GNN encoders, and comparing with our \framework with them as the backbones. 

\begin{table*}[htb!]
    \renewcommand{\arraystretch}{0.95}
    \setlength{\abovecaptionskip}{0.25cm}
    \setlength{\belowcaptionskip}{-0.25cm}
    \caption{Comparison of training time(hr.) of achieving the best performance based on GPU.}
    % \vspace{-2.5mm}
    \label{tab:time comparasion}
    \centering
    % \scalebox{1.0}{
    \setlength{\tabcolsep}{3.6mm}{
        \begin{tabular}{l|ccccccccc}
        \hline
      Method & {Cora} & {Citeseer} & {Pubmed} & {PT} & {TW} & {Actor} & {Cornell} & {Texas} & {Wisconsin} \\
        \cline{1-1}
        \hline     % cora    cite    pub     pt    tw    act      cor      tex     wis
        \framework$_{GCN}$  &  0.071 & 0.213 & 4.574 & 0.178 &  0.374
                     & 1.482 & 0.006 & 0.008 & 0.009 \\
        \framework$_{SAGE}$   & 0.074 & 0.076 & 4.852 & 0.169 &  0.214
                     & 0.817 &  0.006 & 0.007 & 0.009 \\
        \framework$_{GAT}$   & 0.071 & 0.180 & 4.602 & 0.172 &  0.329
                     & 1.273 & 0.006 & 0.008 & 0.009 \\
        \framework$_{APPNP}$   & 0.073 & 0.215 & 4.854 & 0.138 &  0.379
                     & 1.367 & 0.010 & 0.011 & 0.013 \\
        \hline
        \end{tabular}
    }
    % \vspace{-2.5mm}
\end{table*}

% \vspace{-1.0em}
\subsection{Overall algorithm of \framework}
The overall algorithm of \framework~ is shown in Algorithm ~\ref{algorithm:training}. Note that, if choose to retain the connection from the previous iteration, to ensure that the number of edges remains stable during the training, a percentage of edges in the reconstructed graph with low similarity will be dropped in each iteration.

\label{appendix:overall algorithm}
\begin{algorithm}[htb!]
\SetAlgoRefName{1}
\SetAlgoVlined
\KwIn{a graph $G=(V,E)$, features $X$, labels $Y_L$, mode $\in {True,False}$\\
iterations $\eta$, encoding tree height $K$, hyperparameter $\theta$}
\KwOut{optimized graph $G'=(V,E')$, prediction $Y_P$, GNN parameters $\Theta$ }
Initialize $\Theta$;\\
\For{$i=1$ to $\eta$}{
    Update $\Theta$ by classification loss $\mathcal{L}_{cls}(Y_L,Y_P)$;\\
    Getting node representation $X'=\mathrm{GNN}(X)$;\\
    Initialize $k=1$ for $k$-NN structuralization;\\
    Create fusion map $G_{f}$ according to Algorithm~\ref{algorithm:kselector};\\
    Create $K$-dimensional encoding tree $\mathcal{T}^*$ according to Algorithm~\ref{algorithm:KDimSEMinimize};\\
    \For{each non-root node $\alpha$ in $\mathcal{T}^*$}{
        Calculate $H^{\mathcal{T}^*}(G_{f};(\lambda,\alpha])$ through Eq.~\ref{eq:deduct-se};\\
        Assign probability $P(\alpha)$ to $\alpha$ through Eq.~\ref{eq:prob-softmax};\\
    }
    \For{each subtree rooted at $\alpha$ in $\mathcal{T}^*$}{
        Assuming $\alpha$ has $n$ children, set $t = \theta \times n$;\\ 
        \For{$j=1$ to $t$}{
            Sample a node pair $(v_m,v_n)$ according to \S~\ref{step3};\\
            Adding edge $e_{mn}$ to $G'$;\\
        }
    }
    \If{mode}{
        Let $E' = E \cup E'$, where $E'$ and $E$ are the edge set of $G'$ and $G$, respectively;\\
        Drop a percentage of edges in $G'$;
    }
    Update graph structure $G \gets G'$;
    Update node representation: $X \gets X'$;\\
}
Get prediction $Y_P$;\\
Return $G'$, $Y_P$ and $\Theta$;\\
\caption{Model training for~\framework}
\label{algorithm:training}
\end{algorithm}

% \vspace{-1.0em}
\subsection{Algorithm of one-dimensional structural entropy guided graph enhancement}
\label{appendix:1dse algorithm}
The $k$-selector is designed for choosing an optimal $k$ for $k$-NN structuralization under the guidance of one-dimensional structural entropy maximization. The algorithm of $k$-selector and fusion graph construction is shown in Algorithm~\ref{algorithm:kselector}.

\begin{algorithm}[htb!]
\SetAlgoRefName{2}
\SetAlgoVlined
\KwIn{a graph $G=(V,E)$, node representation $X$}
\KwOut{fusion graph $G_{f}$ }
Calculate $S \in \mathbb{R}^{|V|\times |V|}$ via Eq.~\ref{eq:pcc};\\
\For{$k=2$ to $|V|-1$}{
    Generate $G_{knn}$ by $S$;\\
    Generate $G^{(k)}_{f} = \{V,E_{f} = E \cup E_{knn} \}$;\\
    Reweight $G^{(k)}_{f}$ via Eq.~\ref{eq:reweighted};\\
    Calculate $H^1(G^{(k)}_{f})$ via Eq.~\ref{eq:H1};\\
    \If{$H^1(G^{(k)}_{f})$ reaches the maximal optima}{
        $G_{f} \gets G^{(k)}_{f}$;\\
        Return $G_{f}$;
    }
}
\caption{$k$-selector and fusion graph construction}
\label{algorithm:kselector}
\end{algorithm}

% \vspace{-1.0em}
\subsection{Algorithm of high-dimensional structural entropy minimization}
\label{appendix:kdse algorithm}
% \vspace{-0.5em}
% The high-dimensional structural entropy minimization is a heuristic algorithm that adjusts encoding tree structure by greedily executing specific operators. The pseudo-code is shown in Algorithm~\ref{algorithm:KDimSEMinimize}.
The pseudo-code of the high-dimensional structural entropy minimization algorithm is shown in Algorithm~\ref{algorithm:KDimSEMinimize}.
%%伪代码
\begin{algorithm}[htb!]
\SetAlgoRefName{3}
\SetAlgoVlined
\KwIn{a graph $G=(V,E)$, the height of encoding tree $k>1$}
\KwOut{Optimal high-dimensional encoding tree $\mathcal{T}^*$}
//Initialize an encoding tree $\mathcal{T}$ with height $1$ and root $\lambda$\\
Create root node $\lambda$;\\
\For{$v_i \in V$}{
    Create node $\alpha_i$. Let $T_{\alpha_i} = v_i$;\\
    $v_i^- = \lambda$;\\
}
//Generation of binary encoding tree\\
%% problem
\While{$\lambda$ has more than $2$ children}{
    Select $\alpha_i$ and $\alpha_j$ in $\mathcal{T}$, 
    condition on $\alpha_i^-=\alpha_j^-=\lambda$ and $\mathop{\arg\max}\limits_{\alpha_i,\alpha_j}(H^\mathcal{T}(G)-H^\mathcal{T}_{\mathrm{CB}(\alpha_{i},\alpha_{j})}(G))$;\\
    $ \mathrm{CB}(\alpha_{i},\alpha_{j})$ according to Definition ~\ref{def:CBop};\\
}
//Squeezing of encoding tree\\
\While{$\mathrm{height}(\mathcal{T}) > K$}{
    Select non-root node $\alpha$ and $\beta$ in $\mathcal{T}$, 
    condition on $\alpha^-=\beta$ and $\mathop{\arg\max}\limits_{\alpha,\beta}(H^\mathcal{T}(G)-H^\mathcal{T}_{\mathrm{LF}(\alpha,\beta)}(G))$;\\
    $ \mathrm{LF}(\alpha,\beta)$ according to Definition ~\ref{def:LFop};\\
}
Return $\mathcal{T}^* \gets \mathcal{T}$;
\caption{K-dimensional structural entropy minimization}
\label{algorithm:KDimSEMinimize}
\end{algorithm}




\end{document}