\section{Related Work}

\subsection{Deformable Object Manipulation}

Recent advancements in deformable manipulation include cable untangling algorithms ~\cite{viswanath2021disentangling, grannen2020untangling, sundaresan2021untangling, lui2013tangled}, fabric smoothing and folding techniques ~\cite{seita2019deep, weng2022fabricflownet,ganapathi2020learning,hoque2020visuospatial,kollar2022simnet,luv2022,hoque2022reach}, and object placement in bags ~\cite{seita2020learning,lawrence2023bag}. Methods for autonomously manipulating deformable objects range from model-free to model-based, the latter of which estimates the state of the object for subsequent planning. 

\textbf{Model-free approaches} include reinforcement or self-supervised learning for fabric smoothing and folding~\cite{matas2018sim, wu2019learning, lee2020learning,speedfolding} and straightening curved ropes~\cite{wu2019learning}, or directly imitating human actions~\cite{seita2019deep, Schulman2013LearningFD}. Research by \citet{grannen2020untangling} and \citet{sundaresan2021untangling} employs learning-based keypoint detection to untangle isolated knots without state estimation. \citet{viswanath2022autonomously} and \citet{shivakumar2022sgtm} extend this approach to long cables (3m) with a learned knot detection pipeline. However, scaling to arbitrary knot types requires impractical amounts of human labels across many conceivable knot types. Our study focuses on state estimation to address this limitation.

\textbf{Model-based methods} for deformable objects employ methods to estimate the state as well as dynamics. Work on cable manipulation includes that by \citet{sundaresan2020learning}, who use dense descriptors for goal-conditioned manipulation. Descriptors have also been applied to fabric smoothing ~\cite{ganapathi2020learning}, as well as visual dynamics models for non-knotted cables ~\cite{yan2020learning, wang2019learning2} and fabric ~\cite{hoque2020visuospatial, yan2020learning, lin2022learning}. Other approaches include learning visual models for manipulation~\cite{nair2017vismodel}, iterative refinement of dynamic actions~\cite{chi2022irp}, and using approximate state dynamics with a learned error function~\cite{dmitry2020trust}. Fusing point clouds across time has shown success in tracking segments of cable provided they are not tangled on themselves~\cite{abbeeltrackingcable,tracking2}. Works from \citet{lui2013tangled} and \citet{dlotase} estimate splines of cables before untangling them, and we discuss these methods among others below.

%%%%%%%%%%%%%%%%%%%%%%%%%%%%%%%%%%%%%%%%%%%%%% 
\vspace{-0.1in}
\subsection{Tracing Deformable Linear Objects}
\vspace{-0.1in}
%\todo{break this section down by learning and non learning based? I think they are all analytic?}

% Prior tracing methods for 1D objects are primarily analytic approaches, such as ~\citet{dlotase}, \citet{lui2013tangled}, and ~\citet{schaal2022}, which estimate the state of one or multiple ropes against varying backgrounds, given heuristics about the expected bending properties of cables. The works of \citet{nurbtracer} and \citet{britishcvsurgery} trace surgical strings in stereo or mono images by optimizing a continuous spline representation. In contrast, this work primarily focuses on longer cables with a greater variety of configurations, for which analytical methods struggle to differentiate nearby, twisted cables. Some prior work approaches this problem~\cite{Parmar_2013} but do not fully estimate cable state, only identifying crossings with analytic methods.

Prior tracing methods for LDOs, including in our prior work \cite{shivakumar2022sgtm}, primarily employ analytic approaches \cite{kicki2023dloftbs, lui2013tangled, dlotase, schaal2022}. Other works, \citet{nurbtracer} and \citet{britishcvsurgery}, optimize splines to trace surgical threads. Very recently, \citet{kicki2023dloftbs} use a primarily analytic method for fast state estimation and tracking of short cable segments with 0-1 crossings. In contrast to these works, this work focuses on longer cables with a greater variety of configurations including those with over 25 crossings and twisted cable segments.
% , where analytical methods struggle to differentiate nearby, twisted cables.
Other prior work attempts to identify the locations of crossings \cite{Parmar_2013} in cluttered scenes of cables but does not fully estimate cable state.

Several prior works approach sub-parts of the problem using learning. \citet{lui2013tangled} use learning to identify weights on different criteria to score traces. \citet{dlotase} focus on classifying crossings for thick cables. \citet{song2019untangling} predict entire traces for short cables as gradient maps but lack quantitative results for cables and evaluation of long cables with tight knots. \citet{yan2020selfsuperviseddlo} use self-supervised learning to iteratively estimate splines in a coarse-to-fine manner. In very recent work, \citet{caporaliariadne2022,caporalifastdlo2022} use learned embeddings to match cable segments on opposite sides of analytically identified crossings, in scenes with 3 or fewer total crossings, while we test \peralgabbr{} on complex configurations containing over 25 crossings.

\vspace{-0.1in}