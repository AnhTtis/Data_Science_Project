\section{Problem Statement}
\vspace{-0.1in}
\label{sec:ps}

\textbf{Workspace and Assumptions:} The workspace is defined by an $(x, y, z)$ coordinate system with a fixed overhead camera facing the surface that outputs grayscale images.
% Depth images are used only to enable robust grasping.
We assume that 1) the greyscale image includes only cables (no obstacles), 2) each cable is visually distinguishable from the background, 3) each cable has at least one endpoint visible, and 4) the configuration is semi-planar, meaning each crossing contains at most 2 cable segments. We define each cable state for $i=1,...,l$ cables to be $\theta_{i}(s) = \{(x(s), y(s), z(s))\}$ where $s$ is an arc-length parameter that ranges $[0, 1]$, representing the normalized length of the cable. Here, $(x(s), y(s), z(s))$ is the location of a cable point at a normalized arc length of $s$ from the cable's first endpoint. We also define the range of $\theta(s)$---that is, the set of all points on a cable at time $t$---to be $\mathcal{C}_t$. %\todo{rework to work for multiple cables}. 
We assume all cables are visually distinguishable from the background and that the background is monochrome. Additional assumptions we make for the manipulation tasks are stated in Section \ref{sec:untanging_setup}.

\textbf{Objective:} The objective of \peralgabbr{} is to estimate the cable state – a pixel-wise trace as a function of $s$ of one specified cable indicating all over- and under-crossings.
% given an overhead grayscale image and resolve whether each of its crossings is "over" or "under". We define a trace as the most probable sequence of spaced pixels that the cable traverses. This state estimate can facilitate manipulation tasks such as those defined in Sections \ref{sec:untanging_setup} and \ref{sec:demos_setup}.

