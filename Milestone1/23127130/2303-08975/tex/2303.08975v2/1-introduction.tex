\section{Introduction}
\label{sec:intro}

% \kaushik{\st{In industrial and household settings, disorganized deformable linear objects such as cables can impede task success and create tripping hazards. In automation tasks such as testing connector functionality, orderly cables are important to verify success. For humans, messy or faulty wires can pose a risk of entanglement, causing equipment malfunctions and electric shock.}}
% \st{\cite{sanchez2018robotic, mayer2008system, van2010superhuman, yamakawa2007one}}

Tracing long one-dimensional objects such as cables has many applications in robotics. However, this is a challenging task as depth images are prone to noise, and estimating cable state from greyscale images alone is difficult since cables often fall into complex configurations with many crossings. Long cables can also contain a significant amount of free cable (\emph{slack}), which can occlude and inhibit the perception of knots and crossings.

 % industrial and household settings often contain disorganized electrical cables. To ensure functionality and safety are maintained, it can be useful to automate diagnosis and in some cases, even remedial, via robot inspection and manipulation of these long cables}. This \kaushik{process} is \kaushik{often} constrained by challenges \kaushik{in perception}.  Successful manipulation of long cables requires the ability to trace a cable path in complex configurations and handle the wide distribution of states long cables can fall into.

\begin{figure}[!ht]
    \centering
    \includegraphics[width=1.0\linewidth]{figures/splash.pdf}
    \caption{\textbf{Example of \peralgabbr on a single knotted cable with 16 crossings:} \peralgabbr\ includes two networks. One network (pink at the top) is trained to predict the next point in the trace given a prior context window, and the other network (blue at the top) is trained to classify over- and under-crossings. During inference, \peralgabbr{} 1) uses the pink network to autoregressively find the most likely trace (illustrated with a rainbow gradient, from violet to purple, depicting the path of the cable) and 2) performs crossing recognition using the blue network to obtain the full state of the cable (3), where red circles indicate overcrossings and blue circles indicate undercrossings. 4) The state estimate from \peralgabbr{} can be used for inspection and manipulation.}
    \label{fig:splash}
    \vspace*{-0.2in}
\end{figure}

% \kaushik{Can we cut this paragraph down?}
% Much of prior cable manipulation work, such as untangling, on both long and short cables bypasses full state estimation by employing object detection networks and keypoint selection networks to identify knots and grasp points directly based on geometric patterns \cite{viswanath2022autonomously, shivakumar2022sgtm, viswanath2021disentangling, grannen2020untangling, sundaresan2021untangling, song2019untangling}. \citet{sundaresan2020learning} use dense descriptors for estimating cable state for the knot tying task. While successful, the work performs state estimation on a thick short cables with mostly straight to loose overhand knots. On the other hand, our work considers thin long cables containing crossings with near-parallel segments. \citet{dlotase} also performs full state estimation for the untangling task but uses analytic methods and doesn't demonstrate performance on the types of challenging cases we do.

Prior cable tracing research use a combination of learned and analytic methods to trace a cable, but are limited to at most 3 crossings \cite{caporaliariadne2022, kicki2023dloftbs}. Previous cable manipulation work bypasses state estimation with object detection and keypoint selection networks for task-specific points based on geometric patterns \cite{viswanath2022autonomously, shivakumar2022sgtm, viswanath2021disentangling, grannen2020untangling, sundaresan2021untangling, song2019untangling}. \citet{sundaresan2020learning} employ dense descriptors for cable state estimation in knot tying but focus on thick short cables with loose overhand knots. However, this work tackles the challenges of long cables in semi-planar configurations (at most 2 cable segments per crossing), which often contain dense arrangements with over 25 crossings and near-parallel cable segments. Although works like \citet{lui2013tangled} and \citet{dlotase} also perform cable state estimation, they use analytic methods and do not address the challenging cases mentioned above which we consider. While analytic methods require heuristics to score and select from traces, learning to predict cable traces directly from images avoids this problem.

This work considers long cables (up to 3 meters in length) in semi-planar configurations, i.e. where each crossing includes at most 2 cable segments. These cable configurations may include knots within a single cable (e.g. overhand, bowline, etc.) or between multiple cables (e.g. carrick bend, sheet bend, etc.).
% The single-cable semi-planar knots evaluated in this work are the overhand, figure-eight, overhand honda, bowline, linked overhand, and figure-eight honda knots. The double-cable semi-planar knots evaluated in this work are the carrick bend, sheet bend, and square knots.
% \peralgabbr{} uses learning to facilitate high-accuracy tracing, which previously has been done mostly analytically, and crossing identification, to generate a cable state estimate, which we show to be useful for inspection and manipulation tasks. 
This paper contributes:
\begin{enumerate}
    \item \peralgname{} (\peralgabbr{}), shown in Figure \ref{fig:splash}: A tracing algorithm for long deformable linear objects with up to 25 or more crossings.
    % \item An analytic knot detection algorithm and untangling point selection algorithm given the cable state estimates.
    \item Novel methods for cable inspection, state-based imitation, knot detection, and autonomous untangling for cables using \peralgabbr{}.
    %and one-shot learning from demonstrations pipeline that employs \peralgabbr{}. removed: single or multiple
    \item Data from physical robot experiments that suggest \peralgabbr{} can correctly trace long DLOs unseen during training with 85\% accuracy, trace and segment a single cable in multi-cable settings with 80\% accuracy, and detect knots with 77\% accuracy. Robot experiments suggest \peralgabbr{} in a physical system for untangling semi-planar knots achieves 64\% untangling success in under 8 minutes and in learning from demonstrations, achieves 80\% success.
    \item Publicly available code and data (and data generation code) for \peralgabbr{}: \href{https://github.com/vainaviv/handloom}{https://github.com/vainaviv/handloom}.
\end{enumerate}

