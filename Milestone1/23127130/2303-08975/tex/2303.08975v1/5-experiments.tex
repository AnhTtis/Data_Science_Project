% \section{Experiments}
% \label{sec: experiments}
\section{Experiments}
We test the performance of 1) \peralgabbr{}, 2) the learned cable tracer, and 3) \peralgabbr{} applied to autonomous robot untangling. 

\subsection{Workspace}
The workspace consists of a 1 m $\times$ 0.75 m surface with a bimanual ABB YuMi robot and an overhead Photoneo PhoXi camera with $773 \times 1032 \times 4$ RGB-D observations. Although there are 3 color channels, images outputted by the PhoXi are grayscale. Additionally, the workspace is padded with a 5 cm tall piece of foam and covered with a black cloth. 

\subsection{\peralgabbr{} Setup}
To test \peralgabbr{}, we use a single 3 m, white, braided USB-A to micro-USB cable to the workspace.

\begin{figure}[!ht]
    \vspace*{-0.15in}
    \centering
    \includegraphics[width=1.0\linewidth]{figures/start_states.pdf}
    \vspace*{-0.19in}
    \caption{\textbf{Starting configurations for} the 3 categories for \peralgabbr{} experiments and the 3 levels for physical experiments.}
    \label{fig:start-configs}
    \vspace*{-0.12in}
\end{figure}

We test \peralgabbr{} on 3 different categories of cable configurations, shown in Figure \ref{fig:start-configs}. The ordering of the categories for these experiments does not indicate varying difficulty. Rather, they are 3 categories of knot configurations to test \peralgabbr{} on.
\begin{enumerate}
    \item \textbf{Tier A1}: Loose (35-40 cm in diameter) figure 8, overhand, overhand honda, bowline, linked overhand, and figure 8 honda knots. 
    \item \textbf{Tier A2}: Dense (5-10 cm in diameter) figure 8, overhand, overhand honda, bowline, linked overhand, and figure 8 honda knots. 
    \item \textbf{Tier A3}: Fake knots (trivial configurations positioned to appear knot-like from afar). 
\end{enumerate}

We evaluate \peralgabbr{} on the 3 categories across the following ablations:
\begin{enumerate}
    \item SGTM 2.0 perception system: using a Mask R-CNN model trained on overhand and figure-8 knots for knot detection.
    \item \peralgabbr{} (-LT): replacing the \underline{L}earned \underline{T}racer with the same analytic tracer from \citet{shivakumar2022sgtm} as described in Section \ref{subsec:multicablesetup} combined with the topology identification and knot detection methods without learned tracing.
    \item \peralgabbr{} (-CC): using the learned tracer and topology identification scheme to do knot detection without \underline{C}rossing \underline{C}ancellation, the iterative algorithmic application of Reidemeister moves I and II. 
    \item \peralgabbr{}: the full perception system.
\end{enumerate}

We report the success rate of each of these algorithms in the following manner. If a knot is present, the algorithm is successful if it correctly detects the first knot and correctly labels the first undercrossing corresponding to that knot. If there are no knots, the algorithm is successful if it correctly detects no knots. 

\subsection{Tracing in Multi-Cable Settings Setup}
\label{subsec:multicablesetup}
\begin{figure}[!ht]
    \vspace*{-0.1in}
    \centering
    \includegraphics[width=1.0\linewidth]{figures/multicable_tracing.pdf}
    \vspace*{-0.19in}
    \caption{\textbf{Multi-cable tracing}: \textbf{Top row:} illustrative examples of each of the 3 tiers of difficulty for multi-cable tracing experiments. \textbf{Bottom row:} Corresponding successful traces outputted by the learned tracer.}
    \label{fig:multicable}
    \vspace*{-0.18in}
\end{figure}

For this set of perception experiments, the workspace contains a power strip. Attached to the power strip are 3 MacBook adapters, with two 3m USB-C to USB-C cables and one 2m plain white USB-C to MagSafe 3 cable. This setup is depicted in the top row of Figure \ref{fig:multicable}.
We evaluate perception on multi-cable settings on 3 tiers of difficulty.
\begin{enumerate}
    \item \textbf{Tier B1}: No knots; cables are dropped onto the workspace, one at a time. 
    \item \textbf{Tier B2}: Each cable is tied with a single knot that is 5-10 cm in diameter (one of figure 8, overhand, overhand honda, bowline, linked overhand, and figure 8 honda). The cables are then randomly dropped onto the workspace, one after another.
    \item \textbf{Tier B3}: Each cable is tied with another cable through the following 2 cable knot types (square, carrick bend, and sheet bend) with up to 3 knots in the scene. As in the other tiers, the cables are randomly dropped onto the workspace, one by one.  
\end{enumerate}
Across all 3 tiers, we assume the cable of interest cannot exit and re-enter the workspace and that crossings must be semi-planar. Additionally, we pass in the locations of all 3 adapters to the tracer and an endpoint to initialize from. To account for noise in the input images, we take 3 images of each configuration and count the experiment a success if 2/3 have the correct trace and reach their corresponding adapter, otherwise we report a failure.

We evaluate the performance of the learned tracer from \peralgabbr{} against an analytic tracer from \citet{shivakumar2022sgtm} as a baseline with scoring rules inspired by the work of \citet{lui2013tangled} and \citet{schaal2022}. The analytic tracer explores all potential paths and determines the most correct trace through a scoring metric from~\cite{shivakumar2022sgtm}. The scoring metric prefers paths that reach an endpoint, discarding traces that do not reach adapters. This is because the scoring metric sees reaching an endpoint as indicative of completing a trace. Of the paths that reach an endpoint, the trace returned is the one with the least sharp angle deviations and the highest coverage score.

\subsection{Physical Robot Untangling Setup}
Physical experiments are conducted on a single 3 m, white, braided USB-A to micro-USB cable, which is added to the workspace.

We evaluate \peralgabbr{} in untangling performance on the following 3 levels of difficulty (Figure \ref{fig:start-configs}), where all knots are upward of 10 cm in diameter, and compare performance to SGTM 2.0 \cite{shivakumar2022sgtm}:
\begin{enumerate}
    \item \textbf{Tier C1:} A cable consisting of an overhand, figure 8, or overhand honda knot. The full cable configuration has $\leq 6$ crossings.
    \item \textbf{Tier C2:} A cable consisting of a bowline, linked overhand, or figure 8 honda knot. The full cable configuration has $\geq 6$ and $< 10$ crossings.
    \item \textbf{Tier C3:} A cable consisting of 2 knots (one of a knot class from Tier C2 and one of a knot class from Tier C1). The full cable configuration has $\geq 10$ and $< 15$ crossings.
\end{enumerate}
Similar to \citet{shivakumar2022sgtm}, we use a 15-minute timeout on each rollout. We report metrics including success rate for untangling 1 and 2 knots, as well as the time to do so. We also report the success rate for termination, as well as the time required to do so, as a fraction of the number of rollouts that succeeded in fully untangling the cable.

%%%%%%%%%%%%%%%%%%%%%%%%%%%%
\section{Results}
\subsection{\peralgabbr{}}

\begin{table}[!t]
    \centering
    \newcolumntype{?}{!{\vrule width 1.75pt}}
    \label{tab:tusk-results}
    \caption{\peralgabbr{} Experiments}
    \setlength\tabcolsep{4pt}
    \begin{tabular}{|c?c|c|c|c|} \hline 
    % \multicolumn{1}{|c?}{} & \multicolumn{4}{|c|}{{\textbf{Tier A1}}} \\ \hline
    \multicolumn{1}{|c?}{} & \cellcolor{gray!15}{SGTM 2.0} & \cellcolor{gray!15}{\peralgabbr{} (-LT)} & \cellcolor{gray!15}{\peralgabbr{} (-CC)} & \cellcolor{gray!15}{\peralgabbr{}} \\ \hline
    Tier A1 & 2/30 & {14/30} & 20/30 & \textbf{24/30} \\ \hline
    Tier A2 & \textbf{28/30} & 8/30 & 21/30 & 26/30 \\ \hline
    Tier A3 & 12/30 & 14/30 & 0/30 & \textbf{19/30} \\ \hline
    Failures & (A) 30, (B) 18 & (D) 11, (F) 7 & (B) 38, (C) 5, & (B) 11, (D) 8 \\ 
     & & (G) 24, (H) 11 & (E) 6 & (F) 1 \\ \hline
    \end{tabular}
    \label{tab:tusk_exp}
    \vspace*{-0.18in}
\end{table}

As summarized in Table \ref{tab:tusk_exp}, \peralgabbr{} outperforms SGTM 2.0, \peralgabbr{} (-LT), and \peralgabbr{} (-CC) on categories 1 and 2. SGTM 2.0 outperforms \peralgabbr{} in tier A2. This is because the Mask R-CNN is trained on dense overhand and figure 8 knots. While other knots in tier A2 are out of distribution, they visually resemble the overhand and figure 8 knots. The network is, therefore, able to still detect them as knots. 

\begin{table*}[!t]
    \centering
    \newcolumntype{?}{!{\vrule width 1.75pt}}
    \caption{\peralgabbr{} and Physical Robot Experiments (90 total trials)}
    \setlength\tabcolsep{4pt}
    \begin{tabular}{|c?c|c?c|c?c|c|}
    \hline \multicolumn{1}{|c?}{} & \multicolumn{2}{|c?}{\cellcolor{gray!20}{{\textbf{Tier C1}}}} & \multicolumn{2}{|c?}{\cellcolor{gray!20}{\textbf{Tier C2}}} & \multicolumn{2}{|c|}{\cellcolor{gray!20}{\textbf{Tier C3}}} \\ 
    \hline
    \multicolumn{1}{|c?}{} & {SGTM 2.0} & {\peralgabbr{}} & {SGTM 2.0} & \peralgabbr{} & {SGTM 2.0} & {\peralgabbr{}} \\ \hline
    Knot 1 Success Rate & {11/15} & {\textbf{12/15}} & {6/15} & {\textbf{11/15}} & {9/15} & {\textbf{14/15}} \\ \hline
    Knot 2 Success Rate & {-} & {-} & {-} & {-} & 2/15 & \textbf{6/15} \\ 
    \hline
    Verification Rate & {\textbf{11/11}} & 8/12 & \textbf{6/6} & 6/11 & \textbf{1/2} & 2/6 \\
    \hline
    Avg. Knot 1 Time (min) & 1.09$\pm$0.12 & 2.11$\pm$0.25 & 3.45$\pm$0.74 & 3.88$\pm$1.09 & 1.84$\pm$0.38 & 2.00$\pm$0.42 \\
    \hline
    Avg. Knot 2 Time (min) & {-} & {-} & - & - & 3.11$\pm$1.18 & 7.45$\pm$1.55 \\ 
    \hline
    Avg. Verif. Time (min) & 5.71$\pm$0.88 & 6.13$\pm$1.44 & 6.35$\pm$1.81 & 10.10$\pm$0.67 & 5.38 & 9.58$\pm$1.48 \\
    \hline
    \multicolumn{1}{|c?}{Failures} & {(7) 4} & {(1) 2, (2) 1} & {(1) 3, (5) 6} & {(2) 2, (4) 1, (5) 1} & {(1) 3, (2) 3, (5) 3, (6) 2, (7) 2} & {(1) 2, (2) 3, (3) 1, (6) 3} \\
    \hline 
    \end{tabular}
    \label{tab:results_all}
    \vspace{-0.2in}
\end{table*}

\emph{Failure Modes}

\begin{enumerate}[(A)]
    \item The system fails to detect a knot that is present---a false negative.
    \item The system detects a knot where there is no knot present---a false positive. 
    \item The tracer retraces previously traced regions of cable.
    \item The crossing classification and correction schemes fail to infer the correct cable topology.
    \item The knot detection algorithm does not fully isolate the knot, also getting surrounding trivial loops.
    \item The trace skips a section of the true cable path.
    \item The trace is incorrect in regions containing a series of close parallel crossings.
    \item The tracer takes an incorrect turn, jumping to another cable segment.
\end{enumerate}

For SGTM 2.0, the most common failure modes are (A) and (B), where it misses knots or incorrectly identifies knots when they are out of distribution. For \peralgabbr{} (-LT), the most common failure modes are (F), (G), and (H). All 3 failures are trace-related and result in knots going undetected or being incorrectly detected. For \peralgabbr{} (-CC), the most common failure modes are (B) and (E). This is because \peralgabbr{} (-CC) is unable to distinguish between trivial loops and knots without the crossing cancellation scheme. By the same token, \peralgabbr{} (-CC) is also unable to fully isolate a knot from surrounding trivial loops. For \peralgabbr{}, the most common failure mode is (B). However, this is a derivative of failure mode (D), which is present in \peralgabbr{} (-LT), \peralgabbr{} (-CC), and \peralgabbr{}. Crossing classification is a common failure mode across all systems and is a bottleneck for accurate knot detection. In line with this observation, we hope to dig deeper into accurate crossing classification in future work. 

\begin{table}[!t]
    \centering
    \newcolumntype{?}{!{\vrule width 1.75pt}}
    \caption{Multi-Cable Tracing Results}
    \setlength\tabcolsep{4pt}
    \begin{tabular}{|c?c|c|} \hline 
    % \multicolumn{1}{|c?}{} & \multicolumn{2}{|c|}{{\textbf{Tier B1}}} \\ \hline
    \multicolumn{1}{|c?}{} & \cellcolor{gray!15}{Analytic} & \cellcolor{gray!15}{Learned} \\ \hline
    Tier B1 & 3/30 & \textbf{27/30} \\ \hline
    Tier B2 & 2/30 & \textbf{23/30} \\ \hline
    Tier B3 & 1/30 & \textbf{23/30} \\ \hline
    Failures & {(I) 3, (II) 45, (III) 36} & {(I) 14, (II) 1, (III) 2} \\ \hline
    \end{tabular}
    \label{tab:multi-cable}
    \vspace*{-0.12in}
\end{table}

\subsection{Tracing through Multi-Cable Settings} 
Table \ref{tab:multi-cable} shows that the learned tracer significantly outperforms the baseline analytic tracer on all 3 tiers of difficulty with a total of 81\% success across the tiers. 

\emph{Failure Modes:}

\begin{enumerate}[(I)]
    \item Misstep in the trace, i.e. the trace did not reach any adapter.
    \item The trace reaches the wrong adapter.
    \item The trace reaches the correct adapter but  is an incorrect trace.
\end{enumerate}

The most common failure mode for the learned tracer, especially in Tier B3, is (I). One reason for such failures is the presence of multiple twists along the cable path (particularly in Tier B3 setups, which contain more complex inter-cable knot configurations). The tracer is also prone to deviating from the correct path on encountering parallel cable segments. In Tier B2, we observe two instances of failure mode (III), where the trace was almost entirely correct in that it reached the correct adapter but skipped a section of the cable.

The most common failure modes across all tiers for the analytic tracer are (II) and (III). The analytic tracer particularly struggles in regions of close parallel cable segments and twists. As a result of the scoring metric, 87 of the 90 paths that we test reach an adapter; however, 45/90 paths did not reach the correct adapter. Even for traces that reach the correct adapter, the trace is incorrect, jumping to other cables and skipping sections of the true cable path.

\subsection{Physical Robot Untangling}

Results in Table \ref{tab:results_all} show that our \peralgabbr{}-based untangling system (29/45) outperforms SGTM 2.0 (19/45) in untangling success rate across 3 tiers of difficulty. SGTM 2.0 is, however, faster than \peralgabbr{} in each of the 3 tiers. This is due to the fact that \peralgabbr{} requires a full trace of the cable. \peralgabbr{} also requires the full cable to be in view in order to claim termination, which is difficult to achieve as the cable is 3 $\times$ as long as the width of the workspace. Because the cable falls in varying complex configurations, many of which leave the visible workspace, the untangling algorithm performs cable reveal moves before detecting knots. This increases the time needed to untangle and verify that the cable is untangled, causing some runs to time out before verification.

On the other hand, SGTM 2.0 has false termination as its main failure mode because it does not account for the cable exiting the workspace. This is beneficial for speed because the system terminates as early as possible. However, the system fails when an off-workspace knot remains and goes undetected. This allows rollouts to end quickly, even if the cable is not untangled.

\emph{Failure Modes:}
\begin{enumerate}[(1)]
    \item Incorrect actions create a complex knot. 
    \item The system misses a grasp on tight knots.
    \item The cable falls off the workspace.
    \item The cable drapes on the robot, creating an irrecoverable configuration.
    \item False termination. 
    \item Manipulation failure.
    \item Timeout.
\end{enumerate}

The main failure modes in \peralgabbr{} are (1), (2), and (6). Due to incorrect cable topology estimates, failure mode (1) occurs: a bad action causes the cable to fall into complex, irrecoverable states. Additionally, due to the limitations of the cage-pinch dilation and endpoint separation moves, knots sometimes get tighter during the process of untangling. While the perception system is still able to perceive the knot and select correct grasp points, the robot grippers bump the tight knot, moving the entire knot and causing missed grasps (2). Lastly, we experience manipulation failures while attempting some grasps as the YuMi has a conservative controller (6). We hope to resolve these hardware issues in future work. 

The main failure modes in SGTM 2.0 are (5) and (7). Perception experiments indicate that SGTM 2.0 has both false positives and false negatives for cable configurations that are out of distribution. (5) occurs when out-of-distribution knots go undetected. (7) occurs when trivial loops are identified as knots, preventing the algorithm from terminating.

% \begin{table}[!t]
%     \centering
%     \newcolumntype{?}{!{\vrule width 1.75pt}}
%     \caption{\peralgabbr{} Experiments}
%     \setlength\tabcolsep{4pt}
%     \begin{tabular}{|c?c|c|c|c|} \hline 
%     \multicolumn{1}{|c?}{} & \multicolumn{4}{|c|}{{\textbf{Tier A1}}} \\ \hline
%     \multicolumn{1}{|c?}{} & {SGTM 2.0} & \peralgabbr{} (-LT) & \peralgabbr{} (-CC) & {\peralgabbr{}}\\ \hline
%     \cellcolor{gray!15}Succ. Rate & \cellcolor{gray!15}2/30 & \cellcolor{gray!15}{14/30} & \cellcolor{gray!15}20/30 & \cellcolor{gray!15}\textbf{24/30} \\ \hline
%     Failures & {(A) 28} & {(G) 11, (H) 5} & {(B) 3, (C) 4} & {(B) 5, (D) 1} \\
%     & & & {(E) 3} & \\ \hline
    
%     \multicolumn{1}{|c?}{} & \multicolumn{4}{|c|}{\textbf{Tier A2}}\\ \hline 
%     \multicolumn{1}{|c?}{} & {SGTM 2.0} & \peralgabbr{} (-LT) & \peralgabbr{} (-CC) & {\peralgabbr{}}\\ \hline
%     \cellcolor{gray!15}Succ. Rate & \cellcolor{gray!15}\textbf{28/30} & \cellcolor{gray!15}{8/30} & \cellcolor{gray!15}21/30 & \cellcolor{gray!15}26/30 \\ \hline
%     Failures & {(A) 2} & {(D) 3, (F) 7} & {(B) 6, (E) 3} & {(B) 2, (D) 1} \\ 
%     & & {(G) 7, (H) 5} & {} & {(F) 1} \\ \hline
    
%     \multicolumn{1}{|c?}{} & \multicolumn{4}{|c|}{\textbf{Tier A3}} \\ \hline
%     \multicolumn{1}{|c?}{} & {SGTM 2.0} & \peralgabbr{} (-LT) & \peralgabbr{} (-CC) & {\peralgabbr{}}\\ \hline
%     \cellcolor{gray!15}Succ. Rate & \cellcolor{gray!15}12/30 & \cellcolor{gray!15}{14/30} & \cellcolor{gray!15}0/30 & \cellcolor{gray!15}\textbf{19/30} \\ \hline
%     Failures & {(B) 18} & {(D) 9, (G) 6} & {(B) 29, (C) 1} & {(B) 4, (D) 7}\\
%     & & {(H) 1} & {} & {} \\ \hline
%     \end{tabular}
%     \label{tab:tusk_exp}
% \end{table}

% \begin{table}[!t]
%     \centering
%     \newcolumntype{?}{!{\vrule width 1.75pt}}
%     \caption{Multi-Cable Tracing Results}
%     \setlength\tabcolsep{4pt}
%     \begin{tabular}{|c?c|c|} \hline 
%     \multicolumn{1}{|c?}{} & \multicolumn{2}{|c|}{{\textbf{Tier B1}}} \\ \hline
%     \multicolumn{1}{|c?}{} & {Analytic} & Learned\\ \hline
%     \cellcolor{gray!15}Succ. Rate & \cellcolor{gray!15}3/30 & \cellcolor{gray!15}{\textbf{27/30}} \\ \hline
%     Failures & {(II) 14, (III) 13} & {(I) 3} \\ \hline
    
%     \multicolumn{1}{|c?}{} & \multicolumn{2}{|c|}{{\textbf{Tier B2}}} \\ \hline
%     \multicolumn{1}{|c?}{} & {Analytic} & Learned\\ \hline
%     \cellcolor{gray!15}Succ. Rate & \cellcolor{gray!15}2/30 & \cellcolor{gray!15}{\textbf{23/30}} \\ \hline
%     Failures & {(I) 1, (II) 13, (III) 14} & {(I) 4, (II) 1, (III) 2} \\ \hline
    
%     \multicolumn{1}{|c?}{} & \multicolumn{2}{|c|}{{\textbf{Tier B3}}} \\ \hline
%     \multicolumn{1}{|c?}{} & {Analytic} & Learned\\ \hline
%     \cellcolor{gray!15}Succ. Rate & \cellcolor{gray!15}1/30 & \cellcolor{gray!15}{\textbf{23/30}} \\ \hline
%     Failures & {(I) 2, (II) 18, (III) 9} & {(I) 7} \\ \hline
%     \end{tabular}
%     \label{tab:multi-cable}
% \end{table}