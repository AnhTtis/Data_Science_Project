
\section{Limitations and Future Work}

This approach has notable limitations: The robot system executing \peralgabbr{} still depends on a depth camera. Future work will address this and aim to remove depth for the 1D deformable grasping task. \peralgabbr{} is also dependent on a mono-color workspace. Future work will investigate generalizing \peralgabbr{} to function independent of backgrounds, allowing the system to work in a non-solid, multi-color workspace that imitates home environments. Additionally, in real world scenarios, cables vary in color and thickness, so we will also pursue making \peralgabbr{} invariant to cable appearance. Lastly, on the manipulation side, the untangling system struggles to grasp tight loops. Future work will explore servoing methods to improve grasping reliability in tight regions.

\section{Conclusion}
%things to mention in discussion -justin:
%1. assumptions on background (and how you might generalize to diff backgrounds)
%2. generalizing to more types of cables? colors, thicknesses etc
%3. why assume >=5cm loops? (too small for grippers) but how might you deal with that?
%4. 

This work presents \peralgabbr{}, a perception pipeline that iteratively traces and determines the topology of semi-planar cable configurations, detects knots given the cable state, and detects graspable points for untangling the knots. Experiments show that \peralgabbr{} can successfully trace a single cable in a multi-cable setting with 81\% accuracy, significantly outperforming an analytic baseline. \peralgabbr{} is also able to detect knots with 77\% accuracy. Lastly, when \peralgabbr{} is applied to a robot untangling problem, the system is able to achieve 64\% success in untangling.
