\section{Introduction}
\label{sec:intro}

In industrial and household settings, tangled long cables can pose a threat to the safety of individuals, especially older or at-risk adults, by impeding their movement. Additionally, in environments where heavy machinery is operated, cables can get caught in moving parts and cause potential damage or harm \cite{sanchez2018robotic, mayer2008system, van2010superhuman, yamakawa2007one}.

% When cables are tangled, it can be challenging to determine which cables are connected to which devices, making it difficult to diagnose and repair problems

Untangling long cables can be difficult due to challenges in manipulation and perception alike. Developing manipulation primitives that can adapt to different knot topologies is non-trivial since knot dynamics are challenging to predict and can depend on unobservable parameters of the cable like stiffness. Also, estimating cable state from an RGB image is difficult since long cables often fall into complex configurations with many crossings. Long cables can also contain a significant amount of free cable (referred to as \emph{slack}), which can occlude and inhibit the perception of true knots from an overhead image. The task of autonomously untangling cables requires a generalizable system that can track a cable path in complex configurations and handle the wide distribution of knots present in long cables.

\begin{figure}[!ht]
    \centering
    \includegraphics[width=1.0\linewidth]{figures/splash.pdf}
    \caption{\textbf{\peralgabbr:} \peralgabbr\ first performs cable tracing (1, 2). The trace is shown through a rainbow gradient (from violet to purple), depicting the sequence in which the cable is traced. After tracing, \peralgabbr{} does crossing recognition (3) to obtain the full topology of the cable. Next, using crossing cancellation rules from knot theory, it analytically determines knots (4) in the cable. Next, \peralgabbr\ surveys possible cage-pinch points (5) and selects the best candidate points to grasp to execute a cage-pinch dilation action, untangling the knot (6).}
    \label{fig:splash}
    \vspace*{-0.28in}
\end{figure}

Much of prior work bypasses full state estimation by employing object detection networks and keypoint selection networks to identify knots and grasp points directly based on geometric patterns \cite{viswanath2022autonomously, shivakumar2022sgtm}. These works can disentangle 2 types of knots (overhand and figure-8), but the methods do not generalize well to the large number of complex configurations long cables can form. Other prior work is able to achieve some generality for knot disentanglement, but only for short cables up to 15\,cm in length. In such short cables, the knot configurations are less complex, and there is little difficulty with slack management, eliminating the need to estimate the cable path \cite{viswanath2021disentangling, grannen2020untangling, sundaresan2021untangling, song2019untangling}. This work considers long cables up to 3 meters in length consisting of semi-planar knots, i.e. knots comprised of semi-planar crossings, where each crossing consists of at most 2 cable segments when viewed from above. The single-cable semi-planar knots considered in this work are overhand, figure 8, overhand honda, bowline, linked overhand, and figure 8 honda knots. The double-cable semi-planar knots considered in this work are carrick bend, sheet bend, and square knots. This paper focuses on high-accuracy state estimation techniques and algorithms grounded in knot theory to process the results of state estimation and select untangling actions for a broader class of knots.

% \mallika {change (flow of contributions) to this? 1) A high-accuracy state estimation system (real tracer, crossing detection) and 2) algorithms to untangle given state estimation (crossing-based knot detection, untangling point selection) looks good}

% \jainil{I think the split above sounds good. My only question is: what is SUPERMAN? Is it the collection of algorithms to untangle (2 in Mallika's proposal), which is the way it currently stands, or the overall system? The name (which includes trace perception) seems to suggest that it's the full system. }
This paper presents \peralgabbr{}, which makes the following contributions:
\begin{enumerate}
    \item A novel cable state estimator consisting of a learning-based iterative tracer and a crossing classifier with a crossing correction algorithm. 
    \item An analytic knot detection algorithm and untangling point selection algorithm given the cable state estimates.
    % \begin{enumerate}
    %     \item A novel learning-based full cable spline and topology estimator using just RGB inputs.
    %     \item A new analytic, crossing-based knot detection algorithm that, unlike much prior work, avoids unnecessarily disentangling non-essential crossings.
    %     \item An untangling point selection algorithm that determines optimal graspable points for executing untangling action primitives.
    % \end{enumerate}
    \item Data from physical experiments using \peralgabbr{} to untangle semi-planar knots. Results suggest \peralgabbr{} can correctly trace and segment a single cable in multi-cable settings with 81\% accuracy, detect knots with 77\% accuracy, and function in a physical system for untangling semi-planar knots with 64\% untangling success in under 8 minutes.
\end{enumerate}

