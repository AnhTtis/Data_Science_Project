\section{Related Work}

\subsection{Deformable Object Manipulation}
Robot manipulation of deformable objects, such as cables (1D), fabric (2D), and bags (3D), is difficult because they have a near-infinite state space, can form self-occlusions, and are difficult to model. This study focuses on the problem of untangling knots in long cables. Recently, there has been progress in deformable manipulation, including algorithms for untangling cables ~\cite{grannen2020untangling, sundaresan2021untangling, viswanath2021disentangling,lui2013tangled}, smoothing and folding fabric~\cite{seita2019deep, weng2022fabricflownet,ganapathi2020learning,hoque2020visuospatial,kollar2022simnet,luv2022,hoque2022reach}, and placing objects into bags~\cite{seita2020learning,lawrence2023bag}.

Methods for intelligently and autonomously manipulating deformable objects lie on a spectrum ranging from completely model-free, directly perception-driven approaches to those that directly estimate the state of the object of interest and then perform planning on it. 

Examples of model-based methods for deformable objects are dense descriptors ~\cite{florence2018dense}, which have been applied to cable knot tying ~\cite{sundaresan2020learning} and fabric smoothing ~\cite{ganapathi2020learning}, 
as well as visual dynamics models for non-knotted cables ~\cite{yan2020learning, wang2019learning2} and fabric ~\cite{hoque2020visuospatial, yan2020learning, lin2022learning}. Model-free approaches include reinforcement or self-supervised learning for fabric smoothing and folding ~\cite{matas2018sim, wu2019learning, lee2020learning,speedfolding} and straightening curved ropes~\cite{wu2019learning}, or directly imitating human actions~\cite{seita2019deep}.

\subsection{Cable Perception, Manipulation, and Untangling}
Pioneering research in untangling, such as that conducted by Lui and Saxena~\cite{liu2013untangling}, relies on decomposing point clouds of rope into segments which are refined into a graphical representation of the cable's structure based on priors on cable behavior such as bending radius. Other methods learn visual models for cable manipulation over which to plan~\cite{nair2017vismodel}, investigate iteratively refining dynamic actions~\cite{chi2022irp}, or use approximate state dynamics along with a learned error function~\cite{dmitry2020trust}. Fusing point clouds across time has shown success in tracking segments of cable provided they are not tangled on themselves~\cite{abbeeltrackingcable,tracking2}.

Studies on dense knots, such as those by \citet{grannen2020untangling} and \citet{sundaresan2021untangling}, employ learning-based keypoint detection to parameterize action primitives for untangling isolated knots. The work of \citet{viswanath2022autonomously} expands on this idea to long (3m) cables, with the addition of a learned knot detection pipeline. This approach works well for a certain range of knot types within the model's training distribution as it uses end-to-end perception systems trained on human labels. Scaling these methods to arbitrary knot types would require an intractable amount of human labels, motivating the local topology estimation approach in this work.

Prior cable state estimation work includes that led by ~\citet{dlotase}, which estimates the topological state of multiple ropes against varying backgrounds and uses primitives to untangle rope configurations consisting primarily of loops with 2 to 4 crossings. Additional linear deformable tracing work includes that of ~\citet{schaal2022}, which models cables as chains of jointed cylindrical bodies, then uses the model for routing tasks. The works of \citet{nurbtracer} and \citet{britishcvsurgery} trace surgical strings in stereo or mono images by optimizing a continuous spline representation. In contrast, this work primarily focuses on much longer cables with a greater variety of configurations, for which analytical methods struggle to differentiate nearby, twisted cables. Some prior work approaches this problem~\cite{Parmar_2013} but does not fully estimate cable state, only identifying crossings. 