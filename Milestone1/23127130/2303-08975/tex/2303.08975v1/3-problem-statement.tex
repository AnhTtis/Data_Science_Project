\section{Problem Statement}
\label{sec:ps}

The objective is to bring a long (3 m) cable containing semi-planar knots into an untangled configuration, where no knots remain (knots defined in Section \ref{sec:knot_def}).

% The objective for the robot is to untangle a long (3 m) cable consisting of any semi-planar knot from overhead RGB image observations. We use a bimanual robot to execute manipulation primitives until the cable reaches a fully untangled state with no knots.

The workspace is defined by an $(x, y, z)$ coordinate system and consists of a bilateral robot and a foam-padded manipulation surface, which lies in the $(x, y)$ plane. The workspace also contains a fixed overhead RGB-D camera that faces the manipulation surface and outputs grayscale images and depth data. However, depth data is not used in \peralgabbr{}. Rather, it is only used during manipulation. We work with a 300 cm cable. We assume the cable is visually distinguishable from the manipulation surface, its initial configuration has at least one endpoint visible, and is semi-planar as assumed in \citet{grannen2020untangling}, meaning each crossing in the knot has at most 2 intersecting cable segments. For perception experiments, we work with knots as tight as 5\,cm in diameter. For physical experiments, due to robot graspability constraints, we work with knots of varying density, or approximate diameter, upwards of 10 cm in diameter. We define cable state to be $\theta(s) = \{(x(s), y(s), z(s))\}$ where $s$ is an arc-length parameter that ranges $[0, 1]$, representing the normalized length of the cable. Here, $(x(s), y(s), z(s))$ is the location of a cable point at a normalized arc length of $s$ from the cable's first endpoint. We also define the range of $\theta(s)$---that is, the set of all points on the cable at time $t$---to be $\mathcal{C}_t$.  

% \todo{I don't think we should define things like this. We should be looking at the cable configuration as a whole} Note that although the knot itself is semi-planar, crossings in the cable can consist of more than 2 cable segments as slack can fall on top of the knot or form loops elsewhere in the cable such that crossings with more than 2 cable segments appear. \todo{Check note above + definitions below. Also, highlight in methods? That is, the fact that the tracer can handle more than 2 crossings}.

% \subsection{Cable Segment Definition (For Semi-Planar Knots)}
% Suppose there exists a crossing on the cable path at time $t$. Owing to the semi-planar nature of the knot, there will be two sightings of this crossing (once as an undercrossing, once as an overcrossing) on the cable path $P_t(s)$. Assume that the two indices at which this crossing is encountered on $P_t(s)$ are $a$ and $b$ ($a < b$). 

% We define the cable segments intersecting at this crossing to be $s_1$ and $s_2$. Here, $s_1$ and $s_2 \subset C_t$ such that $s_1 = \{P_t(s) \mid s \in [0, a]\}$ and $s_2 = \{P_t(s) \mid s \in [b, 1]\}$.

\subsection{Knot Definition}
\label{sec:knot_def}
Consider a pair of points $p_1$ and $p_2$ on the cable path at time $t$ with ($p_1, p_2 \in \mathcal{C}_t)$. Knot theory strictly operates with closed loops, so to form a loop with the current setup, we construct an imaginary cable segment with no crossings joining $p_1$ to $p_2$ \cite{reidemeister1983knot}. This imaginary cable segment passes above the manipulation surface to complete the loop between $p_1$ and $p_2$ (``$p_1\rightarrow p_2$ loop").
A knot exists between $p_1$ and $p_2$ at time $t$ if no combination of Reidemeister moves I, II (both shown in Figure \ref{fig:reid_cc}), and III can simplify the $p_1 \rightarrow p_2$ loop to an unknot, i.e. a crossing-free loop. In this paper, we aim to untangle semi-planar knots. For convenience, we define an indicator function $k(s):[0,1]\rightarrow\{0,1\}$ which is 1 if the point $\theta(s)$ lies between any such points $p_1$ and $p_2$, and 0 otherwise.

\begin{figure}[!t]
    \centering
    \includegraphics[width=1.0\linewidth]{figures/crossing_cancellation.pdf}
    \caption{\textbf{Reidemeister Moves and Crossing Cancellation}: Top left depicts Reidemeister Move II. Top right depicts Reidemeister Move I. The bottom row shows that by algorithmically applying Reidemeister Moves II and I, we can cancel trivial loops, even if they visually appear as knots.}
    \label{fig:reid_cc}
    % \vspace*{-0.25in}
\end{figure}

 
Based on the above knot definition, this objective is to remove all knots, such that $\int k(s)_0^1=0$. In other words, the cable, if treated as a closed loop from the endpoints, can be deformed into an unknot. We measure the success rate of the system at removing knots, as well as the time taken to remove these knots. 