\documentclass[amsfonts,amsmath,prd,preprint,nofootinbib]{revtex4}
%\documentclass[reprint,amsmath,amssymb,aps]{revtex4-2}
\newcommand{\beq}{\begin{equation}}
\newcommand{\eeq}{\end{equation}}
\newcommand{\eqal}[1]{\begin{align*}#1\end{align*}}
\newcommand{\eqaln}[1]{\begin{align}#1\end{align}}
\newcommand{\bsp}{\begin{split}}
\newcommand{\erfc}{\mathop{\rm erfc}}
\usepackage{epsfig,bbm,cancel,ulem}
%\usepackage{hyperref}
\usepackage[breaklinks=true]{hyperref}
\usepackage{xcolor}
\usepackage{wasysym}
\usepackage[mathscr]{eucal}
\usepackage{multirow}
\usepackage{amsmath}
\usepackage{enumitem}

\newcommand{\ardian}[1]{\textit{#1}}
\newcommand{\mc}[1]{\mathcal{#1}}

\begin{document}

\title{Strong lensing of the nonsingular black hole in nonlinear electrodynamics framework}

\author{M.~F.~Ishlah}
%\email{muhammad.fauzan79@sci.ui.ac.id.}
\author{F.~P.~Pratama}
%\email{fernanda.putra@ui.ac.id.}
\author{I.~Alfredo}
%\email{immanuel.alfredo@ui.ac.id.}
\author{H.~S.~Ramadhan}
%\email{hramad@sci.ui.ac.id}

\affiliation{Departemen Fisika, FMIPA, Universitas Indonesia, Depok, 16424, Indonesia. }
\def\changenote#1{\footnote{\bf #1}}

\begin{abstract}

We model supermassive black hole as the nonsingular black hole viewed from the point of view of Ayon-Beato-Garcia nonlinear electrodynamics (NLED) and present a complete study of their corresponding strong gravitational lensing. The NLED modifies the the photon's geodesic, and our calculations show that such effect increases the corresponding photon sphere radius and image separation, but decreases the magnification. We also present the intensity map of the black hole's shadow image with thin accretion disk.
\end{abstract}

\maketitle
\thispagestyle{empty}
%\section{Introduction}
\setcounter{page}{1}

\section{Introduction}
\label{intro}
Black hole (BH) is one of the most straightforward yet profound prediction of General Relativity (GR). Its extreme gravity distorts its surrounding spacetime and bends light, creating (among many things) the {\it gravitational lensing} phenomenon. The recent observation by Event Horizon Telescope (EHT) that successfully captured the visual images of the superheavy BHs M87*~\cite{EventHorizonTelescope:2019dse} and Sgr A*~\cite{akiyama_first_2022} has established a triumph for the gravitational lensing as a means to empirically prove black hole's existence. By ``image" here is the corresponding {\it shadow}~\cite{Falcke:1999pj} surrounded by accreting materials that emits and lenses light from the nearby background source. 

Theoretically, the study of gravitational lensing is as old as GR itself (see, for example, \cite{schneider:1992, Perlick:2004tq} and the references therein), but it was Darwin who first applied it for Schwarzschild BH~\cite{Darwin_gravity_1959}. His exact calculation on the deflection angle shows that at small impact parameters there exists a critical value (close to the corresponding photon sphere) where the deflection angle suffers from logarithmic divergence~\cite{Bozza:2010xqn}, beyond which photons fall into the horizon. His results were later rediscovered and developed by other authors, for example in~\cite{Luminet:1979nyg, Chandrasekhar:1985kt}. In the last two decades the study of gravitational lensing in the strong deflection limit received revival and extensive elaboration~\cite{Frittelli:1999yf, Virbhadra:1999nm, bozzaStrongFieldLimit2001, bozzaGravitationalLensingStrong2002}. In particular, Bozza shows in~\cite{bozzaGravitationalLensingStrong2002} that the analytical expansion of the strong deflection angle in the limit of $r\rightarrow r_{ps}$ ($r_{ps}$ being the photon sphere radius) is given by
\begin{equation}
	\alpha(b) = -\tilde{a}_1 \log \left( \frac{b}{b_c} - 1\right) + \tilde{a}_2 + O(b - b_c).
	\label{eq:Defleksi_b}
\end{equation}
with $\tilde{a}_1$ and $\tilde{a}_2$ some constants. Upon closer inspection, Tsukamoto gave correction to the higher order expansion~\cite{Tsukamoto:2016qro, Tsukamoto:2016jzh}. This result on Schwarszchild was extended to the case of Reissner-Nordstrom by Eiroa {\it et al}~\cite{Eiroa:2002mk}, while the strong lensing in Kerr BH was studied in~\cite{Bozza:2002af, Vazquez:2003zm, Bozza:2006nm}.%  While gravitational lensing under weak field limit was extensively studied by numerous authors method to calculate strong gravitational lensing was devised, for example, by Bozza to discriminate types of black holes as well as their signature from other compact objects~\cite{bozzaStrongFieldLimit2001, bozzaGravitationalLensingStrong2002}. % These optical observables such as the lensing effect and the emission of the accretion disk near the shadow could help identifying the physical properties such as the existence of event horizon and the black hole mass.

Probably the most intriguing property of black holes is the existence of singularity due to the gravitational collapse~\cite{Oppenheimer:1939ue}. It was believed that such singularity is an inherent solution of general relativity, but the stable ones (like all observable black holes) are disconnected from the observers by event horizon~\cite{Penrose:1964wq}. Nevertheless, Bardeen in 1968 constructed a metric function that produced nonsingular spacetime~\cite{bardeenNonsingularGeneralrelativisticGravitational1968}. The metric and all invariants are devoid of singularity everywhere, including at $r=0$. Instead, we have regular de Sitter space at the core. (For an excellent review on regular BH see, for example, \cite{Ansoldi:2008jw}.) The strong lensing phenomenon around Bardeen BH has been studied in~\cite{Eiroa:2010wm, Wei:2015qca}. 

At first nobody knows what kind of matter that sources the Bardeen geometry, but later Ayon-Beato and Garcia (ABG) realized that this nonsingular metric can be obtained as solutions of Einstein's equations coupled to some nonlinear electrodynamics (NLED) source~\cite{Ayon-Beato:1998hmi, ayon-beatoBardeenModelNonlinear2000}. Invoking NLED turns out to have profound impact on the geodesic of test photon. Novello {\it et al} showed that in nonlinear electrodynamics background, photon moves in an effective modified geometry~\cite{Novello:1999pg}, and this radically modifies the corresponding optical observables. 
%There have been numerous We can infer an object by studying their gravitational signature. Depending on the characteristics of the objects, the gravitational field they produce will be different. Thus, affecting the motion of particles around the object. One consquences of this phenomena are gravitational lensing and black hole shadow radius. A photon that travels near a massive body could be deflected, this will results an object analogous to an optical lense therefore this phenomena is known as gravitational lensing, this phenomena can also occur in compact objects such as black holes. The light that travels too close to a black hole can be trapped into the horizon or travelling around the black hole in circle and cannot escape to the observer, this will create a dark region known as black hole shadow.  
In this work, we discuss the effect of effective geometry to the lensing phenomenon in the Bardeen BH using one of the ABG's NLED model. In particular, we calculate the image separation and magnification. We also investigate its shadow with thin accretion disks. This paper is organized as follows. In Sec.~\ref{sec:bardeen} we briefly review the regular Bardeen solution and its corresponding ABG models. In Sec.~\ref{sec:eff} we present the effective metric of ABG and the corresponding photon sphere. Sec.~\ref{sec:deflection} is devoted to applying the Bozza's and Tsukamoto's strong lensing formalism to our model. Sec.~\ref{sec:lensing} is devoted to calculating the strong lensing observables. In Sec.~\ref{sec:shadow} we compare the shadow images of the NLED model to the RN as well as the ordinary Bardeen in the thin accretion disk approximation. Finally, we summarize our findings in Sec.~\ref{sec:conc}.
%Previously theoretical studies of the shadow of compact bodies with accretion disks such as black hole, wormhole, gravastar, or naked singularity has been studied by \cite{harko_can_2009}, \cite{bambi_can_2013}, \cite{he_feature_2022}, \cite{joshi_shadow_2020}.
%By studying the observational signatures of regular black holes could enlighten us on the existence of regular black holes as astrophysical black hole candidate.

\section{Bardeen Spacetime}
\label{sec:bardeen}
%\subsection{Bardeen}
The Bardeen metric is given by~\cite{bardeenNonsingularGeneralrelativisticGravitational1968}:
\begin{equation}
	ds^2=-f(r)dt^2+f(r)^{-1}dr^2+r^2d\Omega^2,
\end{equation}
with
\begin{equation}
	\label{bardeen}
	f(r)\equiv1 - \frac{2mr^2}{(r^2 + q^2)^\frac{3}{2}},
\end{equation}
and $q$ is the charge. This spacetime is regular at $r = 0$, as can easily be seen from the Kretschmann scalar:
\begin{equation}
	\lim_{r\rightarrow0}R^{\alpha\beta\gamma\delta}R_{\alpha\beta\gamma\delta}=\frac{96m^2}{q^{8/3}},    
\end{equation}
while the metric behaves de Sitter-like
\begin{equation}
	f(r)\approx 1-\frac{2m}{q^3}r^2.
\end{equation}
The horizons $r_h$ are given by the roots of
\begin{equation}
	\left(r_h^2+q^2\right)^3-4m^2r_h^4=0.
\end{equation}
Bardeen black hole can, in general, possess up to two horizons. The extremal condition is achieved when~\cite{ayon-beatoBardeenModelNonlinear2000}
\begin{equation}
	q^2=\frac{16}{27}m^2\ \rightarrow r_{extr}=\sqrt{\frac{32}{27}}m.
\end{equation}
Ayon-Beato and Garcia proposed the NLED matter to source the Bardeen spacetime, given in~\cite{Ayon-Beato:1998hmi}. This model, however, produces a slightly different metric function than the original Bardeen,
\begin{equation}
	f(r)=1 - \frac{2mr^2}{(r^2 + q^2)^\frac{3}{2}}+\frac{q^2 r^2}{\left(r+2+q^2\right)^2}.
	\label{ABGori}
\end{equation}
Strong lensing of this particular model has been discussed in~\cite{ghaffarnejadGravitationalLensingCharged2018}. Later ABG considered a simpler NLED sourced by magnetic monopole as follows~\cite{ayon-beatoBardeenModelNonlinear2000},
\begin{equation}
	\mathcal{L} = \frac{3}{ 2 s q^2} \left( \frac{\sqrt{2q^2 F}}{1 + \sqrt{2q^2 F}}\right)^{5/2},
	\label{eq:LagrangeABG}
\end{equation}
where $F\equiv\frac{1}{4}F^{\mu\nu}F_{\mu\nu}$ and $s\equiv q/2m$. By inserting the monopole ansatz $A_{\mu}=\delta^{\varphi}_{\mu}q\left(1-\cos\theta\right)$, the field strength becomes $F=q^2/2r^4$ and the Lagrangian produces the metric solution Eq.~\eqref{bardeen}.%, for magnetic monopole source the $F$ will become $F = \frac{q^2}{2r^4}$. Substituting the Lagrangian to Einstein's Hilbert action will give us the Bardeen metric in (1).
% \subsection{Hayward}
% Hayward spacetime is represented by the metric below \cite{hayward_formation_2006}.
% \begin{equation}
	% 	f(r) = 1 - \frac{2Mr^2}{r^3 + 2Ml^2}
	% \end{equation}
% where $l$ is a constant related to Hubble Length.
% The spacetime we will evaluate is the modified Hayward spacetime by Kruglov \cite{kruglovMagnetizedBlackHoles2017}. It is modified by changing the $M$ component by mass distribution function $M(r)$
% \begin{equation}
	% 	f(r) = 1 - \frac{2M(r)r^2}{r^3 + 2M(r)l^2}
	% \end{equation}
% The mass distribution follows the Lagrangian proposed by Bronnikov \cite{bronnikovRegularMagneticBlack2001}. It is modified using $\rho = -\mathcal{L}$, where $\mathcal{L}$
% \begin{equation}
	% 	\mathcal{L} = - \frac{F}{\cosh^2\sqrt[4]{|\beta F|}}
	% \end{equation}
% \begin{equation}
	% 	M = M(r) = mo + \int_{0}^{\infty}\rho(r)r^2dr - \int_{r}^{\infty}\rho(r)r^2 dr.
	% \end{equation}
% The magnetic mass is defined as
% \begin{equation}
	% 	\begin{split}
		% 			m_M &= \int_{0}^{\infty}\rho(r)r^2dr - \int_{r}^{\infty}\rho(r)r^2 dr, \
		% 			& =\frac{q^\frac{3}{2}}{2^\frac{3}{4}\beta^\frac{1}{4}}.
		% \end{split}
	% \end{equation}
% Substituting $M(r)$ we obtain the modified Hayward metric
% \begin{eqnarray}
	% 	f(r) = 1 - \frac{2\left(mo + \frac{q^\frac{3}{2}}{2^\frac{3}{4}\beta^\frac{1}{4}} - \frac{q^\frac{3}{2}}{2^\frac{3}{4}\beta^\frac{1}{4}}\tanh\left(\frac{\beta^\frac{1}{4}\sqrt{q}}{2^\frac{1}{4}r}\right)\right)r^2}{r^3 + 2\left(mo + \frac{q^\frac{3}{2}}{2^\frac{3}{4}\beta^\frac{1}{4}} - \frac{q^\frac{3}{2}}{2^\frac{3}{4}\beta^\frac{1}{4}}\tanh\left(\frac{\beta^\frac{1}{4}\sqrt{q}}{2^\frac{1}{4}r}\right)\right)l^2}.\nonumber\
	% \end{eqnarray}

%===================================================================================================%
\section{Effective Geometry}
\label{sec:eff}

The NLED Lagrangian above induces the effective metric tensor~\cite{Novello:1999pg} 
\begin{equation}
	g^{\mu\nu}_{eff} = g^{\mu\nu} - \frac{4\mathcal{L}_{FF}}{\mathcal{L}_{F}}F^{\mu}_{\alpha}F^{\alpha\nu},
\end{equation}
where $\mathcal{L}_{A}\equiv\partial\mathcal{L}/\partial A$. This, in turn, yields the effective length element
\begin{equation}
	ds^2_{eff} = -f(r)dt^2 + f(r)^{-1}dr^2 + h_m(r)r^2d^2\Omega,
\end{equation} 
where
\begin{equation}
	h_m(r) = \left(1+\frac{4\mathcal{L}_{FF}}{\mathcal{L}_F}\frac{q^2}{r^4}\right)^{-1}.
\end{equation}
Inserting the Lagrangian~\eqref{eq:LagrangeABG} we obtain
\begin{equation}
	h_{ABG}(r) = \left( 1 - \frac{2(6q^2 - r^2)}{(q^2 + r^2)} \right)^{-1}.
\end{equation}

From the corresponding geodesic equation it is not difficult to see that the radial equation satisfies
\begin{equation}
	\frac{1}{2}\dot{r}^2+V_{eff}=0,
\end{equation}
%\begin{equation}
%    \dot{r}^2 = E^2 - %\frac{f(r)}{h(r)}\frac{\mathbb{L}^2}{r^2}
%\end{equation}
where we define the effective potential $V_{eff}$ as
\begin{equation}
	V_{eff} \equiv \frac{f(r)}{h(r)}\frac{\mathbb{L}^2}{r^2}.
\end{equation}
The corresponding photon sphere radius is given by the largest positive root of the following condition,~\cite{bozzaGravitationalLensingStrong2002}
\begin{equation}
	\frac{f'(r_{ps})}{f(r_{ps})} - \frac{2}{r_{ps}} -  \frac{h'(r_{ps})}{h(r_{ps})}=0.
\end{equation}
This yields,
%\begin{widetext}
\begin{eqnarray}
	\frac{2}{r_{ps}}+\frac{28q^2r_{ps}}{11q^4+8q^2r_{ps}^2-3r_{ps}^4}+\frac{m\left(4q^2r_{ps}-2r_{ps}^3\right)}{\left(q^2+r_{ps}^2\right)\left[\left(q^2+r_{ps}^2\right)^{3/2}-2mr_{ps}^2\right]}=0.
\end{eqnarray}
%\end{widetext}
\begin{figure}
	\centering
	\includegraphics[scale=0.6]{ABGPhotonSphere.pdf}
	\caption{Photon Sphere of Bardeen with ABG source as function of charge ($q$) for $m=1$.}
	\label{fig:ABGphotonsphere1}
\end{figure}
By solving the roots numerically, the behavior of $r_{ps}$ as a function of q,  $r_{ps}=r_{ps}(m=1,q)$, is shown in Fig.\ref{fig:ABGphotonsphere1}. It is shown that the photon sphere decreases as the charge increases until some critical value $q_{crit}$ where $r_{ps}(m=1,q=q_{crit})$ is minimum, beyond which $r_{ps}$ starts increasing without bound. Interestingly, the critical value $q_{crit}$ is not given by the extremal charge $q=\sqrt{16/27}m$, as in the Bardeen case. Rather, $q_{crit}=16/27\ m=0.592$. \begin{figure}
	\centering
	% \includegraphics[scale=0.38]{rpsm.pdf}
	\includegraphics[scale=0.6]{RadiusOfPhotonSphereForVariousChargesValues.pdf}
	\caption{$r_{ps}$ as function of $m$ for a variation of charge $q$.}
	\label{fig:grafik_rpsm}
\end{figure}
\begin{figure}
	\centering
	\includegraphics[scale=0.8]{rpsfull2.pdf}
	% \includegraphics[scale=0.38]{radius of photon sphere comparisons.pdf}
	\caption{$r_{ps}$ as function of $q$ for different models.}
	\label{fig:grafik_rpsfull}
\end{figure}
In Fig.~\ref{fig:grafik_rpsm} we show how $r_{ps}$ varies with $m$ for several values of $q$. They differ only at small $m$. When the mass is large, $r_{ps}$ for different $q$ asymptote to a single gradient. In Fig.~\ref{fig:grafik_rpsfull} we show the deviation of $r_{ps}$ as a function of $q$ from the Schwarzschild (charge-less condition). The ABG model we consider here falls between the Schwarzschild and the RN at large $q$, unlike the original Bardeen which falls the fastest.
%===================================================================================================%
\section{Deflection Angle in the Strong Field Limit}
\label{sec:deflection}

% For model static, and spherically symmetric metric expressed
% \begin{equation}
	% 	ds^2=-A(r)dt^2 + B(r)dr^2 + C(r)(d\theta^2 + \sin^2\theta d\phi^2)
	% 	\label{eq:metrik_umum}
	% \end{equation}
% Have light deflection
From the spherical symmetry and staticity conditions, Noether's theorem dictates that this spacetime has constants of motion, the total test particle's energy $E$ and angular momentum $\mathbb{L}$, related to the $\partial_t$ and $\partial_{\varphi}$ Killing vectors, respectively. We define the impact parameter for photon as 
\begin{equation}
	\label{critimpact}
	b(r_0)\equiv\frac{\mathbb{L}}{E}=\sqrt{\frac{h_m(r_0)r_0^2}{f(r_0)}}.
\end{equation}
%where for photon we have
%\begin{equation}
%\frac{\mathbb{L}}{E} = \sqrt{\frac{h_m(r_0)r_0^2}{f(r_0)}}.
%\end{equation}
%There exists a minimum value of impact parameter before which photon plunges into the BH, $b_c\left(r_{ps}\right)\equiv\lim_{r_0\rightarrow r_{ps}}b\left(r_0\right)$.


Solving the null geodesic equation, the general expression for  bending angle of light rays can be expressed as (see, for example, in~\cite{Weinberg:1972kfs})
% \begin{equation}
	%     \alpha(r_0) = 2 \int_{r_0}^\infty \sqrt{\frac{B(r)}{C(r)R(r)}} \,dr - \pi
	%    \label{eq:DefleksiSudut_int}
	% \end{equation}
\begin{equation}
	\label{alpha}
	\alpha(r_0) = 2 \int_{r_0}^\infty \sqrt{\frac{1}{r^2 f(r) h(r) R(r)}}dr - \pi, 
	% & = 2 \int_{r_0}^\infty 
\end{equation}
where
\begin{equation}
	% R(r) &= \frac{C(r)}{b^2 A(r) } - 1 \
	R(r) = \frac{r^2 h(r)}{b^2 f(r) } - 1. 
	% & = \frac{f(r_0)}{r_0^2 h(r_0)} \frac{r^2 h(r)}{f(r)} - 1
\end{equation}
The integral is divergent at $r_0\rightarrow r_{ps}$. To circumvent this problem we define $z\equiv1-r_0/r$~\cite{Tsukamoto:2016jzh} 
%\begin{equation}
%    z\equiv1-\frac{r_0}{r},
%\end{equation}
and write Eq.~\eqref{alpha} as
\begin{equation}
	\label{ar0}
	\alpha(r_0) = \int^1_0\mathcal{H}(z,r_0) dz-\pi,
\end{equation}
with 
\begin{equation}
	\mathcal{H}(z,r_0) \equiv  \frac{2\left(1-z\right)^2}{\sqrt{f\left(\frac{r_0}{1-z}\right)h\left(\frac{r_0}{1-z}\right)R\left(\frac{r_0}{1-z}\right)}}.
\end{equation}
The singular part can be isolated by defining
\begin{equation}
	\mathcal{H}(z,r_0)\equiv\mathcal{H}_R(z,r_0)+\mathcal{H}_D(z,r_0),
\end{equation}
where the subscript R(D) refers to the regular (divergent) part, respectively.

%\subsection{Divergent Integral}
To handle the divergent part: 
\begin{equation}
	\mathcal{I}_D(r_0)\equiv\int^1_0\mathcal{H}_D(z,r_0)dz,
\end{equation}
we define $\mathcal{H}(z,r_0)\equiv2r_0/\sqrt{\mathcal{G}(z,r_)}$ and, by expanding it around $z\rightarrow0$, obtain the expression for $\mathcal{G}(z,r_0)$ up to second-order:
\begin{equation}
	\label{g0}
	\mathcal{G}_0(z,r_0)=c_1(r_0)z+c_2(r_0) z^2,
\end{equation}
where
\begin{eqnarray}
	c_1(r_0)&=& C_0\mathcal{D}_0r_0 f(r_0),\nonumber\\
	c_2(r_0)&=&C_0r_0f_0\bigg\{\mathcal{D}_0\left[\mathcal{D}_0\left(\mathcal{D}_0+\frac{f'_0}{f_0^3}\right)r_0-3\right]\nonumber\\
	&&+\frac{\mathcal{D}_0r_0}{2}\bigg\},
\end{eqnarray}
with $X_0\equiv X(r=r_0)$, $C(r)\equiv h(r)r^2$, and
\begin{equation}
	\mathcal{D}(r)\equiv\frac{C''(r)}{C(r)}-\frac{f''(r)}{f(r)}.
\end{equation}
For the ABG model, the values of $c_1$ and $c_2$ are
\begin{widetext}
	\begin{eqnarray}
		c_1(r_0) &=& - \frac{2(r_0^2 (11 q^4 + 22 q^2 r_0^2 - 3r_0^4) \tilde{j}_1^{3/2}+ m(9r_0^8 - 61 q^2 r_0^6)))}{(11 q^2 - 3r^2)^2\tilde{j}_1^{3/2}},\nonumber\\
		c_2(r_0) &=& \frac{m r_0^6\tilde{j}_2 + r_0^2(\tilde{j}_3 \tilde{j}_1^{5/2} - \tilde{j}_3)}{(11q^2 - 3r_0^2)^3 \tilde{j}_1^{5/2}},\nonumber\\
	\end{eqnarray}
\end{widetext}
with
\begin{eqnarray}
	\tilde{j}_1 &\equiv& q^2 + r_0^2 ,\nonumber\\
	\tilde{j}_2 &\equiv& 2013 q^6 + 3104 q^4 r_0^2 + 57 q^2 r_0^4 - 54 r_0^6,\nonumber\\
	\tilde{j}_3 &\equiv& 121q^6 + 825 q^4 r_0^2 - 99 Q^2 r_0^4 + 9 r_0^6.
\end{eqnarray}
Following Bozza~\cite{bozzaGravitationalLensingStrong2002} and Tsukamoto~\cite{Tsukamoto:2016jzh} it can be shown that the divergent integral in the strong field limit $r\rightarrow r_{ps}$ (or equivalently $b\rightarrow b_{ps}$) is expressed as
\begin{eqnarray}
	\label{id}
	\mathcal{I}_D(b)&=&-\sqrt{\frac{2}{f_{ps}C_{ps}''-f''_{ps}C_{ps}}}\nonumber\\
	&&\times\bigg\{\log\left(\frac{b}{b_c}-1\right)+\log\left[r_{ps}^2\left(\frac{C_{ps}''}{C_{ps}}-\frac{f_{ps}''}{f_{ps}}\right)\right]\bigg\}.\nonumber\\
\end{eqnarray}
\begin{figure}
	\centering
	\includegraphics[scale=0.7]{a1a2.pdf}
	\caption{The values of $\tilde{a}_1$ and $\tilde{a}_2$ for Bardeen with ABG source}
	\label{fig:a1a2ABG}
\end{figure}
%\subsection{Regular Integral}

The regular part is
\begin{equation}
	\mathcal{I}_R\equiv\int^1_0\mathcal{H}_R(z,r_0)dz.
\end{equation}
Likewise, by expanding around $r\rightarrow r_{ps}$ the integral can be expressed as
\begin{eqnarray}
	\label{ir}
	\frac{\mathcal{I}_R(r_{ps})}{2r_{ps}}&=&\int^1_0\frac{dz}{F(r_{ps},z)}-\int^1_0\sqrt{\frac{2}{C_{ps}r_{ps}f_r{ps}\mathcal{D}_{ps}z^2}}dz,\nonumber\\
\end{eqnarray}
with
\begin{equation}
	F(r_{ps},z)\equiv\sqrt{C\left(\frac{r_{ps}}{1-z}\right)R\left(\frac{r_{ps}}{1-z}\right)f\left(\frac{r_{ps}}{1-z}\right)\left(1-z\right)^4}.
\end{equation}
For the ABG model, the integral can be evaluated numerically. Putting~\eqref{id} and~\eqref{ir} into~\eqref{ar0}, we can express it as Eq.~\eqref{eq:Defleksi_b} by identifying
\begin{eqnarray}
	\tilde{a}_1&\equiv&\sqrt{\frac{2 A_{ps} B_{ps}}{A_{ps}C''_{ps} - A''_{ps} C_{ps}}},\nonumber\\
	\tilde{a}_2&\equiv&\tilde{a_1} \log \tilde{b} + I_R(r_{ps}) - \pi,
\end{eqnarray}
with
\begin{equation}
	\tilde{b}\equiv r_{ps}^2 \left( \frac{C''_{ps}}{C_{ps}} -\frac{ A''_{ps}}{A_{ps}} \right).
\end{equation}
In terms of the ABG model we consider, their expressions are
\begin{eqnarray}
	\tilde{a}_1&=&\sqrt{\frac{(11q^2 - 3r_{ps}^2)^3 \tilde{k}_1^{5/2}}{\tilde{k}_2 - \tilde{k}_3 \tilde{k}_1^{5/2}}},\nonumber\\
	\tilde{a}_2&=&- \pi+ \mathcal{I}_R(r_{ps})+\sqrt{\frac{(11q^2 - 3r_{ps}^2)^3 \tilde{k}_1^{5/2}}{\tilde{k}_2 - \tilde{k}_3 \tilde{k}_1^{5/2}}}\nonumber\\
	&&\times\log\left[\frac{2 \tilde{k}_1^{5/2} \tilde{k}_3 - 2 \tilde{k}_2}{(11 q^2 - 3 r_{ps}^2)^2 (\tilde{k}_1^{3/2} - 2 m r_{ps}^2) \tilde{k}_1^2}\right],\nonumber\\
\end{eqnarray}
with
\begin{eqnarray}
	\tilde{k}_1 &=& q^2 + r_{ps}^2,\nonumber\\
	\tilde{k}_2 &=& m(3355 q^6 r_{ps}^4 + 466 q^4 r_{ps}^6 + 51 q^2 r_{ps}^8), \nonumber\\
	\tilde{k}_3 &=& (121 q^6 + 825 q^4 r_{ps}^2 - 99 q^2 r_{ps}^4 + 9 r_{ps}^6).
\end{eqnarray}
\begin{figure}
	\centering
	\includegraphics[scale=0.7]{DeflectionAllComparison.pdf}
	\caption{Comparison of light deflection for Regular Bardeen and Bardeen with ABG source}
	\label{fig:LightDeflectionComparison}
\end{figure}

The calculated $\tilde{a}_1$ and $\tilde{a}_2$ are shown in Fig.~\ref{fig:a1a2ABG}. The $\tilde{a}_1$ slowly increases until $q=q_{crit}$ and then decreases linearly. The $\tilde{a}_2$, on the other hand, decreases as $q$ goes up until $q=q_{crit}$, then it starts increasing. The deflection angles are depicted in Fig.~\ref{fig:LightDeflectionComparison}. It is shown that the critical impact parameters for the NLED models are smaller than for the pure Bardeen. This critical value increases with increasing charge. 

%===================================================================================================%
\section{Lensing}
\label{sec:lensing}

The most straightforward effect of light deflection due to gravitational field is the notion of ``gravitational lensing". The lensing mechanism can be inferred from Fig.~\ref{fig:light_deflection}. The straight segment $\overline{SO}$ is the path the light would have taken had it not been deflected due to the lens (BH) at $L$. The angle $\beta$ denotes the angular position of the source $S$ from the observer $O$ if there were no lensing. What the $O$ observes is the ``image" of $S$ located at $I$ whose angular position is given by $\theta$. The deflection angle is given by $\alpha$. From simple geometry the reation between $\beta$ and $\theta$ can be expressed as~\cite{Virbhadra:1999nm, Bozza:2008ev}
\begin{equation}
	\tan \beta = \tan \theta - \frac{D_{LS}}{D_{OL}} \left[ \tan \theta + \tan (\alpha - \theta)\right],
\end{equation}
known as the lens equation.

In the strong field limit ($r_0\rightarrow r_{ps}$) $\beta$ and $\theta$ are small, $\alpha $ can exceed $2\pi$ and light can loop around the black hole several ($n$) times before escaping out to the observer. 
\begin{figure}
	\centering
	\includegraphics[scale=0.6]{Lensing1.pdf}
	\caption{Gravitational lensing diagram. S, L, and O are the source, the lens, and the observer, respectively.}
	\label{fig:light_deflection}
\end{figure}
In this sense, $\alpha = 2n \pi + \Delta \alpha_n$. We can then expand $\tan (\alpha - \theta)\sim\Delta\alpha_n-\theta$~\cite{bozzaStrongFieldLimit2001}. The lens equation thus becomes
\begin{equation}
	\beta = \theta - \frac{D_{LS}}{D_{OL}} \Delta \alpha_n.
	\label{eq:Persamaan_Lensa}
\end{equation}
We also have the relation
\begin{equation}
	b = D_{OL} \theta.
	\label{eq:impactparameter_posisi}
\end{equation}
Substituting it into \eqref{eq:Defleksi_b} and inverting it results in~\cite{bozzaGravitationalLensingStrong2002}
\begin{equation}
	\theta(\alpha) \simeq \frac{b_c}{D_{OL}} \left(1 + e^{\frac{\tilde{a}_2 - \alpha}{\tilde{a}_1}}\right).
	\label{eq:Theta_defleksi}
\end{equation}
%\subsection{Observable}

From Fig.~\ref{fig:light_deflection}, the innermost image is given by
\begin{equation}
	\theta_\infty = \frac{b_c}{D_{OL}}.
	\label{eq:theta_infty}
\end{equation}
Expanding around $\alpha$ yields
\begin{equation}
	\theta_n = \theta_n^0 - \gamma_n \Delta\alpha_n,
	\label{eq:Posisi_bayangan}
\end{equation}
with
\begin{eqnarray}
	\theta_n^0 &=& \frac{b_c}{D_{OL}} \left( 1 + e_n\right),\\
	\gamma_n &=& \frac{b_c}{\tilde{a}_2 D_{OL}}e_n,\\
	e_n &=& e^{\frac{\tilde{a}_2 - 2n \pi}{\tilde{a}_1}}.
\end{eqnarray}
We can eliminate $\Delta\alpha_n$ using the Eq.~\eqref{eq:Persamaan_Lensa}. This results in the equation for the n-th shadow position~\cite{bozzaStrongFieldLimit2001, bozzaGravitationalLensingStrong2002, Tsukamoto:2016qro}
\begin{equation}
	\theta_n = \theta_n^0 + \frac{ D_{LS}}{D_{OS}}\gamma_n(\beta - \theta_n^0),
\end{equation}
where the second term on the right-hand side is small compared to the first one. For the Einstein ring case, $\beta=0$ and
\begin{equation}
	\theta_{nE} = \left(1- \frac{D_{LS}}{D_{OS}} \gamma_n \right) \theta_n^0.
\end{equation}

The observables we wish to calculate are the image separation and the magnification. The separation $s$ is the difference between the outermost and innermost images,
\begin{equation}
	s = \theta_1 -\theta_\infty = \theta_\infty e^{\frac{\tilde{a}_2 - 2 \pi}{\tilde{a}_1}}.
	\label{eq:O_s}
\end{equation}
The magnification $\mu$ is the {\it inverse} of the corresponding Jacobian determinant for the critical curve~\cite{schneider:1992, bozzaStrongFieldLimit2001}. The $n^{th}$ image magnification is defined to be
\begin{equation}
	\mu_n = \frac{1}{|\det J_{\theta_n^0}|} = \frac{1}{\frac{\beta}{\theta_n^0} \left. \frac{\partial \beta}{\partial \theta}\right|_{\theta_n^0}},
\end{equation}
from which the (relativistic) flux ratio is expressed as
\begin{equation}
	r = \frac{\mu_1}{\sum\limits_{n=2}^\infty \mu_n} = e^{\frac{2\pi}{\tilde{a}_1}}.
	\label{eq:O_RasioFluks}
\end{equation}

In this paper we calculate the strong lensing from the Sgr A* black hole modeled as the Bardeen with ABG source. We use data from GRAVITY collaboration where the black hole mass and its distance from the Earth (observer) are $m=4.154\times 10^6 M_\odot$ and $D_{OL} = 8.178\  kpc$, respectively~\cite{abuterGeometricDistanceMeasurement2019}. These values are consistent with the EHT results~\cite{akiyama_first_2022}. In Table~\ref{tabel:data_contohlensa} we show the observables. Here the value the magnification is converted to magnitudes $r_m = 2.5 \log_{10} r$~\cite{bozzaGravitationalLensingStrong2002}. From the Table it can be seen that the observable values for the Bardeen does not differ much from that of RN. However, the observables for the ABG are significantly different from both Bardeen and RN; {\it i.e.,} the ABG's are smaller. This shows that the NLED effect is quite significant here. 
\begin{table}%[H]
	\centering
	\begin{tabular}{|c|c|c|c|c|}
		\hline
		{Model} & $Q/m$ & $\theta_\infty$ ($\mu$ as) & $s$($\mu$ as) & $r_m$
		\\ \hline
		{Schwarzschild} & - & 26.0592 & 0.0327 & 6.8184
		\\ \hline 
		\multicolumn{1}{|c|}{{Bardeen}} &0    & 26.0592 & 0.0327 & 6.8184
		\\ \cline{2-5} 
		\multicolumn{1}{|c|}{} & 0.1 & 26.0156 & 0.0332351 & 6.79937
		\\ \cline{2-5} 
		\multicolumn{1}{|c|}{} & 0.2 & 25.8833 & 0.0349143 & 6.74083
		\\ \cline{2-5} 
		\multicolumn{1}{|c|}{} & 0.3 & 25.6569 & 0.0381989 & 6.63828
		\\ \cline{2-5} 
		\multicolumn{1}{|c|}{} & 0.4 & 25.3266 & 0.044135 & 6.48266
		\\ \cline{2-5} 
		\multicolumn{1}{|c|}{} & 0.5 & 24.8751 & 0.0552564 & 6.25695
		\\ \cline{2-5} 
		\multicolumn{1}{|c|}{} & 0.6 & 24.2718 & 0.0788401 & 5.92697
		\\ \cline{2-5} 
		\multicolumn{1}{|c|}{} & 0.7 & 23.4566 & 0.144124 & 5.41046
		\\ \hline
		\multicolumn{1}{|c|}{{RN}} & 0 & 26.0592 & 0.0327 & 6.8184
		\\ \cline{2-5} 
		\multicolumn{1}{|c|}{} & 0.1 & 26.0157 & 0.0330446 & 6.81081
		\\ \cline{2-5} 
		\multicolumn{1}{|c|}{} & 0.2 & 25.8841 & 0.0340718 & 6.7875
		\\ \cline{2-5} 
		\multicolumn{1}{|c|}{} & 0.3 & 25.6612 & 0.0359467 & 6.74699
		\\ \cline{2-5} 
		\multicolumn{1}{|c|}{} & 0.4 & 25.3412 & 0.0389696 & 6.68646
		\\ \cline{2-5} 
		\multicolumn{1}{|c|}{} & 0.5 & 24.9146 & 0.043696 & 6.60104
		\\ \cline{2-5} 
		\multicolumn{1}{|c|}{} & 0.6 & 24.3668 & 0.0511167 & 6.48246
		\\ \cline{2-5} 
		\multicolumn{1}{|c|}{} & 0.7 & 23.6747 & 0.0628463 & 6.31574
		\\ \hline
		{ABG} & 0 & 15.0453 & 1.01295 & 3.93662
		\\ \cline{2-5} 
		& 0.1 & 15.0593 & 1.015 & 3.93252
		\\ \cline{2-5} 
		& 0.2 & 15.1029 & 1.02079 & 3.92061
		\\ \cline{2-5} 
		& 0.3 & 15.1806 & 1.02911 & 3.90248
		\\ \cline{2-5}
		& 0.4 & 15.3007 & 1.03751 & 3.88228
		\\ \cline{2-5} 
		& 0.5 & 15.4768 & 1.04185 & 3.87016
		\\ \cline{2-5} 
		& 0.6 & 15.7286 & 1.03655 & 3.8886
		\\ \cline{2-5} 
		& 0.7 & 16.0813 & 1.01629 & 3.9788
		\\ \hline
	\end{tabular}
	\caption{Observables for the Sgr A* Schwarzschild, Bardeen, RN, and ABG. The $\theta$ and its corresponding separation are expressed in $\mu$ arc-second (as) unit.}
	\label{tabel:data_contohlensa}
\end{table}

From Table~\ref{tabel:data_contohlensa} it can be inferred that the ABG has $1.5\times$ smaller values of $\theta_{\infty}$ compared to Bardeen. This means that photon can orbit ABG with smaller radius. While the separation $s$ for the ABG is surprisingly $30\times$ larger, its magnification $r_m$ is smaller than for Bardeen. The NLED thus strengthens the gravitational field by decreasing the innermost distance while at the same time increasing its corresponding separation with the outermost image. Interestingly, the observables in the ABG behave in such opposite ways with the the ones in the Bardeen. In Figs.~\ref{fig:thetainf} it is shown that while the $\theta_\infty$ in Bardeen decreases monotonically, in the ABG case it increases.
\begin{figure}
	\centering
	%	\begin{tabular}{cc}
		\includegraphics[scale=0.6]{theta_inf_abg.pdf} \\
		\includegraphics[scale=0.6]{theta_inf_bardeen.pdf}
		%	\end{tabular}
	\caption{$\theta_\infty$ as function of $q$ for: [top] The ABG model and, [down] the original Bardeen.}
	\label{fig:thetainf}
\end{figure}
From Figs.~\ref{fig:swithq} the separation in the Bardeen model increases monotonically, while in the ABG there exists some maximum value $s=s_{max}$ before which it initially increases and after which it starts decreasing.
\begin{figure}
	\centering
	%	\begin{tabular}{cc}
		\includegraphics[scale=0.6]{separation_abg.pdf} \\
		\includegraphics[scale=0.6]{separation_bardeen.pdf}
		%	\end{tabular}
	\caption{$s$ as function of $q$ for: [top] the ABG model, [down] the Bardeen model.}
	\label{fig:swithq}
\end{figure}
Similarly, in Figs.~\ref{fig:rwithq} we see that the $r_m$ decreases unboundedly for the Bardeen as $q$ increases, whereas it decreases to its minimum value before increasing monotonically for the ABG. 
\begin{figure}
	\centering
	%	\begin{tabular}{cc}
		\includegraphics[scale=0.6]{rm_abg.pdf} 
		\includegraphics[scale=0.6]{rm_bardeen.pdf}
		%	\end{tabular}
	\caption{$r_m$ as function of $q$ for: [top] the ABG model, [down] the Bardeen model.}
	\label{fig:rwithq}
\end{figure}
%===============================%

\section{Shadow with Accretion Disk}
\label{sec:shadow}

The radius of the black hole {\it shadow} is defined as the dark region due to the inability of the photon from the background light source to escape the gravitational potential around the black hole. The size of the shadow corresponds to the critical impact parameter of the photon orbit \cite{stuchlik_shadow_2019, bisnovatyi-kogan_gravitational_2017, lu_size_2020}. The idea that 
the Sgr A* shadow can be observed was suggested in~\cite{Falcke:1999pj, Melia:2001dy,  kruglovShadowM87Black2020}.

%For the photon affected by nonlinear electrodynamics caused by magnetic charge we can start from equation (14)
%\begin{equation}
%   \Dot{r}^2 = E^2 - \frac{f(r)L^2}{h(r)r^2}
%\end{equation}
%using the condition $\Dot{r}^2$ = 0, we define $U_{eff}$ as
%\begin{equation}
%\Dot{r}^2 = 
%    U_{eff} = E^2 - \frac{f(r)L^2}{h(r)r^2},
%\end{equation}
%and
%\begin{equation}
%   \begin{split}
	%        0 &= \frac{E^2}{L^2} - \frac{f(r)}{h(r)r^2} \\
	%         &= b^{-2} - \frac{f(r)}{h(r)r^2} \\
	%        b^{-2} &= \frac{f(r)}{h(r)r^2} \\
	%        b &= \sqrt{\frac{h(r)r^2}{f(r)}} \\
	%       b &= r\sqrt{\frac{h(r)}{f(r)}}
	%    \end{split}
%\end{equation}
The shadow radius $R_{sh}$ is represented by the critical impact parameter, {\it i.e.,} the impact parameter evaluated at the photon sphere radius $r_{ps}$
\begin{equation}
	R_{sh} = b_c = r_{ps}\sqrt{\frac{h(r_{ps})}{f(r_{ps})}}.
\end{equation}
For Bardeen with ABG source we will have
\begin{equation}
	R_{shABG} = r_c\sqrt{\frac{\left( 1 - \frac{2(6q^2 - r_c^2)}{(q^2 + r_c^2)} \right)^{-1}}{\left(1 - \frac{2mr_c^2}{(r_c^2 + q^2)^{3/2}}\right)}}.
\end{equation}
% while for the modified Hayward case we will have
% \begin{widetext}
	% \begin{equation}
		%     R_{sh } = r_c\sqrt{\frac{- \frac{- 1 \left(\frac{\beta q^{2}}{2 r^{4}}\right)^{\frac{1}{4}} \sinh{\left(2 \left(\frac{\beta q^{2}}{2 r^{4}}\right)^{\frac{1}{4}} \right)} + 2 \cosh{\left(2 \left(\frac{\beta q^{2}}{2 r^{4}}\right)^{\frac{1}{4}} \right)} + 2}{2 \left(5 \left(\frac{\beta q^{2}}{2 r^{4}}\right)^{\frac{1}{4}} \sinh{\left(2 \left(\frac{\beta q^{2}}{2 r^{4}}\right)^{\frac{1}{4}} \right)} - 2 \left(\frac{\beta q^{2}}{2 r^{4}}\right)^{0.5} \cosh{\left(2 \left(\frac{\beta q^{2}}{2 r^{4}}\right)^{\frac{1}{4}} \right)} + 4 \left(\frac{\beta q^{2}}{2 r^{4}}\right)^{0.5}\right) + \left(\frac{\beta q^{2}}{2 r^{4}}\right)^{\frac{1}{4}} \sinh{\left(2 \left(\frac{\beta q^{2}}{2 r^{4}}\right)^{\frac{1}{4}} \right)} - 2 \cosh{\left(2 \left(\frac{\beta q^{2}}{2 r^{4}}\right)^{\frac{1}{4}} \right)} - 2}}{\left(1 - \frac{2Mr_c^2}{r_c^3 + 2Ml^2}\right)}}
		% \end{equation}
	% \end{widetext}

%The shadow corresponds to the critical impact parameter $b_c$, the impact parameter where the photon will be captured by the black hole or the impact parameter $b$ evaluated at $r_c$.Where $r_c$ is the radius of photon sphere that satisfies $Ueff = 0$ and $Ueff' = 0$.

% The accretion disk model we use is the free falling gas from Cosimo Bambi $[need citation]$.
Due to the effect of NLED source, the emitted photon from the accretion disk will be affected by the effective geometry of the spacetime.  
\
The intensity of the observed emission from the accretion disk can be obtained by integrating the specific emissivity $j(\nu_e)$ along photon path $\gamma$ \cite{bambi_can_2013},
%\begin{equation}
%    ds^2 = -f(r) dt^2 + \frac{1}{g(r)} dr^2 + h(r)r^2 d\Omega^2
%\end{equation}
\begin{equation}
	I_{obs} = \int_\gamma g^3 j(\nu_e) dl_{prop},
	\label{iobs}
\end{equation}
where $dl_{prop}$ is the infinitesimal proper length, and $g$, also known as the redshift factor $g\equiv\nu_{obs}/\nu_e$, can be obtained from
\begin{equation}
	g = \frac{k_\alpha u^\alpha_{obs}}{k_\beta u^\beta_e}.
\end{equation}
We use the model from Bambi \cite{bambi_can_2013} defined as
\begin{equation}
	j(\nu_e) \propto \frac{\delta(\nu_e - \nu_*)}{r^2},
\end{equation}
with $k^\mu=(1,0,0,0)$ the 4-velocity of the photon, $u^\mu_{obs}$ the 4-velocity of the observer, and $u^\mu_{e}$ the 4-velocity of the accreting gas emitting the radiation. For the gas in free fall in a static and spherically symmetric spacetime,
\begin{eqnarray}
	%\begin{split}
	u^{\mu}_{obs} = (1, 0, 0, 0),\nonumber\\
	u^{t}_{e} &=& \frac{1}{f(r)},\nonumber\\
	u^{r}_{e} &=& - \sqrt{\frac{g(r)}{f(r)}[1-f(r)]},\nonumber\\
	u^{\theta}_{e} &=& 0,\nonumber\\
	u^{\phi}_{e} &=& 0.
	%\end{split}.
\end{eqnarray}
%Due to the effect of nonlinear dynamics we have to take the null effective geometry  $ds^2_{eff}$ into account
%\begin{equation}
%    ds^2 = -f(r) dt^2 + \frac{1}{g(r)} dr^2 + h(r)r^2 d\Omega^2
%\end{equation}
The redshift factor $g$, after taking into account the effective metric, is
\begin{eqnarray}
% \begin{split}
	g &=& \frac{k_\alpha u^\alpha_{o}}{k_\beta u^\beta _{e}}\nonumber\\
	% &= \frac{k_t u^t_o}{k_t u^t_e + %k_r u^r_e} \\
	% &= \frac{k_t}{k_t \frac{1}{f(r)} %+ k_r %\left(-\sqrt{\frac{g(r)}{f(r)}- %[1-f(r)]}\right)} \\
	&=& \frac{1}{\frac{1}{f(r)} - \frac{k_r}{k_t}\sqrt{\frac{g(r)}{f(r)}- [1-f(r)]}}.
	%\end{split}
\end{eqnarray}
From $k_\mu k^\mu = 0$ we obtain the relation between $k^\mu$ as
\begin{equation}
	%\begin{split}
	% 0 &= g_{tt}\Dot{t}^2 + g_{rr}\Dot{r}^2 + g_{\theta\theta}\Dot{\theta}^2 + g_{\phi\phi}\Dot{\phi}^2 \\
	\frac{k_r}{k_t} = \pm f(r)\sqrt{g(r) \left(\frac{1}{f(r)} - \frac{b^2}{h(r) r^2}\right)}.
	%\end{split}
\end{equation}

We put all the ingredients together into the Eq.~\eqref{iobs} to obtain the intensity of the accretion disk emission. For this purpose we modify the \textit{EinsteinPy} \cite{einsteinpy} python library to accomodate the metric functions and the null effective geometry. Then, we can compare the result to the known charged solutions: the Reissner-Nordstrom and the original Bardeen; they are shown in Fig.~\ref{fig:my_label}. For the ABG black hole, the shadow radius seems to be smaller compared to the other two.% On the other hand, the Hayward metric modified with Bronnikov Lagrangian black hole appears to be bigger compared to the non NLED counterpart as well as RN.


% \begin{figure}[h]
	%     \centering
	%     \includegraphics[scale = 0.31]{EiBIGM strongdeflection/Reissner-Nordstrom.pdf}
	%     \includegraphics[scale = 0.31]{EiBIGM strongdeflection/Bardeen Black Hole.pdf}
	%     \includegraphics[scale = 0.31]{EiBIGM strongdeflection/Bardeen Black Hole with ABG Source.pdf}
	%     \caption{Comparison of shadow with accretion disk.}
	%     \label{fig:my_label}
	% \end{figure}

%\begin{widetext}
\begin{figure}[h]
	\centering
	\includegraphics[scale = 0.27]{BlackholeShadowsCompareBardeen.pdf}
	\includegraphics[scale = 0.35]{IntensityBlackHoleShadowMerged.pdf}
	\caption{Comparison of shadow of black holes with accretion disk. The lower panels describe the corresponding intensity variation along the $x$-axis.}
	\label{fig:my_label}
\end{figure}
%\end{widetext}

\section{Conclusion}
\label{con}

\section{Conclusion}
\label{sec:conc}

In this work, by assuming that our galaxy's supermassive Sgr A* is a nonsingular magnetically-charged black hole, we study both its strong lensing as well as its shadow image in the thin accretion disk approach. Our approach is different either from~\cite{Eiroa:2010wm, Wei:2015qca} in that we regard the nonsingularity coming from the NLED charge, or from~\cite{ghaffarnejadGravitationalLensingCharged2018} where we used simpler Ayon-Beato-Garcia NLED model. 

The NLED reduces photon sphere radius with increasing $q$ until it reaches $q=q_{crit}$, after which it starts to increase monotonically. The NLED reduces the radius of the photon orbit, and also increases the gravitational field by pulling the innermost distance closer but at the same time stretching its separation distance from the outermost image. Our observable quantities behave differently from the ordinary Bardeen~\cite{Eiroa:2010wm, Wei:2015qca} in terms of the innermost image (Figs.~\ref{fig:thetainf}), the separation between innermost and outermost images (Figs.~\ref{fig:swithq}), and the magnification (Figs.~\ref{fig:rwithq}). Interestingly, the NLED type that we particularly choose to model Bardeen in this paper gives distinct observable results compared to other types. While the model~\cite{ghaffarnejadGravitationalLensingCharged2018} predicts that the angular separation $s$ decreases while the magnification $r_m$ increases with increasing $q$, our results show the opposite. From Table~\ref{tabel:data_contohlensa} we can observe that as the charge increases, the angular separation does increase while the magnification does decrease. Lastly, our analysis with thin accretion disk shows that the nonsingular black hole with ABG metric gives the smallest shadow radius when compared to the RN and ordinary Bardeen.  
%The most surprising results of our work is the observable quantities from the strong lensing. The NLED reduces the radius of the photon orbit, and also increases the gravitational field by pulling the innermost distance closer but at the same time stretching its separation distance from the outermost image.  


%===========================================================
\acknowledgments

We thank Reyhan Lambaga and Imam Huda for the fruitful discussions on the preliminary stage of this work.

\section*{Data Availability Statement}

Data sharing is not applicable to this article as no data sets were generated or analyzed during the current study.

%===========================================================

\begin{thebibliography}{999}
	\bibitem{EventHorizonTelescope:2019dse}Akiyama, K. \& Others First M87 Event Horizon Telescope Results. I. The Shadow of the Supermassive Black Hole. {\em Astrophys. J. Lett.}. \textbf{875} pp. L1 (2019)
	\bibitem{akiyama_first_2022}Akiyama, K. First Sagittarius A* Event Horizon Telescope Results. I. The Shadow of the Supermassive Black Hole in the Center of the Milky Way. {\em The Astrophysical Journal Letters}. pp. 21 (2022)
	\bibitem{Falcke:1999pj}Falcke, H., Melia, F. \& Agol, E. Viewing the shadow of the black hole at the galactic center. {\em Astrophys. J. Lett.}. \textbf{528} pp. L13 (2000)
	\bibitem{schneider:1992}Schneider, P., Ehlers, J. \& Falco, E. Gravitational Lenses. {\em Gravitational Lenses}. (1992)
	\bibitem{Perlick:2004tq}Perlick, V. Gravitational lensing from a spacetime perspective. {\em Living Rev. Rel.}. \textbf{7} pp. 9 (2004)
	\bibitem{Darwin_gravity_1959}Darwin, C. The gravity field of a particle. {\em Proceedings Of The Royal Society Of London. Series A. Mathematical And Physical Sciences}. \textbf{249}
	\bibitem{Bozza:2010xqn}Bozza, V. Gravitational Lensing by Black Holes. {\em Gen. Rel. Grav.}. \textbf{42} pp. 2269-2300 (2010)
	\bibitem{Luminet:1979nyg}Luminet, J. Image of a spherical black hole with thin accretion disk. {\em Astron. Astrophys.}. \textbf{75} pp. 228-235 (1979)
	\bibitem{Chandrasekhar:1985kt}Chandrasekhar, S. The mathematical theory of black holes.  (1985)
	\bibitem{Frittelli:1999yf}Frittelli, S., Kling, T. \& Newman, E. Space-time perspective of Schwarzschild lensing. {\em Phys. Rev. D}. \textbf{61} pp. 064021 (2000)
	\bibitem{Virbhadra:1999nm}Virbhadra, K. \& Ellis, G. Schwarzschild black hole lensing. {\em Phys. Rev. D}. \textbf{62} pp. 084003 (2000)
	\bibitem{bozzaStrongFieldLimit2001}Bozza, V., Capozziello, S., Iovane, G. \& Scarpetta, G. Strong Field Limit of Black Hole Gravitational Lensing. {\em General Relativity And Gravitation}. \textbf{33}, 1535-1548 (2001,9)
	\bibitem{Tsukamoto:2016qro}Tsukamoto, N. Strong deflection limit analysis and gravitational lensing of an Ellis wormhole. {\em Phys. Rev. D}. \textbf{94}, 124001 (2016)
	\bibitem{Tsukamoto:2016jzh}Tsukamoto, N. Deflection angle in the strong deflection limit in a general asymptotically flat, static, spherically symmetric spacetime. {\em Phys. Rev. D}. \textbf{95}, 064035 (2017)
	\bibitem{bozzaGravitationalLensingStrong2002}Bozza, V. Gravitational Lensing in the Strong Field Limit. {\em Phys. Rev. D}. \textbf{66}, 103001 (2002,11)
	\bibitem{Eiroa:2002mk}Eiroa, E., Romero, G. \& Torres, D. Reissner-Nordstrom black hole lensing. {\em Phys. Rev. D}. \textbf{66} pp. 024010 (2002)
	\bibitem{Bozza:2002af}Bozza, V. Quasiequatorial gravitational lensing by spinning black holes in the strong field limit. {\em Phys. Rev. D}. \textbf{67} pp. 103006 (2003)
	\bibitem{Vazquez:2003zm}Vazquez, S. \& Esteban, E. Strong field gravitational lensing by a Kerr black hole. {\em Nuovo Cim. B}. \textbf{119} pp. 489-519 (2004)
	\bibitem{Bozza:2006nm}Bozza, V., De Luca, F. \& Scarpetta, G. Kerr black hole lensing for generic observers in the strong deflection limit. {\em Phys. Rev. D}. \textbf{74} pp. 063001 (2006)
	\bibitem{Oppenheimer:1939ue}Oppenheimer, J. \& Snyder, H. On Continued gravitational contraction. {\em Phys. Rev.}. \textbf{56} pp. 455-459 (1939)
	\bibitem{Penrose:1964wq}Penrose, R. Gravitational collapse and space-time singularities. {\em Phys. Rev. Lett.}. \textbf{14} pp. 57-59 (1965)
	\bibitem{bardeenNonsingularGeneralrelativisticGravitational1968}Bardeen, J. Non-Singular General-Relativistic Gravitational Collapse. {\em Proc. Int. Conf. GR5, Tbilisi}. \textbf{174} (1968)
	\bibitem{Ansoldi:2008jw}Ansoldi, S. Spherical black holes with regular center: A Review of existing models including a recent realization with Gaussian sources. {\em Conference On Black Holes And Naked Singularities}. (2008,2)
	\bibitem{Eiroa:2010wm}Eiroa, E. \& Sendra, C. Gravitational lensing by a regular black hole. {\em Class. Quant. Grav.}. \textbf{28} pp. 085008 (2011)
	\bibitem{Wei:2015qca}Wei, S., Liu, Y. \& Fu, C. Null Geodesics and Gravitational Lensing in a Nonsingular Spacetime. {\em Adv. High Energy Phys.}. \textbf{2015} pp. 454217 (2015)
	\bibitem{Ayon-Beato:1998hmi}Ayon-Beato, E. \& Garcia, A. Regular black hole in general relativity coupled to nonlinear electrodynamics. {\em Phys. Rev. Lett.}. \textbf{80} pp. 5056-5059 (1998)
	\bibitem{ayon-beatoBardeenModelNonlinear2000}Ayón-Beato, E. \& García, A. The Bardeen Model as a Nonlinear Magnetic Monopole. {\em Physics Letters B}. \textbf{493}, 149-152 (2000,11)
	\bibitem{Novello:1999pg}Novello, M., De Lorenci, V., Salim, J. \& Klippert, R. Geometrical aspects of light propagation in nonlinear electrodynamics. {\em Phys. Rev. D}. \textbf{61} pp. 045001 (2000)
	\bibitem{ghaffarnejadGravitationalLensingCharged2018}Ghaffarnejad, H., Amirmojahedi, M. \& Niad, H. Gravitational Lensing of Charged Ayon-Beato-Garcia Black Holes and Nonlinear Effects of Maxwell Fields. {\em Advances In High Energy Physics}. \textbf{2018} pp. 1-18 (2018)
	\bibitem{Weinberg:1972kfs}Weinberg, S. Gravitation and Cosmology: Principles and Applications of the General Theory of Relativity. (John Wiley,1972)
	\bibitem{Bozza:2008ev}Bozza, V. A Comparison of approximate gravitational lens equations and a proposal for an improved new one. {\em Phys. Rev. D}. \textbf{78} pp. 103005 (2008)
	\bibitem{bambi_can_2013}Bambi, C. Can the supermassive objects at the centers of galaxies be traversable wormholes? The first test of strong gravity for mm/sub-mm VLBI facilities. {\em Physical Review D}. \textbf{87}, 107501 (2013,5), http://arxiv.org/abs/1304.5691, arXiv: 1304.5691
	\bibitem{abuterGeometricDistanceMeasurement2019}Abuter, R., Amorim, A., Bauboeck, M., Berger, J., Bonnet, H., Brandner, W., Clenet, Y., Foresto, V., De Zeeuw, P., Dexter, J., Duvert, G., Eckart, A., Eisenhauer, F., Schreiber, N., Garcia, P., Gao, F., Gendron, E., Genzel, R., Gerhard, O., Gillessen, S., Habibi, M., Haubois, X., Henning, T., Hippler, S., Horrobin, M., Jimenez-Rosales, A., Jocou, L., Kervella, P., Lacour, S., Lapeyrere, V., Bouquin, J., Lena, P., Ott, T., Paumard, T., Perraut, K., Perrin, G., Pfuhl, O., Rabien, S., Coira, G., Rousset, G., Scheithauer, S., Sternberg, A., Straub, O., Straubmeier, C., Sturm, E., Tacconi, L., Vincent, F., Von Fellenberg, S., Waisberg, I., Widmann, F., Wieprecht, E., Wiezorrek, E., Woillez, J. \& Yazici, S. A Geometric Distance Measurement to the Galactic Center Black Hole with 0.3% Uncertainty. {\em A&A}. \textbf{625} pp. L10 (2019,5)
	\bibitem{stuchlik_shadow_2019}Stuchlík, Z. \& Schee, J. Shadow of the regular Bardeen black holes and comparison of the motion of photons and neutrinos. {\em The European Physical Journal C}. \textbf{79}, 44 (2019,1), http://link.springer.com/10.1140/epjc/s10052-019-6543-8
	\bibitem{bisnovatyi-kogan_gravitational_2017}Bisnovatyi-Kogan, G. \& Tsupko, O. Gravitational Lensing in presence of Plasma: Strong Lens Systems, Black Hole Lensing and Shadow. {\em Universe}. \textbf{3}, 57 (2017,7), http://arxiv.org/abs/1905.06615, arXiv: 1905.06615
	\bibitem{lu_size_2020}Lu, H. \& Lyu, H. On the Size of a Black Hole: The Schwarzschild is the Biggest. {\em Physical Review D}. \textbf{101}, 044059 (2020,2), http://arxiv.org/abs/1911.02019, arXiv:1911.02019 [gr-qc, physics:hep-th]
	\bibitem{Melia:2001dy}Melia, F. \& Falcke, H. The supermassive black hole at the galactic center. {\em Ann. Rev. Astron. Astrophys.}. \textbf{39} pp. 309-352 (2001)
	\bibitem{einsteinpy}EinsteinPy Development Team EinsteinPy: Python library for General Relativity.  (2021), https:/einsteinpy.org/
	\bibitem{bozzaGravitationalLensingStrong2002}Bozza, V. Gravitational Lensing in the Strong Field Limit. {\em Phys. Rev. D}. \textbf{66}, 103001 (2002,11)
	\bibitem{kruglovShadowM87Black2020}Kruglov, S. The Shadow of M87* Black Hole within Rational Nonlinear Electrodynamics. {\em Mod. Phys. Lett. A}. \textbf{35}, 2050291 (2020,11)
	
\end{thebibliography}


\end{document}
