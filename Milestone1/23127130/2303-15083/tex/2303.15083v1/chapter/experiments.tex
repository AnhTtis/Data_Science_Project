\section{Experiments}
\subsection{Experimental Setup}
\subsubsection{Dataset}
\quad We choose the popular large-scale dataset nuScenes for evaluation in our experiments. This dataset consists of 1,000 driving sequences (700/150/150 for train/val/test) with images from 6 cameras, points from 5 Radars and 1 LiDAR. Each sequence is 20 seconds long and every 10 frames are fully annotated with 3D object bounding boxes. For each frame, there are 6 RGB camera images and 20 LiDAR scanning frames. Following BEVFusion~\cite{liu2022bevfusion}, we set the area inside [-54.0\text{m}, 54.0\text{m}]$\times$[-54.0\text{m}, 54.0\text{m}]$\times$[-5\text{m},  3\text{m}] as our region of interest.
\subsubsection{Evaluation Metrics}
\quad We use the official evaluation metrics of nuScenes, which are the mean Average Precision(mAP), mean Average Translation Error(mATE), mean Average Scale Error(mASE), mean Average Orientation Error(mAOE), mean Average Velocity Error(mAVE), mean Average Attribute Error(mAAE) and the nuScenes detection score(NDS). The mAP measures the recall and precision of predicted bounding boxes. NDS is the composition of other metrics to comprehensively judge the detection capacity. The remaining metrics calculate the positive results of prediction on the corresponding aspects. The data unit of results is “\%”.

\subsubsection{Models and Training}
\quad We adopt BEVDet~\cite{huang2021bevdet} with ResNet-50 as our camera based detector and CenterPoint\cite{yin2021center} as our LiDAR based detector. The LiDAR-camera detector is constructed by fusing the low-level BEV features of BEVDet and CenterPoint, following the method proposed in BEVFusion~\cite{liu2022bevfusion}. The size of voxel to produce BEV features is 0.075\text{m}$\times$0.075\text{m}$\times$0.2\text{m}.

The detectors are trained by an AdamW optimizer with the learning rate to be 1e-4 for LiDAR and LiDAR-camera based detectors and 2e-4 for the camera based detector. The training batch size is 20 and the training epoch is 20 for all detectors. More details are supplemented in the appendix.

UniDistill is evaluated to see whether it can transfer knowledge in four distillation paths: (1) From the LiDAR-camera based teacher detector to the LiDAR based student. (2) From the LiDAR-camera based teacher detector to the camera based student. (3) From the camera based teacher detector to the LiDAR based student. (4) From the LiDAR based teacher detector to the camera based student. The hyperparameters $\lambda_1$,$\lambda_2$ and $\lambda_3$ in each path are: (1) 10, 1, 10. (2) 10, 5, 10. (3) 10, 5, 1. (4) 100, 40, 10. The adaptive layers for feature distillation and relation distillation are introduced when evaluating in path (3).

%We note that since the originial performance of the camera based detector falls behind the LiDAR based one, using the vanilla feature distillation and relation distillation to align the BEV features of the LiDAR based student with the camera based teacher degrades the final performance. Therefore, in this situation, we firsly introduce two additional one-layer convolution networks, $\text{Conv}_1$ and $\text{Conv}_2$, and apply them to $\textit{F}_{\text{ldr}}^{\text{low}}$ and $\textit{F}_{\text{ldr}}^{\text{high}}$ respectively:
%\begin{equation*}
%\textit{F}_{\text{ldr}}^{\text{low'}}=\text{Conv}_1(\textit{F}_{\text{ldr}}^{\text{low'}}), \textit{F}_{\text{ldr}}^{\text{high'}}=\text{Conv}_2(\textit{F}_{\text{ldr}}^{\text{low'}}).
%\end{equation*}
%Then the feature distillation will be calculated with $\textit{F}_{\text{ldr}}^{\text{low'}}$ and $\textit{F}_{\text{cam}}^{\text{low}}$ and the relation distillation will be calculated with $\textit{F}_{\text{ldr}}^{\text{high'}}$ and $\textit{F}_{\text{cam}}^{\text{high}}$.

\subsection{Comparison with the State-of-the-Arts}
We first evaluate the performance of UniDistill on the test dataset of nuScenes and Table \ref{tab:1} reports the results. It is revealed that in all of the four distillation paths, UniDistill helps transfer knowledge from the teacher detector to student and improve its performance. Besides, with the LiDAR-camera based detector as the teacher, the LiDAR based student obtains better performance than other state-of-the-art LiDAR based detectors, proving the effectiveness of UniDistill. We also compare UniDistill with S2M2-SSD~\cite{zheng2022boosting}, which performs cross-modality knowledge distillation from a PointPainting~\cite{vora2020pointpainting} teacher detector also to a CenterPoint student detector. The result shows that UniDistill helps the student obtain better performance.
\begin{table*}[ht]\small
\centering
\caption{Ablation study of three proposed distillation losses on the nuScenes validation dataset. “T” and “S” represent the teacher detector and student detector respectively and the mAP/NDS of the student is reported.}
\begin{tabular}{c|ccc|cccc}
\shline
\multirow{2}{*}{Setting} & \multicolumn{3}{c|}{Loss} & \multicolumn{4}{c}{Modality}                                                                                                                                                                                                                                                                                                  \\ \cline{2-8} 
                     & $\mathcal{L}_{\text{Fea}}$       & $\mathcal{L}_{\text{Rel}}$      & $\mathcal{L}_{\text{Resp}}$      & \multicolumn{1}{c|}{\begin{tabular}[c]{@{}c@{}}T:LiDAR+Camera\\ S:LiDAR\end{tabular}} & \multicolumn{1}{c|}{\begin{tabular}[c]{@{}c@{}}T:LiDAR+Camera\\ S:Camera\end{tabular}} & \multicolumn{1}{c|}{\begin{tabular}[c]{@{}c@{}}T:Camera\\ S:LiDAR\end{tabular}} & \begin{tabular}[c]{@{}c@{}}T:LiDAR\\ S:Camera\end{tabular} \\ \shline
1                    &         &        &        & \multicolumn{1}{c|}{53.5/63.9}                                                        & \multicolumn{1}{c|}{20.3/33.1}                                                         & \multicolumn{1}{c|}{53.5/63.9}                                                  & 20.3/33.1                                                  \\
2                    & \checkmark       &        &        & \multicolumn{1}{c|}{56.1/65.5}                                                        & \multicolumn{1}{c|}{21.6/34.5}                                                         & \multicolumn{1}{c|}{54.3/64.6}                                                  & 21.1/34.3                                                  \\
3                    &         & \checkmark      &        & \multicolumn{1}{c|}{54.1/64.0}                                                        & \multicolumn{1}{c|}{22.3/35.7}                                                         & \multicolumn{1}{c|}{55.2/65.3}                                                  & 21.7/35.0                                                  \\
4                    &         &        & \checkmark      & \multicolumn{1}{c|}{58.7/66.7}                                                        & \multicolumn{1}{c|}{25.7/37.1}                                                         & \multicolumn{1}{c|}{55.7/65.6}                                                  & 24.9/36.3                                                  \\
5                    & \checkmark       & \checkmark      &        & \multicolumn{1}{c|}{-}                                                                & \multicolumn{1}{c|}{-}                                                                 & \multicolumn{1}{c|}{-}                                                          & 23.5/35.4                                                  \\
6                    &         & \checkmark      & \checkmark      & \multicolumn{1}{c|}{-}                                                                & \multicolumn{1}{c|}{-}                                                                 & \multicolumn{1}{c|}{-}                                                          & 25.3/36.7                                                  \\
7                    & \checkmark       &        & \checkmark      & \multicolumn{1}{c|}{-}                                                                & \multicolumn{1}{c|}{-}                                                                 & \multicolumn{1}{c|}{-}                                                          & 25.3/37.0                                                  \\
8                    & \checkmark       & \checkmark      & \checkmark      & \multicolumn{1}{c|}{\textbf{59.7/67.5}}                                               & \multicolumn{1}{c|}{\textbf{26.5/37.8}}                                                & \multicolumn{1}{c|}{\textbf{57.0/66.3}}                                         & \textbf{26.0/37.3}                                         \\ \shline
\end{tabular}
\label{tab:2}
\end{table*}
%\begin{table*}[ht]\small
%\centering
%\caption{Ablation study of three proposed distillation losses on the nuScenes validation dataset. “T” and “S” represent the teacher detector and student detector respectively and the mAP/NDS of the student is reported.}
%\begin{tabular}{ccc|cccc}
%\shline
%\multicolumn{3}{c|}{Loss} & \multicolumn{4}{c}{Modality}                                                                                                                                                                                                                                                                                                 \\ \shline
%$\mathcal{L}_{\text{Fea}}$       & $\mathcal{L}_{\text{Rel}}$      & $\mathcal{L}_{\text{Resp}}$      & \multicolumn{1}{c|}{\begin{tabular}[c]{@{}c@{}}T:LiDAR+Camera\\S:LiDAR\end{tabular}} & \multicolumn{1}{c|}{\begin{tabular}[c]{@{}c@{}} T:LiDAR+Camera\\S:Camera\end{tabular}} & \multicolumn{1}{c|}{\begin{tabular}[c]{@{}c@{}} T:Camera\\S:LiDAR\end{tabular}} & \begin{tabular}[c]{@{}c@{}}T:LiDAR\\S:Camera\end{tabular} \\ \shline
%        &        &        & \multicolumn{1}{c|}{53.5/63.9}                                                        & \multicolumn{1}{c|}{20.3/33.1}                                                         & \multicolumn{1}{c|}{53.5/63.9}                                                  & 20.3/33.1                                                  \\
%\checkmark       &        &        & \multicolumn{1}{c|}{56.1/65.5}                                                        & \multicolumn{1}{c|}{21.6/34.5}                                                         & \multicolumn{1}{c|}{54.3/64.6}                                                  & 21.1/34.3                                                  \\
%        & \checkmark      &        & \multicolumn{1}{c|}{54.1/64.0}                                                        & \multicolumn{1}{c|}{22.3/35.7}                                                         & \multicolumn{1}{c|}{55.2/65.3}                                                  & 21.7/35.0                                                  \\
%        &        & \checkmark      & \multicolumn{1}{c|}{58.7/66.7}                                                        & \multicolumn{1}{c|}{25.7/37.1}                                                         & \multicolumn{1}{c|}{55.7/65.6}                                                  & 24.9/36.3                                                  \\
%\checkmark       & \checkmark      &        & \multicolumn{1}{c|}{-}                                                                & \multicolumn{1}{c|}{-}                                                                 & \multicolumn{1}{c|}{-}                                                          & 23.5/35.4                                                           \\
%        & \checkmark      & \checkmark      & \multicolumn{1}{c|}{-}                                                                & \multicolumn{1}{c|}{-}                                                                 & \multicolumn{1}{c|}{-}                                                          & 25.3/36.7                                                           \\
%\checkmark       &        & \checkmark      & \multicolumn{1}{c|}{-}                                                                & \multicolumn{1}{c|}{-}                                                                 & \multicolumn{1}{c|}{-}                                                          & 25.3/37.0                                                       \\
%\checkmark       & \checkmark      & \checkmark      & \multicolumn{1}{c|}{\textbf{59.7/67.5}}                                                        & \multicolumn{1}{c|}{\textbf{26.5/37.8}}                                                         & \multicolumn{1}{c|}{\textbf{57.0/66.3}}                                                  &\textbf{26.0/37.3}                                                            \\ \shline
%\end{tabular}
%\label{tab:2}
%\end{table*}

\subsection{Ablation Studies}
In this section, some experiments are conducted on the validation dataset to show the effect of each distillation loss and the rationality of specific designs. For efficiency, we turn off the auto-scaling between classification and regression loss in $\mathcal{L}_{\text{Det}}$. We report the overall results in Table \ref{tab:2} to show the effect of each loss.
\subsubsection{Effect of Feature Distillation}
\quad As in the second setting of Table \ref{tab:2}, using the feature distillation improves the NDS and mAP of the student in four paths. Moreover, the fifth, seventh and eighth settings prove that it is complementary to other distillation losses. 

Another experiment in path (4) is conducted to show the rationality of feature distillation to align the features of 9 crucial points. We compare the original feature distillation with two modified ones which align the low-level BEV features (1) completely or (2) inside a Gaussian-like mask like the response distillation. The results in Table \ref{tab:3} show that feature distillation performs better when selecting 9 crucial points for alignment. Moreover, in this situation, the AP of small objects (pedestrians and motors) improves a lot while the influence on large objects(cars and trucks) is minor.
\begin{table}[t]\small
\centering
\caption{Ablation study in path (4) to show that feature distillation performs better when selecting crucial points for alignment.}
\begin{tabular}{c|ccccc|c}
\shline
\multirow{2}{*}{Method} & \multicolumn{5}{c|}{AP}            & \multirow{2}{*}{NDS} \\ \cline{2-6}
                        & car  & truck & ped  & motor & mean &                      \\ \shline
Baseline                & 38.5 & 20.1  & 9.4  & 18.5  & 20.3 & 33.1                 \\
Complete                    & 38.0 & 13.1  & 14.2 & 22.0  & 20.3 & 32.6                 \\
Gaussian                & \textbf{45.3} & \textbf{21.4}  & 10.3 & 16.8  & 20.6 & 32.8                 \\ \shline
Crucial                 & 44.0 & 14.9  & \textbf{21.9} & \textbf{22.4}  & \textbf{21.1} & \textbf{34.3}                 \\ \shline
\end{tabular}
\label{tab:3}
\end{table}

\begin{table}[t]\small
\centering
\caption{Ablation study in path (4) to show that feature distillation performs better when aligning the low-level BEV features.}
\begin{tabular}{c|ccccc}
\shline
Method     & mAP  & mASE & mAOE & mAAE & NDS  \\ \shline
Baseline   & 20.3 & 27.9 & 46.6 & 21.9 & 33.1 \\
High-Level & 20.6 & 28.1 & 46.9 & 23.2 & 32.3 \\ \shline
Low-Level  & \textbf{21.1} & \textbf{27.8} & \textbf{46.3} & \textbf{21.9} & \textbf{34.3} \\ \shline
\end{tabular}
\label{tab:4}
\end{table}

We further compare the original feature distillation with a modified one which aligns the high-level BEV features of 9 crucial points in path (4). The results illustrated in Table \ref{tab:4} reveal that calculating feature distillation with the low-level BEV features obtains better performance.

\subsubsection{Effect of Relation Distillation}
\quad The comparison between the third setting and first setting of Table \ref{tab:2} shows that using the relation distillation improves the NDS and mAP of the student in all paths. And the results in the fifth, sixth and eighth settings further prove that it is complementary to other distillation losses. 

We conduct another experiment in path (4) to show the rationality of selecting 9 crucial points for relation distillation calculation. We compare the original relation distillation with two modified ones which align the relationship between (1) all of the high-level BEV features or (2) the features inside a Gaussian-like mask like the response distillation. The results illustrated in Table \ref{tab:5} show that the relation distillation obtains the best performance when calculating the relationship between 9 crucial points for alignment.

We further compare the original relation distillation with the modified one, which aligns the relationship between the low-level BEV features of 9 crucial points. The results in path (4) are illustrated in Table \ref{tab:6} and reveal that calculating relation distillation with the high-level BEV features obtains better performance.
\begin{figure*}[t]
\centering
\setlength{\belowcaptionskip}{-5pt}
\includegraphics[width=0.85\linewidth]{image/bevdistill_visual-eps-converted-to.pdf}
\caption{Illustration of detection results. The boxes in red and green are the predicted and ground truth bounding boxes respectively. (a), (b) and (c) show the results of the LiDAR-camera based teacher detector, the LiDAR based student detector without UniDistill and the LiDAR based student detector with UniDistill respectively. The results show that UniDistill helps the student detector generate more accurate predictions and fewer false positive results.}
\label{img:3}
\end{figure*} 
\subsubsection{Effect of Response Distillation}
\quad As the fourth setting of Table \ref{tab:2} shows, using the response distillation improves the NDS and mAP of the student in all paths. Furthermore, the sixth, seventh and eighth settings prove that it is complementary to other distillation losses. 

Another experiment in path (4) is conducted to prove the rationality of aligning the response features inside the Gaussian-like mask for response distillation. The original response distillation is compared with two modified ones which align the response features (1) completely or (2) of 9 crucial points like those in feature distillation. The results illustrated in Table \ref{tab:7} show that the response distillation obtains better performance when selecting the response features inside the Gaussian-like mask for alignment.

To form the response features, we first gather the max value of each position in the classification heatmap to obtain a new heatmap. In path (4), we further compare the original response distillation with a modified one based on the response features formed by concatenating the original classification and regression heatmaps. The results in Table \ref{tab:8} show that gathering the max value to form the response features helps response distillation perform better.
\begin{table}[t]\small
\centering
\caption{Ablation study in path (4) to show that relation distillation performs better when selecting crucial points for alignment.}
\begin{tabular}{c|ccccc|c}
\shline
\multirow{2}{*}{Method} & \multicolumn{5}{c|}{AP}            & \multirow{2}{*}{NDS} \\ \cline{2-6}
                        & car  & truck & ped  & motor & mean &                      \\ \shline
Baseline                & 38.5 & \textbf{20.1}  & 9.4  & 18.5  & 20.3 & 33.1                 \\
Complete                     & 44.0 & 18.3  & 15.9 & 17.9  & 20.4 & 33.4                 \\
Gaussian                & \textbf{44.9} & 18.3  & 15.3 & 18.1  & 20.8 & 34.0                 \\ \shline
Crucial                 & 44.1 & 18.3  & \textbf{19.2} & \textbf{18.8}  & \textbf{21.7} & \textbf{35.0}                 \\ \shline
\end{tabular}
\label{tab:5}
\end{table}
\begin{table}[t]\small
\centering
\caption{Ablation study in path (4) to show that relation distillation performs better when aligning the high-level BEV features.}
\begin{tabular}{c|ccccc}
\shline
Method     & mAP  & mASE & mAOE & mAAE & NDS  \\ \shline
Baseline   & 20.3 & 27.9 & 46.6 & 21.9 & 33.1 \\
Low-Level  & 19.7 & \textbf{27.7} & 54.4 & 24.3 & 32.3 \\ \shline
High-Level & \textbf{21.7} & 28.1 & \textbf{46.2} & \textbf{21.2} & \textbf{35.0} \\ \shline
\end{tabular}
\label{tab:6}
\end{table}
\subsubsection{Effect of Adaptive Layers}
\begin{table}[t]\small
\centering
\caption{Ablation study in path (4) to show that response distillation performs better when aligning the features in Gaussian mask.}
\begin{tabular}{c|ccccc|c}
\shline
\multirow{2}{*}{Method} & \multicolumn{5}{c|}{AP}            & \multirow{2}{*}{NDS} \\ \cline{2-6}
                        & car  & truck & ped  & motor & mean &                      \\ \shline
Baseline                & 38.5 & 20.1  & 9.4  & 18.5  & 20.3 & 33.1                 \\
Complete                     & 45.3 & 18.8  & 13.6 & 22.1  & 23.4 & 34.9                 \\
Crucial                 & 46.7 & 17.4  & 14.6 & 22.5  & 23.4 & 35.3                 \\ \shline
Gaussian                & \textbf{47.8} & \textbf{21.1}  & \textbf{26.5} & \textbf{23.4}  & \textbf{24.9} & \textbf{36.3}                 \\ \shline
\end{tabular}
\label{tab:7}
\end{table}
\begin{table}[t]\small
\centering
\caption{Ablation study in path (4) to show that gathering the max value of each position in the classification heatmap helps response distillation perform better.}
\begin{tabular}{c|ccccc}
\shline
Method   & mAP  & mASE & mAOE & mAAE & NDS  \\ \shline
Baseline & 20.3 & 27.9 & 46.6 & \textbf{21.9} & 33.1 \\
W/O Max   & 24.3 & 27.5 & 46.7 & 22.5 & 35.4 \\ \shline
Max      & \textbf{24.9} & \textbf{27.1} & \textbf{42.6} & 24.0 & \textbf{36.3} \\ \shline
\end{tabular}
\label{tab:8}
\end{table}
\begin{table}[t]\small
%\setlength{\belowcaptionskip}{-20pt}
\centering
\caption{Ablation study in path (3) to show that the adaptive layers help feature distillation and relation distillation perform better.}
\resizebox{0.92\width}{!}{
\begin{tabular}{c|ccc|ccc}
\shline
\multirow{2}{*}{Method} & \multicolumn{3}{c|}{$\mathcal{L}_{\text{Fea}}$} & \multicolumn{3}{c}{$\mathcal{L}_{\text{Rel}}$} \\ \cline{2-7} 
                        & mAP    & mAVE  & NDS   & mAP   & mAVE  & NDS   \\ \shline
Baseline                & 53.5   & 22.5  & 63.9  & 53.5  & 22.5  & 63.9  \\
W/O Adapt                & 53.3   & 25.3  & 63.5  & 53.1  & 24.3  & 63.2  \\ \shline
With Adapt              & \textbf{54.3}   & \textbf{21.3}  & \textbf{64.6}  & \textbf{55.2}  & \textbf{21.4}  & \textbf{65.3}  \\ \shline
\end{tabular}
}
\label{tab:9}
\end{table}
\quad When evaluating in path (3), we introduce two adaptive layers $\text{Adapt}_1$ and $\text{Adapt}_2$ after the low- and high-level BEV features to avoid performance degradation. Here we conduct experiments to show the indispensability of these adaptive layers by removing them and re-evaluating the performance. The results in Table \ref{tab:9} show that without the adaptive layers, the feature distillation and relation distillation even worsen the performance of the student detector. Therefore, in this situation, we keep the two adaptive layers for help.

We further compare the detection loss $\mathcal{L}_{\text{Det}}$ with/without the adaptive layers and the baseline is the student without UniDistill. The results in Figure \ref{img:4} show that, with the adaptive layers, the detection loss gradually becomes lower than the baseline. However, without the adaptive layers, the detection loss is always larger than the baseline, leading to worse performance. We think the problem results from that the feature quality of camera based teacher is worse than the LiDAR based student. Without the adaptive layers, aligning the features of student with teacher can decrease their quality. However, with the adaptive layers, the student can decide whether to learn from the teacher so that the performance is improved.
\begin{figure}[t]
\centering
\setlength{\belowcaptionskip}{-10pt}
\includegraphics[width=0.95\linewidth]{image/bevdistill_conv-eps-converted-to.pdf}
\caption{Illustration to show that the adaptive layers help feature distillation and relation distillation decrease the detection loss.}
\label{img:4}
\end{figure} 
\subsection{Visualization}
In this section, we visualize the 3D object detection results to qualitatively show the effectiveness of UniDistill. The results are illustrated in Figure \ref{img:3}, where the teacher and the student are LiDAR-camera based and LiDAR based. From the results, the student detector can localize objects better with the help of UniDistill. Moreover, due to the balance between objects of different sizes, there are fewer false positive predictions on small objects.