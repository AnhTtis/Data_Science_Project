\usepackage{amsmath}
\usepackage{graphicx}
\usepackage{mathtools}
\usepackage{esvect}
\usepackage{bbm}
\usepackage{bm}
\usepackage{lipsum}
\usepackage{xspace}
\usepackage[capitalize,noabbrev]{cleveref}
\usepackage{todonotes}
% \usepackage{tabularray}
% \SetTblrInner{rowsep=2pt}
\usepackage{soul}

\usepackage{subcaption}

% make writing commands easier
\usepackage{xparse}

\ExplSyntaxOn

\makeatletter
\NewDocumentCommand{\multicitep}{m}
 {
  \NAT@open
  \mjb_multicitep:n { #1 }
  \NAT@close
 }
\makeatother

\seq_new:N \l_mjb_multicite_in_seq
\seq_new:N \l_mjb_multicite_out_seq
\seq_new:N \l_mjb_cite_seq

\cs_new_protected:Npn \mjb_multicitep:n #1
 {
  \seq_set_split:Nnn \l_mjb_multicite_in_seq { ; } { #1 }
  \seq_clear:N \l_mjb_multicite_out_seq
  \seq_map_inline:Nn \l_mjb_multicite_in_seq
  {
    \mjb_cite_process:n { ##1 }
  }
  \seq_use:Nn \l_mjb_multicite_out_seq { ;~ }
 }

\cs_new_protected:Npn \mjb_cite_process:n #1
 {
  \seq_set_split:Nnn \l_mjb_cite_seq { , } { #1 }
  \int_compare:nTF { \seq_count:N \l_mjb_cite_seq == 1 }
  {
    \seq_put_right:Nn \l_mjb_multicite_out_seq
     { \citep{#1} }
  }
  {
    \seq_put_right:Nx \l_mjb_multicite_out_seq
     {
      \exp_not:N \citenum{\seq_item:Nn \l_mjb_cite_seq { 1 }},~
      \seq_item:Nn \l_mjb_cite_seq { 2 }
     }
  }
 }
\ExplSyntaxOff

% colored text
\usepackage{color}

% include eps, pdf graphics
\usepackage{graphicx}

% use "H" for floats
\usepackage{float}

% avoid "too many unprocessed floats" error
\usepackage{morefloats}

% FloatBarrier to ensure figures do not jump to different sections
\usepackage{placeins}

% properly handle spaces after defines
\usepackage{xspace}

% in case we need to span rows in our tables
\usepackage{multirow}

% tables across multiple pages
\usepackage{ltablex}

% nice-looking tables
\usepackage{booktabs}

% easy centering for tables
\newcolumntype{Y}{>{\centering\arraybackslash}X}

% math
\usepackage{algorithm}
\usepackage{algorithmicx}
\usepackage{algpseudocode}
% \usepackage{amsmath,amsthm,amssymb}
\usepackage{amsmath,amssymb}
\usepackage[amsmath]{ntheorem}
\usepackage{mathtools}
\usepackage{dsfont}

% more math
\newcommand*{\defeq}{\mathrel{\vcenter{\baselineskip0.5ex \lineskiplimit0pt
                     \hbox{\scriptsize.}\hbox{\scriptsize.}}}%
                     =}

\DeclareMathOperator*{\argmax}{arg\,max}
\DeclareMathOperator*{\argmin}{arg\,min}

\newcommand{\BigO}[1]{\ensuremath{\operatorname{O}\left(#1\right)}}


\newcommand{\indep}{\perp \!\!\! \perp}
\newcommand{\given}{\;|\;}
\newcommand{\mgiven}{\;\middle|\;}

\newtheorem{definition}{Definition}%[section]
\newtheorem{assumption}{Assumption}%[section]
\newtheorem{property}{Property}%[section]
\newtheorem{theorem}{Theorem}%[section]
\newtheorem{corollary}{Corollary}[theorem]
\newtheorem{lemma}[theorem]{Lemma}
\newtheorem{proposition}[theorem]{Proposition}
\newtheorem{postulate}[theorem]{Postulate}
\newtheorem{principle}{Principle}%[section]
\newtheorem{test}{Test}%[section]

\DeclarePairedDelimiterX{\infdivx}[2]{(}{)}{%
  #1\;\delimsize|\delimsize|\;#2%
}
\newcommand{\kld}[2]{\ensuremath{D_{KL}\infdivx{#1}{#2}}\xspace}

\usepackage{optidef}

\newcommand{\methodname}{NCS\xspace}

\algnewcommand\algorithmicforeach{\textbf{for each}}
\algdef{S}[FOR]{ForEach}[1]{\algorithmicforeach\ #1\ \algorithmicdo}

\newcommand{\revadd}{\textcolor{red}}
\newcommand{\revchange}{\textcolor{green}}
\newcommand{\revdel}{\st}