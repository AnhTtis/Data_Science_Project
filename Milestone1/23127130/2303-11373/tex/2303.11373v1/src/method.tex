\section{Proposed Framework: {\ourmodel}}
\label{model}


In this section, we introduce a novel self-supervised co-training framework {\ourmodel}.
The proposed framework is illustrated in Figure~\ref{fig:intro_model} and works in three phases.
Phase one automatically generates two sets of pseudo labels.
We use a combination of off-the-shelf pre-trained POS and NER taggers, knowledge graph, and GPT-2 scorer for generating the first set of pseudo labels automatically without any hand-crafted rules for matching the slot values.
The other set of pseudo labels is acquired through a zero-shot slot filling model~\cite{liu2020coach}, trained on the out-of-domain dataset.
It is critical to emphasize that both sets of labels are noisy and incomplete which poses serious challenges to training effective models for the task of open-domain slot filling.
Phase two fine-tunes the pre-trained BERT to the slot filling task that effectively transfers the knowledge from the pre-trained language model~(LM) to overcome the issue of label incompleteness to some extent. 
Further, we employ the early stopping technique to minimize the noise in the labels.
The output of this phase is two BERT models that can generate soft labels for self-supervision during co-training in phase three.
Phase three leverages the fine-tuned models and further trains them in an iterative fashion.
Specifically, the proposed peer training approach facilitates high-confidence soft label selection for the other peer to perform training. This phase progressively reduces the noise in the labels and enables effective model fitting. 



\subsection{Phase One: Automatic Label Generation}
To acquire the first set of labels, we perform the following steps.
First of all, off-the-shelf trained POS and NER taggers are used to predict initial estimates of the slot values irrespective of the slot types. Then, the type information of the slot values is queried from the KG and the slot value is tagged for the most appropriate slot in the target domain.
This approach, however, produces low recall. 
To expand the candidate slot values, we generate n-grams of the natural language text and employ a partial matching scheme to query the KG for type information (e.g., \myspecial{Jason} \myspecial{Aldean} = \myspecial{American} \myspecial{singer}) of the n-grams if the entry exists.
This process generates multiple overlapping hypotheses about the slot values.
We replace a span of text that corresponds to a slot value by its type information and a GPT-2 based scorer (see Section~\ref{sec:nlpmodels}) is used to select the best candidate based on the fluency of the text.
Naturally, if a token (or span of tokens) is replaced by its type, the sentence should score higher as compared to the case where an inappropriate substitution is performed. 
We select the best hypothesis if the score is greater than the threshold.
Intuitively, the candidate selection threshold can automatically be searched based on a small validation set from the target domain, making the label generation process fully automatic. 
The other set of noisy labels is acquired by the zero-shot slot filling model~\cite{liu2020coach} that has been trained using an out-of-domain dataset. It is important to highlight that the zero-shot slot filling model does not require any labeled in-domain training example. 
To summarize the automatic label generation phase, both sets of labels are acquired in a fully automatic fashion without any hand-crafting.


In contrast to previous work in weak supervision~\cite{ren2015clustype,he2017autoentity,fries2017swellshark,giannakopoulos2017unsupervised} that obtains a single set of noisy labels and then propose techniques to overcome the challenge of fitting an effective model to the noisy labels, we acquire two sets of complementary labels.
The choice of these two sets of labels is guided by the intuition that they should be complementary and the models trained on these sets of labels should be able to share complementary information with the other to improve the performance in the later phases of the framework.
Essentially, the first set of labels carries information from external knowledge sources, whereas the labels generated through the pre-trained zero-shot slot filling model capture how the slot values are mentioned in other domains.
%
To further elaborate on the motivation and our process for the first set of labels (i.e., labels using KG and other NLP models), the pre-trained LMs have been shown to have a great deal of knowledge~\cite{petroni2019language}, thus should be capable of generating automatic labels with no need of external KG. 
To the best of our knowledge, there exists no work that shows that accurate token-level automatic labeling (e.g., slot filling task) is possible with pre-trained LMs. 
Moreover, such approaches would require heavy prompting in each new target domain, whereas our label generation process is fully automatic and only relies on the readily-available pre-trained NLP models and external KG.

\subsection{Phase Two: LM-assisted Weak Supervision}
Since we do not have access to dataset $\{(\mathbf{X}_n,\mathbf{Y}_n)\}_{n=1}^N$ with true ground-truth labels.
We use pseudo labels generated in phase one, $\{(\mathbf{X}_n,\mathbf{D}_n)\}_{n=1}^N$, to learn 
$f_{m,c}(\cdot; \cdot)$ that outputs the probability of the $m$-th token to take on class $c$. 
We learn $f_{m,c}(\cdot; \cdot)$ by minimizing the following loss over the noisy dataset $\{(\mathbf{X}_n,\mathbf{D}_n)\}_{n=1}^N$: 
$$
\hat\theta = \argmin_{\theta}\frac{1}{N}\sum_{n=1}^{N} \ell(\mathbf{D}_n, f(\mathbf{X}_{n}; \theta)),
\label{eq:stage1}
$$
where $\ell(\mathbf{D}_n, f(\mathbf{X}_{n}; \theta)) = \frac{1}{M} \sum_{m=1}^{M} -\log{f_{m,d_{n, m}}(\mathbf{X}_{n}; \theta)}$. 
We employ the pre-trained multilingual BERT with token-level classification head that uses Adam optimizer \cite{kingma2014adam,Liu2019} with early stopping and multiple random initializations. 


Since slot filling task is similar to the MLM training objective of the BERT, we employ pre-trained BERT as the backbone model.
That is, MLM's goal is to predict the masked tokens using bidirectional contexts. Similarly, slot filling tries to predict the label for a token leveraging both left and right contexts simultaneously, which makes the pre-trained BERT an ideal model of choice that greatly facilitates minimizing incomplete labels.
It is important to highlight that our automatically generated labels are not only incomplete but also potentially wrong.
The training strategies employed in this phase minimize the noise in the label to some extent. 
Specifically, early stopping can provide a strong regularization and would not let the model overfit to the noisy labels, especially wrong labels. 
Moreover, early stopping does not let the model forget the knowledge in the pre-trained model.
Similarly, multiple random initializations enforce robustness. 
Since the model is fine-tuned on the noisy labels, averaging the predictions of multiple models for each token ensures that wrong labels end up with low probabilities and true labels consistently achieve high probabilities.
Using the above-mentioned strategies, we train two slot filling models, which we call the peers. The peer one is trained on the first set of pseudo labels that were generated using POS and NER taggers, KG, and the GPT-2 scorer in phase one. Similarly, peer two is trained using the predictions of the zero-shot slot filling model~\cite{liu2020coach}.
Both models have the same architecture and follow the same training procedures.

\begin{table*}[t!]
\centering
\caption{Dataset statistics.}
\vspace{-7pt}
\label{tab:dataset}
\begin{tabular}{lccccc}
\toprule
\textbf{Dataset}  & \textbf{Dataset Size} & \textbf{Vocab. Size} & \textbf{Avg. Length} & \textbf{\# of Domains} & \textbf{\# of Slots} \\ \hline
\textbf{SGD}      & 188K                  & 33.6K                & 13.8                 & 20                     & 240                  \\
\textbf{MultiWoZ} & 67.4K                 & 10.5K                & 13.3                 & 8                      & 61 \\
\bottomrule
\end{tabular}
\vspace{-7pt}
\end{table*}

\subsection{Phase Three: Self-supervised Co-training}
We introduce an iterative peer training algorithm where both peers generate high-confidence soft labels for training the other peer in the next iteration. 
Theoretically, these peers can be anything, but in this work, 
we explore two of the most promising directions that have shown the promise to minimize the need for manual labeling for the task: zero-shot learning and distant supervision.
This phase uses a self-supervised co-training scheme to exploit the patterns of slot values from other domains through the labels generated by the zero-shot filling model (i.e., peer two)~\cite{liu2020coach} as well as utilize the knowledge in external KGs and pre-trained models via labels provided by the peer one.
Specifically, we initialize the peers trained in phase two and use their pseudo labels to kick-start training in this phase.
Specifically, peer one $f_{m,c}(\cdot; \theta_{\textrm{p1}})$ would generate labels $\{\tilde{\mathbf{Y}}^{(t)}_n = [\tilde{y}_{n,1}^{(t)}, ..., \tilde{y}_{n,m}^{(t)}]\}_{n=1}^{N}$ for peer two $f_{m,c}(\cdot; \theta_{\textrm{p2}})$ at the $t$-th iteration by:
$$
\tilde{y}_{n,m}^{(t)} = \argmax_{c}{f_{m,c}(\mathbf{X}_n; \theta_{\textrm{p1}}^{(t)})}. 
\label{eq:pseudo}
$$

Based on these labels, the peer two can be fine-tuned by: 
$$
\hat\theta_{\textrm{p2}}^{(t+1)} = \argmin_{\theta}\frac{1}{N}\sum_{n=1}^N \ell(\tilde{\mathbf{Y}}_n^{(t)}, f(\mathbf{X}_{n}; \theta)).
\label{eq:self_train1}
$$

Similarly, peer two $f_{m,c}(\cdot; \theta_{\textrm{p2}})$ would generate pseudo labels for peer one $f_{m,c}(\cdot; \theta_{\textrm{p1}})$ that are used to fine-tune peer one. 
We also notice that it is beneficial to stop early during this phase as well, to improve the model fitting and gradually reduce the noise associated with the automatically generated labels.
Since pseudo labels are refined gradually in an iterative way, both peers can benefit from the knowledge contained within the labels of the other while avoiding overfitting.
Furthermore, as an alternative to pseudo labels, we also generate soft labels that are used for confidence re-weighting. 
The high-confidence soft label selection strategy enables better model fitting and efficient learning via better quality of the automatic labels.
Specifically, for the given $m$-th token in the $n$-th training example, the probability for all classes $C$ is $[f_{m,1}(\mathbf{X}_n;\theta),...,f_{m,C}(\mathbf{X}_n;\theta)]$. 
Following ~\cite{xie2016unsupervised}, at $t$-th iteration, peer one generates soft labels, $\{\mathbf{S}_n^{(t)} = [\mathbf{s}_{n,m}^{(t)}]_{m=1}^M \}_{n=1}^N$, as given below:
$$
\mathbf{s}_{n,m}^{(t)} = [s_{n,m,c}^{(t)}]_{c=1}^{C} = \Bigg[  \frac{f_{m,c}^2(\mathbf{X}_n;\theta_{\textrm{peer1}}^{(t)})/p_{c}}{\sum_{c'=1}^C f_{m,c'}^2(\mathbf{X}_n;\theta_{\textrm{peer1}}^{(t)})/p_{c'}}\Bigg]_{c=1}^{C}
\label{eq:soft}
$$ 
where $p_{c} = \sum_{n=1}^N \sum_{m=1}^M f_{m,c}(\mathbf{X}_n;\theta_{\textrm{p1}}^{(t)})$ computes the frequency of the tokens for the $c$-th class. 
Then, peer two $f(\cdot; \theta_{\textrm{p2}}^{(t+1)})$ is fine-tuned by:
$$
\theta_{\textrm{p2}}^{(t+1)} = \argmin_{\theta} \frac{1}{N} \sum_{n=1}^{N} \ell_{\rm KL}(\mathbf{S}_n^{(t)}, f(\mathbf{X}_{n}; \theta)),
$$
where $\ell_{\rm KL}(\cdot,\cdot)$ is the KL-divergence-based loss:
$$
\ell_{\rm KL}(\mathbf{S}_n^{(t)}, f(\mathbf{X}_{n}; \theta))=\frac{1}{M}\sum_{m=1}^M\sum_{c=1}^C - s_{n,m,c}^{(t)} \log f_{m,c}(\mathbf{X}_{n}; \theta).
\label{eq:klloss}
$$

Moreover, we also investigate selecting tokens that have high confidence. 
For instance, we pick high-confidence tokens from the $m$-th input example at the $t$-th iteration by  
$
H^{(t)}_n = \{m : \max_{c} s_{n,m,c}^{(t)} > \epsilon \},
$
where $\epsilon\in [0,1]$ is a threshold that can be searched based on a small validation set. 
Then, peer two $f(\cdot; \theta_{\textrm{p2}}^{(t+1)})$ is fine-tuned by:
$$
\theta_{\textrm{p2}}^{(t+1)} %&= \argmin_{\theta} \frac{1}{N} \sum_{n=1}^{N} \ell_{\rm S-KL}(\bS_n^{(t)}, f(\bX_{n}; \theta)) \\
= \argmin_{\theta} \frac{1}{N|H^{(t)}_n|}\sum_{n=1}^{N} \sum_{m\in H^{(t)}_n}\sum_{c=1}^C - s_{n,m,c}^{(t)} \log f_{m,c}(\mathbf{X}_{n}; \theta).
$$

This phase improves the robustness to effectively fit the model for tokens with high confidence. 
Both peers keep sharing information and their confidence by producing soft labels for their counterparts until they approximate to the true labels while employing early stopping and scheduled learning rates.
It is important to remind that phase three is the most important phase that progressively reduces noise from the labels to a great extent and enables superior performance for the task of open-domain slot filling.

\section{Neural Constraint Satisfaction} \label{sec:method}
In \S\ref{sec:problem} we introduced a structured problem formulation for object rearrangement and reduced it to solving the correspondence and combinatorial problems.
We now present our method, Neural Constraint Satisfaction (\methodname) as a method for controlling an agent that plans over and acts with a state transition graph constructed from learned entity representations.
This section is divided into two parts: modeling and control.
The modeling part is further divided into two parts: representation learning and graph construction.
The representation learning part addresses the correspondence problem, while the graph construction and control parts address the combinatorial problem.

\subsection{Modeling}
The modeling component of~\methodname abstracts the experience buffer into a factorized state transition graph that can be reused across different rearrangement problems.
Below we describe how we first train an object-centric world model to infer entities that are independent, symmetric, and factorized and then construct the state transition graph by clustering entities with similar state transitions.
These two steps comprise a two-level abstraction hierarchy over the raw sensorimotor transitions.

\paragraph{Level 1: representation learning}
The first level concerns the unsupervised learning of entity representations that factorizes into their action-invariant features (their \textbf{type}) and their action-dependent features (their \textbf{state}).
Concretely our goal is to model a video transition $o_t, a_t \rightarrow o_{t+1}$ as a transition over entity-sets $\mathbf{h}_t, a_t \rightarrow \mathbf{h}_{t+1}$, where each entity $h^k$ is factorized as a pair $h^k = (z^k, s^k)$.
Given our setting where an action moves only a single object in the environment at a time, successful representation learning implies three criteria: (1) the world model properly identifies the individual entity $h^k$ corresponding to the moved object, (2) only the state $s^k$ of that entity should change, while its type $z^k$ should remain unaffected, and (3) other entity representations $h^{\neq k}$ should also remain unaffected.
Criteria (1) and (3) rule out standard approaches that represent an entire scene with a monolithic representation, so we need an object-centric world model instead of a monolithic world model.
But criterion (2) rules out standard object-centric world models (e.g.~\citep{veerapaneni2020entity,elsayed2022savi++,singh2022simple}), which do not decompose entity representations into action-invariant and action-dependent features.

Because the parameters of a mixture model are independent and symmetric by construction, we propose to construct our factorized object-centric world model as an equivariant sequential Bayesian filter with a mixture model as the latent state, where entity representations are the parameters of the mixture components.
Recall that a filter consists of two major components, latent estimation and latent prediction.
We implement latent estimation with the state-of-the-art slot attention (SA)~\citep{locatello2020slot}, based on the connection between mixture components and SA slots~\citep{chang2022object}.
We implement latent prediction with the transformer decoder (TFD) architecture~\citep{vaswani2017attention} because TFD is equivariant with respect to its inputs.
We denote the SA \texttt{slot}s as $\bm{\lambda}$ and SA \texttt{attn} masks as $\bm{\alpha}$.
We split each \texttt{slot} $\lambda \in \mathbb{R}^n$ into two halves $\lambda^z \in \mathbb{R}^{\frac{n}{2}}$ and $\lambda^s \in \mathbb{R}^{\frac{n}{2}}$.
Given observations $o$ and actions $a$, we embed the actions as $\tilde{a}$ with an feedforward network and implement the filter as:
\begin{align*}
    \hat{\bm{\lambda}}_1 &\sim \text{Gaussian} \qquad
    &\hat{\bm{\lambda}^{s}}_{t+1} &= TFD\left(
        \text{queries}=\bm{\lambda}^{s}_{t}, 
        \text{keys/values}=\left[\bm{\lambda}^{s}, \tilde{a}_t\right]
        \right) \\
    \bm{\lambda}_{t}, \bm{\alpha}_{t} &= SA\left(\hat{\bm{\lambda}}_{t}, o_{t}\right) \qquad
    &\hat{\bm{\lambda}}_{t+1} &= \left[\bm{\lambda}^{z}_t, \hat{\bm{\lambda}^{s}}_{t+1}\right]
\end{align*}
where $[ \cdot, \cdot ]$ is the concatenation operator, $\hat{\bm{\lambda}}$ is the output of the latent prediction step, and $\bm{\lambda}$ is the output of the latent estimation step.
We embed this filter inside the SLATE backbone~\citep{singh2022illiterate} and call this implementation \textbf{dynamic SLATE} (dSLATE).
For a background on SLATE, as well as dSLATE hyperparamters, see Appdx.~\ref{appdx:slate_background}.

By constructing $\hat{\bm{\lambda}^{z}}_{t+1}$ as a copy of $\bm{\lambda}^{z}_t$, dSLATE enforces the information contained $\bm{\lambda}^{z}$ to be action-invariant, hence we treat $\bm{\lambda}^{z}$ as dSLATE's representation of the entities' types.
As for the entities' states, either the action-dependent part of the slots $\bm{\lambda}^{s}$ or the attention masks $\alpha$ can be used.
Using $\alpha$ may be sufficient and more intuitive to analyze if all objects looks similar and there is no occlusion, while $\bm{\lambda}^{s}$ may be more suitable in other cases, and we provide an example of each in the experiments.
To simplify notation going forward and connect with the notation in \S\ref{sec:problem}, we use $\mathbf{h}$ to refer to $(\bm{\lambda}, \bm{\alpha})$, use $\bm{z}$ to refer to $\bm{\lambda}^{z}$, and use $\bm{s}$ to refer to $\bm{\lambda}^{s}$ or $\bm{\alpha}$.
Thus by construction dSLATE satisfies criterion (2).
Empirically we observe that it satisfies criterion (1) as well as SLATE does, and that TFD learns to sparsely edit $\bm{\lambda}^{s}_t$, thereby satisfying criterion (3).

\paragraph{Level 2: graph construction} \label{sec:build_graph}
Having produced from the first level a buffer of entity-set transitions ${\{\mathbf{h}_t, a_t \rightarrow \mathbf{h}_{t+1}\}_{n=1}^N}$, the goal of the second level (Fig.~\ref{fig:model}b) is to use this buffer to construct a factorized state transition graph.
The key to solving the combinatorial problem is to construct the edges of this graph to represent not state transitions of entire entity-sets (i.e. ${\mathbf{s}_t, a_t \rightarrow \mathbf{s}_{t+1}}$) as prior work does~\citep{zhang2018composable}, but state transitions of \emph{individual entities} (i.e. ${s^k_t, a_t \rightarrow s^k_{t+1}}$).
Constructing edges over transitions for individual entities rather than entity sets enables the same transition to be reused with different context entities present.
Constructing edges over state transitions instead of entity transitions enables the same transition to be reused across entities with different types.
This would enable the agent to recompose sequences of previously encountered state transitions for solving new rearrangement problems with different entities in different contexts.
Henceforth our use of ``state'' refers specifically to the state of individual entity unless otherwise stated. 

Given our bisimulation assumption that states can be partitioned into a finite number of groups, we construct our graph such that nodes represent equivalence classes among individual states and the edges represent actions that transform a state from one equivalence class to another.
To implement this we cluster state transitions of individual entities in the buffer, which reduces to clustering the states of individual entities before and after the transition.
We treat each cluster centroid as a node in the graph, and an edge between nodes is tagged with the single action that transforms one node's state to another's.
The algorithm for constructing the graph is shown in Alg.~\ref{alg:build_graph} and involves three steps: (1) isolating the state transition of an individual entity from the state transition of the entity-set, (2) creating graph nodes from state clusters, and (3) tagging graph edges with actions.

\begin{wrapfigure}[20]{r}{0.6\linewidth}
\vspace{-20pt}
\begin{minipage}{0.6\textwidth}
\begin{algorithm}[H]
  \caption{Building the Graph}\label{alg:build_graph}
\small
  \begin{algorithmic}[1]
    \State \textbf{input} \texttt{model}, \texttt{buffer}
    \For{$\{(o_t, a_t, o_{t+1})\}_n$ in \texttt{buffer}} 
        \State{\textcolor{gray}{\footnotesize{\# infer entities from transition}}}
        \State $\{(\mathbf{h}_t, a_t, \mathbf{h}_{t+1})\}_n \gets \texttt{model}\left(\{o_t, a_t, o_{t+1}\}_n\right)$.
        \State{\textcolor{gray}{\footnotesize{\# identify which entity changed in transition}}}
        \State $\{(h^k_t, a_t, h^k_{t+1})\}_n\gets \texttt{isolate}\left(
        \{(\mathbf{h}_t, a_t, \mathbf{h}_{t+1})\}_n\right)$
    \EndFor
    \State{\textcolor{gray}{\footnotesize{\# partition transitions by clustering entities}}}
    \State $\{s_*\}_{m=1}^M \gets \texttt{cluster}\left(\{(s^k_t, a_t, s^k_{t+1})\}_{n=1}^N\right)$ 
    \State{\textcolor{gray}{\footnotesize{\# transitions between clusters are edges}}}
    \State \textbf{initialize} \texttt{graph} with nodes $s_*^{[m]}$, for $m \in [1:M]$
    \ForEach{$\{(h^k_t, a_t, h^k_{t+1})\}_n$}
        \State{\textcolor{gray}{\footnotesize{\# infer cluster assignments}}}
        \State $[i], [j]  \gets \texttt{bind}\left(h_{t}^k\right), \texttt{bind}\left(h_{t+1}^k\right)$
        \State{\textcolor{gray}{\footnotesize{\# tag edge with action $a_t$}}}
        \State \texttt{graph.edges}$[i, j] \gets \texttt{create-edge}\left(s_*^{[i]} \overset{a_t}{\rightarrow} s_*^{[j]}\right)$
    \EndFor
    \State \Return \texttt{graph}
  \end{algorithmic}
\end{algorithm}
\end{minipage}
\vspace{-10pt}
\end{wrapfigure}

The first step is to identify which object was moved in each transition, i.e. identifying the entity $h^k$ that dSLATE predicted was affected by $a_t$ in the transition $(\mathbf{h}_t, a_t, \mathbf{h}_{t+1})$.
We implement a function \texttt{isolate} that achieves this by solving $k = \argmax_{k' \in \{1, ..., K\}} d(s^{k'}_t, s^{k'}_{t+1})$ to identify the index of the entity whose state has most changed during the transition, where $d(\cdot, \cdot)$ is a distance function, detailed in Table~\ref{tab:FACTS_hyperparameters} of the Appendix.
This converts the buffer of transitions over entity-sets ${\mathbf{h}_t, a_t \rightarrow \mathbf{h}_{t+1}}$ into a buffer of transitions over individual entities ${h^k_t, a_t \rightarrow h^k_{t+1}}$.

The second step is to cluster the states before and after each transition.
We implement a function \texttt{cluster} that uses K-means to returns graph nodes as the centroids $\{s_*\}_{m=1}^M$ of these state clusters.

The third step is to connect the nodes with edges that record actions that actually were taken in the buffer to transform one state to the next.
We implement a function \texttt{bind} that, given entity $h^k$, returns the index $[i]$ of the centroid $s_*$ that is the nearest neighbor to the entity's state $s^k$.
For each entity transition $(h_t^k, a_t, h_{t+1}^k)$ we \texttt{bind} entity $h_t^k$ and $h_{t+1}^k$ to their associated nodes $s_*^{[i]}$ and $s_*^{[j]}$ and create an edge between $s_*^{[i]}$ and $s_*^{[j]}$ tagged with action $a$, overwriting previous edges based on the assumption that with a proper clustering there should only be one action per pair of nodes.

In our experiments both \texttt{cluster} and \texttt{bind} use the same distance metric (see Table~\ref{tab:dslate_hyperparameters} in the Appendix), but other clustering algorithms and distance metrics can also be used.
Our experiments (Fig.~\ref{fig:robogym_analysis}) also show that it is also possible to have more than one action primitive per pair of nodes as long as these actions all map between states bound to the same pair of nodes.

\begin{figure}
    \centering
    \includegraphics[width=0.9\textwidth]{figs/planning_and_control_large.pdf}
    \caption{\small{\textbf{Planning and control}.
    Given a rearrangement problem specified only by the current and goal observations $(o_0, o_g)$,~\methodname decomposes the rearrangement problem into one subproblem $(o_t, o_g)$ per entity.
    (a) shows the computations~\methodname uses to solve each subproblem and (b-d) show these steps in context.
    For each subproblem $(o_t, o_g)$,~\methodname infers entities from both the current and goal observations.
    The states of the goal entities indicate constraints on the desired locations of the current entities.
    (b) ~\methodname aligns the indices of the current entities to those of the goal entities with corresponding types.
    (c) It selects the index $k$ of the next goal constraint $s_g^k$ to satisfy, as indicated by the red box.
    The selected goal constraint and current entity are also colored black in (a), and note that their types are the same but states are different: we want to choose the action to transform the state of the current entity to the state of the goal constraint.
    (d) It binds the selected goal constraint and its corresponding current entity to nodes $s_*$ and $s_*'$ in the transition graph.
    Lastly, it identifies the edge connecting those two nodes and executes the action tagged to that edge in the environment.
    }}
    \label{fig:planning_and_control}
    \vspace{-10pt}
\end{figure}

\subsection{Control}
To solve new rearrangement problems, we re-compose sequences of state transitions from the graph.
Specifically, the agent decomposes the rearrangement problem into a set of per-entity subproblems (e.g. initial and goal positions for individual objects), searches the transition graph for a transition that transforms the current entity's state to its goal state, and executes the action tagged with this transition in the environment.
This problem decomposition is possible because the transitions in our graph are constructed to be agnostic to type and context, enabling different rearrangement problems to share solutions to the same subproblems.
The core challenge in deciding which transitions to compose is in determining which transitions are \emph{possible} to compose.
That is, the agent must determine which nodes in the graph correspond to the given goal constraints and which nodes correspond to the entities in the current observation, but the current entities $\mathbf{h}_t$  and goal constraints $\mathbf{h}_g$ must themselves be inferred from the current and goal observations $o_t$ and $o_g$, requiring the agent to infer both what to do and how to do it purely from its sensorimotor interface.

\begin{wrapfigure}[17]{r}{0.45\linewidth}
\vspace{-13pt}
\begin{minipage}{0.45\textwidth}
\begin{algorithm}[H]
  \caption{Action Selection}\label{alg:use_graph}
  \begin{algorithmic}[1]
    \State \textbf{given} \texttt{model}, \texttt{graph}
    \State \textbf{input} goal $o_g$, observation $o_t$
    \State{\textcolor{gray}{\footnotesize{\# infer goal constraints and current entities}}}
    \State $\mathbf{h_g}, \mathbf{h_t} \leftarrow \texttt{model}\left(o_g\right), \texttt{model}\left(o_t\right)$
    \State{\textcolor{gray}{\footnotesize{align entity indices of $\mathbf{h_t}$ with those of $\mathbf{h_g}$}}}
    \State $\pi \gets \texttt{align}\left(\mathbf{h_t}, \mathbf{h_g}\right)$
    \State{\textcolor{gray}{\footnotesize{permute indices of $\mathbf{h_t}$ according to $\pi$}}}
    \State $\mathbf{h_t} \gets (h_t^{\pi[1]}, ..., h_t^{\pi[K]})$
    \State{\textcolor{gray}{\footnotesize{identify $k$th goal constraint to satisfy next}}}
    \State $k \leftarrow \texttt{select-constraint}\left(\mathbf{h_t}, \mathbf{h_g}\right)$
    \State{\textcolor{gray}{\footnotesize{infer cluster assignments}}}
    \State $[i], [j]  \gets \texttt{bind}\left(h_{t}^k\right), \texttt{bind}\left(h_{g}^k\right)$
    \State{\textcolor{gray}{\footnotesize{action that transforms node $[i]$ to node $[j]$}}}
    \State \Return \texttt{graph.edges}$[i, j]$\texttt{.action}
  \end{algorithmic}
\end{algorithm}
\end{minipage}
\end{wrapfigure}

Our approach takes four steps, summarized in Alg.~\ref{alg:use_graph} and Fig.~\ref{fig:planning_and_control}, with further details in Appdx.~\ref{appdx:action_selection}.
In the first step, we use dSLATE to infer $\mathbf{h}_t$ and $\mathbf{h}_g$ from $o_t$ and $o_g$ (e.g. the positions and types of all objects in the initial and goal images).
In the second step (Fig.~\ref{fig:planning_and_control}b), because of the permutation symmetry among entities, we find a bipartite matching that matches each entities in $h_g^j$ with a corresponding entity in $h_t^k$ that shares the same type and permute the indices $k$ of $\mathbf{h}_t$ to match those of $\mathbf{h}_g$.
We implement a function \texttt{align} that uses the Hungarian algorithm to perform this matching over $(z^1_t, ... z^K_t)$ and $(z^1_g, ... z^K_g)$, with Euclidean distance as the matching cost.
The third step selects which goal constraint $h^k_g$ to satisfy next (Fig.~\ref{fig:planning_and_control}c).
W implement this \texttt{select-constraint} procedure by determining which constraint $h^k_g$ has the highest difference in state with its counterpart $h^k_t$, which reduces to solving the same argmax problem as in \texttt{isolate} with the same distance function used in \texttt{isolate}.
The last step chooses an action given the chosen goal constraint $h^k_g$ and its counterpart  $h^k_t$, by \texttt{bind}ing $h^k_t$ and $h^k_g$ to the graph based on their state components and returning the action tagged to the edge between their respective nodes (Fig.~\ref{fig:planning_and_control}d).
If an edge does not exist between the inferred nodes, then we simply take a random action.
