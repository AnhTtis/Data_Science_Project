\begin{figure}[th]
\begin{subfigure}[]{.24\textwidth}
  \centering
    \includegraphics[width=\textwidth]{figs/horizon_rand.png}
  \caption{Rand}
  \label{fig:interaction_horizon:rand}
\end{subfigure}
\hfill
\begin{subfigure}[]{.24\textwidth}
  \centering
    \includegraphics[width=\textwidth]{figs/horizon_mgs.png}  
  \caption{NF}
  \label{fig:interaction_horizon:NF}
\end{subfigure}
\hfill
\begin{subfigure}[]{.24\textwidth}
  \centering
    \includegraphics[width=\textwidth]{figs/horizon_mpc.png}  
  \caption{MPC}
  \label{fig:interaction_horizon:mpc}
\end{subfigure}
\hfill
\begin{subfigure}[]{.24\textwidth}
  \centering
    \includegraphics[width=\textwidth]{figs/horizon.png}
  \caption{\methodname}
  \label{fig:interaction_horizon:hlg}
\end{subfigure}
\caption{\small{\textbf{Varying interaction horizon.}
The performance of the NF (b) and MPC (c) baselines compared to~\methodname (d, reproduced from Fig.~\ref{fig:robogym_analysis}) and the random baseline (a) on \emph{robogym-rearrange} as we vary the interaction horizon (as a multiple of the minimum steps needed to complete the task).
\emph{Note that the scale of the y-axis is not the same.}
While a longer horizon improves performance,~\methodname still achieves at least 50x better accuracy with an interaction horizon multiplier of 1 than the performance obtained by increasing the interaction horizon multiplier for the model-based baselines to 8.
}}
\label{fig:robogym_analysis_baselines}
\end{figure}