\begin{figure}
    \centering
    \includegraphics[width=0.9\textwidth]{figs/planning_and_control_large.pdf}
    \caption{\small{\textbf{Planning and control}.
    Given a rearrangement problem specified only by the current and goal observations $(o_0, o_g)$,~\methodname decomposes the rearrangement problem into one subproblem $(o_t, o_g)$ per entity.
    (a) shows the computations~\methodname uses to solve each subproblem and (b-d) show these steps in context.
    For each subproblem $(o_t, o_g)$,~\methodname infers entities from both the current and goal observations.
    The states of the goal entities indicate constraints on the desired locations of the current entities.
    (b) ~\methodname aligns the indices of the current entities to those of the goal entities with corresponding types.
    (c) It selects the index $k$ of the next goal constraint $s_g^k$ to satisfy, as indicated by the red box.
    The selected goal constraint and current entity are also colored black in (a), and note that their types are the same but states are different: we want to choose the action to transform the state of the current entity to the state of the goal constraint.
    (d) It binds the selected goal constraint and its corresponding current entity to nodes $s_*$ and $s_*'$ in the transition graph.
    Lastly, it identifies the edge connecting those two nodes and executes the action tagged to that edge in the environment.
    }}
    \label{fig:planning_and_control}
    \vspace{-10pt}
\end{figure}